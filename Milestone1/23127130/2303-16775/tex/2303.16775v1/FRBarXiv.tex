\documentclass[twocolumn]{aastex63}
\usepackage[normalem]{ulem}
\usepackage{graphicx}
\usepackage{dcolumn}
\usepackage{bm}
\usepackage{color}
%\usepackage[colorlinks,citecolor=blue,urlcolor=blue,linkcolor=blue]{hyperref}% ÎÄÏ×ÁŽœÓ
\usepackage{array}
\usepackage{multirow}
\usepackage{amsmath}
%\usepackage{makecell}

%\usepackage{multicol}
\usepackage{subfigure}
\usepackage{times}
\usepackage{mathptmx}
\usepackage{sistyle}
\SIthousandsep{\,}
%\newcommand{\ap}{\alpha}
\newcommand{\dm}{{\rm DM}}
\newcommand{\host}{{\rm host}}
\newcommand{\mw}{{\rm MW}}
\newcommand{\igm}{{\rm IGM}}
\newcommand{\halo}{{\rm halo}}

\newcommand{\dd}{{\rm d}}
\newcommand{\bt}{\beta}
\newcommand{\ep}{\epsilon}
\newcommand{\f}{\frac}
\newcommand{\lt}{\left}
\newcommand{\n}{\nonumber}
\newcommand{\p}{\partial}
\newcommand{\rd}{{\rm d}}
\newcommand{\rt}{\right}
\newcommand{\ve}{\varepsilon}
\newcommand{\vp}{\varphi}
\newcommand{\arXg}[1]{\href{http://arxiv.org/abs/#1}{{\ttfamily arXiv:#1[gr-qc]}}}
\newcommand{\arXh}[1]{\href{http://arxiv.org/abs/#1}{{\ttfamily arXiv:#1[hep-th]}}}
\renewcommand\arraystretch{1.3}


\newcommand{\reffg}[1]{Figure~\ref{#1}}
\newcommand{\reftb}[1]{Table~\ref{#1}}
\newcommand{\refeq}[1]{Equation~(\ref{#1})}
\newcommand{\refsc}[1]{Section~\ref{#1}}

\begin{document}
\title{ Fast radio burst energy function in the presence of $\dm_\host$ variation}
\author{Yichao Li}
\affiliation{Key Laboratory of Cosmology and Astrophysics (Liaoning Province) \& Department of Physics, \\
College of Sciences, Northeastern University, Shenyang 110819, China}

\author{Jia-Ming Zou}
\affiliation{Key Laboratory of Cosmology and Astrophysics (Liaoning Province) \& Department of Physics, \\
College of Sciences, Northeastern University, Shenyang 110819, China}

\author{Ji-Guo Zhang}
\affiliation{Key Laboratory of Cosmology and Astrophysics (Liaoning Province) \& Department of Physics, \\
College of Sciences, Northeastern University, Shenyang 110819, China}

\author{Ze-Wei Zhao}
\affiliation{Key Laboratory of Cosmology and Astrophysics (Liaoning Province) \& Department of Physics, \\
College of Sciences, Northeastern University, Shenyang 110819, China}

\author{Jing-Fei Zhang}
\affiliation{Key Laboratory of Cosmology and Astrophysics (Liaoning Province) \& Department of Physics, \\
College of Sciences, Northeastern University, Shenyang 110819, China}

\author{Xin Zhang}
\correspondingauthor{Xin Zhang}
\email{zhangxin@mail.neu.edu.cn}

\affiliation{Key Laboratory of Cosmology and Astrophysics (Liaoning Province) \& Department of Physics, \\
College of Sciences, Northeastern University, Shenyang 110819, China}
\affiliation{Key Laboratory of Data Analytics and Optimization for Smart Industry (Ministry of Education), \\
Northeastern University, Shenyang 110819, China}
\affiliation{National Frontiers Science Center for Industrial Intelligence and Systems Optimization, \\
Northeastern University, Shenyang 110819, China}

%\affiliation{Center for High Energy Physics, Peking University, Beijing 100080, China}
%\affiliation{CAS Key Laboratory of Theoretical Physics, Institute of Theoretical Physics, Chinese Academy of Sciences, Beijing 100190, China}
%\affiliation{Center for Gravitation and Cosmology, Yangzhou University, Yangzhou 225009, China}
%\date{Received: date / Accepted: date}


\begin{abstract}
Fast radio bursts (FRBs) have been found in great numbers but the physical mechanism of these sources
is still a mystery. The redshift evolutions of the FRB energy distribution function and the volumetric rate
shed light on revealing the origin of the FRBs. However, such estimations rely on the 
dispersion measurement (DM)--redshift ($z$) relationship. A few of FRBs detected recently show large excess DM 
beyond the expectation from the cosmological and Milky Way contributions, which indicates large spread of 
DM from their host galaxies. In this work, we adopt the lognormal distributed $\dm_\host$ model and estimate the 
energy function using the non-repeating FRBs selected from the Canadian Hydrogen Intensity Mapping Experiment 
(CHIME)/FRB Catalog 1. By comparing the lognormal distributed $\dm_\host$ model to the constant $\dm_\host$ model, 
the FRB energy function results are consistent within the measurement uncertainty.
We also estimate the volumetric rate of the non-repeating FRBs in three different redshift bins.
The volumetric rate shows that the trend is consistent with the stellar-mass density redshift evolution. 
Since the lognormal distributed $\dm_\host$ model increases the measurement errors,
the inference of FRBs tracking the stellar-mass density is nonetheless undermined.
%but has no significant effect on redshift evolution of volumetric rate.

%Furthermore, the most of them redshift of is still uncertain. We estimate the redshift of the known FRBs, and discuss the  energy distributions of the FRBs in this paper. The method is to use the relationship between dispersion measurement (DM)-redshift (z) and compare it with the redshift of observed galaxies, we model the FRBs energy distribution and fit the parameters of Schechter function of FRBs energy. The fitting parameters are compared with other known astrophysical event rates and the possibility of the FRB origin model is analyzed by the event rates. We infer a characteristic energy cut-off of $\rm{E_{char}}$=1.01×$10^{41}$ erg and a differential power-law index of ${\gamma}$= -0.86. Moreover, we infer a volumetric rate 5.75×$10^{4}$ of bursts Gpc$^{-3}$ year$^{-1}$above a pivot energy of $10^{39}$ erg.



\end{abstract}
%\pacs{}

%\keywords{fast radio bursts, redshift, energy function, event rates}
\keywords{stars: luminosity function, mass function --- methods: data analysis --- radio continuum: stars}

%\maketitle

\section{Introduction} \label{sec:intro}
The fast radio bursts (FRBs) are bright, violent flashes of radio emission with durations of the order of milliseconds. 
In 2007, the first FRB event was discovered by \citet{Lorimer:2007qn} from the archived data of the Parkes telescope
in Australia. Since then, a couple of FRB events were discovered by different radio telescopes, 
e.g., the Green Bank Telescope (GBT) and the Arecibo radio telescope. 
\citet{Chatterjee:2017dqg} for the first time confirmed the host galaxy of one FRB, i.e., FRB121102, which was discovered
by the Arecibo radio telescope \citep{Spitler:2014fla}.
Recently, more than 800 FRBs are discovered by the advanced radio telescopes, such as
the Canadian Hydrogen Intensity Mapping Experiment (CHIME)\footnote{\url{https://www.chime-frb.ca/catalog}}, the Australian Square Kilometre Array Pathfinder
(ASKAP)\footnote{\url{https://www.atnf.csiro.au/projects/askap/index.html}}, etc. 
Most of the FRBs are non-repeating events and dozens of sources are repeating bursts.
Currently, only 21 of the FRBs are localized. In the future, it is expected that a large number of FRBs can be localized, and these localized FRBs have the potential to be developed into a powerful cosmological probe \citep{Zhao:2020ole,Qiu:2021cww,Wu:2022dgy,Zhao:2022bpd}.

There is still an open question for the origin of FRBs. A series of theoretical models have been proposed 
to explain the FRB origin \citep{Platts:hiy}. 
Accurate localizations of FRBs give us hints about the origin of FRBs. 
In 2020, the Galactic magnetar SGR J1935+2154 \citep{Bochenek:2020zxn,CHIMEFRB:2021srp}  was found to produce an FRB 
that coincided in time with a non-thermal X-ray burst from the magnetar. 
It supports the conjecture that some FRBs may be produced by magnetars. 
The repeating FRB 20200120E is localized at the position of a globular cluster in M81 \citep{Bhardwaj:2021xaa}. 
Globular clusters are usually composed of old stars with low metal content, so this FRB source requires an 
old population as the progenitor. 
By contrast, the repeating FRB 20201124A is localized to a massive, star-forming galaxy \citep{Fong:2021xxj} 
at the redshift of $z = 0.098$~\citep{Nimmo:2021ntn}. 
However, only a small number of localized events are actually inadequate to reveal the origin of FRBs.

The FRB energy function is also an effective way to constrain the origin models of FRBs 
\citep{James:2021oep,Zhang:2020ass,Luo:2020wfx}. 
The energy functions allow us to study the redshift evolution of volumetric rate of FRBs. 
%Study the luminosity function or energy function of FRB, which can tell us the volumetric FRB rates%. 
If the origin of FRBs is related to young stellar populations, the volumetric FRB rate should increase with 
increasing redshift, as the density of cosmic star formation rate increases towards higher redshift, 
up to $z$ $\sim$ 2. Conversely, if FRB progenitors are old, such as white dwarfs,
neutron stars or black holes, the volumetric rate follows the evolution of stellar mass,
i.e. it should decrease towards higher redshift.
\cite{Hashimoto:2020acj} found that the number density of non-repeating FRB sources 
does not show any significant redshift evolution, which is consistent with the evolution of 
stellar mass in the universe. In their recent research,  \cite{Hashimoto:2022llm} reported that 
neutron stars and black holes are more likely to be the progenitors of non-repeating FRBs. 

%They use the observed dispersion measurement method to derive the redshift of each FRB and adopt 
%a constant $\dm_\host$ = 50 ${\rm pc}\,{\rm cm}^{-3}$. 
Recently, a few of FRBs show large excess DM beyond the expectation from
the cosmological and Milky Way contributions 
\citep{Spitler:2014fla,Chatterjee:2017dqg,
Hardy:2017djf,Tendulkar:2017vuq,2021ApJ...922..173C}. 
In particular, \citet{Niu:2021bnl} reported a detection of FRB 190520B 
with $\dm_{\host} \approx 903^{+72}_{-111}\,{\rm pc}\,{\rm cm}^{-3}$, 
which is almost an order of magnitude higher than the average $\dm_\host$ of the FRBs discovered so far.
We summarize the values of $\dm_{\rm obs}$ and $\dm_\host$ of all the localized FRBs in \reftb{tab1}.
%We found that the value of $\dm_\host$  affects the  luminosity/energy functions and volumetric rates of FRB. 
In this work, we study the effect of $\dm_\host$ uncertainty on the FRB energy function estimation, 
as well as the redshift evolution of FRB event rate. Note that, in this work, the {\it Planck} 2018 $\Lambda$ cold dark matter model is adopted as a fiducial model, with the best-fit cosmological parameters $H_0=67.36$ km s$^{-1}$ Mpc$^{-1}$, $\Omega_{\rm b}=0.0493$, $\Omega_{\rm m}=0.3153$, and $\Omega_{\Lambda}=0.6847$.

%The $Planck18$ cosmology  is adopted as a fiducial model, i.e. $\Lambda$ cold dark matter cosmology with ($H_0$, $\Omega_{\rm b}$, $\Omega_{\Lambda}$, $\Omega_{\rm m}$) = (67.36, 0.0493, 0.6847, 0.3153).

\begin{table}
{\scriptsize
%\begin{center}
\centering
\caption{$\dm_{\rm obs}$ and $\dm_{\host}$ of the localized FRBs.}\label{tab1}
\begin{tabular}{cccc} \hline \hline
${\rm ID}$  & ${\rm DM}_{\rm obs}\,[{\rm pc}\,{\rm cm}^{-3}]$ & $z$ &  
$\dm_{\host}\,[{\rm pc}\,{\rm cm}^{-3}]$ \\ \hline
FRB20121102A &  557.000 & 0.1927 &   41.921690 \\
FRB20180301A &  536.000 & 0.3305 &   37.579857 \\
FRB20180916B &  348.800 & 0.0337 &   48.369933 \\
FRB20180924B &  362.160 & 0.3212 &   37.838656 \\
FRB20181112A &  589.000 & 0.4755 &   33.886818 \\
FRB20190102C &  364.545 & 0.2913 &   38.720669 \\
FRB20190523A &  760.800 & 0.6600 &   30.120482 \\
FRB20190608B &  340.050 & 0.1178 &   44.730721 \\
FRB20190611B &  332.630 & 0.3778 &   36.289737 \\
FRB20190711A &  592.600 & 0.5217 &   32.857988 \\
FRB20190714A &  504.130 & 0.2365 &   40.436717 \\
FRB20191001A &  506.920 & 0.2340 &   40.518639 \\
FRB20191228A &  297.500 & 0.2432 &   40.218790 \\
FRB20200120E &   87.820 & 0.0008 &   50.007001 \\
FRB20200430A &  380.250 & 0.1608 &   43.073742 \\
FRB20200906A &  577.800 & 0.3688 &   36.528346 \\
FRB20201124A &  412.000 & 0.0979 &   45.541488 \\
FRB20220610A & 1457.600 & 1.0160 & 1304.00000 \\
FRB20190520B & 1204.700 & 0.2400 &  903.000000 \\
FRB20210405L &  565.170 & 0.6600 &   21.000000 \\
FRB20190614D &  959.000 & 0.6000 &   57.000000 \\ \hline
\end{tabular}
%\end{center}
}
\end{table}

This paper is organized as follows;
In \refsc{sec:data}, we describe the FRB catalogue, as well as the selection criteria, 
used in this work.
The Bayesian framework used for the redshift estimation is described in \refsc{sec:method}. 
In \refsc{Results}, we present the energy functions and volumetric rates of non-repeating FRB sources along with their redshift evolution. The conclusions are presented in \refsc{ccc}.



\section{Data}\label{sec:data}

\begin{figure*}
    \centering
    \includegraphics[width=0.9\textwidth]{plot_footprint.pdf}
    \caption{The sky locations of the selected FRBs. The footprint of the galaxy sample used in this work is shown with the green area
    and the FRBs are shown with the black circles. 
    }\label{fig:footprint}
\end{figure*}

\subsection{FRB catalog}\label{sec:frbcat}

We use the first release of CHIME/FRB Catalog 1 \citep{CHIMEFRB:2021srp}\footnote{\url{https://www.chime-frb.ca/home}},
which contains 474 non-repeating bursts and 62 repeating bursts from $18$ repeaters.
The $536$ burst events were detected between from 2018 July 25 to 2019 July 1 
over an effective survey duration of $214.8$ days. 
In order to minimize the selection effects, a number of criteria are suggested
\citep{Shin:2022crt,Hashimoto:2022llm}.

\begin{enumerate}
\item Events with {\tt bonsai\_snr} $<10$ are rejected, where {\tt bonsai\_snr} 
is the signal-to-noise ratio (S/N) recorded in the catalog. 
\cite{Shin:2022crt} suggests S/N cut of $12$ since signal below ${\rm S/N} =12$ may be 
misclassified as radio-frequency interference (RFI).
In this work, we use the bursts with S/N over $10$, which maintains a meaningful number of FRB samples for the 
statistical analysis \citep{Hashimoto:2022llm}.
\item Events with ${\rm DM}_{\rm obs} < 1.5\times {\rm max}\left({\rm DM}_{\rm NE2001}, {\rm DM}_{\rm YMW16}\right)$ are rejected to ensure extragalactic origin of the events.
${\rm DM}_{\rm obs}$ is the measured DM; ${\rm DM}_{\rm NE2001}$ and 
${\rm DM}_{\rm YMW16}$ are the DM of the Milky Way estimated according to 
the NE2001 model \citep{Cordes:2002wz} and the YMW16 model \citep{Yao_2017}, 
respectively.
\item Events with $\log_{10}\left(\tau_{\rm scat}/{\rm ms}\right) > 0.8$ are rejected, 
where $\tau_{\rm scat}$ is the scattering timescale. 
\item Events with $\log_{10}\left(F_\nu/{\rm Jy\,ms}\right) < 0.5$ are rejected, where $F_\nu$
is the fluence of the burst. 
\item Events detected in the side lobes of the telescope primary beam are rejected.
\end{enumerate}

After applying the selection criteria, we have $176$ FRB events selected, including $12$
repeaters. 

Previous analysis with the mock data shows that a significant
fraction of FRBs are missed by the CHIME detection algorithm, i.e., only 39638 out of
84697 injected mock events are detected \citep{Shin:2022crt}. The total number of events needs to be
scaled according to the detection fraction,
\begin{equation}\label{eq:selnum}
N_{\rm FRB} = N_{\rm obs} \times \frac{\num{84697}}{\num{39638}}.
\end{equation}

In addition, the fraction of missed events also depend on the property of the FRB signal.
Longer scattering times or lower fluencies result in a higher number of missed events.
Following \cite{Hashimoto:2022llm}, the relationship between the observed and intrinsic 
data distributions is described by
\begin{equation}
P(\vartheta) = P_{\rm obs}(\vartheta) \times s(\vartheta)^{-1},
\end{equation}
where $P(\vartheta)$ and $P_{\rm obs}(\vartheta)$ represent the intrinsic and observed distributions of the
FRB property $\vartheta$, respectively. $s(\vartheta)$ is the selection function as a function
of different FRB properties. 
The properties considered for deriving the selection function include
the dispersion measure (${\rm DM}_{\rm obs}$), scattering timescale ($\tau_{\rm scat}$), 
intrinsic duration ($w_{\rm int}$) and fluence ($F_{\nu}$); see \citet{CHIMEFRB:2021srp} for details.
We adopt the best-fit selection functions in 
\citet{Hashimoto:2022llm},
\begin{eqnarray}
s({\rm DM}_{\rm obs}) = &-& 0.7707 (\log_{10}{\rm DM}_{\rm obs})^2 \nonumber\\
                        &+& 4.5601 (\log_{10}{\rm DM}_{\rm obs}) - 5.6291, \label{eq:seldm}\\
s(\tau_{\rm scat})    = &-& 0.2922 (\log_{10}\tau_{\rm scat})^2 \nonumber \\
                        &-& 1.0196 (\log_{10}\tau_{\rm scat}) + 1.4592, \label{eq:seltau}\\
s(w_{\rm int})        = &-& 0.0785 (\log_{10}w_{\rm int})^2 \nonumber \\
                        &-& 0.5435 (\log_{10}w_{\rm int}) + 0.9574, \label{eq:selw}\\
\log_{10} s(F_{\nu})  = &\,& 1.7173 (1 - \exp(-2.0348 \log_{10} F_\nu)) \nonumber \\ 
                        &-& 1.7173. \label{eq:selF}
\end{eqnarray}
The sky locations of the selected FRBs are shown in \reffg{fig:footprint} with the black circles.

%We use the catalog of contains a total of 536 bursts from 492 sources, including 474 non-repeaters and 18 repeaters from the first release of CHIME/FRB Catalog 1. Its observation frequency ranges from 300 MHz to 8 GHz and dispersion measure (DM) range from 100 to 2600. However, these data are selected by us, DM of data needs to meet the conditions of \( D M>100   \mathrm{pc}   \mathrm{cm}^{-3} \) and \( D M<1200  \mathrm{pc}  \mathrm{cm}^{-3} \), to make sure the sources are extragalactic and the Macquart relation \citep{Macquart:2020lln} between the dispersion measure and redshift (DM-z) is effective.


\subsection{Galaxy catalog}

%In the analysis of gravitational waves (GW), if the redshift information of the GW source could be obtained by identifying the electromagnetic (EM) counterparts, it is called the bright siren method. While for the GW events without EM counterparts, the statistical analysis of the GW event related to the galaxy catalog can also be applied in obtaining the redshift information. This kind of GW standard sirens is known as the dark sirens \citep{DelPozzo:2011vcw} (see Refs.~\citep{} for recent works). We develop a statistical method to measure the  redshift of FRBs using galaxy catalogs\citep{Zhao:2022yiv}. Thanks to the advanced radio equipment like the Canadian Hydrogen Intensity Mapping Experiment (CHIME)\footnote{https://www.chime-frb.ca/catalog}, the Australian Square Kilometre Array Pathfinder (ASKAP)\footnote{https://www.atnf.csiro.au/projects/askap/index.html}, Square Kilometer Array (SKA). More FRBs and their redshifts can be found through these instruments in the future. In this work, the first question to be answered is how to obtain the redshift of FRBs more accurately.%

In order to evaluate the redshifts of the unlocalized FRBs, we follow the
method developed in \citet{Zhao:2022yiv}, which actually employs the 
dark siren method in gravitational wave cosmology \citep{DelPozzo:2011vcw,Wang:2022oou,Song:2022siz}.
%{Jin:2023zhi,Jin:2022tdf,Jin:2022qnj,Wang:2021srv,Zhang:2019ylr,Zhang:2019ple,Zhao:2019gyk,Qiu:2021cww}. 
It assumes that an FRB always locates in a galaxy and the redshift of the 
FRB can be statistically estimated by associating the FRB event with its 
potential host galaxies according to a underlying galaxy catalogue.

In this work, we adopt the galaxy catalogue from 
Dark Energy Spectroscopic Instrument (DESI) Legacy Surveys.
The Legacy Surveys combine three imaging projects of different telescopes, i.e. 
the Beijing-Arizona Sky Survey \citep[BASS,][]{2017PASP..129f4101Z}, 
the Dark Energy Camera Legacy Survey \citep[DECaLS,][]{2015AJ....150..150F},
and the Mayall z-band Legacy Survey \citep[MzLS,][]{2016AAS...22831702S}, 
covering about $\num{14000}\,{\rm deg}^2$ of the northern
hemisphere and producing the target catalog for the DESI survey;
for an overview of the Legacy Surveys, see \citet{2019AJ....157..168D}.
We use the galaxy sample from the $8$-th public data release of
the Legacy Surveys, i.e., the Data Release $8$ (DR8).
The spectroscopic redshift of the galaxy sample is substituted for the photometric 
redshift, if available, in accordance with the sample selection process in
\citet{Yang:2020eeb}. In total, there are 129.35 million galaxies remaining in 
the galaxy sample. The footprint of the galaxy sample is illustrated with the green
area in \reffg{fig:footprint}, where the galaxies locating in the South Galactic Cap (SGC)
and the North Galactic Cap (NGC) are shown in the left and right panels, respectively.


%We use the catalogGC o f galaxy clusters\citep{Yang:2020eeb} based on the DESI Legacy Imaging Surveys DR8\citep{Yang:2004an} expansion which has a redshift ranging from 0 to 1. The sky coverage of galaxies is divided into the northern Galaxy cap (NGC) and the southern Galaxy cap (SGC), NGC covers about 8580 square degrees and has 59.6 million galaxies, SGC covers about 9673 square degrees with 69.75 million galaxies. It contains the basic properties of the distribution of richness, halo mass, and luminosity.





\section{Methods}\label{sec:method}

%\subsection{redshift estimation}
\subsection{Bayesian framework}
We adopt a Bayesian data analysis scheme to measure the FRBs' redshifts. 
The Bayesian inference relates the probability density functions 
(PDFs) involving data and parameters,
\begin{equation}
P(\vartheta | x) \propto P(\vartheta) P(x | \vartheta),
\end{equation}
where $P(x | \vartheta)$ is the {\it likelihood} function of the data given the
model parameters and $P(\vartheta | x)$ is the {\it posterior} PDF, i.e.
the PDF of the parameters given the data set.
In this work, we shall estimate the posterior PDF of the FRBs' redshifts $z$ 
given the measurement set of DM,
\begin{equation}
P(z|{\rm DM}) \propto P(z) P({\rm DM}|z),
\end{equation}
where $P(\dm | z)$ represents the likelihood function of the measured DM
given the parameter set.
The measured DM is the combination of several components, 
\begin{equation}
\dm = \dm_{\mw} + \dm_{\halo} + \dm_{\igm} + \frac{\dm_{\host}}{1 + z},
\end{equation}
where the contributions are from the interstellar medium of the Milky Way ($\dm_{\mw}$),
the ionized gas in the local halo ($\dm_{\halo}$), the intergalactic medium ($\dm_{\igm}$),
and the FRB host galaxy ($\dm_{\host}$). 

$\dm_{\mw}$ and $\dm_{\halo}$ can be subtracted according to the current model.
The CHIME/FRB catalogue provides the dispersion measure with the Milky Way contribution 
subtracted using the NE2001 model \citep{Cordes:2002wz} and the YMW16 model \citep{Yao_2017},
respectively. We test both of the two different models and find no significant difference 
in the final estimation. In the following analysis, only the results with the YMW16 model
are presented.

The precise contribution of the local halo to DM is uncertain. 
%with estimates of order 30-245 ${\rm {pc~cm^{-3}}}$
\cite{Yamasaki:2019htx} provides the prediction of $\dm_{\halo}$
with a mean value of $43\,{\rm pc}\,{\rm cm}^{-3}$ and a full
range of $30\sim245\,{\rm pc}\,{\rm cm}^{-3}$.
We adopt the mean value of $43\,{\rm pc}\,{\rm cm}^{-3}$ in the following
analysis.

After subtracting $\dm_{\mw}$ and $\dm_{\halo}$ for the total $\dm$, 
the measurement likelihood function is written as
\begin{eqnarray}\label{eq:likeli}
P(\dm|z) = \int &\dd&\dm_{\host}\,\dd\,\dm_{\igm}\,
          P(\dm|\dm_{\host},\dm_{\igm}, z) \nonumber \\
         &\times& P(\dm_{\host}|z) P(\dm_{\igm}| z),
\end{eqnarray}
where $P(\dm_{\host}|z)$ and $P(\dm_{\igm}|z)$ are the 
{likelihood} functions of $\dm_{\host}$ and $\dm_{\igm}$, respectively,
and the integration represents the marginalization  of
$\dm_{\host}$ and $\dm_{\igm}$ {likelihood} function.

\subsection{$P(\dm_{\igm}|z)$}
The DM contribution from the IGM ($\dm_{\igm}$) can be explained 
as the dispersion induced when an FRB is emitted at a random point in the
universe of redshift $z$ and propagates to $z=0$.
The average value of $\dm_{\igm}$ at redshift $z$ is given by the 
integration of the free electron number density $n_{\rm e}$ along the line of sight,
\begin{equation}
\langle \dm_{\igm} \rangle = \int_0^z \dd z^\prime \frac{n_{\rm e}(z^\prime)}{1+z^\prime}
\left(\frac{1}{1+z^\prime}\frac{c}{H_0}\frac{1}{E(z^\prime)}\right),
\end{equation}
where $H_0$ is the Hubble constant. In this work, we consider the standard flat $\Lambda$CDM model
$E(z)=\sqrt{\Omega_{\rm m}(1+z)^3 + \Omega_{\Lambda}}$.
Assuming the universe is fully ionized at $z\lesssim3$, the free electron
number density equals to the total electron number density, 
\begin{equation}
n_{\rm e}(z) = f_{\igm}\bar{\rho}_{{\rm b},0}(1+z)^3 
\left( \frac{Y_{\rm H}\chi_{{\rm e},{\rm H}}(z)}{m_{\rm p}} 
     + 2\frac{Y_{\rm He}\chi_{{\rm e},{\rm He}}(z)}{4m_{\rm p}}
\right),
\end{equation}
where the $Y_{\rm H}\sim3/4$ and $Y_{\rm He}\sim1/4$ denote the primordial mass 
fractions of hydrogen and helium, respectively. 
The ionization fraction, $\chi_{{\rm e}, {\rm H}}(z)$ and
$\chi_{{\rm e}, {\rm He}}(z)$ for hydrogen and helium, are both
set to unity at the redshift of $z\lesssim2$ \citep{Fan:2006dp,McQuinn:2008am}.
$\bar{\rho}_{{\rm b},0} = 3H_0^2\Omega_{\rm b}/8\pi G$ is the 
comoving cosmological baryon density at current epoch, $m_{\rm p}$ 
is the mass of proton and $f_{\igm}=0.83$ represents the fraction 
of the free electrons in the IGM \citep{Deng:2013aga}.

The $\dm_{\igm}$ deviation from $\langle\dm_{\igm}\rangle$ is 
expected to follow the normal distribution. Thus, the {likelihood}
function is expressed as
\begin{equation}\label{eq:ligm}
P(\dm_{\igm} | z ) = \frac{1}{\mathcal{N}_{{\rm IGM}}}\exp\left( -\frac{1}{2}\frac{\left( {\rm DM}_{\rm IGM} -\langle\mathrm{DM}_{\mathrm{IGM}}\rangle \right)^2}{\sigma^2_{{\rm IGM}}} \right),
\end{equation}
where $\mathcal{N}_{\igm}=\sigma_{\igm} \sqrt{2\pi}$ is the normalization factor
and $\sigma_{\igm} = 173.8\,z^{0.4}\,{\rm pc}\,{\rm cm}^{-3}$ 
 \citep{Qiang:2021bwb}.

\subsection{$P(\dm_{\host}|z)$}
The major uncertainty of the FRB redshift measurement comes from the 
variation of the DM contribution from the FRB's host galaxy.
A few of FRBs show large excess DM beyond the expectation from the
the cosmological and Milky Way contributions 
\citep{Spitler:2014fla,Chatterjee:2017dqg,
Hardy:2017djf,Tendulkar:2017vuq,2021ApJ...922..173C}. 
Recently, \citet{Niu:2021bnl} reported a detection of FRB 190520B 
with $\dm_{\host} \approx 903^{+72}_{-111}\,{\rm pc}\,{\rm cm}^{-3}$, 
which is almost an order
of magnitude higher than the average $\dm_\host$ of the FRBs discovered so far. Generally, the large spread of the $\dm_\host$ can be modeled 
using a lognormal distribution and the corresponding {likelihood}
function is expressed as
\begin{equation}\label{eq:lhost}
P(\dm_\host | z) = \frac{1}{{\mathcal N}_{\host}}
\exp\left( -\frac{1}{2} \frac{\left( \ln x - \mu \right)^2}{\sigma^2_{\host} }\right),
\end{equation}
where $x ={{\rm DM}_{\rm host}}\big/{{\rm pc}\,{\rm cm}^{-3}}$, 
${\mathcal N}_{\host}= x \sigma_{\host} \sqrt{2 \pi}$ is the
normalization factor, $\mu$ and $\sigma_\host$ are the lognormal distribution 
parameters. Using the cosmological magnetohydrodynamical simulation,
\citet{Zhang:2020mgq} (hereinafter referred to as the Zhang20 model) provided the fitting results,
\begin{equation}
\mu = \ln \left(32.97 (1 + z) ^{0.84}\right),\,\,\sigma_\host=1.248.
\end{equation}
In addition, \cite{Mo:2022qxz} proposed another detailed analysis for the distribution 
of the $\dm_\host$ for different FRB population models. In this work, we adopt the
fitting results of
\begin{equation}
\mu = \ln (63.55),\,\,\sigma_\host=1.25,
\end{equation}
from \cite{Mo:2022qxz} (hereinafter referred to as the Mo22 model). We compare the 
result differences between using such two $\dm_\host$ distribution models,
as well as using the model of assuming constant $\dm_\host=50\,{\rm pc}\,{\rm cm}^{-3}$. 

\subsection{The posterior distribution of the FRB redshift}

\begin{figure}
    \centering
    \includegraphics[width=0.46\textwidth]{plot_likeli_galcat_onceoff145}
    \caption{
    The redshift posterior probability distribution of each FRB event.
    Each curve represents the posterior probability of one FRB using the
    DESI Legacy Surveys DR8 galaxies' redshift sample that located in 
    the sky area determined by \refeq{eq:beam}.
    The colors of the lines indicate the values of 
    ${\rm DM}_{\rm IGM}+{\rm DM}_{\rm host}$.
    }\label{fig:likeli}
\end{figure}


The DM measurement {likelihood} function is expressed as
\begin{equation}\label{eq:ldm}
P(\dm | \dm_\host, \dm_\igm, z) = \frac{1}{\mathcal{N}} 
\exp\left(-\frac{1}{2}\frac{(\dm - \Theta)^2}{\sigma^2_{\dm}}\right),
\end{equation}
where $\mathcal{N} = \sigma_\dm \sqrt{2\pi}$ is the normalization factor,
$\sigma_\dm$ is the measurement uncertainty and 
$\Theta=\dm_\host + \dm_\igm + \dm_\mw + \dm_\halo$ represents the
DM's theoretical value.
Substituting Equations (\ref{eq:ligm}), (\ref{eq:lhost}) and (\ref{eq:ldm}) 
into \refeq{eq:likeli}, we can estimate the posterior probability at a given redshift.
Assuming the FRBs always locate in the galaxies, we shall use the 
redshifts of the galaxy catalog, i.e. the DESI Legacy Surveys DR8 catalog, 
as the prior distribution. 
For a given FRB, we use the redshifts of all the galaxies with their
celestial coordinate $\alpha$ (Right ascension) and $\delta$ (Declination),
\begin{eqnarray}\label{eq:beam}
| \alpha_{\rm gal} - \alpha_{\rm FRB} | < \theta_\alpha \times \cos(\delta_{\rm FRB}), \,\,\,
| \delta_{\rm gal} - \delta_{\rm FRB} | < \theta_\delta,
\end{eqnarray}
where $\theta_\alpha$ and $\theta_\delta$ are the $68\%$ confidence 
pointing errors of the CHIME beam in the Right ascension and Declination
direction, respectively.
The estimated redshift posterior probabilities are shown in \reffg{fig:likeli}.
Each curve represents the posterior probability of one FRB using the
DESI Legacy Surveys DR8 galaxies' redshift sample locating in 
the sky area determined by \refeq{eq:beam}.
The color of the curve indicates the value of $\dm_\igm + \dm_\host$ of the FRB event.
There is a clear trend that the FRB with larger DM has the 
posterior probability distribution peaking at higher redshift.
This is consistent with the Macquart relation \citep{Macquart:2020}.
However, the posterior probability also shows a wide range of 
distribution, which indicates the large uncertainty of the redshift estimation.
Such large uncertainty is dominated by the large scattering of $\dm_\host$.

\subsection{Energy function}

The FRB fluence ($F_\nu$) is converted to rest-frame isotropic radio energy ($E$) for each FRB via
\citep{Macquart:2018jlq}
\begin{equation}\label{eq:rho}
E = \frac{4\pi d_{\rm L}^2}{(1+z)^{2+\alpha}} F_{\nu} \Delta\nu,
\end{equation}
where $d_{\rm L}$ is the luminosity distance to the FRB, 
$\alpha$ is the spectrum index of 
the FRB's power-law spectrum across the frequencies and $\Delta \nu$ is the 
burst bandwidth. The burst bandwidth is calculated by 
{\tt high\_freq}$-${\tt low\_freq} in the CHIME/FRB catalog, 
where {\tt high\_freq} and {\tt low\_freq} represent the upper and lower bands
of the detection at full-width tenth-maximum (FWTM) \citep{CHIMEFRB:2021srp}. 

The FRB energy function represents the number density of FRB events as a function of
energy. The number density per unit time is estimated via each of the FRB event detection,
\begin{equation}
{\rho}_{\rm obs} = \frac{1}{V_{\rm max} f_{\rm sky} \left( t_{\rm obs}/(1+z) \right)},
\end{equation}
where $f_{\rm sky}=3\times10^{-3}$ is the fraction of the sky covered by the
CHIME's field of view, $t_{\rm obs}=0.59\,{\rm yr}$ is the survey time for
the first CHIME/FRB catalog and the factor of $1+z$ converts the survey time to the rest frame.
$V_{\rm max}$ is defined as the flux limited maximum volume within which 
the FRB event could still be detected \citep{Schmidt:1968kn,1980ApJ...235..694A},
\begin{equation}
V_{\rm max} = \frac{4\pi}{3}\left(\chi_{\rm max}^3 - \chi_{\rm min}^3\right),
\end{equation}
where $\chi_{\rm min}$ and $\chi_{\rm max}$ are the comoving distances at the 
minimum and maximum redshifts. We adopt $z_{\rm min}=0.05$ in this work and 
$z_{\rm max}$ is estimated according the fluence of the FRB,
\begin{equation}
F_{\nu} = \frac{E(1+z_{\rm max})^{2+\alpha}}{4\pi d_{{\rm L},z_{\rm max}}^2\Delta \nu } 
> 10^{0.5}\,{\rm Jy}\,{\rm ms}.
\end{equation}

The number density estimated with \refeq{eq:rho} needs to be corrected for the selection
effect as mentioned in \refsc{sec:frbcat}. Considering the selection function of
Equations~(\ref{eq:seldm})--(\ref{eq:selF}), the corrected number density is
\citep{Hashimoto:2022llm}
\begin{equation}
\rho_{\rm corr} = \frac{1}{\mathcal{N}_{s}}W\rho_{\rm obs},
\end{equation}
where $W=\big( s(\dm_{\rm obs}) s(\tau_{\rm scat}) s(w_{\rm int}) s(F_{\nu}) \big)^{-1}$. 
$\mathcal{N}_s = \sum_i W_i / {N_{\rm FRB}}$ is the normalization
factor, where $i$ denotes the $i$-th FRB event and $N_{\rm FRB}$ is the corrected total number, i.e., \refeq{eq:selnum}.

We divide the full energy range occupied by the FRB detection into a number of 
energy bins ($\phi_j$) in the logarithmic scale and sum $\rho_{\rm corr}$ within
each energy bin,
\begin{equation}
\phi_{j} = \frac{1}{\Delta_j \log_{10} E}\sum_i \rho_{{\rm corr}, i},
\end{equation}
where $\Delta_j \log_{10} E$ is the $j$-th energy bin size.

We perform a Monte Carlo (MC) simulation with $\num{10000}$ realizations of the
redshift sample following the posterior probability distribution of 
\refeq{eq:likeli}. The energy function is estimated using each of the realization.
The estimation uncertainty is evaluated via the the standard deviation.

%a “normalized” Schechter type is selected as described in the Equation (10),
%\begin{equation}
%P(E)dE=\rho\frac{1}{E_{\text {char }}}\left(\frac{E}{E_{\text {char }}}\right)^\gamma \exp \left[-\frac{E}{E_{\text {char }}}\right]dE
%\end{equation}
%where$ \rho $ is a normalization coefficient, E is the radio energy, $E_{\text {pivot }}$ is the pivot energy of FRBs. $E_{\text {char }}$ is the cut-off energy of FRBs. $\gamma$ is the power-law index. \cite{Shin:2022crt} used the Schechter function to infer a characteristic energy cut-off of $\rm{E_{char}}=2.38_{-1.64}^{+5.35}$×$10^{41}$, power-law index of $\gamma=-1.3_{-0.4}^{+0.7}$ and FRBs event rate of $\left.\left[7.3_{-3.8}^{+8.8} \text { (stat.) }{ }_{-1.8}^{+2.0} \text { (sys. }\right)\right] \times 10^4$Gpc$^{-3}$ year$^{-1}$, the above three parameters are fitted to our energy distribution, as is shown in Fig.~\ref{fig:2}. We infer a characteristic energy cut-off of $\rm{E_{char}}$=1.01×$10^{41}$ erg, a power-law index of ${\gamma}$= -0.86 and a volumetric rate 5.75×$10^{4}$ of bursts Gpc$^{-3}$ year$^{-1}$above a pivot energy of $10^{39}$ erg. This result is basically consistent with their research results.
%

\begin{figure*}
\begin{minipage}[t]{0.49\textwidth}
\centering
\includegraphics[width=\textwidth]{plot_EF_onceoff145_ymw16_dmMo.pdf}
\caption{The energy functions  of non-repeating  CHIME FRB sources using Mo22 model. 
The results for three redshift bins are shown with three different colors.
The best-fit Schechter functions are shown with solid lines.}\label{fig:efMo}                             
\end{minipage} \hfill
\begin{minipage}[t]{0.49\textwidth}
\centering
%\includegraphics[width=\textwidth]{plot_EF_onceoff145_ymw16_fixdm.pdf}
\includegraphics[width=\textwidth]{plot_EF_onceoff145_ymw16_dmZhang.pdf}
\caption{The energy functions  of non-repeating  CHIME FRB sources using Zhang20 Model. The results for three redshift bins are shown with three different colors.
The best-fit Schechter functions are shown with solid lines.}\label{fig:efZhang}                             
\end{minipage}
\end{figure*}

\begin{figure}
\centering
\includegraphics[width=0.49\textwidth]{plot_frbrate.pdf}
\caption{
The volumetric rate of CHIME non-repeating FRBs as a function
of redshift. The horizontal errors represent the redshift bin width and the vertical errors are the estimation
uncertainty, which are estimated by 10000 times of the MC simulation. 
The results of the constant $\dm_\host$ model, the Zhang20 model, and the Mo22 model are shown with different colors.
%The blue errorbars shows the result by assuming constant $\dm_\host$, while the orange and green errorbars show the results using $\dm_\host$ followingZhang20 Model and Mo22 Model, respectively.
}\label{fig:frbrate}
\end{figure}

\section{Results and discussions}\label{Results}

Figures \ref{fig:efMo} and \ref{fig:efZhang} show the energy functions estimated using the 
CHIME/FRB catalog. The energy functions are estimated within three redshift bins,
i.e. $0.05<z \leqslant 0.3$, $0.3<z \leqslant 0.68$ and $0.68<z \leqslant 1.38$ (follow Ref.~\citep{Hashimoto:2022llm}).
The estimated energy distribution functions of different redshift bins are shown 
in different colors.
The results with different $\dm_\host$ models, i.e. the Mo22 model and the Zhang20 model, 
 are shown in Figures \ref{fig:efMo} and \ref{fig:efZhang}, respectively. 

The FRB energy distribution is modeled with a Schechter function \citep{1976ApJ...203..297S},
\begin{align}\label{eq:ef}
\phi(\lg E)\,{\rm d}\lg E = \phi^\star \left(\frac{E}{E^\star}\right)^{\gamma + 1}
\exp\left(-\frac{E}{E^\star}\right) {\rm d} \lg E,
\end{align}
where $\phi^\star$ is the normalization factor, $\gamma + 1$ is the faint-end slope
and $E^\star$ is the break energy of the Schechter function. 
The energy distribution functions are fit to the measurements using 
the public package {\tt emcee} 
\footnote{\url{https://emcee.readthedocs.io/en/stable/index.html}}
\citep{2013PASP..125..306F}.
We use $\phi^\star$, $E^\star$ and $\gamma$ as the free parameters for
the first redshift bin, i.e. $0.05 < z < 0.30$.
Due to the lack of FRB data in the higher reshift bins, 
we only use $\phi^\star$ and $E^\star$ as the free parameters in
the rest two redshift bins and fix $\gamma$ to the best-fit value of the
first redshift bin.

\begin{table}
\begin{center}
{\scriptsize
%\centering
\caption{The fit values of parameters in Schechter function for FRB energy function.}\label{tab:bestfit}
\begin{tabular}{ccccc} \hline\hline
& $\lg E^\star$ & $\lg \phi^\star$ & $\gamma$ & $\lg \Phi$ \\ \hline
\multicolumn{5}{c}{$0.05 < z < 0.30$} \\ \hline
Constant $\dm_\host$  & $40.000_{-0.363}^{+0.428}$ & $3.994_{-0.626}^{+0.254}$ & $-1.046_{-0.373}^{+0.516}$ & $4.277_{-0.301}^{+0.139}$ \\
Zhang20 Model         & $40.219_{-0.378}^{+0.592}$ & $3.418_{-0.877}^{+0.358}$ & $-1.382_{-0.224}^{+0.572}$ & $4.035_{-0.192}^{+0.125}$ \\
Mo22 Model            & $40.225_{-0.382}^{+0.589}$ & $3.783_{-0.882}^{+0.369}$ & $-1.427_{-0.238}^{+0.537}$ & $4.438_{-0.181}^{+0.121}$ \\
\hline
\multicolumn{5}{c}{$0.30 < z < 0.68$} \\ \hline
Constant $\dm_\host$  & $40.239_{-0.119}^{+0.169}$ & $4.097_{-0.222}^{+0.164}$ & -- & $4.434_{-0.167}^{+0.129}$ \\
Zhang20 Model         & $40.503_{-0.150}^{+0.177}$ & $3.370_{-0.226}^{+0.207}$ & -- & $4.321_{-0.116}^{+0.113}$ \\
Mo22 Model            & $40.496_{-0.153}^{+0.183}$ & $3.729_{-0.235}^{+0.211}$ & -- & $4.678_{-0.123}^{+0.111}$ \\
\hline
\multicolumn{5}{c}{$0.68 < z < 1.38$} \\ \hline
Constant $\dm_\host$  & $40.408_{-0.451}^{+0.216}$ & $3.772_{-0.847}^{+0.336}$ & -- & $4.167_{-1.033}^{+0.263}$ \\
Zhang20 Model         & $40.553_{-1.549}^{+0.461}$ & $2.970_{-0.470}^{+1.640}$ & -- & $3.956_{-1.869}^{+0.794}$ \\
Mo22 Model            & $40.547_{-1.519}^{+0.534}$ & $3.295_{-0.003}^{+2.204}$ & -- & $4.280_{-1.992}^{+0.813}$ \\ \hline
\end{tabular}}
\end{center}
\end{table}
     
The best-fit energy distribution function at each redshift bin 
is shown with the smooth curve in \reffg{fig:efMo} and \reffg{fig:efZhang},
respectively. The energy function estimated in each redshift bin is consistent
with each other. There is no significant redshift evolution for using either 
the Mo22 model or the Zhang20 model. The fit values of the parameters in 
\refeq{eq:ef} are list in \reftb{tab:bestfit}. The volumetric rate of the FRBs
is estimated by integrating the energy function within the available energy 
range of each redshift bin.
%\begin{align}\label{eq:frbrate}
%\Phi = \int_{E_{\rm min}}^{E_{\rm max}} \phi(E) {\rm d} \ln E,
%\end{align}
The corresponding volumetric rates estimated using the best-fit energy functions
are list in the last column of \reftb{tab:bestfit} and also shown in \reffg{fig:frbrate}.
In \reffg{fig:frbrate}, the blue error bars show the FRB volumetric rates estimated
by assuming a constant $\dm_\host=50\,{\rm pc}\,{\rm cm}^{-3}$, while the 
orange and green error bars show the results using $\dm_\host$ from the Zhang20 model and the Mo22 model, 
respectively.
The horizontal error bars indicate the redshift bin width and the vertical error bars show the
standard deviation. 
The solid black and gray curves show the cases of the star formation rate and the stellar-mass density,
which are estimated using the fitting functions in \cite{Hashimoto:2020acj}.
Both of the two curves are normalized for their amplitudes at redshift $z=0.2$ to the same value 
as the FRB volumetric rates estimated using the constant $\dm_\host$.

With the constant $\dm_\host$ assumption, the FRB volumetric rate shows the same
trend as the stellar-mass density, which is consistent with the previous analysis
\citep{Hashimoto:2022llm}. By releasing the constant $\dm_\host$ assumption, 
the estimation uncertainty increases, especially for the high redshift bin. 
Within the estimation error, there is no significant difference between 
using and not using the constant $\dm_\host$ assumption. 
However, it can be obviously seen that the variation of $\dm_\host$ weakens the conclusion that the volumetric rate is consistent with the stellar-mass density. It is expected that the future much larger FRB and galaxy samples could greatly improve the measurement and draw a more solid conclusion.


%It means that the variation of $\dm_\host$ do not have significant affect on the FRB volumetric rate prediction. Such measurements can be improved in the future with even large FRB catalogue and deeper galaxy catalogue.  

\section{Conclusions}\label{ccc}

In this work, we estimate the energy function and the volumetric rate of the non-repeating
FRBs using CHIME/FRB Catalog 1. We follow the FRB selection criteria as used in the 
literature \citep{Shin:2022crt,Hashimoto:2022llm}. In the meanwhile, we follow the
Bayesian framework data analysis scheme developed in \cite{Zhao:2022yiv} and adopt the
galaxy catalogue from Dark Energy Spectroscopic Instrument (DESI) Legacy Surveys 
to evaluate the redshift of the unlocalized FRBs.

We also consider different $\dm_\host$ models, including the model of constant 
$\dm_\host = 50\,{\rm pc}\,{\rm cm}^{-3}$ and the lognormal distribution model,
which are constrained using the magnetohydrodynamical simulation. 
\citet{Zhang:2020mgq} and \citet{Mo:2022qxz} both provided the fitting functions according to
magnetohydrodynamical simulation (the Zhang20 and Mo22 models). The FRB energy function is estimated with each of the $\dm_\host$ model.

The Schechter function-like energy function model is considered and is fit to the
measurements using the non-repeating FRBs from CHIME/FRB Catalog 1. 
The fit values of parameters are summarized in \reftb{tab:bestfit}.
We do not find significant difference between using the constant $\dm_\host$ model
and the lognormal $\dm_\host$ models (i.e. the Zhang20 and Mo22 models).

We also estimate the FRB volumetric rate according to the best-fit energy 
distribution function and compare the trend of redshift evolution with the
star formation-rate density and the stellar-mass density. 
We find that, with the lognormal $\dm_\host$ model, the estimation uncertainties
increase. The trend of redshift evolution is consistent with the 
stellar-mass density for both the constant $\dm_\host$ model and the lognormal 
$\dm_\host$ models.
However, since the lognormal distributed $\dm_\host$ model increases the measurement errors,
the inference of FRBs tracking the stellar-mass density is nonetheless undermined.
The measurement can be further improved in the future by using a larger FRB catalog and/or
a deeper galaxy survey catalogue.


\section*{Acknowledgments}
We thank Chenhui Niu and Yuhao Zhu for helpful discussions and suggestions.
We are grateful for the support by the National SKA Program of China (Grants Nos. 2022SKA0110200 and 2022SKA0110203)
and the National Natural Science Foundation of China (Grants Nos. 11975072, 11875102, and 11835009).

\bibliography{FRBarXiv}{}
\bibliographystyle{aasjournal}

\end{document}

% So far, the total FRB population now stands at over 802 published sources, among them, only twenty-four FRB have been localized.