\section{Introduction}

\begin{figure}[!t]
\includegraphics[width=\textwidth]{fig/teaser.pdf}
\caption{Blind inpainting model vs. Non-blind inpainting model. The blind inpainting model is capable of effectively restoring corrupted images without the need for mask annotation.}
\label{fig1}
\end{figure}
Recent advancements in artificial intelligence (AI) techniques have led to an increased interest in AI-based diagnostics among researchers~\cite{shen2019artificial}.
Among them, medical imaging plays a pivotal role in the healthcare sector for diagnosis and treatment~\cite{currie2019machine}. 
However, medical images often contain artificial markers added by doctors to highlight lesion regions, as shown in Fig.~\ref{fig1}. While these markers are useful for human diagnosis, they can negatively impact deep learning-based lesion detectors and classifiers. Therefore, there is a critical need to reconstruct clean images by removing such markers and restoring missing visual content, especially if there is a large amount of historical unclean data. 
Inpainting methods are commonly used for image completion~\cite{elharrouss2020image},
and over the past decade, numerous studies in the literature have focused on developing efficient and robust inpainting methods, including GAN-based~\cite{zheng2022image,liu2021pd}, gated convolution-based~\cite{yu2019free}, fourier-based~\cite{suvorov2022resolution}, transformer-based~\cite{li2022mat,dong2022incremental,liu2022reduce} methods, diffusion-based~\cite{wang2022zero,lugmayr2022repaint,liu2022pseudo,cao2023difffashion}, \textit{etc}. 
Inpainting is also widely employed in medical imaging.
Dong et al.~\cite{dong2013x} propose a simultaneous X-Ray CT image reconstruction and Radon domain inpainting model using wavelet frame based regularization. Belli et al.~\cite{belli2018context} demonstrate how the context encoder architecture under adversarial training can effectively be used for inpainting in chest X-ray images. IpA-MedGAN~\cite{armanious2020ipa} is illustrated to perform well for the inpainting of brain MR images. Rouzrokh et al.~\cite{rouzrokh2022multitask} present a leveraging diffusion models for multitask brain tumor inpainting.\\

\indent However, recovering corrupted regions of artificial markers in images often requires manually drawing a mask as input to the inpainting model. This can be inconvenient, time-consuming, and error-prone, which limits its practicality. 
Blind inpainting methods~\cite{liu2019deep} offer a more suitable solution as they do not require additional masks.
For example, combining sparse coding and deep neural networks pre-trained with denoising auto-encoders, Xie et al.~\cite{xie2012image} use a non-linear approach to tackle the problem of blind inpainting of complex patterns. Afonso et al.~\cite{afonso2015blind} present an iterative method based on alternating minimization. 
Based on a fully convolutional neural network, BICNN~\cite{cai2017blind} directly learns an end-to-end mapping between corrupted and ground truth pairs. VC-Net~\cite{wang2020vcnet} performs well against unseen degradation patterns with sequentially connected mask prediction and inpainting networks. However, existing works still have difficulty in localizing the corrupted regions, leading to sub-optimal solutions in the image completion.\\

\indent In this work, we focus on solving this challenging blind inpainting task by designing an efficient network that relaxes the restriction of mask as well as maintaining a good performance. We propose a novel end-to-end blind inpainting framework with mask-free reconstruction and object-aware discrimination. 
With two branches predicting corrupted regions and recovering missing visual content, and an object-aware discriminator for adversarial training, our approach produces reconstructions closely resembling clean images. \\

\indent In summary, this paper makes the following contributions: 1) We propose a novel blind inpainting network for artificial marker removal in medical images. 2) We propose a two-branch reconstruction network for blind inpainting, which can simultaneously detect artificial markers and recover the missing visual contents. 3) We employ the object-aware discrimination by a dense object detector to ensure the reconstructed images are close to clean ones. 4) Our method outperforms other existing blind inpainting methods with a large margin on several medical image datasets with variant modalities, such as ultrasound (US), magnetic resonance imaging (MRI), and electron microscopy (EM), demonstrating the effectiveness of our method.


