\begin{abstract}
Medical images often incorporate doctor-added markers that can hinder AI-based diagnosis. This issue highlights the need of inpainting techniques to restore the corrupted visual contents. However, existing methods require manual mask annotation as input, limiting the application scenarios. In this paper, we propose a novel \textbf{blind inpainting} method that automatically reconstructs visual contents within the corrupted regions without mask input as guidance. Our model includes a blind reconstruction network and an object-aware discriminator for adversarial training. The reconstruction network contains two branches that predict corrupted regions in images and simultaneously restore the missing visual contents. Leveraging the potent recognition capability of a dense object detector, the object-aware discriminator ensures markers undetectable after inpainting. Thus, the restored images closely resemble the clean ones. We evaluate our method on three datasets of various medical imaging modalities, confirming better performance over other state-of-the-art methods.

%Medical images often incorporate doctor-added markers that can hinder AI-based diagnosis. This issue highlights the need of inpainting techniques to restore the corrupted visual contents. However, existing inpainting methods require manual mask annotation as an input, limiting the application scenarios. In this paper, we propose a novel \textbf{blind inpainting} method that can automatically reconstruct visual contents within the corrupted regions without a mask input as guidance. Our model includes a blind reconstruction network and an object-aware discriminator for adversarial training. The reconstruction network consists of two branches that predict corrupted regions in an image and simultaneously restore the missing visual contents. Leveraging the potent recognition capability of a dense object detector, the object-aware discriminator ensures markers undetectable in any local regions after inpainting. As a result, the restored images can closely resemble the clean ones as much as possible. We evaluate our method on three datasets of different medical imaging modalities, and experimental results confirm it achieves better performance over other state-of-the-art methods.

\end{abstract}
\begin{keywords}
Blind image inpainting, generative adversarial networks, image reconstruction, dense object detector
\end{keywords}