\section{Introduction} \label{sec:intro}

Face recognition systems are omnipresent nowadays. Their applications range from classical use cases such as access control of smartphones to newer ones such as tagging a picture by identity or even controlling the output of generative models~\cite{smoothswap, simswap, latent_space_mapping}. The goal of a face recognition method $f$ is to obtain embeddings $\mb{y}$ of face images $\mb{x}$ such that the embeddings of images of the same person are closer to each other than those of images of other people. We refer to this embedding $\mb{y}$ as the \emph{identity vector} or \emph{ID vector}. In this paper, we propose a technique to sample from $p(\mb{x}|\mb{y})$, \ie to produce realistic face images from an ID vector.

By design, the many-to-one mapping of face recognition methods assigns multiple images of a given identity to the same identity representation. The inverse one-to-many problem, \ie producing a high-dimensional image from a low-dimensional ID vector, is extremely challenging. Previous methods often rely on the gradient of face recognition models either directly~\cite{DBLP:journals/corr/ZhmoginovS16} or use it during training in the form of a loss function~\cite{cole, latent_space_mapping}. This gradient or information about the model's architecture and weights is often not available, \eg if using an API of a proprietary model. We therefore focus on the more generally applicable \emph{black-box} setting, where only the resulting ID vectors are available. In addition to being more general, the black-box setting has the additional benefit that we can easily extend our conditioning mechanism to include information from different, even non-differentiable sources (\eg labels, biological signals).

We propose the \underline{id}entity \underline{d}enoising \underline{d}iffusion \underline{p}robabilistic \underline{m}odel (ID3PM), the first method that uses a diffusion model to invert the latent space of a face recognition model, \ie to generate highly realistic, identity-preserving face images conditioned solely on black-box ID vectors as seen in \cref{fig:teaser_small}. We show mathematically that we can effectively invert a model $f$ even without having access to its gradients by using a conditional diffusion model. This allows us to train our method with an easy-to-obtain data set of pairs of images and corresponding ID vectors without an identity-specific loss term.

\begin{figure}[htpb]
    \centering
    \includegraphics[trim = 20mm 30mm 209mm 16mm, clip, width=0.7\linewidth]{images/teaser}
    \caption{Overview. Our method inverts a pre-trained face recognition model (here InsightFace~\cite{insightface}) to produce high-quality identity-preserving images. We are the first black-box method to provide intuitive control over the image generation process.}
    \vspace{-0.5cm}
    \label{fig:teaser_small}
\end{figure}

Our method demonstrates state-of-the-art performance for the inversion task and also allows for controllability over the generation process as seen in \cref{fig:teaser_small}. To the best of our knowledge, our method is the first black-box face recognition model inversion method to do so. Specifically, we can control (1) the diversity among samples generated from the same ID vector via the classifier-free guidance scale, (2) identity-specific features (\eg age) via smooth transitions in the ID vector latent space, which also enables identity interpolations, and (3) identity-agnostic features (\eg pose) via explicit attribute conditioning.

To summarize, our main contributions are:
\begin{enumerate}[itemsep=1pt,topsep=1pt]
    \item Showing that the conditional diffusion model loss naturally emerges from an analysis of the black-box inversion problem.
    \item Applying the resulting framework to invert face recognition models without identity-specific loss functions. 
    \item Demonstrating state-of-the-art performance in generating diverse, identity-preserving face images from black-box ID vectors.
    \item Providing control mechanisms for the face recognition model inversion task.
\end{enumerate}
