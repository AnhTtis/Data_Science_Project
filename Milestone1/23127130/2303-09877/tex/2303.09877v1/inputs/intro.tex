
\begin{figure}
    \centering
    \includegraphics[width=\columnwidth]{fig/fig-main}
    \caption{Overview of the \fwName framework for a two-view dataset. Different colors denote different components. The framework is generalizable to an arbitrary number of views by adding more view specific encoders (\( f \)) and SV-SSL blocks.}
    \label{fig:framework}
\end{figure}

Multi-view clustering (MVC) generalizes the clustering task to data where the instances to be clustered are observed through multiple views, or by multiple modalities.
In recent years, deep learning architectures have seen widespread adoption in MVC, resulting in the \emph{deep MVC} subfield.
Methods developed within this subfield have shown state-of-the-art clustering performance on several multi-view datasets~\cite{zhouEndtoEndAdversarialAttentionNetwork2020,trostenReconsideringRepresentationAlignment2021,xuMultiVAELearningDisentangled2021,maoDeepMutualInformation2021,wangAdversarialMultiviewClustering2022,wangSelfSupervisedInformationBottleneck2022}, largely outperforming traditional, non-deep-learning-based methods~\cite{zhouEndtoEndAdversarialAttentionNetwork2020}.

Despite these promising developments, we identify significant drawbacks with the current state of the field.
Self-supervised learning (SSL) is a crucial component in many recent methods for deep MVC~\cite{zhouEndtoEndAdversarialAttentionNetwork2020,trostenReconsideringRepresentationAlignment2021,xuMultiVAELearningDisentangled2021,maoDeepMutualInformation2021,wangAdversarialMultiviewClustering2022,wangSelfSupervisedInformationBottleneck2022}.
However, the large number of methods, all with unique components and arguments about how they work, makes it challenging to identify clear directions and trends in the development of new components and methods.
Methodological research in deep MVC thus lacks foundation and consistent directions for future advancements.
This effect is amplified by large variations in implementation and evaluation of new methods.
Architectures, data preprocessing and data splits, hyperparameter search strategies, evaluation metrics, and model selection strategies all vary greatly across publications, making it difficult to properly compare methods from different papers.
To address these challenges, we present a unified framework for deep MVC, coupled with a rigorous and consistent evaluation protocol, and an open-source implementation.
Our main contributions are summarized as follows:

\customparagraph{(1)~~\fwName framework.}
    Despite the variations in the development of new methods, we recognize that the majority of recent methods for deep MVC can be decomposed into the following fixed set of components:
    \begin{enumerate*}[label=(\roman*)]
        \item view-specific encoders;
        \item single-view SSL;
        \item multi-view SSL;
        \item fusion; and
        \item clustering module.
    \end{enumerate*}
    The \fwName framework (Figure~\ref{fig:framework}) is obtained by organizing these components into a unified deep MVC model.
    Methods from previous work can thus be regarded as \emph{instances} of \fwName.

\customparagraph{(2)~~Theoretical insight on alignment and number of views.}
    Contrastive alignment of view-specific representations is an MV-SSL component that has demonstrated state-of-the-art performance in deep MVC~\cite{trostenReconsideringRepresentationAlignment2021}.
    We study a simplified case of deep MVC, and find that contrastive alignment can only decrease the number of separable clusters in the representation space.
    Furthermore, we show that this potential negative effect of contrastive alignment becomes worse when the number of views in the dataset increases.

\customparagraph{(3)~~New instances of \fwName.}
    Inspired by initial findings from the \fwName framework, and our theoretical findings on contrastive alignment, we develop \( 6 \) new instances of \fwName, which outperform current state-of-the-art methods on several multi-view datasets.
    The new instances include both novel and well-known types of self-supervision, fusion and clustering modules.

\customparagraph{(4)~~Open-source implementation of \fwName and evaluation protocol.}
    We provide an open-source implementation of \fwName, including several recent methods and our new instances.
    The implementation includes a shared evaluation protocol for all methods, and all datasets used in the experimental evaluation.
    By making the datasets and our implementation openly available, we aim to facilitate simpler development of new methods, as well as rigorous and accurate comparisons between methods and components.

\customparagraph{(5)~~Evaluation of methods and components.}
    We use the implementation of \fwName to evaluate and compare several recent state-of-the-art methods and SSL components -- both against each other, and against our new instances.
    In our experiments, we both provide a consistent evaluation of methods in deep MVC, and systematically analyze several SSL-based components -- revealing how they behave under different experimental settings.

\customparagraph{The main findings from our work are:}
    \begin{itemize}[topsep=0pt, noitemsep, leftmargin=*]
        \item We show that aligning view-specific representations can have a negative impact on cluster separability, especially when the number of views becomes large.
            In our experiments, we find that contrastive alignment of view-specific representations works well for datasets with few views, but \emph{significantly degrades performance when the number of views increases}.
            Conversely, we find that maximization of mutual information performs well with many views, while not being as strong with fewer views.
        \item All methods included in our experiments benefit from at least one form of SSL.
            In addition to contrastive alignment for few views and mutual information maximization for many views, we find that autoencoder-style reconstruction improves overall performance of methods.
        \item Properties of the datasets, such as class (im)balance and the number of views, heavily impact the performance of current MVC approaches.
            There is thus not a single ``state-of-the-art'' -- it instead depends on the datasets considered.
        \item Results reported by the original authors differ significantly from the performance of our re-implementation for some baseline methods, illustrating the necessity of a unified framework with a consistent evaluation protocol.
    \end{itemize}
