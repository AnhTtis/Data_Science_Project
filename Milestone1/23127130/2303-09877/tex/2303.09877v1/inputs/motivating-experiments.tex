
As can be seen in Table~\ref{tab:previousMethods}, SSL components are crucial in recent state-of-the-art methods for deep MVC.
Recent works have focused on aligning view-specific representations~\cite{zhouEndtoEndAdversarialAttentionNetwork2020,trostenReconsideringRepresentationAlignment2021}, and in particular, contrastive alignment~\cite{trostenReconsideringRepresentationAlignment2021}.
We study a simplified setting where, for each view, all observations in a cluster are located at the same point.
This allows us to prove that aligning view-specific representations has a negative impact on the cluster separability after fusion.
This is the same starting point as in~\cite{trostenReconsideringRepresentationAlignment2021}, but we extend the analysis to investigate contrastive alignment when the number of views increases.

\def\propositionMinSeparbleClusters{
    \begin{proposition}[Adapted from~\cite{trostenReconsideringRepresentationAlignment2021}]
        \label{prop:prop1}
        Suppose the dataset consists of \( n \) instances, \( V \) views, and \( k \) ground-truth clusters, and that view-specific representations are computed with view-specific encoders as \( \vec z^{(v)}_i = f^{(v)}(\vec x^{(v)}_i) \).
        Furthermore, assume that:
        \begin{itemize}
            \item For all \( v \in \{1, \dots V\} \) and \( j \in \{1, \dots, k \} \),
                \begin{align}
                    \forall i \in \cl C_j, \vec x^{(v)}_i = \vec c^{(v)} \in \{ \vec c_1^{(v)}, \dots, \vec c_{k_v}^{(v)} \}
                \end{align}
                where \( \cl C_j \) is the set of indices for instances in cluster \( j \), and \( k_v \in \{ 1, \dots, k \} \) is the number of separable clusters in view \( v \).
            \item Representations are fused as \( \vec z_i = \sum_{v=1}^{V} w_v \vec z_i^{(v)} \) where \( w_1, \dots, w_V \) are all unique.
            \item For all \( j \in \{1, \dots, k \} \),
                \begin{align}
                    \forall i \in \cl C_j, \vec z_i = \vec z^{\star} \in \{ \vec z_1^\star, \dots, \vec z_\kappa^\star \}
                \end{align}
        \end{itemize}
        Then if \( \vec z_i^{(1)} = \dots = \vec z_i^{(V)} \) (perfectly aligned view-specific representations),
        \begin{align}
            \kappa = \min \{k, (\min\limits_{v=1,\dots,V} \{k_v \})^V \}
        \end{align}
    \end{proposition}
}

\def\propositionConditionalProbMinimum{
    \begin{proposition}
         \label{prop:conditionalProbMinimum}
         Suppose \( k_v, v \in \naturals \) are random variables taking values in \( \{1, \dots, k \} \).
         Then, for any \( V \ge 1 \),
         \begin{align}
             \Prob \{ \min\limits_{v=1, \dots, V+1} \lrc{k_v} \le \min\limits_{v=1, \dots, V} \lrc{k_v} ~\Big|~ k_1, \dots, k_V \} = 1
         \end{align}
    \end{proposition}
}
\def\propositionExpectationMinimum{
    \begin{proposition}
        \label{prop:expectationMinimum}
        Suppose \( k_v, v \in \naturals \) are iid.\ random variables taking values in \( \{1, \dots, k \} \).
        Then, for any \( V \ge 1 \),
        \begin{align}
            \E (\min\limits_{v=1, \dots, V+1} \lrc{k_v}) \le \E (\min\limits_{v=1, \dots, V} \lrc{k_v})
        \end{align}
    \end{proposition}
}

\propositionMinSeparbleClusters
\begin{proof}
    See~\cite{trostenReconsideringRepresentationAlignment2021}.
\end{proof}

According to Proposition~\ref{prop:prop1}, when the view-specific representations are perfectly aligned, the number of separable clusters after fusion, \( \kappa \), depends on the number of separable clusters in the \emph{least informative view} -- the view with the lowest \( k_v \).
The following propositions show what happens to \( \min \{ k_v \} \) when the number of views increases\footnote{The proofs of Propositions~\ref{prop:conditionalProbMinimum} and~\ref{prop:expectationMinimum} are given in the supplementary}.

\propositionConditionalProbMinimum
\propositionExpectationMinimum

Assuming the view-specific representations are perfectly aligned, Propositions~\ref{prop:conditionalProbMinimum} and~\ref{prop:expectationMinimum} show that:
\begin{enumerate*}[label=(\roman*)]
    \item Given a number of views, adding another view will, with probability \( 1 \), not increase \( \min \{k_v \} \).
    \item Among two datasets with the same distribution for the \( k_v \), the dataset with the \emph{smallest number of views} will have the highest expected value of \( \min \{k_v\} \).
\end{enumerate*}

In summary, we have shown that contrastive alignment-based models perform worse when the number of views in a dataset increases.
These findings are supported by the experimental results in Figure~\ref{fig:motivatingIncviews} and Table~\ref{tab:motivatingMNIST} which show that, when the number of views increases, the contrastive alignment-based model is outperformed by the model without any alignment.

\customparagraph{Alignment as a pretext task.}
    In contrast to our theoretical findings in the simplified case, Figure~\ref{fig:motivatingIncviews} and Table~\ref{tab:motivatingMNIST} show that contrastive alignment can sometimes be beneficial for the performance, particularly when the number of views is small.
    This is because alignment might be a good pretext task that helps the encoders learn informative representations, by learning to represent the information that is shared across views.
    However, we emphasize that this is only true when the number of views is small ($\le 4$ in Figure~\ref{fig:motivatingIncviews}), meaning that alignment should be used with caution when the number of views increases beyond this point.

\begin{figure}
\begin{floatrow}
\ffigbox[0.35\columnwidth]{%
    \centering
    \bgroup
    \def\figwidth{3.7cm}
    \def\figheight{4.2cm}
    \scriptsize
    \begin{tikzpicture}

\definecolor{chocolate217952}{RGB}{217,95,2}
\definecolor{darkcyan27158119}{RGB}{27,158,119}
\definecolor{darkgray176}{RGB}{176,176,176}
\definecolor{lightgray204}{RGB}{204,204,204}

\begin{groupplot}[group style={group size=1 by 1}]
\nextgroupplot[
height=\figheight,
legend cell align={left},
legend style={
  fill opacity=0.8,
  draw opacity=1,
  text opacity=1,
  at={(0.03,0.03)},
  anchor=south west,
  draw=lightgray204
},
tick align=outside,
tick pos=left,
width=\figwidth,
x grid style={darkgray176},
xlabel={Number of views},
xmajorgrids,
xmin=1.8, xmax=6.2,
xtick style={color=black},
y grid style={darkgray176},
ylabel={ACC},
ymajorgrids,
ymin=0.2, ymax=0.5,
ytick style={color=black}
]
\path [draw=darkcyan27158119, fill=darkcyan27158119, opacity=0.2]
(axis cs:2,0.375364035135541)
--(axis cs:2,0.357336074346271)
--(axis cs:3,0.38645462505398)
--(axis cs:4,0.366926940756274)
--(axis cs:5,0.350017003154158)
--(axis cs:6,0.371804667381619)
--(axis cs:6,0.393459903331424)
--(axis cs:6,0.393459903331424)
--(axis cs:5,0.374542005921007)
--(axis cs:4,0.396980789823102)
--(axis cs:3,0.391021626071356)
--(axis cs:2,0.375364035135541)
--cycle;

\path [draw=chocolate217952, fill=chocolate217952, opacity=0.2]
(axis cs:2,0.391650448011561)
--(axis cs:2,0.372257282567815)
--(axis cs:3,0.317603556653463)
--(axis cs:4,0.321921504793895)
--(axis cs:5,0.366895840213459)
--(axis cs:6,0.381128247307615)
--(axis cs:6,0.428912465048952)
--(axis cs:6,0.428912465048952)
--(axis cs:5,0.375302269413311)
--(axis cs:4,0.361931942643392)
--(axis cs:3,0.358108790377177)
--(axis cs:2,0.391650448011561)
--cycle;

\addplot [semithick, darkcyan27158119]
table {%
2 0.366350054740906
3 0.388738125562668
4 0.381953865289688
5 0.362279504537582
6 0.382632285356522
};
\addlegendentry{w/ align}
\addplot [semithick, chocolate217952, dashed]
table {%
2 0.381953865289688
3 0.33785617351532
4 0.341926723718643
5 0.371099054813385
6 0.405020356178284
};
\addlegendentry{w/o align}
\end{groupplot}

\end{tikzpicture}

    \egroup
}{%
  \caption{Clustering accuracy for an increasing number of views on Caltech7.}
  \label{fig:motivatingIncviews}
}
\capbtabbox[0.5\columnwidth]{%
    \bgroup
    \tableFontSize
    \setlength{\tabcolsep}{.9mm}
    
\rowcolors{2}{gray!25}{white}
\begin{tabular}{ccc}\toprule
    Dataset & w/o align & w/ align \\ \midrule
    \ST[c]{Edge-\\MNIST\\(2 views)} & \MTC{0.89} & \MTC{\BEST{0.97}} \\
    \ST[c]{Caltech7\\(6 views)} & \MTC{\BEST{0.41}} & \MTC{0.38} \\
    \ST[c]{Patched-\\MNIST\\(12 views)} & \MTC{\BEST{0.84}} & \MTC{0.73} \\
    \bottomrule
\end{tabular}





    \egroup
}{%
  \caption{Clustering accuracies on datasets with varying number of views.}
  \label{tab:motivatingMNIST}
}
\end{floatrow}
\end{figure}
