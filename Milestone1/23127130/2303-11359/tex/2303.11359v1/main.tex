\documentclass[twocolumn,twocolappendix]{aastex63}

\usepackage{todonotes}
\usepackage{cleveref}
\usepackage{booktabs}
\usepackage{longtable}
\usepackage{array}

% fix for using cleveref: https://tex.stackexchange.com/questions/171796/cref-does-not-work-for-section-in-emulateapj-cls/239694
\makeatletter
\usepackage{etoolbox}
\patchcmd\H@refstepcounter{\protected@edef}{\protected@xdef}{}{}
\makeatother

%% If you want to create your own macros, you can do so 
%% using \newcommand. Your macros should appear before
%% the \begin{document} command.
%%
\newcommand{\vdag}{(v)^\dagger}
\newcommand\aastex{AAS\TeX}
\newcommand\latex{La\TeX}
\newcolumntype{i}{>{\scriptsize}r}

%\received{June 1, 2019}
%\revised{January 10, 2019}
%\accepted{\today}

%\submitjournal{ApJ}

%\definecolor{purple}{rgb}{1, 0, 1}

\newcommand{\ie}{\emph{i.e.,}\xspace}
\newcommand{\eg}{\emph{e.g.,}\xspace}
\newcommand{\abr}{\emph{abbr.}\xspace}
\newcommand{\ea}{\emph{et al.}\xspace}
\newcommand{\gensync}{\emph{GenSync}\xspace}
\newcommand{\colosseum}{\emph{Colosseum}\xspace}
\newcommand{\srep}{\emph{SREP}\xspace} % Set Reconciliation Enhances
\newcommand{\srepsim}{\emph{SREPSim}\xspace}
% Propagation
\newcommand{\esrep}{\emph{E-SREP}\xspace}
\newcommand{\epsrep}{\emph{EP-SREP}\xspace}
\newcommand{\mesrep}{\emph{ME-SREP}\xspace}
\newcommand{\mempoolsync}{\emph{MempoolSync}}

\newcommand{\fref}[1]{Fig.~\ref{#1}}
\newcommand{\tref}[1]{Table~\ref{#1}}
\newcommand{\aref}[1]{Algorithm~\ref{#1}}
\newcommand{\procref}[1]{Procedure~\ref{#1}}
\newcommand{\sref}[1]{Section~\ref{#1}}
\newcommand{\lineref}[1]{line~\ref{#1}}
\newcommand{\appref}[1]{Appendix~\ref{#1}}

% Change \eqref
\LetLtxMacro{\originaleqref}{\eqref}
\renewcommand{\eqref}{Eq.~\originaleqref}

% Theorems and corollaries
\newcounter{theoremcount}
\setcounter{theoremcount}{0}
\DeclareRobustCommand{\theorem}[1]{%
  \refstepcounter{theoremcount}%
  \noindent\textit{\textbf{Theorem \thetheoremcount\label{theorem:#1}: }}%
}
\DeclareRobustCommand{\theoremref}[1]{Theorem~\ref{theorem:#1}}

\DeclareRobustCommand{\proof}{\emph{Proof:}\xspace}
\DeclareRobustCommand{\qqed}{\hfill$\blacksquare$}

\newcounter{corollcount}
\setcounter{corollcount}{0}
\DeclareRobustCommand{\coroll}[1]{%
  \refstepcounter{corollcount}%
  \noindent\textit{\textbf{Corollary \thecorollcount\label{coroll:#1}: }}%
}
\DeclareRobustCommand{\corollref}[1]{Corollary~\ref{coroll:#1}}

\newcounter{lemmacount}
\setcounter{lemmacount}{0}
\DeclareRobustCommand{\lemma}[1]{%
  \refstepcounter{lemmacount}%
  \noindent\textit{\textbf{Lemma \thelemmacount\label{lemma:#1}: }}%
}
\DeclareRobustCommand{\lemmaref}[1]{Lemma~\ref{lemma:#1}}

\newcounter{definitioncount}
\setcounter{definitioncount}{0}
\DeclareRobustCommand{\definition}[1]{%
  \refstepcounter{definitioncount}%
  \noindent\textit{\textbf{Definition \thedefinitioncount\label{definition:#1}: }}%
}
\DeclareRobustCommand{\defref}[1]{Definition~\ref{definition:#1}}

%notes of different authors
\newif\ifnotes
\notestrue
\notesfalse

\newif\ifdiff
\difftrue
\difffalse

\newcommand{\anote}[1]{\ifnotes $\ll$\textsf{\textcolor{purple}{Ari: {#1}}}$\gg$ \fi}
\newcommand{\nnote}[1]{\ifnotes $\ll$\textsf{\textcolor{orange}{Novak: {#1}}}$\gg$ \fi}
\newcommand{\diff}[1]{\ifdiff\textcolor{orange}{#1}\else#1\fi}

%%% Local Variables:
%%% mode: latex
%%% TeX-master: "main"
%%% End:

\shorttitle{A Distinct Population of Jetted-\acp{agn}}
\shortauthors{Kiehlmann et al.}

\graphicspath{{./}{figs/}}

% cleverref abbreviations:
\crefname{equation}{Eq.}{Eqs.}
\Crefname{equation}{Equation}{Equations}
\crefname{figure}{Fig.}{Figs.}
\Crefname{figure}{Figure}{Figures}
\crefname{table}{Table}{Tables}
\Crefname{table}{Table}{Tables}
\crefname{section}{Section}{Sections}
\Crefname{section}{Section}{Sections}

% acronyms
\usepackage[nolist,nohyperlinks]{acronym}
\begin{acronym}
 \acro{ads}[ADS]{Astrophysics Data System}
 \acro{agn}[AGN]{Active Galactic Nucleus}
 \acroplural{agn}[AGN]{Active Galactic Nuclei}
 \acro{bologna}[Bologna]{Bologna Complete Sample of Nearby Radio Sources}
 \acro{cd}[CD]{Compact Double}
 \acro{cjf}[CJF]{Caltech Jodrell Bank flat-spectrum}
 \acro{class}[CLASS]{Cosmic Lens All-Sky Survey}
 \acro{cso}[CSO]{Compact Symmetric Object}
 \acro{css}[CSS]{Compact Steep Spectrum}
 \acro{ddrg}[DDRG]{Double-Double Radio Galaxy}
 \acroplural{ddrg}[DDRGs]{Double-Double Radio Galaxies}
 \acro{emerlin}[eMERLIN]{Enhanced Multi Element Remotely Linked Interferometer Network}
 \acro{fsrq}[FSRQ]{Flat Spectrum Radio Quasar}
 \acro{fwhm}[FWHM]{Full Width at Half Maximum}
 \acro{ps}[PS]{Peaked Spectrum}
 \acro{hfp}[HFP]{High-Frequency Peaker}
 \acro{ks}[KS]{Kolmogorov-Smirnov}
 \acro{mso}[MSO]{Medium Symmetric Object}
 \acro{ned}[NED]{NASA/IPAC Extragalactic Database}
 \acro{ovro}[OVRO]{Owens Valley Radio Observatory}
 \acro{pr}[PR]{Pearson Readhead}
 \acro{ps}[PS]{Peaked-Spectrum}
 \acro{prcj1}[PR+CJ1]{Pearson Readhead and First Caltech Jodrell Bank}
 \acro{rfc}[RFC]{Radio Fundamental Catalog}
 \acro{sfr}[SFR]{Star Formation Rate}
 \acro{smbh}[SMBH]{Supermassive Black Hole}
 \acro{vips}[VIPS]{VLBA Imaging and Polarimetry Survey}
 \acro{vla}[VLA]{Very Large Array}
 \acro{vlba}[VLBA]{Very Long Baseline Array}
 \acro{vlbi}[VLBI]{Very Long Baseline Interferometry}
\end{acronym}


%------------------------------------------------------------------------------
\begin{document}

\title{Compact Symmetric Objects: A Distinct Population of Jetted Active Galaxies}
%\title{A new catalog of Compact Symmetric Objects}
%\title{First Definitive Catalog of Compact Symmetric Objects}
%\title{Definite Catalog of Compact Symmetric Objects}
%\title{Definite Catalog of Extragalactic Compact Symmetric Objects}
%\title{Extragalactic Radio Sources: Definite Catalog of Compact Symmetric Objects}
%\title{First Catalog of Compact Symmetric Objects}

\correspondingauthor{Anthony Readhead}
\email{acr@caltech.edu}

\author{S. Kiehlmann}
\affiliation{Institute of Astrophysics, Foundation for Research and Technology-Hellas, GR-70013 Heraklion, Greece}
\author{A.C.S Readhead}
\affiliation{Owens Valley Radio Observatory, California Institute of Technology, Pasadena, CA 91125, USA}
\author{S. O'Neill}
\affiliation{Owens Valley Radio Observatory, California Institute of Technology, Pasadena, CA 91125, USA}
\author{P.N. Wilkinson}
\affiliation{Jodrell Bank Centre for Astrophysics, University of Manchester, Oxford Road, Manchester M13 9PL, UK} 
\author{M.L. Lister}
\affiliation{Department of Physics and Astronomy, Purdue University, 525 Northwestern Avenue, West Lafayette, IN 47907, USA}
\author{I. Liodakis}
\affiliation{Finnish Center for Astronomy with ESO, University of Turku, Vesilinnantie 5, FI-20014, Finland}
\affiliation{Department of Physics, Univ. of Crete, GR-70013 Heraklion, Greece}
\author{S. Bruzewski}
\affiliation{Department of Physics and Astronomy, University of New Mexico, Albuquerque, NM 87131, USA}
 \author{T. J. Pearson}
\affiliation{Owens Valley Radio Observatory, California Institute of Technology, Pasadena, CA 91125, USA}
\author{E. Sheldahl}
\affiliation{Department of Physics and Astronomy, University of New Mexico, Albuquerque, NM 87131, USA}
\author{A. Siemiginowska}
\affiliation{Center for Astrophysics|Harvard and Smithsonian, 60 Garden St., Cambridge, MA 02138, USA}
\author{K. Tassis} 
\affiliation{Institute of Astrophysics, Foundation for Research and Technology-Hellas, GR-70013 Heraklion, Greece}
\affiliation{Department of Physics and Institute of Theoretical and Computational Physics, University of Crete, 70013 Heraklion, Greece}
\author{G.B. Taylor}
\affiliation{Department of Physics and Astronomy, University of New Mexico, Albuquerque, NM 87131, USA}


\begin{abstract}
\acfp{cso} are a class of compact, jetted \acp{agn} whose jet axes are not aligned close to the line of sight, and whose observed emission is not predominantly relativistically boosted towards us. Using complete samples of \acp{cso}, we present three independent lines of evidence, based on their relative numbers, their redshift distributions, and their size distributions, 
which show conclusively that most \acp{cso} do not evolve into larger-scale radio sources. Thus \acp{cso}  belong to a distinct population of  jetted-\acp{agn}.  This population should be characterized as ``short-lived'', as opposed to ``young''.  We show that there is a sharp upper cutoff in the \ac{cso} size distribution at $\approx 500$ pc, which cannot result from random episodic fueling events. There is clearly something that limits the fueling to $\lesssim 100 M_\odot$. Possible origins of \acp{cso}, if not related to the fueling, must be related to the accretion disk, or  the collimation of the relativistic jets.  CSOs may well have a variety of origins, with each of the above mechanisms producing subsets of CSOs.  Whatever the physical mechanism(s) might be, the distinct differences between \acp{cso} and other jetted-\ac{agn}  provide crucial insights into the formation and evolution of relativistic jets in \acp{agn} and the supermassive black holes that drive them.




\end{abstract}

\keywords{Active Galactic Nucleus, Compact Symmetric Objects, Young Radio Sources}



%------------------------------------------------------------------------------
\section{Introduction}
\label{sec:intro}

The first indication of relativistic motion in the jets of active galaxies was the asymmetric large-scale jet in M87 discovered by \citet{1918PLicO..13....9C}. The next was arguably the discovery of rapid flux density variations in blazars \citep{1965Sci...148.1458D,1965AJ.....70..672D}, which were quickly shown by \citet{1966Natur.211..468R,1967MNRAS.135..345R} to be due to relativistic motion of the emission regions towards the observer. In spite of this development, 
observations of the synchrotron self-absorption cutoff frequencies of radio sources with flat spectra led to the hypothesis that an ``inverse Compton catastrophe'' imposes an upper limit of $\sim 10^{12}\,\mathrm{K}$ on the brightness temperatures of compact radio sources \citep{1969ApJ...155L..71K}. This appeared, at first, to be supported by \ac{vlbi} observations, but in these calculations the possibility of relativistic bulk motion towards the observer \citep{1966Natur.211..468R,1967MNRAS.135..345R} was not taken into account. 

The first astronomical image at any wavelength with resolution $\ll 1$ arc second was produced in the first ``hybrid map'',  which showed a one-sided jet \citep{1977Natur.269..764W}. Such core-jet structures were soon shown to predominate in compact radio sources \citep{1978Natur.276..768R,1980IAUS...92..165R}. making it clear that relativistic beaming determines the apparent morphology of most compact radio sources observed at cm wavelengths. Nevertheless the myth of the inverse Compton catastrophe continued to propagate, but,  as shown by \citet{1994ApJ...426...51R}, when relativistic beaming is taken into account the brightness temperatures drop to $\sim 10^{11}\,\mathrm{K}$, and are consistent with equipartition between the magnetic field and particle energy densities in the emission regions. 

It has been clear for over four decades now that the observed emission regions in most compact radio sources at cm wavelengths are strongly boosted by relativistic beaming, but {\it even today\/} astronomers often still ignore this, leading, as we show below, to the misclassification of objects and to the obfuscation of the underlying astrophysics.


Relativistic beaming greatly complicates the physical analysis of the observed radio emission. 
In order to overcome such complications, which introduce large uncertainties in the physical properties -- magnetic field and particle energy densities, pressures, total energies, etc. -- of the emission regions, \citet{1994ApJ...432L..87W}, hereafter W94, introduced the \acf{cso} classification of compact radio sources.
Due to the morphological symmetry of the emission on either side of the nucleus, these objects are clearly not exhibiting strongly beamed emission towards the observer. 

Unfortunately, however, a number of jetted-\acp{agn} have been misidentified  as \acp{cso} or \ac{cso} candidates, and in the literature many jetted-\acp{agn} whose axes are close to the line of sight, and whose observed emission is strongly beamed towards us, have crept into this class.

This paper is the second of three  on the morphological radio properties of \acp{cso} in which we explore the phenomenology uncontaminated by objects that have been mis-identified as \acp{cso}.  In the first paper we added two new criteria, based on variability and speed, to the \ac{cso} selection criteria and undertook a detailed survey of the literature, which enabled us to identify 79~bona fide \acp{cso} (Kiehlmann et al. submitted, hereafter Paper~1), and we determined the fractions of \acp{cso} in complete flux density limited samples.  In this paper (Paper~2) we show that \acp{cso} form a class of jetted-\acp{agn} that is both distinct from other jetted-\acp{agn} and  exhibits a sharp cutoff in size at $\approx 500$ pc, and in the third paper (Paper~3: Readhead et al. submitted) we discuss the evolution of \acp{cso} and show that while \acp{cso} are nearly all  ``short-lived'' compared to the classes of larger jetted-AGN, only a minority of them are ``young''. 
Just as we would not call a 10-year-old dog a ``young'' dog, we should avoid the implicit assumptions involved in calling  all  \acp{cso} ``young'', which obscures the true nature and the importance of this class of jetted-\acp{agn}. We discuss three possible origins of CSOs based on (i) the capture of single stars by  dormant spinning supermassive black holes (SMBH) in the nuclei of elliptical galaxies, as first suggested by \citet{1994cers.conf...17R}, hereafter R94, (ii) extraction of energy from the spin of the SMBH, catalyzed by stellar capture, and (iii)  an origin in  the accretion disc itself, somewhat  analogous to dwarf novae.

As discussed in detail in Paper~1, in \acp{cso}, two emission regions are seen straddling the center of activity, making it clear that these cannot be strongly relativistically boosted, otherwise the object would be seen as a one-sided asymmetric ``core-jet'' object as is the case in the vast majority of compact radio sources observed at cm wavelengths \citep{2019ApJ...874...43L}. 

Individual \acp{cso} undergo appreciable evolutionary changes on timescales of years that can therefore be studied without the complications of relativistic beaming.  The bulk flows along their jets and their speeds of advance into the interstellar medium can be measured directly. 


 





As we show in this paper, a subset consisting of 17~bona fide \acp{cso}, which is constructed from three statistically complete samples,  provides statistically robust evidence that \acp{cso} constitute a population of jetted-\acp{agn} that is both (i) distinct from all other classes of jetted-\acp{agn}, and (ii) shows a sharp upper cutoff in size at $\approx 500$ pc. As such, there must be separate physical mechanism(s), and hence a separate origin(s) of \acp{cso} compared to other jetted-\acp{agn}.

We argue that \acp{cso} provide a uniquely accessible laboratory for the study of relativistic jets \citep{2019ARAandA..57..467B} and the SMBH central engines that drive them, because they are short-lived, rather than young, and hence pass through all stages of their lives as \acp{cso}, and are  readily accessible to detailed study in all of these phases of their lives. 

It is critically important, therefore, to recognize the distinction between the terms  ``young'' and ``short-lived'', which otherwise obfuscates the phenomenology of the \ac{cso} class. By ``short-lived'' we mean short-lived in comparison to FR~I and FR~II objects, which have sizes in the $\sim$100\,kpc - several Mpc range, and therefore clearly have ages $\gg  10^6$\,yr. 

By the early 1990s, three bona fide \acp{cso} had been definitively identified in the complete sample of 65 radio sources studied by \citet{1988ApJ...328..114P}. Despite the small size of the \ac{cso} sample, and entirely because it was part of a complete sample, this  sample of only three \acp{cso} was enough to enable a number of the most critical questions about \acp{cso} to be addressed by R94, including their relationship to the larger jetted-AGN, their lifetimes, and their energy requirements. R94 concluded that CSOs form a distinct population of compact jetted-\acp{agn}, and that there must be a physical reason for this which provides a unique window on the central engines that drive \acp{agn}. R94 also  suggested that \acp{cso} might be the result of the capture of a single star by a SMBH in an otherwise quiescent
elliptical galaxy nucleus. 

All of these properties of 
 \acp{cso} were discussed in more detail, and confirmed,  in \citet{1996ApJ...460..612R}, hereafter R96. Nevertheless, in spite of their distinction, \acp{cso} have attracted comparatively little attention among jetted-\acp{agn} enthusiasts. We explore the defining properties of \acp{cso} in considerably more detail in this paper and in Paper~3.



In this paper we study the bona fide \acp{cso} found in three complete samples of radio sources, and we conclude that the findings of R94 and R96 are indeed correct.  Of the 79 bona fide CSOs that we identified in Paper 1, 54 have spectroscopic redshifts, and 17 of these are included in one or more of three complete samples, allowing them to be used for statistical studies. These three complete samples have no spectral index filter, and so are fully complete over their given areas of sky and down to their flux density  limits.  These  samples provide statistically robust evidence for a cutoff in \ac{cso} sizes well below 1\,kpc. This prompted our investigations of the number and redshift distributions in addition to the size. These are the main foci of this paper.


Discussions of a size cutoff in \acp{cso} are not new \citep{1998MNRAS.299.1159A,2006MNRAS.368.1411A,2009AN....330..190A}, and early lobe-speed measurements showed that the hotspots of CSOs are rapidly separating  \citep{1998AandA...337...69O,1999NewAR..43..669O, 2002evn..conf..139P}. It was clear, therefore, as pointed out in R94 and R96, that \acp{cso} must be short-lived, since otherwise there would be far more of their longer-lived, larger counterparts. This means that \acp{cso} {\it must} exhibit a size cutoff. As shown in this paper,  we have now determined that this size cutoff occurs between a few hundred~pc and 500~pc. The evolution of \acp{cso} within this size range from ``early-life'' through ``mid-life'' to ``late-life'' is discussed in detail in Paper~3.
 


Throughout this paper we adopt the convention $S_\nu \propto
\nu^{\alpha}$ for spectral index $\alpha$, and use the cosmological
parameters $\Omega_\mathrm{m} = 0.27$, $\Omega_\Lambda = 0.73$ and $H_0 = 71 \;
\mathrm{km\; s^{-1} \;\,Mpc^{-1}}$ \citep{Komatsu09}. We do this for consistency with our other papers. None of the conclusions would be changed were we to adopt the best model of the Planck Collaboration  \citep{2020AandA...641A...6P}.


\begin{figure}[!t]
 \centering
 \includegraphics[width=1.0\linewidth]{Largescalehistogram.pdf}
 \caption{The  distributions of the largest projected sizes of FR~I objects (top panel) and FR~II objects (bottom panel) in the PR+CJ1+PW complete samples. The FR I sizes have been determined from our own angular size measurements. The FR II sizes are based on the largest angular size measurements of \citet{1993ApJ...413..453N} for all but six sources not included in their sample, for which we measured the angular sizes ourselves. 
 There is one FR~II object (3C 236) of size 4.3\,Mpc that is not included in the FR~II plot.  }
 \label{plt:histogramlargescale}
\end{figure}




\section{Complete Samples of Jetted-AGN}\label{sec:complete}

In this paper we make extensive use of samples of radio sources that are complete in the sense that they include all of the radio sources above a specified flux density limit at a specified frequency in a given area of the sky. 

Before introducing the complete samples, we summarize briefly the selection of our bona fide \ac{cso} sample, the details of which can be found in Paper~1. 

In order to recapture the original intent of the \ac{cso} classification and to explore the properties of this class, we carried out the comprehensive survey of the literature described in Paper~1, which resulted in a sample of 79~bona fide \acp{cso}, all of which passed the first three of our \ac{cso} criteria (morphology, size, and variability), and were not known to violate the fourth (apparent speed limit).  Of these, 54~have known spectroscopic redshifts.

The determination of a uniform set of measurements of the largest angular size of these bona fide \acp{cso} is described in Paper~1.
In this paper we focus on the 54~bona fide \acp{cso} with spectroscopic redshifts. As described below and in Paper~1, there are three complete samples of \acp{cso} that are subsets of these 54~bona fide \acp{cso}. These are of prime importance in this study because they enable us to carry out the statistical studies on which this paper is based. 


\begin{deluxetable*}{c@{\hskip 8mm}ccccccccc}
\tablecaption{The Numbers of CSS, FR~I, FR~II, and CSO objects in the Complete Samples. }
\tablehead{Complete& Flux Density&CSS&FR~I&FR~II&Total&CSO2&CSO2/FR~II&CSO2/Total\\
Sample&limits&Number&Number&Number&Number&Number&Percentage&Percentage}
\startdata
PR&$S_5\geq 1.3$\,Jy&6 &3&16&$64^\dag$&6&37.5$\pm$18.0 \%&9.4$\pm$4.0 \%\\
CJ1&1.3\,Jy $\geq S_5 \geq 0.7$\,Jy&23 &6&30&135&$6^\ddag$&20.0$\pm$8.9 \%&4.4$\pm$1.9 \%\\
PR$+$CJ1&$S_5\geq 0.7$\,Jy&29 &8&46&$199^\dag$&$12^\ddag$&26.1$\pm$8.5 \%&6.0$\pm$1.8 \%\\
PW&$S_{2.7}\geq 1.5$\,Jy&26 &15&65&$170^\dag$&12&18.5$\pm$5.8 \%&7.1$\pm$2.1 \%\\
PWS&$S_5\geq 1.3$\,Jy&7 &8&11&50&5&45.5$\pm$24.5 \%&10.0$\pm$4.7 \%\\
PR$+$CJ1$+$PW& - & 43 &16&76&$281^\dag$&$18^\ddag$&23.7$\pm$6.2 \%&6.4$\pm$1.6 \%\\
\enddata
\tablecomments{All of the CSOs in the PR+CJ1+PW sample are CSO 2s. $^\dag$ the numbers exclude 3C~231 (M82), a starburst galaxy, not an AGN. $^\ddag$ the numbers include the bona fide \ac{cso} J1335+5844, for which there is no published spectroscopic redshift. These numbers are taken from the list of the 282~sources in the three complete samples given in \cref{tab:samples} in the Appendix. PWS is the subsample of PW at $10^\circ<\delta<35^\circ$ (B1950) and with $S_{5\,\mathrm{GHz}}> 1.3$\,Jy.  As should be clear in view of the size of the samples, and assuming there is no dependence of  the CSO fraction on flux density,  the most reliable statistic is the final one combining the three full samples PR, CJ1, and PW. }
\label{tab:csonumbers}
\end{deluxetable*} 


There are 282~objects in the union of these three complete samples (enumerated below) and these are listed in \cref{tab:samples} in the Appendix. In our analysis in this paper, we exclude M82 (3C~231) which is in all three  
 samples, but is a starburst galaxy and not a jetted AGN, leaving  281 sources.  The number of \acp{cso} in each of these three complete samples is given in \cref{tab:csonumbers}.

The three complete, flux-density limited  samples used in this study are:
\vskip 6pt
\noindent
1. The \textbf{Pearson--Readhead (PR) sample} \citep{1981ApJ...248...61P,1988ApJ...328..114P} is the complete sample of  65~radio sources with declination $\delta \ge 35^\circ$ (B1950),  Galactic latitude $|b| > 10^\circ$, and flux density $> 1.3$\,Jy at 5\,GHz in the MPIfR/NRAO~S4 and S5~surveys \citep{1978AJ.....83..451P,1981AJ.....86..854K}. After excluding M82 the total number of AGN  in the PR sample is 64.

\vskip 6pt
\noindent
2. The \textbf{first Caltech--Jodrell (CJ1) sample} \citep{1995ApJS...98....1P,1995ApJS...99..297X} covers the same area of sky as the PR sample and includes 135~sources between 0.7 and 1.3\,Jy at 5\,GHz. The combined PR$+$CJ1 sample contains 199 AGN with $S_{\rm 5\,GHz} > 0.7$\,Jy (after excluding M82).

\vskip 6pt
\noindent
3. The \textbf{Peacock--Wall (PW) sample} \citep{1981MNRAS.194..331P,1985MNRAS.216..173W} is complete for $S_{\rm 2.7\,GHz} > 1.5$\,Jy over the region of sky $\delta \geq 10^\circ$ (B1950) and $|b|\ge 10^\circ$. It contains 171 sources (or 170 AGN after excluding M82).\footnote{The original PW sample \citet{1981MNRAS.194..331P} contained 168 sources; to these three (DA~240, 0945+73 = 4C~73.08, and NGC~6251) were added by \citet{1985MNRAS.216..173W}.} All the  sources in the PW sample were mapped using the Cambridge 5 km Telescope by \citet{1981MNRAS.194..331P}, who also classified the large scale structures in  the PW sample according to the following types:~(i) FR~I and FR~II, and an intermediate FR~type (FR?) \citep{1974MNRAS.167P..31F}; (ii) objects unresolved on the 5 km Telescope (U); (iii)~\ac{css} objects having $\alpha \le -0.5$ between 2.7\,GHz and 5\,GHz; and (iv)~double objects with the optical identification coincident with one of the two radio components.  These types are listed in column 7 of \cref{tab:samples} in the Appendix. 

\begin{deluxetable*}{c@{\hskip 8mm}cccccc}
\tablecaption{Redshifts and Sizes of the 17~\acp{cso} with spectroscopic redshifts in the PR, CJ1 and PW Complete Samples}
\tablehead{IAU Name&Redshift&Size (pc)& PR&CJ1&PW\\}
\startdata
J0029+3456	&	0.517	&	259	&		&		&	Y	\\
J0111+3906	&	0.668	&	56	&	Y	&		&		\\
J0119+3210	&	0.0602	&	115	&		&		&	Y	\\
J0405+3803	&	0.05505	&	44	&		&	Y	&		\\
J0713+4349	&	0.518	&	217	&	Y	&		&	Y	\\
J1035+5628	&	0.46	&	221	&	Y	&		&	Y	\\
J1227+3635	&	1.975	&	499	&		&	Y	&	Y	\\
J1244+4048	&	0.8135	&	529	&		&	Y	&		\\
J1326+3154	&	0.37	&	346	&		&		&	Y	\\
J1347+1217	&	0.121	&	215	&		&		&	Y		\\
J1400+6210	&	0.431	&	378	&	Y	&		&	Y	\\
J1407+2827	&	0.077	&	16	&		&		&	Y		\\
J1609+2641	&	0.473	&	362	&		&		&	Y		\\
J1735+5049	&	0.835	&	61	&		&	Y	&			\\
J1944+5448	&	0.263	&	196	&		&	Y	&			\\
J2022+6136	&	0.227	&	105	&	Y	&		&	Y		\\
J2355+4950	&	0.237	&	336	&	Y	&		&	Y	\\
\enddata
\tablecomments{The sizes and redshifts of the CSOs in the three complete samples (PR+CJ1+PW). References for the redshifts are given in Paper~1.}
\label{tab:zandsize}
\end{deluxetable*}

The PW sample was selected at 2.7\,GHz, unlike the PR and CJ1 samples which were selected at 5\,GHz. However, we have 5\,GHz flux densities for all the PW sources \citep{1977IAUS...74...63P}.    Following a suggestion by John Peacock, in order to be able to combine results from these three complete samples without introducing any possible biases due to the different sample selection frequencies, we define a subset of the PW sample that is effectively complete at 5\,GHz. 
For this purpose we compare the GB6 \citep{1996ApJS..103..427G}, PR, and PW samples at 5\,GHz over their common sky area ($35^\circ \leq \delta \leq 75^\circ$, $|b| \geq 10^\circ$, B1950). These surveys were all made on different instruments at different times and since many of the sources are variable the samples change slightly with time. 
In this area of sky, the GB6 survey has 54, the PR survey  has 51, and the PW survey has 54 objects with  $S_{\rm 5\,GHz} \geq 1.3$\,Jy. 
%
It may safely be assumed, therefore, that the PW sample is effectively complete down to 1.3\,Jy at 5\,GHz. %S$_{\rm 5\,GHz}=1.3$\,Jy.  
Of these we define a sub-sample, PWS, where ``S'' stands for ``Subsample'', consisting of the PW sources at declinations $\delta < 35^\circ$ (B1950) and having $S_{\rm5\,GHz} \geq  1.3$\,Jy, for use in our physical size distribution statistical tests in \S \ref{sec:size}.


\begin{figure*}[!t]
 \centering
 \includegraphics[width=1.0\linewidth]{RedshiftHistograms.pdf}
 \caption{The redshift distributions for the PR+CJ1 complete sample (top panel) and the PW complete sample (bottom panel).  The light shaded  distributions show the complete samples. The dark shaded regions show the \acp{cso}.  Note that these distributions are not stacked vertically, so the values on the ordinate represent the total numbers of sources and the numbers of \acp{cso} in each sample. The cumulative distributions and KS statistics are shown in Figs.  \ref{plt:KSredshift} (a) and (b). }
 \label{plt:histogramredshift}
\end{figure*}



A great strength of the PR, CJ1, and PW samples is that {\it all} of the objects are well-studied and their radio properties on both large and small scales are known.   There is, therefore, no danger of unknown selection bias that could compromise the statistics. In \cref{tab:samples} in the Appendix we list all of the sources in the complete PR, CJ1 and PW samples and we provide references to these structure observations.


%\vskip 6pt
%\noindent
%4: The VLBI imaging and polarimetry survey (VIPS) complete sample \citep{2007ApJ...658..203H}: comprises 1127 objects having S$_{4850} > 85$ mJy, spectral indices $\alpha \geq -0.5$ between 4.85\,GHz and a lower frequency, drawn from the parent sample of the cosmic lense all sky survey (CLASS) \citep{2003MNRAS.341....1M}, and resticted to lie in the survey area, or ``footprint'', of the Sloan Digital Sky Survey (SDSS; \citet{2000AJ....120.1579Y}.



In addition to the above three complete samples, there is one other complete sample that is of prime importance to this study:
the GaLactic and Extragalactic All-Sky Murchison Widefield Array (GLEAM) survey \citep{2017ApJ...836..174C}, which covers the sky area $\delta<30^\circ$ (J2000), $|b|>10^\circ$ and defines a complete sample of 11,400~objects exhibiting flux densities greater than 1\,Jy in the 72\,MHz -- 700\,MHz range.


In the next two sections we present three independent arguments that \acp{cso} belong to a distinct class of jetted-\acp{agn} --- two arguments, based on the fractions of \acp{cso} in complete samples and their redshift distributions, are given in \S \ref{sec:statistics} and \S \ref{sec:redshift}, and the third argument, based on the size distribution of \acp{cso}, is given in \S \ref{sec:size}.

\begin{figure*}[!t]
 \centering
 \includegraphics[width=1.0\linewidth]{newKScompositeredshift.pdf}
 \caption{KS Tests on the redshift distributions of the bona fide \acp{cso} in the PR+CJ1, PW, and PWS samples. (a), (b) and (c): comparison of the \ac{cso} cumulative redshift distributions {\it vs.} the non-\acp{cso} in the  complete PR+CJ1, PW, and PWS samples, respectively.  The green bars indicate the maximum differences in the cumulative distributions, corresponding to the values of the KS statistic given by the numbers in green. The corresponding p-values are listed in \cref{tab:ksredshift}. }
 \label{plt:KSredshift}. 
\end{figure*}





\section{The Fractions of CSO\lowercase{s} in Complete Samples and Their Redshift Distributions}\label{sec:numbers}






\subsection{The Statistics of CSOs in Complete Samples}\label{sec:statistics}

 R96 gave a detailed discussion of the \ac{cso} fractions in the PR and CJ1 complete samples.  Here we update this discussion  and, in addition, incorporate the PW sample.

All of the bona fide CSOs in the PR, CJ1 and PW samples are CSO 2s (see Paper 1), because the flux density limits of all three samples a relatively high, making these high-luminosity objects.  So in this paper we discuss only CSO 2s. While the PR \ac{cso} sample is complete, as can be seen in Paper~1, in CJ1 there are 5 class~A \ac{cso} candidates which might possibly be bona fide \acp{cso}.  All of these candidates have sizes less than 500\,pc, and, were we to include these five sources in our analysis, the conclusions below would be strengthened.  We prefer to take the conservative route and not to include any \acp{cso} in the bona fide sample until they have met the CSO criteria laid out in Paper 1.

As can be seen in Paper~1, there are also six class~B \ac{cso} candidates in the PR, CJ1, and PW samples.  These are much less likely to be bona fide \acp{cso}, and all but one have sizes less than 500\,pc.  For these reasons, the conservative approach is again not to include any of these objects in the present analysis.

We see in \cref{tab:csonumbers} that the fraction of \acp{cso} in complete 2.7--5\,GHz samples is $(6.4\pm 1.6) \%$. this would rise to $ (8.2 \pm 1.8)\% $ if all of the class~A \acp{cso} candidates in the CJ1 sample are shown to be bona fide \acp{cso}.

We take as a simple hypothesis to be used throughout this paper, that, between their appearance and disappearance, the separation speed, $v_{\rm sep}$, of the opposing hot spots in \acp{cso}, when averaged over a sufficient interval of time, is constant, and that they continue at the same separation speed if they expand to form larger classes of sources, such as FR~IIs. Under this hypothesis, the number of objects in different size ranges scales  simply in proportion to the size ranges.

It is important to note that, for the purposes of our arguments regarding the fractions of \acp{cso} with respect to classes of larger sources,  this hypothesis is highly conservative. We show in Paper~3 that the separation  speed of the  hot spots for \acp{cso}  is $v_{\rm sep} =(0.36\pm0.04) $c, whereas, for example, \citet{1991ApJ...383..554C} argue convincingly that for the opposing  hot spots in Cygnus~A,   $  0.01 {\rm c } < v_{\rm sep} < 0.05 {\rm c}$. Note that this deduced separation speed for Cygnus~A is typical for FR IIs \citep{1995MNRAS.277..331S}.  Based on these values, the separation speeds in \acp{cso} are approximately an order of magnitude greater than those in FR~IIs, which means that if \acp{cso} do expand to form FR~IIs, they spend far less of their time in the 0\,kpc  to 1\,kpc size range than under the constant speed hypothesis, and so there should be far fewer of them relative to FR~IIs than under the constant speed hypothesis.

Under the constant speed hypothesis  we also assume  that the luminosity does not change enough for the source to drop out of the flux-limited sample. We discuss possible changes in luminosity later.

\begin{deluxetable*}{c@{\hskip 8mm}cccccccc}
\tablecaption{Two-sample KS Tests of \ac{cso} Redshifts as a Distinct Population}
\tablehead{Test&Complete&Sky& Flux Density&Frequency&KS&p-value & Significance\\
Number&Sample&Area&limit&GHz&statistic&&}
\startdata
1& PR+CJ1&$\delta > 35^\circ, |b|> 10^\circ$&0.7\,Jy &5\,GHz&0.34& $1.3 \times 10^{-1}$ &1.1$\sigma$ \\
2& PW&$\delta > 10^\circ, |b|> 10^\circ$&1.5\,Jy &2.7\,GHz&0.41& $3.1 \times 10^{-2}$ & 1.9$\sigma$\\
3&PWS&$10^\circ<\delta < 35^\circ, |b|> 10^\circ$&1.3\,Jy &5\,GHz&0.52& $1.2\times 10^{-1}$ & 1.3$\sigma$\\
4& PR+CJ1+PWS&-&-&-&-& $1.6 \times 10^{-2}$ &2.1$\sigma$ \\
\enddata
\tablecomments{Tests \#1 and \#3 are independent due to their the different sky areas. We can therefore, legitimately, multiply their p-values, which we do in Test \#4. }
\label{tab:ksredshift}
\end{deluxetable*}

We consider three populations of objects that are larger than \acp{cso} and that might, therefore, be the populations that \acp{cso} evolve into:
\vskip 6pt 
\noindent
(i) \ac{css} objects \citep{1982MNRAS.198..843P}, including the subclass of \acp{mso} which have sizes in the range 1\,kpc -- 20\,kpc \citep{1995AandA...302..317F} and R96. Note that MSOs have the same characteristics as CSOs apart from the size range.
\vskip 6pt 
\noindent
(ii) Fanaroff \& Riley Class~I jetted-\acp{agn} \citep{1974MNRAS.167P..31F}, which have sizes that range up to $\approx 1$ Mpc -- see Fig.  \ref{plt:histogramlargescale}(upper panel).
\vskip 6pt 
\noindent
(iii) Fanaroff \& Riley Class~II jetted-\acp{agn} \citep{1974MNRAS.167P..31F},  which also have sizes that range up to $\approx 1$ Mpc -- see Fig.  \ref{plt:histogramlargescale}(lower panel).

Note that \citet{1995AandA...302..317F} and R96 used an upper size limit of $15h^{-1}$\,kpc for \acp{mso}, where $H_o = 100 h \;
\mathrm{km\; s^{-1} \;\,Mpc^{-1}}$. For our adopted cosmology, this translates to 21\,kpc.  However, since the original choice of $15h^{-1}$\,kpc was chosen by \citet{1995AandA...302..317F} and R96 to be a convenient ``round number'',  we will follow that practice and use 20\,kpc  as the upper size limit of \acp{mso} in this study.


 We see from  \cref{tab:csonumbers} that there are 18~\acp{cso} and 43~\ac{css} objects in the combined PR+CJ1+PW complete sample. Note that MSOs are a subset of the CSS class. Thus the fraction of \acp{cso} in the combined \ac{cso}+\ac{css} sample is $(30 \pm 8)\%$.   Assuming an upper limit on \ac{css} and \ac{mso} sizes of 20\,kpc,  on our hypothesis of constant speed of advance, the number of \acp{cso} in complete samples of \ac{css} and \acp{mso} should be  $1/20=5\%$. We therefore reject the hypothesis that \acp{cso} evolve into \ac{css}+\ac{mso} sources.

 The median size of the FR-I sources in the combined PR, CJ1 and PW samples shown in Fig.  \ref{plt:histogramlargescale} is 180 kpc.  This can be compared to the median size of the CSO 2s in these samples of 217 pc.  The ratio in sizes $\approx 830$, so that on the hypothesis of constant expansion speed we would expect there to be $\sim$14,110 FR Is, whereas there are 16 --- i.e., there are $\sim 880 \times$ fewer FR Is than expected.   Conversely, given the number of FR Is in these three complete samples, there are $\approx 880 \times$ more CSO 2s than expected
 
 
 The median size of the FR-II sources in the combined PR, CJ1 and PW samples shown in Fig.  \ref{plt:histogramlargescale} is 305 kpc.  This can be compared to the median size of the CSO 2s in these samples of 217 pc.  The ratio in sizes $\approx 950$, so that on the hypothesis of constant expansion speed we would expect there to be $\sim$16,000 FR IIs, whereas there are 77 --- i.e., there are $\sim 210 \times$ fewer FR IIs than expected. Conversely, given the number of FR IIs in these three complete samples, there are $\approx 210 \times$ more CSO 2s than expected.  

 We see therefore that the numbers of both FR~I and FR~II sources, relative to \acp{cso} are far too small, by  factors of over 800 for the FR~Is and over 200 for the FR~IIs, for  \acp{cso} to evolve into either FR~I or FR~II sources of comparable radio luminosity. At this flux density level the integrated number-flux density counts have a power-law slope of $-1.3$, so that the luminosity would have to drop by  factors of 185 and 80, respectively to accommodate this scenario for FR~I or FR~II objects. 

 \citet{1991Natur.349..138R} have shown that there is a strong correlation between radio jet power and optical narrow line luminosity. Based on observations by \citet{1996ApJS..107..541L}, R96 showed that the narrow line luminosities of the \acp{cso} J0111+3906, J0713+4349 and  J2355+4950 are about a factor 30 below that of typical FR-II galaxies, so that if \acp{cso} are to evolve into FR~II galaxies, then their optical line luminosities must increase by about a factor 30 while their radio luminosities decrease by about a factor 35, which is an unlikely scenario.
 It is interesting to note that R96 show that the jet power for J2355+4950, when corrected for the Hubble constant and different cosmologies, is $\sim 7 \times 10^{43}\; {\rm erg \, s^{-1}}$, and for J0111+3906, and J0713+4349 the similarly corrected jet powers $\sim 10^{45}\; {\rm erg \, s^{-1}}$, which may be compared to the range of jet powers in FR II sources of $\sim 10^{44}\; {\rm erg \, s^{-1}} - 10^{47}\; {\rm erg \, s^{-1}}$ (R96). Thus the jet powers of CSO 2s are similar to those of FR II objects, as is also the case regarding their lumiosities.

Given the agreement in narrow line luminosity between \acp{cso} and FR~I galaxies, the possible evolutionary scenario from \acp{cso} to FR~I galaxies may seem promising, but again, the numbers are off by over a factor 800.

We conclude on the basis of these fractions of \acp{cso} in complete samples, that they do not evolve into any of the above classes of larger jetted-\acp{agn}, and  therefore that they belong to a distinct class of jetted-AGN.

\begin{figure}[!t]
 \centering
 \includegraphics[width=\columnwidth]{NewSizehistogramsPaper22.pdf}
 \caption{ The distributions in size of the bona fide \acp{cso} over the whole \ac{cso} size range, from 0\,pc to 1\,kpc: (a) The heavy black boxes show the histogram of  the sample of 54~bona fide \acp{cso} for which there are spectroscopic redshifts, with the numbers given on the left axis. The dashed curve, marking the border of the shaded region, shows the physical size corresponding to 100 milliarcseconds at the redshift indicated on the right-hand axis. For typical \ac{vlbi} observations at 5\,GHz and above, \acp{cso} in the grayed region to the right of this curve would be hard to observe, so there is a strong selection effect that might account for the drop in numbers of bona fide \acp{cso} with physical size. (b) The 17~bona fide \acp{cso} with spectroscopic redshifts in the complete flux density-limited PR+CJ1+PW sample. Dotted curves show the data binned into 100 pc bins, while solid curves show the data binned into 500 pc bins. }
 \label{plt:histograms}
\end{figure}


\subsection{The Redshift Distribution of CSOs in Complete Samples}\label{sec:redshift}

An independent test of whether or not \acp{cso} are drawn from the same population as the other jetted-\acp{agn} in our complete samples is provided through the redshift distribution. The redshifts are listed in Table \ref{tab:zandsize}.  The redshift distributions of the PR+CJ1 and PW complete samples, and their corresponding \ac{cso} distributions, are shown in \cref{plt:histogramredshift}.

We have carried out the \ac{ks} 2-sample test on the PR+CJ1 sample, the PW sample,  and the PWS sample, with the results given in the  four tests shown in \cref{tab:ksredshift}. The cumulative distributions corresponding to Tests \#1, \#2 and \#3, and their KS statistics, are shown in \cref{plt:KSredshift}~(a), (b) and~(c). In carrying out these tests we have removed the \acp{cso} from the full samples.    The KS statistic is completely determined by the data, but the corresponding p-value depends on the assumptions made in integrating over the parent distribution \citep{1992nrfa.book.....P}. We verified that MATLAB and Numerical Recipes use the same formulae for determining the p-values.  For that reason we use the MATLAB p-values in deriving the significance levels in \cref{tab:ksredshift}.

The first two redshift distribution tests (\#s 1 and 2 in \cref{tab:ksredshift}) show that the probability of the hypothesis that the \acp{cso} and non-\acp{cso} are drawn from the same population is 0.13 for the PR+CJ1 sample;  and 0.03  for the  PW sample. 

\begin{figure*}[!t]
 \centering
 \includegraphics[width=1.0\linewidth]{largeangularsizes.pdf}
 \caption{Demonstration that the complete samples studied in this paper are not restricted by the usual $\sim 100$~milliarcsecond field sizes typical of most \ac{vlbi} observations at 5\,GHz. Shown here are 1.7\,GHz \ac{vlbi} maps of six large angular scale compact AGN from the CJ1 complete sample survey \citep{1995ApJS...98....1P}, all of which have sizes $\gg 100$~milliarcseconds. Note that the structure of 1458+718 (J1459+7140, 3C~309.1) extends over 1\,arc second --- i.e., the map is ten times larger than the typical field of view of \ac{vlbi} maps at 5\,GHz or higher frequencies.}
 \label{plt:largeobjects}. 
\end{figure*}




If we look at the effectively complete PW subsample having S$_{\rm 5\,GHz} >1.3$\,Jy and at declination $\delta < 35^\circ$, which is independent of the PR+CJ1 sample in view of the mutually exclusive declination limits, we see that the  probability is ~0.12. Since these are independent samples, we may legitimately multiply the p-values of Tests~\#1 and~\#3, which yields a probability of $1.6\times 10^{-2}$, which is significant at the $2.1\sigma$ level.

While not at the $3\sigma$ level, these statistics nevertheless provide some independent evidence that \acp{cso} are drawn from a different population compared to that of the other jetted-\acp{agn} in these complete samples. 

This result, which is seen clearly in the redshift distributions shown in \cref{plt:histogramredshift}, is interesting.  If correct, it suggests that \acp{cso} only started forming in significant numbers towards the end of the epoch of maximum galaxy and star formation: The lookback time to the peak in the cosmic star formation rate is $\sim 8$ billion years \citep{2020ARA&A..58..661F}, which is close to the lookback time to  redshift $z \approx 0.9$, when \acp{cso} started to appear in significant numbers, as can be seen in \cref{plt:histogramredshift}(a) and (b). The peak \ac{sfr} occurs from $z \approx 1$ to $z \approx 2$, with the peak \ac{smbh} formation rate \citep{2020ARA&A..58..157T} peaking slightly after the peak \ac{sfr}.

Thus, a possible explanation of the origin of \acp{cso} is that quiescent \acp{smbh} form \acp{cso}  by single star capture, and so becomes significant around $z\sim 1$, when the numbers of both stars and \acp{smbh} in the universe reaches a maximum.  We give a detailed discussion of this hypothesis in Paper~3.

However  results that are significant only at the $\sim  2 \sigma$ level often disappear with the advent of more data, and  this particular  apparent difference between \acp{cso} and other jetted-\ac{agn} may disappear as more bona fide \acp{cso} are accrued through new complete samples.

As a distinct population, and recalling that these are all ``short-lived'' but not all ``young'' sources, it will be of great interest to investigate whether \ac{cso}~2s show the same strong cosmological evolution as do  both high-luminosity  extended steep spectrum sources and compact flat spectrum sources \citep{1981MNRAS.196..611P}, but this is beyond the scope of the present paper.







\section{The Size Distribution of CSO\lowercase{s}}\label{sec:size}


Our third independent test of the hypothesis that \acp{cso} form a distinct class of jetted-\acp{agn} is based on the size distribution of \acp{cso}. This test is more complex and more subject to selection effects than the two tests of the previous section and therefore merits a section of its own.

Selection effects are particularly strong when it comes to consideration of the observed distribution of \ac{cso} sizes, so we discuss first the effectiveness of our approach in dealing with these selection effects, in order to give the reader some confidence in the statistical robustness of our results.


\subsection{The Efficacy of Complete Samples in Dealing with the CSO Size Distribution Selection Effects}\label{sec:efficacy}

The distribution of the physical sizes of the 54~bona fide \acp{cso}, out of our sample of~79, for which we have spectroscopic redshifts is shown in \cref{plt:histograms}~(a). It shows a strong cutoff well below 1\,kpc. However, one has to bear in mind that this sample of 54~bona fide \acp{cso} is a heterogeneous sample gleaned from the literature, and is subject to selection effects. We therefore have to consider carefully whether these selection effects can be eliminated in complete sub-samples of our 54~bona fide \acp{cso}.





\subsubsection{The $\sim 100$ Milliarcsecond Selection Effect}\label{sec:typical}



The first selection effect we consider comes about because the largest angular size that is
measured in most cm-wavelength \ac{vlbi} maps $\sim 100$~milliarcseconds. In \cref{plt:histograms}~(a) we show the upper size cutoff this would impose as a function of redshift. Only \acp{cso} to the left of this curve have angular sizes less than 100~milliarcseconds at the corresponding redshift. Clearly this could well impose a strong selection effect on the observed size distribution of \acp{cso}.

On the face of it, it might appear that this selection effect alone is so strong that the true size distribution of \acp{cso} is impossible to determine from these data. Fortunately this is not the case because one can observe complete samples in which one knows the sizes of all the objects in the sample, and if some objects are too large for \ac{vlbi} mapping at cm wavelengths they can be observed at longer wavelengths, where the $\sim 100\,$mas limit does not apply. We have availed ourselves fully of this strategy: In addition to the observations of compact objects in the PR+CJ1 complete samples at 5\,GHz \citep{1988ApJ...328..114P,1995ApJS...99..297X},   all of these objects were observed at 1.66\,GHz \citep{1995ApJS...98....1P,1995ApJS...99..297X}.
In \cref{plt:largeobjects} we show examples of six AGN from  the CJ1 complete sample with sizes far exceeding the 100\,mas angular size limitation of regular \ac{vlbi} at 5\,GHz and above. As can be seen here, even objects as large as 1 arcsecond were mapped in this survey. This is one of two reasons we can be confident that we have not missed any large \acp{cso} in these complete samples. The other reason is that the large-, by which we mean ($\gtrsim$ 1 arcsec), -and small-scale radio structures of {\it all} of these objects are known, as can be seen in \cref{tab:samples} in the Appendix. 



\subsubsection{Spectral Shape Selection Effect}\label{sec:spectra}

Many CSOs are peaked spectrum (PS) sources\footnote{In this paper we follow the lead of  \citet{2021AandARv..29....3O} in their comprehensive review of peaked spectrum sources, and refer to GPS and MHz peaked spectrum sources as Peaked Spectrum (PS) sources}. Thus in a sample of CSOs selected at a single frequency, we will clearly include all of the sources that peak at that frequency down to the flux density limit.  However, for sources that peak at frequencies significantly higher or lower than the selection frequency, the sample will exclude an increasing number of the CSOs as the separation between the peak frequency and the selection frequency increases.   In this study, we therefore consider not only the PR+CJ1 and PW samples, selected at 5 GHz and 2.7 GHz, but we also consider the GLEAM sample, observed using 20 simultaneous flux density measurements spanning frequencies between 72 MHz and 231 MHz,  the 3CRR sample selected at 178 MHz, and the Jodrell Bank 966 MHz sample.
 
 
%Another selection effect can be introduced by the choice of observing frequency. 

The situation is illustrated in \cref{plt:OQ208}. The blue points show the observed radio spectrum of OQ~208 (J1407+2827) \citep{1997A&A...318..376S}, which has one of the narrowest, most sharply peaked, spectra amongst the bona fide \acp{cso}. The gray points illustrate an object with the same spectral shape as OQ~208, but with the maximum shifted from 5\,GHz down to 1\,GHz, and the peak  flux density shifted down to 1.3\,Jy. This is the point where the object would drop below the GLEAM 1\,Jy limit \citep{2017ApJ...836..174C}, and the CJ1 700\,mJy limit. Because of the drop-off in flux density, relative to the peak,  at both higher and lower frequencies,  such an object would not be included in  the PW, PR, 
 CJ1 or GLEAM samples. Objects of this type with peak flux densities greater than 1.3\,Jy would, however, be included in the GLEAM and CJ1 samples, whose limiting flux densities are indicated by the horizontal brown line in \cref{plt:OQ208}, and the red horizontal bar, respectively.


\begin{figure}[!t]
 \centering
 \includegraphics[width=1.0\linewidth]{OQ208example.pdf}
 \caption{\acp{cso} that might be missed in the PR, CJ1, and PW complete samples due to spectral effects: The blue, green and red arrows indicate the selection frequency and limiting flux densities of the PW, PR, and CJ1 samples, respectively. The horizontal brown line indicates the flux density limit of the GLEAM sample. The blue and grey points show the observed spectrum of OQ~208, and a shifted spectrum of  OQ~208, respectively (see text).}
 \label{plt:OQ208}
\end{figure}








The GLEAM survey provides an excellent complement to the higher frequency \ac{vlbi} surveys that have studied complete samples. From studies of \ac{css} and  \ac{ps} sources, it is clear that the sizes of  PS sources range up to 20\,kpc \citep{2021AandARv..29....3O}.  505 of the 11,400 sources in the complete 1\,Jy GLEAM sample (($4.4 \pm 0.2)$ \%) are peaked-spectrum objects \citep{2017ApJ...836..174C}, with peak flux densities above 1\,Jy in the 72\,MHz -- 700\,MHz range,  and so must also be compact \citep{2021AN....342.1185R}.  

\begin{figure*}[!t]
    \centering
    \includegraphics[width=1.0\linewidth]{BGsample.png}
    \caption{ The distributions of the bona fide CSOs and MSOs as a function of projected physical size in the BG sample. The red distribution shows the BG CSOs+MSOs+CSS$>$20 kpc objects.  The green distribution shows the BG+PW CSOs. }
    \label{plt:bghistograms}
\end{figure*}


It is clear that if the speeds of advance of jetted-\ac{agn} into the interstellar medium are constant, when averaged over sufficient time, the expected fraction of objects which range up to 20\,kpc in size that lie in the 0--1\,kpc range $= 0.05 \times 4.4\% = 0.22\%$. However we see in \cref{tab:csonumbers} that the fraction of \acp{cso} in our 2.7\,GHz and 5\,GHz complete samples lies within the range $(6.4\pm 1.6) \%$ to  $(8.2 \pm 1.8)\%$, and is therefore a factor $\sim 26 \rightarrow 33$ larger than expected (see \S \ref{sec:statistics}). This discrepancy cannot be resolved by allowing for variable  speeds, with the hot spot speeds of advance in the CSO phase  $\lesssim 0.05\, \times $ the hot spot advance speeds of objects in  later phases, since this  is certainly not the case, as discussed above in \S \ref{sec:statistics}.

Our conclusion that CSOs do not evolve into FR I or FR II objects is consistent with interpretation  of \citet{2017ApJ...836..174C}  that the GLEAM PS sources are young precursors of the larger, \ac{css}, FR~I and FR~II, sources, provided that very few of the GLEAM sources are CSOs.


Similarly, this is consistent with a completely independent study of  northern radio sources at low frequencies. The statistics and implications of 373 PS sources, with peaks around 150\,MHz, in the LOFAR surveys have been studied in detail by \citet{2022arXiv221016570S}, who find strong evidence that high-luminosity PS sources, i.e., objects of the type found in the GLEAM survey, evolve into large-scale radio-loud \acp{agn}. 

Note that a small fraction of these GLEAM and LOFAR objects is likely to be \acp{cso} since all classes of larger sources must pass through the \ac{cso} size range in an early phase of their evolution, and those with axes at large angles to the line of sight are likely to have symmetric morphology.  We return to this point in \S \ref{sec:distinct}.

\subsection{A Spectral Shape Lacuna}\label{sec:3CRPW}

As we have seen in previous sections, we are only considering the 17 bona fide CSOs in the PR+CJ2+PW complete sample with known spectroscopic redshifts, and there is only one bona fide CSO in these complete samples  without a spectroscopic redshift. The PR+CJ2+PW complete sample consists of the 282  sources listed in Table \ref{tab:samples}, including M82, so there are 281 AGN in our sample.  In \ref{sec:spectra} we discussed a spectral shape selection effect that can be affecting this sample. 
%We now point out a selection effect lacuna due to the spectral shape of PS CSOs.
The GLEAM survey detected 11,400 sources with flux densities greater than 1 Jy between 70 MHz and 700 MHz.  Of these 505 are PS sources.   In order to double the numbers of CSOs, and hence potentially to have a strong effect on any statistical tests of the size distribution of CSOs, there would need to be $\approx$17 bona fide CSOs in the GLEAM sample.  Thus only a small fraction $\sim 3 \%$ (17/505) of the PS sources in the GLEAM would need to be CSOs in order potentially to have a significant impact on the statistics.  So this is a lacuna that has to be addressed in any size tests.

In the next three subsections we advance two independent arguments to address this lacuna and we suggest a test that could fill the lacuna, but which requires more observations and is  therefore beyond the scope of the present paper. 

\subsubsection{The Range of Peak Frequencies in Our Sample}\label{sec:rangepeak}
  
In Paper 1, Fig. 6, we have plotted the range of peak frequencies observed, and it can be seen  that the peak frequencies range from below 80 MHz to  $\sim 10$ GHz.  The same is true of the objects in our combined PR+CJ1+PW sample - i.e., the lowest peak is below 80 MHz and the highest peak is at  $\sim 10$ GHz, and the distribution of the peaks is roughly uniform between these extremes.  

 Thus the selection procedure of the complete PR+CJ1+PW sample and our bona fide CSO identification method do not appear to have created a bias against CSO peaking anywhere within this range. However, while the (rarer) flat-spectrum CSOs will not suffer from the spectral selection biases described earlier, some peaked-spectrum CSOs could be excluded from the sample for certain redshift ranges.  This could, therefore, influence the size distribution of the observed CSOs in the PR+CJ1+PW samples, particularly if CSO intrinsic size is related to peak frequency and/or luminosity.

 In the next two subsections we first give an argument that shows that spectral shape selection effects are unlikely to have biased the size distribution of the CSOs in the PR+CJ1+PW sample, and we make a prediction that provides a powerful test of the possibility of spectral shape bias, but which is beyond the scope of this paper.

\subsubsection{The 3CRR and PW Complete CSS Double Sample}\label{sec:lowfreq}

In addition to our complete samples of CSOs, described in  the previous sections, there is one other relevant sample of CSOs and MSOs that has been studied extensively by the Bologna Group (BG), the key results of which are given in a series of papers \citep{1985AandA...143..292F,1990AandA...231..333F,1991MNRAS.250..225S,1995AandA...302..317F,1995AA...295...27D,2013MNRAS.433..147D,2021MNRAS.504.2312D}. The BG identified 32 double-lobed CSS objects, given in Table 1 of \citet{1995AandA...302..317F}, in their complete sample drawn from the 3CRR \citep{1983MNRAS.204..151L} and the PW samples. They subsequently added one double-lobed source (1819+396 = 4C $+$39.56) that they had previously missed \citep{2021MNRAS.504.2312D}, bringing the total of double-lobed CSS sources in the BG sample to 33. Given that these are CSS objects, they excluded flat spectrum objects with $\alpha>-0.5$.

 This spectral filter  against flat spectrum sources ($\alpha > -0.5$), as applied to the 3CRR sample, which has a limiting flux density of 10 Jy at 178 MHz, excludes sources
 brighter than 2.57 Jy at 2.7 GHz.  Since these are greater than the flux density limit of the PW complete sample  (1.5 Jy at 2.7 GHz), any such  object can be included in the BG study by adding the flat spectrum PW CSOs to the steep spectrum CSOs detected by the BG, in order to get the total of both flat and steep spectrum compact doubles, including CSOs, in the 3CRR and PW samples (note that there are no flat spectrum doubles of size greater than 1 kpc in PW).  There are four BG CSOs in our sample of 79 bona fide CSOs listed in Paper 1. All four of these BG CSOs are already in our complete sample of CSOs  in the PR+CJ1+PW complete samples. 

 We therefore compensate for the spectral index limit in BG sample by adding the flat spectrum PW bona fide CSOs that were excluded from the BG sample by the spectral index cutoff at $\alpha =-0.5$ to the BG sample of CSOs, thereby bringing the total including the PW flat spectrum CSOs to 40.   In order to apply the same largest angular size filter as that used in Paper 1, we have re-measured the largest angular sizes of the 33 BG CSS double sources at the lowest frequency at which high-quality maps are available in the BG group's publications listed above. The results are shown in Fig. \ref{plt:bghistograms}, together with the 7 flat spectrum PW bona fide CSOs that we have added.

We find that 27 of the 33 objects in the BG sample fit the CSO$+$MSO criteria, with four of the objects, all of which were already in our list from the PR+CJ1+PW complete sample, being bona fide CSOs and the remaining 23 objects being MSOs. The 6 remaining objects  all have largest  projected physical sizes greater than 20 kpc, based on our measured largest angular sizes.  When we add the flat spectrum objects from the PW sample, the number of CSOs increases from 4 to 11, as shown in Fig. \ref{plt:bghistograms}. 

 The four CSOs in the BG CSS sample all have sizes between 300 pc and 500 pc.  These PS CSOs have spectral turnovers that are almost certainly due to synchrotron self absorption as is shown in the paper by \citet{1977MNRAS.180..539S}, who showed that
 the equipartition angular size
$\psi_{_{\rm eq}} \propto  S^{8 \over 17}  \nu^{-{{35+ 2\alpha} \over 34}}   (1+z)^{{15-2\alpha} \over 34}$.  Thus fainter objects that show spectral peaks at higher frequencies will be smaller than the four CSS CSO objects in the BG sample. It therefore is unlikely that there will be a significant number of CSS CSOs with sizes in the 500 pc to 1 kpc range.  We return to this point in \S \ref{sec:compsizes}.

Thus, even though we have now added a sample selected at 178 MHz, we have found that no new CSOs have been discovered, which suggests that our complete PR+CJ1+PW sample has not missed any low-frequency peaked CSOs due to the spectral bias effect.


 As a result, we conclude that we can be fairly confident that we have not missed any CSOs in our complete samples due spectral shape effects. This conclusion is  testable, as described in the next section.



\begin{figure*}[!t]
 \centering
 \includegraphics[width=1.0\linewidth]{newKScompositesize.pdf}
 \caption{KS Tests on the  size distributions of the bona fide \acp{cso} in the PR+CJ1 and PW samples. (a), (b) and (c): comparison of the \ac{cso} cumulative size distributions {\it vs.}  the uniform model for the  PR+CJ1, PW, and PWS samples, respectively. The green bars indicate the maximum differences in the cumulative distributions, corresponding to the values of the KS statistic given by the numbers in green. The corresponding p-values are listed in \cref{tab:kssize}. }
 \label{plt:KSsize}. 
\end{figure*}


\begin{deluxetable*}{c@{\hskip 8mm}cccccccc}
\tablecaption{One-sample KS Tests Against a Uniform Distribution  of \ac{cso} Sizes\label{tab:kssize}}
\tablehead{Test&Complete&Sky& Flux Density&Frequency&KS&p-value & Significance\\
Number&Sample&Area&limit&GHz&statistic&&}
\startdata
5& PR+CJ1&$\delta > 35^\circ, |b|> 10^\circ$&0.7\,Jy &5\,GHz&0.47& $9.3 \times 10^{-3}$ &2.4$\sigma$\\
6&PW&$\delta > 10^\circ, |b|> 10^\circ$&1.5\,Jy&2.7\,GHz&0.54& $8.7 \times 10^{-4}$ & 3.1$\sigma$\\
7& PWS&$10^\circ<\delta < 35^\circ, |b|> 10^\circ$&1.3\,Jy&5\,GHz &0.64& $1.7 \times 10^{-2}$ & 2.1$\sigma$\\
8& PR+CJ1+PWS&-&- &-&-& $1.6 \times 10^{-4}$ & 3.6$\sigma$\\
%5& VIPS&SDSS footprint&0.085\,Jy &5\,GHz&?& $X \times 10^{-y}$ & z\\
\enddata
\tablecomments{Tests of the observed \ac{cso} size distribution compared to a uniform size distribution. The PWS sample is the effectively complete subsample of the PW sample having $10^\circ<\delta < 35^\circ, |b|> 10^\circ$ and  S$_{\rm 5\,GHz} \ge 1.3 $\,Jy (see text).   }
\end{deluxetable*}

\begin{deluxetable*}{c@{\hskip 8mm}cccccccc}
\tablecaption{Binomial Tests of Significance Levels of \ac{cso} Size Distributions in Complete Samples}
\tablehead{Test&Complete&Sky& $\alpha$& N&N&p-value & Significance\\
Number&Sample&Area&limit&$0-500$\,pc&500\,pc-1\,kpc&&}
\startdata
9& PR+CJ1&$\delta > 35^\circ, |b|> 10^\circ$&- & 11& 1& $5.4 \times 10^{-3}$ & 2.6$\sigma$\\
10& PWS&$\delta < 35^\circ, |b|> 10^\circ$&-&6&0& $3.1 \times 10^{-2}$ & 1.9$\sigma$\\
11& PR+CJ1+PWS&-&- & -& -& $1.7 \times 10^{-4}$ & 3.6$\sigma$\\
\enddata
\tablecomments{ The PR+CJ1 and PWS ($\delta<35^\circ, \; {\rm S_5>1.3 \,Jy}$) samples are independent, so we have multiplied their p-values in Test 11 (see text). }
\label{tab:histograms}
\end{deluxetable*} 


\subsubsection{A Prediction Based on the Jodrell Bank 966 MHz Sample}\label{sec:jodrell}

Since the 3CRR sample is complete down to 10 Jy at 178 MHz \citep{1983MNRAS.204..151L},  for comparison with the other samples in this study it would be helpful to have a low frequency sample complete down to $\sim 1$ Jy. Fortunately, such a sample exists for which the radio structures of over 98\% of the objects are known.

Referring back to Fig. \ref{plt:OQ208} and the objects in the lacuna illustrated by the gray spectrum. We can define a complete sample drawn from the Jodrell Bank 966 MHz survey \citep{1977MmRAS..84....1C}, which produced  a radio catalogue and measured arcsec-level positions for the majority of its sources.  We have selected a  sub-sample, consisting of 169 of the strongest sources (S$_{0.966} > 1.5$ Jy)  from \citet{1977MmRAS..84....1C}.   This sub-sample is unbiased, and while the full survey is  not strictly complete due to confusion issues, these apply only at  low flux density levels, and thus the sub-sample  that we have selected is not affected by confusion. We will refer to this unbiased sub-sample of 169 objects as the ``JBS'' sample.

We classified 74 of the JBS sample in the filtering process we carried out in selecting our bona fide CSOs described in paper 1.  We identified six of them as bona fide CSOs.

We have extracted VLASS cutout images of all 169 JBS objects using the CIRADA cutout server\footnote{http://cutouts.cirada.ca}, and we found only 17 of them to be unresolved, with largest angular size $<3$ arc sec, and hence possible CSOs.  Of these 17 compact objects, two are MOJAVE ``core-jet'' objects, one is an FR II object, one is a nearby double-lobed Markarian galaxy, and one is a 2 arc second double. Thus there are 12 possible CSOs in the 966 MHz JBS sample in addition  to the 6 bona fide CSOs we have already identified.

Our prediction, based on the argument of the previous section, is that very few, if any, of these 12 objects will be found to be CSOs with sizes greater than 500 pc.  This is testable with VLBI observations, but beyond the scope of the present paper.

 
\subsection{CSO Sizes in the Complete Samples}\label{sec:compsizes}
 As it happens, {\it all} of the bona fide \acp{cso} in the complete PR, CJ1 and PW samples are \ac{cso}~2s --- i.e., as discussed in Paper~1, they are edge-brightened, high-luminosity objects.
 The size distributions of the PR+CJ1+PW complete samples are shown in \cref{plt:histograms}(b), binned into 100 pc and 500 pc intervals.

Using the \ac{cso} size distributions in the PR+CJ1 and PW complete samples, we have carried out two sets of statistical tests of the hypothesis that \acp{cso} are uniformly distributed in size between 0\,pc and 1000\,pc, as would be expected on the hypothesis that the speed of advance is constant: (i) a set of \ac{ks} 1-sample tests, which yield the cumulative distributions shown in \cref{plt:KSsize}~(a), (b) and~(c) and the p-values shown in Tests \#5 - \#8 in \cref{tab:kssize}; and (ii) binomial tests in which we divided the \acp{cso} into two size bins, from 0\,pc to 500\,pc, and from 500\,pc to 1000\,pc, which yield the results shown in \cref{tab:histograms}.

\begin{figure*}[!t]
 \centering
 \includegraphics[width=1.0\linewidth]{composite.pdf}
 \caption{Example of an \ac{mso} and a \ac{cso}, with similar morphologies.  The MSO:(a) from \citet{1995ApJS...99..297X}, and (b) from \citet{1996ApJ...463...95T}:  0404+768 (J0410+7656), which these authors had classified as a \ac{cso}. The CSO:  from \citet{1995AA...295...27D}, and (d) from \citet{1996ApJ...463...95T}:  1358+624 (J1400+6210). Some \acp{cso} must evolve into \acp{mso}. and 1358+624 may be just such a case. The red cross marks the location of the core in each map.}
 \label{plt:csomso}. 
\end{figure*}

We consider first the \ac{ks} tests shown in \cref{tab:kssize}.  We see there that the uniform hypothesis is rejected by the PR+CJ1 \ac{cso} sample at the $9.3 \times 10^{-3}$ probability level, and by the PW \ac{cso} sample at the $8.7 \times 10^{-4}$ probability level. The independent, effectively complete, PW \ac{cso} sample below declination $35^\circ$ (Test \#7) rejects the uniform hypothesis at the $1.7 \times 10^{-2}$ level.

 Tests \#5 and \#7 are independent -- note the different sky areas -- and at the same observing frequency. We can therefore, legitimately, multiply their p-values. This gives Test \#8, which rejects the uniform size distribution hypothesis at the p-value $1.6 \times 10^{-4}$, or $3.6 \sigma$ significance level.

The results of the binomial tests, shown in \cref{tab:histograms}, show that by similarly combining the PR+CJ1 \ac{cso} sample (Test \#9)  with the independent PWS sample (Test \#10), as shown in Test \#11, the uniform hypothesis is rejected at the $1.7\times 10^{-4}$ ($3.6 \sigma$) level.

In our view, these tests on complete samples constitute strong evidence that the size distribution of \acp{cso} cuts off sharply at $\approx 500$ pc, which is significantly below the 1\,kpc size limit imposed by the defining criteria of \acp{cso}.   As described in \S \ref{sec:jodrell}, the existence of this sharp cutoff can be tested, for example, with VLBI observations of the 12 potential CSOs in the complete JBS sample.




Clearly {\it some} \acp{cso} must grow to larger sizes in order to produce \acp{mso}. An example of a \ac{cso} and \ac{mso} with remarkably similar morphologies is shown in \cref{plt:csomso}, which illustrates this point, especially since 0404+768 (J0410+7656) was originally classified as a \ac{cso}, but fails the  size cutoff (by $\sim$20\%). It is clear that  the majority of \acp{cso} do not grow much above 500\,pc in projected size.





\section{A Distinct Class of Jetted-AGN that cuts of sharply at $\approx$ 500 \lowercase{pc}}\label{sec:distinct}

The number fraction, redshift, and the size statistics discussed in the previous two sections and presented in \cref{tab:ksredshift,tab:kssize,tab:histograms} demonstrate that (i) \acp{cso} are drawn from a population of jetted-\acp{agn} that is distinct from other jetted-\acp{agn}, and (ii) the size distribution of CSOs cuts off sharply at $\approx 500$ pc, a finding that is verifiable (see \S \ref{sec:jodrell}).  



These are  significant findings. They show that there has to be something fundamentally different between \acp{cso} and other jetted-\ac{agn} and the larger sources. While both must be driven by the same type of central engine, since both are producing high-luminosity relativistic jets, there must be some fundamental difference between them to produce two such different outcomes -- one with a size cutoff around 500 pc and the other with a size cutoff $\sim 100 \times$ larger.  

One might think, for example, that the cutoff could be explained simply by random episodic fuelling. But how would random episodic fuelling produce a sharp cutoff? Random episodic fuelling would produce a uniform distribution. There has to be another explanation for the cutoff, such as, for example, an upper limit on the size of the fuel packages,  a change in the jet environment that leads the jets to fade  beyond a certain distance from the central engine, or a mechanism associated with the accretion disk that limits the energy of the jet. 











\section{Discussion}\label{sec:discussion}


Although the selection effects inherent in our literature search for \acp{cso} are significant, we have shown that, with careful use of complete samples, and using the 17~\acp{cso} in the complete PR, CJ1, and PW samples for which we have spectroscopic redshifts, it is possible to carry out a series of rigorous statistical tests that provide what we regard as compelling evidence that \acp{cso} constitute a population of jetted-\acp{agn} that is distinct from, and therefore requires a separate origin to, the larger classes of jetted-\acp{agn}, such as \ac{css} sources,  \acp{mso}, and FR~I and FR~II objects.  

The physical size cutoff is clearly telling us something important about this class of jetted-\acp{agn}. The scenario that produces \acp{cso} must be different in some important way from that which produces the larger symmetric radio sources. We return to this discussion of the origins of \acp{cso} in Paper~3.  

It should be clear, therefore, that, (i) because the observed emission regions in these objects are not relativistically boosted towards the observer, thereby making it possible to determine their detailed physical properties, and (ii) they belong to a distinct class of jetted-\acp{agn}, \acp{cso} provide an important window on the central engines of jetted-\acp{agn} and the supermassive black holes that drive them.  For these reasons it is very important that in the future workers in this area are careful to apply the variability and speed selection criteria, before assigning a jetted-\ac{agn} to the \ac{cso} class.


%------------------------------------------------------------------------------






%------------------------------------------------------------------------------
\acknowledgments


We thank John Peacock for useful discussions. We are grateful for the use of the CATS database of \citet{2005BSAO...58..118V}, of the Special Astrophysical Observatory. 
% ADS:
This research has made use of NASA’s Astrophysics Data System Bibliographic Services.
% NED:
This research has made use of the NASA/IPAC Extragalactic Database (NED) which is operated by the Jet Propulsion Laboratory, California Institute of Technology, under contract with the National Aeronautics and Space Administration.
% OVRO:
This research has made use of data from the OVRO 40-m monitoring program (Richards, J. L. et al. 2011, ApJS, 194, 29), supported by private funding from the California Insitute of Technology and the Max Planck Institute for Radio Astronomy, and by NASA grants NNX08AW31G, NNX11A043G, and NNX14AQ89G and NSF grants AST-0808050, AST-1109911, and AST-1835400.
% MOJAVE:
This research has made use of data from the MOJAVE database that is maintained by the MOJAVE team \citep{MOJAVE_XV}. The MOJAVE program is supported by NASA-{\it Fermi} grant 80NSSC19K1579.
% CIRADA:
This research has made use of the CIRADA cutout service at http://cutouts.cirada.ca, operated by the Canadian Initiative for Radio Astronomy Data Analysis (CIRADA). CIRADA is funded by a grant from the Canada Foundation for Innovation 2017 Innovation Fund (Project 35999), as well as by the Provinces of Ontario, British Columbia, Alberta, Manitoba and Quebec, in collaboration with the National Research Council of Canada, the US National Radio Astronomy Observatory and Australia’s Commonwealth Scientific and Industrial Research Organisation.
% SK
S.K. and K.T. acknowledge support from the European Research Council (ERC) under the European Unions Horizon 2020 research and innovation programme under grant agreement No.~771282.
% KT
KT acknowledges support from the Foundation of Research and Technology - Hellas Synergy Grants Program through project POLAR, jointly implemented by the Institute of Astrophysics and the Institute of Computer Science.
%AS
A.S. was supported by the NASA Contract NAS8-03060 to the Chandra X-ray Center.

This paper depended on a very large amount of \ac{vlbi} data, almost all of which was taken with the Very Long Baseline Array. The National Radio Astronomy Observatory is a facility of the National Science Foundation operated under cooperative agreement by Associated Universities, Inc.



\appendix




\startlongtable
\begin{deluxetable*}{lllccccccccc}
\tablecaption{Structure Types and Literature References for the Combined PR, CJ1 and PW$^\dag$ Samples
\label{tab:samples}}
\tablehead{
    B1950 Name&J2000 Name&Alias&PR&CJ1&PW&Type&CSOcat&Optical&Large-scale&Small-scale\\
    &&&&&&&ID\#&ID$^\ddag$&Structure&Structure\\
    (1)&(2)&(3)&(4)&(5)&(6)&(7)&(8)&(9)&(10)&(11)}
\startdata
 0010+775 	&	 J0013+7748 	&	          S5 0010+77&	   	&	 Y 	&	   	&	 CT,CSS  	&	1996	&	  G 	&	     	&	 41,42  	\\
 0010+405 	&	 J0013+4051 	&	          4C +40.01 &	   	&	 Y 	&	   	&	 CT,CFS  	&	2903	&	  G 	&	     	&	 41,42  	\\
 0013+790 	&	 J0016+7916 	&	            3C~6.1 	&	   	&	 Y 	&	 Y 	&	 FR~II  	&	2904	&	  G 	&	   3,9,62 	&	        	\\
 0016+731 	&	 J0019+7327 	&	           S5 0016+73&	 Y 	&	   	&	 Y 	&	  U,CFS 	&	2906	&	  Q 	&	9	&	31	\\
 0022+390 	&	 J0025+3919 	&	           S4 0022+39&	   	&	 Y 	&	   	&	 CT,CFS  	&	2907	&	  Q 	&	     	&	 41,42  	\\
  0026+34 	&	 J0029+3456 	&	        B2 0026+34 	&	   	&	   	&	 Y 	&	 CSO 	&	6	&	  G 	&	    	&	    58, Paper~1  	\\
  0038+32 	&	 J0040+3310 	&	             3C~19 	&	   	&	   	&	 Y 	&	 FR~II  	&	3145	&	  G 	&	   8,9,62 	&	        	\\
 0040+517 	&	 J0043+5203 	&	             3C~20 	&	 Y 	&	   	&	 Y 	&	 FR~II  	&	2911	&	  G 	&	   8,9,64 	&	        	\\
 0048+509 	&	 J0050+5112 	&	            3C 22.0&	   	&	 Y 	&	   	&	 FR~II  	&	2912	&	  G 	&	8	&	        	\\
 0102+480 	&	 J0105+4819 	&	                   &	   	&	 Y 	&	   	&	 CT,CFS  	&	323	&	    	&	     	&	 41,42  	\\
  0104+32 	&	 J0107+3224 	&	             3C~31 	&	   	&	   	&	 Y$^\dag$ 	&	  FR~I 	&	2915	&	  G 	&	   8,9 	&	        	\\
  0106+13 	&	 J0108+1320 	&	             3C~33 	&	   	&	   	&	 Y$^\dag$ 	&	  FR~II 	&	2916	&	  G 	&	  9,34 	&	        	\\
 0106+729 	&	 J0109+7311 	&	           3C~33.1 	&	   	&	 Y 	&	 Y 	&	 FR~II  	&	2917	&	  G 	&	   8,9 	&	        	\\
 0108+388 	&	 J0111+3906 	&	          S4 0108+388 	&	 Y 	&	   	&	   	&	 CSO 	&	11	&	  G 	&	     	&	    31, Paper~1  	\\
  0116+31 	&	 J0119+3210 	&	        4C 31.04 	&	   	&	   	&	 Y$^\dag$ 	&	 CSO 	&	12	&	  G 	&	     	&	    35, Paper~1  	\\
  0123+32 	&	 J0126+3313 	&	             3C~41 	&	   	&	   	&	 Y$^\dag$ 	&	 FR~II  	&	3146	&	  G 	&	   5,9,62 	&	        	\\
  0125+28 	&	 J0128+2903 	&	             3C~42 	&	   	&	   	&	 Y 	&	 FR~II  	&	1830	&	  G 	&	   8,9 	&	        	\\
  0127+23 	&	 J0129+2338 	&	             3C~43 	&	   	&	   	&	 Y 	&	 CSS  	&	1298	&	  Q 	&	  9,48 	&	        	\\
  0133+20 	&	 J0136+2057 	&	             3C~47 	&	   	&	   	&	 Y 	&	 FR~II  	&	3147	&	  Q 	&	   3,9 	&	        	\\
 0133+476 	&	 J0136+4751 	&	            OC 457 	&	 Y 	&	   	&	 Y 	&	 U,CFS  	&	324	&	  Q 	&	9	&	31	\\
  0134+32 	&	 J0137+3309 	&	             3C~48 	&	   	&	   	&	 Y$^\dag$ 	&	  CSS 	&	540	&	  Q 	&	  9,48 	&	        	\\
  0138+13 	&	 J0141+1353 	&	             3C~49 	&	   	&	   	&	 Y 	&	 CSS  	&	587	&	  G 	&	  9,48 	&	        	\\
 0153+744 	&	 J0157+7442 	&	           S5 0153+744&	 Y 	&	   	&	 Y 	&	 U,CFS  	&	325	&	  Q 	&	9	&	31	\\
  0202+14 	&	 J0204+1514 	&	          4C~15.05 	&	   	&	   	&	 Y$^\dag$ 	&	 U,CFS  	&	3148	&	 Q$^a$ 	&	9	&	58	\\
 0206+355 	&	 J0209+3547 	&	            4C +35.03&	   	&	 Y 	&	   	&	 FR~I  	&	2924	&	  G 	&	3	&	        	\\
 0212+735 	&	 J0217+7349 	&	            S5 0212+73&	 Y 	&	   	&	 Y 	&	 U,CFS  	&	326	&	  Q 	&	9	&	31	\\
 0218+357 	&	 J0221+3556 	&	           B2 0218+357&	   	&	 Y 	&	   	&	  CT,CFS 	&	871	&	  Q 	&	     	&	 41,42  	\\
 0210+860 	&	 J0222+8619 	&	            3C~61.1 	&	 Y 	&	   	&	 Y 	&	 FR~II  	&	2926	&	  G 	&	  9,34 	&	        	\\
 0220+427 	&	 J0223+4259 	&	            3C~66B 	&	 Y 	&	   	&	 Y 	&	 FR~I  	&	2927	&	  G 	&	  9,46 	&	        	\\
 0220+397 	&	 J0223+4000 	&	             3C~65 	&	   	&	 Y 	&	 Y 	&	 FR~II  	&	2928	&	  G 	&	   5,9 	&	        	\\
  0221+27 	&	 J0224+2750 	&	             3C~67 	&	   	&	   	&	 Y 	&	 CSS  	&	600	&	  G 	&	  9,48 	&	        	\\
  0223+34 	&	 J0226+3421 	&	          4C~34.07 	&	   	&	   	&	 Y$^\dag$ 	&	 CSS  	&	591	&	 Q$^a$ 	&	  9,57,56 	&	        	\\
  0235+16 	&	 J0238+1636 	&	           PKS 0235+164 	&	   	&	   	&	 Y$^\dag$ 	&	 U,CFS  	&	1494	&	  Q 	&	9	&	58	\\
 0248+430 	&	 J0251+4315 	&	           B3 0248+430	&	   	&	 Y 	&	   	&	 CT,CFS  	&	560	&	  Q 	&	     	&	 41,42  	\\
 0258+350 	&	 J0301+3512 	&	            NGC 1167&	   	&	 Y 	&	   	&	  FR~I 	&	350	&	  G 	&	28	&	        	\\
  0300+16 	&	 J0303+1626 	&	           3C~76.1 	&	   	&	   	&	 Y$^\dag$ 	&	 FR~I  	&	2933	&	  G 	&	  9,34 	&	        	\\
  0307+16 	&	 J0310+1705 	&	             3C~79 	&	   	&	   	&	 Y$^\dag$ 	&	 FR~II  	&	3149	&	  G 	&	  9,23,62,64 	&	        	\\
 0307+444 	&	 J0310+4435 	&	            4C 44.07&	   	&	 Y 	&	   	&	   CSS	&	1822	&	  Q 	&	53	&	        	\\
 0309+390 	&	 J0312+3916 	&	           4C 39.11&	   	&	 Y 	&	   	&	 CT,CSS  	&	2934	&	  G 	&	     	&	59	\\
 0314+416 	&	 J0318+4151 	&	          3C~83.1B 	&	 Y 	&	   	&	 Y 	&	 FR~I  	&	2936	&	  G 	&	   6,9 	&	        	\\
  0316+16 	&	 J0318+1628 	&	          CTA 21&	   	&	   	&	 Y$^\dag$ 	&	 CSS  	&	28	&	  G 	&	9	&	58	\\
 0316+413 	&	 J0319+4130 	&	             3C~84 	&	 Y 	&	   	&	 Y 	&	 U,CFS  	&	2937	&	  G 	&	  9,33 	&	61,56	\\
  0319+12 	&	 J0321+1221 	&	           PKS 0319+12 	&	   	&	   	&	 Y 	&	 CSS  	&	557	&	 Q$^a$ 	&	9	&	54,56	\\
  0356+10 	&	 J0358+1026 	&	             3C~98 	&	   	&	   	&	 Y$^\dag$ 	&	 FR~II  	&	2942	&	  G 	&	  9,47 	&	        	\\
  0400+25 	&	 J0403+2600 	&	           CTD 26 	&	   	&	   	&	 Y$^\dag$ 	&	 U,CFS  	&	872	&	  Q 	&	9	&	58	\\
 0402+379 	&	 J0405+3803 	&	       4C +37.11  &	   	&	 Y 	&	   	&	 CSO 	&	33	&	  G 	&	     	&	 41,42, Paper~1  	\\
 0404+768 	&	 J0410+7656 	&	          4C~76.03 	&	 Y 	&	   	&	 Y 	&	  CSS 	&	34	&	  G 	&	  9,57 	&	        56	\\
 0407+747 	&	 J0413+7451 	&	          4C~74.08 	&	   	&	 Y 	&	 Y 	&	 FR~II  	&	2944	&	 G? 	&	  9,19 	&	        	\\
  0410+11 	&	 J0413+1112 	&	            3C~109 	&	   	&	   	&	 Y$^\dag$ 	&	 FR~II  	&	3150	&	  G 	&	  9,39 	&	        	\\
  0411+14 	&	 J0414+1416 	&	          4C~14.11 	&	   	&	   	&	 Y 	&	 FR~II  	&	3151	&	  G 	&	  9,34,64 	&	        	\\
  0428+20 	&	 J0431+2037 	&	            PKS 0428+20 	&	   	&	   	&	 Y$^\dag$ 	&	  U,CFS 	&	40	&	 G$^a$	&	9	&	54,56	\\
  0433+29 	&	 J0437+2940 	&	            3C~123 	&	   	&	   	&	 Y$^\dag$ 	&	 FR~II  	&	3152	&	  G 	&	  9,52,62,64 	&	        	\\
  0453+22 	&	 J0456+2249 	&	            3C~132 	&	   	&	   	&	 Y 	&	 FR~II  	&	3153	&	  G 	&	   8,9,62,64	&	        	\\
 0454+844 	&	 J0508+8432 	&	             S5 0454+84&	 Y 	&	   	&	   	&	  CT,CFS 	&	327	&	  Q 	&	     	&	31	\\
  0518+16 	&	 J0521+1638 	&	            3C~138 	&	   	&	   	&	 Y$^\dag$ 	&	 CSS  	&	581	&	  Q 	&	  9,48 	&	57	\\
  0528+13 	&	 J0530+1331 	&	            PKS 0528+134 	&	   	&	   	&	 Y$^\dag$ 	&	 U,CFS  	&	681	&	 Q$^a$ 	&	9	&	58	\\
 0538+498 	&	 J0542+4951 	&	            3C~147 	&	 Y 	&	   	&	 Y 	&	 CSS  	&	348	&	  Q 	&	  9,48 	&	        	\\
 0602+673 	&	 J0607+6720 	&	                   	&	   	&	 Y 	&	   	&	 CT,CFS  	&	2950	&	  Q 	&	     	&	 41,42  	\\
 0605+480 	&	 J0609+4804 	&	            3C~153 	&	 Y 	&	   	&	 Y 	&	 FR~II  	&	2951	&	  G 	&	   3,9,62,64 	&	        	\\
 0620+389 	&	 J0624+3856 	&	            S4 0620+389	&	   	&	 Y 	&	   	&	  CT,CFS 	&	2954	&	  Q 	&	     	&	 41,42  	\\
 0615+820 	&	 J0626+8202 	&	             S5 0615+82	&	   	&	 Y 	&	   	&	 CT,CFS  	&	481	&	  Q 	&	     	&	58	\\
 0642+449 	&	 J0646+4451 	&	             OH 471	&	   	&	 Y 	&	   	&	 CT,CSS  	&	328	&	  Q 	&	     	&	 41,42  	\\
 0646+600 	&	 J0650+6001 	&	             4 0646+60	&	   	&	 Y 	&	   	&	 CT,CFS  	&	49	&	  Q 	&	     	&	 41,42  	\\
 0650+371 	&	 J0653+3705 	&	             S4 0650+371 	&	   	&	 Y 	&	   	&	 CT,CFS  	&	329	&	  Q 	&	     	&	 41,42  	\\
 0651+542 	&	 J0655+5408 	&	            3C~171 	&	   	&	 Y 	&	 Y 	&	 FR~II  	&	2958	&	  G 	&	  9,49,64 	&	        	\\
 0703+426 	&	 J0706+4230 	&	          4C~42.23 	&	   	&	 Y 	&	 Y 	&	 FR~I  	&	3154	&	  G 	&	  9,19 	&	        	\\
 0702+749 	&	 J0709+7449 	&	            3C 173.1	&	   	&	 Y 	&	   	&	  FR~II 	&	2960	&	  G 	&	34	&	        	\\
 0707+476 	&	 J0710+4732 	&	           S4 0707+47&	   	&	 Y 	&	   	&	 CT,CFS  	&	2961	&	  Q 	&	     	&	 41,42  	\\
 0707+689 	&	 J0713+6852 	&	           4C 68.08	&	   	&	 Y 	&	   	&	  CSS 	&	1811	&	  Q 	&	45	&	        	\\
 0710+439 	&	 J0713+4349 	&	       B3 0710+439 	&	 Y 	&	   	&	 Y 	&	 CSO 	&	52	&	  G 	&	     	&	    31,Paper~1  	\\
 0711+356 	&	 J0714+3534 	&	            OI 318 	&	 Y 	&	   	&	 Y 	&	 U,CFS  	&	738	&	  Q 	&	9	&	31	\\
 0716+714 	&	 J0721+7120 	&	           TXS 0716+714 	&	   	&	 Y 	&	   	&	 CT,CFS  	&	2963	&	  Q 	&	     	&	 41,42  	\\
 0723+679 	&	 J0728+6748 	&	            3C~179 	&	 Y 	&	   	&	 Y 	&	 FR~II  	&	2965	&	  Q 	&	  9,19,62 	&	        	\\
  0735+17 	&	 J0738+1742 	&	            OI 158 	&	   	&	   	&	 Y$^\dag$ 	&	 U,CFS  	&	1515	&	  Q 	&	9	&	58	\\
  0738+31 	&	 J0741+3112 	&	            OI 363 	&	   	&	   	&	 Y$^\dag$ 	&	  U,CFS 	&	59	&	  Q 	&	9	&	58	\\
 0734+805 	&	 J0743+8026 	&	          3C~184.1 	&	   	&	 Y 	&	 Y 	&	 FR~II  	&	2966	&	  G 	&	  9,47 	&	        	\\
  0742+10 	&	 J0745+1011 	&	            OI 471 	&	   	&	   	&	 Y$^\dag$ 	&	  U,CFS 	&	61	&	 EF$^a$ 	&	9	&	58	\\
  0744+55 	&	 J0748+5548 	&	            DA~240 	&	   	&	   	&	 Y 	&	 FR~?  	&	3155	&	  G 	&	  9,24 	&	        	\\
 0746+483 	&	 J0750+4814 	&	          S4 0746+483	&	   	&	 Y 	&	   	&	 CT,CFS  	&	330	&	  Q 	&	     	&	 41,42  	\\
  0748+12 	&	 J0750+1231 	&	            OI 280 	&	   	&	   	&	 Y$^\dag$ 	&	 U,CFS  	&	3156	&	  Q 	&	9	&	58	\\
 0740+828 	&	 J0750+8241 	&	            S5 0740+82&	   	&	 Y 	&	   	&	 CT,CSS  	&	2968	&	  Q 	&	     	&	 41,42  	\\
 0755+379 	&	 J0758+3747 	&	          NGC 2484	&	   	&	 Y 	&	 Y 	&	 CSS  	&	2969	&	  G 	&	9	&	 41,42  	\\
  0802+24 	&	 J0805+2409 	&	            3C~192 	&	   	&	   	&	 Y$^\dag$ 	&	 FR~II  	&	2973	&	  G 	&	  9,47 	&	        	\\
 0804+499 	&	 J0808+4950 	&	            OJ 508 	&	 Y 	&	   	&	 Y 	&	  U,CFS 	&	2331	&	  Q 	&	9	&	31	\\
 0805+410 	&	 J0808+4052 	&	            S4 0805+41	&	   	&	 Y 	&	   	&	  CT,CFS 	&	2332	&	  Q 	&	     	&	 41,42  	\\
 0809+483 	&	 J0813+4813 	&	            3C~196 	&	 Y 	&	   	&	 Y 	&	 FR~II  	&	2974	&	  Q 	&	9.25	&	        	\\
 0812+367 	&	 J0815+3635 	&	            B2 0812+36 	&	   	&	 Y 	&	   	&	 CT,CFS  	&	2342	&	  Q 	&	     	&	 41,42  	\\
 0814+425 	&	 J0818+4222 	&	            OJ 425 	&	 Y 	&	   	&	 Y 	&	 U,CFS  	&	2345	&	  Q 	&	9	&	31	\\
 0816+526 	&	 J0819+5232 	&	           4C 52.18	&	   	&	 Y 	&	   	&	 FR~II  	&	2975	&	  G 	&	13	&	        	\\
 0818+472 	&	 J0821+4702 	&	      3C~197.1             	&	   	&	 Y 	&	   	&	  FR~II 	&	2976	&	  G 	&	   4,9,62 	&	        	\\
 0820+560 	&	 J0824+5552 	&	          OJ 535	&	   	&	 Y 	&	   	&	  CT,CFS 	&	2351	&	  Q 	&	     	&	 41,42  	\\
 0821+394 	&	 J0824+3916 	&	        4C +39.23	&	   	&	 Y 	&	   	&	 CT,CFS  	&	2352	&	  Q 	&	     	&	 41,42  	\\
 0827+378 	&	 J0831+3742 	&	        B2 0827+37	&	   	&	 Y 	&	   	&	  CT,CSS 	&	2357	&	  Q 	&	     	&	 41,42  	\\
 0828+493 	&	 J0832+4913 	&	        OJ 448 	&	   	&	 Y 	&	   	&	 CT,CFS  	&	331	&	  Q 	&	     	&	 41,42  	\\
 0831+557 	&	 J0834+5534 	&	          4C~55.16 	&	 Y 	&	   	&	 Y 	&	 U,CFS  	&	79	&	  G 	&	9	&	31	\\
 0833+585 	&	 J0837+5825 	&	           S4 0833+58 	&	   	&	 Y 	&	   	&	  CT,CFS 	&	2109	&	  Q 	&	     	&	 41,42  	\\
  0838+13 	&	 J0840+1312 	&	            3C~207 	&	   	&	   	&	 Y$^\dag$ 	&	 FR~II  	&	3157	&	  Q 	&	   3,9,62 	&	        	\\
 0836+710 	&	 J0841+7053 	&	          4C~71.07 	&	 Y 	&	   	&	 Y 	&	 U,CFS  	&	2980	&	 Q$^a$ 	&	9	&	31	\\
 0844+540 	&	 J0847+5352 	&	          NGC 2656	&	   	&	 Y 	&	   	&	  FR~I 	&	2981	&	  G 	&	12	&	        	\\
 0850+581 	&	 J0854+5757 	&	          4C~58.17 	&	 Y 	&	   	&	   	&	  CT,CFS 	&	2384	&	  Q 	&	     	&	31	\\
  0851+20 	&	 J0854+2006 	&	            OJ 287 	&	   	&	   	&	 Y$^\dag$ 	&	  U,CFS 	&	1447	&	  Q 	&	9	&	58	\\
 0859+470 	&	 J0903+4651 	&	          4C~47.29 	&	 Y 	&	   	&	 Y 	&	 U,CFS  	&	353	&	  Q 	&	9	&	  n,31  	\\
 0900+428 	&	 J0904+4238 	&	          B3 0900+428 	&	   	&	 Y 	&	   	&	 CT,CFS  	&	2394	&	  G 	&	     	&	 41,42  	\\
 0906+430 	&	 J0909+4253 	&	            3C~216 	&	 Y 	&	   	&	 Y 	&	 U,CFS  	&	332	&	  Q 	&	  9,48,62	&	31	\\
 0917+449 	&	 J0920+4441 	&	           TXS 0917+449 	&	   	&	 Y 	&	   	&	 CT,CFS  	&	2416	&	  Q 	&	     	&	 41,42  	\\
 0917+458 	&	 J0921+4538 	&	            3C~219 	&	 Y 	&	   	&	 Y 	&	 FR~II  	&	2988	&	  G 	&	  9,36 	&	        	\\
 0917+624 	&	 J0921+6215 	&	           OK 630	&	   	&	 Y 	&	   	&	  CT,CFS 	&	2417	&	  Q 	&	     	&	 41,42  	\\
 0923+392 	&	 J0927+3902 	&	          4C~39.25 	&	 Y 	&	   	&	 Y 	&	 U,CFS  	&	748	&	  Q 	&	9	&	60	\\
 0936+361 	&	 J0939+3553 	&	            3C~223 	&	   	&	 Y 	&	 Y 	&	 FR~II  	&	2990	&	  G 	&	  9,47 	&	        	\\
 0938+399 	&	 J0941+3944 	&	           3C 223.1 	&	   	&	 Y 	&	   	&	 Complex  	&	2991	&	  G 	&	50	&	        	\\
  0939+14 	&	 J0942+1345 	&	           3C~225B 	&	   	&	   	&	 Y 	&	 FR~II  	&	3158	&	  G 	&	   8,9 	&	        	\\
 0945+408 	&	 J0948+4039 	&	          4C~40.24 	&	 Y 	&	   	&	   	&	  CT,CFS 	&	2439	&	  Q 	&	     	&	31	\\
 0945+664 	&	 J0949+6615 	&	          4C~66.09 	&	   	&	 Y 	&	 Y 	&	 U,CFS  	&	100	&	  G 	&	9	&	 41,42  	\\
  0945+73 	&	 J0949+7314 	&	4C 73.08                   	&	   	&	   	&	 Y 	&	 FR~II  	&	3159	&	  G 	&	7	&	        	\\
  0947+14 	&	 J0950+1420 	&	            3C~228 	&	   	&	   	&	 Y 	&	 FR~II  	&	3160	&	  G 	&	   8,9,62 	&	        	\\
 0951+699 	&	 J0955+6940 	&	      M82$^*$ (3C~231) 	&	 Y 	&	   	&	 Y 	&	 FR~?  	&	2992	&	  G 	&	  9,10 	&	        	\\
 0954+556 	&	 J0957+5522 	&	          4C~55.17 	&	 Y 	&	   	&	 Y 	&	 U,CFS  	&	1048	&	  Q 	&	9	&	31	\\
 0955+476 	&	 J0958+4725 	&	           OK 492	&	   	&	 Y 	&	   	&	 CT,CFS  	&	333	&	  Q 	&	     	&	 41,42  	\\
 0954+658 	&	 J0958+6533 	&	         S4 0954+65 	&	 Y 	&	   	&	   	&	  CT,CFS 	&	2993	&	  Q 	&	     	&	58	\\
  0958+29 	&	 J1001+2847 	&	            3C~234 	&	   	&	   	&	 Y$^\dag$ 	&	 FR~II  	&	3161	&	  G 	&	  9,29,64 	&	        	\\
 1003+351 	&	 J1006+3454 	&	            3C~236 	&	 Y 	&	   	&	 Y 	&	 FR~II  	&	334	&	  G 	&	  9,24 	&	        	\\
 1003+830 	&	 J1010+8250 	&	          S5 1003+83 	&	   	&	 Y 	&	   	&	 CT,CFS  	&	2997	&	  G 	&	     	&	 41,42  	\\
 1007+416 	&	 J1010+4132 	&	         4C 41.21 	&	   	&	 Y 	&	   	&	  FR~II 	&	1831	&	  Q 	&	22	&	22	\\
 1015+359 	&	 J1018+3542 	&	           B2 1015+35B	&	   	&	 Y 	&	   	&	 CT,CFS  	&	2468	&	  Q 	&	     	&	 41,42  	\\
 1020+400 	&	 J1023+3948 	&	          4C +40.25&	   	&	 Y 	&	   	&	 CT,CFS  	&	2475	&	  Q 	&	     	&	 41,42  	\\
 1030+415 	&	 J1033+4116 	&	         IVS B1030+415	&	   	&	 Y 	&	   	&	  CT ,CSF	&	2488	&	  Q 	&	     	&	 41,42  	\\
 1030+585 	&	 J1033+5814 	&	          3C~244.1 	&	   	&	 Y 	&	 Y 	&	 FR~II  	&	1835	&	  G 	&	   8,9 	&	        	\\
 1031+567 	&	 J1035+5628 	&	   JVAS~J1035+5628 	&	 Y 	&	   	&	 Y 	&	 CSO 	&	111	&	 G$^a$ 	&	     	&	    65  	\\
  1040+12 	&	 J1042+1203 	&	            3C~245 	&	   	&	   	&	 Y$^\dag$ 	&	  D 	&	3162	&	  Q 	&	   8,9 	&	        	\\
 1039+811 	&	 J1044+8054 	&	          S5 1039+81 	&	   	&	 Y 	&	   	&	 CT,CFS  	&	884	&	  Q 	&	     	&	42	\\
 1044+719 	&	 J1048+7143 	&	          S5 1044+71&	   	&	 Y 	&	   	&	 CT,CFS  	&	335	&	  Q 	&	     	&	 41,42  	\\
 1053+704 	&	 J1056+7011 	&	         S5 1053+70	&	   	&	 Y 	&	   	&	  CT,CFS 	&	2999	&	  Q 	&	     	&	 41,42  	\\
 1053+815 	&	 J1058+8114 	&	         S5 1053+81 	&	   	&	 Y 	&	   	&	  CT,CFS 	&	3000	&	  G 	&	     	&	 41,42  	\\
  1055+20 	&	 J1058+1951 	&	          4C~20.24 	&	   	&	   	&	 Y 	&	 D  	&	2508	&	  Q 	&	9	&	58	\\
 1056+432 	&	 J1058+4301 	&	            3C~247 	&	   	&	 Y 	&	 Y 	&	  FR~II 	&	3001	&	  G 	&	  9,40,62 	&	        	\\
 1058+726 	&	 J1101+7225 	&	           S5 1058+726 	&	   	&	 Y 	&	   	&	 CT,CSS  	&	3003	&	  Q 	&	     	&	 41,42  	\\
 1100+772 	&	 J1104+7658 	&	         3C 249.1	&	   	&	 Y 	&	   	&	  FR~II 	&	3004	&	  Q 	&	37	&	        	\\
 1101+384 	&	 J1104+3812 	&	        Mrk 421  	&	   	&	 Y 	&	   	&	 CT,CFS  	&	2519	&	  Q 	&	     	&	 41,42  	\\
 1111+408 	&	 J1114+4037 	&	         3C 254   	&	   	&	 Y 	&	   	&	  FR~II 	&	3006	&	  Q 	&	55	&	        	\\
  1116+12 	&	 J1118+1234 	&	          4C~12.39 	&	   	&	   	&	 Y$^\dag$ 	&	 U,CFS&	3163	&	  Q 	&	9	&	21	\\
 1128+385 	&	 J1130+3815 	&	          IVS B1128+385 	&	   	&	 Y 	&	   	&	 CT,CFS  	&	2162	&	  Q 	&	     	&	 41,42  	\\
 1137+660 	&	 J1139+6547 	&	            3C~263 	&	   	&	 Y 	&	 Y 	&	 FR~II  	&	3009	&	  Q 	&	  9,37 	&	        	\\
 1138+594 	&	 J1140+5912 	&	            4C +59.16 	&	   	&	 Y 	&	   	&	  CT,CSS 	&	130	&	    	&	     	&	 41,42  	\\
  1140+22 	&	 J1143+2206 	&	          3C~263.1 	&	   	&	   	&	 Y 	&	 FR~II  	&	3164	&	  G 	&	   8,9 	&	        	\\
  1142+19 	&	 J1145+1936 	&	            3C~264 	&	   	&	   	&	 Y$^\dag$ 	&	 FR~I  	&	3010	&	  G 	&	  9,26 	&	        	\\
 1144+542 	&	 J1146+5356 	&	         S4 1144+54	&	   	&	 Y 	&	   	&	 CT,CSS  	&	336	&	  Q 	&	     	&	 41,42  	\\
 1144+402 	&	 J1146+3958 	&	         S4 1144+40	&	   	&	 Y 	&	   	&	  CT,CFS 	&	2166	&	  Q 	&	     	&	 41,42  	\\
 1150+812 	&	 J1153+8058 	&	         S5 1150+81 	&	   	&	 Y 	&	   	&	 CT,CFS  	&	3011	&	  Q 	&	     	&	58	\\
 1150+497 	&	 J1153+4931 	&	          4C~49.22 	&	   	&	 Y 	&	 Y 	&	 CSS  	&	2569	&	  Q 	&	9	&	 41,42  	\\
 1152+551 	&	 J1155+5453 	&	          4C +55.22 	&	   	&	 Y 	&	   	&	  Complex 	&	3134	&	  G 	&	16	&	        	\\
  1153+31 	&	 J1156+3128 	&	          4C~31.38 	&	   	&	   	&	 Y 	&	 CSS  	&	586	&	  Q 	&	  9,57 	&	        	\\
 1157+732 	&	 J1200+7300 	&	          3C~268.1 	&	 Y 	&	   	&	 Y 	&	 FR~II  	&	3013	&	   G$^a$ 	&	   8,9,62 	&	        	\\
 1203+645 	&	 J1206+6413 	&	          3C~268.3 	&	   	&	 Y 	&	 Y 	&	 CSS  	&	354	&	  G 	&	  9,48 	&	        	\\
 1213+538 	&	 J1215+5335 	&	          4C +53.24	&	   	&	 Y 	&	   	&	  FR~II 	&	1834	&	  Q 	&	44	&	        	\\
 1213+350 	&	 J1215+3448 	&	           S4 1213+350&	   	&	 Y 	&	   	&	 CT,CFS  	&	2591	&	  Q 	&	     	&	 41,42  	\\
 1216+487 	&	 J1219+4829 	&	           S4 1216+48	&	   	&	 Y 	&	   	&	 CT,CFS  	&	2596	&	  Q 	&	     	&	 41,42  	\\
  1218+33 	&	 J1220+3343 	&	          3C~270.1 	&	   	&	   	&	 Y 	&	 FR~II  	&	3165	&	  Q 	&	   8,9,62 	&	        	\\
  1222+13 	&	 J1225+1253 	&	          3C~272.1 	&	   	&	   	&	 Y$^\dag$ 	&	 FR~I  	&	3017	&	  G 	&	   8,9	&	        	\\
 1225+368 	&	 J1227+3635 	&	         B2 1225+36 	&	   	&	 Y 	&	 Y 	&	 CSO 	&	154	&	 Q$^b$	&	     	&	    54, 56, Paper~1  	\\
  1228+12 	&	 J1230+1223 	&	            3C~274 	&	   	&	   	&	 Y$^\dag$ 	&	 FR~I  	&	3018	&	  G 	&	  1,9  	&	        	\\
  1241+16 	&	 J1243+1622 	&	          3C~275.1 	&	   	&	   	&	 Y 	&	 FR~II  	&	3166	&	  Q 	&	   8,9,62 	&	        	\\
 1242+410 	&	 J1244+4048 	&	       B3 1242+410 	&	   	&	 Y 	&	   	&	 CSO 	&	162	&	  Q 	&	     	&	 41,42, Paper~1  	\\
 1250+568 	&	 J1252+5634 	&	          3C~277.1 	&	   	&	 Y 	&	 Y 	&	 CSS  	&	355	&	  Q 	&	  9,43 	&	        	\\
  1251+27 	&	 J1254+2737 	&	          3C~277.3 	&	   	&	   	&	 Y 	&	 FR~II  	&	3020	&	  G 	&	   3,9 	&	        	\\
 1254+476 	&	 J1256+4720 	&	            3C~280 	&	 Y 	&	   	&	 Y 	&	  FR~II 	&	3021	&	  G 	&	  9,17 	&	        	\\
 1311+678 	&	 J1313+6736 	&	           4C +67.22 	&	   	&	 Y 	&	   	&	 CSS  	&	178	&	    	&	9	&	 41,42  	\\
 1317+520 	&	 J1319+5148 	&	           4C +52.27	&	   	&	 Y 	&	   	&	 CT,CFS  	&	1832	&	  Q 	&	     	&	 41,42  	\\
 1319+428 	&	 J1321+4235 	&	           3C 285 	&	   	&	 Y 	&	   	&	  FR~II 	&	3028	&	  G 	&	50	&	        	\\
  1323+32 	&	 J1326+3154 	&	            DA~344 	&	   	&	   	&	 Y$^\dag$ 	&	 CSO 	&	187	&	 G$^a$	&	     	&	    58, 56, Paper~1  	\\
  1328+25 	&	 J1330+2509 	&	            3C~287 	&	   	&	   	&	 Y$^\dag$ 	&	 CSS  	&	558	&	  Q 	&	9	&	    58  	\\
  1328+30 	&	 J1331+3030 	&	            3C~286 	&	   	&	   	&	 Y$^\dag$ 	&	 CSS  	&	604	&	  Q 	&	  9,48 	&	        	\\
 1333+459 	&	 J1335+4542 	&	         S4 1333+459 	&	   	&	 Y 	&	   	&	 CT,CFS  	&	188	&	  Q 	&	     	&	 41,42  	\\
 1333+589 	&	 J1335+5844 	&	   JVAS~J1335+5844 	&	   	&	 Y 	&	   	&	 CSO 	&	189	&	    	&	     	&	 41,42, Paper~1  	\\
 1336+391 	&	 J1338+3851 	&	            3C~288 	&	   	&	 Y 	&	 Y 	&	 FR~?  	&	3032	&	  G 	&	  9,32 	&	        	\\
 1342+663 	&	 J1344+6606 	&	        S4 1342+663 	&	   	&	 Y 	&	   	&	 CT,CFS  	&	3034	&	  Q 	&	     	&	 41,42  	\\
  1345+12 	&	 J1347+1217 	&	     PKS~B1345+125 	&	   	&	   	&	 Y$^\dag$ 	&	 CSO 	&	190	&	  G 	&	     	&	    58, Paper~1  	\\
 1347+539 	&	 J1349+5341 	&	       4C 53.28	&	   	&	 Y 	&	   	&	 CT,CFS  	&	2675	&	  Q 	&	     	&	 41,42  	\\
 1349+647 	&	 J1350+6429 	&	        3C 292  	&	   	&	 Y 	&	   	&	 FR~II  	&	3037	&	  G 	&	30	&	        	\\
  1350+31 	&	 J1352+3126 	&	            3C~293 	&	   	&	   	&	 Y$^\dag$ 	&	 FR~?  	&	3038	&	  G 	&	  9,43 	&	        	\\
  1354+19 	&	 J1357+1919 	&	          4C~19.44 	&	   	&	   	&	 Y$^\dag$ 	&	 FR~II  	&	2684	&	  Q 	&	9	&	58	\\
 1357+769 	&	 J1357+7643 	&	         S5 1357+76 	&	   	&	 Y 	&	   	&	 CT,CFS  	&	337	&	  Q 	&	     	&	 41,42  	\\
 1358+624 	&	 J1400+6210 	&	         4C 62.22 	&	 Y 	&	   	&	 Y 	&	 CSO 	&	197	&	  G 	&	     	&	    56,Paper~1  	\\
  1404+28 	&	 J1407+2827 	&	           OQ +208 	&	   	&	   	&	 Y$^\dag$ 	&	 CSO 	&	198	&	  G 	&	     	&	    58,Paper~1  	\\
 1409+524 	&	 J1411+5212 	&	            3C~295 	&	 Y 	&	   	&	 Y 	&	 FR~II  	&	1150	&	  G 	&	  9,17 	&	        	\\
  1413+34 	&	 J1416+3444 	&	        B2 1413+34 	&	   	&	   	&	 Y 	&	 CSO 	&	561	&	    EF$^a$	&	   	&	      54,56,Paper~1  	\\
  1414+11 	&	 J1416+1048 	&	            3C~296 	&	   	&	   	&	 Y$^\dag$ 	&	 FR~I  	&	3043	&	  G 	&	  9,34 	&	        	\\
 1418+546 	&	 J1419+5423 	&	            OQ 530	&	   	&	 Y 	&	   	&	 CT,CFS  	&	2701	&	  Q 	&	     	&	 41,42  	\\
 1419+419 	&	 J1421+4144 	&	            3C~299 	&	   	&	 Y 	&	 Y 	&	 CSS  	&	1786	&	  G 	&	  9,43 	&	        	\\
  1420+19 	&	 J1422+1935 	&	            3C~300 	&	   	&	   	&	 Y 	&	 FR~II  	&	1839	&	  G 	&	  9,29,64 	&	        	\\
 1435+638 	&	 J1436+6336 	&	          VIPS 0792	&	   	&	 Y 	&	   	&	 CT,CFS  	&	2718	&	  Q 	&	     	&	 41,42  	\\
 1437+624 	&	 J1438+6211 	&	          OQ 663	&	   	&	 Y 	&	   	&	 CT,CSS  	&	216	&	  Q 	&	     	&	 41,42  	\\
 1438+385 	&	 J1440+3820 	&	          S4 1438+38	&	   	&	 Y 	&	   	&	 CT,CFS  	&	2723	&	  Q 	&	     	&	 41,42  	\\
 1441+522 	&	 J1443+5201 	&	            3C~303 	&	   	&	 Y 	&	 Y 	&	 FR~II  	&	3046	&	  G 	&	  9,17 	&	        	\\
  1442+10 	&	 J1445+0958 	&	            OQ 172 	&	   	&	   	&	 Y 	&	 CSS  	&	500	&	  Q 	&	9	&	 51,56  	\\
 1448+634 	&	 J1449+6316 	&	            3C~305 	&	   	&	 Y 	&	 Y 	&	 FR~?  	&	3049	&	  G 	&	  9,17 	&	        	\\
 1458+718 	&	 J1459+7140 	&	          3C~309.1 	&	 Y 	&	   	&	 Y 	&	 CSS  	&	338	&	  Q 	&	  9,48 	&	31	\\
  1502+10 	&	 J1504+1029 	&	          4C~10.39 	&	   	&	   	&	 Y$^\dag$ 	&	 U,CFS  	&	3167	&	  Q 	&	9	&	58	\\
  1502+26 	&	 J1504+2600 	&	            3C~310 	&	   	&	   	&	 Y$^\dag$ 	&	 FR~II  	&	3051	&	  G 	&	27	&	        	\\
 1504+377 	&	 J1506+3730 	&	           B2 1504+37 	&	   	&	 Y 	&	   	&	 CT,CFS  	&	2746	&	  G 	&	     	&	 41,42  	\\
  1511+26 	&	 J1513+2607 	&	            3C~315 	&	   	&	   	&	 Y 	&	 FR~?  	&	3168	&	  G 	&	  9,29 	&	        	\\
  1529+24 	&	 J1531+2404 	&	            3C~321 	&	   	&	   	&	 Y 	&	 FR~II  	&	3057	&	  G 	&	   8,9 	&	        	\\
  1538+14 	&	 J1540+1447 	&	          4C~14.60 	&	   	&	   	&	 Y$^\dag$ 	&	 U,CFS  	&	3169	&	 Q? 	&	9	&	58	\\
 1547+507 	&	 J1549+5038 	&	           S4 1547+507 	&	   	&	 Y 	&	   	&	 CT,CFS  	&	351	&	  Q 	&	     	&	 41,42  	\\
 1549+628 	&	 J1549+6241 	&	            3C~325 	&	   	&	 Y 	&	 Y 	&	 FR~II  	&	1828	&	  G 	&	   8,9 	&	        	\\
 1557+708 	&	 J1557+7041 	&	          NGC 6048 	&	   	&	 Y 	&	 Y 	&	 FR~I  	&	3060	&	  G 	&	  9,19 	&	        	\\
  1600+33 	&	 J1602+3326 	&	          4C +33.38 	&	   	&	   	&	 Y$^\dag$ 	&	 CSS  	&	562	&	   G$^a$ 	&	9	&	54,56	\\
  1607+26 	&	 J1609+2641 	&	            CTD 93 	&	   	&	   	&	 Y$^\dag$ 	&	 CSO 	&	245	&	 G$^a$ 	&	     	&	    58,Paper~1  	\\
 1609+660 	&	 J1609+6556 	&	            3C~330 	&	 Y 	&	   	&	 Y 	&	 FR~II  	&	1838	&	  G 	&	8	&	        	\\
  1611+34 	&	 J1613+3412 	&	           DA 406 	&	   	&	   	&	 Y$^\dag$ 	&	  CT,CFS 	&	3074	&	  Q 	&	     	&	58	\\
 1624+416 	&	 J1625+4134 	&	          4C~41.32 	&	 Y 	&	   	&	 Y 	&	 U,CFS 	&	2821	&	 Q$^a$ 	&	9	&	31	\\
 1627+444 	&	 J1628+4419 	&	            3C~337 	&	   	&	 Y 	&	 Y 	&	 FR~II  	&	3067	&	  G 	&	   8,9 	&	        	\\
 1637+826 	&	 J1632+8232 	&	          NGC~6251 	&	   	&	 Y 	&	 Y 	&	 FR~?  	&	3068	&	  G 	&	  11 	&	        	\\
 1634+628 	&	 J1634+6245 	&	            3C~343 	&	 Y 	&	   	&	 Y 	&	 CSS  	&	352	&	  Q 	&	  9,57 	&	        	\\
 1633+382 	&	 J1635+3808 	&	          4C~38.41 	&	 Y 	&	   	&	 Y 	&	 U,CFS  	&	2828	&	  Q 	&	     	&	31	\\
 1637+574 	&	 J1638+5720 	&	            OS 562 	&	 Y 	&	   	&	   	&	  CT,CFS 	&	2834	&	  Q 	&	     	&	31	\\
 1637+626 	&	 J1638+6234 	&	          3C~343.1 	&	   	&	 Y 	&	 Y 	&	  CSS 	&	349	&	  G 	&	  9,57 	&	        	\\
 1638+398 	&	 J1640+3946 	&	          NRAO 512 	&	   	&	 Y 	&	   	&	 CT,CFS  	&	2273	&	  Q 	&	     	&	 41,42  	\\
 1642+690 	&	 J1642+6856 	&	          4C~69.21 	&	 Y 	&	   	&	   	&	 CT,CFS  	&	3070	&	  Q 	&	     	&	31	\\
 1641+399 	&	 J1642+3948 	&	            3C~345 	&	 Y 	&	   	&	 Y 	&	 U,CFS  	&	2838	&	  Q 	&	9	&	58	\\
  1641+17 	&	 J1643+1715 	&	            3C~346 	&	   	&	   	&	 Y$^\dag$ 	&	 FR~I  	&	1301	&	  G 	&	   3,9 	&	        	\\
 1652+398 	&	 J1653+3945 	&	          Mrk 501 	&	 Y 	&	   	&	   	&	 CT,CFS  	&	251	&	  G 	&	     	&	31	\\
 1656+482 	&	 J1657+4808 	&	          4C +48.41 	&	   	&	 Y 	&	   	&	 CT,CFS  	&	2859	&	    	&	     	&	 41,42  	\\
 1656+477 	&	 J1658+4737 	&	          S4 1656+47 	&	   	&	 Y 	&	   	&	  CT,CFS 	&	2861	&	  Q 	&	     	&	 41,42  	\\
 1658+471 	&	 J1659+4702 	&	            3C~349 	&	   	&	 Y 	&	 Y 	&	 FR~II  	&	3072	&	  G 	&	   8,9,64 	&	        	\\
 1704+608 	&	 J1704+6044 	&	            3C~351 	&	   	&	 Y 	&	 Y 	&	 FR~II  	&	3077	&	  Q 	&	  4,9 	&	        	\\
 1719+357 	&	 J1721+3542 	&	         S4 1719+357 	&	   	&	 Y 	&	   	&	 CT,CFS  	&	256	&	  Q 	&	     	&	 41,42  	\\
  1726+31 	&	 J1728+3145 	&	            3C~357 	&	   	&	   	&	 Y 	&	 FR~II  	&	3170	&	  G 	&	  1,9  	&	        	\\
 1732+389 	&	 J1734+3857 	&	          OT 355 	&	   	&	 Y 	&	   	&	 CT,CFS  	&	339	&	  Q 	&	     	&	 41,42  	\\
 1734+508 	&	 J1735+5049 	&	  	&	   	&	 Y 	&	   	&	 CSO 	&	263	&	    	&	     	&	 41,42,Paper~1  	\\
 1738+476 	&	 J1739+4737 	&	          OT 465  	&	   	&	 Y 	&	   	&	  CT,CFS 	&	340	&	  Q 	&	     	&	 41,42  	\\
 1739+522 	&	 J1740+5211 	&	          4C~51.37 	&	 Y 	&	   	&	 Y 	&	 U,CFS  	&	2893	&	  Q 	&	9	&	31	\\
 1749+701 	&	 J1748+7005 	&	        S4 1749+70 	&	 Y 	&	   	&	 Y 	&	 CSS  	&	3080	&	  Q 	&	9	&	31	\\
 1751+441 	&	 J1753+4409 	&	        S4 1751+441 	&	   	&	 Y 	&	   	&	 CT,CFS  	&	2297	&	  Q 	&	     	&	 41,42  	\\
 1758+388 	&	 J1800+3848 	&	        B3 1758+388B	&	   	&	 Y 	&	   	&	 CT,CFS  	&	726	&	  Q 	&	     	&	 41,42  	\\
 1803+784 	&	 J1800+7828 	&	         S5 1803+784 	&	 Y 	&	   	&	 Y 	&	 U,CFS  	&	890	&	  Q 	&	9	&	31	\\
 1800+440 	&	 J1801+4404 	&	         S4 1800+44	&	   	&	 Y 	&	   	&	 CT,CFS  	&	3082	&	  Q 	&	     	&	 41,42  	\\
 1807+698 	&	 J1806+6949 	&	            3C~371 	&	 Y 	&	   	&	 Y 	&	 U,CFS  	&	3083	&	  G 	&	9	&	31	\\
 1819+396 	&	 J1821+3942 	&	          4C~39.56 	&	   	&	 Y 	&	 Y 	&	 CSS  	&	356	&	 G? 	&	  9,57,56 	&	        	\\
 1823+568 	&	 J1824+5651 	&	          4C~56.27 	&	 Y 	&	   	&	 Y 	&	 U,CFS  	&	3086	&	  Q 	&	9	&	31	\\
 1825+743 	&	 J1824+7420 	&	3C 379.1                   	&	   	&	 Y 	&	   	&	 FR~II  	&	3087	&	  G 	&	1	&	        	\\
 1828+487 	&	 J1829+4844 	&	            3C~380 	&	 Y 	&	   	&	 Y 	&	 D  	&	3089	&	  Q 	&	  9,48 	&	        	\\
  1829+29 	&	 J1831+2907 	&	          4C~29.56 	&	   	&	   	&	 Y 	&	 CSS  	&	603	&	 G$^a$ 	&	  9,57,56 	&	        	\\
 1833+653 	&	 J1833+6521 	&	   3C~383          	&	   	&	 Y 	&	   	&	 FR~II  	&	3090	&	  G 	&	12	&	        	\\
 1832+474 	&	 J1833+4727 	&	            3C~381 	&	   	&	 Y 	&	 Y 	&	 FR~II  	&	3091	&	  G 	&	  9,17,64 	&	        	\\
 1845+797 	&	 J1842+7946 	&	          3C~390.3 	&	 Y 	&	   	&	 Y 	&	 FR~II  	&	3094	&	  G 	&	  9,17 	&	31	\\
 1842+681 	&	 J1842+6809 	&	          TXS 1842+681 	&	   	&	 Y 	&	   	&	 CT,CFS  	&	3095	&	  Q 	&	     	&	 41,42  	\\
 1842+455 	&	 J1844+4533 	&	            3C~388 	&	 Y 	&	   	&	 Y 	&	 FR~II  	&	3096	&	  G 	&	  9,17 	&	        	\\
 1843+356 	&	 J1845+3541 	&	    COINS J1845+3541 	&	   	&	 Y 	&	   	&	 CT,CFS  	&	275	&	  G 	&	     	&	 41,42  	\\
 1926+611 	&	 J1927+6117 	&	    S4 1926+61	&	   	&	 Y 	&	   	&	 CT,CFS  	&	341	&	  Q 	&	     	&	 41,42  	\\
 1928+738 	&	 J1927+7358 	&	          4C~73.18 	&	 Y 	&	   	&	 Y 	&	 U,CFS  	&	3104	&	  Q 	&	9	&	31	\\
 1939+605 	&	 J1940+6041 	&	            3C~401 	&	 Y 	&	   	&	 Y 	&	 FR~II  	&	1824	&	  G 	&	   8,9,64 	&	        	\\
 1940+504 	&	 J1941+5035 	&	              3C~402     	&	   	&	 Y 	&	   	&	FR~?  	&	3106	&	  G 	&	4	&	        	\\
 1943+546 	&	 J1944+5448 	&	     COINS J1944+5448 	&	   	&	 Y 	&	   	&	 CSO 	&	284	&	  G 	&	     	&	 41,42,Paper~1  	\\
 1954+513 	&	 J1955+5131 	&	            OV 591 	&	 Y 	&	   	&	 Y 	&	 U,CFS  	&	3108	&	  Q 	&	9	&	31	\\
 2007+777 	&	 J2005+7752 	&	        S5 2007+77  	&	   	&	 Y 	&	   	&	 CT,CFS  	&	877	&	  Q 	&	     	&	58	\\
 2010+723 	&	 J2009+7229 	&	         4C +72.28 	&	   	&	 Y 	&	   	&	 CT,CFS  	&	342	&	  Q 	&	     	&	 41,42  	\\
 2021+614 	&	 J2022+6136 	&	      TXS 2021+614 	&	 Y 	&	   	&	 Y 	&	 CSO 	&	289	&	  G 	&	     	&	    31,Paper~1  	\\
 2104+763 	&	 J2104+7633 	&	          3C~427.1 	&	   	&	 Y 	&	 Y 	&	 FR~II  	&	3113	&	  G 	&	  9,17 	&	        	\\
  2121+24 	&	 J2123+2504 	&	            3C~433 	&	   	&	   	&	 Y$^\dag$ 	&	 FR~?  	&	3171	&	  G 	&	   3,9 	&	        	\\
  2141+27 	&	 J2144+2810 	&	            3C~436 	&	   	&	   	&	 Y 	&	 FR~II  	&	3172	&	  G 	&	   4,9,64 	&	        	\\
  2145+15 	&	 J2147+1520 	&	            3C~437 	&	   	&	   	&	 Y 	&	 FR~II  	&	3173	&	  G 	&	   5,9 	&	        	\\
 2153+377 	&	 J2155+3800 	&	            3C~438 	&	 Y 	&	   	&	 Y 	&	 FR~II  	&	3118	&	  G 	&	  9,17,64 	&	        	\\
 2200+420 	&	 J2202+4216 	&	            BL Lac 	&	 Y 	&	   	&	 Y 	&	 U,CFS  	&	3119	&	  Q 	&	9	&	  58     	\\
  2203+29 	&	 J2206+2929 	&	            3C~441 	&	   	&	   	&	 Y 	&	 FR~II  	&	3174	&	  G 	&	   5,9 	&	        	\\
 2207+374 	&	 J2209+3742 	&	           S4 2207+37 	&	   	&	 Y 	&	   	&	 CT,CSS  	&	343	&	  Q 	&	     	&	 41,42  	\\
 2214+350 	&	 J2216+3518 	&	            OY 324	&	   	&	 Y 	&	   	&	 CT,CFS  	&	3121	&	  Q 	&	     	&	 41,42  	\\
 2229+695 	&	 J2230+6946 	&	          S5 2229+69  	&	   	&	 Y 	&	   	&	 CT,CFS  	&	344	&	  G 	&	     	&	 41,42  	\\
 2229+391 	&	 J2231+3921 	&	            3C~449 	&	 Y 	&	   	&	 Y 	&	 FR~1  	&	3122	&	  G 	&	  9,18 	&	        	\\
  2230+11 	&	 J2232+1143 	&	          CTA 102 	&	   	&	   	&	 Y$^\dag$ 	&	  U,CFS 	&	690	&	  Q 	&	9	&	58,56	\\
 2243+394 	&	 J2245+3941 	&	            3C~452 	&	 Y 	&	   	&	 Y 	&	 FR~II  	&	3126	&	  G 	&	  1,9  	&	        	\\
  2247+14 	&	 J2250+1419 	&	          4C~14.82 	&	   	&	   	&	 Y 	&	  CSS 	&	764	&	  Q 	&	9	&	38	\\
  2251+15 	&	 J2253+1608 	&	          3C~454.3 	&	   	&	   	&	 Y$^\dag$ 	&	 U,CFS  	&	3175	&	  Q 	&	  9,58 	&	        	\\
  2252+12 	&	 J2255+1313 	&	            3C~455 	&	   	&	   	&	 Y 	&	 FR~II  	&	1816	&	  Q 	&	9	&	63	\\
 2253+417 	&	 J2255+4202 	&	          OY 489 	&	   	&	 Y 	&	   	&	 CT,CFS  	&	345	&	  Q 	&	     	&	 41,42  	\\
 2255+416 	&	 J2257+4154 	&	          4C 41.45 	&	   	&	 Y 	&	   	&	 CT,CSS  	&	346	&	  Q 	&	     	&	 41,42  	\\
 2311+469 	&	 J2313+4712 	&	           4C 46.47	&	   	&	 Y 	&	   	&	 CT,CSS  	&	933	&	  Q 	&	     	&	 41,42  	\\
 2323+435 	&	 J2325+4346 	&	          S4 2323+43 	&	   	&	 Y 	&	   	&	  CSS 	&	347	&	  G 	&	45	&	        	\\
 2324+405 	&	 J2326+4048 	&	            3C~462 	&	   	&	 Y 	&	 Y 	&	 FR~II  	&	1829	&	 G$^a$ 	&	9	&	 19  	\\
  2335+26 	&	 J2338+2701 	&	            3C~465 	&	   	&	   	&	 Y$^\dag$ 	&	 FR~I  	&	3130	&	  G 	&	  1,9  	&	        	\\
 2342+821 	&	 J2344+8226 	&	          S5 2342+82 	&	 Y 	&	   	&	 Y 	&	  CSS 	&	309	&	    Q$^a$	&	  9,57,56 	&	        	\\
 2351+456 	&	 J2354+4553 	&	          4C~45.51 	&	 Y 	&	   	&	 Y 	&	 U,CFS  	&	3132	&	 Q$^a$ 	&	9	&	31	\\
 2352+495 	&	 J2355+4950 	&	      DA 611 	&	 Y 	&	   	&	 Y 	&	 CSO 	&	313	&	  G 	&	     	&	    31,Paper~1	\\
\enddata
\tablecomments{$^*$ M82 s not an active galaxy, and was therefore not included in the statistical analyses in this paper. $^\dag$ indicates objects in the PWS subsample (see text). $^\ddag$ the references for the optical classes are as follows: We used the PW \citep{1981MNRAS.194..331P} optical identifications where available, otherwise  the PR \citep{1981ApJ...248...61P} and CJ1 \citep{1995ApJS...98....1P}, where we replaced  ``SO'' and  ``BL'' entires with ``Q''. The only exceptions are $^a$ from \citet{1994AAS..105..211S}, and $^b$ from \citet{2000ApJ...534...90P}. The  columns are as follows: Source Names (1--3), Membership in the PR, CJ1 and PW Samples (4--6), Source Type (7), The Types listed in column (7)  are as follows: Quasars (Q), galaxies (G), Fanaroff and Riley types~I, II and intermediate (FR~I, FR~II, FR?); objects unresolved by \citet{1981MNRAS.194..331P} on the 5\,km~Telescope (U,CSS or U,CFS for compact steep spectrum and compact flat spectrum sources, respectively), doubles indentified by \citet{1981MNRAS.194..331P} in which the optical ID coincides with one of the radio components (D2); Compact Steep Spectrum objects identified by \citet{1981MNRAS.194..331P} (CSS); Objects found to be compact in various VLBI surveys  other than any identified by \citet{1981MNRAS.194..331P} (CT,CSS or CT,CFS) for compact steep spectrum and compact flat spectrum sources, respectively;  CSO Catalog ID (8), Optical Identifications (9), and Structure References (10+11). No attempt has been made at complete map references for each object since these number in the tens for many objects and the purpose of this Table is solely to provide justification for the claim that the structures of all of these sources are well known.  References: 1 \citet{1968MNRAS.138..259M}; 2 \citet{1972MNRAS.156..377B}; 3 \citet{1974MNRAS.169..477P}; 4 \citet{1975MmRAS..80..105R}; 5 \citet{1975MNRAS.173..309L}; 6 \citet{1975AandA....38..381M};7 \citet{1976Natur.262..179B}; 8 \citet{1977MmRAS..84...61J};  9 \citet{1981MNRAS.194..331P}; 10 \citet{1977MNRAS.178..577C}; 11 \citet{1977MNRAS.181..465W}; 12 \citet{1980AJ.....85..981F};13 \citet{1980Natur.287..208B}; 14 \citet{1981ApJ...248...61P}; 15 \citet{1981Natur.294...47P}; 16 \citet{1981AandAS...43..381K}; 17 \citet{1981MNRAS.195..261L}; 18 \citet{1981MNRAS.197..253B}; 19 \citet{1982MNRAS.198..843P}; 20 \citet{1983MNRAS.204..151L}; 21 \citet{1984AandA...135..289R}; 22 \citet{1984AJ.....89..932O};
23 \citet{1984AJ.....89.1478S}; 24 \citet{1980AandA....85...36S}; 25 \citet{1984MNRAS.208..545L}; 26 \citet{1982MNRAS.200..705J}; 27 \citet{1984ApJ...282L..55V}; 28  \citet{1986AandAS...65..145F}; 29 \citet{1986MNRAS.222..753L}; 30 \citet{1987MNRAS.225....1A}; 31 \citet{1988ApJ...328..114P}; 32 \citet{1989AJ.....97..674B}; 33 \citet{1990MNRAS.246..477P}; 34 \citet{1991AJ....102..537L}; 35 \citet{1995AandAS..114..197A}; 36 \citet{1975MNRAS.172..181T}; 37 \citet{1994AJ....108..766B}; 38 \citet{1994AJ....108..821L}; 39 \citet{1994ApJ...435..116G}; 40  \citet{1994AandAS..105..247A}; 41  \citet{1995ApJS...98....1P} or \citep{1995ApJS...98...33T}; 42  \citet{1995ApJS...99..297X}; 43 \citet{1995AandAS..112..235A}; 44 \citet{1995AandAS..110..213R}; 45 \citet{1995AandA...295..629S}; 46 
 \citet{1996MNRAS.278..273H}; 47  \citet{1997MNRAS.291...20L}; 48 \citet{1998MNRAS.299..467L}; 49 \citet{1995ApJS...99..349N}; 50 \citet{1999MNRAS.304..271D}; 51 \citet{2000ApJS..131...95F}; 52 \citet{2000ApJ...534..172L}; 53 \citet{2001MNRAS.321...37S}; 54 \citet{2013MNRAS.433..147D}:55 \citet{2006MNRAS.372.1607T};
  56 \citet{1995AA...295...27D}; 57 \citet{2021MNRAS.504.2312D}; 58 MOJAVE
website: \url{https://www.cv.nrao.edu/MOJAVE/allsources.html}; 59 Radio Fundamental Catalog (RFC); 60 \citet{1998AJ....115.1295K}; 61 \citet{2018NatAs...2..472G}; 62
\citet{2014ApJS..212...19F}; 63 \citet{1991MNRAS.250..215A}; 64 \citep{1997MNRAS.288..859H};65 \citet{2016MNRAS.459..820T}
}
\end{deluxetable*}


\clearpage
%------------------------------------------------------------------------------
%\bibliography{references}{}
%\bibliographystyle{aasjournal}
\begin{thebibliography}{}
\expandafter\ifx\csname natexlab\endcsname\relax\def\natexlab#1{#1}\fi
\providecommand{\url}[1]{\href{#1}{#1}}
\providecommand{\dodoi}[1]{doi:~\href{http://doi.org/#1}{\nolinkurl{#1}}}
\providecommand{\doeprint}[1]{\href{http://ascl.net/#1}{\nolinkurl{http://ascl.net/#1}}}
\providecommand{\doarXiv}[1]{\href{https://arxiv.org/abs/#1}{\nolinkurl{https://arxiv.org/abs/#1}}}

\bibitem[{{Akujor} \& {Garrington}(1995)}]{1995AandAS..112..235A}
{Akujor}, C.~E., \& {Garrington}, S.~T. 1995, \aaps, 112, 235

\bibitem[{{Akujor} {et~al.}(1994){Akujor}, {Luedke}, {Browne}, {Leahy},
  {Garrington}, {Jackson}, \& {Thomasson}}]{1994AandAS..105..247A}
{Akujor}, C.~E., {Luedke}, E., {Browne}, I.~W.~A., {et~al.} 1994, \aaps, 105,
  247

\bibitem[{{Akujor} {et~al.}(1991){Akujor}, {Spencer}, {Zhang}, {Davis},
  {Browne}, \& {Fanti}}]{1991MNRAS.250..215A}
{Akujor}, C.~E., {Spencer}, R.~E., {Zhang}, F.~J., {et~al.} 1991, \mnras, 250,
  215, \dodoi{10.1093/mnras/250.1.215}

\bibitem[{{Alexander} \& {Leahy}(1987)}]{1987MNRAS.225....1A}
{Alexander}, P., \& {Leahy}, J.~P. 1987, \mnras, 225, 1,
  \dodoi{10.1093/mnras/225.1.1}

\bibitem[{{Altschuler} {et~al.}(1995){Altschuler}, {Gurvits}, {Alef},
  {Dennison}, {Graham}, {Trotter}, \& {Carson}}]{1995AandAS..114..197A}
{Altschuler}, D.~R., {Gurvits}, L.~I., {Alef}, W., {et~al.} 1995, \aaps, 114,
  197

\bibitem[{{Augusto}(2009)}]{2009AN....330..190A}
{Augusto}, P. 2009, Astronomische Nachrichten, 330, 190,
  \dodoi{10.1002/asna.200811153}

\bibitem[{{Augusto} {et~al.}(2006){Augusto}, {Gonzalez-Serrano},
  {Perez-Fournon}, \& {Wilkinson}}]{2006MNRAS.368.1411A}
{Augusto}, P., {Gonzalez-Serrano}, J.~I., {Perez-Fournon}, I., \& {Wilkinson},
  P.~N. 2006, \mnras, 368, 1411, \dodoi{10.1111/j.1365-2966.2006.10227.x}

\bibitem[{{Augusto} {et~al.}(1998){Augusto}, {Wilkinson}, \&
  {Browne}}]{1998MNRAS.299.1159A}
{Augusto}, P., {Wilkinson}, P.~N., \& {Browne}, I.~W.~A. 1998, \mnras, 299,
  1159, \dodoi{10.1046/j.1365-8711.1998.01871.x}

\bibitem[{{Birkinshaw} {et~al.}(1981){Birkinshaw}, {Laing}, \&
  {Peacock}}]{1981MNRAS.197..253B}
{Birkinshaw}, M., {Laing}, R.~A., \& {Peacock}, J.~A. 1981, \mnras, 197, 253,
  \dodoi{10.1093/mnras/197.2.253}

\bibitem[{{Blandford} {et~al.}(2019){Blandford}, {Meier}, \&
  {Readhead}}]{2019ARAandA..57..467B}
{Blandford}, R., {Meier}, D., \& {Readhead}, A. 2019, \araa, 57, 467,
  \dodoi{10.1146/annurev-astro-081817-051948}

\bibitem[{{Branson} {et~al.}(1972){Branson}, {Elsmore}, {Pooley}, \&
  {Ryle}}]{1972MNRAS.156..377B}
{Branson}, N.~J.~B.~A., {Elsmore}, B., {Pooley}, G.~G., \& {Ryle}, M. 1972,
  \mnras, 156, 377, \dodoi{10.1093/mnras/156.4.377}

\bibitem[{{Bridle} {et~al.}(1976){Bridle}, {Davis}, {Meloy}, {Fomalont},
  {Strom}, \& {Willis}}]{1976Natur.262..179B}
{Bridle}, A.~H., {Davis}, M.~M., {Meloy}, D.~A., {et~al.} 1976, \nat, 262, 179,
  \dodoi{10.1038/262179a0}

\bibitem[{{Bridle} {et~al.}(1989){Bridle}, {Fomalont}, {Byrd}, \&
  {Valtonen}}]{1989AJ.....97..674B}
{Bridle}, A.~H., {Fomalont}, E.~B., {Byrd}, G.~G., \& {Valtonen}, M.~J. 1989,
  \aj, 97, 674, \dodoi{10.1086/115013}

\bibitem[{{Bridle} {et~al.}(1994){Bridle}, {Hough}, {Lonsdale}, {Burns}, \&
  {Laing}}]{1994AJ....108..766B}
{Bridle}, A.~H., {Hough}, D.~H., {Lonsdale}, C.~J., {Burns}, J.~O., \& {Laing},
  R.~A. 1994, \aj, 108, 766, \dodoi{10.1086/117112}

\bibitem[{{Burns} \& {Christiansen}(1980)}]{1980Natur.287..208B}
{Burns}, J.~O., \& {Christiansen}, W.~A. 1980, \nat, 287, 208,
  \dodoi{10.1038/287208a0}

\bibitem[{{Callingham} {et~al.}(2017){Callingham}, {Ekers}, {Gaensler}, {Line},
  {Hurley-Walker}, {Sadler}, {Tingay}, {Hancock}, {Bell}, {Dwarakanath}, {For},
  {Franzen}, {Hindson}, {Johnston-Hollitt}, {Kapi{\'n}ska}, {Lenc}, {McKinley},
  {Morgan}, {Offringa}, {Procopio}, {Staveley-Smith}, {Wayth}, {Wu}, \&
  {Zheng}}]{2017ApJ...836..174C}
{Callingham}, J.~R., {Ekers}, R.~D., {Gaensler}, B.~M., {et~al.} 2017, \apj,
  836, 174, \dodoi{10.3847/1538-4357/836/2/174}

\bibitem[{{Carilli} {et~al.}(1991){Carilli}, {Perley}, {Dreher}, \&
  {Leahy}}]{1991ApJ...383..554C}
{Carilli}, C.~L., {Perley}, R.~A., {Dreher}, J.~W., \& {Leahy}, J.~P. 1991,
  \apj, 383, 554, \dodoi{10.1086/170813}

\bibitem[{{Cohen} {et~al.}(1977){Cohen}, {Porcas}, {Browne}, {Daintree}, \&
  {Walsh}}]{1977MmRAS..84....1C}
{Cohen}, A.~M., {Porcas}, R.~W., {Browne}, I.~W.~A., {Daintree}, E.~J., \&
  {Walsh}, D. 1977, \memras, 84, 1

\bibitem[{{Cottrell}(1977)}]{1977MNRAS.178..577C}
{Cottrell}, G.~A. 1977, \mnras, 178, 577, \dodoi{10.1093/mnras/178.4.577}

\bibitem[{{Curtis}(1918)}]{1918PLicO..13....9C}
{Curtis}, H.~D. 1918, Publications of Lick Observatory, 13, 9

\bibitem[{{Dallacasa} {et~al.}(1995){Dallacasa}, {Fanti}, {Fanti}, {Schilizzi},
  \& {Spencer}}]{1995AA...295...27D}
{Dallacasa}, D., {Fanti}, C., {Fanti}, R., {Schilizzi}, R.~T., \& {Spencer},
  R.~E. 1995, \aap, 295, 27

\bibitem[{{Dallacasa} {et~al.}(2021){Dallacasa}, {Orienti}, {Fanti}, \&
  {Fanti}}]{2021MNRAS.504.2312D}
{Dallacasa}, D., {Orienti}, M., {Fanti}, C., \& {Fanti}, R. 2021, \mnras, 504,
  2312, \dodoi{10.1093/mnras/stab1014}

\bibitem[{{Dallacasa} {et~al.}(2013){Dallacasa}, {Orienti}, {Fanti}, {Fanti},
  \& {Stanghellini}}]{2013MNRAS.433..147D}
{Dallacasa}, D., {Orienti}, M., {Fanti}, C., {Fanti}, R., \& {Stanghellini}, C.
  2013, \mnras, 433, 147, \dodoi{10.1093/mnras/stt710}

\bibitem[{{Dennett-Thorpe} {et~al.}(1999){Dennett-Thorpe}, {Bridle}, {Laing},
  \& {Scheuer}}]{1999MNRAS.304..271D}
{Dennett-Thorpe}, J., {Bridle}, A.~H., {Laing}, R.~A., \& {Scheuer}, P.~A.~G.
  1999, \mnras, 304, 271, \dodoi{10.1046/j.1365-8711.1999.02234.x}

\bibitem[{{Dent}(1965{\natexlab{a}})}]{1965Sci...148.1458D}
{Dent}, W.~A. 1965{\natexlab{a}}, Science, 148, 1458,
  \dodoi{10.1126/science.148.3676.1458}

\bibitem[{{Dent}(1965{\natexlab{b}})}]{1965AJ.....70..672D}
---. 1965{\natexlab{b}}, \aj, 70, 672, \dodoi{10.1086/109792}

\bibitem[{{Fanaroff} \& {Riley}(1974)}]{1974MNRAS.167P..31F}
{Fanaroff}, B.~L., \& {Riley}, J.~M. 1974, \mnras, 167, 31P,
  \dodoi{10.1093/mnras/167.1.31P}

\bibitem[{{Fanti} {et~al.}(1995){Fanti}, {Fanti}, {Dallacasa}, {Schilizzi},
  {Spencer}, \& {Stanghellini}}]{1995AandA...302..317F}
{Fanti}, C., {Fanti}, R., {Dallacasa}, D., {et~al.} 1995, \aap, 302, 317

\bibitem[{{Fanti} {et~al.}(1986){Fanti}, {Fanti}, {de Ruiter}, \&
  {Parma}}]{1986AandAS...65..145F}
{Fanti}, C., {Fanti}, R., {de Ruiter}, H.~R., \& {Parma}, P. 1986, \aaps, 65,
  145

\bibitem[{{Fanti} {et~al.}(1985){Fanti}, {Fanti}, {Parma}, {Schilizzi}, \& {van
  Breugel}}]{1985AandA...143..292F}
{Fanti}, C., {Fanti}, R., {Parma}, P., {Schilizzi}, R.~T., \& {van Breugel},
  W.~J.~M. 1985, \aap, 143, 292

\bibitem[{{Fanti} {et~al.}(1990){Fanti}, {Fanti}, {Schilizzi}, {Spencer}, {Nan
  Rendong}, {Parma}, {van Breugel}, \& {Venturi}}]{1990AandA...231..333F}
{Fanti}, R., {Fanti}, C., {Schilizzi}, R.~T., {et~al.} 1990, \aap, 231, 333

\bibitem[{{Fernini}(2014)}]{2014ApJS..212...19F}
{Fernini}, I. 2014, \apjs, 212, 19, \dodoi{10.1088/0067-0049/212/2/19}

\bibitem[{{Fomalont} {et~al.}(2000){Fomalont}, {Frey}, {Paragi}, {Gurvits},
  {Scott}, {Taylor}, {Edwards}, \& {Hirabayashi}}]{2000ApJS..131...95F}
{Fomalont}, E.~B., {Frey}, S., {Paragi}, Z., {et~al.} 2000, \apjs, 131, 95,
  \dodoi{10.1086/317368}

\bibitem[{{Fomalont} {et~al.}(1980){Fomalont}, {Palimaka}, \&
  {Bridle}}]{1980AJ.....85..981F}
{Fomalont}, E.~B., {Palimaka}, J.~J., \& {Bridle}, A.~H. 1980, \aj, 85, 981,
  \dodoi{10.1086/112761}

\bibitem[{{F{\"o}rster Schreiber} \& {Wuyts}(2020)}]{2020ARA&A..58..661F}
{F{\"o}rster Schreiber}, N.~M., \& {Wuyts}, S. 2020, \araa, 58, 661,
  \dodoi{10.1146/annurev-astro-032620-021910}

\bibitem[{{Giovannini} {et~al.}(1994){Giovannini}, {Feretti}, {Venturi},
  {Lara}, {Marcaide}, {Rioja}, {Spangler}, \& {Wehrle}}]{1994ApJ...435..116G}
{Giovannini}, G., {Feretti}, L., {Venturi}, T., {et~al.} 1994, \apj, 435, 116,
  \dodoi{10.1086/174799}

\bibitem[{{Giovannini} {et~al.}(2018){Giovannini}, {Savolainen}, {Orienti},
  {Nakamura}, {Nagai}, {Kino}, {Giroletti}, {Hada}, {Bruni}, {Kovalev},
  {Anderson}, {D'Ammando}, {Hodgson}, {Honma}, {Krichbaum}, {Lee}, {Lico},
  {Lisakov}, {Lobanov}, {Petrov}, {Sohn}, {Sokolovsky}, {Voitsik}, {Zensus}, \&
  {Tingay}}]{2018NatAs...2..472G}
{Giovannini}, G., {Savolainen}, T., {Orienti}, M., {et~al.} 2018, Nature
  Astronomy, 2, 472, \dodoi{10.1038/s41550-018-0431-2}

\bibitem[{{Gregory} {et~al.}(1996){Gregory}, {Scott}, {Douglas}, \&
  {Condon}}]{1996ApJS..103..427G}
{Gregory}, P.~C., {Scott}, W.~K., {Douglas}, K., \& {Condon}, J.~J. 1996,
  \apjs, 103, 427, \dodoi{10.1086/192282}

\bibitem[{{Hardcastle} {et~al.}(1996){Hardcastle}, {Alexander}, {Pooley}, \&
  {Riley}}]{1996MNRAS.278..273H}
{Hardcastle}, M.~J., {Alexander}, P., {Pooley}, G.~G., \& {Riley}, J.~M. 1996,
  \mnras, 278, 273, \dodoi{10.1093/mnras/278.1.273}

\bibitem[{{Hardcastle} {et~al.}(1997){Hardcastle}, {Alexander}, {Pooley}, \&
  {Riley}}]{1997MNRAS.288..859H}
---. 1997, \mnras, 288, 859, \dodoi{10.1093/mnras/288.4.859}

\bibitem[{{Jenkins} {et~al.}(1977){Jenkins}, {Pooley}, \&
  {Riley}}]{1977MmRAS..84...61J}
{Jenkins}, C.~J., {Pooley}, G.~G., \& {Riley}, J.~M. 1977, \memras, 84, 61

\bibitem[{{Jenkins}(1982)}]{1982MNRAS.200..705J}
{Jenkins}, C.~R. 1982, \mnras, 200, 705, \dodoi{10.1093/mnras/200.3.705}

\bibitem[{{Kapahi}(1981)}]{1981AandAS...43..381K}
{Kapahi}, V.~K. 1981, \aaps, 43, 381

\bibitem[{{Kellermann} \& {Pauliny-Toth}(1969)}]{1969ApJ...155L..71K}
{Kellermann}, K.~I., \& {Pauliny-Toth}, I.~I.~K. 1969, \apjl, 155, L71,
  \dodoi{10.1086/180305}

\bibitem[{{Kellermann} {et~al.}(1998){Kellermann}, {Vermeulen}, {Zensus}, \&
  {Cohen}}]{1998AJ....115.1295K}
{Kellermann}, K.~I., {Vermeulen}, R.~C., {Zensus}, J.~A., \& {Cohen}, M.~H.
  1998, \aj, 115, 1295, \dodoi{10.1086/300308}

\bibitem[{{Komatsu} {et~al.}(2009){Komatsu}, {Dunkley}, {Nolta}, {Bennett},
  {Gold}, {Hinshaw}, {Jarosik}, {Larson}, {Limon}, {Page}, {Spergel},
  {Halpern}, {Hill}, {Kogut}, {Meyer}, {Tucker}, {Weiland}, {Wollack}, \&
  {Wright}}]{Komatsu09}
{Komatsu}, E., {Dunkley}, J., {Nolta}, M.~R., {et~al.} 2009, \apjs, 180, 330,
  \dodoi{10.1088/0067-0049/180/2/330}

\bibitem[{{Kuehr} {et~al.}(1981){Kuehr}, {Pauliny-Toth}, {Witzel}, \&
  {Schmidt}}]{1981AJ.....86..854K}
{Kuehr}, H., {Pauliny-Toth}, I.~I.~K., {Witzel}, A., \& {Schmidt}, J. 1981,
  \aj, 86, 854, \dodoi{10.1086/112957}

\bibitem[{{Laing}(1981)}]{1981MNRAS.195..261L}
{Laing}, R.~A. 1981, \mnras, 195, 261, \dodoi{10.1093/mnras/195.2.261}

\bibitem[{{Laing} {et~al.}(1983){Laing}, {Riley}, \&
  {Longair}}]{1983MNRAS.204..151L}
{Laing}, R.~A., {Riley}, J.~M., \& {Longair}, M.~S. 1983, \mnras, 204, 151,
  \dodoi{10.1093/mnras/204.1.151}

\bibitem[{{Lawrence} {et~al.}(1996){Lawrence}, {Zucker}, {Readhead}, {Unwin},
  {Pearson}, \& {Xu}}]{1996ApJS..107..541L}
{Lawrence}, C.~R., {Zucker}, J.~R., {Readhead}, A.~C.~S., {et~al.} 1996, \apjs,
  107, 541, \dodoi{10.1086/192375}

\bibitem[{{Leahy} {et~al.}(1997){Leahy}, {Black}, {Dennett-Thorpe},
  {Hardcastle}, {Komissarov}, {Perley}, {Riley}, \&
  {Scheuer}}]{1997MNRAS.291...20L}
{Leahy}, J.~P., {Black}, A.~R.~S., {Dennett-Thorpe}, J., {et~al.} 1997, \mnras,
  291, 20, \dodoi{10.1093/mnras/291.1.20}

\bibitem[{{Leahy} \& {Perley}(1991)}]{1991AJ....102..537L}
{Leahy}, J.~P., \& {Perley}, R.~A. 1991, \aj, 102, 537, \dodoi{10.1086/115892}

\bibitem[{{Leahy} {et~al.}(1986){Leahy}, {Pooley}, \&
  {Riley}}]{1986MNRAS.222..753L}
{Leahy}, J.~P., {Pooley}, G.~G., \& {Riley}, J.~M. 1986, \mnras, 222, 753,
  \dodoi{10.1093/mnras/222.4.753}

\bibitem[{{Lister} {et~al.}(2018){Lister}, {Aller}, {Aller}, {Hodge}, {Homan},
  {Kovalev}, {Pushkarev}, \& {Savolainen}}]{MOJAVE_XV}
{Lister}, M.~L., {Aller}, M.~F., {Aller}, H.~D., {et~al.} 2018, \apjs, 234, 12,
  \dodoi{10.3847/1538-4365/aa9c44}

\bibitem[{{Lister} {et~al.}(1994){Lister}, {Gower}, \&
  {Hutchings}}]{1994AJ....108..821L}
{Lister}, M.~L., {Gower}, A.~C., \& {Hutchings}, J.~B. 1994, \aj, 108, 821,
  \dodoi{10.1086/117113}

\bibitem[{{Lister} {et~al.}(2019){Lister}, {Homan}, {Hovatta}, {Kellermann},
  {Kiehlmann}, {Kovalev}, {Max-Moerbeck}, {Pushkarev}, {Readhead}, {Ros}, \&
  {Savolainen}}]{2019ApJ...874...43L}
{Lister}, M.~L., {Homan}, D.~C., {Hovatta}, T., {et~al.} 2019, \apj, 874, 43,
  \dodoi{10.3847/1538-4357/ab08ee}

\bibitem[{{Longair}(1975)}]{1975MNRAS.173..309L}
{Longair}, M.~S. 1975, \mnras, 173, 309, \dodoi{10.1093/mnras/173.2.309}

\bibitem[{{Lonsdale}(1984)}]{1984MNRAS.208..545L}
{Lonsdale}, C.~J. 1984, \mnras, 208, 545, \dodoi{10.1093/mnras/208.3.545}

\bibitem[{{Looney} \& {Hardcastle}(2000)}]{2000ApJ...534..172L}
{Looney}, L.~W., \& {Hardcastle}, M.~J. 2000, \apj, 534, 172,
  \dodoi{10.1086/308753}

\bibitem[{{Ludke} {et~al.}(1998){Ludke}, {Garrington}, {Spencer}, {Akujor},
  {Muxlow}, {Sanghera}, \& {Fanti}}]{1998MNRAS.299..467L}
{Ludke}, E., {Garrington}, S.~T., {Spencer}, R.~E., {et~al.} 1998, \mnras, 299,
  467, \dodoi{10.1046/j.1365-8711.1998.01843.x}

\bibitem[{{MacDonald} {et~al.}(1968){MacDonald}, {Kenderdine}, \&
  {Neville}}]{1968MNRAS.138..259M}
{MacDonald}, G.~H., {Kenderdine}, S., \& {Neville}, A.~C. 1968, \mnras, 138,
  259, \dodoi{10.1093/mnras/138.3.259}

\bibitem[{{Miley} {et~al.}(1975){Miley}, {Wellington}, \& {van der
  Laan}}]{1975AandA....38..381M}
{Miley}, G.~K., {Wellington}, K.~J., \& {van der Laan}, H. 1975, \aap, 38, 381

\bibitem[{{Neff} {et~al.}(1995){Neff}, {Roberts}, \&
  {Hutchings}}]{1995ApJS...99..349N}
{Neff}, S.~G., {Roberts}, L., \& {Hutchings}, J.~B. 1995, \apjs, 99, 349,
  \dodoi{10.1086/192190}

\bibitem[{{Nilsson} {et~al.}(1993){Nilsson}, {Valtonen}, {Kotilainen}, \&
  {Jaakkola}}]{1993ApJ...413..453N}
{Nilsson}, K., {Valtonen}, M.~J., {Kotilainen}, J., \& {Jaakkola}, T. 1993,
  \apj, 413, 453, \dodoi{10.1086/173016}

\bibitem[{{O'Dea} \& {Saikia}(2021)}]{2021AandARv..29....3O}
{O'Dea}, C.~P., \& {Saikia}, D.~J. 2021, \aapr, 29, 3,
  \dodoi{10.1007/s00159-021-00131-w}

\bibitem[{{Owen} \& {Puschell}(1984)}]{1984AJ.....89..932O}
{Owen}, F.~N., \& {Puschell}, J.~J. 1984, \aj, 89, 932, \dodoi{10.1086/113589}

\bibitem[{{Owsianik} \& {Conway}(1998)}]{1998AandA...337...69O}
{Owsianik}, I., \& {Conway}, J.~E. 1998, \aap, 337, 69.
\newblock \doarXiv{astro-ph/9712062}

\bibitem[{{Owsianik} {et~al.}(1999){Owsianik}, {Conway}, \&
  {Polatidis}}]{1999NewAR..43..669O}
{Owsianik}, I., {Conway}, J.~E., \& {Polatidis}, A.~G. 1999, \nar, 43, 669,
  \dodoi{10.1016/S1387-6473(99)00075-5}

\bibitem[{{Pauliny-Toth}(1977)}]{1977IAUS...74...63P}
{Pauliny-Toth}, I.~I.~K. 1977, in Radio Astronomy and Cosmology, ed. D.~L.
  {Jauncey}, Vol.~74, 63

\bibitem[{{Pauliny-Toth} {et~al.}(1978){Pauliny-Toth}, {Witzel}, {Preuss},
  {K{\"u}hr}, {Kellermann}, {Fomalont}, \& {Davis}}]{1978AJ.....83..451P}
{Pauliny-Toth}, I.~I.~K., {Witzel}, A., {Preuss}, E., {et~al.} 1978, \aj, 83,
  451, \dodoi{10.1086/112223}

\bibitem[{{Peacock} \& {Gull}(1981)}]{1981MNRAS.196..611P}
{Peacock}, J.~A., \& {Gull}, S.~F. 1981, \mnras, 196, 611,
  \dodoi{10.1093/mnras/196.3.611}

\bibitem[{{Peacock} \& {Wall}(1981)}]{1981MNRAS.194..331P}
{Peacock}, J.~A., \& {Wall}, J.~V. 1981, \mnras, 194, 331,
  \dodoi{10.1093/mnras/194.2.331}

\bibitem[{{Peacock} \& {Wall}(1982)}]{1982MNRAS.198..843P}
---. 1982, \mnras, 198, 843, \dodoi{10.1093/mnras/198.3.843}

\bibitem[{{Pearson} \& {Readhead}(1981)}]{1981ApJ...248...61P}
{Pearson}, T.~J., \& {Readhead}, A.~C.~S. 1981, \apj, 248, 61,
  \dodoi{10.1086/159130}

\bibitem[{{Pearson} \& {Readhead}(1988)}]{1988ApJ...328..114P}
---. 1988, \apj, 328, 114, \dodoi{10.1086/166274}

\bibitem[{{Peck} \& {Taylor}(2000)}]{2000ApJ...534...90P}
{Peck}, A.~B., \& {Taylor}, G.~B. 2000, \apj, 534, 90, \dodoi{10.1086/308746}

\bibitem[{{Pedlar} {et~al.}(1990){Pedlar}, {Ghataure}, {Davies}, {Harrison},
  {Perley}, {Crane}, \& {Unger}}]{1990MNRAS.246..477P}
{Pedlar}, A., {Ghataure}, H.~S., {Davies}, R.~D., {et~al.} 1990, \mnras, 246,
  477

\bibitem[{{Planck Collaboration} {et~al.}(2020){Planck Collaboration},
  {Aghanim}, {Akrami}, {Ashdown}, {Aumont}, {Baccigalupi}, {Ballardini},
  {Banday}, {Barreiro}, {Bartolo}, {Basak}, {Battye}, {Benabed}, {Bernard},
  {Bersanelli}, {Bielewicz}, {Bock}, {Bond}, {Borrill}, {Bouchet}, {Boulanger},
  {Bucher}, {Burigana}, {Butler}, {Calabrese}, {Cardoso}, {Carron},
  {Challinor}, {Chiang}, {Chluba}, {Colombo}, {Combet}, {Contreras}, {Crill},
  {Cuttaia}, {de Bernardis}, {de Zotti}, {Delabrouille}, {Delouis}, {Di
  Valentino}, {Diego}, {Dor{\'e}}, {Douspis}, {Ducout}, {Dupac}, {Dusini},
  {Efstathiou}, {Elsner}, {En{\ss}lin}, {Eriksen}, {Fantaye}, {Farhang},
  {Fergusson}, {Fernandez-Cobos}, {Finelli}, {Forastieri}, {Frailis},
  {Fraisse}, {Franceschi}, {Frolov}, {Galeotta}, {Galli}, {Ganga},
  {G{\'e}nova-Santos}, {Gerbino}, {Ghosh}, {Gonz{\'a}lez-Nuevo}, {G{\'o}rski},
  {Gratton}, {Gruppuso}, {Gudmundsson}, {Hamann}, {Handley}, {Hansen},
  {Herranz}, {Hildebrandt}, {Hivon}, {Huang}, {Jaffe}, {Jones}, {Karakci},
  {Keih{\"a}nen}, {Keskitalo}, {Kiiveri}, {Kim}, {Kisner}, {Knox},
  {Krachmalnicoff}, {Kunz}, {Kurki-Suonio}, {Lagache}, {Lamarre}, {Lasenby},
  {Lattanzi}, {Lawrence}, {Le Jeune}, {Lemos}, {Lesgourgues}, {Levrier},
  {Lewis}, {Liguori}, {Lilje}, {Lilley}, {Lindholm}, {L{\'o}pez-Caniego},
  {Lubin}, {Ma}, {Mac{\'\i}as-P{\'e}rez}, {Maggio}, {Maino}, {Mandolesi},
  {Mangilli}, {Marcos-Caballero}, {Maris}, {Martin}, {Martinelli},
  {Mart{\'\i}nez-Gonz{\'a}lez}, {Matarrese}, {Mauri}, {McEwen}, {Meinhold},
  {Melchiorri}, {Mennella}, {Migliaccio}, {Millea}, {Mitra},
  {Miville-Desch{\^e}nes}, {Molinari}, {Montier}, {Morgante}, {Moss}, {Natoli},
  {N{\o}rgaard-Nielsen}, {Pagano}, {Paoletti}, {Partridge}, {Patanchon},
  {Peiris}, {Perrotta}, {Pettorino}, {Piacentini}, {Polastri}, {Polenta},
  {Puget}, {Rachen}, {Reinecke}, {Remazeilles}, {Renzi}, {Rocha}, {Rosset},
  {Roudier}, {Rubi{\~n}o-Mart{\'\i}n}, {Ruiz-Granados}, {Salvati}, {Sandri},
  {Savelainen}, {Scott}, {Shellard}, {Sirignano}, {Sirri}, {Spencer},
  {Sunyaev}, {Suur-Uski}, {Tauber}, {Tavagnacco}, {Tenti}, {Toffolatti},
  {Tomasi}, {Trombetti}, {Valenziano}, {Valiviita}, {Van Tent}, {Vibert},
  {Vielva}, {Villa}, {Vittorio}, {Wandelt}, {Wehus}, {White}, {White},
  {Zacchei}, \& {Zonca}}]{2020AandA...641A...6P}
{Planck Collaboration}, {Aghanim}, N., {Akrami}, Y., {et~al.} 2020, \aap, 641,
  A6, \dodoi{10.1051/0004-6361/201833910}

\bibitem[{{Polatidis} {et~al.}(2002){Polatidis}, {Conway}, \&
  {Owsianik}}]{2002evn..conf..139P}
{Polatidis}, A.~G., {Conway}, J.~E., \& {Owsianik}, I. 2002, in Proceedings of
  the 6th EVN Symposium, 139

\bibitem[{{Polatidis} {et~al.}(1995){Polatidis}, {Wilkinson}, {Xu}, {Readhead},
  {Pearson}, {Taylor}, \& {Vermeulen}}]{1995ApJS...98....1P}
{Polatidis}, A.~G., {Wilkinson}, P.~N., {Xu}, W., {et~al.} 1995, \apjs, 98, 1,
  \dodoi{10.1086/192152}

\bibitem[{{Pooley} \& {Henbest}(1974)}]{1974MNRAS.169..477P}
{Pooley}, G.~G., \& {Henbest}, S.~N. 1974, \mnras, 169, 477,
  \dodoi{10.1093/mnras/169.3.477}

\bibitem[{{Porcas}(1981)}]{1981Natur.294...47P}
{Porcas}, R.~W. 1981, \nat, 294, 47, \dodoi{10.1038/294047a0}

\bibitem[{{Press} {et~al.}(1992){Press}, {Teukolsky}, {Vetterling}, \&
  {Flannery}}]{1992nrfa.book.....P}
{Press}, W.~H., {Teukolsky}, S.~A., {Vetterling}, W.~T., \& {Flannery}, B.~P.
  1992, {Numerical recipes in FORTRAN. The art of scientific computing}

\bibitem[{{Rawlings} \& {Saunders}(1991)}]{1991Natur.349..138R}
{Rawlings}, S., \& {Saunders}, R. 1991, \nat, 349, 138,
  \dodoi{10.1038/349138a0}

\bibitem[{{Readhead}(1980)}]{1980IAUS...92..165R}
{Readhead}, A.~C.~S. 1980, in Objects of High Redshift, ed. G.~O. {Abell} \&
  P.~J.~E. {Peebles}, Vol.~92, 165--175

\bibitem[{{Readhead}(1994)}]{1994ApJ...426...51R}
{Readhead}, A. C.~S. 1994, \apj, 426, 51, \dodoi{10.1086/174038}

\bibitem[{{Readhead} {et~al.}(1978){Readhead}, {Cohen}, {Pearson}, \&
  {Wilkinson}}]{1978Natur.276..768R}
{Readhead}, A.~C.~S., {Cohen}, M.~H., {Pearson}, T.~J., \& {Wilkinson}, P.~N.
  1978, \nat, 276, 768, \dodoi{10.1038/276768a0}

\bibitem[{{Readhead} {et~al.}(1996){Readhead}, {Taylor}, {Xu}, {Pearson},
  {Wilkinson}, \& {Polatidis}}]{1996ApJ...460..612R}
{Readhead}, A.~C.~S., {Taylor}, G.~B., {Xu}, W., {et~al.} 1996, \apj, 460, 612,
  \dodoi{10.1086/176996}

\bibitem[{{Readhead} {et~al.}(1994){Readhead}, {Xu}, {Pearson}, {Wilkinson}, \&
  {Polatidis}}]{1994cers.conf...17R}
{Readhead}, A.~C.~S., {Xu}, W., {Pearson}, T.~J., {Wilkinson}, P.~N., \&
  {Polatidis}, A.~G. 1994, in Compact Extragalactic Radio Sources, ed. J.~A.
  {Zensus} \& K.~I. {Kellermann}, 17

\bibitem[{{Readhead} {et~al.}(2021){Readhead}, {Kiehlmann}, {Lister},
  {O'Neill}, {Pearson}, {Sheldahl}, {Siemiginowska}, {Taylor}, \&
  {Wilkinson}}]{2021AN....342.1185R}
{Readhead}, A. C.~S., {Kiehlmann}, S., {Lister}, M.~L., {et~al.} 2021,
  Astronomische Nachrichten, 342, 1185, \dodoi{10.1002/asna.20210049}

\bibitem[{{Rees}(1966)}]{1966Natur.211..468R}
{Rees}, M.~J. 1966, \nat, 211, 468, \dodoi{10.1038/211468a0}

\bibitem[{{Rees}(1967)}]{1967MNRAS.135..345R}
---. 1967, \mnras, 135, 345, \dodoi{10.1093/mnras/135.4.345}

\bibitem[{{Reid} {et~al.}(1995){Reid}, {Shone}, {Akujor}, {Browne}, {Murphy},
  {Pedelty}, {Rudnick}, \& {Walsh}}]{1995AandAS..110..213R}
{Reid}, A., {Shone}, D.~L., {Akujor}, C.~E., {et~al.} 1995, \aaps, 110, 213

\bibitem[{{Riley} \& {Pooley}(1975)}]{1975MmRAS..80..105R}
{Riley}, J.~M., \& {Pooley}, G.~G. 1975, \memras, 80, 105

\bibitem[{{Romney} {et~al.}(1984){Romney}, {Padrielli}, {Bartel}, {Weiler},
  {Ficarra}, {Mantovani}, {Baath}, {Kogan}, {Matveenko}, {Moiseev}, \&
  {Nicholson}}]{1984AandA...135..289R}
{Romney}, J., {Padrielli}, L., {Bartel}, N., {et~al.} 1984, \aap, 135, 289

\bibitem[{{Saikia} {et~al.}(2001){Saikia}, {Jeyakumar}, {Salter}, {Thomasson},
  {Spencer}, \& {Mantovani}}]{2001MNRAS.321...37S}
{Saikia}, D.~J., {Jeyakumar}, S., {Salter}, C.~J., {et~al.} 2001, \mnras, 321,
  37, \dodoi{10.1046/j.1365-8711.2001.04017.x}

\bibitem[{{Sanghera} {et~al.}(1995){Sanghera}, {Saikia}, {Luedke}, {Spencer},
  {Foulsham}, {Akujor}, \& {Tzioumis}}]{1995AandA...295..629S}
{Sanghera}, H.~S., {Saikia}, D.~J., {Luedke}, E., {et~al.} 1995, \aap, 295, 629

\bibitem[{{Scheuer}(1995)}]{1995MNRAS.277..331S}
{Scheuer}, P.~A.~G. 1995, \mnras, 277, 331, \dodoi{10.1093/mnras/277.1.331}

\bibitem[{{Scott} \& {Readhead}(1977)}]{1977MNRAS.180..539S}
{Scott}, M.~A., \& {Readhead}, A.~C.~S. 1977, \mnras, 180, 539,
  \dodoi{10.1093/mnras/180.4.539}

\bibitem[{{Slob} {et~al.}(2022){Slob}, {Callingham}, {R{\"o}ttgering},
  {Williams}, {Duncan}, {de Gasperin}, {Hardcastle}, \&
  {Miley}}]{2022arXiv221016570S}
{Slob}, M.~M., {Callingham}, J.~R., {R{\"o}ttgering}, H.~J.~A., {et~al.} 2022,
  arXiv e-prints, arXiv:2210.16570.
\newblock \doarXiv{2210.16570}

\bibitem[{{Spangler} {et~al.}(1984){Spangler}, {Myers}, \&
  {Pogge}}]{1984AJ.....89.1478S}
{Spangler}, S.~R., {Myers}, S.~T., \& {Pogge}, J.~J. 1984, \aj, 89, 1478,
  \dodoi{10.1086/113649}

\bibitem[{{Spencer} {et~al.}(1991){Spencer}, {Schilizzi}, {Fanti}, {Fanti},
  {Parma}, {van Breugel}, {Venturi}, {Muxlow}, \&
  {Rendong}}]{1991MNRAS.250..225S}
{Spencer}, R.~E., {Schilizzi}, R.~T., {Fanti}, C., {et~al.} 1991, \mnras, 250,
  225, \dodoi{10.1093/mnras/250.1.225}

\bibitem[{{Stanghellini} {et~al.}(1997){Stanghellini}, {Bondi}, {Dallacasa},
  {O'Dea}, {Baum}, {Fanti}, \& {Fanti}}]{1997A&A...318..376S}
{Stanghellini}, C., {Bondi}, M., {Dallacasa}, D., {et~al.} 1997, \aap, 318, 376

\bibitem[{{Stickel} {et~al.}(1994){Stickel}, {Meisenheimer}, \&
  {Kuehr}}]{1994AAS..105..211S}
{Stickel}, M., {Meisenheimer}, K., \& {Kuehr}, H. 1994, \aaps, 105, 211

\bibitem[{{Strom} \& {Willis}(1980)}]{1980AandA....85...36S}
{Strom}, R.~G., \& {Willis}, A.~G. 1980, \aap, 85, 36

\bibitem[{{Tacconi} {et~al.}(2020){Tacconi}, {Genzel}, \&
  {Sternberg}}]{2020ARA&A..58..157T}
{Tacconi}, L.~J., {Genzel}, R., \& {Sternberg}, A. 2020, \araa, 58, 157,
  \dodoi{10.1146/annurev-astro-082812-141034}

\bibitem[{{Taylor} {et~al.}(1996){Taylor}, {Readhead}, \&
  {Pearson}}]{1996ApJ...463...95T}
{Taylor}, G.~B., {Readhead}, A.~C.~S., \& {Pearson}, T.~J. 1996, \apj, 463, 95,
  \dodoi{10.1086/177225}

\bibitem[{{Thakkar} {et~al.}(1995){Thakkar}, {Xu}, {Readhead}, {Pearson},
  {Taylor}, {Vermeulen}, {Polatidis}, \& {Wilkinson}}]{1995ApJS...98...33T}
{Thakkar}, D.~D., {Xu}, W., {Readhead}, A.~C.~S., {et~al.} 1995, \apjs, 98, 33,
  \dodoi{10.1086/192153}

\bibitem[{{Thomasson} {et~al.}(2006){Thomasson}, {Saikia}, \&
  {Muxlow}}]{2006MNRAS.372.1607T}
{Thomasson}, P., {Saikia}, D.~J., \& {Muxlow}, T.~W.~B. 2006, \mnras, 372,
  1607, \dodoi{10.1111/j.1365-2966.2006.10955.x}

\bibitem[{{Tremblay} {et~al.}(2016){Tremblay}, {Taylor}, {Ortiz}, {Tremblay},
  {Helmboldt}, \& {Romani}}]{2016MNRAS.459..820T}
{Tremblay}, S.~E., {Taylor}, G.~B., {Ortiz}, A.~A., {et~al.} 2016, \mnras, 459,
  820, \dodoi{10.1093/mnras/stw592}

\bibitem[{{Turland}(1975)}]{1975MNRAS.172..181T}
{Turland}, B.~D. 1975, \mnras, 172, 181, \dodoi{10.1093/mnras/172.1.181}

\bibitem[{{van Breugel} \& {Fomalont}(1984)}]{1984ApJ...282L..55V}
{van Breugel}, W., \& {Fomalont}, E.~B. 1984, \apjl, 282, L55,
  \dodoi{10.1086/184304}

\bibitem[{{Verkhodanov} {et~al.}(2005){Verkhodanov}, {Trushkin}, {Andernach},
  \& {Chernenkov}}]{2005BSAO...58..118V}
{Verkhodanov}, O.~V., {Trushkin}, S.~A., {Andernach}, H., \& {Chernenkov},
  V.~N. 2005, Bulletin of the Special Astrophysics Observatory, 58, 118.
\newblock \doarXiv{0705.2959}

\bibitem[{{Waggett} {et~al.}(1977){Waggett}, {Warner}, \&
  {Baldwin}}]{1977MNRAS.181..465W}
{Waggett}, P.~C., {Warner}, P.~J., \& {Baldwin}, J.~E. 1977, \mnras, 181, 465,
  \dodoi{10.1093/mnras/181.3.465}

\bibitem[{{Wall} \& {Peacock}(1985)}]{1985MNRAS.216..173W}
{Wall}, J.~V., \& {Peacock}, J.~A. 1985, \mnras, 216, 173,
  \dodoi{10.1093/mnras/216.2.173}

\bibitem[{{Wilkinson} {et~al.}(1994){Wilkinson}, {Polatidis}, {Readhead}, {Xu},
  \& {Pearson}}]{1994ApJ...432L..87W}
{Wilkinson}, P.~N., {Polatidis}, A.~G., {Readhead}, A.~C.~S., {Xu}, W., \&
  {Pearson}, T.~J. 1994, \apjl, 432, L87, \dodoi{10.1086/187518}

\bibitem[{{Wilkinson} {et~al.}(1977){Wilkinson}, {Readhead}, {Purcell}, \&
  {Anderson}}]{1977Natur.269..764W}
{Wilkinson}, P.~N., {Readhead}, A.~C.~S., {Purcell}, G.~H., \& {Anderson}, B.
  1977, \nat, 269, 764, \dodoi{10.1038/269764a0}

\bibitem[{{Xu} {et~al.}(1995){Xu}, {Readhead}, {Pearson}, {Polatidis}, \&
  {Wilkinson}}]{1995ApJS...99..297X}
{Xu}, W., {Readhead}, A.~C.~S., {Pearson}, T.~J., {Polatidis}, A.~G., \&
  {Wilkinson}, P.~N. 1995, \apjs, 99, 297, \dodoi{10.1086/192189}

\end{thebibliography}


%------------------------------------------------------------------------------


%------------------------------------------------------------------------------




%------------------------------------------------------------------------------



%------------------------------------------------------------------------------



\end{document}


https://www.overleaf.com/project/5ef77d8c06422a00016b4f73