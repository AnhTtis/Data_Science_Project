While the development of combinatorial mathematics has always been spurred by applied problems, in recent years both the scope and specificity of applied combinatorics has increased dramatically. In concrete and varied ways, combinatorial approaches are tackling problems across the sciences: from using graph clustering to design reduced order models and controls for complex fluid flows~\cite{meena2018network,meena2021identifying}, employing Ramanujan graphs as interconnection topologies of linkages between elements of supercomputers \cite{aksoy2021ramanujan, young2022spectralfly}, predicting drug interactions using graph neural networks (GNNs)~\cite{knutson2022decoding}, to detecting anomalies in cybersecurity data using graph and hypergraph centrality measures \cite{aksoy2021directional, joslyn2020hypergraph}. 

In celebration and promotion of this eclecticism, here we compile seven open problems in applied combinatorics posed by researchers in academia and government. The problems are organized in self-contained sections, authored by the attributed submitters. Each consists of a problem statement, discussion of relevant application areas, and any partial progress or prior work. The problems are as follows:

\begin{itemize}
    \item[\ref{sec:frias}.] {\bf The Dowker Complex in Metric Graphs}: Fr\'{i}as, Segovia-Dominguez, Chen, and Gel propose problems in topological data analysis (TDA) related to Dowker complexes. 
    \item[\ref{sec:kay}.]{\bf An Application of Probabilistic Combinatorics to Quantum Circuit Expressiveness}: Kay and Bennink introduce a problem in probabilistic combinatorics relevant to quantum circuits. 
    \item[\ref{sec:veldt}.] {\bf The Computational Complexity of the 4-uniform Hypergraph Minimum $s$-$t$ Cut Problem}: Veldt outlines a problem on the computational complexity of hypergraph minimum $s$-$t$ cuts.
    \item[\ref{sec:naumann1}-\ref{sec:naumann2}.] {\bf The Edge and Vertex Elimination Problems in Directed Acyclic Graphs} and {\bf Data Flow Reversal Problems}: Naumann poses several problems involving directed acyclic graphs, motivated by algorithmic differentiation of numerical programs.
    \item[\ref{sec:ratfish}.] {\bf Price of Asynchrony}: Ortiz Marrero and Young pose problems to study how asynchronous updates affect the convergence rate of iterative methods for large-scale systems in scientific computing. 
\item [\ref{sec:aksoy}.] {\bf $(n-k)$-contingent Zero-Forcing for Power Grids}: Aksoy, Petyuk, Roy, and Young propose problems on a variant of graph zero-forcing relevant to structural stability in power grids.
\end{itemize}

We have curated this sample of open problems with several criteria in mind. First, we have chosen problems that are timely and new, in the sense that they are motivated by recent research and not likely to be well-known by the larger community. Second, while far from comprehensive, our varied selection aims to illustrate the diversity of areas engaged by applied combinatorics. Third, while much of combinatorics plausibly ``has applications”, we have searched for problems having clear, specific impact to fields outside pure mathematics, including computing, industry, data science, and scientific software. We thank the reader for their interest. 


