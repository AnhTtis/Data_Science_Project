\section{The Edge and Vertex Elimination Problems in Directed Acyclic Graphs}\label{sec:naumann1}

\begin{flushright}
{\it Uwe Naumann}
\end{flushright}
\date{}

\mySub{Introduction}

Impressive progress in the development of computer hard- and software has been 
made over the past decades. Consequently, numerical simulation has become one 
of the pillars of science and engineering. Practically relevant real-world 
phenomena are modelled mathematically. Their numerical evaluation yields 
multivariate vector functions 
$F: \R^n \rightarrow \R^m$
%$$F: \R^n \rightarrow \R^m : \X \mapsto \Y=F(\X)$$ 
implemented as often 
highly complex computer programs. 

Algorithmic Differentiation \cite{Griewank2008EDP,Naumann2012TAo} of numerical 
programs plays a central role in numerous
areas of computational science and engineering including error estimation, uncertainty quantification, parameter sensitivity analysis, model calibration and 
optimization. 
The resulting derivative programs
can be used, for example, to compute the Jacobian (matrix) 
of (a differentiable) $F.$ 

The data flow in a numerical program induces a directed acyclic graph (dag)
$G=(V,E)$ with integer vertices representing all input, intermediate and output values $v_i,$ $i=1,\ldots,|V|$ of the program and directed edges 
$E \subseteq V \times V$ modelling data dependence.
Association of local partial derivatives 
$d_{i,j}=\frac{\partial v_j}{\partial v_i}$ 
with all edges yields the chain rule of differentiation 
as
\begin{equation} \label{eqn:Baur}
\frac{d v_t}{d v_s} = \sum_{(s,*,t)} \prod_{(i,j) \in (s,*,t)} d_{i,j} \; ,
\end{equation}
where summation is over all paths $(s,*,t)$ connecting a vertex $s$ with 
a vertex $t$ \cite{Baur1983TCo}. (Partial) Derivatives are thus defined for 
arbitrary pairs of values represented by both vertices. 

\mySub{Formal Statement of the Problem}

\begin{definition}[Edge Elimination] \label{ee}
	Let $G=(V,E)$ and $(i,j) \in E.$
	{\em Front-elimination} of $(i,j)$ yields
	$G - (i,j) \equiv G^f=(V^f,E^f)$ such that
	$$
	V^f = \begin{cases}
		V & \text{if}~|P_i|>1 \\
		V \setminus j & \text{otherwise} \\
	\end{cases} \quad \text{and} \quad 
	E^f= E \, \cup \, \{(i,k) : k \in S_j\} 
	$$
at the cost of $|S_j|.$

	{\em Back-elimination} of $(i,j)$ yields
	$G - (j,i) \equiv G^b=(V^b,E^b)$ such that
	$$
	V^b = \begin{cases}
		V & \text{if}~|S_i|>1 \\
		V \setminus i & \text{otherwise} \\
	\end{cases} \quad \text{and} \quad 
	E^b= E \, \cup \, \{(k,j) : k \in P_i \}
	$$
at the cost of $|P_i|.$
\end{definition}
{\em Front-[back-]eliminatable} edges are required to have non-empty 
successor [predecessor] sets. By the chain rule of differentiation, all 
{\em complete} edge elimination sequences transform $G$ into a bipartite 
dag representing the Jacobian matrix of the underlying numerical program.
The cost of an edge elimination sequence is the sum of the costs of the 
individual edge eliminations.
We aim to minimize the cost over all complete edge elimination sequences.
\begin{problem}[{\sc Edge Elimination}]
Given a directed acyclic graph $G$ and a positive integer $k \geq 0,$ is there a complete edge
elimination sequence with cost less than or equal to $k?$
\end{problem}
The elimination of a vertex is equivalent to 
front-elimination of its in-edges. 
Similarly, it is equivalent to back-elimination of its out-edges.
\begin{definition}[Vertex Elimination] \label{ve}
	Let $G=(V,E)$ and $j \in V.$
	{\em Elimination} of $j$ yields
	$G - j \equiv G'=(V',E')$ such that
	$$
	V' = V \setminus j 
	\quad \text{and} \quad 
	E'= E \, \cup \, \{ (i,k) : i \in P_j~\text{and}~k \in S_j \}
	$$
at the cost of $|P_j| \cdot |S_j|.$
\end{definition}
{\em Eliminatable} vertices are required to have non-empty 
predecessor and successor sets. By the chain rule of differentiation, all 
{\em complete} vertex elimination sequences transform $G$ into a bipartite 
dag representing the Jacobian matrix of the underlying numerical program.
The cost of a vertex elimination sequence is the sum of the costs of the 
individual vertex eliminations.
We aim to minimize the cost over all complete vertex elimination sequences.
\begin{problem}[{\sc Vertex Elimination}]
Given a directed acyclic graph $G$ and a positive integer $k \geq 0,$ is there a complete vertex
elimination sequence with cost less than or equal to $k?$
\end{problem}

\mySub{What Is [Not] Known}

Both edge and vertex elimination terminate. Structural and numerical 
correctness follows by the chain rule \cite{Naumann2004Oao}.

\begin{theorem}
{\sc Jacobian Accumulation} is NP-complete.
\end{theorem}
The proof, see \cite{Naumann2008OJa}, uses reduction from {\sc Ensemble 
Computation} \cite{Garey1979CaI}. 
It exploits potential algebraic dependences (in particular, 
equality) among local partial derivatives (labels on edges in the dag).
Under the same assumptions both {\sc Edge Elimination} and 
{\sc Vertex Elimination} turn out to be NP-complete. The computational
complexity of the purely structural formulations in Defs.~\ref{ee} and \ref{ve}
is unknown.

Exiting heuristics for computing good vertex and edge elimination sequences are 
based on the structural formulations, for example \cite{Chen2012AIP,Forth2004JCG,Griewank2003AJa,Pryce2008FAD}.

Incomplete elimination sequences are required in the case of {\em scarcity}
\cite{Griewank2005AaE}. The computational complexity of the 
associated combinatorial {\sc Minimum Edge Count} 
problem asking for a (matrix-free) representation of the Jacobian as a dag 
with a minimal number of edges is unknown \cite{Mosenkis2012Oop}. Ultimately,
combinations {\sc Edge Elimination} and {\sc Vertex Elimination} with 
{\sc Minimum Edge Count} should be considered.