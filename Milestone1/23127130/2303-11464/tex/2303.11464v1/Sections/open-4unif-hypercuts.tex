\section{The Computational Complexity of the 4-uniform Hypergraph Minimum $s$-$t$ Cut Problem}\label{sec:veldt}
\begin{flushright}
	{\it Nate Veldt}
\end{flushright}

\mySub{Introduction} 
Finding a minimum $s$-$t$ cut in a graph is one of the most fundamental and well-known combinatorial optimization problems, studied and applied widely in mathematics, computer science, operations research, and data science. The input to the problem is a graph $G = (V,E)$ with a distinguished source node $s$ and sink node $t$. The goal is to delete the minimum number of edges to destroy all paths from $s$ to $t$, or equivalently partition the nodes in a way that separates $s$ and $t$ while minimizing a cut penalty (see Figure~\ref{fig:graph}). A polynomial time solution for this problem has been known since the 1950s~\cite{ford1956maximal}, and many other algorithms are now taught regularly as a standard part of an undergraduate algorithms course. 
\begin{figure}[h]
	\centering
	\begin{minipage}{.35\textwidth}
		\centering
		\includegraphics[width=\linewidth]{Figures/graphst.pdf}
		\caption{Graph $s$-$t$ cut.}
		\label{fig:graph}
	\end{minipage}%
	\begin{minipage}{0.65\textwidth}
		\textbf{Two formulations of the graph $s$-$t$ cut problem}
		\begin{enumerate}
			\item Find a minimum size set of edges whose deletion separates $s$ from $t$ (e.g., three blue edges in Figure~\ref{fig:graph}).
			\item Partition $V$ into $S \subseteq V$ and $\bar{S} = V\backslash S$ (e.g., white and black nodes in Figure~\ref{fig:graph}), with $s \in S$ and $t \in \bar{S}$, in order to minimize the number of \emph{cut} edges (i.e., edges crossing the partition).
		\end{enumerate}
	\end{minipage}
\end{figure}

The $s$-$t$ cut problem can also be defined when the input is a hypergraph $\mathcal{H} = (V,E)$, in which case each edge $e \in E$ can have two \emph{or more} nodes. If we generalize one way of formulating the graph $s$-$t$ cut problem, the goal is simply to delete a minimum sized set of hyperedges to separate a source node $s$ from a sink node $t$. A polynomial time solution for this problem was given half a century ago by Lawler~\cite{lawler1973}, which works by reducing the hypergraph to an $s$-$t$ cut problem in a \emph{directed} graph with an augmented node set. However, generalizing the second formulation of the graph $s$-$t$ cut problem (partition nodes to separate $s$ and $t$ while minimizing a {cut} penalty) leads to alternative problems that are not equivalent to this edge-deletion problem. When a hyperedge has more than two nodes, there is more than one way to separate those nodes across two node sets $S$ and $\bar{S}$. In a 4-uniform hypergraph (see Figure~\ref{fig:4unif}), we can treat $1$-vs-$3$ hyperedge splits differently from $2$-vs-$2$ splits. Recently there has been a growing interest in solving hypergraph cut problems under these generalized types of cut penalties~\cite{veldt2020hyperlocal,veldt2021approximate,veldt2022hypergraph,zhu2022hypergraph,panli_submodular}, since different ways of splitting up the nodes of a hyperedge may be more or less desirable depending on the application. In many machine learning and data mining applications, for example, hyperedges represent evidence that a set of data objects are related and should be associated with the same label or cluster. In these cases, it is more desirable to split hyperedges in such a way that most (even if not all) nodes are on the same side of a cut. At the same time, using different cut penalties leads to drastic differences in the underlying computational complexity of the problem, even when hyperedges are very small. We specifically consider an open question on the computational complexity of a generalized hypergraph $s$-$t$ cut problem in 4-uniform hypergraphs.

%Hypergraph cut and clustering problems are already widely applied in VLSI layout, sparse matrix partitioning, and various other machine learning and data mining applications where datasets are characterized by multiway interactions (e.g., multiway biological interactions or group social interactions). Recently there has been a growing interest in generalized hypergraph cut functions, since different ways of splitting up the nodes of a hyperedge may be more or less desirable depending on the application. 
%
%In these applications, assigning different cut penalties to hyperedge splits may be more or less desirable for different applications. However, different types of cut penalties  can lead to significant differences in the computational complexity of the hypergraph cut problems. Here we  
%
%
%Recently there has been a surge of interest in algorithms for solving hypergraph cut problems under generalized penalties for splitting hyperedges. 
% 
% 
%There has been a recent surge of interest in algorithms for solving hypergraph cut and clustering problems under these generalized notions of hypergraph cut functions, which includes recent work specifically on generalized hypergraph $s$-$t$ cuts~\cite{veldt2020hyperlocal,veldt2021approximate,veldt2022hypergraph,zhu2022hypergraph}. 
%
%
% This research is motivated by data mining and machine learning applications where data objects are organized into multiway interactions that cannot be modeled as well by graphs (e.g., multiway biological interactions and group social interactions). In these applications, assigning different cut penalties to hyperedge splits may be more or less desirable for different applications. However, different types of cut penalties  can lead to significant differences in the computational complexity of the hypergraph cut problems. Here we 

%	Polynomial-time solutions for some generalized cut penalties are known and are being actively used in data science and machine learning applications~\cite{veldt2020hyperlocal,zhu2022hypergraph,panli_submodular}. Many open questions remain, even for 4-uniform hypergraphs.

\mySub{The cardinality-based hypergraph $s$-$t$ cut problem}
Let $\mathcal{H} = (V,\mathcal{E})$ be a hypergraph with a source node $s$ and sink node $t$. Given a bipartition $\{S, \bar{S}\}$ of the nodes, hyperedges can be classified based on the number (i.e., cardinality) of nodes on the small side of the cut. Edges with $i$ nodes on the small side of the cut are denoted by $\partial_i(S) = \{ e \in \mathcal{E} \colon \min \{|e \cap S|, |e \cap \bar{S} |\} = i\}$.
%	\begin{equation}
%		\partial_i(S) = \{ e \in \mathcal{E} \colon \min \{|e \cap S|, |e \cap \bar{S} |\} = i\}.
%	\end{equation}
We take the minimum between $|e \cap S|$ and $|e \cap \bar{S}|$ to ensure the resulting hypergraph cut function is symmetric, generalizing the fact that a graph cut function is symmetric.
Let $w_i \geq 0 $ be the penalty assigned to each hyperedge in $\partial_i(S)$. The value $w_0$ is set to 0 to ensure that a hyperedge only has a penalty if it is actually {cut}. 
The {cardinality-based} hypergraph $s$-$t$ cut problem is given by
\begin{equation}
	\label{eq:cbstcut}
	\min_{S \subseteq V} \;\; \sum_i w_i |\partial_i(S) |, \hspace{.5cm} \text{ subject to $s \in S$ and $t \in \bar{S}$}.
\end{equation}
When $w_i = 1$ for every $i > 0$, this is equivalent to finding a minimum number of hyperedges to remove to separate $s$ from $t$, and can be solved in polynomial time using the reduction of Lawler~\cite{lawler1973}. If $\mathcal{H}$ is 3-uniform, any way of cutting a hyperedge places exactly one node in  $S$  or exactly one node in $\bar{S}$. Since all hyperedge cut penalties are the same, this is solved by Lawler's algorithm.

%	Veldt, Benson, and Kleinberg~\cite{veldt2022hypergraph} recently showed that a generalized hypergraph-to-graph reduction technique can be applied if and only if the hyperedge cut penalties satisfy a certain submodularity condition, which is equivalent to assuming that there is some concave function $f$ such that $f(i) = w_i$ for each integer $i$ from $0$ to $\lfloor r/2 \rfloor$, where $r$ is the maximum hyperedge size. The same authors showed NP-hardness results for some problems outside this concave regime.

\mySub{Results for 4-uniform hypergraphs}
\begin{figure}[h]
	\centering
	\includegraphics[width=.9\linewidth]{Figures/hyper4-examples.pdf}
	\caption{Optimal $s$-$t$ cuts in a 4-uniform hypergraph, depending on the penalty $w_2$ assigned to $2$-vs-$2$ hyperedge splits (i.e., two nodes on each side of a bipartition). All $1$-vs-$3$ hyperedge splits have a penalty $w_1 = 1$. Previous results have shown the problem is NP-hard for $w_2 \in [0,1)$ and polynomial-time solvable when $w_2 \in [1,2]$. The complexity of the problem remains open for $w_2 > 2$.}
	\label{fig:4unif}
\end{figure}
When $\mathcal{H}$ is 4-uniform, there is a distinction between 2-vs-2 splits (penalty of $w_2$) and 1-vs-3 splits (penalty of $w_1$). As long as $w_1 > 0$, parameters can be scaled without loss of generality so that $w_1 = 1$. For certain choices of $w_2$, Veldt, Benson, and Kleinberg~\cite{veldt2022hypergraph} showed how to reduce the hypergraph problem to a graph $s$-$t$ cut problem, by replacing each hyperedge with a \emph{gadget} involving auxiliary vertices and directed edges. The aim is to design gadgets so that cut penalties in the reduced directed graph match cut penalties in the original hypergraph. In the 4-uniform case, it turns out this is possible if \emph{and only if} $1 \leq w_2 \leq 2$. If $0 \leq w_2 < 1$, the problem can be shown to be NP-hard via reduction from the optimization version of maximum cut, one of Karp's 21 NP-complete problems~\cite{karp2010reducibility}. If $G = (V,E)$ is an unweighted graph representing an instance of maximum cut, this reduction constructs a hypergraph with the same node set plus two additional nodes $s$ and $t$. For each edge $(u,v) \in E$, a hyperedge $\{u,v,s,t\}$ is introduced, which must be cut in the resulting hypergraph $s$-$t$ cut problem since it contains both $s$ and $t$. If $w_2 < w_1 = 1$, then a 2-vs-2 split of the hyperedge is cheaper than a 1-vs-3 split. Therefore, the minimum $s$-$t$ cut is the cut maximizing the number of 2-vs-2 hyperedge splits, which is equivalent to finding a bipartition of nodes in $G$ that maximizes the number of cut edges.

\mySub{Open questions and motivation}
Given the above notation and terminology, our open question can be stated formally as follows.
\begin{problem}
	\label{op:4hyper}
	What is the complexity of the cardinality-based 4-uniform hypergraph $s$-$t$ cut problem when $w_2 > 2$ and $w_1 = 1$?
\end{problem}
Nothing is known about the computational complexity of the problem except that the graph reduction strategy no longer applies. A reasonable first step is to address the special case obtained by setting $w_1 = 1$ and taking the limit as $w_2 \rightarrow \infty$. This limit converges to the \textsc{No-Even-Split} 4-uniform $s$-$t$ cut problem: minimize the number of $1$-vs-$3$ hyperedge splits when separating $s$ and $t$, without making any $2$-vs-$2$ hyperedge splits. Placing node $s$ in a cluster by itself always provides one way to separate $s$ from $t$ without having even splits, but the complexity of finding the \emph{minimum} number of 1-vs-3 splits is unknown.
\begin{problem}
	What is the complexity of the \textsc{No-Even-Split} 4-uniform $s$-$t$ cut problem?
\end{problem}
This special case of Open Problem~\ref{op:4hyper} is particularly interesting given its close relationship to an $s$-$t$ cut problem with a very simple solution. More specifically, this problem at first appears equivalent to setting $w_2 = 1$ and taking a limit as $w_1 \rightarrow 0$, but there is a subtle and interesting difference. Setting $w_2 = 1$ and $w_1 = 0$ results in a degenerate problem that is easily optimized, since separating $s$ from the rest of the nodes has a cut penalty of 0. However, if $w_1 = 1$ and $w_2 \rightarrow \infty$, it may never be optimal to separate $s$ (or $t$) from all other nodes. Despite its close relationship to a degenerate and easily solved problem, it is not clear if \textsc{No-Even-Split} 4-uniform $s$-$t$ cut is NP-hard, or if there is a strategy leading to a polynomial time solution.

A solution to the \textsc{No-Even-Split} problem would hopefully shed light on other polynomial time algorithms or hardness results for the parameter region $w_2 \in (2,\infty)$. In hypergraphs where the maximum hyperedge size is greater than 4, there is an even larger gap between known NP-hardness results and polynomial time solutions. When hyperedges are of arbitrary size and cut penalties satisfy $w_i = f(i)$ for some increasing concave function $f$, the hypergraph problem can be reduced to a directed graph $s$-$t$ cut problem~\cite{veldt2022hypergraph}. There are also some parameter regimes where the problem is known to be NP-hard, but there are many parameter settings where neither hardness results nor polynomial time solutions are known. New techniques for the 4-uniform case would hopefully help close these gaps as well. 

In addition to their theoretical value, answers to these questions have the potential to advance the state of the art in practical hypergraph algorithms for downstream applications. Generalized hypergraph cut problems are already being used in data mining and machine learning applications such as node classification~\cite{panli_submodular,veldt2020hyperlocal,zhu2022hypergraph} and image segmentation~\cite{veldt2021approximate}, but existing methods focus only on the regime where cut penalties are submodular. Cut penalties are also cardinality-based in most applications~\cite{veldt2022hypergraph}, in which case this submodularity property exactly corresponds to cut penalties defined by a concave function $f$. If, for example, penalties were instead chosen so that $w_i = g(i)$ for an increasing and \emph{strictly convex} function $g$, this would provide even more incentive to assign nodes from the same hyperedge to the same cluster or node label. However, the computational complexity of finding minimum cuts in this setting is unknown. The $4$-uniform hypergraph $s$-$t$ cut problem with $w_2 > 2$ is the simplest example of this type of convex-penalty hypergraph cut problem. An answer to Open Problem~\ref{op:4hyper} would provide a needed first step in understanding what is algorithmically possible for this and many other types of generalized hyperedge cut penalties.
