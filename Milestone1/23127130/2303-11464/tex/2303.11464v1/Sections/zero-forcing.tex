% \documentclass{amsart}

% \usepackage{fullpage}
% \usepackage[utf8]{inputenc}
% \RequirePackage{hyperref}
% \RequirePackage{tikz}
% \RequirePackage{listings}
% \RequirePackage{wrapfig}
% \RequirePackage{amsmath, amsthm, amsfonts}

% \newcommand{\R}{{I\!\!R}}
% \newcommand{\X}{{\bf x}}
% \newcommand{\Y}{{\bf y}}
% \newcommand{\V}{{\bf v}}
% \newcommand{\set}[1]{\left\{ #1 \right\}}
% \newcommand{\size}[1]{\left| #1 \right|}

% \theoremstyle{plain}
% \newtheorem{theorem}{Theorem}[section]
% \newtheorem*{defn}{Definition}
% \newtheorem*{problem}{Open Problem}
% \newtheorem{lemma}{Lemma}


\newcommand{\LZF}[2][\ell]{Z_{(#1)}\!\paren{#2}}


\section{$(n-k)$-contingent Zero-Forcing for Power Grids}\label{sec:aksoy}
\begin{flushright}
{\it Sinan G.\ Aksoy, Anthony V.\ Petyuk, Sandip Roy, Stephen J.\ Young}
\end{flushright}


\mySub{Introduction}
With the increasing penetration of microgrids, renewable energy sources, power-generation at the edge, and distributed energy resources, maintaining the stability of power grid transmission has changed dramatically.  For example, the majority of grid power generation used to be achieved by spinning large turbines (e.g. hydro-electric turbines spun by falling water, steam powered turbines in gas/coal/nuclear power plants).  However, in 2022 almost 14\% of the utility scale generation was from wind or solar~\cite{electric_power}. In contrast to turbine-based generation, neither wind nor solar power has any ``inertia" and so is subject to rapid changes in the total generation which are not controlled by operators.  This puts significant pressure on control schemes designed to manage the stability of the power grid through the regulation of power generation.  Such control schemes are also made challenging by the increasing penetration of microgrids in the power system: these are small, regional areas of the power grid which contain sufficient internal generation that disconnect themselves from the broader power grid in order to improve local stability.  As microgrids change their connectivity to the power grid, the fundamental electrical equations which govern power flow on the grid also change, impacting control schemes.

The changing grid not only poses challenges for stable control, but also presents several opportunities to increase both resilience and efficiency.  For example, phasor measurement units (PMUs) have been deployed at selected places within the power-grid over the last 25 years, providing measures of the local power grid state (i.e., voltage magnitude and phase angle) up to 120 times a second -- a significant increase over previous methods yielding one measurement every few seconds.  
%This increased temporal visibility into the power-grid could enable novel control schemes for maintaining the stability of the power grid. 
Another opportunity to increase the resilience and stability of the grid is afforded by the increasing penetration of distributed energy resources (DERs), such as grid-scale battery installations.
%presents another opportunity to increase the resilience and stability of the grid.  
Not only do these large battery installations provide a means to temporally arbitrage power from renewable sources (e.g., store excess power generated during the day by solar panels to use at night), they also provide alternative points of control for the power grid via selective charging, islanding, and discharging to rapidly inject power from the overall grid system.  
%of a grid-scale battery provide a rapid (non-initial) means of injecting power from the overall grid system.  
Furthermore, because of the significantly smaller footprint of grid-scale batteries (as compared to traditional power plants) it is possible to deploy these resources in locations which are advantageous to control and stability. %of the power grid.

Given this variety of power-grid structures and operating points, it might seem nearly intractable to decide whether a collections of PMUs provide sufficient observability of the state of grid, or whether a collection DERs are able to provide stabilization for the grid.  Fortunately, both of these problems can be reduced to a combinatorial problem on the structure of the grid known as zero-forcing.

\begin{definition}[Zero-Forcing Move]
Given a graph $G = (V,E)$ and a set $S \subseteq V$ of vertices colored blue.  A vertex $v \in V-S$ can be colored by a \emph{zero-forcing move} if there is a vertex $s$ such that $\set{v}  = N(s) - S$, that is, if there is a vertex $s \in S$ such that  $v$ is the unique neighbor of $s$ not colored blue. If $v$ is colored via a zero-forcing move that relies on $s$, we will say that $s$ is used to color $v$.
\end{definition}

\begin{definition}[Zero-Forcing Set]
Given a graph $G = (V,E)$, a set $Z \subseteq V$ is called a \emph{zero-forcing set} if there is a sequence of zero-forcing moves such that all vertices in $V$ are colored.
\end{definition}

There is an extensive literature around determining the size and structure of minimal zero-forcing sets (see for instance \cite{ZFbook} and the references therein). Concerning power-grid applications, \cite{monshizadeh2014zero} shows the linear dynamics associated to the power-flow equations are controllable by controllers at a zero-forcing set, and \cite{Brueni:PMUplacement,smith2020optimal} shows if phasor-measurement units are placed at a power-dominating set\footnote{A power-dominating set is a set of vertices $S$ such that, combined with their neighborhood, they form a zero-forcing set.} then the phase of every bus in the network is recoverable. 
%Thus looking forward to the control and observability of the future grid, 
Consequently, distributed energy resource placement (such as grid-scale batteries) and phasor-measurement units may enable the real-time steering of the power-grid to increase efficiency and resiliency. In addition, recent work~\cite{Roy:zero_forcing} has shown zero-forcing sets can be used to identify control points for microgrids.  However, these efforts fail to take into account the dynamic nature of the grid.  In particular, zero-forcing is defined for a static graph which does not allow for changes in the underlying network topology, limiting its applicability to the observability and control of the power grid.

\mySub{Contingent Zero-Forcing}

To address the limitations of zero-forcing in the context of the power grid, 
%and taking inspiration from the contingency analysis for other power-grid parameters, 
we propose the following extension of the zero-forcing to identify resilient sets which can achieve the observability and control of the grid.
\begin{definition}[$(n-k)$-contingent Zero Forcing]
Given a graph $G = (V,E)$, a set $Z \subseteq V$ is a $(n-k)$-contingent zero-forcing set if for every set of edges $E'$ of size at most $k$, $Z$ is a zero-forcing set for $G' = (V, E-E')$.
\end{definition}

We note the edge-leaky zero-forcing number introduced independently by Alameda, Kritschgau, and Young~\cite{MYoung:EdgeLeakyZF} is essentially equivalent to this definition\footnote{In contrast to $(n-k)$-contingent zero-forcing, edges in the edge-leaky zero-forcing process aren't removed from the graph, rather their use is prohibited. Consequently, it is possible that in some stage of the zero-forcing process the contingent zero-forcing process can force a strict superset of the vertices forcible in the equivalent edge-leaky process.  However, Alameda, Kritschgau, and Young go on to show the $\ell$-edge-leaky zero-forcing sets are the same as $\ell$-leaky zero-forcing sets. Thus, by Lemma \ref{L:equiv} $\ell$-edge leaky zero-forcing is equivalent to $(n-\ell)$-contingent zero-forcing.}. Since the transmission and distribution level of power grids frequently exhibit ``tree-like" structure, it is natural to first consider nature of the $(n-k)$-contingent zero-forcing sets on trees.  To explore this, we first recall the following fact about the zero-forcing sets of trees.

\begin{theorem}[\cite{Kenter:LeakyZF}]\label{T:treeZF}
If $T$ is a tree with $t \geq 2$ leaves, then any collection of at least $t-1$ leaves is a zero-forcing set for $T.$
\end{theorem}

In the setting of $(n-k)$-contingent zero-forcing, we apply this to prove the following lemma. 

\begin{lemma}\label{L:tree_kZF}
    Let $T = (V,E)$ be a tree. For any $k \geq 1$, a set $Z \subseteq V$ is a $(n-k)$-contingent zero-forcing set if and only if $Z$ contains all vertices of degree at most $k$.  
\end{lemma}

\begin{proof}
    If $Z$ is a $(n-k)$-contingent zero-forcing set for any graph, then it must contain all vertices of degree at most $k$.  In particular, if any vertex $v$ of degree at most $k$ is not in $Z$ then removing all of the incident edges to $v$ results in a graph where $Z$ is not a zero-forcing set.  

    Now let $Z$ be the set of vertices of degree at most $k$ in $T$ and let $F$ be the forest formed by removing a set of $k$ edges from $T$.  We first note that any isolated vertex in $F$ has degree at most $k$ in $T$ as we removed at most $k$ edges. Suppose there is some tree $T'$ which has two leaves $u,v \not\in Z$. We note that since $u$ and $v$ are in the same tree in $F$, there edge $\set{u,v}$ is not in $T$.  Thus, in order for both $u$ and $v$ to be leaves in $F$ at least $k + k > k$ edges must be removed, contradicting the construction of $F$.  Thus, every tree in $F$ is either an isolated vertex belong to $Z$ or a tree with at most one leaf not in $Z$. Thus,  by Theorem \ref{T:treeZF}, every tree in $F$ is forcible by $Z$.  
\end{proof}

The idea of $(n-k)$-contingent zero-forcing is the not the first attempt to develop a notion of zero-forcing that is resilient to changes in the underling graph. For example, Dillman and Kenter~\cite{Kenter:LeakyZF} consider the resilience in the context of water flows where water can ``leak" out at the joints preventing vertices from participating in the forcing process.  More formally, they give the following definition:

\begin{definition}[$\ell$-leaky Zero-Forcing]
Given a graph $G = (V,E)$, a set $Z \subseteq V$ is an $\ell$-leaky zero-forcing set if for every set of vertices $L \subseteq V$ of size at most $\ell$, $Z$ is a zero-forcing set of $G'$, where $G'$ is formed by adding a pendant vertex to each vertex in $L$.
\end{definition}

%The practical effect of the set of vertices $L$ is that 
Observe that any vertex $v \in L$ can not be used to color any other vertex via a zero-forcing move. In other words, the zero-forcing process ``leaks" out of the graph on the vertices in $L$ so they are only able to result in the coloring of new pendant vertices.  In many ways, the leaky zero-forcing process can be thought of as an vertex version of the contingent zero-forcing process: in leaky zero-forcing an unknown set of vertices can not participate in the zero-forcing process, while in the contingent zero-forcing process an unknown set of edges can not participate in the zero-forcing process. With this observation, it is unsurprising these two processes are effectively equivalent, as we show below.

\begin{lemma}\label{L:equiv}
A set $S$ is a $(n-k)$-contingent zero-forcing set if and only if it is an  $k$-leaky zero-forcing set.
\end{lemma}

\begin{proof}
Let $G$ be a graph and suppose $Z$ is a $(n-k)$-contingent zero-forcing set for $G.$ Fix an arbitrary set of $L$ leaks in the graph $G$, where $\size{L} \leq k$.  We will identify a set $\mathcal{E}$ of at most $\ell$ edges such that there is a forcing sequence of $G - \mathcal{E}$ which does not use any vertex in $L$.  To this end, we iteratively build the set $\mathcal{E}$ of removed edges based on $Z$.  Note that as long as $\size{\mathcal{E}} \leq k$, the forcing process from $Z$ can continue as $Z$ is a $(n-k)$-contingent zero-forcing set.   Now, to identify these edges of $\mathcal{E}$, consider sequentially applying the forcing process until the first time a vertex $v \in L$ is used to color $u$.  At this point, add the edge $\set{v,u}$ to $\mathcal{E}$ and remove it from the graph $G$.  Since all of the neighbors of $v$ are now colored, there is no future step in the forcing process in which $v$ colors a neighbor.  Thus we may effectively remove $v$ from $L$.  Repeating this process results in a set $\mathcal{E}$ of size at most $k$ and a sequence of zero-forcing moves for $G - \mathcal{E}$, such that no vertex in $L$ forces any other vertex. Thus $S$ is a forcing set for $G$ with leaks at all the vertices in $L$.  

The converse proceeds similarly.  Let $Z$ be a $k$-leaky zero-forcing set for $G$ and fix an arbitrary set of $\mathcal{E}$ edges in $G$.   We again interatively apply the zero-forcing process and build a set of leaks $L$ as needed.  Specifically, for each edge $\set{u,v} \in E$ we add the first vertex from the edge that is colored to the set of leaks $L$.  Clearly, this ensures that the edge $\set{u,v}$ is not used in the forcing process and adds at most $k$ vertices to the set $L$. 
\end{proof}

In general, determining the size of the minimal zero-forcing set is $\mathcal{NP}$-\texttt{complete}, even for highly-restricted classes of graphs (see \cite{Fallat:PSD_ZeroForcing}, for instance).  Thus, it is exceedingly likely that determining the size of the minimal $(n-k)$-contingent zero-forcing number is also $\mathcal{NP}$-\textrm{complete}.  However, the class of graphs corresponding to power-grid networks is known to have a number of unusual structural properties (see, for instance \cite{Aksoy2018,Young:NoN}), including average degree between 1 and 2, diameter scaling like the square root of the number of vertices, and the presence of long-cycles.  In other words, typical power-grid networks are tree-like by Lemma \ref{L:tree_kZF} the minimal $(n-k)$-contingent zero-forcing sets are known exactly for trees.  This leads naturally to our first two open problems:

\begin{problem}
    Is there a structural characterization of the minimal-sized $(n-k)$-contingent zero-forcing sets for graphs with few cycles? For sparse graphs with large diameter?
\end{problem}

\begin{problem}
    Is there an efficient algorithm to determine the minimal-sized $(n-k)$-contingent zero-forcing set for graphs with few cycles? For sparse graphs with large diameter?
\end{problem}

Even if the structural properties of power-grid networks are insufficient to yield an efficient means of computing a minimal, or nearly so, $(n-k)$-contingent zero-forcing set, if such a set is sufficiently small the additional computational effort may be worth while.  However, in order to assess the value of identifying the minimal $(n-k)$-contingent zero-forcing sets, it would be helpful to have a rough estimate of size of the resulting set.  

\begin{problem}
    What is the typical size, in terms of the number of vertices and edges, of the minimal $(n-k)$-contingent zero-forcing sets in graphs exhibiting structural properties characteristic of power grids?
\end{problem}

While this question may be answerable by restricting attention to the class of graphs with a few hallmark structural properties (e.g., limitation on density, diameter, maximum degree, etc.) it may be more useful to build off the recent work characterizing the zero-forcing number for the Erd\H{o}s-Reny\'i random graph~\cite{Pralat:ZeroForcingRandom} and random regular graphs~\cite{Pralat:ZF_regular,Sudakov:ZeroForcing}.  While the structural properties of these random graph models are far from those of the power-grid, there are a few recent models such as the Chung-Lu Chain~\cite{Aksoy2018} and the Geometric Delaunay~\cite{Young:GeoDe} models which have been shown to capture an array of structural properties of the power grid.

\mySub{Acknowledgements}
Sinan G.\ Aksoy, Anthony V.\ Petyuk, and Stephen J.\ Young were supported by the Resilience through Data-driven Intelligently-Designed Control (RD2C) Initiative, under the Laboratory Directed Research and Development (LDRD) Program at Pacific Northwest National Laboratory (PNNL).  PNNL is a multi-program national laboratory operated for the U.S. Department of Energy (DOE) by Battelle Memorial Institute under Contract No. DE-AC05-76RL01830. Sandip Roy conducted this research during an appointment at the U.S. National Science Foundation, supported by an Intergovermental Personnel Act agreement with Washington State University.
% \bibliographystyle{siam}
% \bibliography{zfr}
% \end{document}