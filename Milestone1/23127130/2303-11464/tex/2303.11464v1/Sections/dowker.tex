
% \documentclass[joc]{ipart}
% \RequirePackage{hyperref}

% \startlocaldefs
% \theoremstyle{plain}
% \newtheorem{thm}{Theorem}[section]
% \def\SM#1#2{\sum_{#1\in #2}}
% \def\FL#1{\left\lfloor #1 \right\rfloor}
% \def\FR#1#2{{\frac{#1}{#2}}}
% \endlocaldefs

% %%%%%%%%%%%%%%%%%%%%%%%%%%%%%%%%%%%%%%%%%%%%%%%%%%%%%%%%%%
% % OUR OWN LIBRARIES AND DEFINITIONS
% %%%%%%%%%%%%%%%%%%%%%%%%%%%%%%%%%%%%%%%%%%%%%%%%%%%%%%%%%
% \usepackage{graphicx} 
% \usepackage{xcolor}

% \newtheorem{theorem}{Theorem}
% \newtheorem{problem}{Problem}
% \newtheorem{example}{Example}

% \newtheorem{corollary}[theorem]{Corollary}
% \newtheorem{definition}[theorem]{Definition}
% \newtheorem{proposition}[theorem]{Proposition}
% \newtheorem{lemma}[theorem]{Lemma}
% \newtheorem{claim}[theorem]{Claim}
% \newtheorem{conjecture}[theorem]{Conjecture}
% \newtheorem{remark}[theorem]{Remark}
% \newtheorem{algorithm}[theorem]{Algorithm}


% \begin{document}

% \begin{frontmatter}

% \title[The Dowker Complex in Metric Graphs]{The Dowker Complex in Metric Graphs}

% \begin{aug}
%     \author{\fnms{Jose} \snm{Frias}\ead[label=e1]{frias4@cimat.mx}},
%     \address{University of Texas at Dallas\\
%              USA\\
%              \printead{e1}}
%     \author{\fnms{Ignacio} \snm{Segovia-Dominguez}\ead[label=e2]{Ignacio.SegoviaDominguez@UTDallas.edu}},
%     \address{University of Texas at Dallas\\
%              Jet Propulsion Laboratory, Caltech\\
%              USA\\
%              \printead{e2}}
%     \author{\fnms{Yuzhou} \snm{Chen}\ead[label=e3]{yuzhou.chen@temple.edu}},
%     \address{Temple University\\
%              USA\\
%              \printead{e3}}
%     \and
%     \author{\fnms{Yulia R.} \snm{Gel}
%             \ead[label=e4]{ygl@utdallas.edu}}
%     \address{University of Texas at Dallas\\
%              National Science Foundation\\
%              USA\\
%              \printead{e4}}
  
% \end{aug}

% \end{frontmatter}

\section{The Dowker Complex in Metric Graphs}\label{sec:frias}
\begin{flushright}
{\it Jos\'{e} Fr\'{i}as, Ignacio Segovia-Dominguez, Yuzhou Chen, Yulia R.\ Gel }
\end{flushright}

\mySub{Background}

Topological data analysis (TDA) is a set of methods in computational topology that continues to receive increasing attention in data sciences. One of the most relevant tools in TDA is persistent homology (PH)~\cite{CEH2006,ELZ2000, ZC2005}. The aim of PH is to obtain topological information of a finite metric space using a \emph{filtered simplicial complex} $\mathcal{K}$ with vertices in the data set, namely, there exists a sequence of simplicial complexes $\mathcal{K}_{0}\subset\mathcal{K}_{1}\subset \cdots\subset \mathcal{K}_{n-1}\subset\mathcal{K}_{n}=\mathcal{K}$. A non-trivial element $\gamma\in H_{p}(\mathcal{K}_{i})$, in the $p$-dimensional homology group of $\mathcal{K}_{i}$, is commonly called a \emph{$p$-dimensional topological feature} and is associated with a point $(i_b,j_d) \in \mathbb{R}^2$,  where $i_b$ and $j_d$ are the \emph{birth} and \emph{death} of $\gamma$, respectively (i.e., the indices of the simplicial complex at which $\gamma$ is first and last observed, respectively, that is, when it becomes trivial in homology or is merged to another topological feature that was born before). The lifespan of the topological feature is $j_d-i_b$.
The \emph{persistence diagram} $PD$ of the filtered simplicial complex $\mathcal{K}$ is the multiset $PD(\mathcal{K})=\{(i_b,j_d) \in \mathbb{R}^2\mid i_b<j_d\}$, where the multiplicity of $(i_b,j_d)$ is the number of topological features in the filtered simplicial complex $\mathcal{K}$ that are born and die at $i_b$ and $j_d$, respectively. A persistence diagram summarizes the evolution of the homology groups of $\mathcal{K}$ along the filtration. Two persistence diagrams can be compared using metrics such as the bottleneck or the Wasserstein distances (please see~\cite{CDO2014,CEH2006, ELZ2000, ZC2005} for a detailed exposition on persistence diagrams and distances in PH). 

Metric spaces induced by weighted graphs are particularly interesting due to their combinatorial foundations and  their applications in data analysis~\cite{ABB2020,LFWX2022}. A weighted graph $\mathcal{G}=(\mathcal{V},\mathcal{E},\omega)$ contains a set of vertices $\mathcal{V}$, an edge set $\mathcal{E}\subset \mathcal{V}\times \mathcal{V}$, and a weight function $\omega:\mathcal{E}\rightarrow \mathbb{R}_{+}$. We say that a graph $\mathcal{G}$ is \emph{simple} if it does not contain  self-edges nor multiple edges. Given a path $\gamma$ in a weighted graph $\mathcal{G}$, the \emph{length} of $\gamma$ is the sum of the weights of the edges in $\gamma$.  A connected weighted graph $\mathcal{G}$ is then endowed with the \emph{geodesic distance} $d_{\mathcal{G}}:\mathcal{ V}\times \mathcal{V} \rightarrow \mathbb{R}_{\geq 0}$, defined on a pair of vertices $u,v\in \mathcal{V}$ as the minimum length among all the paths connecting $u$ and $v$. The set $(\mathcal{V}, d_{\mathcal{G}})$ is a finite metric space. \par 

 One of the most widely used simplicial complexes in TDA is the Vietoris-Rips simplicial complex~\cite{CDO2014, ZC2005}. One important property of this complex is that it is completely determined by its $1$-skeleton, which makes it suitable for computations. 
 
\begin{definition}[{Vietoris-Rips Complex}]\label{df:vrc}
Let $\mathcal{G}=(\mathcal{V},\mathcal{E}, \omega)$ be a weighted simple graph with induced geodesic distance $d_{\mathcal{G}}$. For $\alpha \in \mathbb{R}_{\geq 0}$, we define the \textbf{Vietoris-Rips complex} $VR_{\alpha}(\mathcal{G})$ as the abstract simplicial complex with vertices in $\mathcal{V}$ and, for $k\geq 2$, a $k$-simplex  $\sigma=[x_{0},x_{1},\ldots,x_{k}]\in VR_{\alpha}(\mathcal{G})$ if and only if $d_{\mathcal{G}}(x_{i},x_{j})\leq \alpha$ for $0\leq i\leq j\leq k$.
\end{definition}

However, if the cardinality of $\mathcal{V}$ is large, the number of simplices in $VR_{\alpha}(\mathcal{G})$,  for large values $\alpha$, could be excessively large to be analyzed using computational tools. An alternative to deal with the scalability problem is to construct another simplicial complex with set of vertices $L\subset \mathcal{V}$, where $|L|<|\mathcal{V}|$. One of such simplicial complexes is the witness complex~\cite{ABB2020,BGO2009, DC2004, DFW2015}. 

\begin{definition}[{Witness Complex}]\label{df:wc}
Let $\mathcal{G}=(\mathcal{V},\mathcal{E}, \omega)$ be a weighted simple graph with induced geodesic distance $d_{\mathcal{G}}$, and take subsets $L,W\subset \mathcal{V}$. For $\alpha \in \mathbb{R}_{\geq 0}$, let $Wit_{\alpha}(W,L)$ be the abstract simplicial complex with set of vertices $L$, and a simplex $\sigma\in Wit_{\alpha}(W,L)$ if and only if for every $\tau\subseteq \sigma$ there exists $w\in W$ such that $d_{\mathcal{G}}(w,l)\leq d_{\mathcal{G}}(w,l') +\alpha$ for all $l\in\tau$ and $l'\in L\setminus\tau$.  The complex $Wit_{\alpha}(W,L)$ is called the \textbf{witness complex} of $\mathcal{G}$ with set of \textbf{witnesses} $W$ and \textbf{landmarks} $L$.
\end{definition}

Given a sequence of non-decreasing values $0\leq\alpha_0\leq\alpha_1\leq\cdots \leq\alpha_{n-1}\leq\alpha_n$, the associated Vietoris-Rips complexes constructed on $\mathcal{G}$ satisfy $VR_{\alpha_{0}}(\mathcal{G})\subseteq VR_{\alpha_{1}}(\mathcal{G}),\subseteq \cdots \subseteq VR_{\alpha_{n}}(\mathcal{G})$, namely, the sequence of scale values define a filtered simplicial complex. If we select sets of landmarks and witnesses $L,W\subset\mathcal{V}$, there exists also a filtration  $Wit_{\alpha_{0}}(L,W)\subseteq Wit_{\alpha_{1}}(L,W)\subseteq \cdots \subseteq Wit_{\alpha_{n}}(L,W)$. Then persistent homology can be computed in both filtered simplicial complexes. However, computation of witness complexes on graphs is challenging due to the difficulty to determine a witness of some simplexes, as the next illustrative example shows.

\begin{example}
Suppose we are interested in the $1$-dimensional persistent homology of the $1$-weighted cycle graph $\mathcal{C}_{m}$, $m\geq 4$, with vertices set $\mathcal{V}$. Let $L\subset\mathcal{V}$ be a set of landmarks and $W=\mathcal{V}$ be a set of witnesses (see Figure~\ref{fig:f1}a). It is not difficult to determine the smallest scale value $\alpha> 0$ such that $Wit_{\alpha}(L,W)$ contains the $1$-simplex  $\{l_{i},l_{i+1}\}$, for any  two consecutive landmarks $l_{i},l_{i+1}\in L$ (in Figure~\ref{fig:f1}b the $1$-simplexes are represented by edges connecting consecutive landmarks). These $1$-simplexes form a $1$-dimensional topological feature $\gamma$ that is born at $\alpha$. However, given a sequence of three consecutive landmarks $l_{i-1},l_{i},l_{i+1}\in L$, it is combinatorially more difficult to determine a witness and a scale value at which the $1$-simplex $\{l_{i-1},l_{i+1}\}$ (dotted edge in Figure~\ref{fig:f1}b), appears in the filtered witness complex, or, even more difficult, when $\gamma$ vanishes.  
\end{example}

\begin{figure}
  \centering
    \includegraphics[width=10cm]{Figures/fig1.pdf}
  \caption{Landmarks in a cycle graph.}
  \label{fig:f1}
\end{figure} 


The Dowker complex~\cite{CDO2014, CM2018}  is a construction obtained after relaxing the definition of the witness complex, Definition~\ref{df:wc}, to make it easier to compute:   

\begin{definition}[{Dowker Complex}]\label{df:dc}
Let $\mathcal{G}=(\mathcal{V},\mathcal{E}, \omega)$ be a weighted simple graph with induced geodesic distance $d_{\mathcal{G}}$, and let $L,W\subset \mathcal{V}$.
 For $\alpha \in \mathbb{R}_{\geq 0}$, let $Dow_{\alpha}(W,L)$ be the abstract simplicial complex with set of vertices $L$ and a simplex $\sigma\in Dow_{\alpha}(W,L)$ if and only if there exists $w\in W$ such that $d_{\mathcal{G}}(w,l)\leq \alpha$ for all $l\in\sigma$. The simplicial complex $Dow_{\alpha}(W,L)$ is called the \textbf{Dowker complex} of $\mathcal{G}$ with sets of witnesses and landmarks $W$ and $L$, respectively.
\end{definition}

Note that witness and Dowker complexes constructions strongly depend on the selection of the sets $L$ and $W$. In~\cite{DC2004}, two algorithms to define the set of landmarks $L$ are developed: \emph{random} and \emph{maxmin} algorithms. More recently,
\cite{ABB2020} proposes the $\epsilon$-nets algorithm  to obtain a set of landmarks $L\subset \mathcal{V}$ satisfying both properties of sparsity and proximity to the set of points $\mathcal{V}\setminus L$, thereby allowing for a trade-off between computational complexity and potential information loss. 

\begin{definition}[{$\epsilon$-net}]\label{df:en}
 Let $(\mathcal{V},d_{\mathcal{G}})$ be the finite metric space obtained from the weighted graph $\mathcal{G}=(\mathcal{V},\mathcal{E},\omega)$. Given $L=\{u_{1},u_{2},\ldots,u_{l}\}\subset \mathcal{V}$ and $\epsilon \geq 0$ then:
\begin{itemize}
    \item [(i)] The set $L$ is an \textbf{$\epsilon$-sample} of $\mathcal{G}$ if the collection $\{\mathcal{N}(u_{i})\}_{i=1}^{l}$ of closed $\epsilon$-neighborhoods of points in $L$ covers $\mathcal{V}$, i.e. for any $v\in \mathcal{V}$ there exists $u_{j}\in L$ such that $d_{\mathcal{G}}(v,u_{j})\leq \epsilon$.
    \item[(ii)] $L$ is \textbf{$\epsilon$-sparse} if for any two distinct points $u_{i},u_{j}\in L$, their distance $d_{\mathcal{G}}(u_{i},u_{j})>\epsilon$.
    \item[(iii)] The set $L$ is an \textbf{$\epsilon$-net} of $\mathcal{G}$ if it is an $\epsilon$-sample of $\mathcal{G}$ and is $\epsilon$-sparse.
\end{itemize}
\end{definition}

\mySub{Open Problems}

Along the present section, $\mathcal{G}=(\mathcal{V},\mathcal{E},\omega)$ is a weighted simple graph and $d_{\mathcal{G}}$ is the geodesic distance induce by $\mathcal{G}$ on $\mathcal{V}$. The construction of Dowker simplicial complexes on the metric space $(\mathcal{V},d_{\mathcal{G}})$ relies on the selection of the sets of landmarks and witnesses. Usually the set of witnesses is taken to be the set of vertices, $W=\mathcal{V}$, or the complement of landmarks, $W=\mathcal{V}\setminus L$. 
Furthermore, there are numerous possibilities of selecting a set of landmarks.
The existence of an $\epsilon$-net defining the set of landmarks guarantees some combinatorial and geometric properties, as shown in~\cite{ABB2020}. Furthermore, \cite{ABB2020} proved that for a weighted graph $\mathcal{G}=(\mathcal{V},\mathcal{E},\omega)$ and a particular $\epsilon$, there exists an $\epsilon$-net for $(\mathcal{V},d_{\mathcal{G}})$ whose cardinality admits a bound depending on $\epsilon$, the diameter of the graph and the number of vertices (Theorem~3). However, this bound does not pretend to be any close to a sharp bound, while its further analysis has a number of important implication for graph learning applications in machine learning and statistics. As a result, we formulate the first open problem on the Dowker complexes on graphs as follows:

\begin{problem}\label{pr:1}
Given a weighted simple graph  $\mathcal{G}=(\mathcal{V},\mathcal{E},\omega)$ and $\epsilon>0$, determine  optimal upper and lower bounds for the number of elements of an $\epsilon$-net for $\mathcal{G}$. 
\end{problem} 

Three different methods to obtain $\epsilon$-nets in metric graphs, namely, the greedy, iterative and \textit{SPTpruning} algorithms, are proposed in~\cite{ABB2020}. These algorithms are experimentally compared with respect to the number of landmarks they produce for the same graph, including their execution time. 
As such, the next natural question arises:

\begin{problem}\label{pr:2}
For a weighted simple graph  $\mathcal{G}=(\mathcal{V},\mathcal{E},\omega)$, propose an algorithm to construct an $\epsilon$-net for $\mathcal{G}$ such that:
\begin{itemize}
\item[(i)]  It is based on a lower number of landmarks than that of the existing algorithms;
\item[(ii)] The time of execution of the algorithm is lower than the existing algorithms.
\end{itemize}
 \end{problem} 

Since the real-world datasets are usually noisy and incomplete, a suitable choice of landmarks in the Dowker complexes has multi-fold potential benefits in applications. 
First, it tends to improve performance of graph learning algorithms by focusing on the most essential
topological characteristics (i.e. the ``data skeleton"). Second, it helps enhance robustness to noise/data perturbations.
Third, by reducing computational costs, it opens a perspective of utilizing TDA for downstream tasks in graph learning. Indeed, as computing persistence homology on large graphs is often infeasible due to prohibitive computational costs,
the Dowker complex construction offers a promising alternative by the means of the suitable landmark selection. Additionally, the data separation in landmarks and witnesses trigger an intrinsic spatial decomposition that match with the natural local divisions needed when dealing with modern massive parallel computing~\cite{LEWIS2015}.

Given the current proliferation of the Vietoris-Rips complex in TDA, the natural question to address is
the comparison of persistent homology obtained from the Vietoris-Rips  and  Dowker complexes constructions on the same weighted graph $\mathcal{G}$. We denote with $PD^{i}(\mathcal{K})$ the persistence diagram of dimension $i$ for the filtered simplicial complex $\mathcal{K}$. In the case of metric graphs,  $1$-dimensional homology is relevant. It is known that for a large enough scale value $\alpha$, the persistence diagram $PD^{1}(VR_{\alpha}(\mathcal{G}))$ contains as many points (counting multiple points) as the genus of $\mathcal{G}$, the cardinality of a minimal system of cycles of $\mathcal{G}$~\cite{GGPSWWZ2017}. A main concern on the selection of landmarks to construct the witness or Dowker complexes is the loss of relevant topological information. Then, the next question arises:    
 

\begin{problem}\label{pr:3} Let   $\mathcal{G}=(\mathcal{V},\mathcal{E},\omega)$ be weighted simple graph. Take the set of witnesses $W=\mathcal{V}$.
\begin{itemize}
    \item[(i)] Determine $\epsilon> 0$ and the way to select an $\epsilon$-net $L\subset \mathcal{V}$ such that $PD^{1}(Dow_{\alpha}(W,L))$ has cardinality  equal to the genus of $\mathcal{G}$, for a sufficiently large value $\alpha$ .
    \item[(ii)] Assuming that an $\epsilon$-net for $\mathcal{G}$ satisfying (i) is given. Find a bound for $d_{B}(PD^{1}(VR_{\alpha}(\mathcal{G})), PD^{1}(Dow_{\alpha}(W, L)))$, in terms of the lengths of cycles in $\mathcal{G}$ and $\epsilon$, where $d_{B}$ is the bottleneck distance in persistence diagrams.
\end{itemize}
 \end{problem} 

Finally, another important but largely unexplored topic
for PH obtained from the Dowker complexes is stability. 
Stability ensures that after applying a  ``small perturbation'' to  a metric graph $\mathcal{G}$ in order to obtain a new metric graph $\mathcal{G}'$, the corresponding persistence diagrams are close with respect to some distance, for instance the bottleneck or Wasserstein distance. By perturbation we refer to any transformation that may include edge deletion, addition, cleaving or contraction, or changing the weight function (which automatically modifies the distance $d_{\mathcal{G}}$). In the case of complexes depending on a selection of landmarks and witnesses a perturbation could be   changing the selection of these two sets. For the Dowker complex,  some stability results have been obtained, including the Dowker interleaving and the Dowker duality~\cite[see Section 4.2.3]{CDO2014}. 
In particular, Dowker duality can be interpreted as the property such that if $L,W\subset\mathcal{V}$ are the sets of landmarks and witnesses of $\mathcal{G}$, then $Dow_{\alpha}(W,L)$ and $Dow_{\alpha}(L,W)$ have the same homotopy type and, then, the two filtered simplicial complexes have the same persistent homology.


\begin{problem}\label{pr:4} Let $\mathcal{G}=(\mathcal{V},\mathcal{E},\omega)$ be a weighted graph and let $L,W\subset \mathcal{V}$ be sets of landmarks and witnesses for $\mathcal{G}$. 
\begin{itemize}
\item[(i)] Let $\mathcal{G}'=(\mathcal{V},\mathcal{E},\omega')$ be the metric graph with the same set of vertices and edges of $\mathcal{G}$, but another weight function $\omega'$. Find a bound for the bottleneck distance of the persistence diagrams corresponding to the Dowker complexes of $\mathcal{G}$ and $\mathcal{G}'$ in terms of the supremum distance between $\omega$ and $\omega'$. %\YGL{Jose, infinite distance is not defined.}
\item[(ii)] If two sets of landmarks $L$ and $L'$ for $\mathcal{G}$ are $\epsilon$-nets. Find conditions on the landmarks, such that $Dow_{\alpha}(W,L)$ and $Dow_{\alpha}(W,L')$ have the same homotopy type.
\end{itemize}
\end{problem}

Recent studies have addressed some  properties of the Dowker complexes in directed graphs. For instance, a complete characterization of the Dowker persistence diagrams for cycle networks is presented, and some stability properties have been proven for pair swaps by~\cite{CM2018} and for conjugate and  shift equivalent relations~\cite{Cote2023}. 
While both the theoretical properties and practical utility of the Dowker complexes on graphs yet remain largely unexplored, we believe that the Dowker complexes have a potential to address many bottlenecks that currently preclude broader applications of TDA in analysis of real-world graph datasets. 
\mySub{Acknowledgements}  This work was supported by the Office of Naval Research (ONR) award N00014-21-1-2530. Any opinions, findings, conclusions, or recommendations expressed in this paper are those of the authors and do not necessarily reflect the views of ONR.

% \bibliographystyle{imsart-number}
% \bibliography{references}

% \end{document}
