\documentclass[10pt,twocolumn,letterpaper]{article}

\usepackage{iccv}
\usepackage{times}
\usepackage{epsfig}
\usepackage{graphicx}
\usepackage{amsmath}
\usepackage{amssymb}


\usepackage{pifont}
\usepackage{booktabs}
\usepackage{multirow}
\usepackage{xspace}
\usepackage[dvipsnames]{xcolor}

\usepackage{algorithm}
\usepackage{algpseudocode}
\usepackage[accsupp]{axessibility}


\definecolor{darkergreen}{RGB}{21, 152, 56}
\definecolor{red2}{RGB}{252, 54, 65}
\newcommand{\yesmark}{\textcolor{darkergreen}{\ding{52}}}
\newcommand{\nomark}{\textcolor{red2}{\ding{56}}}
\newcommand{\byesmark}{{\ding{52}}}
\newcommand{\bnomark}{{\ding{56}}}

\usepackage[pagebackref=true,breaklinks=true,letterpaper=true,colorlinks,bookmarks=false]{hyperref}

\usepackage[capitalize]{cleveref}

\newcommand{\oursfull}{\textbf{S}torage-\textbf{e}fficient V\textbf{i}sion \textbf{T}raining (\textbf{\ours})\xspace}
\newcommand{\ours}{SeiT\xspace}
\newcommand{\supp}{Appendix}

\newcommand{\bh}[1]{\textcolor{teal}{#1\xspace}}
\newcommand{\sh}[1]{\textcolor{orange}{#1\xspace}}

\iccvfinalcopy %

\def\iccvPaperID{1251} %
\def\httilde{\mbox{\tt\raisebox{-.5ex}{\symbol{126}}}}

\ificcvfinal\pagestyle{empty}\fi

\begin{document}

\title{SeiT: Storage-Efficient Vision Training with Tokens Using 1\% of Pixel Storage}

\author{Song Park$^*$ \quad Sanghyuk Chun$^*$ \quad Byeongho Heo \quad Wonjae Kim \quad Sangdoo Yun\\
{\small $^*$ Equal contribution}\\
{\small \,}
\\
{NAVER AI Lab}\\
}

\maketitle
\ificcvfinal\thispagestyle{empty}\fi


\begin{abstract}
We need billion-scale images to achieve more generalizable and ground-breaking vision models, as well as massive dataset storage to ship the images (\eg, the LAION-5B dataset needs 240TB storage space). However, it has become challenging to deal with unlimited dataset storage with limited storage infrastructure. A number of storage-efficient training methods have been proposed to tackle the problem, but they are rarely scalable or suffer from severe damage to performance. In this paper, we propose a storage-efficient training strategy for vision classifiers for large-scale datasets (e.g., ImageNet) that only uses 1024 tokens per instance without using the raw level pixels; our token storage only needs $<$1\% of the original JPEG-compressed raw pixels. We also propose token augmentations and a Stem-adaptor module to make our approach able to use the same architecture as pixel-based approaches with only minimal modifications on the stem layer and the carefully tuned optimization settings. Our experimental results on ImageNet-1k show that our method significantly outperforms other storage-efficient training methods with a large gap. We further show the effectiveness of our method in other practical scenarios, storage-efficient pre-training, and continual learning.
Code is available at \url{https://github.com/naver-ai/seit}
\end{abstract}

\section{Introduction}

We need billion-scale data points for more generalizable and ground-breaking vision models, \eg,  400M image-text pairs \cite{clip}, 1.8B image-text pairs \cite{jia2021scaling}, or 3.6B weakly-annotated images \cite{mahajan2018exploring, singh2022revisiting}. However, designing and operating a high-performance but fault-tolerant generic distributed dataset is a very expensive and challenging problem \cite{zaharia2012spark}. This problem has become more challenging for vision datasets compared to language datasets. For example, training GPT-2 with 8M documents only need 40GB of storage \cite{radford2019gpt2}, while the larger GPT-3 is trained with 410B tokens with 570GB of storage \cite{brown2020gpt3}. On the other hand, storing images requires significantly more storage space than storing language. For example, the ImageNet-21k dataset \cite{russakovsky2015imagenet} with 11M images requires a 1.4TB storage size, 2.5 times larger than GPT-3 storage despite containing fewer data points. Larger-scale datasets for large-scale pre-training require even more massive storage, \eg, 240TB for 5B images \cite{laion5b}. Consequently, storage remains a major bottleneck in scaling up vision models compared to language models.

\begin{figure}[t]
    \centering
    \includegraphics[width=\linewidth]{figures/Storage-acc-new.pdf}
    \caption{\small \textbf{Training data storage vs. ImageNet 1k Accuracy.}
    Comparisons on ImageNet-1k \cite{russakovsky2015imagenet} using ViT-B/16 backbone \cite{vit} are shown.
    Our \ours ({\color{red}red lines}) significantly outperforms other storage-efficient methods with the same storage size, achieving 74.0\% and 78.4\% top-1 acc with only 1.36GB utilizing tokenizers trained with ImageNet-1k and OpenImages, respectively. Note that the original pixel-based image storage requires 140GB of storage to achieve 81.8\% top-1 accuracy. Details are
    in \cref{tab:main_results_supp}.
    }
    \label{fig:teaser}
    \vspace{-.5em}
\end{figure}

Why do images require a large storage size than text? This is because while the nature of language is discrete, images are continuous in nature. 
Also, while the text quality is independent of document length, the image quality directly affects the storage size; better quality images require larger storage sizes. 
Although a lossy JPEG compression can reduce the storage size, as witnessed by Rombach \etal, still ``most bits of a digital image corresponds to imperceptible details'' \cite{rombach2022latentdiffusion}. Such imperceptible details (\eg, fine-grained details or high-frequency information of images) could be unnecessary for our desired vision classifiers. However, deep vision models are vulnerable to imperceptible high-frequency perturbations \cite{fgsm, pgd, autoattack} or unreasonably local areas \cite{geirhos2019stylized_imagenet, bahng2019rebias, scimeca2022wcst-ml}, implying that deep vision models attend too much to imperceptible details instead of the true property of objects. Therefore, we can expect that we can still achieve a high-performing vision model with the reduced image dataset by removing the imperceptible details.




There are two major directions to storage-efficient vision model training. The first direction aims to reduce the total number of data points by discarding less important samples \cite{coresets_for_efficient,2021mansheej_inportant_examples_diet,cscores} or synthesizing more ``condensed'' images than natural images \cite{data_condensation,data_condensation_augmentation}. However, this approach shows a significant performance drop compared to the full dataset (the blue and yellow lines in \cref{fig:teaser}) or cannot be applied to large-scale datasets due to their high complexity. Also, as the sampled or synthesized images are still normal images, these methods still suffer from an inefficient compression ratio to express imperceptible details. Furthermore, these methods need to compute the importance score or the sample-wise gradient of each sample by learning models with the full dataset. It makes these approaches not applicable to unseen datasets or newly upcoming data streams.

The other approach involves reducing the size of each image while keeping the total number of images. For example, by learning a more efficient compression method \cite{balle2016end, balle2018variational}. However, the neural compression methods have been mostly studied on extremely small-scale datasets (\eg, 24 images \cite{kodak} or 100 images \cite{asuni2014testimages}), and their generalizability to large-scale datasets is still an open problem. Moreover, the goal of neural compression is to compress an image and recover the original image as perfectly as possible, not to extract the most discriminant features for object recognition tasks. In response to these limitations, no neural compression method has been used to compress large-scale datasets like ImageNet \cite{russakovsky2015imagenet} to train deep vision models.

Due to the difficulty of the practical usage of neural compression, practitioners have attempted to reduce storage usage by controlling image quality. For example, the LAION dataset \cite{laion400m, laion5b} stores each image at 256 $\times$ 256 resolution, which takes up only 36\% of ImageNet images (469 $\times$ 387 resolution on average). Similarly, adjusting the JPEG compression quality can reduce the overall storage. As shown in \cref{fig:teaser} (green and purple lines), these approaches work well in practice compared to sampling-based methods. However, these methods have a limited compression ratio; if the compression ratio becomes less than 25\%, the performances drop significantly. By adjusting the image resolution with a 4\% compression ratio and JPEG quality with a 7\% compression ratio, we achieve 63.3\% and 67.8\% top-1 accuracies, respectively. In contrast, our approach achieves 74.0\% top-1 accuracy with only a 1\% compression ratio.

All shortcomings of the previous methods originate from the fact that too many imperceptible bits are assigned to store a digital image, which is misaligned with our target task. 
Our approach overcomes this limitation by storing images as tokens rather than pixels, using pre-trained vision tokenizers, such as VQGAN \cite{vqgan} or ViT-VQGAN tokenizer \cite{vitvqgan}. Introducing \oursfull, we convert each image to 32 $\times$ 32 tokens. The number of possible cases each token can have (the codebook) is 391, which takes only 1.15KB to store each token (assuming that the number of 391 cases can be expressed in 9 bits). It costs only less than 1.5GB for storing 140GB pixel-based storage of ImageNet. We train Vision Transformer (ViT) models on our tokenized images with minimum modifications. First, a 1024-length tokenized image is converted to a $32 \times 32 \times 32$ tensor by using pre-trained 32-dimensional codebook vectors from ViT-VQGAN. Next, we apply random resized crop (RRC) to the tensor to get a $32 \times 28 \times 28$ tensor. Then, to convert the tensor into a form that ViT can handle, we introduce \textit{Stem-Adapter} module that converts the RRC-ed tensor into a tensor of size $768 \times 14 \times 14$, the same as the first layer input of ViT after the stem layer. Because the image-based augmentations are not directly applicable to tokens, we propose simple token-specific augmentations, including \textit{Token-EDA} (inspired from easy data augmentation (EDA) \cite{wei2019eda} for language), \textit{Emb-Noise} and \textit{Token-CutMix} (inspired from CutMix \cite{yun2019cutmix}). In our experiment, we achieve 74.0\% top-1 accuracy with 1.36GB token storage, where the full image storage requires 140GB to achieve 81.8\% \cite{deit}.

\ours has several advantages over previous storage-efficient methods. First, as we use a frozen pre-trained tokenizer that only requires forward operations to extract tokens from images, we do not need an additional optimization for compressing a dataset, such as importance score-based sampling \cite{cscores}, image synthesis methods \cite{data_condensation, data_condensation_augmentation}, or neural compression \cite{balle2016end, balle2018variational}. Hence, \ours is easily applicable to newly upcoming data streams directly. Second, unlike previous works that use pre-trained feature extractors (\eg, HOG \cite{dalal2005hog} or Faster-RCNN \cite{ren2015fasterrcnn, anderson2018bottomup}), \ours can use the same architecture as pixel-based approaches with only minimal modifications on the stem layer, as well as the carefully tuned optimization settings, such as DeiT \cite{deit}. It becomes a huge advantage when using \ours as an efficient pre-training method; we can achieve 82.6\% top-1 accuracy by fine-tuning the token pre-trained model with images. Moreover, applying an input augmentation for feature extractor-based approaches is not straightforward, limiting their generalizability. Finally, \ours shows a significant compression ratio, with a 1\% compression ratio for ImageNet.

We show the effectiveness of \ours on three image classification scenarios: (1) storage-efficient ImageNet-1k benchmark (2) storage-efficient large-scale pre-training, and (3) continual learning. The overview of storage-efficient results is shown in \cref{fig:teaser}: \ours outperforms comparison methods with a significant gap with the same storage size, 74.0\% accuracy on ImageNet under 1\% of the original storage, where comparison methods need 40\% (uniform sampling, C-score sampling \cite{cscores}), 6\% (adjusting image resolution), and 8\% (adjusting JPEG quality) of the original storage to achieve the similar performance. We also demonstrate that \ours can be applied to large-scale pre-training for an image-based approach; we pre-train a ViT-B/16 model on the tokenized ImageNet-21k (occupying only 14.1GB) and fine-tune the ViT model on the full-pixel ImageNet-1k. By using slightly more storage (156GB vs. 140GB), our storage-efficient pre-training strategy shows 82.8\% top-1 accuracy, whereas the full-pixel ImageNet-1k training shows 81.8\%. Finally, we observe that our token-based approach significantly outperforms the image-based counterpart in the continual learning scenario \cite{rolnick2019experience_replay_er} by storing more data samples in the same size of the memory compared to full-pixel images.

\paragraph{Contributions.} (1) We compress an image to 1024 discrete tokens using a pre-trained visual tokenizer. By applying a simple lossless compression for the tokens, we achieve only 0.97\% storage size compared to images stored in pixels. (2) We propose Stem-Adapter module and augmentation methods for tokens such as Token-RRC, Token-CutMix, Emb-Noise, and Token-EDA in order to enable ViT training with minimal change to the protocol and hyperparameters of existing ViT training. (3) Our storage-efficient training pipeline named \oursfull shows great improvements on the low-storage regime. With only 1\% storage size, \ours achieves 74.0\% top-1 ImageNet 1k validation accuracy. (4) We additionally show that \ours can be applied to a storage-efficient pre-training strategy, and continual learning tasks.

\section{Related Works}

\paragraph{Importance sampling for efficient training.}
Sampling-based methods \cite{coresets_for_efficient,2021mansheej_inportant_examples_diet,coleman2019selection,cscores} aims to idendity a compact, yet representative subset of the training dataset that satisfies the original objectives for efficient model training. This is usually achieved through exploring the early training stage \cite{2021mansheej_inportant_examples_diet}, constructing a proxy model \cite{coleman2019selection}, or utilizing consistency score (C-score) \cite{cscores}. However, the empirical performance gap between sampling-based methods and the baseline approach of random selection is insignificant, particularly in large-scale datasets like ImageNet-1k (See \cref{fig:teaser}). We believe that preserving the diversity of data points in a dataset is crucial, and therefore we endeavor to maintain the number of data points instead of pruning them.

\paragraph{Dataset distillation.}
Dataset distillation \cite{wang2018dataset} aims to generate a compact dataset by transferring the knowledge of the training dataset into a smaller dataset. Recent works \cite{data_condensation, data_condensation_augmentation, lee2022dcc, sangermano2022sample, rosasco2022distilled} have shown that the synthesized data can be effective in efficient model training, especially in scenarios such as continual learning \cite{rolnick2019experience_replay_er}. However, due to their high complexity, they have not yet demonstrated successful cases in large-scale datasets such as ImageNet-1k. We recommend the survey paper \cite{lei2023dataset_distillation_survey} for curious readers.

\paragraph{Neural compression.}
Image compression algorithms have improved with the use of neural network training to minimize quality loss on lossy compression. 
The representative learned image compression methods are based on VAE \cite{balle2016end, balle2018variational}. The compressor encodes an image to discrete latent codes and the codes can be decoded into the image with small losses. Recent studies \cite{cheng2020learned,kim2022joint} have utilized the self-attention mechanism \cite{vaswani2017attention} with heavy CNN architectures to demonstrate superior compression power compared to conventional methods such as JPEG. However, the learned image compression targets high-quality images with complex and detailed contexts, which are distant from ImageNet samples. Thus, it is challenging to apply these methods to compress ImageNet for ViT training.

\paragraph{Learning with frozen pre-extracted features.}
Using extracted visual features for a model has been widely used in the computer vision field. It shows reasonable performances with a low computational cost compared to pixel-based visual encoders. For example, the Youtube-8M \cite{abu2016youtube} dataset consists of frame features extracted from Inception \cite{szegedy2015going} instead of raw pixels, allowing efficient video model training \cite{mao2018hierarchical,bhardwaj2019efficient} with frozen frame features. The pre-extracted features have also been widely used for tasks that need higher knowledge than pixel-level understandings. For example, frozen CNN features \cite{lu2016hierarchical} or bottom-up and top-down (BUTD) \cite{teney2018tips,anderson2018bottomup} features \cite{kim2018bilinear} have been a popular choice for vision-and-language models that aim to understand complex fused knowledge between two modalities, \eg, visual question answering \cite{antol2015vqa,goyal2017making}. These approaches show slightly worse performances than the end-to-end training from raw inputs without pre-extracted features \cite{kim2021vilt, clip}, but show high training efficiency in terms of computations.

However, these methods need feature-specific modules to handle frozen features and specialized optimization techniques rather than standard optimization methods of pixel-based methods. Furthermore, some fundamental augmentations, such as random resized crop (RRC), are not applicable to the frozen features, resulting in inferior generalizability. \ours has major advantages over these methods where it is the almost same training method for ViT (\eg, DeiT \cite{deit}), and yet it can significantly reduce the storage space.





\section{Token-based Storage-Efficient Training}

In this section, we propose \oursfull. \ours aims to learn a high-performing vision classifier at scale (\eg, ImageNet-1k \cite{russakovsky2015imagenet}) with a small storage size (\eg, under 1\% of the original size), a minimal change on the training strategy (\eg, highly optimized training strategy \cite{deit}), and the minimum sacrifice of accuracies. \ours consists of two parts (1) preparing the compressed token dataset and (2) training a model using the tokens.

\begin{figure}[t]
    \centering
    \includegraphics[width=0.9\linewidth]{figures/tokenization.pdf}
    \caption{\small \textbf{Tokenization.} The input image is resized to 256 $\times$ 256 and then divided into non-overlapping $n^2$ patches. The patches are fed into the ViT-VQGAN encoder, which produces a sequence of $d_c$ dimensional vectors from the patches. 
    Finally, the tokens are generated by mapping each vector to the nearest code in a pre-trained codebook. We used 32 for both $n$ and $d_c$ in this paper.}
    \label{fig:tokenization}
\end{figure}



\begin{figure*}
    \centering
    \includegraphics[width=\linewidth]{figures/augmentation.pdf}
    \caption{\small \textbf{The data processing pipeline for token data.} For each image, a $n \times n$-shaped token is loaded from the storage. We apply Token-EDA to augment the token and convert it into a one-hot form. Then, we randomly resize crop the one-hot token to $m \times m$ and process it using the pre-trained ViT-VQGAN codebook to transform it into a $d_c \times m \times m$ tensor. We further apply CutMix and Embedding-noise to this tensor and the augmented tensor is then fed into the Stem-Adapter module, which transforms it into a shape of $d_V \times k \times k$, making it suitable for use with ViT models. Our experimental values for the parameters are $n=32$, $m=28$, $k=14$, $d_c=32$, and $d_V=768$.}
    \label{fig:augmentation}
\end{figure*}


\subsection{Preparing the token dataset}
We extract tokens using the ImageNet-trained ViT-VQGAN tokenizer \cite{vitvqgan} because it shows the best reconstruction quality among the ImageNet-1k only trained tokenizers (See Appendix). In \cref{fig:teaser} and Appendix, our approach performs better if a stronger tokenizer trained with an extra dataset, \eg, the OpenImages-trained VQGAN tokenizer \cite{vqgan}, is used. In the main paper, however, we use the ViT-VQGAN tokenizer for a fair comparison with other storage-efficient methods in terms of the training dataset.



\cref{fig:tokenization} shows the overview of the dataset preparation pipeline.
We first resize the entire ImageNet dataset to 256 $\times$ 256. Then, each resized image is divided into non-overlapping 8 $\times$ 8 image patches. Finally, we encode each patch into a 32-dimensional vector and assign a code index by finding the nearest codeword from the pre-trained codebook. Here, we only use 391 codewords from the 8192 original codewords because we found that only 391 codewords are used for the ImageNet training dataset. As a result, each image is converted to 32 $\times$ 32 tokens where each token belongs to [0, $\ldots$, 390]. We also store the codebook of ViT-VQGAN (a 32 $\times$ 391 vector) to re-use the knowledge of the codebook for better performance.

\begin{table}[t]
\small
\centering
\begin{tabular}{llrr}
\toprule
Format & Encoding & \begin{tabular}[c]{@{}c@{}}Storage\\ size \end{tabular} & \begin{tabular}[c]{@{}c@{}} Avg. size\\ per image\end{tabular} \\
\midrule
Pixels & \texttt{uint8} (uncompressed) & 1471.2 GB & 1.14 MB  \\
Pixels & JPEG (baseline) & 140.0 GB & 109.3 kB  \\
Tokens & \texttt{uint16} (uncompressed) & 2.50 GB & 2.0 kB\\
Tokens & Ours (8 bits encoding) & 1.54 GB & 1.26 kB\\
Tokens & Ours + Huffman coding  & \textbf{1.36 GB} & \textbf{1.11 kB}\\
\midrule
Tokens & \textit{Theoretical optimum} & 1.32 GB & 1.08 kB\\
\bottomrule
\end{tabular}
\vspace{.5em}
\caption{\small {\bf Storage size of the ImageNet-1k training dataset for different formats and encodings.} \texttt{uint8} and \texttt{uint16} denote uncompressed version of each data format. \textit{Theoretical optimum} is estimated by assuming the token population is uniform.}
\label{tab:format_storage}
\end{table}

In theory, as our token indices belong to [0-390], the optimal bit length to store the tokens is $\log_2 391 = 8.61$\footnote{Following the empirical population of the tokens, the ``empirical'' optimal bit length is 8.54 by computing $H(p) = - \sum p_i \log p_i$. However, in the rest of the paper, we assume the population is uniform for simplicity.} by the source coding theorem \cite{cover1999information}. Therefore, the optimal storage size of an image will be 1.08 kB\footnote{We have 1.08 kB = bits per token (8.61) $\times$ token length (1024) / bits per Byte (8). If we follow the actual distribution, it becomes 1.07 kB.}. However, in practice, we cannot store tokens in 8.61 bits because the commonly used data types use Byte for the minimal unit, \eg, 1 Byte (\texttt{uint8}) or 2 Bytes (\texttt{uint16}). To compress the required bits per token to less than 2 Bytes, we propose a simple yet efficient encoding for the tokens. First, we assign each token index following the token popularity, \ie, the most frequent token is assigned to index 0, and the least frequent token is assigned to index 390. Then, we break up token indices larger than 255 into two elements as follows:
\begin{equation}
\label{eq:encoding}
    i = \begin{cases}
        [i] \quad\quad~\text{if}~ i < 255\\
        [255, i] ~\text{if}~ i \geq 255\\
    \end{cases}
\end{equation}
We store multiple tokens in a file to reduce the required storage as small as possible. However, because our encoding process makes the length of each token variable, the naive decoding process for our encoding will need $O(n)$ complexity where $n$ is the number of encoded tokens by \cref{eq:encoding}. We solve the problem by simply storing the start indices of each image. The index storage only requires 9.8 MB for the entire ImageNet training dataset, but it makes the decoding process becomes $O(1)$ and parallelizable. 
Pseudo-codes for the proposed encoding-decoding are in \cref{subsec:pseudocode_supp}.

Our simple encoding strategy reduces 40\% of the overall storage size compared to the naive \texttt{uint16} data type as shown in \cref{tab:format_storage}. Here, as the original baseline storage also employs a compression algorithm, such as JPEG (See the first and the second row of \cref{tab:format_storage}), we also apply a simple compression algorithm, Huffman coding \cite{huffman1952method}. After applying Huffman coding to our token storage, we achieve nearly optimal storage size per image (1.11 kB vs. 1.08 kB). We empirically observe that the entire decoding process, including Huffman decoding, is almost neglectable: while the full-pixel processing requires 0.84s per 100 images, our approach only needs 0.07s.
As a result, full-pixel and SeiT take 5m 40s and 5m 12s for 1 epoch training, respectively.
In the remaining part of this paper, we use the compressed version of our token dataset if there is no specification.


\subsection{Training classifiers with tokens}

Training a classifier with tokenized images is not trivial. For example, an input token has 32 $\times$ 32 dimensions, but a conventional image input has 3 $\times$ 224 $\times$ 224. Furthermore, strong image-level augmentations (\eg, RandAugment \cite{cubuk2020randaugment}, Gaussian blur \cite{touvron2022deit}) have become crucial in large-scale vision classifiers, however, these operations cannot be directly applied to the tokens. One possible direction is to decode tokens to pixel-level images during every forward computation. However, this would impose an additional computational load on the network. 
Instead, we propose simple yet effective token-level augmentations and a simple Stem-Adapter module to train a vision classifier directly on the tokens with minimal modification but small sacrifices.

\subsubsection{Token Augmentations}

\noindent \textbf{Token-EDA.}
We utilize the EDA \cite{wei2019eda}, designed for language models, to augment our token data. EDA originally involves four methods: Synonym Replacement (SR), Random Insertion (RI), Random Swap (RS), and Random Deletion (RD). However, we only adopt SR and RS because the others do not maintain the number of tokens, which is not compatible with the ViT training strategy. For SR, we define synonyms of a token as the five tokens that have the closest Euclidean distance in the ViT-VQGAN codebook space. Then, each token is randomly replaced with one of its synonyms with a certain probability $p_{s}$ during training. For RS, we randomly select two same-sized squares from a 32 $\times$ 32 token and swapped the tokens inside them with each other, with a probability $p_{r}$. We use 0.25 for $p_{s}$ and $p_{r}$ for \ours.

\noindent \textbf{Token-RRC and Token-CutMix.}
In addition to EDA, we apply Random Resized Crop (RRC) and CutMix \cite{yun2019cutmix} to tokens. For RRC, we adopt a standard ImageNet configuration with a scale (0.08, 1) and an aspect ratio (3/4, 4/3). To enable interpolation, we first convert the original 32 $\times$ 32 tokens to one-hot form. Then, apply the random cropping to these one-hot tokens, which are subsequently resized to 28 $\times$ 28 using bicubic interpolation. After RRC, the one-hot tokens are converted to a 32 $\times$ 28 $\times$ 28 tensor using the pre-trained codebook vectors from ViT-VQGAN, where 32 is the size of a pre-trained code vector. Note that tokens that are not in one-hot form due to interpolation are converted to mixed codebooks following their values. CutMix is then applied to these tensors, whereby a patch is randomly selected from one token and replaced with a patch from another token while maintaining the channel dimension. 

\begin{table*}
\begin{center}
\caption{Comparison with \sota\ methods on the public crowd analysis benchmarks: \jhu, ShanghaiTech, UCF, and \nwpu. 
The best results are shown in \first{red}. The second-best results are shown in \second{blue}. 
}
\vspace{\tablegap}
\resizebox{0.95\textwidth}{!}{
\begin{tabular}{l c c c c c c c c c c c c c}
\toprule
 \multirow{2}{*}{Method} & \multirow{2}{*}{Venue} &\multicolumn{2}{c}{\jhu} &\multicolumn{2}{c}{\shha} &\multicolumn{2}{c}{\shhb} &\multicolumn{2}{c}{\ucf} &\multicolumn{2}{c}{\qnrf} &\multicolumn{2}{c}{\nwpu}\\[0.2ex]
 \cmidrule(lr){3-4}\cmidrule(lr){5-6}\cmidrule(lr){7-8}\cmidrule(lr){9-10}\cmidrule(lr){11-12}\cmidrule(lr){13-14}
& & MAE$\downarrow$ & MSE$\downarrow$ & MAE$\downarrow$ & MSE$\downarrow$ & MAE$\downarrow$ & MSE$\downarrow$ & MAE$\downarrow$ & MSE$\downarrow$ & MAE$\downarrow$ & MSE$\downarrow$ & MAE$\downarrow$ & MSE$\downarrow$\\[0.2ex]
\midrule\midrule
TopoCount \cite{abousamra2021localization}	& AAAI'21	& {60.9}	& {267.4}	& {61.2}	& {104.6}	& {7.8}	& {13.7}	& {184.1}	& {258.3}	& {89.0}	& {159.0}	& {107.8}	& {438.5}	\\[0.2ex]
SUA \cite{meng2021spatial}	& ICCV'21	& {80.7}	& {290.8}	& {68.5}	& {121.9}	& {14.1}	& {20.6}	& {-}	& {-}	& {130.3}	& {226.3}	& {111.7}	& {443.2}	\\[0.2ex]
ChfL \cite{shu2022crowd}	& CVPR'22	& {57.0}	& {235.7}	& {57.5}	& {94.3}	& {6.9}	& {11.0}	& {-}	& {-}	& {80.3}	& {137.6}	& {76.8}	& {343.0}	\\[0.2ex]
MAN \cite{lin2022boosting}	& CVPR'22	& {53.4}	& \second{209.9}	& {56.8}	& {90.3}	& {-}	& {-}	& {-}	& {-}	& {77.3}	& {131.5}	& {76.5}	& {323.0}	\\[0.2ex]
GauNet \cite{cheng2022rethinking}	& CVPR'22	& {58.2}	& {245.1}	& {54.8}	& {89.1}	& {6.2}	& {9.9}	& {186.3}	& {256.5}	& {81.6}	& {153.7}	& {-}	& {-}	\\[0.2ex]
CLTR \cite{liang2022end}	& ECCV'22	& {59.5}	& {240.6}	& {56.9}	& {95.2}	& {6.5}	& {10.6}	& {-}	& {-}	& {85.8}	& {141.3}	& {74.3}	& {333.8}	\\[0.2ex]
CrwodHat \cite{wu2023boosting}	& CVPR'23	& \second{52.3}	& {211.8}	& {51.2}	& {81.9}	& \first{5.7}	& {9.4}	& {-}	& {-}	& {75.1}	& \second{126.7}	& {68.7}	& \second{296.9}	\\[0.2ex]
STEERER \cite{han2023steerer}	& ICCV'23	& {54.3}	& {238.3}	& {54.5}	& {86.9}	& {5.8}	& \second{8.5}	& {-}	& {-}	& {74.3}	& {128.3}	& \second{63.7}	& {309.8}	\\[0.2ex]
PET \cite{liu2023point}	& ICCV'23	& {58.5}	& {238.0}	& \second{49.3}	& \second{78.8}	& {6.2}	& {9.7}	& {-}	& {-}	& {79.5}	& {144.3}	& {74.4}	& {328.5}	\\[0.2ex]
\rowcolor{black!10}\method\	& 	& \first{47.3}	& \first{198.9}	& \first{47.4}	& \first{75.0}	& \first{5.7}	& \first{8.2}	& \first{160.8}	& \first{225.0}	& \first{68.9}	& \first{125.6}	& \first{57.8}	& \first{221.2}	\\[0.2ex]
\bottomrule
\end{tabular}
}
\vspace{\tablegap}
\label{table: crowd counting performance}
\end{center}
\end{table*}

\noindent \textbf{Adding channel-wise noise.}
We also developed Emb-Noise, a token augmentation method that mimics color-changing image augmentations, such as color jittering. Inspired by the fact that each channel in an image represents a specific color, we first generate noise of length 32 and add it to each channel of the converted tensor with 32 $\times$ 28 $\times$ 28 dims, and then apply full-size iid noise, i.e. noise size of 32 $\times$ 28 $\times$ 28, to the tensor. All of the noise is sampled from a normal distribution. We have empirically demonstrated that this method brings significant performance improvement despite its simplicity. Moreover, we found that adding channel-wise noise to the tokens in ViT-VQGAN, the tokenizer we used, effectively changes the colors of the decoded images, unlike adding Gaussian noise in entire dimensions. 
Example decoded images by ViT-VQGAN are presented in \cref{subsec:embnoise_vis_supp}.


\subsubsection{Stem-Adapter module}

As the tokens have a smaller size than images, they cannot be directly used for input of networks. We introduce a Stem-Adapter that converts the augmented tensor into ViT/16 to make minimal modifications on the network. Specifically, the Stem-Adapter module converts the 32 $\times$ 28 $\times$ 28 pre-processed tokens into 768 $\times$ 14 $\times$ 14, the same as the input of transformer blocks of ViT after the stem layer. 
We implement the Stem-Adapter module as a convolutional layer with a kernel size of 4 and a stride of 2. This allows the module to capture the spatial relationships of adjacent tokens and produce a tensor that can be used as input to ViT. The comparison among the different Stem-Adapter architectures is included in Section \cref{sec:ablation}.



\section{Experiments}
In this section, we conduct various experiments to demonstrate the effectiveness of token-based training. First, we compare \ours with four image compression methods on ImageNet-1k \cite{russakovsky2015imagenet}. Next, we explore the potential of \ours as a large-scale pre-training dataset by employing the ImageNet-21k dataset. We also provide ablation studies on the proposed token augmentation methods and Stem-Adapter module to determine the effectiveness of each proposed element. Lastly, we evaluate a continual learning scenario on the ImageNet-100 \cite{imagenet100} dataset to demonstrate the benefits of tokens in a limited memory environment.
The fine-grained classification results can be found in Appendix.
 

\subsection{ImageNet-1k classification}

ImageNet-1k classification performances are summarized in \cref{tab:main_results} and \cref{fig:teaser}.
Random sampling (\textcolor{YellowOrange}{yellow} in \cref{fig:teaser}) had the most significant negative impact on performance as storage capacity decreased. On the other hand, sampling by C-score~\cite{cscores} (\textcolor{Cyan}{blue}) also resulted in a noticeable performance drop, but it performed better than random sampling when storage capacity reduced to 10\% of the original. 
Although both sampling-based methods led to a considerable performance drop even with a small decrease in storage, JPEG-based compression methods (\textcolor{ForestGreen}{green}) maintained their performance until storage reached 50\% of the original.
When the quality was set above 50, the performance remained nearly the same as the original, even with 24.3\% of the original storage usage. However, when the quality was set to 1, the performance dropped dramatically to 67.8\%. Adjusting the resolution (\textcolor{Purple}{purple}) achieved better results than reducing the quality as storage became smaller while reducing the quality performed better than reducing the resolution with relatively large storage.
Despite the overall performance decline of image-based methods in low-storage environments, \ours achieved 74.0\% accuracy while using only 1\% of the original storage. Furthermore, by employing ImageNet-1k-5M \cite{yun2021re}, we were able to access more storage on tokens and achieve 78.6\% accuracy at 5\% of the ImageNet-1k storage size, where JPEG-based methods demonstrated performances lower than 75\%. These results highlight the effectiveness of \ours in improving performance in low-storage scenarios.

We also evaluate \ours model and the image-trained model on robustness datasets, such as adding Gaussian noise or Gaussian blur, ImageNet-R \cite{imagenet-r}, and adversarial attacks \cite{pgd, autoattack} in \cref{subsec:robustness_exp_supp}.
We observe that without strong pixel-level augmentations, \ours shows lower performance drops compared to the pixel-trained counterparts on corruptions and distribution shifts. \ours shows a significant gradient-based attack robustness compared to others.



\subsection{Storage-efficient token pre-training}
We extract tokens from ImageNet-21k dataset and pre-trained a ViT-B/16 model on the tokenized ImageNet-21k to determine the effectiveness of tokens as a large-scale pre-training. We then fine-tuned the pre-trained model with both tokenized ImageNet-1k and full-pixel ImageNet-1k, respectively (details are in \cref{subsec:hyperparam_exp_supp}).
Additionally, we extend our storage-efficient pre-training in three stages, namely, 21k token pre-training $\rightarrow$ 1k token pre-training $\rightarrow$ 1k image fine-tuning, following BeiT v2 \cite{peng2022beit2} (details are in \cref{subsec:storage_efficient_pt_exp_supp}).
The results are shown in \cref{tab:token_pretrain}.

The use of large-scale tokens for pre-training improved not only the performance of ImageNet-1k benchmarks using tokens but also the performance of full-pixel images. Pre-training with ImageNet-21k tokens led to a 2.5\% performance gain compared to using ImageNet-1k-5M tokens, using only 8GB more storage. Furthermore, our pre-training strategy improved full-pixel ImageNet-1k performance by 1.0\% using only 11.4\% more storage compared to the original full-pixel ImageNet-1k training.
It is only 27\% storage size compared to the sampling-based image pre-training strategy with a similar accuracy (410\% of IN-1k, showing 82.5\% accuracy) as shown in \cref{tab:pt_comparison}.







\begin{table}[t]
\small
\centering
\begin{tabular}{ccrrl}
\toprule

Pre-training & Fine-tuning  & \multicolumn{2}{c}{Storage}    & \multirow{2}{*}{Acc.} \\
IN-21k & IN-1k  & \multicolumn{1}{c}{Size} & \multicolumn{1}{c}{Ratio}    &  \\ 
\midrule
-         & Pixels & 140 GB & 100.0\% & 81.8$^\dagger$ \\
Tokens & Tokens & 16 GB & 11.1\% & 81.1   \\
Tokens & Pixels & 154 GB & 110.0\%& 82.6   \\
Tokens & Tokens $\rightarrow$ Pixels & 156 GB & 111.4\%& \textbf{82.8}   \\
\bottomrule
\end{tabular}
\vspace{.5em}
\caption{\small \textbf{Impact of storage-efficient pre-training (PT) and fine-tuning (FT).} We show the scenario of storage-efficient PT; we pre-train a model with a tokenized ImageNet-21k with more data points and fine-tune the model on the pixel or the token ImageNet-1k dataset. $^\dagger$ is from the original paper. ``Tokens $\rightarrow$ Pixels'' denotes three-staged FT, Token 21k PT, Token 1k PT and Pixels FT.
}
\label{tab:token_pretrain}
\end{table}

\begin{table}[t]
\small
\centering
\begin{tabular}{c|ccccc}
\toprule
\# PT images & $\times$1.35 & $\times$1.70 & $\times$2.05 & $\times$2.40 & $\times$3.10 \\
IN-1k FT Acc & 79.1 & 81.4 & 81.0 & 80.9 
 & 82.5 \\ \bottomrule 
\end{tabular}
\vspace{.5em}
\caption{\small {\bf Sampling-based pixel PT.} We show the IN-1k FT accuracies by different PTs by subsampling ImageNet-1k-5M \cite{yun2021re}. Pixel-based PT-FT strategy shows comparable accuracy to \ours when 410\% storage size is used (82.5 and 82.8, respectively).}
\label{tab:pt_comparison}
\end{table}

\subsection{Ablation study}
\label{sec:ablation}
We present an analysis of the proposed augmentation methods, Stem-Adapter architectures, and results on convolutional networks. \cref{tab:ablation_augmentation} reports the impact of the proposed augmentations for tokens. 
We found that employing Token-CutMix not only stabilized the overall training procedures but also resulted in the largest performance gain (8.1\%) compared to excluding it. 
The newly proposed methods for tokens, Embedding-Noise and Token-EDA, also showed performance improvements of 0.3\% and 1.4\%, respectively. Interestingly, these methods not only work effectively when used individually but also achieve higher performance when used in combination (74.0\%).

We also assessed the impact of the Stem-Adapter architecture on performance in \cref{tab:ablation:stem}. We compared two different Stem-Adapter architectures with our design choice. Note that, we used a smaller learning rate of 0.0005 for the linear Step-Adapter because of its unstable convergence using a larger learning rate and an input size of 14 $\times$ 14 to match the number of input patches with the convolutional Stem-Adapters. The results validate that our decision to use Conv $4\times4$ as Stem-Adapter for ViT models yields the highest performance among the considered candidates.

We also investigated the applicability of \ours to convolutional networks. The benchmark results on different architectures of ImageNet-1k are presented in \cref{tab:results_other_arch}. Note that token-based training only requires 1.4GB storage, which is merely 1\% of the storage required for pixel-based training. To match the size of features after the stem layer, we used a deconvolutional Stem-Adapter for ResNet \cite{resnet} models. Our findings indicate that \ours can also be used for storage-efficient training of convolutional models.

Finally, we show the impact of the tokenizer in \cref{subsec:more_tokenizers_exp_supp}. In summary, we observe that \ours works well for various tokenizers, \eg, ViT-VQGAN \cite{vitvqgan} and VQGAN \cite{vqgan} variants. We chose ViT-VQGAN considering the trade-off between the performance and the storage size, and it is solely trained on ImageNet-1k without external datasets.




\begin{table}[t]
\small
\centering
\begin{tabular}{@{}cccccc@{}}
\toprule
Token-CutMix & Token-EDA & Emb-Noise & Acc. (ViT-B)\\
\midrule
\nomark & \nomark & \nomark & 63.8 \\
\yesmark & \nomark & \nomark & 71.9 \\
\yesmark & \yesmark & \nomark & 72.2 \\
\yesmark & \nomark & \yesmark & 73.3 \\
\yesmark & \yesmark & \yesmark & \textbf{74.0} \\
\bottomrule
\end{tabular}
\vspace{.5em}
\caption{\small \textbf{Impact of the proposed augmentations.} ImageNet-1k validation accuracies for the combination of the proposed augmentations for tokens are shown.}
\label{tab:ablation_augmentation}
\end{table}

\begin{table}[t]
\small
\centering
\begin{tabular}{lccc}
\toprule
 & Linear & Conv $2\times2$ & Conv $4\times4$  \\
\midrule
Accuracy & 58.6 & 73.1 & \textbf{74.0} \\
\bottomrule
\end{tabular}
\vspace{.5em}
\caption{\small \textbf{Stem-Adapter architectures.} We compare three Stem-Adapter architectures for ViT-B/16 on ImageNet-1k. Note that stride of Convolution layers set to 2.}
\label{tab:ablation:stem}
\end{table}

\begin{table}[t]
\small
\centering
\begin{tabular}{ccc|cc}
\toprule
\multirow{2}{*}{Network} & \multicolumn{2}{c|}{Pixel-based training} & \multicolumn{2}{c}{Token-based training} \\
 & Acc. & Storage & Acc. & Storage \\
\midrule
ViT-S \cite{vit} & 79.9 & 140GB & 73.5 & 1.4GB \\
ResNet-50 \cite{resnet} & 76.1 & 140GB & 67.7 & 1.4GB \\
ResNet-18 & 69.7 & 140GB & 58.0 & 1.4GB \\
\bottomrule
\end{tabular}
\vspace{.5em}
\caption{\small \textbf{Comparisons on various architectures.} We additionally compare the performances of the pixel-training and token-training accuracies of three architectures, including ViT-S, ResNet-50, and ResNet-18, on the ImageNet-1k benchmark.}
\label{tab:results_other_arch}
\end{table}


\subsection{Continual learning}
To demonstrate the effectiveness of \ours in memory-limited settings, we compare \ours with full-pixel datasets in a continual learning scenario. Specifically, we employed the Reduced ResNet-18 architecture on the ImageNet-100 dataset~\cite{imagenet100} and evaluated the results following the Experience Relay~\cite{rolnick2019experience_replay_er}. 
We observed that when using the same memory size, \ours is significantly more memory-efficient than images, with a storage capacity of 147 times that of images. As a result, the total memory required to store the entire dataset in tokens was less than 500MB.
\cref{fig:continual_learning} illustrates the comparison results between using a token dataset and a full-pixel dataset in three different settings.

The left figure shows the performances of the token dataset and the full-pixel dataset by increasing memory size while fixing the number of tasks to ten. \ours outperforms the pixel dataset and shows a neglectable performance drop even when the memory size decreased, as it stored sufficient data even with memory sizes below 100MB.

The center figure presents the results of changing the number of tasks with a fixed memory size of 574MB ($\approx$ 1k images). In this case, both token and full-pixel datasets exhibited decreased performance as the number of tasks increased. However, the performance degradation of the token dataset was less severe than that of the full-pixel dataset.

Finally, with both memory size and the number of tasks fixed, we varied the number of times the dataset was viewed per task (the right figure). When there was only one task, the full-pixel dataset outperformed the token dataset as the epoch increased, consistent with other classification benchmark results. However, when there were ten tasks, the full-pixel dataset had lower performance than the token dataset, even with increased epochs due to insufficient stored data.

\subsection{Implementation details}
We used a pre-trained ViT-VQGAN Base-Base \cite{vitvqgan} model for extracting tokens from the images. Extracting tokens of entire ImageNet-21k dataset took 1.1 hours using 64 A100 GPUs with 2048 batch-size.
We conducted ImageNet-1k benchmark experiments using the ViT-B/16 model \cite{vit, deit} with an input size of 224 x 224. 
For token ImageNet-1k training, we replaced the patch embedding layer in ViT-B/16 model with the proposed Stem-Adapter module and added a global pooling layer before the final norm layer for tokens. 
We used a learning rate of 0.0015 with cosine scheduling and a weight decay of 0.1. The model was trained for 300 epochs with a batch size of 1024. We followed the training recipe proposed in DeiT~\cite{deit} for remaining settings except for the data augmentations.
We also followed the training recipe proposed in DeiT for the full-pixel ImageNet-1k training but made a few adjustments to handle the reduced datasets. We used a smaller learning rate of 0.0009 with a batch size of 1024 compared to the original value of 0.001, as we found that the original learning rate did not converge well on smaller datasets. Also, we increased the number of warm-up epochs and total training iterations when the number of data points decreased to ensure a fair comparison.
For large-scale token pre-training and token fine-tuning, we adopted simple augmentation strategies as suggested in DeiT-III \cite{touvron2022deit}; we excluded Token-EDA and replaced RRC with a simple random crop. 
Following the DeiT-III training recipe, we pre-trained the model with tokenized ImageNet-21k dataset for 270 epochs and then we fine-tuned the model for 100 epochs both of token and full-pixel dataset using learning rates of 0.00001 with 4096 batch-size and 0.0005 with 1024 batch-size, respectively.
We provide the more detailed hyper-parameter setting of our experiments in \cref{subsec:hyperparam_exp_supp}.





\begin{figure}
    \centering
    \includegraphics[width=\linewidth]{figures/CL.pdf}
    \caption{\small \textbf{Comparisons on the continual learning task.} We train two Experience Replay (ER) \cite{rolnick2019experience_replay_er} models on the ImageNet-100 \cite{imagenet100} dataset using the pixel dataset and the token dataset. (a) By varying the memory size while the number of tasks is fixed by 10. (b) By varying the number of tasks while fixing the memory size. (c) By increasing the epochs per task. Note that except (c), we set the epochs per task to 1 following the original setting \cite{rolnick2019experience_replay_er}.}
    \label{fig:continual_learning}
\end{figure}

\section{Conclusion}

In this paper, we propose \oursfull by storing images into tokens. In practice, we store an image into 1kB as a 32$\times$32 token sequence and propose an efficient and fast encoding and decoding strategy for the token data type. We also propose token augmentations and Stem-Adaptor to train vision transformers with minimal modifications from the highly-optimized pixel-based training. Our experiments show that compared to the other storage-efficient training methods, \ours shows significantly large gaps; with the same amount of storage size, \ours shows the best performance among the comparison methods. Our method also shows benefits in other practical scenarios, such as storage-efficient large-scale pre-training and continual learning at scale.

\section{Appendix for Proofs}

\paragraph{Proof of Theorem \ref{thm:main}.}

\begin{proof}
\label{proof:main}
Our proof has two steps. In Step 1, we will show that SimCLR is equivalent to minimizing the cross entropy loss defined in Eqn.~(\ref{eqn:cross-entropy}). 
In Step 2, we will show  that minimizing the cross-entropy loss 
is equivalent to spectral clustering on $\bfpi$. 
Combining the two steps together, we have proved our theorem. 

\textbf{Step 1: } SimCLR is equivalent to minimizing the cross entropy loss.

The cross-entropy loss takes expectation over 
$\bfW_\bfX\sim \mathbb{P}(\cdot ; \bfpi)$, 
which means $\bfW_\bfX$ has exactly one non-zero entry in each row $i$. By Lemma~\ref{lem:multinomial}, we know every row $i$ of $\bfW_\bfX$ is independent of other rows. Moreover, 
$\bfW_{\bfX,i}\sim \mathcal{M}(1, \bfpi_i/\sum_j \bfpi_{i,j})=\mathcal{M}(1, \bfpi_i)$, because $\bfpi_i$ itself is a probability distribution.
Similarly, we know $\bfW_\bfZ$ also has the row-independent property by sampling over $\mathbb{P}(\cdot;\bfK_\bfZ)$.
Therefore, by Lemma~\ref{lem:cross_split}, we know Eqn.~(\ref{eqn:cross-entropy}) is equivalent to:
\[
 -\sum_{i=1}^n \mathbb{E}_{\bfW_{\bfX,i}}[\log \mathbb{P}(\bfW_{\bfZ,i}=\bfW_{\bfX,i};\bfK_\bfZ)],
\]

This expression takes expectation over $\bfW_{\bfX,i}$ for the given row $i$. Notice that 
$\bfW_{\bfX,i}$ has exactly one non-zero entry, which equals $1$ (same for $\bfW_{\bfZ,i}$). 
As a result
we expand the above expression to be:
\begin{equation}
 -\sum_{i=1}^n \sum_{j\neq i} \Pr(\bfW_{\bfX,i,j}=1)\log \Pr(\bfW_{\bfZ,i,j}=1).
\label{eqn:detailed-expansion}    
\end{equation}


By Lemma~\ref{lem:multinomial}, $\Pr(\bfW_{\bfZ,i,j}=1)=\bfK_{\bfZ,i,j}/\|\bfK_{\bfZ,i}\|_1$ for $j\neq i$. Recall that $\bfK_\bfZ=(k(\bfZ_i-\bfZ_j))_{(i,j)\in[n]^2}$, which means 
$\bfK_{\bfZ,i,j}/\|\bfK_{\bfZ,i}\|_1=\frac{\exp(-\|\bfZ_i-\bfZ_j\|^2/{2\tau})}{\sum_{k\neq i}
\exp(-\|\bfZ_i-\bfZ_k\|^2/{2\tau})
}$ for $j\neq i$, when $k$ is the Gaussian kernel with variance $\tau$. 

Notice that $\bfZ_i=f(\bfX_i)$, so we know
\begin{equation}
-\log \Pr(\bfW_{\bfZ,i,j}=1)=
-\log \frac{\exp(-\|f(\bfX_i)-f(\bfX_j)\|^2/{2\tau})}{\sum_{k\neq i}
\exp(-\|f(\bfX_i)-f(\bfX_k)\|^2/{2\tau}),
}
\label{eqn:infonce-equivalence}    
\end{equation}


The right hand side is exactly the InfoNCE loss defined in Eqn.~(\ref{eqn:infonce}).
Inserting Eqn.~(\ref{eqn:infonce-equivalence}) into Eqn.~(\ref{eqn:detailed-expansion}), we get the SimCLR algorithm, which first samples augmentation pairs $(i,j)$ with $\Pr(\bfW_{\bfX,i,j}=1)$ for each row $i$, and then optimize the InfoNCE loss. 

\textbf{Step 2: } minimizing the cross entropy loss 
is equivalent to spectral clustering on $\bfpi$.


By Lemma~\ref{lem:convert_to_spectral}, we may further convert the loss to 
\begin{equation}
\label{eqn:main-theorem-repul-attr}
\min_{\bfZ}
-\sum_{(i,j)\in [n]^2} \mathbf{P}_{i,j}
\log k (\bfZ_i-\bfZ_j)+\log \mathbf{R}(\bfZ).
\end{equation}
Since $k$ is the Gaussian kernel, this reduces to \[
\min_\bfZ \mathrm{tr}(\bfZ^\top \mathbf{L}(\bfpi) \bfZ)
+\log \mathbf{R}(\bfZ),
\]

where we use the fact that $\mathbb{E}_{\bfW_\bfX\sim \mathbb{P}(\cdot; \bfpi)}[\mathbf{L}(\bfW_\bfX)]
=\mathbf{L}(\bfpi)
$, because the Laplacian operator is linear and $
\mathbb{E}_{\bfW_\bfX\sim \mathbb{P}(\cdot; \bfpi)}(\bfW_\bfX)=\bfpi
$.
\end{proof}

\paragraph{Proof of Theorem \ref{thm:clip}.}
\begin{proof}
Since $\bfW_\bfX\sim \mathbb{P}(\cdot;\bfpi_{\mathbf{A}, \mathbf{B}})$, we know 
$\bfW_\bfX$ has exactly one non-zero entry in each row, denoting the pair that got sampled. 
A notable difference compared to the previous proof is we now have $n_\mathcal{A}+n_\mathcal{B}$ objects in our graph. CLIP deals with this by taking a mini-batch of size $2N$, 
such that $n_\mathcal{A}=n_\mathcal{B}=N$, and adding the $2N$ InfoNCE losses together. We label the objects in $\mathcal{A}$ as $[n_\mathcal{A}]$, and the objects in $\mathcal{B}$ as $\{n_\mathcal{A}+1, \cdots, n_\mathcal{A}+n_\mathcal{B}\}$. 

Notice that $\bfpi_{\mathbf{A}, \mathbf{B}}$ is a bipartite graph, so the edges of objects in $\mathcal{A}$ will only connect to object in $\mathcal{B}$ and vice versa. We can define the similarity matrix in $\cZ$ as $\bfK_\bfZ$, 
where $\bfK_\bfZ(i, j+n_\mathcal{A})=\bfK_\bfZ(j+n_\mathcal{A},i)= k(\bfZ_i-\bfZ_j)$ for $i\in [n_\mathcal{A}], j\in [n_\mathcal{B}]$, and otherwise we set $\bfK_\bfZ(i,j)=0$. 
The rest is same as the previous proof. 
\end{proof}

\paragraph{Proof of Theorem \ref{thm:exponential}.}

\begin{proof}
\label{proof:exponential}
Since the objective function consists of a linear term combined with an entropy regularization, which is a strongly concave function, the maximization problem is a convex optimization problem. Owing to the implicit constraints provided by the entropy function, the problem is equivalent to having only the equality constraint. We then introduce the Lagrangian multiplier $\lambda$ and obtain the following relaxed problem:

$$
\widetilde{E}(\boldsymbol{\alpha})=\psi_{1}-\sum_{i=1}^n \alpha_{i} \psi_{i}+\tau \sum_{i=1}^n \alpha_{i}\log \alpha_{i}+\lambda\left(\boldsymbol{\alpha}^{\top} \mathbf{1}_n-1\right).
$$

As the relaxed problem is unconstrained, taking the derivative with respect to $\alpha_{i}$ yields

$$
\frac{\partial \widetilde{E}(\boldsymbol{\alpha})}{\partial \alpha_{i}}=-\psi_{i}+\tau\left(\log \alpha_{i}+\alpha_{i} \frac{1}{\alpha_{i}}\right)+\lambda=0.
$$

Solving the above equation implies that $\alpha_{i}$ takes the form
$
\alpha_{i}=\exp \left(\frac{1}{\tau} \psi_{i}\right) \exp \left(\frac{-\lambda}{\tau}-1\right).
$ Since $\alpha_{i}$ lies on the probability simplex, the optimal $\alpha_{i}$ is explicitly given by
$
\alpha^{*}_{i}=\frac{\exp \left(\frac{1}{\tau} \psi_{i}\right)}{\sum_{i^{\prime}=1}^n \exp \left(\frac{1}{\tau} \psi_{i^{\prime}}\right)} .
$ Substituting the optimal point into the objective function, we obtain
$$
\begin{aligned}
E\left(\boldsymbol{\alpha}^*\right)  &=\psi_1-\sum_{i=1}^n \frac{\exp \left(\frac{1}{\tau} \psi_{i}\right)}{\sum_{i^{\prime}=1}^n \exp \left(\frac{1}{\tau} \psi_{i^{\prime}}\right)} \psi_{i}+\tau \sum_{i=1}^n \frac{\exp \left(\frac{1}{\tau} \psi_{i}\right)}{\sum_{i^{\prime}=1}^n \exp \left(\frac{1}{\tau} \psi_{i^{\prime}}\right)}\log \frac{\exp \left(\frac{1}{\tau} \psi_{i}\right)}{\sum_{i^{\prime}=1}^n \exp \left(\frac{1}{\tau} \psi_{i^{\prime}}\right)} \\
& =\psi_1 - \tau \log \left(\sum_{i=1}^n \exp \left(\frac{1}{\tau} \psi_{i}\right)\right).
\end{aligned}
$$
Thus, the Lagrangian dual function is given by
\begin{equation*}
-E\left(\boldsymbol{\alpha}^*\right)= -\tau \log \frac{\exp \left(\frac{1}{\tau} \psi_{1}\right)}{\sum_{i=1}^n \exp \left(\frac{1}{\tau} \psi_{i}\right)}.\qedhere
\end{equation*}
\end{proof}



\section{More on Experiments} \label{section: experiment_details}

\paragraph{CIFAR-10 and CIFAR-100} CIFAR-10 ~\citep{krizhevsky2009learning} and CIFAR-100 ~\citep{krizhevsky2009learning} are well-known classic image classification datasets. Both CIFAR-10 and CIFAR-100 contain a total of 60k $32 \times 32$ labeled images of different classes, with 50k for training and 10k for testing. CIFAR-10 is similar to CIFAR-100, except there are 10 different classes in CIFAR-10 and 100 classes in CIFAR-100.

\paragraph{TinyImageNet} TinyImageNet ~\citep{le2015tiny} is a subset of ImageNet ~\citep{deng2009imagenet}. There are 200 different object classes in TinyImageNet, with 500 training images, 50 validation images, and 50 test images for each class. All the images in TinyImageNet are colored and labeled with a size of $64 \times 64$.

\textbf{Pseudo-code.} Algorithm \ref{alg:Training Procedure} presents the pseudo-code for our empirical training procedure.

\begin{algorithm}[!htbp]
\caption{Training Procedure}
\label{alg:Training Procedure}
\begin{algorithmic}[1]
\REQUIRE trainable encoder network $f$, batch size $N$, augmentation strategy \textit{aug}, loss function $L$ with hyperparameters \textit{args}
\FOR {sampled minibatch ${x_i}_{i=1}^N$}
\FORALL{$i \in { 1, ..., N }$}
\STATE draw two augmentations $t_i = \textit{aug}\left(x_i\right) $, $t_i' = \textit{aug}\left(x_i\right) $
\STATE $z_i = f\left(t_i\right)$, $z_i' = f\left(t_i'\right)$
\ENDFOR
\STATE compute loss $\mathcal{L} = L(N, z, z', \textit{args})$
\STATE update encoder network $f$ to minimize $\mathcal{L}$
\ENDFOR
\STATE \textbf{Return} encoder network $f$
\end{algorithmic}
\end{algorithm}

We also provide the pseudo-code for our core loss function used in the training procedure in Algorithm \ref{alg:Core loss}. The pseudo-code is almost identical to SimCLR's loss function, with the exception of an extra parameter $\gamma$.

\begin{algorithm}[!htbp]
\caption{Core loss function $\mathcal{C}$}
\label{alg:Core loss}
\begin{algorithmic}[1]
\REQUIRE batch size $N$, two encoded minibatches $z_1, z_2$, $\gamma$, temperature $\tau$
\STATE $z = \textit{concat}\left(z_1, z_2\right)$
\FOR {$i \in {1, ..., 2N }, j \in {1, ..., 2N}$ }
\STATE $s_{i,j} = \Vert z_i - z_j \Vert_2^{\gamma}$
\ENDFOR
\STATE \textbf{define} $l(i, j)$ \textbf{as} $l(i, j) = - \log \frac{exp\left(s_{i,j}/\tau \right)}{\sum_{k=1}^{2N} \mathbf{1}{[k \ne i]} exp\left(s{i, j} / \tau \right)} $
\STATE \textbf{Return} $\frac{1}{2N} \sum_{k=1}^N\left[l(i, i+N) + l(i+N, i)\right]$
\end{algorithmic}
\end{algorithm}

Utilizing the core loss function $\mathcal{C}$, we can define all kernel loss functions used in our experiments in Table \ref{table: loss definition}. For all $z_i \in z$ with even dimensions $n$, we define $z_{L_i} = z_i\left[0:n/2\right]$ and $z_{R_i} = z_i\left[n/2:n\right]$.

\begin{table}[ht]
\centering
\begin{tabular}{{@{}l|l@{}}}
Kernel  &  Loss function \\ \midrule
Laplacian & $\mathcal{C}\left(N, z, z', \gamma=1, \tau\right)$\\ \midrule
Sum       & $\lambda * \mathcal{C}\left(N, z, z', \gamma=1, \tau_1\right) + (1-\lambda) * \mathcal{C}\left(N, z, z', \gamma=2, \tau_2\right)$  \\ \midrule
Concatenation Sum&$\lambda * \mathcal{C}\left(N, z_L, z'_L, \gamma=1, \tau_1\right) + (1-\lambda) * \mathcal{C}\left(N, z_R, z'_R, \gamma=2, \tau_2\right)$\\ \midrule
$\gamma = 0.5$ & $\mathcal{C}\left(N, z, z', \gamma=0.5, \tau\right)$          \\ 

\end{tabular}

\caption{Definition of kernel loss functions in our experiments}
\label {table: loss definition}
\end{table}

\textbf{Baselines.} We reproduce the SimCLR algorithm using PyTorch Lightning~\citep{PytorchLightning}.

\textbf{Encoder details.}
The encoder $f$ consists of a backbone network and a projection network. We employ ResNet50~\citep{ResNet} as the backbone and a 2-layer MLP (connected by a batch normalization~\citep{ioffe2015batch} layer and a ReLU \cite{nair2010rectified} layer) with hidden dimensions 2048 and output dimensions 128 (or 256 in the concatenation kernel case).

\textbf{Encoder hyperparameter tuning.}
For each encoder training case, we randomly sample 500 hyperparameter groups (sample details are shown in Table \ref{table: Hyperparameter sample}) and train these samples simultaneously using Ray Tune ~\citep{RayTune}, with the ASHA scheduler~\citep{li2018massively}. Ultimately, the hyperparameter group that maximizes the online validation accuracy (integrated in PyTorch Lightning) within 5000 validation steps is chosen for the given encoder training case.

\begin{table}[ht]
\centering

\begin{tabular}{@{}l|l|l@{}}
\midrule
Hyperparameter  & Sample Range & Sample Strategy \\ \midrule
start learning rate & $\left[10^{-2}, 10\right]$ & log uniform \\ \midrule
$\lambda$       & $\left[0, 1\right]$ & uniform \\ \midrule
$\tau$, $\tau_1$, $\tau_2$ & $\left[0, 1\right]$ & log uniform \\ \midrule
\end{tabular}

\caption{Hyperparameters sample strategy}
\label {table: Hyperparameter sample}
\end{table}

\textbf{Encoder training.} 
We train each encoder using the LARS optimizer~\citep{LARSOptimizer}, LambdaLR Scheduler in PyTorch, momentum 0.9, weight decay $10^{-6}$, batch size 256, and the aforementioned hyperparameters for 400 epochs on a single A-100 GPU.

\textbf{Image transformation.} The image transformation strategy, including augmentation, is identical to the default transformation strategy provided by PyTorch Lightning.

\textbf{Linear evaluation.}
The linear head is trained using the SGD optimizer with a cosine learning rate scheduler, batch size 64, and weight decay $10^{-6}$ for 100 epochs. The learning rate starts at $0.3$ and ends at $0$.

\textbf{Moco Experiments.} We also tested our method based on MoCo~\citep{he2019moco}. The results are summarized in Table \ref{tab:results-moco}. Here we choose ResNet18~\citep{ResNet} as the backbone and set a temperature of $0.1$ as default. For our simple sum kernel, we set $\lambda=0.8$. The results show that our method outperforms the original MoCo method.

\begin{table}[thb]
\centering
\caption{MoCo Experiment Results on CIFAR-10 and CIFAR-100.}
\label{tab:results-moco}
\resizebox{\textwidth}{!}{%
\begin{tabular}{@{}c|ccc|ccc@{}}
\toprule
\multirow{3}{*}{Method} & \multicolumn{3}{c|}{CIFAR-10} & \multicolumn{3}{c}{CIFAR-100} \\ \cmidrule(lr){2-4} \cmidrule(lr){5-7} 
                        & 200 epochs & 400 epochs    & 1000 epochs   & 200 epochs & 400 epochs & 1000 epochs         \\ \midrule
MoCo (repro.)         & $76.41 \pm 0.12$    & $80.01 \pm 0.15$          & $84.45 \pm 0.08$    & $\mathbf{47.02 \pm 0.11}$ & $52.50 \pm 0.07$ & $57.62 \pm 0.15$            \\
\midrule
Laplacian Kernel        & ${78.09 \pm 0.10}$    & $\mathbf{83.85 \pm 0.09}$          & $\mathbf{88.34 \pm 0.16}$    & $46.12 \pm 0.22$   & $53.44 \pm 0.17$ & $59.10 \pm 0.14$        \\
Simple Sum Kernel & $\mathbf{78.12 \pm 0.15}$   & $83.23 \pm 0.18$ & $87.50 \pm 0.20$ & $46.65 \pm 0.06$ & $\mathbf{53.62 \pm 0.19}$ & $\mathbf{59.83 \pm 0.12}$\\
\bottomrule
\end{tabular}
}
\end{table}



\section{More Experiments on Synthetic Data}


Consider a scenario with $n$ clusters, each containing $k$ vertices. Let the probability of vertices $u$ and $v$ from the same cluster belonging to $\bfpi$ be $p$. Conversely, for vertices $u$ and $v$ from different clusters, let the probability of belonging to $\pi$ be $q$. We generate the graph $\bfpi$ randomly, based on $p$ and $q$. We experiment with values of $k=100$ and $n=6$ for ease of visualization, embedding all points in a two-dimensional space. Each vertex's initial position originates from a normal distribution. In each iteration, we sample a subgraph of $\bfpi$ uniformly, ensuring each vertex has an out-degree of $1$. We then optimize the corresponding vectors using InfoNCE loss with an SGD optimizer and iterate until convergence. Our experimental setup consists of an SGD learning rate of $1$, an InfoNCE loss temperature of $0.5$, and a batch size of $50$. We evaluate two scenarios with different $p$ and $q$ values: $p=1$, $q=0$, and $p=0.75$, $q=0.2$. The results of these experiments are visualized in Figure \ref{fig:vis-spectral-cluster}. The obtained embeddings exhibit the hallmark pattern of spectral clustering of graph $\bfpi$.

\begin{figure}[!tb]
\centering
\subfigure{
\includegraphics[width=1\textwidth]{Figures/cluster_pi.png}
\label{fig:vis-cluster}
}
\subfigure{
\includegraphics[width=1\textwidth]{Figures/noised_cluster_pi.png}
\label{fig:vis-noised-cluster}
}
\caption{Visualizations of the optimization process using InfoNCE Loss on the vectors corresponding to $\bfpi$. Points of identical color belong to the same cluster within $\bfpi$. To showcase the internal structure of $\bfpi$, we randomly select 10 vertices from each cluster to display the edge distribution of $\bfpi$.}
\label{fig:vis-spectral-cluster}
\end{figure}


\clearpage

{
\bibliographystyle{ieee_fullname}
\bibliography{reference}
}

\end{document}