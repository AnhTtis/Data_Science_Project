\section{{Conclusion and Open Questions}}
To the best of our knowledge, this work provides the first instance-dependent sample complexity results for zero-sum normal form games. We have completely characterized the instance dependent sample complexity of finding Nash Equilibrium in $2\times 2$ matrix games. In addition, we have shed some light on the case of $n\times 2$ matrix games.
These results shed light on the properties of a game that make its dynamics converge quickly or slowly. 
The implications of this line of results could be new algorithms designed to take advantage of easy games, where previous minimax optimal algorithms may not.
This more nuanced understanding of instance-dependent sample complexity may also influence mechanism design since our results describe specifically how one could speed up convergence of players to a Nash equilibrium. 

However, our work leaves many questions unresolved as well as revealing new ones. 
The most obvious direction--extending our results to general $(n \times m) \in \mathbb{N} \times \mathbb{N}$--is also one of the most challenging. 
First, unlike our $n\times 2$ case in which the size of the support is trivially at most $k=2$, it is unclear how to identify the true size $k$ of the support of the Nash equilibrium in general, and then how to identify the $k \times k$ sub-matrix within the game matrix.
Second, there does not exist a closed-form expression for the Nash equilibrium of general $(n\times m) \in \mathbb{N} \times \mathbb{N}$ matrix games.
We exploit the existence of the closed-form solution in our $2\times 2$ analysis in many ways, including deriving alternative instances for lower bounds, and also understanding the right notions of gap by considering a perturbation of the optimal solution.  Due to Cramer's rule there exists a closed-form expression for the Nash equilibrium of general $(n\times n) $ matrix game, however, we would have to analyze how determinants of a matrix behave under minor perturbations in order to establish meaningful upper bounds. 

Besides larger game matrices, there are other very natural questions to pursue. 
Given the instance-dependent quantities we introduced in this work, how do these generalize to general-sum games and can our lower bound strategies be extended? 
What is the sample complexity of identifying other kinds of equilibria, such as an $\varepsilon$ (coarse) correlated equilibrium?
Finally, can we derive instance-dependent regret bounds for computationally efficient strategies? 


% This work reveals a number of open problems: \kevin{TODO}\ljr{ @arnab: I dont think we should leave this as a list but rather make it prose}\arn{I agree. Just wrote it separately so that we can write a short discussion quickly later on.}
% \begin{itemize}
%     \item Instance dependent bounds for general $(m\times n) \in \mathbb{N} \times \mathbb{N}$ matrix games is an important open question. This require new techniques, the primary challenge coming from the fact that we do not know the size of the support of the Nash equilibrium. Even knowing the support size (say $k$) does not resolve all the challenges as we do not exactly which $k\times k$ sub-matrix is optimal among all the possible $k\times k$ matrices. Defining a gap parameter like we did for the $n \times 2$ case is a challenge as we need to determine an appropriate re-scaling factor to obtain meaningful concentration bounds for any empirical estimate of the gap parameter. Moreover, there does not exist a closed-form expression for the Nash equilibrium of general $(m\times n) \in \mathbb{N} \times \mathbb{N}$ matrix games and this makes it difficult to decide the alternatives to our input matrix in order to derive a meaningful lower bound like we did in the case of $2\times 2$ matrix games. Due to Cramer's rule there exists a closed-form expression for the Nash equilibrium of general $(n\times n) $ matrix game, however, we would be required to build tools to analyze how determinants of a matrix behave under minor perturbations in order to establish meaningful upper bounds. 
%     \item No regret and no swap regret dynamics exist to find $\varepsilon$-coarse correlated equilibrium and $\varepsilon$-correlated equilibrium, respectively. It would an interesting open problem to extend the results to the stochastic setting and achieve instant dependent guarantees.
%     \item We observe that in some instances, we are able to do much better than the trivial upper bound of $\frac{1}{\varepsilon^2}$. It is an important open problem to characterize the class of general sum games that do better than the trivial upper bound.
%     \item In the stochastic setting, another important problem is to achieve instance dependent bounds for minimizing regret relative to the value of the matrix game.
% \end{itemize}


