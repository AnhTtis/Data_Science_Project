
%% bare_jrnl.tex
%% V1.4b
%% 2015/08/26
%% by Michael Shell
%% see http://www.michaelshell.org/
%% for current contact information.
%%
%% This is a skeleton file demonstrating the use of IEEEtran.cls
%% (requires IEEEtran.cls version 1.8b or later) with an IEEE
%% journal paper.
%%
%% Support sites:
%% http://www.michaelshell.org/tex/ieeetran/
%% http://www.ctan.org/pkg/ieeetran
%% and
%% http://www.ieee.org/

%%*************************************************************************
%% Legal Notice:
%% This code is offered as-is without any warranty either expressed or
%% implied; without even the implied warranty of MERCHANTABILITY or
%% FITNESS FOR A PARTICULAR PURPOSE! 
%% User assumes all risk.
%% In no event shall the IEEE or any contributor to this code be liable for
%% any damages or losses, including, but not limited to, incidental,
%% consequential, or any other damages, resulting from the use or misuse
%% of any information contained here.
%%
%% All comments are the opinions of their respective authors and are not
%% necessarily endorsed by the IEEE.
%%
%% This work is distributed under the LaTeX Project Public License (LPPL)
%% ( http://www.latex-project.org/ ) version 1.3, and may be freely used,
%% distributed and modified. A copy of the LPPL, version 1.3, is included
%% in the base LaTeX documentation of all distributions of LaTeX released
%% 2003/12/01 or later.
%% Retain all contribution notices and credits.
%% ** Modified files should be clearly indicated as such, including  **
%% ** renaming them and changing author support contact information. **
%%*************************************************************************


% *** Authors should verify (and, if needed, correct) their LaTeX system  ***
% *** with the testflow diagnostic prior to trusting their LaTeX platform ***
% *** with production work. The IEEE's font choices and paper sizes can   ***
% *** trigger bugs that do not appear when using other class files.       ***                          ***
% The testflow support page is at:
% http://www.michaelshell.org/tex/testflow/



\documentclass[journal]{IEEEtran}
%
% If IEEEtran.cls has not been installed into the LaTeX system files,
% manually specify the path to it like:
% \documentclass[journal]{../sty/IEEEtran}





% Some very useful LaTeX packages include:
% (uncomment the ones you want to load)


% *** MISC UTILITY PACKAGES ***
%
%\usepackage{ifpdf}
% Heiko Oberdiek's ifpdf.sty is very useful if you need conditional
% compilation based on whether the output is pdf or dvi.
% usage:
% \ifpdf
%   % pdf code
% \else
%   % dvi code
% \fi
% The latest version of ifpdf.sty can be obtained from:
% http://www.ctan.org/pkg/ifpdf
% Also, note that IEEEtran.cls V1.7 and later provides a builtin
% \ifCLASSINFOpdf conditional that works the same way.
% When switching from latex to pdflatex and vice-versa, the compiler may
% have to be run twice to clear warning/error messages.






% *** CITATION PACKAGES ***
%
%\usepackage{cite}
% cite.sty was written by Donald Arseneau
% V1.6 and later of IEEEtran pre-defines the format of the cite.sty package
% \cite{} output to follow that of the IEEE. Loading the cite package will
% result in citation numbers being automatically sorted and properly
% "compressed/ranged". e.g., [1], [9], [2], [7], [5], [6] without using
% cite.sty will become [1], [2], [5]--[7], [9] using cite.sty. cite.sty's
% \cite will automatically add leading space, if needed. Use cite.sty's
% noadjust option (cite.sty V3.8 and later) if you want to turn this off
% such as if a citation ever needs to be enclosed in parenthesis.
% cite.sty is already installed on most LaTeX systems. Be sure and use
% version 5.0 (2009-03-20) and later if using hyperref.sty.
% The latest version can be obtained at:
% http://www.ctan.org/pkg/cite
% The documentation is contained in the cite.sty file itself.






% *** GRAPHICS RELATED PACKAGES ***
%
\ifCLASSINFOpdf
  \usepackage[pdftex]{graphicx}
  \usepackage[nocompress]{cite}
  \usepackage{graphicx}
  \usepackage{amsmath}
  \usepackage{amssymb}
  \usepackage{booktabs}
  \usepackage{algorithm}
  \usepackage{algorithmic}
  \usepackage{multirow}
  \usepackage[dvipsnames,table,xcdraw]{xcolor}
% \usepackage{colortbl}
% \usepackage{color}
  \usepackage{tikz}
  \usepackage{booktabs}
  \usepackage{threeparttable}
  \usepackage{graphicx}                                                           
  \usepackage{float} 
  \usepackage{times}
  \usepackage{epsfig}
  \usepackage{subfigure}
  % declare the path(s) where your graphic files are
  % \graphicspath{{../pdf/}{../jpeg/}}
  % and their extensions so you won't have to specify these with
  % every instance of \includegraphics
  % \DeclareGraphicsExtensions{.pdf,.jpeg,.png}
\else
  % or other class option (dvipsone, dvipdf, if not using dvips). graphicx
  % will default to the driver specified in the system graphics.cfg if no
  % driver is specified.
  % \usepackage[dvips]{graphicx}
  % declare the path(s) where your graphic files are
  % \graphicspath{{../eps/}}
  % and their extensions so you won't have to specify these with
  % every instance of \includegraphics
  % \DeclareGraphicsExtensions{.eps}
\fi
% graphicx was written by David Carlisle and Sebastian Rahtz. It is
% required if you want graphics, photos, etc. graphicx.sty is already
% installed on most LaTeX systems. The latest version and documentation
% can be obtained at: 
% http://www.ctan.org/pkg/graphicx
% Another good source of documentation is "Using Imported Graphics in
% LaTeX2e" by Keith Reckdahl which can be found at:
% http://www.ctan.org/pkg/epslatex
%
% latex, and pdflatex in dvi mode, support graphics in encapsulated
% postscript (.eps) format. pdflatex in pdf mode supports graphics
% in .pdf, .jpeg, .png and .mps (metapost) formats. Users should ensure
% that all non-photo figures use a vector format (.eps, .pdf, .mps) and
% not a bitmapped formats (.jpeg, .png). The IEEE frowns on bitmapped formats
% which can result in "jaggedy"/blurry rendering of lines and letters as
% well as large increases in file sizes.
%
% You can find documentation about the pdfTeX application at:
% http://www.tug.org/applications/pdftex





% *** MATH PACKAGES ***
%
%\usepackage{amsmath}
% A popular package from the American Mathematical Society that provides
% many useful and powerful commands for dealing with mathematics.
%
% Note that the amsmath package sets \interdisplaylinepenalty to 10000
% thus preventing page breaks from occurring within multiline equations. Use:
%\interdisplaylinepenalty=2500
% after loading amsmath to restore such page breaks as IEEEtran.cls normally
% does. amsmath.sty is already installed on most LaTeX systems. The latest
% version and documentation can be obtained at:
% http://www.ctan.org/pkg/amsmath





% *** SPECIALIZED LIST PACKAGES ***
%
%\usepackage{algorithmic}
% algorithmic.sty was written by Peter Williams and Rogerio Brito.
% This package provides an algorithmic environment fo describing algorithms.
% You can use the algorithmic environment in-text or within a figure
% environment to provide for a floating algorithm. Do NOT use the algorithm
% floating environment provided by algorithm.sty (by the same authors) or
% algorithm2e.sty (by Christophe Fiorio) as the IEEE does not use dedicated
% algorithm float types and packages that provide these will not provide
% correct IEEE style captions. The latest version and documentation of
% algorithmic.sty can be obtained at:
% http://www.ctan.org/pkg/algorithms
% Also of interest may be the (relatively newer and more customizable)
% algorithmicx.sty package by Szasz Janos:
% http://www.ctan.org/pkg/algorithmicx




% *** ALIGNMENT PACKAGES ***
%
%\usepackage{array}
% Frank Mittelbach's and David Carlisle's array.sty patches and improves
% the standard LaTeX2e array and tabular environments to provide better
% appearance and additional user controls. As the default LaTeX2e table
% generation code is lacking to the point of almost being broken with
% respect to the quality of the end results, all users are strongly
% advised to use an enhanced (at the very least that provided by array.sty)
% set of table tools. array.sty is already installed on most systems. The
% latest version and documentation can be obtained at:
% http://www.ctan.org/pkg/array


% IEEEtran contains the IEEEeqnarray family of commands that can be used to
% generate multiline equations as well as matrices, tables, etc., of high
% quality.




% *** SUBFIGURE PACKAGES ***
%\ifCLASSOPTIONcompsoc
%  \usepackage[caption=false,font=normalsize,labelfont=sf,textfont=sf]{subfig}
%\else
%  \usepackage[caption=false,font=footnotesize]{subfig}
%\fi
% subfig.sty, written by Steven Douglas Cochran, is the modern replacement
% for subfigure.sty, the latter of which is no longer maintained and is
% incompatible with some LaTeX packages including fixltx2e. However,
% subfig.sty requires and automatically loads Axel Sommerfeldt's caption.sty
% which will override IEEEtran.cls' handling of captions and this will result
% in non-IEEE style figure/table captions. To prevent this problem, be sure
% and invoke subfig.sty's "caption=false" package option (available since
% subfig.sty version 1.3, 2005/06/28) as this is will preserve IEEEtran.cls
% handling of captions.
% Note that the Computer Society format requires a larger sans serif font
% than the serif footnote size font used in traditional IEEE formatting
% and thus the need to invoke different subfig.sty package options depending
% on whether compsoc mode has been enabled.
%
% The latest version and documentation of subfig.sty can be obtained at:
% http://www.ctan.org/pkg/subfig




% *** FLOAT PACKAGES ***
%
%\usepackage{fixltx2e}
% fixltx2e, the successor to the earlier fix2col.sty, was written by
% Frank Mittelbach and David Carlisle. This package corrects a few problems
% in the LaTeX2e kernel, the most notable of which is that in current
% LaTeX2e releases, the ordering of single and double column floats is not
% guaranteed to be preserved. Thus, an unpatched LaTeX2e can allow a
% single column figure to be placed prior to an earlier double column
% figure.
% Be aware that LaTeX2e kernels dated 2015 and later have fixltx2e.sty's
% corrections already built into the system in which case a warning will
% be issued if an attempt is made to load fixltx2e.sty as it is no longer
% needed.
% The latest version and documentation can be found at:
% http://www.ctan.org/pkg/fixltx2e


%\usepackage{stfloats}
% stfloats.sty was written by Sigitas Tolusis. This package gives LaTeX2e
% the ability to do double column floats at the bottom of the page as well
% as the top. (e.g., "\begin{figure*}[!b]" is not normally possible in
% LaTeX2e). It also provides a command:
%\fnbelowfloat
% to enable the placement of footnotes below bottom floats (the standard
% LaTeX2e kernel puts them above bottom floats). This is an invasive package
% which rewrites many portions of the LaTeX2e float routines. It may not work
% with other packages that modify the LaTeX2e float routines. The latest
% version and documentation can be obtained at:
% http://www.ctan.org/pkg/stfloats
% Do not use the stfloats baselinefloat ability as the IEEE does not allow
% \baselineskip to stretch. Authors submitting work to the IEEE should note
% that the IEEE rarely uses double column equations and that authors should try
% to avoid such use. Do not be tempted to use the cuted.sty or midfloat.sty
% packages (also by Sigitas Tolusis) as the IEEE does not format its papers in
% such ways.
% Do not attempt to use stfloats with fixltx2e as they are incompatible.
% Instead, use Morten Hogholm'a dblfloatfix which combines the features
% of both fixltx2e and stfloats:
%
% \usepackage{dblfloatfix}
% The latest version can be found at:
% http://www.ctan.org/pkg/dblfloatfix




%\ifCLASSOPTIONcaptionsoff
%  \usepackage[nomarkers]{endfloat}
% \let\MYoriglatexcaption\caption
% \renewcommand{\caption}[2][\relax]{\MYoriglatexcaption[#2]{#2}}
%\fi
% endfloat.sty was written by James Darrell McCauley, Jeff Goldberg and 
% Axel Sommerfeldt. This package may be useful when used in conjunction with 
% IEEEtran.cls'  captionsoff option. Some IEEE journals/societies require that
% submissions have lists of figures/tables at the end of the paper and that
% figures/tables without any captions are placed on a page by themselves at
% the end of the document. If needed, the draftcls IEEEtran class option or
% \CLASSINPUTbaselinestretch interface can be used to increase the line
% spacing as well. Be sure and use the nomarkers option of endfloat to
% prevent endfloat from "marking" where the figures would have been placed
% in the text. The two hack lines of code above are a slight modification of
% that suggested by in the endfloat docs (section 8.4.1) to ensure that
% the full captions always appear in the list of figures/tables - even if
% the user used the short optional argument of \caption[]{}.
% IEEE papers do not typically make use of \caption[]'s optional argument,
% so this should not be an issue. A similar trick can be used to disable
% captions of packages such as subfig.sty that lack options to turn off
% the subcaptions:
% For subfig.sty:
% \let\MYorigsubfloat\subfloat
% \renewcommand{\subfloat}[2][\relax]{\MYorigsubfloat[]{#2}}
% However, the above trick will not work if both optional arguments of
% the \subfloat command are used. Furthermore, there needs to be a
% description of each subfigure *somewhere* and endfloat does not add
% subfigure captions to its list of figures. Thus, the best approach is to
% avoid the use of subfigure captions (many IEEE journals avoid them anyway)
% and instead reference/explain all the subfigures within the main caption.
% The latest version of endfloat.sty and its documentation can obtained at:
% http://www.ctan.org/pkg/endfloat
%
% The IEEEtran \ifCLASSOPTIONcaptionsoff conditional can also be used
% later in the document, say, to conditionally put the References on a 
% page by themselves.




% *** PDF, URL AND HYPERLINK PACKAGES ***
%
%\usepackage{url}
% url.sty was written by Donald Arseneau. It provides better support for
% handling and breaking URLs. url.sty is already installed on most LaTeX
% systems. The latest version and documentation can be obtained at:
% http://www.ctan.org/pkg/url
% Basically, \url{my_url_here}.




% *** Do not adjust lengths that control margins, column widths, etc. ***
% *** Do not use packages that alter fonts (such as pslatex).         ***
% There should be no need to do such things with IEEEtran.cls V1.6 and later.
% (Unless specifically asked to do so by the journal or conference you plan
% to submit to, of course. )


% correct bad hyphenation here
\hyphenation{op-tical net-works semi-conduc-tor}


\begin{document}
\setcounter{table}{7}  % 将表格编号设置为从5开始
\setcounter{figure}{4}  % 将图片编号设置为从5开始
%
% paper title
% Titles are generally capitalized except for words such as a, an, and, as,
% at, but, by, for, in, nor, of, on, or, the, to and up, which are usually
% not capitalized unless they are the first or last word of the title.
% Linebreaks \\ can be used within to get better formatting as desired.
% Do not put math or special symbols in the title.
\title{Supplementary Materials of Unsupervised \\ Gait Recognition with Selective Fusion}

\author{Xuqian Ren,
        Shaopeng Yang,
        Saihui Hou,
        Chunshui Cao,
        Xu Liu and
        Yongzhen Huang% <-this % stops a space
}

% The paper headers
\markboth{Journal of \LaTeX\ Class Files,~Vol.~14, No.~8, August~2015}%
{Shell \MakeLowercase{\textit{et al.}}: Bare Demo of IEEEtran.cls for IEEE Journals}
% The only time the second header will appear is for the odd numbered pages
% after the title page when using the twoside option.
% 
% *** Note that you probably will NOT want to include the author's ***
% *** name in the headers of peer review papers.                   ***
% You can use \ifCLASSOPTIONpeerreview for conditional compilation here if
% you desire.




% If you want to put a publisher's ID mark on the page you can do it like
% this:
%\IEEEpubid{0000--0000/00\$00.00~\copyright~2015 IEEE}
% Remember, if you use this you must call \IEEEpubidadjcol in the second
% column for its text to clear the IEEEpubid mark.



% use for special paper notices
%\IEEEspecialpapernotice{(Invited Paper)}




% make the title area
\maketitle

% As a general rule, do not put math, special symbols or citations
% in the abstract or keywords.
\textcolor{blue}{
In the supplementary materials, we provide implementation details, ablation studies, and visualizations of t-SNE to further illustrate the effectiveness of our method.}
\section{Implementation Details}
\textcolor{blue}{For CASIA-BN~\cite{yu2006framework} and Outdoor-Gait~\cite{song2019gaitnet}, we achieve a good performance by simply using a fixed momentum value of $m=0.2$. However, due to its unprecedented scale and real-world settings, the utilization of a relatively small fixed value for GREW~\cite{lin2014effects} will result in excessively rapid updates of cluster centroids, compromising the overall consistency of the clusters. Therefore, we use a cosine annealing strategy to update $m$, the new strategy can be formulated as follows:
\begin{equation}
\label{momentum_strategy}
   m_t = m_{\text{min}} + \frac{1}{2} (m_{\text{max}} - m_{\text{min}}) \left(1 + \cos\left(\frac{t \pi}{T}\right)\right)
\end{equation}
where $m_t$ is the momentum at training step $t$, $m_{\text{min}}$ is the minimum momentum value, $m_{\text{max}}$ is the initial value, and $T$ represents the total number of training steps within a single epoch. As training progresses ($t$ increases), the momentum decreases following a cosine curve. We set $m_{\text{max}}=0.5$ and $m_{\text{min}}=0.1$.
}
\section{Ablation Study}
\subsection{Effects of Different Parameters in Baseline}
\label{hyper}
%
Here we research how hyper-parameters $s_{up}$, $n$, $\tau$, $m$ affect the results of baselines. 
%
We adjust one parameter at the one time and keep other hyper-parameters unchanged. 
%
$s_{up}$ regulates the boundary of how far the features can be gathered into one cluster. The smaller $s_{up}$ it is, the tighter the boundary.
%
$n$ is the number of neighbors KNN searched for each sequence.
%
$\tau$ is the temperature parameter in ClusterNCE loss, indicating the entropy of the distribution.
%
$m$, the momentum value, controls the update speed of centroids stored in the Memory Bank.
%
From the results on CASIA-BN, we can see that when $s_{up}=0.7$, $n=40$, $\tau=0.05$, $m=0.2$ we have the overall best results for NM, BG, and CL.
%
When these parameters deviate too much from the current setting, the performance is sub-optimal.
%
Here we show the accuracy of NM, BG, and CL when adopting different parameters in the baseline framework in Figure~\ref{fig:hyperparameter}.

\begin{figure*}[h]
	\centering	 
	\includegraphics[width=\linewidth]{figures/hyper-parameter.pdf}	 	
	\caption{The effect of hyper-parameters $s_{up}/n/\tau/m$ on baselines. In there we choose a set of hyper-parameters that have the best result in our experiments. Other hyper-parameters don't change the result a lot, just lead to sub-optimal.}
	\label{fig:hyperparameter}
\end{figure*}
\subsection{Impact of Rate of Curriculum Learning in SSF}

We test the effect of SSF with a dynamic or constant rate when conducting curriculum learning on CASIA-BN. 
%
Without curriculum learning, linearly clustering the front/back view sequences with sequences in other views will degrade the performance.
%
In Table~\ref{tab:SS-Fusion}, with a dynamic pulling rate, we can relax the requirement when training the model, which can make the model learn from easy to hard better. 
\begin{table}[t]
\centering
\caption{The performance of different kinds of rates when conducting curriculum learning.}
\setlength{\tabcolsep}{12pt}
\begin{tabular}{cccc}
\toprule
Settings & NM & BG & CL \\ \midrule
Ours ($\lambda_{base}$) & 90.3  & 82.7 & 40.7 \\
Ours ($\lambda$) & \textbf{90.3} & \textbf{82.9} & \textbf{40.8} \\ \bottomrule
\end{tabular}%
\label{tab:SS-Fusion}
\end{table}

\section{Visulization of Selective Fusion}
The visualization effect of Selective Fusion is shown in Figure~\ref{fig:tsne}. We select a subject in CASIA-BN, finding that in baseline, BG and CL have a different pseudo label with NM. 
%
At the same time, some sequences in front/back views of NM/BG/CL are also assigned different pseudo labels with sequences in other views.
%
With Selective Fusion, most sequences in various views and different conditions are assigned the same ID.
\begin{figure*}[t]
\centering  %图片全局居中
\subfigure[Baseline]{
\label{Fig.sub.1}
\includegraphics[width=0.3\linewidth]{IEEEtran/figures/tsne_baseline.pdf}}\subfigure[Ours]{
\label{Fig.sub.2}
\includegraphics[width=0.3\linewidth]{IEEEtran/figures/tsne_ours.pdf}}
\caption{The TSNE images of baseline/Ours. The text above each feature point shows the pseudo label. The blue color in the text represents \textit{the sequence is in front/back view}. A type of color in a feature point indicates a type of condition. SF is the Selective Fusion method we proposed. Please scroll in to see the details.}
\label{fig:tsne}
\end{figure*}
%
\section{Discussion and future work}

Our method can be employed with off-the-shelf backbones to train a new unlabeled dataset without a label. 
%
However, there are still some limitations. 
%
First, we utilize prior knowledge about the views to identify the challenging samples with front/back views. Without such knowledge, it is difficult to specifically group these sequences with other views.
%
More automatic methods can be further developed to reduce the dependence on prior knowledge.
%

Second, our method uses data augmentation to simulate cross-cloth samples for gait recognition, which makes the model's recognition performance dependent on the effect of data augmentation to some extent. Therefore, more data augmentation methods can be developed to simulate real-world cloth-changing situations.
%
Also, the intrinsic principles of cross-cloth recognition need further study.
%

Third, the accuracy of unsupervised learning relies much on the pre-trained model's accuracy, so a higher precision model is necessary when conducting unsupervised learning on larger datasets, especially on real-world datasets.
%


Since OU-MVLP~\cite{takemura2018multi} is a dataset that is captured in the lab environment, it provides limited knowledge when realizing cross-view and cross-cloth on real-world data.
%
So further efforts need to be spent on training a more robust and high precision pre-train model to make the unsupervised learning method better adapted to the real world. 

Our work conducts gait recognition with unsupervised learning, which alleviates the human labor requirement in the data collection process, making training gait recognition models with large datasets possible and economical. This is crucial because currently previous datasets are collected with manual labels, which can be a time-consuming and costly process, especially when large datasets are involved. By reducing the amount of human labor required, unsupervised gait recognition makes training gait recognition models with large datasets more feasible and cost-effective. Training with larger datasets can ultimately lead to more accurate and robust gait recognition models, which can have a wide range of applications in fields such as security, healthcare, and sports analysis. 

In a nutshell, for future work, more intelligent methods can be developed to identify sequences taken from front/back views and incorporate them with other views correctly. And more robust architecture can be developed to make the gait recognition method stable when adapting to real-world datasets.

% For peer review papers, you can put extra information on the cover
% page as needed:
% \ifCLASSOPTIONpeerreview
% \begin{center} \bfseries EDICS Category: 3-BBND \end{center}
% \fi
%
% For peerreview papers, this IEEEtran command inserts a page break and
% creates the second title. It will be ignored for other modes.
% \IEEEpeerreviewmaketitle










% if have a single appendix:
%\appendix[Proof of the Zonklar Equations]
% or
%\appendix  % for no appendix heading
% do not use \section anymore after \appendix, only \section*
% is possibly needed

% use appendices with more than one appendix
% then use \section to start each appendix
% you must declare a \section before using any
% \subsection or using \label (\appendices by itself
% starts a section numbered zero.)
%


% \appendices
% \section{Proof of the First Zonklar Equation}
% Appendix one text goes here.

% % you can choose not to have a title for an appendix
% % if you want by leaving the argument blank
% \section{}
% Appendix two text goes here.


% % use section* for acknowledgment
% \section*{Acknowledgment}


% The authors would like to thank...


% Can use something like this to put references on a page
% by themselves when using endfloat and the captionsoff option.
\ifCLASSOPTIONcaptionsoff
  \newpage
\fi



\bibliographystyle{IEEEtran}
\bibliography{Transactions-Bibliography/IEEEabrv,Transactions-Bibliography/egbib}

% trigger a \newpage just before the given reference
% number - used to balance the columns on the last page
% adjust value as needed - may need to be readjusted if
% the document is modified later
%\IEEEtriggeratref{8}
% The "triggered" command can be changed if desired:
%\IEEEtriggercmd{\enlargethispage{-5in}}

% references section

% can use a bibliography generated by BibTeX as a .bbl file
% BibTeX documentation can be easily obtained at:
% http://mirror.ctan.org/biblio/bibtex/contrib/doc/
% The IEEEtran BibTeX style support page is at:
% http://www.michaelshell.org/tex/ieeetran/bibtex/
%\bibliographystyle{IEEEtran}
% argument is your BibTeX string definitions and bibliography database(s)
%\bibliography{IEEEabrv,../bib/paper}
%
% <OR> manually copy in the resultant .bbl file
% set second argument of \begin to the number of references
% (used to reserve space for the reference number labels box)


% You can push biographies down or up by placing
% a \vfill before or after them. The appropriate
% use of \vfill depends on what kind of text is
% on the last page and whether or not the columns
% are being equalized.

%\vfill

% Can be used to pull up biographies so that the bottom of the last one
% is flush with the other column.
%\enlargethispage{-5in}



% that's all folks
\end{document}


