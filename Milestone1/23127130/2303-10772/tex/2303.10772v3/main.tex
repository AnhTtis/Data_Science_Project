
%% bare_jrnl.tex
%% V1.4b
%% 2015/08/26
%% by Michael Shell
%% see http://www.michaelshell.org/
%% for current contact information.
%%
%% This is a skeleton file demonstrating the use of IEEEtran.cls
%% (requires IEEEtran.cls version 1.8b or later) with an IEEE
%% journal paper.
%%
%% Support sites:
%% http://www.michaelshell.org/tex/ieeetran/
%% http://www.ctan.org/pkg/ieeetran
%% and
%% http://www.ieee.org/

%%*************************************************************************
%% Legal Notice:
%% This code is offered as-is without any warranty either expressed or
%% implied; without even the implied warranty of MERCHANTABILITY or
%% FITNESS FOR A PARTICULAR PURPOSE! 
%% User assumes all risk.
%% In no event shall the IEEE or any contributor to this code be liable for
%% any damages or losses, including, but not limited to, incidental,
%% consequential, or any other damages, resulting from the use or misuse
%% of any information contained here.
%%
%% All comments are the opinions of their respective authors and are not
%% necessarily endorsed by the IEEE.
%%
%% This work is distributed under the LaTeX Project Public License (LPPL)
%% ( http://www.latex-project.org/ ) version 1.3, and may be freely used,
%% distributed and modified. A copy of the LPPL, version 1.3, is included
%% in the base LaTeX documentation of all distributions of LaTeX released
%% 2003/12/01 or later.
%% Retain all contribution notices and credits.
%% ** Modified files should be clearly indicated as such, including  **
%% ** renaming them and changing author support contact information. **
%%*************************************************************************


% *** Authors should verify (and, if needed, correct) their LaTeX system  ***
% *** with the testflow diagnostic prior to trusting their LaTeX platform ***
% *** with production work. The IEEE's font choices and paper sizes can   ***
% *** trigger bugs that do not appear when using other class files.       ***                          ***
% The testflow support page is at:
% http://www.michaelshell.org/tex/testflow/



\documentclass[journal]{IEEEtran}
%
% If IEEEtran.cls has not been installed into the LaTeX system files,
% manually specify the path to it like:
% \documentclass[journal]{../sty/IEEEtran}





% Some very useful LaTeX packages include:
% (uncomment the ones you want to load)


% *** MISC UTILITY PACKAGES ***
%
%\usepackage{ifpdf}
% Heiko Oberdiek's ifpdf.sty is very useful if you need conditional
% compilation based on whether the output is pdf or dvi.
% usage:
% \ifpdf
%   % pdf code
% \else
%   % dvi code
% \fi
% The latest version of ifpdf.sty can be obtained from:
% http://www.ctan.org/pkg/ifpdf
% Also, note that IEEEtran.cls V1.7 and later provides a builtin
% \ifCLASSINFOpdf conditional that works the same way.
% When switching from latex to pdflatex and vice-versa, the compiler may
% have to be run twice to clear warning/error messages.






% *** CITATION PACKAGES ***
%
%\usepackage{cite}
% cite.sty was written by Donald Arseneau
% V1.6 and later of IEEEtran pre-defines the format of the cite.sty package
% \cite{} output to follow that of the IEEE. Loading the cite package will
% result in citation numbers being automatically sorted and properly
% "compressed/ranged". e.g., [1], [9], [2], [7], [5], [6] without using
% cite.sty will become [1], [2], [5]--[7], [9] using cite.sty. cite.sty's
% \cite will automatically add leading space, if needed. Use cite.sty's
% noadjust option (cite.sty V3.8 and later) if you want to turn this off
% such as if a citation ever needs to be enclosed in parenthesis.
% cite.sty is already installed on most LaTeX systems. Be sure and use
% version 5.0 (2009-03-20) and later if using hyperref.sty.
% The latest version can be obtained at:
% http://www.ctan.org/pkg/cite
% The documentation is contained in the cite.sty file itself.






% *** GRAPHICS RELATED PACKAGES ***
%
\ifCLASSINFOpdf
  \usepackage[pdftex]{graphicx}
  \usepackage[nocompress]{cite}
  \usepackage{graphicx}
  \usepackage{amsmath}
  \usepackage{amssymb}
  \usepackage{booktabs}
  \usepackage{algorithm}
  \usepackage{algorithmic}
  \usepackage{multirow}
  \usepackage[dvipsnames,table,xcdraw]{xcolor}
% \usepackage{colortbl}
% \usepackage{color}
  \usepackage{tikz}
  \usepackage{booktabs}
  \usepackage{threeparttable}
  \usepackage{graphicx}                                                           
  \usepackage{float} 
  \usepackage{times}
  \usepackage{epsfig}
  \usepackage{subfigure}
  % declare the path(s) where your graphic files are
  % \graphicspath{{../pdf/}{../jpeg/}}
  % and their extensions so you won't have to specify these with
  % every instance of \includegraphics
  % \DeclareGraphicsExtensions{.pdf,.jpeg,.png}
\else
  % or other class option (dvipsone, dvipdf, if not using dvips). graphicx
  % will default to the driver specified in the system graphics.cfg if no
  % driver is specified.
  % \usepackage[dvips]{graphicx}
  % declare the path(s) where your graphic files are
  % \graphicspath{{../eps/}}
  % and their extensions so you won't have to specify these with
  % every instance of \includegraphics
  % \DeclareGraphicsExtensions{.eps}
\fi
% graphicx was written by David Carlisle and Sebastian Rahtz. It is
% required if you want graphics, photos, etc. graphicx.sty is already
% installed on most LaTeX systems. The latest version and documentation
% can be obtained at: 
% http://www.ctan.org/pkg/graphicx
% Another good source of documentation is "Using Imported Graphics in
% LaTeX2e" by Keith Reckdahl which can be found at:
% http://www.ctan.org/pkg/epslatex
%
% latex, and pdflatex in dvi mode, support graphics in encapsulated
% postscript (.eps) format. pdflatex in pdf mode supports graphics
% in .pdf, .jpeg, .png and .mps (metapost) formats. Users should ensure
% that all non-photo figures use a vector format (.eps, .pdf, .mps) and
% not a bitmapped formats (.jpeg, .png). The IEEE frowns on bitmapped formats
% which can result in "jaggedy"/blurry rendering of lines and letters as
% well as large increases in file sizes.
%
% You can find documentation about the pdfTeX application at:
% http://www.tug.org/applications/pdftex





% *** MATH PACKAGES ***
%
%\usepackage{amsmath}
% A popular package from the American Mathematical Society that provides
% many useful and powerful commands for dealing with mathematics.
%
% Note that the amsmath package sets \interdisplaylinepenalty to 10000
% thus preventing page breaks from occurring within multiline equations. Use:
%\interdisplaylinepenalty=2500
% after loading amsmath to restore such page breaks as IEEEtran.cls normally
% does. amsmath.sty is already installed on most LaTeX systems. The latest
% version and documentation can be obtained at:
% http://www.ctan.org/pkg/amsmath





% *** SPECIALIZED LIST PACKAGES ***
%
%\usepackage{algorithmic}
% algorithmic.sty was written by Peter Williams and Rogerio Brito.
% This package provides an algorithmic environment fo describing algorithms.
% You can use the algorithmic environment in-text or within a figure
% environment to provide for a floating algorithm. Do NOT use the algorithm
% floating environment provided by algorithm.sty (by the same authors) or
% algorithm2e.sty (by Christophe Fiorio) as the IEEE does not use dedicated
% algorithm float types and packages that provide these will not provide
% correct IEEE style captions. The latest version and documentation of
% algorithmic.sty can be obtained at:
% http://www.ctan.org/pkg/algorithms
% Also of interest may be the (relatively newer and more customizable)
% algorithmicx.sty package by Szasz Janos:
% http://www.ctan.org/pkg/algorithmicx




% *** ALIGNMENT PACKAGES ***
%
%\usepackage{array}
% Frank Mittelbach's and David Carlisle's array.sty patches and improves
% the standard LaTeX2e array and tabular environments to provide better
% appearance and additional user controls. As the default LaTeX2e table
% generation code is lacking to the point of almost being broken with
% respect to the quality of the end results, all users are strongly
% advised to use an enhanced (at the very least that provided by array.sty)
% set of table tools. array.sty is already installed on most systems. The
% latest version and documentation can be obtained at:
% http://www.ctan.org/pkg/array


% IEEEtran contains the IEEEeqnarray family of commands that can be used to
% generate multiline equations as well as matrices, tables, etc., of high
% quality.




% *** SUBFIGURE PACKAGES ***
%\ifCLASSOPTIONcompsoc
%  \usepackage[caption=false,font=normalsize,labelfont=sf,textfont=sf]{subfig}
%\else
%  \usepackage[caption=false,font=footnotesize]{subfig}
%\fi
% subfig.sty, written by Steven Douglas Cochran, is the modern replacement
% for subfigure.sty, the latter of which is no longer maintained and is
% incompatible with some LaTeX packages including fixltx2e. However,
% subfig.sty requires and automatically loads Axel Sommerfeldt's caption.sty
% which will override IEEEtran.cls' handling of captions and this will result
% in non-IEEE style figure/table captions. To prevent this problem, be sure
% and invoke subfig.sty's "caption=false" package option (available since
% subfig.sty version 1.3, 2005/06/28) as this is will preserve IEEEtran.cls
% handling of captions.
% Note that the Computer Society format requires a larger sans serif font
% than the serif footnote size font used in traditional IEEE formatting
% and thus the need to invoke different subfig.sty package options depending
% on whether compsoc mode has been enabled.
%
% The latest version and documentation of subfig.sty can be obtained at:
% http://www.ctan.org/pkg/subfig




% *** FLOAT PACKAGES ***
%
%\usepackage{fixltx2e}
% fixltx2e, the successor to the earlier fix2col.sty, was written by
% Frank Mittelbach and David Carlisle. This package corrects a few problems
% in the LaTeX2e kernel, the most notable of which is that in current
% LaTeX2e releases, the ordering of single and double column floats is not
% guaranteed to be preserved. Thus, an unpatched LaTeX2e can allow a
% single column figure to be placed prior to an earlier double column
% figure.
% Be aware that LaTeX2e kernels dated 2015 and later have fixltx2e.sty's
% corrections already built into the system in which case a warning will
% be issued if an attempt is made to load fixltx2e.sty as it is no longer
% needed.
% The latest version and documentation can be found at:
% http://www.ctan.org/pkg/fixltx2e


%\usepackage{stfloats}
% stfloats.sty was written by Sigitas Tolusis. This package gives LaTeX2e
% the ability to do double column floats at the bottom of the page as well
% as the top. (e.g., "\begin{figure*}[!b]" is not normally possible in
% LaTeX2e). It also provides a command:
%\fnbelowfloat
% to enable the placement of footnotes below bottom floats (the standard
% LaTeX2e kernel puts them above bottom floats). This is an invasive package
% which rewrites many portions of the LaTeX2e float routines. It may not work
% with other packages that modify the LaTeX2e float routines. The latest
% version and documentation can be obtained at:
% http://www.ctan.org/pkg/stfloats
% Do not use the stfloats baselinefloat ability as the IEEE does not allow
% \baselineskip to stretch. Authors submitting work to the IEEE should note
% that the IEEE rarely uses double column equations and that authors should try
% to avoid such use. Do not be tempted to use the cuted.sty or midfloat.sty
% packages (also by Sigitas Tolusis) as the IEEE does not format its papers in
% such ways.
% Do not attempt to use stfloats with fixltx2e as they are incompatible.
% Instead, use Morten Hogholm'a dblfloatfix which combines the features
% of both fixltx2e and stfloats:
%
% \usepackage{dblfloatfix}
% The latest version can be found at:
% http://www.ctan.org/pkg/dblfloatfix




%\ifCLASSOPTIONcaptionsoff
%  \usepackage[nomarkers]{endfloat}
% \let\MYoriglatexcaption\caption
% \renewcommand{\caption}[2][\relax]{\MYoriglatexcaption[#2]{#2}}
%\fi
% endfloat.sty was written by James Darrell McCauley, Jeff Goldberg and 
% Axel Sommerfeldt. This package may be useful when used in conjunction with 
% IEEEtran.cls'  captionsoff option. Some IEEE journals/societies require that
% submissions have lists of figures/tables at the end of the paper and that
% figures/tables without any captions are placed on a page by themselves at
% the end of the document. If needed, the draftcls IEEEtran class option or
% \CLASSINPUTbaselinestretch interface can be used to increase the line
% spacing as well. Be sure and use the nomarkers option of endfloat to
% prevent endfloat from "marking" where the figures would have been placed
% in the text. The two hack lines of code above are a slight modification of
% that suggested by in the endfloat docs (section 8.4.1) to ensure that
% the full captions always appear in the list of figures/tables - even if
% the user used the short optional argument of \caption[]{}.
% IEEE papers do not typically make use of \caption[]'s optional argument,
% so this should not be an issue. A similar trick can be used to disable
% captions of packages such as subfig.sty that lack options to turn off
% the subcaptions:
% For subfig.sty:
% \let\MYorigsubfloat\subfloat
% \renewcommand{\subfloat}[2][\relax]{\MYorigsubfloat[]{#2}}
% However, the above trick will not work if both optional arguments of
% the \subfloat command are used. Furthermore, there needs to be a
% description of each subfigure *somewhere* and endfloat does not add
% subfigure captions to its list of figures. Thus, the best approach is to
% avoid the use of subfigure captions (many IEEE journals avoid them anyway)
% and instead reference/explain all the subfigures within the main caption.
% The latest version of endfloat.sty and its documentation can obtained at:
% http://www.ctan.org/pkg/endfloat
%
% The IEEEtran \ifCLASSOPTIONcaptionsoff conditional can also be used
% later in the document, say, to conditionally put the References on a 
% page by themselves.




% *** PDF, URL AND HYPERLINK PACKAGES ***
%
%\usepackage{url}
% url.sty was written by Donald Arseneau. It provides better support for
% handling and breaking URLs. url.sty is already installed on most LaTeX
% systems. The latest version and documentation can be obtained at:
% http://www.ctan.org/pkg/url
% Basically, \url{my_url_here}.




% *** Do not adjust lengths that control margins, column widths, etc. ***
% *** Do not use packages that alter fonts (such as pslatex).         ***
% There should be no need to do such things with IEEEtran.cls V1.6 and later.
% (Unless specifically asked to do so by the journal or conference you plan
% to submit to, of course. )


% correct bad hyphenation here
\hyphenation{op-tical net-works semi-conduc-tor}


\begin{document}
%
% paper title
% Titles are generally capitalized except for words such as a, an, and, as,
% at, but, by, for, in, nor, of, on, or, the, to and up, which are usually
% not capitalized unless they are the first or last word of the title.
% Linebreaks \\ can be used within to get better formatting as desired.
% Do not put math or special symbols in the title.
\title{Unsupervised Gait Recognition with \\ Selective Fusion}

\author{Xuqian Ren,
        Shaopeng Yang,
        Saihui Hou,
        Chunshui Cao,
        Xu Liu and
        Yongzhen Huang% <-this % stops a space
\thanks{Xuqian Ren is with the Computer Science Unit, Faculty of Information Technology and Communication Sciences, Tampere University, Tampere 33720, Finland. This work is finished when she was an intern at Watrix Technology Limited Co. Ltd before becoming a Ph.D. candidate at Tampere University.
% note need leading \protect in front of \\ to get a newline within \thanks as
% \\ is fragile and will error, could use \hfil\break instead.
(E-mail: xuqian.ren@tuni.fi)}% <-this % stops a space
\thanks{Shaopeng Yang is with School of Artificial Intelligence,
Beijing Normal University, Beijing 100875, China and also with Watrix Technology Limited Co. Ltd, Beijing 100088, China. He is co-first author.}% <-this % stops a space
\thanks{Saihui Hou and Yongzhen Huang are with School of Artificial Intelligence,
Beijing Normal University, Beijing 100875, China and also with Watrix Technology Limited Co. Ltd, Beijing 100088, China. (E-mail: housai hui@bnu.edu.cn, huangyongzhen@bnu.edu.cn). They are the corresponding authors of this paper.}
\thanks{Chunshui Cao and Xu Liu are with Watrix Technology Limited Co. Ltd, Beijing 100088, China.}
}

% note the % following the last \IEEEmembership and also \thanks - 
% these prevent an unwanted space from occurring between the last author name
% and the end of the author line. i.e., if you had this:
% 
% \author{....lastname \thanks{...} \thanks{...} }
%                     ^------------^------------^----Do not want these spaces!
%
% a space would be appended to the last name and could cause every name on that
% line to be shifted left slightly. This is one of those "LaTeX things". For
% instance, "\textbf{A} \textbf{B}" will typeset as "A B" not "AB". To get
% "AB" then you have to do: "\textbf{A}\textbf{B}"
% \thanks is no different in this regard, so shield the last } of each \thanks
% that ends a line with a % and do not let a space in before the next \thanks.
% Spaces after \IEEEmembership other than the last one are OK (and needed) as
% you are supposed to have spaces between the names. For what it is worth,
% this is a minor point as most people would not even notice if the said evil
% space somehow managed to creep in.



% The paper headers
\markboth{Journal of \LaTeX\ Class Files,~Vol.~14, No.~8, August~2015}%
{Shell \MakeLowercase{\textit{et al.}}: Bare Demo of IEEEtran.cls for IEEE Journals}
% The only time the second header will appear is for the odd numbered pages
% after the title page when using the twoside option.
% 
% *** Note that you probably will NOT want to include the author's ***
% *** name in the headers of peer review papers.                   ***
% You can use \ifCLASSOPTIONpeerreview for conditional compilation here if
% you desire.




% If you want to put a publisher's ID mark on the page you can do it like
% this:
%\IEEEpubid{0000--0000/00\$00.00~\copyright~2015 IEEE}
% Remember, if you use this you must call \IEEEpubidadjcol in the second
% column for its text to clear the IEEEpubid mark.



% use for special paper notices
%\IEEEspecialpapernotice{(Invited Paper)}




% make the title area
\maketitle

% As a general rule, do not put math, special symbols or citations
% in the abstract or keywords.
\begin{abstract}
Previous gait recognition methods primarily relied on labeled datasets, which require a labor-intensive labeling process. To eliminate this dependency, we focus on a new task: Unsupervised Gait Recognition (UGR). We introduce a cluster-based baseline to solve UGR. However, we identify additional challenges in this task. First, sequences of the same person in different clothes tend to cluster separately due to significant appearance changes. Second, sequences captured from $0^{\circ}$ and $180^{\circ}$ views lack distinct walking postures and do not cluster with sequences from other views. To address these challenges, we propose a Selective Fusion method, consisting of Selective Cluster Fusion (SCF) and Selective Sample Fusion (SSF). SCF merges clusters of the same person wearing different clothes by updating the cluster-level memory bank using a multi-cluster update strategy. SSF gradually merges sequences taken from front/back views using curriculum learning. Extensive experiments demonstrate the effectiveness of our method in improving rank-1 accuracy under different clothing and view conditions.
\end{abstract}

% Note that keywords are not normally used for peerreview papers.
\begin{IEEEkeywords}
Gait Recognition, Unsupervised Learning, Contrastive Learning, Curriculum Learning.
\end{IEEEkeywords}






% For peer review papers, you can put extra information on the cover
% page as needed:
% \ifCLASSOPTIONpeerreview
% \begin{center} \bfseries EDICS Category: 3-BBND \end{center}
% \fi
%
% For peerreview papers, this IEEEtran command inserts a page break and
% creates the second title. It will be ignored for other modes.
\IEEEpeerreviewmaketitle



\section{Introduction}
% 介绍步态识别,引出主要无监督步态识别问题
\IEEEPARstart{W}{ith} the growing intelligent security and safety camera systems, gait recognition has gradually gained more attention and exploration for its non-contact, long-term, and long-distance recognition properties. 
%
Several works~\cite{chao2019gaitset,fan2020gaitpart,lin2021gait} attempt to solve gait recognition tasks and have reached significant progress in a laboratory environment. 
%
However, gait recognition in a realistic situation~\cite{ren2022progressive} and will be affected by many factors such as occlusion, dirty labels, labeling, and more. 
%
Labeling, especially, is a major challenge, requiring intensive manual effort for pairwise data. Therefore, training on unlabeled datasets becomes crucial to save resources and address these challenges.
% Among them, the problem of labeling is one of the biggest challenges because it demands intensive manual effort to label pairwise data. 
%
% Moreover, deploying pre-trained models in new test environments without any adaptation often suffers from severe performance deterioration due to the domain gap across different datasets.
%
% So it is necessary to train gait recognition models on unlabeled datasets, which saves a lot of human and financial resources. 

%----------------------------------------
To realize gait recognition using an unlabeled dataset, we focus on a task called \textbf{Unsupervised Gait Recognition} (UGR) to facilitate the research on training gait recognition models with new unlabeled datasets.
%
Here we focus on using silhouettes for gait recognition to illustrate our method.
%
When only using silhouettes of human walking sequences as input, due to lack of enough information, we observe two main challenges in UGR, as shown in Figure~\ref{fig:Introduction}.
%
First, due to the large change in appearance, sequences in different clothes of a subject are hard to gather into one cluster without any label supervision.
%
Second, sequences captured from front/back views, such as views in $0^{\circ}/018^{\circ}/162^{\circ}/180^{\circ}$ in CASIA-B~\cite{yu2006framework}, are challenging to gather with sequences taken from other views of the same person because they lack vital information, such as walking postures.
%
Furthermore, these sequences tend to cluster into small groups based on their views or get mixed with sequences of the same perspective from other subjects.
%
So in this paper, we provide methods to overcome them accordingly.

\begin{figure}[t]
	\centering	 
	\includegraphics[width=\linewidth]{IEEEtran/figures/Introduction.pdf}	 	
	\caption{Two main challenges in UGR. A kind of style in different colors denotes a subject in different clothes, which are usually erroneously assigned with different pseudo labels (\textit{e.g.}, `001', `002'). Also, sequences taken from front/back views of different subjects tend to mix together (\textit{e.g.}, `003').}
	\label{fig:Introduction}
 % \vspace{-0.6cm}
\end{figure}

Currently, some Person Re-identification (Re-ID) works~\cite{lin2019bottom,ge2020self,lin2020unsupervised,wang2020unsupervised,dai2021cluster,9978648} have already touched the field of identifying person in an unsupervised manner.
%
There are also some traditional methods, such as~\cite{ball2012unsupervised,cola2015unsupervised,rida2015unsupervised} employ unsupervised learning to facilitate the development of gait recognition. 
%
However, research directions involving methods based on deep learning in this field are still under-explored.
%
In this paper, we use the state-of-the-art pattern in~\cite{dai2021cluster}, a cluster-based framework with contrastive learning, as a baseline to realize UGR.
%
% However, when directly adopting this framework, we find the performance still needs to be improved, especially in the \textbf{walking with different clothes (CL)} condition.
%
To address the two challenges, we propose a new method called \textbf{S}elective \textbf{F}usion(SF), to gradually pull cross-view and cross-cloth sequences together.


Our method comprises two techniques: Selective Cluster Fusion (SCF), which is used to narrow the distance of cross-cloth clusters, and Selective Sample Fusion (SSF), which is used to pull cross-view pairs nearly, especially helpful for outliers near $0^{\circ}/180^{\circ}$.
%
To be specific, first, in SCF, we use a support set selection module to generate a support set for each cluster.
%
In the support set, there are selected candidate clusters of each corresponding cluster that potentially belong to the same person but are in different clothes.
%
There is another multi-cluster update strategy designed in SCF to help update the cluster centroid of each pseudo cluster in the memory bank.
%
Using this approach, we not only tighten the clusters but also encourage clusters of the same individual in different clothing to be influenced by the current clustered groups and pulled closer toward them.
%
Second, we designed the SSF to deal with samples taken from front/back views. 
%
In SSF, we utilize a view classifier to identify sequences captured from front/back views. We then employ curriculum learning to gradually incorporate these sequences with those captured from other views.
%
Namely, the sequences are absorbed at a dynamic rate, relaxing the aggregate requirement for each cluster.
%
This approach enables us to re-assign pseudo labels for sequences captured from front/back views, thereby encouraging them to cluster with sequences captured from other views.
%
With our method, we gain a large recognition accuracy improvement compared to the baseline (with GaitSet~\cite{chao2019gaitset} backbone: NM + 3.1\%, BG + 8.6\%, CL + 9.7\%; with GaitGL~\cite{lin2021gait} backbone: BG + 1.1\%, GL + 17.2\% on CASIA-BN dataset~\cite{yu2006framework}\footnote{NM: normal walking condition, BG: carrying bags when walking, CL: walking with different coats.}).


% It is worth noting that our method focuses on clustering the features and learning the interior correlations of clusters based on pseudo labels. We do not specifically focus on learning the domain-invariant features or addressing domain shift in particular.
% % It bears emphasizing that our method focuses on clustering the features and learning the interior correlations of clusters based on pseudo labels. 
% % %
% % We don’t focus on learning the domain-invariant features or reducing the domain shift particularly. 
% %
% Also, unlike the Re-ID task, there does not exist a unified pre-trained model like ResNet-50~\cite{he2016deep} pre-trained on ImageNet~\cite{deng2009imagenet} that can be used for the gait community, which is another promising future research direction. 
% %
% Our method is built on a model pre-trained on OU-MVLP~\cite{takemura2018multi}. The OU-MVLP dataset is only used in the pre-training stage and is not involved in the unsupervised learning stage.
% %
% Since no large dataset can be used to learn cross-cloth information, our method mainly focuses on improving the cross-cloth performance. We achieve this without relying on any prior information from the pre-trained model, making it an unsupervised learning task rather than a domain transformation.}

To sum up, our contributions mainly lie in three folds:
\begin{itemize}
    \item[$\bullet$] We focus on Unsupervised Gait Recognition (UGR) using a cluster-based method with contrastive learning. Despite its practicality, it requires careful consideration. To address this task, we establish a baseline using cluster-level contrastive learning.
    
    \item[$\bullet$]  We deeply explore the characteristics of UGR, finding the two main challenges: clustering sequences with different clothes and with front/back views. To address these challenges, we propose a \textbf{S}elective \textbf{F}usion(SF) method. This method involves selecting potentially matched cluster/sample pairs to help them fuse gradually.
    
    \item[$\bullet$] Extensive experiments on three popular gait recognition benchmarks have shown that our method can bring consistent improvement over baseline, especially in walking in different coat conditions. 
\end{itemize}
%%%%%%%%%%%%%%%%%%%%%%%%%%%%%%%%%%%%%%%%%%%%%%%%%%%

\section{Related Work}\label{sec:relatedwork}

Gait recognition plays an important role in enhancing safety and security in the development of intelligent cities~\cite{zhang2024research}. Most existing gait recognition works are trained in a supervised manner, in which cross-cloth and cross-view labeled sequence pairs have been provided.
%
They mainly focus on learning more discriminative features~\cite{liao2020model,li2020jointsgait,chao2019gaitset,fan2020gaitpart,lin2021gait,9916067,9229117,9913216,10042966} or developing gait recognition applications in natural scenes~\cite{hou2022gait,das2023gait,9870842,9928336}.
%
However, obtaining labeled training pairs is challenging in real-world applications. Despite extensive research in gait recognition, further exploration of its practical applications is still needed.
%
In this work, we consider a practical setting and take one of the first steps toward achieving gait recognition without the need for labeled training datasets.

\subsection{Gait Recognition}

\noindent\textbf{Model-based method:} This kind of method encodes poses or skeletons into discriminative features to classify identities.
%
For example, PoseGait~\cite{liao2020model} extracts handcrafted features from 3D poses based on human prior knowledge. 
%
JointsGait~\cite{li2020jointsgait} extracts spatiotemporal features from 2D joints by GCN~\cite{yan2018spatial}, then maps them into discriminative space according to the human body structure and walking pattern.
%
GaitGraph~\cite{gaitgraph} uses human pose estimation to extract robust pose from RGB images, then encode the key points as nodes, and encode skeletons as joints in the Graph Convolutional Network to extract gait information.

\noindent\textbf{Appearance-based method:} This series of methods mostly input silhouettes, extracting identity information from the shape and walking postures.
%
GaitSet~\cite{chao2019gaitset} first extracts frame-level and set-level features from an unordered silhouette set, promoting the set-based method's development.
%
GaitPart~\cite{fan2020gaitpart} further includes part-level pieces of information, mining details from silhouettes.
%
In contrast, the video-based method GaitGL~\cite{lin2021gait} employs 3D CNN for feature extraction based on temporal knowledge.
%
Our method can be used in appearance-based unsupervised gait recognition.
%
% Since our method mainly focuses on realistic gait recognition with a large amount of data, we choose silhouettes as input for its computation saving and robustness.
%
In our framework, we adopt both the set-based method and the video-based method as the backbones to illustrate the generalization of our framework.

\noindent\textbf{Gait Recognition with Contrastive Learning:} The core idea of contrastive learning methods~\cite{chen2020simple, chen2021exploring, he2020momentum} is to construct effective positive and negative sample pairs through data augmentation and to design appropriate loss functions to optimize the model for learning useful data representations. Inspired by such methods as MoCo~\cite{he2020momentum} and SimCLR~\cite{chen2020simple}, GaitSSB~\cite{fan2023learning} proposes a self-supervised framework to learn general gait representations from large-scale unlabeled walking videos. GaitSSB treats each gait sequence as a single instance and aims to learn discriminative instance-level sequence features through contrastive learning. However, the current data augmentation methods are limited. More importantly, the positive pairs are often drawn from the same sequence, resulting in very similar positive sample pairs. This makes it difficult to simulate the variations caused by changes in clothing and camera viewpoints, thereby limiting the ability to provide effective supervisory signals that can guide the model to learn robust features.
To address these challenges, our method adopts a clustering strategy to group unlabeled data and generate pseudo-labels, allowing the model to learn representations based on cluster assignments. This clustering-based approach generates high-quality pseudo-labels and, unlike traditional contrastive methods where positive pairs are drawn from the same sequence, offers a more diverse and reasonable definition of positive and negative samples across different sequences. Defining these samples across multiple sequences proves more effective in addressing the challenges of cross-clothing and cross-camera scenarios in gait recognition tasks.


\subsection{Unsupervised Person Re-identification}
\textbf{Short-term Unsupervised Re-ID:}
Most fully unsupervised learning (FUL) Re-ID methods estimate pseudo labels for sequences, which can be roughly categorized into clustering-based and non-clustering-based methods. 
%----------------------------------------
Clustering-based methods~\cite{zeng2020hierarchical,wang2021camera,chen2021ice,xuan2021intra,zhang2023camera,dai2021cluster} first estimate a pseudo label for each sequence and train the network with sequence similarity.
%
In contrast, non-clustering-based methods~\cite{lin2020unsupervised,wang2020unsupervised} regard each image as a class and use a non-parametric classifier to push each similar image closer and pull all other images further.
%
In total, the accuracy of most non-cluster-based methods does not exceed the latest cluster-based methods, so we use the latter to solve UGR.
%

At present, there are some typical algorithms in clustering-based methods.
%
BUC~\cite{lin2019bottom} utilizes a bottom-up clustering method, gradually clustering samples into a fixed number of clusters. Though there is a need for more flexibility, it is a good starting point. 
%
HCT~\cite{zeng2020hierarchical} adopts triplet loss to BUC to help learn hard samples. 
%
SpCL~\cite{ge2020self} introduces a self-paced learning strategy and memory bank, gradually making generated sample features closer to reliable cluster centroids. 
%
To alleviate the high intra-class variance inside a cluster caused by camera styles, CAP~\cite{wang2021camera} proposes cross-camera proxy contrastive loss to pull instances near their own camera centroids in a cluster. 
%
ICE~\cite{chen2021ice} further explores inter-instance relationships instead of using camera labels to compact the clusters with hard contrastive loss and soft instance consistency loss. 
%
IICS~\cite{xuan2021intra} also considers the difference caused by cameras, decomposing the training pipeline into two phases. First, it categorizes features within each camera and generates labels. Second, according to sample similarity across cameras, inter-camera pseudo labels will be generated based on all instances. These two stages train CNN alternately to optimize features.
%
Cluster-contrast~\cite{dai2021cluster} improves SpCL by establishing a cluster-level memory dictionary, optimizing and updating both CNN and memory bank at the cluster level.
%----------------------------------------

On the contrary, the non-clustering-based methods mainly realize fully unsupervised Re-ID with similarity-based methods. 
%
SSL~\cite{lin2020unsupervised} predicts a soft label for each sample and trains the classification model with softened label distribution. 
%
MMCL~\cite{wang2020unsupervised} formulates FUL Re-ID as a multi-label classification task and classifies each sample into multiple classes by considering their self-similarity and neighbor similarity.

Inspired by the simple but elegant structure of Cluster-contrast~\cite{dai2021cluster}, we start from the Cluster-contrast framework to solve the UGR task.
Unsupervised~\cite{zeng2020hierarchical,dai2021cluster} and semi-supervised learning~\cite{chen2023class,chen2021multimodal} rely on modeling visual features (e.g., color, texture, shape) extracted from static images. Moreover, ReID methods typically depend on visual consistency across camera views, assuming that the same identity retains relatively stable appearance characteristics. However, gait recognition uses silhouette sequences as input, and it needs to identify features not only cross-view but also cross-clothes.
%

In summary, due to the differences in data modalities and the unique challenges faced by gait recognition, directly applying the cluster contrast method from~\cite{dai2021cluster} to the UGR task is not suitable.
%
As a solution, we design the SCF module with a support set that contains several cluster candidates the feature belongs to. Rather than pushing each feature toward only one cluster, we reduce the negative impact of erroneous gradients by minimizing incorrect associations. To tackle the challenge of clustering features from front/back views with those from other views, we also develop a novel method SSF to specifically help clustering these features from sparse views. 

\noindent\textbf{Long-term Unsupervised Re-ID:}
%
Gait recognition is a long-term task with cloth-changing, so long-term FUL Re-ID is more similar to UGR.
%
CPC~\cite{li2022unsupervised} uses curriculum learning~\cite{bengio2009curriculum} strategy to incorporate easy and hard samples and gradually relax the clustering criterion.
%
We do not use the same method in SCF, since each cluster mainly contains sequences with one cloth type, and it is better to pull clusters as a whole.
%
In contrast, we use curriculum learning in SSF to distinguishly deal with sequences in front/back views and gradually re-assign pseudo labels for those sequences.
%%%%%%%%%%%%%%%%%%%%%%%%%%%%%%%%%%%%%%%%%%%%%%%%%%%%%%%%%%%%%%%%%%

\section{Our Method}
In this work, we propose a new task called Unsupervised Gait Recognition (UGR), which is practical when dealing with realistic unlabeled gait datasets.
%
In this section, we first formally define our technique.
%
Next, we will show our baseline based on Cluster-contrast~\cite{dai2021cluster}, trying to solve UGR with an unlabeled training set.
%
Then, we deeply research the problems faced by UGR and find two challenges to improve the accuracy: sequences in different clothes of the same person tend to form different clusters, and the sequences captured from front/back views are difficult to gather with other views.
%
Based on the two problems, we propose Selective Fusion to gradually merge cross-cloth clusters and sequences taken from front/back views to make samples of each.  

% Please add the following required packages to your document preamble:
% \usepackage{graphicx}
\begin{table}[ht]
\centering
\caption{The definition of important symbols}
\resizebox{\columnwidth}{!}{%
\begin{tabular}{c|c}
\toprule
Symbol & Definition \\ \midrule
$\mathcal{X}_u/\mathcal{X}_t$ & \begin{tabular}[c]{@{}c@{}}Unlabeled training dataset/\\ Labeled testing dataset\end{tabular} \\ \hline
$\mathcal{Y}_u/\mathcal{Y}'_u$ & The true/ pseudo label for training dataset \\ \hline
$\mathcal{Y}_t$ & The true label for testing dataset \\ \hline
$f_\theta$ & Gait Recognition backbone \\ \hline
$N$ & The total sequence number of the training dataset \\ \hline
$f_i$ & \begin{tabular}[c]{@{}c@{}}The feature extracted from the $i$-th \\ sequence of the training dataset  \end{tabular}  \\ \hline



$n$ & \begin{tabular}[c]{@{}c@{}} The number of neighbors KNN \\ searched for each sequence  \end{tabular} \\ \hline
$s_{up}$ & The similarity threshold KNN used for clustering \\ \hline
$Q$ & The number of clusters \\ \hline
$\mathcal{C}_k$ & The $k$-th cluster centroid \\ \hline
$\mathcal{M}$ & The memory bank  \\ \hline
$b$ & The mini-batch extracted from pseudo clusters each iteration  \\ \hline
$q$ & One query contained in $b$ \\ \hline
$\tau$ & The temperature hyper-parameter \\ \hline
$m$ & The momentum hyper-parameter \\ \hline
$\mathcal{L}_{q_{CNCE}}$ & The ClusterNCE Loss of query $q$ \\ \hline
$\mathcal{C}_{+}$ & The positive cluster centroid \\ \hline


$f_{i_{CA}}$ & The feature of $i$-th cloth augmented sequence \\ \hline
$\mathcal{C}_{ak}$ & The $k$-th adjusted cluster centroid \\ \hline

$\mathcal{S}_k$ & The support set of the $k$-th cluster \\ \hline
$a$ & The number of pseudo ids $\mathcal{S}_k$ contains.  \\ \hline
$c_{low}$ & The lower bound to judge FVC \\ \hline
$s_{c}/s_{o}$ & \begin{tabular}[c]{@{}c@{}}The current/initial similarity bound \\ when re-assign pseudo labels for sequences in FVC  \end{tabular}  \\ \hline
$\lambda/\lambda_{base}$ & \begin{tabular}[c]{@{}c@{}} Each epoch ratio/base ratio to merge \\ extreme view sequences in each epoch\end{tabular} \\ \hline
$\mathcal{C}_n/\mathcal{C}_o$ & The number of new or old clusters in each epoch \\ \hline

\end{tabular}%
}

\label{tab:symbol}
\end{table}


%----------------------------------------
\subsection{Problem Formulation}

To formulate the unsupervised learning, we first define an unlabeled training dataset, denoted as $\mathcal{X}_u=\{x_{1}, x_2, ..., x_N\}$, including diverse conditions such as changes in clothing and viewpoint, where $N$ is the total sequence number.
%
% Then, we assume its ground truth label is set as $\mathcal{Y}_u$, but we cannot obtain it during training.
%
We want to train a gait recognition backbone $f_\theta$ to classify these sequences according to their similarity. By clustering the features, we generate pseudo-labels $\mathcal{Y}_u$ for the training dataset. Following this, specialized modules are employed to gradually merge samples from different clothes and views.
%
During evaluation, $f_\theta$ will extract features from a labeled test dataset $\{\mathcal{X}_t, \mathcal{Y}_t\}$ and the gallery will rank according to their similarity with the probe, then we gain the rank-1 accuracy for each condition and each view.
%
We aim to train $f_\theta$ and gain the best performance on $\mathcal{X}_t$.


% We first define an unlabeled training dataset with cloth-changing as $\mathcal{X}_u=\{x_{1}, x_2, ..., x_N\}$, where $N$ is the total sequence number.
% %
% Then, we assume its ground truth label is set as $\mathcal{Y}_u$, but we cannot obtain it during training.
% %
% We want to train a gait recognition backbone $f_\theta$ to classify these sequences according to their similarity.
% %
% During evaluation, $f_\theta$ will extract features from a labeled test dataset $\{\mathcal{X}_t, \mathcal{Y}_t\}$ and the gallery will rank according to their similarity with the probe, then we gain the rank-1 accuracy for each condition and each view.
% %
% We aim to train $f_\theta$ and gain the best performance on $\mathcal{X}_t$.


%----------------------------------------
\subsection{Proposed Baseline}
We modified Cluster-contrast~\cite{dai2021cluster} to build our baseline framework. Since a pre-trained model is required to initialize $f_\theta$, we first pre-train the backbones on a large gait recognition dataset OU-MVLP~\cite{takemura2018multi} and then load it when training on the unlabeled dataset. The training pipeline for the unlabeled dataset can be summarized as follows: 
%
% Since there is no pre-trained model in the Gait Recognition field like in Re-ID, we choose OU-MVLP~\cite{takemura2018multi} to train a pre-trained model and gain cross-view prior knowledge, Since OU-MVLP has a large number of subjects and various views, which is an ideal dataset to obtain identity classification ability.
% %
% Then we modify Cluster-contrast~\cite{dai2021cluster} to build our baseline framework to transfer the knowledge learned from OU-MVLP to other datasets. 
%
%

1) At the beginning of each epoch, we first use $f_\theta$ to extract features from each sequence in the training dataset in parts, which has been sliced by Horizontal Pyramid Matching (HPM)~\cite{chao2019gaitset} horizontally and equally. 
%
Then we concatenate all the parts to an embedding and regard it as a sequence feature to participate in the following process, denoted as $f_i, i \in \{1,2,..., N\} $. 
%
In this way, we can consider each sequence's features in parts and details.
%

2) We adopt KNN~\cite{fix1989discriminatory} to search $n$ neighbors for each sequence in feature space and calculate the similarity distances between each other. 
%
Then InfoMap~\cite{rosvall2008maps} is used to cluster $f_i$ with a similarity threshold $s_{up}$, and predict a pseudo label $\mathcal{Y}'_u= \{y'_{1},y'_{2},...,y'_{N} \}$ for each sequence.
%
When mapping features with a pre-trained model, each sequence tends to be mapped closer and not well separated.
%
So we tighten $s_{up}$, aiming to separate each subject into a single cluster.
%

3) With the pseudo labels, we compute the centroid of each cluster $\mathcal{C}_k, k \in \{0,1,...,Q\}$, $Q$ is the number of clusters, and then initialize a memory bank $\mathcal{M}$ at the cluster-level to store these centers $\mathcal{M} = \{\mathcal{C}_0, \mathcal{C}_1, ..., \mathcal{C}_Q\}$.
%

4) During each iteration, a mini-batch $b$ will be randomly selected from pseudo clusters, and during gradient propagation, we update the backbone with a ClusterNCE loss~\cite{dai2021cluster}.
\begin{equation}
\label{ClusterNCE}
    \mathcal{L}_{q_{CNCE}}=-\log \frac{\exp \left( q\cdot \mathcal{C}_+/\tau \right)}{\sum_{k=1}^Q{\exp}\left( q\cdot \mathcal{C}_k/\tau \right)}
\end{equation}
for each query feature $q$ extracted from the samples in the mini-batch, we calculate its similarity with the positive cluster centroid $\mathcal{C}_+$ it belongs to, $\mathcal{C}_k$ is the $k$-th cluster centroid, $\tau$ is the temperature hyper-parameter. Here we use all the query features in the mini-batch to calculate $\mathcal{L}_{q_{CNCE}}$.
%

5) Also, we update the centroids in the memory bank $\mathcal{M}$ that the queries belong to at a cluster level.
\begin{equation}
   \forall q \in \mathcal{C}_k, \mathcal{C}_k\gets m\mathcal{C}_k+(1-m)q
   \label{eq:momentum}
\end{equation}
$m$ is the momentum hyper-parameter used to update the centroids impacted by the batch. Each feature is responsible for updating the cluster centroids it belongs to. Additionally, $m$ serves as a crucial factor in determining the pace of cluster memory updates, whereby a higher value results in a slower update. This momentum value directly influences the consistency between the cluster features and the most recently updated query instance feature.

%----------------------------------------
With this pipeline, we can initially realize UGR.
%
However, some defects still prevent further improvement in both cross-cloth and cross-view situations.
%
First, each cloth condition of a subject has been separated from each other, making it hard to group them together into a single class.
%
This is because of the large intra-class diversity within each subject when the identity changes cloth and subtle inter-class variance between different persons when clothes types of different subjects are similar.
%
For example, NM and CL of one person are less similar in appearance to NM of other persons, leading to the intra-similarity being smaller than the inter-similarity.
%
Second, some sequences in front/back views (such as $0^{\circ}/018^{\circ}/162^{\circ}/180^{\circ}$) cannot correctly gather with sequences in other views, but tend to be confused with front/back views sequences of other subjects.
%
This is because sequences in these views lack enough walking patterns, so the model can only use the shape information to classify these sequences. With more similarity in appearance, sequences of different subjects in these views tend to be classified together.
%
Necessary solutions need to be considered to help the framework solve the two problems.
%----------------------------------------

\subsection{Proposed Method}

To tackle the problems we pointed out for UGR, we developed Selective Fusion, containing Selective Cluster Fusion (SCF) and Selective Sample Fusion (SSF) to solve the two drawbacks separately.
%
The framework of our method is in Figure~\ref{fig:structure}, and the pseudo-code is shown in Algorithm~\ref{our}.
%
\begin{figure*}[t]
	\centering	 
	\includegraphics[width=\linewidth]{IEEEtran/figures/structure.pdf}
	\caption{Overview of the framework with Selective Fusion. The upper two branches generate pseudo labels and initialize a memory bank at the start of each epoch. The lower branch accepts mini-batch extracted from pseudo clusters and calculates ClusterNCE Loss with Memory Bank to update it and the backbone in each iteration. CA is the \textit{Cloth Augmentation} method. InfoMap is employed in the Cluster module. SCF means \textit{Selective Cluster Fusion}. SSF represents \textit{Selective Sample Fusion}. In the Support set, the darker the color, the higher the similarity with the target cluster (Best viewed in color).}
	\label{fig:structure}
\end{figure*}


\begin{algorithm}[t]
\caption{The training procedure of Selective Fusion}
\label{alg:algorithm}
\textbf{Input}:  $\mathcal{X}_u$;  $f_\theta$
\begin{algorithmic}[1]  %1表示每隔一行编号
\REQUIRE Epoch number; iteration number; batch size; \\
hyper-parameters: $M/s_{up}/\tau/m/\mathcal{C}_{low}/\lambda_{base}/k/\mathcal{S}_{i}$
\FOR {epoch in range(0, epoch number+1) }
    \STATE {Apply CA to $\mathcal{X}_u$, get the augmented $\mathcal{X}_u$;}
    \STATE {Extract $f_n$ from $\mathcal{X}_u$ by $f_\theta$;}
    \STATE {Extract $f_{CA}$ from augmented $\mathcal{X}_u$ by $f_\theta$;}
    \STATE {Generate $\mathcal{Y}'_u$ for $f_n$}
    \STATE {Calculate the pseudo centroids with $\mathcal{Y}'_u$ and $f_n$;}
    \STATE {Generate adjusted $\mathcal{Y}'_u$ in SSF;}
    \STATE {Generate adjusted centroids with adjusted $\mathcal{Y}'_u$ and $f_n$;}
    \STATE {Initialize Memory Bank with adjusted centroids;}   
    \STATE {Generate support set in Support Set Selection Module with the centroids of $f_{CA}$ and $f_n$;}
    \FOR {iter in range(0, iteration number+1) }
    \STATE {Extract mini-batch from pseudo clusters;}
    \STATE {Extract sequences feature from batch;}
    \STATE {Calculate ClusterNCE loss with Memory Bank according to Eq.\textcolor{red}{~\ref{ClusterNCE}};}
    \STATE {Update the backbone;}
    \STATE {Update the Memory Bank according to Eq.\textcolor{red}{~\ref{momentum}};}
    \ENDFOR
\ENDFOR \\
\textbf{Output}: $f_\theta$
\end{algorithmic}
\label{our}
\end{algorithm}

 
In our method, at the beginning of each epoch, we use a Cloth Augmentation (CA) method to randomly generate an augmented variant for each sequence in the training dataset, then put them into the same backbone to extract features, named $f_{i}$ and $f_{i_{CA}}$.
%
Second, after getting the original pseudo labels generated by InfoMap, we use SSF to adjust the pseudo labels and then apply them to $f_{i}$ and $f_{i_{CA}}$ to get their adjusted centroids.
%
The adjusted centroids $\mathcal{C}_{ak}$ are used to initialize the memory bank $\mathcal{M}$.
%
Third, we use a support set selection module to generate a support set for each cluster, which will be used in the multi-cluster update strategy during back-propagation to help update the memory bank.
%
The support set selection module and multi-cluster update strategy are the two components of our SCF.
%

Next, we will introduce how we implement our Cloth Augmentation, SCF, and SSF methods.
% After using InfoMap in the Cluster module, we can get pseudo labels for each original feature, then apply them to $f_{n}$ and calculate cluster centroids.
% %
% However, since the pseudo label has many mistakes, we first put centroids and $f_n$ into SSF to re-assign pseudo labels for each feature, assigning the label of other views to the sequence in front/back views and get the adjusted pseudo labels for each sequence. 
% %
% With the new adjusted pseudo labels, we assign them to $f_{n}$ and gain the final adjusted centroids.
% %
% Also, the adjusted pseudo labels can be used to classify $f_{DA}$, and the centroids of $f_{n}$ and $f_{DA}$ will be put into Support Set Selection, generating the support set for each cluster.
% %
% The adjusted centroids of $f_{n}$ will initialize Memory Bank, storing the centers.
% %
% During each iteration, a batch of sequences will be selected according to the adjusted pseudo labels, then input backbones to extract features in the batch.
% %
% Next, all the batch features will be used when calculating ClusterNCE loss with Memory Bank.
% %
% When conducting gradient backpropagation, we update the backbone as well as use a Multi-cluster Update strategy to update the Memory Bank.
% %
% The Suppor Set Selection Module and Multi-cluster Update Strategy are the two components of our SCF.
% %
% Next, I will introduce how we implement our Data Augmentation, SSF, and SCF.





%----------------------------------------
\subsubsection{Cloth Augmentation}

The cloth augmentation is conducted for each sequence in the training set to explicitly get a fuse direction for each cluster, which simulates the potential clusters in other conditions belonging to the same person.
%
Currently, the cloth augmentation methods we use are targeted for silhouette datasets, which the majority of algorithms work on.
%
We randomly dilate or erode the upper/bottom/whole body in the whole sequence with a probability of 0.5\footnote{The kernel size for upper part: $5 \times 5$, lower part: $2 \times 2$}, forming six cloth augmentation types.
%
Also, the upper/middle/bottom has a dynamic edited boundary\footnote{The boundary selected from upper bound: [14, 18], middle bound: [38, 42], bottom bound:[60, 64] for $64 \times 64$ silhouettes}, adding more variance to the augmentation results.
%
Here we visualize some cloth augmented results in Figure~\ref{fig:DA}. When dilating NM, the sequence can simulate its corresponding CL condition, and when eroding CL, the appearance of subjects can be regarded as in NM condition.
\begin{figure}[ht]
	\centering	 
	\includegraphics[width=\linewidth]{IEEEtran/figures/DA.pdf}	 	
	\caption{The visualization of data augmentation on NM and CL conditions. Cloth Augmentation can simulate the potential appearance in different conditions of the same person.}
	\label{fig:DA}
\end{figure}
It is indeed that in real-world cloth-changing situations, the clothes have more diversity, only dilating or eroding cannot fully simulate all the situations.
%
Currently, we first consider the simple cloth-changing situation that walking with or without coats, which is also the cloth-changing method in CASIA-B~\cite{yu2006framework} and Outdoor-Gait~\cite{song2019gaitnet} dataset, to prove our method is valid.
%
In real application, automatic cloth augmentation methods can be employed, to automatically search other cloth augmentation methods, such as sheer the bottom part of the silhouettes to simulate wearing a dress and adding an oval above the head to simulate wearing a hat, which is another promising research direction in the future research.
%
In this work, we use clothing augmentation via dilation or erosion in our method to provide the opportunity to let the cross-cloth sequences have the chance to be closer to each other. 
%
This will facilitate our method of utilizing the close chance to further pull the clusters near through the support set and multi-cluster update strategy.

%
%----------------------------------------
\subsubsection{Selective Cluster Fusion}
SCF aims to pull clusters in different clothes belonging to the same subject closer.
%
It comprises two parts, a support set selection, which is used to generate a support set $\mathcal{S}_k$ for each cluster, aiming to find potential candidate clusters in different clothes, and a multi-cluster update strategy, aiming to decrease the distance between candidate clusters in the support set.
%

\textbf{Support Set:} The input of the support set selection module is the centroids of $\mathcal{C}_{ak}$ and $\mathcal{C}_{k}$.
%
By calculating the similarity between the $k$-th Cloth Augmented centroid $\mathcal{C}_{ak}$ with all the original centroids contained in $\mathcal{M}$, we can get a rank list with these pseudo labels, ranging from highest to lowest according to their similarity distances.
%
We select the top $a$ ids in each rank list, and the first id we set is the cluster itself, formulating the support set $\mathcal{S}_k = {id_0,id_1,...,id_a}$.
%
With the support set, we can concretize the optimization direction when pulling NM and CL together because $f_{i_{CA}}$ can be seen as the cross-cloth sequences in reality to some extent.
%
With the explicit regulation, we will not blindly pull a cluster close to any near neighbor.

\textbf{Multi-cluster update strategy:} When updating the memory bank during backpropagating, we use the support set in the multi-cluster update strategy.
%
Knowing which clusters are the potential conditions of one person, the new strategy can be formulated as follows:
\begin{equation}
\label{momentum}
   \forall \mathcal{C}_{ak} \in \mathcal{S}_k, \forall q \in \mathcal{C}_{ak}, \mathcal{C}_{ak} \leftarrow m \mathcal{C}_{ak} + (1-m) q
\end{equation}
All the candidate clusters in $\mathcal{S}_k$ need to participate in updating the memory bank.
%
By forcing the potential conditions to fuse, we can make the mini-batch influence clusters and clusters in the support set, compressing the distance between cross-cloth pairs.

% % In Equation~\ref{momentum}, $m$ serves as the momentum updating factor, and its value significantly affects the coherence between cluster features and the most recently updated query instance feature. In prior ReID-related work~\cite{dai2021cluster}, a relatively small fixed value was employed, leading to overly rapid updates of cluster centroids and compromising the overall consistency of the clusters. To address this issue, we adaptively use a cosine annealing strategy to update $m$. The new strategy can be formulated as follows:
% % \begin{equation}
% % \label{momentum_strategy}
% %    m_t = m_{\text{min}} + \frac{1}{2} (m_{\text{max}} - m_{\text{min}}) \left(1 + \cos\left(\frac{t \pi}{T}\right)\right)
% % \end{equation}
% % where $m_t$ is the momentum at training step $t$, $m_{\text{min}}$ is the minimum momentum value, $m_{\text{max}}$ is the initial value, and $T$ represents the total number of training steps within a single epoch. As training progresses ($t$ increases), the momentum decreases following a cosine curve. We include the value of $m_{\text{max}}$ and $m_{\text{min}}$  in the supplementary materials.
% }


%----------------------------------------
\subsubsection{Selective Sample Fusion}

Seeing that walking postures are absent in front/back views, sequences taken from these views have less feature similarity with features extracted from other views.
%
So they cannot be appropriately gathered into their clusters like other views, and tend to mix up with sequences of other identities captured from front/back views.
%
If we pull all the clusters towards their candidate clusters in the support set, those clusters mainly composed of sequences taken from front/back views will further gather with clusters with the same view condition, making the situation worse. 
%
To deal with clusters in this condition, we design SSF, in which we use curriculum learning to gradually re-assign pseudo labels to sequences in front/back views, forcing them to fuse with samples taken from other views progressively before conducting the SCF method.
%

\textbf{View classifier :} Specifically, we first train a view classifier on OU-MVLP, classifying whether the sequence is in the front/back view.
%
The view classifier structure we set is as same as the GaitSet structure we used, and we add the BNNeck behind. 
%
We assign the label 1 for sequences in $0^{\circ}/180^{\circ}$ and assign the label 0 for sequences in other views to train the view classifier. 
%
We train the view classifier on OU-MVLP to gain view knowledge, and when we have prior knowledge of sequences' view, we can quickly identify which clusters generated by our framework are composed of sequences taken from front/back views.
%

Indeed, it is true that when adapting the view classifier to other datasets, there may be appearance domain gaps and view classification gaps between different datasets, as other datasets may not label views in the same way as OUVMLP, and some datasets may not provide view labels at all.
%
However from our observations, we found the view classification task is not a hard task, the view classifier can already identify most sequences with fewer walking postures.
%
To further solve the problem that some sequences haven't appropriately assigned the right view label, we set a threshold to relax the requirement when clustering.
%
The criterion is that only when the number of sequences in the front/back view in one cluster is larger than the threshold $c_{low}$, we consider the cluster as Front/Back View Clusters (FVC).
%
%train:0.8865567216391804
%test:0.8850931677018633
By dissolving FVC, we calculate the similarity between each sequence in it with other centroids of non-FVC, and if the similarity is larger than $s_{c}$, we re-assign the nearest non-FVC pseudo label for the sequence.
%
Therefore, even if not all sequences with front or back views are correctly identified, aligning most of these sequences closer to other views can help reduce the feature disparity between these perspectives and other views.
%
And with the training process of curriculum learning, the clusters are tighter. The sequences identified with front/back views will also push the sequences that were misidentified closer to their cluster center automatically since they have appearance similarity. 
%
More intelligent methods can be developed in future research to reduce the dependence of the model on the prior information.

\textbf{Curriculum Learning:} However, we do not incorporate all the sequences in FVC at the same time, instead, we utilize Curriculum Learning~\cite{bengio2009curriculum} to fuse them progressively by enlarging the $s_{c}$ during each epoch:
%
\begin{equation}
    s_{c} = s_{o} - \lambda \times epoch\_number
\end{equation}
%
At the first epoch, $s_{o}$ is high, and only the similarity between sequences in FVC and centroids of other non-FVC higher than $s_{i}$, can be assigned new pseudo labels. 
%
Otherwise, they will be seen as outliers and cannot participate in training.
%
During training, the criterion gradually increases with a speed of $\lambda$, which allows for the gradual incorporation of new knowledge.
%
% To tighten the clusters and make more samples can be fused, we adopt a Center Loss:
% \begin{equation}
%     \mathcal{L} _C=\sum_{i=1}^{batch}{ \| f_{i}-c_{y'_i} \| _{2}^{2}}
% \end{equation}
% $f_i$ is the sequence feature in batch, $c_{y'_i}$ is the centroids of its corresponding pseudo cluster.
%
In response to this, we propose to set $\lambda$ adaptively~\cite{hou2019learning}:
\begin{equation}
\lambda=\lambda_{base} \left| \frac{\mathcal{C} _n}{\mathcal{C} _o} \right|
\end{equation}
where $\left|\mathcal{C}_n\right|$ and $\left|\mathcal{C}_o\right|$ is the number of new and old clusters in each epoch, $\lambda_{base}$ is a fixed constant for each dataset. 
%
Since more clusters are fused in each phase, $\lambda$ increases as the ratio of the number of new clusters to that of old clusters increases.
%


%----------------------------------------
\subsection{Training Strategy}
Overall, Selective Fusion can make separated conditions and scattered sequences taken from front/back views fuse tighter.
%
Here we show our training strategy and represent the feature distribution of each training phase in Figure~\ref{fig:stage}.
%
Our training strategy encompasses three stages.
%
At first, the features extracted by the pre-trained model have a tendency to cluster together, making it hard to distinguish between them.
%
We adopt our baseline to separate these sequences with a strict criterion, making each cluster gathered according to their similarity.
%
Second, we apply Selective Fusion to fuse different conditions of the same person and gradually merge sequences in front/back views with sequences in other views.
%
Finally, we get the clusters with all the cloth conditions and views.
%
% The total loss is summarized as:
% \begin{equation}
%     \mathcal{L} = \mathcal{L}_{CNCE} + \lambda_c \mathcal{L}_{C}
% \end{equation}
% $\lambda_c$ is used to control the weight of $\mathcal{L}_{C}$.

\begin{figure}[htbp]
	\centering	 
	\includegraphics[width=\linewidth]{IEEEtran/figures/multi-stage.pdf}	 	
	\caption{The three stages in our training strategy. First, with narrowed features extracted by the pre-trained model, we first adopt our baseline to separate them further. Then Selective Fusion is used to fuse matched clusters and samples together. Finally, we gain clusters with different clothes and views. The base is the \textit{Baseline}. Each type in a different color indicates \textit{each subject in a different cloth condition.} }
	\label{fig:stage}
\end{figure}

%%%%%%%%%%%%%%%%%%%%%%%%%%%%%%%%%%%%%%%%%%%%%%%%%%%%%%%%%%%%%%%%%%%%%%%%%%%%%%%%%%
\begin{table*}[h]
\centering
\caption{The parameters used in unsupervised learning on OU-MVLP, CASIA-BN, Outdoor-Gait dataset, and GREW.}
\setlength{\tabcolsep}{1pt}
%\resizebox{\columnwidth}{!}{%
\begin{tabular}{ccccccc}
\hline
Param                           & Backbone  &OU-MVLP & CASIA-BN                 & Outdoor-Gait   &GREW  \\  
\hline
\multirow{2}{*}{Model Channel}  & GaitSet   & -        & (32, 64, 128)(128, 256)  & (32, 64, 128)(128, 256) &(32, 64, 128,256)(256, 256)\\
                                & GaitGL    & -        & (32, 64, 128)(128, 128)  & -                       &- \\
Batch Size                      & Both      & (32, 16) & (8, 16)                  & (8, 8)                  & (8,16)                \\
Weight Decay                    & Both      & 5e-4     & 5e-4                     & 5e-4                    &5e-4                  \\
Start LR                        & Both      & 1e-1     & 1e-4                     & 1e-4                    &1e-4                \\
Milestones                      & Both      & -        & {[}3.5k, 8.5k{]}         & {[}3.5k, 8.5k{]}        & {[}3.5k, 8.5k{]}      \\
Epoch                           & Both      & -        & Baseline: 50, SF: 50     & Baseline: 50, SF: 50    & Baseline: 50, SF: 50  \\
Iteration                       & Both      & -        & Baseline: 50, SF: 100    & Baseline: 50, SF: 100   &Baseline: 1000, SF: 1000 \\
\multirow{2}{*}{\begin{tabular}[c]{@{}l@{}}Upper bound Milestones\end{tabular}} 
                                 & GaitSet  &{[}50k, 100k, 125k{]}      & {[}20k, 40k, 60k, 80k{]} & {[}10k, 20k, 30k, 35k{]} & {[}10k, 20k, 30k, 35k{]}\\
                                & GaitGL    & {[}150k, 200k, 210k{]}    & {[}70k, 80k{]}           & -  &-\\ \hline
\end{tabular}%
%}
\label{unsupervise_param}
\end{table*}
\section{Experiments}
%
Our methods can be employed in appearance-based methods. 
%
For simplicity, we take silhouette sequences as input since they are more robust when datasets are collected in the wild.
%
To demonstrate the effectiveness of our framework, we apply our methods to two existing backbones: GaitSet~\cite{chao2019gaitset} and GaitGL~\cite{lin2021gait} to help them train with unlabeled datasets. 
%
We also compare our method with upper bound which is trained with the ground truth label, and with the baseline, which is also trained without supervision.
%
All methods are implemented with PyTorch~\cite{paszke2019pytorch} and trained on TITAN-XP GPUs.
%In this section, 

%%%%%%%%%%%%%%%%%%%%%%%%%%%%%%%%%%%%%%%%%%%%%%%%%%%%%%%%%%%%%%%%%%%%%%%%%%%%%%%%%%
\subsection{Datasets} 
% Here we first pre-train backbones on a large gait recognition dataset OU-MVLP~\cite{takemura2018multi}. 
% %
% Most of the previous unsupervised ReID methods use ResNet-50~\cite{he2016deep} pre-trained on ImageNet~\cite{deng2009imagenet} as the pre-train model, while in gait recognition, there does not exist any pre-train model with strong generalization.
% %
% Fortunately, OU-MVLP contains a large number of subjects with 14 views, which is an ideal dataset for pre-train models.
% %
% We can train backbones on it to gain preliminary information to classify subjects, and with the large dataset volume, the model can be more generalized when adapted to other datasets. 
% %
% However, without cross-cloth pairs in OU-MVLP, the model could only gain cross-view ability.
% %
% So, we need to develop methods to help the model recognize cross-cloth pairs.
% %
% We load the pre-trained model and evaluate the performance of our method on three popular datasets, CASIA-BN~\cite{yu2006framework}, Outdoor-Gait~\cite{song2019gaitnet} and GREW~\cite{lin2014effects}.}
We train and test the performance of our method on three popular datasets, CASIA-BN~\cite{yu2006framework}, Outdoor-Gait~\cite{song2019gaitnet} and GREW~\cite{lin2014effects}.
\subsubsection{CASIA-BN} 
%
The original CASIA-B is a useful dataset with both cross-view and cross-cloth sequence pairs. 
%
It consists of 124 subjects, having three walking conditions: \textbf{normal walking} (NM\#01-NM\#06), \textbf{carrying bags} (BG\#01-BG\#02), and \textbf{walking with different coats} (CL\#01-CL\#02).
%
Each walking condition contains 11 views distributed in $[0^{\circ},180^{\circ}]$. We employ the protocol in the previous research~\cite{chao2019gaitset,fan2020gaitpart}.
% taking 74 subjects as the training dataset and 50 subjects as the testing dataset. 
%
During the evaluation, NM\#01-NM\#04 are the gallery, NM\#05-NM\#06, BG\#01-BG\#02, CL\#01-CL\#02 are the probe.
%
Due to the coarse segmentation of CASIA-B, we collected some pedestrian images and trained a new segmentation model to re-segment CASIA-B, and gain CASIA-BN.

\subsubsection{Outdoor-Gait}
This dataset only has cross-cloth sequence pairs.
%
With 138 subjects, Outdoor-Gait contains three walking conditions: normal walking (NM\#01-NM\#04), carrying bags (BG\#01-BG\#04), and walking with different coats (CL\#01-CL\#04). 
%
There are three capture scenes (Scene\#01-Scene\#03), however, each person only has one view ($90^{\circ}$).
%
69 subjects are used for training and the last 69 subjects for tests.
%
During the test, we use NM\#01-NM\#04 in Scene\#03 as a gallery and all the sequences in Scene\#01-Scene\#02 as probes in different conditions.
\subsubsection{GREW}
To the best of our knowledge, 
%
GREW~\cite{lin2014effects} is the largest gait dataset in real-world conditions. The raw videos are gathered from 882 cameras in a vast public area, encompassing nearly 3,500 hours of 1,080×1,920 streams. Alongside identity information, some attributes such as gender, 14 age groups, 5 carrying conditions, and 6 dressing styles have been annotated as well, ensuring a rich and diverse representation of practical variations.
%
Furthermore, this dataset includes a train set with 20,000 identities and 102,887 sequences, a validation set with 345 identities and 1,784 sequences, and a test set with 6,000 identities and 24,000 sequences. In the test phase, we strictly follow the official test protocols.
%2 sequences per subject are treated as probes and another 2 sequences as the gallery.
%This evaluation strictly follows the official test protocols.
%----------------------------------------
% Please add the following requiBrickRed packages to your document preamble:
% \usepackage{multirow}
\setlength{\tabcolsep}{5pt}
% Please add the following requiBrickRed packages to your document preamble:
% \usepackage{multirow}
% Please add the following requiBrickRed packages to your document preamble:
% \usepackage{multirow}
% Please add the following requiBrickRed packages to your document preamble:
% \usepackage{multirow}
\begin{table*}[ht]
\centering
\caption{The rank-1 accuracy (\%) on CASIA-BN for different probe views excluding the identical-view cases. For evaluation, the sequences of NM\#01-NM\#04 for each subject are taken as the gallery. The probe sequences are divided into three subsets according to the walking conditions (\textit{i.e.} NM, BG, CL). SF is our \textit{Selective Fusion Method}. CC is the \textit{Cluster-contrast framework we followed}. \textcolor{BrickRed}{Red} indicates the upper bound with supervised learning. \textcolor{RoyalBlue}{\textbf{Blue}} indicates the improvements on sequences in front/back views. \textbf{Bold} indicates the total improvements on different conditions.}
\begin{tabular}{ccccccccccccccc}
\toprule
\multirow{2}{*}{Backbone} & \multirow{2}{*}{Condition} & \multirow{2}{*}{Method} & \multicolumn{11}{c}{Probe View} & \multirow{2}{*}{Average} \\ \cline{4-14}
 &  &  & $0^{\circ}$ & $18^{\circ}$ & $36^{\circ}$ & $54^{\circ}$ & $72^{\circ}$ & $90^{\circ}$ & $108^{\circ}$ & $126^{\circ}$ & $134^{\circ}$ & $162^{\circ}$ & $180^{\circ}$ &  \\ \midrule
\multirow{12}{*}{GaitSet} & \multirow{4}{*}{NM} & Upper & 90.5 & 98.1 & 99.0 & 96.9 & 93.5 & 91.0 & 94.9 & 97.8 & 98.9 & 97.2 & 83.4 & \textcolor{BrickRed}{94.7} \\
 &  & Pretrain & 59.6 & 72.9 & 80.1 & 77.4 & 67.4 & 58.8 & 63.9 & 72.4 & 79.2 & 62.2 & 42.7 & 67.0 \\
 &  & Base (CC) & 77.7 & 92.0 & 94.7 & 92.8 & 88.4 & 83.8 & 86.2 & 91.2  & 93.0 & 90.5 & 69.4 & 87.2  \\
 &  & Ours (SF) & \textcolor{RoyalBlue}{\textbf{85.2}} & 93.6 & 96.4 & 93.8 & 90.0 & 84.6 & 89.6 & 92.3 & 96.9 & 93.2 & \textcolor{RoyalBlue}{\textbf{77.4}} & \textbf{90.3} \\ \cline{2-15} 
 
 & \multirow{4}{*}{BG} & Upper & 86.1 & 94.1 & 95.9 & 90.7 & 84.2 & 79.9 & 83.7 & 87.1 & 94.0 & 93.8 & 78.0 & \textcolor{BrickRed}{88.0} \\
 &  & Pretrain & 48.0 & 51.9 & 58.1 & 54.2 & 52.2 & 45.8 & 47.6 & 49.9  & 53.5 & 45.5 & 36.4 & 49.2 \\
 &  & Base (CC) & 70.4 & 81.5 & 84.0 & 78.0 & 74.3  & 67.0 & 71.9 & 73.7  & 77.3 & 76.5 & 62.5 & 74.3 \\
 &  & Ours (SF) & \textcolor{RoyalBlue}{\textbf{78.5}} & \textcolor{RoyalBlue}{\textbf{88.3}} & 89.8 & 88.0 & 83.5 & 76.4 & 80.5 & 83.5 & 85.8 & \textcolor{RoyalBlue}{\textbf{84.8}} & \textcolor{RoyalBlue}{\textbf{72.6}} & \textbf{82.9} \\ \cline{2-15} 
 & \multirow{4}{*}{CL} & Upper & 65.2 & 79.3 & 84.4 & 81.0 & 77.9 & 74.1 & 75.7 & 79.2 & 81.5 & 73.2 & 47.5 & \textcolor{BrickRed}{74.5} \\
 &  & Pretrain & 9.8 & 10.7 & 14.4 & 17.3 & 16.1 & 13.6 & 15.2 & 14.8 & 13.1  & 7.8 & 6.8 & 12.7 \\
 &  & Base (CC) & 27.7 & 32.4 & 37.2 & 37.6 & 33.0 & 29.2 & 32.0 & 32.2  & 32.3 & 27.8 & 20.7 & 31.1 \\
 &  & Ours (SF) & \textcolor{RoyalBlue}{\textbf{33.1}} & \textcolor{RoyalBlue}{\textbf{41.6}} & 46.4 & 47.6 & 46.7 & 41.2 & 44.7 & 43.3 & 45.6 & \textcolor{RoyalBlue}{\textbf{35.8}} & 22.5 & \textbf{40.8} \\ \midrule

 
\multirow{12}{*}{GaitGL} & \multirow{4}{*}{NM} & Upper & 94.2 & 97.5 & 98.7 & 96.7 & 95.1 & 92.9 & 95.9 & 97.9 & 99.0 & 98.0 & 87.1 & \textcolor{BrickRed}{95.7} \\
 &  & Pretrain & 66.2 & 79.9 & 85.3 & 84.9 & 73.5 & 65.8 & 71.8 & 81.7 & 85.8 & 79.1 & 51.0 & 75.0 \\
 &  & Base (CC) & 83.7 & 94.7 & 96.4 & 93.2 & 88.2 & 85.1 & 87.6 & 91.7 & 95.5 & 94.3 & 70.8 & 89.2 \\
 &  & Ours (SF) & 84.2 & 95.7 & 96.2 & 94.7 & 89.9 & 87.0 & 89.0 & 91.8  & 96.7 & 93.2 & 63.6 & \textbf{89.3} \\ \cline{2-15} 
 & \multirow{4}{*}{BG} & Upper & 88.6 & 96.2 & 96.6 & 94.2 & 91.0 & 85.8 & 91.0 & 94.4 & 97.0 & 95.3 & 76.6 & \textcolor{BrickRed}{91.5} \\
 &  & Pretrain & 53.4 & 66.4 & 67.2 & 67.7 & 62.0 & 56.2 & 59.9 & 62.4 & 66.1 & 64.6 & 43.4 & 60.8 \\
 &  & Base (CC) & 74.6 & 88.7 & 87.6 & 85.0 & 83.2 & 79.3 & 80.1 & 83.0 & 87.4 & 86.9 & 63.0 & 81.7 \\
 &  & Ours (SF) & 75.8 & 90.1 & 91.6 & 87.5 & 83.9 & 80.2 & 83.0 & 85.0 & 89.7 & 87.9  & 56.0 & \textbf{82.8} \\ \cline{2-15} 
 & \multirow{4}{*}{CL} & Upper & 71.7 & 87.6 & 91.0 & 88.3 & 85.3 & 81.4 & 82.9 & 85.9 & 87.5 & 85.4 & 53.0 & \textcolor{BrickRed}{81.8} \\
 &  & Pretrain & 18.5 & 25.4 & 28.7 & 29.9 & 28.4 & 23.6 & 25.7 & 24.9 & 24.3 & 21.7 & 11.8 & 23.9 \\
 &  & Base (CC) & 33.7 & 49.6 & 55.0 & 55.1 & 58.2 & 53.4 & 57.3 & 52.6 & 52.2 & 41.7 & 24.1 & 48.4 \\
 &  & Ours (SF) & \textcolor{RoyalBlue}{\textbf{53.6}} & \textcolor{RoyalBlue}{\textbf{71.1}} & 76.9 & 75.5 & 72.9 & 69.1 & 69.8 & 69.0 & 71.7 & \textcolor{RoyalBlue}{\textbf{60.7}} & \textcolor{RoyalBlue}{\textbf{31.2}} & \textbf{65.6} \\ \bottomrule
\end{tabular}

\label{tab:dataset_casiabn}
\end{table*}



%
%----------------------------------------
\subsection{Implementation Details} 
In this section, we provide comprehensive details about the implementation of our method. 
%We begin with the pre-training on the OU-MVLP~\cite{takemura2018multi} dataset, followed by specific considerations for benchmark settings. 
% Further details, such as structure, optimization settings, are included in the supplementary materials. }
% \subsubsection{Pre-training on OU-MVLP}
In the Re-ID task~\cite{dai2021cluster}, ResNet-50~\cite{he2016deep}, pre-trained on ImageNet~\cite{deng2009imagenet}, is commonly used as the backbone for feature extraction.
Similar to unsupervised Re-ID methods, a unified pre-trained model is crucial for initialization in UGR. However, there currently does not exist a unified pre-trained model that can be used for the gait community. Here we first pre-train backbones on a large gait recognition dataset OU-MVLP~\cite{takemura2018multi}, and then load it when training on the unlabeled dataset. OU-MVLP, with its large dataset volume, is an ideal dataset for pre-train models. It contains 10,307 subjects. The sequences of each subject distributed in 14 views between $[0^{\circ},90^{\circ}]$ and $[180^{\circ},270^{\circ}]$, but only have normal walking conditions. We used 5153 subjects to pre-train backbones.

It is worth noting that the OU-MVLP dataset is only used in the pre-training stage and is not involved in the unsupervised learning stage. There are no cloth-changing pairs in this dataset, preventing the transfer of cross-cloth information. And our method focuses on tackling the challenging cross-cloth task by clustering features and learning intrinsic information based on pseudo labels, without relying on any prior information from the pre-trained model. As a result, we abbreviate our method as Unsupervised Gait Recognition rather than a domain transformation.   Moreover, we believe that training a unified pre-trained model for the gait community is a promising direction for future work.

% Since there are no sequences involving walking with different clothes, the model trained on OU-MVLP does not have much cross-cloth ability.}

% It is worth noting that our method focuses on clustering the features and learning the interior correlations of clusters based on pseudo labels. We do not specifically focus on learning the domain-invariant features or addressing domain shift in particular.
% Also, unlike the Re-ID task, there does not exist a unified pre-trained model like ResNet-50~\cite{he2016deep} pre-trained on ImageNet~\cite{deng2009imagenet} that can be used for the gait community, which is another promising future research direction. Our method is built on a model pre-trained on OU-MVLP. The OU-MVLP dataset is only used in the pre-training stage and is not involved in the unsupervised learning stage.Since no large dataset can be used to learn cross-cloth information, our method mainly focuses on improving the cross-cloth performance. We achieve this without relying on any prior information from the pre-trained model, making it an unsupervised learning task rather than a domain transformation.}
%
% We can train backbones on it to gain preliminary information to classify subjects, and with the large dataset volume, the model can be more generalized when adapted to other datasets. 
% %
% However, without cross-cloth pairs in OU-MVLP, the model could only gain cross-view ability.
% %
% So, we need to develop methods to help the model recognize cross-cloth pairs.
% %
% % We load the pre-trained model and evaluate the performance of our method on three popular datasets, CASIA-BN~\cite{yu2006framework}, Outdoor-Gait~\cite{song2019gaitnet} and GREW~\cite{lin2014effects}.}


% OU-MVLP~\cite{takemura2018multi} contains 10,307 subjects. 
% %
% The sequences of each subject distributed in 14 views between $[0^{\circ},90^{\circ}]$ and $[180^{\circ},270^{\circ}]$ but only have normal walking conditions (00, 01). 
% %
% We used 5153 subjects to pre-train backbones.
% %
% In particular, 00 is used as the gallery, and 01 is taken as the probe.
% %
% Without walking with different clothes sequences, the model trained on OU-MVLP does not have much cross-cloth ability, so we need to develop specific methods to fuse cloth-changing pairs.

% We include details such as the structure, optimization settings, and hyperparameter configurations in the supplementary materials.}
% \subsubsection{Hyper-Parameters Setting} 
In Table~\ref{unsupervise_param}, we present the structure and optimization settings used for pre-training and unsupervised learning. Recognizing individuals on GREW proves more challenging due to its larger scale and diverse, real-world settings. Consequently, we employ convolutional layers with increased channel sizes (32, 64, 128, 256). For hyperparameters, we set $s_{up}=0.7$ for the baseline and $s_{up}=0.3$ for Selective Fusion to enlarge the boundary. The remaining parameters are set as $n=40$, $\tau=0.05$, $k=2$, $\lambda_{base}=0.005$, $c_{low} = 0.8$, and $s_{o} = 0.7$. Since the sequences in Outdoor-Gait are fewer, we change $n$ to 20 to avoid overfitting. The batch size is represented as $(B_S, B_T)$, where each mini-batch contains $B_S$ subjects. For each subject, $B_T$ sequences are sampled, and 30 frames are randomly selected from each sequence. Each frame is normalized to a size of $64\times44$. We use a cosine annealing strategy to update $m$ in Equation~\eqref{eq:momentum}, the strategy can be formulated as follows:
\begin{equation}
\label{momentum_strategy}
   m_t = m_{\text{min}} + \frac{1}{2} (m_{\text{max}} - m_{\text{min}}) \left(1 + \cos\left(\frac{t \pi}{T}\right)\right)
\end{equation}
where $m_t$ is the momentum at training step $t$, $m_{\text{min}}$ is the minimum momentum value, $m_{\text{max}}$ is the initial value, and $T$ represents the total number of training steps within a single epoch. As training progresses ($t$ increases), the momentum decreases following a cosine curve. We set $m_{\text{max}}=0.5$ and $m_{\text{min}}=0.1$.
%

%
\subsection{Benchmark Settings} To show the effectiveness, we define several benchmarks:

(1) \textit{Upper}. The upper bound reports the performance of each backbone trained with ground-truth labels.

(2) \textit{Pre-train}. The effect when directly applying the pre-trained model to the target dataset without fine-tuning.

(3) \textit{Base (CC)}. Fine-tuning the pre-trained model with our baseline framework implemented by Cluster-contrast.

(4) \textit{Ours (SF)}. The results of our proposed method.

\subsection{Performance Comparison}
% 
% Before training and testing, we first pre-train backbones on OU-MVLP.
% %
% The effect of the pre-trained model on the OU-MVLP test dataset with GaitSet~\cite{chao2019gaitset} backbone is 77.9, and with GaitGL~\cite{lin2021gait} backbone is 79.1 on rank-1 accuracy of the NM condition. 
% %
% % Training on OUMVLP can make the model gain cross-view knowledge but cannot achieve cross-cloth information.
% %
% It's essential to note that the OU-MVLP dataset is only used in the pre-training stage and is not involved in the unsupervised learning stage.
% So the comparison results can show that our unsupervised method boosts the cross-view performance further and produces decent cross-cloth results.}

%----------------------------------------
\subsubsection{CASIA-BN}
The performance comparison on CASIA-BN is shown in Table~\ref{tab:dataset_casiabn}. 
%
We evaluate the probe in three walking conditions separately. 
%
Since our method aims to improve the rank-1 accuracy of CL and sequences in front/back views, \textbf{ we take the accuracy for CL as the main criteria}.
%
From the results, we can see that our method outperforms the baseline in the CL condition by a remarkable margin (GaitSet: CL + 9.7\%; GaitGL: CL + 17.2\%).
%
It indicates that our Selective Cluster Fusion method can properly identify the potential clusters of the same person with different cloth conditions and pull them together.
%
Moreover, sequences in $0^{\circ}/18^{\circ}/162^{\circ}/180^{\circ}$ also gained large improvement in both cloth conditions.
%
Selective Sample Fusion can gradually gather individual front/back samples that were excluded initially, by assigning them the same pseudo labels as the sequences in other views.
%
Although lacking walking postures, the sequences with front/back views can still provide useful information for identifying a particular person. 
%
It should be noted that the hyper-parameters used for GaitGL are as same as GaitSet and without specific adjustment, which shows the generalization of our method when applying to different backbones. 
%
Both cues indicate that our method is effective when dealing with cloth-changing and front/back views.

%

%----------------------------------------
\subsubsection{Outdoor-Gait}
Although Outdoor-Gait does not consider cross-view data pairs, we can still verify the SCF method on this dataset and show the result with the GaitSet backbone in Table~\ref{tab:outdoor_gait}.
%
% Please add the following requiBrickRed packages to your document preamble:
% \usepackage{multirow}
% Please add the following requiBrickRed packages to your document preamble:
% \usepackage{multirow}

% \setlength{\tabcolsep}{12pt}
% \begin{table}[h]
% \centering
% \caption{The Rank-1 accuracy (\%) on Outdoor-Gait. When evaluation, we take NM\#1-NM\#4 in Scene\#3 as gallery and others as probe.}
% \begin{tabular}{ccccc}
% \hline
% Backbone & Method & NM & BG & CL \\ \hline
% \multirow{4}{*}{GaitSet} & Upper & \textcolor{BrickRed}{97.6} & \textcolor{BrickRed}{90.9} & \textcolor{BrickRed}{90.4} \\
%  & Pretrain & 45.8 &  46.4 & 43.3 \\
%  & Base (CC) & 84.8 & 66.5 & 62.9 \\
%  & Ours (SCF) & \textbf{89.1} & \textbf{73.6} & \textbf{71.9} \\ \hline
% \end{tabular}
% \label{tab:outdoor_gait}
% \end{table}
\setlength{\tabcolsep}{12pt}
\begin{table}[h]
\centering
\caption{The Rank-1 accuracy (\%) on Outdoor-Gait. When evaluation, we take NM\#1-NM\#4 in Scene\#3 as gallery and others as probe.}
\begin{tabular}{ccccc}
\hline
Backbone & Method & NM & BG & CL \\ \hline
\multirow{4}{*}{GaitSet} & Upper & \textcolor{BrickRed}{97.6} & \textcolor{BrickRed}{90.9} & \textcolor{BrickRed}{90.4} \\
 & Pretrain & 45.8 &  46.4 & 43.3 \\
 & Base (CC) & 84.8 & 66.5 & 62.9 \\
 & Ours (SF) & \textbf{90.0} & \textbf{73.7} & \textbf{71.9} \\ \hline
\end{tabular}
\label{tab:outdoor_gait}
\end{table}
%
\begin{table}[t]
\centering
\caption{The Rank-1 and Rank-5 accuracy (\%) on GREW. Trained on the GREW train set and evaluated on the test set.}
\setlength{\tabcolsep}{12pt}
\begin{tabular}{ccccc}
\hline
Backbone & Method &Rank-1 &Rank-5 \\ \hline
\multirow{4}{*}{GaitSet} & Upper & \textcolor{BrickRed}{48.4} & \textcolor{BrickRed}{63.6}  \\
 & Pretrain & 17.0 &  28.5  \\
 & Base (CC) & 18.3 & 30.4  \\
 & Ours (SF) & \textbf{20.2} & \textbf{32.0}\\ \hline
\end{tabular}
\label{tab:grew}
\end{table}
%----------------------------------------
% Please add the following required packages to your document preamble:
% \usepackage{multirow}
% \usepackage{graphicx}
\begin{table*}[h]
\centering
\caption{Ablation study demonstrating the effectiveness of each module in our method on the CASIA-BN, Outdoor-Gait, and GREW datasets based on Rank-1 accuracy (\%)}
\setlength{\tabcolsep}{12pt}
\label{tab:each}
\begin{tabular}{cccccccc}
\hline
\multirow{2}{*}{Setting} & \multicolumn{3}{c}{CASIA-BN}                  & \multicolumn{3}{c}{Outdoor-Gait}              & \multirow{2}{*}{GREW} \\
                         & NM            & BG            & CL            & NM            & BG            & CL            &                       \\ \hline
Base                     & 87.2          & 74.3          & 31.1          & 84.8          & 66.5          & 62.9          & 18.3                  \\
Base + SSF               & 87.5          & 75.8          & 32.8          & 85.3          & 66.9          & 61.6          & 18.7                  \\
Base + SCF               & 83.3          & 75.9          & 39.9          & 89.1          & 73.6          & 71.9          & 19.7                  \\
Ours                     & \textbf{90.3} & \textbf{82.9} & \textbf{40.8} & \textbf{90.0} & \textbf{73.7} & \textbf{71.9} & \textbf{20.2}         \\ \hline
\end{tabular}%
\end{table*}
%
%
\begin{table*}[h]
\centering
\caption{The rank-1 accuracy (\%) on CASIA-BN for different probe views excluding the identical-view cases. The probe sequences are divided into three subsets according to the walking conditions (\textit{i.e.} NM, BG, CL).}
\setlength{\tabcolsep}{9pt}
\resizebox{\textwidth}{!}{%
\begin{tabular}{cccccccccccccc}
\toprule
\multirow{2}{*}{Backbone} & \multirow{2}{*}{Condition} & \multirow{2}{*}{Method} & \multicolumn{10}{c}{Probe View}  \\ \cline{4-14} %%\multirow{2}{*}{Average}
 &  &  & $0^{\circ}$ & $18^{\circ}$ & $36^{\circ}$ & $54^{\circ}$ & $72^{\circ}$ & $90^{\circ}$ & $108^{\circ}$ & $126^{\circ}$ & $134^{\circ}$ & $162^{\circ}$ & $180^{\circ}$  \\ \midrule
\multirow{12}{*}{GaitSet} & \multirow{4}{*}{NM}& Base  & 77.7 & 92.0 & 94.7 & 92.8 & 88.4 & 83.8 & 86.2 & 91.2  & 93.0 & 90.5 & 69.4  \\ %& 87.2 
 &  & Base+SSF & \textcolor{RoyalBlue}{\textbf{80.1}} & \textcolor{RoyalBlue}{93.2} & 94.9 & 92.9 & 87.6 & 83.4 & 86.4 & 91.1 & 92.9 & \textcolor{RoyalBlue}{90.7} & \textcolor{RoyalBlue}{\textbf{69.9}}  \\ 
 &  & Base+SCF & 70.1 &89.3 &91.9 &91.8 &83.7 &82.1 &82.3 &90.8 &90.5 &81.6 &62.2   \\
&  & Ours (SF) & \textcolor{RoyalBlue}{\textbf{85.2}} & 93.6 & 96.4 & 93.8 & 90.0 & 84.6 & 89.6 & 92.3 & 96.9 & 93.2 & \textcolor{RoyalBlue}{\textbf{77.4}} \\ \cline{2-14} 
 
 & \multirow{4}{*}{BG} & Base & 70.4 & 81.5 & 84.0 & 78.0 & 74.3  & 67.0 & 71.9 & 73.7  & 77.3 & 76.5 & 62.5  \\%& 74.3
 &  & Base+SSF & \textcolor{RoyalBlue}{\textbf{72.9}} & \textcolor{RoyalBlue}{\textbf{84.0}} & 83.6 & 77.9 & 74.5 & 67.7 & 70.8 & 73.2 & 79.5 & \textcolor{RoyalBlue}{\textbf{78.1}} & \textcolor{RoyalBlue}{\textbf{71.3}} \\
 &  & Base+SCF &69.3 &79.5 &87.7 &79.8 &74.4 &74.5 &74.6 &77.7 &79.1 &76.1 &61.9 \\
 &  & Ours (SF) & \textcolor{RoyalBlue}{\textbf{78.5}} & \textcolor{RoyalBlue}{\textbf{88.3}} & 89.8 & 88.0 & 83.5 & 76.4 & 80.5 & 83.5 & 85.8 & \textcolor{RoyalBlue}{\textbf{84.8}} & \textcolor{RoyalBlue}{\textbf{72.6}} \\ \cline{2-14}  %& \textbf{82.9} \
 & \multirow{4}{*}{CL} & Base  & 27.7 & 32.4 & 37.2 & 37.6 & 33.0 & 29.2 & 32.0 & 32.2  & 32.3 & 27.8 & 20.7  \\%& 31.1
 &  & Base+SSF & \textcolor{RoyalBlue}{\textbf{32.1}} & \textcolor{RoyalBlue}{\textbf{37.3}} & 39.1 & 37.7 & 37.6 & 29.5 & 30.5 & 31.7 & 32.5 & \textcolor{RoyalBlue}{\textbf{30.9}} & \textcolor{RoyalBlue}{\textbf{21.9}}  \\ 
 &  & Base+SCF & 30.2 &38.0 &46.5 &46.7 &45.9 &41.3 &44.8 &43.7 &45.1 &34.3 &22.0\\
 &  & Ours (SF) & \textcolor{RoyalBlue}{\textbf{33.1}} & \textcolor{RoyalBlue}{\textbf{41.6}} & 46.4 & 47.6 & 46.7 & 41.2 & 44.7 & 43.3 & 45.6 & \textcolor{RoyalBlue}{\textbf{35.8}} & \textcolor{RoyalBlue}{\textbf{22.5}}  \\ 
 %& \textbf{40.8}
 \bottomrule
 \end{tabular}
}
\label{tab:dataset_casiabn_seperate}
\end{table*}
%
We can see that SF surpasses the baseline on both conditions (NM + 4.3\%; BG + 7.1\%; CL + 9.0\%).
%
With the SF method, not only the accuracy of CL condition improved, but also the accuracy of NM, and BG, which means features from CL sequences can also provide useful information when recognizing a person, and they can not be neglected.
%
If the features before and after changing clothes are not correctly associated, the gait recognition model will miss important information, leading to insufficient learning and difficulty in distinguishing between different individuals.
%
However, due to the small dataset volume and lack of views in Outdoor-Gait, the upper bound with GaitGL backbone overfit\footnote{NM: 95.5, BG: 91.3, CL: 86.2 }, so we do not show the results with it.
%
\subsubsection{GREW}
To test the generalization of our method, we apply it to a large wild dataset GREW ~\cite{lin2014effects}. Our method excels in further separating the narrowed features within the feature space as comprehensively as possible. The results, as shown in Table~\ref{tab:grew}, illustrate that our proposed method outperforms the baseline under both conditions (Rank-1 + 1.9\%; Rank-5 + 1.6\%). This affirms that our Selective Fusion method can effectively guide the updating of clusters and finally get the clusters with different conditions. Additionally, it validates the generalization of our method in outdoor scenarios.
%
\begin{figure*}[t]
	\centering	 
	\includegraphics[width=\linewidth]{IEEEtran/figures/hyper-parameter.pdf}	 	
	\caption{The effect of hyper-parameters $s_{up}/n/\tau/m$ on baselines. In there we choose a set of hyper-parameters that have the best result in our experiments. Other hyper-parameters do not change the result a lot, just lead to sub-optimal.}
	\label{fig:hyperparameter}
\end{figure*}
\begin{figure*}[t]
	\centering	 
	\includegraphics[width=\linewidth]{IEEEtran/figures/tbiomtsne.pdf}	 
	\caption{(a) and (b) are the TSNE images of the baseline method and our method(SF), respectively. The text above each feature point shows the pseudo label, and blue text indicates that the sequence is in front/back view. A color in a feature point represents a type of condition. Our method effectively clusters features belonging to the same person. (c) shows the corresponding gait sequence from (b), where the dashed circles indicate examples of clustering errors, and the solid circles indicate correct clustering examples. Please zoom in to see the details.}
	\label{fig:TSNE1}
\end{figure*}
% In GREW, the collection only lasts for one day, resulting in a lack of cross-cloth variation. Although with this limitation, our method still has effective generalization in outdoor scenarios.}

% {To test the generalization of our method, we apply it to a large wild dataset GREW ~\cite{lin2014effects}. Our method excels in further separating the narrowed features within the feature space as comprehensively as possible. The results, as shown in Table~\ref{tab:grew}, illustrate that our proposed method outperforms the baseline under both conditions (Rank-1 + 2.55\%; Rank-5 + 3.25\%). This evaluation follows the protocols used in OpenGait~\cite{fan2022opengait}. During the test phase, using the sequence "00" as the probe and "01" as the gallery sequence for straightforward comparison.}

% We also present the results, strictly following the official test protocols, as shown in Table~\ref{tab:grew1}. We compare our method with the state-of-the-art method GaitSSB~\cite{fan2023learning} in an unsupervised manner. This affirms that our Selective Fusion method can effectively guide the updating of clusters and finally get the clusters with different conditions. Additionally, it validates the generalization of our method in outdoor scenarios. In GREW, the collection only lasts for one day, resulting in a lack of cross-cloth variation. 
% %
% Although with this limitation, our method still has effective generalization in outdoor scenarios.}
% under uncontrolled enviornment, however, we observed that in GREW, the data collection in the wild is usually aided by person
% Re-Identification, which is particularly hard to collect the cross-clothes sequences. In GREW, the collection only lasts for one day,
% and thus the cross-cloth variation is lacking, which is not suitable to verify our method, which mainly targets for solving cross-cloth task with unsupervised learning. Therefore, we only test the first stage of our method on GREW, which used to seperate the
% narrowed feature further in the feature space. Our results, presented in Table~\ref{tab:grew}, illustrate that our proposed method outperforms the baseline under both conditions (Rank-1 + 2.55\%; Rank-5 + 3.25\%), affirming the generalization of our method in outdoor scenarios.}
%
% Please add the following requiBrickRed packages to your document preamble:
% \usepackage{multirow}
% Please add the following requiBrickRed packages to your document preamble:
% \usepackage{multirow}
% \setlength{\tabcolsep}{12pt}
% \begin{table}[h]
% \centering
% \caption{Our method v.s. other unsupervised State-of-the-Art.}}
% \begin{tabular}{ccccc}
% \hline
% Backbone} & Method} &Rank-1} &Rank-5} \\ \hline
% GaitBase} & GaitSSB} & 16.60} &-}\\ \hline
% \multirow{2}{*}{GaitSet}} 
%  % & Pretrain} & 16.96} &  28.46}  \\
%  & Base (CC)} & 18.26} & 30.42}  \\
%  & Ours (SCF)} & \textbf{19.66}} & \textbf{32.10}} \\ \hline
% \end{tabular}
% \label{tab:grew1}
% \end{table}

%%%%%%%%%%%%%%%%%%%%%%%%%%%%%%%%%%%%%%%%%%%%%%%%%%%%%%%%%%%%%%%%%%%%%%%%%%%%%%%%%%
%----------------------------------------
\begin{table}[t]
\centering
\caption{The effect of cluster candidates number $k$ in support set.}
\setlength{\tabcolsep}{12pt}
\begin{tabular}{cccc}
\hline
Settings     & NM & BG & CL \\ \hline
Ours ($a=4$) & 89.2 & 81.2 & 35.1 \\
Ours ($a=3$) & 89.7 & 81.1 & 37.9   \\
Ours ($a=2$) & \textbf{90.3} & \textbf{82.9} & \textbf{40.8}  \\ \hline
\end{tabular}%
\label{tab:SC-Fusion}
\end{table}
%%
% Please add the following required packages to your document preamble:
%%
\begin{table}[t]
\centering
\caption{The performance of different kinds of rates when conducting curriculum learning.}
\setlength{\tabcolsep}{12pt}
\begin{tabular}{cccc}
\toprule
Settings & NM & BG & CL \\ \midrule
Ours ($\lambda_{base}$) & 90.3  & 82.7 & 40.7 \\
Ours ($\lambda$) & \textbf{90.3} & \textbf{82.9} & \textbf{40.8} \\ \bottomrule
\end{tabular}%
\label{tab:SS-Fusion}
\end{table}
%
%%%%%%%%%%%%%%%%%%%%%%%%%%%%%%%%%%%%%%%%%%%%%%%%%%%%%%%%%%%%%%%%%%%%%%%%%%%%%%%%%%
% \usepackage{multirow}
\begin{table}[h!]
\centering
\caption{Comparison of Rank-1 Accuracy (\%) between our method and GaitSSB* on the CASIA-BN and GREW datasets. GaitSSB* denotes the modified version of GaitSSB~\cite{fan2023learning} with an updated backbone.}
\label{tab:SF_gaitSSB}
\begin{tabular}{ccccc}
\hline
\multirow{2}{*}{Method} & \multicolumn{3}{c}{CASIA-BN}                                             & \multirow{2}{*}{GREW} \\
                         & NM            & BG            & CL                                 &                       \\ \hline
GaitSSB*                 & 43.0          & 30.2          & 31.1                               & 17.2                  \\
Ours                     & \textbf{90.3} & \textbf{82.9} & \multicolumn{1}{c}{\textbf{40.8}} & \textbf{20.2}         \\ \hline
\end{tabular}
\end{table}
%%%%%%%%%%%%%%%%%%%%%%%%%%%%%%%%%%%%%%%%%%%%%%%%%%%%%%%%%%%%%%%%%%%%%%%%%%%%%%%%%%
\subsection{Ablation Study}
\subsubsection{Impact of Each Component in Our Selective Fusion}
In this section, we show that both SCF and SSF are essential components of our Selective Fusion method through experiments on the indoor CASIA-BN dataset and the outdoor datasets, Outdoor-Gait and GREW.
%
In Table~\ref{tab:each}, we show the results only using SCF or SSF.
%
Only with SSF, the rank-1 accuracy for each condition in front/back view slightly improved, but still had a poor performance on the cross-cloth problem.
%
When directly applying SCF, clusters mainly composed of sequences in front/back views will also pull closer to other clusters in the same condition, which should be forbidden.
%
Pulling more FVCs closer will further make these sequences merge into their actual clusters with other views, degrading the performance in recognizing sequences with front/back views.
%
Therefore, the best effect can be achieved only when these two methods are effectively combined.
As shown in Table~\ref{tab:dataset_casiabn_seperate}, we demonstrate the impact of SSF on sequences taken from front/back views, which validates its effectiveness. The improvements on sequences with front/back views are highlighted in blue.
%
%Next, we will show the influence of parameters used in each method.
\subsection{Effects of Different Parameters in Baseline}
\label{hyper}
%
Here we research how hyper-parameters $s_{up}$, $n$, $\tau$, $m$ affect the results of baselines. 
%
We adjust one parameter at the one time and keep other hyper-parameters unchanged. 
%
$s_{up}$ regulates the boundary of how far the features can be gathered into one cluster. The smaller $s_{up}$ it is, the tighter the boundary.
%
$n$ is the number of neighbors KNN searched for each sequence.
%
$\tau$ is the temperature parameter in ClusterNCE loss, indicating the entropy of the distribution.
%
$m$, the momentum value, controls the update speed of centroids stored in the Memory Bank.
%
From the results on CASIA-BN, we can see that when $s_{up}=0.7$, $n=40$, $\tau=0.05$, $m=0.2$ we have the overall best results for NM, BG, and CL.
%
When these parameters deviate too much from the current setting, the performance is sub-optimal.
%
Here we show the accuracy of NM, BG, and CL when adopting different parameters in the baseline framework in Figure~\ref{fig:hyperparameter}.
% Please add the following required packages to your document preamble:
% \usepackage{graphicx}
% \begin{table}[h]
% \centering
% \setlength{\tabcolsep}{12pt}
% \caption{The effectiveness of each component in our framework. Combining both methods can maximize their effect.}
% \begin{tabular}{cccc}
% \hline
% Settings & NM & BG & CL \\ \hline
% Base & 87.2 & 74.3 & 31.1 \\
% Base + SSF & 87.5 & 75.8 & 32.8 \\
% Base + SCF & 83.3 & 75.9 & 39.9 \\
% Ours & \textbf{90.3} & \textbf{82.9} & \textbf{40.8} \\ \hline
% \end{tabular}%
% \label{tab:each}
% \end{table}
% Please add the following required packages to your document preamble:
% \usepackage{multirow}
%----------------------------------------
\subsubsection{Impact of Candidate Number in Support Set}
Here we discuss the effect of a parameter in SCF, the candidate number $a$, in the support set.
%
In Table~\ref{tab:SC-Fusion}, we can see that when $a=2$, we have the best performance, which is in line with the fact that the features of NM and BG are easily projected together in the feature space since they have larger similarity and we should drag the features of NM with CL specifically in CASIA-BN.
\subsubsection{Impact of Rate of Curriculum Learning in SSF}
We test the effect of SSF with a dynamic or constant rate when conducting curriculum learning on CASIA-BN. 
%
Without curriculum learning, linearly clustering the front/back view sequences with sequences in other views will degrade the performance.
%
In Table~\ref{tab:SS-Fusion}, with a dynamic pulling rate, we can relax the requirement when training the model, which can make the model learn from easy to hard better. 
\subsubsection{Instance vs. Cluster-based method}
To further validate the effectiveness of our method, we conducted extensive experiments on both the indoor CASIA-BN dataset and the outdoor GREW dataset, comparing our approach against GaitSSB*. For a fair comparison, we replaced the backbone of GaitSSB with GaitSet while keeping all other experimental settings unchanged. As shown in Table~\ref{tab:SF_gaitSSB}, our method consistently outperforms GaitSSB* on both datasets. GaitSSB is an instance-based contrastive learning method that constructs positive pairs at the instance level through data augmentation. However, due to the limitations of current data augmentation techniques for gait sequences, and more importantly, the positive pairs are drawn from the same sequence, resulting in positive sample pairs that are very similar to each other. This makes it difficult to simulate the variations caused by changes in clothing and camera viewpoints, limiting the ability to provide effective supervisory signals that can guide the model to learn robust features. 

In contrast, our method adopts a clustering strategy to group unlabeled data and generate pseudo-labels, allowing the model to learn representations based on cluster assignments. By integrating clustering strategies with selective fusion, our approach effectively addresses the practical challenges of unsupervised gait recognition. As shown in Table 9, the experimental results demonstrate that our method is more robust in handling complex scenarios, such as cross-clothing and cross-view variations.

\subsubsection{Impact of different training dataset scales}
We further conducted related experiments on CASIA-BN to verify the impact of different training dataset scales on our method in Table~\ref{tab:train_scale}. The CASIA-BN training set consists of 8,107 sequences. We randomly selected different training dataset scales (4,000, 6,000, and all) from the CASIA-BN training set to train our method. The performance of the models trained on different scales was evaluated using the standard test set. As shown in Table~\ref{tab:train_scale}, having more training data is beneficial, as a larger dataset enables the model to learn more generalizable features, ultimately improving recognition performance.

\begin{table}[h]
\centering
\caption{Randomly selecting different training dataset scales from the CASIA-BN training set to train our method, and evaluating the Rank-1 accuracy (\%)on the standard test set.}
\label{tab:train_scale}
\begin{tabular}{cccc}
\hline
\multirow{2}{*}{Training dataset Scale} & \multicolumn{3}{c}{CASIA-BN} \\
                                        & NM       & BG      & CL      \\ \hline
4000                                    & 86.9     & 75.8    & 33.6    \\
6000                                    & 89.0     & 79.1    & 37.7    \\
All                                     & 90.3     & 82.9    & 40.8    \\ \hline
\end{tabular}
\end{table}


\subsubsection{Visulization of Selective Fusion}
The visualization effect of Selective Fusion is shown in Figure~\ref{fig:TSNE1}. We selected a subject in CASIA-BN and found that in baseline, BG and CL have a different pseudo label from NM. 
%
At the same time, some sequences in front/back views of NM/BG/CL are also assigned different pseudo labels with sequences in other views.
%
With SF, as shown in Figure~\ref{fig:TSNE1}(b), most sequences from various views and conditions are assigned the same identity. As shown in Figure~\ref{fig:TSNE1}(c), we also visualized some cases of SF.
\section{Limitation and future work}
Our method can be employed with off-the-shelf backbones to train on a new, unlabeled dataset. However, there are still some limitations:
First, our approach relies on prior knowledge in certain areas. For instance, (1) we use knowledge of viewing angles to identify challenging samples with front/back views, and (2) our model requires well-initialized parameters to ensure reasonable clustering performance at the beginning of training. These dependencies on prior knowledge can impact the final accuracy. To overcome this, we need to explore methods that can achieve better unsupervised gait recognition without such reliance.

Additionally, data augmentation plays a crucial role in unsupervised gait recognition. However, our current augmentation method only simulates clothing variations in specific scenarios. To better capture real-world variations in gait sequences, it is necessary to develop more robust data augmentation techniques.

\section{Conclusion}
In this work, we propose a new task, Unsupervised Gait Recognition. 
%
We first design a new baseline with cluster-level contrastive learning.
%
We identified two key challenges in unsupervised gait recognition: (1) sequences with different clothing are not grouped into a single cluster, and (2) sequences captured from front/back views are difficult to merge with those from other views.
To address these challenges, we developed the Selective Fusion method, which includes Selective Cluster Fusion and Selective Sample Fusion. Selective Cluster Fusion helps cluster sequences of the same individual across different clothing conditions, while Selective Sample Fusion progressively merges sequences from front/back views with those from other views.

%
Our experiments demonstrate that the proposed method effectively improves accuracy under both clothing variation and front/back view conditions.
This work reduces reliance on labeled data, enabling us to learn high-quality feature representations from large-scale unlabeled datasets, thereby advancing the development of gait recognition.





% Can use something like this to put references on a page
% by themselves when using endfloat and the captionsoff option.
\ifCLASSOPTIONcaptionsoff
  \newpage
\fi



\bibliographystyle{IEEEtran}
\bibliography{Transactions-Bibliography/IEEEabrv,Transactions-Bibliography/egbib}

% trigger a \newpage just before the given reference
% number - used to balance the columns on the last page
% adjust value as needed - may need to be readjusted if
% the document is modified later
%\IEEEtriggeratref{8}
% The "triggered" command can be changed if desired:
%\IEEEtriggercmd{\enlargethispage{-5in}}

% references section

% can use a bibliography generated by BibTeX as a .bbl file
% BibTeX documentation can be easily obtained at:
% http://mirror.ctan.org/biblio/bibtex/contrib/doc/
% The IEEEtran BibTeX style support page is at:
% http://www.michaelshell.org/tex/ieeetran/bibtex/
%\bibliographystyle{IEEEtran}
% argument is your BibTeX string definitions and bibliography database(s)
%\bibliography{IEEEabrv,../bib/paper}
%
% <OR> manually copy in the resultant .bbl file
% set second argument of \begin to the number of references
% (used to reserve space for the reference number labels box)


% biography section
% 
% If you have an EPS/PDF photo (graphicx package needed) extra braces are
% needed around the contents of the optional argument to biography to prevent
% the LaTeX parser from getting confused when it sees the complicated
% \includegraphics command within an optional argument. (You could create
% your own custom macro containing the \includegraphics command to make things
% simpler here.)
%\begin{IEEEbiography}[{\includegraphics[width=1in,height=1.25in,clip,keepaspectratio]{mshell}}]{Michael Shell}
% or if you just want to reserve a space for a photo:
\section{Biography Section}

\begin{IEEEbiography}[{\includegraphics[width=1in,height=1.25in,clip,keepaspectratio]{IEEEtran/figures/rxq.jpg}}]{Xuqian Ren}
received the B.E. degree from the University of Science and Technology Beijing in 2019, and received the M.S. degree from Beijing Institute of Technology in 2022. She is currently a Ph.D. candidate in the Computer Science, Faculty of Information Technology and Communication Sciences, at Tampere University, Finland. Her current research interests include Novel view synthesis, image generation, and 3d reconstruction.
\end{IEEEbiography}

\vspace{-10mm}
% if you will not have a photo at all:
\begin{IEEEbiography}[{\includegraphics[width=1in,height=1.25in,clip,keepaspectratio]{IEEEtran/figures/ysp.jpg}}]{Shaopeng Yang}
received the B.E. degree from Langfang Normal University in 2015, received the M.S. degree from Beijing Union University in 2019. He is currently a Ph.D. student with School of Artificial Intelligence, Beijing Normal University. He current research interests include contrastive learning and gait recognition.
\end{IEEEbiography}

\vspace{-10mm}
% if you will not have a photo at all:
\begin{IEEEbiography}[{\includegraphics[width=1in,height=1.25in,clip,keepaspectratio]{IEEEtran/figures/hsh.pdf}}]{Saihui Hou}
received the B.E. and Ph.D. degrees from University of Science and Technology of China in 2014 and 2019, respectively. He is currently an Assistant Professor with School of Artificial Intelligence, Beijing Normal University, and works in cooperation with Watrix Technology Limited Co. Ltd. His research interests include computer vision and machine learning. He recently focuses on gait recognition which aims to identify different people according to the walking patterns.
\end{IEEEbiography}

% insert where needed to balance the two columns on the last page with
% biographies
%\newpage
\vspace{-10mm}
\begin{IEEEbiography}[{\includegraphics[width=1in,height=1.25in,clip,keepaspectratio]{IEEEtran/figures/ccs.pdf}}]{Chunshui Cao}
received the B.E. and Ph.D. degrees from University of Science and Technology of China in 2013 and 2018, respectively. During his Ph.D. study, he joined Center for Research on Intelligent Perception and Computing, National Laboratory of Pattern Recognition, Institute of Automation, Chinese Academy of Sciences. From 2018 to 2020, he worked as a Postdoctoral Fellow with PBC School of Finance, Tsinghua University. He is currently a Research Scientist with Watrix Technology Limited Co. Ltd. His research interests include pattern recognition, computer vision and machine learning.
\end{IEEEbiography}

\vspace{-10mm}
\begin{IEEEbiography}[{\includegraphics[width=1in,height=1.25in,clip,keepaspectratio]{IEEEtran/figures/lx.pdf}}]{Xu Liu}
received the B.E. and Ph.D. degrees from University of Science and Technology of China in 2013 and 2018, respectively. He is currently a Research Scientist with Watrix Technology Limited Co. Ltd. His research interests include gait recognition, object detection and image segmentation.
\end{IEEEbiography}

\vspace{-10mm}
\begin{IEEEbiography}[{\includegraphics[width=1in,height=1.25in,clip,keepaspectratio]{IEEEtran/figures/hyz.pdf}}]{Yongzhen Huang}
received the B.E. degree from Huazhong University of Science and Technology in 2006, and the Ph.D. degree from Institute of Automation, Chinese Academy of Sciences in 2011. He is currently a Professor with School of Artificial Intelligence, Beijing Normal University, and works in cooperation with Watrix Technology Limited Co. Ltd. He has published one book and more than 80 papers at international journals and conferences such as TPAMI, IJCV, TIP, TSMCB, TMM, TCSVT, CVPR, ICCV, ECCV, NIPS, AAAI. His research interests include pattern recognition, computer vision and machine learning.
\end{IEEEbiography}
% You can push biographies down or up by placing
% a \vfill before or after them. The appropriate
% use of \vfill depends on what kind of text is
% on the last page and whether or not the columns
% are being equalized.

%\vfill

% Can be used to pull up biographies so that the bottom of the last one
% is flush with the other column.
%\enlargethispage{-5in}



% that's all folks
\end{document}



%% bare_jrnl.tex
%% V1.4b
%% 2015/08/26
%% by Michael Shell
%% see http://www.michaelshell.org/
%% for current contact information.
%%
%% This is a skeleton file demonstrating the use of IEEEtran.cls
%% (requires IEEEtran.cls version 1.8b or later) with an IEEE
%% journal paper.
%%
%% Support sites:
%% http://www.michaelshell.org/tex/ieeetran/
%% http://www.ctan.org/pkg/ieeetran
%% and
%% http://www.ieee.org/

%%*************************************************************************
%% Legal Notice:
%% This code is offered as-is without any warranty either expressed or
%% implied; without even the implied warranty of MERCHANTABILITY or
%% FITNESS FOR A PARTICULAR PURPOSE! 
%% User assumes all risk.
%% In no event shall the IEEE or any contributor to this code be liable for
%% any damages or losses, including, but not limited to, incidental,
%% consequential, or any other damages, resulting from the use or misuse
%% of any information contained here.
%%
%% All comments are the opinions of their respective authors and are not
%% necessarily endorsed by the IEEE.
%%
%% This work is distributed under the LaTeX Project Public License (LPPL)
%% ( http://www.latex-project.org/ ) version 1.3, and may be freely used,
%% distributed and modified. A copy of the LPPL, version 1.3, is included
%% in the base LaTeX documentation of all distributions of LaTeX released
%% 2003/12/01 or later.
%% Retain all contribution notices and credits.
%% ** Modified files should be clearly indicated as such, including  **
%% ** renaming them and changing author support contact information. **
%%*************************************************************************


% *** Authors should verify (and, if needed, correct) their LaTeX system  ***
% *** with the testflow diagnostic prior to trusting their LaTeX platform ***
% *** with production work. The IEEE's font choices and paper sizes can   ***
% *** trigger bugs that do not appear when using other class files.       ***                          ***
% The testflow support page is at:
% http://www.michaelshell.org/tex/testflow/



\documentclass[journal]{IEEEtran}
%
% If IEEEtran.cls has not been installed into the LaTeX system files,
% manually specify the path to it like:
% \documentclass[journal]{../sty/IEEEtran}





% Some very useful LaTeX packages include:
% (uncomment the ones you want to load)


% *** MISC UTILITY PACKAGES ***
%
%\usepackage{ifpdf}
% Heiko Oberdiek's ifpdf.sty is very useful if you need conditional
% compilation based on whether the output is pdf or dvi.
% usage:
% \ifpdf
%   % pdf code
% \else
%   % dvi code
% \fi
% The latest version of ifpdf.sty can be obtained from:
% http://www.ctan.org/pkg/ifpdf
% Also, note that IEEEtran.cls V1.7 and later provides a builtin
% \ifCLASSINFOpdf conditional that works the same way.
% When switching from latex to pdflatex and vice-versa, the compiler may
% have to be run twice to clear warning/error messages.






% *** CITATION PACKAGES ***
%
%\usepackage{cite}
% cite.sty was written by Donald Arseneau
% V1.6 and later of IEEEtran pre-defines the format of the cite.sty package
% \cite{} output to follow that of the IEEE. Loading the cite package will
% result in citation numbers being automatically sorted and properly
% "compressed/ranged". e.g., [1], [9], [2], [7], [5], [6] without using
% cite.sty will become [1], [2], [5]--[7], [9] using cite.sty. cite.sty's
% \cite will automatically add leading space, if needed. Use cite.sty's
% noadjust option (cite.sty V3.8 and later) if you want to turn this off
% such as if a citation ever needs to be enclosed in parenthesis.
% cite.sty is already installed on most LaTeX systems. Be sure and use
% version 5.0 (2009-03-20) and later if using hyperref.sty.
% The latest version can be obtained at:
% http://www.ctan.org/pkg/cite
% The documentation is contained in the cite.sty file itself.






% *** GRAPHICS RELATED PACKAGES ***
%
\ifCLASSINFOpdf
  \usepackage[pdftex]{graphicx}
  \usepackage[nocompress]{cite}
  \usepackage{graphicx}
  \usepackage{amsmath}
  \usepackage{amssymb}
  \usepackage{booktabs}
  \usepackage{algorithm}
  \usepackage{algorithmic}
  \usepackage{multirow}
  \usepackage[dvipsnames,table,xcdraw]{xcolor}
% \usepackage{colortbl}
% \usepackage{color}
  \usepackage{tikz}
  \usepackage{booktabs}
  \usepackage{threeparttable}
  \usepackage{graphicx}                                                           
  \usepackage{float} 
  \usepackage{times}
  \usepackage{epsfig}
  \usepackage{subfigure}
  % declare the path(s) where your graphic files are
  % \graphicspath{{../pdf/}{../jpeg/}}
  % and their extensions so you won't have to specify these with
  % every instance of \includegraphics
  % \DeclareGraphicsExtensions{.pdf,.jpeg,.png}
\else
  % or other class option (dvipsone, dvipdf, if not using dvips). graphicx
  % will default to the driver specified in the system graphics.cfg if no
  % driver is specified.
  % \usepackage[dvips]{graphicx}
  % declare the path(s) where your graphic files are
  % \graphicspath{{../eps/}}
  % and their extensions so you won't have to specify these with
  % every instance of \includegraphics
  % \DeclareGraphicsExtensions{.eps}
\fi
% graphicx was written by David Carlisle and Sebastian Rahtz. It is
% required if you want graphics, photos, etc. graphicx.sty is already
% installed on most LaTeX systems. The latest version and documentation
% can be obtained at: 
% http://www.ctan.org/pkg/graphicx
% Another good source of documentation is "Using Imported Graphics in
% LaTeX2e" by Keith Reckdahl which can be found at:
% http://www.ctan.org/pkg/epslatex
%
% latex, and pdflatex in dvi mode, support graphics in encapsulated
% postscript (.eps) format. pdflatex in pdf mode supports graphics
% in .pdf, .jpeg, .png and .mps (metapost) formats. Users should ensure
% that all non-photo figures use a vector format (.eps, .pdf, .mps) and
% not a bitmapped formats (.jpeg, .png). The IEEE frowns on bitmapped formats
% which can result in "jaggedy"/blurry rendering of lines and letters as
% well as large increases in file sizes.
%
% You can find documentation about the pdfTeX application at:
% http://www.tug.org/applications/pdftex





% *** MATH PACKAGES ***
%
%\usepackage{amsmath}
% A popular package from the American Mathematical Society that provides
% many useful and powerful commands for dealing with mathematics.
%
% Note that the amsmath package sets \interdisplaylinepenalty to 10000
% thus preventing page breaks from occurring within multiline equations. Use:
%\interdisplaylinepenalty=2500
% after loading amsmath to restore such page breaks as IEEEtran.cls normally
% does. amsmath.sty is already installed on most LaTeX systems. The latest
% version and documentation can be obtained at:
% http://www.ctan.org/pkg/amsmath





% *** SPECIALIZED LIST PACKAGES ***
%
%\usepackage{algorithmic}
% algorithmic.sty was written by Peter Williams and Rogerio Brito.
% This package provides an algorithmic environment fo describing algorithms.
% You can use the algorithmic environment in-text or within a figure
% environment to provide for a floating algorithm. Do NOT use the algorithm
% floating environment provided by algorithm.sty (by the same authors) or
% algorithm2e.sty (by Christophe Fiorio) as the IEEE does not use dedicated
% algorithm float types and packages that provide these will not provide
% correct IEEE style captions. The latest version and documentation of
% algorithmic.sty can be obtained at:
% http://www.ctan.org/pkg/algorithms
% Also of interest may be the (relatively newer and more customizable)
% algorithmicx.sty package by Szasz Janos:
% http://www.ctan.org/pkg/algorithmicx




% *** ALIGNMENT PACKAGES ***
%
%\usepackage{array}
% Frank Mittelbach's and David Carlisle's array.sty patches and improves
% the standard LaTeX2e array and tabular environments to provide better
% appearance and additional user controls. As the default LaTeX2e table
% generation code is lacking to the point of almost being broken with
% respect to the quality of the end results, all users are strongly
% advised to use an enhanced (at the very least that provided by array.sty)
% set of table tools. array.sty is already installed on most systems. The
% latest version and documentation can be obtained at:
% http://www.ctan.org/pkg/array


% IEEEtran contains the IEEEeqnarray family of commands that can be used to
% generate multiline equations as well as matrices, tables, etc., of high
% quality.




% *** SUBFIGURE PACKAGES ***
%\ifCLASSOPTIONcompsoc
%  \usepackage[caption=false,font=normalsize,labelfont=sf,textfont=sf]{subfig}
%\else
%  \usepackage[caption=false,font=footnotesize]{subfig}
%\fi
% subfig.sty, written by Steven Douglas Cochran, is the modern replacement
% for subfigure.sty, the latter of which is no longer maintained and is
% incompatible with some LaTeX packages including fixltx2e. However,
% subfig.sty requires and automatically loads Axel Sommerfeldt's caption.sty
% which will override IEEEtran.cls' handling of captions and this will result
% in non-IEEE style figure/table captions. To prevent this problem, be sure
% and invoke subfig.sty's "caption=false" package option (available since
% subfig.sty version 1.3, 2005/06/28) as this is will preserve IEEEtran.cls
% handling of captions.
% Note that the Computer Society format requires a larger sans serif font
% than the serif footnote size font used in traditional IEEE formatting
% and thus the need to invoke different subfig.sty package options depending
% on whether compsoc mode has been enabled.
%
% The latest version and documentation of subfig.sty can be obtained at:
% http://www.ctan.org/pkg/subfig




% *** FLOAT PACKAGES ***
%
%\usepackage{fixltx2e}
% fixltx2e, the successor to the earlier fix2col.sty, was written by
% Frank Mittelbach and David Carlisle. This package corrects a few problems
% in the LaTeX2e kernel, the most notable of which is that in current
% LaTeX2e releases, the ordering of single and double column floats is not
% guaranteed to be preserved. Thus, an unpatched LaTeX2e can allow a
% single column figure to be placed prior to an earlier double column
% figure.
% Be aware that LaTeX2e kernels dated 2015 and later have fixltx2e.sty's
% corrections already built into the system in which case a warning will
% be issued if an attempt is made to load fixltx2e.sty as it is no longer
% needed.
% The latest version and documentation can be found at:
% http://www.ctan.org/pkg/fixltx2e


%\usepackage{stfloats}
% stfloats.sty was written by Sigitas Tolusis. This package gives LaTeX2e
% the ability to do double column floats at the bottom of the page as well
% as the top. (e.g., "\begin{figure*}[!b]" is not normally possible in
% LaTeX2e). It also provides a command:
%\fnbelowfloat
% to enable the placement of footnotes below bottom floats (the standard
% LaTeX2e kernel puts them above bottom floats). This is an invasive package
% which rewrites many portions of the LaTeX2e float routines. It may not work
% with other packages that modify the LaTeX2e float routines. The latest
% version and documentation can be obtained at:
% http://www.ctan.org/pkg/stfloats
% Do not use the stfloats baselinefloat ability as the IEEE does not allow
% \baselineskip to stretch. Authors submitting work to the IEEE should note
% that the IEEE rarely uses double column equations and that authors should try
% to avoid such use. Do not be tempted to use the cuted.sty or midfloat.sty
% packages (also by Sigitas Tolusis) as the IEEE does not format its papers in
% such ways.
% Do not attempt to use stfloats with fixltx2e as they are incompatible.
% Instead, use Morten Hogholm'a dblfloatfix which combines the features
% of both fixltx2e and stfloats:
%
% \usepackage{dblfloatfix}
% The latest version can be found at:
% http://www.ctan.org/pkg/dblfloatfix




%\ifCLASSOPTIONcaptionsoff
%  \usepackage[nomarkers]{endfloat}
% \let\MYoriglatexcaption\caption
% \renewcommand{\caption}[2][\relax]{\MYoriglatexcaption[#2]{#2}}
%\fi
% endfloat.sty was written by James Darrell McCauley, Jeff Goldberg and 
% Axel Sommerfeldt. This package may be useful when used in conjunction with 
% IEEEtran.cls'  captionsoff option. Some IEEE journals/societies require that
% submissions have lists of figures/tables at the end of the paper and that
% figures/tables without any captions are placed on a page by themselves at
% the end of the document. If needed, the draftcls IEEEtran class option or
% \CLASSINPUTbaselinestretch interface can be used to increase the line
% spacing as well. Be sure and use the nomarkers option of endfloat to
% prevent endfloat from "marking" where the figures would have been placed
% in the text. The two hack lines of code above are a slight modification of
% that suggested by in the endfloat docs (section 8.4.1) to ensure that
% the full captions always appear in the list of figures/tables - even if
% the user used the short optional argument of \caption[]{}.
% IEEE papers do not typically make use of \caption[]'s optional argument,
% so this should not be an issue. A similar trick can be used to disable
% captions of packages such as subfig.sty that lack options to turn off
% the subcaptions:
% For subfig.sty:
% \let\MYorigsubfloat\subfloat
% \renewcommand{\subfloat}[2][\relax]{\MYorigsubfloat[]{#2}}
% However, the above trick will not work if both optional arguments of
% the \subfloat command are used. Furthermore, there needs to be a
% description of each subfigure *somewhere* and endfloat does not add
% subfigure captions to its list of figures. Thus, the best approach is to
% avoid the use of subfigure captions (many IEEE journals avoid them anyway)
% and instead reference/explain all the subfigures within the main caption.
% The latest version of endfloat.sty and its documentation can obtained at:
% http://www.ctan.org/pkg/endfloat
%
% The IEEEtran \ifCLASSOPTIONcaptionsoff conditional can also be used
% later in the document, say, to conditionally put the References on a 
% page by themselves.




% *** PDF, URL AND HYPERLINK PACKAGES ***
%
%\usepackage{url}
% url.sty was written by Donald Arseneau. It provides better support for
% handling and breaking URLs. url.sty is already installed on most LaTeX
% systems. The latest version and documentation can be obtained at:
% http://www.ctan.org/pkg/url
% Basically, \url{my_url_here}.




% *** Do not adjust lengths that control margins, column widths, etc. ***
% *** Do not use packages that alter fonts (such as pslatex).         ***
% There should be no need to do such things with IEEEtran.cls V1.6 and later.
% (Unless specifically asked to do so by the journal or conference you plan
% to submit to, of course. )


% correct bad hyphenation here
\hyphenation{op-tical net-works semi-conduc-tor}


\begin{document}
%
% paper title
% Titles are generally capitalized except for words such as a, an, and, as,
% at, but, by, for, in, nor, of, on, or, the, to and up, which are usually
% not capitalized unless they are the first or last word of the title.
% Linebreaks \\ can be used within to get better formatting as desired.
% Do not put math or special symbols in the title.
\title{Unsupervised Gait Recognition with \\ Selective Fusion}

\author{Xuqian Ren,
        Shaopeng Yang,
        Saihui Hou,
        Chunshui Cao,
        Xu Liu and
        Yongzhen Huang% <-this % stops a space
\thanks{Xuqian Ren is with the Computer Science Unit, Faculty of Information Technology and Communication Sciences, Tampere University, Tampere 33720, Finland. This work is finished when she was an intern at Watrix Technology Limited Co. Ltd before becoming a Ph.D. candidate at Tampere University.
% note need leading \protect in front of \\ to get a newline within \thanks as
% \\ is fragile and will error, could use \hfil\break instead.
(E-mail: xuqian.ren@tuni.fi)}% <-this % stops a space
\thanks{Shaopeng Yang is with School of Artificial Intelligence,
Beijing Normal University, Beijing 100875, China and also with Watrix Technology Limited Co. Ltd, Beijing 100088, China. He is co-first author.}% <-this % stops a space
\thanks{Saihui Hou and Yongzhen Huang are with School of Artificial Intelligence,
Beijing Normal University, Beijing 100875, China and also with Watrix Technology Limited Co. Ltd, Beijing 100088, China. (E-mail: housai hui@bnu.edu.cn, huangyongzhen@bnu.edu.cn). They are the corresponding authors of this paper.}
\thanks{Chunshui Cao and Xu Liu are with Watrix Technology Limited Co. Ltd, Beijing 100088, China.}
}

% note the % following the last \IEEEmembership and also \thanks - 
% these prevent an unwanted space from occurring between the last author name
% and the end of the author line. i.e., if you had this:
% 
% \author{....lastname \thanks{...} \thanks{...} }
%                     ^------------^------------^----Do not want these spaces!
%
% a space would be appended to the last name and could cause every name on that
% line to be shifted left slightly. This is one of those "LaTeX things". For
% instance, "\textbf{A} \textbf{B}" will typeset as "A B" not "AB". To get
% "AB" then you have to do: "\textbf{A}\textbf{B}"
% \thanks is no different in this regard, so shield the last } of each \thanks
% that ends a line with a % and do not let a space in before the next \thanks.
% Spaces after \IEEEmembership other than the last one are OK (and needed) as
% you are supposed to have spaces between the names. For what it is worth,
% this is a minor point as most people would not even notice if the said evil
% space somehow managed to creep in.



% The paper headers
\markboth{Journal of \LaTeX\ Class Files,~Vol.~14, No.~8, August~2015}%
{Shell \MakeLowercase{\textit{et al.}}: Bare Demo of IEEEtran.cls for IEEE Journals}
% The only time the second header will appear is for the odd numbered pages
% after the title page when using the twoside option.
% 
% *** Note that you probably will NOT want to include the author's ***
% *** name in the headers of peer review papers.                   ***
% You can use \ifCLASSOPTIONpeerreview for conditional compilation here if
% you desire.




% If you want to put a publisher's ID mark on the page you can do it like
% this:
%\IEEEpubid{0000--0000/00\$00.00~\copyright~2015 IEEE}
% Remember, if you use this you must call \IEEEpubidadjcol in the second
% column for its text to clear the IEEEpubid mark.



% use for special paper notices
%\IEEEspecialpapernotice{(Invited Paper)}




% make the title area
\maketitle

% As a general rule, do not put math, special symbols or citations
% in the abstract or keywords.
\begin{abstract}
Previous gait recognition methods primarily relied on labeled datasets, which require a labor-intensive labeling process. To eliminate this dependency, we focus on a new task: Unsupervised Gait Recognition (UGR). We introduce a cluster-based baseline to solve UGR. However, we identify additional challenges in this task. First, sequences of the same person in different clothes tend to cluster separately due to significant appearance changes. Second, sequences captured from $0^{\circ}$ and $180^{\circ}$ views lack distinct walking postures and do not cluster with sequences from other views. To address these challenges, we propose a Selective Fusion method, consisting of Selective Cluster Fusion (SCF) and Selective Sample Fusion (SSF). SCF merges clusters of the same person wearing different clothes by updating the cluster-level memory bank using a multi-cluster update strategy. SSF gradually merges sequences taken from front/back views using curriculum learning. Extensive experiments demonstrate the effectiveness of our method in improving rank-1 accuracy under different clothing and view conditions.
\end{abstract}

% Note that keywords are not normally used for peerreview papers.
\begin{IEEEkeywords}
Gait Recognition, Unsupervised Learning, Contrastive Learning, Curriculum Learning.
\end{IEEEkeywords}






% For peer review papers, you can put extra information on the cover
% page as needed:
% \ifCLASSOPTIONpeerreview
% \begin{center} \bfseries EDICS Category: 3-BBND \end{center}
% \fi
%
% For peerreview papers, this IEEEtran command inserts a page break and
% creates the second title. It will be ignored for other modes.
\IEEEpeerreviewmaketitle



\section{Introduction}
% 介绍步态识别,引出主要无监督步态识别问题
\IEEEPARstart{W}{ith} the growing intelligent security and safety camera systems, gait recognition has gradually gained more attention and exploration for its non-contact, long-term, and long-distance recognition properties. 
%
Several works~\cite{chao2019gaitset,fan2020gaitpart,lin2021gait} attempt to solve gait recognition tasks and have reached significant progress in a laboratory environment. 
%
However, gait recognition in a realistic situation~\cite{ren2022progressive} and will be affected by many factors such as occlusion, dirty labels, labeling, and more. 
%
Labeling, especially, is a major challenge, requiring intensive manual effort for pairwise data. Therefore, training on unlabeled datasets becomes crucial to save resources and address these challenges.
% Among them, the problem of labeling is one of the biggest challenges because it demands intensive manual effort to label pairwise data. 
%
% Moreover, deploying pre-trained models in new test environments without any adaptation often suffers from severe performance deterioration due to the domain gap across different datasets.
%
% So it is necessary to train gait recognition models on unlabeled datasets, which saves a lot of human and financial resources. 

%----------------------------------------
To realize gait recognition using an unlabeled dataset, we focus on a task called \textbf{Unsupervised Gait Recognition} (UGR) to facilitate the research on training gait recognition models with new unlabeled datasets.
%
Here we focus on using silhouettes for gait recognition to illustrate our method.
%
When only using silhouettes of human walking sequences as input, due to lack of enough information, we observe two main challenges in UGR, as shown in Figure~\ref{fig:Introduction}.
%
First, due to the large change in appearance, sequences in different clothes of a subject are hard to gather into one cluster without any label supervision.
%
Second, sequences captured from front/back views, such as views in $0^{\circ}/018^{\circ}/162^{\circ}/180^{\circ}$ in CASIA-B~\cite{yu2006framework}, are challenging to gather with sequences taken from other views of the same person because they lack vital information, such as walking postures.
%
Furthermore, these sequences tend to cluster into small groups based on their views or get mixed with sequences of the same perspective from other subjects.
%
So in this paper, we provide methods to overcome them accordingly.

\begin{figure}[t]
	\centering	 
	\includegraphics[width=\linewidth]{IEEEtran/figures/Introduction.pdf}	 	
	\caption{Two main challenges in UGR. A kind of style in different colors denotes a subject in different clothes, which are usually erroneously assigned with different pseudo labels (\textit{e.g.}, `001', `002'). Also, sequences taken from front/back views of different subjects tend to mix together (\textit{e.g.}, `003').}
	\label{fig:Introduction}
 % \vspace{-0.6cm}
\end{figure}

Currently, some Person Re-identification (Re-ID) works~\cite{lin2019bottom,ge2020self,lin2020unsupervised,wang2020unsupervised,dai2021cluster,9978648} have already touched the field of identifying person in an unsupervised manner.
%
There are also some traditional methods, such as~\cite{ball2012unsupervised,cola2015unsupervised,rida2015unsupervised} employ unsupervised learning to facilitate the development of gait recognition. 
%
However, research directions involving methods based on deep learning in this field are still under-explored.
%
In this paper, we use the state-of-the-art pattern in~\cite{dai2021cluster}, a cluster-based framework with contrastive learning, as a baseline to realize UGR.
%
% However, when directly adopting this framework, we find the performance still needs to be improved, especially in the \textbf{walking with different clothes (CL)} condition.
%
To address the two challenges, we propose a new method called \textbf{S}elective \textbf{F}usion(SF), to gradually pull cross-view and cross-cloth sequences together.


Our method comprises two techniques: Selective Cluster Fusion (SCF), which is used to narrow the distance of cross-cloth clusters, and Selective Sample Fusion (SSF), which is used to pull cross-view pairs nearly, especially helpful for outliers near $0^{\circ}/180^{\circ}$.
%
To be specific, first, in SCF, we use a support set selection module to generate a support set for each cluster.
%
In the support set, there are selected candidate clusters of each corresponding cluster that potentially belong to the same person but are in different clothes.
%
There is another multi-cluster update strategy designed in SCF to help update the cluster centroid of each pseudo cluster in the memory bank.
%
Using this approach, we not only tighten the clusters but also encourage clusters of the same individual in different clothing to be influenced by the current clustered groups and pulled closer toward them.
%
Second, we designed the SSF to deal with samples taken from front/back views. 
%
In SSF, we utilize a view classifier to identify sequences captured from front/back views. We then employ curriculum learning to gradually incorporate these sequences with those captured from other views.
%
Namely, the sequences are absorbed at a dynamic rate, relaxing the aggregate requirement for each cluster.
%
This approach enables us to re-assign pseudo labels for sequences captured from front/back views, thereby encouraging them to cluster with sequences captured from other views.
%
With our method, we gain a large recognition accuracy improvement compared to the baseline (with GaitSet~\cite{chao2019gaitset} backbone: NM + 3.1\%, BG + 8.6\%, CL + 9.7\%; with GaitGL~\cite{lin2021gait} backbone: BG + 1.1\%, GL + 17.2\% on CASIA-BN dataset~\cite{yu2006framework}\footnote{NM: normal walking condition, BG: carrying bags when walking, CL: walking with different coats.}).


% It is worth noting that our method focuses on clustering the features and learning the interior correlations of clusters based on pseudo labels. We do not specifically focus on learning the domain-invariant features or addressing domain shift in particular.
% % It bears emphasizing that our method focuses on clustering the features and learning the interior correlations of clusters based on pseudo labels. 
% % %
% % We don’t focus on learning the domain-invariant features or reducing the domain shift particularly. 
% %
% Also, unlike the Re-ID task, there does not exist a unified pre-trained model like ResNet-50~\cite{he2016deep} pre-trained on ImageNet~\cite{deng2009imagenet} that can be used for the gait community, which is another promising future research direction. 
% %
% Our method is built on a model pre-trained on OU-MVLP~\cite{takemura2018multi}. The OU-MVLP dataset is only used in the pre-training stage and is not involved in the unsupervised learning stage.
% %
% Since no large dataset can be used to learn cross-cloth information, our method mainly focuses on improving the cross-cloth performance. We achieve this without relying on any prior information from the pre-trained model, making it an unsupervised learning task rather than a domain transformation.}

To sum up, our contributions mainly lie in three folds:
\begin{itemize}
    \item[$\bullet$] We focus on Unsupervised Gait Recognition (UGR) using a cluster-based method with contrastive learning. Despite its practicality, it requires careful consideration. To address this task, we establish a baseline using cluster-level contrastive learning.
    
    \item[$\bullet$]  We deeply explore the characteristics of UGR, finding the two main challenges: clustering sequences with different clothes and with front/back views. To address these challenges, we propose a \textbf{S}elective \textbf{F}usion(SF) method. This method involves selecting potentially matched cluster/sample pairs to help them fuse gradually.
    
    \item[$\bullet$] Extensive experiments on three popular gait recognition benchmarks have shown that our method can bring consistent improvement over baseline, especially in walking in different coat conditions. 
\end{itemize}
%%%%%%%%%%%%%%%%%%%%%%%%%%%%%%%%%%%%%%%%%%%%%%%%%%%

\section{Related Work}\label{sec:relatedwork}

Gait recognition plays an important role in enhancing safety and security in the development of intelligent cities~\cite{zhang2024research}. Most existing gait recognition works are trained in a supervised manner, in which cross-cloth and cross-view labeled sequence pairs have been provided.
%
They mainly focus on learning more discriminative features~\cite{liao2020model,li2020jointsgait,chao2019gaitset,fan2020gaitpart,lin2021gait,9916067,9229117,9913216,10042966} or developing gait recognition applications in natural scenes~\cite{hou2022gait,das2023gait,9870842,9928336}.
%
However, obtaining labeled training pairs is challenging in real-world applications. Despite extensive research in gait recognition, further exploration of its practical applications is still needed.
%
In this work, we consider a practical setting and take one of the first steps toward achieving gait recognition without the need for labeled training datasets.

\subsection{Gait Recognition}

\noindent\textbf{Model-based method:} This kind of method encodes poses or skeletons into discriminative features to classify identities.
%
For example, PoseGait~\cite{liao2020model} extracts handcrafted features from 3D poses based on human prior knowledge. 
%
JointsGait~\cite{li2020jointsgait} extracts spatiotemporal features from 2D joints by GCN~\cite{yan2018spatial}, then maps them into discriminative space according to the human body structure and walking pattern.
%
GaitGraph~\cite{gaitgraph} uses human pose estimation to extract robust pose from RGB images, then encode the key points as nodes, and encode skeletons as joints in the Graph Convolutional Network to extract gait information.

\noindent\textbf{Appearance-based method:} This series of methods mostly input silhouettes, extracting identity information from the shape and walking postures.
%
GaitSet~\cite{chao2019gaitset} first extracts frame-level and set-level features from an unordered silhouette set, promoting the set-based method's development.
%
GaitPart~\cite{fan2020gaitpart} further includes part-level pieces of information, mining details from silhouettes.
%
In contrast, the video-based method GaitGL~\cite{lin2021gait} employs 3D CNN for feature extraction based on temporal knowledge.
%
Our method can be used in appearance-based unsupervised gait recognition.
%
% Since our method mainly focuses on realistic gait recognition with a large amount of data, we choose silhouettes as input for its computation saving and robustness.
%
In our framework, we adopt both the set-based method and the video-based method as the backbones to illustrate the generalization of our framework.

\noindent\textbf{Gait Recognition with Contrastive Learning:} The core idea of contrastive learning methods~\cite{chen2020simple, chen2021exploring, he2020momentum} is to construct effective positive and negative sample pairs through data augmentation and to design appropriate loss functions to optimize the model for learning useful data representations. Inspired by such methods as MoCo~\cite{he2020momentum} and SimCLR~\cite{chen2020simple}, GaitSSB~\cite{fan2023learning} proposes a self-supervised framework to learn general gait representations from large-scale unlabeled walking videos. GaitSSB treats each gait sequence as a single instance and aims to learn discriminative instance-level sequence features through contrastive learning. However, the current data augmentation methods are limited. More importantly, the positive pairs are often drawn from the same sequence, resulting in very similar positive sample pairs. This makes it difficult to simulate the variations caused by changes in clothing and camera viewpoints, thereby limiting the ability to provide effective supervisory signals that can guide the model to learn robust features.
To address these challenges, our method adopts a clustering strategy to group unlabeled data and generate pseudo-labels, allowing the model to learn representations based on cluster assignments. This clustering-based approach generates high-quality pseudo-labels and, unlike traditional contrastive methods where positive pairs are drawn from the same sequence, offers a more diverse and reasonable definition of positive and negative samples across different sequences. Defining these samples across multiple sequences proves more effective in addressing the challenges of cross-clothing and cross-camera scenarios in gait recognition tasks.


\subsection{Unsupervised Person Re-identification}
\textbf{Short-term Unsupervised Re-ID:}
Most fully unsupervised learning (FUL) Re-ID methods estimate pseudo labels for sequences, which can be roughly categorized into clustering-based and non-clustering-based methods. 
%----------------------------------------
Clustering-based methods~\cite{zeng2020hierarchical,wang2021camera,chen2021ice,xuan2021intra,zhang2023camera,dai2021cluster} first estimate a pseudo label for each sequence and train the network with sequence similarity.
%
In contrast, non-clustering-based methods~\cite{lin2020unsupervised,wang2020unsupervised} regard each image as a class and use a non-parametric classifier to push each similar image closer and pull all other images further.
%
In total, the accuracy of most non-cluster-based methods does not exceed the latest cluster-based methods, so we use the latter to solve UGR.
%

At present, there are some typical algorithms in clustering-based methods.
%
BUC~\cite{lin2019bottom} utilizes a bottom-up clustering method, gradually clustering samples into a fixed number of clusters. Though there is a need for more flexibility, it is a good starting point. 
%
HCT~\cite{zeng2020hierarchical} adopts triplet loss to BUC to help learn hard samples. 
%
SpCL~\cite{ge2020self} introduces a self-paced learning strategy and memory bank, gradually making generated sample features closer to reliable cluster centroids. 
%
To alleviate the high intra-class variance inside a cluster caused by camera styles, CAP~\cite{wang2021camera} proposes cross-camera proxy contrastive loss to pull instances near their own camera centroids in a cluster. 
%
ICE~\cite{chen2021ice} further explores inter-instance relationships instead of using camera labels to compact the clusters with hard contrastive loss and soft instance consistency loss. 
%
IICS~\cite{xuan2021intra} also considers the difference caused by cameras, decomposing the training pipeline into two phases. First, it categorizes features within each camera and generates labels. Second, according to sample similarity across cameras, inter-camera pseudo labels will be generated based on all instances. These two stages train CNN alternately to optimize features.
%
Cluster-contrast~\cite{dai2021cluster} improves SpCL by establishing a cluster-level memory dictionary, optimizing and updating both CNN and memory bank at the cluster level.
%----------------------------------------

On the contrary, the non-clustering-based methods mainly realize fully unsupervised Re-ID with similarity-based methods. 
%
SSL~\cite{lin2020unsupervised} predicts a soft label for each sample and trains the classification model with softened label distribution. 
%
MMCL~\cite{wang2020unsupervised} formulates FUL Re-ID as a multi-label classification task and classifies each sample into multiple classes by considering their self-similarity and neighbor similarity.

Inspired by the simple but elegant structure of Cluster-contrast~\cite{dai2021cluster}, we start from the Cluster-contrast framework to solve the UGR task.
Unsupervised~\cite{zeng2020hierarchical,dai2021cluster} and semi-supervised learning~\cite{chen2023class,chen2021multimodal} rely on modeling visual features (e.g., color, texture, shape) extracted from static images. Moreover, ReID methods typically depend on visual consistency across camera views, assuming that the same identity retains relatively stable appearance characteristics. However, gait recognition uses silhouette sequences as input, and it needs to identify features not only cross-view but also cross-clothes.
%

In summary, due to the differences in data modalities and the unique challenges faced by gait recognition, directly applying the cluster contrast method from~\cite{dai2021cluster} to the UGR task is not suitable.
%
As a solution, we design the SCF module with a support set that contains several cluster candidates the feature belongs to. Rather than pushing each feature toward only one cluster, we reduce the negative impact of erroneous gradients by minimizing incorrect associations. To tackle the challenge of clustering features from front/back views with those from other views, we also develop a novel method SSF to specifically help clustering these features from sparse views. 

\noindent\textbf{Long-term Unsupervised Re-ID:}
%
Gait recognition is a long-term task with cloth-changing, so long-term FUL Re-ID is more similar to UGR.
%
CPC~\cite{li2022unsupervised} uses curriculum learning~\cite{bengio2009curriculum} strategy to incorporate easy and hard samples and gradually relax the clustering criterion.
%
We do not use the same method in SCF, since each cluster mainly contains sequences with one cloth type, and it is better to pull clusters as a whole.
%
In contrast, we use curriculum learning in SSF to distinguishly deal with sequences in front/back views and gradually re-assign pseudo labels for those sequences.
%%%%%%%%%%%%%%%%%%%%%%%%%%%%%%%%%%%%%%%%%%%%%%%%%%%%%%%%%%%%%%%%%%

\section{Our Method}
In this work, we propose a new task called Unsupervised Gait Recognition (UGR), which is practical when dealing with realistic unlabeled gait datasets.
%
In this section, we first formally define our technique.
%
Next, we will show our baseline based on Cluster-contrast~\cite{dai2021cluster}, trying to solve UGR with an unlabeled training set.
%
Then, we deeply research the problems faced by UGR and find two challenges to improve the accuracy: sequences in different clothes of the same person tend to form different clusters, and the sequences captured from front/back views are difficult to gather with other views.
%
Based on the two problems, we propose Selective Fusion to gradually merge cross-cloth clusters and sequences taken from front/back views to make samples of each.  

% Please add the following required packages to your document preamble:
% \usepackage{graphicx}
\begin{table}[ht]
\centering
\caption{The definition of important symbols}
\resizebox{\columnwidth}{!}{%
\begin{tabular}{c|c}
\toprule
Symbol & Definition \\ \midrule
$\mathcal{X}_u/\mathcal{X}_t$ & \begin{tabular}[c]{@{}c@{}}Unlabeled training dataset/\\ Labeled testing dataset\end{tabular} \\ \hline
$\mathcal{Y}_u/\mathcal{Y}'_u$ & The true/ pseudo label for training dataset \\ \hline
$\mathcal{Y}_t$ & The true label for testing dataset \\ \hline
$f_\theta$ & Gait Recognition backbone \\ \hline
$N$ & The total sequence number of the training dataset \\ \hline
$f_i$ & \begin{tabular}[c]{@{}c@{}}The feature extracted from the $i$-th \\ sequence of the training dataset  \end{tabular}  \\ \hline



$n$ & \begin{tabular}[c]{@{}c@{}} The number of neighbors KNN \\ searched for each sequence  \end{tabular} \\ \hline
$s_{up}$ & The similarity threshold KNN used for clustering \\ \hline
$Q$ & The number of clusters \\ \hline
$\mathcal{C}_k$ & The $k$-th cluster centroid \\ \hline
$\mathcal{M}$ & The memory bank  \\ \hline
$b$ & The mini-batch extracted from pseudo clusters each iteration  \\ \hline
$q$ & One query contained in $b$ \\ \hline
$\tau$ & The temperature hyper-parameter \\ \hline
$m$ & The momentum hyper-parameter \\ \hline
$\mathcal{L}_{q_{CNCE}}$ & The ClusterNCE Loss of query $q$ \\ \hline
$\mathcal{C}_{+}$ & The positive cluster centroid \\ \hline


$f_{i_{CA}}$ & The feature of $i$-th cloth augmented sequence \\ \hline
$\mathcal{C}_{ak}$ & The $k$-th adjusted cluster centroid \\ \hline

$\mathcal{S}_k$ & The support set of the $k$-th cluster \\ \hline
$a$ & The number of pseudo ids $\mathcal{S}_k$ contains.  \\ \hline
$c_{low}$ & The lower bound to judge FVC \\ \hline
$s_{c}/s_{o}$ & \begin{tabular}[c]{@{}c@{}}The current/initial similarity bound \\ when re-assign pseudo labels for sequences in FVC  \end{tabular}  \\ \hline
$\lambda/\lambda_{base}$ & \begin{tabular}[c]{@{}c@{}} Each epoch ratio/base ratio to merge \\ extreme view sequences in each epoch\end{tabular} \\ \hline
$\mathcal{C}_n/\mathcal{C}_o$ & The number of new or old clusters in each epoch \\ \hline

\end{tabular}%
}

\label{tab:symbol}
\end{table}


%----------------------------------------
\subsection{Problem Formulation}

To formulate the unsupervised learning, we first define an unlabeled training dataset, denoted as $\mathcal{X}_u=\{x_{1}, x_2, ..., x_N\}$, including diverse conditions such as changes in clothing and viewpoint, where $N$ is the total sequence number.
%
% Then, we assume its ground truth label is set as $\mathcal{Y}_u$, but we cannot obtain it during training.
%
We want to train a gait recognition backbone $f_\theta$ to classify these sequences according to their similarity. By clustering the features, we generate pseudo-labels $\mathcal{Y}_u$ for the training dataset. Following this, specialized modules are employed to gradually merge samples from different clothes and views.
%
During evaluation, $f_\theta$ will extract features from a labeled test dataset $\{\mathcal{X}_t, \mathcal{Y}_t\}$ and the gallery will rank according to their similarity with the probe, then we gain the rank-1 accuracy for each condition and each view.
%
We aim to train $f_\theta$ and gain the best performance on $\mathcal{X}_t$.


% We first define an unlabeled training dataset with cloth-changing as $\mathcal{X}_u=\{x_{1}, x_2, ..., x_N\}$, where $N$ is the total sequence number.
% %
% Then, we assume its ground truth label is set as $\mathcal{Y}_u$, but we cannot obtain it during training.
% %
% We want to train a gait recognition backbone $f_\theta$ to classify these sequences according to their similarity.
% %
% During evaluation, $f_\theta$ will extract features from a labeled test dataset $\{\mathcal{X}_t, \mathcal{Y}_t\}$ and the gallery will rank according to their similarity with the probe, then we gain the rank-1 accuracy for each condition and each view.
% %
% We aim to train $f_\theta$ and gain the best performance on $\mathcal{X}_t$.


%----------------------------------------
\subsection{Proposed Baseline}
We modified Cluster-contrast~\cite{dai2021cluster} to build our baseline framework. Since a pre-trained model is required to initialize $f_\theta$, we first pre-train the backbones on a large gait recognition dataset OU-MVLP~\cite{takemura2018multi} and then load it when training on the unlabeled dataset. The training pipeline for the unlabeled dataset can be summarized as follows: 
%
% Since there is no pre-trained model in the Gait Recognition field like in Re-ID, we choose OU-MVLP~\cite{takemura2018multi} to train a pre-trained model and gain cross-view prior knowledge, Since OU-MVLP has a large number of subjects and various views, which is an ideal dataset to obtain identity classification ability.
% %
% Then we modify Cluster-contrast~\cite{dai2021cluster} to build our baseline framework to transfer the knowledge learned from OU-MVLP to other datasets. 
%
%

1) At the beginning of each epoch, we first use $f_\theta$ to extract features from each sequence in the training dataset in parts, which has been sliced by Horizontal Pyramid Matching (HPM)~\cite{chao2019gaitset} horizontally and equally. 
%
Then we concatenate all the parts to an embedding and regard it as a sequence feature to participate in the following process, denoted as $f_i, i \in \{1,2,..., N\} $. 
%
In this way, we can consider each sequence's features in parts and details.
%

2) We adopt KNN~\cite{fix1989discriminatory} to search $n$ neighbors for each sequence in feature space and calculate the similarity distances between each other. 
%
Then InfoMap~\cite{rosvall2008maps} is used to cluster $f_i$ with a similarity threshold $s_{up}$, and predict a pseudo label $\mathcal{Y}'_u= \{y'_{1},y'_{2},...,y'_{N} \}$ for each sequence.
%
When mapping features with a pre-trained model, each sequence tends to be mapped closer and not well separated.
%
So we tighten $s_{up}$, aiming to separate each subject into a single cluster.
%

3) With the pseudo labels, we compute the centroid of each cluster $\mathcal{C}_k, k \in \{0,1,...,Q\}$, $Q$ is the number of clusters, and then initialize a memory bank $\mathcal{M}$ at the cluster-level to store these centers $\mathcal{M} = \{\mathcal{C}_0, \mathcal{C}_1, ..., \mathcal{C}_Q\}$.
%

4) During each iteration, a mini-batch $b$ will be randomly selected from pseudo clusters, and during gradient propagation, we update the backbone with a ClusterNCE loss~\cite{dai2021cluster}.
\begin{equation}
\label{ClusterNCE}
    \mathcal{L}_{q_{CNCE}}=-\log \frac{\exp \left( q\cdot \mathcal{C}_+/\tau \right)}{\sum_{k=1}^Q{\exp}\left( q\cdot \mathcal{C}_k/\tau \right)}
\end{equation}
for each query feature $q$ extracted from the samples in the mini-batch, we calculate its similarity with the positive cluster centroid $\mathcal{C}_+$ it belongs to, $\mathcal{C}_k$ is the $k$-th cluster centroid, $\tau$ is the temperature hyper-parameter. Here we use all the query features in the mini-batch to calculate $\mathcal{L}_{q_{CNCE}}$.
%

5) Also, we update the centroids in the memory bank $\mathcal{M}$ that the queries belong to at a cluster level.
\begin{equation}
   \forall q \in \mathcal{C}_k, \mathcal{C}_k\gets m\mathcal{C}_k+(1-m)q
   \label{eq:momentum}
\end{equation}
$m$ is the momentum hyper-parameter used to update the centroids impacted by the batch. Each feature is responsible for updating the cluster centroids it belongs to. Additionally, $m$ serves as a crucial factor in determining the pace of cluster memory updates, whereby a higher value results in a slower update. This momentum value directly influences the consistency between the cluster features and the most recently updated query instance feature.

%----------------------------------------
With this pipeline, we can initially realize UGR.
%
However, some defects still prevent further improvement in both cross-cloth and cross-view situations.
%
First, each cloth condition of a subject has been separated from each other, making it hard to group them together into a single class.
%
This is because of the large intra-class diversity within each subject when the identity changes cloth and subtle inter-class variance between different persons when clothes types of different subjects are similar.
%
For example, NM and CL of one person are less similar in appearance to NM of other persons, leading to the intra-similarity being smaller than the inter-similarity.
%
Second, some sequences in front/back views (such as $0^{\circ}/018^{\circ}/162^{\circ}/180^{\circ}$) cannot correctly gather with sequences in other views, but tend to be confused with front/back views sequences of other subjects.
%
This is because sequences in these views lack enough walking patterns, so the model can only use the shape information to classify these sequences. With more similarity in appearance, sequences of different subjects in these views tend to be classified together.
%
Necessary solutions need to be considered to help the framework solve the two problems.
%----------------------------------------

\subsection{Proposed Method}

To tackle the problems we pointed out for UGR, we developed Selective Fusion, containing Selective Cluster Fusion (SCF) and Selective Sample Fusion (SSF) to solve the two drawbacks separately.
%
The framework of our method is in Figure~\ref{fig:structure}, and the pseudo-code is shown in Algorithm~\ref{our}.
%
\begin{figure*}[t]
	\centering	 
	\includegraphics[width=\linewidth]{IEEEtran/figures/structure.pdf}
	\caption{Overview of the framework with Selective Fusion. The upper two branches generate pseudo labels and initialize a memory bank at the start of each epoch. The lower branch accepts mini-batch extracted from pseudo clusters and calculates ClusterNCE Loss with Memory Bank to update it and the backbone in each iteration. CA is the \textit{Cloth Augmentation} method. InfoMap is employed in the Cluster module. SCF means \textit{Selective Cluster Fusion}. SSF represents \textit{Selective Sample Fusion}. In the Support set, the darker the color, the higher the similarity with the target cluster (Best viewed in color).}
	\label{fig:structure}
\end{figure*}


\begin{algorithm}[t]
\caption{The training procedure of Selective Fusion}
\label{alg:algorithm}
\textbf{Input}:  $\mathcal{X}_u$;  $f_\theta$
\begin{algorithmic}[1]  %1表示每隔一行编号
\REQUIRE Epoch number; iteration number; batch size; \\
hyper-parameters: $M/s_{up}/\tau/m/\mathcal{C}_{low}/\lambda_{base}/k/\mathcal{S}_{i}$
\FOR {epoch in range(0, epoch number+1) }
    \STATE {Apply CA to $\mathcal{X}_u$, get the augmented $\mathcal{X}_u$;}
    \STATE {Extract $f_n$ from $\mathcal{X}_u$ by $f_\theta$;}
    \STATE {Extract $f_{CA}$ from augmented $\mathcal{X}_u$ by $f_\theta$;}
    \STATE {Generate $\mathcal{Y}'_u$ for $f_n$}
    \STATE {Calculate the pseudo centroids with $\mathcal{Y}'_u$ and $f_n$;}
    \STATE {Generate adjusted $\mathcal{Y}'_u$ in SSF;}
    \STATE {Generate adjusted centroids with adjusted $\mathcal{Y}'_u$ and $f_n$;}
    \STATE {Initialize Memory Bank with adjusted centroids;}   
    \STATE {Generate support set in Support Set Selection Module with the centroids of $f_{CA}$ and $f_n$;}
    \FOR {iter in range(0, iteration number+1) }
    \STATE {Extract mini-batch from pseudo clusters;}
    \STATE {Extract sequences feature from batch;}
    \STATE {Calculate ClusterNCE loss with Memory Bank according to Eq.\textcolor{red}{~\ref{ClusterNCE}};}
    \STATE {Update the backbone;}
    \STATE {Update the Memory Bank according to Eq.\textcolor{red}{~\ref{momentum}};}
    \ENDFOR
\ENDFOR \\
\textbf{Output}: $f_\theta$
\end{algorithmic}
\label{our}
\end{algorithm}

 
In our method, at the beginning of each epoch, we use a Cloth Augmentation (CA) method to randomly generate an augmented variant for each sequence in the training dataset, then put them into the same backbone to extract features, named $f_{i}$ and $f_{i_{CA}}$.
%
Second, after getting the original pseudo labels generated by InfoMap, we use SSF to adjust the pseudo labels and then apply them to $f_{i}$ and $f_{i_{CA}}$ to get their adjusted centroids.
%
The adjusted centroids $\mathcal{C}_{ak}$ are used to initialize the memory bank $\mathcal{M}$.
%
Third, we use a support set selection module to generate a support set for each cluster, which will be used in the multi-cluster update strategy during back-propagation to help update the memory bank.
%
The support set selection module and multi-cluster update strategy are the two components of our SCF.
%

Next, we will introduce how we implement our Cloth Augmentation, SCF, and SSF methods.
% After using InfoMap in the Cluster module, we can get pseudo labels for each original feature, then apply them to $f_{n}$ and calculate cluster centroids.
% %
% However, since the pseudo label has many mistakes, we first put centroids and $f_n$ into SSF to re-assign pseudo labels for each feature, assigning the label of other views to the sequence in front/back views and get the adjusted pseudo labels for each sequence. 
% %
% With the new adjusted pseudo labels, we assign them to $f_{n}$ and gain the final adjusted centroids.
% %
% Also, the adjusted pseudo labels can be used to classify $f_{DA}$, and the centroids of $f_{n}$ and $f_{DA}$ will be put into Support Set Selection, generating the support set for each cluster.
% %
% The adjusted centroids of $f_{n}$ will initialize Memory Bank, storing the centers.
% %
% During each iteration, a batch of sequences will be selected according to the adjusted pseudo labels, then input backbones to extract features in the batch.
% %
% Next, all the batch features will be used when calculating ClusterNCE loss with Memory Bank.
% %
% When conducting gradient backpropagation, we update the backbone as well as use a Multi-cluster Update strategy to update the Memory Bank.
% %
% The Suppor Set Selection Module and Multi-cluster Update Strategy are the two components of our SCF.
% %
% Next, I will introduce how we implement our Data Augmentation, SSF, and SCF.





%----------------------------------------
\subsubsection{Cloth Augmentation}

The cloth augmentation is conducted for each sequence in the training set to explicitly get a fuse direction for each cluster, which simulates the potential clusters in other conditions belonging to the same person.
%
Currently, the cloth augmentation methods we use are targeted for silhouette datasets, which the majority of algorithms work on.
%
We randomly dilate or erode the upper/bottom/whole body in the whole sequence with a probability of 0.5\footnote{The kernel size for upper part: $5 \times 5$, lower part: $2 \times 2$}, forming six cloth augmentation types.
%
Also, the upper/middle/bottom has a dynamic edited boundary\footnote{The boundary selected from upper bound: [14, 18], middle bound: [38, 42], bottom bound:[60, 64] for $64 \times 64$ silhouettes}, adding more variance to the augmentation results.
%
Here we visualize some cloth augmented results in Figure~\ref{fig:DA}. When dilating NM, the sequence can simulate its corresponding CL condition, and when eroding CL, the appearance of subjects can be regarded as in NM condition.
\begin{figure}[ht]
	\centering	 
	\includegraphics[width=\linewidth]{IEEEtran/figures/DA.pdf}	 	
	\caption{The visualization of data augmentation on NM and CL conditions. Cloth Augmentation can simulate the potential appearance in different conditions of the same person.}
	\label{fig:DA}
\end{figure}
It is indeed that in real-world cloth-changing situations, the clothes have more diversity, only dilating or eroding cannot fully simulate all the situations.
%
Currently, we first consider the simple cloth-changing situation that walking with or without coats, which is also the cloth-changing method in CASIA-B~\cite{yu2006framework} and Outdoor-Gait~\cite{song2019gaitnet} dataset, to prove our method is valid.
%
In real application, automatic cloth augmentation methods can be employed, to automatically search other cloth augmentation methods, such as sheer the bottom part of the silhouettes to simulate wearing a dress and adding an oval above the head to simulate wearing a hat, which is another promising research direction in the future research.
%
In this work, we use clothing augmentation via dilation or erosion in our method to provide the opportunity to let the cross-cloth sequences have the chance to be closer to each other. 
%
This will facilitate our method of utilizing the close chance to further pull the clusters near through the support set and multi-cluster update strategy.

%
%----------------------------------------
\subsubsection{Selective Cluster Fusion}
SCF aims to pull clusters in different clothes belonging to the same subject closer.
%
It comprises two parts, a support set selection, which is used to generate a support set $\mathcal{S}_k$ for each cluster, aiming to find potential candidate clusters in different clothes, and a multi-cluster update strategy, aiming to decrease the distance between candidate clusters in the support set.
%

\textbf{Support Set:} The input of the support set selection module is the centroids of $\mathcal{C}_{ak}$ and $\mathcal{C}_{k}$.
%
By calculating the similarity between the $k$-th Cloth Augmented centroid $\mathcal{C}_{ak}$ with all the original centroids contained in $\mathcal{M}$, we can get a rank list with these pseudo labels, ranging from highest to lowest according to their similarity distances.
%
We select the top $a$ ids in each rank list, and the first id we set is the cluster itself, formulating the support set $\mathcal{S}_k = {id_0,id_1,...,id_a}$.
%
With the support set, we can concretize the optimization direction when pulling NM and CL together because $f_{i_{CA}}$ can be seen as the cross-cloth sequences in reality to some extent.
%
With the explicit regulation, we will not blindly pull a cluster close to any near neighbor.

\textbf{Multi-cluster update strategy:} When updating the memory bank during backpropagating, we use the support set in the multi-cluster update strategy.
%
Knowing which clusters are the potential conditions of one person, the new strategy can be formulated as follows:
\begin{equation}
\label{momentum}
   \forall \mathcal{C}_{ak} \in \mathcal{S}_k, \forall q \in \mathcal{C}_{ak}, \mathcal{C}_{ak} \leftarrow m \mathcal{C}_{ak} + (1-m) q
\end{equation}
All the candidate clusters in $\mathcal{S}_k$ need to participate in updating the memory bank.
%
By forcing the potential conditions to fuse, we can make the mini-batch influence clusters and clusters in the support set, compressing the distance between cross-cloth pairs.

% % In Equation~\ref{momentum}, $m$ serves as the momentum updating factor, and its value significantly affects the coherence between cluster features and the most recently updated query instance feature. In prior ReID-related work~\cite{dai2021cluster}, a relatively small fixed value was employed, leading to overly rapid updates of cluster centroids and compromising the overall consistency of the clusters. To address this issue, we adaptively use a cosine annealing strategy to update $m$. The new strategy can be formulated as follows:
% % \begin{equation}
% % \label{momentum_strategy}
% %    m_t = m_{\text{min}} + \frac{1}{2} (m_{\text{max}} - m_{\text{min}}) \left(1 + \cos\left(\frac{t \pi}{T}\right)\right)
% % \end{equation}
% % where $m_t$ is the momentum at training step $t$, $m_{\text{min}}$ is the minimum momentum value, $m_{\text{max}}$ is the initial value, and $T$ represents the total number of training steps within a single epoch. As training progresses ($t$ increases), the momentum decreases following a cosine curve. We include the value of $m_{\text{max}}$ and $m_{\text{min}}$  in the supplementary materials.
% }


%----------------------------------------
\subsubsection{Selective Sample Fusion}

Seeing that walking postures are absent in front/back views, sequences taken from these views have less feature similarity with features extracted from other views.
%
So they cannot be appropriately gathered into their clusters like other views, and tend to mix up with sequences of other identities captured from front/back views.
%
If we pull all the clusters towards their candidate clusters in the support set, those clusters mainly composed of sequences taken from front/back views will further gather with clusters with the same view condition, making the situation worse. 
%
To deal with clusters in this condition, we design SSF, in which we use curriculum learning to gradually re-assign pseudo labels to sequences in front/back views, forcing them to fuse with samples taken from other views progressively before conducting the SCF method.
%

\textbf{View classifier :} Specifically, we first train a view classifier on OU-MVLP, classifying whether the sequence is in the front/back view.
%
The view classifier structure we set is as same as the GaitSet structure we used, and we add the BNNeck behind. 
%
We assign the label 1 for sequences in $0^{\circ}/180^{\circ}$ and assign the label 0 for sequences in other views to train the view classifier. 
%
We train the view classifier on OU-MVLP to gain view knowledge, and when we have prior knowledge of sequences' view, we can quickly identify which clusters generated by our framework are composed of sequences taken from front/back views.
%

Indeed, it is true that when adapting the view classifier to other datasets, there may be appearance domain gaps and view classification gaps between different datasets, as other datasets may not label views in the same way as OUVMLP, and some datasets may not provide view labels at all.
%
However from our observations, we found the view classification task is not a hard task, the view classifier can already identify most sequences with fewer walking postures.
%
To further solve the problem that some sequences haven't appropriately assigned the right view label, we set a threshold to relax the requirement when clustering.
%
The criterion is that only when the number of sequences in the front/back view in one cluster is larger than the threshold $c_{low}$, we consider the cluster as Front/Back View Clusters (FVC).
%
%train:0.8865567216391804
%test:0.8850931677018633
By dissolving FVC, we calculate the similarity between each sequence in it with other centroids of non-FVC, and if the similarity is larger than $s_{c}$, we re-assign the nearest non-FVC pseudo label for the sequence.
%
Therefore, even if not all sequences with front or back views are correctly identified, aligning most of these sequences closer to other views can help reduce the feature disparity between these perspectives and other views.
%
And with the training process of curriculum learning, the clusters are tighter. The sequences identified with front/back views will also push the sequences that were misidentified closer to their cluster center automatically since they have appearance similarity. 
%
More intelligent methods can be developed in future research to reduce the dependence of the model on the prior information.

\textbf{Curriculum Learning:} However, we do not incorporate all the sequences in FVC at the same time, instead, we utilize Curriculum Learning~\cite{bengio2009curriculum} to fuse them progressively by enlarging the $s_{c}$ during each epoch:
%
\begin{equation}
    s_{c} = s_{o} - \lambda \times epoch\_number
\end{equation}
%
At the first epoch, $s_{o}$ is high, and only the similarity between sequences in FVC and centroids of other non-FVC higher than $s_{i}$, can be assigned new pseudo labels. 
%
Otherwise, they will be seen as outliers and cannot participate in training.
%
During training, the criterion gradually increases with a speed of $\lambda$, which allows for the gradual incorporation of new knowledge.
%
% To tighten the clusters and make more samples can be fused, we adopt a Center Loss:
% \begin{equation}
%     \mathcal{L} _C=\sum_{i=1}^{batch}{ \| f_{i}-c_{y'_i} \| _{2}^{2}}
% \end{equation}
% $f_i$ is the sequence feature in batch, $c_{y'_i}$ is the centroids of its corresponding pseudo cluster.
%
In response to this, we propose to set $\lambda$ adaptively~\cite{hou2019learning}:
\begin{equation}
\lambda=\lambda_{base} \left| \frac{\mathcal{C} _n}{\mathcal{C} _o} \right|
\end{equation}
where $\left|\mathcal{C}_n\right|$ and $\left|\mathcal{C}_o\right|$ is the number of new and old clusters in each epoch, $\lambda_{base}$ is a fixed constant for each dataset. 
%
Since more clusters are fused in each phase, $\lambda$ increases as the ratio of the number of new clusters to that of old clusters increases.
%


%----------------------------------------
\subsection{Training Strategy}
Overall, Selective Fusion can make separated conditions and scattered sequences taken from front/back views fuse tighter.
%
Here we show our training strategy and represent the feature distribution of each training phase in Figure~\ref{fig:stage}.
%
Our training strategy encompasses three stages.
%
At first, the features extracted by the pre-trained model have a tendency to cluster together, making it hard to distinguish between them.
%
We adopt our baseline to separate these sequences with a strict criterion, making each cluster gathered according to their similarity.
%
Second, we apply Selective Fusion to fuse different conditions of the same person and gradually merge sequences in front/back views with sequences in other views.
%
Finally, we get the clusters with all the cloth conditions and views.
%
% The total loss is summarized as:
% \begin{equation}
%     \mathcal{L} = \mathcal{L}_{CNCE} + \lambda_c \mathcal{L}_{C}
% \end{equation}
% $\lambda_c$ is used to control the weight of $\mathcal{L}_{C}$.

\begin{figure}[htbp]
	\centering	 
	\includegraphics[width=\linewidth]{IEEEtran/figures/multi-stage.pdf}	 	
	\caption{The three stages in our training strategy. First, with narrowed features extracted by the pre-trained model, we first adopt our baseline to separate them further. Then Selective Fusion is used to fuse matched clusters and samples together. Finally, we gain clusters with different clothes and views. The base is the \textit{Baseline}. Each type in a different color indicates \textit{each subject in a different cloth condition.} }
	\label{fig:stage}
\end{figure}

%%%%%%%%%%%%%%%%%%%%%%%%%%%%%%%%%%%%%%%%%%%%%%%%%%%%%%%%%%%%%%%%%%%%%%%%%%%%%%%%%%
\begin{table*}[h]
\centering
\caption{The parameters used in unsupervised learning on OU-MVLP, CASIA-BN, Outdoor-Gait dataset, and GREW.}
\setlength{\tabcolsep}{1pt}
%\resizebox{\columnwidth}{!}{%
\begin{tabular}{ccccccc}
\hline
Param                           & Backbone  &OU-MVLP & CASIA-BN                 & Outdoor-Gait   &GREW  \\  
\hline
\multirow{2}{*}{Model Channel}  & GaitSet   & -        & (32, 64, 128)(128, 256)  & (32, 64, 128)(128, 256) &(32, 64, 128,256)(256, 256)\\
                                & GaitGL    & -        & (32, 64, 128)(128, 128)  & -                       &- \\
Batch Size                      & Both      & (32, 16) & (8, 16)                  & (8, 8)                  & (8,16)                \\
Weight Decay                    & Both      & 5e-4     & 5e-4                     & 5e-4                    &5e-4                  \\
Start LR                        & Both      & 1e-1     & 1e-4                     & 1e-4                    &1e-4                \\
Milestones                      & Both      & -        & {[}3.5k, 8.5k{]}         & {[}3.5k, 8.5k{]}        & {[}3.5k, 8.5k{]}      \\
Epoch                           & Both      & -        & Baseline: 50, SF: 50     & Baseline: 50, SF: 50    & Baseline: 50, SF: 50  \\
Iteration                       & Both      & -        & Baseline: 50, SF: 100    & Baseline: 50, SF: 100   &Baseline: 1000, SF: 1000 \\
\multirow{2}{*}{\begin{tabular}[c]{@{}l@{}}Upper bound Milestones\end{tabular}} 
                                 & GaitSet  &{[}50k, 100k, 125k{]}      & {[}20k, 40k, 60k, 80k{]} & {[}10k, 20k, 30k, 35k{]} & {[}10k, 20k, 30k, 35k{]}\\
                                & GaitGL    & {[}150k, 200k, 210k{]}    & {[}70k, 80k{]}           & -  &-\\ \hline
\end{tabular}%
%}
\label{unsupervise_param}
\end{table*}
\section{Experiments}
%
Our methods can be employed in appearance-based methods. 
%
For simplicity, we take silhouette sequences as input since they are more robust when datasets are collected in the wild.
%
To demonstrate the effectiveness of our framework, we apply our methods to two existing backbones: GaitSet~\cite{chao2019gaitset} and GaitGL~\cite{lin2021gait} to help them train with unlabeled datasets. 
%
We also compare our method with upper bound which is trained with the ground truth label, and with the baseline, which is also trained without supervision.
%
All methods are implemented with PyTorch~\cite{paszke2019pytorch} and trained on TITAN-XP GPUs.
%In this section, 

%%%%%%%%%%%%%%%%%%%%%%%%%%%%%%%%%%%%%%%%%%%%%%%%%%%%%%%%%%%%%%%%%%%%%%%%%%%%%%%%%%
\subsection{Datasets} 
% Here we first pre-train backbones on a large gait recognition dataset OU-MVLP~\cite{takemura2018multi}. 
% %
% Most of the previous unsupervised ReID methods use ResNet-50~\cite{he2016deep} pre-trained on ImageNet~\cite{deng2009imagenet} as the pre-train model, while in gait recognition, there does not exist any pre-train model with strong generalization.
% %
% Fortunately, OU-MVLP contains a large number of subjects with 14 views, which is an ideal dataset for pre-train models.
% %
% We can train backbones on it to gain preliminary information to classify subjects, and with the large dataset volume, the model can be more generalized when adapted to other datasets. 
% %
% However, without cross-cloth pairs in OU-MVLP, the model could only gain cross-view ability.
% %
% So, we need to develop methods to help the model recognize cross-cloth pairs.
% %
% We load the pre-trained model and evaluate the performance of our method on three popular datasets, CASIA-BN~\cite{yu2006framework}, Outdoor-Gait~\cite{song2019gaitnet} and GREW~\cite{lin2014effects}.}
We train and test the performance of our method on three popular datasets, CASIA-BN~\cite{yu2006framework}, Outdoor-Gait~\cite{song2019gaitnet} and GREW~\cite{lin2014effects}.
\subsubsection{CASIA-BN} 
%
The original CASIA-B is a useful dataset with both cross-view and cross-cloth sequence pairs. 
%
It consists of 124 subjects, having three walking conditions: \textbf{normal walking} (NM\#01-NM\#06), \textbf{carrying bags} (BG\#01-BG\#02), and \textbf{walking with different coats} (CL\#01-CL\#02).
%
Each walking condition contains 11 views distributed in $[0^{\circ},180^{\circ}]$. We employ the protocol in the previous research~\cite{chao2019gaitset,fan2020gaitpart}.
% taking 74 subjects as the training dataset and 50 subjects as the testing dataset. 
%
During the evaluation, NM\#01-NM\#04 are the gallery, NM\#05-NM\#06, BG\#01-BG\#02, CL\#01-CL\#02 are the probe.
%
Due to the coarse segmentation of CASIA-B, we collected some pedestrian images and trained a new segmentation model to re-segment CASIA-B, and gain CASIA-BN.

\subsubsection{Outdoor-Gait}
This dataset only has cross-cloth sequence pairs.
%
With 138 subjects, Outdoor-Gait contains three walking conditions: normal walking (NM\#01-NM\#04), carrying bags (BG\#01-BG\#04), and walking with different coats (CL\#01-CL\#04). 
%
There are three capture scenes (Scene\#01-Scene\#03), however, each person only has one view ($90^{\circ}$).
%
69 subjects are used for training and the last 69 subjects for tests.
%
During the test, we use NM\#01-NM\#04 in Scene\#03 as a gallery and all the sequences in Scene\#01-Scene\#02 as probes in different conditions.
\subsubsection{GREW}
To the best of our knowledge, 
%
GREW~\cite{lin2014effects} is the largest gait dataset in real-world conditions. The raw videos are gathered from 882 cameras in a vast public area, encompassing nearly 3,500 hours of 1,080×1,920 streams. Alongside identity information, some attributes such as gender, 14 age groups, 5 carrying conditions, and 6 dressing styles have been annotated as well, ensuring a rich and diverse representation of practical variations.
%
Furthermore, this dataset includes a train set with 20,000 identities and 102,887 sequences, a validation set with 345 identities and 1,784 sequences, and a test set with 6,000 identities and 24,000 sequences. In the test phase, we strictly follow the official test protocols.
%2 sequences per subject are treated as probes and another 2 sequences as the gallery.
%This evaluation strictly follows the official test protocols.
%----------------------------------------
% Please add the following requiBrickRed packages to your document preamble:
% \usepackage{multirow}
\setlength{\tabcolsep}{5pt}
% Please add the following requiBrickRed packages to your document preamble:
% \usepackage{multirow}
% Please add the following requiBrickRed packages to your document preamble:
% \usepackage{multirow}
% Please add the following requiBrickRed packages to your document preamble:
% \usepackage{multirow}
\begin{table*}[ht]
\centering
\caption{The rank-1 accuracy (\%) on CASIA-BN for different probe views excluding the identical-view cases. For evaluation, the sequences of NM\#01-NM\#04 for each subject are taken as the gallery. The probe sequences are divided into three subsets according to the walking conditions (\textit{i.e.} NM, BG, CL). SF is our \textit{Selective Fusion Method}. CC is the \textit{Cluster-contrast framework we followed}. \textcolor{BrickRed}{Red} indicates the upper bound with supervised learning. \textcolor{RoyalBlue}{\textbf{Blue}} indicates the improvements on sequences in front/back views. \textbf{Bold} indicates the total improvements on different conditions.}
\begin{tabular}{ccccccccccccccc}
\toprule
\multirow{2}{*}{Backbone} & \multirow{2}{*}{Condition} & \multirow{2}{*}{Method} & \multicolumn{11}{c}{Probe View} & \multirow{2}{*}{Average} \\ \cline{4-14}
 &  &  & $0^{\circ}$ & $18^{\circ}$ & $36^{\circ}$ & $54^{\circ}$ & $72^{\circ}$ & $90^{\circ}$ & $108^{\circ}$ & $126^{\circ}$ & $134^{\circ}$ & $162^{\circ}$ & $180^{\circ}$ &  \\ \midrule
\multirow{12}{*}{GaitSet} & \multirow{4}{*}{NM} & Upper & 90.5 & 98.1 & 99.0 & 96.9 & 93.5 & 91.0 & 94.9 & 97.8 & 98.9 & 97.2 & 83.4 & \textcolor{BrickRed}{94.7} \\
 &  & Pretrain & 59.6 & 72.9 & 80.1 & 77.4 & 67.4 & 58.8 & 63.9 & 72.4 & 79.2 & 62.2 & 42.7 & 67.0 \\
 &  & Base (CC) & 77.7 & 92.0 & 94.7 & 92.8 & 88.4 & 83.8 & 86.2 & 91.2  & 93.0 & 90.5 & 69.4 & 87.2  \\
 &  & Ours (SF) & \textcolor{RoyalBlue}{\textbf{85.2}} & 93.6 & 96.4 & 93.8 & 90.0 & 84.6 & 89.6 & 92.3 & 96.9 & 93.2 & \textcolor{RoyalBlue}{\textbf{77.4}} & \textbf{90.3} \\ \cline{2-15} 
 
 & \multirow{4}{*}{BG} & Upper & 86.1 & 94.1 & 95.9 & 90.7 & 84.2 & 79.9 & 83.7 & 87.1 & 94.0 & 93.8 & 78.0 & \textcolor{BrickRed}{88.0} \\
 &  & Pretrain & 48.0 & 51.9 & 58.1 & 54.2 & 52.2 & 45.8 & 47.6 & 49.9  & 53.5 & 45.5 & 36.4 & 49.2 \\
 &  & Base (CC) & 70.4 & 81.5 & 84.0 & 78.0 & 74.3  & 67.0 & 71.9 & 73.7  & 77.3 & 76.5 & 62.5 & 74.3 \\
 &  & Ours (SF) & \textcolor{RoyalBlue}{\textbf{78.5}} & \textcolor{RoyalBlue}{\textbf{88.3}} & 89.8 & 88.0 & 83.5 & 76.4 & 80.5 & 83.5 & 85.8 & \textcolor{RoyalBlue}{\textbf{84.8}} & \textcolor{RoyalBlue}{\textbf{72.6}} & \textbf{82.9} \\ \cline{2-15} 
 & \multirow{4}{*}{CL} & Upper & 65.2 & 79.3 & 84.4 & 81.0 & 77.9 & 74.1 & 75.7 & 79.2 & 81.5 & 73.2 & 47.5 & \textcolor{BrickRed}{74.5} \\
 &  & Pretrain & 9.8 & 10.7 & 14.4 & 17.3 & 16.1 & 13.6 & 15.2 & 14.8 & 13.1  & 7.8 & 6.8 & 12.7 \\
 &  & Base (CC) & 27.7 & 32.4 & 37.2 & 37.6 & 33.0 & 29.2 & 32.0 & 32.2  & 32.3 & 27.8 & 20.7 & 31.1 \\
 &  & Ours (SF) & \textcolor{RoyalBlue}{\textbf{33.1}} & \textcolor{RoyalBlue}{\textbf{41.6}} & 46.4 & 47.6 & 46.7 & 41.2 & 44.7 & 43.3 & 45.6 & \textcolor{RoyalBlue}{\textbf{35.8}} & 22.5 & \textbf{40.8} \\ \midrule

 
\multirow{12}{*}{GaitGL} & \multirow{4}{*}{NM} & Upper & 94.2 & 97.5 & 98.7 & 96.7 & 95.1 & 92.9 & 95.9 & 97.9 & 99.0 & 98.0 & 87.1 & \textcolor{BrickRed}{95.7} \\
 &  & Pretrain & 66.2 & 79.9 & 85.3 & 84.9 & 73.5 & 65.8 & 71.8 & 81.7 & 85.8 & 79.1 & 51.0 & 75.0 \\
 &  & Base (CC) & 83.7 & 94.7 & 96.4 & 93.2 & 88.2 & 85.1 & 87.6 & 91.7 & 95.5 & 94.3 & 70.8 & 89.2 \\
 &  & Ours (SF) & 84.2 & 95.7 & 96.2 & 94.7 & 89.9 & 87.0 & 89.0 & 91.8  & 96.7 & 93.2 & 63.6 & \textbf{89.3} \\ \cline{2-15} 
 & \multirow{4}{*}{BG} & Upper & 88.6 & 96.2 & 96.6 & 94.2 & 91.0 & 85.8 & 91.0 & 94.4 & 97.0 & 95.3 & 76.6 & \textcolor{BrickRed}{91.5} \\
 &  & Pretrain & 53.4 & 66.4 & 67.2 & 67.7 & 62.0 & 56.2 & 59.9 & 62.4 & 66.1 & 64.6 & 43.4 & 60.8 \\
 &  & Base (CC) & 74.6 & 88.7 & 87.6 & 85.0 & 83.2 & 79.3 & 80.1 & 83.0 & 87.4 & 86.9 & 63.0 & 81.7 \\
 &  & Ours (SF) & 75.8 & 90.1 & 91.6 & 87.5 & 83.9 & 80.2 & 83.0 & 85.0 & 89.7 & 87.9  & 56.0 & \textbf{82.8} \\ \cline{2-15} 
 & \multirow{4}{*}{CL} & Upper & 71.7 & 87.6 & 91.0 & 88.3 & 85.3 & 81.4 & 82.9 & 85.9 & 87.5 & 85.4 & 53.0 & \textcolor{BrickRed}{81.8} \\
 &  & Pretrain & 18.5 & 25.4 & 28.7 & 29.9 & 28.4 & 23.6 & 25.7 & 24.9 & 24.3 & 21.7 & 11.8 & 23.9 \\
 &  & Base (CC) & 33.7 & 49.6 & 55.0 & 55.1 & 58.2 & 53.4 & 57.3 & 52.6 & 52.2 & 41.7 & 24.1 & 48.4 \\
 &  & Ours (SF) & \textcolor{RoyalBlue}{\textbf{53.6}} & \textcolor{RoyalBlue}{\textbf{71.1}} & 76.9 & 75.5 & 72.9 & 69.1 & 69.8 & 69.0 & 71.7 & \textcolor{RoyalBlue}{\textbf{60.7}} & \textcolor{RoyalBlue}{\textbf{31.2}} & \textbf{65.6} \\ \bottomrule
\end{tabular}

\label{tab:dataset_casiabn}
\end{table*}



%
%----------------------------------------
\subsection{Implementation Details} 
In this section, we provide comprehensive details about the implementation of our method. 
%We begin with the pre-training on the OU-MVLP~\cite{takemura2018multi} dataset, followed by specific considerations for benchmark settings. 
% Further details, such as structure, optimization settings, are included in the supplementary materials. }
% \subsubsection{Pre-training on OU-MVLP}
In the Re-ID task~\cite{dai2021cluster}, ResNet-50~\cite{he2016deep}, pre-trained on ImageNet~\cite{deng2009imagenet}, is commonly used as the backbone for feature extraction.
Similar to unsupervised Re-ID methods, a unified pre-trained model is crucial for initialization in UGR. However, there currently does not exist a unified pre-trained model that can be used for the gait community. Here we first pre-train backbones on a large gait recognition dataset OU-MVLP~\cite{takemura2018multi}, and then load it when training on the unlabeled dataset. OU-MVLP, with its large dataset volume, is an ideal dataset for pre-train models. It contains 10,307 subjects. The sequences of each subject distributed in 14 views between $[0^{\circ},90^{\circ}]$ and $[180^{\circ},270^{\circ}]$, but only have normal walking conditions. We used 5153 subjects to pre-train backbones.

It is worth noting that the OU-MVLP dataset is only used in the pre-training stage and is not involved in the unsupervised learning stage. There are no cloth-changing pairs in this dataset, preventing the transfer of cross-cloth information. And our method focuses on tackling the challenging cross-cloth task by clustering features and learning intrinsic information based on pseudo labels, without relying on any prior information from the pre-trained model. As a result, we abbreviate our method as Unsupervised Gait Recognition rather than a domain transformation.   Moreover, we believe that training a unified pre-trained model for the gait community is a promising direction for future work.

% Since there are no sequences involving walking with different clothes, the model trained on OU-MVLP does not have much cross-cloth ability.}

% It is worth noting that our method focuses on clustering the features and learning the interior correlations of clusters based on pseudo labels. We do not specifically focus on learning the domain-invariant features or addressing domain shift in particular.
% Also, unlike the Re-ID task, there does not exist a unified pre-trained model like ResNet-50~\cite{he2016deep} pre-trained on ImageNet~\cite{deng2009imagenet} that can be used for the gait community, which is another promising future research direction. Our method is built on a model pre-trained on OU-MVLP. The OU-MVLP dataset is only used in the pre-training stage and is not involved in the unsupervised learning stage.Since no large dataset can be used to learn cross-cloth information, our method mainly focuses on improving the cross-cloth performance. We achieve this without relying on any prior information from the pre-trained model, making it an unsupervised learning task rather than a domain transformation.}
%
% We can train backbones on it to gain preliminary information to classify subjects, and with the large dataset volume, the model can be more generalized when adapted to other datasets. 
% %
% However, without cross-cloth pairs in OU-MVLP, the model could only gain cross-view ability.
% %
% So, we need to develop methods to help the model recognize cross-cloth pairs.
% %
% % We load the pre-trained model and evaluate the performance of our method on three popular datasets, CASIA-BN~\cite{yu2006framework}, Outdoor-Gait~\cite{song2019gaitnet} and GREW~\cite{lin2014effects}.}


% OU-MVLP~\cite{takemura2018multi} contains 10,307 subjects. 
% %
% The sequences of each subject distributed in 14 views between $[0^{\circ},90^{\circ}]$ and $[180^{\circ},270^{\circ}]$ but only have normal walking conditions (00, 01). 
% %
% We used 5153 subjects to pre-train backbones.
% %
% In particular, 00 is used as the gallery, and 01 is taken as the probe.
% %
% Without walking with different clothes sequences, the model trained on OU-MVLP does not have much cross-cloth ability, so we need to develop specific methods to fuse cloth-changing pairs.

% We include details such as the structure, optimization settings, and hyperparameter configurations in the supplementary materials.}
% \subsubsection{Hyper-Parameters Setting} 
In Table~\ref{unsupervise_param}, we present the structure and optimization settings used for pre-training and unsupervised learning. Recognizing individuals on GREW proves more challenging due to its larger scale and diverse, real-world settings. Consequently, we employ convolutional layers with increased channel sizes (32, 64, 128, 256). For hyperparameters, we set $s_{up}=0.7$ for the baseline and $s_{up}=0.3$ for Selective Fusion to enlarge the boundary. The remaining parameters are set as $n=40$, $\tau=0.05$, $k=2$, $\lambda_{base}=0.005$, $c_{low} = 0.8$, and $s_{o} = 0.7$. Since the sequences in Outdoor-Gait are fewer, we change $n$ to 20 to avoid overfitting. The batch size is represented as $(B_S, B_T)$, where each mini-batch contains $B_S$ subjects. For each subject, $B_T$ sequences are sampled, and 30 frames are randomly selected from each sequence. Each frame is normalized to a size of $64\times44$. We use a cosine annealing strategy to update $m$ in Equation~\eqref{eq:momentum}, the strategy can be formulated as follows:
\begin{equation}
\label{momentum_strategy}
   m_t = m_{\text{min}} + \frac{1}{2} (m_{\text{max}} - m_{\text{min}}) \left(1 + \cos\left(\frac{t \pi}{T}\right)\right)
\end{equation}
where $m_t$ is the momentum at training step $t$, $m_{\text{min}}$ is the minimum momentum value, $m_{\text{max}}$ is the initial value, and $T$ represents the total number of training steps within a single epoch. As training progresses ($t$ increases), the momentum decreases following a cosine curve. We set $m_{\text{max}}=0.5$ and $m_{\text{min}}=0.1$.
%

%
\subsection{Benchmark Settings} To show the effectiveness, we define several benchmarks:

(1) \textit{Upper}. The upper bound reports the performance of each backbone trained with ground-truth labels.

(2) \textit{Pre-train}. The effect when directly applying the pre-trained model to the target dataset without fine-tuning.

(3) \textit{Base (CC)}. Fine-tuning the pre-trained model with our baseline framework implemented by Cluster-contrast.

(4) \textit{Ours (SF)}. The results of our proposed method.

\subsection{Performance Comparison}
% 
% Before training and testing, we first pre-train backbones on OU-MVLP.
% %
% The effect of the pre-trained model on the OU-MVLP test dataset with GaitSet~\cite{chao2019gaitset} backbone is 77.9, and with GaitGL~\cite{lin2021gait} backbone is 79.1 on rank-1 accuracy of the NM condition. 
% %
% % Training on OUMVLP can make the model gain cross-view knowledge but cannot achieve cross-cloth information.
% %
% It's essential to note that the OU-MVLP dataset is only used in the pre-training stage and is not involved in the unsupervised learning stage.
% So the comparison results can show that our unsupervised method boosts the cross-view performance further and produces decent cross-cloth results.}

%----------------------------------------
\subsubsection{CASIA-BN}
The performance comparison on CASIA-BN is shown in Table~\ref{tab:dataset_casiabn}. 
%
We evaluate the probe in three walking conditions separately. 
%
Since our method aims to improve the rank-1 accuracy of CL and sequences in front/back views, \textbf{ we take the accuracy for CL as the main criteria}.
%
From the results, we can see that our method outperforms the baseline in the CL condition by a remarkable margin (GaitSet: CL + 9.7\%; GaitGL: CL + 17.2\%).
%
It indicates that our Selective Cluster Fusion method can properly identify the potential clusters of the same person with different cloth conditions and pull them together.
%
Moreover, sequences in $0^{\circ}/18^{\circ}/162^{\circ}/180^{\circ}$ also gained large improvement in both cloth conditions.
%
Selective Sample Fusion can gradually gather individual front/back samples that were excluded initially, by assigning them the same pseudo labels as the sequences in other views.
%
Although lacking walking postures, the sequences with front/back views can still provide useful information for identifying a particular person. 
%
It should be noted that the hyper-parameters used for GaitGL are as same as GaitSet and without specific adjustment, which shows the generalization of our method when applying to different backbones. 
%
Both cues indicate that our method is effective when dealing with cloth-changing and front/back views.

%

%----------------------------------------
\subsubsection{Outdoor-Gait}
Although Outdoor-Gait does not consider cross-view data pairs, we can still verify the SCF method on this dataset and show the result with the GaitSet backbone in Table~\ref{tab:outdoor_gait}.
%
% Please add the following requiBrickRed packages to your document preamble:
% \usepackage{multirow}
% Please add the following requiBrickRed packages to your document preamble:
% \usepackage{multirow}

% \setlength{\tabcolsep}{12pt}
% \begin{table}[h]
% \centering
% \caption{The Rank-1 accuracy (\%) on Outdoor-Gait. When evaluation, we take NM\#1-NM\#4 in Scene\#3 as gallery and others as probe.}
% \begin{tabular}{ccccc}
% \hline
% Backbone & Method & NM & BG & CL \\ \hline
% \multirow{4}{*}{GaitSet} & Upper & \textcolor{BrickRed}{97.6} & \textcolor{BrickRed}{90.9} & \textcolor{BrickRed}{90.4} \\
%  & Pretrain & 45.8 &  46.4 & 43.3 \\
%  & Base (CC) & 84.8 & 66.5 & 62.9 \\
%  & Ours (SCF) & \textbf{89.1} & \textbf{73.6} & \textbf{71.9} \\ \hline
% \end{tabular}
% \label{tab:outdoor_gait}
% \end{table}
\setlength{\tabcolsep}{12pt}
\begin{table}[h]
\centering
\caption{The Rank-1 accuracy (\%) on Outdoor-Gait. When evaluation, we take NM\#1-NM\#4 in Scene\#3 as gallery and others as probe.}
\begin{tabular}{ccccc}
\hline
Backbone & Method & NM & BG & CL \\ \hline
\multirow{4}{*}{GaitSet} & Upper & \textcolor{BrickRed}{97.6} & \textcolor{BrickRed}{90.9} & \textcolor{BrickRed}{90.4} \\
 & Pretrain & 45.8 &  46.4 & 43.3 \\
 & Base (CC) & 84.8 & 66.5 & 62.9 \\
 & Ours (SF) & \textbf{90.0} & \textbf{73.7} & \textbf{71.9} \\ \hline
\end{tabular}
\label{tab:outdoor_gait}
\end{table}
%
\begin{table}[t]
\centering
\caption{The Rank-1 and Rank-5 accuracy (\%) on GREW. Trained on the GREW train set and evaluated on the test set.}
\setlength{\tabcolsep}{12pt}
\begin{tabular}{ccccc}
\hline
Backbone & Method &Rank-1 &Rank-5 \\ \hline
\multirow{4}{*}{GaitSet} & Upper & \textcolor{BrickRed}{48.4} & \textcolor{BrickRed}{63.6}  \\
 & Pretrain & 17.0 &  28.5  \\
 & Base (CC) & 18.3 & 30.4  \\
 & Ours (SF) & \textbf{20.2} & \textbf{32.0}\\ \hline
\end{tabular}
\label{tab:grew}
\end{table}
%----------------------------------------
% Please add the following required packages to your document preamble:
% \usepackage{multirow}
% \usepackage{graphicx}
\begin{table*}[h]
\centering
\caption{Ablation study demonstrating the effectiveness of each module in our method on the CASIA-BN, Outdoor-Gait, and GREW datasets based on Rank-1 accuracy (\%)}
\setlength{\tabcolsep}{12pt}
\label{tab:each}
\begin{tabular}{cccccccc}
\hline
\multirow{2}{*}{Setting} & \multicolumn{3}{c}{CASIA-BN}                  & \multicolumn{3}{c}{Outdoor-Gait}              & \multirow{2}{*}{GREW} \\
                         & NM            & BG            & CL            & NM            & BG            & CL            &                       \\ \hline
Base                     & 87.2          & 74.3          & 31.1          & 84.8          & 66.5          & 62.9          & 18.3                  \\
Base + SSF               & 87.5          & 75.8          & 32.8          & 85.3          & 66.9          & 61.6          & 18.7                  \\
Base + SCF               & 83.3          & 75.9          & 39.9          & 89.1          & 73.6          & 71.9          & 19.7                  \\
Ours                     & \textbf{90.3} & \textbf{82.9} & \textbf{40.8} & \textbf{90.0} & \textbf{73.7} & \textbf{71.9} & \textbf{20.2}         \\ \hline
\end{tabular}%
\end{table*}
%
%
\begin{table*}[h]
\centering
\caption{\textbf{The rank-1 accuracy (\%) on CASIA-BN for different probe views excluding the identical-view cases. The probe sequences are divided into three subsets according to the walking conditions (\textit{i.e.} NM, BG, CL).}}
\setlength{\tabcolsep}{9pt}
\resizebox{\textwidth}{!}{%
\begin{tabular}{cccccccccccccc}
\toprule
\multirow{2}{*}{Backbone} & \multirow{2}{*}{Condition} & \multirow{2}{*}{Method} & \multicolumn{10}{c}{Probe View}  \\ \cline{4-14} %%\multirow{2}{*}{Average}
 &  &  & $0^{\circ}$ & $18^{\circ}$ & $36^{\circ}$ & $54^{\circ}$ & $72^{\circ}$ & $90^{\circ}$ & $108^{\circ}$ & $126^{\circ}$ & $134^{\circ}$ & $162^{\circ}$ & $180^{\circ}$  \\ \midrule
\multirow{12}{*}{GaitSet} & \multirow{4}{*}{NM}& Base  & 77.7 & 92.0 & 94.7 & 92.8 & 88.4 & 83.8 & 86.2 & 91.2  & 93.0 & 90.5 & 69.4  \\ %& 87.2 
 &  & Base+SSF & \textcolor{RoyalBlue}{\textbf{80.1}} & \textcolor{RoyalBlue}{93.2} & 94.9 & 92.9 & 87.6 & 83.4 & 86.4 & 91.1 & 92.9 & \textcolor{RoyalBlue}{90.7} & \textcolor{RoyalBlue}{\textbf{69.9}}  \\ 
 &  & Base+SCF & 70.1 &89.3 &91.9 &91.8 &83.7 &82.1 &82.3 &90.8 &90.5 &81.6 &62.2   \\
&  & Ours (SF) & \textcolor{RoyalBlue}{\textbf{85.2}} & 93.6 & 96.4 & 93.8 & 90.0 & 84.6 & 89.6 & 92.3 & 96.9 & 93.2 & \textcolor{RoyalBlue}{\textbf{77.4}} \\ \cline{2-14} 
 
 & \multirow{4}{*}{BG} & Base & 70.4 & 81.5 & 84.0 & 78.0 & 74.3  & 67.0 & 71.9 & 73.7  & 77.3 & 76.5 & 62.5  \\%& 74.3
 &  & Base+SSF & \textcolor{RoyalBlue}{\textbf{72.9}} & \textcolor{RoyalBlue}{\textbf{84.0}} & 83.6 & 77.9 & 74.5 & 67.7 & 70.8 & 73.2 & 79.5 & \textcolor{RoyalBlue}{\textbf{78.1}} & \textcolor{RoyalBlue}{\textbf{71.3}} \\
 &  & Base+SCF &69.3 &79.5 &87.7 &79.8 &74.4 &74.5 &74.6 &77.7 &79.1 &76.1 &61.9 \\
 &  & Ours (SF) & \textcolor{RoyalBlue}{\textbf{78.5}} & \textcolor{RoyalBlue}{\textbf{88.3}} & 89.8 & 88.0 & 83.5 & 76.4 & 80.5 & 83.5 & 85.8 & \textcolor{RoyalBlue}{\textbf{84.8}} & \textcolor{RoyalBlue}{\textbf{72.6}} \\ \cline{2-14}  %& \textbf{82.9} \
 & \multirow{4}{*}{CL} & Base  & 27.7 & 32.4 & 37.2 & 37.6 & 33.0 & 29.2 & 32.0 & 32.2  & 32.3 & 27.8 & 20.7  \\%& 31.1
 &  & Base+SSF & \textcolor{RoyalBlue}{\textbf{32.1}} & \textcolor{RoyalBlue}{\textbf{37.3}} & 39.1 & 37.7 & 37.6 & 29.5 & 30.5 & 31.7 & 32.5 & \textcolor{RoyalBlue}{\textbf{30.9}} & \textcolor{RoyalBlue}{\textbf{21.9}}  \\ 
 &  & Base+SCF & 30.2 &38.0 &46.5 &46.7 &45.9 &41.3 &44.8 &43.7 &45.1 &34.3 &22.0\\
 &  & Ours (SF) & \textcolor{RoyalBlue}{\textbf{33.1}} & \textcolor{RoyalBlue}{\textbf{41.6}} & 46.4 & 47.6 & 46.7 & 41.2 & 44.7 & 43.3 & 45.6 & \textcolor{RoyalBlue}{\textbf{35.8}} & \textcolor{RoyalBlue}{\textbf{22.5}}  \\ 
 %& \textbf{40.8}
 \bottomrule
 \end{tabular}
}
\label{tab:dataset_casiabn_seperate}
\end{table*}
%
We can see that SF surpasses the baseline on both conditions (NM + 4.3\%; BG + 7.1\%; CL + 9.0\%).
%
With the SF method, not only the accuracy of CL condition improved, but also the accuracy of NM, and BG, which means features from CL sequences can also provide useful information when recognizing a person, and they can not be neglected.
%
If the features before and after changing clothes are not correctly associated, the gait recognition model will miss important information, leading to insufficient learning and difficulty in distinguishing between different individuals.
%
However, due to the small dataset volume and lack of views in Outdoor-Gait, the upper bound with GaitGL backbone overfit\footnote{NM: 95.5, BG: 91.3, CL: 86.2 }, so we do not show the results with it.
%
\subsubsection{GREW}
To test the generalization of our method, we apply it to a large wild dataset GREW ~\cite{lin2014effects}. Our method excels in further separating the narrowed features within the feature space as comprehensively as possible. The results, as shown in Table~\ref{tab:grew}, illustrate that our proposed method outperforms the baseline under both conditions (Rank-1 + 1.9\%; Rank-5 + 1.6\%). This affirms that our Selective Fusion method can effectively guide the updating of clusters and finally get the clusters with different conditions. Additionally, it validates the generalization of our method in outdoor scenarios.
%
\begin{figure*}[t]
	\centering	 
	\includegraphics[width=\linewidth]{IEEEtran/figures/hyper-parameter.pdf}	 	
	\caption{The effect of hyper-parameters $s_{up}/n/\tau/m$ on baselines. In there we choose a set of hyper-parameters that have the best result in our experiments. Other hyper-parameters do not change the result a lot, just lead to sub-optimal.}
	\label{fig:hyperparameter}
\end{figure*}
\begin{figure*}[t]
	\centering	 
	\includegraphics[width=\linewidth]{IEEEtran/figures/tbiomtsne.pdf}	 
	\caption{(a) and (b) are the TSNE images of the baseline method and our method(SF), respectively. The text above each feature point shows the pseudo label, and blue text indicates that the sequence is in front/back view. A color in a feature point represents a type of condition. Our method effectively clusters features belonging to the same person. (c) shows the corresponding gait sequence from (b), where the dashed circles indicate examples of clustering errors, and the solid circles indicate correct clustering examples. Please zoom in to see the details.}
	\label{fig:TSNE1}
\end{figure*}
% In GREW, the collection only lasts for one day, resulting in a lack of cross-cloth variation. Although with this limitation, our method still has effective generalization in outdoor scenarios.}

% {To test the generalization of our method, we apply it to a large wild dataset GREW ~\cite{lin2014effects}. Our method excels in further separating the narrowed features within the feature space as comprehensively as possible. The results, as shown in Table~\ref{tab:grew}, illustrate that our proposed method outperforms the baseline under both conditions (Rank-1 + 2.55\%; Rank-5 + 3.25\%). This evaluation follows the protocols used in OpenGait~\cite{fan2022opengait}. During the test phase, using the sequence "00" as the probe and "01" as the gallery sequence for straightforward comparison.}

% We also present the results, strictly following the official test protocols, as shown in Table~\ref{tab:grew1}. We compare our method with the state-of-the-art method GaitSSB~\cite{fan2023learning} in an unsupervised manner. This affirms that our Selective Fusion method can effectively guide the updating of clusters and finally get the clusters with different conditions. Additionally, it validates the generalization of our method in outdoor scenarios. In GREW, the collection only lasts for one day, resulting in a lack of cross-cloth variation. 
% %
% Although with this limitation, our method still has effective generalization in outdoor scenarios.}
% under uncontrolled enviornment, however, we observed that in GREW, the data collection in the wild is usually aided by person
% Re-Identification, which is particularly hard to collect the cross-clothes sequences. In GREW, the collection only lasts for one day,
% and thus the cross-cloth variation is lacking, which is not suitable to verify our method, which mainly targets for solving cross-cloth task with unsupervised learning. Therefore, we only test the first stage of our method on GREW, which used to seperate the
% narrowed feature further in the feature space. Our results, presented in Table~\ref{tab:grew}, illustrate that our proposed method outperforms the baseline under both conditions (Rank-1 + 2.55\%; Rank-5 + 3.25\%), affirming the generalization of our method in outdoor scenarios.}
%
% Please add the following requiBrickRed packages to your document preamble:
% \usepackage{multirow}
% Please add the following requiBrickRed packages to your document preamble:
% \usepackage{multirow}
% \setlength{\tabcolsep}{12pt}
% \begin{table}[h]
% \centering
% \caption{Our method v.s. other unsupervised State-of-the-Art.}}
% \begin{tabular}{ccccc}
% \hline
% Backbone} & Method} &Rank-1} &Rank-5} \\ \hline
% GaitBase} & GaitSSB} & 16.60} &-}\\ \hline
% \multirow{2}{*}{GaitSet}} 
%  % & Pretrain} & 16.96} &  28.46}  \\
%  & Base (CC)} & 18.26} & 30.42}  \\
%  & Ours (SCF)} & \textbf{19.66}} & \textbf{32.10}} \\ \hline
% \end{tabular}
% \label{tab:grew1}
% \end{table}

%%%%%%%%%%%%%%%%%%%%%%%%%%%%%%%%%%%%%%%%%%%%%%%%%%%%%%%%%%%%%%%%%%%%%%%%%%%%%%%%%%
%----------------------------------------
\begin{table}[t]
\centering
\caption{The effect of cluster candidates number $k$ in support set.}
\setlength{\tabcolsep}{12pt}
\begin{tabular}{cccc}
\hline
Settings     & NM & BG & CL \\ \hline
Ours ($a=4$) & 89.2 & 81.2 & 35.1 \\
Ours ($a=3$) & 89.7 & 81.1 & 37.9   \\
Ours ($a=2$) & \textbf{90.3} & \textbf{82.9} & \textbf{40.8}  \\ \hline
\end{tabular}%
\label{tab:SC-Fusion}
\end{table}
%%
% Please add the following required packages to your document preamble:
%%
\begin{table}[t]
\centering
\caption{The performance of different kinds of rates when conducting curriculum learning.}
\setlength{\tabcolsep}{12pt}
\begin{tabular}{cccc}
\toprule
Settings & NM & BG & CL \\ \midrule
Ours ($\lambda_{base}$) & 90.3  & 82.7 & 40.7 \\
Ours ($\lambda$) & \textbf{90.3} & \textbf{82.9} & \textbf{40.8} \\ \bottomrule
\end{tabular}%
\label{tab:SS-Fusion}
\end{table}
%
%%%%%%%%%%%%%%%%%%%%%%%%%%%%%%%%%%%%%%%%%%%%%%%%%%%%%%%%%%%%%%%%%%%%%%%%%%%%%%%%%%
% \usepackage{multirow}
\begin{table}[h!]
\centering
\caption{Comparison of Rank-1 Accuracy (\%) between our method and GaitSSB* on the CASIA-BN and GREW datasets. GaitSSB* denotes the modified version of GaitSSB~\cite{fan2023learning} with an updated backbone.}
\label{tab:SF_gaitSSB}
\begin{tabular}{ccccc}
\hline
\multirow{2}{*}{Method} & \multicolumn{3}{c}{CASIA-BN}                                             & \multirow{2}{*}{GREW} \\
                         & NM            & BG            & CL                                 &                       \\ \hline
GaitSSB*                 & 43.0          & 30.2          & 31.1                               & 17.2                  \\
Ours                     & \textbf{90.3} & \textbf{82.9} & \multicolumn{1}{c}{\textbf{40.8}} & \textbf{20.2}         \\ \hline
\end{tabular}
\end{table}
%%%%%%%%%%%%%%%%%%%%%%%%%%%%%%%%%%%%%%%%%%%%%%%%%%%%%%%%%%%%%%%%%%%%%%%%%%%%%%%%%%
\subsection{Ablation Study}
\subsubsection{Impact of Each Component in Our Selective Fusion}
In this section, we show that both SCF and SSF are essential components of our Selective Fusion method through experiments on the indoor CASIA-BN dataset and the outdoor datasets, Outdoor-Gait and GREW.
%
In Table~\ref{tab:each}, we show the results only using SCF or SSF.
%
Only with SSF, the rank-1 accuracy for each condition in front/back view slightly improved, but still had a poor performance on the cross-cloth problem.
%
When directly applying SCF, clusters mainly composed of sequences in front/back views will also pull closer to other clusters in the same condition, which should be forbidden.
%
Pulling more FVCs closer will further make these sequences merge into their actual clusters with other views, degrading the performance in recognizing sequences with front/back views.
%
Therefore, the best effect can be achieved only when these two methods are effectively combined.
As shown in Table~\ref{tab:dataset_casiabn_seperate}, we demonstrate the impact of SSF on sequences taken from front/back views, which validates its effectiveness. The improvements on sequences with front/back views are highlighted in blue.
%
%Next, we will show the influence of parameters used in each method.
\subsection{Effects of Different Parameters in Baseline}
\label{hyper}
%
Here we research how hyper-parameters $s_{up}$, $n$, $\tau$, $m$ affect the results of baselines. 
%
We adjust one parameter at the one time and keep other hyper-parameters unchanged. 
%
$s_{up}$ regulates the boundary of how far the features can be gathered into one cluster. The smaller $s_{up}$ it is, the tighter the boundary.
%
$n$ is the number of neighbors KNN searched for each sequence.
%
$\tau$ is the temperature parameter in ClusterNCE loss, indicating the entropy of the distribution.
%
$m$, the momentum value, controls the update speed of centroids stored in the Memory Bank.
%
From the results on CASIA-BN, we can see that when $s_{up}=0.7$, $n=40$, $\tau=0.05$, $m=0.2$ we have the overall best results for NM, BG, and CL.
%
When these parameters deviate too much from the current setting, the performance is sub-optimal.
%
Here we show the accuracy of NM, BG, and CL when adopting different parameters in the baseline framework in Figure~\ref{fig:hyperparameter}.
% Please add the following required packages to your document preamble:
% \usepackage{graphicx}
% \begin{table}[h]
% \centering
% \setlength{\tabcolsep}{12pt}
% \caption{The effectiveness of each component in our framework. Combining both methods can maximize their effect.}
% \begin{tabular}{cccc}
% \hline
% Settings & NM & BG & CL \\ \hline
% Base & 87.2 & 74.3 & 31.1 \\
% Base + SSF & 87.5 & 75.8 & 32.8 \\
% Base + SCF & 83.3 & 75.9 & 39.9 \\
% Ours & \textbf{90.3} & \textbf{82.9} & \textbf{40.8} \\ \hline
% \end{tabular}%
% \label{tab:each}
% \end{table}
% Please add the following required packages to your document preamble:
% \usepackage{multirow}
%----------------------------------------
\subsubsection{Impact of Candidate Number in Support Set}
Here we discuss the effect of a parameter in SCF, the candidate number $a$, in the support set.
%
In Table~\ref{tab:SC-Fusion}, we can see that when $a=2$, we have the best performance, which is in line with the fact that the features of NM and BG are easily projected together in the feature space since they have larger similarity and we should drag the features of NM with CL specifically in CASIA-BN.
\subsubsection{Impact of Rate of Curriculum Learning in SSF}
We test the effect of SSF with a dynamic or constant rate when conducting curriculum learning on CASIA-BN. 
%
Without curriculum learning, linearly clustering the front/back view sequences with sequences in other views will degrade the performance.
%
In Table~\ref{tab:SS-Fusion}, with a dynamic pulling rate, we can relax the requirement when training the model, which can make the model learn from easy to hard better. 
\subsubsection{Instance vs. Cluster-based method}
To further validate the effectiveness of our method, we conducted extensive experiments on both the indoor CASIA-BN dataset and the outdoor GREW dataset, comparing our approach against GaitSSB*. For a fair comparison, we replaced the backbone of GaitSSB with GaitSet while keeping all other experimental settings unchanged. As shown in Table~\ref{tab:SF_gaitSSB}, our method consistently outperforms GaitSSB* on both datasets. GaitSSB is an instance-based contrastive learning method that constructs positive pairs at the instance level through data augmentation. However, due to the limitations of current data augmentation techniques for gait sequences, and more importantly, the positive pairs are drawn from the same sequence, resulting in positive sample pairs that are very similar to each other. This makes it difficult to simulate the variations caused by changes in clothing and camera viewpoints, limiting the ability to provide effective supervisory signals that can guide the model to learn robust features. 

In contrast, our method adopts a clustering strategy to group unlabeled data and generate pseudo-labels, allowing the model to learn representations based on cluster assignments. By integrating clustering strategies with selective fusion, our approach effectively addresses the practical challenges of unsupervised gait recognition. As shown in Table 9, the experimental results demonstrate that our method is more robust in handling complex scenarios, such as cross-clothing and cross-view variations.

\subsubsection{Impact of different training dataset scales}
We further conducted related experiments on CASIA-BN to verify the impact of different training dataset scales on our method in Table~\ref{tab:train_scale}. The CASIA-BN training set consists of 8,107 sequences. We randomly selected different training dataset scales (4,000, 6,000, and all) from the CASIA-BN training set to train our method. The performance of the models trained on different scales was evaluated using the standard test set. As shown in Table~\ref{tab:train_scale}, having more training data is beneficial, as a larger dataset enables the model to learn more generalizable features, ultimately improving recognition performance.

\begin{table}[h]
\centering
\caption{\textbf{Randomly selecting different training dataset scales from the CASIA-BN training set to train our method, and evaluating the Rank-1 accuracy (\%)on the standard test set.}}
\label{tab:train_scale}
\begin{tabular}{cccc}
\hline
\multirow{2}{*}{Training dataset Scale} & \multicolumn{3}{c}{CASIA-BN} \\
                                        & NM       & BG      & CL      \\ \hline
4000                                    & 86.9     & 75.8    & 33.6    \\
6000                                    & 89.0     & 79.1    & 37.7    \\
All                                     & 90.3     & 82.9    & 40.8    \\ \hline
\end{tabular}
\end{table}


\subsubsection{Visulization of Selective Fusion}
The visualization effect of Selective Fusion is shown in Figure~\ref{fig:TSNE1}. We selected a subject in CASIA-BN and found that in baseline, BG and CL have a different pseudo label from NM. 
%
At the same time, some sequences in front/back views of NM/BG/CL are also assigned different pseudo labels with sequences in other views.
%
With SF, as shown in Figure~\ref{fig:TSNE1}(b), most sequences from various views and conditions are assigned the same identity. As shown in Figure~\ref{fig:TSNE1}(c), we also visualized some cases of SF.
\section{Limitation and future work}
Our method can be employed with off-the-shelf backbones to train on a new, unlabeled dataset. However, there are still some limitations:
First, our approach relies on prior knowledge in certain areas. For instance, (1) we use knowledge of viewing angles to identify challenging samples with front/back views, and (2) our model requires well-initialized parameters to ensure reasonable clustering performance at the beginning of training. These dependencies on prior knowledge can impact the final accuracy. To overcome this, we need to explore methods that can achieve better unsupervised gait recognition without such reliance.

Additionally, data augmentation plays a crucial role in unsupervised gait recognition. However, our current augmentation method only simulates clothing variations in specific scenarios. To better capture real-world variations in gait sequences, it is necessary to develop more robust data augmentation techniques.

\section{Conclusion}
In this work, we propose a new task, Unsupervised Gait Recognition. 
%
We first design a new baseline with cluster-level contrastive learning.
%
We identified two key challenges in unsupervised gait recognition: (1) sequences with different clothing are not grouped into a single cluster, and (2) sequences captured from front/back views are difficult to merge with those from other views.
To address these challenges, we developed the Selective Fusion method, which includes Selective Cluster Fusion and Selective Sample Fusion. Selective Cluster Fusion helps cluster sequences of the same individual across different clothing conditions, while Selective Sample Fusion progressively merges sequences from front/back views with those from other views.

%
Our experiments demonstrate that the proposed method effectively improves accuracy under both clothing variation and front/back view conditions.
This work reduces reliance on labeled data, enabling us to learn high-quality feature representations from large-scale unlabeled datasets, thereby advancing the development of gait recognition.





% Can use something like this to put references on a page
% by themselves when using endfloat and the captionsoff option.
\ifCLASSOPTIONcaptionsoff
  \newpage
\fi



\bibliographystyle{IEEEtran}
\bibliography{Transactions-Bibliography/IEEEabrv,Transactions-Bibliography/egbib}

% trigger a \newpage just before the given reference
% number - used to balance the columns on the last page
% adjust value as needed - may need to be readjusted if
% the document is modified later
%\IEEEtriggeratref{8}
% The "triggered" command can be changed if desired:
%\IEEEtriggercmd{\enlargethispage{-5in}}

% references section

% can use a bibliography generated by BibTeX as a .bbl file
% BibTeX documentation can be easily obtained at:
% http://mirror.ctan.org/biblio/bibtex/contrib/doc/
% The IEEEtran BibTeX style support page is at:
% http://www.michaelshell.org/tex/ieeetran/bibtex/
%\bibliographystyle{IEEEtran}
% argument is your BibTeX string definitions and bibliography database(s)
%\bibliography{IEEEabrv,../bib/paper}
%
% <OR> manually copy in the resultant .bbl file
% set second argument of \begin to the number of references
% (used to reserve space for the reference number labels box)


% biography section
% 
% If you have an EPS/PDF photo (graphicx package needed) extra braces are
% needed around the contents of the optional argument to biography to prevent
% the LaTeX parser from getting confused when it sees the complicated
% \includegraphics command within an optional argument. (You could create
% your own custom macro containing the \includegraphics command to make things
% simpler here.)
%\begin{IEEEbiography}[{\includegraphics[width=1in,height=1.25in,clip,keepaspectratio]{mshell}}]{Michael Shell}
% or if you just want to reserve a space for a photo:
\section{Biography Section}

\begin{IEEEbiography}[{\includegraphics[width=1in,height=1.25in,clip,keepaspectratio]{IEEEtran/figures/rxq.jpg}}]{Xuqian Ren}
received the B.E. degree from the University of Science and Technology Beijing in 2019, and received the M.S. degree from Beijing Institute of Technology in 2022. She is currently a Ph.D. candidate in the Computer Science, Faculty of Information Technology and Communication Sciences, at Tampere University, Finland. Her current research interests include Novel view synthesis, image generation, and 3d reconstruction.
\end{IEEEbiography}

\vspace{-10mm}
% if you will not have a photo at all:
\begin{IEEEbiography}[{\includegraphics[width=1in,height=1.25in,clip,keepaspectratio]{IEEEtran/figures/ysp.jpg}}]{Shaopeng Yang}
received the B.E. degree from Langfang Normal University in 2015, received the M.S. degree from Beijing Union University in 2019. He is currently a Ph.D. student with School of Artificial Intelligence, Beijing Normal University. He current research interests include contrastive learning and gait recognition.
\end{IEEEbiography}

\vspace{-10mm}
% if you will not have a photo at all:
\begin{IEEEbiography}[{\includegraphics[width=1in,height=1.25in,clip,keepaspectratio]{IEEEtran/figures/hsh.pdf}}]{Saihui Hou}
received the B.E. and Ph.D. degrees from University of Science and Technology of China in 2014 and 2019, respectively. He is currently an Assistant Professor with School of Artificial Intelligence, Beijing Normal University, and works in cooperation with Watrix Technology Limited Co. Ltd. His research interests include computer vision and machine learning. He recently focuses on gait recognition which aims to identify different people according to the walking patterns.
\end{IEEEbiography}

% insert where needed to balance the two columns on the last page with
% biographies
%\newpage
\vspace{-10mm}
\begin{IEEEbiography}[{\includegraphics[width=1in,height=1.25in,clip,keepaspectratio]{IEEEtran/figures/ccs.pdf}}]{Chunshui Cao}
received the B.E. and Ph.D. degrees from University of Science and Technology of China in 2013 and 2018, respectively. During his Ph.D. study, he joined Center for Research on Intelligent Perception and Computing, National Laboratory of Pattern Recognition, Institute of Automation, Chinese Academy of Sciences. From 2018 to 2020, he worked as a Postdoctoral Fellow with PBC School of Finance, Tsinghua University. He is currently a Research Scientist with Watrix Technology Limited Co. Ltd. His research interests include pattern recognition, computer vision and machine learning.
\end{IEEEbiography}

\vspace{-10mm}
\begin{IEEEbiography}[{\includegraphics[width=1in,height=1.25in,clip,keepaspectratio]{IEEEtran/figures/lx.pdf}}]{Xu Liu}
received the B.E. and Ph.D. degrees from University of Science and Technology of China in 2013 and 2018, respectively. He is currently a Research Scientist with Watrix Technology Limited Co. Ltd. His research interests include gait recognition, object detection and image segmentation.
\end{IEEEbiography}

\vspace{-10mm}
\begin{IEEEbiography}[{\includegraphics[width=1in,height=1.25in,clip,keepaspectratio]{IEEEtran/figures/hyz.pdf}}]{Yongzhen Huang}
received the B.E. degree from Huazhong University of Science and Technology in 2006, and the Ph.D. degree from Institute of Automation, Chinese Academy of Sciences in 2011. He is currently a Professor with School of Artificial Intelligence, Beijing Normal University, and works in cooperation with Watrix Technology Limited Co. Ltd. He has published one book and more than 80 papers at international journals and conferences such as TPAMI, IJCV, TIP, TSMCB, TMM, TCSVT, CVPR, ICCV, ECCV, NIPS, AAAI. His research interests include pattern recognition, computer vision and machine learning.
\end{IEEEbiography}
% You can push biographies down or up by placing
% a \vfill before or after them. The appropriate
% use of \vfill depends on what kind of text is
% on the last page and whether or not the columns
% are being equalized.

%\vfill

% Can be used to pull up biographies so that the bottom of the last one
% is flush with the other column.
%\enlargethispage{-5in}



% that's all folks
\end{document}


