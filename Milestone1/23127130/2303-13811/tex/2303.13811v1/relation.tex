\section{Some Remarks  About  \Apqe And Experiments}
\label{sec:rel}

In~\cite{cert_tech_rep}, we introduced a PQE solver called \Apqe. One
can view \Apqe as a version of \egp where the internal procedure
\ti{PrvClsRed} is implemented using the machinery of certificate
clauses. (Such \ti{PrvClsRed} procedure is more powerful than the one
implemented by \dpqe that we use in this paper.)  As we mentioned
above, we present \Eg because it is a very simple algorithm that still
can efficiently solve some large problems. The reason for introducing
\egp is twofold. First, it helps to emphasize the two advantages of
PQE over QE listed in Subsection~\ref{ssec:disc2}. Second, \egp is a
generic algorithm allowing to get new PQE solvers by varying the
implementation of \ti{PrvClsRed}.

In the following three sections, we describe experiments with \dpqe
(used as a stand-alone PQE algorithm), \Eg and \egp. The first two
sections reproduce the experiments with FIFO buffers and HWMCC-13
benchmarks that we conducted in~\cite{cert_tech_rep} with \Apqe.
Comparing the results of \dpqe,\linebreak\Eg and \egp allows to better
understand which of the features mentioned in
Subsection~\ref{ssec:disc2} makes PQE solving more efficient.  We
implemented local calls to \dpqe in \egp using the source of \dpqe
provided at~\cite{ds_pqe}. The same source was used to build a binary
of \dpqe. The sources of \Eg and \egp are available
at~\cite{eg_pqe,eg_pqe_plus}. We used Minisat2.0~\cite{minisat} as an
internal SAT-solver.

