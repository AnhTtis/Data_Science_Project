\section{Examples Of Problems That Reduce To PQE}
\label{app:using_pqe}
In this section, we give a few examples of how a problem can be
reduced to PQE.


\subsection{SAT-solving by PQE~\normalfont{\cite{hvc-14}}}
\label{ssec:sat_by_pqe}
%\vspace{5pt}
%\noindent\tb{SAT-solving by PQE~\nf{\cite{hvc-14}}.}
%\section{Simple Reduction Of SAT To PQE}
%\label{app:pqe_sat}
Consider the SAT problem of checking if
formula \prob{X}{F(X)} is true. One can view traditional SAT-solving
as proving \ti{all} clauses redundant in \prob{X}{F} e.g. by finding a
satisfying assignment or by deriving an empty clause and adding it to
$F$. The reduction to PQE below facilitates developing an incremental
SAT-algorithm that needs to prove redundancy only for a \ti{fraction}
of clauses.

Let \pnt{x} be a full assignment to $X$ and $G$ denote the clauses of
$F$ falsified by \pnt{x}. Checking the satisfiability of $F$ reduces
to taking $G$ out of the scope of quantifiers i.e. to finding $H$ such
that \mbox{$\prob{X}{F} \equiv H \wedge \prob{X}{F \setminus
    G}$}. Since all variables of $F$ are quantified in \prob{X}{F},
the formula $H$ is a Boolean constant 0 or 1. If \mbox{$H=0$}, then
$F$ is unsatisfiable. If $H\!=\!1$, then $F$ is satisfiable because $F
\setminus G$ is satisfied by \pnt{x}.

%
%  Equvialence checking
%
\subsection{Equivalence checking by PQE~\normalfont{\cite{fmcad16}}}
\label{ssec:pqe_ec}
%\vspace{4pt}
%\noindent\tb{Equivalence checking by PQE~\nf{\cite{fmcad16}}.}  Let
$N'(X',V',w')$ and $N''(X'',V'',w'')$ be single-output combinational
circuits to check for equivalence. Here $X',V'$ are the sets of
internal and input variables and $w'$ is the output variable of
$N'$. (Definition of $X'',V'',w''$ for $N''$ is the same.) The
reduction to PQE below facilitates the design of a \ti{complete}
algorithm able to exploit the similarity of $N'$ and $N''$.  This is
important because the current equivalence checkers exploiting such
similarity are \ti{incomplete}. If $N'$ and $N''$ are not ``similar
enough'', e.g. they have no functionally equivalent internal points,
the equivalence checker invokes a complete (but inefficient) procedure
ignoring similarities between $N'$ and $N''$.

Let $\mi{eq}(V',V'')$ specify a formula such that
\mbox{$\mi{eq}(\pnt{v}\,',\pnt{v}\,'')$ = 1} iff $\pnt{v}\,' =
\pnt{v}\,''$ where $\pnt{v}\,', \pnt{v}\,''$ are full assignments to
$V'$ and $V''$ respectively. Let formulas $G'(X',V',w')$ and
$G''(X'',V'',w'')$ specify $N'$ and $N''$ respectively. (As usual, we
assume that a formula $G$ specifying a circuit $N$ is obtained by
Tseitin transformations~\cite{tseitin}, see
Section~\ref{sec:prop_gen}.)
%
Let $h(w',w'')$ be a formula obtained by taking \ti{eq} out of
\prob{Z}{\mi{eq} \wedge G' \wedge G''} where $Z = X'\cup V' \cup X''
\cup V''$.  That is \prob{Z}{\mi{eq} \!\wedge\!G'\!\wedge\!G''}
$\equiv$ $h \wedge \prob{Z}{G' \wedge G''}$.  If $h \imp (w' \equiv
w'')$, then $N'$ and $N''$ are equivalent. Otherwise, $N'$ and $N''$
are inequivalent, unless they are identical constants
i.e. $w'\!\equiv\!w''\!\equiv 1$ or \mbox{$w'\!\equiv
  w''\!\equiv\!0$}. It is formally proved in~\cite{fmcad16} that the
more similar $N',N''$ are (where similarity is defined in the most
general sense), the easier taking $\mi{eq}$ out of \prob{Z}{\mi{eq}
  \wedge G' \wedge G''} becomes.


%
% Model checking
%
%\vspace{-5pt}
\subsection{Model checking by PQE~\normalfont{\cite{mc_no_inv2}}}
\label{ssec:pqe_mc}
%\vspace{4pt}
%\noindent\tb{Model checking by PQE~\nf{\cite{mc_no_inv2}}.}  One can
use PQE to find the reachability diameter of a sequential circuit
without computing the set of all reachable states. So, one can prove
an invariant by PQE without generating a stronger invariant that is
inductive.
%
Let $T(S',V,S'')$ denote the transition relation of a sequential
circuit $N$ where $S',S''$ are the present and next state variables
and $V$ specifies the (combinational) input variables. Let $I(S)$
specify the initial states of $N$.  For the sake of simplicity, we
assume that $N$ can stutter i.e.  for every state \pnt{s} there exists
a full assignment \pnt{v} to $V$ such that
$T(\pnt{s},\pnt{v},\pnt{s})=1$, (Then the sets of states reachable in
$m$ transitions and \ti{at most} $m$ transitions are identical. If $T$
has no stuttering, it can be easily introduced by adding a variable to
$V$.)

Let \di{I,T} denote the \ti{reachability diameter} for initial states
$I$ and transition relation $T$.  That is every state of the circuit
$N$ can be reached in at most \di{I,T} transitions. Given a number
$m$, one can use PQE to decide if \mbox{$\di{I,T} < m$}. This is done
by checking if $I_1$ is redundant in \linebreak\prob{\Abs{m}}{I_0
  \wedge I_1 \wedge T_m}. Here $I_0$ and $I_1$ specify the initial
states of $N$ in terms of variables of $S_0$ and $S_1$ respectively,
$\Abs{m} = S_0 \cup V_0 \cup \dots \cup S_{m-1} \cup V_{m-1}$ and $T_m
= T(S_0,V_0,S_1) \wedge \dots \wedge T(S_{m-1},V_{m-1},S_m)$.  If
$I_1$ is redundant, then $\di{I,T} < m$.

The idea above can be used, for instance, to prove an invariant $P$
true in an IC3-like manner (i.e. by constraining $P$) but without
generating an inductive invariant. To prove $P$ true, it suffices to
constrain $P$ to a formula $H$ such that\linebreak a) $I \imp H \imp
P$, b) $\di{H,T} < m$ and c) no state falsifying $P$ can be reached
from a state satisfying $H$ in $m\!-\!1$ transitions. The conditions
b) and c) can be verified by PQE and bounded model checking~\cite{bmc}
respectively.  In the special case where $H$ meets the three
conditions above for $m=1$, it is an \ti{inductive invariant}.
%\clearpage
