\section{Brief Description Of \dpqe}
\label{app:ds_pqe}
In this appendix, we give a high-level view of \dpqe and explain how
it works in \egp in more detail. A full description of \dpqe can be
found in~\cite{hvc-14}. \dpqe is based on the machinery of
D-sequents~\cite{fmcad13} ('\ti{DS}' in the name \dpqe stands for
'D-sequent'). Given a formula \prob{X}{F(X,Y)} and an assignment
\pnt{p} to $X \cup Y$, a D-sequent is a record \olds{X}{F}{p}{C}
stating that clause $C$ is redundant in \prob{X}{F} in subspace
\pnt{p}. In \egp, \dpqe is called in subspaces \pnt{y} where $F$ is
satisfiable. (Here \pnt{y} is a full assignment to $Y$.)  \dpqe
terminates upon deriving a D-sequent $(\prob{X}{F},\pnt{y}^*)
\rightarrow C$ where $\pnt{y}^* \subseteq \pnt{y}$. Such derivation
means that $C$ is proved redundant in \prob{X}{F} in subspace \pnt{y}.
Then the plugging clause $D$ falsified by $\pnt{y}^*$ is generated
where $\V{D} = \mi{Vars}(\pnt{y}^*)$.

\dpqe derives the D-sequent $(\prob{X}{F},\pnt{y}^*) \rightarrow C$
above by branching on variables of $X$. A variable is assigned a value
either by a decision or during Boolean Constraint Propagation (BCP). A
branch of the search tree goes on until an atomic D-sequent is derived
for $C$. This occurs when proving $C$ in the current subspace becomes
trivial. When backtracking, \dpqe merges D-sequents derived in
different branches using a resolution like operation called
\ti{join}. For instance, the join operation applied to D-sequents
$(\prob{X}{F},\pnt{p}') \rightarrow C$ where $\pnt{p}' =
(y_1=0,x_1=0)$ and $(\prob{X}{F},\pnt{p}'') \rightarrow C$ where
$\pnt{p}'' = (y_2=1,x_1=1)$ produces the D-sequent \olds{X}{F}{p}{C}
where \pnt{p}=$(y_1=0,y_2=1)$.

\dpqe has three situations where an atomic D-sequent is generated.
First, when $C$ is blocked in the current subspace and hence is
redundant there. Then an atomic D-sequent $(\prob{X}{F},\pnt{p})
\rightarrow C$ is generated where \pnt{p} consists of assignments that
made $C$ blocked in the current subspace. For instance, in
Example~\ref{exmp:eg_pqe+}, \dpqe would generate an atomic D-sequent
$(\prob{X}{F},(y_1\!=\!1)) \rightarrow C$.  Second, an atomic
D-sequent is generated when $C$ is satisfied by an assignment $w=b$
where $w \in X \cup Y$ and $b \in \s{0,1}$. (This can be a decision
assignment or an assignment derived from a clause during BCP.) Then an
atomic D-sequent\linebreak $(\prob{X}{F},(w=b)) \rightarrow C$ is
built.  Third, an atomic D-sequent is generated when a conflict occurs
and a conflict clause \Sub{C}{cnfl} falsified in the current subspace
is derived. Adding \Sub{C}{cnfl} to $F$ makes $C$ redundant in the
current subspace. So, an atomic D-sequent $(\prob{X}{F},\pnt{p})
\rightarrow C$ is generated where \pnt{p} is the shortest assignment
falsifying \Sub{C}{cnfl}.

