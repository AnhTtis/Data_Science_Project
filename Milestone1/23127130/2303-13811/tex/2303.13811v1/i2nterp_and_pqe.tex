\section{PQE And Interpolation}
\label{app:interp}
In this appendix, we recall the observation of~\cite{tech_rep_pc_lor}
that interpolation is a special case of PQE. Let $A(X,Y) \wedge
B(Y,Z)$ be an unsatisfiable formula. Let $I(Y)$ be a formula such that
$A \wedge B \equiv I \wedge B$ and $A \imp I$. Then $I$ is called an
\ti{interpolant}~\cite{craig}.  Now, let us show that interpolation
can be described in terms of PQE. Consider the formula \prob{W}{A
  \wedge B} where $A$ and $B$ are the formulas above and $W = X \cup
Z$. Let $A^*(Y)$ be obtained by taking $A$ out of the scope of
quantifiers i.e. \mbox{\prob{W}{A \wedge B} $\equiv
  A^*\wedge$\prob{W}{B}}. Since $A \wedge B$ is unsatisfiable, $A^*
\wedge B$ is unsatisfiable too. So, \mbox{$A\wedge B \equiv A^* \wedge
  B$}. If $A \imp A^*$, then $A^*$ is an interpolant.

The \ti{general case} of PQE that takes $A$ out of \prob{W}{A \wedge
  B} is different from the instance above in three aspects. First, one
does not assume that $A \wedge B$ is unsatisfiable.  Second, one does
not assume that $\V{B} \subset \V{A \wedge B}$. In other words, in
general, PQE \ti{does not} remove any variables from the original
formula.  Third, a solution $A^*$ is implied by $A \wedge B$ rather
than by $A$ alone.  Summarizing, one can say that interpolation is a
special case of PQE.



 
