\section{Proofs Of Propositions}
 \setcounter{proposition}{0}
 \label{app:proofs}
 %
 % proposition claiming that a solution is implied by 
 % the formula
 %
\begin{proposition}
%\label{prop:sol_impl}
Let $H$ be a solution to the PQE problem of
Definition~\ref{def:pqe_prob}.  That is $\prob{X}{F}\equiv
H\wedge\prob{X}{F \setminus G}$. Then $F \imp H$ (i.e. $F$ implies
$H$).
\end{proposition}
%
%
\begin{proof}
By conjoining both sides of the equality with $H$ one concludes
that\linebreak $H \wedge \prob{X}{F} \equiv
H\wedge\prob{X}{F \setminus G}$ and hence
$H \wedge \prob{X}{F} \equiv \prob{X}{F}$. Then $\prob{X}{F} \imp H$
and thus $F \imp H$.
\end{proof}
%
% proposition about a blocked clause
%
\begin{proposition}
%\label{prop:blk_cls}
Let a clause $C$ be blocked in a formula $F(X,Y)$ with respect to a
variable $x \in X$.  Then $C$ is redundant in \prob{X}{F}
i.e. \prob{X}{F \setminus \s{C}} $\equiv$ \prob{X}{F}.
\end{proposition}
%
%
\begin{proof}
It was shown in~\cite{blocked_clause} that adding a clause $B(X)$
blocked in $G(X)$ to the formula \prob{X}{G} does not change the value
of this formula.  This entails that removing a clause $B(X)$ blocked
in $G(X)$ does not change the value of \prob{X}{G} either. So, $B$ is
redundant in \prob{X}{G}. Let \pnt{y} be a full assignment to
$Y$. Then the clause $C$ is either satisfied by \pnt{y} or \cof{C}{y}
is blocked in \cof{F}{y} with respect to $x$. (The latter follows from
the definition of a blocked clause.) In either case \cof{C}{y} is
redundant in \prob{X}{\cof{F}{y}}. Since \cof{C}{y} is redundant in
\prob{X}{\cof{F}{y}} in every subspace \pnt{y}, $C$ is redundant in
\prob{X}{F}.
\end{proof}
%
% proposition about symbolic simulation
%
\begin{proposition}
\label{prop:symb_sim}
Let $M(X,V,W)$ be a combinational circuit where $X,V,W$ are the
internal, input and output variables. Let $F(X,V,W)$ be a formula
specifying $M$. Let $B(W)$ be a clause.  Let $H(V)$ be a formula
obtained by taking a clause $C \in F$ out of \Prob{X}{W}{F \wedge
  B}. That is \Prob{X}{W}{F \wedge B} $\equiv H \wedge$ \Prob{X}{W}{(F
  \setminus \s{C}) \wedge B}. Let $Q(V)$ be a clause of $H$. Then for
every full assignment \pnt{v} to $V$ falsifying $Q$, the circuit $M$
outputs an assignment \pnt{w} falsifying the clause $B$.
\end{proposition}
%
%
%
\begin{proof}
  From Proposition~\ref{prop:sol_impl} it follows that $F \wedge B
  \imp H$ and hence $F \wedge B \imp Q$. This entails that
  $\overline{Q} \imp \overline{B} \vee \overline{F}$. Let \pnt{v} be a
  full assignment to $V$ i.e. an input to $M$. Let
  (\pnt{x},\pnt{v},\pnt{w}) be the execution trace produced by $M$
  under the input \pnt{v}. Here \pnt{x},\pnt{w} are full assignments
  to $X$ and $W$ respectively. Suppose, \pnt{v} satisfies
  $\overline{Q}$ (and so falsifies $Q$). Then
  (\pnt{x},\pnt{v},\pnt{w}) satisfies $\overline{B} \vee
  \overline{F}$.  Since (\pnt{x},\pnt{v},\pnt{w}) is an execution
  trace, it satisfies $F$ and so falsifies $\overline{F}$.  This
  entails that (\pnt{x},\pnt{v},\pnt{w}) (and specifically \pnt{w})
  satisfies $\overline{B}$ and hence falsifies $B$.
\end{proof}

