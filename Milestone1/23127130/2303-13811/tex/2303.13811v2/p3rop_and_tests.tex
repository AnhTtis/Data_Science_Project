\section{Tests Specified By A Property Generated By PQE}
\label{app:tests_props}
In this appendix, we show the relation between tests specified by a
property obtained via PQE (see Subsection~\ref{ssec:tests_props}) and
those detecting stuck-at faults.  Here, we reuse the notation of
Section~\ref{sec:prop_gen}. Let $M(X,V,W)$ be a combinational circuit
where $X,V,W$ are the internal, input and output variables
respectively.  Let $F(X,V,W)$ be a formula specifying $M$.
%
Let $G$ be an AND gate of $M$ whose functionality is $x_3 = x_1 \wedge
x_2$. That is $x_1,x_2$ are the input variables of $G$ and $x_3$ is
its output variable.  The functionality of $G$ is specified by the
formula $C_1 \wedge C_2 \wedge C_3$ where $C_1 = \overline{x}_1 \vee
\overline{x}_2 \vee x_3$, $C_2 = x_1 \vee \overline{x}_3$, $C_3 = x_2
\vee \overline{x}_3$ (see Example~\ref{exmp:gate_cnf}). The clauses
$C_1,C_2,C_3$ are present in formula $F$. Consider taking $C_1$ out of
\prob{X}{F}. This clause makes $G$ produce the output value 1 when its
input values are 1.  (If $x_1$ and $x_2$ are set to 1, the clause
$C_1$ can be satisfied only by setting $x_3$ to 1.)

Let $H(V,W)$ be the property obtained by taking out $C_1$. That
is\linebreak $\prob{X}{F} \equiv H \wedge \prob{X}{F \setminus
  \s{C_1}}$. Let $Q(V,W)$ be a clause of $H$. As we mentioned earlier,
we assume that $H$ does not have redundant clauses i.e. those implied
by $F \setminus \s{C_1}$. Then the formula $(F \setminus \s{C_1})
\wedge \overline{Q}$ is satisfiable. Let (\pnt{x},\pnt{v},\pnt{w}) be
an assignment satisfying this formula. Note that this assignment
\ti{falsifies} $C_1$.  (Indeed, assume the contrary. Then
(\pnt{x},\pnt{v},\pnt{w}) satisfies $F$ because it already satisfies
$F \setminus \s{C_1}$. Since this assignment falsifies $Q$, we have to
conclude that $F \not\imp Q$ and hence $F \not\imp H$. So we have a
contradiction.)

The fact that (\pnt{x},\pnt{v},\pnt{w}) falsifies $C_1$ and satisfies
$F \setminus \s{C_1}$ means that one can view this assignment as an
execution trace of a faulty version \Sub{M}{flt} of $M$.  Namely, the
output $x_3$ of gate $G$ is stuck at 0 in \Sub{M}{flt}. (The clause
$C_1$ is falsified when $x_1=1,x_2=1,x_3=0$ i.e. if the gate $G$
outputs 0 when its input variables are assigned 1.)  Let
$(\pnt{x}^*,\pnt{v},\pnt{w}^*)$ be the execution trace of $M$ under
the input \pnt{v}. As we showed in Subsection~\ref{ssec:tests_props},
$\pnt{w}^*$ is different from \pnt{w}. So the input \pnt{v} exposes a
stuck-at fault by making \Sub{M}{flt} and $M$ produce different
outputs.
