\vspace{-5pt}
\section{Experiment With Combinational Circuits}
\label{app:symb_sim}
In this appendix, we give more information about the experiment with
property generation for combinational circuits described in
Section~\ref{sec:comb_exper}. We formed PQE problems as follows.  For
each benchmark $N$ we picked the number $k$ of time frames to unroll.
The value of $k$ ranged from 10 to 40. (For larger circuits we picked
a smaller value of $k$.) Then we unrolled $N$ for $k$ time frames to
form a combinational circuit $M_k$ and randomly generated a clause
$B(S_k)$ of 15 literals. So, $B$ depended on output variables of
$M_k$. After that, we constructed the subcircuit $M'_k$ of $M_k$ as
described in Section~\ref{sec:comb_exper}. That is, $M'_k$ was
obtained by removing the logic of $M_k$ that did not feed any output
variable present in $B$.

Let formula $F'_k$ specify the subcircuit $M'_k$. (Here we reuse the
notation of Section~\ref{sec:comb_exper}.)  For every benchmark, we
generated PQE problems of taking different clauses $C$ out of
\prob{S_{1,k}}{F'_k}. That is each PQE problem was to find $H$ such
that $\prob{S_{1,k}}{F'_k} \equiv H \wedge \prob{S_{1,k}}{F'_k
  \setminus \s{C}}$. Each clause $C$ to take out was chosen among the
clauses of $F'_k$ that contained a variable of $S_k$ (i.e. an output
variable of $M'_k$). In this way, we formed a set of 3,254 PQE
problems.  1,668 of these problems were solved by simple formula
preprocessing i.e. turned out to be trivial.  So, in the experiment we
used the remaining 1,586 non-trivial PQE problems.

%
% results of property generation
%
\vspace{5pt}
\begin{wraptable}{l}{2.3in}
%\begin{table}
\centering
%\small
  % \vspace{15pt}
\scriptsize
\captionsetup{justification=centering}
\caption{\small{Property generation}}
\vspace{-5pt}
%\begin{center}
%\renewcommand{\arraystretch}{1.2} % Default value: 1
%\begni{tabular}{|p{22pt}|p{36pt}|c|c|c|c|} \hline
  \begin{tabular}{|p{29pt}|p{22pt}|p{20pt}|p{20pt}|p{20pt}|p{20pt}|} \hline
 pqe & num.  & \multicolumn{4}{c|}{properties were generated}   \\ \cline{3-6}
 solver         & of pqe  &num-   &   &  \multicolumn{2}{c|}{stronger than}     \\
                &prob-    &ber     &~~\%      & \multicolumn{2}{c|}{3-val. sim.} \\ \cline{5-6}
                &lems       &   &   & num. & ~~\%  \\ \hline
\ti{ds-pqe}     & 1,586 & 983 &~~62 &~728  &~~74    \\ \hline
\ti{eg-pqe}     & 1,586 &  450 &~~28 &~361 &~~\tb{80}   \\ \hline
\ti{eg-pqe}$^+$ & 1,586 & \tb{1,046}  &~~\tb{66} &~\tb{817} &~~78   \\ \hline 
\end{tabular}                
%\end{center}
%\vspace{10pt}
\label{tbl:all_prop_gen}
%\end{table}
\end{wraptable}







The time limit for solving a PQE problem was set to 10 sec. Besides,
solving a PQE problem terminated as soon as the size of $H$ reached 5
clauses.  The results of the experiment are summarized in
Table~\ref{tbl:all_prop_gen}. The second column gives the total number
of PQE problems.  The next two columns show the number and percentage
of problems where $H$ was non-empty i.e. had at least one
clause. (Recall, that each clause of $H$ represents a property.) The
last two columns give the number and percentage of cases where a
clause of $H$ represented a property that was stronger than ones
produced by 3-valued simulation, a version of symbolic
simulation~\cite{SymbolSim}.  Consider, for instance, the last line of
the table corresponding to \egp.  For 817 out of 1,046 PQE problems
where $H$ was not empty, at least one clause of $H$ constituted a
property that could not be produced by 3-valued
simulation. Table~\ref{tbl:all_prop_gen} shows that \Eg had the
weakest results generating properties only for 28\% of problems
whereas \dpqe and \egp performed much better producing properties for
62\% and 66\% problems respectively.
