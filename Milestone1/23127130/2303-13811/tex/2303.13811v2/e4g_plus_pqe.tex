\section{Introducing \egp}
\label{sec:eg_pqe+}
In this section, we describe \egp, an improved version of \Eg.  
%
% subsection
%
\vspace{-7pt}
\subsection{Main idea}
%
The pseudocode of \egp is shown in Fig~\ref{fig:eg_pqe+}. It is
different from that of \Eg only in line 11 marked with an asterisk.
The motivation for this change is as follows.  Line 11 describes
proving redundancy of $C$ for the case where \cof{C}{y} is not implied
by \cof{(F \setminus \s{C})}{y} and \cof{F}{y} is satisfiable. Then
\Eg simply uses a satisfying assignment as a proof of redundancy of
$C$ in subspace \pnt{y}. This proof is unnecessarily strong because it
proves that \ti{every} clause of $F$ (including $C$) is redundant in
\prob{X}{F} in subspace \pnt{y}. Such a strong proof is hard to
generalize to other subspaces.

\input{e7g_pqe+.fig}

The idea of \egp is to generate a proof for a much weaker proposition
namely a proof of redundancy of $C$ (and only $C$). Intuitively, such
a proof should be easier to generalize. So, \egp calls a procedure
\ti{PrvClsRed} generating such a proof. \egp is a generic algorithm in
the sense that \ti{any} suitable procedure can be employed as
\ti{PrvClsRed}. In our current implementation, the procedure
\dpqe~\cite{hvc-14} is used as \ti{PrvClsRed}. \dpqe generates a proof
stating that $C$ is redundant in \prob{X}{F} in subspace $\pnt{y}^*
\subseteq \pnt{y}$.  Then the plugging clause $D$ falsified by
$\pnt{y}^*$ is generated. Importantly, $\pnt{y}^*$ can be much shorter
than \pnt{y}. Appendix~\ref{app:ds_pqe} gives a brief description of
\dpqe. 

%
%  EXAMPLE
%
\begin{example}
\label{exmp:eg_pqe+}
Consider the example solved in Subsection~\ref{ssec:exmp}.  That is,
we consider taking clause $C_1$ out of \prob{X}{F(X,Y)} where $F = C_1
\wedge \dots \wedge C_4$, $C_1=\overline{x}_3 \vee x_4$,
$C_2\!=\!y_1\!\vee\!x_3$, $C_3=y_1 \vee \overline{x}_4$,
$C_4\!=\!y_2\!\vee\!x_4$ and $Y=\s{y_1,y_2}$ and $X=\s{x_3,x_4}$.
Consider the step where \Eg proves redundancy of $C_1$ in subspace
$\pnt{y}=(y_1\!=\!1,y_2\!=\!1)$.  \Eg shows that
$(y_1\!=\!1,y_2\!=\!1,\!x_3=0)$ satisfies $F$, thus proving every
clause of $F$ (including $C_1$) redundant in \prob{X}{F} in subspace
\pnt{y}. Then \Eg generates the plugging clause $D = \overline{y}_1
\vee \overline{y}_2$ falsified by \pnt{y}.

In contrast to \Eg, \egp calls \ti{PrvClsRed} to produce a proof of
redundancy for the clause $C_1$ alone.  Note that $F$ has no clauses
resolvable with $C_1$ on $x_3$ in subspace $\pnt{y}^* = (y_1 =
1)$. (The clause $C_2$ containing $x_3$ is satisfied by $\pnt{y}^*$.)
This means that $C_1$ is blocked in subspace $\pnt{y}^*$ and hence
redundant there (see Proposition~\ref{prop:blk_cls}). Since $\pnt{y}^*
\subset \pnt{y}$, \egp produces a more general proof of redundancy
than \Eg. To avoid re-examining the subspace $\pnt{y}^*$, \egp
generates a \ti{shorter} plugging clause $D = \overline{y}_1$.
\end{example}


%
% Subsection
%
\subsection{Discussion}
\label{ssec:disc2}
Consider the PQE problem of taking a clause $C$ out of
\prob{X}{F(X,Y)}.  There are two features of PQE that make it easier
than QE.  The first feature mentioned earlier is that one can ignore
the subspaces \pnt{y} where $F \setminus \s{C}$ implies $C$. The
second feature is that when \cof{F}{y} is satisfiable, one only needs
to prove redundancy of the clause $C$ alone.  Among the three
algorithms we run in experiments, namely, \dpqe, \Eg, and \egp only
the latter exploits both features. (In addition to using \dpqe inside
\egp we also run it as a stand-alone PQE solver.)  \dpqe does not use
the first feature~\cite{hvc-14} and \Eg does not exploit the second
one. As we show in Sections~\ref{sec:fifo_exper}
and~\ref{sec:inv_gen_exper}, this affects the performance of \dpqe and
\Eg.
