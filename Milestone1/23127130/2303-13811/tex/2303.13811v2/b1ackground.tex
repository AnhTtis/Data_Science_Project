%   Plan
%   *) BDDs
%   *) SAT
%   *) Symmetry, circuits and preprocessing
%\pagebreak
%\vspace{-10pt}
\section{Some Background}
\label{sec:bg}
In this section, we discuss some research relevant to PQE and property
generation.  Information on BDD based QE can be found
in~\cite{bryant_bdds1,bdds_qe}. SAT based QE is described in
\cite{blocking_clause,fabio,cofactoring,cav09,cav11,cmu,nik1,nik2,cadet_qe}.
Our first PQE solver called \dpqe was introduced in~\cite{hvc-14}. It
was based on redundancy based reasoning presented in~\cite{fmcad12} in
terms of variables and in~\cite{fmcad13} in terms of clauses. The main
flaw of \dpqe is as follows. Consider taking a clause $C$ out of
\prob{X}{F}. Suppose \dpqe proved $C$ redundant in a subspace where
$F$ is \ti{satisfiable} and some \ti{quantified} variables are
assigned. The problem is that \dpqe cannot simply assume that $C$ is
redundant every time it re-enters this subspace~\cite{qe_learn}. The
root of the problem is that redundancy is a \ti{structural} rather
than semantic property. That is, redundancy of a clause in a formula
$\xi$ (quantified or not) does not imply such redundancy in every
formula logically equivalent to $\xi$. Since our current
implementation of \egp uses \dpqe as a subroutine, it has the same
learning problem.  We showed in~\cite{cert_tech_rep} that this problem
can be addressed by the machinery of certificate clauses. So, the
performance of PQE can be drastically improved via enhanced learning
in subspaces where $F$ is satisfiable.


We are unaware of research on property generation for combinational
circuits. As for invariants, the existing procedures typically
generate some auxiliary \ti{desired} invariants to prove a predefined
property (whereas our goal is to generate invariants that are
\ti{unwanted}). For instance, they generate loop
invariants~\cite{loop_invars} or invariants relating internal points
of circuits checked for equivalence ~\cite{ec_invars}. Another example
of auxiliary invariants are clauses generated by \ict to make an
invariant inductive~\cite{ic3}.  As we showed in
Subsection~\ref{ssec:ic3_invars}, the invariants produced by PQE are,
in general, different from those built by \ict.
