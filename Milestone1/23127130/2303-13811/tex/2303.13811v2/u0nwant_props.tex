\section{Deciding If A Property Is Unwanted}
\label{app:unw_props}
In this appendix, we give the main idea of a procedure for deciding if
a property is unwanted. We reuse the notation of
Section~\ref{sec:prop_gen}. That is, we consider property generation
for a combinational circuit $M(X,V,W)$ specified by formula
$F(X,V,W)$. Here $X,V,W$ denote the sets of the internal, input, and
output variables of $M$ respectively. For the sake of simplicity, we
consider generation of a property $H(W)$ depending only on output
variables of $M$. Such a property can be obtained by taking a clause
$C$ out of formula \Prob{X}{V}{F} (where only variables of $W$ are not
quantified).  So, $\Prob{X}{V}{F} \equiv H \wedge \Prob{X}{V}{F
  \setminus \s{C}}$.

Note that every clause of $H$ is a property too.  $H$ is an unwanted
property if and only if one of its clauses is an unwanted property.
Let $Q \in H$ be a single-clause unwanted property (and so $M$ has a
bug).  This property says that $M$ never outputs an assignment
\ti{falsifying} $Q$. To prove $Q$ unwanted one needs to find a test
\pqnt{v}{be} for which a correct version of $M$ would produce an
output falsifying $Q$. Here 'be' stands for 'bug-exposing'. (If $Q$ is
a \ti{desired} property, such \pqnt{v}{be} cannot be produced. So, the
failure to find \pqnt{v}{be} is an argument in favor of $Q$ being a
desired property.)

A test \pqnt{v}{be} can be found as follows. Since $F$ implies $Q$,
the formula $F \wedge \overline{Q}$ is unsatisfiable (and so $M$
cannot produce an output falsifying $Q$). Let $F'$ be a formula
obtained by removing some clauses from $F$ for which the formula $F'
\wedge \overline{Q}$ is satisfiable. Then means that the circuit $M'$
specified by $F'$ can produce an output falsifying $Q$. The intuition
here is that if $Q$ is unwanted property, the clauses removed from $F$
to obtain $F'$ relate to the buggy part of $M$. (Note that the circuit
$M'$ specified by $F'$ is non-deterministic. The reason is that $F'$ can
be satisfied by assignments (\ppnt{x}{1},\pnt{v},\ppnt{w}{1}) and
(\ppnt{x}{2},\pnt{v},\ppnt{w}{2}) with the same input \pnt{v} and
different outputs \ppnt{w}{1} and \ppnt{w}{2}).


Assume that \ti{all clauses} making up the buggy part of $M$ has been
removed from $F$ when obtaining $F'$. Then there is an assignment
(\pnt{x},\pqnt{v}{be},\pnt{w}) satisfying $F' \wedge
\overline{Q}$. That is a correct version of $M$ would produce the
assignment \pnt{w} falsifying $Q$ under the input \pqnt{v}{be}. So, to
obtain \pqnt{v}{be} one needs to build the formula $F'$ above and
examine the input part \pnt{v} of assignments satisfying $F' \wedge
\overline{Q}$.

The formula $F'$ can be obtained as follows. Recall that by our
assumption $F \setminus \s{C} \not\imp Q$ (see
Subsection~\ref{ssec:pg_gnrl_tsts} and Remark~\ref{rem:noise}).  So
the formula $F' \wedge \overline{Q}$ where $F' = F \setminus \s{C}$ is
satisfiable. However, one may not extract a test \pqnt{v}{be} from
assignments satisfying $F' \wedge \overline{Q}$ because $C$ may not be
the only clause making up the buggy part of $M$. A good heuristic for
forming $F'$ is to remove from $F$ every clause $B$ such that $F
\setminus \s{B} \not\imp Q$. (The latter means that one could obtain
the property $Q$ by taking out the clause $B$ from
\Prob{X}{V}{F}). However, the topic of obtaining formula $F'$ and
producing a test \pqnt{v}{be} needs further research.


