%\vspace{-10pt}
\section{Conclusions And Directions For Future Research}
\label{sec:concl}
%\vspace{-10pt}
We consider Partial Quantifier Elimination (PQE) on propositional CNF
formulas with existential quantifiers. In contrast to \ti{complete}
quantifier elimination, PQE allows to unquantify a \ti{part} of the
formula.  We show that PQE can be used to generate properties of
combinational and sequential circuits. The goal of property generation
is to check if there is an \ti{unwanted} property identifying a bug.
We used PQE to generate an unwanted invariant for a FIFO buffer
exposing a non-trivial bug.  We also applied PQE to invariant
generation for HWMCC benchmarks. Finally, we used PQE to generate
properties of combinational circuits mimicking symbolic simulation.
Our experiments show that PQE can efficiently generate properties for
realistic designs.

There are at least three directions for future research. The first
direction is to improve the performance of PQE solving. As we
mentioned in Section~\ref{sec:bg}, the most promising idea here is to
enhance the power of learning in subspaces where the formula is
satisfiable.  The second direction is to use the improved PQE solvers
to design new, more efficient algorithms for well-known problems like
SAT, model checking and equivalence checking. The third direction is
to look for new problems that can be solved by PQE.
