\vspace{-7pt}
\subsection{Experiment 3}
\label{ssec:exper3}
To prove an invariant $P$ true, \ict conjoins it with clauses
$Q_1,\dots,Q_n$ to make $P\wedge Q_1\!\wedge \dots \wedge Q_n$
inductive.  If \ict succeeds, every $Q_i$ is an invariant. Moreover,
$Q_i$ may be an \ti{unwanted} invariant.  Arguably, the cause of
efficiency of \ict is that $P$ is often close to an inductive
invariant. So, \ict needs to generate a relatively small number of
clauses $Q_i$ to make the constrained version of $P$
inductive. However, this nice feature of \ict drastically limits the
set of invariant clauses it generates. In this subsection, we
substantiate this claim by an experiment. In this experiment, we
picked the HWMCC-13 benchmarks for which one could prove \ti{all}
predefined invariants $P_1, \dots, P_m$ within a time limit. Namely,
for every benchmark we formed the aggregate invariant \mbox{\spe =
  $P_1 \wedge \dots \wedge P_m$} and ran \ict to prove \spe true.

We selected the benchmarks that \ict solved in less than 1,000 sec.
(In addition to dropping the benchmarks not solved in 1,000 sec., we
discarded those where \spe failed because some invariants $P_i$ were
false). Let \bm{\sspe} denote \smallskip the inductive version of \spe
produced by \ict when proving \spe true.  That is, $\sspe$ is \spe
conjoined with the invariant clauses produced by \ict.

\begin{wraptable}{l}{2.4in}
%\begin{table}[h]
\centering
%  \small
\vspace{5pt}
\scriptsize
\caption{\small{Invariants of \egp and \ict}}
%\caption{\small{A sample of HWMCC-13 benchmarks. The time limit is 5\,sec. per PQE problem}}
\vspace{2pt}
%\begin{center}
%\renewcommand{\arraystretch}{1.2} % Default value: 1
%\begni{tabular}{|p{22pt}|p{36pt}|c|c|c|c|} \hline
\begin{tabular}{|p{20pt}|p{20pt}|p{22pt}|p{20pt}|p{25pt}|p{25pt}|} \hline
  name     & lat- & inva-  &\multicolumn{3}{c|}{glob. single cls. invars} \\ \cline{4-6}
           & ches &riants    & glob.    & not & not   \\
           &      & of    & inva-   & impl.   & impl.  \\
           &      & \spe &  riants & by $\spe$  &by \sspe  \\ \hline
  6s135  & 2,307  &~340 &~53 &~53    &~27 \\ \hline
  6s325  & 1,756  &~301 &~99 &~99   &~96\\ \hline
ex1    & 130    &~33  &~25 &~16    &~16 \\ \hline
ex2    & 212    &~32  &~64  &~64    &~47 \\ \hline
6s106  & 135    &~17  &~100   &~96    &~96 \\ \hline
6s256  & 3,141  &~5   &~13  &~13    &~13  \\ \hline
ex3    & 61     &~3   &~4  &~4    &~4  \\ \hline
ex4    & 63     &~3   &~1  &~1    &~1  \\ \hline
6s209  & 5,759  &~2   &~95 &~95   &~89 \\ \hline
6s113  & 994    &~1   &~19 &~16   &~16 \\ \hline
6s143  & 260    &~1   &~97  &~86   &~77  \\ \hline
6s170  & 3,141  &~1   &~13  &~13    &~13  \\ \hline
6s252  & 170   &~1   &~54  &~41   &~34 \\ \hline \hline
\tb{Total}  &       &     &     &  \tb{597}  & \tb{529} \\ \hline
\end{tabular}
%\end{center}
\vspace{5pt}
\label{tbl:ic3_pqe}
%\end{table}
\end{wraptable}
%The actual names of examples \ti{ex1},..,\ti{ex4} in the
%HWMCC-13 set are \ti{pdtvsarmultip}, \ti{bobtuintmulti},
%\ti{nusmvdme1d3multi}, \ti{nusmvdme2d3multi}.)


For each of the selected benchmarks we generated invariants by \egp
exactly as in Experiment 2. That is, we stopped generation of local
single clause invariants when their number exceeded 100. Then we ran
\ict to identify local invariants that were global as well.  After
that we checked which of the global invariants generated by \egp were
not implied by \spe. The difference from Experiment 2 was that we also
checked which global invariants generated by \egp were not implied by
$\sspe$.


The results of the experiment are shown in Table~\ref{tbl:ic3_pqe}.
The first three columns of this table are the same as in
Table~\ref{tbl:sample}. They give the name of a benchmark, the number
of latches and the number of invariants $P_1$,$\dots$,$P_m$ to
prove. (The actual names of examples \ti{ex1},..,\ti{ex4} in the
HWMCC-13 set are \ti{pdtvsarmultip}, \ti{bobtuintmulti},
\ti{nusmvdme1d3multi}, \ti{nusmvdme2d3multi} respectively.) The next
column of Table~\ref{tbl:ic3_pqe} shows the number of local invariants
generated by \egp that turn out to be global.  The last two columns
give the number of global invariants that were not implied by \spe and
\sspe respectively. The last row of the table shows that in 522 cases
out of 590 the invariants not implied by \spe were not implied by
\sspe either. So, in 88\% of cases, the invariant clauses generated by
\egp were \ti{different} from those generated by \ict to form \sspe.
%\clearpage

