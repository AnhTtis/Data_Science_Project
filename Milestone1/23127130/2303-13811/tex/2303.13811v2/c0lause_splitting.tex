\section{Combining PQE With Clause Splitting}
\label{app:cls_split}
In this appendix, we consider combining PQE with clause splitting
mentioned in Subsection~\ref{ssec:eff_comp}. We show that the
corresponding PQE problem of taking out a clause produced by splitting
is solved by \Eg in linear time.  We also show that if this clause is
not redundant, the solution produced by \Eg is a single-test
property.


Here we reuse the notation of Section~\ref{sec:prop_gen} but, for the
sake of simplicity, consider a single-output combinational circuit.
Let $M(X,V,w)$ be such a circuit where $X$ and $V$ specify the
internal and input variables respectively and $w$ is the output
variable of $M$. Let $F(X,V,w)$ be a formula specifying the circuit
$M$ and $C$ be a clause of $F$. Consider the case of
splitting \smallskip $C$ on \ti{all} variables of $V$. That is $C =
C_1 \wedge \dots \wedge C_{p+1}$ where $C_1 = C \vee
\overline{l(v_1)}$,\dots,\,$C_p = C \vee \overline{l(v_p)}$,\,$C_{p+1}
= C \vee l(v_1) \vee \dots \vee l(v_p)$ and $l(v_i)$ is a literal of
$v_i$ and $V = \s{v_1,\dots,v_p}$. Let $F'$ denote the formula
obtained from $F$ by replacing the clause $C$ with $C_1 \wedge \dots
\wedge C_{p+1}$. Denote by \bm{\pqnt{v}{spl}} the input assignment
falsifying the literals $l(v_1),\dots,l(v_p)$ where 'spl' stands for
'splitting'.

Consider applying \Eg to solve the PQE problem of taking the clause
$C_{p+1}$ out of \prob{X}{F'}. \smallskip \Eg starts with looking for
an assignment satisfying\linebreak $(F' \setminus \s{C_{p+1}}) \wedge
\overline{C_{p+1}}$ (to find a subspace where $F' \setminus
\s{C_{p+1}}$ does not imply $C_{p+1}$). Consider the following three
cases. The first case is that the formula above is unsatisfiable. Then
$C_{p+1}$ is trivially redundant in $F'$ and hence in \prob{X}{F'} and
\Eg terminates.

The second case is that there is an assignment
($\pnt{x},\pqnt{v}{spl},w^*$) satisfying $F'$ where \pnt{x} is a full
assignment to $X$ and $w^*$ is the output value taken by $M$ under the
input \pqnt{v}{spl}.  (Note \smallskip that any full assignment to $V$
that is different from \pqnt{v}{spl} falsifies
$\overline{C_{p+1}}$. So, any assignment satisfying $(F' \setminus
\s{C_{p+1}}) \wedge \overline{C_{p+1}}$ has to contain \pqnt{v}{spl}.)
Then formula $F'$ is satisfiable in subspace $(\pqnt{v}{spl},w^*)$ and
\Eg adds the plugging \smallskip clause $D(V,w)$ that is the longest
clause falsified by $(\pqnt{v}{spl},w^*)$. If $(F' \setminus
\s{C_{p+1}}) \wedge \overline{C_{p+1}} \wedge D$ is unsatisfiable,
then $C_{p+1}$ is redundant in \prob{X}{F'} and \Eg terminates.

The third case \smallskip occurs when there is an assignment
($\pnt{x},\pqnt{v}{spl},w^*$) satisfying\linebreak $(F' \setminus
\s{C_{p+1}}) \wedge \overline{C_{p+1}}$ where $w^*$ is the
\ti{negation} of the output value taken by $M$ under input
\pqnt{v}{spl}.  In this case, formula $F'$ is unsatisfiable in
subspace $(\pqnt{v}{spl},w^*)$. Since, $F' \setminus \s{C_{p+1}}$ is
satisfiable in this subspace, $C_{p+1}$ is \ti{not} redundant in
\prob{X}{F'}. To \ti{make} $C_{p+1}$ redundant in subspace
$(\pqnt{v}{spl},w^*)$, \Eg has to add the clause $B(V,w)$ that is the
longest clause falsified by $(\pqnt{v}{spl},w^*)$. The clause $B$ is a
solution to the PQE problem at hand i.e. $\prob{X}{F'} \equiv B \wedge
\prob{X}{F' \setminus \s{C_{p+1}}}$.

The clause $B$ above is implied by $F'$ (and hence $F$) and so, is a
\tb{property} of $M$. This property specifies the input/output
behavior of $M$ under the input \pqnt{v}{spl}. Namely, to satisfy $B$
when the variables of $V$ are assigned as in \pqnt{v}{spl}, one has to
set the variable $w$ to $\overline{w^*}$. The latter is the output
produced by $M$ under the input \pqnt{v}{spl}. So, the property $B$
specifies the behavior of $M$ under a \tb{single test}. In all three
cases above, the SAT problem considered by \Eg is solved just by
initial BCP. (The reason is that the formula at hand contains the unit
clauses produced by negating $C_{p+1}$ or those specifying the
subspace $(\pqnt{v}{spl},w^*)$.) So, \Eg solves the PQE problem above
in \tb{linear} time.
