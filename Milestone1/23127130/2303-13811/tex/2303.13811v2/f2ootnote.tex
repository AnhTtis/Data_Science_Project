Let $P(\hat{S})$ be an invariant for a circuit $N$ depending only on a
subset $\hat{S}$ of the state variables $S$. Identifying $P$ as an
unwanted invariant is much easier if $\hat{S}$ is meaningful from the
high-level view of the design.  Suppose, for instance, that
assignments to $\hat{S}$ specify values of a high-level variable
$v$. Then $P$ is unwanted if it claims unreachability of a value of
$v$ that is supposed to be reachable. Another simple example is that
assignments to $\hat{S}$ specify values of high-level variables $v$
and $w$ that are supposed to be \ti{independent}. Then $P$ is unwanted
if it claims that some combinations of values of $v$ and $w$ are
unreachable. (This may mean, for instance, that an assignment operator
setting the value of $v$ erroneously involves the variable $w$.)
