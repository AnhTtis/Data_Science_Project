%
\documentclass[%
reprint,
10pt,
%superscriptaddress,
%groupedaddress,
%unsortedaddress,
%runinaddress,
%frontmatterverbose, 
%preprint,
%preprintnumbers,
nofootinbib,
%nobibnotes,
%bibnotes,
%footinbib,
amsmath,amssymb,
aps,
%pra,
%prb,
%rmp,
%prstab,
%prstper,
%floatfix,
]{revtex4-2}

\usepackage{graphicx}% Include figure files
\usepackage{dcolumn}% Align table columns on decimal point
\usepackage{bm}% bold math
\usepackage{lipsum}
\usepackage{xfrac}
\usepackage{bigints}
%\usepackage{widetext}
%\usepackage{graphicx}
%\usepackage{epstopdf, epsfig}
\usepackage{amsmath,amssymb} % easier math formulae, align, subequations \ldots
\usepackage{mathrsfs}
\usepackage{amsfonts}
\usepackage{amsmath,MnSymbol}
%\usepackage{amsart}
\usepackage{pgfplots}
\pgfplotsset{compat=1.5}
%\usepackage{subfig}
\usepackage{floatrow}
\usepackage{float}
%\usepackage{dblfloatfix} 
\usepackage{soul}
\usepackage{xcolor}
%\setcitestyle{super}
\usepackage[colorlinks=true,linkcolor=blue,citecolor=Green, urlcolor=teal]{hyperref}%
\usepackage{footmisc}
\usepackage{lineno}
\usepackage{footnote,caption}
\def\bibsection{\section*{\refname}} 
\captionsetup[figure]{labelfont=bf,labelformat={default},labelsep=period,name={\textbf{Fig.}},font=small,justification=Justified}


\def\IB#1{\boldsymbol{#1}} % use \IB{} for italic bold vector
\def\RE#1{\Re_{#1}} % macro for Reynolds number 
\def\W/!i#1{\Wi} % macro for Wi
\def\bten#1{\IB{\mathsf{#1}}}
\def\ang#1{{\langle {#1} \rangle}}  % angular brackets

%\addtolength{\oddsidemargin}{0.1in}
%\addtolength{\evensidemargin}{-0.1in}
%\addtolength{\textwidth}{-0.4in}
%\addtolength{\topmargin}{-0.2in}
%\addtolength{\textheight}{1.3in}

\definecolor{Green}{HTML}{167425}
\definecolor{azure}{HTML}{005FFF}
\newcommand{\ac}[1]{{\color{blue} #1}}
\newcommand{\AC}[1]{\textit{\small \color{azure}{AC: #1}}}
\newcommand{\hs}[1]{{\color{magenta} #1}}

%\renewcommand*{\thefootnote}{\fnsymbol{footnote}}

\begin{document}
	
	%	\linenumbers
	%	\modulolinenumbers[10]
	
	
	\preprint{APS/123-QED}
	
	\title{Orientational dynamics and rheology of  active suspensions in weakly viscoelastic flows}
	
	\author{Akash Choudhary\textsuperscript{1}}
	\homepage{a.choudhary@campus.tu-berlin.de}
	\author{Sankalp Nambiar\textsuperscript{2}}
	\author{Holger Stark\textsuperscript{1}}
	\homepage{holger.stark@tu-berlin.de}
	\affiliation{%
		\textsuperscript{1}Institute of Theoretical Physics, Technische Universit\"{a}t Berlin, Hardenbergstr. 36, 10623 Berlin, Germany\\
		\textsuperscript{2}Nordita, KTH Royal Institute of Technology and Stockholm University, Stockholm 10691, Sweden
	}%
	
	
	\begin{abstract}
		\noindent
		\textbf{Abstract.}
		Microswimmer suspensions are constantly in non-equilibrium and exhibit macroscale properties that are in stark 
		contrast to passive suspensions.  Motivated by ubiquitous microbial systems suspended in biological fluids, % 28
		we analyse the rheological response of 
		%active 
		%\hs{rod}
		%\\ \hs{HS: Spheroids are also rods, but rod is less technical. We need a simple language here.} \\
		%\footnote{\ac{At least one mention of elongation is needed}}
		a suspension of elongated microswimmers to a steady shear flow in a weakly viscoelastic fluid.
		%\\ \hs{HS: Otherwise to many "of"} \\
		% 49
		%
		At the individual level, 
		%\\ \hs{HS: The comma is optional according to grammerly, so we can skip it. The excessive use of commas in
			%English is bad style.} \\
		we find that the viscoelastic stresses generated by activity
		%\footnote{\ac{\st{elastic and active stresses} My mistake. We cannot write them as if they are different}
			substantially modify the Jeffery orbits well-known from Newtonian fluids. The orientational dynamics depends on the swimmer type and, in particular, microswimmers can  resist flow-induced rotation and align at an angle with the flow. %95 
			To analyze its impact on the bulk rheological response, we study a dilute ensemble of microswimmers in the presence of stochastic noise due to tumbling and  rotational diffusion.  % 123
			% 
			%\\\\
			Compared to Newtonian fluids, activity and its
			%the 
			elastic response 
			%of 
			in polymeric fluids alter the orientational distribution and substantially amplify the swimmer-induced viscosity. This suggests that pusher suspensions 
			reach the regime of superfluidity at lower volume fractions.   %159
		\end{abstract}
		\maketitle
		
		
		
		
		\section*{Introduction}
		
		%\hs{HS: I miss citations from our group} \\
		Systems of particulate matter suspended in fluids are prevalent in numerous natural and industrial processes.
		% \hs{[ref] let's see.} 
		Active suspensions are particulate systems that are driven out of equilibrium by converting chemical energy or fuel into mechanical work to achieve self-propulsion \cite{marchetti2013Review,zottl2016emergent}.
		Motile microorganisms are the prototypical example of
		active motion that is generated via metachronal actuation of hairlike cilia (\textit{Paramecium, Volvox}), by whipping cell-attached flagellar appendages (spermatozoa,  algae), or by rotating a bundle of helical flagella (bacteria) \cite{lauga2020fluid}.
		Microswimmers navigate through their environment by sensing or interacting with gradients in hydrodynamic, chemical, thermal, and light 
		fields; a strategy known as taxis \cite{berg1972chemotaxis,demir2012bacterial,stark2018artificial,mathijssen19,jing2020chirality,doan2020trace,ramamonjy2022nonlinear}.
		A particular example is rheotaxis, where microswimmers 
		experience hydrodynamic gradients and swim against the flow, which
		is relevant for biofilm formation and reproduction \cite{suarez2006sperm,zottl2012nonlinear, zottl2016emergent,jing2020chirality}.
		
		
		
		
		Pathogenic microswimmers when infiltrating human and animal bodies have to pass through mucus linings that lubricate 
		and protect our respiratory tracks, eyes, urogenital and gastrointestinal systems \cite{sznitman2015locomotion}. 
		The presence of mucin fibres (typically 3-10 nm in length \cite{mucus_length_shogren1989}) and DNA  makes these mucus linings viscoelastic and shear thinning \cite{mucus_lai2009micro}. 
		%These mucus systems 
		%\ac{These} 
		The 
		%\hs{HS: to much these ... and it is clear by now what linings mean}
		linings are often subjected to shearing motion, for instance, during blinking, coughing, reproduction, and continuos mucociliary 
		clearance in respiratory systems.
		In such microbiological flows, the non-Newtonian fluid properties can alter the swimmer's rheotactic behaviour \cite{mathijssen2016upstream,choudhary2022soft}.
		
		%Bacteria are also known to render Newtonian mediums complex \cite{conrad2018confined,li2021microswimming}. During the production of biofilms,  bacteria can release a mixture of proteins and polysaccharides, which endows the fluid with viscoelastic properties \cite{conrad2018confined}.
		
		%{\footnotesize
			%	\\ \hs{HS: I would not take this as the prime example why we consider viscoelastic enviroments. There should be more direct examples, where the fluid is already complex without that the microswimmer releases biomolecules into the fluid.} \AC{Yes, this can be secondary example. As a prime example we can state state that biological fluids are already complex due to their constituents.}}
		
		%\hs{HS: I am missing here older experimental work by Aronson and Peyla about this subject. So therefore I am not
			%sure about the logic "experiments...have confirmed...theory". We need to be not complete but cite the relevant literature here.} \\
		
		%Microswimmers also exhibit bulk scale modifications to their surroundings. 
		%\footnote{\AC{Should we mention at atleast one place that we do the bulk rheology, as opposed to microrheology?}}
		In this article we study the bulk rheology
		%\\ \hs{in principle, I find bulk unnecessary here since microrheology is a very specialized technique but ok} \\
		of microswimmers in viscoelastic flows. For Newtonian fluids, several rheological experiments on suspensions of extensile swimmers like \emph{E.coli} have shown that activity can drastically reduce the effective viscosity \cite{sokolov2009reduction,gachelin2013non,mcdonnell2015motility}, even down to the `superfluidic' limit \cite{lopez2015turning}. 
		The mechanism, first outlined in Ref.\ \cite{hatwalne2004rheology} and elaborated further in Refs.\ \cite{haines2009three,saintillan2010dilute},
		is as follows. The orientations of elongated swimmers in shear flow follow periodic Jeffery orbits \cite{jeffery1922motion} and are also subject to thermal noise. This competition yields a mean orientation that points in the extensional quadrant of applied shear flow. 
		%This facilitates the 
		With such an orientation
		%\\ \hs{HS: No comma according to grammerly: not more than four words. I am against this excessive use of commas. This is 
			%not how it should be} \\
		extensile microswimmers (pushers) support the applied shear
		flow and thereby reduce the effective shear viscosity, whereas contractile swimmers like \emph{C. reinhardtii} (pullers) resist it and thus increase viscosity \cite{rafai2010effective}.
		
		
		
		
		%\hs{HS: Somehow their could be more references to swimmers/active suspensions in viscoelastic fluids ..... } \\ 
		%\AC{Added review refs below}
		
		Since almost all biological fluids are non-Newtonian, there has been a recent interest towards developing theoretical 
		frameworks that capture the individual and collective dynamics of microswimmers in complex fluids \cite{li2016collective,li2021microswimming,spagnolie2022swimming}.
		%Since almost all biological fluids are non-Newtonian, there has been a recent interest towards developing theoretical frameworks 
		%that predict microscale dynamics
		%\hs{in complex fluids and their rheology}
		%%\hs{in and rheology of}
		%%%bulk-scale behaviour in 
		%%complex fluids 
		%\cite{li2021microswimming,spagnolie2022swimming}. 
		%\AC{the added part reads strange. Should we simply write `` ...predict micro and macroscale dynamics''?}
		%\hs{HS: Meaning of the last part?}
		%we must develop the theoretical frameworks further to understand such macroscale phenomena.
		%Experimental investigations 
		For example, experiments have shown reduced tumbling and increased persistence lengths of bacteria in polymeric 
		%mediums 
		fluids \cite{patteson2015running,zottl2019enhanced,kamdar2022colloidal}.
		%\\ \hs{HS: What about Andreas Z. work of simulating bacterial locomotion in polymeric fluids?????} \\
		Viscoelasticity can also initiate spatiotemporal order in active suspensions.
		For example, a recent study found that DNA polymers trigger oscillatory vortices in confined suspension of \textit{E.coli} \cite{liu2021viscoelastic}.
		However,  determining the effective shear viscosity of an active suspension in a viscoelastic fluid
		%\ac{impact on shear viscosity} 
		has been an uncharted territory because of its complexity:  the elastic relaxation of non-Newtonian fluids and their 
		shear-thinning/thickening property.
		%\\ \hs{HS: the repetition here makes sense}\\
		% % WHY SAY NON-NEWTONIAN fluids IN THE SAME SENTENCE WHEN WE ALREADY WRITE "VISCOELASTIC"?
		%\hs{HS: I thought the Oldroyd-B fluid is not shear-thinning but it has an intrinsic time ......} \\
		%introduces an intrinsic time scale, \ac{and furthermore}, these fluids are shear-thinning/thickening. \AC{The previous 
			%statement [\hs{elastic relaxation of non-Newtonian fluids introduces an intrinsic time scale and these fluids are 
				%shear-thinning/thickening}] reads as if shear thickening/thinning is not associated with a time scale. 
			%But it is!  (see carreau fluid wiki and $\lambda$ therein). To me ``Furthermore'' helps us separate.} \\
		
		To gain better insights into biological and artificial motility in biological fluids, this communication investigates the role of non-linear polymeric stresses in non-Newtonian fluids. Specifically, this non-linearity allows us to
		%\footnote{ {\ac{allows who? us? ``..allows direct coupling between..''}}} 
		directly couple a background shear flow to the active disturbance flow generated by a microswimmer. 
		For spherical microswimmers in a Poiseuille flow, we already showed that 
		due to such a coupling they experience a swimming lift force that
		%the \hs{swimming lift}
		%\footnote{\ac{We can't use ``\textbf{the} swimming lift'' as if it is well known. ``For spherical microswimmers in a Poiseuille flow, we have shown that due to such a coupling they experience swimming lift that depends on swimmer type.''}} force 
		depends on the swimmer type \cite{choudhary2022soft}.
		Here, we will show that this nonlinear coupling influences the Jeffery orbit of an elongated swimmer in a shear flow, which thereby also fundamentally alters the bulk rheological response. Such activity-induced changes in the orientational dynamics
		% \hs{???}dynamical changes\hs{???} 
		cannot be observed in Newtonian fluids because the viscous stress tensor does not permit such a nonlinear coupling.
		
		%\\ \hs{HS: We need to look at the previous again.} \AC{I understand that it can be a bold statement because something 
			%like chirality brings in changes to swimmer trajectory, even in Newtonian shear. But that is due to inherent asymmetry of 
			%swimmer geometry and there the type of activity won't play a role.} \\
		%\ac{Shall we write "These modifications to the dynamical behaviour are sensitive to swimmer type and cannot be observed 
			%in Newtonian fluids because the viscous stress tensor does not permit such a nonlinear coupling."?} 
		%\AC{If you still feel it is too general, we can remove it.}
		%\\
		
		
		The current work uses the model of a second-order fluid with a single elastic relaxation time that captures the dynamics of 
		polymeric fluids 
		%with \hs{small} \ac{dilute} polymer concentration
		in the dilute limit (Boger fluids) \cite{bird1987dynamics,james2009boger}. 
		%\\ \hs{HS: For the below: What should be the main message?}  \\
		For weak elasticity, quantified by the Weissenberg number,
		%\\ \hs{HS: My understanding is that all the hydrodynamic calculations use Wi, so why not mention it here. It is
			%so central!} \\
		%(\AC{Commas: because ``quantified by the Weissenberg number'' is an interruption})
		we perform a perturbative analysis.
		% and 
		Thereby we show that  
		%even then 
		the inherent non-linearity in the polymeric stress tensor together with activity
		significantly alters the Jeffrey orbits known from the Newtonian fluid
		and of passive rods in a viscoelastic fluid \cite{leal1975slow,brunn1977slow}.
		%
		%\\ \hs{HS: Shall we cite here the old work on passive rods, we did not mention it at all so far.}\\
		%\AC{Fluid peers will be unhappy to see us casually ignoring log-rolling here and claiming that  \underline{we found out} 
			%that weak viscoelasticity yields deviation from Jeffery. We actually modify log-rolling, which is well-known from 50-year old 
			%passive viscoelastic literature. But I see your motivation in keeping things light in intro. I suggest that we add some nuances: 
			%associate "active" effects to "we perform/we show that".}
		%\\ \AC{I also recommend that we avoid any dimensionless numbers here. Note that we switch from Wi to De.}
		%\\ 
		%\ac{How about:} For even weak elastic perturbations, risen from the non-linear polymeric stress tensor, we find that 
		%microswimmer's activity significantly modifies the Jeffrey orbits known from the Newtonian flows. \\
		To evaluate the influence of thermal noise, we combine our results with the orientational Smoluchowski equation and find that the altered deterministic dynamics
		also substantially affects the orientational distribution, known as the suspension microstructure \cite{hinch1975constitutive}.
		It strongly differs between extensile (pushers) and contractile (pullers) microswimmers. The orientational distribution allows
		us to directly determine the effective viscosity of the active suspension from an orientational average over the swimmer stresslets.
		Our analysis shows that  fluid elasticity reduces the effective viscosity of active suspensions for both pushers and pullers
		compared to Newtonian fluids. The reduction increases with activity and allows to reach the superfluidic limit for smaller
		swimmer densities.
		%\\ \hs{HS: Let's see later if this are our main messages.}
		
		%
		%aims to step in this direction by considering a steady viscoelastic flow that introduces a single relaxation time scale owing 
		%to fluid elasticity \cite{bird1987dynamics}. 
		%%First, we employ the perturbation theory to analytically derive the impact of this intrinsic relaxation mechanism on the 
		%%orientational dynamics of an active spheroid.  
		%Although the considered relaxation mechanism is weakly elastic, we show that even weak non-linearity fundamentally alters 
		%the microscale dynamics.
		%We then formulate an ensemble of such swimmers by using the mean-field kinetic theory, and find that the alterations to orbits also %substantially affect the averaged orientational distribution, known as the suspension microstructure \cite{hinch1975constitutive}.
		%We show that the microstructure for extensile swimmers (pushers) is distinct from contractile swimmers (pullers). 
		%%We find that this is because the bulk viscosity of the suspension is influenced in a twofold manner: directly via the active 
		%%stresses and through the modification of microstructure.
		%Our results for the rheological response are found using a numerical approach and are valid across a range of applied shear 
		%strengths; these are supported by analytical expressions in the limit of weak flows.
		%\ac{The analysis suggests that, in comparison to Newtonian mediums, fluid's elasticity overall reduces the effective viscosity 
			%of active suspensions, for both pushers and pullers.}
		%
		
		
		
		%The current work aims to step in this direction by considering a steady viscoelastic flow that introduces a single relaxation 
		%time scale owing to fluid elasticity \cite{bird1987dynamics}. 
		%%First, we employ the perturbation theory to analytically derive the impact of this intrinsic relaxation mechanism on the orientational %dynamics of an active spheroid.  
		%Although the considered relaxation mechanism is weakly elastic, we show that even weak non-linearity fundamentally alters the 
		%microscale dynamics.
		%We then formulate an ensemble of such swimmers by using the mean-field kinetic theory, and find that the alterations to orbits 
		%also substantially affect the averaged orientational distribution, known as the suspension microstructure \cite{hinch1975constitutive}.
		%We show that the microstructure for extensile swimmers (pushers) is distinct from contractile swimmers (pullers). 
		%%We find that this is because the bulk viscosity of the suspension is influenced in a twofold manner: directly via the active stresses 
		%%and through the modification of microstructure.
		%Our results for the rheological response are found using a numerical approach and are valid across a range of applied shear 
		%strengths; these are supported by analytical expressions in the limit of weak flows.
		%\ac{The analysis suggests that, in comparison to Newtonian mediums, fluid's elasticity overall reduces the effective viscosity 
			%of active suspensions, for both pushers and pullers.}
		
		%We first explore its impact on both the orientational dynamics and the bulk rheological response of active suspensions.
		%which can further modify in cases of strong viscoelasticity \cite{cohen1987orientation}.
		
		
		%This can get further complicated if the medium is strongly viscoelastic, 
		%render the issue quite cumbersome. For instance, 
		%the rheological impact of activity has not been explored in complex fluids.
		
		
		\begin{figure}[t]
			%	{0.25\textwidth}
			\centering
			\includegraphics[width=1\textwidth]{schem_1.pdf}
			\caption{\textbf{Microswimmers in viscoelastic shear flow}. Schematics showing (a) the dilute suspension of microswimmers in an external shear flow $ \IB{V}^{\infty} $ of a viscoelastic fluid, (b) coordinate system 
				moving with an individual swimmer that is modelled as an active prolate spheroid of major axis $ a $ and minor axis $ b $. Here $ \IB{p} $ and $ U_s $ denote the orientation and swimming speed, respectively.}
			\label{schematic}
		\end{figure}
		
		
		
		
		%
		
		
		
		\section*{Results}
		
		\noindent
		\textbf{Setup and second-order fluid}.
		% \textbf{Active spheroid in viscoelastic shear flow}.
		We consider a dilute suspension of microswimmers in a shear flow of a viscoelastic fluid, for instance, consisting of polymers 
		dissolved in a Newtonian fluid as shown in Fig.\ \ref{schematic}(a). Figure\ \ref{schematic}(b) depicts the coordinate system 
		moving with the microswimmer that is modelled as an active prolate spheroid and swims with speed $ U_s $. 
		The uniformly distributed polymers are much smaller than the microswimmers and hence modeled within a continuum description. 
		%\\ \hs{HS: I am not sure if the previous sentence is necessary} \\
		The inertia-less hydrodynamics is governed by the mass and momentum conservation as $ {\IB{\nabla}} \cdot \IB{V} = 0$ and $ {\IB{\nabla}} \cdot \bten{T}=0  $, respectively, where $ \IB{V} $ is the  velocity field and $ \bten{T} $ is the total stress tensor.
		It follows the second-order fluid (SOF) model: 
		$ \bten{T} =-P \, \bten{I} + 2 \, \bten{E} + \text{Wi} \, \bten{S} $ \cite{bird1987dynamics}.
		Here, $ \bten{E} $ denotes the rate-of-strain tensor and
		$
		\bten{S} =	  4 \bten{E}\cdot \bten{E} + 2 \delta \overset{\Delta}{\bten{E}}
		$
		is the polymeric stress tensor, which is quadratic in $ \bten{E} $ and contains the lower-convected time derivative.
		The SOF model not only allows us to capture the elastic effects pertinent to Boger fluids, but also to obtain analytical results  for small $ \text{Wi} $.
		Since we consider a steady shear rate in this work, we disregard the partial time derivative of $\bten{E}$.
		In above governing equations, length, velocity, and pressure are  already non-dimensionalized by the swimmer length ($ l=2a $), $ \dot{\gamma} l$, and $ \mu_f \dot{\gamma} $, respectively, where $ \mu_f $ is the fluid shear viscosity. 
		%The 
		Also, the stress tensor is written in dimensionless units using characteristic numbers (\text{Wi} and $ \delta $).
		% where
		%\hs{HS: Too many where's.}
		In particular, $ \text{Wi} =  t_{\text{relax}} \dot{\gamma} $ is the shear based Weissenberg number that quantifies the importance of elasticity in the medium.
		It compares the shear rate $ \dot{\gamma} $ or inverse shearing time to the polymer relaxation time $ t_{\text{relax}} = (\Psi_{1}+\Psi_{2})/\mu_f $, where $ \Psi_{1}$ and $\Psi_{2} $ are the normal stress coefficients.  
		Furthermore, $ \delta = -\Psi_{1}/2(\Psi_{1}+\Psi_{2}) $ is the viscometric parameter that typically varies between $ -0.7 $ to $ -0.5 $. 
		In `Methods: Hydrodynamic model' we explain in detail how we solve the governing equations using a  systematic perturbation expansion in $ \rm{Wi} $ to address the limit of weak viscoelasticity.
		
		%implement the steady SOF model.
		%it multiplies the lower convected derivative of $ \bten{E} $. 
		%and 
		%$\Psi_1,\Psi_2$ represent the dimensional steady-shear normal stress coefficients that are measured experimentally \cite{bird1987dynamics}.
		%The polymeric stress tensor $ \bten{S} =
		%4 \bten{E} \cdot \bten{E} + 2 \delta \overset{\Delta}{\bten{E}} $ is non-linear in $\bten{E}$ and contains the lower-convected time derivative of  $\bten{E}$ denoted by
		
		
		
		\textbf{Active spheroid in viscoelastic shear flow}.
		To determine the dynamics of an active spheroid, we first note that a swimmer disturbs the flow field both passively (due to its rigid body) and actively (due to self-propulsion). 
		The disturbance fields are implemented using hydrodynamic multipoles
		\cite{chwang1975hydromechanics,spagnolie_lauga_2012}.
		Flagellated microswimmers like \emph{E.coli} and \emph{Chlamydomonas} generate a force-dipole flow field and 
		higher order disturbances: source dipole, rotlet dipole, and force quadrupole.
		We included all four of them and found that only the force-dipole flow field affects the swimmer dynamics in leading order in $ \text{Wi} $. 
		%\\ \hs{HS: This sounds very unspecific.  ... current oder of approximation ...} \\
		The force-dipole field around the active spheroid is $ \sigma \frac{\hat{\IB{r}}}{r^2} [3 (\hat{\IB{r}} \cdot \IB{p})^2 -1]$ 
		with $\hat{\IB{r}} = \IB{r}/r$.
		%
		%$ \sigma \big[   \frac{-  \IB{r} }{r^{3}} +  \frac{3  \IB{r} \left(\IB{r} \cdot \IB{p}  \right)^{2} }{r^{5 }}  \big] $.
		%
		%Here we consider pushers and pullers that are representative of a broad class of motile organisms \cite{lauga2020fluid}.
		%The 
		Here, $\sigma$ is a non-dimensional parameter equal to the ratio of the force-dipole strength ($\sigma^*$)
		to the stresslet imposed by the shear flow 
		($ 8\pi \mu_f \dot{\gamma} l^{3} $), where $\sigma>0$ represents pushers 
		and  $ \sigma<0 $ pullers. 
		For the current work, we focus on swimmers of $ 5 \mu $m size and shear rates of order $ 1-10\text{s}^{-1}$, which corresponds to $ \sigma $ of typical wild-type \emph{E.coli} being roughly $ 0.04-0.1 $ \cite{berke2008hydrodynamic,drescher2010direct}.
		%This force-dipole strength ($ \sigma $) is normalized using $ \textcolor{red}{8\pi} \mu \dot{\gamma} a^{3} $.
		%The spheroid passively disturbs the flow that is captured by the far-field spheroidal multipoles, and active disturbance is the force-dipole flow field\footnote{The impact of higher order disturbances (source-dipole, rotlet-dipole and force-quadrupole) is also included and found to be null at the present order of approximation. See Supplementary material.}
		
		\begin{figure}[t]
			\centering
			\includegraphics[width=1\textwidth]{orbits.pdf}
			%		\includegraphics[width=1\textwidth]{orbits-new.pdf}
			\caption{\textbf{Orientation dynamics} of passive $(\sigma =0)$	and active particles in weakly viscoelastic shear flow. The blue curves show time traces of the orientation vector $\IB{p}$ on the unit sphere starting at the red dot.
				(a) $\sigma = 0 $, (b) $ \sigma=0.06 $, (c) $ \sigma=0.2 $, (d) $ \sigma=-0.2 $, (e) $ \sigma=-0.8 $, (f) $ \sigma =-2.4 $ . Other 
				parameters: $ \text{Wi}=0.2$, $\lambda=5$, $\theta_{0}=\pi/5$, $\phi_{0}=\pi/2$, $\alpha_{1}=4.62$, $\alpha_{2}=-0.33$, 
				$\beta_{1}=0.68$, and $\beta_{2}=1.46 $, $ \delta=-0.6 $.}
			\label{fig:orbits}
		\end{figure}
		
		In Newtonian fluids, the equation of motion for the orientation $ \IB{p} \, (\theta, \phi)$ gives the Jefferey orbits. 
		To formulate this equation for viscoelastic shear, we note that  the polymeric stress tensor $ \bten{S} $ is quadratic in the rate-of-strain tensor and vorticity \cite{bird1987dynamics}. 
		Thus, similar to Ref.\ \cite{einarsson2015rotation} we determine all terms that by symmetry contribute to the rate of change $ \dot{\IB{p}} $ up to first order in $ \text{Wi} $. Neglecting small terms, we arrive at 
		%\ac{the }
		\begin{align}\label{EOM}
			\dot{\IB{p}} &=   \left(  \bten{I} -  \IB{pp}  \right) \cdot  \left(   \bten{E}^{\infty} \cdot \IB{p}   \right) \left[  \Lambda + \text{Wi}  \, \sigma  \alpha_{1} + \text{Wi} \, \beta_1 \,  \bten{E}^{\infty}: \IB{pp} \right] 
			\nonumber \\
			& \; + \IB{\Omega}^{\infty} \times \IB{p} \left[  1 + \text{Wi} \, \sigma  \alpha_{2} + \text{Wi} \, \beta_2 \,  \bten{E}^{\infty}: \IB{pp}   \right] + O(\text{Wi}^{2})
		\end{align}
		in non-dimensional form, where $ \IB{\Omega} $ is the angular velocity and the superscript $ \infty $ denotes the quantities that belong to the prescribed 
		shear flow.
		% (the 
		%The derivation of Eq.\ (\ref{EOM}) is detailed in `Methods: Orientation dynamics'.
		%
		The shape factor $ \Lambda = \frac{-1+\lambda^{2}}{1+\lambda^{2}}$ contains the aspect ratio
		$ \lambda = a/b$,
		the ratio of major to minor axis.
		%\ac{, called as the aspect ratio.}
		% we need to define "aspect ratio", as we use it later. 
		% It (I explicitly write shape factor so that the reader does not confuse "it" with aspect ratio.)
		The shape factor approaches  $ +1 $ and $ -1 $ for needle and disk-like particles, respectively. 
		Since microswimmers are usually elongated, we focus on prolate spheroids of $ \Lambda > 0.9 $, which corresponds to $\lambda > 4$; 
		it also helps in simplifying the calculations (see `Methods: Hydrodynamic model').
		The terms with coefficients $ \alpha_i$ and $ \beta_i $ represent the active and passive viscoelastic contributions, respectively. 
		These coefficients are evaluated explicitly as outlined in `Methods: Orientation dynamics' using the Lorentz reciprocal theorem, 
		where we also derive
		%show the derivation of 
		Eq.\ (\ref{EOM}).
		For our relevant parameters, we find $\alpha_1 \gg |\alpha_2|$. Since the coefficients vary only weakly with $\lambda$ and $ \delta $, they are treated as constants in the following discussion of results (see also Supplementary Note 2A). 
		
		
		
		In the absence of polymers ($ \text{Wi} = 0 $), Eq.\ (\ref{EOM}) reduces to the well-known Jeffery equation \cite{jeffery1922motion} 
		that has an infinite number of neutrally stable solutions, \emph{i.e.,} the microswimmer's orientation traces periodic orbits that 
		depend on initial conditions. During orbital time period $ T = 2\pi (\lambda+\lambda^{-1})/\dot{\gamma} $, they spent the majority 
		of their time  ($ \propto \lambda $) aligned near the flow-vorticity plane.
		% ($ \propto \lambda $).
		When polymers are present, the passive viscoelastic effect breaks the degeneracy of Jeffery orbits and the microswimmer 
		slowly drifts towards an alignment with the vorticity axis, known as the ``log-rolling'' state \cite{gauthier1971_shear-flow,brunn1977slow,davino2015particle}. 
		Figure \ref{fig:orbits}a shows the curve traced by the orientation of a spheroid on a unit sphere while drifting towards the vorticity axis.
		This dynamics can be deduced from Eq. (\ref{EOM}) for passive spheroids ($\sigma = 0$) when the terms with coefficients 
		$ \beta_{1} $ and $ \beta_{2} $ are present.
		%
		%\ac{This can be observed in Eq. (\ref{EOM}), where, for $ \sigma=0 $, terms corresponding to coefficients $ \beta_{1} $ and 
			%$ \beta_{2} $ are responsible for log-rolling.}
		%
		%See 
		In Supplementary Note 2C  we also compare our results of passive spheroids with \citet{brunn1977slow}.
		
		%\AC{Below, I emphasised a bit on the fact that modifications to log-rolling are made possible by the fluid's elasticity.}
		
		%\hs{HS: This I will further adjust once we clarify two issues} \\
		Now, we illustrate the influence of activity.
		For weak pushers ($0 < \sigma \lesssim 0.1 $), the orbits towards log rolling look similar to the ones of passive spheroids,
		albeit with a larger aspect ratio (see Fig.\ \ref{fig:orbits}b) since they exhibit more skewed trajectories.
		This can directly be
		% deduced 
		inferred from Eq.\ (\ref{EOM}),
		%\\\AC{"can be deduced from Eq.\ (\ref{EOM})" is used just few lines ago. How about "inferred"? } \\
		where  activity ($\sigma$) in the term with $\mathrm{Wi} \, \sigma \alpha_1$  modifies the shape factor.
		%
		Since $ \alpha_{1}>0 $, a pusher ($ \sigma > 0 $) effectively increases $\Lambda$, which corresponds to a higher aspect ratio. 
		The opposite occurs for weak pullers ($ \sigma < 0 $) as they  behave like a passive spheroid with reduced aspect ratio.
		Here the trajectories are more circular as shown in Fig. \ref{fig:orbits}d.
		
		
		\begin{figure}[t]
			\centering
			\includegraphics[width=1\textwidth]{phase-diag.pdf}
			\caption{\textbf{State diagram} showing the different states of the orientational dynamics for three $\mathrm{Wi}$ values. 
				With increasing dipole strength $\sigma$ from left to right, we observe  alignment (Align.), shear-plane rotation (SPR), 
				log rolling (LR) and again alignment.
				%	for different $\mathrm{Wi}$. 
				In the alignment state, the alignment angle $\phi_{eq}$ is plotted versus $\sigma$  as solid lines.
				The dots on the horizontal axis indicate the critical values $\sigma_c$ at which alignment (Align.) occurs
				either to the left of the dashed-dotted lines ($\sigma < 0$) or to the right ($\sigma>0$). The dashed lines separate
				SPR and LR from each other at $\sigma = -1.93, \, -0.49, \, -0.19 $ for increasing $ \text{Wi}$.
				The inset shows the variation of Lyapunov exponent with $\sigma$.
				Other parameters are chosen as in Fig.\ \ref{fig:orbits}.
			}
			\label{fig:phase}
		\end{figure}
		
		As the dipole strength of a pusher increases beyond a critical value $ \sigma_{c} $, the orientation drifts to the shear plane ($ \theta=\pi/2 $) and aligns at an angle $ \phi_{eq} $ with the flow direction (see Fig.\ \ref{fig:orbits}c).  
		We quantify this transition in Fig.\ \ref{fig:phase} and show  that $ \sigma_{c} $ (dots close to 0) decreases with increasing ${\rm Wi}$ while $ \phi_{eq} $ increases with both $\sigma$ and $ \rm{Wi} $. 
		%
		This dynamical behaviour can again be discerned from Eq.\ (\ref{EOM}), which essentially balances the effect of rotational ($\IB{\Omega}^{\infty}$) and elongational ($\bten{E}^{\infty} $) flow on $\IB{p}$.
		Now, activity together with viscoelasticity allows to control this balance. In particular, for our case of $\alpha_1 \gg |\alpha_2|$, we expect the elongational flow to dominate the dynamics for increasing activity. Indeed, when the effective shape factor $\Lambda + \text{Wi}  \, \sigma  \alpha_{1}$ exceeds one at a critical value $\sigma_c$, the dynamics of $\IB{p}$ transitions from the orbital to the alignment state as favored by the elongational flow.  
		In `Methods: Dynamical analysis 1', we show for a  modified dynamical system that the relevant eigenvalue of the dynamical matrix becomes real at $\sigma_c$, as expected for 
		such a transition, and the corresponding eigenvector gives the alignment angle $\phi_{eq}$. 
		%\hs{For increasing $\sigma$} \ac{
			As $\sigma$ increases, 
			%  NOTE: increases <-> approaches 
			%it 
			$\phi_{eq}$ grows from zero and approaches $\pi/4$ for large $\sigma$, since in this limit, the elongational flow ($\bten{E}^{\infty}$) with its principle axis along $\phi=\pi/4$ completely dominates the dynamics of $\IB{p}$.
			
			
			%\ac{Viscoelasticity in the flow allows activity to alter the local flow and perturb this balance.
				%	%%%%%%%%%%%%%%%
				%	%
				%% 			To HS: the above message is important, it connects to the conclusions of this paragraph (see "Hence, for a fixed...."). 
				%%			Otherwise it feels are explaining Stokes/Newtonian regime.
				%	%
				%	%%%%%%%%%%%%%%%
				%As we increase  $ \sigma $ beyond a critical value, this balance is disturbed to an extent that $  \bten{E}^{\infty} $  dominates the orbital dynamics. 
				%We demonstrate this by analysing the modification brought by the activity in Eq. (\ref{EOM}), as it is solely responsible for alignment in the shear plane.
				%In `Methods: Dynamical analysis 1', we demonstrate the solution in terms of a fundamental matrix:
				%	\begin{equation} %\label{active_EOM_main}
					%		\IB{q}(t) = \exp \big[ \left[  \bten{E}^{\infty}  (\Lambda+ \text{Wi} \sigma \alpha_{1}) + \bten{O}^{\infty} (1+ \text{Wi} \sigma 
					%\alpha_{2}) \right] t   ] \IB{q}(0), \nonumber
					%	\end{equation}
				%	where $ \frac{\IB{q}(t)}{|\IB{q}(t)|} = \IB{p}(t)  $.
				%Since the eigensystem % meaning Eigenvalues and Eigenvector combined
				%	of this matrix contains the information of orbital dynamics, we analyse the eigenvalues and find that they turn real when $ \sigma > 
				%\frac{ 1 - \Lambda}{\text{Wi} (\alpha_{1} - \alpha_{2})} $ or $ \sigma < \frac{ -1 - \Lambda}{\text{Wi} (\alpha_{1} + \alpha_{2})}  $. The %normalized eigenvector corresponding to the positive eigenvalue provides the angle of alignment ($ \phi_{eq} $).
				%%%%%%%%%%%%%%%%%%%
				%%
				%%	\begin{align}
					%%	0, \, \pm \frac{1}{2} \sqrt{\Lambda ^2 -1 +2 \sigma  \text{Wi} \left(\text{$\alpha_{1} $} \Lambda  -\text{$\alpha_{2} $}\right) + \text{Wi}^{2} %\sigma^{2} \left( \alpha_{1}^{2} - \alpha_{2}^{2} \right)} \nonumber .
					%%\end{align}
					%%%%%%%%%%%%%%%%%%%
					%	We verify that this eigenvector matches the alignment obtained from numerically evaluated $ \dot{\IB{p}} $, which is used in Fig. \ref{fig:phase}.
					%	Furthermore, the onset of alignment is accurately predicted by the aforementioned criteria of eigenvalues.
					%%
					%Hence, for a fixed $ \text{Wi} $, there is a $ |\sigma_{c}| $ beyond which the interaction between active stress and  the background flow alters the local flow as being effectively elongational, whose principle axis points in $ \phi_{eq} $.}
				%The equilibrium angle increases with the pusher's strength because the elongation prefactor becomes stronger.
				%The alignment angle asymptotes at $ \pi/4 $ because the system approaches dynamics governed \ac{completely} by $  \bten{E}^{\infty} $, %whose principle axis points at $\phi=\pi/4$.
				
				%\\ \hs{HS: I am not getting all this. How do you know that $\Lambda>1$ means shear alignment under an angle ????????
					%\\ And: Also the coefficient for the rotation contains $\sigma$ and thus gets stronger.
					%\\ And: You have all the numbers: so what happens at $\Lambda + Wi \sigma \alpha_1 = 1$ and beyond.}
				%\\ \hs{HS: The last sentence is too vague ..... and what is this issue with the Lyapunov exponent???????}
				
				% \hs{HS:  $\sigma_c$ is small. Also $\sigma_c$ is not shown in the figure.}
				% \hs{HS: 1. What do we characterize? Does $ \mathcal{L} < 0$ mean the limit cycle is stable and  $ \mathcal{L} > 0$ unstable	since it moves to the log-rolling state?????}\\
				%\\ 2. Are you sure one calls this eigenvalue of the dynamic matrix Lyapunov exponent? \\
				
				For pullers, Fig.\ \ref{fig:orbits}e shows that increasing the activity induces a transition from log rolling towards rotation in the shear plane.
				This orbit is also observed for passive oblate particles in viscoelastic flows 
				\cite{gauthier1971_shear-flow,brunn1977slow,davino2015particle} and acts as an attracting limit cycle.
				%
				We have performed a stability analysis for the shear-plane rotation in `Methods: Dynamical analysis 2' and calculated the Lyapunov stability exponent $ \mathcal{L} $. 
				It determines the exponential time variation of the disturbed limit cycle. 
				In the inset of Fig.\ \ref{fig:phase}, we show the Lyapunov exponent as a function of $\sigma$.
				For weak and moderate pullers, $ \mathcal{L} >0$ shows that shear-plane rotation is an unstable limit cycle, where the system drifts towards stable log rolling.
				As the dipole strength of the puller becomes more negative, $ \mathcal{L}$ 
				%assumes
				%\\\AC{can we use something apart from "assumes"? attains, exhibits?For me it sounds a thing related to an assumption.} \\
				turns negative meaning that shear-plane rotation is stable. 
				Upon analysing the orbital dynamics in the shear-plane rotation, we find that increasing $ |\sigma| $ gradually slows down the 
				swimmer's rotation near the $y$-axis, along which the flow gradient is applied.
				%the flow-gradient axis. 
				Eventually, a transition to permanent alignment with $\phi_{eq} = \pi/2 $ occurs, which can be similarly analyzed as the alignment transition of pushers. However, here it occurs at the effective shape factor $\Lambda + \text{Wi}  \, \sigma_c  \alpha_{1} = -1$. With further increasing $|\sigma|$ the 
				active spheroid tilts against the flow, in contrast to pushers, and asymptotes at $ 3\pi/4 $, which is the direction of the second principal
				axis of the elongational flow ($\bten{E}^{\infty}$).
				% THE FLOW IS SHEAR FLOW
				This behavior is illustrated in Fig.\ \ref{fig:orbits}f and Fig.\ \ref{fig:phase}. Finally,  Fig.\ \ref{fig:phase} also shows that as
				%As 
				$ \text{Wi} $ increases, the regime of shear-plane rotation shrinks.
				
				
				\vspace{4mm}
				\noindent
				%%%%%%%
				\textbf{Impact of noise on orientational dynamics.}
				The deterministic orientational dynamics 
				%in steady state 
				% NOT JUST STEADY STATE, AT ALL TIMES. TAKE LOG-ROLLING, NOISE WON'T EVEN LET IT ATTAIN THE DETERMINISTIC STEADY STATE
				is disturbed by two types of stochastic reorientations of the  swimming direction, which we now address with the help of the Smoluchowski equation. 
				First, a bacterium tumbles, which is triggered when the rotation of one of its flagella reverses so that it leaves the  flagellar bundle
				%the  sense of rotation of one of its flagella reverses so that it leaves the flagellar bundle
				\cite{berg1983random,lauga2016bacterial,adhyapak2016dynamics}. 
				%\\\AC{"sense of rotation of one of its flagella" too many ``of".}\\
				%We now weigh these deterministic dynamical states against the two stochastic reorientations. 
				%One kind are the internally caused `tumbling' events that are triggered by polymorphic transition of flagella from bundling to 
				%unbundling \cite{berg1983random,lauga2016bacterial}. 
				For a wild type \emph{E.coli}, tumbling occurs roughly every $1\mathrm{s} $, where it attains a new random orientation.
				Although tumbling is biased in the forward direction with a mean tumbling angle of roughly $68.5^\circ$, we model it to be unbiased for computational ease because it only affects our results marginally as suggested by \citet{nambiar2017stress}.
				% AS was occuring in short span
				Second, thermal rotational diffusion continuously reorients a microswimmer but its effect is small compared to tumbling.
				%\AC{Use of ``continuously" is a subtle reminder to the reader that the two stochastic process are qualitatively different. 
					%Tumbling is a random and non-local event, whereas diffusion is a local reorientation event. For the same reason, I removed "also". 
					%Also, I made the sentence for a single swimmer, like the "First,.." point.}\\
				% that are a few microns or smaller in size. 
				
				
				
				
				We now evaluate the orientational distribution of an ensemble of non-interacting microswimmers in steady state, caused by the three mechanisms discussed so far:  deterministic motion in background shear flow, tumbling, and rotational diffusion.
				The steady-state probability distribution function $ \psi(\IB{p}) $ for the orientation vector of non-interacting microswimmers
				%an isolated microswimmer on a unit sphere 
				is governed by the Smoluchowski equation \cite{doi1988theory}
				\begin{equation}\label{Smol}
					\text{Pe}_f \nabla_{p} \cdot \left( \IB{\dot{p}} {\psi} \right)    - \tau D_r  \nabla_{p}^{2} {\psi} +  \left( {\psi} -  \frac{1}{{4\pi}} \right) = 0,
				\end{equation}
				%where the distribution function is conserved by $\int_{S} \psi(\IB{p}) d \IB{p} =1 $. 
				where $\psi(\IB{p})$ is normalized to one.
				%one. 
				The first term describes the orientational drift using $\nabla_{p} $ as the gradient operator on the unit sphere. To compare the 
				strength of flow-induced reorientation to the mean time $\tau$ between two tumbling events,
				we introduce the flow P\'eclet number $\mathrm{Pe}_f = \dot{\gamma} \tau$. Tumbling away from the swimming
				direction $\IB{p}$ is handled by the third term and rotational diffusion by the second term. 
				For bacteria, the latter is typically  small compared to tumbling since $ \tau \sim 1 \mathrm{s} $ \cite{berg1983random} and $ D_r  \lesssim 0.1 \mathrm{s}^{-1} $, as an estimate from the Stokes-Einstein relation shows.
				In restricting ourselves to Eq.\ (\ref{Smol}), we assume a spatially uniform system valid when hydrodynamic and steric
				interactions with bounding surfaces can be neglected \cite{spagnolie_lauga_2012,spagnolie2022swimming}. 
				In particular, this means that the mean length of persistent swimming, $U_s \tau$, where $U_s$ is the swimming speed, is much
				smaller than the spatial extent of the system. 
				Finally, we consider weak fluid elasticity with relaxation time $t_{\mathrm{relax}} \ll \tau, D_r^{-1}$, 
				%\AC{i have two options in mind:}
				%\\\ac{so that the fluid relaxes on the time scales} of stochastic reorientations
				so that the fluid relaxes faster than the time scales given by stochastic reorientations
				\cite{cohen1987orientation}.
				Thus, we do not  need to take into account any memory in the rotational noise.
				
				In the following, % comma 
				we want to emulate a rheological experiment and vary the shear rate $ \dot{\gamma} $ via $ \text{Pe}_f $,
				%%%%%%%%
				%\AC{``$ \dot{\gamma} \propto \text{Pe}_{f}$ via $ \text{Pe}_{f}$'' seems redundant as we just defined $ \text{Pe}_{f} $ }\\
				%%%%%%%%
				while the fluid and swimmer properties are kept constant. Therefore, in Eq.\ (\ref{EOM}) for the rotational 
				drift velocity $\dot{\IB{p}}$, we rewrite $\text{Wi} $ as $ \text{Pe}_{f} \text{De} $, where the Deborah number 
				$ \text{De} = t_{\mathrm{relax}}/\tau$ compares the fluid relaxation time to $\tau$.
				Furthermore, the second relevant parameter, $ \text{Wi} \sigma $, does not % explicitely
				explicitly depend on $\dot \gamma$ but rather quantifies the strength of activity relative to fluid elasticity. 
				%%%%%%%%%%%%%%%
				% WE SHOULD NOTE THAT THERE'S t_relax
				%%%%%%%%%%%%%%%
				To vary activity  independent of other parameters, we replace $ \text{Wi} \sigma $ by
				$ \text{Pe}_{a} \text{De}/8\pi$. Here, $ {\rm Pe}_a $ is the signed activity Peclet number that is positive for pushers and negative for pullers. 
				Using $U_s \sim \sigma^{*} / (\mu_f  l^{2})$ for the swimming speed with the force dipole moment $\sigma^{*}$\cite{berke2008hydrodynamic}, we can write it in the familiar form $ {\rm Pe}_a \sim U_s \tau/l $. 
				Thus, its magnitude compares the persistence length to the body length $l$. A wild-type \emph{E.coli} typically has 
				$ \text{Pe}_{a} \lesssim 5 $ \cite{berg1983random,chattopadhyay2006swimming}. 
				%Finally
				We numerically solve Eq.\ (\ref{Smol}) for arbitrary $Pe_f$ by expanding $\Psi$ in 
				spherical harmonics and 
				% sentence needs a conjunction
				taking into account the first 100 harmonics. The numerical solution is also verified analytically in the limit of 
				$ \text{Pe}_f \ll 1 $ (see further details in `Methods: Kinetic model').
				
				
				
				Figure \ref{fig: microstructure} shows the orientational probability distribution of passive and active particles
				for various cases.
				It quantifies the orientational `microstructure'  of a suspension of non-interacting and orientable particles subject
				to shear flow
				\cite{hinch1976constitutive,chen1996rheology,nambiar2017stress}.
				We begin with discussing the case of weak shear rate ($\text{Pe}_f  < 1$) in Figs.\ \ref{fig: microstructure}a-d.
				In the absence of shear flow, the microstructure is solely governed by rotational noise and
				%\hs{therefore, it}
				is therefore isotropic.
				For weak shear rates the extensional part ($ \bten{E}^{\infty} $) of the shear flow primarily distorts the microstructure,
				which peaks near the extensional axis as noted for Newtonian fluids in Ref.\ \cite{hinch1972effect}. 
				We quantify this in `Methods: Kinetic model 1' by solving Eq.\ (\ref{Smol}) via a perturbation expansion in $\text{Pe}_f$ and obtain 
				%for   % we should write either "obtain the following for the first order correction" or simply ommit "for"
				% I verified this in Grammarly® english check as well
				the first-order correction:
				\begin{equation}\label{pert_Smol}
					%\psi^{(1)}	=  \frac{1 }{8 \pi} (\Lambda+\alpha_{1} \text{De} \text{Pe}_{a}/8\pi) \,  \bten{E}^{\infty}: \IB{p}\IB{p} \, .
					\psi_{(1)}	=  \frac{3 }{4 \pi} (\Lambda+\alpha_{1} \text{De} \text{Pe}_{a}/8\pi)      
					%\left[   
					\frac{   \bten{E}^{\infty}: \IB{p}\IB{p} }{1+ 6\tau D_r}  
					%\right].
				\end{equation}
				%Note
				In the absence of activity, there is no deviation from the Newtonian microstructure at this order. 
				% The slow relaxation towards the deterministic log-rollingstate observed in viscoelastic uids (see Fig. 2a) is alwaysinterrupted by the dominating stochastic reorientation \cite{cohen1987orientation,chung1987orientation, iso1996orientation}.
				
				In Figs.\ \ref{fig: microstructure}a-d we use a weak shear rate of $\text{Pe}_{f}=0.5$ 
				%with $ |\text{Pe}_{a}| = 20$, 
				and $ |\text{Pe}_{a}| = 20$,
				which is
				% "AND WITH"?
				%\\ \hs{HS: Why should there be a semicolon????} \\
				an activity close to mutated strains of tumbling \textit{E.\ coli} (pusher) \cite{wolfe1988acetyladenylate}
				and the algae \textit{C.\ reinhardtii} (puller).
				%\AC{I know this was my text but now this is 4 times the max limit of wild-type E.coli. Should we use ``close'' for $ Pe_a =20 $?}\\
				%\ac{How about:} \hs{HS: Let's decide what to do here}\\
				%In Figs.\ \ref{fig: microstructure}a-d we use a weak shear rate of $\text{Pe}_{f}=0.5$ with $ |\text{Pe}_{a}| = 20$, 
				%which is \ac{a persistence length pertinent to \textit{C.\ reinhardtii} and mutated strains of \textit{E.coli} 
					%\cite{wolfe1988acetyladenylate,lauga2016bacterial,jeanneret2016brief}.} \\
				%	\AC{I also did quick simulations and found that we can show $ Pe_a =10$ with $\tau D_r =0.05$. The results are very similar; 
					%Dr for 5 micron slender particle is 0.0723 (for ellipsoid it's 0.051). Shall we try that and \textbf{justify the closeness to wild-type}? 
					%or write ``mutant''.}
				%\noindent
				The observed distributions follow the trend of Eq.\ (\ref{pert_Smol}). While pushers  ($ \text{Pe}_a >0 $) enhance the alignment along the extensional axis of the applied shear flow, pullers ($ \text{Pe}_a < 0 $) weaken it. 
				%\hs{The distributions in the shear plane (plot d) show a small deviation from the expected angle of $\pi/4$ due to the rotational flow ($\IB{\Omega}^{\infty}$), otherwise they confirm the discussion so far.} 
				%
				% AC: I found the latter part of the commented sentence informal 
				We note that the peak of the orientational distributions in the shear plane (plot d) exhibits a small deviation from $\phi=\pi/4$. This is due to the rotational flow 
				($\IB{\Omega}^{\infty}$), which contributes to $\psi(\IB{p})$ in second order in $ \text{Pe}_f$ \cite{chen1996rheology}.
				%
				%\ac{We note that the peak of orientation distribution in the shear plane (plot d) exhibits a small deviation away from $\phi=\pi/4$,  
					%which is expected from Eq. \ref{pert_Smol}. 
					%This is because, at $ \text{Pe}_f=0.5 $, $ O(\text{Pe}_f^{2}) $ terms related to $\IB{\Omega}^{\infty}$ have some contribution 
					%\cite{chen1996rheology}.}
				%\\ 
				%\AC{$ \bten{E}^{\infty}.\bten{E}^{\infty} \; \& \; \bten{E}^{\infty}.\bten{O}^{\infty} $ terms arrive at $ O(Pe_f^{2}) $.} \\
				%
				
				\begin{figure}[!t]
					\centering
					\includegraphics[width=0.98\textwidth]{PDF_new1.pdf}
					\includegraphics[width=0.99\textwidth]{PDF_new2.pdf}
					\includegraphics[width=0.95\textwidth]{PDF_ABP.pdf}
					\caption{%\textbf{Probability distribution of orientations}. 
						\textbf{Orientational probability distribution $\psi(\IB{p})$} for different parameters:
						(a-d) weak shear flow at $ \text{Pe}_f = 0.5 $, (e-j) strong shear flow at $ \text{Pe}_f = 5 $. 
						%		(a,e) $ \text{Pe}_a=0 $, $ D_r \hs{\tau} =0.1 $; (b,f) $ \text{Pe}_a=20 $, $ D_r \hs{\tau} =0.1 $; 
						%		(c,g) $ \text{Pe}_a=-20 $, $ D_r \=0.1 $;
						(a,e) $ \text{Pe}_a=0 $, (b,f) $ \text{Pe}_a=20 $, (c,g) $ \text{Pe}_a=-20 $ and always with $ D_r \tau =0.1 $,
						(d,h) $ \psi(\IB{p}) $ in the shear plane; the inset in (h) shows $ \psi(\IB{p}) $ in the flow-vorticity plane.
						%            the insets in (d,h) show $\psi(\IB{p})$ in the shear plane.    INCORRECT: there is no inset in d and the inset in h shows psi along x-z plane
						Non-tumbling particles: (i) $ \text{Pe}_{a}=100 $, (j) $ \text{Pe}_{a}=-100 $. The insets in (e,g,j) show $\phi(t)$ for the shear-plane rotation state corresponding to the respective distributions.
						Other parameters are chosen as in Fig.\ \ref{fig:orbits} with $ \text{De}=0.1 $.
						To locate these distributions in Fig.\  \ref{fig:phase}, we give the corresponding Weissenberg and activity numbers:	
						(a-d) $ \text{Wi}=0.05 $, $ \sigma=\pm 1.6 $; (e-h) $ \text{Wi}=0.5 $, $ \sigma=\pm0.16 $; (i,j) $ \text{Wi}=0.5 $, 
						$ \sigma=\pm0.8 $.
					}
					\label{fig: microstructure}
				\end{figure}
				
				As the shear rate increases, the deterministic dynamics becomes more visible. We illustrate this in Figs.\ \ref{fig: microstructure}e-h for  $\text{Pe}_f=5 $. 
				Compared to Fig.\ \ref{fig: microstructure}a, the peak in $\psi(\IB{p})$ for passive particles 
				%moves % "moving" gives a dynamic feel like velocity. We should use shifts or rotates, because Pe_f is fixed; if it was a video with changing pe_f, "moves" would be ok.
				shifts towards the flow axis and becomes more anisotropic along the flow-vorticity plane, as the deterministic velocity $\dot{\IB{p}}$ is smallest in this plane.
				Eventually, $\dot{\IB{p}}$ becomes zero in the deterministic log-rolling state observed in viscoelastic fluids (Fig.\ \ref{fig:orbits}a).
				%\\ \hs{HS: Let's keep this sentence here. There is nothing wrong with it and we need it for the following sentences.}   \\
				%\ac{AC: Also, this statement doesn't fit here. $\dot{\IB{p}}$ being zero (when perfectly aligned with z-axis) doesn't explain fig.4 at all 
					%or the larger picture of this section. We will not see this in microstructure until $ \text{Pe}_{f} \to \infty $.\\ Shouldn't this subtle 
					%detail on deterministic dynamics be shifted in the previous section?}\\
				However, this state is not completely reflected in the distribution because the slow relaxation towards the log-rolling axis is always 
				interrupted by the dominating stochastic reorientations
				\cite{cohen1987orientation,chung1987orientation, iso1996orientation}.
				Note that due to the fluid elasticity, the peak of $\psi(\IB{p})$ is slightly closer to the flow axis compared to the Newtonian case as Fig. \ref{fig: microstructure}h shows.
				Furthermore, the inset therein also illustrates a broader distribution in the flow-vorticity plane ($\phi = 0 $). 
				These findings are consistent with  earlier studies  on passive fibre suspensions at moderate shear rates \cite{cohen1987orientation,chung1987orientation}.
				%Further increasing $ \text{Pe}_{f}$ spreads the distribution even more in the flow-vorticity plane and, \ac{ultimately for $ \text{Pe}_{f} \gtrsim 100 $, a peak develops along the vorticity axis, which is entirely reminiscent of log rolling.}
				%\\ \AC{Alternative (I split sentence in two):} \\
				Further increasing $ \text{Pe}_{f}$ spreads the distribution even more in the flow-vorticity plane and, ultimately, for $ \text{Pe}_{f} \gtrsim 100 $ the peak develops along the vorticity axis, which is entirely reminiscent of log rolling.
				%\\ \hs{HS: One should be careful with peak splitting since we also have two weaks long the x axis.} \\
				%.  Ultimately for $ \text{Pe}_{f} \gtrsim 100 $, \ac{the peak splits in two parts that move towards the positive and negative vorticity 
					%axes, which is reminiscent of log rolling.}
				%However, this is a regime which we cannot strictly capture \ac{here because} our analysis requires $\text{De}$, $\text{Wi} < 1$.
				%since
				However, this is a regime which we cannot strictly capture 
				%within our perturbation theory  that 
				as our analysis requires $\text{De}$, $\text{Wi} < 1$, which means $ \text{Pe}_{f} < 10$.
				%\\ \hs{HS: I would like to keep it as it is. All these numbers I find too much. This does not help us.}
				%
				%\ac{, which puts a restriction on our $ \text{Pe}_{f} $}. 
				%\\ \hs{HS: We should keep this as it is and not get lost in giving too much numbers. This disturbs the flow of reading.}\AC{I agree but the %reader also won't be able to relate the constraint on De, Wi with constraints on $ Pe_{f} $ (unless we hint it).}
				%
				% \ac{in Fig. \ref{fig: microstructure}e, where $ \text{Pe}_{f}=5 $ is much smaller. 
					%Our analysis requires $\text{De}$, $\text{Wi} < 1$, which in turn imposes a constraint on the largest $ \text{Pe}_{f} $ we can consider 
					%%\emph{i.e.} $ \text{Pe}_{f}=10 $.}
				%
				
				Figure\ \ref{fig: microstructure}f demonstrates that the activity of pushers in conjunction with the higher shear rate at 
				$ \text{Pe}_{f}=5$ makes the distribution more anisotropic and, according to Fig. \ref{fig: microstructure}h, strongly focused 
				% focuses at 
				% {"something" makes "something" FOCUSES at an angle} is strange in english}
			at an angle $\phi=18^\circ$. 
			This is close to the alignment angle $\phi_{eq}=21^\circ$ of the deterministic alignment state with the parameters $ \text{Wi}=0.5$, $\sigma=0.16$ in Fig.\ \ref{fig:phase}. 
			Conversely, pullers reduce the anisotropy in the orientational distribution as shown in Fig. \ref{fig: microstructure}g. 
			Similar to the passive case, we can directly relate this observation to the deterministic velocity $\dot{\IB{p}}$ with parameters $ \text{Wi}=0.5$, $\sigma=-0.16$ that indicate again the log-rolling state; the resulting orbit  is similar to that depicted in Fig. \ref{fig:orbits}d. 
			%\AC{Why are we going back to log-rolling when we already said that `we can't capture the log-rolling in orientation distribution at these $ Pe_f $' for passive. Why not say: }\\
			% \\ \hs{HS: It is an orbit not an orbital drift.}\\
			% %orbital drift 
			%\\ \hs{HS: The hint to the passive case is useful here, since we again argument with $\dot{p}$!!!!!} \\
			% AC: "similar to passive case" is a jarring transition. And it is not really needed.
			%
			%We can relate it to the deterministic  velocity
			%We can \ac{understand this behaviour by following the} deterministic \ac{dynamics} ($\dot{\IB{p}}$) 
			% and the orientation is not as constrained to the flow-vorticity plane
			%\\ \hs{HS: I find the following to detailed and hard to read. I think we overdue it here. But I tried to work on it.} \\
			Compared to the passive case, the orbit is more circular, which we already related to a reduced effective aspect ratio.
			Therefore, the orientation is less constrained to the flow-vorticity plane (see also the inset of  Fig. \ref{fig: microstructure}g) and, as a result, the distribution is broader and less anisotropic \cite{leal1971effect}.   
			%Furthermore  % The next statement is a consequence of low aspect ratio
			Consequently, the particle is less susceptible to the effect of stochastic reorientations and thus, its distribution has a peak closer 
			to the flow axis.
			%\\ \hs{HS: I guess I don't get the reason for the shift .....} \\
			%\ac{As shown in the inset of Fig. \ref{fig: microstructure}g, these orbits are such that the time spent} constrained to the flow-vorticity 
			%plane is \ac{reduced compared to a passive particle because the effective aspect ratio is reduced.}
			%%, which is directly reflected in the distribution.   % AC: VAGUE:  what is "reflected"? shifting of peak or broadening of peak?
			%\ac{As a consequence, the particle is less susceptible to the effect of stochastic reorientations and thus, its distribution has a peak closer to %the flow axis. 
				%Additionally, the broadening of distribution is consistent with the fact that as the aspect ratio reduces, the anisotropy in the 
				%distribution reduces, until it becomes isotropic for a spherical particle \cite{leal1971effect}.}
			Hence, unlike active Newtonian suspensions \cite{saintillan2010dilute}, the microstructure of microswimmers in a viscoelastic fluid is clearly sensitive to their hydrodynamic signature being either pushers or pullers.
			
			We also examined the orientational distributions of microswimmers that do not tumble and only experience rotational thermal noise. 
			Therefore, in this case we eliminate  the last term on the right-hand side of Eq.\ (\ref{Smol}) and redefine our parameters
			$ \text{Pe}_{f} $, $ \text{Pe}_{a} $, and $ \text{De}$ replacing $ \tau $ by $ D_{r}^{-1} $. 
			%\AC{Should we mention somewhere that this is temporary redefining of parameters. Next para is with old (tumble) parameters!}
			For passive  particles and activity $ \text{Pe}_a \sim \text{O}(10) $, the microstructure is qualitatively similar to the distributions in Figs.\ \ref{fig: microstructure}a-h.
			However, microswimmers that do not tumble can have larger persistence lengths \cite{elgeti2015physics}. 
			In particular, one can genetically modify \emph{E.\ Coli} such that tumbling does not occur \cite{berke2008hydrodynamic,junot2019swimming,wolfe1988acetyladenylate}.
			Therefore, in Figs.\ \ref{fig:phase}i and j we show the orientational distributions for $\text{Pe}_{a} = \pm 100$.
			Compared to Fig.\ \ref{fig: microstructure}f, increasing the activity of a pusher moves the peak in the distribution function
			(at $ \phi=34^{\circ} $) even closer to the alignment angle ($ \phi_{\text{eq}}=36^{\circ} $) of the deterministic alignment state.
			For pullers we observe an even more drastic change upon increasing the activity. 
			The peak in the orientational distribution shifts to the quadrant where the compressional part of the shear flow occurs (see Fig. \ref{fig: microstructure}j).
			Interestingly, this is not due to the deterministic alignment of pullers as the parameters ($ \text{Wi}=0.5, \, \sigma=-0.8 $) belong to the shear-plane rotation state (see Fig. \ref{fig:phase}). 
			%However, in the discussion of Fig.\ \ref{fig:phase} we already explained  that
			As already explained in the discussion of Fig.\ \ref{fig:phase}, the rotational velocity $\dot{\IB{p}}$ slows down near $ \phi=\pi/2 $ 
			when the effective shape  factor $\Lambda + \text{Wi}  \, \sigma_c  \alpha_{1}$ becomes negative. Therefore, the active particle resembles a passive oblate spheroid \cite{brown2000orientational}.  
			Indeed, the inset of Fig.\ \ref{fig: microstructure}j shows how the puller orientation spends more time near $ \phi=\pi/2$. This again shows that even in the presence of significant noise the  microstructure is determined by the deterministic dynamics.
			%
			%distorts towards the compression axis of the shear flow, in a manner inverse to Fig. \ref{fig: microstructure}e.
			%\\ \hs{HS: Fig. 4e ??????}
			%
			
			
			
			%
			%We also examined the orientational distributions of microswimmers
			%%of active Brownian particles (ABPs) 
			%that do not tumble and only experience rotational thermal noise.
			%\\ \hs{HS: We should not use ABPs here, since they don't include flow fields typically.}\\
			%% as a decorrelation mechanism. 
			%Therefore, 
			%\hs{we skip the last term on the right-hand side of Eq.\ (\ref{Smol}) and redefine our parameters
				%$ \text{Pe}_{f} $, $ \text{Pe}_{a} $ and $ \text{De}$ with $ \tau $ replaced by $ D_{r}^{-1} $.}
			%%we replace $ \tau $ by $ D_{r}^{-1} $ in Eq. \ref{Smol} and 
			%%\hs{the definitions of}
			%%$ \text{Pe}_{f} $, $ \text{Pe}_{a} $ and $ \text{De} $.
			%For passive  
			%\hs{particles and activity}
			%%and active particles with 
			%$ \text{Pe}_a \sim \text{O}(10) $, the microstructure is qualitatively similar to
			%\hs{the distributions in Figs.\ \ref{fig: microstructure}a-h.
				%However, microswimmers that do not tumble can have larger persistence lengths \cite{???}.}
			%%Since ABPs are representative of active suspensions that can be tuned for large persistence lengths \cite{?}, 
			%\hs{Therefore, in Fig. \ref{fig:phase}i and j,} 
			%we show the 
			%\hs{orientational} 
			%distribution for
			%% such cases (
			%$ |\text{Pe}_{f}| =100 $.
			%%).
			%Similar to Fig. \ref{fig: microstructure}e, increased activity of pusher further aligns the peak ($ \phi=0.6 $) even closer to the angle 
			%that is determined by the deterministic orbits of single particle ($ \phi_{eq}=0.625 $).
			%For pullers, we observe the shift of microstructure in the compression quadrant of the shear flow \emph{i.e.,} along 
			%$ -\bten{E}^{\infty} $ as shown in Fig. \ref{fig: microstructure}j.
			%Interestingly, this is not due to the deterministic alignment of pullers, as the parameters ($ \text{Wi}=0.5, \, \sigma=-0.8 $) 
			%lie in the shear-plane rotation regime of Fig. \ref{fig:phase}. Upon examining the deterministic orbital dynamics, we find that 
			%the rotation of particle slows down near $ \phi=\pi/2 $ because the effective shape factor resembles an oblate spheroid.
			%In contrast to prolate spheroids, they are known to spend larger times with their normal orientation aligned to the gradient axis 
			%\cite{brown2000orientational}. 
			%We observe the same for this puller and show the temporal variation of $ \phi $ in the inset. 
			%It suggests that in the presence of significant noise, the microstructure distorts towards the compression axis of the shear flow, 
			%in a manner inverse to Fig. \ref{fig: microstructure}e.
			%\\ \hs{HS: Fig. 4e ??????}
			%\\ \AC{Do you have reference in mind on large persistence length of ABP?}
			%
			
			%
			%\begin{figure}[t]
			%	\centering
			%	\includegraphics[width=0.9\textwidth]{visc_Part1.pdf}
			%	\includegraphics[width=0.9\textwidth]{visc_Part2.pdf}
			%	\caption{\textbf{Particle-induced viscosity modification} $ \mu_{p}/\mu_{f} $. (a) Variation of modified viscosity ratio with shear 
				%strength ($ \text{Pe}_f $) for pusher ($ \text{Pe}_{a}>0 $), puller ($ \text{Pe}_{a}<0 $), and passive particle ($ \text{Pe}_a = 0 $). 
				%		Dashed lines correspond to viscosity ratios for Newtonian suspensions.
				%		(b) Components of total modified viscosity ratio for $ \text{Pe}_{a}=20 $.
				%		(c) Impact of activity on the modified viscosity ratio in $ \text{Pe}_{f} \ll1 $ regime for tumbling and non-tumbling particles.
				%		(d) Effect of shear strength on viscosity modification for non-tumbling particles.
				%		Dotted black lines near $ \text{Pe}_f = 0.1 $ in (a,d) correspond to the analytical results for $ \text{Pe}_{f} \ll1 $. 
				%		Other parameters: $ \text{De}=0.1, \; \lambda=10,  \; \alpha_{1}=5.44, \; \alpha_{2}=-0.75 \; \beta_{1}=0.63, \; \beta_{2}=1.18 $.}
			%	\label{fig: viscosity}
			%\end{figure}
			%
			
			
			%\vspace{4mm}
			%\noindent
			%\hs{HS: The following paragraph is reworked.}\\
			
			%\hs{HS: For the following I understood that you wanted to use $\varphi = n a^3 $} \AC{Yes} \\
			\textbf{%Bulk 
				Shear viscosity of active suspensions}.
			Finally, we evaluate the effective shear viscosity of a dilute active suspension from the total stress tensor,
			%of the \hs{active} suspension, 
			%is 
			$ \bten{\Sigma} = \bten{\Sigma}_{f} + \bten{\Sigma}_{p} $, where subscripts $f$ and $p$ refer to
			fluid and particle contributions, respectively. 
			%Then,  % Then? Do you mean "Here"?
			The deviatoric component of $\bten{\Sigma}$ provides the shear viscosity,  $ \mu = \mu_f + \varphi \mu_p $, where $ \varphi \mu_{p} = \Sigma_{p\,xy}/\dot{\gamma} $ is due to the suspended microswimmers and 
			%\ac{$ \varphi = n l^3 $} \textcolor{Green}{
				$ \varphi = n a^3 $
				%}
			is proportional to their volume fraction with $n$ being the particle density.
			% 
			%  \ac{Here, $ \varphi \ll 1 $ as we consider a dilute suspension.} 
			%\\ \hs{HS: 1. What is the meaning of Fig.5 and 6 .... Let's talk.\\
				%2. I think it is sufficient to write $ \varphi < 1 $ since $a^3$ is not the actual volume of the rod but conderably larger.} \\
			%
			Note that the shear viscosity $\mu_f$ of the second-order fluid does not depend on the shear rate, while $\mu_p$ is dependent on it.
			%\hs{as we show below.}
			To evaluate $\mu_p$, we need to average over the stresslets $\bten{\Pi} (\IB{p})$  generated by the suspended particles. Using the orientational distribution, we obtain
			%$  \ang{  \left[\IB{pppp} - \IB{pp} /3\right] : \bten{E}^{\infty}  } $.
			%Mathematically, this extra particle stress relates to the ensemble as
			\begin{equation}\label{ensemble}
				\bten{\Sigma}_{p} = n \ang{ \bten{\Pi}} = n\int_{S} \bten{\Pi} (\IB{p}) \psi(\IB{p}) \text{d}\IB{p}, 
			\end{equation}
			where $\bten{\Pi} (\IB{p})$ is the sum of three contributions \cite{saintillan2010dilute}, which give the following stress tensors:
			$ \bten{\Sigma}^{A}_{p} = n   \sigma^{*}  \left[\ang{\IB{pp}} - \bten{I} /3\right] /8\pi $ due to activity,
			$ \bten{\Sigma}^{T}_{p} = 3 n k_B T   \left[\ang{\IB{pp}} - \bten{I} /3\right] $ due to thermal reorientations, and
			$ \bten{\Sigma}^{F}_{p} $ due to the resistance of passive particles to shear. 
			For Newtonian fluids, the latter was first derived
			%calculated 
			by \citet{einstein1906new} for spherical particles, and then later generalized to elongated particles
			by \citet{hinch1975constitutive,hinch1976constitutive}.
			% for elongated particles. 
			% Sorry, but we have to give detail & credit where absolutely necessary. Otherwise a general reader will incorrectly believe that Einstien gave a general expression that is true for all particles.
			For the second-order fluid, % Definitely a comma here 
			we follow  \citet{ferec2017steady} and approximate it using the Geisekus form,
			$\bten{\Sigma}^{F}_{p} = -\mu_f n l^{3} A \dot{\gamma}  \overset{\nabla}{\ang{\IB{pp}}}/2 $, where $ A = \pi/6 \ln(2\lambda)$ is a shape factor 
			%\\ \hs{HS: In $A$ could you please check the argument $2 \lambda$.  In my expression for the rotational friction coefficient. I have only $\lambda$. Please also check the prefactors $a^3$ and $1/2$ in the expression for $\bten{\Pi}^{F}$.} \\
			to access the friction of a long slender body and $ \overset{\nabla}{\ang{\IB{pp}}} $ is the  upper-convected time derivative of the second moment of $\psi(\IB{p})$. 
			The expression for $ \bten{\Pi}^{F} $ was originally derived for dumbbells suspended in a Newtonian fluid \cite{birdTWO1987dynamics}. 
			As in Ref.\ \cite{ferec2017steady} we use it here as an approximation for the stress response of particles in a weakly viscoelastic fluid, as there are no closed form expressions available. 
			The consequences
			%\ac{effect} 
			of the second-order fluid come in through the orientational dynamics $ \dot{\IB{p}}$ 
			in Eq.\ (\ref{EOM}) and further terms in the upper-convected derivative (see `Methods: Rheology'). 
			%\\ \hs{HS: I would stay with properties. Effect one says if one does not know what to say.} \AC{`properties' is misleading because
				% properties of SOF (elasticity) are already used in the constitutive model (see first section) and we are not capturing that. We are 
				%capturing the effect of prescribed SOF's property (elasticity) on extra particle stress generated by active rods. The `effect' is clear 
				%to me \& it's the alteration of Jeffery orbits $\to$ altered extra stress.}\\
			%
			%\AC{Makes it sound like we do something in addition to $  \overset{\nabla}{\ang{\IB{pp}}} $. 
				%When we expand the upper-convected derivative we obtain the $\dot{\IB{p}}$ terms and other terms.}\\
			%\ac{The properties of the second-order fluid come in through the time derivative (orientational dynamics $ \dot{\IB{p}}$ in Eq.\ [\ref{EOM}]) %and the steady terms of the upper-convected derivative.}
			The expressions for the three contributions to the particle stress tensor
			%component of stresses 
			are pertinent to studies on slender rods and fibres (\emph{i.e.,} $ \Lambda \to 1 $),  and henceforth, we will be focusing on spheroids with larger aspect ratios
			%spheroids 
			($ \lambda=10 $).
			%
			%\\ \hs{HS: Let's discuss this point again, why we need slenderness; it's about the thermal stresses.} \AC{We can discuss again.} \\
			%
			%\\ \hs{HS: I would claim that in all three expressions for the stress tensor the terms quadratic in $\IB{p}$ are due to the
				%fact that the force dipoles and their flow fields are considered, which has nothing to do with the slender-body approximation.
				%The latter comes in only through the shape factor $A$, which should be easily changeable but we don't have to do it.
				%\\ Let's discuss this.}\AC{Ok}\\
			% AC: OKAY, I CAN IMAGINE THIS FOR ACTIVE AND THERMAL STRESS. MY ISSUE IS MOSTLY WITH THE PREFACTOR, FOR THERMAL STRESS WE HAVE A REFERENCE OR SOMEONE ALREADY DERIVED IT.
			%FOR HYDRODYNAMIC STRESS, THE DERIVATION IS ONLY DONE FOR TWO DUMBELLS ATTACHED WITH A SPRING. THE PREFACTOR MIGHT JUST NOT BE SIMPLY SUBSTITUTING ANALOGOUS FRICTIONAL COEFF FROM THERMAL STRESS. ONE HAS TO DERIVE AND FIND OUT.
			%
			%
			%In Supplementary Note B, we show that the orientational dynamics for $ \lambda=5 $ and $ 10 $ are close, and therefore results 
			%here can be related to previous sections for a qualitative understanding.
			%\\ \hs{HS: ?????? If  $ \lambda=5 $ and $ 10 $ do not make a lot of difference, why do we then use 10 ?} \\
			Using Eq.\ (\ref{ensemble}) in the definition of the particle-induced contribution to shear viscosity, $ \varphi \mu_{p} = \Sigma_{p\,xy}/\dot{\gamma} $, 
			together with our 
			%%\ac{three stress components} and 
			characteristic parameters, 
			%\\ \hs{HS: Why do we need the three stress components? The formula gives a clear description how to calculated $\mu$} \\
			we obtain
			%\ac{
				%	\begin{equation} \label{visc_res}
					%	\frac{\mu_{p}}{\mu_f} = 
					%	-\frac{A}{2} \overset{\nabla}{  \ang{\IB{pp}}  }_{xy}+ \frac{1}{ \text{Pe}_{f} } \left( 6 \tau D_r A - \frac{\text{Pe}_a}{8\pi} \right)
					%\left[ \ang{\IB{pp}} - \frac{\bten{I}}{3} \right]_{xy}.
					%\end{equation}
					%\AC{for $\varphi=nl^{3}$}}
				%\textcolor{Green}{
					\begin{equation} \label{visc_res}
						\frac{\mu_{p}}{\mu_f} = 
						-4 A \overset{\nabla}{  \ang{\IB{pp}}  }_{xy}+ \frac{8}{ \text{Pe}_{f} } \left( 6 \tau D_r A - \frac{\text{Pe}_a}{8\pi} \right) \left[ \ang{\IB{pp}} - \frac{\bten{I}}{3} \right]_{xy}. 
					\end{equation}
					%\AC{for $\varphi=na^{3}$}.}\\
				%
				%\begin{equation} \label{visc_res}
				%	\frac{\mu_{p}}{\mu_f} = 
				%	- \frac{\overset{\nabla}{  \ang{\IB{pp}}  }_{xy}}{2} + \frac{1}{ \text{Pe}_{f} } \left( 6 \tau D_r - \frac{\text{Pe}_a}{8\pi A} \right) 
				%\left[ \ang{\IB{pp}} - \frac{\bten{I}}{3} \right]_{xy}.
				%\end{equation}
				%
				%Here we have used $ D_{r} = k_B T/(2\mu l^{3} A)$
				\noindent
				We use numerical solutions of Eq.\ (\ref{Smol}) for the orientational distribution $\psi(\IB{p})$ to evaluate $\mu_p$ for
				varying $\mathrm{Pe}_f$ and $\mathrm{Pe}_a$. We also match the numerical values of $\mu_p$ with an expression, which
				we obtain using the analytic form of $\psi(\IB{p})$ from Eq.\ (\ref{pert_Smol}) in the limit of small $ \text{Pe}_{f} $ (the derivation 
				is provided in `Methods: Rheology'). In Supplementary Note 3, we also show that our results agree 
				with \citet{saintillan2010dilute} in the Newtonian limit.
				
				%
				%\hs{Finally, we evaluate the effective shear viscosity from the}
				%%The 
				%total 
				%%dimensional
				%\\ \hs{HS: Do we need dimensional here????} \\
				%stress 
				%\hs{tensor}
				%in the suspension, 
				%%is 
				%$ \bten{\Sigma} = \bten{\Sigma}_{f} + \bten{\Sigma}_{p} $, where subscript $ f $ and $ p $
				%% represent 
				%\hs{refer to}
				%fluid and particle 
				%\hs{contributions.} 
				%%to stress. 
				%Deviatoric component of the latter provides modification to the shear viscosity as $ \mu = \mu_f + \varphi \mu_p $, where $ \varphi \mu_{p} = \Sigma_{p\,xy}/\dot{\gamma} $ and $ \varphi $ is the particle volume fraction.
				%Such a distinction of stresses and viscosity is possible for the SOF as suspending medium because it does not exhibit shear-rate dependent viscosity.
				%%  - P \bten{I} + 2\mu \bten{E} + \left( \psi_{(1)}^{*} + \Psi_{2}^{*} \right) \bten{S}
				%Since the modification to viscosity arises from the stress response of suspended particles, we consider an ensemble of them as
				%%$  \ang{  \left[\IB{pppp} - \IB{pp} /3\right] : \bten{E}^{\infty}  } $.
				%%Mathematically, this extra particle stress relates to the ensemble as
				%\begin{equation}\label{ensemble}
				%	\bten{\Sigma}_{p} = n \ang{ \bten{\Pi}} = n\int_{S} \bten{\Pi} (\IB{p}) \psi(\IB{p}) \text{d}\IB{p}, 
				%\end{equation}
				%and account for the following stresses ($ \bten{\Pi} $) \cite{saintillan2010dilute}:
				%%$ \left[\ang{\IB{pppp}}- \ang{\IB{pp}} /3\right] : \bten{E}^{\infty}    $, 
				%force-dipole stress arising from activity $ \bten{\Pi}^{A} =  \sigma^{*}  \left[\ang{\IB{pp}} - \bten{I} /3\right] /8\pi $, the stress generated from thermal reorientation torque $ \bten{\Pi}^{T} = 3 k_B T   \left[\ang{\IB{pp}} - \bten{I} /3\right]  $, and particle's passive resistance to strain in the background flow $ \bten{\Pi}^{F}  $.
				%%
				%%This $ \bten{\Pi}^{P} $ 
				%Following \citet{ferec2017steady}, we approximate it by using the Geisekus form
				%% NOT GEISEKUS FLUID
				% $   \bten{\Pi}^{F} = -\mu_f a^{3} A \dot{\gamma}  \overset{\nabla}{\ang{\IB{pp}}}/2 $ that was derived for a dumbbell suspension \cite{birdTWO1987dynamics}.
				%Here $ A $ is the shape factor ($ \pi/6 \ln[2\lambda] $) and $ \overset{\nabla}{\ang{\IB{pp}}} $ is the upper-convected time derivative of the second orientation moment and is an approximation to the stress response of a particle in weakly viscoelastic fluid, as there are no closed form expressions available. 
				%It represents the Newtonian stress response corresponding to the orientational dynamics $ \dot{\IB{p}} $ \AC{\emph{i.e.,} stress response in an effective Newtonian flow that corresponds to $ \dot{\IB{p}} $}.
				%%It represents the stress response of a particle in weakly viscoelastic fluid as being approximated to an effectively modified Newtonian stress response corresponding to $ \dot{\IB{p}} $. 
				%The expressions for the three component of stresses are pertinent to studies on slender rods and fibres (\emph{i.e.,} $ \Lambda \to 1 $), and henceforth, we will be focusing on larger aspect ratio spheroids ($ \lambda=10 $).
				%In Supplementary Note B, we show that the orientational dynamics for $ \lambda=5 $ and $ 10 $ are close, and therefore results here can be related to previous sections for a qualitative understanding.
				%Using Eq. \ref{ensemble} in the definition of particle-induced shear viscosity ($ \varphi \mu_{p} = \Sigma_{p\,xy}/\dot{\gamma} $), we obtain
				%\begin{equation} \label{visc_res}
				%	\frac{\mu_{p}}{\mu_f} = 
				%	- \frac{\overset{\nabla}{  \ang{\IB{pp}}  }_{xy}}{2} + \frac{1}{ \text{Pe}_{f} } \left( 6 \tau D_r - \frac{\text{Pe}_a}{8\pi A} \right) 
				%\left[ \ang{\IB{pp}} - \frac{\bten{I}}{3} \right]_{xy}.
				%\end{equation}
				%%Here we have used $ D_{r} = k_B T/(2\mu a^{3} A)$ 
				%The numerical solution of the orientation distribution $ (\psi) $ from Eq. \ref{Smol} is used to evaluate Eq. \ref{visc_res} for an arbitrary $ Pe_f $. 
				%We also match the numerical solutions with analytical solution in the regime of small $ \text{Pe}_{f} $ (derivation provided in `Methods: Near-equilibrium rheology'). In Supplementary Note 3, we show that our results agree with \citet{saintillan2010dilute} in the Newtonian limit.
				%%\begin{equation}
				%%	\frac{\mu_{p}}{\mu_f} \approx \frac{1}{15} + \frac{6 \tau D_r - \text{Pe}_a}{5(1+6\tau D_r)} (1+\text{De} \, \text{Pe}_a \alpha_1/8\pi),
				%%\end{equation}
				%%and show that approximated viscoelastic modification $ \bten{\Pi}^{P} $ is marginal, and thus one can use the Newtonian description: $ \%left[\ang{\IB{pppp}} - \ang{\IB{pp}} /3\right]  :  \bten{E}^{\infty}  $
				
				%%\AC{Tumbling $ \to $ Why activity contribution decreases with $ Pe_f $? $ \psi $ flattens on x-axis and }
				
				%\bigskip
				%\AC{Explaining fig.5a is most crucial, (b,c,d) follow easily after that.}\\
				
				%\AC{Newtonian picture}
				
				
				
				%\hs{HS: 1. I would not do the boldface in the following. People can read and should understand it.
					%They also could think, that we think they are "stupid". \ac{The boldface was only meant for our discussion :) To convey that we use 
						%\textbf{resist} to label both passive and puller's contribution.} \AC{are you sure we go ahead with `resist'? I will remove boldface then.}
					%\\ 2. I also find your distinction between resist and oppose a bit artificial.} 
				
				
				Figure\ \ref{fig: viscosity}a shows the 
				%ratio of modified viscosity to fluid viscosity 
				particle-induced contribution to shear viscosity normalized by the bare fluid viscosity $\mu_f$,
				which is plotted versus shear strength $\text{Pe}_f$. We begin with discussing suspensions in a Newtonian fluid (dashed lines) \cite{saintillan2010dilute}.
				When a 
				%passive 
				suspension of passive rods (black) is sheared weakly ($ \text{Pe}_{f} \ll 1 $), the orientational microstructure is governed by Eq.\ (\ref{pert_Smol}) with $\text{De} = 0$ and resembles  Fig.\ \ref{fig: microstructure}a. 
				Here, the 
				%particles
				rods are aligned along the extensional axis of the applied shear flow and resist shearing ($\mu_p > 0$), which enhances viscosity.
				% and resist the shear flow, yielding a positive contribution \emph{i.e.,} viscosity enhancement.
				%
				For a suspension of pushers (brown dashed) in the weak-shear regime, the microstructure still resembles Fig.\ \ref{fig: microstructure}a,
				%\\ \hs{HS: Why is this and not (b)????} \\
				but now the extensile force dipoles support the elongational part of the shear flow, which results in $\mu_p < 0$ so that 
				the total shear viscosity is smaller than $\mu_f$.
				%and an introduction of extensile activity aids the elongational component of the shear flow, which is reflected in a negative 
				%contribution to the modified viscosity.
				Conversely, 
				%as pullers are contractile swimmers, they contribute to the viscosity enhancement. 
				pullers (green dashed) with their contractile force dipoles additionally resist shear flow and thereby enhance viscosity.
				Note that, since the microstructure of active suspensions in a Newtonian fluid
				%active Newtonian suspensions 
				is unchanged by the hydrodynamic signature of microswimmers, 
				%\\ \hs{HS: Is this so obvious?}   \AC{Yes, look at eq. 5, if the microstructure (i.e., $ \ang{} $) is independent of $ Pe_a $, viscosity ratio is %only linearly dependent on it.} \\
				%\\ \hs{HS: Why do we then have Fig. 4(b) and (c)?} \\
				the magnitude of the activity contribution
				%  \hs{!!!!} contribution from activity \hs{!!!!} 
				to $\mu_p$
				%viscosity modification 
				is identical for pushers and pullers.
				%\\ \hs{HS: I do not see this in the figure. Pullers are at 1 and Pushers at  -0.6 !!!!!!}\AC{\textbf{[clarified]} Note the positive 
					%passive contribution and see eq 5.} 
				As the shear rate increases, the magnitude of 
				%modified viscosity 
				$\mu_p$ for both passive and active rods reduces, which resembles a typical shear-thinning/thickening behaviour.
				For passive particles the microstructure is similar to Fig.\ \ref{fig: microstructure}e.
				The alignment close to the flow axis explains why the resistance of particles to the applied shear flow
				is reduced.
				% and it 
				%depicts 
				%shows that   \hs{???? skip the blue part} \ac{shear-thinning occurs because} \AC{(Connection was missing)} \hs{????}  
				%particles offer reduced \textbf{resistance} to the applied shear flow. 
				% \hs{HS: Why do I see this from $\psi(\mathbf{p})$?} \AC{\textbf{[clarified]} They align more to the flow direction compared to 
					%low shear case (fig.4a).}
				%
				For active particles, the activity-induced flow  becomes  less and less important with increasing shear rate as quantified by the inverse proportionality to $ \text{Pe}_{f} $ in Eq. (\ref{visc_res}).
				
				%\AC{To Sankalp: In your suggestion, we go back and forth in terms of particle (if not high and low shear): Passive $\to$ Active$\to$ Passive $\to$ Active (in the same paragraph). Also see that next para has Passive $\to$ Active. Thus, it is a matter of preference reg. what we want to highlight. I chose yours because we can go in order of fig. 4.}
				
				%\AC{VE picture.}
				
				Now, we discuss the results for the second-order fluid (SOF) in Fig. \ref{fig: viscosity}a (solid lines). For the suspension of 
				passive rods (black),
				%For a passive suspension
				% in second-order fluid (SOF), 
				%\ac{similar to Refs. \cite{cohen1987orientation,chung1987orientation}}, 
				we do not find an appreciable deviation from the 
				%aforementioned 
				Newtonian case until $ \text{Pe}_{f} \approx 10 $, similar to Refs. \cite{cohen1987orientation,chung1987orientation}.
				% \hs{HS: Why the citations?} \AC{They did the passive problem \& also found little deviation.}
				However, for active suspensions  the modifications are substantial and qualitatively different for 
				pushers (brown) and pullers (green).
				%Fig. \ref{fig: viscosity}a shows that in SOF, pushers' viscosity reduction 
				For pushers, the reduction in viscosity is further enhanced compared to Newtonian fluids, whereas the viscosity enhancement of pullers is  reduced. We understand this using the orientational distributions in
				%This can be understood from 
				Fig. \ref{fig: microstructure}.
				% that shows how pushers and pullers distinctly alter the orientational microstructure. 
				Pushers are even more aligned in the extensional quadrant of the applied shear flow 
				(Fig. \ref{fig: microstructure}b,d,f), which explains their further increased support of shear flow. 
				In contrast, pullers are less aligned compared to the Newtonian case (Fig. \ref{fig: microstructure}c,d,g) and, therefore,
				%\\ \hs{HS: Please leave the comma as they are.} \\
				%\textbf{resist}
				oppose shear flow less.
				%the} 
			%\\ \hs{no the here} \\
			% \AC{We should use `resist' only for passive particles (as in resistance to strain). Here we can use oppose (opp. of support).}
			%Pushers concentrate the orientation distribution in the extension quadrant of the applied shear (Fig. \ref{fig: microstructure}b,d,f), 
			%which enhances their net impact.
			%On the other hand, the impact of pullers is diminished as they diffuse the orientation distribution (Fig. \ref{fig: microstructure}c,g).
			Figure \ref{fig: viscosity}b for pushers clearly shows that the active stress ($ \bten{\Sigma}^{A} $) dominates and hence is primarily responsible for the observed viscous response.
			
			%
			%\noindent
			%\\
			%\AC{Effect of $ \text{Pe}_a $ and possibility of negative puller viscosity! :}
			In the limit of shear rate much smaller than tumbling rate, $\text{Pe}_{f} \ac{\ll} 1$
			%$\text{Pe}_{f} \rightarrow 0$
			, we can calculate $\mu_p/\mu_f$ analytically using Eq.\ (\ref{pert_Smol}) (see `Methods: Rheology')
			and obtain:
			\begin{align}
				\label{eq.mup}
				\frac{\mu_{p}}{  \mu_{f} } =  	&\frac{ 4A(5-3\Lambda) }{15} +   \frac{  48 A \tau D_{r} - \text{Pe}_{a} / \pi}{5\left( 1+ 6 \tau D_r \right)}  \left( \Lambda + \frac{\text{De} \, \text{Pe}_{a} \alpha_{1}}{8\pi}  \right)  \nonumber \\
				& -  \frac{\text{De} \, \text{Pe}_{a} A \alpha_{1}}{10\pi}. 
			\end{align}
			%\AC{The last component is quite small in comparison and hinders the main message here. Shall we remove it here (by using 
				%$ \approx $ sign?) and show it  fully in appendix.}
			%\hs{HS: As I understand you calculate only the constant term in $\mu_p/\mu_f$ and the term linear in $Pe_f$ vanishes. Correct?} 
			%\AC{Yes.}
			We discuss the result here since the influence of passive and active rods on the shear viscosity is the strongest in this limit.
			%
			%In Fig.\ \ref{fig: viscosity}c, 
			%%\\ \hs{????? This is the $\psi$ for pullers ?????} \\
			%we plot the $O(\text{Pe}_{f})$ perturbation solution that is obtained in the limit of shear rates much weaker than tumbling 
			%(see `Methods: Near-equilibrium rheology'). \AC{We can show Eq. \ref{near-eq-rheo} here.}
			In Fig.\ \ref{fig: viscosity}c we plot $\mu_p/\mu_f$ \emph{vs.} $\text{Pe}_{a}$ for tumbling microswimmers (solid lines).
			As Eq.\ (\ref{eq.mup}) shows, elasticity in the fluid adds a quadratic dependence in $\text{Pe}_{a}$, while in a Newtonian fluid the dependence is only linear \cite{saintillan2010dilute}.
			%
			%The viscosity modification in SOF is primarily quadratic in $ \text{Pe}_{a} $, unlike the linear dependency in Newtonian suspensions 
			%%\\ %\hs{HS: Why do we loose the linear dependence? Or do you mean, we have a stronger dependence on $Pe_a$ since it is
				%%quadratic.} \\
			%
			For 
			%virtually 
			all $\text{Pe}_{a}$,
			except a small region close to
			% (\hs{?}\AC{Yes, see attachment})
			$\text{Pe}_{a} =0$, the values of $\mu_p/\mu_f$
			%% "virtually" and not "always" because there is a very tiny region where modifications are above(not visible in curve)
			% the modifications 
			remain below 
			%than 
			those obtained in the Newtonian limit. 
			Thus,
			%which suggests that net impact of 
			fluid elasticity reduces the shear viscosity for both pusher and puller suspensions.
			Note that, as we increase the activity of pullers ($ \text{Pe}_{a} \lesssim -50 $), 
			%\\ \hs{HS: You write increase but then you put an upper bound on $Pe_a$.} \\
			they too can reduce the shear viscosity ($\mu_p/\mu_f < 0$) like pushers.
			%\\
			%
			%\noindent
			%This is due to the fact that for such large activities 
			%\\ \hs{the commas disturbed the flow of reading} \\
			%pullers align preferentially in the compressional quadrant of the shear flow and thereby also support the shearing fluid similar to pushers %aligned in the extensional quadrant. 
			%We have already showed this alignment in the orientational distribution of non-tumbling pullers in Fig. \ref{fig: microstructure}j.
			%
			%\hs{??} \AC{Alternative to above sentences. Much clearer.}\\
			This is a consequence of the orientational
			%orientation 
			distribution  shown in Fig. \ref{fig: microstructure}j; pullers with large activity align preferentially in the compressional 
			quadrant of the shear flow and thereby also support the shearing fluid similar to pushers aligned in the extensional quadrant.
			
			
			
			\begin{figure}[t]
				\centering
				\includegraphics[width=1.0\textwidth]{visc_Part1_new.pdf}
				\includegraphics[width=1.0\textwidth]{visc_Part2_new.pdf}
				\caption{%\hs{Caption from old Fig. 5. Changes needed?}
					\textbf{Particle-induced contribution to viscosity 
					} $ \mu_{p}/\mu_{f} $ in a second-order fluid.
					(a) Plotted \emph{vs.} shear rate $ \text{Pe}_f$
					%\hs{HS: No brackets here, also below}
					%(a) Variation of modified viscosity ratio with shear strength ($ \text{Pe}_f $) 
					for pusher ($ \text{Pe}_{a}>0 $), puller ($ \text{Pe}_{a}<0 $), and  a passive rod ($ \text{Pe}_a = 0 $) with solid lines.
					Dashed lines correspond to the Newtonian case.
					%viscosity ratios for Newtonian suspensions.
					(b) Contributions to
					%Components of \ac{
						particle-induced viscosity for $ \text{Pe}_{a}=20 $.
						(c) $ \mu_{p}/\mu_{f} $ \emph{vs.} activity $\text{Pe}_{a}$ in the limit $ \text{Pe}_{f} \rightarrow 0 $
						%Impact of activity on the modified viscosity ratio in $ \text{Pe}_{f} \ll1 $ regime 
						for tumbling and non-tumbling active rods.
						(d) $ \mu_{p}/\mu_{f} $ \emph{vs.} shear rate $ \text{Pe}_f$
						%Effect of shear strength on viscosity modification 
						for non-tumbling active rods with large persistence length, $ |\text{Pe}_{a} | = 100 $.
						Dotted black lines near $ \text{Pe}_f = 0.1 $ in (a,d) correspond to the analytical results for $ \text{Pe}_{f} \ll1 $. 
						Other parameters: $ \text{De}=0.1, \; \lambda=10,  \; \alpha_{1}=5.92, \; \alpha_{2}=-0.91 \; \beta_{1}=0.63, \; \beta_{2}=1.18 $.	
						%	
						%	\textcolor{Green}{if $\varphi=na^{3}$} \protect\includegraphics[height=1.5cm]{vol_frac.pdf} i.e., hydrodynamic volume of rod 
						% of length $ 2a $. Compare y-axis of fig.6 and fig.5.}
				}
				\label{fig: viscosity}
			\end{figure}
			
			
			
			
			
			As described earlier, we can treat the case of non-tumbling microswimmers by replacing $ \tau $ by $ D_{r}^{-1} $ in the definitions of $ \text{Pe}_{f} $, $ \text{Pe}_{a} $, and $ \text{De}$. 
			With these parameters, Eq.\ (\ref{eq.mup}) can be formulated in the limit $\tau \rightarrow \infty$ to show that the viscosity contribution of non-tumbling active rods
			%To explore this regime, we consider non-tumbling microswimmers that exhibit high persistence lengths. 
			%Fig. \ref{fig: viscosity}c shows that these particles 
			also follows a quadratic dependence in $ \text{Pe}_{a} $ (Fig.\ \ref{fig: viscosity}c, dashed lines). 
			%Since 
			Non-tumbling microswimmers can exhibit high persistence lengths.
			%\hs{?? connection of both sentences}
			Thus, in Fig.\ \ref{fig: viscosity}d we show for  $|\text{Pe}_{a}| =100 $
			%, 
			%\\ \hs{HS: There is no comma before that. This is the Phys. Rev. rule and we should keep it here.} \\
			that their contribution to viscosity is negative over a wide range of shear rates  for both pushers and pullers in a second-order fluid.
			%This viscosity reduction of pullers can be understood from the orientational microstructure (see Fig. \ref{fig: microstructure}j), 
			%where pullers were observed to align in the compressional quadrant of the shear flow.
			%In such a configuration, the contractile fluid stress of pullers are akin to the extensile stress of pushers aligned in extensional quadrant (c.f. %Fig. \ref{fig: microstructure}i).
			%
			%Hence for the systems of sheared active suspensions, non-Newtonian fluids allow the microswimmers to alter the orientational %microstructure, which can cascade into anomalous viscous response.
			%
			%For the case of second-order fluid considered here, we find that fluid's elasticity 
			%\hs{HS: Let's decide later if we need a concluding sentence here and take something of the following.} \\ 
			Hence, we find that the elasticity of a fluid always results in a reduced total viscosity as compared to suspensions in a Newtonian fluid. 
			
			
			
			
			
			
			\section*{Discussion}
			In this article we show how activity influences the dynamics of a sheared microswimmer suspension in a viscoelastic fluid at an individual level and in the bulk.
			%and bulk scale.
			%When the suspending media is Newtonian, elongated passive particles obey Jeffery orbits \cite{jeffery1922motion}, which drift to a `log-rolling' state in the presence of weak viscoelasticity \cite{gauthier1971_shear-flow,leal1975slow,brunn1977slow}.
			At the individual level, the orientational dynamics of passive rods is well known from Jeffery's orbits in Newtonian shear 
			\cite{jeffery1922motion} and `log-rolling' orbits in viscoelastic shear flow \cite{leal1975slow,brunn1977slow}.
			Our analytical result  [Eq.\ (\ref{EOM})], % I have always used "Analytical"
			derived for elongated active spheroids in a second-order fluid, demonstrates how the active flow field of a swimmer modifies the orientational dynamics. 
			%extensile 
			Extensile swimmers like \emph{E.coli} (pushers) drift to the shear plane and align at an angle to the flow direction. For contractile swimmers like \emph{C. reinhardtii} (pullers), activity effectively lowers 
			%the  
			their aspect ratio.
			% of the  
			%\hs{microswimmer}.
			%\ac{spheroid}. % We could say ROD if we were doing slender body theory.
			With increasing dipole strength, pullers show log rolling,
			%\\ \hs{HS: I just mention all the three possibilities!!!!} \\
			%(\AC{repetition not needed})} 
		transition to shear-plane rotation, and ultimately align at an angle against the flow direction. 
		The latter occurs for strong pullers at dipole strengths  relevant to artificial swimmers with large propulsion speed  \cite{ren20193d,aghakhani2020acoustically}.
		
		To demonstrate the impact of the individual swimmer dynamics on the bulk rheological behaviour, we employ the Smoluchowski equation to evaluate the orientational probability distribution of  an ensemble of  active spheroids
		called the suspension microstructure \cite{hinch1975constitutive}. 
		Accounting for tumbling and rotary diffusion, we find that the microstructure substantially depends on the hydrodynamic signature of the swimmer unlike suspensions in Newtonian fluids \cite{hatwalne2004rheology,saintillan2010dilute,nambiar2017stress}.
		%We find that pushers are more likely to align in the extension quadrant, 
		Pushers align more strongly in the extensional quadrant of the applied shear flow, while the alignment of pullers is significantly weaker.
		%%whereas 
		%the distribution of pullers is more 
		%\hs{diffuse}
		%diffused.
		%
		%As a consequence, the bulk rheological response is modified because it is sensitive to how particles are arranged in relation to the applied shear.
		This activity-specific microstructure 
		%\hs{strongly determines} 
		significantly modifies the effective shear viscosity compared to Newtonian 
		fluids \cite{lopez2015turning}.
		In particular, the viscosity reduction of a dilute pusher suspension is more pronounced under viscoelastic shear flow,
		% compared to Newtonian fluids \cite{lopez2015turning}, 
		while the viscosity enhancement of pullers is weaker. Thus, the activity of a microswimmer contributes in 
		two ways to the rheology of swimmer suspensions in a viscoelastic fluid; directly through its active stresses and indirectly by coupling to
		the elasticity of the fluid and thereby influencing the orientational microstructure. 
		In particular, in the weak-shear limit the particle-induced contribution to viscosity scales quadratically with the 
		% square of
		activity $ \text{Pe}_{a} $.
		%
		%We find that the modification scales as $ \text{Pe}_{a}^{2} $ in the weak-shear limit, where $ \text{Pe}_{a} $ is the swimmer persistence length.}
	%\\ \hs{HS: I would leave the sentence out, it disturbs the reading.} \\
	%
	% As a result, the elasticity of a second-order fluid always reduces the viscous response of microswimmers
	% Reduces is an ambigous term, let's be more direct
	%
	%As a result, 
	In total, we find that the elasticity of a second-order fluid always reduces the total viscosity of the microswimmer suspension.
	%% \ac{always brings out a negative} 
	%viscous response 
	%\hs{of} 
	%microswimmers. 
	Especially for pushers, this can help to reach the regime of superfluidity at lower volume fractions compared to 
	Newtonian fluids.
	%
	%We show how these activity-specific modifications in the microstructure play a role in modifying the shear viscosity.
	%We find that viscosity reduction of a dilute pusher suspension (well-known from the Newtonian studies \cite{lopez2015turning}) 
	%is now stronger in viscoelastic shear, whereas pullers' viscosity enhancement is weaker.
	%We correlate these results with the modifications in orientation distribution.
	%Hence, we find that elasticity of the medium allows the activity to participate in shear rheology in a twofold manner: directly 
	%via the active stresses and through the direct modification of the microstructure.
	%
	%For second-order fluid, the elasticity of a fluid always results in a reduced viscous response of microswimmers. Especially 
	%for pushers, this can help to reach the regime of superfluidity at lower volume fractions than predicted under Newtonian assumption.
	%
	
	
	Biologically relevant fluids are viscoelastic. In this article we presented a first systematic study of the individual dynamics and the bulk 
	rheology of microswimmers suspended in such fluids. Earlier investigations on active suspensions in quiescent \cite{bozorgi2011effect} 
	and vortical viscoelastic flows \cite{ardekani2012emergence} assumed conventional Newtonian Jeffery orbits.  In the light of our results, 
	these studies on the collective behaviour need to be revisited. 
	%\hs{need to be reconsidered}  RECONSIDERED can be seen as strong words for our potential reviewers
	% Our investigations makes \hs{two?} main \AC{Certainly not two, even if they are considered `main'. We need to show/accept that this 
		%is a highly idealized problem. There are other strong assumptions of our work when we compare with actual microbial systems. Shape of 
		%actual swimmers is not an axisymmetric spheroid; flagella chirality; flow field around the swimmer is assumed to be force dipole but we 
		%don't know what it is in viscoelastic fluid.} 
	% \\
	Our investigations makes several
	assumptions to address the complexity of microbial flows,  which offers ample opportunities for future developments. Biological fluids can be more complex and exhibit shear-thinning/-thickening properties or several characteristic relaxation times, which requires more evolved modeling using, for example, the 
	%Oldroyd-B 
	FENE-P or Giesekus model.
	%\\ \hs{HS: What do these models describe.\\
		%For me: In what sense is Oldroyd-B better than second-order fluid.} \\
	%
	%What are the single models good for????} \AC{What do you mean by single model? Also, I would suggest FENE-P instead of Oldroyd B. The %former is more robust and latter is mostly used for weak viscoelastic effects.}\\
%
Furthermore, mucus besides being significantly shear-thinning has relaxation times larger than 1 second, which is of the order of
the bacterial mean free tumble time \cite{li2021microswimming,spagnolie2022swimming,mucus_lai2009micro}.
Thus, noise becomes non-Markovian and memory needs to be incorporated in the stochastic description, for example, within
a generalized Langevin equation \cite{narinder2018memory}. 
%Finally, bacteria often move in a heterogeneous environment
%such as biofilms, which needs .... \cite{zhang2018reduced}. 
%\\ \hs{HS: I am not sure why we should mention the biofilms. How is this connected to rheology.} \AC{Right, not needed.}
%
%Furthermore, highly heterogeneous microstructure of biofilms may render the continuum approximation frail, where Brownian 
%dynamics simulations can be employed \cite{zhang2018reduced}. 
%
%\\




%
%Earlier investigations on active suspensions in quiescent \cite{bozorgi2011effect,li2016collective} and vortical viscoelastic 
%flows \cite{ardekani2012emergence} had assumed that particles obey Newtonian Jeffery orbits. 
%In the light of our results, revisiting these mean-field theories of collective behaviour should be considered \cite{li2016collective}.
%%
%We also note that  in order to obtain qualitative insights we make several assumptions and only scrape the surface of quantifying 
%the complexity of microbiological flows. 
%For instance, mucus have larger relaxation times than assumed here ($ >1 s $) and are also significantly shear-thinning 
%\cite{li2021microswimming,spagnolie2022swimming,mucus_lai2009micro}.
%We envision that insights from current work will spawn future developments in employing the stochastic effects in strongly non-Newtonian 
%fluids. This is not straightforward because of the stronger memory effects that need to be accounted in the non-Markovian Langevin 
%framework \cite{narinder2018memory}.
%Furthermore, highly heterogeneous microstructure of biofilms may render the continuum approximation frail, where Brownian 
%dynamics simulations can be employed \cite{zhang2018reduced}. 
%

\newpage

\section*{Methods}
\label{sec.methods}	



\noindent
\textbf{Hydrodynamic model}\\
The hydrodynamics around the spheroidal particle is governed by the mass and momentum conservation as
$ \nabla \cdot \IB{V} = 0, \;  \nabla \cdot \bten{T}=0. $
Here $ \bten{T} $ is the total stress tensor defined as $ \bten{T} =-P \, \bten{I} + 2 \, \bten{E} + \text{Wi} \, \bten{S} $ \cite{bird1987dynamics}, where 
$
\bten{S} =	  4 \bten{E}\cdot \bten{E} + 2 \delta \overset{\Delta}{\bten{E}}
$. The lower-convected derivative is defined as
\begin{equation}
	\overset{\Delta}{\bten{E}} = \left[\frac{\partial\bten{E}}{\partial t}  + \left( \IB{V} \cdot \IB{\nabla} \right) \bten{E} + \bten{E} \cdot \IB{\nabla} \IB{V} + \IB{\nabla V}^{T} \cdot \bten{E}  \right] \,.
\end{equation}
Since we consider a steady shear flow and a steadily active swimmer, we can disregard the temporal derivative in the above stress tensor.
In particle frame of reference, the boundary condition on its surface is the no-slip condition
$ 	\IB{V} = \IB{\Omega}^{p} \times \IB{r} $,
where the rotational velocity $ \IB{\Omega}^{p} $ is currently unknown and will be evaluated by using the torque-free condition.
We split the velocity field into the disturbance field $\IB{v}$ and background flow field $\IB{V}^{\infty}  = \bten{E}^{\infty} \cdot \IB{r} + \IB{\Omega}^\infty \times \IB{r} $ \emph{i.e.,} $ \IB{V}=\IB{v}+\IB{V^{\infty}} $.
The equations governing the disturbance flow field are 
%\begin{equation}\label{GE}
%	\nabla \cdot \IB{v} = 0, \quad  - \nabla p + \nabla^{2} \IB{v} = - \text{Wi} \, \bten{s}, \text{ where}
%\end{equation}
%\begin{equation}
%	\bten{s}= \bten{e} \cdot \bten{e} + \bten{w} + \delta(\overset{\Delta}{\bten{e}}+\overset{\Delta}{\bten{w}}),
%	\label{s}
%\end{equation}
\begin{eqnarray}  \label{GE_supp}
	\nabla \cdot \IB{v} &=& 0, \quad  - \nabla p + \nabla^{2} \IB{v} = - \text{Wi} \, (\nabla \cdot \bten{s}), \text{ where} \nonumber \\ 
	\bten{s} &=& 4 (\bten{e} \cdot \bten{e} + \bten{w}) + 2 \, \delta(\overset{\Delta}{\bten{e}}+\overset{\Delta}{\bten{w}}).
\end{eqnarray}
Here $ \bten{e} $ is the rate of strain tensor for the disturbance flow $ (\nabla \IB{v} + \nabla \IB{v}^{\dagger})/2 $, whereas $ \bten{w} $, $ \overset{\Delta}{\bten{e}} $ and $ \overset{\Delta}{\bten{w}} $ are the different constituents of the polymeric tensor ($ \bten{s}$) due to the disturbance flow field.
Specifically, $ \bten{w} $ is the interaction tensor defined as $  \bten{E}^{\infty}\cdot \bten{e} + \bten{e} \cdot  \bten{E}^{\infty} $, arising from the interaction between background flow and disturbance field;
%Here $ \overset{\Delta}{\bten{e}} $ is the lower-convected derivative of $ \bten{e}$, 
and $ \overset{\Delta}{\bten{w}} $ is the lower-convected derivative of $ \bten{e} $ and $  \bten{E}^{\infty} $ with respect to $ \IB{V}^{\infty} $ and $ \IB{v} $, respectively:
\begin{eqnarray}
	%	\overset{\Delta}{\bten{e}} &=& \IB{v} \cdot \nabla \bten{e} + \bten{e} \cdot \nabla \IB{v}^{\dagger} + \nabla \IB{v} \cdot \bten{e}, \nonumber \\
	&\overset{\Delta}{\bten{w}}  = {\IB{V}^\infty} \cdot \nabla \bten{e} + \bten{e} \cdot \nabla {\IB{V}^\infty}^{\dagger} + \nabla {\IB{V}^\infty} \cdot \bten{e} \\
	&\qquad \qquad \qquad+  \IB{v} \cdot \nabla  \bten{E}^{\infty}+  \bten{E}^{\infty}\cdot \nabla \IB{v}^{\dagger} + \nabla \IB{v} \cdot  \bten{E}^{\infty}. \nonumber
\end{eqnarray}
The boundary conditions of the disturbance field at particle surface and far away are
\begin{equation}
	\IB{v} =  \Delta \IB{\Omega} \times \IB{r}  -  \bten{E}^{\infty} \cdot \IB{r}  \quad \mbox{at\ } \IB{r} \in S \quad \text{and}  \quad   \IB{v} \to \IB{0} \quad \mbox{as\ } \; \IB{r} \to \infty,
\end{equation}
respectively. Here $  \Delta \IB{\Omega}  $ is the difference in particle angular velocity and background vorticity ($   \IB{\Omega}^{p} - \IB{\Omega}^{\infty} $). 
%	%disturbance tensor
%	derivative of $ \bten{e}$ (also known as the Rivlin-Eriksen tensor), $ \bten{w} $ is the `interaction tensor' (arising from the interaction between background flow and disturbance field), and $ \overset{\Delta}{\bten{w}} $ is its lower-convected derivative.



%\medskip

For small values of $ \text{Wi} $, the disturbance field variables are expanded as $ 	f=f^{(0)} + \text{Wi} \; f^{(1)} + \cdots. $
%\begin{equation}	\label{pert}
%
%\end{equation}
Here, $ f $ is a generic field variable that represents velocity ($ \IB{v} $), pressure ($ p $), and angular velocity ($ \IB{\Omega}^{p} $). 
We substitute this expansion in the Eq. (\ref{GE_supp}) and obtain the $ O(1) $ Stokes problem as
\begin{eqnarray}\label{GE}
	\nabla \cdot \IB{v}^{(0)} &=& 0, \quad  - \nabla p^{(0)} + \nabla^{2} \IB{v}^{(0)} = 0
\end{eqnarray}
with the boundary condition at particle surface:
\begin{equation}
	\IB{v}^{(0)} =  \Delta \IB{\Omega}^{(0)} \times \IB{r}  -  \bten{E}^{\infty} \cdot \IB{r}  \quad \mbox{at\ } \IB{r} \in S,
\end{equation}
and a decaying condition at infinity ($ \IB{v}^{(0)} \to 0 $).
Using the finite multipole expansion around the spheroid \cite{chwang1975hydromechanics,einarsson2015rotation,abtahi2019jeffery}, we obtain the disturbance velocity for a passive spheroid at $ O(1) $ as 
%	\begin{widetext}
	{
		\small
		\begin{align}\label{dist}
			v^{(0)}_{i}  &=  \mathcal{Q}_{ij,k}^{R} \varepsilon_{jkl} \Bigl\{  \left[  \mathbb{A}^{R} p_{l} p_{m} + \mathbb{B}^{R} (\delta_{lm}-p_{l} p_{m})  \right] \Delta\Omega_{m}  
			\nonumber \\
			&  \!\!\!\!\!\!	+\mathbb{C}^{R} \varepsilon_{lmn} p_{m} \mathsf{E}^{\infty}_{no} p_{o} \Bigr\} 
			\nonumber \\
			& \!\!\!\!\!\! + \left(\mathcal{Q}_{ij,k}^{S} + \chi \mathcal{Q}_{ij,llk}^{Q}\right) \Bigl\{  \left[ \mathbb{A}^{S} {\mathsf{p}^{A}_{jklm}} + \mathbb{B}^{S} {\mathsf{p}^{B}_{jklm}} + \mathbb{C}^{S} {\mathsf{p}^{C}_{jklm}}  \right]  \mathsf{E}_{lm}^{\infty}
			\nonumber \\
			&\!\!\!\!\!\!  - \mathbb{C}^{R} \left( \varepsilon_{jlm} p_{k}p_{m} +\varepsilon_{klm} p_{j} p_{m} \right) \Delta \Omega_{l} \Bigr\} .
		\end{align}
	}
	%	\end{widetext}
Here $ \mathcal{Q}$  represent the spheroidal multipoles \emph{i.e.} integral representation of fundamental singularities spread on a line extending from one foci to another ($ -c $ to $ c $, where $ c=\sqrt{\lambda^{2}-1}/\lambda $);
$ \mathcal{Q}^{R}, \; \mathcal{Q}^{S} $ represent rotlet and stresslet around the spheroid, respectively.
Here $ \chi = \frac{1}{8(\lambda^{2}-1)} $ represents the strength of higher order quadrupolar field ($ \mathcal{Q}_{ij,llk}^{Q} $). Since we focus on elongated particles, we exclude this component for computational ease ($ \lambda=\lbrace 5,10 \rbrace $ correspond to $ \chi = \lbrace0.005,0.001 \rbrace $).
Following \citet{einarsson2015rotation}, we simplify these multipoles as
\begin{subequations}\label{spheroid_multipole}
	\begin{align}
		\mathcal{Q}_{ij,k}^{R} &= \left( \delta_{ik} r_{j} - \delta_{ij} r_{k} \right) J_{3}^{0} + \left( \delta_{ij} p_{k} - \delta_{ik} p_{j} \right)J_{3}^{1} 
		\\
		\mathcal{Q}_{ij,k}^{S} &= \delta_{jk} r_{i} J_{3}^{0} - \delta_{jk} p_{i} J_{3}^{1} - 3  r_{i} r_{j} r_{k} J_{5}^{0}  \\ \nonumber
		&+ 3(p_{i}r_{j}r_{k} + p_{j}r_{i} r_{k} +p_{k} r_{i} r_{j} ) J_{5}^{1} \\ \nonumber
		&- 3(r_{i}p_{j}p_{k} + r_{j}p_{i} p_{k} +r_{k} p_{i} p_{j} ) J_{5}^{2} 
		+ 3 p_{i} p_{j} p_{k} J_{5}^{3}.
	\end{align}
	%\begin{align}
	%	\mathcal{Q}_{ij,k}^{Q} &= -6(\delta_{jk} r_{i} + \delta_{ik} {r}_{j} + \delta_{ij} r_{k}) K_{5}^{0}  + 30 r_{i}r_{j} r_{k} K_{7}^{0} \\ \nonumber
	%	&+ 6 (\delta_{jk} p_{i} + \delta_{ik} p_{j} + \delta_{ij} p_{k} ) K_{5}^{1}  \\ \nonumber
	%	&- 30(p_{i}r_{j}r_{k} + p_{j}r_{i} r_{k} +p_{k} r_{i} r_{j} ) K_{7}^{1} \\ \nonumber
	%	&+ 30(r_{i}p_{j}p_{k} + r_{j}p_{i} p_{k} +r_{k} p_{i} p_{j} ) K_{7}^{2} 
	%	- 30 p_{i} p_{j} p_{k} K_{7}^{3}
	%\end{align}
\end{subequations}
%	Here $ I $, $ J $ and $ K $ represent various integrals defined as
%	\begin{align}
	%		I_{a}^{b} = \int_{-c}^{c}  \frac{\xi^{b}}{|\IB{r}-\xi \IB{p}|^{a}} \text{d}\xi ; \; J_{a}^{b} = c^{2} I_{a}^{b}  -  I_{a}^{b+2};
	%		\;  K_{a}^{b} = c^{2} J_{a}^{b}  -  J_{a}^{b+2}. \nonumber
	%	\end{align}
Here $ I $ and $ J $ represent various integrals defined as
\begin{align}
	I_{a}^{b} = \int_{-c}^{c}  \frac{\xi^{b}}{|\IB{r}-\xi \IB{p}|^{a}} \text{d}\xi ; \; J_{a}^{b} = c^{2} I_{a}^{b}  -  I_{a}^{b+2}. 
\end{align}
In Eq. (\ref{dist}), the fourth order orientation tensors ($ \mathsf{p}^{A} $, $ \mathsf{p}^{B} $, $ \mathsf{p}^{C} $) and other coefficients ($ \mathbb{A}, \mathbb{B}, \mathbb{C} $) are described in the Supplementary Note 1.

At $ O(1) $, we add the activity using the far-field descriptions, which consists of a force-dipole (FD), source-dipole (SD), rotlet-dipole (RD), and force-quadrupole (FQ) \cite{spagnolie_lauga_2012}:
{
	\small
	\begin{subequations}\label{active_fields}
		\begin{equation}
			\IB{v}^{(0)}_{FD}	= \sigma_{FD} \left(\frac{-  \IB{r} }{r^{3}} +  \frac{3  \IB{r} \left(\IB{r} \cdot \IB{p}  \right)^{2} }{r^{5 }} \right)
		\end{equation}
		\begin{equation}
			\IB{v}^{(0)}_{SD}	= \sigma_{SD} \left( \frac{ 3 \IB{rr}}{r^{2}}  -\bten{I} \right)  \frac{\IB{p}}{2 r^{3}}
		\end{equation}
		\begin{equation}
			\IB{v}^{(0)}_{RD}	= \sigma_{RD} \left[  \frac{ 3 \IB{p}\cdot \IB{r} }{r^{5}}   \left( \IB{p} \times \IB{r} \right)  \right]
		\end{equation}
		\begin{equation}
			\IB{v}^{(0)}_{FQ}	= \sigma_{FQ}  \left\lbrace \frac{\IB{p}}{r^{3}} -  \frac{3}{r^{5}} \left[ 3(\IB{p}\cdot \IB{r} )  \IB{r}  + (\IB{p}\cdot \IB{r})^{2} \IB{p}  - \frac{5 (\IB{p}\cdot \IB{r})^{3} \IB{r} }{r^{2}} \right]   \right\rbrace
		\end{equation}
	\end{subequations}
}

%	$ \Delta \Omega $ is the difference between the angular velocities of flow and particle: $ \IB{\Omega}^{\infty} - \IB{\Omega}^{p} $.





%\medskip

%	\subsection{First order problem}
At $ O(\text{Wi}) $, the governing equation is
\begin{eqnarray}\label{dist_Wi}
	\nabla \cdot \IB{v}^{(1)} &=& 0, \quad  - \nabla p^{(1)} + \nabla^{2} \IB{v}^{(1)} = -  \, (\nabla \cdot \bten{s}^{(0)}), \text{ where} \nonumber \\ 
	\bten{s}^{(0)} &=& 4 (\bten{e}^{(0)} \cdot \bten{e}^{(0)} + \bten{w}^{(0)}) + 2 \, \delta(\overset{\Delta}{\bten{e}}^{(0)}+\overset{\Delta}{\bten{w}}^{(0)}),
\end{eqnarray}
The boundary condition at the particle surface being
\begin{equation}\label{BC_Wi}
	\IB{v}^{(1)} = \IB{U}_{p}^{(1)} + \IB{\Omega}_{p}^{(1)} \times \IB{r}  \quad \mbox{at\ } \IB{r} \in S,
\end{equation}
with a decaying condition at infinity ($ \IB{v}^{(1)} \to 0 $).
%
Conventionally, a solution to Eqs. (\ref{dist_Wi}) and (\ref{BC_Wi})  (\emph{i.e.,} $ \IB{v}^{(1)} $) is sought, which, on the implementation of torque-free condition, reveals the modification of Jeffery orbits ($ \IB{\Omega}_{p}^{(1)} $).
We employ the reciprocal theorem \cite{elfring2015theory,abtahi2019jeffery} to avoid solving for the first order flow field and directly obtain the rotation velocity as
\begin{equation}\label{reciprocal}
	\IB{\Omega}_{p}^{(1)} \cdot \IB{T}^{t} = \int_{V_{f}} \bten{s}^{(0)} : \nabla \IB{v}^{t} \; dV. \quad  
	%		\ac{evaluated \; for \; fixed \; \theta, \phi}
\end{equation}
Here, superscript $ t $ refers to the test/auxiliary problem where the particle rotates at a unit velocity $ \IB{\Omega}^{t} = \IB{e}_{i} $, where $ i $ corresponds to either of the three Cartesian coordinates.
The test torque is:
\begin{equation}
	\IB{T}^{t} = \bten{R} \cdot \IB{\Omega}^{t}, \, \mbox{where\ } \bten{R} =  \frac{64 \pi c^{3}}{3}  \left[  \mathbb{A}^{R} \IB{pp} + \mathbb{B}^{R}(\bten{I}-\IB{pp})  \right] 
\end{equation}
%
We solve Eq.\ (\ref{reciprocal}) for three test field torques (along three Cartesian coordinates) and obtain the following system of equations:
$ \bten{R}^{\dagger} \cdot \IB{\Omega}_{p}^{(1)} = \left\{ \mathcal{I}_{1}, \mathcal{I}_{2}, \mathcal{I}_{3} \right\} $,
where $ \dagger $ represents transpose and $ \mathcal{I}_{i} $ is the solution of volume integral (\ref{reciprocal}) for the test field in $ i^{\text{th}} $ direction. We further simplify the above relation by noting that $ \bten{R} $ is a symmetric matrix and
%\begin{equation}
%	\IB{\Omega}_{p}^{(1)} = \bten{R}^{-1} \cdot \left\{ \mathcal{I}_{1}, \mathcal{I}_{2}, \mathcal{I}_{3} \right\}.
%\end{equation}
%To obtain the viscoelastic modification to the equation of motion (i.e. $ \dot{\IB{p}}^{(1)} $), we note that 
$ \dot{\IB{p}}^{(1)}=\IB{\Omega}_{p}^{(1)} \times \IB{p} $:
\begin{equation}\label{pdot}
	\dot{\IB{p}}^{(1)}	= \left( \bten{R}^{-1} \cdot \left\{ \mathcal{I}_{1}, \mathcal{I}_{2}, \mathcal{I}_{3} \right\} \right)	\times \IB{p}
\end{equation}
%
Eq.\ (\ref{pdot}) is evaluated over a discretized angular grid, where each point requires solving the volume integral in Eq. (\ref{reciprocal}). 
The polymeric stress  therein ($ \bten{s}^{(0)} $) is given by Eq. (\ref{GE_supp}) and is entirely dependent on $ O(1) $ disturbance flow fields \emph{i.e.,} passive disturbance in Eq.\ (\ref{dist}) and active disturbance in Eq. (\ref{active_fields}).
%
We find that out of all active fields, only force-dipole ($ \sim r^{-2} $) provides a modification at the current order of approximation; the rest decays in odd powers of distance ($ \sim r^{-3} $), and due to antisymmetry, give zero contribution to the volume integral at $ O(Wi) $.






\medskip

\textbf{Orientation dynamics}\\
%In addition to a numerical solution to Eq. (\ref{pdot}), 
We evaluate 
%the equation of orientation modification 
$ \dot{\IB{p}}^{(1)} $ analytically by noting that it stems from the leading order polymeric stress [specifically $ \bten{s}^{(0)} $ in Eq.\ (\ref{reciprocal})];
its form in Eq.\ (\ref{GE_supp}) suggests that the modification of a passive particle will be quadratic in the flow gradient tensor, which can be decomposed in symmetric $ \bten{E}^{\infty} $ and antisymmetric $ \bten{O}^{\infty} $ (rotation-rate tensor) components.
Following \citet{einarsson2015rotation}, this can be written in the general form as:
\begin{equation}\label{p_passive}
	\dot{p}_{i}^{(1,P)} = \mathsf{K}_{ijklm}^{(1)} \mathsf{E}^{\infty}_{jk} \mathsf{E}^{\infty}_{lm}  +  \mathsf{K}_{ijklm}^{(2)} \mathsf{E}^{\infty}_{jk} \mathsf{O}^{\infty}_{lm}  + \mathsf{K}_{ijklm}^{(3)} \mathsf{O}^{\infty}_{jk} \mathsf{O}^{\infty}_{lm},
\end{equation}
where superscript $ P $ denotes passive contribution.
The coefficients of the fifth-order tensor $ \bten{K} $ are composed of all possible permutations of the orientation vector $ \IB{p} $ with $ \bten{E}^{\infty} $ and $ \bten{O}^{\infty} $:
\begin{align}
	\mathsf{K}_{ijklm}^{(i)} = \sum_{n=5!} \left( \beta_{n}^{[1]} p_{n_{1}} p_{n_{2}} p_{n_{3}} p_{n_{4}} p_{n_{5}} + \beta_{n}^{[2]} p_{n_{1}} p_{n_{2}} p_{n_{3}} \delta_{n_{4}n_{5}}  \right.
	\nonumber	\\ 
	\left.
	+ \beta_{n}^{[3]} p_{n_{1}} \delta_{n_{2}n_{3}} \delta_{n_{4}n_{5}}  \right) \nonumber.
\end{align}
Here $ \beta $ represents  the unique coefficients for all the $ 5! $ terms.

When activity is added in the hydrodynamics, the form of polymeric stress tensor in Eq. (\ref{GE_supp}) suggests that the gradients of disturbance velocity (directed with $ \IB{p} $) will multiply with gradients of passive disturbance (which originate from $ \bten{E}^{\infty} $ and $ \bten{O}^{\infty} $) to yield further modification to $\dot{\IB{p}}^{(1)}$.
Thus, the contribution resulting from interaction of active disturbances ($ \IB{p} $) with background flow ($ \bten{E}^{\infty} $ and $ \bten{O}^{\infty} $) takes the following form at $ O(\text{Wi}) $:
\begin{equation}\label{p_active}
	\dot{p}_{i}^{(1,A)} = \sigma \left[  \mathsf{L}_{ijk}^{(1)} \mathsf{E}^{\infty}_{jk}  +  \mathsf{L}_{ijk}^{(2)} \mathsf{O}^{\infty}_{jk} \right]  ,
\end{equation}
where coefficients of the third-order tensor $ \bten{L} $ are composed of the following combinations
\begin{equation}
	\mathsf{L}_{ijk}^{(i)} = \sum_{n=5!} \left( \alpha_{n}^{[1]} p_{n_{1}} p_{n_{2}} p_{n_{3}} + \alpha_{n}^{[2]} p_{n_{1}}  \delta_{n_{2}n_{3}}  \right). \nonumber
\end{equation}
Here $ \alpha $ represent the unique coefficients for all the $ 3! $ terms.
We simplify Eq.\ (\ref{p_passive}) and (\ref{p_active}) by using the symmetry arguments and noting that magnitude of $ \IB{p} $ is always unity. We obtain the following six irreducible terms:
\begin{align}\label{pdot_all}
	\dot{\IB{p}}^{(1)} &=  \bten{E}^{\infty}: \IB{pp} \left[ \beta_{1} (\bten{I}-\IB{pp}) \cdot ( \bten{E}^{\infty} \cdot \IB{p}  ) +  \beta_{2} \, \bten{O}^{\infty} \cdot \IB{p}   \right] \nonumber  \\ 
	& \; + (\bten{I}-\IB{pp})   \cdot   \left[ \beta_{3} (\bten{E}^{\infty} \cdot  \bten{E}^{\infty}) \cdot \IB{p} 
	+ \beta_{4}  (\bten{O}^{\infty} \cdot  \bten{E}^{\infty}) \cdot \IB{p}   \right]  \nonumber  \\ 
	& \; + \sigma \alpha_{1}  (\bten{I}-\IB{pp})  \cdot \left( \bten{E}^{\infty} \cdot \IB{p} \right) + \sigma  \alpha_{2}  \,  \bten{O}^{\infty} \cdot \IB{p}
	.
\end{align}
We calculate the coefficients ($ \alpha_{i} $ and $ \beta_{i} $) by calculating $ \dot{\IB{p}}^{(1)} $ numerically for six independent orientations ($ \IB{p} $) and solve the system of equations from Eq.\ (\ref{pdot_all}) to extract the coefficients.
We find that $ \beta_{3} $ and $\beta_{4}$ are nearly zero ($ \lesssim 10^{-5} $) and thus neglect them.
We obtain the analytical approximation as
\begin{align}\label{pdot_supp}
	\dot{\IB{p}}^{(1)} &=   \left(  \bten{I} -  \IB{pp}  \right) \cdot  \left(   \bten{E}^{\infty} \cdot \IB{p}   \right) \left[   \sigma  \alpha_{1} + \beta_1 \,  \bten{E}^{\infty}: \IB{pp} \right] 
	\nonumber \\
	& \; + \bten{O}^{\infty} \cdot \IB{p} \left[   \sigma  \alpha_{2} +  \beta_2 \,  \bten{E}^{\infty}: \IB{pp}   \right] ,
\end{align}
We use the above equation in the main text as Eq. (\ref{EOM}), where we write $ \bten{O}^{\infty}\cdot \IB{p}$ as $ \IB{\Omega}^{\infty} \times \IB{p} $.
In Supplementary Note 2A, we show the weak variation of these coefficients with particle aspect ratio $ \lambda $ and the viscometric coefficient $ \delta $.
%the match between the predictions of alignment angle obtained from Eq. (\ref{pdot_supp}) with that obtained by solving Eq. (\ref{pdot}) numerically; therein we also show the weak variation of these coefficients with particle aspect ratio $ \lambda $ and the viscometric coefficient $ \delta $.


\medskip

\textbf{Dynamical analysis}\\
\textit{1. Eigenvalue analysis of alignment dynamics}. Following \citet{bretherton1962}, we explain the alignment of a particle in the shear plane by extracting a fundamental matrix solution of Eq. (\ref{EOM}).
For this, we first consider the non-dimensional Jeffery equation in the expanded form as
\begin{align}
	\dot{\IB{p}} & 
	%	= \bten{O}^{\infty}\cdot \IB{p} + \Lambda \left[ (\bten{I}-\IB{pp}) \cdot (\bten{E}^{\infty} \cdot \IB{p})  \right] \nonumber \\ 
	= \bten{O}^{\infty}\cdot \IB{p} + \Lambda \left[ \bten{E}^{\infty} \cdot \IB{p} - \IB{p} (\bten{E}^{\infty}:\IB{pp}) \right],
\end{align}
which can be represented in an unconserved form that facilitates a fundamental matrix solution 
\begin{equation}\label{EOM_unconserved}
	\dot{\IB{q}} = \left( \Lambda \bten{E}^{\infty} + \bten{O}^{\infty}  \right) \cdot \IB{q}.
\end{equation}
Here $ \frac{\IB{q}(t)}{|\IB{q}(t)|} = \IB{p}(t)  $ and $ |\IB{q}(t)| = \exp[ \Lambda \left( \bten{E}^{\infty} : \IB{pp} \right) t ] $ represents the exponential elongation of $ \IB{q} $. 
We study the angular dynamics of $ \IB{q} $, as its normalization yields back the orientation vector $ \IB{p} $.
The solution to Eq. (\ref{EOM_unconserved}) is
\begin{equation}
	\IB{q}(t) = \exp\big[ ( \Lambda \bten{E}^{\infty}+\bten{O}^{\infty}  )t ] \IB{q}(0).
\end{equation}
The eigensystem of the above exponential matrix determines the orbital dynamics of spheroid.
For the two-dimensional shear flow, the eigenvalues of the matrix in the exponent ($ \Lambda \bten{E}^{\infty} + \bten{O}^{\infty} $) are: $ 0, \, \pm  \sqrt{\Lambda ^2-1}/2  $. The non-zero eigenvalues are purely imaginary as $ 0 < \Lambda < 1 $, where $\Lambda=1$ for an infinitely slender particle. 
As a consequence of this imaginary pair, we observe degenerate infinite solutions of orbits in Newtonian fluids.
For the case of pure elongational flow ($ \bten{O}^{\infty} \equiv 0 $),  the eigenvalues are real: $ 0, \, \pm \Lambda/2 $, which reveals the absence of periodic orbits.
The normalized eigenvector corresponding to the positive eigenvalue determines the equilibrium orientation: $ \left\lbrace {1}/{\sqrt{2}}, \, {1}/{\sqrt{2}}, \, 0  \right\rbrace $ \emph{i.e.,} $ 45 $\textdegree$\,$ or $ 225 $\textdegree$\,$ in the two extension quadrants.



We now use Eq. (\ref{EOM}) for the second-order fluid and consider only the active viscoelastic component because the passive effects do not contribute to the alignment [this assumption is later verified by matching the results with numerical integration of complete Eq. (\ref{EOM})].
In this case, the equation of motion for the unconserved vector $ \IB{q} $ is
\begin{equation}\label{EOM_avtive_unconserved}
	\dot{\IB{q}} = \left[   \bten{E}^{\infty} (\Lambda+ \text{Wi} \sigma \alpha_{1}) + \bten{O}^{\infty} (1+ \text{Wi} \sigma \alpha_{2})  \right] \cdot \IB{q},
\end{equation}
which yields the solution as
%	\begin{equation}
	%	\IB{q}(t) = \exp \left\lbrace \left[  \bten{E}^{\infty}  (\Lambda+ \text{Wi} \sigma \alpha_{1}) + \bten{O}^{\infty} (1+ \text{Wi} \sigma \alpha_{2}) \right] t   \right\rbrace \IB{q}(0).
	%\end{equation}
	\begin{equation} \label{active_EOM}
		\IB{q}(t) = \exp \big[ \left[  \bten{E}^{\infty}  (\Lambda+ \text{Wi} \sigma \alpha_{1}) + \bten{O}^{\infty} (1+ \text{Wi} \sigma \alpha_{2}) \right] t   ] \IB{q}(0).
	\end{equation}
	We obtain the eigenvalues of the fundamental matrix as
	\begin{align}
		0, \, \pm \frac{1}{2} \sqrt{\Lambda ^2 -1 +2 \sigma  \text{Wi} \left(\text{$\alpha_{1} $} \Lambda  -\text{$\alpha_{2} $}\right) + \text{Wi}^{2} \sigma^{2} \left( \alpha_{1}^{2} - \alpha_{2}^{2} \right)} \nonumber .
	\end{align}
	These eigenvalues turn real when $ \sigma > \frac{ 1 - \Lambda}{\text{Wi} (\alpha_{1} - \alpha_{2})} $ or $ \sigma < \frac{ -1 - \Lambda}{\text{Wi} (\alpha_{1} + \alpha_{2})}  $. 
	%This result matches exactly with $ \sigma_c $ depicted in Fig. \ref{fig:phase}.
	For instance, for a particle of aspect ratio $ \lambda=5 $, the negative and positive critical values are $ \sigma_{crit} = -0.448/\text{Wi}; \;  0.016/\text{Wi} $, which matches with Fig. \ref{fig:phase} generated from the integration of full Eq. (\ref{EOM}) (see Supplementary Note 2B). 
	In the alignment regimes of Fig. \ref{fig:phase}, we further find that the normalized eigenvector corresponding to the positive eigenvalue matches the angle of alignment ($ \phi_{eq} $) obtained numerically.
	
	In addition to validating our results of Fig.\ \ref{fig:phase}, Eq. (\ref{active_EOM}) provides a key insight that the contribution from active disturbances is to effectively alter the strength of elongation (prefactor of $ \bten{E}^{\infty} $) and the rotation rate (prefactor of $ \bten{O}^{\infty} $).
	%Using this approximation, we find the condition for alignment of a pusher as $ \sigma > (1 - \Lambda)/(\text{Wi} \alpha_{1}) $. 
	The first prefactor is the more decisive as $ \alpha_{1} \gg |\alpha_{2}| $.
	It suggests that a pusher ($ \sigma>0 $) disturbs the local shear flow in order to increase the weight of elongation. 
	Once the activity exceeds the critical limit, the local shear flow transforms into an effective elongation flow whose axis of elongation points in the direction of the normalized eigenvector associated with the positive eigenvalue.
	%
	As pusher's activity further increases, the locally elongated flow asymptotically approaches the pure elongation state where the impact of $ \bten{O}^{\infty} $ is negligible in comparison (similar to our discussion of $ \bten{O}^{\infty} \equiv 0 $ case in previous paragraph).
	%
	%We also observe this for pullers ($ \sigma<0 $), however, the required magnitude of dipole strength is far more than that required for pullers because the factor $ \text{Wi} \sigma \alpha_{1} $ competes against the elongation factor $ \Lambda $.
	
	
	
	
	\medskip
	
	\noindent
	\textit{2. Stability analysis of orbits}. 
	Here we find the stability exponent of the complete non-linear Eq. (\ref{EOM}) to analyze the onset of shear-plane rolling state as shown in Fig. \ref{fig:orbits},\ref{fig:phase}.
	For Newtonian fluids, $ \theta = \pi/2 $ is one of the infinite neutral Jeffery orbits.
	For SOF, the polymeric stress in the fluid lifts this degeneracy. 
	To quantify this, we write down the angular dynamics as
	\begin{equation}
		\dot{\theta}=f(\theta,\phi), \qquad  \dot{\phi} = g(\theta,\phi), \tag{\theequation a,b}
	\end{equation}
	and perform a Taylor expansion near the shear-plane: $ \theta(t) = \pi/2 + \epsilon(t) $. Here $ \epsilon $ represents the deviation from Jeffery orbit and and we determine its growth \emph{i.e.,} whether it grows or shrinks with time.
	%
	At $ O(\epsilon) $ we obtain
	\begin{equation}
		\frac{d \epsilon}{d t} =  \epsilon \,  \left. \frac{\partial f}{\partial \theta} \right\vert_{\theta=\pi/2},  
	\end{equation}
	which can be simplified and integrated over an orbit to obtain
	\begin{equation} \label{eps}
		\epsilon  = \epsilon_{0} \; \exp\left[  
		\bigintssss_{0}^{-2\pi} \left( \frac{1}{g(\theta,\phi)} \frac{\partial f}{\partial \theta} \right)_{\theta=\pi/2} \, d\phi
		\right] .
	\end{equation}
	Here the integration is over $ -2\pi $ because the particle's rotation due to shear is in $ -\phi $ direction.
	Eq. (\ref{eps}) can be expressed in the form of Lyapunov exponent ($ \epsilon = \epsilon_{0} \exp[\mathcal{L}  T] $) \cite{strogatz} as
	\begin{equation}\label{lya}
		\mathcal{L}  \sim \frac{1}{T} \bigintsss_{0}^{-2\pi}    \left[ \frac{1}{\dot{\phi}} \frac{\partial \dot{\theta}}{\partial \theta} \right]_{\theta=\pi/2}  \;    d\phi ,
	\end{equation}
	where $ T $ is the time period of a Jeffery orbit $  2\pi (\lambda+\lambda^{-1})/\dot{\gamma} $.
	We show the solution to Eq. (\ref{lya}) in the inset of Fig. \ref{fig:phase}.
	
	
	
	\medskip
	
	\textbf{Kinetic model}\\
	%Eq. (\ref{Smol}) determines the probability distribution  of microswimmer's orientation $ {\psi}(\IB{p}) $.
	\noindent
	\textit{1. Near-equilibrium microstructure}.
	First, we detail the evaluation of microstructure near  equilibrium ($ \text{Pe}_{f} = \dot{\gamma} \tau \ll 1 $) \emph{i.e.,} when stochastic reorientation dominates the shear-induced reorientation.
	In this limit, we expand the orientation distribution as
	\begin{equation}
		\psi = \psi_{(0)} + \text{Pe}_{f} \, \psi_{(1)} + \cdots.
	\end{equation}
	We substitute this in Eq. (\ref{Smol}) and collect the zeroth and first order terms in $ \text{Pe}_{f} $. At $ O(1) $, we get $ \psi_{(0)} = 1/4\pi $ as the isotropic microstructure. 
	At $ O(\text{Pe}_{f}) $,  we have
	\begin{equation}\label{Kinetic_model_supp}
		\nabla_{p} \cdot\Bigl\{
		\left[   \dot{\IB{p}}^{(0)} + {\rm De} \, {\rm Pe}_a \, \dot{\IB{p}}^{(1)}_{A}/8\pi  \right] \psi_{(0)}
		\Bigr\}
		- \tau D_r \nabla_{p}^{2} \psi_{(1)} + \psi_{(1)} = 0.
	\end{equation}
	Here we denote the equation of motion (Eq. \ref{EOM}) as having three parts: $ \dot{\IB{p}} = \dot{\IB{p}}^{(0)} + \text{De} \text{Pe}_{f} \dot{\IB{p}}^{(1)}_{P} + \text{De} \text{Pe}_{a} \dot{\IB{p}}^{(1)}_{A} $, where subscripts represent passive and active components.
	Upon substituting this in Eq. (\ref{Kinetic_model_supp}), we find that the $ \dot{\IB{p}}^{(1)}_{P}  $ term only contributes at $ O(\text{Pe}_{f}^{2}) $, and is thus neglected. Upon simplification, we find that the microstructure at $ O(\text{Pe}_{f}) $ is governed by
	%\begin{equation}
	%	\nabla_{p} \cdot\Bigl\{
	%\left[   \dot{\IB{p}}_n + {\rm De} \, {\rm Pe}_a \, \dot{\IB{p}}_{nn}^{A}/8\pi  \right] \psi_{(0)}  \Bigr\}
	%=	-\frac{3(\Lambda+\alpha_{1} \text{De} \text{Pe}_{a}) }{4 \pi} \bten{E}^{\infty} : \IB{pp} .
	%\end{equation}
	%Thus, Eq. (\ref{Kinetic_model_supp}) becomes
	\begin{equation}
		- \tau D_r \nabla_{p}^{2} \psi_{(1)} + \psi_{(1)} = \frac{3 }{4 \pi} (\Lambda+\alpha_{1} \text{De} \text{Pe}_{a}/8\pi) \bten{E}^{\infty} : \IB{pp} .
	\end{equation}
	Following \cite{chen1996rheology,nambiar2017stress}, we solve the above inhomogeneous linear differential equation using the Green's function $ G(\IB{p}|\IB{p}') $, which is governed by 
	\begin{equation} \label{Green_GE}
		- \tau D_r \nabla_{p}^{2} G(\IB{p}|\IB{p}') + G(\IB{p}|\IB{p}') = \delta(\IB{p}-\IB{p}') ,
	\end{equation}
	where $ \delta $ represents the Dirac-delta function. 
	Following \cite{arfken1999mathematical}, we expand this $ G $ into spherical harmonic series as
	\begin{equation}\label{Green_expansion}
		G(\IB{p}|\IB{p}')  = \sum_{n=0}^{\infty} \sum_{m=-n}^{n} C_{n,m}(\IB{p}') Y_{n}^{m}(\IB{p}).
	\end{equation}
	Here $ C_{n,m} $ are the series coefficients and $ Y_{n}^{m} (\IB{p}) $ represent the spherical harmonics. 
	Similarly, the Dirac-delta function can also be related to the spherical harmonics as $ \delta(\IB{p}-\IB{p}') =  \sum_{n=0}^{\infty} \sum_{m=-n}^{n} Y_{n}^{m}(\IB{p}) \overline{Y_{n}^{m}(\IB{p}')} $, where $ \overline{Y_{n}^{m}} $ represents the corresponding complex conjugate spherical harmonic \cite{arfken1999mathematical}.
	Eq. (\ref{Green_GE}) is now expressible as
	\begin{align}\label{Green_GE_expansion}
		\sum_{n=0}^{\infty} \sum_{m=-n}^{n} C_{n,m}(\IB{p}') \Bigl\{ -   \tau D_r & \nabla_{p}^{2} \left[ Y_{n}^{m}(\IB{p}) \right]  + Y_{n}^{m}(\IB{p})  \Bigr\} = \nonumber \\
		&  \sum_{n=0}^{\infty} \sum_{m=-n}^{n} Y_{n}^{m}(\IB{p}) \overline{Y_{n}^{m}(\IB{p}')}    	.
	\end{align}
	Using the properties of spherical harmonics \cite{arfken1999mathematical}, we substitute $ \nabla_{p}^{2} Y_{n}^{m} = -n(n+1) Y_{n}^{m} $, take the inner product with respect to $ \overline{Y_{N}^{M}(\IB{p})} $ on both sides of Eq. (\ref{Green_GE_expansion}),	and use the orthogonality property to obtain
	\begin{align}
		C_{n,m} (\IB{p}') =  \frac{\overline{Y_{n}^{m}(\IB{p}')}}{1+ \tau D_r n(n+1)},
	\end{align}
	where the normalization condition yields $ C_{0,0} = 1/\sqrt{4\pi} $.
	We finally use the Green's function solution to find $ \psi_{(1)} $:
	%	\begin{align}\label{psi_1_expanded}
		%		\psi_{(1)} &= \frac{3(\Lambda+\alpha_{1} \text{De} \text{Pe}_{a}/8\pi) }{4 \pi}  \int_{S}  G(\IB{p}|\IB{p}') (\bten{E}^\infty : \IB{p}' \IB{p}') \text{d}\IB{p}',   \nonumber \\
		%		\psi_{(1)} &= \frac{3(\Lambda+\alpha_{1} \text{De} \text{Pe}_{a}/8\pi) }{4 \pi}  \, \times \nonumber \\
		%		& \bigintsss_{S} \sum_{n=0}^{\infty} \sum_{m=-n}^{n}  Y_{n}^{m}(\IB{p})   \frac{\overline{Y_{n}^{m}(\IB{p}')}}{1+ \tau D_r n(n+1)}
		%		\left( \bten{E}^{\infty} :  \IB{p}'\IB{p}' \right) \text{d}\IB{p}',
		%	\end{align}
	%	We expand $ \bten{E}^{\infty}: \IB{p}'\IB{p}' $ in spherical harmonics as $ \text{i} \sqrt{2 \pi/15} [Y_{2}^{-2}(\IB{p}') -Y_{2}^{2}(\IB{p}')] $. This suggests that only $ n=2 $ and $ m=\pm 2 $ modes yield non-zero terms in Eq. (\ref{psi_1_expanded}). Upon simplification, we obtain the first order solution as
	%	\begin{equation}
		%		\psi_{(1)}	=  \frac{3(\Lambda+\alpha_{1} \text{De} \text{Pe}_{a}) }{4 \pi}      \left[   \text{i} \sqrt{\frac{2 \pi}{15}}  \frac{  Y_{2}^{-2}(\IB{p}) -Y_{2}^{2}(\IB{p})  }{1+ 6\tau D_r}  \right].
		%	\end{equation}
	%	This near equilibrium solution can be recasted back into the vectorial form as
	%	which yields the following first order solution
	\begin{equation} \label{psi_near_eq}
		\psi_{(1)}	=  \frac{3 }{4 \pi} (\Lambda+\alpha_{1} \text{De} \text{Pe}_{a}/8\pi)      \left[   \frac{   \bten{E}^{\infty}: \IB{p}\IB{p} }{1+ 6\tau D_r}  \right].
	\end{equation}
	In the limit of weak shear, this solution matches with the numerically obtained $ \psi $ for arbitrary $ \text{Pe}_{f} $, whose evaluation is discussed next. 
	
	
	
	\medskip
	
	
	\noindent
	\textit{2. Numerical solution valid for arbitrary $ \text{Pe}_{f} $}.
	Now we detail the numerical solution of Eq.\ (\ref{Smol}), which uses the decomposition of $ \psi $ as $ \sum_{n=0}^{\infty} \sum_{m=-n}^{n} C_{n,m}(\IB{p}) Y_{n}^{m}(\IB{p}). $
	%	\begin{equation}
		%		\psi(\theta,\phi) =  \sum_{n=0}^{\infty} \sum_{m=-n}^{n} C_{n,m}(\IB{p}) Y_{n}^{m}(\IB{p}).
		%	\end{equation}
	We substitute this expansion in Eq. (\ref{Smol}) and use the properties of spherical harmonics to obtain
	\begin{align}\label{Smol_Numerical}
		-\frac{1}{4\pi} + \sum_{n=0}^{\infty} \sum_{m=-n}^{n} C_{n,m} \left[ 1+\tau D_r n(n+1) \right] Y_{n}^{m} +& 
		\nonumber \\
		\text{Pe}_{f}  \sum_{n=0}^{\infty} \sum_{m=-n}^{n} C_{n,m}  \mathcal{H}(Y_{n}^{m}) & = 0  .
	\end{align}
	Here $ \mathcal{H}(Y_{n}^{m})  $ is a collection of expressions obtained by simplifying $ \nabla_{p} \cdot \left(  \dot{\IB{p}} \psi  \right)  $ . These are represented in terms of angular momentum operators for computational convenience \cite{chen1996rheology} (detailed in Supplementary Note 3).
	%This helps us in using orthogonal properties of Clebsch-Gordon coefficients.
	We take an inner product with $ \overline{Y_{i}^{j}} $ on both sides of Eq. (\ref{Smol_Numerical}) and use the orthogonality property to obtain
	\begin{align}
		-\frac{1}{4\pi} \int_{S} \overline{Y_{i}^{j}}  \text{d}\IB{p}
		+  C_{i,j} \left[ 1+\tau D_r i(i+1) \right] +& 
		\nonumber \\
		\text{Pe}_{f}  \sum_{n=0}^{\infty} \sum_{m=-n}^{n} C_{n,m} \int_{S} \overline{Y_{i}^{j}}  \mathcal{H}(Y_{n}^{m})  \text{d}\IB{p}  &  =0.
	\end{align}
	Since the $ n=0 $ mode is already known from normalization, the above equations can be recasted as the following linear system of equations where $ C_{i,j} $ (for $ i \geq 1 $) is unknown:
	\begin{align}
		C_{i,j} \left[ 1+\tau D_r i(i+1) \right] + \; \; & 
		\nonumber \\
		\text{Pe}_{f}  \sum_{n=1}^{100} \sum_{m=-n}^{n} C_{n,m} \int_{S} \overline{Y_{i}^{j}} & \mathcal{H}(Y_{n}^{m})  \text{d}\IB{p}    =
		\nonumber \\
		-	& \text{Pe}_{f}  C_{0,0} \int_{S} \overline{Y_{i}^{j}}  \mathcal{H}(Y_{0}^{0})  \text{d}\IB{p}  .
		,
	\end{align}
	To find $ C_{i,j} $ coefficients, the above equations are solved using the Clebsch-Gordon coefficient formulation, where first 100 modes in $ n $ are used.
	We note that although the numerical approach is valid for arbitrary $ \text{Pe}_{f} $, there is an indirect upper bound of weak viscoelasticity 
	that we must adhere to, as we are using the SOF model. 
	In non-dimensional terms this bound is: $ \text{De} \, \text{Pe}_{f}  < 1 $.
	Thus, for $ \text{De}=0.1 $, we explore the results within an upper bound of $ \text{Pe}_f = 10 $.
	
	
	
	
	\medskip
	
	
	
	\textbf{Rheology.}
	First we show the expanded version of the flow-induced component of the particle stress 
	\begin{equation}
		\bten{\Sigma}^{F}_{p} = -\mu_f n l^{3} A \dot{\gamma}    {\overset{\nabla}{\bten{a}_{2}  }} /2 
		%		\\
		%		\frac{\bten{\Sigma}^{f}_{p}}{n \mu a^{3}A/\tau} = - { \text{Pe}_{f} } \, {\overset{\nabla}{\bten{a}_{2}  }}    / {2} ,
	\end{equation} where $ \bten{a}_{i} $ is a shorthand notation for the ensemble of orientation moment of $ i^{\text{th}} $ order: $ \ang{\IB{p}^{\otimes i}} $. Expanding the upper-convected derivative, we obtain
	\begin{align} \label{GF}
		\!   {\overset{\nabla}{\bten{a}_2 }}  
		= &  \left(\Lambda - 1 \right)  \left(  \bten{E}^{\infty} \cdot \bten{a}_{2}  + \bten{a}_{2} \cdot  \bten{E}^{\infty}  \right)  -  2 \Lambda \bten{a}_{4} : \bten{E}^{\infty}
		\nonumber \\
		& \!\!\! \!\!\!\!\!\!    + \text{De} \Big[ \text{Pe}_{f} \beta_{1}  \left(  \bten{E}^{\infty} \cdot \bten{a}_{4} : \bten{E}^{\infty}  - 2 \bten{E}^{\infty} : \bten{a}_{6} : \bten{E}^{\infty} +   \bten{E}^{\infty} :  \bten{a}_{4} \cdot \bten{E}^{\infty}  \right)  	\nonumber \\
		& \!\!\! \!\!\!\!\!\!  
		+ \text{Pe}_{f} \beta_{2} \left( \bten{O}^{\infty} \cdot \bten{a}_{4} : \bten{E}^{\infty}  -  \bten{E}^{\infty} : \bten{a}_{4} \cdot \bten{O}^{\infty}   \right)
		\nonumber \\
		& \!\!\! \!\!\!\!\!\!    + \text{Pe}_{a} \alpha_{1}  \left(  \bten{E}^{\infty} \cdot \bten{a}_{2}   -  2 \bten{a}_{4}:\bten{E}^{\infty}   + \bten{a}_{2} \cdot \bten{E}^{\infty}    \right)
		\nonumber \\
		& \!\!\! \!\!\!\!\!\!    + \text{Pe}_{a} \alpha_{2}  \left(  \bten{O}^{\infty} \cdot \bten{a}_{2}   -  \bten{a}_{2} \cdot \bten{O}^{\infty}
		\right)
		\Big].
	\end{align}
	This $ \bten{\Sigma}^{F}_{p}  $ matches with the Newtonian bulk stress for $ \text{De}=0 $ \cite{hinch1976constitutive}.
	For near-equilibrium results ($ \text{Pe}_{f} \ll 1 $), the above expression is substituted in Eq. (\ref{visc_res}) and orientation distribution from Eq. $ (\ref{psi_near_eq}) $ is used to perform the ensemble integral.
	Since the passive viscoelastic contribution is $ O(\text{Pe}_{f}^{2}) $, it does not contribute at the $ O(\text{Pe}_{f}) $ calculations.
	The viscosity relates to the deviatoric particle-induced stress as $ 	\varphi\mu_{p} = {\Sigma_{p \, xy}} /{\dot{\gamma}}  , $ where $ \varphi=n  a^{3} $ is the volume fraction.
	This viscometeric relation can be simplified to obtain the near-equilibrium viscosity ratio
	%	\begin{align}
		%		\frac{\mu_{p}}{  \mu_{f} }   =  \left( \frac{\Sigma_{p \, xy}  }{ \varphi \mu_{f}/\tau} \right) \frac{1}{\text{Pe}_{f}},
		%		%	\nonumber \\
		%		%	 &  \!\!\! \! \! \! \! \! \!\!\!\!\!\!\!   
		%	\end{align}
	%	where the dimensionless stress in the bracket can be substituted from Eq. (\ref{visc_res}) to obtain the dimensionless viscosity as
	%\ac{
		%		\begin{align}\label{near-eq-rheo}
			%	\frac{\mu_{p}}{  \mu_{f} } =  &A	\frac{ 5-3\Lambda }{30} +   \frac{  6 A \tau D_{r} - \frac{\text{Pe}_{a}}{8 \pi } }{5\left( 1+ 6 \tau D_r \right)}  \left( \Lambda + \frac{\text{De} \, \text{Pe}_{a} \alpha_{1}}{8\pi}  \right) \nonumber \\
			%	& -  \frac{\text{De} \, \text{Pe}_{a} A \alpha_{1}}{80\pi}. 
			%\end{align}
			%}
		\begin{align}
			\frac{\mu_{p}}{  \mu_{f} } =  	&\frac{ 4A(5-3\Lambda) }{15} +   \frac{  48 A \tau D_{r} - \text{Pe}_{a} / \pi}{5\left( 1+ 6 \tau D_r \right)}  \left( \Lambda + \frac{\text{De} \, \text{Pe}_{a} \alpha_{1}}{8\pi}  \right)  \nonumber \\
			& -  \frac{\text{De} \, \text{Pe}_{a} A \alpha_{1}}{10\pi}. 
		\end{align}
		The last term in the above expression is negligible and is generated from the non-Newtonian component of $ \bten{\Sigma}_{p}^{F} $. 
		Thus, the major modifications due to fluid elasticity scale as $ \text{Pe}_{a}^{2} $, as also depicted in Fig. \ref{fig: viscosity}d.
		
		\medskip
		
		\section*{Acknowledgements}
		Support from the Alexander von Humboldt Foundation is gratefully acknowledged. We also acknowledge financial support from the Collaborative Research Center 910 funded by the Deutsche Forschungsgemeinschaft.
		S.N acknowledges support from the Swedish Research Council (under grant no. 638-2013-9243) and Nordita, which is partially supported by NordForsk.
		A.C thanks Jonas Einarsson for his advice on the effective equation of motion.
		
		
		\section*{Author contributions}
		A.C.  and H.S. planned the research. A.C performed the calculations. A.C., S.N.,  and H.S. analyzed the results and wrote the manuscript.
		
		\section*{Conflicts of interest}
		There are no conflicts to declare.
		
		
		
		
%		\bibliography{Akash}
		
		
		\clearpage
		
		
		
		
		
		
		
		
		
		
		
		
		
		
		
		
		
		
		
		
		
		
		
		
		
		
		
		
		
		
		
		
		
		
		
		
		
		
		
		
		
		
		
		
		
		
		
		
		
		
		
		
		
		
		
		
		
		
		
		
		
		
		
		
		
		
		
		
		
		
		
%%%%%%%%%% Merge with supplemental materials %%%%%%%%%%
\pagebreak
\widetext
\begin{center}
	\textbf{\large {Supplementary material for: ``Orientational dynamics and rheology of  active suspensions in weakly viscoelastic flows''}}
\end{center}
%%%%%%%%%% Merge with supplemental materials %%%%%%%%%%
%%%%%%%%%% Prefix a "S" to all equations, figures, tables and reset the counter %%%%%%%%%%
\setcounter{equation}{0}
\setcounter{figure}{0}
\setcounter{table}{0}
\setcounter{page}{1}
\makeatletter
\renewcommand{\theequation}{S\arabic{equation}}
\renewcommand{\thefigure}{S\arabic{figure}}
%\renewcommand{\bibnumfmt}[1]{[S#1]}
%\renewcommand{\citenumfont}[1]{S#1}
%%%%%%%%%% Prefix a "S" to all equations, figures, tables and reset the counter %%%%%%%%%%


\section{Coefficients of the multipole expansion}
The tensorial coefficients of the spheroidal multipoles are
\begin{subequations}
	\begin{equation}
		\mathsf{p}^{a}_{jklm} = (p_{j} p_{k} - \delta_{jk}/3) (p_{l} p_{m} - \delta_{lm}/3),
	\end{equation}
	\begin{equation}
		\mathsf{p}^{b}_{jklm} = p_{j} p_{l} \delta_{km} + p_{j} p_{m} \delta_{kl} + p_{k} p_{l} \delta_{jm} + p_{k} p_{m} \delta_{jl} - 4p_{j} p_{k} p_{l} p_{m},
	\end{equation}
	\begin{equation}
		\mathsf{p}^{c}_{jklm} = -\delta_{jk} \delta_{lm} + \delta_{jl} \delta_{km} + \delta_{jm} \delta_{kl} + p_{l} p_{m} \delta_{jk} + p_{j} p_{k} \delta_{lm} - p_{j} p_{m} \delta_{kl} - p_{k} p_{m} \delta_{jl} - p_{j} p_{l} \delta_{km} - p_{k} p_{l} \delta_{jm} + p_{j} p_{k} p_{l} p_{m}.
	\end{equation}
\end{subequations}
The other scalar coefficients are
\begin{subequations}
	\begin{equation}
		\mathbb{A}^{R} = \frac{\sqrt{\lambda^{2}-1}}{4(C-\lambda^{3}+\lambda)},
		\;\; \mathbb{B}^{R} = \frac{\sqrt{\lambda^{2}-1}(\lambda^{2}+1)}{4(C+\lambda^{3}-\lambda-2C\lambda^{2})},
		\;\; \mathbb{C}^{R} = \frac{(\lambda^{2}-1)^{3/2}}{4(C+\lambda^{3}-\lambda-2C\lambda^{2})},
	\end{equation}
	\begin{equation}
		\mathbb{A}^{S} = \frac{(\lambda^{2}-1)^{3/2}}{4(C-3\lambda^{3}+3\lambda+2C\lambda^{2})},
		\;\; \mathbb{B}^{S} = \frac{(\lambda^{2}-1)^{3/2} (C\lambda+\lambda^{4}-3\lambda^{2}+2)}{8(C+\lambda^{3}-\lambda-2C\lambda^{2})(-3C\lambda + \lambda^{4} + \lambda^{2}-2)},
		\;\; \mathbb{C}^{S} = \frac{(\lambda^{2}-1)^{3/2}}{2(3C+\lambda^{3} +2\lambda^{5}-7\lambda^{3}+5\lambda)}.
	\end{equation}	
\end{subequations}
Here $ C=\sqrt{\lambda^{2}-1} \coth^{-1}\left[ \lambda/\sqrt{\lambda^{2}-1} \right] $.










\section{Further results of orientation dynamics}



\subsection{Coefficients of equation of motion}


Fig. \ref{fig:supp_coeff} shows the coefficients of the analytical equation of motion. They vary weakly with aspect ratio $ \lambda $ and the viscometric parameter $ \delta $, which typically lies between $ -0.7 $ and $ -0.5 $.
%Aspect ratio smaller than $ 5 $ are erroneous with the elongation approximation ($ \chi \to 0 $) implemented in our simulations. 
Since the coefficients vary weakly with  $ \lambda $ and the integration time is very large for $  \lambda >10 $, we consider aspect ratios between 5 and 10.

\begin{figure}[h]
	\centering
	\includegraphics[width=0.7\textwidth]{supp_coeff.pdf}
	\caption{Coefficients of equation of motion (Eq. 1 in main text). $ \alpha_{i} $ represent active coefficients and $ \beta_{i} $ are passive coefficients.}	
	\label{fig:supp_coeff}
\end{figure}

\subsection{Alignment angle}

\begin{figure}[h]
	\centering
	\includegraphics[width=0.4\textwidth]{supp_phase.pdf}
	\caption{Comparison of alignment angles obtained by integrating the full equation of motion (dots) with those found using analytical approximation (line). Blue: $ \lambda=5 $, red: $ \lambda=10 $. Vertical dashed lines separate the regions of alignment. 
		Other parameters: $\delta=-0.6$, $ \text{Wi}=0.1 $ .}
	\label{fig:supp_phase}
\end{figure}
Fig. \ref{fig:supp_phase} shows the comparison of angles of alignment obtained by integrating the analytical equation of motion (Eq. \textcolor{red}{1} of main text) and those from the eigenvalue analysis. 
Note that the phase diagram has a weak dependency on aspect ratio.
%Numerical approach: we evaluate Eq. (\textcolor{red}{23}) over an angular grid of $ 160 \times 160 $ points, where each point requires solving Eq. (\textcolor{red}{19}). 



\subsection{Passive spheroid in weakly viscoelastic shear flow}


The non-dimensional equation of motion for the orbits is 
\begin{equation}
	\dot{\IB{p}} =  \left(  \bten{I} -  \IB{pp}  \right) \cdot  \left(   \bten{E}^{\infty} \cdot \IB{p}   \right) \left[  \Lambda + \text{Wi}  \, \sigma  \alpha_{1} + \text{Wi} \, \beta_1 \,  \bten{E}^{\infty}: \IB{pp} \right] 
	+ \IB{\Omega}^{\infty} \times \IB{p} \left[  1 + \text{Wi} \, \sigma  \alpha_{2} + \text{Wi} \, \beta_2 \,  \bten{E}^{\infty}: \IB{pp}   \right].
\end{equation}
This can be written in the trigonometric form as
\begin{subequations}
	\begin{equation}
		\dot{\theta}	=  \frac{\Lambda}{4} \sin 2 \theta  \sin 2 \phi  +   \text{Wi} \,\beta_{1} \sin ^3\theta \cos \theta \sin ^2\phi \cos ^2\phi
	\end{equation}
	\begin{equation}
		\dot{\phi}	=   -\frac{1 }{\lambda^2+1} \left[  \lambda^2 \sin ^2\phi+\cos ^2\phi\right]  + \frac{\text{Wi}}{4}  \left[ \beta_{1} \cos (2 \phi )- \beta_{2} \right] \sin ^2\theta  \sin 2 \phi  .
	\end{equation}
\end{subequations}
To quantify the log-rolling and compare it with \citet{brunn1977slow}, we use the following transformation \cite{leal1971effect}:
\begin{equation}
	\theta = \arctan \left( \frac{C \lambda}{\sqrt{\lambda^{2} \cos^{2}\phi + \sin^{2} \phi}} \right), \quad  \tau = \arctan \left( \frac{\tan \phi}{\lambda} \right)   .
\end{equation}
Here $ C $ represents the orbit constant ($ C=0 $ and $ C=\infty $ being the log-rolling and shear-plane rotation orbits, respectively) and $ \tau $ is essentially the time coordinate that is scaled with $ {T}/{2\pi}  $, where $ T $ is the Jeffery orbit.
Using the above relation, we obtain the evolution equation of the orbit constant as
\begin{align} \label{Ctau}
	\dot{C}(t) &= \text{Wi} \frac{ \left[C^3 \left(\beta_{1} \lambda^2-\beta_{2}\lambda^2+\beta_{1}+\beta_{2}\right) \sin ^2\tau \cos ^2\tau\right]}{2 \left(\lambda^2 C^2 \cos ^2\tau+C^2 \sin ^2\tau+1\right)}
	, \nonumber \\
	\dot{\tau}(t) &= \frac{\lambda}{1+\lambda^{2}}  +  \text{Wi} \left[ \frac{ \sin \tau\cos \tau\left(2 \beta_{1} \lambda^{2} C^2 \cos ^2\tau+\beta_{1}+\beta_{2}\right)}{2 C^2 \left(\lambda^{2} \cos ^2\tau+\sin ^2\tau\right)+2} -
	\frac{1}{4}  (\beta_{1}+\beta_{2}) \sin 2 \tau \right] .
\end{align}
We now compare our result with \citet{brunn1977slow} who, for a rigid tri-dumbbell, showed that the particle slowly drifts towards the log-rolling orbit.
%For the tri-dumbbell, his equations can be written as [refs]
%\begin{align}\label{brunn}
%		\dot{C}(t) =  -\frac{C H_2}{4}-\frac{C^3 H_1 \left(\lambda ^2+1\right) \sin ^2\tau \cos ^2\tau}{2 \left(C^2 \left(\lambda ^2 \sin ^2\tau+\cos ^2\tau\right)+1\right)}, \quad
%\dot{\tau}(t) = \frac{\lambda }{\lambda ^2+1}
%+ \frac{C^2 H_1 \sin \tau \cos \tau \left(\lambda ^2 \sin ^2\tau-\cos ^2\tau\right)}{2 \left(C^2 \left(\lambda ^2 \sin ^2\tau+\cos ^2\tau\right)+1\right)},
%\end{align}
%where $ H_{1} =- \text{Wi} (3 \delta +1) \left(\frac{\lambda^2-1}{\lambda^2+1}\right)^2  $ and $ H_{2} = - \text{Wi} (3 \delta +1)  \left( \frac{\lambda^2-1}{\left(\lambda^2+1\right)^2} \right) $. 
We show the match between orbit drift from Eq. (\ref{Ctau}) and Eq. (4.17) from Brunn in Fig. \ref{fig: Brunn}. 
Note that over one period the particle's orbit declines in a wiggly manner. 
Interestingly, these wiggles have a physical significance and can be characterized as orbit reduction with two distinct plateaus. 
Fig. \ref{fig: Brunn}b shows that the long  plateau (around $ t/T = 1/4 $) occurs first and is the state where particle is in the vicinity of flow-vorticity plane (where it slows down and spends most of its time).
Once the particle exits the alignment state, it flips quickly to the other side ($ -x $ axis). This flip generates the short plateau around $ t/T = 1/2 $. 


\begin{figure}[h]
	\centering
	\includegraphics[width=0.67\textwidth]{Brunn.pdf}
	\caption{Comparison with \citet{brunn1977slow} for the variation of the orbit constant $ C/C_{0} $, where $ C_{0} $ is the orbit constant for $ t=0 $. (a) Variation over 10 periods. (b) Variation over one period.
		Other parameters: $\lambda=5$, $ T= 2\pi \frac{\lambda^{2}+1}{\lambda} $, $\delta=-0.6$, $ \text{Wi}=0.1 $.}
	\label{fig: Brunn}
\end{figure}


.

\section{Details of Kinetic model}
Here we provide full expressions and further details involved in the computational steps of kinetic model.	
The spectral decomposition of $ \psi $ is carried out as
\begin{equation}
	\psi(\theta,\phi) =  \sum_{n=0}^{\infty} \sum_{m=-n}^{n} C_{n,m}(\IB{p}) Y_{n}^{m}(\IB{p}).
\end{equation}
Substituting this in the Smoluchowski equation of main text and using the properties of spherical harmonics yields
\begin{align}\label{Smol_Numerical_supp}
	-\frac{1}{4\pi} + \sum_{n=0}^{\infty} \sum_{m=-n}^{n} C_{n,m} \left[ 1+\tau D_r n(n+1) \right] Y_{n}^{m} +
	\text{Pe}_{f}  \sum_{n=0}^{\infty} \sum_{m=-n}^{n} C_{n,m}  \mathcal{H}(Y_{n}^{m})  = 0  .
\end{align}
Here $ \mathcal{H}(Y_{n}^{m})  $ is a term obtained by simplifying $ \nabla_{p} \cdot \left(  \dot{\IB{p}} \psi  \right)  $ and is represented in terms of angular momentum operators  for computational convenience \cite{chen1996rheology}. 
%This helps us in using orthogonal properties of Clebsch-Gordon coefficients.
\begin{align}\label{Hfunc}
	\mathcal{H}(Y_{n}^{m}) & =	 \left[ -\frac{\text{i} \, \mathcal{L}_{z}}{2} + \frac{ \text{i}\Lambda}{2}  \left(3\text{i} {Y_{n}^{m}}  \sin ^2(\theta ) \sin (2 \phi )+\mathcal{L}_{y} \sin (2 \theta ) \sin (\phi )+\mathcal{L}_{z} \cos (2 \theta ) \sin ^2(\phi )+\mathcal{L}_{z} \cos ^2(\phi )\right)  \right] + 
	\nonumber \\
	& \!\!\!\!\!\!\!\!\!\!\!\!\!\!   \text{De} \lbrace \frac{\text{i} \, \text{Pe}_{a} }{16\pi} \left[  \alpha_{1} \left( 3\text{i}Y_{n}^{m}  \sin ^2(\theta ) \sin (2 \phi )+\mathcal{L}_{y} \sin (2 \theta ) \sin (\phi )+\mathcal{L}_{z} \cos (2 \theta ) \sin ^2(\phi )+\mathcal{L}_{z} \cos ^2(\phi ) \right)  -\alpha_2 \mathcal{L}_{z} \right] + 
	\nonumber \\
	& \!\!\!\!\!\!\!  \frac{\text{i} \, \text{Pe}_{f}}{16}  \Big[
	\beta_{1} \sin ^2(\theta ) \Big(-i Y_{n}^{m}  (3 + 5 \cos (2 \theta ))-10\text{i}Y_{n}^{m}  \sin ^2(\theta ) \cos (4 \phi )+2 \mathcal{L}_{y} \sin (2 \theta ) \cos (\phi )-2 \mathcal{L}_{y} \sin (2 \theta ) \cos (3 \phi )+
	\nonumber \\
	&   \quad
	2 \mathcal{L}_{z} \sin ^2(\theta ) \sin (4 \phi )  +2 \mathcal{L}_{z} \cos (2 \theta ) \sin (2 \phi )+2 \mathcal{L}_{z} \sin (2 \phi )\Big)
	+ \beta_{2} \sin ^2(\theta ) \left(8\text{i}Y_{n}^{m}  \cos (2 \phi ) -4 \mathcal{L}_{z} \sin (2 \phi ) \right)
	\Big]
\rbrace .
\end{align}
Here, the angular momentum operators ($ \mathcal{L} $) are shorthand notation for the following gradients on harmonic $ Y_{n}^{m} $:
\begin{align}
	\mathcal{L}_{y} =  L_{y}|Y_{n}^{m}\rangle = -\text{i} \cos \phi \frac{\partial Y_{n}^{m}}{\partial \theta} + \text{i} \cot \theta \sin \phi \frac{\partial Y_{n}^{m}}{\partial \phi} \quad \mbox{and\ } \quad \mathcal{L}_{z}=  L_{z}|Y_{n}^{m}\rangle  = - \text{i} \frac{\partial Y_{n}^{m}}{\partial \phi}.
\end{align}
The trigonometric functions in the Eq. ({\ref{Hfunc}}) are converted into spherical harmonics. For example:
\begin{equation}
	\sin \theta \cos \theta \sin \phi = \text{i} \sqrt{\frac{2\pi}{15}} (Y_{2}^{-1}+Y_{2}^{1}), \quad \cos^{2} \theta = \frac{1}{3} \left( 2 \sqrt{4\pi/5} Y_{2}^{0} + \sqrt{4\pi} Y_{0}^{0} \right).
\end{equation}
Using such transformations, Eq. (\ref{Hfunc}) yields
{\begin{align}\label{Hfunc_harmonic}
		\mathcal{H}(Y_{n}^{m}) & =  [ -i \sqrt{\pi } \mathcal{L}_{z} Y^{0}_{0} + \Lambda ( -\sqrt{\frac{2 \pi }{15}} \mathcal{L}_{y} Y_{2}^{-1}-\sqrt{\frac{2 \pi }{15}} \mathcal{L}_{y} Y_{2}^{1}+i \sqrt{\frac{2 \pi }{15}} Y_{2}^{-2} (\mathcal{L}_{z}-3 Y_{n}^{m} )+i \sqrt{\frac{2 \pi }{15}} Y_{2}^{2} (\mathcal{L}_{z}+3 Y_{n}^{m} )
		\nonumber \\
		& \qquad \qquad \qquad \qquad \quad +\frac{1}{3} i \sqrt{\pi } \mathcal{L}_{z} Y_{0}^{0}+\frac{2}{3} i \sqrt{\frac{\pi }{5}} \mathcal{L}_{z} Y_{2}^{0} ) ] \; + 
		\nonumber \\
		&   \text{De}  \lbrace  \text{Pe}_{a}  [  -\frac{\text{$\alpha $1} \mathcal{L}_{y} Y_{2}^{-1}}{4 \sqrt{30 \pi }}-\frac{\text{$\alpha $1} \mathcal{L}_{y} Y_{2}^{1}}{4 \sqrt{30 \pi }}+\frac{i \mathcal{L}_{z} Y_{0}^{0} (\text{$\alpha $1}-3 \text{$\alpha $2})}{24 \sqrt{\pi }} +\frac{i \text{$\alpha $1} \mathcal{L}_{z} Y_{2}^{-2}}{4 \sqrt{30 \pi }}+\frac{i \text{$\alpha $1} \mathcal{L}_{z} Y_{2}^{0}}{12 \sqrt{5 \pi }}  +\frac{i \text{$\alpha $1} \mathcal{L}_{z} Y_{2}^{2}}{4 \sqrt{30 \pi }} + 
		\nonumber \\
		&  \qquad\qquad -\frac{1}{4} i \sqrt{\frac{3}{10 \pi }} \text{$\alpha $1} Y_{n}^{m}  Y_{2}^{-2}+ \frac{1}{4} i \sqrt{\frac{3}{10 \pi }} \text{$\alpha $1} Y_{n}^{m}  Y_{2}^{2}
		] \; + 
		\nonumber \\
		&  \qquad \;  \text{Pe}_{f} [ \frac{1}{7} i \sqrt{\frac{2 \pi }{15}} \beta_{1} \mathcal{L}_{y} Y_{2}^{-1}-\frac{1}{7} i \sqrt{\frac{2 \pi }{15}} \beta_{1} \mathcal{L}_{y} Y_{2}^{1}-\frac{1}{3} i \sqrt{\frac{\pi }{35}} \beta_{1} \mathcal{L}_{y} Y_{4}^{-3}-\frac{1}{21} i \sqrt{\frac{\pi }{5}} \beta_{1} \mathcal{L}_{y} Y_{4}^{-1}+ \frac{1}{21} i \sqrt{\frac{\pi }{5}} \beta_{1} \mathcal{L}_{y} Y_{4}^{1}+
		\nonumber \\
		& \qquad \qquad \;    \frac{1}{3} i \sqrt{\frac{\pi }{35}} \beta_{1} \mathcal{L}_{y} Y_{4}^{3}-\frac{1}{7} \sqrt{\frac{\pi }{30}} \mathcal{L}_{z} Y_{2}^{-2} (\beta_{1}-7 \beta_{2})+\frac{1}{7} \sqrt{\frac{\pi }{30}} \mathcal{L}_{z} Y_{2}^{2} (\beta_{1}-7 \beta_{2})  -\frac{1}{3} \sqrt{\frac{2 \pi }{35}} \beta_{1} Y_{4}^{-4} (\mathcal{L}_{z}-5 Y_{n}^{m} )+
		\nonumber \\
		&  \qquad\quad \;
		\frac{1}{3} \sqrt{\frac{2 \pi }{35}} \beta_{1} Y_{4}^{4} (\mathcal{L}_{z}+5 Y_{n}^{m} )-\frac{1}{21} \sqrt{\frac{2 \pi }{5}} \beta_{1} \mathcal{L}_{z} Y_{4}^{-2}+\frac{1}{21} \sqrt{\frac{2 \pi }{5}} \beta_{1} \mathcal{L}_{z} Y_{4}^{2}+\frac{2}{7} \sqrt{\frac{\pi }{5}} \beta_{1} Y_{n}^{m}  Y_{2}^{0} -\frac{2}{21} \sqrt{\pi } \beta_{1} Y_{n}^{m}  Y_{4}^{0} +
		\nonumber \\
		&  \qquad \quad\;
		-\sqrt{\frac{2 \pi }{15}} \beta_{2} Y_{n}^{m}  Y_{2}^{-2}-\sqrt{\frac{2 \pi }{15}} \beta_{2} Y_{n}^{m}  Y_{2}^{2}
		]
		\rbrace
\end{align}}
The angular momentum operators are further simplified using the following relations \cite{arfken1999mathematical}
\begin{align}
	\mathcal{L}_{y} =  L_{y}|Y_{n}^{m}\rangle = \frac{1}{2} \sqrt{n(n+1)-m(m+1)} Y_{n}^{m+1} - \frac{1}{2} \sqrt{n(n+1)-m(m-1)} Y_{n}^{m-1}
	\quad \mbox{and\ } \quad \mathcal{L}_{z}=  L_{z}|Y_{n}^{m}\rangle  =  m Y_{n}^{m}.
\end{align}
We substitute above relations in Eq. (\ref{Hfunc_harmonic}) to simplify $ \mathcal{H}(Y_{n}^{m}) $ and use that in Eq. (\ref{Smol_Numerical_supp}).
Next, we take an inner product with $ \overline{Y_{i}^{j}} $ on both sides of Eq. (\ref{Smol_Numerical_supp}) and use the orthogonality property to obtain
\begin{align}
	-\frac{1}{4\pi} \int_{S} \overline{Y_{i}^{j}}  \text{d}\IB{p}
	+  C_{i,j} \left[ 1+\tau D_r i(i+1) \right] +
	\text{Pe}_{f}  \sum_{n=0}^{\infty} \sum_{m=-n}^{n} C_{n,m} \int_{S} \overline{Y_{i}^{j}}  \mathcal{H}(Y_{n}^{m})  \text{d}\IB{p}   =0.
\end{align} 
The first term is only non-zero for the zeroth mode ($ i=0 $), which yields $ -1/\sqrt{4\pi} $.
Since the $ C_{0,0}=1/\sqrt{4\pi} $ is already known from normalization, the first term cancels the $ i=0 $ mode of second term.
Using these identities, Eq. (14) can be recasted as the following linear system of equations in terms of $ C_{i,j} $:
\begin{align}
	C_{i,j} \left[ 1+\tau D_r i(i+1) \right] + \; \; 
	\text{Pe}_{f}  \sum_{n=1}^{100} \sum_{m=-n}^{n} C_{n,m} \int_{S} \overline{Y_{i}^{j}}  \mathcal{H}(Y_{n}^{m})  \text{d}\IB{p}    =      -   \text{Pe}_{f}  C_{0,0} \int_{S} \overline{Y_{i}^{j}}  \mathcal{H}(Y_{0}^{0}) \text{d}\IB{p},
	\quad   \mbox{ for\ }  i \geq 1,
\end{align}
where the right hand side is known.
The two integral terms in the above set of equations contain the inner product of three spherical harmonics. These are represented in terms of Clebsch-Gordon coefficients that exploit the orthogonal relations between harmonics for computational ease \cite{arfken1999mathematical}:
%The idea is to decompose the highest harmonic in terms of  combination of other two harmonics such that orthogonality conditions can be exploited.
\begin{equation}
	\int_{0}^{\pi} \int_{0}^{2\pi}  Y_{N}^{M} Y_{a}^{b} Y_{n}^{m} \, \sin \theta \, \text{d}\phi \text{d}\theta = \sqrt{\frac{(2 a+1) (2 n+1)}{4 \pi  (2 \text{N}+1)}} 
	\begin{bmatrix}
		a & n & N \\
		0 & 0 & 0
	\end{bmatrix}
	\begin{bmatrix}
		a & n & N \\
		b & m & M
	\end{bmatrix}
	.
\end{equation}
The bracket terms here are the Clebsch-Gordan coefficients inbuilt in Mathematica 13.



skip


Below we show the match between our results (obtained from the above kinetic model) in the limit $ De=0 $ and those obtained by \citet[Fig. 5a, $ \tilde{\tau}=1 $]{saintillan2010dilute} .







\begin{figure}[h]
	\centering
	\includegraphics[width=0.3\textwidth]{saintillan-2.pdf}
	\caption{Comparing the shear viscosity as a function of $ \text{Pe}_f $ from our analysis in the Newtonian fluid limit with that of \citet{saintillan2010dilute}. for $ \frac{|\text{Pe}_{a}|}{8\pi A}= 1 $, $\Lambda= 1 $, $\tau=1$, $ D_{r}=0.1 $. Lines (blue is puller, red is pusher) indicate our viscosity ratio and dots correspond to \citet{saintillan2010dilute}, whose viscosity is defined here with respect to the particle volume fraction $ \varphi= n a^{3} $. \citet{saintillan2010dilute} used $ \varphi = n l^{3} A $.}
	\label{fig: viscosity_supp}
\end{figure}




%\bibliography{Akash}

%merlin.mbs apsrev4-1.bst 2010-07-25 4.21a (PWD, AO, DPC) hacked
%Control: key (0)
%Control: author (8) initials jnrlst
%Control: editor formatted (1) identically to author
%Control: production of article title (-1) disabled
%Control: page (0) single
%Control: year (1) truncated
%Control: production of eprint (0) enabled
\begin{thebibliography}{75}%
	\makeatletter
	\providecommand \@ifxundefined [1]{%
		\@ifx{#1\undefined}
	}%
	\providecommand \@ifnum [1]{%
		\ifnum #1\expandafter \@firstoftwo
		\else \expandafter \@secondoftwo
		\fi
	}%
	\providecommand \@ifx [1]{%
		\ifx #1\expandafter \@firstoftwo
		\else \expandafter \@secondoftwo
		\fi
	}%
	\providecommand \natexlab [1]{#1}%
	\providecommand \enquote  [1]{``#1''}%
	\providecommand \bibnamefont  [1]{#1}%
	\providecommand \bibfnamefont [1]{#1}%
	\providecommand \citenamefont [1]{#1}%
	\providecommand \href@noop [0]{\@secondoftwo}%
	\providecommand \href [0]{\begingroup \@sanitize@url \@href}%
	\providecommand \@href[1]{\@@startlink{#1}\@@href}%
	\providecommand \@@href[1]{\endgroup#1\@@endlink}%
	\providecommand \@sanitize@url [0]{\catcode `\\12\catcode `\$12\catcode
		`\&12\catcode `\#12\catcode `\^12\catcode `\_12\catcode `\%12\relax}%
	\providecommand \@@startlink[1]{}%
	\providecommand \@@endlink[0]{}%
	\providecommand \url  [0]{\begingroup\@sanitize@url \@url }%
	\providecommand \@url [1]{\endgroup\@href {#1}{\urlprefix }}%
	\providecommand \urlprefix  [0]{URL }%
	\providecommand \Eprint [0]{\href }%
	\providecommand \doibase [0]{http://dx.doi.org/}%
	\providecommand \selectlanguage [0]{\@gobble}%
	\providecommand \bibinfo  [0]{\@secondoftwo}%
	\providecommand \bibfield  [0]{\@secondoftwo}%
	\providecommand \translation [1]{[#1]}%
	\providecommand \BibitemOpen [0]{}%
	\providecommand \bibitemStop [0]{}%
	\providecommand \bibitemNoStop [0]{.\EOS\space}%
	\providecommand \EOS [0]{\spacefactor3000\relax}%
	\providecommand \BibitemShut  [1]{\csname bibitem#1\endcsname}%
	\let\auto@bib@innerbib\@empty
	%</preamble>
	\bibitem [{\citenamefont {Marchetti}\ \emph {et~al.}(2013)\citenamefont
		{Marchetti}, \citenamefont {Joanny}, \citenamefont {Ramaswamy}, \citenamefont
		{Liverpool}, \citenamefont {Prost}, \citenamefont {Rao},\ and\ \citenamefont
		{Simha}}]{marchetti2013Review}%
	\BibitemOpen
	\bibfield  {author} {\bibinfo {author} {\bibfnamefont {M.~C.}\ \bibnamefont
			{Marchetti}}, \bibinfo {author} {\bibfnamefont {J.~F.}\ \bibnamefont
			{Joanny}}, \bibinfo {author} {\bibfnamefont {S.}~\bibnamefont {Ramaswamy}},
		\bibinfo {author} {\bibfnamefont {T.~B.}\ \bibnamefont {Liverpool}}, \bibinfo
		{author} {\bibfnamefont {J.}~\bibnamefont {Prost}}, \bibinfo {author}
		{\bibfnamefont {M.}~\bibnamefont {Rao}}, \ and\ \bibinfo {author}
		{\bibfnamefont {R.~A.}\ \bibnamefont {Simha}},\ }\href@noop {} {\bibfield
		{journal} {\bibinfo  {journal} {Rev. Mod. Phys.}\ }\textbf {\bibinfo {volume}
			{85}},\ \bibinfo {pages} {1143} (\bibinfo {year} {2013})}\BibitemShut
	{NoStop}%
	\bibitem [{\citenamefont {Z{\"o}ttl}\ and\ \citenamefont
		{Stark}(2016)}]{zottl2016emergent}%
	\BibitemOpen
	\bibfield  {author} {\bibinfo {author} {\bibfnamefont {A.}~\bibnamefont
			{Z{\"o}ttl}}\ and\ \bibinfo {author} {\bibfnamefont {H.}~\bibnamefont
			{Stark}},\ }\href@noop {} {\bibfield  {journal} {\bibinfo  {journal} {Journal
				of Physics: Condensed Matter}\ }\textbf {\bibinfo {volume} {28}},\ \bibinfo
		{pages} {253001} (\bibinfo {year} {2016})}\BibitemShut {NoStop}%
	\bibitem [{\citenamefont {Lauga}(2020)}]{lauga2020fluid}%
	\BibitemOpen
	\bibfield  {author} {\bibinfo {author} {\bibfnamefont {E.}~\bibnamefont
			{Lauga}},\ }\href@noop {} {\emph {\bibinfo {title} {The fluid dynamics of
				cell motility}}},\ Vol.~\bibinfo {volume} {62}\ (\bibinfo  {publisher}
	{Cambridge University Press},\ \bibinfo {year} {2020})\BibitemShut {NoStop}%
	\bibitem [{\citenamefont {Berg}\ and\ \citenamefont
		{Brown}(1972)}]{berg1972chemotaxis}%
	\BibitemOpen
	\bibfield  {author} {\bibinfo {author} {\bibfnamefont {H.~C.}\ \bibnamefont
			{Berg}}\ and\ \bibinfo {author} {\bibfnamefont {D.~A.}\ \bibnamefont
			{Brown}},\ }\href@noop {} {\bibfield  {journal} {\bibinfo  {journal}
			{nature}\ }\textbf {\bibinfo {volume} {239}},\ \bibinfo {pages} {500}
		(\bibinfo {year} {1972})}\BibitemShut {NoStop}%
	\bibitem [{\citenamefont {Demir}\ and\ \citenamefont
		{Salman}(2012)}]{demir2012bacterial}%
	\BibitemOpen
	\bibfield  {author} {\bibinfo {author} {\bibfnamefont {M.}~\bibnamefont
			{Demir}}\ and\ \bibinfo {author} {\bibfnamefont {H.}~\bibnamefont {Salman}},\
	}\href@noop {} {\bibfield  {journal} {\bibinfo  {journal} {Biophysical
				journal}\ }\textbf {\bibinfo {volume} {103}},\ \bibinfo {pages} {1683}
		(\bibinfo {year} {2012})}\BibitemShut {NoStop}%
	\bibitem [{\citenamefont {Stark}(2018)}]{stark2018artificial}%
	\BibitemOpen
	\bibfield  {author} {\bibinfo {author} {\bibfnamefont {H.}~\bibnamefont
			{Stark}},\ }\href@noop {} {\bibfield  {journal} {\bibinfo  {journal}
			{Accounts of Chemical Research}\ }\textbf {\bibinfo {volume} {51}},\ \bibinfo
		{pages} {2681} (\bibinfo {year} {2018})}\BibitemShut {NoStop}%
	\bibitem [{\citenamefont {Mathijssen}\ \emph {et~al.}(2019)\citenamefont
		{Mathijssen}, \citenamefont {Figueroa-Morales}, \citenamefont {Junot},
		\citenamefont {Cl{\'e}ment}, \citenamefont {Lindner},\ and\ \citenamefont
		{Z{\"o}ttl}}]{mathijssen19}%
	\BibitemOpen
	\bibfield  {author} {\bibinfo {author} {\bibfnamefont {A.~J.}\ \bibnamefont
			{Mathijssen}}, \bibinfo {author} {\bibfnamefont {N.}~\bibnamefont
			{Figueroa-Morales}}, \bibinfo {author} {\bibfnamefont {G.}~\bibnamefont
			{Junot}}, \bibinfo {author} {\bibfnamefont {{\'E}.}~\bibnamefont
			{Cl{\'e}ment}}, \bibinfo {author} {\bibfnamefont {A.}~\bibnamefont
			{Lindner}}, \ and\ \bibinfo {author} {\bibfnamefont {A.}~\bibnamefont
			{Z{\"o}ttl}},\ }\href@noop {} {\bibfield  {journal} {\bibinfo  {journal}
			{Nat. Comm.}\ }\textbf {\bibinfo {volume} {10}},\ \bibinfo {pages} {1}
		(\bibinfo {year} {2019})}\BibitemShut {NoStop}%
	\bibitem [{\citenamefont {Jing}\ \emph {et~al.}(2020)\citenamefont {Jing},
		\citenamefont {Z{\"o}ttl}, \citenamefont {Cl{\'e}ment},\ and\ \citenamefont
		{Lindner}}]{jing2020chirality}%
	\BibitemOpen
	\bibfield  {author} {\bibinfo {author} {\bibfnamefont {G.}~\bibnamefont
			{Jing}}, \bibinfo {author} {\bibfnamefont {A.}~\bibnamefont {Z{\"o}ttl}},
		\bibinfo {author} {\bibfnamefont {{\'E}.}~\bibnamefont {Cl{\'e}ment}}, \ and\
		\bibinfo {author} {\bibfnamefont {A.}~\bibnamefont {Lindner}},\ }\href@noop
	{} {\bibfield  {journal} {\bibinfo  {journal} {Science advances}\ }\textbf
		{\bibinfo {volume} {6}},\ \bibinfo {pages} {eabb2012} (\bibinfo {year}
		{2020})}\BibitemShut {NoStop}%
	\bibitem [{\citenamefont {Doan}\ \emph {et~al.}(2020)\citenamefont {Doan},
		\citenamefont {Saingam}, \citenamefont {Yan},\ and\ \citenamefont
		{Shin}}]{doan2020trace}%
	\BibitemOpen
	\bibfield  {author} {\bibinfo {author} {\bibfnamefont {V.~S.}\ \bibnamefont
			{Doan}}, \bibinfo {author} {\bibfnamefont {P.}~\bibnamefont {Saingam}},
		\bibinfo {author} {\bibfnamefont {T.}~\bibnamefont {Yan}}, \ and\ \bibinfo
		{author} {\bibfnamefont {S.}~\bibnamefont {Shin}},\ }\href@noop {} {\bibfield
		{journal} {\bibinfo  {journal} {ACS nano}\ }\textbf {\bibinfo {volume}
			{14}},\ \bibinfo {pages} {14219} (\bibinfo {year} {2020})}\BibitemShut
	{NoStop}%
	\bibitem [{\citenamefont {Ramamonjy}\ \emph {et~al.}(2022)\citenamefont
		{Ramamonjy}, \citenamefont {Dervaux},\ and\ \citenamefont
		{Brunet}}]{ramamonjy2022nonlinear}%
	\BibitemOpen
	\bibfield  {author} {\bibinfo {author} {\bibfnamefont {A.}~\bibnamefont
			{Ramamonjy}}, \bibinfo {author} {\bibfnamefont {J.}~\bibnamefont {Dervaux}},
		\ and\ \bibinfo {author} {\bibfnamefont {P.}~\bibnamefont {Brunet}},\
	}\href@noop {} {\bibfield  {journal} {\bibinfo  {journal} {Physical Review
				Letters}\ }\textbf {\bibinfo {volume} {128}},\ \bibinfo {pages} {258101}
		(\bibinfo {year} {2022})}\BibitemShut {NoStop}%
	\bibitem [{\citenamefont {Suarez}\ and\ \citenamefont
		{Pacey}(2006)}]{suarez2006sperm}%
	\BibitemOpen
	\bibfield  {author} {\bibinfo {author} {\bibfnamefont {S.~S.}\ \bibnamefont
			{Suarez}}\ and\ \bibinfo {author} {\bibfnamefont {A.}~\bibnamefont {Pacey}},\
	}\href@noop {} {\bibfield  {journal} {\bibinfo  {journal} {Hum. Reprod.
				Update}\ }\textbf {\bibinfo {volume} {12}},\ \bibinfo {pages} {23} (\bibinfo
		{year} {2006})}\BibitemShut {NoStop}%
	\bibitem [{\citenamefont {Z{\"o}ttl}\ and\ \citenamefont
		{Stark}(2012)}]{zottl2012nonlinear}%
	\BibitemOpen
	\bibfield  {author} {\bibinfo {author} {\bibfnamefont {A.}~\bibnamefont
			{Z{\"o}ttl}}\ and\ \bibinfo {author} {\bibfnamefont {H.}~\bibnamefont
			{Stark}},\ }\href@noop {} {\bibfield  {journal} {\bibinfo  {journal} {Phys.
				Rev. Lett.}\ }\textbf {\bibinfo {volume} {108}},\ \bibinfo {pages} {218104}
		(\bibinfo {year} {2012})}\BibitemShut {NoStop}%
	\bibitem [{\citenamefont {Sznitman}\ and\ \citenamefont
		{Arratia}(2015)}]{sznitman2015locomotion}%
	\BibitemOpen
	\bibfield  {author} {\bibinfo {author} {\bibfnamefont {J.}~\bibnamefont
			{Sznitman}}\ and\ \bibinfo {author} {\bibfnamefont {P.~E.}\ \bibnamefont
			{Arratia}},\ }\href@noop {} {\emph {\bibinfo {title} {Complex Fluids in
				Biological Systems}}}\ (\bibinfo  {publisher} {Springer},\ \bibinfo {year}
	{2015})\ pp.\ \bibinfo {pages} {245--281}\BibitemShut {NoStop}%
	\bibitem [{\citenamefont {Shogren}\ \emph {et~al.}(1989)\citenamefont
		{Shogren}, \citenamefont {Gerken},\ and\ \citenamefont
		{Jentoft}}]{mucus_length_shogren1989}%
	\BibitemOpen
	\bibfield  {author} {\bibinfo {author} {\bibfnamefont {R.}~\bibnamefont
			{Shogren}}, \bibinfo {author} {\bibfnamefont {T.~A.}\ \bibnamefont {Gerken}},
		\ and\ \bibinfo {author} {\bibfnamefont {N.}~\bibnamefont {Jentoft}},\
	}\href@noop {} {\bibfield  {journal} {\bibinfo  {journal} {Biochemistry}\
		}\textbf {\bibinfo {volume} {28}},\ \bibinfo {pages} {5525} (\bibinfo {year}
		{1989})}\BibitemShut {NoStop}%
	\bibitem [{\citenamefont {Lai}\ \emph {et~al.}(2009)\citenamefont {Lai},
		\citenamefont {Wang}, \citenamefont {Wirtz},\ and\ \citenamefont
		{Hanes}}]{mucus_lai2009micro}%
	\BibitemOpen
	\bibfield  {author} {\bibinfo {author} {\bibfnamefont {S.~K.}\ \bibnamefont
			{Lai}}, \bibinfo {author} {\bibfnamefont {Y.-Y.}\ \bibnamefont {Wang}},
		\bibinfo {author} {\bibfnamefont {D.}~\bibnamefont {Wirtz}}, \ and\ \bibinfo
		{author} {\bibfnamefont {J.}~\bibnamefont {Hanes}},\ }\href@noop {}
	{\bibfield  {journal} {\bibinfo  {journal} {Advanced drug delivery reviews}\
		}\textbf {\bibinfo {volume} {61}},\ \bibinfo {pages} {86} (\bibinfo {year}
		{2009})}\BibitemShut {NoStop}%
	\bibitem [{\citenamefont {Mathijssen}\ \emph {et~al.}(2016)\citenamefont
		{Mathijssen}, \citenamefont {Shendruk}, \citenamefont {Yeomans},\ and\
		\citenamefont {Doostmohammadi}}]{mathijssen2016upstream}%
	\BibitemOpen
	\bibfield  {author} {\bibinfo {author} {\bibfnamefont {A.~J.}\ \bibnamefont
			{Mathijssen}}, \bibinfo {author} {\bibfnamefont {T.~N.}\ \bibnamefont
			{Shendruk}}, \bibinfo {author} {\bibfnamefont {J.~M.}\ \bibnamefont
			{Yeomans}}, \ and\ \bibinfo {author} {\bibfnamefont {A.}~\bibnamefont
			{Doostmohammadi}},\ }\href@noop {} {\bibfield  {journal} {\bibinfo  {journal}
			{Phys. Rev. Lett.}\ }\textbf {\bibinfo {volume} {116}},\ \bibinfo {pages}
		{028104} (\bibinfo {year} {2016})}\BibitemShut {NoStop}%
	\bibitem [{\citenamefont {Choudhary}\ and\ \citenamefont
		{Stark}(2022)}]{choudhary2022soft}%
	\BibitemOpen
	\bibfield  {author} {\bibinfo {author} {\bibfnamefont {A.}~\bibnamefont
			{Choudhary}}\ and\ \bibinfo {author} {\bibfnamefont {H.}~\bibnamefont
			{Stark}},\ }\href@noop {} {\bibfield  {journal} {\bibinfo  {journal} {Soft
				Matter}\ }\textbf {\bibinfo {volume} {18}},\ \bibinfo {pages} {48} (\bibinfo
		{year} {2022})}\BibitemShut {NoStop}%
	\bibitem [{\citenamefont {Sokolov}\ and\ \citenamefont
		{Aranson}(2009)}]{sokolov2009reduction}%
	\BibitemOpen
	\bibfield  {author} {\bibinfo {author} {\bibfnamefont {A.}~\bibnamefont
			{Sokolov}}\ and\ \bibinfo {author} {\bibfnamefont {I.~S.}\ \bibnamefont
			{Aranson}},\ }\href@noop {} {\bibfield  {journal} {\bibinfo  {journal}
			{Physical review letters}\ }\textbf {\bibinfo {volume} {103}},\ \bibinfo
		{pages} {148101} (\bibinfo {year} {2009})}\BibitemShut {NoStop}%
	\bibitem [{\citenamefont {Gachelin}\ \emph {et~al.}(2013)\citenamefont
		{Gachelin}, \citenamefont {Mino}, \citenamefont {Berthet}, \citenamefont
		{Lindner}, \citenamefont {Rousselet},\ and\ \citenamefont
		{Cl{\'e}ment}}]{gachelin2013non}%
	\BibitemOpen
	\bibfield  {author} {\bibinfo {author} {\bibfnamefont {J.}~\bibnamefont
			{Gachelin}}, \bibinfo {author} {\bibfnamefont {G.}~\bibnamefont {Mino}},
		\bibinfo {author} {\bibfnamefont {H.}~\bibnamefont {Berthet}}, \bibinfo
		{author} {\bibfnamefont {A.}~\bibnamefont {Lindner}}, \bibinfo {author}
		{\bibfnamefont {A.}~\bibnamefont {Rousselet}}, \ and\ \bibinfo {author}
		{\bibfnamefont {{\'E}.}~\bibnamefont {Cl{\'e}ment}},\ }\href@noop {}
	{\bibfield  {journal} {\bibinfo  {journal} {Physical review letters}\
		}\textbf {\bibinfo {volume} {110}},\ \bibinfo {pages} {268103} (\bibinfo
		{year} {2013})}\BibitemShut {NoStop}%
	\bibitem [{\citenamefont {McDonnell}\ \emph {et~al.}(2015)\citenamefont
		{McDonnell}, \citenamefont {Gopesh}, \citenamefont {Lo}, \citenamefont
		{O'Bryan}, \citenamefont {Yeo}, \citenamefont {Friend},\ and\ \citenamefont
		{Prabhakar}}]{mcdonnell2015motility}%
	\BibitemOpen
	\bibfield  {author} {\bibinfo {author} {\bibfnamefont {A.~G.}\ \bibnamefont
			{McDonnell}}, \bibinfo {author} {\bibfnamefont {T.~C.}\ \bibnamefont
			{Gopesh}}, \bibinfo {author} {\bibfnamefont {J.}~\bibnamefont {Lo}}, \bibinfo
		{author} {\bibfnamefont {M.}~\bibnamefont {O'Bryan}}, \bibinfo {author}
		{\bibfnamefont {L.~Y.}\ \bibnamefont {Yeo}}, \bibinfo {author} {\bibfnamefont
			{J.~R.}\ \bibnamefont {Friend}}, \ and\ \bibinfo {author} {\bibfnamefont
			{R.}~\bibnamefont {Prabhakar}},\ }\href@noop {} {\bibfield  {journal}
		{\bibinfo  {journal} {Soft Matter}\ }\textbf {\bibinfo {volume} {11}},\
		\bibinfo {pages} {4658} (\bibinfo {year} {2015})}\BibitemShut {NoStop}%
	\bibitem [{\citenamefont {L{\'o}pez}\ \emph {et~al.}(2015)\citenamefont
		{L{\'o}pez}, \citenamefont {Gachelin}, \citenamefont {Douarche},
		\citenamefont {Auradou},\ and\ \citenamefont
		{Cl{\'e}ment}}]{lopez2015turning}%
	\BibitemOpen
	\bibfield  {author} {\bibinfo {author} {\bibfnamefont {H.~M.}\ \bibnamefont
			{L{\'o}pez}}, \bibinfo {author} {\bibfnamefont {J.}~\bibnamefont {Gachelin}},
		\bibinfo {author} {\bibfnamefont {C.}~\bibnamefont {Douarche}}, \bibinfo
		{author} {\bibfnamefont {H.}~\bibnamefont {Auradou}}, \ and\ \bibinfo
		{author} {\bibfnamefont {E.}~\bibnamefont {Cl{\'e}ment}},\ }\href@noop {}
	{\bibfield  {journal} {\bibinfo  {journal} {Phys. Rev. Lett.}\ }\textbf
		{\bibinfo {volume} {115}},\ \bibinfo {pages} {028301} (\bibinfo {year}
		{2015})}\BibitemShut {NoStop}%
	\bibitem [{\citenamefont {Hatwalne}\ \emph {et~al.}(2004)\citenamefont
		{Hatwalne}, \citenamefont {Ramaswamy}, \citenamefont {Rao},\ and\
		\citenamefont {Simha}}]{hatwalne2004rheology}%
	\BibitemOpen
	\bibfield  {author} {\bibinfo {author} {\bibfnamefont {Y.}~\bibnamefont
			{Hatwalne}}, \bibinfo {author} {\bibfnamefont {S.}~\bibnamefont {Ramaswamy}},
		\bibinfo {author} {\bibfnamefont {M.}~\bibnamefont {Rao}}, \ and\ \bibinfo
		{author} {\bibfnamefont {R.~A.}\ \bibnamefont {Simha}},\ }\href@noop {}
	{\bibfield  {journal} {\bibinfo  {journal} {Phys. Rev. Lett.}\ }\textbf
		{\bibinfo {volume} {92}},\ \bibinfo {pages} {118101} (\bibinfo {year}
		{2004})}\BibitemShut {NoStop}%
	\bibitem [{\citenamefont {Haines}\ \emph {et~al.}(2009)\citenamefont {Haines},
		\citenamefont {Sokolov}, \citenamefont {Aranson}, \citenamefont {Berlyand},\
		and\ \citenamefont {Karpeev}}]{haines2009three}%
	\BibitemOpen
	\bibfield  {author} {\bibinfo {author} {\bibfnamefont {B.~M.}\ \bibnamefont
			{Haines}}, \bibinfo {author} {\bibfnamefont {A.}~\bibnamefont {Sokolov}},
		\bibinfo {author} {\bibfnamefont {I.~S.}\ \bibnamefont {Aranson}}, \bibinfo
		{author} {\bibfnamefont {L.}~\bibnamefont {Berlyand}}, \ and\ \bibinfo
		{author} {\bibfnamefont {D.~A.}\ \bibnamefont {Karpeev}},\ }\href@noop {}
	{\bibfield  {journal} {\bibinfo  {journal} {Physical Review E}\ }\textbf
		{\bibinfo {volume} {80}},\ \bibinfo {pages} {041922} (\bibinfo {year}
		{2009})}\BibitemShut {NoStop}%
	\bibitem [{\citenamefont {Saintillan}(2010)}]{saintillan2010dilute}%
	\BibitemOpen
	\bibfield  {author} {\bibinfo {author} {\bibfnamefont {D.}~\bibnamefont
			{Saintillan}},\ }\href@noop {} {\bibfield  {journal} {\bibinfo  {journal}
			{Exp. Mech.}\ }\textbf {\bibinfo {volume} {50}},\ \bibinfo {pages} {1275}
		(\bibinfo {year} {2010})}\BibitemShut {NoStop}%
	\bibitem [{\citenamefont {Jeffery}(1922)}]{jeffery1922motion}%
	\BibitemOpen
	\bibfield  {author} {\bibinfo {author} {\bibfnamefont {G.~B.}\ \bibnamefont
			{Jeffery}},\ }\href@noop {} {\bibfield  {journal} {\bibinfo  {journal}
			{Proceedings of the Royal Society of London.}\ }\textbf {\bibinfo {volume}
			{102}},\ \bibinfo {pages} {161} (\bibinfo {year} {1922})}\BibitemShut
	{NoStop}%
	\bibitem [{\citenamefont {Rafa{\"\i}}\ \emph {et~al.}(2010)\citenamefont
		{Rafa{\"\i}}, \citenamefont {Jibuti},\ and\ \citenamefont
		{Peyla}}]{rafai2010effective}%
	\BibitemOpen
	\bibfield  {author} {\bibinfo {author} {\bibfnamefont {S.}~\bibnamefont
			{Rafa{\"\i}}}, \bibinfo {author} {\bibfnamefont {L.}~\bibnamefont {Jibuti}},
		\ and\ \bibinfo {author} {\bibfnamefont {P.}~\bibnamefont {Peyla}},\
	}\href@noop {} {\bibfield  {journal} {\bibinfo  {journal} {Phys. Rev. Lett.}\
		}\textbf {\bibinfo {volume} {104}},\ \bibinfo {pages} {098102} (\bibinfo
		{year} {2010})}\BibitemShut {NoStop}%
	\bibitem [{\citenamefont {Li}\ and\ \citenamefont
		{Ardekani}(2016)}]{li2016collective}%
	\BibitemOpen
	\bibfield  {author} {\bibinfo {author} {\bibfnamefont {G.}~\bibnamefont
			{Li}}\ and\ \bibinfo {author} {\bibfnamefont {A.~M.}\ \bibnamefont
			{Ardekani}},\ }\href@noop {} {\bibfield  {journal} {\bibinfo  {journal}
			{Phys. Rev. Lett.}\ }\textbf {\bibinfo {volume} {117}},\ \bibinfo {pages}
		{118001} (\bibinfo {year} {2016})}\BibitemShut {NoStop}%
	\bibitem [{\citenamefont {Li}\ \emph {et~al.}(2021)\citenamefont {Li},
		\citenamefont {Lauga},\ and\ \citenamefont {Ardekani}}]{li2021microswimming}%
	\BibitemOpen
	\bibfield  {author} {\bibinfo {author} {\bibfnamefont {G.}~\bibnamefont
			{Li}}, \bibinfo {author} {\bibfnamefont {E.}~\bibnamefont {Lauga}}, \ and\
		\bibinfo {author} {\bibfnamefont {A.~M.}\ \bibnamefont {Ardekani}},\
	}\href@noop {} {\bibfield  {journal} {\bibinfo  {journal} {Journal of
				Non-Newtonian Fluid Mechanics}\ }\textbf {\bibinfo {volume} {297}},\ \bibinfo
		{pages} {104655} (\bibinfo {year} {2021})}\BibitemShut {NoStop}%
	\bibitem [{\citenamefont {Spagnolie}\ and\ \citenamefont
		{Underhill}(2022)}]{spagnolie2022swimming}%
	\BibitemOpen
	\bibfield  {author} {\bibinfo {author} {\bibfnamefont {S.~E.}\ \bibnamefont
			{Spagnolie}}\ and\ \bibinfo {author} {\bibfnamefont {P.~T.}\ \bibnamefont
			{Underhill}},\ }\href@noop {} {\bibfield  {journal} {\bibinfo  {journal}
			{arXiv preprint arXiv:2208.03537}\ } (\bibinfo {year} {2022})}\BibitemShut
	{NoStop}%
	\bibitem [{\citenamefont {Patteson}\ \emph {et~al.}(2015)\citenamefont
		{Patteson}, \citenamefont {Gopinath}, \citenamefont {Goulian},\ and\
		\citenamefont {Arratia}}]{patteson2015running}%
	\BibitemOpen
	\bibfield  {author} {\bibinfo {author} {\bibfnamefont {A.}~\bibnamefont
			{Patteson}}, \bibinfo {author} {\bibfnamefont {A.}~\bibnamefont {Gopinath}},
		\bibinfo {author} {\bibfnamefont {M.}~\bibnamefont {Goulian}}, \ and\
		\bibinfo {author} {\bibfnamefont {P.}~\bibnamefont {Arratia}},\ }\href@noop
	{} {\bibfield  {journal} {\bibinfo  {journal} {Sci. rep.}\ }\textbf {\bibinfo
			{volume} {5}},\ \bibinfo {pages} {15761} (\bibinfo {year}
		{2015})}\BibitemShut {NoStop}%
	\bibitem [{\citenamefont {Z{\"o}ttl}\ and\ \citenamefont
		{Yeomans}(2019)}]{zottl2019enhanced}%
	\BibitemOpen
	\bibfield  {author} {\bibinfo {author} {\bibfnamefont {A.}~\bibnamefont
			{Z{\"o}ttl}}\ and\ \bibinfo {author} {\bibfnamefont {J.~M.}\ \bibnamefont
			{Yeomans}},\ }\href@noop {} {\bibfield  {journal} {\bibinfo  {journal}
			{Nature Physics}\ }\textbf {\bibinfo {volume} {15}},\ \bibinfo {pages} {554}
		(\bibinfo {year} {2019})}\BibitemShut {NoStop}%
	\bibitem [{\citenamefont {Kamdar}\ \emph {et~al.}(2022)\citenamefont {Kamdar},
		\citenamefont {Shin}, \citenamefont {Leishangthem}, \citenamefont {Francis},
		\citenamefont {Xu},\ and\ \citenamefont {Cheng}}]{kamdar2022colloidal}%
	\BibitemOpen
	\bibfield  {author} {\bibinfo {author} {\bibfnamefont {S.}~\bibnamefont
			{Kamdar}}, \bibinfo {author} {\bibfnamefont {S.}~\bibnamefont {Shin}},
		\bibinfo {author} {\bibfnamefont {P.}~\bibnamefont {Leishangthem}}, \bibinfo
		{author} {\bibfnamefont {L.~F.}\ \bibnamefont {Francis}}, \bibinfo {author}
		{\bibfnamefont {X.}~\bibnamefont {Xu}}, \ and\ \bibinfo {author}
		{\bibfnamefont {X.}~\bibnamefont {Cheng}},\ }\href@noop {} {\bibfield
		{journal} {\bibinfo  {journal} {Nature}\ }\textbf {\bibinfo {volume} {603}},\
		\bibinfo {pages} {819} (\bibinfo {year} {2022})}\BibitemShut {NoStop}%
	\bibitem [{\citenamefont {Liu}\ \emph {et~al.}(2021)\citenamefont {Liu},
		\citenamefont {Shankar}, \citenamefont {Marchetti},\ and\ \citenamefont
		{Wu}}]{liu2021viscoelastic}%
	\BibitemOpen
	\bibfield  {author} {\bibinfo {author} {\bibfnamefont {S.}~\bibnamefont
			{Liu}}, \bibinfo {author} {\bibfnamefont {S.}~\bibnamefont {Shankar}},
		\bibinfo {author} {\bibfnamefont {M.~C.}\ \bibnamefont {Marchetti}}, \ and\
		\bibinfo {author} {\bibfnamefont {Y.}~\bibnamefont {Wu}},\ }\href@noop {}
	{\bibfield  {journal} {\bibinfo  {journal} {Nature}\ }\textbf {\bibinfo
			{volume} {590}},\ \bibinfo {pages} {80} (\bibinfo {year} {2021})}\BibitemShut
	{NoStop}%
	\bibitem [{\citenamefont {Bird}\ \emph
		{et~al.}(1987{\natexlab{a}})\citenamefont {Bird}, \citenamefont {Armstrong},\
		and\ \citenamefont {Hassager}}]{bird1987dynamics}%
	\BibitemOpen
	\bibfield  {author} {\bibinfo {author} {\bibfnamefont {R.~B.}\ \bibnamefont
			{Bird}}, \bibinfo {author} {\bibfnamefont {R.~C.}\ \bibnamefont {Armstrong}},
		\ and\ \bibinfo {author} {\bibfnamefont {O.}~\bibnamefont {Hassager}},\
	}\href@noop {} {\emph {\bibinfo {title} {Dynamics of polymeric liquids. Vol.
				1: Fluid mechanics}}}\ (\bibinfo {year} {1987})\BibitemShut {NoStop}%
	\bibitem [{\citenamefont {James}(2009)}]{james2009boger}%
	\BibitemOpen
	\bibfield  {author} {\bibinfo {author} {\bibfnamefont {D.~F.}\ \bibnamefont
			{James}},\ }\href@noop {} {\bibfield  {journal} {\bibinfo  {journal} {Annual
				Review of Fluid Mechanics}\ }\textbf {\bibinfo {volume} {41}},\ \bibinfo
		{pages} {129} (\bibinfo {year} {2009})}\BibitemShut {NoStop}%
	\bibitem [{\citenamefont {Leal}(1975)}]{leal1975slow}%
	\BibitemOpen
	\bibfield  {author} {\bibinfo {author} {\bibfnamefont {L.}~\bibnamefont
			{Leal}},\ }\href@noop {} {\bibfield  {journal} {\bibinfo  {journal} {J.
				Fluid. Mech.}\ }\textbf {\bibinfo {volume} {69}},\ \bibinfo {pages} {305}
		(\bibinfo {year} {1975})}\BibitemShut {NoStop}%
	\bibitem [{\citenamefont {Brunn}(1977)}]{brunn1977slow}%
	\BibitemOpen
	\bibfield  {author} {\bibinfo {author} {\bibfnamefont {P.}~\bibnamefont
			{Brunn}},\ }\href@noop {} {\bibfield  {journal} {\bibinfo  {journal} {J.
				Fluid Mech.}\ }\textbf {\bibinfo {volume} {82}},\ \bibinfo {pages} {529}
		(\bibinfo {year} {1977})}\BibitemShut {NoStop}%
	\bibitem [{\citenamefont {Hinch}\ and\ \citenamefont
		{Leal}(1975)}]{hinch1975constitutive}%
	\BibitemOpen
	\bibfield  {author} {\bibinfo {author} {\bibfnamefont {E.}~\bibnamefont
			{Hinch}}\ and\ \bibinfo {author} {\bibfnamefont {L.}~\bibnamefont {Leal}},\
	}\href@noop {} {\bibfield  {journal} {\bibinfo  {journal} {J. Fluid. Mech.}\
		}\textbf {\bibinfo {volume} {71}},\ \bibinfo {pages} {481} (\bibinfo {year}
		{1975})}\BibitemShut {NoStop}%
	\bibitem [{\citenamefont {Chwang}\ and\ \citenamefont
		{Wu}(1975)}]{chwang1975hydromechanics}%
	\BibitemOpen
	\bibfield  {author} {\bibinfo {author} {\bibfnamefont {A.~T.}\ \bibnamefont
			{Chwang}}\ and\ \bibinfo {author} {\bibfnamefont {T.~Y.-T.}\ \bibnamefont
			{Wu}},\ }\href@noop {} {\bibfield  {journal} {\bibinfo  {journal} {Journal of
				Fluid mechanics}\ }\textbf {\bibinfo {volume} {67}},\ \bibinfo {pages} {787}
		(\bibinfo {year} {1975})}\BibitemShut {NoStop}%
	\bibitem [{\citenamefont {Spagnolie}\ and\ \citenamefont
		{Lauga}(2012)}]{spagnolie_lauga_2012}%
	\BibitemOpen
	\bibfield  {author} {\bibinfo {author} {\bibfnamefont {S.~E.}\ \bibnamefont
			{Spagnolie}}\ and\ \bibinfo {author} {\bibfnamefont {E.}~\bibnamefont
			{Lauga}},\ }\href@noop {} {\bibfield  {journal} {\bibinfo  {journal} {J.
				Fluid Mech.}\ }\textbf {\bibinfo {volume} {700}},\ \bibinfo {pages}
		{105–147} (\bibinfo {year} {2012})}\BibitemShut {NoStop}%
	\bibitem [{\citenamefont {Berke}\ \emph {et~al.}(2008)\citenamefont {Berke},
		\citenamefont {Turner}, \citenamefont {Berg},\ and\ \citenamefont
		{Lauga}}]{berke2008hydrodynamic}%
	\BibitemOpen
	\bibfield  {author} {\bibinfo {author} {\bibfnamefont {A.~P.}\ \bibnamefont
			{Berke}}, \bibinfo {author} {\bibfnamefont {L.}~\bibnamefont {Turner}},
		\bibinfo {author} {\bibfnamefont {H.~C.}\ \bibnamefont {Berg}}, \ and\
		\bibinfo {author} {\bibfnamefont {E.}~\bibnamefont {Lauga}},\ }\href@noop {}
	{\bibfield  {journal} {\bibinfo  {journal} {Phys. Rev. Lett.}\ }\textbf
		{\bibinfo {volume} {101}},\ \bibinfo {pages} {038102} (\bibinfo {year}
		{2008})}\BibitemShut {NoStop}%
	\bibitem [{\citenamefont {Drescher}\ \emph {et~al.}(2010)\citenamefont
		{Drescher}, \citenamefont {Goldstein}, \citenamefont {Michel}, \citenamefont
		{Polin},\ and\ \citenamefont {Tuval}}]{drescher2010direct}%
	\BibitemOpen
	\bibfield  {author} {\bibinfo {author} {\bibfnamefont {K.}~\bibnamefont
			{Drescher}}, \bibinfo {author} {\bibfnamefont {R.~E.}\ \bibnamefont
			{Goldstein}}, \bibinfo {author} {\bibfnamefont {N.}~\bibnamefont {Michel}},
		\bibinfo {author} {\bibfnamefont {M.}~\bibnamefont {Polin}}, \ and\ \bibinfo
		{author} {\bibfnamefont {I.}~\bibnamefont {Tuval}},\ }\href@noop {}
	{\bibfield  {journal} {\bibinfo  {journal} {Phys. Rev. Lett.}\ }\textbf
		{\bibinfo {volume} {105}},\ \bibinfo {pages} {168101} (\bibinfo {year}
		{2010})}\BibitemShut {NoStop}%
	\bibitem [{\citenamefont {Einarsson}\ \emph {et~al.}(2015)\citenamefont
		{Einarsson}, \citenamefont {Candelier}, \citenamefont {Lundell},
		\citenamefont {Angilella},\ and\ \citenamefont
		{Mehlig}}]{einarsson2015rotation}%
	\BibitemOpen
	\bibfield  {author} {\bibinfo {author} {\bibfnamefont {J.}~\bibnamefont
			{Einarsson}}, \bibinfo {author} {\bibfnamefont {F.}~\bibnamefont
			{Candelier}}, \bibinfo {author} {\bibfnamefont {F.}~\bibnamefont {Lundell}},
		\bibinfo {author} {\bibfnamefont {J.}~\bibnamefont {Angilella}}, \ and\
		\bibinfo {author} {\bibfnamefont {B.}~\bibnamefont {Mehlig}},\ }\href@noop {}
	{\bibfield  {journal} {\bibinfo  {journal} {Physics of Fluids}\ }\textbf
		{\bibinfo {volume} {27}},\ \bibinfo {pages} {063301} (\bibinfo {year}
		{2015})}\BibitemShut {NoStop}%
	\bibitem [{\citenamefont {Gauthier}\ \emph {et~al.}(1971)\citenamefont
		{Gauthier}, \citenamefont {Goldsmith},\ and\ \citenamefont
		{Mason}}]{gauthier1971_shear-flow}%
	\BibitemOpen
	\bibfield  {author} {\bibinfo {author} {\bibfnamefont {F.}~\bibnamefont
			{Gauthier}}, \bibinfo {author} {\bibfnamefont {H.}~\bibnamefont {Goldsmith}},
		\ and\ \bibinfo {author} {\bibfnamefont {S.}~\bibnamefont {Mason}},\
	}\href@noop {} {\bibfield  {journal} {\bibinfo  {journal} {Rheo.l Acta}\
		}\textbf {\bibinfo {volume} {10}},\ \bibinfo {pages} {344} (\bibinfo {year}
		{1971})}\BibitemShut {NoStop}%
	\bibitem [{\citenamefont {D’Avino}\ and\ \citenamefont
		{Maffettone}(2015)}]{davino2015particle}%
	\BibitemOpen
	\bibfield  {author} {\bibinfo {author} {\bibfnamefont {G.}~\bibnamefont
			{D’Avino}}\ and\ \bibinfo {author} {\bibfnamefont {P.~L.}\ \bibnamefont
			{Maffettone}},\ }\href@noop {} {\bibfield  {journal} {\bibinfo  {journal}
			{Journal of Non-Newtonian Fluid Mechanics}\ }\textbf {\bibinfo {volume}
			{215}},\ \bibinfo {pages} {80} (\bibinfo {year} {2015})}\BibitemShut
	{NoStop}%
	\bibitem [{\citenamefont {Berg}(1983)}]{berg1983random}%
	\BibitemOpen
	\bibfield  {author} {\bibinfo {author} {\bibfnamefont {H.}~\bibnamefont
			{Berg}},\ }\href@noop {} {\emph {\bibinfo {title} {Random Walks in Biology.
				Princeton Univ. Press, Princeton, NJ}}}\ (\bibinfo {year} {1983})\BibitemShut
	{NoStop}%
	\bibitem [{\citenamefont {Lauga}(2016)}]{lauga2016bacterial}%
	\BibitemOpen
	\bibfield  {author} {\bibinfo {author} {\bibfnamefont {E.}~\bibnamefont
			{Lauga}},\ }\href@noop {} {\bibfield  {journal} {\bibinfo  {journal} {Ann.
				Rev. Fluid Mech.}\ }\textbf {\bibinfo {volume} {48}},\ \bibinfo {pages} {105}
		(\bibinfo {year} {2016})}\BibitemShut {NoStop}%
	\bibitem [{\citenamefont {Adhyapak}\ and\ \citenamefont
		{Stark}(2016)}]{adhyapak2016dynamics}%
	\BibitemOpen
	\bibfield  {author} {\bibinfo {author} {\bibfnamefont {T.~C.}\ \bibnamefont
			{Adhyapak}}\ and\ \bibinfo {author} {\bibfnamefont {H.}~\bibnamefont
			{Stark}},\ }\href@noop {} {\bibfield  {journal} {\bibinfo  {journal} {Soft
				Matter}\ }\textbf {\bibinfo {volume} {12}},\ \bibinfo {pages} {5621}
		(\bibinfo {year} {2016})}\BibitemShut {NoStop}%
	\bibitem [{\citenamefont {Nambiar}\ \emph {et~al.}(2017)\citenamefont
		{Nambiar}, \citenamefont {Nott},\ and\ \citenamefont
		{Subramanian}}]{nambiar2017stress}%
	\BibitemOpen
	\bibfield  {author} {\bibinfo {author} {\bibfnamefont {S.}~\bibnamefont
			{Nambiar}}, \bibinfo {author} {\bibfnamefont {P.}~\bibnamefont {Nott}}, \
		and\ \bibinfo {author} {\bibfnamefont {G.}~\bibnamefont {Subramanian}},\
	}\href@noop {} {\bibfield  {journal} {\bibinfo  {journal} {Journal of Fluid
				Mechanics}\ }\textbf {\bibinfo {volume} {812}},\ \bibinfo {pages} {41}
		(\bibinfo {year} {2017})}\BibitemShut {NoStop}%
	\bibitem [{\citenamefont {Doi}\ \emph {et~al.}(1988)\citenamefont {Doi},
		\citenamefont {Edwards},\ and\ \citenamefont {Edwards}}]{doi1988theory}%
	\BibitemOpen
	\bibfield  {author} {\bibinfo {author} {\bibfnamefont {M.}~\bibnamefont
			{Doi}}, \bibinfo {author} {\bibfnamefont {S.~F.}\ \bibnamefont {Edwards}}, \
		and\ \bibinfo {author} {\bibfnamefont {S.~F.}\ \bibnamefont {Edwards}},\
	}\href@noop {} {\emph {\bibinfo {title} {The theory of polymer dynamics}}},\
	Vol.~\bibinfo {volume} {73}\ (\bibinfo  {publisher} {Oxford University
		Press},\ \bibinfo {year} {1988})\BibitemShut {NoStop}%
	\bibitem [{\citenamefont {Cohen}\ \emph {et~al.}(1987)\citenamefont {Cohen},
		\citenamefont {Chung},\ and\ \citenamefont {Stasiak}}]{cohen1987orientation}%
	\BibitemOpen
	\bibfield  {author} {\bibinfo {author} {\bibfnamefont {C.}~\bibnamefont
			{Cohen}}, \bibinfo {author} {\bibfnamefont {B.}~\bibnamefont {Chung}}, \ and\
		\bibinfo {author} {\bibfnamefont {W.}~\bibnamefont {Stasiak}},\ }\href@noop
	{} {\bibfield  {journal} {\bibinfo  {journal} {Rheo. Acta}\ }\textbf
		{\bibinfo {volume} {26}},\ \bibinfo {pages} {217} (\bibinfo {year}
		{1987})}\BibitemShut {NoStop}%
	\bibitem [{\citenamefont {Chattopadhyay}\ \emph {et~al.}(2006)\citenamefont
		{Chattopadhyay}, \citenamefont {Moldovan}, \citenamefont {Yeung},\ and\
		\citenamefont {Wu}}]{chattopadhyay2006swimming}%
	\BibitemOpen
	\bibfield  {author} {\bibinfo {author} {\bibfnamefont {S.}~\bibnamefont
			{Chattopadhyay}}, \bibinfo {author} {\bibfnamefont {R.}~\bibnamefont
			{Moldovan}}, \bibinfo {author} {\bibfnamefont {C.}~\bibnamefont {Yeung}}, \
		and\ \bibinfo {author} {\bibfnamefont {X.}~\bibnamefont {Wu}},\ }\href@noop
	{} {\bibfield  {journal} {\bibinfo  {journal} {Proc. Natl. Acad. Sci.}\
		}\textbf {\bibinfo {volume} {103}},\ \bibinfo {pages} {13712} (\bibinfo
		{year} {2006})}\BibitemShut {NoStop}%
	\bibitem [{\citenamefont {Hinch}\ and\ \citenamefont
		{Leal}(1976)}]{hinch1976constitutive}%
	\BibitemOpen
	\bibfield  {author} {\bibinfo {author} {\bibfnamefont {E.}~\bibnamefont
			{Hinch}}\ and\ \bibinfo {author} {\bibfnamefont {L.}~\bibnamefont {Leal}},\
	}\href@noop {} {\bibfield  {journal} {\bibinfo  {journal} {J. Fluid. Mech.}\
		}\textbf {\bibinfo {volume} {76}},\ \bibinfo {pages} {187} (\bibinfo {year}
		{1976})}\BibitemShut {NoStop}%
	\bibitem [{\citenamefont {Chen}\ and\ \citenamefont
		{Koch}(1996)}]{chen1996rheology}%
	\BibitemOpen
	\bibfield  {author} {\bibinfo {author} {\bibfnamefont {S.~B.}\ \bibnamefont
			{Chen}}\ and\ \bibinfo {author} {\bibfnamefont {D.~L.}\ \bibnamefont
			{Koch}},\ }\href@noop {} {\bibfield  {journal} {\bibinfo  {journal} {Physics
				of Fluids}\ }\textbf {\bibinfo {volume} {8}},\ \bibinfo {pages} {2792}
		(\bibinfo {year} {1996})}\BibitemShut {NoStop}%
	\bibitem [{\citenamefont {Hinch}\ and\ \citenamefont
		{Leal}(1972)}]{hinch1972effect}%
	\BibitemOpen
	\bibfield  {author} {\bibinfo {author} {\bibfnamefont {E.}~\bibnamefont
			{Hinch}}\ and\ \bibinfo {author} {\bibfnamefont {L.}~\bibnamefont {Leal}},\
	}\href@noop {} {\bibfield  {journal} {\bibinfo  {journal} {Journal of Fluid
				Mechanics}\ }\textbf {\bibinfo {volume} {52}},\ \bibinfo {pages} {683}
		(\bibinfo {year} {1972})}\BibitemShut {NoStop}%
	\bibitem [{\citenamefont {Wolfe}\ \emph {et~al.}(1988)\citenamefont {Wolfe},
		\citenamefont {Conley},\ and\ \citenamefont
		{Berg}}]{wolfe1988acetyladenylate}%
	\BibitemOpen
	\bibfield  {author} {\bibinfo {author} {\bibfnamefont {A.~J.}\ \bibnamefont
			{Wolfe}}, \bibinfo {author} {\bibfnamefont {M.~P.}\ \bibnamefont {Conley}}, \
		and\ \bibinfo {author} {\bibfnamefont {H.~C.}\ \bibnamefont {Berg}},\
	}\href@noop {} {\bibfield  {journal} {\bibinfo  {journal} {Proceedings of the
				National Academy of Sciences}\ }\textbf {\bibinfo {volume} {85}},\ \bibinfo
		{pages} {6711} (\bibinfo {year} {1988})}\BibitemShut {NoStop}%
	\bibitem [{\citenamefont {Chung}\ and\ \citenamefont
		{Cohen}(1987)}]{chung1987orientation}%
	\BibitemOpen
	\bibfield  {author} {\bibinfo {author} {\bibfnamefont {B.}~\bibnamefont
			{Chung}}\ and\ \bibinfo {author} {\bibfnamefont {C.}~\bibnamefont {Cohen}},\
	}\href@noop {} {\bibfield  {journal} {\bibinfo  {journal} {J. Nonnewton.
				Fluid Mech.}\ }\textbf {\bibinfo {volume} {25}},\ \bibinfo {pages} {289}
		(\bibinfo {year} {1987})}\BibitemShut {NoStop}%
	\bibitem [{\citenamefont {Iso}\ \emph {et~al.}(1996)\citenamefont {Iso},
		\citenamefont {Koch},\ and\ \citenamefont {Cohen}}]{iso1996orientation}%
	\BibitemOpen
	\bibfield  {author} {\bibinfo {author} {\bibfnamefont {Y.}~\bibnamefont
			{Iso}}, \bibinfo {author} {\bibfnamefont {D.~L.}\ \bibnamefont {Koch}}, \
		and\ \bibinfo {author} {\bibfnamefont {C.}~\bibnamefont {Cohen}},\
	}\href@noop {} {\bibfield  {journal} {\bibinfo  {journal} {J. Nonnewton.
				Fluid Mech.}\ }\textbf {\bibinfo {volume} {62}},\ \bibinfo {pages} {115}
		(\bibinfo {year} {1996})}\BibitemShut {NoStop}%
	\bibitem [{\citenamefont {Leal}\ and\ \citenamefont
		{Hinch}(1971)}]{leal1971effect}%
	\BibitemOpen
	\bibfield  {author} {\bibinfo {author} {\bibfnamefont {L.}~\bibnamefont
			{Leal}}\ and\ \bibinfo {author} {\bibfnamefont {E.}~\bibnamefont {Hinch}},\
	}\href@noop {} {\bibfield  {journal} {\bibinfo  {journal} {Journal of Fluid
				Mechanics}\ }\textbf {\bibinfo {volume} {46}},\ \bibinfo {pages} {685}
		(\bibinfo {year} {1971})}\BibitemShut {NoStop}%
	\bibitem [{\citenamefont {Elgeti}\ \emph {et~al.}(2015)\citenamefont {Elgeti},
		\citenamefont {Winkler},\ and\ \citenamefont {Gompper}}]{elgeti2015physics}%
	\BibitemOpen
	\bibfield  {author} {\bibinfo {author} {\bibfnamefont {J.}~\bibnamefont
			{Elgeti}}, \bibinfo {author} {\bibfnamefont {R.~G.}\ \bibnamefont {Winkler}},
		\ and\ \bibinfo {author} {\bibfnamefont {G.}~\bibnamefont {Gompper}},\
	}\href@noop {} {\bibfield  {journal} {\bibinfo  {journal} {Reports on
				progress in physics}\ }\textbf {\bibinfo {volume} {78}},\ \bibinfo {pages}
		{056601} (\bibinfo {year} {2015})}\BibitemShut {NoStop}%
	\bibitem [{\citenamefont {Junot}\ \emph {et~al.}(2019)\citenamefont {Junot},
		\citenamefont {Figueroa-Morales}, \citenamefont {Darnige}, \citenamefont
		{Lindner}, \citenamefont {Soto}, \citenamefont {Auradou},\ and\ \citenamefont
		{Cl{\'e}ment}}]{junot2019swimming}%
	\BibitemOpen
	\bibfield  {author} {\bibinfo {author} {\bibfnamefont {G.}~\bibnamefont
			{Junot}}, \bibinfo {author} {\bibfnamefont {N.}~\bibnamefont
			{Figueroa-Morales}}, \bibinfo {author} {\bibfnamefont {T.}~\bibnamefont
			{Darnige}}, \bibinfo {author} {\bibfnamefont {A.}~\bibnamefont {Lindner}},
		\bibinfo {author} {\bibfnamefont {R.}~\bibnamefont {Soto}}, \bibinfo {author}
		{\bibfnamefont {H.}~\bibnamefont {Auradou}}, \ and\ \bibinfo {author}
		{\bibfnamefont {E.}~\bibnamefont {Cl{\'e}ment}},\ }\href@noop {} {\bibfield
		{journal} {\bibinfo  {journal} {EPL (Europhysics Letters)}\ }\textbf
		{\bibinfo {volume} {126}},\ \bibinfo {pages} {44003} (\bibinfo {year}
		{2019})}\BibitemShut {NoStop}%
	\bibitem [{\citenamefont {Brown}\ \emph {et~al.}(2000)\citenamefont {Brown},
		\citenamefont {Clarke}, \citenamefont {Convert},\ and\ \citenamefont
		{Rennie}}]{brown2000orientational}%
	\BibitemOpen
	\bibfield  {author} {\bibinfo {author} {\bibfnamefont {A.}~\bibnamefont
			{Brown}}, \bibinfo {author} {\bibfnamefont {S.}~\bibnamefont {Clarke}},
		\bibinfo {author} {\bibfnamefont {P.}~\bibnamefont {Convert}}, \ and\
		\bibinfo {author} {\bibfnamefont {A.}~\bibnamefont {Rennie}},\ }\href@noop {}
	{\bibfield  {journal} {\bibinfo  {journal} {Journal of rheology}\ }\textbf
		{\bibinfo {volume} {44}},\ \bibinfo {pages} {221} (\bibinfo {year}
		{2000})}\BibitemShut {NoStop}%
	\bibitem [{\citenamefont {Einstein}(1906)}]{einstein1906new}%
	\BibitemOpen
	\bibfield  {author} {\bibinfo {author} {\bibfnamefont {A.}~\bibnamefont
			{Einstein}},\ }\href@noop {} {\bibfield  {journal} {\bibinfo  {journal} {Ann.
				Phys.}\ }\textbf {\bibinfo {volume} {19}},\ \bibinfo {pages} {289} (\bibinfo
		{year} {1906})}\BibitemShut {NoStop}%
	\bibitem [{\citenamefont {F{\'e}rec}\ \emph {et~al.}(2017)\citenamefont
		{F{\'e}rec}, \citenamefont {Bertevas}, \citenamefont {Khoo}, \citenamefont
		{Ausias},\ and\ \citenamefont {Phan-Thien}}]{ferec2017steady}%
	\BibitemOpen
	\bibfield  {author} {\bibinfo {author} {\bibfnamefont {J.}~\bibnamefont
			{F{\'e}rec}}, \bibinfo {author} {\bibfnamefont {E.}~\bibnamefont {Bertevas}},
		\bibinfo {author} {\bibfnamefont {B.}~\bibnamefont {Khoo}}, \bibinfo {author}
		{\bibfnamefont {G.}~\bibnamefont {Ausias}}, \ and\ \bibinfo {author}
		{\bibfnamefont {N.}~\bibnamefont {Phan-Thien}},\ }\href@noop {} {\bibfield
		{journal} {\bibinfo  {journal} {Journal of Non-Newtonian Fluid Mechanics}\
		}\textbf {\bibinfo {volume} {239}},\ \bibinfo {pages} {62} (\bibinfo {year}
		{2017})}\BibitemShut {NoStop}%
	\bibitem [{\citenamefont {Bird}\ \emph
		{et~al.}(1987{\natexlab{b}})\citenamefont {Bird}, \citenamefont {Curtiss},
		\citenamefont {Armstrong},\ and\ \citenamefont
		{Hassager}}]{birdTWO1987dynamics}%
	\BibitemOpen
	\bibfield  {author} {\bibinfo {author} {\bibfnamefont {R.~B.}\ \bibnamefont
			{Bird}}, \bibinfo {author} {\bibfnamefont {C.~F.}\ \bibnamefont {Curtiss}},
		\bibinfo {author} {\bibfnamefont {R.~C.}\ \bibnamefont {Armstrong}}, \ and\
		\bibinfo {author} {\bibfnamefont {O.}~\bibnamefont {Hassager}},\ }\href@noop
	{} {\emph {\bibinfo {title} {Dynamics of polymeric liquids, volume 2: Kinetic
				theory}}}\ (\bibinfo  {publisher} {Wiley},\ \bibinfo {year}
	{1987})\BibitemShut {NoStop}%
	\bibitem [{\citenamefont {Ren}\ \emph {et~al.}(2019)\citenamefont {Ren},
		\citenamefont {Nama}, \citenamefont {McNeill}, \citenamefont {Soto},
		\citenamefont {Yan}, \citenamefont {Liu}, \citenamefont {Wang}, \citenamefont
		{Wang},\ and\ \citenamefont {Mallouk}}]{ren20193d}%
	\BibitemOpen
	\bibfield  {author} {\bibinfo {author} {\bibfnamefont {L.}~\bibnamefont
			{Ren}}, \bibinfo {author} {\bibfnamefont {N.}~\bibnamefont {Nama}}, \bibinfo
		{author} {\bibfnamefont {J.~M.}\ \bibnamefont {McNeill}}, \bibinfo {author}
		{\bibfnamefont {F.}~\bibnamefont {Soto}}, \bibinfo {author} {\bibfnamefont
			{Z.}~\bibnamefont {Yan}}, \bibinfo {author} {\bibfnamefont {W.}~\bibnamefont
			{Liu}}, \bibinfo {author} {\bibfnamefont {W.}~\bibnamefont {Wang}}, \bibinfo
		{author} {\bibfnamefont {J.}~\bibnamefont {Wang}}, \ and\ \bibinfo {author}
		{\bibfnamefont {T.~E.}\ \bibnamefont {Mallouk}},\ }\href@noop {} {\bibfield
		{journal} {\bibinfo  {journal} {Sci. Adv.}\ }\textbf {\bibinfo {volume}
			{5}},\ \bibinfo {pages} {3084} (\bibinfo {year} {2019})}\BibitemShut
	{NoStop}%
	\bibitem [{\citenamefont {Aghakhani}\ \emph {et~al.}(2020)\citenamefont
		{Aghakhani}, \citenamefont {Yasa}, \citenamefont {Wrede},\ and\ \citenamefont
		{Sitti}}]{aghakhani2020acoustically}%
	\BibitemOpen
	\bibfield  {author} {\bibinfo {author} {\bibfnamefont {A.}~\bibnamefont
			{Aghakhani}}, \bibinfo {author} {\bibfnamefont {O.}~\bibnamefont {Yasa}},
		\bibinfo {author} {\bibfnamefont {P.}~\bibnamefont {Wrede}}, \ and\ \bibinfo
		{author} {\bibfnamefont {M.}~\bibnamefont {Sitti}},\ }\href@noop {}
	{\bibfield  {journal} {\bibinfo  {journal} {Proc. Natl. Acad. Sci.}\ }\textbf
		{\bibinfo {volume} {117}},\ \bibinfo {pages} {3469} (\bibinfo {year}
		{2020})}\BibitemShut {NoStop}%
	\bibitem [{\citenamefont {Bozorgi}\ and\ \citenamefont
		{Underhill}(2011)}]{bozorgi2011effect}%
	\BibitemOpen
	\bibfield  {author} {\bibinfo {author} {\bibfnamefont {Y.}~\bibnamefont
			{Bozorgi}}\ and\ \bibinfo {author} {\bibfnamefont {P.~T.}\ \bibnamefont
			{Underhill}},\ }\href@noop {} {\bibfield  {journal} {\bibinfo  {journal}
			{Physical Review E}\ }\textbf {\bibinfo {volume} {84}},\ \bibinfo {pages}
		{061901} (\bibinfo {year} {2011})}\BibitemShut {NoStop}%
	\bibitem [{\citenamefont {Ardekani}\ and\ \citenamefont
		{Gore}(2012)}]{ardekani2012emergence}%
	\BibitemOpen
	\bibfield  {author} {\bibinfo {author} {\bibfnamefont {A.~M.}\ \bibnamefont
			{Ardekani}}\ and\ \bibinfo {author} {\bibfnamefont {E.}~\bibnamefont
			{Gore}},\ }\href@noop {} {\bibfield  {journal} {\bibinfo  {journal} {Phys.
				Rev. E}\ }\textbf {\bibinfo {volume} {85}},\ \bibinfo {pages} {056309}
		(\bibinfo {year} {2012})}\BibitemShut {NoStop}%
	\bibitem [{\citenamefont {Narinder}\ \emph {et~al.}(2018)\citenamefont
		{Narinder}, \citenamefont {Bechinger},\ and\ \citenamefont
		{Gomez-Solano}}]{narinder2018memory}%
	\BibitemOpen
	\bibfield  {author} {\bibinfo {author} {\bibfnamefont {N.}~\bibnamefont
			{Narinder}}, \bibinfo {author} {\bibfnamefont {C.}~\bibnamefont {Bechinger}},
		\ and\ \bibinfo {author} {\bibfnamefont {J.~R.}\ \bibnamefont
			{Gomez-Solano}},\ }\href@noop {} {\bibfield  {journal} {\bibinfo  {journal}
			{Phys. Rev. Lett.}\ }\textbf {\bibinfo {volume} {121}},\ \bibinfo {pages}
		{078003} (\bibinfo {year} {2018})}\BibitemShut {NoStop}%
	\bibitem [{\citenamefont {Abtahi}\ and\ \citenamefont
		{Elfring}(2019)}]{abtahi2019jeffery}%
	\BibitemOpen
	\bibfield  {author} {\bibinfo {author} {\bibfnamefont {S.~A.}\ \bibnamefont
			{Abtahi}}\ and\ \bibinfo {author} {\bibfnamefont {G.~J.}\ \bibnamefont
			{Elfring}},\ }\href@noop {} {\bibfield  {journal} {\bibinfo  {journal}
			{Physics of Fluids}\ }\textbf {\bibinfo {volume} {31}},\ \bibinfo {pages}
		{103106} (\bibinfo {year} {2019})}\BibitemShut {NoStop}%
	\bibitem [{\citenamefont {Elfring}\ and\ \citenamefont
		{Lauga}(2015)}]{elfring2015theory}%
	\BibitemOpen
	\bibfield  {author} {\bibinfo {author} {\bibfnamefont {G.~J.}\ \bibnamefont
			{Elfring}}\ and\ \bibinfo {author} {\bibfnamefont {E.}~\bibnamefont
			{Lauga}},\ }in\ \href@noop {} {\emph {\bibinfo {booktitle} {Complex fluids in
				biological systems}}}\ (\bibinfo  {publisher} {Springer},\ \bibinfo {year}
	{2015})\ pp.\ \bibinfo {pages} {283--317}\BibitemShut {NoStop}%
	\bibitem [{\citenamefont {Bretherton}(1962)}]{bretherton1962}%
	\BibitemOpen
	\bibfield  {author} {\bibinfo {author} {\bibfnamefont {F.~P.}\ \bibnamefont
			{Bretherton}},\ }\href@noop {} {\bibfield  {journal} {\bibinfo  {journal} {J.
				Fluid Mech.}\ }\textbf {\bibinfo {volume} {14}},\ \bibinfo {pages} {284}
		(\bibinfo {year} {1962})}\BibitemShut {NoStop}%
	\bibitem [{\citenamefont {Strogatz}(2018)}]{strogatz}%
	\BibitemOpen
	\bibfield  {author} {\bibinfo {author} {\bibfnamefont {S.~H.}\ \bibnamefont
			{Strogatz}},\ }\href@noop {} {\emph {\bibinfo {title} {Nonlinear dynamics and
				chaos: with applications to physics, biology, chemistry, and engineering}}}\
	(\bibinfo  {publisher} {CRC Press},\ \bibinfo {year} {2018})\BibitemShut
	{NoStop}%
	\bibitem [{\citenamefont {Arfken}\ and\ \citenamefont
		{Weber}(1999)}]{arfken1999mathematical}%
	\BibitemOpen
	\bibfield  {author} {\bibinfo {author} {\bibfnamefont {G.~B.}\ \bibnamefont
			{Arfken}}\ and\ \bibinfo {author} {\bibfnamefont {H.~J.}\ \bibnamefont
			{Weber}},\ }\href@noop {} {\emph {\bibinfo {title} {Mathematical methods for
				physicists}}}\ (\bibinfo  {publisher} {American Association of Physics
		Teachers},\ \bibinfo {year} {1999})\BibitemShut {NoStop}%
\end{thebibliography}%


\end{document} 
	%
	% ****** End of file apssamp.tex ******
	%	($ \dot{\gamma} t_{\text{relax}} < 1 $) 
