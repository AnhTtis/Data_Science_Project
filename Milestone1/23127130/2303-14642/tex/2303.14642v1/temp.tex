\documentclass[preprintnumbers, showpacs, amsmath, showkeys, amssymb, aps, superscriptaddress, twocolumn, longbibliography, letterpaper]{revtex4-2}
%\bibliographystyle{apsrev4-2}
%\documentclass[prd,showpacs,preprintnumbers,amsmath,amssymb,nofootinbib, aps, letterpaper]{revtex4-1}
\usepackage{lipsum, babel}
\usepackage{color}
%\usepackage{epstopdf}
\usepackage{hyperref}
\usepackage{bm}
\usepackage{amsfonts}
\usepackage{mathrsfs}
\usepackage{graphicx}
\usepackage{amsmath}
\usepackage{float}
\usepackage{braket}
%\usepackage{biblatex}
\usepackage{natbib}

%\usepackage[english]{babel}
\newcommand{\beginsupplement}{%
       \setcounter{table}{0}
        \renewcommand{\thetable}{S\arabic{table}}%
        \setcounter{figure}{0}
        \renewcommand{\thefigure}{S\arabic{figure}}%
     }
     \hypersetup{
    colorlinks=true,
    linkcolor=red,
    citecolor=blue,
} 



\begin{document}

\newcommand{\sgn}{\operatorname{sgn}}
\newcommand{\hhat}[1]{\hat {\hat{#1}}}
\newcommand{\pslash}[1]{#1\llap{\sl/}}
\newcommand{\kslash}[1]{\rlap{\sl/}#1}
%\newcommand{\lab}[1]{\hypertarget{lb:#1}}
\newcommand{\lab}[1]{}
% to create labels.
%\newcommand{\iref}[2]{\footnote{\hyperlink{lb:#1}{\textit{$^\spadesuit$#2}}}}
%\newcommand{\iref}[2]{}
% reference to other part in this notes.
\newcommand{\sto}[1]{\begin{center} \textit{#1} \end{center}}
\newcommand{\rf}[1]{{\color{blue}[\textit{#1}]}}
% Reference
\newcommand{\eml}[1]{#1}
% Emphasis
\newcommand{\el}[1]{\label{#1}}
% Equation labeling
\newcommand{\er}[1]{Eq.\eqref{#1}}
% Equation Reference
\newcommand{\df}[1]{\textbf{#1}}
% Temporarily replace \textbf
\newcommand{\mdf}[1]{\pmb{#1}}
% Use for vectors etc.
\newcommand{\ft}[1]{\footnote{#1}}
% footnote.
\newcommand{\n}[1]{$#1$}
% Use for numbers etc.
% \newcommand{\cjktext}[1]{\begin{CJK}{GB}{gbsn} #1 \end{CJK}} 
% Language support
\newcommand{\fals}[1]{$^\times$ #1}
% wrong statement
\newcommand{\new}{{\color{red}$^{NEW}$ }}
% update
% \newcommand{\ci}[1]{\cite{#1}}
\newcommand{\ci}[1]{}
\newcommand{\de}[1]{{\color{green}\underline{#1}}}
\newcommand{\ke}{\rangle}
\newcommand{\br}{\langle}
\newcommand{\lb}{\left(}
\newcommand{\rb}{\right)}
\newcommand{\lbk}{\left[}
\newcommand{\rbk}{\right]}
\newcommand{\blb}{\Big(}
\newcommand{\brb}{\Big)}
\newcommand{\nn}{\nonumber \\}
\newcommand{\p}{\partial}
\newcommand{\pd}[1]{\frac {\partial} {\partial #1}}
\newcommand{\cd}{\nabla}
\newcommand{\cc}{$>$}
% ##### ###### ###### ######
\newcommand{\bqa}{\begin{eqnarray}}
\newcommand{\eqa}{\end{eqnarray}}
\newcommand{\bqe}{\begin{equation}}
\newcommand{\eqe}{\end{equation}}
\newcommand{\bay}[1]{\left(\begin{array}{#1}}
\newcommand{\eay}{\end{array}\right)}
\newcommand{\eg}{\textit{e.g.} }
\newcommand{\ie}{\textit{i.e.}, }
\newcommand{\iv}[1]{{#1}^{-1}}
\newcommand{\st}[1]{|#1\ke}
\newcommand{\at}[1]{{\Big|}_{#1}}
\newcommand{\zt}[1]{\texttt{#1}}
\newcommand{\non}{\nonumber}
\newcommand{\m}{\mu}
% ##### ###### ###### ######
% Greek Letters
\def\xa{{m}}
\def\xA{{m}}
\def\xb{{\beta}}
\def\xB{{\Beta}}
\def\xd{{\delta}}
\def\xD{{\Delta}}
\def\xe{{\epsilon}}
\def\xE{{\Epsilon}}
\def\xve{{\varepsilon}}
\def\xg{{\gamma}}
\def\xG{{\Gamma}}
\def\xk{{\kappa}}
\def\xK{{\Kappa}}
\def\xl{{\lambda}}
\def\xL{{\Lambda}}
\def\xo{{\omega}}
\def\xO{{\Omega}}
\def\xvp{{\varphi}}
\def\xs{{\sigma}}
\def\xS{{\Sigma}}
\def\xt{{\theta}}
\def\xvt{{\vartheta}}
\def\xT{{\Theta}}
% ##### ###### ###### ######
\def \Tr {{\rm Tr}}
\def\CA{{\cal A}}
\def\CC{{\cal C}}
\def\CD{{\cal D}}
\def\CE{{\cal E}}
\def\CF{{\cal F}}
\def\CH{{\cal H}}
\def\CJ{{\cal J}}
\def\CK{{\cal K}}
\def\CL{{\cal L}}
\def\CM{{\cal M}}
\def\CN{{\cal N}}
\def\CO{{\cal O}}
\def\CP{{\cal P}}
\def\CQ{{\cal Q}}
\def\CR{{\cal R}}
\def\CS{{\cal S}}
\def\CT{{\cal T}}
\def\CV{{\cal V}}
\def\CW{{\cal W}}
\def\CY{{\cal Y}}
%
\def\BC{\mathbb{C}}
\def\BR{\mathbb{R}}
\def\BZ{\mathbb{Z}}
%
\def\sA{\mathscr{A}}
\def\sB{\mathscr{B}}
\def\sF{\mathscr{F}}
\def\sG{\mathscr{G}}
\def\sH{\mathscr{H}}
\def\sJ{\mathscr{J}}
\def\sL{\mathscr{L}}
\def\sM{\mathscr{M}}
\def\sN{\mathscr{N}}
\def\sO{\mathscr{O}}
\def\sP{\mathscr{P}}
\def\sR{\mathscr{R}}
\def\sQ{\mathscr{Q}}
\def\sS{\mathscr{S}}
\def\sX{\mathscr{X}}

\def\slz{SL(2,Z)}
\def\slr{$SL(2,R)\times SL(2,R)$ }
\def\ads{${AdS}_5\times {S}^5$ }
\def\adst{${AdS}_3$ }
%
\def\sun{SU(N)}
\def\ad#1#2{{\frac \delta {\delta\sigma^{#1}} (#2)}}
% for SU(N) SYM
\def\bqf{\bar Q_{\bar f}}
\def\nf{N_f}
\def\sunf{SU(N_f)}

\def\dcirc{{^\circ_\circ}}

\author{Morgan H. Lynch}
\email{morganlynch1984@gmail.com}
\affiliation{Center for Theoretical Physics,
Seoul National University, Seoul 08826, Korea}



\title{Experimental observation of a Rindler horizon}
\date{\today}

\begin{abstract}
In this manuscript we confirm the presence of a Rindler horizon at CERN-NA63 by exploring its thermodynamics induced by the Unruh effect in their high energy channeling radiation experiments. By linking the entropy of the emitted radiation to the photon number, we find the measured spectrum to be a simple manifestation of the second law of Rindler horizon thermodynamics and thus a direct measurement of the recoil Fulling-Davies-Unruh (FDU) temperature. Moreover, since the experiment is born out of an ultra-relativistic positron, and the FDU temperature is defined in the proper frame, we find that temperature boosts as a length and thus fast objects appear colder. The spectrum also provides us with a simple setting to measure fundamental constants, and we employ it to measure the positron mass. 
\end{abstract}

%\pacs{04.60.Bc, 04.62.+v, 04.70.Dy}

\maketitle

\section{Introduction}

The pursuits of quantum field theory in curved spacetime have led to a wide array of surprising discoveries. Most notably, it completely changed our understanding of such simple questions as how many particles are present in a given system. This ambiguity in particle number gave rise to a new class of kinematic particle production induced by horizons; namely the Parker, Hawking, and the Fulling-Davies-Unruh effects \cite{Parker:1968mv, hawking1974black, Davies:1976hi, Davies:1977yv, Unruh:1976db, lynch2021experimental, lynchgood}. Most surprising of all is the fact that the particles which are produced via these effects are thermalized at a temperature which is characterized by the acceleration scale of the system. In all cases, this acceleration characterizes the location of a horizon, either real or apparent, which is also present. 

When considering these horizons, one can also ascribe thermodynamic quantities to the system. Most notably, via the second law of thermodynamics, the connection between the area of the horizon, or area change, and the entropy encoded within it \cite{Bekenstein:1973ur, hawking1974black, bianchi2013mechanical}. All of these notions combined paint a rather surprising picture; that for an accelerated observer, the presence of these thermalized particles along with the energy/momentum flux of particles through the associated horizon gives rise to an area change in the horizon, which is completely determined by general relativity; i.e. the Einstein equation is the thermodynamic equation of state of these effervescent particles \cite{Jacobson:1995ab}. The implication being that gravitation is an emergent quantity born out of these vaccuum fluctuations. 

The recent experimental observation of radiation reaction and the Unruh effect by CERN-NA63 \cite{Wistisen:2017pgr, lynch2021experimental} provides us with an unprecendented opportunity to investigate the various tenets of quantum field theory in curved spacetime. In particular, we will now examine the data via the thermodynamics of the Rindler horizon. 

	
\section{Rindler horizon thermodynamics}

Thermodynamics offers a surprisingly simple technique to examine the properties of both black holes as well as Rindler horizons \cite{Bekenstein:1973ur,Jacobson:1995ab, bianchi2013mechanical}. The temperatures of these systems can also be explored via radiation reaction. There we have a FDU temperature which is set by the recoil kinetic energy, $T_{FDU} =\frac{(\hbar \omega)^2}{2mc^2k_{B}}$. Systems that are thermalized at this temperature will then, by necessity, obey the second law of thermodynamics,
\bqe
dQ = k_{B}T dS.
\eqe
As we shall see, this statement of the second law not only describes the relationship between energy flux and entropy generation, but it also describes the spectrum of thermalized particles due to the Unruh effect. Ultimately, this second law, and its application to Rindler horizons, lies at the heart of our understanding of gravity. As it has been shown \cite{Jacobson:1995ab}, this second law along with the Bekenstein-Hawking area-entropy relation, $S = \frac{1}{4 \ell_{P}^2}A$, with $\ell_{P}^{2}$ being the Planck area, is equivalent to the Einstein equation of general relativity.

\subsection{Experimental methods}
The CERN-NA63 experimental site studies various aspects of strong field QED \cite{2019PhRvR...1c3014W, 2023PhRvL.130g1601N}. Here we analyse their high energy channeling radiation experiement which successfully measured radiation reaction \cite{Wistisen:2017pgr}. There, an ultra-relativistic $178.2$ GeV positron traverses a 3.8 mm thick sample of single crystal silicon along the $\braket{111}$ axis. These ``channeled" positrons are repelled by the atomic lattice and undergo a transverse harmonic oscillation which causes a photon to build up around the positron as it is pumped up the ladder of harmonic oscillator states. This photon becomes so energetic that its energy is comparable to the positron rest mass and upon emission, the positron experiences an enormous recoil acceleration. This acceleration is sufficiently strong enough to thermalize the system via the Unruh effect \cite{lynch2021experimental}.

In order to analyze the NA63 data set using the second law, we must first transform their power spectrum, $\frac{dE}{d(\hbar \omega)dt}$ into a photon spectrum, $\frac{dN}{d(\hbar \omega)}$. We shall take the crystal crossing time to be the time interval, $dt = (3.8$ mm)$/c$. Next, the single particle spectrum, is given by $\frac{dE_{sp}}{d(\hbar \omega)} \frac{1}{\hbar \omega}$ and the differential bin spectrum is given by $\frac{dE}{d(\hbar \omega)} \frac{1}{\Delta E_{b}}$, where $\Delta E_{b} =1.007$ GeV is the bin width. \textit{It is this differential bin spectrum which we will analyze.} Thus we have,
\bqe
\frac{dN}{d(\hbar \omega)} = \frac{dE_{data}}{d(\hbar \omega)dt} \frac{3.8\;mm}{c\Delta E_{b}}
\label{db}
\eqe
A comparison of the different spectra is presented in Fig. (\ref{plot1}). We will now turn to the theoretical description of this data based on Rindler horizon thermodynamics.
\begin{figure}[H]
\centering  
\includegraphics[scale=.24]{specbin.pdf}
\caption{A comparison of the single particle photon spectrum and the differential bin spectrum, Eqn. (\ref{db}), of the CERN-NA63 3.8 mm radiation reaction data set \cite{Wistisen:2017pgr}.} 	
\label{plot1}
\end{figure}

\subsection{The Rindler entropy photon spectrum}

Although it is comonly believed that thermal radiation contains no information, there is, in fact, a persistance of information in the particles radiated \cite{2016PhLB..757..383A, 2018PhLB..776...10A}. Although small, each photon emitted will carry away a portion of the information content present prior to burning. This provides us with a method to link the spectrum of a thermal system to its information content via the second law of thermodynamics.

We begin by exploring the information content of the Rindler horizon. In particular, we shall examine the entropy-information content which is ``embedded" into the the surface of the horizon by the radiation that passes through it. In consideration of the entropy, $S$, we will then have $\alpha_S$ nats of information per $N$ photons, i.e. $S = \alpha_{S} N$. Finally taking the energy that crosses the horizon to be that of one photon, $dQ = d(\hbar \omega)$, our second law can be written as
\bqe
\frac{dN}{d(\hbar \omega)} = \frac{1}{\alpha_{S}} \frac{1}{k_{B}T}.
\eqe
The above expression is the photon spectrum which has been mapped from the entropy of the Rindler horizon to the photons which have passed through it. These photons will be thermalized at the FDU temperature, $T = T_{FDU}$. We must note that there is still a slight ambiguity in the recoil temperature due to the photon emission time. From the relativistic Newtons law, we have a proper acceleration of $a = \frac{1}{m}\frac{\Delta p}{\Delta t }$. We have $\Delta p = \hbar \omega/c$, however for radiation reaction we also expect the time scale of emission to be on the order of a photon period, i.e. $\Delta t \sim 1/ \omega$. What the precise proportionality is, remains unknown. In the radiation analysis of the Unruh effect \cite{lynch2021experimental}, $\Delta t = \pi / \omega$ was used, and here we have used $\Delta t = 1/(\pi \omega)$ such that the temperature is set to the recoil kinetic energy. The true form of the recoil acceleration, and thus temperature remains an open problem however. We simply note, that there is still a constant of order unity which remains to be fixed. As such, we will adopt a coefficient $\alpha_{T}$ to rescale our emission time, $\Delta t = \alpha_{T}/(\pi \omega)$ in the temperature, $T_{FDU} =\frac{(\hbar \omega)^2}{\alpha_{T}2mc^2k_{B}}$. 

Finally, we note the fact that this is an ultra-relativistic system that is thermalized and thus gives us insight into how a temperature boosts \cite{einstein.., 1966Natur.212..571L, 1967Natur.214..903L, 2017NatSR...717657F}. \textit{We find that temperature boosts like a length, i.e.} $T_{lab} = T_{proper}/\gamma$. As such, in order to match the data set, our temperature must be boosted into the lab frame via, $T_{lab} = T_{FDU}/\gamma$. Thus, our spectrum is given by,
\bqe
\frac{dN}{d(\hbar \omega)} =\alpha_{TS}\frac{2mc^2 \gamma	}{(\hbar \omega)^2}.
\eqe
By taking a best fit of $\alpha_{TS} = \alpha_{T}/\alpha_{S}$, we can gain insight into the number of bits associated with each photon as well as the unknown emission timescale. With a threshold of $22$ GeV based on the thermalization time of the system and using the energy range of $30$ GeV to $120$ GeV, where the chi-squared statistic lies within the 1 standard deviation threshold in the radiation analysis \cite{lynch2021experimental}, we find $\alpha_{TS} = 1.61$ with a reduced chi-squared statistic of 1.90. We must comment on the fact that although this chi-squared is excellent, it is not below the threshold of 1.26. This is most likely due to the fact there are, in principle, additional terms in our second law which are not included in this analysis that would correspond to chemical potentials. These reflect the various terms which comprise the energy gap of the Unruh-DeWitt detector. We also note the entropy content per photon of blackbody radiation is given by \cite{2016PhLB..757..383A, 2018PhLB..776...10A}, $\alpha_{S} = 2.7 \pm 1.7$. Thus if we take $\alpha_{S} = 2.7$ then this would imply that $\alpha_{T} = 4.38$ and thus $\Delta t = 1.39/\omega$, i.e. approximately one fourth of a photon period. Therefore, the ambiguity due to these two parameters is an overall factor of order unity. Given one, the other is then fixed. What is important to note, is that the entropy content, $\alpha_{S} = 2.7$, is for a 3-dimensional blackbody and its value depends crucially on the dimensionality of the system \cite{1989JPhA...22.1073L}. It is an important concept to consider, that the dimensionality that governs the thermodynamics of the Rindler horizon may be 1 or perhaps 2-dimensional \cite{Bekenstein:2001tj}.
\begin{figure}[H]
\centering  
\includegraphics[scale=.24]{bita.pdf}
\caption{Here we present the measured photon bin spectrum compared to the theoretical prediction based on the second law of Rindler horizon thermodynamics. Here, the 2nd law spectrum is with $\alpha_{TS} = 1$ and the $\alpha_{TS}$-2nd law spectrum is the best fit with $\alpha_{TS} = 1.61$. The reduced chi-squared per degree of freedom for this fit is 1.90.} 	
\label{plot2}
\end{figure}
\subsection{Measurement of the positron mass}
Given a succesful description of the photon spectrum based on the second law of thermodynamics, it is interesting to note that within the temperature, one is able to isolate any of the physical constants present; namely $c$, $k_{B}$, $\hbar$, or the positron mass $m_{p}$. This provides an interesting technique for measuring physical constants. Although many of these constants are used as experimental inputs prior to the measurement, it seems reasonable that one, say the positron mass, $m_{p}$,  may be left out of any preliminary experimental inputs and then be measured via the overall coefficient on the frequency dependent temperature, $T_{FDU}$. Thus, we can solve for the positron mass to yield,
\bqe
m =\frac{1}{ \alpha_{TS}} \frac{(\hbar \omega)^2	}{2c^2\gamma}\frac{dN}{d(\hbar \omega)}.
\eqe
The measurement of the positron mass is presented in Fig. (\ref{plot3}). As such, we find, via the recoil FDU temperature, a thermometer capable of measuring mass and/or other physical constants. From the data set, we average the measured positron mass over the region where the chi-squared statistic remains below the 1-standard deviation threshold from the radiation analysis; $30-120$ GeV. We obtain a value of $m_{p}=1.04 \pm .221 \times 10^{-30}$ kg or $.583 \pm .124$ MeV/$c^2$. Here the error comes from the standard deviation of the mean. Note, the statistical error here is the dominant source of error as the experimental error bars are smaller than the data points. The accepted value of the positron mass is $m_{p} = 9.11 \times 10^{-31}$ kg or $.511$ MeV/$c^{2}$ \cite{Workman:2022ynf}. \\

\begin{figure}[H]
\centering  
\includegraphics[scale=.24]{electron.pdf}
\caption{Here we see the convergence of the experimental data set to the positron rest mass. Given a recoil FDU temperature, $T_{FDU} =\frac{(\hbar \omega)^2}{2mc^2k_{B}} $, one can use an Unruh-thermalized spectrum to measure a selection of fundamental constants.} 	
\label{plot3}
\end{figure}


\section{conclusions}
In this manuscript, we have analyzed the Unruh-thermalized photon spectrum of channeled positrons undergoing the extreme accelerations of radiation reaction measured by CERN-NA63. We employed an analysis based entirely on the 2nd law of Rindler horizon thermodynamics, at the recoil FDU temperature, and found an excellent agreement with the data. Moreover, the ultra-relativistic nature of the system revealed that temperatures boost like a length, i.e. $T_{lab} = T_{proper}/\gamma$. The technique also provides a novel way to experimentally measure the positron mass. 

\section*{Acknowledgments}
This work has been supported by the National Research Foundation of Korea under Grants No.~2017R1A2A2A05001422 and No.~2020R1A2C2008103.

\goodbreak


\bibliography{temp} 





\end{document}

