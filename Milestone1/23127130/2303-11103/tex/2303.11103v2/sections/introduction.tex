\section{Introduction}

Many 6G research topics require the simulation of specific radio environments by ray tracing. Examples are \gls{ISAC} \cite{sensing}, multi-modal sensing \cite{alkhateeb2022deepsense}, \glspl{RIS} \cite{ris}, radio-based localization \cite{studer2018channel}, \gls{ML}-based transceiver algorithms \cite{aiai}, as well as most of the use-cases of the recently started 3GPP study-item on AI/ML for the air interface \cite{sidai2021}. The reason for this is that a spatially consistent correspondence between a physical location in a scene and the \gls{CIR} is required which the widely used stochastic channel models such as \cite{3gpp38901} cannot provide. 
In addition, there is an increasing interest in ray tracing for the creation of digital twin networks \cite{alkhateeb2023real, lin20226g}. For these reasons, we have added a differentiable ray tracing module (RT) in release v0.14 of our open-source link-level simulator Sionna\texttrademark{} \cite{sionna} that we will introduce in this article.

Although ray tracing for radio propagation is a mature field \cite{iskander2015}, it has lately received renewed interest due to the development of \gls{RIS} \cite{9713744} and the potential of \gls{ML} techniques to improve accuracy and speed up computations \cite{deepray, ray-launching-wave-prop}. Several papers use \glspl{NN} for path loss prediction \cite{zhang2020cellular, yapar-u-net, 9954403, 9722715}, while \cite{winert} attempts to model ray-surface interactions by \glspl{NN}. The prevalence of \gls{ML} and \glspl{NN} is also apparent in the field of computer vision which is currently disrupted by neural rendering techniques such as \glspl{NeRF} \cite{muller2022instant}, while inverse rendering and differentiable ray tracing methods gain in popularity \cite{diffrt2018, Jakob2020DrJit, Mitsuba3}.

Driven by the aforementioned fusion of \gls{ML} and ray tracing, we have decided to develop Sionna RT as the world's first differentiable ray tracer for radio propagation  modeling. That means that functions of the generated field components, such as \glspl{CIR} and coverage maps, can be differentiated with respect to most parameters that are involved in their computation. These comprise the constituent material properties (conductivity $\sigma$ and relative permittivity $\varepsilon_r$), parameters of scattering functions, array orientations, positions, and geometries, as well as antenna patterns. In the future, we will include the configuration of \glspl{RIS} and multiple other ray-object interactions.
