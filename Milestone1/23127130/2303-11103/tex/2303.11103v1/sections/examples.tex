\section{Example Applications} 
In this section, we will present two applications of differentiable ray tracing that are enabled by Sionna RT. The code to reproduce the results is available.\footnote{\url{https://github.com/NVlabs/diff-rt}}

\subsection{Learning Radio Materials}
Scene construction is an essential part of the ray tracing process. While scene geometries can be rather easily obtained today, e.g., from building databases such as OpenStreetMap \cite{osm}, there is no straightforward process to obtain the material properties. It is therefore important to develop techniques that automatically assign material properties, such as permittivity, conductivity, permeability, roughness, and scattering functions, to all objects in a scene.

Differentiable ray tracing allows us to optimize material properties based on data using gradient-based learning techniques. In this example application, we use Sionna RT to generate a data set of \glspl{CIR} from a scene with preconfigured radio materials. Pretending that the material properties
are unknown, we initialize them with default values and optimize them by gradient descent on the normalized mean squared error between the original \glspl{CIR} and those computed based on the \emph{trainable} materials. 

\begin{figure}[!t]
    \centering
    \includegraphics[width=\columnwidth]{figures/learning_materials.png}
    \caption{Data set generation for learning radio materials. The transmitter is located on a roof (blue) while 200 receivers (green) are located within the scene.}
    \label{fig:learning_materials}
\end{figure}

Fig.~\ref{fig:learning_materials} shows the locations of the transmitter and 200 receivers in the scene \texttt{sionna.rt.scene.etoile} for which \glspl{CIR} where generated. The original scene uses only four different materials: concrete for the streets, marble for the building walls, metal for the roofs, and wood for the ceiling underneath the Arc de Triomphe. We first generate 200 \glspl{CIR} that are transformed into the frequency domain assuming an \gls{OFDM} system with 128 subcarriers spaced 30kHz apart. To start, the radio materials are replaced by trainable ones with an initial conductivity $\sigma$ of 0.1S/m and a relative permittivity $\varepsilon_r$ of 3.0. Then, we iteratively compute channel frequency responses and update the material parameters using gradient descent until convergence. Fig.~\ref{fig:training_curves} shows the evolution of the learned radio materials during training. After around \num{100} iterations, the learned materials have converged to the targeted values indicated by dashed lines. Note that the convergence for wood is rather slow because only very few paths interact with objects made of it. We also do not show results for metal, as in our example there is no reflected path coming from a rooftop.

A few remarks are in order: First, our approach of training on channel frequency responses is unlikely to work well with measured data, as accurate phase information can rarely be predicted via ray tracing. A more practical approach is to train on coverage maps. This is an open field of research and many other options are possible.
Second, we have assumed that many objects in the scene share the same material. In practice, this may not be the case and one may need to assign a different material to each object or groups of objects instead. In turn, this will increase the training complexity. Third, only specular reflections are considered. Studying the impact of other propagation effects on the learning process, such as refraction, diffraction, and scattering, are interesting future investigations enabled by Sionna RT.

\begin{figure}[!t]
    \vspace{-8.5pt}
    \centering
    \begin{subfigure}{\columnwidth}
        \centering
        \includegraphics[width=\columnwidth]{figures/permittivity.png}
    \end{subfigure}
    \begin{subfigure}{\columnwidth}
        \vspace{-8.5pt}
        \centering
        \includegraphics[width=\columnwidth]{figures/conductivity.png}
    \end{subfigure}
    \caption{Training curves for radio materials.}
    \label{fig:training_curves}
\end{figure}


In practice, the data set for training may be obtained from geolocalized channel measurements. Note that the goal is not necessarily to obtain the correct material properties, but rather those that achieve the best prediction of the actually measured \glspl{CIR} in combination with the applied ray tracing algorithm. One may think about this as a sort of calibration of the ray tracer and environment model with respect to measurements.
Rather than optimizing material properties which are then used by physical models in the ray tracer to predict radio propagation, one may try to directly learn the interactions between rays and scene objects. A first exploration in this direction has been undertaken in \cite{winert}.

\subsection{Optimization of Transmitter Orientation}
Our second example application shows how differentiable ray tracing can be used to optimize the orientation of a transmit array in order to maximize the average received signal power in a specific region of the scene. The same approach may be used to optimize the joint orientation of multiple transmitters to maximize the weighted sum of \glspl{SINR} in different regions of the scene, or to optimize antenna geometries, precoding codebooks, configurations of \glspl{RIS}, as well as many other system parameters with respect to various metrics. It is also possible to solve inverse problems by gradient-based methods, such as determining a user's position based on measured \gls{CIR}.

In this example, we place a transmitter on a building within the example scene \texttt{sionna.rt.scene.etoile} and compute a coverage map for a small region located behind the Arc de Triomphe (indicated by an orange ring). Then gradients of the average received power in this region with respect to the orientation of the transmitter are computed and used to optimize the latter via gradient ascent. Fig.~\ref{fig:orientation} shows the coverage map of the full scene before and after the optimization. Note that such optimization problems may become quickly non-convex and one would need to resort to other methods, such as Bayesian learning \cite{maggi2021bayesian}.

\begin{figure}
    \centering
    \begin{subfigure}{\columnwidth}
        \centering
        \includegraphics[width=0.9\textwidth]{figures/learned_orientation_start.png}
        \caption{Before optimization}
        \label{fig:orientation_a}
    \end{subfigure}
    \begin{subfigure}{\columnwidth}
        \centering
        \includegraphics[width=0.9\textwidth]{figures/learned_orientation_end.png}
        \caption{After optimization}
        \label{fig:orientation_b}
    \end{subfigure}
    \caption{Gradient-based optimization of the orientation of a transmitter (see the inset) with respect to the average received power within a small region of the scene (orange ring).}
    \label{fig:orientation}
\end{figure}
