\documentclass[%
 reprint,
superscriptaddress,
%groupedaddress,
%unsortedaddress,
%runinaddress,
%frontmatterverbose, 
%preprint,
%preprintnumbers,
%nofootinbib,
%nobibnotes,
%bibnotes,
 amsmath,amssymb,aps,
 prl,
%pra,
%prb,
%rmp,
%prstab,
%prstper,
%floatfix,
]{revtex4-2}

\usepackage{graphicx}% Include figure files
\usepackage{dcolumn}% Align table columns on decimal point
\usepackage{bm}% bold math
\usepackage{xcolor}% color text
\usepackage{amsmath,amssymb}
\usepackage{float}
\usepackage{mathtools}

\begin{document}

\title{Multistable topological  edge states in double-wave chains of polariton condensates}

\author{Tobias Schneider}
 \affiliation{Department of Physics and Center for Optoelectronics and Photonics Paderborn (CeOPP), Paderborn University, 33098 Paderborn, Germany}

\author{Wenlong Gao}
 \affiliation{EIT Institute for Advanced Study, Ningbo, China}

\author{Thomas Zentgraf}
\affiliation{Department of Physics and Center for Optoelectronics and Photonics Paderborn (CeOPP), Paderborn University, 33098 Paderborn, Germany}%
\affiliation{Institute for Photonic Quantum Systems (PhoQS), Paderborn University, 33098 Paderborn, Germany}

\author{Stefan Schumacher}
\affiliation{Department of Physics and Center for Optoelectronics and Photonics Paderborn (CeOPP), Paderborn University, 33098 Paderborn, Germany}%
\affiliation{Institute for Photonic Quantum Systems (PhoQS), Paderborn University, 33098 Paderborn, Germany}
\affiliation{Wyant College of Optical Sciences, University of Arizona, Tucson, AZ 85721, USA}%

\author{Xuekai Ma}
\affiliation{Department of Physics and Center for Optoelectronics and Photonics Paderborn (CeOPP), Paderborn University, 33098 Paderborn, Germany}%


\begin{abstract}
Topological edge states have been widely investigated in different types of lattices. In the present work, we report on topological edge states in double-wave potential chains, which can be described by a generalized Aubry-Andr{\' e}-Harper (AAH) model, applied to a driven-dissipative exciton polariton system. We show that in such potential chains, different types of edge states can form. For resonant optical excitation, the nonlinearity resulting from the repulsive polariton-polariton interaction leads to a multistability of different edge states that are stabilized for the same set of system and excitation parameters. This includes topologically protected edge states evolved directly from the individual linear eigenmodes of the system as well as additional edge states that originate from the localization of bulk states in the presence of nonlinearity. Generally, the interplay of topological localization and nonlinearity opens up exciting prospects for functional topological structures.
\end{abstract}

\maketitle

\textit{Introduction} -- Recent years have seen a surge on the study of phases of matter classified as topological in various physical platforms~\cite{xiao2010berry,hasan2010colloquium,qi2011topological,lu2014topological}. Despite being originally conceived in condensed matter solids for electrons, topological states have been reported in various photonic systems such as photonic crystals, coupled waveguides, metamaterials, optomechanics, and exciton-polaritons~\cite{ozawa2019topological}. Such topological states have shown prospect for bringing the coveted topologically protected robustness into various applications, for instance unidirectional signal propagation~\cite{haldane2008possible, wang2009observation}, optical communication, single mode lasing~\cite{st2017lasing}, vertical-cavity surface-emitting lasers (VCSELs)~\cite{yang2022topological}, quantum information~\cite{PhysRevLett.94.166802,alicea2011non}, just to name a few. Morover, photonic techniques were used to further extend the realm of topological states, some of which are not readily accessible in condensed matter systems, for instance Floquet topological insulators \cite{Rechtsman2013}, synthetic space topological insulators \cite{zilberberg2018photonic,lohse2018exploring}, Thouless pumping \cite{Citro2023}, and nonlinear topological insulators \cite{maczewsky2020nonlinearity}.     

Besides the study of novel types of topological structures, one aspect that deserves further attention is the intricate interplay between topology and nonlinearity. One platform of significant current interest in which both topology and nonlinearity can be conveniently investigated and controlled are so-called exciton polaritons. Exciton polaritons are partial-light partial-matter quasi-particles, originating from the strong coupling of excitons and photons in (planar) microresonators~\cite{kavokin2017microcavities}. The hybrid nature gives rise to a strong polariton-polariton interaction, which is one of the main factors for the realization of polariton condensation~\cite{deng2002condensation,kasprzak2006bose}, nonlinear functional polaritonic elements \cite{Luo2023,Zasedatelev2019,ma2020realization, luk2021all}, and enables the nonlinear control and trapping of macroscopically coherent polariton states in optically induced potential landscapes~\cite{schmutzler2015, schneider2016exciton}. In tailored lattice structures, different topological edge states can be realized in the non-equilibrium polariton system such as in honeycomb lattices~\cite{klembt2018exciton,milicevic2015edge,Ma:20}, Lieb lattices~\cite{PhysRevLett.120.097401,PhysRevB.97.081103}, kagome lattices~\cite{gulevich2017exploring,PhysRevB.94.115437}, and Su–Schrieffer–Heeger (SSH) chains~\cite{st2017lasing,su2021optical,harder2021coherent}. It was also reported that the strong polariton nonlinearity can induce novel topological phenomena such as multistable vortices~\cite{ma2020realization,PhysRevLett.121.227404}, topologically protected solitons~\cite{Kartashov:16,gulevich2017exploring}, gap solitons~\cite{pernet2022gap}, coupling between corner modes in higher-order polariton insulators~\cite{PhysRevLett.124.063901}, bistable topological insulators~\cite{PhysRevLett.119.253904,zhang2019finite}, and the double-sided skin effect~\cite{PhysRevB.103.235306}.

One prominent finding in quasi one-dimensional (1D) topological systems is the characteristic 0D localization of states on the edges. To systematically obtain such topological states in 1D systems, one possible route is by referring to the periodic table that classifies the ten discrete symmetry classes that coincide with the Altland-Zirnbauer classification of random matrices~\cite{chiu2016classification}. For instance the topological group of the BDI class is $\mathbb{Z}$, corresponding to a winding number in the first homotopy group. Such states, however, often find obstacles in their realizations in photonic systems, due to their stringent symmetry requirements. A more feasible realization is the renowned SSH model, in which the introduction of both chiral and geometrical mirror symmetry endows a finer topological class beyond the Altland-Zirnbauer table. With its generalizations, the SSH model has been extensively studied in various photonic systems \cite{blanco2016topological,st2017lasing,zhao2018topological,parto2018edge,han2019lasing}. Finally, a less intensely studied but not less promising route is the Aubry-Andr{\' e}-Harper (AAH) model~\cite{aubry1980analyticity} which translates into a Chern class in  synthetic space~\cite{ganeshan2013topological} and that has not been investigated for polariton systems with their significant nonlinearity.  

\begin{figure*}[t]
  \centering
   \includegraphics[width=2.0\columnwidth]{fig1.png}
  \caption{{\bf Edge states in double-wave chains.} (a) Double-wave potential chain with $a=3$ $\mu$m, $A=2$ $\mu$m, and wave modulation amplitude $d=0.5$ $\mu$m. (b) Dependence of the energies (relative to the polariton dispersion minimum) of the linear eigenmodes in the double-wave chain on the inter-wave separation $A$ with $a=3$ $\mu$m and $d=0.5$ $\mu$m fixed. Red lines indicate the edge states and black lines are the bulk states. (c-f) Amplitude (left) and phase (right) distributions of the edge states marked in (b).
  }
  \label{fig:1}
\end{figure*}

In this work, we introduce and investigate a class of lattices to realize topological edge states: double-wave chains as sketched in Fig.~1(a), which can be described in the framework of a coupled AAH model, i.e., cosinusoidal modulated potentials. We find that the edge states in such chains become clearly distinguishable from the bulk states when the intra-wave separation between the potential wells is larger than the separation of the two waves, i.e., the inter-wave separation. By changing the specific pattern of the waves, different edge states can be obtained, and they can even appear at different edges. Besides these edge states intrinsic to the structure, we find that the nonlinearity can transform specific bulk states into states that are localized at the edges. Moreover, in the nonlinear regime, the eigenstates on the edge with different phase distributions can be excited simultaneously, resulting in a final state where asymmetric edge states are stabilized by the nonlinearity. Several of these edge states can be stabilized for the same optical pumping parameters, which offers potential applications of this multistability in all-optical switching. It is important to emphasize that the topological structure and states introduced are of very general nature and can be realized in different physical implementations.

\begin{figure*}[htp]
  \centering
   \includegraphics[width=2\columnwidth]{fig2.png}
  \caption{{\bf Edge states in differently structured double-wave chains.} (a) Dependence of the energies of the linear eigenmodes on the phase shift $\theta$ of the chains, with $a=3$ $\mu$m, $A=2$ $\mu$m, and $d=0.5$ $\mu$m. Red lines indicate the right edge states, green lines indicate the left edge states, and black lines are the bulk states. (b) Enlarged view of the selected area in (a). (c-e) Different double-wave chain structures (upper panels) and amplitudes [same colorbar as in Fig.~1(c-f)] of the corresponding selected edge states (lower panels) at (c) $\theta=1.75\pi$, (d) $\theta=1.5\pi$, and (e) $\theta=\pi$.
  }
  \label{fig:2}
\end{figure*}

\textit{Theoretical model} -- The low-energy resonant optical excitations of polariton condensates can be described by a generalized Gross-Pitaevskii equation with loss and gain, i.e.,
\begin{equation}
\begin{aligned}
\label{GP_psi}
    i\hbar\frac{\partial \psi(\textbf{r},t)}{\partial t} =&\left[-\frac{\hbar^2}{2m}\nabla^2-i\hbar\frac{\gamma}{2}+g|\psi(\textbf{r},t)|^2+V(\textbf{r})\right]\psi(\textbf{r},t) \\
    \\
    &+E(\textbf{r},t)\,.
\end{aligned}
\end{equation}
Here, $\psi(\textbf{r},t)$ is the polariton wavefunction, $m=10^{-4}m_\text{e}$ (with free electron mass $m_\text{e}$) is the effective polariton mass, $\gamma$ is the loss rate in quasi-mode approximation \cite{Carcamo:20}, and $g$ is the nonlinearity coefficient representing the strength of the polariton interaction. $E(\textbf{r},t)$ is a coherent pump for the creation of polaritons. $V(\textbf{r})$ is the external potential, which can be fabricated in semiconductor microcavities in different ways \cite{schneider2016exciton}, and satisfies
\begin{equation}
\begin{aligned}
\label{potential}
    V(x,y)={\sum_n}V_0 e^{-\left(\frac{(x-x_n)^2+(y\pm y_n)^2}{w^2}\right)^{10}}.
\end{aligned}
\end{equation}
Here, $V_0=-5$ meV is the potential depth, $n$ is the index labelling the potential wells, and $w=1$ $\mu$m is the radius of each potential well. In this work we study the edge states in a double-wave chain with a finite number of potential wells that are distributed at $(x_n, {\pm}y_n)$ with $n\in[1,N]$. The positions of the potential wells are given by 
\begin{equation}
\begin{aligned}
\label{pillars}
    x_n=a(n-1), \ \ y_n=A+d-d\sin\left((n-1)\frac{\pi}{2}+\theta\right)
\end{aligned}
\end{equation}
Here, $a$ is the separation of the neighbouring wells along the horizontal ($x$) direction, i.e., intra-wave separation, $A$ represents the separation of the closest wells in the upper and lower waves, i.e., inter-wave separation, and $d$ is the amplitude of each modulated wave. Figure \ref{fig:1}(a) shows the potential distribution with the wave constant $4a$, i.e., each four potential wells form a wave unit, and four periods along $x$ direction with a total of 16 potential wells. The separation of the upper and lower waves, center to center, is $A+2d$. The phase shift $\theta$ ($\in[0, 2\pi]$) in Eq.~\eqref{pillars} indicates the phase of the first (leftmost) well in the modulation and $\theta=0$ in Fig. \ref{fig:1}(a). All the parameters we choose are based on typical experiments in GaAs-based semiconductor microcavities~\cite{schneider2016exciton}.

Heuristically, our double-wave chain can be approximated by a tight-binding (TB) model with the Hamiltonian
\begin{equation}
\label{chain TBM}
\begin{multlined}
    \hat{H}=\sum_{j}^{N-1} (J_{j,j+1}A_j^{\dag}A_{j+1}+J_{j,j+1}B_j^{\dag}B_{j+1}+h.c.)\\
    +\sum_{j}^{N} (t_{j,j} A_j^{\dag}B_{j}+h.c.)
\end{multlined}
\end{equation}
Here only the fundamental mode of each well and their nearest neighbour couplings are considered. $X_j^{\dag} (X_j)$ is the creation (annihilation) operator on the $j$th site ($X=A,B$) and $A (B)$ corresponds to the upper (lower) chain. Furthermore, the Hamiltonian $\hat{H}$ has mirror exchange symmetry with the corresponding exchange operator $\hat{M}$ defined as $\hat{M}\psi=\psi'=\begin{pmatrix} B_j\\ A_j\end{pmatrix}$. The Hamiltonian can be block-diagonalized onto the basis of $\hat{M}$ under the transformation $\mathbf{U}\hat{H}\mathbf{U}^{-1}=\hat{H}_{BD}$, in which $\hat{H}_{BD}=\begin{pmatrix}h_+ & 0 \\ 0 & h_-\end{pmatrix}$. 
Finally, if we assume the inter-wave couplings to take the form $t_{j,j}=t+\lambda \cos(2\pi \beta j+\theta)$ where $\beta =p/q$ is a rational parameter ($p$ and $q$ are co-prime), then $h_{+(-)}$ is inscribed as an AAH model, while $\hat{H}_{BD}$ governs two uncoupled AAH models with oppositely signed modulation. The Chern index associated with each energy band can be defined in the synthetic 2D parameter space $(\theta, k_x)$ for each mirror subspace as $C_{\pm,n} \mbox{ for } n=1,...,N$. Here $+(-)$ refers to the even(odd) subspace and n is the index of bands whose energy are arranged in ascending order. Owning to the reversely signed modulation between the two subspaces, the relation $C_{+,n}=C_{-,N-n+1}$ is satisfied (details in Supplemental Material). Considering an individual wave unit with $N=4$, we have $C_{+,n}=C_{-,N-n+1}=\{1,1,-1,-1 \}$.

\textit{Eigenmodes and edge states} -- We start by studying the linear eigenmodes of the double-wave chain. The eigenvalue problem can be solved by substituting the ansatz $\psi(\textbf{r},t)=\psi(\textbf{r})e^{-i{\omega}t}$ into Eq.~\eqref{GP_psi} and neglecting the loss, gain, and nonlinearity. From the eigenenergies in Fig.~\ref{fig:1}(b), one can see that the edge states (red lines) become more isolated from the bulk states (black lines) when the intra-wave separation $a$ is larger than the inter-wave separation $A$, akin to the appearance of the edge states in SSH chains when the intra-cell coupling is larger than the inter-cell coupling~\cite{RevModPhys.60.781}. In this configuration, all the edge states are located at the right edge of the chain. For each edge state four potential wells are occupied, and there are in total four different types of edge states as shown in Fig.~\ref{fig:1}(c-f). These edge states can be grouped into two subgroups, i.e., 0-states [Fig.~\ref{fig:1}(d,f)] and $\pi$-states [Fig.~\ref{fig:1}(c,e)]. For 0-states the phases in the equivalent wells in the two waves are the same, while for $\pi$-states there is a $\pi$-phase difference between them. In each group, there are two edge states where one shows the same phase for the last two wells in the same wave [Fig.~\ref{fig:1}(e,f)] and the other shows a $\pi$-phase jump [Fig.~\ref{fig:1}(c,d)]. Therefore, we denote the four edge states as 0-0 state [Fig.~\ref{fig:1}(f)], $\pi$-0 state [Fig.~\ref{fig:1}(e)], 0-$\pi$ state [Fig.~\ref{fig:1}(d)], and $\pi$-$\pi$ state [Fig.~\ref{fig:1}(c)]. Besides the separation of the two waves, the properties of the edge states are also related to the modulation amplitude of the waves $d$. As $d$ decreases, the inter-wave interaction in the whole chain becomes stronger, leading to edge states that are less distinguishable from the bulk states (see Fig.~S1 in the SM). Such edge states hold even if more integer potential periods are added to the left or right side of the chain. While we note that our study shows representative results for a specific choice of system parameters, band gaps can be broadened significantly by increasing the potential depth and hopping between adjacent potential wells can be enhanced by moving them closer together~\cite{schneider2016exciton}.

\begin{figure}[t]
  \centering
   \includegraphics[width=1.0\columnwidth]{fig3.png}
  \caption{\textbf{Nonlinearity and multistable edge states.} (a) Dependence of the peak densities of the edge states on the amplitude of the continuous wave, spatially homogeneous pump with photon energy $\hbar\omega=-3.5995$ meV [indicated by the red square in Fig.~\ref{fig:1}(b) for $a=3$ $\mu$m, $A=2.5$ $\mu$m, $d=0.5$ $\mu$m, and $\theta=0$]. (b-d) Density profiles of multistable edge states corresponding to the points marked in panel (a). (e) Normalized peak densities in each potential well from the upper wave of the state in (d) for different nonlinear coefficients $g$.
  }
  \label{fig:3}
\end{figure}

The reason why the edge states in Fig.~\ref{fig:1} appear at the right edge rather than the left one is the specific arrangement of the potential wells. How the potential wells are arranged determines which edge (left or right or both) is occupied as illustrated in Fig.~\ref{fig:2} in which the phase shift $\theta$ is systematically varied in panels (a) and (b) from zero to 2$\pi$. When $\theta=1.75\pi$, the chain becomes symmetric along $x$ direction [Fig.~\ref{fig:2}(c)], and consequently both edges are occupied. In this case there appear two pairs of degenerate edge states [cf. Fig.~\ref{fig:2}(b)], because the wavefunctions at the two edges are spatially well separated from each other with virtually zero overlap, resulting in the degeneracy of the 0-0 and 0-$\pi$ states and $\pi$-0 and $\pi$-$\pi$ states. When $\theta=1.5\pi$ [Fig.~\ref{fig:2}(d)], the chain is spatially reversed in $x$ direction, compared with the one in Fig.~\ref{fig:1}(a). Under this condition the four edge states relocate to the left edge of the chain. When the chain is in the configuration shown in Fig.~\ref{fig:2}(e) where $\theta=\pi$, only two edge states that are localized in the rightmost wells appear. From the results presented in Fig.~\ref{fig:2} (see also the extended version of this figure, Fig.~S2), it is clear that the edge states arise only when the edge wells are not at the exact antinodes of the double wave structure, i.e., only when $\sin\left((n-1)\frac{\pi}{2}+\theta\right)\neq0$ in Eq.~\eqref{fig:3}. That is why the left edge in Fig.~\ref{fig:1} and Fig.~\ref{fig:2}(e) and the right edge in Fig.~\ref{fig:2}(d) are not occupied. The appearance of the edge states shown in Fig. 2 is consistent with the inscribed Chern number calculated from the tight-binding model \eqref{chain TBM} and thus the bulk-edge correspondence \cite{hatsugai2016bulk}. The edge states can also be reconstructed by directly adding or cutting specific potential wells at the left or right side of the chain in Fig.~\ref{fig:1}(a) (see Fig.~S3). 

\textit{Multistable edge states} -- To optically excite the edge states and study their nonlinear dynamics, a resonant continuous-wave (CW) pump $E(\textbf{r},t)=E_0{e^{-i{\omega}t}}$, where $E_0$ is the constant amplitude and $\omega$ is the frequency, is used to continuously drive the system until the stationary solution is reached. In our calculations, the polaritons have a loss rate of $\gamma=0.005$ ps$^{-1}$ and a nonlinear coefficient of $g=2$ $\mu$eV $\mu$m$^{2}$. The time evolution of Eq.~(1) is computed by using a 4th order Runge-Kutta method with zero boundary conditions. The stability of the stationary solutions is tested by adding white noise into the solutions and letting them evolve over a long time (10 ns).

From Fig.~\ref{fig:1}(b), one can see that two edge states are degenerate at around $a/A=1.17$. Therefore, we choose the inter-wave separation of the chain at $a/A=1.2$ , where the 0-$\pi$ and $\pi$-$\pi$ states are very close to each other, and tune the photon energy of the pump to $\hbar\omega=-3.5995$ meV (see the red square in Fig.~\ref{fig:1}(b)). In this scenario, because of the finite polariton lifetime and nonlinearity, the 0-$\pi$ and $\pi$-$\pi$ states can be simultaneously excited, and their phase difference leads to destructive interference of the condensate density in the upper wave [see the density at the right edge in Fig.~\ref{fig:3}(c)]. Meanwhile, the left edge is also occupied by the condensate; however, this state is not one of the linear eigenmodes as known from Fig.~\ref{fig:1}. This edge state can even be separately excited as shown in Fig.~\ref{fig:3}(d) especially when the pump intensity is lower. To illustrate this point further, we use the left-edge state shown in Fig.~\ref{fig:3}(d) as the initial condition and gradually decrease the nonlinear coefficient while keeping the pump fixed. It is clearly seen from Fig.~\ref{fig:3}(e) that the nonlinearity strongly affects this edge state. With the decrease of the nonlinearity, the condensate starts to move to the bulk of the system, forming a progressively decreasing density distribution from left to right [the blue line in Fig.~\ref{fig:3}(e)], which is similar to the linear eigenmodes with frequency below the 0-$\pi$ edge state (see Fig.~S4). In conclusion the nonlinearity pushes the bulk states to the edge, leading to the formation of nonlinearity-enhanced edge states, analogous to surface gap solitons~\cite{PhysRevLett.96.073901}. As the pump intensity increases [Fig.~\ref{fig:3}(a,b)], the right edge can be fully excited, which means only one of the intrinsic edge states is excited here. If we excite the edge state in the lattice with $a/A=1.5$, where the edge states are more isolated from the bulk states, by choosing the photon energy of the pump between the 0-$\pi$ and $\pi$-$\pi$ states, it is possible that the condensate is loaded only into the wells at the right edge, leaving the potential wells at the left edge empty [see Fig.~S5]. From Fig.~\ref{fig:3}(a), one can see that all these edge states are stable in a broad pump intensity range but at different peak densities. We note that this finding of edge-state multistability is of general nature in our structure as it can occur at different lattice constants and pump frequencies according to the results shown in Fig.~S5.

\textit{Conclusion} -- In summary, we report on different topological edge states in polaritonic double-wave chains. Under specific pumping conditions more than one edge state can be stabilized, including the combination of different intrinsic edge states as well as nonlinearity-enhanced ones. These states can appear at different edges for differently structured double-wave potentials. Although implemented in a specific system, our theoretical model and main results are of general nature and similar observations of topologically protected edge states in double-wave potentials and their multistability are expected in other physical systems such as atomic condensates and nonlinear optical systems, in which topological edge states have been widely studied in  other types of lattices.

\begin{acknowledgments}
This work was supported by the Deutsche Forschungsgemeinschaft (DFG) (Grant No. 467358803 and No. 519608013) and by the Paderborn Center for Parallel Computing, PC$^2$.
\end{acknowledgments}

%\bibliography{refs}
%apsrev4-2.bst 2019-01-14 (MD) hand-edited version of apsrev4-1.bst
%Control: key (0)
%Control: author (8) initials jnrlst
%Control: editor formatted (1) identically to author
%Control: production of article title (0) allowed
%Control: page (0) single
%Control: year (1) truncated
%Control: production of eprint (0) enabled
\begin{thebibliography}{52}%
\makeatletter
\providecommand \@ifxundefined [1]{%
 \@ifx{#1\undefined}
}%
\providecommand \@ifnum [1]{%
 \ifnum #1\expandafter \@firstoftwo
 \else \expandafter \@secondoftwo
 \fi
}%
\providecommand \@ifx [1]{%
 \ifx #1\expandafter \@firstoftwo
 \else \expandafter \@secondoftwo
 \fi
}%
\providecommand \natexlab [1]{#1}%
\providecommand \enquote  [1]{``#1''}%
\providecommand \bibnamefont  [1]{#1}%
\providecommand \bibfnamefont [1]{#1}%
\providecommand \citenamefont [1]{#1}%
\providecommand \href@noop [0]{\@secondoftwo}%
\providecommand \href [0]{\begingroup \@sanitize@url \@href}%
\providecommand \@href[1]{\@@startlink{#1}\@@href}%
\providecommand \@@href[1]{\endgroup#1\@@endlink}%
\providecommand \@sanitize@url [0]{\catcode `\\12\catcode `\$12\catcode
  `\&12\catcode `\#12\catcode `\^12\catcode `\_12\catcode `\%12\relax}%
\providecommand \@@startlink[1]{}%
\providecommand \@@endlink[0]{}%
\providecommand \url  [0]{\begingroup\@sanitize@url \@url }%
\providecommand \@url [1]{\endgroup\@href {#1}{\urlprefix }}%
\providecommand \urlprefix  [0]{URL }%
\providecommand \Eprint [0]{\href }%
\providecommand \doibase [0]{https://doi.org/}%
\providecommand \selectlanguage [0]{\@gobble}%
\providecommand \bibinfo  [0]{\@secondoftwo}%
\providecommand \bibfield  [0]{\@secondoftwo}%
\providecommand \translation [1]{[#1]}%
\providecommand \BibitemOpen [0]{}%
\providecommand \bibitemStop [0]{}%
\providecommand \bibitemNoStop [0]{.\EOS\space}%
\providecommand \EOS [0]{\spacefactor3000\relax}%
\providecommand \BibitemShut  [1]{\csname bibitem#1\endcsname}%
\let\auto@bib@innerbib\@empty
%</preamble>
\bibitem [{\citenamefont {Xiao}\ \emph {et~al.}(2010)\citenamefont {Xiao},
  \citenamefont {Chang},\ and\ \citenamefont {Niu}}]{xiao2010berry}%
  \BibitemOpen
  \bibfield  {author} {\bibinfo {author} {\bibfnamefont {D.}~\bibnamefont
  {Xiao}}, \bibinfo {author} {\bibfnamefont {M.-C.}\ \bibnamefont {Chang}},\
  and\ \bibinfo {author} {\bibfnamefont {Q.}~\bibnamefont {Niu}},\ }\bibfield
  {title} {\bibinfo {title} {{B}erry phase effects on electronic properties},\
  }\href@noop {} {\bibfield  {journal} {\bibinfo  {journal} {Rev. Mod. Phys.}\
  }\textbf {\bibinfo {volume} {82}},\ \bibinfo {pages} {1959} (\bibinfo {year}
  {2010})}\BibitemShut {NoStop}%
\bibitem [{\citenamefont {Hasan}\ and\ \citenamefont
  {Kane}(2010)}]{hasan2010colloquium}%
  \BibitemOpen
  \bibfield  {author} {\bibinfo {author} {\bibfnamefont {M.~Z.}\ \bibnamefont
  {Hasan}}\ and\ \bibinfo {author} {\bibfnamefont {C.~L.}\ \bibnamefont
  {Kane}},\ }\bibfield  {title} {\bibinfo {title} {Colloquium: topological
  insulators},\ }\href@noop {} {\bibfield  {journal} {\bibinfo  {journal} {Rev.
  Mod. Phys.}\ }\textbf {\bibinfo {volume} {82}},\ \bibinfo {pages} {3045}
  (\bibinfo {year} {2010})}\BibitemShut {NoStop}%
\bibitem [{\citenamefont {Qi}\ and\ \citenamefont
  {Zhang}(2011)}]{qi2011topological}%
  \BibitemOpen
  \bibfield  {author} {\bibinfo {author} {\bibfnamefont {X.-L.}\ \bibnamefont
  {Qi}}\ and\ \bibinfo {author} {\bibfnamefont {S.-C.}\ \bibnamefont {Zhang}},\
  }\bibfield  {title} {\bibinfo {title} {Topological insulators and
  superconductors},\ }\href@noop {} {\bibfield  {journal} {\bibinfo  {journal}
  {Rev. Mod. Phys.}\ }\textbf {\bibinfo {volume} {83}},\ \bibinfo {pages}
  {1057} (\bibinfo {year} {2011})}\BibitemShut {NoStop}%
\bibitem [{\citenamefont {Lu}\ \emph {et~al.}(2014)\citenamefont {Lu},
  \citenamefont {Joannopoulos},\ and\ \citenamefont
  {Solja{\v{c}}i{\'c}}}]{lu2014topological}%
  \BibitemOpen
  \bibfield  {author} {\bibinfo {author} {\bibfnamefont {L.}~\bibnamefont
  {Lu}}, \bibinfo {author} {\bibfnamefont {J.~D.}\ \bibnamefont
  {Joannopoulos}},\ and\ \bibinfo {author} {\bibfnamefont {M.}~\bibnamefont
  {Solja{\v{c}}i{\'c}}},\ }\bibfield  {title} {\bibinfo {title} {Topological
  photonics},\ }\href@noop {} {\bibfield  {journal} {\bibinfo  {journal} {Nat.
  Photon.}\ }\textbf {\bibinfo {volume} {8}},\ \bibinfo {pages} {821} (\bibinfo
  {year} {2014})}\BibitemShut {NoStop}%
\bibitem [{\citenamefont {Ozawa}\ \emph {et~al.}(2019)\citenamefont {Ozawa},
  \citenamefont {Price}, \citenamefont {Amo}, \citenamefont {Goldman},
  \citenamefont {Hafezi}, \citenamefont {Lu}, \citenamefont {Rechtsman},
  \citenamefont {Schuster}, \citenamefont {Simon}, \citenamefont {Zilberberg},\
  and\ \citenamefont {Carusotto}}]{ozawa2019topological}%
  \BibitemOpen
  \bibfield  {author} {\bibinfo {author} {\bibfnamefont {T.}~\bibnamefont
  {Ozawa}}, \bibinfo {author} {\bibfnamefont {H.~M.}\ \bibnamefont {Price}},
  \bibinfo {author} {\bibfnamefont {A.}~\bibnamefont {Amo}}, \bibinfo {author}
  {\bibfnamefont {N.}~\bibnamefont {Goldman}}, \bibinfo {author} {\bibfnamefont
  {M.}~\bibnamefont {Hafezi}}, \bibinfo {author} {\bibfnamefont
  {L.}~\bibnamefont {Lu}}, \bibinfo {author} {\bibfnamefont {M.~C.}\
  \bibnamefont {Rechtsman}}, \bibinfo {author} {\bibfnamefont {D.}~\bibnamefont
  {Schuster}}, \bibinfo {author} {\bibfnamefont {J.}~\bibnamefont {Simon}},
  \bibinfo {author} {\bibfnamefont {O.}~\bibnamefont {Zilberberg}},\ and\
  \bibinfo {author} {\bibfnamefont {I.}~\bibnamefont {Carusotto}},\ }\bibfield
  {title} {\bibinfo {title} {Topological photonics},\ }\href
  {https://doi.org/10.1103/RevModPhys.91.015006} {\bibfield  {journal}
  {\bibinfo  {journal} {Rev. Mod. Phys.}\ }\textbf {\bibinfo {volume} {91}},\
  \bibinfo {pages} {015006} (\bibinfo {year} {2019})}\BibitemShut {NoStop}%
\bibitem [{\citenamefont {Haldane}\ and\ \citenamefont
  {Raghu}(2008)}]{haldane2008possible}%
  \BibitemOpen
  \bibfield  {author} {\bibinfo {author} {\bibfnamefont {F.~D.~M.}\
  \bibnamefont {Haldane}}\ and\ \bibinfo {author} {\bibfnamefont
  {S.}~\bibnamefont {Raghu}},\ }\bibfield  {title} {\bibinfo {title} {Possible
  realization of directional optical waveguides in photonic crystals with
  broken time-reversal symmetry},\ }\href@noop {} {\bibfield  {journal}
  {\bibinfo  {journal} {Phys. Rev. Lett.}\ }\textbf {\bibinfo {volume} {100}},\
  \bibinfo {pages} {013904} (\bibinfo {year} {2008})}\BibitemShut {NoStop}%
\bibitem [{\citenamefont {Wang}\ \emph {et~al.}(2009)\citenamefont {Wang},
  \citenamefont {Chong}, \citenamefont {Joannopoulos},\ and\ \citenamefont
  {Solja{\v{c}}i{\'c}}}]{wang2009observation}%
  \BibitemOpen
  \bibfield  {author} {\bibinfo {author} {\bibfnamefont {Z.}~\bibnamefont
  {Wang}}, \bibinfo {author} {\bibfnamefont {Y.}~\bibnamefont {Chong}},
  \bibinfo {author} {\bibfnamefont {J.~D.}\ \bibnamefont {Joannopoulos}},\ and\
  \bibinfo {author} {\bibfnamefont {M.}~\bibnamefont {Solja{\v{c}}i{\'c}}},\
  }\bibfield  {title} {\bibinfo {title} {Observation of unidirectional
  backscattering-immune topological electromagnetic states},\ }\href@noop {}
  {\bibfield  {journal} {\bibinfo  {journal} {Nature}\ }\textbf {\bibinfo
  {volume} {461}},\ \bibinfo {pages} {772} (\bibinfo {year}
  {2009})}\BibitemShut {NoStop}%
\bibitem [{\citenamefont {St-Jean}\ \emph {et~al.}(2017)\citenamefont
  {St-Jean}, \citenamefont {Goblot}, \citenamefont {Galopin}, \citenamefont
  {Lema{\^\i}tre}, \citenamefont {Ozawa}, \citenamefont {Le~Gratiet},
  \citenamefont {Sagnes}, \citenamefont {Bloch},\ and\ \citenamefont
  {Amo}}]{st2017lasing}%
  \BibitemOpen
  \bibfield  {author} {\bibinfo {author} {\bibfnamefont {P.}~\bibnamefont
  {St-Jean}}, \bibinfo {author} {\bibfnamefont {V.}~\bibnamefont {Goblot}},
  \bibinfo {author} {\bibfnamefont {E.}~\bibnamefont {Galopin}}, \bibinfo
  {author} {\bibfnamefont {A.}~\bibnamefont {Lema{\^\i}tre}}, \bibinfo {author}
  {\bibfnamefont {T.}~\bibnamefont {Ozawa}}, \bibinfo {author} {\bibfnamefont
  {L.}~\bibnamefont {Le~Gratiet}}, \bibinfo {author} {\bibfnamefont
  {I.}~\bibnamefont {Sagnes}}, \bibinfo {author} {\bibfnamefont
  {J.}~\bibnamefont {Bloch}},\ and\ \bibinfo {author} {\bibfnamefont
  {A.}~\bibnamefont {Amo}},\ }\bibfield  {title} {\bibinfo {title} {Lasing in
  topological edge states of a one-dimensional lattice},\ }\href@noop {}
  {\bibfield  {journal} {\bibinfo  {journal} {Nat. Photon.}\ }\textbf {\bibinfo
  {volume} {11}},\ \bibinfo {pages} {651} (\bibinfo {year} {2017})}\BibitemShut
  {NoStop}%
\bibitem [{\citenamefont {Yang}\ \emph {et~al.}(2022)\citenamefont {Yang},
  \citenamefont {Li}, \citenamefont {Gao},\ and\ \citenamefont
  {Lu}}]{yang2022topological}%
  \BibitemOpen
  \bibfield  {author} {\bibinfo {author} {\bibfnamefont {L.}~\bibnamefont
  {Yang}}, \bibinfo {author} {\bibfnamefont {G.}~\bibnamefont {Li}}, \bibinfo
  {author} {\bibfnamefont {X.}~\bibnamefont {Gao}},\ and\ \bibinfo {author}
  {\bibfnamefont {L.}~\bibnamefont {Lu}},\ }\bibfield  {title} {\bibinfo
  {title} {Topological-cavity surface-emitting laser},\ }\href@noop {}
  {\bibfield  {journal} {\bibinfo  {journal} {Nat. Photon.}\ }\textbf {\bibinfo
  {volume} {16}},\ \bibinfo {pages} {279} (\bibinfo {year} {2022})}\BibitemShut
  {NoStop}%
\bibitem [{\citenamefont {Das~Sarma}\ \emph {et~al.}(2005)\citenamefont
  {Das~Sarma}, \citenamefont {Freedman},\ and\ \citenamefont
  {Nayak}}]{PhysRevLett.94.166802}%
  \BibitemOpen
  \bibfield  {author} {\bibinfo {author} {\bibfnamefont {S.}~\bibnamefont
  {Das~Sarma}}, \bibinfo {author} {\bibfnamefont {M.}~\bibnamefont
  {Freedman}},\ and\ \bibinfo {author} {\bibfnamefont {C.}~\bibnamefont
  {Nayak}},\ }\bibfield  {title} {\bibinfo {title} {Topologically protected
  qubits from a possible non-abelian fractional quantum hall state},\ }\href
  {https://doi.org/10.1103/PhysRevLett.94.166802} {\bibfield  {journal}
  {\bibinfo  {journal} {Phys. Rev. Lett.}\ }\textbf {\bibinfo {volume} {94}},\
  \bibinfo {pages} {166802} (\bibinfo {year} {2005})}\BibitemShut {NoStop}%
\bibitem [{\citenamefont {Alicea}\ \emph {et~al.}(2011)\citenamefont {Alicea},
  \citenamefont {Oreg}, \citenamefont {Refael}, \citenamefont {Von~Oppen},\
  and\ \citenamefont {Fisher}}]{alicea2011non}%
  \BibitemOpen
  \bibfield  {author} {\bibinfo {author} {\bibfnamefont {J.}~\bibnamefont
  {Alicea}}, \bibinfo {author} {\bibfnamefont {Y.}~\bibnamefont {Oreg}},
  \bibinfo {author} {\bibfnamefont {G.}~\bibnamefont {Refael}}, \bibinfo
  {author} {\bibfnamefont {F.}~\bibnamefont {Von~Oppen}},\ and\ \bibinfo
  {author} {\bibfnamefont {M.~P.}\ \bibnamefont {Fisher}},\ }\bibfield  {title}
  {\bibinfo {title} {Non-abelian statistics and topological quantum information
  processing in 1d wire networks},\ }\href@noop {} {\bibfield  {journal}
  {\bibinfo  {journal} {Nat. Phys.}\ }\textbf {\bibinfo {volume} {7}},\
  \bibinfo {pages} {412} (\bibinfo {year} {2011})}\BibitemShut {NoStop}%
\bibitem [{\citenamefont {Rechtsman}\ \emph {et~al.}(2013)\citenamefont
  {Rechtsman}, \citenamefont {Zeuner}, \citenamefont {Plotnik}, \citenamefont
  {Lumer}, \citenamefont {Podolsky}, \citenamefont {Dreisow}, \citenamefont
  {Nolte}, \citenamefont {Segev},\ and\ \citenamefont
  {Szameit}}]{Rechtsman2013}%
  \BibitemOpen
  \bibfield  {author} {\bibinfo {author} {\bibfnamefont {M.~C.}\ \bibnamefont
  {Rechtsman}}, \bibinfo {author} {\bibfnamefont {J.~M.}\ \bibnamefont
  {Zeuner}}, \bibinfo {author} {\bibfnamefont {Y.}~\bibnamefont {Plotnik}},
  \bibinfo {author} {\bibfnamefont {Y.}~\bibnamefont {Lumer}}, \bibinfo
  {author} {\bibfnamefont {D.}~\bibnamefont {Podolsky}}, \bibinfo {author}
  {\bibfnamefont {F.}~\bibnamefont {Dreisow}}, \bibinfo {author} {\bibfnamefont
  {S.}~\bibnamefont {Nolte}}, \bibinfo {author} {\bibfnamefont
  {M.}~\bibnamefont {Segev}},\ and\ \bibinfo {author} {\bibfnamefont
  {A.}~\bibnamefont {Szameit}},\ }\bibfield  {title} {\bibinfo {title}
  {Photonic {F}loquet topological insulators},\ }\href@noop {} {\bibfield
  {journal} {\bibinfo  {journal} {Nature}\ }\textbf {\bibinfo {volume} {496}},\
  \bibinfo {pages} {196} (\bibinfo {year} {2013})}\BibitemShut {NoStop}%
\bibitem [{\citenamefont {Zilberberg}\ \emph {et~al.}(2018)\citenamefont
  {Zilberberg}, \citenamefont {Huang}, \citenamefont {Guglielmon},
  \citenamefont {Wang}, \citenamefont {Chen}, \citenamefont {Kraus},\ and\
  \citenamefont {Rechtsman}}]{zilberberg2018photonic}%
  \BibitemOpen
  \bibfield  {author} {\bibinfo {author} {\bibfnamefont {O.}~\bibnamefont
  {Zilberberg}}, \bibinfo {author} {\bibfnamefont {S.}~\bibnamefont {Huang}},
  \bibinfo {author} {\bibfnamefont {J.}~\bibnamefont {Guglielmon}}, \bibinfo
  {author} {\bibfnamefont {M.}~\bibnamefont {Wang}}, \bibinfo {author}
  {\bibfnamefont {K.~P.}\ \bibnamefont {Chen}}, \bibinfo {author}
  {\bibfnamefont {Y.~E.}\ \bibnamefont {Kraus}},\ and\ \bibinfo {author}
  {\bibfnamefont {M.~C.}\ \bibnamefont {Rechtsman}},\ }\bibfield  {title}
  {\bibinfo {title} {Photonic topological boundary pumping as a probe of 4d
  quantum hall physics},\ }\href@noop {} {\bibfield  {journal} {\bibinfo
  {journal} {Nature}\ }\textbf {\bibinfo {volume} {553}},\ \bibinfo {pages}
  {59} (\bibinfo {year} {2018})}\BibitemShut {NoStop}%
\bibitem [{\citenamefont {Lohse}\ \emph {et~al.}(2018)\citenamefont {Lohse},
  \citenamefont {Schweizer}, \citenamefont {Price}, \citenamefont
  {Zilberberg},\ and\ \citenamefont {Bloch}}]{lohse2018exploring}%
  \BibitemOpen
  \bibfield  {author} {\bibinfo {author} {\bibfnamefont {M.}~\bibnamefont
  {Lohse}}, \bibinfo {author} {\bibfnamefont {C.}~\bibnamefont {Schweizer}},
  \bibinfo {author} {\bibfnamefont {H.~M.}\ \bibnamefont {Price}}, \bibinfo
  {author} {\bibfnamefont {O.}~\bibnamefont {Zilberberg}},\ and\ \bibinfo
  {author} {\bibfnamefont {I.}~\bibnamefont {Bloch}},\ }\bibfield  {title}
  {\bibinfo {title} {Exploring 4d quantum hall physics with a 2d topological
  charge pump},\ }\href@noop {} {\bibfield  {journal} {\bibinfo  {journal}
  {Nature}\ }\textbf {\bibinfo {volume} {553}},\ \bibinfo {pages} {55}
  (\bibinfo {year} {2018})}\BibitemShut {NoStop}%
\bibitem [{\citenamefont {Citro}\ and\ \citenamefont
  {Aidelsburger}(2023)}]{Citro2023}%
  \BibitemOpen
  \bibfield  {author} {\bibinfo {author} {\bibfnamefont {R.}~\bibnamefont
  {Citro}}\ and\ \bibinfo {author} {\bibfnamefont {M.}~\bibnamefont
  {Aidelsburger}},\ }\bibfield  {title} {\bibinfo {title} {Thouless pumping and
  topology},\ }\href@noop {} {\bibfield  {journal} {\bibinfo  {journal} {Nat.
  Rev. Phys.}\ }\textbf {\bibinfo {volume} {5}},\ \bibinfo {pages} {87}
  (\bibinfo {year} {2023})}\BibitemShut {NoStop}%
\bibitem [{\citenamefont {Maczewsky}\ \emph {et~al.}(2020)\citenamefont
  {Maczewsky}, \citenamefont {Heinrich}, \citenamefont {Kremer}, \citenamefont
  {Ivanov}, \citenamefont {Ehrhardt}, \citenamefont {Martinez}, \citenamefont
  {Kartashov}, \citenamefont {Konotop}, \citenamefont {Torner}, \citenamefont
  {Bauer} \emph {et~al.}}]{maczewsky2020nonlinearity}%
  \BibitemOpen
  \bibfield  {author} {\bibinfo {author} {\bibfnamefont {L.~J.}\ \bibnamefont
  {Maczewsky}}, \bibinfo {author} {\bibfnamefont {M.}~\bibnamefont {Heinrich}},
  \bibinfo {author} {\bibfnamefont {M.}~\bibnamefont {Kremer}}, \bibinfo
  {author} {\bibfnamefont {S.~K.}\ \bibnamefont {Ivanov}}, \bibinfo {author}
  {\bibfnamefont {M.}~\bibnamefont {Ehrhardt}}, \bibinfo {author}
  {\bibfnamefont {F.}~\bibnamefont {Martinez}}, \bibinfo {author}
  {\bibfnamefont {Y.~V.}\ \bibnamefont {Kartashov}}, \bibinfo {author}
  {\bibfnamefont {V.~V.}\ \bibnamefont {Konotop}}, \bibinfo {author}
  {\bibfnamefont {L.}~\bibnamefont {Torner}}, \bibinfo {author} {\bibfnamefont
  {D.}~\bibnamefont {Bauer}}, \emph {et~al.},\ }\bibfield  {title} {\bibinfo
  {title} {Nonlinearity-induced photonic topological insulator},\ }\href@noop
  {} {\bibfield  {journal} {\bibinfo  {journal} {Science}\ }\textbf {\bibinfo
  {volume} {370}},\ \bibinfo {pages} {701} (\bibinfo {year}
  {2020})}\BibitemShut {NoStop}%
\bibitem [{\citenamefont {Kavokin}\ \emph {et~al.}(2017)\citenamefont
  {Kavokin}, \citenamefont {Baumberg}, \citenamefont {Malpuech},\ and\
  \citenamefont {Laussy}}]{kavokin2017microcavities}%
  \BibitemOpen
  \bibfield  {author} {\bibinfo {author} {\bibfnamefont {A.~V.}\ \bibnamefont
  {Kavokin}}, \bibinfo {author} {\bibfnamefont {J.~J.}\ \bibnamefont
  {Baumberg}}, \bibinfo {author} {\bibfnamefont {G.}~\bibnamefont {Malpuech}},\
  and\ \bibinfo {author} {\bibfnamefont {F.~P.}\ \bibnamefont {Laussy}},\
  }\href@noop {} {\emph {\bibinfo {title} {Microcavities}}},\ Vol.~\bibinfo
  {volume} {21}\ (\bibinfo  {publisher} {Oxford university press},\ \bibinfo
  {year} {2017})\BibitemShut {NoStop}%
\bibitem [{\citenamefont {Deng}\ \emph {et~al.}(2002)\citenamefont {Deng},
  \citenamefont {Weihs}, \citenamefont {Santori}, \citenamefont {Bloch},\ and\
  \citenamefont {Yamamoto}}]{deng2002condensation}%
  \BibitemOpen
  \bibfield  {author} {\bibinfo {author} {\bibfnamefont {H.}~\bibnamefont
  {Deng}}, \bibinfo {author} {\bibfnamefont {G.}~\bibnamefont {Weihs}},
  \bibinfo {author} {\bibfnamefont {C.}~\bibnamefont {Santori}}, \bibinfo
  {author} {\bibfnamefont {J.}~\bibnamefont {Bloch}},\ and\ \bibinfo {author}
  {\bibfnamefont {Y.}~\bibnamefont {Yamamoto}},\ }\bibfield  {title} {\bibinfo
  {title} {Condensation of semiconductor microcavity exciton polaritons},\
  }\href@noop {} {\bibfield  {journal} {\bibinfo  {journal} {Science}\ }\textbf
  {\bibinfo {volume} {298}},\ \bibinfo {pages} {199} (\bibinfo {year}
  {2002})}\BibitemShut {NoStop}%
\bibitem [{\citenamefont {Kasprzak}\ \emph {et~al.}(2006)\citenamefont
  {Kasprzak}, \citenamefont {Richard}, \citenamefont {Kundermann},
  \citenamefont {Baas}, \citenamefont {Jeambrun}, \citenamefont {Keeling},
  \citenamefont {Marchetti}, \citenamefont {Szyma{\'n}ska}, \citenamefont
  {Andr{\'e}}, \citenamefont {Staehli}, \citenamefont {Savona}, \citenamefont
  {Littlewood}, \citenamefont {Deveaud},\ and\ \citenamefont
  {Dang}}]{kasprzak2006bose}%
  \BibitemOpen
  \bibfield  {author} {\bibinfo {author} {\bibfnamefont {J.}~\bibnamefont
  {Kasprzak}}, \bibinfo {author} {\bibfnamefont {M.}~\bibnamefont {Richard}},
  \bibinfo {author} {\bibfnamefont {S.}~\bibnamefont {Kundermann}}, \bibinfo
  {author} {\bibfnamefont {A.}~\bibnamefont {Baas}}, \bibinfo {author}
  {\bibfnamefont {P.}~\bibnamefont {Jeambrun}}, \bibinfo {author}
  {\bibfnamefont {J.~M.~J.}\ \bibnamefont {Keeling}}, \bibinfo {author}
  {\bibfnamefont {F.~M.}\ \bibnamefont {Marchetti}}, \bibinfo {author}
  {\bibfnamefont {M.~H.}\ \bibnamefont {Szyma{\'n}ska}}, \bibinfo {author}
  {\bibfnamefont {R.}~\bibnamefont {Andr{\'e}}}, \bibinfo {author}
  {\bibfnamefont {J.~L.}\ \bibnamefont {Staehli}}, \bibinfo {author}
  {\bibfnamefont {V.}~\bibnamefont {Savona}}, \bibinfo {author} {\bibfnamefont
  {P.~B.}\ \bibnamefont {Littlewood}}, \bibinfo {author} {\bibfnamefont
  {B.}~\bibnamefont {Deveaud}},\ and\ \bibinfo {author} {\bibfnamefont {L.~S.}\
  \bibnamefont {Dang}},\ }\bibfield  {title} {\bibinfo {title}
  {Bose-\uppercase{E}instein condensation of exciton polaritons},\ }\href@noop
  {} {\bibfield  {journal} {\bibinfo  {journal} {Nature}\ }\textbf {\bibinfo
  {volume} {443}},\ \bibinfo {pages} {409} (\bibinfo {year}
  {2006})}\BibitemShut {NoStop}%
\bibitem [{\citenamefont {Luo}\ \emph {et~al.}(2023)\citenamefont {Luo},
  \citenamefont {Zhou}, \citenamefont {Zhang},\ and\ \citenamefont
  {Chen}}]{Luo2023}%
  \BibitemOpen
  \bibfield  {author} {\bibinfo {author} {\bibfnamefont {S.}~\bibnamefont
  {Luo}}, \bibinfo {author} {\bibfnamefont {H.}~\bibnamefont {Zhou}}, \bibinfo
  {author} {\bibfnamefont {L.}~\bibnamefont {Zhang}},\ and\ \bibinfo {author}
  {\bibfnamefont {Z.}~\bibnamefont {Chen}},\ }\bibfield  {title} {\bibinfo
  {title} {Nanophotonics of microcavity exciton–polaritons},\ }\href@noop {}
  {\bibfield  {journal} {\bibinfo  {journal} {Applied Physics Reviews}\
  }\textbf {\bibinfo {volume} {10}},\ \bibinfo {pages} {011316} (\bibinfo
  {year} {2023})}\BibitemShut {NoStop}%
\bibitem [{\citenamefont {Zasedatelev}\ \emph {et~al.}(2019)\citenamefont
  {Zasedatelev}, \citenamefont {Baranikov}, \citenamefont {Urbonas},
  \citenamefont {Scafirimuto}, \citenamefont {Scherf}, \citenamefont
  {Stöferle}, \citenamefont {Mahrt},\ and\ \citenamefont
  {Lagoudakis}}]{Zasedatelev2019}%
  \BibitemOpen
  \bibfield  {author} {\bibinfo {author} {\bibfnamefont {A.~V.}\ \bibnamefont
  {Zasedatelev}}, \bibinfo {author} {\bibfnamefont {A.~V.}\ \bibnamefont
  {Baranikov}}, \bibinfo {author} {\bibfnamefont {D.}~\bibnamefont {Urbonas}},
  \bibinfo {author} {\bibfnamefont {F.}~\bibnamefont {Scafirimuto}}, \bibinfo
  {author} {\bibfnamefont {U.}~\bibnamefont {Scherf}}, \bibinfo {author}
  {\bibfnamefont {T.}~\bibnamefont {Stöferle}}, \bibinfo {author}
  {\bibfnamefont {R.~F.}\ \bibnamefont {Mahrt}},\ and\ \bibinfo {author}
  {\bibfnamefont {P.~G.}\ \bibnamefont {Lagoudakis}},\ }\bibfield  {title}
  {\bibinfo {title} {A room-temperature organic polariton transistor},\
  }\href@noop {} {\bibfield  {journal} {\bibinfo  {journal} {Nature Photonics}\
  }\textbf {\bibinfo {volume} {13}},\ \bibinfo {pages} {378} (\bibinfo {year}
  {2019})}\BibitemShut {NoStop}%
\bibitem [{\citenamefont {Ma}\ \emph {et~al.}(2020{\natexlab{a}})\citenamefont
  {Ma}, \citenamefont {Berger}, \citenamefont {A{\ss}mann}, \citenamefont
  {Driben}, \citenamefont {Meier}, \citenamefont {Schneider}, \citenamefont
  {H{\"o}fling},\ and\ \citenamefont {Schumacher}}]{ma2020realization}%
  \BibitemOpen
  \bibfield  {author} {\bibinfo {author} {\bibfnamefont {X.}~\bibnamefont
  {Ma}}, \bibinfo {author} {\bibfnamefont {B.}~\bibnamefont {Berger}}, \bibinfo
  {author} {\bibfnamefont {M.}~\bibnamefont {A{\ss}mann}}, \bibinfo {author}
  {\bibfnamefont {R.}~\bibnamefont {Driben}}, \bibinfo {author} {\bibfnamefont
  {T.}~\bibnamefont {Meier}}, \bibinfo {author} {\bibfnamefont
  {C.}~\bibnamefont {Schneider}}, \bibinfo {author} {\bibfnamefont
  {S.}~\bibnamefont {H{\"o}fling}},\ and\ \bibinfo {author} {\bibfnamefont
  {S.}~\bibnamefont {Schumacher}},\ }\bibfield  {title} {\bibinfo {title}
  {Realization of all-optical vortex switching in exciton-polariton
  condensates},\ }\href@noop {} {\bibfield  {journal} {\bibinfo  {journal}
  {Nat. Commun.}\ }\textbf {\bibinfo {volume} {11}},\ \bibinfo {pages} {897}
  (\bibinfo {year} {2020}{\natexlab{a}})}\BibitemShut {NoStop}%
\bibitem [{\citenamefont {Luk}\ \emph {et~al.}(2021)\citenamefont {Luk},
  \citenamefont {Vergnet}, \citenamefont {Lafont}, \citenamefont {Lewandowski},
  \citenamefont {Kwong}, \citenamefont {Galopin}, \citenamefont {Lemaitre},
  \citenamefont {Roussignol}, \citenamefont {Tignon}, \citenamefont
  {Schumacher}, \citenamefont {Binder},\ and\ \citenamefont
  {Baudin}}]{luk2021all}%
  \BibitemOpen
  \bibfield  {author} {\bibinfo {author} {\bibfnamefont {S.~M.}\ \bibnamefont
  {Luk}}, \bibinfo {author} {\bibfnamefont {H.}~\bibnamefont {Vergnet}},
  \bibinfo {author} {\bibfnamefont {O.}~\bibnamefont {Lafont}}, \bibinfo
  {author} {\bibfnamefont {P.}~\bibnamefont {Lewandowski}}, \bibinfo {author}
  {\bibfnamefont {N.~H.}\ \bibnamefont {Kwong}}, \bibinfo {author}
  {\bibfnamefont {E.}~\bibnamefont {Galopin}}, \bibinfo {author} {\bibfnamefont
  {A.}~\bibnamefont {Lemaitre}}, \bibinfo {author} {\bibfnamefont
  {P.}~\bibnamefont {Roussignol}}, \bibinfo {author} {\bibfnamefont
  {J.}~\bibnamefont {Tignon}}, \bibinfo {author} {\bibfnamefont
  {S.}~\bibnamefont {Schumacher}}, \bibinfo {author} {\bibfnamefont
  {R.}~\bibnamefont {Binder}},\ and\ \bibinfo {author} {\bibfnamefont
  {E.}~\bibnamefont {Baudin}},\ }\bibfield  {title} {\bibinfo {title}
  {All-optical beam steering using the polariton lighthouse effect},\
  }\href@noop {} {\bibfield  {journal} {\bibinfo  {journal} {ACS Photonics}\
  }\textbf {\bibinfo {volume} {8}},\ \bibinfo {pages} {449} (\bibinfo {year}
  {2021})}\BibitemShut {NoStop}%
\bibitem [{\citenamefont {Schmutzler}\ \emph {et~al.}(2015)\citenamefont
  {Schmutzler}, \citenamefont {Lewandowski}, \citenamefont {A\ss{}mann},
  \citenamefont {Niemietz}, \citenamefont {Schumacher}, \citenamefont {Kamp},
  \citenamefont {Schneider}, \citenamefont {H\"ofling},\ and\ \citenamefont
  {Bayer}}]{schmutzler2015}%
  \BibitemOpen
  \bibfield  {author} {\bibinfo {author} {\bibfnamefont {J.}~\bibnamefont
  {Schmutzler}}, \bibinfo {author} {\bibfnamefont {P.}~\bibnamefont
  {Lewandowski}}, \bibinfo {author} {\bibfnamefont {M.}~\bibnamefont
  {A\ss{}mann}}, \bibinfo {author} {\bibfnamefont {D.}~\bibnamefont
  {Niemietz}}, \bibinfo {author} {\bibfnamefont {S.}~\bibnamefont
  {Schumacher}}, \bibinfo {author} {\bibfnamefont {M.}~\bibnamefont {Kamp}},
  \bibinfo {author} {\bibfnamefont {C.}~\bibnamefont {Schneider}}, \bibinfo
  {author} {\bibfnamefont {S.}~\bibnamefont {H\"ofling}},\ and\ \bibinfo
  {author} {\bibfnamefont {M.}~\bibnamefont {Bayer}},\ }\bibfield  {title}
  {\bibinfo {title} {All-optical flow control of a polariton condensate using
  nonresonant excitation},\ }\href@noop {} {\bibfield  {journal} {\bibinfo
  {journal} {Phys. Rev. B}\ }\textbf {\bibinfo {volume} {91}},\ \bibinfo
  {pages} {195308} (\bibinfo {year} {2015})}\BibitemShut {NoStop}%
\bibitem [{\citenamefont {Schneider}\ \emph {et~al.}(2016)\citenamefont
  {Schneider}, \citenamefont {Winkler}, \citenamefont {Fraser}, \citenamefont
  {Kamp}, \citenamefont {Yamamoto}, \citenamefont {Ostrovskaya},\ and\
  \citenamefont {H{\"o}fling}}]{schneider2016exciton}%
  \BibitemOpen
  \bibfield  {author} {\bibinfo {author} {\bibfnamefont {C.}~\bibnamefont
  {Schneider}}, \bibinfo {author} {\bibfnamefont {K.}~\bibnamefont {Winkler}},
  \bibinfo {author} {\bibfnamefont {M.~D.}\ \bibnamefont {Fraser}}, \bibinfo
  {author} {\bibfnamefont {M.}~\bibnamefont {Kamp}}, \bibinfo {author}
  {\bibfnamefont {Y.}~\bibnamefont {Yamamoto}}, \bibinfo {author}
  {\bibfnamefont {E.}~\bibnamefont {Ostrovskaya}},\ and\ \bibinfo {author}
  {\bibfnamefont {S.}~\bibnamefont {H{\"o}fling}},\ }\bibfield  {title}
  {\bibinfo {title} {Exciton-polariton trapping and potential landscape
  engineering},\ }\href@noop {} {\bibfield  {journal} {\bibinfo  {journal}
  {Rep. Prog. Phys.}\ }\textbf {\bibinfo {volume} {80}},\ \bibinfo {pages}
  {016503} (\bibinfo {year} {2016})}\BibitemShut {NoStop}%
\bibitem [{\citenamefont {Klembt}\ \emph {et~al.}(2018)\citenamefont {Klembt},
  \citenamefont {Harder}, \citenamefont {Egorov}, \citenamefont {Winkler},
  \citenamefont {Ge}, \citenamefont {Bandres}, \citenamefont {Emmerling},
  \citenamefont {Worschech}, \citenamefont {Liew}, \citenamefont {Segev},
  \citenamefont {Schneider},\ and\ \citenamefont
  {H{\"o}fling}}]{klembt2018exciton}%
  \BibitemOpen
  \bibfield  {author} {\bibinfo {author} {\bibfnamefont {S.}~\bibnamefont
  {Klembt}}, \bibinfo {author} {\bibfnamefont {T.}~\bibnamefont {Harder}},
  \bibinfo {author} {\bibfnamefont {O.}~\bibnamefont {Egorov}}, \bibinfo
  {author} {\bibfnamefont {K.}~\bibnamefont {Winkler}}, \bibinfo {author}
  {\bibfnamefont {R.}~\bibnamefont {Ge}}, \bibinfo {author} {\bibfnamefont
  {M.}~\bibnamefont {Bandres}}, \bibinfo {author} {\bibfnamefont
  {M.}~\bibnamefont {Emmerling}}, \bibinfo {author} {\bibfnamefont
  {L.}~\bibnamefont {Worschech}}, \bibinfo {author} {\bibfnamefont
  {T.}~\bibnamefont {Liew}}, \bibinfo {author} {\bibfnamefont {M.}~\bibnamefont
  {Segev}}, \bibinfo {author} {\bibfnamefont {C.}~\bibnamefont {Schneider}},\
  and\ \bibinfo {author} {\bibfnamefont {S.}~\bibnamefont {H{\"o}fling}},\
  }\bibfield  {title} {\bibinfo {title} {Exciton-polariton topological
  insulator},\ }\href@noop {} {\bibfield  {journal} {\bibinfo  {journal}
  {Nature}\ }\textbf {\bibinfo {volume} {562}},\ \bibinfo {pages} {552}
  (\bibinfo {year} {2018})}\BibitemShut {NoStop}%
\bibitem [{\citenamefont {Mili{\'c}evi{\'c}}\ \emph {et~al.}(2015)\citenamefont
  {Mili{\'c}evi{\'c}}, \citenamefont {Ozawa}, \citenamefont {Andreakou},
  \citenamefont {Carusotto}, \citenamefont {Jacqmin}, \citenamefont {Galopin},
  \citenamefont {Lemaitre}, \citenamefont {Le~Gratiet}, \citenamefont {Sagnes},
  \citenamefont {Bloch},\ and\ \citenamefont {Amo}}]{milicevic2015edge}%
  \BibitemOpen
  \bibfield  {author} {\bibinfo {author} {\bibfnamefont {M.}~\bibnamefont
  {Mili{\'c}evi{\'c}}}, \bibinfo {author} {\bibfnamefont {T.}~\bibnamefont
  {Ozawa}}, \bibinfo {author} {\bibfnamefont {P.}~\bibnamefont {Andreakou}},
  \bibinfo {author} {\bibfnamefont {I.}~\bibnamefont {Carusotto}}, \bibinfo
  {author} {\bibfnamefont {T.}~\bibnamefont {Jacqmin}}, \bibinfo {author}
  {\bibfnamefont {E.}~\bibnamefont {Galopin}}, \bibinfo {author} {\bibfnamefont
  {A.}~\bibnamefont {Lemaitre}}, \bibinfo {author} {\bibfnamefont
  {L.}~\bibnamefont {Le~Gratiet}}, \bibinfo {author} {\bibfnamefont
  {I.}~\bibnamefont {Sagnes}}, \bibinfo {author} {\bibfnamefont
  {J.}~\bibnamefont {Bloch}},\ and\ \bibinfo {author} {\bibfnamefont
  {A.}~\bibnamefont {Amo}},\ }\bibfield  {title} {\bibinfo {title} {Edge states
  in polariton honeycomb lattices},\ }\href@noop {} {\bibfield  {journal}
  {\bibinfo  {journal} {2D Materials}\ }\textbf {\bibinfo {volume} {2}},\
  \bibinfo {pages} {034012} (\bibinfo {year} {2015})}\BibitemShut {NoStop}%
\bibitem [{\citenamefont {Ma}\ \emph {et~al.}(2020{\natexlab{b}})\citenamefont
  {Ma}, \citenamefont {Kartashov}, \citenamefont {Ferrando},\ and\
  \citenamefont {Schumacher}}]{Ma:20}%
  \BibitemOpen
  \bibfield  {author} {\bibinfo {author} {\bibfnamefont {X.}~\bibnamefont
  {Ma}}, \bibinfo {author} {\bibfnamefont {Y.~V.}\ \bibnamefont {Kartashov}},
  \bibinfo {author} {\bibfnamefont {A.}~\bibnamefont {Ferrando}},\ and\
  \bibinfo {author} {\bibfnamefont {S.}~\bibnamefont {Schumacher}},\ }\bibfield
   {title} {\bibinfo {title} {Topological edge states of nonequilibrium
  polaritons in hollow honeycomb arrays},\ }\href
  {https://doi.org/10.1364/OL.405844} {\bibfield  {journal} {\bibinfo
  {journal} {Opt. Lett.}\ }\textbf {\bibinfo {volume} {45}},\ \bibinfo {pages}
  {5311} (\bibinfo {year} {2020}{\natexlab{b}})}\BibitemShut {NoStop}%
\bibitem [{\citenamefont {Whittaker}\ \emph {et~al.}(2018)\citenamefont
  {Whittaker}, \citenamefont {Cancellieri}, \citenamefont {Walker},
  \citenamefont {Gulevich}, \citenamefont {Schomerus}, \citenamefont
  {Vaitiekus}, \citenamefont {Royall}, \citenamefont {Whittaker}, \citenamefont
  {Clarke}, \citenamefont {Iorsh}, \citenamefont {Shelykh}, \citenamefont
  {Skolnick},\ and\ \citenamefont {Krizhanovskii}}]{PhysRevLett.120.097401}%
  \BibitemOpen
  \bibfield  {author} {\bibinfo {author} {\bibfnamefont {C.~E.}\ \bibnamefont
  {Whittaker}}, \bibinfo {author} {\bibfnamefont {E.}~\bibnamefont
  {Cancellieri}}, \bibinfo {author} {\bibfnamefont {P.~M.}\ \bibnamefont
  {Walker}}, \bibinfo {author} {\bibfnamefont {D.~R.}\ \bibnamefont
  {Gulevich}}, \bibinfo {author} {\bibfnamefont {H.}~\bibnamefont {Schomerus}},
  \bibinfo {author} {\bibfnamefont {D.}~\bibnamefont {Vaitiekus}}, \bibinfo
  {author} {\bibfnamefont {B.}~\bibnamefont {Royall}}, \bibinfo {author}
  {\bibfnamefont {D.~M.}\ \bibnamefont {Whittaker}}, \bibinfo {author}
  {\bibfnamefont {E.}~\bibnamefont {Clarke}}, \bibinfo {author} {\bibfnamefont
  {I.~V.}\ \bibnamefont {Iorsh}}, \bibinfo {author} {\bibfnamefont {I.~A.}\
  \bibnamefont {Shelykh}}, \bibinfo {author} {\bibfnamefont {M.~S.}\
  \bibnamefont {Skolnick}},\ and\ \bibinfo {author} {\bibfnamefont {D.~N.}\
  \bibnamefont {Krizhanovskii}},\ }\bibfield  {title} {\bibinfo {title}
  {Exciton polaritons in a two-dimensional {L}ieb lattice with spin-orbit
  coupling},\ }\href {https://doi.org/10.1103/PhysRevLett.120.097401}
  {\bibfield  {journal} {\bibinfo  {journal} {Phys. Rev. Lett.}\ }\textbf
  {\bibinfo {volume} {120}},\ \bibinfo {pages} {097401} (\bibinfo {year}
  {2018})}\BibitemShut {NoStop}%
\bibitem [{\citenamefont {Li}\ \emph {et~al.}(2018)\citenamefont {Li},
  \citenamefont {Ye}, \citenamefont {Chen}, \citenamefont {Kartashov},
  \citenamefont {Ferrando}, \citenamefont {Torner},\ and\ \citenamefont
  {Skryabin}}]{PhysRevB.97.081103}%
  \BibitemOpen
  \bibfield  {author} {\bibinfo {author} {\bibfnamefont {C.}~\bibnamefont
  {Li}}, \bibinfo {author} {\bibfnamefont {F.}~\bibnamefont {Ye}}, \bibinfo
  {author} {\bibfnamefont {X.}~\bibnamefont {Chen}}, \bibinfo {author}
  {\bibfnamefont {Y.~V.}\ \bibnamefont {Kartashov}}, \bibinfo {author}
  {\bibfnamefont {A.}~\bibnamefont {Ferrando}}, \bibinfo {author}
  {\bibfnamefont {L.}~\bibnamefont {Torner}},\ and\ \bibinfo {author}
  {\bibfnamefont {D.~V.}\ \bibnamefont {Skryabin}},\ }\bibfield  {title}
  {\bibinfo {title} {Lieb polariton topological insulators},\ }\href
  {https://doi.org/10.1103/PhysRevB.97.081103} {\bibfield  {journal} {\bibinfo
  {journal} {Phys. Rev. B}\ }\textbf {\bibinfo {volume} {97}},\ \bibinfo
  {pages} {081103(R)} (\bibinfo {year} {2018})}\BibitemShut {NoStop}%
\bibitem [{\citenamefont {Gulevich}\ \emph {et~al.}(2017)\citenamefont
  {Gulevich}, \citenamefont {Yudin}, \citenamefont {Skryabin}, \citenamefont
  {Iorsh},\ and\ \citenamefont {Shelykh}}]{gulevich2017exploring}%
  \BibitemOpen
  \bibfield  {author} {\bibinfo {author} {\bibfnamefont {D.~R.}\ \bibnamefont
  {Gulevich}}, \bibinfo {author} {\bibfnamefont {D.}~\bibnamefont {Yudin}},
  \bibinfo {author} {\bibfnamefont {D.~V.}\ \bibnamefont {Skryabin}}, \bibinfo
  {author} {\bibfnamefont {I.~V.}\ \bibnamefont {Iorsh}},\ and\ \bibinfo
  {author} {\bibfnamefont {I.~A.}\ \bibnamefont {Shelykh}},\ }\bibfield
  {title} {\bibinfo {title} {Exploring nonlinear topological states of matter
  with exciton-polaritons: Edge solitons in kagome lattice},\ }\href@noop {}
  {\bibfield  {journal} {\bibinfo  {journal} {Sci. Rep.}\ }\textbf {\bibinfo
  {volume} {7}},\ \bibinfo {pages} {1780} (\bibinfo {year} {2017})}\BibitemShut
  {NoStop}%
\bibitem [{\citenamefont {Gulevich}\ \emph {et~al.}(2016)\citenamefont
  {Gulevich}, \citenamefont {Yudin}, \citenamefont {Iorsh},\ and\ \citenamefont
  {Shelykh}}]{PhysRevB.94.115437}%
  \BibitemOpen
  \bibfield  {author} {\bibinfo {author} {\bibfnamefont {D.~R.}\ \bibnamefont
  {Gulevich}}, \bibinfo {author} {\bibfnamefont {D.}~\bibnamefont {Yudin}},
  \bibinfo {author} {\bibfnamefont {I.~V.}\ \bibnamefont {Iorsh}},\ and\
  \bibinfo {author} {\bibfnamefont {I.~A.}\ \bibnamefont {Shelykh}},\
  }\bibfield  {title} {\bibinfo {title} {Kagome lattice from an
  exciton-polariton perspective},\ }\href
  {https://doi.org/10.1103/PhysRevB.94.115437} {\bibfield  {journal} {\bibinfo
  {journal} {Phys. Rev. B}\ }\textbf {\bibinfo {volume} {94}},\ \bibinfo
  {pages} {115437} (\bibinfo {year} {2016})}\BibitemShut {NoStop}%
\bibitem [{\citenamefont {Su}\ \emph {et~al.}(2021)\citenamefont {Su},
  \citenamefont {Ghosh}, \citenamefont {Liew},\ and\ \citenamefont
  {Xiong}}]{su2021optical}%
  \BibitemOpen
  \bibfield  {author} {\bibinfo {author} {\bibfnamefont {R.}~\bibnamefont
  {Su}}, \bibinfo {author} {\bibfnamefont {S.}~\bibnamefont {Ghosh}}, \bibinfo
  {author} {\bibfnamefont {T.~C.~H.}\ \bibnamefont {Liew}},\ and\ \bibinfo
  {author} {\bibfnamefont {Q.}~\bibnamefont {Xiong}},\ }\bibfield  {title}
  {\bibinfo {title} {Optical switching of topological phase in a perovskite
  polariton lattice},\ }\href@noop {} {\bibfield  {journal} {\bibinfo
  {journal} {Sci. Adv.}\ }\textbf {\bibinfo {volume} {7}},\ \bibinfo {pages}
  {eabf8049} (\bibinfo {year} {2021})}\BibitemShut {NoStop}%
\bibitem [{\citenamefont {Harder}\ \emph {et~al.}(2021)\citenamefont {Harder},
  \citenamefont {Sun}, \citenamefont {Egorov}, \citenamefont {Vakulchyk},
  \citenamefont {Beierlein}, \citenamefont {Gagel}, \citenamefont {Emmerling},
  \citenamefont {Schneider}, \citenamefont {Peschel}, \citenamefont {Savenko},
  \citenamefont {Klembt},\ and\ \citenamefont
  {H{\"o}fling}}]{harder2021coherent}%
  \BibitemOpen
  \bibfield  {author} {\bibinfo {author} {\bibfnamefont {T.~H.}\ \bibnamefont
  {Harder}}, \bibinfo {author} {\bibfnamefont {M.}~\bibnamefont {Sun}},
  \bibinfo {author} {\bibfnamefont {O.~A.}\ \bibnamefont {Egorov}}, \bibinfo
  {author} {\bibfnamefont {I.}~\bibnamefont {Vakulchyk}}, \bibinfo {author}
  {\bibfnamefont {J.}~\bibnamefont {Beierlein}}, \bibinfo {author}
  {\bibfnamefont {P.}~\bibnamefont {Gagel}}, \bibinfo {author} {\bibfnamefont
  {M.}~\bibnamefont {Emmerling}}, \bibinfo {author} {\bibfnamefont
  {C.}~\bibnamefont {Schneider}}, \bibinfo {author} {\bibfnamefont
  {U.}~\bibnamefont {Peschel}}, \bibinfo {author} {\bibfnamefont {I.~G.}\
  \bibnamefont {Savenko}}, \bibinfo {author} {\bibfnamefont {S.}~\bibnamefont
  {Klembt}},\ and\ \bibinfo {author} {\bibfnamefont {S.}~\bibnamefont
  {H{\"o}fling}},\ }\bibfield  {title} {\bibinfo {title} {Coherent topological
  polariton laser},\ }\href@noop {} {\bibfield  {journal} {\bibinfo  {journal}
  {ACS Photonics}\ }\textbf {\bibinfo {volume} {8}},\ \bibinfo {pages} {1377}
  (\bibinfo {year} {2021})}\BibitemShut {NoStop}%
\bibitem [{\citenamefont {Ma}\ and\ \citenamefont
  {Schumacher}(2018)}]{PhysRevLett.121.227404}%
  \BibitemOpen
  \bibfield  {author} {\bibinfo {author} {\bibfnamefont {X.}~\bibnamefont
  {Ma}}\ and\ \bibinfo {author} {\bibfnamefont {S.}~\bibnamefont
  {Schumacher}},\ }\bibfield  {title} {\bibinfo {title} {Vortex multistability
  and {B}essel vortices in polariton condensates},\ }\href
  {https://doi.org/10.1103/PhysRevLett.121.227404} {\bibfield  {journal}
  {\bibinfo  {journal} {Phys. Rev. Lett.}\ }\textbf {\bibinfo {volume} {121}},\
  \bibinfo {pages} {227404} (\bibinfo {year} {2018})}\BibitemShut {NoStop}%
\bibitem [{\citenamefont {Kartashov}\ and\ \citenamefont
  {Skryabin}(2016)}]{Kartashov:16}%
  \BibitemOpen
  \bibfield  {author} {\bibinfo {author} {\bibfnamefont {Y.~V.}\ \bibnamefont
  {Kartashov}}\ and\ \bibinfo {author} {\bibfnamefont {D.~V.}\ \bibnamefont
  {Skryabin}},\ }\bibfield  {title} {\bibinfo {title} {Modulational instability
  and solitary waves in polariton topological insulators},\ }\href
  {https://doi.org/10.1364/OPTICA.3.001228} {\bibfield  {journal} {\bibinfo
  {journal} {Optica}\ }\textbf {\bibinfo {volume} {3}},\ \bibinfo {pages}
  {1228} (\bibinfo {year} {2016})}\BibitemShut {NoStop}%
\bibitem [{\citenamefont {Pernet}\ \emph {et~al.}(2022)\citenamefont {Pernet},
  \citenamefont {St-Jean}, \citenamefont {Solnyshkov}, \citenamefont
  {Malpuech}, \citenamefont {Carlon~Zambon}, \citenamefont {Fontaine},
  \citenamefont {Real}, \citenamefont {Jamadi}, \citenamefont {Lema{\^\i}tre},
  \citenamefont {Morassi}, \citenamefont {Gratiet}, \citenamefont {Baptiste},
  \citenamefont {Harouri}, \citenamefont {Sagnes}, \citenamefont {Amo},
  \citenamefont {Ravets},\ and\ \citenamefont {Bloch}}]{pernet2022gap}%
  \BibitemOpen
  \bibfield  {author} {\bibinfo {author} {\bibfnamefont {N.}~\bibnamefont
  {Pernet}}, \bibinfo {author} {\bibfnamefont {P.}~\bibnamefont {St-Jean}},
  \bibinfo {author} {\bibfnamefont {D.~D.}\ \bibnamefont {Solnyshkov}},
  \bibinfo {author} {\bibfnamefont {G.}~\bibnamefont {Malpuech}}, \bibinfo
  {author} {\bibfnamefont {N.}~\bibnamefont {Carlon~Zambon}}, \bibinfo {author}
  {\bibfnamefont {Q.}~\bibnamefont {Fontaine}}, \bibinfo {author}
  {\bibfnamefont {B.}~\bibnamefont {Real}}, \bibinfo {author} {\bibfnamefont
  {O.}~\bibnamefont {Jamadi}}, \bibinfo {author} {\bibfnamefont
  {A.}~\bibnamefont {Lema{\^\i}tre}}, \bibinfo {author} {\bibfnamefont
  {M.}~\bibnamefont {Morassi}}, \bibinfo {author} {\bibfnamefont {L.~L.}\
  \bibnamefont {Gratiet}}, \bibinfo {author} {\bibfnamefont {T.}~\bibnamefont
  {Baptiste}}, \bibinfo {author} {\bibfnamefont {A.}~\bibnamefont {Harouri}},
  \bibinfo {author} {\bibfnamefont {I.}~\bibnamefont {Sagnes}}, \bibinfo
  {author} {\bibfnamefont {A.}~\bibnamefont {Amo}}, \bibinfo {author}
  {\bibfnamefont {S.}~\bibnamefont {Ravets}},\ and\ \bibinfo {author}
  {\bibfnamefont {J.}~\bibnamefont {Bloch}},\ }\bibfield  {title} {\bibinfo
  {title} {Gap solitons in a one-dimensional driven-dissipative topological
  lattice},\ }\href@noop {} {\bibfield  {journal} {\bibinfo  {journal} {Nat.
  Phys.}\ }\textbf {\bibinfo {volume} {18}},\ \bibinfo {pages} {678} (\bibinfo
  {year} {2022})}\BibitemShut {NoStop}%
\bibitem [{\citenamefont {Banerjee}\ \emph {et~al.}(2020)\citenamefont
  {Banerjee}, \citenamefont {Mandal},\ and\ \citenamefont
  {Liew}}]{PhysRevLett.124.063901}%
  \BibitemOpen
  \bibfield  {author} {\bibinfo {author} {\bibfnamefont {R.}~\bibnamefont
  {Banerjee}}, \bibinfo {author} {\bibfnamefont {S.}~\bibnamefont {Mandal}},\
  and\ \bibinfo {author} {\bibfnamefont {T.~C.~H.}\ \bibnamefont {Liew}},\
  }\bibfield  {title} {\bibinfo {title} {Coupling between exciton-polariton
  corner modes through edge states},\ }\href
  {https://doi.org/10.1103/PhysRevLett.124.063901} {\bibfield  {journal}
  {\bibinfo  {journal} {Phys. Rev. Lett.}\ }\textbf {\bibinfo {volume} {124}},\
  \bibinfo {pages} {063901} (\bibinfo {year} {2020})}\BibitemShut {NoStop}%
\bibitem [{\citenamefont {Kartashov}\ and\ \citenamefont
  {Skryabin}(2017)}]{PhysRevLett.119.253904}%
  \BibitemOpen
  \bibfield  {author} {\bibinfo {author} {\bibfnamefont {Y.~V.}\ \bibnamefont
  {Kartashov}}\ and\ \bibinfo {author} {\bibfnamefont {D.~V.}\ \bibnamefont
  {Skryabin}},\ }\bibfield  {title} {\bibinfo {title} {Bistable topological
  insulator with exciton-polaritons},\ }\href
  {https://doi.org/10.1103/PhysRevLett.119.253904} {\bibfield  {journal}
  {\bibinfo  {journal} {Phys. Rev. Lett.}\ }\textbf {\bibinfo {volume} {119}},\
  \bibinfo {pages} {253904} (\bibinfo {year} {2017})}\BibitemShut {NoStop}%
\bibitem [{\citenamefont {Zhang}\ \emph {et~al.}(2019)\citenamefont {Zhang},
  \citenamefont {Chen}, \citenamefont {Kartashov}, \citenamefont {Skryabin},\
  and\ \citenamefont {Ye}}]{zhang2019finite}%
  \BibitemOpen
  \bibfield  {author} {\bibinfo {author} {\bibfnamefont {W.}~\bibnamefont
  {Zhang}}, \bibinfo {author} {\bibfnamefont {X.}~\bibnamefont {Chen}},
  \bibinfo {author} {\bibfnamefont {Y.~V.}\ \bibnamefont {Kartashov}}, \bibinfo
  {author} {\bibfnamefont {D.~V.}\ \bibnamefont {Skryabin}},\ and\ \bibinfo
  {author} {\bibfnamefont {F.}~\bibnamefont {Ye}},\ }\bibfield  {title}
  {\bibinfo {title} {Finite-dimensional bistable topological insulators: From
  small to large},\ }\href@noop {} {\bibfield  {journal} {\bibinfo  {journal}
  {Laser Photonics Rev.}\ }\textbf {\bibinfo {volume} {13}},\ \bibinfo {pages}
  {1900198} (\bibinfo {year} {2019})}\BibitemShut {NoStop}%
\bibitem [{\citenamefont {Xu}\ \emph {et~al.}(2021)\citenamefont {Xu},
  \citenamefont {Xu}, \citenamefont {Mandal}, \citenamefont {Banerjee},
  \citenamefont {Ghosh},\ and\ \citenamefont {Liew}}]{PhysRevB.103.235306}%
  \BibitemOpen
  \bibfield  {author} {\bibinfo {author} {\bibfnamefont {X.}~\bibnamefont
  {Xu}}, \bibinfo {author} {\bibfnamefont {H.}~\bibnamefont {Xu}}, \bibinfo
  {author} {\bibfnamefont {S.}~\bibnamefont {Mandal}}, \bibinfo {author}
  {\bibfnamefont {R.}~\bibnamefont {Banerjee}}, \bibinfo {author}
  {\bibfnamefont {S.}~\bibnamefont {Ghosh}},\ and\ \bibinfo {author}
  {\bibfnamefont {T.~C.~H.}\ \bibnamefont {Liew}},\ }\bibfield  {title}
  {\bibinfo {title} {Interaction-induced double-sided skin effect in an
  exciton-polariton system},\ }\href
  {https://doi.org/10.1103/PhysRevB.103.235306} {\bibfield  {journal} {\bibinfo
   {journal} {Phys. Rev. B}\ }\textbf {\bibinfo {volume} {103}},\ \bibinfo
  {pages} {235306} (\bibinfo {year} {2021})}\BibitemShut {NoStop}%
\bibitem [{\citenamefont {Chiu}\ \emph {et~al.}(2016)\citenamefont {Chiu},
  \citenamefont {Teo}, \citenamefont {Schnyder},\ and\ \citenamefont
  {Ryu}}]{chiu2016classification}%
  \BibitemOpen
  \bibfield  {author} {\bibinfo {author} {\bibfnamefont {C.-K.}\ \bibnamefont
  {Chiu}}, \bibinfo {author} {\bibfnamefont {J.~C.~Y.}\ \bibnamefont {Teo}},
  \bibinfo {author} {\bibfnamefont {A.~P.}\ \bibnamefont {Schnyder}},\ and\
  \bibinfo {author} {\bibfnamefont {S.}~\bibnamefont {Ryu}},\ }\bibfield
  {title} {\bibinfo {title} {Classification of topological quantum matter with
  symmetries},\ }\href@noop {} {\bibfield  {journal} {\bibinfo  {journal} {Rev.
  Mod. Phys.}\ }\textbf {\bibinfo {volume} {88}},\ \bibinfo {pages} {035005}
  (\bibinfo {year} {2016})}\BibitemShut {NoStop}%
\bibitem [{\citenamefont {Blanco-Redondo}\ \emph {et~al.}(2016)\citenamefont
  {Blanco-Redondo}, \citenamefont {Andonegui}, \citenamefont {Collins},
  \citenamefont {Harari}, \citenamefont {Lumer}, \citenamefont {Rechtsman},
  \citenamefont {Eggleton},\ and\ \citenamefont
  {Segev}}]{blanco2016topological}%
  \BibitemOpen
  \bibfield  {author} {\bibinfo {author} {\bibfnamefont {A.}~\bibnamefont
  {Blanco-Redondo}}, \bibinfo {author} {\bibfnamefont {I.}~\bibnamefont
  {Andonegui}}, \bibinfo {author} {\bibfnamefont {M.~J.}\ \bibnamefont
  {Collins}}, \bibinfo {author} {\bibfnamefont {G.}~\bibnamefont {Harari}},
  \bibinfo {author} {\bibfnamefont {Y.}~\bibnamefont {Lumer}}, \bibinfo
  {author} {\bibfnamefont {M.~C.}\ \bibnamefont {Rechtsman}}, \bibinfo {author}
  {\bibfnamefont {B.~J.}\ \bibnamefont {Eggleton}},\ and\ \bibinfo {author}
  {\bibfnamefont {M.}~\bibnamefont {Segev}},\ }\bibfield  {title} {\bibinfo
  {title} {Topological optical waveguiding in silicon and the transition
  between topological and trivial defect states},\ }\href
  {https://doi.org/10.1103/PhysRevLett.116.163901} {\bibfield  {journal}
  {\bibinfo  {journal} {Phys. Rev. Lett.}\ }\textbf {\bibinfo {volume} {116}},\
  \bibinfo {pages} {163901} (\bibinfo {year} {2016})}\BibitemShut {NoStop}%
\bibitem [{\citenamefont {Zhao}\ \emph {et~al.}(2018)\citenamefont {Zhao},
  \citenamefont {Miao}, \citenamefont {Teimourpour}, \citenamefont {Malzard},
  \citenamefont {El-Ganainy}, \citenamefont {Schomerus},\ and\ \citenamefont
  {Feng}}]{zhao2018topological}%
  \BibitemOpen
  \bibfield  {author} {\bibinfo {author} {\bibfnamefont {H.}~\bibnamefont
  {Zhao}}, \bibinfo {author} {\bibfnamefont {P.}~\bibnamefont {Miao}}, \bibinfo
  {author} {\bibfnamefont {M.~H.}\ \bibnamefont {Teimourpour}}, \bibinfo
  {author} {\bibfnamefont {S.}~\bibnamefont {Malzard}}, \bibinfo {author}
  {\bibfnamefont {R.}~\bibnamefont {El-Ganainy}}, \bibinfo {author}
  {\bibfnamefont {H.}~\bibnamefont {Schomerus}},\ and\ \bibinfo {author}
  {\bibfnamefont {L.}~\bibnamefont {Feng}},\ }\bibfield  {title} {\bibinfo
  {title} {Topological hybrid silicon microlasers},\ }\href@noop {} {\bibfield
  {journal} {\bibinfo  {journal} {Nat. Commun.}\ }\textbf {\bibinfo {volume}
  {9}},\ \bibinfo {pages} {981} (\bibinfo {year} {2018})}\BibitemShut {NoStop}%
\bibitem [{\citenamefont {Parto}\ \emph {et~al.}(2018)\citenamefont {Parto},
  \citenamefont {Wittek}, \citenamefont {Hodaei}, \citenamefont {Harari},
  \citenamefont {Bandres}, \citenamefont {Ren}, \citenamefont {Rechtsman},
  \citenamefont {Segev}, \citenamefont {Christodoulides},\ and\ \citenamefont
  {Khajavikhan}}]{parto2018edge}%
  \BibitemOpen
  \bibfield  {author} {\bibinfo {author} {\bibfnamefont {M.}~\bibnamefont
  {Parto}}, \bibinfo {author} {\bibfnamefont {S.}~\bibnamefont {Wittek}},
  \bibinfo {author} {\bibfnamefont {H.}~\bibnamefont {Hodaei}}, \bibinfo
  {author} {\bibfnamefont {G.}~\bibnamefont {Harari}}, \bibinfo {author}
  {\bibfnamefont {M.~A.}\ \bibnamefont {Bandres}}, \bibinfo {author}
  {\bibfnamefont {J.}~\bibnamefont {Ren}}, \bibinfo {author} {\bibfnamefont
  {M.~C.}\ \bibnamefont {Rechtsman}}, \bibinfo {author} {\bibfnamefont
  {M.}~\bibnamefont {Segev}}, \bibinfo {author} {\bibfnamefont {D.~N.}\
  \bibnamefont {Christodoulides}},\ and\ \bibinfo {author} {\bibfnamefont
  {M.}~\bibnamefont {Khajavikhan}},\ }\bibfield  {title} {\bibinfo {title}
  {Edge-mode lasing in 1d topological active arrays},\ }\href
  {https://doi.org/10.1103/PhysRevLett.120.113901} {\bibfield  {journal}
  {\bibinfo  {journal} {Phys. Rev. Lett.}\ }\textbf {\bibinfo {volume} {120}},\
  \bibinfo {pages} {113901} (\bibinfo {year} {2018})}\BibitemShut {NoStop}%
\bibitem [{\citenamefont {Han}\ \emph {et~al.}(2019)\citenamefont {Han},
  \citenamefont {Lee}, \citenamefont {Callard}, \citenamefont {Seassal},\ and\
  \citenamefont {Jeon}}]{han2019lasing}%
  \BibitemOpen
  \bibfield  {author} {\bibinfo {author} {\bibfnamefont {C.}~\bibnamefont
  {Han}}, \bibinfo {author} {\bibfnamefont {M.}~\bibnamefont {Lee}}, \bibinfo
  {author} {\bibfnamefont {S.}~\bibnamefont {Callard}}, \bibinfo {author}
  {\bibfnamefont {C.}~\bibnamefont {Seassal}},\ and\ \bibinfo {author}
  {\bibfnamefont {H.}~\bibnamefont {Jeon}},\ }\bibfield  {title} {\bibinfo
  {title} {Lasing at topological edge states in a photonic crystal l3
  nanocavity dimer array},\ }\href@noop {} {\bibfield  {journal} {\bibinfo
  {journal} {Light Sci. Appl.}\ }\textbf {\bibinfo {volume} {8}},\ \bibinfo
  {pages} {40} (\bibinfo {year} {2019})}\BibitemShut {NoStop}%
\bibitem [{\citenamefont {Aubry}\ and\ \citenamefont
  {Andr{\'e}}(1980)}]{aubry1980analyticity}%
  \BibitemOpen
  \bibfield  {author} {\bibinfo {author} {\bibfnamefont {S.}~\bibnamefont
  {Aubry}}\ and\ \bibinfo {author} {\bibfnamefont {G.}~\bibnamefont
  {Andr{\'e}}},\ }\bibfield  {title} {\bibinfo {title} {Analyticity breaking
  and {A}nderson localization in incommensurate lattices},\ }\href@noop {}
  {\bibfield  {journal} {\bibinfo  {journal} {Ann. Israel Phys. Soc}\ }\textbf
  {\bibinfo {volume} {3}},\ \bibinfo {pages} {18} (\bibinfo {year}
  {1980})}\BibitemShut {NoStop}%
\bibitem [{\citenamefont {Ganeshan}\ \emph {et~al.}(2013)\citenamefont
  {Ganeshan}, \citenamefont {Sun},\ and\ \citenamefont
  {Das~Sarma}}]{ganeshan2013topological}%
  \BibitemOpen
  \bibfield  {author} {\bibinfo {author} {\bibfnamefont {S.}~\bibnamefont
  {Ganeshan}}, \bibinfo {author} {\bibfnamefont {K.}~\bibnamefont {Sun}},\ and\
  \bibinfo {author} {\bibfnamefont {S.}~\bibnamefont {Das~Sarma}},\ }\bibfield
  {title} {\bibinfo {title} {Topological zero-energy modes in gapless
  commensurate {A}ubry-{A}ndr{\'e}-{H}arper models},\ }\href@noop {} {\bibfield
   {journal} {\bibinfo  {journal} {Phys. Rev. Lett.}\ }\textbf {\bibinfo
  {volume} {110}},\ \bibinfo {pages} {180403} (\bibinfo {year}
  {2013})}\BibitemShut {NoStop}%
\bibitem [{\citenamefont {Carcamo}\ \emph {et~al.}(2020)\citenamefont
  {Carcamo}, \citenamefont {Schumacher},\ and\ \citenamefont
  {Binder}}]{Carcamo:20}%
  \BibitemOpen
  \bibfield  {author} {\bibinfo {author} {\bibfnamefont {M.}~\bibnamefont
  {Carcamo}}, \bibinfo {author} {\bibfnamefont {S.}~\bibnamefont
  {Schumacher}},\ and\ \bibinfo {author} {\bibfnamefont {R.}~\bibnamefont
  {Binder}},\ }\bibfield  {title} {\bibinfo {title} {Transfer function
  replacement of phenomenological single-mode equations in semiconductor
  microcavity modeling},\ }\href {https://doi.org/10.1364/AO.392014} {\bibfield
   {journal} {\bibinfo  {journal} {Appl. Opt.}\ }\textbf {\bibinfo {volume}
  {59}},\ \bibinfo {pages} {G112} (\bibinfo {year} {2020})}\BibitemShut
  {NoStop}%
\bibitem [{\citenamefont {Heeger}\ \emph {et~al.}(1988)\citenamefont {Heeger},
  \citenamefont {Kivelson}, \citenamefont {Schrieffer},\ and\ \citenamefont
  {Su}}]{RevModPhys.60.781}%
  \BibitemOpen
  \bibfield  {author} {\bibinfo {author} {\bibfnamefont {A.~J.}\ \bibnamefont
  {Heeger}}, \bibinfo {author} {\bibfnamefont {S.}~\bibnamefont {Kivelson}},
  \bibinfo {author} {\bibfnamefont {J.~R.}\ \bibnamefont {Schrieffer}},\ and\
  \bibinfo {author} {\bibfnamefont {W.~P.}\ \bibnamefont {Su}},\ }\bibfield
  {title} {\bibinfo {title} {Solitons in conducting polymers},\ }\href
  {https://doi.org/10.1103/RevModPhys.60.781} {\bibfield  {journal} {\bibinfo
  {journal} {Rev. Mod. Phys.}\ }\textbf {\bibinfo {volume} {60}},\ \bibinfo
  {pages} {781} (\bibinfo {year} {1988})}\BibitemShut {NoStop}%
\bibitem [{\citenamefont {Hatsugai}\ and\ \citenamefont
  {Fukui}(2016)}]{hatsugai2016bulk}%
  \BibitemOpen
  \bibfield  {author} {\bibinfo {author} {\bibfnamefont {Y.}~\bibnamefont
  {Hatsugai}}\ and\ \bibinfo {author} {\bibfnamefont {T.}~\bibnamefont
  {Fukui}},\ }\bibfield  {title} {\bibinfo {title} {Bulk-edge correspondence in
  topological pumping},\ }\href {https://doi.org/10.1103/PhysRevB.94.041102}
  {\bibfield  {journal} {\bibinfo  {journal} {Phys. Rev. B}\ }\textbf {\bibinfo
  {volume} {94}},\ \bibinfo {pages} {041102(R)} (\bibinfo {year}
  {2016})}\BibitemShut {NoStop}%
\bibitem [{\citenamefont {Kartashov}\ \emph {et~al.}(2006)\citenamefont
  {Kartashov}, \citenamefont {Vysloukh},\ and\ \citenamefont
  {Torner}}]{PhysRevLett.96.073901}%
  \BibitemOpen
  \bibfield  {author} {\bibinfo {author} {\bibfnamefont {Y.~V.}\ \bibnamefont
  {Kartashov}}, \bibinfo {author} {\bibfnamefont {V.~A.}\ \bibnamefont
  {Vysloukh}},\ and\ \bibinfo {author} {\bibfnamefont {L.}~\bibnamefont
  {Torner}},\ }\bibfield  {title} {\bibinfo {title} {Surface gap solitons},\
  }\href {https://doi.org/10.1103/PhysRevLett.96.073901} {\bibfield  {journal}
  {\bibinfo  {journal} {Phys. Rev. Lett.}\ }\textbf {\bibinfo {volume} {96}},\
  \bibinfo {pages} {073901} (\bibinfo {year} {2006})}\BibitemShut {NoStop}%
\end{thebibliography}%

\newpage

\section*{I. Dependence of edge states on the modulation amplitude of the double waves}
Besides the ratio of the intra-wave and inter-wave separations, the modulation amplitude $d$ of the chain waves also strongly influences the edge states as can be seen in Fig. \ref{fig:s1}(a). When $d$ is decreased to the point at which the modulated waves cannot be easily recognized, that is, when the two chains are almost parallel to each other [Fig. \ref{fig:s1}(b)], 0-0 state and $\pi$-$\pi$ state slightly permeate through the bulk potential wells [Fig. \ref{fig:s1}(c,f)], whereas the $\pi$-0 and 0-$\pi$ edge states become more localized in the edge potential wells as shown in Fig. \ref{fig:s1}(d,e). 
\begin{figure}[H]
  \centering
   \includegraphics[width=1\columnwidth]{figS1.png}
  {FIG. S1. {\bf Influence of the modulation amplitude of the double-wave chain on the edge states.} (a) Distribution of the double-wave potential chain with $a=3$ $\mu$m, $A=2$ $\mu$m, and $d=0.1$ $\mu$m. (b) Dependence of the eigenenergies on the modulation amplitude $d$. Here $A=2$ $\mu$m and $a=3$ $\mu$m. Red lines indicate the edge states and black lines are the bulk states. (c-f) Density (left) and phase (right) distributions of the eigenstates in the double-wave chain presented in (a), corresponding to the symbols in (b), respectively.
  }
  \label{fig:s1}
\end{figure}

\newpage

\section*{II. Configuring different edge states}
Figure \ref{fig:s2} is the extended version of Fig. 2, where more examples of the chain distributions and the corresponding edge states are shown. It is clearly seen that the edge states appear as long as the edge potential wells are the most or lest separated ones in the double waves, even if the phase shift $\theta$ is an arbitary value [see Fig. S2(j)]. 
\begin{figure}[H]
  \centering
   \includegraphics[width=1.0\columnwidth]{figS2.png}
 {FIG. S2. {\bf Edge states in differently structured double-wave chains.} (a) Dependence of the eigenenergies of the linear modes on the phase shift of the chains $\theta$. Here $a=3$ $\mu$m, $A=2$ $\mu$m, and $d=0.5$ $\mu$m. Red lines indicate the right edge states, green lines indicate the left edge states, and black lines are the bulk states. (b) Enlarged view of the selected area in (a). (c-k) Profiles of the chain (upper) and the corresponding density of the selected edge states (lower) at different phase shifts: (c) $\theta=0.1\pi$, (d) $\theta=0.25\pi$, (e) $\theta=0.45\pi$, (f) $\theta=0.5\pi$, (g) $\theta=0.75\pi$, (h) $\theta=\pi$, (i) $\theta=1.25\pi$, (j) $\theta=1.525\pi$, and (k) $\theta=1.9\pi$.
  }
  \label{fig:s2}
\end{figure}

\newpage

\section*{III. Edge states in double-wave chains with different lengths}
To realize different edge states, one can also cut or add some potential wells in the chain. Here we consider the chain in Fig. 1(a) as the standard one. By cutting four potential wells at the right edge as shown in Fig. \ref{fig:s3}(a), similar edge states like the one shown in Fig. \ref{fig:s2}(h) form [Fig. \ref{fig:s3}(d,g)]. If two potential wells at the left edge is removed [Fig. \ref{fig:s3}(b)], two types of edge states appear in the same chain with one at the left edge, where the potential wells are the least separated, and the other one at the right edge, where the potential wells are the most separated [Fig. \ref{fig:s3}(e,h)]. With adding two more potential wells to the left side of the chain to make it symmetric along $x$ direction [Fig. \ref{fig:s3}(c)], both edges are allowed to be occupied by the condensates with the same profile as shown in [Fig. \ref{fig:s3}(f,i)].
\begin{figure}[H]
  \centering
   \includegraphics[width=1.0\columnwidth]{figS3.png}
  {FIG. S3. \textbf{Potential chains with different lengths and the corresponding edge states.} (a-c) Profiles of the double-wave chains with the same lattice parameters, $a=3$ $\mu$m, $A=2$ $\mu$m, and $d=0.5$ $\mu$m, but different lengths along the horizontal ($x$) direction. (d-f) Eigenfrequencies of the linear eigenmodes versus the eigenstate numbers. (g-i) Selected edge states of the chains (a-c), corresponding to the symbols in (d-f), respectively. 
  }
  \label{fig:s3}
\end{figure}

\newpage

\section*{IV. Eigenmodes at the inter-wave separation $A=2.5$ $\mu$m}
When the inter-wave separation $A=2.5$ $\mu$m, the density of the condensate of the edge states in the four potential wells at the right edge are almost identical [Fig. \ref{fig:s4}(b,c)]. Figure \ref{fig:s4}(d-g) shows the eigenmodes (bulk states) with the energies below the two edge states. The condensates in such bulk states can be suppressed and consequently the condensate at the left edge is enhanced, forming new edge states, i.e., nonlinearity-induced edge states (see Fig. 3 in the main text). 
\begin{figure}[H]
  \centering
   \includegraphics[width=1.0\columnwidth]{figS4.png}
 {FIG. S4. \textbf{Linear eigenmodes.} (a) Dependence of the eigenenergies of the linear eigenmodes in the double-wave chain on the inter-wave separation $A$. Here $a=3$ $\mu$m, $d=0.5$ $\mu$m, and $\theta=0$. Red lines indicate the edge states and black lines are the bulk states. (b-g) Density (left) and phase (right) distributions of the selected eigenmodes, corresponding to the symbols in (a), respectively.
  }
  \label{fig:s4}
\end{figure}

\newpage

\section*{V. Multistable edge states}
From Fig. 1(b) or Fig. \ref{fig:s4}(a), one can see that when $a/A=1.5$, the edge states are more isolated from the bulk states, so that in this case, the edge state [Fig. \ref{fig:s5}(c,e)] can be solely excited without the excitation of the left edge state [c.f. Fig. 3(b)]. The left edge can still be excited when the pump intensity is stronger as shown in Fig. \ref{fig:s5}(a,f). When the pump intensity is stronger, the 0-$\pi$ edge state at the right edge is still slightly influenced by the $\pi$-$\pi$ state [see the density distribution in Fig. \ref{fig:s5}(d)], even there is a significant energy gap between them. It is clear that under some specific pump intensity range, the bistability remains in this case as shown in Fig. \ref{fig:s5}(a).
\begin{figure}[H]
  \centering
   \includegraphics[width=1.0\columnwidth]{figS5.png}
{FIG. S5. \textbf{Bistable edge states.} (a) Dependence of the peak density of the nonlinear edge states on the amplitudes of the pump with photon energy $\hbar\omega=$-3.5982 meV [the chain's parameters are $a=3$ $\mu$m, $A=2$ $\mu$m, $d=0.5$ $\mu$m, and $\theta=0$, the same with the one shown in Fig. 1(a)]. (b-f) Density (left) and phase (right) profiles of the states marked in (a), respectively.
  }
  \label{fig:s5}
\end{figure}

\newpage

\section*{VI. Chern index of the even and odd subspaces}
In this section, the Chern index relation between the even and odd subspaces is derived. Following the formula used in the main text, the vertical couplings are $t+\lambda \cos(2\pi p /(N+1)n+\theta)$  and the horizontal couplings are $1$  . We know that the total Hamiltonian of two uncoupled AAH models reads:
$\hat{H}_{BD}=\begin{pmatrix}h_+ & 0 \\ 0 & h_-\end{pmatrix}$.
It is beneficial to write the position informed momentum space Hamiltonian of the odd subspace:
\begin{equation}
 h_-(k_x)=t+\begin{bmatrix} \Lambda_1 &e^{ik_x/N} & 0 & \dots &0&  e^{-ik_x/N} \\ e^{-ik_x/N} & \Lambda_2  \\ 0 & & & & \ddots\\\vdots &  & &\ddots & \ddots & e^{ik_x/N} \\ 0\\ e^{ik_x/N} & 0& \dots & 0 & e^{-ik_x/N} & \Lambda_{N}\end{bmatrix}\,.   
\end{equation}
In this equation, we have $\Lambda_n=\lambda \cos(2\pi P/(N+1)n+\theta)$. Furthermore for the even subspace we have, $h_+: h_- |\Lambda_n \rightarrow -\Lambda_n, t \rightarrow -t$, meaning the modulations are $\pi$ in phase between two subspaces. Next, eigenstates of the odd subspace satisfy $h_- |\Psi_{-,n}\rangle=E_{-,n}|\Psi_{-,n}\rangle$. It is apparent we have the following equivalence relation: $h_-(k_x)|\Psi_{-,n}(k_x)\rangle=-h_+(k_x+N\pi)|\Psi_{-,n}(k_x)\rangle=E_{-,n}(k_x)|\Psi_{-,n}(k_x)\rangle $, which leads us to equivalence: 
\begin{equation}
\begin{aligned}
    E_{+,n}(k_x)=-E_{-,N-n+1}(k_x+N\pi)\,, \\
    |\Psi_{+,n}(k_x)\rangle=|\Psi_{-,N-n+1}(k_x+N\pi)\rangle\,.
\end{aligned}
\end{equation}
This relation allows us to conclude $C_{+,n}=C_{-,N-n+1}$.
\end{document}


