% ****** Start of file apssamp.tex ******
%
%   This file is part of the APS files in the REVTeX 4.2 distribution.
%   Version 4.2a of REVTeX, December 2014
%
%   Copyright (c) 2014 The American Physical Society.
%
%   See the REVTeX 4 README file for restrictions and more information.
%
% TeX'ing this file requires that you have AMS-LaTeX 2.0 installed
% as well as the rest of the prerequisites for REVTeX 4.2
%
% See the REVTeX 4 README file
% It also requires running BibTeX. The commands are as follows:
%
%  1)  latex apssamp.tex
%  2)  bibtex apssamp
%  3)  latex apssamp.tex
%  4)  latex apssamp.tex
%
\documentclass[%
 reprint,
%superscriptaddress,
%groupedaddress,
%unsortedaddress,
%runinaddress,
%frontmatterverbose, 
%preprint,
%preprintnumbers,
%nofootinbib,
%nobibnotes,
%bibnotes,
 amsmath,amssymb,
 aps,
 prl,
%pra,
%prb,
%rmp,
%prstab,
%prstper,
%floatfix,
]{revtex4-2}

\usepackage{graphicx}% Include figure files
\usepackage{dcolumn}% Align table columns on decimal point
\usepackage{bm}% bold math
\usepackage{xcolor}


%\usepackage{hyperref}% add hypertext capabilities
%\usepackage[mathlines]{lineno}% Enable numbering of text and display math
%\linenumbers\relax % Commence numbering lines

%\usepackage[showframe,%Uncomment any one of the following lines to test 
%%scale=0.7, marginratio={1:1, 2:3}, ignoreall,% default settings
%%text={7in,10in},centering,
%%margin=1.5in,
%%total={6.5in,8.75in}, top=1.2in, left=0.9in, includefoot,
%%height=10in,a5paper,hmargin={3cm,0.8in},
%]{geometry}

\newcommand{\sarah}[1]{\textcolor{red}{\textbf{SL: #1}}}


\begin{document}



\preprint{APS/123-QED}

\title{Manuscript Title:\\with Forced Linebreak}% Force line breaks with \\
\thanks{A footnote to the article title}%

\author{Ann Author}
 \altaffiliation[Also at ]{Physics Department, XYZ University.}%Lines break automatically or can be forced with \\
\author{Second Author}%
 \email{Second.Author@institution.edu}
\affiliation{%
 Authors' institution and/or address\\
 This line break forced with \textbackslash\textbackslash
}%

\collaboration{MUSO Collaboration}%\noaffiliation

\author{Charlie Author}
% \homepage{http://www.Second.institution.edu/~Charlie.Author}
\affiliation{
 Second institution and/or address\\
 This line break forced% with \\
}%
\affiliation{
 Third institution, the second for Charlie Author
}%
\author{Delta Author}
\affiliation{%
 Authors' institution and/or address\\
 This line break forced with \textbackslash\textbackslash
}%

%\collaboration{CLEO Collaboration}%\noaffiliation

\date{\today}% It is always \today, today,
             %  but any date may be explicitly specified

\begin{abstract}
An article usually includes an abstract, a concise summary of the work
covered at length in the main body of the article. 
\begin{description}
\item[Usage]
Secondary publications and information retrieval purposes.
\item[Structure]
You may use the \texttt{description} environment to structure your abstract;
use the optional argument of the \verb+\item+ command to give the category of each item. 
\end{description}
\end{abstract}

%\keywords{Suggested keywords}%Use showkeys class option if keyword
                              %display desired
\maketitle

%\tableofcontents

\section{\label{sec:level1}Introduction}
Dissipative structures  \cite{Goldbeter18,Tiezzi2008,Belintsev_1983} are an example of ordered non-equilibrium (quasi-) stationary states (NESS) forming in systems operating far from thermodynamic equilibrium. They appear as spatio-temporal patterns \cite{Cross93,Gollub1999}, stationary \cite{Maini2012}, moving in space\cite{travelling waves} or oscillating\cite{swift-hohenberg}. Their formation in continuous systems described by an order parameter $\phi$ can be modeled by macroscopic kinetic equations for, which have the form of PDEs \cite{MALOMED1984}
\begin{equation}\label{equ:dyn}
    \partial_t \phi = F(\phi,\nabla\phi)\,.
\end{equation}
Permanent structured states are accordingly
found as fixed points of Eq. \eqref{equ:dyn} respectively also as solutions of periodic character. Understanding the conditions for the occurrence of different classes of such stationary states, has motivated a lot of research in the past \cite{}.
When it comes to the question of where dissipation of energy plays a decisive role in the formation of these structures, temporal stationary solutions are of special interest. While the formation of ordered static states is also widely observed in equilibrium, they are an exclusive feature of of non-equilibrium. More precisely of a particularly drastic type of violation of equilibrium. These are possible only if the PDE is not of gradient type \cite{Newell2003} 
i.e., when the dynamical operator cannot be expressed in the form $F=K\delta_\phi \mathcal{F}$, where $K$ is a differential operator and $\mathcal{F}$ is a functional of $\phi$. This condition excludes any kind of system in thermodynamic equilibrium, where the corresponding $\mathcal{F}$ would simply be the coarse-grained free energy.
%and linearly stable fixed points would be its local mimima.
%implicitly implying the existence of non-relaxational currents
However, systems that are non-equilibrium on the microscopic scale but have a
continuum-representation in terms of a pseudo free energy \cite{MIPS} also exist.
% siehe ersten absatz bei Nardini
%, meaning that all traces of microscopic dissipation have been washed out in the coarse graining procedure. 
%Usually the influence of noise is not explicitly considered
Therefore the formation of temporal steady states is restricted to a special class of non-equilbrium we call non-integrable. This is characterized by the existence of non-relaxational fluxes also on the mesoscopic scale.
% as opposed to a form of pseudo-equilibrium where all traces of microscopic dissipation have been lost in the coarse graining procedure.

Our goal is analyzing this special form of violation of thermodynamic equilibrium from the perspective of fluctuations with respect to a stable stationary state.
For this purpose we study the fluctuating hydrodynamic equivalent of Eq. \eqref{equ:dyn} which is obtained by adding an appropriate noise term, such that for $F=K\delta_\phi \mathcal{F}$, the resulting statistical field theory would obey a fluctuation dissipation relation. Hereby we make use of a quantity that only has recently introduced in the context of statistical field theories, called the stochastic entropy production rate (EPR). This is formally defined identically to the entropy production from stochastic thermodynamics of individual degrees of freedom. A particular focus is on conditions in the vicinity of a phase transition and their irreversible character quantifiable by the EPR.
We are the first to study dynamical phase transitions in non-equilibrium models from entropic perspective

Specifically,
%For this purpose
we investigate a stochastic PDE that is an non-equilibrium version of the two-component Cahn-Hilliard-Cook equation. It belongs to a more general class of models that have the following structure.
\begin{equation}\label{equ:mod}
    \partial_t \phi=-\nabla(\boldsymbol{J}^\mathrm{H}+\sqrt{2\epsilon}\boldsymbol{\Lambda})
    \qquad
    \boldsymbol{J}^\mathrm{H}=-\nabla\left( \frac{\delta\mathcal{F}}{\delta\phi}+ \mu_a\right)\,.
\end{equation}
The non-equilibrium part of the hydrodynamic current $\boldsymbol{J}^\mathrm{H}$ here originates from a non-equilibrium chemical potential $\mu_a$, which is distinguished a Jacobian that is not totally symmetric with respect to at least one state $\phi$.  

To define a phase transition between to NESS, we use the concept of bifurcation, which comes from the theory of dynamical systems. Accordingly, a phase transition occurs when the fixed point structure of $\nabla\boldsymbol{J}^\mathrm{H}$ qualitatively changes, e.g. the number of fixed points increases or decreases, meaning that als the qualitative behaviour of the stochastic process will drastically change. Since in the following we deal with structured states that form from a homogeneous state, we assume that $\phi=0$ is always a fixed point, which is not necessarily stable. Thus, we exclude the coupling of $\phi$ to an external state-independent potential.

%Noise considered if finite scale or finite temperature. 

%From the statistical mechanics/fluctuating hydrodynamics perspectives non-existence of (pseudo) free energy for the continuum description marks a 
\begin{figure}[t]
\centering
\includegraphics[width=.45\textwidth]{phase_diagram.png}% Here is how to import EPS art
\caption{\label{fig:pd} Phase diagram for $\kappa=0.01$, $\gamma_A=0.015$ and $\chi_B=0.05$. The mixed phase is marked in yellow, the static demixed phase in blue and the travelling wave phase in red. The dotted region indicates configuration where a non-zero fixed point exists but is unstable.}
\end{figure}
\section{Model}
We study a SDE of the form of Eq. \eqref{equ:mod} in one spatial dimension is $d=1$ with a two-component field $\phi=(\phi_A,\phi_B)^T$ on the domain $I=[0,L]$ where we assume periodic boundary conditions.
The integrable current stems from a $\phi^4$-free-energy of the concrete form
\begin{align}
    \mathcal{F}=
    \int\limits_I\mathrm{d}r\left(f(\phi_A,\phi_B)+\gamma_A\vert \nabla\phi_A\vert^2 \right)\,.
\end{align}
with
\begin{align}
    f(x,y)=\frac{1}{2}x^2+\frac{1}{12}x^4+\frac{1}{2}y^2+\kappa xy
\end{align}
The non-equilibrium chemical potential is linear and given by 
\begin{align}
    \mu^\mathrm{a}=\delta(- \phi_A, \phi_B )\,.
\end{align}
In component form the SDE reads
\begin{align}\label{equ:1}\nonumber
\partial_t\phi_A =& \nabla\left( (\chi_A+ \phi_A^2 )\nabla \phi_A-\gamma_A\nabla^3\phi_A +(\kappa-\delta)\nabla\phi_B  \right)\\
&+\sqrt{ 2\epsilon}\nabla\Lambda_A \,,\nonumber
\\\nonumber
\partial_t\phi_B = &\nabla\left(\chi_B\nabla \phi_A +(\kappa+\delta)\nabla\phi_A  \right)
\\
&+\sqrt{ 2\epsilon}\Lambda_B \,.
\end{align}

The fixed point structure of Eq. \eqref{equ:1} for $\epsilon=0$ has already been studied by \cite{You_2020}. They find that stationary solutions are approximately of the form
$\phi_\mu(t)=\mathcal{A}^0_\mu\cos(x+\theta_\mu(t))$ and can be categorized as
mixed ($\mathcal{A}^0_\mu=0$), static demixed   ($\mathcal{A}^0_\mu>0$, $\theta(t)=\mathrm{const}$) and travelling ($\mathcal{A}^0_\mu>0$,   $\partial_t\theta=\mathfrak{v}$). While both types of static solution are found for any strength of non-reciprocal coupling $\delta$, travelling wave solutions are only observed above a critical threshold given by $\delta_c=\sqrt{\chi_B^2+\kappa^2}$. 
%The dynamical state can be reached either coming from the demixed state or from the mixed state. The phase diagram is presented in Fig. \ref{fig:pd}.
Based on the knowledge of the zero noise solution, the covariance function of $\phi$ for small $\epsilon$ can be calculated via a small noise approximation, as detailed in App. \ref{app:cov}.
\section{Entropy production}
Recently, it has been shown that the concept of stochastic entropy production, which was originally introduced for microscopic systems consisting of single degrees of freedom \cite{Seifert2005}, can also be applied to field models \cite{Li_2021,Nardini2017}. Analogously, the entropy production $\Delta\mathcal{S}[\phi,0,T]$ assigned to single trajectory $\{\phi(t)_{t\in [0,T]}\}$ is defined as the logarithmic difference of forward and backward path probabilities. A detailed definition is given in App. \ref{app:ep}. 
Instead of the entropy production often the average entropy production rate (EPR) is considered which is defined as
\begin{equation}\label{equ:diff1}
  \overline{\frac{\mathrm{d}}{\mathrm{d}t}\mathcal{S}}= \lim\limits_{h\rightarrow 0}\frac{\langle\Delta \mathcal{S}[\phi,t,t+h]\rangle}{h}\,.
\end{equation}
This quantity serves as a measure of the breaking of detailed balance at time $t$. While for gradient type models the EPR vanishes in for $t\rightarrow\infty$, non-integrable models have a strictly positive steady state EPR. 
Besides the stationary probability $\mathcal{P}_{\mathrm{ss}}[\phi]$, the EPR is another property of the stationary state/ensemble that can be studied for such models. 

As a central result, it was found that for a model of the form of Eq. \eqref{equ:mod} the stationary EPR is given by \cite{Nardini2017}
\begin{align}\label{equ:EPR1}
   \epsilon \overline{\frac{\mathrm{d}}{\mathrm{d}t}\mathcal{S}}=\int\limits_V\mathrm{d}r\langle \boldsymbol{J}^H\cdot\nabla\mu_a \rangle
\end{align}
Alternatively the steady state EPR can also be expressed in terms of the so called assymetric current \cite{Li_2021} which is defined as
\begin{align}
   \boldsymbol{J}^{\mathrm{A}}(\phi)\equiv\boldsymbol{J}^{\mathrm{H}}(\phi)- \epsilon\frac{\nabla\frac{\delta }{\delta\phi}\mathcal{P}[\phi]}{\mathcal{P}[\phi]}\,.
\end{align}
Here the expression reads
\begin{align}
    \epsilon\overline{\frac{\mathrm{d}}{\mathrm{d}t}\mathcal{S}}=\int\limits_V\mathrm{d}r\langle \vert\boldsymbol{J}^\mathrm{A} \vert^2\rangle\,,
\end{align}
This representation does not require a decomposition of the form of Eq. \eqref{equ:mod}. 
%It illustrate that irreversibility is associated with the presence of a non-zero asymmetric current $\boldsymbol{J}^\mathrm{A}\equiv \boldsymbol{J}^H-\boldsymbol{J}^\mathrm{S}$in which the hydrodynamic part of the dynamics differs from its time reversed version \cite{Bouchet_2016}. 
For Our two-component model we can explicitly calculate the asymmetric current and therefore study the relation ship to its counterpart the symmetric current $\boldsymbol{J}^\mathrm{S}\equiv
 \boldsymbol{J}^H-\boldsymbol{J}^\mathrm{A}$.

Finally we introduce another decomposition-independent representation of the EPR given by
\begin{align}\label{equ:eprown}
    \nonumber
  \overline{\frac{\mathrm{d}}{\mathrm{d}t}\mathcal{S}}= \int\limits_V\mathrm{d}r
  \langle\,\frac{\vert\boldsymbol{J}^{\mathrm{H}}\vert^2}{\epsilon}
   +   \mathrm{tr}\frac{\delta}{\delta\phi}\nabla\cdot\boldsymbol{J}^{\mathrm{H}}\,\rangle
    \,.
\end{align}
The proof is given in App. \ref{app:ei}. From this representation it is apparent there is a close connection between a sub-linear scaling of $\vert\boldsymbol{J}^{\mathrm{H}}\vert^2$ in $\epsilon$
which is the characteristic of a continuous transition from a static to a dynamical state and a divergence of the EPR for $\epsilon \rightarrow 0$. 
This we will further investigate in Sec. \ref{sec:sing}.

\section{Results}
We find that the general expression for the EPR is given by (See App. \ref{sec:eprg})
\begin{align}
    \epsilon\overline{\frac{\mathrm{d}}{\mathrm{d}t}\mathcal{S}}=2\delta\int\limits_V\mathrm{d}r\langle\, ((\kappa+\delta)\nabla\phi_A+\chi_B\nabla\phi_B) \cdot\nabla \phi_A,\rangle
\end{align}
In terms of the Fourier representation this reads \begin{align}
    \epsilon\overline{\frac{\mathrm{d}}{\mathrm{d}t}\mathcal{S}}=4L\delta\sum\limits_{j>0} q_j^2(\alpha_B\mathrm{Re}\langle  \hat{\phi}^j_{A}\hat{\phi}^{-j}_{B}\rangle+ (\kappa+\delta) \langle \vert\hat{\phi}_{A}^j\vert^2\rangle)
\end{align}
As shown in App. \ref{app:cov} the terms in this series approach a finite positive value for $j\rightarrow \infty$.
Therefore the above expression turns out to be divergent. This fact is not surprising, considering that the implementation of randomness in our model by a space-time white noise allows states with arbitrarily large variation, all contributing to entropy production. Therefore, the divergence is already caused by the fact that,
formally are dealing with an infinite dimensional process where each dimension adds a similar large contribution to the total EPR and is no exclusive feature of the specific model. 

Clearly the possibility of states with infinite variation conflicts with the idea of our model representing the idealized version of a coarse grained description of a microscopic model that is one the only valid above certain cut-off length scale. And in fact, modifying the noise kernel such the cut-off is respected, would also render the EPR finite. 
However, in our case it is even reasonable to truncate the series after the first term, with the following argument
We must not forget that our model represents only qualitative properties of a physical system.
Therefore we mus distinguish sharply between physically relevant results and mathematical artifacts, also with regard to entropy production. For our model we are dealing with the special situation that the essential information about the macroscopic demixing, such as magnitude or position of the demixed state are encoded in the first mode \cite{You_2020}. Although the model also provides information about higher modes, which represent deformations of the idealized state, we cannot assume that their properties are correctly described in it.
%The higher modes, on the other hand, only correspond to corrections or deformations, which, however, have no essential character with respect to the formation of the demixed phase. 
Therefore we only consider the EPR of trajectories projected to the first mode.
An exception is the mixed state, where the higher modes are also relevant, since there is no dominant mode.

%What is interesting is the contribution of the low modes to the EPR 
%thererfore it would be more appropriate to To 
%Die Einführung eines cut-off ist jedoch nicht unbedingt notwendig. idenfikation relevanter moden

In the following, we will show that also for the stochastic model, within certain approximations, it is possible to consider the dynamics of individual modes in isolation which makes it possible to calculate their EPR directly.
%This corresponds to a kind of scale separation. For example, if we consider only the first mode for our model, this provides information about the essential macroscopic properties of the demixed state, such as its amplitude or phase. 

The central assumption is the small noise strength, such that all of the following results are valid only in the limit $\epsilon\rightarrow 0$. 


%This is the EPR of a process consisting of an an infinite number of coupled sub-processes 
%In principle requires the solution of an infinite dimensional linear equation also containing higher moments. We give arguments why for $\epsilon\rightarrow 0$ the dynamics of each mode is approximately decoupled also for $\phi_0\neq 0$. 
%This is done by identifying groups of slow and fast variables. As a result moments can be obtain by solving only a finite dimensional matrix equation for each $j$. At the same time the original process can be replaced by one that has the same moments and generates the same overall and is a collection of mutually independent processes for each $q$. This allows to interpret the components of the sum as the EPR assigned to each mode.
\subsection{$\phi_0=0$}\label{sec:hom}
Applying the standard procedure of the small noise approximation we find that for $\epsilon\rightarrow0$ the dynamics in Fourier space is decoupled for different wavenumbers. The dynamical equation for each mode is given by 
\begin{align}\label{equ:eomm}
\partial_t\hat{\phi}^j(t)=-q^2_jA_j\hat{\phi}^j(t)+\xi^j\,,
\end{align}
with
\begin{align}
 A_j=
\begin{pmatrix}
\chi_A+\gamma_A q_j^2  & \kappa-\delta\\
\kappa+\delta & \chi_B\\
\end{pmatrix} \,.
\end{align}
and $\langle\xi^j,\xi^k\rangle=2\epsilon/L q_j^2\delta_{j,-k}$\,.
Therefore, in this case the EPR is the sum of contributions $\overline{\frac{\mathrm{d}}{\mathrm{d}t}\mathcal{S}^j}$ stemming from independent finite dimensional sub-processes.
Applying the results of \cite{loos2020thermodynamic} on entropy production of system of individual degrees of freedom we find
\begin{align}
    \overline{\frac{\mathrm{d}}{\mathrm{d}t}\mathcal{S}^j}=8\delta^2\frac{q_j^2}{\mathrm{tr}A_j}\,,
\end{align}
which also corresponds to the expected value of the squared components of the asymmetric current.
On the other hand, for the average intensity of the components of the symmetric current we find
\begin{align}
     \langle \vert\boldsymbol{J}^\mathrm{S}(j)\vert^2\rangle=q_j^2\mathrm{tr}\,A_j\,.
\end{align}
Therefore of the asymmetric and the symmetric current are inversely related.
Since the the squared symmetric current is a measure of tendency to restore the ground state it can be interpreted as the stiffness parameter of each mode. From this we can derive the first general conclusion that in a less stiff
configuration, the occurrence of irreversible dynamics is promoted.

However, we can also ask how the irreversible dynamic, i.e. the source(s) of entropy production presents itself in detail.
Which physical phenomenon is opposed to the increase, respectively the divergence of the EPR?
Here a distinction can be made based on the basis of the properties of the eigenvalues of the matrix $A_j$, denoted by $\{\lambda_i(j)\}$. 
If the EPR at the transition remains finite, there is always a neighborhood in which it holds that $\mathrm{Im}\lambda_i=0$.
In this case is is possible to find an orthogonal \footnote{The orthogonality results in the fact that the transformed entries of the noise term remain independent and at equal temperature. } transformation $\hat{\phi}^j\mapsto \tilde{\phi}^j$ such that equations of motion in the transformed coordinates read 
\begin{align}
\partial_t\tilde{\phi}^j(t)=-q^2_jA'_j\tilde{\phi}^j(t)+\xi^j\,,
\end{align}
with
\begin{align}
    A'=
    \begin{pmatrix}
     &\lambda_- &2\delta\\
     &0 &\lambda_+
    \end{pmatrix}\,.
\end{align}
This is shown in App ?.
This motion can be characterized as a persistent random walk in a harmonic potential. In context of active matter it is known as (one possible formulation of) the active Ornstein-Uhlenbeck process (AOU) \cite{Dabelow_2021}. 
For the AOU-particles, special forms of dynamics only enabled by the violation of thermodynamic equilibrium (particles are equipped with an energy consuming self-propulsion mechanism ) can be observed \cite{}.
We observe something similar for the dynamics of demixing in the non-reciprocal CHC model. In particular, it turns out that where the trajectories differ particularly strongly from the equilibrium counterpart, the irreversible
character also increasingly dominates. This is illustrated in figure \ref{fig:epsart}.

%Acts like the controller of an drive that is neither backcoupled to the deterministic part of the motion nor to the noise and must therefore be considered external  

For $\vert \mathrm{Im} \lambda \vert>0$ the matrix operator has a diagonal representation of the form
\begin{align}
    A'=\mathrm{Re}\lambda\sigma_0+\vert\mathrm{Im}\lambda\vert\sigma_1
\end{align} 
where $\boldsymbol{\sigma}$ is the vector of Pauli matrices. In App. we show that the solutions to Eq. \eqref{equ:eomm} for $\epsilon=0$ take the form of damped oscillating patterns, travelling waves or combinations of these state.
This is in line with the fact that close to the transition, we find that the EPR is given by
\begin{align}
    \epsilon\overline{\frac{\mathrm{d}}{\mathrm{d}t}\mathcal{S}^j}\sim 2L(q_j\mathrm{Im}\lambda)^2\langle\vert\phi_\mu\vert^2\rangle\,.
\end{align}
This coincides with the average EPR of a travelling state $\phi(x)=A \cos(q_jx \pm q_j^2\mathrm{Im}\lambda t)$ with $A_\mu=2\sqrt{\langle\vert\phi_\mu\vert^2\rangle}$ or a standing wave state $\phi(x)=A(t) \cos(q_jx )$ with $A_\mu(t)=4\sqrt{\langle\vert\phi_\mu\vert^2\rangle}\cos( q_j^2\mathrm{Im}\lambda t)$, caused by an external horizontally or vertically acting (periodic) force,
\begin{figure}[b]
\centering
\includegraphics[width=.45\textwidth]{prw.png}% Here is how to import EPS art
\caption{\label{fig:epsart} Trajectories of $\mathrm{Re} \phi_A$ (red) and $\mathrm{Re} \phi_B $ (blue) for $\delta=0$, $\chi_A= -0.01159$  (A) and $\delta=0.047$, $\chi_A=-0.057$ (B)
Note the different scales of the y-axis. The parameters are chosen such that $\lambda_1(A)=\lambda_1(B)$ and $\lambda_2(A)>\lambda_2(B)$,
i.e. the damping of single perturbations is weaker for the (A) configuration. Nevertheless the characteristic scale of fluctuations differs by one
magnitude. Also the increased persistence for the (B) configuration is clearly visible. }
\end{figure}
\begin{figure}[b]
\centering
\includegraphics[width=.45\textwidth]{drive.png}% Here is how to import EPS art
\caption{\label{fig:epsart2} Trajectories of $\mathrm{Re} \phi_A$ (red) and $\mathrm{Re} \phi_B $ (blue) for $\delta=0$, $\chi_A= -0.01159$  (A) and $\delta=0.047$, $\chi_A=-0.057$ (B). The periodic structure of the trajectories is clearly visible}
\end{figure}
The functional dependence of the entropy production near the phase boundary thus mirrors the fact that the dynamics, that is set in motion each time the system is found to be away from the fixed point is reminiscent of that of a driven system.
The divergence of the EPR testifies to the fact that fluctuations, also excited by random noise attain a strongly irreversible, almost deterministic character (note that the symmetrical current disappears completely).

As detailed in App \ref{app:gen} the dynamics of fluctuations with respect to $\phi_0=0$ in any non-equilbrium field theory that does not belong to the class of externally driven system is described by a non-Hermitian matrix. Since $A_j$ parametrizes all real non-Hermitian matrices in two dimensions, all conclusions drawn from our model are general for any models with two field components. Thus, for example, for a two-component system, we have found the general condition for the occurrence of fluctuations with quasi-deterministic character and we know that these will appear as periodic oscillations with a partly random character. 
\subsection{$\phi_0\neq 0$}
Here we must distinguish between temporal and static solutions.
In the travelling wave state projections of trajectories contain a deterministic motion, resulting in a divergent EPR for $\epsilon\rightarrow 0$ with
\begin{align}
    \epsilon\overline{\frac{\mathrm{d}}{\mathrm{d}t}\mathcal{S}^1}\sim \mathfrak{v}^2\int\mathrm{d}r\vert\phi_0(r)\vert^2
\end{align}
Therefore the EPR does not provide actual information about fluctuations, but about the deterministic mass flow in the dynamical ground state. 

In the following, we treat the static demixed region of the phase diagram. Here the situation is more complicated than for $\phi_0=0$. Since the mean position of the demixed state is not fixed it holds that $\langle \phi\rangle=0$ even though $\vert \phi \vert =\mathcal{A}_0+\mathcal{O}(\epsilon) $. Therefore, a small noise ansatz can only be performed after switching to the representation
\begin{align}
    \hat{\phi}^j_\mu=\mathcal{A}_\mu^j e^{iq_j \theta_\mu^j}\,.
\end{align}
Here, we find that fluctuations with different wave numbers in general do not decouple.
However, we can show that the stochastic dynamics can be separated into a vertical part that represents the fluctuations of the amplitude $\{\mathcal{A}^j\}$ of the demixed state (which we do not consider especially relevant, at least far away from the mixed-demixed phase boundary) and a horizontal part representing the motion of the mean position of the demixing profile (and the relative postion of its components) encoded in $\{\theta^j\}$ (See App. \ref{app:ap}).
Further, we argue that by making a division into fast and slow degrees of freedom, at least for the horizontal part, the original stochastic dynamics projected on the first mode can be replaced by an approximation coming in closed form. The equations of motion read
%Again, we empathize that the first mode captures the essential properties of the demixed state, while higher modes modes are associated with deformations of the profile shape, with regard to which the model has no informative value and which we therefore do not explicitly consider here. 
\begin{align}\label{equ:dynp}
    &\partial_t\theta_A^1=-q_1^2\frac{\kappa^2-\delta^2}{\alpha_B}(\theta_A^1-\theta_B^1)+\xi_{\theta_A^1}\,,\nonumber
    \\
    &\partial_t\theta_B^1=q_1^2\alpha_B(\theta_A^1-\theta_B^1)+\xi_{\theta_B^1}\,.
\end{align}
with $\langle\xi_{\theta_\mu^1}\xi_{\theta_\nu^1}\rangle=\frac{\epsilon}{L(\mathcal{A}_\mu^0)^2}\delta_{\mu\nu}$ and $\theta_B^1$ redefined as $\theta_B^1+\pi$.
Again the EPR of this process can be computed using the results of $\cite{loos2020thermodynamic}$. We find
\begin{align}
    \overline{\frac{\mathrm{d}}{\mathrm{d}t}\mathcal{S}^1_\theta}= 4\delta^2\frac{\alpha_B}{\delta_c^2-\delta^2}\,.
\end{align}
In App. \ref{sec:amp} we detail that in non-Hermitian systems with a continuous symmetry a mechanism of  amplification of perturbations is present.
There we also calculate the amplification factor $\mathfrak{a}$ associated with Eq. \eqref{equ:dynp} and find $\overline{\frac{\mathrm{d}}{\mathrm{d}t}\mathcal{S}^1_\theta}\propto\mathfrak{a}$. This suggests that the mechanism that leads to the amplification of fluctuations also generates the irreversible part of the phase dynamics. To show that this is indeed the case, we have to explicitly treat the case of continuously acting source of thermal noise.
We apply a transformation of the coordinates is necessary that resembles a change to the center-of-mass frame for a mechanical system, where role of the mass is here taken by the squared amplitude of the ground state.
It is defined as
\begin{align}
    &\theta_c:=\frac{\gamma_A\theta_A^1+\gamma_B\theta_B^1}{\bar{\gamma}}\,,
    &\Delta\theta:=\theta_A^1-\theta_B^1\,,
\end{align}
\sarah{$\Delta \theta$ same as $\Theta$}
with $\bar{\gamma}=\gamma_A+\gamma_B$ and $\gamma_\mu=\vert \mathcal{A}_\mu^0\vert^2$.
In these coordinates the equations of motion now read
\begin{align}
    \bar{\gamma}\partial_t\theta_c&=F(\Theta)+\sqrt{\bar{\gamma}\tau}\xi_{\theta_c}\,,\\\nonumber
    \\
    \partial_t\Theta&=-\lambda_2\Theta+\sqrt{\frac{\bar{\gamma}}{\gamma_A\gamma_B}\tau}\xi_{\Delta \theta}\,,
\end{align}
with
\begin{align}
    &\lambda_2=\frac{\delta_c^2-\delta^2}{\alpha_B}\,,
    &F(\Theta)=2\sqrt{\gamma_A\gamma_B}\Theta\,\delta\,,
\end{align}
and $\langle \xi_\mu(t)\xi_\nu(t')\rangle=\delta_{\mu\nu}\delta(t-t')$.
%Therefore, we have obtained an exact mapping of the linearized phase dynamics onto the dynamics of an overdamped Brownian particle with mobility $\bar{\gamma}^{-1}$ at temperature $\tau/2$ which is in addition subject to a driving force . That force is proportional to the degree of non-reciprocal coupling and the value a statistically independent 
Hence, similar to the demixed\sarah{homogeneous} phase we find that the dynamics can be mapped onto 
a persistent random walk. But this time the walker is only driven by the persistent force $F(\Theta)$ and thermal noise and is not additionally subject to a potential force. 
Here the random walk motion has a very vivid geometrical interpretation. It represents the motion of the demixing profiles. This motion changes direction after an average time $t_0=1/\lambda_2$. At the same time $1/\lambda_2$ also controls the modulus of the mean squared deterministic velocity. 
%The EPR of such a random walk es easily computed as \cite{ Masoliver2017}
%\begin{align}\label{equ:drive}
%    \overline{\frac{\mathrm{d}}{\mathrm{d}t}\mathcal{S}_{\theta_c}}=\frac{2}{\bar{\gamma}}\langle F(\Theta)^2\rangle= 4\delta^2\frac{\alpha_B}{\delta_c^2-\delta^2}
%\end{align}
%which coincides with the above result for the EPR of the phase contribution. From Eq. \eqref{equ:drive} one can see that the motion generated by the driving force $F$ is to be regarded as completely irreversible.
Having identified the persistent random walk as the source entropy production we can represent the EPR in the followin form.
Introducing the mean drift speed 
\begin{align}\label{equ:dri}
    \mathfrak{v}_\mathrm{d}:=\sqrt{\langle (F/\bar{\gamma})^2\rangle} =\sqrt{2\frac{\delta^2}{\bar{\gamma}}\mathfrak{a}}\,,
\end{align}
it can be expressed as
\begin{align}
    \epsilon\overline{\frac{\mathrm{d}}{\mathrm{d}t}\mathcal{S}_{\theta_c}}=\mathfrak{v}_\mathrm{d}^2\int\mathrm{d}r\vert \phi_0(r)\vert^2\,.
\end{align}
This expression resembles the one for the EPR of a dynamical stationary stationary (See App. ?). Only the deterministic drift speed is here replaced by the mean stochastic drift speed due to non-equilibrium driving. Therfore also in the static phase the EPR associated with the dynamics of the phase is a measure of irreversible mass flow.

A quantity with which the dynamical properties of the persiten Random Walk motion can be quantified is the mean squared displacement (MSD) of the mean phase $\theta_\mathrm{m}\equiv(\theta_A-\theta_B)/2$ defined by $\mathrm{MSD}(\theta_m)\equiv \langle(\theta_m(t_0+t)-\theta_m(t_0))^2\rangle$.
We find
\begin{align}\label{equ:difshort}
    \mathrm{MSD}(\theta_m)=
    \begin{cases}
    \left(\left(1+\frac{1}{4}\frac{(\gamma_B-\gamma_A)^2}{\gamma_A\gamma_B}\right)t+\frac{1}{2}\frac{\delta^2}{\lambda_2}t^2\right)\frac{\tau}{\bar{\gamma}} 
    &t\ll t_0\\
     \left(\left(1+\frac{\delta^2}{\lambda_2^2}\right)t\right)\frac{\tau}{\bar{\gamma}}
    &t\gg t_0
    \end{cases}
\end{align}
Therefore, on timescales shorter than the persistence time, nonzero non-reciprocal coupling leads to a combination of diffusive and ballistic motion. The ballistic speed is given by $\mathfrak{v}_B=\delta^2\tau/(2\lambda\bar{\gamma})$. 

On time scales much larger than the persistence time non-reciprocal coupling only leads to enhanced diffusive motion with diffusion constant
$D=(1+\delta^2/\lambda^2)/(2\bar{\gamma})$.
%The inverse dependence of the drift speed on the parameter $\bar{\gamma}$ that measures demixing could also be observed in the data from the numerical simulation of Eq. \eqref{equ:1}. Exemplary space time plots of this data are presented in Fig. \ref{fig:snap1}.
%Symmetric current in this base. 
\section{Singularities of the EPR}\label{sec:sing}
As the analysis of our model has shown, there seems to be a close relationship between different types of dynamic phase transitions and a divergence of the EPR. We now want to illuminate this from a more general point of view.
For this purpose we assume that, as for our model within a certain approximation, a mapping to a finite dimensional system of linearly coupled degrees of freedom is possible such that the dynamical equation in generalized coordinates $\phi$ reads
\begin{align}
    \partial_t\phi_\mu=\sum\limits_{\nu}M_{\mu\nu}\phi_\nu+\xi_\mu
\end{align}
with $\langle\xi_\mu(t)\xi_\nu(t')\rangle=\delta_{\mu\nu}\delta(t-t')$. The detailed analysis is carried out in App. ?.
As a general result we find that a divergence of the EPR can only occur at a phase transition, i.e. for a configuration where at least for a single eigenvalue of $M$ it holds that $\mathrm{Re}\lambda_0=0$ resulting in a divergence of fluctuations of this mode. Transitions where it simultaneously holds that $\mathrm{Im}\lambda_0=0$ and which therefore resemble classical critical points are marked by a finite EPR. Eq. \eqref{equ:ptrate} implies the ratio of the square of generalized hydrodynamic mass flow $\varphi=M\phi$ and noise strength remains finite which suggest that the transition connects to static states.
On the other hand, where this ratio diverges, which is the case at a continuous transition from a static to a dynamical state (which is  characterized by non-zero mass flow for $\epsilon=0$), also a divergence of the EPR must be observed. 
$\varphi/\epsilon$ can be seen as a measure of the ability of the system to generate mass flow if triggered by noise. Therefore an unlimited increase in this capability is always accompanied by a divergence of the EPR, i.e. by dominance of irreversible dynamics and vice versa. 

In detail there are two scenarios where a divergence of the EPR can occur. 
The first is a transitions that can be characterized as Hopf-Bifurcations, i.e. where it holds that $\vert\mathrm{Im}\lambda_0\vert>0$ Here we find that close to the transition it holds that
\begin{align}
    \overline{\frac{\mathrm{d}}{\mathrm{d}t}\mathcal{S}}= \mathrm{Im}\lambda_i^2\frac{\langle\,\vert\hat{\phi}\vert^2\,\rangle}{\epsilon}+\mathcal{O}(1)
\end{align}
The second scenario is a so called exceptional point (EP) transition. Here we find that close to the transition it holds that
\begin{align}
    \overline{\frac{\mathrm{d}}{\mathrm{d}t}\mathcal{S}}\sim\frac{1}{\lambda_0}
\end{align}
In case $M$ has a zero eigenvalue ,i.e. the system has a continuous symmetry and the EP  occurs as the fusion of a formerly linear independent mode with the Goldstone mode $1\lambda_0$ can be interpreted a the strength of a effective non-equilibrium driving. Therefore also transitions from a static symmetry broken state to a dynamical state that can be characterized as dynamical restoration of broken symmetry are always accompanied by an unbounded increase of entropy production.
\section{Discussion}
%Mention that static-dynamical transition happens at EP.\\
%\\
%First of all, it should be noted that the EPR does not vanish in the segregated phase as it does for active model B \cite{}, a related model that has the same general structure and the same free energy part.
%\\
%\\
%(Remember that the OU process in our case does not represent a velocity unlike for the actual AOUP. This explains the difference between our results for the EPR and the results of 
%\cite{Martin21,Caprini_2019}.)
%\\
%Therefore, we find that in the presence of thermal noise, the amplification of a single perturbation has a dynamical equivalent in the form of a random motion characterized by a certain persistence of its velocity.
\begin{acknowledgments}
We wish to acknowledge the support of the author community in using
REV\TeX{}, offering suggestions and encouragement, testing new versions,
\dots.
\end{acknowledgments}

\appendix
\section{Exceptional points, amplification of fluctuations and symmetry breaking}
Here we describe the mechanism that leads to amplification of fluctuations near an exceptional phase transition. This connection has already been discussed by \cite{Hanai2020}. We calculate a characteristic amplification factor for a generic two-dimensional system and for the EP found in the non-reciprocal Cahn-Hilliard model.
\subsection{Amplification of perturbations}\label{sec:amp}
Let the linearized dynamics of small perturbations $\delta \boldsymbol{x}$ around a steady state $\boldsymbol{x}_0$ be determined by
\begin{align}\label{equ:ex3}
    \partial_t \delta \boldsymbol{x} = L \delta \boldsymbol{x}\,,
\end{align}
with $L=\frac{\partial H}{\partial \boldsymbol{x}}\vert_{\boldsymbol{x}=\boldsymbol{x}_0}$.

Now let $\hat{e}_1$ and $\hat{e}_2$ be the (right) eigenvectors of $L$ where $\hat{e}_1$ is associated with the Goldstone mode. In terms of the eigenbase representation
\begin{align}
    \delta \boldsymbol{x}=\alpha\hat{e}_1+\beta\hat{e}_2\,,
\end{align}
Eq. \eqref{equ:ex3} reads
\begin{align}
\partial_t
    \begin{pmatrix}
    \alpha\\
    \beta
    \end{pmatrix}
    =\begin{pmatrix}
    &0 &0\\
    &0 &\lambda_2
    \end{pmatrix}
    \begin{pmatrix}
    \alpha\\
    \beta
    \end{pmatrix}\,,
\end{align}
where $\lambda_2$ is the eigenvalue associated to $\hat{e}_2$ which for reasons of stability of $\boldsymbol{x}_0$ is always negative.

Another representation enables us to  
see the amplification mechanism, namely, a representation with respect to the orthogonal base given by
\begin{align}
     \boldsymbol{x}=\alpha'\hat{e}_0+\beta'\hat{e}_0^\perp\,,
\end{align}
where $\hat{e}_0^\perp$ is the normal vector orthogonal to $\hat{e}_0$ which is not an eigenvector of $L$. \\
The transformation Matrix that connects both bases reads
\begin{align}
    T=
    \begin{pmatrix}
    &1 &-\frac{\hat{e}_2\cdot\hat{e}_1}{\hat{e}_2\cdot\hat{e}_1^\perp}\\
    &0 &\frac{1}{\hat{e}_2\cdot\hat{e}_1^\perp}
    \end{pmatrix}\,.
\end{align}
Therefore applying the standard transformation rule for matrices one finds that the dynamical equation with respect to this basis is given by 
\begin{align}\label{equ:epdy}
    \partial_t
    \begin{pmatrix}
    \alpha'\\
    \beta'
    \end{pmatrix}
    =
    \lambda_2
    \begin{pmatrix}
    &0 &\frac{\hat{e}_2\cdot\hat{e}_1}{\hat{e}_2\cdot\hat{e}_1^\perp}\\
    &0 &1
    \end{pmatrix}
    \begin{pmatrix}
    \alpha'\\
    \beta'
    \end{pmatrix}\,.
\end{align}
One can see that the dynamics in  $\hat{e}_1^\perp$ direction is an overdamped motion in a pseudo potential $U$ which in first harmonic approximation is given $U(x)(\beta)=-(\lambda/2)(\beta')^2$. The term ``pseudo" refers to the fact that $U$ controls only the dynamics in $\hat{e}_1^\perp$ direction which in turn completely controls the dynamics in the subspace of the Goldstone mode.

Now one is in the position to determine the evolution of an orthogonal perturbation $x(0)=\beta_0'\hat{e}_1^\perp$. The solution of Eq. \eqref{equ:epdy} with respect to this initial condition is given by
\begin{align}\label{equ:orth}
    \alpha'(t)&=\beta_0'\frac{\hat{e}_2\cdot\hat{e}_1}{\hat{e}_2\cdot\hat{e}_1^\perp}(e^{-\vert\lambda\vert t}-1)\,,\\
    \beta'(t)&=\beta_0'e^{-\vert\lambda\vert t}\,.
\end{align}
Therefore one finds that as the perturbation $\beta_0'$ in the direction $\hat{e}_1^\perp$ decays for $t\rightarrow \infty$ it leads to an effective shift in the direction of the Goldstone mode given by
\begin{align}
    \lim\limits_{t\rightarrow\infty}\alpha'(t)=-\beta_0'\frac{\hat{e}_2\cdot\hat{e}_1}{\hat{e}_2\cdot\hat{e}_1^\perp}\,.
\end{align}
Therefore, disturbances in the direction orthogonal to the Goldstone mode are forwarded in the direction of the Goldstone mode, and their magnitude is thereby amplified by a factor of  
\begin{align}
    \mathfrak{a}:=\vert \frac{\hat{e}_2\cdot\hat{e}_1}{\hat{e}_2\cdot\hat{e}_1^\perp} \vert\,.
\end{align}
%The geometrical conditions, which lead to the activation of the Goldstone mode, are also depicted graphically in Fig. \ref{fig:gold} (B).
Hence forwarding of perturbations is observed in any non-Hermitian system described by a linear operator that is not normal and therefore necessarily has a non-orthogonal eigenbase \cite{Ashida2020}.
Amplification is observed for $\vert\hat{e}_2\cdot\hat{e}_1\vert > \vert \hat{e}_2\cdot\hat{e}_1^\perp \vert$ . This is found close to an exceptional point. 
Here one can expand
\begin{align}\label{equ:EPap}
    \hat{e}_2=\hat{e}_1+\lambda_2\hat{f} +\mathcal{O}((\lambda_2)^2)
\end{align}
which is possible because $\hat{e}_1$ and $\hat{e}_2$ align while $\lambda_2$ simultaneously vanishes. One finds
\begin{align}
    \mathfrak{a}=\frac{1}{\lambda_2}(\hat{e}_1^\perp\cdot \hat{f})^{-1}+\mathcal{O}(1)\,.
\end{align}
Therefore, near an EP, perturbations aligned with the orthogonal direction are strongly amplified, such that the contribution of perturbations originally directed in the Goldstone mode direction become negligible. 
Also the characteristic time scale for the decay of the motion given by $t_0:=1/\vert \lambda_2 \vert $ diverges.

The solution of Eq. \eqref{equ:epdy} with respect to the initial condition $x(0)=\beta_0'\hat{e}_1^\perp$ can be expressed as
\begin{align}\label{equ:qss}
    \alpha'(t)&=\beta_0'(\hat{e}_1^\perp\cdot \hat{f})^{-1}t+\mathcal{O}(\frac{t}{t_0})\,,\\
    \beta'(t)&=\beta_0'+\mathcal{O}(\frac{t}{t_0})\,.
\end{align}
Therefore one finds that at the exceptional point the damped motion is transformed into a sustained motion.

We note that this analysis can be performed completely analogously in any other basis as long as it is related to the basis $\{\hat{e}_1,\hat{e}_1^\perp\}$ via a transformation matrix of the form
\begin{align}
    T'=
    \begin{pmatrix}
    &u &v \\
    &0 &1
    \end{pmatrix},
    \qquad
    \text{with}
    \qquad
    u,v \in \mathbb{R}\,,
\end{align}
i.e as long as $\hat{e}_1^\perp$ is also a basis vector in the new basis.
The respective equations of motion for the new coefficients $(\alpha'',\beta'')$ are then given by
\begin{align}\label{equ:epdy2}
   \partial_t
    \begin{pmatrix}
    \alpha''\\
    \beta''
    \end{pmatrix}
    =
    \lambda_2
    \begin{pmatrix}
    &0 &\frac{1}{u}\left(\frac{\hat{e}_2\cdot\hat{e}_1}{\hat{e}_2\cdot\hat{e}_1^\perp}-v\right)\\
    &0 &1
    \end{pmatrix}
    \begin{pmatrix}
    \alpha''\\
    \beta''
    \end{pmatrix} 
\end{align}
and the respective amplification factor is given by
\begin{align}
    \mathfrak{a}':=\vert \frac{1}{u}\left(\frac{\hat{e}_2\cdot\hat{e}_1}{\hat{e}_2\cdot\hat{e}_1^\perp}-v\right) \vert\,.
\end{align}
\\
An important fact about the type of phase transition described above is that it does not only require a symmetry breaking state as a basic requirement, but is itself a spontaneous symmetry breaking. In this case it is 
$\mathcal{P}\mathcal{T}$-symmetry. 
For quantum systems, this fact has been known for some time \cite{Bender1999}, while this relation for classical, non-reciprocally coupled systems has only recently been studied in full detail \cite{Fruchart2021}. To understand the nature of this connection, one must first define what exactly is meant by $\mathcal{P}\mathcal{T}$-symmetry in the context of a two-dimesnional non-reciprcally coupled system.

The action of time reversal operator is in the classical sense defined by
\begin{align}
    (\partial_t \boldsymbol{x},\boldsymbol{x})\underset{\mathcal{T}}{\longrightarrow}(-\partial_t \boldsymbol{x},\boldsymbol{x})\,,
\end{align}
 i.e. this implies that $\boldsymbol{x}$ position-like variable.

The action of the parity operator defined with respect to the  $\{\hat{e}_1,\hat{e}_1^\perp\}$-base is given by
\begin{align}
    \{\hat{e}_1,\hat{e}_1^\perp\}\underset{\mathcal{P}}{\longrightarrow}\{\hat{e}_1,-\hat{e}_1^\perp\}\,,
\end{align}
leading to the coordinate representation 
\begin{align}
    (\partial_t \alpha',\partial_t \beta',\alpha',\beta')\underset{\mathcal{P}}{\longrightarrow}(\partial_t \alpha',\partial_t \beta',\alpha',-\beta')\,,
\end{align}
i.e. time derivatives are even with respect to $\mathcal{P}$ but the sign of a certain component of the position-like variable $\boldsymbol{x}$ is inverted. 
However, whether $\hat{e}_1^\perp$ represents a geometrical direction depends on what kind of system is actually represented by $\boldsymbol{x}$. Therefore $\mathcal{P}$ must be understood as a generalized parity inversion operator \footnote{On can imagine that in the normal case it is the representation of a geometric parity operator.}, which is nevertheless named so because it always satisfies the formal requirement $\mathcal{P}\mathcal{P}=\mathrm{id}$ i.e. it is unitary. 

Based on this definition one clearly sees that Eq. \eqref{equ:epdy} (and also Eq. \eqref{equ:epdy2}) are invariant with respect to the simultaneous action of $\mathcal{P}$ and $\mathcal{T}$, but not with respect to the action of $\mathcal{T}$ or $\mathcal{P}$ alone. 
One also sees that $\mathcal{PT}$ applied to any quasi-stationary solution close to the EP given by Eq. \eqref{equ:qss} generates again a quasi-stationary solution. This is however running along the path of degenerate steady states in opposite direction. Therefore the solution is not an eigenstate of $\mathcal{PT}$ and thus breaks the symmetry.
Hence, it is to be expected that the actual stable solution in the  dynamical phase which emerges after crossing the EP is also a state in which the symmetry is spontaneously broken.

Finally, we would like to emphasize that this relationship between symmetry breaking and exceptional critical points derives exclusively from very general geometrical conditions, which can be realized in different ways in a variety of very different systems. Phenomena that can be traced back to the existence of an EP can therefore be observed in a wide variety of systems.
\subsection{Fluctuations close to the EP of the non-reciprocal Cahn-Hilliard model}
We define $\Theta_A=\theta_A$ and $\Theta_B=\theta_B+\pi$. In this base the  operator in Eq. \eqref{equ:dynp} reads
\begin{align}
    L=
    \begin{pmatrix}
     &\frac{\kappa^2-\delta^2}{\chi_B}&-\frac{\kappa^2-\delta^2}{\chi_B}\\
     &-\chi_B &\chi_B
    \end{pmatrix}\,.
\end{align}
Its eigenvalues are and eigenvectors are given by given by
\begin{align}
    \lambda_1=0\,,
    \qquad
    \text{and}
    \qquad
    \lambda_2=\frac{\delta_c^2-\delta^2}{\chi_B}
\end{align}
and 
\begin{align}
    \hat{e}_1=
    \frac{1}{\sqrt{2}}
    \begin{pmatrix}
     1\\
     1
    \end{pmatrix}\,,
    \qquad
    \text{and}
    \qquad
    \hat{e}_2=\frac{\begin{pmatrix}
     1\\
     1-\frac{\lambda_2}{\chi_B^2}
    \end{pmatrix}}{\sqrt{1+\left(1-\frac{\lambda_2}{\chi_B}\right)^2}}
    \,.
\end{align}
The eigenvector $\hat{e}_1$ corresponds to a simultaneous shift of both phases, which is the symmetry operation with respect to the equations of motion. Therefore it is the Goldstone mode eigenvector.

From the above representation it can be seen that for $\lambda_2\rightarrow0 $
the eigenvector $\hat{e}_2$ co-aligns with the Goldstone mode eigenvector which is the conditions for the occurrence of an exceptional point. We also find that the vector perpendicular to the Goldstone mode eigenvector is $\hat{e}_0^\perp=\hat{e}_{\Delta\theta}$ representing deviations of the phase difference with respect to the deterministic value $\Delta\theta_0$.

According to the results of Sec. \ref{sec:amp}, it is perturbations pointing in this directions that are forwarded and amplified via the mechanism described there.
For the amplification factor we find
\begin{align}
    \mathfrak{a}=\frac{1}{\lambda_2}\,.
\end{align}
\section{Entropy production of travelling and oscillating states}
For a system with a dynamical ground state $\phi_0(t)$, from Eq. \eqref{equ:eprown} it directly follows that the EPR is divergent. The singular part is given by
\begin{align}
    \epsilon\overline{\frac{\mathrm{d}}{\mathrm{d}t}\mathcal{S}_\mathrm{sing}}= \int\limits_V\mathrm{d}r
 \vert \nabla^{1}\partial_t\phi_0\vert^2
\end{align}
For a travelling state of the form $\phi_0(r,t)=\phi_0(r-vt)$ this results in
\begin{align}
    \epsilon\overline{\frac{\mathrm{d}}{\mathrm{d}t}\mathcal{S}_\mathrm{sing}}=v^2 \int\limits_V\mathrm{d}r
 \vert \phi_0\vert^2
\end{align}
For an oscillating state of the form $\phi_0(r,t)=\sin(\omega t)\phi_0(r)$ this results in a time dependent EPR with long time average
\begin{align}
    \epsilon\lim\limits_{t\rightarrow\infty}\frac{1}{t}\int\limits_0^t \mathrm{d}t\overline{\frac{\mathrm{d}}{\mathrm{d}t}\mathcal{S}_\mathrm{sing}}=\frac{1}{2} \omega^2 \int\limits_V\mathrm{d}r
 \vert \phi_0\vert^2
\end{align}
\section{Properties of the mean phase}
A quantity with which the dynamical properties of the persiten Random Walk motion can be quantified is the mean squared displacement (MSD) of the mean phase $\theta_\mathrm{m}:=(\theta_A-\theta_B)/2$.
The mean phase and the center of mass are related
via 
\begin{align}
   \theta_m = \theta_c+\frac{1}{2}\frac{\gamma_B-\gamma_A}{\gamma_A+\gamma_B}\Theta \,.
\end{align}
Therefore the mean the (MSD) is given by
\begin{align}
\nonumber
    &\langle(\theta_m(t_0+t)-\theta_m(t_0))^2\rangle=\\\nonumber
    &\langle(\theta_c(t_0+t)-\theta_c(t_0))^2\rangle\\\nonumber
    &+\frac{1}{4}\left(\frac{\gamma_B-\gamma_A}{\gamma_A+\gamma_B}\right)^2
    \langle(\Theta(t_0+t)-\Theta(t_0))^2\rangle\\
    &+\frac{\gamma_B-\gamma_A}{\gamma_A+\gamma_B}\langle(\theta_c(t_0+t)-\theta_c(t_0))(\Theta(t_0+t)-\Theta(t_0))\rangle\,.
    \end{align}
Using
\begin{align}
    \langle(\theta_c(t_0+t)-\theta_c(t_0))^2\rangle&=
    \left(t+\frac{\delta^2}{\lambda^2_2}\left(t+\frac{e^{-\lambda_2 t}-1}{\lambda_2}\right)\right)\frac{\tau}{\bar{\gamma}}\,,
\end{align}
\begin{align}\nonumber
    &\langle(\Theta(t_0+t)-\Theta(t_0))^2\rangle
    =\\
    &\left(\frac{\gamma_B-\gamma_A}{\gamma_A+\gamma_B}\right)^{-2}\left(\frac{1}{\lambda_2}\frac{(\gamma_B-\gamma_A)^2}{\gamma_A\gamma_B}(1-e^{-\lambda_2 t})\frac{\tau}{\bar{\gamma}}\right)
\end{align}
and 
\begin{align}
    \langle(\theta_c(t_0+t)-\theta_c(t_0))(\Theta(t_0+t)-\Theta(t_0))\rangle=0\,,
\end{align}
We find that for $t\ll t_0$ the MSD is given by
\begin{align}\label{equ:difshort}
    &\langle(\theta_m(t_0+t)-\theta_m(t_0))^2\rangle=\\
    &\left(\left(1+\frac{1}{4}\frac{(\gamma_B-\gamma_A)^2}{\gamma_A\gamma_B}\right)t+\frac{1}{2}\frac{\delta^2}{\lambda_2}t^2\right)\frac{\tau}{\bar{\gamma}}\,.
\end{align}
For $t\gg t_0$ the MSD is given by
\begin{align}\label{equ:diflong}
    \langle(\theta_m(t_0+t)-\theta_m(t_0))^2\rangle= \left(\left(1+\frac{\delta^2}{\lambda_2^2}\right)t\right)\frac{\tau}{\bar{\gamma}}\,.
\end{align}
Therefore, on timescales shorter than the persistence time, nonzero non-reciprocal coupling leads to a combination of diffusive and ballistic motion. The ballistic speed is given by $\mathfrak{v}_B=\delta^2\tau/(2\lambda\bar{\gamma})$. 

On time scales much larger than the persitence time non-reciprocal coupling only leads to enhanced diffusive motion with diffusion constant
$D=(1+\delta^2/\lambda^2)/(2\bar{\gamma})$.

%Plots of the MSD for three representative point in parameter space are presented in Fig.  \ref{fig:MSD1} and in Fig. \ref{fig:MSD2} together with visual data from the numerical simulation of Eq. \eqref{equ:1} which clearly shows the change in the character of the phase dynamics.
\section{General}\label{app:gen}
Considerations on the general validity of the model:\\
Assuming mass conserving dynamics:\\
We study the dynamics of fluctuations $\Delta\phi$ with respect to a fixed point $\phi_0$. For $\phi_0=0$ we have $\Delta\phi=\phi$ and in the limit $\epsilon\rightarrow 0$ the dynamics is determined by
\begin{align}
    \partial_t\phi_\alpha=-\nabla\sum\limits_{n=0}\sum\limits_\beta\mathcal{J}^n_{\alpha\beta}\nabla^n\phi_\beta+\sqrt{2\epsilon}\nabla\Lambda_\alpha
\end{align}
with
\begin{align}
    \mathcal{J}^n_{\alpha\beta}=\frac{\partial\boldsymbol{J}_\alpha}{\partial\nabla^n\phi_\beta}\vert_{\phi=0}
\end{align}
In a suitable Fourier basis this can be expressed as
\begin{align}
    \partial_t\hat{\phi}_\alpha(j)=-q_j^2\sum\limits_\beta(P_{\alpha\beta}(j)+iQ_{\alpha\beta}(j))\hat{\phi}_\beta(j)+\nabla\xi_\alpha(j)
\end{align}
with
\begin{align}
P_{\alpha\beta}(j)&=\sum\limits_{n=0}-(q_j)^{2n}\mathcal{J}^n_{\alpha\beta}\\
Q_{\alpha\beta}(j)&=\sum\limits_{n=0}(-q_j)^{2n-1}\mathcal{J}^n_{\alpha\beta}
\end{align}
and $\langle \xi_\alpha(j), \xi_\beta(k)\rangle=2\tau\delta_{k,l}\delta_{\alpha,\beta}$
From this representation it is clear that the part of the dynamics controlled by $P$ can be derived from a free energy in case $P$ is symmetric, while for $Q$ this is impossible. In particular, e.g. the non-intolerable term  $\mathcal{J}^0_{\alpha\alpha}$ represents driving of the field component $\alpha$ by an external force. Therefore we imagine $Q$ as the type of non-equilibrium caused by an external source. Internal non-equilibrium is encoded in the specific asymmetry of $P$. Therefore properties of fluctuations = properties of non-Hermitian real finite dimensional matrices. \\
%For $\phi\neq0$ we show that formally properties of fluctuations = properties of non-Hermitian real finite dimensional matrices.
\section{Additional Calculations}
\subsection{Asymmetric \& symmetric current}
\subsubsection{Computation of currents in the demixed state}
For $\epsilon\rightarrow 0$ it holds that $\frac{\partial \boldsymbol{J}^\mathrm{S}_\mu(j)}{\partial\hat{\phi}_\nu(j)}=C^-1_{\mu\nu}(j)$. Therefore 
\begin{align}
    \mathrm{tr}\frac{\partial \boldsymbol{J}^\mathrm{S}(j)}{\partial\hat{\phi}(j)}=\mathrm{tr}\,C^{-1}=\frac{\mathrm{tr}\,C}{\det C}=\mathrm{tr}\, A_j=\mathrm{tr}\frac{\partial \boldsymbol{J}^\mathrm{H}(j)}{\partial\hat{\phi}(j)}
\end{align}
where we have used the results of App. \ref{app:cov}. Therefore
\begin{align}
    \mathrm{tr}\, A_j=\mathrm{tr}\frac{\partial \boldsymbol{J}^\mathrm{A}(j)}{\partial\hat{\phi}(j)}=0
\end{align}
and 
\begin{align}
    \langle \boldsymbol{J}^\mathrm{S}(j)^2\rangle=q_j^2\mathrm{tr}\,A_j
\end{align}
\subsection{Covariance matrix demixed state and for $j\rightarrow\infty$}\label{app:cov}
To obtain the covariance matrix $C$ we have to solve the Lyaponov equation assigned to Eq. \eqref{equ:eomm} for each $j$ which reads
\begin{align}
    C(j)A^T_j+A_jC(j)=2\frac{\epsilon}{L}\mathrm{1}
\end{align}
The solution reads
\begin{align}
    &\langle \mathrm{Re}\hat{\phi}^j_A(\hat{\phi}^j_B)^*\rangle=-\frac{\kappa(\alpha_A^j+\alpha_B)+\delta(\alpha_B-\alpha_A^j)}{(\alpha_A^j+\alpha_B)(\delta^2-\kappa^2+\alpha_A^j\alpha_B)}\frac{\epsilon}{L}\\
    &\langle\vert \hat{\phi}^j_A\vert^2\rangle=\frac{\delta-\kappa}{\alpha_A^j}\langle \mathrm{Re}\hat{\phi}^j_A(\hat{\phi}^j_B)^*\rangle+\frac{1}{\alpha_A^j}\frac{\epsilon}{L}\\
    &\langle\vert \hat{\phi}_B^j\vert^2\rangle=-\frac{\delta+\kappa}{\alpha_B}\langle \mathrm{Re}\hat{\phi}^j_A(\hat{\phi}^j_B)^*\rangle+\frac{1}{\alpha_B}\frac{\epsilon}{L}
\end{align}
We note that for $j\rightarrow\infty$ we have
\begin{align}
 A_j=
\begin{pmatrix}
\chi_A+\gamma_A q_j^2  & \kappa-\delta\\
\kappa+\delta & \chi_B\\
\end{pmatrix} \,,
\end{align}
i.e. the properties of fluctuations become independent of $\chi_A$ and the underlying ground state. We find
\begin{align}
    &\langle \mathrm{Re}\hat{\phi}^j_A(\hat{\phi}^j_B)^*\rangle\sim\frac{1}{\alpha_A^j}\frac{\delta-\kappa}{\alpha_B}\frac{\epsilon}{L}\\
    &\langle\vert \hat{\phi}^j_A\vert^2\rangle\sim\frac{1}{\alpha_A^j}\frac{\epsilon}{L}\\
    &\langle\vert \hat{\phi}_B^j\vert^2\rangle\sim-\frac{1}{\alpha_A^j}\frac{\delta^2-\kappa^2}{\alpha_B^2}\frac{\epsilon}{L}
\end{align}
\subsection{Oscillating instability}
We want to find the solution of Eq. \eqref{equ:eomm} for $\epsilon=0$ and the initial conditions $\hat{\phi}^l_\mu(0)=\mathcal{A}_mu e^{i\theta_\mu}\delta_{jl}+\mathcal{A}_mu e^{-i\theta_\mu}\delta_{-jl}$, in the region of the phase diagram wher $\vert\mathrm{Im}\lambda\vert>0$. 
The transformation to the eigenbase coordinates is given by 
\begin{align}
    \begin{pmatrix}
     &\hat{\phi}^j_-\\
     &\hat{\phi}^j_+
    \end{pmatrix}
    = T^{-1}
    \begin{pmatrix}
     &\hat{\phi}^j_A\\
     &\hat{\phi}^j_N
    \end{pmatrix}
\end{align}
with $T=(e,e^*)$ being the column matrix of eigenvectors. The equations of motions read
\begin{align}
    \partial_t \hat{\phi}^j=-q_j^2
    \begin{pmatrix}
     &\mathrm{Re}\lambda-\mathrm{Im}\lambda &0\\
     &0 &\mathrm{Re}\lambda+\mathrm{Im}\lambda
    \end{pmatrix}
    \hat{\phi}^j
\end{align}
with solution
\begin{align}
    \hat{\phi}^j_{\pm}(t)=\hat{\phi}^j_{\pm}(0)e^{-q_j^2(\mathrm{Re}\lambda\pm i\mathrm{Im}\lambda)t}
\end{align}
The coordinates in the original base fulfill $\hat{\phi}_\mu^{-j}=(\hat{\phi}_\mu^{j})^*$. Therefore
\begin{align}
    \begin{pmatrix}
     &\hat{\phi}^{-j}_-\\
     &\hat{\phi}^{-j}_+
    \end{pmatrix}
    = T^{-1}
    \begin{pmatrix}
     &\hat{\phi}^j_A\\
     &\hat{\phi}^j_A
    \end{pmatrix}^*
    =T^{-1}T^*
    \begin{pmatrix}
     &\hat{\phi}^{j}_-\\
     &\hat{\phi}^{j}_+
    \end{pmatrix}^*\,.
\end{align}
We find
\begin{align}
    T^{-1}T^* =
    \begin{pmatrix}
     &0&1\\
     &1&0
    \end{pmatrix}\,.
\end{align}
Hence
\begin{align}
    &\phi(x)=e^{-q_j^2\mathrm{Re}\lambda t} \times\\
    &T
    \begin{pmatrix}
     \hat{\phi}^j_{-}(0) e^{q_j^2 \mathrm{Im}\lambda t+iq_jx}+
     (\hat{\phi}^j_{+}(0))^* e^{q_j^2 \mathrm{Im}\lambda t-iq_jx}\\
     \hat{\phi}^j_{+}(0) e^{-q_j^2 \mathrm{Im}\lambda t+iq_jx}+
     (\hat{\phi}^j_{-}(0))^* e^{-q_j^2 \mathrm{Im}\lambda t-iq_jx}
    \end{pmatrix}\\
    &=e^{-q_j^2\mathrm{Re}\lambda t} \times
    \\
    &2\mathrm{Re}\begin{pmatrix}
     e_1\hat{\phi}^j_{-}(0) e^{q_j^2 \mathrm{Im}\lambda t+iq_jx}+e_1^*
     \hat{\phi}^j_{+}(0) e^{q_j^2 \mathrm{Im}\lambda t-iq_jx}\\
     e_2\hat{\phi}^j_{-}(0) e^{q_j^2 \mathrm{Im}\lambda t+iq_jx}+e_2^*
     \hat{\phi}^j_{+}(0) e^{q_j^2 \mathrm{Im}\lambda t-iq_jx}
    \end{pmatrix}
    \\
\end{align}
Using $\mathrm{Re}(uv)=\mathrm{Re}u\mathrm{Re}v-\mathrm{Im}u\mathrm{Im}v$ we find
\begin{align}
\nonumber
\phi_A(x) =&2e^{-q_j^2\mathrm{Re}\lambda t} (\mathcal{A}_A\mathrm{Im}(e_1e_2^*)\cos(q_jx+\theta_A)\cos(q_j^2 \mathrm{Im}\lambda t)\\
\nonumber
-&
\mathcal{A}_A\mathrm{Re}(e_1e_2^*)\sin(q_jx+\theta_A)\sin(q_j^2 \mathrm{Im}\lambda t)\\
-&
\mathcal{A}_B\vert e_1 \vert ^2\sin(q_jx+\theta_B)\sin(q_j^2 \mathrm{Im}\lambda t))/\mathrm{Im}(e_1e_2^*)\\
\nonumber
\phi_B(x) =&2e^{-q_j^2\mathrm{Re}\lambda t} (\mathcal{A}_B\mathrm{Im}(e_1e_2^*)\cos(q_jx+\theta_B)\cos(q_j^2 \mathrm{Im}\lambda t)\\
\nonumber
-&
\mathcal{A}_B\mathrm{Re}(e_1e_2^*)\sin(q_jx+\theta_A)\sin(q_j^2 \mathrm{Im}\lambda t)\\
-&
\mathcal{A}_A\vert e_1 \vert ^2\sin(q_jx+\theta_A)\sin(q_j^2 \mathrm{Im}\lambda t))/\mathrm{Im}(e_1e_2^*)
\end{align}
The solution takes the form of damped oscillations, standing and travelling wave patterns.
\subsection{General expression for the steady state EPR}\label{sec:eprg}
Using Eq. \eqref{equ:EPR1} we find
\begin{align}
    \epsilon\overline{\frac{\mathrm{d}}{\mathrm{d}t}\mathcal{S}}&=
    \langle\,
    \boldsymbol{J}^{\mathrm{det}}\cdot 
    \nabla
    \begin{pmatrix}
     \delta\phi_B\\
     -\delta\phi_A
    \end{pmatrix}
    \,\rangle\\
    &=
    \langle\,
    \boldsymbol{J}^{\mathrm{det}}\cdot 
    \left[\delta
    \nabla
    \begin{pmatrix}
     \phi_B\\
     \phi_A
    \end{pmatrix}
    +
    \nabla
    \begin{pmatrix}
     0\\
     -2\delta\phi_A
    \end{pmatrix}
    \right]
    \,\rangle
    \\
    &=\left(-2\delta\langle\, \boldsymbol{J}^{\mathrm{det}}_B\cdot \nabla \phi_A,\rangle+ \underset{=0}{\delta\langle\, \frac{\mathrm{d}}{\mathrm{d}t}(\phi_A\phi_B)\, \rangle}\right)\\
    &=2\delta\langle\, ((\kappa+\delta)\nabla\phi_A+\chi_B\nabla\phi_B) \cdot\nabla \phi_A,\rangle
\end{align}
In terms of the Fourier weights we find
\begin{align}
    \epsilon\overline{\frac{\mathrm{d}}{\mathrm{d}t}\mathcal{S}}=4L\delta\sum\limits_{j>0} q_j^2(\alpha_B\mathrm{Re}\langle  \hat{\phi}^j_{A}\hat{\phi}^{-j}_{B}\rangle+ (\kappa+\delta) \langle \vert\hat{\phi}_{A}^j\vert^2\rangle)
\end{align}
\section{Entropy Production}\label{app:ep}

Let  $\boldsymbol{X}:\Omega\times \mathbb{R}^+\rightarrow S$ be a time-independent stochastic process. For a continuous trajectory $\{\boldsymbol{X}(t,\omega)_{t\in [0,T]}\}$ connecting the states of the system at time $0$ and time $T$ and a partition $p=\{0,t_1,t_2,\dots,t_{n-1},T\}$ 
the entropy difference is defined as
\begin{equation}\label{equ:diff1}
    \Delta \mathcal{S}[\boldsymbol{X}(\omega),p]=\log \frac{\mathbb{P}\left[\{\boldsymbol{X}(s,\omega)\}_{s\in p},p\right]}{\mathbb{P}\left[\{\boldsymbol{X}^R(s,\omega)\}_{s\in p^R},p^R\right]}\,.
\end{equation}
$\mathbb{P}[\cdot,p]$ is the finite dimensional distribution of $X$ with respect to $p$, $p^R\equiv\{0,T-t_{n-1},T-t_{n-2},T-t_{1},T\}$ is the reversed partition and $\boldsymbol{X}^R(t,\omega)\equiv\boldsymbol{X}(T-t,\omega)$ is the reversed version of the trajectory. The entropy difference of the trajectory $\boldsymbol{X}(\omega)$ is then defined as
\begin{equation}\label{equ:diff1}
    \Delta \mathcal{S}[\boldsymbol{X}(\omega),0,T]=\lim\limits_{n\rightarrow\infty}\Delta \mathcal{S}[\boldsymbol{X}(\omega),p^n]
    \,,
\end{equation}
where $p^n$ is any sequence of partitions with $\lVert p^n \rVert\underset{n\rightarrow\infty}{\rightarrow}0$. The entropy difference vanishes whenever 
\begin{equation}
\lim\limits_{n\rightarrow\infty}
    \frac{\mathbb{P}\left[\{\boldsymbol{X}(s,\omega)\}_{s\in p_n},p_n \vert \boldsymbol{X}(0,\omega) \right]}{\mathbb{P}\left[\{\boldsymbol{X}(s,\omega)\}_{s\in p_n^R},p_n^R \vert \boldsymbol{X}(T,\omega) \right]}=\frac{\mathcal{P}[\boldsymbol{X}(T,\omega)]}{\mathcal{P}[\boldsymbol{X}(0,\omega)]}
\end{equation}
where $\mathcal{P}[\boldsymbol{X},t]$ is the propabilitx for oberserving the configuration $\boldsymbol{X}$ at time $t$ and  $\mathbb{P}\left[\{\boldsymbol{X}(s,\omega)\}_{s\in p},p \vert \boldsymbol{X}(0,\omega) \right]$ is the conditional probability for observing the sequence of configurations $\{\boldsymbol{X}(s,\omega)\}_{s\in p}$ given the initial configuration $\boldsymbol{X}(0,\omega)$.
Therefore, by construction it is a measure of the the violation of the detailed balance condition (DB).
One can also define the instantaneous entropy production rate (EPR)
\begin{equation}
  \frac{\mathrm{d}}{\mathrm{d}t}\mathcal{S}[\boldsymbol{X}(t,\omega),\partial_t \boldsymbol{X}(t,\omega)]= \lim\limits_{h\rightarrow 0}\frac{\Delta \mathcal{S}[\boldsymbol{X}(\omega),t,t+h]}{h}\,.
\end{equation}
Taking the expected value one obtains a measure for the tendency of the process to break time-reversal-symmetry (TRS). This can also be expressed in terms of an average (EPR) which is defined as
\begin{equation}\label{equ:diff1}
  \overline{\frac{\mathrm{d}}{\mathrm{d}t}\mathcal{S}[\boldsymbol{X}(t)]}= \lim\limits_{h\rightarrow 0}\frac{\langle\Delta \mathcal{S}[\boldsymbol{X}(\omega),t,t+h]\rangle}{h}\,.
\end{equation}
\subsection{Entropy production in non-equilibrium field theories}
In \cite{Nardini2017,Gradenigo2012} it was shown that for field theories of the type defined in section \ref{sec:nefd}, the average EPR in the steady state is given by 
\begin{equation}\label{equ:ecen}
    \overline{\frac{\mathrm{d}}{\mathrm{d}t}\mathcal{S}[\phi]}=
    \frac{1}{\epsilon M}\langle\,(\boldsymbol{J}_\mathrm{det}-\boldsymbol{\bar{J}}_\mathrm{det})\cdot(\boldsymbol{J}^{\mathrm{a}}-\boldsymbol{\bar{J}}^{\mathrm{a}})\,\rangle\,,
\end{equation}
In \cite{Li_2021} it was shown that for any trajectory $\{\phi(t,\omega)_{t\in [0,+\infty]}\}$ it holds
\begin{align}
    \lim\limits_{t\rightarrow\infty}\Delta \mathcal{S}[\phi(\omega),t,t+T]=\frac{1}{M}
    \int\mathrm{d}\phi(\omega)\,\cdot \nabla^{-1}\boldsymbol{J}^{\mathrm{A}}\,.
\end{align}
Therefore, already on the level of a single trajectory only the asymmetric current contributes to the EPR and hence to the breaking of detailed balance. Consequently, for zero active current the steady state entropy production does not only vanish on average.
Here also another representation for the steady state EPR in terms of the asymmetric current was obtained.
\begin{equation}
    \overline{\frac{\mathrm{d}}{\mathrm{d}t}\mathcal{S}[\phi]}=
    \frac{1}{\epsilon M}\langle \boldsymbol{J}^\mathrm{A}\cdot\boldsymbol{J}^\mathrm{A}\,\rangle\,
\end{equation}
The above results where obtained using Stratonovich Calculus.
In App. ? we show how to obtain the same results using the Ito calculus. Advantageous: Appears more consistent all in all. It is not necessary to use model specific discretization schemes as in \cite{Li_2021, Nardini2017}. In addition, we find another representation of the EPR.  
\begin{eqnarray}\label{equ:eprl}
    \langle\,\frac{\mathrm{d}}{\mathrm{d}t}\mathcal{S}[\phi]\,\rangle=&&
    \frac{1}{\epsilon M}\langle\,(\boldsymbol{J}^{\mathrm{det}}-\boldsymbol{\bar{J}}^{\mathrm{det}})(\phi)\cdot(\boldsymbol{J}^{\mathrm{det}}-\boldsymbol{\bar{J}}^{\mathrm{det}})(\phi)\,\rangle
    \nonumber
    \\\nonumber
    \\
    &&+M \langle\, \mathrm{tr}\frac{\delta}{\delta\phi}\nabla^2\frac{\delta\mathcal{F}}{\delta\phi}\,\rangle
    \,.
\end{eqnarray}
This representation allows to study the relationship between EPR and phase transition in a systematic way. 


\subsection{Singularities of the EPR}
Eq. \eqref{equ:eprl} allows a systematic examination of the EPR for possible singularities.
Precondition: Dynamics of the phase transition can be described with a finite set of Fourier modes, i.e. SDE for fluctuations decomposes in Fourier space into sets of independent SDEs. This is the case if $\boldsymbol{J}^{\mathrm{det}}$ is linear in $\phi$ and its gradients. Then each mode is described by a separate SDE. It is also the case if the operator of the linearized dynamics becomes block diagonal in a certain basis. This is approximately the case e.g. for the non-reciprocal Cahn-Hilliard model. We first treat the linear case. 
Under this condition the measure of path probabilities decomposes into a product of single mode path probability measures and therefore it is reasonable to define a mode-wise EPR which is then equal to the Fourier components of the full EPR. By Eq. \eqref{equ:eprl} it is given by
\begin{align}\label{equ:ptrate}
    \langle\,\frac{\mathrm{d}}{\mathrm{d}t}\mathcal{S}^j\,\rangle=
    &\frac{1}{\epsilon M}\langle\,\boldsymbol{J}^{\mathrm{det}}(j)\cdot\boldsymbol{J}^{\mathrm{det}}(j)\,\rangle
    -q_j^2\mathrm{Tr}\frac{\delta^2\mathcal{F}}{\delta\hat{\phi}^j\delta\hat{\phi}^j}\,.
\end{align}
with
\begin{align}
    \mathrm{Tr}\frac{\delta^2\mathcal{F}}{\delta\hat{\phi}^j\delta\hat{\phi}^j}=\sum\limits_n\lambda_n(j)
    =\sum\limits_n\mathrm{Re}\lambda_n(j)\,,
\end{align}
where $\{\lambda_{n}\}$ are the eigenvalues of the real matrix operator $A^j$ with components 
\begin{align}
 A^j_{\nu\eta}=\frac{\partial}{\partial\phi_\eta^j}\mu^\nu(j)\,,
 \qquad
 \text{and}
 \qquad
 \mu =\nabla^{-1}\frac{\boldsymbol{J}^\mathrm{det}}{M}\,.
\end{align}
For the expected value of the squared deterministic current I find
\begin{align}
   \boldsymbol{J}^{\mathrm{det}}(j)\cdot\boldsymbol{J}^{\mathrm{det}}(j)&=
   \psi^*\mathrm{dig}(\lambda^*)T^TT\mathrm{dig}(\lambda)\psi\\&=q_j^2
   \sum\limits_{lmn}\psi_{-l}\lambda^*_{l}T_{ml}T_{mn}\lambda_n\psi_n\,,
\end{align}
where the transformation matrix $T$ be implicitly defined by
\begin{align}
    T^{-1}AT=\mathrm{dig}(\lambda)\,,
\end{align}
and the coordinates in the eigenvector base are given by
\begin{align}
    \psi^j=T^{-1}\phi^j\,.
\end{align}
Note that $T^j$ is not orthogonal as long as $A$ is non-Hermitian i.e if $\boldsymbol{J}^{\mathrm{a}}\neq 0$.
Inserting the steady state condition
\begin{align}
    0&=\langle\,\frac{\mathrm{d}}{\mathrm{d}t} (\psi_{l}^{-j}\psi_{n}^j)\,\rangle\\
    &=-q_j^2(\lambda^*_l+\lambda_n)\langle\, \psi_{l}^{-j}\psi_{n}^j\,\rangle +2\epsilon q_j^2\sum\limits_{k} T^{-1}_{lk}T_{nk}^{-1}\,,
\end{align}
we find
\begin{eqnarray}\label{equ:eppt2}
     \langle\,\frac{\mathrm{d}}{\mathrm{d}t}\mathcal{S}^j\,\rangle=&&q_j^2\sum\limits_{ln}\left(2\frac{\lambda_l^*\lambda_n}{\lambda_l^*+\lambda_n}\frac{C_{ln}}{(\det T)^2}-\mathrm{Re}\lambda_n \delta_{ln}\right)\\\nonumber
     \\
     =&&q_j^2\sum\limits_{l}\left(\frac{\mathrm{Re}\lambda_l^2+\mathrm{Im}\lambda_l^2}{\mathrm{Re}\lambda_l}\frac{C_{ll}}{(\det T)^2}-\mathrm{Re}\lambda_l \right)\\\nonumber
     \\&&+
     q_j^2\sum\limits_{l\neq n}\left(2\frac{\lambda_l^*\lambda_n}{\lambda_l^*+\lambda_n}\frac{C_{ln}}{(\det T)^2}\right)
\end{eqnarray}
with
\begin{align}
C_{ln}&=\sum\limits_{mk}T_{kl}T_{kn}T^{-1}_{lm}T^{-1}_{nm}(\det T)^2\\
&=\sum\limits_{mk}T_{kl}T_{kn}(\mathrm{adj} T^T)_{ml}(\mathrm{adj}T^T)_{mn}\,.
\end{align}
Stability of the steady state requires 
\begin{align}
   & \lambda_n(j)>0\,,
   &\forall j, n\,.
\end{align}
Accordingly a phase transition is defined as a path $\{\gamma(s)\}_{s\in [0,s_0)}$ with $\mathrm{Re}\lambda_i(\gamma(s_0))=0$. In the following always the re-parametrization with $s'=\mathrm{Re}\lambda_i$ will be used.

From the representation of EPR in Eq. \eqref{equ:eppt2} it can be seen that a discontinuity of an eigenvalue with respect to a certain path must also manifest itself as a discontinuity in the EPR. This can be a divergence or only a kink.
Three scenarios can be distinguished:
\begin{itemize}
    \item $\mathrm{Im}\lambda_i=0$ and $\vert \det T(0) \vert  > 0$
\end{itemize}
Here we find that the EPR is given by
\begin{align}
\langle\,\frac{\mathrm{d}}{\mathrm{d}t}\mathcal{S}^j\,\rangle&=
    q_j^2\sum\limits_{l\neq i}\left(\frac{\mathrm{Re}\lambda_l^2+\mathrm{Im}\lambda_l^2}{\mathrm{Re}\lambda_l}\frac{C_{ll}}{(\det T)^2}-\mathrm{Re}\lambda_l \right)
    \\
     &+q_j^2\sum\limits_{l\neq n}\left(2\frac{\lambda_l^*\lambda_n}{\lambda_l^*+\lambda_n}\frac{C_{ln}}{(\det T)^2}\right)\,.
\end{align}
Thus, the entropy remains finite but discontinuity in the first derivative with respect to the control parameter might occur.
\begin{itemize}
    \item $\mathrm{Im}\lambda_i>0$ and $\vert \det T(0) \vert  > 0$
\end{itemize}
Here I find that the EPR close to the transition is approximately given by 
\begin{align}
\langle\,\partial_t\mathcal{S}^j\,\rangle\approx
q_i^2 \frac{\mathrm{Im}\lambda_i^2}{\mathrm{Re}\lambda_i}\frac{C_{ii}}{(\det T)^2}\,.
\end{align}
Since it always holds that $C_{ii}>0$ the EPR hence diverges as the transition is approached. \\
Close to the transition it holds that 
\begin{align}
   \sum\limits_n \langle\, (\hat{\phi}_n^j)^*\hat{\phi}_n^j\,\rangle&=\epsilon\sum\limits_{nkms}2\frac{T_{nk}T_{nm}(T^{-1})_{ks}(T^{-1})_{ms}}{\lambda_k^*+\lambda_m}\\
   &\approx
    \epsilon\sum\limits_{ns}\frac{T_{ni}T_{ni}(T^{-1})_{is}(T^{-1})_{is}}{\mathrm{Re}\lambda_i}
    \\
    &=\frac{\epsilon}{\mathrm{Re}\lambda_i}\frac{C_{ii}}{(\det T)^2}
\end{align}
and therefore the EPR can be expressed in terms of the imaginary part of the eignevalue and the expected value of fluctuations only.
\begin{align}
    \langle\,\partial_t\mathcal{S}^j\,\rangle\approx q_i^2\mathrm{Im}\lambda_i^2\frac{\langle\,(\hat{\phi}^j)^*\hat{\phi}^j\,\rangle}{\epsilon}
\end{align}
This kind of transition can be characterized as a Hopf-Bifurcation.
\begin{itemize}
    \item $\mathrm{Im}\lambda_i=0$ and $\vert \det T(0) \vert  = 0$
\end{itemize}
In this scenario the phase transition is accompanied by the co-aligning of two eigenvectors. If in addition the model has a continuous symmetry the aligning can take place between Goldstone mode eigenvector $\hat{e}_1$ resp. the Goldstone mode eigenspace which always has eigenvalue $\lambda_1=0$ and another linearly independent eigenvector $\hat{e}_i$ with eigenvalue $\lambda_i$. 

Such type of phase transition is referred to as an exceptional point transition. Its properties will be discussed in Detail in Sec \ref{sec:epd}. Note that the Goldstone eigenvector remains fixed as long as the symmetry properties of the system are not altered along the path $\gamma$.
For the EPR I find 
\begin{align}
   \langle\,\partial_t\mathcal{S}^j\,\rangle&=q_j^2\sum\limits_{l,n>1}\left(2\frac{\lambda_l^*\lambda_n}{\lambda_l^*+\lambda_n}\frac{C_{ln}}{(\det T)^2}-\mathrm{Re}\lambda_n \delta_{ln}\right) 
\end{align}
which already suggests that the EPR will diverge at the transition point, but this can also explicitly be shown.

Choosing an ordering of the eigenbase such that $i=2$, one has
\begin{align}
    T(0)&=\begin{pmatrix}
    \hat{e}_1,\hat{e}_1,\hat{e}_3,\dots,\hat{e}_n
    \end{pmatrix}\,,\\
    \mathrm{adj}T^T&=
    \begin{pmatrix}
    \hat{d}_1,-\hat{d}_1,0,\dots,0
    \end{pmatrix}\,,
\end{align}
with
\begin{align}
    \hat{d}(k)=M_{1,k}(T(0))
\end{align}
where $M$ is the set of minors of $T$. All other minors are zero because of the linear dependence of the columns of $T$.
This results in 
\begin{align}
    &C(0)=\\
    &\begin{pmatrix}
    \sum_k M_{1,k}(T(0))^2 & -\sum_k  M_{1,k}(T(0))^2 &0 &\dots & 0\\
    -\sum_k M_{1,k}(T(0))^2 & \sum_k M_{1,k}(T(0))^2 &0 &\dots & 0\\
     0 & 0 &0 &\dots & 0\\
      : & : &: &\dots & :\\
       0 & 0 &0 &\dots & 0
    \end{pmatrix}\,.
\end{align}
Therefore
\begin{align}
    \langle\,\partial_t\mathcal{S}^j\,\rangle&\sim q_j^2\frac{\lambda_2}{(\det T)^2}\sum_k M_{1,k}(T(0))^2
    .
\end{align}
For the determinant of the transformation matrix one can write
\begin{align}
    T(\lambda_2)&=\begin{pmatrix}
    \hat{e}_1,\hat{e}_1+\lambda_2 f_2,\hat{e}_3(0)+\lambda_2 f_3,\dots,\hat{e}_n(0)+\lambda_2 f_n
    \end{pmatrix}+\\
    &\mathcal{O}((\lambda_2)^2)
    \end{align}
Using the Laplace expansion rule with respect to the second column an repeating the same step for the remaining columns one finds
\begin{align}
    \det T(\lambda_2)&=\det T(0) \\
    &+\lambda_2 \det 
    \begin{pmatrix}
    \hat{e}_1, f_2,\hat{e}_3(0)+\lambda_2 f_3,\dots,\hat{e}_n(0)+\lambda_2 f_n
    \end{pmatrix}\\
    &+
    \mathcal{O}(T^2)\nonumber
    \\\nonumber
    &\dots\\
    &=\lambda_2\det 
    \begin{pmatrix}
    \hat{e}_1, f_2,\hat{e}_3(0),\dots,\hat{e}_n(0)
    \end{pmatrix}+\mathcal{O}((\lambda_2)^2)\,.
\end{align}
Consequently, I find that the EPR diverges when approaching the exceptional point as $\sim 1/\lambda_2$. 
\section{Entropy production: Derivation in Ito Calculus}\label{app:ei}
Also main result can be obtained otherwise, we present here a derivation of concrete analytical expressions for the EPR using Ito Calculus. Starting from Eq. \eqref{equ:diff1} and using the definitions of the quasipotential $\mathcal{V}$ and the action $\mathbb{A}$ 
\begin{eqnarray}
    &&\mathcal{P}[\phi(t,\omega),t]\propto \mathrm{exp}\{-\frac{\mathcal{V}[\phi(t,\omega),t]}{\epsilon}\}\,,\label{equ:qua}
    \\\nonumber
    \\
    &&\mathbb{P}\left[\{\phi(s,\omega)\}_{s\in p},p \vert \phi(0,\omega) \right]\propto \mathrm{exp}\{-\frac{\mathbb{A}[\phi(\omega),p]}{\epsilon}\}\,.
\end{eqnarray}
the total entropy difference can be decomposed into an internal and external part $\Delta\mathcal{S}=\Delta\mathcal{S}_{\mathrm{int}}+\Delta\mathcal{S}_{\mathrm{ext}}$, with
\begin{eqnarray}
\epsilon\Delta\mathcal{S}_{\mathrm{int}}&&=
\mathcal{V}[\phi(T),T]-\mathcal{V}[\phi(0),0]\label{equ:enin}\,,\\\nonumber
\\
\epsilon\Delta\mathcal{S}_{\mathrm{ext}}&&=-\mathbb{A}\left[\phi,p\right]+\mathbb{A}\left[\phi^R,p^R\right]\,,\label{equ:enex}
\end{eqnarray}
For an equilibrium field theory $\Delta\mathcal{S}_{\mathrm{int}}$ in the steady state simply measures the fluctuation of the free energy $\mathcal{F}$. The detailed balance condition can therefore be rephrased as $\Delta\mathcal{S}_{\mathrm{ext}}= - \Delta\mathcal{F}$. As a consequence, a trajectory decreasing the free energy is more probable to be traversed in the forward direction than the backwards direction. If the conditional path probabilities are not tied anymore to the probabilities for the occurrence of the configurations at its endpoints DB is broken. The dynamics of $\phi$ becomes irreversible which is expressed in an average EPR that is always non-negative.
\begin{eqnarray}
    \nonumber
    &&\langle\Delta \mathcal{S}[\phi]\rangle=
    \int\mathcal{D}\{\phi\}\log \frac{\mathbb{P}\left[\{\phi\}\right]}{\mathbb{P}\left[\{\phi^R\}\right]}\mathbb{P}\left[\{\phi\}\right]\\\nonumber
    \\
   &&=\frac{1}{2}\int\mathcal{D}\{\phi\}\log \frac{\mathbb{P}\left[\{\phi\}\right]}{\mathbb{P}\left[\{\phi^R\}\right]} \left(\mathbb{P}\left[\{\phi\}\right]-\mathbb{P}\left[\{\phi^R\}\right]\right) \geq 0\nonumber
   \\
\end{eqnarray}
 This means that a given sequence of configurations is more likely to be observed in one direction or the other. Extreme example: deterministic dynamics and vanishing noise: trajectories are only observed. Average EPR measures the average difference. We want to examine this in more detail.
For an Ito-SDE of the form of Eq. \eqref{equ:esta} the action with is given by \cite{Bach78}
\begin{widetext}
\begin{equation}
    \mathbb{A}[\{\phi(t,\omega)_{t\in [t,t+T]}\},p]=
    \frac{1}{2}\int \mathrm{d}\boldsymbol{r}\frac{1}{M} \sum\limits_{i=1}^n \Delta p_i\left(\frac{\nabla^{-1}\phi(p_{i+1})-\nabla^{-1}\phi(p_i)}{\Delta p_i}+\nabla^{-1}\nabla\boldsymbol{J}^{\mathrm{det}}(p_i)\right)^2+ C(p)\,,
\end{equation}
\end{widetext}
where $\Delta p_i\equiv p_i-p_{i-1}$ and $C$ is normalization constant with $C(p)=C(p^R)$.
Accordingly the external EPR with respect to a trajectory $\{\phi(t,\omega)_{t\in [0,T]}\}$ and the partition $p$ is given by
\begin{align}
    \nonumber
    &\epsilon\Delta\mathcal{S}_{\mathrm{ext}}[\phi(\omega),p]=
    \\\nonumber
    \\
    &\int \mathrm{d}\boldsymbol{r} \sum\limits_{i=1}^n\left(\phi(p_{i+1})-\phi(p_i)\right)\nabla^{-1}\frac{\boldsymbol{J}^{\mathrm{det}}(p_i)-\boldsymbol{\bar{J}}^{\mathrm{det}}(p_i)}{M}\,.
\end{align}
Hence the external EPR with respect to the trajectory can be expressed as an integral 
\begin{align}\label{equ:eprdif}
    \epsilon\Delta\mathcal{S}_{\mathrm{ext}}[\phi(\omega),t,t+T]=\int\mathrm{d}\boldsymbol{r}\int\limits_t^{t+T}\mathrm{d}\phi(\omega)\nabla^{-1}\frac{\boldsymbol{J}^{\mathrm{det}}-\boldsymbol{\bar{J}}^{\mathrm{det}}}{M}
\end{align}
To make the notation more compact, we introduce the following abbreviations.
The scalar product of two $m$-position fields $\boldsymbol{F}$ and $\boldsymbol{G}$ is defined as
\begin{align}
    \nonumber
    &\boldsymbol{F}\cdot \boldsymbol{G}\equiv\\
    &\int \mathrm{d}\boldsymbol{r}_1\dots \mathrm{d}\boldsymbol{r}_m\sum\limits_{\mu\dots\mu_l}F_{\mu_1\dots\mu_l}(\boldsymbol{r}_1,..\,,\boldsymbol{r}_m)G_{\mu_1\dots\mu_l}(\boldsymbol{r}_1,..\,,\boldsymbol{r}_m)
\end{align}
The trace operation is defined as
\begin{align}
    \mathrm{tr}\boldsymbol{F}=\int \mathrm{d}\boldsymbol{r}\sum\limits_{\mu}F_{\mu\dots\mu}(\boldsymbol{r},..\,,\boldsymbol{r})G_{\mu\dots\mu}(\boldsymbol{r},..\,,\boldsymbol{r})
\end{align}
Using the decomposition Eq. \eqref{equ:deco2}, the integral can be further transformed as
\begin{align}\label{equ:inth}
    \nonumber
    &\int\limits_t^{t+T}\mathrm{d}\phi(\omega)\cdot\nabla^{-1}\frac{\boldsymbol{J}^{\mathrm{det}}-\boldsymbol{\bar{J}}^{\mathrm{det}}}{M}=\\
    &\mathcal{F}[\phi(T+t,\omega)]-\mathcal{F}[\phi(t,\omega)]+\int\limits_t^{t+T}\mathrm{d}\phi(\omega)\cdot\nabla^{-1}\frac{\boldsymbol{J}^{\mathrm{a}}-\boldsymbol{\bar{J}}^{\mathrm{a}}}{M}
\end{align}
together with the prescription for the internal entropy difference, the EPR is given by
\begin{align}
    \nonumber
    &\epsilon\frac{\mathrm{d}}{\mathrm{d}t}\mathcal{S}[\phi(\omega),t]=
    \\
    &\frac{\mathrm{d}}{\mathrm{d}t}\left(F[\phi(t,\omega)]-\mathcal{V}[\phi(t,\omega),t]\right)+\partial_t\phi(t,\omega)\cdot\frac{\boldsymbol{J}^{\mathrm{a}}-\boldsymbol{\bar{J}}^{\mathrm{a}}}{M}
\end{align}
In the absence of an active current it holds that
\begin{align}
    \lim\limits_{t\rightarrow\infty}\mathcal{V}[\cdot,t]=\mathcal{F}[\cdot]\,.
\end{align}
Hence, the steady state EPR is identically zero for any trajectory.
Just as with a random variable that is a function of a stochastic process, the statistical properties of the entropy difference can be represented by a stochastic integral. The expression for the external entropy difference is a path functional of $\phi$. Therefore its correct representation in terms of the random path $\{\phi(t)_{t\in [0,T]}\}$ is determined by the Ito formula for functionals that can be found in \cite{Itofunc}. We apply it to the integral in Eq. \eqref{equ:inth} and find
\begin{align}
    \nonumber
    &\int\limits_t^{t+T}\mathrm{d}\phi(\omega)\cdot\nabla^{-1}\frac{\boldsymbol{J}^{\mathrm{a}}-\boldsymbol{\bar{J}}^{\mathrm{a}}}{M}=\int\limits_t^{t+T}\mathrm{d}t\,(\boldsymbol{J}-\boldsymbol{\bar{J}})\cdot\frac{\boldsymbol{J}^{\mathrm{a}}-\boldsymbol{\bar{J}}^{\mathrm{a}}}{M}\\
    \\&+\int\limits_t^{t+T}\mathrm{d}t[\nabla\mathbf{\Lambda}(t),\nabla\mathbf{\Lambda}(t)]\cdot\epsilon\frac{\delta}{\delta\phi}\nabla^{-1}(\boldsymbol{J}^{\mathrm{a}}-\boldsymbol{\bar{J}}^{\mathrm{a}})
\end{align}
Using $[\nabla\Lambda_\mu(\boldsymbol{r},t),\nabla\Lambda_\nu(\boldsymbol{r'},t)]=-\delta_{\mu\nu}\nabla^2\delta(r-r')\,$, we find
\begin{align}
    \nonumber
    &\epsilon\Delta\mathcal{S}_{\mathrm{ext}}[\phi,t,t+T]=\\
    &\int\limits_t^{t+T}\mathrm{d}t\,\left((\boldsymbol{J}-\boldsymbol{\bar{J}})\cdot\frac{\boldsymbol{J}^{\mathrm{a}}-\boldsymbol{\bar{J}}^{\mathrm{a}}}{M}-\epsilon\mathrm{tr}\nabla^2\frac{\delta}{\delta\phi}\nabla^{-1}(\boldsymbol{J}^{\mathrm{a}}-\boldsymbol{\bar{J}}^{\mathrm{a}})\right)
\end{align}
According to the definition of the active current it holds that 
\begin{equation}\label{equ:trf}
   \mathrm{tr}\nabla^2\frac{\delta}{\delta\phi}\nabla^{-1}(\boldsymbol{J}^{\mathrm{a}}-\boldsymbol{\bar{J}}^{\mathrm{a}})=0 
\end{equation}
Therefore
\begin{align}
    \epsilon\langle\Delta\mathcal{S}_{\mathrm{ext}}[\phi,t,t+T]\rangle=
    \langle\int\limits_t^{t+T}\mathrm{d}t\,(\boldsymbol{J}^\mathrm{det}-\boldsymbol{\bar{J}}^\mathrm{det})\cdot\frac{\boldsymbol{J}^{\mathrm{a}}-\boldsymbol{\bar{J}}^{\mathrm{a}}}{M}\rangle
\end{align}
where we have made use of that 
\begin{align}
    \langle\int\mathrm{d}t\,(\boldsymbol{J}^\mathrm{N}-\boldsymbol{\bar{J}}^\mathrm{N})\cdot \boldsymbol{Y}\rangle=0
\end{align}
holds for any (semi?) martingale $\boldsymbol{Y}$. Consequently
\begin{align}
    \nonumber
  \epsilon\overline{\frac{\mathrm{d}}{\mathrm{d}t}\mathcal{S}[\boldsymbol{X}(t)]}= &\frac{\mathrm{d}}{\mathrm{d}t}\left(\langle F[\phi(t,\omega)]\rangle-\langle\mathcal{V}[\phi(t,\omega),t]\rangle\right)\\
  &+\frac{1}{M}\langle(\boldsymbol{J}^\mathrm{det}-\boldsymbol{\bar{J}}^\mathrm{det})(t)\cdot\left(\boldsymbol{J}^{\mathrm{a}}-\boldsymbol{\bar{J}}^{\mathrm{a}}\right)(t)\rangle\,,
\end{align}
such that we recover the result of \cite{Nardini2017} for the steady state EPR. The functional Ito formula can also be directly applied to the expression for the entropy difference in Eq. \eqref{equ:eprdif}. Here in the same way as before we obtain 
\begin{align}
    \nonumber
    \epsilon\Delta\mathcal{S}_{\mathrm{ext}}[\phi,t,t+T]=&\int\limits_t^{t+T}\mathrm{d}t\,(\boldsymbol{J}-\boldsymbol{\bar{J}})\cdot\frac{\boldsymbol{J}^{\mathrm{det}}-\boldsymbol{\bar{J}}^{\mathrm{det}}}{M}
    \\
    -&\int\limits_t^{t+T}\mathrm{d}t\,\epsilon\mathrm{tr}\nabla^2\frac{\delta}{\delta\phi}\nabla^{-1}(\boldsymbol{J}^{\mathrm{det}}-\boldsymbol{\bar{J}}^{\mathrm{det}})
\end{align}
Again using Eq. \eqref{equ:trf} we find
\begin{align}
    \nonumber
  \epsilon\overline{\frac{\mathrm{d}}{\mathrm{d}t}\mathcal{S}[\boldsymbol{X}(t)]}= &-\frac{\mathrm{d}}{\mathrm{d}t}\langle\mathcal{V}[\phi(t,\omega),t]\rangle\\
  &+\frac{1}{ M}\langle\,(\boldsymbol{J}^{\mathrm{det}}-\boldsymbol{\bar{J}}^{\mathrm{det}})\cdot(\boldsymbol{J}^{\mathrm{det}}-\boldsymbol{\bar{J}}^{\mathrm{det}})\,\rangle
   \\
   &-\epsilon M \langle\, \mathrm{tr}\frac{\delta}{\delta\phi}\nabla^2\frac{\delta\mathcal{F}}{\delta\phi}\,\rangle
    \,.
\end{align}
\\
Using the decomposition of the deterministic current into a symmetric and asymmetric part defined as 
\begin{equation}\label{equ:asc}
    \boldsymbol{J}^\mathrm{S}=-M\nabla \frac{\delta \mathcal{V}}{\delta\phi}\,,
   \quad
   \text{and}
   \quad
   \boldsymbol{J}^\mathrm{A}=\boldsymbol{J}^\mathrm{det}-\boldsymbol{\bar{J}}^\mathrm{det}+M\nabla \frac{\delta \mathcal{V}}{\delta\phi}\,,
\end{equation}
as introduced in $\cite{Li_2021}$ we can obtain a third representation for the entropy difference and the average EPR. First we find
\begin{align}
    \nonumber
    &\epsilon\Delta\mathcal{S}_{\mathrm{ext}}[\phi,t,t+T]=\\
    &\int\limits_t^{t+T}\mathrm{d}t\,\left(\frac{\mathrm{d}}{\mathrm{d}t}\mathcal{V}[\phi(t,\omega),t]-\partial_t\mathcal{V}[\phi(t,\omega),t]\right)\\\nonumber
    &\int\limits_t^{t+T}\mathrm{d}t\,\left((\boldsymbol{J}-\boldsymbol{\bar{J}})\cdot\frac{\boldsymbol{J}^{\mathrm{A}}}{M}-\epsilon\mathrm{tr}\nabla^2\frac{\delta}{\delta\phi}\nabla^{-1}\boldsymbol{J}^{\mathrm{A}}\right)\label{equ:e3}
\end{align}This expression can be further transformed with the help of the functional Fokker-Plank equation \cite{Chavanis2015} that determines the evolution of $\mathcal{P}(t)$ and which reads
\begin{align}
   \partial_t\mathcal{P}[\phi,t] =-M\mathrm{tr}\frac{\delta }{\delta\phi}\left(\epsilon\nabla^2\frac{\delta }{\delta\phi}\mathcal{P}[\phi,t] -\frac{1}{M}\nabla\boldsymbol{J}_\mathrm{det}\mathcal{P}[\phi,t] \right)\,.
\end{align}
Inserting the quasipotential representation Eq. \eqref{equ:qua} one finds
\begin{align}
    \nonumber
     \partial_t\mathcal{V}[\phi,t]=
     &-M\left(\nabla\frac{\delta \mathcal{V}}{\delta\phi}\cdot\nabla\frac{\delta\mathcal{V} }{\delta\phi}+\frac{1}{M}\nabla\frac{\delta\mathcal{V} }{\delta\phi}\cdot(\boldsymbol{J}^\mathrm{det}-\boldsymbol{\bar{J}}^\mathrm{det})\right)\\
     &-\epsilon\mathrm{tr}\frac{\delta}{\delta\phi}\nabla\left(\boldsymbol{J}^\mathrm{det}-\boldsymbol{\bar{J}}^\mathrm{det}+M\nabla \frac{\delta \mathcal{V}}{\delta\phi}\right)\,.
\end{align}
Therefore
\begin{align}\label{equ:quaev}
    \partial_t \mathcal{V}=\frac{1}{M}\boldsymbol{J}^\mathrm{S}\cdot\boldsymbol{J}^\mathrm{A}-\epsilon\mathrm{tr}\nabla^2\frac{\delta}{\delta\phi}\nabla^{-1}\boldsymbol{J}^{\mathrm{A}}\,,
\end{align}
This means that the symmetric and the asymmetric part of the deterministic current become perpendicular whenever $\delta_\phi\nabla\boldsymbol{J}^\mathrm{a}$ is trace free. 
Inserting this result in Eq. \eqref{equ:e3} we obtain 
\begin{align}
    \nonumber
    \epsilon\Delta\mathcal{S}[\phi,t,t+T]=
    \int\limits_t^{t+T}\mathrm{d}t\,(\boldsymbol{J}^\mathrm{A}+\boldsymbol{J}^\mathrm{N})\cdot\frac{\boldsymbol{J}^{\mathrm{A}}}{M}
\end{align}
Therefore 
\begin{equation}
  \overline{\frac{\mathrm{d}}{\mathrm{d}t}\mathcal{S}[\boldsymbol{X}(t)]}=\frac{1}{M}\langle\boldsymbol{J}^{\mathrm{A}}\cdot \boldsymbol{J}^{\mathrm{A}}\rangle
\end{equation}
which is the result already found in \cite{Li_2021} for the steady state.

\section{Properties of the symmetric and asymmetric current}\label{app:prop}
The SDE for $\mathcal{V}$ is obtained using the ordinary Ito formula 
\begin{align}
    \nonumber
    \frac{\mathrm{d}}{\mathrm{d}t}\mathcal{V}&=
    \partial_t\mathcal{V}[\phi,t]+\nabla\frac{\delta\mathcal{V} }{\delta\phi}\cdot(\boldsymbol{J}^\mathrm{det}-\boldsymbol{\bar{J}}^\mathrm{det})-\epsilon M\mathrm{tr}\frac{\delta }{\delta\phi}\nabla^2\frac{\delta \mathcal{V}}{\delta\phi}\\
    &+\nabla\frac{\delta\mathcal{V} }{\delta\phi}\cdot \boldsymbol{J}_{\mathrm{N}}
\end{align}
Inserting the definition of the symmetric and the asymmetric current Eq. \eqref{equ:asc} and using the result Eq. \eqref{equ:quaev} we find
\begin{align}\label{equ:zw}
    \frac{\mathrm{d}}{\mathrm{d}t}\mathcal{V}=-\frac{1}{M}\boldsymbol{J}^\mathrm{S}\cdot\boldsymbol{J}^\mathrm{S}-\frac{1}{M}\boldsymbol{J}^\mathrm{S}\cdot\boldsymbol{J}^\mathrm{N}+\epsilon \mathrm{tr}\frac{\delta }{\delta\phi}\nabla(\boldsymbol{J}^\mathrm{S}-\boldsymbol{J}^\mathrm{A})\,,
\end{align}
In case we have $\langle\mathrm{tr}\frac{\delta }{\delta\phi}\nabla\boldsymbol{J}^\mathrm{A}\rangle=0$ the steady state, i.e. the asymmetric and the symmetric part of the deterministic current become on average perpendicular, the expected value of the squared symmetric current is given by
\begin{align}
    \langle \boldsymbol{J}^\mathrm{S}\cdot\boldsymbol{J}^\mathrm{S} \rangle &= \epsilon M^2\langle \mathrm{tr}\frac{\delta }{\delta\phi}\nabla^2\frac{\delta \mathcal{F}}{\delta\phi}\rangle\,.
\end{align}
On average the intensity of the symmetric current is determined by the properties of the free energy. 
\section{Non-equilibrium field dynamics}\label{sec:nefd}
\begin{itemize}
    \item boundary conditions?
    \item set $M$ to unity to save space?
    \item make it more general? include relaxational models by replacing $\nabla$ with a general noise operator $\sigma$? 
\end{itemize}
We study a generic multi-component field model with conserved mass flow. The evolution of $\phi=(\phi_1, \phi_2 \dots, \phi_d )^T$ is given by the continuity equation
\begin{equation}\label{equ:esta}
    \partial_t \phi = -\nabla\cdot\boldsymbol{J}
    \qquad
    \text{with}
    \qquad
    \boldsymbol{J}=\boldsymbol{J}^{\mathrm{det}}+ \boldsymbol{J}^{\mathrm{N}}\,.
\end{equation}
The noise current is given by $\boldsymbol{J}^{\mathrm{N}} = \sqrt{2M\epsilon}\mathbf{\Lambda}$, where $\boldsymbol{\Lambda}$ is a space time white noise process with covariance function
\begin{equation}
    \langle\, \Lambda_\mu(\boldsymbol{r},t)\Lambda_\eta(\boldsymbol{r}',t')\rangle=\delta_{\eta\mu}\delta(\boldsymbol{r}-\boldsymbol{r}')\delta(t-t')\,.
\end{equation}
The deterministic current can be decomposed
as 
\begin{equation}\label{equ:deco2}
    \boldsymbol{J}^{\mathrm{det}}=-M\nabla\frac{\delta \mathcal{F}}{\delta \phi}+\boldsymbol{J}^{\mathrm{a}}\,.
\end{equation}
The active current  $\boldsymbol{J}^{\mathrm{a}}$ 
satisfies
\begin{align}\label{equ:trfe}
    &\frac{\delta }{\delta\phi_\eta(\boldsymbol{r}')}\nabla^{-1}\boldsymbol{J}_\mu^{\mathrm{a}}(\phi(\boldsymbol{r}))+\frac{\delta }{\delta\phi_\mu(\boldsymbol{r}')}\nabla^{-1}\boldsymbol{J}_\eta^{\mathrm{a}}(\phi(\boldsymbol{r}))=0\,,
\end{align}
for all $\eta,\mu,\boldsymbol{r},\boldsymbol{r}'$.
Therefore, the presence of the active current renders the full deterministic current non-integrable, i.e. there is no free energy functional $\mathcal{F}'$ such that $\boldsymbol{J}^\mathrm{det}=-M\nabla\frac{\delta\mathcal{F}'}{\delta\phi}$.
The probability of observing a configuration $\phi$ at time $t$ is denoted by $\mathcal{P}[\phi,t]$. Its evolution is determined by the functional Fokker-Plank equation \cite{Chavanis2015} 
\begin{equation}
   \partial_t\mathcal{P}[\phi,t] =-M\mathrm{tr}\frac{\delta }{\delta\phi}\left(\epsilon\nabla^2\frac{\delta }{\delta\phi}\mathcal{P}[\phi,t] -\frac{1}{M}\nabla\boldsymbol{J}^\mathrm{det}\mathcal{P}[\phi,t] \right)\,.
\end{equation}
Inserting the representation $\mathcal{P}[\phi,t]=\mathrm{exp}\{-\mathcal{V}/\epsilon\}$ in terms of the quasipotential $\mathcal{V}$ one finds
\begin{align}\label{equ:quap}
    \nonumber
     \partial_t\mathcal{V}[\phi,t]=
     &-M\left(\nabla\frac{\delta \mathcal{V}}{\delta\phi}\cdot\nabla\frac{\delta\mathcal{V} }{\delta\phi}+\frac{1}{M}\nabla\frac{\delta\mathcal{V} }{\delta\phi}\cdot(\boldsymbol{J}^\mathrm{det}-\boldsymbol{\bar{J}}^\mathrm{det})\right)\\
     &-\epsilon\mathrm{tr}\frac{\delta}{\delta\phi}\nabla\left(\boldsymbol{J}^\mathrm{det}-\boldsymbol{\bar{J}}^\mathrm{det}+M\nabla \frac{\delta \mathcal{V}}{\delta\phi}\right)\,.
\end{align}
where $\boldsymbol{\bar{\cdot}}$ denotes the spatial average.  Defining the symmetric and the asymmetric current 
\begin{equation}
    \boldsymbol{J}^\mathrm{S}=-M\nabla \frac{\delta \mathcal{V}}{\delta\phi}\,,
   \quad
   \text{and}
   \quad
   \boldsymbol{J}^\mathrm{A}=\boldsymbol{J}^\mathrm{det}-\boldsymbol{\bar{J}}^\mathrm{det}+M\nabla \frac{\delta \mathcal{V}}{\delta\phi}\,,
\end{equation}
as introduced in \cite{Li_2021}, this can be rephrased as
\begin{align}
    \partial_t \mathcal{V}=\frac{1}{M}\boldsymbol{J}^\mathrm{S}\cdot\boldsymbol{J}^\mathrm{A}-\epsilon\mathrm{tr}\nabla^2\frac{\delta}{\delta\phi}\nabla^{-1}\boldsymbol{J}^{\mathrm{A}}\,,
\end{align}
This suggests two things. First, Eq. \eqref{equ:quap} is not independent of $\epsilon$. Still, any ground state $\phi_0$ of the model, i.e., a maximum of $\mathcal{P}[\phi,t]$ is characterized by $\delta_\phi \mathcal{V}\vert_{\phi=\phi_0}=0$. Secondly, in the limit $\epsilon\rightarrow 0$ the necessary condition for
 $\mathcal{V}$ represents a steady state is that symmetric and the asymmetric part of the deterministic current become perpendicular. On the other hand if $\boldsymbol{J}^\mathrm{S}\cdot\boldsymbol{J}^\mathrm{A}=0$ also holds for $\epsilon>0$, the symmetric current is related to the evolution of the quasipotential in a similar way as the deterministic current is related to the evolution of the free energy in an equilibrium field theory (see Appendix \ref{app:sym}).
\begin{equation}\label{equ:zw}
    \frac{\mathrm{d}}{\mathrm{d}t}\mathcal{V}=-\frac{1}{M}\boldsymbol{J}^\mathrm{S}\cdot\boldsymbol{J}^\mathrm{S}-\frac{1}{M}\boldsymbol{J}^\mathrm{S}\cdot\boldsymbol{J}^\mathrm{N}+\epsilon\mathrm{tr}\frac{\delta}{\delta\phi}\nabla\boldsymbol{J}^{\mathrm{S}}\,,
\end{equation}
It represents that part of the dynamics which is determined by the tendency of the system to assume its ground state, i.e. to minimize the quasipotential. Thus, it represents the aspects of the model's behavior that are equilibrium-like. Its counterpart, the asymmetric current represents that part of the dynamics that does not help to restore the ground state. Such currents do not occur in an equilibrium model.
\section{Coupling of fluctuations}\label{app:ap}
 In complex amplitude-phase form the Fourier coefficients can be expressed as  
\begin{align}
    \hat{\phi}^1_\mu=\mathcal{A}_\mu e^{iq_j \theta_\mu}\,.
\end{align}
In order to keep the calculations clear, we will assume in the following that the domain size $L=2\pi$. By applying the Ito formula to
\begin{align}
    \theta_\mu&=\arctan\frac{\mathrm{Im}\hat{\phi}_\mu}{\mathrm{Re}\hat{\phi}_\mu}\,,\\\nonumber
    \\
    \mathcal{A}_\mu&=\sqrt{\mathrm{Re}\hat{\phi}_\mu+\mathrm{Im}\hat{\phi}_\mu}\,,
\end{align}
and noting that the quadratic variations for the amplitude and the phase processes are determined by
\begin{align}
    \nabla^2 \sqrt{x^2+y^2}=\frac{1}{\sqrt{x^2+y^2}}\,,\\
    \nabla^2 \arctan\left(\frac{x}{y}\right)=0\,,
\end{align}
I obtain the respective stochastic equations of motions for the amplitude and the phase dynamics.
\begin{align}\nonumber
    &\partial_t\mathcal{A}_A=-\left[\alpha_A^1 \mathcal{A}_A+\mathrm{Re}(\mathrm{B}e^{-i\theta_A})+(\kappa-\delta)\mathcal{A}_B\cos(\theta_A-\theta_B)\right]+\frac{\tau}{\mathcal{A}_A}+\xi_{\mathcal{A}_A}\,,\\\nonumber
    &\partial_t\mathcal{A}_B=-\left[(\alpha_B^1 )\mathcal{A}_B+(\kappa+\delta)\mathcal{A}_A\cos(\theta_A-\theta_B)\right]+\frac{\tau}{\mathcal{A}_A}+\xi_{\mathcal{A}_B}\,,\\\nonumber
    &\partial_t\theta_A=\sin(\theta_A-\theta_B)\frac{\mathcal{A}_B}{\mathcal{A}_A}(\kappa-\delta)+\mathrm{Im}(\mathrm{B}e^{-i\theta_A})\frac{1}{\mathcal{A}_A^1}+\xi_{\theta_A}\,,\\\nonumber
    &\partial_t\theta_B=-\sin(\theta_A-\theta_B)\frac{\mathcal{A}_A}{\mathcal{A}_B}(\kappa+\delta)+\xi_{\theta_B}\,,
\end{align}
with
\begin{align}
    \mathrm{B}= \sum\limits_{k,l}(q_k^2+2q_kq_l)\mathcal{A}^k\mathcal{A}^l\mathcal{A}^{1-k-l}e^{i(\theta_k+\theta_l+\theta_{1-k-l})}\,.
\end{align}
For the transformed noise terms I find
\begin{align}
    \xi_{\mathcal{A}_\mu}=\sqrt{\tau}\frac{\mathrm{Re}\hat{\phi}^1_\mu\mathrm{Re}\xi_\mu+\mathrm{Im}\hat{\phi}^1_\mu\mathrm{Im}\xi_\mu}{\mathcal{A}_\mu}\,,\\\nonumber
    \\
    \xi_{\theta_\mu}=\sqrt{\tau}\frac{\mathrm{Re}\hat{\phi}^1_\mu\mathrm{Im}\xi_\mu-\mathrm{Im}\hat{\phi}^1_\mu\mathrm{Re}\xi_\mu}{\mathcal{A}_\mu^2}\,,
\end{align}
such that the co-variance matrix again turns out to be diagonal. The transformation has rendered the noise  multiplicative. The diagonal elements of its co-variances  matrix are given by $(\tau,\tau,\frac{\tau}{\mathcal{A}_A^2},\frac{\tau}{\mathcal{A}_B^2})$. 

Coupling of fluctuations of different modes via a non-linearity requires a well defined relationship between the phases of both modes. This is only possible if both modes are activated in the groundstate. 
\bibliography{bib}% Produces the bibliography via BibTeX.

\end{document}
%
% ****** End of file apssamp.tex ******
