\begin{lemma}
	Let $\rho_K : G(K) \longrightarrow H(K)$ be a natural transformation  between $\Phi$-group functors associated to vector groups such that 
	\[ \rho_K(\lambda \cdot_i g) = \lambda \cdot_i \rho_K(g)\]
	for all $g \in G(K), \lambda \in K$.
	If $1/2 \in \Phi$, then $L = \ker \rho$ and $I = \Ima \rho$ are vector groups.
	Moreover, if $1/2 \notin \Phi$,  $\ker \rho_K \cap G_2(K)$ is the $K$-linear span of $(\ker \rho \cap G_2(\Phi)) \otimes 1$, and if $\rho(G_2(K)) = I_2(K)$
	for all $K \in \Phi\textbf{-alg,}$
	then $L$ and $I$ are vector groups as well.
	\begin{proof}
		
		It is trivial to see that $L$ is closed under both scalar multiplications. So, we conclude that $L$ is a vector group, using an additional assumption if $1/2 \notin \Phi$.
		
		The group $I$ is closed under both scalar multiplications as well. So, $I$ is definitely a vector group if $1/2 \in \Phi$.
		For the third axiom of vector groups, note that $I_2(K) = \rho(G_2(K)) = \rho(G_2 \otimes K) = I_2 \otimes K.$
	\end{proof}
\end{lemma}

\begin{definition}
	We call a natural transformation $\rho : G \longrightarrow H$ between the $\Phi\textbf{-grp}$ functors associated to vector groups $G,H$, a \textit{vector group homomorphism} if 
	\[ \rho(\lambda \cdot_i g) = \lambda \cdot_i \rho(g),\]
	and $\rho^{-1}(H_2) \subseteq G_2 \ker\rho.$
	If $1/2 \notin \Phi$ we require, additionally, that 
	\[ \ker \rho_K \cap G_2(K) = \langle g \otimes 1 \in B \otimes K | g \in G_2 \cap \ker \rho \rangle ,\]
	where the right-hand side is the span as a $K$-module.
	We use $\textbf{VecGrp}$ to denote the category of vector groups over $\Phi$ with these morphisms. If there is any doubt about the base ring, we write $\Phi\textbf{-VecGrp}$.
\end{definition}

\begin{remark}
	All the conditions, except $\rho^{-1}(H_2) \subset G_2 \ker\rho$, are justified by the fact that $\ker \rho$ and $\Ima \rho$ must be vector groups as well.
	This other condition translates into $\rho^{-1}(H_2) \subset G_2$ for injective maps, i.e., injective morphisms preserve the second component.
	The more general condition is obtained by requiring that this holds and
	that $G \longrightarrow H$ factors through some kind of vector group $L = G/\ker \rho$ with $L_2$ obtained by projecting $G_2$.
	
	At this moment, we should be careful if we write $G/K$ for normal $K \le G$, since we did not prove that $G/K$ is a vector group for each normal vector subgroup $K$ in general. However, we can always consider it as a $\Phi$-group with well-defined scalar endomorphisms and a well-defined subgroup $(G/K)_2$. Later, we will see that $G/K$ is in fact a vector group.
\end{remark}

\begin{lemma}
	\label{Lemma reparam}
	Let $L_1 \oplus L_2$ be a $\mathbb{Z}$-graded Lie algebra over a ring $\Phi$ with $1/2 \in \Phi$. Then $L_1 \oplus L_2$ forms a vector group with operation $$(a,b)(c,d) = (a+c,b+d + [a,c]/2).$$
	Moreover, each vector group $G$ over $\Phi$ is isomorphic to the vector group associated to $\Lie(G)$ by mapping 
	$$ (a,b) \longmapsto (a,b - \psi(a,a)/2).$$
	Hence, the third axiom for vector groups will also hold if $1/2 \in \Phi$.
		\begin{proof}
			Note that $G = L_1 \oplus L_2$ so that all axioms for vector groups are trivially satisfied.
			
			Now, we prove that each vector group can be reparametrized in this fashion. Lemma \ref{lemma 2g2 - g1sq} proves that $2b - \psi(a,a) \in G_2 \subset \text{Lie}(G)$.
			So, we conclude that 
			$(a,b - \psi(a,a)/2) \in \text{Lie}(G).$ To observe the surjectivity of this map, take arbitrary $(a,b) \in G$ and note that the coset $(a,b) G_2$ maps to all $(a,s) \in \text{Lie}(G)$.
			On the other hand, the reparametrization is obviously injective.
			
			A direct computation shows that this reparametrization is a group homomorphism if one uses that $\psi(a,c) - \psi(c,a) = [a,c]$.
			Moreover, the scalar multiplications and $G_2$ are preserved, so that we have a vector group isomorphism.
		\end{proof}
\end{lemma}

\begin{remark}
	The fact that the operation in the previous lemma is given by
	$$(a,b)(c,d) = (a+c,b+d + [a,c]/2)$$
	shows that vector groups, if $1/2 \in \Phi$, correspond directly to groups of exponentials. Specifically, if we consider the Lie algebra
	$$ L = \Phi \zeta \oplus L_1 \oplus L_2,$$
	with $[\zeta ,l_i] = il_i$ for $l_i \in L_i$, then the group 
	$$ \exp(L_1 \oplus L_2) \le \text{Aut}(L)$$ has the aforementioned operation.
\end{remark}

Let $G$ be a vector group over $\Phi$, then $G(K)$ is a vector group over $K$. We write $L \longmapsto G_K(L)$ to denote the $K\textbf{-Grp}$ functor associated to the vector group $G_K = G(K)$.

\begin{lemma}
	\label{lemma G(K) is Lie(G)K}
	Let $G$ be a vector group and $K \in \Phi\textbf{-alg}$, then  $\text{Lie}(G_K) \cong \Lie(G) \otimes K$.
		\begin{proof}
			First, recall the equality $G_2(K) = G_2 \otimes K$ which is assumed in the definition of a vector group if $1/2 \notin \Phi$. We can also obtain the equality from Lemma \ref{Lemma reparam} if $1/2 \in \Phi$. Secondly, the possible first coordinates of elements of $G_K$ are $K$-linear combinations of possible first coordinates of $G$. We conclude that $\text{Lie}(G_K) \cong \text{Lie}(G) \otimes K$ by Lemma \ref{lemma Lie G}.
		\end{proof}
\end{lemma}