\section*{Introduction}

Structurable algebras were introduced over fields of characteristic different from $2$ and $3$ and classified over fields of characteristic $0$ by Bruce Allison \cite{ALL78}. Oleg Smirnov \cite{smirnov1990} generalized the classification of these algebras to fields with characteristic different from $2$, $3$ and $5$. More recently, Anastasia Stavrova \cite{STAV20} classified the structurable algebras over fields of characteristic different from $2$ and $3$ in a different fashion.
These algebras are useful in the study of $5$-graded Lie algebras and algebraic groups \cite{Gar01,Kru07,BOE19, Cuy21}, which is the point of view we shall take throughout this paper. %However, these algebras can also be related to Lie algebras with different gradings, e.g., $D_4$ gradings \cite{Eld07}.
Furthermore, the definition of structurable algebras can be extended to arbitrary rings \cite[Section 5]{ALLFLK93}, in a manner suitable for the study of $5$-graded Lie algebras. 
However, a good definition for structurable algebras over arbitrary rings, which allows for a generalization of the links with (algebraic) groups, was still lacking.

We will not only focus on structurable algebras but also Kantor pairs and Kantor triple systems as introduced ---under the name generalized Jordan triple systems of second order--- by Isaiah Kantor \cite{Kan72}. 
Kantor pairs are useful for the study of $5$-graded Lie algebras and related groups as well \cite{ALLFLK99,ALLFLK17}.
Just like for structurable algebras, there is no good theory yet for Kantor pairs over arbitrary rings.
Since each structurable algebra induces a Kantor pair with the same $5$-graded Lie algebra, we will focus on generalizing Kantor pairs to ``operator Kantor pairs'' as a means to generalize both.

We introduce the notion of an operator Kantor pair.
The operator Kantor pairs relate to Kantor pairs, like quadratic Jordan pairs \cite{Loos74,Loos75} relate to linear Jordan pairs (``verbundene Paare'' in \cite{Mey70Kon}). To be specific, instead of the $3$-linear map $(a,b,c) \mapsto V_{a,b} c$ associated to a Kantor pair, we will consider certain operators $Q^\text{grp}$, $T$ and $P$ which can be constructed from $V$ over rings with\footnote{We will often talk about ``a ring $\Phi$ with $1/n$'' or write ``$1/n \in \Phi$'' when we mean that we require $n$ to be invertible in the ring $\Phi$.} $1/6$.

We construct, for each operator Kantor pair $(G^+,G^-)$, an associated $5$-graded Lie algebra. If $1/2 \in \Phi$, then this $5$-graded Lie algebra can, alternatively, be constructed from a genuine Kantor pair. We also construct a group $G(G^+,G^-)$ generalizing the projective elementary group. We show that the group $G(G^+,G^-)$ has the properties which are desired for the projective elementary group if $G^+$ and $G^-$ are projective modules. Namely, we prove that it is not only a group of automorphisms of a Lie algebra, but that the groups $G^+$ and $G^-$ correspond exactly to the ``positive" and ``negative" part of $G(G^+,G^-)$.


For rings with $1/30 \in \Phi$, we show that each Kantor pair induces an operator Kantor pair.
Moreover, we prove that all classes of central simple structurable algebras induce operator Kantor pairs.
More precisely, we show:
\begin{enumerate}
	\item That a quadratic Jordan algebra becomes an operator Kantor pair with operators $Q^\text{grp} = Q,$ $T = 0,$ and $P_xy = Q_xQ_yx$ by establishing the link between the operators and parts of quasi-inverses.
	\item Similarly, we establish that a Kantor pair associated to a Hermitian form, as studied by John Faulkner and Bruce Allison \cite{ALLFLK99}, induces an operator Kantor pair over arbitrary rings using an appropriate generalization of quasi-inverses.
	This establishes that each associative algebra with involution induces an operator Kantor pair.
	\item For structurable algebra associated to a Hermitian form \cite[8.iii]{ALL78} we also construct operator Kantor pairs, by making use of a Peirce context in the sense of \cite{ALLFLK99} for this class of algebras.
	\item For the other classes, i.e.,
	\begin{enumerate}
		\item forms of the tensor products of composition algebras,
		\item forms of matrix structurable algebras, described in terms of Hermitian cubic norm structures \cite{DeMedts2019},
		\item Smirnov algebras (if $1/2 \in \Phi$),
	\end{enumerate}
    we also prove that these induce operator Kantor pairs using certain algebras defined over $\mathbb{Z}$. However, for the tensor product of composition algebras and Smirnov algebras, we will not consider arbitrary forms.
\end{enumerate}


\paragraph{Outline.}

In \cref{se:vectorgroups}, we introduce vector groups. These groups are $\Phi$-group functors $G$ equipped with a subgroup functor $G_2$, together with some sort of scalar multiplications. These vector groups can be identified with $\Phi$-groups that can be represented as formal power series $1 + tg_1 + t^2g_2 + \dotsm$. For each vector group, we construct a universal representation, which is a representation in a Hopf algebra $H$, in terms of such formal power series. We also define homogeneous maps on such groups and develop a method that will allow us to derive necessary equalities involving homogeneous maps using the universal representation.

In \cref{se:pairs}, we introduce some notation to work with a pair of vector group representations, formulate Theorem \ref{thm main} which we prove in the appendix, and list a multitude of equations in Lemma \ref{Lemma equations} that hold for pairs of vector group representations. Theorem \ref{thm main} and Lemma \ref{Lemma equations} are technical results that will be very useful when proving that certain pairs of vector group representations correspond to operator Kantor pairs.

In \cref{se:okpairs}, we introduce operator Kantor pairs and prove the aforementioned results  on general operator Kantor pairs.
Namely, first we introduce pre-Kantor pairs $(G^+,G^-)$, which are pairs of vector groups with homogeneous operators. For such a pair we introduce a Lie algebra $L$, and we construct vector group representations to the endomorphism algebra of $L$ by making use of the operators.
Secondly, we define operator Kantor pairs , which are pre-Kantor pairs satisfying some additional equations on those operators. We prove for operator Kantor pairs that the constructed vector group representations induce maps $G^\pm \longrightarrow \text{Aut}(L)$. Lastly, in Theorem \ref{thm weights lie algebra} we prove that the group $G \le \text{Aut}(L)$ generated by $G^+$ and $G^-$ is similar to the projective elementary group of quadratic Jordan pairs if $G^+$ and $G^-$ are projective.

 In \cref{se:lie}, we prove the precise link between Kantor pairs, pre-Kantor pairs and operator Kantor pairs. Namely, we show that these three classes are in a one to one correspondence if $1/30 \in \Phi$. Finally, in \cref{se:struct}, we use the different classes of central simple structurable algebras to define operator Kantor pairs.

\bigskip

Throughout the paper, $\Phi$ denotes the base ring over which we work.
Unless stated otherwise, $\Phi$ is an arbitrary commutative unital ring.
The category $\Phi\textbf{-alg}$ is the category of associative, unital, commutative $\Phi$-algebras. 
All $\Phi$-modules are assumed to be left modules unless stated explicitly otherwise (which will occur only in paragraph \ref{sss:herm}.)

