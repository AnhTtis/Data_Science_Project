\section{Computations}

\subsection{Lemma 1}

For a pre-Kantor pair $(G^+,G^-)$ we have an operator $V$.
We will often write $V_{x,y}z$ with $z \in A^+ \cong G^+/(G^+_2)$, i.e., the space of possible first coordinates.
In that case, the definition of $V$ becomes
\[ V_{x,y }z = - Q^{(1,1)}_{z,x} y.\]

\begin{lemma}
	\label{Lemma further linearizations P,R}
	Let $(G^+,G^-)$ be a pre-Kantor pair. The following equations hold for all $g,h,x,a \in G^+, y,y',b,c \in G^-$:
	\begin{align*}
		P^{((1,2),(1,1))}_{a,g}(y,y') = &  V_{Q_g y, y'} a + V_{g,y'} V_{g,y} a , \\
		R^{((2,2),(1,1))}_{g,h}(c,b) = & - V_{g,b}T^{(1,2)}_{g,h} c - V_{h,b} T^{(2,1)}_{g,h} c  + \psi(Q_g c,Q_h b) + \psi(Q^{(1,1)}_{g,h} c, Q^{(1,1)}_{g,h} b) + \psi(Q_h c,Q_g b), \\
		P^{(1,1,1)}_{g,x,a}y = & \; Q^{(1,1)}_{g,x}Q_{y^{-1}} a - V_{a,y} Q^{(1,1)}_{g,x} y, \\
		R^{(2,1,1)}_{g,x,a} y = &T^{(2,1)}_{g,x}Q_{y^{-1}} a - V_{a,y} T^{(2,1)}_{g,x} y + \psi(Q_g y, Q^{(1,1)}_{x,a} y) + \psi(Q^{(1,1)}_{g,x}y,Q^{(1,1)}_{g,a} y).
	\end{align*}
	\begin{proof}
		These equations are obtained by linearizing the expressions in the Definition \ref{Definition pre-Kantor pair} for $P^{(3,(1,1))}$, $R^{(4,(1,1))},$  $P^{(2,1)}$ and $R^{(3,1)}$ respectively.
		For the first equation, linearizing only yields
		\[ Q_{T^{(1,2)}_{a,g}y} y' - V_{a,y'} Q_g y - V_{g,y'} Q^{(1,1)}_{a,g} y.  \]
		Applying $T^{(1,2)}_{a,g} y = [a,Q_gy],$ $Q_{[u,v]} w = Q^{(1,1)}_{u,v} w - Q^{(1,1)}_{v,u} w$, and $Q^{(1,1)}_{a,g} y = - V_{g,y} a$ yields the desired result. The other expressions are obtained more easily.
	\end{proof}
\end{lemma}

\begin{lemma}
	Let $(G^+,G^-)$ be a pre-Kantor pair. For $x \in G^+, a,b \in G^-$ we have
	\label{Lemma linerizations tau}
	$$ \tau^{(2,(1,1))}_{x,a,b}  = - V_{b,Q_{x}a} + V_{a,x}V_{b,x},$$
	$$ \tau^{((1,1),2)}_{b,a,x^{-1}} =  - V_{Q_{x}a,b} + V_{b,x}V_{a,x}.$$
	\begin{proof}
		In order to avoid subtleties with the second coordinate, we assume that $d(g_1,g_2) = (d_1(g_1),d_2(g_2))$ with $d_i$ a linear function for derivations $d$. This can be assumed if we work with the first $2$ coordinates of the universal representation.
		Consequently, we can say that
		$$ \delta(\psi(a,b)) = \psi(\delta(a),b) + \psi(a,\delta(b)),$$
		without ever running into identification issues.
		
		
		Both equations that we want to prove are equivalent. To observe that this is the case set $g = (t \cdot_1 a)b \in G(\Phi[t])$, and consider the equation
		$$ \tau_{x,g} + \tau_{g^{-1},x^{-1}} - V_{x,g}^2 = 0$$
		used in the definition of $\tau$, cfr., Definition \ref{definition V tau}.
		Comparing the terms belonging to $t$ shows that both equations are equivalent.
		
		So, suppose that $x \in G^+, a,b \in G^-$.
		We show that the first equality holds on $G^-$ and the second on $G^+$, in order to prove that both equations hold in $\text{InStr}(G)$.
		
		We first evaluate the $\tau^{(2,(1,1))}_{x,a,b}$ at $g \in G^-$. 
		The first coordinate is
		$$ P^{(1,1,1)}_{g,b,a}x^{-1} = (V_{Q_{x}a,b} + V_{a,x}V_{b,x})_1 g,$$
		by Lemma \ref{Lemma further linearizations P,R}. Hence, the left and right-hand side agree on the first coordinate.
		The second coordinate is
		$$ R^{(2,1,1)}_{g,b,a} x^{-1} - \psi(Q_g x, Q^{(1,1)}_{b,a} x) + \psi(P^{(1,1,1)}_{g,b,a}x^{-1},g)$$
		which equals
		$$ T^{(2,1)}_{g,b}Q_{x} a - V_{a,x} T^{(2,1)}_{g,b} x^{-1} + \psi(Q^{(1,1)}_{g,b}x,Q^{(1,1)}_{g,a} x) + \psi(P^{(1,1,1)}_{g,b,a}x^{-1},g).$$ 
		We can factor $V_{a,x}$, using the expression for $P^{(1,1,1)}$ of Lemma \ref{Lemma further linearizations P,R} and the fact that $\delta(\psi(a,b))$ equals $\psi(\delta a,b) + \psi(a, \delta b)$ to obtain
		$$  T^{(2,1)}_{g,b}Q_{x} a + \psi(V_{b,Q_xa}g,g) + V_{a,x}(T^{(2,1)}_{g,b}x + \psi(V_{b,x} g,g)).$$
		So, by evaluating the right-hand side of the equation we want to prove, we conclude that the the left-hand side and the right-hand side have the same action on $G^-$.
		
		Now, we prove that the second equality holds as functions on $G^+$.
		The first coordinate of $\tau^{(1,1)}_{b,a,x^{-1}} g$ equals
		$$ P^{((1,2),(1,1))}_{g,x}(a,b) = V_{Q_{x}a,b}g + V_{x,b}V_{x,a}g.$$
		This proves the equality on the first coordinate.
		The second coordinate equals
		$$ R^{((2,2),(1,1))}_{g,x}(a,b) - \psi(Q_ga,Q_xb) - \psi(Q_gb,Q_xa) + \psi(P^{((1,2),(1,1))}_{g,x}(a,b),g).$$
		By substituting the the linearisation of $R$ appearing in Lemma \ref{Lemma further linearizations P,R}, we can conclude that the second coordinate equals
		\begin{equation}
			\label{eq1}
			- V_{g,b}T^{(1,2)}_{g,x} a - V_{x,b} T^{(2,1)}_{g,x} a  + \psi(Q^{(1,1)}_{g,x} a, Q^{(1,1)}_{g,x} b) + [Q_x a,Q_g b] +  \psi(P^{((1,2),(1,1))}_{g,x}(a,b),g).
		\end{equation}
		We want to prove that this expression equals the second coordinate of $(V_{Q_{x}a,b} + V_{b,x}V_{a,x})(g),$ i.e., that it equals
		$$ T^{(2,1)}_{g,Q_xa} b + \psi(V_{Q_xa,b}g,g) + V_{b,x}(T^{(2,1)}_{g,x}a + \psi(V_{a,x}g,g))) .$$
		We use the equality
		$$ T^{(2,1)}_{g,Q_xa} b = T^{(1,2)}_{Q_xa,g} b + T^{(2,1)}_{[Q_xa,g],g} b  = [Q_xa,Q_gb] - V_{g,b} T^{(1,2)}_{g,x} a, $$
		to rewrite the expression above as
		$$ - V_{g,b}T^{(1,2)}_{g,x} a + [Q_x a,Q_g b]  + \psi(V_{Q_xa,b}g,g) + V_{b,x}(- T^{(2,1)}_{g,x}a + \psi(V_{a,x}g,g))).$$
		So, to conclude the equality between that and Equation (\ref{eq1}), we must check
		$$ V_{x,b}T^{(2,1)}_{g,x} a -  \psi(V_{x,a}g, V_{x,b}g) - \psi(V_{Q_xa,b}g + V_{x,b}V_{x,a}g,g) + \psi(V_{Q_xa,b}g,g ) + V_{b,x}(T^{(2,1)}_{g,x}a + \psi(V_{a,x}g,g))) = 0,$$
		as it is the previous expression minus Expression (\ref{eq1}) in which we substituted the known expression for the $P$ and replaced all the $Q^{(1,1)}$'s with $V$'s. We see that this is the case, using the fact that $V_{x,b}$ acts as a derivation on $\psi$.
	\end{proof}                                 
\end{lemma}

\subsection{Computation of $T$}
\label{appendix T}
We work with the structurable algebra and corresponding vector group associated to a Hermitian cubic norm structure.
Set $g = ((a,j),(u,aj + j^\sharp)) \in G$.
First, we compute 
\begin{align*}
	Q_g(b,k)  = & ((a,j)(\bar{b},k))(a,j) - (u,aj + j^\sharp)(b,k) \\
	= & (a\bar{b} + T(j,k), ak + bj + j \times k)(a,j) \\ & - (ub + aT(j,k)+ T(j^\sharp,k), uk + a\bar{b}j + \bar{b}j^\sharp + \bar{a}\cdot j \times k + j^\sharp \times k) \\
	 = & (a^2\bar{b} - ub + aT(k,j) + bT(j,j) + T(j^\sharp,k),\\ & a\bar{a}k - uk + T(j,k)j + b\bar{a}j - \bar{a} \cdot j \times k + \bar{b} j^\sharp + (j \times k) \times j - j^\sharp \times k).	
\end{align*}
By linearising, we obtain that $l =  Q^{(1,1)}_{(a,j),(a,j)}(b,k)$ equals
\begin{align*}
	(& 2a^2\bar{b} - a\bar{a}b + 2aT(k,j) + bT(j,j) + T(j\times j,k), \\ & a\bar{a}k - T(j,j)k + 2T(j,k)j + 2b\bar{a}j - 2\bar{a}\cdot j\times k + 2\bar{b} j^\sharp + 2(j\times k)\times j - (j\times j)\times k).
\end{align*}
We want to compute $[l,(a,j)]$.
This is easily computed by computing $(t,\ldots) = l(\bar{a},j)$, and then using $[l,(a,j)] = (t - \bar{t},0)$.
We get that
\begin{align*}
	t = & \; 2a^2\bar{b}\bar{a} - ab\bar{a}^2 + 3a\bar{a}T(k,j) + b\bar{a}T(j,j) + \bar{a}T(j \times j,k) - T(j,j)T(k,j) + 2T(j,k)T(j,j)\\ & \;  + 2b\bar{a}T(j,j) + 2\bar{a}T(j \times k,j) + \bar{b}T(j\times j,j) + 2T((j \times k) \times j,j) - T((j\times j)\times k,j).
\end{align*}
Using that $$T(j \times k,j) = T(j \times j,k), \quad T((j\times k)\times j,j) = T(j,(j \times j) \times k), \quad 2 j^\sharp = j \times j, \quad 3N(j) = T(j,j^\sharp)$$ we obtain 
$$ t - \bar{t}= 3(s - \bar{s})$$
with
$$ s =  -ab\bar{a}^2 + a\bar{a}T(k,j) + \bar{a}bT(j,j) - aT(k,j \times j) - T(j,j)T(k,j) - 2 bN(j) + T(j\times(j\times j),k).$$
We can rewrite this $s$ as
$$ s= (a\bar{a} - T(j,j))(T(k,j) - b\bar{a}) - 2 \left( T(ak,j^\sharp) + bN(j) - T(j^\sharp \times k,j)\right).$$
Finally, we can compute $T$ if there is no $3$-torsion using the formula
$$ T_g(b,k) = [(a,j),Q_g(b,k)] + s - \bar{s},$$
proved in Lemma \ref{lemma constructing structurable algebras}, using that $3(\bar{s} - s)$ coincides with the right-hand side of the defining expression of $T$. To see that this $s$ coincides with that expression, observe that the mentioned expression in Lemma \ref{lemma constructing structurable algebras}, is obtained by computing $[(a,j),Q^{(1,1)}_{(a,j),(a,j)}(b,k)] = 3(\bar{s} - s)$.
Using the same technique as before, we compute that
$$ [Q_g(b,k),(a,j)] = v - \bar{v},$$
with
$$ v = b\bar{a}(2T(j,j) - u - a\bar{a}) + T(k,j)(2a\bar{a} - T(j,j) - u) - T(ak,j^\sharp) - 3\left(bN(j) - T(j^\sharp \times k,j)\right).$$
So, we conclude that
$$ T_g(b,k) = w - \bar{w},$$
with $w = s - v = (u - T(j,j))(b\bar{a}) + (u - a\bar{a})T(k,j) - T(ak,j^\sharp) + bN(j) - T(j^\sharp \times k,j)$. 