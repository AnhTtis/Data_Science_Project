\section{5-graded Lie algebras over rings $\Phi$ with $1/6$ and $1/30$}\label{se:lie}

We assume throughout this section that $1/6 \in \Phi$.
We use the definition of a Kantor pair $(P^+,P^-)$ with operators $V$ and associated Lie algebra as given by Allison and Faulkner \cite[section 3]{ALLFLK99}.



\begin{lemma}
	\label{lemma kantor pair implies pre kantor pair}
	Let $(P^-,P^+)$ be a Kantor pair with associated Lie algebra $$ L  = [P^-,P^-] \oplus  P^- \oplus (\Phi\zeta \oplus V_{P^+,P^-}) \oplus P^+ \oplus [P^+,P^+].$$
	Then the following hold
	\begin{itemize}
		\item $\exp(P^+ \oplus [P^+,P^+]) = G^+$ is a vector group\footnote{This is the usual exponential $\exp(l) = \sum_{i = 0}^4 (\text{ad} \; l)^i/(i!)$.}. This vector group can be coordinatized by $\exp(e) = 1 + e_1 + e_2 + e_3 + e_4 \mapsto (e_1,e_2) \in \End_\Phi(L)^2$ with $e_i$ the part which acts as $+i$ on the grading.
		The similarly defined $G^-$ is also a vector group.
		\item On these pairs $(e_1,e_2)$ the operators $Q^\text{grp},T,P$ are given by
		\begin{itemize}
			\item 
			$Q^{\text{grp}}_g h = (\nu_{2,1}(g,h), \nu_{4,2}(g,h)),$
			\item $T_g h = (0,\nu_{3,1}(g,h))$,
			\item $P_gh = \nu_{3,2}(g,h)$
		\end{itemize}
		and satisfy the definition of a pre-Kantor pair.
	\end{itemize}
	\begin{proof}
		The map  $\exp : P^+ \oplus [P^+,P^+] \longrightarrow \text{End}_\Phi(L)$ is injective since $1/6 \in \Phi$, as can be observed by evaluating the exponentials on the grading derivation $\zeta$. Mapping $\exp(g) = 1 + e_1 + e_2 + \ldots \mapsto (e_1,e_2)$ is injective, since it is injective on $\text{Lie}(G^+)$. 
		From
		$$\exp(a)\exp(b) = \exp(a+b+[a,b]/2),$$
		which holds by the Baker-Campbell-Hausdorff Theorem, we conclude that $P^+ \oplus [P^+,P^+]$ is a vector group with representation $\exp$. The description as pairs $(e_1,e_2)$ is exactly the image of $P^+ \oplus [P^+,P^+]$ under this representation.
		We choose to work with this second coordinatization since this allows us to define the operators more easily.
		
		We want to check that the mentioned operators $Q^{\text{grp}}, T$ and $P$ map to $G^+, G^+_2$ and $G^+/G^+_2$. If this is the case, then these operators automatically satisfy the part of Definition \ref{Definition pre-Kantor pair} that determines the linearisations of these operators by Lemma \ref{Lemma equations} and Lemma \ref{lemma linearizations mu}. For a more precise correspondence between those lemmas and the definition, see the proof of Theorem \ref{thm main 2}.
		
		
		In order to prove that the operators map to the right spaces, we assume $g = (a,b) \in G^+$ and $h = (c,d) \in G^-$.
		 We will use that $g_i [a,b] = \sum_{k + l = i}[g_k a,g_l b]$ for $i \le 4$, which must hold since we have $1/6 \in \Phi$.
		Note that this implies $\nu_{i,1}(g,h) = \mu_{i,1}(g,h) = \text{ad}(g_i \cdot h_p),$ where we use $h_p$ to denote the part in $P^-$ without the $\text{ad}$.
		In particular, this implies that $Q$ and $T$ already map to the right spaces.
		We note that $\mu_{1,1}(g,h)$ corresponds to the usual $V_{g,h}$ operator for Kantor pairs and that $\mu_{4,1} = 0$.
		
		For the $\nu_{i,2}$ we will use that $h_2$ is of the form
		$$ (h_1)^2/2 + \text{ad} \; v,$$
		for some $ v \in [P^-,P^-].$
		It is easy to check that 
		$$\mu_{(n,2)}(g,h) = \sum_{i + j = n} \mu_{(i,1)}(g,h)\mu_{(j,1)}(g,h)/2 + \text{ad} \; (g_n \cdot v),$$
		for $n \le 4$.
		Using this, we obtain that
		$$ \nu_{(4,2)}(g,h) =  \mu_{(2,1)}(g,h)^2/2 + \text{ad} \; (g_4 \cdot v) - \text{ad} (V_{g,h} T_g h)/2,$$
		which shows that $G^\text{grp}$ maps to $G^+$.
		Similarly, one proves that $P_g h$ is an inner derivation.
		
		Note that $1 + \epsilon V_{x,y} = 1 + \epsilon \text{ad} \; [x,y]$, which we let act using conjugation on the groups, is a vector group automorphism that interacts nicely with the defined operators.
		Lastly, observe that $G^+_2 = [G^+,G^+]$ and that $G^+_2$ acts faithfully on $P^-$ under the adjoint action, proving that it acts faithfully under $Q$ (since this coincides with adjoint action).
	\end{proof}
\end{lemma}

\begin{definition}
	We call the pre-Kantor pair associated to a Kantor pair, as constructed in the previous lemma, the \textit{associated pre-Kantor pair}.
\end{definition}

\begin{theorem}
	\label{theorem kantor implies operator kantor}
	Suppose that $1/30 \in \Phi$ and that $P = (P^+,P^-)$ is a Kantor pair over $\Phi$.
	The pre-Kantor pair associated to $P$ is an operator Kantor pair. 
	\begin{proof}
		Note that we are working with the usual exponentials and one checks that these are necessarily automorphisms using that $1/30 \in \Phi$.
		Note that 
		$$ \mu_{m,2}(x,y)= \nu_{m,2}(x,y) + \begin{cases}
			y_1\nu_{m,1}(x,y) & 1 \le m \le 2 \\
			y_1\nu_{3,1}(x,y) + \nu_{1,1}(x,y)\nu_{2,1}(x,y)&  m = 3 \\
			\nu_{1,1}(x,y)\nu_{3,1}(x,y) + \nu_{0,1}(x,y)\nu_{4,1}(x,y)& m = 4 \\
			\nu_{2,1}(x,y)\nu_{3,1}(x,y) + \nu_{0,1}(x,y)\nu_{5,1}(x,y) + \nu_{1,1}(x,y)\nu_{4,1}(x,y) & m = 5 \\
			0 & \text{otherwise}
		\end{cases} $$
		and that $\nu_{k,1}(x,y) = \text{ad} (x_k \cdot y_1)$, with $x_k$ the part of $\exp(x)$ that is $k$-graded. This immediately ensures that $\nu_{4,1} = \nu_{5,1} = 0$.
		We prove that the sufficient conditions listed in Remark \ref{remark sufficient condition} are satisfied, in order to prove that we have an operator Kantor pair.
		The first, second, and third condition follow immediately.
		
		Similar to Lemma \ref{lemma grouplike}, one can prove that $\exp(o_{i,j}(x,y))$ is an automorphism for all $x,y$, using that $\exp(x)$ and $\exp(y)$ are automorphisms.
		For the fourth, we only need to see that $\nu_{5,2}$ acts trivially on $P^-$, which is the case since it is a derivation $d$ (as it is the first not necessarily zero component of $o_{5,3}$) that acts as $+3 = 5 - 2$ on the grading.
		Namely, for $i$-graded $u$, we compute that
		$$  id(u) = d[\zeta,u] = [\zeta, d(u)] = (3 + i)d(u),$$
		so that $3d(u) = 0$.
		
		For $\nu_{6,3}(x,y)$, recall that 
		$$ \exp(o_{2,1}(x,y)) = \sum_{k = 1}^4 \nu_{2k,k}(x,y)$$
		is an automorphism. We know already that $\nu_{2,1}(x,y)$ and $\nu_{4,2}(x,y)$ correspond to the parts of $\exp(Q^\text{grp}_xy)$ which act as $+1$ and $+2$ on the grading.
		This proves that 
		$$ \nu_{6,3}(x,y) = \nu_{2,1}(x,y)\nu_{4,2}(x,y) - \nu_{2,1}(x,y)^3/3 + d$$
		with $d$ a derivation. Since there are no derivations which act as $+3$ on the grading, we conclude that $\nu_{6,3}(x,y)$ coincides with the part of $\exp(Q^\text{grp}_xy)$ which acts as $+3$ on the grading.
	\end{proof}
\end{theorem}


\begin{remark}
	We did only use $1/5 \in \Phi$ to prove that the exponentials are automorphisms of the Lie algebra.
\end{remark}

\begin{corollary}
	\label{bijection Kp preKP opKP}
	If $1/6 \in \Phi$, then the Kantor pairs and pre-Kantor pairs  stand in a bijective relation given by considering the associated pre-Kantor pair to a Kantor pair.
	If $1/30 \in \Phi$, then each pre-Kantor pair is an operator Kantor pair.
	\begin{proof}
		In the previous theorem and lemma, we established that each Kantor pair $P = (P^+,P^-)$ induces a pre-Kantor pair $G = (G^+,G^-)$ if $1/6 \in \Phi$. 
		So, it is sufficient to prove that $P \mapsto G$ establishes a bijection between Kantor pairs and pre-Kantor pairs if $1/6 \in \Phi$.
		
		We know that we can associate a Kantor pair $P'$ to each pre-Kantor pair $G$ by Lemma \ref{lemma preKantor induces Kantor pair}, from which we can uniquely recover the operators of $G$. Furthermore, we know that $P \cong P'$ since both pairs can be understood as $(G^+/G^+_2, G^-/G^-_2)$ with the induced action of $V$.
		
		If $1/30 \in \Phi$, Theorem \ref{theorem kantor implies operator kantor} proves that $P$ is also an operator Kantor pair.
	\end{proof}
\end{corollary}


\section{Structurable algebras}\label{se:struct}

We will construct operator Kantor systems corresponding to each of the classes of central simple structurable algebras as determined by Allison and Smirnov \cite{ALL78, SMI90Example,smirnov1990}, i.e., associative algebras, structurable algebras associated to a Hermitian form, tensor products of composition algebras, Smirnov algebras, skew dimension one structurable algebras, and Jordan algebras. Allison and Faulkner \cite{ALLFLK93} proved that these classes of algebras induce $5$-graded Lie algebras over arbitrary rings of scalars. The operator Kantor pairs will have roughly the same Lie algebras. There are certain small divergences, however. Firstly, for associative algebras and structurable algebras associated to Hermitian forms, it will prove useful to embed these in bigger structurable algebras of the same type if $1/2 \notin \Phi$. Secondly, we will not consider Smirnov algebras if $1/2 \notin \Phi$.

\subsection{(Quadratic) Jordan algebras}

Consider a quadratic Jordan algebra $(J,Q)$. Set $G = J \times 0$, $Q^\text{grp} = Q = (Q, 0)$, and $T = 0$. 
We also set $P_x y = Q_xQ_yx$. 

Now, we show that this forms an operator Kantor system.
First, observe that $(sx,ty)$ is quasi-invertible in $J[[s,t]]$ for $x,y \in J$ since the Bergmann operator $B(sx,ty)$ is invertible \cite[Proposition 3.2]{Loos75}, and that quasi inverses $(sx)^{ty}, (ty)^{sx}$ are determined by the symmetry principle \cite[Proposition 3.3]{Loos75}:
$$ (sx)^{ty} = sx + s^2Q_{x}(ty)^{sx}, \quad (ty)^{sx} = ty + t^2Q_{y}(sx)^{ty}.$$
Loos \cite[Theorem 1.4]{Loos95} proved that
$$ \exp(sx)\exp(ty) = \exp(ty)^{sx}\beta(sx,ty)\exp(sx)^{ty},$$
holds in the TKK representation.
We conclude that
$$ o_{2i+1,2i}(x,y) = \exp((Q_xQ_y)^ix), \quad o_{2i+2,2i+1}(x,y) = \exp((Q_xQ_y)^iQ_xy), \quad o_{i,j}(x,y) = 1 \text{ if $|i - j| > 1$}$$
hold in the TKK representation. Theorem \ref{thm main 2} shows that $(G,Q^{\text{grp}},T,P)$ forms an operator Kantor system.


\subsection{Associative algebras}

Let $A$ be an associative algebra with involution over $\Phi$.
If $1/2 \notin \Phi$, we set $B = A \otimes \Phi[s]/(s^2 - s)$ with involution determined by $a \otimes s \mapsto \bar{a} \otimes (1 -s) $. If $1/2 \in \Phi$ we also use $B$ to denote $A$.

The algebra $B$ induces an operator Kantor system of the form of Example \ref{example hermitian form}, using $(a,b) \mapsto a\bar{b}$ as the Hermitian form. 
The vector group is given by
$$ G_B = \{ (a,b) \in B \times B | b + \bar{b} = a\bar{a}\}$$
 and we obtain as operators
\begin{enumerate}
	\item $Q_{(a,b)}c = -a\bar{c}a + bc,$
	\item $T_{(a,b)}c = a\bar{c}\bar{b} - bc\bar{a},$
	\item $P_{(a,b)}(c,d) = a\bar{c}a\bar{c}a - a\bar{c}bc - bda,$
	\item $R_{(a,b)}(c,d) = bd\bar{b} + a\bar{c}b\bar{c}a - a\bar{c}a\bar{c}b$.
\end{enumerate}

The reason we chose to extend $A$ if $1/2 \notin \Phi$, is to guarantee that condition (\ref{Kantor pair equation}) of the definition of a pre-Kantor pair holds (or its equivalent formulation in Example \ref{example hermitian form}), i.e.,
$$ x - \bar{x} \in [G_B,G_B]$$
holds for all $x \in B$. This condition holds in the extension, since
$$ [(x, x\bar{x}s), (1,s)] = (0,x - \bar{x}),$$
for all $x \in B$.

\begin{remark}
	In order to obtain formulas compatible with the the operators of following sections, one must apply the automorphism $((a,b) \mapsto (-a,b), (c,d) \mapsto (-c,d))$ to this operator Kantor system.
\end{remark}

\subsection{Structurable algebras associated to Hermitian forms}\label{sss:herm}

Consider an associative algebra $C$ with involution $c \mapsto \bar{c}$ and a right $C$-module $M$ with (right-)Hermitian form $h :  M \times M \longrightarrow C,$ i.e., it is the same as a left-hermitian form except that $M$ is a right $C$-module and $h$ should satisfy $h(a,bc) = h(a,b)c$ for $c \in C, a,b \in M$.
Set $A = C \times M$ with operation
$$ (a,m)(b,n) = (ab + h(n,m), na + m\bar{b})$$
and involution
$$ \overline{(a,m)} = (\bar{a},m).$$
This algebra is isomorphic to the structurable algebra associated to a Hermitian form constructed by Allison \cite[8.iii]{ALL78}, by still using $h$ as the (now left) Hermitian form with $M$ considered as a left module under $a \cdot m = m\bar{a}.$ When we speak of $C$ as a left $A$ module, we consider $M$ with this second module structure (and $A$ with the obvious structure). The reason we work with the right Hermitian description, is that there appear less involutions in computations.


We shall consider
\[ G_{A} = \{ ((c,m),(d,mc)) \in A \times A| d + \bar{d} = c\bar{c} + h(m,m)\}.\]
We assume that $G_{A}$ satisfies condition (\ref{Kantor pair equation}), i.e., $x - \bar{x} \in [G_A,G_A]$ for all $x \in A$.
As for associative algebras, we can assume that this holds for $A$ or $A \otimes \Phi[s]/(s^2 - s)$ with $\bar{s} = 1 - s$.

We shall construct all the operators by making use of a representation. This approach also establishes that $A$ corresponds to a $3$-special Kantor pair in the sense of \cite[section 6]{ALLFLK99}.
Define \[h^\pm : A \times A \longrightarrow C : ((a,m),(b,n)) \mapsto a\bar{b} \pm h(m,n).\] These are (left-)Hermitian maps if we consider $A$ as a left $C$-module.
We let $M_2(C)$ act on $A^{2 \times 1}$ using
\[ \begin{pmatrix}
	a & b \\
	c & d
\end{pmatrix} \cdot \begin{pmatrix}
(e,f) \\ (g,h)
\end{pmatrix} = \begin{pmatrix}
(ae + bg, af - bh) \\
(ce + dg, -cf + dh)
\end{pmatrix},\]
i.e., we act normally on $C^{2 \times 1} \subset A^{2 \times 1}$ and as
\[ \begin{pmatrix}
	a & -b \\
	-c & d \\
\end{pmatrix}\]
on $M^{2 \times 1}$. We write $b \cdot_{\epsilon} (g,h)$ for $(bg, - bh)$.
So, 
\[ f: A^{2 \times 1} \otimes A^{1\times2} \longrightarrow M_2(C) : \begin{pmatrix}
	a \\ b
\end{pmatrix} \otimes \begin{pmatrix}
	c & d
\end{pmatrix} \longmapsto \begin{pmatrix}
	h^+(a,c) & h^-(a,d) \\
	h^-(b,c) & h^+(b,d) 
\end{pmatrix}\]
is an $M_2(C)$-bimodule map if we let $M_2(C)$ act on the right on $A^{1 \times 2}$ using $x \cdot M = (M^* \cdot x^*)^*$ with the $a^*$ the Hermitian transpose of $a$,
since
\[ h^-(c \cdot_\epsilon a,b) = c h^+(a,b), \quad ch^-(a,b) = h^+(c \cdot_\epsilon a,b),\]
for all $a,b \in A, c \in C$.
Set \[\mathcal{E} = \{ (a,b) \in \text{End}(A^{2\times1}) \times \text{End}(A^{1\times2}) |f(a(x),y) = f(x,b(y))\}.\]
Define
\[ g : A^{1\times2} \otimes_{M_2(C)} A^{2\times1} \longrightarrow \mathcal{E}\]
as
\[ g( x \otimes y)(z) = \begin{cases}
	xf(y,z) & z \in A^{1,2}\\
	f(z,x)y & z \in A^{2,1}
\end{cases}.\]
It is not hard to check that $g$ is a well defined $\mathcal{E}$-bimodule map. This proves that
\[ (M_2(C) \oplus \mathcal{E}) \oplus ( A^{2\times1} \oplus A^{1\times2}),\]
forms an associative algebra if all undefined multiplications are seen as $0$.
Using this definition, we can interpret elements as matrices in
\[ \begin{pmatrix}
	C & A & C \\
	A & \mathcal{E} & A \\
	C & A & C\\
\end{pmatrix},\]
with the obvious embeddings of subspaces.

We have two representations of the associated vector group, namely
\[ ((a,m),(u,ma)) \mapsto \begin{pmatrix}
	1 & (a,m) & -u + h(m,m) \\
	0 & 1 & (-a,-m) \\
	0 & 0 & 1\\
\end{pmatrix}\]
and 
\[ ((a,m),(u,ma)) \mapsto \begin{pmatrix}
	1 & 0 &  0 \\
	(-a,-m) & 1 & 0 \\
	-u + h(m,m) & (a,m) & 1\\
\end{pmatrix}. \]

Given an element $g = ((a,m),(u,ma))$, we write $\tilde{u}$ to represent $u - h(m,m)$ in operators involving $g$.
This representation yields us operators
\begin{enumerate}
	\item $Q_{(x,y)} b= h^+(x,b)\cdot x - \tilde{y} \cdot_\epsilon b,$
	\item $T_{(x,y)} b =  h^-( \tilde{y} \cdot_\epsilon b,x) -h^-( x,\tilde{y} \cdot_\epsilon b),$
	\item $P_{(x,y)} (u,v) = h^+(x,u) \cdot Q_{(x,y)} u - \tilde{y}\tilde{v} \cdot (a,m),$
	\item $\tilde{R}_{(x,y)} (u,v) = \tilde{u}\tilde{v}\overline{\tilde{u}} - h^+(x,b)T_{(x,y)}b,$
\end{enumerate}
with $(Q_xy , \tilde{R}_{x}y)$ the image of $Q^\text{grp}_xy$ under $(a,b) \mapsto (a, \tilde{b})$.
So, we see that these operators form an operator Kantor pair on the structurable algebra $A$, using Theorem \ref{thm main 2}.
It is not hard to compute that
\[ (x\bar{y})x - uy = Q_{(x,u)} y\]
holds for $(x,u) \in G_A$ and $y \in A$, so that
\[ (x\bar{y})z + (z\bar{y})x - (z\bar{x})y = Q^{(1,1)}_{z,x}y.\]
Hence, the constructed operator Kantor pair corresponds precisely to the structurable algebra.


We remark that this construction works more generally for $M_1 \times M_2$ with $M_i$ left $C$-modules with Hermitian forms $h_i$.
This corresponds to the Kantor triple system
\[ V_{x,y}(z) = h^+(x,y) \cdot z + h^+(z,y) \cdot x - h^-(z,x) \cdot_\epsilon y,\]
with $h^\pm = h_1 \pm h_2$. Using as underlying vector group
\[ G = \{(a,b) \in (M_1 \times M_2) \times C | b + \bar{b} = h^-(a,a), b - \bar{b} \in \langle h^-(x,y) - h^-(y,x) | x,y \in M_1 \times M_2 \rangle \}\]
contained in $(M_1 \times M_2) \times C$ with $\psi = h^-$, we get an operator Kantor pair if $1/2 \in \Phi$. If $1/2 \notin \Phi$ we need to make the extra assumption that \[ \langle h^-(x,y) - h^-(y,x) | x,y \in M_1 \times M_2 \rangle = [G,G]. \]

%We construct those operators for $G_{A}$ using a free model of $A$.
%Specifically, suppose that $C$ is generated by $S_C$ as a $\mathbb{Z}$-module and that $M$ is generated by $S_M$ over $\mathbb{Z}$.
%Consider the free unital associative algebra $F_C$ over $\mathbb{Z}$ generated by 
%$$ S_C \cup \{ h_s | s \in S_C \} \cup \{t\} \cup \{ h(m,n) | m,n \in S_M\}$$
%with $t$ central.
%We endow $F_C$ with the unique involution defined by
%\begin{enumerate}
%	\item $\bar{s} + s = h_s$ for $s \in S_C$,
%	\item $\bar{t}+ t = 1$,
%	\item $\overline{h(m,n)} = h(n,m)$.
%\end{enumerate}
%We set $F_{M}$ to be the free $F_C$-module over $S_M$.
%The map $h : S_M \times S_M \longrightarrow S_C$ induces a hermitian form $F_M \times F_M \longrightarrow F_C$.
%We want to prove that $A_{F} = F_C \times F_M$ induces an operator Kantor system over $\mathbb{Z}$.
%
%\begin{lemma}
%	The algebra $A_{F}\otimes \mathbb{Q}$ induces a unique operator Kantor system $G_{A_{F_{C}}}$ with
%	$$ Q_{(a,b)}c = (a\bar{c})a - bc.$$
%	Furthermore, all operations can be expressed in terms of multiplications and involutions on $F_C$, hermitian forms on $F_M$, and module actions of $F_C$ on $F_M$.
%	\begin{proof}		
%		Note that $G_{A_{F_C}}(\mathbb{Q})$ forms an operator Kantor system.
%		This is the case since each structurable algebra, and thus in particular $(A_F \otimes \mathbb{Q}),$ over a field of characteristic different from $2,3$ induces a Kantor triple system with operator $V_{x,y} z = - Q^{(1,1)}_{z,x} y$ and since each Kantor triple system over $\mathbb{Q}$ induces an operator Kantor system.
%		
%		Remark that for each $(a,m) \in A_F$, we have
%		$$ ((a,m),(t(a\bar{a} + h(m,m)),ma)) \in G_{A_F}$$
%		so that $$\text{Lie}(G_{A_F}) = A_F \oplus \{(x,0) \in A_F | x + \bar{x} = 0\}.$$
%		So, we can apply Lemma \ref{lemma constructing structurable algebras}. Hence it is sufficient to prove that $T$ restricts to $$G_{A_F} \times A_F \longrightarrow \{x - \bar{x} | x \in A_F\}.$$
%		We prove that this is the case.
%		Using a direct computation one can show that
%	$$2((g\bar{h})g)\bar{g} - 2g(\bar{g}(h\bar{g})) - ((g\bar{g})h)\bar{g} + g(\bar{h}(g\bar{g}))$$
%	is contained in $3 A_F$ for $g,h \in A_F$, which proves that $T$ restricts as required by Lemma \ref{lemma constructing structurable algebras}.
%	Specifically, if $g = (a,u)$ and $h = (b,v)$, one computes\footnote{For a way to compute this, consider the similar computation in Appendix \ref{appendix T} for a different kind of structurable algebra. That computation yields a very similar expression, $s$ in that section, if one sets $j \times k$, $j^\sharp$, and $N(j)$ equal to $0$ for all $j, k$. In fact, the computation of that appendix can be translated to this context, as long as one is careful with the order of the multiplication in $C$ (which is unnecessary, and thus has not been done, in the context of the appendix, but necessary here) and uses that $T$ is left-hermitian instead of right hermitian. More precisely, setting $h(x,y) = T(y,x)$ and applying the induced sign changes yields the formulas from here. } that
%	$$2((g\bar{h})g)\bar{g} - 2g(\bar{g}(h\bar{g})) - ((g\bar{g})h)\bar{g} + g(\bar{h}(g\bar{g})) = 3[(a\bar{b} + h(u,v))(a\bar{a} - h(u,u))),1],$$
%	where we use $[x,1] = x - \bar{x}$.
%	\end{proof}
%\end{lemma}
%
%The exact formula for $T$ is given by
%$$ T_{((a,u),(c,ua))}(b,v) = [(a\bar{b} + h(u,v))(a\bar{a} - c),1] = [(h(u,u)- c)(b\bar{a} + h(v,u)),1].$$
%
%
%\begin{corollary}
%	For a structurable algebra $A$ associated to a hermitian form over $\Phi$ containing $1/2$, we have an operator Kantor system $G_{A}$.
%	If $1/2 \notin \Phi$, then $G_{A \otimes \Phi[s]/(s^2 - s)}$ forms an operator Kantor system over $\Phi$ with Lie algebra 
%	$$ (A \otimes \Phi[s]/(s^2 - s)) \oplus \{ (x - \bar{x}, 0) | (x,0) \in A \otimes \Phi[s]/(s^2 - s) \}.$$
%	\begin{proof}
%		Suppose that there exists central $t \in A$ such that $t + \bar{t} = 1$.
%		Then there exists a map $A_F \otimes \Phi \longrightarrow A$ preserving the multiplication, hermitian form, module action and involution, proving that $G_A$ is an operator Kantor system formed by taking the quotient of $G_{A_F}(\Phi)$.
%		The lemma follows if $1/2 \in \Phi$ by putting $t = 1/2$.
%		
%		If $1/2 \notin \Phi$, then we can certainly extend $A$ with an idempotent $s$ and involution determined by $\bar{s} = 1 -s $. This extension is a quotient of $A_{F} \otimes \Phi$.	
%	\end{proof}
%\end{corollary}

\subsection{Constructing operator structurable algebras using $\mathbb{Z}$-substructures}

We want to construct subsystems over $\mathbb{Z}$ from some operator Kantor systems over $\mathbb{Q}$. We want to do this specifically for systems associated to structurable algebras. In this setting, the associated operator Kantor system over $\mathbb{Q}$ means the operator Kantor pair associated to the $5$-graded Lie algebra \cite[section 3]{ALL79} corresponding to the structurable algebra over $\mathbb{Q}$.

The advantage of these $\mathbb{Z}$-subsystems is that we do not need to check the axioms of operator Kantor pairs, but only need to check that the operators are internal.
Remark \ref{Remark uniqueness P and R} shows that if $P$ and $R$ exist for an operator Kantor pair, then they are unique. In this section, we start by first constructing $P$ and $R$ as homogeneous maps from $Q, T$ and $V$. After this construction, it becomes sufficient to construct the primary operators $Q$ and $T$ to construct all operators. 
Later we will construct $\mathbb{Z}$-analogues for different classes of structurable algebras. With those analogues we will be able to construct operator Kantor pairs for all classes of central simple structurable algebras (except Smirnov algebras if $1/2 \notin \Phi$).


We consider an algebra $C$ with involution $a \mapsto \bar{a}$.
Consider $\psi : C \times C \longrightarrow C, (a,b) \mapsto a\bar{b}$. This endows 
$$ G_C = \{ (a,b) \in C \times C | b + \bar{b} = a\bar{a}\},$$
with a vector group structure.
We will be interested in defining an operator Kantor pair structure with operator
\[ Q_{(a,b)}(c,d) = (a\bar{c})a - bc.\]
We assume that 
$$ \text{Lie}(G_C) = C \times (G_C)_2,$$
i.e., for each $a \in C$ there exists $b$ such that $b + \bar{b} = a\bar{a}$. The $(G_C)_2$ appearing in this assumption does not impose anything as proved in Lemma \ref{lemma Lie G}. We set $S = (G_C)_2$.
Note that $x - \bar{x} \in S$ for all $x \in C$ and that $2S = \{ s - \bar{s} | s \in S\}$ since $s = - \bar{s}$ for all $s \in S$.
So, the assumption that $ \text{Lie}(G_C) = C \times (G_C)_2$ guarantees that conditions (\ref{Kantor pair equation}),(\ref{Kantor pair equation 2}) for pre-Kantor pairs always hold. This assumption is only important if $1/2 \notin \Phi$ and will, for us, only play a role in structurable algebras that can have an arbitrary associative part.

We want to prove that there exists at most one operator Kantor system with
$$ Q_{(a,b)}(c,d) = (a\bar{c})a - bc,$$
and a given operator $T : G_C \times C \longrightarrow \{ x - \bar{x} | x \in C \} = \{ [a,b] = a\bar{b}- b \bar{a} | a,b \in C\}$ of bidegree $[3,1]$. We remark that we will use $V$ and $\tau$ without making reference to the operator Kantor pair structure on which they are defined. In that case, we use operators as
$$V_{x,y} g = ( - Q^{(1,1)}_{g,x} y, - T^{(2,1)}_{g,x} y - \psi(Q^{(1,1)}_{g,x}y,g))$$ and $$\tau_{y,x} g = (P^{(1,2)}_{g,x^{-1}}y^{-1}, R^{(2,2)}_{g,x^{-1}}y^{-1} + \psi(P^{(2,1)}_{g,x^{-1}}y, g) - \psi(Q_gy,Q_{x^{-1}}y)),$$
i.e, we identify the operator Kantor $V$ with  its restriction to $C^+ \times C^- \longrightarrow \text{Nat}(G^+)$ and $\tau$ to its restriction $G^- \times G^+ \longrightarrow \text{Nat}(G^+)$. There is an incongruity in the order of arguments with the purpose of making $V$ correspond to the usual $V$ for structurable algebras, while still taking the version of $\tau$ that is not only the most easy to define, but also the only one which we will need in this section.

\begin{lemma}
	\label{lemma constructing structurable algebras}
	Let $C$ be a unital algebra with involution and consider a vector group $G \le G_C$ with $\Lie(G) = C \times S$ for some $S$.
	Define $Q_{(a,u)}b = (a\bar{b})a - ub$. Suppose that there exist operators $T,P,R$ such that $(G,Q^\text{grp},T,P)$ forms an operator Kantor system. Use $(V_{x,y})_i$ to denote the action of $V$ on the $i$-graded parts of $\text{Lie}(G_C)$. The operators $P,R$ are uniquely determined by $Q,T$, as indicated by the following formulas, in which $x = (a,b)$, and $g, h \in G$:
	\begin{enumerate}
		\item $P_g (x(-x))= T_g(b) - (T_g1)b - (V_{g,1})_1Q_g b + (V_{g,b})_1 Q_g 1,$
		\item $(\tau_{h,g(-g)} g)_1 = - Q_{g(-g)}Q_h g + V_{g,h} Q_{g(-g)}h ,$
		\item $ P_{g} h = - P_{g} (h(-h)) - Q_{g}Q_hg - \tau_{h,g(-g)}g + 2 Q_{T_{g}h}h - (V_{g,h})_1Q_{g}h,$
		\item $(\tau_{h,T_{g^{-1}}1} g)_1 = [Q_{T_{g^{-1}}1},Q_{h^{-1}}g] - Q_{[g,Q_{T_{g^{-1}}1}h]}h,$
		\item $ R_g h = (Q_gh)^2 - P_g(T^{(2,1)}_{h,1}g) - Q_gQ_hQ_{g^{-1}} 1 + (\tau_{hT_{g^{-1}}1} g)_1 + (V_{g,h})_1 T_gh.$
	\end{enumerate}
	 Moreover, if there is no $3$ torsion, then $T$ is uniquely determined by
	 $$ 3(T_g h - [g,Q_g h]) = 2((g\bar{h})g)\bar{g} - 2g(\bar{g}(h\bar{g})) - ((g\bar{g})h)\bar{g} + g(\bar{h}(g\bar{g})).$$
	 \begin{remark}
	 	Substituting equations $1$ and $2$ in equation $3$ of the lemma yields that $P_g(a,b)$ must equal to
	 	\[- T_g(b) + (T_{g}1)b + (V_{g,1})Q_gb - (V_{g,b})Q_g1 - Q_gQ_{(a,b)}g - Q_{g(-g)}Q_{(- a,\bar{b})}g + V_{g,a}Q_{g(-g)}a + 2Q_{T_ga}a - (V_{g,a})_1Q_ga, \] 
	 	which is a homogeneous map of bidegree $[3,2]$ in $g$ and $(a,b)$.
	 	Similarly, substituting equation $4$ in equation $5$ yields us a defintion of $R_g h$ as a homogeneous map of bidegree $[4,2]$ in terms of $V$, $Q$, $T$ and $P$.
	 \end{remark}
	\begin{proof}
		
		We work with the Lie algebra $L$ associated to an operator Kantor pair, and prove that these equalities hold.
		
		We know that $x(-x) = (0,2b - a\bar{a}) = (0,b - \bar{b}) = [b,1]$ for $x = (a,b)$.
		The first equation is obtained by evaluating
		$$ \text{ad} \; g_3 ([b,1]) = \nu_{3,2}(g,x(-x)),$$
		on the grading element $\zeta$ of $L$. Namely,
		using that $\exp(g)$ is necessarily an automorphism so that
		\[\text{ad} \; g_3 ([b,1]) \cdot \zeta = ([T_gb,1] + [b,T_g1] + [Q_gb,V_{g,1}] + [V_{g,b},Q_g1]) \cdot \zeta = [T_gb,1] + [b,T_g1] + [Q_gb,V_{g,1}] + [V_{g,b},Q_g1],\]
		and that
		\[ \nu_{3,2}(x,y) = \text{ad} \; P_x y,\]
		we obtain the first equation.
		The second equation is obtained from $$(\tau_{a,b}c)_1 = - P^{(1,2)}_{c,b} a^{-1} =- P^{(2,1)}_{b,c} a^{-1} = - Q_{b} Q_a c + V_{c,a} Q_{b} a$$
		which holds if $b \in G_2$,
		by first applying the definition of $\tau$, using that $f^{(2,1)}_{b,a} = f^{(1,2)}_{a,b}$ if $b \in G_2$ for all $f$ homogeneous of degree $3$, and the expression of $P^{(2,1)}$ for pre-Kantor pairs.
		
		The third equation follows from evaluating 
		$$ \nu_{3,2}(g,h) = \mu_{3,2}(g,h) - h_1\mu_{3,1}(g,h) - \mu_{1,1}(g,h)\mu_{2,1}(g,h)$$
		on the grading element $\zeta$.
		The last equation follows from evaluating 
		$$ \nu_{4,2}(g,h) = \mu_{4,2}(g,h) - \mu_{1,1}(g,h)\mu_{3,1}(g,h),$$
		on the $1$ contained in the $-1$-graded copy of $C$ in $L$.
		Namely, we know that
		\[ (\nu_{4,2}(g,h) + \mu_{1,1}(g,h)\mu_{3,1}(g,h)) \cdot 1 = (Q_gh)^2 - R_gh + (V_{g,h})_2 T_g h,\]
		while we also know that
		\[ (g_3 h_2 g_1^{-1} + g_2 h_2 g_2^{-1} + g_1 h_2 g_3^{-1}) \cdot 1 =  P_g\left(T^{2,1}_{(h,1)} g\right) + Q_gQ_h Q_{g^{-1}} 1 - \tau_{h,T_{g^{-1}}1} g\]
		since
		$$ g_3 h_2 g_1^{-1} \cdot 1 = g_3 ( h_2 \cdot  V_{1,g}) = P_g(T^{(2,1)}_{(h,1)}g).$$
		The fourth equation follows from evaluating the equation
		$$ (g_1h_2g^{-1}_3 + h^{-1}_2g_1g^{-1}_3 - h_1g_1h_1g^{-1}_3 - (Q_{h^{-1}}g)_1 g^{-1}_3) = 0,$$
		on the $-1$-graded $1$. This last equation holds since
		$$ \mu_{2,1}(h^{-1},g) = (Q_{h^{-1}}g)_1.$$
		
		Note that equations $1,2$ and $4$ uniquely determine their left-hand sides, since these only use $Q,T$ and $V$ (with $V$ itself also being defined in terms of $Q$ and $T$).
		Now, we see that the third equation, uniquely determines $P_g h$.
		Now that $P$ is defined, we see that the last equation determines $R_gh$ uniquely as well.	
		
		We use Equation (\ref{equation homogeneous of degree 3}) to obtain
		$$ 3T_g - 3T^{(1,2)}_{g,g} = - T^{(1,1,1)}_{g,g,g},$$
	as $T$ must be homogeneous of degree $3$ in $g$. Using that $T^{(1,2)}_{g,g}h = [g,Q_gh]$ and evaluating the $(1,1)$-linearisation $T^{(1,1,1)}_{g,g,g}h$ yields the expression for $T$ given in the statement of this lemma, since 
		\[ 	- T^{(1,1,1)}_{g,g,g} h = - [g,Q^{(1,1)}_{g,g} h] = - [g,2(g\bar{h})g - (g\bar{g})h] = - g\overline{(2(g\bar{h})g - (g\bar{g})h)} + (2(g\bar{h})g - (g\bar{g})h	)\bar{g}. \qedhere \]
	
	\end{proof}
\end{lemma}

\begin{remark}
	The previous lemma is not that useful if $1/2 \in \Phi$, since the operators are then uniquely defined by
	\[ 2 P_g h = P^{(3,(1,1))}_g (h,h) + P_g^{(3,(1,1))}(h_2,1) - P_g^{(3,(1,1))}(1,h_2),\]
	and
	\[ 2 R_g h = R^{(4,(1,1))}_g (h,h) + R_g^{(4,(1,1))}(h_2,1) - R_g^{(4,(1,1))}(1,h_2),\]
	using the known linearisations of these operators.
\end{remark}




\subsubsection{Tensor product composition algebras}

Let $O_1$, $O_2$ be two composition algebras with norms $N_1,N_2$.
Consider $A = O_1 \otimes O_2$ with components-wise involution and suppose that there exists $t_i \in O_i$ such that $t_i + \bar{t}_i = 1$.
Note that if $a = \sum_{i = 1}^n \lambda_i a_i \otimes b_i,$ corresponding to a basis formed by pure tensors $\{a_i \otimes b_i | i \in \{1,\ldots,n\}\}$ of $A$, we can define
$$q(a) = \sum_{i = 1}^n \lambda_i^2 N_1(a_i)N_2(b_i) t_1 \otimes 1 + \sum_{i < j} \lambda_i\lambda_j a_i\bar{a_j} \otimes b_i\bar{b_j},$$ so that
$$ q(a) + \overline{q(a)} = a\bar{a}.$$
This map $q$ is clearly quadratic.
Set
$$ G_A = \{ (a,s + q(a)) \in A \times A | s \in (1 \otimes S_{O_2} + S_{O_1} \otimes 1) \text{ or } \; \{ x - \bar{x} | x \in O_1 \otimes O_2 \} \},$$
with $S_{O_i} = \{ x \in O_i | x \perp 1,t_i\}$. This means that $G_2$ contains all $x - \bar{x}$ and all $1 \otimes s_1 + s_2 \otimes 1$ with $s_i \perp 1,t_i$, i.e., all elements that could reasonably be called skew.


We want to prove that there exists an operator Kantor system associated with $G_A$ and 
$$ Q_{(a,b)} h = (a\bar{h})a - bh.$$

We do this by considering universal variants of $O_1,O_2$ and we will still refer to these variants as $O_1,O_2$.
Consider $R = \mathbb{Z}[a,b,c,d,e,f]$.
Set $O_1 = CD(R[t_1]/(t_1^2 - t_1 + a),b,c)$ with which we mean the algebra obtained by applying the Cayley-Dickson process $2$ times using the parameters $b,c$ starting from $R[t]/(t^2 - t +a)$.
Similarly, we set $O_2 = CD(R[t_2]/(t_2^2 - t_2 + d),e,f)$.
Note that $O_i$ is a subalgebra of an octonion algebra $P_i$ over $\mathbb{Q}(a,b,c,d,e,f)$.


We want to prove that $A = O_1 \otimes O_2 \le P_1 \otimes P_2$ induces an operator Kantor system.
We first check whether $T$ restricts to $A$.
We know that $T_gh \in A$ for $h \in A$ and $g = (o_1 \otimes o_2, \ldots)$ by using the alternativity of the multiplication for octonion algebras and the expression for $T$ proved in Lemma \ref{lemma constructing structurable algebras}. More precisely, for these elements
we can compute that $$ T_{(a \otimes b, u)} c = [a \otimes b, - uc ] = - (a \otimes b)(\bar{c}\bar{u}) + (uc)(\bar{a} \otimes \bar{b}).$$
Moreover, the linearisations of $T$ restrict nicely to $A$ since they satisfy $T^{(1,2)}_{x,g} y = [x,Q_gy]$ and $T^{(2,1)}_{x,g} = T^{(1,2)}_{g,x} + T^{(1,2)}_{x,[x,g]}$.
Hence $T$ maps $$G_A \times G_A \longrightarrow [G_A,G_A] \subset (G_A)_2.$$
So, the operators $Q,P$ restrict to maps from $G_A \times G_A$ to $A$.
Observe that
$$ \langle (t - \bar{t}) \otimes 1, t \otimes t - \bar{t} \otimes \bar{t} \rangle_{\mathbb{Z}} \subset A \otimes \mathbb{Q} = \langle (t - \bar{t}) \otimes 1, 1/2 ((t - \bar{t}) \otimes 1 + 1 \otimes (t - \bar{t})) \rangle_{\mathbb{Z}}.$$
We also note that $(1 \otimes S_2 + S_1 \otimes 1) \otimes \mathbb{Q} \cap A = (1 \otimes S_2 + S_1 \otimes 1)$. Using that $$ x - \bar{x} \in (1 \otimes S_2 \oplus S_1 \otimes 1 \oplus (1 - 2t_1)\Phi \otimes 1 + 1 \otimes \Phi(1 - 2t_2)) \otimes \mathbb{Q}$$ for $x \in A \otimes \mathbb{Q}$ since $1 - 2t_i = \bar{t}_i - t_i$, and that $A \ni R_gh - q(Q_gh) \in \langle x - \bar{x} | x \in A \otimes \mathbb{Q} \rangle$, we conclude that $Q^\text{grp}$ restricts to $G_A$.
This proves that all operators restrict nicely to $G_A$.

Now, consider $G_A(g)(G_A(k[a,b,c,d,e,f]))$ for $g: k[a,b,c,d,e,f] \longrightarrow k$ corresponding to actually considering octonion algebras $O_1,O_2$ obtained by a Cayley Dickson process over a field $k$. This induces an operator Kantor pair structure associated to the tensor product of octonion algebras.
For general composition algebras, we only need to consider subsystems of the just constructed operator Kantor pair.

%In general it is not that easy to compute what the operators $P$ and $R$ are.
%However, for relatively easy group elements, the operators can often be computed using the formulas for associative algebras, e.g.,
%$$ P_{(a \otimes b, 0)}(x \otimes y,0 ) = a\bar{x}a\bar{x}a \otimes b\bar{y}b\bar{y}b,$$
%since each subalgebra generated by 2 elements of a composition algebra is associative,
%we can use that $P^{(2,1)}_{g,a} h = Q_gQ_{h^{-1}} a - V_{a,h} Q_g h,$ to obtain
%$$ P_{(0,s) \cdot (a \otimes b, 0)}(x \otimes y,0 ) = a\bar{x}a\bar{x}a \otimes b\bar{y}b\bar{y}b + s(x\bar{a}x \otimes y\bar{b}y) - (a\bar{x} \otimes b\bar{y})(s(x \otimes y)) - ((s(x \otimes y))\bar{a} \otimes \bar{b})(x \otimes y).$$

\subsubsection{Smirnov algebras}

Allison and Faulkner generalized the Smirnov algebra already to arbitrary rings \cite[example 6.7]{ALLFLK93} combined with \cite{ALLFLK93_2} in a setting suitable for the study of Lie algebras. We show that if $1/2 \in \Phi$, then there is a operator Kantor system that yields the same Lie algebra.


Consider an octonion algebra $O$ with norm $N$. The Smirnov algebra is a quotient of the subalgebra $B$ of $O \otimes O$ generated by $o \otimes o$ for $o \in O$ by an ideal $I$ of Hermitian elements.
The ideal $I$ corresponds to the scalar multiplications on $O$ under the embedding $$B \longrightarrow \text{End}(O) : c \otimes c \mapsto (x \mapsto N(c,x)c).$$ If $O$ is split with basis $\{e_1,e_2,e_3,e_4,e^*_1,e^*_2,e^*_3,e^*_4\}$ such that $N(e_i,e_j) = N(e^*_i,e^*_j) = 0$ for all $i,j$ and $N(e_i,e^*_j) = \delta_{ij}$, then $I$ is linearly generated by
$$ \sum_{i = 1}^4 e_i \otimes e^*_i + e^*_i \otimes e_i,$$

We set $A = B/I$.
We can take the fixed elements under the exchange involution of the operator Kantor system associated to the tensor product of composition algebras.
This yields an operator Kantor system associated to $B$ if $1/2 \in \Phi$. 
It is not hard to check that all operators are compatible with the quotient with respect to $I$, as they are formed by multiplications, involutions and the operator $T$.
For the operator $T$ this follows from the formula for $T$ and the linearizations of $T$, if there is no $3$-torsion. It also holds in general, as can be proved using the model of $O \otimes O$, now over $\mathbb{Z}[1/2]$, from the previous subsection.

\begin{remark}
	If $1/2 \notin \Phi$, it is better let the unit go. Let $O$ be the split octonion algebra over $\mathbb{Z}$. If necessary we represent elements as Zorn matrices.	
	 Consider \[A = \{ x \in \langle o \otimes o | o \in \mathbb{O} \rangle : n(x) = 0\},\] with $n(o \otimes o) = o\bar{o} = N(o)$, which is a nonunital algebra. 
	Let $G_A$ be the subgroup of $A \times A$ generated by the set $\{ (o \otimes o, 0) | N(o) = 0\}$,
	and elements $ (\tilde{t} - \tilde{e}_i,\tilde{t} - \tilde{e}_i),$
	with $\tilde{t} = t \otimes \bar{t} + \bar{t} \otimes t$, $\tilde{e}_i = e_i \otimes e_i^T + e_i^T\otimes e_i$ for a standard generator $e_i \in \mathbb{Z}^3$ using
	\[ t = \begin{pmatrix}
		1 & 0 \\
		0 & 0
	\end{pmatrix}, \quad e_i = \begin{pmatrix}
	0 & e_i \\
	0 & 0 \\
\end{pmatrix}.\]
It is not hard to check that $\text{Lie}((G_{A})_\Phi) = (A \oplus \{ x - \bar{x} | x \in A\}) \otimes \Phi.$ using that $A$ is the direct sum of the rank $32$ subspace $\langle o \otimes o| N(o) = 0\rangle$ and rank $3$ subspace $\langle \tilde{t} - \tilde{v} | v = e_1,e_2,e_3 \rangle$, proving that $G_A$ is a vector group.
By embedding $G_A \subset G_{O \otimes O}$ we inherit the operators $Q,T,P$ and $R$ mapping $G_A \times G_A \longrightarrow G_{O \otimes O}$.
On $G_A(\mathbb{Q})$ the operators are $\mathbb{Q}$-polynomial expressions of $A$-elements, proving that the operators map to $G_A$ instead of $G_{O \otimes O}$.
This construction differs form $B/I$ since
\[ A \otimes \mathbb{F}_2 \ni  \tilde{t} + \tilde{e}_1 + \tilde{e}_2 + \tilde{e}_3 \in I \otimes \mathbb{F}_2. \]
\end{remark}


\subsubsection{Structurable algebras of skew-dimension $1$}

Consider a Hermitian cubic norm structure $(J,N,T,\sharp,\times,e)$, as described for example by the second author \cite[Definition 4.1]{DeMedts2019}\footnote{ We will not explicitly use the additional axioms of \cite[Remark 4.4]{DeMedts2019} that are required for fields of characteristic $2$ or $3$. These will still implicitly play a role, however, in constraining the classes of hermitian cubic norm structures that which will consider. }, over a quadratic étale extension $A = E = \Phi[t]/(t^2 - t + \alpha)$ of $\Phi$, for some $\alpha \in \Phi$, and associated algebra
$ E \times J$ with operation $$(a,b)(c,d) = (ac + T(b,d), ad + \bar{c}b + b \times d)$$
and involution
$$ (a,c) \mapsto (\bar{a},c).$$
Over fields of characteristic different from $2$ and $3$ each structurable algebra of skew dimension $1$, i.e., a structurable algebra such that the space $\{ x | x + \bar{x} = 0\}$ is $1$-dimensional, can be described in such a way as proved in \cite[Theorem 5.14]{DeMedts2019}\footnote{	This theorem says that specific forms of matrix structurable algebra can be described as a Hermitian cubic norm structure. These specific forms are all skew-dimension $1$-structurable algebras \cite[Theorem 1.13]{ALL90}}. Recall that an arbitrary cubic norm structure $(J,N,T,\sharp,\times,e)$ over $F$ induces a hermitian cubic norm structure $(J \oplus J, N', T', \sharp', \times', e')$ uniquely determined by $N '(a,b) = (N(a),N(b)), T'((a,b),(c,d))= (T(a,d),T(b,c)), (a,b)^{\sharp'} = (b^\sharp,a^\sharp)$ \cite[Example 3.10]{DeMedts2019}. We call this Hermitian cubic norm structure a \textit{split Hermitian cubic norm structure}.


We consider the group
$$ G_A = \{ ((a,b),((u,b^\sharp + ab)) \in (E \times J)^2| u + \bar{u} = a\bar{a} + T(b,b)\}.$$
Remark that $u,v$ satisfy $u + \bar{u} = v + \bar{v} = a\bar{a} + T(b,b)$ if and only if $(u - v) + \overline{(u - v)} = 0$. If we work over a field $\Phi$ the space of $x \in E$ such that $x + \bar{x} = 0$ coincides with the span of all $y - \bar{y}$ since both spaces are $1$ dimensional and contain $t - \bar{t} \neq 0$. 
One can compute, see Appendix \ref{appendix T}, that
$$ T_{((a,j),(u,j^\sharp + aj))} (b,k) = [(u - T(j,j))(b\bar{a}) + (u - a\bar{a})T(k,j) - T(ak,j^\sharp) + bN(j) - T(j^\sharp \times k,j),1],$$
using $[a,1] = a - \bar{a}$,
if one works over a ring $\Phi$ without $3$-torsion.

\begin{lemma}
	Suppose that $A$ is an algebra over a field $\Phi$ for which there exist a quadratic extension $K/\Phi$ such that (1) $A \otimes K$ forms a matrix structurable algebra, i.e., $A \otimes K$ can be constructed from a split Hermitian cubic norm structure, (2) there exists $t \in A$ such that $t + \bar{t} = 1$ and $t - \bar{t} = 1 - 2t \neq 0$, and (3) there exists a vector group $G_A$ such that $(G_A)_K \cong G_{A \otimes K}$. Then $A$ is an algebra constructed from a Hermitian cubic norm structure.
	\begin{proof}
		This way proved under less strict conditions if $1/6 \in \Phi$ \cite[Theorem 5.14]{DeMedts2019}.
		We work slightly differently. We will try to reconstruct all operators uniquely from the assumptions we made in the lemma. All axioms for Hermitian cubic norm structures will hold, since we are considering a substructure of a split Hermitian cubic norm structure for which these axioms necessarily hold.
		We set $E$ to be the submodule of $A$ generated by $t$ and $\bar{t}$.
	We observe that $E \otimes K \cong K \oplus K$ (as an algebra) since $\langle t \in A \otimes K | t - \bar{t} \neq 0 \rangle = K \oplus K$, so that $t^2 - t = - a \in \Phi$ and $E \cong \Phi[t]/(t^2 - t + a)$.
	We put
	$$ J = \langle j \in A | ej = j \bar{e}, \forall e \in E \rangle.$$
We know that that $A_K \cong E_K \oplus J_K$. 
	So, for $b \in A$, we write $b = e_k + j_k$ and compute $tb - b\bar{t} = e_k(t - \bar{t})$.
	We obtain that $(tb - b\bar{t})(t - \bar{t}) = e_k(1 - 4a) \in \Phi^\times e_k$ since $4a \neq 1$ ($4a = 1$ implies that $a = 1/4$, $t = 1/2$, $t - \bar{t} = 0$). We see that $e_k \in E$ and $j_k \in J$ and thus $A = J \oplus K$. 
	The $E$-module structure of $J$ is given by left multiplication in $A$. 
	
	We immediately obtain the existence of $\sharp$ restricted from $J_K \times J_K \longrightarrow J_K$ using the third assumption.
	We remark that $\sharp$ defines all operations uniquely. Firstly, $\times $ is just the linearization of $\sharp$.
	Secondly, if $E$ is a field, then $N$ is uniquely defined from $N(a)a = (a^\sharp)^\sharp$. If it is not a field, then there exists an idempotent $e \in E$ such that $E = \Phi e \oplus \Phi (1 - e)$. In that case, $N(a)e, N(a)(1 -e)$ are uniquely defined from evaluating $N$ on $ea,$ $(1 -e)a$.
	Namely, $eJ$ and $(1 - e)J$ are $\Phi$ vector spaces, $\Phi \cong e\Phi \subset E$ acts on the former by scalar multiplication and acts trivially on the latter while $(1 - e)\Phi$ acts only nontrivially on the latter, so $eN(a), (1-e)N(a)$ are easily recovered if $ea, (1-e)a \neq 0$.
	If $ea = 0$, however, then we know that $eN(a) = N(ea) = 0$.
	
	Lastly, we can recover $T$ using $T(a,b) = ab - a \times b$.
 	\end{proof}
\end{lemma}

Using similar observations as for the tensor product of composition algebras, one can prove that $(G,Q,T,P,R)$ forms an operator Kantor system for certain classes of Hermitian cubic norm structures. 

Namely, over fields $\Phi$ a non degenerate (split Hermitian) cubic norm structure is either of the form $\Phi \oplus J(Q,1)$ for some quadratic form with basepoint $1$, $\mathcal{H}_3(\mathcal{C},\gamma)$ for a composition algebra $\mathcal{C}$, or it is anisotropic \cite[Theorem 4.1.59]{MUL22}. 
In the first two cases, \cite[example 2.2 and 2.3]{PETRAC86} provide versions of cubic norm structures defined over arbitrary rings of scalars, which one can use prove that the split Hermitian cubic norm structures corresponding to isotropic cubic norms induce operator kantor pairs associated to the skew dimension $1$ algebras over arbitrary fields.

For anisotropic cubic norm structures, one can also sometimes construct variants over arbitrary rings. For example, any ring $\Phi$ forms a cubic norm structure over itself with $$N(x) = x^3, \quad T(x,y) = 3xy, \quad x^\sharp = x^2.$$ In particular, $\mathbb{Z}[t_i | i \in I] \subset \mathbb{Q}(t_i | i \in I)$ forms such a cubic norm structure and thus induces an operator Kantor pair for all $I$. The split Hermitian cubic norm structure related to a inseperable field extension $E/F$ of degree $3$ is a substructure of the split Hermitian cubic norm structure related to $E$, which is an operator Kantor pair as it is the quotient of the operator Kantor pair induced by $\mathbb{Z}[t_i | i \in E]$ (we should pass via $E[t_i | i \in E]$ in order to obtain an operator Kantor pair over $E$). For the other classes of anisotropic cubic norm structures, which can be denoted as 27/F, 27K/F, 9/F, 9K/F and 3/F using the notation of \cite[Section 15]{Tits2002} and \cite[Theorem 17.6]{Tits2002}, it is easier to identify them with anisotropic substructures of isotropic structures, so that we work with cubic norms on split Albert algebras (27(K)/F), $3 \times 3$ Hermitian matrices over $\Phi[t]/(t^2 - t)$ (9(K)/F), or $\Phi \oplus \Phi \oplus \Phi$ (3/F). 



