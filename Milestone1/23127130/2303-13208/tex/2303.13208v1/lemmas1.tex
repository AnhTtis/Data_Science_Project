

In what follows, will just write $$(\lambda g_1,\lambda^2 g_2) = \lambda \cdot_1 (g_1,g_2),$$ etc., for elements of $G(K)$ instead of using tensor products.

\begin{lemma}
	\label{lemma 2g2 - g1sq}
	Let $G$ be a vector group. If $(g_1,g_2)$ is an element of $G$, then $s_g = (0,2g_2 - \psi(g_1,g_1))$ is an element of $G$ as well.
	Furthermore, for $K \in \Phi\textbf{-alg}$, $\lambda, \mu \in K$ and $g \in G$, we have 
	$$ (\lambda + \mu) \cdot_1 g = (\lambda \cdot_1 g)(\mu \cdot_1 g)(\lambda \mu \cdot_2 s_g).$$
			\begin{proof}
				We compute for $g = (g_1,g_2) \in G$ that
				$$ G \ni g(-1 \cdot_1 g) = (0,2g_2 - \psi(g_1,g_1)) = s_g.$$
				For the second claim, we compute
				\begin{align*}
					(\lambda \cdot_1 g) (\mu \cdot_1 g)(\lambda \mu \cdot_2 s_g) 
					& = (\lambda g_1, \lambda^2 g_2) (\mu g_1, \mu^2 g_2)  (0, 2 \lambda \mu g_2 - \lambda \mu \psi(g_1,g_1)) \\ &= ((\lambda + \mu)g_1,(\lambda^2 + \mu^2)g_2 + \lambda\mu \psi(g_1,g_1)) (0, 2 \lambda \mu g_2 - \lambda \mu \psi(g_1,g_1)) \\
					 & = ((\lambda + \mu)g_1, (\lambda^2 + \mu^2 + 2 \lambda\mu) g_2) \\
					 & = (\lambda + \mu) \cdot_1 (g_1,g_2). \qedhere
				\end{align*}
			\end{proof}
\end{lemma}

\begin{remark}
	We will often write $-g$ for $(-1) \cdot_1 g$.
	So, the element $s_g$ of the previous lemma becomes $g(-g)$.
\end{remark}

\begin{lemma}
	\label{lemma Lie G}
	Let $G$ be a vector group. The underlying $\Phi$-module of $\Lie(G)$ is isomorphic to $$L = G/G_2 \oplus G_2,$$
	which we can identify with a submodule of $A \times B$ using $(a,b)G_2 \mapsto (a,0)$.
			\begin{proof}
				We note that $G/G_2 \subset A$ is an actual module since it is closed under the group operation, and thus addition, and since it is closed under $\cdot_1$, and thus scalar multiplication.
				
				We prove that $G(\Phi[\epsilon])$ is a semi-direct product $G \ltimes (\epsilon L)$.
				This immediately proves that $\text{Lie}(G) \cong L$.
				First, note that
				$$ \epsilon L \subset G(\Phi[\epsilon]),$$
				since $$\epsilon L = (\epsilon \cdot_1 G) \cup (\epsilon \cdot_2 G_2).$$
				We see that $\epsilon L$ is an abelian subgroup.
				Second, one can compute that $\epsilon L$ is stabilized by $G$ under the conjugation action.
				By applying Lemma \ref{lemma 2g2 - g1sq} on $(a + b \epsilon) \cdot_1 g$ we obtain that $\Phi[\epsilon] \cdot G \le G \ltimes \epsilon L$. Since $(a + b \epsilon)\cdot_1$ is an endomorphism, we see that $G \ltimes (\epsilon L)$ is closed under scalar multiplications.
			\end{proof}
\end{lemma}