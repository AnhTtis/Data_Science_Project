\section{Vector groups}\label{se:vectorgroups}
	
	In this section, we introduce vector groups. These can be thought of as a generalization to the $\Phi$-group functor $K \mapsto M \otimes_\Phi K = M(K)$ associated to a module; which is a  $\Phi$-group functor $K \mapsto M(K)$ endowed with a natural transformation $K \otimes M(K) \longrightarrow M(K)$ corresponding to the scalar multiplication, with both the functor and scalar multiplication determined from their analogs over $\Phi$ alone. 
	Vector groups generalize this notion in the following way: we consider a subset $G$ of a module $A \times B$ endowed with a specific group structure and certain scalar multiplications, and ask that $G$ is closed under these scalar multiplications and consider a naturally associated $\Phi$-group contained in $K \mapsto (A \times B) \otimes K.$ 
	
	In the first subsection, we introduce vector groups and prove that the Lie algebra of a vector group is preserved under scalar extensions. In the second subsection, we introduce homogeneous maps on and representations of vector groups, and introduce the corresponding universal objects. In the last subsection, we determine the primitive elements of the universal representation for projective vector groups and recover each projective vector group from its universal representation.
	
	\subsection{Definition and basic properties}

Consider $\Phi$-modules $A$ and $B$ and a bilinear form $$\psi \colon A \times A \longrightarrow B.$$ We endow $A \times B$ with a group structure $$(a,b)(c,d) = (a + c , b + d +\psi(a,c)),$$
a first type of scalar multiplication
$$ \lambda \cdot_1 (a,b) = (\lambda a , \lambda^2 b),$$
and a second one 
$$ \lambda \cdot_2 (0,b) = (0, \lambda b),$$
where the second scalar multiplication is only defined on the subgroup $0 \times B$. Since $\psi$ is a bilinear form, we see that $\lambda \cdot_1$ is an endomorphism of $A \times B$.

\begin{definition}
	\label{definition vecgroup}
	Consider a subgroup $G$ of  $A \times B$. We use $G(K)$ to denote  the minimal subgroup of $(A \times B)\otimes K$ that contains $(g_1 \otimes 1_K, g_2 \otimes 1_K)$ for all $(g_1,g_2) \in G$ and is closed under both types of $K$-scalar multiplications (note that $k \cdot_2 g$ is only defined for $g \in B \otimes K$ and $k \in K$). We call $G$ a \textit{vector group} if
	\begin{enumerate}
		\item $G$ is closed under $g \mapsto \lambda \cdot_1 g$,
		\item $G_2 = G \cap 0 \times B$ is closed under $g \mapsto \lambda \cdot_2 g$,
		\item if $1/2 \notin \Phi$, then we also ask that $G_2(K) \subseteq G_2 \otimes K.$
	\end{enumerate}
Remark that we can restate conditions (1) and (2) as $G = G(\Phi)$.
\end{definition}

\begin{remark}	
	There are no known counterexamples for the third condition.
	However, it is quite an important assumption to ensure that the vector groups behave uniformly.
	Specifically, the last assumption will ensure that the Lie algebra that we will associate to a vector group is preserved over scalar extensions. 
	In the context of quadratic Jordan pairs, one requires that the axioms hold over all scalar extensions. Nevertheless, what we require may not be entirely similar; if one works over fields of characteristic $2$, then the third condition always holds.
	Later, cfr. Lemma \ref{Lemma reparam}, we shall see that $G_2(K) = G_2 \otimes K$ follows from the first and second conditions for rings containing $1/2$ as well.
\end{remark} 

\begin{lemma}
	\label{lemma G(K)}
	The group $G(K)$ is generated as a subgroup of $(A \times B) \otimes K$ by the sets of generators $K \cdot_1 (g \otimes 1)$ for $g \in G$ and $K \cdot_2 (G_2 \otimes 1)$.
	\begin{proof}
		By definition, both  $K \cdot_1 (g \otimes 1)$ for $g \in G$ and $K \cdot_2 (G_2 \otimes 1)$ should be contained in $G(K)$. So, we need to prove that the group generated by those sets is closed under scalar multiplications.
		This is the case since scalar multiplications are endomorphisms of $G(K)$ such that
		$$ \lambda \cdot_i (\mu \cdot_i g) = (\lambda\mu \cdot_i g),$$
		for $i = 1,2$ and 
		$$ \lambda_i \cdot_i (\lambda_j \cdot_j g) = \lambda_i^j \lambda^i_j \cdot_2 g,$$
		for $i,j$ such that $\{i,j\} = \{1,2\}.$
	\end{proof}
\end{lemma}

From the previous lemma, we can conclude immediately that $K \longmapsto G(K)$ is a functor from $$\Phi\textbf{-alg} \longrightarrow \textbf{Grp}$$ and thus a $\Phi$-group.
Consider the ring of \textit{dual numbers} $\Phi[\epsilon]$, i.e., the ring obtained by adding a generator $\epsilon$ satisfying $\epsilon^2 = 0$, and consider the projection $\pi : \Phi[\epsilon] \longrightarrow \Phi$.
We use $\text{Lie}(G)$ to denote the Lie algebra associated to $G$: $$ \text{Lie}(G) = \ker G(\pi) = \{(\epsilon a, \epsilon b) \in G(\Phi[\epsilon])\}$$ with as bracket the restriction of the commutator on $A \times B$, or equivalently
$$ [(a,b),(c,d)] = (0,\psi(a,c) - \psi(c,a)).$$
 This is the Lie algebra one typically associates to a linear algebraic group, cfr. \cite[the definition preceding Example 3.1]{Milne} (even though the Lie bracket is typically introduced in a different fashion).



In what follows, will just write $$(\lambda g_1,\lambda^2 g_2) = \lambda \cdot_1 (g_1,g_2),$$ etc., for elements of $G(K)$ instead of using tensor products.

\begin{lemma}
	\label{lemma 2g2 - g1sq}
	Let $G$ be a vector group. If $(g_1,g_2)$ is an element of $G$, then $s_g = (0,2g_2 - \psi(g_1,g_1))$ is an element of $G$ as well.
	Furthermore, for $K \in \Phi\textbf{-alg}$, $\lambda, \mu \in K$ and $g \in G$, we have 
	$$ (\lambda + \mu) \cdot_1 g = (\lambda \cdot_1 g)(\mu \cdot_1 g)(\lambda \mu \cdot_2 s_g).$$
			\begin{proof}
				We compute for $g = (g_1,g_2) \in G$ that
				$$ G \ni g(-1 \cdot_1 g) = (0,2g_2 - \psi(g_1,g_1)) = s_g.$$
				For the second claim, we compute
				\begin{align*}
					(\lambda \cdot_1 g) (\mu \cdot_1 g)(\lambda \mu \cdot_2 s_g) 
					& = (\lambda g_1, \lambda^2 g_2) (\mu g_1, \mu^2 g_2)  (0, 2 \lambda \mu g_2 - \lambda \mu \psi(g_1,g_1)) \\ &= ((\lambda + \mu)g_1,(\lambda^2 + \mu^2)g_2 + \lambda\mu \psi(g_1,g_1)) (0, 2 \lambda \mu g_2 - \lambda \mu \psi(g_1,g_1)) \\
					 & = ((\lambda + \mu)g_1, (\lambda^2 + \mu^2 + 2 \lambda\mu) g_2) \\
					 & = (\lambda + \mu) \cdot_1 (g_1,g_2). \qedhere
				\end{align*}
			\end{proof}
\end{lemma}

\begin{remark}
	We will often write $-g$ for $(-1) \cdot_1 g$.
	So, the element $s_g$ of the previous lemma becomes $g(-g)$.
\end{remark}

\begin{lemma}
	\label{lemma Lie G}
	Let $G$ be a vector group. The underlying $\Phi$-module of $\Lie(G)$ is isomorphic to $$L = G/G_2 \oplus G_2,$$
	which we can identify with a submodule of $A \times B$ using $(a,b)G_2 \mapsto (a,0)$.
			\begin{proof}
				We note that $G/G_2 \subset A$ is an actual module since it is closed under the group operation, and thus addition, and since it is closed under $\cdot_1$, and thus scalar multiplication.
				
				We prove that $G(\Phi[\epsilon])$ is a semi-direct product $G \ltimes (\epsilon L)$.
				This immediately proves that $\text{Lie}(G) \cong L$.
				First, note that
				$$ \epsilon L \subset G(\Phi[\epsilon]),$$
				since $$\epsilon L = (\epsilon \cdot_1 G) \cup (\epsilon \cdot_2 G_2).$$
				We see that $\epsilon L$ is an abelian subgroup.
				Second, one can compute that $\epsilon L$ is stabilized by $G$ under the conjugation action.
				By applying Lemma \ref{lemma 2g2 - g1sq} on $(a + b \epsilon) \cdot_1 g$ we obtain that $\Phi[\epsilon] \cdot G \le G \ltimes \epsilon L$. Since $(a + b \epsilon)\cdot_1$ is an endomorphism, we see that $G \ltimes (\epsilon L)$ is closed under scalar multiplications.
			\end{proof}
\end{lemma}

\begin{lemma}
	Let $\rho_K : G(K) \longrightarrow H(K)$ be a natural transformation  between $\Phi$-group functors associated to vector groups such that 
	\[ \rho_K(\lambda \cdot_i g) = \lambda \cdot_i \rho_K(g)\]
	for all $g \in G(K), \lambda \in K$.
	If $1/2 \in \Phi$, then $L = \ker \rho$ and $I = \Ima \rho$ are vector groups.
	Moreover, if $1/2 \notin \Phi$,  $\ker \rho_K \cap G_2(K)$ is the $K$-linear span of $(\ker \rho \cap G_2(\Phi)) \otimes 1$, and if $\rho(G_2(K)) = I_2(K)$
	for all $K \in \Phi\textbf{-alg,}$
	then $L$ and $I$ are vector groups as well.
	\begin{proof}
		
		It is trivial to see that $L$ is closed under both scalar multiplications. So, we conclude that $L$ is a vector group, using an additional assumption if $1/2 \notin \Phi$.
		
		The group $I$ is closed under both scalar multiplications as well. So, $I$ is definitely a vector group if $1/2 \in \Phi$.
		For the third axiom of vector groups, note that $I_2(K) = \rho(G_2(K)) = \rho(G_2 \otimes K) = I_2 \otimes K.$
	\end{proof}
\end{lemma}

\begin{definition}
	We call a natural transformation $\rho : G \longrightarrow H$ between the $\Phi\textbf{-grp}$ functors associated to vector groups $G,H$, a \textit{vector group homomorphism} if 
	\[ \rho(\lambda \cdot_i g) = \lambda \cdot_i \rho(g),\]
	and $\rho^{-1}(H_2) \subseteq G_2 \ker\rho.$
	If $1/2 \notin \Phi$ we require, additionally, that 
	\[ \ker \rho_K \cap G_2(K) = \langle g \otimes 1 \in B \otimes K | g \in G_2 \cap \ker \rho \rangle ,\]
	where the right-hand side is the span as a $K$-module.
	We use $\textbf{VecGrp}$ to denote the category of vector groups over $\Phi$ with these morphisms. If there is any doubt about the base ring, we write $\Phi\textbf{-VecGrp}$.
\end{definition}

\begin{remark}
	All the conditions, except $\rho^{-1}(H_2) \subset G_2 \ker\rho$, are justified by the fact that $\ker \rho$ and $\Ima \rho$ must be vector groups as well.
	This other condition translates into $\rho^{-1}(H_2) \subset G_2$ for injective maps, i.e., injective morphisms preserve the second component.
	The more general condition is obtained by requiring that this holds and
	that $G \longrightarrow H$ factors through some kind of vector group $L = G/\ker \rho$ with $L_2$ obtained by projecting $G_2$.
	
	At this moment, we should be careful if we write $G/K$ for normal $K \le G$, since we did not prove that $G/K$ is a vector group for each normal vector subgroup $K$ in general. However, we can always consider it as a $\Phi$-group with well-defined scalar endomorphisms and a well-defined subgroup $(G/K)_2$. Later, we will see that $G/K$ is in fact a vector group.
\end{remark}

\begin{lemma}
	\label{Lemma reparam}
	Let $L_1 \oplus L_2$ be a $\mathbb{Z}$-graded Lie algebra over a ring $\Phi$ with $1/2 \in \Phi$. Then $L_1 \oplus L_2$ forms a vector group with operation $$(a,b)(c,d) = (a+c,b+d + [a,c]/2).$$
	Moreover, each vector group $G$ over $\Phi$ is isomorphic to the vector group associated to $\Lie(G)$ by mapping 
	$$ (a,b) \longmapsto (a,b - \psi(a,a)/2).$$
	Hence, the third axiom for vector groups will also hold if $1/2 \in \Phi$.
		\begin{proof}
			Note that $G = L_1 \oplus L_2$ so that all axioms for vector groups are trivially satisfied.
			
			Now, we prove that each vector group can be reparametrized in this fashion. Lemma \ref{lemma 2g2 - g1sq} proves that $2b - \psi(a,a) \in G_2 \subset \text{Lie}(G)$.
			So, we conclude that 
			$(a,b - \psi(a,a)/2) \in \text{Lie}(G).$ To observe the surjectivity of this map, take arbitrary $(a,b) \in G$ and note that the coset $(a,b) G_2$ maps to all $(a,s) \in \text{Lie}(G)$.
			On the other hand, the reparametrization is obviously injective.
			
			A direct computation shows that this reparametrization is a group homomorphism if one uses that $\psi(a,c) - \psi(c,a) = [a,c]$.
			Moreover, the scalar multiplications and $G_2$ are preserved, so that we have a vector group isomorphism.
		\end{proof}
\end{lemma}

\begin{remark}
	The fact that the operation in the previous lemma is given by
	$$(a,b)(c,d) = (a+c,b+d + [a,c]/2)$$
	shows that vector groups, if $1/2 \in \Phi$, correspond directly to groups of exponentials. Specifically, if we consider the Lie algebra
	$$ L = \Phi \zeta \oplus L_1 \oplus L_2,$$
	with $[\zeta ,l_i] = il_i$ for $l_i \in L_i$, then the group 
	$$ \exp(L_1 \oplus L_2) \le \text{Aut}(L)$$ has the aforementioned operation.
\end{remark}

Let $G$ be a vector group over $\Phi$, then $G(K)$ is a vector group over $K$. We write $L \longmapsto G_K(L)$ to denote the $K\textbf{-Grp}$ functor associated to the vector group $G_K = G(K)$.

\begin{lemma}
	\label{lemma G(K) is Lie(G)K}
	Let $G$ be a vector group and $K \in \Phi\textbf{-alg}$, then  $\text{Lie}(G_K) \cong \Lie(G) \otimes K$.
		\begin{proof}
			First, recall the equality $G_2(K) = G_2 \otimes K$ which is assumed in the definition of a vector group if $1/2 \notin \Phi$. We can also obtain the equality from Lemma \ref{Lemma reparam} if $1/2 \in \Phi$. Secondly, the possible first coordinates of elements of $G_K$ are $K$-linear combinations of possible first coordinates of $G$. We conclude that $\text{Lie}(G_K) \cong \text{Lie}(G) \otimes K$ by Lemma \ref{lemma Lie G}.
		\end{proof}
\end{lemma}



\begin{definition}
	\label{def free}
	We call a vector group $G$ \textit{free (resp. projective)} if $G_2$ and $G/G_2$ are free (resp. projective) as $\Phi$-modules.	
	Similarly, we say that $G$ is \textit{finitely generated} if $G/G_2$ and $G_2$ finitely generated as $\Phi$-modules. 
\end{definition}

Note that there exist non-canonical bijections between $G$ and $\text{Lie}(G)$, since $(a,b), (a,c)$ being elements of $G$ implies that $(a,b)(-a,c) = (0,b + c - \psi(a,a)) \in G_2$ so that for each possible first coordinate $a$ of $G$, the possible second coordinates stand in bijection with $\psi(a,a) - b + G_2 \cong G_2$.
So, Lemma \ref{lemma G(K) is Lie(G)K} shows that $G$ can be identified with a $\textbf{Grp}$-functor which has $$K \longmapsto \text{Lie}(G) \otimes K$$ as underlying $\textbf{Set}$-functor.

If $G/G_2$ is projective, then this identification is given by a natural transformation. Namely, since this module is projective, we have a generating set $(a_i)_{i \in I}$ of $G/G_2$ and dual generating set $(\alpha_i)_{i \in I}$ such that
$$ v = \sum_{i \in I} \alpha_i(v)a_i,$$
with finite $\alpha_i(v) \neq 0$. By using a total order on $I$ and taking $(a_i,b_i) \in G(\Phi)$ for each $i$, we obtain a natural transformation 
$$ \text{Lie}(G)(K) \longmapsto G(K) : (v,s) \longmapsto \left(v, s + \sum_{i \in I} \alpha_i(v)^2b_i + \sum_{i < j} \alpha_i(v)\alpha_j(v) \psi(a_i,a_j)\right) \in G(K).$$


If $G$ is finitely generated projective, then we know that $G$ is an affine algebraic group. 
Specifically, set $V = \text{Lie}(G)$ then we know that $K \mapsto V \otimes K$ is an affine algebraic scheme underlying $G \mapsto G(K)$.
To see that this is the case, use the module $V^*$ which is finitely generated and projective \cite[Chapter 2.6]{BouAlg1}, to consider the algebra $\text{Sym}(V^*)$ (where we do not use the (usual) Hopf algebra structure of this algebra) which is a finitely generated as an algebra. 
Using that 
$$ \text{Hom}_\Phi(\text{Sym}(V^*),K) \cong \text{Hom}_\Phi(V^*,K) \cong  V \otimes K,$$
for finitely generated projective modules $V$ \cite[Chapter 2.7, Corollary 4]{BouAlg1}, we obtain that $G$ is an affine algebraic group scheme.


\begin{lemma}
	\label{Lemma quotient free}
	Each vector group $G$ is a quotient of a free vector group.
		\begin{proof}
			We construct a free vector group over sets $S_1, S_2$ in the usual categorical sense of freeness, i.e, it is a vector group $F(S_1,S_2)$ such that for each vector group $G$ and $f_i : S_i \longrightarrow G_i$ (where we use $G_1$ to denote the $g \in G\setminus G_2 \cup \{(0,0)\}$), there exists a unique vector group morphism $f:F(S_1,S_2) \longrightarrow G$. We will also prove that this vector group is free in the vector group sense.
			
			First, fix a total order of $S_1$.
			Second, we define $T_2 = S_2 \cup \{[a,b] \in S_1 \times S_1 | a < b\} \cup \{ s(t) | t \in S_1\},$
			where we wrote $[a,b] \in S_1 \times S_1$ as they will be used to represent commutators, and use the symbol $s(t)$ to represent $t(-t)$ for $t \in S_1$.
			Fix a total order of $T_2$ as well.
			We will endow the functor
			$$ K \longrightarrow F(S_1,S_2)(K) = \left\{ \prod_{s \in S_1} k_s \cdot s \prod_{t \in T_2} l_t \cdot t | \text{ only finitely many nonzero }  k_s,l_t \in K\right\}$$
			where the product is a symbolic product respecting the ordering of $S_1$ and $T_2$,
			with a vector group structure, and we wrote products as this will represent a product of group elements.
			
			 If we can do that, we have defined a free vector group in the categorical sense.
			Note that this vector group, if it is a vector group, is free in the usual sense of Definition \ref{def free}, since the Lie algebra is freely generated by $S_1$ and $T_2$.
			
			We will define $F(S_1,S_2)(K)$ as a vector group contained of the free $K$-module $F_1 \times F_2$ with $F_1$ free over $S_1$ and and $F_2$ free over $U_2 = S_2 \cup (S_1 \times S_1) \cup t(S_1)$ (where we write $(s_1,s_2) \in S_1 \times S_2$ as usual for products of sets), endowed with the group operation
			$$ (s,t)(s',t') = (s + s',t + t' + (s,s')),$$
			where $(s,s')$ is the bilinear form defined from extending the identity map $$S_1 \times S_1 \longrightarrow S_1 \times S_1 \longrightarrow F(S_1 \times S_1)$$ bilinearly. In this module, we use $2t(l) - (l,l)$ to represent the element $s(l) \in T_2$ for $l \in S_1$, and use $(a,b) - (b,a)$ to represent $[a,b] \in T_2$.
			
			The map which sends each element $g$ to its inverse $g^{-1}$ defined on $F_1 \times F_2$ can be projected to only act on the second component $F_2$; we obtain a map $i$ which acts as $l \mapsto -l, (a,b) \mapsto (b,a), t(k) \mapsto - t(k) + (k,k)$ for $a,b,k \in S_1, l \in S_2$. The submodule containing all $x$ in $F_2$ such that $x + i(x) = 0$ is precisely the free module generated by $T_2$, using the identifications made between generators.
			We define
			$$ \hat{F}(S_1,S_2)(K) = \left\{ (a,b) \in (F_1 \times F_2) \otimes K | b + i(b) = (a,a)\right\}.$$
			Note that for each $s \in S_1$ the element $(ks_1,k^2 t(s_1))$ lies in $\hat{F}(S_1,S_2)(K)$ for $k \in K$, so that the underlying set functor $K \mapsto \hat{F}(S_1,S_2)(K)$ coincides with the earlier definition of $F$.
			Using the fact that $[a,b] = (a,b) - (b,a)$ and $s(a) = 2t(a) - (a,a)$, we see that the elements $[a,b] , s(a)$ of $F$ coincide with what they should be, which proves the freeness in the categorical sense.
			
			Consider a vector group $G \le A \times B$, and take generating set $S_1$ of $G/G_2$ and $S_2$ of $G_2$ where we identify $S_1$ with a transversal set in $G$, i.e., a set $S$ mapping bijectively to $S_1$ under $s \mapsto sG_2$. The quotient $F(S_1,S_2) \longrightarrow G$ is a quotient in the category $\textbf{VecGrp}$, since the morphism induces a graded surjection $\text{Lie}(F(S_1,S_2)_K) \longrightarrow \text{Lie}(G_K)$.
		\end{proof}
\end{lemma}

\subsection{Representations and homogeneous maps}

\begin{definition}Let $M$ be a module and $G$ a vector group.
		Identify $M$ with the functor $$\Phi\textbf{-alg} \longrightarrow \textbf{Set}: K \longmapsto M \otimes K.$$
		Similarly, identify $G$ with $G \longrightarrow G(K).$
		We shall define homogeneous maps of degree $n$ using recursion on the degree.	
		We call a natural transformation $f : G \longrightarrow  M$ a \textit{homogeneous map of degree $n$} if there exist linearisations $$f^{(i,j)} : G \times G \longrightarrow M,$$ for all $i,j \in \mathbb{N}_{>0}$ with $i + j = n$, which are homogeneous of degree $i$ in the first component, homogeneous of degree $j$ in the second component, such that
		$$ f_K((\lambda \cdot_a g)(\mu \cdot_b h)) = f_K(\lambda \cdot_a g) + f_K(\mu \cdot_b h) + \sum_{ai + bj = n} \lambda^{i}\mu^j f_K^{(ai,bj)}(g,h),$$
		for all $\lambda, \mu \in K$, in which we ask that 
		$$ f_K(\lambda \cdot_1 g) = \lambda^n f_K(g),$$
		and
		$$ f_K(\lambda \cdot_2 g) = \begin{cases}
			0 & n \text{ is odd }\\
			\lambda^{(n/2)} f_K(g) & \text{otherwise}
		\end{cases}.
		$$
		Remark that homogeneous maps of degree $0$ are constant maps\footnote{To prove this, use that $0^0$ means $0$ multiplied by itself $0$ times, i.e., $1$, in this context.}, and homogeneous maps of degree $1$ are maps that factor as linear maps through $G/G_2$.
	\end{definition}
	\begin{definition}Let $C$ be an associative unital algebra and $t$ a variable over which we consider formal power series. 
		We identify $(1 + tC[[t]])$ with the $\Phi\textbf{-Grp}$ functor
		$$K \mapsto 1 + t(C \otimes K)[[t]].$$
		Consider a sequence $(\rho_{i} : G \longrightarrow C)_i$ of maps.
		Assume that these define a natural transformation of $\Phi$-groups $\rho_{[t]} : G \longrightarrow (1 + tC[[t]])$ using
		$$  \rho_{[t]}(g) = \sum \rho_i(g) t^i.$$ 
		Note that this satisfies, as it is a natural transformation of $\Phi$-groups, the equation
		$$ \rho_{[t]}(gh) = \rho_{[t]}(g)\rho_{[t]}(h)$$
		for all $g,h \in G(K)$.
		We call $(\rho_{i} : G \longrightarrow C)_i$ a \textit{vector group representation} if
		\begin{enumerate}
			\item $\rho_{[t]}(\lambda \cdot_1 g) = \rho_{[{\lambda t}]}(g)$, i.e., the first scalar multiplication corresponds to substituting $\lambda t$ for $t$ in the formal power series,
			\item $\rho_{[t]}(G_2) \le (1 + t^2C[[t^2]])$, so that we can take $$\sigma_{[t]} : G_2 \longrightarrow (1 + tC[[t]])$$ such that $\sigma_{[t^2]} = \rho_{[t]}$, 
			\item $\sigma_{[t]}(\lambda \cdot_2 g) = \sigma_{[\lambda t]}(g)$,
			\item $\rho_{[t]}(g) \in (1 + t^2C[[t^2]])$ means that 
			$$ g \in (\ker \rho) \cdot G_2,$$
			i.e., there exists $h \in G_2$ such that $gh \in \ker \rho$.
		\end{enumerate}	
\end{definition}

\begin{lemma}
	Suppose that $(\rho_i : G \longrightarrow C)_i$ is a vector group representation, then each $\rho_i$ is a homogeneous map of degree $i$.
	\begin{proof}
		This is obviously the case for $i = 0,1$.
		The rest follows easily by induction using that  \[ \rho^{(i,j)}_n(g,h) = \rho_i(g)\rho_j(h). \qedhere \]
	\end{proof}
\end{lemma}

\begin{lemma}
	\label{Lemma homogeneous map on commutator}
	Let $G$ be a vector group, and $M$ a module
	Suppose that $f : G \longrightarrow M$ is a homogeneous map of degree $2$, then 
	$$ f^{(1,1)}(a,b) - f^{(1,1)}(b,a) = f([a,b]) = f(a^{-1}b^{-1}ab),$$
	holds for all $a,b \in G$.
	\begin{proof}
		Using the $(1,1)$-linearisation, one obtains 
		\begin{align*}
		f([a,b]) = &  f(a^{-1}) + f(b^{-1}) + f(a) + f(b) + f^{(1,1)}(a^{-1},a) + f^{(1,1)}(b^{-1},b) \\ &  + f^{(1,1)}(a^{-1},b^{-1}) + f^{(1,1)}(a^{-1},b) + f^{(1,1)}(a,b) + f^{(1,1)}(b^{-1},a).
		\end{align*}
	We use that $f(a^{-1}) + f^{(1,1)}(a^{-1},a) + f(a)  = f(0) = 0$, and that $f^{(1,1)}$ is linear if we interpret it as a map defined on $(G/G_2)^2$ so that $f^{(1,1)}(a,b^{-1}) = -f^{(1,1)}(a,b)$, etc., to obtain
	\[ f([a,b]) = f^{(1,1)}(a,b) - f^{(1,1)}(b,a). \qedhere \]
	\end{proof}
\end{lemma}

Note that $\rho_{[t]} : G \longrightarrow \Phi \oplus (A \times B)[[t]] : (g_1,g_2) \mapsto 1 + (tg_1,t^2g_2)$ can be seen as a vector group representation if we endow $\Phi \oplus (A \times B)$ with the multiplication $$(\lambda + (a,b))(\mu + (c,d)) = \lambda\mu + ((\mu a + \lambda c  ,\mu b + \lambda d + \psi(a,c)).$$

\begin{remark}
	\label{Remark vecgroup repres}
	With a vector group representation $\rho_i : G \longrightarrow C$, one can associate the vector group $\Ima \rho \le C \times C$, where the latter vector group is associated to the bilinear form $(a,b) \longrightarrow ab$ and is formed by the elements $(\rho_1(g),\rho_2(g))$. A vector group representation induces a morphism $$G \longrightarrow C \times C.$$
\end{remark}


\begin{lemma}
	Let $C$ be an associative unital algebra and $G$ a vector group.
	Consider a vector group representation $(\rho_i : G \longrightarrow C)_i$.
	For all $g,h \in G$ and $k \in G_2$ the following equations, in which we write
	$g_i$ for $\rho_i(g)$, hold:
	\begin{enumerate}
		\item $$\binom{i + j}{i}k_{2(i+j)} = k_{2i}k_{2j},$$
		\item $$\binom{i+j}{i} g_{i+j} = \sum_{\substack{a + c = i\\ b + c = j}} g_ag_b (g(-g))_{2c},$$
		\item $$g_i h_j = \sum_{\substack{a + c = i\\b + c = j}} h_bg_a [g,h]_c.$$
	\end{enumerate}
	\begin{proof}
		The last two equations are obtained by comparing the terms belonging to the scalar $\lambda^i\mu^{j}$ in
		\begin{align*}
			((\lambda + \mu) \cdot_1 g)_{i+j} & = ((\lambda \cdot_1 g)(\mu \cdot_1 g)(\lambda\mu \cdot_2 (g(-g)))_{i+j}, \\
			& \text{and} \\
			((\lambda \cdot_1 g)(\mu \cdot_1 h))_{i+j} & = ((\mu \cdot_1 h)(\lambda \cdot_1 g)(\lambda\mu \cdot_1 [g,h]))_{i+j},
		\end{align*}
		where the first of these equations must hold by Lemma \ref{lemma 2g2 - g1sq}, and the second follows from a direct computation.
		Similarly, one proves the first equation of this lemma.
	\end{proof}
\end{lemma}

\begin{definition}
	Let $G$ be a vector group.
	We say that $(\rho_i : G \longrightarrow U)_i$ is a \textit{universal vector group representation} if for each vector group representation $(\gamma_i : G \longrightarrow V)_i$, there exists a unique algebra homomorphism $f : U \longrightarrow V$ such that $f \circ \rho_i = \gamma_i$.
	Similarly, we call a homogeneous map $\rho : G \longrightarrow M$ of degree $n$ a \textit{universal homogeneous map of degree} $n$, if for each $\gamma : G \longrightarrow N$ of degree $n$ there exist a unique linear map $f : M \longrightarrow N$ such that $\gamma = f \circ \rho$.
\end{definition}


\begin{remark}
	Universal vector group representations and universal homogeneous maps are unique up to isomorphism.
\end{remark}

\begin{construction}
	\label{construction universal representation}
We define $\mathcal{U}(G)$ as the unital associative algebra generated by symbols $g_i$ for $g \in G$ and $i \in \mathbb{N}$ and relations
\begin{enumerate}
	\item $g_0 = 1$ for all $g$
	\item $g_{2i+1} = 0$ for all $g \in G_2$
	\item $\sum_{i + j = n} g_ih_j = (gh)_n$
	\item $(\lambda \cdot_1 g)_{j} = \lambda^j g_{j}$
	\item $(\lambda \cdot_2 g)_{2j} = \lambda^j g_{2j}$ for $g \in G_2$
	\item $\binom{i + j}{i}g_{2(i+j)} = g_{2i}g_{2j}$ for $g \in G_2$ \label{cons:g2 scalar}
	\item $\binom{i+j}{i} g_{i+j} = \sum_{\substack{a + c = i\\ b + c = j}} g_ag_b (g(-g))_{2c}$ \label{cons:g scalar}
	\item $g_i h_j = \sum_{\substack{a + c = i\\b + c = j}} h_bg_a [g,h]_c$
\end{enumerate}
\end{construction}

Note that $\mathcal{U}(G)$ is an $\mathbb{N}$-graded algebra if we set $g_j$ to be $j$-graded. Observe, additionally, that all the relations imposed on $\mathcal{U}(G)$ are necessary to have a representation
$$ \rho_i : G \longrightarrow \mathcal{U}(G): g \longmapsto g_i.$$
So, if there exists a vector group representation to $\mathcal{U}(G)$, then it is obviously a universal one.
We can write $g \mapsto \rho_{[t]}(g) = 1 + tg_1 + t^2g_2 + \ldots$, as if there exists a representation, for $g \in G(\Phi)$.

We shall prove that $\mathcal{U}(G)$ is the universal representation for arbitrary $G$ by first proving it for free $G$.

\begin{lemma}
	The algebra $\mathcal{U}(G)$ is an $\mathbb{N}$-graded Hopf algebra with operations
	\[ \Delta(g_n) = \sum_{i +j = n} g_i \otimes g_j, \quad \epsilon(g_i) = \delta_{i0}, \quad S(g_n) = (g^{-1})_n.\]
	\begin{proof}
		We note that all operations are compatible with the grading, so we only need to check whether we have a Hopf algebra.
		
		The comultiplication $\Delta$ is compatible with all the relations to which $\mathcal{U}(G)$ is subject as well.
		There are only $2$ relations that are not straightforwardly satisfied, as they involve binomial coefficients.
		We prove the compatibility with the comultiplication for the most difficult relation of them, namely
		$$ \binom{i+j}{i} g_{i+j} = \sum_{\substack{a + c = i\\ b + c = j}} g_ag_b (g(-g))_{2c}.$$
		We compute
		\begin{align*}
			\Delta \left(  \sum_{\substack{a + c = i\\ b + c = j}} g_ag_b (g(-g))_{2c} \right)			
			& = \sum_{\substack{a_1 + c_1 + a_2 + c_2 = i\\ b_1 + c_1 + b_2 + c_2 = j}} g_{a_1}g_{b_1} (g(-g))_{2c_1} \otimes g_{a_2}g_{b_2} (g(-g))_{2c_2}\\
			& = \sum_{\substack{k + l = i + j\\ m + n = i}} \binom{k}{m} g_{k} \otimes \binom{l}{n} g_{l} \\ 
			& = \binom{i+ j}{i}\sum_{k + l = i + j} g_{k} \otimes g_l,
		\end{align*}
	in which the second to last equality comes from applying the relation to the sum where we observe that the equations over which we sum are equivalent to 
	$$ \sum a_i + \sum b_i + 2 \sum c_i = i +j \quad \sum a_i + \sum c_i = i,$$
	and where the last equality corresponds to 
	$$  \sum_{\substack{ m + n = i}} \binom{k}{m} \binom{l}{n} = \binom{i + j}{i}$$
	for $k,l$ such that $k + l = i + j$.
This equality between binomial coefficients corresponds to computing the term belonging to $a^ib^j$ in
	$$ (a + b)^k(a + b)^l = (a + b)^{i +j}$$
	over $\mathbb{Z}[a,b]$.
	All other relations are either trivial or depend on the same binomial formula.
	
	The antipode and counit are fine as well since $\rho_{[t]}(g)\rho_{[t]}(g^{-1}) = 1$ with $\rho_{[t]}(g) = \sum t^i g_i$.
	\end{proof}
\end{lemma}



\begin{lemma}
	\label{lemma free universal implies universal}
	If $\mathcal{U}(G)$ is the universal representation of $G$ for all free $G$ over $\mathbb{Z}$, then $\mathcal{U}(G)$ is the universal representation of $G$ for all $G$.
	\begin{proof}
		Suppose that $G$ is an arbitrary vector group. Write it as a quotient $H/K$ of a free vector group $H$ using Lemma \ref{Lemma quotient free} and note that $H = L_\Phi$ for some free vector group $L$ over $\mathbb{Z}$.		
		We have a representation corresponding to $\rho_{i,\Phi} : L(\Phi) \longrightarrow \mathcal{U}(L) \otimes \Phi  \longrightarrow \mathcal{U}(G)$ of $L_\Phi$ which factors through $L_\Phi/K \cong G$ so that $g \mapsto g_i$ is a representation of $G$.
	\end{proof}
\end{lemma}



Now, we prove that $\mathcal{U}(G)$ is the universal representation for all free $G$ over $\mathbb{Z}$. To achieve that, we construct a similar algebra $\mathcal{X}(G)$ which will be isomorphic to $\mathcal{U}(G)$.

\begin{construction}Suppose that $G$ is free, let $B_1$ be a set of free generators for $ G/G_2$ and let $B_2$ be a set of free generators for $G_2$. Take for each $b \in B_1$ an element $\hat{b} \in G$ so that $b = \hat{b}G_2$.
We assume that $B_1$ and $B_2$ are totally ordered.
Consider the unital associative algebra $F(G)$ with generators $b_i$ for $b \in B_1$ and $i \ge 1$ and $b_{2i}$ for $b \in B_2$ and $i \ge 1$. We use $b_0$ to denote $1$ for all $b \in B_1 \cup B_2$.
For $g = \sum_{j = 1}^k \lambda_j \cdot_2 b_{j} \in G_2$ with $b_i < b_j$ for $i < j$, we write
$$g_{2n} = \sum_{i_1 + \ldots + i_k = n} \prod_{j = 1}^k \lambda^{i_j}(b_j)_{2i_j}.$$

We impose the following relations for $b < c \in B_1$, $d < e \in B_2$ on $F$ to obtain $\mathcal{X}(G)$.
\begin{enumerate}
	\item $b_ib_j = \sum_{k + 2l = i + j} \binom{k}{i - l} b_k (-1 \cdot_2 (-b)b)_{2l}$.
	\item $d_{2i}d_{2j} = \binom{i + j}{i} d_{2(i+j)}$
	\item $c_jb_i = \sum_{\substack{k + m = i\\l + m = i}} b_k c_l [\hat{c},\hat{b}]_{2m}$
	\item $d_{2j}b_i = b_id_{2j}$
	\item $e_{2i}d_{2j} = d_{2j}e_{2i}.$
\end{enumerate}
\end{construction}

All of these relations hold in $\mathcal{U}(G)$. This is only non-trivially the case for the first equation, which can be restated as
$$ \rho_{[t]}(b)\rho_{[s]}(b) = \rho_{[t + s]}(b)\sigma_{[ts]}(-1 \cdot_2 (-b)b),$$
where $\sigma_{[t^2]} = \rho_{[t]}$ on $G_2$. This relation holds in the algebra $\mathcal{U}(G)$ since we have imposed relations that ensure that
$$  \rho_{[t]}(\lambda \cdot_1 b)\rho_{[t]}(\mu \cdot_1 b)\sigma_{[ t^2]}(\lambda \mu \cdot_2 (-b)b) = \rho_{[t]}((\lambda + \mu) \cdot_1 b),$$
and 
$$ \rho_{[t]}(g)\rho_{[t]}(h) = \rho_{[t]}(gh)$$
hold in $\mathcal{U}(G)[\lambda,\mu][[t]]$.
So, we know that there is a unique morphism $\mathcal{X}(G) \longrightarrow \mathcal{U}(G)$ which maps on generators as $b_i \mapsto b_i$ for $b \in B_1 \cup B_2$. 
Furthermore, note that if there exists a vector group representation  $G$ in $\mathcal{X}(G)$ which maps the generators $b \in B_1 \cup B_2$ as expected, then we obtain $\mathcal{U}(G) \longrightarrow \mathcal{X}(G)$ mapping $b_i \mapsto b_i$. This is the case as each vector group representation $(\rho_i: G \longrightarrow A)_i$ induces a map $\mathcal{U}(G) \longrightarrow A$, since all relations imposed to obtain $\mathcal{U}(G)$ must hold in $A$ if one writes $g_i$ for $\rho_i(g)$.
This will prove that $\mathcal{X}(G) \cong \mathcal{U}(G)$ and that both are universal representations.

We will prove that $\mathcal{X}(G)$ is a universal representation by proving that the elements, in which the order of the products respects the orders we fixed for $B_1$ and $B_2$,
$$ B_{f,g} = \prod_{b \in B_1} b_{f(b)} \prod_{b \in B_2} b_{g(b)},$$
with $f: B_1 \longrightarrow \mathbb{N}$, $g : B_2 \longrightarrow \mathbb{N}$ so that $\sup(f) = \{ b \in B_1 | f(b) \neq 0\}$ and $\sup(g) =\{ b \in B_2 | g(b) \neq 0\}$ are finite sets, form a basis of $\mathcal{X}(G)$  utilizing the universal enveloping algebra of Lie algebras over fields of characteristic $0$.

\begin{lemma}
	The module 
	$$ M = \langle B_{f,g} \rangle \subset F,$$
	is precisely the submodule of $F$ generated by monomials containing no expressions that are a left-hand side of a relation imposed on $F$ to obtain $\mathcal{X}(G)$. Furthermore, any element of $F$ is equivalent to an element of $M$ after applying a finite amount of relations.
	\begin{proof}
		We note that the elements $B_{f,g}$ are precisely the generating monomials of $F$ such that (1) no $b \in B_1 \cup B_2$ occurs multiple times as a $b_i$ in $B_{f,g}$ and (2) there are no $b,c \in B_1 \cup B_2$ which occur as $b_i, c_j$ which are not the same order as the order on $B_1 \cup B_2$, which is the order of $B_1$ and $B_2$ extended by $B_1 < B_2$. We note that the first two left-hand sides of relations correspond exactly with possible violations of (1) and that the last three left-hand correspond to possible violations of (2). This proves the first part of the lemma.
		
		Now, we prove the furthermore-part.
		First, we will associate a pair of natural numbers $(a,b)$ to a generating monomial which counts in a certain way to what degree the monomial violates (1) and (2).
		Second, we will prove that applying a relation (in the context of the lemma that is substituting a right-hand side for a left-hand side of a relation) to such a monomial creates a sum of monomials associated to pairs $(c,d) < (a,b)$ under the lexicographic order on $\mathbb{N}^2$.
		This will allow us to conclude that the furthermore part holds as those pairs $(a,b)$ are well-ordered, i.e., there cannot exist an infinite decreasing sequence.
		
		With a monomial $m = m_1\ldots m_n$ we associate the pair $(a,b)$ where $a$ counts the degree to which (1) and (2) are violated with at least one $b \in B_1$: $$a = |\{(i,j) | 1 \le i \le j \le n, m_i \text{ corresponds to } b \in B_1, m_j \text{ corresponds to } c < b \}|,$$
		while $b$ is the total degree in which (1) and (2) are violated:
		$$ b = |\{(i,j) | 1 \le i \le j \le n, m_i \text{ corresponds to } b, m_j \text{ corresponds to } c < b \}|.$$
		
		Observe that applying relations $1,2,4,5$ decreases $a + b$ while preserving $a$ or $b$. Applying relation $3$ decreases $a$. So, $(a,b)$ always decreases.
		Using the earlier remarked fact that the pairs $(a,b)$ are well-ordered, we can conclude that the lemma holds.
	\end{proof}
\end{lemma}

\begin{lemma}
	\label{lemma universal representation free over Z}
	Suppose that $G$ is a free vector group over $\mathbb{Z}$, the algebras $\mathcal{X}(G)$ and $\mathcal{U}(G)$ are isomorphic and are universal representations of $G$.
	\begin{proof}
		Consider the universal enveloping algebra $U$ of $\text{Lie}(G) \otimes \mathbb{Q}$. We know that $$\exp_{[t]} : \text{Lie}(G) \otimes \mathbb{Q} \longrightarrow U : (g,h) \mapsto \exp(tg,t^2h) = 1 + tg + t^2 (g^2/2 + h) + \ldots $$ is a vector group representation of $\text{Lie}(G) \otimes \mathbb{Q} \cong G(\mathbb{Q})$. This induces a vector group representation $$(\rho_i: G \longrightarrow U)_i.$$
		
		This means that there exists a map $$\mathcal{X}(G) \longrightarrow \mathcal{U}(G) \longrightarrow U,$$ using the earlier described map $\mathcal{X}(G) \longrightarrow \mathcal{U}(G)$ and the fact that there exists for each vector group representation a map with domain $\mathcal{U}(G)$ corresponding to that representation. Using the Poincarré-Birkhoff-Witt basis, we can conclude that this map is injective since $M$ embeds into $U$ and since each element of $\mathcal{X}(G)$ is contained in $M$ modulo the relations.
		Specifically, a generating monomial $B_{f,g} \in M$ is associated to a pair of functions $(f,g)$.
		Using the lexicographical order on $\mathbb{N}^{B_1 \cup B_2}_{\text{fin sup}}$, we can order these $B_{f,g}$.
		Similarly, one one orders the (slightly modified) Poincarré-Birkhoff-Witt basis elements $$B'_{f,g} = \prod_{b \in B_1} b^{f(b)}/(f(b)!) \prod_{b \in B_2} b^{g(b)}/(g(b)!).$$
		Since $b < c$ for $b \in B_1,c \in B_2$, one can check that $B_{f,g} \mapsto B'_{f,g} + \sum_{(f,g) < (k,l)} c_{f,g,k,l} B'_{k,l}$ for certain $c_{f,g,k,l}$ using the relations involving binomial coefficients on $\mathcal{X}(G)$, and $b_i \mapsto b_i$ for $b_i \in B_i$ so that
		$$ (b_i)_{ni} \mapsto (b_i)^{n}/(n!) \mod \text{ ($B'_{f,g}$ with bigger $(f,g)$ than $(b_i)_{n_i}$)}$$
		for $b \in B_i$.
		The representation $(\rho_i : G \longrightarrow U)_i$ maps to the embedding of $\mathcal{X}(G)$ in $U$.
		Hence, $$(\rho_i: G \longrightarrow \mathcal{X}(G))_i$$ is a vector group representation. 
		As exposed earlier, this proves that $\mathcal{X}(G) \cong \mathcal{U}(G)$ is the universal representation. Specifically, we have a unique map from $\mathcal{U}(G)$ corresponding to the vector group representation $(\rho_i)_i$,  we have a map $\mathcal{X}(G) \longrightarrow \mathcal{U}(G)$ corresponding to the fact that all relations on $\mathcal{X}(G)$ also hold on $\mathcal{U}(G)$, and these maps interact nicely as they send $b_i \mapsto b_i$ for $b \in B_1 \cup B_2, i \in \mathbb{N}$.
	\end{proof}
\end{lemma}

\begin{theorem}
	Let $G$ be a vector group. There exists a universal vector group representation $$(\gamma_i : G \longrightarrow \mathcal{U}(G))_i.$$
	\begin{proof}
		For free vector groups $G$ over $\mathbb{Z}$, this is Lemma \ref{lemma universal representation free over Z}.
		For arbitrary $G$, this now follows from Lemma \ref{lemma free universal implies universal}.
	\end{proof}
\end{theorem}

\begin{corollary}
	\textbf{VecGrp} is closed under quotients, i.e., suppose that $K \le G$ are vector groups with $K$ normal in $G$, then $L \longrightarrow G(L)/K(L)$ corresponds to a vector group as well.
	\begin{proof}
		Consider $A = \mathcal{U}(G)/I$ with $I$ the ideal generated by $\mathcal{U}(K) \cap \ker \epsilon$ with $\epsilon$ the counit of the Hopf algebra. The universal representation $(\gamma_i: G \longrightarrow \mathcal{U}(G))_i$ induces a representation $(\rho_i: G/K \longrightarrow A)$. The representation $(\rho_i)_i$ is injective since it is injective on 
		$ \text{Lie}(G)/\text{Lie}(K)$ (since $K$ is normal).
		
		So, this means that $G/K$ can be given a vector group structure in $A \times A$ as in Remark \ref{Remark vecgroup repres}
	\end{proof}
\end{corollary}

\begin{theorem}
	\label{theorem universal homogeneous}
	Suppose that $\rho : G \longrightarrow M$ is a homogeneous map of degree $n$, then it factors uniquely through the $n$-th grading component of $\mathcal{U}(G)$.
	Hence, the mapping $g \longrightarrow g_n$ to $\mathcal{U}(G)_n$ defines the universal homogeneous map of degree $n$.
	\begin{proof}
		We proceed by induction.
		For $n = 0$, this is trivial as every homogeneous map $f$ of degree $0$ is constant, so it uniquely factors through $\Phi = \mathcal{U}(G)_0$, mapping $1$ to $f(G)$.
		
		Let $f : G \longrightarrow M$ be a homogeneous map of degree $n$.
		We prove by induction on $n$ that there exists a representation	
		$$ \rho_{[t]}(g) = 1 + tg_1 + \ldots t^{n-1}g_{n-1} + t^nf(g)$$	
		 of $G$ in an algebra $A = \mathcal{U}(G)_{<n} \oplus M.$
		From this, we can obtain an algebra map $$\mathcal{U}(G) \longrightarrow A$$ using the universality of $\mathcal{U}(G)$ as a vector group representation. The restriction $\mathcal{U}(G)_n \longrightarrow M$ will map $g_n \mapsto f(g)$ which will prove the theorem.
		
		So, consider a homogeneous map $f : G \longrightarrow M$ of degree $n$ with linearisations $f^{i,j} : G \times G \longrightarrow M$, which inductively correspond to linear maps
		$g^{i,j}: \mathcal{U}(G)_i \otimes \mathcal{U}(G)_j \longrightarrow M.$
		These maps are associative in the following sense $g^{(i,j),k}(x,y,z) = g^{i,(j,k)}(x,y,z)$ where both expressions are $(i,j,k)$ linearisations of $f$ obtained by linearizing either $g^{i+j,k}$ or $g^{i,j+k}$, since both correspond to a certain term belonging to the coefficient $a^ib^jc^k$ in
		$$ f((a \cdot_1 x)(b \cdot_1 y)(c \cdot_1 z)).$$
		We endow $A$ with the restriction of the product of $\mathcal{U}(G)$ to $\mathcal{U}{(G)}_{<n}$ (where we throw away the result if it has too high a grading), and with the obvious left and right action of $\Phi$ on $M$, while using 
		$g^{i,j}$ for the rest of the products $\mathcal{U}(G)_i \otimes \mathcal{U}(G)_j \longrightarrow M$. Note that $A$ algebra is associative because the associativity of the linearisations.
		
		It is easy to see that 
		$$ \rho_{[t]}(g) = 1 + tg_1 + \ldots + t^{n-1}g_{n - 1} + t^n f(g),$$
		forms a vector group representation.
		As argued earlier, this proves that $g \longmapsto g_n$ is the universal homogeneous map of degree $n$. 
	\end{proof}
\end{theorem}

\begin{remark}The following equality for homogeneous maps $f$ degree $3$ will be useful:
	\begin{equation}
		\label{equation homogeneous of degree 3}
		3(f(g) - f^{(1,2)}({g,g})) + f^{(1,1,1)}({g,g,g}) = 0.
	\end{equation}
	This equality can be obtained from the equality
	$$ 3g_3 - 3g_1g_2 + g_1^3 = 3g_3 - 2g_2g_1 + g_1(g^{-1})_2 = 0$$
	which holds in the universal representation. 
	This can either be proved using the relations on $\mathcal{U}(G)$, or one can use the following computation:
	$$ \epsilon(3g_3 - 2g_2g_1 + g_1(g^{-1})_2) = (((1 + \epsilon) \cdot_1 g)g^{-1})_3 = (\epsilon g_1, 2 \epsilon g_2 - \epsilon \psi(g_1,g_1))_3 = 0$$ 
	over the dual numbers $\Phi[\epsilon]$.
\end{remark}

\subsection{The primitive elements of the universal representation}


\begin{definition}	Let $H$ be a Hopf algebra, we use $P(H)$ to denote the set of \textit{primitive} elements, i.e., the $x \in H$ such that 
		$$ \Delta(x) = x \otimes 1 + 1 \otimes x.$$
		We call an element $x \in H$ \textit{group-like} if $\Delta(x) = x \otimes x$ and $\epsilon(x) = 1$.
\end{definition}


We want to prove that $P(\mathcal{U}(G)) = \text{Lie}(G)$ for projective $G$ and be able to recover $G(K)$ from $\mathcal{U}(G) \otimes K$.
In order to achieve that, we will consider concrete filtrations of $\mathcal{U}(G)$ which will coincide for projective vector groups $G$. These filtrations are chosen in such a way that we know the primitives of $\mathcal{U}(G)$ coincide with $\text{Lie}(G)$ if the first $2$ parts of the filtration are the same. Although we will not explicitly prove it, the question of whether these two filtrations coincide only depends on the module $\text{Lie}(G)$.

\begin{definition}
	Consider the space
	\[ Y_m = \left\langle  \prod_{i = 1}^{k_1} (v_{1i})_{m_{1i}} \prod_{j = 1}^{k_2} (v_{2j})_{2m_{2j}}| v_{1i} \in G, m_{2j} \in G_2, \sum_{i=1}^{k_1} m_{1,i} + \sum_{j = 2}^{k_2} v_{2,j} \le m, k_i \in \mathbb{N} \right\rangle.\]
	We also define
	\[ Z_m =  \ker (\text{Id} - \epsilon)^{\otimes m + 1} \Delta^{m},\]
	writing $\text{Id} - \epsilon$ for $\text{Id}- \eta \circ \epsilon$ with $\eta : \Phi \longrightarrow \mathcal{U}(G)$ the structure morphism and using the operator  $\Delta^{m+1} = (\Delta^{m} \otimes \text{Id}) \circ \Delta$ with $\Delta^1 = \Delta$. 
	The coassociativity guarantees that $\Delta^{m+1} = (\Delta^{i} \otimes \Delta^{j}) \circ \Delta$ for all $i + j = m$.
	We will call vector groups $G$ with universal representation $\mathcal{U}(G)$ \textit{well-behaved} if $Y_m = Z_m$ for all $m$.
\end{definition}

\begin{remark}
	By construction the spaces $Y_i$ satisfy
	\[ \Phi = Y_0 \subsetneq Y_1 \subsetneq Y_2 \subsetneq \dots \]
	and $\bigcup_m Y_m = \mathcal{U}(G)$.
\end{remark}

\begin{lemma}
	The spaces $Z_i$ satisfy $Z_i \subset Z_{i + 1}$ for all $i$. 
	\begin{proof}
		We must prove that
		\[ \ker (\text{Id} - \epsilon)^{\otimes m + 1} \Delta^{m} \subset \ker (\text{Id} - \epsilon)^{\otimes m + 2} \Delta^{m + 1}.\]
		For $p \in \mathcal{U}(G)$, we will use that
		$$ (\text{Id} - \epsilon)^{\otimes 2} \Delta(p) = \Delta((\text{Id} - \epsilon)(p)) - (1 - \epsilon)p \otimes 1 - 1 \otimes (1 - \epsilon)p = (\Delta - I_1 - I_2)(1 - \epsilon)(p),$$
		with $I_1(x) = x \otimes 1$ and $I_2(x) = 1 \otimes x$.
		This last equation is easily checked for $p$ such that $p = \epsilon(p)$, while one can use the $\mathbb{N}$-grading for $p$ such that $\epsilon(p) = 0$ to prove that $\Delta(p) = p \otimes 1 + 1 \otimes p + (\text{Id} - \epsilon)^{\otimes 2} \Delta(p)$.
		Hence, we obtain
		\begin{align*}
			(\text{Id} - \epsilon)^{\otimes m + 2} \Delta^{m + 1} (x) = & ((\Delta  - I_1 - I_2) \otimes \text{Id}^{\otimes m}) (\text{Id} - \epsilon)^{\otimes m + 1} \Delta^m (x).
		\end{align*} This proves the lemma.
	\end{proof}
\end{lemma}

\begin{lemma}
	\label{lemma filtration}
	Suppose that $G$ is a vector group such that $$Z_{m-1} \cap Y_m = Y_{m-1},$$ then $G$ is a well-behaved vector group.
	\begin{proof}
		For each $m$, induction on $m - k$ shows that
		$$Z_k \cap Y_m = Y_{k},$$
		for $k \le m$, using $Z_k \subset Z_{k+1}$.
		This proves the lemma since $\bigcup_m Y_m = \mathcal{U}(G)$.
	\end{proof}
\end{lemma}

Consider $$T_{n_1,\ldots,n_k} = \{ a \in \{1,\ldots,k\}^n | i \text{ appears } n_i \text{ times in } a\}. $$
We consider a set of coset representatives $S_{n_1,\ldots,n_k} \subset S_{\sum n_i}$ corresponding to $S_{\sum n_i}/S_{n_1} \times \ldots \times S_{n_k}$, note that this set maps bijectively to $T_{n_1,\ldots,n_k}$ using permutation action of $S_m$ on the element $$(1,\ldots,1,2,\ldots,2,\ldots,k,\ldots,k)$$ formed by taking $n_i$ times $i$ in for each $i \in \{1,\ldots,k\}$.

\begin{lemma}
	\label{Lemma well behaved}
	Each projective vector group is well-behaved.
	\begin{proof}
		Let $G$ be a projective vector group.
		Consider dual bases $(b_i^*)_{i \in I}$ and $(b_i)_{i \in I}$ of $G/G_2$ and $G_2$ so that each $v \in \text{Lie}(G)$ equals
		$$ v = \sum_{i \in I} b_i^*(v)b_i,$$
		with finitely $b_i^*(v) \neq 0$.
		Suppose that $I$ is totally ordered.
		
		We define a map $\text{Im}(\text{Id} - \epsilon)^{\otimes m} \Delta^{m-1}_{|Y_m} \longrightarrow Y_m/Y_{m-1}$ inverse to the map $Y_m/Y_{m-1} \longrightarrow \text{Im}(\text{Id} - \epsilon)^{\otimes m} \Delta^{m-1}_{|Y_m}$ induced by the map $(\text{Id} - \epsilon)^{\otimes m} \Delta^{m-1}_{|Y_m}$ defined on $Y_m$. 
		It is sufficient to construct a map $f$ which  maps as
		$$ \sum_{\sigma \in S_{n_1,\ldots,n_k}} \sigma \cdot b_{i_1} \otimes \cdots \otimes b_{i_1} \otimes \cdots \otimes b_{i_k} \otimes \cdots \otimes b_{i_k} \longmapsto (b_{i_1})_{n_1}\cdots(b_{i_k})_{n_k} \mod Y_{m-1},$$
		on linear generators 
		$$\sum_{\sigma \in S_{n_1,\ldots,n_k}} \sigma \cdot b_{i_1} \otimes \cdots \otimes b_{i_1} \otimes \cdots \otimes b_{i_k} \otimes \cdots \otimes b_{i_k}$$
		of $\text{Im}(\text{Id} - \epsilon)^{\otimes m} \Delta^{m-1}_{|Y_m}$ in which $n_i$ denotes the amount of times $b_i$ appears. To avoid redundant generators we suppose that $i_1,\ldots,i_k \in I$ form a strictly increasing sequence.
		We denote the values we want for $f$ on the set of generators as $f(b_{i_1},n_1,\ldots,b_{i_k},n_k)$.
		For arbitrary elements of $\Ima (\text{Id} - \epsilon)^{\otimes m} \Delta^{m-1}_{|Y_m}$ we define $f$ using
		$$ x \mapsto \sum_{\text{generators}} f(b_{i_1},n_1,\ldots,b_{i_k},n_k) (b_{i_1}^*)^{\otimes n_1} \otimes \ldots \otimes (b_{i_k}^*)^{\otimes n_k} (x) \mod Y_{m-1}.$$
		We remark, to justify the dual maps in the previous definition, that we can use the $b^*_i$ on $\mathcal{U}(G)$. For $b_i \in \text{Lie}(G)_1$ this is obvious, since we have a projection operator $\mathcal{U}(G) \longrightarrow \mathcal{U}(G)_1 \cong \text{Lie}(G)_1$. 
		If $b_i \in \text{Lie}(G)_2$, this is a bit trickier to see. We can use $\pi : \mathcal{U}(G)_2 \longrightarrow \text{Lie}(G)_2 : x \mapsto x - \hat{f} \circ (\text{Id} - \epsilon)^{\otimes 2} \circ \Delta^2 (x)$, with $\hat{f}$ the map with the same definition of $f$, except that it maps to $Y_2$ instead of $Y_2/Y_1$. The image of $\pi$ is $\text{Lie}(G)_2$ since it acts on linear generators as \[v_2 \mapsto v_2 - q(v) = ((v_1,v_2) \cdot (-v_1, - q(v) + v_1^2))_2\] and \[a_1b_1 \mapsto a_1b_1 - q(a,b) = ((a_1,q(a)) \cdot (b_1,q(b)) \cdot (-(a_1 + b_1),-q(a + b)+(a_1+b_1)^2))_2,\] with $q : \text{Lie}(G)_1 \longrightarrow \mathcal{U}(G)_2 : v \mapsto \hat{f}(v_1 \otimes v_1)$ a quadratic map such that $(a_1,q(a))$ in the image of $G$ for all $a \in G$ under the universal representation and $q(a,b) = q(a + b) - q(a) - q(b)$ its linearisation.
		
		Now, we prove that $f$ acts as expected.
		Inductively applying the relations on $\mathcal{U}(G)$ yields $(b_i)_{j_in} \equiv  \prod_{b_j \in I} (b^*_j(b_i)b_j)_{j_in} \mod Y_{m-1}$. Thus, we conclude that the function acts as expected on $b_i \otimes \ldots \otimes b_i$.
		Furthermore, one sees that
		\begin{align}
			\nonumber 
			& \left(\prod_i\binom{n_i + m_i}{n_i}\right)f(b_{i_1},n_1 + m_1, \ldots,b_{i_k},n_k + m_k) \\ \label{equation modulo} \equiv & f(b_{i_1},n_1, \ldots,b_{i_k},n_k)f(b_{i_1},m_1, \ldots,b_{i_k},m_k) \mod Y_{m-1}
		\end{align} using the relations on $\mathcal{U}(G)$.  
		Note that 
		$$ (1 - \epsilon)^{\otimes m+l} \Delta^{m+l-1}(ab) = \sum_{\sigma \in S_{l,m}} \sigma \cdot (1 - \epsilon)^{\otimes l}\Delta^{l-1}(a) \otimes (1 - \epsilon)^{\otimes m}\Delta^{m-1}(b)$$
		for elements $a \in Y_l, b \in Y_m$.
		A single term
		$$f(b_{i_1},n_1,\ldots,b_{i_k},n_k) (b_{i_1}^*)^{\otimes n_1} \otimes \ldots \otimes (b_{i_k}^*)^{\otimes n_k}(1 - \epsilon)^{\otimes k + l}\Delta^{k + l - 1}(ab)$$
	is computed by considering all possible sums $\sum_{i = 1}^k o_j = l$ with $o_j \le n_j$ and for each such sum considering all possible ways to choose of $o_j$ of the $b^*_{i_j}$ and evaluating the chosen $b^*$ on $(1 - \epsilon)^{\otimes l}\Delta^{l-1}(a)$ while evaluating the remaining $n_j - o_j$ of the $b^*_{i_j}$ on $(1 - \epsilon)^{\otimes m}\Delta^{m-1}(b)$. This yields the same result as directly evaluating
	$$ \sum_{\substack{\sum_j o_j = l\\ \sum_j p_j = m\\ o_j + p_j = n_j}} f(b_{i_1},o_1,\ldots)(b_{i_1}^*)^{\otimes o_1} \otimes \ldots \otimes (b_{i_k}^*)^{\otimes o_k} \bigotimes f(b_{i_1},p_1,\ldots)(b_{i_1}^*)^{\otimes p_1} \otimes \ldots \otimes (b_{i_k}^*)^{\otimes p_k}$$
	on 
	$$ (1 - \epsilon)^{\otimes l}\Delta^{l-1}(a) \otimes (1 - \epsilon)^{\otimes m}\Delta^{m-1}(b)$$
	by Equation (\ref{equation modulo}).
	This proves that $$f(1 - \epsilon)^{\otimes m + l}\Delta^{m + l - 1}(ab) \equiv f(1 - \epsilon)^{\otimes l}\Delta^{l-1}(a) f(1 - \epsilon)^{\otimes m}\Delta^{m-1}(b) \mod Y_{m + l - 1}$$
	for $a \in Y_{l}$ and $b \in Y_{m}$.
	So, $f$ takes the right values on all generators since $Y_nY_m \subset Y_{n+m}$ and since $Y_{n+m}$ is generated by all $Y_nY_m, nm \neq 0$ and the $g_{n+m}, h_{2(n+m)}$ for $g \in G, h \in G_2$. This proves that $$Y_{m-1} = \ker (1 - \epsilon)^{\otimes m} \Delta^{m-1} \cap Y_m.$$
		  Therefore, Lemma \ref{lemma filtration} shows $G$ to be well-behaved.
	\end{proof}
\end{lemma}

\begin{lemma}
	\label{Lemma Y1}
	 $Y_1 \cap \ker \epsilon \cong \Lie(G)$.
	 \begin{proof}
	 	We know that $\Lie(G) \cong \langle g_1, h_2 | g \in G, h \in G_2\rangle$ and that $Y_1$ is generated by these elements and the $g_0$, which are equal to $1$.
	 \end{proof}
\end{lemma}

\begin{lemma}
	\label{lemma Z1}
	$Z_1 \cap \ker \epsilon \cong \text{P}(\mathcal{U}(G))$
	\begin{proof}
		We know that $P(\mathcal{U}(G)) \subset \ker \epsilon$.
		For $x \in \ker \epsilon$ we know that $x$ is primitive if and only if
		\begin{align*}
			& \Delta(x) - x \otimes 1 - 1 \otimes x = 0 \\
			\iff&  \Delta(x) - (\text{Id}\otimes \epsilon) \Delta(x) - (\epsilon \otimes \text{Id}) \Delta(x) + (\epsilon \otimes \epsilon ) \Delta(x) = 0 \\
			\iff & (\text{Id}- \epsilon)^{\otimes 2} \Delta(x) = 0 \\
			\iff&  x \in Z_1,
		\end{align*}
	using that $\mu(\epsilon \otimes \text{Id}) \Delta = \text{Id} = \mu(\text{Id} \otimes \epsilon ) \Delta$ and $\epsilon^{\otimes 2} \Delta = \Delta \epsilon$.
	\end{proof}
\end{lemma}

\begin{theorem}
	\label{theorem PUG}
	If $G$ is a projective vector group, then $P(\mathcal{U}(G))$ is isomorphic to $\Lie(G)$. Consequently, the group $G(K)$ can be recovered from the universal representation, using
	$$ \{ g = 1 + tg_1 + t^2 g_2 + \ldots \in (\mathcal{U}(G) \otimes K)[[t]] \; | \; \Delta(g) = g \otimes g, g_i \in \mathcal{U}(G)_i \otimes K\} \cong G(K).$$
	
	
	\begin{proof}
		Lemmas \ref{Lemma well behaved}, \ref{Lemma Y1}, and \ref{lemma Z1} prove that $P(\mathcal{U}(G)) \cong \text{Lie}(G)$.
		
		 Now, we prove that each group-like element 
		 $$ g = \sum_{i = 1}^\infty t^ig_i,$$
		 with $g_i \in \mathcal{U}(G)_i \otimes K$ is contained in the image of $G(K)$ under the universal representation.
		 Remark that the first nonzero $g_i$ must be primitive, so that the first nonzero $g_i$ must be contained in $\text{Lie}(G) \otimes K$.
		 We can use this to write
		 $$ g = h_1h_2,$$
		 with $h_1 \in G(K), \; h_2 \in G_2(K)$ such that $h_1$ and $g$ share the first coordinate and $h_2$ and $h_1^{-1}g$ share the second coordinate. 
		\end{proof}
\end{theorem}

\begin{remark}
	In this section we defined vector groups using
	$$ G \le A_1 \times \ldots \times A_n,$$
	with $n = 2$ since these are the only vector groups we need in this article.
	It is possible to generalize what we have done here and give an exact way to construct group functors $K \mapsto G(K)$ from $G(\Phi)$ alone that are equivalent as a class of groups to groups of formal power series $ K \mapsto \{1 + tg_1 + t^2g_2 + \ldots | g \in G(K) \}$ closed under $$1 + t^ig_i + t^{2i}g_{2i} +  \ldots \mapsto 1 + \lambda t^i g_i + \lambda^2 t^{2i} g_{2i}\ldots$$ satisfying
	$$ G_i/G_{i+1} \otimes K \cong G_i(K)/G_{i+1}(K),$$
	with $G_i(K) = \{ g \in G(K) | g_{k} = 0 , k = 1,\ldots,i-1\}$ and for which there exists $i$ such that $G_i = 0$ (this restriction can be lifted by considering direct limits). 
	In that case, we have $\text{Lie}(G) = G/G_2 \times G_2/G_3 \times \ldots.$
\end{remark}