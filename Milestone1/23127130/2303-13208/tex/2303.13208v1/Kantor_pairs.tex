\section{Operator Kantor pairs}\label{se:okpairs}
	
	\subsection{Pre-Kantor pairs}
	
	Let $G^+ \le A^+ \times B^+$ and $G^- \le A^- \times B^-$ be a pair of vector groups with associated bilinear forms $\psi : A \times A \longrightarrow B$ (we will not write uninformative $\pm$ signs). For  ease of notation, we assume that 
	$$ \text{Lie}(G^+) = A^+ \oplus G^+_2,$$
	i.e., we assume that $A^+$ is as small as possible (one can always modify $A^+ \times B^+$ so that this is the case). We assume the same for $G^-$ and $A^-$.
	
	
	\begin{lemma}
		Let $G$ be a vector group.
		The space $[G,G] \le G_2$ is a $\Phi$-submodule.
		\begin{proof}
			By definition $[G,G] \le G_2$ is the subgroup generated by all the commutators of $G$, so it is an additive subspace.
			We prove that it is closed under $\cdot_2$ as well.
			This is the case, since the equation
			$$ \lambda \cdot_2 [a,b] = [\lambda \cdot_1 a, b]$$
			holds for all $a, b \in G$.
		\end{proof}
	\end{lemma}

Let $M$ be a module. We call a natural transformation $f : G^+ \times G^- \longrightarrow M$ homogeneous of \textit{bidegree} $[i,j]$ if it is homogeneous of degree $i$ in the first component and of degree $j$ in the second component.
We often use $f_x y$ to denote $f(x,y)$ for a map that is homogeneous in both arguments.



\begin{assumption}
Throughout this section we will be working with certain operators $Q,R,T,P, Q^\text{grp}$ defined for $G^+,G^-$. We assume, in this section, that these are of the following form: 
\begin{itemize}
	\item $ Q^{\text{grp}} : G^+ \times G^- \longrightarrow G^+ : (g,h) \mapsto Q^{\text{grp}}_gh = (Q_g h, R_g h)$
	with $Q$ of bidegree $[2,1]$ and $R$ of bidegree $[4,2]$,
	\item $ T : G^+ \times G^- \longrightarrow [G^+,G^+] : (g,h) \mapsto (0,T_g h)$
	of bidegree $[3,1]$,
	\item $ P : G^+ \times G^- \longrightarrow A^+: (g,h) \mapsto P_g h,$ 
	of bidegree $[3,2]$.
\end{itemize}
We also use $Q,R,T,P,Q^\text{grp}$ to denote operators $G^-\times G^+ \longrightarrow M$ for the appropriate $M$.
These operators $Q^\text{grp},T$ will in a precise sense play the role of $o_{2,1}$ and $o_{3,1}$ of Theorem \ref{thm main} and $P$ will play the role of $\nu_{3,2}.$
\end{assumption}


\begin{definition}
	Let $C$ be an associative algebra with involution $a \mapsto \overline{a}$ and consider a left $C$-module $M$. We call a $\Phi$-bilinear map $h : M \times M \longrightarrow C$ a (left-)Hermitian form if it satisfies $h(cm,n) = ch(m,n)$ and $\overline{h(m,n)} = h(n,m)$.
\end{definition}

\begin{example}
	\label{example hermitian form}
	Now, we will consider an example of such a set of operators that will form an operator Kantor pair. This will be proved after we define operator Kantor pairs.
	Consider a Hermitian form $h : M \times M \longrightarrow C$ and assume that $C$ acts faithfully on $M$.
	The Kantor triple system $(M,V_{x,y})$, with $V_{x,y} : M \longrightarrow M$ for $x,y \in M$ (for a definition of a Kantor triple system, see \cite[section 3]{ALLFLK99}), associated to this Hermitian form is given by
	$$ V_{x,y} z = h(x,y)z + h(z,y)x - h(z,x)y.$$
	
	
	Set $G = \{ (m,a) \in M \times C | a + \bar{a} = h(m,m), a - \bar{a} \in \langle  h(x,y) - h(y,x) |x,y \in M \rangle\}$ with operation $$(m,a)(n,b) = (m + n, a + b +h(m,n)).$$  
	Define 
	\begin{enumerate}
		\item $Q_{(m,a)}(n) = - h(m,n)m + an,$
		\item $T_{(m,a)}(n) =  h(m,an) -h(an,m),$
		\item $P_{(m,a)}(n,b) =  - h(m,n)Q_{(m,a)}(n) - abm,$
		\item $R_{(m,a)}(n,b) = ab\bar{a} + h(m,n)ah(n,m) - h(m,n)h(m,an).$
	\end{enumerate}
	A direct computation shows that $Q^{\text{grp}}_g h = (Q_g h,R_g h)$ is an operator mapping to $G$. At the end of this section we will prove that $(G,G,Q^\text{grp},T,P)$ forms an operator Kantor pair if $1/2 \in \Phi$ or if
	\begin{equation}
		\label{condition kantor pair hermitian form}
		\langle c - \bar{c} | (a,c) \in G \rangle \subseteq [G,G].
	\end{equation}
One can recover the original Kantor triple system using
	$$ V_{x,y} z = Q^{(1,1)}_{z,x} y,$$
	with $Q^{(1,1)}$ the $(1,1)$-linearisation of $Q$.
\end{example}


\begin{definition}
	\label{definition V tau}
	We want to define certain new operators $V, \tau : G^+ \times G^- \longrightarrow \text{End}(G^+) \times \text{End}(G^-)$, and also define them on $G^- \times G^+$.
	In this definition we write $V^+_{g,h}, V^-_{g,h}$ for the projections of $V_{g,h}$ as endomorphisms of $G^+, G^-$ (outside the scope of this definition, we will just write $V$). We do the same for $\tau$.
	Use $f^{(i,j)}$ to denote the $(i,j)$-linearisation $f$ to the first component.
	For $x,g \in G^\epsilon$ and $y \in G^{-\epsilon}$, $\epsilon \in \{\pm\}$, we put
	$$ V^\epsilon_{y^{-1},x^{-1}} g = (Q^{(1,1)}_{g,x} y, T^{(2,1)}_{g,x} y + \psi(Q^{(1,1)}_{g,x} y,g))$$
	and 
	$$ \tau^\epsilon_{y^{-1},x^{-1}} g = (P^{(1,2)}_{g,x} y, R^{(2,2)}_{g,x} y - \psi(Q_g y,Q_x y) + \psi(P^{(1,2)}_{g,x} y, g)).$$	
		So far we have defined $V^{\epsilon},\tau^{\epsilon}$ on $G^{-\epsilon} \times G^\epsilon$. 
		Now, we want to define $V^{\epsilon}, \tau^{\epsilon}$ when these take arguments in $G^{\epsilon} \times G^{-\epsilon}$.
	
By working over $\Phi[\eta]$ with $\eta^3 = 0$ and taking elements in $G^\pm \le G^\pm(\Phi[\eta])$ under the canonical embedding, we can rewrite the definition of $V^+,\tau^+$ as
	\begin{align*}
		& \left(1 + \eta V^{+}_{y^{-1},x^{-1}} + \eta^2 \tau^{+}_{y^{-1},x^{-1}}\right)(g) = \left(\eta Q^{(1,1)}_{g,x} y + \eta^2P^{(1,2)}_{g,x} y,  \eta T^{(2,1)}_{g,x} y + \eta^2 R^{(2,2)}_{g,x} y - \eta^2\psi(Q_gy,Q_xy)\right) \cdot g,
	\end{align*}
	for $x,g \in G^+$, $y \in G^-$.
	

	We use $$(1 + \eta V^\epsilon_{x,y} + \eta^2 \tau^\epsilon_{x,y})(1 + \eta V^\epsilon_{y^{-1},x^{-1}} + \eta^2 \tau^\epsilon_{y^{-1},x^{-1}}) = 1$$
	to define $V^\epsilon_{x,y}$ and $\tau^{\epsilon}_{x,y}$ from the already defined $V^\epsilon_{y^{-1},x^{-1}}, \tau^\epsilon_{y^{-1},x^{-1}}$.
\end{definition}

\begin{remark}
	Note that the definition of $V$ and $\tau$ is not that easy to give, although it could be easier for $V$ since $V_{x,y} = - V_{y^{-1},x^{-1}}$. We extended the definition using
	$$(1 + \eta V_{x,y} + \eta^2 \tau_{x,y})(1 + \eta V_{y^{-1},x^{-1}} + \eta^2 \tau_{y^{-1},x^{-1}}) = 1,$$
	which encodes the relationship between the very difficult to define operator $\tau$ and the more easily defined $V$.
	This relationship corresponds to the fact that $(G^+,G^-) \longrightarrow (G^-,G^+):(x,y) \longmapsto (y^{-1},x^{-1})$ corresponds to $o_{i,j}(x,y) \mapsto o_{j,i}(y^{-1},x^{-1})^{-1}$ for any pair of vector group representations.
\end{remark}

\begin{remark}
	If one were to use $Q = \nu_{2,1}, T = \nu_{3,1}, P = \nu_{3,2}, R = \nu_{4,2}$ for some vector group representations (later we shall see that the operators can  always be understood in this way), then one can check that the definition of $V_{y^{-1},x^{-1}}$ and $\tau_{y^{1},x^{-1}}$ is defined from the conjugation action of $\exp(o_{1,1}(y^{-1},x^{-1},\eta)).$
	We can uniquely recover the definition of $V$ from equations $2$ and $4$ of Lemma \ref{Lemma equations}, where one can think of the left-hand side of those equations as representing 
	$$ [V_{y^{-1},x^{-1}},\rho_i(g)].$$
	Similarly, one can uniquely recover the definition of $\tau$ from equations $3$ and $8.a$, where we gave expressions $\nu_{2,i}(o_{1,1}(y^{-1},x^{-1}),g)$ instead of the usual conjugation action $\mu_{2,i}(o_{1,1}(y^{-1},x^{-1}),g)$ (which is no problem since one can easily go back and forth between $\nu_{i,j}$'s and $\mu_{i,j}$'s).
\end{remark}

From now onward we do not make a distinction between homogeneous maps $f :G^\pm \longrightarrow M$ of degree $1$, and linear maps $A^\pm \longrightarrow M$.
In particular, we can write $V_{y,x} = V_{-y,-x} = V_{y^{-1},x^{-1}}$.
We use the same convention to identify the commutator on $\text{Lie}(G)$ and the commutator on $G$, using $[a,b] = [(a,\cdot),(b,\cdot)].$


\begin{definition}
	\begin{enumerate}
		\item Let $f : G^\pm \times G^\mp \longrightarrow M$ be a map defined on a product of vector groups which is homogeneous of degree $n$ in the first component and of degree $m$ in the second component. 
		We use $f^{(i,j)}$ to denote the linearisation of $f$ in the first component, $f^{(n,(i,j))}$ to denote the linearisation of $f$ in the second component, and $f^{((i,j),(k,l))}$ if we use linearisations in both components.
		The reason for this asymmetry in notation is that the first component in $f$ plays conceptually a more important role. The used notation $g \mapsto f_g$ can be interpreted as corresponding to a natural transformation $G^\pm \longrightarrow \text{Nat}(G^\mp,M)$.
		\item 
		Consider a pair $(G^+,G^-)$ of vector groups with the assumed operators $Q,T,P,R$.
		A pair of automorphisms $(a,b)$ of $G^+$ and $G^-$ is called an \textit{automorphism} if it preserves the operators, i.e.,
		$a(O_g h) = O_{ag} (bh), \quad b(O_hg) = O_{bh}(ag),$
		for $O = Q^{\text{grp}},T,P$ and $g \in G^+, h \in G^-$.  To say that $Q^\text{grp}$ is preserved can be restated as $a_1(Q_gh) = Q_{(ag)}(bh)$ and $a_2(Q^\text{grp}_gh) = R_{a(g)}(bh)$, if one writes $a(g) = (a_1(g),a_2(g))$.
	\end{enumerate}
\end{definition}
\begin{definition}
		\label{Definition pre-Kantor pair}
		Let $(G^+,G^-)$ be a pair of vector groups such that
		\begin{equation}
			\label{Kantor pair equation}
			g(-g) \in [G^\pm,G^\pm] \text{ for } g \in G^\pm.
		\end{equation}
We also assume that
		\begin{equation}
			\label{Kantor pair equation 2}
			G_2^\pm \longrightarrow \text{Hom}_\Phi(A^\mp,A^\pm): s \longmapsto Q_s \quad \text{is injective.}
		\end{equation}  
		We call $(G^+,G^-,Q^\text{grp}, T, P)$ a \textit{pre-Kantor} pair if
		\begin{enumerate}
			\item $1 + \epsilon V_{x,y}$ is an automorphism over $\Phi[\epsilon]$ with $\epsilon^2 = 0$
			\item \label{PKP LIN T} $T^{(1,2)}_{a,g} y = [a, Q_g y]$,
			\item \label{PKP LIN P} $P^{(2,1)}_{g,a} y = Q_gQ_{y^{-1}} a - V_{a,y} Q_g y$,
			\item $P^{(3,(1,1))}_g (y_1,y_2) = Q_{T_g y_1} y_2 - V_{g,y_2} Q_g y_1 $,
			\item $R^{(3,1)}_{g,a} y = T_g Q_{y^{-1}} a - V_{a,y} T_g y + \psi(Q_gy,Q^{(1,1)}_{g,a} y)$,
			\item $R^{(1,3)}_{a,g} y = [a, P_g y] + \psi(Q^{(1,1)}_{a,g} y,Q_g y)$,
			\item $R^{(4,(1,1))}_g(a,b) = - V_{g,b} T_g a + \psi(Q_g a,Q_g b)$.		
		\end{enumerate}
		We assume that these equalities hold over all scalar extensions.
\end{definition}

\begin{remark}
	If $1/2 \in \Phi$, then Condition (\ref{Kantor pair equation}) ensures that $G^\pm_2 = [G^\pm,G^\pm]$ since $$(0,2s) = (0,s)(0,s) = (0,s)(-1 \cdot_1 (0,s)) \in [G^\pm,G^\pm].$$
	Later we will see that this equation ensures the existence of a Lie algebra with grading element associated to a Kantor pair such that both $G^\pm$ are groups of exponentials (in the usual sense if $1/6 \in \Phi$ and in a more general sense if only $1/2 \in \Phi$).
	Conditions (\ref{Kantor pair equation}), (\ref{Kantor pair equation 2}) correspond thus, if $1/6 \in \Phi$, precisely to the conditions \cite[Proposition 7.5]{BENSMI03} for a Jordan-Kantor pair to come from a Kantor pair (using that the action of the Jordan part will be faithful if and only if $s \mapsto Q_s$ is injective on $G_2$).
\end{remark}

\begin{lemma}
	\label{lemma uniqueness P}
	Let $(G^+,G^-)$ be a pre-Kantor pair.
	Let $P'$ be a homogeneous map $G^+ \times G^- \longrightarrow A^+$ with the same linearisations as $P$, 
	then 
	$$ P_xy = P'_xy$$
	for all $x \in G^+, y \in G^-$.
	Furthermore, $P$ is determined uniquely by $Q$ and $T$.
	\begin{proof}
		Recall Theorem \ref{theorem universal homogeneous}, i.e.,   $g \mapsto g_n \in \mathcal{U}(G)_n$ is the universal homogeneous map of degree $n$, so that equalities in the universal representation induce equalities for all homogeneous maps.
		First, we use that $P$ is homogeneous of degree $3$ in the first argument, so that
		$$ 3 P_xy = 3 P^{(1,2)}_{x,x} y - P^{(1,1,1)}_{x,x,x}y,$$
		since
		$$ 3x_3 = 3x_1x_2 - x_1^3$$
		holds in the universal representation.
		Second, we use that $P$ is homogeneous of degree $2$ in the second component and that there exists $a_i,b_i$ such that $y(-y) = \sum_{i = 1}^n [a_i,b_i]$ by (\ref{Kantor pair equation}), to obtain
		$$ 2P_xy = P^{(3,(1,1))}_x(y,y) +  \sum_{i = 1}^n P^{(3,(1,1))}_x(a_i,b_i) - P^{(3,(1,1))}_x(b_i,a_i),$$
		since $2y_2 - y_1^2 = \sum_{i = 1}^n [a_i,b_i]$ in the universal representation.
		These two combined yield that
		$$ P_xy = (3 - 2)P_xy = 3 P^{(1,2)}_{x,x} y - P^{(1,1,1)}_{x,x,x}y - P^{(3,(1,1))}_x(y,y) - \sum_{i = 1}^n P^{(3,(1,1))}_x(a_i,b_i) - P^{(3,(1,1))}_x(b_i,a_i).$$
		Hence, $P$ is uniquely determined by its linearisations. So, we conclude that $P_xy = P'_xy$.
		For the furthermore part, note that the linearisations of $P$ given in the definition of a pre-Kantor pair are expressions only involving $Q, T,$ and linearisations of these operators ($V$ is defined in terms of these linearisations).
	\end{proof}
\end{lemma}

\begin{definition}
	Consider an automorphism $ 1 + \epsilon \delta = (1 + \epsilon \delta^+, 1 + \epsilon \delta^-)$ over $\Phi[\epsilon]$ with $\epsilon^2 = 0$, of a pre-Kantor pair. We call $\delta = (\delta^+,\delta^-)$ a \textit{derivation} of the pre-Kantor pair.  
	For derivations $\delta_1, \delta_2$ we define
	$[\delta_1,\delta_2]$ using
	$$ [1 + \epsilon_1 \delta_1, 1 + \epsilon_2 \delta_2] = 1 + \epsilon_1\epsilon_2 [\delta_1,\delta_2],$$
	over $\Phi[\epsilon_1,\epsilon_2]$ with $\epsilon_i^2 = 0$ for $i = 1,2$.
	On the universal representations of the vector groups $G^+, G^-$, we have
	$$ [\delta_1,\delta_2] = \delta_1\delta_2 - \delta_2\delta_1.$$
\end{definition}

\begin{lemma}
	\label{Lemma tau commutator}
	Let $(G^+,G^-)$ be a pre-Kantor pair.
	Take, $x,u,c,d \in G^+,y,v,a,b \in G^-$.
	The equations
	\begin{enumerate}
		\item $[V_{x,y},V_{u,v}] = V_{V_{x,y} u,v} - V_{u,V_{y,x} v},$
		\item $ \tau_{x,[a,b]} = V_{Q_x a, b} + [V_{x,a},V_{x,b}] - V_{Q_x b, a}$,
		\item $ \tau_{[c,d],y^{-1}} = V_{Q_y d, c} + [V_{y,d},V_{y,c}] - V_{Q_y c,d}$
	\end{enumerate}
	hold.
	\begin{proof}
		The first equation holds since $V$ is a derivation by item (3.a) of Definition \ref{Definition pre-Kantor pair}, $V_{x,y} = - V_{y,x}$ and since $V$ is entirely defined using linearisations of the operators.
		We note that the second and third equations are equivalent, since $\tau_{x,[a,b]} = - \tau_{[a,b],x^{-1}} = \tau_{[b,a],x^{-1}}$ holds by definition.
		We prove the second (and also third equation) in Appendix \ref{Lemma linerizations tau} by proving that $\tau^{(2,(1,1))}_{x,a,b} = V_{Q_xa,b} + V_{x,a}V_{x,b}$, which is sufficient since $\tau_{x,[a,b]} = \tau^{(2,(1,1))}_{x,a,b} - \tau^{(2,(1,1))}_{x,b,a}$ as can be proved by applying Lemma \ref{Lemma homogeneous map on commutator} to the homogeneous map $g \mapsto \tau_{x,g}$.
	\end{proof}
\end{lemma}

\begin{definition}
	Let $(G^+,G^-)$ be a pre-Kantor pair. For each invertible $\lambda$ in $K$ we have an automorphism of $(G^+_K,G^-_K)$, namely $(g \mapsto \lambda \cdot_1 g, g \mapsto \lambda^{-1} \cdot_1 g)$ by the assumptions on the degrees of the operators.
	Consider the automorphism associated to $(1 + \epsilon)$, and write its action on $G^+$ and $G^-$ additively as $(1 + \epsilon\zeta,1 + \epsilon \zeta)$. We see that $\zeta (g_1,g_2) = (\pm g_1, \pm 2 g_2)$ for $g \in G^\pm$.
	We call $$ \text{InStr}(G) = \langle V_{x,y} | x \in G^\pm, y \in G^\mp \rangle + \langle \zeta \rangle,$$
	with operation $[a,b]$ the \textit{inner structure algebra}\footnote{We do not call it the inner derivation algebra, as there could be some more morphisms which should fall under that name  if $1/2 \notin \Phi$, namely if $g$ or $t$ in $G_2$ then $\tau(g,t)$ should correspond to a derivation.} and $\zeta$ the \textit{grading element}.
	The previous lemma proves that derivations $\tau_{x,[a,b]}, \tau_{[a,b],x}$ are contained in InStr$(G)$. 
\end{definition}

\begin{lemma}
	The module $\text{InStr}(G)$ forms a Lie algebra with operation $[\cdot, \cdot]$.
	\begin{proof}
		For derivations $D,D'$ of $(G^+,G^-,Q^\text{grp},T,P)$ and arbitrary $x \in G^+, y \in G^-$, we compute
		$$ [D,[D', V_{x,y}]] - [D',[D,V_{x,y}]] = V_{(DD' - D'D)x,y} + V_{x,(DD' - D'D)y} = V_{[D,D']x,y} + V_{x,[D,D']y} = [[D,D'],V_{x,y}]$$
		using that $[E,V_{x,y}] = V_{Ex,y} + V_{x,Ey}$ for all derivations $E$.
		The Jacobi identity now follows for arbitrary triples $a + \lambda_a \zeta, b + \lambda_ b \zeta, c + \lambda_c \zeta$ of elements, with $a,b,c$ linear combinations of $V_{x,y}$'s and $\lambda_a, \lambda_b, \lambda_c \in \Phi$, since the bracket is alternating and linear.
	\end{proof}
\end{lemma}

We want to define a $5$-graded Lie algebra structure on
$$ L(G^+,G^-) = [G^-,G^-] \oplus A^- \oplus \text{InStr}(G^+,G^-) \oplus A^+ \oplus [G^+,G^+].$$
To achieve that, we endow $L(G^+,G^-)$ with the unique alternating operation $[\cdot,\cdot]$ such that
\begin{itemize}
	\item $A^+ \oplus [G^+,G^+]$ is a Lie subalgebra of $\text{Lie}(G^+)$, i.e.,
	$[(a,b),(c,d)] = [a,c] = \psi(a,c) - \psi(c,a),$
	\item $A^- \oplus [G^-,G^-]$ is a Lie subalgebra of $\text{Lie}(G^-),$
	\item $\text{InStr}$ embeds into $L(G^+,G^-)$ as a Lie algebra,
	\item $ [\delta, g] = \delta(g)$ for $g \in [G^\pm,G^\pm], \delta \in \text{InStr}(G)$,
	\item $[\delta,a] = \delta_1(a),$ for $ a \in A^\pm,\delta \in \text{InStr}(G)$ i.e.,
	we compute the action of $\delta$ by only looking at the first component so that 
	$$ [V_{x,y} ,a ] = -Q^{(1,1)}_{a,x} y,$$
	and 
	$$ [\zeta,a] = \pm a,$$
	with the sign corresponding to whether $a \in A^+$ or $A^-$,
	\item $[a,b] = V_{a,b}$ for $a \in A^+,b \in A^-$, 
	\item $[g,h] = \tau_{g,h}$ for $g \in [G^+,G^+]$, $h \in [G^-,G^-]$,
	\item $[g,b] = Q_{g} b$ for $g \in [G^\pm,G^\pm], b \in A^\mp$.
\end{itemize}
We remark that $[b,a] = V_{b,a} = -V_{a,b}$ and $[g,h] = \tau_{g,h} = -\tau_{h,g}$, so that $L(G^+,G^-) \cong L(G^-,G^+)$.

\begin{lemma}
	Consider a pre-Kantor pair $(G^+,G^-)$. The algebra $L(G^+,G^-)$ is a Lie algebra.
	\begin{proof}
		Set $L_+ = A^+ \oplus [G^+,G^+], L_0 = \text{InStr}(G^+,G^-)$ and $L_- = A^- \oplus [G^-,G^-].$
		Since $L_0$ is a Lie algebra that acts as derivations on the Lie algebra $L_+$, we know that $$ L_0 \oplus L_+$$
		forms a Lie algebra. 
		
		We only need to check the Jacobi identity.	As we check this identity we can assume without loss of generality that $a,b \in L_+ \cup L_0$ and that $c \in L_-$.
		Suppose that $a, b \in L_0$, then we are already finished since we know that $L_- \oplus L_0$ forms a Lie algebra.
		So, suppose that $b \in L_0$ and $a \in L_+$.
		Note that all automorphisms of the pre-Kantor pair preserve the Lie bracket of $L$, so that 
		$$ (1 + \epsilon b)[a,c] = [(1 + \epsilon b)a,(1 + \epsilon b)c],$$
		for $b \in L_0$. This proves the Jacobi identity if $b \in L_0$.
		
		So, we can assume that $a,b \in L_+$, $ c \in L_-$.
		
		Suppose first that $c \in A^-$
		For $a,b \in A^+$, the Jacobi identity follows from  $$[[a,b],c] = Q_{[a,b]} c = Q^{(1,1)}_{a,b}c - Q^{(1,1)}_{b,a}c = - V_{b,c} a + V_{a,c} b = [[a,c],b] + [a,[b,c]]$$
		by using Lemma \ref{Lemma homogeneous map on commutator} for the second step.
		If $a, b \in A^+ \cup [G^+,G^+]$ but not both in $A^+$, then we need to check that
		$$ [a,[b,c]] = [b,[a,c]].$$
		Assume that $a \in A^+, b \in [G^+,G^+]$. This is equivalent to checking that
		$$ T^{(1,2)}_{a,b} c = [a,Q_b c] \overset{?}{=} - V_{a,c} b =T^{(2,1)}_{b,a}$$
		using Definition \ref{Definition pre-Kantor pair}.(\ref{PKP LIN T}) for the linearization of $T$ and Definition \ref{definition V tau} for the definition of $V$.
		The previous equation holds since $T_{(t \cdot_1 g)b} = T_{b(t \cdot_1 g)}$ for all $g \in G(\Phi[t])$ with first coordinate $a$ since $b \in G_2$.
		If both $a,b \in [G^+,G^+]$, then the Jacobi-identity is trivially satisfied by the grading.
		
		Assume now that $c \in [G^-,G^-]$. If both $a,b \in A^+$, then Lemma \ref{Lemma tau commutator} shows that the Jacobi identity holds, since
		$$ [c,[a,b]] = \tau_{c,[a,b]} = V_{Q_ca,b}- V_{Q_c b,a} = [[c,a],b] - [[c,b],a].$$
	    Lastly, if both $a,b \in A^+ \cup [G^+,G^+]$ and if it is not the case that both are contained in $A^+,$ then we must show that
	    $$ [a,[b,c]] = [b,[a,c]],$$
	    which is the case if both $a,b \in [G^+,G^+]$ since the previous equation becomes
	    $$ - \tau_{b,c} a = R^{(2,2)}_{a,b} c = R^{(2,2)}_{b,a} c = - \tau_{a,c} b,$$
	    using Definition \ref{definition V tau} for $\tau$, which holds since $\Phi[t]$-scalar multiples of $a$ and $b$ commute. 
	    So, suppose that $a \in A^+$ and $b \in [G^+,G^+]$, then the equation reduces to
	    $$ P^{(1,2)}_{a,b} c = -Q_bQ_ca,$$
	    which is the case since $P^{(1,2)}_{a,b} c = P^{(2,1)}_{b,a} c = - Q_bQ_c a$ using Definition \ref{Definition pre-Kantor pair}.(\ref{PKP LIN P}).
	\end{proof}
\end{lemma}

\begin{lemma}
	\label{lemma preKantor induces Kantor pair}
	Consider a pre-Kantor pair $(G^+,G^-,Q^\text{grp},T,P)$. Define $A^\pm = G^\pm/G_2^\pm$, the triple $(A^+,A^-,V_{|A^+ \times A^-})$ forms a Kantor pair. Moreover, if $1/6 \in \Phi$ then $(G^+,G^-,Q^\text{grp},T,P)$ can be uniquely recovered from $(A^+,A^-,V_{|A^+ \times A^-})$.
	\begin{proof}
		This immediately follows from the fact that $L = L(G^+,G^-)$ is a Lie algebra. More precisely, since $L$ is a $\mathbb{Z}/2\mathbb{Z}$-graded algebra with as $1$-graded part $A^+ \oplus A^-$, we can conclude that $A^+ \oplus A^-$ is a Lie triple system with operation $[a^\epsilon,b^{-\epsilon},c^\epsilon] = V_{a,b}c,\; [a^\epsilon,b^{\epsilon},c^\epsilon] = 0$ where $x^\epsilon$ is contained in $A^\epsilon$ for $x = a,b,c$. Allison and Faulkner proved \cite[Theorem 7]{ALLFLK99} that this implies that $(A^+,A^-,V)$ is a Kantor pair.
		
		If $1/6 \in \Phi$, then we know that $G^+ \cong A^+ \oplus \{ a \mapsto V_{x,a}y - V_{y,a} x | x,y \in A^+\}$ with group operation determined by $$\psi(a,b) = c \mapsto (V_{a,c}b - V_{b,c} a)/2$$ since $G_2 = [G,G]$, $G_2 \cong Q_{G_2}$ as modules, and since $Q$ must satisfy $$Q_{[a,b]} c = Q^{(1,1)}_{a,b} c - Q^{(1,1)}_{b,a} c = V_{a,c} b - V_{b,c} a.$$
		The operators $Q,T,P$ and $R$ are determined by their linearisations.
		To be precise, if $x(-x) = \sum_{i = 1}^n [a_i,b_i],$ $ y(-y) = \sum_{i = 1}^m [c_i,d_i]$, then we know that
		\begin{enumerate}
			\item $2 Q_x y = Q^{(1,1)}_{x,x} y + \sum_{i = 1}^n Q^{(1,1)}_{a_i,b_i}y - Q^{(1,1)}_{b_i,a_i} y,$
			\item $ 3 T_x y = 3 T^{(1,2)}_{x,x} y - T^{(1,1,1)}_{x,x,x} y,$
			\item $ 2 P_x y =  P^{(3,(1,1))}_x(y,y) + \sum_{i = 1}^m P^{(3,(1,1))}_x(c_i,d_i) - P^{(3,(1,1))}_x(d_i,c_i),$
			\item $ 2 R_x y = R^{(4,(1,1))}_x(y,y)+ \sum_{i = 1}^m R^{(3,(1,1))}_x(c_i,d_i) - R^{(3,(1,1))}_x(d_i,c_i)$
		\end{enumerate}
	since $2x_2 = x_1^2 + \sum_{i = 1}^n (a_i)_1(b_i)_1 - (b_i)_1(a_i)_1,$ $2y_2 = y_1^2 + \sum_{i = 1}^m (c_i)_1(d_i)_1 - (d_i)_1(c_i)_1$, and $3x_3 = 3x_1x_2 - x_1^3$ hold for the universal homogeneous maps of degree $2$ and $3$ on vector groups.
	Using $V_{x,y} z = - Q^{(1,1)}_{z,x} y$ and the linearisations of the operators for pre-Kantor pairs ($T^{(1,1,1)}$ can be obtained by linearising $T^{(1,2)}$), these equations uniquely determine the pre-Kantor pair.
	\end{proof}
\end{lemma}

We define an action of $G^\pm$ on $L(G^+,G^-)$.
Namely, for $g = (a,b) \in G^\pm $ we define
$$ \exp(g) = 1 + g_1 + g_2 + g_3 + g_4,$$
using the following endomorphisms of $L(G^+,G^-)$:
\begin{align*}
	g_1 \cdot z & =  [a,z] \\
	g_2 \cdot z & = \begin{cases} \tau_{g,z} & z \in [G^\mp,G^\mp]  \\
		Q_g z & z \in A^\mp \\
		\hat{z}(g) & z \in \text{InStr}(G^+,G^-) \\
		0 & \text{otherwise}
	\end{cases}, & \text{where } \hat{z}(g) = - (z(g))_2 + \psi(z(g)_1,a)\\
 g_3 \cdot z & = \begin{cases} P_g z & z \in [G^\mp,G^\mp] \\
	T_g z & z \in A^\mp \\
	0 & \text{otherwise} 
\end{cases}, \\
	 g_4 \cdot z & = \begin{cases} R_g z  & z \in [G^\mp,G^\mp]\\
		0 & \text{otherwise} 
	\end{cases}.
\end{align*}
Remark that $(\epsilon z(g)_1,-\epsilon \hat{z}(g)) = ((1 + \epsilon z)(g))g^{-1}$ for all derivations $z$.
\begin{remark}
	Observe that $g_2 \cdot \zeta \in [G^+,G^+]$ for $g \in G^+$ since $g(-g) = (\zeta(g))_2 - \psi(\zeta(g),g)$. We also note that $R_g z \in [G^+,G^+]$ for $z \in [G^-,G^-]$. This can be observed by considering
	$$ R_g [a,b] = R^{(4,(1,1))}_{g}(a,b) - R^{(4,(1,1))}(b,a)$$
	and the $(4,(1,1))$ linearisation of $R$ given in the definition of a pre-Kantor pair.
\end{remark}

In what follows, we will often write $L$ to denote the Lie algebra $L(G^+,G^-)$ if $G$ is clear from the context.

\begin{lemma}
	The maps $(G \longrightarrow \text{End}_\Phi(L) : g \mapsto g_i)_i$ form a vector group representation, with $g_i = 0$ for $i > 4$.
	\begin{proof}
		It is sufficient to prove that $g \mapsto g_i$ is a homogeneous map with $(k,l)$-linearisation $(g,h)_{k,l} = g_kh_l$.
		Since the $g_i$ are defined using homogeneous maps, we only need to prove that the $(k,l)$-linearisations are what they should be.
		
		We already know that $g \mapsto g_1$ is linear.
		For $g_2$ we check whether the $(1,1)$-linearisation evaluated in $(a,b) \in A^+ \times A^+$ applied to the definition coincides with $a_1b_1$. So, we check whether
		\begin{align*}
			\tau^{((1,1),2)}_{a,b,z}&  = - V_{a,Q_z b}, \\
			Q^{(1,1)}_{a,b} z & = - V_{b,z} a, \\
			-(z(a,b))_{(1,1)} + \psi(z(a),b) + \psi(z(b),a) & = [a,z(b)],
		\end{align*}
		i.e., we check all the nontrivial cases in the definition of $g_2 \cdot z$.
		The first case holds by Lemma \ref{Lemma linerizations tau} and since $z \in G_2$.
		The second case holds by definition. The last case follows from $(z(a,b))_{(1,1)} = \psi(z(a),b) + \psi(a,z(b))$, which can be proved using that
		$ (1 + \epsilon z)$ is an automorphism.
		
		Most linearisations of $g_3$ and $g_4$ are directly observed to be what they should be, as can be seen from the linearisations of $T, \; P$ and $R$ that are fixed in the definition pre-Kantor pairs. The other linearisations play a role in the definition of $V$ and $\tau$ involving the linearisations of $T,\;P$ and $R$. In these cases a direct computation shows that the linearisations are what they should be.
		For example, we can compute for $T$ that
		$$ T^{(2,1)}_{g,x}(y) = (V_{y,x} g)_2 - \psi(V_{y,x}g,g) = - g_2 \cdot V_{y,x} = g_2x_1 \cdot y.$$
		 The homogeneous maps $g_n = 0$ for all $n > 4$ have trivial linearisations, as implied by the grading of the Lie algebra. 
	\end{proof}
\end{lemma}

\subsection{Operator Kantor pairs}

\begin{definition}
	We call a pre-Kantor pair $(G^+,G^-)$ an \textit{operator Kantor pair} if
	\begin{enumerate}
		\item $ \tau_{x,T^{(2,1)}_{y,a}(x)} +  V_{x,Q_yQ_{x^{-1}}a} + \tau_{y,T_{x^{-1}}a} = V_{P_xy,a} + V_{Q_{T_xy}a,y} - [V_{x,y},V_{Q_xy,a}],$
		\item   \label{operator Kantor pair uniqueness R}	$ P_xT^{(2,1)}_{y,a}(x) + Q_xQ_yQ_{x^{-1}}a - \tau_{y,T_{x^{-1}}a}x = Q_{Q^{\text{grp}}_xy} a + V_{x,y}Q_{T_xy} a,$
		\item
		$ Q^\text{grp}_x T^{(2,1)}_{y,a}(x) + T_xQ_yQ_{x^{-1}}a - (\tau_{y,T_{x^{-1}}a}(x))_2 + \psi(\tau_{y,T_{x^{-1}}a}(x),x)  =  [Q_x y, Q_{T_xy} a],$
		\item $ T_{Q^{\text{grp}}_x y} a + [P_x y , Q_{T_x y} a ] + V_{x,y} [Q_x y, Q_{T_xy} a] = R_x T_y Q_{x^{-1}} a+ T_x P_y T_{x^{-1}} a,$
	\end{enumerate}
hold for all $x \in G^\epsilon(K), y \in G^{-\epsilon}(K), a \in A^{-\epsilon} \otimes K$.
The first three equations express, using the representations $g \mapsto g_i \in \text{End}_\Phi(L)$ and associated operators $\mu_{i,j}, \nu_{i,j}$, precisely that
\begin{enumerate}
	\item $ \mu_{3,2}(x,y)$ acts as $a \mapsto V_{P_xy,a}  + V_{{Q_{T_xy}a},y} - [V_{x,y},V_{Q_xy,a}] $ on $A^{-\epsilon}$,
		\item  $ \mu_{4,2}(x,y)$ acts as $a \mapsto Q_{Q^{\text{grp}}_xy} a + V_{x,y}Q_{T_xy} a$  on $A^{-\epsilon}$,
		\item $\mu_{5,2}(x,y)$ acts as $a \mapsto [Q_x y, Q_{T_xy} a]$ on $A^{-\epsilon}$.
\end{enumerate}
The fourth equation expresses that $\nu_{6,3}(x,y)$ acts as $a \mapsto T_{Q^\text{grp}_x y} a$ on $A^{-\epsilon}$, if $\mu_{4,1}(x,y) = 0,\; \nu_{2,1}(x,y) = \text{ad} \; Q_xy, \;\nu_{1,1}(x,y) = \text{ad} \; V_{x,y},  \;\nu_{3,1}(x,y) = \text{ad} \; T_x y,$ and $\nu_{3,2}(x,y) = \text{ad} \; P_x y$ hold.
\end{definition}


\begin{remark}
	If $1/30 \in \Phi$, then each pre-Kantor pair is an operator Kantor pair. We prove this in Corollary \ref{bijection Kp preKP opKP}. 
\end{remark}


\begin{remark}
	\label{remark sufficient condition}
	A sufficient condition, which will be easier to use in practice, for the previous equations to hold, is given by the following list of properties:
	\begin{enumerate}
		\item $\exp(g)$ is an automorphism for $g \in G^+$ and $G^-$,
		\item $ \nu_{3,2}(x,y) = \text{ad} \; P_x y,$
		\item $ \nu_{4,2}(x,y) = (Q^{\text{grp}}_x y)_2,$
		\item $ \nu_{5,2}(x,y) = 0,$
		\item $ \nu_{6,3}(x,y) = (Q^{\text{grp}}_x y)_3.$
	\end{enumerate}
	These conditions are sufficient since one obtains the axioms of an operator Kantor pair by (1) expressing the $\nu_{i,j}$ as a sum of products $\mu_{k,l}$ and (2) evaluating the endomorphisms on $A^-$.
	The $\mu_{i,j}$ that must be of a specific form to obtain the precise equation of the definition, are $\mu_{1,1}(x,y) = \text{ad} \; V_{x,y},\; \mu_{2,1}(x,y) = \text{ad} \; Q_x y$, $\mu_{3,1}(x,y) = \text{ad} \; T_x y$ and $\mu_{4,1}(x,y) = 0$.
	Using the assumption that $\exp(g)$ is an automorphism one sees that the equation $\mu_{i,1}(x,y) = \text{ad} \; \exp(x)_i(y)$ always holds.
	We remark that these conditions are not only sufficient but necessary as well. In the next lemma we prove that $\exp(g)$ is an automorphism and we argue in the second part of Theorem \ref{thm weights lie algebra}, in which we prove that we can apply Theorem \ref{thm main}, that the other equations hold.
\end{remark}



\begin{remark}
	\label{Remark uniqueness P and R}
	For operator Kantor pairs, the operators $P$ and $R$ are uniquely determined from $Q$, $T$ and $V$.
	We already proved this for $P$ in Lemma \ref{lemma uniqueness P}.
	We want to prove the uniqueness of $R$. So, consider the set $S = \{ (Q_gh, t) \in G\}$.
	There exists a $k \in S$, namely $ (Q^{\text{grp}}_x y)$, which satisfies
	\[ k_2 \cdot a = \nu_{4,2}(x,y) \cdot a = (\mu_{4,2}(x,y) - \mu_{1,1}(x,y) \mu_{3,1}(x,y)) \cdot a\]
	for all $a \in A^-$, using that $ \nu_{4,2}(x,y) = (Q^{\text{grp}}_x y)_2$.
	This $k$ is unique by Equation \ref{Kantor pair equation 2}, since
	\[ (k(0,s))_2 \cdot a = k_2 \cdot a + Q_{(0,s)} \cdot a.\]
	So, if
	\[ (\mu_{4,2}(x,y) - \mu_{1,1}(x,y) \mu_{3,1}(x,y)) \cdot a \]
	is an element of $A^+$ which can be expressed using $Q$, $T$, $P$, $V$, then we know that $R$ is uniquely determined by the other operators.
	Evaluating this normally yields such an expression if one substitutes $ - \tau_{y,T_{x^{-1}} a} x = P^{(1,2)}_{x, T_{x^{-1}} a} y^{-1},$ for the contribution of
	$ x_1y_2x^{-1}_3 \cdot a$ in $\mu_{4,2}(x,y) \cdot a$.
\end{remark}






\begin{lemma}
	Let $(G^+,G^-)$ be an operator Kantor pair and take $g \in G^\pm$. The map $\exp(g)$ is an automorphism. 
	\begin{proof}
		We note that $\exp(g)$ is an automorphism if 
		$$ \exp(g)[a,b] = [\exp(g)a,\exp(g)b],$$
		for all $a,b \in L$.
		Suppose that $g \in G^+$, the previous identity holds if either one of $a$ and $b$ is an element of $L_+$. We illustrate this if $a \in L_+$. In that case, the equation $\exp(g)(\text{ad} \; a )\exp(g^{-1}) = \text{ad} \; \exp(g) a$ holds since we have a vector group representation, which proves that $ \exp(g)[a,b] = [\exp(g)a,\exp(g)b].$
		
		Similarly, using that $1 + \epsilon l$, with $\epsilon^2 = 0$, is an automorphism of the operator Kantor pair for $l \in L_0$ and that $\exp(g) \cdot \epsilon l = \epsilon l - ( (1 + \epsilon l)(g)\cdot g^{-1})$ holds by definition of the action, we conclude that $\exp(g)$ interacts nicely with brackets involving an $l \in L_0$ by observing that
		$$ \epsilon l - \text{ad} \; \exp(g) \cdot \epsilon l= \exp((1 + \epsilon l) g) \exp(g^{-1}) - 1= (1 + \epsilon l)\exp(g)(1 - \epsilon l)\exp(g^{-1}) - 1 = \epsilon l - \epsilon \exp(g) l \exp(g^{-1}),$$
		as maps on $L$ (remark that we can write $l$ for $\text{ad} \; l$ if we interpret $l$ as an element of $\text{InStr}(G)$).
		
		So, the only thing left to prove is that
		$$ \exp(g)[a,b] = [\exp(g)a,\exp(g)b],$$
		for $a,b \in L_-$.
		
		If $a,b \in A^-$ this follows from the Jacobi identity, the $(n,(1,1))$-linearisations $P,R$ of Definition \ref{Definition pre-Kantor pair} using Lemma \ref{Lemma homogeneous map on commutator}, and the equation
		$$ \tau_{x,[a,b]} = V_{Q_xa,b} + V_{a,Q_xb} + [V_{x,a},V_{x,b}]$$
		proved in Lemma \ref{Lemma tau commutator}.
		
		Now, we assume that $a \in [G^-,G^-]$ and $b \in A^-$.
		We will prove that 
		\begin{equation}
			\label{eq ad a}
			\exp(g)[a,\exp(g^{-1})(b)] = [\exp(g)(a),b],
		\end{equation} which proves that 
		\begin{equation}
			\label{eq ad b}
			\exp(g)(\text{ad} \; \exp(g^{-1})b)\exp(g^{-1}) = \text{ad} \; b
		\end{equation} holds. Using that we already proved that $\exp(g)$ conjugates as expected with $$\exp(g^{-1})(b) - b \in \text{InStr}(G) \oplus A^+ \oplus [G^+,G^+]$$
		we will be able to conclude from Equation (\ref{eq ad b}) that $$\exp(g)(\text{ad} \; b)\exp(g^{-1}) = \text{ad} \; \exp(g)(b).$$
		So, if we prove Equation (\ref{eq ad a}), then we are able to conclude that
		$\exp(g)(\text{ad} \; b)\exp(g^{-1}) = \text{ad} \; \exp(g)(b)$ for all $b \in A^- \cup A^+ \cup \zeta$, i.e., a generating set of $L$. This proves that $\exp(g)(\text{ad} \; c)\exp(g^{-1}) = \text{ad} \; \exp(g)(c)$ for all $c \in L$, since we can write $\text{ad} \; c$ as a polynomial function of elements $\text{ad} \; b$ for which it holds.
		So, proving Equation (\ref{eq ad a}) proves the lemma, without needing to consider the case where $a,b \in [G^-,G^-]$.
		So, we try to prove Equation (\ref{eq ad a}).
		The first $3$ axioms for operator Kantor pairs show that $\exp(g)a_2\exp(g^{-1}) = \text{ad} \; \exp(g)(a)$ on $A^-$ if and only if
		$$ \tau_{g,a} = g_2a_2 - g_1a_2g_1 + a_2(g^{-1})_2$$ on $A^-$.
	    Applying the definition of $\tau$ shows us that we need to prove that
	    $$ -P^{(1,2)}_{b,a}(g^{-1})  = \tau(g,a)(b) \overset{?}{=} (g_2a_2 - g_1a_2g_1 + a_2(g^{-1})_2) \cdot b = Q_aQ_{g^{-1}}b + Q_{V_{b,g}a} g.$$
	    Using that $P^{(2,1)}_{a,b}(g^{-1}) = Q_aQ_g b - V_{b,g}Q_a g$ by Definition \ref{Definition pre-Kantor pair}.(\ref{PKP LIN P}) and that $P^{(1,2)}_{b,a} = P^{(2,1)}_{a,b} + P^{(2,1)}_{[b,a],a}$ as any homogeneous map of degree $3$, transforms this to
	    $$ Q_a(Q_g + Q_{g^{-1}})b - V_{b,g}Q_ag + Q_{V_{b,g}a}g + Q_{[b,a]}Q_ga - V_{a,g}Q_{[b,a]}g \overset{?}{=} 0.$$
	    Using that $(Q_g + Q_{g^{-1}})(b) = V_{b,g} g$ since $g_2 + (g^{-1})_2 = - g_1^2$, and that $[Q_a,V_{b,g}]g = -Q_{V_{b,g}a}g$ for $a \in G_2$ since $V_{b,g}$ is a derivation, reduces the equation to
	    $$ V_{a,g}Q_{[a,b]}g = Q_{[a,b]}Q_ga$$
	    which holds since $a \in [G^-,G^-]$ so that $V_{a,g} = 0$ and $Q_{[a,b]} = 0$. This proves that Equation (\ref{eq ad a}) holds and thus the lemma.
 	\end{proof}
\end{lemma}

\begin{theorem}
	\label{thm weights lie algebra}
	Let $(G^+,G^-)$ be an operator Kantor pair and let $G^+$ and $G^-$ be projective. The $\Phi$-group functor $K \longrightarrow G(K) = \langle G^+(K),G^-(K) \rangle \le \text{End}_\Phi(L(G^+,G^-)) \otimes K$ has as corresponding Lie algebra $$L = \text{Lie}(G^-) \oplus L_0 \oplus \text{Lie}(G^+)$$ for some Lie algebra $L_0$. This Lie algebra is $5$-graded by the weights of the action of $\Phi_m(K) = K^\times$ defined by $\lambda \cdot (g_+,g_-) = (\lambda \cdot_1 g_+, \lambda^{-1} \cdot_1 g_-)$ of $\Phi_m$  on $(G^+,G^-)$ and $L_0$ is the $0$ graded component of this action. Moreover, if $E = 1 + e_1 + e_2 + e_3 + e_4 \in G(K)$ with $e_i(L_j) \subset L_{i+j}$ for all $i,j$, then $E$ must be an element of $G^+(K)$. 
	\begin{proof}
		We want to apply Theorem \ref{thm main}. 
		Suppose that we can use the conclusions of the theorem.
		This proves that the vector group representations of $G^+,G^-$ factor through some $\mathbb{Z}$-graded bialgebra $H = \mathcal{U}(G^-)\mathcal{H}\mathcal{U}(G^+)$ such that each $h \in H$ can be uniquely written as $\sum_{i = 1}^n u_{i}h_iv_i$ with all $u_i \in \mathcal{U}(G^-), v_i \in \mathcal{U}(G^+)$ and all $h_i$ contained in a $0$-graded Hopf subalgebra $\mathcal{H}$.
		Furthermore, $G(K)$ can be constructed by a quotient of the group $I(K) = \langle \rho^+_{[t]}(G^+(K)), \rho^-_{[t]}(G^-(K)) \rangle$ defined by embedding the universal representations $\mathcal{U}(G^\pm)$ in $H$.
		Specifically, take the unique map defined by $$1 + tg_1 + t^2g_2 + \ldots \longmapsto 1 + g_1 + g_2 + g_3 + g_4$$ on the generators $\rho^+_{[t]}(G^+(K))$ and $\rho^-_{{t}}(G^-(K))$ which maps $I(K) \longrightarrow G(K)$.
		Suppose that $1 + \epsilon d \in \text{Lie}(G)$. We can lift this to an element $F$ of $I(K[\epsilon])$. Furthermore, we can assume that $F$ is of the form
		$$ 1 + \epsilon f_1 + \epsilon f_2 + \epsilon f_3 + \ldots,$$
		using the fact that the image of $F$ under $\epsilon \mapsto 0$ lies in $I(K).$
		Since all the generators of $I(K)$ are group like, i.e., satisfy $\Delta(i) = i \otimes i, \epsilon(i) = 1$, we conclude that all $f_i$ satisfy $\Delta(f_i) = f_i \otimes 1 + 1 \otimes f_i$. Lastly, note that $g_k \mapsto 0$ for all $k > 4$ and all $g \in G^+,G^-$, i.e., $1 + ti_1 + t^2i_2 + \ldots \in I(K)$ implies that there exists some $N$ such that $i_k \mapsto 0 $ if $k \ge N$.
		So, we conclude that there exists a primitive element 
		$ \sum_{i = 1}^N f_i$ which maps to $d$.
		Since each primitive element of $H$ can be uniquely written as a sum $\sum u_{i}h_iv_i$, it is not hard to check that each primitive element can be uniquely decomposed as $p = u_1 + h_2 + v_3$ with $u_1,h_2,v_3$ primitive using $(\pi_{\mathcal{U}(G^-)} \otimes \pi_\mathcal{H} \otimes \pi_{\mathcal{U}(G^+)})\Delta^2(yhx) = y \otimes h \otimes x$.
		If we assume that $d$ has no $0$-graded component, then we can conclude that $d \in \text{Lie}(G^+) \oplus \text{Lie}(G^-)$.
		
		If the conditions of Theorem \ref{thm main} apply, then one also proves the moreover part in a similar fashion. Namely, one can find a lift for $E$ in $I(K)$ using the fact that all group-like elements $h$ in $H$ such that $h - 1$ has only positively graded components, must lie in $\mathcal{U}(G^+)$ and then apply Theorem \ref{theorem PUG} to conclude that $h \in G^+(K)$.
		
		So, now we prove that the conditions of Theorem \ref{thm main} apply.
		It is sufficient to prove that 
		$$ \exp(O^{i1}_xy) = 1 + \sum_{k = 1}^4 \nu_{(ki),k}(x,y),$$
		for $O^{21} = Q^{\text{grp}}$ and $O^{31} = T$, that
		$$ \nu_{4,1}, \nu_{5,1}, \nu_{5,2}, \nu_{7,3} = 0$$
		and that
		$$ \nu_{3,2}(x,y) =\text{ad} \; P_x y$$
		for $x \in G^+$ and $y \in G^-$ (or with the roles of $x$ and $y$ reversed) and $l \in A^+$.
		To simplify this task, note that there are no nontrivial derivations $d: L(G^+,G^-) \longrightarrow L(G^+,G^-)$ so that $d(L(G^+,G^-)_i) \subset L(G^+,G^-)_{i + k}$ with $|k| \ge 4$ since $L(G^+,G^-)$ is generated by $A^+,A^-$ and $\zeta$.
		Furthermore, one computes, using the fact that $\exp(g)$ is an automorphism for all $g$, that 
		$$ \exp(o_{i,j}(x,y)) = \sum_{k = 0}^\infty \nu_{ki,kj}(x,y),$$
		is an automorphism as well (similar to Lemma \ref{lemma grouplike}), proving, in particular, that the $\nu_{i,j}$ with $|i - j| \ge 4$ are uniquely determined by the $\nu_{k,l}$ with $|k - l| < 4$ and $i/j = k/l$, as all $\nu_{i,j}$ are determined by those smaller $\nu_{k,l}$ up to a derivation.
		The previous observations reduce what we need to prove to
		\begin{enumerate}
			\item $\nu_{2,1}(x,y) = \text{ad} \; Q_x y$
			\item $\nu_{4,2}(x,y) = (Q^\text{grp}_x y)_2$,
			\item $\nu_{6,3}(x,y) = (Q^\text{grp}_xy)_3$
			\item $\nu_{3,1}(x,y) = \text{ad} \; T_xy$
			\item $\nu_{4,1}(x,y) = \nu_{5,2}(x,y) = 0$
			\item $\nu_{3,2}(x,y) = \text{ad} \; l = \text{ad} \; P_x y$.
		\end{enumerate}
	For $\nu_{i,1}(x,y) = \mu_{i,1}(x,y)$ this follows from the fact that $\exp(x)$ is an automorphism. 
	Note that $\nu_{5,2}(x,y)$ is a derivation on $L(G^+,G^-)$ which acts as $+3$ on the grading. So, the action given in the definition of an operator Kantor pair determines $\nu_{5,2}(x,y)$ uniquely since that the third axiom for operator Kantor pairs implies that $\mu_{5,2}(x,y) - \nu_{2,1}(x,y)\nu_{3,1}(x,y) = 0$ on $A^-$. Similarly, the fourth axiom guarantees that $\nu_{6,3}(x,y) = (Q^\text{grp}_xy)_3$ if we can prove that the other equations hold.
	So, we only need to consider $\nu_{4,2}(x,y)$ and $\nu_{3,2}(x,y)$.
	The element $\nu_{4,2}$ acts trivially on $A^+$ and acts as expected on $A^-$. Evaluating on the grading derivation gives us an element that acts as an inner derivation of the Lie algebra. This evaluation yields $\nu_{4,2}(x,y) \cdot \zeta$ which is an element that acts like the endomorphism $-2\nu_{4,2}(x,y) + \nu_{2,1}(x,y)^2$. This last endomorphism acts exactly as $(0,-2R_xy + \psi(Q_xy,Q_xy))_2$ on $A^-$, which proves that $\nu_{4,2}(x,y)$ acts as $(Q^\text{grp}_x y)_2$ since $G^+_2$ acts faithfully on $A^-$.
	
	Note that 
	$$ \nu_{3,2}(x,y) \cdot \zeta = - l \in A^+.$$
	Using the fact that $\nu_{3,2}$ is a derivation, we conclude $\text{ad} \; l = \nu_{3,2}(x,y)$.
	We know that $(x,y) \mapsto  - \nu_{3,2}(x,y) \cdot l$ is a homogeneous map with the same linearisations, by Lemma \ref{Lemma equations}, as $P$, hence $l = P_xy$ by Lemma \ref{lemma uniqueness P}. 
	 This finishes the proof.
	\end{proof}
\end{theorem}

\begin{remark}
	It could very well be true that the previous theorem holds without the assumption that $G^+$ and $G^-$ are projective.
	One can, however, also drop the projective assumption in order to get a weaker result. Namely, if there exists an element $E \in G(K)$ of the form of the previous theorem which is not contained in $G^+(K)$, then it still corresponds to a group-like element in a Hopf quotient $U^+$ of the Hopf algebra $\mathcal{U}(G^+)$. Similarly, the positively graded elements of the Lie algebra $\text{Lie}(G)$ that are not contained in $\text{Lie}(G^+)$, correspond to primitive elements of $U^+$.
\end{remark}





\begin{theorem}
	\label{thm main 2}
	Suppose that $\rho^\pm : G^\pm \longrightarrow A$ are vector group representations such that for all $x \in G^\pm, y \in G^\mp$,
	\begin{itemize}
		\item $\exp(o_{2,1}(x,y),s) \in \rho^\pm_s(G^\pm)$, 
		\item $\exp(o_{3,1}(x,y),s^2) \in \rho^\pm_s([G^\pm,G^\pm])$ ,
		\item $\exp(o_{i,j}(x,y),s) = 1$ for $i > 2j$ and $i \neq 3j,$
		\item there exists $z \in G^\pm$ such that $\rho_s(z) = 1 + s\nu_{3,2}(x,y) + O(s^2)$.
	\end{itemize}
	Set $K^\pm = \{(\rho^\pm_1(g),\rho_2^\pm(g)) \in A \times A | g \in G^\pm\}.$
	Then $$(K^+,K^-, Q^\text{grp} = o_{2,1}, T = o_{3,1}, P = \nu_{3,2})$$ forms an operator Kantor pair if and only if $\rho_2(G_2) \ni (0,s) \mapsto Q_{(0,s)}(\cdot)$ is injective and $\rho^\pm(g(-g)) \in \rho^\pm[G^\pm,G^\pm]$.
	\begin{proof}
		The conditions that the $(0,s) \in G$ act faithfully and that $$\rho^\pm(g(-g)) \in \rho^\pm[G^\pm,G^\pm]$$ are necessary.
		Furthermore, the equations of Lemma \ref{Lemma equations} prove that we have a pre-Kantor pair since these express firstly, in equations \ref{REP EQ V1}, \ref{REP EQ V 2}, \ref{REP EQ Tau1} and \ref{REP EQ Tau2}, that the definition of $V_{x,y},$ and $\tau_{x,y}$ is such that the action of $1 + \eta V_{x,y} + \eta^2 \tau_{x,y}$ coincides with the conjugation action of $\exp(o_{1,1}(x,y),\eta)$ over $\Phi[\eta]/(\eta^3)$. These equations also prove that $1 + \epsilon V_{x,y}$ is an automorphism since the operators $Q,$ $T,$ $R,$ and $P$ are defined using multiplications in $A$. Thirdly, the rest of the equations also express that the linearisations of the operators $Q,$ $R,$ and $P$ are as required; the equations can be matched to axioms for pre-Kantor pairs by matching the degrees. The linearization for $T$ follows from Lemma \ref{lemma linearizations mu} and $\nu_{k,1} = \mu_{k,1}$.
		
		We will prove that it is an actual operator Kantor pair in Appendix \ref{section proof}.
	\end{proof}
\end{theorem}

We use the previous theorem to prove that what we constructed in Example \ref{example hermitian form} is in fact an operator Kantor pair.

\begin{definition}
	We call a vector group $G$ with operators $Q^\text{grp}, T, P$  such that $(G,G,Q^\text{grp},T,P)$ forms an operator Kantor pair, an \textit{operator Kantor system}.
\end{definition}

\begin{lemma}
	The quadruple $(G,Q^{\text{grp}},T,P)$ of Example \ref{example hermitian form} forms an operator Kantor system if condition (\ref{condition kantor pair hermitian form}) holds or if $1/2 \in \Phi$.
	\begin{proof}
		We will apply Theorem \ref{thm main 2}. If we construct vector group representations so that the operators of Example \ref{example hermitian form} correspond to the operators of Theorem \ref{thm main 2}, then it is sufficient to check that $g(-g) \in [G,G]$ in order to prove that we have an operator Kantor pair.
		Condition (\ref{condition kantor pair hermitian form}) says precisely this, namely $(m,a) \in G$ implies that
		$$ (m,a)(-m,a) = (0,2a - h(m,m)) = (0,a - \bar{a}) \in [G,G].$$
		We do not need this condition if $1/2 \in \Phi$ since $(m,a) \in G$ implies that $a - \bar{a} \in \langle h(x,y) - h(y,x) | x,y \in M \rangle$, so that $$(0, a - \bar{a}) \in \langle (0,h(x,y) - h(y,x))  \rangle = \langle [(x,h(x,x)/2),(y,h(y,y)/2)] \rangle.$$
		
		We work with the algebra $\mathcal{A}$ associated with a Kantor pair associated to a Hermitian form in \cite[section 8]{ALLFLK99}. This algebra is of the form $\begin{pmatrix}
			C & M & C \\
			\bar{M} & \mathcal{E} & \bar{M} \\
			C & M & C\\
		\end{pmatrix}$ as a $\Phi$ module with $\bar{M}$ and $M$ equal as sets but opposite $C$-module structures and $\mathcal{E}$ a subalgebra of pairs of endomorphisms of $M,\bar{M}$; the product of $\mathcal{A}$ corresponds to a matrix product and product that not involve $\mathcal{E}$ are given by the actions of $C$ on $M$ and the bilinear map $M \times \bar{M} \longrightarrow C$. We do not need the products involving $\mathcal{E}$ in this proof.
		
		We also consider the vector group representations
		$$(m,a) \mapsto \begin{pmatrix}
			1 & m & a \\ 
			0 & (\text{Id},\text{Id}) & \bar{m} \\
			0 & 0 & 1 \\
		\end{pmatrix} \in \mathcal{A}, \quad 
		(n,b) \mapsto \begin{pmatrix}
			1 & 0 & 0 \\ 
			\bar{n} & (\text{Id},\text{Id}) & 0 \\
			b & \bar{n} & 1 \\
		\end{pmatrix} \in \mathcal{A}.$$
		One can compute that this gives us a pre-Kantor pair with the given operators $Q,T,P,R$ corresponding to $\nu_{ij}$ using Theorem \ref{thm main 2}.
		The decomposition of $\exp_+(sx)\exp_-(ty)$ as $\exp_-(tY)h\exp_+(sX)$ for certain $X,Y$with $h$ diagonal, can be used to compute certain formal power series in $s$ and $t$. Note that for such a decomposition
		 $\exp_-(ty^{-1})\exp_+(tx^{-1}) = \exp_+(sX^{-1})h^{-1}\exp(tY^{-1})$ 
		 always holds.
		The first decomposition yields that $\exp_+(tX)$ equals
		$$\begin{pmatrix}
			1 & tm + t^2s Q_xy + t^3s^2P_xy + O(s^3) & t^2a + t^3s(T_xy + (Q_xy) \bar{x}) + t^4s^2(R_xy + (P_xy) \bar{x} )  + O(s^3)\\
			& (\text{Id},\text{Id}) & \ldots \\
			& & 1 \\			
		\end{pmatrix},$$
		which shows the necessity of the formulas for $Q,T,P,R$.
		With the second decomposition, one can easily show that the  omitted (and difficult) expression is what it should be, which proves that the formulas hold.
		Theorem \ref{thm main 2} shows that we have an operator Kantor pair.
	\end{proof}
\end{lemma}

\begin{remark}	
	The fact that we have an operator Kantor pair, can be used to prove that the element $sX$ of the previous theorem is a well-defined element of $s \cdot_1 G^+[[st]]$ which denotes the inverse limit of the groups $s \cdot_1 G^+(\Phi[st]/[(st)^n])$.
	This $sX$ can be thought of as the quasi inverse $(sx)^{ty}$ which one often considers in the context of Jordan pairs.
\end{remark}