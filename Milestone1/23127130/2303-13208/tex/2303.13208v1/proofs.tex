	\section{Proof of Theorems \ref{thm main} and \ref{thm main 2}}
\label{section proof}


We shall prove Theorems \ref{thm main} and \ref{thm main 2} using a universal construction.
Specifically, let $G^\pm$ be vector groups and $U^\pm$ be their universal representations $\mathcal{U}(G^\pm)$.
Set $F$ equal to the free product $U^+*U^-$ of associative algebras where we identify the units of $U^+, U^-$ so that these units correspond to the unit in $F$. Recall that $\mathcal{H}(G^+,G^-,F)$ is the subalgebra of $F$ generated by all the $\nu_{i,i}(x,y)$ for $x \in G^+$ and $y \in G^-$ in $F$ and that the universal representation is given by $\exp_{[s]}(x) = \sum_{i = 0}^\infty s^i x_i$ for $x \in G^\pm$. We also write $\exp(sx)$ for $\exp_{[s]}(x)$.

We will mod an ideal $I$ out that corresponds to imposing the conditions of Theorem \ref{thm main} on $F$ and nothing more.
In doing so, we will also lift the functions $o_{2,1}, o_{3,1}, \nu_{3,2}$ which are maps that depend on the representation in $A$, to maps $o_{2,1}, o_{3,1}, \nu_{3,2}$ to $F/I$.

We consider the ideal $I$ corresponding to equalities
$$\nu_{(i_1,\ldots,i_n),(j_1,\ldots,j_m)}(x_1,\ldots,x_n,y_1,\ldots,y_m) = u_{(i_1,\ldots,i_n),(j_1,\ldots,j_m),x_1,\ldots,x_n,y_1,\ldots,y_m} \in U^\pm_{i - j}$$ (where we still need to define the $u$) for  $$\quad \sum_{k = 1}^n i_k \ge 2 \sum_{k =1}^m j_k \text{ or } \left(\sum i_k,\sum j_k\right) = (3,2).$$ Note that the $(i_1,\ldots,i_n)$ in $\nu$ denote linearisations; we use it just for indexing purposes in $u$.
Mostly, it does not matter what the $u$ are, except that they lie in the right space $U^\pm$.
However, concrete forms for the $u$ are given by
\begin{itemize}
	\item $u_{2i,i,x,y} = (Q^\text{grp}_xy)_i$,
	\item $u_{3i,i,x,y} = (T_xy)_{2i}$,
	\item $u_{i,j,x,y} = 0$ if $i > 2j$ and $i \neq 3j$,
	\item $u_{3,2,x,y} = (P_x y)_1$.
\end{itemize}
in which $Q^\text{grp}_xy$ denotes an element in $G^\pm$ that maps to $o_{2,1}(x,y)$ in $A$ and in which $T$ and $P$ are defined similarly.
In the case that $Q^\text{grp}, T,P$ are defined from an injective representation we know that $(x,y) \mapsto u_{i,j,x,y} \in U^\pm$ is a homogeneous map, so we can linearize $u_{i,j,\cdot,\cdot}$ to obtain the $$ u_{(i_1,\ldots,i_n),(j_1,\ldots,j_m),x_1,\ldots,x_n,y_1,\ldots,y_m}.$$
If there is no injective representation, then one can take random elements in $U^\pm$ which map to \[\nu_{(i_1,\ldots,i_n),(j_1,\ldots,j_m)}(x_1,\ldots,x_n,y_1,\ldots,y_m) \in A \] under the non-injective representation. We assume that these random elements have the right grading, i.e., $u_{(i_1,\ldots,i_n),(j_1,\ldots,j_m),x_1,\ldots,x_n,y_1,\ldots,y_m}$ is $\sum i_k - \sum j_k$ graded.

\begin{lemma}
	For injective representations the algebra $F/I$ is a Hopf algebra.
	\begin{proof}
		One easily sees that the generators of the ideal $I$ form a coideal using the fact that
		$$ \Delta(u_{(ki_1,\ldots,ki_n),(kj_1,\ldots,kj_m),\ldots}) = \sum_{m + n = k}u_{(mi_1,\ldots,mi_n),(mj_1,\ldots,mj_m),\ldots} \otimes u_{(ni_1,\ldots,ki_n),(nj_1,\ldots,nj_m),\ldots}$$
		if $\gcd(i_1,\ldots,j_m) = 1$ and that the same holds for $\nu_{(i_1,\ldots,i_n),(j_1,\ldots,j_m)}(\ldots)$.
		These equalities for $u$ are easily obtained by observing that $U_{2,1,t,x,y} = \sum t^i u_{2i,i,x,y}$ and the similarly defined $U_{3,1,t,x,y}$ are group-like and by using comparison of scalars for the linearisations. For the primitive element $u_{3,2,x,y}$ and its linearisations, we can argue similarly.
		 For $\nu$ these equations follow from Lemma \ref{lemma grouplike}.
	\end{proof}
\end{lemma}


We argue along the lines of Faulkner \cite[section 6]{FLK00} to prove Theorem \ref{thm main}.

For a monomial $m = \prod_{i=1}^k {(u_i)}_{n_i} \in F/I$ with $u_i \in G^\pm$ and $n_i \in \mathbb{N}$ and $k \in \mathbb{N}$, we define the \textit{$\sigma$-degree} as
$$ \deg_\sigma{m} = \sum_{u_i \in G^\sigma} n_i. $$
Additionally, the \textit{level} of $m$ is defined as
$$ \lambda(m) = \sum_{(i,j) \in L} n_in_j, $$
where
$$ L = \{ (i,j) : i < j, u_i \in G^+, u_j \in G^- \}.$$

Something useful to note is, if $f(m_2) \le f(m_2)'$ for $f = \deg_\pm$ and $f = \lambda$, then
\begin{equation}
	\label{equation level}
	\lambda(m_1m_2m_3) \le \lambda(m_1m_2'm_3),
\end{equation}
holds, for all $m_1,m_3,$
since $\lambda(m_1m_2) = \lambda(m_1) + \lambda(m_2) + \deg_+(m_1)\deg_-(m_2)$ holds for all $m_1,m_2$.

Let $\mathcal{M}_{pq}(r)$ be the span of all the monomials $m$ with $\deg_+(m) \le p$, $\deg_-(m) \le q$ and $\lambda(m) \le r$.
Define $\mathcal{H}$ as the image of $\mathcal{H}(G^+,G^-,F)$ under the projection $F \longrightarrow F/I$, i.e., it is the subalgebra of $F/I$ generated by all $\nu_{i,i}(x,y)$ for $x \in G^+$ and $y \in G^-$.

Consider the statement $I_n$ for $F/I$: ``all the $\nu_{i,j}(g^+,g^-)$ with $\min(i,j) \le n$ are contained in $\mathcal{U}^+ \mathcal{H} \mathcal{U}^-$".
We know that $I_1$ holds.
We will be using induction to prove $I_n$, as part of a proof of Theorem \ref{thm main}.

Recall that we use $\exp(sx) = \exp_{[s]}(x)$ to denote the image of $x \in G^\pm$ under the universal representation in $F$. 
We work in $F/I$ from now onward.


\begin{lemma}
	\label{lemma xpyq equiv wpq}
	For all $p,q \in \mathbb{N}$ and all $x \in G^+, y \in G^-$ we have
	$$ x_py_q \equiv \nu_{p,q}(x,y) \mod \mathcal{M}_{pq}(pq-1).$$
	\begin{proof}
		We look at the terms with coefficient $s^pt^q$ in $$\exp(sx) \cdot \exp(ty) \cdot \exp(sx^{-1}) = \prod \exp( o_{a,b}(x,y), s^at^b),$$
		and realize that $x_py_q$ is the only term that is not necessarily $0$ mod $\mathcal{M}_{pq}(pq-1)$ on the left-hand side. If we look at the right-hand side, $\nu_{pq}(x,y)$ is the only term which is not a product of multiple terms.
		Each of these products has a lower level since the last contributing term has nonzero $\deg_+$.	
	\end{proof}
\end{lemma}

\begin{lemma}
	\label{lemma Faulkner}
	Consider $n \in \mathbb{N}$ and $d = \text{gcd}\{ \binom{n}{i} | 1 < i < n\}$. In that case $d \neq 1$ if and only if $n = p^e$ for a prime $p$ and $d = p$.
	\begin{proof}
		Faulkner \cite[Lemma 21]{FLK00} proved this.
	\end{proof}
\end{lemma}

\begin{lemma}
	\label{lemma gcds}
	Consider even $n > 2$ and $d = \text{gcd}\{ \binom{n}{i} | 1 < i < n, 2i \neq n\},$ then $d = 2$ or $d = 4$ if $n = 2^e$, $d = p$ if $n = 2p^e$ for an odd prime $p$, and $d = 1$ in all other cases.
	\begin{proof}
		We write $2k = n$.
		
		Set $S = \{ \binom{2k}{i} | 1 < i < 2k, i \neq k\}$.
		Suppose that there is an odd prime $p$ such that $p | S$ and write $2k = p^e l$ with $p \nmid l$.
		We have
		$$ 1 + \binom{2k}{k} t^k + t^{2k} \equiv (1 + t)^{2k} \equiv 1 + \binom{l}{1}t^{p^e} + \binom{l}{2}t^{2p^e} + \ldots \mod p.$$
		So, $l = 2$, $k = p^e$ and $p^i \mid S$ for some $1 \le i \le e$.
		We know that there exists $0 < j < k$ such that
		$$(1 + t)^k = 1 + a_j t^j + O(t^{j+1}) \mod p^2$$ with $a_j \neq 0 \mod p^2$, by Lemma \ref{lemma Faulkner}.
		Hence, we obtain $$(1 + t)^{2k} \equiv 1 + 2 a_j t^j + O(t^{j+1}) \mod p^2.$$
		So, we conclude that $p^2 \nmid S$ and thus $d = p$.
		
		Now, suppose that $n = 2^i = 2k.$
		Take $j$ such that
		$$ (1 + t)^{k} \equiv 1 + a_jt^j + O(t^{j+1}) \mod 4,$$
		with $a_j \equiv 2 \mod 4$, which exists by Lemma \ref{lemma Faulkner}.
		We compute that
		$$ (1 + t)^{2k} \equiv (1 + 2 a_j t^j + O(t^{j+1})) \equiv 1 + 4 t^{j} + O(t^{j+1}) \mod 8.$$
		Hence, the maximal power of $2$ that could divide all of $S$ is $4$.
	\end{proof}
\end{lemma}

\begin{lemma}
	\label{lemma xpyq in Mpqpq-1}
	If $I_n$ holds and if we consider $p \neq q, \min(p,q) \le n + 1$, $x \in G^+,$ and  $y \in G^-$, then we know
	$$ x_py_q \in \mathcal{M}_{pq}(pq - 1).$$
	\begin{proof}
		If $\min(p,q) \le n$, this is true since $I_n$ holds.
		So, suppose that $q = n + 1$ (the case $p = n + 1 $ is similar).
		We can assume that $n + 1 < p < 2n + 2$, as $ p \ge 2q$, either learns us that we imposed $\nu_{p,q}(x,y) = (T_xy)_q$ if $p = 3q$, $\nu_{p,q}(x,y) = (Q^{\text{grp}}_{x}y)_q$ if $p = 2q$, or $\nu_{p,q} = 0$ if $p \neq 3q$, which implies that $x_py_q \in \mathcal{M}_{pq}(pq-1)$ by Lemma \ref{lemma xpyq equiv wpq}.
		
		Suppose first that $x \in G^+_2$ and $y \in G^-_2$.
		We can assume that $p = 2l, q = 2k$ since $x_py_q = 0$ otherwise.
		Using the fact that
		\[ \binom{l}{a} x_p = x_{2a}x_{p-2a},\]
		which holds by Construction \ref{construction universal representation}.(\ref{cons:g2 scalar}),
		and the fact that $x_{2a}x_{p-2a}y_q = x_{p-2a}x_{2a}y_q \in \mathcal{M}_{pq}(pq-1),$ which holds by induction since $\min(a,p-1) \le n$, we see that all $\binom{l}{i}$-multiples of $x_py_q$ with $0 < i < l$ and all $\binom{k}{j}$-multiples with $0<j<k$ are contained in $\mathcal{M}_{pq}(pq-1)$. The greatest common divisor of all those binomial coefficients is $1$ by Lemma \ref{lemma Faulkner} since there are no powers $a,b$ of the same prime such that $a < b < 2a$. So, we conclude that $x_py_q \in \mathcal{M}_{pq}(pq-1)$, since $\gcd(a_1,\ldots,a_n) = k$, for $a_i \in \mathbb{Z}$, implies that there exist integers $b_1,\ldots,b_n$ such that
		\[ \sum_{i = 1}^k a_ib_i = k.\]	
		Suppose now that $x \in G^+$ and $y \in G^-_2$.
		We will use that
		$$\binom{m}{k} x_m= \sum_{\substack{b + c = k\\ a +c = m}} x_a x_b (x(-x))_{2c}, \quad \binom{m}{k} y_{m} = \sum_{\substack{b + c = k\\ a +c = m}} y_a y_b (2 \cdot_2 y)_{2c},$$
		i.e., \ref{construction universal representation}.(\ref{cons:g scalar}).
		Analoguous to the previous case, one argues that all $\binom{p}{i}, 0 < i < p, \binom{q}{j}, 0 < j < q$ multiplies of $x_py_q$ are contained in  $\mathcal{M}_{pq}(pq-1)$, except possibly the $\binom{q}{q/2}$ multiples since the expression for $\binom{q}{q/2} y_mq$ contains $(2 \cdot_2 y)_q$.
		If $q = 2$, and thus $p = 3$, we can use that $x_3y_2 \equiv \nu_{3,2}(x,y) \mod \mathcal{M}_{32}(5)$ and $\nu_{3,2}(x,y) \in \mathcal{M}_{10}(0) \subset \mathcal{M}_{32}(5)$.
		If $q > 2$, then the set $S = \{ \binom{q}{a} | a \neq 1,q,q/2\}$ is nonempty. Since $p$ and $q$ cannot both be powers of the same prime, we can use Lemma \ref{lemma gcds} and Lemma \ref{lemma Faulkner} to conclude that $x_py_q \in \mathcal{M}_{pq}(pq-1)$.
		
		
		
		Lastly, suppose that $x \in G^+$ and $y \in G^-$.
		Using the previous case and analogous argumentation to the first case one proves that $x_py_q \in \mathcal{M}_{pq}(pq-1)$.
	\end{proof}
\end{lemma}

\begin{lemma}
	\label{Lemma decomposition}
	Each element of $F/I$ can be written as an element in $U^-\mathcal{H}U^+ \le F$.
	\begin{proof}
		We will show that $I_n$ implies that the statement $J_r$: ``$\mathcal{M}_{pq}(r)$ is contained in $U^-\mathcal{H}U^+$ for all $p,q$ such that $\min(p,q) \le n + 1$", holds for all $r$.
		Since $I_1$ holds, proving that all $J_r$ holds, lets us conclude that the lemma holds, as it proves by induction (i) that all $I_n$ hold and (ii) that the lemma holds if all $I_n$ hold since $\bigcup_{p,q,r} \mathcal{M}_{p,q}(r) = F$.
		
		We use induction on $r$ to prove $J_r$.
		If $r = 0$, this is trivial.
		So, suppose that $r > 0$.
		Take a monomial $m \in \mathcal{M}_{pq}(r)$ but not in $\mathcal{M}_{pq}(r-1)$.
		This monomial cannot factor as
		$ m = m_1 x_a y_b m_2,$
		with $0 \neq a \neq b \neq 0$, $x \in G^+$ and $y \in G^-$ by Lemma \ref{lemma xpyq in Mpqpq-1} and (\ref{equation level}).
		Now, consider a factorization $m_1 x_iz_jy_cm_2,$ with $i,j,c \neq 0$ with $x,z \in G^+$ and $y \in G^-$. Using the previous case, we conclude that $j = c$. Now, we use that $i \neq 0$, to obtain that
		$$ m_1x_iz_cy_cm_2 = m_1 \left((xz)_{i+c} - \sum_{k+l = i+c, l \neq c} x_kz_l\right)y_c m_2 \in \mathcal{M}_{pq}(r-1)$$
		using the previous case.
		Similarly, it cannot factor as $m_1 x_a y_c w_dm_2$
		with $cd \neq 0$ or $0 \neq c + d \neq a$ for $x \in G^+$ and $y,w \in G^-$. 
		
		Hence, $m$ decomposes as a product
		$$ m_1 \cdot \prod (x_i)_{p_i}(y_i)_{p_i} \cdot m_2,$$
		with $m_1 \in U^-$, $m_2 \in U^+$, $x_i \in G^+$ and $y_i \in G^-$.
		Applying Lemma \ref{lemma xpyq equiv wpq} learns us that this product can be rewritten as 
		$$ m_1 \cdot \prod (x_i)_{p_i}(y_i)_{p_i} \cdot m_2 \equiv m_1 \cdot \prod \nu_{p_i,p_i}(x,y) \cdot m_2 \mod \mathcal{M}_{pq}(r - 1).$$
		So, $m$ is an element of $U^- \mathcal{H}U^+$ by the induction hypothesis.
	\end{proof}
\end{lemma}

\begin{proof}[Proof of Theorem \ref{thm main}]
	The non-moreover part is Lemma \ref{Lemma decomposition}.
	For the moreover part, we first prove that the image of $V^+$ of $U^+ = \mathcal{U}(G^+)$ is a direct summand of $F/I$ (without needing the projectivity).
	Suppose that we have an element 
	$$ \sum_{i = 1}^k y_i h_i x_i,$$
	with $y_i \in V^-$, $x_i \in V^+$ and $h_i \in \mathcal{H}$ and assume that all the $y_i,x_i$ are nicely graded (i.e., homogeneous). We want to prove that 
	$$ \sum_{i = 1}^k y_i h_i x_i \in V^+,$$
	implies that $y_ih_i\in \Phi$ for all $i$.
	First of all, we can assume that $\epsilon(x_i) = 0$ for all $i$, since $\sum y_ih_i \in V^+$ implies, using the fact that this element is necessarily $0$-graded, that $\epsilon (\sum y_ih_i) = \sum y_ih_i$.
	Take the terms with the $y_i$ with minimal grading in $F/I$ (or equivalently with maximal grading in $V^-$), which we assume to be $-n$, and take of those terms the ones with $x_i$ maximally graded, which we assume to be $m$ graded. Note that $m > 0$.
	We call the set of those terms $S$.
	We compute if $-n < 0$, using $\pi_i$ the projection on the $i$th grading component of $F/I$, that
	$$ 0 = \mu \left((\pi_{-n} \otimes S \otimes \pi_m)\Delta^3\left(\sum_{i = 1}^k y_i h_i x_i\right)\right) = \sum_{yhx \in S} yh_{(1)}S(h_{(2)})h_{(3)}x = \sum_{t \in S} t$$
	using that the function of the left-hand side must evaluate to $0$ on $U^+$, the fact that $\Delta(h) = h \otimes 1 + 1 \otimes h \mod \ker \epsilon \otimes \epsilon$, and where we used Sweedler summation notation so that
	$$ h_{(1)}S(h_{(2)})h_{(3)} = h,$$
	for all $h \in \mathcal{H}$.
	
	Now, to resolve the $n = 0$ case, consider $\sum_i h_ix_i$ with $h_i \notin \Phi$. We can assume that $\epsilon(h_i) = 0$ for all $i$. We use
	$$ 0 = \mu \left( S(1 - \epsilon)\pi_0 \otimes \pi_m) \Delta\left(\sum_i h_ix_i\right)\right) = \sum_i S(h_{i,(1)})h_{i,(2)}x_i - \sum_i h_ix_i = -\sum_i h_ix_i. $$
	This proves that $V^+$ is a direct summand. Note that the projection $\pi_{V^+}$ acts as
	$$ \pi_{V^+}(yhx) = \epsilon(yh)x$$
	for $y \in V^-, h \in \mathcal{H}, x \in V^+$.
	
	For projective $G^+$, we can see that Hopf algebra map $f : \mathcal{U}(G^+) \longrightarrow F/I$ is an embedding because
	$$  (f(1 - \epsilon))^{\otimes n+1}\Delta^n(y) = (1 - \epsilon)^{\otimes n+1}\Delta^n(f(y)),$$
	for all $y \in \mathcal{U}(G^\pm)$ and since $f$ induces an embedding of $\text{Lie}(G^+) \longrightarrow F/I$ (the map $U^+ \longrightarrow A$ factors through $F/I$). More precisely, using the well-behavedness of projective vectorgroups this can be used to prove that 
	$ (1 - \epsilon)^{\otimes n+1}\Delta^nf$ corresponds to an injective map $Y_{n+1}/Y_n \longrightarrow (F/I)^{\otimes n+1}$ for all $n$ (namely, one can use the argumentation of Lemma \ref{Lemma well behaved}) which can be used to prove the injectivity $Y_{n+1} \longrightarrow F/I$ using induction on $n$. Thus, we see that $V^\pm \cong \mathcal{U}(G^\pm)$.
	
	Now, we only need to prove that $F/I \cong V^- \otimes \mathcal{H} \otimes V^+$ as modules since this implies that both spaces are isomorphic as coalgebras.
	It is easy to check that $$\mu \circ (S \otimes \text{Id} \otimes S) \circ (\pi_{V_-} \otimes \text{Id} \otimes \pi_{V_+}) \circ \Delta^2 (yhx)  = \epsilon(x)\epsilon(y)h$$
	for all $y \in U^-, h \in \mathcal{H}, x \in U^+$. This yields a projection $\pi_h : F/I \longrightarrow \mathcal{H}$.
	
	Note that $yhx = \mu(\pi_{V^-} \otimes \pi_h \otimes \pi_{V^+})\Delta^2(yhx)$ for all $y \in V^-, h \in \mathcal{H}, x \in V^+$. So, we conclude that 
	$$ yhx \mapsto (\pi_{V^-} \otimes \pi_h \otimes \pi_{V^+})\Delta^2(yhx)$$
	forms a coalgebra isomorphism $F/I \longrightarrow V^- \otimes \mathcal{H} \otimes V^+$.
\end{proof}




Now, we assume that we are working with the vector groups $\rho^+(G^+)$ and $\rho^-(G^-)$ in order to prove Theorem \ref{thm main 2}.
So, we use the previously considered $F/I$ with $U^+$ the universal representation of $\rho^+(G^+)$ and similarly defined $U^-$.

Consider the Lie subalgebra $L$ of $F/I$ generated by the $g_1,h_1$ for $g \in G^+$ and $h \in G^-$.
We add a grading element of $F/I$, namely take $F/I \otimes \Phi[\zeta]$ with associative product defined from
$$ k \otimes \zeta \cdot l \otimes p(\zeta) = kl \otimes \zeta p(\zeta) + i kl \otimes p(\zeta),$$
for $l$ in the $i$-th grading component of $F/I$ and $p(\zeta) \in \Phi[\zeta]$.
We effectively added a single generator $\zeta$ and relations $[\zeta , l] = i l$ for $i$-graded $l$.
Given this interpretation, we do not write the $\otimes$-sign anymore.
We use $J$ to denote the ideal $$Z(L \oplus \Phi\zeta) \cap \{ z \in (F/I)_0 | \sum_{i + j = 2} g_izg^{-1}_{j} = 0, \forall g \in G^\pm \},$$ i.e., the elements $e$ of the center of the Lie algebra $L \oplus \Phi\zeta$ which are $0$-graded so that $\exp(g) e \exp(g^{-1}) = e$ for all $g \in G^\pm$, with $\exp(g) = \sum_{i = 0}^\infty g_i$ (one can prove that only $g_i$ with $i \le 2$ contribute in the conjugation action).

\begin{lemma}
	The Lie algebra $(L \oplus \Phi \zeta)/J$ is isomorphic to the Lie algebra associated to the pre-Kantor pair $(\rho^+(G^+),\rho^-(G^-))$.
	\begin{proof}
		We first prove that the underlying modules of both Lie algebras are the same. We see that $\text{Lie}(\rho^+(G^+))$ embeds into $F/I$ and the same holds for $G^-$. The $0$-graded part of the Lie algebra can be identified with certain maps $G^\pm \longrightarrow \text{Lie}(G^\pm)$ using $[\delta,g_1] = \delta(g)_1$ and $g_2\delta - g_1\delta g_1 + \delta \times (g^{-1})_2 = - \delta(g)_2 + \delta(g)_1g_1$, and this identification is bijective because we divided the ideal $J$ out.
		To prove that all the $V_{x,y}$ are contained in $L_0$, one uses Lemma \ref{Lemma equations} to prove that $[x,y]$ corresponds exactly to $V_{x,y}$.
		
		One easily checks that the other brackets coincide, proving the isomorphism.
	\end{proof}
\end{lemma}

We call the action
$$ x \mapsto (a \mapsto x \cdot a = x_{(1)}aS(x_{(2)}))$$
of $F/I$ on itself, using Sweedler summation notation, the \textit{adjoint action}.

\begin{lemma}
	The adjoint action of $F/I$ on $L \oplus \Phi \zeta$ induces an action of the pre-Kantor pair $(\rho^+(G^+),\rho^-(G^-))$ on $(L \oplus \Phi \zeta)/J$ which coincides with the usual action of a pre-Kantor pair on its associated Lie algebra.
	\begin{proof}
		We only need to check that certain elements act as expected on $L \oplus \Phi \zeta$ under the adjoint action.
		For the elements $g_1$, this is the case since the Lie algebras are isomorphic.
		For elements $g_3,g_4$ this is easily checked using that $T,P,R$ coincide, on the Lie algebra, with $\mu_{i,j}(g,h)$ for $(i,j) = (3,1), (3,2),(4,2)$ respectively.
		For $g_2$ one notes that the map $\tau_{g,v}$ acts as the same as $g_2v_2 - g_1v_2g_1 + v_2(g^{-1})_2$ for $v_2 \in [G,G]$ and that $g_2 \cdot \delta  = - \delta(g)_2 + (\delta(g)_1)g_1$ is used as part of the definition of $\delta(g)$ for $0$-graded primitive $\delta$.
	\end{proof}
\end{lemma}

\begin{proof}[Proof of Theorem \ref{thm main 2}]	
	We only need to prove the axioms for operator Kantor pairs, as the rest was already proved below the statement of Theorem \ref{thm main 2}.
	All the axioms for operator Kantor pairs follow from the fact that $\nu_{3,2}(x,y) = (P_xy)_1$, $\nu_{2i,i}(x,y) = (Q^\text{grp}_xy)_i$ for $i = 2,3$ and $\nu_{5,2}(x,y) = 0$, using Remark \ref{remark sufficient condition}, since the adjoint action satisfies 
	\[ x \cdot ab = (x_{(1)} \cdot a)(x_{(2)} \cdot b). \qedhere \]
\end{proof}