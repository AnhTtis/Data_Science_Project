% ****** Start of file apssamp.tex ******
%
%   This file is part of the APS files in the REVTeX 4.2 distribution.
%   Version 4.2a of REVTeX, December 2014
%
%   Copyright (c) 2014 The American Physical Society.
%
%   See the REVTeX 4 README file for restrictions and more information.
%
% TeX'ing this file requires that you have AMS-LaTeX 2.0 installed
% as well as the rest of the prerequisites for REVTeX 4.2
%
% See the REVTeX 4 README file
% It also requires running BibTeX. The commands are as follows:
%
%  1)  latex apssamp.tex
%  2)  bibtex apssamp
%  3)  latex apssamp.tex
%  4)  latex apssamp.tex
%
\documentclass[%
 reprint,
%preprint, %for correction
superscriptaddress,
showkeys,
%groupedaddress,
%unsortedaddress,
%runinaddress,
%frontmatterverbose, 
%preprintnumbers,
%nofootinbib,
%nobibnotes,
%bibnotes,
 amsmath,amssymb,
 aps,
%pra,
%prb,
%rmp,
%prstab,
%prstper,
%floatfix,
]{revtex4-2}



\usepackage{graphicx}% Include figure files
\usepackage{dcolumn}% Align table columns on decimal point
\usepackage{bm}% bold math
\usepackage{amsthm}
\usepackage{floatrow}

\usepackage{color}
%\newcommand{\blue}{\color{blue}} %
\newcommand{\red}{\color{red}} %
%\newcommand{\green}{\color{green}} %
%\newcommand{\magenta}{\color{magenta}} %


%\usepackage{hyperref}% add hypertext capabilities
%\usepackage[mathlines]{lineno}% Enable numbering of text and display math
%\linenumbers\relax % Commence numbering lines

%\usepackage[showframe,%Uncomment any one of the following lines to test 
%%scale=0.7, marginratio={1:1, 2:3}, ignoreall,% default settings
%%text={7in,10in},centering,
%%margin=1.5in,
%%total={6.5in,8.75in}, top=1.2in, left=0.9in, includefoot,
%%height=10in,a5paper,hmargin={3cm,0.8in},
%]{geometry}
\newtheorem{lemma}{Lemma}
\newtheorem{definition}{Definition}
\newtheorem{theorem}{Theorem}
\newtheorem{proposition}{Proposition}
\newtheorem{corollary}{Corollary}

\newcommand{\bra}[1]{\langle {#1} |}
\newcommand{\ket}[1]{| {#1} \rangle}
\newcommand{\braket}[2]{\langle {#1} |{#2} \rangle}
\newcommand{\ketbra}[1]{| {#1} \rangle\langle {#1} |}

\newcommand{\lpnorm}[2]{\left\|#2\right\|_{#1}}%Lp norm
\newcommand{\trdist}[1]{\left\|#1\right\|_{\text{tr}}}%Tr distance
\newcommand{\fidelity}[2]{F\left(#1,#2\right)}
\newcommand{\tr}[1]{\text{tr}\left[#1\right]}%trace
\newcommand{\ptr}[2]{\text{tr}_{#1}\left[#2\right]}%partial trace
\newcommand{\diamondnorm}[1]{\left\|#1\right\|_{\diamond}}%Lp norm

\newcommand{\conv}[1]{\text{conv}\left(#1\right)}%trace

\newcommand{\linop}[1]{\mathbf{L}\left(#1\right)}
\newcommand{\pos}[1]{\mathbf{Pos}\left(#1\right)}
\newcommand{\unitary}[1]{\mathbf{U}\left(#1\right)}
\newcommand{\dop}[1]{\mathbf{S}\left(#1\right)}%set of density operator
\newcommand{\puredop}[1]{\mathbf{P}\left(#1\right)}%set of rank-1 density operator

\newcommand{\hh}{\mathcal{H}}%Hilbert space
\newcommand{\vv}{\mathcal{V}}%subspace of Hilbert space
\newcommand{\cd}{\mathbb{C}^d}%Hilbert space
\newcommand{\cdim}[1]{\mathbb{C}^{#1}}
\newcommand{\rdim}[1]{\mathbb{R}^{#1}}
\newcommand{\vspan}[1]{\text{span}\left(#1\right)}
\newcommand{\nn}{\mathbb{N}}%natural number}
\newcommand{\rr}{\mathbb{R}}%real number}
\newcommand{\cc}{\mathbb{C}}%complex number}

\newcommand{\re}[1]{\text{Re}\left(#1\right)}%real part

\newcommand{\phiI}[1]{\phi_{#1}}
\newcommand{\psiI}[1]{\psi_{#1}}

\newcommand{\indicator}[1]{\text{I}\left[#1\right]}
\newcommand{\extpoint}[1]{\text{ext}\left(#1\right)}

\newcommand{\idop}{\mathbb{I}}%identity operator

\newcommand{\eball}[2]{B_{#1}\left(#2\right)}%epsilon ball
\newcommand{\polylog}[1]{polylog\left(#1\right)}

\newcommand{\SEP}{\mathbf{SEP}}

\begin{document}

\preprint{APS/123-QED}

\title{Quadratic reduction of approximation error of pure quantum states\\ by optimal probabilistic state synthesis}% Force line breaks with \\

\author{Seiseki Akibue}
\affiliation{%
NTT Communication Science Labs., NTT Corporation.\\
3--1, Morinosato-Wakamiya, Atsugi, Kanagawa 243-0198, Japan
}%
 %\altaffiliation[Also at ]{Physics Department, XYZ University.}%Lines break automatically or can be forced with \\
\author{Go Kato}%
\affiliation{%
 Advanced ICT Research Institute, NICT.\\
  4--2--1, Nukui-Kitamachi, Koganei, Tokyo 184-8795, Japan
}
\author{Seiichiro Tani}%
\affiliation{%
NTT Communication Science Labs., NTT Corporation.\\
3--1, Morinosato-Wakamiya, Atsugi, Kanagawa 243-0198, Japan
}%

%\date{\today}% It is always \today, today,
             %  but any date may be explicitly specified
\begin{abstract}
When preparing a pure state with a quantum circuit, there is an inevitable coherent error since each unitary gate suffers from the discretized coherent error due to fault-tolerant implementation. A recently proposed approach called probabilistic state synthesis, where the circuit is probabilistically sampled to turn such coherent errors into incoherent ones, is able to reduce the order of the approximation error compared to conventional deterministic synthesis.
In this paper, we demonstrate that the optimal probabilistic synthesis quadratically reduces the approximation error with respect to the trace distance. 
We also show that a deterministic synthesis algorithm can be efficiently converted into a probabilistic one to achieve quadratic error reduction.
To estimate how the error reduction affects the circuit size, we show that probabilistic encoding asymptotically halves the length of the classical bit string required to approximately encode a pure state.
As a byproduct of our technique, we provide a proof for conjectures about the minimum trace distance between an entangled state and the set of separable states and alternate proof about a recently identified coincidence between an entanglement measure and a coherence measure.
\end{abstract}

\keywords{state synthesis, convex approximation, covering number, entanglement, coherence, trace distance}%Use showkeys class option if keyword
                              %display desired
\maketitle

%\tableofcontents
\onecolumngrid
\section{Introduction}
The latest quantum computer applications require various nontrivial quantum states for computation, secure communication, and the fundamental investigation of quantum mechanics. Examples include the ground state (or its approximation) of a Hamiltonian, which is used to compute the ground energy in quantum chemistry \cite{LC19}, a graph state (or a hypergraph state), which has a wide range of applications such as measurement-based quantum computation \cite{BBRN09}, blind computation \cite{AJE09}, and secret-sharing \cite{DB08}, and data-hiding states, which are utilized for quantum data hiding \cite{DLT02} and the study of local indistinguishability \cite{BDCTEPSW99, MSA09}. If we run a subroutine in the computation or communication protocols that prepares a target pure state $\phi$ with a quantum circuit, there is an inevitable coherent error and it is only possible to prepare an approximated pure state $\hat{\phi}_x$ generated by a circuit $\mathcal{C}_x$. This is because each unitary gate in the circuit suffers from the approximation error caused by the decomposition of the unitary gate into a sequence of gates chosen from a finite gate set that is almost perfectly implementable due to the quantum error correction \cite{B15} or the nature of the system \cite{K03}.

To suppress the effect of decoherence or overhead caused by the fault-tolerant implementation of each gate, various {\it state synthesis} algorithms \cite{MMAM21, ZWY21, ZYY21, FTEAYJPF22, AYYS22} have been proposed for minimizing the approximation error or the circuit size. The final goal of these algorithms is to deterministically find one of the best circuits for the approximation (even if an algorithm \cite{ZYY21} succeeds probabilistically). Thus, the minimum approximation error obtained by such {\it deterministic state synthesis} is given by $\min_{x\in X}\trdist{\phi-\hat{\phi}_x}$, where $\trdist{\rho-\sigma}$ is the trace distance between two states $\rho$ and $\sigma$ and $X$ is the label set of pure states generated by circuits with a given cost, e.g., the circuit size, depth, or number of $T$ gates.

While it makes sense to approximate a target pure state by utilizing an approximated state generated by a single circuit, a recently proposed approach called {\it probabilistic state synthesis} probabilistically samples a circuit for the approximation. 
Suppose that the probabilistic algorithm independently samples a circuit $\mathcal{C}_x$ (generating $\hat{\phi}_x$) in accordance with a probability distribution $p(x)$ each time the subroutine preparing $\phi$ is called. Then, each generated state is described by a mixed state $\sum_{x}p(x)\hat{\phi}_{x}$.
This can be interpreted as the transition from coherent errors to incoherent errors \cite{H17, WE16, KN23}, and recent studies have experimentally demonstrated that this transition reduces the approximation error \cite{HNMVMMDSIOHWES21}. A similar technique has been applied to unitary gate synthesis \cite{C17, H17, KLMPP22, SGS23} and quantum simulation \cite{C19}. 

Despite extensive studies, the limitation of probabilistic state synthesis, especially the minimum approximation error $\min_p\trdist{\phi-\sum_{x}p(x)\hat{\phi}_x}$, remains unknown, nor is it clear how to find the optimal probability distribution $p$. Indeed, minimax optimization to compute the minimum approximation error makes all analyses except for a qubit state \cite{S17,LHYFH20,ZYY21conv} quite difficult.


\subsection{Our contributions}
\begin{figure}[h]
\includegraphics[height=.19\textheight]{Bloch2.eps}% Here is how to import EPS art
\caption{\label{fig:Bloch} Quadratic reduction of the approximation error by using probabilistic synthesis.
We assume that we can exactly generate an eigenstate $\hat{\phi}_x$ of the Pauli operators, represented by the six extreme points of the octahedron. We represent the Bloch sphere by a sphere with radius $\frac{1}{2}$, where the trace distance between two quantum states equals the Euclidean distance between the corresponding points. (a) We can compute $\min_p\trdist{\phi-\sum_xp(x)\hat{\phi}_x}=\epsilon^2=\frac{1}{2\sqrt{3}}\left(\sqrt{3}-1\right)$ and $\min_x\trdist{\phi-\hat{\phi}_x}=\epsilon$, where $\phi$ is the furthest state from $\{\hat{\phi}_x\}_{x=1}^6$, represented as a large red point. (b) Suppose that the target state is chosen from $S_G:=\{\phi:\ket{\phi}=\cos t\ket{0}+\sin t\ket{1},t\in\rr\}$, represented by a meridian. We can compute $\min_p\trdist{\phi-\sum_xp(x)\hat{\phi}_x}=\tilde{\epsilon}^2=\frac{1}{2}\left(1-\frac{1}{\sqrt{2}}\right)$ and $\min_{x}\trdist{\phi-\hat{\phi}_x}=\tilde{\epsilon}$, where $\phi$ is the furthest state in $S_G$ from $\{\hat{\phi}_x\}_{x=1}^6$, represented as a large red point.}
\end{figure}

Before presenting our results, we provide intuitive examples demonstrating the capability of probabilistic synthesis in Fig.~\ref{fig:Bloch}. As a generalization of the qubit examples, we obtain the fundamental relationship between the minimum approximation errors obtained by the deterministic synthesis and the probabilistic one in the following theorem.

{\bf Theorem \ref{thm:main1}.} (simplified version)
{\it 
 Let $G$ be a finite subgroup of unitary and antiunitary operators and $S_G:=\{\phi:\forall U\in G,[U,\phi]=0\}$ be the set of pure states invariant under the action of $G$.
  If $\{\hat{\phi}_x\}_{x\in X}=\{U\hat{\phi}_x U^\dag\}_{x\in X}$ for all $U\in G$ with a finite set $X$, it holds that
\begin{equation}
 \max_{\phi\in S_G} \min_p\trdist{\phi-\sum_{x\in X}p(x)\hat{\phi}_x}=\max_{\phi\in S_G}\min_{x\in X}\trdist{\phi-\hat{\phi}_x}^2.
\end{equation}
}

This theorem compares the {\it worst} approximation errors occurring when one synthesizes the target state in a subset $S_G$ that is most difficult to approximate by using $\{\hat{\phi}_x\}_{x}$, which does not need to be a subset of $S_G$. It implies that the optimal probabilistic synthesis at least quadratically reduces the approximation error for any target state $\phi\in S_G$ compared to the worst approximation error caused by the optimal deterministic synthesis. This theorem holds for various $S_G$ by tailoring $G$. For example, $S_G$ coincides with the set of pure states when $G=\{\idop\}$. In such a case, Theorem \ref{thm:main1} is applicable to any  $\{\hat{\phi}_x\}_x$. When $G=\{\idop,\theta\}$, where $\theta$ represents the complex conjugation with respect to the computational basis, $S_G$ coincides with $\{\cos t\ket{0}+\sin t\ket{1}:t\in\rr\}$, which is generated by an axial rotation $\exp(-it\sigma_Y)$ along a fixed axis from $\ket{0}$.
Such one-parameter pure states, more generally known as conjugation-invariant pure states, are often utilized in the optimal parameter estimation \cite{JK22}. For the last example, $S_G$ coincides with the set of pure states in a subspace $\vv$ or its orthogonal complement $\vv_\bot$ when $G=\{\idop,2\Pi_\vv-\idop\}$, where $\Pi_\vv$ is the Hermitian projector whose range is $\vv$. Preparing a state in a particular subspace is an extensively used subroutine in various quantum information processing tasks.

The technique used to prove Theorem \ref{thm:main1} is also applicable to analyzing the minimum trace distance between a general mixed state $\rho$ and a convex hull of $\{\hat{\phi}_x\}_{x}$, both of which are invariant under the action of $G$. For example, we can analyze the entanglement measure by setting $G$ and $\{\hat{\phi}_x\}_{x\in X}$ to be a subset of non-entangling unitary operators and the set of pure product states, respectively.
 As a byproduct, we analytically compute $\min_{\sigma\in\text{SEP}}\trdist{\rho-\sigma}$ when $\rho$ is the isotropic state or the Werner state, which coincides with a conjecture numerically found in \cite{GBK22}. Moreover, we provide alternate succinct proof about a recently identified coincidence between the entanglement measure and coherence measure \cite{JSNCS16}.

We also show an efficient way to convert a deterministic state synthesis algorithm into a probabilistic one that achieves quadratic error reduction. We assume there exists a deterministic state synthesis algorithm $\mathcal{D}$ with

INPUT: a target pure state $\phi$ and target approximation error $\epsilon$,

OUTPUT: circuit $\mathcal{C}_x$ (generating $\hat{\phi}_x$)

\noindent
such that $\trdist{\phi-\hat{\phi}_x}\leq\epsilon$ and a matrix representation of $\hat{\phi}_x$ can be obtained within runtime $\polylog{\frac{1}{\epsilon}}$. Such a deterministic algorithm has been constructed in \cite[Lemma~4.5]{ZWY21}, or we can construct it by combining exact synthesis algorithms allowing the use of arbitrary unitary operations on a constant qubit \cite{MMAM21, ZYY21} with the Solovay-Kitaev algorithm \cite{KBook} to decompose the unitary operations into a sequence of gates in a finite gate set. The efficient conversion is shown in the following theorem.
 
{\bf Theorem \ref{thm:main2}.} (informal version)
{\it 
There exists a probabilistic state synthesis algorithm $\mathcal{P}$ that calls a deterministic state synthesis algorithm $\mathcal{D}$ as an oracle, and has

{\rm INPUT}: a target pure state $\phi$ and target approximation error $\epsilon\in\left(0,1\right)$

{\rm OUTPUT}: circuit $\mathcal{C}_x$ (generating $\hat{\phi}_x$) sampled in accordance with probability distribution $\hat{p}:\hat{X}\rightarrow[0,1]$

\noindent
such that $\mathcal{P}$ satisfies the following properties:
\begin{itemize}
 \item {\rm Efficiency}: $\mathcal{P}$ calls $\mathcal{D}$ constant times, and runtime of $\mathcal{P}$ is $\polylog{\frac{1}{\epsilon}}$,
 
 \item {\rm Quadratic improvement}: The approximation error $\trdist{\phi-\sum_{x\in \hat{X}}\hat{p}(x)\hat{\phi}_x}$ obtained with this algorithm is upper bounded by $\epsilon^2$, whereas $\min_{x\in \hat{X}}\trdist{\phi-\hat{\phi}_{x}}\leq\epsilon$.
\end{itemize}
}

Since probabilistic state synthesis reduces the approximation error, it also reduces the size of a circuit to approximately generate a target state for a given approximation error. However, the reduction rate depends on the circuit's construction, e.g., which gate set and synthesis algorithm are used. There is a universal lower bound on the circuit size obtained by regarding a synthesized circuit as a {\it classical encoding} of a pure state, where a description of a circuit $\mathcal{C}_x$ and the state $\hat{\phi}_x$ generated by $\mathcal{C}_x$ correspond to a label encoding a pure state and the reconstructed state by a decoder, respectively. To analyze how probabilistic synthesis reduces this lower bound, we investigate the minimum length of classical bit strings that encodes a pure state $\phi$ so as to approximately reconstruct the original state as shown in Fig.~\ref{fig:Cencoding}. 

\begin{figure}[h]
\includegraphics[height=.045\textheight]{CQchannel.eps}% Here is how to import EPS art
\caption{\label{fig:Cencoding} Probabilistic encoding of pure state $\phi$ on a $d$-dimensional system using $n$-bit strings and a decoder $\Gamma$ that generates an approximated state $\hat{\rho}$. State $\phi$ is probabilistically encoded in label $x$ in a finite set $X$ in accordance with probability distribution $p_\phi:X\rightarrow[0,1]$. As a special case of probabilistic encoding, we also consider deterministic encoding that utilizes probability distribution $p_\phi:X\rightarrow\{0,1\}$. Note that the length of classical bit strings to represent $x\in X$ is given by $n=\lceil\log_2|X|\rceil$.}
\end{figure}

We compare two types of encoding: (1) deterministic encoding that associates each $\phi$ to a single label $x$, and (2) probabilistic encoding that associates each $\phi$ to a label $x$ in accordance with a probability distribution $p_\phi(x)$. The decoder $\Gamma$ generates, in general, a mixed state $\hat{\rho}_x$ based on the input label $x$. Thus, the reconstructed state in the deterministic and probabilistic encoding is given by $\hat{\rho}=\hat{\rho}_x$ and $\hat{\rho}=\sum_xp_\phi(x)\hat{\rho}_x$, respectively.
In the following theorem, we show that probabilistic encoding exactly halves the bit length required for deterministic encoding in the asymptotic limits.

{\bf Theorem \ref{thm:main3}.} (simplified version)
{\it 
 Let $n_{det}$ (or $n_{prob}$) be the minimum bit length required for deterministic (or probabilistic) encoding that reconstructs a state $\hat{\rho}$ satisfying $\trdist{\phi-\hat{\rho}}\leq\epsilon$ for any pure state $\phi$ in a $d$-dimensional Hilbert space. Then, it holds that
 \begin{equation}
 \lim_{\epsilon\rightarrow0}\frac{n_{prob}}{n_{det}}= \lim_{d\rightarrow\infty}\frac{n_{prob}}{n_{det}}=\frac{1}{2}.
\end{equation}
}

\if0
 While this theorem implies the half reduction of a lower bound on the size of a synthesized circuit, the same reduction of the minimum number of $T$ gates is possible if we assume a conjecture about the minimum number \cite{RS16}.
\fi

\subsection{Technical Outline}
\label{sec:tech}
In the proof of Theorem \ref{thm:main1}, we analyze the minimum approximation error $\min_p\trdist{\phi-\sum_{x}p(x)\hat{\phi}_x}=\min_p\max_{0\leq M\leq\idop}\tr{M(\phi-\sum_{x}p(x)\hat{\phi}_x)}$, which contains minimax optimization by definition. The first tool for the analysis is the strong duality of semidefinite programming. This enables us to formulate the minimum approximation error as a semidefinite program (SDP). The second tool is the symmetrization of an element $M$ of POVM, which appears in the optimization for the minimum approximation error, by exploiting the symmetry of $\phi$ and $\{\hat{\phi}_x\}_x$. This dramatically simplifies the optimization. As discussed in the previous subsection, these techniques can be utilized to analyze the minimum trace distance between a general mixed state and a convex set, such as the set of separable states.

The reformulation of the minimum approximation error as an SDP enables us to compute the optimal probability distribution to achieve it efficiently. By using Theorem \ref{thm:main1}, we can verify that by solving this SDP with $\{\hat{\phi}_x\}_{x\in X}$ satisfying $\max_\phi\min_{x\in X}\trdist{\phi-\hat{\phi}_x}\leq\epsilon$, which is called an {\it $\epsilon$-covering}, we obtain a probability distribution $\hat{p}$ that achieves quadratic reduction of the approximation error, i.e., $\trdist{\phi-\sum_{x\in X}\hat{p}(x)\hat{\phi}_x}\leq\epsilon^2$. However, the size of this SDP is too large to achieve the efficiency shown in Theorem \ref{thm:main2}, since the size $|X|$ of the $\epsilon$-covering is $\left(\frac{1}{\epsilon}\right)^{\Omega(1)}$. This problem can be resolved by proving that any $\hat{\phi}_x$ in the support of the optimal probability distribution in the minimum approximation error is close to $\phi$; more precisely, $\trdist{\phi-\hat{\phi}_x}\leq2\epsilon$. This enables us to construct a modified SDP whose size is independent of $\epsilon$.

Theorem \ref{thm:main3} is obtained by combining Theorem \ref{thm:main1} with the estimation of the minimum size of the $\epsilon$-covering. Due to its prominent role in algorithm design and asymptotic geometric analysis, the order of the minimum size of the $\epsilon$-covering has been well-studied \cite{HLW06, GT13, ABmeetBanach}. However, to obtain Theorem \ref{thm:main3}, we precisely analyze the constant factor in the order, which refines the previous estimations \cite{HLW06, GT13}.

\subsection{Organization}
Section \ref{sec:preliminaries} of this paper introduces the basic notations in quantum information theory and the antiunitary operator, which we use to represent symmetry. In Section \ref{sec:quad}, we show the quadratic reduction of approximation error by probabilistic synthesis. In
Section \ref{sec:PSSalgorithm}, we show an efficient conversion from a deterministic synthesis algorithm into a probabilistic one.
In Section \ref{sec:Cbit}, we show that probabilistic encoding asymptotically halves the length of a classical bit string required to approximately encode a pure state.
As a byproduct of our technique, we provide proof for conjectures and alternate concise proof for a recently found fact about an entanglement measure in Section \ref{sec:application}.
We conclude Section \ref{sec:conclusion} with a brief summary.

\section{Preliminaries}
\label{sec:preliminaries}
We consider only finite-dimensional Hilbert spaces in this paper. The two-dimensional Hilbert space $\cdim{2}$ is called a qubit.
$\linop{\hh}$ and $\pos{\hh}$ represent the set of linear operators and positive semidefinite operators on Hilbert space $\hh$, respectively. 
$\idop\in\pos{\hh}$ represents the identity operator.
For Hermitian operators $A$ and $B$ on $\hh$, $A\geq B$ represents $A-B\in\pos{\hh}$, and $A>B$ means $A-B$ is positive definite.
$\dop{\hh}:=\left\{\rho\in\pos{\hh}:\tr{\rho}=1\right\}$ and $\puredop{\hh}:=\left\{\rho\in\dop{\hh}:\tr{\rho^2}=1\right\}$ represent the set of quantum states and pure states, respectively. Pure state $\phi\in\puredop{\hh}$ is sometimes alternatively represented by complex unit vector $\ket{\phi}\in\hh$ satisfying $\phi=\ketbra{\phi}$. 
\if0
We denote Pauli operators by $\sigma_X$, $\sigma_Y$ and $\sigma_Z$ whose matrix representations with respect to the computational basis are given by
\begin{equation}
 \sigma_X=
 \begin{pmatrix}
 0&1\\
 1&0
\end{pmatrix},
 \sigma_Y=
 \begin{pmatrix}
 0&-i\\
 i&0
\end{pmatrix},
 \sigma_Z=
 \begin{pmatrix}
 1&0\\
 0&-1
\end{pmatrix}.
\end{equation}
\fi

The trace distance $\trdist{\rho-\sigma}$ of two quantum states $\rho,\sigma\in\dop{\hh}$ is defined as $\trdist{M}:=\frac{1}{2}\tr{\sqrt{MM^\dag}}$ for $M\in\linop{\hh}$. It represents the maximum total variation distance between probability distributions obtained by measurements performed on two quantum states. Thus, it satisfies $\trdist{\rho-\sigma}=\max_{0\leq M\leq\idop}\tr{M(\rho-\sigma)}$.
A similar notion measuring the distinguishability of $\rho$ and $\sigma$ is the fidelity function, defined by $\fidelity{\rho}{\sigma}:=\max\tr{\Phi^\rho\Phi^\sigma}$, where $\Phi^\rho\in\puredop{\hh\otimes\hh'}$ is a purification of $\rho$, i.e., $\rho=\ptr{\hh'}{\Phi^\rho}$, and the maximization is taken over all the purifications. Fuchs-van de Graaf inequalities \cite{FG99} provide relationships between the two measures with respect to the distinguishability as follows:
\begin{equation}
\label{ineq:FG}
 1-\sqrt{\fidelity{\rho}{\sigma}}\leq\trdist{\rho-\sigma}\leq\sqrt{1-\fidelity{\rho}{\sigma}}
\end{equation}
holds for any states $\rho,\sigma\in\dop{\hh}$, where the equality of the right inequality holds when $\rho$ and $\sigma$ are pure. 

An operator $A:\hh\rightarrow\hh$ is called antilinear if it satisfies $A(\alpha\ket{\phi}+\beta\ket{\psi})=\alpha^*A\ket{\phi}+\beta^*A\ket{\psi}$, where $\alpha^*$ represents the complex conjugate of $\alpha\in\cc$. The Hermitian adjoint $A^\dag$ of an antilinear operator $A$ is defined by $\bra{\psi}A^\dag\ket{\phi}=\bra{\phi}A\ket{\psi}$. An antilinear operator $U$ is called antiunitary if it satisfies $U^\dag U=\idop$.
An antiunitary operator $\Theta$ is called a conjugation if it satisfies $\Theta^\dag=\Theta$.
An example of a conjugation is the complex conjugation $\theta$ with respect to the computational basis.
Note that for Hermitian operators $M_1$ and $M_2$ and an antilinear operator $A$, the cyclic property $\tr{M_1AM_2A^\dag}=\tr{A^\dag M_1 AM_2}$ of the trace holds.




\section{Quadratic reduction of approximation error}
\label{sec:quad}
We first show the lower bound of the approximation error obtained by the optimal probabilistic synthesis in the following lemma.
\begin{lemma}
\label{lemma:lowerbound}
For a finite set $\{\hat{\phi}_x\}_{x\in X}\subseteq\puredop{\hh}$ of pure states and a pure state $\phi\in\puredop{\hh}$, it holds that
 \begin{equation}
 \label{ineq:lowerbound}
 \min_p\trdist{\phi-\sum_{x\in X}p(x)\hat{\phi}_x}\geq\min_{x\in X}\trdist{\phi-\hat{\phi}_x}^2.
\end{equation}
\end{lemma}
\begin{proof}
Let $p$ minimize the left-hand side of Eq.~\eqref{ineq:lowerbound}. The following calculation completes the proof.
 \begin{eqnarray}
 (L.H.S.)\geq \left(1-\sum_{x\in X}p(x)\tr{\phi\hat{\phi}_x}\right)\geq\min_{x\in X}\left(1-\fidelity{\phi}{\hat{\phi}_x}\right)=(R.H.S.),
\end{eqnarray}
where we use $\trdist{\rho-\sigma}\geq\max_{\phi\in\puredop{\hh}}\tr{\phi(\rho-\sigma)}$ in the first inequality and use the right equality in Ineq.~\eqref{ineq:FG} in the last equality.
\end{proof}

This lemma shows that the reduction rate of the approximation error by using probabilistic synthesis is, at best, quadratic. However, the two examples given in Fig.~\ref{fig:Bloch} indicate that a precisely quadratic reduction is possible if we consider the worst approximation error occurring when we synthesize the target state that is most difficult to approximate in a particular subset of states. This observation is generalized and proven in the following lemma.

\begin{lemma}
\label{lemma:worstcase}
  Let $X$ be a finite set, $G$ be a finite subgroup of unitary and antiunitary operators, and $S_G:=\{\phi\in\puredop{\hh}:\forall U\in G,[U,\phi]=0\}$ be the set of pure states invariant under the action of $G$. 
  If a set $\{\hat{\phi}_x\in\puredop{\hh}\}_{x\in X}$ of pure states is invariant under the action of $G$, i.e., $\{\hat{\phi}_x\}_{x\in X}=\{U\hat{\phi}_x U^\dag\}_{x\in X}$ for all $U\in G$, it holds that
\begin{equation}
 \max_{\phi\in S_G} \min_p\trdist{\phi-\sum_{x\in X}p(x)\hat{\phi}_x}=\max_{\phi\in S_G}\min_{x\in X}\trdist{\phi-\hat{\phi}_x}^2.
\end{equation}
\end{lemma}

Lemma \ref{lemma:worstcase} is a direct consequence of the following lemma for computing the minimum trace distance between a mixed state and a convex subset of mixed states.

\begin{lemma}
\label{lemma:mindist}
 Let $X$ be a finite set and $G$ be a finite subgroup of unitary and antiunitary operators.
Let $P_G$ be the set of positive semidefinite operators invariant under the action of $G$, i.e., $P_G:=\{P\in\pos{\hh}:\forall U\in G,[U,P]=0\}$.
  If $\rho\in P_G\cap\dop{\hh}$ and a set $\{\hat{\rho}_x\in\dop{\hh}\}_{x\in X}$ of mixed states is invariant under the action of $G$, i.e., $\{\hat{\rho}_x\}_{x\in X}=\{U\hat{\rho}_x U^\dag\}_{x\in X}$ for all $U\in G$, it holds that
\begin{equation}
\label{eq:witness1}
 \min_p\trdist{\rho-\sum_{x\in X}p(x)\hat{\rho}_x}=\max_{\substack{0\leq M\leq\idop\\ M\in P_G}}\left(\tr{M\rho}-\max_{x\in X}\tr{M\hat{\rho}_x}\right),
\end{equation}
where the minimization is taken over a probability distribution $p$ over $X$.
In particular, when $\rho$ is a pure state $\phi$, it holds that
 \begin{equation}
\label{eq:witness2}
 \min_p\trdist{\phi-\sum_{x\in X}p(x)\hat{\rho}_x}=\max_{\psi\in P_G\cap\puredop{\hh}}\left(\tr{\psi\phi}-\max_{x\in X}\tr{\psi\hat{\rho}_x}\right).
\end{equation}
\end{lemma}

\begin{proof}
We start from a mixed state $\rho$. By using the minimax theorem, we obtain
\begin{eqnarray}
 (L.H.S.\  of\  Eq.~\eqref{eq:witness1})&=&\min_p\max_{0\leq M\leq\idop}\left(\tr{M\rho}-\sum_{x\in X}p(x)\tr{M\hat{\rho}_x}\right)\\
 &=&\max_{0\leq M\leq\idop}\min_p\left(\tr{M\rho}-\sum_{x\in X}p(x)\tr{M\hat{\rho}_x}\right)\\
\label{eq:subspacelemma1}
 &=&\max_{0\leq M\leq\idop}\left(\tr{M\rho}-\max_{x\in X}\tr{M\hat{\rho}_x}\right).
\end{eqnarray}
This proves $(L.H.S.)\geq(R.H.S.)$.
Let $M$ maximize Eq.~\eqref{eq:subspacelemma1}. Due to the invariance of $\rho$ and $\{\hat{\rho}_x\}_x$ under the action of $G$, we can verify that $U^\dag MU$ also maximizes Eq.~\eqref{eq:subspacelemma1}. By defining $\hat{M}=\frac{1}{|G|}\sum_{U\in G}U^\dag MU$, we obtain
\begin{eqnarray}
 (R.H.S.\  of\  Eq.~\eqref{eq:witness1})&\geq&\tr{\hat{M}\rho}-\max_{x\in X}\tr{\hat{M}\hat{\rho}_x}
=\tr{M\rho}-\max_{x\in X}\left(\frac{1}{|G|}\sum_{U\in G}\tr{MU\hat{\rho}_xU^\dag}\right)\\
&\geq&\tr{M\rho}-\frac{1}{|G|}\sum_{U\in G}\max_{x\in X}\tr{MU\hat{\rho}_xU^\dag}
=\tr{M\rho}-\max_{x\in X}\tr{M\hat{\rho}_x}=(L.H.S.\  of\  Eq.~\eqref{eq:witness1}),
\end{eqnarray}
where we use Eq.~\eqref{eq:subspacelemma1} in the last equality.

When $\rho$ is a pure state $\phi$, we can derive 
\begin{equation}
\label{eq:subspacelemma2}
  (L.H.S.\  of\  Eq.~\eqref{eq:witness2})=\max_{\sigma\in P_G\cap\dop{\hh}}\left(\tr{\sigma\phi}-\max_{x\in X}\tr{\sigma\hat{\rho}_x}\right)
\end{equation}
by the same argument starting from
\begin{equation}
 (L.H.S.\  of\  Eq.~\eqref{eq:witness2})=\min_p\max_{\sigma\in\dop{\hh}}\tr{\sigma\left(\phi-\sum_{x\in X}p(x)\hat{\rho}_x\right)}=\max_{\sigma\in\dop{\hh}}\left(\tr{\sigma\phi}-\max_{x\in X}\tr{\sigma\hat{\rho}_x}\right),
\end{equation}
where we use the fact that the dimension of the eigenspace of $\phi-\sum_{x\in X}p(x)\hat{\rho}_x$ associated with positive eigenvalues is zero or one in the first equality, and use the minimax theorem in the second equality. We complete the proof of Eq.~\eqref{eq:witness2} by using the following observation: When $(L.H.S.\  of\  Eq.~\eqref{eq:witness2})=0$, Eq.~\eqref{eq:witness2} holds since there exists $x\in X$ such that $\hat{\rho}_x=\phi$. When $(L.H.S.\  of\  Eq.~\eqref{eq:witness2})>0$, $\sigma$ maximizing Eq.~\eqref{eq:subspacelemma2} is a pure state. For if $\sigma$ with $\lpnorm{\infty}{\sigma}<1$ maximizes Eq.~\eqref{eq:subspacelemma2}, we can show a contradiction by setting $\rho=\phi$ and $M=\frac{\sigma}{\lpnorm{\infty}{\sigma}}$ in Eq.~\eqref{eq:subspacelemma1}.

\end{proof}

\begin{proof}[Proof of Lemma \ref{lemma:worstcase}]
By setting $\hat{\rho}_x$ in Eq.~\eqref{eq:witness2} to be $\hat{\phi}_x$, we obtain
\begin{eqnarray}
 \max_{\phi\in S_G} \min_p\trdist{\phi-\sum_{x\in X}p(x)\hat{\phi}_x}=\max_{\psi\in S_G}\left(\max_{\phi\in S_G}\tr{\psi\phi}-\max_{x\in X}\tr{\psi\hat{\phi}_x}\right)=1-\min_{\psi\in S_G}\max_{x\in X}\tr{\psi\hat{\phi}_x}=\max_{\psi\in S_G}\min_{x\in X}\trdist{\psi-\hat{\phi}_x}^2,\nonumber\\
 \end{eqnarray}
 where we use the right equality in Ineq.~\eqref{ineq:FG} in the last equality.
\end{proof}



As consequences of Lemma \ref{lemma:worstcase} or Lemma \ref{lemma:mindist}, we obtain the following implications.
\begin{enumerate}
 \item When $G=\{\idop\}$, we obtain $S_G=\puredop{\hh}$. This case is applicable to any $\{\hat{\phi}_x\}_{x\in X}$ and proves the quadratic reduction of the approximation error given in Fig.~\ref{fig:Bloch}(a).
 
  \item When $G=\{\idop,\theta\}$ with the complex conjugation $\theta$, we obtain $S_G=\{\phi\in\puredop{\cdim{2}}:\ket{\phi}=\cos t\ket{0}+\sin t\ket{1},t\in\rr\}$. In this case, the quadratic reduction of the worst approximation error occurring when we synthesize a target state in $S_G$ is possible if $\{\hat{\phi}_x\}_x$ is reflection-symmetric with respect to the XZ-plane in the Bloch representation. This proves the quadratic reduction of the approximation error given in Fig.~\ref{fig:Bloch}(b). In general, conjugation-invariant pure states are often utilized in the optimal parameter estimation \cite{JK22}.
  
 
  \item When $G=\{\idop,2\Pi-\idop\}$ with Hermitian projector $\Pi$ whose range is $\vv$, $S_G=\{\phi\in\puredop{\hh}:\ket{\phi}\in\vv\vee\ket{\phi}\in\vv_{\bot}\}$. In this case, the quadratic reduction of the worst approximation error occurring when we synthesize a target state in $\vv$ is possible if $\{\hat{\phi}_x\}_x$ is reflection-symmetric under the action of $2\Pi-\idop$. This is because 
 \begin{eqnarray}
\max_{\ket{\phi}\in\vv}\min_p\trdist{\phi-\sum_{x\in X}p(x)\hat{\phi}_x}&=&\max_{\psi\in S_G}\left(\max_{\ket{\phi}\in \vv}\tr{\psi\phi}-\max_{x\in X}\tr{\psi\hat{\phi}_x}\right)=\max_{\ket{\psi}\in \vv}\left(\max_{\ket{\phi}\in \vv}\tr{\psi\phi}-\max_{x\in X}\tr{\psi\hat{\phi}_x}\right)\\
&=&1-\min_{\ket{\psi}\in\vv}\max_{x\in X}\tr{\psi\hat{\phi}_x}=\max_{\ket{\phi}\in\vv}\min_{x\in X}\trdist{\phi-\hat{\phi}_x}^2,
\end{eqnarray}
where we use Eq.~\eqref{eq:witness2} in the first equation.
In general, preparing a state in a particular subspace is a widely used subroutine in various quantum information processing tasks.
\end{enumerate}


We obtain the following theorem as a summary of Lemmas \ref{lemma:lowerbound} and \ref{lemma:worstcase}.
\begin{theorem}
\label{thm:main1}
 Let $X$ be a finite set, $G$ be a finite subgroup of unitary and antiunitary operators, and $S_G:=\{\phi\in\puredop{\hh}:\forall U\in G,[U,\phi]=0\}$ be the set of pure states invariant under the action of $G$.
  If $\phi\in S_G$ and $\{\hat{\phi}_x\in\puredop{\hh}\}_{x\in X}=\{U\hat{\phi}_x U^\dag\}_{x\in X}$ for all $U\in G$, it holds that
 \begin{equation}
 \epsilon_\phi^2\leq \min_p\trdist{\phi-\sum_{x\in X}p(x)\hat{\phi}_x}\leq\epsilon_G^2\ \ \ {\rm with}\ \epsilon_\phi=\min_{x\in X}\trdist{\phi-\hat{\phi}_x},\ \epsilon_G=\max_{\psi\in S_G}\min_{x\in X}\trdist{\psi-\hat{\phi}_x}.
\end{equation}
 \end{theorem}

\section{Efficient probabilistic state synthesis algorithm}
\label{sec:PSSalgorithm}
While the probabilistic unitary synthesis achieves the quadratic reduction of the approximation error of a target unitary \cite{SGS23}, it does not imply the quadratic reduction of the approximation error of a generated state, since the generated state is obtained by applying a unitary transformation to a fixed input state such as $\ket{0}$ while the approximation error of a unitary transformation is quantified for the worst input state. Moreover, for probabilistic state synthesis, it may be advantageous to probabilistically mix unitary transformations whose behaviors are totally different, except for the fixed input state.
The algorithm is based on the following proposition and lemma.

\begin{proposition}
\label{prop:SDP}
Let $\rho$ and $\{\hat{\rho}_x\}_{x\in X}$ be a target mixed state and a finite set of mixed states in $\dop{\hh}$, respectively. Then, distance $\min_{p}\trdist{\rho-\sum_{x\in X}p(x)\hat{\rho}_x}$ and the optimal probability distribution $\{p(x)\}_{x\in X}$, which minimizes the distance, can be computed with the following SDP:
\begin{equation}
\label{eq:SDP}
\begin{tabular}{rlcrl}
\multicolumn{2}{c}{\underline{{\rm Primal problem}}} &\ \ \ \ \ \ \ \ \ \ \ \ \ \  
 &\multicolumn{2}{c}{\underline{{\rm Dual problem}}}\\
{\rm maximize:}&$\tr{M\rho}-t$&&{\rm minimize:}&$\tr{Y}$ \\
{\rm subject to:}& $0\leq M\leq\idop$,&&
{\rm subject to:}& $Y\geq0\wedge Y\geq \rho-\sum_{x\in X}p(x)\hat{\rho}_x$,\\
&$\forall x\in X,\tr{M\hat{\rho}_x}\leq t$.&&&$\forall x\in X,p(x)\geq0$,\\
&&&&$\sum_{x\in X}p(x)\leq 1$.
\end{tabular} 
\end{equation}
Note that the strong duality holds in this SDP, i.e., the optimum primal and dual values are equal.

\end{proposition}
\begin{proof}
Recall that for two states $\rho$ and $\sigma$, $\trdist{\rho-\sigma}$ can be computed by the following SDP:
\begin{center}
 \begin{tabular}{rlcrl}
\multicolumn{2}{c}{\underline{Primal problem}} &\ \ \ \ \ \ \ \ \ \ \ \ \ \ \ \ \ \ \ \ \ \ \ \ \ \ \ \ \ \ \ \  & \multicolumn{2}{c}{\underline{Dual problem}}\\
maximize: & $\tr{M(\rho-\sigma)}$ && minimize:& $\tr{Y}$\\
subject to: & $0\leq M\leq\idop$, && subject to: & $Y\geq0\wedge Y\geq \rho-\sigma$.\\
\end{tabular}
\end{center}

A formal SDP and the verification of the strong duality are provided in Appendix \ref{appendix:SDP}.

By extending the dual problem of this SDP to include the minimization of probability distribution $\{p(x)\}_{x\in X}$, we obtain Eq.~\eqref{eq:SDP}. Note that the last condition $\sum_{x\in X}p(x)\leq 1$ in the dual problem is different from the condition $\sum_{x\in X}p(x)=1$ of a probability distribution; however, the optimum dual value can be achieved under the latter condition.
Again, a formal SDP and the verification of the strong duality are provided in Appendix \ref{appendix:SDP}.
\end{proof}

\begin{lemma}
\label{lemma:support}
Let $G$ be a finite subgroup of unitary and antiunitary operators, and $S_G:=\{\phi\in\puredop{\hh}:\forall U\in G,[U,\phi]=0\}$ be the set of pure states invariant under the action of $G$. 
For a positive number $\epsilon>0$, if $\phi\in S_G$ and $\{\hat{\phi}_x\}_{x\in X}$ is a finite $\epsilon$-covering of $S_G$ that is invariant under the action of $G$, i.e., $\max_{\psi\in S_G}\min_{x\in X}\trdist{\psi-\hat{\phi}_x}\leq\epsilon$ and $\{\hat{\phi}_x\}_{x\in X}=\{U\hat{\phi}_xU^\dag\}_{x\in X}$ for all $U\in G$, then
\begin{equation}
\label{eq:support}
\min_{p}\trdist{\phi-\sum_{x\in X}p(x)\hat{\phi}_x}=\min_{\hat{p}}\trdist{\phi-\sum_{x\in \hat{X}}\hat{p}(x)\hat{\phi}_x}
\end{equation}
holds, where $\hat{X}:=\{x\in X:\trdist{\phi-\hat{\phi}_x}\leq2\epsilon\}$ and the minimization of $p$ and $\hat{p}$ are taken over probability distributions over $X$ and $\hat{X}$, respectively.
\end{lemma}
\begin{proof}
We show the nontrivial inequality $(L.H.S.)\geq(R.H.S.)$. We also consider the nontrivial case when $\epsilon<\frac{1}{2}$. By using Lemma \ref{lemma:mindist}, we obtain
 \begin{eqnarray}
 \label{eq:supp1}
 (L.H.S.)&=&\max_{\psi\in S_G}\left(\fidelity{\psi}{\phi}-\max_{x\in X}\fidelity{\psi}{\hat{\phi}_x}\right)=\max_{\psi\in S_G}\left(\min_{x\in X}\trdist{\psi-\hat{\phi}_x}^2-\trdist{\psi-\phi}^2\right),\\
 \label{eq:supp2}
  (R.H.S.)&=&\max_{\psi\in S_G}\left(\fidelity{\psi}{\phi}-\max_{x\in \hat{X}}\fidelity{\psi}{\hat{\phi}_x}\right)=\max_{\psi\in S_G}\left(\min_{x\in \hat{X}}\trdist{\psi-\hat{\phi}_x}^2-\trdist{\psi-\phi}^2\right),
\end{eqnarray}
where we use the right equality in Ineq.~\eqref{ineq:FG} in the last equality.

If $\psi$ in $\eball{\epsilon}{\phi}:=\{\psi\in S_G:\trdist{\psi-\phi}\leq \epsilon\}$ maximizes Eq.~\eqref{eq:supp2}, there exists $x\in \hat{X}$ such that $\min_{x\in X}\trdist{\psi-\hat{\phi}_x}=\trdist{\psi-\hat{\phi}_x}$ since $\min_{x\in X}\trdist{\psi-\hat{\phi}_x}\leq\epsilon$  and $\trdist{\phi-\psi}\leq\epsilon$. This implies $Eq.~\eqref{eq:supp1}\geq Eq.~\eqref{eq:supp2}$.

If $\psi\in S_G\setminus\eball{\epsilon}{\phi}$ maximizes Eq.~\eqref{eq:supp2}, we can show $(R.H.S.)=0$ as follows.
First, if there exists a unitary operator $W\in G$ such that $W\ket{\phi}=\omega\ket{\phi}$,  $W\ket{\psi}=\omega'\ket{\psi}$ and $\omega\neq\omega'$, $\trdist{\phi-\psi}=1$ since $\ket{\phi}$ and $\ket{\psi}$ are eigenvectors associated with distinct eigenvalues. Thus, $(R.H.S.)=0$.
Next, we assume $W\ket{\phi}=\omega\ket{\phi}$ and  $W\ket{\psi}=\omega\ket{\psi}$ for any unitary operator $W\in G$.
If there exist an antiunitary operator $V\in G$, we fix global phases of $\ket{\phi}$ and $\ket{\psi}$ to satisfy the following three conditions: (i) $V\ket{\phi}=\ket{\phi}$, (ii) $V\ket{\psi}=\ket{\psi}$ and (iii) $\braket{\phi}{\psi}\geq0$. For there exist $\ket{\phi'}$ and $\ket{\psi'}$ satisfying $V\ket{\phi'}=\ket{\phi'}$ and $V\ket{\psi'}=\ket{\psi'}$, $\braket{\phi'}{\psi'}=\bra{\phi'}V\ket{\psi'}=\bra{\psi'}V^\dag\ket{\phi'}=\braket{\psi'}{\phi'}$ implies $\braket{\phi'}{\psi'}\in\rr$, and we can choose $\ket{\phi}$ (or $\ket{\psi}$) from $\{\pm\ket{\phi'}\}$ (or $\{\pm\ket{\psi'}\}$) to satisfy the three conditions (i)-(iii). 
In this case, we can verify that $V'\ket{\phi}=\omega\ket{\phi}$ and $V'\ket{\psi}=\omega\ket{\psi}$ for any antiunitary operator $V'\in G$ since $\ket{\phi}$ and $\ket{\psi}$ are eigenvectors of a unitary operator $V'V\in G$ associated with the same eigenvalue. 
If $G$ contains no antiunitary operator, we can set arbitrary global phases for $\ket{\phi}$ and $\ket{\psi}$ such that $\braket{\phi}{\psi}\geq0$ holds.
Then, we can verify that any pure state $\ket{\eta}=a\ket{\phi}+b\ket{\psi}$ with real coefficients $a,b\in\rr$ satisfies $\eta\in S_G$ since $U\ket{\eta}=\omega\ket{\eta}$ for any unitary or antiunitary operator $U\in G$.

Let $\ket{\psi}=\cos t_1\ket{\phi}+\sin t_1\ket{\phi_\bot}$, where $\braket{\phi}{\phi_\bot}=0$ and $t_1\in\left[0,\frac{\pi}{2}\right]$. 
Note that $\phi_\bot\in S_G$ and $\psi\notin\eball{\epsilon}{\phi}$ implies $\sin t_1>\epsilon$. 
Define $\ket{\hat{\phi}}=\cos t_2\ket{\phi}+\sin t_2\ket{\phi_\bot}$ and $\ket{\hat{\phi}_\bot}=-\sin t_2\ket{\phi}+\cos t_2\ket{\phi_\bot}$, where $\sin t_2=\epsilon$ and $t_2\in\left[0,\frac{\pi}{2}\right]$.
Since $\hat{\phi}\in S_G$ and $\{\hat{\phi}_x\}_{x}$ is an $\epsilon$-covering of $S_G$, there exists $x\in X$ such that $\trdist{\hat{\phi}-\hat{\phi}_x}\leq\epsilon$.
Moreover, $x\in\hat{X}$ since $\trdist{\hat{\phi}-\phi}=\epsilon$. 
By letting $(\hat{\phi}+\hat{\phi}_\bot)\ket{\hat{\phi}_x}=a\ket{\hat{\phi}}+\beta\ket{\hat{\phi}_\bot}$ with $a\geq\sqrt{1-\epsilon^2}$ and $\beta\in\cc$, we obtain
\begin{eqnarray}
 \min_{x\in \hat{X}}\trdist{\psi-\hat{\phi}_x}^2&\leq&\trdist{\psi-\hat{\phi}_x}^2=1-\fidelity{\psi}{\hat{\phi}_x}\\
 &=&1-\left|a\cos(t_1-t_2)+\beta\sin(t_1-t_2)\right|^2\\
 &\leq&1-\re{a\cos(t_1-t_2)+\beta\sin(t_1-t_2)}^2\\
 \label{eq:supportproof1}
 &=&1-\left(
\begin{pmatrix}
 a\\\re{\beta}
\end{pmatrix}\cdot
\begin{pmatrix}
 \cos(t_1-t_2)\\\sin(t_1-t_2)
\end{pmatrix}
\right)^2,
\end{eqnarray}
where $\re{\beta}$ represents the real part of a complex number $\beta$. We can draw the possible region of $
\begin{pmatrix}
 a\\\re{\beta}
\end{pmatrix}
$ as the shaded region in Fig.~\ref{fig:support}. By using $0< t_1-t_2\leq\frac{\pi}{2}-t_2$ and Fig.~\ref{fig:support}, we obtain
\begin{equation}
 Eq.~\eqref{eq:supportproof1}\leq1-\left(
\begin{pmatrix}
 \cos t_2\\-\sin t_2
\end{pmatrix}\cdot
\begin{pmatrix}
 \cos(t_1-t_2)\\\sin(t_1-t_2)
\end{pmatrix}
\right)^2=1-\fidelity{\phi}{\psi}=\trdist{\psi-\phi}^2.
\end{equation}
This implies $(R.H.S.)=0$, which completes the proof.

\begin{figure}[h]
\includegraphics[height=.2\textheight]{support.eps}% Here is how to import EPS art
\caption{\label{fig:support} Relationship between $\ket{\psi}$, $\ket{\phi}$, $\ket{\hat{\phi}}$ and $\ket{\hat{\phi}_x}$. $\ket{\psi}$, $\ket{\phi}$, $\ket{\hat{\phi}}$ and $\ket{\hat{\phi}_x}$ correspond to $
\begin{pmatrix}
 \cos(t_1-t_2)\\\sin(t_1-t_2)
\end{pmatrix}
$, $
\begin{pmatrix}
 \cos t_2\\-\sin t_2
\end{pmatrix}
$, $
\begin{pmatrix}
 1\\0
\end{pmatrix}
$ and $
\begin{pmatrix}
 a\\\re{\beta}
\end{pmatrix}
$, respectively.}
\end{figure}

\end{proof}


By combining Proposition \ref{prop:SDP} and Lemma \ref{lemma:support}, we can efficiently convert a deterministic state synthesis algorithm into a probabilistic one. We assume there exists a deterministic state synthesis algorithm $\mathcal{D}$ with

INPUT: a pure state $\phi\in S_G$ and an approximation error $\epsilon\in\left(0,1\right)$

OUTPUT: a set $\{\mathcal{C}_x^{(U)}\}_{U\in G}$ of circuits such that $\mathcal{C}_x^{(U)}$ generates $U\hat{\phi}_xU^\dag$

such that $\trdist{\phi-\hat{\phi}_x}\leq\epsilon$ and a matrix representation of $U\hat{\phi}_xU^\dag$ can be obtained within runtime $\polylog{\frac{1}{\epsilon}}$, where $G$ is a finite subgroup of unitary and antiunitary operators and $S_G$ is the set of pure states invariant under the action of $G$.
When $G=\{\idop\}$, concrete constructions of $\mathcal{D}$ are given in the introduction.
When $G=\{\idop,\theta\}$ with the complex conjugation $\theta$ and every output circuit of $\mathcal{D}$ consists of the Hadamard gate, controlled-NOT, $T$ gate and $S(=T^2)$ gate, we can construct $\mathcal{C}_x^{(\theta)}$ by replacing a gate $T$ and $S$ by a gate sequence $S^3T(=T^*)$ and $S^3(=S^*)$, respectively, in $\mathcal{C}_x^{(\idop)}$ without increasing the number of $T$ gates.

\begin{theorem}
\label{thm:main2}
For a given gate set, there exists a probabilistic state synthesis algorithm $\mathcal{P}$ that calls a deterministic synthesis algorithm $\mathcal{D}$ as an oracle, and has

{\rm INPUT}: a pure state $\phi\in S_G$, an approximation error $\epsilon\in\left(0,1\right)$, and precision $\delta\in\left(0,1\right)$

{\rm OUTPUT}: circuit $\mathcal{C}_x$ (generating $\hat{\rho}_x$) sampled from a set $\hat{X}$ in accordance with probability distribution $\hat{p}:\hat{X}\rightarrow[0,1]$

\noindent
such that $\mathcal{P}$ satisfies the following properties:
\begin{itemize}
 \item {\rm Efficiency}: $\mathcal{P}$ calls $\mathcal{D}$ a constant number of times, and runtime of $\mathcal{P}$ is $poly\left(\log\left(\frac{1}{\epsilon}\right),\log\left(\frac{1}{\delta}\right)\right)$,
 
 \item {\rm Quadratic improvement}: The approximation error $\trdist{\phi-\sum_{x\in\hat{X}}\hat{p}(x)\hat{\rho}_x}$ obtained by $\mathcal{P}$ is upper bounded by $\epsilon^2+\delta$, whereas $\min_{x\in\hat{X}}\trdist{\phi-\hat{\rho}_x}\leq\epsilon$.
 
\end{itemize}
\end{theorem}
\begin{proof}
 In the following, we explicitly construct the algorithm.

\noindent \textbf{Efficient probabilistic state synthesis algorithm}
\begin{enumerate}
\item Set free parameters $c>0$ and $c'>0$ satisfying $c+c'\leq1$.

 \item Generate a list $\{\phi_x\}_{x\in \tilde{X}}\subseteq S_G$ such that for any $\psi\in S_G$, $\min_{x\in \tilde{X}}\trdist{\psi-\phi_x}\leq c\epsilon$ if $\trdist{\phi-\psi}\leq 2\epsilon$. 
  
 \item Call $\mathcal{D}$ to find $\mathcal{C}_x^{(U)}$ generating $U\hat{\phi}_x U^\dag$ such that $\trdist{\phi_x-\hat{\phi}_x}\leq c'\epsilon$ for all $x\in\tilde{X}$ and all $U\in G$.
 
 \item Numerically solve the SDP shown in Proposition \ref{prop:SDP} by setting $\rho=\phi$ and $\{\hat{\rho}_x\}_{x\in\hat{X}}=\{U\hat{\phi}_xU^\dag\}_{x\in\tilde{X},U\in G}$ and obtain a probability distribution $\hat{p}$, which causes the approximation error $\delta$-close to $\min_p\trdist{\phi-\sum_{x\in\hat{X}}p(x)\hat{\rho}_x}$.
 
 \item Sample $\mathcal{C}_x^{(U)}$ in accordance with $\hat{p}$, whose domain is $\hat{X}=\tilde{X}\times G$.
\end{enumerate}
The two properties can be verified as follows:
\begin{itemize}
 \item {\it Efficiency}: We can verify that all steps of the algorithm take $poly\left(\log\left(\frac{1}{\epsilon}\right),\log\left(\frac{1}{\delta}\right)\right)$-time by using the following observations: We can construct a list $\{\phi_x\}_{x\in\tilde{X}}$ whose size is independent to $\epsilon$. From the assumption on $\mathcal{D}$, we can also obtain a list of matrix representations of $\{U\phi_xU^\dag\}_{x\in\tilde{X},U\in G}$ within $\polylog{\frac{1}{\epsilon}}$-time. 
The ellipsoid method guarantees that the optimal value of our SDP can be computed in $poly\left(\log\left(\frac{1}{\epsilon}\right),\log\left(\frac{1}{\delta}\right)\right)$-time within an approximation error $\delta$ \cite{L03}.
  
 \item {\it Quadratic improvement}: The minimum approximation error $\min_p\trdist{\phi-\sum_{x\in\hat{X}}p(x)\hat{\rho}_x}$ is at most $\epsilon^2$ since $\{\hat{\rho}_x\}_{x\in\hat{X}}$ is a subset of an $\epsilon$-covering $\{\hat{\rho}_x\}_{x\in\hat{X}}\cup\{\psi_y\}_y$ of $S_G$, where $\{\psi_y\in S_G\}_y$ is a finite $\epsilon$-covering of $\{\psi\in S_G:\trdist{\phi-\psi}>2\epsilon\}$ and $\trdist{\phi-\psi_y}> 2\epsilon$ for any $y$, $\{\hat{\rho}_x\}_{x\in\hat{X}}\cup\{\psi_y\}_y$ is invariant under the action of $G$, and we can thus apply Theorem \ref{thm:main1} and Lemma \ref{lemma:support}.
 
\end{itemize}

\end{proof}

\section{Halving bit representation of pure states}
\label{sec:Cbit}
We verify that the existence of probabilistic and deterministic encoding given in Fig.~\ref{fig:Cencoding} can be reduced into a property of output states of the decoder $\Gamma$, as shown in the following propositions.
\begin{proposition}
\label{prop:probenc}
 A probabilistic encoding of $\puredop{\cd}$ with approximation error $\epsilon$ and a label set $X$ exists if and only if there exists set $\{\hat{\rho}_x\in\dop{\cd}\}_{x\in X}$ of mixed states satisfying
\begin{equation}
\label{eq:convcovering}
\max_{\phi\in\puredop{\cd}}\min_{p}\trdist{\phi-\sum_{x\in X}p(x)\hat{\rho}_x}\leq\epsilon,
\end{equation}
where the minimization is taken over a probability distribution $p$ over $X$.
\end{proposition}

\begin{proposition}
 A deterministic encoding of $\puredop{\cd}$ with approximation error $\epsilon$ and a label set $X$ exists if and only if there exists set $\{\hat{\rho}_x\in\dop{\cd}\}_{x\in X}$ of mixed states satisfying
\begin{equation}
\label{eq:covering}
\max_{\phi\in\puredop{\cd}}\min_{x\in X}\trdist{\phi-\hat{\rho}_x}\leq\epsilon.
\end{equation}
\end{proposition}
A set $\{\hat{\rho}_x\}_{x\in X}$ of mixed states satisfying Eq.~\eqref{eq:covering} is called an {\it external} $\epsilon$-covering of $\puredop{\cd}$. A set $\{\hat{\rho}_x\in\puredop{\cd}\}_{x\in X}$ of pure states satisfying Eq.~\eqref{eq:covering} is called an {\it internal} $\epsilon$-covering of $\puredop{\cd}$. The minimum size of internal (or external) $\epsilon$-coverings is called the internal (or external) {\it covering number} and denoted by $I_{in}$ (or $I_{ex}$). Note that $I_{ex}\leq I_{in}$ by definition and the minimum bit length $n_{det}$ required for deterministic encodings is equal to $\lceil\log_2 I_{ex}\rceil$.
We obtain the following lemma by using the volume consideration and applying the construction of an $\epsilon$-covering shown in \cite{ABmeetBanach}.
\begin{lemma}
 \label{lemma:detencoding}
For any $\epsilon\in\left(0,\frac{1}{2}\right]$ and an integer $d\geq2$ specified below,
the internal and external covering numbers $I_{in}$ and $I_{ex}$ of an $\epsilon$-covering of $\puredop{\cd}$ are bounded by 
\begin{equation}
2\cdot l(d,2\epsilon)\leq\log_2 I_{ex}\leq \log_2I_{in}\ \ \wedge\  \ 2\cdot l(d,\epsilon)\leq\log_2 I_{in}\leq 2\cdot l(d,\epsilon)+\log_2(5d\ln d),
\end{equation}
where $l(d,\epsilon):=\left(d-1\right)\log_2\left(\frac{1}{\epsilon}\right)$. Moreover, if $d\geq4$, the first lower bound can be strengthened as $2\cdot l(d,\epsilon)\leq\log_2 I_{ex}$.
\end{lemma}
The details of the proof are given in Appendix \ref{appendix:covering}. By combining this lemma with Theorem \ref{thm:main1}, we obtain the following theorem about the minimum bit length.
\begin{theorem}
\label{thm:main3}
For any $\epsilon\in\left(0,\frac{1}{2}\right]$ and an integer $d\geq2$ specified below,
the minimum bit length $n_{det}$ (or $n_{prob}$) of the deterministic (or probabilistic) encoding of $\puredop{\cd}$ with approximation error $\epsilon$ is bounded by 
\begin{eqnarray}
2\cdot l(d,2\epsilon)\leq &n_{det}&\leq 2\cdot l(d,\epsilon)+\log_2(5d\ln d),\\
l(d,\epsilon)-\log_2d\leq&n_{prob}&\leq l(d,\epsilon)+\log_2(5d\ln d),
\end{eqnarray}
where $l(d,\epsilon):=\left(d-1\right)\log_2\left(\frac{1}{\epsilon}\right)$. Moreover, if $d\geq4$, the first lower bound can be strengthened as $2\cdot l(d,\epsilon)\leq n_{det}$.
\end{theorem}
\begin{proof}
 Since the bounds on $n_{det}$ are a direct consequence of Lemma \ref{lemma:detencoding}, we show the bounds on $n_{prob}$.
The upper bound is obtained by setting $\{\hat{\rho}_x\}_{x}$ in Proposition \ref{prop:probenc} to be the minimum internal $\sqrt{\epsilon}$-covering of $\puredop{\cd}$. This is because Theorem \ref{thm:main1} with $G=\{\idop\}$ guarantees that $\{\hat{\rho}_x\}_{x}$ satisfies Eq.~\eqref{eq:convcovering}, and an upper bound on the size of the internal $\sqrt{\epsilon}$-covering is given by Lemma \ref{lemma:detencoding}.
 
Next, we show the lower bound on $n_{prob}$. Let $\{\hat{\rho}_x\in\dop{\cd}\}_{x\in X}$ satisfy Eq.~\eqref{eq:convcovering}. We obtain
\begin{eqnarray}
 \epsilon\geq\max_{\phi\in\puredop{\cd}}\min_p\trdist{\phi-\sum_{x\in X}p(x)\hat{\rho}_x}\geq\max_{\phi\in\puredop{\cd}}\min_p\left(1-\sum_xp(x)\tr{\phi\hat{\rho}_x}\right)=1-\min_{\phi\in\puredop{\cd}}\max_{x\in X}\fidelity{\hat{\rho}_x}{\phi},
\end{eqnarray}
where we use $\trdist{\rho-\sigma}\geq\max_{\phi\in\puredop{\hh}}\tr{\phi(\rho-\sigma)}$ in the second inequality.

By letting $\hat{\rho}_x=\sum_{i=1}^{d}p(i|x)\phiI{i|x}$, we ensure that for any $\phi\in\puredop{\cd}$, there exists $i$ and $x$ such that
\begin{eqnarray}
 1-\epsilon\leq\fidelity{\hat{\rho}_x}{\phi}=\sum_{j=1}^{d}p(j|x)\fidelity{\phiI{j|x}}{\phi}\leq\fidelity{\phiI{i|x}}{\phi}=1-\trdist{\phi-\phiI{i|x}}^2.
\end{eqnarray}
Thus, $\{\phiI{i|x}\}_{i,x}$ is an internal $\sqrt{\epsilon}$-covering of $\puredop{\cd}$. Hence, the lower bound can be obtained by applying Lemma \ref{lemma:detencoding} as $\log_2(|X|d)\geq2\cdot l(d,\sqrt{\epsilon})=l(d,\epsilon)$.

\end{proof}

\if0
While this theorem implies the half reduction of a lower bound on the size of a synthesized circuit, the same reduction for $T$ gates is possible if we assume a conjecture given in \cite{RS16}. The conjecture states that the minimum number of $T$ gates for synthesizing arbitrary single qubit axial rotation is typically given as $3\log_2\left(\frac{1}{\epsilon}\right)+\Theta(1)$.
\fi

\section{Applications for analysis on entanglement measure}
\label{sec:application}
As byproducts of Lemma \ref{lemma:mindist}, we can derive the following implications for the entanglement measure based on the trace distance.
Recall that the set of separable states is defined as follows.
\begin{definition}
 $\SEP:=\{\sigma\in\dop{\cd\otimes\cd}:\sigma=\sum_xp(x)\phi_x\otimes\psi_x\wedge \phi_x,\psi_x\in\puredop{\cd}\}$.
\end{definition}

{\bf Analytical proof for the distance from separable states}: 
Despite its computational hardness \cite{G10}, the computation of $\min_{\sigma\in\SEP}\trdist{\rho-\sigma}$ for a given mixed state $\rho$ plays an important role in the foundation of entanglement theory \cite{CKMR07}.
In \cite{ANT22}, Girardin et al. used a neural network to numerically obtain the following conjectures:
\begin{eqnarray}
\label{eq:entconj}
\min_{\sigma\in \SEP}\trdist{\rho_q^{WER}-\sigma}=q-\frac{1}{2},\ \ 
 \min_{\sigma\in \SEP}\trdist{\rho_q^{ISO}-\sigma}=\frac{d^2-1}{d^2}\left(q-\frac{1}{d+1}\right),
\end{eqnarray}
where $\rho^{WER}_q\in\dop{\cd\otimes\cd}$ is the Werner state defined as $\rho_q^{WER}:=\frac{2(1-q)}{d(d+1)}\Pi_\vee+\frac{2q}{d(d-1)}\Pi_\wedge$ with Hermitian projectors $\Pi_\vee$ and $\Pi_\wedge$ whose ranges are the symmetric subspace and antisymmetric subspace and $\rho^{ISO}_q\in\dop{\cd\otimes\cd}$ is the isotropic state defined as $\rho_q^{ISO}:=\frac{1-q}{d^2}\idop+q\Phi^+$ with $\ket{\Phi^+}=\frac{1}{\sqrt{d}}\sum_{i=0}^{d-1}\ket{ii}$, respectively. 
Since the Werner (or isotropic) state is entangled if and only if $\frac{1}{2}<q\leq1$ (or $\frac{1}{d+1}<q\leq1$), we assume they are entangled in Eqs.~\eqref{eq:entconj}.
Moreover, the Werner (or isotropic) state is the only state that is invariant under the action of $u\otimes u$ (or $u\otimes u^*$), where $u$ is a unitary operator on $\cd$. Thus, by setting $G=\{u\otimes u\}$ (or $G=\{u\otimes u^*\}$), we obtain $\rho^{WER}_q\in P_G=\{a\Pi_\vee+b\Pi_\wedge:a,b\geq0\}$ (or $\rho^{ISO}_q\in P_G=\{a\idop+(b-a)\Phi^+:a,b\geq0\}$), where $P_G:=\{P\in\pos{\hh}:\forall U\in G,[U,P]=0\}$.
By setting $\{\hat{\rho}_x\}_{x}=\{\phi\otimes\psi:\phi,\psi\in\puredop{\cd}\}$, which is invariant under the action of $G$, we can verify that the minimum approximation error in Eq.~\eqref{eq:witness1} coincides with $ \min_{\sigma\in \SEP}\trdist{\rho-\sigma}$.
Rigorously speaking, we cannot directly apply Lemma \ref{lemma:mindist} since neither $G$ nor $\{\hat{\rho}_x\}_{x}$ are finite sets, but their compactness allows us to extend Lemma \ref{lemma:mindist} and apply it. Details are provided in Appendix \ref{appendix:compactextension}.

For the Werner state, we obtain
\begin{eqnarray}
  \min_{\sigma\in \SEP}\trdist{\rho_q^{WER}-\sigma}&=&\max_{0\leq a,b\leq 1}\left(\tr{(a\Pi_\vee+b\Pi_\wedge)\rho_q^{WER}}-\max_{\phi,\psi\in\puredop{\cd}}\tr{(a\Pi_\vee+b\Pi_\wedge)(\phi\otimes\psi)}\right)\\
  &=&\max_{0\leq a,b\leq 1}\left(a(1-q)+bq-\frac{a+b}{2}-\max_{\phi,\psi\in\puredop{\cd}}\frac{a-b}{2}\fidelity{\phi}{\psi}\right)=q-\frac{1}{2},
\end{eqnarray}
where the last equality is obtained by setting $a=0$ and $b=1$.

For the isotropic state, we obtain
\begin{eqnarray}
  \min_{\sigma\in \SEP}\trdist{\rho_q^{ISO}-\sigma}&=&\max_{0\leq a,b\leq 1}\left(\tr{(a\idop+(b-a)\Phi^+)\rho_q^{ISO}}-\max_{\phi,\psi\in\puredop{\cd}}\tr{(a\idop+(b-a)\Phi^+)(\phi\otimes\psi)}\right)\\
  &=&\max_{0\leq a,b\leq 1}\left(a+(b-a)\left(\frac{1-q}{d^2}+q\right)-a-\max_{\phi,\psi\in\puredop{\cd}}\frac{b-a}{d}\fidelity{\phi}{\psi}\right)\\
  &=&\frac{d^2-1}{d^2}\left(q-\frac{1}{d+1}\right),
\end{eqnarray}
where the last equality is obtained by setting $a=0$ and $b=1$.

{\bf Alternate proof about the coincidence between entanglement and coherence measures}: 
Due to its clear operational meaning, the resource measure based on trace distance has been investigated for various resource theories, including entanglement and coherence \cite{BP14}. Lemma \ref{lemma:mindist} provides an alternate concise proof for the following recently identified coincidence between entanglement and coherence measures.
\begin{proposition} \cite[Theorem 3]{JSNCS16}
For pure states $\ket{\Phi}=\sum_{i=0}^{d-1}\alpha_i\ket{ii}$ and $\ket{\phi}=\sum_{i=0}^{d-1}\alpha_i\ket{i}$, it holds that
\begin{equation}
 \min_{\sigma\in \SEP}\trdist{\Phi-\sigma}=\min_{\rho\in I}\trdist{\phi-\rho},
\end{equation}
where $I:=\conv{\{\ketbra{i}\}_{i=0}^{d-1}}$ is called a set of incoherent states and $\{\ket{i}\}_{i=0}^{d-1}$ is an orthonormal basis.
\end{proposition}

\begin{proof}
By using the extended Lemma \ref{lemma:mindist} (see Appendix \ref{appendix:compactextension}) with $G=\{u\otimes u^*,U_{SWAP}(u\otimes u^*)\}$ and $\{\hat{\rho}_x\}_{x}=\{\phi\otimes\psi:\phi,\psi\in\puredop{\cd}\}$, where $U_{SWAP}$ is the swap operator, $u$ is taken from a set $\left\{\sum_{k=0}^{d-1}e^{i\theta_k}\ketbra{k}:\forall k,\theta_k\in\{1,e^{i\frac{2\pi}{3}},e^{i\frac{4\pi}{3}}\}\right\}$ of diagonal unitary operators and $u^*$ is the complex conjugate of $u$ with respect to the basis $\{\ket{k}\}_{k=0}^{d-1}$,
\begin{eqnarray}
 (L.H.S.)=\max_{\ket{\Psi}\in\vspan{\{\ket{ii}\}_i}}\left(\tr{\Psi\Phi}-\max_{\phi,\psi\in\puredop{\cd}}\tr{\Psi(\phi\otimes\psi)}\right)=\max_p\left(\left(\sum_{i=0}^{d-1}\sqrt{p(i)}|\alpha_i|\right)^2-\max_ip(i)\right),
\end{eqnarray}
where we obtain the last equality by setting $\ket{\Psi}=\sum_{i=0}^{d-1}\sqrt{p(i)}\frac{\alpha_i}{|\alpha_i|}\ket{ii}$.

By using Eq.~\eqref{eq:witness2} with $G=\{\idop\}$ and $\{\hat{\rho}_x\}_{x}=\{\ketbra{i}\}_{i=0}^{d-1}$, we obtain
\begin{eqnarray}
 (R.H.S.)=\max_{\psi\in\puredop{\cd}}\left(\tr{\psi\phi}-\max_{i}\bra{i}\psi\ket{i}\right)=\max_p\left(\left(\sum_{i=0}^{d-1}\sqrt{p(i)}|\alpha_i|\right)^2-\max_ip(i)\right),
\end{eqnarray}
where we set $\ket{\psi}=\sum_{i=0}^{d-1}\sqrt{p(i)}\frac{\alpha_i}{|\alpha_i|}\ket{i}$ for the maximization.
This completes the proof.
\end{proof}

\section{Conclusion}
\label{sec:conclusion}
We investigated the limitation of the optimal probabilistic state synthesis and its potential for reducing the size of a synthesized circuit.
As a main result, we verified the tight relationship between the approximation error obtained by the optimal probabilistic state synthesis and the optimal deterministic one.
We also constructed an efficient way to convert a deterministic synthesis algorithm into a probabilistic one that quadratically reduces the approximation error.
To estimate how the error reduction affects the size of a synthesized circuit, we evaluated the length of the classical bit string required to approximately encode a pure state and found that probabilistic encoding asymptotically halves the bit length.

In many quantum resource theories \cite{HO13, BG15, CG19}, a set of quantum states that are relatively easy to prepare, which is termed a {\it free state}, is often convex. A mathematical tool is required to analyze the optimal convex approximation to quantify or approximately prepare a target state using the free state (sometimes assisted by a resource state). The application of our techniques to analyze the entanglement measure outlined in this work demonstrates their good potential for use as such a tool.

\begin{acknowledgments}
We thank Yuki Takeuchi, Yasunari Suzuki, Yasuhiro Takahashi, and Adel Sohbi for their helpful discussions.
This work was partially supported by JST Moonshot R\&D MILLENNIA Program (Grant no.JPMJMS2061).
SA was partially supported by JST, PRESTO Grant nos.JPMJPR2111 and JPMXS0120319794.
GK was supported in part by the Grant-in-Aid for Scientific Research (C) no.20K03779, (C) no.21K03388, and (S) no.18H05237 of JSPS, and CREST (Japan Science and Technology Agency) Grant no.JPMJCR1671.
ST was partially supported by JSPS KAKENHI Grant nos. JP20H05966 and JP22H00522.
\end{acknowledgments}

\bibliography{references.bib}% Produces the bibliography via BibTeX.

\appendix
\section{Formal SDP}
\label{appendix:SDP}
After briefly restating the formal definition of semidefinite programming \cite{WBook}, we show the formal SDPs provided in Proposition \ref{prop:SDP} and verify their strong duality.
\subsection{Preliminaries}
Let $\Xi:\linop{\hh_1}\rightarrow\linop{\hh_2}$ be a linear Hermitian-preserving mapping and $A$ and $B$ be Hermitian operators on $\hh_1$ and $\hh_2$, respectively.
SDP is an optimization problem formally defined with a triple $(\Xi,A,B)$ as follows:
\begin{equation}
\begin{tabular}{rlcrl}
\multicolumn{2}{c}{\underline{{\rm Primal problem}}} &\ \ \ \ \ \ \ \ \ \ \ \ \ \  
 &\multicolumn{2}{c}{\underline{{\rm Dual problem}}}\\
{\rm maximize:}&$\tr{AX}$&&{\rm minimize:}&$\tr{BY}$ \\
{\rm subject to:}& $X\in\pos{\hh_1}$,&&
{\rm subject to:}& $Y\text{\ is\ a\ Hermitian\ operator\ on\ }\hh_2$,\\
&$\Xi(X)=B$&&&$\Xi^\dag(Y)\geq A$,
\end{tabular} 
\end{equation}
where $\Xi^\dag:\linop{\hh_2}\rightarrow\linop{\hh_1}$ is the adjoint of $\Xi$, defined as the linear mapping satisfying $\tr{Y^\dag\Xi(X)}=\tr{(\Xi^\dag(Y))^\dag X}$ for all $X\in\linop{\hh_1}$ and $Y\in\linop{\hh_2}$.
We can easily verify that the solution to the primal problem is smaller than or equal to that of the dual problem. The situation when the two solutions coincide is called a {\it strong duality}. Slater's theorem states that the strong duality holds if either of the following conditions are met:
\begin{enumerate}
 \item The solution to the primal problem is finite, and there exists a Hermitian operator $Y$ on $\hh_2$ such that $\Xi^\dag(Y)>A$.
 \item The solution to the dual problem is finite, and there exists a positive definite operator $X$ on $\hh_1$ such that $\Xi(X)=B$.
\end{enumerate}

\subsection{Formal SDP and verification of strong duality}
A formal SDP to compute $\trdist{\rho-\sigma}$ is defined with a triple $(\Xi,A,B)$ such that
\begin{eqnarray}
A=\left(
\begin{matrix}
 \rho-\sigma&0\\
 0&0
\end{matrix}
\right),\ \ 
B=\idop,\ \ 
\Xi\left(\left(
\begin{matrix}
 M&*\\
 *&M'
\end{matrix}
\right)\right)=
M+M'
\end{eqnarray}
holds for any linear operators $M,M'\in\linop{\hh}$, where the asterisks in the argument to $\Xi$ represent arbitrary linear operators upon which $\Xi$ does not depend, and we identify a linear operator and its matrix representation with respect to a fixed orthonormal basis. The dual problem is obtained by observing that the adjoint of $\Xi$ satisfies
\begin{equation}
 \Xi^\dag\left(Y\right)=
\left(
\begin{matrix}
 Y&0\\
 0&Y
 \end{matrix}
\right)
\end{equation}
for any linear operator $Y\in\linop{\hh}$. We can verify the strong duality of this SDP by observing $\Xi\left(\frac{\idop}{2}\oplus\frac{\idop}{2}\right)=B$ and applying Slater's theorem.

The formal SDP shown in Proposition \ref{prop:SDP} is defined with a triple $(\Xi,A,B)$ such that
\begin{eqnarray}
A&=&\left(
\begin{matrix}
 \rho&0&0&0\\
 0&0&0&0\\
 0&0&0&0\\
 0&0&0&-1
\end{matrix}
\right)\\
B&=&\left(
\begin{matrix}
 \idop&0\\
 0&0
\end{matrix}
\right)\\
  \Xi^\dag\left(\left(
\begin{matrix}
 Y&*\\
*&P
\end{matrix}
\right)\right)&=&
\left(
\begin{matrix}
 Y+\sum_{x\in X}P(x)\hat{\rho}_x&0&0&0\\
 0&Y&0&0\\
 0&0&P&0\\
 0&0&0&-\tr{P}
\end{matrix}
\right)
\end{eqnarray}
holds for any linear operators $Y\in\linop{\hh}$ and $P\in\linop{\cdim{|X|}}$,
where $P(x)$ represents a diagonal element $\bra{x}P\ket{x}$. The primal problem is obtained by observing that the adjoint of $\Xi^\dag$ satisfies
\begin{equation}
 \Xi\left(
 \left(
\begin{matrix}
 M&*&*&*\\
 *&M'&*&*\\
 *&*&Q&*\\
 *&*&*&t
\end{matrix}
\right)
 \right)=
 \left(
\begin{matrix}
 M+M'&0\\
 0&\sum_{x\in X}\tr{M\hat{\rho}_x}\ketbra{x}+Q-t\idop
\end{matrix}
\right)
\end{equation}
for any linear operators $M,M'\in\linop{\hh}$ and $Q\in\linop{\cdim{|X|}}$.
We can verify the strong duality of this SDP by observing $\Xi\left(\frac{\idop}{2}\oplus\frac{\idop}{2}\oplus\frac{\idop}{2}\oplus 1\right)=B$ and applying Slater's theorem.



\section{Proof of Lemma \ref{lemma:detencoding}}
\label{appendix:covering}
\begin{itemize}
 \item {\bf Proof for} $2(d-1)\log_2\left(\frac{1}{\epsilon}\right)\leq\log_2 I_{in}$: Let $\mu$ be the unitarily invariant probability measure on the Borel sets of $\puredop{\cd}$. As shown in subsection \ref{appendix:vol1}, the volume of $\epsilon$-ball $\eball{\epsilon}{\phi}:=\left\{\psi\in\puredop{\cd}:\trdist{\psi-\phi}\leq\epsilon\right\}$ can be calculated as $\mu(\eball{\epsilon}{\phi})=\epsilon^{2(d-1)}$ for any $\phi\in\puredop{\cd}$. Since the union of $\eball{\epsilon}{\hat{\phi}_x}$ covers $\puredop{\cd}$ when $\{\hat{\phi}_x\}_{x\in X}$ is an $\epsilon$-covering of $\puredop{\cd}$, we obtain $\mu(\eball{\epsilon}{\phi})|X|\geq1$.
 
 \item {\bf Proof for} $2(d-1)\log_2\left(\frac{1}{\epsilon}\right)\leq\log_2 I_{ex}$ if $d\geq4$ and $2(d-1)\log_2\left(\frac{1}{2\epsilon}\right)\leq\log_2 I_{ex}$ if $d\in\{2,3\}$: Similar to the previous proof, this bound is a consequence of the upper bound on the volume of $\epsilon$-ball $\eball{\epsilon}{\rho}:=\left\{\psi\in\puredop{\cd}:\trdist{\psi-\rho}\leq\epsilon\right\}$ shown in subsection \ref{appendix:vol2}.

 \item {\bf Proof for} $\log_2 I_{in}\leq 2(d-1)\log_2\left(\frac{1}{\epsilon}\right)+\log_2(5d\ln d)$:
 We construct an internal $\epsilon$-covering ($\epsilon\in(0,1]$) following the proof in \cite[Corollary 5.5]{ABmeetBanach}. The construction is based on the fact that sufficiently many pure states randomly sampled form an $\epsilon$-covering. However, since the probability of a new random pure state residing in the uncovered region decreases when many random $\epsilon$-balls are sampled, it is better to stop sampling a pure state and change the construction strategy.

In the proof, we represent some parameters explicitly, which are tailored to the $\epsilon$-covering with respect to the trace distance.
 Assume $d\geq2$ and let $D=2(d-1)(\geq2)$. Let $\{\phiI{j}\in\puredop{\cd}\}_{j=1}^{J_R}$ be a set of finite randomly sampled pure states with respect to product measure $\mu^{J_R}$. The expected volume of the region not covered by $A:=\cup_{j=1}^{J_R}\eball{\epsilon_R}{\phiI{j}}$ ($0<\epsilon_R\leq 1$) can be calculated as follows:
 \begin{eqnarray}
 &&\int d\mu^{J_R}\mu\left(A^c\right)\nonumber\\
 &&= \int d\mu^{J_R}\int d\mu(\psi)\prod_{j=1}^{J_R}\indicator{\trdist{\psi-\phiI{j}}>\epsilon_R}\nonumber\\
 &&=\int d\mu(\psi)\prod_{j=1}^{J_R} \int d\mu(\phiI{j})\indicator{\trdist{\psi-\phiI{j}}>\epsilon_R}\nonumber\\
 &&=\left(1-\epsilon_R^D\right)^{J_R}\leq\exp\left(-J_R\epsilon_R^D\right),
\end{eqnarray}
where we use Fubini's theorem and Eq.~\eqref{eq:volume_of_eball} in the second and the third equations, respectively. Note that $\indicator{X}\in\{0,1\}$ is the indicator function, i.e., $\indicator{X}=1$ iff $X$ is true.

Thus, there exists $\{\phiI{j}\}_{j=1}^{J_R}$ such that $\mu\left(A^c\right)\leq\exp\left(-J_R\epsilon_R^D\right)$. Pick $\{\psiI{j}\}_{j=1}^{J_P}$ as much as possible such that $\eball{\epsilon_P}{\psiI{j}}$ are disjoint and contained in $A^c$. When $0<\epsilon_P\leq\epsilon_R\leq 1$, we can verify that $\{\phiI{j}\}_{j=1}^{J_R}\cup\{\psiI{j}\}_{j=1}^{J_P}$ is an $(\epsilon_R+\epsilon_P)$-covering  and its size $J:=J_R+J_P$ is upper bounded as
\begin{equation}
 J\leq J_R+\frac{\exp\left(-J_R\epsilon_R^D\right)}{\epsilon_P^D}.
\end{equation}
By setting $J_R=\left\lceil\frac{D}{\epsilon_R^D}\ln\left(\frac{\epsilon_R}{\epsilon_P}\right)\right\rceil$, $\epsilon_P=\frac{\epsilon_R}{x}$, and $\epsilon_R=\frac{x}{1+x}\epsilon$ with $x\geq1$, we obtain the following upper bound:
\begin{equation}
 J\leq\left\lceil \frac{D\ln x}{\epsilon_R^D}\right\rceil+\frac{1}{\epsilon_R^D}
\leq\frac{1}{\epsilon^D}\left\{\left(1+\frac{1}{x}\right)^D(D\ln x+1)+1\right\}=\frac{2d\ln d}{\epsilon^D}\cdot\frac{\alpha(d,x)}{2d\ln d},
\end{equation}
where $\alpha(d,x)=\left(1+\frac{1}{x}\right)^D(D\ln x+1)+1$.
Since $\lim_{d\rightarrow\infty}\frac{\alpha(d,D\ln D)}{2d\ln d}=1$, we obtain that for any $r>2$ there exists $d_0\in\nn$ such that
\begin{equation}
\forall d\geq d_0,\forall\epsilon\in(0,1], J\leq rd(\ln d)\left(\frac{1}{\epsilon}\right)^{2(d-1)}.
\end{equation}
For example, if $r=5$, we can set $d_0=2$.
This completes the proof. 
 
\end{itemize}

\subsection{Volume analysis 1}
\label{appendix:vol1}
We compute the volume of $\eball{\epsilon}{\phi}$ in $\puredop{\cd}$ as follows:
\begin{equation}
\label{eq:volume_of_eball}
\forall d\in\nn,\forall \epsilon\in(0,1],\forall \phi\in\puredop{\cd}, \mu(\eball{\epsilon}{\phi})=\epsilon^{2(d-1)},
\end{equation}
where $\mu$ is the unitarily invariant probability measure on the Borel sets of $\puredop{\cd}$.

When $d=1$, Eq.~\eqref{eq:volume_of_eball} holds. By assuming $d\geq2$, we proceed as follows:
\begin{eqnarray}
&&\mu(\eball{\epsilon}{\phi})\nonumber\\
&&=\mu\left(\left\{\psi\in\puredop{\cd}:\trdist{\ketbra{0}-\psi}\leq\epsilon\right\}\right)\nonumber\\
&&=\mu\left(\left\{\psi\in\puredop{\cd}:|\braket{0}{\psi}|^2\geq1-\epsilon^2\right\}\right)\nonumber\\
&&=\xi\left(\left\{\vec{x}\in\rdim{2d}:\lpnorm{2}{\vec{x}}=1\wedge x_1^2+x_2^2\geq1-\epsilon^2\right\}\right),
\end{eqnarray}
where the first equality uses fixed pure state $\ket{0}$ and the unitary invariance of $\mu$ and the trace distance, the second equality uses Eq.~\eqref{ineq:FG}, and the third equality uses the relationship between $\mu$ and the uniform spherical probability measure $\xi$. Using a spherical coordinate system, we can proceed as follows:
\begin{eqnarray}
\mu(\eball{\epsilon}{\phi})&=&\frac{V(\epsilon)}{V(1)},\\
\text{where}\ V(\epsilon)&:=&\int_{D_\epsilon}\sin^{2d-2}\theta\sin^{2d-3}\phi d\theta d\phi\nonumber\\
&=&4\int_{\hat{D}_\epsilon}\sin^{2d-2}\theta\sin^{2d-3}\phi d\theta d\phi
\end{eqnarray}
and the domain of the integration $D_\epsilon$ is given by $\{(\theta,\phi):\theta,\phi\in(0,\pi),\sin\theta\sin\phi<\epsilon\}$. Since this domain and that of the integrand have reflection symmetries for two lines $\theta=\frac{\pi}{2}$ and $\phi=\frac{\pi}{2}$, it is sufficient to perform the integration in domain $\hat{D}_\epsilon:=\{(\theta,\phi):\theta,\phi\in\left(0,\frac{\pi}{2}\right),\sin\theta\sin\phi<\epsilon\}$. By changing the variables as
$ \begin{pmatrix}
 x\\y
\end{pmatrix}
=
\begin{pmatrix}
 \sin\theta\sin\phi\\
 \sin\theta
\end{pmatrix}$, we obtain
\begin{eqnarray}
 V(\epsilon)&=&4\int_0^{\epsilon}dxx^{2d-3}\int_x^1dy\frac{y}{\sqrt{1-y^2}\sqrt{y^2-x^2}}\nonumber\\
 &=&4\int_0^{\epsilon}dxx^{2d-3}\left[\arcsin\sqrt{\frac{1-y^2}{1-x^2}}\right]_1^x\nonumber\\
 &=&\frac{\pi}{d-1}\epsilon^{2(d-1)}
\end{eqnarray}
for $\epsilon\in[0,1]$. This completes the calculation.

\subsection{Volume analysis 2}
\label{appendix:vol2}
We show the following upper bound on the volume of $\epsilon$-ball $\eball{\epsilon}{\rho}$. For any $\epsilon\in\left(0,\frac{1}{2}\right]$, it holds that
\begin{eqnarray}
\label{appp:volbound1}
  \mu(\eball{\epsilon}{\rho})&\leq&  (2\epsilon)^{2(d-1)}\ \ \text{for}\  d\in\{1,2,3\}\\
  \label{appp:volbound2}
    \mu(\eball{\epsilon}{\rho})&\leq & \epsilon^{2(d-1)}\ \ \ \ \ \text{for}\ d\geq4.
\end{eqnarray}
This bound and $\mu(\eball{\epsilon}{\phi})=\epsilon^{2(d-1)}$ imply that the volume of the $\epsilon$-ball can be maximized by setting its center as a pure state if $d\geq4$, which is contrary to what happens in a qubit ($d=2$), where $\eball{\epsilon}{\rho}$ corresponds to the intersection of the Bloch sphere and a ball centered at $\rho$ and the intersection is maximized not by a ball centered at a point {\it on} the Bloch sphere but by a ball centered at a point {\it inside} the Bloch ball. The qubit case also implies that the condition $d\geq4$ for the second inequality cannot be fully relaxed. $\mu(\eball{\epsilon}{\sigma})=1$ if $\epsilon\geq1-\frac{1}{d}$ with the maximally mixed state $\sigma=\frac{1}{d}\idop$ implies that another condition $\epsilon\in\left(0,\frac{1}{2}\right]$ is also not fully removable. 

{\bf Proof of Eq.~\eqref{appp:volbound1}}: 
By defining $\phi:=\arg\min_{\phi\in\puredop{\cd}}\trdist{\phi-\rho}$, we obtain $\eball{\epsilon}{\rho}\subseteq\eball{2\epsilon}{\phi}$. This is because $\trdist{\psi-\phi}\leq\trdist{\psi-\rho}+\trdist{\phi-\rho}\leq2\trdist{\psi-\rho}<2\epsilon$ for any pure state $\psi\in\eball{\epsilon}{\rho}$. This completes the proof because $\mu(\eball{\epsilon}{\rho})\leq\mu(\eball{2\epsilon}{\phi})=(2\epsilon)^{2(d-1)}$ for any $\epsilon\in\left(0,\frac{1}{2}\right]$.

{\bf Proof of Eq.~\eqref{appp:volbound2}}: 
We assume $d\geq4$.
Let $\rho=\sum_{i=0}^{d-1}p_i\ketbra{i}$, where $\{\ket{i}\}_i$ is a set of eigenvectors of $\rho$ and eigenvalues are arranged in decreasing order, i.e., $p_0\geq p_1\geq \cdots$. Since $\mu(\eball{\epsilon}{\rho})$ depends not on the eigenvectors but on the eigenvalues of $\rho$, it is sufficient to consider only diagonal $\rho$ with respect to a fixed basis. However, it is difficult to precisely calculate $\mu(\eball{\epsilon}{\rho})$ due to a complicated relationship between $\psi$ and the largest eigenvalue of $\psi-\rho$, resulting from the condition $\epsilon\geq\trdist{\psi-\rho}=\lambda_{\max}(\psi-\rho)$. 

We derive the lower bound $f_\rho(\psi)$ of $\trdist{\psi-\rho}$ and use the relationship $\mu(\eball{\epsilon}{\rho})\leq\mu\left(\{\psi:f_\rho(\psi)\leq\epsilon\}\right)$ to show Eq.~\eqref{appp:volbound2}, where $f_\rho$ is a measurable function. Since the simple bound $f_\rho(\psi)=1-\fidelity{\psi}{\rho}$ is too loose to show Eq.~\eqref{appp:volbound2}, we derive a tighter lower bound as follows: 
Let $\Pi$ and $\Pi^\bot$ be the Hermitian projectors on two-dimensional subspace $\vv\supseteq\vspan{\{\ket{0},\ket{\psi}\}}$ and its orthogonal complement, respectively. We then obtain
\begin{eqnarray}
\trdist{\psi-\rho}&\geq&\trdist{\Pi(\psi-\rho)\Pi+\Pi_\bot(\psi-\rho)\Pi_\bot}\nonumber\\
\label{eq:fpsi}
&=&\trdist{\psi-\Pi\rho\Pi}+\trdist{\Pi_\bot\rho\Pi_\bot},
\end{eqnarray}
where we use the monotonicity of the trace distance under a CPTP mapping in the first inequality. Define $f_\rho(\psi)$ as the value in Eq.~\eqref{eq:fpsi}, which can be explicitly written as
 \begin{equation}
 f_\rho(\psi)= \frac{1}{2}\sqrt{(1+p_0-q)^2-4(p_0-q)|\braket{0}{\psi}|^2}+\frac{1}{2}(1-p_0-q),
\end{equation}
where $q=\bra{0_\bot}\rho\ket{0_\bot}$ and $\{\ket{0},\ket{0_\bot}\}$ is an orthonormal basis of $\vv$. The explicit formula implies $f_\rho$ is  uniquely defined (although neither $\vv$ nor $q$ is uniquely defined if $\psi=\ketbra{0}$) and continuous, and thus measurable. 

We assume $\epsilon\in\left(0,\frac{1}{2}\right]$.
Since $\mu(\eball{\epsilon}{\rho})=0$, satisfying Eq.~\eqref{appp:volbound2}, if $p_0\leq1-\epsilon$, we consider the case $p_0>1-\epsilon$. In this case, we obtain
\begin{equation}
 f_\rho(\psi)\leq \epsilon\Leftrightarrow
 |\braket{0}{\psi}|^2\geq \frac{(\epsilon+p_0)(1-q-\epsilon)}{p_0-q},
\end{equation}
where the condition is not trivial, i.e., $\frac{(\epsilon+p_0)(1-q-\epsilon)}{p_0-q}\in(0,1)$.
We calculate an upper bound on $\mu(\eball{\epsilon}{\rho})$ as follows: 
By defining $\unitary{\hh}$ as the set of unitary operators on $\hh$ and $C:\unitary{\cdim{d-1}}\rightarrow\unitary{\cd}$ as $C(U):=\ketbra{0}\oplus U$, we can show that for any unitary operator $U\in\unitary{\cdim{d-1}}$,
\begin{eqnarray}
 &&\mu\left(\{\psi\in\puredop{\cd}:f_\rho(\psi)\leq\epsilon\}\right)\nonumber\\
&&=\mu\left(\{\psi:f_\rho(C(U)\psi C(U)^\dag)\leq\epsilon\}\right)\nonumber\\
\label{eq:uinv}
&&= \int_{\puredop{\cd}}d\mu(\psi)\indicator{|\braket{0}{\psi}|^2\geq\frac{(\epsilon+p_0)(1-q_U-\epsilon)}{p_0-q_U}},
\end{eqnarray}
where we use the unitary invariance of $\mu$ in the first equality, $q_U=\bra{0_\bot} C(U)^\dag\rho C(U)\ket{0_\bot}$, and $\indicator{X}\in\{0,1\}$ is the indicator function, i.e., $\indicator{X}=1$ iff $X$ is true. By integrating Eq.~\eqref{eq:uinv} with respect to the unitarily invariant probability measure on $\unitary{\cdim{d-1}}$ and using Fubini's theorem, we obtain
 \begin{equation}
 \mu\left(\{\psi\in\puredop{\cd}:f_\rho(\psi)<\epsilon\}\right)=\int_{\puredop{\cd}}d\mu(\psi)\int_{\puredop{\cdim{d-1}}}d\mu(\phi) \indicator{|\braket{0}{\psi}|^2\geq\frac{(\epsilon+p_0)(1-\fidelity{\rho}{\phi}-\epsilon)}{p_0-\fidelity{\rho}{\phi}}},
\end{equation}
where $\phi\in\puredop{\cdim{d-1}}$ is identified with a pure state on $\puredop{\cd}$ acting on subspace $\vspan{\{\ket{1},\cdots,\ket{d-1}\}}$. Using Fubini's theorem again and Eq.~\eqref{eq:volume_of_eball}, we can proceed with the calculation:
\begin{eqnarray}
 &&=\int_{\puredop{\cdim{d-1}}}d\mu(\phi)\delta(\fidelity{\rho}{\phi})^{2(d-1)}\nonumber\\
 &&=\int_{\puredop{\cdim{d-1}}}d\mu(\phi)\delta\left(\sum_{i=1}^{d-1}p_i|\braket{i}{\phi}|^2\right)^{2(d-1)}\nonumber\\
 &&\leq\int_{\puredop{\cdim{d-1}}}d\mu(\phi)\delta\left((1-p_0)|\braket{1}{\phi}|^2\right)^{2(d-1)}\nonumber\\
 &&=(d-2)\int_0^1(1-x)^{d-3}\delta((1-p_0)x)^{2(d-1)}dx=:g_{d,\epsilon}(p_0),
\end{eqnarray}
where $\delta(q)=\sqrt{\frac{(\epsilon+q)(p_0+\epsilon-1)}{p_0-q}}$. We use the convexity of $\delta^{2(d-1)}$ and the unitary invariance of $\mu$ in the last inequality, and we use the probability density of $x=|\braket{1}{\phi}|^2$ derived by Eq.~\eqref{eq:volume_of_eball} in the last equality.
To confirm the calculation, we plot a comparison between $\mu(\eball{\epsilon}{\rho})$ and its upper bound $g_{d,\epsilon}(p_0)$ for a particular $\rho$ in Fig.~\ref{fig:boundcomparison}, where we use the following explicit expression of $g_{4,\epsilon}(p_0)$:
\begin{eqnarray}
 \label{eq:g4expression}
 g_{4,\epsilon}(p_0)&=&2(p_0+\epsilon-1)^3\bigg\{\frac{1-6b-ab^2}{2ab^2}+\frac{3(a+1)}{a(b-a)}\left(1-\frac{a}{b-a}\log\frac{b}{a}\right)\bigg\},
\end{eqnarray}
where $a=\frac{2p_0-1}{\epsilon+p_0}$, $b=\frac{p_0}{\epsilon+p_0}$, and $p_0\in(1-\epsilon,1)$. Note that $g_{4,\epsilon}(1)=\epsilon^6=\lim_{p_0\rightarrow1}g_{4,\epsilon}(p_0)$.

\begin{figure}[h]
\includegraphics[height=.23\textheight]{g4.eps}% Here is how to import EPS art
\caption{\label{fig:boundcomparison} Plots of estimated values of $\mu(\eball{\frac{1}{2}}{\rho})$ (dots) and $g_{4,\frac{1}{2}}(p_0)$ (curve) for $\rho=p_0\ketbra{0}+(1-p_0)\ketbra{1}\in\dop{\cdim{4}}$. $\mu(\eball{\frac{1}{2}}{\rho})$ is estimated by uniformly sampling $10^7$ pure states. The plots indicate $\mu(\eball{\frac{1}{2}}{\rho})$ is accurately upper bounded by $g_{4,\frac{1}{2}}(p_0)$.}
\end{figure}

It is sufficient to show that under the two conditions $\epsilon\in\left(0,\frac{1}{2}\right]$ and $d\geq4$,
\begin{equation}
\forall p_0\in(1-\epsilon,1),\frac{dg_{d,\epsilon}}{dp_0}\geq0,
\end{equation}
since $g_{d,\epsilon}(1)=\epsilon^{2(d-1)}$. Since the integrand of $g_{d,\epsilon}$ and its partial derivative with respect to $p_0$ are continuous, we can interchange the partial differential and integral operators:
\begin{eqnarray}
 \frac{dg_{d,\epsilon}}{dp_0}&=&(d-2)\int_0^1(1-x)^{d-3}\frac{\partial}{\partial p_0}\delta((1-p_0)x)^{2(d-1)}dx\nonumber\\
 &=&\alpha_{d,\epsilon}(p_0)\int_0^1\beta_{\epsilon}(p_0,x)\gamma_{d,\epsilon}(p_0,x)dx,
\end{eqnarray}
where $\alpha_{d,\epsilon}(p_0)=(d-2)(d-1)(p_0+\epsilon-1)^{d-2}$, $\beta_{\epsilon}(p_0,x)=-(1-p_0)^2x^2+(1-\epsilon-\epsilon^2-p_0^2)x+(1-\epsilon)\epsilon$ and $\gamma_{d,\epsilon}(p_0,x)=\frac{(1-x)^{d-3}(\epsilon+(1-p_0)x)^{d-2}}{(p_0-(1-p_0)x)^d}$. Since $\alpha_{d,\epsilon}$ and $\gamma_{d,\epsilon}$ are non-negative in the entire considered region $R:=\{(p_0,x):p_0\in(1-\epsilon,1)\wedge x\in[0,1]\}$, $\frac{dg_{d,\epsilon}}{dp_0}\geq0$ if $\beta_{\epsilon}$ is non-negative for all $x\in[0,1]$. However, $\beta_{\epsilon}$ can be negative for some $x\in[0,1]$ if and only if $\beta_{\epsilon}(p_0,1)<0$. Taking account of considered region $R$, it is sufficient to show $\frac{dg_{d,\epsilon}}{dp_0}\geq0$ for all $p_0\in\left(\frac{1+\sqrt{1-4\epsilon^2}}{2},1\right)(\subseteq(1-\epsilon,1))$, where $\beta_{\epsilon}$ can be negative. 

For fixed $p^*\in\left(\frac{1+\sqrt{1-4\epsilon^2}}{2},1\right)$, let $x^*\in(0,1)$ satisfy $\beta_{\epsilon}(p^*,x^*)=0$. Since $\beta_{\epsilon}(p^*,x)$ is monotonically decreasing in $x\geq0$, $x^*$ is uniquely defined, $\beta_{\epsilon}(p^*,x)>0$ if $x\in[0,x^*)$ and $\beta_{\epsilon}(p^*,x)<0$ if $x\in(x^*,1]$. Thus, showing
\begin{equation}
\label{eq:condition1}
 \forall d\geq4,\exists c>0,
 \left\{
\begin{array}{ll}
 \gamma_{d+1,\epsilon}(p^*,x)\geq c\gamma_{d,\epsilon}(p^*,x)&\text{for}\ x\in[0,x^*)\\
  \gamma_{d+1,\epsilon}(p^*,x)\leq c\gamma_{d,\epsilon}(p^*,x)&\text{for}\ x\in(x^*,1]
\end{array}
 \right.
\end{equation}
and
\begin{equation}
\label{eq:condition2}
\frac{dg_{4,\epsilon}}{dp_0}\bigg|_{p_0=p^*}\geq0
\end{equation}
is sufficient for $\forall d\geq4,\frac{dg_{d,\epsilon}}{dp_0}\Big|_{p_0=p^*}\geq0$.
This is because
\begin{eqnarray}
 \alpha_{d+1,\epsilon}(p^*)^{-1}\frac{dg_{d+1,\epsilon}}{dp_0}\bigg|_{p_0=p^*}&=&\int_0^{x^*}\beta_{\epsilon}(p^*,x)\gamma_{d+1,\epsilon}(p^*,x)dx+\int_{x^*}^{1}\beta_{\epsilon}(p^*,x)\gamma_{d+1,\epsilon}(p^*,x)dx\nonumber\\
 &\geq& c\biggl\{\int_0^{x^*}\beta_{\epsilon}(p^*,x)\gamma_{d,\epsilon}(p^*,x)dx+\int_{x^*}^{1}\beta_{\epsilon}(p^*,x)\gamma_{d,\epsilon}(p^*,x)dx\biggr\}\nonumber\\
 &=&c\alpha_{d,\epsilon}(p^*)^{-1}\frac{dg_{d,\epsilon}}{dp_0}\bigg|_{p_0=p^*}
\end{eqnarray}
holds for any $d\geq4.$

First, we show Eq.~\eqref{eq:condition1}. By observing that for any $d\geq4$,
\begin{equation}
\gamma_{d+1,\epsilon}(p^*,x)-c \gamma_{d,\epsilon}(p^*,x)= \gamma_{d,\epsilon}(p^*,x)\left(\frac{(1-x)(\epsilon+(1-p^*)x)}{p^*-(1-p^*)x}-c\right),
\end{equation}
 $h_{\epsilon,p^*}(x):=\frac{(1-x)(\epsilon+(1-p^*)x)}{p^*-(1-p^*)x}$ is monotonically decreasing in $x\in[\hat{x},1]$ and $h_{\epsilon,p^*}(x)\geq h_{\epsilon,p^*}(\hat{x})$ for $x\in[0,\hat{x}]$ with $\hat{x}:=\max\left\{0,1-\frac{2p^*-1}{p^*(1-p^*)}\epsilon\right\}(\leq x^*)$, setting $c=h_{\epsilon,p^*}(x^*)(>0)$ implies Eq.~\eqref{eq:condition1}.

Next, Eq.~\eqref{eq:condition2} can be verified by using the explicit expression in Eq.~\eqref{eq:g4expression}.

\section{Extension of Lemma \ref{lemma:mindist}}
\label{appendix:compactextension}
In this section, we extend Lemma \ref{lemma:mindist} for the case when both $G$ and $X$ are infinite sets. In the following extended lemma, we assume the Borel set $\mathcal{B}_G$ is defined in a subgroup $G$ of unitary and antiunitary operators on $\hh$. Since the Borel set in $G$ is the smallest $\sigma$-algebra containing the topology $\tau$ in $G$, we define $\tau$ as follows. First, we can verify that the set $\vv$ of summations of linear and antilinear operators is a complex vector space. We define a norm in the space as follows: $\lpnorm{}{A+B}:=\max_{\phi\in\puredop{\hh},t\in\rr}\lpnorm{2}{(A+B)e^{it}\ket{\phi}}$, where $A$ and $B$ are a linear and antilinear operator, respectively. Since $\lpnorm{}{A+B}^2=\max_{\phi\in\puredop{\hh}}\left(\lpnorm{2}{A\ket{\phi}}^2+\lpnorm{2}{B\ket{\phi}}^2+2|\bra{\phi}A^\dag B\ket{\phi}|\right)$, we can verify that a nontrivial condition of a norm, $\lpnorm{}{A+B}=0\Leftrightarrow A=B=0$.
Since $G$ is a subset of a normed linear space $\vv$, we can define a topology $\tau$ in $G$ induced by the norm $\lpnorm{}{\cdot}$.
Note that a function $f:G\rightarrow\linop{\hh}$ defined as $f(U):= U^\dag AU$ with $A\in\linop{\hh}$ is continuous.

{\bf Lemma \ref{lemma:mindist}.} (extended version)
{\it
 Let $G$ be a subgroup of unitary and antiunitary operators where a $G$-invariant probability measure $\mu:\mathcal{B}_G\rightarrow[0,1]$ is defined, i.e., $\mu(UE)=\mu(EU)=\mu(E)$ for any $U\in G$ and Borel set $E\in\mathcal{B}_G$.
Let $P_G$ be the set of positive semidefinite operators invariant under the action of $G$, i.e., $P_G:=\{P\in\pos{\hh}:\forall U\in G,[U,P]=0\}$.
  If $\rho\in P_G\cap\dop{\hh}$ and a compact set $\{\hat{\rho}_x\in\dop{\hh}\}_{x\in X}$ of mixed states is invariant under the action of $G$, i.e., $\{\hat{\rho}_x\}_{x\in X}=\{U\hat{\rho}_x U^\dag\}_{x\in X}$ for all $U\in G$, it holds that
\begin{equation}
\label{eq:exwitness1}
 \min_{\sigma\in\conv{\{\hat{\rho}_x\}_{x}}}\trdist{\rho-\sigma}=\max_{\substack{0\leq M\leq\idop\\ M\in P_G}}\left(\tr{M\rho}-\max_{x\in X}\tr{M\hat{\rho}_x}\right),
\end{equation}
where $\conv{A}$ represents the set of finite convex combinations of a set $A$.
In particular, when $\rho$ is a pure state $\phi$, it holds that
 \begin{equation}
 \label{eq:exwitness2}
 \min_{\sigma\in\conv{\{\hat{\rho}_x\}_{x}}}\trdist{\phi-\sigma}=\max_{\psi\in P_G\cap\puredop{\hh}}\left(\tr{\psi\phi}-\max_{x\in X}\tr{\psi\hat{\rho}_x}\right).
\end{equation}
}

\begin{proof}
We start from a mixed state $\rho$. Since we can apply the minimax theorem due to the compactness of $\conv{\{\hat{\rho}_x\}_{x}}$, we obtain
\begin{equation}
\label{eq:exsubspacelemma1}
 (L.H.S.\  of\  Eq.~\eqref{eq:exwitness1})=\max_{0\leq M\leq\idop}\left(\tr{M\rho}-\max_{x\in X}\tr{M\hat{\rho}_x}\right).
\end{equation}
This proves $(L.H.S.)\geq(R.H.S.)$.
Let $M$ maximize Eq.~\eqref{eq:exsubspacelemma1}. Due to the invariance of $\rho$ and $\{\hat{\rho}_x\}_x$ under the action of $G$, we can verify that $U^\dag MU$ also maximizes Eq.~\eqref{eq:exsubspacelemma1}. By defining $\hat{M}=\int_GU^\dag MU d\mu(U)$, we obtain
\begin{eqnarray}
 (R.H.S.\  of\  Eq.~\eqref{eq:exwitness1})&\geq&\tr{\hat{M}\rho}-\max_{x\in X}\tr{\hat{M}\hat{\rho}_x}
=\tr{M\rho}-\max_{x\in X}\left(\int_{G}\tr{MU\hat{\rho}_xU^\dag}d\mu(U)\right)\\
&\geq&\tr{M\rho}-\int_G\max_{x\in X}\tr{MU\hat{\rho}_xU^\dag}d\mu(U)
=\tr{M\rho}-\max_{x\in X}\tr{M\hat{\rho}_x}\\
&=&(L.H.S.\  of\  Eq.~\eqref{eq:exwitness1}).
\end{eqnarray}

We can prove the case when $\rho$ is a pure state $\phi$ in the same way as the proof of Lemma \ref{lemma:mindist}.

\end{proof}

Note that $\mu$ is the counting measure when $G$ is finite. Thus, when $X$ is also finite, this extended lemma is reduced into the original Lemma \ref{lemma:mindist}.
Let $\unitary{\cd}$ and $\nu$ be the set of unitary operators on $\cd$ and the unitarily invariant probability measure on $\unitary{\cd}$, respectively.
In Section \ref{sec:application}, we use this extended lemma in the following three cases:
\begin{enumerate}
 \item When $G=\{u\otimes u: u\in\unitary{\cd}\}$, we can verify that $\mu(E):=\nu\left(\{u\in\unitary{\cd}:u\otimes u\in E\}\right)$ is a $G$-invariant probability measure.
 
  \item When $G=\{u\otimes u^*: u\in\unitary{\cd}\}$, we can verify that $\mu(E):=\nu\left(\{u\in\unitary{\cd}:u\otimes u^*\in E\}\right)$  is a $G$-invariant probability measure.

\end{enumerate}

\end{document}
%
% ****** End of file apssamp.tex ******