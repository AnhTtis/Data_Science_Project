\begin{figure*}
\centering
\setlength\tabcolsep{1pt}
\begin{tabular}{ccc}
\includegraphics[width=0.33\textwidth]{iccv2023AuthorKit/Supp_figures/limitations/snapshot00-min.jpg} &
\includegraphics[width=0.33\textwidth]{iccv2023AuthorKit/Supp_figures/limitations/snapshot01-min.jpg} &
\includegraphics[width=0.33\textwidth]{iccv2023AuthorKit/Supp_figures/limitations/snapshot02-min.jpg} \\
(a) created scene & (b) overly smoothed geometry & (c) stretched geometry
\end{tabular}
\caption{
\textbf{Limitations of our method.}
(a) Our approach creates scenes with compelling textures and complete structure like walls, floor and ceiling.
(b) Our completion stage
(see Section~3.4)
might not be able to inpaint all holes, if no suitable camera pose could be sampled (e.g. small areas behind an object that are close to a wall). The hole is still closed through Poisson reconstruction~\cite{kazhdan2006poisson}, but the geometry may become smoothed.
(c) Our fusion stage
(see Section~3.3)
might not remove all stretched-out faces, because we use fixed thresholds.
}
\label{fig:supp-limitation}
\end{figure*}