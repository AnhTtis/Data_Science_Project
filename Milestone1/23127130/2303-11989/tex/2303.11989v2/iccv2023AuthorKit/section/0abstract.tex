\begin{figure}

\twocolumn[{
\renewcommand\twocolumn[1][]{#1}
\maketitle

\centering
\setlength\tabcolsep{1pt}
\begin{tabular}{cc}
\includegraphics[width=0.766\textwidth]{iccv2023AuthorKit/figures/teaser/teaser_left.pdf} &
\includegraphics[width=0.233\textwidth]{iccv2023AuthorKit/figures/teaser/teaser_right.pdf} \\
(a) 3D Mesh Generation from Text &(b) Rendered Image + Mesh
\end{tabular}
\vspace{1mm}
\caption{
\textbf{Textured 3D mesh generation from text prompts.}
We generate textured 3D meshes from a given text prompt using 2D text-to-image models.
(a) The scene is iteratively created from different viewpoints (marked in blue).
(b) Our generated mesh contains compelling textures and geometry.
We remove the ceiling in the top-down views for better visualization of the scene layout.
}
\vspace{3mm}
\label{fig:teaser}
}]
\end{figure}




\begin{abstract}
We present \OURS \footnote[2]{\url{https://lukashoel.github.io/text-to-room}}, a method for generating room-scale textured 3D meshes from a given text prompt as input.
To this end, we leverage pre-trained 2D text-to-image models to synthesize a sequence of images from different poses.
In order to lift these outputs into a consistent 3D scene representation, we combine monocular depth estimation with a text-conditioned inpainting model.
The core idea of our approach is a tailored viewpoint selection such that the content of each image can be fused into a seamless, textured 3D mesh.
More specifically, we propose a continuous alignment strategy that iteratively fuses scene frames with the existing geometry to create a seamless mesh.
Unlike existing works that focus on generating single objects~\cite{poole2022dreamfusion, lin2022magic3d} or zoom-out trajectories~\cite{fridman2023scenescape} from text, our method generates complete 3D scenes with multiple objects and explicit 3D geometry.
We evaluate our approach using qualitative and quantitative metrics, demonstrating it as the first method to generate room-scale 3D geometry with compelling textures from only text as input.

\end{abstract}