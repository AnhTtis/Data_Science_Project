\begin{figure*}
\centering
\setlength\tabcolsep{1pt}
\begin{tabular}{ccccc}
\includegraphics[width=0.19\textwidth]{iccv2023AuthorKit/Supp_figures/mask_dilation/rendered.jpg} &
\includegraphics[width=0.19\textwidth]{iccv2023AuthorKit/Supp_figures/mask_dilation/mask.jpg} &
\includegraphics[width=0.19\textwidth]{iccv2023AuthorKit/Supp_figures/mask_dilation/inpaint_naive.jpg} &
\includegraphics[width=0.19\textwidth]{iccv2023AuthorKit/Supp_figures/mask_dilation/mask_dilated.jpg} &
\includegraphics[width=0.19\textwidth]{iccv2023AuthorKit/Supp_figures/mask_dilation/inpaint_dilated.jpg} \\
(a) rendered image & (b) rendered mask & (c) inpaint na\"ive & (d) dilated mask & (e) inpaint dilated
\end{tabular}
\caption{
\textbf{Importance of mask dilation during completion.}
In our second stage, we complete the scene mesh by filling in unobserved regions (see Section~\ref{subsec:Trajectory Generation}).
First, we sample camera poses that view such unobserved regions (a).
The unobserved regions can have arbitrary size (b).
Directly inpainting only the masked regions from (b) gives distorted results, because the holes can be too small for reasonable inpainting results (c).
Instead, we inpaint small holes with a classical inpainting method~\cite{telea2004image} and dilate remaining holes to a larger size (d).
The resulting image after inpainting contains more reasonable structure (e).
}
\label{fig:supp-mask-dilation}
\end{figure*}