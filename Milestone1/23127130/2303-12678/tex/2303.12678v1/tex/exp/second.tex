\newcommand{\fabImSize}{.19}
\begin{figure*}[htbp]
	\centering
	\setlength{\tabcolsep}{0.1em}
	\renewcommand{\arraystretch}{.1}
	\begin{tabular}{|c | c |c  | c | c | c |}
		\toprule 
		& \textbf{scene0568\_00} & \textbf{scene0249\_00} & \textbf{scene0435\_00} & \textbf{office3} & \textbf{room0}\\
		\midrule
		\rotatebox{90}{\textbf{Color}} &
		\raisebox{-.5\height}{\includegraphics[width=\fabImSize\linewidth]{im/exp/fab/scannet/0568_color.png}} & %\includegraphics[width=\fabImSize\linewidth]{im/exp/fab/scannet/0164_color.png} &
		\raisebox{-.5\height}{\includegraphics[width=\fabImSize\linewidth]{im/exp/fab/scannet/0249_color.png}} & \raisebox{-.5\height}{\includegraphics[width=\fabImSize\linewidth]{im/exp/fab/scannet/0435_color.png}}
		& %\raisebox{-.5\height}{\includegraphics[width=\fabImSize\linewidth]{im/exp/fab/scannet/0050_color.png}}&
		 \raisebox{-.5\height}{\includegraphics[width=\fabImSize\linewidth]{im/exp/fab/replica/office3_color.png}}&
		 \raisebox{-.5\height}{\includegraphics[width=\fabImSize\linewidth]{im/exp/fab/replica/room0_color.png}}
		\\	
		\midrule
		\rotatebox{90}{\textbf{Saliency}} &
		\raisebox{-.5\height}{\includegraphics[width=\fabImSize\linewidth]{im/exp/fab/scannet/0568_saliency.png}}&
		%\includegraphics[width=\fabImSize\linewidth]{im/exp/fab/scannet/0164_saliency.png}&
		\raisebox{-.5\height}{\includegraphics[width=\fabImSize\linewidth]{im/exp/fab/scannet/0249_saliency.png}}&
		\raisebox{-.5\height}{\includegraphics[width=\fabImSize\linewidth]{im/exp/fab/scannet/0435_saliency.png}}&
		%\raisebox{-.5\height}{\includegraphics[width=\fabImSize\linewidth]{im/exp/fab/scannet/0050_saliency.png}}&
		\raisebox{-.5\height}{\includegraphics[width=\fabImSize\linewidth]{im/exp/fab/replica/office3_saliency.png}}&
		\raisebox{-.5\height}{\includegraphics[width=\fabImSize\linewidth]{im/exp/fab/replica/room0_saliency.png}}
		\\
	
		\hline
	\end{tabular}
	%\captionof{figure}
	\caption{Saliency transfer on the 3D canvas.
       The upper row is the result of the colored mesh.
       The lower row is saliency mesh.}
	\label{fig:saliency}
	\vspace{-.5cm}
\end{figure*}

\subsection{Results for 2D-to-3D transfer}

Except for surface and color, Uni-Fusion also supports continuous mapping of other fabricated data.
%
Therefore, we implement Uni-Fusion on style-transfer data and saliency data to approach a style- and saliency-transfer on the 3D canvas.

In~\cref{fig:style}, we demonstrate artistic painting in 3D canvas. 
We use MSG-Net~\cite{zhang2018multi} to transfer style for each fed frame and use Uni-Fusion to construct the style LIM for the surface style field.
The test scene is office0 of the Replica dataset from one view angle.
$20$ images are used to provide the style and are attached at bottom-left corner correspondingly.
We see that our Uni-Fusion well-constructs style meshes in the taste of the provided style images.
For example, with pure style or abstract painting, the 3D ``canvas'' demonstrates a very close style. 
Our favorite style is in the middle of the fourth row, it even produces a high quality 3D sketch painting. 

In~\cref{fig:saliency}, we demonstrate saliency detection in 3D.
In this figure we select three scenes from ScanNet and two scenes from Replica.
Similarly, we detect saliency on each frame using InSPyReNet\cite{kim2022revisiting} and construct saliency LIM for surface saliency field.

In the second row, high salient regions are colored in yellow. 
Here we see the object that is interest.
This further directs a robot in 3D scene.
For example, in first column, sofa, chair, curtain and television in the room surely gets more attention in daily life.
In the second, a meeting room, obviously the long desk and chairs are the major components.
Also for the bed in the third column of the hotel room, the sofa, desk, lamp, television in the fourth and the sofas and chairs in the last.
