%
% 3_method
%
Our experiments are conducted using UCLA's RoMag device \cite{king2012heat,cheng2015laboratory,vogt2018jump,grannan2022experimental,xu2022thermoelectric}. Figure \ref{fig:Schematics}a) - c) show schematics of the diagnostics and the apparatus. The container consists of two copper end blocks and a stainless steel sidewall. Two sets of sidewalls, $\Gamma = 1.0$ and $2.0$, have been used in this study to investigate MC heat transfer from $10^5 \lesssim Ra \lesssim 10^8$. An external solenoid generates a steady vertical magnetic field, $0 < |\boldsymbol{B}_0| < 800$ Gauss, with a vertical component that varies within $\pm 0.5$\% over the field volume \citep{king2015magnetostrophic}. The tank is placed at the center of the solenoid's bore. A non-inductively wound electrical resistance pad heats the bottom of the lower copper end block at a constant rate, $0 < P \lesssim 2000 \mathrm{W}$, and a thermostated water-cooled heat exchanger maintains a constant temperature at the top of the upper copper end block. This setup can reach up to $Ra \approx 10^9$ and $Ch = 3\times 10^5$ in the $\Gamma = 1.0$ tank.  
%
\begin{figure}[t]
   \centering
    \makebox[\textwidth][c] {\includegraphics[width=0.95\columnwidth]{figures/SWthermistors2.pdf}}
    \makebox[\textwidth][c]{\includegraphics[width=0.95\columnwidth]{figures/Romagphotos.pdf}}
 	  \caption{ a) Thermometry schematic for the $\Gamma = 2R/H = 2.0$ tank ($R = 98.6\, \mathrm{mm}$): there are six thermistors in the top lid at $0.71R$, six thermistors at the midplane sidewall, and six thermistors in the bottom lid at $0.71R$. These thermistors at different heights align with each other azimuthally. The thermometry is in a similar layout on the aspect ratio one tank ($\Gamma = 1.0$). b) Photo of the convection cell of the RoMag device. c) Schematics of the convection cell of the RoMag device. For further device details, see \citet{xu2022thermoelectric}.}
   \label{fig:Schematics}
\end{figure}
%

Twelve thermistors in total are placed inside the top and bottom boundaries about $28.9\ \mathrm{mm}$ radially inwards from the sidewall, as shown in figure \ref{fig:Schematics}a). These end-block thermistors are used to measure the heat transfer efficiency of the system, characterised by the Nusselt number, 
%
\begin{equation}
    Nu = \frac{q H}{\lambda \Delta T},
\end{equation}
%
where $q = 4 P/(\pi D^2)$ is the heat flux, $P$ is the heating power, and $\lambda = 31.4 \ \mathrm{W/(m \cdot K)}$ is the thermal conductivity of gallium \cite{aurnou2018rotating}. The Nusselt number describes the total to conductive heat transfer ratio across the fluid layer, and $Nu=1$ corresponds to the conductive state. The vertical temperature difference across the fluid layer, $\Delta T$, is indirectly controlled by the constant basal heat flux. Six thermistors are attached to the sidewall midplane to detect wall modes and any thermal imprints of the bulk fluid structures at the sidewall. 

We have also conducted direct numerical simulations (DNS) using the finite volume code \textsc{goldfish} \goldfish \citep{horn2015toroidal,shishkina2015thermal, shishkina2016thermal, horn2017prograde, horn2018regimes, horn2019rotating, horn2021tornado}. The nondimensional equations governing quasi-static, Oberbeck-Boussinesq \citep{oberbeck1879warmeleitung,boussinesq1903theorie} magnetoconvection  are:
%
\begin{align}
\hspace*{-.45em}
\vect{\nabla} \cdot \widetilde{\vec{u}} &= 0, \label{eq:NS1}\\
D_{\widetilde{t}} \widetilde{\vec{u}} &= - \vect{\nabla} \widetilde{p}  + \sqrt{\frac{Pr}{Ra}}\, \vect{\nabla}^2 \widetilde{\vec{u}}  + \sqrt{\frac{Pr}{Ra \, Ek^2}}\, \widetilde{\vec{u}} \times  \vec{\hat{e}_z} + \sqrt{ \frac{Ch^2 Pr }{Ra}}\,  \widetilde{\vec{j}}\times \vec{\hat{e}_z} + \widetilde{T} \vec{\hat{e}_z}, \label{eq:NS2} \\
D_{\widetilde{t}} \widetilde{T} &=  \sqrt{\frac{1}{Ra Pr}}\, \vect{\nabla}^2 \widetilde{T},\label{eq:NS3}\\[.7em]
&\hspace*{-3.14em} \left. 
{\begin{array}{ll}
    \hspace*{.2em} 
    \vect{\nabla} \cdot \widetilde{\vec{j}} &= 0 \\[1em]
    \hspace*{1.7em} \widetilde{\vec{j}} &= - \vec{\nabla} \widetilde{\Phi} +  (\widetilde{\vec{u}} \times \vec{\hat{e}_z})
\end{array}
} \right \} \; \vec{\nabla}^2 \widetilde{\Phi} = \vec{\nabla} \cdot (\widetilde{\vec{u}} \times \vec{\hat{e}_z}), 
    \label{eq:NS4}
\end{align}
%
where $\widetilde{\vec{u}}$ denotes nondimensional velocity, $\widetilde{T}$ the temperature, $\widetilde{p}$ the pressure, $\widetilde{\vec{j}}$ the current density, and $\widetilde{\Phi}$ the electrostatic potential. The scales used for the nondimensionalisation are the free-fall speed $U_{f \! f} = \sqrt{\alpha g \Delta T H}$ \citep{aurnou2020connections}, the temperature difference between top and bottom $\Delta T$, the reference pressure $\rho_0 U_{f \! f}^2$, the reference current density $\sigma B_0 U_{f \! f}$ and the reference potential $B_0 H U_{f \! f}$.

Our non-linear DNS solves these equations in a cylindrical domain $(r, \phi, z)$ with  $\Gamma = 2.0$. The sidewall is assumed to be perfectly thermally insulating, $\partial_r T\mid_{\text{r=R}} = 0$, and the top and bottom plates are isothermal with $T_t = -0.5$ and $T_b = 0.5$, respectively.  All boundaries are assumed to be impermeable and no-slip, $\vect{u}\hspace*{-3pt}\mid_{\text{wall}} = 0$, and electrically insulating, $\vect{j}\hspace*{-3pt}\mid_{\text{wall}} = 0$, i.e. the current forms closed loops inside the domain. The DNS control parameters are set to $Pr = 0.027$,  $Ch = 4.0\times 10^4$, and $Ra = \{1.5\times 10^5,\ 2.0\times 10^5,\ 3.0 \times 10^5,\ 4.0 \times 10^5,\ 7.0\times 10^5,\ 1.0\times 10^6,\ 1.5\times 10^6,\ 4.0\times 10^6 \}$. The numerical mesh resolution is $N_r \times N_\phi \times N_z = 240 \times 256 \times 240$. This choice of mesh was verified by running simulations at twice the resolution for the highest $Ra$ for a shorter time, indicating the grid independence of the solution.