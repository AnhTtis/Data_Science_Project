%
% 4_results

\subsection{Comparing onset predictions}
 %
\begin{figure}[ht]
   \centering
    \includegraphics[width=\textwidth]{figures/4panels_v15.pdf}
 	\caption{a) $Nu - Ra$ survey with various $Ch$ from current study and one set of $Ch = 10^6$ data by \citet{king2015magnetostrophic}. Different colors correspond to different $Ch$ values. Different shapes correspond to different aspect ratios and sets of experiments, marked in the legend of panel b). Error bars are shown on the experimental data from the current study. The dashed line is the heat transfer scaling acquired from non-magnetic Rayleigh-B\'enard convection data in the $\Gamma = 1.0$ tank, marked by dark blue circles. See equation (\ref{eq:nurbc}a). Panel b)-d) show ratios of convective heat transfer to conduction ($Nu-1$) versus the reduced bifurcation parameter \citep{zhong1993rotating,horn2017prograde} using three different predicted critical Rayleigh numbers: b) infinite-plane stress-free critical $Ra$ defined in \citet{chandrasekhar1961hydrodynamic}; c) magnetowall mode critical $Ra$ by \citet{busse2008asymptotic}. d) \citet{houchens2002rayleigh}'s critical $Ra$ for magnetowall modes in two different aspect ratios, $\Gamma = 1$ and $2$. The linear fit in panel c) uses data close to onset at $\varepsilon_W \leq 5$. The $\Gamma = 1.0$ data in panel d) are shown in the smaller subplot, which only appears at $\varepsilon \gtrsim 25$.}
   \label{fig:RaRac2}
\end{figure}
%
The validity and accuracy of the onset predictions discussed in Section \ref{sec:2_theory} are tested here using both laboratory and DNS data at $10^5 \lesssim Ra \lesssim 10^8$, $0\lesssim Ch \lesssim 3\times 10^5$, and in $\Gamma = 1.0$ and $2.0$ cylindrical cells. 

Figure \ref{fig:RaRac2}a) shows measurements of heat transfer efficiency, $Nu$, as a function of the buoyancy forcing $Ra$. (All the detailed measurement data are provided in the tables in the Appendix). Vertical error bars based on heat loss and accuracy of the thermometry, are shown in the lab data from this study. We also include \citet{king2015magnetostrophic}'s $Ch=10^6$ data (red stars) made in the same $\Gamma = 1.0$ experimental setup used in this study. 
Figure \ref{fig:RaRac2}b) - d) show convective heat transfer data ($Nu-1$) as a function of supercriticality of the convection, as described by the reduced bifurcation parameter $\varepsilon = (Ra-Ra_{crit})/Ra_{crit}$, following the convention of \citep[][]{ecke1992hopf,zhong1993rotating,horn2017prograde}. Three different $Ra_{crit}$ are examined in panel b) to d): for convection in an infinite plane layer with two rigid boundaries, $Ra_{NS}^\infty$ (\ref{eq:ns}), wall-attached convection, $Ra_{W}^\infty$ (\ref{eq:Busse}), and convection in a cylinder with aspect ratio 1 and 2, $Ra_{cyl}$ (\ref{eq:cyl}). If the $Ra_{crit}$ prediction is accurate, the onset of convection occurs at $\varepsilon = 0$, and $Nu$ and follows an approximate linear scaling for sufficiently small $\varepsilon$. 

Figure \ref{fig:RaRac2}b) presents the convective heat transfer data, $Nu-1$, as a function of the reduced bifurcation parameter $\varepsilon_{NS} = (Ra-Ra_{NS}^\infty)/Ra_{NS}^\infty$ calcuated using eq.~\eqref{eq:ns}. The laboratory-numerical $Nu-1$ data exceeds 0 at $\varepsilon_{NS}<0$. This implies that the $Ra_{NS}^\infty$ predictions do not capture the onset of MC in our system. Moreover, the increased scatter and variation in $Nu$ for different $Ch$ as $\varepsilon_{NS}$ increases suggest a low correlation between the data and the expected linear $\varepsilon_{NS}$ scaling. As the infinite-plane $Ra_{crit}$ is associated with the bulk onset of convection, our heat transfer data implies that the MC flow does not initiate in the fluid bulk.

Figure \ref{fig:RaRac2}c) tests \citet{busse2008asymptotic}'s asymptotic onset predictions for magnetowall modes  as a function of $\varepsilon_W = (Ra-Ra_W)/Ra_W$ calculated using \eqref{eq:Busse}. The nonzero $Nu-1$ data start approximately at the origin of the graph, being only slightly below $\varepsilon_W = 0$ and show a good data collapse up to moderately high supercriticalities of  $\varepsilon \lesssim 10$. Thus, our data provide evidence that in our system the onset of convection occurs in the form of wall-attached modes. This is further quantified by two different linear least-square fits for the nonzero $Nu-1$ data for $\varepsilon_W \leq 5$. The first fit `fit1' (blue line) makes no assumptions on the onset and yields $Nu-1 = 0.419\ \varepsilon_W + 0.490$.  The second `fit2' (red line) is forced to pass through the origin, i.e. it assumes the onset prediction $\varepsilon_W = (Ra-Ra_W)/Ra_W$ is exact and yields $Nu-1 = 0.582\ \varepsilon_W$. Thus, in agreement with the theoretical predictions and as also found by \citet{zurner2020flow}, our MC onset data is consistent with the wall mode predictions. Furthermore, both linear fits hold well up to $\varepsilon_W \lesssim 10$, suggesting that the dynamics and the heat transfer in our system are largely controlled by linear magnetowall modes within this supercriticality range \citep[cf.][]{ecke1992hopf,zhong1993rotating,horn2017prograde,lu2021heat}. 

Figure \ref{fig:RaRac2}d) tests the hybrid theoretical-numerical predictions of \citet{houchens2002rayleigh} for magnetowall modes in cylindrical geometries by plotting $Nu - 1$ versus $\varepsilon_{cyl} = (Ra-Ra_{cyl})/Ra_{cyl}$ calculated using \eqref{eq:cyl}. The underlying assumptions for these predictions best match the experimental and numerical setup. Therefore, they should best capture the measured onset of convection. The $\Gamma = 2.0$ data (green and yellow hues) show an excellent agreement with theory. 
%JMA: RETOOLED: 
The $\Gamma = 1.0$ data (inset, red and orange hues) does not have a low enough supercriticality ($\varepsilon_{cyl} > 28$) to reliably test the exact $Ra_{cyl}$ value. Thus, we are currently unable to disambiguate which magnetowall mode onset predictions are more accurate, even though \citet{houchens2002rayleigh}'s $\Gamma = 1.0$ predictions differ from \citet{busse2008asymptotic}'s by nearly an order of magnitude.
%
\begin{figure}[b!]
    \centering
    \includegraphics[width=\textwidth]{figures/multipanel_v5.pdf}
    \caption{Temperature and velocity fields from $\Gamma = 2.0$ laboratory experiments at $Ch = 4.0\times 10^4$. The vertical columns show cases at a) $Ra = 2.0\times 10^5$, b) $Ra = 4.0\times 10^5$, c) $Ra = 7.0\times 10^5$, d) $Ra = 1.5\times 10^6$, and e) $Ra = 4.0\times 10^6$, respectively ($Ra$ is only approximate for the laboratory cases; their exact values are given in tables \ref{table1} and \ref{table1b}). The first row shows the azimuthal-temporal temperature contours at the midplane interpolated by lab data over $5$ thermal diffusion times, $\tau_\kappa = H^2/\kappa$. The color represents the dimensionless temperature, $(T-\overline T)/\Delta T$, where $\overline T$ is the mean temperature obtained by averaging the top and bottom temperatures. The second row consists of snapshots of the normalised DNS temperature field $\widetilde {T}$ and the third row presents snapshots of normalised DNS vertical velocity fields $\widetilde{u}_z$ at the same moment in time as the temperature field. The vertical black dashed line between b) and c) separates between cases below bulk onset (based on $Ra_{NS}^{\infty}$) to the left and above bulk onset to the right. The vertical green dash-dotted line between c) and d) indicates the transition from an azimuthal mode number of $m=3$ to $m\leq 2$ seen in the laboratory cases.}
   \label{fig:multipanel}
\end{figure}
%

\subsection{Transition to multimodality}
%
\begin{figure}[b!]
   \centering
    \includegraphics[width=\textwidth]{figures/NuRa4e4_b.pdf}
    \caption{a) Azimuthal mode number $m$ as a function of $Ra$ at $Ch = 4\times 10^4$, $\Gamma = 2.0$ case for both lab (green triangles) and numerical data (red triangles). b) Nusselt number $Nu$ as a function of $Ra$ for both laboratory experimental and numerical data. c) $Nu$-$Ra$ Data plotted on detrended curves. Here, $\widetilde{Nu}$ is $Nu$ normalised by the linear fits (\ref{eq:detrendlab}, \ref{eq:detrenddns}) of each respective data set in panel b). In parallel to figure \ref{fig:multipanel}, the black dashed line between b) and c) indicates the predicted $Ra_{NS}^{\infty}$, whereas the green dash-dotted line between c) and d) marks the average $Ra$ between two adjacent laboratory data with $m=3$ and $m=2$. The kinks in $Nu$-$Ra$ data in b) are shown as the fluctuations around $\widetilde{Nu} = 1$ in the $\widetilde{Nu}-Ra$ trend, likely associated with mode switching.}
   \label{fig:modenumber}
\end{figure}
%
We analysed temperature and velocity field data to elucidate both the wall modes and the transition to multi-modal flow at higher supercriticality. Figure \ref{fig:multipanel} shows temperature and velocity fields of five laboratory numerical cases with $Ch \simeq 4\times 10^4$ in the $\Gamma = 2.0$ tank. Each column represents the laboratory case (top row) and its corresponding numerical case (middle and bottom rows) at a similar $Ra$. The detailed parameters are given in table \ref{table1} and \ref{table1b} in the appendix. The top rows show nondimensional temperature $(T-\overline T)/\Delta T$ Hovm\"oller diagrams, a time evolution of the sidewall midplane temperature field. The temperature fields $T$ are interpolated by the mid-plane thermistor data taken $60^\circ$ apart in azimuth. The mean temperature $\overline T$ is measured by averaging the top and bottom boundaries' temperature; the vertical temperature difference across the fluid layer is denoted as $\Delta T$. The second and third rows of figure \ref{fig:multipanel} show snapshots of numerical 3D isosurfaces of the dimensionless temperature fields $\widetilde T = (T-\left<T\right>)/\Delta T$ and corresponding vertical velocity fields $\widetilde{\boldsymbol{u}}_z$ at the same instant in the time. The mean temperature $\left<T\right>$ is calculated by averaging the temperature fields over the entire domain. 

The velocity and temperature fields in figure \ref{fig:multipanel} all show magnetowall modes, manifesting as azimuthally alternating upwelling warm and downwelling cold patches located close to the sidewall. The snapshots of the DNS velocity fields in figures \ref{fig:multipanel}a) and b) further reveal that the magnetowall modes have a two-layer, `nose-like' flow pattern attached to the sidewall with alternating $\pm \widetilde{\boldsymbol{u}}_z$. \citet{liu2018wall} observed similar structures in their simulations in a rectangular box at $Ra = 10^7,\ Ch = 4\times 10^6$. They found that these noses scale approximately with a Shercliff boundary layer thickness $\delta_{Sh} \propto Ch^{-1/4}$ \citep{liu2018wall}. 

Figure \ref{fig:multipanel}a) to b) show that these noses also grow gradually towards the interior as the supercriticality increases, while the interior remains otherwise quiescent. This ``Pinocchio effect"  persist until $Ra \gtrsim Ra_{NS}^\infty$, when the bulk fluid starts convecting from the top and bottom boundaries and then interacts with the inward-extended wall modes. 

Figure \ref{fig:multipanel}c) shows this extending nose behavior for $Ra = 7\times 10^5$, which is just above the bulk onset $Ra > Ra_{NS}^\infty$. The DNS velocity field visualises how two noses with positive/negative $u_z$ (pink/blue) connect across the entire diameter of the tank via the convecting upwelling/downwelling fluid in the interior. The laboratory and numerical temperature field on the sidewall agree perfectly and show that close to the sidewall the wall modes are virtually unaffected by this interior dynamics.
In total there are six alternating cold and hot patches along the sidewall azimuth, i.e., the azimuthal mode number is $m=3$. The magnetowall mode number $m$ is defined as the number of repeating azimuthal structures along the lateral surface.

Figure \ref{fig:multipanel}d) and e) show that these nonlinear interactions become more complicated and chaotic as $Ra$ increases further. The nose-like structures interact and impinge on each other. The nonlinear behaviour also affects the flow close to the sidewall as visible in the temperature Hovm\"oller diagram from sidewall thermometry in d) for $Ra = 1.5\times 10^6$ and even more so in e) for $Ra = 4.0\times 10^6$.

For $Ra = 4.0\times 10^6$ (Figure \ref{fig:multipanel}e), the experimental temperature Hovm\"oller diagram shows that the magnetowall modes are transient between $m=1$ and $m=2$ in a chaotic sequence. The velocity field of the DNS further demonstrates that the bulk flow dominates the dynamics, and, hence the flow for $Ra = 4.0\times 10^6$ significantly differs from the ones at lower $Ra$ shown in figure \ref{fig:multipanel}a) to d). 

There is, however, a small discrepancy between the number of azimuthal structures between the lab and DNS data, being $m = 2$ and $m = 3$, respectively for $Ra = 1.5\times 10^6$ (Figure \ref{fig:multipanel}d). This may be because $m$ is sensitive to small changes in $Ra$ and $\Gamma$, and there are slight differences in parameters between the lab and the DNS, or because of the sidewall boundary conditions which are not perfectly adiabiatic in the experiment. It is also possible that the DNS snapshots do not capture fully equilibrated flow patterns whilst the lab experiments revealed more averaged dynamics of MC, as, unlike the DNS, they can be run for many thermal diffusion times.

Figure \ref{fig:modenumber}a) shows how the time-averaged azimuthal mode numbers $m$ observed in the laboratory experiment and the DNS velocity fields depend on the Rayleigh number $Ra$. The $m$ values generally decrease with increasing $Ra$, which qualitatively agrees with previous studies \citep{liu2018wall,akhmedagaev2020turbulent,zurner2020flow}. For $Ra < 10^4$, we only present DNS data in this plot and no lab data due to a combination of both precision and spatial aliasing issues of the sidewall thermistor array. The temperature variation between each wall mode structure near the midplane is $\lesssim 0.2\ \mathrm{K}$, which is too small to be resolved by our thermometers. Additionally, with only six thermistors evenly spaced at the azimuth, we can only resolve up to $m = 3$ according to the Nyquist-Shannon sampling theorem. Thus, even though the first-row temperature contour in figure \ref{fig:modenumber}b) shows an $m = 3$ structure, it was omitted in figure \ref{fig:modenumber}a) and only the $m = 4$ from the velocity field from the DNS is shown.

The changes in mode number also affect the global heat transport. Figure \ref{fig:modenumber}b) shows that $Nu$ increases monotonically with $Ra$, but not at a constant rate. Instead, kinks exist in the $Nu$-$Ra$ trends in both lab experiments and DNS for $Ch = 4\times 10^4$ (and $Ch = 10^5$, see figure \ref{fig:RaRac2}a), a phenomenon which has not been reported in previous MC experiments \citep{cioni2000effect, zurner2020flow}. To further investigate this behaviour, we normalised $Nu$ by power laws obtained by separate fits to the $Ch = 4 \times 10^4$ laboratory and DNS $Nu$-$Ra$ data sets.  For the laboratory data, the best fit is
    %
    \begin{equation}
       \widetilde{Nu} = Nu/(0.0029Ra^{0.493}),
     \label{eq:detrendlab}
    \end{equation}
    %
    whereas for the DNS, it is found that
    %
    \begin{equation}
        \widetilde{Nu} = Nu/(0.0014Ra^{0.541}).
        \label{eq:detrenddns}
    \end{equation}
    %

Figure \ref{fig:modenumber}c) shows the normalised $\widetilde{Nu}$. The non-monotonicity of the trend manifests as fluctuations around $\widetilde{Nu} \approx 1$ with an amplitude of approximately $0.05$. The increase after the first local minimum in the experimental $\widetilde{Nu}$ data curve coincides with the bulk onset, $Ra = Ra_{NS}^{\infty}$, and is marked by the vertical black dashed line. This suggests that bulk convection enhances heat transfer efficiency. The decrease after the first local maximum in the experimental $\widetilde{Nu}$ data curve coincides with the change of mode numbers from $m=3$ to $m=2$ observed in the laboratory cases (cf. figure \ref{fig:multipanel}) and is marked by the green dash-dotted line. The transition to a smaller mode number appears to suppress the heat transfer efficiency temporarily. A similar behaviour was observed in \citet{horanyi1999turbulent}'s liquid metal Rayleigh-B\'enard convection experiments. The second enhancement in $\widetilde{Nu}$ after the second local minimum happens when the highly-nonlinear flow structures in the bulk fluid start to dominate the convective dynamics. This corresponds to flow behaviors somewhere between $Ra = 1.5 \times 10^6$ (figure \ref{fig:multipanel}d) and $Ra = 4 \times 10^6$ (figure \ref{fig:multipanel}e). The DNS data in figure \ref{fig:modenumber}c) match the first enhancement near $Ra = Ra_{NS}^{\infty}$. Because no mode switch  from $m = 3$ to $m = 2$ was found in the DNS, no kink shows up in the  $Nu$-$Ra$ trend in the DNS data at this point. 