%
% 2_theory
%
Laboratory-scale liquid metal magnetoconvection usually has a negligible induced magnetic field $\boldsymbol{b}$ with respect to the external applied magnetic field $\vec{B}_0$, so that $|\vec{b}|\ll |\vec{B}_0|$. Moreover, any induced field is considered temporally invariant, $\partial_t \boldsymbol{b}\approx 0$. The magnetic Reynolds number, defined as $Rm = UH/\eta$ \citep{julien1996rapidly,glazier1999evidence,davidson_2016,akhmedagaev2020turbulent}, is well below unity, where $U$ and $H$ are the characteristic velocity and length scales, respectively, and $\eta$ is the magnetic diffusivity. This parameter represents the ratio between induction and diffusion of the magnetic field. Thus, the so-called low-$Rm$ quasistatic approximation is valid in most liquid metal experimental and industrial applications \citep{sarris2006limits,knaepen2008magnetohydrodynamic,davidson_2016,cioni2000effect}. In the quasistatic limit, $Rm$ and magnetic Prandtl number $Pm$ formally drop out of the problem, so it is not necessary to solve the magnetic induction equation explicitly, and the system is greatly simplified. In addition, the Oberbeck-Boussinesq approximation is commonly applied in the governing equations for liquid metal MC systems \citep[e.g.][]{cioni2000effect,liu2018wall,vogt2018jump,yan2019heat,xu2022thermoelectric}.

Four nondimensional control parameters govern quasistatic Oberbeck-Boussinesq magnetoconvection \citep{liu2018wall,yan2019heat,xu2022thermoelectric}. The Prandtl number $Pr$ describes the thermo-mechanical properties of the fluid, 
%
\begin{equation}
    Pr =\frac{\nu}{\kappa}, 
\end{equation}
%
where $\nu$ is the kinematic viscosity and $\kappa$ is the thermal diffusivity. In this study, $Pr \approx 0.027$ for liquid gallium. The Rayleigh number $Ra$ characterises the buoyancy forcing with respect to thermo-viscous diffusion and is defined as
%
\begin{equation}
   Ra =  \frac{\alpha g \Delta T H^3}{\kappa \nu}.
\end{equation}
%
Here, $\alpha$ is the thermal expansion coefficient, $g$ is the magnitude of the vertically oriented ($\vec{\hat e_z}$) gravitational acceleration, $\Delta T$ is the bottom-to-top vertical temperature difference across the fluid layer, and the characteristic length scale is the layer height $H$. The Chandrasekhar number $Ch$ denotes the ratio of quasistatic Lorentz forces and viscous forces,
%
\begin{equation}
   Ch = \frac{\sigma B_0^2 H^2}{\rho_0 \nu},
\end{equation}
%
where $\sigma$ is the electric conductivity of the fluid, $B_0$ is the magnitude of the applied vertical magnetic field, and $\rho_0$ is the mean density of the fluid. The Chandrasekhar number is the square of the Hartmann number, $Ch = Ha^2$ \citep[e.g.][]{davidson_2016,moreau1999fundamentals,roberts1967introduction}. Additionally, the cylindrical container has a diameter-to-height aspect ratio
%
\begin{equation}
    \Gamma = \frac{D}{H},
\end{equation}
%
where $D$ is the diameter of the container. Here, $\Gamma$ is fixed to $1.0$ and $2.0$, respectively.

The onset of convection is controlled by the critical Rayleigh denoted as $Ra_{crit}$, which characterises the buoyancy forcing needed for a particular convective mode in the system \citep{plumley2019scaling}. Figure \ref{fig:onset} shows different $Ra_{crit}$ predictions. Linear analysis has shown that the convection driven by buoyancy forces must balance the viscous and Joule dissipation \citep{chandrasekhar1961hydrodynamic}. Thus, in general, the magnetic field inhibits the onset of the convection. \citet{chandrasekhar1961hydrodynamic} derived the onset for the bulk stationary magnetoconvection in an infinite plane layer ($\infty$). With free-slip (FS) boundaries on both ends, the dispersion relation expresses the marginal Rayleigh number $Ra_M$ as,
%
\begin{equation}
   Ra_M =\frac{\pi^2+a^2}{a^2} \left[ \left( \pi^2 + a^2 \right)^2 + \pi^2 Ch\right],
   \label{eq:FS}
\end{equation}
%
where $a$ is the characteristic cell aspect ratio \citep{davidson_2016}, defined as $a \equiv \pi H /L$, where $H$ is the height of the fluid layer, and $2L$ is the horizontal wavelength of the convection flow, assuming the form of two-dimensional rolls with each roll having diameter $L$. By minimizing equation (\ref{eq:FS}), setting $\partial Ra/\partial a = 0$, we obtain the critical Rayleigh number for the bulk stationary magnetoconvection in an infinite layer with free-slip boundaries on both ends, $Ra^{\infty}_{FS}$, and its critical mode number $a_{FS}$. In the limit of $ Ch \rightarrow \infty$, we have $Ra^{\infty}_{FS} \rightarrow \pi^2 Ch$, and $a_{FS} \rightarrow (\pi^4 Ch/2)^{1/6}$ \citep{chandrasekhar1961hydrodynamic,davidson_2016}.

Magnetoconvection with no-slip (NS) rigid horizontal boundaries has a dispersion relation \citep{chandrasekhar1952xlvi}
%
\begin{equation}
   Ra_M =\frac{\left(\pi^{2}+a^{2}\right)\left[\left(\pi^{2}+a^{2}\right)^{2}+\pi^{2} Ch\right]}{a^{2}\left[1-4 \pi^{2} \delta\left(q_{1}^{2}-q_{2}^{2}\right) /\left(\pi^{2}+q_{1}^{2}\right)\left(\pi^{2}+q_{2}^{2}\right)\right]}
   \label{eq:ns}
\end{equation}
%
where 
%
\begin{equation}
    q_1=\frac{1}{2}\left( \sqrt{Ch+4 a^{2}}+\sqrt{Ch}\right), \quad
    q_2=\frac{1}{2}\left( \sqrt{Ch+4 a^{2}}-\sqrt{Ch}\right),
\end{equation}
%
and
\begin{equation}
    \delta = \left(q_1 \tanh (q_1/2) - q_2 \tanh(q_2/2)\right)^{-1}.
\end{equation}
%
By assuming a single structure in the vertical direction and minimizing $Ra$ in (\ref{eq:ns}), we obtain the first approximation of critical $Ra$ of magnetoconvection with two rigid boundaries \cite{chandrasekhar1952xlvi}. Equation (\ref{eq:ns}) also predicts that $Ra^{\infty}_{NS}\rightarrow \pi^2 Ch$ and $a_{NS} \rightarrow (\pi^4 Ch/2)^{1/6}$ with $Ch\rightarrow \infty$. Thus, both critical Rayleigh numbers with free-slip boundaries $Ra^{\infty}_{FS}$ and no-slip boundaries $Ra^{\infty}_{NS}$ asymptote to $\pi^2 Ch$ above $Ch \gtrsim 10^4$, as shown in figure \ref{fig:onset}a). These asymptotic bulk onset predictions agree with previous experimental results \citep{nakagawa1955experiment,cioni2000effect,yan2019heat,zurner2020flow}. 

\citet{busse2008asymptotic} theoretically analysed the side wall modes in MC and derived an asymptotic solution along a straight vertical sidewall in a semi-infinite domain with free-slip top-bottom boundaries. The critical Rayleigh number $Ra_{W}$ for these so-called magnetowall modes is,
%
\begin{equation}
   Ra_{W} = 3\pi^2\sqrt{3\pi/2}\left(1+3Ch^{-1/4}\sqrt{3\pi/2}\right) Ch^{3/4}.
   \label{eq:Busse}
\end{equation}
% 
The asymptotic onset of the wall modes is generally lower than the onset in the bulk fluid at large $Ch$, since $Ch^{3/4}\ll Ch$ as $Ch \rightarrow \infty$. These magnetowall modes are non-drifting and extend into the fluid bulk with a distance that scales as the magnetic boundary layer thickness, which scales with the Shercliff boundary layer thickness $\delta_{Sh} \sim Ch^{-1/4}$ \citep{shercliff1953steady, liu2018wall}. The stationary wall modes of MC differ from those found in rotating convection, where wall modes drift in azimuth \citep{ecke1992hopf, herrmann1993asymptotic, horn2017prograde}.

\citet{houchens2002rayleigh} performed a hybrid linear stability analysis combining the analytical solution for the $\delta_{Ha} \sim Ch^{-1/2}$ Hartmann layers \citep{hartmann1937hg} at top-bottom boundaries and numerical solutions for the rest of the domain in $\Gamma = 1$ and $2$ cylindrical geometries. They also presented a linear asymptotic analysis for large $Ch$. Their asymptotic solutions for critical $Ra$ for $\Gamma = 1$ and $2$ are, respectively,
%
\begin{equation}
   Ra_{cyl,\Gamma=1} = 8.302 Ch^{3/4};\quad Ra_{cyl,\Gamma=2} = 67.748 Ch^{3/4}.
   \label{eq:cyl}
\end{equation}
%
Figure \ref{fig:onset} summarises all the critical Rayleigh predictions mentioned above. \citet{houchens2002rayleigh}'s $Ra_{cyl,\Gamma = 1}$ values (marked by the purple dashed line) are approximately an order of magnitude lower than the rest of the onset predictions that are not aspect-ratio dependent. To test the validity of these predictions, we combine laboratory experiments and direct numerical simulations (DNS) to investigate the five different $Ch$ shown by the vertical dashed lines in figure \ref{fig:onset}b). The values of these five $Ch$ numbers and their corresponding critical $Ra$ are summarised in table \ref{table0}. 
%
\begin{table}[ht]
\begin{ruledtabular}
\begin{tabular}{l|ccccc}
$\bm{Ch}$    & $\bm{Ra_{\mathrm{FS}}^\infty} $  & $\bm{Ra_{\mathrm{NS}}^\infty}$ & $\bm{Ra_{W}} $ & $\bm{Ra_{\mathrm{cyl},\ \Gamma =1}}$ & $\bm{Ra_{\mathrm{cyl},\ \Gamma =2}}$\\[0.5ex]
\hline\\[-1.5ex]
$1\times 10^4$   & $ 1.20\times 10^5$    &$ 1.25\times 10^5$      & $ 1.06\times 10^5$   & $8.30\times 10^3 $ & $6.77\times 10^4$ \\[1.5ex]
$4\times 10^4$      & $ 4.46\times 10^5$   & $ 4.54\times 10^5$      & $2.66\times 10^5 $    & $ 2.35\times 10^4$  & $1.92\times 10^5 $ \\[1.5ex]
$1\times 10^5$      & $ 1.08\times 10^6$   & $ 1.09\times 10^6$   &    $4.94\times 10^5 $   & $ 4.67\times 10^4$ & $	3.81\times 10^5 $  \\[1.5ex]
$3\times 10^5$      & $ 3.15\times 10^6$   & $ 3.17\times 10^6$  &    $	1.05\times 10^6$   & $	1.06\times 10^5 $ & $8.68\times 10^5 $ \\[1.5ex]
$1\times 10^6$    & $1.03\times 10^7 $   & $ 1.03\times 10^7$   &    $2.45\times 10^6 $   & $2.63\times 10^5 $ & $	2.14\times 10^6 $ \\[1.5ex]
\end{tabular}
\end{ruledtabular}
\caption{Values of different predicted critical $Ra$ at $Ch = \{10^4$, $4\times 10^4$, $10^5$, $3\times 10^5$, $10^6\}$, which have been examined experimentally.}
\label{table0}
\end{table}
%
\begin{figure}[t]
  \centering
    \makebox[\textwidth][c] {\includegraphics[width=\textwidth]{figures/Rac_Ch_v3.pdf}}
 	  \caption{a) Different prediction of critical Rayleigh number ($Ra_{crit}$) as a function of the Chandrasekhar number. The black dashed curve shows the $Ra_{cirt}$. For infinite plane MC system with free-slip boundaries at the top and bottom, $Ra_{crit} = Ra_{FS}^{\infty}$ \citep{chandrasekhar1961hydrodynamic}, as shown in the black dashed curve; for infinite plane MC system with no-slip boundaries, $Ra_{crit} = Ra_{NS}^{\infty}$ \citep{chandrasekhar1952xlvi}, as shown in the blue curve. Note that both $Ra_{FS}^{\infty}$ and $Ra_{NS}^{\infty}$ asymptote to $\pi^2Ch$ as $Ch \rightarrow \infty$; the grayish-blue curve shows that $Ra_{W}$ is the asymptotic $Ra_{crit}$ for the wall-mode onset in a half-infinite plane with a vertical boundary \cite{busse2008asymptotic}; the purple curve and the purple dotted curve are \citet{houchens2002rayleigh}'s predicted $Ra_{crit}$, namely $Ra_{cyl,\ \Gamma = 2}$ and $Ra_{cyl,\ \Gamma = 1}$, for MC in cylindrical containers with aspect ratio $1$ and $2$, respectively. b) The zoom-in view of the region circumscribed by the black rectangular box in panel a). The colored vertical dashed lines correspond to five $Ch$ numbers employed in our study. }
   \label{fig:onset}
\end{figure}
%