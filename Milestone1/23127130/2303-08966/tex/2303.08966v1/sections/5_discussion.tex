%
% 5_discussion
%
\subsection{Wall modes stability and the cellular flow regime}

In our laboratory-numerical experiments, the magnetowall modes are stationary and do not drift over dynamically long time scales ($\gg 5 \tau_\kappa$), in contrast to the drifting wall modes in rotating convection systems \citep[e.g.,][]{ecke1992hopf}. This is because the quasi-static Lorentz force ($f_L \propto B_0^2$) does not break the system's azimuthal symmetry, unlike the Coriolis force \citep{ecke1992hopf}). The stationarity of magnetowall modes has been confirmed in both numerical simulations \citep{liu2018wall} and laboratory experiments \citep{zurner2020flow}. Furthermore, \citet{liu2018wall} showed that the magnetowall modes can inject jets into the bulk. This phenomenon was also found in the numerical MC simulations of \citet{akhmedagaev2020turbulent}, where strong, axially invariant wall mode injections were accompanied by a net azimuthal drift of the flow field with random orientations. We believe that the collisional interaction of the jets in a small aspect ratio cylinder ($\Gamma = 1.0$), rather than any innate azimuthal motion of magnetowall modes, is responsible for the drifting motions observed by \citet{akhmedagaev2020turbulent}. 

The fully three-dimensional flow fields from our DNS facilitated the investigation of the bulk flow patterns in this study. Thus, we are also able to compare multiple cases at similar parameters with \citet{zurner2020flow} who inferred the interior structure solely from linewise Ultrasonic Doppler Velocimetry (UDV) and pointwise temperature measurements along the sidewalls and within the top and bottom plate. Our identified flow structures and corresponding flow changes match well with their observations. Specifically, what they denoted as the 'cellular regime' corresponds to our case with extended wall mode noses with interior bulk modes. Our $Ra = 7 \times 10^5$ and $Ra = 1.5 \times 10^6$ cases (figure \ref{fig:multipanel}c, d) resemble the inferred '3-cell' and '4-cell' patterns of figure 3 in \citet{zurner2020flow}. Our $Ra = 1.5\times 10^6,\ m=2$ thermal data in figure \ref{fig:multipanel}d) also agrees with their '2 cell' pattern on the sidewall. Moreover, the transition range from the 'cellular regime' to the non-rotating LSC regime in their experiment occurred approximately at $Ra \gtrsim 4\times 10^6$ for $Ch = 4 \approx 10^4$, which is consistent with our observation of a more chaotic interior and unsteady and irregular wall mode behaviour, as shown in figure \ref{fig:multipanel}e).

\subsection{The $Nu$ vs. $Ra$ MC party}
Figure \ref{fig:NuRaParty} presents a broad compilation of laboratory MC heat transfer measurements in different aspect ratios and geometries, all in the presence of an external vertical magnetic field. 
\citet{cioni2000effect} (open circles) studied liquid mercury in a $\Gamma = 1.0$ cylindrical cell up to $Ch \approx 4\times 10^6$ and $Ra \approx 3\times 10^9$. \citet{aurnou2001experiments} (open triangles pointing right) carried out near onset liquid gallium experiments in a $\Gamma = 8$ rectangular cell. \citet{burr2001rayleigh} (open triangles pointing left) investigated sodium-potassium alloy in a $20:10:1$ rectangular cell. \citet{king2015magnetostrophic} studied liquid gallium MC in a $\Gamma = 1.0$ cylinder on the same device (RoMag) as this study. \citet{zurner2020flow} studied both heat and momentum transfer behaviors of liquid GaInSn in a $\Gamma = 1$ cylinder. The results from our current $\Gamma = \{1.0,\ 2.0\}$ cylindrical liquid gallium experiments and simulations are demarcated by the filled symbols. 
%
\begin{figure}[t!]
   \centering
    \makebox[\textwidth][c] {\includegraphics[width=\columnwidth]{figures/NuRaPartyMCHX_final2.pdf}} 
     	  \caption{Collection of $Nu$-$Ra$ data from this study and previous MC laboratory experiments in liquid metal \cite{chandrasekhar1961hydrodynamic,cioni2000effect,aurnou2001experiments,burr2001rayleigh,king2015magnetostrophic,zurner2020flow}. Color represents $Ch$. The filled symbols mark the data in this study. The five-pointed stars at $Nu = 1$ mark the $Ra_W$ (\ref{eq:Busse}) for all different $Ch>0$. The two asterisk symbols from left to right mark $Ra^{\infty}_{NS}$ for $Ch=2\times 10^6$ and $4\times 10^6$, respectively, and corresponding to the \citet{cioni2000effect}'s two $Ch$ data set. The non-filled color symbols are selected heat transfer data from prior liquid metal laboratory experiments. All data displayed here are included in appendix tables \ref{table1} - \ref{table7}.}
   \label{fig:NuRaParty}
\end{figure}
%

In addition, we have included the Nusselt number data for $Ch=0$, i.e. pure Rayleigh-B\'enard convection (RBC). Best fits to the RBC cases yield
%
\begin{subequations}
\begin{eqnarray}
    Nu_{0} &\approx (0.191 \pm 0.088) Ra^{0.248 \pm 0.025} \quad \text{for}\ \Gamma = 1.0,  \label{eq:nurbca} \\
    Nu_{0} &\approx (0.176 \pm 0.081) Ra^{0.246 \pm 0.028} \quad \text{for}\ \Gamma = 2.0,  \label{eq:nurbcb}
\end{eqnarray}
    \label{eq:nurbc} 
\end{subequations}
%
\hspace{-3pt}which are in good agreement with previous studies on the same device (RoMag) \citep{king2013flow,king2015magnetostrophic,vogt2018jump,aurnou2018rotating}. The differences between these two scaling laws lie within their error bars but may be due to the different tank aspect ratios \citep{king2013turbulent,vogt2018jump,aurnou2018rotating}.  

As discussed in section \ref{sec:4_results}, our data show that the onset of MC in a cylinder occurs via wall modes. The five-point stars at $Nu = 1$ mark the magnetowall mode onset predictions $Ra_W$ \eqref{eq:Busse} for the different $Ch$. Our near-onset data at $Ch = 10^5$ (yellow triangles) is in good agreement with the onset prediction by \citet{busse2008asymptotic} (yellow star). However, \citet{cioni2000effect}'s heat transfer data at $Ch = 2\times 10^6$ and $Ch = 4\times 10^6$ have $Ra_{crit} \approx 3 Ra_W$. The lowest $Nu$ data from \citet{cioni2000effect}'s $Ch = 2\times 10^6$ and $Ch = 4\times 10^6$ align well with the bulk onset prediction, $Ra^{\infty}_{NS}$. \citet{zurner2020flow} have analysed the $Nu$-$Ra$ trends and also found a large deviation between the experimental results of \citet{cioni2000effect} and \citet{king2015magnetostrophic}. This discrepancy is likely due to \citet{cioni2000effect}'s thermometry setup, which used a single thermistor at the center of each top and bottom boundary to measure $\Delta T$. This setup was not designed to characterise wall modes and could only detect the convective heat transfer occurring near the center of the tank.  Thus, top and bottom end wall temperature measurements nearer to the sidewall are required in order to detect the onset of wall modes and to measure their contributions to  the total heat transfer \citep[cf.][]{akhmedagaev2020turbulent, zurner2020flow, grannan2022experimental}.

\subsection{Comparison between magnetoconvection and rotating convection in liquid metal}
The goal of this work is to provide a better understanding of the pathway from convective onset to multimodal turbulence in liquid metal magnetoconvection.  Thus far, we have compared our laboratory-numerical data with the results of other MC studies. Here we expand on this by comparing our MC data against rotating convection data. Although the Lorentz and Coriolis forces both act to constrain the convection in these systems \citep{julien2007reduced}, their data are rarely closely compared since the vast majority of rotating convection (RC) studies are carried out in moderate to high Prandtl fluids (non-metals), whereas MC studies are nearly always made using low $Pr$ liquid metals \citep[cf.][]{Aujogue18}.


\begin{figure}[b!]
   \centering
    \includegraphics[width=\textwidth]{figures/MCvsRC_v9.pdf}
 	  \caption{
 	  a) Normalised Nusselt number ($Nu/Nu_{0}$) versus magnetic Rossby number ($Ro_m$) for MC lab data in $\Gamma = 1.0$ and $2.0$ tanks. Here, $Nu_{0}$ are the best-fit power laws in (\ref{eq:nurbc}) and $Ro_m$ is the inverse of interaction parameter $N$, as defined in (\ref{eq:Rom}). Symbols in thick black outlines represent $\Gamma = 2.0$ data, and those in thin grey outlines are $\Gamma=1.0$ data. The color of the symbols indicates $\mathrm{log_{10}}(Ch)$. The white symbols are subcritical cases according to wall mode onset ($Ra<Ra_{W}$). The symbol shapes do not contain information but help differentiate different $Ch$. b) Rotating convective heat transfer data adapted from the $\Gamma = 1.0$ liquid gallium experiments of \citet{king2013turbulent}. The color indicates $\mathrm{log_{10}}(Ek^{-1})$. The vertical axis shows a reduced Nusselt number $Nu/Nu_{0}= Nu/(0.185 Ra^{0.25})$, following \citep{king2013turbulent}. The horizontal axis is convective Rossby number $Ro_c$, as defined in (\ref{eq:Rom}).} 
   \label{fig:MCvsRC}
\end{figure}
%

Figure \ref{fig:MCvsRC} shows a side-by-side comparison of the convective heat transfer efficiency $Nu/Nu_0$ as a function of the normalised buoyancy forcing in a) our present liquid gallium $\Gamma = 1.0$ and $2.0$ MC experiments and b) the liquid gallium $\Gamma = 1.0$ rotating convection data from King \& Aurnou (2013) \cite{king2013turbulent}. The thicker outlined symbols in panel a) demarcate the $\Gamma = 2.0$ MC cases. The liquid gallium convection data in Figure \ref{fig:MCvsRC} was all obtained using the same experimental apparatus and setup. The fill color in a) denotes $\log_{10}(Ch)$, whereas it denotes $\log_{10}(Ek^{-1})$ in panel b). The Ekman number, $Ek = \nu / 2 \Omega H^2$, is the ratio of viscous and Coriolis forces in rotating systems and $\Omega$ is the RC system's angular rotation rate.  

The best co-collapse of the $Nu/Nu_0$ data sets was found when the buoyancy force was normalised by the appropriate constraining force, that being Lorentz in MC and Coriolis forces in RC.  In MC, this non-dimensional ratio is called the magnetic Rossby number, $Ro_m$, which formally describes the ratio of convective inertia and the Lorentz force: 
%
\begin{equation}
    Ro_m= \frac{\text{Inertia}}{\text{Lorentz}} = Re_{f\!f}Ch^{-1}= \sqrt{\frac{RaCh^{-2}}{Pr}},
    \label{eq:Rom}
\end{equation}
%
where $Re_{f\!f} = U_{f\!f}H/\nu$. In the MHD literature, the reciprocal of this ratio, which is called the interaction parameter $N = Ro_m^{-1}$ is often employed \citep[e.g.,][]{xu2022thermoelectric}.  In RC, this non-dimensional ratio is called the convective Rossby number, $Ro_c$, 
%
\begin{equation}
    Ro_c= \frac{\text{Inertia}}{\text{Coriolis}} = Re_{f\!f}Ek = \sqrt{\frac{RaEk^2}{Pr}},
    \label{eq:Roc}
\end{equation}
%
which shows up as a collapse parameter in a broad array of rotating convection problems \citep[e.g.][]{gastine2014zonal,aurnou2020connections, Landin23}. Lorentz forces dominate in MC when $Ro_m = 1/N$ is small; Coriolis forces dominate in RC when $Ro_c$ is small. When these Rossby numbers exceed unity, buoyancy-driven inertial forces should be dominant and the convection is expected to be effectively unconstrained on all available length scales in the system. 

Comparing Figures \ref{fig:MCvsRC}a) and \ref{fig:MCvsRC}b), it is clear that the liquid metal MC and RC data have similar gross morphologies.  The $Nu/Nu_0$ is near unity and effectively flat for both $Ro_m \gtrsim 1$ and $Ro_c \gtrsim 1$. Thus, when the constraining Lorentz or Coriolis forces become subdominant to inertia in either system, the heat transfer is similar to that found in unconstrained RBC experiments. 

The basic structures of MC and RC data are also similar at $Ro_m \lesssim 1$ and $Ro_c \lesssim 1$: the normalised heat transfer trends relatively sharply downwards with decreasing Rossby number.  However, the detailed structures of the low $Ro_m$ and low $Ro_c$ data differ substantively. The data fall off more steeply with Rossby in the rotating case, then it greatly flattens out in the lowest $Nu/Nu_0$ RC cases. The differences in slope may be due to the difference in Ekman pumping (EP) effects in both systems \citep{julien2016nonlinear}, although heat transfer enhancement by EP is typically weak in metals since it is hard to modify the thermal boundary layers in low $Pr$ flows. 

Alternatively, these differences may be caused by the differences in critical $Ra$ values and their scalings. For instance, in the parameter ranges explored in Figure \ref{fig:MCvsRC}, oscillatory bulk convection first onsets in RC \citep{aurnou2018rotating}, whereas it is the wall modes that develop first in MC.  Further, the bulk magnetoconvective onset scales asymptotically as $Ra_{crit} \sim Ch^1$ whereas bulk oscillatory convective onset asymptotically scales as $Ra_{crit} \sim Ek^{-4/3}$.  This $1/3$ difference in the scaling exponents may imply that the available range of $Nu/Nu_0$ will be larger in the RC cases. Further, the flat tail in the lowest $Nu/Nu_0$ RC data is likely due to the low convective heat transfer efficiency of oscillatory rotating convection.

Thus, the gross structures of the two data compilations are similar in Figure \ref{fig:MCvsRC}. We hypothesise that their differences in our current data are likely due to the various modal onset phenomena, as are clearly present in Figure \ref{fig:modenumber}, that alter the low Rossby branches of each figure panel. However, it may be that differences in MC and RC supercritical dynamics better explain these data \citep[cf.][]{yan2019heat, Oliver23}. Regardless of the root cause, these low Rossby differences have thus far thwarted our attempts to create a unified plot in which all the $Nu/Nu_0$ data are simultaneously collapsed \citep[cf.][]{Chong17}. 

\subsection{Summary}

We have conducted a suite of laboratory thermal measurements of liquid gallium magnetoconvection in cylindrical containers of aspect ratios $\Gamma = 1.0$ and $2.0$. Our data allow us to characterise liquid metal MC from wall mode onset to multimodality. We performed a fixed $Ch = 4\times 10^4$ survey of direct numerical simulation for the same system in a $\Gamma = 2.0$ cylindrical geometry. Both laboratory and numerical methods obtained similar heat transfer behaviors, with possible subtle differences in flow morphology. Together with previous studies, our liquid metal heat transfer data comprise a convective heat transfer survey over six orders of magnitude in both $Ra$ and in $Ch$ (Figure \ref{fig:NuRaParty}).

\citet{busse2008asymptotic}'s asymptotic solutions for magnetowall modes best collapse all our MC heat transfer data, whereas the hybrid theoretical-numerical solutions by \citet{houchens2002rayleigh} captures the exact onset for $\Gamma = 1.0$, but the onset for $\Gamma = 2.0$ remains unverified. Better theoretical onset predictions are needed for liquid metal MC in a cylindrical cell as a function of $\Gamma$. This differs from liquid metal rotating convection where accurate theoretical predictions currently exist for low-$Pr$ fluids in cylindrical geometries \citep{zhang2009onset, zhang_liao_2017}. 

The MC flow morphology was characterised experimentally using a sidewall thermistor array as well as the DNS temperature and velocity fields. The onset of convection was verified to occur in the form of stationary (non-drifting) magnetowall modes. These magnetowall modes develop nose-like protuberances that extend into the fluid bulk with increasing supercriticality. At Rayleigh numbers beyond the critical value for steady bulk convection, the noses interact with the interior bulk modes, likely resulting in the apparent cell-like flow patterns observed by \citet{zurner2020flow}. Our data show that MC convective heat transport is sensitive to the flow morphology, with the Nusselt number $Nu$-$Ra$ data containing distinct kinks at these points where the dominant convection mode appears to change. 

Lastly, liquid metal heat transfer trends in magnetoconvection were compared with rotating convection. The gross behavior of the heat transfer is controlled by the magnetic and convective Rossby numbers, $Ro_m$ and $Ro_c$, in the respective systems, with the normalised heat transport $Nu/Nu_0$ approaching the RBC scaling as $Ro_m$ and $Ro_c$ approach unity from below. The detailed trends at Rossby values less than unity show clear differences between MC and RC.  We have not yet deduced a scheme by which it is possible to collapse all the liquid metal MC and RC data in a unified way.  