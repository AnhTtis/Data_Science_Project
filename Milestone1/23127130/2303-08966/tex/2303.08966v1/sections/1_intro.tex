%
% 1_intro
%
Convection influenced by ambient magnetic fields is called magnetoconvection (MC), which arises in many areas of fluid dynamics. In geophysics and planetary physics, motions of turbulent convective flows in planetary liquid metal outer cores generate planetary-scale magnetic fields via dynamo processes. Studying the effects of magnetic fields in MC is essential to understand these processes \citep[e.g.,][]{jones2011planetary,roberts2013genesis, aurnou2017cross,moffatt2019self}. In astrophysics, MC is associated with the sunspot umbra on the outer layer of the Sun and other stars \citep[e.g.,][]{proctor1982magnetoconvection,schussler2006magnetoconvection,rempel2009radiative}. Furthermore, MC has an essential role in numerous industrial and engineering applications including but not limited to liquid metal batteries \citep{kelley2018fluid, cheng2022laboratory}, crystal growth \citep{moreau1999fundamentals, rudolph2008travelling}, nuclear fusion liquid-metal cooling blanket designs \citep{barleon1991mhd,abdou2001exploration,salavy2007overview}, and induction heating and casting \citep{taberlet1985turbulent,davidson1999magnetohydrodynamics}. These systems have drastically different ratios between electromagnetic and inertial forces. Therefore, it is crucial to investigate the effects of a wide range of magnetic forces in MC systems. 

The canonical model of MC is a convection system with an electrically-conducting fluid layer heated from below, cooled from above, and in the presence of an external vertical magnetic field \cite{chandrasekhar1961hydrodynamic,yan2019heat}. It is most fundamentally understood in an extended plane layer geometry \citep[e.g.,][]{chandrasekhar1961hydrodynamic,nakagawa1955experiment,yan2019heat}. But there is an increasing interest in MC systems with defined sidewall boundaries because of their many experimental and industrial applications. Various numerical and laboratory studies have been carried out in rectangular \citep{schussler2006magnetoconvection, liu2018wall} and cylindrical geometries \citep{cioni2000effect,akhmedagaev2020turbulent,zurner2019combined,zurner2020flow}. 

In weakly supercritical, near-onset regimes, MC systems tend to develop steady wall modes in the sidewall Shercliff boundary layer \citep{houchens2002rayleigh,busse2008asymptotic,liu2018wall,zurner2020flow,akhmedagaev2020turbulent} while the bulk remains quiescent. As the magnetoconvective supercriticality increases, convective flows self-organise into multi-cellular bulk flow structures \citep{yan2019heat, zurner2020flow}. Eventually, at very large supercriticalities, the buoyancy forces dominate and magnetic field effects become subdominant. Large-scale circulations (LSCs) then form and the heat and momentum transfer asymptote to that of turbulent RBC \citep[e.g.][]{lim2019quasistatic,zurner2020flow,xu2022thermoelectric,grannan2022experimental}. 

The pathway from the onset of convection to fully developed turbulence in liquid metal MC is not well characterised. To address this deficit, we present a suite of laboratory-numerical coupled MC experiments in liquid gallium to investigate how MC transitions from near-onset wall modes to turbulent multimodality in cylindrical cells. The paper is organised as follows. Section \ref{sec:2_theory} introduces control parameters, and reviews established onset predictions for the magnetoconvection system. Section \ref{sec:3_method} presents our experimental setup, numerical schemes, diagnostics, and the physical properties of liquid gallium. Section \ref{sec:4_results} compares different theoretical onsets and observations of the transition to multimodality in a survey with fixed magnetic field strength and varying convective vigor. Section \ref{sec:5_discussion} shows heat transfer results combining  previous studies of MC and compares the gross heat transfer behaviors between liquid metal magnetoconvection and liquid metal rotating convection systems.