\documentclass[
amsmath,
amssymb,
groupedaddress,
notitlepage,
floatfix,
aps,
prf,
dvipsnames,
%preprint
%reprint
%superscriptaddress,
%unsortedaddress,
%runinaddress,
%frontmatterverbose, 
%showpacs,
%preprintnumbers
%nofootinbib,
%nobibnotes,
%bibnotes,
%aps,
%pra,
%prb,
%rmp,
%prstab,
%prstper,
%floatfix,
]{revtex4-2}
\usepackage{CJK}
\usepackage[letterpaper,margin=1in]{geometry}
\usepackage{graphicx}% Include figure files
\usepackage{dcolumn}% Align table columns on a decimal point
\usepackage{bm}% bold math
\usepackage[FIGTOPCAP]{subfigure}
\usepackage{CJK}
\usepackage{xcolor}
\usepackage{rotating}
\usepackage{braket}
\usepackage{float}
\usepackage{hyperref}
\usepackage{protosem}

%% Commenting
\newcommand{\red}{\textcolor{red}}

% goldfish
\definecolor{rbctop}{RGB}{0,103,154}
\definecolor{rbcbottom}{RGB}{152,0,76}
\newcommand{\goldfish}{%
\begin{minipage}[c][1em]{3.ex}{\vspace*{0.1em}
\begin{tabular}{@{}c@{}} {\color{rbctop} \textproto{\Amem}}\\[-1.6ex] \textproto{\Adaleth}\\[-1.6ex] {\color{rbcbottom} \textproto{\Amem}}\end{tabular}
}\end{minipage}}
\newcommand{\vect}{\boldsymbol}
\renewcommand{\vec}{\boldsymbol}
%%%%%%%%%%%%%%%%%%%%%%%%%%%%%%%%%%%%%%%%%%%%%%%%%%%%%%%%%%%%%%%%%%%%%%%%%%%%%%%%%
\begin{document}   
%\preprint{APS/123-QED}

\bibliographystyle{apsrev4-1}

\begin{CJK*}{UTF8}{gbsn}
\title{The Transition from Wall Modes to Multimodality in Liquid Gallium Magnetoconvection}
\author{Yufan Xu (徐宇凡)} %(徐宇凡)
\email{yufanxu@g.ucla.edu}
\affiliation{Department of Earth, Planetary, and Space Science, University of California, Los Angeles, California 90095, USA}
\author{Susanne Horn (苏珊娜·霍恩)}%(苏珊娜·霍恩)
% \email{Second.Author@institution.edu}
\affiliation{Centre for Fluid and Complex Systems, Coventry University, CV1 2NL Coventry, UK}
\author{Jonathan M. Aurnou(乔纳森·奥诺)}%(乔纳森·奥诺)
%\collaboration{MUSO Collaboration}%\noaffiliation
\affiliation{Department of Earth, Planetary, and Space Science, University of California, Los Angeles, California 90095, USA } %Lines break automatically or can be forced with \\
\date{\today}% It is always \today, today,
             %  but any date may be explicitly specified
\maketitle
\end{CJK*}
%%%%%%%%%%%%%%%%%%%%%%%%%%%%%%%%%%%%%%%%%%%%%%%%%%%%%%%%%%%%%%%%%%%%%%%%%%%%%%%%%
\begin{abstract}
Coupled laboratory-numerical experiments of Rayleigh-B\'enard convection (RBC) in liquid gallium subject to a vertical magnetic field are presented. The experiments are carried out in two cylindrical containers with diameter-to-height aspect ratio $\Gamma = 1.0$ and $2.0$ at varying thermal forcing (Rayleigh numbers $10^5 \lesssim Ra \lesssim 10^8$) and magnetic field strength (Chandrasekhar numbers $0\lesssim Ch \lesssim 3\times 10^5$). Laboratory measurements and numerical simulations confirm that magnetoconvection in our finite cylindrical tanks onsets via non-drifting wall-attached modes, in good agreement with asymptotic predictions for a semi-infinite domain. With increasing supercriticality, the experimental and numerical thermal measurements and the numerical velocity data reveal transitions between wall mode states with different azimuthal mode numbers and between wall-dominated convection to wall and interior multimodality. These transitions are also reflected in the heat transfer data, which combined with previous studies, connect onset to supercritical turbulent behaviors in liquid metal magnetoconvection over a large parameter space. The gross heat transfer behaviors between magnetoconvection and rotating convection in liquid metals are compared and discussed.
\end{abstract}
\maketitle
%\tableofcontents

%%%%%%%%%%%%%%%%%%%%%%%%%%%%%%%%%%%%%%%%%%%%%%%%%%%%%%%%%%%%%%%%%%%%%%%%%%%%%%%%%
\section{Introduction} \label{sec:1_intro}
\section{Introduction}
\label{sec:intro}
\begin{figure}[t]
\begin{center}
    \includegraphics[width=1\linewidth]{figures/teaser.pdf}
\end{center}
\vspace{-0.1in}
\caption{\textbf{{\em Foggy} vs {\em Clear} NeRF.} Our \ournerf gets rid of reconstruction errors manifested as foggy ``floaters" in the density volume without additional input or significant computational overhead. 
%
Below are density profiles along a given ray before and after our geometry correction procedure, where we discard density peaks corresponding to floaters.
}
\label{fig:teaser}
\vspace{-0.2in}
\end{figure}



%The emergence of 
Neural Radiance Fields (NeRFs)~\cite{mildenhall2020nerf}  %and its variants 
have made revolutionary contributions in %photo-realistic 
novel view synthesis~\cite{barron2021mip,barron2022mip}, 
autonomous driving~\cite{rematas2022urban,tancik2022block}, digital human~\cite{hong2022headnerf,zhao2022humannerf}, and 3D content generation~\cite{eg3d,poole2022dreamfusion,lin2022magic3d}.
%by leveraging a multi-layer perceptron (MLP) to implicitly model the mapping from input 5D coordinates (i.e., 3D coordinates $\mathbf{x} = (x,y,z)$ and 2D viewing directions $\mathbf{d}=(\theta,\phi)$) to volume density $\sigma$ and view-dependent emitted radiance color $\mathbf{c} = (r,g,b)$. 
%
%They then use traditional volume rendering mechanisms on the obtained continuous 5D function (i.e., MLP) to generate novel views. 
To date, unfortunately, most NeRF-based methods encounter challenges when tackling large-scale cluttered scenes (e.g., Fig.~\ref{fig:teaser}):
\begin{enumerate}[leftmargin=0.16in, topsep=2pt,itemsep=-1ex,partopsep=1ex,parsep=1ex]
\item Input observations used for NeRF are often too sparse  compared to forward-facing or synthetic looking-inward scenes;
%\item Recovering fine-grained objects within a large volume is challenging for NeRF; %in capturing details accurately.
\item View-dependent visual effects give rise to ambiguity, resulting in a ``foggy" density field as shown in Fig.~\ref{fig:teaser}. 
%
Such artifacts are particularly pronounced in indoor scenes strewn with view-dependent appearances, such as specular highlights, glossy surface reflections from man-made objects. 
\end{enumerate}

Despite attempts to enhance NeRF's rendering quality given suboptimal input, such as using 3D conical frustums~\cite{barron2021mip,barron2022mip}, physically-grounded augmentations~\cite{chen2022aug}, and misalignment correction~\cite{jiang2022alignerf},  these challenges have yet to be fully resolved.
%
Depth supervision~\cite{deng2022depth, wei2021nerfingmvs} or proxy geometry~\cite{xu2021scalable,wu2022scalable} images can help alleviate the challenges in handling large-scale with sparse input, at the expense of %but they come at the cost of requiring 
expensive pre-processing or additional input.
%
Another line of work~\cite{wang2021neus, oechsle2021unisurf, wang2022neuris} achieves better reconstruction of surface geometry by using signed distances instead of volume density as scene representation. However, they sacrifice the ability to synthesize photo-realistic novel views.

%We observe that NeRF has been suffering from foggy ``floater" artifacts in large-scale cluttered scenes.
%
%Such artifacts are particularly pronounced in indoor scenes strewn with view-dependent appearances from man-made objects. 
%
To address the above issues, we propose an extension to NeRF, dubbed as {\bf \ournerf}, which enforces effective {\em appearance} and {\em geometry} constraints conducive to accurate colors and 3D densities estimation. We believe \ournerf can contribute beyond novel view synthesis, such as NeRF object detection~\cite{hu2022nerf}, NeRF object segmentation~\cite{zhi2021place, liu2022unsupervised, fan2022nerf,ren2022neural}, and NeRF registration~\cite{goli2022nerf2nerf}, where the rooms for improvement are substantial if more accurate color and density estimation are available.

Correspondingly, there are two steps in \ournerf. First, for appearance correction, the view-independent and view-dependent color components are predicted from the underlying 3D scene, which is combined to produce the final color estimation (Fig.~\ref{fig:toaster}).
%
The view-independent component (diffuse color and shading) captures the overall scene color, while the view-dependent component (highlights or reflections) captures color variations due to changes in viewing angle.
%
\ournerf then discards these view-dependent appearances in the training views to prevent them from interfering with the density estimation.
%
Second, a simple and effective geometry correction procedure will be performed to further eliminate the foggy ``floaters" or density errors. This geometry correction procedure is based on an assumption in line with traditional ray tracing in computer graphics.
\begin{comment}
% xh: basically copying method
On the other hand, ClearNeRF performs a geometric correction procedure performed on each traced ray during inference to refine the density estimation and better tackle the floater artifacts. 
%
The geometry correction procedure assumes that there should only be one salient peak along each traced ray during NeRF inference. 
Only the salient peak closest to the ray origin (the camera center) corresponds to  true geometry while the others will be manifested as foggy floaters hovering in the density volume. 
%
This assumption is in line with traditional ray tracing in computer graphics where in the absence of noise, only one intersection per ray should be returned to indicate the closest ray-object intersection.
%
\end{comment}
%%%%%%%%%%%
%As shown in Fig.~\ref{fig:teaser}, when reconstructing an indoor scene with sparse input and highly view-dependent objects, NeRF produces severe floating artifacts due to its attempt to explain view-dependent appearances.
%
Experiments verify that our proposed \ournerf can effectively get rid of floater artifacts without additional input.% or significant computational overhead. 


In summary, our contributions include the following:
\begin{itemize}[leftmargin=0.16in, topsep=2pt,itemsep=-1ex,partopsep=1ex,parsep=1ex]
    \item We propose a concise method for decomposing view-independent and view-dependent appearance during NeRF training and eliminate the interference of view-dependent appearance.
    \item We propose a geometric correction procedure performed on each traced ray during inference to refine the density estimation and better tackle the floater artifacts.
    \item Extensive experiments and ablations verify the effectiveness of our core designs and results in improvements over the vanilla NeRF and other state-of-the-art alternatives.
    %without additional computational resources or other inputs.
\end{itemize}





%%%%%%%%%%%%%%%%%%%%%%%%%%%%%%%%%%%%%%%%%%%%%%%%%%%%%%%%%%%%%%%%%%%%%%%%%%%%%%%%%
\section{Control parameters and linear prediction} \label{sec:2_theory}
%
% 2_theory
%
Laboratory-scale liquid metal magnetoconvection usually has a negligible induced magnetic field $\boldsymbol{b}$ with respect to the external applied magnetic field $\vec{B}_0$, so that $|\vec{b}|\ll |\vec{B}_0|$. Moreover, any induced field is considered temporally invariant, $\partial_t \boldsymbol{b}\approx 0$. The magnetic Reynolds number, defined as $Rm = UH/\eta$ \citep{julien1996rapidly,glazier1999evidence,davidson_2016,akhmedagaev2020turbulent}, is well below unity, where $U$ and $H$ are the characteristic velocity and length scales, respectively, and $\eta$ is the magnetic diffusivity. This parameter represents the ratio between induction and diffusion of the magnetic field. Thus, the so-called low-$Rm$ quasistatic approximation is valid in most liquid metal experimental and industrial applications \citep{sarris2006limits,knaepen2008magnetohydrodynamic,davidson_2016,cioni2000effect}. In the quasistatic limit, $Rm$ and magnetic Prandtl number $Pm$ formally drop out of the problem, so it is not necessary to solve the magnetic induction equation explicitly, and the system is greatly simplified. In addition, the Oberbeck-Boussinesq approximation is commonly applied in the governing equations for liquid metal MC systems \citep[e.g.][]{cioni2000effect,liu2018wall,vogt2018jump,yan2019heat,xu2022thermoelectric}.

Four nondimensional control parameters govern quasistatic Oberbeck-Boussinesq magnetoconvection \citep{liu2018wall,yan2019heat,xu2022thermoelectric}. The Prandtl number $Pr$ describes the thermo-mechanical properties of the fluid, 
%
\begin{equation}
    Pr =\frac{\nu}{\kappa}, 
\end{equation}
%
where $\nu$ is the kinematic viscosity and $\kappa$ is the thermal diffusivity. In this study, $Pr \approx 0.027$ for liquid gallium. The Rayleigh number $Ra$ characterises the buoyancy forcing with respect to thermo-viscous diffusion and is defined as
%
\begin{equation}
   Ra =  \frac{\alpha g \Delta T H^3}{\kappa \nu}.
\end{equation}
%
Here, $\alpha$ is the thermal expansion coefficient, $g$ is the magnitude of the vertically oriented ($\vec{\hat e_z}$) gravitational acceleration, $\Delta T$ is the bottom-to-top vertical temperature difference across the fluid layer, and the characteristic length scale is the layer height $H$. The Chandrasekhar number $Ch$ denotes the ratio of quasistatic Lorentz forces and viscous forces,
%
\begin{equation}
   Ch = \frac{\sigma B_0^2 H^2}{\rho_0 \nu},
\end{equation}
%
where $\sigma$ is the electric conductivity of the fluid, $B_0$ is the magnitude of the applied vertical magnetic field, and $\rho_0$ is the mean density of the fluid. The Chandrasekhar number is the square of the Hartmann number, $Ch = Ha^2$ \citep[e.g.][]{davidson_2016,moreau1999fundamentals,roberts1967introduction}. Additionally, the cylindrical container has a diameter-to-height aspect ratio
%
\begin{equation}
    \Gamma = \frac{D}{H},
\end{equation}
%
where $D$ is the diameter of the container. Here, $\Gamma$ is fixed to $1.0$ and $2.0$, respectively.

The onset of convection is controlled by the critical Rayleigh denoted as $Ra_{crit}$, which characterises the buoyancy forcing needed for a particular convective mode in the system \citep{plumley2019scaling}. Figure \ref{fig:onset} shows different $Ra_{crit}$ predictions. Linear analysis has shown that the convection driven by buoyancy forces must balance the viscous and Joule dissipation \citep{chandrasekhar1961hydrodynamic}. Thus, in general, the magnetic field inhibits the onset of the convection. \citet{chandrasekhar1961hydrodynamic} derived the onset for the bulk stationary magnetoconvection in an infinite plane layer ($\infty$). With free-slip (FS) boundaries on both ends, the dispersion relation expresses the marginal Rayleigh number $Ra_M$ as,
%
\begin{equation}
   Ra_M =\frac{\pi^2+a^2}{a^2} \left[ \left( \pi^2 + a^2 \right)^2 + \pi^2 Ch\right],
   \label{eq:FS}
\end{equation}
%
where $a$ is the characteristic cell aspect ratio \citep{davidson_2016}, defined as $a \equiv \pi H /L$, where $H$ is the height of the fluid layer, and $2L$ is the horizontal wavelength of the convection flow, assuming the form of two-dimensional rolls with each roll having diameter $L$. By minimizing equation (\ref{eq:FS}), setting $\partial Ra/\partial a = 0$, we obtain the critical Rayleigh number for the bulk stationary magnetoconvection in an infinite layer with free-slip boundaries on both ends, $Ra^{\infty}_{FS}$, and its critical mode number $a_{FS}$. In the limit of $ Ch \rightarrow \infty$, we have $Ra^{\infty}_{FS} \rightarrow \pi^2 Ch$, and $a_{FS} \rightarrow (\pi^4 Ch/2)^{1/6}$ \citep{chandrasekhar1961hydrodynamic,davidson_2016}.

Magnetoconvection with no-slip (NS) rigid horizontal boundaries has a dispersion relation \citep{chandrasekhar1952xlvi}
%
\begin{equation}
   Ra_M =\frac{\left(\pi^{2}+a^{2}\right)\left[\left(\pi^{2}+a^{2}\right)^{2}+\pi^{2} Ch\right]}{a^{2}\left[1-4 \pi^{2} \delta\left(q_{1}^{2}-q_{2}^{2}\right) /\left(\pi^{2}+q_{1}^{2}\right)\left(\pi^{2}+q_{2}^{2}\right)\right]}
   \label{eq:ns}
\end{equation}
%
where 
%
\begin{equation}
    q_1=\frac{1}{2}\left( \sqrt{Ch+4 a^{2}}+\sqrt{Ch}\right), \quad
    q_2=\frac{1}{2}\left( \sqrt{Ch+4 a^{2}}-\sqrt{Ch}\right),
\end{equation}
%
and
\begin{equation}
    \delta = \left(q_1 \tanh (q_1/2) - q_2 \tanh(q_2/2)\right)^{-1}.
\end{equation}
%
By assuming a single structure in the vertical direction and minimizing $Ra$ in (\ref{eq:ns}), we obtain the first approximation of critical $Ra$ of magnetoconvection with two rigid boundaries \cite{chandrasekhar1952xlvi}. Equation (\ref{eq:ns}) also predicts that $Ra^{\infty}_{NS}\rightarrow \pi^2 Ch$ and $a_{NS} \rightarrow (\pi^4 Ch/2)^{1/6}$ with $Ch\rightarrow \infty$. Thus, both critical Rayleigh numbers with free-slip boundaries $Ra^{\infty}_{FS}$ and no-slip boundaries $Ra^{\infty}_{NS}$ asymptote to $\pi^2 Ch$ above $Ch \gtrsim 10^4$, as shown in figure \ref{fig:onset}a). These asymptotic bulk onset predictions agree with previous experimental results \citep{nakagawa1955experiment,cioni2000effect,yan2019heat,zurner2020flow}. 

\citet{busse2008asymptotic} theoretically analysed the side wall modes in MC and derived an asymptotic solution along a straight vertical sidewall in a semi-infinite domain with free-slip top-bottom boundaries. The critical Rayleigh number $Ra_{W}$ for these so-called magnetowall modes is,
%
\begin{equation}
   Ra_{W} = 3\pi^2\sqrt{3\pi/2}\left(1+3Ch^{-1/4}\sqrt{3\pi/2}\right) Ch^{3/4}.
   \label{eq:Busse}
\end{equation}
% 
The asymptotic onset of the wall modes is generally lower than the onset in the bulk fluid at large $Ch$, since $Ch^{3/4}\ll Ch$ as $Ch \rightarrow \infty$. These magnetowall modes are non-drifting and extend into the fluid bulk with a distance that scales as the magnetic boundary layer thickness, which scales with the Shercliff boundary layer thickness $\delta_{Sh} \sim Ch^{-1/4}$ \citep{shercliff1953steady, liu2018wall}. The stationary wall modes of MC differ from those found in rotating convection, where wall modes drift in azimuth \citep{ecke1992hopf, herrmann1993asymptotic, horn2017prograde}.

\citet{houchens2002rayleigh} performed a hybrid linear stability analysis combining the analytical solution for the $\delta_{Ha} \sim Ch^{-1/2}$ Hartmann layers \citep{hartmann1937hg} at top-bottom boundaries and numerical solutions for the rest of the domain in $\Gamma = 1$ and $2$ cylindrical geometries. They also presented a linear asymptotic analysis for large $Ch$. Their asymptotic solutions for critical $Ra$ for $\Gamma = 1$ and $2$ are, respectively,
%
\begin{equation}
   Ra_{cyl,\Gamma=1} = 8.302 Ch^{3/4};\quad Ra_{cyl,\Gamma=2} = 67.748 Ch^{3/4}.
   \label{eq:cyl}
\end{equation}
%
Figure \ref{fig:onset} summarises all the critical Rayleigh predictions mentioned above. \citet{houchens2002rayleigh}'s $Ra_{cyl,\Gamma = 1}$ values (marked by the purple dashed line) are approximately an order of magnitude lower than the rest of the onset predictions that are not aspect-ratio dependent. To test the validity of these predictions, we combine laboratory experiments and direct numerical simulations (DNS) to investigate the five different $Ch$ shown by the vertical dashed lines in figure \ref{fig:onset}b). The values of these five $Ch$ numbers and their corresponding critical $Ra$ are summarised in table \ref{table0}. 
%
\begin{table}[ht]
\begin{ruledtabular}
\begin{tabular}{l|ccccc}
$\bm{Ch}$    & $\bm{Ra_{\mathrm{FS}}^\infty} $  & $\bm{Ra_{\mathrm{NS}}^\infty}$ & $\bm{Ra_{W}} $ & $\bm{Ra_{\mathrm{cyl},\ \Gamma =1}}$ & $\bm{Ra_{\mathrm{cyl},\ \Gamma =2}}$\\[0.5ex]
\hline\\[-1.5ex]
$1\times 10^4$   & $ 1.20\times 10^5$    &$ 1.25\times 10^5$      & $ 1.06\times 10^5$   & $8.30\times 10^3 $ & $6.77\times 10^4$ \\[1.5ex]
$4\times 10^4$      & $ 4.46\times 10^5$   & $ 4.54\times 10^5$      & $2.66\times 10^5 $    & $ 2.35\times 10^4$  & $1.92\times 10^5 $ \\[1.5ex]
$1\times 10^5$      & $ 1.08\times 10^6$   & $ 1.09\times 10^6$   &    $4.94\times 10^5 $   & $ 4.67\times 10^4$ & $	3.81\times 10^5 $  \\[1.5ex]
$3\times 10^5$      & $ 3.15\times 10^6$   & $ 3.17\times 10^6$  &    $	1.05\times 10^6$   & $	1.06\times 10^5 $ & $8.68\times 10^5 $ \\[1.5ex]
$1\times 10^6$    & $1.03\times 10^7 $   & $ 1.03\times 10^7$   &    $2.45\times 10^6 $   & $2.63\times 10^5 $ & $	2.14\times 10^6 $ \\[1.5ex]
\end{tabular}
\end{ruledtabular}
\caption{Values of different predicted critical $Ra$ at $Ch = \{10^4$, $4\times 10^4$, $10^5$, $3\times 10^5$, $10^6\}$, which have been examined experimentally.}
\label{table0}
\end{table}
%
\begin{figure}[t]
  \centering
    \makebox[\textwidth][c] {\includegraphics[width=\textwidth]{figures/Rac_Ch_v3.pdf}}
 	  \caption{a) Different prediction of critical Rayleigh number ($Ra_{crit}$) as a function of the Chandrasekhar number. The black dashed curve shows the $Ra_{cirt}$. For infinite plane MC system with free-slip boundaries at the top and bottom, $Ra_{crit} = Ra_{FS}^{\infty}$ \citep{chandrasekhar1961hydrodynamic}, as shown in the black dashed curve; for infinite plane MC system with no-slip boundaries, $Ra_{crit} = Ra_{NS}^{\infty}$ \citep{chandrasekhar1952xlvi}, as shown in the blue curve. Note that both $Ra_{FS}^{\infty}$ and $Ra_{NS}^{\infty}$ asymptote to $\pi^2Ch$ as $Ch \rightarrow \infty$; the grayish-blue curve shows that $Ra_{W}$ is the asymptotic $Ra_{crit}$ for the wall-mode onset in a half-infinite plane with a vertical boundary \cite{busse2008asymptotic}; the purple curve and the purple dotted curve are \citet{houchens2002rayleigh}'s predicted $Ra_{crit}$, namely $Ra_{cyl,\ \Gamma = 2}$ and $Ra_{cyl,\ \Gamma = 1}$, for MC in cylindrical containers with aspect ratio $1$ and $2$, respectively. b) The zoom-in view of the region circumscribed by the black rectangular box in panel a). The colored vertical dashed lines correspond to five $Ch$ numbers employed in our study. }
   \label{fig:onset}
\end{figure}
%

%%%%%%%%%%%%%%%%%%%%%%%%%%%%%%%%%%%%%%%%%%%%%%%%%%%%%%%%%%%%%%%%%%%%%%%%%%%%%%%%%
\section{Methods} \label{sec:3_method}
\section{Method}
\label{sec: method}
% This section introduces the rendering pipeline of our proposed hierarchical compositional scene. 
% our pipeline consists of three processes, including decomposing the text into editable 3D layout, rendering the compositional views with local (object) NeRFs and global (scene) NeRF and the joint optimization on these hierarchical 3D representations.

% Note that the transformation between the object and the scene frame is defined by ${p}_o$ and ${D}_o$. 
%
% Next, we build a residual connection to add ${\sigma}_o$ and the referenced global color, and the rendering result will be used to calculate the SDS loss based on the global text.  
% Fig.~\ref{fig:framework} illustrates our pipeline, which consists of three main components, including the editable 3D scene layout based on multi-object text (Sec.~\ref{ssec:layout}), the scene rendering pipeline that composites the predictions from all local NeRFs (Sec.~\ref{ssec:render}), and the joint optimization on both local and global representation models (Sec.~\ref{sec:optimization}).
% To elaborate, our editable 3D scene layout represents a global frame of the scene by decomposing it into a set of local frames, where each is parameterized by a local NeRF, a 3D bounding box, and a corresponding local text prompt.
% For instance, the text prompt `A teddy bear and a stuffed monkey sit side by side' is interpreted as a 3D scene layout, as shown in Fig.~\ref{fig:framework}.  
% The whole 3D layout, \ie, scene frame, consists of two 3D bounding boxes, \ie local frames \#1 and \#2, with specific local text prompts, \ie, `a teddy bear' and `a stuffed monkey'. 
% %
% To render the scene view, we first calculate the ray-box intersections between the boxes and rays $({\boldsymbol{r}}_o, \boldsymbol{\phi}_d, {\boldsymbol{\theta}}_d)$, where the ${\boldsymbol{r}}_o$ is the ray origin and the $({\boldsymbol{r}}_o, \boldsymbol{\phi}_d)$ is its direction.
% Then, to infer each object's properties in local NeRFs, we sample the global points $({\boldsymbol{x}}_g, {\boldsymbol{y}}_g, {\boldsymbol{z}}_g)$ in the global frame within the ray-box intersection intervals and project them into the normalized local location $({\boldsymbol{x}}_l, {\boldsymbol{y}}_l, {\boldsymbol{z}}_l)$ in the local frame.
% %
% Given the local sampling points $({\boldsymbol{x}}_l, {\boldsymbol{y}}_l, {\boldsymbol{z}}_l)$, the implicit local NeRF ${\boldsymbol{\theta}}_l$ outputs four pseudo-color channels ${\boldsymbol{C}}_l$ and density $\boldsymbol{\sigma}$, which can be used to render a local view of the local frame to match its local text prompt.
% %
% We further calibrate the predicted pseudo-color $\boldsymbol{C}_l$ from local frames by adding the global embeddings ${\boldsymbol{emb}}_g$ to improve the global view consistency.
% Then, the calibrated predictions after composition are used to reconstruct the scene view by volumetric rendering along the rays.
% %
% Lastly, the rendered views based on local and global frames are guided by score distillation sampling loss $\nabla \mathcal{L}_{\text{SDS}}$~\cite{poole2022dreamfusion} to optimize all the learnable parameters. 
To resolve the issue of guidance collapse, our principal strategy is to \textit{decompose the scene into reusable components and compose/recompose them into a unified and consistent one}.
This enables flexible control over the generated content with direct use of prompts and box layouts, as illustrated in \cref{fig:teaser}.
%
Our proposed CompoNeRF confers several key benefits:
1) \textbf{Semantic Coherence}: It reliably creates 3D objects with detailed textures and global consistency, exemplified by authentic light interactions, such as reflections on the bed surface.
2) \textbf{Modularity and Reusability}: CompoNeRF functions as an ensemble of independently trained NeRF models. These can be efficiently stored and later retrieved from a cached dataset, enabling their reuse in various cases.
3) \textbf{Editability}: Our approach allows for flexible scene modification, such as interchanging the lamp for a vase filled with sunflowers or altering its scale, by simply adjusting the box dimensions for later finetuning. This feature enhances flexibility and creative possibilities. 


% Furthermore, the usage of layout boxes enables more flexible control over the generated content compared with the intricate sketch shape in Latent-NeRF\cite{metzer2022latent}. 
\begin{figure*}[t]
    \centering
    \includegraphics[width=0.9\linewidth]{figures/method.pdf}
    % \vspace{-12pt}
    \caption{\textbf{Framework Overview}.
The CompoNeRF model unfolds in three stages: 1) Editing 3D scene, which initiates the process by structuring the scene with 3D boxes and textual prompts; 2) Scene rendering, which encapsulates the composition/recomposition process, facilitating the transformation of NeRFs to a global frame, ensuring cohesive scene construction. Here, we specify design choices between density-based or color-based(without refining density) composition; 3) Joint Optimization, which leverages textual directives to amplify the rendering quality of both global and local views, while also integrating revised text prompts and NeRFs for refined scene depiction.
  % The model is structured into three components: Composition, Decomposition, and Recomposition. Composition deals with the foundational setup, detailed with choices for density-based and color-based composition. Decomposition utilizes the modularity of the CompoNeRF feature, caching each NeRF module offline for efficient recalibration. Recomposition reuses these cached NeRFs and adjusts the semantic context, providing a revised output with the inclusion of the offline NeRF enhancements.
    % Our model consists of two branches where the upper part is individual NeRFs, and the lower part denotes global calibration with our tailored composition model. The specific designs for density-based and color-based composition modules are highlighted. 
    % CompoNeRF consists of three parts: 1). The editable 3D scene layout configures the scene representations with 3D boxes and text prompts; 2).  The scene rendering includes the global calibration and the compositional process; 3). The joint optimization applies global and local text guidance on global and local render views.
    % The global frame (scene space) contains a set of local frames. Each is  represented by a local NeRF associated with a 3D box and text prompt defined by the editable 3D layout.
    % The scene view is volumetric rendered by sampling the points $({\boldsymbol{x}}_g, \boldsymbol{y}_g, \boldsymbol{z}_g)$ intersected with any local frame along the ray $(\boldsymbol{r}_o, {\boldsymbol{\phi}}_d, \boldsymbol{\theta}_d)$.
    % The sampling points are first inferred through the local NeRF with the local frame locations $({\boldsymbol{x}}_l, \boldsymbol{y}_l, \boldsymbol{z}_l)$ projected from the global location $({\boldsymbol{x}}_g, \boldsymbol{y}_g, \boldsymbol{z}_g)$.
    % And then, all the local predictions are calibrated by a global MLP with conditional input to render the scene view.
    % During the optimization, the text guidance is applied to both local views predicted by local frames only and global views predicted by the composition of all local frame predictions.
    }
    \label{fig:framework}
    % \vspace{-8pt}
\end{figure*}

\subsection{Preliminaries}
Defining individual object bounding boxes as \textit{local frames} and the overall scene coordinate system as the \textit{global frame}, we build the foundation of NeRF and diffusion processes.

\label{sec:background}
\noindent \textbf{3D Representation in Latent Space.}
Our methodology capitalizes on the state-of-the-art text-to-image generative model—Stable Diffusion as described by Rombach et al\cite{rombach2022high}.
We build upon the Latent-NeRF framework~\cite{metzer2022latent}, which computes latent colors for individual objects by considering their sample positions within a localized frame. Specifically, it maps a three-dimensional point in local coordinates \(\boldsymbol{x}_l = (x_l, y_l, z_l)\) to a volumetric density \(\boldsymbol{\sigma}_l\) and an associated color \(\boldsymbol{C}_l\), expressed as \((\boldsymbol{C}_l, \boldsymbol{\sigma}_l) = f_{\boldsymbol{\theta}_l}(x_l, y_l, z_l)\). Here, \(f\) represents a Multi-Layer Perceptron (MLP) characterized by parameters \(\boldsymbol{\theta}_l\).
 This NeRF-generated color is then assessed in the context of the Stable Diffusion model, using text prompts to guide NeRF toward spatially coherent inference with intricate context.
% to infer pseudo-color for each object using local NeRF.
% Specifically, the representation maps a point $\boldsymbol{x}_l = \left({x}_l, {y}_l, {z}_l\right)\in [-1, 1]$ in the local frame to its corresponding volumetric density $\boldsymbol{\sigma}_l$ and emitted color $\boldsymbol{C}_l$, \ie,  $\left(\boldsymbol{C}_l, {\boldsymbol{\sigma}_l}\right)=\boldsymbol{\theta}_{_l}\left({x_l}, {y}_l, {z}_l\right)$.
% The predicted pseudo-color is fed forward into the decoder of the Stable Diffusion model to obtain the final rendering result.

\noindent \textbf{Volume Rendering with Multiple Objects.}
% For each local frame $j$ with NeRF parameterized as $\theta_j$, we follow original NeRF design\cite{nerf} to integrate $(\boldsymbol{C}_l, \boldsymbol{\sigma}_l)$ of   sampled points from any hit ray $r_l=(\boldsymbol{o}_l, \boldsymbol{d}_l)$ by,
% For consistent scene rendering, object transmittance $T_k$ must be recalculated in the global frame based on independent properties inferred from local NeRFs. Hence, we sort predictions according to their distance to $\boldsymbol{o}_g$. 
% Similar to \cref{eq:volrend}, global color $\hat{\boldsymbol{C}}_g$ of ray $\boldsymbol{r}_g=(\boldsymbol{o}_g, \boldsymbol{d}_g)$ is predicted by the volumetric rendering integrating over $m$ objects,
We extend the volume rendering process to accommodate multiple objects by assigning each a local frame, denoted as $j$, with NeRF parameters $\boldsymbol{\theta}_{l, j}$. Drawing from the foundational NeRF approach \cite{nerf}, in each local frame, we integrate the color $\boldsymbol{C}_l$ and density $\boldsymbol{\sigma}_l$ for points $\boldsymbol{x}_l$ sampled along a ray $\boldsymbol{r}_l$, emanates from the camera origin $\boldsymbol{o}_l$ in direction $\boldsymbol{d}_l$. This is formalized in the predicted color integration for $\hat{\boldsymbol{C}}_l$ as:
{\setlength\abovedisplayskip{2pt}
\setlength\belowdisplayskip{2pt}
\begin{equation}
\label{eq:volrend}
{\hat{\boldsymbol{C}}_l}({\boldsymbol{r}_l})=\sum_{k=1}^{N} T_{l, k} \left(1-\exp \left(-\sigma_{l, k} \delta_k\right) \right) {\boldsymbol{C}}_{l,k},
\end{equation}}where $T_{l, k}=\exp \left(-\sum_{j=1}^{k-1} \sigma_{l,j} \delta_j\right)$ represents the transmittance to the $k$-th of total $N$ sample, calculated exponentially over the cumulative density along $\boldsymbol{r}_l$, and $\delta_k$ is the interval between adjacent samples.
%
To synthesize a coherent scene, we transition from processing individual local frames to a collective global frame. Within this global context, we reconcile object attributes inferred from their individual local NeRFs for refined $\boldsymbol{\sigma}_g, \boldsymbol{C}_g$ along with $T_{g, k}$. The samples $\boldsymbol{x}_g$ are ordered based on their spatial distances from the origin $\boldsymbol{o}_g$ following the coordinate transformation. We then express the volumetric rendering of a ray $\boldsymbol{r}_g$ integrating $m$ objects within the global frame as follows:
{
\setlength\abovedisplayskip{2pt}
\setlength\belowdisplayskip{2pt}
\begin{equation}
\label{eq:multi_volrend}
{\hat{\boldsymbol{C}}_g}({\boldsymbol{r}_g})=\sum_{k=1}^{m*N} T_{g, k} \left(1-\exp \left(-\sigma_{g, k} \delta_k\right) \right) {\boldsymbol{C}}_{g,k}. 
\end{equation}}

\noindent \textbf{Score Distillation Sampling.}
% During the SDS process, a noise image $\boldsymbol{X}_t$ is first generated by adding a sampled noise $\epsilon \sim \mathcal{N}(0, I)$ in noise level $t$ into a rendered view $\boldsymbol{X}$ from a NeRF.
To facilitate the conversion from text descriptions to 3D models, DreamFusion~\cite{poole2022dreamfusion} utilizes Score Distillation Sampling (SDS), leveraging the generative capabilities of a diffusion model, denoted as $\phi$, to guide the optimization of NeRF parameters, symbolized as $\boldsymbol{\theta}$.
%
Initially, SDS creates a noisy image $\boldsymbol{X}_t$ by infusing a randomly sampled noise $\epsilon$, which follows a normal distribution $\mathcal{N}(0, I)$, into a NeRF-rendered image $\boldsymbol{X}$ at a given noise level $t$.
The diffusion model $\phi$ then estimates the noise $\epsilon_\phi\left(\boldsymbol{X}_t, t, T\right)$ from this noisy image, conditioned by the noise level $t$ and an optional text prompt $T$. 
The key step in SDS involves calculating the gradient of the loss function, which measures the discrepancy between the estimated noise and the originally added noise:
{\setlength\abovedisplayskip{2pt}
\setlength\belowdisplayskip{2pt}
\begin{equation}
\label{eq:sds_loss}
\nabla_\theta \mathcal{L}_{\text{SDS}}(\boldsymbol{X}_t, T)=  w(t)\left(\epsilon_\phi\left(\boldsymbol{X}_t, t, T\right)-\epsilon\right),
\end{equation}}where $w(t)$ is a weighting function that adjusts the influence of the gradient based on the noise level. 
The gradients across all rendered views direct the update of $\boldsymbol{\theta}$, ensuring that the NeRF-generated images align with the text descriptions. Additionally, we incorporate the 'perturb and average' technique from SJC for more robust $\mathcal{L}_{\text{SDS}}$. For a comprehensive understanding of these methods, the reader is directed to the detailed explanations provided in \cite{poole2022dreamfusion,wang2022score}.

%
%
% \subsection{Editable 3D Scene Layout}
% \label{ssec:layout}
% The 3D scene layout explicitly combines language structures with 3D layouts in an editable way.
% Given the input text prompt $T$, the attribute-object pairs can be easily obtained based on user control.
% Note that the text prompt indicates the multi-object text prompt by default.
% % available for free in many structured representations, such as the constituency tree.
% As shown in Fig.~\ref{fig:framework}, we can extract multiple noun phrases with their binding attributes and map these local text prompts into corresponding regions.
% Specifically, we define the scene structure with $m$ local frames, each employs a local NeRF $\boldsymbol{\theta}_l$ as representation, the local text prompt $T_{l} \subseteq{T}$ and its spatial layout with 3D boxes $\mathbf{b} = \{\mathbf{p}, \mathbf{s}\} \in  \mathbb{R}^6$ of each object entity, where $\mathbf{p}=\{p_x, p_y, p_z\}$ refers to the center point and $\mathbf{s}=\{s_x, s_y, s_z\}$ denotes the box scale. 
% \textit{Our editable 3D layout is easy to be collected and edited with its simplicity, allowing for versatile and interactive user control by modifying the box's or text's properties to define a new scene}.
% Moreover, as depicted in Fig.~\ref{fig:teaser}, each component in a 3D scene layout can be replaced or re-composited with other trained local NeRFs, which is more friendly for flexible user editions compared with using only text prompts.
% We fine-tuned the new layout by global rendering, which enables scalable re-editing.
% Each relationship $r_k \in R$ is a triplet in a <subject-predictive object> format, where a subject node is. After we generate the scene graph from the complex prompts, we can sample the closest relationship with the 2d spatial layout as the initial 3D position. fine-tuned the new layout by global rendering, which enables scalable re-editing
%
% \subsection{Scene Rendering Pipeline}
% \label{ssec:render}
% In CompoNeRF, the scene images are rendered by a ray-casting approach following the design of NeRF.
% % Each ray to be cast is generated based on the camera pose, intrinsic, and transformation.
% The camera is defined by a pinhole camera model, casting a set of rays $(\boldsymbol{r}_o, \boldsymbol{\phi}_d, {\boldsymbol{\theta}}_d)=\boldsymbol{o}+t\boldsymbol{d}$ through each pixel on the frame of size $H \times W$, where the $\boldsymbol{r}_o \in  \mathbb{R}^3$ is the origin and the $(\boldsymbol{\phi}_d, \boldsymbol{\theta}_d)$ is the viewing direction.
% Along this ray, we sample all the points intersected with any layout box of local frames.
% For each hit sampled point, the color and volumetric density are computed through the local NeRF of the hit local frame.
% The ray color perdition is calculated by the differentiable integration applied on all the point-predicted colors and volumetric density along the ray.
%
% \noindent \textbf{Ray-box Intersection with Local Frames.}
% Given a ray $\boldsymbol{r}_i$, each box $\boldsymbol{b}_j$ of the local frame is applied with the AABB ray intersection test algorithm to check the intersections.
% When the ray $r_i$ is hit with a box $\boldsymbol{b}_j$ of the local frame, we use the entrance and exit points as near $\boldsymbol{t}_{in}$ and far $\boldsymbol{t}_{out}$ bounds to sample $N$ equidistant quadrature points, $
% \boldsymbol{t}_{i,j,n}=\frac{n-1}{N-1}\left(\boldsymbol{t}_{out}-\boldsymbol{t}_{in}\right)+\boldsymbol{t}_{in} , n \in \left[1, N\right]$
% % Despite each local frame only having a small number of hit rays compared to the scene, we observe that it is enough to represent each object accurately while maintaining short rendering times.
% Note that the coordinates of sampled points are first projected into normalized coordinates using the box scale of local frames to enable each local NeRF to learn the scale-independent representation.
% The bounding box $\mathbf{b}$ of the local frame in global coordinate can be transformed into a canonical bounding box by ${(\mathbf{b}} - \boldsymbol{p}) / \mathbf{s}$.
% Considering the rendering efficiency, we only calculate the valid points, interacted with the boxes, and set all the empty points with a constant background color.
%
% The appearance of a set object representations depends on its interaction with the scene and illumination which should be decided by the local frame location.
% To ensure the volumetric consistency, we only calibrate the emitted color with scene location, while the gradient still can be propagated.
% Since the overall color depends on both the global  positions $({x}_w, {y}_w, {z}_w)$ and ray directions $({\phi}_d, {\theta}_d)$, the global color embedding is learned based on both the positions and ray directions.
% Since the overall color depends on both the global  positions $({x}_w, {y}_w, {z}_w)$ and ray directions $({\phi}_d, {\theta}_d)$, the global color embedding is learned based on both the positions and ray directions.
% \subsection{The Proposed CompoNeRF}
% \subsubsection{Composition Module}
% CompoNeRF aims to composite multiple NeRFs to reconstruct multi-object scenes with both box and prompt guidance.
% %
% Our framework, as shown in \cref{fig:framework}, applies the AABB ray intersection test algorithm to check for intersections on each box in the global frame. We then samples $\boldsymbol{x}_g$ within the ray box intervals, and project them to $\boldsymbol{x}_l$ to infer  $\left(\boldsymbol{C}_l, {\boldsymbol{\sigma}_l}\right)$ in separate NeRF models. 
% %
% We then utilize volume rendering to obtain rendered views for each local frame respectively. 
% %
% After that, they would be passed on to our tailored composition Module to infer 
% $\left(\boldsymbol{C}_g, {\boldsymbol{\sigma}_g}\right)$
% for global rendering. 
% Next, we match local and global texts with their corresponding image outputs by SDS losses. 
% We also support recomposition by passing samples from cached models into $\boldsymbol{x}_l$ to continue the above process.
\begin{figure}[t!]
    \centering
    \includegraphics[width=\linewidth]{figures/abls.pdf}
    % \vspace{-22pt}
    % \caption{Ablation study on text guidance. (a) without local SDS losses. (b) without global SDS losses. (c) vanilla SDS losses without perturb and average scoring~\cite{wang2022score}. (d) full model.}
    \caption{\textbf{Design Impact Comparison: Density vs. Color-based Methods.} The top row illustrates the density-based approach's detailed rendering and quick convergence in the 'table wine' scene. The bottom row highlights the color-based method's enhancements and its drawbacks, such as geometric and shadow inaccuracies, particularly in close-up views and slow convergence.
    % \textbf{(a)} global text guidance(integrating local frames by \cref{eq:multi_volrend}) and global calibration(integrating local frames, then aligning the rendering result directly with the full text). 
    }
    \label{fig:abls}
    % \vspace{-20pt}
\end{figure}
\subsection{The Proposed CompoNeRF}
\subsubsection{Composition Module}
CompoNeRF is designed to composite multiple NeRFs to reconstruct scenes featuring multiple objects, utilizing guidance from both bounding boxes and textual prompts. Within our framework, depicted in \cref{fig:framework}, the Axis-Aligned Bounding Box (AABB) ray intersection test algorithm is applied to ascertain intersections across each box in the global frame. Subsequently, we sample points \(\boldsymbol{x}_g\) within the intervals of the ray-box and project them to \(\boldsymbol{x}_l\) to deduce the corresponding color \(\boldsymbol{C}_l\) and density \(\boldsymbol{\sigma}_l\) within individual NeRF models.
%
These properties are processed through our composition module to infer the global color \(\boldsymbol{C}_g\) and density \(\boldsymbol{\sigma}_g\), crucial for the global rendering.
%
Volume rendering techniques~\cite{kajiya1984ray} are then employed to procure the rendered views for both local and global frames. We propose dual SDS losses to ensure coherence between the image outputs and their corresponding textual descriptions. Additionally, our approach facilitates recomposition by channeling samples from cached models back into local frames along with the text revision, thereby streamlining the integration.

% As shown in \cref{fig:abls}(a), we verify its necessity by dropping $\nabla \mathcal{L}_{\text{SDS}_g}$. 
% %
% Compared with our full model, its layout does not fit our shared sense of a room, \ie, \emph{nightstand} is usually lower than \emph{bed}; \emph{lamp} needs a base to support it. Additionally,  it lacks global consistency, such as light reflection, to make it more realistic. 
% %
% Therefore, we leverage the full text semantics to ensure consistent global rendering across local frames. 
% %
% Instead of conditioning the global rendering view with the full prompt directly, we note that global calibration is necessary for geometry and color to be learned sufficiently.
% For example, we observe that geometric completeness and texture of \emph{nightstand} are not ideal. Although reflection appears around \emph{nightstand}, \emph{bed} is stripped of the light. 
% %
% Therefore, we opt to leverage the correlation between the rendering output of the combined NeRFs and the overall semantics to perform multi-object scene reconstruction.  
%

\noindent\textbf{Global Composition.}
The independent optimization of each local frame may inadvertently result in a lack of global coherence within the scene. To address this, our scene composition process is designed to integrate these frames, thereby achieving a more consistent result.
%
Before exploring the specifics of the module, it is imperative to discuss two critical design decisions within the composition module, as depicted in \cref{fig:framework}.
%
Upon integrating the properties inferred from \(\boldsymbol{x}_g\) into the composition module, they are fine-tuned through gradients derived from the global SDS loss.  This process leads to a critical consideration: the necessity and implications of refining the global density \(\boldsymbol{\sigma}_g\). This can be divided into two approaches: \textbf{1) Density-based:} The advantage of adjusting \(\boldsymbol{\sigma}_g\) is that it can adjust geometry, thus yielding a scene more congruent with the global text prompt. 
However, this comes at the cost of potentially compromising the optimal color \(\boldsymbol{C}_g\), as calibrating \(\boldsymbol{\sigma}_g\) introduces more uncertainty for subsequent color refinement as it requires prior density features $\boldsymbol{h}$ as shown at \cref{fig:compo}. 
\textbf{2) Color-based:} Conversely, directly employing \(\boldsymbol{\sigma}_l\) mitigates this uncertainty but at the expense of reduced geometric control, presenting a challenging balance to strike in the pursuit of precise scene composition.
% , which may lead to suboptimal outcomes.
%
After thorough experiments, exemplified in \cref{fig:abls}, we have opted for the density-based approach to refine \(\boldsymbol{\sigma}_g\)  prioritizing both \textbf{accuracy and efficiency}. The test revealed that it excels in rendering intricate details, such as enhanced wood grain textures and more naturally contoured 'salad', as accentuated by boxes. This method also demonstrated a swifter convergence rate. Conversely, while the color-based improved reflections and reduced flickering on the 'wine cup', it was plagued by issues such as sparse density, which adversely brings holes at the base of the 'cup' and the corner of the 'table'.
Furthermore, upon close examination, it becomes evident that shadow artifacts of 'wine' on the 'table' are pronounced, suggesting that its disadvantages outweigh its advantages.
%  in this context
% \textbf{Global Composition.}
% Each local frame is optimized independently, causing a lack of global connections for scene composition.
% Before delving into module details, there are two choices (see \cref{fig:framework}) on the composition module design we need to elaborate on first. 
% %
% In \cref{fig:framework}, by taking $\boldsymbol{x}_g$ into the composition module, their inferred properties are calibrated with gradients propagated from the global SDS loss. 
% However, it remains unclear whether $\boldsymbol{\sigma}_g$ should be refined or not. 
% %
% The trade-off on its usage is the density adjustment bringing a more reasonable layout and more geometric details that fit the global text prompt. While its potential downside is that $\boldsymbol{C}_g$ may not be optimal as $\boldsymbol{\sigma}_g$ has more uncertainty compared to $\boldsymbol{\sigma}_l$, bringing sub-optimal rendering results. 

% We choose the density-based method after comparing them with the experiment shown in \cref{fig:abls}. 
% %
% Specifically, we test both designs on the scene \emph{table wine} and discover that the density-based design provides more intrinsic details(as indicated by green boxes), \eg, enriched wood grains, and a more natural shape for \emph{salad} and has much faster convergence speed. In contrast, the color-based method enhances the reflection and smooths flickering on \emph{wine cup}, (as indicated by red boxes), but it suffers from 1) sparse density, resulting in poorly generated geometry at the base of  \emph{cup} and the wood \emph{table} corner. Additionally, shadow artifacts appeared on \emph{table} when viewed up close, outweighing benefits of the color-based method.

\begin{figure}[t!]
    \centering
    \includegraphics[width=\linewidth]{figures/compo_module.pdf}
    % \vspace{-24pt}
    % \caption{Ablation study on text guidance. (a) without local SDS losses. (b) without global SDS losses. (c) vanilla SDS losses without perturb and average scoring~\cite{wang2022score}. (d) full model.}
    \caption{\textbf{Detail of Composition module}: density-based design. 
    }
    \label{fig:compo}
    % \vspace{-18pt}
\end{figure}
\noindent\textbf{Network Design.}
The compositional framework of our network, as delineated in \cref{fig:compo}, is predicated on an architecture that employs a suite of MLPs, represented as \(\{\boldsymbol{\theta}_l\}_{l=1}^{m}\),  each dedicated to a distinct local frame. To harmonize \(\boldsymbol{\sigma}_l\) and \(\boldsymbol{C}_l\), we incorporate global MLPs, including density calibrator $f_{\boldsymbol{\theta}_{g_d}}$ and color calibrator $f_{\boldsymbol{\theta}_{g_c}}$.
%
A transformation module complements this system, tasked with maintaining the spatial coherence between the global and local frames. It governs the transformation of sampling points $\boldsymbol{x}$, ray directions $\boldsymbol{d}$, and adjacent sampling distances $\delta$. This module also orders the points $\{\boldsymbol{x}_{g,j}\}_j$ by their distance to the global camera origin $\boldsymbol{o}_g$, ensuring that each local point $\boldsymbol{x}_l$ is accurately matched with its corresponding global point $\boldsymbol{x}_g$ for subsequent volume rendering. 
%
The network design is:
{
\setlength\abovedisplayskip{4.5pt}
\setlength\belowdisplayskip{4.5pt}
\begin{align}
\label{eq:g_c_d}
{\boldsymbol{\sigma}_g}  &= \alpha_d f_{\boldsymbol{\theta}_{g_d}}({\boldsymbol{x}_g}) + \boldsymbol{\sigma}_l, \\
{\boldsymbol{C}_g}  &= \alpha_c f_{\boldsymbol{\theta}_{g_c}}(\boldsymbol{h}, {\boldsymbol{d}_g}) + \boldsymbol{C}_l. 
\end{align}}In contrast to the local frames, the global frame's color output $\boldsymbol{C}_g$ is inferred based on $\boldsymbol{h}$ and conditional on $\boldsymbol{d}_g$ to enable a view-dependent lighting effect.
% Denote the density features as $\boldsymbol{h}$. 
%
%
Residual learning is leveraged here, where \(\boldsymbol{\sigma}_l, \boldsymbol{C}_l\) serve as foundational elements that support the learning of global density \(\boldsymbol{\sigma}_g\) and color \(\boldsymbol{C}_g\). The parameters \(\alpha_d, \alpha_c\) are adjustable, allowing fine-tuning of the influence that local components exert on the global outputs.
%
It is imperative to acknowledge that in our color-based method, density calibration is intentionally excluded to concentrate solely on the refinement of color dynamics as shown at \cref{fig:framework}. This is achieved by conditioning the process on both spatial and directional global inputs \((\boldsymbol{x}_g, \boldsymbol{d}_g)\), as demonstrated in the following equations:
\begin{align}
\setlength\abovedisplayskip{4.5pt}
\setlength\belowdisplayskip{4.5pt}
\label{eq:g_c_c}
\boldsymbol{\sigma}_g = \boldsymbol{\sigma}_l, \quad
{\boldsymbol{C}_g} = \alpha_c f_{\boldsymbol{\theta}_{g_c}}({\boldsymbol{x}_g}, {\boldsymbol{d}_g}) + \boldsymbol{C}_l.
\end{align}
The integration of extra $\boldsymbol{x}_g$ aims to facilitate a fair comparison under same inputs with the density-based. It enhances the visual appeal of effects like the wine cup's reflection, as demonstrated in \cref{fig:abls}. However, this method is not without its compromises. It tends to produce artifacts and is characterized by a slower convergence rate. Additionally, this approach limits the ability to precisely control density, subsequently impacting the intricate geometric details.


\begin{figure*}[t!]
    \centering
    \includegraphics[width=\linewidth]{figures/sota.pdf}
    % \vspace{-24pt}
    \caption{\textbf{Qualitative comparison with other text-to-3D methods using multi-object text prompts}. Cases 1-3 demonstrate simpler settings characterized by compositions involving two objects. In contrast, Cases 4-8 delve into more intricate scenarios featuring compositions with more than two objects. Smaller images are presented to illustrate the generated local NeRFs(partially shown in Cases 4-8).}
    \label{fig:sota}
    % \vspace{-5pt}
\end{figure*}
%
% \begin{table*}[t!]
% \centering
% \resizebox{\textwidth}{!}
% {
% \begin{tabular}{cccccccc}
% \toprule
% Method            & \rotatebox{60}{table wine}  & \rotatebox{60}{teddy monkey} & \rotatebox{60}{computer mouse} & \rotatebox{60}{bed room}  & \rotatebox{60}{chess} & \rotatebox{60}{pisa tower} & \rotatebox{60}{astronaut} & \rotatebox{60}{tesla}  \\ \midrule
% LatentNeRF  & 21.55 & 27.38 & 17.13 & 21.86 & 31.19 & 24.31 & 27.07 & 25.16 \\
% SJC & 23.33 & 27.37 & 18.00 & 22.54 & 30.53 & \textbf{26.18 }& 27.84 & 23.55 \\
% CompoNeRF & \textbf{32.68} & \textbf{28.57}	 &\textbf{ 22.34} &\textbf{ 28.65} & \textbf{31.45} & \textbf{28.96} & 25.82 & 25.95 & 24.42 & \textbf{32.71} & \textbf{26.13 }& \textbf{26.38} & \textbf{30.98} & \textbf{33.37} \\
% \bottomrule
% \end{tabular}
% }
% \vspace{-10pt}
% \caption{Performance of our CompoNeRF in different 3D scenes. We use CLIP score \cite{parmar2023zero,zhang2023sine,wang2023imagen} as our evaluation metric, which is a common evaluation metric in text-to-image generation tasks to evaluate the similarity of the generated image to the text prompt. }
% \label{perclass}
% \end{table*}
%
\begin{table*}[t!]
% \scalebox{0.8}
\renewcommand{\arraystretch}{1.2}
\fontsize{4pt}{4pt}
\selectfont 
\centering
% \vspace{-8pt}
\resizebox{\textwidth}{!}
{
% \begin{tabular}{lcccccccc}
% \hline
% Method     & table\_wine    & tesla          & pyramid        & chess          & apple and banana      & astronaut      & glass\_balls   & Eiffel\_tower    \\ \hline
% LatentNeRF & 21.55          & 25.16          & 27.43          & 31.19          & 27.69          & 27.07          & 29.51          & 26.32          \\
% SJC        & 23.33          & 23.55          & 25.62          & 30.53          & 28.21          & 27.84          & 28.76          &27.41 \\
% \textbf{CompoNeRF(Ours)}     & \textbf{32.68} & \textbf{26.13} & \textbf{28.96} & \textbf{31.45} & \textbf{33.37} & \textbf{32.71} & \textbf{30.98} & \textbf{28.44}          \\ \hline
% \end{tabular}
\begin{tabular}{lcccccccc}
\hline
Method                   & Case 1         & Case 2         & Case 3         & Case 4         & Case 5         & Case 6         & Case 7         & Case 8         \\ 
\hlineB{1.1}
LatentNeRF               & 25.16          & 27.07          & 27.69          & 31.19          & 21.55          & 26.32          & 27.43          & 29.51          \\
SJC                      & 23.55          & 27.84          & 28.21          & 30.53          & 23.33          & 27.41          & 25.62          & 28.76          \\
\textbf{CompoNeRF (Ours)} & \textbf{26.13} & \textbf{32.71} & \textbf{33.37} & \textbf{31.45} & \textbf{36.06} & \textbf{28.44} & \textbf{28.96} & \textbf{30.98} \\ \hlineB{1.1}
\end{tabular}
}

% \vspace{-6pt}
\caption{\textbf{Performance comparison of our CompoNeRF in different 3D scenes}. For our evaluation metric, we utilize the average of CLIP scores~\cite{parmar2023zero,zhang2023sine,wang2023imagen} across different views, which serve to assess the similarity between the generated images and the global text prompt. }
\label{tb:perclass}
\end{table*}
% \cref{fig:framework} depicts the network architecture of the composition module. Denote $m$ as local MLP $\{\boldsymbol{\theta}_l\}_{l=1}^{m}$ for each local frame. Then, we introduce the global MLPs including density $\boldsymbol{\theta}_{g_d}$ and $\boldsymbol{\theta}_{g_c}$ calibrators to refine $\boldsymbol{\sigma}_l$ and $\boldsymbol{C}_l$. 
% %
% In detail, the network design is, 
% {
% % \setlength\abovedisplayskip{4.5pt}
% % \setlength\belowdisplayskip{4.5pt}
% \begin{align}
% \label{eq:g_c_d}
% {\boldsymbol{\sigma}_g}  &= \alpha_d \boldsymbol{\theta}_{g_d}({\boldsymbol{\sigma}_l}) + \boldsymbol{\sigma}_l, \\  
% {\boldsymbol{C}_g}  &= \alpha_c \boldsymbol{\theta}_{g_c}({\boldsymbol{C}_l},  {\boldsymbol{d}_g}) + \boldsymbol{C}_l, 
% \end{align}}
% %
% where residual $\boldsymbol{\sigma}_l, \boldsymbol{C}_l$ assist in learning $\boldsymbol{\sigma}_g$ and $\boldsymbol{C}_g$, while $\alpha_d, \alpha_c$ balance their contribution as learnable parameters.
% %
% Note that the color-based omits density calibration, and simply uses the shared color refinement.



% The 3D boxes are only used for the spatial configuration of local NeRFs, while the implicit representation of local NeRFs is inferred by the canonical samples inside the local frame without considering the global relationship across different objects.
% To relieve such location-dependent effects, we further calibrate the output color and density from the local NeRF with global coordinates $({\boldsymbol{x}}_g, {\boldsymbol{y}}_g, {\boldsymbol{z}}_g)$ and ray directions $\left({\boldsymbol{\phi}}_{d}, {\boldsymbol{\theta}}_{d}\right)$ as the conditional input.
% % to inject the global visual clues.
% %
% %
% Specifically, we adopt a shared MLP $\boldsymbol{\theta}_{g}$ to calibrate all the predicted object colors, that is,
% {\setlength\abovedisplayskip{4.5pt}
% \setlength\belowdisplayskip{4.5pt}
% \begin{align}
% \label{eq:MLP_dyn_2}
% {\boldsymbol{C}_g} = {\boldsymbol{C}_l} + \boldsymbol{emb}_{g} &= {\boldsymbol{C}_l} + \boldsymbol{\theta}_{g}({\boldsymbol{x}}_g, {\boldsymbol{y}}_g, {\boldsymbol{z}}_g, {\boldsymbol{\phi}}_{d}, {\boldsymbol{\theta}}_{d}),
% \end{align}}
% where ${\boldsymbol{C}_l}$ is the color predicted by the local NeRF.
% Therefore, the scene color can preserve the view-consistent behavior from the original architecture and add consistency across poses for the volumetric density.
% Since the color and density values share the same latent expression in $({\boldsymbol{x}}_l, {\boldsymbol{y}}_l, {\boldsymbol{z}}_l)$, we only calibrate the emitted scene color explicitly with the scene location, as the densities of local NeRFs also are implicitly adjusted during optimization.

% \noindent \textbf{Global and Local Volumetric Rendering.}
% After compositing all the interacted points, each ray $\boldsymbol{r}_i$ collects a set sampling points by $\{\boldsymbol{t}_{i,j,n} \}_{j=1, n=1}^{m_j, N}$, where $m_j$ is the number of the hit object.
% For each sampling point, the inference results with the respective 3D representations are the local color $\boldsymbol{c}_{l}$, global color $\boldsymbol{c}_{g}$, and density $\sigma$.

% In fact, the local view $\hat{C}_{l,j}$ of single object $j$ also can be rendered by the sampled points  belongs to the same local frames as shown at Fig.~\ref{fig:framework}.

\subsubsection{Recomposition}
Our architecture advances scene reconstruction by providing an intuitive interface for layout manipulation.  This capability is crucial for the reconfiguration of scene elements into novel scenes, as depicted in \cref{fig:framework}. Here, the input panel allows for adjustments in the attributes of bounding boxes, such as modifying the position and scale of the 'apple' bounding box prior to composition. The refinement process further involves sampling ray-box intervals from the global frame, leading to transformed coordinates with the corresponding ray samples that are then incorporated into the pipeline, as demonstrated in \cref{fig:compo}.
%
Each bounding box represents an individual NeRF, providing the flexibility to move, scale, or remove elements as needed. CompoNeRF's capabilities also extend to textual edits, exemplified by the transformation of 'wine' into 'juice'.
%
Since NeRFs have been well trained, we only finetune \(\theta_g, \theta_l\) to align text prompts to promote consistency of both local and global views.
%
Moreover, the NeRFs once retrained within the edited scene, are also structured to be decomposable and cacheable in future scene compositions.
% Our CompoNeRF architecture facilitates the seamless reconstruction of scenes leveraging existing models. It enables precise editing of bounding boxes parameterized by \(\{\boldsymbol{\theta}_l\}_{l=1}^{m}\), allowing for their reconfiguration into new layouts. Refer to \cref{fig:framework}, the input panel permits the modification of attributes such as the position and scale of the 'apple' node's bounding box prior to composition. The process is further refined by sampling from the updated ray-box intervals within the global frame, which are then projected onto \(\boldsymbol{x}_l\), ensuring a streamlined reconstruction that integrates the 'apple' effectively. This addition is executed with careful attention to color consistency, positioning the 'apple' adjacent to the 'French bread' to complement the scene's overall palette. Each bounding box represents an individual NeRF, which means they can be manipulated through moving, scaling, and removal operations. CompoNeRF also extends its editing prowess to textual modifications, as evidenced by the 'wine cup' now appearing filled with juice—a change propagated through both subtexts and the global test. 
% %
% Since NeRFs have been well trained, we only finetune $\theta_g, \theta_l$ to align text prompts to promote consistency of both local and global views . 
% %
% Moreover, the NeRFs, once retrained within the reimagined scene, are also structured to be decomposable and cacheable for subsequent scene compositions.

% , as shown in Fig.~\ref{fig:framework}.
% For each scene described by the multi-object text prompt $T$, we
% To enhance the guidance of local representations, we use the local text prompt $T_l \subseteq T$ of a single object to optimize the local NeRFs in local views.
% The scene views $\hat{\boldsymbol{X}}_g=\{\hat{\boldsymbol{C}}_{g,i}\}_{i=1}^{H\times W}$ is obtained from the predicted pixel values of $H \times W$ rays by compositing all the ray-box interaction values.
% Similarly, the rendered view $\hat{\boldsymbol{X}}_{l,j}$ of the local frame $\boldsymbol{\theta}_j$ without compositing other objects can be calculated by $\hat{\boldsymbol{C}}_{l,j}$, as depicted in Sec.~\ref{ssec:render}.
% We use the local color instead of the globally calibrated color to obtain a local view because the local NeRF should learn the object identity unrelated to its placed position, as the position can be different during user edition.
% % Compared to cropping the local region from a global view for training, separate rendering can avoid the undesired information from other objects brought by the occlusion and resolution adjustments.
% Formally, we employ the following loss as the learning objective,
\begin{figure*}[t!]
    \centering
    \includegraphics[width=\linewidth]{figures/editing.pdf}
    % \vspace{-23pt}
    \caption{\textbf{Scene Editing Outcome:} Demonstrated here are the stages of our recomposition, utilizing cached source scenes. Each NeRF is individually identified by colorful labels. These decomposed nodes are then positioned in the initial layout and subsequently calibrated to form the final composition. The detailed description of the ambient environment is underscored, enhancing the scene's realism.}
    \label{fig:app}
    % \vspace{-12pt}
\end{figure*} 
\subsubsection{Optimization}
\label{sec:optimization}
During optimization, our method employs dual text guidance to align rendering results with both global and local textual descriptions. The optimization objective is:
{
\small
\setlength\abovedisplayskip{2pt}
\setlength\belowdisplayskip{2pt}
\begin{equation}
\label{eqn:loss_f}
\mathcal{L}= {\alpha_g}\nabla\mathcal{L}_{\text{SDS}}(\hat{\boldsymbol{X}}_{g}, T) + {\alpha_l}\sum_{j=1}^{m} \nabla\mathcal{L}_{\text{SDS}}(\hat{\boldsymbol{X}}_{l,j}, T_{l,j}) + \beta\mathcal{L}_{\text{sparse}},\nonumber
\end{equation}
}where $T$ signifies the global text prompt, while $T_{l}$ pertains to a specific object within the global context. The hyperparameters $\alpha_{g}, \alpha_{l}$, and $\beta$ modulate the respective loss weights. 
% $\nabla \mathcal{L}_{\text{SDS}}$ is the score distillation sampling loss, as described in Sec.~\ref{sec:background}.
As suggested in~\cite{metzer2022latent}, we use $L_{\text{sparse}}$ included to penalize the binary entropy of local NeRFs' densities, thereby mitigating the issue of extraneous floating radiance.
Additionally, incorporating directional cues such as "front view" or "side view" into the input text, as suggested by \cite{poole2022dreamfusion,metzer2022latent} proves beneficial in specifying camera poses during the training phase, further enhancing the alignment of our generated scenes with the intended perspectives.
% Note that the global calibration in the scene frame can adaptively revise both $({C}_l, {\sigma})$ in local NeRF with $\nabla \mathcal{L}_{SDS}$ along with the back-propagating gradient.


%%%%%%%%%%%%%%%%%%%%%%%%%%%%%%%%%%%%%%%%%%%%%%%%%%%%%%%%%%%%%%%%%%%%%%%%%%%%%%%%%
\section{Results} \label{sec:4_results}
%%%%%%%%%%%%%%%%%%%%%%%%%%%%%%%%%%%%%%%%%%%%%%%
%%%%%%%        4. Results         %%%%%%%
%%%%%%%%%%%%%%%%%%%%%%%%%%%%%%%%%%%%%%%%%%%%%%%

\section{Results}
\label{sec:results}

\subsection{MOS prediction results}
\label{subsec:mos_results}
We first evaluate our MOS-prediction performance in comparison with other approaches. In particular, we compare against NISQA~\cite{mittag2019non}, which we modified to estimate human-accessed MOS. Originally, they estimate perceptual objective listening quality assessment (POLQA)~\cite{beerends2013perceptual} scores using a CNN and BLSTM architecture. We also compare against the PMOS model proposed in~\cite{dong2020pyramid}, which is identical in structure to our PMOS model. Finally, we include our proposed SE+PMOS approach~\cite{nayem2021incorporating} (no joint training), where our PMOS model is held fixed while the SE model is training using the embeddings from the PMOS encoder. 

We use four metrics to evaluate MOS-estimation performance: mean absolute error (MAE), epsilon insensitive root mean squared error (RMSE)~\cite{rec2012p}, Pearson’s correlation coefficient $\gamma$ (PCC), and Spearman’s rank correlation coefficient $\rho$ (SRCC). 

%    Later, both models are jointly-trained for fine tuning. Our proposed PMOS model is similar of \cite{nayem2021incorporating}, however, SE models are different in structure.

%%%%%%%%%%%%%%%%%%%%%%%%%%%%%%%%%%%%%%%%%%%%%%%%%%%%
% Table 1, MOS results
%%%%%%%%%%%%%%%%%%%%%%%%%%%%%%%%%%%%%%%%%%%%%%%%%%%%
\begin{table}[t!]

\centering
\caption{Performance comparison with MOS prediction models {comparing against the ground truth MOS obtained from human subjects}. Best results are shown in \textbf{bold}.}
\label{tab:mos_results}
% \vspace{-0.5em}
\resizebox{\columnwidth}{!}{%
\begin{tabular}{| l | c c c c | }
\cline{2-5}
   \multicolumn{1}{c|}{}         & {MAE}$\downarrow$ & {RMSE}$\downarrow$ & {PCC ($\gamma$)}$\downarrow$ & {SRCC ($\rho$)}$\downarrow$ \\ \hline
   
NISQA~\cite{mittag2019non}    & 0.62 ($\pm$0.18)        & 0.7 ($\pm$0.16)      & 0.71 ($\pm$0.14)           & 0.79 ($\pm$0.15)            \\
PMOS~\cite{dong2020pyramid}                      & 0.51 ($\pm$0.15)         & 0.57 ($\pm$0.12)          & 0.88 ($\pm$0.17)           & 0.88 ($\pm$0.14)           \\
SE+PMOS~\cite{nayem2021incorporating}                     & \textbf{0.45} ($\pm$0.08) & \textbf{0.52} ($\pm$0.09) & \textbf{0.9} ($\pm$0.12) & \textbf{0.91} ($\pm$0.1)           \\
Proposed                     & \textbf{0.45} ($\pm$0.08) & \textbf{0.52} ($\pm$0.09) & \textbf{0.9} ($\pm$0.12) & \textbf{0.91} ($\pm$0.1)         \\
\hline
\end{tabular}
}
% \vspace{-2em}
\end{table}

Table~\ref{tab:mos_results} shows the results, where our proposed approach and SE+PMOS clearly outperform the other MOS prediction models according to all metrics. MAE is minimized by $0.6$ compared to the original PMOS~\cite{dong2020pyramid} approach. There is also a $0.05$ reduction in RMSE. This justifies our proposed approach that combines MOS estimation and speech enhancement tasks. Note, however, that similar results are obtained for our proposed approach and the SE+PMOS approach, which suggests that joint training (e.g., fine tuning) may help speech enhancement more than MOS prediction.  




\subsection{Speech enhancement model}
\label{subsec:se_results}
%%%%%%%%%%%%%%%%%%%%%%%%%%%%%%%%%%%%%%%%%%%%%%%%%%%%
% Table 2, SE comparison results on COSINE & VOiCES
%%%%%%%%%%%%%%%%%%%%%%%%%%%%%%%%%%%%%%%%%%%%%%%%%%%%

% Please add the following required packages to your document preamble:
% \usepackage{multirow}
% \usepackage[table,xcdraw]{xcolor}
% If you use beamer only pass "xcolor=table" option, i.e. \documentclass[xcolor=table]{beamer}
\begin{table*}[t!]
\centering
\caption{Average results of the speech enhancement models in different performance metrics. Best results are shown in \textbf{bold}.}
\label{tab:results_cosineVoices}
\resizebox{\linewidth}{!}{%
\begin{tabular}{ | l | l | c c c c | c c c c | }
\cline{3-10}
\multicolumn{1}{l}{\multirow{2}{*}{}} &                    & \multicolumn{4}{ c |}{{COSINE}}                             & \multicolumn{4}{ c |}{{VOiCES}} 
\\ \hline
\multicolumn{1}{|l|}{{models}}                 & \multicolumn{1}{c|}{{loss func.}} & {PESQ}$\uparrow$ & {SI-SDR}$\uparrow$ & {ESTOI}$\uparrow$ & {MOS-LQO}$\uparrow$ & {PESQ}$\uparrow$ & {SI-SDR}$\uparrow$ & {ESTOI}$\uparrow$ & {MOS-LQO}$\uparrow$ \\ \hline
\multicolumn{1}{|l|}{{Mixture}}                                        & {-}                                                       & {1.46} & {0.53}   & {0.62}  & {4.04}    & {1.26} & {-1.3}   & {0.48}  & {2.74}    \\ \hline
\multicolumn{1}{|l|}{}                                                        & mse                                                              & 2.68          & 2.8             & 0.8            & 3.2              & 2.3           & 1.2             & 0.69           & 3.5              \\ 
\multicolumn{1}{|l|}{}                                                        & mos~\cite{fu2019learning}                                                              & 2.8           & 3.8             & 0.82           & 4.2              & 2.37          & 1.66            & 0.74           & 5.3              \\ 
\multicolumn{1}{|l|}{}                                                        & mse+sa                                                           & 2.72          & 3.1             & 0.82           & 4                & 2.35          & 1.6             & 0.7            & 3.8              \\ 
\multicolumn{1}{|l|}{}                                                        & mos+sa                                                           & 2.89          & 4.1             & 0.85           & 4.4              & 2.42          & 1.72            & 0.77           & 5.7              \\ 
\multicolumn{1}{|l|}{\multirow{-5}{*}{SE}}                                    & sdr~\cite{kawanaka2020stable}                                                              & 2.7           & 4.5             & 0.82           & 3.4                & 2.32          & 2.01            & 0.72           & 3              \\ \hline
\multicolumn{1}{|l|}{{ }}                                 & mse                                                              & 3.1           & 4               & 0.85           & 4.2              & 2.48          & 1.8             & 0.8            & 6                \\ 
\multicolumn{1}{|l|}{{}}                                 & mse+sa                                                           & 3.19          & 4.6             & 0.93           & 4.8              & 2.54          & 2.08            & 0.86           & 6.3              \\  
\multicolumn{1}{|l|}{\multirow{-3}{*}{SE+PMOS~\cite{nayem2021incorporating}}}        & mse+sa+mos                                                       & 3.19          & 4.5             & 0.92           & \textbf{5.1}     & 2.53          & 2.06            & 0.84           & \textbf{6.5}     \\ \hline
\multicolumn{1}{|l|}{}                                                        & pesq                                                             & \textbf{3.28} & 4.4             & 0.9            & 5                & \textbf{2.67} & 2.01            & 0.83           & 6.1              \\ 
\multicolumn{1}{|l|}{\multirow{-2}{*}{MetricGAN~\cite{fu2019metricGAN}} }                             & stoi                                                             & 3.19          & 4.3             & \textbf{0.94}  & 4.8              & 2.5           & 2               & \textbf{0.87}  & 5.8              \\ \hline
\multicolumn{1}{|l|}{SSEMS~\cite{zezario2019specialized}}                                                   & qnet ($\phi=0dB$)                                                       & 2.85          & 2.99            & 0.83           & 3                & 2.4           & 1.8             & 0.7            & 2.8              \\ \hline
\multicolumn{1}{|l|}{{Chi++\textsubscript{fQSM,bS}~\cite{nayem2021towards}}}                     &    dc+cls+sa                                                              & 2.9           & 3.3             & 0.84           & 3.4              & 2.44          & 1.78            & 0.7            & 3                \\ \hline
\multicolumn{1}{|l|}{}                                 & mse+sa                                                           & 3.25          & 4.8             & \textbf{0.94}  & 4.75             & 2.64          & 2.1             & \textbf{0.87}  & 6.2              \\ 
\multicolumn{1}{|l|}{\multirow{-2}{*}{Proposed}} & mse+sa+mos                                                       & 3.25          & \textbf{4.82}   & \textbf{0.94}  & 5.04             & 2.64          & \textbf{2.13}   & \textbf{0.87}  & 6.47             \\ \hline
\end{tabular}
}
\end{table*}
For speech enhancement, we compare against a baseline approach without an attention mechanism \cite{graves2013speech}. We denote this baseline approach as SE. Five separate loss functions are applied to optimize this approach, and they are MSE, MSE plus signal approximation, MOS, signal approximation with MOS, and SDR. To compute the MOS loss function, we utilize the SE loss function from \cite{fu2019learning} which leverages objective-MOS (oMOS) ratings learned from a speech assessment model~\cite{fu2018quality}. SDR~\cite{kawanaka2020stable} loss functions are proposed in literature previously with different enhancement architectures. For the SDR loss function, the SE model is optimized using the following cost function:
\begin{align}
    \mathcal{L}_{SDR} = \sum_{n=1}^N \mathcal{K}_{\theta}  \Big( 10 \log \frac{\Vert s^n\Vert^2}{\Vert s^n-\hat{s}^n\Vert^2} \Big)
\end{align}
where $\mathcal{K}_\theta(a)=\theta\cdot \tanh(\frac{a}{\theta})$, $\theta$ is a clipping parameter, $N$ is the mini-batch size, and $s^n$ and $\hat{s}^n$ are the n\textsuperscript{th} sample of the clean and estimated speech signal in time. We use $\theta=20$ in our training. We also compare against a generative adversarial network (GAN) approach that individually optimizes with PESQ and STOI~\cite{fu2019metricGAN}. We denote this model as MetricGAN. 
% They estimate the IRM conditioned on continuous space of the discriminator label based on either PESQ or STOI target label. 
They estimate the IRM for a speech mixture conditioned on a GAN discriminator that outputs evaluation scores in continuous space (i.e. scores between 0 and 1) based on either normalized PESQ or STOI target metrics. 
We compare our model with the ensemble-based Specialized Speech Enhancement Model Selection (SSEMS) approach~\cite{zezario2019specialized} that uses Quality-Net~\cite{fu2018quality} as its objective function in a black-box manner. Quality-Net is an oMOS approach that estimates the Perceptual Evaluation of Speech Quality (PESQ) score. The SSEMS approach uses an ensemble of enhancement models, each trained on audio at specific SNRs and speaker genders. During inference, it selects the output with the highest PESQ score. SSEMS uses a SNR threshold of $20$ dB, while we use a threshold of $0$ dB for balanced training and better performance. Additionally, we conduct a comparison with our initial approach that integrates MOS embeddings in speech enhancement, as presented in \cite{nayem2021incorporating}. This model is referred to as SE+PMOS, and it does not involve joint training or the QSM language model. We evaluate SE+PMOS with varying combinations of loss functions. %We compare against a quantized speech enhancement model which utilizes a spectral language model~\cite{nayem2021towards}. This model is motivated from chimera++~\cite{wang2018alternative} in structure with BLSTM layers and deep clustering (dc) loss.
%Traditional chimera++ model estimates a phase-sensitive mask which has been applied in the task of speech enhancement in non-speech noisy conditions with multi-talker speech~\cite{wichern2019wham, yang2019improved}. However, in \cite{nayem2021incorporating}, they estimate quantized speech signal, not mask; they use cross-entropy classification (cls) loss, and signal approximation loss altogether. They report best results using per-frequency quantized spectral model (fQSM) as language model for beam search (bS) with beam size $100$. We use this model as our comparison model denoting as Chi++\textsubscript{fQSM,bS}. 
All models are trained using the experimental setup that is previously mentioned. We modify the comparison models using the code provided by the original authors.

We assess speech enhancement performance using PESQ~\cite{rix2001perceptual}, scale-invariant SDR (SI-SDR)~\cite{le2019sdr}, and extended STOI (ESTOI)~\cite{jensen2016algorithm}. In the absence of actual human quality objective, we measure the predicted MOS score of the enhanced speech, using our proposed PMOS model, since we aim to improve human-assessed speech quality. We denote this metric as MOS listener quality objective (MOS-LQO). Table~\ref{tab:results_cosineVoices} shows the average results of the different enhancement models, according to each of the performance metrics on COSINE and VOiCES dataset. As the scores of the unprocessed mixtures show, the VOiCES corpus is  more challenging than the COSINE corpus. 
With the baseline SE model, we experiment with 5 different combination of loss functions. Using the MSE loss only in SE:mse, we see improvements in objective scores, except with MOS-LQO for the COSINE data. Then we apply a MOS loss $\mathcal{L}_{mos}$ as the sole objective criterion, as proposed in \cite{fu2019learning}. Our experimental results show that this approach results in an overall improvement of $1.4$ in MOS-LQO compared to SE:mse. %We apply MOS-LQO scores of enhanced speech to calculate MOS loss $\mathcal{L}_{mos}$ as the only objective criteria as proposed in \cite{fu2019learning}, which gives improves MOS-LQO by $1.4$ overall compared with SE:mse. 
Then we separately combine the signal approximation loss with the mse loss and MOS loss (e.g., mse+sa and mos+sa). In PESQ, we gain an average of $\ge0.05$ and $\ge0.07$ compared to the models that use only the MSE loss and only the MOS loss, respectively. Furthermore, the model trained with the mos+sa loss function achieves the highest MOS-LQO score of $4.4$ and $5.7$ among all five loss functions tested with the SE model in COSINE and VOiCES dataset, respectively. This result is on average $1.15$ MOS-LQO higher than that obtained with the mse+sa loss function. These scores suggest that $\mathcal{L}_{mse}$ and $\mathcal{L}_{sa}$ maximize the overall speech intelligibility, whereas $\mathcal{L}_{mos}$ guides the model towards perceptual speech quality. Note that in all these $\mathcal{L}_{mos}$ calculations, we use a separately trained PMOS model's output without joint learning.
Lastly, we apply the SDR loss function as proposed in \cite{kawanaka2020stable}, which is used as the pre-training stage for model training. We observe an average gain of $0.9$ in SI-SDR, however, it yields a poor score according to other metrics, especially a $0.7$ loss in MOS-LQO compared to SE with mse and sa loss terms. 

SE+PMOS is separately investigated with 3 combinations of loss functions, i.e. mse, mse+sa, and mse+sa+mos. Compared with SE models, SE+PMOS with mse loss achieves $0.9$ SI-SDR and $1.75$ MOS-LQO improvements on average, which shows the benefit of incorporating the PMOS model. The SE+PMOS:mse+sa model improves the performance further with an average of $0.14$ ESTOI gain over the SE:mse+sa model. The inclusion of the mos loss gives the best MOS-LQO scores of $5.1$ and $6.5$ over all the comparison models in noisy and reverberant conditions, respectively.

%%%%%%%%%%%%%%%%%%%%%%%%%%%%%%%%%%%%%%%%%%%%%%%%%%%%
% Table 3, SE comparison test results on CHiME 5+4 
%%%%%%%%%%%%%%%%%%%%%%%%%%%%%%%%%%%%%%%%%%%%%%%%%%%%

% Please add the following required packages to your document preamble:
% \usepackage{multirow}
\begin{table*}[t!]
\centering
\caption{Average testing results of the speech enhancement models on CHiME-5 and CHiME-4 datasets. Best results are shown in \textbf{bold}.}
\label{tab:results_chime}
\resizebox{\linewidth}{!}{%
\begin{tabular}{| l | l | c c c c c | c c c c c |}
\cline{3-12}
\multicolumn{1}{l}{\multirow{2}{*}{}} &                    & \multicolumn{5}{ c |}{{CHiME-5}}                             & \multicolumn{5}{ c |}{{CHiME-4}} 
\\ \hline
\multicolumn{1}{|l|}{models}                          & \multicolumn{1}{c|}{loss func.} & PESQ$\uparrow$          & SI-SDR$\uparrow$       & ESTOI$\uparrow$         & MOS-LQO$\uparrow$      & WER\%$\downarrow$         & PESQ$\uparrow$          & SI-SDR$\uparrow$        & ESTOI$\uparrow$         & MOS-LQO$\uparrow$      & WER\%$\downarrow$         \\ \hline
\multicolumn{1}{|l|}{Mixture}                & -                               & 1.7           & 2.4          & 0.52          & 3.8          & 152.1         & 1.96          & 2.86          & 0.6           & {4.6} & {33.7} \\ \hline
\multicolumn{1}{|l|}{SE}                              & mos+sa                          & {2.25} & {3.9} & {0.62} & {4}   & {96.4} & {2.32} & {5.22} & {0.63} & {5}   & {25.6} \\ \hline
\multicolumn{1}{|l|}{SE+PMOS}                         & mse+sa+mos                      & 2.37          & 6.1          & 0.67          & 4.4          & 84.5          & 2.45          & 7.6           & 0.7           & 5.8          & 22.6          \\ \hline
\multicolumn{1}{|l|}{\multirow{2}{*}{MetricGAN}}      & pesq                            & \textbf{2.44} & {6.3} & {0.65} & {4.1} & {94.8} & \textbf{2.51} & {7}    & {0.68} & {5.3} & {19.7} \\ 
\multicolumn{1}{|l|}{}                                & stoi                            & 2.39          & 6.2          & \textbf{0.71} & 4.1          & 91.3          & 2.45          & {6.45} & \textbf{0.73} & 5.6          & 21.5          \\ \hline
\multicolumn{1}{|l|}{\multirow{2}{*}{Proposed}} & mse+sa                          & 2.41          & 7.1          & {0.68} & 4.7          & \textbf{78.3} & 2.5           & {7.9}  & 0.72          & 5.76         & \textbf{18.1} \\ 
\multicolumn{1}{|l|}{}                                & mse+sa+mos                      & 2.41          & \textbf{7.3} & {0.68} & \textbf{4.9} & 79.4          & {2.5}  & \textbf{8.61} & \textbf{0.73} & \textbf{6}   & 18.9          \\ \hline
\end{tabular}}
\end{table*}
MetricGAN optimizes PESQ or STOI, therefore, it outperforms other comparison models in terms of PESQ and ESTOI, although the scores for the SE+PMOS approaches are higher according to the other evaluation metrics even though these metrics are not leveraged during training. 
SSEMS yields the lowest scores across all metrics compared with SE+PMOS and MetricGAN approaches, though we do parameter tuning for this model.
Chi++\textsubscript{fQSM,bS} estimates quantized speech, and the results show that it affects the traditional objective functions. This performs poorly compared with the SE+PMOS and MetricGAN approaches, however, on average, it outperforms SSEMS in all criteria, and the SE models in terms of PESQ. With the MOS-LQO criteria, it fails to produce good scores. This points out the importance of incorporating perceptual features during enhancement, which Chi++\textsubscript{fQSM,bS} clearly lacks.

We calculate the performance of our proposed model using two combinations of loss functions. 
Using only mse and sa loss terms, we achieve the highest ESTOI scores for both corpora, though these results are nearly identical to the model trained with all three loss terms. Using $\mathcal{L}$ (eq:\ref{eq:loss}) in our proposed model, we obtain the highest SI-SDR scores while maintaining similar PESQ and ESTOI performance as compared to the best-performing model. Specifically, our proposed model achieves the highest ESTOI score and an average PESQ score that is only $0.03$ less than that of the best performing MetricGAN:pesq model.
Contrasting with the Chi++\textsubscript{fQSM,bS} model, which uses spectral language model to estimate quantized speech, our proposed approach outperforms the quantized model according to all metrics, which proves the significance of joint learning.% to direct speech enhancement model towards perceptually better speech using a speech quality assessment model.
When comparing MOS-LQO scores, our proposed:mse+sa+mos model achieves better scores than the other models except the SE+PMOS:mse+sa+mos model with an average of only $0.05$ declination. Thus, the inclusion of a spectral language model helps the model proposed (e.g., mse+sa+mos) to estimate better quality speech according to the overall evaluation criteria. 
It is important to note that our proposed approach performs best according to SI-SDR in both noisy and reverberant environments, where this metric is not used by any of the approaches during optimization.  

We further examine our approaches using completely unseen corpora. We test models with the CHiME-5 and CHiME-4 corpora where the models are trained from the COSINE dataset according to the system setup mentioned in section~\ref{subsec:setup}. Table~\ref{tab:results_chime} shows the performance evaluated according to PESQ, SI-SDR, ESTOI, MOS-LQO, and word error rate (WER). To calculate WER, we use the conventional ASR baseline that is provided with CHiME-5 and CHiME-4 dataset. We investigate WER with both GMM based ASR and end-to-end ASR, however, we find that the end-to-end approach results in a higher error compared to the GMM baseline. This might happen due to larger data requirements of the end-to-end ASR system as mentioned in \cite{barker2018fifth}. Therefore, we use the GMM ASR approach to compare the WER performance of the enhancement models.
From the scores of mixtures, we find that CHiME-5 is more challenging than CHiME-4 with a $118.8\%$ higher WER and a $0.46$ lower SI-SDR. Our proposed approach yields the best MOS-LQO scores with $4.9$ with CHiME-5 and $6$ with CHiME-4 data. The proposed mse+sa model results in the lowest WER of $78.3$ and $18.1$ using CHiME-5 and CHiME-4, respectively. Note that the WER of the GMM baseline ASR for the CHiME-5 challenge is $72.8$ in binaural and $91.7$ in single array conditions. Here our approaches enhance monaural speech, a more challenging condition. Our proposed approach outperforms other comparison models in terms of SI-SDR with a $5.29$ average improvement compared to others. According to PESQ and ESTOI metrics, MetricGAN variants give the best performace, however, proposed model's performance is $0.02$  and $ 0.015$ lower according to PESQ and ESTOI, respectively, for the best performing MetricGAN models. Hence, our proposed approach is effective on out-of-vocabulary scenario trained by a comparable dataset.


% \nayem{*** Possibly add graphs of evaluation metrics vs SNRs.}

%%%%%%%%%%%%%%%%%%%%%%%%%%%%%%%%%%%%%%%%%%%%%%%%%%%%
% Table 3, DNSMOS results
%%%%%%%%%%%%%%%%%%%%%%%%%%%%%%%%%%%%%%%%%%%%%%%%%%%%
% \begin{table}[thb!]

% \centering
% \caption{Average MOS ratings of the speech enhancement modes on CHiME-4 and CHiME-5 datasets using DNSMOS P.835~\cite{reddy2022dnsmos}. Best results are shown in \textbf{bold}.}
% \label{tab:dnsmos_results}
% % \vspace{-0.5em}
% \resizebox{\columnwidth}{!}{%
% \begin{tabular}{| l | c c | }
% \cline{2-3}
%   \multicolumn{1}{c|}{}         & {CHiME-4} & {CHiME-5} \\ \hline
   
% Mixture   & 1.54 ($\pm$0.85)         & 1.3 ($\pm$1.1)                \\
% PMOS+SE                      & 4.28 ($\pm$0.9)       & 3.67 ($\pm$1.3)\\
% MetricGAN                    & 4.26 ($\pm$0.87) & 3.5 ($\pm$1.34)          \\
% Proposed                     & \textbf{4.32} ($\pm$0.8)& \textbf{3.8} ($\pm$1.41)            \\ \hline
% Clean                     & 4.67 ($\pm$1.2) & -      \\
% \hline
% \end{tabular}
% }
% % \vspace{-2em}
% \end{table}

%%%%%%%%%%%%%%%%%%%%%%%%%%%%%%%%%%%%%%%%%%%%%%%%%%%%
% Fig 3, DNSMOS results plot
%%%%%%%%%%%%%%%%%%%%%%%%%%%%%%%%%%%%%%%%%%%%%%%%%%%%

\begin{figure}[b!]
    \centering
\begin{tikzpicture}
	\begin{axis}[
	    cycle list/Dark2-4,
		boxplot/draw direction = y,
		boxplot/box extend=0.8,
% 		x=3em,
% 		x axis line style = {opacity=0.6},
		axis x line* = bottom,
		axis y line = left,
		enlarge y limits,
		ymajorgrids,
		xtick = {1, 2, 3, 4, 5, 6, 7, 8},
		xticklabel style = {align=center, font=\small, rotate=60, alias={xtick-\ticknum}},
		xticklabels = {Mixture, SE+PMOS, MetricGAN, Proposed, Mixture, SE+PMOS, MetricGAN, Proposed},
% 		xtick style = {draw=none}, % Hide tick line
		ylabel = {MOS},
		ytick = {1, 2, 3, 4, 5},
	]
	
	\addplot+[
        boxplot prepared={
        lower whisker=1, lower quartile=1.45,
        median=1.74,
        upper quartile=2.5, upper whisker=4.05, }, fill, draw=black]
        coordinates {}
        node[above, color=black] at
        (boxplot box cs: \boxplotvalue{median},.5)
        {\scriptsize \pgfmathprintnumber{\boxplotvalue{median}}};
    \addplot+[
        boxplot prepared={
        lower whisker=1.38, lower quartile=1.84,
        median=2.28,
        upper quartile=3.1, upper whisker=4.3, }, fill, draw=black]
        coordinates {}
        node[above, color=black] at
        (boxplot box cs: \boxplotvalue{median},.5)
        {\scriptsize \pgfmathprintnumber{\boxplotvalue{median}}};
    \addplot+[
        boxplot prepared={
        lower whisker=1.3, lower quartile=1.75,
        median=2.13,
        upper quartile=3.2, upper whisker=4.1, }, fill, draw=black]
        coordinates {}
        node[above, color=black] at
        (boxplot box cs: \boxplotvalue{median},.5)
        {\scriptsize \pgfmathprintnumber{\boxplotvalue{median}}};
    \addplot+[
        boxplot prepared={
        lower whisker=1.4, lower quartile=1.9,
        median=2.46,
        upper quartile=3.16, upper whisker=4.34, }, fill, draw=black]
        coordinates {}
        node[above, color=black] at
        (boxplot box cs: \boxplotvalue{median},.5)
        {\scriptsize \pgfmathprintnumber{\boxplotvalue{median}}};
        
    \addplot+[
        boxplot prepared={
        lower whisker=1.0, lower quartile=1.35,
        median=1.64,
        upper quartile=2.39, upper whisker=4.18, }, fill, draw=black]
        coordinates {}
        node[above, color=black] at
        (boxplot box cs: \boxplotvalue{median},.5)
        {\scriptsize \pgfmathprintnumber{\boxplotvalue{median}}};
    \addplot+[
        boxplot prepared={
        lower whisker=1.31, lower quartile=1.8,
        median=2.18,
        upper quartile=2.76, upper whisker=4.24, }, fill, draw=black]
        coordinates {}
        node[above, color=black] at
        (boxplot box cs: \boxplotvalue{median},.5)
        {\scriptsize \pgfmathprintnumber{\boxplotvalue{median}}};
    \addplot+[
        boxplot prepared={
        lower whisker=1.26, lower quartile=1.71,
        median=2.06,
        upper quartile=3.17, upper whisker=4.32, }, fill, draw=black]
        coordinates {}
        node[above, color=black] at
        (boxplot box cs: \boxplotvalue{median},.5)
        {\scriptsize \pgfmathprintnumber{\boxplotvalue{median}}};
    \addplot+[
        boxplot prepared={
        lower whisker=1.34, lower quartile=1.85,
        median=2.25,
        upper quartile=3.07, upper whisker=4.48, }, fill, draw=black]
        coordinates {}
        node[above, color=black] at
        (boxplot box cs: \boxplotvalue{median},.5)
        {\scriptsize \pgfmathprintnumber{\boxplotvalue{median}}};
        
	\end{axis}
	
	\path (0,0) coordinate (P);
    \draw [thick,decoration={brace,mirror,raise=5em},decorate] (xtick-0|-P) -- (xtick-3.5|-P) 
        node[midway,yshift=-6em]{CHiME-4};
    \draw [thick,decoration={brace,mirror,raise=5em},decorate] (xtick-4|-P) -- (xtick-7.5|-P) 
        node[midway,yshift=-6em]{CHiME-5};

    % \node[text width=3cm] at (1.54,0.5) 
    % {\scriptsize 1.54};

\end{tikzpicture}

\caption{MOS ratings of the speech enhancement modes on CHiME-4 and CHiME-5 datasets using DNSMOS P.835.}
    % \vspace{-2em}
\label{fig:dnsmos_results}
    % \vspace{-0.4cm}
\end{figure}

\subsection{Perceptual quality evaluation}
\label{subsec:dnsmos}

We finally evaluate our model using P.835 metric~\cite{reddy2022dnsmos} to measure perceptual quality. We calculate the DNSMOS score on a scale of $[1-5]$ ($1$ = worst, $5$ = best) for the mixture, PMOS+SE, MetricGAN, and our proposed models using the CHiME-4~\cite{vincent2017analysis} and CHiME-5~\cite{barker2018fifth} datasets (simulated and real-recording). Figure~\ref{fig:dnsmos_results} shows the scores. With CHiME-4, the original mixture scores range from $1.45$ to $2.5$ with a median of $1.74$. Our proposed model achieves a median MOS of $2.46$, which is higher than the others. Fon CHiME-5, the original mixture scores range from $1.0$ to $4.18$. Our proposed model outperforms the others with a median of $2.25$. Our proposed model and PMOS+SE have smaller standard deviations compared to MetricGAN. Overall, our proposed model improves noisy speech in both the acoustic and perceptual aspects. 




% \subsection{Listening results}
% \label{subsec:listening_results}

% We conduct an IRB-approved listening study using Amazon Mechanical Turk to conceive the perceptual quality of enhanced speech assessed by normal-hearing listeners. 

% This study follows the design structure of \cite{nayem2021towards} and figure~\ref{fig:survey} shows the actual listener study interface of a single question. The study is conducted as follows, the participant will listen to two audio signals, one is enhanced and the other is clean audio as reference.  Then they provide a preference score using a Likert scale. The scale ranges from $-3$ to $+3$, where $-3$ refers to a strong preference towards the first signal, $+3$ refers to a strong preference towards the second signal, and $0$ refers to no preference. Before providing a score, the participant can listen to the signals as many as times they like, where the scores are not limited to integer values. The two signals are randomly selected, and the participant listens to different audio clips in each question. The audio clips are chosen from the CHiME-5 and CHiME-4 corpus spoken by both males and females in equal proportion. Prior to actual survey questions, each participants has to pass eligibility test and make themselves familiar with the upcoming study session by going through a practice session. The structure of this practice session is similar to the actual study, however, speakers' voice and audio clips which participants hear in practice session are not used in the actual study. A tentative feedback is provided in the practice session to give a guideline to the participants, however, to avoid biases and leading answers, the feedback is provided in a form of range where the expected answer should reside.



%  \begin{figure}[thb!]
%     \centering
%     \includegraphics[width = 0.5\linewidth]{IEEEtran/figs/survey.png}
%     % \vspace{-2em}
%     \caption{A question of actual listener study interface conducted on MTurk.}
%     \label{fig:survey}
%     % \vspace{-2em}
% \end{figure}

% \nayem{***One paragraph on the statistics of the conducted study.}
% The study session contains total 30 questions, which is preceded by a practice session of 7 questions. Ten participants (9 male, 1 female) who are native English speakers over the age of 18 participated, where a headset/headphone was required to be worn. On average, participants took 14 minutes to complete the study, they were given $\$3$ monetary incentive.


\section{Discussion}
\label{sec:discuss}

Our proposed model outperforms all comparison models on SI-SDR metrics for both seen and unseen datasets, without optimization of any of the models (Table \ref{tab:results_cosineVoices}, \ref{tab:results_chime}). This means that our approach improves speech quality by minimizing the distortion ratio when separated from the noise component. Additionally, our models yield the best MOS-LQO ratings on real-world captured audios (CHiME datasets, Table \ref{tab:results_chime}). These results are consistent with the findings of \cite{zezario2022deep, nayem2021incorporating} that incorporating embeddings from a speech assessment model improves SE performance, and the results of \cite{braun2022effect} that using MOS loss during model optimization leads to higher MOS-LQO scores. Our proposed approach achieves PESQ and ESTOI scores that are only slightly lower than those of the best-performing model, with a difference of only $0.03$ and $0.01$, respectively. This indicates that speech quality and intelligibility metrics are closely related to the subjective speech quality metric (MOS-LQO), and that these metrics can be improved without explicit optimization. Furthermore, our proposed model achieves the best average DNSMOS scores with low standard deviations on CHiME datasets (Figure \ref{fig:dnsmos_results}), indicating that it is effective in a wide range of real-world noise levels. This is a desirable quality for an effective SE model to be effective not only in high SNRs and limited noisy environments, but also in large SNR ranges and real-world conditions such as those offered by the CHiME dataset.

When comparing our proposed model that uses mse+sa+mos loss to the PMOS+SE model (as shown in Table \ref{tab:results_chime}), we can observe significant improvements in all performance metrics. As both models use the same loss function, the improvements are attributed to the incorporation of LM and the joint learning method. Moreover, we found that these two models exhibit similar performance on the MOS prediction (Table \ref{tab:mos_results}), indicating that the benefits of joint learning mostly impact the enhancement part of the model.

An intriguing finding is that our proposed model shows a decline in WER\% when MOS loss is incorporated, especially for larger real-world recordings such as CHiME-5, with degradation up to $1.1$. Although our study is not primarily concerned with ASR performance, this suggests a potential trade-off between ASR accuracy and subjective speech quality scores. Further investigation is needed to comprehend this relationship.

Our proposed method demonstrates that training a speech enhancement (SE) model and a MOS-based speech assessment model jointly can lead to better speech quality measured by objective metrics such as perceptual quality, intelligibility, and MOS ratings. However, we acknowledge that our study's use of subjective MOS (sMOS) estimation instead of actual human listeners may introduce discrepancies between MOS-LQO and human-rated MOS, which could impact our findings. To address this limitation, we plan to conduct sMOS evaluation by human listeners in future work. Although we used the same MOS prediction model for all comparison models, we believe that incorporating human-rated sMOS evaluations will provide more robust insights into our proposed method's effectiveness.
For computing loss terms, we opt for the MSE loss function along with a bi-gram language model that considers only time-along transitions. Our aim is to keep the model simple and focus on the effectiveness of our approach. However, we acknowledge that using different loss functions for different loss components and employing a more complex language model that considers both temporal and spectral transition levels can be beneficial. We plan to explore these possibilities in our future work.




%%%%%%%%%%%%%%%%%%%%%%%%%%%%%%%%%%%%%%%%%%%%%%%%%%%%%%%%%%%%%%%%%%%%%%%%%%%%%%%%%
\section{Discussion} \label{sec:5_discussion}
% !TEX root = root.tex

\section{Conclusions and Future Work}
\label{sec:5_discussion}
Our work unifies the SF-GPI and value composition to the continuous concurrent composition framework and allows reconstructing task policy from a set of primitives. The proposed method was extended to composition at the action component level. We demonstrate in the Pointmass environment that our multi-task agents can reconstruct the task policy from a set of primitives in real time and transfer the skills to solve unseen tasks while the single-task performance is competitive with SAC.
This flexible framework incorporates well with the reward-shaping techniques, such as entropy regularization, curiosity\cite{pmlr-v70-pathak17a}, etc. In addition, the task-agnostic property should benefit the autotelic framework \cite{colas2022autotelic} where agents can set goals and curriculum for themselves \cite{narvekar2020curriculum}. 

However, the primary concern at this stage is whether the proposed approach can scale to higher dimensional problems. Additionally, two important topics are left as future works. First, look for the corresponding value composition for DAC. A good starting point might be thinking of the MSF composition with weights evaluated by GPE. 
Second, the optimality of each composition method. One might start with bounding the loss incurred by the policy and value composition. 

%%%%%%%%%%%%%%%%%%%%%%%%%%%%%%%%%%%%%%%%%%%%%%%%%%%%%%%%%%%%%%%%%%%%%%%%%%%%%%%%%
\section*{Acknowledgments}

We gratefully acknowledge the support of the NSF Geophysics Program (EAR awards \#1853196 and \#2143939) and S.H. also thanks the EPSRC (grant \#EP/V047388/1). Thanks also go to Jewel Abbate and Taylor Lonner for providing assistance with the experimental setup of RoMag, and to Meredith Plumley for generously sharing her extracted $Nu$-$Ra$ data from King \& Aurnou (2015) and Burr \& Muller (2011). 

%%%%%%%%%%%%%%%%%%%%%%%%%%%%%%%%%%%%%%%%%%%%%%%%%%%%%%%%%%%%%%%%%%%%%%%%%%%%%%%%%
% References

\bibliography{bib}

\clearpage
%%%%%%%%%%%%%%%%%%%%%%%%%%%%%%%%%%%%%%%%%%%%%%%%%%%%%%%%%%%%%%%%%%%%%%%%%%%%%%%%%
\section{Appendix} \label{sec:appendix}
\section{Appendix for Proofs}

\paragraph{Proof of Theorem \ref{thm:main}.}

\begin{proof}
\label{proof:main}
Our proof has two steps. In Step 1, we will show that SimCLR is equivalent to minimizing the cross entropy loss defined in Eqn.~(\ref{eqn:cross-entropy}). 
In Step 2, we will show  that minimizing the cross-entropy loss 
is equivalent to spectral clustering on $\bfpi$. 
Combining the two steps together, we have proved our theorem. 

\textbf{Step 1: } SimCLR is equivalent to minimizing the cross entropy loss.

The cross-entropy loss takes expectation over 
$\bfW_\bfX\sim \mathbb{P}(\cdot ; \bfpi)$, 
which means $\bfW_\bfX$ has exactly one non-zero entry in each row $i$. By Lemma~\ref{lem:multinomial}, we know every row $i$ of $\bfW_\bfX$ is independent of other rows. Moreover, 
$\bfW_{\bfX,i}\sim \mathcal{M}(1, \bfpi_i/\sum_j \bfpi_{i,j})=\mathcal{M}(1, \bfpi_i)$, because $\bfpi_i$ itself is a probability distribution.
Similarly, we know $\bfW_\bfZ$ also has the row-independent property by sampling over $\mathbb{P}(\cdot;\bfK_\bfZ)$.
Therefore, by Lemma~\ref{lem:cross_split}, we know Eqn.~(\ref{eqn:cross-entropy}) is equivalent to:
\[
 -\sum_{i=1}^n \mathbb{E}_{\bfW_{\bfX,i}}[\log \mathbb{P}(\bfW_{\bfZ,i}=\bfW_{\bfX,i};\bfK_\bfZ)],
\]

This expression takes expectation over $\bfW_{\bfX,i}$ for the given row $i$. Notice that 
$\bfW_{\bfX,i}$ has exactly one non-zero entry, which equals $1$ (same for $\bfW_{\bfZ,i}$). 
As a result
we expand the above expression to be:
\begin{equation}
 -\sum_{i=1}^n \sum_{j\neq i} \Pr(\bfW_{\bfX,i,j}=1)\log \Pr(\bfW_{\bfZ,i,j}=1).
\label{eqn:detailed-expansion}    
\end{equation}


By Lemma~\ref{lem:multinomial}, $\Pr(\bfW_{\bfZ,i,j}=1)=\bfK_{\bfZ,i,j}/\|\bfK_{\bfZ,i}\|_1$ for $j\neq i$. Recall that $\bfK_\bfZ=(k(\bfZ_i-\bfZ_j))_{(i,j)\in[n]^2}$, which means 
$\bfK_{\bfZ,i,j}/\|\bfK_{\bfZ,i}\|_1=\frac{\exp(-\|\bfZ_i-\bfZ_j\|^2/{2\tau})}{\sum_{k\neq i}
\exp(-\|\bfZ_i-\bfZ_k\|^2/{2\tau})
}$ for $j\neq i$, when $k$ is the Gaussian kernel with variance $\tau$. 

Notice that $\bfZ_i=f(\bfX_i)$, so we know
\begin{equation}
-\log \Pr(\bfW_{\bfZ,i,j}=1)=
-\log \frac{\exp(-\|f(\bfX_i)-f(\bfX_j)\|^2/{2\tau})}{\sum_{k\neq i}
\exp(-\|f(\bfX_i)-f(\bfX_k)\|^2/{2\tau}),
}
\label{eqn:infonce-equivalence}    
\end{equation}


The right hand side is exactly the InfoNCE loss defined in Eqn.~(\ref{eqn:infonce}).
Inserting Eqn.~(\ref{eqn:infonce-equivalence}) into Eqn.~(\ref{eqn:detailed-expansion}), we get the SimCLR algorithm, which first samples augmentation pairs $(i,j)$ with $\Pr(\bfW_{\bfX,i,j}=1)$ for each row $i$, and then optimize the InfoNCE loss. 

\textbf{Step 2: } minimizing the cross entropy loss 
is equivalent to spectral clustering on $\bfpi$.


By Lemma~\ref{lem:convert_to_spectral}, we may further convert the loss to 
\begin{equation}
\label{eqn:main-theorem-repul-attr}
\min_{\bfZ}
-\sum_{(i,j)\in [n]^2} \mathbf{P}_{i,j}
\log k (\bfZ_i-\bfZ_j)+\log \mathbf{R}(\bfZ).
\end{equation}
Since $k$ is the Gaussian kernel, this reduces to \[
\min_\bfZ \mathrm{tr}(\bfZ^\top \mathbf{L}(\bfpi) \bfZ)
+\log \mathbf{R}(\bfZ),
\]

where we use the fact that $\mathbb{E}_{\bfW_\bfX\sim \mathbb{P}(\cdot; \bfpi)}[\mathbf{L}(\bfW_\bfX)]
=\mathbf{L}(\bfpi)
$, because the Laplacian operator is linear and $
\mathbb{E}_{\bfW_\bfX\sim \mathbb{P}(\cdot; \bfpi)}(\bfW_\bfX)=\bfpi
$.
\end{proof}

\paragraph{Proof of Theorem \ref{thm:clip}.}
\begin{proof}
Since $\bfW_\bfX\sim \mathbb{P}(\cdot;\bfpi_{\mathbf{A}, \mathbf{B}})$, we know 
$\bfW_\bfX$ has exactly one non-zero entry in each row, denoting the pair that got sampled. 
A notable difference compared to the previous proof is we now have $n_\mathcal{A}+n_\mathcal{B}$ objects in our graph. CLIP deals with this by taking a mini-batch of size $2N$, 
such that $n_\mathcal{A}=n_\mathcal{B}=N$, and adding the $2N$ InfoNCE losses together. We label the objects in $\mathcal{A}$ as $[n_\mathcal{A}]$, and the objects in $\mathcal{B}$ as $\{n_\mathcal{A}+1, \cdots, n_\mathcal{A}+n_\mathcal{B}\}$. 

Notice that $\bfpi_{\mathbf{A}, \mathbf{B}}$ is a bipartite graph, so the edges of objects in $\mathcal{A}$ will only connect to object in $\mathcal{B}$ and vice versa. We can define the similarity matrix in $\cZ$ as $\bfK_\bfZ$, 
where $\bfK_\bfZ(i, j+n_\mathcal{A})=\bfK_\bfZ(j+n_\mathcal{A},i)= k(\bfZ_i-\bfZ_j)$ for $i\in [n_\mathcal{A}], j\in [n_\mathcal{B}]$, and otherwise we set $\bfK_\bfZ(i,j)=0$. 
The rest is same as the previous proof. 
\end{proof}

\paragraph{Proof of Theorem \ref{thm:exponential}.}

\begin{proof}
\label{proof:exponential}
Since the objective function consists of a linear term combined with an entropy regularization, which is a strongly concave function, the maximization problem is a convex optimization problem. Owing to the implicit constraints provided by the entropy function, the problem is equivalent to having only the equality constraint. We then introduce the Lagrangian multiplier $\lambda$ and obtain the following relaxed problem:

$$
\widetilde{E}(\boldsymbol{\alpha})=\psi_{1}-\sum_{i=1}^n \alpha_{i} \psi_{i}+\tau \sum_{i=1}^n \alpha_{i}\log \alpha_{i}+\lambda\left(\boldsymbol{\alpha}^{\top} \mathbf{1}_n-1\right).
$$

As the relaxed problem is unconstrained, taking the derivative with respect to $\alpha_{i}$ yields

$$
\frac{\partial \widetilde{E}(\boldsymbol{\alpha})}{\partial \alpha_{i}}=-\psi_{i}+\tau\left(\log \alpha_{i}+\alpha_{i} \frac{1}{\alpha_{i}}\right)+\lambda=0.
$$

Solving the above equation implies that $\alpha_{i}$ takes the form
$
\alpha_{i}=\exp \left(\frac{1}{\tau} \psi_{i}\right) \exp \left(\frac{-\lambda}{\tau}-1\right).
$ Since $\alpha_{i}$ lies on the probability simplex, the optimal $\alpha_{i}$ is explicitly given by
$
\alpha^{*}_{i}=\frac{\exp \left(\frac{1}{\tau} \psi_{i}\right)}{\sum_{i^{\prime}=1}^n \exp \left(\frac{1}{\tau} \psi_{i^{\prime}}\right)} .
$ Substituting the optimal point into the objective function, we obtain
$$
\begin{aligned}
E\left(\boldsymbol{\alpha}^*\right)  &=\psi_1-\sum_{i=1}^n \frac{\exp \left(\frac{1}{\tau} \psi_{i}\right)}{\sum_{i^{\prime}=1}^n \exp \left(\frac{1}{\tau} \psi_{i^{\prime}}\right)} \psi_{i}+\tau \sum_{i=1}^n \frac{\exp \left(\frac{1}{\tau} \psi_{i}\right)}{\sum_{i^{\prime}=1}^n \exp \left(\frac{1}{\tau} \psi_{i^{\prime}}\right)}\log \frac{\exp \left(\frac{1}{\tau} \psi_{i}\right)}{\sum_{i^{\prime}=1}^n \exp \left(\frac{1}{\tau} \psi_{i^{\prime}}\right)} \\
& =\psi_1 - \tau \log \left(\sum_{i=1}^n \exp \left(\frac{1}{\tau} \psi_{i}\right)\right).
\end{aligned}
$$
Thus, the Lagrangian dual function is given by
\begin{equation*}
-E\left(\boldsymbol{\alpha}^*\right)= -\tau \log \frac{\exp \left(\frac{1}{\tau} \psi_{1}\right)}{\sum_{i=1}^n \exp \left(\frac{1}{\tau} \psi_{i}\right)}.\qedhere
\end{equation*}
\end{proof}



\section{More on Experiments} \label{section: experiment_details}

\paragraph{CIFAR-10 and CIFAR-100} CIFAR-10 ~\citep{krizhevsky2009learning} and CIFAR-100 ~\citep{krizhevsky2009learning} are well-known classic image classification datasets. Both CIFAR-10 and CIFAR-100 contain a total of 60k $32 \times 32$ labeled images of different classes, with 50k for training and 10k for testing. CIFAR-10 is similar to CIFAR-100, except there are 10 different classes in CIFAR-10 and 100 classes in CIFAR-100.

\paragraph{TinyImageNet} TinyImageNet ~\citep{le2015tiny} is a subset of ImageNet ~\citep{deng2009imagenet}. There are 200 different object classes in TinyImageNet, with 500 training images, 50 validation images, and 50 test images for each class. All the images in TinyImageNet are colored and labeled with a size of $64 \times 64$.

\textbf{Pseudo-code.} Algorithm \ref{alg:Training Procedure} presents the pseudo-code for our empirical training procedure.

\begin{algorithm}[!htbp]
\caption{Training Procedure}
\label{alg:Training Procedure}
\begin{algorithmic}[1]
\REQUIRE trainable encoder network $f$, batch size $N$, augmentation strategy \textit{aug}, loss function $L$ with hyperparameters \textit{args}
\FOR {sampled minibatch ${x_i}_{i=1}^N$}
\FORALL{$i \in { 1, ..., N }$}
\STATE draw two augmentations $t_i = \textit{aug}\left(x_i\right) $, $t_i' = \textit{aug}\left(x_i\right) $
\STATE $z_i = f\left(t_i\right)$, $z_i' = f\left(t_i'\right)$
\ENDFOR
\STATE compute loss $\mathcal{L} = L(N, z, z', \textit{args})$
\STATE update encoder network $f$ to minimize $\mathcal{L}$
\ENDFOR
\STATE \textbf{Return} encoder network $f$
\end{algorithmic}
\end{algorithm}

We also provide the pseudo-code for our core loss function used in the training procedure in Algorithm \ref{alg:Core loss}. The pseudo-code is almost identical to SimCLR's loss function, with the exception of an extra parameter $\gamma$.

\begin{algorithm}[!htbp]
\caption{Core loss function $\mathcal{C}$}
\label{alg:Core loss}
\begin{algorithmic}[1]
\REQUIRE batch size $N$, two encoded minibatches $z_1, z_2$, $\gamma$, temperature $\tau$
\STATE $z = \textit{concat}\left(z_1, z_2\right)$
\FOR {$i \in {1, ..., 2N }, j \in {1, ..., 2N}$ }
\STATE $s_{i,j} = \Vert z_i - z_j \Vert_2^{\gamma}$
\ENDFOR
\STATE \textbf{define} $l(i, j)$ \textbf{as} $l(i, j) = - \log \frac{exp\left(s_{i,j}/\tau \right)}{\sum_{k=1}^{2N} \mathbf{1}{[k \ne i]} exp\left(s{i, j} / \tau \right)} $
\STATE \textbf{Return} $\frac{1}{2N} \sum_{k=1}^N\left[l(i, i+N) + l(i+N, i)\right]$
\end{algorithmic}
\end{algorithm}

Utilizing the core loss function $\mathcal{C}$, we can define all kernel loss functions used in our experiments in Table \ref{table: loss definition}. For all $z_i \in z$ with even dimensions $n$, we define $z_{L_i} = z_i\left[0:n/2\right]$ and $z_{R_i} = z_i\left[n/2:n\right]$.

\begin{table}[ht]
\centering
\begin{tabular}{{@{}l|l@{}}}
Kernel  &  Loss function \\ \midrule
Laplacian & $\mathcal{C}\left(N, z, z', \gamma=1, \tau\right)$\\ \midrule
Sum       & $\lambda * \mathcal{C}\left(N, z, z', \gamma=1, \tau_1\right) + (1-\lambda) * \mathcal{C}\left(N, z, z', \gamma=2, \tau_2\right)$  \\ \midrule
Concatenation Sum&$\lambda * \mathcal{C}\left(N, z_L, z'_L, \gamma=1, \tau_1\right) + (1-\lambda) * \mathcal{C}\left(N, z_R, z'_R, \gamma=2, \tau_2\right)$\\ \midrule
$\gamma = 0.5$ & $\mathcal{C}\left(N, z, z', \gamma=0.5, \tau\right)$          \\ 

\end{tabular}

\caption{Definition of kernel loss functions in our experiments}
\label {table: loss definition}
\end{table}

\textbf{Baselines.} We reproduce the SimCLR algorithm using PyTorch Lightning~\citep{PytorchLightning}.

\textbf{Encoder details.}
The encoder $f$ consists of a backbone network and a projection network. We employ ResNet50~\citep{ResNet} as the backbone and a 2-layer MLP (connected by a batch normalization~\citep{ioffe2015batch} layer and a ReLU \cite{nair2010rectified} layer) with hidden dimensions 2048 and output dimensions 128 (or 256 in the concatenation kernel case).

\textbf{Encoder hyperparameter tuning.}
For each encoder training case, we randomly sample 500 hyperparameter groups (sample details are shown in Table \ref{table: Hyperparameter sample}) and train these samples simultaneously using Ray Tune ~\citep{RayTune}, with the ASHA scheduler~\citep{li2018massively}. Ultimately, the hyperparameter group that maximizes the online validation accuracy (integrated in PyTorch Lightning) within 5000 validation steps is chosen for the given encoder training case.

\begin{table}[ht]
\centering

\begin{tabular}{@{}l|l|l@{}}
\midrule
Hyperparameter  & Sample Range & Sample Strategy \\ \midrule
start learning rate & $\left[10^{-2}, 10\right]$ & log uniform \\ \midrule
$\lambda$       & $\left[0, 1\right]$ & uniform \\ \midrule
$\tau$, $\tau_1$, $\tau_2$ & $\left[0, 1\right]$ & log uniform \\ \midrule
\end{tabular}

\caption{Hyperparameters sample strategy}
\label {table: Hyperparameter sample}
\end{table}

\textbf{Encoder training.} 
We train each encoder using the LARS optimizer~\citep{LARSOptimizer}, LambdaLR Scheduler in PyTorch, momentum 0.9, weight decay $10^{-6}$, batch size 256, and the aforementioned hyperparameters for 400 epochs on a single A-100 GPU.

\textbf{Image transformation.} The image transformation strategy, including augmentation, is identical to the default transformation strategy provided by PyTorch Lightning.

\textbf{Linear evaluation.}
The linear head is trained using the SGD optimizer with a cosine learning rate scheduler, batch size 64, and weight decay $10^{-6}$ for 100 epochs. The learning rate starts at $0.3$ and ends at $0$.

\textbf{Moco Experiments.} We also tested our method based on MoCo~\citep{he2019moco}. The results are summarized in Table \ref{tab:results-moco}. Here we choose ResNet18~\citep{ResNet} as the backbone and set a temperature of $0.1$ as default. For our simple sum kernel, we set $\lambda=0.8$. The results show that our method outperforms the original MoCo method.

\begin{table}[thb]
\centering
\caption{MoCo Experiment Results on CIFAR-10 and CIFAR-100.}
\label{tab:results-moco}
\resizebox{\textwidth}{!}{%
\begin{tabular}{@{}c|ccc|ccc@{}}
\toprule
\multirow{3}{*}{Method} & \multicolumn{3}{c|}{CIFAR-10} & \multicolumn{3}{c}{CIFAR-100} \\ \cmidrule(lr){2-4} \cmidrule(lr){5-7} 
                        & 200 epochs & 400 epochs    & 1000 epochs   & 200 epochs & 400 epochs & 1000 epochs         \\ \midrule
MoCo (repro.)         & $76.41 \pm 0.12$    & $80.01 \pm 0.15$          & $84.45 \pm 0.08$    & $\mathbf{47.02 \pm 0.11}$ & $52.50 \pm 0.07$ & $57.62 \pm 0.15$            \\
\midrule
Laplacian Kernel        & ${78.09 \pm 0.10}$    & $\mathbf{83.85 \pm 0.09}$          & $\mathbf{88.34 \pm 0.16}$    & $46.12 \pm 0.22$   & $53.44 \pm 0.17$ & $59.10 \pm 0.14$        \\
Simple Sum Kernel & $\mathbf{78.12 \pm 0.15}$   & $83.23 \pm 0.18$ & $87.50 \pm 0.20$ & $46.65 \pm 0.06$ & $\mathbf{53.62 \pm 0.19}$ & $\mathbf{59.83 \pm 0.12}$\\
\bottomrule
\end{tabular}
}
\end{table}



\section{More Experiments on Synthetic Data}


Consider a scenario with $n$ clusters, each containing $k$ vertices. Let the probability of vertices $u$ and $v$ from the same cluster belonging to $\bfpi$ be $p$. Conversely, for vertices $u$ and $v$ from different clusters, let the probability of belonging to $\pi$ be $q$. We generate the graph $\bfpi$ randomly, based on $p$ and $q$. We experiment with values of $k=100$ and $n=6$ for ease of visualization, embedding all points in a two-dimensional space. Each vertex's initial position originates from a normal distribution. In each iteration, we sample a subgraph of $\bfpi$ uniformly, ensuring each vertex has an out-degree of $1$. We then optimize the corresponding vectors using InfoNCE loss with an SGD optimizer and iterate until convergence. Our experimental setup consists of an SGD learning rate of $1$, an InfoNCE loss temperature of $0.5$, and a batch size of $50$. We evaluate two scenarios with different $p$ and $q$ values: $p=1$, $q=0$, and $p=0.75$, $q=0.2$. The results of these experiments are visualized in Figure \ref{fig:vis-spectral-cluster}. The obtained embeddings exhibit the hallmark pattern of spectral clustering of graph $\bfpi$.

\begin{figure}[!tb]
\centering
\subfigure{
\includegraphics[width=1\textwidth]{Figures/cluster_pi.png}
\label{fig:vis-cluster}
}
\subfigure{
\includegraphics[width=1\textwidth]{Figures/noised_cluster_pi.png}
\label{fig:vis-noised-cluster}
}
\caption{Visualizations of the optimization process using InfoNCE Loss on the vectors corresponding to $\bfpi$. Points of identical color belong to the same cluster within $\bfpi$. To showcase the internal structure of $\bfpi$, we randomly select 10 vertices from each cluster to display the edge distribution of $\bfpi$.}
\label{fig:vis-spectral-cluster}
\end{figure}



%%%%%%%%%%%%%%%%%%%%%%%%%%%%%%%%%%%%%%%%%%%%%%%%%%%%%%%%%%%%%%%%%%%%%%%%%%%%%%%%%
\end{document}
%
% ****** End of file apssamp.tex ******
