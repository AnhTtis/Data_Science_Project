\subsection{Simplifying assumptions\label{sec:assumptions}}

\noindent In Sec. \ref{sec:Pes}, we will analytically calculate  the back-to-back performance of $M$-ary MVM over $N$ spatial and polarization degrees of freedom in the amplified spontaneous emission (ASE) noise-limited regime. For mathematical tractability, we neglect all transmission impairments other than ASE noise and random carrier phase shifts, as well as transceiver imperfections and implementation penalties. These simplifying assumptions are justified in the sense that we want to  quantify the ultimate potential of MVM for use in optical interconnects. 

Nevertheless, it is worth discussing upfront about the anticipated impact of the most prominent transmission effects. 

In general, the extension of SVM to MVM requires similar conditions for transmission, i.e., negligible chromatic dispersion (CD), modal dispersion (MD), and mode-dependent loss (MDL), or their full compensation, either in the optical or the electronic domain, before making decisions on the received symbols at the receiver. Let us briefly contemplate how feasible it would be to satisfy these requirements in the case of practical homogeneous  multicore fibers with single-mode cores. 

As a starting point, consider transmission over homogeneous MCFs with uncoupled or weekly-coupled single-mode cores. These fibers typically exhibit static and dynamic intercore skew \cite{Puttnam2014}. The static differential mode group delay (DMGD) spread is on the order of 0.5 ns/km and grows linearly with the transmission distance. The DMGD spread due to the dynamic component of the intercore skew is of the order of 0.5 ps/km and also grows linearly with the transmission distance. 

On the other hand, coupled-core MCFs exhibit modal dispersion and strong coupling among their supermodes and the DMGD grows with the square root of the transmission distance \cite{Hayashi:17, Yoshida:19}. From  published values based on the characterization of several  coupled-core MCFs used in MDM experiments, we conclude that the MD coefficient is currently on the order of 3-6 ps/$\sqrt{\mathrm{km}}$. These values are much higher than typical PMD coefficient values for SMFs, e.g., from the data sheet of Corning\textregistered \, {SMF-28 Ultra}\textregistered \,  optical fiber \cite{SMF28u_datasheet}, we notice that the PMD coefficient  is less than 0.1 ps/$\sqrt{\mathrm{km}}$.

Transmission impairments can be compensated using a combination of optical and electronic techniques at the transmitter and the receiver. These techniques are out of the scope of this paper, since we are interested in the back-to-back performance of MVM systems, and will be part of future work. For simplicity, in the depiction of the optically-preamplified MVM direct-detection receiver in Fig.~\ref{TxRx:sub2}, we assume ideal optical post-compensation of all transmission impairments.