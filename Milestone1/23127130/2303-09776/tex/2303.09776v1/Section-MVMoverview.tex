\subsection{MVM signal representation}
\noindent  MVM  can be used together with multimode  and multicore fibers, as well as for free-space transmission. In this section, for the description of the operation of the  MVM transceiver, without loss of generality, we examine the special case of MVM transmission over an ideal homogeneous  multicore fiber with identical single-mode cores and negligible differential  group delay among cores.

We assume that we select a subset of $K$  single-mode cores of the multicore fiber (Fig.~\ref{fig:MVM_MCFs}).
MVM modulation consists in sending optical pulses  over all these cores simultaneously with the same shape but different amplitudes and initial phases (Fig.~\ref{fig:intensityPlot_M64_N8.png}). Similar to SVM over single-mode fibers, wherein the optical wave can be analyzed in two orthogonal states of polarization, e.g., $x$ and $y$, the composite optical wave of MVM over a homogeneous  single-mode-core multicore fiber can be described by $N=2K$ orthogonal states of polarization, e.g., $x$ and $y$ in each core.


\begin{figure}[!htb]
	\centering
	\includegraphics[width=3in]{Figs/LaunchedWavesMCF.png}
	\caption{MVM over homogeneous MCFs with single-mode cores.}
	\label{fig:MVM_MCFs}
\end{figure}

\begin{figure}[!htb]\centering
\captionsetup[subfigure]{justification=centering}
    \begin{subfigure}[b]{.45\columnwidth}
    \centering
	    \includegraphics[width=1.5in]{Figs/intensityPlot_M64_N8.png}
	    \caption{}
	\end{subfigure}
	\begin{subfigure}[b]{.45\columnwidth}
	    \centering
	    \includegraphics[width=1.5in]{Figs/Polarization_Ellipses_8_64_MVM.png}
	    \caption{}
	\end{subfigure}
	\caption{(a)  Intensity plot and (b) Polarization ellipses  of an MVM signal propagating over an ideal four-core MCF with identical uncoupled single-mode cores.} 
	\label{fig:intensityPlot_M64_N8.png}
\end{figure}

The mathematical representation of the MVM signals at the  fiber input  is written as
\begin{equation} 
	{\bf E}_{m} (t)=A_{m} \exp\left( \iota \phi _{m} \right) g(t) \ket{ s_m},
	\label{eq:MVMsigs} 
\end{equation}
where $m=1,\ldots,M$, $A_m$ and $\phi_{m}$ denote the common amplitude  and phase, respectively, $g(t)$ is a real function describing the pulse shape, and $\ket{ s_m } $ is a generalized unit Jones vector collecting the complex excitations of the cores, i.e., the amplitudes and phases of electric fields of the optical waves \cite{Antonelli:12,Roudas_PJ_17,Roudas_JLT_18}. 
    \nomenclature[$emt$]{$\mathbf{E}_m (t)$}{Electric field at fiber input}
    \nomenclature[$eg$]{$\mathcal{E}_g$}{Pulse energy}
	\nomenclature[$am$]{$A_m$}{Common signal amplitude}
	\nomenclature[$phim$]{$\phi_m$}{Common signal phase}
	\nomenclature[$gt$]{$g(t)$}{Pulse shape}
	\nomenclature[$sket$]{$\ket{s}$}{Generalized unit Jones vector} 

The signal energy $\mathcal{E}_s$ is given by \cite{Proakis} \nomenclature[$es$]{$\mathcal{E}_s$}{Signal energy}
\begin{align}
\mathcal{E}_s :=& \frac{1}{2}\int_{-\infty}^\infty {\bf E}_{m} (t)^\dagger {\bf E}_{m} (t) dt 
= \frac{A_m^2}{2}\int_{-\infty}^\infty g(t)^2dt= \frac{A_m^2}{2}\mathcal{E}_g , 
    \label{eq:sigEnergy}
\end{align}
where $\dagger$ denotes the adjoint (i.e., conjugate transpose) of a matrix and  $\mathcal{E}_g$ denotes the pulse energy defined as
\begin{equation}
\mathcal{E}_g := \int_{-\infty}^\infty g(t)^2dt.
    \label{eq:pulseEnergy} 
\end{equation}





In the remainder of the article, without loss of generality,  we consider that the common amplitude $A_m$ and phase $\phi_{m}$ in \eqref{eq:MVMsigs} are constant. In other words, we focus exclusively on a special case of MVM that is a generalization of PolSK to higher dimensions.


