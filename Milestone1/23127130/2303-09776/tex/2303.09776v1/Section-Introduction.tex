Over the last few years, there has been a keen interest in developing inter-data-center optical interconnects with bit rates approaching or exceeding 1 Tb/s %for information exchange between hyperscale data centers
\cite{Eiselt18,Perin2018,PerinJLT21,hu_ultrahigh-net-bitrate_2022}. 
%In particular, spectrally-efficient modulation formats for short-haul optical communications systems \cite{plant2018trends}, in conjunction  with multi-branch, self-homodyne, direct-detection receivers \cite{chen2020high,yoshida_phase-retrieving_2020,chen_dual_2020}, have become one of the most active  research areas in contemporary optical communications. 
Low cost, low energy consumption, and high spectral efficiency compared to binary intensity modulation/direct-detection are prerequisites for Tb/s short-haul applications \cite{Eiselt18}.  To satisfy the first two specifications, 
%several parallel fiber lanes (or separate wavelengths), each carrying $M$-ary modulation formats with symbol rates in the range 25--50 GBd  are currently used.  In addition, since inter-data-center link distances are typically smaller than 10 km, 
direct-detection optical receivers are currently preferred %due to their attractive price and low operating cost compared 
to coherent optical receivers. %Finally, 
To increase spectral efficiency, binary intensity modulation (IM) is gradually replaced by $M$-ary pulse amplitude modulation ($M$-PAM) \cite{Pang:20, Le:JLT:2020, wettlin_dsp_2020, ozolins_100_2021}. %Looking forward, 


Given the forecasted exponential increase in data traffic in the near future due to broadband applications \cite{Cisco20}, to accommodate traffic demands, it will be important to keep increasing  the spectral efficiency per fiber lane or per wavelength channel of short-haul optical links in an energy-efficient manner. The main disadvantage of $M$-PAM is that its energy consumption scales quadratically with the number of amplitude levels $M$ \cite{Proakis},  since the $M$-PAM constellation is one-dimensional. Therefore, it is imperative to adopt higher-dimensional modulation formats other than PAM, which are still amenable to direct detection, to keep upfront cost down.

Therefore, looking forward, we anticipate that it will be necessary to modulate additional attributes  of the optical wave other than the amplitude, e.g., the phase or the polarization, in order to increase spectral efficiency beyond today's values. Consequently, it will be necessary to recover the information imprinted in the electric field of the optical wave using self-homodyning \cite{mecozzi2016kramers, chen2020high, yoshida_JLT_19, Chen_JLT_20, ji_theoretical_2021}. %\footnote{
This is predicated on the assumption that the price of phase- and polarization-diversity coherent optical receivers will continue to be prohibitive for short-haul applications in the medium- to long-term, which is currently subject of debate  \cite{Perin2018, PerinJLT21, zhou_beyond_2020, morsy-osman_dsp-free_2018, jia_coherent_2021, rizzelli_scaling_2021}. Based on the above considerations, spectrally-efficient modulation formats for short-haul optical communications systems with higher dimensionality than $M$-PAM 
\cite{plant2018trends}, in conjunction  with multi-branch, self-homodyne, direct-detection receivers \cite{chen2020high,yoshida_JLT_19,Chen_JLT_20}, have become one of the most active research areas in contemporary optical communications. 

Let us discuss self-homodyning in direct-detection-based interferometry and  direct-detection-based polarimetry. Direct-detection-based interferometric receivers employ delay interferometers, like the ones used in differential receivers  for Differential Phase Shift Keying (DPSK) and Differential Quadrature Phase-Shift Keying (DQPSK) \cite{Ho2005}. Direct-detection polarimeters, on the other hand, are based on optical components performing polarization transformations (e.g., polarization beam splitters, polarization rotators) \cite{Huard1997} but the final result is also interferometric in nature, as explained  below. 

\begin{figure*}[!ht]
	\centering
		\includegraphics[width=1.0\textwidth]{Figs/MVM_DD_theory_organization_chart_color_2.png}
	\caption{Hierarchical organization of the paper (Abbreviations: MVM=Mode vector modulation, Tx=transmitter, Rx=receiver, UB=Union Bound, BER=bit error rate, SER=symbol error rate, MAP= maximum a posteriori, ML=maximum likelihood, MZM=Mach-Zehnder modulator, B2S mapping=bit-to-symbol mapping).}
	\label{fig:VisualAbstract}
\end{figure*}

In both direct-detection-based interferometry and  direct-detection-based polarimetry, the optical wave reaching the optical receiver is initially divided into multiple copies, which are subsequently modified by a series of optical devices, and then added pairwise before detection. The last step is crucial: Since photodiodes can be modeled as square-law detectors, the intensities of these sums of terms contain beating products which {depend on} %preserve 
 the phase of the optical wave. In fact, these beating terms are mathematically very similar to the ones provided  by coherent homodyne receivers. In other words,  under certain conditions, interferometric  and polarimetric direct-detection receivers enable access to all the attributes of the optical wave {apart from the overall phase} at relatively low cost and complexity compared to their coherent optical counterparts.

%much higher sensitivity than the conventional direct-detection receivers, like the ones used to detect amplitude modulation schemes, e.g., intensity modulation and 4-PAM, and may approach the performance of coherent homodyning under certain conditions, despite their relatively low cost and complexity compared to their coherent optical counterparts. 

%In this paper, we focus on polarization modulations and direct-detection-based polarimetric receivers. 

Let us turn our attention to digital polarization modulation formats. 
%They were first considered and studied during the late 80s and early 90s [Betti, Benedetto]. They were never used in practical applications, however, due to their complexity and the worldwide adoption and deployment of  wavelength-division-multiplexed binary intensity modulation direct-detection systems with erbium-doped fiber amplifiers in the mid 90s. 
Polarization shift keying (PolSK) was first studied in the late 1980s  \cite{betti1990multilevel, betti1992polarization} and early 1990s \cite{Benedetto_1992, benedetto1994multilevel, Benedetto_1995} before later falling into obscurity. It was recently revived as a subset of Stokes vector modulation (SVM) \cite{Kikuchi2014, Cercos2015a, Che2015, Che2016a, Dong2016, MorsyOsman2016, Chagnon2016, Chagnon2017, Sowailem2017, Ishimura2018, Ghosh2018, Hoang2018, TANEMURA2018, MorsyOsman2019, Ghosh2019, Ishimura2019, Cui2019, Kikuchi2020, Tanemura2020, Huo2020, Jin2020} when research in direct-detection systems was rekindled. This renewed interest in digital polarization modulation formats 
has been fueled by the maturity and low-cost of integrated photonics components and the possibility of using adaptive electronic equalizers in the  direct-detection optical receivers to compensate  for polarization rotations introduced by short optical fibers\footnote{Other applications of the Stokes space formalism were also proposed in combination with various modulation formats  \cite{Nazarathy_2006, Nazarathy2006, Nazarathy2007, perrone2018multidimensional, Ziaie2019, Ji2019, Feuer2020}, as well as  in combination with digital signal processing (DSP) \cite{Visintin2014, Caballero2017, Fernandes_2017, Che2019, Eltaieb2020}.}. 


%Optical polarimeters based on direct-detection, as scientific instruments, have a history of at least 170  years since the publication of the original work of G. Stokes [Stokes]. The idea of using optical      polarimeters as  receivers in optical telecommunications is much more recent.    Polarization shift keying (PolSK) was first studied in the late 1980s  [Betti]     and early 90s  [Benedetto] and later fell into oblivion but was recently revived in the form of Stokes vector modulation (SVM) [Kikuchi], when interest in direct-detection receivers was rekindled. 

%In order to increase the spectral efficiency of short-reach intensity-modulation direct-detection (IM-DD) optical communications systems, various flavors of Stokes vector modulation (SVM) formats have been considered. In the exclusive presence of thermal noise, 
SVM allows for more power-efficient signaling than $M$-PAM. This is achieved by spreading the constellation points in the three-dimensional Stokes space, as opposed to the one-dimensional $M$-PAM  signal space. 
%At the same time, it is possible to avoid the use of coherent receivers by employing a multi-branch direct-detection receiver with phase rotators and polarization beam splitters to detect the received signal SOP. 

To further increase the energy-efficiency of SVM formats, a transition to a higher-dimensional Stokes space is necessary, which can be achieved by using SVM in conjunction with few-mode and multicore fibers or free space \cite{Ji2019}. We call this format mode vector modulation (MVM).

SVM and MVM are spatial modulation formats \cite{Wen2019}. It is worth mentioning here that several papers studied spatial modulation formats for {\em coherent} optical communication systems over MCFs. For instance, Eriksson et al. \cite{eriksson2014k} analyzed multidimensional position modulation (MDPM) with multiple pulses per frame ($K$-over-$L$-MDPM) in combination with quadrature phase shift keying (QPSK), polarization multiplexed QPSK (PM-QPSK) and polarization-switched QPSK (PS-QPSK)  to increase both the spectral efficiency and the asymptotic power efficiency compared to conventional modulation formats. In companion  papers, Puttnam et al. \cite{Puttnam2014, puttnam2014energy, Puttnam2017} reviewed spatial modulation formats for high-capacity coherent optical systems using homogeneous multicore fibers.


In this paper, we study, for the first time, short-haul,  optical interconnects using MVM, in conjunction with optically-preamplified  direct-detection receivers. A  visual abstract of the paper is given in Fig. \ref{fig:VisualAbstract}. In the remainder of the paper, we elaborate on the following topics: 
\begin{enumerate}
    \item An overview of MVM along with the necessary mathematical formalism, notation, and simplifying assumptions (Sec. \ref{sec:MVMoverview});
    \item The optimal MVM transceiver architecture (Sec. \ref{sec:txrx_design});
    \item The performance limits of MVM optically-preamplified, direct-detection receivers using both Monte Carlo simulation and a new analytical formula that we derived for the union bound (Sec. \ref{sec:Pes});
    \item The  design of geometrically-shaped constellations with arbitrary cardinality $M$, obtained by numerical optimization of  various objective functions using the method of gradient descent (Sec. \ref{sec:GeometricShaping});
    \item The bit-to-symbol mapping optimization using simulated annealing (Sec. \ref{sec:b2s});
    \item The investigation of various constellation designs and bit encodings using analytical and numerical methods (Sec. \ref{sec:results});
    \item The use of simplex MVM constellations based on symmetric, informationally complete, positive operator-valued measure (SIC-POVM) vectors \cite{Fuchs} (Sec. \ref{sec:results}).
\end{enumerate}
Early results were presented in \cite{Roudas:ECOC21, Fink:IPC21, Kwapisz:CLEO22, Kwapisz:IPC22}.






%%%%OLD%%%%

% Cost-efficient high-data-rate optical interconnects constitute one of the most prominent research areas of contemporary optical communications. Consequently, during the last few years, there was a  surge in publications on short- and medium-haul optical communications systems that make use of direct-detection receivers and novel modulation formats. 

% Most recently, there has been a keen interest in developing such systems with bit rates approaching 1 Tb/s for the interconnection of hyperscale data centers   \cite{Eiselt18}, \cite{Perin2018}, \cite{Cheng:OFC:2019}, \cite{Le:JLT:2020}, \cite{Kahn:20}, \cite{PerinJLT21}. The requirements for such systems are, in order of decreasing importance, low cost, low energy consumption,  and high spectral efficiency \cite{Eiselt18}.

% In the case of data center buildings within the same geographical location, optical network  nodes are distanced by a few miles. For such applications, the use of direct-detection optical receivers is beneficial due to the low capital expenditure and operating cost compared to coherent optical receivers. 
% Subsequently, in conjunction with  direct-detection optical receivers, binary and quaternary digital amplitude modulation formats are typically used (e.g., binary intensity modulation (IM), quaternary pulse amplitude modulation (PAM-4)). However, these systems typically have low spectral efficiencies (theoretically up to 1-2 b/s/Hz for Nyquist pulse shaping). 

% Given the forecasted exponential increase in data traffic in the near future due to broadband applications \cite{Cisco20}, more sophisticated modulation formats might be used instead, which need to be amenable to direct detection, to  increase spectral efficiency compared to the aforementioned conventional digital amplitude modulation formats.

% Therefore, looking forward, it will be necessary to modulate the phase  of the optical wave, as well as the amplitude, in order to increase spectral efficiency beyond a few b/s/Hz. Assuming, as a working hypothesis, that the price of phase- and polarization-diversity coherent optical receiver will continue to be prohibitive for short-haul applications in the medium- to long-term, it will be necessary to recover the optical phase using either direct-detection-based interferometry or  direct-detection-based polarimetry. 

% Direct-detection-based interferometric receivers may employ delay interferometers, like the ones used in differential receivers  for Differential Phase Shift Keying (DPSK) and Differential Quadrature Phase-Shift Keying (DQPSK) \cite{Ho2005}. Direct-detection-based polarimeters on the other hand are based on optical components performing polarization transformations (e.g., polarization beam splitters, polarization rotators) \cite{Huard1997} but the final result is also interferometric in nature, as explained  below. 

% In both direct-detection-based interferometry and  direct-detection-based polarimetry, the optical wave reaching the optical receiver is initially divided into multiple copies, which are subsequently modified by series of optical devices, and then added pairwise before detection. The last step is crucial: Since photodiodes can be modeled as square-law detectors, the intensities of these sums of terms contain beating products which preserve the phase of the optical wave. In fact, these beating terms are mathematically very similar to the ones provided  by coherent homodyne receivers. Therefore, interferometric  and polarimetric direct-detection receivers exhibit much higher sensitivity than the conventional direct-detection receivers, like the ones used to detect amplitude modulation schemes, e.g., intensity modulation and 4-PAM, and may approach the performance of coherent homodyning under certain conditions, despite their relatively low cost and complexity compared to their coherent optical counterparts. 

% %In this paper, we focus on polarization modulations and direct-detection-based polarimetric receivers. 

% Let us turn our attention to digital polarization modulation formats. 
% %They were first considered and studied during the late 80s and early 90s [Betti, Benedetto]. They were never used in practical applications, however, due to their complexity and the worldwide adoption and deployment of  wavelength-division-multiplexed binary intensity modulation direct-detection systems with erbium-doped fiber amplifiers in the mid 90s. 
% Polarization shift keying (PolSK) was first studied in the late 1980s  \cite{betti1990multilevel}, \cite{betti1992polarization}     and early 90s \cite{Benedetto_1992}, \cite{benedetto1994multilevel}, \cite{Benedetto_1995},   and later fell into oblivion. It was recently revived in the form of Stokes vector modulation (SVM) \cite{Kikuchi2014}, \cite{Cercos2015a}, \cite{Che2015}, \cite{Che2016a}, \cite{Dong2016}, \cite{MorsyOsman2016},  \cite{Chagnon2016},  \cite{Chagnon2017}, \cite{Sowailem2017}, \cite{Ishimura2018}, \cite{Ghosh2018}, \cite{Hoang2018}, \cite{TANEMURA2018}, \cite{MorsyOsman2019},\cite{Ghosh2019}, \cite{Ishimura2019}, \cite{Cui2019}, \cite{Kikuchi2020}, \cite{Tanemura2020}, \cite{Huo2020}, \cite{Jin2020} when research in direct-detection systems was rekindled. This renewed interest in digital polarization modulation formats 
% has been fueled by the maturity and low-cost of integrated photonics components and the possibility of using adaptive electronic equalizers in the  direct-detection optical receivers to compensate  for polarization rotations introduced by short optical fibers. Other applications of the Stokes space formalism were also proposed in combination with various modulation formats  \cite{Nazarathy_2006}, \cite{Nazarathy2006}, \cite{Nazarathy2007}, \cite{perrone2018multidimensional}, \cite{Ziaie2019}, \cite{Ji2019}, \cite{Feuer2020}, as well as  in combination with digital signal processing (DSP) \cite{Visintin2014}, \cite{Caballero2017}, \cite{Fernandes_2017}, \cite{Che2019}, \cite{Eltaieb2020}. 


% %Optical polarimeters based on direct-detection, as scientific instruments, have a history of at least 170  years since the publication of the original work of G. Stokes [Stokes]. The idea of using optical      polarimeters as  receivers in optical telecommunications is much more recent.    Polarization shift keying (PolSK) was first studied in the late 1980s  [Betti]     and early 90s  [Benedetto] and later fell into oblivion but was recently revived in the form of Stokes vector modulation (SVM) [Kikuchi], when interest in direct-detection receivers was rekindled. 

% %In order to increase the spectral efficiency of short-reach intensity-modulation direct-detection (IM-DD) optical communications systems, various flavors of Stokes vector modulation (SVM) formats have been considered. In the exclusive presence of thermal noise, 
% SVM allows for more power-efficient signaling than $M$-ary pulse amplitude modulation ($M$-PAM). This is achieved by spreading the constellation points in the three-dimensional Stokes space as opposed to the one-dimensional $M$-PAM  signal space. 
% %At the same time, it is possible to avoid the use of coherent receivers by employing a multi-branch direct-detection receiver with phase rotators and polarization beam splitters to detect the received signal SOP. 
% To further increase the energy-efficiency of SVM formats, a transition to a higher-dimensional Stokes space is necessary, which can be achieved by using SVM in conjunction with few-mode or multicore fibers \cite{Ji2019}. We call this format mode vector modulation (MVM).

% It is worth mentioning here that several papers studied spatial modulation formats for {\em coherent} communication systems over MCFs. For instance, Eriksson et al. \cite{eriksson2014k} analyzed multidimensional position modulation (MDPM) with multiple pulses per frame ($K$-over-$L$-MDPM) in combination with quadrature phase shift keying (QPSK), polarization multiplexed QPSK (PM-QPSK) and polarization-switched QPSK (PS-QPSK)  to increase both the spectral efficiency and the asymptotic power efficiency compared to conventional modulation formats. In a companion invited paper, Puttnam et al. \cite{Puttnam2014}, \cite{puttnam2014energy}, \cite{Puttnam2017} reviewed spatial modulation formats for high-capacity coherent optical systems using homogeneous multi-core fibers.


% In this paper, we study short-haul,  optical interconnects using MVM, in conjunction with optically-preamplified  direct-detection receivers. In the remainder of the paper, we elaborate on the following topics: (i) The optimal MVM transceiver architecture; (ii) The use of simplex MVM constellations based on symmetric, informationally complete, positive operator valued measure (SIC-POVM) vectors \cite{Fuchs}; (iii) The  design of $M$-ary geometrically-shaped constellations obtained by numerical optimization of  various objective functions using the method of gradient descent; (iv) The optimal bit-to-symbol mapping using simulated annealing; and (v) The performance limits of MVM optically-preamplified, direct-detection receivers using both Monte Carlo simulation and a new analytical formula that we derived for the union bound. Early results were presented at \cite{Roudas:ECOC21}, \cite{Fink:IPC21}.


% \section{Mode Vector Modulation Overview}
% \subsection{MVM signal representation}
% \noindent  MVM modulation can be used together with multimode  and multicore fibers, as well as for free-space transmission. In this section, for the description of the operation of the  MVM transceiver, without loss of generality, we examine the special case of MVM transmission over an ideal homogeneous  multicore fiber with identical single-mode cores.

% We assume that we select a subset of $K$  single-mode cores of the multicore fiber (Fig.~\ref{fig:MVM_MCFs}).
% MVM modulation consists in sending optical pulses  over all these cores simultaneously with the same shape but different amplitudes and initial phases (Fig.~\ref{fig:intensityPlot_M64_N8.png}). Similar to SVM over single-mode fibers, where the optical wave can be analyzed in two orthogonal states of polarization, e.g., $x$ and $y$,  in the case of MVM over a homogeneous  single-mode-core multicore fiber, the composite optical wave can now be described by $N=2K$ orthogonal states of polarization, e.g., $x$ and $y$, in each core.


% \begin{figure}[!htb]
% 	\centering
% 	\includegraphics[width=.9\columnwidth]{Figs/LaunchedWavesMCF.png}
% 	\caption{MVM over homogeneous MCFs with single-mode cores.}
% 	\label{fig:MVM_MCFs}
% \end{figure}

% \begin{figure}[!htb]\centering
% 	\includegraphics[width=0.2\textwidth]{Figs/intensityPlot_M64_N8.png}
% 	\caption{Intensity plot of an MVM signal propagating over a four-core MCF with uncoupled single-mode cores.} 
% 	\label{fig:intensityPlot_M64_N8.png}
% \end{figure}

% The mathematical representation of the MVM signals at the  fiber input  is written as
% \begin{equation} 
% 	{\bf E}_{m} (t)=A_{m} e^{\iota \phi _{m} } g(t){\left| s_{m}  \right\rangle}  ,
% 	\label{eq:MVMsigs}
% \end{equation} 
% where $m=1,\ldots,M$, $A_m$ and $\phi_{m}$ denote the common amplitude  and phase, respectively, $g(t)$ is a real function that denotes the pulse shape, and ${\left| s_{m}  \right\rangle} $ is a generalized unit Jones vector with elements the complex excitations of the cores, i.e., the amplitudes and phases of electric fields of the optical waves \cite{Antonelli:12,Roudas_PJ_17,Roudas_JLT_18}.

% The signal energy is given by \cite{Proakis}
% \begin{eqnarray}
% \mathcal{E}_s &:=& \frac{1}{2}\int_{-\infty}^\infty {\bf E}_{m} (t)^\dagger {\bf E}_{m} (t) dt \nonumber\\
% &=& \frac{A_m}{2}\int_{-\infty}^\infty g(t)^2dt= \frac{A_m}{2}\mathcal{E}_g ,
%     \label{eq:sigEnergy}
% \end{eqnarray}
% where $\dagger$ denotes the adjoint (i.e., conjugate transpose) of a matrix and  $\mathcal{E}_g$ denotes the pulse energy defined as
% \begin{equation}
% \mathcal{E}_g := \int_{-\infty}^\infty g(t)^2dt.
%     \label{eq:pulseEnergy} 
% \end{equation}

% We can parameterize a unit Jones vector $\ket{s}$ using $2N-2$ hyperspherical coordinates \cite{Roudas_JLT_18}, i.e., 
% \begin{align}
% 	\ket{s} &:= \left[\cos (\phi _{1} ),\sin (\phi _{1} )\cos (\phi _{2} )e^{i\theta _{1} } ,\ldots , \right.  \notag\\
% 	&\quad \left. \sin (\phi _{1} )\cdots \sin (\phi _{N-2} )\sin (\phi _{N-1} )e^{i\theta _{N-1} } \right]^{T} .
% 	\label{eq:jv}
% \end{align}

% In Sec. \ref{sec:txrx_design}, we will be based on \eqref{eq:MVMsigs}, \eqref{eq:jv} in order to build the MVM transmitter using $N$ parallel branches of concatenated electro-optic Y-junctions and phase shifters. 

% Alternatively, the launched states  can be geometrically represented by real vectors in a  generalized $(N^2-1)-$dimensional Stokes space \cite{Antonelli:12,Roudas_PJ_17,Roudas_JLT_18}. Unit generalized Stokes vectors are defined by the quadratic form \cite{Roudas_JLT_18}
% \begin{equation} 
% 	\hat s : = \sqrt {\frac{N}{{2(N - 1)}}} \left\langle s \right|{\bf{\Lambda }}\left| s \right\rangle ,
% 	\label{eq:StokesVectorsdf}
% \end{equation}
% where ${\bf{\Lambda}}$ denotes the  generalized Gell-Mann spin vector  \cite{Roudas_JLT_18}. 

% Notice that the $(N^2-1)$ components of $\hat s$ are functions of the $2N-2$ hyperspherical coordinates of $\ket{s}$ and, therefore, they are  interdependent.

% In Sec. \ref{sec:txrx_design}, we will be based on \eqref{eq:StokesVectorsdf} in order to build the MVM direct-detection receiver using $N^2-1$ parallel branches. In addition, it is possible to take advantage of the interdependence of the components of $\hat s$ in order to reduce the hardware complexity of the receiver.

% From the generalized Stokes vector definition \eqref{eq:StokesVectorsdf}, we notice that the dimensionality of the generalized Stokes space grows quadratically with the number of spatial and polarization modes in MMFs/MCFs. Therefore, instead of using SVM-DD in conjunction with the conventional 3D Stokes space, we can generate more energy-efficient  constellations by spreading the constellation points in the generalized Stokes space. 




% In the remainder of the article, without loss of generality,  we consider that the common amplitude $A_m$ and phase $\phi_{m}$ in \eqref{eq:MVMsigs} are constant. In other words, we focus exclusively on a special case of MVM which is a generalization of PolSK in higher dimensions.

% Regarding mathematical notation, we follow the conventions of \cite{Roudas_PJ_17,Roudas_JLT_18}, where Dirac's bra-ket vectors represent both unit and non-unit vectors in the generalized Jones space, while hats represent unit vectors and arrows represent non-unit vectors in the generalized Stokes space. A full list of symbols is given at the end of the paper.

% \subsection{Simplifying assumptions\label{sec:assumptions}}

% \noindent In the following, we calculate analytically the back-to-back performance of $M$-ary MVM over $N$ spatial degrees of freedom in the amplified spontaneous emission (ASE) noise-limited regime. For mathematical tractability, we neglect all transmission impairments (other than ASE noise and random carrier phase shifts), as well as transceiver imperfections and implementation penalties. These simplifying assumptions are justified in the sense that we want to  quantify the ultimate potential of MVM for use in optical interconnects. 

% Nevertheless, it is worth contemplating right upfront about the anticipated impact of the most prominent of these effects. 

% In general, the extension of SVM to MVM requires similar conditions for transmission, i.e., negligible chromatic dispersion (CD), modal dispersion (MD), and mode-dependent loss (MDL), or their full compensation, either in the optical or the electronic domain, before making decisions on the received symbols at the receiver. Let us briefly contemplate how feasible it would be to satisfy these requirements in the case of practical homogeneous  multicore fibers with single-mode cores. 

% As a starting point, consider transmission over homogeneous MCFs with uncoupled or weekly-couple single-mode cores. These fibers typically exhibit static and dynamic intercore skew \cite{Puttnam2014}. The static differential mode group delay (DMGD) spread is on the order of 0.5 ns/km and grows linearly with the transmission distance. The DMGD spread due to the dynamic component of the intercore skew is of the order of 0.5 ps/km and also grows linearly with the transmission distance. 

% On the other hand, coupled-core MCFs exhibit modal dispersion among their supermodes and the DMGD grows with the square root of the transmission distance \cite{Hayashi:17}, \cite{Yoshida:19}. From  published values based on the characterization of several  coupled-core MCFs used in MDM experiments, we conclude that the MD coefficient is currently of the order of 3-6 ps/$\sqrt{\mathrm{km}}$. These values are much higher than typical PMD coefficient values for SMFs, e.g., from the data sheet of Corning\textregistered {SMF-28}\textregistered Ultra optical fiber \cite{SMF28u_datasheet}, we notice that the PMD coefficient  is less than 0.1 ps/$\sqrt{\mathrm{km}}$.

% Transmission impairments can be compensated using a combination of optical and electronic techniques at the transmitter and the receiver. These techniques are out of the scope of this paper, since we are interested in the back-to-back performance of MVM systems, and will be part of future work. For simplicity, in the depiction of the optically-preamplified MVM direct-detection receiver in Fig.~ \ref{fig:TxRx}~(b), we assume ideal optical post-compensation of all transmission impairments.
