In Sec. \ref{sec:Pes}--\ref {sec:GeometricShaping}, we derived an upper bound for the symbol error probability of $(N,M)$-MVM and discussed accelerated geometric constellation shaping in the generalized Stokes space using an electrostatic analog (i.e., an extension of the Thomson problem to higher dimensions). In Sec. \ref {sec:b2s}, we proposed a method to optimize the bit-to-symbol mapping of arbitrary MVM constellations using simulated annealing. In this section, we navigate the reader through the steps of the  formalism presented in Sec. \ref{sec:Pes}--\ref{sec:b2s}  by providing illustrative examples for specific $N$, $M$. 

\subsection{Constellation design}
\noindent As a starting point, to develop some physical intuition by visualization, we consider  constellation shaping and  bit-to-symbol mapping in the three-dimensional Stokes space.

We  first examine the optimal distribution of eight points on the surface of the Poincar\'e sphere $S^2$. From \cite{benedetto1994multilevel}, we know that the optimal constellation corresponds to a square antiprism inscribed in the sphere as shown in Fig. \ref{fig:polyhedronN_2_M_8.png} (rather than a cube as proposed by \cite{Kikuchi2014}). We want to test whether the solution of the Thomson problem using the method of gradient descent coincides with the solution of \cite{benedetto1994multilevel}.

Fig. \ref{fig:ThomsonConvergence_N2_M8_N2.png} shows the evolution of the potential energy given by \eqref{eq:ColumbPot:eq} as a function of the number of gradient descent iterations associated with 100 different random initial configurations of $M=8$ point charges on $S^2$. After about 1,000 iterations, all cases converge to  essentially identical square antiprisms (up to arbitrary 3D rotations), like the one shown in Fig. \ref{fig:polyhedronN_2_M_8.png}. 

\begin{figure}[!htb]
	\centering
	\includegraphics[width=2in]{Figs/polyhedronN_2_M_8.png}
	\caption{Square antiprism (Conditions: $N=2$, $M=8$).}
	\label{fig:polyhedronN_2_M_8.png}
\end{figure}

Close inspection reveals that the Euclidean distances between constellation points provided by the solution of the Thomson problem using the method of gradient descent in Fig. \ref{fig:ThomsonConvergence_N2_M8_N2.png} are slightly different from the ones provided by \cite{benedetto1994multilevel}. Actually, the constellation of \cite{benedetto1994multilevel} is unstable from an electrostatic point-of-view. In other words, if the constellation of \cite{benedetto1994multilevel} is provided as an initial configuration for the Thomson problem, the gradient in \eqref{eq:gradient_Thomson} of the potential energy in \eqref{eq:ColumbPot:eq} is non-zero, and, therefore, the constellation points experience Coulomb forces that move them to slightly different final positions. The same holds if one uses as initial guesses for the Thomson problem various point configurations provided by the minimization of alternative cost functions, e.g., for the Tammes problem \cite{jasper2019game}.



In conclusion, the polytopes provided by the minimization of different cost functions  for $N=2, M=8$ correspond to slightly different square antiprisms. For practical engineering purposes, however, we consider that these differences among various constellation configurations are immaterial and that the numerical solution of the Thomson problem using the method of gradient descent provides sufficient optimization effectiveness at low computational cost.
 
 \begin{figure}[!htb]
	\centering
	\includegraphics[width=3in]{Figs/ThomsonConvergence_N2_M8_N2.png}
	\caption{Thomson algorithm convergence for 100 distinct initial configurations (Conditions: $N=2$, $M=8$). After roughly 1,000 iterations, all cases have converged to square antiprisms.}
	\label{fig:ThomsonConvergence_N2_M8_N2.png}
\end{figure}



Next, we  shift our focus to the optimal bit-to-symbol mapping for the square antiprism. To facilitate visualization, we can represent the configuration of constellation points on the surface of the Poincar\'e sphere by a two-dimensional graph whose vertices represent the constellation points and its edges represent closest neighbors. For the case of the square antiprism of Fig. \ref{fig:polyhedronN_2_M_8.png}, we obtain the graph shown in Fig. \ref{fig:Graph_ThomsonqubicAntiprism_N2_M8.png}. The two square faces on opposite sides of the square antiprism are shown in red and green respectively, and the edges interconnecting them are shown with dotted black lines. The two square faces have sides equal to 1.17 and the edges interconnecting them are 1.29 long.

In Gray coding, closest neighbors at distance $1.17$ are assigned binary words that differ in only one bit, i.e., they have  a Hamming distance of one. Since each vertex in Fig. \ref{fig:Graph_ThomsonqubicAntiprism_N2_M8.png} has only two closest neighbors belonging to the same square face, it is straightforward to Gray label the vertices of the square faces using all binary words of three bits. For instance, one can Gray code the green square using the binary words with their most significant bit (msb) equal to zero and then use the remaining binary words with their most significant bit equal to one for the red square. The  proposed bit-to-symbol mapping in Fig. \ref{fig:Graph_ThomsonqubicAntiprism_N2_M8.png} is just one of  many possible Gray mappings.

However,  since the second-closest neighbors at distance $1.29$ are not very different distance-wise compared to the first neighbors at distance $1.17$, we have to take into account that erroneous symbol decisions can lead to second-closest neighbors with significant probability. 
%Considering the first- and second-closest neighbors, %in the bit-to-symbol mapping scheme, each point in Fig. \ref{fig:Graph_ThomsonqubicAntiprism_N2_M8.png}  should have a code that is %bit-wise as similar as possible to the codes of the four neighbors (connecting to it by an edge). 
% If we include the second-closest neighbors as well in our bit-to-symbol mapping scheme,  each point in Fig. \ref{fig:Graph_ThomsonqubicAntiprism_N2_M8.png}  %is connected by an edge to 
%  has four closest neighbors. 
 The proposed bit-to-symbol mapping in Fig. \ref{fig:Graph_ThomsonqubicAntiprism_N2_M8.png} offers almost all the benefits of Gray coding.
Each symbol error leads to 3 neighboring nodes that differ by one bit and to only one neighboring node that differs by two bits.


In  this particular case, the problem of assigning binary words to constellation points in order to minimize the bit error probability can be solved manually as follows: starting with the green square, we go into the clockwise direction and  
assign bits to symbols using all Gray words of zero msb. Then, starting from the vertex between 000 and 001, we trace the red square into the counterclockwise direction and  
assign bits to symbols using all Gray words of unit msb. We verified that the solution obtained via the simulated annealing algorithm is indeed the one found manually in Fig. \ref{fig:Graph_ThomsonqubicAntiprism_N2_M8.png}. This is evidence that the simulated annealing algorithm  performs adequately. 

\begin{figure}[!htb]
	\centering	\includegraphics[width=2in]{Figs/Graph_ThomsonqubicAntiprism_N2_M8.png}
	\caption{Bit-to-symbol mapping for the square antiprism (Conditions: $N=2$, $M=8$). The red, green, and black edges have lengths of approximately $1.17$, $1.17$, and $1.29$, respectively.}
	\label{fig:Graph_ThomsonqubicAntiprism_N2_M8.png}
\end{figure}

In order to further validate bit-to-symbol mappings provided by our simulated annealing algorithm, we ran benchmarking tests on constellations that admit Gray coding \cite{grayCode} (e.g., M-PSK and M-QAM). Our implementation of the simulated annealing algorithm displayed strong performance in these tests, often finding the global minimum for small constellation sizes. 

For larger $M$, the computational complexity of assigning binary words to constellation points in order to minimize the bit error probability grows exponentially. Let us see why that is: 
There are $M!$ ways that we can assign $M$ words of $k$ bits to the $M$ nodes. Using the dominant term in Stirling's approximation for factorials, we see that $M!\sim M^{M}e^{-M}$ for $M \gg 1$. Computing the objective function for all possible arrangements and selecting the bit-to-symbol mapping that yields the global minimum is clearly computationally prohibitive for large values of $M$. Simulated annealing can be used to solve such combinatorial optimization problems. 
While it may not find globally optimal solutions, evidence from tests performed on small constellation sizes suggests that simulated annealing can produce bit-to-symbol mappings that are sufficiently nearly-optimal.
 
% \textcolor{gray}{OLD: Even though one cannot  guarantee that  bit-to-symbol mappings found by simulated annealing indeed correspond to globally optimal solutions, evidence provided by  tests on small constellation sizes indicates that simulated annealing should be able to achieve nearly-optimal solutions, at least in principle.}

We continue by examining constellation shaping and  bit-to-symbol mapping in higher-dimensional Stokes spaces based on the physical intuition provided by  
%Figs. \ref{fig:polyhedronN_2_M_8.png}-\ref {fig:Graph_ThomsonqubicAntiprism_N2_M8.png} for 
the three-dimensional Stokes space.

Examples of optimized  constellations for $N=2$ and $N=4$ and $M=256$ are shown in Fig. \ref{fig:constellations}(b), (d), respectively.
\begin{figure}[ht] 
\captionsetup[subfigure]{justification=centering}
	\begin{subfigure}[b]{0.25\linewidth}
		\centering
		\includegraphics[width=0.75\linewidth]{Figs/SMF.png}  
		\caption{} 
		\label{fig7:a} 
		%\vspace{4ex}
	\end{subfigure}%% 
	\begin{subfigure}[b]{0.25\linewidth}
		\centering
		\includegraphics[width=0.75\linewidth]{Figs/polyhedronN_2_M_256.png}  
		\caption{} 
		\label{fig7:b} 
		%\vspace{4ex}
	\end{subfigure} 
	\begin{subfigure}[b]{0.25\linewidth}
		\centering
		\includegraphics[width=0.75\linewidth]{Figs/MCF2c.png} 
		\caption{} 
		\label{fig7:c} 
	\end{subfigure}%%
	\begin{subfigure}[b]{0.25\linewidth}
		\centering
		\includegraphics[width=0.65\linewidth]{Figs/plot2DN4M256.png}  
		\caption{} 
		\label{fig7:d} 
	\end{subfigure} 
	\caption{Illustration of optimized constellations (b), (d),  with $M$=256 points for (a) SMF ($N$=2); and (c) dual-core MCF ($N$=4), respectively.}
	\label{fig:constellations}
\end{figure}


To illustrate the difficulties of bit-to-symbol mapping in higher-dimensional Stokes spaces, let us take a closer look at the optimized MVM constellation for $N=4$, $M=32$. The histograms of internodal distances for the Thomson problem and the Tammes problem \cite{Math_Colorado_State_website} are shown in Fig. \ref{fig:HistogramComparison_M32_N4.png}. Notice that the constellations found by solving these two problems are not identical. For instance, there are 343 closest neighbor pairs at distance 1.33 in the Tammes problem, whereas the Thomson problem gives a continuum-like distribution of internodal distances in the range 1.1-1.6 for the closest neighbors. Choosing the Tammes problem solution due to its high degree of symmetry, we make a 2D graph of the 32 vertices with edges interconnecting closest neighbors only (Fig. \ref{fig:GraphSloanes4x32_2.png}). Since the average vertex degree in the graph is 21 (Fig. \ref{fig:HistogramVertexDegrees4x32.png}), it is obvious that Gray coding cannot be applied. For 32 constellation points, the number of possible codings is $32!\approx 2.6\times 10^{35}$, so a brute force optimization by exhaustive enumeration is impossible. 
The bit-to-symbol mapping given by simulated annealing is shown in Fig.  \ref{fig:GraphSloanes4x32_2.png}.

\begin{figure}[!htb]\centering
	\includegraphics[width=2.5in]{Figs/HistogramComparison_M32_N4.png}
	\caption{Histograms of internodal distances. (Yellow: Thomson problem; Black: Tammes problem \protect\cite{jasper2019game}).  (Conditions: $N=4$, $M=32$).} 
	\label{fig:HistogramComparison_M32_N4.png}
\end{figure}


\begin{figure}[!htb]\centering
	\includegraphics[width=2.5in]{Figs/GraphSloanes4x32_2.png}
	\caption{Constellation graph and bit-to-symbol mapping  (Conditions: $N=4$, $M=32$).} 
	\label{fig:GraphSloanes4x32_2.png}
\end{figure}

\begin{figure}[!htb]\centering
	\includegraphics[width=2.5in]{Figs/HistogramVertexDegrees4x32.png}
	\caption{Histogram of vertex degrees for the constellation graph of Fig. \ref{fig:GraphSloanes4x32_2.png} (Conditions: $N=4$, $M=32$).} 
	\label{fig:HistogramVertexDegrees4x32.png}
\end{figure}


\subsection{Validity of the error probability upper bounds}
The symbol error probability for equienergetic signals is bounded by using  the analytical union bound of Corollary \ref{cor:UBsumExact}. We want to gain insight into the validity and the tightness of this bound %accuracy of this formula 
 at various bit SNRs. In Fig. \ref{fig:Aymptotics_peb_vs_bit_snr_N_8_M_64_SpherePoints.png}, we check the validity of Corollary \ref{cor:UBsumExact} and the asymptotic expressions of Corollary \ref{asympt:cor} %for the upper bound for the symbol error probability 
 by Monte Carlo simulation. %\textcolor{red}{[JK: Are we sure this figure is not for the symbol error rate?]}
  We observe that the union bound is asymptotically tight and spot-on for bit error probabilities below the order of $10^{-3}$. Otherwise,   the union bound  overestimates larger error probabilities due to the significant overlap between the pairwise decision regions. %of pairwise half-planes. 

\begin{figure}[!htb]
	\centering
	% \includegraphics[width=.9\textwidth]{Images/Smale1/FischerCover.png}
	\includegraphics[width=3in]{Figs/Figs/Asymptotics_peb_vs_bit_snr_N_8_M_64_SpherePoints_v02.png}
	%	\includegraphics[scale=1.3]{Images/Smale1/Smale1FischerCover}
	\caption{Bit error probability vs bit SNR (dB) per SDOF (Points: Monte Carlo simulation; Blue line: union bound (Corollary \ref{cor:UBsumExact}); dashed lines: asymptotics (Corollary \ref{asympt:cor})(Conditions: $N=8$, $M=64$).}
	\label{fig:Aymptotics_peb_vs_bit_snr_N_8_M_64_SpherePoints.png}
\end{figure}

\subsection{Impact of bit-to-symbol mapping}
Fig.  \ref{fig:BERvRandom} compares the bit error rates (computed using Monte Carlo simulation) of the optimized bit-to-symbol mapping provided by simulated annealing (in blue) against multiple randomized encodings (in gray) for the same $(4,64)$-MVM constellation. We observe a performance gain of the optimized encoding over randomized encodings across a wide range of bit SNRs. In particular, we note that our bit-to-symbol mapping optimization requires a concrete choice of noise level $\sigma^2$ in defining the objective function of  \eqref{eq:SimulatedAnnealObjective} for simulated annealing. Hence it is possible that the suitability of an encoding might change with the noise level, requiring different optimizations for different noise levels. However, Fig. \ref{fig:BERvRandom} shows that a bit-to-symbol mapping optimized at one noise level (in this case, a bit SNR per SDOF of $10$ dB) performs well across a range of SNRs, showing that this concern is immaterial in practice. 

%This effect is particular pronounced at lower SNRs, where there is a correspondingly higher rate of symbol errors. 


\begin{figure}[!htb]
	\centering
	\includegraphics[width=3in]{Figs/BERvRandom.pdf}
	\caption{Bit error rate (BER) versus bit SNR per SDOF \nomenclature[$sdof$]{SDOF}{Spatial degrees of freedom}  for the optimized bit-to-symbol mapping (in blue) and randomized bit-to-symbol mappings (in gray).}
	\label{fig:BERvRandom}
\end{figure}



\subsection{Potential selection for constellation optimization}

%(\ref{mainExact:eq}) in the union bound (\ref{eq:UBgeneralForm})
We use the union bound of Corollary \ref{cor:UBsumExact} to compare the performance of various $(4,64)$-MVM constellations obtained via different optimization methods. Fig. \ref{fig:potential comparison} shows the symbol error probability $\Pe$\nomenclature[$pes$]{$\Pe$}{Symbol Error Probability} as a function of the symbol SNR per SDOF. The blue and orange curves correspond to constellations obtained using the gradient descent method with a Thomson (Coulomb) potential and with the union bound based on Corollary \ref{cor:UBsumExact} as an objective function, respectively. The green curve is a numerical approximation of a solution to the Tammes problem using the Matlab code provided by \cite{jasper2019game}. Finally, as a baseline for our analysis, the red curve corresponds to a standard Jones hypercube constellation (cf. Sec. \ref{subsection: numerical details}).

Given the different algorithmic approaches and computational complexities of these methods, the parameters are selected in such a way that each implementation takes roughly the same amount of computing time in order to provide a fair comparison. Using the union bound as the objective function yields the best performing constellation, as befits its intrinsic nature, despite its high computational complexity resulting in fewer gradient descent steps in the allotted time. Belying its extrinsic motivation, the Thomson method performs remarkably well, with only a slightest penalty compared to the Union Bound potential. The Tammes Problem method also performs quite well, with only a marginal performance loss compared to the Union Bound method.  Finally, we observe that all three numerical optimizations outperform the standard Jones hypercube constellation by nearly 3 dB. 

\begin{figure}[!htb]
    \centering
    \includegraphics[width=3in]{Figs/PotentialComparison.png}
    \caption{Performance comparison of different $(4,64)$-MVM constellations optimized using various potential functions and algorithms.}
    \label{fig:potential comparison}
\end{figure}

\subsection{Simplex MVM constellations}
 In Fig. \ref{fig:MVMperformance}, we plot the upper limit of the bit error probability, given by the union bound, for an optically-preamplified SIC-POVM  MVM DD receiver with matched optical filters, as a function of the electronic bit signal-to-noise ratio (SNR) per spatial degree of freedom. The Jones space dimension varies in the interval $N=2$--$16$ in power-of-two increments for different lines from top to bottom. The accuracy of the curves has also been checked by Monte Carlo simulation and the numerical data agree asymptotically with the analytical curves, but the Monte Carlo simulation results have been omitted from Fig. \ref{fig:MVMperformance} to avoid clutter. We observe that the  bit SNR required to achieve a given bit error probability decreases as $N$ increases, since the Euclidean distance between two SIC-POVM Stokes vectors increases as $\mathrm{d}=\sqrt{{2N^{2} }/({N^{2} -1}) }$. %We also notice that increasing the number of spatial degrees of freedom offers very little improvement in performance from one point onwards. This behavior is characteristic of simplex and orthogonal modulation formats \cite{Proakis2008}.
 
 \begin{figure}[!htb]
	\centering
	\includegraphics[width=3in]{Figs/Figs/pes_vs_snr_N2to16_SICs_v03.png}
	\caption{Bit error probability of $M-$ary MVM based on SIC-POVM vectors vs the bit SNR per spatial degree of freedom.}
	\label{fig:MVMperformance}
\end{figure}

\subsection{Performance comparison of various modulation formats}
Armed with Corollary \ref{cor:UBsumExact}, we want to compare the performance of MVM with that of conventional modulation formats for short-haul transmission and optically-preamplified direct-detection. 

For a fair comparison, we want to select the MVM constellation cardinality so that MVM exhibits the same spectral efficiency as conventional modulation formats. In single-mode transmission, spectral efficiency is defined as the ratio of the net bit rate after FEC to the channel bandwidth. Here, we use the following definition of the spectral efficiency per SDOF: Let the symbol interval be $T_s$ and the symbol rate be $R_s=T_s^{-1}$. Assuming ideal Nyquist pulses and carrier modulation, the signal bandwidth $B_s$ is equal to the symbol rate $R_s$. Suppose that the bit interval is $T_b$ and the bit rate is $R_b$. Let $M$ be the number of constellation symbols. Then, $k=\log_2 M$ bits are transmitted per symbol interval. We define the spectral efficiency per SDOF as $\eta:= {R_b}/{(N B_s)}$. Since $T_s=k T_b$ and  $R_b=k R_s$, the spectral density per spatial degree of freedom is $\eta=k N^{-1}$. 

For instance, for SIC-POVMs, there are $M=N^2$ constellation points and, therefore, the spectral density per spatial degree of freedom is $\eta=2 N^{-1} \log N $. 
%Therefore, for SIC-POVMs, the spectral efficiency increases in proportion to $N$ as $2\log_2N$. 
Consequently, by increasing the dimensionality $N$ of Jones space, the normalized spectral efficiency per SDOF decreases. 




For illustration, suppose we have an ideal homogeneous MCF with eight identical single-mode cores. The most straightforward way to use this fiber is to transmit $8$ independent parallel channels, each carrying a binary signal, e.g., based on either intensity modulation (IM), binary DPSK (DBPSK), or binary SVM (BSVM). When ideal Nyquist pulses with zero roll-off factor are used, all the aforementioned modulation formats can achieve a theoretical spectral efficiency of $0.5 \, \, \mathrm{b/s/Hz/SDOF}$.

Alternatively, rather than using the $8$ cores independently, we can transmit a single MVM channel by sending pulses over all eight cores in parallel, i.e., simultaneously utilizing all 16 available spatial degrees of freedom (SDOFs) .   Therefore, we should choose $256$ MVM, which results in a spectral  efficiency is 0.5 b/s/Hz/SDOF as well. In the $255$-dimensional generalized Stokes space, the optimal (16,256)-MVM constellation corresponds to a 256-simplex \cite{Roudas:ECOC21}. 



In Fig. \ref{fig:AMFs comparison}, we present analytical plots of the bit error probability vs. the bit SNR per SDOF at the decision device. Single-polarization, optically-preamplified, direct-detection receivers require $15.83\,  \, \mathrm{dB}$, $13\,\,  \mathrm{dB}$, and $16\, \, \mathrm{dB}$ for IM \cite{humblet1991bit}, BDPSK \cite{humblet1991bit}, and BSVM \cite{Kikuchi2020}, respectively, to achieve a bit error probability of $10^{-9}$. In contrast, the $(16, 256)$-MVM optically-preamplified, direct-detection receiver requires only $8.84 \,\, \mathrm{dB}$, to achieve the same bit error probability. This corresponds to bit SNR gains of $4.16\,\, \mathrm{dB}$, $7\, \, \mathrm{dB}$, and $7.16 \, \, \mathrm{dB}$ over BDSPK, IM, and BSVM, respectively. We conclude that the use of MVM can greatly improve system performance over conventional modulation formats at the expense of transceiver complexity \cite{Roudas:ECOC21}. 



\begin{figure}[!htb]
	\centering
	\includegraphics[width=3in]{Figs/Figs/peb_vs_snrb_AMFs_comparison_03.png}
	\caption{Bit error probability for $(16,256)$-MVM in comparison to conventional modulation formats for an $8$-core MCF.}
	\label{fig:AMFs comparison}
\end{figure}

\subsection{MVM performance for various $(N,M)$ pairs}
In this subsection, we present geometrically-optimized  signal sets  that correspond to the densest sphere packing in the generalized Stokes space. We show that the best trade-off between spectral and energy efficiency occurs for simplex constellations.

Fig. \ref{fig:SEplot_MN_MVM} shows the MVM spectral efficiency per SDOF vs the bit SNR per SDOF required to achieve a bit error probability of $10^{-4}$. Each curve corresponds to a different degree of freedom $N$, and each point within a curve corresponds to a different constellation cardinality $M$.  It is worth mentioning that these graphs represent geometrically-shaped constellations with optimized bit-to-symbol mapping. Non-optimized constellations lie on the right of these graphs. %That is to say, these graphs are reminiscent of the efficient frontiers in modern portfolio theory \cite{Markowitz}.
Furthermore, the vertices of different graphs in Fig. \ref{fig:SEplot_MN_MVM}  correspond to simplex constellations. Interestingly, the best combination of  spectral efficiency per SDOF and receiver sensitivity is achieved for SIC-POVMs. Higher spectral efficiencies can be obtained with a modest bit SNR penalty by switching to a constellation with more points, especially in higher-dimensional settings. 

% Fig. \ref{fig:SEplot_MN_MVM} shows the MVM spectral efficiency per SDOF vs the bit SNR per SDOF required to achieve a bit error probability of $10^{-4}$. Each curve corresponds to a different number of degrees of freedom $N$, and each point within a curve corresponds to a different constellation cardinality $M$.  It is worth stressing that these graphs represent geometrically-shaped constellations with optimized bit-to-symbol mapping. Suboptimal MVM constellations lie on the right of these graphs. %That is to say, these graphs are reminiscent of the efficient frontiers in modern portfolio theory \cite{Markowitz}.
% %Furthermore, the vertices of the curves for $N>2$ correspond to simplex constellations. 
% \begin{figure}[t!]
%     \centering
%     \begin{tabular}{cc}
%     \begin{subfigure}{0.25\textwidth}
%         \centering
%         \includegraphics[height=1.5in]{PoincareSphereVoronoiCellsN2M256_Thomson_cambridge.jpg}
%         \caption{}
%     \end{subfigure}%
%     &
%     \multirow{2}{*}{
%         \begin{subfigure}{0.7\textwidth}
%         \centering
%         \vspace{-1.3in}
%         \includegraphics[width=4.3in]{SEplot_MN_MVM_OFC23_trimmed.jpg}
%         \caption{}
%     \end{subfigure}%
%     }
%     \\
%     % uncomment next line
%     % \multicolumn{x}{c}{%
%     \begin{subfigure}{0.25\textwidth}
%         \centering
%         \includegraphics[height=1.5in]{MVM_SICs_N4.jpg}
%         \caption{}
%     \end{subfigure}%}// <- uncomment
%     % comment next line
%     & \\
%     \end{tabular}
%     \vspace{-0.2in}
%     \caption{(a) Optimized constellation and spherical Voronoi cells for $N=2, M=256$, obtained by solving the Thomson problem; (b) MVM spectral efficiencies per SDOF vs the bit SNR per SDOF at an average bit error probability of $10^{-4}$ for different degrees of freedom $N$ and constellation cardinalities $M$; (c) Intensity plots of the optimal MVM signal set for $N=4$, $M=16$,  over a two-core multicore fiber with identical uncoupled single-mode cores.}
%     \label{label}
%     \vspace{-0.2in}
% \end{figure}

For the qualitative interpretation of results of Fig. \ref{fig:SEplot_MN_MVM}, we need to take a closer look at the evaluation of error probability. The leading term of the asymptotic expression for the pairwise symbol error probability based on the union bound is given by \eqref{eq:asymptoticBehavior}.
% \begin{equation}\label{mainAsymptOld:eq}
% 	\boxed{     P_{e|s}^{ m'| m}
% 		\sim \frac{1}{2} \frac{1}{\sqrt{\pi}}
% 		% \frac{\sqrt{1+\gamma}}{\sqrt{1-\gamma}}
% 		\sqrt{\frac{1+\gamma}{1-\gamma}}
% 		\frac{1}{\sqrt{\gamma}\sqrt{\gamma_s}} \exp \left[{-\frac{\gamma_s (1 - \gamma)}{2}}\right] .}
% \end{equation}
% where $\gamma_s$ is the symbol signal-to-noise ratio (SNR) per SDOF and $\gamma:= |\ip{s_m}{s_{m'}}|$.

For $M>N$, the Welch–Rankin bound on $\gamma$ is written as \cite{jasper2019game}
$\gamma\geq \sqrt{{M-N}/{N(M-1)} }.$ The Welch–Rankin bound on $\gamma$ is not tight when the signal set cardinality tends to infinity. Below, we  estimate $\gamma$ from geometric arguments.

Fig. \ref{fig:fig17Aux} shows the optimal Thomson constellation and the partitioning of the sphere into Dirichlet (Voronoi) cells for $N=2, M=256$. In general, for $N=2$ and for large $M$’s, the Dirichlet cells for an optimal configuration are mostly hexagonal \cite{Saff}. For simplicity, let's assume that the constellation points form an ideal hexagonal lattice. The Dirichlet cell for a two-dimensional hexagonal lattice is a regular hexagon of side $d/\sqrt{3}$, where $d$ is the minimum Euclidean distance between pairs of points.  The area of each cell is $\delta A=\sqrt{3}d^2/2$. We can estimate $d$ if we divide the area of the unit sphere $S^2$, equal to $A= 4\pi$, by the total area of $M$ cells. We obtain the estimate
$d^2 \approx { {8\pi}/({\sqrt{3}M})}.$ 
We observe that, in the asymptotic limit of large $M$'s, the Euclidean distance is inversely proportional to the square root of the number of points $M$. By combining the formulas \cite{Gordon2000}
$d^2=\norm{\hat{s}-\hat{s}'}^2=2(1-\hat{s}\cdot \hat{s}')$,$
    \gamma^2 =|\ip{s}{s'}|^2=(1+\hat{s}\cdot \hat{s}')/2,$
and using the first-order Taylor expansion of the square root of $\gamma^2$, we obtain the average symbol error probability
$\bar{P}_{e|s}\sim\exp\left[-{\pi\gamma_s}/({2\sqrt{3}M)}\right].$

%For Gray-like bit-to-symbol mapping, the average bit error probability is related to the average symbol error probability by $\bar{P}_{e|b}\simeq\bar{P}_{e|s}/k$, where $k:=\log_2 M$, while in the case of quasi-orthogonal signal sets $\bar{P}_{e|b}\simeq M\bar{P}_{e|s}/[2(M-1)]\simeq \bar{P}_{e|s}/2$ \cite{Proakis}. We expect that Gray-like bit-to-symbol mapping will be applicable at large constellation cardinalities $M$, while orthogonal signal sets will occur for $M<N$ and quasi-orthogonal signal sets will occur for $N<M<N^2$.
The average bit error probability for a Gray-like bit-to-symbol mapping is related to the average symbol error probability by $\bar{P}{e|b}\simeq\bar{P}{e|s}/k$, where $k:=\log_2 M$. For quasi-orthogonal signal sets, it is related by $\bar{P}{e|b}\simeq M\bar{P}{e|s}/[2(M-1)]\simeq \bar{P}_{e|s}/2$. Gray-like bit-to-symbol mappings are expected at large constellation cardinalities $M$, while orthogonal signal sets exist for $M<N$ and quasi-orthogonal signal sets occur for $N<M<N^2$. 
%Asymptotically, the difference in SNR in dB for all these bit-to-symbol mappings is small and, for a qualitative interpretation of the results of  Fig. \ref{fig:SEplot_MN_MVM}, we can assume that $\bar{P}_{e|b}\simeq\bar{P}_{e|s}$.
In general, the difference in SNR between the different bit-to-symbol mappings is asymptotically small. For the purposes of qualitatively understanding the results shown in Fig. \ref{fig:SEplot_MN_MVM}, it is reasonable to assume that $\bar{P}_{e|b}\simeq\bar{P}_{e|s}$.

The spectral efficiency per SDOF for MVM is defined as $\eta:=k/N$ and the symbol SNR per SDOF is related to the bit SNR per SDOF via $\gamma_s:=k \gamma_b$.
 For a given average bit error probability, we can write for $N=2$ (SVM case) that
\begin{align}
\eta\sim \frac{\gamma_b \mathrm{(dB)}}{20\log{2}} .
\label{eq:asymptoticBehaviorN2}
\end{align}
Using a similar geometric argument for $N>2$ (MVM case), we find that $d^2\sim M^{-\frac{1}{N-1}}$ and we can write
\begin{align}
	\eta\sim \frac{N-1}{N}\frac{\gamma_b \mathrm{(dB)}}{10\log{2}}
	\label{eq:asymptoticBehaviorNlt2}
\end{align}
for a given average bit error probability.

Rephrasing the above expressions, we expect that the slope $\eta/{\gamma_b (dB)} \sim 0.16$  for $N=2$ at large constellation cardinalities $M$ and will increase towards 0.33 as $N\rightarrow{\infty}$, which is approximately the slope of the Shannon capacity formula. 

At the opposite extreme, $\gamma= 0$ for orthogonal signal sets  $M<N$ and we expect that
\begin{align}
	\eta\sim 10^{-\gamma_b \mathrm{(dB)} / 10}
	\label{eq:asymptoticBehaviorQO}
\end{align}
for a given average bit error probability. %$\bar{P}_{e|b}$,

%We can interpret the results of Fig. \ref{fig:SEplot_MN_MVM} using the above asymptotic analysis. In general, we expect that the MVM spectral efficiency $\eta$ as a function of the bit SNR per SDOF in dB $\gamma_b (dB)$ will be a C-shaped curve, with the upper section increasing linearly with $\gamma_b (dB)$ according to \eqref{eq:asymptoticBehaviorN2}, \eqref{eq:asymptoticBehaviorNlt2}, and the lower section decreasing exponentially with $\gamma_b (dB)$ according to \eqref{eq:asymptoticBehaviorQO}. The apex of each C-shaped curve  occurs for simplex constellations $M=N^2$, where $\gamma^2=(N+1)^{-1}$. We conclude that simplex constellations represent the best compromise between spectral and energy efficiency for $N>2$. A simplex signal set for $N=4$, $M=16$, is shown in Fig. \ref{fig:fig17Aux}.

Using the preceding asymptotic analysis, we consider the results shown in Fig. \ref{fig:SEplot_MN_MVM}. The MVM spectral efficiency $\eta$ is generally expected to follow a C-shaped curve when plotted against the bit SNR per SDOF $\gamma_b\, \mathrm{(db)}$. The upper part of the curve will increase linearly with the bit SNR per \eqref{eq:asymptoticBehaviorN2} and \eqref{eq:asymptoticBehaviorNlt2}, while the lower part of the curve will decrease exponentially with the bit SNR per \eqref{eq:asymptoticBehaviorQO}. Each curve's apex occurs for simplex constellations with $M=N^2$, where $\gamma^2=(N+1)^{-1}$. We thus conclude that simplex constellations are best balance between energy and spectral efficiency for $N>2$. An example of a simplex signal set for $N=4$ and $M=16$ is shown in Fig. \ref{fig:fig17Aux}.



% Of peculiar interest is the non-monotonic behavior of each curve, especially for the $N=2$ and $M=128, 256, 512$ points. Re-running the analysis in terms of symbol SNR and ignoring bit-to-symbol mappings yields the same qualitative behavior, which we further confirmed via Monte Carlo simulation. We hypothesize that this behavior is a result of packing efficiencies on the Poincar\'e sphere and the number of local neighbors for each constellation point, but a rigorous explanation is left for future work. 



 \begin{figure}[!htb]
	\centering
	\includegraphics[width=3in]{Figs/SEplot_MN_MVM_OFC23_trimmed.jpg}
	\caption{MVM spectral efficiencies per spatial degree of freedom (SDOF) vs the bit SNR per SDOF required to achieve a bit error probability of $10^{-4}$ for different degrees of freedom $N$ and constellation cardinalities $M$. Blue, red, and black curves correspond to $N=2,4,8$, respectively. The number listed next to each point corresponds to the constellation cardinality.}
	\label{fig:SEplot_MN_MVM}
\end{figure}

\begin{figure}[ht] 
	\centering
\captionsetup[subfigure]{justification=centering}
	\begin{subfigure}{0.5\linewidth}
		\centering
		\includegraphics[width=1.5in]{Figs/PoincareSphereVoronoiCellsN2M256_Thomson_cambridge.jpg}  
		\caption{} 
		%\vspace{4ex}
	\end{subfigure}%% 
 \hfill
	\begin{subfigure}{0.5\linewidth}
		\centering
		\includegraphics[width=1.5in]{Figs/MVM_SICs_N4.jpg}  
		\caption{} 
		\label{fig:MVM_SIC} 
		%\vspace{4ex}
  	\end{subfigure}%% 
	\caption{(a) Optimized constellation and spherical Voronoi cells for $N=2, M=256$, obtained by solving the Thomson problem; (b) Intensity plots of the optimal MVM signal set for $N=4$, $M=16$,  over a two-core multicore fiber with identical uncoupled single-mode cores.}
	\label{fig:fig17Aux}
\end{figure}

\subsection{Spectral efficiency vs energy efficiency trade-offs}
In Fig. \ref{fig:SE}, we plot  the change in spectral efficiency per SDOF for SIC-POVM MVM for different $N$ as a function of the bit SNR per SDOF required to achieve a bit error probability of $10^{-4}$ (in blue). On the same figure, we graph Shannon's formula for the spectral efficiency of an additive white Gaussian noise (AWGN) channel (in red) \cite{Proakis}. 
The maximum spectral efficiency for simplex MVM is equal to $1.06$ b/s/Hz/SDOF and occurs for $N = 3$. Similarly, the spectral efficiency for $N = 2$ and $N=4$ is 1 b/s/Hz/SDOF. This means that, at best, the spectral density of the simplex MVM is approximately equal to that of binary intensity modulation per SDOF for low $N$’s and decreases thereafter with increasing $N$. 

Notice that simplex MVM DD over SDM fibers offers 6.6 dB sensitivity improvement compared to conventional simplex SVM over SMFs ($N=2$), at the expense of spectral efficiency per SDOF. 

Based on our analysis, we conclude that using MVM DD over SDM fibers could potentially be beneficial, since the spatial degrees of freedom in SDM fibers are utilized as one channel instead of as individual channels, as is standard engineering practice. In comparison to SVM DD over SMFs, MVM offers a greater degree of flexibility for balancing energy consumption and spectral efficiency.



 \begin{figure}[!htb]
	\centering
	\includegraphics[width=3in]{Figs/SEplot_SICS_corrected_2.png}
	\caption{SIC-POVM MVM spectral efficiencies per spatial degree of freedom (SDOF) vs the bit SNR per SDOF required to achieve a bit error probability of $10^{-4}$ (in blue). Results for coherent PAM, QAM, and for SVM DD for various constellation cardinalities are also shown in magenta, green, and black, respectively.}
	\label{fig:SE}
\end{figure}


