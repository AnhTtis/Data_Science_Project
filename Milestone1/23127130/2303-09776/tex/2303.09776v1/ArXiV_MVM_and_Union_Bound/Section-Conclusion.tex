Renewed interest in direct-detection systems  is based on the assumption that the cost and energy consumption of coherent receivers designed for long-haul transmission will be prohibitive  for short-reach optical interconnects in the near and medium term \cite{Perin2018}, \cite{PerinJLT21}.

 Adopting the optimistic view that the price of SVM transceivers  will eventually become affordable  through photonic integration before that of coherent receivers, we went one step further and envisioned the use of SVM-like spatial modulations (referred to as {\em MVM}) over  multimode and multicore  fibers or free-space.

In this paper, we  investigated the merits of MVM with equipower signal sets, which is a direct extension of PolSK for the generalized Stokes space. In other words, we limited ourselves to a subset of the full spatial modulation/direct-detection family set. We derived an analytical upper limit for the back-to-back performance of $M$-ary MVM over $N$ spatial degrees of freedom in the amplified spontaneous emission (ASE) noise-limited regime. 

We also elaborated on the following topics: (i) The optimal MVM transceiver architecture; (ii) The use of simplex MVM constellations based on symmetric, informationally complete, positive operator valued measure (SIC-POVM) vectors; (iii) The design of M-ary geometrically-shaped constellations obtained by numerical optimization  of  various  objective  functions  using  the  method of  gradient  descent; and (iv)  The  optimal  bit-to-symbol  mapping using simulated annealing. 

We showed that it is potentially beneficial to use  MVM DD over SDM fibers, i.e., to use spatial degrees of freedom in SDM fibers together as a single channel instead of individually as separate channels, per standard engineering practice. Compared to SVM DD over SMFs, MVM DD over SDM fibers offers greater flexibility for better trade-offs between energy consumption and spectral efficiency.

 The successful  commercialization  of MVM eventually depends on technoeconomics. MVM, like other advanced direct-detection techniques for spectrally-efficient transmission \cite{Kikuchi2014}, \cite{mecozzi2016kramers}, \cite{yoshida_JLT_19}, \cite{Chen_JLT_20}, requires several parallel optical branches followed by ADCs and DSP, all of which increase cost and energy consumption in comparison to $M$-ary PAM and approach or even exceed the complexity of coherent receivers. Therefore, MVM's future commercial viability depends on the development of inexpensive  silicon photonic (SiP) integrated circuits and application-specific integrated circuits (ASICs) for DSP. In the end, it is likely that low-power, "light"  coherent receivers that avoid high-speed  analog-to-digital converters (ADCs) and digital signal processing (DSP) might eventually prevail against all direct-detection alternatives in intra- and inter-data center links \cite{Perin2018}, \cite{PerinJLT21}, \cite{zhou_beyond_2020}, \cite{morsy-osman_dsp-free_2018}, \cite{jia_coherent_2021}, \cite{rizzelli_scaling_2021}.