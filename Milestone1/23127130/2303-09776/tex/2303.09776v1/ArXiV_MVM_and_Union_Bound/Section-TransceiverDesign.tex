 In Fig. \ref{TxRx:sub1}, we draw the block diagram of an MVM transmitter for an ideal homogeneous multicore fiber with two identical single-mode cores ($N=4$). The schematic shows the optical components required for a single wavelength, but the architecture can be easily generalized for wavelength division multiplexing (WDM). Our goal is to generate the $N$ spatial and polarization components of the MVM signal as described by \eqref{eq:MVMsigs}. 
 
 The transmitter design begins with a single semiconductor laser diode. Subsequently, a Mach-Zehnder modulator, followed by a phase modulator, can be employed to alter the pulse shape $g(t)$, as well as the common amplitude $A_m$ and phase  $\phi_m$ of the MVM signal according to  \eqref{eq:MVMsigs}. After that, electro-optic splitters  can be used to partition the signal into $N$ parallel branches. By adjusting the control voltage of each Y-junction, an arbitrary power splitting ratio between its two output ports can be achieved. Recalling the hyper-spherical parametrization of the unit Jones vector $\ket{s_m}$ in \eqref{eq:jv}, the power splitting ratio is $\cos^2 (\phi _{1} ):\sin^2 (\phi _{1} )$ at the first Y-junction, $\cos^2 (\phi _{2} ):\sin^2 (\phi _{2} )$ at the second Y-junction, and so forth\footnote{An alternative design, similar to the one proposed by Kikuchi and Kawakami for SVM \cite{Kikuchi:14}, would entail the use of a passive 1:$N$ splitter, followed by an array of $N$ parallel Mach-Zehnder modulators, one at each individual transmitter branch.}.  Then, an array of phase modulators is used to generate phase differences among vector components. Finally, polarization controllers and polarization beam combiners are used to merge pairs of signals originating from different optical paths to create orthogonal states of polarization (SOPs) that are launched into separate fiber cores.
\begin{figure*}[ht!]
	\centering
	\captionsetup[subfigure]{justification=centering}
	\begin{subfigure}[t]{0.5\textwidth}
		\centering
		\includegraphics[width=.8\linewidth]{Figs/MVM_Tx_JLT}
		\caption{}
		\label{TxRx:sub1}
	\end{subfigure}%
	\begin{subfigure}[t]{0.5\textwidth}
		\centering
		\includegraphics[width=.8\linewidth]{Figs/MVM_Rx_JLT}
		\caption{}
		\label{TxRx:sub2}
	\end{subfigure}
\vspace{0.1 cm}
	\caption{Schematic of the proposed (a) MVM transmitter and (b) Optically-preamplified MVM DD receiver. Symbols: LD=laser diode, MZM=Mach-Zehnder modulator, OA=optical amplifier, OF=optical filter,  PCTR=polarization controller, PBC/S=polarization beam combiner/splitter, PD=photodiode, A/D=Analog-to-digital converter. (Condition: $N=4$.)}
	\label{fig:TxRx}
\end{figure*}

Fig. \ref{TxRx:sub2} shows the direct-detection receiver for $N=4$. The purpose of the optical front-end is to measure the  Stokes components of the incoming spatial superchannel, which are given by \eqref{eq:StokesVectorsdf}. To begin, it is necessary to separate the spatial and polarization components of the individual tributaries of the spatial superchannel using mode demultiplexers and polarization beam splitters. Then, an array of $N$ photodiodes are used to measure their powers. In addition, polarization controllers and power splitters/couplers are used to combine the different spatial and polarization components pairwise in order to create $N(N-1)/2$ distinct combinations. The real and imaginary parts of the latter are measured using an array of $2N(N-1)$ identical photodiodes grouped in pairs. In total, $2N^2-N$ photodiodes are needed to measure all the elements of the dyad  ${\bf S}=\dyad{s}$ independently \cite{Kwapisz:IPC22}. 

However, by taking advantage of the interdependence of Stokes components, as they are functions of the $2N-2$ hyperspherical coordinates of $\ket{s}$, it is possible to reduce the direct-detection receiver front-end complexity. In \cite{Kwapisz:IPC22}, we showed that $O(N)$ photodiodes are sufficient to estimate the Stokes parameters of the spatial superchannel.

For the purposes of this article, we assume that the simplifying assumptions of Sec. \ref{sec:assumptions} hold, i.e., the optically-equalized communication channel exhibits negligible  CD, MD, and MDL  so that the residual transmission effects (random modal birefringence and random differential carrier phase shifts) can be modeled by a frequency-independent random unitary matrix. The  action of this transfer matrix is a dynamic rotation of the received mode vector that varies slowly over time. 

Stokes receiver DSP can estimate Stokes vector rotations caused by fiber propagation and can counteract them by multiplying, in Stokes space, the received generalized Stokes vector with a compensating generalized M\"uller matrix. Alternatively, it is possible to perform optical derotation and MD/PMD compensation of the received generalized Jones vector, which is driven by the Stokes vector receiver DSP.
%Nevertheless, this equalization process might create noise enhancement in Stokes space \cite{Huo2020}.To simplify the mathematical analysis, we assume here optical derotation of the received generalized Stokes vector, which is driven by the Stokes vector receiver DSP. This equalization process does not result in noise enhancement in the absence of MDL.  Indeed, the incoming ASE noise affects the Jones vector, and the DSP compensation is equivalent to unitary rotation of this vector, which does not change the statistical properties of the ASE noise.

The decisions of the Stokes vector receiver in Fig. \ref{TxRx:sub2} are based on the maximum a posteriori (MAP) criterion \cite{Proakis}, which is equivalent to the maximum-likelihood criterion for equiprobable signals \cite{Proakis} (see Sec. \ref{sec:decision_criterion}). Applying the maximum-likelihood criterion in the generalized Jones space, we will show that the optimum decision maximizes 
the modulus of the inner product of Jones vectors (cf. \eqref{eq:optScheme}).

% Applying the maximum-likelihood criterion in the generalized Jones space, we will show that the optimum decision maximizes %JKApr metric is
%  the modulus of the inner product of Jones vectors (cf. \eqref{eq:optScheme}). Subsequently, this can be transformed to the minimum Euclidean distance criterion in Stokes space. This is optimal for the case of equipower MVM signals, which is the focus of this paper. Benedetto and Poggiolini \cite{benedetto1994multilevel} showed that this decision criterion is not optimal for the case of non-equipower signals and give an improved criterion for non-equipower signal constellations or in the presence of PDL for $N=2$. The extension of our formalism to non-equipower signals or equipower signals in the presence of residual  PDL/MDL will be part of future work.


%%%OLD%%%

%  Fig. \ref{fig:TxRx} (a) shows an implementation example of the MVM transmitter for a two-core homogeneous multicore fiber with identical single-mode cores ($N=4$). We display the components needed for a single wavelength tributary but the architecture is amenable to wavelength division multiplexing (WDM). A single semiconductor laser diode is used to create the $N$ spatial and polarization components of the MVM signal given by \eqref{eq:MVMsigs}, \eqref{eq:jv}. First, the common amplitude and phase in \eqref{eq:MVMsigs} can be modulated using a Mach-Zehnder modulator and a phase modulator, respectively. Then, the signal is divided into $N$ parallel branches using electro-optic splitters. The first Y-junction splits the incoming signal into its two output ports with an arbitrary power splitting ratio $\cos^2 (\phi _{1} ):\sin^2 (\phi _{1} )$, the second Y-junction splits the incoming signal into its two output ports with an arbitrary power splitting ratio $\cos^2 (\phi _{2} ):\sin^2 (\phi _{2} )$, and so forth. An alternative design, similar to the one proposed by Kikuchi and Kawakami for SVM \cite{Kikuchi:14}, would entail the use of a passive 1:$N$ splitter, followed by an array of $N$ parallel Mach-Zehnder modulators, one at each individual transmitter branch.  The phase modulators at different branches generate the phase differences $\theta_{1}, \theta_{2}, \dots$ among Jones vector components. Signals from two subsequent signal paths are set into orthogonal states of polarization (SOPs) using polarization controllers and polarization beam combiners and are launched into different fiber cores.
% \begin{figure*}[ht!]
% 	\centering
% 	\begin{subfigure}[t]{0.5\textwidth}
% 		\centering
% 		\includegraphics[width=.8\linewidth]{Figs/MVM_Tx_JLT}
% 		\caption{}
% 		\label{TxRx:sub1}
% 	\end{subfigure}%
% 	\begin{subfigure}[t]{0.5\textwidth}
% 		\centering
% 		\includegraphics[width=.8\linewidth]{Figs/MVM_Rx_JLT}
% 		\caption{}
% 		\label{TxRx:sub2}
% 	\end{subfigure}
% \vspace{0.1 cm}
% 	\caption{Schematic of the proposed (a) MVM transmitter and (b) Optically-preamplified MVM DD receiver. Symbols: LD=laser diode, MZM=Mach-Zehnder modulator, OA=optical amplifier, OF=optical filter,  PCTR=polarization controller, PBC/S=polarization beam combiner/splitter, PD=photodiode, A/D=Analog-to-digital converter. (Condition: $N=4$.)}
% 	\label{fig:TxRx}
% \end{figure*}

% The direct-detection receiver front-end for $N=4$ is shown in Fig. \ref{fig:TxRx} (b). The receiver measures the Stokes components of the incoming optical signal based on \eqref{eq:StokesVectorsdf}: First, the spatial and polarization components of the Jones vectors representing the received MVM signals are separated. Then, they are mixed pairwise to create $2(N^2-1)$ combinations, using a network of power splitters/couplers, polarization controllers, and polarization beam splitters. Finally, an array of $2(N^2-1)$ identical photodiodes are used pairwise for direct-detection of the $N^2-1$ Stokes components of the received signals. It can be shown that the front-end complexity can be reduced significantly taking into account that Stokes vector components are interdependent since they are functions of the  $2N-2$ hyperspherical coordinates in  $\ket{s}$. The number of required photodiodes can therefore be reduced to only $3N-2$ without loss of information.

% For the purposes of this article, we assume that the simplifying assumptions of Sec. \ref{sec:assumptions} hold, i.e., the optically-equalized communication channel exhibits negligible  CD, MD, and MDL  so that the residual transmission effects (random modal birefringence, random differential carrier phase shifts) can be modeled by a frequency-independent random unitary matrix. The  action of this matrix is a dynamic rotation of the received mode vector that varies slowly over time. 

% Stokes receiver DSP can estimate Stokes vector rotations caused by fiber propagation and can counteract them by multiplying in Stokes space the received generalized Stokes vector with a compensating generalized M\"uller matrix. Nevertheless, this equalization process might create noise enhancement in Stokes space \cite{Huo2020}.
% To simplify the mathematical analysis, we assume here optical derotation of the received generalized Stokes vector, which is driven by the Stokes vector receiver DSP. This equalization process does not result in noise enhancement in the absence of MDL.  Indeed, the incoming ASE noise affects the Jones vector, and the DSP compensation is equivalent to unitary rotation of this vector, which does not change the statistical properties of the ASE noise.

% The decisions of the Stokes vector receiver in Fig. \ref{fig:TxRx} (b) are based on the maximum a posteriori (MAP) criterion \cite{Proakis}, which is equivalent to the maximum-likelihood criterion for equiprobable signals \cite{Proakis}. Applying the maximum-likelihood criterion in the generalized Jones space, we will show that the optimum decision metric is the modulus of the inner product of Jones vectors (cf. \eqref{eq:optScheme}). Subsequently, this can be transformed to the minimum Euclidean distance criterion in Stokes space. This is optimal for the case of equipower MVM signals, which is the focus of this paper. Benedetto and Poggiolini \cite{benedetto1994multilevel} showed that this decision criterion is not optimal for the case of non-equipower signals and give an improved criterion for non-equipower signal constellations or in the presence of PDL for $N=2$. The extension of our formalism to non-equipower signals or equipower signals in the presence of residual  PDL/MDL will be part of future work.
