 As a prelude to the Marcum function asymptotics  in  Appendix \ref{App:Marcum}, let us rederive  the modified Bessel function asymptotic expression   
  \begin{equation}
  I_0(x) := \frac{1}{\pi} \int_0^\pi e^{x \cos{\theta}} \, d\theta \sim \frac{1}{\sqrt{2\pi x}} e^{x}, 
\end{equation}
giving rigorous error bounds.

Anticipating $e^x$ behavior,  we multiply $I_0(x)$ by $e^{-x}$ and integrate, substituting 
$u:=1-\cos \theta$ (with $\sin \theta = \sqrt{1-\cos^2 \theta} = \sqrt{u(2-u)}$), to get 
\begin{align}
  g(x)&:=e^{-x}I_0(x) \notag\\
  &= \frac{1}{\pi} \int_0^\pi e^{-x (1-\cos{\theta})} \, d\theta \notag\\
  &= \frac{1}{\pi} \int_0^2 \frac{e^{-xu}}{\sqrt{u(2-u)}} \, du. 
\end{align}



The main contribution is from near $u=0$ so, to start,  we replace $\sqrt{2-u}$ by $\sqrt{2}$ and approximate (via substitution $y:=ux$ with $du=x^{-1}dy$),
\begin{align}
%  \int_0^2 \frac{e^{-xu}}{\sqrt{u}} \, du & =
    g(x) &=      \frac{1}{\pi\sqrt{2}}\int_0^2 u^{-\frac{1}{2}}e^{-xu} \, du  \notag\\
                                          %  = \int_0^1  x^{\frac{1}{2}}(ux)^{-\frac{1}{2}}e^{-xu} \, x^{-1}d(ux) \\
     &= \frac{1}{\pi\sqrt{2}}\int_0^{2x} x^{-\frac{1}{2}}y^{-\frac{1}{2}}e^{-y} \, dy \notag \\
                                          % &= x^{-\frac{1}{2}} \left(  \Gamma\left(\frac{1}{2}\right) + \text{Error less than $e^{-x}$} \right).
     &= \frac{1}{\pi\sqrt{2}} x^{-\frac{1}{2}} \Gamma\left(\frac{1}{2}\right) \left( 1  +  O\left( e^{-2x} \right) \right).
     \label{eq: g_x}
\end{align}
%where we used a crude estimate $\int_{2x}^\infty y^{-\frac{1}{2}}e^{-y} \, dy$
Above, we used the Gamma function
\begin{align*}
  \Gamma(s):=\int_0^{\infty} y^{s-1}e^{-y} \, dy.
\end{align*}
Notice its well-known value $\Gamma\left(\frac{1}{2}\right)=\sqrt{\pi}$. The term  $O\left( e^{-2x} \right)$ in \eqref{eq: g_x} is precisely quantified by a primitive tail bound (via $v:=y-2x$) 
\begin{align}
%&\int_+^\infty u^{-\frac{1}{2}}e^{-xu} \, du = e^{-2x} \int_0^\infty (v+2)^{-\frac{1}{2}}e^{-xv} \, dv \leq \Gamma(\frac{1}{2}) e^{-2x}  
    \int_{2x}^\infty y^{-\frac{1}{2}}e^{-y} \, dy  &= e^{-2x} \int_0^\infty (v+2x)^{-\frac{1}{2}}e^{-v} \, dv \notag \\
    &\leq  \Gamma\left(\frac{1}{2}\right) e^{-2x} \notag\\
    &= \sqrt{\pi}   e^{-2x}.  
\end{align}

% \textcolor{lightgray}{So if we can justify dropping $\sqrt{2-u}$ then we would get asymptotics
% \begin{align*}
%   I_0(x) &\approx \frac{1}{\pi}\frac{1}{\sqrt{2}} \frac{e^x}{\sqrt{x}}\left(\sqrt{\pi}  + \text{Error less than $e^{-x}$} \right) = \frac{1}{\sqrt{2 \pi }} \frac{e^x}{\sqrt{x}}\left(1 + O(e^{-x}) \right)   
% \end{align*}
% which is unrealistically good having exponentially small relative error.
% }
% \bigskip

The above accounts for the constant term in the linearization $\frac{1}{\sqrt{2-u}} = \frac{1}{\sqrt{2}} + O(u)$, which is valid for any fixed $u \in [0,\delta]$ where $\delta < 2$ is fixed to avoid the singularity.
Note that
\begin{align*}
   \int_\delta^2 \frac{e^{-xu}}{\sqrt{u(2-u)}} \, du \leq  e^{-x\delta} \int_0^2 \frac{1}{\sqrt{u(2-u)}} \, du
\end{align*}
is exponentially small, so restricting the integration below to $[0,\delta]$ does not give up anything and small $\delta>0$ facilitate better control on the error term $O(u)$. 
% and the the exponentially small error term, $O\left( e^{-2x} \right) \leq \sqrt{\pi}   e^{-2x}$, is dwarfed by that indiced by
This term produces relative error in $g(x)$  of order $x^{-\frac{3}{2}}$:
\begin{align}
  \int_0^\delta O(u) u^{-\frac{1}{2}}e^{-xu} \, du &= 
  O(1) \cdot \int_0^\delta u^{\frac{1}{2}}e^{-xu} \, du \notag \\
  &=  O(1) \cdot x^{-\frac{3}{2}} \cdot \int_0^{\delta x} y^{\frac{1}{2}}e^{-y} \, dy \notag \\
    &\leq  O(1) \cdot \Gamma\left(\frac{3}{2}\right)  x^{-\frac{3}{2}}.
\end{align}

Putting things together, we get the desired asymptotics 
\begin{equation}
  I_0(x) = \frac{1}{\sqrt{2 \pi }} \frac{e^x}{\sqrt{x}}\left(1 + O(x^{-1}) \right)   
\end{equation}
with relative error  $O(x^{-1})$, which we can precisely bound if asked by instantiating $\delta$, etc.. 
More precise asymptotics can be obtained in an analogous way by using the longer Taylor approximation  
\begin{align*}
  \frac{1}{\sqrt{2-u}} = % = \frac{1}{\sqrt{2}} + O(u)
  \frac{1}{\sqrt{2}}+\frac{u}{4 \sqrt{2}}+\frac{3 u^2}{32
   \sqrt{2}}+\frac{5 u^3}{128 \sqrt{2}}+ \ldots. \addtag
\end{align*}

For instance, the $x^{-1}$ correction arises as:
\begin{align}
  I_0(x) &= \frac{1}{\sqrt{2 \pi }} \frac{e^x}{\sqrt{x}}\left(1 + \frac{1}{8}x^{-1} + \ldots \right).   
\end{align}
%But we do not need this. 

% [STOP]


% \pagebreak 

% $\frac{1}{\sqrt{2-u}} = \frac{1}{\sqrt{2}} + O(u)$

% to write
% \begin{align*}
%   &\int_0^1 \frac{e^{-xu}}{\sqrt{u(2-u)}} \, du  \\
%   &= \frac{1}{\sqrt{2}} \int_0^1 u^{-\frac{1}{2}}e^{-xu} \, du +  O(1) \cdot \int_0^1 u^{\frac{1}{2}}e^{-xu} \, du \\
%   % &= \frac{1}{\sqrt{2}} x^{-\frac{1}{2}} \left(  \Gamma\left(\frac{1}{2}\right) + \text{Error less than $e^{-x}$} \right) +  O(1) \cdot \int_0^1 u^{\frac{1}{2}}e^{-xu} \, du \\
%   % &= \frac{1}{\sqrt{2}} x^{-\frac{1}{2}} \left(  \Gamma\left(\frac{1}{2}\right) + \text{Error less than $e^{-x}$} \right)
%   %   +  O(1) \cdot x^{-\frac{3}{2}} \cdot \int_0^x y^{\frac{1}{2}}e^{-y} \, dy \\
%   &= \frac{1}{\sqrt{2}} \cdot x^{-\frac{1}{2}} \cdot \int_0^x y^{-\frac{1}{2}}e^{-y} \, dy
%     +  O(1) \cdot x^{-\frac{3}{2}} \cdot \int_0^x y^{\frac{1}{2}}e^{-y} \, dy
% \end{align*}

% Thus we obtained a very relaxed bound 
% \begin{align*}
%  I_0(x) &= \frac{1}{\sqrt{2 \pi }} \frac{e^x}{\sqrt{x}}\left(1 + O(x^{-1}) \right).   
% \end{align*}

% By taking longer Taylor expansions of $ \frac{1}{\sqrt{2-u}}$ we could get better asymptotics.


% % The true integral, via sub $y:=ux$ with $du=x^{-1}dy$, is 
% % \begin{align*}
% %   \int_0^1 \frac{e^{-xu}}{\sqrt{u(2-u)}} \, du  = \int_0^x \frac{e^{-y}}{\sqrt{y(2x-y)}} \, dy 
% % \end{align*}
 

%\bigskip
%
%\bigskip