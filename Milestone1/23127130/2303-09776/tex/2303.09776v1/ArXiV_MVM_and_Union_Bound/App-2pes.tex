For ease of reference, we rederive here the union bound for $N=2$, initially  presented in \cite{betti1990multilevel}, \cite{betti1992polarization}. We use the formalism of Gordon and Kogelnik \cite{Gordon2000} and work exclusively in Stokes space.

Based on the decision metric \eqref{eq:optScheme} and  relationship [3.11] of \cite{Gordon2000} that links the dot products in Jones and Stokes space
\begin{equation}
\abs{\bra{r}\ket{s_m}}^2=\frac{1}{2}\left(1+ \vec{r}\cdot\hat{s}_m' \right),
\end{equation} 
we can express the pairwise symbol error probability as follows
\begin{equation}
	{P}_{e|s|{\rm pair}}=P(\vec{r}\cdot\hat{s}_m \leq \vec{r}\cdot\hat{s}_m'),
\end{equation} 
or, equivalently,
\begin{equation}
	{P}_{e|s|{\rm pair}}=P\left[\vec{r}\cdot(\hat{s}_m-\hat{s}_m') \leq 0 \right].
	\label{eq:pes_Pair_Stokes}
\end{equation} 

We assume that the unit vector $\hat{x}$ along the $x-$axis in Stokes space is aligned with $\hat{s}_m-\hat{s}_m'$
\begin{equation}
	\hat{x}:=\frac{\hat{s}_m-\hat{s}_m'}{\norm{\hat{s}_m-\hat{s}_m'}}.
\end{equation} 

Then, \eqref{eq:pes_Pair_Stokes} can be rewritten in compact form
\begin{equation}
	{P}_{e|s|{\rm pair}}=P(\vec{r}\cdot\hat{x} \leq 0).
\end{equation} 

The projection of $\vec{r}$ on $\hat{x}$ yields the $x-$Stokes component of $\vec{r}$, which is the difference in the intensities of the $x$ and $y$ SOP's (cf. relationship [2.2] of \cite{Gordon2000})
\begin{equation}
	{P}_{e|s|{\rm pair}}=P(\vec{r}\cdot\hat{x} \leq 0)
	=P(|r_x|^2-|r_y|^2 \leq 0),
\end{equation} 
where $r_x$, $r_y$ are the components of the received Jones vector.

Rewriting the above inequality in terms of the quadrature components of $r_x$, $r_y$, denoted by $r_{xr}, r_{xi}, r_{yr}, r_{yi}$, where the subscripts $r, i$ indicate the real and imaginary parts, respectively,  we obtain
\begin{equation}
	{P}_{c|s|{\rm pair}}=P(r_{xr}^2+r_{xi}^2 \leq r_{yr}^2+r_{yi}^2).
	\label{eq:Pes_inequality}
\end{equation} 

We define the auxiliary random variables
\begin{subequations}
	\begin{align}
		y_1 & :=  r_{xr}^2+r_{xi}^2  \\
		y_2 & :=  r_{yr}^2+r_{yi}^2
	\end{align}
\end{subequations}  

In the above, $r_{xr},r_{xi},r_{yr},r_{yi}
$ are i.i.d. Gaussian random variables with means $A_{xr},A_{xi},A_{yr},A_{yi}
$ and variance $\sigma^2$. Then, $y_1$,$y_2$ follow non-central $\chi^2$-distributions with two degrees of freedom \cite{Proakis}
\begin{equation}
	p_{y_i}(y_i)=\frac{1}{2\sigma^2}\exp\left(-\frac{A_i^2+y_i}{2\sigma^2}\right)I_0\left(\frac{A_i\sqrt{y_i}}{\sigma^2}\right),
	\label{eq:chi2}
\end{equation} 
where we defined the means
\begin{subequations}
	\begin{align}
		A_1^2 := & A_{xr}^2+A_{xi}^2 =|A_x|^2,\\
		A_2^2 := & A_{yr}^2+A_{yi}^2 =|A_y|^2.
			\end{align}
\end{subequations}

The means in \eqref{eq:chi2} can be rewritten in Stokes space as
\begin{subequations}
	\begin{align}
		A_1^2= & |A_x|^2=|\bra{s_m}\ket{x}|^2 =\frac{1}{2}\left(1+\hat{s}_m\cdot\hat{x}\right),\\
		A_2^2 = & |A_y|^2=|\bra{s_m}\ket{y}|^2 =\frac{1}{2}\left(1+\hat{s}_m\cdot\hat{y}\right) .
		\end{align}
\end{subequations}

Assume that $\hat{s}_m,\hat{s}_m'$ form an angle $\theta$ in Stokes space
\begin{equation}
	\hat{s}_m\cdot\hat{s}_m'=\cos{\theta}.
\end{equation} 

Then, 
\begin{equation}
	\hat{s}_m\cdot\hat{x}= \hat{s}_m\cdot\frac{\hat{s}_m-\hat{s}_m'}{\norm{\hat{s}_m-\hat{s}_m'}}=\sin\frac{\theta}{2}.
\end{equation} 

Given that $\hat{x},\hat{y}$ are antipodal in Stokes space
\begin{equation}
	\hat{s}_m\cdot\hat{x}=-\hat{s}_m\cdot\hat{y} ,
\end{equation} 
relationships \eqref{eq:chi2} can be rewritten in their final form
\begin{subequations}
\begin{alignat}{3}
		A_1^2=& \frac{1}{2}\left(1+\hat{s}_m\cdot\hat{x}\right) =\frac{1}{2}\left(1+\sin\frac{\theta}{2}\right) ,\\
		A_2^2 = & \frac{1}{2}\left(1-\hat{s}_m\cdot\hat{x}\right) =\frac{1}{2}\left(1-\sin\frac{\theta}{2}\right).
\end{alignat} 
\end{subequations}

From \eqref{eq:Pes_inequality}, \eqref{eq:chi2}, we obtain the following integral expression for the pairwise symbol error probability
\begin{equation}
	{P}_{e|s|{\rm pair}}=\int_0^\infty Q_1\left( \frac{A_2}{\sigma},\frac{\sqrt{y_1}}{\sigma}\right)p_{y_1}(y_1)dy_1,
	\label{eq:pes_N2_integral}
\end{equation} 
where we used the Marcum Q function of the first order \eqref{eq:MarcumDef}.

By performing a transformation of random variables \cite{Benedetto1987}
\begin{equation}
	\xi_i = \sqrt{y_i},
\end{equation} 
we get the following Rician distributions
\begin{equation}
	p_{\xi_i}(\xi_i)=\frac{\xi}{2\sigma^2}\exp\left(-\frac{A_i^2+\xi_i^2}{2\sigma^2}\right)I_0\left(\frac{A_i\xi}{\sigma^2}\right).
\end{equation} 

Then, we can obtain a closed form for \eqref{eq:pes_N2_integral} \cite[Appendix~G]{helstrom2013}, \cite{Benedetto1987}
\begin{eqnarray}
	P_{e|s|pair}&=&P(y_2 > y_1)=P(\xi_2 > \xi_1)\nonumber\\&=& Q_1(\sqrt{a},\sqrt{b})-\frac{1}{2}\exp\left(-\frac{a+b}{2}\right)I_0\left(\sqrt{ab}\right),\notag\\&&
	\label{eq:proof_ub_N2}
\end{eqnarray} 
where we defined
\begin{subequations}
	    \begin{align}
		a:=& \frac{A_2^2}{2\sigma^2},\\
		b:=&  \frac{A_1^2}{2\sigma^2} .
		\end{align}
\end{subequations}

