\subsection{Transmission channel model}
We formulate the transmission channel by a discrete-time model \cite{Divsalar_TCOM_90}. All optical and electronic signals from now on are represented by their samples taken once per symbol.

After optical post-compensation, we assume that all transmission impairments are fully compensated. For instance, let ${\bf U} $ denote the unitary {Jones transfer matrix} of the optical fiber due to modal birefringence. We  assume that the unitary fiber transfer matrix  ${\bf U}$ is fully compensated, up to a random phase $\theta$,  by a  zero-forcing optical adaptive equalizer with transfer matrix ${\bf W} $ so that
 \begin{equation}
  {\bf W U} = e^{\iota \theta} {\bf I}_N,
 \end{equation}
where ${\bf I}_N$ denotes the $N \times N$ identity Jones matrix.

After the optical front-end, at the $0-$th sampling instant, in the absence of noise, the incoherent receiver recovers  the transmitted Jones vector $\ket{s_m}$ up to phase $\theta$, which one could denote by $e^{\iota \theta}\ket{s_m}$.
 
 
  Optical amplifiers introduce amplified spontaneous emission (ASE) noise, which is modelled as additive white Gaussian noise (AWGN). 
  
  The received vector at the $0-$th sampling instant before photodetection equals \cite{Divsalar_TCOM_90} 
 %{\bit Jones channel} formula  % {\bf Jones Channel} (AWGN unitary): % with DSP:   
  \begin{equation}
  \label{eq:AWGNJoneschannelBis}
    \ket{r} = A_{m} e^{\iota \phi _{m} } e^{\iota \theta} \ket{s_m} + \ket{n}, %\qquad ( \ket{r}, \ket{s_m}, \ket{n} \in \C^\d, \ m=1, \ldots , \mm). \notag 
 \end{equation}
\nomenclature[$rket$]{$\ket{r}$}{Received Jones vector, $\ket{r} = A_{m} e^{\iota \phi _{m} } e^{\iota \theta} \ket{s_m} + \ket{n}$}%
\nomenclature[$nket$]{$\ket{n}$}{Noise vector in Jones space}%
 where  $\theta$ is uniformly distributed over $[0, 2\pi)$ and $\ket{n}$ is a  noise vector in Jones space. Notice that $\ket{r},  \ket{n}$ are non-normalized Jones vectors, whereas $\ket{s_m}$ is a unit Jones vector.
 
 Assuming identical optical amplifiers at the output of all cores,  $\ket{n}$ has independent and identically distributed (i.i.d.) entries following a complex Gaussian distribution. The probability density function (pdf) of $\ket{n}$ is
 \begin{equation}
 \label{gaussianDistro:eq}
%p(\ket{n})=\frac{1}{(2 \pi \nvar)^\d} \exp\left(-\frac{\ip{n}^2}{2\nvar}\right),
\Pden(\ket{n})=\frac{1}{(2 \pi \nvar)^\d} \exp\left(-\frac{\ip{n}}{2\nvar}\right),
 \end{equation} 
where $\nvar$ denotes the noise variance per quadrature after the matched optical filter. 

The noise energy in each spatial degree of freedom is
\begin{equation}
  \mathcal{N}_0=  \frac{1}{2}\int_{-\infty}^\infty {\bf n}_{\nu} (t)^\dagger {\bf n}_{\nu} (t) dt =\nvar %.
\end{equation}
 where ${\bf n}_{\nu} (t)$ is the complex noise component in a single quadrature plane of the electric field (given by any $\ket{\nu} \in \C^N$). %{\tiny [JKtoJohn: fix to orthodox wording}}

%%%%%%%%%%%%%%%%%%%%%%%%%%%%%%%%%%%%%%%%%%%%%%%%%%%%%%%%%%%%%%%%%%%%%%%%%%%%%%%%%%%%%%%%%
The symbol SNR $\gamma_s$, taking into account the noise over a spatial degree of freedom of the signal $\ket{s_m}$, is defined as
 \begin{equation}
  \gamma_s := \frac{\mathcal{E}_s}{\mathcal{N}_0}=\frac{A_m^2}{2\nvar}\mathcal{E}_g.
 \end{equation}
 In the following, without any loss of generality, we assume that $A_m=1$ and $\mathcal{E}_g=1$, so %\textcolor{red}{{\tiny [JK: decided to display but this may be redundant}} %$\gamma_s=1/(2\nvar)$.
  \begin{equation}\label{gammasSigma:eq}
  \gamma_s=\frac{1}{2\nvar}.
 \end{equation}
%  %%%%%%%%%%%%%%%%%%%%%%%%%%%%%%%%%%%%%%%%%%%%%%%%%%%%%%%%%%%%%%%%%%%%%%%%%%%%%%%%%%%%%%%%%%%%%%%%%%%%%%%%%%
%  \textcolor{red}{ JK 12 Dec PROPOSED FIX OF THE GRAY ABOVE:
%  The total noise energy is $2N \mathcal{N}_0}$, so the symbol SNR is
%  \begin{equation}
%   \gamma_s := \frac{\mathcal{E}_s}{2N \mathcal{N}_0}=\frac{A_m}{2\mathcal{E}_g}}\frac{1}{2N \mathcal{N}_0}.
%  \end{equation}
%  In the following, without any loss of generality, we assume that  $\mathcal{E}_g=1$ and $A_m=2N$, 
% so that $\mathcal{E}_s=N$, which is to say that the average signal energy allocated to a degree of freedom is $1/2$. 
% [Note: Could have taken $A_m=2N$ to make it one.]
% In particular, the symbol SNR is simply 
%   \begin{equation}\label{gammasSigma:eq}
%   \gamma_s=\frac{1}{2\nvar}.
%  \end{equation}
%  NOW GRAPHS HAVE TO BE CORRECTED TO REFLECT $\mathcal{E}_s=N$.
%  } % end red 
%  %%%%%%%%%%%%%%%%%%%%%%%%%%%%%%%%%%%%%%%%%%%%%%%%%%%%%%%%%%%%%%%%%%%%%%%%%%%%%%%%%%%%%%%%%%%%%%%%%%%%%%%%%%
%\subsection{Dyad notation}
%\noindent

The received dyad ${\bf R}$ is related to the transmitted dyad ${\bf S}_m=\dyad{s_m}$ by the \emph{Stokes channel} formula:
 \begin{equation}
  \label{eq:AWGNStokesChannelIncoherent}
  % \dyad{r} = U \dyad{s_m}U^* +  e^{\iota \theta}U \ketbra{s_m}{n} + e^{-\iota \theta}\ketbra{n}{s_m}U^* + \dyad{n}
   \underset{{\bf R}}{\underbrace{\dyad{r}}} = \underset{{\bf S}_m}{\underbrace{\dyad{s_m}}} +   \underset{{\bf N}_m}{\underbrace{2\Re\left(\ketbra{s_m}{n} \right) + \dyad{n} }}.
 \end{equation}
 Observe that, in this formulation, the last two terms form a non-Gaussian %take the form of a {\it Ricean noise}
  noise ${\bf N}_m$ exhibiting   signal-noise and noise-noise beating %,  the transformed Gaussian noise:
 \begin{equation}
   \label{eq:StokesProjNoise}
   {\bf N}_m := 2\Re\left( \ketbra{s_m}{n} \right) + \dyad{n}.
   \end{equation}

   We  note  that Jones vectors, up to phase, are often represented by generalized real Stokes vectors $\sv$ in a higher-dimensional real vector space $\R^{\d^2-1}$. The latter arise as  coefficients with respect to a fixed Gell-Mann basis for \emph{trace neutralized dyad} ${\bf S}$, specifically, ${\bf S}-\frac{1}{\d}{\bf I}_N$ assuming normalization $\ip{s}=1$. We will use the Jones vector up to phase $e^{\iota \theta}\ket{r}$, the dyad  ${\bf R}=\dyad{r}$, and the Stokes vector $\rv$ interchangeably, depending on which one is more convenient.
   In particular, even when we use Jones vectors to express points of a constellation, we refer to it as a (generalized)  Stokes constellation when it is deployed in the incoherently received communication.

 
%\pagebreak

%\section{Symbol error probability} 
\subsection{Optimum decision criterion for equipower signals\label{sec:decision_criterion}} 

% \noindent The decision scheme  at the receiver uses the \emph{maximum a posteriori  probability} (MAP) criterion  \cite{Proakis} to select a signal $\sv_m$ out of the set of $M$ transmitted signals given that $\rv$ was received. 
% The conditional probability distribution of $\sv_m$ given $\rv$ is 
% \begin{equation}
%   \Pden(\sv_m \vert\rv) = \frac{P_m \Pden(\rv \vert\sv_m)}{\Pden(\rv)},
% \end{equation}
% where $P_m$ is the probability of sending $\sv_m$ (which we take as $P_m=\frac{1}{\mm}$),  $\Pden(\rv)$ is the marginal pdf of receiving $\rv$, and $\Pden(\rv \mid \sv_m)$ is the  likelihood pdf of receiving $\rv$ given that $\sv_m$ was sent.
% The latter is obtained   by integrating the pdf of the difference $e^{\iota \theta}\ket{r}-\ket{s_m}$ \cite{Proakis}:  
% %, using the gaussian distribution of the noise $\ket{n}=e^{\iota \theta}\ket{r}-\ket{s_m}$ in the Jones channel,
% % Jones picture the conditional probability $\Pden(\rv \ | \ \sv_m)$ is the probability that sending $\ket{s_m}$ produces $e^{\iota \theta}\ket{r}$ for some $\theta$. Using the gaussian distribution of the noise $\ket{n}=e^{\iota \theta}\ket{r}-\ket{s_m}$ we get \cite{Proakis}
%  %%%% end footnote l
% \begin{align}\label{mainrsProb:eq}
%   &\quad \Pden(\rv \vert \sv_m) \notag \\
%   &= \frac{1}{(2 \pi \nvar)^\d} \int_0^{2 \pi} \exp\left({-\frac{\| \ket{r} - e^{\iota \theta}\ket{s_m} \|^2}{2 \nvar}}\right) \, d\theta \notag \\
%      % &=    \int e^{-\frac{ \braket{r} - 2 \Re \left( e^{\iota \theta}\ip{r}{s_m} \right) + \braket{s_m} }{2 \nvar}} \, d\theta \notag \\
%      %   &=        e^{-\frac{\braket{s_m} }{2 \nvar}} 
%      %          \int e^{\frac{2 \Re \left( e^{\iota (\theta-\phi_m) } |\ip{r}{s_m}| \right)}{2 \nvar}} \, d\theta \notag \\
%      %   &=     e^{-\frac{\braket{s_m} }{2 \nvar}} 
%      %          \int e^{\frac{2 \cos (\theta-\phi_m)  |\ip{r}{s_m}| }{2 \nvar}} \, d\theta \notag \\
%       & =   \frac{1}{(2 \pi \nvar)^\d} \exp\left(-\frac{\braket{r} }{2 \nvar} -\frac{\braket{s_m} }{2 \nvar} \right)
%               I_0\left( \frac{|\ip{r}{s_m}| }{\nvar}  \right),
% \end{align}
%  where we used the \emph{modified Bessel function of the first kind} of zero order:
%   \begin{equation}
%     \label{eq:modBesselZero}
%     I_0(x):= % \frac{1}{2\pi} \int_0^{2 \pi} e^{x \cos \theta} \, d\theta =
%     \frac{1}{\pi} \int_0^{\pi} \exp\left({x \cos \theta}\right) \, d\theta.
%   \end{equation}
% %%%%%%%%%%%%%%%%%%%%%%%%%%%%%%%%%%%%%%%%%%%%%%%%%%%%%%%%%%%%%%%%%%%%%%%%%%%%%%%
% %JK WORK ZONE: 
% The integral in \eqref{mainrsProb:eq} reflects the fact that, for incoherent detection, we can model $\theta$ as
%  a random variable uniformly distributed between 0 and $2\pi$ and, for any fixed $\theta$, the difference $e^{\iota \theta}\ket{r}-\ket{s_m}$ is distributed like $\ket{n}$ (i.e., per \eqref{gaussianDistro:eq}). Hence, equality \eqref{mainrsProb:eq} is seen by expanding the exponent  (cf. \eqref{eq:AWGNStokesChannelIncoherent}): % Following \eqref{eq:AWGNStokesChannelIncoherent}, the exponent expands to 
% 	\begin{align*}
% 		&\quad {\frac{\| \ket{r} - e^{\iota \theta}\ket{s_m} \|^2}{2 \nvar}} \notag \\
% 		%&=  
% 		%\frac{ \braket{r} - 2 \Re \left( e^{\iota \theta}\ip{r}{s_m} \right) + \braket{s_m} }{2 \nvar} \notag \\
% 		%%&= \argmax_{1 \leq m \leq \mm} \   P_m  e^{-\frac{\braket{s_m} }{2 \nvar}} \textcolor{gray}{e^{-\frac{\braket{r} }{2 \nvar}}}
% 		%% \int e^{\frac{2 \Re \left( e^{\iota \theta}\ip{r}{s_m} \right)}{2 \nvar}} \, d\theta \notag \\
% 		&=  \frac{\braket{r} }{2 \nvar} + \frac{\braket{s_m} }{2 \nvar} -
% 		\frac{2 \Re \left( e^{\iota (\theta-\phi_m) } |\ip{r}{s_m}| \right)}{2 \nvar},    \addtag
% 	\end{align*}
% where $\phi_m$ denotes the argument of $\ip{r}{s_m}$ (which is immaterial). The integration domain reduces to $[0,\pi]$ due to the symmetry of cosine.  
% %STOP


% %END JK WORK ZONE 

% % %%%%%%%%%%%%%%%%%%%%%%%%%%%%%%%%%%%%%%%%%%%%%%%%%%%%%%%%%%%%%%%%%%%%%%%%%%%%%%%
% % \begin{proof}
% % Typically, for noncoherent detection,	we can model $\theta$ as
% % a random variable uniformly distributed between 0 and $2\pi$. By averaging over all values of $\theta$, we obtain
% % %%%%%%%%%%%%%%%%%%%%%%%%%%%%%%%%%%%%%%%%%%%%%%%%%%%%%%%%%%%%%%%%%%%%%%%%%%%%%%%%%%%%%%%%%%%%%%

% % %%%%%%%%%%%%%%%%%%%%%%%%%%%%%%%%%%%%%%%%%%%%%%%%%%%%%%%%%%%%%%%%%%%%%%%%%%%%%%%%%%%%%%%%%%%%%%%%
% % 	\begin{align*}
% % 		&\quad \int \exp\left({-\frac{\| \ket{r} - e^{\iota \theta}\ket{s_m} \|^2}{2 \nvar}}\right) \, d\theta \notag \\
% % 		&=  \int \exp \left({-\frac{ \braket{r} - 2 \Re \left( e^{\iota \theta}\ip{r}{s_m} \right) + \braket{s_m} }{2 \nvar}} \right) \, d\theta \notag \\
% % 		%&= \argmax_{1 \leq m \leq \mm} \   P_m  e^{-\frac{\braket{s_m} }{2 \nvar}} \textcolor{gray}{e^{-\frac{\braket{r} }{2 \nvar}}}
% % 		% \int e^{\frac{2 \Re \left( e^{\iota \theta}\ip{r}{s_m} \right)}{2 \nvar}} \, d\theta \notag \\
% % 		&=   \exp \left({-\frac{\braket{s_m} }{2 \nvar}} \right)
% % 		\int \exp \left({\frac{2 \Re \left( e^{\iota (\theta-\phi_m) } |\ip{r}{s_m}| \right)}{2 \nvar}}\right) \, d\theta \notag \\
% % 		&=  \exp\left({-\frac{\braket{s_m} }{2 \nvar}} \right) 
% % 		\int \exp \left( {\frac{2 \cos (\theta-\phi_m)  |\ip{r}{s_m}| }{2 \nvar}} \right) \, d\theta \notag \\
% % 		%&= \argmax_{1 \leq m \leq \mm} \   P_m  e^{-\frac{\braket{s_m} }{2 \nvar}}  \int e^{\frac{2 \cos (\theta-\phi_m)  |\ip{r}{s_m}| }{2 \nvar}} \, d\theta
% % 		&=   \exp\left({-\frac{\braket{s_m} }{2 \nvar}} \right)
% % 		I_0\left( \frac{|\ip{r}{s_m}| }{\nvar}  \right).   \addtag
% % 	\end{align*}
% % \end{proof}


% %%%%%%%%%%%%%%%%%%%%%%%%%%%%%%%%%%%%%%%%%%%%%%%%%%%%%%%%%%%%%%%%%%%%%%%%%%%%%%%
%   % \section{Maximum Likelihood Detection}
 


 
% %The optimal detection scheme, the \emph{maximum a posteriori probability rule} \cite{Proakis}, is %with the {\it Bayesian flip} is  
% Based on \eqref{mainrsProb:eq}, the MAP rule guesses the sent symbol as $\sv_{\hat{m}}$ with
% \begin{subequations}
% \begin{align}
%   %\label{eq:optScheme}
%   \hat{m} &:= \argmax_{1 \leq m \leq \mm} \Pden(\sv_m \vert \rv)\\
%   &= \argmax_{1 \leq m \leq \mm} \frac{P_m \Pden(\rv \vert \sv_m)}{\Pden(\rv)}\\
%   &= \argmax_{1 \leq m \leq \mm} \lvert \ip{r}{s_m} \rvert ,
%   \label{eq:optScheme}
% \end{align}
% \end{subequations}  
% where the last equality uses the monotonicity of $I_0(x)$ (and $\ip{s_m}=1$). 
% In particular,  in the Jones space, the %acceptance region 
%  decision region ${\mathcal D}_m \subset \C^N$ for $\ket{s_m}$ is %indeed the {\it Voroni cells} around the complex lines generated by the symbol vectors: 
% \begin{equation}
%   \label{eq:Voronoi}
%   {\mathcal D}_m = \left\{ \ket{r} : \   |\ip{r}{s_m}| \geq   |\ip{r}{s_{m'}}| \ \text{ for all $m' \neq m$}  \right\}.
% \end{equation}
% Note, in passing, that,  viewed in the Stokes space,  ${\mathcal D}_m$ is the  \emph{Voroni cell} around $\sv_m$, and this is so irrespective of which applicable distance,  $\dD$ or $\dStok$, is used (cf. Sec. \ref{sec:Pes}\ref{sec:distances}).
% % The associated {\bit decision regions} (collecting the received vectors $\rv$ from which the correct message $m$ is recovered) are 
% % \begin{equation}
% %   \label{eq:MAPdecisionRegions}
% %   D_m := \left\{ \rv \in \R^\d: \ m \neq m' \implies \Pden(\sv_m \ | \ \rv \ ) \geq \Pden(\sv_{m'} \ | \ \rv) \right\}.
% % \end{equation}

% The \emph{symbol error probability} $\Pe$ is the expected probability of missing the right region, expressed by the following sum of integrals with respect to the $2\d$-dimensional volume in $\C^\d$: 
% \begin{equation}
%   \label{eq:errProb}
%   \Pe = \sum_{m=1}^M P_m \sum_{m' \neq m} \int_{{\mathcal D}_{m'}}  \Pden(\rv \ | \ \sv_m) \, d \! \ket{r}. %d\rv.
%   %\Pden(\rv \ | \ \sv_m ) \, d\rv.
% \end{equation}
% In principle, $\Pe$ can be computed based on the channel model, but its analytic evaluation is intractable for all but the simplest constellations due to the complex geometry of the regions ${\mathcal D}_m$. Therefore, readily computable analytic bounds on $\Pe$ are of value. 



% %%%%%%%%%%%%%%%%%%%%%%%%%%%%%%%%%%%%%%%%%%%%%%%%%%%%%%%%%%%%%%%%%%%%%%%%%%%%%%%%%%%%%%%%%%%%%%%%%%%%%% 
% %%%%%%%%%%%%%%%%%%%%%%%%%%%%%%%%%%%%%%%%%%%%%%%%%%%%%%%%%%%%%%%%%%%%%%%%%%%%%%%%%%%%%%%%%%%%%%%%%%%%%%

% \subsection{Union Bound}


% \noindent The general form of the union bound is  \cite{Proakis}
% \begin{equation}
%   \label{eq:UBgeneralForm}
%   \Pe %= \sum_{m=1}^\mm P_m \sum_{m' \neq m}  \Pe^{m' | m}
%   \leq \sum_{m=1}^\mm P_m \sum_{m' \neq m}   \Pbin,
% \end{equation}
% where $ \Pbin$ is the pairwise probability of deciding on $\ket{s_{m'}}$ when $\ket{s_{m}}$ was sent in a binary decision (when no other symbols are considered). Therfore,  
% \begin{equation}
%   \label{eq:mPrimeUnionBoundBis}
%   %\Pbin =  \int_{D^{ m'| m}} \Pden(\rv \ | \ \sv_m ) \, d\rv.
% \Pbin =  \int_{{\mathcal D}^{ m'| m}} \Pden\left( \ \ket{r} \, \mid \, \ket{s_m} \, \right) \, d\ket{r}.
% \end{equation}
% Above,  the \emph{pairwise error decision region} 
% %\footnote{One can see that it is the Cartesian product of an exterior of a cone in 4D and a Cartesian space.}
% ${\mathcal D}^{ m'| m}$ is where 
% $\Pden( \rv  \ | \ \sv_{m'} \ ) \geq \Pden( \rv  \ | \ \sv_{m} \ )$, i.e., 
% % \begin{align}
% %   \label{eq:mmprimeRegionBis}
% %   % D^{ m'| m}&:= \left\{ \rv: \  \Pden( \rv  \ | \ \sv_{m'} \ ) \geq \Pden( \rv  \ | \ \sv_{m} \ ) \right\}.
% %                 \Pden( \rv  \ | \ \sv_{m'} \ ) \geq \Pden( \rv  \ | \ \sv_{m} \ ).
% % \end{align}
% % This used that $P_m=1$. Because also $\ip{s_m}=1$, a (skipped) argument hinging on  monotonicity of the Bessel function $I_0$ simplifies the inequality and yields 
% % % resolves to  $|\ip{r}{s_{m'}}| \geq |\ip{r}{s_{m}}|$
% \begin{align}
%   \label{eq:mmprimeRegionBis}
%                                             {\mathcal D}^{ m'| m} &= \left\{ \ket{r} \in \C^\d: \  |\ip{r}{s_{m'}}| \geq |\ip{r}{s_{m}}| \right\}.
% \end{align}
% The inequality of \eqref{eq:UBgeneralForm} follows from the manifest inclusion ${\mathcal D}_{m'} \subset {\mathcal D}^{ m'| m}$. 
% % Note that, although we chose to express $D^{ m'| m}$ as a subset of the Jones space, 
% %  it is invariant under phase rotations and thus defines a region worth of $\rv$ in the Stokes space.

\noindent The decision scheme  at the receiver uses the \emph{maximum a posteriori  probability} (MAP) criterion  \cite{Proakis} to select a signal $\sv_{\hat{m}}$ out of the set of $M$ transmitted signals given that $\rv$ was received
\begin{align}
  %\label{eq:optScheme}
  \hat{m} &:= \argmax_{1 \leq m \leq M} \Pden(\sv_m \ | \ \rv).
  \label{eq:MAP_criterion}
\end{align}

From Bayes' theorem, the conditional probability distribution  of $\sv_m$ given $\rv$ is 
\begin{equation}
  \Pden(\sv_m \ | \ \rv) = \frac{P_m \Pden(\rv \ | \ \sv_m)}{\Pden(\rv)},
  \label{eq:Bayes}
\end{equation}
where $P_m$ is the probability of sending $\sv_m$ ,  $\Pden(\rv)$ is the marginal pdf of receiving $\rv$, and $\Pden(\rv \ | \ \sv_m)$ is the  likelihood pdf of receiving $\rv$ given that $\sv_m$ was sent.

% By substituting (\ref{eq:Bayes}) into (\ref{eq:MAP_criterion}), we obtain
% \begin{subequations}
% \begin{align}
%   %\label{eq:optScheme}
%   \hat{m}   &= \argmax_{1 \leq m \leq M} \frac{P_m \Pden(\rv \ | \ \sv_m)}{\Pden(\rv)}\\&= \argmax_{1 \leq m \leq M} {P_m \Pden(\rv \ | \ \sv_m)},
%   \label{eq:MAP_criterion_2}
% \end{align}
% \end{subequations}

% In (\ref{eq:MAP_criterion_2}), we omitted the denominator $\Pden(\rv)$ because it is the same for all candidate signals $\sv_m$ and does not influence the decision.

By substituting \eqref{eq:Bayes} into \eqref{eq:MAP_criterion} and omitting the common denominator $\Pden(\rv)$ (which does not influence the decision) we obtain
%\begin{subequations}
\begin{align}
  %\label{eq:optScheme}
  \hat{m}   %&= \argmax_{1 \leq m \leq M} \frac{P_m \Pden(\rv \ | \ \sv_m)}{\Pden(\rv)}\\
  &= \argmax_{1 \leq m \leq M} {P_m \Pden(\rv \ | \ \sv_m)}.
  \label{eq:MAP_criterion_2}
\end{align}
%\end{subequations}

In this paper, we focus exclusively on equiprobable symbols, and, therefore, $P_m={1}/{M}.$ In this case, the   \emph{maximum a posteriori  probability} (MAP) criterion of \eqref{eq:MAP_criterion_2} becomes equivalent to the \emph{maximum likelihood} (ML) criterion  \cite{Proakis} 
\begin{align}
  \hat{m}   &= \argmax_{1 \leq m \leq M} {\Pden(\rv \ | \ \sv_m)}.
  \label{eq:ML_criterion}
\end{align}

The likelihood pdf in \eqref{eq:ML_criterion} is obtained by considering the Jones vectors corresponding to $\rv$ and $\sv_m$,  noting that $\ket{r}-e^{\iota \theta}\ket{s_m} = \ket{n}$ by \eqref{eq:AWGNJoneschannelBis}, and averaging the pdf \eqref{gaussianDistro:eq} 
 %of the difference 
 % $e^{\iota \theta}\ket{r}-\ket{s_m}$ %(see (\ref{eq:AWGNJoneschannelBis})) 
  over all $\theta$ \cite{Proakis}: % by using the Gaussian (\ref{gaussianDistro:eq}):  %{eq:AWGNJoneschannelBis}
%, using the gaussian distribution of the noise $\ket{n}=e^{\iota \theta}\ket{r}-\ket{s_m}$ in the Jones channel,
% Jones picture the conditional probability $\Pden(\rv \ | \ \sv_m)$ is the probability that sending $\ket{s_m}$ produces $e^{\iota \theta}\ket{r}$ for some $\theta$. Using the gaussian distribution of the noise $\ket{n}=e^{\iota \theta}\ket{r}-\ket{s_m}$ we get \cite{Proakis}
 %%%% end footnote l
%Since $\ket{r}-e^{\iota \theta}\ket{s_m} = \ket{n}$ by (\ref{eq:AWGNJoneschannelBis}) we use the Gaussian pdf (\ref{gaussianDistro:eq}):
\begin{align}\label{mainrsProb:eq}
  &\quad \Pden(\rv \ | \ \sv_m) \notag \\ 
  &= \frac{1}{(2 \pi \nvar)^\d} \int_0^{2 \pi} \exp\left({-\frac{\| \ket{r} - e^{\iota \theta}\ket{s_m} \|^2}{2 \nvar}}\right) \, \frac{d\theta}{2\pi} \notag \\
     % &=    \int e^{-\frac{ \braket{r} - 2 \Re \left( e^{\iota \theta}\ip{r}{s_m} \right) + \braket{s_m} }{2 \nvar}} \, d\theta \notag \\
     %   &=        e^{-\frac{\braket{s_m} }{2 \nvar}} 
     %          \int e^{\frac{2 \Re \left( e^{\iota (\theta-\phi_m) } |\ip{r}{s_m}| \right)}{2 \nvar}} \, d\theta \notag \\
     %   &=     e^{-\frac{\braket{s_m} }{2 \nvar}} 
     %          \int e^{\frac{2 \cos (\theta-\phi_m)  |\ip{r}{s_m}| }{2 \nvar}} \, d\theta \notag \\
      %& =   \frac{e^{-\frac{\braket{r}-\braket{s_m} }{2 \nvar}}}{(2 \pi \nvar)^\d} \\
      %\frac{1}{2 \nvar}}}{(2 \pi \nvar)^\d} e^{-\frac{\braket{r}-\braket{s_m} 
      & =   \frac{  1 }{(2 \pi \nvar)^\d}  \exp\left( -\frac{\braket{r}-\braket{s_m} }{2 \nvar} \right)  
              I_0\left( \frac{|\ip{r}{s_m}| }{\nvar}  \right),
\end{align}
 where we used the \emph{modified Bessel function of the first kind} of zero order
  \begin{equation}
    \label{eq:modBesselZero}
    I_0(x):= % \frac{1}{2\pi} \int_0^{2 \pi} e^{x \cos \theta} \, d\theta =
    \frac{1}{\pi} \int_0^{\pi} \exp\left({x \cos \theta}\right) \, d\theta.
  \end{equation}
%%%%%%%%%%%%%%%%%%%%%%%%%%%%%%%%%%%%%%%%%%%%%%%%%%%%%%%%%%%%%%%%%%%%%%%%%%%%%%%
%JK WORK ZONE: 
% The integral in \eqref{mainrsProb:eq} reflects the fact that, for incoherent detection, we can model $\theta$ as
%  a random variable uniformly distributed between 0 and $2\pi$ and, for any fixed $\theta$, the difference $e^{\iota \theta}\ket{r}-\ket{s_m}$ is distributed like $\ket{n}$ (i.e.,\ per \eqref{gaussianDistro:eq}). Hence, 
\noindent Equality \eqref{mainrsProb:eq} is obtained by expanding in the exponent  (cf. \eqref{eq:AWGNStokesChannelIncoherent}): % Following \eqref{eq:AWGNStokesChannelIncoherent}, the exponent expands to 
% 	\begin{align*}
% 		&\quad {\frac{\| \ket{r} - e^{\iota \theta}\ket{s_m} \|^2}{2 \nvar}} \notag \\
% 		%&=  
% 		%\frac{ \braket{r} - 2 \Re \left( e^{\iota \theta}\ip{r}{s_m} \right) + \braket{s_m} }{2 \nvar} \notag \\
% 		%%&= \argmax_{1 \leq m \leq \mm} \   P_m  e^{-\frac{\braket{s_m} }{2 \nvar}} \textcolor{gray}{e^{-\frac{\braket{r} }{2 \nvar}}}
% 		%% \int e^{\frac{2 \Re \left( e^{\iota \theta}\ip{r}{s_m} \right)}{2 \nvar}} \, d\theta \notag \\
% 		&=  \frac{\braket{r} }{2 \nvar} + \frac{\braket{s_m} }{2 \nvar} -
% 		\frac{2 \Re \left( e^{\iota (\theta-\theta_m) } |\ip{r}{s_m}| \right)}{2 \nvar},    \addtag
% 		\label{eq:likelihood_exponent}
% 	\end{align*}	
	\begin{align*}
		&\quad 
  {  \| \ket{r} - e^{\iota \theta}\ket{s_m} \|^2 } \notag \\
		& =  \braket{r}  + \braket{s_m}  -
		\underset{2 \cos(\theta-\theta_m) |\ip{r}{s_m}|}{\underbrace{2 \Re \left( e^{\iota (\theta-\theta_m) } |\ip{r}{s_m}| \right)}},    \addtag
		\label{eq:likelihood_exponent}
	\end{align*}	
where $\theta_m$ denotes the argument of $\ip{r}{s_m}$ (which is immaterial). The integration domain reduces to $[0,\pi]$ due to the symmetry of cosine.  
%STOP


%END JK WORK ZONE 

% %%%%%%%%%%%%%%%%%%%%%%%%%%%%%%%%%%%%%%%%%%%%%%%%%%%%%%%%%%%%%%%%%%%%%%%%%%%%%%%
% \begin{proof}
% Typically, for noncoherent detection,	we can model $\theta$ as
% a random variable uniformly distributed between 0 and $2\pi$. By averaging over all values of $\theta$, we obtain
% %%%%%%%%%%%%%%%%%%%%%%%%%%%%%%%%%%%%%%%%%%%%%%%%%%%%%%%%%%%%%%%%%%%%%%%%%%%%%%%%%%%%%%%%%%%%%%

% %%%%%%%%%%%%%%%%%%%%%%%%%%%%%%%%%%%%%%%%%%%%%%%%%%%%%%%%%%%%%%%%%%%%%%%%%%%%%%%%%%%%%%%%%%%%%%%%
% 	\begin{align*}
% 		&\quad \int \exp\left({-\frac{\| \ket{r} - e^{\iota \theta}\ket{s_m} \|^2}{2 \nvar}}\right) \, d\theta \notag \\
% 		&=  \int \exp \left({-\frac{ \braket{r} - 2 \Re \left( e^{\iota \theta}\ip{r}{s_m} \right) + \braket{s_m} }{2 \nvar}} \right) \, d\theta \notag \\
% 		%&= \argmax_{1 \leq m \leq \mm} \   P_m  e^{-\frac{\braket{s_m} }{2 \nvar}} \textcolor{gray}{e^{-\frac{\braket{r} }{2 \nvar}}}
% 		% \int e^{\frac{2 \Re \left( e^{\iota \theta}\ip{r}{s_m} \right)}{2 \nvar}} \, d\theta \notag \\
% 		&=   \exp \left({-\frac{\braket{s_m} }{2 \nvar}} \right)
% 		\int \exp \left({\frac{2 \Re \left( e^{\iota (\theta-\phi_m) } |\ip{r}{s_m}| \right)}{2 \nvar}}\right) \, d\theta \notag \\
% 		&=  \exp\left({-\frac{\braket{s_m} }{2 \nvar}} \right) 
% 		\int \exp \left( {\frac{2 \cos (\theta-\phi_m)  |\ip{r}{s_m}| }{2 \nvar}} \right) \, d\theta \notag \\
% 		%&= \argmax_{1 \leq m \leq \mm} \   P_m  e^{-\frac{\braket{s_m} }{2 \nvar}}  \int e^{\frac{2 \cos (\theta-\phi_m)  |\ip{r}{s_m}| }{2 \nvar}} \, d\theta
% 		&=   \exp\left({-\frac{\braket{s_m} }{2 \nvar}} \right)
% 		I_0\left( \frac{|\ip{r}{s_m}| }{\nvar}  \right).   \addtag
% 	\end{align*}
% \end{proof}


%%%%%%%%%%%%%%%%%%%%%%%%%%%%%%%%%%%%%%%%%%%%%%%%%%%%%%%%%%%%%%%%%%%%%%%%%%%%%%%
  % \section{Maximum Likelihood Detection}
 


%\noindent 
%The optimal detection scheme, the \emph{maximum a posteriori probability rule} \cite{Proakis}, is %with the {\it Bayesian flip} is  

Based on  \eqref{mainrsProb:eq}, we can rewrite the {maximum likelihood} (ML) criterion \eqref{eq:ML_criterion} as \cite{McCloud}
%\begin{subequations}
\begin{align}
  %\label{eq:optScheme}
  \hat{m} 
  &= \argmax_{1 \leq m \leq M} |\ip{r}{s_m}|,
  \label{eq:optScheme}
\end{align}
%\end{subequations}  
where we used the monotonicity of $I_0(x)$ and the fact that $\ip{s_m}=1$. 

In particular,  in the Jones space, the %acceptance region 
 ML decision region ${\mathcal D}_m \subset \C^N$ for $\ket{s_m}$ is %indeed the {\it Voroni cells} around the complex lines generated by the symbol vectors: 
\begin{equation}
  \label{eq:Voronoi}
  {\mathcal D}_m = \left\{ \ket{r}: \   |\ip{r}{s_m}| \geq   |\ip{r}{s_{m'}}| \ \text{ for all $m' \neq m$}  \right\}.
\end{equation}

In passing, note that ${\mathcal D}_m$ viewed in the Stokes space is the  \emph{Voronoi cell} around $\sv_m$, and this is so irrespective of which applicable concept of distance,  $\dD$ or $\dStok$, is used (cf. Sec. \ref{sec:distances}).
% The associated {\bit decision regions} (collecting the received vectors $\rv$ from which the correct message $m$ is recovered) are 
% \begin{equation}
%   \label{eq:MAPdecisionRegions}
%   D_m := \left\{ \rv \in \R^\d: \ m \neq m' \implies \Pden(\sv_m \ | \ \rv \ ) \geq \Pden(\sv_{m'} \ | \ \rv) \right\}.
% \end{equation}

The \emph{symbol error probability} is the expected probability of missing the right ML decision region, expressed by the following sum of integrals%\footnote{\textcolor{red}{John: In the integral, as usual, $\rv$ is the Stokes vector associated to Jones vector $\ket{r}$. And  I used $d \! \ket{r}$ for the $2\d$-dimensional volume element in $\C^\d$, if that is fine.}} 
with respect to the $2\d$-dimensional volume %element $d \! \ket{r}$
 in $\C^\d$: 
\begin{equation}
  \label{eq:errProb}
  \Pe = \sum_{m=1}^M P_m \sum_{m' \neq m} \int_{{\mathcal D}_{m'}}  \Pden(\rv \ | \ \sv_m) \, d \! \ket{r}. %d\rv.
  %\Pden(\rv \ | \ \sv_m ) \, d\rv.
\end{equation}

In principle, $\Pe$ can be computed based on the channel model, but its analytic evaluation is impossible for all but the simplest constellations due to the complex geometry of the ML decision regions ${\mathcal D}_m$. Therefore, readily  computable analytic bounds on $\Pe$ are of value. 



%%%%%%%%%%%%%%%%%%%%%%%%%%%%%%%%%%%%%%%%%%%%%%%%%%%%%%%%%%%%%%%%%%%%%%%%%%%%%%%%%%%%%%%%%%%%%%%%%%%%%% 
%%%%%%%%%%%%%%%%%%%%%%%%%%%%%%%%%%%%%%%%%%%%%%%%%%%%%%%%%%%%%%%%%%%%%%%%%%%%%%%%%%%%%%%%%%%%%%%%%%%%%%

\subsection{Union Bound}


\noindent The general form of the union bound is  \cite{Proakis}
\begin{equation}
  \label{eq:UBgeneralForm}
  \Pe %= \sum_{m=1}^\mm P_m \sum_{m' \neq m}  \Pe^{m' | m}
  \leq \sum_{m=1}^M P_m \sum_{m' \neq m}   \Peb^{m' | m} \ ,
\end{equation}
where $ \Peb^{m' | m}$ is the pairwise error probability of deciding on $\ket{s_{m'}}$ when $\ket{s_{m}}$ was sent in a binary fashion, while no other symbols are considered. Therefore,  
\begin{equation}
  \label{eq:mPrimeUnionBoundBis}
  %\Peb^{m' | m} =  \int_{D^{ m'| m}} \Pden(\rv \ | \ \sv_m ) \, d\rv.
\Peb^{m' | m} =  \int_{{\mathcal D}^{ m'| m}} \Pden(\rv \ | \ \sv_m) \, d \! \ket{r}. %d\rv.
\end{equation}
Above,  the \emph{pairwise error decision region}
%\footnote{One can see that it is the Cartesian product of an exterior of a cone in 4D and a Cartesian space.} 
${\mathcal D}^{ m'| m}$ is where 
$P_{m'} \Pden( \rv  \ | \ \sv_{m'} \ ) \geq P_{m} \Pden( \rv  \ | \ \sv_{m} \ )$, i.e., for equiprobable symbols,
% \begin{align}
%   \label{eq:mmprimeRegionBis}
%   % D^{ m'| m}&:= \left\{ \rv: \  \Pden( \rv  \ | \ \sv_{m'} \ ) \geq \Pden( \rv  \ | \ \sv_{m} \ ) \right\}.
%                 \Pden( \rv  \ | \ \sv_{m'} \ ) \geq \Pden( \rv  \ | \ \sv_{m} \ ).
% \end{align}
% This used that $P_m=1$. Because also $\ip{s_m}=1$, a (skipped) argument hinging on  monotonicity of the Bessel function $I_0$ simplifies the inequality and yields 
% % resolves to  $|\ip{r}{s_{m'}}| \geq |\ip{r}{s_{m}}|$
\begin{align}
  \label{eq:mmprimeRegionBis}
                                            {\mathcal D}^{ m'| m} &= \left\{ \ket{r} \in \C^\d: \  |\ip{r}{s_{m'}}| \geq |\ip{r}{s_{m}}| \right\}.
\end{align}
Inequality \eqref{eq:UBgeneralForm} follows from the manifest inclusion ${\mathcal D}_{m'} \subset {\mathcal D}^{ m'| m}$. 
% Note that, although we chose to express $D^{ m'| m}$ as a subset of the Jones space, 
%  it is invariant under phase rotations and thus defines a region worth of $\rv$ in the Stokes space.

Our main result gives an explicit form for the terms in the union bound.

\begin{tcolorbox}
%\begin{framed}
 \begin{thm}\label{main theorem Pbin}
   The binary error probability between two equiprobable non-orthogonal unit vectors $\ket{s_m}, \ket{s_{m'}} \in \C^\d$ is
   %subjected to AWG noise with variance $\nvar$ (per dimension) is given by  
\begin{empheq}[box=\fbox]{align}
     \label{mainExact:eq}
  \Peb^{ m'| m} 
  &= Q_1\left( \sqrt{\gamma_s}{\rho_-},\sqrt{\gamma_s}{\rho_+} \right) \notag \\
  &\quad -\frac{1}{2} \exp\left({-\frac{\gamma_s}{2}}\right) I_0\left(\frac{\gamma_s\gamma}{2}\right),
\end{empheq}
   where we defined the length parameters
\begin{subequations}
	\begin{align}
 \rho^2_\mp &:=\frac{1 \mp \delta}{2},\\ \delta &:= \sqrt{1 - \gamma^2},\\ \gamma & := |\ip{s_m}{s_{m'}}|>0,
\end{align}
\end{subequations}
    \nomenclature[$rho$]{$\rho^2_\mp$}{Length parameter, $\rho^2_\mp:= \frac{1\mp \delta}{2}$}
    \nomenclature[$delta$]{$\delta$}{Length parameter, $\delta:= \sqrt{1-\gamma^2}$}
%\begin{eqnarray}
% \rho^2_\mp &:=&\frac{1 \mp \delta}{2},\\ \delta &:=& \sqrt{1 - \gamma^2},\\ \gamma & :=& |\ip{s_m}{s_{m'}}|>0,
%  \end{eqnarray} 
 and $Q_1$ stands for the \emph{Marcum Q-function} of the first order defined by \cite{Proakis}
 \begin{equation}
   \label{eq:MarcumDef}
   Q_1(\aa, \bb) := \int_{\bb}^\infty x \exp\left({-\frac{x^2+\aa^2}{2}}\right) I_0(\aa x) \, dx.
 \end{equation}
\end{thm}
%\end{framed}
\end{tcolorbox}
The formulas extend to case of orthogonal signals, when $\gamma=0$  and one can fall back onto $Q_1(0, \bb)=\exp\left({-\frac{\bb^2}{2}}\right)$. To improve readability, we leave the proof of Theorem \ref{main theorem Pbin} to Appendix \ref{app:mainProofPes}.
%It is worth noting that the difference $\rho^2_2-\rho^2_1$ coincides with the Hilbert-Schmidt distance$\dHS(s,s') = \sqrt{2}\sqrt{1 - \gamma^2}$ (up to scaling by $\sqrt{2}$).
%(The incoherent metric  $\dD(s,s') = \sqrt{2}\sqrt{1 - \gamma}$ is not prominently displayed.)
% \textcolor{blue}{Comment/cite on the case when $\gamma=0$. Then $Q_1(0,1/\sqrt{2 \nvar})$ comes up, which is undefined per se and one should mention the limit. The non-central chi-squared/Rice distributions become central and things are more classical.}

For ease of reference, we instantiate \eqref{eq:UBgeneralForm} with \eqref{mainExact:eq} and record the following corollary. 
\begin{tcolorbox}
\begin{cor}
\label{cor:UBsumExact}
Given a  constellation of equiprobable unit vectors $(\ket{s_i})_{i=1}^M \in \C^N$ % no two of which are mutually orthogonal
 the symbol error probability $\Pe$ is bounded as 
\begin{empheq}[box=\fbox]{align}
    \label{UBsumExact}
    \Pe &
    \leq \frac{1}{M} \sum_{m=1}^M \sum_{m' \neq m}  \Big[ Q_1\left( \sqrt{\gamma_s}{\rho_-},\sqrt{\gamma_s}{\rho_+} \right) \notag\\
     &\quad -\frac{1}{2} \exp\left(-\frac{\gamma_s}{2}\right) I_0\left(\frac{\gamma_s\gamma}{2}\right)\Big]. 
\end{empheq}
\end{cor}
\end{tcolorbox}


Below, we present asymptotic expressions that are valid for larger values of the symbol SNR $\gamma_s$ and are obtained by implementing the results reported in \cite{GilSeguraTemme2014}  (proven in \cite{Temme1993} and based on \cite{Temme1986, Goldstein1953}). The derivation of these formulas is given in Appendix  \ref{app:Asymptotics}.
\begin{tcolorbox}
\begin{cor}
  \label{asympt:cor}
  For large values of the symbol SNR $\gamma_s$, when $\gamma_s \rho_+\rho_- = \gamma_s \gamma/2$ is sufficiently large, we can use asymptotic expansions for $\Peb^{m' | m}$ and the zeroth and first order approximations are as follows: 
%   \textcolor{red}{RE-CHECK and decide $\nvar$ or $\gamma_s$ at the ``center stage''}
%%%%%%%%%%%%%%%%%%%%%%%%%%%%%%%%%%%%% NEW: %%%%%%%%%%%%%%%%%%%%%%%%%%%%%%%%%%%%%%%%%%%%%%%%%%%%%%%%%%%%%%
\begin{empheq}[box=\fbox]{align} \label{mainAsymptZero:eq}
    \Pbin 
    &\sim   \frac{1}{2} \sqrt{\frac{\gamma}{1-\delta}}        
         \erfc\left(\frac{\sqrt{\gamma_s}\sqrt{1-\gamma}}{\sqrt{2}} \right) \notag \\
    &\quad  - \frac{1}{2\sqrt{\pi \gamma \gamma_s}} \exp\left( {-\frac{\gamma_s(1-\gamma)}{2}}\right)
\end{empheq}
%  %%%%%%%%%%%%%%%%%%%%%%%%%%%%%%%%%%%%% OLD: %%%%%%%%%%%%%%%%%%%%%%%%%%%%%%%%%%%%%%%%%%%%%%%%%%%%%%%%%%%%%%
%  \begin{empheq}[box=\widefbox]{align} \label{mainAsymptZero:eq}
%  &~ \Pbin \notag\\
%   &\sim \frac{1}{2} \left\{ \sqrt{\frac{1+\delta}{\gamma}} \ \erfc\left(\frac{\sqrt{\gamma_s}\sqrt{1-\gamma}}{\sqrt{2}} \right)\right.\nonumber \\& \left.- e^{-\frac{\gamma_s(1-\gamma)}{2}}\frac{1}{\sqrt{2\pi \gamma \gamma_s}} \right\}
%  \end{empheq}
%   % \begin{align}
%   %   \Pbin
%   %   &\sim \frac{1}{2} \left\{ \sqrt{\frac{\bb}{\aa}} \ \erfc\left(\frac{\bb-\aa}{\sqrt{2}} \right) - e^{-\frac{(\bb-\aa)^2}{2}}\frac{1}{\sqrt{2\pi\aa \bb}} \right\}.
%   %  \end{align}
and
  %%%%%%%%%%%%%%%%%%%%%%%%%%%%%%%%%%%%% NEW: %%%%%%%%%%%%%%%%%%%%%%%%%%%%%%%%%%%%%%%%%%%%%%%%%%%%%%%%%%%%%%     
\begin{empheq}[box=\fbox]{align} 
  \label{mainAsympt:eq}
    &\quad \Pbin \notag \\
    &\sim   \left[ \frac{1}{2} \sqrt{\frac{\gamma}{1-\delta}}
               - \sqrt{\frac{1-\gamma}{\gamma}}  \frac{1}{8\sqrt{2}}\left(\frac{1+\delta}{\gamma} + 3 \right) \right]    \notag \\ 
    &\quad  \times \erfc\left(\frac{\sqrt{\gamma_s}\sqrt{1-\gamma}}{\sqrt{2}} \right) \notag \\
    &\quad + \frac{1}{8\sqrt{\pi}}\left[\sqrt{2} \sqrt{\frac{1-\gamma}{1-\delta}}\left(\gamma \gamma_s\right)^{-\frac{1}{2}}   - \left(\gamma \gamma_s\right)^{-\frac{3}{2}}   \right]  \notag \\
    &\quad  \times \exp\left({-\frac{\gamma_s(1-\gamma)}{2}} \right) . 
\end{empheq}
%%%%%%%%%%%%%%%%%%%%%%%%%%%%%%%%%%%%% OLD: %%%%%%%%%%%%%%%%%%%%%%%%%%%%%%%%%%%%%%%%%%%%%%%%%%%%%%%%%%%%%%
% \begin{empheq}[box=\widefbox]{align} 
%   \label{mainAsympt:eq}
%   &~ \Pbin \notag \\
%   &\sim  \frac{1}{8}
%      \left(\frac{1+\delta}{\gamma} -1  \right) \frac{1}{\sqrt{2 \pi}\sqrt{\gamma \gamma_s}} e^{-\frac{\gamma_s(1-\gamma)}{2}} \notag \\
%     &+ \frac{1}{16}
%      \left( 3\frac{\gamma}{1+\delta}+6-\frac{1+\delta}{\gamma}  \right) \times\nonumber\\&
%      \ \sqrt{\frac{1+\delta}{\gamma}} \ \erfc\left(\frac{\sqrt{\gamma_s}\sqrt{1-\gamma}}{\sqrt{2}}\right).
% \end{empheq}
% Both formulas are valid when $\gamma_s \rho_+\rho_- = \gamma_s \gamma/2$ is sufficiently large. % and are then 
% valid with uniform error bounds for all $\gamma \in (0,1)$.\footnote{\textcolor{red}{I have yet to reproduce this uniformity proof.}}
\end{cor}
\end{tcolorbox}


      
We note that the particular value of the asymptotic expressions in the corollary above is their ability to handle poorly separated vectors (with $\gamma \approx 1$). Pairs of vectors with small separation contribute the bulk of the $\Pe$.  Moreover, MVM constellations with  a given spectral efficiency per spatial degree of freedom (e.g., analogous to QPSK) have diminishing minimal distance with the increase of the number of cores/modes.
When $\gamma$ is not too close to $1$ and SNR is large, we have a simpler%\footnote{When $\gamma \approx 1$, we can use $\frac{1+\delta}{\gamma} = \frac{\sqrt{1+\gamma}+\sqrt{1-\gamma}}{\gamma}\sqrt{1-\gamma}$. When $\gamma \approx 0$, $\frac{1+\delta}{\gamma}$ explodes ... }
asymptotic expression (with a straight forward derivation given in Appendix \ref{app:Asymptotics}): %\textcolor{red}{CHECKED OK 23 Dec 2021 by JK}
\begin{align}\label{mainAsymptOld:eq}
\boxed{     \Peb^{ m'| m}
  \sim \frac{1}{2} \frac{1}{\sqrt{\pi}}
  % \frac{\sqrt{1+\gamma}}{\sqrt{1-\gamma}}
  \sqrt{\frac{1+\gamma}{1-\gamma}}
  \frac{1}{\sqrt{\gamma}\sqrt{\gamma_s}} \exp \left[{-\frac{\gamma_s (1 - \gamma)}{2}}\right] .}
\end{align}

In any case,  the leading exponential asymptotics is
\begin{align}
\exp\left[-\frac{\gamma_s (1 - \gamma)}{2}\right] = \exp\left[-\frac{1}{2} \gamma_s \frac{\dD(\sv_m, \sv_{m'})^2}{2} \right],
\label{eq:asymptoticBehavior}
\end{align}
where $\dD(\sv_m, \sv_{m'})$ is the incoherent distance between $\sv_m$ and $\sv_{m'}$ (as defined in Sec. \ref{sec:distances}, ahead). This indicates that the distance $\dD$ is a natural way of expressing the proximity of the symbols in our context. 
%This indicates that the distance $\dD$ is appropriate 
%the most appropriate performance metric in this context (see distance definitions in Sec. \ref{sec:distances}).
%\textcolor{red}{The dependence of the non-exponential power pre-factor on $\gamma$ has to be discussed once we are sure the above is correct.} 
%%%%%%%%%%%%%%%%%%%%%%%%%%%%%%%%%%%%%%%%%%%%%%%%%%%%%%%%%%%%%%%%%%%%%%%%%%%%%%%%%%%%%%%%%%%%%%%%%%%%%%%%%%%%%%%%%%%%

% \textcolor{red}{COMMENT OUT IN FINAL PASS: NEW TEMP FOR DISCUSSION PURPOSES ONLY 13 Dec 2021 of ONE application of the rough asymptotics: Using the definition of $\gamma_s := \frac{\mathcal{E}_s}{\mathcal{N}_0}$ the exponent in equation  (\ref{eq:asymptoticBehavior}) rewrites as 
% \begin{align}
%   \frac{1}{2}\frac{\mathcal{E}_s}{\mathcal{N}_0} \frac{\dD^2}{2} 
%   = \frac{1}{2}\frac{\mathcal{E}_b}{\mathcal{N}_0} \log_2 M \frac{\dD^2}{2}
% %\label{eq:asymptoticBehaviorBis}
% \end{align}
% where $\mathcal{E}_b$ is the energy per bit. 
% Following Aggrell's methodology, after breaking out the bit SNR,  $\frac{\mathcal{E}_b}{\mathcal{N}_0}$, we find  that the asymptotic power efficiency of a constellation is
% \begin{equation}
%   \gamma_{A, \text{dd}} := \log_2 M \frac{\dD_{\text{\rm min}}^2}{2}
% \end{equation}
% where $\dD_{\text{\rm min}}$ is the minimum distance separation between points of the constellation. 
% }

%%%%%%%%%%%%%%%%%%%%%%%%%%%%%%%%%%%%%%%%%%%%%%%%%%%%%%%%%%%%%%%%%%%%%%%%%%%%%%%%%%%%%%%%%%%%%%%%%%%%%%%%%%%%%%%%%%%%%
% We add that, if $\gamma=0$, \eqref{mainAsymptZero:eq}-\eqref{mainAsymptOld:eq} are not valid; however, %in \eqref{mainExact:eq}, the term
%  then $\Pbin$ equals $\frac{1}{2}e^{-\gamma_s /2}$ and thus is eclipsed by the terms with $\gamma >0$ in the sum giving the union bound (\ref{eq:UBgeneralForm}).  Although, (\ref{mainAsymptZero:eq}) and  (\ref{mainAsympt:eq}) work well even for moderately small values of $\gamma$ one can safely drop the terms with the smaller $\gamma \approx 0$.%[QUANTIFY]   

We add that, if $\gamma=0$, \eqref{mainAsymptZero:eq}--\eqref{mainAsymptOld:eq} are not valid. However, then $\Peb^{m' | m}$ equals $\frac{1}{2}\exp(-\gamma_s /2)$ and is eclipsed by the terms with $\gamma >0$ in the sum giving the union bound \eqref{eq:UBgeneralForm}.  Even though \eqref{mainAsymptZero:eq} and  \eqref{mainAsympt:eq} work well even for moderately small values of $\gamma$, one can safely drop the terms with the smaller $\gamma \approx 0$.
 
Appendices \ref{app:mainProofPes} and \ref{app:Asymptotics} are devoted to the proofs of the Theorem \ref{main theorem Pbin} and the Corollary \ref{asympt:cor}, respectively. Sec. \ref{sec:results} shows comparisons of the union bound for  $\Pe$ obtained by using the above theoretical approximations and numerically computed Monte Carlo based values of $\Pe$ for several example constellations.

% \textcolor{gray}{
% \begin{cor}[OLD]
%   For sufficiently large values of SNR parameter $\gamma_s=\frac{1}{2\nvar}$ we have asymptotic formula
% \begin{align}\label{mainAsymptOldBis:eq}
%      \Peb^{ m'| m}
%   \sim \frac{1}{2} \frac{1}{\sqrt{2\pi}}
%   % \frac{\sqrt{1+\gamma}}{\sqrt{1-\gamma}}
%   \sqrt{\frac{1+\gamma}{1-\gamma}}
%   \frac{1}{\sqrt{\gamma}\sqrt{\gamma_s}} e^{-\frac{\gamma_s (1 - \gamma)}{2}}.
% \end{align}
% The error of the expansion is of order $e^{-\frac{\gamma_s (1 - \gamma)}{2}} O\left( \frac{1}{\gamma \gamma_s}\right)$ where the $O\left( \frac{1}{\gamma \gamma_s}\right)$ factor can be rigorously bounded...
% \end{cor}
% }



%%%%%%%%%%%%%%%%%%%%%%%%%%%%%%%%%%%%%%%%%%%%%%%%%%%%%%%%%%%%%%%%%%%%%%%%%%%%%%%%%%%%%%%

% \subsection{June 16: Digression for Eric: True Gradient (for the descent)}

% One can perform the gradient descent for the potential given by the formula (\ref{mainExact:eq}) for $\Peb$. The gradient is readily evaluated by using the formulas  
%   \begin{align}
%     \frac{\partial Q_1(\aa,\bb)}{\partial \bb} =  \bb e^{-\frac{\bb^2+\aa^2}{2}}I_0(\aa \bb)   \quad \text{ and } \quad
%      \frac{\partial Q_1(\aa,\bb)}{\partial \aa} = \aa \left( Q_2(\aa,\bb) - Q_1(\aa,\bb) \right)
%   \end{align}
%   where the latter comes from differentiating under the integral and using  $I_0'(x)=I_1(x)$: 
%  \begin{align}
%   \frac{\partial Q_1(\aa,\bb)}{\partial \aa}
%   &=  \int_{\bb}^\infty x^2 e^{-\frac{x^2+\aa^2}{2}} I_1(\aa x) \, dx
%      - \int_{\bb}^\infty x \aa e^{-\frac{x^2+\aa^2}{2}} I_0(\aa x) \, dx \\
%   &= \aa \left( Q_2(\aa,\bb) - Q_1(\aa,\bb) \right).
%   \end{align}
%   Above, $Q_2$ is the second Marcum function
%   $Q_2(x) = \int_{\bb}^\infty \frac{x^2}{\aa} e^{-\frac{x^2+\aa^2}{2}} I_1(\aa x) \, dx$. 
% Now Eric can program the gradient...
  
%   \bigskip

%%%%%%%%%%%%%%%%%%%%%%%%%%%%%%%%%%%%%%%%%%%%%%%%%%%%%%%%%%%%%%%%%%%%%%%%%%%%%%%%%%%%%%%%%%
%%%%%%%%%%%%%%%%%%%%%%%%%%%%%%%%%%%%%%%%%%%%%%%%%%%%%%%%%%%%%%%%%%%%%%%%%%%%%%%%%%%%%%%%%%%%%%%%%%%%%%%%%%%%%


\subsection{Distance definitions}\label{sec:distances}

\noindent For the optimal geometric shaping of an MVM constellation, which is discussed in Sec. \ref{sec:GeometricShaping},  it is necessary to adopt some function of distance between constellation points.
The choice of distance function depends on the detection scheme and the nature of the dominant channel impairments. For the back-to-back performance evaluation of  optically-preamplified MVM direct-detection receivers, we consider that ASE noise is the dominant impairment. In this case, for equienergetic MVM constellations,  the suitable metric turns out to be the \emph{chordal Fubini-Study distance}, which is a special case of what we call below \emph{incoherent/direct-detection distance}.
We note that there are several arguments advocating naturality of this metric. Perhaps the strongest is based on the way it enters the previously derived asymptotic form of the union bound \eqref{eq:asymptoticBehavior} for the symbol error probability.  

One quick takeaway is that there is a better distance than the ordinary Euclidean Stokes distance, which is often the default choice and was also initially used in our computations.  Below, we define the \emph{incoherent/direct detection distance} and relate it to other common distance functions. 

\subsubsection{Coherent Distance }
\noindent In  Jones space $\C^\d$, we have the standard Euclidean distance between MVM symbols, which can be written as a function of the Hermitian inner product 
\begin{align}
	\label{eq:EuclidJOnesDist}
	\dC( \ket{s}, \ket{s'})&:=
	\|\ket{s}-\ket{s'}\| \notag\\
	&=\sqrt{ \ip{s}- 2\Re \ip{s}{s'}  + \ip{s'}}.
\end{align}
For unit vectors, \eqref{eq:EuclidJOnesDist} is  expressed in terms of the \emph{coherent angle}
$\thetaC \in [0, \pi]$ as 
\begin{equation}\label{eq:EuclidJOnesDistUnit}
	\dC( \ket{s}, \ket{s'})=\sqrt{2}\sqrt{1 - \cos \thetaC}, 
\end{equation}
where $\cos \thetaC :=\Re \ip{s}{s'}$.
% \footnote{
%   In general we have Law of cosines expression 
% \begin{equation}\label{eq:EuclidJOnesDist}
%   \dC(s,r):=\|\ket{r}-\ket{s}\|
%   % =\sqrt{ \ip{s}- 2\frac{ \Re \ip{s}{r} }{ \sqrt{\ip{s}\ip{r}}}\sqrt{\ip{s}\ip{r}}  }  + \ip{r} }.
%   % =\sqrt{ \ip{s}- 2 \sqrt{\ip{s}\ip{r}}   \cos \thetaC   + \ip{r} }.
%   =\sqrt{ \|s\|^2 - 2 \|s\|\|r\|  \cos \thetaC   + \|r\|^2 }.
% \end{equation}}
%The coherent distance reflects

We  refer to this distance as \emph{coherent distance}, since the probability of making a binary error between $\ket{s}$ and $\ket{s'}$ in a coherent receiver depends on $\dC$  in a natural way.% (i.e., roughly like $e^{-\dC^2}$ although we do not go into detail here now.)  

\subsubsection{Incoherent Distance}

\noindent In the case of incoherent detection, a transmitted MVM symbol is abstractly represented by a Jones vector up to phase, $e^{\iota \theta}\ket{s}$, with indeterminate $\theta \in [0, 2\pi)$. Mathematically, as long as $\ket{s}\neq0$, this is a circle in Jones space $\C^\d$.    
From this standpoint, one might guess that the natural distance between symbols $s$ and $s'$ is the minimum coherent Jones distance between the two circles:
\begin{align}
	\dD( \ket{s}, \ket{s'})&:= \min_{\theta,\theta'} \left\vert e^{\iota \theta}\ket{s} -  e^{\iota \theta'} \ket{s'} \right\vert \notag\\
	&= \sqrt{\|s\|^2 - 2 \left| \ip{s}{s'} \right|  + \|s'\|^2}.
\end{align}
When $\|s\|=\|s'\|=1$, we can use the \emph{incoherent/direct detection angle} $\thetaD \in [0,\pi/2]$: 
\begin{equation}\label{eq:ddDistUnit}
	\dD( \ket{s}, \ket{s'})=\sqrt{2}\sqrt{1 - \gamma},
\end{equation}
where $\gamma:= \cos \thetaD:= |\ip{s}{s'}|$. 

In this case, the distance  coincides with the {\it chordal Fubini-Study distance}%\footnote{And is a special case of {\it Bures distance}} 
on the complex projective space.
In Appendices \ref{app:mainProofPes} and \ref{app:Asymptotics},  the incoherent cosine $\gamma = \cos \thetaD = |\ip{s}{s'}|$ of two normalized symbols under consideration  will make frequent appearance.

Of course, if only from $\Re \ip{s}{s'} \leq \vert \ip{s}{s'}\vert$, we have
\begin{equation}
	\dD(\ket{s}, \ket{s'}) \leq \dC(\ket{s}, \ket{s'}) \quad \text{ and } \quad  \thetaD \leq \thetaC.
\end{equation}
The loss of phase information degrades one's ability to distinguish symbols.

\subsubsection{Hilbert-Schmidt and Stokes Distance}
%Traditionally incoherently received symbols are represented by Stokes vectors $\sv \in \R^{\d^2-1}$.

%Note that by eschewing the normalization we are not in the projective space but in a sligtly larger space of vectors in $\C^\d$ modulo a phase rotation. We call it the {\bit extended projective space} [CHECK terminology]. Its elements can be identified with dyads $\dyad{s}$, which are $\d \times \d$ complex matrices.

\noindent Another way to represent  incoherently-received symbols is with dyads ${\bf S}:=\dyad{s}$. %, which are $\d \times \d$ complex matrices.
Their natural ambient linear space is 
${\mathrm M}_{\d \times \d}(\C)$ of all $\d \times \d$ complex matrices, which can be used together with the Hilbert-Schmidt Hermitian inner product $\trace({\bf A}^\dagger {\bf B})$, where the operator $\trace(~)$ denotes the trace of a matrix.
%(Note that $\|A\|_{HS}^2 = \trace(A^*A)$ is the sum of the squared singular values of $A$.)

The \emph{Hilbert-Schmidt distance} on ${\mathrm M}_{\d \times \d}(\C)$ is defined as  
\begin{align}
	\dHS({\bf A}, {\bf B})&:=\left\| {\bf A} - {\bf B}\right\|_{HS} \notag\\
	&= \sqrt{\trace({\bf A}^\dagger {\bf A}) - 2 \Re \trace({\bf A}^\dagger {\bf B}) + \trace({\bf B}^\dagger {\bf B})}.
\end{align}
%Observe that the units of $\dHS(A,B)$ are energy.

Restricted to dyads, since $\left\| {\bf S} \right\|_{HS}^2 = \ip{s}^2$ and $\tr({\bf S}^\dagger{\bf S'})=|\ip{s}{s'}|^2$ is already real, we get 
\begin{equation}
	\dHS({\bf S},{\bf S'}) = \sqrt{\ip{s}^2 - 2|\ip{s}{s'}|^2 + \ip{s'}^2}.
\end{equation}
When $\|\ket{s} \|=\| \ket{s'} \|=1$, we could speak of \emph{Hilbert-Schmidt angle} and 
\begin{equation}
	\dHS({\bf S},{\bf S'})=\sqrt{2}\sqrt{1 - \gamma^2} \quad \text{where} \quad \gamma:= |\ip{s}{s'}|. 
\end{equation}

Traditionally, incoherently-received MVM symbols are represented by Stokes vectors $\sv \in \R^{\d^2-1}$, whose entries are the coefficients of the expansion of the \emph{trace neutralized dyad} ${\bf S}$, e.g., ${\bf S}-\frac{1}{\d}{\bf I}_N$ (assuming normalization $\ip{s}=1$), with respect to the Gell-Mann matrix basis \cite{Roudas_PJ_17}.
The Euclidean distance in  Stokes space, called \emph{Stokes distance},  coincides with the Hilbert-Schmidt distance up to scaling (having to do with the said trace adjustment and the conventions for the  Gell-Mann matrix basis): 
\begin{equation}
\label{stokDis:eq}
	\dStok(\sv,\sv') =  2C_{\d}\sqrt{1-\gamma^2}.
\end{equation}
% where as before
% \begin{equation}
% \gamma:= |\ip{s}{s'}|. 
% \end{equation}	

The Stokes distance should be better suited for thermal noise-limited direct-detection receivers, not for their ASE noise-limited counterparts, which  have been the focus  of this work.
