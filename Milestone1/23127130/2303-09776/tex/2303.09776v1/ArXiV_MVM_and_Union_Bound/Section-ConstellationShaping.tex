A quintessential problem in digital communications systems is the optimal selection of signal sets to minimize the symbol error probability under various noise distributions and channel impairments. The term geometric constellation shaping means that the positions of constellation points
in the signal space are selected appropriately in order to minimize the  error probability. As a prototypical example, Foschini et al. \cite{Foschini1974OptimizationOT} numerically optimized  the shapes of two-dimensional signal constellations with arbitrary cardinality in the case of additive white Gaussian noise and coherent detection.  Extending this work to optical communications, Karlsson and Agrell \cite{karlsson2016multidimensional} investigated optimized power-efficient multidimensional modulation formats for coherent optical communications systems. For relatively small dimensions $N$, they used sphere-packing algorithms to optimize the constellation points. For larger dimensions, their design strategy was to select points from $N$-dimensional lattices \cite{karlsson2016multidimensional}.




In the case of SVM ($N=2$), geometric constellation shaping for equipower signal  sets (PolSK) was performed numerically initially by Betti et al. \cite{betti1990multilevel} by maximizing the minimum Euclidean distance among signals in Stokes space and then by Benedetto and Poggiolini \cite{benedetto1994multilevel} by using  the exact symbol error probability of $M$-ary PolSK modulation schemes for constellations of equipower signals as an objective function. To derive a formula for the symbol error probability, Benedetto and Poggiolini calculated the boundaries of the decision regions initially considering signal vectors in Stokes space that were placed at the vertices of a regular polyhedron inscribed within the Poincar\'e sphere, and then extended the analysis to generic equipower constellations with constellation points at the vertices of irregular polyhedra \cite{benedetto1994multilevel}. Optimum signal constellations for the case of $N=2$ and $M=$ 4, 8, 16 and 32 signals were derived \cite{benedetto1994multilevel}.  Kikuchi \cite{Kikuchi2014} used suboptimal 2D quaternary and cubic octary constellations for implementation simplicity. Morsy-Osman et al.\cite{MorsyOsman2019} designed intensity/polarization SVM constellations based on the face-centered cubic (FCC) lattice to achieve maximum packing density, assuming a thermal-noise-limited scenario and using the minimum Euclidean distance criterion.



%We developed efficient and flexible Mathematica code for performing  gradient descent based geometric shaping of  (generalized) Stokes vector constellations used as {\it symbols} by the transmitter. %  using arbitrary interaction potentials between points of a constellation of Stokes vectors.
 Our goal here is to  spread out  the MVM constellation points in the generalized Stokes space and, thus, improve the symbol error probability in a direct-detection-based link. %We discussed our initial AWGN channel model in the first report. (Here it is only touched on in the subsequent section about the Union Bound.) 
 Since the adoption of symbol error probability as objective function leads to a computationally-intensive numerical optimization, a suitably-selected  simplified objective function is used instead.    The gradient-descent method \cite{boyd2004convex} is used for the minimization of the simplified objective function.
 
 To facilitate calculations, we consider an objective function from  electrostatics \cite{wiki:Thomson}, wherein the constellation points are assumed to be identical charges on the surface of a perfectly conducting Poincar\'e hypersphere. Starting from given initial positions, the charges are allowed to equilibriate under the action of Coulomb forces. In other words, we recast the original three-dimensional Thomson problem \cite{wiki:Thomson} to higher-dimensional Stokes space.	This adaptation requires constraining the $M$ constellation points to a $(2N-2)$--dimensional manifold  due to the relationships \eqref{eq:jv}, \eqref{eq:StokesVectorsdf} relating the higher-dimensional Jones and Stokes spaces  \cite{Antonelli:12, Roudas_PJ_17, Roudas_JLT_18}. 
 
 It is worth saying a few words here about the extensive literature on the Thomson problem. Since the original publication of the problem by  J. J. Thomson’s in 1904,  numerous papers were written on this topic and its variants. Saff and Kuijlaars \cite{Saff} give a comprehensive survey of the literature in the two dimensional case $N=2$, with an emphasis on the case when $M$ is large. Global minima for the Thomson Problem for $N=2$ are posted on the Cambridge website \cite{Cambridge_Thomson_website}. 

The function \lstset{language=Mathematica}{SpherePoints[n]}
%\begin{lstlisting}
%	SpherePoints[n]
%\end{lstlisting} % 
in Wolfram \textit{Mathematica}\textregistered \, \cite{kogan2017} gives the positions of $n$ approximately uniformly distributed points on the surface of the $S^2$ unit sphere in three dimensions, with exact values for certain small $n$ and a spiral-based approximation for large $n$ \cite{kogan2017}.


Closely related to Thomson's problem is the Tammes problem whose goal is to find the arrangement of $M$ points on a unit sphere which
maximizes the minimum distance between any two points. Jasper et al. \cite{jasper2019game} studied the Tammes problem in the complex projective space and maintain a website listing the current best-known numerical approximations \cite{Math_Colorado_State_website}.
% such that the minimum distance between any two points is as large as possible.

%Ballinger et al. \cite{Ballinger} give a survey analytical and numerical results for $N>2$. 
%
%Many potentials are investigated as a proxy for the error probability.
%
%Gradient descent is a local optimization technique. It starts by an initial configuration and iteratively adjusts the point coordinates in order to minimize a given cost-function. 
%
%Initial conditions to avoid clustering.
%
%The polyhedra given by different techniques differ.
%
%This is a  coarse optimization for engineering purposes. We call these constellations optimized instead of optimal, since it is unlikely that a global optimum was achieved.
%

\subsection{Gradient computation}


\noindent Consider a perfectly conducting Poincar\'e hypersphere with identical charges at the positions of the constellation points. As charges repel each other with Coulomb forces, they move on the surface of the Poincar\'e hypersphere until they reach an equilibrium distribution with minimum potential energy. 

The electrostatic potential energy $\Omega(d_{ij})$ of two charges $i,j$ separated by a distance $d_{ij}$ is inversely proportional to their distance $\Omega(d_{ij})\sim d_{ij}^{-1}$. The total electrostatic potential energy $U$ of a system of $M$ charges can be obtained by calculating the  potential energy $\Omega(d_{ij})$ for each individual pair of charges $i,j$ and adding the  potential energies for all distinct combinations of charge pairs
% Given %a constellation recorded as Jones vectors $\ket{s_i}$ where $i=1, \ldots, M$, and
%  a pairwise interaction potential energy $\Omega(d_{ij})$, the associated \emph{configuration potential energy} $U$ of the constellation is 
\begin{equation}
  \label{eq:ColumbPot:eq}
  U = \sum_{i=1}^{M}\sum_{j=i+1}^{M}\Omega(d_{ij}).
\end{equation}

The distances $d_{ij}$ can be calculated in terms of the corresponding unit Jones vectors $\ket{s_i} \in \C^\d$, $i=1, \ldots, M$. 
\begin{equation}
\label{dijDistGenFunction:eq}
d_{ij} = \psi\left(\lvert \ip*{s_i}{s_j} \rvert^2\right) = \psi \left( \gamma^2 \right)
\end{equation}
%$d_{ij} = \psi\left(\left|\ip{s_i}{s_j}\right|^2\right)$
 We leave the function $\psi$ unspecified for now to allow use of various distances between Stokes vectors (cf. Sec. \ref{sec:distances}).
 
To compute the gradient, we first assume that the Jones vectors depend on a certain parameter $t$ and compute  
\begin{align}
  \label{partDerRec:eq}
     \frac{\partial U}{\partial t} 
    &=   \sum_{i<j} \Omega'(d_{ij})  \frac{\partial d_{ij}}{\partial t} \notag\\
    &=   \sum_{i<j}  \Omega'(d_{ij})  \psi'\left( |\ip*{s_i}{s_j}|^2 \right) \frac{\partial |\ip*{s_i}{s_j}|^2}{\partial t} \notag\\
  &=  \sum_{i<j} \Omega'(d_{ij}) \psi'\left( |\ip*{s_i}{s_j}|^2 \right) \cdot 2 \Re\left(  \ip*{s_j}{s_i} \ip{\frac{\partial s_i}{\partial t}}{s_j} \right.\notag\\
    &\qquad \left. +   \ip*{s_i}{s_j} \ip{\frac{\partial s_j}{\partial t}}{s_i} \right),
\end{align}
where we used multilinearity to evaluate 
\begin{align}
  \frac{\partial |\ip*{s_i}{s_j}|^2}{\partial t}  &=  \frac{\partial}{\partial t} \ip*{s_j}{s_i} \ip*{s_i}{s_j} \notag\\
    &= \ip{\frac{\partial s_j}{\partial t}}{s_i} \ip*{s_i}{s_j} +  \ip*{s_i}{s_j} \ip{\frac{\partial s_j}{\partial t}}{s_i} .
\end{align}

Taking $t$ to be the real and imaginary parts of $s_{im}=x_{im} + \iota y_{i,}$, the components of the gradient of $U$ are found as   
\begin{align}
  \frac{\partial U}{\partial x_{im}}
  &=  \sum_{j: \ j \neq i} \Omega'(d_{ij}) \psi'\left( |\ip*{s_i}{s_j}|^2 \right)
    2 \Re\left(  \ip*{s_j}{s_i} s_{jm} \right)
\end{align}
and
\begin{align}
  \frac{\partial U}{\partial y_{im}}
  &=  \sum_{j: \ j \neq i} \Omega'(d_{ij}) \psi'\left( |\ip*{s_i}{s_j}|^2 \right)
    2 \Im \left(  \ip*{s_j}{s_i} s_{jm} \right).
\end{align}
To state the end result, the gradient\footnote{N.B.: This is not a complex derivative as $U$ is not necessarily analytic.} $\nabla U$ is the vector of real and imaginary parts of the (complex) vector
$\left( \frac{\partial U}{\partial s_{im}} \right)_{i,m} \in \C^{M\times\d}$ given by 
\begin{align}
  \label{mainGradientFormula:eq}
  \frac{\partial U}{\partial s_{im}}
  &=   2 \sum_{j:\ j \neq i} \Omega'(d_{ij}) \psi'\left( |\ip*{s_i}{s_j}|^2 \right)
    \ip*{s_j}{s_i} s_{jm}.
\end{align}




\subsection{Example: Coulomb Potential}


\noindent In the following, we adapt the three-dimensional Thomson problem \cite{wiki:Thomson} to the generalized Stokes space. It is true that the use of the electrostatic potential energy as an objective function in lieu of the symbol error probability  is not justified by the underlying physics of the problem under study. Nevertheless, as shown in Fig. \ref{fig:potential comparison}, the minimization of the electrostatic potential yields nearly optimal results that are very close to the ones obtained by minimizing the symbol error probability.    

For the Thomson problem, we use  the Euclidean distance $\dStok$ in the Stokes space,  per \eqref{stokDis:eq},  %\footnote{not a natural choice, it turns out}
 so that  \eqref{dijDistGenFunction:eq} is written as
\begin{equation}
  \psi(t) :=  2C_{\d}\sqrt{1-t}, % \textcolor{gray}{ = \sqrt{2}\sqrt{1- 2C_\d^2 (t -1/N)}}. %\frac{1}{\sqrt{2}\sqrt{1-1/\d}},
 \end{equation}
 where now $t=\gamma^2$.
 
 From $\psi(t)^2 = -4C_\d^2t + \text{Const}$, we get %$2 \psi(t) \psi'(t) = -4C_\d^2$ and
  $\psi'(t) = -2C_\d^2 \psi(t)^{-1}$, so 
% \begin{equation}
%    \psi'(t) = -2C_\d^2 \psi(t)^{-1}
% \end{equation}
\begin{align}
  \frac{\partial U}{\partial s_{im}}
  &=   -4C_\d^2 \sum_{j:\ j\neq i} \Omega'(d_{ij}) \psi\left( |\ip*{s_i}{s_j}|^2 \right)^{-1} \ip*{s_j}{s_i} s_{jm} \notag \\
  &=   -4C_\d^2 \sum_{j:\ j\neq i} \Omega'(d_{ij}) d_{ij}^{-1} \ip*{s_j}{s_i} s_{jm}.
\end{align}
Furthermore, for the case of electrostatic Coulomb forces acting in the Stokes space, we have the inverse distance potential 
\begin{equation}
  \Omega(d_{ij}) = d_{ij}^{-1}, \quad \Omega'(d_{ij}) = -d_{ij}^{-2}.
\end{equation}
Thus, instantiating \eqref{mainGradientFormula:eq} yields
\begin{equation}
\boxed{\frac{\partial U}{\partial s_{im}}
  =   4C_\d^2 \sum_{j:\ j\neq i} d_{ij}^{-3}  \ip*{s_j}{s_i} s_{jm}.}
  \label{eq:gradient_Thomson}
\end{equation}

 \subsection{Numerical details}
 \label{subsection: numerical details}

 % \noindent We wrote an efficient (partially-compiled) Mathematica program implementing gradient descent for arbitrary potentials.
 % For instance, using a personal computer, it is possible to optimize MVM constellations with thousands of points for practical signal space dimensionalities.  
 
     
 % The gradient descent algorithm is seeded by either a randomly-generated constellation or a small random perturbation of a geometrically constructed \emph{standard constellation}, as exemplified by the Cartesian product of independent constellations. 
 
 % To give an example, taking the independent product of $\{\pm 1, \pm \iota\}$ in each of the $N$ quadrature planes gives $4^\d$ Jones vectors $\ket{s_m}$, each in a set of four vectors equivalent by the overall phase rotation. Taking one vector from each set of four (say by fixing the phase of the first component) and normalizing yields a constellation with $M=4^{\d-1}$ vectors $\sv_m$. Mathematically, it consists of the orbits of the vertices of a  hypercube\footnote{N.B.: Not all hypercubes of the same size are unitarily equiavalent.} in $\C^\d$ under the circle action by the phase rotation.  This constellation  amounts to simple multiplexing of binary %(BPSK)
 %  encoding in each (real) quadrature  degree of freedom. It is built from four points per core/polarization mode quadrature, subjected to a 4-fold phase symmetry reduction to account for incoherent detection. (Thus for $N/2$ cores, it has $4^N/4=4^{N-1}$ points.)  
 
 \noindent We developed an efficient, partially-compiled Mathematica code implementing the gradient-descent optimization algorithm for arbitrary potentials. This implementation  is adequately fast on a personal computer to enable the design of MVM constellations with up to $M=1024$ points for up to $N=8$  spatial degrees of freedom (SDOFs) (see Fig. \ref{fig:SEplot_MN_MVM}).  
 
     
 The gradient-descent optimization algorithm starts with either a randomly-generated constellation or a small random perturbation of a deterministic constellation.
 %, as exemplified by the Cartesian product of independent two-dimensional constellations, one for the quadrature plane of each SDOF. 
 To give an example, consider the following deterministic constellation of Jones vectors: Their first component is set equal to unity, while their remaining $N-1$ components take all possible combinations of values in $\{\pm 1, \pm \iota\}$. Finally, the Jones vector length is normalized to unity. This process yields an MVM constellation with $M=4^{\d-1}$ vectors. We call it the \emph{standard constellation}. 
 %\emph{standard Jones hypercube constellation}
 Mathematically, it represents %JKApr consists of
  the orbits of the vertices of a  hypercube
%   \footnote{N.B.: Not all hypercubes of the same size are unitarily equivalent.} 
in $\C^\d$ under the circle action by the %(overall) 
 phase rotation. {For this reason, it is possible to refer to it as the \emph{standard reduced hypercube constellation} or the \emph{standard reduced Jones hypercube constellation}.}
