 The use of multidimensional modulations can %dramatically
 decrease the energy consumption of optical links. In this paper, we propose and study Mode Vector Modulation (MVM), a generalized polarization modulation scheme for transmission over multimode/multicore optical fibers or free space. Similar to Polarization Shift Keying (PolSK) and Stokes Vector Modulation (SVM), MVM can be used in conjunction with direct detection and, therefore,  is suitable for next-generation, short-haul optical interconnects.
This paper focuses on the MVM transceiver architecture, the back-to-back performance of optically-preamplified MVM direct-detection (DD) receivers, the optimized geometric shaping of the MVM constellation, and the related bit-to-symbol mapping.
We show that MVM DD outperforms conventional single-mode, direct-detection-compliant, digital modulation formats by several dB's in terms of receiver sensitivity and the SNR gain increases with the number of spatial degrees of freedom (SDOFs) $N$. 
%The MVM DD receiver hardware complexity is on the order of $N^2$ but can be reduced by taking advantage of the interdependence of Stokes parameters and eventually lies in between that of single-branch direct-detection receivers  and polarization and phase-diversity, intradyne, coherent optical receivers. In addition, the MVM spectral efficiency can be adjusted by selecting the constellation cardinality. For illustration, we consider optimized constellations with nominal spectral efficiencies in the range 0.1--2 b/s/Hz/SDOF and evaluate the trade-offs between receiver sensitivity and spectral efficiency.

    % Multiple optical waves launched into orthogonal propagation modes in multimode and multicore  fibers or free-space can be altered jointly to create spatial light modulations. Spatially-modulated signals can be represented geometrically as generalized polarization vectors (hereby referred to as \emph{mode vectors}) in multidimensional Jones or Stokes spaces. Similar to the case of conventional polarization measurements, the Stokes components of mode vectors can be accurately determined using direct detection. 
    
    % In this paper, for the first time,  we propose and study theoretically the back-to-back performance of \emph{Mode Vector Modulation (MVM)}, a new energy-efficient, multidimensional modulation format for short-haul,  optical interconnects using direct detection and  multimode and multicore  fibers. MVM is a  generalization into higher dimensions of previously-proposed polarization modulation schemes for single-mode fibers, i.e., \emph{Polarization Shift Keying (PolSK)} and  \emph{Stokes Vector Modulation (SVM)}. 
    
    % In the following, we design the architecture of the MVM transceiver, we numerically optimize the   MVM constellation using geometric shaping, we determine the corresponding bit-to-symbol mapping, and we evaluate by analysis and simulation  the performance of  optically-preamplified MVM direct-detection receivers. 
    
    % We show that MVM greatly improves bit SNR performance compared to SVM at the expense of transceiver complexity and spectral efficiency.