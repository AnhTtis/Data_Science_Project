\noindent The goal of this appendix is to  compute  the  pairwise error probability  $\Peb^{m' | m}$   defined by \eqref{eq:mPrimeUnionBoundBis} and prove \eqref{mainExact:eq}.


\subsection{Geometric Setup}

\noindent 
We fix $m \neq m'$ and,  to simplify expressions, set $\ket{s}:=\ket{s_m}$ and $\ket{s'}:=\ket{s_{m'}}$. 
We assume that
\begin{subequations}
	\begin{align}
	\ip{s} &=\ip{s'}=1, \\ \gamma & :=\ip{s}{s'}>0.
	\label{eq:decision_inequality}
\end{align}
\end{subequations}
% (i.e. the vectors are not orthogonal).
In \eqref{eq:decision_inequality},  we dropped the absolute value on $\ip{s}{s'}$ because, since we deal with two vectors in isolation and only the projection operators ${\bf S}=\dyad{s}$ and ${\bf S}'=\dyad{s'}$ matter,  we can adjust the phase of $\ket{s'}$ so that  $\ip{s'}{s}$ is a positive real.
As the first step, we reduce the considerations to the two-dimensional complex subspace $\Sigma$ spanned by $\ket{s}$ and $\ket{s'}$ and derive analytical expressions in a convenient orthonormal basis for $\Sigma$. 

We  express $\Peb^{m' | m}$ as 
\begin{align}
%\label{eq:mPrimeUnionBoundBis}
\label{PebQuad:eq}
	\Peb^{m' | m} &= \Prob\left(  |\ip{r}{s'}|^2 -  |\ip{r}{s}|^2 \geq 0 \right) \notag\\
	&= \Prob\left( \mel{r}{{\bf {\bf \Delta}}}{r} \geq 0 \right),
\end{align}
where we introduced the difference of dyads \nomenclature[$delta$]{${\bf \Delta}$}{Difference of dyads, ${\bf \Delta}:= \ket{s'}\bra{s'}-\ket{s}\bra{s}$}
${\bf \Delta} := {\bf S'}-{\bf S} $ and the associated quadratic form\footnote{Note that this formulation makes it clear that ${\mathcal D}^{m' | m}$ is bounded by $3D$-cone in $\Sigma$ treated as a 4D real space.}   
\begin{align}
	|\ip{r}{s'}|^2 - |\ip{r}{s}|^2 &= \trace\left( {\bf R}{\bf S}'\right) - \trace\left( {\bf R}{\bf S}\right) \notag \\
	&= \trace\left( {\bf R}{\bf \Delta} \right) %\notag \\
	= \mel{r}{{\bf \Delta}}{r}. 
\end{align}

% This is a quadratic form in gaussian $\ket{n}$ shifted by $\ket{s}$ (as $\ket{r} =  \ket{s} +  \ket{n}$) so we know that we  can understand the random variable $\mel{r}{\Delta}{r}$ as a generalized chi-squared variable.
% Let's articulate this in detail.



%\bigskip
%\pagebreak

%\subsection{Auxilliary orthonormal basis} % [NEW 2 Dec 2020]} 

%Our goal is to evaluate $\Prob\left( \mel{r}{\Delta}{r} \geq 0 \right)$.
\noindent The following three real \emph{length parameters} will play a key role: 
\begin{subequations}
\begin{align}
	\label{deltarhodef:eq}
	\delta &:= \sqrt{1-\gamma^2},  \\
	\rho_\pm &:= \sqrt{\frac{1 \pm \delta}{2}}.
	\label{deltarhodef:eq2}
\end{align}
\end{subequations}

%Note that $\Delta$ is a  Hermitian rank two matrix.
For ease of reference we record that 
\begin{subequations}
\begin{align}
	\label{prels:eq}
	\gamma &= \sqrt{1-\delta^2} = 2\rho_+ \rho_-  \\
	\rho_\pm - \rho_\mp \gamma &= (1-2\rho_\mp^2) \rho_\pm  = \pm \delta \rho_\pm. \label{prels2:eq}
\end{align}
\end{subequations}
% We claim that the vectors $\ket{u_+}$ and $\ket{u_-}$ below are unit eigenvectors of $\Delta$,  
We also introduce two  vectors in $\Sigma$, $\ket{u_+}$ and $\ket{u_-}$,   defined as
\begin{align}
	\label{eq:ONbasis}
	% \ket{u_\pm}   &:=\frac{\rho_\pm \ket{s'} - \rho_\mp \ket{s} }{\delta}.
	\ket{u_\pm} 
	&:=\pm \frac{\rho_\pm \ket{s'} - \rho_\mp \ket{s} }{\delta}.
\end{align}
Their scalar components along $\ket{s'}$ and $\ket{s}$ are found, via \eqref{prels:eq} and \eqref{prels2:eq}, to be
\begin{subequations}
\begin{align}
	\label{eq:inProdS}
	% \ip{s}{u_\pm} =\frac{1}{\delta  }\left( - \rho_\mp +  \rho_\pm \gamma \right) = \frac{ \rho_\mp}{ } \quad \text{and} \quad
	\ip{s'}{u_\pm} &=\pm \frac{\rho_\pm - \rho_\mp\gamma}{\delta  } = \rho_\pm  \\ 
	\ip{s}{u_\pm}& =\pm \frac{  \rho_\pm\gamma - \rho_\mp}{\delta  } =  \rho_\mp.
		\label{eq:inProdS2}
\end{align}
\end{subequations}
%We then observe the difference of the projections of $\ket{u_\pm}$ onto $\ket{s'}$ and $\ket{s}$ as
Computing ${\bf \Delta} \ket{u_\pm}$ as the difference %${\bf S'} \ket{u_\pm} - {\bf S} \ket{u_\pm}$ 
 of the projections onto  $\ket{s'}$ and $\ket{s}$ (and then using \eqref{eq:ONbasis}) gives 
\begin{align}
	%\label{eq:ONbasisInv}
	{\bf \Delta} \ket{u_\pm} %= S' \ket{u_\pm}-S\ket{u_\pm}
	=   \rho_\pm\ket{s'} -  \rho_\mp\ket{s} = \pm \delta \ket{u_\pm}.
\end{align}
Thus  $\ket{u_\pm}$ are eigenvectors of ${\bf \Delta}$ with eigenvalues $\pm \delta$, respectively. 
%we use the above (and (\ref{eq:ONbasis})) to compute 
They are normalized since  combining  \eqref{eq:ONbasis} and \eqref{eq:inProdS} yields
%we use the above (and \eqref{eq:ONbasis}) to compute 
They are normalized since  combining  \eqref{eq:ONbasis} and \eqref{eq:inProdS}, \eqref{eq:inProdS2} yields
\begin{align}
	\ip{u_\pm} &= \pm \frac{1}{\delta} \ip{ \ (\rho_\pm \ket{s'} - \rho_\mp \ket{s}) \ }{u_\pm} \notag \\
	%&= \frac{1}{\delta \sqrt{2}} \left( \rho_\pm \ip{s'}{u_\pm} - \rho_\mp \ip{s}{u_\pm} \right)  \\
	&= \pm \frac{1}{\delta} \left( \rho_\pm^2 -  \rho_\mp^2 \right) %\notag \\
	=1. 
\end{align}

Because ${\bf \Delta}$ is Hermitian  of  rank two, the eigenvectors $\ket{u_\pm}$ are orthogonal and the remaining eigenvalue of ${\bf \Delta}$, other than $\pm \delta$, is zero (with the orthogonal complement $\Sigma^\perp$ as its eigenspace).   
The underlying geometry is simple: Examining \eqref{eq:ONbasis}, we see that  $\ket{u_\pm}$ sit in the real sub-plane inside $\Sigma$  spanned by $\ket{s'}$ and $\ket{s}$.
The vectors  $\ket{s'}$ and $\ket{s}$ form an acute angle  (by virtue of our initial phase rotation).
From \eqref{eq:inProdS} and \eqref{eq:inProdS2},   $\ip{s'}{u_+}= \rho_+ = \ip{s}{u_-}$,  so this acute angle  is positioned symmetrically within the right angle formed by  $\ket{u_\pm}$. One could say that $\ket{u_\pm}$ are the result of \emph{symmetrically opening up} $\ket{s'}$ and $\ket{s}$ to be orthogonal.
%(Fig. \ref{fig:vector_diagram}).

% \begin{figure}[!htb]\centering
% 	\includegraphics[width=0.3\textwidth]{Figs/vector_diagram.png}
% 	\caption{Vector diagram displaying $\ket{u_\pm}$, $\ket{s}$ and $\ket{s'}$.} 
% 	\label{fig:vector_diagram}
% \end{figure}

%[STOP EDITING 11 June 2021]


% \bigskip
% \bigskip

%\pagebreak

%%%%%%%%%%%%%%%%%%%%%%%%%%%%%%%%%%%%%%%%%%%%%%%%%%%%%%%%%%%%%%%%%%%%%%%%%%%%%%%%%%%%%%%%%%%
%%%%%%%%%%%%%%%%%%%%%%%%%%%%%%%%%%%%%%%%%%%%%%%%%%%%%%%%%%%%%%%%%%%%%%%%%%%%%%%%%%%%%%%%%%%

\subsection{Signal Decomposition}

% Our first goal is to
%Continuing the work towards computing  the  probability  $\Peb^{m' | m}$ 

\noindent With our orthonormal basis $\ket{u_\pm}$ of $\Sigma$ in hand, we  
% spanned by the $\ket{s}$ and $\ket{s'}$
orthogonally decompose the noise
\begin{align}
	\label{noisedecomp:eq}
	\ket{n} = \ket{n_+} + \ket{n_-} + \ket{\tilde{n}},
\end{align}
where  the components along  $\ket{u_\pm}$ are 
\begin{equation}
	\label{eq:ONbasisTwoDim}
	\ket{n_+} := \ip{n}{u_+} \ket{u_+} \quad \text{and} \quad
	\ket{n_-} := \ip{n}{u_-} \ket{u_-},
\end{equation}
and $\ket{\tilde{n}}$ is the component orthogonal to $\Sigma$.
(Going forward, tilde indicates components orthogonal to $\Sigma$.)
Inverting \eqref{eq:ONbasis}, we get the analogous decomposition % \footnote{by taking inner products with $\ket{u_{\pm}}$ or inverting the matrix
% $$ \left( \frac{1}{\delta} \begin{bmatrix} -\rho_- & \rho_+ \\ -\rho_+ & \rho_- \end{bmatrix}\right)^{-1} = \frac{\delta}{\delta} \begin{bmatrix} \rho_- & - \rho_+ \\ \rho_+ & -\rho_- \end{bmatrix} 
% $$ where we used the determinant $-\rho_-^2 + \rho_+^2 = -(1-\delta)/2 + (1+\delta)/2 =  \delta  $.}
of the symbols
\begin{subequations}
\begin{align}
	\label{symbDecomp:eq}
	\ket{s} &=  \rho_- \ket{u_+} + \rho_+ \ket{u_-}, \\
	\ket{s'} &=  \rho_+ \ket{u_+} + \rho_- \ket{u_-}.
	\label{symbDecomp:eq2}
	%\quad \text{and} \quad \ket{n} = \ket{n_+} + \ket{n_-}.
\end{align}
\end{subequations}

%Disregarding the immaterial random phase,  per \eqref{eq:AWGNJoneschannelBis}, the received signal is $\ket{r} = \ket{s} + \ket{n}$, 
Because the Gaussian noise $\ket{n}$ is symmetric with respect to phase rotations, we can disregard the random phase in \eqref{eq:AWGNJoneschannelBis} and express the Jones vector representing the incoherently received signal as $\ket{r} = \ket{s} + \ket{n}$. 
Putting together \eqref{noisedecomp:eq} and \eqref{symbDecomp:eq}, \eqref{symbDecomp:eq2}, reveals its components along  $\ket{u_\pm}$ as 
equal to 
\begin{subequations}
\label{rcomp:eq}
\begin{align}
	% \ket{r_\pm} = \frac{1}{\sqrt{2}}\ket{u_\pm} + \ket{n_\pm}
	\ket{r_+} &= \rho_-\ket{u_+} + \ket{n_+},
	\\ 
	\ket{r_-} &= \rho_+\ket{u_-} + \ket{n_-},
\end{align}
\end{subequations}
with the squared magnitudes consequently given by 
\begin{equation}
\label{rpmSize:eq}
  \ip{r_\pm}  = \rho_{\mp}^2 + 2 \rho_{\mp} \Re \ip{n_\pm}{u_\pm} + \ip{n_\pm}.
\end{equation}

The full $\ket{r}$ decomposes into orthogonal components,
\begin{equation}
	\ket{r} = \ket{s} + \ket{n} %= \frac{1}{\sqrt{2}}\left( \rho_- \ket{u_+} - \rho_+ \ket{u_-} \right) + \ket{n_+} + \ket{n_-}
	= \ket{r_+} + \ket{r_-} + \ket{\tilde{r}},
\end{equation}
along the eigenspaces of ${\bf \Delta}$ for eigenvalues $\delta$, $-\delta$, and $0$, respectively. 

Using  ${\bf \Delta} \ket{\tilde{r}}=0$ as well as ${\bf \Delta} \ket{r_\pm}= \pm \delta \ket{r_\pm}$ and  $\ip{r_-}{r_+}=0$,            %$\mel{r_+}{{\bf \Delta}}{r_-}=0$ 
 the quadratic form simplifies to  
\begin{align}\label{pmDeltaProd:eq}
	\mel{r}{{\bf \Delta}}{r}
	&= \mel{\ \ket{r_+} + \ket{r_-} + \ket{\tilde{r}} \ }{\ {\bf \Delta} \ }{ \  \ket{r_+} + \ket{r_-} + \ket{\tilde{r}} \ } \notag\\
	&=  \mel{r_+}{{\bf \Delta}}{r_+}  + \mel{r_-}{\bf \Delta}{r_-} \notag\\
		&= \delta \ip{r_+} - \delta \ip{r_-}. 
\end{align}
%%%%%%%%%%%%%%%%%%%%%%%%%%%%%%%%%%% NEW %%%%%%%%%%%%%%%%%%%%%%%%%%%%%%%%%%%%%%%%%%%%%%%%%%
Finally, substituting \eqref{rpmSize:eq}, yields 
\begin{align}
	\label{mainMelQuadBis:eq}
	 \mel{r}{{\bf \Delta}}{r} %\notag\\
	%&= \mel{\ \ket{r_+} + \ket{r_-} + \ket{\tilde{r}} \ }{\ \Delta \ }{ \  \ket{r_+} + \ket{r_-} + \ket{\tilde{r}} \ } \\
	%&=  \mel{r_+}{\Delta}{r_+}  + \mel{r_-}{\Delta}{r_-}  \\
	&= \delta \bigg[\rho_-^2 -\rho_+^2 
	+ 2 \Re \left\{ \rho_-  \ip{u_+}{n_+}  - \rho_+ \ip{u_-}{n_-} \right\} \notag\\
    &\qquad +\ip{n_+} - \ip{n_-}\bigg].
\end{align}
%(Mind you $\rho_+^2-\rho_-^2 =\delta$.)

%%%%%%%%%%%%%%%%%%%%%%%%%%%%%%%%%%%%%%%%%%%%%%%%%%%%%%%%%%%%%%%%%%%%%%%%%%%%%%%%%%%%%%%%%%%%%%%%%%%%%%%%%%%%



\subsection{Pairwise symbol error probability calculation}
\noindent We are ready to derive the closed form \eqref{mainExact:eq} for  $\Peb^{m' | m}$ by identifying the relevant probability distributions associated to the quadratic form.  
We can describe the points of $\Sigma$ by their components $x_-+\iota y_-$ and $x_++\iota y_+$ with respect to the orthonormal basis $\ket{u_\mp}$. 
Accordingly, we have four independent real Gaussian random variables with variance $\nvar$:
\begin{align}
    x_\mp := \Re \ip{n}{u_\mp} \quad \text{ and } \quad 
    y_\mp := \Im \ip{n}{u_\mp}. 
\end{align}
%$$x_-:=\Re \ip{n}{u_+}$, $x_+:=\Re \ip{n}{u_-}$,
%$y_-:=\Im \ip{n}{u_+}$, $y_+:=\Im \ip{n}{u_-}$.
The last equation of the previous section, \eqref{mainMelQuadBis:eq}, reads
\begin{align}
  \frac{1}{\delta} \mel{r}{{\bf \Delta}}{r}
  &= \rho_-^2 -\rho_+^2 +  2 \rho_-x_+ - 2 \rho_+ x_- \notag\\
     &\quad +x_+^2 + y_+^2 - x_-^2 -y_-^2.
  % %&= -\delta + \left( x_- +\frac{\rho_-}{\sqrt{2}}\right)^2-\frac{\rho_-^2}{2} - \left( x_+ -\frac{\rho_+}{\sqrt{2}}\right)^2 + \frac{\rho_+^2}{2}+ y_-^2 - y_+^2 \\
  % &= \left( x_- +\rho_-\right)^2
  %   - \left( x_+ -\rho_+\right)^2 + y_-^2 - y_+^2.
\end{align}
So, upon completing the squares,  the sought pairwise error probability in \eqref{PebQuad:eq}  %\footnote{Note that $\rho_-<\rho_+$ so this is the probability that the {\it nearby Gaussian variable} (corresponding to $\ket{s}$) has larger modulus than the {\it remote Gaussian variable} (corresponding to $\ket{s'}$).} 
is 
\begin{align}
%  \label{eq:mPrimeUnionBoundBis}
    \Pbin
    &= \Prob\left( \mel{r}{{\bf \Delta}}{r} \geq 0 \right) \notag\\
    &= \Prob\left(\left( \rho_- + x_+ \right)^2 + y_+^2 \geq 
    \left(\rho_+ + x_- \right)^2 +  y_-^2  \right).
\end{align}
This is to say that  
\begin{align}
  \label{PebViariceIneqProb:eq}
  \Pbin = \Prob\left( \psi_- \geq \psi_+  \right),
  %= \int_0^\infty \left[ \int_0^{\psi_-} f_+(\psi_+)\, d\psi_+ \right]  f_-(\psi_-)\, d\psi_- 
\end{align}
 where we introduced two independent Rice-distributed random variables  
\begin{subequations}
\begin{align}
  \psi_- &:= \sqrt{\left(x_+ +\rho_-\right)^2 + y_+^2},  \\
  \psi_+ &:= \sqrt{\left(x_- +\rho_+\right)^2 + y_-^2}%  \psi_- := \sqrt{\left( \frac{x_-}{\sqrt{\nvar}} +\frac{\rho_-}{\sqrt{\nvar}}\right)^2 + \left(\frac{y_-}{\sqrt{\nvar}}\right)^2} \quad \text{and} \quad
  % \psi_+ := \sqrt{\left( \frac{x_+}{\sqrt{\nvar}} -\frac{\rho_+}{\sqrt{\nvar}}\right)^2 + \left(\frac{y_+}{\sqrt{\nvar}}\right)^2}
\end{align}
\end{subequations}
with  reference distances $\rho_-$ and  $\rho_+$, respectively,   and a common scale parameter $\sigma$. 
The PDFs of $\psi_\pm$ are 
\begin{align}
  \label{eq:RiceCDFbis}
 f_\pm(x) = \frac{x}{\nvar}
    \exp\left({-\frac{x^2+\rho_\pm^2}{2\nvar}}\right)
    I_0\left( \frac{x \rho_\pm}{\nvar} \right) ,
\end{align}
 % via the {\it modified Bessel function of the first kind}:
 %  \begin{equation}
 %    \label{eq:modBesselZero}
 %  I_0(x):= \frac{1}{2\pi} \int_0^{2 \pi} e^{x \cos \theta} \, d\theta = \frac{1}{\pi} \int_0^{\pi} e^{x \cos \theta} \, d\theta.
 %  \end{equation}
 with the corresponding tail (complementary) distribution functions \cite{Proakis}
\begin{equation}
   \label{eq:nonCentchiSquaredCDF}
   \Prob(\psi_\pm \geq x) = Q_{1}\left( \frac{\rho_\pm}{\sqrt{\nvar}},\frac{x}{\sqrt{\nvar}} \right).
 \end{equation}
 Recall that $Q_1$  stands for the \emph{Marcum function} defined by \eqref{eq:MarcumDef}.

 %Note that $a$ represents the non-centrality length and $b$ is the tail cutoff length.  
 Thus, formula \eqref{PebViariceIneqProb:eq} can be represented by a single integral:
 \begin{align}
  \label{singleInRep:eq}
  &\quad \Pbin \notag\\
  &= \int_0^\infty \Prob\left( \psi_- \geq x  \right)\, f_+(x) dx \notag \\
  &= \int_0^\infty Q_{1}\left( \frac{\rho_-}{\sqrt{\nvar}},\frac{x}{\sqrt{\nvar}} \right)
    \frac{x}{\nvar} 
    \exp\left({-\frac{x^2+\rho_+^2}{2\nvar}}\right) 
    I_0\left( \frac{x \rho_+}{\nvar} \right) \,
    dx. 
 \end{align}
 The integral can be computed
 % there is a closed form representation of the above probability for two Rice distributed variables:
 and is given by a formula from \cite{Benedetto1987}
 %\textcolor{red}{[PUT LIT REF John suggests: Digital Transmission Theory 1st Edition  by Sergio Benedetto, Exio Biglieri, Valentino Castellani]}
 to the effect that 
 \begin{align}%\label{benedettoClosed:eq}
    \Pbin
    &= \Prob(\psi_- \geq \psi_+) \notag\\
    &= Q_1(\aa,\bb)-\frac{1}{2}
    \exp\left({-\frac{\aa^2+\bb^2}{2}}\right)
    I_0(\aa \bb),
    \label{eq:proof_ub}
 \end{align}
 where  (recalling  $\gamma_s = \frac{1}{2\nvar}$, per \eqref{gammasSigma:eq})
 \begin{subequations}
 \begin{align}
   \aa  &:=\frac{\rho_-}{\sqrt{2 \nvar}}=\rho_-\sqrt{\gamma_s}, \\
   \quad  \bb&:=\frac{\rho_+}{\sqrt{2 \nvar}}=\rho_+\sqrt{\gamma_s}.
   %JKApr , \\ \quad \nu&:=1. %\frac{\sigma_-}{\sigma_+}
 \end{align}
\end{subequations}
%\textcolor{red}{ $a$ and $b$ were switched.}

 Using $\rho^2_- + \rho^2_+ = 1$ and $\rho^2_- \rho^2_+ = \frac{1-\delta^2}{4}=\frac{\gamma^2}{4}$ from \eqref{deltarhodef:eq} through \eqref{prels2:eq} gives
 \begin{subequations}
 \begin{align}
   \label{aabbRels:eq}
   \aa^2 + \bb^2 &=(\rho^2_-+\rho^2_+)\gamma_s = \gamma_s, \\
    2\aa\bb &=2\rho_-\rho_+\gamma_s = \gamma \gamma_s. 
 \end{align}
\end{subequations}
Thus \eqref{eq:proof_ub} coincides with the promised formula \eqref{mainExact:eq}.

%JK 4 Dec As a sanity check, we note that, for $N=2$, expression \eqref{eq:proof_ub} coincides with \eqref{eq:proof_ub_N2}.

 
%  \textcolor{gray}{
%    We arrive at the final closed form expression: 
%  \begin{thm}
%    The binary error probability between two equiprobable unit vectors $\ket{s}, \ket{s'} \in \C^\d$ subjected to AWG noise with variance $\nvar$ (per dimension) is given by  
%  \begin{equation}
%    \Peb = Q_1\left( \sqrt{\frac{\rho^2_-}{2 \nvar}},\sqrt{\frac{\rho^2_+}{2 \nvar}} \right)
%      -\frac{1}{2}e^{-\frac{1}{4\nvar}}I_0\left(\frac{\gamma}{4 \nvar}\right)
%    \end{equation}
%    where
%  \begin{equation}
%   \rho^2_{1,2} =\frac{1 \mp \delta}{2}  = \frac{ 1 \mp \sqrt{1 - \gamma^2}}{2} \quad \text{ and } \quad \gamma = |\ip{s}{s'}|.
% \end{equation}
% \end{thm}
% It is worth noting that the difference $\rho^2_+-\rho^2_-$ coincides with the Hilbert-Schmidt distance
% $\dHS(s,s') = \sqrt{2}\sqrt{1 - \gamma^2}$ (up to scaling by $\sqrt{2}$).
% (The incoherent metric  $\dD(s,s') = \sqrt{2}\sqrt{1 - \gamma}$ is not prominently displayed.)
% }

%  \pagebreak