\subsection{Mathematical notation}
\noindent  Throughout the paper, we follow the conventions of \cite{Gordon2000, Roudas_PJ_17, Roudas_JLT_18}, where Dirac's bra-ket vectors represent both unit and non-unit vectors in the generalized Jones space, while hats indicate unit vectors and arrows indicate non-unit vectors in the generalized Stokes space. %A full list of symbols is given at the end of the paper.

We can parameterize a unit Jones vector
$\ket{s}$, up to phase, using $2N-2$ hyperspherical coordinates \cite{Roudas_JLT_18}, i.e., 
% \begin{align}
% 	\ket{s} &:= \left[\cos (\phi _{1} ),\sin (\phi _{1} )\cos (\phi _{2} )e^{i\theta _{1} } ,\ldots , \right.  \notag\\
% 	&\quad \left. \sin (\phi _{1} )\cdots \sin (\phi _{N-2} )\sin (\phi _{N-1} )e^{i\theta _{N-1} } \right]^{T} .
% 	\label{eq:jv}
% \end{align}
% \begin{align}
% 	\ket{s} &:= 
% 	\left[
% 	\cos \phi_{1}, \ \sin \phi_{1}\cos\phi_{2}\, e^{i\theta _{1}}, \ \sin\phi_{1}\sin\phi_{2}\cos\phi_{3}\, e^{i\theta _{2}} ,\ldots , \right.  \notag\\
% 	&\quad \left. 
% 	%\sin\phi_{1} \cdots \sin\phi_{N-2}\cos\phi_{N-1}\, e^{i\theta _{N-2}}, 
% 	\sin\phi_{1}\cdots \sin\phi_{N-2}\sin\phi_{N-1}\, e^{i\theta _{N-1}} 
% 	\right]^{T} .
% 	\label{eq:jv}
% \end{align}
\begin{align}
	\ket{s} &:= 
	\left[
	\cos \phi_{1}, \ \sin \phi_{1}\cos\phi_{2}\, e^{\iota \theta _{1}},  \sin\phi_{1}\sin\phi_{2}\cos\phi_{3}\, e^{\iota \theta _{2}},\ldots ,  \right.  \notag\\
	&\quad \left. 
	\sin\phi_{1} \cdots \cos\phi_{N-1}\, e^{\iota \theta _{N-2}}, \
	\sin\phi_{1}\cdots \sin\phi_{N-1}\, e^{\iota \theta _{N-1}} 
	\right]^{T} 
	\label{eq:jv}
\end{align}
% \begin{align}
% 	\ket{s} &:= 
% 	\left[
% 	\cos \phi_{1}, \ \sin \phi_{1}\cos\phi_{2}\, e^{i\theta _{1}}, \ \sin\phi_{1}\sin\phi_{2}\cos\phi_{3}\, e^{i\theta _{2}} ,\ldots , \right.  \notag \\
% 	&\quad 
% 	\sin\phi_{1} \cdots  \sin\phi_{N-2} \cos\phi_{N-1}\, e^{i\theta _{N-2}}, \notag \\
% 	&\quad \left.  \sin\phi_{1}\cdots  \sin\phi_{N-2} \sin\phi_{N-1}\, e^{i\theta _{N-1}} 
% 	\right]^{T} 
% 	\label{eq:jv} 
% \end{align}
where the superscript $T$ indicates transposition.

%In Sec. \ref{sec:txrx_design}, we will be based on \eqref{eq:MVMsigs}, \eqref{eq:jv} in order to build the MVM transmitter using $N$ parallel branches of concatenated electro-optic Y-junctions and phase shifters. 

%Alternatively, the launched states  can be geometrically represented by real vectors in a  generalized $(N^2-1)-$dimensional Stokes space \cite{Antonelli:12,Roudas_PJ_17,Roudas_JLT_18}. 

Unit Jones vectors, up to phase, are often represented by generalized real unit Stokes vectors $\hat s$ in a higher-dimensional real vector space $\R^{\d^2-1}$. Generalized unit Stokes vectors are defined by the quadratic form \cite{Roudas_JLT_18}
\begin{equation} 
	\hat s : = C_N \mel{s}{\mathbf{\Lambda }}{s},
	\label{eq:StokesVectorsdf}
\end{equation} 
where $\mathbf{\Lambda}$ denotes the  generalized Gell-Mann spin vector  and $C_N$ denotes the normalization coefficient \cite{Roudas_JLT_18} 
\begin{equation}
C_N := \sqrt{ \frac{N}{ 2\left(N-1\right)}}.
\end{equation}
    \nomenclature[$shat$]{$\hat{s}$}{Generalized unit Stokes vector}
	\nomenclature[$lambda$]{$\mathbf{\Lambda}$}{Generalized Gell-Mann spin vector}
	\nomenclature[$n$]{$N$}{Number of spatial and polarization degrees of freedom}
	\nomenclature[$cn$]{$C_N$}{Normalization coefficient, $C_N:= \sqrt{N/[2(N-1)]}$}

From the generalized Stokes vector definition \eqref{eq:StokesVectorsdf}, we notice that the dimensionality of the generalized Stokes space grows quadratically with the number of spatial and polarization modes in MMFs/MCFs. Therefore, instead of using SVM-DD in conjunction with the conventional 3D Stokes space, we can generate more energy-efficient  constellations by spreading the constellation points in the generalized Stokes space. 



Also notice that the $(N^2-1)$ components of $\hat s$ are functions of the $2N-2$ hyperspherical coordinates of $\ket{s}$  in \eqref{eq:jv} and, therefore,  are  interdependent.

%In Sec. \ref{sec:txrx_design}, we will be based on \eqref{eq:StokesVectorsdf} in order to build the MVM direct-detection receiver using $N^2-1$ parallel branches. In addition, it is possible to take advantage of the interdependence of the components of $\hat s$ in order to reduce the hardware complexity of the receiver.



For each Jones vector $\left.|s\right\rangle $, we can define the associated projection operator ${\bf S} := | s \rangle \langle s |$, which represents a mode filter, i.e., the equivalent of a polarizer in the two-dimensional case. 
This projection operator can be expressed in terms of the identity matrix and the generalized Gell-Mann matrices \cite{Roudas_JLT_18} 
\begin{equation}
\label{jonesToop:eq}
{\bf S} = \frac{1}{N}{\bf I }+\frac{1}{2 C_N} \hat{s}\cdot {\bf{\Lambda }}.
\end{equation}
    \nomenclature[$s$]{$\mathbf{S}$}{Projection operator dyad, $\mathbf{S}:=\ket{s}\bra{s}$}

By rearranging the terms in \eqref{jonesToop:eq}, we obtain
\begin{equation}
\label{StokesCoefficients:eq}
\hat{s}\cdot {\bf{\Lambda }}=2 C_N\left({\bf S} - \frac{1}{N}{\bf I }\right) .
\end{equation}
 From \eqref{StokesCoefficients:eq}, we see that Stokes vectors arise as  coefficients with respect to a fixed Gell-Mann basis for the \emph{trace neutralized dyad} ${\bf S}-\frac{1}{\d}{\bf I}_N$, assuming  $\ip{s}=1$. 
 
 In the remainder of the article, we will use the Jones vector up to phase $e^{\iota \theta}\ket{s}$, the dyad  ${\bf S}=\dyad{s}$, and the Stokes vector $\hat s$ interchangeably, depending on which one is more convenient.
  In particular, even when we use Jones vectors to represent points in a constellation, since we consider noncoherent detection, we refer to it as a generalized Stokes constellation.

   
