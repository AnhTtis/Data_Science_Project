\noindent We derive the approximate formulas for the error probability stated in Corollary~\ref{asympt:cor} and valid for large SNR parameters $\gamma_s$, as well as the simplified approximation \eqref{mainAsymptOld:eq}. %show the approximation \eqref{mainAsympt:eq} to the error probability for large SNR parameters $\gamma_s$.

%In terms of $\gamma_s$, 
The exact formula \eqref{mainExact:eq} reads 
   \begin{equation}
     \label{mainExact:eqBis}
  \Pbin = Q_1\left( \rho_-\sqrt{\gamma_s},\rho_+\sqrt{\gamma_s} \right)
     -\frac{1}{2} \exp\left({-\frac{\gamma_s}{2}}\right) I_0\left(\frac{\gamma \gamma_s}{2}\right).
   \end{equation}


   For large $x:=\frac{\gamma \gamma_s}{2}$, %{=\aa\bb}$,
    taking the first $n+1$ terms of the Hankel asymptotics given by 
   \cite{benedetto2006principles} % A 11 in the appendix or [Wikipedia],
    \begin{align*}
  I_0(x) &= \frac{e^x}{\sqrt{2\pi x}}
           \left(1 + \frac{1}{8x} + \frac{1 \cdot 9}{2!(8x)^2} + \frac{1 \cdot 9 \cdot 25}{3!(8x)^3}  + \ldots \right),   
   \end{align*}
   yields an approximation to the Bessel term in \eqref{mainExact:eqBis}: 
   \begin{align}
     \label{BesselAsympt:eq}
  \exp\left({-\frac{\gamma_s}{2}}\right) I_0\left(\frac{\gamma \gamma_s}{2}\right)   \sim \Bc_n    \exp \left( {-\frac{\gamma_s(1-\gamma)}{2}}\right),
\end{align}
with 
\begin{align}
     \label{BesselAsymptCoef:eq}
  \Bc_n:= \frac{1}{\sqrt{\pi}}
  \left( \frac{1}{(\gamma \gamma_s)^{\frac{1}{2}}} + \frac{\frac{1}{4}}{(\gamma \gamma_s)^{\frac{3}{2}}} + \frac{\frac{9}{32}}{(\gamma \gamma_s)^{\frac{5}{2}}} + \ldots \right)   
  %\left( \frac{1}{(\gamma \gamma_s)^{\frac{1}{2}}} + \frac{\frac{1}{4}}{(\gamma \gamma_s)^{\frac{3}{2}}} + \frac{\frac{9}{32}}{(\gamma \gamma_s)^{\frac{5}{2}}} + \ldots + \frac{\frac{(2n-3)!!^2}{4^{n-1}(n-1)!}}{(\gamma \gamma_s)^{\frac{2n-1}{2}}} \right).   
\end{align}
where the sum is terminated on  $\frac{(2n-1)!!^2}{4^{n}n!}(\gamma \gamma_s)^{-n-1/2}$ for $n \geq 1$.
%\textcolor{gray}{\tiny Clearly, once  $\gamma_s$ is sufficiently large, this approximation holds uniformly for all $\gamma$ bounded away from $0$(i.e. $\gamma \in [\epsilon_0,1)$), as then $\gamma \gamma_s$ is large.}

A similar asymptotic expansion for the Marcum term $Q_1\left( \rho_-\sqrt{\gamma_s},\rho_+\sqrt{\gamma_s} \right)$ in \eqref{mainExact:eqBis} is more subtle and can be extracted from \cite{GilSeguraTemme2014} in the  form of  a linear combination of the exponential  $\exp\left({-\frac{\gamma_s(1-\gamma)}{2}}\right)$ and the error function $\erfc\left(\frac{\sqrt{\gamma_s}\sqrt{1-\gamma}}{\sqrt{2}} \right)$ with the coefficients described below. (Here $\erfc(x):=\frac{2}{\sqrt{\pi}} \int_x^\infty e^{-t^2}\, dt$.)  %[The details of error bounds and instabilities can be found in \cite{Temme1986}.]

%%%%%%%%%%%%%%%%%%%%%%%%%%%%%%%%%%%%%%%%%%%%%%%%%%%%%%%%%%%%%%%%%%%%%%%%%%%%%%%%%%%%%%%%%%%%%%%%%%%%%%%%%%%




% Legendre duplication formula [Wikipedia] $\Gamma(z)\Gamma(z+\frac{1}{2})=2^{1-2z}\sqrt{\pi}\Gamma(2z)$, with $z=k+\frac{1}{2}$ where $k \in \Z$ gives
% \begin{equation}
%   \Gamma(k+\frac{1}{2}) = 2^{-2k}\sqrt{\pi}\frac{\Gamma(2k+1)}{\Gamma(k+1)}
% \end{equation} 


To start, define $\Ce_n$ and $\Cf_n$ recursively: Let $\Ce_0:=0$ and follow with    
\begin{equation}
  \label{Ce:eq}
  \Ce_n:= \frac{1}{\frac{1}{2}-n}\left( \frac{1-\gamma}{\gamma} \Ce_{n-1} - \left(\frac{\gamma \gamma_s}{2}\right)^{\frac{1}{2}-n} \right) \quad(n\geq 1). 
\end{equation}
 Let $ \Cf_0 := \sqrt{\pi}\sqrt{\frac{\gamma}{1-\gamma}}$ and follow with 
\begin{equation}\label{Cf:eq}
 % \Cf_0 := \sqrt{\pi}\sqrt{\frac{\gamma}{1-\gamma}}, \quad
  \Cf_n := \frac{1}{\frac{1}{2}-n} \frac{1-\gamma}{\gamma} \Cf_{n-1} \quad (n\geq 1).
\end{equation}
%(-1)^n  ((\[Rho]p/\[Rho]m)^m/(2 Sqrt[2 Pi]))   ( AA[n, m - 1] - ( \[Rho]m/ \[Rho]p ) AA[n, m]  ) Ce[n]
%\textcolor{blue}{\scriptsize The above recursions come from recursion (36) in \cite{GilSeguraTemme2014} broken up into exp and erfc components and taking into account that their $\sigma=(1-\gamma)/\gamma$ and their $\xi=\frac{\gamma \gamma_s}{2}$.}
Then, using constants %\textcolor{blue}{\scriptsize from (32) in  \cite{GilSeguraTemme2014}}
% \footnote{\textcolor{gray}{\tiny DERIVATION:  For $n\geq 1$, taking $w:=\frac{1}{2}+m-n$, and using $\Gamma(z+1)=z\Gamma(z)$, 
% \begin{align*}
%   A_{n,m}&=\frac{1}{n!2^n}\frac{\Gamma(w+2n)}{\Gamma(w)} \\
%   &= \frac{1}{n!2^n}\frac{\Gamma(w+1)}{\Gamma(w)}\frac{\Gamma(w+2)}{\Gamma(w+1)}\ldots \frac{\Gamma(w+2n)}{\Gamma(w+2n-1)}\\
%   &= \frac{1}{n!2^n}w(w+1) \ldots (w+2n-1)  \\
%          &= \frac{1}{n!2^n} \underset{\text{$2n$ factors}}{\underbrace{\left(m+1/2-n\right) \ldots \left(m+1/2+(n-1)\right)}}.
% \end{align*}
% For $n=0$, we get
% %\begin{equation}
%   $A_{0,m}:=\frac{\Gamma(1/2+m)}{\Gamma(1/2+m)} =  1.$
% %\end{equation}
% }}
\begin{equation}\label{Acoef:eq}
  A_{n,m}:=\frac{1}{n!2^n}\frac{\Gamma(\frac{1}{2}+m+n)}{\Gamma(\frac{1}{2}+m-n)}
  = \frac{1}{n!2^n} \prod_{i=-n}^{n-1}\left(m+i+\frac{1}{2}\right),  %\quad (n=0,1,2, \ldots)  
\end{equation}
define a multiplier %\textcolor{blue}{\scriptsize extracted from (38) in \cite{GilSeguraTemme2014} and using that their $\rho= \rho_+/\rho_-$} 
\begin{align}
           %            \lambda_n:=\frac{(-1)^n}{2 \sqrt{2\pi}}\frac{\rho_+}{\rho_-} \left( A_{n,0}- \frac{\rho_-}{\rho_+} A_{n,1} \right),
  \lambda_n:=\frac{(-1)^n}{2 \sqrt{2\pi}}\left( \frac{\rho_+}{\rho_-}  A_{n,0}- A_{n,1} \right),
\end{align}
and set
\begin{subequations}
\begin{align}
  \Ce'_n &:=\lambda_n \Ce_n, \\ \Cf'_n &:=\lambda_n \Cf_n, \\ \Ce''_n &:=\sum_{i=0}^n \Ce'_i, \\  \Cf''_n &:=\sum_{i=0}^n \Cf'_i.
\end{align} 
\end{subequations}
We note that $\Cf''_n$ only depends on $\gamma$ while $\Ce''_n$ is a linear combination of the powers %$1$ through $2n-1$ of $(\gamma \gamma_s)^{-\frac{1}{2}}$, namely
$(\gamma\gamma_s)^{-\frac{1}{2}}$, $(\gamma\gamma_s)^{-\frac{3}{2}}$, $\ldots$, $(\gamma\gamma_s)^{-\frac{2n-1}{2}}$ with $\gamma$-dependent coefficients  (for $n \geq 1$).

The approximation given by formula (37) in \cite{GilSeguraTemme2014} reads then
\begin{align}
  \label{MarcumPerGil:eq}
    &\quad Q_1\left( \rho_-\sqrt{\gamma_s},\rho_+\sqrt{\gamma_s} \right) \notag\\
    &\sim \Ce''_n \exp\left({-\frac{\gamma_s(1-\gamma)}{2}}\right) + \Cf''_n \erfc\left(\frac{\sqrt{\gamma_s}\sqrt{1-\gamma}}{\sqrt{2}} \right). 
\end{align}


An important feature of \eqref{MarcumPerGil:eq} is that it is valid
%\textcolor{green}{\scriptsize which equals $2\aa\bb$ and the $2\xi$ in Gil}
uniformly across $\gamma \in (0,1)$ (as long as $\gamma \gamma_s$ is sufficiently large).
%\textcolor{green}{\scriptsize i.e. uniformly for all Gil's $\sigma = \frac{1-\gamma}{\gamma} \in (0, \infty)$}.
In \cite{Temme1986} %\textcolor{green}{\tiny Section 3, just above (3.12)}
 explicit error bounds are discussed together with suitable expansion termination criteria. For our purposes, using $n = 1$ gives excellent results. 

% \textcolor{blue}{\scriptsize Let us comment about the somewhat non-standard nature of the expansion.
%   One may be tempted to also expand the error function itself in the powers of $(\gamma \gamma_s)^{-\frac{1}{2}}$ in order to view this as an asymptotic power expansion but this will not result in a stable expansion as some early terms are altered as one increases $n$. This phenomenon is discussed in  \cite{Temme1986} [Section 3, just above (3.12)]].
% Here we only note that using  the standard asymptotic expansion for the error function (valid for large arguments) would lead to loss of uniformity near $\gamma=1$.
% Indeed, per Wikipedia, the standard asymptotic expansion for large $x>0$ is
% \begin{equation}
%   \label{erfcStdAsym:eq}
%   \erfc(x) \sim \frac{e^{-x^2}}{x\sqrt{\pi}}\left(1-\frac{1}{2x^2}+\frac{3}{(2x^2)^2}-\frac{3 \cdot 5}{(2x^2)^3}+\ldots \right)
% \end{equation}
% and it gives 
% \begin{align*}
%   &\erfc\left(\frac{\sqrt{\gamma_s}\sqrt{1-\gamma}}{\sqrt{2}} \right) \\
%   &\sim \frac{\sqrt{2}}{\sqrt{\pi}} \frac{e^{-\frac{(1-\gamma) \gamma_s}{2}}}{\sqrt{(1-\gamma) \gamma_s}}
%   \left(1-\frac{1}{(1-\gamma)\gamma_s}+\frac{3}{((1-\gamma)\gamma_s)^2}-\ldots \right)
% \end{align*}
% which has all terms blowing up at $\gamma = 1$.
% This is the rationale for keeping an unexpanded erfc in the approximation to the Marcum function.
%   Expressing the expansion in terms of the error function is crucial to achieving a uniform error bound when $\gamma \approx 1$ and the ergument  $\sqrt{(1-\gamma)\gamma_s}$ is too small to rely on the traditional asymptotics \eqref{erfcStdAsym:eq}.}

% %\textcolor{red}{[STOP] Add this expansion per wikipedia}




% \textcolor{blue}{\scriptsize Let us give more detail on the nature of $\Ce''_n$ and $\Cf''_n$.
% The coefficient $\Cf''_n$ does not depend on $\gamma_s$ but does depend on $\gamma$ and can be expressed as an expansion in the powers of $\sqrt{1-\gamma}$: \textcolor{green}{\scriptsize see JKreforgingGilAsymptotics*.nb} 
% \begin{align*}
%   \Cf''_n &= \frac{1}{2\sqrt{2}} \sqrt{\frac{\gamma}{1-\gamma}} \left\{
%             \left( \frac{\rho_+}{\rho_-}-1 \right) + (1-\gamma)\ldots \right\} \\
%           &= \frac{1}{2\sqrt{2}} \sqrt{\frac{\gamma}{1-\gamma}} \left\{
%             \sqrt{2}\sqrt{\frac{1-\gamma}{1-\delta}} + (1-\gamma) \ldots \right\}\\
%           % &= \frac{1}{2\sqrt{2}} \sqrt{\frac{\gamma}{1-\delta}}
%           %   - \Hc_1(\gamma) (1-\gamma)^{\frac{1}{2}} + \Hc_2(\gamma) (1-\gamma)^{\frac{3}{2}} - \ldots + (-1)^n \Hc_n(\gamma) (1-\gamma)^{n-\frac{1}{2}} \\
%           &= \frac{1}{2} \sqrt{\frac{\gamma}{1-\delta}} \\
%           &- \Hc_{n,1}(\gamma) \left(\frac{1-\gamma}{\gamma}\right)^{\frac{1}{2}} + \Hc_{n,2}(\gamma) \left(\frac{1-\gamma}{\gamma}\right)^{\frac{3}{2}} - \ldots \\
%   &+ (-1)^n \Hc_{n,n}(\gamma) \left(\frac{1-\gamma}{\gamma}\right)^{n-\frac{1}{2}} 
% \end{align*}
% where the (sub-)coefficients $\Hc_{n,i}(\gamma)$ are regular with non-zero (actually positive) limits at $\gamma=1$.
% The coefficient $\Ce''_n$ depends on $\gamma_s$ and can be developed in the powers of $\sqrt{\gamma \gamma_s}$:
%   %is a linear combination of $\gamma_s^{-\frac{1}{2}}, \gamma_s^{-\frac{3}{2}} \ldots, \gamma_s^{-\frac{2n-1}{2}}$ 
% \begin{align*}
%   \Ce''_n &= \Jc_{n,1}(\gamma) \left(\gamma \gamma_s \right)^{-\frac{1}{2}} + \Jc_{n,2}(\gamma) \left(\gamma \gamma_s \right)^{-\frac{3}{2}} +  \ldots + \Jc_{n,n}(\gamma) \left(\gamma \gamma_s \right)^{-n+\frac{1}{2}} 
% \end{align*}
% where the (sub-)coefficients $\Jc_{n,i}(\gamma)$ are regular with non-zero (actually positive) limits at $\gamma=1$.
% } % END gray 
% %%%%%%%%%%%%%%%%%%%%%%%%%%%%%%%%%%%%%%%%%%%%%%%%%%%%%%%%%%%%%%%%%%%%%%%%%%%%%%%%%%%%%%%%%%%%%%%%%%%%%%%%%%


To approximate the error probability $\Pbin$, as given by \eqref{mainExact:eqBis}, we combine the Marcum and Bessel approximations, \eqref{MarcumPerGil:eq} and \eqref{BesselAsympt:eq}, and obtain:
\begin{align}
  \label{ultimateApprox:eq}
     \Pbin  %&= Q_1\left( \rho_-\sqrt{\gamma_s},\rho_+\sqrt{\gamma_s} \right)
               % -\frac{1}{2}e^{-\frac{\gamma_s}{2}}I_0\left(\frac{\gamma \gamma_s}{2}\right) \\
  &\sim   \Cf''_n \erfc\left(\frac{\sqrt{\gamma_s}\sqrt{1-\gamma}}{\sqrt{2}} \right) \notag\\
  &\quad + \left(\Ce''_n -\frac{1}{2}\Bc_{n} \right) \exp\left({-\frac{\gamma_s(1-\gamma)}{2}}\right).
\end{align}
%where  $n, n'$ are chosen to achieve a desired level of accuracy. Taking $n=n'=1$ is sufficient for our purposes. 
%The coeficient $\Ce'''_n -\frac{1}{2}\Bc_n$ %where $\Ce''_n:=\Ce''+\Ce^{\text{Bessel}}$
% \textcolor{blue}{\scriptsize For $n=0$, $\Cf''_0 \neq 0$ so the erfc term in \eqref{ultimateApprox:eq} would have the leading term of order
% $\gamma_s^{-\frac{1}{2}}e^{-\frac{\gamma_s(1-\gamma)}{2}}$, should we use the standard expansion \eqref{erfcStdAsym:eq} of erfc. The Bessel term would also have the leading order  $\gamma_s^{-\frac{1}{2}}e^{-\frac{\gamma_s(1-\gamma)}{2}}$. This pattern continues for $n \geq 1$ if we keep on adding the terms to the erfc expansion and justifies using the first $n+1$ (not $n$) terms of the Hankel expansion \eqref{BesselAsymptCoef:eq}. However, see Remark~\ref{truncEffect:rmk} for a surprise.} 
%%%%%%%%%%%%%%%%%%%%%%%%%%%%%%%%%%%%%%%%%%%%%%%%%%%%%%%%%%%%%%%%%%%%%%%%%%%%%%%%%%%%%%%%%%%%%%%%%%%%%%%%%%
Corollary~\ref{asympt:cor} will follow now by using  $n=0,1$ in  \eqref{ultimateApprox:eq}. % for $n=0,1$.
 

%%%%%%%%%%%%%%%%%%%%%%%%%%%%%%%%%%%%%%%%%%% PREAMBLE  %%%%%%%%%%%%%%%%%%%%%%%%%%%%%%%%%%%%%%%%%%%%%%%%%%%%%%%%%%%%%%%

%PREAMBLE

To streamline formulas we reach back to \eqref{deltarhodef:eq}, \eqref{deltarhodef:eq2}, and  note
\begin{equation}
  \label{rhoRatio2:eq}
  \frac{\rho_+}{\rho_-}%=\frac{\sqrt{1+\delta}}{\sqrt{1-\delta}}
  % =\frac{\sqrt{1+\delta}\textcolor{gray}{\sqrt{1+\delta}}}{\sqrt{1-\delta}\textcolor{gray}{\sqrt{1+\delta}}}
  =\frac{\sqrt{1+\delta}}{\sqrt{1-\delta}}
 =\frac{1+\delta}{\sqrt{1-\delta^2}}=\frac{1+\delta}{\gamma}.
\end{equation}
Also,  squaring as follows 
\begin{equation}
    (\rho_+ \pm \rho_-)^2 = \left( \sqrt{\frac{1+\delta}{2}} \pm \sqrt{\frac{1-\delta}{2}}\right)^2 = 1 \pm \sqrt{1-\delta^2},
\end{equation}
 gives \begin{equation}
  \label{rhoPMaux:eq}
   \rho_+ \pm \rho_- = \sqrt{1 \pm \gamma}.
 \end{equation}
In particular, 
\begin{equation}
  \label{rhoRatio1:eq}
   \frac{\rho_+}{\rho_-}-1=\frac{\rho_+-\rho_-}{\rho_-}= \sqrt{2}\sqrt{\frac{1 - \gamma}{1-\delta}}.
 \end{equation}
% Also
% \begin{equation}
%   \label{rhoRatio2:eq}
%   \frac{\rho_+}{\rho_-}%=\frac{\sqrt{1+\delta}}{\sqrt{1-\delta}}
%   % =\frac{\sqrt{1+\delta}\textcolor{gray}{\sqrt{1+\delta}}}{\sqrt{1-\delta}\textcolor{gray}{\sqrt{1+\delta}}}
%   =\frac{\sqrt{1+\delta}/2}{\sqrt{1-\delta}/2}
%  =\frac{1+\delta}{\sqrt{1-\delta^2}}=\frac{1+\delta}{\gamma}.
%  \end{equation}


%%%%%%%%%%%%%%%%%%%%%%%%%%%%%%%%%%%%%%%%%%% n=0 %%%%%%%%%%%%%%%%%%%%%%%%%%%%%%%%%%%%%%%%%%%%%%%%%%%%%%%%%%%%%%%

Taking $n=0$, we find  %\eqref{Acoef:eq}
$A_{0,0}=1$ and $A_{0,1}=1$, so %\textcolor{green}{\tiny Verified Against JKreforgingGilAsymptotics1.0.nb +}
\begin{align}
  \label{CF0:eq}
  \Cf_0'' =  \Cf_0' = \lambda_0 \Cf_0  &= \frac{1}{2\sqrt{2\pi}} \left(\frac{\rho_+}{\rho_-}-1\right)\sqrt{\pi}\sqrt{\frac{\gamma}{1-\gamma}} \notag \\
                                     %  %&= \frac{1}{2\sqrt{2}} \left(\frac{\rho_+}{\rho_-}-1\right)\sqrt{\frac{\gamma}{1-\gamma}}\\
                                % &\textcolor{gray}{= \frac{1}{2\sqrt{2}} \sqrt{2}\sqrt{\frac{1 - \gamma}{1-\delta}}\sqrt{\frac{\gamma}{1-\gamma}} } \notag \\
    &= \frac{1}{2}\sqrt{\frac{\gamma}{1-\delta}} \ ,
\end{align}
where we used \eqref{rhoRatio1:eq}. 
% \begin{equation}
%   \label{rhoRatio1:eq}
%    \frac{\rho_+}{\rho_-}-1=\frac{\rho_+-\rho_-}{\rho_-}= \sqrt{2}\sqrt{\frac{1 - \gamma}{1-\delta}}.
%  \end{equation}
% The latter is seen by reaching back to \eqref{deltarhodef:eq} and  squaring
% $$(\rho_+ \pm \rho_-)^2 = \left( \sqrt{\frac{1+\delta}{2}} \pm \sqrt{\frac{1-\delta}{2}}\right)^2 = 1 \pm \sqrt{1-\delta^2},$$
%  which gives $\rho_+ \pm \rho_- = \sqrt{1 \pm \gamma}$.
 Plugging \eqref{CF0:eq} and $\Ce''_0=0$ and $\Bc_0=(\gamma \gamma_s)^{-1/2}/\sqrt{\pi}$ (from \eqref{BesselAsymptCoef:eq}) into  \eqref{ultimateApprox:eq}  reproduces \eqref{mainAsymptZero:eq}, the first formula in  Corollary~\ref{asympt:cor}:
  \begin{align}
    \label{mainAsymZeroBis:eq}
    \Pbin
    \sim  
     &\frac{1}{2} \sqrt{\frac{\gamma}{1-\delta}} \erfc\left(\frac{\sqrt{\gamma_s}\sqrt{1-\gamma}}{\sqrt{2}} \right) \notag \\
    &\quad - \frac{1}{2\sqrt{\pi \gamma \gamma_s}} \exp\left({-\frac{\gamma_s(1-\gamma)}{2}}\right).
 \end{align}
 % \begin{align*}
 %    \Pbin &\sim 
 %                 \frac{1}{2} \sqrt{\frac{1+\delta}{\gamma}}        
 %                    \erfc\left(\frac{\sqrt{\gamma_s}\sqrt{1-\gamma}}{\sqrt{2}} \right)
 %                    - \frac{1}{2\sqrt{\pi \gamma \gamma_s}} e^{-\frac{\gamma_s(1-\gamma)}{2}}
 %                     \end{align*}
 %                  }
 % end gray 

 %%%%%%%%%%%%%%%%%%%%%%%%%%%%%%%%%%%%%%%%%%% END n=0 %%%%%%%%%%%%%%%%%%%%%%%%%%%%%%%%%%%%%%%%%%%%%%%%%%%%%%%%%%%%%%%


%%%%%%%%%%%%%%%%%%%%%%%%%%%%%%%%%%%%%%%%%%% n=1 %%%%%%%%%%%%%%%%%%%%%%%%%%%%%%%%%%%%%%%%%%%%%%%%%%%%%%%%%%%%%%%

 
 Taking $n=1$, we find
 $A_{1,0}=-\frac{1}{8}$ and $A_{1,1}=\frac{3}{8}$, %(per \eqref{Acoef:eq})
  so  %\textcolor{green}{\tiny Verified Against JKreforgingGilAsymptotics1.0.nb +}% \footnote{
  % \textcolor{gray}{KILL detail:
  %   $$A_{1,0}=\frac{1}{2}\Gamma(3/2)/\Gamma(-1/2)=\frac{1}{2}\Gamma(3/2)/\Gamma(1/2) \cdot \Gamma(1/2)/\Gamma(-1/2)=-\frac{1}{8}$$ and
  % $$A_{1,1}=\frac{1}{2}\Gamma(5/2)/\Gamma(1/2)=\frac{1}{2}\Gamma(5/2)/\Gamma(3/2)\Gamma(3/2)/\Gamma(1/2)=\frac{3}{8}$$ PER explicit values or last footnote}}
%$\Cf_0 := \sqrt{\pi}\sqrt{\frac{\gamma}{1-\gamma}}$, followed by 
\begin{align}
  \Cf_1' = \Cf_1 \lambda_1 &= \frac{1}{\frac{1}{2}-1} \frac{1-\gamma}{\gamma} \sqrt{\pi}\sqrt{\frac{\gamma}{1-\gamma}}
                             \frac{(-1)^1}{2 \sqrt{2\pi}}\left( -\frac{\rho_+}{\rho_-}\frac{1}{8}- \frac{3}{8} \right)    \notag  \\
                           %&\textcolor{gray}{=  \frac{1-\gamma}{\gamma}\sqrt{\frac{\gamma}{1-\gamma}} \frac{1}{8\sqrt{2}}\left( - \frac{\rho_+}{\rho_-}- 3 \right) \notag } \\
                           &=  - \sqrt{\frac{1-\gamma}{\gamma}}
                             \frac{1}{8\sqrt{2}}\left( \frac{\rho_+}{\rho_-}+ 3 \right) \notag \\
   &= - \sqrt{\frac{1-\gamma}{\gamma}}  \frac{1}{8\sqrt{2}}\left(\frac{1+\delta}{\gamma} + 3 \right)
\end{align}
where we used \eqref{rhoRatio2:eq}. 
% \begin{equation}
%   \label{rhoRatio2:eq}
%   \frac{\rho_+}{\rho_-}%=\frac{\sqrt{1+\delta}}{\sqrt{1-\delta}}
%   =\frac{\sqrt{1+\delta}\textcolor{gray}{\sqrt{1+\delta}}}{\sqrt{1-\delta}\textcolor{gray}{\sqrt{1+\delta}}}=\frac{1+\delta}{\gamma}.
%  \end{equation}
Thus, using \eqref{CF0:eq}, we arrive with 
\begin{align}
  \label{cfff:eq}
  \Cf_1'' = \Cf_0' +  \Cf_1'%  &=  \frac{1}{2\sqrt{2}} \left(\frac{\rho_+}{\rho_-}-1\right)\sqrt{\frac{\gamma}{1-\gamma}} \\
  %                  &- \sqrt{\frac{1-\gamma}{\gamma}}  \frac{1}{8\sqrt{2}}\left( \frac{\rho_+}{\rho_-}+ 3 \right) \\
  % &=\frac{1}{2\sqrt{2}} \sqrt{2}\frac{\sqrt{1 - \gamma}}{\sqrt{1-\delta}}\sqrt{\frac{\gamma}{1-\gamma}} \\
  %                    &- \sqrt{\frac{1-\gamma}{\gamma}}  \frac{1}{8\sqrt{2}}\left(\frac{1+\delta}{\gamma} + 3 \right)\\
  % &=\frac{1}{2} \sqrt{\frac{\gamma}{1-\delta}} \\
  %                  &- \sqrt{\frac{1-\gamma}{\gamma}}  \frac{1}{8\sqrt{2}}\left(\frac{1+\delta}{\gamma} + 3 \right) \\
                   &=\frac{1}{2} \sqrt{\frac{\gamma}{1-\delta}}
                     - \sqrt{\frac{1-\gamma}{\gamma}}  \frac{1}{8\sqrt{2}}\left(\frac{1+\delta}{\gamma} + 3 \right).
\end{align}

Turning attention to $\Ce''_1=\Ce_1' = \Ce_1 \lambda_1$, we have 
                     %\textcolor{green}{\tiny Verified Against JKreforgingGilAsymptotics1.0.nb, but the notebook now evolved into JKreforgingGilAsymptoticsMinimalistic3.1.nb}
\begin{align}
  \Ce_1''  &= \frac{1}{\frac{1}{2}-1}\left( - \left(\frac{\gamma \gamma_s}{2}\right)^{\frac{1}{2}-1} \right)\frac{(-1)^1}{2 \sqrt{2\pi}}\left( -\frac{\rho_+}{\rho_-}\frac{1}{8}- \frac{3}{8} \right)  \notag \\
    &=  \frac{1}{8 \sqrt{\pi}}\left( \frac{\rho_+}{\rho_-}+3 \right)\left(\gamma \gamma_s\right)^{-\frac{1}{2}}.
\end{align}
Fetching  $\Bc_0 $ from  %the Bessel expansion
 \eqref{BesselAsymptCoef:eq} and then using \eqref{rhoRatio1:eq} gives  
\begin{align}\label{cfffBis:eq}
  \Ce_1'' - \frac{1}{2}\Bc_0 &=  \frac{1}{8 \sqrt{\pi}}\left( \frac{\rho_+}{\rho_-}+3 \right) \left(\gamma \gamma_s\right)^{-\frac{1}{2}}-
                              \frac{1}{2} \frac{1}{\sqrt{\pi}} (\gamma \gamma_s)^{-\frac{1}{2}} \notag \\
  %&=   \left\{ \frac{1}{8 \sqrt{\pi}}\left( \frac{\rho_+}{\rho_-}+3 \right) - \frac{1}{2} \frac{1}{\sqrt{\pi}} \right\} \left(\gamma \gamma_s\right)^{-\frac{1}{2}} \notag \\
                            %%&=  \left(\gamma \gamma_s\right)^{-\frac{1}{2}} \left\{ \frac{1}{8 \sqrt{\pi}} \frac{\rho_+}{\rho_-} -  \frac{1}{8\sqrt{\pi}} \right\}  \\
                            &=  
                              \frac{1}{8\sqrt{\pi}} \left( \frac{\rho_+}{\rho_-} - 1 \right)\left(\gamma \gamma_s\right)^{-\frac{1}{2}} 
  \notag \\
                            &= 
                              \frac{\sqrt{2}}{8\sqrt{\pi}} \sqrt{\frac{1-\gamma}{1-\delta}} \left(\gamma \gamma_s\right)^{-\frac{1}{2}}. 
\end{align}
Subtracting one more term of the Bessel expansion  \eqref{BesselAsymptCoef:eq} yields
\begin{align}\label{ceee:eq}
  \Ce_1'' - \frac{1}{2}\Bc_1  &=  
                               \frac{\sqrt{2}}{8\sqrt{\pi}} \sqrt{\frac{1-\gamma}{1-\delta}}\left(\gamma \gamma_s\right)^{-\frac{1}{2}}   - \frac{1}{8\sqrt{\pi}}\left(\gamma \gamma_s\right)^{-\frac{3}{2}}.
\end{align}

%%%%%%%%%%%%%%%%%%%%%%%%%%%%%%%%%%%%%%%%%%%
%%%%%%%%%%%%%%%%%%%%%%%%%%%%%%%%%%%%%%%%%%%

One can check now that plugging \eqref{ceee:eq} and \eqref{cfff:eq} into \eqref{ultimateApprox:eq}  reproduces  \eqref{mainAsympt:eq}, the second formula
  in  Corollary~\ref{asympt:cor}.
% \textcolor{red}{[NEW 14 Nov 2021 TRANSPLANT: 
%  \begin{align*}
%    &\Pbin \\
%    &\sim     
%              \left\{ \frac{1}{2} \sqrt{\frac{\gamma}{1-\delta}}
%      - \sqrt{\frac{1-\gamma}{\gamma}}  \frac{1}{8\sqrt{2}}\left(\frac{1+\delta}{\gamma} + 3 \right) \right\}
%      \erfc\left(\frac{\sqrt{\gamma_s}\sqrt{1-\gamma}}{\sqrt{2}} \right) \\
%              &+ \frac{1}{8\sqrt{\pi}}\left\{\sqrt{2} \sqrt{\frac{1-\gamma}{1-\delta}}\left(\gamma \gamma_s\right)^{-\frac{1}{2}}   - \left(\gamma \gamma_s\right)^{-\frac{3}{2}}   \right\}  e^{-\frac{\gamma_s(1-\gamma)}{2}}
%  \end{align*}
 % end gray 

  \begin{rmk}
    \label{truncEffect:rmk}
   Dropping the $\left(\gamma \gamma_s\right)^{-\frac{3}{2}}$ term in the second formula  in Corollary~\ref{asympt:cor} yields 
\begin{align}
   \Pbin 
   &\sim     
             \left[ \frac{1}{2} \sqrt{\frac{\gamma}{1-\delta}}
     - \sqrt{\frac{1-\gamma}{\gamma}}  \frac{1}{8\sqrt{2}}\left(\frac{1+\delta}{\gamma} + 3 \right) \right]  \notag \\ 
    &\quad\times\erfc\left(\frac{\sqrt{\gamma_s}\sqrt{1-\gamma}}{\sqrt{2}} \right) \notag \\
    &\quad + \frac{\sqrt{2}}{8\sqrt{\pi}}\sqrt{\frac{1-\gamma}{1-\delta}}\left(\gamma \gamma_s\right)^{-\frac{1}{2}} \exp\left({-\frac{\gamma_s(1-\gamma)}{2}}\right).
 \end{align}
This is a somewhat looser approximation for very large $\gamma_s$
  but works well for moderate values of $\gamma_s$ of interest in our applications. %(This effect seems to hold uniformly over $\gamma$.)
 \end{rmk}


 
 %\pagebreak
%%%%%%%%%%%%%%%%%%%%%%%%%%%%%%%%%%%%%%%%%%%%%%%%%%%%%%%%%%%%%%%%%%%%%%%%%%%%%%%%%%%%%%%%%%%%%%%%%%%%%%%%%%%%%%%%%%%
%%%%%%%%%%%%%%%%%%%%%%%%%%%%%%%%%%%%%%%%%%%%%% END WORK ZONE %%%%%%%%%%%%%%%%%%%%%%%%%%%%%%%%%%%%%%%%%%%%%%%%%%%%%%%%%%%%%%%%%%%%%
%%%%%%%%%%%%%%%%%%%%%%%%%%%%%%%%%%%%%%%%%%%%%%%%%%%%%%%%%%%%%%%%%%%%%%%%%%%%%%%%%%%%%%%%%%%%%%%%%%%%%%%%%%%%%%%%%%%

 %%%%%%%%%%%%%%%%%%%%%%%%%%%%%%%%%%%%%%%%%%%%%%%%%%%%%%%%%%%%%%%%%%%%%%%%%%%%%%%%%%%%%%%%%%%%%%%%%%%%%%%%%%%%%%%%%%%
%%%%%%%%%%%%%%%%%%%%%%%%%%%%%%%%%%%%%%%%%%%%%% BEGIN SECONDARY WORK ZONE %%%%%%%%%%%%%%%%%%%%%%%%%%%%%%%%%%%%%%%%%%%%%%%%%%%%%%%%%%%%%%%%%%%%%
%%%%%%%%%%%%%%%%%%%%%%%%%%%%%%%%%%%%%%%%%%%%%%%%%%%%%%%%%%%%%%%%%%%%%%%%%%%%%%%%%%%%%%%%%%%%%%%%%%%%%%%%%%%%%%%%%%%
 It remains to derive the crude approximation \eqref{mainAsymptOld:eq}. When  $\sqrt{x}=\frac{\sqrt{\gamma_s}\sqrt{1-\gamma}}{\sqrt{2}}$ is large, which happens for large $\gamma_s$ when $\gamma$ is not too close to $1$, the simple standard asymptotics
 %$\erfc(x)\sim \frac{1}{\sqrt{\pi}x}e^{-x^2}$ 
 $\erfc(\sqrt{x})\sim \frac{1}{\sqrt{\pi x}}e^{-x}$ 
 is viable and, when substituted into  \eqref{mainAsymZeroBis:eq}, yields  \eqref{mainAsymptOld:eq}: %\eqref{mainAsymptOld:eq} as follows:
  \begin{align}
    \label{mainAsymptOldBis:eq}
    &\quad\Pbin \notag\\ 
         % &\sim \frac{1}{2} \sqrt{\frac{\gamma}{1-\delta}}        
         %            \erfc\left(\frac{\sqrt{\gamma_s}\sqrt{1-\gamma}}{\sqrt{2}} \right)
         %   - \frac{1}{2\sqrt{\pi \gamma \gamma_s}} e^{-\frac{\gamma_s(1-\gamma)}{2}} \notag \\
  &\sim 
                 \frac{1}{2}\left[\sqrt{\frac{\gamma}{1-\delta}}  \frac{\sqrt{2}}{\sqrt{\pi}\sqrt{\gamma_s}\sqrt{1-\gamma}}  
    - \frac{1}{\sqrt{\pi \gamma \gamma_s}} \right] 
    \exp\left( {-\frac{\gamma_s(1-\gamma)}{2}}\right) \notag   \\
         &= \frac{1}{2}\left[ \frac{\sqrt{2}\gamma}{\sqrt{1-\delta}\sqrt{1-\gamma}}  
           - 1 \right] \frac{1}{\sqrt{\pi \gamma \gamma_s}} 
           \exp\left( {-\frac{\gamma_s(1-\gamma)}{2}}\right)  \notag \\
         % &= 
         %         \frac{1}{2}\left\{\frac{\sqrt{2}\sqrt{1+\delta}\textcolor{gray}{\sqrt{1-\delta}} }{\sqrt{1-\gamma}\textcolor{gray}{\sqrt{1-\delta}}}  
         %            - 1 \right\} \frac{1}{\sqrt{\pi \gamma \gamma_s}} e^{-\frac{\gamma_s(1-\gamma)}{2}}  \notag  \\
         &=
                 \frac{1}{2} \frac{\sqrt{1+\gamma}}{\sqrt{1-\gamma}}  
                  \frac{1}{\sqrt{\pi \gamma \gamma_s}} 
                  \exp\left( {-\frac{\gamma_s(1-\gamma)}{2}}\right) ,
\end{align}
where the last equality can be seen by using 
\begin{equation}
  \sqrt{1+\gamma} + \sqrt{1-\gamma} = \sqrt{2}\sqrt{1+\delta},
\end{equation}
which itself is evident  (from $\delta=\sqrt{1-\gamma^2}$) after squaring.

It is worth recording (cf. \eqref{eq:asymptoticBehavior})  
 that \eqref{mainAsymptOldBis:eq} %is the desired approximation \eqref{mainAsymptOld:eq},
 can be  also rewritten as
  \begin{align}
   \boxed{ \Peb^{m' | m}  %\Prob(\sqrt{\phi_-} \geq \sqrt{\phi_+})
    \sim \frac{1}{\sqrt{2\pi}}
    \frac{   \sqrt{1+\frac{1}{1-\dD^2/2}} }{ \dD } \frac{1}{\sqrt{\gamma_s}}  
    \exp\left( -\frac{1}{2}\gamma_s\frac{\dD^2}{2} \right)
    } % end boxed
  \end{align}
%   \begin{align}
%     \Peb^{m' | m}  %\Prob(\sqrt{\phi_-} \geq \sqrt{\phi_+})
%     \sim \frac{1}{2} \frac{1}{\sqrt{2\pi}}
%     \frac{\sqrt{1+\frac{1}{1-\dD^2}}}{\dD}  \frac{1}{\sqrt{\gamma_s}}  \exp\left( {\frac{-\gamma_s \dD^2}{2}} \right)
%   \end{align}
  where $\dD$ stands for $\dD(s_m,s_{m'})=\sqrt{2}\sqrt{1-\gamma}$, the incoherent distance 
  %$\dD(\sv_m, \sv_{m'})$ 
  between the two symbols 
  %$\sv_m$ and $\sv_{m'}$ 
  $s_m$ and $s_{m'}$ (as given by \eqref{eq:ddDistUnit}).
  
    
%     \bigskip
%   \textcolor{red}{JK CORRECTED to account for the factor $\sqer{2}$ in def of $\dD$ CHECK
%     \begin{align}
%     \Peb^{m' | m}  %\Prob(\sqrt{\phi_-} \geq \sqrt{\phi_+})
%     \sim \frac{1}{\sqrt{2\pi}}
%     \frac{   \sqrt{1+\frac{1}{1-\dD^2/2}} }{ \dD } \frac{1}{\sqrt{\gamma_s}}  
%     \exp\left( -\frac{1}{2}\gamma_s\frac{\dD^2}{2} \right)
%   \end{align}
%   Looking at this we should have removed the $\sqrt{2}4 from the definition of $\dD$ ... but then we would lose the linck with $\dC$. THINK WHAT's BEST marketing wise.
%   }