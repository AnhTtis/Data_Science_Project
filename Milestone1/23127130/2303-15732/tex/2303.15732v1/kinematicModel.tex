\subsection{Kinematic Model}\label{sec:kinematic}

\begin{figure}
    \centering
    \includegraphics[width=\columnwidth]{images/SnakeHelix.png}
    \caption{The kinematic model of the \robot{}.}
    \label{fig:kinematicModel}
\end{figure}

To build a kinematic model, we utilize an n-link approximation of the continuous body. The approximation assists in predicting the shape change of the \robot{} by manipulating the joint angles within the n-link approximation. Since sidewinding motion is characterized by static ground contacts, we use this assumption to constrain the movement of the body. This constraint upon the shape change of the body determines how the \robot{} locomotes. We first establish a sinusoidal actuation cycle for the actuator pairs, offset by a phase of $\frac{\pi}{2}$,
%
\begin{equation}
    \begin{aligned}
        a_1 = \cos(t), \\
        a_2 = \sin(t),
    \end{aligned}
\end{equation}
%
where $a_1$ and $a_2$ are the actuation values of each pair of actuators. We create a relation between actuation values and curvature,
%
\begin{equation}
    \begin{aligned}
        \alpha_1 &= a_1\cos(s_i)-a_2\sin(s_i), \\
        \alpha_2 &= a_1\sin(s_i)+a_2\cos(s_i),
    \end{aligned}
\end{equation}
%
where $\alpha_1$ is denoted as the \emph{curvature} along the body, $\alpha_2$ is the \emph{bicurvature}, and $s_i$ is the link index along the backbone of the body. We assume a torsion-free model, in which the shape of the backbone is entirely determined by curvature, with zero torsional effects. This allows us to create a flow vector solely influenced by the \emph{curvature} and \emph{bicurvature},
%
\begin{equation}
    \label{eq:flowvector}
    \omega =
    \begin{bmatrix}
        0 \\
        \alpha_1 \\
        \alpha_2
    \end{bmatrix}.
\end{equation}
%
We can then place \cref{eq:flowvector} into a skew-symmetric matrix, describing the angular velocity $\widehat{\omega}$,
%
\begin{equation}
    \widehat{\omega} = 
    \begin{bmatrix}
        0 & -\omega_3 & \omega_2 \\
        \omega_3 & 0 & -\omega_1 \\
        -\omega_2 & \omega_1 & 0
    \end{bmatrix}.
\end{equation}
%
To obtain the rotation matrices at each link, we take the matrix exponential of the angular velocity multiplied by the flow rate along the backbone,
%
\begin{equation}
    R_{i} = \exp{(\widehat{\omega}\Delta s)},
\end{equation}
%
in which $\Delta s$ represents the flow rate along the backbone. With the resulting set of rotation matrices, we can then take its cumulative product to build up the body of the kinematic model, as shown in \cref{fig:kinematicModel}.

With the kinematic model, we can then determine how the body displaces itself as it changes shape. We begin by identifying the points along the body that contact the ground, with the aim of keeping them static after each time step. We then calculate the Jacobian of the joint angles $\alpha$ with respect to the actuation amplitudes $a$,
%
\begin{equation}
    \label{eq:jacobianalpha}
    J_{\alpha/a} = 
    \begin{bmatrix}
        \frac{\partial\alpha_1}{\partial a_1} & \frac{\partial\alpha_1}{\partial a_2} \\
        \frac{\partial\alpha_2}{\partial a_1} & \frac{\partial\alpha_2}{\partial a_2}
    \end{bmatrix}.
\end{equation}
%
We also determine how each ground contact moves with the change in joint angles by taking the cross product of the axes and the link vector,
%
\begin{equation}
    \label{eq:jacobianmotion}
    J_{p/\alpha} = 
    \begin{bmatrix}
        \Omega_y\times r_i & \Omega_z\times r_i
    \end{bmatrix},
\end{equation}
%
in which $\Omega_y$, $\Omega_z$ are the y and z axes in the local body frame. $r_i$ represents the vector from the base of the \robot{} to the $i$th joint. Combining \cref{eq:jacobianalpha} and \cref{eq:jacobianmotion},
%
\begin{equation}
    \label{eq:shapepfaffian}
    J_{p/a} = J_{p/\alpha}J_{\alpha/a},
\end{equation}
%
we determine how each ground contact would move with changes in the actuation amplitudes. These ground contacts can also move with changes in position of the initial link, or the \emph{tail}, determined by the spatial velocity Jacobian,
%
\begin{equation}
    \label{eq:positionpfaffian}
    J_{p/g} = 
    \begin{bmatrix}
        \begin{pmatrix}
            1 & & \\
            & 1 & \\
            & & 1
        \end{pmatrix}
        &
        \begin{pmatrix}
            0 & z & -y \\
            -z & 0 & x \\
            y & -x & 0
        \end{pmatrix}
    \end{bmatrix}.
\end{equation}

In order to keep the ground contacts static, we show that \cref{eq:shapepfaffian} and \cref{eq:positionpfaffian} must result in a net zero displacement,
%
\begin{equation}
    \label{eq:gcirc}
    J_{p/g}\groupderiv{g} + J_{p/a}\dot{a} = 0,
\end{equation}
%
where $\groupderiv{g}$ represents the body velocity of the tail, and $\dot{a}$ is the change in the actuation amplitudes. We then isolate $\groupderiv{g}$,
%
\begin{equation}
    \label{eq:gcirc}
    \groupderiv{g} = J_{p/g}^{-1}J_{p/a}\dot{a},
\end{equation}
%
to determine the movement of the tail. The body velocity, $\groupderiv{g}$, is then applied to the orientation of the next time step to determine how the \robot{} would move.