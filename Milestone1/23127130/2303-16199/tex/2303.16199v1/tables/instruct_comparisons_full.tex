% Please add the following required packages to your document preamble:
% \usepackage{graphicx}
\begin{framed}
\centering
 \begin{minipage}{0.98\textwidth}
{\bf Instruction:} Tell me about alpacas. \\

{\bf LLaMA-Adapter:} Alpacas are members of the camelid family and are native to the Andes Mountains of South America. They are typically found in herds of 10-20 animals and are known for their gentle nature and their ability to thrive in harsh climates. Alpacas are typically white or grey in color and have long, silky coats. They are also known for their ability to produce high-quality wool, which is used in the production of clothing and other products. \\

{\bf Alpaca-LoRA:} Alpacas are members of the camelid family and are native to the Andes Mountains of South America. They are known for their soft, luxurious fleece, which is used to make clothing, blankets, and other items. Alpacas are herbivores and graze on grasses and other plants. They are social animals and live in herds of up to 20 individuals.  \\

{\bf Alphaca:} Alpacas are small, fluffy animals related to camels and llamas. They are native to Peru and Bolivia, and were first domesticated around 5,000 years ago. They are kept mainly for their fine, soft fleece, which is used to make knitwear and other garments. Alpacas are herd animals and live in small family groups, led by an older male. They are highly social animals and can form strong bonds with their owners.      \\

{\bf GPT-3:} Alpacas are a domesticated species of South American camelid, similar to llamas. They are native to the Andes Mountains in South America and are kept as livestock for their soft, luxurious wool. Alpacas are social animals and live in herds of up to 20 individuals. They are typically kept in small herds of two to five animals, and are relatively easy to care for. Alpacas are herbivores and feed on grass, hay, and other vegetation. They are also known for their gentle and friendly nature, making them popular as pets.
\end{minipage}
\end{framed}

\begin{framed}
\centering
 \begin{minipage}{0.98\textwidth}
{\bf Instruction:} Tell me about the president of Mexico in 2019. \\

{\bf LLaMA-Adapter:} The president of Mexico in 2019 was Andrés Manuel López Obrador, who took office on December 1st, 2018. He is a member of the left-wing party Morena and is known for his populist policies. He has focused on tackling corruption, reducing poverty, and improving the country's infrastructure. He has also been a vocal critic of the United States' immigration policies. \\

{\bf Alpaca-LoRA:} The president of Mexico in 2019 was Andrés Manuel López Obrador, who took office on December 1, 2018. He is a member of the National Regeneration Movement (MORENA) political party and is the first left-wing president of Mexico since 1946. He is known for his anti-corruption and anti-neoliberal policies, as well as his commitment to improving the living conditions of the Mexican people.  \\

{\bf Alphaca:} In 2019, Andres Manuel Lopez Obrador was sworn in as President of Mexico. He is the country’s 58th president and the first from the left-wing Morena party. He was elected in 2018 on a promise to combat corruption and reduce poverty and has faced criticism for his handling of the economy, crime and the country’s relationship with the United States. He is also known for his ambitious social welfare programs and has been praised for raising the minimum wage and providing aid to low-income families. He is seen as a champion of democracy and human rights in Mexico.      \\

{\bf GPT-3:} The president of Mexico in 2019 is Andrés Manuel López Obrador, commonly known as AMLO. He is a Mexican politician who has been the President of Mexico since December 1, 2018. He is a member of the National Regeneration Movement (MORENA) party and previously served as the Head of Government of Mexico City from 2000 to 2005. He is a left-wing populist and has promised to reduce corruption and inequality in Mexico. He has also promised to increase public spending on social programs and infrastructure projects.
\end{minipage}
\end{framed}

\begin{framed}
\centering
 \begin{minipage}{0.98\textwidth}
{\bf Instruction:} List all Canadian provinces in alphabetical order. \\

{\bf LLaMA-Adapter:} Alberta, British Columbia, Manitoba, New Brunswick, Newfoundland and Labrador, Nova Scotia, Ontario, Prince Edward Island, Quebec, Saskatchewan, Yukon. \\

{\bf Alpaca-LoRA:} Alberta, British Columbia, Manitoba, New Brunswick, Newfoundland and Labrador, Nova Scotia, Ontario, Prince Edward Island, Quebec, Saskatchewan.  \\

{\bf Alphaca:} Alberta, British Columbia, Manitoba, New Brunswick, Newfoundland and Labrador, Nova Scotia, Ontario, Prince Edward Island, Quebec, Saskatchewan, Northwest Territories, Nunavut, Yukon.      \\

{\bf GPT-3:} 1. Alberta
2. British Columbia
3. Manitoba
4. New Brunswick
5. Newfoundland and Labrador
6. Northwest Territories
7. Nova Scotia
8. Nunavut
9. Ontario
10. Prince Edward Island
11. Quebec
12. Saskatchewan
13. Yukon
\end{minipage}
\end{framed}

\begin{framed}
\centering
 \begin{minipage}{0.98\textwidth}
{\bf Instruction:} Tell me five words that rhyme with 'shock'. \\

{\bf LLaMA-Adapter:} Flock, rock, lock, cook, and stock. \\

{\bf Alpaca-LoRA:} Flock, clock, dock, stock, and sock. \\

{\bf Alphaca:} Five words that rhyme with shock are: rock, pop, shock, cook, and snock.\\

{\bf GPT-3:}
1. Block
2. Lock
3. Dock
4. Clock
5. Rock
\end{minipage}
\end{framed}


\begin{framed}
\centering
 \begin{minipage}{0.98\textwidth}
{\bf Instruction:} Translate the sentence 'I have no mouth but I must scream' into Spanish. \\

{\bf LLaMA-Adapter:} No tengo boca pero tengo que gritar. \\

{\bf Alpaca-LoRA:} No tengo boca pero tengo que gritar. \\

{\bf Alphaca:} No tengo boca, pero debo gritar. \\

{\bf GPT-3:} No tengo boca pero debo gritar.
\end{minipage}
\end{framed}


\begin{framed}
\centering
 \begin{minipage}{0.98\textwidth}
{\bf Instruction:} Write a Python program that prints the first 10 Fibonacci numbers. \\

{\bf LLaMA-Adapter:}
\begin{lstlisting}
def fibonacci():
    a = 0
    b = 1
    for i in range(10):
        a, b = b, a + b
        print(a, b)
    print("Fibonacci numbers: ", a, b)

fibonacci()
\end{lstlisting}

{\bf Alpaca-LoRA:}
\begin{lstlisting}
def fibonacci(n):
    if n == 0:
        return 0
    elif n == 1:
        return 1
    else:
        return fibonacci(n-1) + fibonacci(n-2)

print(fibonacci(10))
\end{lstlisting}

{\bf Alphaca:} 
\begin{lstlisting}
numbers = [0, 1]

for i in range(2, 11):
    numbers.append(numbers[i-2] + numbers[i-1])

print(numbers)
# Output: [0, 1, 1, 2, 3, 5, 8, 13, 21, 34]
\end{lstlisting}

{\bf GPT-3:}
\begin{lstlisting}
# Fibonacci numbers are the numbers in the following integer sequence, called the Fibonacci sequence, and characterized by the fact that every number after the first two is the sum of the two preceding ones.

# Fibonacci sequence: 0, 1, 1, 2, 3, 5, 8, 13, 21, 34, ...

# Program:

# Initialize the first two numbers of the Fibonacci sequence
a = 0
b = 1

# Print the first 10 numbers of the Fibonacci sequence
print(a)
print(b)

for i in range(2,10):
    c = a + b
    a = b
    b = c
    print(c)
\end{lstlisting}
\end{minipage}
\end{framed}


% \begin{framed}
% \centering
% \begin{minipage}{0.98\textwidth}
% {\bf Instruction:} Write a program that prints the numbers from 1 to 100. But for multiples of three print 'Fizz' instead of the number and for the multiples of five print 'Buzz'. For numbers which are multiples of both three and five print 'FizzBuzz'. \\

% {\bf LLaMA-Adapter:}
% \begin{lstlisting}
% def print_numbers(n):
%     for i in range(1, 101):
%         if i % 3 == 0 and i % 5 == 0:
%             print("FizzBuzz")
%         elif i % 3 == 0:
%             print("Fizz")
%         elif i % 5 == 0:
%             print("Buzz")
%         else:
%             print(i)
% \end{lstlisting} \\

% {\bf Alpaca-LoRA:}
% \begin{lstlisting}
% for i in range(1, 101):
%     if i % 3 == 0 and i % 5 == 0:
%         print("FizzBuzz")
%     elif i % 3 == 0:
%         print("Fizz")
%     elif i % 5 == 0:
%         print("Buzz")
%     else:
%         print(i)
% \end{lstlisting} \\

% {\bf Stanford Alpaca:}
% \begin{lstlisting}
% for i in range(1, 101):
%     if i % 3 == 0 and i % 5 == 0:
%         print('FizzBuzz')
%     elif i % 3 == 0:
%         print('Fizz')
%     elif i % 5 == 0:
%         print('Buzz')
%     else:
%         print(i)
% \end{lstlisting} \\

% {\bf GPT-3:}
% \begin{lstlisting}
% for (let i = 1; i <= 100; i++) {
%   if (i % 3 === 0 && i % 5 === 0) {
%     console.log("FizzBuzz");
%   } else if (i % 3 === 0) {
%     console.log("Fizz");
%   } else if (i % 5 === 0) {
%     console.log("Buzz");
%   } else {
%     console.log(i);
%   }
% }
% \end{lstlisting}
% \end{minipage}
% \end{framed}