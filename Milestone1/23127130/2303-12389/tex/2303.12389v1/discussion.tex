\section{Discussion}

We have presented two ways to explore numerically the optimization of eigenvalues of the Laplace-Beltrami operator
with Neumann boudndary conditions for domains in the sphere and the torus. The first way generalizes the notion
of eigenvalues of domains and thus is independant of topological consideration. On the other hand, the second
method relies on the level set representation of the domain, which allows topological changes.

In the case of the optimization on the sphere,  this flexibility turned out to be an important feature, regarding the topological complexity of certain optimal domains for $\mu_1$ on the sphere. Indeed, while the density optimization leads to optima that does not corresponds to domains, the level set procedure tries to create areas with a lot of holes, which might indicate non-existence of optimal domains for large enough surface area. Oppositely, it seems clear that for small enough surface area, the optimal density (and thus, optimal domain) for $\mu_1$ on the sphere is the one of a geodesic ball.
It has then been witnessed that the behaviour of $\mu_2$ on the sphere was completely clear. Indeed, the optimal domains are always the union of two geodesic balls as it have been shown in \cite{bucur_sharp_2022}.
For $\mu_3$ , it might be difficult to decribe theoretically the way optima acts depending on the total surface area but it would be interesting to prove some necessary conditions such domains have to meet, such as symmetry.

In the case of the optimization on the torus, we noticed that while homogenization happened this time for $\mu2$ and not for $\mu_1$. Further investigations may be needed to understand better how this phenomenon evolves with the size of the torus, and if it such homogenization happens in flat tori like $(\R^n/\Z^n)$.

In any case, an interesting result would be to get a better grasp on the existence or non-existence of optimal domains on manifold.

\section{Acknowledgement}

The author was supported by the ANR SHAPO (ANR-18-CE40-0013).
