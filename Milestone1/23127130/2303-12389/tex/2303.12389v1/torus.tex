\section{Explorations on a torus}

As it has been mentioned in the introduction, this last part is devoted to the optimization of eigenvalues on a torus. Specifically, we consider the torus $\mathbb{T}$ in $\R^3$ parametrized by
\[
  (u,v) \mapsto \left( (R+r\cos{v})\cos{u}, r \sin{v}, (R+r\cos{v})\sin{u} \right),
\]
that is, the torus with major radius $R$ and minor radius $r$. In the sequel, we will choose $R=2$ and $r=1$. Knowing that the surface area of $\mathbb{T}$ is given by
\[
  |\mathbb{T}| = 4 \pi^2 Rr
\]
we have in our case $|\mathbb{T}| \approx 78.96$.

The problem we consider now is analogous to the one on the sphere. With $\Om \subset \mathbb{T}$ being a Lipschitz open domain, the eigenvalue problem is
\[
  \begin{cases}
    -\Delta u = \mu_k(\Om) u \mbox { in } \Om,\\
    \frac{\partial u}{\partial n} = 0  \mbox { on } \partial \Om,
  \end{cases}.
\]

We also use the same notion of generalized eigenvalues of a density $\rho : \mathbb{T} \to [0,1]$, we define
\begin{equation*}
  \mu _k(\rho) := \inf_{V\in{\mathcal V}_{k+1}} \max_{u \in V \sm \{0\}} \frac{\int_{\mathbb{T}} \rho|\nabla u|^2}{\int_{\mathbb{T}} \rho u^2},
\end{equation*}
where ${\mathcal V}_{k+1}$ is the family of subspaces of dimension $k+1$ in
\[
  \{u\cdot 1_{\{\rho (x)>0\}}: u \in C^\infty_c (\mathbb{T})\}.
\]

We then consider the two optimization problems
\begin{equation*}
  \sup \left\{ \mu_k(\Om) \mbox{ s.t. } \Om \subset \mathbb{T}, | \Om | = m, \Om \mbox{ bounded, open and Lipschitz} \right\}
\end{equation*}

and

\begin{equation*}
  \sup \left\{ \mu_k(\rho) \mbox{ s.t. }  \rho : \mathbb{T} \rightarrow [0,1], \int_{\mathbb{T}}\rho =m\right\}.
\end{equation*}

with $0 < m < |\mathbb{T}|$ and $k>0$.

This last formulation allows to perform the same density method performed on the sphere. These are the results presented hereafter, followed by the results obtained by the level set method.

\subsection{Density optimization}

We precise that the optimization parameters that have been used are the same than the ones used for the sphere. We begin to depict some of the optimal densities in Figure \ref{fig:tore_mu_density_examples}. For a better visualization, we recall that all the meshes and densities are available at \url{https://github.com/EloiMartinet/Neumann_Sphere/}.

\begin{figure}
    \centering
    \includegraphics[width=0.24\textwidth]{density_tore_mu_1_2.59.png}
    \includegraphics[width=0.24\textwidth]{density_tore_mu_1_21.68.png}
    \includegraphics[width=0.24\textwidth]{density_tore_mu_1_43.95.png}
    \includegraphics[width=0.24\textwidth]{density_tore_mu_1_67.72.png}

    \includegraphics[width=0.24\textwidth]{density_tore_mu_2_2.59.png}
    \includegraphics[width=0.24\textwidth]{density_tore_mu_2_21.68.png}
    \includegraphics[width=0.24\textwidth]{density_tore_mu_2_43.95.png}
    \includegraphics[width=0.24\textwidth]{density_tore_mu_2_67.81.png}

    \includegraphics[width=0.24\textwidth]{density_tore_mu_3_2.40.png}
    \includegraphics[width=0.24\textwidth]{density_tore_mu_3_21.04.png}
    \includegraphics[width=0.24\textwidth]{density_tore_mu_3_41.95.png}
    \includegraphics[width=0.24\textwidth]{density_tore_mu_3_67.69.png}

    \caption{Example of optimal densities for $\mu_1, \mu_2$ and $\mu_3$ (resp. first, second and third row) for $m \in \{3, 22, 43, 68\}$ approximately.}
    \label{fig:tore_mu_density_examples}
\end{figure}

We can notice that for small enough masses, the optimal densities seems to be the characteristic function of geodesic balls for $k\in\{1,2,3\}$. However, by comparison with the case of the plane and the sphere, the case of $\mu_3$ must be taken with caution.
A striking fact is that contrary to the case of the sphere, the optimal density for $\mu_1$ seems to stay a charateristic function whereas we cans
witness some homogenization for $\mu_2$.
The optimal eigenvalues plotted as functions of $m$ are shown in Figure \ref{fig:tore_mu_density}.

\begin{figure}
    \centering
    \includegraphics[width=0.4\textwidth]{density_tore_chart_mu_1_left.png}
    \includegraphics[width=0.4\textwidth]{density_tore_chart_mu_1_right.png}
    \includegraphics[width=0.4\textwidth]{density_tore_chart_mu_2_left.png}
    \includegraphics[width=0.4\textwidth]{density_tore_chart_mu_2_right.png}
    \includegraphics[width=0.4\textwidth]{density_tore_chart_mu_3_left.png}
    \includegraphics[width=0.4\textwidth]{density_tore_chart_mu_3_right.png}
    \caption{Optimal values of $\mu_1, \mu_2$ and $\mu_3$ (resp. first, second and third row) obtained by the density method.}
    \label{fig:tore_mu_density}
\end{figure}


\subsection{Level set optimization}

In Figure \ref{fig:tore_mu_ls_examples} are the optimal domains obtained by the level set method for $\mu_1, \mu_2, \mu_3$ and various masses.

\begin{figure}
    \centering
    \includegraphics[width=0.24\textwidth]{levelset_tore_mu_1_4.png}
    \includegraphics[width=0.24\textwidth]{levelset_tore_mu_1_22.png}
    \includegraphics[width=0.24\textwidth]{levelset_tore_mu_1_42.png}
    \includegraphics[width=0.24\textwidth]{levelset_tore_mu_1_60.png}

    \includegraphics[width=0.24\textwidth]{levelset_tore_mu_2_4.png}
    \includegraphics[width=0.24\textwidth]{levelset_tore_mu_2_22.png}
    \includegraphics[width=0.24\textwidth]{levelset_tore_mu_2_42.png}
    \includegraphics[width=0.24\textwidth]{levelset_tore_mu_2_60.png}

    \includegraphics[width=0.24\textwidth]{levelset_tore_mu_3_4.png}
    \includegraphics[width=0.24\textwidth]{levelset_tore_mu_3_22.png}
    \includegraphics[width=0.24\textwidth]{levelset_tore_mu_3_42.png}
    \includegraphics[width=0.24\textwidth]{levelset_tore_mu_3_60.png}

    \caption{Example of optimal domains for $\mu_1, \mu_2$ and $\mu_3$ (resp. first, second and third row) for $m \in \{4, 22, 42, 60\}$ approximately.}
    \label{fig:tore_mu_ls_examples}
\end{figure}

The optimal eigenvalues plotted as functions of $m$ are shown in Figure \ref{fig:tore_mu_ls}.

\begin{figure}
    \centering
    \includegraphics[width=0.4\textwidth]{levelset_tore_chart_mu_1_left.png}
    \includegraphics[width=0.4\textwidth]{levelset_tore_chart_mu_1_right.png}
    \includegraphics[width=0.4\textwidth]{levelset_tore_chart_mu_2_left.png}
    \includegraphics[width=0.4\textwidth]{levelset_tore_chart_mu_2_right.png}
    \includegraphics[width=0.4\textwidth]{levelset_tore_chart_mu_3_left.png}
    \includegraphics[width=0.4\textwidth]{levelset_tore_chart_mu_3_right.png}
    \caption{Optimal values of $\mu_1, \mu_2$ and $\mu_3$ (resp. first, second and third row) obtained by the level set method.}
    \label{fig:tore_mu_ls}
\end{figure}
