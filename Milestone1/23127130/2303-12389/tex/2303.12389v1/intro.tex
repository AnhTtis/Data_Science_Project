\section{Introduction}

Let $n\geq 1$. The problem we consider in the first part of this paper involves the spectrum of the Laplace operator on domain in the unit sphere $\S^n = \{ x \in \R^{n+1} : \|x\| = 1 \}$. Let $\Omega \subset \S^n$ be an open, bounded and Lipschitz set. The spectral theorem assures that the problem
$$
\begin{cases}
-\Delta u = \mu_k(\Om) u \mbox { in } \Om,\\
\frac{\partial u}{\partial \nu} = 0  \mbox { on } \partial \Om,
\end{cases}
$$
with $u \in H^1(\Om)\sm \{0\}$ has a sequence of eigenvalues
$$
0 = \mu_0(\Om) \leq \mu_1(\Om) \leq \mu_2(\Om) \leq ... \to +\infty.
$$
The corresponding eigenfunctions satisfy the well-known variational formula
\begin{equation}\label{domain_var}
    \mu_k(\Om) = \min_{V\in{\mathcal V}_{k+1}} \max_{u \in V\sm \{0\}} \frac{\int_\Om |\nabla u|^2}{\int_\Om u^2},
\end{equation}
where ${\mathcal V}_k$ is the family of subspaces of dimension $k$ in $H^1(\Om)$. We are interested in the following problem : for $m > 0$ and $k \in \N$, solve
\begin{equation}\label{domain_problem}
\sup \{ \mu_k(\Om) : \Om \subset \S^n, | \Om | = m, \Om \mbox{ bounded, open and Lipschitz} \}.
\end{equation}
The aim of this paper will be to study this optimization problem from a numerical point of view.

While the numerical shape optimization of Neumann eigenvalues of domains in the Euclidian space have drawn a lot of attention in the past years (see for instance \cite{antunes2012numerical} \cite{antunes_numerical_2017} \cite{antunes_oudet} \cite{BMO_2022}), the litterature on the optimization of those eigenvalues for domains in curved spaces is sparse. The present work will address this problem by considering the optimization of several Neumann eigenvalues of domains in the sphere $\S^n$ as well as on a torus.

Old and recent theoretical works have been conducted to prove the optimality of the spherical cap for domain in the sphere for the first Neumann eigenvalue, either by assuming simple connectedness of the optimal domain \cite{bandle_isoperimetric_1972} \cite{langford2022maximizers} or, following the seminal work of Weinberger \cite{weinberger_isoperimetric_1956}, using the so-called "mass transplantation technique". However all these results require some restriction on the domain : for instance, to lie on the hemisphere \cite{ashbaugh_sharp_1995} or to lie outside of some spherical cap \cite{bucur_sharp_2022}.

In this work, we will provide some numerical evidence of properties of the optimal domain in the sphere. Especially, we will see that some density-based method strongly suggest that the problem of the optimization of the first Neumann eigenvalue on domains can not be tackled by mass transplantation arguments for domains and densities of masses large enough. Moreover, we will see how the numerical study of the second non-trivial Neumann eigenvalue suggests that the optimal shape is two disjoint geodesic balls \cite{bucur_sharp_2022}. We will also numerically witness the rich variety of optimal shapes for the third non-trivial eigenvalue. In a last part we will consider the same questions on a torus.

As it has already been said, one of the numerical method will rely on some relaxation of the original problem by extending the relation \eqref{domain_var} to the class of densities $\rho \in \L^\infty(\S^n, [0,1])$ in the following way
\begin{equation}\label{density_var}
 \mu _k(\rho) := \inf_{V\in{\mathcal V}_{k+1}} \max_{u \in V \sm \{0\}} \frac{\int_{\S^n} \rho|\nabla u|^2}{\int_{\S^n} \rho u^2},\end{equation}
where ${\mathcal V}_{k+1}$ is the family of subspaces of dimension $k+1$ in
\begin{equation}
\{u\cdot 1_{\{\rho (x)>0\}}: u \in C^\infty_c (\S^n)\}.
\end{equation}

This relaxation, which has already been extensively used in \cite{BMO_2022}, can also be found in greater generality in \cite{colbois_spectrum_2019}.

The original problem will then be replaced by

\begin{equation} \label{density_problem}
\sup \left\{ \mu_k(\rho) :  \rho : \S^n \rightarrow [0,1], \int_{\S^n}\rho=m\right\}.
\end{equation}
which well-posedness is established below.

This formulation allows to perform some classical optimization methods such as gradient descent over the variable $\rho$ instead of considering difficult shape optimization problem involving changes of topology. However, the density method is strictly more general than the problem \eqref{domain_problem} in the sense that optimal density may not correspond to characteristic functions of domains. This led to an implementation of a shape optimization method, namely the level set method, which solves directly the problem \eqref{domain_problem} with possible changes in topology.

In the next sections, we first see the theoretical aspects brought by the relaxation \eqref{density_var}. Then we discuss the practical implementation of the two methods cited above and provide numerical results related to the first three eigenvalues. We also address some possible consequences these computations have at a theoretical level.
