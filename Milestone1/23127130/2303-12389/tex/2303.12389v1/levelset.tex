\section{Level-set method}

In this section we focus on the optimization of the original problem (\ref{domain_problem}) through the level-set method. This allows to give more informations for the original shape optimization problem, when optima lead by the previous density method didn't match the characteristic function of an actual shape.
The level-set method consists in representing the domain $\Omega$ as the level-set of a function has been extensively used for shape optimization, either for compliance or eigenvalue optimization (see for example \cite{de_gournay_velocity_2006}, \cite{allaire_level-set_2005}, \cite{allaire_shape_2014} among others). In the line of \cite{allaire_shape_2014} and in order to fix the notations, we recall main ideas of the level set approach.

Let $t\in[0,T]$ and $\Omega(t) \subset \S^n$ be a domain evolving in time according to a velocity field $V : [0,T]\times\S^n \to \mathbb{T}\S^n$. More precisely, if $\Omega_0$ is a domain in $\S^n$ and $\chi$ is the flow of $V$ defined as the solution of the differential equation
\begin{equation*}
    \begin{cases}
    \chi'(x_0,t)= V(t,\chi(x_0,t)) \\
    \chi(x_0,0)= x_0 \in \Omega_0
    \end{cases}
\end{equation*}
then we can define
\[
\Omega(t) := \chi(\Omega_0,t).
\]

In the framework of the level-set method, we represent our domain by a function $\phi : [0,T] \times \S^N \to \R$ such that
\begin{equation}
    \forall x \in \S^N, \forall t\in[0,T],
    \begin{cases}
    \phi(t,x) < 0 & \mbox{ if } x \in \Omega(t) \\
    \phi(t, x) = 0 & \mbox{ if } x \in \partial \Omega(t) \\
    \phi(t,x) > 0 & \mbox{ if } x \in \S^N \sm \Omega(t)
    \end{cases}.
\end{equation}

The motion of $\Omega(t)$ is equivalent to the advection of $\phi$ by the equation
\begin{equation}
    \partial_t \phi(t,x) + V(t,x) \cdot \nabla \phi(t,x) = 0 \mbox{ on } (0,T)\times \S^n.
\end{equation}

Observing that $n_{\Omega(t)}:=\frac{\nabla \phi(t,.)}{|\nabla \phi(t,.)|}$ is an extension of the unitary outward normal of $\Omega(t)$ for all $t$ and if $V$ is of the kind $V(t,x) = v(t,x)n_{\Omega(t)}(x)$ with $v(t,x)\in \R$, we can re-write the previous equation as

\begin{equation}
     \partial_t \phi(t,x) + v(t,x) |\nabla \phi(t,x)| = 0 \mbox{ on } (0,T)\times \S^n.
\end{equation}

In our case, $[0,T]$ represents a single timestep and therefore $T$ is small enough to consider that $v(t,x) \approx v(x)$ for all $t \in [0,T]$ which finally leads to the so called \textit{Hamilton-Jacobi} equation :

\begin{equation}
    \label{eqn:advection}
     \partial_t \phi(t,x) + v(x) |\nabla \phi(t,x)| = 0 \mbox{ on } (0,T)\times \S^n.
\end{equation}

Numerically, this equation is solved by the method of characteristics, thanks to the \textit{"advect"} toolbox that can be found at \url{https://github.com/ISCDtoolbox/Advection/}. This method supports $\P_1$ functions defined on surface meshes. Many thanks to the authors for their precious work.


\subsection{Shape derivative}

At each time step, the purpose is to find some vector field $v$ such that advecting $\phi$ by (\ref{eqn:advection}) on this small time step increase the considered eigenvalue. In order to address this issue we need to compute the shape derivative of an eigenvalue of a domain of $\S^n$. This shape derivative is given by the following result (see for instance \cite{ZANGER200139}):

\begin{theorem}[Shape derivative] We assume that $\Omega$ is $C^1$ with non-empty boundary. Let $k \in \N$ and $V : \S^N \to \mathbb{T}\S^n$ be a smooth vector field with compact support in the neighborhood of $\Omega_0$. We denote
\[
\mu_k'(\Omega_0, V) := \lim_{t \to 0^+} \frac{\mu_k(\Omega(t))-\mu_k(\Omega_0)}{t}.
\]
Moreover we assume that $\mu_k(\Omega_0)$ is simple. Then this limit exists and
\begin{equation}
    \label{eqn:shape_derivative}
    \mu_k'(\Omega_0, V) = \int_{\partial \Omega_0} \left(|\nabla u|^2 - \mu_k(\Omega_0) u^2\right)(V.n) \dd \sigma
\end{equation}
with $u$ an eigenfunction associated to $\mu_k(\Omega_0)$ with unitary $\L^2$ norm and $n$ the outward normal.
\end{theorem}

The previous theorem allows us to only consider normal variations of the boundary and from this consideration $\mu_k'(\Omega_0,V) = \mu_k'(\Omega_0,v)$ where $v = V.n$. Moreover, it shows that we can choose $v=|\nabla u|^2 - \mu_k(\Omega_0)u^2$ as a gradient direction. This, however, may not be the best choice as we see hereafter.


\subsection{Handling area constraint}

Despite being able to compute a gradient direction (\ref{eqn:shape_derivative}), we also have to fullfill the constraint $|\Omega| = m$ in the original problem (\ref{domain_problem}). We could choose to add a penalization term and maximize the function
\[
\Omega \mapsto \mu_k(\Omega) - b (|\Omega| - m')^2
\]
instead, with $b>0$ and $m'$ a parameter allowing to control for the total mass $m$. However, if we consider the sequence of geodesic balls $B(\eps)$ of radius $\eps>0$ then we can fin that
\[
\mu_k(B(\eps)) - b (|B(\eps)|-m')^2 \xrightarrow[\eps\to 0]{} +\infty
\]
which establishes that the optimum is never attained for a positive area.
However, a result by Strichartz allows to replace the unbounded quantity $\mu_k(\Omega)$ by quantity $|\Omega|^\frac{2}{n} \mu_k(\Omega)$ even if we do not have invariance by dilation as in the Euclidian case. Indeed, using formulas (3.15) and (3.16) of \cite{strichartz_estimates_1996} in the special case $n=2$, we get the following property :

\begin{proposition}
Let $\Om \subset \S^2$. Then
\begin{equation} \label{ineq:strichartz_bound}
    |\Om|\mu_k(\Om) \leq 2\pi k^2.
\end{equation}
\end{proposition}

This generic bound allows to maximize the function
\begin{equation}
    \label{eqn:cost}
    J(\Omega) := |\Omega| \mu_k(\Omega) - b (|\Omega| - m')^2
\end{equation}
which prevents the function to blow-up. Then if $\Omega^*$ maximizes (\ref{eqn:cost}), it is solution of (\ref{domain_problem}) with $m=|\Omega^*|$.

There is two way to implement the level set method. The so-called ersatz material approach involves a fixed mesh where the "void" part is filled with some material with good properties. The other one involves to remesh the domain at each step according to the level-set function. While the second one is more accurate, it suffers two main drawbacks, the most obvious one being its computational cost. The second one is related to the connectivity of $\Omega(t)$: suppose that we want to optimize $\mu_k$, starting from a topologically complex domain. The level-set method allowing topological changes, it is very likely that at one point $t$, $\Omega(t)$ splits into $k+1$ connected components. Then $\mu_0(\Omega(t))= ... = \mu_k(\Omega(t)) = 0$ and the associated eigenfunctions are constant on each connected component. This leads the shape derivative to be equal to
\[
  \mu_k'(\Omega,v) = - \mu_k(\Omega) \int_{\partial \Omega} v \dd \sigma.
\]
The reader may recognise that this is proportionnal to the shape derivative of the function $\Om \to |\Om|$. This implies that the optimization process will only optimize on the volume. One the other hand, the "ersatz material" approach allows transparent topological changes and is faster than the second one since it doesn't require remeshing at each iteration. This is why we perform a first optimization using the ersatz material method and then use a remeshing approach for a final optimization of higher accuracy.


\subsection{Level-set with ersatz material}

In this section we assume the eigenvalues to be simple. According to the approximation theorem \ref{th:approximation}, we fix a small $\eps>0$ and solve the problem
\[
 -\div[(\1_\Omega+\eps)\nabla u] = \mu_k^\eps(\1_\Omega) (\1_\Omega+\eps^2) u
\]
at each step. Thanks to the above-mentioned theorem, we expect this eigenvalue to be close to the actual one for small $\eps$.
However, the function $\1_\Omega = 1_{\{\phi < 0 \}}$, as defined on the mesh, may be highly irregular. This is why we approximate it in the following way
\[
  \1_\Omega \approx \frac{1}{2}\left( 1- \frac{\phi}{\sqrt{\phi^2 + \sigma^2}} \right)
\]
with $\sigma>0$ small. This avoids degeneracy in the denominator. Similar regularizations have been used in this framework, see for instance \cite{allaire2014multi} For the same reasons, the extended normal field is approximated by
\[
n_\Omega \approx \frac{\nabla\phi}{\sqrt{|\nabla \phi|^2 + \sigma^2}}
\]


\subsection{Initialization}

It is well-known that the levelset method is prone to fall into local optima because of its sensitivity to initialization. To tackle this problem, the levelset function is initilized with a randomized trigonometric sum of the type
\begin{equation*}
  \phi(\theta, \psi) = \Re \left\{\sum_{j=0}^{p} \sum_{k=0}^q c_{j,k} \exp{i(j\theta + k\psi)}\right\}
\end{equation*}
where the $c_{i,j}$ are chosen at random and $\theta,\psi \in [0,\pi]\times[0,2\pi]$ are respectively the latitude and longitude on $\S^2$. It is expected that the larger $q$ and $p$, the more complex $\phi$ is and, by extension, $\Omega$.

\subsection{Multiple eigenvalues}

The case of multiple eigenvalues, which always occursin practice, is handled in the same way as in the density case.

\subsection{Numerical considerations}

In our simulations, we took $\eps = 10^{-4}$ and $\sigma = 10^{-5}$. Moreover, to capture the variations of $\Omega(t)$ with good accuracy and because the level-set function tends to steepen near $\partial \Omega(t)$ over the iterations, we remesh the domain thanks to the MMG library \cite{dapogny2014three} and recompute the signed distance function every $20$ iterations thanks to the mshdist tool \cite{dapogny2012computation}. The maximal size of an element is $h_{max} = 10^{-1}$ and the minimal size is $h_{min} = 10^{-3}$. Just as previously, we use $\P_1$ finite elements and the FE computations are performed in GetFEM. The optimization algorithm is a simple gradient algorithm with a fixed number of $N=600$ steps. The step size $\delta t$ is chosen such that, if $v$ is the gradient direction at a given moment then $\delta t = \frac{\gamma}{\|v\|_{\infty}}$ with $\gamma=3.10^{-2}$. The penalty term $b$ is chosen to be equal to $5$ and $m'$ takes multiple values between $0.5$ and $4\pi$. The algorithm is presented in algorithm \ref{alg:ersatz} for clarity purposes. Note that this algorithm is the one performed for a fixed $m'$. Finally, as in the density case, the optimization is performed multiple times with different initial level set functions and the best one is kept.

\begin{algorithm}
\caption{ersatz material levelset algorithm.}\label{alg:ersatz}
\begin{algorithmic}
  \Require $k > 0$ \Comment{The eigenvalue we optimize}
  \Require A mesh in MEDIT format
  \State Initialize the levelset $\phi$
  \For{$i$ from $0$ to $N$}
    \If{$N=0 \mod 20$}
        \State Remesh using MMG.
        \State Reinitialize the levelset function $\phi$.
    \EndIf
    \State Compute $\mu_k^\eps(\Omega)$ and the associated eigenfunctions using FE.
    \State Compute $v$ the maximizing direction.
    \State Advect $\phi$ during a time $\delta t$.
  \EndFor
\end{algorithmic}
\end{algorithm}


\subsection{Level-set with remeshing}

The following method is triggered once the previous one has converged, hence we don't expect major changes in topology which could be problematic as discussed before. In this procedure, we remesh the sphere  such that $\partial \Omega = \{\phi = 0\}$ is a polygonal line of the mesh at each timestep. This then allows us to extract the mesh describing $\Omega$ and solve the original eigenvalue problem on it, without having to compute the approximation $\mu_k^\eps$. But then the optimization direction $v = |\nabla u|^2 - \mu_k(\Omega)u^2$ (with $u$ an eigenfunction of $\mu_k(\Omega)$) is only defined on $\Omega$ while we need it to be defined in the whole sphere $\S^n$ in order to advect the level-set function. In this purpose we use the well-known "extension-regularization" method which allows - as its name suggests - to extend the velocity field on all $\S^n$ and regularize it at the same time \cite{de_gournay_velocity_2006}. Still assuming that the eigenvalue is simple, we see that $v \mapsto \mu_k'(\Omega_0,v)$ is a continuous linear form in $v$. Hence, we can find $w$ the unique solution to the variational problem
\begin{equation}
    \forall v \in \H^1(\S^n), \int_{\S^n} \alpha \nabla v \nabla w + vw = \dd \mu_k(\Omega_0, v)
\end{equation}
where $\alpha > 0$. Then $w$ is indeed an extension of $|\nabla u|^2 - \mu_k u^2$ on $\H^1(\S^n)$ with regularity depending on $\alpha$. Moreover, $w$ is a valid gradient direction since
\[
\dd \mu_k(\Omega_0, w) = \int_{\S^n} \alpha |\nabla w|^2 + w^2 \geq 0.
\]

\subsection{Numerical considerations}

The numerical values chosen for this second optimization are mostly the same as the previous procedure. One add the regularization parameter $\alpha=0.1$ and that $h_{max}$ is now equal to $5.10^{-2}$. The step size $\delta t_i$ is now adaptative : if at a given iteration $i$ we have that $\mu_k(\Omega_i) > \mu_k(\Omega_{i+1})$ then $\delta t_{i+1} = \delta t_{i}/2$. Otherwise $\delta t_{i+1} = 1.1 \delta t_i$. The optimization stops when $\delta t_i < 10^{-7}$ and the mesh with the best cost is kept. The pseudocode of the algorithm for a fixed $m'$ is provided in algorithm ~\ref{alg:remesh}.


\begin{algorithm}
\caption{ersatz material levelset algorithm.}\label{alg:remesh}
\begin{algorithmic}
  \Require $k > 0$ \Comment{The eigenvalue we optimize}
  \Require The mesh and optimal domain $\Omega$ obtained by the previous procedure.
  \State Initialize the levelset $\phi$ as the one of $\Omega$.
  \While{$\delta t > 10^{-7}$}
    \State Remesh using MMG.
    \State Reinitialize the levelset function $\phi$.
    \State Extract the submesh $\Omega$
    \State Compute $\mu_k(\Omega)$ and the associated eigenfunctions using FE.
    \State Compute $v$ the maximizing direction on $\partial \Omega$
    \State Extend $v$ to the whole mesh of $\S^2$ by extension-regularization
    \State Advect $\phi$ during a time $\delta t$
    \If{The cost function increased}
      \State $\delta t \gets 1.1 \delta t$
    \Else
    \State $\delta t \gets \delta t/2$
    \EndIf
  \EndWhile
\end{algorithmic}
\end{algorithm}



\section{Results : level-set method.}

We report here the optimization results for $k \in \{1,2,3\}$. We denote by $\Omega^m$ the optimum computed with the levelset procedure verifying $|\Omega^m|=m$. The optimal eigenvalues $\mu_k(|\Omega^m|)$  are plotted in green, against the corresponding surface area $m$. As for the density method, we also plot in red the eigenvalue $\mu_k\left(UB(|\Omega^m|, k)\right)$ of an union of $k$ disjoints balls of total area $m$. Since the eigenvalue goes to $\infty$ as $|\Omega^m|$ goes to $0$, we divide the plot on two parts $0<|\Omega^m| \leq 2\pi$ and $2\pi < |\Omega^m| \leq 4\pi$ for better readability.

\subsection{Optimization of $\mu_1$}

\begin{figure}
    \centering
    \includegraphics[width=0.49\textwidth]{levelset_chart_mu_1_left.png}
    \includegraphics[width=0.49\textwidth]{levelset_chart_mu_1_right.png}
    \caption{Optimal value of $\mu_1$ obtained by the level-set method.}
    \label{fig:mu_1_ls}
\end{figure}

In Figure \ref{fig:mu_1_ls} are displayed the results for the optimization of $\mu_1$. The spherical cap seems to be the optimal shape up to $m \approx 8.0$, after which it clearly ceases to be the optimal shape. From that point up to $m=4\pi$, complex shapes arises, consisting in a plain hemisphere and a lot of holes in the opposite one. Different views of one of those shapes can be seen in Figure \ref{fig:mu_1_views}, where $m \approx 11.13 $ and $\mu_1(\Omega^m) \approx 1.77$ (for instance, a spherical cap of this surface area would give $\mu_1(\B^m) = 1.62$). This strange behaviour, combined with the density approach above, may suggest that the actual optimal may be attained by some kind of homogenization procedure. Some simple, non conclusive numerical test have been performed in this direction but this problem surely needs further investigation and may lead to interesting numerical and theoretical developements.

\begin{figure}
    \centering
    \includegraphics[width=0.24\textwidth]{levelset_mu_1_example_0.png}
     \includegraphics[width=0.24\textwidth]{levelset_mu_1_example_1.png}
      \includegraphics[width=0.24\textwidth]{levelset_mu_1_example_2.png}
       \includegraphics[width=0.24\textwidth]{levelset_mu_1_example_3.png}
    \caption{Rotationnal view of the optimal shape obtained by the level-set method for large $m$.}
    \label{fig:mu_1_views}
\end{figure}


More optimal shapes are displayed in Figure \ref{fig:mu_1_levelset_examples}. Looking at $\Omega^m$ for $m=8.0$ (the third one from the left), one can imagine that it would be possible for the geodesic cap to cease to be optimal for $m$ way lower than $8.0$ but the numerical procedure wouldn't be able to "see" it because it would be necessary to create details thinner than the size of an element of the mesh. However, it seems unlikely that the spherical cap ceases to be optimal for the same mass $m$ as the density method:

\begin{conjecture}Let $\delta$ be the same as in Conjecture 2. Then there exists $\delta'>\delta$ such that for all $0 < m < \delta'$, $\Omega^m=B^m$
\end{conjecture}

\begin{remark}The fact that $\delta' \geq \delta$ is obvious; the interesting part would be to show that the inequality is strict.
\end{remark}


\begin{figure}
    \centering
    \includegraphics[width=0.24\textwidth]{levelset_mu_1_2.03.png}
    \includegraphics[width=0.24\textwidth]{levelset_mu_1_5.1.png}
    \includegraphics[width=0.24\textwidth]{levelset_mu_1_8.0.png}
    \includegraphics[width=0.24\textwidth]{levelset_mu_1_10.85.png}
    \caption{Example of optimal domains for $\mu_1$ for $m \in \{2.03, 5.1, 8.0, 10.85\}$.}
    \label{fig:mu_1_levelset_examples}
\end{figure}


\subsection{Optimization of $\mu_2$}

\begin{figure}
    \centering
    \includegraphics[width=0.49\textwidth]{levelset_chart_mu_2_left.png}
    \includegraphics[width=0.49\textwidth]{levelset_chart_mu_2_right.png}
    \caption{Optimal value of $\mu_2$ obtained by the level-set method.}
    \label{fig:mu_2_ls}
\end{figure}

The optimal results for $\mu_2$ are displayed Figure \ref{fig:mu_2_ls}. We can clearly see that the optimal shape is always the union of two spherical caps, as it has been proven in \cite{bucur_sharp_2022}. Hence this case can be considered as a test case to support the validity of the method. In Figure \ref{fig:mu_2_levelset_examples} are shown the optimal shapes. For $m=1.17$, we see that the computed shape isn't a union of two disjoint disks. Indeed, for large $m$, the first levelset procedure struggled to disconnect one domain into two disks due to numerical instabilities. The eigenvalue is however really close to the one of two disks.

\begin{figure}
    \centering
    \includegraphics[width=0.24\textwidth]{levelset_mu_2_2.12.png}
    \includegraphics[width=0.24\textwidth]{levelset_mu_2_5.10.png}
    \includegraphics[width=0.24\textwidth]{levelset_mu_2_8.13.png}
    \includegraphics[width=0.24\textwidth]{levelset_mu_2_11.17.png}
    \caption{Example of optimal domains for $\mu_2$ for $m \in \{2.12, 5.1, 8.13, 11.17\}$.}
    \label{fig:mu_2_levelset_examples}
\end{figure}

\subsection{Optimization of $\mu_3$}

\begin{figure}
    \centering
    \includegraphics[width=0.49\textwidth]{levelset_chart_mu_3_left.png}
    \includegraphics[width=0.49\textwidth]{levelset_chart_mu_3_right.png}
    \caption{Optimal value of $\mu_3$ obtained by the level-set method.}
    \label{fig:mu_3_ls}
\end{figure}

In Figure \ref{fig:mu_3_ls} is displayed the results for the optimization of $\mu_3$. As for the density case, this eigenvalue shows a rich variety of behaviours depending on the value of $m$ (see Figure \ref{fig:mu_3_levelset_examples}).

\begin{figure}
    \centering
    \includegraphics[width=0.24\textwidth]{levelset_mu_3_2.0.png}
    \includegraphics[width=0.24\textwidth]{levelset_mu_3_5.22.png}
    \includegraphics[width=0.24\textwidth]{levelset_mu_3_8.0.png}
    \includegraphics[width=0.24\textwidth]{levelset_mu_3_11.04.png}
    \caption{Example of optimal domains for $\mu_3$ for $m \in \{2.0, 5.22, 8.0, 11.04\}$.}
    \label{fig:mu_3_levelset_examples}
\end{figure}

The results seems to be in accordance with the ones given by the density method.

\subsection{Data.}

As for the results of the density method, all the final solutions are available in MEDIT format at \url{https://github.com/EloiMartinet/Neumann_Sphere/}, with a FreeFem++ script allowing to compute the eigenvalue and surface area of each solution.
