\vspace{-5pt}
\section{Conclusion}
\label{sec:conclusion}
In this paper, we tackle the problem of compositional zero-shot learning (CZSL). To disentangle visual concepts from the attribute-object composition, we propose \framework adopting cross-attentions to learn the individual concept from paired concept-sharing images. To constrain the disentanglers to learn the concept of interest, we employ a regularization term adapted from the earth mover's distance (EMD), which is used as a feature similarity metric in the cross-attention module. Moreover, to exploit the attribute and object prediction ability of \framework, we improve the inference process by combining attribute, object, and composition probabilities into the final prediction score. We empirically demonstrate \framework outperforms the current state-of-the-art methods under both closed- and open-world settings. We also conduct a comprehensive qualitative analysis to validate the disentanglement ability of attention disentanglers in \framework. 
\paragraph{Limitations} \hsznew{Like existing CZSL methods, it is time- and computation-consuming to derive all composition embeddings when the numbers of attributes and objects are large. 
Moreover, it remains an open challenge to exploit concepts based on the actual semantics rather than solely on text; for instance, the “open” attribute in “open curtain” and “open computer” has completely different meanings.} 
\paragraph{Acknowledgements} 
This work is partially supported by Hong Kong Research
Grant Council - Early Career Scheme (Grant No. 27208022) and HKU Seed Fund for Basic Research.