\section{Comparing to CORSIKA IACT}\label{CORSIKA}
To demonstrate the veracity of this Cherenkov universality model, showers were generated using CORSIKA's IACT extension, with IACT's defined at increasing distances from the shower core \cite{CORSIKA}. The particular shower shown in figures \ref{LateralDistribution} and \ref{ArrivalTimes} is a proton shower with a primary energy of $10^8$ GeV, \xmax\ of $666 \, \textrm{g/cm}^2$, \nmax\ of $\num{6.3e7}$ particles, and a zenith angle of 30$^\circ$. \xmax\  and \nmax\  refer to the atmospheric depth at shower maximum and the maximum number of secondary shower particles, respectively. The observation level was defined at sea level. Shower development stages were calculated at each step in slant depth based on CORSIKA's longitudinal profile compared to \nmax. Atmospheric $\delta$s along the profile were calculated via interpolation from depth along the axis as a function of altitude. The U.S.\ Standard Atmosphere of 1976 was used for all altitude and density calculations.

Using the method described in section \ref{table}, Cherenkov signals were calculated along this shower's longitudinal profile toward each counting location. We compare this signal to that generated by CORSIKA IACT. The total number of photons collected at increasing distances from the shower core are compared in Figure~\ref{LateralDistribution}.

When considering Cherenkov production, CORSIKA's clock starts when the primary particle enters the atmosphere \cite{CORSIKA}. To calculate when our Cherenkov signals arrive at each counting location, we approximate the shower front as traveling along the axis at $c$, thus calculating the time it takes for the shower front to move from the top of the atmosphere to each axis location $\va{r}_a$. We then compute the amount of time it would take for something moving at $c$ to travel along travel vectors $\va{r}_t$. This time is adjusted by the delay the Cherenkov photons experience as they propagate along $\va{r}_t$. A vertical delay is calculated as the sum of delays through refractive indices at intervals surrounding each sampled height, then divided by the cosine of the polar angle of vector $\va{r}_t$. The arrival time distributions for a counter 303 m from the shower core are compared in Figure~\ref{ArrivalTimes}.

\begin{figure}
	\begin{center}
		\includegraphics[width=\columnwidth]{cwld.png}
	\end{center}
	\caption{Comparison of Cherenkov lateral distribution, the total number of photons collected from the whole shower at each spherical counting volume, from both CORSIKA and universality.}
	\label{LateralDistribution}
\end{figure}

\begin{figure}
	\begin{center}
		\includegraphics[width=\columnwidth]{ATD.png}
	\end{center}
	\caption{Comparison of arrival time distribution of Cherenkov photons at just one counter from both CORSIKA and universality. The fluctuation in the CORSIKA IACT signal is due to the concentration of charged particles into sub-showers.}
	\label{ArrivalTimes}
\end{figure}