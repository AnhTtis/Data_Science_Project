\section{Introduction}
Extensive air showers (EAS) from cosmic rays with energies at or below the knee of the cosmic ray energy spectrum produce a limited amount of fluorescence light. The Cherenkov light produced by such EAS, while similar in total number of photons to fluorescence light, is primarily beamed forward, giving a much larger flux in that direction and permitting the optical detection of EAS at lower energies\cite{TALE}. In a non-imaging Cherenkov detector array, the properties an air shower can be reconstructed using both the time-integrated lateral distribution of the light, as well as the width of the temporal signal, as seen in different detectors of the array. The reconstruction of showers using Cherenkov signals traditionally involves a comparison between data and phenomenologically-determined distributions. A deterministic model of the Cherenkov light distribution from an air shower with given parameters would allow one to perform an Inverse Monte Carlo (IMC) analysis, fitting shower parameters based on collected signals. Such a model will not reproduce the shower-to-shower fluctuations due to hadronic subshowers, and thus may not be useful in the discrimination of photonic from hadronic air showers. The model is conceived in the context of non-imaging Cherenkov detectors at air-shower energies above $10^{14}$ eV.

The plan of this paper is as follows: in Section~\ref{universality} we present a method  to convolve charged-particle energy and angular distributions with the charged-particle Cherenkov to generate a universal, Cherenkov-photon angular distribution. In Section~\ref{cmc} we present a Monte Carlo verification of the analytic solution. In section~\ref{YieldTable} we present a method of tabulating the average Cherenkov photon yield of an air shower. In section~\ref{CherenkovUniversality} we present a public repository where these various calculations have been implemented. In Section~\ref{table} we present a method of using the tabulated Cherenkov angular distribution to reproduce the signal of an air shower in a surface non-imaging Cherenkov array. In Section~\ref{CORSIKA} we present a comparison of the distributions generated to those generated by CORSIKA\cite{CORSIKA} with its IACT extension. Finally, in Section~\ref{CHASM} we describe the CHerenkov Air Shower Model (CHASM), a python module where users can input shower parameters and determine the Cherenkov signal at desired locations.