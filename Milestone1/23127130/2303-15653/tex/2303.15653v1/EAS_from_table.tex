\section{Calculating EAS Cherenkov Signal from Distribution Tables} \label{table}

For a given EAS trajectory, vectors $\va{r}_a$ from the origin (where the shower axis meets the Earth's surface) to evenly spaced points on the axis are calculated. Based on the locations of supposed photon counters $\va{r}_c$, travel vectors $\va{r}_t = \va{r}_c - \va{r}_a$ from the axis to the counters are calculated. A shower profile, either given directly or calculated from parameters, as a function of slant depth is then assigned to the axis based on the depth intervals between axis points. A shower development stage $t$ is assigned to each axis point according to this profile. An atmospheric delta $\delta$ is also calculated based on the altitude of each sampled axis point.

The total number of Cherenkov photons produced per meter per charged particle at a shower stage is found by accessing the pre-compiled table described above in Section~\ref{YieldTable}. To first order, we assume that the distance each charged particle travels during each sampled stage of shower development is similar to the spatial interval represented by each corresponding axis interval $\ud r_a \approx \ud x_e$. We multiply the spatial intervals $\ud r_a$ by the tabulated yields per particle per meter from equation \ref{stage_yield}. This is the total number of Cherenkov photons produced over all solid angles at each sampled stage. 

The fractions of photons going from the specific point in the EAS toward specified counting locations are found by sampling the table of angular distributions described in Section~\ref{universality}. Based on the shower stages $t$, as well as the atmospheric delta $\delta$ at each axis point, a Cherenkov angular distribution is calculated via interpolation from the table. This distribution is sampled at the angles which the vectors $\va{r}_t$ make with the shower axis.

Before factoring in detector response and acceptance, each counter simply represents an amount of solid angle $\ud \Omega$ as seen by the production point on the axis. When multiplied by the photon density found from the tables, we find the total number of photons arriving from a specific axis point. 
\begin{equation}\label{Ng}
    N_\gamma(\theta, \; t, \; \delta) = 2 \; \expval{\frac{\ud^2 N_\gamma}{\ud x_e \; \ud N_e}(t,\delta)} \; N_e(t) \; g_\gamma(\theta; \; t, \; \delta) \; \ud \Omega \; \ud r_a
\end{equation}


% \begin{figure}
% 	\begin{center}
% 		\includegraphics[width=\columnwidth]{lateral_spread_geometry.png}
% 	\end{center}
% 	\caption{geometry}
% 	\label{lateral_spread_geometry}
% \end{figure}