\section{Lateral Distribution}
As EAS develop, they spread out in space. To first order, the source locations of Cherenkov light in an EAS can be assigned to points along the shower axis. For Cherenkov telescopes far away from the axis, this approximation is sufficient. However, in the regime where the distance between the shower axis and the counter location is similar to the lateral spread of the shower itself, it is necessary to spread the Cherenkov production vectors away from the shower axis. The universality of shower width has been established in the domain of dimensionless Moliere units $X$, to account for shower development in various atmospheric densities, and has been parameterized by several groups \cite{Lafebre} \cite{NKG}. 

\begin{equation}
    X = \frac{r}{r_m}
\end{equation}

\begin{equation}
    r_m \approx \frac{96 } {\rho(h) } \frac{kg/m^2}{kg/m^3}
\end{equation}

For this study, it is sufficient to use the energy independent lateral distribution parameterized by \cite{NKG}. This distribution is normalized over all log Moliere units $l_x = \ln{X}$ \cite{Lafebre}.

\begin{equation}
  G_M(t,l_X) = \frac{\partial N_e}{\partial l_X} = C_0 X^{\zeta_0}(X_1+X)^{\zeta_1}
\end{equation}

To find the relative density of particles per unit area we perform two changes of variables. First from the $l_X$ domain to $X$.

\begin{eqnarray}
  \frac{\partial N_e}{\partial X} &=& \frac{\partial N_e}{\partial l_X} \frac{\partial l_X}{\partial X}\\
                        &=& \frac{\partial N_e}{\partial l_X} \frac{1}{X}
\end{eqnarray}

Now we convert to density of particles per unit area.

\begin{eqnarray}
  \frac{\partial N_e}{\partial r} &=& \frac{\partial N_e}{\partial X} \frac{\partial X}{\partial r}\\
                        &=& \frac{\partial N_e}{\partial X} \frac{1}{r_m}
\end{eqnarray}

\begin{eqnarray}
  \frac{\partial N_e}{\partial A} &=& \frac{\partial N_e}{\partial r} \frac{1}{2 \pi r}\\
                        &=& \frac{\partial N_e}{\partial r} \frac{1}{2 \pi X r_m}\\
                        &=& \frac{\partial N_e}{\partial l_X} \frac{1}{2 \pi X^2 r_m^2}\\
\end{eqnarray}

To find the average distance from the shower axis, we integrate $\frac{\partial N_e}{\partial A}$ over all area in the shower cross section for a given stage $t$ and Moliere radius $r_m(h)$.

\begin{eqnarray}
    \overline{r} &=& 2 \pi \int_0^\infty \ud r \; r^2 \frac{\partial N_e}{\partial A} \\
                 &=& \int_0^\infty \ud r \; \frac{\partial N_e}{\partial l_X}
\end{eqnarray}

where $l_X(r,r_m) = \ln{X} = \ln{\frac{r}{r_m}}$. Converting back to Moliere units so this average can be used independently of atmospheric density:

\begin{equation}
    \frac{\partial N_e}{\partial A_X} = \frac{\partial N_e}{\partial X} \frac{1}{2 \pi X} = \frac{\partial N_e}{\partial l_X} \frac{1}{2 \pi X^2}
\end{equation}

\begin{eqnarray}
    \overline{X} &=& 2 \pi \int_0^\infty \ud X \; X^2 \frac{\partial N_e}{\partial A_x} \\
                 &=& \int_0^\infty \ud X \; \frac{\partial N_e}{\partial l_X}
\end{eqnarray}