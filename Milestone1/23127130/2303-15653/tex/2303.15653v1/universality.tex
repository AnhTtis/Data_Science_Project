\section{Universal Cherenkov Angular Distribution}\label{universality}
Although EAS develop in a stochastic cascade, certain characteristics of the particles in the cascade, including energy and angle, have been shown to represent samples from universal probability distributions \cite{Lafebre,Nerling,Giller}. This is only the case for showers large enough that the distributions of shower-particle properties are meaningful. Both the energy and angular distributions of charged particles are universal in shower stage $t$, and are independent of primary cosmic-ray species \cite{Lafebre}. The charged particles in an EAS will produce Cherenkov light if their energy exceeds the Cherenkov threshold at that point in the atmosphere. The Cherenkov light is produced in a cone of given angle according to the atmospheric index-of-refraction in which they propagate.

The energy of a charged particle in a shower is drawn from a universal distribution $f_e(E_e;t) = \frac{\ud n_e}{\ud l}$ where $t$ is the shower stage and $l=\ln E_e$ with $E_e$ in MeV. For convenience we use $l=\ln E$ to represent the log-energy of secondary particles. The charged particle energy distribution is only dependent on the stage of shower development. To be specific, we will use shower stage as parameterized, $t=\frac{X-X_{\rm max}}{X_o}$\cite{Lafebre}, where $X_o$ is the radiation length of the medium, though other parameterizations could be used such as shower age, $s=\frac{3X}{X+2X_{\rm max}}$\cite{Nerling}.

In the absence of significant geomagnetic field effects, we assume azimuthal symmetry. The angle $\theta_e$ that a charged particle of a given energy makes with the shower axis is drawn from the universal angular distribution $g_e(\theta_e;l_e) = \frac{\ud n_e}{\ud \Omega_e}$. This distribution depends only on the charged particle energy and is independent of stage \cite{Giller}.

At a given shower stage, $t$, and atmospheric index-of-refraction, $n=1+\delta$, we wish to calculate the relative Cherenkov photon distribution at an angle $\theta$ from the shower axis direction \nhat. We define this direction to be \ghat. A charged particle of log-energy $l_e$ may emit Cherenkov photons toward \ghat, if its direction from shower axis, \ehat, makes an angle with \ghat\ matching the local Cherenkov cone angle, $\theta_{\rm \check{C}}$. The Cherenkov cone angle is a function of both the electron energy and the local index-of-refraction. The angle between \nhat\ and \ehat\ is denoted $\theta_e$. These angles, and the corresponding interior angles $\phi$, $\phi_e$, and $\phi_\gamma$, form a spherical triangle as shown in Figure~\ref{geometry}.

\begin{figure}
	\begin{center}
		\includegraphics[width=\columnwidth]{sphericaltriangle.png}
	\end{center}
	\caption{The angles and directions involved in producing a Cherenkov photon at an angle $\theta$ from the shower axis. The interior ($\phi$) angles are labeled according angle on the opposite side ($\theta$).}
	\label{geometry}
\end{figure}

The number of electrons (charged particles) of a given energy going in direction $\ehat $ is found from the known energy and angular distributions.
\begin{equation}\label{dNe}
    \ud N_e = f_e(E_e; t)\,g_e(\theta_e;E_e)\;\ud l\;\ud\Omega_e
\end{equation}

Charged particles going towards $\ehat $ with $E_e > E_{\rm Th}$, where $E_{\rm Th}$ is the Cherenkov threshold for the given index-of-refraction, will produce Cherenkov photons according to their path length and energy. If we divide out the maximum photon yield of a hyper-relativistic charged particle, $\ud N_\gamma/\ud X (E_e\gg E_{\rm Th})$, we have the relative photon yield 
\begin{equation} \label{yield}
    Y_{\rm \check{C}}(E_e > E_{\rm Th}) = 1-(E_{\rm Th}/E_e)^2
\end{equation}
This is the relative probability that a given charged particle with energy $E_e$ will produce a Cherenkov photon while propagating through a medium with Cherenkov threshold $E_{\rm Th}$. Some of these photons will go into the solid angle about \ghat\ as long as $\theta_\gamma$, the angle a Cherenkov photon could have, matches $\theta_{\rm \check{C}}$. Thus the fraction of photons going towards $\hat{\gamma}$ from electrons going towards \ehat\ is
\begin{equation}\label{dNg}
  \ud N_\gamma = Y_{\rm \check{C}}\,
  \delta(\theta_\gamma-\theta_{\rm \check{C}})\,
  \frac{\ud \Omega_\gamma}{2\pi}
\end{equation}

We convolve the distributions given in equations \ref{dNe} and \ref{dNg} to produce a single relative value of $g_\gamma(\theta;t,\delta) = \frac{\ud n_\gamma}{\ud \Omega_\gamma}$, the angular distribution of Cherenkov photons, at a given $\theta$ away from the shower axis and at a given $t$ and $\delta$. After the integration, the distribution must be normalized because Cherenkov cones from individual electrons simultaneously contribute to multiple increments of $\Omega_\gamma$.
\begin{equation} \label{gg1}
    g_\gamma \propto \int\ud\Omega_e \int\displaylimits_{l_{\rm Th}}^\infty\ud l\ Y_{\rm \check{C}}(l)\,f_e(l)\,g_e(\theta_e;\, l)\,\delta(\theta_\gamma-\theta_{\rm \check{C}})\\
\end{equation}

% \begin{equation} \label{gg1}
%     P_{l_x}(l) = \int\ud l_x\ n(t; l, l_x)
% \end{equation}

The key to easily performing the integral is to realize that $\ud\Omega_e$ can be defined in terms of variables related to \ghat\ rather than with respect to \nhat. Thus, $\ud\Omega_e = \sin\theta_\gamma\,\ud\phi_e\,\ud\theta_\gamma$. Also, the domain of $\ud\Omega_e$, which contributes to the solid angle part of the integral above, depends on variables related to \ghat. For a given energy increment, with $\theta$ fixed, the allowed values of $\theta_e$ are found through geometric constraints. When $\nhat \cdot \ehat$ is calculated, the spherical law-of-cosines is recovered for the spherical triangle shown in Figure~\ref{geometry}. 
\begin{equation}\label{loc}
    \cos{\theta_e(\phi_e, \theta_\gamma)} = \nhat \cdot \ehat = \cos{\theta}\cos{\theta_\gamma} + \sin{\theta}\sin{\theta_\gamma}\cos{\phi_e}
\end{equation}
Thus, $g_e$ is a function of $\phi_e$ and $\theta_\gamma$ for a given $l$.

Now we can leverage the delta function to compute the $\theta_\gamma$ part of the angular integral, since for a given $l$, $\theta_{\rm \check{C}}$ is a constant.
% \begin{align} \label{gg_int}
%     g_\gamma &\propto
%     \int\displaylimits_{l_{\rm Th}}^\infty\ud l\ Y_{\rm \check{C}}(l)\,f_e(l)         \int\displaylimits_{0}^{2\pi}\ud \phi_e \int\displaylimits_{0}^{\pi}\ud \theta_\gamma\ \sin{\theta_\gamma}\,g_e(\phi_e,\, \theta_\gamma;\, l)\,\delta(\theta_\gamma-\theta_{\rm \check{C}})\\
%     &= \int\displaylimits_{l_{\rm Th}}^\infty\ud l\ \sin{\theta_{\rm \check{C}}(l)}\, Y_{\rm \check{C}}(l)\,f_e(l) \int\displaylimits_{0}^{2\pi}\ud \phi_e\ g_e(\phi_e,\, \theta_{\rm \check{C}}(l);\, l)
% \end{align}
\begin{eqnarray}\label{gg_int}
    g_\gamma &\propto&
        \int\displaylimits_{l_{\rm Th}}^\infty\ud l\ Y_{\rm \check{C}}(l)\,f_e(l) \times\\
    && \int\displaylimits_{0}^{2\pi}\ud \phi_e \int\displaylimits_{0}^{\pi}\ud \theta_\gamma\ \sin{\theta_\gamma}\,g_e(\phi_e,\, \theta_\gamma;\, l)\,\delta(\theta_\gamma-\theta_{\rm \check{C}})\\
    &=& \int\displaylimits_{l_{\rm Th}}^\infty\ud l\ \sin{\theta_{\rm \check{C}}(l)}\, Y_{\rm \check{C}}(l)\,f_e(l) \times\\\label{gg_int2}
    && \int\displaylimits_{0}^{2\pi}\ud \phi_e\ g_e(\phi_e,\, \theta_{\rm \check{C}}(l);\, l)
\end{eqnarray}

The double integral in \ref{gg_int2} is performed numerically for a range of shower states and indices-of-refraction and tabulated. This requires specification of the $f_e$ and $g_e$ distributions. We use the distributions given in \cite{Bergman}, but others could be used. The distributions of \cite{Lafebre} should work as well. With others, e.g. \cite{Giller}, care must be taken not to extend the integral into parts of the parameters space where the phenomenologically determined distributions are not valid.

After the integration the distribution is normalized over all solid angle. However, as all charged particle azimuthal angles are equally likely, for every charged particle, there exists a second charged particle with the same $\theta_e$ but a different azimuthal angle $\phi_\gamma$ whose Cherenkov cone can also emit Cherenkov photons towards $\ghat$. This is related to the fact that each charged particle's Cherenkov cone intersects twice with the spherical annulus representing $\ud \Omega$. In other words, equation \ref{loc} has two roots with different values of $\phi_e$. This property implies that the normalization constant is effectively doubled. The projection of the spherical triangle from Figure~\ref{geometry} onto a unit sphere, as well as the Cherenkov cones of equally energetic charged particles, is shown in Figure~\ref{UnitSphere}.

\begin{figure}
	\begin{center}
		\includegraphics[width=\columnwidth]{unit_sphere.pdf}
	\end{center}
	\caption{The projection of the spherical triangle \ref{geometry} onto a unit sphere, showing directions \nhat, \ehat, and \ghat, as well as the Cherenkov cones of two charged particles with the same $\theta_e$.}
	\label{UnitSphere}
\end{figure}



