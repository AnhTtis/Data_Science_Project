\begin{abstract}
    The reconstruction of cosmic-ray-induced extensive air showers with a  non-imaging Cherenkov detector array requires knowledge of the Cherenkov yield of any given air shower for a given set of shower parameters. Although air showers develop in a stochastic cascade, certain characteristics of the particles in the shower have been shown to come from universal probability distributions, a property known as shower universality. Both the energy and the angular distributions of charged particles within a shower have been parameterized. One can use these distributions to calculate the Cherenkov photon yield as an angular distribution from the Cherenkov cones of charged particles at various stages of shower development. This Cherenkov photon yield can then be tabulated for use in the reconstruction of air showers. In this work, we develop the calculation of both the Cherenkov angular distribution and Cherenkov yield per shower particle, and show how a look-up table was constructed to capture the relevant features of these distributions for general use. We compare the results of our calculations with the results of full, particle-stack, Monte Carlo simulation of the Cherenkov light produced in extensive air showers using CORSIKA-IACT. We make comparisons of both the lateral distribution of the Cherenkov photon flux amongst several detectors and of the arrival-time distribution of the Cherenkov photons in a single detector.
\end{abstract}
