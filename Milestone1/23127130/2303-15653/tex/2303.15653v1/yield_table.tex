\section{Tabulation of Average Photon Yield}\label{YieldTable}
The number of Cherenkov photons produced by a charged particle in a given propagation interval depends on its energy relative to the local Cherenkov threshold. The photon yield of a charged particle at a given stage $t$ of an EAS and atmospheric $\delta$ is a function of particle energy. Thus, the average value can be found by using the charged particle energy distribution $f_e(E_e, t)$:
\begin{equation}\label{stage_yield}
    \expval{\frac{\ud^2 N_\gamma}{\ud x_e \; \ud N_e}(t,\delta)} = \int\displaylimits_{l_{\rm Th}}^\infty\ud l \; \frac{\ud N_\gamma}{\ud x_e}(E_e,\delta) \; f_e(E_e, t)
\end{equation}
with
\begin{equation} \label{FrankTamm}
    \frac{\ud N_\gamma}{\ud x_e} = 2 \pi \alpha \; \bigg[1 - \frac{1}{(n \beta)^2} \bigg] \; \bigg( \frac{1}{\lambda_{min}} - \frac{1}{\lambda_{max}} \bigg) \;\;\; \left[\gamma \; \mathrm{m^{-1} e^{-1}}\right]
\end{equation}
where $x_e$ is the charged particle propagation interval (in meters), $\alpha$ is the fine structure constant, $\beta$ is the electron's speed in units of $c$ at energy $E_e$, $n$ is the local index of refraction, and $\lambda_{min}$ and $\lambda_{max}$ are the minimum and maximum Cherenkov light wavelengths accepted by the detectors. Since the accepted wavelength interval factor is separable, the value of the rest of the integral has been tabulated at various values of $t$ and $\delta$.