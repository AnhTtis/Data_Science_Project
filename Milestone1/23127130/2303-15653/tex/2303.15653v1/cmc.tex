\section{Cherenkov Monte Carlo}\label{cmc}

To show the validity of the integration in the previous section, Cherenkov photons were generated via Monte Carlo for comparison. At a given stage of shower development and with a given index-of-refraction (atmospheric $\delta$, which also gives a particular Cherenkov threshold energy), charged particles were drawn from the universal energy distribution $f_e(E_e;t)$ and $g_e(\theta_e;l_e)$. For each of these particles a Cherenkov photon was or was not produced by drawing from the relative Cherenkov yield, equation~\ref{yield}. If produced, a Cherenkov photon was given random azimuthal angle $\phi$. For the generated Cherenkov photons, the angle with respect to the shower axis is now constrained geometrically. In this case, when we take $\nhat \cdot \ghat$ we again recover the spherical law of cosines for figure \ref{geometry}, this time solving for $\theta$.
\begin{equation}\label{mcloc}
    \cos{\theta(\theta_e,\phi, \theta_\gamma)} = \nhat \cdot \ghat = \cos{\theta_e}\cos{\theta_\gamma} + \sin{\theta_e}\sin{\theta_\gamma}\cos{\phi}
\end{equation}

The generated $\theta$ values are collected in bins centered around the angles tabulated by the convolution integral, weighted based on the amount of differential solid angle they represent, and the distribution values are normalized. Since the energy distribution only depends on the stage of shower development, the same set of particle energies can be used to compute the Cherenkov distribution of that stage occurring at various heights in the atmosphere. In Figure~\ref{mc_compare} we show the results of throwing $\num{5e8}$ charged particles and compare the distribution to one with the same parameters generated by convolution.

\begin{figure}
	\begin{center}
		\includegraphics[width=\columnwidth]{mc_table_comparison.png}
	\end{center}
	\caption{Comparison between a Cherenkov distribution at $t=0$ and $\delta=10^{-4}$ where the integration is performed by convolution and Monte Carlo respectively. The bins used to collect the Monte Carlo data were chosen to center around the angles tabulated by the convolution integral. While there are slight differences at small angles the peak angle matches exactly.}
	\label{mc_compare}
\end{figure}