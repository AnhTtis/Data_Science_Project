% \begin{figure*}[t]
%     \centering
%     \includegraphics[scale=0.078]{figs/pa/res_pa.jpg}
%     \caption{Samples of perturbation attacks applied to sentences in the test dataset. The original images and perturbed images are generated from the same seeds.}
%     \label{fig:res_pa}
% \end{figure*}

\begin{figure*}[t]
    \centering
    \subfloat[\label{fig:bike}]{\includegraphics[scale=0.078]{figs/pa/pa1.pdf}}
    \hspace{1mm}
    \subfloat[]{\includegraphics[scale=0.078]{figs/pa/pa2.pdf}}
    \hspace{1mm}
    \subfloat[]{\includegraphics[scale=0.078]{figs/pa/pa3.pdf}}
    \hspace{1mm}
    \subfloat[]{\includegraphics[scale=0.078]{figs/pa/pa4.pdf}}
    \hfill
    \subfloat[]{\includegraphics[scale=0.078]{figs/pa/chess.pdf}}
    \hspace{1mm}
    \subfloat[]{\includegraphics[scale=0.078]{figs/pa/sea.pdf}}
    \hspace{1mm}
    \subfloat[]{\includegraphics[scale=0.078]{figs/pa/frog.pdf}}
    \hspace{1mm}
    \subfloat[]{\includegraphics[scale=0.078]{figs/pa/dog.pdf}}
    \hfill

    \caption{Illustrations of the effect of \textit{untargeted} query-free attacks. In each group, the first row of images is generated using the original prompts vs. the second row using the perturbed ones. The perturbations found by our method are highlighted in \textcolor[RGB]{22,113,250}{blue} in the prompt. Images in the same column share the same random seed.}
    \label{fig:res_pa}
\end{figure*}