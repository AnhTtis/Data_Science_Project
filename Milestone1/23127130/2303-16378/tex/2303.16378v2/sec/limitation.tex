% \section{Limitations}
% %Although the proposed query-free attacks have demonstrated promising performances, they may not always perform well due to the limited available information in the black box setting. Specifically, 
% Although the proposed query-free attack is able to identify patterns that may cause low loss (cosine similarity) with original sentences, such perturbation patterns may not always succeed in misleading the model. In the black box setting, the parameters in the cross attention module are not available, which allows diffusion models to generate robust images even with low-loss perturbation inputs like `-E3=7' causing $loss = 0.44$ that is shown in Fig.\,\ref{fig:loss}. In the contract, there are prompts that keep high loss (cosine similarity) with original sentences but shift their semantic meanings like `\#K\&21' causing $loss = 0.80$. \YH{Do we really need this section?}
% \begin{figure}[tb]
%     \centering
%     \subfloat[Original]{\includegraphics[scale=0.12]{figs/man/man.png}}\hspace{2mm}
%     \subfloat[`-3=E7']{\includegraphics[scale=0.12]{figs/man/E7.png}}\hspace{2mm}
%     \subfloat[`\#K\&21']{\includegraphics[scale=0.12]{figs/man/k21.png}}\hfill
%     % \caption{Examples for prompts ``a young man with a yellow hat'' followed by perturbation prompts ``-3=E7'' with low perturbation loss (0.43) but causing no change on meanings and perturbation prompt ``\#K\&21'' with high perturbation loss (0.8) but changing semantic meanings.}
%     \caption{Samples for the lower-loss perturbation word outperforming higher-loss perturbation word. The left image is the original image generated by the prompt `a young man with a yellow hat'. The prompt is followed by two perturbation prompts: the middle one `-3=E7' with a low perturbation loss (0.43) that causes no change in meaning, and the right one `\#K\&21' with a high perturbation loss (0.8) that changes the semantic meaning.}
%     \label{fig:loss}
% \end{figure}
