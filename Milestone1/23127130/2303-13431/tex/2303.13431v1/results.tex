
\section{Results}
\label{sec:results}

The inclusive $\Jpsi$ pair production cross section in the kinematic interval $2.5<y<4.0$ and $\pt>0$ is measured to be 
\begin{equation}
     \sigma ({\rm J}/\psi \,\, {\rm J}/\psi ) = 10.3 \pm 2.3 \, {\rm (stat.)} \pm 1.3 \, {\rm (syst.)} \,\, {\rm nb}. \nonumber
\end{equation}
%
The ratio of the production cross section of the inclusive $\Jpsi$ pair to that of the inclusive $\Jpsi$ is %computed to be 
\begin{equation}
     \frac{ \sigma ({\rm J}/\psi \,\, {\rm J}/\psi ) } { \sigma ({\rm J}/\psi) } = ( 9.1 \pm  2.0 \, {\rm (stat.)} \pm  1.3 \, {\rm (syst.)} ) \times 10^{-4} \,, \nonumber
\end{equation}
considering ${\rm d}\sigma ({\rm J}/\psi) / {\rm d} y = (7533.3 \pm 26.7 \, {\rm(stat.)} \pm 491.6 \, {\rm (syst.)} )$~nb for $\pt>0$ and $2.5<y<4.0$~\cite{ALICE:2017leg}, and assuming the systematic uncertainties to be uncorrelated. 
Likewise, the ratio 
\begin{equation}
    \frac{1}{2} \frac{ \sigma ({\rm J}/\psi)^2 } { \sigma ({\rm J}/\psi \,\, {\rm J}/\psi ) } = 6.2 \pm \, 1.4 \, {\rm (stat.)} \pm  1.1 \, {\rm (syst.)}  \,\, {\rm mb}, \nonumber
\end{equation}
can be calculated and interpreted as an effective cross section, according to Eq.~(\ref{eq:DPSred}). This interpretation assumes that all $\Jpsi$ pairs are produced via DPS processes. 
The relative contribution of SPS and DPS processes to $\Jpsi$ pair production is a topic of debate and intense studies, see e.g. Ref.~\cite{LHCb:2016wuo}. 
In addition, the understanding of this ratio gets challenged by the contribution of both the prompt and non-prompt components to the measured inclusive $\Jpsi$ cross section, where the non-prompt contribution originates from beauty-hadron decays. 


The contamination from beauty-hadron decays to the $\Jpsi$ pair cross section is evaluated to get a grasp of its influence to the measurement according to
\begin{equation}
    \sigma_{\rm non-prompt} ({\rm J}/\psi \,\, {\rm J}/\psi ) = \sigma_{\bbbar}^{\rm total} \times \alpha \times B^2(h_b \rightarrow \Jpsi\ \  + X)  \,.
    \nonumber
\end{equation}
The total beauty-hadron production cross section was measured to be
\begin{displaymath}
\sigma_{\bbbar}^{\rm total} =  502 \pm 16~{\rm (stat.)} \pm 51~{\rm (syst.)} _{-3}^{+2}~{\rm (extr.)}~\mu\text{b}
\end{displaymath}
 in Ref.~\cite{ALICE:2021edd}. 
The branching ratio of a beauty hadron into a $\Jpsi$ is $B(h_b \rightarrow \Jpsi + X) = (1.16 \pm 0.10) \%$~\cite{Workman:2022ynf}, 
and the acceptance correction factor $\alpha$ is estimated using PYTHIA 8.3~\cite{Bierlich:2022pfr} simulations. 
Beauty hadrons are simulated according to three different configurations and forced to decay into $\Jpsi$. 
The three configurations use the Monash 2013 tune for the calculation~\cite{Skands:2014pea}. Two of them also include a tuning of the parameters to get a good agreement with the NLO calculation by Mangano, Nason, and Ridolfi for the $\bbbar$ single and double differential distributions~\cite{Mangano:1991jk}. The difference between the latter two is that one of them adds the ATLAS tune settings for multiple parton interactions~\cite{TheATLAScollaboration:2014rfk}. 
The $\alpha$ factor is obtained from the ratio of the $\Jpsi$ pair counts in the acceptance to the number of all $\Jpsi$ pairs in the simulation. 
The value of $\alpha = 0.044^{+0.005}_{-0.007}$ is determined as the average of the factors obtained with all configurations, and the systematic uncertainty is conservatively set to the full spread of the values. 
This gives a non-prompt contribution of
\begin{equation}
    \sigma_{\text{non-prompt}} ({\rm J}/\psi \,\, {\rm J}/\psi ) = 2.97 \pm \, 0.09 \, {\rm (stat.)} ^{+0.68}_{-0.76} {\rm (syst.)}  \,\, {\rm nb}, \nonumber
    \nonumber
\end{equation}
and correspondingly the prompt $\Jpsi$ pair cross section is
\begin{equation}
    \sigma_{\rm prompt} ({\rm J}/\psi \,\, {\rm J}/\psi )  
    =  \sigma ({\rm J}/\psi \,\, {\rm J}/\psi )  - \sigma_{\text{non-prompt}} ({\rm J}/\psi \,\, {\rm J}/\psi ) 
    = 7.3 \pm \, 1.7 \, {\rm (stat.)} ^{+1.9}_{-2.1} {\rm (syst.)}  \,\, {\rm nb}. \nonumber
\end{equation}
Analogously, for the single $\Jpsi$ case, the computed extrapolation factor to account for the number of $\Jpsi$ from beauty decays in the acceptance is $\beta=0.121 ^{+0.001}_{-0.002}$. Thus, the non-prompt contribution to the $\Jpsi$ production cross section is 
\begin{equation}
    \sigma_{\text{non-prompt}} ({\rm J}/\psi ) = 2 \times \sigma_{\bbbar}^{\rm total} \times \beta \times B(h_{\rm b} \rightarrow \Jpsi\ \  + X)  = 1.41 \pm \, 0.04 \, {\rm (stat.)} \pm 0.19 {\rm (syst.)}  \,\, \mu{\rm b}, \nonumber  
\end{equation}
and the prompt component is evaluated to be 
\begin{equation}
    \sigma_{\rm prompt} ({\rm J}/\psi ) 
    = \sigma ({\rm J}/\psi)  - \sigma_{\text{non-prompt}} ({\rm J}/\psi) 
     = 9.89 \pm \, 0.32 \, {\rm (stat.)} ^{+1.47}_{-1.48} {\rm (syst.)}  \,\, \mu{\rm b}. \nonumber
    \nonumber
\end{equation}
Therefore, the ratios discussed earlier in this section can be evaluated for the prompt case. 
The ratio of the prompt $\Jpsi$ pair production cross section to that of $\Jpsi$ equals
\begin{equation}
     \frac{ \sigma_{\rm prompt} ({\rm J}/\psi \,\, {\rm J}/\psi ) } { \sigma_{\rm prompt}  ({\rm J}/\psi) }
       = ( 7.4 \pm  1.7 \, {\rm (stat.)} \pm  2.2 \, {\rm (syst.)} ) \times 10^{-4} \,, \nonumber
\end{equation}
and the ratio related to the effective DPS cross section becomes
\begin{equation}
    \frac{1}{2} \frac{ \sigma_{\rm prompt}  ({\rm J}/\psi)^2 } { \sigma_{\rm prompt}  ({\rm J}/\psi \,\, {\rm J}/\psi ) } 
       = 6.7  \pm 1.6 \, {\rm (stat.)} \pm  2.7 \, {\rm (syst.)}  \,\, {\rm mb}. \nonumber
\end{equation}
%
A differential measurement of the prompt $\Jpsi$ pair production cross section and the corresponding ratios were previously reported by the LHCb collaboration in a slightly different kinematic interval, $2.0<y<4.5$ and $\pt<10$~GeV/$c$~\cite{LHCb:2011kri,LHCb:2016wuo}. The results presented here are in agreement with the LHCb ones within uncertainties. 


Despite the caveat caused by the calculation of this effective value considering both the SPS and DPS contributions to the production cross section, this value is consistent with the values obtained from quarkonium-pair production measurements, with $\seff$ values ranging from 2.2 to 12.5~mb~\cite{D0:2014vql,ATLAS:2016ydt,LHCb:2016wuo,D0:2015dyx}, 
and with the values obtained for quarkonium associated production at central rapidity (in the range 2.3--6.1 mb)~\cite{Lansberg:2016muq,Lansberg:2017chq,Lansberg:2019adr}. 
It is smaller than the values obtained for associated heavy-flavour production at large rapidity by LHCb (ranging from 12.8 to 18.0 mb)~\cite{LHCb:2012aiv,LHCb:2015wvu}, or those from jet or electroweak associated production (whose values are between 12.0 and 21.3 mb)~\cite{CDF:1997yfa,D0:2009apj,CDF:1993sbj,CMS:2015wcf,CMS:2013huw,ATLAS:2013aph,CMS:2019jcb}.





\section{Conclusion}

The production cross section of $\Jpsi$ pairs at large rapidity in \pp collisions at \thirteen was studied by ALICE. The measurement exploits the full Run~2 data sample collected by ALICE. 
The production cross section of inclusive $\Jpsi$ pairs is reported to be $10.3 \pm 2.3 \, {\rm (stat.)} \pm 1.3 \, {\rm (syst.)}$~nb,  for $\Jpsi$ in the rapidity interval $2.5<y<4.0$ and for $\pt>0$. 
The results are compatible with analogous measurements performed by the LHCb collaboration in a similar kinematic interval~\cite{LHCb:2011kri,LHCb:2016wuo}. 


The Run~3 data taking, with the upgraded ALICE detector and the larger accumulated luminosity~\cite{ALICE:2023udb}, will allow us to perform this measurement with increased precision and separating the prompt and non-prompt contributions. This will also enable studying the kinematics of these events and probe model calculations. 
