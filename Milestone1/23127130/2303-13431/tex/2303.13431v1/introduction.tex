\section{Introduction}
\label{sec:intro}

%The production of particle pairs

In the quantum chromodynamics (QCD) parton model~\cite{Bjorken:1969ja}, hadrons are composed of elementary constituents, the partons. Due to the composite nature of hadrons, multiple parton hard scatterings can occur in a hadron--hadron collision. Thus, it is possible to have two or more hard parton interactions simultaneously. 
% 
%
%
Multiple parton interactions (MPI) have been studied since the introduction of the parton model~\cite{Takagi:1979wn,Goebel:1979mi}. 
Further studies included the generalization of the QCD evolution equations into multiparton distribution and fragmentation functions 
\cite{Kirschner:1979im,Shelest:1982dg}, 
and a discussion on the possible correlations in the colour and spin degrees of freedom~\cite{Mekhfi:1985dv}. 
%
%
Double-parton scatterings (DPS) are the simplest case of MPI, 
and were found to play the most important role in processes with final states such as four jets, four leptons or $n$-jet + W/$\gamma$ measurements 
\cite{Paver:1982yp,Paver:1983hi,Mekhfi:1983az,Humpert:1983pw,Humpert:1983fy,Humpert:1984ay,Ametller:1985tp,Halzen:1986ue,Godbole:1989ti}. 
These studies were complemented by several measurements in hadron collisions at center-of-mass energies ($\s$) ranging from 63 GeV to 1.96 TeV ~\cite{AxialFieldSpectrometer:1986dfj,UA2:1991apc,CDF:1993sbj,CDF:1997lmq,D0:2009apj,D0:2014owy,D0:2014vql}. 


At the LHC energies, the probability to have multiple parton interactions increases: as with the increase of collision energy, partons with smaller momentum fraction $x$ are probed, with larger fluxes.
Recent measurements have shown the relevance of MPI at the LHC 
\cite{LHCb:2012aiv,ATLAS:2013aph,CMS:2013huw,ATLAS:2014ofp,LHCb:2015wvu,ALICE:2022wpn}, 
and have contributed to stimulate recent progresses in the theoretical understanding of MPI~\cite{Blok:2011bu,Diehl:2011yj,Gaunt:2014ska,Diehl:2014vaa,Gaunt:2014rua}. 
Nevertheless, a quantitative estimate of the DPS impact on observables remains challenging. 
Neglecting the parton correlations in the proton, the DPS contribution to a final state \mbox{$A + B$} can be evaluated as the product of the parton level cross sections ($\widehat{\sigma}$) divided by an effective cross section ($\seff$)~\cite{Humpert:1984ay,Ametller:1985tp,DelFabbro:2000ds} 
\begin{equation}
\sigma^{\rm DPS}_{A,B} = \frac{m}{2} \,\frac{\widehat{\sigma}^{A} \, \widehat{\sigma}^{B}}{\seff}   \,,
\label{eq:DPSred} 
\end{equation}
 where the parameter $m$ is a symmetry factor, $m=1$ if $A=B$, and 2 otherwise. The effective cross section is a phenomenological parameter related to the transverse overlap function between the partons of the proton, and is thought to be universal. 
It was found to range between 2 and 25 mb 
\cite{D0:2014vql,ATLAS:2016ydt,LHCb:2016wuo,D0:2015dyx,Lansberg:2016muq,Lansberg:2017chq,LHCb:2015wvu,LHCb:2012aiv,CDF:1997yfa,D0:2009apj,CDF:1993sbj,CMS:2015wcf,CMS:2013huw,ATLAS:2013aph,CMS:2019jcb}. 


Double particle production is typically exploited to study DPS. 
A non exhaustive list of these studies are the measurements of the production cross sections of double quarkonium,  i.e.\ ${\rm J}/\psi$ pairs~\cite{NA3:1982qlq,PLB198585,ATLAS:2016ydt,CMS:2014cmt,LHCb:2011kri,LHCb:2016wuo,D0:2014vql}, $\Upsilon$ pairs~\cite{CMS:2016liw}, or ${\rm J}/\psi + \Upsilon$~\cite{D0:2015dyx}, 
electroweak boson plus quarkonium %\cite{D0:2014vql,ATLAS:2014jkm,ATLAS:2014ofp,LHCb:2016wuo,D0:2015dyx,Lansberg:2016muq,Lansberg:2017chq,LHCb:2015wvu}, 
\cite{D0:2014vql,ATLAS:2014jkm,ATLAS:2014ofp,Lansberg:2016muq,Lansberg:2017chq,LHCb:2015wvu}, 
double charm production~\cite{LHCb:2012aiv},
charmed hadrons plus quarkonium~\cite{LHCb:2015wvu,LHCb:2012aiv}, 
electroweak boson plus open charm ~\cite{LHCb:2012aiv,LHCb:2014kke}, 
 as well as measurements with jets in the final state, multi-jets~\cite{AxialFieldSpectrometer:1986dfj,UA2:1991apc,CDF:1993sbj,ATLAS:2016rnd}, 
$\gamma + 3$-jets~\cite{CDF:1997lmq,D0:2009apj,D0:2014owy}, 
$2\gamma + 2$-jets~\cite{D0:2015rpo}, and
W + 2-jets~\cite{CMS:2013huw}. 
The recent observation of triple J/$\psi$ production proposes an additional channel to study double and triple parton scatterings 
\cite{CMS:2021qsn}. 


 In the quarkonium sector, quarkonium-pair production is a golden tool to probe the production mechanism of heavy quarkonia~\cite{Ko:2010xy,Sun:2014gca,Baranov:2015cle}. 
 The production mechanism of heavy quarkonia is not fully understood after more than forty years of study, and considered a long-standing puzzle of QCD. 
The colour-singlet model (CSM), which assumes the formation of an intermediate $\QQbar$ state with the quantum numbers of the final state, underestimates the production cross section at high $\pt$ both at leading-order (LO) and next-to-leading-order (NLO)~\cite{Lansberg:2011hi,Campbell:2007ws,Gong:2008sn}. 
The recent CSM 
next-to-next-to-leading-order
NNLO$^\star$ calculations have reduced the discrepancies~\cite{Lansberg:2011hi,Lansberg:2019adr}. 
Non-relativistic QCD (NRQCD) calculations consider both colour-singlet (CS) and colour-octet (CO) states of the $\QQbar$ pair~\cite{Bodwin:1994jh}, but fail to predict at the same time the production cross section and polarisation~\cite{LHCb:2013izl,LHCb:2014brf,CMS:2012bpf,CMS:2013gbz,ALICE:2011gej}. 
The selection rules for pair production in the CS process of LO NRQCD forbid the feed-down from cascade decays of excited charge-conjugate-even states, e.g.\ $\chi_{\rm c} \rightarrow {\rm J}/\psi \, \gamma$, whose contribution is significant in single quarkonium production, and makes difficult the comparison of data to model calculations.
As a consequence, quarkonium-pair production provides stringent tests of model calculations. 


In this letter we report the measurement of inclusive $\Jpsi$ pair production cross section in \pp collisions at \mbox{$\s=13$~TeV} at large rapidity ($2.5<y<4.0$) with ALICE. 
Inclusive $\Jpsi$ results correspond to the sum two contributions: the prompt contribution, originated from direct charm decays or decays of higher-mass excited states; 
and the non-prompt contribution, steaming from beauty decays.
The results corroborate analogous measurements performed in a similar rapidity interval by LHCb~\cite{LHCb:2016wuo}. 
They constitute a probe to study quarkonium production mechanisms and the DPS contribution.