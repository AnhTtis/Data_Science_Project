
\section{Experimental apparatus and data sample}
\label{sec:detector}

A description of the ALICE detector and its performances can be found in Refs.~\cite{ALICE:2008ngc,ALICE:2014sbx}.
At forward rapidity ($2.5<y<4.0$) the production of quarkonium states is measured in the muon spectrometer down to $\pt = 0$ via their dimuon decay channel. 
The muon spectrometer of ALICE consists of a ten interaction length thick  front absorber to filter muons, five tracking stations of two planes of cathode pad chambers each (MCH), a dipole magnet with a field integral of 3 Tm surrounding the third tracking station, a 1.2 m thick iron wall to absorb secondary hadrons escaping from the front absorber and low momentum muons coming mainly from $\pi$ and K decays, and two trigger stations made of two planes of resistive plate chambers each (MTR)~\cite{ALICE:2011zqe}. 
The silicon pixel detector (\SPD) and scintillator arrays (\VZERO) are also used in this analysis. The \VZERO counters, two arrays of 32 scintillator tiles each, cover $2.8 \leq \eta \leq 5.1$ (\VZEROA) and $-3.7 \leq \eta  \leq -1.7 $  (\VZEROC) and provide trigger information. The minimum bias (MB) trigger requirement consists of a logical AND of  a signal in \VZEROA and in \VZEROC. The \SPD,  two cylindrical layers covering $|\eta| \leq  2.0$ and $|\eta| \leq 1.4$ for the inner and outer layers, respectively, is dedicated to the  vertex reconstruction and allows estimating pile-up. 
The maximum interaction rate for the analysed data sample is 260 kHz, and the maximum pile-up probability is about $5\times10^{-3}$, negligible for this measurement.


The $\Jpsi$ pair analysis is performed using data from pp collisions at \thirteen collected from 2016 to 2018. 
The event sample was selected using the dimuon trigger condition, which 
is defined as  the  coincidence between the MB requirement and two opposite-charge sign track segments in the muon spectrometer trigger stations. Each track segment in the trigger stations is required to have a transverse momentum, evaluated online, larger than about 0.5~\GeVc. 
%
Only events passing a selection criterion to remove beam--background collisions contamination, based on the timing information from the V0 arrays, are considered in the analysis. 

When multiple primary vertices are reconstructed by the \SPD, the event is tagged as pile-up and removed from this analysis. 
In order to avoid acceptance biases on the reconstructed \SPD tracklets, events with a displaced vertex with respect to centre of the \SPD detector along the beam direction are discarded according to the requirement $|v_{z}| \leq 10$~cm. 
These selections allowed us to keep the pile-up below $0.3\%$ for the analysed events, also for events with two muon pairs with an invariant mass above 2 \GeVmass. 
Considering the above selections, the total number of dimuon triggered events in the sample sums up to $587.4 \times 10^{6}$ events and corresponds to an integrated luminosity of $24.11 \pm  0.01 (\rm{stat.}) \pm 0.80 (\rm{syst.}) \, \rm{pb}^{-1}$.

