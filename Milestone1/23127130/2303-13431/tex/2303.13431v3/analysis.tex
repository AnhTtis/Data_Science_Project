\section{Analysis}
\label{sec:analysis}

$\Jpsi$ candidates are built from  muon pairs of opposite-charge sign. 
Muons are identified by requiring that selected tracks in the MCH have a matching track segment in MTR. 
Only muon tracks within the detector acceptance are kept for analysis. Tracks are required to be within $-4.0 < \eta^{\mu} < -2.5$, and the radial distance from the beam axis at the end of the front absorber, $R_{\rm abs}$, is limited to $17.6 < R_{\rm abs} < 89.5$~cm~\cite{ALICE:2017leg}. 
$\Jpsi$ pair candidates are reconstructed from all combinations of double dimuon pairs (each dimuon consisting of an opposite-charge sign muon pair) per event.

The production cross section of inclusive $\Jpsi$ pairs is determined as
\begin{equation}
    \sigma ({\rm J}/\psi \,\, {\rm J}/\psi ) 
     = \frac{N}{ \mathcal{L}_{\rm int} \times \epsilon \times B^2( {\rm J}/\psi \rightarrow \mu^+ \mu^-)} \, ,
\end{equation}
where $N$ is the signal estimate, $\epsilon$ is the acceptance-times-efficiency correction, $B(\Jpsi \rightarrow \mu^+ \mu^-)=(5.961 \pm 0.033) \%$ is the branching fraction of $\Jpsi \rightarrow \mu^+ \mu^-$~\cite{Workman:2022ynf}, and $\mathcal{L}_{\rm int} $ is the integrated luminosity. 
%
% Describe also luminosity


%
% Describe fit
The $\Jpsi$ pair signal is evaluated from a fit to the 2-dimensional invariant mass distribution. A two-step procedure was chosen. 
The first step exploits the 1-dimensional distribution of all $\Jpsi$ candidates in the data sample analysed, to obtain a good description of the $\Jpsi$ line shape from data. 
A fit is performed with a superposition of $\Jpsi$ and $\psi(2S)$ signal functions and a background function. 
The $\Jpsi$ mass, width, and normalisation are left free in the procedure. 
Instead, the $\psi(2S)$ mass and width are bound to those of $\Jpsi$ as described in Ref.~\cite{ALICE:2015pgg}. 
The 2-dimensional invariant mass distribution of $\Jpsi$ pair candidates ($m_1(\mu_1^+ \mu_1^-)$, $m_2(\mu_2^+ \mu_2^-)$) is fit in the second step via $F(m_1,m_2)$. 
\begin{eqnarray}
F(m_1,m_2) & = & N \times S_1 (m_1) \times S_2 (m_2) 
        + R_{B_1, S_2} \times B_1(m_1) \times S_2(m_2) \\
      &&  + R_{S_1, B_2} \times S_1(m_1) \times B_2(m_2) 
        + R_{B_1, B_2} \times B_1(m_1) \times B_2(m_2) \,, \nonumber
\end{eqnarray}
where $N$ and $R$ are the corresponding normalisation parameters. 
%
The $\psi(2S)$ contribution is neglected in the 2-dimensional fit. 
The $\Jpsi$ pole mass and width determined from the first step are fixed in the second step of the fit, the rest of the fit parameters are left free. 
Different combinations of functional forms are used to determine the raw yield and its uncertainties. 
The signal $S$ is modelled by a Crystal Ball function including a Gaussian core and two asymmetric power-law tails~\cite{ALICE-PUBLIC-2015-006}. 
The power-law tail parameters are obtained both from data or Monte Carlo and fixed in the fits~\cite{ALICE:2017leg}. 
%
The background $B$ contribution is described by either the sum of two exponentials, an exponential of a second order polynomial, or the ratio of a first-order to a second-order polynomials. 
%
The mass distribution is fit in two different mass intervals to test the results stability, i.e. [2.0, 4.5] and [2.2, 4.9]~GeV/$c^2$.
%
As the candidates were assigned randomly, the fit function is symmetric under exchange of $m_1$ and $m_2$. 
The projections of one of the fits on $m_1(\mu_1^+ \mu_1^-)$ and $m_2(\mu_2^+ \mu_2^-)$ are shown in Fig.~\ref{fig:invmass}. 
%
The $\Jpsi$ pair signal and statistical uncertainty are evaluated as the average of the values obtained in the twelve fit configurations. The systematic uncertainty is given by their standard deviation. 
The raw yield is $N = 59.3 \pm 13.5 \, {\rm (stat)}\, \pm 4.4 \, {\rm (syst)}$. 

\begin{figure}[!htbp]
    \begin{center}
    \includegraphics[width = 0.98\textwidth]{figures/c2DprojMeas.pdf}
    \end{center}
    \caption{Projections of a fit to the 2-dimensional invariant mass distribution for inclusive $\Jpsi$ pair candidates for (a) $m_1(\mu_1^+ \mu_1^-)$ and (b) $m_2(\mu_2^+ \mu_2^-)$. 
    The (black) markers show data. 
    The (black) solid line represents the total fit function. 
    The (blue and green) dashed-dotted lines indicate the background contribution from a combination of a real $\Jpsi$ signal with a combinatorial candidate. 
    The (yellow) dotted line represents muon pairs from combinatorial background. 
    \label{fig:invmass}
    }
\end{figure}


% 
% Describe efficiency
The acceptance, reconstruction and selection efficiency is evaluated assuming factorisation of the corrections of the $\Jpsi$ pair as 
\begin{equation}
    \epsilon  ({\rm J}/\psi \,\, {\rm J}/\psi )  = 
 \epsilon  ({\rm J}/\psi)  
    \times
 \epsilon  ({\rm J}/\psi)  \,.
\end{equation}
The $\Jpsi$ $\epsilon$,  $\epsilon  ({\rm J}/\psi)$, is computed from Monte Carlo simulations as described in Ref.~\cite{ALICE:2017leg}. 
An iterative procedure is used to generate input rapidity ($y$) and transverse momentum ($\pt$) distributions from data. 
The $\Jpsi$ are decayed into pairs of muons using EVTGEN~\cite{Lange:2001uf} and PHOTOS~\cite{Barberio:1993qi}. 
A GEANT3~\cite{Brun:1994aa} simulation is performed to transport the decay muons through the apparatus including a realistic description of the detector conditions during data taking. 
The validity of the factorisation approach for the efficiency calculation was tested. The invariant mass distribution was compared with the corresponding one after applying a 2-dimensional ($y
$, $\pt$) acceptance-times-efficiency correction per  $\Jpsi$ candidate. The shapes of the 2-dimensional invariant mass distribution, and their projections are not modified by the correction, confirming the validity of our assumption. 
In addition, a toy Monte Carlo was developed to study the possible influence of angular correlations between the two $\Jpsi$ of the pair. Two  $\Jpsi$ were simulated per event, according to a ($y$, $\pt$) distribution extracted from single  $\Jpsi$ measurements. To mimic possible correlations among the  $\Jpsi$, their rapidity difference was forced to follow
 either a triangular or a flat distribution. The average pair efficiency was computed for both cases. The resultant pair efficiency was found to be in agreement with the calculation from the factorisation approach for both cases.


Various sources of systematic uncertainties on the $\Jpsi$ pair production cross section are considered: 
\begin{enumerate*}[label=(\roman*)]
    \item the signal extraction, 
    \item the branching fraction uncertainty, 
    \item the luminosity normalisation, and
    \item the acceptance-times-efficiency correction. 
\end{enumerate*}
%
% yield
Details on the signal extraction procedure were given previously in this Letter. 
The systematic uncertainty on the signal extraction, obtained as described above, amounts to 7.4\%. 
%
% BR
The branching fraction uncertainty is 0.6\% for single $\Jpsi$~\cite{Workman:2022ynf}, thus 1.1\% for $\Jpsi$ pairs.
%
% Luminosity
The influence of the luminosity normalisation factor is evaluated by computing the equivalent number of minimum-bias events in the analysed dimuon sample with different methods as described in Ref.~\cite{ALICE:2022gpu}, which amounts to 2.9\%. 
The uncertainty on the minimum bias cross section, evaluated in a van der Meer scan (1.6\%), is also taken into account in the calculation~\cite{ALICE-PUBLIC-2021-005}. 
These two sources lead to a 3.3\% systematic uncertainty for the luminosity. 
%
% A x E
The systematic uncertainty on the acceptance-times-efficiency correction contains contributions from
\begin{enumerate*}[label=(\roman*)]
    \item the input $\pt$ and $y$ distributions, 
    \item the tracking efficiency in the MCH, 
    \item the MTR trigger efficiency, and
    \item the matching of the reconstructed tracks in the MCH with the track segments in the MTR.
\end{enumerate*}
%
The influence of the simulated $\Jpsi$ $\pt$ and $y$ distributions is tested by comparing the corrected yield obtained via the iterative procedure, with the one obtained from an efficiency-corrected invariant mass distribution.  
For this exercise, a 2-dimensional $\epsilon(\pt,y)$ correction is applied to each $\Jpsi$ candidate in order to build the efficiency-corrected invariant mass distribution, which was then fit to obtain the corresponding corrected yield. A 0.5\% uncertainty is assigned to the MC input for $\Jpsi$~\cite{ALICE:2017leg}. 
% 
The systematic uncertainties on the tracking efficiency in the MCH, the MTR trigger efficiency and the matching between the MCH and MTR are evaluated comparing single muon data and MC, as described in Ref.~\cite{ALICE:2014uja}. The differences are then propagated to the dimuon case, being 4\%, 2\% and 1\% respectively for the $\Jpsi$~\cite{ALICE:2017leg}. 
This results in a 4.6\% acceptance-times-efficiency uncertainty for $\Jpsi$, and is propagated to a 9.2\% uncertainty for $\Jpsi$ pairs. 
The analysis requirement that all selected tracks in the MCH should match track segments in the MTR removes any possible dependence on which pair of tracks activated the trigger. 
%
Table~\ref{tab:unc} summarises the systematic uncertainties on the measurement of the $\Jpsi$ pair production cross section.

\begin{table}[!htbp]
    \centering
    \caption{Sources of systematic uncertainty on the $\Jpsi$ pair production cross section measurement.
    \label{tab:unc}
    }
    \begin{tabular}{l|c}
    Source                                      &  Uncertainty (\%) \\ \hline
    Signal extraction                            &  7.4  \\
    Acceptance-times-efficiency                 &  9.2  \\
    $B( {\rm J}/\psi \rightarrow \mu^+ \mu^-)$  &  1.1  \\ 
    Luminosity                                  &  3.3  \\ \hline
    Total                                       &  12.3
    \end{tabular}
\end{table}