\begin{table*}[tb]
\centering
\setlength{\tabcolsep}{3pt}
\begin{adjustbox}{max width=\textwidth}
\begin{tabular}{lrrrrr}
\toprule
\multirow{2}{*}{Model}  & \multirow{2}{*}{Params} & \multirow{2}{*}{MRR$\uparrow$} & \multicolumn{3}{c}{Hits$\uparrow$} \\\cmidrule(r{4pt}){4-6}
      & & & @1 & @3 & @10 \\\midrule
TransE~\cite{bordes2013transe}   & 2.4T & .253 & .170 & .311 & .392 \\
DistMult~\cite{yang2014distmult}  & 2.4T & .253 & .208 & .278 & .334 \\
ComplEx~\cite{trouillon2016complexe}  &2.4T & .281 & .228 & .310 & .373 \\
SimpIE~\cite{kazemi2018simple}  &2.4T & .296 & .252 & .317 & .377 \\
RotatE~\cite{sun2018rotate}   &2.4T & .290 & .234 & .322 & .390 \\ \midrule
KEPLER~\cite{wang2021KEPLER}   & 110M & .210 & .173 & .224 & .277\\ 
MLMLM~\cite{clouatre-etal-2021-mlmlm}   & 355M & .223 & .201 & .232 & .264\\ 
KGT5~\cite{saxena-etal-2022-sequence}   & 60M & .300 & .267 & .318 & .365\\ 
KGT5-ComplEx Ensemble~\cite{saxena-etal-2022-sequence}   & 674M & .336 & .286 & .362 & .426\\ 

\hitctx (Ours)    & 110M & \textbf{.377} & \textbf{.332} & \textbf{.398} & \textbf{.456} \\
\bottomrule
\end{tabular}
\end{adjustbox}
\caption{Comparison between the proposed method and baseline methods on Wikidata5M under the transductive setting. Results of transductive KGE methods (first section) are taken from \citet{wang2021KEPLER}. Numbers in \textbf{bold} represent the best results.}
\label{tab:transductive_results}
\end{table*}