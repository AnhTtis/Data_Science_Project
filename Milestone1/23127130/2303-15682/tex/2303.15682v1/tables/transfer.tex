\begin{table*}[]
\centering
\begin{adjustbox}{max width=\textwidth}
\begin{tabular}{lcccccccc}
\toprule
 & \multicolumn{4}{c}{Transductive} & \multicolumn{4}{c}{Inductive} \\\cmidrule(r{3pt}){2-5} \cmidrule(){6-9}
 & \multicolumn{2}{c}{FB15K-237} & \multicolumn{2}{c}{WN18RR} & \multicolumn{2}{c}{FB15K-237} & \multicolumn{2}{c}{WN18RR} \\\cmidrule(r{3pt}){2-3} \cmidrule(){4-5} \cmidrule(l{3pt}r{3pt}){6-7} \cmidrule(){8-9}
Model  &      MRR         &     H@10         &    MRR  &    H@10  &      MRR         &     H@10         &    MRR  &    H@10       \\\midrule
 HittER~\cite{chen-etal-2021-hitter} & 37.3 & 55.8 & 50.3 & 58.4 & - & - & - & - \\
 BLP-TransE~\cite{daza2021inductive} & - & - & - & - & 19.5 & 36.3 & 28.5 & 58.0 \\\midrule
\hitent & 34.5 & 53.0 & 44.8 & 67.5 & 23.0 & 39.1 & 30.9 & 51.4 \\
\hitent-WD     & 37.6 & 56.2 & 52.1 & 74.0 & 27.5 & 44.7 & 39.5 & 60.5 \\\midrule
\hitctx   &  34.4 & 52.8 & 50.0 & 71.0 & - & - & - & - \\
\hitctx-WD     & 37.4  & 56.3 & 55.0 & 74.9 & - & - & - & - \\
                        \bottomrule
\end{tabular}
\end{adjustbox}
\caption{Our test results on two regular-scale datasets for transductive and inductive KG completion. The first part of the table presents the results of the state-of-the-art models from previous work, while the second/third parts show the results of our \hit models using different initialization methods. WD stands for Wikidata5M pretraining.}
\label{tab:transfer}
\end{table*}