% ****** Start of file apssamp.tex ******
%
%   This file is part of the APS files in the REVTeX 4.2 distribution.
%   Version 4.2a of REVTeX, December 2014
%
%   Copyright (c) 2014 The American Physical Society.
%
%   See the REVTeX 4 README file for restrictions and more information.
%
% TeX'ing this file requires that you have AMS-LaTeX 2.0 installed
% as well as the rest of the prerequisites for REVTeX 4.2
%
% See the REVTeX 4 README file
% It also requires running BibTeX. The commands are as follows:
%
%  1)  latex apssamp.tex
%  2)  bibtex apssamp
%  3)  latex apssamp.tex
%  4)  latex apssamp.tex
%
\documentclass[%
 reprint,
%superscriptaddress,
%groupedaddress,
%unsortedaddress,
%runinaddress,
%frontmatterverbose, 
%preprint,
%preprintnumbers,
%nofootinbib,
%nobibnotes,
%bibnotes,
 amsmath,amssymb,
 aps,
%pra,
%prb,
%rmp,
%prstab,
%prstper,
%floatfix,
]{revtex4-2}

\usepackage{amsmath,amsxtra,amssymb,amsthm,amsfonts}
%etoolbox,setspace}
%showkeys}
\usepackage{mathrsfs}
\usepackage[T1]{fontenc}
\usepackage[utf8]{inputenc}
\usepackage{mathtools}   % loads »amsmath«
\usepackage{multirow}
\usepackage{tikz,quantikz}
\usepackage{multirow}
\usepackage{bbm}
\usepackage{subfigure}
\usepackage{comment}
\usetikzlibrary{arrows, automata}
\usepackage{color,hyperref}
\definecolor{dblue}{rgb}{0, 0, 0.72}
%\usepackage{amsmath,amsxtra,amssymb,showkeys}
\def\theequation{\thesection.\arabic{equation}}

\allowdisplaybreaks

%\usepackage[active]{srcltx} % SRC Specials: DVI [Inverse] Search
% ----------------------------------------------------------------
\vfuzz2pt % Don't report over-full v-boxes if over-edge is small
\hfuzz2pt % Don't report over-full h-boxes if over-edge is small
% THEOREMS -------------------------------------------------------
\numberwithin{equation}{section}
\newtheorem{lemma}{Lemma}[section]
\newtheorem{prop}[lemma]{Proposition}
\newtheorem{theorem}[lemma]{Theorem}
\newtheorem{ttheorem}{Theorem}
\newtheorem{cor}[lemma]{Corollary}
\newtheorem{res}[lemma]{Resum\'{e}:}
\newtheorem{conj}[lemma]{Conjecture}
\newtheorem{proposition}{Proposition}
\newtheorem{quest}[lemma]{Question}
\newtheorem{prob}[lemma]{Problem}
\newtheorem{rem}[lemma]{Remark}
\newtheorem{remark}[lemma]{Remark}
\newtheorem{frem}[lemma]{Final remark}
\newtheorem{fact}[lemma]{Fact}
\newtheorem{conv}[lemma]{Convention}
\newtheorem{conc}[lemma]{Conclusion}
\newtheorem{obs}[lemma]{Observation}
\newtheorem{war}[lemma]{Warning}
\newtheorem{hypo}[lemma]{Hypothesis}
\newtheorem{example}[lemma]{Example}
\newtheorem{definition}[lemma]{Definition}
\newtheorem{corollary}[lemma]{Corollary}
\newtheorem{axiom}{Axioms}

\newenvironment{claim}[1]{\par\noindent\underline{Claim:}\space#1}{}
\newenvironment{claimproof}[1]{\par\noindent\underline{Proof:}\space#1}{\hfill $\blacksquare$}


\newcommand{\re}{\begin{rem}\rm}
	\newcommand{\mar}{\end{rem}}
\newtheorem{exam}[lemma]{Example}
\newtheorem{exams}[lemma]{Examples}
\newtheorem{defi}[lemma]{Definition}
\newtheorem{ddefi}{Definition}

\newtheorem{defir}[lemma]{Definition and Remark}

\newcommand{\br }{|}
\newcommand{\kla}{\left ( }
\newcommand{\mer}{\right ) }

\newcommand{\ee }{\mathrm{I}\!\!1}

\newcommand{\ketbra}[1]{|{#1}\rangle\langle{#1}|}


\newcommand{\und}{\quad\mbox{and}\quad}
\newcommand{\mitt}{\left | { \atop } \right.}
\newcommand{\kl}{\pl \le \pl}
\newcommand{\gl}{\pl \ge \pl}
\newcommand{\kll}{\p \le \p}
\newcommand{\gll}{\p \ge \p}
\newcommand{\nach}{\rightarrow}
\newcommand{\lel}{\pl = \pl}
\newcommand{\lell}{\p=\p}
\newcommand{\wt}{\widetilde}

\newcommand{\ew}{{\mathbb E}}
\newcommand{\ez}{{\mathbb E}}

\newcommand{\BB}{{\mathbb B}}
\renewcommand{\L}{\mathcal{L}}
\newcommand{\eiz}{{\rm 1}\!\!{\rm 1}}

\newcommand{\nz}{{\mathbb N}}
\newcommand{\nen}{n \in \nz}
\newcommand{\ken}{k \in \nz}
\newcommand{\rz}{{\mathbb R}}
\newcommand{\zz}{{\mathbb Z}}
\newcommand{\qz}{{\mathbb Q}}
\newcommand{\Mz}{{\mathbb M}}
\newcommand{\rn}{\rz^n}
\newcommand{\cz}{{\mathbb C}}
\newcommand{\kz}{{\rm  I\! K}}
\newcommand{\pee}{{\rm I\! P}}
\newcommand{\dop}{{\mathbb D}}
\newcommand{\dt}{{\rm det}}
\newcommand{\ob}{\bar{\om}}
\newcommand{\ten}{\otimes}
\newcommand{\sz}{\scriptstyle}
\newcommand{\wet}{\stackrel{\wedge}{\otimes}}

\newcommand{\gga}{{\rm I}\!\Gamma}

\newcommand{\dd}{\mathbb{D}}

\DeclareMathOperator{\dom}{dom}

\DeclareMathOperator{\Img}{Im}

\DeclareMathOperator{\Rep}{Re}

\DeclareMathOperator{\rc}{rc}

\DeclareMathOperator{\Exp}{Exp}

\DeclareMathOperator{\gr}{gr}

\DeclareMathOperator{\rg}{rg}
\DeclareMathOperator{\vol}{vol}

\DeclareMathOperator{\Hs}{H}
\DeclareMathOperator{\Vs}{V}
\DeclareMathOperator{\Ks}{K}

\DeclareMathOperator{\diam}{diam}

\DeclareMathOperator{\Proj}{Pr}

\DeclareMathOperator{\sgn}{sgn}


\DeclareMathOperator{\err}{err}

\DeclareMathOperator{\tg}{tg}

\DeclareMathOperator{\spa}{span}

\DeclareMathOperator{\Sh}{Shift}

\DeclareMathOperator{\Pe}{Perm}

\DeclareMathOperator{\Hessian}{Hessian}

\DeclareMathOperator{\Ric}{Ric}
\DeclareMathOperator{\Ker}{Ker}


\DeclareMathOperator{\Tan}{Tan}

\DeclareMathOperator{\grad}{grad}

\DeclareMathOperator{\fix}{fix}

\DeclareMathOperator{\Tr}{Tr}
\DeclareMathOperator{\tr}{tr}

\DeclareMathOperator{\Ent}{Ent}

\DeclareMathOperator{\spread}{spread}

\DeclareMathOperator{\spann}{span}

\DeclareMathOperator{\grd}{grad}

\DeclareMathOperator{\grr}{gr}

\DeclareMathOperator{\Res}{Res_{\small{W}}}

\DeclareMathOperator{\cl}{cl}

\DeclareMathOperator{\Ricci}{Ricci}

\DeclareMathOperator{\krn}{ker}

\DeclareMathOperator{\CLSI}{CLSI}
\DeclareMathOperator{\MLSI}{MLSI}
\DeclareMathOperator{\MpSI}{MpSI}
\DeclareMathOperator{\CpSI}{C_pSI}
\DeclareMathOperator{\CpqSI}{C_{pq}SI}

\DeclareMathOperator{\Lin}{\text{\fontsize{2.5}{4}\selectfont {Lin}}}
\newcommand{\suppl}{{\sup}^+}

\newcommand{\p}{\hspace{.05cm}}
\newcommand{\pll}{\hspace{.3cm}}
\newcommand{\pla}{\hspace{1.5cm}}
\newcommand{\hz}{\vspace{0.5cm}}
\newcommand{\hoch}{ \vspace{-0.86cm}}
\newcommand{\hhz}{\vspace{0.3cm}}
%\newcommand{\qed}{\hspace*{\fill}$\Box$\hz\pagebreak[1]}
%\renewcommand{\qed}{\hspace*{\fill} {\vrule height7pt width7pt
%depth0pt} \hz\pagebreak[1]}
%\renewcommand{\qed}{\hspace*{\fill}$\Box$\hz\pagebreak[1]}
\newcommand{\ceil}[1]{\lceil {#1} \rceil}

%\newcommand{\qed}{\hspace*{\fill}$\Box$\hz\pagebreak[1]}

\newcommand{\qd}{\end{proof}\vspace{0.5ex}}

\renewcommand{\qedsymbol}{\vrule height7pt width7pt depth0pt \hz}

\newcommand{\Om}{\Omega}
\newcommand{\om}{\omega}
\renewcommand{\a}{\alpha}
\newcommand{\al}{\alpha}
\newcommand{\de}{\delta}
\newcommand{\si}{\sigma}
\newcommand{\J}{{\mathcal J}}
%\newcommand{\H}{{\mathcal H}}
\newcommand{\Si}{\Sigma}
\newcommand{\tet}{\theta}
\newcommand{\ttett}{\vartheta}
\newcommand{\fid}{\hat{\Phi}}
\newcommand{\La}{\Lambda}
\newcommand{\la}{\lambda}
\newcommand{\eps}{\varepsilon}
\newcommand{\vare}{\varepsilon}
\newcommand{\ds}{D_{\si}}
\newcommand{\lzn}{\ell_2^n}
\newcommand{\len}{\ell_1^n}
\newcommand{\lin}{\ell_{\infty}^n}
\newcommand{\lif}{\ell_{\infty}}
\newcommand{\lih}{\ell_{\infty,\infty,1/2}}
\newcommand{\id}{\iota_{\infty,2}^n}
\newcommand{\zp}{\stackrel{\circ}{Z}}
\newcommand{\ce}{{\tt c}_o}
\newcommand{\leb}{{\mathcal L}}
\newcommand{\pot}{{\mathcal P}(}
\newcommand{\es}{{\mathcal S}}
\newcommand{\ha}{{\mathcal H}}
\newcommand{\hp}{\frac{1}{p}}
%{{\rm c}$_0$ }
%\renewcommand{\L}{{\mathcal L}}
\newcommand{\F}{{\mathcal F}}
\newcommand{\E}{{\mathcal E}}
\newcommand{\A}{{\mathcal A}}
%\newcommand{\H}{{\mathcal H}}

\newcommand{\D}{{\mathcal D}}
\newcommand{\C}{{\mathcal C}}
\newcommand{\K}{{\mathcal K}}
\newcommand{\R}{{\mathcal R}}
\newcommand{\Ha}{{\mathcal H}}
\renewcommand{\S}{{\mathcal S}}
\newcommand{\SH}{{\mathcal S}{\mathcal H}}

\newcommand{\Ma}{{\mathbb M}}

\newcommand{\MM}{{\mathcal M}}
\newcommand{\m}{{\bf \sl m}}
\newcommand{\OL}{{\mathcal OL}}
\newcommand{\COL}{{\mathcal COL}}
\newcommand{\COS}{{\mathcal COS}}
\newcommand{\Uu}{{\mathbb U}}
\newcommand{\Vv}{{\mathbb   V}}
\newcommand{\Pp}{{\mathbb   P}}
\newcommand{\Pd}{{\mathbb I}\!{\mathbb   P}}
\newcommand{\LSI}{\operatorname{LSI}}
\newcommand{\mm}{{\mathbb M}}
\newcommand{\amo}{ \ll \!\! \scriptsize{a}}
\newcommand{\ttau}{\si}
\newcommand{\Kk}{{\rm  K}}
\newcommand{\Ww}{{\mathbb W}}

\newcommand{\Gb}{{\rm I}\!\Gamma}
\renewcommand{\P}{\mathcal P}
\newcommand{\PP}{{\rm I\! P}}
\newcommand{\N}{{\mathcal N}}
\newcommand{\G}{\Gamma}
\newcommand{\g}{\gamma}
\newcommand{\T}{\mathcal T}
\newcommand{\I}{{\mathcal I}}
\newcommand{\U}{{\mathcal U}}
\newcommand{\B}{{\mathcal B}}
\newcommand{\Hh}{{\mathcal H}}

\newcommand{\pf}{\begin{proof}}
\newcommand{\AB}{{\mathbb A}}


%\newcommand{\pxq}{\pi_{X,q}}
%\newcommand{\pxe}{\pi_{X,1}}
%\newcommand{\pxz}{\pi_{X,2}}

\newcommand{\noo}{\left \|}
\newcommand{\rrm}{\right \|}



\newcommand{\bet}{\left |}
\newcommand{\rag}{\right |}
\newcommand{\be}{\left|{\atop}}
\newcommand{\ra}{{\atop}\right|}


\newcommand{\intt}{\int\limits}
\newcommand{\summ}{\sum\limits}
\newcommand{\limm}{\lim\limits}
\newcommand{\ver}{\bigcup\limits}
\newcommand{\durch}{\bigcap\limits}
\newcommand{\prodd}{\prod\nolimits}
\newcommand{\su}{\subst}

\newcommand{\xspace}{\hbox{\kern-2.5pt}}
\newcommand{\xyspace}{\hbox{\kern-1.1pt}}

\newcommand\lge{\langle}
\newcommand\rge{\rangle}

\newcommand\ltriple{ltriple}



%\newcommand\DD[1]{\mathbb{D}( #1)}
\newcommand\DD[1]{\mathbb{D}#1}
\renewcommand\dd{\p\mathbb{D}}
%\newcommand\[1]{\mathbb{d}(#1)}

\newcommand\tnorm[1]{\left\vert\xspace\left\vert\xspace\left\vert\mskip2mu
	#1\mskip2mu \right\vert\xspace\right\vert\xspace\right\vert}

\newcommand\ttnorm[1]{\left\vert\xspace\left\vert\xspace\left\vert\mskip2mu
	#1\mskip2mu \right\vert\xspace\right\vert\xspace\right\vert}

\newcommand{\ch}{\stackrel{\circ{}}{H}{\atop}\!\!\!}

\newcommand{\lb}{\langle \langle }
\newcommand{\rb}{\rangle \rangle }

\newcommand{\TT}{{\mathbb T}}
\newcommand{\Hm}{{\mathcal H}}

\newcommand{\X}{\mathcal{X}}
\newcommand{\Y}{\mathcal{Y}}

\newcommand{\supn}{{\sup_n}^+}
\newcommand{\supt}{{\sup_t}^+}

\newcommand{\sint}{\pl {\si\!\!\int}}

\newcommand{\kk}{<}
%\newcommand{\gr}{>}
\newcommand{\st}{|}

\newcommand{\pr}{{\mathrm P\mathrm r}}

%\newcommand{\Gammab}{{\rm I}\!\Gamma}

\newcommand{\8}{\infty}

\newcommand{\ff}{{\mathbb  F}}

\newcommand{\norm}[2]{\parallel \! #1 \! \parallel_{#2}}


\newcommand{\ssubset} {\!\!\subset\! \!}
%\topmargin -1 cm

%\renewcommand{\baselinestretch}{1.05 }

\hyphenation{comm-ut-ta-ti-ve}

%\mathchardef\mhyphen
%\topmargin -1cm
%\oddsidemargin 0.3 cm

%\evensidemargin 0.3 cm

\allowdisplaybreaks

\textwidth 16 cm \textheight 22.7cm
%\parindent0em
%\include{addti}
\newcommand{\ttt}{{\bf t}}

\definecolor{LightGray}{rgb}{0.94,0.94,0.94}
\definecolor{VeryLightBlue}{rgb}{0.9,0.9,1}
\definecolor{LightBlue}{rgb}{0.8,0.8,1}
\definecolor{DarkBlue}{rgb}{0,0,0.6}
\definecolor{LightGreen}{rgb}{0.88,1,0.88}
\definecolor{MidGreen}{rgb}{0.6,1,0.6}
\definecolor{DarkGreen}{rgb}{0,0.6,0}
\definecolor{DarkGrreen}{rgb}{0,0.8,0}

\definecolor{VeryLightYellow}{rgb}{1,1,0.9}
\definecolor{LightYellow}{rgb}{1,1,0.6}
\definecolor{MidYellow}{rgb}{1,1,0.5}
\definecolor{DarkYellow}{rgb}{0.8,1,0.3}
\definecolor{VeryLightRed}{rgb}{1,0.9,0.9}
\definecolor{LightRed}{rgb}{1,0.8,0.8}
\definecolor{DarkRed}{rgb}{0.8,0.2,0}
\definecolor{DarkRedb}{rgb}{0.6,0.2,0}
\definecolor{DarkLila}{rgb}{0.8,0,1}
\definecolor{Beige}{rgb}{0.96,0.96,0.86}
\definecolor{Gold}{rgb}{1.,0.84,0.}
\definecolor{Goldb}{rgb}{0.7,0.3,0.5}
\definecolor{MyYellow}{rgb}{1.,0.84,0.8}


\newcommand{\VeryLightBlue}[1]{{\color{VeryLightBlue}{#1}}}
\newcommand{\LightBlue}[1]{{\color{LightBlue}{#1}}}
\newcommand{\Blue}[1]{{\color{blue}{#1}}}
\newcommand{\DarkBlue}[1]{{\color{DarkBlue}{#1}}}
\newcommand{\DarkGreen}[1]{{\color{DarkGreen}{#1}}}
\newcommand{\LightGreen}[1]{{\color{LightGreen}{#1}}}

\newcommand{\DarkGrreen}[1]{{\color{DarkGrreen}{#1}}}
\newcommand{\DarkRedb}[1]{{\color{DarkRedb}{#1}}}
\newcommand{\red}[1]{{\color{DarkRedb}{#1}}}


\newcommand{\VeryLightRed}[1]{{\color{VeryLightRed}{#1}}}
\newcommand{\LightRed}[1]{{\color{LightRed}{#1}}}
\newcommand{\DarkYellow}[1]{{\color{DarkYellow}{#1}}}
\newcommand{\LightYellow}[1]{{\color{LightYellow}{#1}}}

\newcommand{\DarkRed}[1]{{\color{DarkRed}{#1}}}
\newcommand{\DarkLila}[1]{{\color{DarkLila}{#1}}}
\newcommand{\Beige}[1]{{\color{Beige}{#1}}}
\newcommand{\Gold}[1]{{\color{Gold}{#1}}}
\newcommand{\Goldb}[1]{{\color{Goldb}{#1}}}

\newcommand{\MyYellow}[1]{{\color{MyYellow}{#1}}}


\newcommand{\Red}[1]{{\color{red}{#1}}}
\newcommand{\Gray}[1]{{\color{gray}{#1}}}
\newcommand{\Black}[1]{{\color{black}{#1}}}

\newcommand{\lan}{\langle}
\newcommand{\ran}{\rangle}
\newcommand{\lgae}{$\la$-$\Gamma \E$}
\newcommand{\lgg}{\mathfrak{g}}
\newcommand{\lgc}{\mathfrak{g}_{\cz}}

\def\ch{{\cal H}}
\def\cm{{\cal M}}
\def\II{\mathbb I}
\def\cb{{\cal B}}
\def\cn{{\cal N}}
\def\ce{{\cal E}}
\def\cf{{\cal F}}
\def\cg{{\cal G}}
\def\ca{{\cal A}}
\def\cv{{\cal V}}
\def\cc{{\cal C}}
\def\cp{{\cal P}}
\def\calr{{\cal R}}
\def\mod{\,\, {\rm mod}\,\,}
\def\ma{\mathbf{a}}
\def\mx{\mathbf{x}}
\def\mapr{\mathbf{a'}}
\def\dom{\operatorname{dom}}

\def\CC{\mathbb{C}}
\def\RR{\mathbb{R}}
\def\ZZ{\mathbb{Z}}
\def\QQ{\mathbb{Q}}
\def\NN{\mathbb{N}}
\def\Mm{\mathcal{M}}

\def\LL{\mathbb{L}}
\def\nbits{\mathbb{Z}_2^n}
\def\11{\mathbb{I}}
\def\LL{\mathcal{L}}
\def\PP{\mathbb{P}}
\def\EE{\mathbb{E}}
\def\RRR{\mathcal{R}}

\def\h{\mathfrak h}
\def\ind{1\hspace{-0.27em}\mathrm{l}}

\newcommand{\half}{\mbox{$\textstyle \frac{1}{2}$} }


\newcommand{\inner}[2]{ \langle #1 | #2 \rangle}
\newcommand{\melement}[2]{ \langle #1 | #2 | #1 \rangle}

\newcommand{\vc}[1]{\underline{#1}}

\newcommand{\mmatrix}[4]{\left( \begin{array}{cc} #1 & #2 \\ #3 &
		#4 \end{array} \right)}
\newcommand{\vvec}[2]{\left( \begin{array}{cc} #1  \\ #2 \end{array} \right)}
\newcommand{\sign}[1]{{\rm sign}(#1)}

\newcommand{\acc}{{I_{acc}}}
\renewcommand{\prob}{{\rm prob}}
\newcommand{\up}{{\underline{p}}}
\newcommand{\ut}{{\underline{t}}}
\newcommand{\bP}{{\overline{P}}}

\newcommand{\bfi}{{\textbf{i}}}
\newcommand{\bfj}{{\textbf{j}}}
\newcommand{\bfk}{{\textbf{k}}}
\newcommand{\bfx}{{\textbf{x}}}

%\usepackage{pst-node}
\usepackage{tikz-cd}
\newcommand{\EP}{\operatorname{EP}}
\newcommand{\Op}{\operatorname{Op}}
\newcommand{\Ri}{\operatorname{Ricci}}
\newcommand{\Av}{\operatorname{Ricci}}

\newcommand{\CLSII}{\CLSI}
\newcommand{\mb}{\mathbb}
\newcommand{\mc}{\mathcal}
\usepackage{graphicx}
%\usepackage{setspace}
%\usepackage{verbatim}
%\usepackage{subfig}

%\newcommand{\EP}{\operatorname{EP}}
%\newcommand{\GNS}{\operatorname{GNS}}

\DeclareRobustCommand\openone{\leavevmode\hbox{\small1\normalsize\kern-.33em1}}
\newcommand{\identity}{I}
\renewcommand{\id}{\rm{id}}
\renewcommand{\be}{\begin{equation}}
	\renewcommand{\ee}{\end{equation}}
\newcommand{\bea}{\begin{eqnarray}}
	\newcommand{\eea}{\end{eqnarray}}
\newcommand{\beas}{\begin{eqnarray*}}
	\newcommand{\eeas}{\end{eqnarray*}}
\def\cDH{\cD(\cH)}
\def\cPH{\cP(\cH)}

%\setcounter{Maxaffil}{1}

%%%%%%%%%%%%%%%%%%%%%%%%%
%\apptocmd{\thebibliography}{\fontsize{8}{13}\selectfont}{}{}%
%\onehalfspace
\usepackage{amsmath}
\usepackage{ams math}
\usepackage{amssymb}
\usepackage{amsthm}
\usepackage{dsfont}
\usepackage{upgreek}
%\usepackage{scalerel}
\usepackage{enumitem}

\usepackage{array}
\usepackage{ams fonts}
%\usepackage{tikz-cd}
\usepackage{graphics}
%\DeclareMathOperator{\Hessian}{Hess}
%\usepackage{eps fig}
%\usepackage{tikz}
%\usepackage{pdfpages}
\usepackage[utf8]{inputenc}
%\usepackage[english]{babel}
%\usepackage{comment}
%\newtheorem{theorem}{Theorem}[section]
\newtheorem*{theorem*}{Theorem}
\newtheorem*{remark*}{Remark}
\newtheorem*{lemma*}{Lemma}
\newtheorem*{notation*}{Notation}
\newtheorem*{cor*}{Corollary}
\newtheorem*{note*}{Note}
\newtheorem*{prop*}{Proposition}
\newtheorem*{example*}{Example}
%\newtheorem{lemma}[theorem]{Lemma}
%\newtheorem{definition}[theorem]{Definition}
%\newtheorem{remark}[theorem]{Remark}
%\newtheorem{prop}[theorem]{Proposition}
%\newtheorem{example}[theorem]{Example}
%\newtheorem{corollary}[theorem]{Corollary}
\newcommand{\RM}{\mathbb{R}}
\renewcommand{\Ric}{\mbox{Ric}}

\newcommand{\Rc}{\mbox{Rc}}
\newcommand{\Rcu}{\ensuremath{\Rc_{\text{\fontsize{2.5}{4}\selectfont {U}}}}}
\newcommand{\Du}{\ensuremath{\Delta_{\text{\fontsize{2.5}{4}\selectfont {U}}}}}
\newcommand{\Dv}{\ensuremath{\Delta_{\text{\fontsize{2.5}{4}\selectfont {V}}}}}
\newcommand{\du}{\ensuremath{d_{\text{\fontsize{2.5}{4}\selectfont {U}}}}}
\newcommand{\Lu}{\ensuremath{L_{\text{\fontsize{2.5}{4}\selectfont {U}}}}}
\newcommand{\Zu}{\ensuremath{Z_{\text{\fontsize{2.5}{4}\selectfont {U}}}}}
\newcommand{\Zv}{\ensuremath{Z_{\text{\fontsize{2.5}{4}\selectfont {V}}}}}
\newcommand{\Hess}{\mbox{Hess}}
\renewcommand{\R}{\mbox{R}}
\newcommand{\DN}{\ensuremath{D_{\mathcal{N}_{\fix}} }}
\renewcommand{\T}{\mbox{T}}
\newcommand{\M}{\ensuremath{\mathds{M}}}
\newcommand{\ad}{\mbox{ad}}
\newcommand{\op}{\mbox{\tiny{op}}}
\newcommand{\trans}{\mbox{\scalebox{.5}{trans}}}
\newcommand{\End}{\mbox{End}}
\renewcommand{\dom}{\mbox{dom}}
\newcommand{\Div}{\mbox{div}}
\newcommand{\Spin}{\mbox{\tiny{Spin}}}
\newcommand{\Spinn}{\mbox{Spin}}
\newcommand{\triple}{\ensuremath{(\mathcal{N}\ssubset \mathcal{M},\tau,\delta)}}
\newcommand{\tripleo}{\ensuremath{(\mathcal{N}_1\ssubset \mathcal{M}_1,\tau_1,\delta_1)}}
\newcommand{\triplet}{\ensuremath{(\mathcal{N}_2\ssubset \mathcal{M}_2,\tau_2,\delta_2)}}
\renewcommand{\ltriple}{\ensuremath{(\tilde{\mathcal{N}}\ssubset \tilde{\mathcal{M}},\tau,\delta)}}
\newcommand{\triplei}{\ensuremath{(M\subset \hat{M},\tau_{1},\delta)}}
\newcommand{\tripleii}{\ensuremath{(M\subset \hat{M},\tau_{2},\delta)}}
\newcommand{\QED}{\hfill\ensuremath{\blacksquare}}%
\newcommand{\Rcb}{\ensuremath{Rc_{\text{\tiny{B}}}}}
\newcommand{\bott}{\ensuremath{^{\text{\tiny{B}}}\nabla}}



\DeclareMathOperator*{\esssup}{ess\,sup}
\DeclareMathOperator*{\essinf}{ess\,inf}

\DeclareMathOperator{\Ep}{Ep}

%\usepackage{geometry}
%\usepackage{listings}
%\usepackage{dsfont}
\usepackage{tikz-cd}
\usepackage{amsthm}

%\usepackage[backend=biber,style=alphabetic,sorting=ynt]{biblatex}
%%%%%%%%%%%%%%%%

\usepackage[margin=1.0 in]{geometry}


\begin{document}
\title{Resource-Dependent Complexity of Quantum Channels}
	\author{Roy Araiza$^{\ddag}$, Yidong Chen$^{\$}$, Marius Junge$^{\ddag}$, Peixue Wu$^{\dag}$}
\affiliation{$^{\ddag}$Department of Mathematics, IQUIST,$^{\$}$Department of Physics, $^{\dag}$Department of Mathematics ;University of Illinois at Urbana Champaign}
 
	\email{raraiza@illinois.edu}
 \email{mjunge@illinois.edu}
	\email{yidongc2@illinois.edu}
	\email{peixuew2@illinois.edu}
	
	\thanks{RA was supported as a JL. Doob research assistant professor}
	
	\thanks{MJ was partially supported by NSF Grant DMS 2247114, NSF Grant DMS 1800872 and NSF RAISE-TAQS 1839177}

\begin{abstract}
We introduce a notion of complexity for quantum channels, depending on a suitably chosen resource set. This new class of convex functions on quantum channels is suitable to study the complexity of both open and closed quantum systems. Crucial properties of this class of complexities are derived from Lipschitz norms motivated by noncommutative geometry. We prove linear growth of our complexity for Hamiltonian simulation and random circuits, up to a Brown-S\"usskind threshold. 
\end{abstract}

	\maketitle

\section{Introduction}\label{section:intro}
A fundamental question in quantum information theory is to determine the resources required to perform a quantum task. The aim of this letter is to provide certificates for quantum operations. 
Indeed, we obtain standard certificates consisting of measuring after one operation is applied.
\begin{comment}

\textcolor{red}{
A fundamental question in quantum information theory is to quantify the hardness of a quantum task given finite resources. \textcolor{blue}{(Since infinite resources would render every task trivial.)} The aim of this letter is to present one (mathematically rigorous) \textcolor{blue}{(Contrast with various complexity measures in chaotic QFT)} approach to this problem. In addition, we provide concrete algorithms to certify the hardness of a given quantum operation.}
    
\end{comment}
A difficult task in circuit complexity is to certify that a large number of universal gates is required to approximate a given unitary. Moreover, there is a plethora of \textit{nonequivalent} notions of quantum complexity measures \cite{OT, RW, JW, FR, PD, KH, BT, BV, BASST, BGS, BSh}. Further investigations have gone beyond the notion of circuit complexity, and in particular, notions of complexity have been used in quantum many-body systems \cite{JM, HR, BS1, BS2, MA} and in chaotic quantum systems \cite{PCASA, DS1, ADS, RSSS1, RSSS2}. It was conjectured by Brown and Süsskind in \cite{BS1,BS2} that the complexity of a typical quantum circuit grows linearly with depth and saturates at a value that is exponential in the system size. It highlights the fundamental incompressibility of quantum circuits and inherent hardness of most quantum operations given finite resources. This led to a push by many authors to put sharper (lower) bounds on circuit complexity. Notably, in \cite{HFKEH, Haferkamp23}, the \textit{Brown-Süsskind} conjecture was proven for a particular design of random quantum circuits. 

\begin{comment}

\textcolor{red}{Following the classical complexity theory, a natural way to quantify quantum complexity is to count the minimal number of elementary gates required to approximate a given quantum operation. This leads to the notion of quantum circuit complexity. \textcolor{blue}{Here maybe we should quote the original papers by Deutsch (e.g. Quantum Theory, the Church-Turing Principle and the Universal Quantum Computer)} A plethora of complexity measures have been developed based on the central idea of circuit complexity \cite{ OT, RW, JW, FR, PD, KH, BT, BV, BASST, BGS, BSh}. Apart from complexity in finite quantum systems, there have been attempts to characterize complexity in quantum many-body systems \cite{JM, HR, BS1, BS2, MA} and in chaotic quantum systems \cite{PCASA, DS1, ADS, RSSS1, RSSS2}. The fundamental conjecture (the Brown-Susskind conjecture \cite{BS1,BS2} ) about the nature of quantum circuit complexity states that the complexity of a typical quantum circuit grows linearly with depth and saturates at a value that is exponential in the system size. It highlights the fundamental incompressibility of quantum circuits and inherent hardness of most quantum operations given finite resources. This led to a push by many authors to put sharper (lower) bounds on circuit complexity. Notably, in \cite{HFKEH}, the \textit{Brown-Süsskind} conjecture was proven for a particular design of random quantum circuits. 

%\textcolor{blue}{There's another paper 'proving' Brown-Susskind conjecture in the case of random circuits.}}%
    
\end{comment}
We construct a robust resource-dependent quantum complexity measure, which is inspired by an axiomatic approach to complexity considered in \cite{LBKJL}. The novelty in our approach consists of the ability to treat both closed and open quantum systems simultaneously. A fortiori, it is necessary to define a complexity measure for quantum channels. To this end, given a finite-dimensional von Neumann algebra $\mathcal N \subset \mathbb B(H)$, then for any set $S \subset \mathcal \mathbb{B}(\mathcal H),$ we have an induced \textit{Lipschitz semi-norm} given as 
\begin{equation}
	|||f|||_S:= \sup_{s \in S}\|[s,f]\|_{\infty},\ f\in \mc N.
	\end{equation}
By taking a proper quotient by the mean zero elements, $|||\cdot|||_S$ becomes a norm. Given a quantum channel $\Phi: \mathcal{N}_{*} \to \mathcal{N}_{*}$, the \textit{Lipschitz complexity} of $\Phi$, is defined as 
\begin{equation}
C_S(\Phi):= \norm{\Phi^* - \id: (\mathcal N, |||\cdot|||_S) \to \mathbb B(H)}.
\end{equation}
For instance, if the resource set is given by unitaries, then $C_S(Ad_{s_{i_k}\cdots s_{i_1}}) \leq k$ if $s_{i_j}, 1\le j \le k$ are in $S$, which provides a lower bound for the gate length.  

For resource sets given by Hamiltonians, we compare our notion of complexity with Nielsen's geometric complexity initiated in \cite{N1, N2, N3}. In particular, for a unitary $U$, we prove the Lipschitz complexity of $Ad_U$ is upper bounded by geometric complexity of $U$ with respect to $S$. Lipschitz norms are geometric in nature. This is particularly evident in Connes' notion of a spectral triple in noncommutative geometry \cite{connes1994noncommutative}, where $S$ has cardinality 1. Moreover, different choice of preferred direction in the Lie algebra is fundamental in sub-Riemannian geometry underlying Nielsen's work, see also \cite{goh2023lie}. 

In the case that the resource set is a Pauli gate set, our Lipschitz complexity is equivalent to the \textit{Wasserstein-1} complexity introduced in \cite{LBKJL}. 
\begin{comment}
    Lipschitz norms are geometric in nature. This is particularly evident in Connes notion of a spectral triple in noncommutative geometry, where $S$ has cardinality 1. Moreover, different choice of preferred direction in the Lie algebra is fundamental in sub-Riemannian geometry(underlying Nielsen's work)
\end{comment}

We prove a version of the Brown-S\"usskind conjecture about the linear growth of a random unitary drawn from the unitary resource set $S$. We do not expect exponential length using such a na\"ive design. We also prove a similar lower bound on the complexity of Hamiltonian simulation via the quantum stochastic drift (qDRIFT) protocal considered in \cite{Campbell19}, implemented in Qiskit \cite{Qiskit}. The mathematical details are outside the scope of this letter, and are presented in \cite{Lipschitz}. 
 



Throughout this letter, $E_{fix}$ is a conditional expectation onto the subspace which is the commutant of $S$. We drop the dependence on the resource set for simplicity.
\section{Comparison of complexities}
\subsection{Circuit complexity}
For quantum circuits, a natural definition of complexity is defined by the depth/length. Suppose a gate set $S \subseteq U(d)$ is given. Then the exact length $l_S(U)$ of a unitary $U$ is defined by, see also \cite{HFKEH,Haferkamp23,Nietner23}
\begin{equation*}
l_S(U):= \inf \{l \ge 1: U = V_1\cdots V_l,\ V_i \in S\}.
\end{equation*}
Similarly, for any $\delta>0$, the approximate $\delta$-complexity is defined by 
\begin{equation*}
l_{S,\delta}(U):= \inf \{l \ge 1: \|U - V_1\cdots V_l\|_{\infty}\le \delta,\ V_i \in S\}.
\end{equation*}
Our notion of complexity provides a lower bound to the above complexity:
\begin{theorem}
Given a unitary $U$. The resource-dependent complexity $C_S(U)$ of is upper bounded by $l_S(U)$. Moreover, under reasonable assumptions on the resource set $S$, we find an $f$ such that 
\begin{equation}
C_S(U) \le l_{S,\delta}(U) + f(\delta).
\end{equation}
\end{theorem}


\subsection{Wasserstein complexity}
Quantum Wasserstein complexity of order 1 for quantum channels was first introduced in \cite{LBKJL}. We show that by choosing our resource set $S$ appropriately (Pauli gates), $C_S(\cdot)$ is equivalent to $C_{W_1}(\cdot)$ introduced in \cite{LBKJL}. 
Indeed, \cite{PMTL} defined a Wasserstein-1 distance, $\|\cdot\|_{W_1}$, on the state space and identify its dual norm
\begin{equation}
\begin{aligned}
\|H\|_L= 2\max_{1\le i \le n} \min\big\{\|H - id_i \otimes H^{(i)}\| \big\},
\end{aligned}
\end{equation}
where $H^{(i)}$ is a traceless selfadjoint operator acting on $n-1$ qudits avoiding the $i-$th register.


\begin{prop}
Suppose $S = \{X_j,Y_j,Z_j:1\le j \le n\}$ is the Pauli gate set. Then for any $H$ self adjoint, 
\begin{equation}
\frac{1}{2}|||H|||_S \le \|H\|_L \le \frac{3}{2}|||H|||_S, 
\end{equation}
where $|||H|||_S = \sup_{s\in S}\|[s,H]\|$.
\end{prop}

By duality, we obtain a dual norm for states(resource-dependent earth-mover distance):
\begin{equation*}
    \|\rho\|_{S}^* = \sup\{|\tr(\rho H)|: E_{fix}(H) = 0, \||H\||_S\le 1\}
\end{equation*}
Thus our resource complexity $C_S(\Phi)$ is given by 
\begin{equation*}
    C_S(\Phi) = \|\Phi - id: S_1(\mc H) \to (S_1(\mc H), \|\cdot\|_S^*)\|.
\end{equation*}
Then the equivalence between resource-dependent complexity for Pauli resource and Wasserstein complexity of order 1 is a direct corollary:
\begin{cor}
Suppose $S$ is the Pauli gate set. Then for any quantum channel $\Phi$,
$$\frac{1}{2} C_S(\Phi) \le C_{W_1}(\Phi) \le \frac{3}{2} C_S(\Phi).$$
\end{cor}

\subsubsection*{Generalizations}
%\textcolor{blue}{I think the simplest generalization is to toric codes. But we should probably not mention it here.}%
We may replace the resource set $S$ by the smaller set, $\{X_j,Y_j: 1\le j \le n\}$. 
This example is a paradigm for our theory, which applies to all bracket-generating families of Hamiltonians, see also \cite{Lipschitz}. 
%\textcolor{red}{Another interesting generalization is that we may replace the resource set by the stabilizer operators generating toric codes \cite{Kitaev}. In fact, define $S = \{A_v, B_p\}$, 

%where $v$ ranges over the vertice set and $p$ ranges over %the face set of a 2D lattice with boundary condition. We %leave it open for future research.`
%}

\subsection{Geometric complexity}
\begin{comment}
    \color{red} 
Maybe we should talk about Hormander system in a more intuitive way, for the benefits of the physics audience. For example, something like: Because we have defined a resource set, whatever directions outside of the resource set are forbidden.

\color{black}
\end{comment}

Nielsen discovered the link between the geometric distance and circuit complexity, now referred to geometric complexity \cite{N1,N2,N3}.

To be more precise, fix a subspace $H \subseteq \mathfrak{su(2^n)}$ and $X_1,\cdots, X_m$ be a linearly independent set such that $span\{X_1,\cdots, X_m\} = H$. We consider $S = \{X_1,\cdots,X_m\}$ as the infinitesimal resource to construct the target unitary in $SU(2^n)$. For any element $h\in H$, define the induced norm of $h$ by 
\begin{equation}
\|h\|_H = \sum_j |\alpha_j|^2,\ h = \sum_{j} \alpha_j X_j.
\end{equation}
Then the corresponding distance introduced by \cite{Chow}, see also \cite{Caratheodory, Gromov}.
\begin{comment}
    Refer to Caratheodory, Chow and Gromov's paper.
\end{comment}
is given by 
\begin{equation}
\begin{aligned}
d_H(g,h)& := \inf \{\int_0^1 \|X_{\gamma(t)}\|_H dt : \\
& \gamma'(t)= X_{\gamma(t)} \gamma(t), X_{\gamma(t)} \in H\},
\end{aligned}
\end{equation}
where $\gamma(t)$ ranges over all the piecewise smooth curve such that $\gamma(0) = g,\gamma(1) = h,\ g,h\in SU(2^n)$. $\gamma'(t)= X_{\gamma(t)} \gamma(t), X_{\gamma(t)} \in H$ should be understood as the admissible directions of the curve are restricted within $H$.

Then by choosing the $\{X_1,\cdots, X_m\}$ as weighted Pauli basis, we can recover Nielsen's complexity. We note that our notion provides a lower bound for it.
\begin{theorem}\label{Geometric complexity: single}
For any $U \in SU(2^n)$, we have 
\begin{equation}
C_{H}(Ad_{U}) \le d_H(U,I). \label{eq: nielsen}
\end{equation}
\end{theorem}
In some cases, e.g., choosing the left regular representation, one obtains equality in Equation~\eqref{eq: nielsen}, via approximation using the components of the left regular representation.

\begin{comment}
\begin{center}
\begin{tikzpicture}
% Circle A
\draw [red](0,0) circle (3cm) ;
% Circle B
\draw (0,0) circle (1cm) node[left] {Wasserstein };
\draw (0,-3) node[below]{Resource dependent};

\draw (8,3) circle (2cm) node[right] {length};
\draw[blue](3,0) node[right]{Lower bound};
\draw[->] (3,1) -- (6,2) ;
\draw (8,-3) circle (2cm) node[right] {Geometric};
\draw[->] (3,-1) -- (6,-3) ;
\end{tikzpicture}
\end{center}




    
\end{comment}
\section{Linear growth and Brown-Susskind conjecture}
\subsection{Random circuits}

In this subsection we construct random circuits which exhibit linear growth of Lipschitz complexity, a lower bound of circuit complexity. To be more precise, let us fix a set of resource gates $S$. Our na\"ive circuit model consists of concatenating random samples from $S$.
\begin{quantikz}[scale=1.5]
\lstick{$|\psi\rangle$} & \gate{s_{i_1}} & \gate{s_{i_2}} & \qw \cdots & \qw & \gate{s_{i_l}}& \meter{}
\end{quantikz}


The lower bound for Lipschitz complexity is obtained by a suitable choice of $|\psi \rangle$, and measurement with small Lipschitz norm. In addition, we assume $S = S^\dag$, and a fully supported probability measure $\nu$ on S. 
\begin{theorem}
There exists constants $L,c$, such that for $l \leq L$, \[
\mathbb P_{\nu^l}(C_S(ad_{s_{i_1}}\cdots ad_{s_{i_l}}) > cl) \geq \frac{l}{4l+4L}.
\]
\end{theorem}

\begin{table*}[t]\centering
 \begin{tabular}{|c|c|} \hline 
\text{Gate sets}& \text{Complexity, threshold, probability} \\ \hline
 Sufficiently connected circuits \cite{HFKEH} &  
   Exact circuit complexity,\\ &exponential, \\ & linear growth with probability 1\\
 \hline
Universal gate sets \cite{Haferkamp23} & 
   Approximate circuit complexity,\\ &exponential, \\ & linear growth with probability close to 1 \\
\hline
 This letter: Arbitrary gate set  &    
 Lipschitz complexity,\\with minor assumption& Brown-S\"usskind threshold, \\& linear growth with positive probability \\
\hline
  \end{tabular}
\end{table*}
\subsection{Hamiltonian simulation}
In this subsection, we illustrate our notion of complexity by an application to Hamiltonian simulation via the `qDRIFT' protocol introduced by \cite{Campbell19}. The starting point is a Hamiltonian given by
 \[ H = \sum_{j=1}^m h_j H_j \] 
where $H_j$ are `elementary' components such that $\|H\|_{\infty} \le 1$ and $h_j$ are some positive weights. Denote
\begin{equation}
\lambda: = \sum_{j=1}^m h_j.
\end{equation}
In this setting, we consider the \textit{resource set} 
\begin{equation}
S = \{H_j|1\le j\le m\}.
\end{equation}
 The goal is to approximate the unitary $U(t) = \exp(itH)$ using a product of $U_j(\tau) = \exp(i\tau H_j), 1\le j \le L$ up to some desired precision. In fact, $\tau= \frac{t\sum_j h_j}{N}$, where $N$ is the length of the approximating random circuit. By defining the \emph{gate set} $U_\tau:= \{\exp(i\tau H_j): 1 \leq j \leq m\}$, we may apply the random model from above for the probability measure $\nu_j=\frac{h_j}{\sum_k h_k}$. According to Campbell \cite{Campbell19}, \[
\|(\mathbb E_\nu(ad_u))^N - ad_{U(t)}\|_{\diamond} \le 2\frac{\lambda^2 t^2}{N}\exp(2\lambda t/N),
\] provides a channel approximation of the Hamiltonian evolution. Using this approximation we obtain:
\begin{theorem}
Define $\lambda = \sum_{j=1}^L h_j$. Then 
\begin{equation}\label{Hamilton: upper bound}
C_S(ad_{U(t)}) \le \lambda t.
\end{equation}
Moreover, suppose $H \neq 0$, then for $0\le t \le \frac{1}{4(e-2)\|H\|_{\infty}}$, we have 
\begin{equation}
C_S(ad_{U(t)}) \ge \frac{1}{2}\delta t,
\end{equation}
where $\delta>0$ is given by 
\begin{equation}
\delta:= \sup_{x: E_{fix}(x)=0}\frac{\|[H,x]\|}{\sup_{j}\|[H_j,x]\|}.
\end{equation}
\end{theorem}
Not only the Hamiltonian simulation admits lower complexity bounds, this remains true for the random approximations (even if the Hamiltonian is trivial!). Indeed, there exists constants $c$ and $L$ depending on $S$ such that 
\begin{align} \label{random}
\mathbb{P}&(\bigcap_{l = 2^{k}<2^L}\{ C_S(ad_{u_{i_1}}\cdots ad_{u_{i_l}}) > c l \}) > \frac{1}{4} 2^{-L}   .  
\end{align}
In many applications the random gates $u_j=e^{i\tau H_j}$ are stopped local evolutions. 
The estimate for increasing length on dyadic integers also holds in the random gate setting. Similar results for the diamond norm appear in \cite{chen2021}.

\subsection{Simple certificate}\label{Calculation}
We illustrate how to certify growth of complexity for the Pauli resource set   \begin{align*}
S&= \{X_j: 1 \leq j \leq n\} \cup \{Y_j: 1 \leq j \leq n\}, \\ H &= \sum_{j=1}^{2n} H_j, H_{2k-1} = X_k, H_{2k} = Y_k, 1\le k \le n.
\end{align*} The observables $\mathrm{X}=\sum_{j=1}^n X_j, \mathrm{Y}=\sum_{j=1}^n Y_j$, optimize the Lipschitz complexity for $E_{fix}^*$ and also $\delta=n\sim C_{S}(E_{fix})$. In particular the estimate \eqref{random} applies with $1/4$ replaced by a constant. Choosing an alternating random gate  
\[ U  = e^{i\tau X_1}e^{i\tau Y_2} e^{i\tau X_3}\cdots \] 
the Lipschitz function $f:= \begin{pmatrix} X & 0 \\ 0& Y \end{pmatrix}$, and the state $|\psi_n\rangle:= \frac{1}{\sqrt{2}} \kla \begin{array}{c}
  \ket{0}^{\otimes n}   
\\ \ket{0}^{\otimes n} 
\end{array} \mer$, we find  %$n\tau \geq C_S^{cb}(ad_U)$ and 
\begin{align*}
    C_S^{cb}(ad_U)& \geq
 | \tr\big( (\ad_{U\otimes I} |\psi_n\rangle \langle \psi_n| - |\psi_n\rangle \langle \psi_n|) f\big)| \\
 & \geq n|\sin \tau|.
\end{align*}

\begin{comment}
    This lower bound also confirms the optimal order $C_S(E_{fix})\sim n$.
\end{comment}


%Letting $V_j \in \{X_j, Y_j\}, 1 \leq j \leq n,$ consider %the unitary $U:= \prod e^{i\tau k_j V_j}$, where $k_j = |k: H_{i_k} \in \{X_j, Y_j\}|.$ Choosing $f:= \begin{pmatrix} X & 0 \\ 0& Y \end{pmatrix}$, and the state $|\psi_n\rangle:= \frac{1}{\sqrt{2}} \kla \begin{array}{c}
%  \ket{0}^{\otimes n}   
%\\ \ket{0}^{\otimes n} 
%\end{array} \mer$. A standard calculation then produces \[
%\tau l \geq C_S^{cb}(ad_U) \geq | \tr( \ad_U |\psi_n\rangle \langle \psi_n| - |\psi_n\rangle \langle \psi_n| f)| \sim \tau l/2, 
%\] where $l$ is the length of the unitary $U$ and $\tau l \le \tau_0$ for $\tau_0$ small enough.}

\subsection{Toric code and certificate}
 The resource set is given by the stabilizer operators generating toric codes \cite{Kitaev}. In fact, define $S = \{A_v, B_p\}$, 
\begin{align*}
    A_v = \prod_{i\in \text{star}(v)}X_i, \\
    B_p = \prod_{e\in \text{bdry}(p)}Y_e
\end{align*}
where $v$ ranges over the vertices set and $p$ ranges over the face set of a 2D lattice with periodic boundary condition, so that the $X_i$ are on the edges adjacent to $p$ and $Y_e$ on the edges.  
The generators of the Hamiltonian $H=\sum_{v} A_v +\sum_{p} B_p$ commute. 

We find eight witnesses for the lower bound of $C_S(E_{fix}) \gtrsim d^2$. Indeed, we choose one edge $e_p$ for every face $p$ and define $Y_1 = \sum_{p}Y_{e_p}$. 
By choosing single edges for faces or vertices, we obtain the other seven witnesses. As above we take the direct sum of eight such Lipschitz operators and pick
$\ket{\psi_n}=\frac{1}{\sqrt{8}}(\ket{0}^{\ten d^2},\cdots,
\ket{0}^{\ten d^2})^{T}$. For every 
 $$ U  \exp(i\tau H_{j_1}) \cdots \exp(i\tau H_{j_l})$$
we find 
\begin{align*}
   \tau l \geq C_S^{cb}(ad_U) \geq |\tr ((ad_{U\ten I}(\psi_n) - \psi_n )f)| \geq c \tau l
\end{align*}
for $|\tau l|\le \pi/8$ and some universal constant. For this special case of commuting Hamiltonian we find a simple length certificate. Using the spectral gap estimate \cite{lucia} combined with the order estimate \cite{GJLL22}, we obtain  $C_S(E_{fix})\lesssim d^3$. A uniform estimate of the $\CLSI\sim {\rm gap}$ for this highly non-ergodic system would imply the expected qubit order for $C_S(E_{fix})$, see \eqref{CLSI}. 


\subsection{Open systems}
Our notion can be naturally generalized to open systems since the complexity measures are defined for quantum channels. Suppose we have quantum Markov semigroup $\{T_t = \exp(tL)\}$, where $$  Lx = \sum_{j=1}^m a_j^*xa_j - \frac{1}{2}(a_j^*a_jx + xa_j^*a_j)$$ is a symmetric or unitary Lindbladian generator. We now use the \textit{jump operators} as the \textit{resource set} $S = \cup_{j=1}^m \{a_j,a_j^*\}$. Using elementary properties of Lipschitz complexity, 
\begin{enumerate}
    \item $C_S(\Phi\circ \Psi) \le C_S(\Phi) + C_S(\Psi)$ for two quantum channels.
    \item $C_S(\Phi) \le C_S(E_{fix}) \|\Phi - id\|_{\diamond}$.
\end{enumerate}
An immediate consequence of the above properties is $$C_S(T_t^*) \le C_S(E_{fix}) \|L\|_{cb} t$$

To obtain lower bound, we use a \textit{mixing time} trick. Indeed, recall the mixing time of the semigroup $T_t$ is defined by 
\begin{equation}
k^{cb}(\varepsilon):= \inf\{t >0: \|T_t-E_{fix}\|_{cb} \le \varepsilon\}.
\end{equation}
If the Lindbladian generator is symmetric or unitary, the return time is finite.
\begin{theorem}
The lower bound estimate of $C_S(T_t)$ grows linearly before $k^{cb}(\varepsilon)$ for any $\varepsilon>0$ and is comparable to $C_S(E_{fix})$ after $k^{cb}(\varepsilon)$.
\end{theorem}	
Using deep mathematical tools, we prove upper estimates for $C_S(E_{fix})$, in particular:
\begin{enumerate}
    \item If $L$ is induced via a finite dimensional representation $\pi$ of a Lie group $G$ with a H\"ormander system $H$, then 
    \begin{align*}
        C_S(E_{fix}) \le \sup_{g\in G} d_H(g,id).
    \end{align*}
    \item Using the tools in \cite{Fisher}, we can get 
    \begin{align}\label{CLSI}
    C_S(E_{fix}) \le c_0 \text{CLSI}(L)^{-1/2} \sqrt{m}.
    \end{align}
\end{enumerate}
Using the Pauli gates as jump operators of the Lindbladian generator, see Section \ref{Calculation}, we recover the upper and lower qubit bounds obtained by \cite{LBKJL}.

\medskip
\section{Conclusion}
Our flexible notion for quantum channels falls into the axiomatic regime from \cite{LBKJL}. By choosing the resource set suitably, we can show upper and lower bounds for the complexity of random circuits and continuous time evolution. The linear aspect of the Brown-S\"usskind conjecture is confirmed in our context. The new Brown-S\"usskind threshold  is determined by the geometric properties of the resource set.



 
	
\bibliography{complexity}% Produces the bibliography via BibTeX.

\end{document}
%
% ****** End of file apssamp.tex ******


