% ----------------------------------------------------------------
% AMS-LaTeX Paper ************************************************
% **** -----------------------------------------------------------
\documentclass[12pt]{amsart}

\usepackage{amsmath,amsxtra,amssymb,amsthm,amsfonts}
%etoolbox,setspace}
%showkeys}
\usepackage{mathrsfs}
\usepackage[T1]{fontenc}
\usepackage[utf8]{inputenc}
\usepackage{mathtools}   % loads »amsmath«
\usepackage{tikz}
\usepackage{bbm}
\usepackage{comment}
\usetikzlibrary{arrows, automata}
\usepackage{color,hyperref}
\definecolor{dblue}{rgb}{0, 0, 0.72}
%\usepackage{amsmath,amsxtra,amssymb,showkeys}
\def\theequation{\thesection.\arabic{equation}}

\allowdisplaybreaks

%\usepackage[active]{srcltx} % SRC Specials: DVI [Inverse] Search
% ----------------------------------------------------------------
\vfuzz2pt % Don't report over-full v-boxes if over-edge is small
\hfuzz2pt % Don't report over-full h-boxes if over-edge is small
% THEOREMS -------------------------------------------------------
\numberwithin{equation}{section}
\newtheorem{lemma}{Lemma}[section]
\newtheorem{prop}[lemma]{Proposition}
\newtheorem{theorem}[lemma]{Theorem}
\newtheorem{ttheorem}{Theorem}
\newtheorem{cor}[lemma]{Corollary}
\newtheorem{res}[lemma]{Resum\'{e}:}
\newtheorem{conj}[lemma]{Conjecture}
\newtheorem{proposition}{Proposition}[section]
\newtheorem{quest}[lemma]{Question}
\newtheorem{prob}[lemma]{Problem}
\newtheorem{rem}[lemma]{Remark}
\newtheorem{remark}[lemma]{Remark}
\newtheorem{frem}[lemma]{Final remark}
\newtheorem{fact}[lemma]{Fact}
\newtheorem{conv}[lemma]{Convention}
\newtheorem{conc}[lemma]{Conclusion}
\newtheorem{obs}[lemma]{Observation}
\newtheorem{war}[lemma]{Warning}
\newtheorem{hypo}[lemma]{Hypothesis}
\newtheorem{example}[lemma]{Example}
\newtheorem{definition}[lemma]{Definition}
\newtheorem{corollary}[lemma]{Corollary}
\newtheorem{axiom}{Axioms}

\newenvironment{claim}[1]{\par\noindent\underline{Claim:}\space#1}{}
\newenvironment{claimproof}[1]{\par\noindent\underline{Proof:}\space#1}{\hfill $\blacksquare$}


\newcommand{\re}{\begin{rem}\rm}
	\newcommand{\mar}{\end{rem}}
\newtheorem{exam}[lemma]{Example}
\newtheorem{exams}[lemma]{Examples}
\newtheorem{defi}[lemma]{Definition}
\newtheorem{ddefi}{Definition}

\newtheorem{defir}[lemma]{Definition and Remark}

\newcommand{\br }{|}
\newcommand{\kla}{\left ( }
\newcommand{\mer}{\right ) }

\newcommand{\ee }{\mathrm{I}\!\!1}

\newcommand{\ketbra}[1]{|{#1}\rangle\langle{#1}|}

%\newcommand{\for}{\begin{eqnarray*}}
\renewcommand{\for}{\begin{eqnarray*}}
	
	\newcommand{\mel}{\end{eqnarray*}}

\newcommand{\und}{\quad\mbox{and}\quad}
\newcommand{\mitt}{\left | { \atop } \right.}
\newcommand{\kl}{\pl \le \pl}
\newcommand{\gl}{\pl \ge \pl}
\newcommand{\kll}{\p \le \p}
\newcommand{\gll}{\p \ge \p}
\newcommand{\nach}{\rightarrow}
\newcommand{\lel}{\pl = \pl}
\newcommand{\lell}{\p=\p}


\newcommand{\ew}{{\mathbb E}}
\newcommand{\ez}{{\mathbb E}}

\newcommand{\BB}{{\mathbb B}}
\renewcommand{\L}{\mathcal{L}}
\newcommand{\eiz}{{\rm 1}\!\!{\rm 1}}

\newcommand{\nz}{{\mathbb N}}
\newcommand{\nen}{n \in \nz}
\newcommand{\ken}{k \in \nz}
\newcommand{\rz}{{\mathbb R}}
\newcommand{\zz}{{\mathbb Z}}
\newcommand{\qz}{{\mathbb Q}}
\newcommand{\Mz}{{\mathbb M}}
\newcommand{\rn}{\rz^n}
\newcommand{\cz}{{\mathbb C}}
\newcommand{\kz}{{\rm  I\! K}}
\newcommand{\pee}{{\rm I\! P}}
\newcommand{\dop}{{\mathbb D}}
\newcommand{\dt}{{\rm det}}
\newcommand{\ob}{\bar{\om}}
\newcommand{\ten}{\otimes}
\newcommand{\sz}{\scriptstyle}
\newcommand{\wet}{\stackrel{\wedge}{\otimes}}

\newcommand{\gga}{{\rm I}\!\Gamma}

\newcommand{\dd}{\mathbb{D}}

\DeclareMathOperator{\dom}{dom}

\DeclareMathOperator{\Img}{Im}

\DeclareMathOperator{\Rep}{Re}

\DeclareMathOperator{\rc}{rc}

\DeclareMathOperator{\Exp}{Exp}

\DeclareMathOperator{\gr}{gr}

\DeclareMathOperator{\rg}{rg}
\DeclareMathOperator{\vol}{vol}

\DeclareMathOperator{\Hs}{H}
\DeclareMathOperator{\Vs}{V}
\DeclareMathOperator{\Ks}{K}

\DeclareMathOperator{\diam}{diam}

\DeclareMathOperator{\Proj}{Pr}

\DeclareMathOperator{\sgn}{sgn}


\DeclareMathOperator{\err}{err}

\DeclareMathOperator{\tg}{tg}

\DeclareMathOperator{\spa}{span}

\DeclareMathOperator{\Sh}{Shift}

\DeclareMathOperator{\Pe}{Perm}

\DeclareMathOperator{\Hessian}{Hessian}

\DeclareMathOperator{\Ric}{Ric}
\DeclareMathOperator{\Ker}{Ker}


\DeclareMathOperator{\Tan}{Tan}

\DeclareMathOperator{\grad}{grad}

\DeclareMathOperator{\fix}{fix}

\DeclareMathOperator{\Tr}{Tr}
\DeclareMathOperator{\tr}{tr}

\DeclareMathOperator{\Ent}{Ent}

\DeclareMathOperator{\spread}{spread}

\DeclareMathOperator{\spann}{span}

\DeclareMathOperator{\grd}{grad}

\DeclareMathOperator{\grr}{gr}

\DeclareMathOperator{\Res}{Res_{\small{W}}}

\DeclareMathOperator{\cl}{cl}

\DeclareMathOperator{\Ricci}{Ricci}

\DeclareMathOperator{\krn}{ker}

\DeclareMathOperator{\CLSI}{CLSI}
\DeclareMathOperator{\MLSI}{MLSI}
\DeclareMathOperator{\MpSI}{MpSI}
\DeclareMathOperator{\CpSI}{C_pSI}
\DeclareMathOperator{\CpqSI}{C_{pq}SI}

\DeclareMathOperator{\Lin}{\text{\fontsize{2.5}{4}\selectfont {Lin}}}
\newcommand{\suppl}{{\sup}^+}

\newcommand{\p}{\hspace{.05cm}}
\newcommand{\pl}{\hspace{.1cm}}
\newcommand{\pll}{\hspace{.3cm}}
\newcommand{\pla}{\hspace{1.5cm}}
\newcommand{\hz}{\vspace{0.5cm}}
\newcommand{\lz}{\vspace{0.0cm}}
\newcommand{\hoch}{ \vspace{-0.86cm}}
\newcommand{\hhz}{\vspace{0.3cm}}
%\newcommand{\qed}{\hspace*{\fill}$\Box$\hz\pagebreak[1]}
%\renewcommand{\qed}{\hspace*{\fill} {\vrule height7pt width7pt
%depth0pt} \hz\pagebreak[1]}
%\renewcommand{\qed}{\hspace*{\fill}$\Box$\hz\pagebreak[1]}


%\newcommand{\qed}{\hspace*{\fill}$\Box$\hz\pagebreak[1]}

\newcommand{\qd}{\end{proof}\vspace{0.5ex}}

\renewcommand{\qedsymbol}{\vrule height7pt width7pt depth0pt \hz}

\newcommand{\Om}{\Omega}
\newcommand{\om}{\omega}
\renewcommand{\a}{\alpha}
\newcommand{\al}{\alpha}
\newcommand{\de}{\delta}
\newcommand{\si}{\sigma}
\newcommand{\J}{{\mathcal J}}
%\newcommand{\H}{{\mathcal H}}
\newcommand{\Si}{\Sigma}
\newcommand{\tet}{\theta}
\newcommand{\ttett}{\vartheta}
\newcommand{\fid}{\hat{\Phi}}
\newcommand{\La}{\Lambda}
\newcommand{\la}{\lambda}
\newcommand{\eps}{\varepsilon}
\newcommand{\vare}{\varepsilon}
\newcommand{\ds}{D_{\si}}
\newcommand{\lzn}{\ell_2^n}
\newcommand{\len}{\ell_1^n}
\newcommand{\lin}{\ell_{\infty}^n}
\newcommand{\lif}{\ell_{\infty}}
\newcommand{\lih}{\ell_{\infty,\infty,1/2}}
\newcommand{\id}{\iota_{\infty,2}^n}
\newcommand{\zp}{\stackrel{\circ}{Z}}
\newcommand{\ce}{{\tt c}_o}
\newcommand{\leb}{{\mathcal L}}
\newcommand{\pot}{{\mathcal P}(}
\newcommand{\es}{{\mathcal S}}
\newcommand{\ha}{{\mathcal H}}
\newcommand{\hp}{\frac{1}{p}}
%{{\rm c}$_0$ }
%\renewcommand{\L}{{\mathcal L}}
\newcommand{\F}{{\mathcal F}}
\newcommand{\E}{{\mathcal E}}
\newcommand{\A}{{\mathcal A}}
%\newcommand{\H}{{\mathcal H}}

\newcommand{\D}{{\mathcal D}}
\newcommand{\C}{{\mathcal C}}
\newcommand{\K}{{\mathcal K}}
\newcommand{\R}{{\mathcal R}}
\newcommand{\Ha}{{\mathcal H}}
\renewcommand{\S}{{\mathcal S}}
\newcommand{\SH}{{\mathcal S}{\mathcal H}}

\newcommand{\Ma}{{\mathbb M}}

\newcommand{\MM}{{\mathcal M}}
\newcommand{\m}{{\bf \sl m}}
\newcommand{\OL}{{\mathcal OL}}
\newcommand{\COL}{{\mathcal COL}}
\newcommand{\COS}{{\mathcal COS}}
\newcommand{\Uu}{{\mathbb U}}
\newcommand{\Vv}{{\mathbb   V}}
\newcommand{\Pp}{{\mathbb   P}}
\newcommand{\Pd}{{\mathbb I}\!{\mathbb   P}}
\newcommand{\LSI}{\operatorname{LSI}}
\newcommand{\mm}{{\mathbb M}}
\newcommand{\amo}{ \ll \!\! \scriptsize{a}}
\newcommand{\ttau}{\si}
\newcommand{\Kk}{{\rm  K}}
\newcommand{\Ww}{{\mathbb W}}

\newcommand{\Gb}{{\rm I}\!\Gamma}
\renewcommand{\P}{\mathcal P}
\newcommand{\PP}{{\rm I\! P}}
\newcommand{\N}{{\mathcal N}}
\newcommand{\G}{\Gamma}
\newcommand{\g}{\gamma}
\newcommand{\T}{\mathcal T}
\newcommand{\I}{{\mathcal I}}
\newcommand{\U}{{\mathcal U}}
\newcommand{\B}{{\mathcal B}}
\newcommand{\Hh}{{\mathcal H}}

\newcommand{\pf}{\begin{proof}}
\newcommand{\AB}{{\mathbb A}}


%\newcommand{\pxq}{\pi_{X,q}}
%\newcommand{\pxe}{\pi_{X,1}}
%\newcommand{\pxz}{\pi_{X,2}}

\newcommand{\noo}{\left \|}
\newcommand{\rrm}{\right \|}



\newcommand{\bet}{\left |}
\newcommand{\rag}{\right |}
\newcommand{\be}{\left|{\atop}}
\newcommand{\ra}{{\atop}\right|}


\newcommand{\intt}{\int\limits}
\newcommand{\summ}{\sum\limits}
\newcommand{\limm}{\lim\limits}
\newcommand{\ver}{\bigcup\limits}
\newcommand{\durch}{\bigcap\limits}
\newcommand{\prodd}{\prod\nolimits}
\newcommand{\su}{\subst}

\newcommand{\xspace}{\hbox{\kern-2.5pt}}
\newcommand{\xyspace}{\hbox{\kern-1.1pt}}

\newcommand\lge{\langle}
\newcommand\rge{\rangle}

\newcommand\ltriple{ltriple}

\newcommand\bra[1]{\langle  #1|}
\newcommand\ket[1]{| #1\rangle}

%\newcommand\DD[1]{\mathbb{D}( #1)}
\newcommand\DD[1]{\mathbb{D}#1}
\renewcommand\dd{\p\mathbb{D}}
%\newcommand\[1]{\mathbb{d}(#1)}

\newcommand\tnorm[1]{\left\vert\xspace\left\vert\xspace\left\vert\mskip2mu
	#1\mskip2mu \right\vert\xspace\right\vert\xspace\right\vert}

\newcommand\ttnorm[1]{\left\vert\xspace\left\vert\xspace\left\vert\mskip2mu
	#1\mskip2mu \right\vert\xspace\right\vert\xspace\right\vert}

\newcommand{\ch}{\stackrel{\circ{}}{H}{\atop}\!\!\!}

\newcommand{\lb}{\langle \langle }
\newcommand{\rb}{\rangle \rangle }

\newcommand{\TT}{{\mathbb T}}
\newcommand{\Hm}{{\mathcal H}}

\newcommand{\X}{\mathcal{X}}
\newcommand{\Y}{\mathcal{Y}}

\newcommand{\supn}{{\sup_n}^+}
\newcommand{\supt}{{\sup_t}^+}

\newcommand{\sint}{\pl {\si\!\!\int}}

\newcommand{\kk}{<}
%\newcommand{\gr}{>}
\newcommand{\st}{|}

\newcommand{\pr}{{\mathrm P\mathrm r}}

%\newcommand{\Gammab}{{\rm I}\!\Gamma}

\newcommand{\8}{\infty}

\newcommand{\ff}{{\mathbb  F}}

\newcommand{\norm}[2]{\parallel \! #1 \! \parallel_{#2}}


\newcommand{\ssubset} {\!\!\subset\! \!}
%\topmargin -1 cm

%\renewcommand{\baselinestretch}{1.05 }

\hyphenation{comm-ut-ta-ti-ve}

%\mathchardef\mhyphen
%\topmargin -1cm
%\oddsidemargin 0.3 cm

%\evensidemargin 0.3 cm

\allowdisplaybreaks

\textwidth 16 cm \textheight 22.7cm
%\parindent0em
%\include{addti}
\newcommand{\ttt}{{\bf t}}

\definecolor{LightGray}{rgb}{0.94,0.94,0.94}
\definecolor{VeryLightBlue}{rgb}{0.9,0.9,1}
\definecolor{LightBlue}{rgb}{0.8,0.8,1}
\definecolor{DarkBlue}{rgb}{0,0,0.6}
\definecolor{LightGreen}{rgb}{0.88,1,0.88}
\definecolor{MidGreen}{rgb}{0.6,1,0.6}
\definecolor{DarkGreen}{rgb}{0,0.6,0}
\definecolor{DarkGrreen}{rgb}{0,0.8,0}

\definecolor{VeryLightYellow}{rgb}{1,1,0.9}
\definecolor{LightYellow}{rgb}{1,1,0.6}
\definecolor{MidYellow}{rgb}{1,1,0.5}
\definecolor{DarkYellow}{rgb}{0.8,1,0.3}
\definecolor{VeryLightRed}{rgb}{1,0.9,0.9}
\definecolor{LightRed}{rgb}{1,0.8,0.8}
\definecolor{DarkRed}{rgb}{0.8,0.2,0}
\definecolor{DarkRedb}{rgb}{0.6,0.2,0}
\definecolor{DarkLila}{rgb}{0.8,0,1}
\definecolor{Beige}{rgb}{0.96,0.96,0.86}
\definecolor{Gold}{rgb}{1.,0.84,0.}
\definecolor{Goldb}{rgb}{0.7,0.3,0.5}
\definecolor{MyYellow}{rgb}{1.,0.84,0.8}


\newcommand{\VeryLightBlue}[1]{{\color{VeryLightBlue}{#1}}}
\newcommand{\LightBlue}[1]{{\color{LightBlue}{#1}}}
\newcommand{\Blue}[1]{{\color{blue}{#1}}}
\newcommand{\DarkBlue}[1]{{\color{DarkBlue}{#1}}}
\newcommand{\DarkGreen}[1]{{\color{DarkGreen}{#1}}}
\newcommand{\LightGreen}[1]{{\color{LightGreen}{#1}}}

\newcommand{\DarkGrreen}[1]{{\color{DarkGrreen}{#1}}}
\newcommand{\DarkRedb}[1]{{\color{DarkRedb}{#1}}}
\newcommand{\red}[1]{{\color{DarkRedb}{#1}}}


\newcommand{\VeryLightRed}[1]{{\color{VeryLightRed}{#1}}}
\newcommand{\LightRed}[1]{{\color{LightRed}{#1}}}
\newcommand{\DarkYellow}[1]{{\color{DarkYellow}{#1}}}
\newcommand{\LightYellow}[1]{{\color{LightYellow}{#1}}}

\newcommand{\DarkRed}[1]{{\color{DarkRed}{#1}}}
\newcommand{\DarkLila}[1]{{\color{DarkLila}{#1}}}
\newcommand{\Beige}[1]{{\color{Beige}{#1}}}
\newcommand{\Gold}[1]{{\color{Gold}{#1}}}
\newcommand{\Goldb}[1]{{\color{Goldb}{#1}}}

\newcommand{\MyYellow}[1]{{\color{MyYellow}{#1}}}


\newcommand{\Red}[1]{{\color{red}{#1}}}
\newcommand{\Gray}[1]{{\color{gray}{#1}}}
\newcommand{\Black}[1]{{\color{black}{#1}}}

\newcommand{\lan}{\langle}
\newcommand{\ran}{\rangle}
\newcommand{\lgae}{$\la$-$\Gamma \E$}
\newcommand{\lgg}{\mathfrak{g}}
\newcommand{\lgc}{\mathfrak{g}_{\cz}}

\def\ch{{\cal H}}
\def\cm{{\cal M}}
\def\II{\mathbb I}
\def\cb{{\cal B}}
\def\cn{{\cal N}}
\def\ce{{\cal E}}
\def\cf{{\cal F}}
\def\cg{{\cal G}}
\def\ca{{\cal A}}
\def\cv{{\cal V}}
\def\cc{{\cal C}}
\def\cp{{\cal P}}
\def\calr{{\cal R}}
\def\mod{\,\, {\rm mod}\,\,}
\def\ma{\mathbf{a}}
\def\mx{\mathbf{x}}
\def\mapr{\mathbf{a'}}
\def\dom{\operatorname{dom}}

\def\CC{\mathbb{C}}
\def\RR{\mathbb{R}}
\def\ZZ{\mathbb{Z}}
\def\QQ{\mathbb{Q}}
\def\NN{\mathbb{N}}
\def\Mm{\mathcal{M}}

\def\LL{\mathbb{L}}
\def\nbits{\mathbb{Z}_2^n}
\def\11{\mathbb{I}}
\def\LL{\mathcal{L}}
\def\PP{\mathbb{P}}
\def\EE{\mathbb{E}}
\def\RRR{\mathcal{R}}

\def\h{\mathfrak h}
\def\ind{1\hspace{-0.27em}\mathrm{l}}

\newcommand{\half}{\mbox{$\textstyle \frac{1}{2}$} }
\newcommand{\proj}[1]{\ket{#1}\!\bra{#1}}
\newcommand{\outerp}[2]{\ket{#1}\!\bra{#2}}
\newcommand{\inner}[2]{ \langle #1 | #2 \rangle}
\newcommand{\melement}[2]{ \langle #1 | #2 | #1 \rangle}

\newcommand{\vc}[1]{\underline{#1}}

\newcommand{\mmatrix}[4]{\left( \begin{array}{cc} #1 & #2 \\ #3 &
		#4 \end{array} \right)}
\newcommand{\vvec}[2]{\left( \begin{array}{cc} #1  \\ #2 \end{array} \right)}
\newcommand{\sign}[1]{{\rm sign}(#1)}

\newcommand{\acc}{{I_{acc}}}
\renewcommand{\prob}{{\rm prob}}
\newcommand{\up}{{\underline{p}}}
\newcommand{\ut}{{\underline{t}}}
\newcommand{\bP}{{\overline{P}}}

\newcommand{\bfi}{{\textbf{i}}}
\newcommand{\bfj}{{\textbf{j}}}
\newcommand{\bfk}{{\textbf{k}}}
\newcommand{\bfx}{{\textbf{x}}}

%\usepackage{pst-node}
\usepackage{tikz-cd}
\newcommand{\EP}{\operatorname{EP}}
\newcommand{\Op}{\operatorname{Op}}
\newcommand{\Ri}{\operatorname{Ricci}}
\newcommand{\Av}{\operatorname{Ricci}}

\newcommand{\CLSII}{\CLSI}
\newcommand{\mb}{\mathbb}
\newcommand{\mc}{\mathcal}
\usepackage{graphicx}
%\usepackage{setspace}
%\usepackage{verbatim}
%\usepackage{subfig}

%\newcommand{\EP}{\operatorname{EP}}
%\newcommand{\GNS}{\operatorname{GNS}}

\DeclareRobustCommand\openone{\leavevmode\hbox{\small1\normalsize\kern-.33em1}}
\newcommand{\identity}{I}
\renewcommand{\id}{\rm{id}}
\renewcommand{\be}{\begin{equation}}
	\renewcommand{\ee}{\end{equation}}
\newcommand{\bea}{\begin{eqnarray}}
	\newcommand{\eea}{\end{eqnarray}}
\newcommand{\beas}{\begin{eqnarray*}}
	\newcommand{\eeas}{\end{eqnarray*}}
\def\cDH{\cD(\cH)}
\def\cPH{\cP(\cH)}

%\setcounter{Maxaffil}{1}

%%%%%%%%%%%%%%%%%%%%%%%%%
%\apptocmd{\thebibliography}{\fontsize{8}{13}\selectfont}{}{}%
%\onehalfspace
\usepackage{amsmath}
\usepackage{ams math}
\usepackage{amssymb}
\usepackage{amsthm}
\usepackage{dsfont}
\usepackage{upgreek}
%\usepackage{scalerel}
\usepackage{enumitem}

\usepackage{array}
\usepackage{ams fonts}
%\usepackage{tikz-cd}
\usepackage{graphics}
%\DeclareMathOperator{\Hessian}{Hess}
%\usepackage{eps fig}
%\usepackage{tikz}
%\usepackage{pdfpages}
\usepackage[utf8]{inputenc}
%\usepackage[english]{babel}
%\usepackage{comment}
%\newtheorem{theorem}{Theorem}[section]
\newtheorem*{theorem*}{Theorem}
\newtheorem*{remark*}{Remark}
\newtheorem*{lemma*}{Lemma}
\newtheorem*{notation*}{Notation}
\newtheorem*{cor*}{Corollary}
\newtheorem*{note*}{Note}
\newtheorem*{prop*}{Proposition}
\newtheorem*{example*}{Example}
%\newtheorem{lemma}[theorem]{Lemma}
%\newtheorem{definition}[theorem]{Definition}
%\newtheorem{remark}[theorem]{Remark}
%\newtheorem{prop}[theorem]{Proposition}
%\newtheorem{example}[theorem]{Example}
%\newtheorem{corollary}[theorem]{Corollary}
\newcommand{\RM}{\mathbb{R}}
\renewcommand{\Ric}{\mbox{Ric}}

\newcommand{\Rc}{\mbox{Rc}}
\newcommand{\Rcu}{\ensuremath{\Rc_{\text{\fontsize{2.5}{4}\selectfont {U}}}}}
\newcommand{\Du}{\ensuremath{\Delta_{\text{\fontsize{2.5}{4}\selectfont {U}}}}}
\newcommand{\Dv}{\ensuremath{\Delta_{\text{\fontsize{2.5}{4}\selectfont {V}}}}}
\newcommand{\du}{\ensuremath{d_{\text{\fontsize{2.5}{4}\selectfont {U}}}}}
\newcommand{\Lu}{\ensuremath{L_{\text{\fontsize{2.5}{4}\selectfont {U}}}}}
\newcommand{\Zu}{\ensuremath{Z_{\text{\fontsize{2.5}{4}\selectfont {U}}}}}
\newcommand{\Zv}{\ensuremath{Z_{\text{\fontsize{2.5}{4}\selectfont {V}}}}}
\newcommand{\Hess}{\mbox{Hess}}
\renewcommand{\R}{\mbox{R}}
\newcommand{\DN}{\ensuremath{D_{\mathcal{N}_{\fix}} }}
\renewcommand{\T}{\mbox{T}}
\newcommand{\M}{\ensuremath{\mathds{M}}}
\newcommand{\ad}{\mbox{ad}}
\newcommand{\op}{\mbox{\tiny{op}}}
\newcommand{\trans}{\mbox{\scalebox{.5}{trans}}}
\newcommand{\End}{\mbox{End}}
\renewcommand{\dom}{\mbox{dom}}
\newcommand{\Div}{\mbox{div}}
\newcommand{\Spin}{\mbox{\tiny{Spin}}}
\newcommand{\Spinn}{\mbox{Spin}}
\newcommand{\triple}{\ensuremath{(\mathcal{N}\ssubset \mathcal{M},\tau,\delta)}}
\newcommand{\tripleo}{\ensuremath{(\mathcal{N}_1\ssubset \mathcal{M}_1,\tau_1,\delta_1)}}
\newcommand{\triplet}{\ensuremath{(\mathcal{N}_2\ssubset \mathcal{M}_2,\tau_2,\delta_2)}}
\renewcommand{\ltriple}{\ensuremath{(\tilde{\mathcal{N}}\ssubset \tilde{\mathcal{M}},\tau,\delta)}}
\newcommand{\triplei}{\ensuremath{(M\subset \hat{M},\tau_{1},\delta)}}
\newcommand{\tripleii}{\ensuremath{(M\subset \hat{M},\tau_{2},\delta)}}
\newcommand{\QED}{\hfill\ensuremath{\blacksquare}}%
\newcommand{\Rcb}{\ensuremath{Rc_{\text{\tiny{B}}}}}
\newcommand{\bott}{\ensuremath{^{\text{\tiny{B}}}\nabla}}



\DeclareMathOperator*{\esssup}{ess\,sup}
\DeclareMathOperator*{\essinf}{ess\,inf}

\DeclareMathOperator{\Ep}{Ep}

%\usepackage{geometry}
%\usepackage{listings}
%\usepackage{dsfont}
\usepackage{tikz-cd}
\usepackage{amsthm}

%\usepackage[backend=biber,style=alphabetic,sorting=ynt]{biblatex}
%%%%%%%%%%%%%%%%

\usepackage[margin=1.0 in]{geometry}

\title{Resource-Dependent Complexity of Quantum Channels}
\begin{document}

	\author[R. Araiza]{Roy Araiza}
	\address{Department of Mathematics, University of Illinois at Urbana-Champaign}
	\email{raraiza@illinois.edu}
	\author[Y. Chen]{Yidong Chen}
	\address{Department of Physics, University of Illinois at Urbana-Champaign}
	\email{yidongc2@illinois.edu}
	\author[M. Junge]{Marius Junge}
	\address{Department of Mathematics, University of Illinois at Urbana-Champaign}
	\email{mjunge@illinois.edu}
	\author[P. Wu]{Peixue Wu}
	\address{Department of Mathematics, University of Illinois at Urbana-Champaign}
	\email{peixuew2@illinois.edu}
	
	\thanks{RA was supported as a JL. Doob research assistant professor}
	
	\thanks{MJ was partially supported by NSF Grant DMS 2247114, NSF Grant DMS 1800872 and NSF RAISE-TAQS 1839177}


	\maketitle
	\begin{abstract}
		Quantum complexity theory is concerned with the amount of elementary quantum resources needed to build a quantum system or a quantum operation. The fundamental question in quantum complexity is to define and quantify suitable complexity measures. This non-trivial question has attracted the attention of quantum information scientists, computer scientists, and high energy physicists alike. In this paper, we combine the approach in \cite{LBKJL} and well-established tools from noncommutative geometry \cite{AC, MR, CS} to propose a unified framework for \textit{resource-dependent complexity measures of general quantum channels}. This framework is suitable to study the complexity of both open and closed quantum systems. We explore the mathematical consequences of the proposed axioms. The central class of examples in this paper is the so-called \textit{Lipschitz complexity} \cite{LBKJL, PMTL}. We use geometric methods to provide upper and lower bounds on this class of complexity measures \cite{N1,N2,N3}. Finally, we study the dynamics of Lipschitz complexity in open quantum systems. In particular, we show that generically the Lipschitz complexity grows linearly in time and then saturates at a maximum value. This is the same qualitative behavior conjecture by Brown and Susskind \cite{BS1, BS2}.
	\end{abstract}
	\tableofcontents
	\section{Introduction}\label{section:intro}
	Quantum complexity is a fundamental concept in quantum information and quantum computation theory. It is a generalization of classical computational complexity and characterizes the inherent difficulty of various quantum information tasks. Compared with classical complexity, quantum complexity is much more difficult to quantify and measure. By now, there are numerous inequivalent quantum complexity measures \cite{ OT, RW, JW, FR, PD, KH, BT, BV, BASST, BGS, BSh}. On one hand, quantum information specialists primarily focus on the quantum circuit complexity, which emphasizes on the number of elementary gates required to simulate a complex unitary transformation \cite{N1, N2, N3, LBKJL, HFKEH, FGHZ}. On the other hand, high energy physicists have adapted the notion of complexity to systems with infinite degrees of freedom \cite{JM, HR, BS1, BS2, MA} and to chaotic quantum systems \cite{PCASA, DS1, ADS, RSSS1, RSSS2}. 
	
	From information theoretic perspective, any computational task can be characterized as an information channel. This is certainly the case in classical computers, where different logical and memory components communicate with each other and accomplish complex computational tasks in coordination. Therefore, when studying quantum complexity measures, it is reasonable to focus on the complexity of general quantum channels. In addition, any reasonable complexity measure must take into consideration the amount of available computational resources. Plainly, if given infinite resources (e.g. all possible unitary gates), then any computational task would have low complexity. The resources for quantum computation typically come in the form of a set of simple unitary gates or a set of simple infinitesimal generators that can drive the time evolution of a quantum system. When considering time evolution in this paper, we do not restrict to the unitary evolution of closed quantum systems. Since any realistic quantum system must interact with the environment, it is more natural to consider time evolution of open quantum systems. 

	With these considerations in mind and following the axiomatic approach in \cite{LBKJL}, we propose a set of general axioms that any resource-dependent complexity measure of quantum channels should satisfy. The main novelty of our axiomatic framework is a combination of complexity theory \cite{LBKJL} and well-established operator algebraic tools \cite{AC, MR, CS, Fisher, CM1, CM2, Wir2}. Our proposed axioms do not require the resources to have a particular form. Therefore it provides a unified framework that can be applied to unitary gate sets and infinitesmial generators. In addition, the axioms consider general quantum channels and hence are suitable to study the complexity of time evolution of both closed and open systems. Finally, our axioms can also be used to study the dynamics of complexity measures. 
	
	The main example of complexity measure in this paper is the so-called Lipschitz complexity \cite{LBKJL, PMTL}. This is a complexity measure based on a noncommutative analog of the classical Lipschitz norm \cite{CM1, CM2, Wir2, Fisher}. It is well-known that the Wasserstein 1-norm is dual to the Lipschitz norm. Our definition of the Lipschitz norm is based on noncommutative geometry \cite{AC, MR}. And it is a generalization of the existing notion \cite{PMTL}. Since the noncommutative Lipschitz norm (or dually the Wasserstein norm) studies the noncommutative differential structure on the state space (or dually the observables), the Lipschitz complexity naturally adopts the geometric perspective on quantum complexity first introduced by Nielsen \cite{N1, N2, N3}. Depending on the applications, we will introduce several different notions of noncommutative Lipschitz norms. Each definition leads to a different quantum complexity measure satisfying the proposed axioms. 
	
	One of the central problems in quantum complexity theory is to provide tight estimates on the complexity measures. Typically a lower bound on the complexity is more informative for the purpose of classification. In this paper, we will provide various upper and lower bounds for the Lipschitz complexity. The upper bound estimates are typically derived from geometric arguments similar to the original Nielsen's estimates \cite{N1, N2, N3}, while the lower bound estimate relies on the noncommutative return time estimate and a notion of size of the Lipschitz space. In the noncommutative setting, it is simply the Lipschitz complexity of the conditional expectation onto the fixed point of a quantum channel. Since the conditional expectation itself is a special type of quantum channel, we will have better control over its Lipschitz complexity. And by relating the Lipschitz complexity of general channels to that of the conditional expectations, we will be able to bound the complexity of general channeles. 
	
	As mentioned above, our approach to quantum complexity is suitable to study the evolution of complexity with time. Given a time-dependent quantum channel (e.g. a quantum Markov semigroup), we can study the time evolution of its Lipschitz complexity. This is closely related to the Brown-Susskind conjecture \cite{BS1, BS2, HFKEH}. The conjecture states that for a generic quantum circuit, its circuit complexity grows linearly with time (measured in terms of the depth of the circuit and the number of gates used in the circuit) and the complexity saturates at a value that is exponential in the system size \cite{BS1, BS2}. This conjecture was proved for random quantum circuits \cite{HFKEH}. We will show that, for a quantum Markov semigroup, the Lipschitz complexity first grows linearly in time and then saturates at a maximal value. In addition, the time scale to saturate the maximum can be calculated by the return time of the quantum Markov semigroup. Qualitatively, this is the same behavior as conjectured by Brown and Susskind \cite{BS1,BS2}. Our result suggests that for any generic time-dependent quantum channel (not necessarily given by a quantum circuit), its complexity should have the qualitative behavior of the Brown-Susskind type. 
	
	\section{Axioms of Resource-Dependent Complexity Measures of Quantum Channels}\label{section:axiom}
	In this section, we present the axioms of resource-dependent complexity measures of quantum channels. We state the axioms for finite-dimensional quantum systems. But it is possible to generalize these axioms to any finite von Neumann algebras as in \cite{Fisher}.
	
	Let $\mc H$ be a Hilbert space and  let $\mb B(\mc H)$ be the bounded operators on $\mc H$. Suppose $\mc N \subset \mb B(\mc H)$ is a finite dimensional von Neumann algebra. Denote the canonical trace on $\mc N$ by $\tr$. Recall that a quantum channel is a normal, trace-preserving, completely positive map: 
	\begin{align*}
	\Phi: \mc N_* \to \mc N_*,
	\end{align*}
where $\mc N_* \cong L_1(\mc N)$ is the space of normal density functionals.	

    The dual map $\Phi^*: \mc N \to \mc N$ defined via $\tr(\Phi^*(\rho)x) = \tr(\rho \Phi(x))$
for any $\rho \in L_1(\mc N), x\in \mc N$, is a normal unital completely positive map.	
	
    As mentioned in the introduction, typically the resources for quantum computations can be characterized by a set of simple unitaries or a set of simple infinitesimal generators. Heuristically, the goal of resource-dependent complexity measure is to quantify the amount of resources needed to construct a given quantum channel. Mathematically, motivated by \cite{LBKJL, HR, PMTL} our goal is to find an appropriate complexity function $C$ defined on quantum channels $$C: CP(\mc N)\to \mb R_{\ge 0},$$
    where $CP(\mc N)$ is the set of all the quantum channels on $\mc N$,
	such that it satisfies some of the following axioms: 
	\begin{axiom}
\begin{enumerate}
\item $C(\Phi) = 0$ if and only if $\Phi = \id$. 
\item (\textit{subadditivity under concatenation}) $C(\Phi \Psi) \le C(\Phi)+C(\Psi)$. 
\item (\textit{Convexity}) For any probability distribution $\{p_i\}_{i\in I}$ and any channels $\{\Phi_i\}_{i \in I}$, we have 
\begin{equation}
C(\sum_{i\in I} p_i\Phi_i) \le \sum_{i\in I} p_iC(\Phi_i).
\end{equation} 
\item (\textit{Tensor additivity}) For finitely many channels $\{\Phi_i\}_{1\le i \le m}$, we have 
\begin{equation}
C(\bigotimes_{i=1}^m \Phi_i) = \sum_{i=1}^m C(\Phi_i).
\end{equation}
\item (\textit{Normalization}) $C(\Phi) \le \log d$, where $d$ is the dimension of the underlying Hilbert space. 
\end{enumerate}	
	\end{axiom}
	The physical intuition behind these axioms is clear. The first clause axiomatizes the fact that the trivial identity channel should have no complexity. The subadditivity clause axiomatizes the fact that, when building complex channels from simpler ones, the overall complexity cannot be larger than the sum of the individual complexities. In other words, one cannot create complexity out of thin air. On the other hand, a decrease in complexity should be allowed. For example, when concatenating two unitary channels $Ad_u$ and $Ad_{u^*}$, the product is the identity channel. Such cancelation occurs generically. The convexity clause is another manifestation of the principle that complexity cannot be created out of thin air. Hence a statistical ensemble of channels cannot create more complexity than the statistical mean of the individual complexities. Tensor additivity clause is the fact that the complexity of building independent quantum channels should be the same as the sum of complexity of building each channel. Finally, the normalization clause 
gives us a limit based on the number of qubits we have.	
	To propose such a complexity function, motivated by \cite{LBKJL}, we will introduce a \textit{Lipschitz norm}, which is widely used in \cite{AC, MR, CS, Fisher, CM1, CM2, Wir2} and fruitful theories were built there.
	
	Recall that a derivation triple is given by $(\mc A, D, \mc H)$, where $D: \mc A \to \mc M$ where $\mc M$ is a larger von Neumann algebra, is a non-commutative differential operator and $\mc A \subset \mb B(\mc H)$ is a $*$-subalgebra such that $\|D(a)\|<\infty$ for any $a \in \mc A$. Typically, $D(x) = [d,x]$ where $d$ is a self-adjoint operator. 
	
	
	
	
	
%------------------	
	\begin{comment}
	In this section, we present the axioms of resource-dependent complexity measures of quantum channels. We state the axioms for finite-dimensional quantum systems. But it is possible to generalize these axioms to any finite von Neumann algebras \cite{Fisher}.
	
	Let $N$ be a finite-dimensional von Neumann algebra with the canonical trace $\tr$. Recall a quantum channel is a normal unital completely positive map:
	\begin{equation}
		\Phi:N\rightarrow N\pl.
	\end{equation}
	The dual channel is a normal trace-preserving completely positive map:
	\begin{equation}
		\Phi^*:L_1(N)\rightarrow L_1(N)
	\end{equation}
	such that $\tr(\Phi^*(\rho)x) = \tr(\rho \Phi(x))$ for any $\rho \in L_1(N), x\in N$. Here $L_1(N)\cong N_*$ is the space of normal functionals on $N$ and the space of normalized states on $N$ is given by: $S(N):= \{\rho\in L_1(N): \rho \geq 0, \tr(\rho) = 1\}$. 
	
	As mentioned in the introduction, typically the resources for quantum computations can be characterized by a set of simple unitaries or a set of simple infinitesimal generators. Heuristically, the goal of resource-dependent complexity measure is to quantify the amount of resources needed to construct a given quantum channel. Mathematically, there exists a unified framework to describe both types of resources (unitaries and infinitesimal generators) because both types of resources naturally provide a noncommutative differential structure on the algebra $N$ \cite{CS, Fisher}. 
	\begin{definition}[Noncommutative Lipschitz norms]\label{definition:Lipschitz}
		Let $N$ be a finite-dimensional von Neumann algebra, and let $\mathcal{S} = \{\partial_j: N\rightarrow M\}_{1\leq j\leq n}$ be a finite set of derivations on $N$. Here $M$ is another finite-dimensional von Neumann algebra which is the noncommutative analog of tangent spaces \cite{AC, MR, CS, Fisher, CM1, CM2}. Then on the orthogonal subspace $\mathcal{A}:= \{x\in N: \tr(x^*y) = 0 \text{ for all }y \text{ such that }\partial_j(y) = 0, \forall j\}$the noncommutative $\ell_\infty$-Lipschitz norm is defined by:
		\begin{equation}
			|||x|||_\infty :=  \sup_{j}||\partial_jx||_\infty\pl.
		\end{equation}
		Analogously, the noncommutative $\ell_2$-Lipschitz norm is defined by:
		\begin{equation}
			|||x|||_2 := ||\big(\sum_j|\partial_j x|^2\big)^{1/2}||_\infty\pl.
		\end{equation}
		The subspace $\mathcal{A}$ is called the Lipschitz space of the derivation set $\mathcal{S}$.
	\end{definition}
	Given a finite set of unitaries $\mathcal{U} = \{u_j\in U(N)\}_{1\leq j \leq n}$ where $U(N)$ is the group of unitaries in $N$, we define the set of derivations:
	\begin{equation}
		\partial^\mathcal{U}_j: N\rightarrow N: x\mapsto [u_j, x] = u_jx - xu_j\pl.
	\end{equation}
	The corresponding $\ell_\infty$-Lipschitz norm can be written as: $|||x|||_\infty = \sup_j ||[u_j,x]||_\infty = \sup_j||u_jxu_j^* - x||_\infty$. Written in this form, we can generalize the Lipschitz norm to any finite set of normal automorphisms:
	\begin{definition}[Lipschitz norms of automorphisms]\label{definition:LipAuto}
		Using the same notation as Definition \ref{definition:Lipschitz}, let $\mathcal{G}:=\{\alpha_j\in Aut(N)\}_{1\leq j \leq n}$ be a finite set of normal automorphisms on $N$. Then the associated $\ell_\infty$-Lipschitz norm is given by:
		\begin{equation}
			|||x|||_\infty:= \sup_j ||\alpha_j(x) - x||_\infty\pl.
		\end{equation}
		The associated $\ell_2$-Lipschitz norm is given by:
		\begin{equation}
			|||x|||_2 := ||\big(\sum_j|\alpha_j(x) - x|^2\big)^{1/2}||\pl.
		\end{equation}
	\end{definition}
	Similarly, given a finite set of infinitesimals (self-adjoint elements in $N$) $\Delta = \{a_j \in N_{s.a.}\}_{1\leq j \leq n}$, we define the set of derivations:
	\begin{equation}
		\partial^\Delta_j:N\rightarrow N: x\mapsto [a_j, x] = a_jx - xa_j\pl.
	\end{equation}
	The corresponding $\ell_\infty$-Lipschitz norm can be written as:
	\begin{equation}
		|||x|||_\infty = \sup_j||[a_j,x]|| = \sup_j ||[a_j,x]^*[a_j,x]||^{1/2}\pl.
	\end{equation}
	And $\ell_2$-Lipschitz norm can be written as:
	\begin{equation}
		|||x|||_2 = ||\big(\sum_j [a_j,x]^*[a_j,x]\big)^{1/2}||
	\end{equation}
	Such Lipschitz norm naturally shows up in the study of (symmetric) quantum Markov semigroups \cite{Fisher, CM1, CM2, Wir2}. Recall a self-adjoint Lindbladian $L$ on a finite-dimensional von Neumann algebra $N$ is the generator of a (symmetric) quantum Markov semigroup $\{\exp(t\mathcal{L})\}$. In addition, a self-adjoint Lindbladian has the canonical form:
	\begin{equation}
		Lx = \sum_{1\leq j \leq n} a_jxa_j - \frac{1}{2}(a_j^2x + xa_j^2) = -\frac{1}{2}\sum_{1\leq j \leq n}[a_j, [a_j, x]]
	\end{equation}
	where $a_j^* = a_j$. The noncommutative gradient form of $L$ is given by:
	\begin{equation}
		\Gamma_L(x,y):= L(x^*y) - x^*L(y) - L(x^*)y = \sum_{1\leq j \leq n}[a_j,x]^*[a_j,y]\pl.
	\end{equation}
	Therefore the $\ell_2$-Lipschitz norm associated with the set of derivations $\Delta_L := \{\partial_j:N\rightarrow N: x\mapsto [a_j, x]\}_{1\leq j \leq n}$ can be written as:
	\begin{equation}
		|||x|||_2  = ||\Gamma_L(x,x)^{1/2}||\pl.
	\end{equation}
	Typically, we want $x$ and its adjoint $x^*$ to have the same Lipschitz norm. This motives the slightly modified definition of $\ell_2$-Lipschitz norm associated with the self-adjoint Lindbladian $L$:
	\begin{equation}
		|||x|||_L := \max\{||\Gamma_L(x,x)^{1/2}, \Gamma_L(x^*, x^*)^{1/2}\}\pl.
	\end{equation}
	Given the definition of Lipschitz norm, we can define the noncommutative Wasserstein 1-norm as its dual:
	\begin{definition}[Noncommutative Wasserstein 1-norm]\label{definition:wasserstein}
		Let $(\mathcal{A},|||\cdot|||)\ssubset N$ be the Lipschitz space in $N$ where $|||\cdot|||$ is a Lipschitz norm. Then the dual norm on $L_1(N)$ is the noncommutative Wasserstein 1-norm:
		\begin{equation}
			||\rho||_W := \sup\{|\tr(\rho x)|: x\in \mathcal{A}, |||x|||\leq 1\}\pl.
		\end{equation}
	\end{definition}
	\begin{remark}
		When restricted to the state space $S(N)$, the Wasserstein 1-norm \ref{definition:wasserstein} gives $S(N)$ the structure of a metric space. Moreover, if $N = M_{2^n}(\mathbb{C}) = Cl_n(\mathbb{C})$ is the Clifford algebra generated by $n$ generators $\{Q_j\}_{1\leq j \leq n}$, the Wasserstein metric dual to the Lipschitz norm associated with the set of derivations: $\{[Q_j, \cdot]\}_{1\leq j \leq n}$ is equivalent to the Wasserstein metric defined in \cite{PMTL}.
	\end{remark}
	\end{comment}
	
%----------------------	
	
	
		
	\subsection{Single-System Resource-Dependent Complexity}\label{subsection:singlesystemComplexity}
	We are ready to introduce the resource-dependent complexity function for single system.  The resources are nothing but a set of operators. To be specific, suppose $S \subset \mc N$ is a subset containing identity, the induced Lipschitz (semi-)norm is defined by 
	\begin{equation}
	|||f|||_S:= \sup_{s \in S}\|[s,f]\|_{\infty},\ f\in \mc N.
	\end{equation}
	To make $|||f|||_S$ a true norm, we define the space of mean 0 Lipschitz space $\mc A$ as the quotient space $$\mc A:= \mc N / \{f \in \mc N: |||f|||_S = 0\}.$$ 
	Recall the commutant of $S$ is given by 
	\begin{equation}
	S'= \{f \in \mc N: [s,f]=0, \ \forall s\in S\}.
	\end{equation}
	Denote the trace-preserving conditional expectation onto $S'$ as $E_{S'}: \mc N \to S'$. Then it is easy to see that 
	\begin{equation}
	\mc A \cong \{f \in \mc N: E_{S'}(f)=0\}.
	\end{equation}
	In other word, $\mc A$ is orthogonal to the algebra $S'$. The definition of complexity for single system is given as follows:
	\begin{definition}\label{complexity:first}
	For a quantum channel $\Phi: \mc N_* \to \mc N_*$, the complexity is defined as 
	\begin{equation}
	C_S(\Phi):= \|\Phi^* - \id: (\mc A, |||\cdot|||_S) \to \mb B(\mc H)\|.
	\end{equation}
	\end{definition}	
\noindent We denote the above norm by $\|\phi^* - \id\|_{Lip \to \infty}$ for simplicity. The elementary properties are summarized in the following lemma:
	\begin{lemma}
	\begin{enumerate}
	\item $C_S(\Phi)=0$ if and only if $\Phi = \id$. 
	\item $C_S(\Phi \Psi) \le C_S(\Phi)+C_S(\Psi)$ for any quantum channels $\Phi,\Psi$.
	\item For any probability distribution $\{p_i\}_{i\in I}$ and any channels $\{\Phi_i\}_{i \in I}$, we have 
\begin{equation}
C_S(\sum_{i\in I} p_i\Phi_i) \le \sum_{i\in I} p_iC_S(\Phi_i).
\end{equation} 
    \item $C_S(\Phi) \le C_S(E_{S'}) \|\Phi^*-\id : \mc N \to \mc N\|$.
	\end{enumerate}
	\end{lemma}
	\begin{proof}
	Property $(1)$ follows from the faithfulness of the norm and property $(3)$ follows from the triangle inequality of norm. We only need to show $(2),(4)$. To show $(2)$, note that \begin{align*}
			\begin{split}
				C_{S}(\Phi\circ\Psi) &= ||\Psi^*(\Phi^*-id) + \Psi^* - id||_{Lip \to \infty} \leq ||\Psi^*||_{\infty\rightarrow\infty}||\Phi^*-id||_{Lip\rightarrow\infty} + ||\Psi^*-id||_{Lip\rightarrow\infty} \\&\leq C_{S}(\Psi) + C_{S}(\Phi)
			\end{split}
		\end{align*}
		where for the first inequality, we use the following diagram and submultiplicativity of norms:
		\begin{center}
\begin{tikzcd}
(\mc A, |||\cdot|||_{S}) \arrow[r, "\Phi^*-id"]
 \arrow[dr,"\Psi^*(\Phi^*-id)" description]
& \mc N \arrow[d, "\Psi^*"]\\
& \mc N
\end{tikzcd}
\end{center}			
		and for the last inequality, we used the fact that a unital completely positive map $\Psi^*$ satisfies $||\Psi^*||_{\infty \to \infty}\leq ||\Psi^*||_{cb} \leq 1$. To show $(4)$, we use the following diagram and submultiplicativity of norms:
\begin{center}
\begin{tikzcd}
(\mc A, |||\cdot|||_{S}) \arrow[r, "\Phi^*-id"]
 \arrow[dr,"\id" description]
& \mc N \\
& \mc N \arrow[u,"\Phi^*-id" description]
\end{tikzcd}
\end{center}	
\begin{align*}
||\Phi^*-id||_{Lip\rightarrow\infty} & \le ||id||_{Lip\rightarrow\infty} ||\Phi^*-id||_{\infty\rightarrow\infty} \\
& = ||id - E_{S'}^*||_{Lip\rightarrow\infty} ||\Phi^*-id||_{\infty\rightarrow\infty} = C_S(E_{S'})||\Phi^*-id||_{\infty\rightarrow\infty},
\end{align*}   
where for the second last equality, we used the fact that for any $x \in \mc A, E_{S'}(x)=0$.   \end{proof}
The effects of the environment on quantum systems are usually taken into consideration, which means we need to allow our system to couple with an additional system. Next we define a complete version of the complexity, or \textit{correlation assisted} version. 

Given a resource set $S \subset \mc N$, the $n$-th amplification of the Lipschitz norm is defined as
\begin{equation}
|||f|||^n_S = \sup_{s\in S}\|[I_n \otimes s, f]\|, f\in \mb M_n(\mc A).
\end{equation}
This enables us to define complete complexity:
\begin{definition}
The complete complexity of a quantum channel $\Phi$ is defined by 
\begin{equation}
C_S^{cb}(\Phi):=\sup_{n\ge 1}\|\id_n\otimes (\Phi^* - \id):(\mb M_n(\mc A), |||\cdot|||^n_S) \to \mb M_n(\mb B(\mc H))\|.
\end{equation}
\end{definition}
All the above elementary properties hold true by the same argument. We summarize the results as follows:
	\begin{lemma}\label{elementary:cb}
	\begin{enumerate}
	\item $C_S^{cb}(\Phi)=0$ if and only if $\Phi = \id$. 
	\item $C_S^{cb}(\Phi \Psi) \le C_S^{cb}(\Phi)+C_S^{cb}(\Psi)$ for any quantum channels $\Phi,\Psi$.
	\item For any probability distribution $\{p_i\}_{i\in I}$ and any channels $\{\Phi_i\}_{i \in I}$, we have 
\begin{equation}
C_S^{cb}(\sum_{i\in I} p_i\Phi_i) \le \sum_{i\in I} p_iC_S^{cb}(\Phi_i).
\end{equation} 
    \item $C_S^{cb}(\Phi) \le C_S^{cb}(E_{S'}) \|\Phi-\id\|_{\diamond}$.
	\end{enumerate}
	\end{lemma}	
	Before ending this subsection, we emphasize that it is not possible to construct all possible channels out of the given resources $S$. Thus the class of channels must be restricted given the resources. We denote $CP_S(\mc N)$ as the subset of all quantum channels $CP(\mc N)$,  which can be constructed from the resource $S$.  In this paper, we focus on two distinct resource-dependent channel sets:
	\begin{enumerate}
		\item(Discrete case) The resource set $\mathcal{U} = \{u_i\in U(\mc H)\}_{1\leq i \leq 2n+1}$ is a symmetric set of unitaries in $\mc N$ \footnote{Here symmetric means that if $u\in \mathcal{U}$, then $u^*\in \mathcal{U}$. In addition, we always assume that $id\in\mathcal{U}$}. Then
		\begin{align}\label{equation:singlesystemUnitary}
			\begin{split}
				&CP_\mathcal{U}(\mc N):=\{\Phi\in CP(\mc N): \exists \{u_{j_i}\in \mathcal{U}\}_{1\leq i \leq N}\text{ such that }\\&\Phi \text{ is a convex combination of products of unitary channels } \prod_{1\leq i \leq N}Ad_{u_{j_i}}\}\pl.
			\end{split}
		\end{align}
		In other words, the channel set depending on $\mathcal{U}$ contains all channels that can be built from finite iterations of the simple unitary gates. 
		\item(Continuous case) The resource set $\Delta = \{a_i\in \mc N\}_{1\leq i\leq n}$ is a set of elements in $\mc N$ \footnote{In the application, the $a_j$'s are either self-adjoint (related to symmetric Lindbladian) or unitary (related to the decoherence Lindbladian: $Lx = \sum_j a_jxa_j^* - x$)}. They characterize the infinitesimal generators (e.g. Hamiltonians) that can be used to construct complex channels. Then
		\begin{align}\label{equation:singlesystemHamiltonian}
			\begin{split}
				&CP_\Delta(\mc N):=\{\Phi\in CP(\mc N): \exists\{X_i = \sum_{j_i}\alpha_{j_i}b_{j_i}\ \text{self-adjoint}\}_{1\leq i \leq N}\text{ such that }\\&\Phi \text{ is a convex combination of products of the form } \prod_{1\leq i \leq N}Ad_{e^{iX_i}}\}
			\end{split}
		\end{align}
		where the self-adjoint $X_i$ is a linear combination of the $b_{j_i}$'s with coefficients $\alpha_{j_i}$ and each $b_{j_i}$ is a finite iterated brackets of the generators $a_j$'s.
	\end{enumerate}
	As an example, consider the symmetric quantum Markov semigroup $\{\Phi_t = e^{tL}\}$ generated by $Lx = -\frac{1}{2}\sum_{1\leq j \leq n}[a_j,[a_j,x]]$. At each time $t\geq 0$, the channel $\Phi_t$ is in the set $UCP_\Delta(N,N)$ where $\Delta = \{a_j\}_{1\leq j \leq n}$. In general, the discrete case is closely related to the quantum circuits since each quantum circuit is a finite concatenation of unitary channels. On the other hand, the continuous case is closely related to the Trotter approximation of Hamiltonian evolution and quantum Markov semigroups. This set of channels have been considered in the geometric complexity theory of Nielsen \cite{N1,N2,N3}.
\begin{remark}
Consider $\mc N = \mb B(\mb C_2^{\otimes n})$, and define the resource set $$\mc U:= \{X_i,Y_i,Z_i: 1\le i \le n\},$$
where $$X_i = I_2\otimes \cdots \otimes I_2 \otimes\underbrace{X}_{i-th\ position}  \otimes I_2 \otimes \cdots \otimes I_2$$
and $X,Y,Z$ are Pauli operators. Then one can show that our definition of complexity is equivalent up to a dimension-independent constant to the Wasserstein complexity introduced in \cite{LBKJL}. More details are given in Section \ref{section:Clifford}.
\end{remark}
We can also consider some other examples of single-system resource-dependent complexity measure. First, recall that our resource-dependent Lipschitz norm is given by 
\begin{align*}
|||f|||_S = \sup_{s\in S}\|[s,f]\|_{\infty}.
\end{align*}  	
This can be seen as an $l^{\infty}$ Lipschitz norm. We can also consider an $l^2$ Lipschitz norm when $S = \{s_1,\cdots, s_k\}$ is a discrete set:
\begin{equation}
|||f|||_{S,2} = \|(\sum_{i=1}^k|[s_i,f]|^2)^{1/2}\|_{\infty}.
\end{equation}	
We can define complexity using this norm and get some estimates. The discussion will be in later sections.	
	
Another example of complexity is a simple universal construction.
	\begin{definition}\label{definition:maxComplexity}
		Let $\mathcal{U} = \{u_j\in U(\mc H)\}_{1\leq j \leq 2n+1}$ be the resource set of unitaries in $\mc N$ containing $id$. Then the maximum complexity is defined as:
		\begin{equation}
			C_\mathcal{U}^{max}:CP_\mathcal{U}(\mc N)\rightarrow \mathbb{R}_{\geq 0}:\Phi \mapsto \inf_{\mu, \{\mu_i(\omega)\}}\{\int \ell(\omega)d\mu(\omega):\Phi = \int \prod_{\substack{1\leq i \leq \ell(\omega)\\ u_i(\omega)\in \mathcal{U}}}Ad_{u_i(\omega)}d\mu(\omega)\}
		\end{equation}
		where $\mu$ is a probability measure on some standard probability space $(\Omega, \mathcal{F},\mu)$. The infimum is taken over all possible products of simple unitary channels whose convex combination is the channel $\Phi$.
	\end{definition}
	Intuitively, the maximum complexity measures the minimum expected number of simple unitaries in order to construct the channel $\Phi$. For unitary channels $Ad_{u_j}$ where $u_j\in \mathcal{U}$, the complexity $C_\mathcal{U}^{max}(Ad_{u_j}) = 1$. This is intuitively correct, since these channels only require one piece of the resource. In addition, the subadditivity axiom is satisfied by the definition of infimum. And the convexity axiom is satisfied by the construction. It is not difficult to see that all possible $\mathcal{U}$-dependent complexity measure is bounded above by $C_\mathcal{U}^{max}$.	
	
	
	
	
	
	
	
	
%%--------------	
	\begin{comment}
	Given these preparations, we are now ready to state the axioms for resource-dependent complexity measures of general quantum channels. Although our main example will be the Lipschitz complexity, the axioms themselves do not depend on the Lipschitz norm. We first state the set of axioms for single-system channels. 
	\begin{axiom}[Resource-Dependent Quantum Complexity Measure of Single-System Channels]\label{axiom:singlequbit}
		Let $UCP(N,N)$ be the set of quantum channels on $N$ \footnote{Equipped with bounded-weak topology, the set $UCP(N,N)$ is a compact topological space. Hence we can study whether a given complexity measure is continuous with respect to this topology. We shall reserve the problem for a future paper.}. Fix a finite "resource" set $\mathcal{S}\ssubset Der(N)$ where $Der(N)$ is the space of derivations on $N$. Let $UCP_\mathcal{S}(N,N)$ be the set of quantum channels that can be generated using the resources $\mathcal{S}$. Then a $\mathcal{S}$-dependent quantum complexity measure is a function:
		\begin{equation}
			C_\mathcal{S}: UCP_\mathcal{S}(N,N)\rightarrow \mathbb{R}_{\geq 0}
		\end{equation}
		such that:
		\begin{enumerate}
			\item(Normalization) $C_\mathcal{S}(id) = 0$;
			\item(Subadditivity under composition) For any two channels $\Phi,\Psi\in UCP_\mathcal{S}(N,N)$, we have: $C_\mathcal{S}(\Phi\circ\Psi) \leq C_\mathcal{S}(\Phi) + C_\mathcal{S}(\Psi)$;
			\item(Convexity) For any finite set of channels $\{\Phi_i\in UCP_{\mathcal{S}}(N,N)\}_{i\in I}$ and any probability measure $\{p_i\}_{i\in I}$, the complexity measure $C_\mathcal{S}$ is convex:
			\begin{equation}
				C_\mathcal{S}(\sum_{i\in I}p_i\Phi_i) \leq \sum_{i\in I}p_iC_\mathcal{S}(\Phi_i)\pl.
			\end{equation}
		\end{enumerate} 
	\end{axiom}
	The physical intuition behind these axioms is clear. The normalization clause axiomatizes the fact that the trivial identity channel should have no complexity. The subadditivity clause axiomatizes the fact that, when building complex channels from simpler ones, the overall complexity cannot be larger than the sum of the individual complexities. In other words, one cannot create complexity out of thin air. On the other hand, a decrease in complexity should be allowed. For example, when concatenating two unitary channels $Ad_u$ and $Ad_{u^*}$, the product is the identity channel. Such cancelation occurs generically. The convexity clause is another manifestation of the principle that complexity cannot be created out of thin air. Hence a statistical ensemble of channels cannot create more complexity than the statistical mean of the individual complexities. 
	
	In stating the axioms, it is necessary to restrict to the subset of channels $UCP_\mathcal{S}(N,N)$ because it is not always possible to construct all possible channels out of the given resources $\mathcal{S}$. In the extreme case where the resource $\mathcal{S} = \{id\}$ is a single identity channel, one can only construct the trivial identity channel. In this case, it is meaningless to discuss $\mathcal{S}$-dependent complexity measure for other channels. In applications, the definition of $UCP_\mathcal{S}(N,N)$ depends on the resource set $\mathcal{S}$. The precise definition of $UCP_\mathcal{S}(N,N)$ cannot be fixed a priori without the context of $\mathcal{S}$. In this paper, we focus on two distinct cases resource-dependent channel sets:
	\begin{enumerate}
		\item(Discrete case) The resource set $\mathcal{U} = \{u_i\in U(N)\}_{1\leq i \leq 2n+1}$ is a symmetric set of unitaries in $N$ \footnote{Here symmetric means that if $u\in \mathcal{U}$, then $u^*\in \mathcal{U}$. In addition, we always assume that $id\in\mathcal{U}$}. Then
		\begin{align}\label{equation:singlesystemUnitary}
			\begin{split}
				&UCP_\mathcal{U}(N,N):=\{\Phi\in UCP(N,N): \exists \{u_{j_i}\in \mathcal{U}\}_{1\leq i \leq N}\text{ such that }\\&\Phi \text{ is a convex combination of products of unitary channels } \prod_{1\leq i \leq N}Ad_{u_{j_i}}\}\pl.
			\end{split}
		\end{align}
		In other words, the channel set depending on $\mathcal{U}$ contains all channels that can be built from finite iterations of the simple unitary gates. 
		\item(Continuous case) The resource set $\Delta = \{a_i\in N\}_{1\leq i\leq n}$ is a set of elements in $N$ \footnote{In the application, the $a_j$'s are either self-adjoint (related to symmetric Lindbladian) or unitary (related to the decoherence Lindbladian: $Lx = \sum_j a_jxa_j^* - xd$)}. They characterize the infinitesimal generators (e.g. Hamiltonians) that can be used to construct complex channels. Then
		\begin{align}\label{equation:singlesystemHamiltonian}
			\begin{split}
				&UCP_\Delta(N,N):=\{\Phi\in UCP(N,N): \exists\{X_i = \sum_{j_i}\alpha_{j_i}b_{j_i}\in N_{s.a.}\}_{1\leq i \leq N}\text{ such that }\\&\Phi \text{ is a convex combination of products of the form } \prod_{1\leq i \leq N}Ad_{e^{iX_i}}\}
			\end{split}
		\end{align}
		where the self-adjoint $X_i$ is a linear combination of the $b_{j_i}$'s with coefficients $\alpha_{j_i}$ and each $b_{j_i}$ is a finite iterated brackets of the generators $a_j$'s.
	\end{enumerate}
	As an example, consider the symmetric quantum Markov semigroup $\{\Phi_t = e^{tL}\}$ generated by $Lx = -\frac{1}{2}\sum_{1\leq j \leq n}[a_j,[a_j,x]]$. At each time $t\geq 0$, the channel $\Phi_t$ is in the set $UCP_\Delta(N,N)$ where $\Delta = \{a_j\}_{1\leq j \leq n}$. In general, the discrete case is closely related to the quantum circuits since each quantum circuit is a finite concatenation of unitary channels. On the other hand, the continuous case is closely related to the Trotter approximation of Hamiltonian evolution and quantum Markov semigroups. This set of channels have been considered in the geometric complexity theory of Nielsen \cite{N1,N2,N3}.
	\end{comment}
	
%%---------------	
	

	
	\subsection{Composite-System Resource-Dependent Quantum Complexity}\label{subsection:compositesystemComplexity}
	For composite systems $\mb B(\mc H_1) \otimes \mb B(\mc H_2)$, the available resources are given by the union of single-system resources. To be more specific, let $$S_1 \subset \mb B(\mc H_1), S_2 \subset \mb B(\mc H_2),$$ 
   the resource set for the composite system is given by 
   \begin{equation}
   S_1 \vee S_2:= \{s_1\otimes id_{\mc H_2}, id_{\mc H_1}\otimes s_2: s_1\in S_1, s_2\in S_2.\}
   \end{equation}
In general, for $n$-composite system $\mb B(\mc H_1\otimes \cdots \otimes \mc H_n)$, suppose we have resources for each single system $S_i \subset \mb B(\mc H_i), 1\le i \le n$, the resource set for the composite system is given by 
   \begin{equation}
   \vee_{j=1}^n S_j:= \{\widetilde{X_j}\big|X \in S_j,\ \forall 1\le j\le n\}, 
   \end{equation}
where $$\widetilde{X_j} = \id\otimes \cdots \otimes \id \otimes\underbrace{X}_{j-th\ position}  \otimes \id \otimes \cdots \otimes \id.$$
	Then we can define the Lipschitz norm by 
	\begin{equation}
	|||f|||_{\vee_{j=1}^n S_j}= \sup_{\widetilde{X} \in \vee_{j=1}^n S_j}\|[\widetilde{X},f]\|_{\infty},\ \forall f\in \mb B(\mc H_1\otimes \cdots \otimes \mc H_n).
	\end{equation}
    Similar to single system, we need to define the subspace of mean zero elements. Define $\mc A_i:= \{x \in \mb B(\mc H_i): E_{S_i'}(x)=0\}$ and 
    \begin{equation}
    \widetilde{\mc A}:= \bigotimes_{i=1}^n \mc A_i.
\end{equation}    	
	Then it can be shown that $\widetilde{\mc A}= \{x\in \mb B(\mc H_1\otimes \cdots \mc H_n):E_{S_1'}\otimes \cdots \otimes E_{S_n'}(x)= 0\}$ and $|||\cdot|||_{\vee_{j=1}^n S_j}$ is a norm on $\widetilde{\mc A}$. We are ready to give the formal definition of resource-dependent quantum complexity on composite systems: 
	\begin{definition}
	For any quantum channel $\Phi$ on $\mb B(\mc H_1\otimes \cdots \mc H_n)$, with resources $S_i \subset \mb B(\mc H_i)$ individually, the complete complexity of $\Phi$ is defined as 
	\begin{equation}
	\begin{aligned}
	C_{\vee_{j=1}^n S_j}^{cb}(\Phi)& = \|\Phi^*-\id:(\widetilde{\mc A}, |||\cdot|||_{\vee_{j=1}^n S_j}) \to \mb B(\mc H_1\otimes \cdots \mc H_n)\|_{cb} \\
	& = \sup_{m \ge 1}\|\id_m\otimes (\Phi^*-\id): (\mb M_m(\widetilde{\mc A}), |||\cdot|||_{\vee_{j=1}^n S_j}^m) \to \mb M_m(\mb B(\mc H_1\otimes \cdots \mc H_n))\|.
	\end{aligned}
	\end{equation}
	\end{definition} 
The advantage of complete complexity is that the tensor additivity holds: 
\begin{prop}
Suppose for all $1\le j \le n$, $\Phi_j: \mb B(\mc H_j)_* \to \mb B(\mc H_j)_*$ are quantum channels and $S_j \subset \mb B(\mc H_j)$ are the resources for each single system , then 
\begin{equation}
C_{\vee_{j=1}^n S_j}^{cb}(\Phi_1 \otimes \cdots \otimes \Phi_n) = \sum_{j=1}^nC_{S_j}^{cb}(\Phi_j).
\end{equation}
\end{prop}
\begin{proof}
We only need to show the case when $n=2$ because the general case follows from induction. For subadditivity, we will show \begin{align*}
C_{S_1\vee S_2}(\Phi_1 \otimes \Phi_2) \le C_{S_1}(\Phi_1)+C_{S_2}(\Phi_2),
\end{align*}
and the complete version follows by the same argument. For each individual $\Phi_j: \mb B(\mc H_j)_* \to \mb B(\mc H_j)_*$, we have an induced channel on the composite system: $\widetilde{\Phi_j}: \mb B(\mc H_1\otimes \mc H_2)_* \to \mb B(\mc H_1\otimes \mc H_2)_*$, where $\widetilde{\Phi_j}$ acts nontrivially only on the $j$-th subsystem as $\Phi_j$ and trivially on the other subsystems. We claim that 
\begin{equation}
C_{S_1\vee S_2}(\widetilde{\Phi_j}) \le C_{S_j}(\Phi_j),
\end{equation}
which is equivalent to for any $f\in \widetilde{A}$, we have 
$$
\|(\widetilde{\Phi_j}^* - id)f\|_{\infty} \le C_{S_j}(\Phi_j) ||| f |||_{S_1\vee S_2} .
$$
By convexity, we only need to show the inequality for $f\in \widetilde{A}$, $f = f_1 \otimes f_2$. Without loss of generality, assume $j=1$, then
\begin{align*}
\|(\widetilde{\Phi_1}^* - \id )(f) \|_{\infty} & = \|(\Phi_1^*(f_1) - f_1)\otimes f_2\|_{\infty} \\
& = \|\Phi_1^*(f_1) - f_1\|_{\infty}\|f_2\|_{\infty} \\
& \le C_{S_1}(\Phi_1)|||f_1|||_{S_1}\|f_2\|_{\infty} \\
& =  C_{S_1}(\Phi_1) \sup_{X\in S_1}\|[X\otimes \id_{\mc H_2}, f_1\otimes f_2]\|_{\infty} \\
& \le C_{S_1}(\Phi_1)  \sup_{\widetilde{X} \in S_1\vee S_2} \|[\widetilde{X},f]\|_{\infty} = C_{S_1}(\Phi_1)|||f|||_{S_1\vee S_2}.
\end{align*}
Therefore, by subadditivity under concatenation, we have 
\begin{align*}
C_{S_1\vee S_2}(\Phi_1 \otimes \Phi_2) = C_{S_1\vee S_2}(\widetilde{\Phi_1} \circ \widetilde{\Phi_2}) \le C_{S_1\vee S_2}(\widetilde{\Phi_1}) + C_{S_1\vee S_2}(\widetilde{\Phi_2}) \le C_{S_1}(\Phi_1)+C_{S_2}(\Phi_2).
\end{align*}
For the superadditivity, by definition, for any $\varepsilon\in (0,1)$, there exist $m_1, m_2 \in \mb N$, and $f_i \in \mb M_{m_i}(\mb B(\mc H_i))$, $i=1,2$, such that 
\begin{align*}
id_{m_i}\otimes E_{S_i'}(f_i) = 0,\ |||f_i|||_{S_i}^{m_i} \le 1,
\end{align*}
and states $\rho_i \in \mb M_{m_i}(\mb B(\mc H_i))$, such that 
\begin{equation}
(1-\varepsilon)C_{S_i}^{cb}(\Phi_i)\le \tr(((\Phi_i^*)_n - id)(f_i)\rho_i),
\end{equation}
where $(\Phi_i^*)_n$ denotes the $n$-th amplification. Now we define 
\begin{equation}
F = f_1 \otimes I_{m_2}\otimes id_{\mc H_2} + I_{m_1} \otimes id_{\mc H_{1}}\otimes f_2 \in \mb M_{m_1m_2}(\mb B(\mc H_1\otimes \mc H_2)).
\end{equation}
It is routine to check that 
\begin{align*}
& F\in \mb M_{m_1m_2}(\widetilde{A}), i.e., id_{m_1m_2}\otimes E_{S_1'}\otimes E_{S_2'}(F)= 0. \\
& |||F|||_{S_1 \vee S_2}^{m_1m_2} = \max\{\sup_{s_1\in S_1} \|[I_{m_1m_2}\otimes s_1\otimes id_{\mc H_2}, F]\|, \sup_{s_2\in S_2} \|[I_{m_1m_2}\otimes id_{\mc H_1}\otimes s_2, F]\|\} \le 1.
\end{align*}
Moreover, we have 
\begin{align*}
& \tr\big(((\Phi_1^*)_{m_1}\otimes (\Phi_2^*)_{m_2} - id)(F)\cdot (\rho_A\otimes \rho_B)\big) \\
& = \tr\big(F\cdot((\Phi_1)_{m_1}\otimes (\Phi_2)_{m_2} - id) (\rho_A\otimes \rho_B)\big) \\
& = \tr\big( (f_1 \otimes I_{m_2}\otimes id_{\mc H_2})\cdot ((\Phi_1)_{m_1}\otimes (\Phi_2)_{m_2} - id) (\rho_A\otimes \rho_B)\big) \\
& + \tr\big( (I_{m_1} \otimes id_{\mc H_{1}}\otimes f_2\cdot ((\Phi_1)_{m_1}\otimes (\Phi_2)_{m_2} - id) (\rho_A\otimes \rho_B)\big) \\
& = \tr(f_1\cdot ((\Phi_1)_{m_1} -id )(\rho_A)) + \tr(f_2\cdot ((\Phi_2)_{m_2} - id)(\rho_B)) \\
& \ge (1-\varepsilon)(C_{S_1}^{cb}(\Phi_1)+ C_{S_2}(\Phi_2)).
\end{align*}
For the last equality, we used trace-preserving property of $(\Phi_1)_{m_1}, (\Phi_2)_{m_2}$.
\end{proof}	
	
	
	\subsection{Expected length and its estimates}\label{subsection:Lipschitz}
%-------------------
\begin{comment}	
	In the last subsection of the introduction, we introduce the main class of complexity measures that we will study in depth in this paper. In addition, we will define the notion of expected length of a noncommutative Lipschitz space. It turns out that the expected length is closely related to the Lipschitz complexity and we will provide some useful a priori upper and lower estimates for the expected length at the end of this subsection.
	\begin{definition}(Lipschitz complexity)\label{definition:LipschitzComplexity}
		Let $(\mathcal{A},|||\cdot|||)\ssubset N$ be a noncommutative Lipschitz space associated with the resource set $\mathcal{S}$. Then the $\mathcal{S}$-dependent Lipschitz complexity of a quantum channel $\Phi \in UCP_\mathcal{S}(N, N)$ is given by:
		\begin{equation}
			C_{Was, \mathcal{S}}(\Phi) := ||\Phi - id:\mathcal{A}\rightarrow N|| = \sup_{|||x|||\leq 1}||\Phi(x) - x||_\infty\pl.
		\end{equation}
		In addition, the $\mathcal{S}$-dependent correlation-assisted Lipschitz complexity is given by:
		\begin{equation}
			C^{cb}_{Was,\mathcal{S}}(\Phi):=\sup_{m\geq 0}||\Phi\otimes id_m - id_N\otimes id_m: \mathcal{A}\otimes M_m(\mathbb{C})\rightarrow M_m(N)||  = \sup_{m\geq 0} C_{Was,\mathcal{S}\otimes id_m}(\Phi\otimes id_m)\pl.
		\end{equation}
	\end{definition}
	If the resource set is clear from the context, we will omit the subscript $\mathcal{S}$. Notice in the definition of correlation-assisted complexity, we are allowed to use an auxillary system $M_m(\mathbb{C})$ but we are not allowed to use additional resources. Hence the correct notion of a resource set in this setting is $\mathcal{S}\otimes id_m = \{s\otimes id_m:s\in \mathcal{S}\}$. 
	
	As stated, the Lipschitz complexity is defined as a single-system complexity measure. Later we will discuss its extension to composite systems. It is easy to check that the Lipschitz complexity indeed satisfies all the axioms of a single-system resource-dependent complexity measure.
	\begin{lemma}\label{lemma:LipschitzAxiom}
		Given a resource set $\mathcal{S}$ and an associated Lipschitz space $(\mathcal{A},|||\cdot|||)\ssubset N$, the following statements hold for Lipschitz complexity:
		\begin{enumerate}
			\item $C_{Was}(id) = 0$;
			\item $C_{Was}$ is convex;
			\item Given two channels $\Phi,\Psi\in UCP_\mathcal{S}(N, N)$ such that $\Phi\circ\Psi\in UCP_\mathcal{S}(N,N)$ \footnote{Typically the set of resource generated quantum channels are closed under composition. But there might be exotic resource sets such that the set of channels generated by these resources is not closed under composition. In this paper, we will not consider such exotic examples.}, then we have subadditivity: $C_{Was}(\Phi\circ\Psi) \leq C_{Was}(\Phi) + C_{Was}(\Psi)$.
		\end{enumerate}
	\end{lemma}
	\begin{proof}
		The first statement follows directly from the definition since $C_{Was}(id) = ||id - id:\mathcal{A}\rightarrow N|| = 0$. The second state follows from the convexity of the norm. The third statement follows from a simple triangle inequality:
		\begin{align*}
			\begin{split}
				C_{Was}(\Phi\circ\Psi) &= ||\Phi(\Psi-id) + \Phi - id|| \leq ||\Phi||_{\infty\rightarrow\infty}||\Psi-id||_{Lip\rightarrow\infty} + ||\Phi-id||_{Lip\rightarrow\infty} \\&\leq C_{Was}(\Psi) + C_{Was}(\Phi)
			\end{split}
		\end{align*}
		where for a quantum channel (unital completely positive) $||\Phi||\leq ||\Phi||_{cb} \leq 1$.
	\end{proof}
	In addition to the Lipschitz space, the annihilator space of the resource set $\mathcal{S}$ is defined as:
	\begin{equation}
		\mathcal{A}^\perp :=\{x\in N: \partial_j(x) = 0 \text{ , }\forall \partial_j\in \mathcal{S}\}\pl.
	\end{equation}
	The annihilator is dual to the Lipschitz space in the Hilbert-Schmidt inner-product: $tr(x^*y) = 0$ for all $x\in\mathcal{A}, y\in \mathcal{A}^\perp$. For finite dimensional $N$, the orthogonal projection onto the annihilator can be extended to $L_1(N)$: $E_{fix}: L_1(N)\rightarrow \mathcal{A}^\perp\ssubset L_1(N)$. If $E_{fix}$ is completely positive, then $E^*_{fix}:N\rightarrow N$ is a unital channel since $1$ is annihilated by any derivation. 
	\end{comment}
	
%-------------------	
	Recall that $C_{S}^{cb}(E_{S'})$ provides an upper bound of the complexity, see (4) of Lemma \ref{elementary:cb}. In this subsection, we aim to provide several estimates for this quantity. 
	We also name this quantity as the \textit{expected length} of the noncommutative Lipschitz space:
	\begin{definition}\label{definition:expLength}
		Given the resource set $S$, the expected length of the Lipschitz space $\mathcal{A}$ is given by the "Lipschitz complexity" \footnote{Although it is not clear that $E_{fix}^*$ is guaranteed to be in the set of admissible channels $CP_S(\mc N)$, the definition of expected length is completely analogous to the definition of the Lipschitz complexity. Hence we choose to use the same terminology but put the Lipschitz complexity in quotation marks.} of the dual channel:
		\begin{equation}
			EL_\mathcal{S}(\mathcal{A}):=||E^*_{fix}-id:\mathcal{A}\rightarrow N|| = ||id:\mathcal{A}\rightarrow N||\pl.
		\end{equation}
		Similarly, the correlation-assisted expected length is given by:
		\begin{equation}
			EL^{cb}_\mathcal{S}(\mathcal{A}):=\sup_{m\geq 0}||E^*_{fix}\otimes id_m -id_{\mc N}\otimes id_m:\mathcal{A}\otimes M_m(\mathbb{C})\rightarrow M_m(\mc N)||\pl.
		\end{equation}
	\end{definition}
	In later applications, we often focus on channels $\Phi\in CP_S(\mc N)$ where the fixed point space of the channel $\Phi:L_1(\mc N)\rightarrow L_1(\mc N)$ is given by the annihilator $\mathcal{A}^\perp = S'$. 

	
	Before presenting the a priori estimates on $EL_\mathcal{S}(\mathcal{A})$, we explain why the expected length deserves its name. Recall any $\mathbb{B}(\mc H)$ contains a maximal Abelian subalgebra $\ell_\infty(\mc H)$ (the diagonal subalgebra). As operators on $H$, $\ell_\infty(\mc H)$ acts by multiplication. It is a well-known fact that the operator norm of the multiplication $||M_f||$ is equal to the sup-norm of the function $||f||$. In the proof, we will freely use this identification.	\begin{lemma}\label{lemma:diameter}
		Let $G$ be a finitely generated finite group and let $\mathcal{S} = \{s^{\pm 1}, e\}_{\{s\}\text{ generates $G$}}$ be the symmetric set of generators along with the identity. Consider the commutative algebra $\mc N:=\ell_\infty(G)\ssubset \mathbb{B}(\ell_2(G))$ and the left regular representation $\lambda:G\rightarrow U(\ell_2(G))$. Then the set $\mathcal{S}$ defines a resource set made up of derivations:
		\begin{equation}
			\partial_s(M_f) := \lambda_s\circ M_f - M_f\circ\lambda_s = (Ad_s(M_f) - M_f)\circ\lambda_s= (M_{\lambda_sf} - M_f)\circ\lambda_s\pl.
		\end{equation}
		where $M_f\in \mc N$ is the multiplication operator by the function $f$. The associated Lipschitz space is $\mathcal{A} = \{f\in \mc N: \frac{1}{|G|}\sum_g f(g) = 0\}$ and the Lipschitz norm is given by:
		\begin{equation}
			|||f||| = \sup_{s\in \mathcal{S}}||\partial_s M_f||_\infty = \sup_s||M_{\lambda_s f} - M_f||_\infty\pl.
		\end{equation} 
		The associated annihilator is $\mathcal{A}^\perp = \mathbb{C} = \text{span}\{\mathbbm{1}_G\}$. Finally, we have:
		\begin{equation}\label{equation:diameter}
			\frac{1}{|G|}\sum_g \ell_\mathcal{S}(g) = EL_\mathcal{S}(\mathcal{A})
		\end{equation}
		where $\ell_\mathcal{S}(g)$ is the word-length function induced by the generating set $\mathcal{S}$. \footnote{The same equation holds for the correlation-assisted expected length, since $N$ is a commutative algebra.}
	\end{lemma}
	\begin{proof}
		All the claim follows directly from definitions. We only need to show that last equation \ref{equation:diameter}. Let $\mu$ be the normalized counting measure on $G$ and let $\mathbb{E}$ be the expectation with respect to $\mu$. Then for $f\in \mathcal{A}$, we have $\mathbb{E}f = 0$. Since $\mathcal{S}$ generates $G$, for each $g$ we can fix a reduced word representation: $g = \prod_{1\leq i \leq \ell_\mathcal{S}(g)}s_i$ where $s_i\in \mathcal{S}$ for each $1\leq i \leq \ell_\mathcal{S}(g)$. First we make the following observation:
		\begin{equation}
			\mathbb{E}_g \lambda_gf (h) = \frac{1}{|G|}\sum_g f(g^{-1}h) = \frac{1}{|G|}\sum_g f(g) = \mathbb{E}(f)
		\end{equation}
		where the equation holds point-wise at each $h\in G$. Then for $f\in\mathcal{A}$, we have:
		\begin{align}
			\begin{split}
				||f|| &= ||f - \mathbb{E}_g\lambda_gf|| \leq \mathbb{E}_g||f - \lambda_gf|| = \mathbb{E}_g||f - (\prod_{1\leq i \leq \ell_\mathcal{S}(g)}\lambda_{s_i})f||
				\\
				& \leq \mathbb{E}_g\sum_{1\leq i \leq \ell_\mathcal{S}(g)}||f - \lambda_{s_i}f|| \leq \big(\mathbb{E}_g\ell_\mathcal{S}(g)\big)\sup_s||f - \lambda_sf|| = \mathbb{E}(\ell_\mathcal{S})|||f|||\pl.
			\end{split}
		\end{align}
		Therefore we have $EL_\mathcal{S}(\mathcal{A})\leq \mathbb{E}(\ell_\mathcal{S})$. 
		
		On the other hand, consider the function: $f_\mathcal{S}(g):=\mathbb{E}(\ell_\mathcal{S}) - \ell_\mathcal{S}(g)$. Clearly $\mathbb{E}f_\mathcal{S} = 0$. In addition, $|||f_\mathcal{S}||| = \sup_s ||\lambda_sf_\mathcal{S} - f_\mathcal{S}|| = ||\lambda_s\ell_\mathcal{S} - \ell_\mathcal{S}||$. By triangle inequality of the word length, for all $g\in G$ we have $|\ell_\mathcal(s^{-1}g) - \ell_\mathcal{g}| \leq \ell_\mathcal(s^{-1}) = 1$. Hence $|||f_\mathcal{S}|||\leq 1$. Then we have:
		\begin{equation}
			EL_\mathcal{S}(\mathcal{A})\geq EL_\mathcal{S}(\mathcal{A})|||f_\mathcal{S}|||\geq||f_\mathcal{S}|| \geq |f_\mathcal{S}(e)| = \mathbb{E}(\ell_\mathcal{S})\pl.
		\end{equation}
		Hence Equation \ref{equation:diameter} holds.
	\end{proof}
	For a finite group, the expected value of the word length function is equivalent to the classical diameter $diam_G = \sup_{g,h}\ell_\mathcal{S}(h^{-1}g)$. In this sense, $EL_\mathcal{S}(\mathcal{A})$ provides geometric information about the size of the Lipschitz space $\mathcal{A}$.
	
	As mentioned before, there are two types of resources that are typical in quantum information science: the discrete type $\mathcal{U}$ and the continuous type $\Delta$. Quantum circuits are built using the discrete type of resources while time evolutions (for both open and closed systems) are built using the continuous type of resources. We will conclude this section by some a priori estimates on the expected length for these two types of resources. The proof of these estimates uses similar ideas as the proof of Lemma \ref{lemma:diameter}.
	\subsubsection{A Priori Estimates of Expected Lengths for Continuous Resources}\label{subsubsection:continuousEst}
	Recall the continous type of resource is given by a finite set of infinitesimal generators $\Delta = \{a_j\}_{1\leq j \leq n}$. For self-adjoint Lindbladians or Hamiltonians, $a_j$'s are self-adjoint operators. When the underlying quantum system is finite dimensional, $\mc N$ is a finite dimensional von Neumann algebra. In this case, the group of unitaries $G:=U(\mc N)$ is a real semisimple Lie algebra. In this case, recall the definition of Hörmander systems.
	\begin{definition}\label{definition:Hörmander}
		Let $\mathfrak{g}$ be a semisimple Lie algebra. In addition, let $H = \{a_1,...,a_n\}\ssubset\mathfrak{g}$ be a finite set of elements in the Lie algebra. Then $H$ is a Hörmander system if there exist a finite number $\ell_H$ such that:
		\begin{equation}\label{equation:generation}
			\text{span}\{[a_{i_1},[a_{i_2},...,[a_{i_{j-1}}, a_{i_j}]]]: j \leq \ell_H\text{ and }a_i\in H\} = \mathfrak{g}\pl.
		\end{equation}
		The number $\ell_H$ is called the Hörmander length.
	\end{definition}
	For a finite dimensional quantum system, the Lie group of unitaries $G$ is semisimple and compact. Therefore we can use the norm induced by the Killing-Cartan form on the associated Lie algebra $\mathfrak{g}$. The norm is bi-invariant under the adjoint action of $G$ and it is equivalent to the Euclidean norm. Using this norm and given a Hörmander system, one can define a compatible Carnot-Caratheodory metric on $G$:
	\begin{definition}\label{definition:CarnotCaratheodory}
		Let $H$ be a Hörmander system of a Lie group $G$, the induced Carnot-Caratheodory metric is given by:
		\begin{equation}
			d_H(x,y):=\inf\{\int||\gamma'(t)||dt: \gamma(0) = x, \text{ }\gamma(1) = y, \text{ }L_{\gamma(t)^{-1}}\gamma'(t)\in \text{span}\{H\} \}
		\end{equation}
		where $\gamma(t)$ is a piecewise-differentiable curve connecting $x,y$ such that its derivative is in the subspace spanned by the Hörmander system. And the norm $||\cdot||$ is induced by the Killing-Cartan form on the Lie algebra.
	\end{definition}
	Hence the Carnot-Caratheodory distance is nothing but the geodesic distance under the pseudo-Riemannian structure induced by the left translation of the Hörmander system. 
	
	If the set of iterated brackets of the infinitesimal generators does not span the entire Lie algebra, operationally it means that the resources are not enough for adequate control over the entire quantum systems. This is a pathological situation that we shall not consider in this paper. 
	\begin{lemma}\label{lemma:continuousEst}
		For a finite dimensional quantum system $\mc N$. Assume $\mc N$ is a factor (i.e. $\mc N = \mathbb{B}(H)$ for some finite dimensional Hilbert space $H$). Let $\mu$ be the normalized Haar measure on the group of unitaries $G$. Assume the resource set $\Delta$ forms a Hörmander system for $G$ and the generators $a_j$'s are orthonormal with respect to the Killing-Cartan inner product. Consider the Lipschitz space: $(\mathcal{A},|||\cdot|||_{\Delta})$ where the norm is the $\ell_\infty$-Lipschitz norm associated with $\Delta$ (c.f. Definition \ref{complexity:first}). Then we have:
		\begin{equation}
			\mathbb{E}_\mu(\ell_\Delta) \leq EL_\Delta(\mathcal{A})\leq |\Delta|^{1/2}\mathbb{E}_\mu(\ell_\Delta)\pl.
		\end{equation}
		where $|\Delta|$ is the cardinality of the generating set. And a similar statement holds for correlation-assisted expected length.
	\end{lemma}
	\begin{proof}
		First observe that since $\Delta$ is a Hörmander system and the annihilator $\mathcal{A}^\perp$ commutes with all generators in $\Delta$, then we have $\mathcal{A}^\perp = \{x\in \mc N: uxu^* = x\text{ for all }u\in G\} = \mathbb{C}$. In particular, for all $x\in \mc N$, $\mathbb{E}_\mu x:=\int_G d\mu(u)uxu^*\in\mathcal{A}^\perp$. And for $x\in\mathcal{A}$, $\mathbb{E}_\mu x = 0$. In addition, for all $u\in G$, since $G$ is a compact Lie group and $\Delta$ is a Hörmander system, there exists a piecewise smooth path $\gamma_u:[0,1]\rightarrow G$ such that $\gamma_u(0) = id, \gamma_u(1) = u$ and $\gamma'_u(t) \in (L_{\gamma_u(t)})_*\Delta$ where $\gamma_u'(t)$ is the derivative and $(L_{\gamma_u(t)})_*\Delta$ is the left-translated Hörmander system. In particular, at each $t\in [0,1]$, $\gamma'_u(t)$ can be written as a linear combination of left-translated vector fields $(L_{\gamma_u(t)})_*\Delta = \{(L_{\gamma_u(t)})_*a_1,...,(L_{\gamma_u(t)})_*a_n\}$:
		\begin{equation}
			\gamma_u'(t) = \sum_{1\leq j \leq n} u_j(t)(L_{\gamma_u(t)})_*a_j
		\end{equation}
		where $u_j(t)$'s are coefficients and depend piece-wise smoothly on $t$. The Killing-Cartan norm is given by $||\gamma_u'(t)||^2 = \sum_j |u_j(t)|^2$. Now fix a length-minimizing path for each $u\in G$. 
		
		Using these observations, for all $x\in \mathcal{A}$ we have:
		\begin{align}
			\begin{split}
				||x|| &= ||x - \mathbb{E}_\mu x|| \leq \mathbb{E}_\mu||\int_0^1 dt [\gamma_u'(t), \gamma_u(t)x\gamma^*_u(t)]|| \leq \mathbb{E}_\mu \int_0^1 dt \sum_j |u_j(t)||\gamma_u(t)[a_j, x]\gamma_u^*(t)||
				\\
				&\leq \mathbb{E}_\mu\int_0^1 dt||\gamma_u'(t)||\big(\sum_j ||[a_j, x]||^2\big)^{1/2} = \mathbb{E}_\mu(\ell_\Delta)\sup_j||[a_j,x]|||\Delta|^{1/2} = |\Delta|^{1/2}\mathbb{E}_\mu(\ell_\Delta)|||x|||\pl.
			\end{split}
		\end{align}
		Therefore $EL_\Delta(\mathcal{A})\leq |\Delta|^{1/2}\mathbb{E}_\mu(\ell_\Delta)$.
		
		On the other hand, consider the function: $f_\Delta(g):=\mathbb{E}_\mu(\ell_\Delta) - \ell_\Delta(g)$. By triangle inequality, $f_\Delta$ is Lipschitz continuous and hence differentiable almost everywhere. In addition for every $a_j\in H$ and $g\in G$ we have:
		\begin{equation}
			|\lim_{t\rightarrow 0}\frac{1}{t}(f(e^{ta_j}g) - f(g))| \leq \lim_{t\rightarrow 0}\frac{1}{t}\ell_\Delta(e^{ta_j}) \leq ||a_j|| = 1
		\end{equation}
		where we used triangle inequality and the fact that all generators $a_j$'s are normalized in the Killing-Cartan norm. Hence $|||f_\Delta|||\leq 1$. Since $||f_\Delta|| \geq |\mathbb{E}_\mu(\ell_\Delta) - \ell_\Delta(e)| = \mathbb{E}_\mu(\ell_\Delta)$. We have the lower bound: $\mathbb{E}_\mu(\ell_\Delta) \leq EL_\Delta(\mathcal{A})$.
	\end{proof}
	\subsubsection{A Priori Estimates of Expected Lengths for Discrete Resources}\label{subsubsection:discreteEst}
	For discrete resources $\mathcal{U}$, we assume this is a symmetric set of unitaries (i.e. if $s\in \mathcal{U}$ then $s^{-1}\in\mathcal{U}$) and we assume this set contains identity. Two cautionary comments are in order. 
	\begin{enumerate}
		\item The group $\Gamma$ generated by $\mathcal{U}$ is a discrete subgroup of the group of unitaries in $\mc N$. Even when $\mathcal{U}$ is a set of universal unitaries in the sense of Kitaev-Solovay \cite{DN}, $\Gamma$ does not contain all unitaries in $\mc N$. In particular, $\Gamma$ has measure zero under the Haar measure on the unitary group.
		\item The Haar measure on $\Gamma$ is the counting measure. If $\Gamma$ is not finite (which is typically the case), the expected length under the Haar measure will be infinite.
	\end{enumerate}
	Hence we must specify another measure on $\Gamma$. Recall the symmetric generating set $\mathcal{S}:=\mathcal{U}-\{e\}$ defines a symmetric random walk on $\Gamma$ whose transition function is given by convolution with the finite measure: $\nu:=\frac{1}{2n}\sum_{s\in\mathcal{S}}\delta_s$ where $2n$ is the cardinality of $\mathcal{S}$. The Cesaro mean of $\nu^{\star n}$ weakly converges to a probability measure:
	\begin{equation}
		\mu := \lim_{N\rightarrow \infty}\frac{1}{N}\sum_{0\leq n \leq N-1}\nu^{\star n}\pl.
	\end{equation}
	For any function $f\in \ell_\infty(\Gamma)$, the convolution with $\mu$ projects $f$ onto the fixed point subspace: $\{f\in\ell_\infty(\Gamma): \nu\star f = f\}$. Denote this projection as $P_\mu$. Consider the transference map:
	\begin{equation}
		\pi: \mc N\rightarrow \ell_\infty(\Gamma,\mc N):\pi(x)(g):= g^*xg\pl.
	\end{equation}
	This is a faithful normal $*$-homomorphism that intertwines the group action: $\pi(hxh^*) = \lambda_h\pi(x)$ where $\lambda_h$ is the left-regular action of $\Gamma$. Consequently, $\pi$ intertwines the projections onto the fixed point subspaces. Recall $E_{fix}:\mc N\rightarrow \mathcal{A}^\perp = \{x\in \mc N: g^*xg = x \text{ for all }g\in\Gamma\}$ is the projection onto the annihilator subspace. Then we have: $\pi\circ E_{fix} = P_\mu\circ \pi$. In particular, for $x\in\mathcal{A}$, since $E_{fix}x = 0$, then we have: $0 = P_\mu(\pi(x)) = \lim_{N\rightarrow\infty}\frac{1}{N}\sum_{0\leq n\leq N-1}\nu^{\star n}\star \pi(x)$. In addition, since $\pi$ is a faithful $*$-homomorphism and it intertwines the group action, we have:
	\begin{equation}
		|||x||| = \sup_{s\in\mathcal{S}}||x - sxs^*|| = \sup_{s\in\mathcal{S}}||\pi(x) - \lambda_s\pi(x)|| = |||\pi(x)|||\pl.
	\end{equation}
	Using these observations, we can now prove:
	\begin{lemma}\label{lemma:discreteEst}
		For a finite dimensional quantum system $\mc N$. Again for simplicity assume $\mc N =\mathbb{B}(\mc H)$ for some finite dimensional Hilbert space $H$. Let $\mathcal{U}$ be a symmetric set of discrete resources and let $\mathcal{S} = \mathcal{U} - \{e\}$ be the set of generators that generate a discrete subgroup $\Gamma$ in the group of unitaries in $\mc N$. Let $\nu,\mu$ be the measures defined in the discussion before this lemma. Then we have:
		\begin{equation}
			 EL_\mathcal{U}(\mathcal{A})\leq\mathbb{E}_\mu(\ell_\mathcal{S})
		\end{equation}
		where $\ell_\mathcal{S}$ is the word-length function induced by the generators $\mathcal{S}$. And a similar result holds for the correlation-assisted expected length.
	\end{lemma}
	\begin{proof}
		Using the same notation as the discussion before this lemma, the Lipschitz space $\mathcal{A} = \{x\in N: E_{fix}x = 0\} = \{x\in N: g^*xg = x \text{ for all }g\in\Gamma\}$. The Lipschitz norm is given by: $|||x||| = \sup_{s\in\mathcal{S}}||x - sxs^*||$. Let $V_n:=\{g\in \Gamma: \ell_\mathcal{S}(g)\leq n\}$. Then on $\ell_\infty(\Gamma,N)$, the convolution by $\nu^{\star n}$ can be written as:\begin{equation*}
			\nu^{\star n}\star f = \sum_{g\in V_n}\mathbb{P}_e^n(g)\lambda_g f
		\end{equation*}
		where the probability measure $\mathbb{P}_e^n(g)$ is the probability of hitting $g$ in $n$-steps starting from the identity $e$. This measure has support in $V_n$. For each $g$, fix the reduced word representation $g = \prod_{1\leq i \leq \ell_\mathcal{S}(g)}s_i$ where $s_i\in \mathcal{S}$. Then using the observations before this lemma, for $x\in\mathcal{A}$ we have:
		\begin{align}
			\begin{split}
				||x|| &= ||x - E_{fix}x|| = ||\pi(x) -P_\mu\pi(x)|| \leq \lim_{N\rightarrow \infty}\frac{1}{N}\sum_{0\leq n\leq N-1}||\pi(x) - \nu^{\star n}\star\pi(x)||
				\\
				&\leq \lim_{N\rightarrow\infty}\frac{1}{N}\sum_{0\leq n\leq N-1}\sum_{g\in V_n}\mathbb{P}^n_e(g)||\pi(x) - \lambda_g\pi(x)||
				\\&
				=\lim_{N\rightarrow\infty}\frac{1}{N}\sum_{0\leq n \leq N-1}\sum_{g\in V_n}\mathbb{P}_e^n(g)\sum_{1\leq i \leq \ell_\mathcal{S}}||\pi(x) - \lambda_{s_i}\pi(x)||
				\\
				&\leq \lim_{N\rightarrow \infty}\frac{1}{N}\sum_{0\leq n \leq N-1}\sum_{g\in V_n}\mathbb{P}^n_e(g)\ell_\mathcal{S}(g)\sup_{s\in\mathcal{S}}||\pi(x) - \pi(sxs^*)|| = \mathbb{E}_\mu(\ell_\mathcal{S})|||x|||\pl.
			\end{split}
		\end{align}
		Therefore we have $EL_\mathcal{U}(\mathcal{A}) \leq \mathbb{E}_\mu(\ell_\mathcal{S})$.
	\end{proof}
	Compared with the case of continuous resources, a lower bound by the expected length is missing for discrete resources. The main difficulty to mimick the construction in the continuous case is the transference principle. More specifically, one would like to consider the following function: $f_{\mathcal{S}}(g) := \mathbb{E}_\mu(\ell_\mathcal{S}) - \ell_\mathcal{S}(g)$ for $\ell_\mathcal{S}(g) \leq d_{min}:=\min\{d\in\mathbb{N}: |d - \mathbb{E}_\mu(\ell_\mathcal{S})| < 1\}$ and $f_{\mathcal{S}}(g) = 0$ for $\ell_\mathcal{S}(g) > d_{min}$. It is clear that for all $s\in S$, we have:
	\begin{equation*}
		||\lambda_sf_\mathcal{S} - f_{\mathcal{S}}|| \leq 1\pl.
	\end{equation*}
	Then it is not difficult to see that if $\mc N = \mathbb{C}$ is trivial, then we have the lower bound: $\mathbb{E}_\mu(\ell_{S}) \leq EL_{\mathcal{U}}(\ell_\infty(\Gamma))$. Indeed, given Lemma \ref{lemma:diameter}, this lower bound should not come as a surprise. However, $f_\mathcal{S}\in\ell_\infty(\Gamma)$ is not in the image of the transference map $\pi(\mc N)$. Therefore, this lower bound on $\ell_\infty(\Gamma)$ cannot be transfered back to a lower bound on $\mc N$.


\section{Resource-dependent complexity estimates for open systems}	
\subsection{Discrete gates as resource}\label{discrete resource}
Suppose $S = \{U_1,\cdots, U_l\}\subset \mb M_d$ is a set of unitaries. Moreover, we assume it to be symmetric. The set of channels we will consider is $CP_{S}(\mb M_d)$. 
For dynamics, recall that the Hamiltonian-free Lindblad generator is given by the jump operators from the above gate set. Recall that the Lipschitz norm is given as follows: 
\begin{equation}
|||X|||_S: = \sup_{j=1,\cdots, l} \|[U_j,X]\|_{\infty}.
\end{equation}
The mean zero space is given by $\mc A = \{x\in \mb M_d: E_{S'}(x)=0\}$. The complete complexity of any channel $\Phi$ is given by 
\begin{align*}
C_S^{cb}(\Phi):= \|\Phi^*-id: (\mc A, |||\cdot|||_S) \to \mb M_d\|_{cb}.
\end{align*}

Suppose we are given a unitary channel from $S$, i.e., $$\Phi_{\mu}:= \sum_{i}\mu_i U_i\cdot U_i^*,$$
where $\{\mu_i\}_i$ is a probability measure. Then we are able to give an upper bound of the complexity of the dynamics driven by $\Phi_{\mu}$:
\begin{prop}
The evolution 
\begin{equation}\label{evolution:discrete}
T_t:= \exp(t(\Phi_{\mu}-I))
\end{equation}
 satisfies 
\begin{equation}
C_S^{cb}(T_t) \le t.
\end{equation}
\end{prop}
\begin{proof}
First we claim that \begin{equation}
C_S^{cb}(\Phi_{\mu}) \le 1.
\end{equation}
In fact, for any $X \in \mc A$, 
\begin{align*}
\|\Phi_{\mu}^*(X) - X\|_{cb}& = \|\sum_{i}\mu_i(U_i^*XU_i - X)\|_{cb} \le \sum_{i}\mu_i\|U_i^*XU_i - X\|_{cb} \\
& = \sum_{i}\mu_i\|U_i^*X - XU_i^*\|_{cb} = \sum_{i}\mu_i\|[U_i^*,X]\|_{cb} \le |||X|||^{cb}_{S}.
\end{align*}
Recall that 
\begin{equation}
T_t:= \exp(t(\Phi_{\mu}-I)) = \sum_{k=0}^{\infty}\frac{t^k}{k!}e^{-t} \Phi_{\mu}^k.
\end{equation}
Then by \textit{convexity and subadditivity under concatenation}, we have 
\begin{align*}
C_S^{cb}(T_t) \le \sum_{k=0}^{\infty}\frac{t^k}{k!}e^{-t} C_S^{cb}(\Phi_{\mu}^k) \le \sum_{k=0}^{\infty}\frac{t^k}{k!}e^{-t} k = t.
\end{align*}
\end{proof}
For the lower bound estimate, we first connect the complexity of a class of channels to its return time to its fixed point.

Suppose $\Phi$ is an arbitrary quantum channel with the same fixed point algebra $\mc N_{fix} = S'$ and we define its \textit{(complete) return time} $t^{(cb)}(\varepsilon, \Phi)$:
\begin{equation}
t^{(cb)}(\varepsilon, \Phi):= \inf\{k \in \mb N: \|(\Phi^*)^{k} - E_{fix}\|^{(cb)}_{\infty \to \infty} \le \varepsilon\},
\end{equation}
where $E_{fix} = E_{S'}$ is the trace-preserving conditional expectation onto the fixed point algebra.
\begin{prop}
For any channel $\Phi$ with the same fixed point algebra $\mc N_{fix}$, and we assume that for some $\varepsilon>0$, the (complete) return time is finite. Then we have 
\begin{equation}
(1-\varepsilon)C^{(cb)}_{S}(E_{fix}) \le t^{(cb)}(\varepsilon, \Phi)C^{(cb)}_{S}(\Phi).
\end{equation}
\end{prop}
\begin{proof}
We only show $(1-\varepsilon)C_{S}(E_{fix}) \le t(\varepsilon, \Phi)C_{S}(\Phi)$ and the correlation-assisted version holds true by just replacing every channel by its $m$-th amplification. Without loss of generality, we assume $k=t(\varepsilon, \Phi)<\infty$, then we have 
\begin{equation}
\|(\Phi^*)^k - E_{fix}\|_{\infty \to \infty} \le \varepsilon.
\end{equation}
Calculating $C_{Lip_L}(E_{fix})$, we have
\begin{align*}
C_{S}(E_{fix}) & = \|E_{fix} - id : (\mc A, |||\cdot|||_{S}) \to L_{\infty}(\mb M_d)\| \\
& = \|E_{fix} - (\Phi^*)^k + (\Phi^*)^k - id : (\mc A, |||\cdot|||_{S}) \to L_{\infty}(\mb M_d)\| \\
& \le C_{S}(\Phi^k) + \|E_{fix} - (\Phi^*)^k  : (\mc A, |||\cdot|||_{S}) \to L_{\infty}(\mb M_d)\|.
\end{align*}
Recall that the fixed point algebra of $\Phi$ is $\mc N_{fix}$, we have $(\Phi^*)^k E_{fix} = E_{fix} (\Phi^*)^k = E_{fix}$, then 
\begin{equation}
E_{fix} - (\Phi^*)^k = (id - E_{fix})(E_{fix} - (\Phi^*)^k).
\end{equation}
Using the following diagram 
\begin{center}
\begin{tikzcd}
(\mc A, \|\cdot\|_{S}) \arrow[r, "id - E_{fix}"]
 \arrow[dr,"E_{fix} - (\Phi^*)^k" description]
&  L_{\infty}(\mb M_d) \arrow[d, "E_{fix} - (\Phi^*)^k"]\\
& L_{\infty}(\mb M_d)
\end{tikzcd}
\end{center}
we have 
\begin{align*}
\|E_{fix} - (\Phi^*)^k  : (\mc A, |||\cdot|||_{S}) \to L_{\infty}(\mb M_d)\| & \le C_{S}(E_{fix}) \|(\Phi^*)^k - E_{fix}\|_{\infty \to \infty} \\
& \le \varepsilon C_{S}(E_{fix}).
\end{align*}
In summary, we showed that 
\begin{align*}
C_{S}(E_{fix}) \le \varepsilon C_{S}(E_{fix}) + C_{S}(\Phi^k) \le \varepsilon C_{S}(E_{fix}) + k C_{S}(\Phi),
\end{align*}
which implies $(1-\varepsilon)C_{Lip_L}(E_{fix}) \le t(\varepsilon, \Phi)C_{Lip_L}(\Phi)$.
\end{proof}
\noindent To provide a lower bound, we need to assume $\Phi_{\mu}$ is symmetric such that we get bounded return time. The dynamical evolution $T_t = \exp(t(\Phi_{\mu}-id))$ converges to $E_{fix}$ as $t$ goes to infinity and moreover \cite{GJLL22},
\begin{equation}
k^{cb}(\frac{1}{2}):= \inf\{t >0: \|T_t-E_{fix}\|_{cb} \le \frac{1}{2}\}<\infty.
\end{equation}
Now we are ready to provide the lower bound of complexity of $T_t$:
\begin{theorem}
Suppose $T_t$ given by \eqref{evolution:discrete} is symmetric, i.e., $T_t = T_t^*$, then 
\begin{equation}
C^{cb}_{Lip_L}(T_t) \asymp \begin{cases}
t,\ t\le k^{cb}(\frac{1}{2}), \\
C^{cb}_{S}(E_{fix}),\ t\ge k^{cb}(\frac{1}{2}).
\end{cases}
\end{equation}
Here $a\asymp b$ means there exist universal constants $C_1,C_2>0$ such that $C_1b \le a \le C_2 b$.
\end{theorem}
\begin{proof}
We only show the case for $C_{S}(T_t)$, and the complete version has exactly the same proof. If $t\le k^{cb}(\frac{1}{2})$, the upper bound follows from the previous proposition. For the lower bound, we choose $k \ge 1$ such that 
\begin{equation}\label{approximate k}
(k-1)t \le k^{cb}(\frac{1}{2}) \le kt.
\end{equation}
Then \begin{align*}
C_{S}(E_{fix}) & = \|E_{fix} - id : (\mc A, |||\cdot|||_S) \to L_{\infty}(\mb M_d)\| \\
& = \|E_{fix} - (T_t)^k + (T_t)^k - id : (\mc A, |||\cdot|||_S) \to L_{\infty}(\mb M_d)\| \\
& \le C_{S}(T_t^k) + \|E_{fix} - (T_t)^k  : (\mc A, |||\cdot|||_S) \to L_{\infty}(\mb M_d)\| \\
& \le kC_{S}(T_t)+ C_{S}(E_{fix})\|E_{fix} - (T_t)^k\|_{\infty\to \infty} \\
& \le kC_{S}(T_t) + \frac{1}{2}C_{S}(E_{fix}).
\end{align*}
Therefore, 
\begin{align*}
C_{S}(T_t) \ge \frac{C_{S}(E_{fix})}{2k} \ge \frac{t C_{S}(E_{fix})}{2(k^{cb}(\frac{1}{2})/t+1)}\ge \frac{t C_{S}(E_{fix})}{4k^{cb}(\frac{1}{2})},
\end{align*}
where in the last two inequalities, we used \eqref{approximate k} and the fact that $t \le k^{cb}(\frac{1}{2})$.

If $t\ge k^{cb}(\frac{1}{2})$, we have $\|E_{fix} - T_t\|_{\infty\to \infty} \le \frac{1}{2}$,
\begin{align*}
C_{S}(E_{fix}) & = \|E_{fix} - id : (\mc A, |||\cdot|||_S) \to L_{\infty}(\mb M_d)\| \\
& = \|E_{fix} - T_t + T_t - id : (\mc A, |||\cdot|||_S) \to L_{\infty}(\mb M_d)\| \\
& \le C_{S}(T_t) + \frac{1}{2}C_{S}(E_{fix}),
\end{align*} 
thus $C_{S}(T_t) \ge \frac{1}{2}C_{S}(E_{fix})$. Similarly, we can show that $C_{S}(T_t) \le \frac{3}{2}C_{S}(E_{fix})$, by bounding $C_{S}(T_t)$.
\end{proof}

\subsection{Continuous case: infinitesimal generators as resource}    \label{continuous resource}
	We take $\mc N = \mb M_d$ and $L:\mb M_d \to \mb M_d $ as a self-adjoint Lindbladian, i.e., 
\begin{equation}\label{Lindblad: continuous}
L(x) = \sum_{j}\big(a_jxa_j - \frac{1}{2}(a_j^2x+xa_j^2)\big),
\end{equation}
where $a_j$ are self-adjoint. Denote $\tau$ as the normalized trace, $\mc N_{fix} = \{X\in \mb M_d: [a_j,X]=0, \forall j\}$ and $E_{fix}$ is the trace-preserving conditional expectation onto $\mc N_{fix}$. The resource set, denoted by $\Delta$ is given by 
\begin{equation}
\Delta:= \{a_j\}.
\end{equation}
We can define the resource-dependent Lipschitz norm and complexity $|||\cdot|||_{\Delta}, C_{\Delta}$ as before. The mean 0 space is given by $\mc A = \{X \in \mb M_d: E_{fix}(X) = 0\}.$ In order to estimate $C_{\Delta}(T_t)$, where $T_t = e^{tL}$, we need to introduce the another Lipschitz norm, induced by the gradient form. Recall the gradient form of $L$ is defined by
\begin{equation}
\Gamma_L(x,y):= L(x^*y)-x^*L(y) - L(x^*)y,\ x,y\in \mb M_d.
\end{equation} 
We define the Lipschitz norm via the gradient form: 
\begin{equation}
|||X|||_{\Gamma}: = \max\{\|\Gamma_L(X,X)\|_{\infty}^{\frac{1}{2}}, \|\Gamma_L(X^*,X^*)\|_{\infty}^{\frac{1}{2}}\}.
\end{equation}
Recall that for any $X\in \mc A$, $|||X|||_\Delta = \sup_j ||[a_j,X]|| = \sup_j||[a_j,X^*]|| =|||X^*|||_\Delta$, by definition, we can show that 
\begin{equation}
|||X|||_{\Gamma} \le \sqrt{|\Delta|}\cdot |||X|||_\Delta.
\end{equation}
The Lipschitz norm $|||\cdot|||_{\Gamma}$ induces a quantum metric on the state space:
\begin{equation}
\|\rho\|_{\Gamma^*} = \sup\{|\tau(\rho X)| \big| E_{fix}(X) = 0, \|X\|_{\Gamma} \le 1\}.
\end{equation}
To establish an upper bound, we make use of \textit{modified logarithmic Sobolev inequality}. Recall that $\{T_t:=\exp(tL)\}$ satisfies the $\lambda$-modified logarithmic Sobolev inequality ($\lambda$-MLSI) if there exists a constant $\lambda > 0$ such that the relative entropy has exponential decay:
	\begin{equation}
		D(T_t\rho \| E_{fix}\rho) \leq e^{-\lambda t}D(\rho \| E_{fix}\rho)
	\end{equation}
 The largest constant $\lambda$ for which the relative entropy decays exponentially is denoted as $\text{MLSI}(L)$ (the MLSI constant of $L$). We denote the best constant as $\text{CLSI}(L)$ if we consider the complete version. By definition, the fixed point algebra $\mc N_{fix}$ of $\{T_t\}_{t\ge 0}$ is tracial-orthogonal to the Lipschitz subspace $\mc A$. 
 
Recall the notation of index of a von Neumann subalgebra $\mc M\subset \mc N$. Let $E_{\mc M}:\mc N\rightarrow \mc M$ be the conditional expectation onto $M$ (with respect to the trace):
	\begin{equation}
		\text{Ind}(\mc N:\mc M) :=\inf\{\alpha > 0: \rho \leq \alpha E_{\mc M}^*(\rho)\text{ , }\forall \rho\in L_1(\mc N)_+\}\pl.
	\end{equation}
	For example, $\text{Ind}(M_n(\mathbb{C}):\mathbb{C}) = n$ and $\text{Ind}(M_n(\mathbb{C}): \ell_{\infty}^n) = n$. Hence the index is a proper notion of relative size of $\mc M$ in $\mc N$. 
	
	We recall the following result from \cite[Corollary 6.8]{Fisher}:
	\begin{prop}\label{proposition:TA}
		If the Lindbladian $L$ generates a semigroup that satisfies $\lambda$-MLSI, then we have:
		\begin{equation}
			||\rho - E_{fix}^*\rho||_{\Gamma^*} \leq 4\sqrt{\frac{2D(\rho \| E^*_{fix}\rho)}{\lambda}}
		\end{equation} 
		for any state $\rho$, where $||\cdot||_{\Gamma^*}$ is dual to the Lipschitz norm $|||\cdot|||_{Lip_{L}}$.
	\end{prop}
		Note that in the tracial setting, $E_{fix}^*=E_{fix}$. Then we can get an upper estimate for the expected length:
	
	
	\begin{lemma}\label{corollary:expLengthCLSI}
		The expected length of $\mathcal{A}$ is bounded above by:
		\begin{equation}
			EL_\Delta^{cb}(\mathcal{A}):= C_{\Delta}^{cb}(E_{fix}) \le 4\sqrt{\frac{2 |\Delta|\sup_{\rho} D^{cb}(\rho \| E_{fix}\rho)}{\text{CLSI}(L)}} \le 4\sqrt{\frac{2|\Delta|\log (\text{Ind}^{cb}(\mc N:\mc N_{fix}))}{\text{CLSI}(L)}}.
		\end{equation}
		
	\end{lemma}	
	\begin{proof}
		We only prove the case for $EL_\Delta(\mathcal{A})$ and the complete version follows from the same argument. 
		
		First, using the relation between the two Lipschitz norms $|||\cdot|||_\Gamma$ and $|||\cdot|||_{\Delta}$, we have:
		\begin{equation}
			||id:(\mathcal{A},|||\cdot|||_\Delta)\rightarrow (\mathcal{A},|||\cdot|||_\Gamma)|| \leq \sqrt{|\Delta|}\pl.
		\end{equation}
		Then using the definition of expected length, we have:
		\begin{align}
			\begin{split}
				EL_\Delta(\mathcal{A}) &= ||id: (\mathcal{A},|||\cdot|||_\Delta)\rightarrow \mc N||
				\leq \sqrt{|\Delta|}||id:(\mathcal{A}, |||\cdot|||_\Gamma)\rightarrow \mc N||\\&
				 = \sqrt{|\Delta|}\cdot ||E_{fix}-id:(\mathcal{A}, |||\cdot|||_\Gamma)\rightarrow \mc N||  \\
				 & = \sqrt{|\Delta|}\cdot ||E^*_{fix}-id: L_1(\mc N)\rightarrow (\mathcal{A}, |||\cdot|||_\Gamma)^*|| \\
				 &\le  4\sqrt{\frac{2 |\Delta|\sup_{\rho} D(\rho \| E_{fix}\rho)}{\text{MLSI}(L)}}
			\end{split}
		\end{align}
		
		where the equation in the second line follows from $E_{fix}(\mathcal{A})= 0$ and the inequality follows from Proposition \ref{proposition:TA}. This gives the first upper bound. For the second upper bound, we used the result in \cite{LGMJNL}:
		\begin{equation}
			D_{\mc N_{fix}} \leq \log \text{Ind}(\mc N:\mc N_{fix})\pl.
		\end{equation}
	where $D_{\mc N_{fix}} := \sup_{\rho: \tr\rho = 1}D(\rho | E^*_{fix}\rho)$ is the maximum entropy \cite{LGMJNL, LGCR} and the maximum entropy is bounded above by the log-index \cite{LGMJNL, LGCR}.
	\end{proof}
	
Combining the above lemma and the elementary property (4) of Lemma \ref{elementary:cb}, the upper estimate of $C_{\Delta}^{cb}(T_t)$ is clear: 
\begin{theorem}
Suppose $T_t = \exp(tL)$ where $L$ is given by \eqref{Lindblad: continuous}. Then we have the following upper estimate: for any $t\ge0$,
\begin{equation}
C_{\Delta}^{cb}(T_t) \le 4\sqrt{\frac{2|\Delta|\log (\text{Ind}^{cb}(\mc N:\mc N_{fix}))}{\text{CLSI}(L)}} \|L\|_{cb}t.
\end{equation}
\end{theorem}

Since the return time argument works for any symmetric quantum channel, the lower bound can also be derived, as in the last subsection:
\begin{theorem}
Define the return time for $T_t = \exp(tL)$ with L given by \eqref{Lindblad: continuous}:\begin{equation}
k^{cb}(\frac{1}{2}):= \inf\{t >0: \|T_t-E_{fix}\|_{cb} \le \frac{1}{2}\}<\infty.
\end{equation}
Then the lower bound estimate of $C_{\Delta}^{cb}(T_t)$ is given as follows:
\begin{align}
& C_{\Delta}^{cb}(T_t) \ge \frac{t C_{S}(E_{fix})}{4k^{cb}(\frac{1}{2})}, t \le k^{cb}(\frac{1}{2}). \\
& C_{\Delta}^{cb}(T_t) \asymp C^{cb}_{\Delta}(E_{fix}),\ t > k^{cb}(\frac{1}{2}).
\end{align}
\end{theorem}
	
	

	
	\section{Hörmander Systems, Carnot-Caratheodory Distance, and Nielsen's Complexity}\label{section:hormander}
	In this section, we apply the Lipschitz complexity measure to the canonical Lipschitz norm associated with a Hörmander system on a compact Lie group. The geometric approach to characterize the complexity of a unitary operator by its distance to the identity operator is due to Nielsen \cite{N1,N2,N3}. In some cases, the geometric notion of complexity is equivalent to the more familiar quantum circuit cost. The key idea in defining the geometric quantum complexity measure is to regard the infinitesmial generators of the unitary operators as resources. Using a given set of infinitesimal generators, one can iteratively construct an efficient realization of unitary operators. This is the familiar idea of Hamiltonian simulation. From this perspective, it is natural and important for us to recast Nielsen's geometric complexity in the language of quantum metric spaces and Lipschitz norms. One advantage of this new framework is the added ability to characterize the Markovian dynamics of certain open quantum systems. Such dissipative dynamics has been used to design and engineer entangled states \cite{C1}\cite{K1}. Therefore it is necessary to be able to characterize the complexity of the dissipative channels used to prepare these states. 
	
	Given the Killing-Cartan inner product on $\mathfrak{g}$, we can define the following Lipschitz norm:
	\begin{definition}\label{definition:HörmanderLipschitz}
		Let $H = \{X_1,...,X_n\}\ssubset \mathfrak{g}$ be a Hörmander system (c.f. Definition \ref{definition:Hörmander}) where the generators are orthonormal under the Killing-Cartan form. In addition, fix a finite von Neumann algebra along with its canonical trace $(\mc N, \tau)$. The Lipschitz norm on $L_\infty(G, \mc N)$ is given by:
		\begin{equation}
			\Gamma_H(f):= ||\sum_{1\leq i \leq n}|[X_i, f]|^2|| = ||\sum_{1\leq i\leq n}\langle[X_i, f], [X_i, f]\rangle_G||
		\end{equation}
		where $\langle\cdot,\cdot\rangle_G$ denotes the bi-invariant metric on $G$ induced by the Killing-Cartan form and $||\cdot||$ is the norm on the von Neumann algebra $L_\infty(G,\mc N)$.
	\end{definition}
	In fact, the definition only depends on the subspace spanned by the generators in the Hörmander system.
	\begin{lemma}\label{lemma:basisIndep}
		Let $H':=\{Y_j\}_{1\leq j \leq n}$ be another orthonormal basis of the subspace $\mathfrak{g}_H:=\text{span}\{X_i: X_i\in H\}$. Then $\Gamma_{H'}(f) = \Gamma_H(f)$ for any finite von Neumann algebra $\mc N$ and any function $f\in L_\infty(G,\mc N)$.
	\end{lemma}
	\begin{proof}
		This is simple change-of-basis calculation. Since the norm we used through-out is the Killing-Cartan norm, we have:
		\begin{align*}
			\begin{split}
				\sum_{1\leq j \leq n}|[Y_j, f]|^2 = \sum_{1\leq j \leq n, 1\leq i \leq n}\langle\gamma_{ji}[X_i, f], \gamma_{ji}[X_i, f]\rangle = \sum_{1\leq i \leq n}\langle[X_i, f], [X_i, f]\rangle = \sum_{1\leq i \leq n}|[X_i, f]|^2
			\end{split}
		\end{align*}
		where $Y_j = \sum_{1\leq i \leq n}\gamma_{ji}X_i$ is the change of basis, and the matrix $(\gamma_{ji})_{ji}$ is an orthogonal matrix under the Killing-Cartan form. 
	\end{proof}
	Using the Carnot-Caratheodory metric, we can define a notion of diameter of a Lie group:
	\begin{definition}\label{CarnotCaratheodoryDiam}
		Let $H$ be a Hörmander system of $G$, then the Carnot-Caratheodory diameter of $G$ is given by:
		\begin{equation}
			diam_H(G) := \sup_{x,y\in G}d_H(x,y)\pl.
		\end{equation}
	\end{definition}
	The Carnot-Caratheodory metric gives us a resource-dependent way to measure the distance of a given unitary operator to the identity: $d_H(x,1)$. This distance depends on the Hörmander system and thus depends on the resources. The longer the distance, the more resource one needs to prepare the operator $x$. This is exactly the original geometric insight of Nielsen \cite{N1}. 
	\begin{remark}\label{remark:subspaceFreedom}
		One intriguing fact is that the Lipschitz complexity induced by the Lipschitz metric $\Gamma_H$ only depends on the subspace spanned by $H$ (c.f. Lemma \ref{lemma:basisIndep}). This observation gives us additional degrees of freedom to potentially provide tighter bounds on the complexity. 
	\end{remark}
	Given a finite dimensional unitary representation $\pi:G\rightarrow U(n)$, $n \in \mb N$ the Hörmander system $H$ induces another system $d\pi(H)$ in the Lie algebra $\mathfrak{su}(n)$. $d\pi(H)$ is a Hörmander system for the image $\pi(G)$, but it may not be a Hörmander system for the full Lie group $U(n)$. However, the following dimension-free estimates hold (independent of the dimension of the representation):
	\begin{theorem}\label{theorem:complexityDiam}
		Let $f:G\rightarrow \mathbb{R}^+$ be a normalized function $\int_G f(g)dg = 1$. Let $\pi:G\rightarrow U(n)$ be a finite-dimensional unitary representation and let $H\ssubset \mathfrak{g}$ be a Hörmander system of $G$. Then we have:
		\begin{enumerate}
			\item Consider the channel:
			\begin{equation}
				\Phi_{f,\pi}:L_1(\mb M_n)\rightarrow L_1(\mb M_n):\rho \mapsto \int_G f(g)\pi(g)\rho\pi(g)^*dg
			\end{equation}
			where $L_1(\mb M_n)$ is the trace-class operators on the $n$-dimensional Hilbert space. Then we have:
			\begin{equation}
				C_H(\Phi_{f,\pi}) \leq diam_H(G)
			\end{equation}
			where $C_H(\Phi_{f,\pi}) = ||\Phi_{f,\pi} - id: L_1(\mb M_n)\rightarrow (\mathbb{B}(N), ||\cdot||_{\Gamma_H})||$ is the complexity of the channel $\Phi_{f,\pi}$ using the Lipschitz norm $||\cdot||_H$ induced by the Hörmander system $H$.
			\item Fix an orthonormal basis $\{X_i\}_{1\leq i \leq n}$ of the subspace $\text{span}\{H\}$ spanned by the Hörmander system. Consider the associated symmetric Lindbladian:
			\begin{equation}
				\mathcal{L}_H:\mb M_n\rightarrow \mb M_n: x\mapsto \sum_{1\leq i \leq n}[d\pi(X_i),[d\pi(X_i), x]]\pl.
			\end{equation}
			Then we have:
			\begin{equation}
				C_H(\exp(-t\mathcal{L}_H)) \leq C\sqrt{t}\Gamma(\frac{\delta_H +3}{2})
			\end{equation}
			where $\Gamma(\cdot)$ is the usual Gamma function, $\delta_H$ is the Hausdorff dimension of $G$ under the metric topoology induced by the Carnot-Caratheodory metric, and $C>0$ is an absolute constant. 
		\end{enumerate} 
	\end{theorem}
	\begin{proof}
		For the first statement, the proof is analogous to the original Nielsen's argument of complexity upper bound \cite{N1}. For every unitary $\pi(g)\in U(n)$, fix a geodesic $\gamma_u(t)$ subordinated to the Carnot-Caratheodory metric such that $\gamma_g(0) = 1$ and $\gamma_g(1) = \pi(g)$. Then for $x \in \mathcal{A}:=\{x\in \mb M_n: E_{fix}x = 0\}$ where $E_{fix}$ is the projection onto the fixed point algebra of $\Phi_{f,\pi}$, we consider $x_g(t) := \gamma_g(t)^*x\gamma_g(t)$. And we have:
		\begin{align}
			\begin{split}
				||x_g(1) - x_g(0)||_\infty &\leq \int_0^1 ||[\gamma'_g(t), x]||_\infty dt = \int_0^1 ||\sum_{1\leq i \leq n}[g_i(t)X_i(t), x]||_\infty dt
				\\
				&\leq \Gamma_H(x)^{1/2}\int_0^1||\sum_{1\leq i \leq n}|g_i(t)|^2||^{1/2}dt = \Gamma_H(x)^{1/2}\int_0^1||\gamma'_g(t)|| dt
			\end{split}
		\end{align}
		where $\{X_i\}_{1\leq i \leq n}$ is an orthonormal basis of the subspace spanned by the Hörmander system $H$ and $\gamma'_g(t) = \sum_{1\leq i \leq n}g_i(t)X_i(t)$ is the orthonormal expansion of the derivative in terms of the left-translated basis $\{X_i(t)\}_{1\leq i \leq n}$. Then by convexity, we have:
		\begin{align}
			\begin{split}
				||\Phi_{f,\pi}^*(x) - x||_\infty &\leq  \int_G dg f(g)||\pi(g)^*x\pi(g) - x||_\infty = \int_G dg f(g)||x_g(1) - x_g(0)||_\infty  \\&\leq \int_Gdg f(g)\Gamma_H(x)^{1/2}d_H(x, 1)
			\end{split}
		\end{align}
		where $d_H(\cdot, 1)$ is the Carnot-Caratheodory distance from the identity. Hence by the definition of the Lipschitz complexity, we have: $C_H(\Phi_{f,\pi}) \leq \int_G dgf(g)d_H(g,1) \leq diam_H(G)\int_G dgf(g) = diam_H(G)$.
		
		For the second statement, we need the point-wise heat kernel estimates on a unimodular Lie group \cite{VSCC}. By a transference principle \cite{GJL}, we consider the unital $*$-homomorphism: $\varphi: \mb M_n\rightarrow L_\infty(G,\mb M_n):x\mapsto \varphi(x)(g) = \pi(g)^*x\pi(g)$. And the transfered Lindbladian is given by: $\mathcal{L}_\Delta(f)(g) = \Delta_Hf(g)$ where $\Delta_H = \sum_{1\leq i \leq n}X_i^2$ is the Laplacian associated with the orthonormal basis $\{X_i\}_i$. Denote the corresponding heat kernel as $h_t^\Delta(g,h)$, then the semi-group acts on $x\in\mb M_n$ by convolution:
		\begin{equation*}
			\exp(-t\mathcal{L}^*_H)x = \int_Gdg h_t^\Delta(1, g^{-1})\pi(g)^*x\pi(g)\pl.
		\end{equation*}
		Thus by the first statement, we have:
		\begin{align}
			\begin{split}
				C_H(\exp(-t\mathcal{L}_H)) \leq \int_G dg h_t^\Delta(1,g^{-1})d_H(1,g)\pl.
			\end{split}
		\end{align}
		Using the pointwise heat kernel estimate \cite{VSCC}, we have:
		\begin{align}
			\begin{split}
				C_H(\exp(-t\mathcal{L}_H)) &\leq C\int_G dg t^{-\delta_H/2}\exp(-\frac{d_H(1,g)^2}{Ct})d_H(1,g)
				\\
				&\lessapprox \int_0^{diam_H(G)}dr r^{\delta_H-1} t^{-\delta_H/2}\exp(-\frac{r^2}{Ct})r
				\\
				&\lessapprox \sqrt{t}\int_0^\infty ds s^{\delta_H/2}\exp(-s) = C\sqrt{t}\Gamma(\frac{\delta_H +2}{2})
			\end{split}
		\end{align} 
		where $s:= r^{2}/t$. 
	\end{proof}	
	
	\section{More examples: Clifford Gates as Resources}\label{section:Clifford}
	In this section, we consider a special type of resources, namely the Clifford gates. In the fundamental representation of a complex Clifford algebra $C\ell_n(\mathbb{C})\cong M_{2^{[n/2]}}(\mathbb{C})$, the generators of the Clifford algebra can be taken to satisfy the following conditions:
	\begin{equation}
		\{c_j,c_k\} = 2\delta_{jk} \text{ , }c_j = c_j^*\pl.
	\end{equation}
	In particular, each $c_j$ is both unitary and self-adjoint. As such, the set $\{c_j\}_{1\leq j \leq n}$ induces a set of discrete quantum resources and a set of continuous quantum resources (c.f. Section \ref{section:axiom}). For $1\leq j \leq n$, let $a_j := -i\log c_j$ be the self-adjoint generator of the unitary $c_j$. Then denote the discrete Clifford resource set as: $\mathcal{S}_n:=\{c_j\}_{1\leq j \leq n}$ and denote the continuous Clifford resource set as: $\Delta_n:=\{a_j\}_{1\leq j\leq n}$. The two sets of resources define two different Lipschitz norms on the matrix algebra $M_{2^{[n/2]}(\mathbb{C})}$:
	\begin{equation}
		|||x|||_\mathcal{S} := \sup_{1\leq j \leq n}||c_jxc_j^* - x|| = \sup_j||[c_j,x]||\pl,
	\end{equation}
	\begin{equation}
		|||x|||_\Delta := \sup_{1\leq j \leq n}||[a_j,x]||\pl.
	\end{equation}
	The two corresponding notions of Lipschitz complexities have a simple relation:
	\begin{lemma}\label{lemma:discreteContClifford}
		On the matrix algebra $M_{2^{[n/2]}}(\mathbb{C})$ the following statements are true:
		\begin{enumerate}
			\item The annihilator subspaces of $\mathcal{S}_n$ and $\Delta_n$ are the same:
			\begin{equation}
				[a_j, x] = 0 \text{ , }\forall 1\leq j \leq n \iff [c_j,x] = 0 \text{ , }\forall 1\leq j \leq n\pl;
			\end{equation}
			\item There exists a contraction:
			\begin{equation}
				id:(\mathcal{A},|||\cdot|||_\Delta)\rightarrow(\mathcal{A},|||\cdot|||_\mathcal{S})
			\end{equation}
			where $\mathcal{A}$ is the tracial-orthogonal subspace to the annihilator subspace;
			\item For channel $\Phi \in CP_{\mathcal{S}}(M_{2^{[n/2]}}(\mathbb{C}), M_{2^{[n/2]}}(\mathbb{C}))\cap CP_\Delta(M_{2^{[n/2]}}(\mathbb{C}), M_{2^{[n/2]}}(\mathbb{C}))$, the Lipschitz complexity associated with the continuous Clifford resource is bounded above by the Lipschitz complexity associated with the discrete Clifford resource:
			\begin{equation}
				C_\Delta(\Phi) \leq C_\mathcal{S}(\Phi)\pl.
			\end{equation}
		\end{enumerate}
		Similar statements hold for correlation-assisted complexity where the identity map is a complete contraction.
	\end{lemma}
	\begin{proof}
		The first statement follows directly from the definition of the Clifford generators. Then the Lipschitz space is given by $\mathcal{A} = \{x\in M_{2^{[n/2]}}(\mathbb{C}): [a_j, x] = 0 \text{ , }\forall 1\leq j \leq n\} = \{x\in M_{2^{[n/2]}}(\mathbb{C}): [c_j,x ] = 0 \text{ , }\forall 1\leq j \leq n\}$. To see the contraction, we have:
		\begin{align}
			\begin{split}
				|||x|||_\mathcal{S} &= \sup_j ||c_jxc_j^* - x|| = \sup_j ||\exp(ia_j)x\exp(-ia_j) - x||\\&
				\leq \sup_j \int_0^1 dt ||\exp(ita_j)[a_j, x]\exp(-ita_j)|| = \sup_j \int_0^1 dt||[a_j,x]|| = |||x|||_\Delta\pl.
			\end{split}
		\end{align}
		A similar proof shows that the identity map is a complete contraction. 
		
		For the last statement, we have:
		\begin{align}
			\begin{split}
				C_\Delta(\Phi) &= ||\Phi^*-id:(\mathcal{A},|||\cdot|||_\Delta)\rightarrow M_{2^{[n/2]}}(\mathbb{C})||
				\\
				&=||(\Phi^*-id)\circ id:(\mathcal{A},|||\cdot|||_\Delta)\rightarrow(\mathcal{A},|||\cdot|||_\mathcal{S})\rightarrow M_{2^{[n/2]}}(\mathbb{C})||
				\\
				&\leq C_\mathcal{S}(\Phi)||id:(\mathcal{A},|||\cdot|||_\Delta)\rightarrow(\mathcal{A},|||\cdot|||_\mathcal{S})|| \leq C_\mathcal{S}(\Phi)\pl.
			\end{split}
		\end{align}
		Using complete contraction, the analogous statement for correlation-assisted complexity also holds.
	\end{proof}
	The Clifford resources have been considered before \cite{PMTL} where the authors used Pauli gates (a particular example of Clifford resources) to define a noncommutative Lipschitz norm. In the last part of this section, we consider the discrete Clifford resources made up of Pauli gates on $n$ qubits: $\mathcal{P}:=\{\sigma^X_j, \sigma^Y_j, \sigma^Z_j\}_{1\leq j \leq n}$. We show that the induced Lipschitz norm $|||\cdot|||_\mathcal{P}$ is equivalent to the Lipschitz norm considered in \cite{PMTL}. Recall the partical trace on $j$-th qubit can be realized as a simple average of adjoint actions by Pauli gates:
	\begin{equation}
		\tr_j(x) = \frac{x + \sigma^X_jx\sigma^X_j + \sigma^Y_jx\sigma^Y_j + \sigma^Z_jx\sigma^Z_j}{4}
	\end{equation}
	where $x\in M_{2^n}(\mathbb{C}) = \bigotimes_{1\leq i\leq n}M_2(\mathbb{C})$. The Lipschitz norm considered in \cite[Section V]{PMTL} can be written in the following equivalent form \footnote{The original definition considers the dual version: $||x||_L = \sup_j \sup_{\rho:\tr_j\rho = 0}|\tr(\rho x)|$. Since the partial trace is tracial self-adjoint, we can rewrite the norm as: $\sup_{\rho:tr_j\rho = 0}|tr(\rho x)| = \sup_\rho|tr((x_tr_j x)\rho)| = ||x-\tr_jx||$.}:
	\begin{equation}
		||x||_L:= \sup_{1\leq j \leq n}||x-\tr_jx||_\infty\pl.
	\end{equation}
	Then it is clear that the following upper bound holds:
	\begin{align}
		\begin{split}
			||x||_L &\leq\frac{1}{4}\sup_j\big( ||x-\sigma^X_jx\sigma^X_j||+ ||x - \sigma^Y_jx\sigma^Y_j||+ ||x - \sigma^Z_jx\sigma^Z_j||\big)\\&\leq \frac{3}{4}\sup_j\max\{||x-\sigma^X_jx\sigma^X_j||, ||x - \sigma^Y_jx\sigma^Y_j||, ||x - \sigma^Z_jx\sigma^Z_j||\}\\& =\frac{3}{4}|||x|||_\mathcal{P}\pl.
		\end{split}
	\end{align}
	On the other hand, we can also bound $||x||_L$ from below by the Lipschitz norm $||x||_\mathcal{P}$. Recall that for $x\in M_{2^n}(\mathbb{C})$ and each $1\leq j \leq n$, we can decompose $x$ into a sum:
	\begin{equation}
		x = x^{\widehat{j}}_{id}\otimes id_j + x^{\widehat{j}}_X\otimes \sigma^X_j + x^{\widehat{j}}_Y\otimes\sigma^Y_j + x_Z^{\widehat{j}}\otimes\sigma^Z_j
	\end{equation}
	where the superscript $\hat{j}$ indicates that the matrix acts on all qubits except the $j$-th one. Then we have: $x - \tr_jx = x^{\widehat{j}}_X\otimes \sigma^X_j + x^{\widehat{j}}_Y\otimes\sigma^Y_j + x_Z^{\widehat{j}}\otimes\sigma^Z_j$. In addition, the following inequality holds:
	\begin{align}
		\begin{split}
			||x - \sigma^X_j x\sigma^X_j || &= 2||x^{\widehat{j}}_Y\otimes\sigma^Y_j + x_Z^{\widehat{j}}\otimes\sigma^Z_j|| \leq 2||x - \tr_jx|| + 2|\tr(\sigma^X_j x)|\pl.
		\end{split}
	\end{align}
	Similarl inequalities hold for $\sigma^Y_j$ and $\sigma^Z_j$. Therefore we have:
	\begin{align}
		\begin{split}
			|||x|||_\mathcal{P}|&=\sup_j\max\{||x- \sigma^X_jx\sigma^X_j||, ||x - \sigma^Y_jx\sigma^Y_j||, ||x - \sigma^Z_jx\sigma^Z_j||\} \\&\leq 2\sup_j \big(||x-\tr_jx|| + \max\{|\tr(\sigma^X_jx)|, |\tr(\sigma^Y_jx)|, |\tr(\sigma^Z_jx)|\}\big)\\
			&\leq 2\sup_j\big(||x-\tr_jx|| + \sup_{\rho:\tr_j\rho = 0 }|\tr(\rho x)|\big) =4\sup_j||x-\tr_jx|| = 4||x||_L\pl.
		\end{split}
	\end{align}
	Combining these estimates, we have that:
	\begin{lemma}\label{lemma:normEquivalence}
		Independent on the number of qubits, the Lipschitz norm in \cite{PMTL} is equivalent to the Lipschitz norm associated with the discrete Pauli resources:
		\begin{equation}
			\frac{1}{4}|||x|||_\mathcal{P}\leq ||x||_L \leq \frac{3}{4}|||x|||_\mathcal{P}
		\end{equation}
		where $x\in M_{2^n}(\mathbb{C})$ and $\mathcal{P} = \{\sigma^X_j,\sigma^Y_j, \sigma^Z_j\}_{1\leq j \leq n}$ is the set of discrete Pauli resources.
	\end{lemma}
	In particular, the Wasserstein complexity considered in \cite{PMTL} is a particular example of the complexity considered in this paper. Hence our axiomatic framework can be understood as a generalization of the results in \cite{PMTL}.
	\nocite{*}
	\bibliographystyle{alpha}
	\bibliography{complexity}
	
\end{document}