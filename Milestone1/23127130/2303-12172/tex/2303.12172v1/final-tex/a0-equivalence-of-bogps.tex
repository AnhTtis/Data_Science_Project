\section{Equivalence of $\BOGP$ and $\BOGP_{\loc,0}$}
\label{sec:equivalence-of-bogps}

In this section, we prove Proposition~\ref{prop:bogp-equivalent} that $\BOGP = \BOGP_{\loc,0}$.
We introduce two other limits $\BOGP_{\den}$ and $\BOGP_{\loc}$, as follows (restating $\BOGP$ and $\BOGP_{\loc,0}$ for convenience). 
\begin{align*}
    \BOGP
    &= 
    \lim_{D\to\infty}
    \lim_{\eta\to 0}
    \lim_{k\to\infty}
    \sup_{\vchi \in \bbI(0,1)^\sS}
    \inf_{\uvphi=\vchi(\up)}
    \limsup_{N\to\infty}
    \fr{1}{N} 
    \bbE \sup_{\ubsig \in \cQ(\eta)}
    \cH_N(\ubsig), \\
    \BOGP_{\den}
    &= 
    \lim_{D\to\infty}
    \lim_{\eta\to 0}
    \lim_{k\to\infty}
    \sup_{\substack{
        \vchi \in \bbI(0,1)^\sS \\ 
        \text{$1/D^2$-separated}
    }}
    \inf_{\substack{
        \uvphi=\vchi(\up) \\
        \text{$6r/D$-dense}
    }}
    \limsup_{N\to\infty}
    \fr{1}{N} 
    \bbE \sup_{\ubsig \in \cQ(\eta)}
    \cH_N(\ubsig), \\
    \BOGP_{\loc}
    &=
    \lim_{D\to\infty}
    \lim_{\eta\to 0}
    \lim_{k\to\infty}
    \sup_{\substack{
        \vchi \in \bbI(0,1)^\sS \\ 
        \text{$1/D^2$-separated}
    }}
    \inf_{\substack{
        \uvphi=\vchi(\up) \\
        \text{$6r/D$-dense}
    }}
    \limsup_{N\to\infty}
    \fr{1}{N} 
    \bbE \sup_{\ubsig \in \cQ_{\loc}(\eta)}
    \cH_N(\ubsig), \\
    \BOGP_{\loc,0}
    &=
    \lim_{D\to\infty}
    \lim_{k\to\infty}
    \sup_{\substack{
        \vchi \in \bbI(0,1)^\sS \\ 
        \text{$1/D^2$-separated}
    }}
    \inf_{\substack{
        \uvphi=\vchi(\up) \\ 
        \text{$6r/D$-dense}
    }}
    \limsup_{N\to\infty}
    \fr{1}{N} 
    \bbE \sup_{\ubsig \in \cQ_{\loc}(0)}
    \cH_N(\ubsig).
\end{align*}
In the last three lines, the limits in $k, \eta$ are clearly decreasing, but the limits in $D$ are not, so the existence of these limits needs to be proven.
Proposition~\ref{prop:bogp-equivalent} follows from the following propositions.
\begin{proposition}
    \label{prop:bogp-den}
    The limit $\BOGP_{\den}$ exists and $\BOGP = \BOGP_{\den}$. 
\end{proposition}

\begin{proposition}
    \label{prop:bogp-loc}
    The limit $\BOGP_{\loc}$ exists and $\BOGP_{\den} = \BOGP_{\loc}$. 
\end{proposition}

\begin{proposition}
    \label{prop:bogp-loc0}
    The limit $\BOGP_{\loc,0}$ exists and $\BOGP_{\loc} = \BOGP_{\loc,0}$. 
\end{proposition}

\subsection{Equivalence of $\BOGP$ and $\BOGP_{\den}$}

Let $\up' = (p_0,\ldots,p_{D'})$ and $\uvphi' = (\vphi'_0,\ldots,\vphi'_{D'})$. 
Say $(\up',\uvphi')$ \textbf{refines} $(\up,\uvphi)$ if there exists $0 \le a_0 < \cdots < a_D \le D'$ such that $(p_d,\vphi_d) = (p'_{a_d}, \vphi'_{a_d})$ for all $0\le d\le D$.

\begin{lemma}
    \label{lem:refinement}
    The value $\bbE \max_{\ubsig \in \cQ(\eta)} \cH_N(\ubsig)$ is decreasing under refinement. 
    That is, if $(\up',\uvphi')$ refines $(\up,\uvphi)$, then for any $k,\eta$,
    \[
        \bbE \max_{\ubsig \in \cQ^{k,D,\uvphi}(\eta)} \cH_N^{k,D,\up} (\ubsig)
        \ge \bbE \max_{\ubsig \in \cQ^{k,D',\uvphi'}(\eta)} \cH_N^{k,D',\up'} (\ubsig).
    \]
\end{lemma}
\begin{proof}
    Let $I = \{a_0,\ldots,a_D\}$ and $J = [D'] \setminus I$. 
    Define an equivalence relation $\bowtie$ on $\bbL(k,D')$ by $u\bowtie v$ if $u_d=v_d$ for all $d\in J$. 
    Let 
    \[
        \cQ' = 
        \lt\{
            \ubsig \in \cB_N^{\bbL(k,D')} : 
            \norm{\vR(\bsig(u^1),\bsig(u^2)) - \vphi'_{u^1\wedge u^2}}_\infty 
            \le \eta,
            ~\forall u^1 \bowtie u^2
        \rt\}
    \]
    be the superset of $\cQ^{k,D',\uvphi'}$ where we only enforce the overlap constraint for $u^1\bowtie u^2$.
    Then
    \begin{align*}
        \bbE \max_{\ubsig \in \cQ^{k,D',\uvphi'}(\eta)} \cH_N^{k,D',\up'} (\ubsig)
        &\le \bbE \max_{\ubsig \in \cQ'} \cH_N^{k,D',\up'} (\ubsig) \\
        &= \bbE \max_{\ubsig \in \cQ'} \fr{1}{k^{D'-D}} \sum_{u_J \in [k]^{D'-D}} \fr{1}{k^D} \sum_{u_I \in [k]^D} \cH_N^{k,D',\up'} (\ubsig) \\
        &= \bbE \max_{\ubsig \in \cQ^{k,D,\uvphi}(\eta)} \cH_N^{k,D,\up} (\ubsig).
    \end{align*}
\end{proof}

\begin{proof}[Proof of Proposition~\ref{prop:bogp-den}]
    Let $\BOGP_{\den}^+$ and $\BOGP_{\den}^-$ be $\BOGP_{\den}$ where the outer limit in $D$ is replaced by $\limsup$ and $\liminf$, respectively.
    We will separately prove $\BOGP \ge \BOGP_{\den}^+$ and $\BOGP \le \BOGP_{\den}^-$.

    Fix any $D,k,\eta$, $1/D^2$-separated $\vchi$, and (not necessarily $6r/D$-dense) $\up,\uvphi$ satisfying $\uvphi = \vchi(\up)$.
    Let $\delta = (r+1)/D$.
    Let $\tp_0 = p_0$, and define a sequence $\tp_1,\ldots,\tp_{\tilde D}$, where $\tp_{d+1}$ is the smallest $p\in [\tp_d,1]$ such that 
    \[
        \max\lt(p-\tp_d, \norm{\vchi(p) - \vchi(\tp_d)}_\infty\rt) \ge \delta
    \]
    if such $p$ exists.
    Let $\tilde D$ be the first index $d$ such that no such $p$ exists.
    Note that if $\Sigma_d = \tp_d + \tnorm{\vchi(\tp_d)}_1$, then $0\le \Sigma_d \le r+1$ and $\Sigma_{d+1} - \Sigma_{d} \ge \delta$ for all $d$.
    Thus $\tilde D \le (r+1)/\delta = D$.

    Consider either $D'=2D$ or $D'=2D+1$.
    Let $\up'$ be the sorted union of $\{p_0,\ldots,p_D\}$, $\{\tp_1,\ldots,\tp_{\tilde D}\}$, and (if necessary) additional arbitrary $p\in [0,1]$, so that $\up'$ has length $D'$.
    Define $\uvphi' = \vchi(\up')$.
    Since $(\up',\uvphi')$ refines $(\up,\uvphi)$, Lemma~\ref{lem:refinement} implies
    \[
        \bbE \max_{\ubsig \in \cQ^{k,D,\uvphi}(\eta)} \cH_N (\ubsig)
        \ge \bbE \max_{\ubsig \in \cQ^{k,D',\uvphi'}(\eta)} \cH_N^{k,D',\up'} (\ubsig).
    \]
    Moreover, one can check that $\delta \le 6r/D'$, so $(\up',\uvphi')$ is $6r/D'$-dense.
    Thus, if $f(D)$ and $g(D)$ are the quantities inside the outer limits of $\BOGP$ and $\BOGP_{\den}$, we have shown $f(D) \ge g(2D), g(2D+1)$ (as taking the supremum over not necessarily $1/D^2$-separated $\vchi$ can only increase $f(D)$). 
    This implies $\BOGP \ge \BOGP_{\den}^+$.

    For the other direction, fix $D,k,\eta$ and (not necessarily $1/D^2$-separated) $\vchi$. 
    Define
    \[
        \vchi'(p) = (1-D^{-2}) \vchi(p) + D^{-2} \vone,
    \]
    so $\vchi'$ is $1/D^2$-separated.
    Consider any $6r/D$-dense $(\up,\uvphi')$ with $\uvphi' = \vchi'(\up)$, and let $\uvphi = \vchi(\up)$. 

    Let $\ubsig \in \cQ^{k,D,\uvphi}(\eta)$.
    Let $\bx$ satisfy $\vR(\bx,\bx) = \vone$ and $\vR(\bx,\bsig(u))=\vzero$ for all $u\in \bbL$.
    Define
    \[
        \brho(u) = \sqrt{1-D^{-2}}\bsig(u) + D^{-1} \bx,
    \]
    so that for all $u, v \in \bbL$,
    \[
        \norm{\vR(\brho(u),\brho(v)) - \vphi'_{u\wedge v}}_\infty
        = (1-D^{-2}) \norm{\vR(\bsig(u),\bsig(v)) - \vphi_{u\wedge v}}_\infty
        \le \eta.
    \]
    Thus $\ubrho \in \cQ^{k,D,\uvphi'}(\eta)$, and we can easily check that
    \[
        \fr{1}{\sqrt{N}} \norm{\brho(u) - \bsig(u)}_2 = O(D^{-2})
    \]
    for all $u\in \bbL$.
    By Proposition~\ref{prop:gradients-bounded}, with probability $1-e^{-\Omega(N)}$ we have $\HNp{u}\in K_N$ for all $u\in \bbL$. 
    On this event, 
    \[
        \lt|\fr{1}{N} \cH_N(\ubrho) - \fr{1}{N} \cH_N(\ubsig)\rt| \le CD^{-2}
    \]
    for some $C>0$, and so 
    \[
        \fr{1}{N} \sup_{\ubrho \in \cQ^{k,D,\uvphi'}(\eta)} \cH_N(\ubrho)
        + CD^{-2}
        \ge 
        \fr{1}{N} \sup_{\ubsig \in \cQ^{k,D,\uvphi}(\eta)} \cH_N(\ubsig).
    \]
    By Lemma~\ref{lem:bogp-subgaussian}, both sides of this inequality are subgaussian with fluctuations $O(N^{-1/2})$, so the contribution from the complement of this event is $o_N(1)$, and 
    \[
        \limsup_{N\to\infty} 
        \fr{1}{N} \bbE \sup_{\ubrho \in \cQ^{k,D,\uvphi'}(\eta)} \cH_N(\ubrho)
        + CD^{-2}
        \ge 
        \limsup_{N\to\infty}
        \fr{1}{N} \bbE \sup_{\ubsig \in \cQ^{k,D,\uvphi}(\eta)} \cH_N(\ubsig).
    \]
    Thus $f(D) \le g(D) + CD^{-2}$ (as taking the infimum over $(\up,\uvphi)$ that are not necessarily the image of a $6r/D$-dense $(\up,\uvphi')$ under the above transformation can only decrease $f(D)$).
    This implies $\BOGP \le \BOGP_{\den}^-$.
\end{proof}

\subsection{Equivalence of $\BOGP_{\den}$ and $\BOGP_{\loc}$}

\begin{lemma}
    \label{lem:recursive-barycenter}
    We have that $\cQ(\eta) \subseteq \cQ_{\loc}(\eta+\fr{2}{k})$.
\end{lemma}
\begin{proof}
    Consider any $\ubsig \in \cQ(\eta)$. 
    We define $\ubrho \in \cB_N^\bbT$ by $\brho(u) = \bsig(u)$ if $u\in \bbL$, and
    \[
        \brho(u) = \fr{1}{k} \sum_{i=1}^k \brho(ui)
    \]
    for $u\in \bbT \setminus \bbL$.
    We will show that $\ubrho \in \cQ_{\loc+}(\eta+\fr{2}{k})$, and so $\ubsig \in \cQ_{\loc}(\eta+\fr{2}{k})$.
    
    Let $v\succeq u$ denote that $v$ is a descendant of $u$ in $\bbT$.
    Consider any non-leaf $u\in \bbT$ and two of its children $ui,uj$, for $i\neq j$. 
    For any $s\in \sS$,
    \begin{equation}
        \label{eq:sibling-overlap}
        |R_s(\brho(ui),\brho(uj)) - \phi_{|u|}^s|
        \le 
        \fr{1}{k^{2(D-|u|)}}
        \sum_{\substack{v,v' \in \bbL \\ v \succeq ui, v' \succeq uj}}
        |R_s(\bsig(v),\bsig(v')) - \phi_{|u|}^s|
        \le \eta.
    \end{equation}
    Moreover,
    \[
        |R_s(\brho(ui),\brho(u)) - \phi_{|u|}^s|
        \le 
        \fr1k 
        \sum_{j=1}^k 
        |R_s(\bsig(ui),\bsig(uj)) - \phi_{|u|}^s|
        \le 
        \eta + \fr{2}{k},
    \]
    where we bounded the terms $j\neq i$ by \eqref{eq:sibling-overlap} and the term $j=i$ crudely by $2$.
    Thus,
    \[
        |R_s(\brho(u),\brho(u)) - \phi_{|u|}^s|
        \le 
        \fr{1}{k}
        \sum_{i=1}^k
        |R_s(\bsig(ui),\bsig(u)) - \phi_{|u|}^s|
        \le 
        \eta + \fr{2}{k}.
    \]
\end{proof}

For $k'\le k$, define a \emph{$k'$-ary subtree} of $\bbT$ to be a subset $T\subseteq \bbT$ isometric to $\bbT(k',D)$.
The following fact is clear from the definition of $\cQ_{\loc}(\eta)$. 
\begin{fact}
    \label{fac:subtree-satisfy-local}
    Let $T\subseteq \bbT$ be a $k'$-ary subtree with leaf set $L$.
    If $\ubsig \in \cQ_{\loc}(\eta)$, then $(\bsig(u))_{u\in L} \in \cQ_{\loc}^{k',D,\uvphi}(\eta)$.
\end{fact}
\begin{proof}
    There exists $\ubrho \in \cQ_{\loc+}(\eta)$ such that $\brho(u) = \bsig(u)$ for all $u\in \bbL$.
    Then $(\brho(u))_{u\in T} \in \cQ_{\loc+}^{k',D,\uvphi}(\eta)$, which implies the result.
\end{proof}

\begin{lemma}
    \label{lem:prune-global-constraints}
    Let $k'$ be the largest integer solution to $D(k')^D \le \min(\sqrt{k}, \eta^{-1})$. 
    If $\ubsig \in \cQ_{\loc}(\eta)$, there exists a $k'$-ary subtree $T$ of $\bbT$ with leaf set $L$ such that $(\bsig(u))_{u\in L} \in \cQ^{k',D,\uvphi}(CD^2 (k^{-1/4} + \eta^{1/4}))$, for some $C >0$. 
\end{lemma}
\begin{proof}
    Let $\ubrho \in \cQ_{\loc+}(\eta)$ such that $\brho(u) = \bsig(u)$ for all $u\in \bbL$. 
    We will construct $T$ by a breadth-first search: we start from $T = \{\emptyset\}$ and each step \emph{process} a leaf $u$ of $T$ by adding $k'$ children of $u$ to $T$, until all leaves of $T$ are of depth $D$.
    
    Suppose we are currently processing vertex $u$. 
    Let $V =  \{\brho(v) : v\in T\}$ and $S = \text{span}(V)$; note that $|V| \le D(k')^D \le \min(\sqrt{k}, \eta^{-1})$.
    Let $P_S$ denote the projection operator onto $S$. 
    For $i\in [k]$, write $\bx^i = \fr{1}{\sqrt{N}} (\brho(ui) - \brho(u))$, and note $\norm{\bx^i}_2 \le 2$.
    Then
    \[
        \fr{1}{N} \sum_{i=1}^k \norm{P_S(\brho(ui)-\brho(u))}_2^2
        = 
        \sum_{i=1}^k \norm{P_S \bx^i}_2^2
    \]
    is upper bounded by the sum of the top $|V|$ eigenvalues of the Gram matrix $\bM = (\la \bx^i,\bx^j\ra)_{i,j=1}^k$.
    However, for $i\neq j$, 
    \[
        |\la \bx^i,\bx^j\ra|
        = \fr{1}{N} |\la \brho(ui)-\brho(u), \brho(uj)-\brho(u)\ra|
        \le \sum_{s\in \sS} \lambda_s |R_s(\brho(ui)-\brho(u), \brho(uj)-\brho(u))| 
        \le 4\eta,
    \]
    while $|\la \bx^i,\bx^i\ra| \le 4$.
    So, if we let $\bM = \bD + \bA$ where $\bD = \diag(\bM)$, and let $a_1 \ge \cdots \ge a_{|V|}$ be the top $|V|$ eigenvalues of $\bA$, then the sum of the top $|V|$ eigenvalues of $\bM$ is upper bounded by $4|V| + \sum_{i=1}^{|V|} a_i$. 
    However,
    \[
        \sum_{i=1}^{|V|} a_i
        \le 
        \sqrt{|V| \sum_{i=1}^{|V|} a_i^2}
        \le 
        \sqrt{|V| \norm{\bA}_F^2} 
        \le 4k\eta \sqrt{|V|}.
    \]
    It follows that
    \[
        \fr{1}{kN} \sum_{i=1}^k \norm{P_S(\brho(ui)-\brho(u))}_2^2
        \le 
        \fr{4|V|}{k} + 4\eta \sqrt{|V|}
        \le 4(k^{-1/2} + \eta^{1/2}),
    \]
    where the last step follows from $|V| \le \min(\sqrt{k}, \eta^{-1})$.
    Thus there are $k'$ children $ui$ of $u$ such that
    \[
        \fr{1}{\sqrt{N}} \norm{P_S(\brho(ui)-\brho(u))}_2 
        \le 3(k^{-1/4} + \eta^{1/4}).
    \]
    We choose these as the children of $u$ in $T$.
    By constructing $T$ in this manner, we get that for all distinct edges $(u,ui)$, $(v,vj)$ in $T$, 
    \[
        \fr{1}{N} |\la \bsig(ui)-\bsig(u), \bsig(vj)-\bsig(v) \ra|, 
        \fr{1}{N} |\la \bsig(ui)-\bsig(u), \bsig(\emptyset) \ra| 
        \le 6(k^{-1/4} + \eta^{1/4}).
    \]
    whence 
    \[
        \norm{\vR(\bsig(ui),\bsig(u), \bsig(vj)-\bsig(v))}_\infty, 
        \norm{\vR(\bsig(ui),\bsig(u), \bsig(\emptyset))}_\infty 
        \le \fr{6(k^{-1/4} + \eta^{1/4})}{\min_s \lambda_s}.
    \]
    We now verify that $(\bsig(u))_{u\in L} \in \cQ^{k',D,\uvphi}(CD^2(k^{-1/4} + \eta^{1/4}))$.
    Consider any $u,v\in L$ with least common ancestor $w$, and let $|w|=d$. 
    Let $(u_0,\ldots,u_{D-d})$ and $(v_0,\ldots,v_{D-d})$ be the paths from $w$ to $u,v$, with $u_0=v_0=w$ and $u_{D-d}=u$, $v_{D-d}=v$, and let $(w_0,\ldots,w_d)$ be the path from $\emptyset$ to $w$, with $w_0=\emptyset$, $w_d = w$. 
    Also define as convention $\bsig(w_{-1}) = \bzero$.
    Then,
    \begin{align*}
        \norm{\vR(\bsig(u),\bsig(v)) - \vphi_d}_\infty 
        &\le 
        \norm{\vR(\bsig(w),\bsig(w)) - \vphi_d}_\infty
        + \sum_{i=1}^{D-d} \sum_{\ell=0}^d \norm{\vR(\bsig(w_\ell) - \bsig(w_{\ell-1}), \bsig(u_i)-\bsig(u_{i-1}))}_\infty \\
        &\quad + \sum_{j=1}^{D-d} \sum_{\ell=0}^d \norm{\vR(\bsig(w_\ell) - \bsig(w_{\ell-1}), \bsig(v_j)-\bsig(v_{j-1}))}_\infty \\
        &\quad + \sum_{i,j=1}^{D-d} \norm{\vR(\bsig(u_i)-\bsig(u_{i-1}),\bsig(v_j)-\bsig(v_{j-1}))}_\infty \\
        &\le CD^2(k^{-1/4} + \eta^{1/4}).
    \end{align*}
\end{proof}

\begin{lemma}
    \label{lem:prune-all-leaves-good}
    There exists $C>0$ such that with probability $1-e^{-\Omega(N)}$ over the Hamiltonians $\HNp{u}$ the following holds.
    If $\eps > 0$, $\ubsig \in \cQ_{\loc}(\eta)$, and 
    \[
        \fr{1}{N} \cH(\ubsig) \ge E,
    \]
    then for $k' = \lfloor k\eps/3CD\rfloor$, there exists a $k'$-ary subtree $T$ of $\bbT$ with leaf set $L$ such that
    \[
        \fr{1}{N} \HNp{u}(\bsig(u)) \ge E-\eps
    \]
    for all $u\in L$.
\end{lemma}
\begin{proof}
    We consider the event that $\HNp{u} \in K_N$ for all $u\in \bbL$, for $K_N$ defined in Proposition~\ref{prop:gradients-bounded}. 
    This holds with probability $1-e^{-\Omega(N)}$, and on this event, $|\HNp{u}(\bsig(u))| \le C$ for all $u\in \bbL$.
    For $u\in \bbT$ define
    \[
        F(u) = \fr{1}{Nk^{D-|u|}} 
        \sum_{\substack{v\in \bbL \\ v\succeq u}}
        H^{(v)}(\bsig(v)).
    \]
    We will show that for any $u\in \bbT \setminus \bbL$, we may find $k'$ distinct children $ui_1,\ldots,ui_{k'}$ such that $F(ui_j) \ge F(u) - \eps/D$ for all $j$.
    Indeed, we have
    \[
        F(u) = \fr{1}{k} \sum_{i=1}^k F(ui),
    \]
    and $|F(ui)| \le C$ for all $i$, so the claim follows from Markov's inequality.

    We construct the subtree $T$ recursively starting from $\emptyset$, using the above claim to select the $k'$ children of each node.
    Thus, for all $u,ui\in T$ with $ui$ a child of $u$, we have $F(ui) \ge F(u) - \eps/D$. 
    Since $F(\emptyset) = \fr{1}{N} \cH_N(\ubsig) \ge E$, the result follows.
\end{proof}

\begin{proof}[Proof of Proposition~\ref{prop:bogp-loc}]
    Let $\BOGP_{\loc}^+$ and $\BOGP_{\loc}^-$ be $\BOGP_{\loc}$ where the outer limit in $D$ is replaced by $\limsup$ and $\liminf$, respectively.
    Lemma~\ref{lem:recursive-barycenter} gives $\BOGP_{\den} \le \BOGP_{\loc}^-$, so it suffices to prove $\BOGP_{\den} \ge \BOGP_{\loc}^+$.

    Fix arbitrary $\eps>0$, $D,k,\eta$, $1/D^2$-dense $\vchi$, and $6r/D$-dense $(\up,\uvphi)$ satisfying $\uvphi = \vchi(\up)$.
    If $\ubsig \in \cQ_{\loc}(\eta)$ and $\fr{1}{N} \cH_N(\ubsig) \ge E$, then on an event with probability $1-e^{-\Omega(N)}$, Lemma~\ref{lem:prune-all-leaves-good} gives a $k'$-ary subtree $T\subseteq \bbT$ with leaf set $L$ such that $\fr{1}{N} \HNp{u}(\bsig(u)) \ge E-\eps$ for all $u\in L$.
    However, $(\bsig(u))_{u\in L}$ is itself an element of $\cQ_{\loc}^{k',D,\uvphi}(\eta)$ by Fact~\ref{fac:subtree-satisfy-local}, so Lemma~\ref{lem:prune-global-constraints} gives a $k''$-ary subtree $T' \subseteq T$ with leaf set $L'$ such that $(\bsig(u))_{u \in L'} \in \cQ^{k'',D,\uvphi}(\eta')$.
    Here $k' = \lfloor \eps / 3CD \rfloor$, $k''$ is the largest solution to $D(k'')^D \le \min(\sqrt{k'}, \eta^{-1})$, and $\eta' = CD^2 ((k')^{-1/4} + \eta^{1/4})$.
    
    It follows that for all $E$,
    \[
        \bbP\lt[
            \fr{1}{N}
            \sup_{\ubsig \in \cQ^{k'',D,\uvphi}(\eta')}
            \cH_N^{k'',D,\up}(\ubsig) \ge E - \eps
        \rt]
        \ge 
        \binom{k}{k''}^{-D}
        \bbP\lt[
            \fr{1}{N}
            \sup_{\ubsig \in \cQ_{\loc}^{k,D,\uvphi}(\eta)}
            \cH_N^{k,D,\up}(\ubsig) \ge E
        \rt]
        - e^{-\Omega(N)}.
    \]
    By Lemma~\ref{lem:bogp-subgaussian}, the random variables in these two probabilities are both subgaussian with fluctuations $O(N^{-1/2})$.
    So
    \[
        \limsup_{N\to\infty} 
        \fr{1}{N}
        \bbE 
        \sup_{\ubsig \in \cQ^{k'',D,\uvphi}(\eta')}
        \cH_N^{k'',D,\up}(\ubsig)
        + \eps
        \ge 
        \limsup_{N\to\infty} 
        \fr{1}{N}
        \bbE
        \sup_{\ubsig \in \cQ_{\loc}^{k,D,\uvphi}(\eta)}
        \cH_N^{k,D,\up}(\ubsig).
    \]
    For fixed $D$, as $k\to\infty$ and $\eta\to 0$, we have $k'' \to \infty$ and $\eta' \to 0$.
    Then taking $D\to\infty$ shows $\BOGP_{\den} + \eps \ge \BOGP^+_{\loc}$. 
    Since $\eps$ was arbitrary, the result follows. 
\end{proof}

\begin{remark}
    \label{rem:min-BOGP}
    A byproduct of Lemma~\ref{lem:prune-all-leaves-good} is that defining $\cH_N$ as the  \textbf{minimum} over $u\in\bbL$ of the energies $H_N^{(u)}(\bsig(u))$, and $\BOGP$ in terms of this $\cH_N$, gives the same threshold as our definition \eqref{eq:grand-hamiltonian} of $\cH_N$ as the average of these energies. The minimal energy is actually more directly connected to our proof of Theorem~\ref{thm:main-ogp-oc}, as seen in the definition \eqref{eq:S-events} of $\Ssolve$. However the average energy is more convenient for our analysis in Section~\ref{sec:uc}.
\end{remark}

\subsection{Equivalence of $\BOGP_{\loc}$ and $\BOGP_{\loc,0}$}

\begin{lemma}
    \label{lem:gram-schmidt}
    Let $k\in \bbN$, $0 < q_0 \le q \le 1$ and $q',\eps \in [0,1]$. 
    There exists $\eps' = \eps'(\eps,k,q_0)$, where $\eps'\to 0$ as $\eps\to 0$ for fixed $k,q_0$, such that the following holds for all $q,q'$. 
    Suppose that $\bx,\by^1,\ldots,\by^k \in \bbR^N$ and 
    \[
        \bY = \begin{bmatrix}\bx & \by^1 & \cdots & \by^k\end{bmatrix}
    \]
    satisfies $\bY^\top \bY = D + E$, where $D = \diag(q,q',\ldots,q')$, all entries of $E$ have magnitude at most $\eps$, and $E_{1,1}=0$.
    There exist $\bz^1,\ldots,\bz^k$ such that for
    \[
        \bZ = \begin{bmatrix}\bx & \bz^1 & \cdots & \bz^k \end{bmatrix},
    \]
    we have $\bZ^\top \bZ = D$ and $\norm{\bz^i-\by^i}_2 \le \eps'$ for all $i\in [k]$.
\end{lemma}
\begin{proof}
    We will take 
    \[
        \eps' = 
        \begin{cases}
            2 & k^2 \eps \ge q_0, \\
            3k^{3/2} \eps^{1/2} & \text{otherwise}.
        \end{cases}
    \]
    If $k^3 \eps \ge q_0$, we let $\bz^1,\ldots,\bz^k$ be any orthogonal vectors of norm $\sqrt{q'}$ orthogonal to $\bx$ and each other.
    As $\norm{\by^i}_2,\norm{\bz^i}_2 \le 1$, the result follows.
    Similarly, if $k^3 \eps \ge q'$, then 
    \[
        \norm{\by^i-\bz^i}_2
        \le 
        \norm{\by^i}_2 + \norm{\bz^i}_2
        = \sqrt{q'+\eps} + \sqrt{q'}
        \le 3k^{3/2} \eps^{1/2}.
    \]
    It remains to address the case $k^3 \eps \le \min(q_0,q')$.
    We define $\bz^1,\ldots,\bz^k$ by the Gram-Schmidt algorithm, i.e. 
    \[
        \hbz^i = \by^i - \fr{\la \by^i, \bx \ra}{\norm{\bx}_2^2} \bx - \sum_{j=1}^{i-1} \fr{\la \by^i, \bz^j\ra}{\norm{\bz^j}_2} \bz^j,
        \qquad
        \bz^i = \fr{\sqrt{q'}}{\norm{\hbz^i}_2} \hbz^i.
    \]
    Let $\eps_i = \eps (1 + 3k^{-2})^i$, and note that $\eps \le \eps_i \le 2\eps$ for all $0\le i\le k$.
    We will show by induction over $i$ that for all $j\le i<\ell$,
    \begin{equation}
        \label{eq:gram-schmidt-induction-goal}
        |\la \by^\ell, \bz^j\ra| \le \eps_i,
    \end{equation}
    where as the base case this vacuously holds for $i=0$.
    Suppose the inductive hypothesis holds for $i-1$.
    It suffices to prove \eqref{eq:gram-schmidt-induction-goal} for $j=i$ because the assertion for the remaining $j$ is implied by the inductive hypothesis, as $\eps_{i-1} \le \eps_i$.
    We have
    \[
        \norm{\hbz^i}_2^2
        = 
        \norm{\by^i}_2^2
        - \fr{\la \by^i, \bx\ra^2}{\norm{\bx}_2^2}
        - \sum_{j=1}^{i-1} \fr{\la \by^i, \bz^j \ra^2}{\norm{\bz^j}_2^2}
    \]
    so
    \[
        \lt|\fr{\norm{\hbz^i}_2^2}{q'}-1\rt|
        \le 
        \lt|\fr{\norm{\by^i}_2^2}{q'}-1\rt|
        + \fr{\la \by^i, \bx\ra^2}{q'\norm{\bx}_2^2}
        + \sum_{j=1}^{i-1} \fr{\la \by^i, \bz^j \ra^2}{q'\norm{\bz^j}_2^2}
        \le 
        \fr{\eps_{i-1}}{q'} + \fr{\eps_{i-1}^2}{q'q_0} + \fr{k\eps_{i-1}^2}{(q')^2}
        \le \fr{2}{k^3}.
    \]
    Thus $\norm{\hbz^i}_2 \ge \sqrt{q'}(1-2k^{-3})$.
    For any $\ell > i$,
    \begin{align*}
        |\la \hbz^i, \by^\ell \ra| 
        &\le 
        |\la \by^i, \by^\ell\ra| 
        + \fr{|\la \by^i, \bx \ra||\la \bx, \by^\ell \ra|}{\norm{\bx}_2^2}  
        + \sum_{j=1}^{i-1} \fr{|\la \by^i, \bz^j\ra| |\la \bz^j, \by^\ell \ra|}{\norm{\bz^j}_2^2} \\
        &\le \eps_{i-1}  + \fr{\eps_{i-1}^2}{q_0} + \fr{k\eps_{i-1}^2}{q'_0} 
        \le \eps_{i-1} \lt(1 + \fr{2}{k^2}\rt)
    \end{align*}
    Thus 
    \[
        |\la \bz^i, \by^\ell \ra| 
        \le \eps_{i-1} \cdot \fr{1 + 2k^{-2}}{1 - 2k^{-3}} 
        \le \eps_i,
    \]
    completing the induction.
    Finally, note that
    \[
        \norm{\hbz^i-\by^j}_2^2
        =
        \fr{\la \by^i, \bx \ra^2}{\norm{\bx}_2^2} 
        + \sum_{j=1}^{i-1} \fr{\la \by^i, \bz^j\ra^2}{\norm{\bz^j}_2^2}
        \le \fr{\eps_{i-1}^2}{q_0} + \fr{k\eps_{i-1}^2}{q'} 
        \le \fr{5\eps}{k^2}
    \]
    and
    \[
        \lt|\norm{\hbz^i}_2 - \sqrt{q'}\rt|
        \le 
        \fr{\lt|\norm{\hbz^i}_2^2 - q'\rt|}{\sqrt{q'}}
        \le 
        \eps_{i-1} + \fr{\eps_{i-1}^2}{q'q_0} + \fr{k\eps_{i-1}^2}{(q')^2}
        \le 3\eps.
    \]
    Thus
    \[
        \norm{\bz^i-\by^i}_2
        \le 
        \norm{\hbz^i-\by^i}_2 + \lt|\norm{\hbz^i}_2 - \sqrt{q'}\rt|
        \le 
        \fr{\sqrt{5\eps}}{k} + 3\eps
        \le \eps'.
    \]
\end{proof}

\begin{proof}[Proof of Proposition~\ref{prop:bogp-loc0}]
    Let $\BOGP_{\loc,0}^+$ and $\BOGP_{\loc,0}^-$ be $\BOGP_{\loc,0}$ where the outer limit in $D$ is replaced by $\limsup$ and $\liminf$, respectively.
    It is clear that $\BOGP_{\loc} \ge \BOGP_{\loc,0}^+$, so it suffices to prove $\BOGP_{\loc} \le \BOGP_{\loc,0}^-$. 

    Fix $D,k,\eta$, $1/D^2$-separated $\vchi$, and $6r/D$-dense $(\up,\uvphi)$ with $\uvphi = \vchi(\up)$.
    Consider $\ubsig \in \cQ_{\loc}(\eta)$ and let $\ubrho \in \cQ_{\loc+}(\eta)$ such that $(\brho(u))_{u\in \bbL} = \ubsig$.
    Define $\eps_0 = \eta D$ and $\eps_d = \eps'(6\eps_{d-1} + 4\eta,k,D^{-2})$ for $1\le d\le D$, where $\eps'$ is given by Lemma~\ref{lem:gram-schmidt}.
    We will now construct $\ubtau \in \cQ_{\loc+}(0)$ approximating $\ubrho$ in the sense that for all $u\in \bbT, s\in \sS$,
    \begin{equation}
        \label{eq:gram-schmidt-invariant}
        \sqrt{R_s(\btau(u)-\brho(u),\btau(u)-\brho(u))}
        \le \eps_{|u|}.
    \end{equation}
    We define $\btau(\emptyset)$ by
    \[
        \btau(\emptyset)_s 
        = \brho(\emptyset)_s 
        \sqrt{\fr{\phi_0^s}{R_s(\brho(\emptyset),\brho(\emptyset))}}
    \]
    for all $s\in \sS$.
    Thus $R_s(\btau(\emptyset),\btau(\emptyset)) = \phi_0^s$ and 
    \[
        \sqrt{R_s(\btau(\emptyset)-\brho(\emptyset),\btau(\emptyset)-\brho(\emptyset))}
        =
        \lt|\sqrt{R_s(\brho(\emptyset),\brho(\emptyset))} - \sqrt{\phi_0^s}\rt|
        \le 
        \fr{\eta}{\sqrt{\phi_0^s}}
        \le 
        \eps_0,
    \]
    where the second-last inequality holds for all sufficiently small $\eta > 0$.
    This proves \eqref{eq:gram-schmidt-invariant} for $u=\emptyset$.
    We construct $\btau(u)$ for the remaining $u\in \bbT$ recursively.
    Suppose we have constructed $\btau(u)$ satisfying \eqref{eq:gram-schmidt-invariant}. 
    Then, for each $s\in \sS$, $i,j\in [k]$, 
    \begin{align*}
        R_s(\brho(ui)-\btau(u), \brho(uj)-\btau(u))
        &=
        R_s(\brho(ui)-\brho(u), \brho(uj)-\brho(u)) \\
        &\quad + R_s(\brho(ui)-\brho(u), \brho(u)-\btau(u)) \\
        &\quad + R_s(\brho(u)-\btau(u), \brho(uj)-\brho(u)) \\
        &\quad + R_s(\brho(u)-\btau(u), \brho(u)-\btau(u)),
    \end{align*}
    so 
    \[
        |R_s(\brho(ui)-\btau(u), \brho(uj)-\btau(u)) - (\phi_{|u|+1}^s - \phi_{|u|}^s) \ind\{i=j\}| \le 6\eps_{|u|} + 4\eta.
    \]
    Similarly,
    \begin{align*}
        |R_s(\brho(ui)-\btau(u), \btau(u))|
        &= 
        |R_s(\brho(ui)-\brho(u), \brho(u))| \\
        &\quad + |R_s(\brho(u)-\btau(u), \brho(u))| \\
        &\quad + |R_s(\brho(ui)-\brho(u), \btau(u)-\brho(u))| \\
        &\quad + |R_s(\brho(u)-\btau(u), \btau(u)-\brho(u))| \\
        &\le 6\eps_{|u|} + 4\eta.
    \end{align*}
    We apply Lemma~\ref{lem:gram-schmidt} on the vectors
    \[
        \fr{\btau(u)_s}{\sqrt{\lambda_s N}},
        \fr{\brho(u1)_s-\btau(u)_s}{\sqrt{\lambda_s N}},
        \ldots,
        \fr{\brho(uk)_s-\btau(u)_s}{\sqrt{\lambda_s N}}
    \]
    with $q = \phi_{|u|}^s \ge D^{-2}$, $q' = \phi_{|u|+1}^s - \phi_{|u|}^s$, and $\eps = 6\eps_{|u|} + 4\eta$. 
    This gives us $\btau(u1)_s,\ldots,\btau(uk)_s$ satisfying \eqref{eq:gram-schmidt-invariant}, such that 
    \[
        R_s(\btau(ui)-\btau(u), \btau(ui)-\btau(u))
        = \phi_{|u|+1}^s - \phi_{|u|}^s
    \]
    and the vectors $\btau(u)_s$, $\btau(u1)_s-\btau(u)_s$, $\btau(uk)_s-\btau(u)_s$ are pairwise orthogonal. 
    From this we can see that
    \begin{align*}
        \vR(\btau(ui),\btau(u)) &= \vphi_{|u|}, \\
        \vR(\btau(ui),\btau(ui)) &= \vphi_{|u|+1}, \\
        \vR(\btau(ui),\btau(uj)) &= \vphi_{|u|} \quad \text{if}~i\neq j. 
    \end{align*}
    Thus the $\ubtau$ constructed this way is an element of $\cQ_{\loc+}(0)$.
    Finally, let $\ubsig' = (\btau(u))_{u\in \bbL}$, so $\ubsig' \in \cQ_{\loc}(0)$.
    Equation \eqref{eq:gram-schmidt-invariant} implies that for all $u\in \bbL$, 
    \[
        \fr{1}{\sqrt{N}} \norm{\bsig'(u)-\bsig(u)}_2 \le \eps_D.
    \]
    By Proposition~\ref{prop:gradients-bounded}, with probability $1-e^{-\Omega(N)}$ we have $\HNp{u}\in K_N$ for all $u\in \bbL$. 
    On this event, 
    \[
        \lt|\fr{1}{N} \cH_N(\ubsig') - \fr{1}{N} \cH_N(\ubsig)\rt| \le C\eps_D
    \]
    for some $C >0$, and so 
    \[
        \fr{1}{N} \sup_{\ubsig' \in \cQ_{\loc}^{k,D,\uvphi'}(\eta)} \cH_N(\ubsig')
        + C\eps_D
        \ge 
        \fr{1}{N} \sup_{\ubsig \in \cQ_{\loc}^{k,D,\uvphi}(0)} \cH_N(\ubsig).
    \]
    By Lemma~\ref{lem:bogp-subgaussian}, both sides of this inequality are subgaussian with fluctuations $O(N^{-1/2})$, so the contribution from the complement of this event is $o_N(1)$, and 
    \[
        \limsup_{N\to\infty} 
        \fr{1}{N} \bbE \sup_{\ubsig' \in \cQ_{\loc}^{k,D,\uvphi'}(0)} \cH_N(\ubsig')
        + C\eps_D
        \ge 
        \limsup_{N\to\infty}
        \fr{1}{N} \bbE \sup_{\ubsig \in \cQ_{\loc}^{k,D,\uvphi}(\eta)} \cH_N(\ubsig).
    \]
    Taking $\eta \to 0$ (which forces $\eps_D \to 0$) followed by $D,k\to\infty$ implies $\BOGP_{\loc} \le \BOGP_{\loc,0}^-$, as desired.
\end{proof}
