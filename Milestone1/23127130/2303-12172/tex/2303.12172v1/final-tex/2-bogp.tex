\section{Algorithmic Thresholds from Branching OGP}

We begin this section by recalling some fundamental definitions and constructions from \cite{huang2021tight}. 
We then review the details of the branching overlap gap property introduced in \cite{huang2021tight}, and in particular the link to hardness for overlap concentrated algorithms. 

\subsection{Correlation Functions and Overlap Concentration}
\label{subsec:correlation}

For any $p\in [0,1]$, we may construct two correlated copies $\HNp{1}, \HNp{2}$ of $H_N$ as follows.
Construct three i.i.d. copies $\wtH_N^{[0]}, \wtH_N^{[1]}, \wtH_N^{[2]}$ of $\wtH$ as in \eqref{eq:def-hamiltonian-no-field}.
For $i=1,2$ define
\begin{align*}
    \HNp{i}(\bsig) 
    &= 
    \la \bh, \bsig \ra 
    + \wtHNp{i}(\bsig), \quad \text{where} \\
    \wtHNp{i}(\bsig) 
    &= 
    \sqrt{p} \wtH_N^{[0]}(\bsig) + 
    \sqrt{1-p} \wtH_N^{[i]}(\bsig).
\end{align*}
We say $\HNp{1}, \HNp{2}$ are $p$-correlated.
Note that pairs of corresponding entries in $\bg(\HNp{1})$ and $\bg(\HNp{2})$ are Gaussian with covariance $\lt[\begin{smallmatrix}1 & p \\ p & 1\end{smallmatrix}\rt]$.

Given a function $\cA_N : \sH_N \to \cB_N$ (always assumed to be measurable) define $\vchi:[0,1] \to \bbR^{\sS}$ by
\begin{equation}
    \label{eq:def-correlation-fn}
    \vchi(p) 
    = 
    \E \vR \lt(
        \cA(\HNp{1}), 
        \cA(\HNp{2})
    \rt),
\end{equation}
where $\HNp{1}, \HNp{2}$ are $p$-correlated copies of $H_N$. 
We say that $\vchi$ is the \textbf{correlation function} of $\cA$.
Let $\chi_s$ denote the $s$-coordinate of $\vchi$.

\begin{proposition}
    \label{prop:correlation-fn-properties}
    We have $\vchi \in \bbI(0,1)^\sS$.
\end{proposition}
\begin{proof}
    Identically to \cite[Proposition 3.1]{huang2021tight}, Hermite expanding $R_s \lt(\cA(\HNp{1}), \cA(\HNp{2}) \rt)$ shows that $\chi_s$ is continuous and increasing.
    The same Hermite expansion shows $\chi_s$ is continuously differentiable.
\end{proof}
The other properties of correlation functions proved in \cite[Proposition 3.1]{huang2021tight} also hold, namely that $\chi_s$ is convex and either strictly increasing or constant; however they are not needed in this paper. 

We will determine the maximum energy attained by algorithms $\cA_N : \sH_N \to \cB_N$ obeying the following overlap concentration property. 

\begin{definition}
    \label{defn:oc}
    Let $\eta, \nu > 0$. 
    An algorithm $\cA = \cA_N$ is $(\eta,\nu)$ overlap concentrated if for any $p\in [0,1]$ and $p$-correlated Hamiltonians $\HNp{1}, \HNp{2}$,
    \begin{equation}
        \label{eq:overlap-concentrated}
        \P
        \lt[
            \norm{ 
                \vR \lt( \cA(\HNp{1}), \cA(\HNp{2}) \rt) -
                \vchi(p)
            }_{\infty}
            \ge \eta
        \rt]
        \le \nu.
    \end{equation}
\end{definition}

Our main hardness result is the following bound on the performance of overlap concentrated algorithms.

\begin{theorem}
    \label{thm:main-ogp-oc}
    Consider a multi-species spherical spin glass Hamiltonian $H_N$ with parameters $(\xi,\vh)$. 
    Let $\ALG$ be given by \eqref{eq:alg}.
    For any $\eps > 0$ there are $\eta, c, N_0$ depending only on $\xi, \vh, \eps$ such that the following holds for any $N\ge N_0$ and $\nu \in [0,1]$. 
    For any $(\eta,\nu)$-overlap concentrated $\cA_N : \sH_N \to \cB_N$, 
    \[
        \bbP\lt[H_N(\cA_N(H_N))/N \ge \ALG + \eps\rt]
        \le 
        \exp(-cN) + 
        \nu^c.
    \]
\end{theorem}
By Gaussian concentration of measure (see \cite[Propositon 8.2]{huang2021tight}), any $\tau$-Lipschitz algorithm is $(\eta, e^{-c(\eta,\tau)N})$-overlap concentrated for any $\eta>0$ and appropriate $c(\eta,\tau)>0$.
Thus Theorem~\ref{thm:main-ogp-oc} implies Theorem~\ref{thm:main-ogp}.

\subsection{Ultrametrically Correlated Hamiltonians}
\label{subsec:ultra-corr-H}

Next we define the hierarchically correlated ensemble of Hamiltonians used to define the branching overlap gap property.
Let $k\ge 2$, $D\ge 1$ be positive integers.
For each $0\le d \le D$, let $V_d = [k]^d$ denote the set of length $d$ sequences of elements of $[k]$. 
The set $V_0$ consists of the empty tuple, which we denote $\emptyset$.
Let $\bbT(k,D)$ denote the depth $D$ tree rooted at $\emptyset$ with depth $d$ vertex set $V_d$, where $u\in V_d$ is the parent of $v\in V_{d+1}$ if $u$ is the length $d$ initial substring of $v$.
For nodes $u^1,u^2\in \bbT(k,D)$, let 
\[
    u^1 \wedge u^2
    =
    \max \lt\{
        d \in \bbZ_{\ge 0}: 
        \text{$u^1_{d'} = u^2_{d'}$ for all $1\le d' \le d$}
    \rt\},
\]
where the set on the right-hand side always contains $0$ vacuously.
This is the depth of the least common ancestor of $u^1$ and $u^2$.
Let $\bbL(k,D) = V_D$ denote the set of leaves of $\bbT(k,D)$.
When $k,D$ are clear from context, we denote $\bbT(k,D)$ and $\bbL(k,D)$ by $\bbT$ and $\bbL$.
Finally, let $K = |\bbL| = k^D$.

Let the sequences $\up = (p_0, p_1, \ldots, p_D)\in\bbR^{D+1}$ 
and $\uvphi = (\vphi_0, \vphi_1, \ldots, \vphi_D)\in (\bbR^{\sS})^{D+1}$ 
satisfy
\begin{align*}
    0 = p_0 \le p_1 \le \cdots \le p_D &= 1, \\
    \vzero \preceq \vphi_0 \preceq \vphi_1 \preceq \cdots \preceq \vphi_D &\preceq \vone.
\end{align*}
The sequence $\up$ controls the correlation structure of our ensemble of Hamiltonians while the sequence $\uvphi$ controls the overlap structure of their inputs.
For each $u\in \bbT$, including interior nodes, let $\wtH_N^{[u]}$ be an independent copy of $\wtH_N$ generated by \eqref{eq:def-hamiltonian-no-field}, and let
\begin{equation}
    \label{eq:def-correlated-disorder}
    \wtHNp{u} = 
    \sum_{d=1}^{|u|}
    \sqrt{p_d - p_{d-1}} \cdot
    \wtH_N^{[(u_1,\ldots,u_d)]}
\end{equation}
where $|u|$ is the length of $u$ and $(u_1,\ldots,u_d)$ is the length-$d$ prefix of $u$.
For $u\in \bbL$, define
\[
    \HNp{u}(\bsig) = 
    \la \bh, \bsig \ra + 
    \wtHNp{u}(\bsig).
\]
This constructs a Hamiltonian ensemble $(\HNp{u})_{u\in \bbL}$ where each $\HNp{u}$ is marginally distributed as $H_N$ and each pair of Hamiltonians $\HNp{u^1}, \HNp{u^2}$ is $p_{u^1\wedge u^2}$-correlated.
We define a grand Hamiltonian on states
\[
    \ubsig = (\bsig(u))_{u\in \bbL} \in (\bbR^N)^\bbL.
\]
by
\begin{equation}
    \label{eq:grand-hamiltonian}
    \cH_N^{k,D,\up}(\ubsig) 
    =
    \fr{1}{K}
    \sum_{u\in \bbL}
    \HNp{u}(\bsig(u)).
\end{equation}
We denote this by $\cH_N$ when $k,D,\up$ are clear from context.
Note that we have thus far not used the definition of $\wtHNp{u}$ for interior nodes $u\in \bbT \setminus \bbL$; these Hamiltonians will be useful in our analysis of the branching OGP threshold in Section~\ref{sec:uc}.
The branching OGP is defined by a maximization of $\cH_N$ over the overlap-constrained set
\begin{equation}
\label{eq:cQ}
    \cQ^{k,D,\uvphi}(\eta)
    =
    \lt\{
        \ubsig \in \cB_N^\bbL : 
        \norm{\vR(\bsig(u^1),\bsig(u^2)) - \vphi_{u^1\wedge u^2}}_\infty 
        \le \eta,
        ~\forall u^1,u^2 \in \bbL
    \rt\}.
\end{equation}
We denote this set $\cQ(\eta)$ when $k,D,\uvphi$ are clear from context.

\subsection{The Branching OGP Threshold}

We will show that overlap concentrated algorithms cannot outperform a \emph{branching OGP} energy $\BOGP$ defined as the ground state energy of the grand Hamiltonian \eqref{eq:grand-hamiltonian} in the limit of ``continuously branching" ultrametrics.

\begin{definition}[Branching OGP energy]
    \label{defn:bogp}
    The energy $\BOGP = \BOGP(\xi,\vh)$ is the infimum of energies $E$ such that the following holds.
    Choose sufficiently large $D$, followed by small $\eta$ and then large $k$. For any $\vchi \in \bbI(0,1)^\sS$ there exists $\up$ such that for $\uvphi = \vchi(\up)$ element-wise (i.e. $\vphi_d=\vchi(p_d)$),
    \begin{equation}
        \label{eq:bogp}
        \limsup_{N\to\infty}
        \fr{1}{N} 
        \bbE \sup_{\ubsig \in \cQ(\eta)}
        \cH_N(\ubsig)
        \le 
        E.
    \end{equation}
    More explicitly,
        \begin{equation}
        \label{eq:bogp-explicit}
        \BOGP(\xi,\vh)
        \equiv
        \lim_{D\to\infty}
        \lim_{\eta\to 0}
        \lim_{k\to\infty}
        \sup_{\vchi \in \bbI(0,1)^\sS}
        \inf_{\uvphi=\vchi(\up)}
        \limsup_{N\to\infty}
        \fr{1}{N} 
        \bbE \sup_{\ubsig \in \cQ^{k,D,\uvphi}(\eta)}
        \cH_N^{k,D,\up}(\ubsig).
    \end{equation}
\end{definition}
Our previous work \cite{huang2021tight} implicitly considered the same quantity. Note that the limits in $(D,k,\eta)$ are decreasing, so they could actually be taken in any order (and moreover the limiting value $\BOGP$ exists apriori). Additionally the role of the infimum over $(\uvphi,\up)$ is quite simple: the only important thing is to ensure both sequences increase in uniformly small steps (see Definition~\ref{def:delta-dense}).

Section~\ref{sec:uc} proves the following proposition identifying $\BOGP$ with the formula \eqref{eq:alg} for $\ALG$.
\begin{proposition}
    \label{prop:bogp-alg}
    For all $(\xi,\vh)$, we have $\BOGP = \ALG$.
\end{proposition}
Let us first prove Theorem~\ref{thm:main-ogp-oc} assuming Proposition~\ref{prop:bogp-alg}.
Let $\eps > 0$ be arbitrary and $k,D,\eta$ be given by Definition~\ref{defn:bogp} for $E = \ALG + \eps/4$. 
Let $\cA = \cA_N : \sH_N \to \cB_N$ be a $(\eta,\nu)$-overlap concentrated algorithm with correlation function $\vchi$.
Let $\up$ and $\uvphi$ be given by Definition~\ref{defn:bogp} (depending on $\vchi$). 
Since $\BOGP = \ALG$ by Proposition~\ref{prop:bogp-alg}, for sufficiently large $N$
\[
    \fr{1}{N} 
    \bbE \sup_{\ubsig \in \cQ(\eta)}
    \cH_N(\ubsig)
    \le 
    \ALG
    + \eps/2.
\]
Let 
\[
    \alpha_N = \bbP\lt[
        H_N(\cA(H_N)) \ge \ALG + \eps      
    \rt].
\]
Let $\bsig(u) = \cA(\HNp{u})$ and $\ubsig = (\bsig(u))_{u\in \bbL}$. 
Define the events
\begin{equation}
\label{eq:S-events}
\begin{aligned}
    \Ssolve &= \lt\{\HNp{u}(\bsig(u)) / N \ge \ALG + \eps ~\forall u\in \bbL\rt\}, \\
    \Soverlap &= \lt\{\ubsig \in \cQ(\eta)\rt\}, \\
    \Sogp &= \lt\{ \sup_{\ubsig \in \cQ(\eta)} \cH_N(\ubsig) / N < \ALG + \eps \rt\}.
\end{aligned}
\end{equation}
\begin{proposition}
    \label{prop:prob-ineqs}
    The following inequalities hold. 
    \begin{enumerate}[label=(\alph*), ref=\alph*]
        \item \label{itm:ssolve} $\bbP(\Ssolve) \ge \alpha_N^K$.
        \item \label{itm:soverlap} 
        $\bbP(\Soverlap) \ge 1 - K^2\nu$.
        \item \label{itm:sogp} $\bbP(\Sogp) \ge 1 - 2 \exp(-cN)$ for suitable $c=c(\eps) > 0$.
    \end{enumerate}
\end{proposition}
\begin{proof}[Proof of (\ref{itm:ssolve})]
    Use Jensen's inequality $D$ times as in \cite[Proof of Proposition 3.6(a)]{huang2021tight}. 
\end{proof}
\begin{proof}[Proof of (\ref{itm:soverlap})]
    For each $u^1,u^2\in\bbL$, $\bbE \vR(\bsig(u^1),\bsig(u^2)) = \vchi(p_{u^1\wedge u^2}) = \vphi_{u^1\wedge u^2}$.
    So,
    \[
        \bbP\lt[
            \norm{\vR(\bsig(u^1),\bsig(u^2)) - \vphi_{u^1\wedge u^2}}_\infty \le \eta
        \rt]
        \ge 1-\nu.
    \]
    The result follows by a union bound on $u^1,u^2$.
\end{proof}
\begin{proof}[Proof of (\ref{itm:sogp})]
    Use the Borell-TIS inequality on the random variable $Y = \fr{1}{N} \sup_{\ubsig \in \cQ(\eta)} \cH_N(\ubsig)$, as in \cite[Proof of Proposition 3.6(d)]{huang2021tight}.
\end{proof}

\begin{proof}[Proof of Theorem~\ref{thm:main-ogp-oc}]
    Note that $\Ssolve \cap \Soverlap \cap \Sogp = \emptyset$.
    So, $\bbP(\Ssolve) + \bbP(\Soverlap) + \bbP(\Sogp) \le 2$.
    The bounds in Proposition~\ref{prop:prob-ineqs} imply
    \[
        \alpha_N^K \le 2\exp(-cN) + K^2\nu
    \]
    By adjusting the constant $c$,
    \[
        \alpha_N 
        \le 
        \exp(-cN) + \nu^c.
    \]
\end{proof}







\subsection{An Alternate Definition for the $\BOGP$ Threshold}
\label{subsec:alt-bogp}

The overlap-constrained input set $\cQ(\eta)$ used to define $\BOGP$ was designed to capture the properties of $\ubsig=(\cA(H_N^{(u)}))_{u\in\bbL}$.
In this set, overlap constraints are enforced \emph{globally}, between each pair of states, and the constraints are \emph{approximate}, within a tolerance $\eta > 0$.

In this subsection, we define a variant $\BOGP_{\loc,0}$ of $\BOGP$, based on an input set $\cQ_{\loc}(0)$, in which overlap constraints are enforced \emph{locally}, between only adjacent and sibling nodes in $\bbT$, and the constraints are \emph{exact}.
We also enforce that the sequences $p_d$, $\vphi_d$ increase in small steps.
To define the local constraints, we introduce the extended states
\[
    \ubrho = (\brho(u))_{u\in \bbT} \in \cB_N^{\bbT}
\]
whose indices now also include interior $u\in \bbT$. 
For $u,v\in \bbT$, let $u\sim v$ indicate that $u=v$, or one of $u,v$ is the parent of the other, or $u,v$ are siblings.
Define
\begin{align*}
    \cQ_{\loc+}^{k,D,\uvphi}(\eta)
    &= \lt\{
        \ubrho \in \cB_N^\bbT : 
        \norm{\vR(\brho(u), \brho(v)) - \vphi_{u\wedge v}}_\infty \le \eta,~\forall u\sim v
    \rt\} \\
    \cQ_{\loc}^{k,D,\uvphi}(\eta)
    &= \lt\{
        \ubsig \in \cB_N^\bbL : 
        \exists \ubrho \in \cQ_{\loc+}^{k,D,\uvphi}(\eta)~\text{such that}~(\brho(u))_{u\in \bbL} = \ubsig
    \rt\}.
\end{align*}
We similarly omit the superscript $k,D,\uvphi$ when this is clear from context.
The following definition captures the property that $p_d$, $\vphi_d$ increase in small steps. 
\begin{definition}
    \label{def:delta-dense}
    The pair of sequences $(\up,\uvphi)$ is \textbf{$\delta$-dense} if $p_d-p_{d-1} \le \delta$ and $\vphi_d - \vphi_{d-1} \preceq \delta \vone$ for all $d$.
\end{definition}
The following technical condition ensures continuous dependence of orthogonal bands on their centers. 
\begin{definition}
    \label{def:separated}
    The function $\vchi \in \bbI(0,1)^\sS$ is \textbf{$\delta$-separated} if $\vchi(0) \succeq \delta \vone$.
\end{definition}
Define
\begin{equation}
    \label{eq:bogp-loc}
        \BOGP_{\loc,0}
        =
        \lim_{D\to\infty}
        \lim_{k\to\infty}
        \sup_{\substack{
            \vchi \in \bbI(0,1)^\sS \\ 
            \text{$1/D^2$-separated}
        }}
        \inf_{\substack{
            \uvphi=\vchi(\up) \\ 
            \text{$6r/D$-dense}
        }}
        \limsup_{N\to\infty}
        \fr{1}{N} 
        \bbE \sup_{\ubsig \in \cQ_{\loc}(0)}
        \cH_N(\ubsig).
\end{equation}
Note that the limit in $D$ is no longer obviously decreasing, so the existence of this limit also needs to be proven. 

The following proposition, which we prove in Appendix~\ref{sec:equivalence-of-bogps}, shows that $\BOGP_{\loc,0}$ is an equivalent characterization of $\BOGP$.
This characterization will be more convenient for the proof of Proposition~\ref{prop:bogp-alg} carried out in the next section. 
We note that in the proof we define several more variants of $\BOGP$ and show all are equal, and it also follows that the average in the definition \eqref{eq:grand-hamiltonian} of $\cH_N$ can be replaced by a minimum with no change. This illustrates some flexibility in using the branching OGP.

\begin{proposition}
    \label{prop:bogp-equivalent}
    The limit $\BOGP_{\loc,0}$ exists and $\BOGP=\BOGP_{\loc,0}$.
\end{proposition}

Finally we record two useful facts.
\begin{lemma}
    \label{lem:loc0-barycenter}
    If $\ubrho \in \cQ_{\loc,+}(0)$ and $\bar \brho = \fr{1}{K} \sum_{u\in \bbL} \brho(u)$, then $\fr{1}{\sqrt{N}} \norm{\brho(\emptyset) - \bar \brho}_2 \le \sqrt{D/k}$.
\end{lemma}
\begin{proof}
    Define $\ubtau \in (\bbR^N)^\bbT$ by $\btau(u) = \brho(u)$ for $u\in \bbL$ and otherwise recursively $\btau(u) = \fr{1}{k} \sum_{i=1}^k \btau(ui)$. 
    By bilinearity of $\vR$, for all $u\in \bbT \setminus \bbL$ with $|u|=d$,
    \[
        \vR\lt(
            \brho(u) - \fr1k \sum_{i=1}^k \brho(ui),
            \brho(u) - \fr1k \sum_{i=1}^k \brho(ui)
        \rt)
        = \fr1k (\vphi_{d+1} - \vphi_d),
    \]
    so
    \[
        \fr{1}{\sqrt{N}} \norm{\brho(u) - \fr1k \sum_{i=1}^k \brho(ui)}_2 
        = \sqrt{\fr{q_{d+1}-q_d}{k}},
    \]
    where $q_d = \la \vlam, \vphi_d \ra$.
    It is easy to see by induction on $d$ that 
    \begin{align*}
        \fr{1}{\sqrt{N}} \norm{\brho(u) - \btau(u)}_2 
        &\le \fr{1}{\sqrt{N}} \norm{\brho(u) - \fr1k \sum_{i=1}^k \brho(ui)}_2 
        + \fr{1}{k} \sum_{i=1}^k \fr{1}{\sqrt{N}} \norm{\brho(ui) - \btau(ui)}_2 \\
        &\le \sum_{\ell=d}^{D-1} \sqrt{\fr{q_{\ell+1}-q_\ell}{k}}.
    \end{align*}
    Since $\bar \brho = \btau(\emptyset)$, 
    \[
        \fr{1}{\sqrt{N}} \norm{\brho(\emptyset) - \bar \brho}_2
        \le \sum_{d=0}^{D-1} \sqrt{\fr{q_{d+1}-q_d}{k}}
        \le \sqrt{\fr{D}{k}}
    \]
    by Cauchy-Schwarz.
\end{proof}

\begin{lemma}
    \label{lem:bogp-subgaussian}
    For any $S \subseteq \cB_N^{\bbL}$, $\fr{1}{N} \sup_{\ubsig \in S} \cH_N(\ubsig)$ is $O(N^{-1/2})$-subgaussian, in particular
    \[
    \bbP\lt[
    \lt|
    \sup_{\ubsig \in S} \cH_N(\ubsig)
    -
    \bbE[
    \sup_{\ubsig \in S} \cH_N(\ubsig)
    ]
    \rt|
    \geq tN^{1/2}
    \rt]
    \leq 
    Ce^{-t^2/C}
    \]
    for a constant $C$ and all $t\geq 0$.
\end{lemma}
\begin{proof}
    We calculate identically to \cite[Proof of Proposition 3.6(d)]{huang2021tight} that for any fixed $\ubsig \in \cB_N^\bbL$, $\Var \cH_N(\ubsig) = O(N)$.
    The result follows from the Borell-TIS inequality, whose statement and proof hold for noncentered Gaussian processes with no modification.
\end{proof}
