\section{Asymmetric Maximizers for Symmetric $\xi$ }
\label{sec:non-unique}

Let 
\[
    \xi(x,y)=x^4+y^4+24xy.
\]
This function is symmetric in $(x,y)$. We will show that the natural minimizer is not symmetric via a 2nd derivative expansion.


We wish to compute $\deriv{\delta}G$. Here we fix $\chi\in C_c^{\infty}((0,1);\mathbb R)$ (hence $\chi(0)=\chi(1)=0$) and set
\[
    \Phi_1^{\delta\chi}(q)
    =
    \Phi^*_1(q) 
    +
    \delta
    \chi(q)
\]
so that
\[
    (\Phi_1^{\delta\chi})'(q)
    =
    (\Phi^*_1)'(q) + \delta \chi'(q)
\]
for all $q\in [0,1]$. As before let $\Phi_s^{\delta\chi}=\Phi_s^*$ for $s\neq 1$. Similarly to before we compute:
\begin{align*}
    \deriv{\delta} (\Phi_1^{\delta\chi})'(q)
    &=
    \chi'(q)
    ;
    \\
    \deriv{\delta} \Phi_1^{\delta\chi}(q)
    &=
    \chi(t)
    ;
    \\
    \deriv{\delta} (\xi^s \circ \Phi)'(q)
    &=
    \deriv{q} \lt((\partial_1\xi^s \circ \Phi)(q) \chi(q)\rt).
\end{align*} 
We will use the computation
\[
    \lt(\sqrt{\frac{f}{g}}\rt)' = \frac{f'g-fg'}{2f^{1/2}g^{3/2}}.
\] 
Recall the previous calculation and setting $p\equiv 1$, we find
\begin{align*}
    G(\Phi,\chi)&
    \equiv 2\deriv{\delta} F(\Phi^{\delta\chi}) \Big|_{\delta = 0} \\
    &=
    \lambda_1
    \underbrace{
        \int_0^1
        \sqrt{\frac{(\xi^1 \circ \Phi)'(q)}{\Phi'_1(q)}}
        \chi'(q)
        \diff{q}
    }_{T_0}
    +
    \sum_{s\in \sS}
    \lambda_s
    \underbrace{
        \int_0^1
        \sqrt{\fr{\Phi'_s(q)}{( \xi^s \circ \Phi)'(q)}}
        \deriv{\delta} ( \xi^s \circ \Phi^{\delta\chi})'(q)
        \Big|_{\delta=0}
        \diff{q}
    }_{T_s}.
\end{align*}



We now begin the main computation with:
\begin{align*}
    \deriv{\delta} T_0
    &=
    \int_0^1 
    \deriv{\delta}
    \lt(
        \sqrt{
        \frac
        {(\xi^1\circ\Phi^{\delta})'(q)}
        {(\Phi_1^{\delta})'(q)}
        }
    \rt)
    \chi'(q)
    \de q
    \\
    &=
    \frac{1}{2}
    \int_0^1 
    \frac{
        \deriv{q} \lt((\partial_1\xi^1 \circ \Phi)(q) \chi(q)\rt)\Phi_1'(q)
        -
        (\xi^1\circ\Phi^{\delta})'(q)\chi'(q)
    }
    {
    (\xi^1\circ\Phi^{\delta})'(q)^{1/2}
    ~
    \Phi_1'(q)^{3/2}
    }
    \chi'(q)
    \de q
\end{align*}
When $\Phi(q)=(q,q)$ this simplifies to:
\begin{align*}
\deriv{\delta} T_0
    &=
    \frac{1}{2}
    \int_0^1 
    \frac{
        \deriv{q} \lt((\partial_1\xi^1 \circ \Phi)(q) \chi(q)\rt)\Phi_1'(q)
        -
        (\xi^1\circ\Phi^{\delta})'(q)\chi'(q)
    }
    {
    (\xi^1\circ\Phi^{\delta})'(q)^{1/2}
    ~
    \Phi_1'(q)^{3/2}
    }
    \chi'(q)
    \de q
    \\
    &=
    \frac{1}{2}
    \int_0^1 
    \frac{
        \deriv{q} \lt((\partial_1\xi^1)(q,q) \chi(q)\rt)
        -
        (\xi^1)'(q,q)\chi'(q)
    }
    {
    (\xi^1)'(q,q)^{1/2}
    }
    \chi'(q)
    \de q
    \\
    &=
    \frac{1}{2}
    \int_0^1 
    \frac{
        (\partial_1\xi^1)'(q,q) \chi(q)
        +
        \big(
            \partial_1\xi^1(q,q)-(\xi^1)'(q,q)
        \big)
        \chi'(q)
    }
    {
    (\xi^1)'(q,q)^{1/2}
    }
    \chi'(q)
    \de q
    \\
    &=
    \frac{1}{2}
    \int_0^1 
    \frac{
        (\partial_1\xi^1)'(q,q) \chi(q)
        -
        \partial_2\xi^1(q,q)
        \chi'(q)
    }
    {
    (\xi^1)'(q,q)^{1/2}
    }
    \chi'(q)
    \de q
\end{align*}



Next we write things out for $T_s$:
\begin{align*}
    \deriv{\delta} T_s
    &=
    \int_0^1
    \deriv{\delta}
    \lt(
    \sqrt{
        \frac
        {(\Phi_s^{\delta})'(q)}
        {(\xi^s\circ\Phi^{\delta})'(q)}
    }
    \deriv{q} \lt((\partial_1\xi^s \circ \Phi^{\delta})(q) \chi(q)\rt)
    \rt)
    \de q
    \\
    &=
    \int_0^1
    \lt(
    \sqrt{
        \frac
        {(\Phi_s)'(q)}
        {(\xi^s\circ\Phi)'(q)}
    }
    \deriv{q} \lt((\partial_{1,1}\xi^s \circ \Phi)(q) \chi(q)^2\rt)
    \rt)
    \de q
    \\
    &\quad\quad\quad
    +
    \int_0^1
    \deriv{\delta}
    \lt(
    \sqrt{
        \frac
        {(\Phi_s^{\delta})'(q)}
        {(\xi^s\circ\Phi^{\delta})'(q)}
    }
    \rt)
    \deriv{q} \big((\partial_1\xi^s \circ \Phi)(q) \chi(q)\big)
    \de q
    .
\end{align*}
With $\Phi(q)=(q,q)$ we have the intermediate computations
\begin{align*}
    \deriv{q} \lt((\partial_{1,1}\xi^1 \circ \Phi)(q) \chi(q)^2\rt)
    &=
    (\partial_{1,1}\xi^s\circ\Phi)'(q)\chi(q)^2
    +
    2(\partial_{1,1}\xi^s\circ\Phi)(q)\chi(q)\chi'(q)
    \\
    &=
    \partial_{1,1}\xi^s(q,q)'\chi(q)^2 + \big(2\partial_{1,1}\xi^s(q,q)\big)\chi(q)\chi'(q)
    .
\end{align*}
For the other term, in the case $s=1$ we have
\begin{align*}
    \deriv{\delta}
    \lt(
    \sqrt{
        \frac
        {(\Phi_1^{\delta})'(q)}
        {(\xi^1\circ\Phi^{\delta})'(q)}
    }
    \rt)
    &=
    \frac{
        \deriv{\delta}(\Phi_1^{\delta})'(q)
        \times
        (\xi^1\circ\Phi^{\delta})'(q)
        -
        (\Phi_1^{\delta})'(q)
        \deriv{\delta}
        (\xi^1\circ\Phi^{\delta})'(q)
    }
    {2(\Phi_1^{\delta})'(q)^{1/2}~(\xi^1\circ\Phi^{\delta})'(q)^{3/2}}
    \\
    &=
    \frac{
        \chi'(q)
        (\xi^1\circ\Phi)'(q)
        -
        (\Phi_1)'(q)
        \deriv{q} \lt((\partial_1\xi^1 \circ \Phi)(q) \chi(q)\rt)
    }
    {2(\Phi_1)'(q)^{1/2}~(\xi^1\circ\Phi)'(q)^{3/2}}
     \\
    &=
    \frac{
        \chi'(q)
        \xi^1(q,q)'
        -
        \deriv{q} \lt((\partial_1\xi^1(q,q) \chi(q)\rt)
    }
    {2(\xi^1(q,q)')^{3/2}}
     \\
    &=
    \frac{
        \chi'(q)
        \partial_2\xi^1(q,q)
        -
        \chi(q)\partial_1\xi^1(q,q)' 
    }
    {2(\xi^1(q,q)')^{3/2}}
\end{align*}
while for $s=2$
\begin{align*}
    \deriv{\delta}
    \lt(
    \sqrt{
        \frac
        {(\Phi_2^{\delta})'(q)}
        {(\xi^2\circ\Phi^{\delta})'(q)}
    }
    \rt)
    &=
    \frac{
        \deriv{\delta}(\Phi_2^{\delta})'(q)
        \times
        (\xi^2\circ\Phi^{\delta})'(q)
        -
        (\Phi_2^{\delta})'(q)
        \deriv{\delta}
        (\xi^2\circ\Phi^{\delta})'(q)
    }
    {2(\Phi_2^{\delta})'(q)^{1/2}~(\xi^2\circ\Phi^{\delta})'(q)^{3/2}}
    \\
    &=
    \frac{
        -
        \Phi_2'(q)
        \deriv{q} \lt((\partial_1\xi^2 \circ \Phi)(q) \chi(q)\rt)
    }
    {2\Phi_2'(q)^{1/2}~(\xi^2\circ\Phi)'(q)^{3/2}}
     \\
    &=
    \frac{
        -
        \deriv{q} \lt((\partial_1\xi^2(q,q) \chi(q)\rt)
    }
    {2(\xi^2(q,q)')^{3/2}}
     \\
    &=
    \frac{
        -
        \partial_1\xi^2(q,q)' \chi(q)
        -
        \partial_1\xi^2(q,q)\chi'(q)
    }
    {2(\xi^2(q,q)')^{3/2}}.
\end{align*}
Additionally,
\begin{align*}
    \deriv{q} \lt((\partial_1\xi^s \circ \Phi)(q) \chi(q)\rt)
    &=
    \partial_1\xi^s(q,q)'\chi(q) + \partial_1\xi^s(q,q)\chi'(q).
\end{align*}

We found above that
\begin{align*}
    \deriv{\delta} T_s
    &=
    \int_0^1
    \lt(
    \sqrt{
        \frac
        {(\Phi_s)'(q)}
        {(\xi^s\circ\Phi)'(q)}
    }
    \deriv{q} \lt((\partial_{1,1}\xi^s \circ \Phi)(q) \chi(q)^2\rt)
    \rt)
    \de q
    \\
    &\quad\quad\quad+
    \int_0^1
    \deriv{\delta}
    \lt(
    \sqrt{
        \frac
        {(\Phi_s^{\delta})'(q)}
        {(\xi^s\circ\Phi^{\delta})'(q)}
    }
    \rt)
    \deriv{q} \lt((\partial_1\xi^s \circ \Phi)(q) \chi(q)\rt)
    \de q
    .
\end{align*}
In particular $s=1$ gives
\begin{align*}
    \deriv{\delta} T_1
    &=
    \int_0^1
    \frac
    {1}
    {(\xi^1(q,q)')^{1/2}}
    \lt(
        (\partial_{1,1}\xi^1(q,q))'\chi(q)^2 + 2\big(\partial_{1,1}\xi^1(q,q)\big)\chi(q)\chi'(q)
    \rt)
    \de q
    \\
    &\quad\quad\quad+
    \int_0^1
    \frac{
        \chi'(q)
        \partial_2\xi^1(q,q)
        -
        \chi(q)\partial_1\xi^1(q,q)' 
    }
    {2(\xi^1(q,q)')^{3/2}}
    \lt(
        \partial_1\xi^1(q,q)'\chi(q) + \partial_1\xi^1(q,q)\chi'(q)
    \rt)
    \de q
\end{align*}
and $s=2$ gives
\begin{align*}
    \deriv{\delta} T_2
    &=
    \int_0^1
    \frac
    {1}
    {(\xi^2(q,q)')^{1/2}}
    \lt(
        (\partial_{1,1}\xi^2(q,q))'\chi(q)^2 + 2\big(\partial_{1,1}\xi^2(q,q)\big)\chi(q)\chi'(q)
    \rt)
    \de q
    \\
    &\quad\quad\quad+
    \int_0^1
    \frac{
        -
        \partial_1\xi^2(q,q)' \chi(q)
        -
        \partial_1\xi^2(q,q)\chi'(q)
    }
    {2(\xi^2(q,q)')^{3/2}}
    \lt(
        \partial_1\xi^2(q,q)'\chi(q) + \partial_1\xi^2(q,q)\chi'(q)
    \rt)
    \de q.
\end{align*}
Moreover since $\xi(x,y)=x^4+y^4+24xy$, we have
\begin{align*}
    \partial_2\xi^1(q,q)
    =
    \partial_1\xi^2(q,q)
    =
    \frac{24}{\lambda_s}
    &=
    48
    \\
    \partial_{1}\xi^2(q,q)'&=0
    \\
    \partial_{1,1}\xi^2(q,q)&=0.
\end{align*}
Additionally,
\begin{align*}
    (\partial_1\xi^1)'(q,q)&=
    \partial_{1,1}\xi^1(q,q)
    \\
    &=
    2\partial_{x,x,x}\xi(x,y)|_{x=y=q} 
    \\
    &= 48q;
    \\
    (\partial_{1,1}\xi^1(q,q))'&=48;
    \\
    \xi^1(q,q)'=\xi^2(q,q)'&=24q^2+48;
    \\
    \partial_1\xi^1(q,q)&=24q^2
    .
\end{align*}


We combine our computations below. The red line vanishes since $\partial_{1,1}\xi^2(x,y)\equiv 0$.
\begin{align*}
    2\deriv{\delta}G
    &=
    \deriv{\delta}(T_0+T_1+T_2)
    \\
    &=
    \frac{1}{2}
    \int_0^1 
    \frac{
    (\partial_1\xi^1)'(q,q) \chi(q)
    -
    \partial_2\xi^1(q,q)
    \chi'(q)
    }
    {
    (\xi^1)'(q,q)^{1/2}
    }
    \chi'(q)
    \de q
    \\
    &\quad\quad\quad+
    \int_0^1
    \frac
    {1}
    {(\xi^1(q,q)')^{1/2}}
    \lt(
        (\partial_{1,1}\xi^1(q,q))'\chi(q)^2 + 2\big(\partial_{1,1}\xi^1(q,q)\big)\chi(q)\chi'(q)
    \rt)
    \de q
    \\
    &\quad\quad\quad+
    \int_0^1
    \frac{
        \chi'(q)
        \partial_2\xi^1(q,q)
        -
        \chi(q)\partial_1\xi^1(q,q)' 
    }
    {2(\xi^1(q,q)')^{3/2}}
    \lt(
        \partial_1\xi^1(q,q)'\chi(q) + \partial_1\xi^1(q,q)\chi'(q)
    \rt)
    \de q
    \\
    &\quad\quad\quad+
    {
        \color{red}
        \int_0^1
        \frac
        {1}
        {(\xi^2(q,q)')^{1/2}}
        \lt(
            (\partial_{1,1}\xi^2(q,q))'\chi(q)^2 + 2\big(\partial_{1,1}\xi^2(q,q)\big)\chi(q)\chi'(q)
        \rt)
        \de q 
    }
    \\
    &\quad\quad\quad+
    \int_0^1
    \frac{
        -
        \partial_1\xi^2(q,q)' \chi(q)
        -
        \partial_1\xi^2(q,q)\chi'(q)
    }
    {2(\xi^2(q,q)')^{3/2}}
    \lt(
        \partial_1\xi^2(q,q)'\chi(q) + \partial_1\xi^2(q,q)\chi'(q)
    \rt)
    \de q.
    \\
    %%%%%%%%
    &=
    %%%%%%%%
    \frac{1}{2}
    \int_0^1 
    \frac{
    48q\chi(q)
    -
    48
    \chi'(q)
    }
    {
    (24q^2+48)^{1/2}
    }
    \chi'(q)
    \de q
    \\
    &\quad\quad\quad+
    \int_0^1
    \frac
    {
    \lt(
        48\chi(q)^2 + 96q\chi(q)\chi'(q)
    \rt)
    }
    {(24q^2+48)^{1/2}}
    \de q
    \\
    &\quad\quad\quad+
    \int_0^1
    \frac{
        48\chi'(q)
        -
        48q\chi(q)
    }
    {2(24q^2+48)^{3/2}}
    \lt(
        48q\chi(q) + 24q^2\chi'(q)
    \rt)
    \de q
    \\
    &\quad\quad\quad+
    \int_0^1
    \frac{
    -48\chi'(q)
    }
    {2(24q^2+48)^{3/2}}
    \lt(
        48\chi'(q)
    \rt)
    \de q    
    \\
    %%%%%%%%v2
    &=
    %%%%%%%%v2
    \sqrt{24}
    \int_0^1 
    \frac{
    q\chi(q)
    -
    \chi'(q)
    }
    {
    (q^2+2)^{1/2}
    }
    \chi'(q)
    \de q
    \\
    &\quad\quad\quad+
    \sqrt{24}
    \int_0^1
    \frac
    {
    \lt(
        2\chi(q)^2 + 4q\chi(q)\chi'(q)
    \rt)
    }
    {(q^2+2)^{1/2}}
    \de q
    \\
    &\quad\quad\quad+
    \sqrt{24}
    \int_0^1
    \frac{1}
    {(q^2+2)^{3/2}}
    \Big(
    \big(
        \chi'(q)-q\chi(q)
    \big)
    \cdot
    \big(
        2q\chi(q) + q^2\chi'(q)
    \big)
    -
    2\chi'(q)^2
    \Big)
    \de q.
\end{align*}


Simplifying further,
\begin{align*}
    \frac{\deriv{\delta}G}{\sqrt 6}
    &=
    \int_0^1 
    \frac{
    q\chi(q)
    -
    \chi'(q)
    }
    {
    (q^2+2)^{1/2}
    }
    \chi'(q)
    \de q
    \\
    &\quad\quad\quad+
    \int_0^1
    \frac
    {
    \lt(
        2\chi(q)^2 + 4q\chi(q)\chi'(q)
    \rt)
    }
    {(q^2+2)^{1/2}}
    \de q
    \\
    &\quad\quad\quad+
    \int_0^1
    \frac{1}
    {(q^2+2)^{3/2}}
    \Big(
    \big(
        \chi'(q)-q\chi(q)
    \big)
    \cdot
    \big(
        2q\chi(q) + q^2\chi'(q)
    \big)
    -
    2\chi'(q)^2
    \Big)
    \\
    %%%
    &=
    %%%
    \int_0^1 
    \frac{
    1
    }
    {
    (q^2+2)^{1/2}
    }
    \lt(
    2\chi(q)^2 + 5q\chi(q)\chi'(q)-\chi'(q)^2
    \rt)
    \de q
    \\
    &\quad\quad\quad+
    \int_0^1
    \frac{1}
    {(q^2+2)^{3/2}}
    \Big(
    \big(
        \chi'(q)-q\chi(q)
    \big)
    \cdot
    \big(
        2q\chi(q) + q^2\chi'(q)
    \big)
    -
    2\chi'(q)^2
    \Big)
    \de q
    .
\end{align*}
Note that this is a quadratic form in $(\chi(q),\chi'(q))$ as we expect. Now we want to find an example function $\chi$ making this strictly positive. This will show that $\Phi(q)=(q,q)$ is \textbf{not} a local maximum for the ALG functional. 

We try
\begin{align*}
    \chi(q)&=\sin(\pi cq);
    \\
    \chi'(q)&=\pi c\cos(\pi cq).
\end{align*}
on $q\in [0,1/c]$, for $c=\frac{1}{10}$ and $c=\frac{1}{20}$. 

\begin{figure}[h]
\includegraphics[width=0.45\linewidth]{imgs/wolfram-pic-10.png}
\includegraphics[width=0.45\linewidth]{imgs/wolfram-pic-20.png}
\caption{This calculation showcases the beauty of the multi-species setting.}
\end{figure}


It would be good to numerically verify the above calculations by manually computing ALG. Once that seems ok, it would be nice to look into a rigorous numerical integrator. (Although the integrand seems tame enough to trust Wolfram Alpha.)
