\section{Deferred Proofs From Section~\ref{sec:alg}}
\label{app:alg}

\subsection{Existence of a Maximizer: Proof of Proposition~\ref{prop:F-max}}
\label{subsec:maximizer-existence}

\propFmax*


Given $(p,\Phi,q_0)\in\cM$ we extend $p,\Phi$ to domain $[0,1]$ by setting $p(q)=0$ for $q\in [0,q_0)$ and making $\Phi$ linear on $[0,q_0]$ with $\Phi(0)=\vzero$. Using this canonical extension we equip $\cM$ with the metric 
    \begin{equation}
    \label{eq:metric}
    d\lt((p^1,\Phi^1,q_0^1),(p^2,\Phi^2,q_0^2)\rt)
    =
    \norm{p^1-p^2}_{L^1([0,1])}
    +
    \norm{\Phi^1-\Phi^2}_{L^1([0,1])}
    + 
    |q_0^1-q_0^2|.
\end{equation}


We will prove that $\cM$ is a compact space on which $\bbA$ is upper semi-continuous. Existence of a triple $(p,\Phi;q_0)\in\cM$ maximizing \eqref{eq:alg} within this space then follows.


\begin{proposition}
\label{prop:compact-space}
The space $\cM$ with metric \eqref{eq:metric} is compact.
\end{proposition}

\begin{proof}
Given an infinite sequence $(p^n,\Phi^n,q_0^n)_{n\geq 0}$ of points in $\cM$, we show there is a limit point. First find a subsequence $(a_n)$ along which the convergence $q_0^{a_n}\to q_0$ holds. Then the subsequence $(p^{a_n})_{n\geq 0}$ has a subsubsequential limit in the space $L^1([q_0,1])$; similarly for $(\Phi_s^{a_n})_{n\geq 0}$, for each $s\in\sS$. Thus we may choose a subsequence $b_n$ of $a_n$ on which $p^{b_n}\to p$ and $\Phi_s^{b_n}\to \Phi_s$ (for all $s\in \sS$) in $L^1([q_0,1])$. It is easy to see that $p$ and each $\Phi_s$ vanishes on $[0,q_0)$, and that $\Phi$ satisfies admissibility. It is easy to see that 
\[
    \|p^{b_n}-p\|_{L^1([0,1])}
    \leq
    \|p^{b_n}-p\|_{L^1([q_0,1])}
    +
    |q_0-q^{b_n}_0|
\]
and 
\[
    \|\Phi_s^{b_n}-\Phi_s\|_{L^1([0,1])}
    \leq
    \|\Phi_s^{b_n}-\Phi_s\|_{L^1([q_0,1])}
    +
    |q_0-q^{b_n}_0|.
\]
It follows that $(p^{b_n},\Phi^{b_n},q_0^{b_n})\to (p,\Phi,q_0)$ in $\cM$. This completes the proof.
\end{proof}


\begin{proposition}
\label{prop:F-bounded}
The function $\bbA$ is uniformly bounded on $\cM$.
\end{proposition}


\begin{proof}
For any admissible $\Phi$ we have by Cauchy-Schwarz
\begin{align}
\nonumber
    \sum_{s\in\sS}\lambda_s\int_0^1 \sqrt{{\Phi}'_s(q)(p\times \xi^s\circ\Phi)'(q)}\de q
    &\leq 
    \sum_{s\in\sS}\lambda_s\int_0^1 \lt({\Phi}_s'(q)+(p\times \xi^s\circ\Phi)'(q)\rt)\de q
    \\
\label{eq:F-bounded}
    &\leq
    \sum_{s\in\sS}\lambda_s \lt(1+\xi^s(\vone)-\xi^s(\vzero)\rt).
\end{align}
The first term of $\bbA$ is clearly uniformly bounded, so the result follows.
\end{proof}


\begin{proposition}
\label{prop:F-usc}
$\bbA$ is upper semi-continuous on $\cM$.
\end{proposition}


\begin{proof}
Suppose $(p^{b_n},\Phi^{b_n},q_0^{b_n})\to (p,\Phi,q_0)$ in $\cM$. We write
\begin{align*}
    |\bbA(p^{b_n},\Phi^{b_n};q_0^{b_n})-\bbA(p,\Phi;q_0)|
    &\leq
    \int_{q_0}^1
    \lt|\sqrt{(\Phi_s^{b_n})'(q)(p^{b_n}\times \xi^s\circ\Phi^{b_n})'(q)}
    -
    \sqrt{\Phi'_s(q)(p\times \xi^s\circ\Phi)'(q)}\rt|\de q
    \\
    &  \quad +
    C_{\lambda}
    \lt(
    \lt|\int_{q_0^{b_n}}^{q_0}
    \sqrt{p'(q)+1}
    \,\de q\,
    \rt|
    +
    \sum_{s\in \sS}
    \lt|\sqrt{\Phi_s^n(q_0^n)}-\sqrt{\Phi_s(q_0)}\rt|
    \rt)
    .
\end{align*}
The sum over $s\in\sS$ obviously tends to $0$. Moreover by Cauchy--Schwarz, 
\begin{align*}
    \lt|\int_{q_0^{b_n}}^{q_0}
    \sqrt{p'(q)+1}
    \,\de q\,
    \rt|
    &\leq
    |q_0-q_0^{b_n}|^{1/2}\cdot \sqrt{p(q_0)-p(q_0^{b_n})+1}
    \\
    &\leq
    C'_{\lambda}(q_0-q_0^{b_n})^{1/2}.
\end{align*}
Therefore it suffices to show the first term above tends to $0$. Since the map
\[
    (p,\Phi)\mapsto (p\times \xi^s\circ\Phi)
\]
from $L^1([0,1])^{|\sS|+1}\to L^1([0,1])$ is continuous and returns a non-decreasing function, it suffices to show that
\[
    G(f,g)=\int_0^1 \sqrt{f'(q)g'(q)}\,\de q
\]
is upper semi-continuous on $L^1([0,1])\times L^1([0,1])$ when restricted to non-decreasing functions. This is essentially equivalent to upper semi-continuity of Hellinger distance which is well-known.
\end{proof}

Combining the results above implies Proposition~\ref{prop:F-max}.


\subsection{A Priori Regularity of Maximizers}
\label{subsec:regularity-for-4.1}

Let $(p,\Phi,q_0)\in\cM$ be a maximizer of $\bbA$, which exists by Proposition~\ref{prop:F-max}. In this subsection we will prove the following two propositions. 

\propBasicRegularity*

\propPBasic*

\begin{lemma}
    \label{lem:p-AC}
    The function $p$ is absolutely continuous and $p(1)=1$. Moreover $p'$ is uniformly bounded on compact subsets of $(q_0,1)$.
\end{lemma}
\begin{proof}
    Given any increasing $p:[q_0,1]\to [q_0,1]$, we may view $p'$ as a positive measure of the form
    \begin{equation}
    \label{eq:positive-measure-of-the-form}
        p'(x)dx = f(x)dx + \mu(dx)
    \end{equation}
    for $\mu$ a singular-plus-atomic measure and $f\in L^1([q_0,1];\mathbb R_{\geq 0})$. We may then replace $p$ by $\bar p$ such that 
    \[
        \bar p'(x)dx=f(x)dx,\quad\text{ and }\quad \bar p(1)=1.
    \]
    Then $\bar p(x)\geq p(x)$ for all $x\in [q_0,1]$, and $\bar p'(x)$ agrees with $p'(x)$ except for a singular-plus-atomic part. It follows that 
    \[
        \bbA(p,\Phi;q_0)\leq \bbA(\bar p,\Phi;q_0).
    \]
    Moreover it is easy to see that strict inequality $\bbA(p,\Phi;q_0)< \bbA(\bar p,\Phi;q_0)$ holds whenever $p\neq \bar p$. We conclude that $p$ is absolutely continuous and $p(1)=1$.
    
    
    To show the latter statement, we use a similar argument with more care. Let $q\in (q_0,1)$ and choose a large constant $C=C(q_0,q)$. Recalling \eqref{eq:positive-measure-of-the-form}, suppose $\|f(x)\|_{L^{\infty}([q,1])}> C$ for a large constant $C$ and let 
    \[
    c\equiv\frac{\int_q^1 (f(x)-C)_+~\de x}{q-q_0}.
    \]
    We may replace $f$ by
    \[
        f_C(x)=
        \begin{cases}
        f(x),\quad x\in [0,q_0)\\
        f(x)+c,\quad x\in [q_0,q)\\
        \min(C,f(x)),\quad x\in [q,1]
        \end{cases}
    \]
    and similarly replace $p$ by $p_C$ with 
    \[
        \bar p_C'(x)dx=f_C(x)~\de x,\quad\text{ and }\quad \bar p(1)=1.
    \]
    It is easy to see that $p_C(x)\geq p(x)$ for each $x\in [0,1]$. Keeping $\Phi$ the same, we consider the change in $\bbA$. The decrease in $\bbA$ on $[q,1]$ is at most
    \begin{equation}
    \label{eq:q-1}
    \begin{aligned}
        &\sum_{s\in\sS}
        \lambda_s
        \int_q^1
        \sqrt{\Phi_s'(x)(p\times \xi^s\circ\Phi)'(x)}
        -
        \sqrt{\Phi_s'(x)(p_C\times \xi^s\circ\Phi)'(x)}
        ~\de x
        \\
        &=
        \sum_{s\in\sS}
        \lambda_s
        \int_q^1
        \sqrt{\Phi_s'(x)\lt(p'(x)\xi^s\big(\Phi(x)\big)+p(x)\langle \Phi'(x),\nabla\xi^s\big(\Phi(x)\big)\rangle\rt)}
        \\
        &\quad\quad\quad\quad\quad\quad -
        \sqrt{\Phi_s'(x)\lt(p_C'(x)\xi^s\big(\Phi(x)\big)+p_C(x)\langle \Phi'(x),\nabla\xi^s\big(\Phi(x)\big)\rangle\rt)}
        ~\de x
        \\
        &\leq
        \sum_{s\in\sS}
        \lambda_s
        \int_q^1
        \sqrt{\Phi_s'(x)\lt(p'(x)\xi^s\big(\Phi(x)\big)+p(x)\langle \Phi'(x),\nabla\xi^s\big(\Phi(x)\big)\rangle\rt)}
        \\
        &\quad\quad\quad\quad\quad\quad -
        \sqrt{\Phi_s'(x)\lt(p_C'(x)\xi^s\big(\Phi(x)\big)+p(x)\langle \Phi'(x),\nabla\xi^s\big(\Phi(x)\big)\rangle\rt)}
        ~\de x
        \\
        &\leq
        \sum_{s\in\sS}
        \lambda_s
        \int_q^1
        \sqrt{\Phi_s'(x) p'(x)\cdot\xi^s\big(\Phi(x)\big)}
        -
        \sqrt{\Phi_s'(x)p_C'(x)\cdot\xi^s\big(\Phi(x)\big)}
        ~\de x
        \\
        &\leq
        O(1)\cdot
        \int_q^1
        \sqrt{p'(x)}-\sqrt{p_C'(x)}
        \de x
        \\
        &\leq
        O(1)\cdot \int_{q}^1 C^{-1/2}(f(x)-C)_+ ~\de x
        \\
        &\leq
        O\lt(\frac{c(q-q_0)}{\sqrt{C}}\rt)
        .
    \end{aligned}
    \end{equation}
    (In the second inequality we used $\sqrt{x+z}-\sqrt{y+z}\leq \sqrt{x}-\sqrt{y}$ for $x\geq y\geq 0$, and in the third we used that $\Phi_s'$ is uniformly bounded by admissibility.)
    On $x\in [q_0,q]$, we find that changing from $p$ to $p_C$ increases the value of $\bbA$:
    \begin{align*}
        &\sum_{s\in\sS}
        \lambda_s
        \int_{q_0}^q
        \sqrt{\Phi_s'(x)(p_C\times \xi^s\circ\Phi)'(x)}
        -
        \sqrt{\Phi_s'(x)(p\times \xi^s\circ\Phi)'(x)}
        ~\de x
        \\
        &=
        \sum_{s\in\sS}
        \lambda_s
        \int_{q_0}^q
        \sqrt{\Phi_s'(x)\lt(p_C'(x)\xi^s\big(\Phi(x)\big)+p_C(x)\langle \Phi'(x),\nabla\xi^s\big(\Phi(x)\big)\rangle\rt)}
        \\
        &\quad\quad -
        \sqrt{\Phi_s'(x)\lt(p'(x)\xi^s\big(\Phi(x)\big)+p(x)\langle \Phi'(x),\nabla\xi^s\big(\Phi(x)\big)\rangle\rt)}
        ~\de x
        \\
        &\geq
        \sum_{s\in\sS}
        \lambda_s
        \int_{q_0}^q
        \sqrt{\Phi_s'(x)\lt((p'(x)+c)\xi^s\big(\Phi(x)\big)+p(x)\langle \Phi'(x),\nabla\xi^s\big(\Phi(x)\big)\rangle\rt)}
        \\
        &\quad\quad -
        \sqrt{\Phi_s'(x)\lt(p'(x)\xi^s\big(\Phi(x)\big)+p(x)\langle \Phi'(x),\nabla\xi^s\big(\Phi(x)\big)\rangle\rt)}
        ~\de x
        \\
        &\geq
        \Omega(c)
        \cdot
        \int_{q_0}^q  
        \sum_{s\in\sS}
        \frac{\de x}{ \sqrt{\Phi_s'(x)\lt((p'(x)+c)\xi^s\big(\Phi(x)\big)+p(x)\langle \Phi'(x),\nabla\xi^s\big(\Phi(x)\big)\rangle\rt)}}\,.
    \end{align*}
    By Markov's inequality, $p'(x)\leq \frac{2}{(q-q_0)}$ on a set of $x\in [q_0,q]$ of measure at least $\frac{q-q_0}{2}$. For each such $x$, we have $\Phi_s'(x)\leq O(1)$ and $\xi^s(\Phi(x))\leq O(1)$. We thus find
    \[
        \Omega(c)
        \cdot
        \int_{q_0}^q  
        \sum_{s\in\sS}
        \frac{\de x}{ \sqrt{\Phi_s'(x)\lt((p'(x)+c)\xi^s\big(\Phi(x)\big)+p(x)\langle \Phi'(x),\nabla\xi^s\big(\Phi(x)\big)\rangle\rt)}}
        \geq
        \Omega\lt(\frac{c(q-q_0)}{\sqrt{q-q_0+c}}\rt)
    \]
    Since $c\leq \frac{1}{q-q_0}$, for $C$ sufficiently large, combining with \eqref{eq:q-1} above implies that
    \[
    \sum_{s\in\sS}
        \lambda_s
        \int_{q}^1
        \sqrt{\Phi_s'(x)(p_C\times \xi^s\circ\Phi)'(x)}
        -
        \sqrt{\Phi_s'(x)(p\times \xi^s\circ\Phi)'(x)}
        ~\de x>0.
    \]
    Since $p(x)=p_C(x)$ for $x\leq q_0$, we find $\bbA(p,\Phi;q_0)<\bbA(p_C,\Phi;q_0)$, contradicting maximality of $\bbA(p,\Phi;q_0)$. Having reached a contradiction for $C$ sufficiently large, we conclude that $p'$ is uniformly bounded on $[q,1]$ for each $q\in (q_0,1)$ as desired.
\end{proof}


\begin{lemma}
    \label{lem:p-positive}
    $p(q)>0$ holds for all $q>q_0$.
\end{lemma}
\begin{proof}
    Suppose not. Then $p(q)=0$ for all $q\in [q_0,q_0+\eps]$, for some $\eps>0$. For $\delta>0$ small, define
    \[
    p_{\delta}(q)=\delta+(1-\delta)p(q).
    \]
    Then 
    \begin{align*}
        \sum_{s\in \sS}\lambda_s\int_{q_0}^{q_0+\eps}\sqrt{\Phi_s'(q)(p_{\delta}\times\xi^s\circ\Phi)'(q)}\de q
        &=
        \delta^{1/2}\sum_{s\in \sS}\lambda_s\int_{q_0}^{q_0+\eps} \sqrt{\Phi_s'(q)(\xi^s\circ\Phi)'(q)}\de q
        \\
        &\geq
        \delta^{1/2}c(\xi)\sum_{s\in\sS}\lambda_s\int_{q_0}^{q_0+\eps}\sqrt{\Phi_s'(q)^2}\de q\\
        &=
        \delta^{1/2}c(\xi).
    \end{align*}
    while
    \[
        \sum_{s\in \sS}\lambda_s\int_{q_0}^{q_0+\eps}\sqrt{\Phi_s'(q)(p\times\xi^s\circ\Phi)'(q)}\de q
        =
        0.
    \]
    On the other hand since $p_{\delta}(q)\geq p(q)$ for all $q\in [q_0,1]$ and $(p_{\delta})'=(1-\delta)(p)'$ as measures, we obtain
    \begin{align*}
        \sum_{s\in \sS}\lambda_s\int_{q_0+\eps}^1\sqrt{\Phi_s'(q)(p_{\delta}\times\xi^s\circ\Phi)'(q)}\de q
        &\geq 
        (1-\delta)\sum_{s\in \sS}\lambda_s\int_{q_0+\eps}^1 \sqrt{\Phi_s'(q)(p\times \xi^s\circ\Phi)'(q)}\de q.
    \end{align*}
    Combining the above implies $\bbA(p_{\delta},\Phi;q_0)>\bbA(p,\Phi;q_0)$ for small enough $\delta$, a contradiction. 
\end{proof}


\begin{lemma}
\label{lem:Phi-q0-neq-1}
    For all $s\in\sS$ and $q\in (0,1)$, we have $\Phi_s(q)<1$.
\end{lemma}

\begin{proof}
    Suppose $\Phi_{s_0}(q_*)=1$; this implies $0<q_0\leq q_*<1$. For small $\delta>0$ we consider the perturbation $\Phi_{\delta}$ with $\Phi_{\delta,s}=\Phi_s$ for $s\neq s_0$ and:
    \[
    \Phi_{\delta,s_0}'(q)
    =
    \begin{cases}
    \Phi_{s_0}'(q)\cdot (1-\delta(1-q_*)),\quad q\in [0,q_*],
    \\
    \delta,\quad\quad\quad\quad\quad\quad\quad\quad\quad\quad q\in [q_*,1]
    \end{cases}
    \]
    Note that $\Phi_{\delta,s}'(q)\geq (1-O(\delta))\Phi_{s}'(q)$ and so also $\Phi_{\delta,s}(q)\geq (1-O(\delta))\Phi_{s}(q)$ for all $s\in\sS$ and $q\in [0,1]$. As $\bbA$ is uniformly bounded, we can thus bound
    \begin{align*}
    \bbA(p,\Phi_{\delta};q_0)-\bbA(p,\Phi;q_0)
    &=
    \sum_{s\in \sS}
    h_s \lambda_s \sqrt{\Phi_{\delta,s}(q_0)}
    +
    \lambda_s 
    \int_{q_0}^1
    \sqrt{\Phi'_{\delta,s}(q) (p\times \xi^s \circ \Phi_{\delta})'(q)}
    ~\de q
    \\
    &\quad\quad
    -
    \sum_{s\in \sS}
    h_s \lambda_s \sqrt{\Phi_s(q_0)}
    -
    \lambda_s 
    \int_{q_0}^1
    \sqrt{\Phi'_s(q) (p\times \xi^s \circ \Phi)'(q)}
    ~\de q
    \\
    &\geq
    -O(\delta)
    +
    \lambda_{s_0}
    \int_{\frac{1+q_*}{2}}^1
    \sqrt{\Phi'_{\delta,s}(q) (p\times \xi^s \circ \Phi)'(q)}
    -
    \sqrt{\Phi'_s(q) (p\times \xi^s \circ \Phi)'(q)}
    ~\de q
    .
    \end{align*}
    Using Lemma~\ref{lem:p-positive}, admissibility and non-degeneracy of $\xi$, we find that $(p\times \xi^s \circ \Phi)'(q)\geq \Omega(q)$ for all $q\geq \frac{1+q_*}{2}$. Therefore 
    \[
    \lambda_{s_0}
    \int_{\frac{1+q_*}{2}}^1
    \sqrt{\Phi'_{\delta,s}(q) (p\times \xi^s \circ \Phi)'(q)}
    -
    \sqrt{\Phi'_s(q) (p\times \xi^s \circ \Phi)'(q)}
    ~\de q
    \geq \Omega(\delta^{1/2})
    \]
    for small $\delta$. Since $\delta^{1/2}$ is of larger order than $\delta$ we conclude that $\bbA(p,\Phi_{\delta};q_0)>\bbA(p,\Phi;q_0)$. This is a contradiction (recall Lemma~\ref{lem:admissible-optional}) and completes the proof.
\end{proof}



Next we turn our attention to $\Phi'$. Similarly to Lemma~\ref{lem:p-positive}, the idea is that the square root function has infinite derivative at $0$. 
 


\begin{lemma}
    \label{lem:Phi-inc}
    There exists $\eta>0$ such that $\Phi'(q) \succeq \eta \vone$ almost everywhere in $q\in [q_0,1]$.
\end{lemma}
\begin{proof}
    First, given $(p,\Phi;q_0)$ choose for some $s\in\sS$ (specified below) a Lebesgue point $q_s\in (q_0,1)$ of $\Phi'$ with 
    \begin{equation}
    \label{eq:Phi-s-1}
        \Phi_s'(q_s)\geq a
    \end{equation}
    for $a>0$. Lemma~\ref{lem:Phi-q0-neq-1} ensures this is possible for some $a$ depending only on $q_0$ and $\Phi(q_0)$ (as long as $q_0<1$, else there is nothing to prove). In fact we can actually find two distinct such points $q_s^{(1)},q_s^{(2)}$ (which will be helpful below).

    Next for small $\eps>0$ depending only on $(p,\Phi)$, define the interval
    \begin{align*}
        J_{s,\eps}&=(q_s-\eps,q_s+\eps).
    \end{align*}
    By \eqref{eq:Phi-s-1} and the fact that $q_s$ is a Lebesgue point of $\Phi'$, there is a subset $I_{s,\eps}\subseteq J_{s,\eps}$ of Lebesgue measure at least $|I_{s,\eps}|\geq \frac{|J_{s,\eps}|}{2}=\eps$ such that
    \begin{equation}
    \label{eq:Phi-s-half}
        \Phi_s'(q)\geq \frac{a}{2} 
        ,\quad
        \forall q\in I_{s,\eps}.
    \end{equation}
    as long as $\eps>0$ is chosen sufficiently small.
    A simple consequence is the estimate
    \begin{equation}
    \label{eq:C-eps}
        C_{\eps}:=\Phi_s(q_s+\eps)-\Phi_s(q_s-\eps)=\int_{q_s-\eps}^{q_s+\eps} \Phi_s'(q)\de q\geq \frac{a\eps}{2}.
    \end{equation}


    With the setup above complete (except that $s$ is not yet specified), suppose the conclusion is false let $\eta$ be sufficiently small depending on $(p,\Phi,\eps)$ for $\eps$ as above. Then there exist $s,{s_0}\in\sS$ and 
    $\hat q_{s_0}\in (q_0,1)$ which is a Lebesgue point for $\nabla \Phi$ such that 
    \begin{align}
    \label{eq:Phi-s-0}
        \Phi_s'(\hat q_{s_0})&\leq \eta,
        \\
    \label{eq:Phi-r-1}
        \Phi_{s_0}'(\hat q_{s_0})&\geq 1.
    \end{align}
    Indeed if $q$ is any Lebesgue point of $\nabla \Phi$ satisfying \eqref{eq:Phi-s-0} for some $s$, then \eqref{eq:Phi-r-1} holds for some ${s_0}\neq s$ by admissibility and we define $\hat q_{s_0}=q$ this way. The bound \eqref{eq:Phi-s-0} determines the species $s$ chosen initially.
    
    As $\hat q_{s_0}$ is also a Lebesgue point of $\Phi'$, in light of \eqref{eq:Phi-s-0} and \eqref{eq:Phi-r-1}, there exists a set $I_{{s_0},\eta}\subseteq J_{{s_0},\eps}=(\hat q_{s_0}-\eps,\hat q_{s_0}+\eps)$ of positive Lebesgue measure such that the inequalities
    \begin{align}
    \label{eq:Phi-s-eta}
        \Phi_s'(q)&\leq 2\eta,
        \\
    \label{eq:Phi-r-half}
        \Phi_{s_0}'(q)&\geq \frac{a}{2}.
    \end{align}
    both hold for all $q\in I_{{s_0},\eta}$.
    Moreover we can assume $J_{s,\eps},J_{s_0,\eps}$ are disjoint, i.e. $|q_s-\hat q_{s_0}|> 2\eps$. Indeed as noted earlier we can choose two candidate points $q_s^{(1)},q_s^{(2)}$. If $\eps<|q_s^{(1)}-q_s^{(2)}|/5$ is taken, at least one of them suffices for any $\hat q_{s_0}\in (q_0,1)$.
    
    
    Next choose $\delta\in (0,\eta)$ small and consider the perturbation $\Phi_{\delta}$ with $\Phi_{\delta}(q_0)=\Phi(q_0)$ and
    \[
       \Phi_{\delta,s}'(q)
        =
        \begin{cases}
        \Phi_s'(q)+\delta,\quad q\in I_{s_0,\eta}
        \\
        \Phi_s'(q)\lt(1-\frac{\delta |I_{s_0,\eta}|}{C_{\eps}}\rt),\quad \forall q\in J_{s,\eps}
        \\
        \Phi_s'(q),\quad \text{otherwise}
        \end{cases}
    \]
    and $\Phi_{\delta,s'}=\Phi_{s'}$ for all $s'\in \sS\backslash \{s\}$. (Note we used disjointness of $J_{s,\eps},J_{s_0,\eps}$ for this definition to make sense.)
    By Lemma~\ref{lem:admissible-optional}, we must have $\bbA(p,\Phi_{\delta};q_0)\geq \bbA(p,\Phi;q_0)$ although $\Phi_{\delta}$ may not be admissible. Then for $\delta\leq\eta$,
    \begin{align}
    \nonumber
        \int_{I_{s_0,\eta}}
        \sqrt{\Phi_{\delta,s}'(q)(p\times\xi^s\circ\Phi)'(q)}
        -
        &
        \sqrt{\Phi_s'(q) (p\times\xi^s\circ\Phi)'(q)}
        \de q
        \\
    \nonumber
        &\stackrel{\eqref{eq:Phi-s-eta}}{\geq}
        (\sqrt{2\eta+\delta}-\sqrt{2\eta})
        \int_{I_{s_0,\eta}}\sqrt{(p\times\xi^s\circ\Phi)'(q)}\de q
        \\
    \nonumber
        &\geq
        \frac{\delta p(q_s-\eps)^{1/2}}{10\eta^{1/2}}
        \int_{I_{s_0,\eta}}
        \sqrt{(\xi^s\circ\Phi)'(q)}\de q
        \\
    \nonumber
        &\geq 
        \frac{\delta p(q_s/2)^{1/2}c(\xi)}
        {10\eta^{1/2}}
        \int_{I_{s_0,\eta}}
        \sqrt{\Phi_{s_0}'(q)}\de q
        \\
    \label{eq:Phi-gain}
        &\stackrel{\eqref{eq:Phi-r-half}}{\geq}
        \frac{\delta a^{1/2} p(q_s/2)^{1/2}c(\xi)|I_{s_0,\eta|}}
        {20\eta^{1/2}}
        .
    \end{align}
    We used non-degeneracy of $\xi$ in the penultimate step. On the other hand recalling \eqref{eq:C-eps}, it follows that for all $\wt{s}\in\sS$ and almost all $q\in [q_0,1]$:
    \begin{equation}
    \label{eq:Phi-comp-1}
        \Phi_{\delta,\wt{s}}'(q)\geq \lt(1-O\lt(\frac{\delta|I_{s_0,\eta}|}{\eps}\rt)\rt)
        \Phi_{\wt{s}}'(q).
    \end{equation}
    Integrating on $[q_0,q]$, we find 
    \begin{equation}
    \label{eq:Phi-comp-2}
        \Phi_{\delta,\wt{s}}(q)\geq \lt(1-O\lt(\frac{\delta|I_{s_0,\eta}|}{\eps}\rt)\rt)
        \Phi_{\wt{s}}(q)
    \end{equation}
    for all $q\in [q_0,1]$. By the chain rule we similarly obtain that for all $\wt{s}\in\sS$,
    \begin{align}
    \label{eq:Phi-comp-3}
        (p\times \xi^{\wt{s}}\circ\Phi_{\delta})'
        &\geq 
        \lt(1-O\lt(\frac{\delta|I_{s_0,\eta}|}{\eps}\rt)\rt)(p\times \xi^{\wt{s}}\circ\Phi)',
        \\
    \label{eq:Phi-comp-4}
        (p\times \xi^{\wt{s}}\circ\Phi_{\delta})(q)
        &\geq 
        \lt(1-O\lt(\frac{\delta|I_{s_0,\eta}|}{\eps}\rt)\rt)(p\times \xi^{\wt{s}}\circ\Phi)(q).
    \end{align}
    It follows from \eqref{eq:Phi-comp-1}, \eqref{eq:Phi-comp-2}, \eqref{eq:Phi-comp-3}, \eqref{eq:Phi-comp-4} that
    \begin{align}
    \nonumber
        \int_{I_{s_0,\eta}}
        \sqrt{\Phi_{\delta,s}'(q)(p\times\xi^s\circ\Phi_{\delta})'(q)}
        &\geq
        \lt(1-O\lt(\frac{\delta|I_{s_0,\eta}|}{\eps}\rt)\rt)
        \int_{I_{s_0,\eta}}
        \sqrt{\Phi_{\delta,s}'(q) (p\times\xi^s\circ\Phi)'(q)}
        \de q
        \\
    \label{eq:Phi-extra-term}
        &\stackrel{\eqref{eq:F-bounded}}{\geq}
        \int_{I_{s_0,\eta}}
        \sqrt{\Phi_{\delta,s}'(q) (p\times\xi^s\circ\Phi)'(q)} \de q
        - 
        O\lt(\frac{\delta|I_{s_0,\eta}|}{\eps}\rt).
    \end{align}
    Since $\Phi_{\delta}$ and $\Phi$ differ only inside $[q_0,1]$ we use $\bbA_{[q_0,1]}$ below to denote the second term of $\bbA$. We have:
    \begin{align*}
        \bbA_{[q_0,1]}(p,\Phi)
        &=
        \sum_{\wt{s}\in\sS}\lambda_{\wt{s}}\int_{q_0}^1 \sqrt{\Phi_{\wt{s}}'(q)(p\times \xi^{\wt{s}}\circ\Phi)'(q)}\de q
        \\
        &=
        \lambda_s\int_{I_{s_0,\eta}} \sqrt{\Phi_s'(q)(p\times \xi^{s}\circ\Phi)'(q)}\de q
        +
        \lambda_s\int_{[q_0,1]\backslash I_{s_0,\eta}} \sqrt{\Phi_s'(q)(p\times \xi^{s}\circ\Phi)'(q)}\de q
        \\
        &
        \qquad 
        +
        \sum_{\wt{s}\in\sS\backslash\{s\}}\lambda_{\wt{s}}\int_{q_0}^1 \sqrt{\Phi_{\wt{s}}'(q)(p\times \xi^{\wt{s}}\circ\Phi)'(q)}\de q
        \\
        &\equiv
        \I+\II+\III.
    \end{align*}
    Similarly for $J$ instead of $I$, 
    \begin{align*}
        \bbA_{[q_0,1]}(p,\Phi_{\delta})
        &=
        \sum_{\wt{s}\in\sS}\lambda_{\wt{s}}\int_{q_0}^1 \sqrt{\Phi_{\delta,\wt{s}}'(q)(p\times \xi^{\wt{s}}\circ\Phi_{\delta})'(q)}\de q
        \\
        &=
        \lambda_s\int_{I_{s_0,\eta}} \sqrt{(\Phi_{\delta,s})'(q)(p\times \xi^{s}\circ\Phi_{\delta})'(q)}\de q
        +
        \lambda_s\int_{[q_0,1]\backslash I_{s_0,\eta}} \sqrt{(\Phi_{\delta,s})'(q)(p\times \xi^{s}\circ\Phi_{\delta})'(q)}\de q
        \\
        &
        \qquad 
        +
        \sum_{\wt{s}\in\sS\backslash\{s\}}\lambda_{\wt{s}}\int_{q_0}^1 \sqrt{\Phi_{\delta,\wt{s}}'(q)(p\times \xi^{\wt{s}}\circ\Phi_{\delta})'(q)}\de q
        \\
        &\equiv
        \I_{\delta}+\II_{\delta}+\III_{\delta}.
    \end{align*}
    Using \eqref{eq:Phi-comp-1}, \eqref{eq:Phi-comp-2}, \eqref{eq:Phi-comp-3}, \eqref{eq:Phi-comp-4} again, we obtain
    \begin{align*}
        \II_{\delta}&\geq \lt(1-O\lt(\frac{\delta|I_{s_0,\eta}|}{\eps}\rt)\rt)\II,
        \\
        \III_{\delta}&\geq \lt(1-O\lt(\frac{\delta|I_{s_0,\eta}|}{\eps}\rt)\rt)\III.
    \end{align*}
    Meanwhile \eqref{eq:Phi-gain} and \eqref{eq:Phi-extra-term} imply that for $\delta$ small compared to $\eta$,
    \[
    \I_{\delta}\geq \lt(1-O\lt(\frac{\delta|I_{s_0,\eta}|}{\eps}\rt)\rt)\I + \frac{\delta a^{1/2}p(q_s/2)^{1/2}c(\xi)|I_{s_0,\eta}|}
        {20\eta^{1/2}}.
    \]
    Combining, we find
    \[
        \bbA_{[q_0,1]}(p,\Phi_{\delta})\geq \bbA_{[q_0,1]}(p,\Phi)+
        \frac{\delta a^{1/2}p(q_s/2)^{1/2}c(\xi)|I_{s_0,\eta}|}
        {20\eta^{1/2}}-O\lt(\frac{\delta|I_{s_0,\eta}|}{\eps}\rt).
    \]
    Taking $\eta\ll \eps^2 ap(q_s/2)c(\xi)^2$ and then $\delta$ sufficiently small contradicts the maximality of $(p,\Phi,q_0)$, thus completing the proof.
\end{proof}




\begin{proposition}
    \label{prop:p-q0-0}
    If $q_0 > 0$, then $p(q_0)=0$.
\end{proposition}
\begin{proof}
    Assume that $p(q_0)>0$. Consider the perturbation
    \[
        \wtp(q)=
        \begin{cases}
        p(q)+(q-q_0-\eps)\delta,\quad q<q_0+\eps
        \\
        p(q),\quad\quad\quad\quad\quad\quad\quad q\geq q_0+\eps.
        \end{cases}
    \]
    The function $\tilde p$ is increasing, and is non-negative for sufficiently small $\eps,\delta>0$. For $q<q_0+\eps$ we find
    \begin{align*}
        \deriv{\delta} (p\times \xi^s\circ\Phi)'(q)
        &=
        \deriv{\delta}\lt(
            p'(q)\xi^s(\Phi(q))
            +
            p(q)(\xi^s\circ\Phi)'(q)
        \rt)
        \\
        &=
        \xi^s(\Phi(q))
        -
        (q_0+\eps-q)
        (\xi^s\circ\Phi)'(q)
        \\
        &\geq 
        \xi^s(\Phi(q))-O(\eps).
    \end{align*}
    If $q_0>0$, then $\xi^s(\Phi(q))\geq c(q_0)>0$ by admissibility and non-degeneracy of $\Phi$. This contradicts optimality of $(p,\Phi,q_0)$ and completes the proof.
\end{proof}


\begin{proof}[Proof of Proposition~\ref{prop:p-basic}]
    Follows from Lemmas~\ref{lem:p-AC} and \ref{lem:p-positive} and Proposition~\ref{prop:p-q0-0}.
\end{proof}



\subsubsection{Continuous Differentiability on $(q_0,1]$}
\label{subsubsec:C1-on-compact-subsets}

Here we show that $p$ and $\Phi$ are continuously differentiable on compact subsets of $(q_0,1]$ using another local perturbation argument. 

\begin{lemma}
    \label{lem:sqrt-xy-concave}
    The function $f(x,y)=\sqrt{xy}$ is concave on $\bbR_{>0}^2$, with strict concavity on all lines except for those passing through the origin.
\end{lemma}

\begin{proof}
    Given $x_0,y_0,x_1,y_1>0$ with $(x_0,y_0)\neq (x_1,y_1)$ and $c\in (0,1)$, we have
    \begin{align*}
        (x_0 y_1 - x_1 y_0)^2
        &\geq 
        0
        \\
        \implies 
        x_0^2 y_1^2 + x_1^2 y_0^2 + 2x_0 x_1 y_0 y_1
        &\geq 
        4 x_0 x_1 y_0 y_1
        \\
        \implies 
        (x_0 y_1 + x_1 y_0)
        &\geq 
        2\sqrt{x_0 x_1 y_0 y_1}
        \\
        \implies 
        c(1-c)\cdot (x_0 y_1 + x_1 y_0)
        &\geq 
        2c(1-c)\sqrt{x_0 x_1 y_0 y_1}
        \\
        \implies 
        c^2 x_0 y_0 + (1-c)^2 x_0 y_0) + c(1-c)\cdot (x_0 y_1 + x_1 y_0)
        &\geq 
        c^2 x_0 y_0 + (1-c)^2 x_0 y_0) + 
        2c(1-c)\sqrt{x_0 x_1 y_0 y_1}
        \\
        \implies 
        \sqrt{(cx_0+(1-c)x_1)(cy_0+(1-c)y_1) }
        &\geq 
        c\sqrt{x_0y_0}+(1-c)\sqrt{x_1y_1}.
    \end{align*}
    Moreover equality holds if and only if it holds in the first step.
\end{proof}

\begin{lemma}
    \label{lem:derivatives-continuous-apriori}
    Both $p$ and $\Phi$ are continuously differentiable on compact subsets of $(q_0,1]$.
\end{lemma}

\begin{proof}


We assume that $q_0<1$ (else there is nothing to prove), and recall Lemma~\ref{lem:p-positive} throughout. Admissibility implies that $\Phi$ is uniformly Lipschitz, and Lemma~\ref{lem:p-AC} shows that $p$ is uniformly Lipschitz on compact subsets of $(q_0,1)$. Hence both $p'(x)$ and $\Phi'_s$ exist as non-negative, integrable functions which are uniformly bounded away from $q_0$.  


By an elementary result of \cite{zaanen1986continuity}, if a measurable function $[q_0,1]\to\bbR$ does not agree with any continuous function on a full measure set, then it possesses a genuine point of discontinuity $q_*\in (q_0,1)$ such that $F$ cannot be made continuous at $q_*$ even by modification on a measure zero set. We fix such a point $q_*$ for sake of contradiction. By definition, this means that for some $\eta>0$ depending only on $(p,\Phi,q_*)$ and for arbitrarily small $\eps>0$, there exist measurable sets $I,J\subseteq (q_*-\eps,q_*+\eps)$ and $a\in \bbR$ such that:
\begin{equation}
\label{eq:eta-discontinuous-f}
\begin{aligned}
    |I|&= \eps_1>0,\\
    |J|&= \eps_1>0,\\
    f(q)&\geq a+\eta,\quad \forall q\in I,\\
    f(q)&\leq a-\eta,\quad \forall q\in J.
\end{aligned}
\end{equation}
Here $f(q)=p'(q)$ or $f(q)=\Phi'_s(q)$ for some $s\in\sS$.

Let $\gamma_I:[0,\eps_1]\to I$ and $\gamma_J:[0,\eps_1]\to J$ be increasing, measure-preserving bijections (and note that their inverse functions are also measurable). For convenience we set $q_{I,x}=\gamma_I(x)$ and $q_{J,x}=\gamma_J(x)$. We construct perturbations $\tilde p,\tilde\Phi$ of $p$ and $\Phi$ by averaging derivatives on $q_{I,x}$ and $q_{J,x}$:
\begin{align*}
   \tilde p'(q_{I,x})&= 
   \tilde p'(q_{J,x})
   = \frac{p'(q_{I,x})+ p'(q_{J,x})}{2}
    \,;\\
    \tilde p'(q) &= p'(q),\quad q\notin I\cup J
    \,;\\
    \tilde \Phi_s'(q_{I,x})&= 
   \tilde \Phi_s'(q_{J,x})
   = \frac{\Phi_s'(q_{I,x})+ \Phi_s'(q_{J,x})}{2}
    \,;\\
    \tilde \Phi_s'(q) &= \Phi_s'(q),\quad q\notin I\cup J.
\end{align*}
We claim that for fixed $q_*,\eta$ and sufficiently small $\eps>0$, we have
\begin{equation}
\label{eq:derivative-continuous}
    \bbA(\tilde p,\tilde\Phi;q_0)
    >
    \bbA(p,\Phi;q_0).
\end{equation}
This contradicts maximality of $(p,\Phi)$ and thus implies the desired continuity of $(p',\Phi')$.  



To begin proving \eqref{eq:derivative-continuous}, recall from Lemma~\ref{lem:p-AC} that $p'$ is uniformly bounded away from $q_0$, hence on $(q_*-\eps,q_*+\eps)$. Moreover $\Phi'$ is uniformly bounded by definition. It follows that for all $s\in\sS$ and $q\in (q_*-\eps,q_*+\eps)$,
\begin{equation}
\label{eq:almost-constant}
\begin{aligned}
    |p(q)-\tilde p(q)|&\leq O(\eps_1),\\
    |p(q)-p(q_*)|&\leq O(\eps),
    \\
    |\Phi_s(q)-\tilde\Phi_{s}(q)|&\leq O(\eps_1),\\
    |\xi^s(\Phi(q))-\xi^s(\tilde\Phi(q))|&\leq O(\eps_1),
    \\
    |\Phi_s(q)-\Phi_s(q_*)|&\leq O(\eps),
    \\
    |\xi^s(\Phi(q))-\xi^s(\Phi(q_*))|&\leq O(\eps).
\end{aligned}
\end{equation}
These estimates will let us treat the above functions as almost constant while proving \eqref{eq:derivative-continuous}, so we can focus on the more important changes in their derivatives. First for $q\notin [q_*-\eps,q_*+\eps]$, we have $p(q)=\tilde p(q)$ and $\Phi(q)=\tilde\Phi(q)$, so it suffices to analyze the discrepancy within $q\in [q_*-\eps,q_*+\eps]$. Next, the estimates \eqref{eq:almost-constant} together with the fact that $\Phi_s'$ is uniformly bounded below (by Lemma~\ref{lem:Phi-inc}) imply that
\begin{equation}
\label{eq:good-outside-IJ}
    \lt|
    \sqrt{\Phi_s'(q)(p\times \xi^s\circ\Phi)'(q)}
    -
    \sqrt{\tilde \Phi_{s}'(q)(\tilde p\times \xi^s\circ\tilde \Phi)'(q)}
    \rt|
    \leq
    O(\eps_1),
    \quad
    \forall~
    q\in [q_*-\eps,q_*+\eps]\backslash (I\cup J).
\end{equation}
Integrating, we obtain
\begin{equation}
\label{eq:good-outside-IJ-2}
    \int_{q\in [q_*-\eps,q_*+\eps]\backslash (I\cup J)}\lt|
    \sqrt{\Phi_s'(q)(p\times \xi^s\circ\Phi)'(q)}
    -
    \sqrt{\tilde\Phi_{s}'(q)(\tilde p\times \xi^s\circ\tilde \Phi)'(q)}
    \rt|
    ~\de q
    \leq
    O(\eps_1 \eps).
\end{equation}
Next we fix $x\in [0,\eps_1]$ and analyze the joint effect of the pertubation at the pair of points $q_{I,x}$ and $q_{J,x}$. This is given by
\begin{equation}
\label{eq:joint-effect}
\begin{aligned}
    &\sqrt{\tilde\Phi_{s}'(q_{I,x})(\tilde p\times \xi^s\circ\tilde\Phi)'(q_{I,x})}
    -
    \sqrt{\Phi_s'(q_{I,x})(p\times \xi^s\circ\Phi)'(q_{I,x})}
    \\
    \quad\quad
    &+
    \sqrt{\tilde\Phi_{s}'(q_{J,x})(\tilde p\times \xi^s\circ\tilde\Phi)'(q_{J,x})}
    -
    \sqrt{\Phi_s'(q_{J,x})(p\times \xi^s\circ\Phi)'(q_{J,x})}
    .
\end{aligned}
\end{equation}
Recalling again \eqref{eq:almost-constant}, we have 
\begin{equation}
\label{eq:recalling-again-blah}
\begin{aligned}
    (\tilde p\times \xi^s\circ\tilde\Phi)'(q_{I,x})
    &=
    \tilde p (q_{I,x}) \sum_{s'\in \sS}\partial_{x_{s'}}\xi^s(\tilde \Phi(q_{I,x}))\cdot \tilde\Phi_{s'}'(q_{I,x})
    +
    \tilde p'(q_{I,x})\cdot \xi^s(\tilde\Phi(q_{I,x}))
    \\
    &=
    p (q_*) \sum_{s'\in \sS}\partial_{x_{s'}}\xi^s(\Phi(q_*))\cdot \tilde\Phi_{s'}'(q_{I,x})
    +
    p'(q_{I,x})\cdot \xi^s(\Phi(q_*))
    \pm O(\eps).
\end{aligned}
\end{equation}
Similarly to \eqref{eq:good-outside-IJ}, we now control the first two terms of \eqref{eq:joint-effect}:
\begin{equation}
\label{eq:approx-Phi-diff}
\begin{aligned}
    &
    \sqrt{\tilde\Phi_{s}'(q_{I,x})(\tilde p\times \xi^s\circ\tilde\Phi)'(q_{I,x})}
    -
    \sqrt{\Phi_s'(q_{I,x})(p\times \xi^s\circ\Phi)'(q_{I,x})}
    +
    O(\eps)
    \\
    &\stackrel{\eqref{eq:recalling-again-blah}}{\geq}
    \sqrt{\tilde\Phi_{s}'(q_{I,x})\cdot \lt(p (q_*) \sum_{s'\in \sS}\partial_{x_{s'}}\xi^s(\Phi(q_*))\cdot \tilde\Phi_{s'}'(q_{I,x})
    +
    \tilde p'(q_{I,x})\cdot \xi^s(\Phi(q_*))\rt)}
    \\
    &\quad -
    \sqrt{\Phi_{s}'(q_{I,x})\cdot \lt(p(q_*) \sum_{s'\in \sS}\partial_{x_{s'}}\xi^s(\Phi(q_*))\cdot \Phi'_{s'}(q_{I,x})
    +
    p'(q_{I,x})\cdot \xi^s(\Phi(q_*))\rt)}
\end{aligned}
\end{equation}
and analogously for $J$ instead of $I$.

It remains to lower-bound the right hand side of \eqref{eq:approx-Phi-diff}. We break into cases depending on whether $\Phi'$ is continuous (if so, then $p'$ must be discontinuous). In both cases, the idea is to argue that the concavity of the square root function yields an increase in the value of $\bbA$.

\paragraph{Case $1$: $\Phi'$ is continuous at $q_*$}


In this case $p'$ is discontinuous, and \eqref{eq:eta-discontinuous-f} applies with $f=p$. We estimate the right-hand side of \eqref{eq:approx-Phi-diff}: as $|\Phi_s'(q)-\tilde \Phi_{s}'(q')|\leq o_{\eps\to 0}(1)$ uniformly in $q,q'\in (q_*-\eps,q_*+\eps)$ by definition, 
\begin{align}
    \nonumber
    &
    \sqrt{\tilde\Phi_{s}'(q_{I,x})\cdot \lt(p (q_*) \sum_{s'\in \sS}\partial_{x_{s'}}\xi^s(\Phi(q_*))\cdot \tilde\Phi_{s'}'(q_{I,x})
    +
    \tilde p'(q_{I,x})\cdot \xi^s(\Phi(q_*))\rt)}
    \\
    \nonumber
    &\quad -
    \sqrt{\Phi_{s}'(q_{I,x})\cdot \lt(p(q_*) \sum_{s'\in \sS}\partial_{x_{s'}}\xi^s(\Phi(q_*))\cdot \Phi'_{s'}(q_{I,x})
    +
    p'(q_{I,x})\cdot \xi^s(\Phi(q_*))\rt)}
    \\
    \nonumber
    &=
    \sqrt{\Phi_{s}'(q_*)\cdot \lt(p (q_*) \sum_{s'\in \sS}\partial_{x_{s'}}\xi^s(\Phi(q_*))\cdot \Phi'_{s'}(q_*)
    +
    \tilde p'(q_{I,x})\cdot \xi^s(\Phi(q_*))\rt)}
    \\
    \label{eq:I-big-long-equation}
    &\quad -
    \sqrt{\Phi_{s}'(q_*)\cdot \lt(p(q_*) \sum_{s'\in \sS}\partial_{x_{s'}}\xi^s(\Phi(q_*))\cdot \Phi'_{s'}(q_*)
    +
    p'(q_{I,x})\cdot \xi^s(\Phi(q_*))\rt)} \pm o_{\eps\to 0}(1)
    .
\end{align}
We analyze the last term, combined with the analogous expression for $J$, using the strict concavity in Lemma~\ref{lem:sqrt-xy-concave} of $x\mapsto \sqrt{x}$ together with \eqref{eq:eta-discontinuous-f} applied to $p$. We find that
\begin{equation}
\label{eq:big-gain-case-1}
\begin{aligned}
    &\sqrt{\Phi_{s}'(q_*)\cdot \lt(p (q_*) \sum_{s'\in \sS}\partial_{x_{s'}}\xi^s(\Phi(q_*))\cdot \Phi'_{s'}(q_*)
    +
    \tilde p'(q_{I,x})\cdot \xi^s(\Phi(q_*))\rt)}
    \\
    &\quad -
    \sqrt{\Phi_{s}'(q_*)\cdot \lt(p(q_*) \sum_{s'\in \sS}\partial_{x_{s'}}\xi^s(\Phi(q_*))\cdot \Phi'_{s'}(q_*)
    +
    p'(q_{I,x})\cdot \xi^s(\Phi(q_*))\rt)}
    \\
    &\quad
    +
    \sqrt{\Phi_{s}'(q_*)\cdot \lt(p (q_*) \sum_{s'\in \sS}\partial_{x_{s'}}\xi^s(\Phi(q_*))\cdot \Phi'_{s'}(q_*)
    +
    \tilde p'(q_{J,x})\cdot \xi^s(\Phi(q_*))\rt)}
    \\
    &\quad -
    \sqrt{\Phi_{s}'(q_*)\cdot \lt(p(q_*) \sum_{s'\in \sS}\partial_{x_{s'}}\xi^s(\Phi(q_*))\cdot \Phi'_{s'}(q_*)
    +
    p'(q_{J,x})\cdot \xi^s(\Phi(q_*))\rt)}
    \\
    &\geq
    c(\eta).
\end{aligned}
\end{equation}
Indeed, all quantities except $p'(\cdot)$ and $\tilde p'(\cdot)$ are the same in the four expressions and are bounded away from $0$ and infinity. Furthermore all other expressions differ by $O(\eps_1)$ thanks to \eqref{eq:almost-constant}, which is small compared to the discrepancy $\eta$ between the values of $p'$ and $\wt p$'. Hence for $\eta$ fixed and $\eps$ small enough, they are bounded away from the equality cases of Lemma~\ref{lem:sqrt-xy-concave}.




Combining \eqref{eq:approx-Phi-diff}, \eqref{eq:I-big-long-equation}, and \eqref{eq:big-gain-case-1} implies that for each $x\in [0,\eps_1]$ and small enough $\eps$,
\begin{align*}
    &\sqrt{\tilde\Phi_{s}'(q_{I,x})(\tilde p\times \xi^s\circ\tilde\Phi)'(q_{I,x})}
    -
    \sqrt{\Phi_s'(q_{I,x})(p\times \xi^s\circ\Phi)'(q_{I,x})}
    \\
    \quad\quad
    &+
    \sqrt{\tilde\Phi_{s}'(q_{J,x})(\tilde p\times \xi^s\circ\tilde\Phi)'(q_{J,x})}
    -
    \sqrt{\Phi_s'(q_{J,x})(p\times \xi^s\circ\Phi)'(q_{J,x})}
    \\
    &\geq c(\eta)-o_{\eps\to 0}(1)
    \\
    &\geq 
    c(\eta)/2.
\end{align*}
Integrating over $x\in [0,\eps_1]$ and combining with \eqref{eq:good-outside-IJ-2}, we conclude that \eqref{eq:derivative-continuous} holds in Case $1$.


\paragraph{Case $2$: $\Phi'$ is discontinuous at $q_*$.} 
(Note that $p'$ might also be discontinuous.) 

Define for each $s\in\sS$ the function
\[
    F_s(A_1,\dots,A_r,B)
    =
    \sqrt{A_s\cdot \lt(p(q_*) \sum_{s'\in \sS}\partial_{x_{s'}}\xi^s(\Phi(q_*)) A_{s'}
    +
    B\xi^s(\Phi(q_*))\rt)}.
\]
Lemma~\ref{lem:sqrt-xy-concave} implies that each function $F_s$ is concave on $\bbR_{\geq 0}^{r+1}$, since both $A_s$ and $p(q_*) \sum_{s'\in \sS}\partial_{x_{s'}}\xi^s(\Phi(q_*)) A_{s'}+B\xi^s(\Phi(q_*))$ are linear functions of $(A_1,\dots,A_r,B)$. In particular, for each $(s,x)\in \sS\times [0,\eps_1]$ the function
\[
    f_{s,x}(t)
    \equiv
    F_s\lt(\frac{(1-t)\Phi_{1}'(q_{I,x})+t\Phi_{1}'(q_{J,x})}{2},\dots,
    \frac{(1-t)\Phi_{r}'(q_{I,x})+t\Phi_{r}'(q_{J,x})}{2},
    \frac{(1-t)p'(q_{I,x})+tp'(q_{J,x})}{2}\rt)
\]
is concave for $t\in [0,1]$. Recalling the definitions of $\tilde p$ and $\tilde \Phi$, we expand the inequality $2f_{s,x}(1/2)\geq f_{s,x}(0)+f_{s,x}(1)$ to obtain
\begin{equation}
\label{eq:concave-in-each-s}
\begin{aligned}
    &
    \sqrt{\tilde\Phi_{s}'(q_{I,x})\cdot \lt(p (q_*) \sum_{s'\in \sS}\partial_{x_{s'}}\xi^s(\Phi(q_*))\cdot \tilde\Phi_{s'}'(q_{I,x})
    +
    \tilde p'(q_{I,x})\cdot \xi^s(\Phi(q_*))\rt)}
    \\
    &\quad -
    \sqrt{\Phi_{s}'(q_{I,x})\cdot \lt(p(q_*) \sum_{s'\in \sS}\partial_{x_{s'}}\xi^s(\Phi(q_*))\cdot \Phi'_{s'}(q_{I,x})
    +
    p'(q_{I,x})\cdot \xi^s(\Phi(q_*))\rt)}
    \\
    &\quad +
    \sqrt{\tilde\Phi_{s}'(q_{J,x})\cdot \lt(p (q_*) \sum_{s'\in \sS}\partial_{x_{s'}}\xi^s(\Phi(q_*))\cdot \tilde\Phi_{s'}'(q_{J,x})
    +
    \tilde p'(q_{J,x})\cdot \xi^s(\Phi(q_*))\rt)}
    \\
    &\quad -
    \sqrt{\Phi_{s}'(q_{J,x})\cdot \lt(p(q_*) \sum_{s'\in \sS}\partial_{x_{s'}}\xi^s(\Phi(q_*))\cdot \Phi'_{s'}(q_{J,x})
    +
    p'(q_{J,x})\cdot \xi^s(\Phi(q_*))\rt)}
    \\
    &\geq
    0.
\end{aligned}
\end{equation}
In light of \eqref{eq:approx-Phi-diff}, this means that perturbing $(p,\Phi)\to(\tilde p,\tilde \Phi)$ can only hurt the contribution from a given $s\in\sS$ by $O(\eps)$. To complete the proof we will show that the contribution from some $s\in\sS$ is positive and of a larger order. Which of these must occur will depend on the ratio $\frac{p'(q_{I,x})}{p'(q_{J,x})}$. 


We will get this contribution from either $s_{\max}$ or $s_{\min}$, defined now. For each $x\in[0,\eps_1]$, let
\begin{align*}
    s_{\max}(x)
    &=
    \argmax_{s\in\sS}
    \frac{\Phi_s'(q_{I,x})}{\Phi_s'(q_{J,x})}
    ,
    \\
    s_{\min}(x)
    &=
    \argmin_{s\in\sS}
    \frac{\Phi_s'(q_{I,x})}{\Phi_s'(q_{J,x})}
    .
\end{align*}
(Both are defined up to almost everywhere equivalence if ties are broken lexicographically.) Recall the functions $\Phi_s'(x)$ are uniformly bounded above and below. It follows from \eqref{eq:eta-discontinuous-f} that 
\begin{equation}
\label{eq:Phi-s-minmax}
    \frac{\Phi_{s_{\min}}'(q_{I,x})}{\Phi_{s_{\min}}'(q_{J,x})}
    \leq
    1-\eta'
    \leq
    1+\eta'
    \leq
    \frac{\Phi_{s_{\max}}'(q_{I,x})}{\Phi_{s_{\max}}'(q_{J,x})}
\end{equation}
for some $\eta'$ depending only on $(\eta,\xi,h)$. (Discontinuity of $\Phi'$ gives one side, and admissibility forces another $s\in\sS$ to change in the opposite direction.)


Without loss of generality, suppose that
\begin{equation}
\label{eq:p'-shrink-wlog}
    \frac{p'(q_{I,x})}{p'(q_{J,x})}\leq 1.
\end{equation}
In this case, the assumption \eqref{eq:p'-shrink-wlog} implies
\[
    \frac{
    p(q_*) \sum_{s'\in \sS}\partial_{x_{s'}}\xi^{s_{\max}}(\Phi(q_*))\cdot \Phi'_{s'}(q_{I,x})
    +
    p'(q_{I,x})\cdot \xi^{s_{\max}}(\Phi(q_*))
    }
    {
    p(q_*) \sum_{s'\in \sS}\partial_{x_{s'}}\xi^{s_{\max}}(\Phi(q_*))\cdot \Phi'_{s'}(q_{J,x})
    +
    p'(q_{J,x})\cdot \xi^{s_{\max}}(\Phi(q_*))
    }
    \leq
    \frac{\Phi_{s_{\max}}'(q_{I,x})}
    {\Phi_{s_{\max}}'(q_{J,x})}-\eta_1
\]
for a constant $\eta_1>0$ depending only on $(\eta,q_*,\xi,h)$. Since all quantities are bounded away from $0$ and infinity, applying a simple compactness argument to the equality case in Lemma~\ref{lem:sqrt-xy-concave} implies
\begin{equation}
\label{eq:f-concave-with-eta1-gain}
    2f_{s_{\max},x}(1/2)\geq f_{s_{\max},x}(0)+f_{s_{\max},x}(1)+c(\eta_1).
\end{equation}
Similarly if \eqref{eq:p'-shrink-wlog} does not hold, then we find \eqref{eq:f-concave-with-eta1-gain} with $s_{\min}$ in place of $s_{\max}$.


Combining the above with $\eps\ll \eta$, we find that for each $x\in[0,\eps_1]$,
\begin{align*}
    \sum_{s\in\sS} 2f_{s,x}(1/2)
    &\geq
    \sum_{s\in\sS} 
    \Big(f_{s,x}(0)+f_{s,x}(1)\Big)+c(\eta_1)/2.
\end{align*}
Integrating over $x$ and recalling \eqref{eq:good-outside-IJ-2} and \eqref{eq:approx-Phi-diff}, we conclude that \eqref{eq:derivative-continuous} also holds in Case $2$. This completes the proof.
\end{proof}

\begin{proof}[Proof of Proposition~\ref{prop:basic-regularity}]
    Follows from Lemmas~\ref{lem:p-AC}, \ref{lem:Phi-inc}, and \ref{lem:derivatives-continuous-apriori}.
    The upper bound on $\Phi'$ comes from admissibility \eqref{eq:admissible}, which implies that $\Phi'_s \le \lambda_s^{-1}$.
\end{proof}



\subsection{Type $\II$ Solutions}
\label{subsec:type-II-Lipschitz}

Here we show that the type $\II$ equation implicitly takes the form of a second order ordinary differential equation in which $\Phi''(q)$ is Lipschitz in $(\Phi(q),\Phi'(q))$. It follows that a unique type $\II$ solution exists given any first-order initial condition $(\Phi(q_1),\Phi'(q_1))$, and that the type $\II$ ODE is satisfied at \emph{all} points in $(q_1,1)$. We will often enforce the admissibility conditions
\begin{align}
\label{eq:derivative-admissible}
    \langle \vlam, \vec\Phi'(q)\rangle &=1,
    \\
\label{eq:second-derivative-admissible}
    \langle \vlam, \vec\Phi''(q)\rangle &=0.
\end{align}
In particular, we denote by $A_{\geq 0}$ the set of vectors $v\in \bbR_{\geq 0}^{\sS}$ satisfying $\langle \vlam, v \rangle=1$. The following important but rather lengthy Lemma~\ref{lem:type-II-Lipschitz} ensures that type $\II$ solutions are described by a Lipschitz ODE. In it, the value $q$ is actually irrelevant and just serves as a placeholder. Importantly there is no issue when $\Phi_s(q)$ or $\Phi_s'(q)$ is near zero, thanks to non-degeneracy.




\lemtypeIILipschitz*



\begin{proof}
Write $\Psi(q)$ for $\Psi_s(q)$, which is independent of $s\in\sS$ by assumption. We assume throughout that $\Phi(q)$ lies in a bounded set in writing $O(\cdot)$ and $\Omega(\cdot)$ expressions. Note that $\vec\Phi''(q)$ exists as an $L^1$ function for $q\in (q_1,1]$ since $\vec\Phi'$ is absolutely continuous. We write
\begin{equation}
\label{eq:expand-Psi-type-II}
\begin{aligned}
    2\Psi(q)
    &=
    \frac{2}{\Phi_s'(q)}
    \deriv{q}{\sqrt{\frac{\Phi'_s(q)}{(\xi^s\circ\Phi)'(q)}}}
    \\
    &=
    \sqrt{\frac{(\xi^s\circ\Phi)'(q)}{\Phi_s'(q)^3}}
    \deriv{q}{\frac{\Phi'_s(q)}{(\xi^s\circ\Phi)'(q)}}
    \\
    &=
    \sqrt{\frac{(\xi^s\circ\Phi)'(q)}{\Phi_s'(q)^3}}
    \cdot
    \frac{\Phi_s''(q)(\xi^s\circ \Phi)'(q) - \Phi_s'(q)(\xi^s\circ\Phi)''(q)}{(\xi^s\circ\Phi)'(q)^2}
    \\
    &=
    \frac{1}{\sqrt{\Phi_s'(q)^3 (\xi^s\circ\Phi)'(q)^3}}
    \cdot
    \lt(
    \Phi_s''(q)(\xi^s\circ \Phi)'(q) - \Phi_s'(q)(\xi^s\circ\Phi)''(q)
    \rt)\,.
\end{aligned}
\end{equation}
%
Moreover we have
\begin{align*}
    &\Phi_s''(q)(\xi^s\circ \Phi)'(q) - \Phi_s'(q)(\xi^s\circ\Phi)''(q)
    \\
    &=
    \Phi_s''(q)
    \sum_{s'\in\sS}\partial_{x_{s'}}\xi^s(\Phi(q)) \cdot \Phi'_{s'}(q) 
    \\
    &\quad
    - 
    \Phi_s'(q)
    \lt(
    \sum_{s'\in\sS}\partial_{x_{s'}}\xi^s(\Phi(q)) \cdot \Phi''_{s'}(q) 
    +
    \sum_{s',s''\in\sS} 
    \partial_{x_{s'}}\partial_{x_{s''}}
    \xi^s(\Phi(q)) \cdot \Phi'_{s'}(q) 
    \rt)
\end{align*}
Let 
\[
B_s(q)=\sum_{s'\in\sS}\partial_{x_{s'}}\xi^s(\Phi(q)) \cdot \Phi'_{s'}(q).
\]
Note that by non-degeneracy each $\partial_{x_{s'}}\xi^s(\Phi(q))$ is bounded away from $0$ and $\infty$ for all $\Phi(q)\in [0,1]^{\sS}$. Meanwhile $\sum_{s\in\sS}\lambda_s\Phi'_s(q)=1$. Thus for $\Phi'(q)$ obeying \eqref{eq:derivative-admissible}, each $B_s(q)$ is uniformly bounded away from $0$ and $\infty$.


Next let $M(q)\in\mathbb R^{\sS\times\sS}$ be a square matrix with entries
\[
    M(q)_{s,s'}
    =
    \frac{\Phi_s'(q) \cdot \partial_{x_{s'}}\xi^s(\Phi(q))}{B_s(q)}
    \,
\]
and let $I$ denote the identity $\sS\times\sS$ matrix. Then the above equations for all $s\in\sS$ can be expressed more succinctly as
\begin{equation}
\label{eq:type-II-Lipschitz}
    (M-I)\Phi''(q) 
    =
    -w_1(\Phi(q),\Phi'(q))-\Psi(q)\cdot w_2(\Phi(q),\Phi'(q))
\end{equation}
for Lipschitz functions $w_1,w_2:[0,1]^{2r}\to\mathbb R_{>0}^{r}$ given explicitly by
\begin{equation}
\label{eq:w1w2}
\begin{aligned}
    (w_1)_s
    &=
    \frac{
    \Phi_s'(q)\sum_{s',s''\in\sS} 
    \partial_{x_{s'}} \partial_{x_{s''}}
    \xi^s(\Phi(q)) \cdot \Phi'_{s'}(q)  
    }
    {B_s(q)}
    ;
    \\
    (w_2)_s
    &=
    \frac{2\sqrt{\Phi_s'(q)^3 (\xi^s\circ\Phi)'(q)^3}}{B_s(q)}.
\end{aligned}
\end{equation}
Since $B_s$ is bounded below, both $w_1$ and $w_2$ have uniformly bounded entries. Moreover $B$ and $w_1,w_2$ are uniformly Lipschitz in $(\Phi(q),\Phi'(q))$. Note also that $w_2$ is entry-wise non-negative.


As a first observation, observe that
\[
    (M-I)\Phi'(q)=0.
\]
Because $\Phi'(q)\succeq 0$ and $M$ has positive entries, this means $\Phi'(q)$ is the unique right Perron-Frobenius eigenvector of $M$, and thus $\rank(M-I)=r-1$. It follows that for given $(\Phi(q),\Phi'(q))$, a unique solution $(\Phi''(q),\Psi(q))$ to \eqref{eq:type-II-Lipschitz} exists so long as 
\begin{equation}
\label{eq:w2-notin-range-MI}
w_2\notin \range(M-I).
\end{equation}
In fact \eqref{eq:w2-notin-range-MI} is always true. To see this, note that $M$ has a left Perron-Frobenius eigenvector $v\in\mathbb R_{>0}^{r}$ with $v(M-I)=0$. Then if $w_2=(M-I)w$ for $w\in\bbR^{\sS}$, we find $\langle v,w_2\rangle=0$.
This is a contradiction: $\langle v,w_2\rangle>0$ since all entries are strictly positive in both vectors. We denote by $\Lambda(q)\in\bbR^\sS$ the value of $\Phi''(q)$ in the aforementioned unique solution.




Our primary aim is now to show that $\Lambda(q)$ is a Lipschitz function of $(\Phi(q),\Phi'(q))\in\bbR^{\sS}\times A_{\geq 0}$. We would like to apply Perron-Frobenius arguments to $M$, but the fact that $M_{s,s'}\asymp \Phi_s'(q)$ may be very small poses an issue. To rectify this, we define $\wt M(q)$ with entries
\begin{equation}
\label{eq:wtM}
    \wt M(q)_{s,s'}
    =
    \frac{\Phi'_{s'}(q)\partial_{x_{s'}}\xi^s(\Phi(q))}{B_s(q)}
    \,
    .
\end{equation}
Then defining the diagonal $\sS\times\sS$ matrix $D(\Phi'(q))$ with entries
\[
    D(\Phi'(q))_{s,s}=\Phi_s'(q)
\]
we have
\[
\wt M(q)=D(\Phi'(q))^{-1}\, M \, D(\Phi'(q)).
\]
The key property obeyed by $\wt M$ but not $M$ is that for any $v\in \bbR_{>0}^{\sS}$, the entries of $\wt M v$ are of the same order. Namely, all ratios $\frac{(\wt M v)_s}{(\wt M v)_{s'}}$ are uniformly bounded because the ratios $M_{s,s'}/M_{s'',s'}$ are uniformly bounded.
In particular Lemma~\ref{lem:Phi'=0isOK} and hence Lemma~\ref{lem:PF-closed-range} (see below) apply to $\wt M$.

Note that $\wt M$ has Perron-Frobenius eigenvector $\vone$ and $\wt M$ is Lipschitz in $(\Phi(q),\Phi'(q))$. We set 
\begin{equation}
\label{eq:wtV-defn}
\begin{aligned}
    \wt V(q)&=D(\Phi'(q))^{-1}\Lambda(q),\quad \text{i.e.}~ \wt V(q)_s=\frac{\Lambda_s(q)}{\Phi_s'(q)};
    \\
    V(q)_s&=\wt V(q)_s - \frac{\sum_{s'\in\sS} \wt V(q)_{s'}}{r}.
\end{aligned}
\end{equation}
By construction, $\sum_s V(q)_s=0$. Moreover 
\begin{equation}
\label{eq:V-wtV-behave-same}
(\wt M(q)-I)V(q)=(\wt M(q)-I)\wt V(q)
\end{equation}
since $V(q)-\wt V(q)$ is proportional to $\vone$. 

\paragraph{A priori estimate on $\Lambda(q)$}
We now prove \eqref{eq:Phi-stays-increasing}, which will also serve as a useful intermediate step. Note first that $w_1$ satisfies $|(w_1)_s|= O(\Phi'_s(q))$ (recall that $B_s$ is bounded below), while all entries of $w_2$ are non-negative. Therefore the entries of $w_1(q)+\Psi(q) w_2(q)$
are bounded either above or below by $O(\Phi'_s(q))$. Furthermore by definition, 
\begin{align*}
    -w_1(q)-\Psi(q) w_2(q)
    &=
    (M(q)-I)\Lambda(q)
    \\
    &=
    D(\Phi'(q))(\wt M(q)-I)\wt V
    \\
    &\stackrel{\eqref{eq:V-wtV-behave-same}}{=}
    D(\Phi'(q))(\wt M(q)-I)V.
\end{align*}
We conclude that
\[
    \min\lt(\|((\wt M(q)-I) V)_+\|_1,\|((\wt M(q)-I) V)_-\|_1\rt) \leq O(1).
\]
Lemma~\ref{lem:PF-closed-range} below now implies that 
\begin{equation}
\label{eq:V-bounded}
\|V(q)\|_1\leq O(1).
\end{equation}



Note that $\langle \wt V,\lambda\odot\Phi'(q)\rangle=0$ by \eqref{eq:second-derivative-admissible} and \eqref{eq:wtV-defn}. The second part of the latter also implies $V(q)-\wt V(q)$ is proportional to $\vone$, and so
\begin{equation}
\label{eq:V-wtV-1}
\begin{aligned}
    \lt|(V(q)-\wt V(q))_s\rt|
    &=
    \lt|
    \langle 
    V(q)-\wt V(q)
    ,
    \lambda\odot\Phi'(q)
    \rangle
    \rt|
    \\
    &=
    \lt|
    \langle 
    V(q)
    ,
    \lambda\odot\Phi'(q)
    \rangle
    \rt|
    \\
    &\stackrel{\eqref{eq:V-bounded}}{\leq} O(1)
\end{aligned}
\end{equation}
Using again \eqref{eq:V-bounded} and \eqref{eq:wtV-defn} we find that $\|\wt V(q)\|_1\leq O(1)$ as well. Finally since $\Lambda(q)=\wt V(q)\odot \Phi'(q)$, we get \eqref{eq:Phi-stays-increasing} as desired. 


\paragraph{Controlling $\Psi$}
We take a second detour to show that $\Psi(q)$ is bounded and Lipschitz.
Using that $\|w_1\|_1\leq O(1)$ and $\|w_2\|_1\geq \Omega(1)$ in the first step below, we find
\begin{align*}
    \Omega(|\Psi(q)|)-O(1)
    &\leq 
    \|w_1(q)+\Psi(q) w_2(q)\|_1
    \\
    &=
    \|(M(q)-I)\Lambda(q)\|_1
    \\
    &\stackrel{\eqref{eq:type-II-Lipschitz}}{\leq}
    O(1).
\end{align*}
The just-proved estimate \eqref{eq:Phi-stays-increasing} implies the weaker bound $\|\Lambda(q)\|_1\leq O(1)$, which was used in the last step.

We conclude that $\Psi(q)$ is uniformly bounded:
\begin{equation}
\label{eq:Psi-bounded}
    |\Psi(q)|\leq O(1).
\end{equation}
Next we show that $\Psi(q)$ is Lipschitz in $(\Phi(q),\Phi'(q))$. We begin by writing
\begin{align*}
    (M(q)-I)\Lambda(q)
    -
    (M(q')-I)\Lambda(q')
    &=
    w_1(q')-w_1(q)
    +
    \Psi(q')w_2(q')
    -
    \Psi(q)w_2(q)
    \\
    &=
    w_1(q')-w_1(q)
    +
    \Psi(q')\big(w_2(q')-w_2(q)\big)
    +
    \big(\Psi(q')-\Psi(q)\big)w_2(q)
    \\
    &=
    O\big(\|\Phi(q)-\Phi(q')\|+\|\Phi'(q)-\Phi'(q')\|\big)
    +
    \big(\Psi(q')-\Psi(q)\big)w_2(q)
    .
\end{align*}
(Note that the latter $O(\cdot)$ notation hides a vector in $\bbR^r$.) We will rely on the fact that $w_2(q)$ is entrywise positive and $\|w_2(q)\|\geq \Omega(1)$. To analyze the left-hand side above, we write
\begin{align*}
    (M(q)-I)\Lambda(q)
    -
    (M(q')-I)\Lambda(q')
    &=
    (M(q)-M(q'))\Lambda(q)
    +
    (M(q')-I)\big(\Lambda(q)-\Lambda(q')\big)
    \\
    &\leq
    O\big(\|\Phi(q)-\Phi(q')\|+\|\Phi'(q)-\Phi'(q')\|\big)
    +
    (M(q')-I)\big(\Lambda(q)-\Lambda(q')\big)
    .
\end{align*}
The latter step holds since $M(q)$ is Lipschitz in $(\Phi(q),\Phi(q'))$ and $\|\Lambda(q)\|_1\leq O(1)$ from \eqref{eq:Phi-stays-increasing}. Now, let $v$ be the left Perron-Frobenius eigenvector of $M(q')$, so $v(M(q')-I)=0$, normalized so that $v\succeq 0$ and $\|v\|_1=1.$ Combining the previous displays implies that 
\[
    (\Psi(q')-\Psi(q))\cdot \langle v, w_2(q)\rangle 
    =
    O\big(\|\Phi(q)-\Phi(q')\|+\|\Phi'(q)-\Phi'(q')\|\big).
\]
Finally we show that $\langle v,w_2(q)\rangle$ is bounded away from $0$. Indeed both vectors are entrywise positive, and $\|w_2(q)\|_1\geq \Omega(1)$ while $\min_s v_s\geq \Omega(1)$. The latter statement holds for similar reasons to the right eigenvector properties of $\wt M$ explained above: for \emph{any} $v\in\bbR_{>0}^{\sS}$, the ratios $\frac{(vM)_{s}}{(vM)_{s'}}$ are uniformly bounded, and this ratio is simply $v_s/v_{s'}$ when $v$ is the left Perron-Frobenius eigenvector. We conclude that
\begin{equation}
\label{eq:Psi-Lip}
    |\Psi(q)-\Psi(q')|\leq O\big(\|\Phi(q)-\Phi(q')\|+\|\Phi'(q)-\Phi'(q')\|\big)
\end{equation}
which ends this second detour.


\paragraph{Finishing the Proof}
Having established \eqref{eq:Phi-stays-increasing} and \eqref{eq:Psi-bounded}, we return to showing that $\Lambda(q)$ is Lipschitz in $(\Phi(q),\Phi'(q))$. Fix a different pair 
\[
(\Phi(q'),\Phi'(q'))\neq (\Phi(q),\Phi'(q)).
\]
Accordingly define $w_1(q'),w_2(q'),M(q'),V(q')$ and so on using $(\Phi(q'),\Phi'(q'))$. (Since we don't require admissibility but only its differential version \eqref{eq:derivative-admissible}, there is no loss of generality here; $q'$ like $q$ is just a place-holder variable so e.g. $\Phi(q)=\Phi(q')$ is possible.)


Then Lemma~\ref{lem:PF-closed-range} implies:
\begin{equation}
\label{eq:type-II-reverse-Lipschitz}
    \|(\wt M(q)-I)V(q)-(\wt M(q)-I)V(q')\|_1 
    \geq
    \Omega\big(\|V(q)-V(q')\|_1\big).
\end{equation}
Using the reverse triangle inequality in the first step, we find the lower bound
%
\begin{align*}
    \|(\wt M(q)-I) V(q)
    -
    (\wt M(q')-I) V(q')\|_1
    &\geq
    \|(\wt M(q)-I) V(q)
    -
    (\wt M(q)-I) V(q')\|_1
    \\
    &\quad\quad
    -
    \|(\wt M(q)-I) V(q')
    -
    (\wt M(q')-I) V(q')\|_1
    \\
    &\stackrel{\eqref{eq:type-II-reverse-Lipschitz}}{\geq}
    \Omega\big(\| V(q)- V(q')\|_1\big)
    -
    O\big(\|\wt M(q)- \wt M(q')\|_1\big)
    \\
    &\geq
    \Omega\big(\| V(q)- V(q')\|_1\big)
    -
    O\big(\|\Phi(q)-\Phi(q')\|_1 + \|\Phi'(q)-\Phi'(q')\|_1\big).
\end{align*}
By \eqref{eq:type-II-Lipschitz}, \eqref{eq:w1w2}, \eqref{eq:Psi-bounded} \eqref{eq:Psi-Lip}, and the simple estimate $\max\big(|w_1(q)_s|,|w_2(q)_s|\big)\leq O(\Phi'_s(q))$, the left-hand side above is upper bounded by
\begin{align*}
   &\|(\wt M(q)-I) V(q)
    -
    (\wt M(q')-I) V(q')\|_1
    \\
    &=
    \|(\wt M(q)-I) \wt V(q)
    -
    (\wt M(q')-I) \wt V(q')\|_1  
    \\
    &=\Big\|
    D(\Phi'(q))^{-1}
    \Big(
    (M(q)-I) \Lambda(q)
    \Big)
    -
    D(\Phi'(q'))^{-1}
    \Big(
    (M(q')-I) \Lambda(q')\Big)
    \Big\|_1
    \\
    &=
    \Big\|
    D(\Phi'(q))^{-1}
    \Big(
    w_1(q)+\Psi(q)w_2(q)
    \Big)
    -
    D(\Phi'(q'))^{-1}
    \Big(
    w_1(q')+\Psi(q')w_2(q')
    \Big)
    \Big\|_1
    \\
    &\leq
    O\big(\|\Phi(q)-\Phi(q')\|_1 + \|\Phi'(q)-\Phi'(q')\|_1\big).
\end{align*}
Rearranging the previous two displays implies that
\begin{equation}
\label{eq:V-initial-bound}
    \|V(q)-V(q')\|_1
    \leq
    O\big(\|\Phi(q)-\Phi(q')\|_1 + \|\Phi'(q)-\Phi'(q')\|_1\big).
\end{equation}
It remains to unwind the transformations to conclude the same for $\Lambda$. Mimicking \eqref{eq:V-wtV-1} in the first step, 
\begin{align*}
    \lt|\big(V_s(q)-V_s(q')\big)-\big(\wt V_s(q)-\wt V_s(q')\big)\rt|
    &=
    \lt|\sum_s \lambda_s \big(\Phi_s'(q) V(q)-\Phi_s'(q')V(q')\big)\rt| 
    \\
    &\leq
    O\big(\|\Phi'(q)\|\cdot \|V(q)-V(q')\| \big)
    +
    O\big(\|\Phi'(q)-\Phi'(q')\|\cdot \|V(q')\| \big)
    \\
    &\stackrel{\eqref{eq:V-initial-bound},\eqref{eq:V-bounded}}{\leq}
    O\big(\|\Phi(q)-\Phi(q')\|_1 + \|\Phi'(q)-\Phi'(q')\|_1\big)
    \\
    &\quad\quad
    +
    O\big(\|\Phi'(q)-\Phi'(q')\big)
    .
\end{align*}
Combining the previous two displays, we conclude that 
\begin{align*}
    \|\wt V(q)-\wt V(q')\|_1
    &\leq
    \|V(q)-V(q')\|_1
    +
    \|(V(q)-V(q'))-(\wt V(q)-\wt V(q'))\|
    \\
    &\leq 
    O\big(\|\Phi(q)-\Phi(q')\|_1 + \|\Phi'(q)-\Phi'(q')\|_1\big).
\end{align*}
Finally since $\Lambda(q)=\wt V(q) \odot \Phi'(q)$ and $\|\wt V(q')\|_1, \|\Phi'(q)\|_1\leq O(1)$, we obtain the desired:
\begin{align*}
    \|\Lambda(q)-\Lambda(q')\|_1
    &\leq
    O\big(\|\wt V(q)-\wt V(q')\|_1 \cdot \|\Phi'(q)\|_1\big)
    +
    O\big(\|\wt V(q')\|_1 \cdot \|\Phi'(q)-\Phi'(q')\|_1\big)
    \\
    &\leq
    O\big(\|\wt V(q)-\wt V(q')\|_1 + \|\Phi'(q)-\Phi'(q')\|_1\big)
    \\
    &\leq
    O\big(\|\Phi(q)-\Phi(q')\|_1 + \|\Phi'(q)-\Phi'(q')\|_1\big).
\end{align*}
This concludes the proof.
\end{proof}




\begin{lemma}
\label{lem:PF-closed-range}
Let $\cM\subseteq \bbR_{\geq 0}^{\sS\times\sS}$ be a compact set of entry-wise non-negative matrices with unique Perron-Frobenius eigenvector $\vone$ and associated eigenvalue $1$.

Then for all  $v\in\bbR^{\sS}$ with $\sum_{s\in\sS} v_s=0$, we have
\begin{align*}
    \|((M-I)v)_+\|_1 &\geq \Omega_{\cM,r}(\|v\|_1),
    \\
    \|((M-I)v)_-\|_1 &\geq \Omega_{\cM,r}(\|v\|_1).
\end{align*}
\end{lemma}


\begin{proof}
The two statements are equivalent under negation so we assume the first is false and derive a contradiction. If it is false, by taking a convergent sequence of approximate counterexamples $(M^i,v^i)\to (\hM,\hv)$ with $M^i\in\cM$ and $\|v^i\|_1=1$, we have:
\begin{enumerate}
    \item $\hM\in \cM$.
    \item $\hM$ has Perron-Frobenius eigenvector $\vone$ and eigenvalue $1$. 
    \item $\sum_{s\in\sS}\hv_s= 0$.
    \item $\|\hv\|_1=1$.
    \item $\hM\hv\preceq \hv$ (since $((\hM-I)\hv)_+=0$).
\end{enumerate}
Since $\hM$ has simple Perron-Frobenius eigenvalue $1$, for $\hM\hv\preceq \hv$ to hold we must actually have $\hM \hv=\hv$. Therefore $\hv=\vone/r$ is a multiple of the right Perron-Frobenius eigenvector, contradicting $\sum_{s\in\sS}\hv_s= 0$.
\end{proof}


\begin{lemma}
\label{lem:Phi'=0isOK}
For $C>0$, let $\cM_C\subseteq \bbR_{\geq 0}^{\sS\times\sS}$ consist of all matrices $M$ such that:
\begin{enumerate}
    \item $M_{s,s'}\in [0,C]$ for all $s,s'\in\sS$.
    \item $M_{s,s'}\leq CM_{s'',s'}$ for all $s,s',s''\in\sS$.
    \item $M\vone=\vone$.
    \item $\sum_{s,s'\in\sS}M_{s,s'}\geq 1/C$.
\end{enumerate}
Then $\cM=\cM_C$ satisfies the conditions of Lemma~\ref{lem:PF-closed-range}.
\end{lemma}

\begin{proof}
    The only thing to show is that $\vone$ is the \textbf{unique} right Perron-Frobenius eigenvector associated to the eigenvalue $1$ of any $M\in\cM_C$, even though $M$ may include zero entries. Thus, suppose that $w\in\bbR^{\sS}$ satisfies $Mw=w$; we will show that $w$ has all equal entries. Let $S'\subseteq \sS$ be the non-empty set of $s'$ such that $M_{s,s'}>0$ (which does not depend on $s$ by definition of $\cM_C$). Then letting $M'$ and $w'$ be the $S'\times S'$ and $S'$-dimensional restrictions of $M$ and $w$, we have $M'w'=w'$. Since $M'$ has strictly positive entries, we conclude that $w'$ has all entries proportional. Hence for some $a\geq 0$, we have $w_s=a$ for all $s\in S'$. By definition of $S'$ we obtain $w=Mw=Ma^{\sS}=a^{\sS}$. This concludes the proof.
\end{proof}




