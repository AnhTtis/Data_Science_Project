\section{Approximate Message Passing}
\label{sec:amp}


\mscomment{
This section will be split off. Update things after writing augmented AMP in the appendix.
}

\mscomment{
Augmented state evolution can be used to analyze pairs of outputs by just doing both in the same sequence, so the suggestion of \cite{alaoui2020algorithmic} actually works now. It is worth commenting that for isoperimetric randomness, one can do AMP algorithms along a rooted tree of finite depth but exponential breadth.
}


In this section we prove Theorem~\ref{thm:main-alg} by exhibiting an approximate message passing (AMP) algorithm.
Throughout this section, Assumption~\ref{as:nondegenerate} on non-degeneracy of $\xi$ will be enforced. 
This is without loss of generality since one can always increase the quadratic and cubic coefficients $\gamma_{s,s'}, \gamma_{s,s',s''}$ by small constants and invoke continuity of $\ALG$.


As in \cite{alaoui2022algorithmic,sellke2021optimizing}, our algorithm has two phases. The first phase identifies the root of the ``algorithmic ultrametric tree'' and the second descends it in small orthogonal steps.
The structure of the first phase is similar to the original AMP algorithm of \cite{bolthausen2014iterative} for the SK model at high-temperature, while the latter \emph{incremental} AMP technique was introduced in \cite{mon18}. 




\subsection{Review of Approximate Message Passing}

Here we recall the class of approximate message passing algorithms, specialized to our setting of interest. We initialize AMP with a deterministic vector $\bw^0$  with coordinates
\begin{equation}
\label{eq:AMP-init-concrete}
    w^0_i = C_{s(i)}
\end{equation}
depending only on the species.
Let $f_{t,s}:\bbR^{t+1}\to\bbR$ be a Lipschitz function for each $(t,s)\in \bbZ_{\geq 0}\times \sS$. For $(\bw^0,\bw^1,\dots,\bw^t)\in\bbR^{N\times (t+1)}$, let 
$f_{t}(\bw^0,\bw^1,\dots,\bw^t)\in\bbR^N$ be given by
\[
    f_{t}(\bw^0,\bw^1,\dots,\bw^t)_i
    =
    f_{t,s(i)}(w^1_i,\dots,w^t_i),\quad i\in [N].
\]
We generate subsequent iterates through recursions of the form, where $\ONS_t$ is known as the \emph{Onsager correction term}:
\begin{align}
\label{eq:AMP-body}
    \bw^{t+1}
    &=
    \nabla H_N(\bm^t)
    -
    \ONS_t
    ;
    \\
\nonumber
    \bm^t
    &=
    f_{t}(\bw^0,\bw^1,\dots,\bw^t);
    \\
\label{eq:ONS-body}
    \ONS_t
    &=
    \sum_{t'\leq t}
    d_{t,t'}
    \diamond
    f_{t'-1}(\bw^1,\dots,\bw^{t'-1});
    \\
\label{eq:dts-def}
    d_{t,t',s}
    &=
    \sum_{s'\in\sS}
    \partial_{x_{s'}}
    \xi^s
    \lt(
    \lt(
    \bbE[M^t_{s''} M^{t'-1}_{s''}]\rt)_{s''\in\sS}
    \rt)
    \cdot
    \bbE
    \lt[
    \partial_{X^{t'}_{s'}}f_{t,s'}(X^0_{s'},\dots,X^t_{s'})
    \rt]
    .
\end{align}
Here $W^t_s,M^t_s$ are defined as follows. $W^0_s=C_s$ and the variables
$(\wt W^t_s)_{(t,s)\in \bbZ_{\geq 1}\times \sS}$ form a centered Gaussian process with covariance defined recursively by
\begin{equation}
\label{eq:state-evolution-basic}
\begin{aligned}
    \bbE[\wt W^{t+1}_s \wt W^{t'+1}_{s}]
    &=
    \xi^s\lt(\bbE[f_{t,s}(W^0_s,\dots,W^{t}_s)f_{t',s}(W^0_s,\dots,W^{t'}_s)]\rt),
    \\
    W^t_s
    &=
    \wt W^t_s+h_s;
    \\
    M^t_s
    &=
    f_{t,s}(W^0_s,\dots,W^t_s)
\end{aligned}
\end{equation}
and $\bbE[\wt W^{t+1}_{s} \wt W^{t'+1}_{s'}]=0$ if $s\neq s'$ (i.e. different species are independent). 


The following \emph{state evolution} characterizes the behavior of the above iterates. It states that for each $s\in\sS$, when $i\in \cI_s$ is uniformly random the sequence of coordinates $(w^1_i,w^2_i,\dots,w^t_i)$ has the same law as $(W^1_s,\dots,W^t_s)$. 


\begin{proposition}
\label{prop:state_evolution}
For any pseudo-Lipschitz function $\psi$ and $\ell\in\bbZ_{\geq 0}$, $s\in\sS$,
\begin{equation}
\label{eq:SE-body}
    \plim_{N\to\infty}\frac{1}{N_s}
    \sum_{i\in\cI_s}
    \psi(\bw^0_i,\dots,\bw^{\ell}_i)
    =
    \bbE
    [
    \psi(W^0_s,\dots,W^{\ell}_s)
    ]
    .
\end{equation}
\end{proposition}


This proposition allows us to read off normalized inner products of the AMP iterates, since e.g.
\[
    \langle \bw^k,\bw^{\ell}\rangle_N
    \simeq
    \sum_{s\in\sS}
    \lambda_s
    \bbE[W^k_s W^{\ell}_s].
\]



Proposition~\ref{prop:state_evolution} is proved in Appendix~\ref{sec:ProofSE}. For random matrices (i.e. the case of quadratic $H$) there is a considerable literature establishing state evolution in many settings beginning with \cite{bolthausen2014iterative,BM-MPCS-2011} and later \cite{bayati2015universality,berthier2019state,chen2020universality,fan2020approximate,dudeja2022universality} (see also \cite{feng2022unifying} for a survey of many statistical applications). The generalization to tensors was introduced in \cite{richard2014statistical} and proved in \cite{ams20}, whose approach we follow. 






\subsection{Stage $\I$: Finding the Root of the Ultrametric Tree}


Our goal in this subsection will be to compute a vector $\bm^{\ol}$ satisfying
\[
    \vR(\bm^{\ol},\bm^{\ol})\approx\Phi(q_0)
\]  
and with the correct energy value (as stated in Lemma~\ref{lem:sphereenergy} below). 




We take as given an maximizer $(p,\Phi;q_0)$ to $\bbA$ with associated transition point $q_1$. Recall this means there is a unique-up-to-scaling vector $v\in\mathbb R_{\geq 0}^{\sS}$ such that
\begin{equation}
\label{eq:v-equation}
    \lt(\xi^s(\Phi(q_1))+h_s^2\rt)v_s=
    \sum_{s'\in\sS}
    \partial_{x_s}\xi^{s'}
    (\Phi(q_1))
    \,
    \Phi_s(q_1)v_{s'}.
\end{equation}
We use the initialization
\[
    w^0_i = \sqrt{\xi^s(\Phi(q_1))+h_s^2},\quad i\in \cI_s.
\]
Define the vector $\va\in\mathbb R^{\sS}$ by 
\[
    a_s=
    \sqrt{\frac{\Phi_s(q_1)}{\xi^s(\Phi(q_1))+h_s^2}}
    =\sqrt{\frac{v_s}
    {\sum_{s'\in\sS} 
    \partial_{x_s} \xi^{s'}(\Phi(q_1))v_{s'}}}.
\]
Subsequent iterates are defined via the following recursion.
\begin{align}
\label{eq:RSsphere}
  \bw^{k+1}
  &=
  \nabla H_N(\bm^k)
  -
  \lt(\va\odot
    \zeta\big(\vR(\bm^k,\bm^{k-1})\big)
  \rt)
  \diamond
  \bm^{k-1}
  \\
\nonumber
  &=
  \bh
  +
  \nabla \wtH_N(\bm^k)
  -
  \lt(\va\odot
    \zeta\big(\vR(\bm^k,\bm^{k-1})\big)
  \rt)
  \diamond
  \bm^{k-1};
  \\
\label{eq:mk-def}
  \bm^k
  &=
  \va\diamond \bw^k
\\
\label{eq:zeta-defn}
    \zeta^s(\vx)
    &\equiv
    \sum_{s'\in \sS}
    \partial_{x_{s'}} \xi^s
    (\vx)
    .
\end{align}
The last term in \eqref{eq:RSsphere} comes from specializing the formula \eqref{eq:ONS-body} for the Onsager term.


Next recalling \eqref{eq:state-evolution-basic}, let $(W^j_s,M^j_s)_{j\geq 0,s\in\sS}$ be the state evolution limit of the coordinates of 
\[
    (\bw^{0},\bm^{0},\dots,\bw^k,\bm^k)
\]
as $N\to\infty$. Concretely, each $W^j_s$ is Gaussian with mean $h_s$ and 
\[
    M^{j}_s=\sqrt{\frac{\Phi_s(q_1)}{\xi^s(\Phi(q_1))+h_s^2}
    }
    \cdot 
    W^j_s,
    \quad
    j\geq 0,~
    s\in\sS.
\]
We next determine the covariance structure of the Gaussians $\wt W^j_s$. Define the (deterministic) $\mathbb R_{\geq 0}^{\sS}$-valued sequence $(R^0,R^1,\dots)$ of asymptotic overlaps recursively by $R^0=0$ and 
\begin{equation}
\label{eq:overlap-recursion-AMP}
    R^{k+1}_s=\alpha_s(R^k)=\lt(\xi^s(R^k)+h_s^2\rt)\cdot \lt(\frac{\Phi_s(q_1)}{\xi^s(\Phi(q_1))+h_s^2}\rt),\quad
    k\geq 0,s\in \sS.
\end{equation}



\begin{lemma}
\label{lem:RSconverge}
For integers $0\leq j<k$, the following equalities hold (the first in distribution):
\begin{align} 
\label{eq:id1.0}
    W^j_s&\stackrel{d}{=} h_s+Z\sqrt{\xi^s(\Phi(q_1))},\quad Z\sim \cN(0,1)\\
\label{eq:id2.0}
    \mathbb E[\wt W^j_s \wt W^k_s]&=\xi^s(R^j)\\
\label{eq:id3.0}
    \mathbb E[(M^j_s)^2]&=\Phi_s(q_1)\\
\label{eq:id4.0}
    \mathbb E[M^j_s M^k_s]&=R^{j+1}_s.
\end{align}
\end{lemma}



\begin{proof}
We proceed by induction on $j$, first showing \eqref{eq:id1.0} and \eqref{eq:id3.0} together. As a base case, \eqref{eq:id1.0} holds for $j=0$ by initialization. For the inductive step, assume first that \eqref{eq:id1.0} holds for $j$. Then by the definition \eqref{eq:mk-def},
\begin{align*}
  \mathbb E\lt[(M^j_s)^2\rt]&=
  \lt(\xi^s(\Phi(q_1))+h_s^2\rt)\cdot \lt(\frac{\Phi_s(q_1)}{\xi^s(\Phi(q_1))+h_s^2}\rt)
  \\
  &=\Phi_s(q_1)
\end{align*}
so that \eqref{eq:id1.0} implies \eqref{eq:id3.0} for each $j\geq 0$. On the other hand, state evolution directly implies that if \eqref{eq:id3.0} holds for $j$ then \eqref{eq:id1.0} holds for $j+1$. This establishes \eqref{eq:id1.0} and \eqref{eq:id3.0} for all $j\geq 0$.


We similarly show \eqref{eq:id2.0} and \eqref{eq:id4.0} together by induction, beginning with \eqref{eq:id2.0}. When $j=0$ it is clear because $\wt W^k_s$ is mean zero and independent of $\wt W^0_s$. 
Just as above, it follows from state evolution that \eqref{eq:id2.0} for $(j,k)$ implies \eqref{eq:id4.0} for $(j,k)$ which in turn implies \eqref{eq:id2.0} for $(j+1,k+1)$. Hence induction on $j$ proves \eqref{eq:id2.0} and \eqref{eq:id4.0} for all $(j,k)$.
\end{proof}


The next lemma is crucial and uses super-solvability of $\Phi(q_1)$.


\begin{lemma}
\label{lem:Rj-to-Phiq1}
    $\lim_{j\to\infty} R^j=\Phi(q_1).$
\end{lemma}


\begin{proof}
    First we observe that $\alpha$ is coordinate-wise strictly increasing in the sense that if $0\preceq x\prec y$ then $\alpha(x)\prec \alpha(y)$. 
    Moreover $\alpha(\vzero)\succ 0$ (assuming $\vh\neq 0$, else the result is trivial) and $\alpha(\Phi(q_1))=\Phi(q_1)$. Therefore $\oR=\lim_{j\to\infty} R^j$ exists, $\alpha(\oR)=\oR$, and
    \[
        \vzero\preceq \oR\preceq\Phi(q_1).
    \]
    It remains to show that the above forces $\oR=\Phi(q_1)$ to hold.
    
    
    Let $M\in\bbR^{\sS\times \sS}$ be the matrix with entries $M_{s,s'}=\deriv{t}\alpha_s'(\Phi(q_1)+te_{s'})|_{t=0}$ for $e_{s'}$ a standard basis vector. Then $M$ is the derivative matrix for $\alpha$ at $\Phi(q_1)$ in the sense that for any $u\in\bbR^{\sS}$,
    \[
        \deriv{t}\alpha'(\Phi(q_1)+tu)|_{t=0}=Mu.
    \]
    It is easy to see that all entries of $M$ are strictly positive, and that $Mv=v$ for $v$ solving \eqref{eq:v-equation}. By Perron-Frobenius theory, it follows that for any entry-wise non-negative vector $w\in\mathbb R_{\geq 0}^{\sS}$, we have
    \begin{equation}
    \label{eq:Mws} 
        (Mw)_s\leq w_s
    \end{equation}
    for some $s\in \sS$. Now suppose for sake of contradiction that $\oR\prec \Phi(q_1)$, let $w=\Phi(q_1)-\oR$, and choose $s^*\in\sS$ such that \eqref{eq:Mws} holds. Write $f(t)=\alpha_{s^*}(\Phi(q_1)+tw)$. Since $\alpha_{s^*}$ is a polynomial with non-negative coefficients and $\xi$ is non-degenerate, $f$ is strictly convex and strictly increasing on $[-1,0]$. Hence
    \begin{align*}
        \alpha_{s^*}(\oR)
        &= f(-1)
        \\
        &>
        f(0)-f'(0)
        \\
        &\geq
        \Phi_{s^*}(q_1)-(Mw)_{s^*}
        \\
        &\stackrel{\eqref{eq:Mws}}{\geq}
        \Phi_{s^*}(q_1)-w_{s^*}
        \\
        &=
        \oR_{s^*}.
    \end{align*}
    The first inequality above is strict, so we deduce that $\alpha(\oR)\neq\oR$ if $\oR\prec\Phi(q_1)$. This contradicts the definition of $\oR$. Therefore $\oR=\Phi(q_1)$, completing the proof.
\end{proof}


\begin{remark}
Super-solvability of $\Phi(q_1)$ is precisely the required condition for the above to go through. It is equivalent to the matrix $M=\alpha'$ above having Perron-Frobenius eigenvalue at most $1$. Indeed suppose that $\Phi(q_1)$ was chosen so that $\lambda_1(M)>1$. Then there exists $w\in\bbR_{>0}^{\sS}$ with $Mw\succ w$.
Letting $x=\Phi(q_1)-\eps w$ for small $\eps>0$, we find $\alpha(x)\prec x$.
Monotonicity implies that $\alpha$ maps the compact, convex set
\[
    K=\{y\in[0,1]^{\sS}~:~0\preceq y\preceq x\}
\]
into itself. By the Brouwer fixed point theorem, a fixed point of $\alpha$ strictly smaller than $\Phi(q_1)$ exists whenever $\Phi(q_1)$ is strictly subsolvable.
\end{remark}


We finish our analysis of the first AMP phase by computing the asymptotic energy it achieves. As expected, the resulting value agrees with the first term in the formula \eqref{eq:alg-for-optimizer} for $\ALG$.
%
\begin{lemma}
\label{lem:sphereenergy}
\[
  \lim_{k\to\infty} \plim_{N\to\infty}\frac{H_N(\bm^k)}{N}
  = 
    \sum_{s\in \sS}
    \lambda_s
      \sqrt{
      \Phi_s(q_1)
      \cdot
      \lt(h_s^2+\xi^s(\Phi(q_1))\rt)
      }
    \,.
\]
\end{lemma}

\begin{proof}

We use the identity
\begin{equation}
  \frac{H_N(\bm^k)}{N}=\big\langle \bh,\bm^k\rangle_N+\int_0^1 \langle \bm^k,\nabla \widetilde H_N(t\bm^k)\big\rangle_N \de t
\end{equation}
and interchange the limit in probability with the integral. To compute $\plim_{N\to\infty}\langle \bm^k,\nabla \widetilde H_N(t\bm^k)\rangle$ we introduce an auxiliary AMP step 
\[
    \by^{k+1}=\nabla \widetilde H_N(t\bm^k)-
    t\cdot
    \lt(\va\odot
    \zeta\big(
        t\cdot \vR(\bm^k,\bm^{k-1})
    \big)
    \rt)
    \diamond
    \bm^{k-1}
\] 
which depends implicitly on $t\in [0,1]$.
Rearranging yields
\begin{align*}
  \vR(\bm^k,\nabla \widetilde H_N(t\bm^k)) 
  &= 
  \vR(\bm^k,\by^{k+1})
  +
  t\cdot
  \lt(
  \vR(\bm^k,\bm^{k-1} )
  \odot \va
  \odot 
  \zeta(t\cdot \vR(\bm^k,\bm^{k-1}) )
  \rt)  
  \\
  &\simeq   
  \vR( \bm^k,\by^{k+1})
  +
  t\cdot
  \lt(
  \vR^k
  \odot \va
  \odot 
  \zeta(t\cdot \vR^k )
  \rt)  
  .
\end{align*}

For the first term, Gaussian integration by parts with $g_s(x)=(x+h_s)a_s$ yields
\begin{align*}
    R_s(\bm^k,\by^{k+1})
    &=
  \mathbb E[g_s(W^k)Y^{k+1}_s]
  \\
  &=
  \mathbb E[g_s'(W^k)]
  \cdot
  \mathbb E[W^k_s Y^{k+1}_s]
  \\
  &= 
  a_s
  \cdot 
  \xi^s(t R^k)
  .
\end{align*}
Integrating with respect to $t$, we find
\begin{align*} 
  \int_0^1 \langle \bm^k,\nabla \widetilde H_N(t\bm^k)\rangle_N \de t
  &\simeq 
  \sum_{s\in\sS}
  \lambda_s 
  \int_0^1
  R_s(\bm^k,\nabla \widetilde H_N(t\bm^k))
  \de t
  \\
  &\simeq
  \sum_{s\in\sS}
  \lambda_s
  a_s
  \int_0^1
  \xi^s(t\cdot \vR^k)
  +
  t R^k_s \zeta^s(t\cdot \vR^k)
  \de t 
  \\
  &=
  \sum_{s\in\sS}
  \lambda_s
  a_s
  \int_0^1
  \frac{\de ~}{\de t}
  \lt(t\cdot \xi^s(t\cdot \vR^k)\rt)
  \de t
  \\
  &=
  \sum_{s\in\sS}
  \lambda_s a_s \xi^s(\vR^k).
\end{align*}
Finally the external field $\bh$ gives energy contribution
\[
  \langle \bh, \bm^k\rangle_N 
  \simeq 
  \sum_{s\in\sS}
  \lambda_s
  h_s\bbE[M^k_s]
  =
  \sum_{s\in\sS}
  \lambda_s
  a_s
  h_s^2.
\]
Since $\lim_{k\to\infty} R^{k}=\Phi(q_1)$ by Lemma~\ref{lem:Rj-to-Phiq1}, we conclude
\begin{align*}
  \lim_{k\to\infty} \plim_{N\to\infty}\frac{H_N(\bm^k)}{N}
  &= 
  \sum_{s\in\sS}
  \lambda_s
  a_s\big(h_s^2 + \xi^s(\Phi(q_1))\big)
  \\
  &=
  \sum_{s\in\sS}
  \lambda_s
  \sqrt{
    \Phi_s(q_1)
  \cdot
  \lt(h_s^2+\xi^s(\Phi(q_1))\rt)
  }
  .
\end{align*}
\end{proof}


\subsection{Stage $\II$: Incremental Approximate Message Passing}


We now turn to the second phase which uses incremental approximate message passing. Choose a large constant $\ul$ and set
\begin{align*}
    \eps_s
    &=
    \sqrt{\frac{\Phi(q_0)_s}{R^{\ul}_s}}
    -1,
    \\
    \bn^{\ul}
    &=
    (1+\eps)
    \diamond 
    \bm^{\ul}
\end{align*}
so that 
\[
    \vR(\bn^{\ul},\bn^{\ul})\simeq \Phi(q_1).
\]
The point $\bn^{\ul}$ will be the ``root'' of our IAMP algorithm.\footnote{If $\vh=0$, one can instead set $n^{\ul}_i=\sqrt{\Phi_{s(i)}(\delta)}\diamond \bg$ for $\bg\sim\cN(0,I_N)$ and use the same algorithm with $q_0=0$.} We fix a small constant $\delta>0$ and consider the sequence
\[
    q^{\delta}_{\ell} 
    =
    q_1
    +
    \ell\delta,\quad \ell\geq 0.
\]
Moreover we set $\ol=\max\{\ell\in\bbZ_+~:~q_{\ell}^{\delta}\leq 1-2\delta\}.$
We also define for $s\in\sS$ and $\ul\leq \ell\leq \ol$ the constants
\begin{equation}
\label{eq:u-def}
    u_{\ell,s}^{\delta}
    =
    \sqrt{
    \frac{\Phi_s(q^{\delta}_{\ell+1})-\Phi_s(q^{\delta}_{\ell})}
    {
    \xi^s(\Phi(q_{\ell+1}^{\delta}))
    -
    \xi^s(\Phi(q_{\ell}^{\delta}))
    }
    }
    .
\end{equation}



Set $\bz^{\ul}=\wt\bw^{\ul}=\bw^{\ul}-\bh$.
So far, we have defined $(\bw^{\ul},\bz^{\ul},\bn^{\ul})$. We turn to inductively defining the triples $(\bw^{\ell},\bz^{\ell},\bn^{\ell})$ for $\ul\leq\ell\leq\ol$. First, the values $(\bz^{\ell})_{\ell\geq \ul}$ are defined as AMP iterates via
\begin{equation}
\label{eq:general_amp}
\begin{aligned}
    \bz^{\ell+1} 
    &= 
    \nabla \widetilde H_N(f_{\ell}(\bz^{\ul},\cdots,\bz^\ell)) - \sum_{j=0}^\ell d_{\ell, j}\diamond f_{j-1}(\bz^{\ul},\cdots,\bz^{j-1})
    .
\end{aligned}
\end{equation}
The Onsager coefficients $d_{\ell,j}$ are given by \eqref{eq:dts-def} and will not appear explicitly in any calculations.
Here the variables $Z^k$ are the state evolution limits. To complete the definition of the iteration \eqref{eq:general_amp}, for $s(i)=s$ and $\ell\geq \ul$ we set
\[
    f_{\ell,s}(z^{\ul}_i,\dots,z^{\ell}_i)
    =
    n^{\ell}_i,
\]
where
\begin{equation}
\label{eq:IAMP}
    \bn^{\ell+1} 
    =
    \bn^{\ell}+ 
    u_{\ell}^{\delta}
    \diamond
    \lt(\bz^{\ell+1}-\bz^{\ell}
    \rt).
\end{equation} 
Finally the algorithm $\cA$ outputs
\begin{equation}
\label{eq:round-final-output}
    \cA(H_N)
    =
    R(\bn^{\ol},\bn^{\ol})^{-1/2}\diamond \bn^{\ol}
    \in\cB_N
\end{equation}
where the power $-1/2$ is taken entry-wise.

The state evolution limits are described by time-changed Brownian motions with total variance $\Phi_s(q^{\delta}_{\ell})$ in species $s$ after iteration $\ell$. This is made precise below.

\begin{lemma}
\label{lem:BMlimit}
Fix $s\in\sS$. The sequence $(Z^{\delta}_{\ul,s},Z^{\delta}_{\ul+1,s},\dots)$ is a Gaussian process satisfying
\begin{align} 
\label{eq:BM1}
    \mathbb E[(Z^{\delta}_{\ell+1,s}-Z^{\delta}_{\ell,s})Z^{\delta}_{j,s}]
    &=
    0,\quad \text{for all }\ul+1\leq j\leq \ell
    \\
\label{eq:BM2}
    \mathbb E\big[
    (Z^{\delta}_{\ell+1,s}-Z^{\delta}_{\ell,s})^2
    \big]
    &=
    \xi^s(\Phi(q_{\ell+1}^{\delta}))
    -
    \xi^s(\Phi(q_{\ell}^{\delta}))
    \\
\label{eq:BM3}
    \mathbb E[Z^{\delta}_{\ell,s}Z^{\delta}_{j,s}]
    &=
    \xi^s(\Phi(q_{j\wedge \ell}^{\delta}))
    \\
\label{eq:BM4}
    \mathbb E[N^{\delta}_{\ell,s}N^{\delta}_{j,s}]
    &=
    \Phi_s(q^{\delta}_{j\wedge \ell})
    .
\end{align}
\end{lemma}


\begin{proof}
The fact that $(Z^{\delta}_{\ul,s},Z^{\delta}_{\ul+1,s},\dots)$ is a Gaussian process is a general fact about state evolution. We proceed by induction on $\ell\geq \ul$; in fact the proof is very similar to \cite[Section 8]{sellke2021optimizing} so we give only the main points. For the base case, the main computation is that 
\begin{align*}
    \bbE\big[\big(Z^{\delta}_{\ul+1,s}-Z^{\delta}_{\ul,s}\big)Z^{\delta}_{\ul,s}\big]
    &=
    \xi^s\lt(\bbE[N^{\delta}_{\ul,s}M^{\ul-1}_s]\rt)
    -
    \xi^s\lt(\bbE[M^{\ul-1}_{s}M^{\ul-1}_s]\rt)
    \\
    &=
    \xi^s\lt((1+\eps_s)\bbE[M^{\ul}_s M^{\ul-1}_s]\rt)-\xi^s(\Phi(q_1))
    \\
    &=
    \xi^s(\Phi(q_1))-\xi^s(\Phi(q_1))
    \\
    &=
    0.
\end{align*}
For inductive steps, we always have by state evolution
\[
    \bbE[Z^{\delta}_{\ell+1,s}Z^{\delta}_{j+1,s}]
    \simeq
    \xi^s\big(\vR(\bn^{\ell},\bn^{j})\big).
\]
It follows by the inductive hypothesis of \eqref{eq:BM1} that for $j\leq \ell$,
\begin{align*}
    R_s(\bn^{\ell},\bn^{j})
    &=
    R_s(\bn^{\ul},\bn^{\ul})
    +
    \sum_{k=\ul}^{j-1}
    (u_k^{\delta})^2
    R_s(\bz^{k+1}-\bz^k,\bz^{k+1}-\bz^k)
    \\
    &=
     R_s(\bn^{\ul},\bn^{\ul})
     +
     \sum_{k=\ul}^{j-1}
     (u_k^{\delta})^2
     \lt(
      \xi^s(\Phi(q_{k+1}^{\delta}))
    -
    \xi^s(\Phi(q_{k}^{\delta}))
    \rt)
    \\
    &=
    \Phi_s(q_1)
    +
    \sum_{k=\ul}^{j-1}
    \Big(
    \Phi_s(q^{\delta}_{k+1})
    -
    \Phi_s(q^{\delta}_{k})
    \Big)
    \\
    &=
    \Phi_s(q^{\delta}_j).
\end{align*}
Plugging into the above yields that for $j\leq \ell$,
\[
    \bbE[Z^{\delta}_{\ell+1,s}Z^{\delta}_{j+1,s}]
    =
    \xi^s(\Phi(q^{\delta}_j)).
\]  
This depends only on $\min(j,\ell)$, so \eqref{eq:BM1} follows. The others are proved by similar computations.
\end{proof}


Equation~\eqref{eq:BM4} implies that $\vR(\bn^{\delta}_{\ell},\bn^{\delta}_{j})\simeq \Phi(q^{\delta}_{\ell\wedge j})$, which exactly corresponds to the previous sections of the paper. In particular it implies that the final iterate $\bn^{\delta}_{\ol}$ satisfies
\begin{equation}
\label{eq:final-overlap-alg}
    (1-O(\delta))\cdot\vone\preceq \vR(\bn^{\delta}_{\ol},\bn^{\delta}_{\ol})\preceq \vone
\end{equation}
so the rounding step \eqref{eq:round-final-output} causes only an $O(\delta)$ change in the Hamiltonian value. Finally we compute in Lemma~\ref{lem:iampenergy} below the energy gain from the second phase, which matches the second term in \eqref{eq:alg-for-optimizer}. 


\begin{lemma}
\label{lem:iampenergy}
\begin{equation}
    \label{eq:iampenergy}
    \lim_{\ul\to\infty}
    \plim_{N \to \infty}
    \frac{H_{N}\lt(\bn^{\ol}\rt)-H_{N}\lt(\bn^{\ul}\rt)}{N} 
    = 
    \sum_{s\in\sS}
  \lambda_s
  \int_{q_1}^{1} 
    \sqrt{\Phi'_s(t) (\xi^s \circ \Phi)'(t)}
    \,
    \de t
\end{equation}
\end{lemma}



\begin{proof}
Recall that $\delta=\delta(\ul)\to 0$ as $\ul\to\infty$, which we will implicitly use throughout the proof. Observe also that $\langle h,\bn^{\ol}-\bn^{\ul}\rangle_N\simeq 0$ because the values $(N_{\ell,s}^{\delta})_{\ell\geq\ul}$ form a martingale sequence for each $s\in\sS$. 
Therefore it suffices to compute the in-probability limit of $\frac{\widetilde{H}_{N}\lt(\bn^{\ol}\rt)-\widetilde{H}_{N}\lt(\bn^{\ul}\rt)}{N}$. The key is to write 
\[
    \frac{\widetilde{H}_{N}\lt(\bn^{\ol}\rt)-\widetilde{H}_{N}\lt(\bn^{\ul}\rt)}{N}=\sum_{\ell=\ul}^{\ol-1}\frac{\widetilde{H}_{N}\lt(\bn^{\ell+1}\rt)-\widetilde{H}_{N}\lt(\bn^{\ell}\rt)}{N}
\]
and use a Taylor series approximation for each term. In particular for $F\in C^3(\mathbb R;\bbR)$, applying Taylor's approximation theorem twice yields
\begin{align*}
    F(1)-F(0)
    &=
    F'(0)+\frac{1}{2}F''(0)+O(\sup_{a\in [0,1]}|F'''(a)|)
    \\
    &= 
    F'(0)+\frac{1}{2}(F'(1)-F'(0))+O(\sup_{a\in [0,1]}|F'''(a)|)
    \\
    &=
    \frac{1}{2}(F'(1)+F'(0))+O(\sup_{a\in [0,1]}|F'''(a)|) .
\end{align*}

Assuming $\sup_{\ell}\frac{\|\bn^{\ell}\|}
{\sqrt{N}}\leq 1$, which holds with probability $1-o_N(1)$ by state evolution and the definition of $\ol$, we apply this estimate with 
\[
    F(a)=\widetilde{H}_N\lt((1-a)\bn^{\ell}+a\bn^{\ell+1}\rt).
\]
The result is:
\begin{align*}
    \lt|
    \widetilde{H}_{N}
    \lt(\bn^{\ell+1}\rt)-\widetilde{H}_{N}\lt(\bn^{\ell}\rt) -\frac{1}{2}\lt\langle \nabla \widetilde{H}_N(\bn^{\ell})+\nabla \widetilde{H}_N(\bn^{\ell+1}),\bn^{\ell+1}-\bn^{\ell}\rt\rangle \rt|
    &\leq 
    O\lt(
    \underline{C}N^{-1/2}
    \|\bn^{\ell+1}-\bn^{\ell}\|^3
    \rt)
    ;
    \\
    \underline{C}
    &=
    N^{1/2}
    \sup_{\|\bsig\|\leq \sqrt{N}}\lt\|\nabla^3 \widetilde{H}_N(\bsig)\rt\|_{\op}
    .
\end{align*}
Proposition~\ref{prop:gradients-bounded} implies that for deterministic constants $c,C$,
\[
    \bbP[\underline{C}\leq C]\geq 1-e^{-cN}.
\]
On the other hand for each $\ul\leq \ell\leq \ol-1$ we have
\begin{align*}
    \plim_{N\to\infty}\|\bn^{\ell+1}-\bn^{\ell}\|&=
    \sqrt{
    \sum_{s\in\sS}\lambda_s R_s(\bn^{\ell+1}-\bn^{\ell},\bn^{\ell+1}-\bn^{\ell})
    }
    \\
    &=
    \sqrt{
    \sum_{s\in\sS}\lambda_s
    \Phi_s(q^{\delta}_{\ell+1}-q^{\delta}_{\ell})
    }
    \\
    &=
    \sqrt{\delta N}.
\end{align*}
Summing and recalling that $\ol-\ul\leq \delta^{-1}$ yields the high-probability estimate
\begin{align*}
  \sum_{\ell=\ul}^{\ol-1}
  &
  \lt|
  \widetilde{H}_{N}\lt(\bn^{\ell+1}\rt)
  -
  \widetilde{H}_{N}\lt(\bn^{\ell}\rt) 
  -
  \frac{1}{2}\lt\langle  \nabla \widetilde{H}_N(\bn^{\ell})+\nabla \widetilde{H}_N(\bn^{\ell+1}),\bn^{\ell+1}-\bn^{\ell}\rt\rangle  
  \rt|
  \\ 
  &\leq 
  O(N^{-1/2})
  \cdot 
  \sum_{\ell=\ul}^{\ol-1} 
  \|\bn^{\ell+1}-\bn^{\ell}\|^3
  \\
  &\leq O(N\sqrt{\delta}).
\end{align*}
Because $\delta\to 0$ as $\ul\to\infty$, this term vanishes in the outer limit. It remains to prove
\[
  \lim_{\ul\to\infty}
  \plim_{N \to \infty}
  \sum_{\ell=\ul}^{\ol-1}
  \lt\langle 
  \nabla \widetilde{H}_N(\bn^{\ell})
  +
  \nabla \widetilde{H}_N(\bn^{\ell+1})
  ,
  \bn^{\ell+1}-\bn^{\ell}
  \rt\rangle_N  
  \stackrel{?}{=}
  2
  \sum_{s\in\sS}
  \lambda_s
  \int_{q_1}^{1} 
    \sqrt{\Phi'_s(t) (\xi^s \circ \Phi)'(t)}
  \de t.
\]
To establish this it suffices to show for each species $s\in\sS$ the equality
\begin{equation}
\label{eq:IAMP-energy-per-species}
  \lim_{\ul\to\infty}
  \plim_{N \to \infty}
  \sum_{\ell=\ul}^{\ol-1}
  R\lt(
  \nabla \widetilde{H}_N(\bn^{\ell})
  +
  \nabla \widetilde{H}_N(\bn^{\ell+1})
  ,
  \bn^{\ell+1}-\bn^{\ell}
  \rt)_s 
  \stackrel{?}{=}
  2
  \int_{q_1}^{1} 
    \sqrt{\Phi'_s(t) (\xi^s \circ \Phi)'(t)}
  \de t.
\end{equation}
This is what we do. Observe by \eqref{eq:general_amp} that:
%
\begin{equation}
\label{eq:amprearrange}
    \nabla \widetilde{H}_N(\bn^{\ell})
    =
    \bz^{\ell+1}- \sum_{j=0}^\ell d_{\ell, j}\diamond \bn^{j-1}.
%
\end{equation}
%
Passing to the limiting Gaussian process $(Z^{\delta}_k)_{k\in\mathbb Z^+}$ via state evolution,
\begin{align*}
  \plim_{N\to\infty}
  R\lt(\nabla \widetilde{H}_N(\bn^{\ell}),\bn^{\ell+1}-\bn^{\ell}\rt)_s
  &=
  \mathbb E\lt[ Z^{\delta}_{\ell+1,s}(N^{\delta}_{\ell+1,s}-N^{\delta}_{\ell,s})\rt]
  - 
  \sum_{j=0}^{\ell}
  d_{\ell,j,s}
  \mathbb E\lt[
  N^{\delta}_{j-1,s}(N^{\delta}_{\ell+1,s}-N^{\delta}_{\ell,s})
  \rt],
  \\
  \plim_{N\to\infty}
  R\lt(\nabla \widetilde{H}_N(\bn^{\ell+1}),\bn^{\ell+1}-\bn^{\ell}\rt)_s
  &=
  \mathbb E\lt[ Z^{\delta}_{\ell+2,s}(N^{\delta}_{\ell+1,s}-N^{\delta}_{\ell,s})\rt]
  - 
  \sum_{j=0}^{\ell+1} 
  d_{\ell+1,j,s}
  \mathbb E\lt[
  N^{\delta}_{j-1}(N^{\delta}_{\ell+1,s}-N^{\delta}_{\ell,s})
  \rt].
\end{align*}


As $(N^{\delta}_k)_{k\geq \mathbb Z^+}$ is a martingale process, it follows that the right-hand expectations all vanish. Similarly it holds that
\begin{align*}
  \mathbb E[Z_{\ell+2}^{\delta}(N_{\ell+1}^{\delta}-N_{\ell}^{\delta})]&=\mathbb E[Z_{\ell+1}^{\delta}(N_{\ell+1}^{\delta}-N_{\ell}^{\delta})]
  \\
  \mathbb E[Z_{\ell}^{\delta}(N_{\ell+1}^{\delta}-N_{\ell}^{\delta})]&=0.
\end{align*}
%
We conclude that
%
\begin{align*}
  \plim_{N\to\infty}
  R\lt(\nabla \widetilde{H}_N(\bn^{\ell})+\nabla \widetilde{H}_N(\bn^{\ell+1}),\bn^{\ell+1}-\bn^{\ell}\rt)_s
  &=
  2\,\mathbb E[
  (Z_{\ell+1,s}^{\delta}-Z^{\delta}_{\ell,s})(N_{\ell+1,s}^{\delta}-N_{\ell,s}^{\delta})
  ]
  \\
  &=
  2\,\mathbb E[u_{\ell,s}^{\delta}(Z^{\delta}_{\ell,s})(Z_{\ell+1,s}^{\delta}-Z^{\delta}_{\ell,s})^2]
  \\
  &=
  2\,\mathbb E[u_{\ell,s}^{\delta}(Z^{\delta}_{\ell,s})]
  \cdot
  \bbE[(Z_{\ell+1,s}^{\delta}-Z^{\delta}_{\ell,s})^2]
  \\
&=
  2\,\sqrt{
  \Big(
  \Phi_s(q^{\delta}_{\ell+1,s})-\Phi_s(q^{\delta}_{\ell,s})
  \Big)
  \cdot
    \Big(\xi^s(\Phi(q_{\ell+1}^{\delta}))-\xi^s(\Phi(q_{\ell}^{\delta}))
    \Big)
    }
    .
\end{align*}
In the second-to-last step we used the fact that $Z^{\delta}_{\ell,s}$ has independent increments, which follows from Lemma~\ref{lem:BMlimit}, while the last step used \eqref{eq:u-def} and \eqref{eq:BM2}.
Combining with Lemma~\ref{lem:ALG-from-continuum} implies \eqref{eq:IAMP-energy-per-species}.
\end{proof}


\begin{proof}[Proof of Theorem~\ref{thm:main-alg}]
We take $\cA$ as in \eqref{eq:round-final-output} for $\ul$ a large constant depending on $(\eps,\xi,h,\lambda)$. The fact that 
\begin{equation}
\label{eq:1-o1-prob-alg}
    \bbP[H_N(\cA(H_N))/N \geq \ALG-\eps/2]
    \ge 
    1-o_N(1)
\end{equation}
follows from combining Lemma~\ref{lem:sphereenergy}, Lemma~\ref{lem:iampenergy} and the fact that (recall \eqref{eq:final-overlap-alg}) 
\[
H_N(\cA(H_N))/N \simeq H_N(\bn^{\ul})/N+o_{\bbP}(1).
\]

Next, let $K_N\subseteq\sH_N$ be as in Proposition~\ref{prop:gradients-bounded}. We recall that $\bbP[H_N\in K_N]\geq 1-e^{-cN}$.
Exactly as in \cite[Theorem 10]{huang2021tight} it follows that there is a $C(\eps)$-Lipschitz function $\wt\cA:\sH_N\to\bbR$ such that $\wt\cA$ and $\cA$ agree on $K_N$.
Moreover \eqref{prop:gradients-bounded} and concentration of measure on Gaussian space imply that $H_N(\wt\cA(H_N))$ is $O(N^{1/2})$-sub-Gaussian. In light of \eqref{eq:1-o1-prob-alg} and since $\bbP[\wt\cA(H_N)=\cA(H_N)]\geq \bbP[H_N\in K_N]\geq 1-e^{-cN}$, we deduce that 
\[
    \bbP[H_N(\cA(H_N))/N \geq \ALG-\eps]
    \ge 
    1-e^{-cN}.
\]
This concludes the proof.
\end{proof}


\begin{remark}
    Analogously to \cite[Section 4]{sellke2021optimizing}, the second stage of our IAMP algorithm can be slightly modified to give a full ultrametric tree of outputs, whose pairwise overlaps are given by values of $\Phi$. More precisely, for any finite ultrametric space $X=(x_1,\dots,x_m)$ of diameter at most $1-q_1$, a branching variant of our algorithm outputs $(\bsig_1,\dots,\bsig_n)$ with
    \[
        \plim_{N\to\infty}
        \max_{1\leq i,j\leq m}\tnorm{
            \vR(\bsig_i,\bsig_j)
            - \Phi\big(1-d_X(x_i,x_j)\big)
        }_{\infty}=0.
    \]
    We omit the details of this extension.
\end{remark}

