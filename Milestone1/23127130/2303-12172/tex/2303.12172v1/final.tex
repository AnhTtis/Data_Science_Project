\documentclass[10pt]{article}
%\usepackage{setspace}
%\usepackage{cprotect}

% \usepackage{polski}

% define RedeclareMathOperator
\makeatletter
\newcommand\RedeclareMathOperator{%
  \@ifstar{\def\rmo@s{m}\rmo@redeclare}{\def\rmo@s{o}\rmo@redeclare}%
}
% this is taken from \renew@command
\newcommand\rmo@redeclare[2]{%
  \begingroup \escapechar\m@ne\xdef\@gtempa{{\string#1}}\endgroup
  \expandafter\@ifundefined\@gtempa
     {\@latex@error{\noexpand#1undefined}\@ehc}%
     \relax
  \expandafter\rmo@declmathop\rmo@s{#1}{#2}}
% This is just \@declmathop without \@ifdefinable
\newcommand\rmo@declmathop[3]{%
  \DeclareRobustCommand{#2}{\qopname\newmcodes@#1{#3}}%
}
\@onlypreamble\RedeclareMathOperator
\makeatother

\usepackage{witharrows}
\usepackage{tikz}
\usepackage{graphicx,wrapfig,lipsum}   

\usepackage{xspace}

\usepackage{framed}

\usepackage{amsmath,amssymb}
\usepackage{amsthm}
\usepackage{mathtools}
%\usepackage{fancybox}
\usepackage[noend]{algorithmic}
\usepackage[ruled,vlined]{algorithm2e}
\usepackage{url}
% \def\UrlBreaks{\do\/\do-}
\usepackage{fullpage}
\usepackage{makeidx}
\usepackage{enumerate}
\usepackage[top=1in, bottom=1.1in, left=0.8in, right=0.8in]{geometry}
\usepackage{graphicx,float,psfrag,epsfig,caption}

%\definecolor{darkgreen}{rgb}{0.0,0,0.9}
\usepackage{hyperref}
\usepackage{epstopdf}
%\usepackage[pagebackref,letterpaper=true,colorlinks=true,pdfpagemode=none,citecolor=OliveGreen,linkcolor=BrickRed,urlcolor=BrickRed,pdfstartview=FitH]{hyperref}
\usepackage{color}
%\usepackage{lmodern}% http://ctan.org/pkg/lm
\usepackage{xr}
% \usepackage{subfig}
% \usepackage{caption}
\usepackage{subcaption}
\usepackage[utf8]{inputenc}

\usepackage{scalerel,stackengine}

\usepackage{thm-restate}

\usepackage{enumitem}
\usepackage{bbm}

%\usepackage[usenames,dvipsnames,svgnames,table]{xcolor}

\stackMath
\newcommand\reallywidehat[1]{%
\savestack{\tmpbox}{\stretchto{%
  \scaleto{%
    \scalerel*[\widthof{\ensuremath{#1}}]{\kern.1pt\mathchar"0362\kern.1pt}%
    {\rule{0ex}{\textheight}}%WIDTH-LIMITED CIRCUMFLEX
  }{\textheight}% 
}{2.4ex}}%
\stackon[-6.9pt]{#1}{\tmpbox}%
}
\parskip 1ex
\usepackage[mathscr]{euscript} 

% Copied from mathrsfs.sty
\DeclareSymbolFont{rsfs}{U}{rsfs}{m}{n}
\DeclareSymbolFontAlphabet{\mathscrsfs}{rsfs}
\usepackage{mathrsfs}

\numberwithin{equation}{section}

\renewcommand{\paragraph}[1]{\noindent\textbf{#1}.\quad}

\newtheoremstyle{myexample} % name
    {\topsep}                    % Space above
    {\topsep}                    % Space below
    {\rm }                   % Body font
    {}                           % Indent amount
    {\bf }                   % Theorem head font
    {.}                          % Punctuation after theorem head
    {.5em}                       % Space after theorem head
    {}  % Theorem head spec (can be left empty, meaning normal)

\newtheoremstyle{myremark} % topsep
    {\topsep}                    % Space above
    {\topsep}                    % Space below
    {\rm}                        % Body font
    {}                           % Indent amount
    {\bf}                        % Theorem head font
    {.}                          % Punctuation after theorem head
    {.5em}                       % Space after theorem head
    {}  % Theorem head spec (can be left empty, meaning normal)


\newtheorem{claim}{Claim}[section]
\newtheorem{lemma}[claim]{Lemma}
\newtheorem{fact}[claim]{Fact}
\newtheorem{assumption}{Assumption}[]
\newtheorem{conjecture}[claim]{Conjecture}
\newtheorem{theorem}{Theorem}
\newtheorem{proposition}[claim]{Proposition}
\newtheorem{corollary}[claim]{Corollary}

\theoremstyle{definition}
\newtheorem{definition}[claim]{Definition}


\theoremstyle{myremark}
\newtheorem{remark}{Remark}[section]

\theoremstyle{myremark}
\newtheorem{open}{Open Problem}[section]

\theoremstyle{myexample}
\newtheorem{example}[remark]{Example}

\newtheorem{exercise}[remark]{Exercise}

\definecolor{darkgreen}{rgb}{0.0, 0.5, 0.0}
\newcommand{\mscomment}[1]{\noindent{\textcolor{red}{[\textbf{MS:} #1]}}}

\newcommand{\bhcomment}[1]{\noindent{\textcolor{blue}{[\textbf{BH:} #1]}}}

\hypersetup{
  colorlinks   = true, %Colours links instead of ugly boxes
  urlcolor     = blue, %Colour for external hyperlinks
  linkcolor    = blue, %Colour of internal links
  citecolor   = red %Colour of citations
}

\DeclareMathOperator*{\veccat}{%
    \mathchoice%
        {\Bigg\Vert}%
        {\Big\Vert}%
        {\Vert}%
        {\Vert}%
}%


\title{Algorithmic Threshold for Multi-Species Spherical Spin Glasses}
\author{
    Brice Huang
    % \thanks{
    %     Department of Electrical Engineering and Computer Science,
    %     Massachusetts Institute of Technology.
    % }
    \and
    Mark Sellke
    % \thanks{
    %     School of Mathematics,
    %     Institute for Advanced Study.
    % }
}
\date{}

\begin{document}

\maketitle

\begin{abstract}
As models continue to grow in size, the development of memory optimization methods (MOMs) has emerged as a solution to address the memory bottleneck encountered when training large models. To comprehensively examine the practical value of various MOMs, we have conducted a thorough analysis of existing literature from a systems perspective. 
% Furthermore, we have evaluated the most widely adopted MOMs employed in mainstream frameworks for both vision and language models.
Our analysis has revealed a notable challenge within the research community: the absence of standardized metrics for effectively evaluating the efficacy of MOMs. The scarcity of informative evaluation metrics hinders the ability of researchers and practitioners to compare and benchmark different approaches reliably. Consequently, drawing definitive conclusions and making informed decisions regarding the selection and application of MOMs becomes a challenging endeavor.
To address the challenge, this paper summarizes the scenarios in which MOMs prove advantageous for model training. We propose the use of distinct evaluation metrics under different scenarios. By employing these metrics, we evaluate the prevailing MOMs and find that their benefits are not universal. We present insights derived from experiments and discuss the circumstances in which they can be advantageous.

\end{abstract}
\setcounter{tocdepth}{1}
\tableofcontents
% \newpage

\section{Introduction}
\IEEEPARstart{T}{he} method Neural Radiance Fields (NeRF)~\cite{mildenhall2020nerf} is proposed for photorealistic novel view synthesis. Given many views of the scene, it creates implicit multi-view geometry and learns for view synthesis. However, it has poor generalizations to new scenes and requires retraining or fine-tuning on each scene. 
 
 Recent work~\cite{Yu_2021_CVPR,Trevithick_2021_ICCV} has explored the ways of using a single image to train NeRF. They introduce a convolutional feature encoder to learn the image representation which gives it some limited generalization abilities to unseen scenes.  But, without fine-tuning, these methods produce many floats and artifacts in rendering novel views. 
 
  Multi-Plane Images (MPI) representation that learns multiple RGB images from a single image is also used in \cite{Wu_2021_ICCV,Tucker_2020_CVPR,wu2022remote} for  novel view synthesis. However, MPI heavily relies on the qualities of the planar images and needs plenty of image planes to avoid blurs. There is no strong 3D geometry constraint and it fails in many complex scenes.
  
  MINE~\cite{Li_2021_ICCV2} introduces the volume rendering of NeRF into the MPI. It runs faster and produces better depth rendering quality compared with single-view NeRFs~\cite{Yu_2021_CVPR,Trevithick_2021_ICCV}. However, the rendering quality heavily relies on the number of image planes. It needs high-resolution 4D volumes to store the 4-channel  (RGB and volume density) image planes that cost a large amount of GPU memory in both training and 
 prediction.  
 

 
 \begin{figure}[t]
\setlength{\abovecaptionskip}{7pt}
\setlength{\belowcaptionskip}{0pt}
	\centering
% 	\subfigure[MINE (PSNR:14.9)]{  % for AAAI
	\subfloat[MINE (PSNR:14.9)]{
%			\centering
			\includegraphics[width=0.23\textwidth]{figure/intro/DJI_20200223_163206_598_0_MINE.png}
%			\label{subfig:pixelnerf}
	}\subfloat[MINE (depth)]{
%			\centering
			\includegraphics[width=0.23\textwidth]{figure/intro/MINE_disp.png}
%			\label{subfig:mpi}
	}
	\\[-3mm]
	\subfloat[Ours (PSNR:17.0)]{
%			\centering
			\includegraphics[width=0.23\textwidth]{figure/intro/DJI_20200223_163206_598_0_ours.png} 
	}\subfloat[Ours (depth)]{
%			\centering
			\includegraphics[width=0.23\textwidth]{figure/intro/ours_disp.png}
	}
	\caption{Comparison with state-of-the-art methods. (a-b) RGB and depth rendering results of  \cite{Li_2021_ICCV2}. It produces many blurs and floats in the occluded regions and at the object/depth edges. 
	(c-d) Our method employs a joint rendering mechanism that preserves more image details and predicts sharp depth edges.}
	\label{fig:performance_illustration}
\end{figure}
 
 In this paper, we propose a joint rendering mechanism that takes the MPI strategy for coarse sampling proposals and the MLP\&volume-based rendering~\cite{mildenhall2020nerf} for fine sampling and rendering. Then, both the coarse point samples and the fine samples are combined according to their geometry distribution to realize a more accurate joint rendering. More importantly, we introduce a depth teacher net that serves as the guidance for the joint rendering. The monocular depth teacher predicts dense pseudo depth maps that assist the consistent 3D geometry learning between the MPI, the fine volume, and the joint rendering. It also boosts the multi-view geometry consistency between the source view and the target novel views that 
helps handle the occlusions, reduce the blurs and floats, and render accurate depths. 
 
In the experiments,  we verify the effectiveness of our method on three challenging real-scene datasets (RealEstate10K~\cite{zhou2018stereo}, NYU~\cite{silberman2012indoor} and  NeRF-LLFF~\cite{mildenhall2020nerf}) for novel view synthesis or depth estimation. Given a single image as input, our method is shown able to produce higher qualities in both the RGB image rendering and depth map prediction. It far outperforms state-of-the-art methods~\cite{Li_2021_ICCV2,Yu_2021_CVPR} with improvements of 5$\sim$20\% in PSNR and SSIM for the RGB rendering and reduces 20$\sim$50\% of the errors for the depth prediction.
\section{Algorithmic Thresholds from Branching OGP}

We begin this section by recalling some fundamental definitions and constructions from \cite{huang2021tight}. 
We then review the details of the branching overlap gap property introduced in \cite{huang2021tight}, and in particular the link to hardness for overlap concentrated algorithms. 

\subsection{Correlation Functions and Overlap Concentration}
\label{subsec:correlation}

For any $p\in [0,1]$, we may construct two correlated copies $\HNp{1}, \HNp{2}$ of $H_N$ as follows.
Construct three i.i.d. copies $\wtH_N^{[0]}, \wtH_N^{[1]}, \wtH_N^{[2]}$ of $\wtH$ as in \eqref{eq:def-hamiltonian-no-field}.
For $i=1,2$ define
\begin{align*}
    \HNp{i}(\bsig) 
    &= 
    \la \bh, \bsig \ra 
    + \wtHNp{i}(\bsig), \quad \text{where} \\
    \wtHNp{i}(\bsig) 
    &= 
    \sqrt{p} \wtH_N^{[0]}(\bsig) + 
    \sqrt{1-p} \wtH_N^{[i]}(\bsig).
\end{align*}
We say $\HNp{1}, \HNp{2}$ are $p$-correlated.
Note that pairs of corresponding entries in $\bg(\HNp{1})$ and $\bg(\HNp{2})$ are Gaussian with covariance $\lt[\begin{smallmatrix}1 & p \\ p & 1\end{smallmatrix}\rt]$.

Given a function $\cA_N : \sH_N \to \cB_N$ (always assumed to be measurable) define $\vchi:[0,1] \to \bbR^{\sS}$ by
\begin{equation}
    \label{eq:def-correlation-fn}
    \vchi(p) 
    = 
    \E \vR \lt(
        \cA(\HNp{1}), 
        \cA(\HNp{2})
    \rt),
\end{equation}
where $\HNp{1}, \HNp{2}$ are $p$-correlated copies of $H_N$. 
We say that $\vchi$ is the \textbf{correlation function} of $\cA$.
Let $\chi_s$ denote the $s$-coordinate of $\vchi$.

\begin{proposition}
    \label{prop:correlation-fn-properties}
    We have $\vchi \in \bbI(0,1)^\sS$.
\end{proposition}
\begin{proof}
    Identically to \cite[Proposition 3.1]{huang2021tight}, Hermite expanding $R_s \lt(\cA(\HNp{1}), \cA(\HNp{2}) \rt)$ shows that $\chi_s$ is continuous and increasing.
    The same Hermite expansion shows $\chi_s$ is continuously differentiable.
\end{proof}
The other properties of correlation functions proved in \cite[Proposition 3.1]{huang2021tight} also hold, namely that $\chi_s$ is convex and either strictly increasing or constant; however they are not needed in this paper. 

We will determine the maximum energy attained by algorithms $\cA_N : \sH_N \to \cB_N$ obeying the following overlap concentration property. 

\begin{definition}
    \label{defn:oc}
    Let $\eta, \nu > 0$. 
    An algorithm $\cA = \cA_N$ is $(\eta,\nu)$ overlap concentrated if for any $p\in [0,1]$ and $p$-correlated Hamiltonians $\HNp{1}, \HNp{2}$,
    \begin{equation}
        \label{eq:overlap-concentrated}
        \P
        \lt[
            \norm{ 
                \vR \lt( \cA(\HNp{1}), \cA(\HNp{2}) \rt) -
                \vchi(p)
            }_{\infty}
            \ge \eta
        \rt]
        \le \nu.
    \end{equation}
\end{definition}

Our main hardness result is the following bound on the performance of overlap concentrated algorithms.

\begin{theorem}
    \label{thm:main-ogp-oc}
    Consider a multi-species spherical spin glass Hamiltonian $H_N$ with parameters $(\xi,\vh)$. 
    Let $\ALG$ be given by \eqref{eq:alg}.
    For any $\eps > 0$ there are $\eta, c, N_0$ depending only on $\xi, \vh, \eps$ such that the following holds for any $N\ge N_0$ and $\nu \in [0,1]$. 
    For any $(\eta,\nu)$-overlap concentrated $\cA_N : \sH_N \to \cB_N$, 
    \[
        \bbP\lt[H_N(\cA_N(H_N))/N \ge \ALG + \eps\rt]
        \le 
        \exp(-cN) + 
        \nu^c.
    \]
\end{theorem}
By Gaussian concentration of measure (see \cite[Propositon 8.2]{huang2021tight}), any $\tau$-Lipschitz algorithm is $(\eta, e^{-c(\eta,\tau)N})$-overlap concentrated for any $\eta>0$ and appropriate $c(\eta,\tau)>0$.
Thus Theorem~\ref{thm:main-ogp-oc} implies Theorem~\ref{thm:main-ogp}.

\subsection{Ultrametrically Correlated Hamiltonians}
\label{subsec:ultra-corr-H}

Next we define the hierarchically correlated ensemble of Hamiltonians used to define the branching overlap gap property.
Let $k\ge 2$, $D\ge 1$ be positive integers.
For each $0\le d \le D$, let $V_d = [k]^d$ denote the set of length $d$ sequences of elements of $[k]$. 
The set $V_0$ consists of the empty tuple, which we denote $\emptyset$.
Let $\bbT(k,D)$ denote the depth $D$ tree rooted at $\emptyset$ with depth $d$ vertex set $V_d$, where $u\in V_d$ is the parent of $v\in V_{d+1}$ if $u$ is the length $d$ initial substring of $v$.
For nodes $u^1,u^2\in \bbT(k,D)$, let 
\[
    u^1 \wedge u^2
    =
    \max \lt\{
        d \in \bbZ_{\ge 0}: 
        \text{$u^1_{d'} = u^2_{d'}$ for all $1\le d' \le d$}
    \rt\},
\]
where the set on the right-hand side always contains $0$ vacuously.
This is the depth of the least common ancestor of $u^1$ and $u^2$.
Let $\bbL(k,D) = V_D$ denote the set of leaves of $\bbT(k,D)$.
When $k,D$ are clear from context, we denote $\bbT(k,D)$ and $\bbL(k,D)$ by $\bbT$ and $\bbL$.
Finally, let $K = |\bbL| = k^D$.

Let the sequences $\up = (p_0, p_1, \ldots, p_D)\in\bbR^{D+1}$ 
and $\uvphi = (\vphi_0, \vphi_1, \ldots, \vphi_D)\in (\bbR^{\sS})^{D+1}$ 
satisfy
\begin{align*}
    0 = p_0 \le p_1 \le \cdots \le p_D &= 1, \\
    \vzero \preceq \vphi_0 \preceq \vphi_1 \preceq \cdots \preceq \vphi_D &\preceq \vone.
\end{align*}
The sequence $\up$ controls the correlation structure of our ensemble of Hamiltonians while the sequence $\uvphi$ controls the overlap structure of their inputs.
For each $u\in \bbT$, including interior nodes, let $\wtH_N^{[u]}$ be an independent copy of $\wtH_N$ generated by \eqref{eq:def-hamiltonian-no-field}, and let
\begin{equation}
    \label{eq:def-correlated-disorder}
    \wtHNp{u} = 
    \sum_{d=1}^{|u|}
    \sqrt{p_d - p_{d-1}} \cdot
    \wtH_N^{[(u_1,\ldots,u_d)]}
\end{equation}
where $|u|$ is the length of $u$ and $(u_1,\ldots,u_d)$ is the length-$d$ prefix of $u$.
For $u\in \bbL$, define
\[
    \HNp{u}(\bsig) = 
    \la \bh, \bsig \ra + 
    \wtHNp{u}(\bsig).
\]
This constructs a Hamiltonian ensemble $(\HNp{u})_{u\in \bbL}$ where each $\HNp{u}$ is marginally distributed as $H_N$ and each pair of Hamiltonians $\HNp{u^1}, \HNp{u^2}$ is $p_{u^1\wedge u^2}$-correlated.
We define a grand Hamiltonian on states
\[
    \ubsig = (\bsig(u))_{u\in \bbL} \in (\bbR^N)^\bbL.
\]
by
\begin{equation}
    \label{eq:grand-hamiltonian}
    \cH_N^{k,D,\up}(\ubsig) 
    =
    \fr{1}{K}
    \sum_{u\in \bbL}
    \HNp{u}(\bsig(u)).
\end{equation}
We denote this by $\cH_N$ when $k,D,\up$ are clear from context.
Note that we have thus far not used the definition of $\wtHNp{u}$ for interior nodes $u\in \bbT \setminus \bbL$; these Hamiltonians will be useful in our analysis of the branching OGP threshold in Section~\ref{sec:uc}.
The branching OGP is defined by a maximization of $\cH_N$ over the overlap-constrained set
\begin{equation}
\label{eq:cQ}
    \cQ^{k,D,\uvphi}(\eta)
    =
    \lt\{
        \ubsig \in \cB_N^\bbL : 
        \norm{\vR(\bsig(u^1),\bsig(u^2)) - \vphi_{u^1\wedge u^2}}_\infty 
        \le \eta,
        ~\forall u^1,u^2 \in \bbL
    \rt\}.
\end{equation}
We denote this set $\cQ(\eta)$ when $k,D,\uvphi$ are clear from context.

\subsection{The Branching OGP Threshold}

We will show that overlap concentrated algorithms cannot outperform a \emph{branching OGP} energy $\BOGP$ defined as the ground state energy of the grand Hamiltonian \eqref{eq:grand-hamiltonian} in the limit of ``continuously branching" ultrametrics.

\begin{definition}[Branching OGP energy]
    \label{defn:bogp}
    The energy $\BOGP = \BOGP(\xi,\vh)$ is the infimum of energies $E$ such that the following holds.
    Choose sufficiently large $D$, followed by small $\eta$ and then large $k$. For any $\vchi \in \bbI(0,1)^\sS$ there exists $\up$ such that for $\uvphi = \vchi(\up)$ element-wise (i.e. $\vphi_d=\vchi(p_d)$),
    \begin{equation}
        \label{eq:bogp}
        \limsup_{N\to\infty}
        \fr{1}{N} 
        \bbE \sup_{\ubsig \in \cQ(\eta)}
        \cH_N(\ubsig)
        \le 
        E.
    \end{equation}
    More explicitly,
        \begin{equation}
        \label{eq:bogp-explicit}
        \BOGP(\xi,\vh)
        \equiv
        \lim_{D\to\infty}
        \lim_{\eta\to 0}
        \lim_{k\to\infty}
        \sup_{\vchi \in \bbI(0,1)^\sS}
        \inf_{\uvphi=\vchi(\up)}
        \limsup_{N\to\infty}
        \fr{1}{N} 
        \bbE \sup_{\ubsig \in \cQ^{k,D,\uvphi}(\eta)}
        \cH_N^{k,D,\up}(\ubsig).
    \end{equation}
\end{definition}
Our previous work \cite{huang2021tight} implicitly considered the same quantity. Note that the limits in $(D,k,\eta)$ are decreasing, so they could actually be taken in any order (and moreover the limiting value $\BOGP$ exists apriori). Additionally the role of the infimum over $(\uvphi,\up)$ is quite simple: the only important thing is to ensure both sequences increase in uniformly small steps (see Definition~\ref{def:delta-dense}).

Section~\ref{sec:uc} proves the following proposition identifying $\BOGP$ with the formula \eqref{eq:alg} for $\ALG$.
\begin{proposition}
    \label{prop:bogp-alg}
    For all $(\xi,\vh)$, we have $\BOGP = \ALG$.
\end{proposition}
Let us first prove Theorem~\ref{thm:main-ogp-oc} assuming Proposition~\ref{prop:bogp-alg}.
Let $\eps > 0$ be arbitrary and $k,D,\eta$ be given by Definition~\ref{defn:bogp} for $E = \ALG + \eps/4$. 
Let $\cA = \cA_N : \sH_N \to \cB_N$ be a $(\eta,\nu)$-overlap concentrated algorithm with correlation function $\vchi$.
Let $\up$ and $\uvphi$ be given by Definition~\ref{defn:bogp} (depending on $\vchi$). 
Since $\BOGP = \ALG$ by Proposition~\ref{prop:bogp-alg}, for sufficiently large $N$
\[
    \fr{1}{N} 
    \bbE \sup_{\ubsig \in \cQ(\eta)}
    \cH_N(\ubsig)
    \le 
    \ALG
    + \eps/2.
\]
Let 
\[
    \alpha_N = \bbP\lt[
        H_N(\cA(H_N)) \ge \ALG + \eps      
    \rt].
\]
Let $\bsig(u) = \cA(\HNp{u})$ and $\ubsig = (\bsig(u))_{u\in \bbL}$. 
Define the events
\begin{equation}
\label{eq:S-events}
\begin{aligned}
    \Ssolve &= \lt\{\HNp{u}(\bsig(u)) / N \ge \ALG + \eps ~\forall u\in \bbL\rt\}, \\
    \Soverlap &= \lt\{\ubsig \in \cQ(\eta)\rt\}, \\
    \Sogp &= \lt\{ \sup_{\ubsig \in \cQ(\eta)} \cH_N(\ubsig) / N < \ALG + \eps \rt\}.
\end{aligned}
\end{equation}
\begin{proposition}
    \label{prop:prob-ineqs}
    The following inequalities hold. 
    \begin{enumerate}[label=(\alph*), ref=\alph*]
        \item \label{itm:ssolve} $\bbP(\Ssolve) \ge \alpha_N^K$.
        \item \label{itm:soverlap} 
        $\bbP(\Soverlap) \ge 1 - K^2\nu$.
        \item \label{itm:sogp} $\bbP(\Sogp) \ge 1 - 2 \exp(-cN)$ for suitable $c=c(\eps) > 0$.
    \end{enumerate}
\end{proposition}
\begin{proof}[Proof of (\ref{itm:ssolve})]
    Use Jensen's inequality $D$ times as in \cite[Proof of Proposition 3.6(a)]{huang2021tight}. 
\end{proof}
\begin{proof}[Proof of (\ref{itm:soverlap})]
    For each $u^1,u^2\in\bbL$, $\bbE \vR(\bsig(u^1),\bsig(u^2)) = \vchi(p_{u^1\wedge u^2}) = \vphi_{u^1\wedge u^2}$.
    So,
    \[
        \bbP\lt[
            \norm{\vR(\bsig(u^1),\bsig(u^2)) - \vphi_{u^1\wedge u^2}}_\infty \le \eta
        \rt]
        \ge 1-\nu.
    \]
    The result follows by a union bound on $u^1,u^2$.
\end{proof}
\begin{proof}[Proof of (\ref{itm:sogp})]
    Use the Borell-TIS inequality on the random variable $Y = \fr{1}{N} \sup_{\ubsig \in \cQ(\eta)} \cH_N(\ubsig)$, as in \cite[Proof of Proposition 3.6(d)]{huang2021tight}.
\end{proof}

\begin{proof}[Proof of Theorem~\ref{thm:main-ogp-oc}]
    Note that $\Ssolve \cap \Soverlap \cap \Sogp = \emptyset$.
    So, $\bbP(\Ssolve) + \bbP(\Soverlap) + \bbP(\Sogp) \le 2$.
    The bounds in Proposition~\ref{prop:prob-ineqs} imply
    \[
        \alpha_N^K \le 2\exp(-cN) + K^2\nu
    \]
    By adjusting the constant $c$,
    \[
        \alpha_N 
        \le 
        \exp(-cN) + \nu^c.
    \]
\end{proof}







\subsection{An Alternate Definition for the $\BOGP$ Threshold}
\label{subsec:alt-bogp}

The overlap-constrained input set $\cQ(\eta)$ used to define $\BOGP$ was designed to capture the properties of $\ubsig=(\cA(H_N^{(u)}))_{u\in\bbL}$.
In this set, overlap constraints are enforced \emph{globally}, between each pair of states, and the constraints are \emph{approximate}, within a tolerance $\eta > 0$.

In this subsection, we define a variant $\BOGP_{\loc,0}$ of $\BOGP$, based on an input set $\cQ_{\loc}(0)$, in which overlap constraints are enforced \emph{locally}, between only adjacent and sibling nodes in $\bbT$, and the constraints are \emph{exact}.
We also enforce that the sequences $p_d$, $\vphi_d$ increase in small steps.
To define the local constraints, we introduce the extended states
\[
    \ubrho = (\brho(u))_{u\in \bbT} \in \cB_N^{\bbT}
\]
whose indices now also include interior $u\in \bbT$. 
For $u,v\in \bbT$, let $u\sim v$ indicate that $u=v$, or one of $u,v$ is the parent of the other, or $u,v$ are siblings.
Define
\begin{align*}
    \cQ_{\loc+}^{k,D,\uvphi}(\eta)
    &= \lt\{
        \ubrho \in \cB_N^\bbT : 
        \norm{\vR(\brho(u), \brho(v)) - \vphi_{u\wedge v}}_\infty \le \eta,~\forall u\sim v
    \rt\} \\
    \cQ_{\loc}^{k,D,\uvphi}(\eta)
    &= \lt\{
        \ubsig \in \cB_N^\bbL : 
        \exists \ubrho \in \cQ_{\loc+}^{k,D,\uvphi}(\eta)~\text{such that}~(\brho(u))_{u\in \bbL} = \ubsig
    \rt\}.
\end{align*}
We similarly omit the superscript $k,D,\uvphi$ when this is clear from context.
The following definition captures the property that $p_d$, $\vphi_d$ increase in small steps. 
\begin{definition}
    \label{def:delta-dense}
    The pair of sequences $(\up,\uvphi)$ is \textbf{$\delta$-dense} if $p_d-p_{d-1} \le \delta$ and $\vphi_d - \vphi_{d-1} \preceq \delta \vone$ for all $d$.
\end{definition}
The following technical condition ensures continuous dependence of orthogonal bands on their centers. 
\begin{definition}
    \label{def:separated}
    The function $\vchi \in \bbI(0,1)^\sS$ is \textbf{$\delta$-separated} if $\vchi(0) \succeq \delta \vone$.
\end{definition}
Define
\begin{equation}
    \label{eq:bogp-loc}
        \BOGP_{\loc,0}
        =
        \lim_{D\to\infty}
        \lim_{k\to\infty}
        \sup_{\substack{
            \vchi \in \bbI(0,1)^\sS \\ 
            \text{$1/D^2$-separated}
        }}
        \inf_{\substack{
            \uvphi=\vchi(\up) \\ 
            \text{$6r/D$-dense}
        }}
        \limsup_{N\to\infty}
        \fr{1}{N} 
        \bbE \sup_{\ubsig \in \cQ_{\loc}(0)}
        \cH_N(\ubsig).
\end{equation}
Note that the limit in $D$ is no longer obviously decreasing, so the existence of this limit also needs to be proven. 

The following proposition, which we prove in Appendix~\ref{sec:equivalence-of-bogps}, shows that $\BOGP_{\loc,0}$ is an equivalent characterization of $\BOGP$.
This characterization will be more convenient for the proof of Proposition~\ref{prop:bogp-alg} carried out in the next section. 
We note that in the proof we define several more variants of $\BOGP$ and show all are equal, and it also follows that the average in the definition \eqref{eq:grand-hamiltonian} of $\cH_N$ can be replaced by a minimum with no change. This illustrates some flexibility in using the branching OGP.

\begin{proposition}
    \label{prop:bogp-equivalent}
    The limit $\BOGP_{\loc,0}$ exists and $\BOGP=\BOGP_{\loc,0}$.
\end{proposition}

Finally we record two useful facts.
\begin{lemma}
    \label{lem:loc0-barycenter}
    If $\ubrho \in \cQ_{\loc,+}(0)$ and $\bar \brho = \fr{1}{K} \sum_{u\in \bbL} \brho(u)$, then $\fr{1}{\sqrt{N}} \norm{\brho(\emptyset) - \bar \brho}_2 \le \sqrt{D/k}$.
\end{lemma}
\begin{proof}
    Define $\ubtau \in (\bbR^N)^\bbT$ by $\btau(u) = \brho(u)$ for $u\in \bbL$ and otherwise recursively $\btau(u) = \fr{1}{k} \sum_{i=1}^k \btau(ui)$. 
    By bilinearity of $\vR$, for all $u\in \bbT \setminus \bbL$ with $|u|=d$,
    \[
        \vR\lt(
            \brho(u) - \fr1k \sum_{i=1}^k \brho(ui),
            \brho(u) - \fr1k \sum_{i=1}^k \brho(ui)
        \rt)
        = \fr1k (\vphi_{d+1} - \vphi_d),
    \]
    so
    \[
        \fr{1}{\sqrt{N}} \norm{\brho(u) - \fr1k \sum_{i=1}^k \brho(ui)}_2 
        = \sqrt{\fr{q_{d+1}-q_d}{k}},
    \]
    where $q_d = \la \vlam, \vphi_d \ra$.
    It is easy to see by induction on $d$ that 
    \begin{align*}
        \fr{1}{\sqrt{N}} \norm{\brho(u) - \btau(u)}_2 
        &\le \fr{1}{\sqrt{N}} \norm{\brho(u) - \fr1k \sum_{i=1}^k \brho(ui)}_2 
        + \fr{1}{k} \sum_{i=1}^k \fr{1}{\sqrt{N}} \norm{\brho(ui) - \btau(ui)}_2 \\
        &\le \sum_{\ell=d}^{D-1} \sqrt{\fr{q_{\ell+1}-q_\ell}{k}}.
    \end{align*}
    Since $\bar \brho = \btau(\emptyset)$, 
    \[
        \fr{1}{\sqrt{N}} \norm{\brho(\emptyset) - \bar \brho}_2
        \le \sum_{d=0}^{D-1} \sqrt{\fr{q_{d+1}-q_d}{k}}
        \le \sqrt{\fr{D}{k}}
    \]
    by Cauchy-Schwarz.
\end{proof}

\begin{lemma}
    \label{lem:bogp-subgaussian}
    For any $S \subseteq \cB_N^{\bbL}$, $\fr{1}{N} \sup_{\ubsig \in S} \cH_N(\ubsig)$ is $O(N^{-1/2})$-subgaussian, in particular
    \[
    \bbP\lt[
    \lt|
    \sup_{\ubsig \in S} \cH_N(\ubsig)
    -
    \bbE[
    \sup_{\ubsig \in S} \cH_N(\ubsig)
    ]
    \rt|
    \geq tN^{1/2}
    \rt]
    \leq 
    Ce^{-t^2/C}
    \]
    for a constant $C$ and all $t\geq 0$.
\end{lemma}
\begin{proof}
    We calculate identically to \cite[Proof of Proposition 3.6(d)]{huang2021tight} that for any fixed $\ubsig \in \cB_N^\bbL$, $\Var \cH_N(\ubsig) = O(N)$.
    The result follows from the Borell-TIS inequality, whose statement and proof hold for noncentered Gaussian processes with no modification.
\end{proof}

\section{Branching OGP from Uniform Concentration}
\label{sec:uc}

We now turn to the proof of Proposition~\ref{prop:bogp-alg}. 
In light of Proposition~\ref{prop:bogp-equivalent}, it suffices to prove $\BOGP_{\loc,0}=\ALG$.
We begin with a very general argument that due to the ``many orthogonal increments'' property at each layer of the branching tree, it suffices to consider ``greedy'' embeddings in some sense. 
This argument is essentially elementary and relies on an idea of Subag \cite{subag2018free} applied recursively down the tree.

\subsection{Uniform Concentration}

For $\vx \in [0,1]^\sS$, define the product of spheres 
\[
    \cS_N(\vx) 
    = \lt\{
        \bsig \in \bbR^N : 
        \vR(\bsig,\bsig) = \vx
    \rt\}
    = \lt\{
        \bsig \in \bbR^N : 
        \norm{\bsig_s}_2^2 = \lambda_s x_s N
        ~\forall s\in \sS
    \rt\}.
\]
For $\bsig^0 \in \cS_N(\vx)$ and $\vy \succeq \vx$, define 
\begin{equation}
    \label{eq:band-defn}
    B(\bsig^0, \vy, k)
    =
    \lt\{\begin{array}{l}
        \ubsig =
        (\bsig^1,\bsig^2,\dots,\bsig^k) \in \cS_N(\vy)^k: 
        \\
        \vR(\bsig^i-\bsig^0,\bsig^0)
        = \vR(\bsig^i-\bsig^0,\bsig^j-\bsig^0)
        = \vzero
        \quad \forall i,j\in [k], i\neq j
    \end{array}\rt\}.
\end{equation}
Let $0\le \pminus < \pplus \le 1$. 
Generate $k+1$ i.i.d. copies $\hH_N^{[0]}, \hH_N^{[1]}, \ldots, \hH_N^{[k]}$ of $\wtH_N$ as in \eqref{eq:def-hamiltonian-no-field}.
Set 
\begin{align}
    \label{eq:def-hH(0)}
    \hH_N^{(0)}(\bsig) &= \sqrt{\pminus} \hH_N^{[0]}(\bsig) \quad \text{and} \\
    \label{eq:def-hH(i)}
    \hH_N^{(i)}(\bsig) &= \sqrt{\pminus} \hH_N^{[0]}(\bsig) + \sqrt{\pplus-\pminus} \hH_N^{[i]}(\bsig),\quad 1\le i\le k.
\end{align}
Define
\[
    F_{\pminus,\pplus}(\bsig^0,\vy, k)
    = 
    \fr{1}{kN}
    \max_{\ubsig \in B(\bsig^0, \vy, k)}
    \sum_{i=1}^k
    \lt(\hH_N^{(i)}(\bsig^i) - \hH_N^{(0)}(\bsig^0)\rt).
\]

\begin{lemma}
    \label{lem:F-lip}
    There exists $C$ such that the following holds. 
    Suppose that $\delta \vone \preceq \vx \preceq \vy \preceq \vone$ and $\bsig^0, \brho^0 \in \cS_N(\vx)$ satisfy $\norm{\bsig^0-\brho^0}_2 \le \iota\sqrt{N}$.
    If $\hH_N^{[0]}, \ldots, \hH_N^{[k]} \in K_N$ for the event $K_N$ in Proposition~\ref{prop:gradients-bounded}, then
    \begin{equation}
        \label{eq:F-lip}
        |F_{\pminus,\pplus}(\bsig^0,\vy,k) - F_{\pminus,\pplus}(\brho^0,\vy,k)|
        \leq
        \frac{C\iota}{\sqrt{\delta}}. 
    \end{equation}
\end{lemma}

\begin{proof}
    Let $T:\bbR^N\to\bbR^N$ be a product of rotation maps in the $r$ factors $\bbR^{\cI_s}$ such that $T(\bsig^0)=\brho^0$. Then
    \[
        T\lt(B(\bsig^0, \vy, k)\rt)
        = B(T(\bsig^0), \vy, k)
        = B(\brho^0, \vy, k).
    \]
    In particular, we take $T$ to be obtained using geodesic rotations from each $\bsig^0_s$\ to $\brho^0_s$.
    Thus, if $\ubsig \in B(\bsig^0,\vy,k)$ and $\ubrho = (\brho^1,\ldots,\brho^k) \in B(\brho^0,\vy,k)$ for $\brho^i = T\bsig^i$, then for all $i\in [k]$
    \[
        \fr{\norm{\brho^i-\bsig^i}_2}{\norm{\bsig^i}_2}
        \le \fr{\norm{\brho^0-\bsig^0}_2}{\norm{\bsig^0}_2}
        \le \fr{\iota}{\sqrt{\delta}},
    \]
    so $\fr{1}{\sqrt{N}} \norm{\brho^i-\bsig^i}_2 \le \iota / \sqrt{\delta}$.
    On the event $\hH_N^{[0]}, \ldots, \hH_N^{[k]} \in K_N$, it follows that
    \[
        \lt|\hH_N^{(i)}(\bsig^i) - \hH_N^{(i)}(\brho^i)\rt|
        \le \fr{C\iota}{\sqrt{\delta}}
    \]
    for $1\le i\le k$ and 
    \[
        \lt|\hH_N^{(0)}(\bsig^0) - \hH_N^{(0)}(\brho^0)\rt|
        \le C\iota,
    \]
    which implies the conclusion (after adjusting $C$).
\end{proof}

\begin{lemma}
\label{lem:unif-main}
    There exist constants $c,C>0$ such that for all $k\in \bbN$ and $\delta,\eps > 0$ the following holds. For any $\vx,\vy$ satisfying $\delta \vone \preceq\vx \preceq \vy$,
    \begin{align*}
        &\bbP\lt(
            \sup_{\bsig^0 \in \cS_N(\vx)}
            |F_{\pminus,\pplus}(\bsig^0, \vy, k) - \bbE F_{\pminus,\pplus}(\bsig^0, \vy, k)|
            \le \eps
        \rt) \\
        &\qquad 
        \ge 
        1 - 
        \exp\lt(
            C\log\lt(\frac{1}{\delta\eps}\rt) N -
            ck\eps^2 N
        \rt)
        - e^{-cN}
    \end{align*}
\end{lemma}
\begin{proof}
    Fix for now $\bsig^0 \in \cS_N(\vx)$ and $\ubsig = (\bsig^1, \ldots, \bsig^k) \in B(\bsig^0, \vy, k)$. 
    Using the definition \eqref{eq:band-defn} in the final step, we find that for small $c>0$,
    \begin{align*}
        &\bbE \lt[\lt(
            \sum_{i=1}^k
            (\hH_N^{(i)}(\bsig^i) - \hH^{(0)}(\bsig^0))
        \rt)^2\rt] \\
        &= 
        \bbE \lt[\lt(
            \sum_{i=1}^k
            \sqrt{\pminus} (\hH_N^{[0]}(\bsig^i) - \hH_N^{[0]}(\bsig^0))
            + 
            \sqrt{\pplus-\pminus} \hH_N^{[i]} (\bsig^i)
        \rt)^2\rt] \\
        &=
        \pminus 
        \sum_{i,j=1}^k
        \bbE\lt[
            (\hH_N^{[0]}(\bsig^i) - \hH_N^{[0]}(\bsig^0))
            (\hH_N^{[0]}(\bsig^j) - \hH_N^{[0]}(\bsig^0))
        \rt]
        +
        (\pplus-\pminus)
        \sum_{i=1}^k
        \bbE\lt[
            \hH_N^{[i]} (\bsig^i)^2
        \rt]
        \\
        &=
        \pminus
        \sum_{i,j=1}^k
        \xi(\vR(\bsig^i,\bsig^j))
        -
        \xi(\vR(\bsig^i,\bsig^0))
        -
        \xi(\vR(\bsig^0,\bsig^j))
        +
        \xi(\vR(\bsig^0,\bsig^0))
        +
        (\pplus-\pminus) \sum_{i=1}^k \xi(\vR(\bsig^i,\bsig^i))
        \\
        &\le
        \frac{k}{8c}.
    \end{align*}
    By the Borell-TIS inequality, for each fixed $\bsig^0 \in \cS_N(\vx)$
    \begin{equation}
        \label{eq:concentration-one-sigma}
        \bbP\lt[
            |F_{\pminus,\pplus}(\bsig^0,\vy,k) 
            -\bbE F_{\pminus,\pplus}(\bsig^0,\vy,k)|
            \le \eps/2
        \rt]
        \ge 
        1 - 
        2\exp\lt(-ck\eps^2 N\rt).
    \end{equation}
    Choose $\iota = \Theta(\eps \sqrt{\delta})$ so that the right-hand side of \eqref{eq:F-lip} is bounded by $\eps/2$, and let $\cN$ be an $\iota \sqrt{N}$-net of $\cS_N(\vx)$ with size $|\cN|\le (1/(\delta\eps))^{CN}$.
    Define the events 
    \begin{align*}
        S_{\mathrm{conc}}
        &= 
        \lt\{
            \,|F_{\pminus,\pplus}(\brho^0, \vy, k) - \bbE F_{\pminus,\pplus}(\brho^0, \vy, k)|
            \le \eps/2
            ~~\forall~
            \brho^0 \in \cN
        \rt\}, \\
        S_{\mathrm{lip}}
        &= 
        \lt\{
            \,\hH_N^{[0]},\ldots,\hH_N^{[k]} \in K_N
        \rt\},
    \end{align*}
    where $K_N$ is defined in Proposition~\ref{prop:gradients-bounded}.
    By a union bound (after adjusting $c,C$), 
    \begin{equation}
        \label{eq:concentration-many-sigma}
        \bbP\lt(
            S_{\mathrm{conc}}
            \cap
            S_{\mathrm{lip}}
        \rt)
        \ge 
        1 - 
        \exp\lt(C\log \lt(\fr{1}{\delta\eps}\rt) N - ck\eps^2 N\rt)-e^{-cN}.
    \end{equation}
    Suppose $S_{\mathrm{conc}} \cap S_{\mathrm{lip}}$ holds.
    For any $\bsig^0 \in \cS_N(\vx)$, there exists $\brho^0 \in \cN$ such that $\tnorm{\bsig^0-\brho^0}_2 \le \iota \sqrt{N}$, and so
    \begin{align*}
        &|F_{\pminus,\pplus}(\bsig^0, \vy, k) - \bbE F_{\pminus,\pplus}(\bsig^0, \vy, k)| \\
        &\le |F_{\pminus,\pplus}(\bsig^0, \vy, k) - F_{\pminus,\pplus}(\brho^0, \vy, k)| 
        + |F_{\pminus,\pplus}(\brho^0, \vy, k) - \bbE F_{\pminus,\pplus}(\brho^0, \vy, k)|
        \le \fr{\eps}{2} + \fr{\eps}{2} = \eps.
    \end{align*}
\end{proof}

For now, let $k, D, \uvphi, \up$ (recall Definition~\ref{defn:bogp}) be arbitrary.
In Proposition~\ref{prop:uc-bogp} below, we obtain an estimate for
$\fr{1}{N} \bbE \max_{\ubsig\in \cQ_{\loc}(0)}\cH_N(\ubsig)$ by applying Lemma~\ref{lem:unif-main} repeatedly at each internal vertex $u\in \bbT\backslash \bbL$.
This maximum will take the form of an abstract sum of energy increments.
In the next subsection we will take a continuum limit of this bound, which will yield the variational formula \eqref{eq:alg} for $\ALG$ and prove Proposition~\ref{prop:bogp-alg}.

Spherical symmetry implies that $\bbE F_{\pminus,\pplus}\lt(\bsig,\vy,k\rt)$ depends on $\bsig$ only through $\vR(\bsig,\bsig)$. Hence for $\vphiminus=\vR(\bsig,\bsig)$ we may define
\begin{equation}
    \label{eq:f}
    f(\vphiminus, \vphiplus; \pminus, \pplus; k)
    =
    \bbE 
    F_{\pminus,\pplus}\lt(\bsig,\vphiplus,k\rt).
\end{equation}

\begin{proposition}
    \label{prop:uc-bogp}
    Fix $D \in \bbN$ and $\eps, \delta > 0$. 
    Suppose that $\vphi_0 \succeq \delta \vone$. 
    There exists $k_0 = k_0(D,\eps,\delta)$ such that for all $k\ge k_0$, there exists $c = c(D,\eps,\delta,k)$ such that
    \[
        \bbP\lt[\lt|
            \fr1N
            \sup_{\ubsig \in \cQ_{\loc}(0)}
            \cH_N(\ubsig)
            - \lt(
                \sum_{s\in \sS}  h_s\lambda_s \sqrt{\phi_0^s} +
                \sum_{d=0}^{D-1}
                f\lt(\vphi_d, \vphi_{d+1}; p_d, p_{d+1}; k\rt)
            \rt)
        \rt|\le 2D\eps\rt]
        \ge
        1-e^{-cN}.
    \]
\end{proposition}

\begin{proof}
    Let $C,c$ be as in Lemma~\ref{lem:unif-main}, and $k_0$ large enough that
    \begin{align}
        \label{eq:uc-exp-rate}
        C \log \lt(\fr{1}{\delta\eps}\rt) - ck_0\eps^2
        &\le -c, \\
        \label{eq:uc-barycenter-close}
        \tnorm{\vh}_\infty / \sqrt{k_0} &\le \eps.
    \end{align}
    Recall the construction of $\wtHNp{u}$ from \eqref{eq:def-correlated-disorder}.
    For any $u\in V_d$, $0\le d\le D-1$, let $\cE_u$ denote the event in Lemma~\ref{lem:unif-main}, with $(\pminus,\pplus) = (p_d, p_{d+1})$, $(\vx,\vy) = (\vphi_d,\vphi_{d+1})$, and
    \begin{equation}
        \label{eq:uc-hamiltonian-choice}
        (\hH_N^{(0)},\hH_N^{(1)},\ldots,\hH_N^{(k)})
        = \lt(\wtHNp{u},\wtHNp{u1},\ldots,\wtHNp{uk}\rt).
    \end{equation}
    Let $\cE = \bigcap_{u\in \bbT \setminus \bbL} \cE_u$. 
    Lemma~\ref{lem:unif-main} and equation \eqref{eq:uc-exp-rate} imply $\bbP(\cE^u) \ge 1-2e^{-cN}$ for all $u\in \bbL$. 
    By a union bound, $\bbP(\cE) \ge 1 - e^{-cN}$ (after adjusting $c$).
    
    Denote by $F^u_{p_d,p_{d+1}}$ the function $F_{p_d,p_{d+1}}$ defined with Hamiltonians \eqref{eq:uc-hamiltonian-choice}.
    Let $\ubsig \in \cQ_{\loc}(0)$, so there exists $\ubrho \in \cQ_{\loc+}(0)$ with $(\brho(u))_{u\in \bbL} = \ubsig$. 
    On the event $\cE$,
    \begin{align*}
        \fr{1}{N} \cH_N(\ubsig) 
        - \fr{1}{KN} \sum_{v\in \bbL} \la \bh, \bsig(u) \ra
        &=
        \fr{1}{KN}
        \sum_{u\in \bbL} \wtHNp{u}(\bsig(u)) \\
        &=
        \sum_{d=0}^{D-1}
        \fr{1}{k^{d}}
        \sum_{u\in V_d} 
        \fr{1}{kN}
        \sum_{i=1}^{k}
        \lt( 
            \wtHNp{ui}(\brho(ui))-\wtHNp^{u}(\brho(u))
        \rt) \\
        &\le
        \sum_{d=0}^{D-1}
        \fr{1}{k^{d}}
        \sum_{u\in V_d} 
        F_{p_d,p_{d+1}}^{u}
        \lt(\brho(u),\vphi_{d+1},k\rt) \\
        &\stackrel{Lem.~\ref{lem:unif-main}}{\le}
        D\eps
        + \sum_{d=0}^{D-1}
        f(\vphi_d, \vphi_{d+1}; p_d, p_{d+1}; k).
    \end{align*}
    In the telescoping sum, we used that $\wtH_N^{(\emptyset)}$ is the zero function. 
    By Lemma~\ref{lem:loc0-barycenter} and equation \eqref{eq:uc-barycenter-close},
    \begin{align*}
        \lt|
            \fr{1}{KN} \sum_{v\in \bbL} \la \bh, \bsig(u) \ra
            - \fr{1}{N} \la \bh, \brho(\emptyset) \ra
        \rt|
        &\le 
        \fr{1}{\sqrt{N}} \tnorm{\bh}_2 \cdot 
        \fr{1}{\sqrt{N}} \norm{\brho(\emptyset) - \fr{1}{K} \sum_{u\in \bbL} \bsig(u)}_2 \\
        &\le 
        \tnorm{\vh}_\infty \sqrt{\fr{D}{k}} 
        \le D\eps.
    \end{align*}
    Finally,
    \begin{equation}
        \label{eq:root-energy}
        \fr{1}{N} \la \bh, \brho(\emptyset) \ra
        = \fr{1}{N} \sum_{s\in \sS} h_s \norm{\brho(\emptyset)_s}_1 
        \le \fr{1}{N} \sum_{s\in \sS} h_s \sqrt{|\cI_s|} \norm{\brho(\emptyset)_s}_2 
        = \sum_{s\in \sS}  h_s \lambda_s \sqrt{\phi_0^s}.
    \end{equation}
    This completes the proof of the upper bound for $\fr1N \sup_{\ubsig\in \cQ_{\loc}(0)}\cH_N(\ubsig)$. 
    Finally, observe that equality holds above (up to the same $2D\eps$ error) if we choose $\brho(\emptyset) = \sqrt{\vphi_0} \diamond \bone$ and then recursively choose $(\brho(ui))_{i\in[k]}$ given $\brho(u)$ so that, for $|u|=d$, 
    \[
        \fr{1}{Nk}
        \sum_{i=1}^k \lt(\wtHNp{ui}(\brho(ui))-\wtHNp{u}(\brho(u))\rt)
        =
        F^u_{p_d,p_{d+1}}(\brho(u),\vphi_{d+1},k).
    \] 
\end{proof}

\subsection{The Algorithmic Functional}

Our next objective is to estimate the terms $f(\vphiminus, \vphiplus; \pminus, \pplus; k)$ appearing in Proposition~\ref{prop:uc-bogp}. 
The key point is that when the differences $\vphiplus-\vphiminus$ and $\pplus-\pminus$ are small, which is ensured by $\delta$-denseness of $(\up, \uvphi)$, this estimate only requires Taylor approximating the relevant Hamiltonians to second order. 
We take advantage of this using the following lemma, which (for $k=1$) gives the ground state energy $\bGS(W, \vv, 1)$ of a quadratic multi-species spin glass with Gaussian external field.
For general $k$, this lemma gives the limiting ground state energy $\bGS(W, \vv, k)$ of a $k$-replica Hamiltonian \eqref{eq:k-replica-hamiltonian} with shared quadratic component $W \diamond \bG$ and independent external fields $\vv \diamond \bg^i$, whose inputs \eqref{eq:bbtperp} are $k$ pairwise orthogonal elements of $\cS_N(\vone)$.
Note that $\bGS(W,\vv,k) \le \bGS(W,\vv,1)$ by definition. 
In fact equality holds, i.e. there exist orthogonal $\bsig^1, \ldots, \bsig^k$ such that each $\bsig^i$ approximately maximizes $H_{N,k}^i(\bsig^i)$.
We prove this lemma in Appendix~\ref{sec:sk-ext-field} by combining a known formula for the case $(k,\vv)=(1,\vzero)$ with an elementary recursive argument along subspaces.

\begin{lemma}
    \label{lem:sk-ext-field}
    Let $W = (w_{s,s'})_{s,s\in \sS} \in \bbR_{\ge 0}^{\sS\times \sS}$ be symmetric and $\vv = (v_s)_{s\in \sS} \in \bbR_{\ge 0}^\sS$.
    Let $k\in \bbN$ and sample independent $\bg^1,\ldots,\bg^k \in \bbR^N$ and $\bG \in \bbR^{N\times N}$ with i.i.d. standard Gaussian entries. 
    Consider the $k$-replica Hamiltonian
    \begin{equation}
        \label{eq:k-replica-hamiltonian}
        H_{N,k}(\ubsig)
        = 
        \fr{1}{k}
        \sum_{i=1}^k
        H_{N,k}^i(\bsig^i),
        \qquad
        H_{N,k}^i(\bsig^i)
        =
        \la \vv \diamond \bg^i, \bsig^i\ra 
        + 
        \fr{1}{\sqrt{N}}
        \la W \diamond \bG, (\bsig^i)^{\otimes 2} \ra
    \end{equation}
    on the input space of orthogonal replicas
    \begin{equation}
        \label{eq:bbtperp}
        \cS_N^{k, \perp}
        = 
        \lt\{
            \ubsig = (\bsig^1, \ldots, \bsig^k) \in \cS_N(\vone):
            \vR(\bsig^i,\bsig^j) = \vzero 
            ~\forall i\neq j
        \rt\}.
    \end{equation}
    Define the $k$-replica ground state energy 
    \begin{equation}
        \label{eq:sk-gsn}
        \GS_N(W,\vv,k)
        \equiv 
        \fr{1}{N}
        \max_{\ubsig \in \cS_N^{k, \perp}}
        H_{N,k}(\ubsig).
    \end{equation}
    Then $\bGS(W,\vv,k) \equiv \lim_{N\to\infty} \bbE \GS_N(W,\vv,k)$ exists, does not depend on $k$, and is given by
    \[
        \bGS(W,\vv,k) 
        = 
        \sum_{s\in \sS}
        \lambda_s
        \sqrt{v_s^2 + 2 \sum_{s'\in \sS} \lambda_s w_{s,s'}^2 }.
    \]
\end{lemma}

\begin{proposition}
    \label{prop:what-F-is}
    Suppose $0\le \pminus \le \pplus \le 1$, $\vzero \preceq \vphiminus \preceq \vphiplus \preceq \vone$ and
    \begin{equation}
        \label{eq:delta-discrete}
        \pplus - \pminus \le \delta,
        \qquad
        \vphiplus - \vphiminus \preceq \delta \vone.
    \end{equation}
    Then,
    \begin{align*}
        f(\vphiminus,\vphiplus; \pminus,\pplus; k)
        &=
        \sum_{s\in\sS}
        \lambda_s
        \sqrt{(\vphiplus^s - \vphiminus^s)\lt((\pplus-\pminus) \xi^s(\vphiminus) + \pminus \sum_{s'\in \sS} \partial_{x_{s'}} \xi^s(\vphiminus) (\vphiplus^s - \vphiminus^s)\rt)}
        \\
        &\qquad + 
        O\big(\delta^{3/2} + (\delta/k)^{1/2}\big)+o_N(1),
    \end{align*}
    where $o_N(1)$ denotes a term tending to $0$ as $N\to\infty$.
\end{proposition}

\begin{proof}
    Fix $\bsig^0$ such that $\vR(\bsig^0, \bsig^0) = \vphiminus$. 
    Let $\ubsig = (\bsig^1,\ldots,\bsig^k) \in B(\bsig^0,\vphiplus,k)$. 
    Let $\Delta \vphi = \vphiplus - \vphiminus$ and $\ubx = (\bx^1,\ldots,\bx^k)$ for $\bx^i = (\Delta \vphi)^{-1/2} \diamond (\bsig^i - \bsig^0)$.
    Define
    \begin{align*}
        \cS_{\bullet} &= \lt\{
            \by \in \cS_N(\vone) : \vR(\by,\bsig^0) = \vzero
        \rt\}, \\
        \cS_\bullet^{k,\perp} &= \lt\{
            \uby = (\by^1,\ldots,\by^k) \in \cS_{\bullet}^k : 
            \vR(\by^i,\by^j) = \vzero ~\forall i\neq j
        \rt\}.
    \end{align*}
    Note that $\ubx \in \cS_\bullet^{k,\perp}$. 
    Recall that $\hH^{[0]}_N,\ldots,\hH^{[k]}_N$ are i.i.d. copies of $\wtH_N$, and that $\hH^{(0)}_N,\ldots,\hH^{(k)}_N$ are defined by \eqref{eq:def-hH(0)}, \eqref{eq:def-hH(i)}.
    Let
    \[
        \oH^i_N(\bx^i) 
        = \hH_N^{[i]}(\bsig^i) - \hH_N^{[i]}(\bsig^0) 
        = \hH_N^{[i]}\lt(\bsig^0 + \sqrt{\Delta \vphi} \diamond \bx^i\rt) - \hH_N^{[i]}(\bsig^0).
    \]
    Then
    \begin{align}
        \notag
        f(\vphiminus,\vphiplus; \pminus,\pplus; k)
        &= 
        \fr{1}{kN}
        \bbE 
        \max_{\ubsig \in B(\bsig^0,\vphiplus,k)}
        \sum_{i=1}^k 
        \lt(\hH^{(i)}_N(\bsig^i) - \hH^{(0)}_N(\bsig^0)\rt) \\
        \notag
        &= 
        \fr{1}{kN}
        \bbE 
        \max_{\ubsig \in B(\bsig^0,\vphiplus,k)}
        \sum_{i=1}^k 
        \bigg(
            \sqrt{\pminus} \lt(\hH_N^{[0]}(\bsig^i) - \hH_N^{[0]}(\bsig^0)\rt) \\
            \notag
            &\qquad 
            + \sqrt{\pplus-\pminus} \lt(\hH_N^{[i]}(\bsig^i) - \hH_N^{[i]}(\bsig^0)\rt) 
            + \sqrt{\pplus-\pminus}\,
            \hH_N^{[i]}(\bsig^0)
        \bigg) \\
        \label{eq:f-expansion-into-hamiltonians}
        &= 
        \fr{1}{kN}
        \bbE 
        \max_{\ubx \in \cS_\bullet^{k,\perp}}
        \sum_{i=1}^k 
        \lt(
            \sqrt{\pminus}\, \oH^0 (\bx^i) + \sqrt{\pplus - \pminus}\, \oH^i (\bx^i)
        \rt) 
    \end{align}
    where we note that $\bbE \hH_N^{[i]}(\bsig^0) = 0$. 
    Let $\oH^{i,\tay}_N$ denote the degree $2$ Taylor expansion of $\oH_N^i$ around $\bzero$. 
    By Proposition~\ref{prop:gradients-bounded} (recalling \eqref{eq:delta-discrete}),
    \[
        \bbE \sup_{\bx \in \cS_{\bullet}} |\oH^i(\bx)_N - \oH^{i,\tay}_N(\bx)| = O(N\delta^{3/2}).
    \]
    So, for all $0\le i\le k$, we have as processes on $\cS_{\bullet}$
    \begin{equation}
        \label{eq:oH-taylor-expansion}
        \oH^i_N(\bx) =_d \la \vv \diamond \bg^i, \bx\ra
        + 
        \la W \diamond \bG^i, \bx^{\otimes 2}\ra
        + O_{\bbP}(N\delta^{3/2}),
    \end{equation}
    where $O_{\bbP}(N\delta^{3/2})$ denotes a $\cS_{\bullet}$-valued process $X(\bx)$ with $\bbE \sup_{\bx \in \cS_{\bullet}} |X(\bx)| = O(N\delta^{3/2})$ and $\vv = (v_s)_{s\in \sS}$ and $W = (w_{s,s'})_{s,s'\in \sS}$ are given by
    \[
        v_s = \sqrt{\xi^s(\vphiminus) (\Delta \vphi)^s }, 
        \qquad
        w_{s,s'} = \fr{1}{\sqrt{2}} \sqrt{\lambda_{s'}^{-1} \partial_{x_{s'}} \xi^s(\vphiminus) (\Delta \vphi)^s (\Delta \vphi)^{s'}}.
    \]
    Next we observe some simplifications. 
    Because $\Delta \vphi \preceq \delta \vone$, we have $v_s = O(\delta^{1/2})$, $w_{s,s'} = O(\delta)$ uniformly over $s,s'$.
    The linear contribution to $\oH^0_N$ in \eqref{eq:oH-taylor-expansion} is small because
    \[
        \fr{1}{kN} \sum_{i=1}^k \la \vv \diamond \bg^0, \bx^i\ra
        \le \fr{1}{kN} \norm{\vv \diamond \bg^0}_2 \norm{\sum_{i=1}^k \bx^i}_2
        = O_{\bbP}((\delta/k)^{1/2})
    \]
    by orthogonality of the $\bx^i$.
    Because $\pplus - \pminus \le \delta$, the quadratic contributions to $\oH^i_N$ for $i\ge 1$ are also small:
    \[
        \fr{\sqrt{\pplus-\pminus}}{N}
        \la W \diamond \bG^i, (\bx^i)^{\otimes 2}\ra
        = O_{\bbP}(\delta^{3/2}).
    \]
    Combining these estimates with \eqref{eq:f-expansion-into-hamiltonians} and \eqref{eq:oH-taylor-expansion}, we find
    \begin{align*}
        f(\vphiminus,\vphiplus; \pminus,\pplus; k)
        &=
        \fr{1}{kN}
        \bbE \max_{\ubx \in \cS_\bullet^{k,\perp}}
        \sum_{i=1}^k 
        \sqrt{\pplus - \pminus} \lt\la \vv \diamond \bg^i, \bx^i\rt\ra +
        \sqrt{\pminus} \lt\la W \diamond \bG^0, (\bx^i)^{\otimes 2} \rt\ra \\
        &\qquad + O((\delta/k)^{1/2} + \delta^{3/2}).
    \end{align*}
    By Lemma~\ref{lem:sk-ext-field} (applied in dimension $N-r$ due to the linear constraint $\vR(\bx^i,\bsig^0)=\vzero$ in $\cS_{\bullet}$), this remaining expectation is given up to $o_N(1)$ error by
    \begin{align*}
        &\sum_{s\in \sS}
        \lambda_s
        \sqrt{(\pplus - \pminus) v_s^2 + 2\sum_{s'\in \sS} \lambda_{s'} \pminus w_{s,s'}^2} \\
        &\quad = 
        \sum_{s\in \sS}
        \lambda_s
        \sqrt{(\Delta \vphi)^s \lt((\pplus - \pminus) \xi^s(\vphiminus) + \pminus \sum_{s'\in \sS} \partial_{x_{s'}} \xi^s(\vphiminus) (\Delta \vphi)^{s'}\rt)}.
    \end{align*}
    This implies the result. 
\end{proof}

We now evaluate $\BOGP_{\loc,0}$ by taking a continuous limit of Propositions~\ref{prop:uc-bogp} and \ref{prop:what-F-is}. 
Fix $D,k$ and $\delta = 6r/D$, and let $(\up,\uvphi)$ be $\delta$-dense. 
We parametrize time by $q_d = \la \vlam, \vphi_d\ra$, so in particular $q_0 = \la \vlam, \vphi_0\ra$.
Let the functions $\tp:[q_0,1]\to [0,1]$ and $\tPhi:[q_0,1]\to [0,1]^{\sS}$ satisfy 
\begin{equation}
    \label{eq:continuous-def}
    \tp(q_d) = p_d,
    \qquad
    \tPhi(q_d) = \vphi_d.
\end{equation}
and be linear on each interval $[q_d,q_{d+1}]$. 
These are piecewise linear approximations of inputs $(p,\Phi)$ to the algorithmic functional $\bbA$.
Define
\begin{equation}
    \label{eq:def-Ads}
    A_d^s = 
    \sqrt{(\phi^s_{d+1} - \phi^s_d)\lt(
        (p_{d+1}-p_d) \xi^s(\vphi_d) + 
        p_d \sum_{s'\in\sS}\partial_{x_{s'}}\xi^s(\vphi_d)(\phi_{d+1}^{s'}-\phi_d^{s'})
    \rt)}.
\end{equation}
This term appears in the estimate of $f\lt(\vphi_d,\vphi_{d+1};p_{d+1},p_d;k\rt)$ obtained from Proposition~\ref{prop:what-F-is}.
\begin{lemma}
    \label{lem:ALG-derivation}
    We have
    \begin{align*}
        \lt|
            \sum_{d=0}^{D-1} A_d^s - 
            \int_{q_0}^{q_D}
            \sqrt{\tPhi_s'(q)(\tp\times \xi^s\circ\tPhi)'(q)}
            ~\de q
        \rt|
        \le CD^{-1/2}
    \end{align*}
    for a constant $C>0$ independent of $D,\up,\uvphi$.
\end{lemma}
\begin{proof}
Until the end, we focus on estimating the difference
\[
    \Delta_d^s \equiv 
    \lt|
        A_d^s - 
        \int_{q_d}^{q_{d+1}}
        \sqrt{\tPhi_s'(q)(\tp\times \xi^s\circ\tPhi)'(q)}
        ~\de q
    \rt|.
\]
Note the general inequality
\begin{align}
    \notag
    \int_{q_d}^{q_{d+1}}\sqrt{a(q)}\cdot |\sqrt{b(q)}-\sqrt{c(q)}| \de q
    &\leq
    \lt(\int_{q_d}^{q_{d+1}}a(q)\de q\rt)^{1/2}
    \cdot
    \lt(\int_{q_d}^{q_{d+1}}\big(\sqrt{b(q)}-\sqrt{c(q)}\big)^2\de q\rt)^{1/2} \\
    \label{eq:integral-bound}
    &\leq
    \lt(\int_{q_d}^{q_{d+1}}a(q)\de q\rt)^{1/2}
    \cdot
    \lt(\int_{q_d}^{q_{d+1}}|b(q)-c(q)|\de q\rt)^{1/2}.
\end{align}
Thus
\begin{align*}
    \Delta^s_d
    &=
    \lt|
        \int_{q_d}^{q_{d+1}}
        \sqrt{\tPhi_s'(q)}
        \lt(
            \sqrt{(\tp\times \xi^s\circ\tPhi)'(q)}
            -
            \sqrt{\fr{
                (p_{d+1}-p_d) \xi^s(\vphi_d) + 
                p_d \sum_{s'\in\sS}
                \partial_{x_{s'}}\xi^s(\vphi_d)
                (\phi_{d+1}^{s'}-\phi_d^{s'})
            }{q_{d+1}-q_d}}
        \rt)
        ~\de q
    \rt| \\
    &\le
    \sqrt{
        (\phi_{d+1}^s-\phi_d^s)
        \int_{q_d}^{q_{d+1}}
        \lt|
            (\tp\times \xi^s\circ\tPhi)'(q)
            -
            \fr{
                (p_{d+1}-p_d) \xi^s(\vphi_d) + 
                p_d \sum_{s'\in\sS}
                \partial_{x_{s'}}\xi^s(\vphi_d)
                (\phi_{d+1}^{s'}-\phi_d^{s'})
            }{q_{d+1}-q_d}
        \rt|
        \de q
    }.
\end{align*}
In the first step we used that $\tPhi'(q)=(\vphi_{d+1}-\vphi_d) / (q_{d+1}-q_d)$ by definition, and in the second we used \eqref{eq:integral-bound}.
Let $(\tp\times \xi^s\circ\tPhi)'(q_d)$ and $(\tp\times \xi^s\circ\tPhi)'(q_{d+1})$ denote the right and left derivatives at these points, respectively.
The definitions of $\tp'$ and $\tPhi'$ imply
\[
    \fr{
        (p_{d+1}-p_d) \xi^s(\vphi_d) + 
        p_d \sum_{s'\in\sS}
        \partial_{x_{s'}}\xi^s(\vphi_d)
        (\phi_{d+1}^{s'}-\phi_d^{s'})
    }{q_{d+1}-q_d}
    = (\tp\times \xi^s\circ\tPhi)'(q_d),
\]
so in fact
\begin{align}
    \notag
    \Delta^s_d
    &\le 
    \sqrt{
        (\phi_{d+1}^s-\phi_d^s)
        \int_{q_d}^{q_{d+1}}
        \lt|
            (\tp\times \xi^s\circ\tPhi)'(q)
            - (\tp\times \xi^s\circ\tPhi)'(q_d)
        \rt|
        \de q
    } \\
    \label{eq:one-step-integral-estimation}
    &\le
    \sqrt{
        (\phi_{d+1}^s-\phi_d^s)
        (q_{d+1}-q_d)
        \lt(
            (\tp\times \xi^s\circ\tPhi)'(q_{d+1})
            - (\tp\times \xi^s\circ\tPhi)'(q_d)
        \rt)
    } .
\end{align}
Let $\nabla \vphi_d = (\vphi_{d+1}-\vphi_d) / (q_{d+1}-q_d)$ be the constant value of $\nabla \tPhi$ on $[q_d,q_{d+1}]$.
Then 
\begin{align*}
    (\tp\times \xi^s\circ\tPhi)'(q_{d+1})-(\tp\times \xi^s\circ\tPhi)'(q_d)
    &=
    \lt(\fr{p_{d+1}-p_d}{q_{d+1}-q_d}\rt)
    \lt(\xi^s(\vphi_{d+1})-\xi^s(\vphi_d)\rt) \\
    &\qquad
    + p_{d+1} \la \nabla \xi^s(\vphi_{d+1}),\nabla \vphi_d \ra
    - p_{d} \la \nabla \xi^s(\vphi_d),\nabla \vphi_d \ra.
\end{align*}
We thus obtain
\begin{align*}
    &(q_{d+1}-q_d)\lt((\tp\times \xi^s\circ\tPhi)'(q_{d+1})-(\tp\times \xi^s\circ\tPhi)'(q_d)\rt) \\
    &= (p_{d+1}-p_d) \lt(\xi^s(\vphi_{d+1})-\xi^s(\vphi_d)\rt) 
    + (q_{d+1}-q_d) (p_{d+1}-p_d) \la \nabla \xi^s(\vphi_{d+1}),\nabla \vphi_d \ra \\
    &\qquad + (q_{d+1}-q_d) p_d 
    \la \nabla \xi^s(\vphi_{d+1}) -  \nabla \xi^s(\vphi_d),\nabla \vphi_d \ra \\
    &\le O\lt(
        (p_{d+1}-p_d) \norm{\vphi_{d+1}-\vphi_d}_2 
        + \norm{\vphi_{d+1}-\vphi_d}_2^2
    \rt) = O(\delta^2).
\end{align*}
Combining with \eqref{eq:one-step-integral-estimation} gives the estimate $\Delta^s_d = O(\delta) \sqrt{\phi_{d+1}^s - \phi_d^s}$.
Summing this over $0\le d\le D-1$ gives the final estimate
\begin{align*}
    \lt|
        \sum_{d=0}^{D-1}
        A_d^s -
        \int_{q_0}^{q_D}
        \sqrt{\tPhi_s'(q)(\tp\times \xi^s\circ\tPhi)'(q)}
        ~\de q
    \rt|
    &\le \sum_{d=0}^{D-1} \Delta^s_d
    \le O(\delta) \sum_{d=0}^{D-1} \sqrt{\phi_{d+1}^s-\phi_d^s} \\
    &\le O(\delta \sqrt{D}) = O(D^{-1/2}).
\end{align*}
by Cauchy-Schwarz.
\end{proof}

We next show that discretizing any $C^1$ functions $(p,\Phi)$ preserves the value of $\bbA$. 


\begin{lemma}
    \label{lem:ALG-from-continuum}
    Suppose $q_0\in [0,1]$, $p\in \bbI(q_0,1)$, and $\Phi \in \Adm(q_0,1)$.
    Consider any $\uq = (q_0,\ldots,q_D)$ with $q_0<\cdots<q_D=1$, such that the $(\up, \uvphi)$ defined by $p_d = p(q_d)$ and $\vphi_d = \Phi(q_d)$ is $6r/D$-dense. 
    Then, for all $s\in \sS$,
    \[
        \lt|
            \int_{q_0}^1 \sqrt{\Phi'_s(q) (p\times \xi^s \circ \Phi)'(q)}~\de q 
            - \int_{q_0}^1 \sqrt{\tPhi'_s(q) (\tp\times \xi^s \circ \tPhi)'(q)}~\de q 
        \rt|
        = o_D(1),
    \]
    where $\tp, \tPhi$ are the piecewise linear interpolations defined by \eqref{eq:continuous-def} and $o_D(1)$ is a term tending to $0$ as $D\to\infty$ (for fixed $(p,\Phi)$).
\end{lemma}
\begin{proof}
    The functions $\Phi,\Phi',p,p'$ are uniformly continuous because they are continuous on $[q_0,1]$.
    So,
    \[
        \tnorm{\Phi-\tPhi}_\infty,
        \tnorm{\Phi'-\tPhi'}_\infty,
        \tnorm{p-\tp}_\infty,
        \tnorm{p'-\tp'}_\infty
        = o_D(1).
    \]
    Bounded convergence implies the result. 
\end{proof}


\begin{proof}[Proof of Proposition~\ref{prop:bogp-alg}]
    By Proposition~\ref{prop:bogp-equivalent} it suffices to prove that $\BOGP_{\loc,0}=\ALG$. 
    We will separately show $\BOGP_{\loc,0} \le \ALG$ and $\BOGP_{\loc,0} \ge \ALG$.

    We first show $\BOGP_{\loc,0} \le \ALG$. 
    Let $\iota > 0$.
    Let $D$ be sufficiently large, $\eps = D^{-2}$ and $\delta = 6r/D$, and $k$ be sufficiently large depending on $D$ such that the following holds. 
    First, $k \ge k_0 (D,\eps,\delta)$ for $k_0$ defined in Proposition~\ref{prop:uc-bogp}. 
    Second, for some $1/D^2$-separated $\vchi \in \bbI(0,1)^\sS$ and all $\delta$-dense $(\up,\uvphi)$ with $\uvphi = \vchi(\up)$, we have
    \[
        \fr1N \bbE \sup_{\ubsig \in \cQ_\loc(0)} \cH_N(\ubsig)
        \ge \BOGP_{\loc,0} - \iota/2.
    \]
    Let $q_d = \la \vlam, \vphi_d\ra$.
    Let $\tp,\tPhi$ be the piecewise linear interpolations defined by \eqref{eq:continuous-def} and $A_d^s$ be defined by \eqref{eq:def-Ads}.
    Then 
    \begin{align*}
        &\lt|
            \fr1N \sup_{\ubsig \in \cQ_{\loc}(0)} \cH_N(\ubsig) 
            - \sum_{s\in \sS} \lt(
                h_s \lambda_s \sqrt{\tPhi_s(q_0)}
                + \lambda_s \int_{q_0}^{q_D}
                \sqrt{\tPhi'_s(q) (p\times \xi^s \circ \Phi)(q)}
                ~\de q
            \rt)
        \rt| \\
        &\le \lt|
            \fr1N \sup_{\ubsig \in \cQ_{\loc}(0)} \cH_N(\ubsig) 
            - \sum_{s\in \sS} \lt(
                h_s \lambda_s \sqrt{\phi_0^s}
                + \sum_{d=0}^{D-1} 
                f\lt(\vphi_d,\vphi_{d+1};p_d,p_{d+1};k\rt)
            \rt)
        \rt| \\
        &\quad + 
        \sum_{d=0}^{D-1} 
        \lt|
            f\lt(\vphi_d,\vphi_{d+1};p_d,p_{d+1};k\rt)
            - \sum_{s\in \sS} \lambda_s A_d^s
        \rt|
        + \sum_{s\in \sS} \lambda_s \lt|
            \sum_{d=0}^{D-1} A_d^s 
            - \int_{q_0}^{q_D}
            \sqrt{\tPhi'_s(q) (p\times \xi^s \circ \Phi)(q)}
            ~\de q
        \rt|.
    \end{align*}
    By Propositions~\ref{prop:uc-bogp}, \ref{prop:what-F-is} and Lemma~\ref{lem:ALG-derivation}, on an event with probability $1-e^{-cN}$ this is bounded by
    \[
        2D\eps + O(D\delta^{3/2} + D(\delta/k)^{1/2}) + O(D^{-1/2}) + o_N(1)
        = O(D^{-1/2} + (D/k)^{1/2}) + o_N(1)
        \le \iota/4,
    \]
    for sufficiently large $N,D,k$.
    Because $\fr1N \sup_{\ubsig \in \cQ_{\loc}(0)} \cH_N(\ubsig) $ is subgaussian with fluctuations $O(N^{-1/2})$ by Lemma~\ref{lem:bogp-subgaussian}, the contributions of the complement of this event are $o_N(1)$, and so
    \begin{equation}
        \label{eq:bogp-final-estimate}
        \lt|
            \fr1N \bbE \sup_{\ubsig \in \cQ_{\loc}(0)} \cH_N(\ubsig) 
            - \sum_{s\in \sS} \lt(
                h_s \lambda_s \sqrt{\tPhi_s(q_0)}
                + \lambda_s \int_{q_0}^{q_D}
                \sqrt{\tPhi'_s(q) (p\times \xi^s \circ \Phi)(q)}
                ~\de q
            \rt)
        \rt|
        \le \iota/4.
    \end{equation}
    Let $p \in \bbI(q_0,1)$ and $\Phi \in \Adm(q_0,1)$ approximate the piecewise linear functions $(\tp,\tPhi)$ on $[q_0,q_D]$, in the sense that
    \begin{equation}
        \label{eq:approximate-piecewise-linear}
        \lt|
            \sum_{s\in \sS}
            \lambda_s
            \int_{q_0}^{q_D}
            \lt(\sqrt{\tPhi_s'(q)(\tp\times \xi^s\circ\tPhi)'(q)}
            - \sqrt{\Phi_s'(q)(p\times \xi^s\circ\Phi)'(q)}\rt)
            ~\de q
        \rt|
        \le \iota/4.
    \end{equation}
    It is clear that such $p,\Phi$ exist. 
    Thus
    \[
        \BOGP_{\loc,0} \le \bbA(p,\Phi;q_0) - \iota \le \ALG - \iota.
    \]
    Since $\iota$ was arbitrary, we conclude $\BOGP_{\loc,0} \le \ALG$. 

    Next, we will show $\BOGP_{\loc,0} \ge \ALG$.
    Let $\iota > 0$, and let $D$ be sufficiently large and $k$ be sufficiently large depending on $D$.
    There exist $q_0\in [0,1]$, $p\in \bbI(q_0,1)$, and $\Phi \in \Adm(q_0,1)$ such that 
    \[
        \bbA(p,\Phi;q_0) \ge \ALG - \iota/2.
    \]
    By replacing $\Phi$ with $(1-D^{-2})\Phi + D^{-2} \vone$ we may assume $\Phi(q_0) \succeq \vone/D^2$, as this replacement affects the left-hand side by $o_D(1)$.
    Similarly, by replacing $p(q)$ with $(1-D^{-1})p(q) + D^{-1}q$, we may assume $p$ is strictly increasing.
    We choose $\vchi = \Phi \circ p^{-1}$, which is $1/D^2$-separated.
    Consider any $\uq = (q_0,q_1,\ldots,q_D)$ with $q_0<q_1<\cdots<q_D=1$ such that for $p_d = p(q_d)$, $\vphi_d = \Phi(q_d)$, the pair $(\up,\uvphi)$ is $6r/D$-dense.
    Similarly to above, we have \eqref{eq:bogp-final-estimate} for sufficiently large $N,D,k$.
    By Lemma~\ref{lem:ALG-from-continuum}, \eqref{eq:approximate-piecewise-linear} holds for $D$ sufficiently large. 
    This implies 
    \[
        \bbA(p,\Phi;q_0) \le \BOGP_{\loc,0}+\iota/2,
    \]
    and so $\ALG \le \BOGP_{\loc,0} + \iota$. 
    Because $\iota$ was arbitrary, we have $\ALG \le \BOGP_{\loc,0}$. 
\end{proof}

\section{Optimization of the Algorithmic Variational Principle}
\label{sec:alg}

In this section we will prove Propositions~\ref{prop:root-finding-trajectory} and \ref{prop:tree-descending-trajectory} and Theorem~\ref{thm:alg-optimizer}. 
Throughout this section we assume Assumption~\ref{as:nondegenerate} except where stated. 


To ensure a priori existence of a maximizer in \eqref{eq:alg}, we work in the following compact space which removes the constraint that $p$ and $\Phi$ are continuously differentiable.


\begin{definition}
\label{defn:cM}
    The space $\cM$ consists of all triples $(p,\Phi,q_0)$ such that:
    \begin{itemize}
    \item $q_0\in [0,1]$.
    \item $p:[q_0,1]\to [0,1]$ is non-decreasing and right-continuous (we write $p\in \hbbI(q_0,1)$).
    \item $\Phi=(\Phi_s)_{s\in\sS}$ consists of $r$ non-decreasing functions $\Phi_s:[q_0,1]\to [0,1]$ satisfying admissibility \eqref{eq:admissible} (we write $\Phi\in \hAdm(q_0,1)$).
    \end{itemize}
\end{definition}



Because we assume almost no regularity for elements of $\cM$, we formally define the integral in \eqref{eq:alg-functional} as follows.
Since $(p\times \xi^s \circ \Phi)$ is a bounded increasing function, it has a positive measure valued distributional derivative 
\begin{equation}
    \label{eq:careful-alg}
    (p\times \xi^s \circ \Phi)'(q) = f(q)~\de q + \de \mu(q)
\end{equation}
where $f\in L^1([q_0,1])$ and $\mu$ is an atomic-plus-singular measure supported in $[q_0,1]$.
Moreover, \eqref{eq:admissible} implies $\Phi_s$ is $\lambda_s^{-1}$-Lipschitz, hence has distributional derivative $\Phi'_s \in L^\infty([q_0,1])$.


\begin{definition}
For $(p,\Phi,q_0)\in\cM$, define 
\begin{equation}
    \label{eq:halg}
    \hALG
    \equiv
    \sup_{(p,\Phi,q_0)\in \cM}
    \bbA(p,\Phi; q_0).
\end{equation}
where the second term of $\bbA$ is given (with $f$ as in \eqref{eq:careful-alg}) by:
\[
    \int_{q_0}^1
    \sqrt{\Phi'_s(q) (p\times \xi^s \circ \Phi)'(q)}
    ~\de q
    =
    \int_{q_0}^1
    \sqrt{\Phi'_s(q) f(q)}
    ~\de q.
\]
\end{definition}


It will follow from our results in this section that for non-degenerate $\xi$, all maximizers to the extended variational problem are continuously differentiable on $[q_0,1]$. The equality $\ALG=\hALG$ follows in general since both are continuous in $(\xi,\vh)$. 



\begin{remark}
    A related (for the most part, simpler) variational problem was considered in \cite{deuschel1995limiting}. 
    There, after showing existence and other basic properties, the general result \cite[Theorem 5.1]{cesari2012optimization} was used to derive an ordinary differential equation \cite[Theorem 4]{deuschel1995limiting} for the optimal $\Phi$.
    The same general result applies in our setting, and essentially yields Proposition~\ref{prop:psi}, assuming $f_s$ defined in \eqref{eq:f-s-q} are absolutely continuous for all $s\in\sS$. 
    More precisely, under this assumption one finds (cf. \eqref{eq:Psi-s}, \eqref{eq:derivative-stability}):
    \[
    \sum_{s\in\sS}
    \Psi_s(q)
    \big(p\times \partial_s \xi^{s'}\circ\Phi\big)(q)\Phi_s'(q)
    =
    \Psi_{s'}(q)
    \big(p\times \partial_s \xi^{s'}\circ\Phi\big)'(q).
    \]
    Viewing this as a linear system in the variables $\Psi_s(q)$, Corollary~\ref{cor:rank} implies that if $p'(q)>0$, then $\Psi_1(q)=\Psi_2(q)=\dots=\Psi_r(q)=0$.
    Similarly if $p'(q)=0$, Lemma~\ref{lem:rank-v2} with $\eps=0$ implies $\Psi_1(q)=\Psi_2(q)=\dots=\Psi_r(q)$.
    However the only way we could establish absolute continuity of $f_s$ was by going through the full proof of Proposition~\ref{prop:psi}.
\end{remark}



\subsection{Linear Algebraic and Analytic Preliminaries}

We first prove Corollary~\ref{cor:solvability-equivalent} below, an equivalent characterization of (super, strict sub)-solvability.
\begin{proposition}
    \label{prop:smallest-eigenvalue}
    Let $M \in \bbR^{\sS \times \sS}$ be diagonally signed.
    Then 
    \begin{equation}
        \label{eq:VM-def}
        \Lambda(M)=\sup_{\vv\in \mathbb R_{>0}^{\sS}} \min_{s\in\sS} \frac{(M\vv)_s}{v_s}
    \end{equation}
    equals the smallest eigenvalue $\lambda_{\min}(M)$ of $M$. 
\end{proposition}
\begin{proof}
    Let $\vw$ be a (unit) minimal eigenvector of $M$.
    Note that
    \[
        \vw^\top M \vw
        = \sum_{s,s' \in \sS} M_{s,s'} w_s w_{s'}
        \ge \sum_{s,s' \in \sS} M_{s,s'} |w_s| |w_{s'}|.
    \]
    Since $\vw$ minimizes $\vw^\top M \vw$, all entries of $\vw$ are the same sign.
    We may thus assume $\vw \in \bbR_{\ge 0}^\sS$.
    Moreover, if $w_s=0$ for any $s$, then $(M\vw)_s < 0$ so $\vw$ is not an eigenvector; thus $\vw \in \bbR_{>0}^\sS$.
    Because $M\vw = \lambda_{\min}(M)\vw$, clearly $\Lambda(M) \ge \lambda_{\min}(M)$.
    For any other $\vv \in \bbR_{>0}^\sS$,
    \[
        \min_{s\in \sS}
        \fr{(M\vv)_s}{v_s}
        \le 
        \la \vw,\vv \ra^{-1} \sum_{s\in \sS} w_sv_s \cdot \fr{(M\vv)_s}{v_s}
        = \fr{\la \vw, M\vv \ra}{\la \vw,\vv \ra}
        = \fr{\la M\vw, \vv \ra}{\la \vw,\vv \ra}
        = \lambda_{\min}(M),
    \]
    so $\Lambda(M) \le \lambda_{\min}(M)$.
    Thus $\Lambda(M) = \lambda_{\min}(M)$.
\end{proof}
\begin{corollary}
    \label{cor:solvability-equivalent}
    For $\vx \in (0,1]^\sS \cup \{\vzero\}$ define
    \[
        M^*(\vx) 
        = \diag\lt((\xi^s(\vx) + h_s^2)_{s\in \sS}\rt)
        - \lt(x_s \partial_{x_{s'}} \xi^s(\vx)\rt)_{s,s'\in \sS}.
    \]
    Then $\vx$ is super-solvable (resp. solvable, strictly sub-solvable) if and only if $\Lambda(M^*(\vx)) \ge 0$ (resp. $=0$, $<0$).
\end{corollary}
\begin{proof}
    Suppose first $\vx \in (0,1]^\sS$. 
    By Proposition~\ref{prop:smallest-eigenvalue}, $\vx$ is super-solvable (resp. solvable, strictly sub-solvable) if and only if $\Lambda(M^*_\sym(\vx)) \ge 0$ (resp. $=0$, $<0$). 
    Note that 
    \begin{equation}
        \label{eq:M*sym-to-M*}
        M^*(\vx) = \diag\lt((\lambda_s x_s)_{s\in \sS}\rt) M^*_\sym(\vx),
    \end{equation}
    so $\Lambda(M^*(\vx))$ has the same sign as $\Lambda(M^*_\sym(\vx))$, as desired.
    If $\vx = \vzero$, then clearly $\Lambda(M^*(\vx)) \ge 0$ with equality at $\vh = \vzero$, which agrees with the convention from Definition~\ref{defn:solvable}.
\end{proof}


The following proposition is clear.

\begin{proposition}
\label{prop:VM}
Let $\Lambda$ be as in \eqref{eq:VM-def} and let $M\in\bbR_{\geq 0}^{r\times r}$ (not necessarily diagonally signed). Then $\Lambda(M)$ is non-negative and locally bounded. Moreover if for some $c\in\bbR$ we have $M'_{s,s'}\geq M_{s,s'}+c\cdot 1_{s=s'}$ for all $s,s'$, then $\Lambda(M')\geq \Lambda(M)+c$.
\end{proposition}

Many perturbation arguments used to establish regularity rely on the following basic fact.

\begin{proposition}[{\cite[Theorem 7.7]{rudin1970real}}]
    \label{prop:lebesgue}
    For $f\in L^{1}([0,1])$, almost all $x\in [0,1]$ are \textbf{Lebesgue points}: 
    \[
      \lim_{\eps\to 0}
      \frac{1}{2\eps} 
      \int_{x-\eps}^{x+\eps}
      |f(y)-f(x)|
      ~\de y = 0.
    \]
\end{proposition}

The next fact ensures that Lipschitz ordinary differential equations are well-posed (even if they are only required to hold almost everywhere).

\begin{proposition}[{\cite[Theorem 1.45, Part (ii)]{roubivcek2013nonlinear}}]
\label{prop:ODE-well-posed}
Suppose $Y_1,Y_2:[0,1]\to\bbR^d$ are each absolutely continuous with $Y_1(0)=Y_2(0)$ and solve the ODE $Y_i'(q)=F(Y_i(q))$ at almost all $q$ for $F:\bbR^d\to\bbR^d$ Lipschitz. Then $Y_1,Y_2$ agree and solve the ODE for all $q$.
\end{proposition}

\subsection{A Priori Regularity of Maximizers}
\label{subsec:basic-regularity}

We first show that for the optimization problem \eqref{eq:alg}, admissibility \eqref{eq:admissible} is just a convenient choice of normalization. 
This makes variational arguments more convenient because we do not need to worry about preserving admissibility of $\Phi$ under perturbations.
Let $\tbbI(q_0,1) \subseteq \hbbI(q_0,1)$ be the set of increasing and Lipschitz functions $f:[q_0,1] \to [0,1]$ with no explicit bound on the Lipschitz constant and with $f(1)=1$.
Note that the algorithmic functional $\bbA$ \eqref{eq:alg-functional} remains well-defined for $\Phi \in \tbbI(q_0,1)^\sS$. 
\begin{lemma}
    \label{lem:admissible-optional}
    We have that
    \begin{equation}
        \label{eq:admissible-optional}
        \hALG =
        \sup_{q_0\in [0,1]}
        \sup_{\substack{p\in \hbbI(q_0,1) \\ \Phi \in \tbbI(q_0,1)^\sS}} 
        \bbA(p,\Phi; q_0).
    \end{equation}
\end{lemma}
\begin{proof}
    Let $\hALG'$ be the right-hand side of \eqref{eq:admissible-optional}.
    We will show that $\hALG \geq \hALG'$ (the opposite implication being trivial).
    
    Consider any $q_0 \in [0,1]$, $p\in \hbbI(q_0,1)$, and $\Phi \in \tbbI(q_0,1)^\sS$.
    For small $\delta > 0$, consider
    \[
        \Phi_{\delta} (q) = \delta q \vone + (1-\delta) \Phi(q)
    \]
    and let $\alpha(q) = \la \vlam, \Phi_{\delta}(q)\ra$, so $\alpha'(q) \ge \delta$. 
    Thus $\alpha^{-1}$ exists and is $\delta^{-1}$-Lipschitz. 
    Consider $(\wtp,\tPhi,\wtq_0)$ given by
    \[
        \wtp(q) = p(\alpha^{-1}(q)),
        \quad 
        \tPhi(q) = \Phi(\alpha^{-1}(q)),
        \quad 
        \wtq_0 = \alpha(q_0).
    \]
    By construction, $\tPhi \in \hAdm(\wtq_0,1)$.
    By the chain rule, $\bbA(\wtp,\tPhi;\wtq_0) = \bbA(p,\Phi_{\delta};q_0)$.
    Thus
    \[
        \hALG 
        \ge 
        \limsup_{\delta\downarrow 0}
        \bbA(\wtp,\tPhi,\wtq_0)
        =
        \limsup_{\delta\downarrow 0}
        \bbA(p,\Phi_{\delta};q_0)
        \ge 
        \bbA(p,\Phi;q_0).
    \]
    Since $p,\Phi,q_0$ were arbitrary the conclusion follows.
\end{proof}

A routine compactness argument given in Appendix~\ref{subsec:maximizer-existence} yields the following. 
\begin{restatable}{proposition}{propFmax}
\label{prop:F-max}
There exists a maximizer $(p,\Phi,q_0)\in\cM$ for $\bbA$ and $\bbA(p,\Phi;q_0)<\infty$.
\end{restatable}

From now on, we let $(p,\Phi,q_0)\in\cM$ denote any maximizer and study the behavior of $(p,\Phi,q_0)$. 
While almost no regularity on $(p,\Phi)$ is assumed, it is possible to establish a priori regularity using variational arguments. 
We defer the proofs of the following two propositions to Appendix~\ref{subsec:regularity-for-4.1}. 
Proposition~\ref{prop:basic-regularity} implies that the discussion following \eqref{eq:alg} is not necessary to define $\bbA(p,\Phi;q_0)$.
\begin{restatable}{proposition}{propBasicRegularity}
    \label{prop:basic-regularity}
    The functions $p,\Phi$ are continuously differentiable on $[q_0+\eps,1]$ for any $\eps>0$.
    Moreover, there exists $L>0$ (possibly depending on $(p,\Phi;q_0)$ as well as $\xi$) such that $L^{-1} \vone \preceq \Phi'(q) \preceq L \vone$ for almost all $q\in (q_0,1]$.
\end{restatable}

\begin{restatable}{proposition}{propPBasic}
    \label{prop:p-basic}
    The function $p$ satisfies $p(q)>0$ for all $q>q_0$, $p(1)=1$, and $p(q_0)=0$ if $q_0>0$.
\end{restatable}


Throughout the next subsection we will use $\eps>0$ as in Proposition~\ref{prop:basic-regularity}. Later we slightly improve the result of Proposition~\ref{prop:basic-regularity} to continuity on $[q_0,1]$ using more detailed properties of the maximizers. 

\subsection{Identification of Root-Finding and Tree-Descending Phases}
\label{subsec:Psi-calculation}

In this subsection we will prove the following result. Recall that the Sobolev space $W^{2,\infty}([q_0+\eps,1])$ consists of $C^1$ functions with Lipschitz derivative on the interval.
\begin{proposition}
    \label{prop:type-12}
    The restrictions of $p$ and $\Phi_s$, for all $s\in \sS$, lie in the space $W^{2,\infty}([q_0+\eps,1])$ for any $\eps>0$. 
    There exists $q_1\in [q_0,1]$ such that the following holds. 
    \begin{enumerate}[label=(\alph*), ref=\alph*]
        \item On $[q_0,q_1]$, $p'>0$ almost everywhere and the quantities $\fr{\Phi'_s(q)}{(p\times \xi^s \circ \Phi)'(q)}$ are constant.
        Moreover $p(q_1)=1$.
        \item On $[q_1,1]$, the ODE \eqref{eq:tree-descending-ode} is satisfied for all $s,s'\in \sS$ almost everywhere and $p=1$.
    \end{enumerate}
\end{proposition}

We begin with a result on diagonally dominant matrices. Variants especially with $\eps=0$ have been used many times, see e.g. \cite{taussky1949recurring}.
Related linear algebraic statements will appear later in Lemmas~\ref{lem:positive-linalg-with-p} and \ref{lem:pos-linalg-diagonal-must-grow} as, roughly speaking, $r$-dimensional analogs of monotonicity. 

\begin{lemma}
    \label{lem:rank-v2}
    Let $A=(a_{i,j})_{i,j\in [r]} \in \bbR^{r\times r}$ satisfy $a_{i,i}>0$ and $a_{i,j} < 0$ for all $i\neq j$.
    \begin{enumerate}[label=(\alph*), ref=\alph*]
        \item 
        \label{it:zero-sum-v2}
        If $\sum_{j=1}^r a_{i,j}=0$ for all $i\in [r]$, then all solutions $\vv \in \bbR^r$ to $A\vv \preceq \eps\vone$ satisfy $|v_i-v_j| \le \eps / a_{\min}$ for all $i,j$, where $a_{\min} = \min_{i\neq j} |a_{i,j}|$.
        \item 
        \label{it:positive-sum-v2}
        If $\sum_{j=1}^r a_{i,j}\geq d_{\min}>0$ for all $i\in [r]$, then all solutions $\vv \in \bbR^r$ to $\tnorm{A\vv}_\infty \le \eps$ satisfy $\|v_i\|_{\infty} \le \eps / d_{\min}$.
    \end{enumerate}
\end{lemma}
\begin{proof}[Proof of Lemma~\ref{lem:rank-v2}]
    Assume without loss of generality that $v_1\geq v_s$ for all $s$. 
    If $\sum_{j=1}^r a_{i,j}=0$ for all $i\in [r]$, then
    \[
        \eps 
        \ge (A\vv)_1 
        = a_{1,1}v_1 + \sum_{j=2}^r a_{1,j}v_i
        = \sum_{j=2}^r |a_{1,j}|(v_1-v_j)
        \ge a_{\min} (v_1-v_i)
    \]
    for all $i\ge 2$.
    Thus $v_1-v_i \le \eps / a_{\min}$, proving the first part.
    For the second part, we will first show $v_1 \le \eps / d_{\min}$. 
    If $v_1 < 0$ there is nothing to prove, and otherwise
    \[
        \eps
        \ge 
        (A\vv)_1 
        = 
        a_{1,1}v_1 + \sum_{j=2}^r a_{1,j}v_j 
        \ge 
        \lt(a_{1,1} - \sum_{j=2}^r a_{1,j}\rt) v_1
        \ge 
        d_{\min} v_1.
    \]
    So $v_1 \le \eps / d_{\min}$, as claimed.
    Finally, note that if $\tnorm{A\vv}_\infty \le \eps$, the same is true for $-\vv$.
    By the same argument we find the largest entry of $-\vv$ is at most $\eps / d_{\min}$.
    This implies the second part.
\end{proof}
\begin{corollary}
    \label{cor:rank}
    Let $A=(a_{i,j})_{i,j\in [r]} \in \bbR^{r\times r}$ satisfy $a_{i,i}>0$ and $a_{i,j} < 0$ for all $i\neq j$.
    If $\sum_{j=1}^r a_{i,j}>0$ for all $i\in [r]$, then the only solution to $A\vv = \vzero$ is $\vv=\vzero$.
\end{corollary}
\begin{proof}
    Apply Lemma~\ref{lem:rank-v2}(\ref{it:positive-sum-v2}) with $\eps=0$. 
\end{proof}

To establish additional regularity we use the following fact on distributional derivatives.

\begin{lemma}[{See e.g. \cite[Theorem 2.2.1]{ziemer2012weakly}}]
    \label{lem:manual-integration-by-parts}
    If $A,B \in L^{\infty}([q_0,1])$ satisfy 
    \[
        \int_{q_0}^1 
        A(q) \psi(q) + B(q) \psi'(q)
        ~\de q 
        = 0
    \]
    for all $\psi \in C_c^\infty((q_0,1);\bbR)$, then there exists $C\in \bbR$ such that for all $q\in [q_0,1]$, 
    \[
        B(q) = \int_{q_0}^q A(t)~\de t + C.
    \]
\end{lemma}


We will make use of the functions
\begin{equation}
\label{eq:f-s-q}
    f_s(q) = \sqrt{\fr{\Phi'_s(q)}{(p\times \xi^s \circ \Phi)'(q)}}
\end{equation}
Note that Propositions~\ref{prop:basic-regularity} and \ref{prop:p-basic} imply $f_s$ is continuous on $[q_0+\eps,1]$.


\begin{proposition}
    \label{prop:psi}
    The functions $f_s$ are Lipschitz on $[q_0+\eps,1]$.
    Thus (recall Proposition~\ref{prop:basic-regularity}) the functions 
    \begin{equation}
    \label{eq:Psi-s}
    \Psi_s(q) = f'_s(q)/\Phi'_s(q)
    \end{equation}
    are measurable and locally bounded on $(q_0,1]$.
    Moreover for almost all $q\in (q_0,1]$, the following holds:
    \begin{equation}    
        \label{eq:psi-equality}
        \Psi_1(q)=\cdots=\Psi_s(q),
    \end{equation}
    and furthermore this common value is $0$ if $p'(q)>0$.
\end{proposition}
\begin{proof}
    Let $\psi \in C_c^\infty((q_0,1);\bbR)$.
    Consider the perturbation
    \[
        \tPhi_1(q) = \Phi_1(q) + \delta \psi(q),
    \]
    and let $\tPhi_s(q) = \Phi_s(q)$ for $s\neq 1$.
    By Proposition~\ref{prop:basic-regularity}, $\tPhi$ remains coordinate-wise increasing and Lipschitz for small positive and negative $\delta$.
    Although $\tPhi \not \in \hAdm(q_0,1)$, recalling Lemma~\ref{lem:admissible-optional} we nonetheless have $\bbA(p,\tPhi;q_0) \le \bbA(p,\Phi;q_0)$.
    Thus,
    \[
        F_1 \equiv \fr{\de}{\de \delta} \bbA(p,\tPhi;q_0) \Big|_{\delta=0} = 0.
    \]
    We now calculate $F_1$. 
    Note that
    \begin{equation}
        \label{eq:phi1-deriv}
        \fr{\de}{\de \delta} (p\times \xi^s \circ \tPhi)'(q) 
        \Big|_{\delta=0}
        = 
        (p\psi \times \partial_{x_1}\xi^s \circ \Phi)'(q)
        =
        \fr{\lambda_1}{\lambda_s}
        (p\psi \times \partial_{x_s}\xi^1 \circ \Phi)'(q).
    \end{equation}
    So,
    \begin{align*}
        0 = \fr{2}{\lambda_1} F_1
        &= 
        \int_{q_0}^1
        f_1(q)^{-1}
        \psi'(q)~\de q
        +
        \sum_{s\in \sS}
        \int_{q_0}^1
        f_s(q)
        (p\psi \times \partial_{x_s}\xi^1 \circ \Phi)'(q)
        ~\de q \\
        &= 
        \int_{q_0}^1
        A_1(q) \psi(q) + B_1(q) \psi'(q) 
        ~\de q
    \end{align*}
    where
    \[
        A_1(q)
        \equiv
        \sum_{s\in \sS}
        f_s(q)
        (p\times \partial_{x_s} \xi^1 \circ \Phi)'(q), 
        \quad 
        B_1(q) 
        \equiv
        f_1(q)^{-1}
        + \sum_{s\in \sS}
        f_s(q)
        (p\times \partial_{x_s} \xi^1 \circ \Phi)(q).
    \]
    By Proposition~\ref{prop:basic-regularity}, for all $\eps>0$ $A_1(q)$ and $B_1(q)$ are bounded for $q\in[q_0+\eps,1]$.
    Lemma~\ref{lem:manual-integration-by-parts} implies that $B_1(q)$ is absolutely continuous and $B'_1(q) = A_1(q)$ for all $q\in (q_0,1]$. In fact by Proposition~\ref{prop:basic-regularity}, $A_1$ is bounded and continuous on $[q_0+\eps,1]$, so $B_1\in C^1([q_0+\eps,1])$ (for all $\eps>0$).

    Fix $q\in (q_0,1]$. 
    For $\iota \in \bbR$ with $|\iota|$ small, let $\Delta^\iota_s = f_s(q+\iota) - f_s(q)$. 
    By Proposition~\ref{prop:basic-regularity} all $f_s$ are continuous, so $\Delta^\iota_s =o(1)$; here are below we use $o(\cdot)$ for limits as $\iota\to 0$.
    Thus,
    \begin{align*}
        B_1(q+\iota) - B_1(q)
        &= 
        \fr{1}{f_1(q)+\Delta^\iota_1} - \fr{1}{f_1(q)}
        + \sum_{s\in \sS}
        \Delta^\iota_s
        \cdot
        (p\times \partial_{x_s} \xi^1 \circ \Phi)(q) \\
        &\quad + \sum_{s\in \sS} f_s(q+\iota)\lt((p\times \partial_{x_s} \xi^1 \circ \Phi)(q+\iota) - (p\times \partial_{x_s} \xi^1 \circ \Phi)(q)\rt).
    \end{align*}
    Since $(p\times \partial_{x_s} \xi^1 \circ \Phi)$ is differentiable and $f_s$ is continuous,
    \begin{align*}
        \sum_{s\in \sS} 
        f_s(q+\iota)
        \cdot
        \big(
        (p\times \partial_{x_s} \xi^1 \circ \Phi)(q+\iota) - (p\times \partial_{x_s} \xi^1 \circ \Phi)(q)
        \big)
        &=
        \iota 
        \sum_{s\in \sS} f_s(q)(p\times \partial_{x_s} \xi^1 \circ \Phi)'(q) + o(|\iota|) \\
        &= \iota A_1(q) + o(|\iota|).
    \end{align*}
    Moreover,
    \begin{align*}
        \fr{1}{f_1(q)+\Delta^\iota_1} - \fr{1}{f_1(q)}
        &= 
        -\fr{\Delta^\iota_1}{f_1(q)(f_1(q)+\Delta^\iota_1)} 
        = 
        \fr{(\Delta^\iota_1)^2}{f_1(q)^2(f_1(q)+\Delta^\iota_1)}
        -\fr{\Delta^\iota_1}{f_1(q)^2} \\
        &= 
        \fr{(\Delta^\iota_1)^2}{f_1(q)^2(f_1(q)+\Delta^\iota_1)}
        - \fr{\Delta^\iota_1}{\Phi'_1(q)} \lt(
            p'(q)(\xi^1\circ \Phi)(q) + 
            \sum_{s\in \sS} (p\times \partial_{x_s} \xi^1 \circ \Phi)(q)\Phi'_s(q)
        \rt).
    \end{align*}
    We also have $B_1(q+\iota) - B_1(q) = A_1\iota + o(|\iota|)$ (recall $A_1$ is continuous).
    Thus
    \begin{equation}
        \label{eq:derivative-stability}
        \sum_{s\in \sS}
        (p\times \partial_{x_s} \xi^1 \circ \Phi)(q) \Phi'_s(q)
        \lt[\fr{\Delta^\iota_1}{\Phi'_1(q)} - \fr{\Delta^\iota_s}{\Phi'_s(q)}\rt]
        + p'(q) (\xi^1 \circ \Phi)(q) \fr{\Delta^\iota_1}{\Phi'_1(q)}
        - \fr{(\Delta^\iota_1)^2}{f_1(q)^2(f_1(q)+\Delta^\iota_1)}
        =
        o(|\iota|).
    \end{equation}
    We get similar equations from perturbing any $\Phi_{s}$ instead of $\Phi_1$.
    If $p'(q)>0$, then we can write the last two terms on the left-hand side of \eqref{eq:derivative-stability} as 
    \[
        \fr{\Delta^\iota_1}{\Phi'_1(q)} \lt(
            p'(q) (\xi^1 \circ \Phi)(q)
            - \fr{\Delta^\iota_1 \Phi'_1(q)}{f_1(q)^2(f_1(q)+\Delta^\iota_1)}
        \rt)
        = \fr{\Delta^\iota_1}{\Phi'_1(q)} \lt(p'(q) (\xi^1 \circ \Phi)(q) + o(1)\rt).
    \]
    Then, \eqref{eq:derivative-stability} and its analogs form a linear system in variables $x_s\equiv\Delta^\iota_s/\Phi'_s(q)$ with all row sums positive. (E.g. in \eqref{eq:derivative-stability}, the first term gives zero coefficient sum so the total coefficient sum is just $\lt(p'(q) (\xi^1 \circ \Phi)(q) + o(1)\rt)>0$.) Moreover the diagonal coefficients of this system are e.g.
    \[
    a_{1,1}
    =
    \lt(p'(q) (\xi^1 \circ \Phi)(q) + o(1)\rt)
    +
    \sum_{s\in \sS}
        (p\times \partial_{x_s} \xi^1 \circ \Phi)(q) \Phi'_s(q)>0
    \]
    while the off-diagonal coefficients are e.g.
    \[
    a_{1,s}
    =
    -
    (p\times \partial_{x_s} \xi^1 \circ \Phi)(q) \Phi'_s(q)
    <0.
    \]
    Applying Lemma~\ref{lem:rank-v2}(\ref{it:positive-sum-v2}), we obtain 
    \[
    |\Delta^\iota_s/\Phi'_s(q)| = o(|\iota|)
    \]
    for all $s\in \sS$.
    Taking $\iota \to 0$ we conclude that $f'_s(q)$ is well-defined and equals $0$.
    This implies the conclusion for $p'(q)>0$.


    
    Otherwise $p'(q)=0$, and \eqref{eq:derivative-stability} implies that
    \[
        \sum_{s\in \sS}
        (p\times \partial_{x_s} \xi^1 \circ \Phi)(q) \Phi'_s(q)
        \lt[\fr{\Delta^\iota_1}{\Phi'_1(q)} - \fr{\Delta^\iota_s}{\Phi'_s(q)}\rt]
        \ge 
        -o(|\iota|)
    \]
    and analogously with any $s\in \sS$ in place of $1$.
    This is a linear system of inequalities in variables 
    $-\Delta^\iota_s/\Phi'_s(q)$, so Lemma~\ref{lem:rank-v2}(\ref{it:zero-sum-v2}) implies that 
    \begin{equation}
        \label{eq:psi-equality-discretized}
        \lt|\fr{\Delta^\iota_s}{\Phi'_s(q)} - \fr{\Delta^\iota_{s'}}{\Phi'_{s'}(q)}\rt| \le o(|\iota|)
    \end{equation}
    for all $s,s'\in \sS$.
    The result now follows if we find a constant $C=C(\eps)$ such that 
    \begin{equation}
        \label{eq:psi-goal}
        \max_{s\in \sS} |\Delta^\iota_s/\Phi'_s(q)| \le C|\iota|
    \end{equation}
    for all sufficiently small $\iota$, and $q\in [q_0+\eps,1]$.
    Indeed, this would imply by Proposition~\ref{prop:basic-regularity} that $f_s$ is Lipschitz on $[q_0+\eps,1]$. It would then follow that $\Psi\in L^{\infty}([q_0+\eps,1])$, and we would conclude from \eqref{eq:psi-equality-discretized} that $\Psi_1=\cdots=\Psi_r$ almost everywhere. 
    
    Since $p$ and $\partial_{x_s}\xi^1$ are differentiable and $p'(q)=0$, we have $p(q+\iota) = p(q) + o(|\iota|)$ and $\partial_{x_s}\xi^1(q+\iota) = \partial_{x_s}\xi^1(q) + O(|\iota|)$.
    Suppose first that $\iota > 0$.
    Using $p'(q)=0$ and $p'(q+\iota)\geq 0$, we find
    \begin{align}
        \label{eq:use-p'-zero}
        \Delta^\iota_1
        &\le
        \sqrt{\fr{\Phi'_1(q+\iota)}{p(q+\iota)(\xi^1\circ \Phi)'(q+\iota)}}
        -\sqrt{\fr{\Phi'_1(q)}{p(q)(\xi^1\circ \Phi)'(q)}} \\
        \notag
        &= 
        \fr{1}{\sqrt{p(q)}} \lt(
            \sqrt{\fr{\Phi'_1(q+\iota)}{\sum_{s\in \sS} (\partial_{x_s}\xi^1 \circ \Phi)(q) \Phi'_s(q+\iota)}}
            - \sqrt{\fr{\Phi'_1(q)}{\sum_{s\in \sS} (\partial_{x_s}\xi^1 \circ \Phi)(q) \Phi'_s(q)}}
        \rt) + O(\iota)
    \end{align}
    where we used that $p'(q)=0$ and $p'(q+\iota) \ge 0$. 
    The hidden constants are uniform on any interval $[q_0+\eps,1]$.
    Analogous bounds hold for $\Delta^\iota_s$. 
    We claim that we cannot have 
    \begin{equation}
        \label{eq:all-ratios-bigger}
        \fr{\Phi'_s(q+\iota)}{\sum_{s'\in \sS} (\partial_{x_{s'}}\xi^s \circ \Phi)(q) \Phi'_{s'}(q+\iota)}
        > \fr{\Phi'_s(q)}{\sum_{s'\in \sS} (\partial_{x_{s'}}\xi^s \circ \Phi)(q) \Phi'_{s'}(q)}
    \end{equation}
    for all $s\in \sS$. 
    Indeed, suppose this holds and let 
    \begin{align*}
        b_s &= \fr{\sum_{s'\in \sS} (\partial_{x_{s'}}\xi^s \circ \Phi)(q) \Phi'_{s'}(q)}{\Phi'_s(q)}, \\
        b'_s &= \fr{\sum_{s'\in \sS} (\partial_{x_{s'}}\xi^s \circ \Phi)(q) \Phi'_{s'}(q+\iota)}{\Phi'_s(q+\iota)},
    \end{align*}
    so $b'_s < b_s$. 
    The linear system given by
    \[
        b'_s\Phi'_s(q+\iota)x_s - \lt(\sum_{s'\in \sS} (\partial_{x_{s'}}\xi^s \circ \Phi)(q) \Phi'_{s'}(q+\iota)x_{s'}\rt) = 0
    \]
    for all $s\in \sS$ has solution $\vx = \vone$, and thus has row sums zero. 
    The linear system given by 
    \[
        b_s\Phi'_s(q+\iota)x_s - \lt(\sum_{s'\in \sS} (\partial_{x_{s'}}\xi^s \circ \Phi)(q) \Phi'_{s'}(q+\iota)x_{s'}\rt) = 0
    \]
    has solution $x_s = \Phi'_s(q)/\Phi'_s(q+\iota)$.
    However, by Corollary~\ref{cor:rank} its only solution is $\vx=\vzero$, contradiction.
    Thus \eqref{eq:all-ratios-bigger} does not hold for all $s\in \sS$. 
    Assume without loss of generality \eqref{eq:all-ratios-bigger} does not hold for $s=1$. 
    Then, $\Delta^\iota_1 \le O(\iota)$.
    In conjunction with \eqref{eq:psi-equality-discretized}, this implies $\max_{s\in \sS} \Delta^\iota_s/\Phi'_s(q) \le C\iota$.
    
    For the matching lower bound, first consider the case $p'(q+\iota)=0$. 
    In this case, the inequality in \eqref{eq:use-p'-zero} is an equality.
    We similarly cannot have
    \[
        \fr{\Phi'_s(q+\iota)}{\sum_{s'\in \sS} (\partial_{x_{s'}}\xi^s \circ \Phi)(q) \Phi'_{s'}(q+\iota)}
        < \fr{\Phi'_s(q)}{\sum_{s'\in \sS} (\partial_{x_{s'}}\xi^s \circ \Phi)(q) \Phi'_{s'}(q)}
    \]
    for all $s\in \sS$, so the same argument implies $\min_{s\in \sS} \Delta^\iota_s/\Phi'_s(q) \ge -C\iota$, which implies \eqref{eq:psi-goal}.
    Otherwise assume $p'(q+\iota)>0$. 
    Let $\iota_1 \in (0, \iota/2)$ be small enough that 
    \begin{equation}
        \label{eq:p'-bdd-from-0}
        p'(q')\ge \fr12 p'(q+\iota) \quad \text{for all}~q'\in [q+\iota-\iota_1,q+\iota]
    \end{equation} 
    which exists by continuity of $p'$. 
    Let $\psi \in C_c^\infty((q_0,1);\bbR)$ satisfy that $|\psi'|\le 1$ and $\psi'$ is supported on $[q,q+\iota_1]\cup [q+\iota-\iota_1,q+\iota]$, positive on $[q,q+\iota_1]$, and negative on $[q+\iota-\iota_1,q+\iota]$.
    (Note that $\psi'$ integrates to zero because $\psi$ has bounded support, and that $\psi$ is clearly nonnegative.)
    Let $\iota_2 = \psi(q+\iota_1)$.
    Consider the perturbation $\wtp = p + \delta \psi$, which is increasing for small $\delta>0$ by \eqref{eq:p'-bdd-from-0}.
    Let $o_{\iota_1}(1)$ denote a term tending to $0$ as $\iota_1\to0$. 
    We compute that
    \begin{align*}
        F &\equiv 
        \fr{\de}{\de \delta} \bbA(\wtp,\Phi;q_0) \Big|_{\delta=0} \\
        &= 
        \sum_{s\in \sS}
        \lambda_s
        \int_{q_0}^1
        f_s(q) (\psi \times \xi^s \circ \Phi)'(q) \\
        &\ge 
        \sum_{s\in \sS}
        \lambda_s
        \int_{q_0}^1
        \psi'(q) f_s(q) (\xi^s \circ \Phi)(q) && \text{(positivity of $\psi$)}\\
        &=
        \sum_{s\in \sS}
        \lambda_s \cdot
        \iota_2 \lt(
            f_s(q) (\xi^s \circ \Phi)(q)
            - f_s(q+\iota) (\xi^s \circ \Phi)(q+\iota)
            +o_{\iota_1}(1)
        \rt) && \text{(continuity of $f_s, \xi^s\circ \Phi$)} \\
        &= \iota_2 \lt(
            -\sum_{s\in \sS} 
            \lambda_s
            \Delta^\iota_s (\xi^s \circ \Phi)(q)
            +o_{\iota_1}(1)
            +O(\iota)
        \rt) && \text{(continuity of $\xi^s\circ \Phi$)} \\
        &= \iota_2 \lt(
            -\Delta^\iota_1
            \sum_{s\in \sS} 
            \frac{\Phi_s'(q)}{\Phi_1'(q)}\cdot
            \lambda_s
            (\xi^s \circ \Phi)(q)
            +o_{\iota_1}(1)
            +O(\iota)
        \rt). && \text{(by \eqref{eq:psi-equality-discretized})}
    \end{align*}
    Since $(p,\Phi,q_0)$ is a maximizer, $F \le 0$. 
    This implies $\Delta^\iota_1 \ge -C\iota$, and by \eqref{eq:psi-equality-discretized}, $\min \Delta^\iota_1 \ge -C\iota$.
    This proves \eqref{eq:psi-goal} for $\iota > 0$. 
    The proof for $\iota < 0$ is analogous.
\end{proof}

\begin{lemma}
    \label{lem:positive-linalg-with-p}
    Let $A = (a_{i,j}) \in \bbR_{>0}^{r\times r}$, $\va, \vb \in \bbR_{>0}^r$, $\vc \in \bbR_{>0}^r$, and $c\in \bbR_{>0}$.
    Let $A_{\min},a_{\min},b_{\min}$ denote the minimal entries of $A,\va,\vb$, and $a_{\max}$ denote the maximal entry of $\va$.
    Suppose the linear system
    \[
        A\vx + \va y = \vc \odot \vx, 
        \quad 
        \la \vb,\vx\ra = c
    \]
    has solution $(y,\vx) = (y_0,\vone)$.
    If ${\vc\,}' \in \bbR_{>0}^r$ satisfies $\tnorm{\vc-{\vc\,}'}_\infty \le \eps$, then any solution $y\in \bbR_{\ge 0}$, $\vx \in \bbR_{\ge 0}^r$ to 
    \[
        A\vx + \va y = {\vc\,}' \odot \vx, 
        \quad 
        \la \vb,\vx\ra = c
    \]
    satisfies 
    \[
        |y-y_0| \le \fr{\eps c}{a_{\min}b_{\min}},
        \quad
        \tnorm{\vx-\vone}_\infty \le \fr{2a_{\max}}{a_{\min}} \cdot \fr{\eps c}{A_{\min}b_{\min}}.
    \]
\end{lemma}
\begin{proof}
    Without loss of generality let $x_1$, $x_2$ be the largest and smallest entries of $\vx$. 
    As $\la \vb,\vone\ra = \la \vb,\vx\ra = c$, 
    \[
    \fr{c}{b_{\min}}\ge x_1\ge 1\ge x_2.
    \]
    Then
    \begin{align*}
        0 
        &= a_1 y + \sum_{i=1}^r a_{1,i}x_i - c'_1x_1 \\
        &\le a_1 y + \lt(\sum_{i=1}^r a_{1,i}-c'_1\rt)x_1 - A_{\min}(x_1-x_2) \\
        &= a_1 y + \lt(c_1-c'_1-a_1y_0\rt)x_1 - A_{\min}(x_1-x_2) \\
        &\le \eps x_1 - a_1y_0(x_1-1) + a_1(y-y_0) - A_{\min}(x_1-x_2) \\
        &\le \fr{\eps c}{b_{\min}} + a_1(y-y_0) - A_{\min}(x_1-x_2).
    \end{align*}
    Analogously
    \begin{align*}
        0 
        &= a_2 y + \sum_{i=1}^r a_{2,i}x_i - c'_2x_2 \\
        &\ge a_2 y + \lt(\sum_{i=1}^r a_{2,i}-c'_2\rt)x_2 + A_{\min}(x_1-x_2) \\
        &= a_2 y + \lt(c_2-c'_2-a_2y_0\rt)x_2 + A_{\min}(x_1-x_2) \\
        &\ge -\eps x_2 - a_2y_0(x_2-1) + a_2(y-y_0) + A_{\min}(x_1-x_2) \\
        &\ge -\fr{\eps c}{b_{\min}} + a_2(y-y_0) + A_{\min}(x_1-x_2).
    \end{align*}
    Since $x_1-x_2\ge 0$, this implies
    \[
        y-y_0 
        \ge -\fr{\eps c}{a_1b_{\min}}
        \ge -\fr{\eps c}{a_{\min}b_{\min}},
        \quad 
        y-y_0 
        \le \fr{\eps c}{a_2b_{\min}}
        \le \fr{\eps c}{a_{\min}b_{\min}},
    \]
    which proves the first conclusion.
    Thus,
    \[
        A_{\min}(x_1-x_2)
        \le 
        \lt(\fr{\eps c}{b_{\min}} + a_1(y-y_0)\rt) 
        \le 
        \fr{2a_{\max}}{a_{\min}} \cdot \fr{\eps c}{b_{\min}}.
    \]
    Since $x_1\ge 1\ge x_2$, we have $\tnorm{\vx - \vone}_\infty \le x_1-x_2$ which implies the second conclusion.
\end{proof}

\begin{proposition}
    \label{prop:twice-diff}
    The functions $p'$ and $\Phi'$ are Lipschitz on $[q_0+\eps,1]$ for all $\eps>0$. 
    Thus $p''$ and $\Phi''$ are well-defined as bounded measurable functions on $[q_0+\eps,1]$.
\end{proposition}
\begin{proof}
    By Proposition~\ref{prop:psi}, $f_s$ is Lipschitz on $[q_0+\eps,1]$.
    Since it is also bounded on $[q_0+\eps,1]$ by Proposition~\ref{prop:basic-regularity}, $f_s^{-2}$ is Lipschitz as well. 
    Thus, for $q\in [q_0+\eps,1]$, $C = C(q)$, and sufficiently small $\iota \in \bbR$,
    \begin{align*}
        O(\iota)
        &\ge 
        |f_1(q+\iota)^{-2} - f_1(q)^{-2}| \\
        &= \bigg|
            \fr{p'(q+\iota)(\xi^1\circ \Phi)(q+\iota) + p(q+\iota)\sum_{s\in \sS} (\partial_{x_s} \xi^1 \circ \Phi)(q+\iota) \Phi'_s(q+\iota)}{\Phi'_1(q+\iota)} \\
            &\qquad -\fr{p'(q)(\xi^1\circ \Phi)(q) + p(q)\sum_{s\in \sS} (\partial_{x_s} \xi^1 \circ \Phi)(q) \Phi'_s(q)}{\Phi'_1(q)}
        \bigg| \\
        &= |C'_1-C_1+O(\iota)|
    \end{align*}
    for
    \begin{align*}
        C_1 &= \fr{p'(q)(\xi^1\circ \Phi)(q) + p(q)\sum_{s\in \sS} (\partial_{x_s} \xi^1 \circ \Phi)(q) \Phi'_s(q)}{\Phi'_1(q)}, \\
        C'_1 &= \fr{p'(q+\iota)(\xi^1\circ \Phi)(q) + p(q)\sum_{s\in \sS} (\partial_{x_s} \xi^1 \circ \Phi)(q) \Phi'_s(q+\iota)}{\Phi'_1(q+\iota)}.
    \end{align*}
    Thus $|C_1-C'_1| \le O(\iota)$.
    Similarly, $|C_s-C'_s|\le O(\iota)$ for analogously defined $C_s,C'_s$.
    Note that the system given by
    \begin{align}
        \label{eq:linear-eq-admissibility}
        1 &= \sum_{s\in \sS} \lambda_s \Phi'_s(q) x_s \\
        \label{eq:linear-eq-gs}
        C_1 \Phi'_1(q) x_1 &= (\xi^1\circ \Phi)(q) y + p(q)\sum_{s\in \sS} (\partial_{x_s} \xi^1 \circ \Phi)(q) \Phi'_s(q) x_s
    \end{align}
    and analogous equations to \eqref{eq:linear-eq-gs} with $s\in \sS$ in place of $1$ has solution $y=p'(q)$, $x_1=\cdots=x_r=1$.
    Moreover, the system given by \eqref{eq:linear-eq-admissibility},
    \begin{equation}
        \label{eq:linear-eq-g's}
        C'_1 \Phi'_1(q) x_1 = (\xi^1\circ \Phi)(q) y + p(q)\sum_{s\in \sS} (\partial_{x_s} \xi^1 \circ \Phi)(q) \Phi'_s(q) x_s
    \end{equation}
    and analogous equations to \eqref{eq:linear-eq-g's} with $s\in \sS$ in place of $1$ has solution $y=p'(q+\iota)$, $x_s = \Phi'_s(q+\iota)/\Phi'_s(q)$.
    Since $|C_s-C'_s|\le O(\iota)$ for all $s$, we may apply Lemma~\ref{lem:positive-linalg-with-p} 
    with $\vc=\vC,{\vc\,}'=\vC'$, $y$ taking the place of $p'(q)$ or $p'(q+\iota)$, and $A$ corresponding to the last term of \eqref{eq:linear-eq-gs} or \eqref{eq:linear-eq-g's}. The result is that
    \[
        |p'(q+\iota)-p'(q)|,
        \lt|\fr{\Phi'_s(q+\iota)}{\Phi'_s(q)}-1\rt| \le O(\iota).
    \]
    (The required constants $A_{\min},a_{\min},b_{\min},a_{\max}$ are bounded thanks to Propositions~\ref{prop:basic-regularity} and \ref{prop:p-basic}.)
    
    Since $\Phi'_s$ is bounded below by Proposition~\ref{prop:basic-regularity}, we conclude that $p', \Phi'$ are Lipschitz in a neighborhood of $q\in (q_0,1]$. 
    This Lipschitz constant is uniform on any $[q_0+\eps,1]$, thus $p',\Phi'$ are Lipschitz on these sets. 
\end{proof}

\begin{lemma}
    \label{lem:pos-linalg-diagonal-must-grow}
    Suppose $A = (a_{i,j}) \in \bbR_{>0}^{r\times r}$ and $\vb \in \bbR_{>0}^r$. 
    Let $A_{\max},A_{\min}$ be the largest and smallest entries of $A$. 
    Suppose the linear system $A\vx = \vb \odot \vx$ admits the solution $\vx = \vone$. 
    If $\vb' \preceq \vb + \eps \vone$, $A'\ge A$ entry-wise, and the system $A'\vx = \vb' \odot \vx$ admits a nontrivial solution $\vx \in \bbR^r_{\ge 0}$, then all entries of $A'-A$ are at most $\eps \cdot \fr{r A_{\max} + A_{\min} + \eps}{A_{\min}}$.
\end{lemma}
\begin{proof}
    Assume without loss of generality that $x_1$ is the smallest entry of $\vx$. 
    Let $\Delta_i = b'_i - b_i$ and $\Delta_{i,j} = a'_{i,j} - a_{i,j}$, so $\Delta_i \le \eps$, $\Delta_{i,j} \ge 0$.
    We have 
    \[
        0 
        = (b_1+\Delta_1)x_1 - \sum_{i=1}^r a'_{i,j}x_i
        \le \Delta_1 x_1 - \sum_{i=1}^r a_{i,j}(x_i-x_1).
    \]
    Thus $a_{i,j}(x_i-x_1) \le \Delta_1 x_1$ for all $i$. 
    If $x_1=0$, this implies $\vx = \vzero$, contradiction.
    Thus $x_1>0$ and we may scale $\vx$ such that $x_1=1$. 
    This implies
    \[
        1 \le x_i \le 1 + \fr{\Delta_1}{a_{i,j}} \le 1 + \fr{\eps}{A_{\min}}
    \]
    for all $i$.
    The equation $b'_jx_j = (A'\vx)_j$ implies
    \begin{align*}
        \sum_{i=1}^r \Delta_{j,i}x_i 
        = 
        b_jx_j + \Delta_jx_j - \sum_{i=1}^r a_{j,i}x_i 
        &\le 
        \lt(1 + \fr{\eps}{A_{\min}}\rt)
        \sum_{i=1}^r a_{j,i} 
        + \eps \lt(1 + \fr{\eps}{A_{\min}}\rt) - \sum_{i=1}^r a_{j,i} \\
        &\le \eps \cdot \fr{r A_{\max} + A_{\min} + \eps}{A_{\min}}.
    \end{align*}
    Since $x_i\ge 1$ for all $i$, this implies the result.
\end{proof}

Let $S\subseteq (q_0,1)$ be the set of $q$ for which \eqref{eq:psi-equality} holds, and for $q\in S$ let $\Psi(q)$ be the common value of the $\Psi_s(q)$.
Let $S_1 = \{q\in S : p'(q) > 0\}$ and $S_2 = S\setminus S_1$.
\begin{proposition}
    \label{prop:psi-negativity}
    Almost everywhere in $S_2$, $\Psi(q)<0$.
\end{proposition}
\begin{proof}
    Suppose for the sake of contradiction that $\Psi(q)\ge 0$ holds for a positive-measure set $T\subseteq S_2$. 
    Let $U\subseteq [q_0,1]$ be the set of $q$ which are Lebesgue points of $f'_s(q)$ for all $s\in \sS$.
    Since these functions are measurable and integrable on $[q_0+\eps,1]$ for all $\eps > 0$, $U$ is almost all of $[q_0,1]$.
    So $T\cap U$ has positive measure.
    Let $q \in T \cap U$.
    Thus
    \[
        \lim_{\iota \to 0^+}
        \fr{f_1(q+\iota)-f_1(q)}{\iota}
        =
        f'_1(q)
        = \Phi'_1(q)\Psi(q),
    \]
    which implies that for small $\iota>0$,
    \[
        f_1(q+\iota) 
        = 
        f_1(q) + \Phi'_1(q)\Psi(q)\iota + o(\iota)
        \ge 
        f_1(q) - o(\iota).
    \]
    Define
    \begin{align*}
        C_1 &= \fr{p(q) (\xi^1 \circ \Phi)'(q)}{\Phi'_1(q)} = f_1(q)^{-2}, \\
        C'_1 &= \fr{p(q+\iota) (\xi^1 \circ \Phi)'(q+\iota)}{\Phi'_1(q+\iota)} \le f_1(q+\iota)^{-2}.
    \end{align*}
    Thus $C'_1 \le C_1 + o(\iota)$.
    For analogously defined $C_s,C'_s$ we have $C'_s \le C_s + o(\iota)$. 
    Note that the system given by
    \[
        C_1\Phi'_1(q) x_1 = \sum_{s\in \sS} p(q) (\partial_{x_s} \xi^1 \circ \Phi)(q) \Phi'_s(q) x_q
    \]
    and analogous equations with $s\in \sS$ in place of $1$ has solution $\vx = \vone$, while the system
    \[
        C'_1\Phi'_1(q) x_1 = \sum_{s\in \sS} p(q+\iota) (\partial_{x_s} \xi^1 \circ \Phi)(q+\iota) \Phi'_s(q) x_q
    \]
    and analogous equations with $s\in \sS$ in place of $1$ has solution $x_s = \Phi'_s(q+\iota)/\Phi'_s(q)$.
    By Lemma~\ref{lem:pos-linalg-diagonal-must-grow} this implies that for all $s,s'\in \sS$, 
    \[
        p(q+\iota) (\partial_{x_s} \xi^{s'} \circ \Phi)(q+\iota) \le p(q) (\partial_{x_s} \xi^{s'} \circ \Phi)(q) + o(\iota).
    \]
    However, since $\xi$ is non-degenerate, $(\partial_{x_s} \xi^{s'} \circ \Phi)(q+\iota) \ge (\partial_{x_s} \xi^{s'} \circ \Phi)(q) + \Omega(\iota)$ for some $s,s'$. 
    This is a contradiction.
\end{proof}

\begin{lemma}
    \label{lem:s1-s2-separate}
    There exists $q_1\in [q_0,1]$ such that, up to modification by a measure zero set, $S_1 = [q_0,q_1]$ and $S_2 = [q_1,1]$.
\end{lemma}
\begin{proof}
    We will show that there do not exist positive measure subsets $I\subseteq S_1$, $J\subseteq S_2$ with $\sup J \le \inf I$.
    Suppose for contradiction that such subsets exist. 
    Define $q^* = \sup J$, $m = \int_I p'(q)~\de q$, and 
    \[
        \psi(q) = 
        \begin{cases}
            m (\int_{[q_0,q]\cap J} \de q)/(\int_J \de q) & q \le q^*, \\
            m - \int_{[q^*,q]\cap I} p'(q)~\de q & q > q^*.
        \end{cases}
    \]
    Note that $\psi$ is absolutely continuous, nonnegative-valued, and positive-valued almost everywhere in $J$. 
    Moreover $\psi(q_0) = \psi(1)=0$, and for small $\delta > 0$, the perturbation
    \begin{equation}
        \label{eq:perturb-p}
        \wtp(q) = p(q) + \delta \psi(q)
    \end{equation}
    remains increasing.
    Note that
    \[
        \fr{\de}{\de \delta}
        (p\times \xi^s \circ \Phi)'(q) = (\psi \times \xi^s \circ \Phi)'(q).
    \]
    Thus, integrating by parts,
    \begin{align*}
        F
        \equiv 
        2 \fr{\de}{\de \delta}
        \bbA(\wtp,\Phi;q_0) \Big|_{\delta=0} 
        &= 
        \sum_{s\in \sS}
        \int_{q_0}^1
        \sqrt{\fr{\Phi'_s(q)}{(p\times \xi^s \circ \Phi)'(q)}} 
        (\psi \times \xi^s \circ \Phi)'(q) 
        ~\de q\\
        &= 
        -\sum_{s\in \sS}
        \int_{q_0}^1
        \psi(q)(\xi^s\circ\Phi)(q) \Phi_s'(q) \Psi_s(q)
        ~\de q \\
        &= 
        -\sum_{s\in \sS}
        \int_{S_2}
        \psi(q)(\xi^s\circ\Phi)(q) \Phi_s'(q) \Psi(q)
        ~\de q.
    \end{align*}
    By Proposition~\ref{prop:psi-negativity}, $\Psi(q) < 0$ almost everywhere in $S_2$.
    Therefore $F > 0$ and the perturbation \eqref{eq:perturb-p} improves the value of $\bbA(p,\Phi;q_0)$, a contradiction. 
    
    Finally, define measures 
    \[
        \mu([q_0,q]) = \int_{[q_0,q]\cap S_1} \de q,
        \qquad 
        \nu([q_0,q]) = \int_{[q_0,q]\cap S_2} \de q.
    \]
    The non-existence of $I,J$ implies that $\max \supp (\mu) \le \min \supp(\nu)$.
    Since $S_1\cup S_2$ is almost all of $[q_0,1]$ the result follows.
\end{proof}

\begin{proof}[Proof of Proposition~\ref{prop:type-12}]
    That $p,\Phi_s\in W^{2,\infty}([q_0+\eps,1])$ follows from Proposition~\ref{prop:twice-diff}.
    By Lemma~\ref{lem:s1-s2-separate}, $p'>0$ almost everywhere on $[q_0,q_1]$.
    By Proposition~\ref{prop:psi}, $\Psi_s=0$ almost everywhere on $[q_0,q_1]$. 
    Since $f_s$ is Lipschitz, for all $q\in [q_0,q_1]$ we have
    \[
        f_s(q)-f_s(q_0) = \int_{q_0}^q f'_s(q)~\de q = \int_{q_0}^q \Phi'_s(q) \Psi_s(q)~\de q = 0.
    \]
    Thus $f_s(q)^{-2} = \fr{(p\times \xi^s \circ \Phi)'(q)}{\Phi'_s(q)}$ is constant on $[q_0,q_1]$.
    By Lemma~\ref{lem:s1-s2-separate} we have $p'=0$ almost everywhere on $[q_1,1]$, hence everywhere by Proposition~\ref{prop:twice-diff}. And by Proposition~\ref{prop:p-basic} we have $p(1)=1$. 
    Thus, for all $q\in [q_1,1]$, 
    \[
        p(1)-p(q) = \int_q^1 p'(q)~\de q = 0,
    \]
    so $p(q)=1$ for all $q\in [q_1,1]$. 
    Finally, by Proposition~\ref{prop:psi} and Lemma~\ref{lem:s1-s2-separate}, \eqref{eq:tree-descending-ode} is satisfied for all $s,s'$ almost everywhere on $[q_1,1]$.
\end{proof}


Given Proposition~\ref{prop:type-12}, it remains to study the behavior of $(p,\Phi)$ separately on $[q_0,q_1]$ and $[q_1,1]$ and establish the root-finding and tree-descending descriptions in Propositions~\ref{prop:root-finding-trajectory} and \ref{prop:tree-descending-trajectory}. We have seen that $(p,\Phi)$ are described by explicit differential equations on $[q_0,q_1]$ and $[q_1,1]$, and it will be important to understand both. We will refer to them as the type $\I$ and $\II$ equations respectively in Subsections~\ref{subsec:type-I-well-posed} and \ref{subsec:type-II}.

\subsection{Behavior in the Root-Finding Phase $1$: Super-solvability of $\Phi(q_1)$}

Let $q_0,q_1$ be given by Proposition~\ref{prop:type-12}, and let $L_s$ be the constant value of $(p\times \xi^s \circ \Phi)'(q)/\Phi'_s(q)$ on $[q_0,q_1]$, which exists by Proposition~\ref{prop:type-12}. 
The goal of this subsection is to prove that $\Phi(q_1)$ is super-solvable.

\begin{lemma}
    \label{lem:phi-q0-positivity}
    We have $\Phi_s(q_0)=0$ if and only if $h_s=0$. 
\end{lemma}
\begin{proof}
    Assume without loss of generality that $s=1$.
    First, suppose $h_1=0$ and $\Phi_1(q_0)>0$. 
    By admissibility, $q_0>0$.
    Consider the perturbation $\wtq_0 = q_0-\delta$, 
    \[
        \wtp(q)=
        \begin{cases}
            q-\wtq_0 & q\in [\wtq_0,q_0] \\
            \delta + (1-\delta)p(q) & q\in [q_0,1] \\
        \end{cases}
        \quad 
        \tPhi_s(q)=
        \begin{cases}
            \fr{q-\wtq_0}{\delta}\Phi_s(q_0) & q\in [\wtq_0,q_0], s=1 \\
            \Phi_s(q_0) & q\in [\wtq_0,q_0], s\neq 1 \\
            \Phi_s(q) & q\in [q_0,1] 
        \end{cases}
    \]
    for all $s\in \sS$.
    Then,
    \[
        \lambda_s
        \int_{\wtq_0}^{q_0}
        \sqrt{\tPhi'_s(q)(\wtp\times \xi^s \circ \tPhi)'(q)}~\de q 
        \ge 
        \begin{cases}
            \Omega(\delta^{1/2}) & s=1 \\
            0 & s\neq 1
        \end{cases}
    \]
    while for all $s\in \sS$, 
    \begin{align*}
        \lambda_s
        \int_{q_0}^1
        \sqrt{\tPhi'_s(q)(\wtp\times \xi^s \circ \tPhi)'(q)}~\de q 
        &\ge 
        \lambda_s
        \int_{q_0}^1
        \sqrt{\tPhi'_s(q)(\wtp\times \xi^s \circ \tPhi)'(q)}~\de q 
        -O(\delta) \\
        h_s\lambda_s \sqrt{\tPhi_s(\wtq_0)} 
        &= 
        h_s\lambda_s \sqrt{\Phi_s(q_0)}.
    \end{align*}
    Thus for small $\delta>0$ the perturbation improves the value of $\bbA$, contradiction.
    
    Conversely, suppose $h_1>0$ and $\Phi_1(q_0)=0$. 
    Consider the perturbation $(\wtp,\tPhi,\wtq_0)$ where $\wtq_0=q_0+\delta$ and $\wtp,\tPhi$ are $p,\Phi$ restricted to $[q_0+\delta,1]$.
    Note that $\tPhi_1(q_0) \ge \Omega(\delta)$ by Proposition~\ref{prop:basic-regularity}. 
    Thus 
    \begin{align*}
        h_1\lambda_1 \sqrt{\tPhi_1(q_0)} - h_1\lambda_1 \sqrt{\Phi_1(q_0)} &\ge \Omega(\delta^{1/2}), \\
        h_s\lambda_s \sqrt{\tPhi_s(q_0)} - h_s\lambda_s \sqrt{\Phi_s(q_0)} &\ge 0 \quad \forall s\neq 1.
    \end{align*}
    Furthermore, for all $s\in \sS$, 
    \begin{align*}
        &\lambda_s \int_{\wtq_0}^1\sqrt{\tPhi'_s(q)(\wtp\times \xi^s \circ \tPhi)'(q)}~\de q 
        - \lambda_s \int_{q_0}^1\sqrt{\Phi'_s(q)(p\times \xi^s \circ \Phi)'(q)}~\de q \\
        &=
        \lambda_s \int_{q_0}^{q_0+\delta}\sqrt{\Phi'_s(q)(p\times \xi^s \circ \Phi)'(q)}~\de q 
        = O(\delta).
    \end{align*}
    Thus for small $\delta>0$ the perturbation improves the value of $\bbA$, contradiction.
\end{proof}
\begin{corollary}
    \label{cor:q0-q1-nonzero}
    If $\vh \neq \vzero$, then $0<q_0<q_1$ and $\Phi(q_1) \in (0,1]^\sS$.
\end{corollary}
\begin{proof}
    Lemma~\ref{lem:phi-q0-positivity} implies $0<q_0$, so Proposition~\ref{prop:p-basic} implies $p(q_0)=0$. 
    Since $p(q_1)=1$ by Proposition~\ref{prop:type-12}, we have $q_0<q_1$.
    Proposition~\ref{prop:basic-regularity} gives $\Phi'(q) \succeq L^{-1}\vone$ for $q\in [q_0,q_1]$, so all coordinates of $\Phi(q_1)$ are positive.
\end{proof}

\begin{lemma}
    \label{lem:q0-q1-zero}
    If $\vh = \vzero$, then $q_0=q_1=0$ (and $\Phi(q_1)=\vzero$).
\end{lemma}
\begin{proof}
    By Lemma~\ref{lem:phi-q0-positivity}, $\Phi(q_0)=\vzero$ so $q_0=0$.
    Suppose that $q_1 > 0$.
    Then, for all $q\in [0,q_1]$, we have $L_s \Phi_s'(q) = (p\times \xi^s \circ \Phi)'(q)$, and by integrating $L_s \Phi_s(q) = p(q)(\xi^s \circ \Phi)(q)$.
    By Assumption~\ref{as:nondegenerate}, we can write $\xi^s(\vx) = \sum_{s'\in \sS} P_{s,s'}(\vx)x_{s'}$ where each $P_{s,s'}$ is a polynomial with nonnegative coefficients and positive constant and linear terms.
    Thus the functions $P_{s,s'} \circ \Phi$ are all strictly increasing.
    Let $0<q<q'<q_1$.
    The linear system
    \[
        L_s\Phi_s(q) x_s
        = 
        \sum_{s'\in \sS}
        p(q)(P_{s,s'} \circ \Phi)(q) \Phi_{s'}(q) x_s
        \quad 
        \forall s\in \sS
    \]
    has solution $\vx = \vone$, while the linear system
    \[
        L_s\Phi_s(q) x_s
        = 
        \sum_{s'\in \sS}
        p(q')(P_{s,s'} \circ \Phi)(q') \Phi_{s'}(q) x_s
        \quad 
        \forall s\in \sS
    \]
    has solution $x_s = \Phi_s(q')/\Phi_s(q)$.
    Monotonicity of $P_{s,s'} \circ \Phi$ implies $p(q')(P_{s,s'} \circ \Phi)(q') \ge p(q)(P_{s,s'} \circ \Phi)(q)$, so Lemma~\ref{lem:pos-linalg-diagonal-must-grow} (with $\eps=0$) implies that $p(q')(P_{s,s'} \circ \Phi)(q') = p(q)(P_{s,s'} \circ \Phi)(q)$ for all $s,s'$.
    This contradicts that the $P_{s,s'} \circ \Phi$ are strictly increasing.
\end{proof}

\begin{lemma}
    \label{lem:gs-value-with-field}
    If $h_s>0$, then $L_s = \fr{h_s^2}{\Phi_s(q_0)}$.
\end{lemma}
\begin{proof}
    Assume without loss of generality that $s=1$. 
    Consider the following perturbation $\tPhi$ of $\Phi$. 
    For all $s\neq 1$, $\tPhi_s=\Phi_s$, and $\tPhi_1(q)=\Phi_1(q)+\delta\psi(q)$  where $\psi \in C^\infty([q_0,1])$ with $\psi(q_0)=1$ and $\psi=0$ on $[q_1,1]$.
    This perturbation is not admissible, but we nonetheless have $\bbA(p,\tPhi;q_0) \le \bbA(p,\Phi;q_0)$ by Lemma~\ref{lem:admissible-optional}.
    
    Recall the calculation \eqref{eq:phi1-deriv}. 
    Integrating by parts, 
    \begin{align*}
        F_1
        &\equiv 2\lambda_1^{-1} \fr{\de}{\de \delta} \bbA(p,\tPhi;q_0)
        \Big|_{\delta=0} \\
        &= 
        \fr{h_1}{\sqrt{\Phi_1(q_0)}}
        +
        \int_{q_0}^1 
        L_1^{1/2} \psi'(q)~\de q
        +
        \sum_{s\in \sS}
        \int_{q_0}^1
        L_s^{1/2}
        (p\psi \times \partial_{x_s}\xi^1 \circ \Phi)'(q)
        = 
        \fr{h_1}{\sqrt{\Phi_1(q_0)}}
        - L_1^{1/2}.
    \end{align*}
    Recall that $\Phi'_1(q)$ is uniformly lower bounded by Proposition~\ref{prop:basic-regularity} and $\Phi_1(q_0)>0$ by Lemma~\ref{lem:phi-q0-positivity}. 
    So, this perturbation is valid for small positive and negative $\delta$. 
    Thus $F_1=0$ which implies the result. 
\end{proof}


\begin{proposition}
    \label{prop:gs-value}
    If $\vh\neq\vzero$, then for all $s$,
    \begin{equation}
        \label{eq:Ls-formula-final}
        L_s = \fr{(\xi^s \circ \Phi)(q_1) + h_s^2}{\Phi_s(q_1)},
    \end{equation}
    which is well-defined by Corollary~\ref{cor:q0-q1-nonzero}.
    Thus, $(p,\Phi)$ satisfies \eqref{eq:root-finding-ode} for all $s\in \sS$, $q\in [q_0,q_1]$ with $\vx = \Phi(q_1)$.
\end{proposition}

\begin{proof}
    Note that $\Phi_s(q_1)>0$ for all $s$ by Corollary~\ref{cor:q0-q1-nonzero} and Proposition~\ref{prop:basic-regularity}.
    Integrating the equation $(p\times \xi^s \circ \Phi)'(q) = L_s\Phi'_s(q)$ on $[q_0+\eps,q]$ and using continuity of $p$ and $\Phi$ and that $p(q_0)=0$, we find
    \begin{equation}
        \label{eq:type1-p-formula}
        p(q)(\xi^s \circ \Phi)(q) = L_s(\Phi_s(q)-\Phi_s(q_0)).
    \end{equation}
    Since $p(q_1)=1$ by Proposition~\ref{prop:type-12}, we have
    \begin{equation}
        \label{eq:type1-ode-aux}
        (\xi^s \circ \Phi)(q_1) = L_s(\Phi_s(q_1)-\Phi_s(q_0)).
    \end{equation}
    If $h_s=0$, by Lemma~\ref{lem:phi-q0-positivity} $\Phi_s(q_0)=0$, so $L_s = (\xi^s \circ \Phi)(q_1)/\Phi_s(q_1)$ as desired. 
    Otherwise, by Lemma~\ref{lem:gs-value-with-field}, $L_s = h_s^2 / (\lambda_s \Phi_s(q_0))$.
    Plugging this into \eqref{eq:type1-ode-aux} implies
    \begin{equation}
        \label{eq:type1-q0-formula}
        \Phi_s(q_0)\lt((\xi^s \circ \Phi)(q_1) + h_s^2\rt) = h_s^2 \Phi_s(q_1).
    \end{equation}
    Thus
    \[
        L_s = \fr{(\xi^s \circ \Phi)(q_1)}{\Phi_s(q_1)-\Phi_s(q_0)}
        = \fr{(\xi^s \circ \Phi)(q_1) + h_s^2}{\Phi_s(q_1)}
    \]
    as desired.
\end{proof}


\begin{corollary}
    \label{cor:alg-value}
    For $(p,\Phi;q_0)$ maximizing $\bbA$, we have
     \begin{equation}
        \label{eq:alg-functional-type1-simplification}
        \bbA(p,\Phi;q_0)
        =
        \sum_{s\in \sS}
        \lambda_s \lt[
            \sqrt{\Phi_s(q_1) (\xi^s(\Phi(q_1)) + h_s^2)}
            +
            \int_{q_1}^1
            \sqrt{\Phi'_s(q)(\xi^s \circ \Phi)'(q)} ~\de q
        \rt].
    \end{equation}
\end{corollary}
\begin{proof}
    If $\vh = \vzero$, then $q_1=0$ by Lemma~\ref{lem:q0-q1-zero}. 
    Thus, $p=1$ on $[0,1]$ by Proposition~\ref{prop:type-12}.
    Thus $(p\times \xi^s \circ \Phi)' = (\xi^s \circ \Phi)'$ and the result is clear. 
    Otherwise $\vh \neq \vzero$, and Corollary~\ref{cor:q0-q1-nonzero} implies $q_1>q_0$. 
    
    If $h_s=0$, then by Lemma~\ref{lem:phi-q0-positivity}, $\Phi_s(q_0)=0$. 
    So,
    \begin{align*}
        h_s\lambda_s \sqrt{\Phi_s(q_0)}
        +
        \lambda_s
        \int_{q_0}^{q_1}
        \sqrt{\Phi'_s(q) (p\times \xi^s \circ \Phi)'(q)}
        ~\de q 
        &= 
        \lambda_s
        \int_{q_0}^{q_1}
        \Phi'_s(q) \sqrt{L_s}
        ~\de q \\
        &= \lambda_s \Phi_s(q_1) \sqrt{L_s} = 
        \lambda_s \sqrt{\Phi_s(q_1) (\xi^s \circ \Phi)(q_1)},
    \end{align*}
    as desired. The last step uses Proposition~\ref{prop:gs-value}. 
    If $h_s>0$, then by Lemma~\ref{lem:gs-value-with-field} and Proposition~\ref{prop:gs-value},
    \begin{align*}
        h_s\lambda_s \sqrt{\Phi_s(q_0)}
        +
        \lambda_s
        \int_{q_0}^{q_1}
        \sqrt{\Phi'_s(q) (p\times \xi^s \circ \Phi)'(q)}
        ~\de q 
        &= 
        \lambda_s \lt[
            \Phi_s(q_0)\sqrt{L_s} + 
            \int_{q_0}^{q_1}\Phi'_s(q) \sqrt{L_s}~\de q 
        \rt] \\
        &= \lambda_s \Phi_s(q_1) \sqrt{L_s} \\
        &= \lambda_s \sqrt{\Phi_s(q_1) \lt((\xi^s \circ \Phi)(q_1) + h_s^2\rt)}.
    \end{align*}
\end{proof}
The following variant of this calculation determines the energy attained by $(p,\Phi;q_0)$ partway through the root-finding phase, and is used in Remark~\ref{rem:type1-partway}.
\begin{corollary}
    \label{cor:type1-partway}
    If $(p,\Phi;q_0)$ maximizes $\bbA$ and $q \in [q_0,q_1]$, then
    \[
        \sum_{s\in \sS} \lambda_s \lt[
            h_s \sqrt{\Phi_s(q_0)} + 
            \int_{q_0}^q \sqrt{\Phi'_s(t) (p\times \xi^s \circ \Phi)'(t)} ~\de t
        \rt]
        = \sum_{s\in \sS} \lambda_s \sqrt{\Phi_s(q) (p(q)(\xi^s \circ \Phi)(q) + h_s^2)}.
    \]
\end{corollary}
\begin{proof}
    If $h_s=0$, then by Lemma~\ref{lem:phi-q0-positivity}, $\Phi_s(q_0)=0$. 
    Then \eqref{eq:type1-p-formula} implies $L_s = p(q)(\xi^s \circ \Phi)(q) / \Phi_s(q)$. 
    So
    \[
        h_s \sqrt{\Phi_s(q_0)} + 
        \int_{q_0}^q \sqrt{\Phi'_s(t) (p\times \xi^s \circ \Phi)'(t)} ~\de t
        = (\Phi_s(q) - \Phi_s(q_0)) \sqrt{L_s} 
        = \sqrt{\Phi_s(q) p(q)(\xi^s \circ \Phi)(q)}.
    \]
    If $h_s>0$, \eqref{eq:type1-p-formula} implies and Lemma~\ref{lem:gs-value-with-field} imply
    \[
        p(q) (\xi^s \circ \Phi)(q) = \fr{h_s^2}{\Phi_s(q_0)} (\Phi_s(q) - \Phi_s(q_0)),
    \]
    which rearranges to 
    \[
        \fr{h_s^2 \Phi_s(q)}{\Phi_s(q_0)} = p(q)(\xi^s \circ \Phi)(q) + h_s^2.
    \]
    Then
    \begin{align*}
        h_s \sqrt{\Phi_s(q_0)} + 
        \int_{q_0}^q \sqrt{\Phi'_s(t) (p\times \xi^s \circ \Phi)'(t)} ~\de t
        &= h_s \sqrt{\Phi_s(q_0)} + (\Phi_s(q)-\Phi_s(q_0)) \sqrt{\fr{h_s^2}{\Phi_s(q_0)}} \\
        &= \fr{h_s \Phi_s(q)}{\sqrt{\Phi_s(q_0)}}
        = \sqrt{\Phi_s(q) (p(q)(\xi^s \circ \Phi)(q) + h_s^2)}.
    \end{align*}
    Summing over $s\in \sS$ completes the proof.
\end{proof}

\begin{lemma}
    \label{lem:q1-(super)-solvable}
    If $q_1=1$, then $\Phi(q_1)=\vone$ is super-solvable.
    If $q_1<1$, then $\Phi(q_1)$ is solvable.
\end{lemma}
\begin{proof}
    First suppose $q_1=1$. 
    Admissibility and the fact that $\Phi(1) \in [0,1]^\sS$ implies $\Phi(q_1)=\vone$.
    We have $p(q_1)=1$ by Proposition~\ref{prop:p-basic} and also $p'(q_1)\ge 0$.
    By Proposition~\ref{prop:gs-value},
    \begin{equation}
        \label{eq:prove-supersolvable}
        \fr{(\xi^s \circ \Phi)(q_1) + h_s^2}{\Phi_s(q_1)}
        = \fr{(p\times \xi^s \circ \Phi)'(q_1)}{\Phi'_s(q_1)}
        \ge \fr{\sum_{s'\in \sS} (\partial_{x_{s'}}\xi^s \circ \Phi)(q_1)\Phi'_{s'}(q_1)}{\Phi'_s(q_1)}.
    \end{equation}
    This implies via Corollary~\ref{cor:solvability-equivalent} (with $\Phi'$ in the role of $\vv$) that $\Phi(q_1)$ is super-solvable. 

    Now suppose $q_1<1$.
    If $\vh=\vzero$ the result follows from Lemma~\ref{lem:q0-q1-zero}, so assume $\vh\neq\vzero$. 
    Because $p(q)=1$ on $[q_1,1]$ and $p'$ is continuous (Proposition~\ref{prop:basic-regularity}), $p'(q_1)=0$. 
    So, the inequality in \eqref{eq:prove-supersolvable} is an equality.
    Thus $\Phi'(q_1)$ is in the null space of $M^*(\Phi(q_1))$, and thus (by \eqref{eq:M*sym-to-M*}) of $M^*_\sym(\Phi(q_1))$.
    So $M^*_\sym(\Phi(q_1))$ is singular and $\Phi(q_1)$ is solvable.
\end{proof}




\subsection{Behavior in the Root-Finding Phase $2$: Well-Posedness}
\label{subsec:type-I-well-posed}

In this subsection we prove Proposition~\ref{prop:root-finding-trajectory} and give a detailed characterization of $(p,\Phi)$ on $[q_0,q_1]$ in Proposition~\ref{prop:type-1}. Recalling Propositions~\ref{prop:gs-value} and \ref{lem:q1-(super)-solvable}, we consider a path $(p,\Phi)$ defined by the \textbf{type $\I$ equation}
\begin{equation}
\label{eq:type-1-traj}
\begin{aligned}
    \fr{(p\times \xi^s \circ \Phi)'(q)}{\Phi'_s(q)} 
    &= L_s= \fr{(\xi^s \circ \Phi)(q_1) + h_s^2}{\Phi_s(q_1)}
    ,\quad\forall s\in\sS
    \\  
    \Phi_s'(q)&\geq 0, \quad 
    \la \vlam, \Phi'(q)\ra = 1
\end{aligned}
\end{equation}
with super-solvable initial condition $\Phi(q_1)$ and $p(q_1)=1$.  We start by verifying the first part of Proposition~\ref{prop:root-finding-trajectory}, namely that $\vh \neq \vzero$ if and only if there exists a super-solvable point $\vx \in [0,1]^\sS$ with $\la \vlam, \vx\ra >0$.


\begin{proof}[Proof of Proposition~\ref{prop:root-finding-trajectory} (first claim)]
    First, assume $\vh \neq \vzero$.
    We will show that all $\vx \in [\delta/2,\delta]^\sS$ are super-solvable for $\delta > 0$ sufficiently small. 
    Assume without loss of generality that $h_1>0$.
    Note that for all $s\in \sS$, 
    \[
        (M^*(\vx) \vx)_s 
        = x_s \lt( h_s^2 + \xi^s(\vx) - \sum_{s'\in \sS} x_{s'} \partial_{x_{s'}} \xi^s(\vx)\rt)
        = x_s \lt( h_s^2 - O(\delta^2)\rt).
    \]
    Moreover, $(M^*(\vx) \ve_1)_1 \le h_1^2 + O(\delta)$, while for $s\neq 1$,
    \[
        (M^*(\vx) \ve_1)_s
        = 
        - x_s \partial_{x_1}\xi^s(\vx).
    \]
    Thus, for $\vv = \vx - \fr12 x_1 \ve_1$, we have 
    \[
        (M^*(\vx) \vv)_1
        \ge x_1 \lt( \fr12 h_1^2 - O(\delta)\rt)
        \ge 0
    \]
    and for $s\neq 1$, 
    \[
        (M^*(\vx) \vv)_s
        \ge 
        x_s\lt( x_1 \partial_{x_1}\xi^s(\vx) - O(\delta^2)\rt)
        \ge 0.
    \]
    This implies by Corollary~\ref{cor:solvability-equivalent} that $\vx$ is super-solvable.

    If $\vh=\vzero$, fix any $\vx \in (0,1]^\sS$. 
    Note that 
    \[
        \vx^\top M^*_\sym(\vx) \vx 
        = \sum_{s\in \sS} x_s \partial_{x_s}\xi(\vx) 
        - \sum_{s,s'\in \sS} x_s x_{s'} \partial_{x_s,x_{s'}}\xi(\vx) < 0,
    \]
    as any monomial of $\xi(\vx)$ with total degree $p\ge 2$ appears with multiplicity $p$ in the first sum and $p(p-1) \ge p$ in the second, with strict inequality for any $p>2$.
    Thus $\vx$ is strictly sub-solvable. 
\end{proof}


\begin{proposition}
\label{proposition:Lambda-def-of-type-I}
    Define for $(p(q),p'(q),\Phi(q))\in [0,1]^{\sS}\times [0,1]\times \mathbb R$ the $\sS\times\sS$ matrix $M(p(q),p'(q),\Phi(q))$ with entries
\[
    M(p(q),p'(q),\Phi(q))_{s,s'}=\frac{p(q)\partial_{x_{s'}}\xi^s\lt(\Phi(q)\rt)
    +
    \lambda_{s'} p'(q)\xi^s(\Phi(q))}{L_s}
    ,\quad\quad
    s,s'\in\sS.
\]
If $(p,\Phi)$ solves \eqref{eq:type-1-traj} then $\Lambda(M(p,p',\Phi))=1$ with Perron-Frobenius eigenvector $\Phi'(q)$.
\end{proposition}

\begin{proof}
    It suffices to expand the left-hand side of the top line of \eqref{eq:type-1-traj}:
\begin{equation}
\label{eq:type-1-traj-rewrite} 
    p'(q)\xi^s(\Phi(q))\left(\sum_{s'\in\sS}\lambda_{s'} \Phi_{s'}'(q)\right) + p(q)\sum_{s'\in\sS} \Phi_{s'}'(q) \partial_{x_{s'}} \xi^s \lt(\Phi(q)\rt)=L_s\Phi_s'(q),\quad \forall s\in\sS.
\end{equation}
    Rearranging shows that $M(p,p',\Phi)\Phi'(q)=\Phi'(q)$, and it is clear that $M$ has non-negative entries.
\end{proof}




We now show the ODE \eqref{eq:type-1-traj} is well-posed.


\begin{lemma}
\label{lem:ODE-Lipschitz}
Fix $q\in [0,1]$ and let $Y(q)=(p(q),\Phi(q))$ and $L_s>0$ be arbitrary. The equation \eqref{eq:type-1-traj} for any fixed $q$ is equivalent to
\[
    Y'(q)=F(Y(q))
\]
for a locally Lipschitz function $F:[0,1]\times \lt([0,1]^r\backslash \vzero\rt)\to \bbR^{r+1}$.
\end{lemma}


\begin{proof}
Let $M$ be as in Proposition~\ref{proposition:Lambda-def-of-type-I}. Because $\xi$ is non-degenerate,
Propositions~\ref{prop:VM} and \ref{proposition:Lambda-def-of-type-I} imply existence of $c>0$ such that 
\[
M(p,x+y,\Phi)\geq M(p,x,\Phi)+cy
\]
holds entrywise for all $x,y\geq 0$, as long as $\Phi(q)\in \bbR_{\geq 0}^{r}\backslash [0,\eps]^{r}$. Therefore a unique value $p'(q)$ solving \eqref{eq:type-1-traj} exists. Moreover $M$ is locally Lipschitz in $(p,\Phi)$, so if
\[
    M(p,p',\Phi)=M(\wt\Phi,\wtp,\wtp')
\]
then
\[
    |p'-\wtp'|\leq O(\|\Phi-\wt\Phi\|_{L^{\infty}}+|p-\wtp|).
\]
(With implicit constant depending on $\eps$ as introduced above.)
This shows that $p'$ has locally Lipschitz dependence on $Y=(p,\Phi)$.
It remains to show $\Phi'$, defined by the resulting solution to \eqref{eq:type-1-traj-rewrite}, also has locally Lipschitz dependence on $Y$. This follows by Proposition~\ref{prop:perron-eigenvector-lipschitz} below. (Note that all entries of $M$ are of the same order up to constants for $\Phi(q)\in \bbR_{\geq 0}^{r}\backslash [0,\eps]^{r}$ by non-degeneracy of $\xi$.)
\end{proof}







\begin{proposition}[{\cite[Lemma 27]{yeo2018frozen}}]
\label{prop:perron-eigenvector-lipschitz}
Let $\cM\subseteq \bbR_{\geq 0}^{r\times r}$ be a compact set of square matrices all of whose Perron-Frobenius eigenvalues have multiplicity $1$. Let $M,\wt M\in\cM$ have entrywise positive Perron-Frobenius eigenvectors $v,\wt v$, normalized so that $\|v\|_1=\|\wt v\|=1$. Then 
\[
    \|v-\wt v\|_{1}\leq O_{\cM}(\|M-\wt M\|_1).
\]
In particular, this holds for $\cM=[c,C]^{r\times r}$ for any $0<c<C<\infty$.
\end{proposition}



Lemma~\ref{lem:ODE-Lipschitz} shows that for any right endpoint $(p(q_1),\Phi(q_1))$, it is possible to solve \eqref{eq:type-1-traj} backwards in time until $q_*$ when $\Phi(q)$ reaches the boundary of $\bbR_{\geq 0}^r$, or at which $p(q)$ reaches $0$. We now show that the latter occurs first.





\begin{lemma}
\label{lem:Phi-linear-LB}
There exists $c>0$ such that for any super-solvable point $\Phi(q_1)$, the solution to the type $\I$ equation \eqref{eq:type-1-traj} on $[q_*,q_1]$ satisfies
\[
    \Phi_{s}(q)\geq cp(q)q.
\]
Moreover $p(q_*)=0$ and Lemma~\ref{lem:phi-q0-positivity} holds for $q_*$, i.e. $h_s>0$ if and only if $\Phi_s(q_*)>0$.
\end{lemma}


\begin{proof}
Observe that in \eqref{eq:type-1-traj}, we have
\[
    L_s\geq K_s\equiv \frac{\xi^s(\Phi(q_1))}{\Phi_s(q_1)}.
\]
Therefore on $q\in[q_{\eps},q_1]$, the left-hand equation in \eqref{eq:type-1-traj} implies
\[
    \frac{(p\times \xi^s\circ \Phi)(q)}{\Phi_s(q)}
    \leq 
    K_s.
\]
Recall that $\xi^s$ is non-degenerate, and so admissibility and $\Phi\succeq 0$ implies $\xi^s(\Phi(q))=\Theta(q).$ Hence for some $c>0$ and all $s\in\sS$,
\[
    \Phi_{s}(q)\geq \Omega(p(q)q/K_s)\geq cp(q)q.
\]
This concludes the proof of the first statement, which implies that $p(q_*)=0$. 

For the second, note that strict inequality holds in the first step if $h_s>0$, and so $p$ must reach $0$ before $\Phi_s$ does. On the other hand if $h_s=0$, then it is easy to see from \eqref{eq:type-1-traj} that $p$ cannot reach zero strictly sooner than $\Phi_s$, hence the numerator and denominator on the left-hand side in \eqref{eq:type-1-traj} both reach zero at time $q_*$.
\end{proof}

\begin{lemma}
\label{lem:p-concave}
If $\Phi(q_1)$ is super-solvable, then the $p$ solving \eqref{eq:type-1-traj-rewrite} is increasing and concave on $[q_*,q_1]$.  Moreover $p,\Phi_s\in C^1([q_*,q_1])$.
\end{lemma}

\begin{proof}
We claim that $p'$ is decreasing. The key point is that with $M$ as in Proposition~\ref{proposition:Lambda-def-of-type-I}, 
\[
    M(p,p',\Phi)< M(\wt\Phi,\wtp,\wtp')
\]
if $\Phi \preceq \wt\Phi$, $p \leq \wtp$ and $p'<\wtp'$. 
Indeed this is immediate by Proposition~\ref{prop:VM}.
It follows that $p'$ must increase backward in time, i.e. $p'(q)$ is a decreasing function. Since $p'(q_1)\geq 0$ by super-solvability, this completes the proof.
\end{proof}


\begin{proof}[Proof of Proposition~\ref{prop:root-finding-trajectory}, parts (\ref{it:unique-root-finding},\ref{it:unique-root-finding-q0})]
    Existence and uniqueness of the root-finding trajectory follows from Lemma~\ref{lem:ODE-Lipschitz} and Proposition~\ref{prop:ODE-well-posed}. Lemma~\ref{lem:Phi-linear-LB} ensures that the solution exists until $p$ reaches $0$. Concavity of $p$ was just shown in Lemma~\ref{lem:p-concave}. 
    This proves part (\ref{it:unique-root-finding}). 
    Part (\ref{it:unique-root-finding-q0}) follows from Lemma~\ref{lem:phi-q0-positivity} or \ref{lem:Phi-linear-LB}.
\end{proof}


\begin{lemma}
    \label{lem:vone-super-solvable}
    If $\vone$ is super-solvable, then $q_1=1$ and $\Phi(q_1)=\vone$.
    Otherwise $q_1 < 1$.
\end{lemma}
\begin{proof}
    If $\vone$ is strictly sub-solvable, Lemma~\ref{lem:q1-(super)-solvable} implies that $q_1<1$.
    Suppose $\vone$ is super-solvable.
    Let $(p^*,\Phi^*,q_0^*)$ be the root-finding trajectory with endpoint $\vone$, which exists by Proposition~\ref{prop:root-finding-trajectory}.
    By Corollary~\ref{cor:alg-value},
    \[
        \bbA(p^*,\Phi^*;q_0^*) 
        = 
        \sum_{s\in \sS} 
        \lambda_s 
        \sqrt{\xi^s(\vone) + h_s^2}.
    \]
    Suppose for contradiction that there is a different maximizer $(p,\Phi,q_0)$ of $\bbA$ with $\bbA(p,\Phi;q_0) \ge \bbA(p^*,\Phi^*;q_0^*)$. The maximizer $(p,\Phi,q_0)$ has its own value $q_1$, and we must have $q_1<1$ since for this to be a different maximizer.
    Note that for each $s\in \sS$, 
    \begin{align*}
        &\sqrt{\xi^s(\vone) + h_s^2} - \sqrt{\Phi_s(q_1)(\xi^s(\Phi(q_1)) + h_s^2)} 
        = \int_{q_1}^1 
        \fr{\de}{\de q}
        \sqrt{\Phi_s(q)((\xi^s\circ\Phi)(q) + h_s^2)}
        ~\de q \\
        &= \fr12 \int_{q_1}^1 \lt(
            \Phi'_s(q)
            \sqrt{\fr{(\xi^s\circ\Phi)(q) + h_s^2}{\Phi_s(q)}}
            + (\xi^s \circ \Phi)'(q) 
            \sqrt{\fr{\Phi_s(q)}{(\xi^s\circ\Phi)(q) + h_s^2}}
        \rt) ~\de q.
    \end{align*}
    By Corollary~\ref{cor:alg-value},
    \begin{align*}
        F &\equiv 
        \bbA(p^*,\Phi^*;q_0^*)
        -
        \bbA(p,\Phi;q_0) \\
        &=
        \sum_{s\in \sS}
        \fr{\lambda_s}{2}
        \int_{q_1}^1
        (\xi^s \circ \Phi)'(q) 
        \sqrt{\fr{\Phi_s(q)}{(\xi^s\circ\Phi)(q) + h_s^2}}
        \lt(
            \sqrt{
                \fr{\Phi'_s(q)}{(\xi^s\circ\Phi)'(q)}
                \cdot 
                \fr{(\xi^s\circ\Phi)(q) + h_s^2}{\Phi_s(q)}
            }
            - 1
        \rt)^2
        \de q
        \ge 0.
    \end{align*}
    Since $\bbA(p,\Phi;q_0) \ge \bbA(p^*,\Phi^*;q_0^*)$, we have $F=0$.
    So, for all $s\in \sS$, and almost all $q\in (q_1,1]$
    \[
        \fr{(\xi^s\circ\Phi)'(q)}{(\xi^s\circ\Phi)(q) + h_s^2}
        =
        \fr{\Phi'_s(q)}{\Phi_s(q)}
        \qquad 
        \Rightarrow 
        \qquad 
        \fr{\de}{\de q} \log \lt((\xi^s\circ\Phi)(q) + h_s^2\rt) 
        = 
        \fr{\de}{\de q} \log \Phi_s(q).
    \]
    Both sides of this equation are continuous on $(q_1,1]$, so in fact it holds for all $q\in (q_1,1]$.
    Thus there exist constants $C_s$ such that
    \[
        (\xi^s\circ\Phi)(q) + h_s^2
        = 
        C_s \Phi_s(q).
    \]
    Thus, for $q_1 < q < q+\iota \le 1$, we have
    \begin{align*}
        C_s\Phi'_s(q)
        &= 
        (\xi^s \circ \Phi)'(q) 
        =
        \sum_{s'\in \sS}
        (\partial_{x_{s'}} \xi^s \circ \Phi)(q)
        \Phi'_{s'}(q)
        \quad \forall s\in \sS, \\ 
        C_s\Phi'_s(q+\iota)
        &=
        (\xi^s \circ \Phi)'(q+\iota) 
        =
        \sum_{s'\in \sS}
        (\partial_{x_{s'}} \xi^s \circ \Phi)(q+\iota)
        \Phi'_{s'}(q+\iota)
        \quad \forall s\in \sS,
    \end{align*}
    We treat these equations as linear systems in $\Phi'(q)$ and $\Phi'(q+\iota)$.
    Since both linear systems have nonnegative solutions and $(\partial_{x_{s'}} \xi^s \circ \Phi)(q+\iota) \ge (\partial_{x_{s'}} \xi^s \circ \Phi)(q)$ for all $s,s'$, Lemma~\ref{lem:pos-linalg-diagonal-must-grow} (with $\eps=0$) implies that $(\partial_{x_{s'}} \xi^s \circ \Phi)(q+\iota) = (\partial_{x_{s'}} \xi^s \circ \Phi)(q)$ for all $s,s'$. 
    This contradicts that $\xi$ is non-degenerate and completes the proof.
\end{proof}


\begin{proposition}
    \label{prop:type-1}
    The following assertions hold. 
    \begin{enumerate}[label=(\alph*), ref=\alph*]
        \item \label{itm:s4-supsolvable} If $\vone$ is super-solvable, then $0 < q_0 < q_1 = 1$ (and thus $\Phi(q_1)=\vone$).
        \item \label{itm:s4-subsolvable-with-field} If $\vone$ is sub-solvable and $\vh \neq \vzero$, then $0 < q_0 < q_1 < 1$ and $\Phi(q_1) \in (0,1]^\sS$.
        \item \label{itm:s4-subsolvable-no-field} If $\vh = \vzero$, then $\vone$ is sub-solvable and $0=q_0=q_1$ (and thus $\Phi(q_1) = \vzero$).
    \end{enumerate}
    In cases (\ref{itm:s4-subsolvable-with-field}, \ref{itm:s4-subsolvable-no-field}), $\Phi(q_1)$ is solvable. 
    In cases (\ref{itm:s4-supsolvable}, \ref{itm:s4-subsolvable-with-field}) (and vacuously in case (\ref{itm:s4-subsolvable-no-field})) $(p,\Phi)$ restricted to $[q_0,q_1]$ is the root-finding trajectory with endpoint $\Phi(q_1)$.
\end{proposition}

\begin{proof}[Proof of Proposition~\ref{prop:type-1}]
    If $\vone$ is super-solvable, Lemma~\ref{lem:vone-super-solvable} implies $q_1=1$. 
    Comparing Corollary~\ref{cor:q0-q1-nonzero} and Lemma~\ref{lem:q0-q1-zero} gives $q_0>0$.
    If $\vone$ is sub-solvable and $\vh\neq\vzero$, Lemma~\ref{lem:vone-super-solvable} implies $q_1<1$ while Corollary~\ref{cor:q0-q1-nonzero} implies $0<q_0<q_1$ and $\Phi(q_1) \in (0,1]^\sS$. 
    If $\vh=\vzero$, Lemma~\ref{lem:q0-q1-zero} implies $0=q_0=q_1$.
    This proves assertions (\ref{itm:s4-supsolvable}, \ref{itm:s4-subsolvable-with-field}, \ref{itm:s4-subsolvable-no-field}). 

    In cases (\ref{itm:s4-subsolvable-with-field}, \ref{itm:s4-subsolvable-no-field}), since $q_1<1$, Lemma~\ref{lem:q1-(super)-solvable} implies $\Phi(q_1)$ is solvable.
    In cases (\ref{itm:s4-supsolvable}, \ref{itm:s4-subsolvable-with-field}), Proposition~\ref{prop:gs-value} implies $(p,\Phi)$ restricted to $[q_0,q_1]$ is the root-finding trajectory with endpoint $\Phi(q_1)$. 
\end{proof}

\subsection{Behavior in the Tree-Descending Phase}
\label{subsec:type-II}


The next lemma, proved in Appendix~\ref{subsec:type-II-Lipschitz}, shows the tree-descending ODE is also well-posed. 

\begin{restatable}{lemma}{lemtypeIILipschitz}
\label{lem:type-II-Lipschitz}
Fix $\eps>0$. For $\Phi(q)\in \bbR_{\geq 0}^{\sS}$ and $\Phi'(q)\in A_{\geq 0}(q)$, the \textbf{type $\II$ equation}
\begin{align*}
    \Psi_s(q) &=\Psi_{s'}(q)
    \quad\forall s,s'\in\sS
    ;
    \\
    \la \vec\lambda,\Phi''(q)\ra &= 0
\end{align*}
is equivalent (for each fixed $q$) to 
\[
    \Phi''(q)=F(\Phi(q),\Phi'(q))
\]
for a locally Lipschitz function $F:\mathbb R_{\geq 0}^{\sS}\times A_{\geq 0}^{\sS}\to\mathbb R^{\sS}$. Moreover, \begin{equation}
\label{eq:Phi-stays-increasing}
|\Phi''_s(q)|\leq O(|\Phi_s'(q)|),\quad \forall s\in \sS.
\end{equation}
with a uniform constant for bounded $\Phi'(q)$.
\end{restatable}



\begin{lemma}
\label{lem:type-II-well-posed-appendix}
    The type $\II$ equation has a unique solution on $q\in [q_1,1]$ for any initial condition $(\Phi(q_1),\Phi'(q_1))\in \bbR_{\geq 0}^{\sS}\times A_{\geq 0}$. This solution satisfies $\Phi'(q)\succeq 0$ for all $q$.
\end{lemma}


\begin{proof}
The result now follows from Proposition~\ref{prop:ODE-well-posed}, since \eqref{eq:Phi-stays-increasing} implies that $\Phi'_s(q)$ stays non-negative for all $s$, and stays strictly positive if $\Phi_s'(q_1)>0$. 
\end{proof}


\begin{proof}[Proof of Proposition \ref{prop:tree-descending-trajectory}]
    Given the above, it only remains to show existence and uniqueness of $\vv$. Consider the matrix
    \[
    M(\vx)_{s,s'}
    =
    \frac{\partial_{x_{s'}}\xi^s(\vx)}
    {\xi^s(\vx)+h_s^2}.
    \]
    Then $M$ has strictly positive entries by non-degeneracy.
    The equation $M^*_\sym(\vx)\vv=\vzero$ is equivalent to $M^*(\vx)\vv=\vzero$ by \eqref{eq:M*sym-to-M*}, which is in turn equivalent to
    \[
    M(\vx)\vv=\vv.
    \]
    Since $\vx\neq \vzero$, non-degeneracy of $\xi$ implies that $\xi^s(\vx)>0$ so there is no division by $0$. Hence any such $\vv$ is uniquely determined as the Perron-Frobenius eigenvector of $M$.
    Conversely it is easy to see that if $M$ has Perron-Frobenius eigenvector \textbf{not} equal to $1$ then $M^*$ would not be solvable, which ensures that $\vv$ as above exists. 
\end{proof}



\begin{corollary}
\label{cor:regular-final}
    $p,\Phi_s\in C^1([q_0,1])$ and their restrictions to $[q_1,1]$ are $C^2$.
\end{corollary}

\begin{proof}
    From Proposition~\ref{prop:basic-regularity}, for the first statement it suffices to verify continuity of $p',\Phi_s'$ at $q_0$. If $\vh\neq \vzero$ this follows by Lemmas~\ref{lem:ODE-Lipschitz} and \ref{lem:p-concave}. If $\vh=\vzero$ this and the second conclusion both follow from Lemmas~\ref{lem:q0-q1-zero} and \ref{lem:type-II-well-posed-appendix}.
\end{proof}

The statement of Theorem~\ref{thm:alg-optimizer} is a combination of many of the results established in this section.

\begin{proof}[Proof of Theorem~\ref{thm:alg-optimizer}]
Existence of a maximizer $(p,\Phi;q_0)$ was shown in Proposition~\ref{prop:F-max}, and such $p,\Phi$ are continuously differentiable on $[q_0,1]$ by Corollary~\ref{cor:regular-final}.
The value $q_1$ was identified in Lemma~\ref{lem:s1-s2-separate}. 
The behavior on $S_1=[q_0,q_1]$ and $S_2=[q_1,1]$ comes directly from the well-posedness of the corresponding ODEs as shown in Lemmas~\ref{lem:ODE-Lipschitz} and \ref{lem:type-II-well-posed-appendix}. 
The formula~\eqref{eq:alg-for-optimizer} was proved in Corollary~\ref{cor:alg-value}.
The last assertions follow from Proposition~\ref{prop:type-1}.
\end{proof}


We finally prove a slight generalization of Proposition~\ref{prop:type-II-locally-unique}. Recall that $\Delta^r\subseteq \bbR_{\geq 0}^r$ denotes the simplex of admissible $\Phi'$ vectors. For any initial point $\vx$ and time-increment $t>0$, solving the type $\II$ equation yields a map $F_{\vx,t}:\Delta^r\to \Delta^r$ given by
\begin{equation}
\label{eq:F-vx-t}
    F_{\vx,t}(\vv)=(\Phi(q+t)-\vx)/t
\end{equation}
where $\Phi$ solves the type $\II$ equation with initial condition $\Phi(q)=\vx$, $\Phi'(q)=\vv$. 



We remark that in the case $\vx=0$ of Proposition~\ref{prop:type-II-locally-unique}, surjectivity also follows simply by taking $(p,\Phi;q_0)$ maximizing a version of $\bbA$ rescaled to have an arbitrary endpoint. 


\begin{corollary}
\label{cor:type-II-locally-unique}
Assume $\xi$ is non-degenerate. For $C>0$, there exists $\eps=\eps(C)$ such that the map $F_{\vx,t}$ defined in \eqref{eq:F-vx-t} is injective for $t\in [0,\eps]$ and $\|\vx\|_1\leq C$. Moreover $F_{\vx,t}$ is always surjective.
\end{corollary}


\begin{proof}
    An easy Gr{\"o}nwall argument using \eqref{eq:Phi-stays-increasing} implies that for $0\leq t\leq \eps$, 
    \[
    \la \Phi(q+t)-\wt\Phi(q+t),\Phi'(q)-\wt\Phi'(q)\ra>0
    \]
    for any pair $(\Phi,\wt\Phi)$ of solutions to the type $\II$ equation with $\Phi(q)=\wt\Phi(q)$ and $\Phi'(q)\neq\wt\Phi'(q)$. This implies injectivity. Surjectivity follows from Lemma~\ref{lem:topology} since \eqref{eq:Phi-stays-increasing} implies that if $\vv_s=0$ then $F_{\vx,t}(\vv)_s=0$.
\end{proof}



\begin{lemma}[{\cite[Lemma 2.1]{jamison1976factoring} or \cite[Lemma 1]{karasev2009kkm}}]
\label{lem:topology}
    Let $F$ be a continuous map from $\Delta^r$ to itself such that $F(\vv)_s=0$ if $v_s=0$. Then $F$ is surjective.
\end{lemma}





\subsection{Explicit Solution for Pure Models}
\label{subsec:pure}


In this subsection we prove Theorem~\ref{thm:pure} and Corollary~\ref{cor:pure}, obtaining an explicit description of $\hALG$ in the important special case of \emph{pure} models for which
\begin{equation}
\label{eq:pure-mixture}
    \xi(x_1,\dots,x_r)=\prod_{s\in\sS} x_s^{a_s}.
\end{equation}
Due to the homogeneity and lack of external field, it is natural to expect that the optimal $(p,\Phi)$ is given by $p\equiv 1$ and $\Phi(q)=(q^{b_1},\dots,q^{b_r})$ for positive constants $b_s$. (Here we do not require $\Phi$ to be admissible, which by Lemma~\ref{lem:admissible-optional} does not make a difference.) Most of our previous results do not apply directly because $\xi$ violates the non-degeneracy condition, however as mentioned previously we can apply them after adding a small perturbation.


\begin{lemma}
\label{lem:pure-p=1}
For a pure model described by $\xi$, there exists $\Phi^*$ such that with $p\equiv 1$,
\[
   \bbA(p,\Phi^*;0)=\hALG. 
\]
\end{lemma}

\begin{proof}
Let 
\[
\xi^{(\eps)}(\vx)=\xi(\vx)+\eps\sum_{s,s'\in\sS} x_sx_{s'}+\eps\sum_{s,s',s''\in\sS}x_s x_{s'}x_{s''}.
\]
Then the preceding results show that optimal solutions $(\Phi^{(\eps)},p^{(\eps)},q_0^{(\eps)})$ for $\xi^{(\eps)}$ satisfy $p^{(\eps)}\equiv 1$ and $q_0^{\eps(\eps)}=0$. Taking a convergent subsequence $\Phi^{(\eps)}\to\Phi^*$ as $\eps\to 0$ in the space $\cM$ (shown to be compact in Appendix~\ref{subsec:maximizer-existence}) implies the result since $\hALG$ is continuous in $\xi$.
\end{proof}



We first non-rigorously guess the solution by assuming it is of the form \eqref{eq:pure-mixture} and also solves the type $\II$ equation. By homogeneity, we may assume 
\begin{equation}
\label{eq:B=1}
    \sum_{s\in\sS}a_{s}b_{s}=1.
\end{equation}
Then
\begin{align*}
    \Phi_s'(q)&=b_s q^{b_s-1},
    \\
    (\xi^s\circ\Phi)(q)
    &=
    \frac{a_s}{\lambda_s}q^{1-b_{s}}.
\end{align*}
We thus expect that for some constant $L$ independent of $s$,
\begin{align*}
    \Psi_s(q) &=
    b_s^{-1} q^{1-b_s}
    \deriv{q}\sqrt{\frac{b_s q^{b_s-1}}{q^{1-b_s-1}a_s(1-b_s)/\lambda_s}}
    \\
    &=
    \sqrt{\frac{\lambda_s}{a_s(1-b_s)b_s}}q^{1-b_s}
    \deriv{q}{q^{-\frac{1}{2}+b_s}}
    \\
    &=
    \lt(-\frac{1}{2}+b_s\rt)\sqrt{\frac{\lambda_s}{a_s(1-b_s)b_s}}
    q^{-1/2}
    \\
    &=
    -L^{-1/2}q^{-1/2}.
\end{align*}
(Recall that $\Psi_s$ should be negative.) The resulting quadratic equation in $b_s$ has solution
\begin{equation}
\label{eq:pure-L}
    b_s=\frac{1- \sqrt{\frac{a_s}{a_s+L\lambda_s}}}{2}.
\end{equation}
Finally $L$ is chosen to satisfy \eqref{eq:B=1}; it is easy to see there is a unique such choice. 


Our next step is to verify the computation above and prove uniqueness. 


\begin{proof}[Proof of Theorem~\ref{thm:pure}]
~\\
\paragraph{Part $1$: Value of $\ALG$}
%
Here we assume $p\equiv 1$, relying on Lemma~\ref{lem:pure-p=1}, and determine the value $\ALG$. Using the purity of $\xi$, a simple scaling argument shows the $\hALG$ value with endpoint $\vx=(x_1,\dots,x_r)$ (cf. Remark~\ref{rem:normalization})
is given by
\begin{equation}
\label{eq:pure-scaling}
    \hALG(\vx)=\hALG(\vone)
    \cdot
    \prod_{s\in \sS}x_s^{a_s/2}.
\end{equation}
(Recall that $\xi$ is a covariance, hence the factor $1/2$ in the exponent on the right-hand side.)
Set $\phi_{D-1}^s=1-b_s \delta$ for small $\delta$ and $\vb\succeq 0$ satisfying \eqref{eq:B=1}. This is a fully general choice for $\phi_{D-1}$ as in Section~\ref{sec:uc}. In light of Proposition~\ref{prop:what-F-is}, we obtain that for small $\delta>0$,
\begin{equation}
\label{eq:pure-DP}
    \hALG(\vone)
    =
    \max_{\vb\,:\,\eqref{eq:B=1}}
    \big(\hALG(\phi_{D-1})
    +
    \delta\sum_{s\in\sS} \lambda_s\sqrt{\lambda_s^{-1}a_sb_s(1-b_s)}
    \big)
    +o(\delta).
\end{equation}
Denoting $\hALG=\hALG(\vone)$ and using \eqref{eq:pure-scaling}, we find 
\begin{align*}
    \hALG&=
    \max_{\vb\,:\,\eqref{eq:B=1}}
    \Big(\hALG\cdot\prod_{s\in\sS} (1-b_s\delta)^{a_i/2} + \delta\sum_{s\in\sS} \lambda_s\sqrt{\lambda_s^{-1}a_sb_s(1-b_s)}
    \Big)
    +o(\delta)
    \\
    &=
    \max_{\vb\,:\,\eqref{eq:B=1}}
    \bigg(\lt(1-\frac{\delta}{2}\rt) \hALG+\delta\sum_{s\in\sS} \sqrt{\lambda_s a_s b_s (1-b_s)}
    \bigg)+o(\delta).
\end{align*}
Rearranging and sending $\delta\to 0$ yields
\begin{equation}
\label{eq:ALG-pure-max}
    \hALG=
    2\max_{\vb\,:\,\eqref{eq:B=1}} 
    \sum_{s\in\sS} 
    \sqrt{\lambda_s a_s b_s (1-b_s)}.
\end{equation}
First, it is easy to see that any maximizing $\vb^*$ has $b_s^*>0$ for all $s$, since otherwise the derivative of the right-hand side in $b_s$ would be infinite. By Lagrange multipliers, for some $C>0$ any solution will have
\begin{equation}
\label{eq:LM-formula}
\begin{aligned}
    \sqrt{\frac{a_s}{L\lambda_s}}
    &=\deriv{b_s}\lt(\sqrt{b_s(1-b_s)}\rt)
    \\
    &=\frac{\frac{1}{2}-b_s}{\sqrt{b_s(1-b_s)}}
\end{aligned}
\end{equation}
for some $L\in [0,\infty]$ (where division by $\infty$ gives $0$).

 
Let us first assume $\sum_{s\in\sS} a_s\geq 3$. Then \eqref{eq:B=1} implies that $b_s<1/2$ for some $s$, hence for all $s$ since the signs have to match in \eqref{eq:LM-formula}. In particular we have $L<\infty$, and \eqref{eq:pure-L} above easily follows from \eqref{eq:LM-formula}.
The resulting formula is as desired:
\begin{align*}
    \hALG
    &=
    2\sum_{s\in \sS}
    \sqrt{\lambda_s a_s}\cdot \lt(\frac{1}{2}-b_s\rt)\sqrt{\frac{L\lambda_s}{a_s}}
    \\
    &=
    \sum_{s\in\sS}
    \lambda_s
    \sqrt{\frac{ L a_s}{L\lambda_s+a_s}}
    .
\end{align*}



The only remaining case is $\xi(x_1,x_2)=x_1 x_2$. Then it is clear from \eqref{eq:ALG-pure-max} that $b_1=b_2=1/2$ and 
\[
\hALG=\sqrt{\lambda_1}+\sqrt{\lambda_2}.
\]
(This case of Theorem~\ref{thm:pure} is stated with $b_1=b_2=1$ which is an equivalent parametrization.)

\paragraph{Part $2$: Uniqueness Assuming $p\equiv 1$}
%
Next we show the optimal trajectory $\Phi^*(q)=(q^{b_1},\dots,q^{b_r})$ is unique up to reparametrization when $p\equiv 1$. The maximization problem in \eqref{eq:ALG-pure-max} is strictly convex on the affine subspace defined by \eqref{eq:B=1}, and hence has a unique minimizer. It follows that if $\phi_d$ in the preceding equation is defined by any choice $\vb$ bounded away from the optimal one, the obtained value would be strictly worse than $\ALG$. In other words, any optimal trajectory where $p\equiv 1$ must satisfy $\Phi'(1)=\vb$. By scale-invariance, we conclude that $\Phi(q)=(q^{b_1},\dots,q^{b_r})$ is the unique optimal such trajectory.



\paragraph{Part $3$: Uniqueness of Optimal $p$}
%
Finally we prove that all optimal solutions actually satisfy $p\equiv 1$.
Suppose another maximizer $(p,\Phi)$ exists. Let
\[
    q_*=\inf_{q>0}\{q~:~\min_{s\in\sS}\Phi_s(q)>0\}.
\]
The definition of $p$ on $[0,q_*)$ is irrelevant so we assume without loss of generality that $p$ is constant on $[0,q_*]$ and continuous at $q_*$. It is easy to see that such a maximizing $p$ must be continuous on all of $[0,1]$ and satisfy $p(1)=1$; otherwise $p$ could be strictly increased while keeping $p'$ constant for the purposes of $\bbA$. The proof of Lemma~\ref{lem:p-AC} implies that $p$ is uniformly Lipschitz on $[q_*+\eps,1]$ for any $\eps>0$, so that $p'$ makes sense as a measurable function. 

We have seen that if $p\equiv 1$ then $\ALG$ is achieved by a unique $\Phi$, so we remains to show that no optimal $(p,\Phi)$ satisfies $p\not\equiv 1$  Assuming that $p\not\equiv 1$ we may choose $q>q_*$ a Lebesgue point for both $p'$ and $\Phi'$ such that
\[
    p'(q)>0.
\]
We now derive a contradiction by expanding $\hALG$ around $q$ as in \eqref{eq:pure-DP}. In particular, consider $\phi_d=\Phi(q-\delta)$ and $p_d=p(q-\delta)$.
Let $\Delta_s=\Phi_s(q)-\phi_{d,s}$ and $\Delta_p=p(q)-p_d$. Since $q$ is a Lebesgue point, we have $\Delta_s=\Phi_s'(q)\delta+o(\delta)$ and $\Delta_p=p'(q)+o(\delta)$.


The computation above for the value $\hALG$ implies 
\[
    \hALG(p_d,\phi_d)
    =
    \hALG(\phi_d)\sqrt{p_d}
    .
\]
Here $\hALG(p_d,\phi_d)$ denotes the analog of \eqref{eq:alg} with endpoint value $\Phi(q)=\phi_d$ rather than $q=1^{\sS}$, and $p(q)=p_d$.
Therefore
\begin{align*}
    \hALG(p(q),\Phi(q))
    &=
    \hALG(\phi_d)
    \sqrt{p_d}
    +
    \sum_{s\in\sS}
    \lambda_s
    \sqrt{\Delta_s \lt(\Delta_p \xi^s(\phi_d)+p_d \sum_{s'\in\sS} \partial_{x_{s'}}\xi^s(\phi_d)\Delta_{s'} \rt)}
    +
    o(\delta)
    \\
    &=
     \hALG(p(q),\Phi(q))
    \cdot
    \lt(1-\frac{\delta}{2}\times\lt(\frac{p'(q)}{p(q)}+\sum_{s\in\sS} \frac{a_s \Phi_s'(q)}{\Phi_s(q)}\rt)\rt)
    \\
    &\quad\quad
    +
    \delta\sum_{s\in\sS}
    \lambda_s
    \sqrt{\Phi'_s(q) \lt(p'(q) (\xi^s\circ\Phi)(q)+p(q) \sum_{s'\in\sS} \partial_{x_{s'}}(\xi^s\circ\Phi)(q)\Phi_{s'}'(q) \rt)}
    +
    o(\delta)
    .
\end{align*}
Rearranging and sending $\delta\to 0$ implies
\begin{equation}
\label{eq:ALG-pure-recursion}
    \hALG(p(q),\Phi(q))/2
    =
    \frac{
    \sum_{s\in\sS}
    \lambda_s
    \sqrt{\Phi'_s(q) \lt(p'(q) (\xi^s\circ\Phi)(q)+p(q) \sum_{s'\in\sS} \partial_{x_{s'}}(\xi^s\circ\Phi)(q)\Phi_{s'}'(q) \rt)}
    }
    {
    \frac{p'(q)}{p(q)}+\sum_{s\in\sS} \frac{a_s \Phi_s'(q)}{\Phi_s(q)}
    }
    \,.
\end{equation}
We claim that \eqref{eq:ALG-pure-recursion} forces $p'(q)=0$, which completes the proof of uniqueness since $q$ was an arbitrary choice of Lebesgue point. Note that from any solution to \eqref{eq:ALG-pure-recursion} we immediately get a maximizing $(\Phi,p)$ for $\bbA$ where $p(q)$ and each $\Phi_s(q)$ is a monomial of the form $aq^b$. 



The right-hand side above has maximum value $\hALG(p(q),\Phi(q))/2$, and we already know from Lemma~\ref{lem:pure-p=1} there exists $(p'(q),\Phi'(q))$ achieving this value with $p'(q)=0$. Supposing another maximizing $(\wt p'(q),\wt\Phi'(q))$ with $\wt p'(q)>0$ exists, 
we suppress the input $q$ and consider a general solution 
\[
    (p_a',\Phi_a')=\big(ap_1'-(a-1)p_0',a\Phi_1'-(a-1)\Phi_0' \big).
\]
We always restrict to $a$ such that all derivatives are non-negative. The denominator of the right-hand side of \eqref{eq:ALG-pure-recursion} is affine in $a$, while Lemma~\ref{lem:sqrt-xy-concave} implies the numerator is concave. Since $(p_0',\Phi_0')$ and $(p_1',\Phi_1')$ both maximize the right-hand side we deduce that it takes the constant value $\hALG(p(q),\Phi(q))/2$ on $(p_a',\Phi_a')$ for all $a\in [0,1]$. In particular using again Lemma~\ref{lem:sqrt-xy-concave} we find that each of the $r$ terms in the numerator is actually a linear function of $a$ on the interval such that 
\begin{equation}
\label{eq:good-a}
    p_a'(q)\geq 0, \quad\text{and}\quad 
    \min_s \Phi_{a,s}'(q)\geq 0.
\end{equation}
This means equality is achieved for $p_a$ for $a$ satisfying \eqref{eq:good-a} (even if $a>1$) and implies that $\Phi'(q)\neq \wt\Phi'(q)$. Let $a_*>0$ be the maximal value satisfying \eqref{eq:good-a}, so that $\min_s \Phi_{a_*,s}'(q)=\Phi_{a_*,s_*}'(q)=0$. Then clearly the $s_*$ term of the numerator is not affine on $a\in [a_*-\eps,a_*]$; since $\Phi_{a_*}'$ satisfies admissibility it does not equal $\vzero$. This gives a contradiction, so we conclude that $p\equiv 1$ holds for all optimal $(p,\Phi)$.  
\end{proof}

\begin{proof}[Proof of Corollary~\ref{cor:pure} ]
    Here we have $\lambda_s=\frac{a_s}{\sum_{s\in\sS} a_s}$ in the preceding formulas. It is easy to see from \eqref{eq:pure-L} that the values $b_s$ are all equal. From \eqref{eq:B=1} we find $b_s=\frac{1}{\sum_{s\in\sS} a_s}$ and so
\begin{align*}
    \hALG
    &=
    2\sum_{s\in \sS}\sqrt{\lambda_s a_sb_s(1-b_s)}
    \\
    &=
    2\sum_{s\in \sS} \frac{a_s}{\sqrt{\sum_{s\in\sS} a_s}}\cdot \frac{\sqrt{\big(\sum_{s\in\sS} a_s\big)-1}}{\sum_{s\in\sS} a_s}
    \\
    &=
    2\sqrt{\frac{\big(\sum_{s\in\sS} a_s\big)-1}{\sum_{s\in\sS} a_s}}.
\end{align*}
\end{proof}



We finally show Corollary~\ref{cor:E-infty}, recalling the formula for $E_{\infty}$ from \cite{mckenna2021complexity} and verifying it equals $\ALG$ for pure models. It is given as follows, where $\bbH=\{z\in\bbC~:~{\mathsf {Im}}(z)> 0\}$ denotes the complex open upper half plane. We recall (a slight generalization of) \cite[Lemma 2.2]{mckenna2021complexity}; as written
only the bipartite case was considered therein but the general multi-species case is no different.
Additionally we point out that the constants $\alpha_s$ appearing in \cite{mckenna2021complexity} continue to vanish in pure models for general $r$, which we take advantage of in the statement below.


Informally, the point below is simply that $\sum_s \lambda_s M_s$ is the Stieltjes transform of the bulk spectral distribution of an $N\times N$ random matrix with variance profile $\partial_{x_s,x_{s'}}\xi$ with diagonal species-dependent shift $E\xi^s(\vone)$. This essentially corresponds to the behavior of the Riemannian Hessian $\nabla^2_{\sph}H_N(\bsig)$ at a point $\bsig$ with $H_N(\bsig)=E$, where the diagonal shift corresponds to the induced radial derivative of $H_N$.


\begin{proposition}[{Adaptation of \cite[Lemma 2.2]{mckenna2021complexity} with $r$ species and pure $\xi$}]
\label{prop:E-infty-mckenna}
    For $z\in\bbH$ (resp. $-\bbH$), there is a unique solution $\vec M\in \bbH^{\sS}$ (resp. $-\bbH^{\sS}$) to the matrix Dyson equation
    \[
    1+M_s\lt(
    \big(z-E\xi^s(\vone) \big) 
    +
    \partial_{x_{s}}\xi^{s}(\vone)
    M_{s}
    +
    \sum_{s'\neq s}
    (\partial_{x_{s'}}\xi^s(\vone))
    M_{s'}
    \rt)
    =0,\quad\forall s\in\sS.
    \]
    The threshold $E_{\infty}\geq 0$ is the smallest value such that with $z=0$, $\vec M(E)$ extends analytically and continuously at the boundary to $E\in [E_{\infty},\infty)$ (and is real-valued on this interval).
\end{proposition}

When $\xi(\vx)=\prod_{s\in\sS} x_s^{a_s}$ is pure and $z=0$, the vector Dyson equation simplifies to
\begin{equation}
\label{eq:matrix-dyson-pure}
    1+a_s M_s\bigg(
    E-\lambda_s M_s + 
    \sum_{s'\in\sS}
    \lambda_{s'}a_{s'}M_{s'}
    \bigg)
    =0,\quad
    \forall\,s\in\sS.
\end{equation}


\begin{proof}[Proof of Corollary~\ref{cor:E-infty}]
For convenience we omit the case $\xi(x_1,x_2)=x_1x_2$ and assume $\sum_s a_s\geq 3$.
Setting
\begin{align*}
    K_s&=a_sM_s,
    \\
    K&=\sum_{s\in\sS} \lambda_s K_s    
\end{align*}
the system \eqref{eq:matrix-dyson-pure} can be rearranged to
\[
    A\equiv K+E = \frac{\lambda_s K_s}{a_s}-\frac{1}{K_s},\quad \forall\,s\in\sS.
\]
With $B_s=\frac{a_s}{\lambda_s}$ we find that at $E=E_{\infty}$,
\[
    K_s=\frac{A B_s - \sqrt{A^2 B_s^2+4B_s}}{2}.
\]
Here the choice of sign is forced by $M_s<0$; this easily holds for sufficiently large $E$ (where one can give a power series expansion), and follows by continuity since $K_s\neq 0$ in general.

Note that $A$ above determines each $K_s$, hence $K$ and hence $E=A-K$.
Viewing $E$ as a function the $A$, its derivative must vanish and so:
\begin{align}
\nonumber
    0&=\frac{\de E}{\de A}
    \\
\label{eq:A-positive}
    &=
    1-\frac{1}{2}
    \sum_{s\in\sS}
    a_s\lt(
    1-\frac{A B_s}{\sqrt{A^2 B_s^2 + 4B_s}}
    \rt)
    \\
\label{eq:A-zero}
    &=
    1-\frac{1}{2}
    \sum_{s\in\sS}
    a_s\lt(
    1-\frac{1}{\sqrt{1 + 4/(A^2 B_s)}}
    \rt)
    \\
\nonumber
    &\stackrel{\eqref{eq:B=1}}{=}
    1-\frac{1}{2}
    \sum_{s\in\sS}
    a_s\lt(
    1-\sqrt{\frac{a_s}{a_s+L\lambda_s}}
    \rt)
    \\
\label{eq:L-zero}
    &=
    1-\frac{1}{2}
    \sum_{s\in\sS}
    a_s\lt(
    1-\sqrt{\frac{1}{1+L/B_s}}
    \rt)
    .
\end{align}
Here we used $\sum_s a_s\geq 3$ to deduce from \eqref{eq:A-positive} that $A>0$, thus implying the next line.
By monotonicity, equality of \eqref{eq:A-zero} and \eqref{eq:L-zero} now implies $A=2/\sqrt{L}$. 
Turning to the desired equality, we first write
\begin{align*}
    E_{\infty}&=A-K
    \\
    &=
    \frac{2}{\sqrt{L}}
    -
    \frac{1}{2}
    \sum_s 
    \lambda_s 
    \lt(
    \frac{2a_s}{\lambda_s \sqrt{L}}
    -
    \sqrt{
    \frac{4a_s^2}{L\lambda_s^2}
    +
    \frac{4a_s}{\lambda_s}
    }
    \rt)
    \\
    &=
    \frac{2}{\sqrt{L}}
    -
    \sum_s 
    \frac{a_s}{\sqrt{L}}
    \lt(1-
    \sqrt{
    \frac{a_s+L\lambda_s}{a_s}
    }
    \rt).
\end{align*}
With $V_s\equiv\sqrt{a_s+L\lambda_s}$, adding and subtracting $\sum_s \frac{a_s^{3/2}}{V_s\sqrt{L}}$ to get the second equality, we compute
\begin{align*}
    E_{\infty}-\ALG
    &=
    \frac{2}{\sqrt{L}}
    +\sum_s
    \lt(
    -\frac{a_s}{\sqrt{L}}
    +
    \frac{V_s\sqrt{a_s}}{\sqrt{L}}
    -
    \frac{\lambda_s \sqrt{L a_s}}{V_s}
    \rt)
    \\
    &=
    \frac{1}{\sqrt{L}}\lt(
    2
    +
    \sum_s
    \lt(
    -a_s
    +
    \frac{a_s^{3/2}}
    {V_s}
    \rt)
    \rt)
    +
    \sum_s
    \frac{\sqrt{a_s}}{V_s \sqrt{L}}
    \lt(
    V_s^2 - a_s-L\lambda_s
    \rt)
    \\
    &=0.
\end{align*}
Here in the last step, we used \eqref{eq:B=1} to handle the first contribution (summed over $s\in\sS$) and the definition of $V_s$ for the second (for each $s\in\sS$). 
\end{proof}



% \section{Approximate Message Passing}
\label{sec:amp}


\mscomment{
This section will be split off. Update things after writing augmented AMP in the appendix.
}

\mscomment{
Augmented state evolution can be used to analyze pairs of outputs by just doing both in the same sequence, so the suggestion of \cite{alaoui2020algorithmic} actually works now. It is worth commenting that for isoperimetric randomness, one can do AMP algorithms along a rooted tree of finite depth but exponential breadth.
}


In this section we prove Theorem~\ref{thm:main-alg} by exhibiting an approximate message passing (AMP) algorithm.
Throughout this section, Assumption~\ref{as:nondegenerate} on non-degeneracy of $\xi$ will be enforced. 
This is without loss of generality since one can always increase the quadratic and cubic coefficients $\gamma_{s,s'}, \gamma_{s,s',s''}$ by small constants and invoke continuity of $\ALG$.


As in \cite{alaoui2022algorithmic,sellke2021optimizing}, our algorithm has two phases. The first phase identifies the root of the ``algorithmic ultrametric tree'' and the second descends it in small orthogonal steps.
The structure of the first phase is similar to the original AMP algorithm of \cite{bolthausen2014iterative} for the SK model at high-temperature, while the latter \emph{incremental} AMP technique was introduced in \cite{mon18}. 




\subsection{Review of Approximate Message Passing}

Here we recall the class of approximate message passing algorithms, specialized to our setting of interest. We initialize AMP with a deterministic vector $\bw^0$  with coordinates
\begin{equation}
\label{eq:AMP-init-concrete}
    w^0_i = C_{s(i)}
\end{equation}
depending only on the species.
Let $f_{t,s}:\bbR^{t+1}\to\bbR$ be a Lipschitz function for each $(t,s)\in \bbZ_{\geq 0}\times \sS$. For $(\bw^0,\bw^1,\dots,\bw^t)\in\bbR^{N\times (t+1)}$, let 
$f_{t}(\bw^0,\bw^1,\dots,\bw^t)\in\bbR^N$ be given by
\[
    f_{t}(\bw^0,\bw^1,\dots,\bw^t)_i
    =
    f_{t,s(i)}(w^1_i,\dots,w^t_i),\quad i\in [N].
\]
We generate subsequent iterates through recursions of the form, where $\ONS_t$ is known as the \emph{Onsager correction term}:
\begin{align}
\label{eq:AMP-body}
    \bw^{t+1}
    &=
    \nabla H_N(\bm^t)
    -
    \ONS_t
    ;
    \\
\nonumber
    \bm^t
    &=
    f_{t}(\bw^0,\bw^1,\dots,\bw^t);
    \\
\label{eq:ONS-body}
    \ONS_t
    &=
    \sum_{t'\leq t}
    d_{t,t'}
    \diamond
    f_{t'-1}(\bw^1,\dots,\bw^{t'-1});
    \\
\label{eq:dts-def}
    d_{t,t',s}
    &=
    \sum_{s'\in\sS}
    \partial_{x_{s'}}
    \xi^s
    \lt(
    \lt(
    \bbE[M^t_{s''} M^{t'-1}_{s''}]\rt)_{s''\in\sS}
    \rt)
    \cdot
    \bbE
    \lt[
    \partial_{X^{t'}_{s'}}f_{t,s'}(X^0_{s'},\dots,X^t_{s'})
    \rt]
    .
\end{align}
Here $W^t_s,M^t_s$ are defined as follows. $W^0_s=C_s$ and the variables
$(\wt W^t_s)_{(t,s)\in \bbZ_{\geq 1}\times \sS}$ form a centered Gaussian process with covariance defined recursively by
\begin{equation}
\label{eq:state-evolution-basic}
\begin{aligned}
    \bbE[\wt W^{t+1}_s \wt W^{t'+1}_{s}]
    &=
    \xi^s\lt(\bbE[f_{t,s}(W^0_s,\dots,W^{t}_s)f_{t',s}(W^0_s,\dots,W^{t'}_s)]\rt),
    \\
    W^t_s
    &=
    \wt W^t_s+h_s;
    \\
    M^t_s
    &=
    f_{t,s}(W^0_s,\dots,W^t_s)
\end{aligned}
\end{equation}
and $\bbE[\wt W^{t+1}_{s} \wt W^{t'+1}_{s'}]=0$ if $s\neq s'$ (i.e. different species are independent). 


The following \emph{state evolution} characterizes the behavior of the above iterates. It states that for each $s\in\sS$, when $i\in \cI_s$ is uniformly random the sequence of coordinates $(w^1_i,w^2_i,\dots,w^t_i)$ has the same law as $(W^1_s,\dots,W^t_s)$. 


\begin{proposition}
\label{prop:state_evolution}
For any pseudo-Lipschitz function $\psi$ and $\ell\in\bbZ_{\geq 0}$, $s\in\sS$,
\begin{equation}
\label{eq:SE-body}
    \plim_{N\to\infty}\frac{1}{N_s}
    \sum_{i\in\cI_s}
    \psi(\bw^0_i,\dots,\bw^{\ell}_i)
    =
    \bbE
    [
    \psi(W^0_s,\dots,W^{\ell}_s)
    ]
    .
\end{equation}
\end{proposition}


This proposition allows us to read off normalized inner products of the AMP iterates, since e.g.
\[
    \langle \bw^k,\bw^{\ell}\rangle_N
    \simeq
    \sum_{s\in\sS}
    \lambda_s
    \bbE[W^k_s W^{\ell}_s].
\]



Proposition~\ref{prop:state_evolution} is proved in Appendix~\ref{sec:ProofSE}. For random matrices (i.e. the case of quadratic $H$) there is a considerable literature establishing state evolution in many settings beginning with \cite{bolthausen2014iterative,BM-MPCS-2011} and later \cite{bayati2015universality,berthier2019state,chen2020universality,fan2020approximate,dudeja2022universality} (see also \cite{feng2022unifying} for a survey of many statistical applications). The generalization to tensors was introduced in \cite{richard2014statistical} and proved in \cite{ams20}, whose approach we follow. 






\subsection{Stage $\I$: Finding the Root of the Ultrametric Tree}


Our goal in this subsection will be to compute a vector $\bm^{\ol}$ satisfying
\[
    \vR(\bm^{\ol},\bm^{\ol})\approx\Phi(q_0)
\]  
and with the correct energy value (as stated in Lemma~\ref{lem:sphereenergy} below). 




We take as given an maximizer $(p,\Phi;q_0)$ to $\bbA$ with associated transition point $q_1$. Recall this means there is a unique-up-to-scaling vector $v\in\mathbb R_{\geq 0}^{\sS}$ such that
\begin{equation}
\label{eq:v-equation}
    \lt(\xi^s(\Phi(q_1))+h_s^2\rt)v_s=
    \sum_{s'\in\sS}
    \partial_{x_s}\xi^{s'}
    (\Phi(q_1))
    \,
    \Phi_s(q_1)v_{s'}.
\end{equation}
We use the initialization
\[
    w^0_i = \sqrt{\xi^s(\Phi(q_1))+h_s^2},\quad i\in \cI_s.
\]
Define the vector $\va\in\mathbb R^{\sS}$ by 
\[
    a_s=
    \sqrt{\frac{\Phi_s(q_1)}{\xi^s(\Phi(q_1))+h_s^2}}
    =\sqrt{\frac{v_s}
    {\sum_{s'\in\sS} 
    \partial_{x_s} \xi^{s'}(\Phi(q_1))v_{s'}}}.
\]
Subsequent iterates are defined via the following recursion.
\begin{align}
\label{eq:RSsphere}
  \bw^{k+1}
  &=
  \nabla H_N(\bm^k)
  -
  \lt(\va\odot
    \zeta\big(\vR(\bm^k,\bm^{k-1})\big)
  \rt)
  \diamond
  \bm^{k-1}
  \\
\nonumber
  &=
  \bh
  +
  \nabla \wtH_N(\bm^k)
  -
  \lt(\va\odot
    \zeta\big(\vR(\bm^k,\bm^{k-1})\big)
  \rt)
  \diamond
  \bm^{k-1};
  \\
\label{eq:mk-def}
  \bm^k
  &=
  \va\diamond \bw^k
\\
\label{eq:zeta-defn}
    \zeta^s(\vx)
    &\equiv
    \sum_{s'\in \sS}
    \partial_{x_{s'}} \xi^s
    (\vx)
    .
\end{align}
The last term in \eqref{eq:RSsphere} comes from specializing the formula \eqref{eq:ONS-body} for the Onsager term.


Next recalling \eqref{eq:state-evolution-basic}, let $(W^j_s,M^j_s)_{j\geq 0,s\in\sS}$ be the state evolution limit of the coordinates of 
\[
    (\bw^{0},\bm^{0},\dots,\bw^k,\bm^k)
\]
as $N\to\infty$. Concretely, each $W^j_s$ is Gaussian with mean $h_s$ and 
\[
    M^{j}_s=\sqrt{\frac{\Phi_s(q_1)}{\xi^s(\Phi(q_1))+h_s^2}
    }
    \cdot 
    W^j_s,
    \quad
    j\geq 0,~
    s\in\sS.
\]
We next determine the covariance structure of the Gaussians $\wt W^j_s$. Define the (deterministic) $\mathbb R_{\geq 0}^{\sS}$-valued sequence $(R^0,R^1,\dots)$ of asymptotic overlaps recursively by $R^0=0$ and 
\begin{equation}
\label{eq:overlap-recursion-AMP}
    R^{k+1}_s=\alpha_s(R^k)=\lt(\xi^s(R^k)+h_s^2\rt)\cdot \lt(\frac{\Phi_s(q_1)}{\xi^s(\Phi(q_1))+h_s^2}\rt),\quad
    k\geq 0,s\in \sS.
\end{equation}



\begin{lemma}
\label{lem:RSconverge}
For integers $0\leq j<k$, the following equalities hold (the first in distribution):
\begin{align} 
\label{eq:id1.0}
    W^j_s&\stackrel{d}{=} h_s+Z\sqrt{\xi^s(\Phi(q_1))},\quad Z\sim \cN(0,1)\\
\label{eq:id2.0}
    \mathbb E[\wt W^j_s \wt W^k_s]&=\xi^s(R^j)\\
\label{eq:id3.0}
    \mathbb E[(M^j_s)^2]&=\Phi_s(q_1)\\
\label{eq:id4.0}
    \mathbb E[M^j_s M^k_s]&=R^{j+1}_s.
\end{align}
\end{lemma}



\begin{proof}
We proceed by induction on $j$, first showing \eqref{eq:id1.0} and \eqref{eq:id3.0} together. As a base case, \eqref{eq:id1.0} holds for $j=0$ by initialization. For the inductive step, assume first that \eqref{eq:id1.0} holds for $j$. Then by the definition \eqref{eq:mk-def},
\begin{align*}
  \mathbb E\lt[(M^j_s)^2\rt]&=
  \lt(\xi^s(\Phi(q_1))+h_s^2\rt)\cdot \lt(\frac{\Phi_s(q_1)}{\xi^s(\Phi(q_1))+h_s^2}\rt)
  \\
  &=\Phi_s(q_1)
\end{align*}
so that \eqref{eq:id1.0} implies \eqref{eq:id3.0} for each $j\geq 0$. On the other hand, state evolution directly implies that if \eqref{eq:id3.0} holds for $j$ then \eqref{eq:id1.0} holds for $j+1$. This establishes \eqref{eq:id1.0} and \eqref{eq:id3.0} for all $j\geq 0$.


We similarly show \eqref{eq:id2.0} and \eqref{eq:id4.0} together by induction, beginning with \eqref{eq:id2.0}. When $j=0$ it is clear because $\wt W^k_s$ is mean zero and independent of $\wt W^0_s$. 
Just as above, it follows from state evolution that \eqref{eq:id2.0} for $(j,k)$ implies \eqref{eq:id4.0} for $(j,k)$ which in turn implies \eqref{eq:id2.0} for $(j+1,k+1)$. Hence induction on $j$ proves \eqref{eq:id2.0} and \eqref{eq:id4.0} for all $(j,k)$.
\end{proof}


The next lemma is crucial and uses super-solvability of $\Phi(q_1)$.


\begin{lemma}
\label{lem:Rj-to-Phiq1}
    $\lim_{j\to\infty} R^j=\Phi(q_1).$
\end{lemma}


\begin{proof}
    First we observe that $\alpha$ is coordinate-wise strictly increasing in the sense that if $0\preceq x\prec y$ then $\alpha(x)\prec \alpha(y)$. 
    Moreover $\alpha(\vzero)\succ 0$ (assuming $\vh\neq 0$, else the result is trivial) and $\alpha(\Phi(q_1))=\Phi(q_1)$. Therefore $\oR=\lim_{j\to\infty} R^j$ exists, $\alpha(\oR)=\oR$, and
    \[
        \vzero\preceq \oR\preceq\Phi(q_1).
    \]
    It remains to show that the above forces $\oR=\Phi(q_1)$ to hold.
    
    
    Let $M\in\bbR^{\sS\times \sS}$ be the matrix with entries $M_{s,s'}=\deriv{t}\alpha_s'(\Phi(q_1)+te_{s'})|_{t=0}$ for $e_{s'}$ a standard basis vector. Then $M$ is the derivative matrix for $\alpha$ at $\Phi(q_1)$ in the sense that for any $u\in\bbR^{\sS}$,
    \[
        \deriv{t}\alpha'(\Phi(q_1)+tu)|_{t=0}=Mu.
    \]
    It is easy to see that all entries of $M$ are strictly positive, and that $Mv=v$ for $v$ solving \eqref{eq:v-equation}. By Perron-Frobenius theory, it follows that for any entry-wise non-negative vector $w\in\mathbb R_{\geq 0}^{\sS}$, we have
    \begin{equation}
    \label{eq:Mws} 
        (Mw)_s\leq w_s
    \end{equation}
    for some $s\in \sS$. Now suppose for sake of contradiction that $\oR\prec \Phi(q_1)$, let $w=\Phi(q_1)-\oR$, and choose $s^*\in\sS$ such that \eqref{eq:Mws} holds. Write $f(t)=\alpha_{s^*}(\Phi(q_1)+tw)$. Since $\alpha_{s^*}$ is a polynomial with non-negative coefficients and $\xi$ is non-degenerate, $f$ is strictly convex and strictly increasing on $[-1,0]$. Hence
    \begin{align*}
        \alpha_{s^*}(\oR)
        &= f(-1)
        \\
        &>
        f(0)-f'(0)
        \\
        &\geq
        \Phi_{s^*}(q_1)-(Mw)_{s^*}
        \\
        &\stackrel{\eqref{eq:Mws}}{\geq}
        \Phi_{s^*}(q_1)-w_{s^*}
        \\
        &=
        \oR_{s^*}.
    \end{align*}
    The first inequality above is strict, so we deduce that $\alpha(\oR)\neq\oR$ if $\oR\prec\Phi(q_1)$. This contradicts the definition of $\oR$. Therefore $\oR=\Phi(q_1)$, completing the proof.
\end{proof}


\begin{remark}
Super-solvability of $\Phi(q_1)$ is precisely the required condition for the above to go through. It is equivalent to the matrix $M=\alpha'$ above having Perron-Frobenius eigenvalue at most $1$. Indeed suppose that $\Phi(q_1)$ was chosen so that $\lambda_1(M)>1$. Then there exists $w\in\bbR_{>0}^{\sS}$ with $Mw\succ w$.
Letting $x=\Phi(q_1)-\eps w$ for small $\eps>0$, we find $\alpha(x)\prec x$.
Monotonicity implies that $\alpha$ maps the compact, convex set
\[
    K=\{y\in[0,1]^{\sS}~:~0\preceq y\preceq x\}
\]
into itself. By the Brouwer fixed point theorem, a fixed point of $\alpha$ strictly smaller than $\Phi(q_1)$ exists whenever $\Phi(q_1)$ is strictly subsolvable.
\end{remark}


We finish our analysis of the first AMP phase by computing the asymptotic energy it achieves. As expected, the resulting value agrees with the first term in the formula \eqref{eq:alg-for-optimizer} for $\ALG$.
%
\begin{lemma}
\label{lem:sphereenergy}
\[
  \lim_{k\to\infty} \plim_{N\to\infty}\frac{H_N(\bm^k)}{N}
  = 
    \sum_{s\in \sS}
    \lambda_s
      \sqrt{
      \Phi_s(q_1)
      \cdot
      \lt(h_s^2+\xi^s(\Phi(q_1))\rt)
      }
    \,.
\]
\end{lemma}

\begin{proof}

We use the identity
\begin{equation}
  \frac{H_N(\bm^k)}{N}=\big\langle \bh,\bm^k\rangle_N+\int_0^1 \langle \bm^k,\nabla \widetilde H_N(t\bm^k)\big\rangle_N \de t
\end{equation}
and interchange the limit in probability with the integral. To compute $\plim_{N\to\infty}\langle \bm^k,\nabla \widetilde H_N(t\bm^k)\rangle$ we introduce an auxiliary AMP step 
\[
    \by^{k+1}=\nabla \widetilde H_N(t\bm^k)-
    t\cdot
    \lt(\va\odot
    \zeta\big(
        t\cdot \vR(\bm^k,\bm^{k-1})
    \big)
    \rt)
    \diamond
    \bm^{k-1}
\] 
which depends implicitly on $t\in [0,1]$.
Rearranging yields
\begin{align*}
  \vR(\bm^k,\nabla \widetilde H_N(t\bm^k)) 
  &= 
  \vR(\bm^k,\by^{k+1})
  +
  t\cdot
  \lt(
  \vR(\bm^k,\bm^{k-1} )
  \odot \va
  \odot 
  \zeta(t\cdot \vR(\bm^k,\bm^{k-1}) )
  \rt)  
  \\
  &\simeq   
  \vR( \bm^k,\by^{k+1})
  +
  t\cdot
  \lt(
  \vR^k
  \odot \va
  \odot 
  \zeta(t\cdot \vR^k )
  \rt)  
  .
\end{align*}

For the first term, Gaussian integration by parts with $g_s(x)=(x+h_s)a_s$ yields
\begin{align*}
    R_s(\bm^k,\by^{k+1})
    &=
  \mathbb E[g_s(W^k)Y^{k+1}_s]
  \\
  &=
  \mathbb E[g_s'(W^k)]
  \cdot
  \mathbb E[W^k_s Y^{k+1}_s]
  \\
  &= 
  a_s
  \cdot 
  \xi^s(t R^k)
  .
\end{align*}
Integrating with respect to $t$, we find
\begin{align*} 
  \int_0^1 \langle \bm^k,\nabla \widetilde H_N(t\bm^k)\rangle_N \de t
  &\simeq 
  \sum_{s\in\sS}
  \lambda_s 
  \int_0^1
  R_s(\bm^k,\nabla \widetilde H_N(t\bm^k))
  \de t
  \\
  &\simeq
  \sum_{s\in\sS}
  \lambda_s
  a_s
  \int_0^1
  \xi^s(t\cdot \vR^k)
  +
  t R^k_s \zeta^s(t\cdot \vR^k)
  \de t 
  \\
  &=
  \sum_{s\in\sS}
  \lambda_s
  a_s
  \int_0^1
  \frac{\de ~}{\de t}
  \lt(t\cdot \xi^s(t\cdot \vR^k)\rt)
  \de t
  \\
  &=
  \sum_{s\in\sS}
  \lambda_s a_s \xi^s(\vR^k).
\end{align*}
Finally the external field $\bh$ gives energy contribution
\[
  \langle \bh, \bm^k\rangle_N 
  \simeq 
  \sum_{s\in\sS}
  \lambda_s
  h_s\bbE[M^k_s]
  =
  \sum_{s\in\sS}
  \lambda_s
  a_s
  h_s^2.
\]
Since $\lim_{k\to\infty} R^{k}=\Phi(q_1)$ by Lemma~\ref{lem:Rj-to-Phiq1}, we conclude
\begin{align*}
  \lim_{k\to\infty} \plim_{N\to\infty}\frac{H_N(\bm^k)}{N}
  &= 
  \sum_{s\in\sS}
  \lambda_s
  a_s\big(h_s^2 + \xi^s(\Phi(q_1))\big)
  \\
  &=
  \sum_{s\in\sS}
  \lambda_s
  \sqrt{
    \Phi_s(q_1)
  \cdot
  \lt(h_s^2+\xi^s(\Phi(q_1))\rt)
  }
  .
\end{align*}
\end{proof}


\subsection{Stage $\II$: Incremental Approximate Message Passing}


We now turn to the second phase which uses incremental approximate message passing. Choose a large constant $\ul$ and set
\begin{align*}
    \eps_s
    &=
    \sqrt{\frac{\Phi(q_0)_s}{R^{\ul}_s}}
    -1,
    \\
    \bn^{\ul}
    &=
    (1+\eps)
    \diamond 
    \bm^{\ul}
\end{align*}
so that 
\[
    \vR(\bn^{\ul},\bn^{\ul})\simeq \Phi(q_1).
\]
The point $\bn^{\ul}$ will be the ``root'' of our IAMP algorithm.\footnote{If $\vh=0$, one can instead set $n^{\ul}_i=\sqrt{\Phi_{s(i)}(\delta)}\diamond \bg$ for $\bg\sim\cN(0,I_N)$ and use the same algorithm with $q_0=0$.} We fix a small constant $\delta>0$ and consider the sequence
\[
    q^{\delta}_{\ell} 
    =
    q_1
    +
    \ell\delta,\quad \ell\geq 0.
\]
Moreover we set $\ol=\max\{\ell\in\bbZ_+~:~q_{\ell}^{\delta}\leq 1-2\delta\}.$
We also define for $s\in\sS$ and $\ul\leq \ell\leq \ol$ the constants
\begin{equation}
\label{eq:u-def}
    u_{\ell,s}^{\delta}
    =
    \sqrt{
    \frac{\Phi_s(q^{\delta}_{\ell+1})-\Phi_s(q^{\delta}_{\ell})}
    {
    \xi^s(\Phi(q_{\ell+1}^{\delta}))
    -
    \xi^s(\Phi(q_{\ell}^{\delta}))
    }
    }
    .
\end{equation}



Set $\bz^{\ul}=\wt\bw^{\ul}=\bw^{\ul}-\bh$.
So far, we have defined $(\bw^{\ul},\bz^{\ul},\bn^{\ul})$. We turn to inductively defining the triples $(\bw^{\ell},\bz^{\ell},\bn^{\ell})$ for $\ul\leq\ell\leq\ol$. First, the values $(\bz^{\ell})_{\ell\geq \ul}$ are defined as AMP iterates via
\begin{equation}
\label{eq:general_amp}
\begin{aligned}
    \bz^{\ell+1} 
    &= 
    \nabla \widetilde H_N(f_{\ell}(\bz^{\ul},\cdots,\bz^\ell)) - \sum_{j=0}^\ell d_{\ell, j}\diamond f_{j-1}(\bz^{\ul},\cdots,\bz^{j-1})
    .
\end{aligned}
\end{equation}
The Onsager coefficients $d_{\ell,j}$ are given by \eqref{eq:dts-def} and will not appear explicitly in any calculations.
Here the variables $Z^k$ are the state evolution limits. To complete the definition of the iteration \eqref{eq:general_amp}, for $s(i)=s$ and $\ell\geq \ul$ we set
\[
    f_{\ell,s}(z^{\ul}_i,\dots,z^{\ell}_i)
    =
    n^{\ell}_i,
\]
where
\begin{equation}
\label{eq:IAMP}
    \bn^{\ell+1} 
    =
    \bn^{\ell}+ 
    u_{\ell}^{\delta}
    \diamond
    \lt(\bz^{\ell+1}-\bz^{\ell}
    \rt).
\end{equation} 
Finally the algorithm $\cA$ outputs
\begin{equation}
\label{eq:round-final-output}
    \cA(H_N)
    =
    R(\bn^{\ol},\bn^{\ol})^{-1/2}\diamond \bn^{\ol}
    \in\cB_N
\end{equation}
where the power $-1/2$ is taken entry-wise.

The state evolution limits are described by time-changed Brownian motions with total variance $\Phi_s(q^{\delta}_{\ell})$ in species $s$ after iteration $\ell$. This is made precise below.

\begin{lemma}
\label{lem:BMlimit}
Fix $s\in\sS$. The sequence $(Z^{\delta}_{\ul,s},Z^{\delta}_{\ul+1,s},\dots)$ is a Gaussian process satisfying
\begin{align} 
\label{eq:BM1}
    \mathbb E[(Z^{\delta}_{\ell+1,s}-Z^{\delta}_{\ell,s})Z^{\delta}_{j,s}]
    &=
    0,\quad \text{for all }\ul+1\leq j\leq \ell
    \\
\label{eq:BM2}
    \mathbb E\big[
    (Z^{\delta}_{\ell+1,s}-Z^{\delta}_{\ell,s})^2
    \big]
    &=
    \xi^s(\Phi(q_{\ell+1}^{\delta}))
    -
    \xi^s(\Phi(q_{\ell}^{\delta}))
    \\
\label{eq:BM3}
    \mathbb E[Z^{\delta}_{\ell,s}Z^{\delta}_{j,s}]
    &=
    \xi^s(\Phi(q_{j\wedge \ell}^{\delta}))
    \\
\label{eq:BM4}
    \mathbb E[N^{\delta}_{\ell,s}N^{\delta}_{j,s}]
    &=
    \Phi_s(q^{\delta}_{j\wedge \ell})
    .
\end{align}
\end{lemma}


\begin{proof}
The fact that $(Z^{\delta}_{\ul,s},Z^{\delta}_{\ul+1,s},\dots)$ is a Gaussian process is a general fact about state evolution. We proceed by induction on $\ell\geq \ul$; in fact the proof is very similar to \cite[Section 8]{sellke2021optimizing} so we give only the main points. For the base case, the main computation is that 
\begin{align*}
    \bbE\big[\big(Z^{\delta}_{\ul+1,s}-Z^{\delta}_{\ul,s}\big)Z^{\delta}_{\ul,s}\big]
    &=
    \xi^s\lt(\bbE[N^{\delta}_{\ul,s}M^{\ul-1}_s]\rt)
    -
    \xi^s\lt(\bbE[M^{\ul-1}_{s}M^{\ul-1}_s]\rt)
    \\
    &=
    \xi^s\lt((1+\eps_s)\bbE[M^{\ul}_s M^{\ul-1}_s]\rt)-\xi^s(\Phi(q_1))
    \\
    &=
    \xi^s(\Phi(q_1))-\xi^s(\Phi(q_1))
    \\
    &=
    0.
\end{align*}
For inductive steps, we always have by state evolution
\[
    \bbE[Z^{\delta}_{\ell+1,s}Z^{\delta}_{j+1,s}]
    \simeq
    \xi^s\big(\vR(\bn^{\ell},\bn^{j})\big).
\]
It follows by the inductive hypothesis of \eqref{eq:BM1} that for $j\leq \ell$,
\begin{align*}
    R_s(\bn^{\ell},\bn^{j})
    &=
    R_s(\bn^{\ul},\bn^{\ul})
    +
    \sum_{k=\ul}^{j-1}
    (u_k^{\delta})^2
    R_s(\bz^{k+1}-\bz^k,\bz^{k+1}-\bz^k)
    \\
    &=
     R_s(\bn^{\ul},\bn^{\ul})
     +
     \sum_{k=\ul}^{j-1}
     (u_k^{\delta})^2
     \lt(
      \xi^s(\Phi(q_{k+1}^{\delta}))
    -
    \xi^s(\Phi(q_{k}^{\delta}))
    \rt)
    \\
    &=
    \Phi_s(q_1)
    +
    \sum_{k=\ul}^{j-1}
    \Big(
    \Phi_s(q^{\delta}_{k+1})
    -
    \Phi_s(q^{\delta}_{k})
    \Big)
    \\
    &=
    \Phi_s(q^{\delta}_j).
\end{align*}
Plugging into the above yields that for $j\leq \ell$,
\[
    \bbE[Z^{\delta}_{\ell+1,s}Z^{\delta}_{j+1,s}]
    =
    \xi^s(\Phi(q^{\delta}_j)).
\]  
This depends only on $\min(j,\ell)$, so \eqref{eq:BM1} follows. The others are proved by similar computations.
\end{proof}


Equation~\eqref{eq:BM4} implies that $\vR(\bn^{\delta}_{\ell},\bn^{\delta}_{j})\simeq \Phi(q^{\delta}_{\ell\wedge j})$, which exactly corresponds to the previous sections of the paper. In particular it implies that the final iterate $\bn^{\delta}_{\ol}$ satisfies
\begin{equation}
\label{eq:final-overlap-alg}
    (1-O(\delta))\cdot\vone\preceq \vR(\bn^{\delta}_{\ol},\bn^{\delta}_{\ol})\preceq \vone
\end{equation}
so the rounding step \eqref{eq:round-final-output} causes only an $O(\delta)$ change in the Hamiltonian value. Finally we compute in Lemma~\ref{lem:iampenergy} below the energy gain from the second phase, which matches the second term in \eqref{eq:alg-for-optimizer}. 


\begin{lemma}
\label{lem:iampenergy}
\begin{equation}
    \label{eq:iampenergy}
    \lim_{\ul\to\infty}
    \plim_{N \to \infty}
    \frac{H_{N}\lt(\bn^{\ol}\rt)-H_{N}\lt(\bn^{\ul}\rt)}{N} 
    = 
    \sum_{s\in\sS}
  \lambda_s
  \int_{q_1}^{1} 
    \sqrt{\Phi'_s(t) (\xi^s \circ \Phi)'(t)}
    \,
    \de t
\end{equation}
\end{lemma}



\begin{proof}
Recall that $\delta=\delta(\ul)\to 0$ as $\ul\to\infty$, which we will implicitly use throughout the proof. Observe also that $\langle h,\bn^{\ol}-\bn^{\ul}\rangle_N\simeq 0$ because the values $(N_{\ell,s}^{\delta})_{\ell\geq\ul}$ form a martingale sequence for each $s\in\sS$. 
Therefore it suffices to compute the in-probability limit of $\frac{\widetilde{H}_{N}\lt(\bn^{\ol}\rt)-\widetilde{H}_{N}\lt(\bn^{\ul}\rt)}{N}$. The key is to write 
\[
    \frac{\widetilde{H}_{N}\lt(\bn^{\ol}\rt)-\widetilde{H}_{N}\lt(\bn^{\ul}\rt)}{N}=\sum_{\ell=\ul}^{\ol-1}\frac{\widetilde{H}_{N}\lt(\bn^{\ell+1}\rt)-\widetilde{H}_{N}\lt(\bn^{\ell}\rt)}{N}
\]
and use a Taylor series approximation for each term. In particular for $F\in C^3(\mathbb R;\bbR)$, applying Taylor's approximation theorem twice yields
\begin{align*}
    F(1)-F(0)
    &=
    F'(0)+\frac{1}{2}F''(0)+O(\sup_{a\in [0,1]}|F'''(a)|)
    \\
    &= 
    F'(0)+\frac{1}{2}(F'(1)-F'(0))+O(\sup_{a\in [0,1]}|F'''(a)|)
    \\
    &=
    \frac{1}{2}(F'(1)+F'(0))+O(\sup_{a\in [0,1]}|F'''(a)|) .
\end{align*}

Assuming $\sup_{\ell}\frac{\|\bn^{\ell}\|}
{\sqrt{N}}\leq 1$, which holds with probability $1-o_N(1)$ by state evolution and the definition of $\ol$, we apply this estimate with 
\[
    F(a)=\widetilde{H}_N\lt((1-a)\bn^{\ell}+a\bn^{\ell+1}\rt).
\]
The result is:
\begin{align*}
    \lt|
    \widetilde{H}_{N}
    \lt(\bn^{\ell+1}\rt)-\widetilde{H}_{N}\lt(\bn^{\ell}\rt) -\frac{1}{2}\lt\langle \nabla \widetilde{H}_N(\bn^{\ell})+\nabla \widetilde{H}_N(\bn^{\ell+1}),\bn^{\ell+1}-\bn^{\ell}\rt\rangle \rt|
    &\leq 
    O\lt(
    \underline{C}N^{-1/2}
    \|\bn^{\ell+1}-\bn^{\ell}\|^3
    \rt)
    ;
    \\
    \underline{C}
    &=
    N^{1/2}
    \sup_{\|\bsig\|\leq \sqrt{N}}\lt\|\nabla^3 \widetilde{H}_N(\bsig)\rt\|_{\op}
    .
\end{align*}
Proposition~\ref{prop:gradients-bounded} implies that for deterministic constants $c,C$,
\[
    \bbP[\underline{C}\leq C]\geq 1-e^{-cN}.
\]
On the other hand for each $\ul\leq \ell\leq \ol-1$ we have
\begin{align*}
    \plim_{N\to\infty}\|\bn^{\ell+1}-\bn^{\ell}\|&=
    \sqrt{
    \sum_{s\in\sS}\lambda_s R_s(\bn^{\ell+1}-\bn^{\ell},\bn^{\ell+1}-\bn^{\ell})
    }
    \\
    &=
    \sqrt{
    \sum_{s\in\sS}\lambda_s
    \Phi_s(q^{\delta}_{\ell+1}-q^{\delta}_{\ell})
    }
    \\
    &=
    \sqrt{\delta N}.
\end{align*}
Summing and recalling that $\ol-\ul\leq \delta^{-1}$ yields the high-probability estimate
\begin{align*}
  \sum_{\ell=\ul}^{\ol-1}
  &
  \lt|
  \widetilde{H}_{N}\lt(\bn^{\ell+1}\rt)
  -
  \widetilde{H}_{N}\lt(\bn^{\ell}\rt) 
  -
  \frac{1}{2}\lt\langle  \nabla \widetilde{H}_N(\bn^{\ell})+\nabla \widetilde{H}_N(\bn^{\ell+1}),\bn^{\ell+1}-\bn^{\ell}\rt\rangle  
  \rt|
  \\ 
  &\leq 
  O(N^{-1/2})
  \cdot 
  \sum_{\ell=\ul}^{\ol-1} 
  \|\bn^{\ell+1}-\bn^{\ell}\|^3
  \\
  &\leq O(N\sqrt{\delta}).
\end{align*}
Because $\delta\to 0$ as $\ul\to\infty$, this term vanishes in the outer limit. It remains to prove
\[
  \lim_{\ul\to\infty}
  \plim_{N \to \infty}
  \sum_{\ell=\ul}^{\ol-1}
  \lt\langle 
  \nabla \widetilde{H}_N(\bn^{\ell})
  +
  \nabla \widetilde{H}_N(\bn^{\ell+1})
  ,
  \bn^{\ell+1}-\bn^{\ell}
  \rt\rangle_N  
  \stackrel{?}{=}
  2
  \sum_{s\in\sS}
  \lambda_s
  \int_{q_1}^{1} 
    \sqrt{\Phi'_s(t) (\xi^s \circ \Phi)'(t)}
  \de t.
\]
To establish this it suffices to show for each species $s\in\sS$ the equality
\begin{equation}
\label{eq:IAMP-energy-per-species}
  \lim_{\ul\to\infty}
  \plim_{N \to \infty}
  \sum_{\ell=\ul}^{\ol-1}
  R\lt(
  \nabla \widetilde{H}_N(\bn^{\ell})
  +
  \nabla \widetilde{H}_N(\bn^{\ell+1})
  ,
  \bn^{\ell+1}-\bn^{\ell}
  \rt)_s 
  \stackrel{?}{=}
  2
  \int_{q_1}^{1} 
    \sqrt{\Phi'_s(t) (\xi^s \circ \Phi)'(t)}
  \de t.
\end{equation}
This is what we do. Observe by \eqref{eq:general_amp} that:
%
\begin{equation}
\label{eq:amprearrange}
    \nabla \widetilde{H}_N(\bn^{\ell})
    =
    \bz^{\ell+1}- \sum_{j=0}^\ell d_{\ell, j}\diamond \bn^{j-1}.
%
\end{equation}
%
Passing to the limiting Gaussian process $(Z^{\delta}_k)_{k\in\mathbb Z^+}$ via state evolution,
\begin{align*}
  \plim_{N\to\infty}
  R\lt(\nabla \widetilde{H}_N(\bn^{\ell}),\bn^{\ell+1}-\bn^{\ell}\rt)_s
  &=
  \mathbb E\lt[ Z^{\delta}_{\ell+1,s}(N^{\delta}_{\ell+1,s}-N^{\delta}_{\ell,s})\rt]
  - 
  \sum_{j=0}^{\ell}
  d_{\ell,j,s}
  \mathbb E\lt[
  N^{\delta}_{j-1,s}(N^{\delta}_{\ell+1,s}-N^{\delta}_{\ell,s})
  \rt],
  \\
  \plim_{N\to\infty}
  R\lt(\nabla \widetilde{H}_N(\bn^{\ell+1}),\bn^{\ell+1}-\bn^{\ell}\rt)_s
  &=
  \mathbb E\lt[ Z^{\delta}_{\ell+2,s}(N^{\delta}_{\ell+1,s}-N^{\delta}_{\ell,s})\rt]
  - 
  \sum_{j=0}^{\ell+1} 
  d_{\ell+1,j,s}
  \mathbb E\lt[
  N^{\delta}_{j-1}(N^{\delta}_{\ell+1,s}-N^{\delta}_{\ell,s})
  \rt].
\end{align*}


As $(N^{\delta}_k)_{k\geq \mathbb Z^+}$ is a martingale process, it follows that the right-hand expectations all vanish. Similarly it holds that
\begin{align*}
  \mathbb E[Z_{\ell+2}^{\delta}(N_{\ell+1}^{\delta}-N_{\ell}^{\delta})]&=\mathbb E[Z_{\ell+1}^{\delta}(N_{\ell+1}^{\delta}-N_{\ell}^{\delta})]
  \\
  \mathbb E[Z_{\ell}^{\delta}(N_{\ell+1}^{\delta}-N_{\ell}^{\delta})]&=0.
\end{align*}
%
We conclude that
%
\begin{align*}
  \plim_{N\to\infty}
  R\lt(\nabla \widetilde{H}_N(\bn^{\ell})+\nabla \widetilde{H}_N(\bn^{\ell+1}),\bn^{\ell+1}-\bn^{\ell}\rt)_s
  &=
  2\,\mathbb E[
  (Z_{\ell+1,s}^{\delta}-Z^{\delta}_{\ell,s})(N_{\ell+1,s}^{\delta}-N_{\ell,s}^{\delta})
  ]
  \\
  &=
  2\,\mathbb E[u_{\ell,s}^{\delta}(Z^{\delta}_{\ell,s})(Z_{\ell+1,s}^{\delta}-Z^{\delta}_{\ell,s})^2]
  \\
  &=
  2\,\mathbb E[u_{\ell,s}^{\delta}(Z^{\delta}_{\ell,s})]
  \cdot
  \bbE[(Z_{\ell+1,s}^{\delta}-Z^{\delta}_{\ell,s})^2]
  \\
&=
  2\,\sqrt{
  \Big(
  \Phi_s(q^{\delta}_{\ell+1,s})-\Phi_s(q^{\delta}_{\ell,s})
  \Big)
  \cdot
    \Big(\xi^s(\Phi(q_{\ell+1}^{\delta}))-\xi^s(\Phi(q_{\ell}^{\delta}))
    \Big)
    }
    .
\end{align*}
In the second-to-last step we used the fact that $Z^{\delta}_{\ell,s}$ has independent increments, which follows from Lemma~\ref{lem:BMlimit}, while the last step used \eqref{eq:u-def} and \eqref{eq:BM2}.
Combining with Lemma~\ref{lem:ALG-from-continuum} implies \eqref{eq:IAMP-energy-per-species}.
\end{proof}


\begin{proof}[Proof of Theorem~\ref{thm:main-alg}]
We take $\cA$ as in \eqref{eq:round-final-output} for $\ul$ a large constant depending on $(\eps,\xi,h,\lambda)$. The fact that 
\begin{equation}
\label{eq:1-o1-prob-alg}
    \bbP[H_N(\cA(H_N))/N \geq \ALG-\eps/2]
    \ge 
    1-o_N(1)
\end{equation}
follows from combining Lemma~\ref{lem:sphereenergy}, Lemma~\ref{lem:iampenergy} and the fact that (recall \eqref{eq:final-overlap-alg}) 
\[
H_N(\cA(H_N))/N \simeq H_N(\bn^{\ul})/N+o_{\bbP}(1).
\]

Next, let $K_N\subseteq\sH_N$ be as in Proposition~\ref{prop:gradients-bounded}. We recall that $\bbP[H_N\in K_N]\geq 1-e^{-cN}$.
Exactly as in \cite[Theorem 10]{huang2021tight} it follows that there is a $C(\eps)$-Lipschitz function $\wt\cA:\sH_N\to\bbR$ such that $\wt\cA$ and $\cA$ agree on $K_N$.
Moreover \eqref{prop:gradients-bounded} and concentration of measure on Gaussian space imply that $H_N(\wt\cA(H_N))$ is $O(N^{1/2})$-sub-Gaussian. In light of \eqref{eq:1-o1-prob-alg} and since $\bbP[\wt\cA(H_N)=\cA(H_N)]\geq \bbP[H_N\in K_N]\geq 1-e^{-cN}$, we deduce that 
\[
    \bbP[H_N(\cA(H_N))/N \geq \ALG-\eps]
    \ge 
    1-e^{-cN}.
\]
This concludes the proof.
\end{proof}


\begin{remark}
    Analogously to \cite[Section 4]{sellke2021optimizing}, the second stage of our IAMP algorithm can be slightly modified to give a full ultrametric tree of outputs, whose pairwise overlaps are given by values of $\Phi$. More precisely, for any finite ultrametric space $X=(x_1,\dots,x_m)$ of diameter at most $1-q_1$, a branching variant of our algorithm outputs $(\bsig_1,\dots,\bsig_n)$ with
    \[
        \plim_{N\to\infty}
        \max_{1\leq i,j\leq m}\tnorm{
            \vR(\bsig_i,\bsig_j)
            - \Phi\big(1-d_X(x_i,x_j)\big)
        }_{\infty}=0.
    \]
    We omit the details of this extension.
\end{remark}


% \section{Attaining the Ground State Energy in Full-RSB models}

\begin{align*}
    \CS(\zeta,\Phi,\vDelta)
    &= 
    \sum_{s\in \sS}
    \fr{\lambda_s}{2}
    \lt[
        \lambda_s^{-1}h_s^2\Delta_s(0)
        + 
        \int_0^1 \fr{\Phi'_s(q)}{\Delta_s(q)} + (\xi^s\circ \Phi)'(q)\Delta_s(q)~\de q
    \rt],\quad \text{where} \\
    \Delta_s(q) &= \Delta_s + \int_q^1 \zeta(t)\Phi'_s(t)~\de t.
\end{align*}

Suppose $\zeta$ is supported on $[q_1,1]$.
Note that because $\Delta_s(q)$ is constant for $q\in [0, q_1]$, $\CS(\zeta,\Phi,\vDelta) = \CS_1 + \CS_2$ for
\begin{align*}
    \CS_1 &= 
    \sum_{s\in \sS}
    \fr{\lambda_s}{2} 
    \lt[
        \fr{\Phi_s(q_1)}{\Delta_s(q_1)}
        +
        ((\xi^s \circ \Phi)(q) + h_s^2/\lambda_s) \Delta_s(q_1)
    \rt], \\
    \CS_2 &= 
    \sum_{s\in \sS}
    \fr{\lambda_s}{2} 
    \int_{q_1}^1
    \fr{\Phi'_s(q)}{\Delta_s(q)} + 
    (\xi^s\circ \Phi)'(q)\Delta_s(q)~\de q.
\end{align*}
Consider the perturbation
\[
    \tPhi_1(q) = \Phi_1(q) + \delta \psi(q)
\]
and $\tPhi_s(q) = \Phi_s(q)$ for $s\neq 1$.
Here $\psi$ is supported on $[q_1,1]$ with $\psi(q_1) = \psi(1)=0$.
Then 
\[
    \fr{\de}{\de \delta} (\xi^s \circ \Phi)(q) 
    = 
    (\psi \times \partial_1 \xi^s \circ \Phi)(q)
    =
    \fr{\lambda_s}{\lambda_1} (\psi \times \partial_s \xi^1 \circ \tPhi)(q),
    \qquad
    \fr{\de}{\de \delta} \Delta_1(q) 
    = 
    \int_q^1 \zeta(t) \psi'(t)~\de t
\]
and $\Delta_s(q)$ for $s\neq 1$ does not change under the perturbation.
Now,
\[
    F_1 
    \equiv
    \fr{\de}{\de \delta}
    2 \lambda_1^{-1} \CS_1 \Big|_{\delta=0} 
    = 
    \lt(
        -\fr{\Phi_1(q_1)}{\Delta_1(q_1)^2} 
        +\lt((\xi^1 \circ \Phi)(q_1) + h_1^2/\lambda_1\rt)
    \rt) \lt(
        \int_{q_1}^1 \zeta(q) \psi'(q) ~\de q
    \rt).
\]
Moreover,
\[
    F_2 
    \equiv 
    \fr{\de}{\de \delta}
    2 \lambda_1^{-1} \CS_2 \Big|_{\delta=0} 
    = F_{2,1} + F_{2,2} + F_{2,3} 
\]
for
\begin{align*}
    F_{2,1}
    &= 
    \int_{q_1}^1 \fr{\psi'(q)}{\Delta_1(q)} ~\de q 
    = 
    -\int_{q_1}^1 
    \fr{\psi(q) \zeta(q) \Phi'_1(q)}{\Delta_1(q)^2} ~\de q \\
    F_{2,2}
    &= 
    \sum_{s\in \sS}
    \int_{q_1}^1 
    (\psi \times \partial_s\xi^1 \circ \Phi)'(q) 
    \Delta_s(q) ~\de q 
    =
    \sum_{s\in \sS} 
    \int_{q_1}^1 
    \psi(q) \zeta(q) (\partial_s \xi^1 \circ \Phi)(q) \Phi'_s(q) 
    ~\de q\\
    &=
    \int_{q_1}^1 
    \psi(q) \zeta(q) (\xi^1 \circ \Phi)'(q) 
    ~\de q \\
    F_{2,3}
    &= 
    \int_{q_1}^1 \lt(
        -\fr{\Phi'_1(q)}{\Delta_1(q)^2}
        + (\xi^1 \circ \Phi)'(q)
    \rt)\lt(
        \int_q^1 \zeta(t) \psi'(t) ~\de t
    \rt) ~\de q.
\end{align*}
Thus
\begin{align*}
    &\fr{\de}{\de \delta}
    \CS(\zeta,\tPhi,\vDelta) \Big|_{\delta = 0} \\
    &= \lt(
        -\fr{\Phi_1(q_1)}{\Delta_1(q_1)^2} 
        +\lt((\xi^1 \circ \Phi)(q_1) + h_1^2/\lambda_1\rt)
    \rt) \lt(
        \int_{q_1}^1 \zeta(q) \psi'(q) ~\de t
    \rt) \\
    &\quad +
    \int_{q_1}^1 
    \lt(
        -\fr{\Phi'_1(q)}{\Delta_1(q)^2}
        + (\xi^1 \circ \Phi)'(q)
    \rt)\lt(
        \psi(q) \zeta(q) +
        \int_q^1 \zeta(t) \psi'(t) ~\de t
    \rt) ~\de q \\
    &= 
    \lt(
        \fr{\Phi_1(q_1)}{\Delta_1(q_1)^2} 
        -\lt((\xi^1 \circ \Phi)(q_1) + h_1^2/\lambda_1\rt)
    \rt) \lt(
        \int_{q_1}^1 \psi(q) \zeta'(q) ~\de t
    \rt) \\
    &\quad +
    \int_{q_1}^1 
    \lt(
        \fr{\Phi'_1(q)}{\Delta_1(q)^2}
        - (\xi^1 \circ \Phi)'(q)
    \rt)\lt(
        \int_q^1 \psi(t) \zeta'(t) ~\de t
    \rt) ~\de q \\
    &= 
    \int_{q_1}^1 
    \psi(q) \zeta'(q) \lt(
        \fr{\Phi_1(q_1)}{\Delta_1(q_1)^2} 
        -\lt((\xi^1 \circ \Phi)(q_1) + h_1^2/\lambda_1\rt)
        + 
        \int_{q_1}^q 
        \lt(
            \fr{\Phi'_1(t)}{\Delta_1(t)^2}
            - (\xi^1 \circ \Phi)'(t)
        \rt)
        ~\de t
    \rt) 
    ~\de q.
\end{align*}
Since $\zeta'>0$ on all of $[q_1,1]$, we have that
\[
    \fr{\Phi_1(q_1)}{\Delta_1(q_1)^2} 
    -\lt((\xi^1 \circ \Phi)(q_1) + h_1^2/\lambda_1\rt)
    + 
    \int_{q_1}^q 
    \lt(
        \fr{\Phi'_1(t)}{\Delta_1(t)^2}
        - (\xi^1 \circ \Phi)'(t)
    \rt)
    ~\de t
    = 0
\]
almost everywhere on $[q_1,1]$.
This function is obviously continuous so it is zero everywhere on $[q_1,1]$.
Evaluating at $q=q_1$, we find
\[
    \Delta_1(q_1) 
    = 
    \sqrt{\fr{\Phi_1(q_1)}{(\xi^1 \circ \Phi)(q_1) + h_1^2/\lambda_1}}.
\]
Moreover, we have 
\[
    \Delta_1(q) = \sqrt{\fr{\Phi'_1(q)}{(\xi^1 \circ \Phi)'(q)}}
\]
almost everywhere on $[q_1,1]$.
Analogous equations hold with $s$ in place of $1$.
\bhcomment{appeal to some regularity here, below not totally rigorous}

The definitions of $\Delta_s(q)$ imply that
\[
    \zeta(q) = -\fr{1}{\Phi'_s(q)} \fr{\de}{\de q} \sqrt{\fr{\Phi'_1(q)}{(\xi^1 \circ \Phi)'(q)}}
\]
for all $s\in \sS$, i.e. $\Phi$ solves the type II ode on $[q_1,1]$.
Moreover comparing the two formulas for $\Delta$ above we get
\[
    \fr{\Phi'_1(q_1)}{(\xi^1 \circ \Phi)'(q_1)} = \fr{\Phi_1(q_1)}{(\xi^1 \circ \Phi)(q_1) + h_1^2/\lambda_1}
\]
and similarly for all $s\in \sS$, i.e. $\Phi(q_1)$ is solvable.
Construct $(p,\Phi,q_0)$ by making $p=1$ and $\Phi$ as above on $[q_1,1]$, and define $q_0$ and $(p,\Phi)$ on $[q_0,q_1]$ as the type I extension of $\Phi(q_1)$.
We can plug this into the alg functional, so $\ALG = \OPT$.

% Here $\psi$ is supported on $[q_1,1]$ with $\psi(1)=0$ (but not necessarily $\psi(q_1)=0$).
% Then 
% \[
%     \fr{\de}{\de \delta} (\xi^s \circ \Phi)(q) 
%     = 
%     (\psi \times \partial_1 \xi^s \circ \Phi)(q)
%     =
%     \fr{\lambda_s}{\lambda_1} (\psi \times \partial_s \xi^1 \circ \tPhi)(q),
%     \qquad
%     \fr{\de}{\de \delta} \Delta_1(q) 
%     = 
%     \int_q^1 \zeta(t) \psi'(t)~\de t
% \]
% and $\Delta_s(q)$ for $s\neq 1$ does not change under the perturbation.
% Now,
% \begin{align*}
%     F_1 
%     &\equiv
%     \fr{\de}{\de \delta}
%     2 \lambda_1^{-1} \CS_1 \Big|_{\delta=0} \\
%     &= 
%     \fr{\psi(q_1)}{\Delta_1(q_1)} 
%     + \sum_{s\in \sS} 
%     (\psi \times \partial_s \xi^1 \circ \Phi)(q_1) \Delta_s(q_1)
%     + \lt(
%         -\fr{\Phi_1(q_1)}{\Delta_1(q_1)^2} 
%         +\lt((\xi^1 \circ \Phi)(q_1) + h_1^2/\lambda_1\rt)
%     \rt) \lt(
%         \int_{q_1}^1 \zeta(t) \psi'(t) ~\de t
%     \rt).
% \end{align*}
% Moreover,
% \[
%     F_2 
%     \equiv 
%     \fr{\de}{\de \delta}
%     2 \lambda_1^{-1} \CS_2 \Big|_{\delta=0} 
%     = F_{2,1} + F_{2,2} + F_{2,3} 
% \]
% for
% \begin{align*}
%     F_{2,1}
%     &= 
%     \int_{q_1}^1 \fr{\psi'(q)}{\Delta_1(q)} ~\de q 
%     = 
%     -\fr{\psi(q_1)}{\Delta_1(q_1)} 
%     -\int_{q_1}^1 
%     \fr{\psi(q) \zeta(q) \Phi'_1(q)}{\Delta_1(q)^2} ~\de q \\
%     F_{2,2}
%     &= 
%     \sum_{s\in \sS}
%     \int_{q_1}^1 
%     (\psi \times \partial_s\xi^1 \circ \Phi)'(q) 
%     \Delta_s(q) ~\de q \\
%     &=
%     - \sum_{s\in \sS}
%     (\psi \times \partial_s\xi^1 \circ \Phi)(q_1)\Delta_s(q_1) 
%     + \sum_{s\in \sS} 
%     \int_{q_1}^1 
%     \psi(q) \zeta(q) (\partial_s \xi^1 \circ \Phi)(q) \Phi'_s(q) 
%     ~\de q\\
%     &=
%     - \sum_{s\in \sS}
%     (\psi \times \partial_s\xi^1 \circ \Phi)(q_1)\Delta_s(q_1) 
%     +
%     \int_{q_1}^1 
%     \psi(q) \zeta(q) (\xi^1 \circ \Phi)'(q) 
%     ~\de q \\
%     F_{2,3}
%     &= 
%     \int_{q_1}^1 \lt(
%         -\fr{\Phi'_1(q)}{\Delta_1(q)^2}
%         + (\xi^1 \circ \Phi)'(q)
%     \rt)\lt(
%         \int_q^1 \zeta(t) \psi'(t) ~\de t
%     \rt) ~\de q.
% \end{align*}
% Thus
% \begin{align*}
%     &\fr{\de}{\de \delta}
%     \CS(\zeta,\tPhi,\vDelta) \Big|_{\delta = 0} \\
%     &= 
%     \lt(
%         -\fr{\Phi_1(q_1)}{\Delta_1(q_1)^2} 
%         +\lt((\xi^1 \circ \Phi)(q_1) + h_1^2/\lambda_1\rt)
%     \rt) \lt(
%         \int_{q_1}^1 \zeta(t) \psi'(t) ~\de t
%     \rt) \\
%     &\quad +
%     \int_{q_1}^1 
%     \lt(
%         -\fr{\Phi'_1(q)}{\Delta_1(q)^2}
%         + (\xi^1 \circ \Phi)'(q)
%     \rt)\lt(
%         \psi(q) \zeta(q) +
%         \int_q^1 \zeta(t) \psi'(t) ~\de t
%     \rt) ~\de q \\
%     &= 
%     \lt(
%         \fr{\Phi_1(q_1)}{\Delta_1(q_1)^2} 
%         -\lt((\xi^1 \circ \Phi)(q_1) + h_1^2/\lambda_1\rt)
%     \rt) \lt(
%         \psi(q_1) \zeta(q_1) 
%         + \int_{q_1}^1 \psi(t) \zeta'(t) ~\de t
%     \rt) \\
%     &\quad +
%     \int_{q_1}^1 
%     \lt(
%         \fr{\Phi'_1(q)}{\Delta_1(q)^2}
%         - (\xi^1 \circ \Phi)'(q)
%     \rt)\lt(
%         \int_q^1 \psi(t) \zeta'(t) ~\de t
%     \rt) ~\de q \\
%     &= 
%     \psi(q_1) \zeta(q_1) \lt(
%         \fr{\Phi_1(q_1)}{\Delta_1(q_1)^2} 
%         -\lt((\xi^1 \circ \Phi)(q_1) + h_1^2/\lambda_1\rt)
%     \rt) \\
%     &\quad + 
%     \int_{q_1}^1 
%     \psi(q) \zeta'(q) \lt(
%         \fr{\Phi_1(q_1)}{\Delta_1(q_1)^2} 
%         -\lt((\xi^1 \circ \Phi)(q_1) + h_1^2/\lambda_1\rt)
%         + 
%         \int_{q_1}^q 
%         \lt(
%             \fr{\Phi'_1(t)}{\Delta_1(t)^2}
%             - (\xi^1 \circ \Phi)'(t)
%         \rt)
%         ~\de t
%     \rt) 
%     ~\de q.
% \end{align*}

\section*{Acknowledgements}

We thank Mehtaab Sawhney for pointing us to \cite{bandeira2021matrix}, and Jean-Christophe Mourrat and Nike Sun for helpful feedback. 
B.H. was supported by an NSF Graduate Research Fellowship, a Siebel scholarship, NSF awards CCF-1940205 and DMS-1940092, and NSF-Simons collaboration grant DMS-2031883.
M.S. was supported by an NSF
graduate research fellowship, the William R. and Sara Hart Kimball Stanford graduate fellowship, and NSF
award CCF-2006489 and was a member at the IAS while parts of this work were completed. 
The initial ideas for this work were generated while the authors were visiting the Computational Complexity of Statistical Inference program at the Simons Institute in Fall 2021. 

\small
\bibliographystyle{alpha}
\bibliography{all-bib}
\normalsize

\appendix
\section{Equivalence of $\BOGP$ and $\BOGP_{\loc,0}$}
\label{sec:equivalence-of-bogps}

In this section, we prove Proposition~\ref{prop:bogp-equivalent} that $\BOGP = \BOGP_{\loc,0}$.
We introduce two other limits $\BOGP_{\den}$ and $\BOGP_{\loc}$, as follows (restating $\BOGP$ and $\BOGP_{\loc,0}$ for convenience). 
\begin{align*}
    \BOGP
    &= 
    \lim_{D\to\infty}
    \lim_{\eta\to 0}
    \lim_{k\to\infty}
    \sup_{\vchi \in \bbI(0,1)^\sS}
    \inf_{\uvphi=\vchi(\up)}
    \limsup_{N\to\infty}
    \fr{1}{N} 
    \bbE \sup_{\ubsig \in \cQ(\eta)}
    \cH_N(\ubsig), \\
    \BOGP_{\den}
    &= 
    \lim_{D\to\infty}
    \lim_{\eta\to 0}
    \lim_{k\to\infty}
    \sup_{\substack{
        \vchi \in \bbI(0,1)^\sS \\ 
        \text{$1/D^2$-separated}
    }}
    \inf_{\substack{
        \uvphi=\vchi(\up) \\
        \text{$6r/D$-dense}
    }}
    \limsup_{N\to\infty}
    \fr{1}{N} 
    \bbE \sup_{\ubsig \in \cQ(\eta)}
    \cH_N(\ubsig), \\
    \BOGP_{\loc}
    &=
    \lim_{D\to\infty}
    \lim_{\eta\to 0}
    \lim_{k\to\infty}
    \sup_{\substack{
        \vchi \in \bbI(0,1)^\sS \\ 
        \text{$1/D^2$-separated}
    }}
    \inf_{\substack{
        \uvphi=\vchi(\up) \\
        \text{$6r/D$-dense}
    }}
    \limsup_{N\to\infty}
    \fr{1}{N} 
    \bbE \sup_{\ubsig \in \cQ_{\loc}(\eta)}
    \cH_N(\ubsig), \\
    \BOGP_{\loc,0}
    &=
    \lim_{D\to\infty}
    \lim_{k\to\infty}
    \sup_{\substack{
        \vchi \in \bbI(0,1)^\sS \\ 
        \text{$1/D^2$-separated}
    }}
    \inf_{\substack{
        \uvphi=\vchi(\up) \\ 
        \text{$6r/D$-dense}
    }}
    \limsup_{N\to\infty}
    \fr{1}{N} 
    \bbE \sup_{\ubsig \in \cQ_{\loc}(0)}
    \cH_N(\ubsig).
\end{align*}
In the last three lines, the limits in $k, \eta$ are clearly decreasing, but the limits in $D$ are not, so the existence of these limits needs to be proven.
Proposition~\ref{prop:bogp-equivalent} follows from the following propositions.
\begin{proposition}
    \label{prop:bogp-den}
    The limit $\BOGP_{\den}$ exists and $\BOGP = \BOGP_{\den}$. 
\end{proposition}

\begin{proposition}
    \label{prop:bogp-loc}
    The limit $\BOGP_{\loc}$ exists and $\BOGP_{\den} = \BOGP_{\loc}$. 
\end{proposition}

\begin{proposition}
    \label{prop:bogp-loc0}
    The limit $\BOGP_{\loc,0}$ exists and $\BOGP_{\loc} = \BOGP_{\loc,0}$. 
\end{proposition}

\subsection{Equivalence of $\BOGP$ and $\BOGP_{\den}$}

Let $\up' = (p_0,\ldots,p_{D'})$ and $\uvphi' = (\vphi'_0,\ldots,\vphi'_{D'})$. 
Say $(\up',\uvphi')$ \textbf{refines} $(\up,\uvphi)$ if there exists $0 \le a_0 < \cdots < a_D \le D'$ such that $(p_d,\vphi_d) = (p'_{a_d}, \vphi'_{a_d})$ for all $0\le d\le D$.

\begin{lemma}
    \label{lem:refinement}
    The value $\bbE \max_{\ubsig \in \cQ(\eta)} \cH_N(\ubsig)$ is decreasing under refinement. 
    That is, if $(\up',\uvphi')$ refines $(\up,\uvphi)$, then for any $k,\eta$,
    \[
        \bbE \max_{\ubsig \in \cQ^{k,D,\uvphi}(\eta)} \cH_N^{k,D,\up} (\ubsig)
        \ge \bbE \max_{\ubsig \in \cQ^{k,D',\uvphi'}(\eta)} \cH_N^{k,D',\up'} (\ubsig).
    \]
\end{lemma}
\begin{proof}
    Let $I = \{a_0,\ldots,a_D\}$ and $J = [D'] \setminus I$. 
    Define an equivalence relation $\bowtie$ on $\bbL(k,D')$ by $u\bowtie v$ if $u_d=v_d$ for all $d\in J$. 
    Let 
    \[
        \cQ' = 
        \lt\{
            \ubsig \in \cB_N^{\bbL(k,D')} : 
            \norm{\vR(\bsig(u^1),\bsig(u^2)) - \vphi'_{u^1\wedge u^2}}_\infty 
            \le \eta,
            ~\forall u^1 \bowtie u^2
        \rt\}
    \]
    be the superset of $\cQ^{k,D',\uvphi'}$ where we only enforce the overlap constraint for $u^1\bowtie u^2$.
    Then
    \begin{align*}
        \bbE \max_{\ubsig \in \cQ^{k,D',\uvphi'}(\eta)} \cH_N^{k,D',\up'} (\ubsig)
        &\le \bbE \max_{\ubsig \in \cQ'} \cH_N^{k,D',\up'} (\ubsig) \\
        &= \bbE \max_{\ubsig \in \cQ'} \fr{1}{k^{D'-D}} \sum_{u_J \in [k]^{D'-D}} \fr{1}{k^D} \sum_{u_I \in [k]^D} \cH_N^{k,D',\up'} (\ubsig) \\
        &= \bbE \max_{\ubsig \in \cQ^{k,D,\uvphi}(\eta)} \cH_N^{k,D,\up} (\ubsig).
    \end{align*}
\end{proof}

\begin{proof}[Proof of Proposition~\ref{prop:bogp-den}]
    Let $\BOGP_{\den}^+$ and $\BOGP_{\den}^-$ be $\BOGP_{\den}$ where the outer limit in $D$ is replaced by $\limsup$ and $\liminf$, respectively.
    We will separately prove $\BOGP \ge \BOGP_{\den}^+$ and $\BOGP \le \BOGP_{\den}^-$.

    Fix any $D,k,\eta$, $1/D^2$-separated $\vchi$, and (not necessarily $6r/D$-dense) $\up,\uvphi$ satisfying $\uvphi = \vchi(\up)$.
    Let $\delta = (r+1)/D$.
    Let $\tp_0 = p_0$, and define a sequence $\tp_1,\ldots,\tp_{\tilde D}$, where $\tp_{d+1}$ is the smallest $p\in [\tp_d,1]$ such that 
    \[
        \max\lt(p-\tp_d, \norm{\vchi(p) - \vchi(\tp_d)}_\infty\rt) \ge \delta
    \]
    if such $p$ exists.
    Let $\tilde D$ be the first index $d$ such that no such $p$ exists.
    Note that if $\Sigma_d = \tp_d + \tnorm{\vchi(\tp_d)}_1$, then $0\le \Sigma_d \le r+1$ and $\Sigma_{d+1} - \Sigma_{d} \ge \delta$ for all $d$.
    Thus $\tilde D \le (r+1)/\delta = D$.

    Consider either $D'=2D$ or $D'=2D+1$.
    Let $\up'$ be the sorted union of $\{p_0,\ldots,p_D\}$, $\{\tp_1,\ldots,\tp_{\tilde D}\}$, and (if necessary) additional arbitrary $p\in [0,1]$, so that $\up'$ has length $D'$.
    Define $\uvphi' = \vchi(\up')$.
    Since $(\up',\uvphi')$ refines $(\up,\uvphi)$, Lemma~\ref{lem:refinement} implies
    \[
        \bbE \max_{\ubsig \in \cQ^{k,D,\uvphi}(\eta)} \cH_N (\ubsig)
        \ge \bbE \max_{\ubsig \in \cQ^{k,D',\uvphi'}(\eta)} \cH_N^{k,D',\up'} (\ubsig).
    \]
    Moreover, one can check that $\delta \le 6r/D'$, so $(\up',\uvphi')$ is $6r/D'$-dense.
    Thus, if $f(D)$ and $g(D)$ are the quantities inside the outer limits of $\BOGP$ and $\BOGP_{\den}$, we have shown $f(D) \ge g(2D), g(2D+1)$ (as taking the supremum over not necessarily $1/D^2$-separated $\vchi$ can only increase $f(D)$). 
    This implies $\BOGP \ge \BOGP_{\den}^+$.

    For the other direction, fix $D,k,\eta$ and (not necessarily $1/D^2$-separated) $\vchi$. 
    Define
    \[
        \vchi'(p) = (1-D^{-2}) \vchi(p) + D^{-2} \vone,
    \]
    so $\vchi'$ is $1/D^2$-separated.
    Consider any $6r/D$-dense $(\up,\uvphi')$ with $\uvphi' = \vchi'(\up)$, and let $\uvphi = \vchi(\up)$. 

    Let $\ubsig \in \cQ^{k,D,\uvphi}(\eta)$.
    Let $\bx$ satisfy $\vR(\bx,\bx) = \vone$ and $\vR(\bx,\bsig(u))=\vzero$ for all $u\in \bbL$.
    Define
    \[
        \brho(u) = \sqrt{1-D^{-2}}\bsig(u) + D^{-1} \bx,
    \]
    so that for all $u, v \in \bbL$,
    \[
        \norm{\vR(\brho(u),\brho(v)) - \vphi'_{u\wedge v}}_\infty
        = (1-D^{-2}) \norm{\vR(\bsig(u),\bsig(v)) - \vphi_{u\wedge v}}_\infty
        \le \eta.
    \]
    Thus $\ubrho \in \cQ^{k,D,\uvphi'}(\eta)$, and we can easily check that
    \[
        \fr{1}{\sqrt{N}} \norm{\brho(u) - \bsig(u)}_2 = O(D^{-2})
    \]
    for all $u\in \bbL$.
    By Proposition~\ref{prop:gradients-bounded}, with probability $1-e^{-\Omega(N)}$ we have $\HNp{u}\in K_N$ for all $u\in \bbL$. 
    On this event, 
    \[
        \lt|\fr{1}{N} \cH_N(\ubrho) - \fr{1}{N} \cH_N(\ubsig)\rt| \le CD^{-2}
    \]
    for some $C>0$, and so 
    \[
        \fr{1}{N} \sup_{\ubrho \in \cQ^{k,D,\uvphi'}(\eta)} \cH_N(\ubrho)
        + CD^{-2}
        \ge 
        \fr{1}{N} \sup_{\ubsig \in \cQ^{k,D,\uvphi}(\eta)} \cH_N(\ubsig).
    \]
    By Lemma~\ref{lem:bogp-subgaussian}, both sides of this inequality are subgaussian with fluctuations $O(N^{-1/2})$, so the contribution from the complement of this event is $o_N(1)$, and 
    \[
        \limsup_{N\to\infty} 
        \fr{1}{N} \bbE \sup_{\ubrho \in \cQ^{k,D,\uvphi'}(\eta)} \cH_N(\ubrho)
        + CD^{-2}
        \ge 
        \limsup_{N\to\infty}
        \fr{1}{N} \bbE \sup_{\ubsig \in \cQ^{k,D,\uvphi}(\eta)} \cH_N(\ubsig).
    \]
    Thus $f(D) \le g(D) + CD^{-2}$ (as taking the infimum over $(\up,\uvphi)$ that are not necessarily the image of a $6r/D$-dense $(\up,\uvphi')$ under the above transformation can only decrease $f(D)$).
    This implies $\BOGP \le \BOGP_{\den}^-$.
\end{proof}

\subsection{Equivalence of $\BOGP_{\den}$ and $\BOGP_{\loc}$}

\begin{lemma}
    \label{lem:recursive-barycenter}
    We have that $\cQ(\eta) \subseteq \cQ_{\loc}(\eta+\fr{2}{k})$.
\end{lemma}
\begin{proof}
    Consider any $\ubsig \in \cQ(\eta)$. 
    We define $\ubrho \in \cB_N^\bbT$ by $\brho(u) = \bsig(u)$ if $u\in \bbL$, and
    \[
        \brho(u) = \fr{1}{k} \sum_{i=1}^k \brho(ui)
    \]
    for $u\in \bbT \setminus \bbL$.
    We will show that $\ubrho \in \cQ_{\loc+}(\eta+\fr{2}{k})$, and so $\ubsig \in \cQ_{\loc}(\eta+\fr{2}{k})$.
    
    Let $v\succeq u$ denote that $v$ is a descendant of $u$ in $\bbT$.
    Consider any non-leaf $u\in \bbT$ and two of its children $ui,uj$, for $i\neq j$. 
    For any $s\in \sS$,
    \begin{equation}
        \label{eq:sibling-overlap}
        |R_s(\brho(ui),\brho(uj)) - \phi_{|u|}^s|
        \le 
        \fr{1}{k^{2(D-|u|)}}
        \sum_{\substack{v,v' \in \bbL \\ v \succeq ui, v' \succeq uj}}
        |R_s(\bsig(v),\bsig(v')) - \phi_{|u|}^s|
        \le \eta.
    \end{equation}
    Moreover,
    \[
        |R_s(\brho(ui),\brho(u)) - \phi_{|u|}^s|
        \le 
        \fr1k 
        \sum_{j=1}^k 
        |R_s(\bsig(ui),\bsig(uj)) - \phi_{|u|}^s|
        \le 
        \eta + \fr{2}{k},
    \]
    where we bounded the terms $j\neq i$ by \eqref{eq:sibling-overlap} and the term $j=i$ crudely by $2$.
    Thus,
    \[
        |R_s(\brho(u),\brho(u)) - \phi_{|u|}^s|
        \le 
        \fr{1}{k}
        \sum_{i=1}^k
        |R_s(\bsig(ui),\bsig(u)) - \phi_{|u|}^s|
        \le 
        \eta + \fr{2}{k}.
    \]
\end{proof}

For $k'\le k$, define a \emph{$k'$-ary subtree} of $\bbT$ to be a subset $T\subseteq \bbT$ isometric to $\bbT(k',D)$.
The following fact is clear from the definition of $\cQ_{\loc}(\eta)$. 
\begin{fact}
    \label{fac:subtree-satisfy-local}
    Let $T\subseteq \bbT$ be a $k'$-ary subtree with leaf set $L$.
    If $\ubsig \in \cQ_{\loc}(\eta)$, then $(\bsig(u))_{u\in L} \in \cQ_{\loc}^{k',D,\uvphi}(\eta)$.
\end{fact}
\begin{proof}
    There exists $\ubrho \in \cQ_{\loc+}(\eta)$ such that $\brho(u) = \bsig(u)$ for all $u\in \bbL$.
    Then $(\brho(u))_{u\in T} \in \cQ_{\loc+}^{k',D,\uvphi}(\eta)$, which implies the result.
\end{proof}

\begin{lemma}
    \label{lem:prune-global-constraints}
    Let $k'$ be the largest integer solution to $D(k')^D \le \min(\sqrt{k}, \eta^{-1})$. 
    If $\ubsig \in \cQ_{\loc}(\eta)$, there exists a $k'$-ary subtree $T$ of $\bbT$ with leaf set $L$ such that $(\bsig(u))_{u\in L} \in \cQ^{k',D,\uvphi}(CD^2 (k^{-1/4} + \eta^{1/4}))$, for some $C >0$. 
\end{lemma}
\begin{proof}
    Let $\ubrho \in \cQ_{\loc+}(\eta)$ such that $\brho(u) = \bsig(u)$ for all $u\in \bbL$. 
    We will construct $T$ by a breadth-first search: we start from $T = \{\emptyset\}$ and each step \emph{process} a leaf $u$ of $T$ by adding $k'$ children of $u$ to $T$, until all leaves of $T$ are of depth $D$.
    
    Suppose we are currently processing vertex $u$. 
    Let $V =  \{\brho(v) : v\in T\}$ and $S = \text{span}(V)$; note that $|V| \le D(k')^D \le \min(\sqrt{k}, \eta^{-1})$.
    Let $P_S$ denote the projection operator onto $S$. 
    For $i\in [k]$, write $\bx^i = \fr{1}{\sqrt{N}} (\brho(ui) - \brho(u))$, and note $\norm{\bx^i}_2 \le 2$.
    Then
    \[
        \fr{1}{N} \sum_{i=1}^k \norm{P_S(\brho(ui)-\brho(u))}_2^2
        = 
        \sum_{i=1}^k \norm{P_S \bx^i}_2^2
    \]
    is upper bounded by the sum of the top $|V|$ eigenvalues of the Gram matrix $\bM = (\la \bx^i,\bx^j\ra)_{i,j=1}^k$.
    However, for $i\neq j$, 
    \[
        |\la \bx^i,\bx^j\ra|
        = \fr{1}{N} |\la \brho(ui)-\brho(u), \brho(uj)-\brho(u)\ra|
        \le \sum_{s\in \sS} \lambda_s |R_s(\brho(ui)-\brho(u), \brho(uj)-\brho(u))| 
        \le 4\eta,
    \]
    while $|\la \bx^i,\bx^i\ra| \le 4$.
    So, if we let $\bM = \bD + \bA$ where $\bD = \diag(\bM)$, and let $a_1 \ge \cdots \ge a_{|V|}$ be the top $|V|$ eigenvalues of $\bA$, then the sum of the top $|V|$ eigenvalues of $\bM$ is upper bounded by $4|V| + \sum_{i=1}^{|V|} a_i$. 
    However,
    \[
        \sum_{i=1}^{|V|} a_i
        \le 
        \sqrt{|V| \sum_{i=1}^{|V|} a_i^2}
        \le 
        \sqrt{|V| \norm{\bA}_F^2} 
        \le 4k\eta \sqrt{|V|}.
    \]
    It follows that
    \[
        \fr{1}{kN} \sum_{i=1}^k \norm{P_S(\brho(ui)-\brho(u))}_2^2
        \le 
        \fr{4|V|}{k} + 4\eta \sqrt{|V|}
        \le 4(k^{-1/2} + \eta^{1/2}),
    \]
    where the last step follows from $|V| \le \min(\sqrt{k}, \eta^{-1})$.
    Thus there are $k'$ children $ui$ of $u$ such that
    \[
        \fr{1}{\sqrt{N}} \norm{P_S(\brho(ui)-\brho(u))}_2 
        \le 3(k^{-1/4} + \eta^{1/4}).
    \]
    We choose these as the children of $u$ in $T$.
    By constructing $T$ in this manner, we get that for all distinct edges $(u,ui)$, $(v,vj)$ in $T$, 
    \[
        \fr{1}{N} |\la \bsig(ui)-\bsig(u), \bsig(vj)-\bsig(v) \ra|, 
        \fr{1}{N} |\la \bsig(ui)-\bsig(u), \bsig(\emptyset) \ra| 
        \le 6(k^{-1/4} + \eta^{1/4}).
    \]
    whence 
    \[
        \norm{\vR(\bsig(ui),\bsig(u), \bsig(vj)-\bsig(v))}_\infty, 
        \norm{\vR(\bsig(ui),\bsig(u), \bsig(\emptyset))}_\infty 
        \le \fr{6(k^{-1/4} + \eta^{1/4})}{\min_s \lambda_s}.
    \]
    We now verify that $(\bsig(u))_{u\in L} \in \cQ^{k',D,\uvphi}(CD^2(k^{-1/4} + \eta^{1/4}))$.
    Consider any $u,v\in L$ with least common ancestor $w$, and let $|w|=d$. 
    Let $(u_0,\ldots,u_{D-d})$ and $(v_0,\ldots,v_{D-d})$ be the paths from $w$ to $u,v$, with $u_0=v_0=w$ and $u_{D-d}=u$, $v_{D-d}=v$, and let $(w_0,\ldots,w_d)$ be the path from $\emptyset$ to $w$, with $w_0=\emptyset$, $w_d = w$. 
    Also define as convention $\bsig(w_{-1}) = \bzero$.
    Then,
    \begin{align*}
        \norm{\vR(\bsig(u),\bsig(v)) - \vphi_d}_\infty 
        &\le 
        \norm{\vR(\bsig(w),\bsig(w)) - \vphi_d}_\infty
        + \sum_{i=1}^{D-d} \sum_{\ell=0}^d \norm{\vR(\bsig(w_\ell) - \bsig(w_{\ell-1}), \bsig(u_i)-\bsig(u_{i-1}))}_\infty \\
        &\quad + \sum_{j=1}^{D-d} \sum_{\ell=0}^d \norm{\vR(\bsig(w_\ell) - \bsig(w_{\ell-1}), \bsig(v_j)-\bsig(v_{j-1}))}_\infty \\
        &\quad + \sum_{i,j=1}^{D-d} \norm{\vR(\bsig(u_i)-\bsig(u_{i-1}),\bsig(v_j)-\bsig(v_{j-1}))}_\infty \\
        &\le CD^2(k^{-1/4} + \eta^{1/4}).
    \end{align*}
\end{proof}

\begin{lemma}
    \label{lem:prune-all-leaves-good}
    There exists $C>0$ such that with probability $1-e^{-\Omega(N)}$ over the Hamiltonians $\HNp{u}$ the following holds.
    If $\eps > 0$, $\ubsig \in \cQ_{\loc}(\eta)$, and 
    \[
        \fr{1}{N} \cH(\ubsig) \ge E,
    \]
    then for $k' = \lfloor k\eps/3CD\rfloor$, there exists a $k'$-ary subtree $T$ of $\bbT$ with leaf set $L$ such that
    \[
        \fr{1}{N} \HNp{u}(\bsig(u)) \ge E-\eps
    \]
    for all $u\in L$.
\end{lemma}
\begin{proof}
    We consider the event that $\HNp{u} \in K_N$ for all $u\in \bbL$, for $K_N$ defined in Proposition~\ref{prop:gradients-bounded}. 
    This holds with probability $1-e^{-\Omega(N)}$, and on this event, $|\HNp{u}(\bsig(u))| \le C$ for all $u\in \bbL$.
    For $u\in \bbT$ define
    \[
        F(u) = \fr{1}{Nk^{D-|u|}} 
        \sum_{\substack{v\in \bbL \\ v\succeq u}}
        H^{(v)}(\bsig(v)).
    \]
    We will show that for any $u\in \bbT \setminus \bbL$, we may find $k'$ distinct children $ui_1,\ldots,ui_{k'}$ such that $F(ui_j) \ge F(u) - \eps/D$ for all $j$.
    Indeed, we have
    \[
        F(u) = \fr{1}{k} \sum_{i=1}^k F(ui),
    \]
    and $|F(ui)| \le C$ for all $i$, so the claim follows from Markov's inequality.

    We construct the subtree $T$ recursively starting from $\emptyset$, using the above claim to select the $k'$ children of each node.
    Thus, for all $u,ui\in T$ with $ui$ a child of $u$, we have $F(ui) \ge F(u) - \eps/D$. 
    Since $F(\emptyset) = \fr{1}{N} \cH_N(\ubsig) \ge E$, the result follows.
\end{proof}

\begin{proof}[Proof of Proposition~\ref{prop:bogp-loc}]
    Let $\BOGP_{\loc}^+$ and $\BOGP_{\loc}^-$ be $\BOGP_{\loc}$ where the outer limit in $D$ is replaced by $\limsup$ and $\liminf$, respectively.
    Lemma~\ref{lem:recursive-barycenter} gives $\BOGP_{\den} \le \BOGP_{\loc}^-$, so it suffices to prove $\BOGP_{\den} \ge \BOGP_{\loc}^+$.

    Fix arbitrary $\eps>0$, $D,k,\eta$, $1/D^2$-dense $\vchi$, and $6r/D$-dense $(\up,\uvphi)$ satisfying $\uvphi = \vchi(\up)$.
    If $\ubsig \in \cQ_{\loc}(\eta)$ and $\fr{1}{N} \cH_N(\ubsig) \ge E$, then on an event with probability $1-e^{-\Omega(N)}$, Lemma~\ref{lem:prune-all-leaves-good} gives a $k'$-ary subtree $T\subseteq \bbT$ with leaf set $L$ such that $\fr{1}{N} \HNp{u}(\bsig(u)) \ge E-\eps$ for all $u\in L$.
    However, $(\bsig(u))_{u\in L}$ is itself an element of $\cQ_{\loc}^{k',D,\uvphi}(\eta)$ by Fact~\ref{fac:subtree-satisfy-local}, so Lemma~\ref{lem:prune-global-constraints} gives a $k''$-ary subtree $T' \subseteq T$ with leaf set $L'$ such that $(\bsig(u))_{u \in L'} \in \cQ^{k'',D,\uvphi}(\eta')$.
    Here $k' = \lfloor \eps / 3CD \rfloor$, $k''$ is the largest solution to $D(k'')^D \le \min(\sqrt{k'}, \eta^{-1})$, and $\eta' = CD^2 ((k')^{-1/4} + \eta^{1/4})$.
    
    It follows that for all $E$,
    \[
        \bbP\lt[
            \fr{1}{N}
            \sup_{\ubsig \in \cQ^{k'',D,\uvphi}(\eta')}
            \cH_N^{k'',D,\up}(\ubsig) \ge E - \eps
        \rt]
        \ge 
        \binom{k}{k''}^{-D}
        \bbP\lt[
            \fr{1}{N}
            \sup_{\ubsig \in \cQ_{\loc}^{k,D,\uvphi}(\eta)}
            \cH_N^{k,D,\up}(\ubsig) \ge E
        \rt]
        - e^{-\Omega(N)}.
    \]
    By Lemma~\ref{lem:bogp-subgaussian}, the random variables in these two probabilities are both subgaussian with fluctuations $O(N^{-1/2})$.
    So
    \[
        \limsup_{N\to\infty} 
        \fr{1}{N}
        \bbE 
        \sup_{\ubsig \in \cQ^{k'',D,\uvphi}(\eta')}
        \cH_N^{k'',D,\up}(\ubsig)
        + \eps
        \ge 
        \limsup_{N\to\infty} 
        \fr{1}{N}
        \bbE
        \sup_{\ubsig \in \cQ_{\loc}^{k,D,\uvphi}(\eta)}
        \cH_N^{k,D,\up}(\ubsig).
    \]
    For fixed $D$, as $k\to\infty$ and $\eta\to 0$, we have $k'' \to \infty$ and $\eta' \to 0$.
    Then taking $D\to\infty$ shows $\BOGP_{\den} + \eps \ge \BOGP^+_{\loc}$. 
    Since $\eps$ was arbitrary, the result follows. 
\end{proof}

\begin{remark}
    \label{rem:min-BOGP}
    A byproduct of Lemma~\ref{lem:prune-all-leaves-good} is that defining $\cH_N$ as the  \textbf{minimum} over $u\in\bbL$ of the energies $H_N^{(u)}(\bsig(u))$, and $\BOGP$ in terms of this $\cH_N$, gives the same threshold as our definition \eqref{eq:grand-hamiltonian} of $\cH_N$ as the average of these energies. The minimal energy is actually more directly connected to our proof of Theorem~\ref{thm:main-ogp-oc}, as seen in the definition \eqref{eq:S-events} of $\Ssolve$. However the average energy is more convenient for our analysis in Section~\ref{sec:uc}.
\end{remark}

\subsection{Equivalence of $\BOGP_{\loc}$ and $\BOGP_{\loc,0}$}

\begin{lemma}
    \label{lem:gram-schmidt}
    Let $k\in \bbN$, $0 < q_0 \le q \le 1$ and $q',\eps \in [0,1]$. 
    There exists $\eps' = \eps'(\eps,k,q_0)$, where $\eps'\to 0$ as $\eps\to 0$ for fixed $k,q_0$, such that the following holds for all $q,q'$. 
    Suppose that $\bx,\by^1,\ldots,\by^k \in \bbR^N$ and 
    \[
        \bY = \begin{bmatrix}\bx & \by^1 & \cdots & \by^k\end{bmatrix}
    \]
    satisfies $\bY^\top \bY = D + E$, where $D = \diag(q,q',\ldots,q')$, all entries of $E$ have magnitude at most $\eps$, and $E_{1,1}=0$.
    There exist $\bz^1,\ldots,\bz^k$ such that for
    \[
        \bZ = \begin{bmatrix}\bx & \bz^1 & \cdots & \bz^k \end{bmatrix},
    \]
    we have $\bZ^\top \bZ = D$ and $\norm{\bz^i-\by^i}_2 \le \eps'$ for all $i\in [k]$.
\end{lemma}
\begin{proof}
    We will take 
    \[
        \eps' = 
        \begin{cases}
            2 & k^2 \eps \ge q_0, \\
            3k^{3/2} \eps^{1/2} & \text{otherwise}.
        \end{cases}
    \]
    If $k^3 \eps \ge q_0$, we let $\bz^1,\ldots,\bz^k$ be any orthogonal vectors of norm $\sqrt{q'}$ orthogonal to $\bx$ and each other.
    As $\norm{\by^i}_2,\norm{\bz^i}_2 \le 1$, the result follows.
    Similarly, if $k^3 \eps \ge q'$, then 
    \[
        \norm{\by^i-\bz^i}_2
        \le 
        \norm{\by^i}_2 + \norm{\bz^i}_2
        = \sqrt{q'+\eps} + \sqrt{q'}
        \le 3k^{3/2} \eps^{1/2}.
    \]
    It remains to address the case $k^3 \eps \le \min(q_0,q')$.
    We define $\bz^1,\ldots,\bz^k$ by the Gram-Schmidt algorithm, i.e. 
    \[
        \hbz^i = \by^i - \fr{\la \by^i, \bx \ra}{\norm{\bx}_2^2} \bx - \sum_{j=1}^{i-1} \fr{\la \by^i, \bz^j\ra}{\norm{\bz^j}_2} \bz^j,
        \qquad
        \bz^i = \fr{\sqrt{q'}}{\norm{\hbz^i}_2} \hbz^i.
    \]
    Let $\eps_i = \eps (1 + 3k^{-2})^i$, and note that $\eps \le \eps_i \le 2\eps$ for all $0\le i\le k$.
    We will show by induction over $i$ that for all $j\le i<\ell$,
    \begin{equation}
        \label{eq:gram-schmidt-induction-goal}
        |\la \by^\ell, \bz^j\ra| \le \eps_i,
    \end{equation}
    where as the base case this vacuously holds for $i=0$.
    Suppose the inductive hypothesis holds for $i-1$.
    It suffices to prove \eqref{eq:gram-schmidt-induction-goal} for $j=i$ because the assertion for the remaining $j$ is implied by the inductive hypothesis, as $\eps_{i-1} \le \eps_i$.
    We have
    \[
        \norm{\hbz^i}_2^2
        = 
        \norm{\by^i}_2^2
        - \fr{\la \by^i, \bx\ra^2}{\norm{\bx}_2^2}
        - \sum_{j=1}^{i-1} \fr{\la \by^i, \bz^j \ra^2}{\norm{\bz^j}_2^2}
    \]
    so
    \[
        \lt|\fr{\norm{\hbz^i}_2^2}{q'}-1\rt|
        \le 
        \lt|\fr{\norm{\by^i}_2^2}{q'}-1\rt|
        + \fr{\la \by^i, \bx\ra^2}{q'\norm{\bx}_2^2}
        + \sum_{j=1}^{i-1} \fr{\la \by^i, \bz^j \ra^2}{q'\norm{\bz^j}_2^2}
        \le 
        \fr{\eps_{i-1}}{q'} + \fr{\eps_{i-1}^2}{q'q_0} + \fr{k\eps_{i-1}^2}{(q')^2}
        \le \fr{2}{k^3}.
    \]
    Thus $\norm{\hbz^i}_2 \ge \sqrt{q'}(1-2k^{-3})$.
    For any $\ell > i$,
    \begin{align*}
        |\la \hbz^i, \by^\ell \ra| 
        &\le 
        |\la \by^i, \by^\ell\ra| 
        + \fr{|\la \by^i, \bx \ra||\la \bx, \by^\ell \ra|}{\norm{\bx}_2^2}  
        + \sum_{j=1}^{i-1} \fr{|\la \by^i, \bz^j\ra| |\la \bz^j, \by^\ell \ra|}{\norm{\bz^j}_2^2} \\
        &\le \eps_{i-1}  + \fr{\eps_{i-1}^2}{q_0} + \fr{k\eps_{i-1}^2}{q'_0} 
        \le \eps_{i-1} \lt(1 + \fr{2}{k^2}\rt)
    \end{align*}
    Thus 
    \[
        |\la \bz^i, \by^\ell \ra| 
        \le \eps_{i-1} \cdot \fr{1 + 2k^{-2}}{1 - 2k^{-3}} 
        \le \eps_i,
    \]
    completing the induction.
    Finally, note that
    \[
        \norm{\hbz^i-\by^j}_2^2
        =
        \fr{\la \by^i, \bx \ra^2}{\norm{\bx}_2^2} 
        + \sum_{j=1}^{i-1} \fr{\la \by^i, \bz^j\ra^2}{\norm{\bz^j}_2^2}
        \le \fr{\eps_{i-1}^2}{q_0} + \fr{k\eps_{i-1}^2}{q'} 
        \le \fr{5\eps}{k^2}
    \]
    and
    \[
        \lt|\norm{\hbz^i}_2 - \sqrt{q'}\rt|
        \le 
        \fr{\lt|\norm{\hbz^i}_2^2 - q'\rt|}{\sqrt{q'}}
        \le 
        \eps_{i-1} + \fr{\eps_{i-1}^2}{q'q_0} + \fr{k\eps_{i-1}^2}{(q')^2}
        \le 3\eps.
    \]
    Thus
    \[
        \norm{\bz^i-\by^i}_2
        \le 
        \norm{\hbz^i-\by^i}_2 + \lt|\norm{\hbz^i}_2 - \sqrt{q'}\rt|
        \le 
        \fr{\sqrt{5\eps}}{k} + 3\eps
        \le \eps'.
    \]
\end{proof}

\begin{proof}[Proof of Proposition~\ref{prop:bogp-loc0}]
    Let $\BOGP_{\loc,0}^+$ and $\BOGP_{\loc,0}^-$ be $\BOGP_{\loc,0}$ where the outer limit in $D$ is replaced by $\limsup$ and $\liminf$, respectively.
    It is clear that $\BOGP_{\loc} \ge \BOGP_{\loc,0}^+$, so it suffices to prove $\BOGP_{\loc} \le \BOGP_{\loc,0}^-$. 

    Fix $D,k,\eta$, $1/D^2$-separated $\vchi$, and $6r/D$-dense $(\up,\uvphi)$ with $\uvphi = \vchi(\up)$.
    Consider $\ubsig \in \cQ_{\loc}(\eta)$ and let $\ubrho \in \cQ_{\loc+}(\eta)$ such that $(\brho(u))_{u\in \bbL} = \ubsig$.
    Define $\eps_0 = \eta D$ and $\eps_d = \eps'(6\eps_{d-1} + 4\eta,k,D^{-2})$ for $1\le d\le D$, where $\eps'$ is given by Lemma~\ref{lem:gram-schmidt}.
    We will now construct $\ubtau \in \cQ_{\loc+}(0)$ approximating $\ubrho$ in the sense that for all $u\in \bbT, s\in \sS$,
    \begin{equation}
        \label{eq:gram-schmidt-invariant}
        \sqrt{R_s(\btau(u)-\brho(u),\btau(u)-\brho(u))}
        \le \eps_{|u|}.
    \end{equation}
    We define $\btau(\emptyset)$ by
    \[
        \btau(\emptyset)_s 
        = \brho(\emptyset)_s 
        \sqrt{\fr{\phi_0^s}{R_s(\brho(\emptyset),\brho(\emptyset))}}
    \]
    for all $s\in \sS$.
    Thus $R_s(\btau(\emptyset),\btau(\emptyset)) = \phi_0^s$ and 
    \[
        \sqrt{R_s(\btau(\emptyset)-\brho(\emptyset),\btau(\emptyset)-\brho(\emptyset))}
        =
        \lt|\sqrt{R_s(\brho(\emptyset),\brho(\emptyset))} - \sqrt{\phi_0^s}\rt|
        \le 
        \fr{\eta}{\sqrt{\phi_0^s}}
        \le 
        \eps_0,
    \]
    where the second-last inequality holds for all sufficiently small $\eta > 0$.
    This proves \eqref{eq:gram-schmidt-invariant} for $u=\emptyset$.
    We construct $\btau(u)$ for the remaining $u\in \bbT$ recursively.
    Suppose we have constructed $\btau(u)$ satisfying \eqref{eq:gram-schmidt-invariant}. 
    Then, for each $s\in \sS$, $i,j\in [k]$, 
    \begin{align*}
        R_s(\brho(ui)-\btau(u), \brho(uj)-\btau(u))
        &=
        R_s(\brho(ui)-\brho(u), \brho(uj)-\brho(u)) \\
        &\quad + R_s(\brho(ui)-\brho(u), \brho(u)-\btau(u)) \\
        &\quad + R_s(\brho(u)-\btau(u), \brho(uj)-\brho(u)) \\
        &\quad + R_s(\brho(u)-\btau(u), \brho(u)-\btau(u)),
    \end{align*}
    so 
    \[
        |R_s(\brho(ui)-\btau(u), \brho(uj)-\btau(u)) - (\phi_{|u|+1}^s - \phi_{|u|}^s) \ind\{i=j\}| \le 6\eps_{|u|} + 4\eta.
    \]
    Similarly,
    \begin{align*}
        |R_s(\brho(ui)-\btau(u), \btau(u))|
        &= 
        |R_s(\brho(ui)-\brho(u), \brho(u))| \\
        &\quad + |R_s(\brho(u)-\btau(u), \brho(u))| \\
        &\quad + |R_s(\brho(ui)-\brho(u), \btau(u)-\brho(u))| \\
        &\quad + |R_s(\brho(u)-\btau(u), \btau(u)-\brho(u))| \\
        &\le 6\eps_{|u|} + 4\eta.
    \end{align*}
    We apply Lemma~\ref{lem:gram-schmidt} on the vectors
    \[
        \fr{\btau(u)_s}{\sqrt{\lambda_s N}},
        \fr{\brho(u1)_s-\btau(u)_s}{\sqrt{\lambda_s N}},
        \ldots,
        \fr{\brho(uk)_s-\btau(u)_s}{\sqrt{\lambda_s N}}
    \]
    with $q = \phi_{|u|}^s \ge D^{-2}$, $q' = \phi_{|u|+1}^s - \phi_{|u|}^s$, and $\eps = 6\eps_{|u|} + 4\eta$. 
    This gives us $\btau(u1)_s,\ldots,\btau(uk)_s$ satisfying \eqref{eq:gram-schmidt-invariant}, such that 
    \[
        R_s(\btau(ui)-\btau(u), \btau(ui)-\btau(u))
        = \phi_{|u|+1}^s - \phi_{|u|}^s
    \]
    and the vectors $\btau(u)_s$, $\btau(u1)_s-\btau(u)_s$, $\btau(uk)_s-\btau(u)_s$ are pairwise orthogonal. 
    From this we can see that
    \begin{align*}
        \vR(\btau(ui),\btau(u)) &= \vphi_{|u|}, \\
        \vR(\btau(ui),\btau(ui)) &= \vphi_{|u|+1}, \\
        \vR(\btau(ui),\btau(uj)) &= \vphi_{|u|} \quad \text{if}~i\neq j. 
    \end{align*}
    Thus the $\ubtau$ constructed this way is an element of $\cQ_{\loc+}(0)$.
    Finally, let $\ubsig' = (\btau(u))_{u\in \bbL}$, so $\ubsig' \in \cQ_{\loc}(0)$.
    Equation \eqref{eq:gram-schmidt-invariant} implies that for all $u\in \bbL$, 
    \[
        \fr{1}{\sqrt{N}} \norm{\bsig'(u)-\bsig(u)}_2 \le \eps_D.
    \]
    By Proposition~\ref{prop:gradients-bounded}, with probability $1-e^{-\Omega(N)}$ we have $\HNp{u}\in K_N$ for all $u\in \bbL$. 
    On this event, 
    \[
        \lt|\fr{1}{N} \cH_N(\ubsig') - \fr{1}{N} \cH_N(\ubsig)\rt| \le C\eps_D
    \]
    for some $C >0$, and so 
    \[
        \fr{1}{N} \sup_{\ubsig' \in \cQ_{\loc}^{k,D,\uvphi'}(\eta)} \cH_N(\ubsig')
        + C\eps_D
        \ge 
        \fr{1}{N} \sup_{\ubsig \in \cQ_{\loc}^{k,D,\uvphi}(0)} \cH_N(\ubsig).
    \]
    By Lemma~\ref{lem:bogp-subgaussian}, both sides of this inequality are subgaussian with fluctuations $O(N^{-1/2})$, so the contribution from the complement of this event is $o_N(1)$, and 
    \[
        \limsup_{N\to\infty} 
        \fr{1}{N} \bbE \sup_{\ubsig' \in \cQ_{\loc}^{k,D,\uvphi'}(0)} \cH_N(\ubsig')
        + C\eps_D
        \ge 
        \limsup_{N\to\infty}
        \fr{1}{N} \bbE \sup_{\ubsig \in \cQ_{\loc}^{k,D,\uvphi}(\eta)} \cH_N(\ubsig).
    \]
    Taking $\eta \to 0$ (which forces $\eps_D \to 0$) followed by $D,k\to\infty$ implies $\BOGP_{\loc} \le \BOGP_{\loc,0}^-$, as desired.
\end{proof}

\section{Ground State Energy of Multi-Species Spherical SK With External Field}
\label{sec:sk-ext-field}

We adopt the notations of Lemma~\ref{lem:sk-ext-field}.
In this section, we will prove this lemma by showing that 
\[
    \limsup_{N\to\infty} \bbE \GS_N(W,\vv,k)
    \le 
    \sum_{s\in \sS}
    \lambda_s
    \sqrt{v_s^2 + 2\sum_{s'\in \sS} \lambda_{s'} w_{s,s'}^2}
    \le 
    \liminf_{N\to\infty} \bbE \GS_N(W,\vv,k).
\]

\subsection{Upper Bound for $\vv=\vzero$, $k=1$}

The following (exact) upper bound for the case $\vv = \vzero$, $k=1$ follows from the results of \cite{bandeira2021matrix}.
We will prove Lemma~\ref{lem:sk-ext-field} using only this result and elementary techniques.

\begin{proposition}
    \label{prop:free-prob}
    For $W$ as in Lemma~\ref{lem:sk-ext-field},
    \[
        \limsup_{N\to\infty} \bbE \GS_N(W,\vzero,1)
        \le
        \sum_{s\in \sS} \lambda_s \sqrt{2\sum_{s'\in \sS} \lambda_{s'} w_{s,s'}^2}.
    \]
\end{proposition}

\begin{proof}
    In this proof, abbreviate $\GS_N = \GS_N(W,\vv,k)$.
    Let $\bG \in \bR^{N\times N}$ have i.i.d. standard Gaussian entries.
    Thus $G = \fr12 \lt(\bG + \bG^\top\rt)$ is symmetric with $\cN(0,1)$ diagonal entries, $\cN(0, 1/2)$ off-diagonal entries, and independent entries on and above the diagonal.
    Define $M\in \bbR^{N\times N}$ by $M_{i,j} = N^{-1/2} w_{s(i),s(j)} G_{i,j}$.
    It is clear by homogeneity that 
    \[
        \GS_N =
        \fr{1}{N} 
        \max_{\bsig \in \cB_N}
        \bsig^\top M \bsig.
    \]
    Let $\vC \in \bbR_{>0}^\sS$ be a vector of constants we will set later.
    We consider the rescaled matrix $\wtM = \sqrt{\vC}^{\otimes 2} \diamond M$.
    This can be generated by $\wtM = \hM + \oM$, where $\hM$ is a random symmetric matrix with independent entries on and above the diagonal 
    \[
        \hM_{i,j} \sim \cN\lt(0, \fr{C_{s(i)}C_{s(j)}w_{s(i),s(j)}^2}{2N}\rt)
    \]
    and $\oM$ is a random diagonal matrix with independent entries
    \[
        \oM_{i,i} \sim \cN\lt(0, \fr{C_{s(i)}^2 w_{s(i),s(i)}^2}{2N}\rt).
    \]
    Clearly $\bbE \tnorm{\oM}_{\op} = O(\sqrt{N^{-1} \log N})$. 
    \cite[Theorem 1.2]{bandeira2021matrix} states that
    \[
        \bbE\tnorm{\hM}_{\op}
        \leq
        \tnorm{X_{\free}}_{\op}+O\lt(v^{1/2}\sigma^{1/2}(\log N)^{3/4}\rt),
    \]
    where $\tnorm{X_{\free}}_{\op}, \sigma, v$ are defined as follows.
    We have
    \[
        \sigma
        =
        \sqrt{\bbE\tnorm{\hM^2}_{\op}}=O(1),
        \qquad
        v
        =
        \sqrt{\tnorm{\Cov(\hM)}_{\op}},
    \]
    where $\Cov(\hM)\in \bbR^{N^2\times N^2}$ is the covariance matrix of the entries of $\hM$ and has operator norm $O(1/N)$. 
    It follows that the error term $v^{1/2}\sigma^{1/2}(\log N)^{3/4}$ contributes $o_N(1)$. Finally \cite[Lemma 3.2]{bandeira2021matrix} states that in our setting,
    \[
        \norm{X_{\free}}_{\op}=
        2\sup_{\substack{a\in [0,1]^N\\ \sum_i a_i=1}}
        \sum_{i\in [N]}
        \sqrt{a_{i}\sum_{i'\in [N]} \frac{C_{s(i)}C_{s(i')}w_{s(i),s(i')}^2 a_{i'}}{2N}}
    \]
    It is not difficult to see by concavity of the square-root that, for $\lambda_{s,N} = |\cI_s|/N$ (so $\lambda_{s,N} \to \lambda_s$) replacing all $a_i$ such that $i\in \cI_s$ with
    \[
        A_s=\lambda_{s,N}^{-1}\sum_{i:s(i)=s} a_i
    \]
    only improves the right-hand side. Substituting $B_s=C_sA_s$, we conclude that
    \begin{align*}
        \norm{X_{\free}}_{\op}
        &=
        \sup_{\substack{\vA\in \bbR_{\geq 0}^{\sS}\\ \sum_s \lambda_{s,N} A_s=1}}
        \sum_{s\in\sS}
        \lambda_{s,N}
        \sqrt{2A_s\sum_{s'\in\sS}\lambda_{s'} C_s C_{s'}w_{s,s'}^2 A_{s'}} \\
        &=
        \sup_{\substack{\vB\in \bbR_{\geq 0}^{\sS}\\ \sum_s C_s^{-1}\lambda_{s,N} B_s=1}}
        \sum_{s\in\sS}
        \lambda_{s,N}
        \sqrt{2B_s\sum_{s'\in\sS}\lambda_{s'}  w_{s,s'}^2 B_{s'}}.
    \end{align*}
    From the above discussion, $\tnorm{\wtM}_{\op} \le \tnorm{X_{\free}}_{\op} + o_N(1)$.
    Moreover we observe that
    \begin{align*}
        \GS_N
        &=
        \frac{1}{N}
        \max_{\norm{\bsig_s}_2^2 \le \lambda_s N}
        \bsig^{\top}M\bsig
        =
        \frac{1}{N}
        \max_{\norm{\bsig_s}_2^2 \le C_s^{-1}\lambda_s N}
        \bsig^{\top}\wtM\bsig \\
        &\leq
        \frac{1}{N}
        \max_{\norm{\bsig}_2^2 \le \sum_{s\in\sS} C_s^{-1}\lambda_s N}
        \bsig^{\top}\wtM\bsig
        =
        \lt(\sum_{s\in\sS} C_s^{-1}\lambda_s\rt)
        \tnorm{\wtM}_{\op} 
        \,.
    \end{align*}
    Combining and using homogeneity, we find
    \begin{align}
        \notag
        \bbE \GS_N
        &\leq 
        \lt(\sum_{s\in\sS} C_s^{-1}\lambda_s\rt)
        \bbE \tnorm{\wtM}_{\op} \\
        \notag
        &=
        \lt(\sum_{s\in\sS} C_s^{-1}\lambda_s\rt)
        \sup_{\substack{
            \vB \in \bbR_{\geq 0}^{\sS} \\ 
            \sum_s C_s^{-1}\lambda_{s,N} B_s=1
        }}
        \sum_{s\in\sS}
        \lambda_{s,N}
        \sqrt{2B_s \sum_{s'\in\sS} \lambda_{s',N} w_{s,s'}^2 B_{s'}}
        +o_N(1) \\
        \label{eq:rvh-supremum}
        &=
        \fr{\sum_{s\in\sS} C_s^{-1}\lambda_s}
        {\sum_{s\in\sS} C_s^{-1}\lambda_{s,N}}
        \cdot 
        \sup_{\substack{
            \vD \in \bbR_{\geq 0}^{\sS} \\ 
            \sum_s C_s^{-1} \lambda_{s,N} D_s = \sum_{s\in\sS} C_s^{-1} \lambda_{s,N}
        }}
        \sum_{s\in\sS}
        \lambda_{s,N}
        \sqrt{2 D_s \sum_{s'\in\sS} \lambda_{s',N} w_{s,s'}^2 D_{s'}}
        +o_N(1).
    \end{align}
    If the supremum in \eqref{eq:rvh-supremum} is attained at $\vD = \vone$, then (because $\lambda_{s,N} \to \lambda_s$) we get the desired bound
    \[
        \bbE \GS_N 
        \le 
        \sum_{s\in\sS} \lambda_s \sqrt{2\sum_{s'\in\sS} \lambda_{s'} w_{s,s'}^2} 
        + o_N(1).
    \]
    Crucially, we observe that the expression 
    \begin{equation}
    \label{eq:concave-dude}
        F(\vD)
        =
        \sum_{s\in\sS}
        \lambda_{s,N}
        \sqrt{2 D_s \sum_{s'\in\sS} \lambda_{s',N} w_{s,s'}^2 D_{s'}}
    \end{equation}
    is concave in $\vD$. 
    Therefore if $\vD=\vone$ is a critical point of $F$ within the set satisfying $\sum_s C_s^{-1} \lambda_{s,N} D_s = \sum_{s\in\sS} C_s^{-1} \lambda_{s,N}$, then it also attains the supremum in \eqref{eq:rvh-supremum}.
    For the choice $C_s=\frac{\lambda_{s,N}}{\partial_{D_s} F}$, $\vD=\vone$ is a critical point of $F$. 
    This concludes the proof.
\end{proof}

\subsection{General Upper Bound}

In this subsection, we will prove the following upper bound for the case $k=1$.
\begin{proposition}
    \label{prop:sk-ub-one-rep}
    For $W, \vv$ as in Lemma~\ref{lem:sk-ext-field},
    \[
        \limsup_{N\to\infty} \bbE \GS_N(W,\vv,1)
        \le
        \sum_{s\in \sS} \lambda_s \sqrt{v_s^2 + 2\sum_{s'\in \sS} \lambda_{s'} w_{s,s'}^2}.
    \]
\end{proposition}
By slight abuse of notation, let $H_N = H_{N,1}^1$ and $\GS_N(W,\vv) = \GS_N(W,\vv,1)$.
Recall that 
\[
    H_N(\bsig)
    =
    \la \vv \diamond \bg, \bsig \ra
    + \wtH_N(\bsig),
    \qquad
    \wtH_N(\bsig)
    =
    \fr{1}{\sqrt{N}}
    \la W \diamond \bG, \bsig^{\otimes 2} \ra 
\]
where $\bg \in \bbR^N$, $\bG\in \bbR^{N\times N}$ have i.i.d. standard Gaussian entries. 
Define
\[
    A(W,\vv) = \limsup_{N\to\infty} 
    \bbE \GS_N(W, \vv).
\]
We first establish some basic properties of this limit.
\begin{lemma}
    \label{lem:A-basic-properties}
    $A$ satisfies the following properties.
    \begin{enumerate}[label=(\alph*), ref=\alph*]
        \item \label{itm:A-homogenity} For any $c>0$, $A(cW, c\vv) = cA(W,\vv)$.
        \item \label{itm:A-linear} $A(0,\vv) = \sum_{s\in \sS} \lambda_s v_s$.
        \item \label{itm:A-quadratic} $A(W,\vzero) \le \sum_{s\in \sS} \lambda_s \sqrt{2\sum_{s'\in \sS} \lambda_{s'} w_{s,s'}^2}$.
        \item \label{itm:A-subadditive} $A(W,\vv) \le A(W,\vzero) + A(0,\vv)$.
    \end{enumerate}
\end{lemma}
\begin{proof}
    Part (\ref{itm:A-homogenity}) is obvious. Part (\ref{itm:A-linear}) follows from
    \[
        \bbE \GS_N(0,\vv) 
        =
        \fr{1}{N} 
        \bbE \max_{\bsig \in \cS_N}
        \la \vv \diamond \bg, \bsig \ra
        =
        \fr{1}{N} 
        \sum_{s\in \sS}
        \sqrt{\lambda_s N} v_s
        \bbE \norm{\bg_s}_2
        =
        \sum_{s\in \sS}
        \lambda_s v_s 
        + o_N(1).
    \]
    Part (\ref{itm:A-quadratic}) follows from Proposition~\ref{prop:free-prob}.
    Part (\ref{itm:A-subadditive}) follows from 
    \begin{align}
        \notag
        \GS_N(W,\vv)
        &=
        \fr{1}{N} 
        \max_{\bsig \in \cS_N}
        \lt(
            \la \vv \diamond \bg, \bsig \ra + 
            \fr{1}{\sqrt{N}} 
            \la W \diamond \bG, \bsig^{\otimes 2} \ra
        \rt) \\
        \notag
        &\ge
        \fr{1}{N} 
        \max_{\bsig \in \cS_N}
        \la \vv \diamond \bg, \bsig \ra 
        +
        \fr{1}{N} 
        \max_{\bsig \in \cS_N}
        \fr{1}{\sqrt{N}} 
        \la W \diamond \bG, \bsig^{\otimes 2} \ra \\
        \label{eq:A-subadditive}
        &=
        \GS_N(W,\vzero) + \GS_N(0,\vv).
    \end{align}
\end{proof}

Next we show some a priori regularity conditions on $A$.

\begin{proposition}
    \label{prop:sk-concentration}
    Let
    \[
        C(W,\vv)
        =
        4 \lt(
            \sum_{s\in \sS}
            \lambda_{s}
            v_s^2 
            +
            \sum_{s,s'\in \sS}
            \lambda_{s}
            \lambda_{s'}
            w_{s,s'}^2
        \rt).
    \]
    Then, for sufficiently large $N$ and all $t>0$,
    \[
        \bbP\lt[
            \lt|\GS_N(W,\vv) - \bbE \GS_N(W,\vv)\rt| > t
        \rt]
        \le 
        2\exp\lt(-\fr{Nt^2}{C(W,\vv)}\rt).
    \]
\end{proposition}
\begin{proof}
    Let $C = C(W,\vv)$.
    For any $\bsig \in \cS_N$, 
    \begin{align*}
        \bbE H_N(\bsig)^2 
        &= 
        \norm{\vv \diamond \bsig}_2^2 + \fr{1}{N} \norm{W \diamond \bsig^{\otimes 2}}_F^2 \\
        &=
        N \lt(
            \sum_{s\in \sS}
            \lambda_{s,N}
            v_s^2 
            +
            \sum_{s,s'\in \sS}
            \lambda_{s,N}
            \lambda_{s',N}
            w_{s,s'}^2
        \rt) 
        \le 
        \fr{CN}{2}.
    \end{align*}
    for large enough $N$.
    By the Borell-TIS inequality, $\max_{\bsig \in \cS_N} H_N(\bsig)$ is $CN/2$-subgaussian, so $\GS_N(W,\vv)$ is $C/2N$-subgaussian, which implies the result.
\end{proof}

For $\va = (a_s)_{s\in \sS'} \in [0,1]^\sS$, define  $W(W,\vv,\va) = (w'_{s,s'})_{s,s'\in \sS}$ and $\vv(W,\vv,\va) = (v'_s)_{s\in \sS}$ where
\[
    w'_{s,s'} 
    = 
    \sqrt{(1-a_s)(1-a_{s'})} 
    w_{s,s'},
    \qquad
    v'_s 
    = 
    \sqrt{
        2(1-a_s) \lt(
            \sum_{s'\in \sS} 
            \lambda_{s'} a_{s'} 
            w_{s,s'}^2
        \rt)
    }.
\]
We will prove Proposition~\ref{prop:sk-ub-one-rep} using the following recursive upper bound in $A$.
\begin{lemma}
    \label{lem:A-fn-ineq}
    For $W, \vv$ as in Lemma~\ref{lem:sk-ext-field},
    \begin{equation}
        \label{eq:A-fn-ineq}
        A(W,\vv) 
        \le 
        \max_{\va \in [0,1]^{\sS}}
        \sum_{s\in \sS}
        \lambda_s v_s \sqrt{a_s}
        +
        A\lt(W(W,\vv,\va), \vv(W,\vv,\va)\rt).
    \end{equation}
\end{lemma}
\begin{proof}
    Define $\hbg \in \cS_N$ by $\hbg_s = \fr{\sqrt{\lambda_s N} \bg_s}{\norm{\bg_s}_2}$ for each $s\in \sS$.
    For $\va\in [0,1]^{\sS}$, define
    \[
        \GS_N(W,\vv;\va)
        = 
        \fr{1}{N} 
        \max_{\bsig \in \cR_N(\va)}
        H_N(\bsig),
        \qquad
        \cR_N(\va) = \lt\{
            \bsig \in \cS_N: 
            R(\bsig, \hbg) = \sqrt{\va}
        \rt\}.
    \]
    For a non-random $\va$ and any $\bsig \in \cR_N(\va)$,
    \[
        \la \vv \diamond \bg, \bsig \ra
        =
        N \sum_{s\in \sS}
        \lambda_s v_s \sqrt{a_s}
        \fr{\norm{\bg_s}_2}{\sqrt{\lambda_s N}}.
    \]
    For $\bsig\in \cR_N(\va)$, we may write $\bsig = \sqrt{\va} \diamond \hbg + \sqrt{\vone-\va} \diamond \brho$ for $\brho \in \cR_N(\vzero)$.
    Define the Gaussian process $\hH_N^{\va}(\brho) = \wtH_N\big(\sqrt{\va} \diamond \hbg + \sqrt{\vone - \va} \diamond \brho\big)$, which is supported on $\cR_N(\vzero)$.
    We next calculate the covariance of this process.
    Recall that the covariance of $\wtH_N$ is 
    \[
        \bbE \wtH_N(\bsig)\wtH_N(\bsig') = N\xi(R(\bsig,\bsig')), 
        \qquad
        \xi(\vx) = \lt\la W \odot W, (\vlam \odot \vx)^{\otimes 2} \rt\ra.
    \]
    Because $\bg, \bG$ are independent, the covariance of $\hH_N^{\va}$ is
    \begin{equation}
        \label{eq:sk-hH-covariance}
        \bbE \hH_N^{\va}(\brho)\hH_N^{\va}(\brho')
        = 
        N\xi_{\va}(R(\brho,\brho')), 
    \end{equation}
    where, for $W' = W(W, \vv, \va)$ and $\vv' = \vv(W,\vv,\va)$,
    \begin{align}
        \notag
        \xi_{\va}(\vx) 
        &= 
        \xi\lt(\va + (\vone - \va) \odot \vx\rt)
        = 
        \lt\la
            W \odot W,
            (\vlam \odot \va + \vlam \odot (1-\va) \odot \vx)^{\otimes 2}
        \rt\ra \\
        \label{eq:sk-hH-covariance-calculation}
        &= 
        \lt\la 
            W' \odot W',
            (\vlam \odot \vx)^{\otimes 2}
        \rt\ra
        +
        \lt\la
            \vv' \odot \vv', \vlam \odot \vx
        \rt\ra
        +
        \lt\la
            W \odot W,
            (\vlam \odot \va)^{\otimes 2}
        \rt\ra.
    \end{align}
    We may construct a Gaussian process $\oH_N^{\va}$ (conditional on $\bg$) on $\cS_N$ with covariance \eqref{eq:sk-hH-covariance} whose restriction to $\cR_N(\vzero)$ agrees with $\hH_N^{\va}$.
    Thus
    \begin{align*}
        \GS_N(W,\vv;\va)
        &=
        \sum_{s\in \sS}
        \lambda_s v_s \sqrt{a_s} \fr{\norm{\bg_s}_2}{\sqrt{\lambda_s N}}
        +
        \fr{1}{N} 
        \max_{\brho \in \cR_N(\vzero)}
        \hH_N^{\va}(\brho) \\
        & \le 
        \sum_{s\in \sS}
        \lambda_s v_s \sqrt{a_s} \fr{\norm{\bg_s}_2}{\sqrt{\lambda_s N}}
        +
        \fr{1}{N} 
        \max_{\brho \in \cS_N}
        \oH_N^{\va}(\brho).
    \end{align*}
    Moreover, 
    \[
        \fr{1}{N} \max_{\brho \in \cS_N} \oH_N^{\va}(\brho)
        =_d 
        \GS(W(W,\vv,\va), \vv(W,\vv,\va)) + 
        \fr{1}{\sqrt{N}} 
        \lt\la
            W \odot W,
            (\vlam \odot \va)^{\otimes 2}
        \rt\ra^{1/2}
        Z
    \]
    for an independent $Z \sim \cN(0, 1)$.
    Let $\cD = \{0, \fr{1}{N}, \ldots, \fr{N-1}{N}, 1\}^\sS$.
    Let $\cE$ be the event that
    \begin{enumerate}[label=(\alph*), ref=\alph*]
        \item \label{itm:condition-lipschitz} For a constant $L$, $H_N(\bsig)$ is $L\sqrt{N}$-Lipschitz on $\bsig \in \cS_N$. 
        By Proposition~\ref{prop:gradients-bounded}, this occurs with probability $1-\exp(-CN)$.
        \item For all $s\in \sS$, $|\norm{\bg_s}_2 - \sqrt{\lambda_{s,N}N}| \le N^{1/4}$; by standard concentration inequalities this holds with probability $1-r\exp(-CN^{1/2})$.
        \item For all $\va \in \cD$, $|\fr{1}{N} \max_{\brho \in \cS_N} \oH_N^{\va}(\brho) - \bbE \GS_N(W(W,\vv,\va), \vv(W,\vv,\va))| \le N^{-1/4}$; by Proposition~\ref{prop:sk-concentration} and standard tail bounds on $Z$ this holds with probability $1-2(N+1)^r \exp(-CN^{1/2})$.
        Here we use that for $\va \in \cD$, the constants $C(W(W,\vv,\va), \vv(W,\vv,\va))$ in Proposition~\ref{prop:sk-concentration} are uniformly upper bounded.
    \end{enumerate}
    By adjusting $C$, $\bbP(\cE) \ge 1-\exp(-CN^{1/2})$.
    On $\cE$, if $\bsig\in \cS_N$ maximizes $H_N$, we can find $\bsig' \in \bigcup_{\va \in \cD} \cR_N(\va)$ with $\norm{\bsig'-\bsig}_2 \le O(1/\sqrt{N})$.
    By the Lipschitz condition (\ref{itm:condition-lipschitz}), $|H(\bsig)-H(\bsig')| \le O(1)$.
    So, 
    \begin{align*}
        \GS_N(W,\vv)
        &= 
        \fr{1}{N}
        H_N(\bsig)
        \le 
        \fr{1}{N} H_N(\bsig') + O(1/N) \\
        &\le 
        \max_{\va \in \cD}
        \GS_N(W,\vv;\va) + O(1/N) \\
        &\le 
        \max_{\va \in \cD} \lt(
            \sum_{s\in \sS} \lambda_s v_s \sqrt{a_s}
            +
            \bbE \GS_N(W(W,\vv,\va), \vv(W,\vv,\va))
        \rt)
        + o_N(1).
    \end{align*}
    The subgaussianity from Proposition~\ref{prop:sk-concentration} implies that the contribtion to $\bbE \GS_N(W,\vv)$ from $\cE^c$ is $o_N(1)$, so 
    \begin{align*}
        \bbE \GS_N(W,\vv)
        &\le 
        \max_{\va \in \cD} \lt(
            \sum_{s\in \sS} \lambda_s v_s \sqrt{a_s}
            +
            \bbE \GS_N(W(W,\vv,\va), \vv(W,\vv,\va))
        \rt)
        + o_N(1) \\
        &\le 
        \max_{\va \in [0,1]^\sS} \lt(
            \sum_{s\in \sS} \lambda_s v_s \sqrt{a_s}
            +
            \bbE \GS_N(W(W,\vv,\va), \vv(W,\vv,\va))
        \rt)
        + o_N(1).
    \end{align*}
    Taking $\limsup_{N\to\infty}$ on both sides yields the result.
\end{proof}

\begin{proof}[Proof of Proposition~\ref{prop:sk-ub-one-rep}]
    We will show that any $A$ satisfying the properties in Lemma~\ref{lem:A-basic-properties} and the bound \eqref{eq:A-fn-ineq} must satisfy
    \[
        A(W,\vv) 
        \le 
        A_*(W,\vv)
        \equiv
        \sum_{s\in \sS} 
        \lambda_s
        \sqrt{v_s^2 + 2\sum_{s'\in \sS} \lambda_{s'} w_{s,s'}^2}.
    \]
    Clearly $A_*$ satisfies the conclusions of Lemma~\ref{lem:A-basic-properties}, with equality in assertion (\ref{itm:A-quadratic}).
    For any $\va \in [0,1]^{\sS}$,
    \begin{align*}
        A_*\lt(W(W,\vv,\va),\vv(W,\vv,\va)\rt)
        &= 
        \sum_{s\in \sS}
        \lambda_s
        \sqrt{
            2(1-a_s) \sum_{s'\in \sS} a_{s'} \lambda_{s'} w_{s,s'}^2 + 
            2\sum_{s'\in \sS} \lambda_{s'} (1-a_s)(1-a_{s'}) w_{s,s'}^2
        } \\
        &= 
        \sum_{s\in \sS}
        \lambda_s
        \sqrt{
            2(1-a_s) \sum_{s'\in \sS} \lambda_{s'} w_{s,s'}^2
        },
        \\
        \implies
        \sum_{s\in \sS}
        \lambda_s v_s \sqrt{a_s}
        +
        A_*\lt(W(W,\vv,\va),\vv(W,\vv,\va)\rt)
        &= 
        \sum_{s\in \sS}
        \lambda_s \lt(
            \sqrt{a_s} v_s + 
            \sqrt{1-a_s}
            \sqrt{2\sum_{s'\in \sS} \lambda_{s'} w_{s,s'}^2} 
        \rt) \\
        &\le 
        \sum_{s\in \sS}
        \lambda_s \sqrt{
            v_s^2 +
            2\sum_{s'\in \sS} \lambda_{s'} w_{s,s'}^2
        } 
        = 
        A_*(W,\vv)
    \end{align*}
    by Cauchy-Schwarz. Equality holds when
    \begin{equation}
        \label{eq:sk-a-opt}
        a_s = \fr{v_s^2}{v_s^2 + 2\sum_{s'\in \sS} \lambda_{s'} w_{s,s'}^2}
    \end{equation}
    for all $s\in \sS$, and so $A_*$ satisfies \eqref{eq:A-fn-ineq} with equality.
    
    Suppose $A$ satisfies the conclusions of Lemma~\ref{lem:A-basic-properties} and the inequality \eqref{eq:A-fn-ineq}, and there exists $(W,\vv)$ with $A(W,\vv) > A_*(W,\vv)$.
    By homogeneity (Lemma~\ref{lem:A-basic-properties}(\ref{itm:A-homogenity})), we can assume $1 = \norm{W}_1 \equiv \sum_{s,s'\in \sS} w_{s,s'}$.
    For any small $\delta > 0$, we may choose $(W^*, \vv^*)$ such that $\norm{W^*}_1 = 1$ and 
    \[
        A(W^*,\vv^*) - A_*(W^*,\vv^*)
        \ge 
        (1-\delta)
        \sup_{(W,\vv) : \norm{W}_1=1}
        \lt(
            A(W,\vv) - A_*(W,\vv)
        \rt) 
        > 0.
    \]
    Set 
    \[
        \va^* 
        = 
        \argmax_{\va \in [0,1]^\sS}
        \sum_{s\in \sS}
        \lambda_s v^*_s \sqrt{a_s} + A\lt(W(W^*,\vv^*,\va), \vv(W^*,\vv^*,\va)\rt),
    \]
    and $W' = W(W^*,\vv^*,\va^*), \vv' = \vv(W^*,\vv^*,\va^*)$, so
    \begin{align*}
        A(W^*,\vv^*) 
        &\le 
        \sum_{s\in \sS} 
        \lambda_s v^*_s \sqrt{a_s^*} +
        A(W',\vv'), \\
        A_*(W^*,\vv^*)
        &\ge 
        \sum_{s\in \sS} 
        \lambda_s v^*_s \sqrt{a_s^*} +
        A_*(W',\vv').
    \end{align*}
    Here, the second inequality uses that $A_*$ satisfies \eqref{eq:A-fn-ineq} with equality.
    Therefore
    \begin{equation}
        \label{eq:A-max-diff}
        A(W',\vv') - A_*(W',\vv')
        \ge 
        A(W^*,\vv^*) - A_*(W^*,\vv^*)
        \ge 
        (1-\delta)
        \sup_{(W,\vv) : \norm{W}_1=1}
        \lt(
            A(W,\vv) - A_*(W,\vv)
        \rt).
    \end{equation}
    By homogeneity, this implies $\norm{W'}_1 \ge 1-\delta$.
    Let $\sS_0 \subseteq \sS$ be the set of $s$ for which there exists $s'$ with $w_{s,s'} \ge \delta_1 \equiv \sqrt{2\delta}$.
    For such $s,s'$, 
    \[
        \delta 
        \ge 
        \norm{W^*}_1 - \norm{W'}_1
        \ge 
        w^*_{s,s'} - w'_{s,s'}
        \ge 
        \lt(1 - \sqrt{1-a_s}\rt) w_{s,s'}
        \ge 
        \fr12 a_s w_{s,s'}
        \ge 
        \fr12 a_s \delta_1.
    \]
    Thus, for $s\in \sS_0$, $a_s \le \delta_1$.
    Of course, for $s\in \sS \setminus \sS_0$, $w_{s,s'} \le \delta_1$ for all $s'\in \sS$.
    Thus for all $s\in \sS$, 
    \[
        v'_s \le \sqrt{2\sum_{s'\in \sS} \lambda_{s'} \delta_1} = \sqrt{2\delta_1} \equiv \delta_2.
    \]
    By parts (\ref{itm:A-subadditive}), (\ref{itm:A-linear}), and (\ref{itm:A-quadratic}) of  Lemma~\ref{lem:A-basic-properties},
    \[
        A(W',\vv') 
        \le 
        A(W',\vzero) + A(0,\vv')
        \le 
        A(W',\vzero) + \delta_2
        \le 
        A_*(W', \vzero) + \delta_2.
    \]
    By inspection, $A_*(W',\vv') \ge A_*(W',\vzero)$.
    Thus
    \[
        A(W',\vv') - A_*(W',\vv')
        \le 
        \delta_2.
    \]
    For small enough $\delta > 0$, this contradicts \eqref{eq:A-max-diff}.
\end{proof}

Finally, the upper bound for $k=1$ directly implies the upper bound for general $k$.
\begin{corollary}
    \label{cor:sk-ub}
    For $W, \vv$ as in Lemma~\ref{lem:sk-ext-field},
    \[
        \limsup_{N\to\infty} \bbE \GS_N(W,\vv,k)
        \le
        \sum_{s\in \sS} \lambda_s \sqrt{v_s^2 + 2\sum_{s'\in \sS} \lambda_{s'} w_{s,s'}^2}.
    \]
\end{corollary}
\begin{proof}
    Note that (recall \eqref{eq:bbtperp})
    \[
        \GS_N(W,\vv,k)
        = 
        \fr{1}{kN} 
        \max_{\vbsig \in \cS_N^{k,\perp}} 
        H_{N,k}
        \le 
        \fr{1}{k} 
        \sum_{i=1}^k
        \fr{1}{N}
        \max_{\bsig^i \in \cS_N} 
        H^i_{N,k}(\bsig^i).
    \]
    Taking expectations yields $\bbE \GS_N(W,\vv,k) \le \bbE \GS_N(W,\vv,1)$.
    This and Proposition~\ref{prop:sk-ub-one-rep} imply the result.
\end{proof}

\begin{remark}
    The proof of Proposition~\ref{prop:sk-ub-one-rep} via the recursive inequality \eqref{eq:A-fn-ineq} extends to the ground state energies in multi-species spherical spin glasses with general (non-quadratic) interactions. 
    It thus gives an elementary way to upper bound the ground state energy for spin glasses with external field given the ground state energy of spin glasses without external field, when the latter is known. 
    As we will see in the next subsection, it is possible to construct points where this recursive inequality holds with (approximate) equality, so the upper bound is sharp.
\end{remark}


\subsection{Lower Bound}

In this subsection, we will constructively prove the matching lower bound to Corollary~\ref{cor:sk-ub}.
\begin{proposition}
    \label{prop:sk-lb}
    For $W,\vv$ as in Lemma~\ref{lem:sk-ext-field}, 
    \[
        \liminf_{N\to\infty} \bbE \GS_N(W,\vv,k)
        \ge
        \sum_{s\in \sS} \lambda_s \sqrt{v_s^2 + 2\sum_{s'\in \sS} \lambda_{s'} w_{s,s'}^2}.
    \]
\end{proposition}

\begin{lemma}   
    \label{lem:sk-gram-schmidt}
    Let $S_N = \{\bx\in \bbR^N : \norm{\bx}_2 = \sqrt{N}\}$.
    Suppose $\by^1,\ldots,\by^k \in S_N$ satisfy $|\la \by^i, \by^j\ra| \le N^{2/3}$ for all $i\neq j$. 
    Then there exist pairwise orthogonal $\bz^1,\ldots,\bz^k \in S_N$ such that $\Span(\bz^1,\ldots,\bz^k) = \Span(\by^1,\ldots,\by^k)$ and $\la \by^i,\bz^i\ra \ge N - 4kN^{1/3}$.
\end{lemma}
\begin{proof}
    We define $\bz^1,\ldots,\bz^k$ by applying the Gram-Schmidt algorithm to $\by^1,\ldots,\by^k$: let $\bz^1=\by^1$, and for $2\le i\le k$, let
    \[
        \tby^i = \by^i - \sum_{j=1}^{i-1} \fr{\la \bz^j, \by^i\ra}{N} \bz^j,
        \qquad
        \bz^i = \fr{\sqrt{N}}{\norm{\tby^i}_2} \tby^i.
    \]
    Clearly $\Span(\bz^1,\ldots,\bz^k) = \Span(\by^1,\ldots,\by^k)$.
    We will prove by induction on $i$ that for all $\ell >i$, $|\la \bz^i, \by^\ell\ra| \le 2N^{2/3}$.
    The base case $i=1$ is true by hypothesis.
    For $i>1$, we have
    \[
        \norm{\tby^i}_2^2
        =
        N \lt(1 - \sum_{j=1}^{i-1} \fr{\la \bz^j, \by^i\ra^2}{N}\rt)
        \in 
        \lt[N(1-4kN^{-2/3}),N\rt],
    \]
    using the inductive hypothesis.
    Moreover, for $\ell >i$, 
    \[
        |\la \tby^i, \by^\ell\ra|
        \le 
        |\la \by^i, \by^\ell\ra|
        +
        \sum_{j=1}^{i-1} \fr{|\la \bz^j, \by^i\ra| |\la \bz^j, \by^\ell\ra|}{N}
        \le 
        N^{2/3}\lt(1 + 4kN^{-1/3}\rt).
    \]
    Therefore
    \[
        |\la \bz^i, \by^\ell\ra| 
        \le 
        N^{2/3} 
        \lt(1 + 4kN^{-1/3}\rt) \lt(1-kN^{-2/3}\rt)^{-1/2}
        \le 2N^{2/3},
    \]
    completing the induction.
    Now $\la \tby^i, \by^i\ra = \norm{\tby^i}_2^2$, so 
    \[
        \la \bz^i, \by^i\ra 
        =
        \sqrt{N} \norm{\tby^i}_2
        \ge 
        N \lt(1-4kN^{-2/3}\rt)^{1/2}
        \ge 
        N-4kN^{1/3}.
    \]
\end{proof}
Recall that $\lambda_{s,N} = |\cI_s|/N$.
Let $\delta_N = \max_{s\in \sS} |\fr{\lambda_{s,N}}{\lambda_s} - 1|$.
\begin{lemma}
    \label{lem:sk-multi-gram-schmidt}
    There exists an event $\cE \in \sigma(\bg^1,\ldots,\bg^k)$ with $\bbP(\cE) \ge 1-\exp(-CN^{1/3})$ such that on this event, there exists $\vbx = (\bx^1,\ldots,\bx^k)\in \cS_N^{k,\perp}$ such that the following properties hold.
    \begin{enumerate}[label=(\alph*), ref=\alph*]
        \item \label{itm:x-norms-good} For all $i$, $\norm{R(\bg^i,\bg^i) - \vone}_\infty \le \delta_N + N^{-1/4}$.
        \item \label{itm:x-approx-g} For all $i$, $\norm{R(\bg^i,\bx^i) - \vone}_\infty \le \delta_N + N^{-1/4}$. 
        \item \label{itm:x-unit} For all $i$, $R(\bx^i,\bx^i) = \vone$.
        \item \label{itm:x-span} For all $s\in \sS$, $\Span(\bx^1_s,\ldots,\bx^k_s) = \Span(\bg^1_s,\ldots,\bg^k_s)$.
    \end{enumerate}
\end{lemma}
\begin{proof}
    By standard concentration inequalities, for each $i\in [k]$ and $s\in \sS$, $|\la \bg^i_s, \bg^i_s\ra - \lambda_{s,N} N| \le N^{2/3}$ with probability $1-\exp(-CN^{1/3})$, which implies
    \[
        \lt|\fr{\la \bg^i_s, \bg^i_s\ra}{\lambda_s N}-1\rt|
        \le 
        \lt|\fr{\lambda_{s,N}}{\lambda_s}-1\rt| + \fr{N^{2/3}}{\lambda_s N}
        \le 
        \delta_N + N^{-1/4}.
    \]
    When this holds for all $i\in [k]$, $s\in \sS$, part (\ref{itm:x-norms-good}) follows.
    
    For each $i\in [k]$, define $\hbg^i \in \bbR^N$ by $\hbg^i_s = \fr{\sqrt{\lambda_{s,N} N}}{\norm{\bg^i_s}_2} \bg^i_s$ for all $s\in \sS$.
    Note that each $\hbg^i_s$ is a uniformly random point on the sphere of radius $\sqrt{\lambda_{s,N} N}$ supported on the coordinates $\cI_s$.
    
    Fix $s\in \sS$. 
    By standard concentration inequalities, for each pair of distinct $i, j\in [k]$, $|\la \hbg^i_s, \hbg^j_s\ra| \le (\lambda_{s,N}N)^{2/3}$ with probability $1-\exp(-CN^{1/3})$. 
    If this holds for all $s,i,j$, Lemma~\ref{lem:sk-gram-schmidt} implies the existence of orthogonal $\bz^1_s,\ldots,\bz^k_s$ on the sphere of radius $\sqrt{\lambda_{s,N}N}$ supported on coordinates $\cI_s$ with 
    \begin{equation}
        \label{eq:sk-span}
        \Span(\bz^1_s,\ldots,\bz^k_s) = \Span(\hbg^1_s,\ldots,\hbg^k_s)
    \end{equation}
    and
    \[
        \lambda_{s,N}N - 4k(\lambda_{s,N}N)^{1/3} \le \la \bz^k_s, \hbg^i_s\ra \le \lambda_{s,N}N.
    \]
    Let $\bx^i_s = \bz^i_s \cdot \sqrt{\lambda_s / \lambda_{s,N}}$, so
    \[
        \fr{\la \bx^i_s, \bg^i_s\ra}{\lambda_s N}
        = 
        \fr{\la \bz^i_s, \hbg^i_s\ra}{\lambda_{s,N} N} \cdot \sqrt{\fr{\lambda_{s,N}}{\lambda_s }} \cdot \fr{\norm{\bg^i_s}}{\sqrt{\lambda_{s,N}N}}
        =
        (1 + O(N^{-1/3}) \sqrt{\fr{\lambda_{s,N}}{\lambda_s }}.
    \]
    Thus
    \[
        \lt|\fr{\la \bx^i_s, \bg^i_s\ra}{\lambda_s N}-1\rt|
        \le 
        \lt|\sqrt{\fr{\lambda_{s,N}}{\lambda_s }} - 1\rt| + O(N^{-1/3}) 
        \le 
        \delta_N + N^{-1/4}.
    \]
    If this holds for all $s$, part (\ref{itm:x-approx-g}) follows.
    By a union bound, adjusting $C$ as necessary, the above events simultaneously hold with probability $1-\exp(-CN^{1/3})$.
    By construction, $R(\bx^i,\bx^i) = \vone$ and $R(\bx^i,\bx^j) = \vzero$ for all $i\neq j$, which implies part (\ref{itm:x-unit}) and $\vbx \in \cS_N^{k,\perp}$.
    The relation \eqref{eq:sk-span} implies part (\ref{itm:x-span}).
\end{proof}

The following recursive lower bound for $\bbE \GS_N(W,\vv,k)$ is a converse to Lemma~\ref{lem:A-fn-ineq} and is the main step in the proof of Proposition~\ref{prop:sk-lb}.
\begin{lemma}
    \label{lem:sk-fe-lb}
    Let $W, \vv$ be as in Lemma~\ref{lem:sk-ext-field} and $\va \in [0,1]^\sS$, and set $W' = W(W,\vv,\va)$, $\vv' = \vv(W,\vv,\va)$. Then,
    \[
        \bbE \GS_N(W,\vv,k)
        \ge 
        \sum_{s\in \sS}
        \lambda_s v_s \sqrt{a_s}
        + 
        \bbE \GS_{N-kr}(W',\vv',k)
        - o_N(1),
    \]
    where $\GS_{N-kr}$ denotes the ground state energy (see \eqref{eq:sk-gsn}) of a dimension $N-kr$ multi-species quadratic spin glass with species sizes $\tilde \cI_s = \cI_s - k$.
\end{lemma}
\begin{proof}
    Suppose for now the event $\cE$ in Lemma~\ref{lem:sk-multi-gram-schmidt} holds and let $\vbx = (\bx^1,\ldots,\bx^k)$ be as in this lemma.
    Let
    \begin{equation}
        \label{eq:bbtnperp}
        \cS_{N,\perp} 
        \equiv
        \lt\{
            \brho \in \cS_N : 
            R(\brho, \bx^i) = \vzero
            ~\forall i\in [k]
        \rt\}
        =
        \lt\{
            \brho \in \cS_N : 
            R(\brho, \bg^i) = \vzero
            ~\forall i\in [k]
        \rt\}
    \end{equation}
    where the second equality follows from Lemma~\ref{lem:sk-multi-gram-schmidt}(\ref{itm:x-span}) and 
    \begin{equation}
        \label{eq:bbtnperpkperp}
        \cS_{N,\perp}^{k,\perp}
        \equiv
        \lt\{
            \vbrho = (\brho^1, \ldots, \brho^k) \in \cS_{N,\perp}^k : 
            R(\brho^i,\brho^j) = \vzero ~\forall i\neq j
        \rt\}.
    \end{equation}
    For each $i\in [k]$ let 
    \begin{equation}
        \label{eq:sk-recursive-bsig}
        \bsig^i = \sqrt{\va} \diamond \bx^i + \sqrt{\vone - \va} \diamond \brho^i    
    \end{equation}
    where $\vbrho = (\brho^1, \ldots, \brho^k) \in \cS_{N,\perp}^{k,\perp}$.
    The orthogonality relations in \eqref{eq:bbtnperp} and \eqref{eq:bbtnperpkperp} imply $\vbsig = (\bsig^1, \ldots, \bsig^k) \in \cS_N^{k,\perp}$.
    Then, 
    \begin{equation}
        \label{eq:sk-k-step}
        \fr{1}{N} H_{N,k}(\vbsig)
        =
        \fr{1}{kN}
        \sum_{i=1}^k
        \la \vv \diamond \bg^i, \sqrt{\va} \diamond \bx^i \ra + 
        \fr{1}{kN^{3/2}}
        \sum_{i=1}^k
        \lt\la W \diamond \bG, (\sqrt{\va} \diamond \bx^i + \sqrt{\vone - \va} \diamond \brho^i)^{\otimes 2} \rt\ra.
    \end{equation}
    By Lemma~\ref{lem:sk-multi-gram-schmidt}(\ref{itm:x-norms-good}, \ref{itm:x-approx-g}),
    \[
        \fr{1}{N} 
        \la W \diamond \bG, \sqrt{\va} \diamond \bx^i\ra
        = 
        \sum_{s\in \sS}
        \lambda_s v_s \sqrt{a_s}
        + o_N(1).
    \]
    Note that the state space $\cS_{N,\perp}$ is $\cS_N$ with $k$ fewer dimensions in each species, and these dimensions (and $\vbx$) are independent of $\bG$. 
    So, optimizing the second term of \eqref{eq:sk-k-step} over $\vbrho \in \cS_{N,\perp}^{k,\perp}$ is equivalent to optimizing a dimension $N-kr$ multi-species quadratic spin glass.
    The same covariance calculation as \eqref{eq:sk-hH-covariance-calculation} shows that
    \[
        \sup_{\vbrho \in \cS_{N,\perp}^{k,\perp}}
        \fr{1}{N^{3/2}}
        \lt\la W \diamond \bG, (\sqrt{\va} \diamond \bx^i + \sqrt{\vone - \va} \diamond \brho^i)^{\otimes 2} \rt\ra
        =_d
        \sqrt{\fr{N-kr}{N}}
        \GS_{N-kr}(W', \vv')
        +
        O(N^{-1/2}) Z,
    \]
    where $Z\sim \cN(0,1)$ is independent of $\GS_{N-kr}$.
    Thus
    \[
        \bbE \GS_N(W,\vv,k)
        \ge 
        \bbE \ind\{\cE\} \fr{1}{N} H_{N,k}(\vbsig)
        \ge 
        \sum_{s\in \sS}
        \lambda_s v_s \sqrt{a_s}
        + 
        \bbE \GS_{N-kr}(W', \vv',k)
        - o_N(1).
    \]    
\end{proof}

Lemma~\ref{lem:sk-fe-lb} suggests a natural way to construct an approximate ground state of $H_{N,k}$.
First, use Gram-Schmidt orthogonalization to produce $\vbx = (\bx^1,\ldots,\bx^k)$ from the external fields $\bg^1,\ldots,\bg^k$, as in Lemma~\ref{lem:sk-multi-gram-schmidt}.
Choose $\va \in [0,1]^\sS$ and set $\vbsig$ as in \eqref{eq:sk-recursive-bsig}, for $\vbrho \in \cS_{N,\perp}^{k,\perp}$ to be determined.
The correlations of the $\bsig^i$ with the external fields $\bg^i$ contribute energy $\sum_{s\in \sS} \lambda_s v_s \sqrt{a_s}$, while the optimization over $\vbrho$ is equivalent to optimizing another quadratic multi-species spin glass, whose parameters depend on $\va$.
Finally, recursively optimize $\vbrho$.
The following proof demonstrates that when $\vv > \vzero$, there exists a sequence of choices of $\va$ such that running this algorithm to a large constant recursion depth finds a near ground state $\vbsig \in \cS_N^{k,\perp}$ of $H_{N,k}$.
(If some entries of $\vv$ are zero, the algorithm succeeds after first introducing a small artificial external field.)

\begin{proof}[Proof of Proposition~\ref{prop:sk-lb}]
    Assume for now that $\vv \succ \vzero$ where the inequality is strict in each coordinate.
    Define $W^{(0)} = W$, $\vv^{(0)}=\vv$. 
    Denote the relation \eqref{eq:sk-a-opt} by $\va = \va(W,\vv)$.
    Let $T$ be a large constant to be determined, and for $0\le t\le T-1$ define
    \[
        \va^{(t)} = \va(W^{(t)}, \vv^{(t)}),
        \qquad
        W^{(t+1)} = W(W^{(t)}, \vv^{(t)}, \va^{(t)}),
        \qquad
        \vv^{(t+1)} = \vv(W^{(t)}, \vv^{(t)}, \va^{(t)}).
    \]
    Further define
    \[
        E^{(t)} = \sum_{s\in \sS} \lambda_s \sqrt{(v^{(t)}_s)^2 + 2\sum_{s'\in \sS} \lambda_{s'} (w_{s,s'}^{(t)})^2}, 
        \qquad
        F^{(t)} = \sum_{s\in \sS} \lambda_s v_s^{(t)} \sqrt{a_s^{(t)}}.
    \]
    Let $\delta > 0$ be arbitrary; we will show that $\bbE \GS_N(W,\vv) \ge E^{(0)} - \delta$ for all sufficiently large $N$.
    Lemma~\ref{lem:sk-fe-lb} with the choice $\va = \va^{(t)}$ implies that
    \[
        \bbE \GS_{N-tkr}(W^{(t)}, \vv^{(t)}) 
        \ge 
        F^{(t)} + \bbE \GS_{N-(t+1)kr}(W^{(t+1)}, \vv^{(t+1)}) - o_N(1),
    \]
    and summing yields
    \[
        \bbE \GS_N(W,\vv)
        \ge 
        \sum_{t=0}^{T-1} F^{(t)} - o_N(1).
    \]
    Note that
    \begin{align*}
        F^{(t)} &= \sum_{s\in \sS} \lambda_s \sqrt{(v^{(t)}_s)^2 + 2\sum_{s'\in \sS} \lambda_{s'} (w_{s,s'}^{(t)})^2} \cdot a_s^{(t)}, \\
        E^{(t+1)} &= \sum_{s\in \sS} \lambda_s \sqrt{(v^{(t)}_s)^2 + 2\sum_{s'\in \sS} \lambda_{s'} (w_{s,s'}^{(t)})^2} \cdot (1-a_s^{(t)}), 
    \end{align*}
    so $F^{(t)} = E^{(t)} - E^{(t+1)}$.
    Thus
    \[
        \bbE \GS_N(W,\vv)
        \ge 
        E^{(0)} - E^{(T)} - o_N(1).
    \]
    Since
    \[
        (v_s^{(t+1)})^2 + 2\sum_{s'\in \sS} \lambda_{s'} (w_{s,s'}^{(t+1)})^2 
        = 
        2 (1-a_s^{(t)}) \sum_{s'\in \sS} \sum_{s'} \lambda_{s'} (w_{s,s'}^{(t)})^2
    \]
    we have
    \[
        a_s^{(t+1)} 
        = 
        \fr{(v_s^{(t+1)})^2}{(v_s^{(t+1)})^2 + 2\sum_{s'\in \sS} \lambda_{s'} (w_{s,s'}^{(t+1)})^2}
        =
        \fr{\sum_{s'\in \sS} a_{s'}^{(t)} \lambda_{s'} (w_{s,s'}^{(t)})^2}{\sum_{s'\in \sS} \lambda_{s'} (w_{s,s'}^{(t)})^2}.
    \]
    It follows that, for $\alpha^{(t)} = \min_{s\in \sS} a_s^{(t)}$, we have $\alpha^{(t+1)} \ge \alpha^{(t)}$.
    The assumption $\vv > \vzero$ ensures $\alpha^{(0)} > 0$.
    Because $E^{(t+1)}/E^{(t)} \le 1 - \alpha^{(t)}$, we have
    \[
        E^{(T)} \le E^{(0)} (1 - \alpha^{(0)})^T < \delta/2
    \]
    for sufficiently large constant $T$.
    This implies $\bbE \GS_N(W,\vv) \ge E^{(0)} - \delta/2 - o_N(1) \ge E^{(0)} - \delta$.
    This proves the result when $\vv \succ \vzero$.
    
    If some coordinates of $\vv$ are zero, we apply this result to $\vv + \eta \vone$ for small $\eta > 0$.
    By \eqref{eq:A-subadditive},
    \[
        \GS_N(W,\vv)
        \ge 
        \GS_N(W,\vv+\eta \vone)
        -
        \GS_N(0,\eta \vone),
    \]
    so for sufficiently large $N$,
    \[
        \bbE \GS_N(W,\vv)
        \ge 
        \sum_{s\in \sS}
        \lambda_s 
        \sqrt{v_s^2 + 2\sum_{s'\in \sS} \lambda_{s'} w_{s,s'}^2}
        - \delta - \eta.
    \]
    As this holds for any $\delta, \eta > 0$ the result follows.
\end{proof}


\begin{proof}[Proof of Lemma~\ref{lem:sk-ext-field}]
    Follows from Corollary~\ref{cor:sk-ub} and Proposition~\ref{prop:sk-lb}.
\end{proof}

\section{Deferred Proofs From Section~\ref{sec:alg}}
\label{app:alg}

\subsection{Existence of a Maximizer: Proof of Proposition~\ref{prop:F-max}}
\label{subsec:maximizer-existence}

\propFmax*


Given $(p,\Phi,q_0)\in\cM$ we extend $p,\Phi$ to domain $[0,1]$ by setting $p(q)=0$ for $q\in [0,q_0)$ and making $\Phi$ linear on $[0,q_0]$ with $\Phi(0)=\vzero$. Using this canonical extension we equip $\cM$ with the metric 
    \begin{equation}
    \label{eq:metric}
    d\lt((p^1,\Phi^1,q_0^1),(p^2,\Phi^2,q_0^2)\rt)
    =
    \norm{p^1-p^2}_{L^1([0,1])}
    +
    \norm{\Phi^1-\Phi^2}_{L^1([0,1])}
    + 
    |q_0^1-q_0^2|.
\end{equation}


We will prove that $\cM$ is a compact space on which $\bbA$ is upper semi-continuous. Existence of a triple $(p,\Phi;q_0)\in\cM$ maximizing \eqref{eq:alg} within this space then follows.


\begin{proposition}
\label{prop:compact-space}
The space $\cM$ with metric \eqref{eq:metric} is compact.
\end{proposition}

\begin{proof}
Given an infinite sequence $(p^n,\Phi^n,q_0^n)_{n\geq 0}$ of points in $\cM$, we show there is a limit point. First find a subsequence $(a_n)$ along which the convergence $q_0^{a_n}\to q_0$ holds. Then the subsequence $(p^{a_n})_{n\geq 0}$ has a subsubsequential limit in the space $L^1([q_0,1])$; similarly for $(\Phi_s^{a_n})_{n\geq 0}$, for each $s\in\sS$. Thus we may choose a subsequence $b_n$ of $a_n$ on which $p^{b_n}\to p$ and $\Phi_s^{b_n}\to \Phi_s$ (for all $s\in \sS$) in $L^1([q_0,1])$. It is easy to see that $p$ and each $\Phi_s$ vanishes on $[0,q_0)$, and that $\Phi$ satisfies admissibility. It is easy to see that 
\[
    \|p^{b_n}-p\|_{L^1([0,1])}
    \leq
    \|p^{b_n}-p\|_{L^1([q_0,1])}
    +
    |q_0-q^{b_n}_0|
\]
and 
\[
    \|\Phi_s^{b_n}-\Phi_s\|_{L^1([0,1])}
    \leq
    \|\Phi_s^{b_n}-\Phi_s\|_{L^1([q_0,1])}
    +
    |q_0-q^{b_n}_0|.
\]
It follows that $(p^{b_n},\Phi^{b_n},q_0^{b_n})\to (p,\Phi,q_0)$ in $\cM$. This completes the proof.
\end{proof}


\begin{proposition}
\label{prop:F-bounded}
The function $\bbA$ is uniformly bounded on $\cM$.
\end{proposition}


\begin{proof}
For any admissible $\Phi$ we have by Cauchy-Schwarz
\begin{align}
\nonumber
    \sum_{s\in\sS}\lambda_s\int_0^1 \sqrt{{\Phi}'_s(q)(p\times \xi^s\circ\Phi)'(q)}\de q
    &\leq 
    \sum_{s\in\sS}\lambda_s\int_0^1 \lt({\Phi}_s'(q)+(p\times \xi^s\circ\Phi)'(q)\rt)\de q
    \\
\label{eq:F-bounded}
    &\leq
    \sum_{s\in\sS}\lambda_s \lt(1+\xi^s(\vone)-\xi^s(\vzero)\rt).
\end{align}
The first term of $\bbA$ is clearly uniformly bounded, so the result follows.
\end{proof}


\begin{proposition}
\label{prop:F-usc}
$\bbA$ is upper semi-continuous on $\cM$.
\end{proposition}


\begin{proof}
Suppose $(p^{b_n},\Phi^{b_n},q_0^{b_n})\to (p,\Phi,q_0)$ in $\cM$. We write
\begin{align*}
    |\bbA(p^{b_n},\Phi^{b_n};q_0^{b_n})-\bbA(p,\Phi;q_0)|
    &\leq
    \int_{q_0}^1
    \lt|\sqrt{(\Phi_s^{b_n})'(q)(p^{b_n}\times \xi^s\circ\Phi^{b_n})'(q)}
    -
    \sqrt{\Phi'_s(q)(p\times \xi^s\circ\Phi)'(q)}\rt|\de q
    \\
    &  \quad +
    C_{\lambda}
    \lt(
    \lt|\int_{q_0^{b_n}}^{q_0}
    \sqrt{p'(q)+1}
    \,\de q\,
    \rt|
    +
    \sum_{s\in \sS}
    \lt|\sqrt{\Phi_s^n(q_0^n)}-\sqrt{\Phi_s(q_0)}\rt|
    \rt)
    .
\end{align*}
The sum over $s\in\sS$ obviously tends to $0$. Moreover by Cauchy--Schwarz, 
\begin{align*}
    \lt|\int_{q_0^{b_n}}^{q_0}
    \sqrt{p'(q)+1}
    \,\de q\,
    \rt|
    &\leq
    |q_0-q_0^{b_n}|^{1/2}\cdot \sqrt{p(q_0)-p(q_0^{b_n})+1}
    \\
    &\leq
    C'_{\lambda}(q_0-q_0^{b_n})^{1/2}.
\end{align*}
Therefore it suffices to show the first term above tends to $0$. Since the map
\[
    (p,\Phi)\mapsto (p\times \xi^s\circ\Phi)
\]
from $L^1([0,1])^{|\sS|+1}\to L^1([0,1])$ is continuous and returns a non-decreasing function, it suffices to show that
\[
    G(f,g)=\int_0^1 \sqrt{f'(q)g'(q)}\,\de q
\]
is upper semi-continuous on $L^1([0,1])\times L^1([0,1])$ when restricted to non-decreasing functions. This is essentially equivalent to upper semi-continuity of Hellinger distance which is well-known.
\end{proof}

Combining the results above implies Proposition~\ref{prop:F-max}.


\subsection{A Priori Regularity of Maximizers}
\label{subsec:regularity-for-4.1}

Let $(p,\Phi,q_0)\in\cM$ be a maximizer of $\bbA$, which exists by Proposition~\ref{prop:F-max}. In this subsection we will prove the following two propositions. 

\propBasicRegularity*

\propPBasic*

\begin{lemma}
    \label{lem:p-AC}
    The function $p$ is absolutely continuous and $p(1)=1$. Moreover $p'$ is uniformly bounded on compact subsets of $(q_0,1)$.
\end{lemma}
\begin{proof}
    Given any increasing $p:[q_0,1]\to [q_0,1]$, we may view $p'$ as a positive measure of the form
    \begin{equation}
    \label{eq:positive-measure-of-the-form}
        p'(x)dx = f(x)dx + \mu(dx)
    \end{equation}
    for $\mu$ a singular-plus-atomic measure and $f\in L^1([q_0,1];\mathbb R_{\geq 0})$. We may then replace $p$ by $\bar p$ such that 
    \[
        \bar p'(x)dx=f(x)dx,\quad\text{ and }\quad \bar p(1)=1.
    \]
    Then $\bar p(x)\geq p(x)$ for all $x\in [q_0,1]$, and $\bar p'(x)$ agrees with $p'(x)$ except for a singular-plus-atomic part. It follows that 
    \[
        \bbA(p,\Phi;q_0)\leq \bbA(\bar p,\Phi;q_0).
    \]
    Moreover it is easy to see that strict inequality $\bbA(p,\Phi;q_0)< \bbA(\bar p,\Phi;q_0)$ holds whenever $p\neq \bar p$. We conclude that $p$ is absolutely continuous and $p(1)=1$.
    
    
    To show the latter statement, we use a similar argument with more care. Let $q\in (q_0,1)$ and choose a large constant $C=C(q_0,q)$. Recalling \eqref{eq:positive-measure-of-the-form}, suppose $\|f(x)\|_{L^{\infty}([q,1])}> C$ for a large constant $C$ and let 
    \[
    c\equiv\frac{\int_q^1 (f(x)-C)_+~\de x}{q-q_0}.
    \]
    We may replace $f$ by
    \[
        f_C(x)=
        \begin{cases}
        f(x),\quad x\in [0,q_0)\\
        f(x)+c,\quad x\in [q_0,q)\\
        \min(C,f(x)),\quad x\in [q,1]
        \end{cases}
    \]
    and similarly replace $p$ by $p_C$ with 
    \[
        \bar p_C'(x)dx=f_C(x)~\de x,\quad\text{ and }\quad \bar p(1)=1.
    \]
    It is easy to see that $p_C(x)\geq p(x)$ for each $x\in [0,1]$. Keeping $\Phi$ the same, we consider the change in $\bbA$. The decrease in $\bbA$ on $[q,1]$ is at most
    \begin{equation}
    \label{eq:q-1}
    \begin{aligned}
        &\sum_{s\in\sS}
        \lambda_s
        \int_q^1
        \sqrt{\Phi_s'(x)(p\times \xi^s\circ\Phi)'(x)}
        -
        \sqrt{\Phi_s'(x)(p_C\times \xi^s\circ\Phi)'(x)}
        ~\de x
        \\
        &=
        \sum_{s\in\sS}
        \lambda_s
        \int_q^1
        \sqrt{\Phi_s'(x)\lt(p'(x)\xi^s\big(\Phi(x)\big)+p(x)\langle \Phi'(x),\nabla\xi^s\big(\Phi(x)\big)\rangle\rt)}
        \\
        &\quad\quad\quad\quad\quad\quad -
        \sqrt{\Phi_s'(x)\lt(p_C'(x)\xi^s\big(\Phi(x)\big)+p_C(x)\langle \Phi'(x),\nabla\xi^s\big(\Phi(x)\big)\rangle\rt)}
        ~\de x
        \\
        &\leq
        \sum_{s\in\sS}
        \lambda_s
        \int_q^1
        \sqrt{\Phi_s'(x)\lt(p'(x)\xi^s\big(\Phi(x)\big)+p(x)\langle \Phi'(x),\nabla\xi^s\big(\Phi(x)\big)\rangle\rt)}
        \\
        &\quad\quad\quad\quad\quad\quad -
        \sqrt{\Phi_s'(x)\lt(p_C'(x)\xi^s\big(\Phi(x)\big)+p(x)\langle \Phi'(x),\nabla\xi^s\big(\Phi(x)\big)\rangle\rt)}
        ~\de x
        \\
        &\leq
        \sum_{s\in\sS}
        \lambda_s
        \int_q^1
        \sqrt{\Phi_s'(x) p'(x)\cdot\xi^s\big(\Phi(x)\big)}
        -
        \sqrt{\Phi_s'(x)p_C'(x)\cdot\xi^s\big(\Phi(x)\big)}
        ~\de x
        \\
        &\leq
        O(1)\cdot
        \int_q^1
        \sqrt{p'(x)}-\sqrt{p_C'(x)}
        \de x
        \\
        &\leq
        O(1)\cdot \int_{q}^1 C^{-1/2}(f(x)-C)_+ ~\de x
        \\
        &\leq
        O\lt(\frac{c(q-q_0)}{\sqrt{C}}\rt)
        .
    \end{aligned}
    \end{equation}
    (In the second inequality we used $\sqrt{x+z}-\sqrt{y+z}\leq \sqrt{x}-\sqrt{y}$ for $x\geq y\geq 0$, and in the third we used that $\Phi_s'$ is uniformly bounded by admissibility.)
    On $x\in [q_0,q]$, we find that changing from $p$ to $p_C$ increases the value of $\bbA$:
    \begin{align*}
        &\sum_{s\in\sS}
        \lambda_s
        \int_{q_0}^q
        \sqrt{\Phi_s'(x)(p_C\times \xi^s\circ\Phi)'(x)}
        -
        \sqrt{\Phi_s'(x)(p\times \xi^s\circ\Phi)'(x)}
        ~\de x
        \\
        &=
        \sum_{s\in\sS}
        \lambda_s
        \int_{q_0}^q
        \sqrt{\Phi_s'(x)\lt(p_C'(x)\xi^s\big(\Phi(x)\big)+p_C(x)\langle \Phi'(x),\nabla\xi^s\big(\Phi(x)\big)\rangle\rt)}
        \\
        &\quad\quad -
        \sqrt{\Phi_s'(x)\lt(p'(x)\xi^s\big(\Phi(x)\big)+p(x)\langle \Phi'(x),\nabla\xi^s\big(\Phi(x)\big)\rangle\rt)}
        ~\de x
        \\
        &\geq
        \sum_{s\in\sS}
        \lambda_s
        \int_{q_0}^q
        \sqrt{\Phi_s'(x)\lt((p'(x)+c)\xi^s\big(\Phi(x)\big)+p(x)\langle \Phi'(x),\nabla\xi^s\big(\Phi(x)\big)\rangle\rt)}
        \\
        &\quad\quad -
        \sqrt{\Phi_s'(x)\lt(p'(x)\xi^s\big(\Phi(x)\big)+p(x)\langle \Phi'(x),\nabla\xi^s\big(\Phi(x)\big)\rangle\rt)}
        ~\de x
        \\
        &\geq
        \Omega(c)
        \cdot
        \int_{q_0}^q  
        \sum_{s\in\sS}
        \frac{\de x}{ \sqrt{\Phi_s'(x)\lt((p'(x)+c)\xi^s\big(\Phi(x)\big)+p(x)\langle \Phi'(x),\nabla\xi^s\big(\Phi(x)\big)\rangle\rt)}}\,.
    \end{align*}
    By Markov's inequality, $p'(x)\leq \frac{2}{(q-q_0)}$ on a set of $x\in [q_0,q]$ of measure at least $\frac{q-q_0}{2}$. For each such $x$, we have $\Phi_s'(x)\leq O(1)$ and $\xi^s(\Phi(x))\leq O(1)$. We thus find
    \[
        \Omega(c)
        \cdot
        \int_{q_0}^q  
        \sum_{s\in\sS}
        \frac{\de x}{ \sqrt{\Phi_s'(x)\lt((p'(x)+c)\xi^s\big(\Phi(x)\big)+p(x)\langle \Phi'(x),\nabla\xi^s\big(\Phi(x)\big)\rangle\rt)}}
        \geq
        \Omega\lt(\frac{c(q-q_0)}{\sqrt{q-q_0+c}}\rt)
    \]
    Since $c\leq \frac{1}{q-q_0}$, for $C$ sufficiently large, combining with \eqref{eq:q-1} above implies that
    \[
    \sum_{s\in\sS}
        \lambda_s
        \int_{q}^1
        \sqrt{\Phi_s'(x)(p_C\times \xi^s\circ\Phi)'(x)}
        -
        \sqrt{\Phi_s'(x)(p\times \xi^s\circ\Phi)'(x)}
        ~\de x>0.
    \]
    Since $p(x)=p_C(x)$ for $x\leq q_0$, we find $\bbA(p,\Phi;q_0)<\bbA(p_C,\Phi;q_0)$, contradicting maximality of $\bbA(p,\Phi;q_0)$. Having reached a contradiction for $C$ sufficiently large, we conclude that $p'$ is uniformly bounded on $[q,1]$ for each $q\in (q_0,1)$ as desired.
\end{proof}


\begin{lemma}
    \label{lem:p-positive}
    $p(q)>0$ holds for all $q>q_0$.
\end{lemma}
\begin{proof}
    Suppose not. Then $p(q)=0$ for all $q\in [q_0,q_0+\eps]$, for some $\eps>0$. For $\delta>0$ small, define
    \[
    p_{\delta}(q)=\delta+(1-\delta)p(q).
    \]
    Then 
    \begin{align*}
        \sum_{s\in \sS}\lambda_s\int_{q_0}^{q_0+\eps}\sqrt{\Phi_s'(q)(p_{\delta}\times\xi^s\circ\Phi)'(q)}\de q
        &=
        \delta^{1/2}\sum_{s\in \sS}\lambda_s\int_{q_0}^{q_0+\eps} \sqrt{\Phi_s'(q)(\xi^s\circ\Phi)'(q)}\de q
        \\
        &\geq
        \delta^{1/2}c(\xi)\sum_{s\in\sS}\lambda_s\int_{q_0}^{q_0+\eps}\sqrt{\Phi_s'(q)^2}\de q\\
        &=
        \delta^{1/2}c(\xi).
    \end{align*}
    while
    \[
        \sum_{s\in \sS}\lambda_s\int_{q_0}^{q_0+\eps}\sqrt{\Phi_s'(q)(p\times\xi^s\circ\Phi)'(q)}\de q
        =
        0.
    \]
    On the other hand since $p_{\delta}(q)\geq p(q)$ for all $q\in [q_0,1]$ and $(p_{\delta})'=(1-\delta)(p)'$ as measures, we obtain
    \begin{align*}
        \sum_{s\in \sS}\lambda_s\int_{q_0+\eps}^1\sqrt{\Phi_s'(q)(p_{\delta}\times\xi^s\circ\Phi)'(q)}\de q
        &\geq 
        (1-\delta)\sum_{s\in \sS}\lambda_s\int_{q_0+\eps}^1 \sqrt{\Phi_s'(q)(p\times \xi^s\circ\Phi)'(q)}\de q.
    \end{align*}
    Combining the above implies $\bbA(p_{\delta},\Phi;q_0)>\bbA(p,\Phi;q_0)$ for small enough $\delta$, a contradiction. 
\end{proof}


\begin{lemma}
\label{lem:Phi-q0-neq-1}
    For all $s\in\sS$ and $q\in (0,1)$, we have $\Phi_s(q)<1$.
\end{lemma}

\begin{proof}
    Suppose $\Phi_{s_0}(q_*)=1$; this implies $0<q_0\leq q_*<1$. For small $\delta>0$ we consider the perturbation $\Phi_{\delta}$ with $\Phi_{\delta,s}=\Phi_s$ for $s\neq s_0$ and:
    \[
    \Phi_{\delta,s_0}'(q)
    =
    \begin{cases}
    \Phi_{s_0}'(q)\cdot (1-\delta(1-q_*)),\quad q\in [0,q_*],
    \\
    \delta,\quad\quad\quad\quad\quad\quad\quad\quad\quad\quad q\in [q_*,1]
    \end{cases}
    \]
    Note that $\Phi_{\delta,s}'(q)\geq (1-O(\delta))\Phi_{s}'(q)$ and so also $\Phi_{\delta,s}(q)\geq (1-O(\delta))\Phi_{s}(q)$ for all $s\in\sS$ and $q\in [0,1]$. As $\bbA$ is uniformly bounded, we can thus bound
    \begin{align*}
    \bbA(p,\Phi_{\delta};q_0)-\bbA(p,\Phi;q_0)
    &=
    \sum_{s\in \sS}
    h_s \lambda_s \sqrt{\Phi_{\delta,s}(q_0)}
    +
    \lambda_s 
    \int_{q_0}^1
    \sqrt{\Phi'_{\delta,s}(q) (p\times \xi^s \circ \Phi_{\delta})'(q)}
    ~\de q
    \\
    &\quad\quad
    -
    \sum_{s\in \sS}
    h_s \lambda_s \sqrt{\Phi_s(q_0)}
    -
    \lambda_s 
    \int_{q_0}^1
    \sqrt{\Phi'_s(q) (p\times \xi^s \circ \Phi)'(q)}
    ~\de q
    \\
    &\geq
    -O(\delta)
    +
    \lambda_{s_0}
    \int_{\frac{1+q_*}{2}}^1
    \sqrt{\Phi'_{\delta,s}(q) (p\times \xi^s \circ \Phi)'(q)}
    -
    \sqrt{\Phi'_s(q) (p\times \xi^s \circ \Phi)'(q)}
    ~\de q
    .
    \end{align*}
    Using Lemma~\ref{lem:p-positive}, admissibility and non-degeneracy of $\xi$, we find that $(p\times \xi^s \circ \Phi)'(q)\geq \Omega(q)$ for all $q\geq \frac{1+q_*}{2}$. Therefore 
    \[
    \lambda_{s_0}
    \int_{\frac{1+q_*}{2}}^1
    \sqrt{\Phi'_{\delta,s}(q) (p\times \xi^s \circ \Phi)'(q)}
    -
    \sqrt{\Phi'_s(q) (p\times \xi^s \circ \Phi)'(q)}
    ~\de q
    \geq \Omega(\delta^{1/2})
    \]
    for small $\delta$. Since $\delta^{1/2}$ is of larger order than $\delta$ we conclude that $\bbA(p,\Phi_{\delta};q_0)>\bbA(p,\Phi;q_0)$. This is a contradiction (recall Lemma~\ref{lem:admissible-optional}) and completes the proof.
\end{proof}



Next we turn our attention to $\Phi'$. Similarly to Lemma~\ref{lem:p-positive}, the idea is that the square root function has infinite derivative at $0$. 
 


\begin{lemma}
    \label{lem:Phi-inc}
    There exists $\eta>0$ such that $\Phi'(q) \succeq \eta \vone$ almost everywhere in $q\in [q_0,1]$.
\end{lemma}
\begin{proof}
    First, given $(p,\Phi;q_0)$ choose for some $s\in\sS$ (specified below) a Lebesgue point $q_s\in (q_0,1)$ of $\Phi'$ with 
    \begin{equation}
    \label{eq:Phi-s-1}
        \Phi_s'(q_s)\geq a
    \end{equation}
    for $a>0$. Lemma~\ref{lem:Phi-q0-neq-1} ensures this is possible for some $a$ depending only on $q_0$ and $\Phi(q_0)$ (as long as $q_0<1$, else there is nothing to prove). In fact we can actually find two distinct such points $q_s^{(1)},q_s^{(2)}$ (which will be helpful below).

    Next for small $\eps>0$ depending only on $(p,\Phi)$, define the interval
    \begin{align*}
        J_{s,\eps}&=(q_s-\eps,q_s+\eps).
    \end{align*}
    By \eqref{eq:Phi-s-1} and the fact that $q_s$ is a Lebesgue point of $\Phi'$, there is a subset $I_{s,\eps}\subseteq J_{s,\eps}$ of Lebesgue measure at least $|I_{s,\eps}|\geq \frac{|J_{s,\eps}|}{2}=\eps$ such that
    \begin{equation}
    \label{eq:Phi-s-half}
        \Phi_s'(q)\geq \frac{a}{2} 
        ,\quad
        \forall q\in I_{s,\eps}.
    \end{equation}
    as long as $\eps>0$ is chosen sufficiently small.
    A simple consequence is the estimate
    \begin{equation}
    \label{eq:C-eps}
        C_{\eps}:=\Phi_s(q_s+\eps)-\Phi_s(q_s-\eps)=\int_{q_s-\eps}^{q_s+\eps} \Phi_s'(q)\de q\geq \frac{a\eps}{2}.
    \end{equation}


    With the setup above complete (except that $s$ is not yet specified), suppose the conclusion is false let $\eta$ be sufficiently small depending on $(p,\Phi,\eps)$ for $\eps$ as above. Then there exist $s,{s_0}\in\sS$ and 
    $\hat q_{s_0}\in (q_0,1)$ which is a Lebesgue point for $\nabla \Phi$ such that 
    \begin{align}
    \label{eq:Phi-s-0}
        \Phi_s'(\hat q_{s_0})&\leq \eta,
        \\
    \label{eq:Phi-r-1}
        \Phi_{s_0}'(\hat q_{s_0})&\geq 1.
    \end{align}
    Indeed if $q$ is any Lebesgue point of $\nabla \Phi$ satisfying \eqref{eq:Phi-s-0} for some $s$, then \eqref{eq:Phi-r-1} holds for some ${s_0}\neq s$ by admissibility and we define $\hat q_{s_0}=q$ this way. The bound \eqref{eq:Phi-s-0} determines the species $s$ chosen initially.
    
    As $\hat q_{s_0}$ is also a Lebesgue point of $\Phi'$, in light of \eqref{eq:Phi-s-0} and \eqref{eq:Phi-r-1}, there exists a set $I_{{s_0},\eta}\subseteq J_{{s_0},\eps}=(\hat q_{s_0}-\eps,\hat q_{s_0}+\eps)$ of positive Lebesgue measure such that the inequalities
    \begin{align}
    \label{eq:Phi-s-eta}
        \Phi_s'(q)&\leq 2\eta,
        \\
    \label{eq:Phi-r-half}
        \Phi_{s_0}'(q)&\geq \frac{a}{2}.
    \end{align}
    both hold for all $q\in I_{{s_0},\eta}$.
    Moreover we can assume $J_{s,\eps},J_{s_0,\eps}$ are disjoint, i.e. $|q_s-\hat q_{s_0}|> 2\eps$. Indeed as noted earlier we can choose two candidate points $q_s^{(1)},q_s^{(2)}$. If $\eps<|q_s^{(1)}-q_s^{(2)}|/5$ is taken, at least one of them suffices for any $\hat q_{s_0}\in (q_0,1)$.
    
    
    Next choose $\delta\in (0,\eta)$ small and consider the perturbation $\Phi_{\delta}$ with $\Phi_{\delta}(q_0)=\Phi(q_0)$ and
    \[
       \Phi_{\delta,s}'(q)
        =
        \begin{cases}
        \Phi_s'(q)+\delta,\quad q\in I_{s_0,\eta}
        \\
        \Phi_s'(q)\lt(1-\frac{\delta |I_{s_0,\eta}|}{C_{\eps}}\rt),\quad \forall q\in J_{s,\eps}
        \\
        \Phi_s'(q),\quad \text{otherwise}
        \end{cases}
    \]
    and $\Phi_{\delta,s'}=\Phi_{s'}$ for all $s'\in \sS\backslash \{s\}$. (Note we used disjointness of $J_{s,\eps},J_{s_0,\eps}$ for this definition to make sense.)
    By Lemma~\ref{lem:admissible-optional}, we must have $\bbA(p,\Phi_{\delta};q_0)\geq \bbA(p,\Phi;q_0)$ although $\Phi_{\delta}$ may not be admissible. Then for $\delta\leq\eta$,
    \begin{align}
    \nonumber
        \int_{I_{s_0,\eta}}
        \sqrt{\Phi_{\delta,s}'(q)(p\times\xi^s\circ\Phi)'(q)}
        -
        &
        \sqrt{\Phi_s'(q) (p\times\xi^s\circ\Phi)'(q)}
        \de q
        \\
    \nonumber
        &\stackrel{\eqref{eq:Phi-s-eta}}{\geq}
        (\sqrt{2\eta+\delta}-\sqrt{2\eta})
        \int_{I_{s_0,\eta}}\sqrt{(p\times\xi^s\circ\Phi)'(q)}\de q
        \\
    \nonumber
        &\geq
        \frac{\delta p(q_s-\eps)^{1/2}}{10\eta^{1/2}}
        \int_{I_{s_0,\eta}}
        \sqrt{(\xi^s\circ\Phi)'(q)}\de q
        \\
    \nonumber
        &\geq 
        \frac{\delta p(q_s/2)^{1/2}c(\xi)}
        {10\eta^{1/2}}
        \int_{I_{s_0,\eta}}
        \sqrt{\Phi_{s_0}'(q)}\de q
        \\
    \label{eq:Phi-gain}
        &\stackrel{\eqref{eq:Phi-r-half}}{\geq}
        \frac{\delta a^{1/2} p(q_s/2)^{1/2}c(\xi)|I_{s_0,\eta|}}
        {20\eta^{1/2}}
        .
    \end{align}
    We used non-degeneracy of $\xi$ in the penultimate step. On the other hand recalling \eqref{eq:C-eps}, it follows that for all $\wt{s}\in\sS$ and almost all $q\in [q_0,1]$:
    \begin{equation}
    \label{eq:Phi-comp-1}
        \Phi_{\delta,\wt{s}}'(q)\geq \lt(1-O\lt(\frac{\delta|I_{s_0,\eta}|}{\eps}\rt)\rt)
        \Phi_{\wt{s}}'(q).
    \end{equation}
    Integrating on $[q_0,q]$, we find 
    \begin{equation}
    \label{eq:Phi-comp-2}
        \Phi_{\delta,\wt{s}}(q)\geq \lt(1-O\lt(\frac{\delta|I_{s_0,\eta}|}{\eps}\rt)\rt)
        \Phi_{\wt{s}}(q)
    \end{equation}
    for all $q\in [q_0,1]$. By the chain rule we similarly obtain that for all $\wt{s}\in\sS$,
    \begin{align}
    \label{eq:Phi-comp-3}
        (p\times \xi^{\wt{s}}\circ\Phi_{\delta})'
        &\geq 
        \lt(1-O\lt(\frac{\delta|I_{s_0,\eta}|}{\eps}\rt)\rt)(p\times \xi^{\wt{s}}\circ\Phi)',
        \\
    \label{eq:Phi-comp-4}
        (p\times \xi^{\wt{s}}\circ\Phi_{\delta})(q)
        &\geq 
        \lt(1-O\lt(\frac{\delta|I_{s_0,\eta}|}{\eps}\rt)\rt)(p\times \xi^{\wt{s}}\circ\Phi)(q).
    \end{align}
    It follows from \eqref{eq:Phi-comp-1}, \eqref{eq:Phi-comp-2}, \eqref{eq:Phi-comp-3}, \eqref{eq:Phi-comp-4} that
    \begin{align}
    \nonumber
        \int_{I_{s_0,\eta}}
        \sqrt{\Phi_{\delta,s}'(q)(p\times\xi^s\circ\Phi_{\delta})'(q)}
        &\geq
        \lt(1-O\lt(\frac{\delta|I_{s_0,\eta}|}{\eps}\rt)\rt)
        \int_{I_{s_0,\eta}}
        \sqrt{\Phi_{\delta,s}'(q) (p\times\xi^s\circ\Phi)'(q)}
        \de q
        \\
    \label{eq:Phi-extra-term}
        &\stackrel{\eqref{eq:F-bounded}}{\geq}
        \int_{I_{s_0,\eta}}
        \sqrt{\Phi_{\delta,s}'(q) (p\times\xi^s\circ\Phi)'(q)} \de q
        - 
        O\lt(\frac{\delta|I_{s_0,\eta}|}{\eps}\rt).
    \end{align}
    Since $\Phi_{\delta}$ and $\Phi$ differ only inside $[q_0,1]$ we use $\bbA_{[q_0,1]}$ below to denote the second term of $\bbA$. We have:
    \begin{align*}
        \bbA_{[q_0,1]}(p,\Phi)
        &=
        \sum_{\wt{s}\in\sS}\lambda_{\wt{s}}\int_{q_0}^1 \sqrt{\Phi_{\wt{s}}'(q)(p\times \xi^{\wt{s}}\circ\Phi)'(q)}\de q
        \\
        &=
        \lambda_s\int_{I_{s_0,\eta}} \sqrt{\Phi_s'(q)(p\times \xi^{s}\circ\Phi)'(q)}\de q
        +
        \lambda_s\int_{[q_0,1]\backslash I_{s_0,\eta}} \sqrt{\Phi_s'(q)(p\times \xi^{s}\circ\Phi)'(q)}\de q
        \\
        &
        \qquad 
        +
        \sum_{\wt{s}\in\sS\backslash\{s\}}\lambda_{\wt{s}}\int_{q_0}^1 \sqrt{\Phi_{\wt{s}}'(q)(p\times \xi^{\wt{s}}\circ\Phi)'(q)}\de q
        \\
        &\equiv
        \I+\II+\III.
    \end{align*}
    Similarly for $J$ instead of $I$, 
    \begin{align*}
        \bbA_{[q_0,1]}(p,\Phi_{\delta})
        &=
        \sum_{\wt{s}\in\sS}\lambda_{\wt{s}}\int_{q_0}^1 \sqrt{\Phi_{\delta,\wt{s}}'(q)(p\times \xi^{\wt{s}}\circ\Phi_{\delta})'(q)}\de q
        \\
        &=
        \lambda_s\int_{I_{s_0,\eta}} \sqrt{(\Phi_{\delta,s})'(q)(p\times \xi^{s}\circ\Phi_{\delta})'(q)}\de q
        +
        \lambda_s\int_{[q_0,1]\backslash I_{s_0,\eta}} \sqrt{(\Phi_{\delta,s})'(q)(p\times \xi^{s}\circ\Phi_{\delta})'(q)}\de q
        \\
        &
        \qquad 
        +
        \sum_{\wt{s}\in\sS\backslash\{s\}}\lambda_{\wt{s}}\int_{q_0}^1 \sqrt{\Phi_{\delta,\wt{s}}'(q)(p\times \xi^{\wt{s}}\circ\Phi_{\delta})'(q)}\de q
        \\
        &\equiv
        \I_{\delta}+\II_{\delta}+\III_{\delta}.
    \end{align*}
    Using \eqref{eq:Phi-comp-1}, \eqref{eq:Phi-comp-2}, \eqref{eq:Phi-comp-3}, \eqref{eq:Phi-comp-4} again, we obtain
    \begin{align*}
        \II_{\delta}&\geq \lt(1-O\lt(\frac{\delta|I_{s_0,\eta}|}{\eps}\rt)\rt)\II,
        \\
        \III_{\delta}&\geq \lt(1-O\lt(\frac{\delta|I_{s_0,\eta}|}{\eps}\rt)\rt)\III.
    \end{align*}
    Meanwhile \eqref{eq:Phi-gain} and \eqref{eq:Phi-extra-term} imply that for $\delta$ small compared to $\eta$,
    \[
    \I_{\delta}\geq \lt(1-O\lt(\frac{\delta|I_{s_0,\eta}|}{\eps}\rt)\rt)\I + \frac{\delta a^{1/2}p(q_s/2)^{1/2}c(\xi)|I_{s_0,\eta}|}
        {20\eta^{1/2}}.
    \]
    Combining, we find
    \[
        \bbA_{[q_0,1]}(p,\Phi_{\delta})\geq \bbA_{[q_0,1]}(p,\Phi)+
        \frac{\delta a^{1/2}p(q_s/2)^{1/2}c(\xi)|I_{s_0,\eta}|}
        {20\eta^{1/2}}-O\lt(\frac{\delta|I_{s_0,\eta}|}{\eps}\rt).
    \]
    Taking $\eta\ll \eps^2 ap(q_s/2)c(\xi)^2$ and then $\delta$ sufficiently small contradicts the maximality of $(p,\Phi,q_0)$, thus completing the proof.
\end{proof}




\begin{proposition}
    \label{prop:p-q0-0}
    If $q_0 > 0$, then $p(q_0)=0$.
\end{proposition}
\begin{proof}
    Assume that $p(q_0)>0$. Consider the perturbation
    \[
        \wtp(q)=
        \begin{cases}
        p(q)+(q-q_0-\eps)\delta,\quad q<q_0+\eps
        \\
        p(q),\quad\quad\quad\quad\quad\quad\quad q\geq q_0+\eps.
        \end{cases}
    \]
    The function $\tilde p$ is increasing, and is non-negative for sufficiently small $\eps,\delta>0$. For $q<q_0+\eps$ we find
    \begin{align*}
        \deriv{\delta} (p\times \xi^s\circ\Phi)'(q)
        &=
        \deriv{\delta}\lt(
            p'(q)\xi^s(\Phi(q))
            +
            p(q)(\xi^s\circ\Phi)'(q)
        \rt)
        \\
        &=
        \xi^s(\Phi(q))
        -
        (q_0+\eps-q)
        (\xi^s\circ\Phi)'(q)
        \\
        &\geq 
        \xi^s(\Phi(q))-O(\eps).
    \end{align*}
    If $q_0>0$, then $\xi^s(\Phi(q))\geq c(q_0)>0$ by admissibility and non-degeneracy of $\Phi$. This contradicts optimality of $(p,\Phi,q_0)$ and completes the proof.
\end{proof}


\begin{proof}[Proof of Proposition~\ref{prop:p-basic}]
    Follows from Lemmas~\ref{lem:p-AC} and \ref{lem:p-positive} and Proposition~\ref{prop:p-q0-0}.
\end{proof}



\subsubsection{Continuous Differentiability on $(q_0,1]$}
\label{subsubsec:C1-on-compact-subsets}

Here we show that $p$ and $\Phi$ are continuously differentiable on compact subsets of $(q_0,1]$ using another local perturbation argument. 

\begin{lemma}
    \label{lem:sqrt-xy-concave}
    The function $f(x,y)=\sqrt{xy}$ is concave on $\bbR_{>0}^2$, with strict concavity on all lines except for those passing through the origin.
\end{lemma}

\begin{proof}
    Given $x_0,y_0,x_1,y_1>0$ with $(x_0,y_0)\neq (x_1,y_1)$ and $c\in (0,1)$, we have
    \begin{align*}
        (x_0 y_1 - x_1 y_0)^2
        &\geq 
        0
        \\
        \implies 
        x_0^2 y_1^2 + x_1^2 y_0^2 + 2x_0 x_1 y_0 y_1
        &\geq 
        4 x_0 x_1 y_0 y_1
        \\
        \implies 
        (x_0 y_1 + x_1 y_0)
        &\geq 
        2\sqrt{x_0 x_1 y_0 y_1}
        \\
        \implies 
        c(1-c)\cdot (x_0 y_1 + x_1 y_0)
        &\geq 
        2c(1-c)\sqrt{x_0 x_1 y_0 y_1}
        \\
        \implies 
        c^2 x_0 y_0 + (1-c)^2 x_0 y_0) + c(1-c)\cdot (x_0 y_1 + x_1 y_0)
        &\geq 
        c^2 x_0 y_0 + (1-c)^2 x_0 y_0) + 
        2c(1-c)\sqrt{x_0 x_1 y_0 y_1}
        \\
        \implies 
        \sqrt{(cx_0+(1-c)x_1)(cy_0+(1-c)y_1) }
        &\geq 
        c\sqrt{x_0y_0}+(1-c)\sqrt{x_1y_1}.
    \end{align*}
    Moreover equality holds if and only if it holds in the first step.
\end{proof}

\begin{lemma}
    \label{lem:derivatives-continuous-apriori}
    Both $p$ and $\Phi$ are continuously differentiable on compact subsets of $(q_0,1]$.
\end{lemma}

\begin{proof}


We assume that $q_0<1$ (else there is nothing to prove), and recall Lemma~\ref{lem:p-positive} throughout. Admissibility implies that $\Phi$ is uniformly Lipschitz, and Lemma~\ref{lem:p-AC} shows that $p$ is uniformly Lipschitz on compact subsets of $(q_0,1)$. Hence both $p'(x)$ and $\Phi'_s$ exist as non-negative, integrable functions which are uniformly bounded away from $q_0$.  


By an elementary result of \cite{zaanen1986continuity}, if a measurable function $[q_0,1]\to\bbR$ does not agree with any continuous function on a full measure set, then it possesses a genuine point of discontinuity $q_*\in (q_0,1)$ such that $F$ cannot be made continuous at $q_*$ even by modification on a measure zero set. We fix such a point $q_*$ for sake of contradiction. By definition, this means that for some $\eta>0$ depending only on $(p,\Phi,q_*)$ and for arbitrarily small $\eps>0$, there exist measurable sets $I,J\subseteq (q_*-\eps,q_*+\eps)$ and $a\in \bbR$ such that:
\begin{equation}
\label{eq:eta-discontinuous-f}
\begin{aligned}
    |I|&= \eps_1>0,\\
    |J|&= \eps_1>0,\\
    f(q)&\geq a+\eta,\quad \forall q\in I,\\
    f(q)&\leq a-\eta,\quad \forall q\in J.
\end{aligned}
\end{equation}
Here $f(q)=p'(q)$ or $f(q)=\Phi'_s(q)$ for some $s\in\sS$.

Let $\gamma_I:[0,\eps_1]\to I$ and $\gamma_J:[0,\eps_1]\to J$ be increasing, measure-preserving bijections (and note that their inverse functions are also measurable). For convenience we set $q_{I,x}=\gamma_I(x)$ and $q_{J,x}=\gamma_J(x)$. We construct perturbations $\tilde p,\tilde\Phi$ of $p$ and $\Phi$ by averaging derivatives on $q_{I,x}$ and $q_{J,x}$:
\begin{align*}
   \tilde p'(q_{I,x})&= 
   \tilde p'(q_{J,x})
   = \frac{p'(q_{I,x})+ p'(q_{J,x})}{2}
    \,;\\
    \tilde p'(q) &= p'(q),\quad q\notin I\cup J
    \,;\\
    \tilde \Phi_s'(q_{I,x})&= 
   \tilde \Phi_s'(q_{J,x})
   = \frac{\Phi_s'(q_{I,x})+ \Phi_s'(q_{J,x})}{2}
    \,;\\
    \tilde \Phi_s'(q) &= \Phi_s'(q),\quad q\notin I\cup J.
\end{align*}
We claim that for fixed $q_*,\eta$ and sufficiently small $\eps>0$, we have
\begin{equation}
\label{eq:derivative-continuous}
    \bbA(\tilde p,\tilde\Phi;q_0)
    >
    \bbA(p,\Phi;q_0).
\end{equation}
This contradicts maximality of $(p,\Phi)$ and thus implies the desired continuity of $(p',\Phi')$.  



To begin proving \eqref{eq:derivative-continuous}, recall from Lemma~\ref{lem:p-AC} that $p'$ is uniformly bounded away from $q_0$, hence on $(q_*-\eps,q_*+\eps)$. Moreover $\Phi'$ is uniformly bounded by definition. It follows that for all $s\in\sS$ and $q\in (q_*-\eps,q_*+\eps)$,
\begin{equation}
\label{eq:almost-constant}
\begin{aligned}
    |p(q)-\tilde p(q)|&\leq O(\eps_1),\\
    |p(q)-p(q_*)|&\leq O(\eps),
    \\
    |\Phi_s(q)-\tilde\Phi_{s}(q)|&\leq O(\eps_1),\\
    |\xi^s(\Phi(q))-\xi^s(\tilde\Phi(q))|&\leq O(\eps_1),
    \\
    |\Phi_s(q)-\Phi_s(q_*)|&\leq O(\eps),
    \\
    |\xi^s(\Phi(q))-\xi^s(\Phi(q_*))|&\leq O(\eps).
\end{aligned}
\end{equation}
These estimates will let us treat the above functions as almost constant while proving \eqref{eq:derivative-continuous}, so we can focus on the more important changes in their derivatives. First for $q\notin [q_*-\eps,q_*+\eps]$, we have $p(q)=\tilde p(q)$ and $\Phi(q)=\tilde\Phi(q)$, so it suffices to analyze the discrepancy within $q\in [q_*-\eps,q_*+\eps]$. Next, the estimates \eqref{eq:almost-constant} together with the fact that $\Phi_s'$ is uniformly bounded below (by Lemma~\ref{lem:Phi-inc}) imply that
\begin{equation}
\label{eq:good-outside-IJ}
    \lt|
    \sqrt{\Phi_s'(q)(p\times \xi^s\circ\Phi)'(q)}
    -
    \sqrt{\tilde \Phi_{s}'(q)(\tilde p\times \xi^s\circ\tilde \Phi)'(q)}
    \rt|
    \leq
    O(\eps_1),
    \quad
    \forall~
    q\in [q_*-\eps,q_*+\eps]\backslash (I\cup J).
\end{equation}
Integrating, we obtain
\begin{equation}
\label{eq:good-outside-IJ-2}
    \int_{q\in [q_*-\eps,q_*+\eps]\backslash (I\cup J)}\lt|
    \sqrt{\Phi_s'(q)(p\times \xi^s\circ\Phi)'(q)}
    -
    \sqrt{\tilde\Phi_{s}'(q)(\tilde p\times \xi^s\circ\tilde \Phi)'(q)}
    \rt|
    ~\de q
    \leq
    O(\eps_1 \eps).
\end{equation}
Next we fix $x\in [0,\eps_1]$ and analyze the joint effect of the pertubation at the pair of points $q_{I,x}$ and $q_{J,x}$. This is given by
\begin{equation}
\label{eq:joint-effect}
\begin{aligned}
    &\sqrt{\tilde\Phi_{s}'(q_{I,x})(\tilde p\times \xi^s\circ\tilde\Phi)'(q_{I,x})}
    -
    \sqrt{\Phi_s'(q_{I,x})(p\times \xi^s\circ\Phi)'(q_{I,x})}
    \\
    \quad\quad
    &+
    \sqrt{\tilde\Phi_{s}'(q_{J,x})(\tilde p\times \xi^s\circ\tilde\Phi)'(q_{J,x})}
    -
    \sqrt{\Phi_s'(q_{J,x})(p\times \xi^s\circ\Phi)'(q_{J,x})}
    .
\end{aligned}
\end{equation}
Recalling again \eqref{eq:almost-constant}, we have 
\begin{equation}
\label{eq:recalling-again-blah}
\begin{aligned}
    (\tilde p\times \xi^s\circ\tilde\Phi)'(q_{I,x})
    &=
    \tilde p (q_{I,x}) \sum_{s'\in \sS}\partial_{x_{s'}}\xi^s(\tilde \Phi(q_{I,x}))\cdot \tilde\Phi_{s'}'(q_{I,x})
    +
    \tilde p'(q_{I,x})\cdot \xi^s(\tilde\Phi(q_{I,x}))
    \\
    &=
    p (q_*) \sum_{s'\in \sS}\partial_{x_{s'}}\xi^s(\Phi(q_*))\cdot \tilde\Phi_{s'}'(q_{I,x})
    +
    p'(q_{I,x})\cdot \xi^s(\Phi(q_*))
    \pm O(\eps).
\end{aligned}
\end{equation}
Similarly to \eqref{eq:good-outside-IJ}, we now control the first two terms of \eqref{eq:joint-effect}:
\begin{equation}
\label{eq:approx-Phi-diff}
\begin{aligned}
    &
    \sqrt{\tilde\Phi_{s}'(q_{I,x})(\tilde p\times \xi^s\circ\tilde\Phi)'(q_{I,x})}
    -
    \sqrt{\Phi_s'(q_{I,x})(p\times \xi^s\circ\Phi)'(q_{I,x})}
    +
    O(\eps)
    \\
    &\stackrel{\eqref{eq:recalling-again-blah}}{\geq}
    \sqrt{\tilde\Phi_{s}'(q_{I,x})\cdot \lt(p (q_*) \sum_{s'\in \sS}\partial_{x_{s'}}\xi^s(\Phi(q_*))\cdot \tilde\Phi_{s'}'(q_{I,x})
    +
    \tilde p'(q_{I,x})\cdot \xi^s(\Phi(q_*))\rt)}
    \\
    &\quad -
    \sqrt{\Phi_{s}'(q_{I,x})\cdot \lt(p(q_*) \sum_{s'\in \sS}\partial_{x_{s'}}\xi^s(\Phi(q_*))\cdot \Phi'_{s'}(q_{I,x})
    +
    p'(q_{I,x})\cdot \xi^s(\Phi(q_*))\rt)}
\end{aligned}
\end{equation}
and analogously for $J$ instead of $I$.

It remains to lower-bound the right hand side of \eqref{eq:approx-Phi-diff}. We break into cases depending on whether $\Phi'$ is continuous (if so, then $p'$ must be discontinuous). In both cases, the idea is to argue that the concavity of the square root function yields an increase in the value of $\bbA$.

\paragraph{Case $1$: $\Phi'$ is continuous at $q_*$}


In this case $p'$ is discontinuous, and \eqref{eq:eta-discontinuous-f} applies with $f=p$. We estimate the right-hand side of \eqref{eq:approx-Phi-diff}: as $|\Phi_s'(q)-\tilde \Phi_{s}'(q')|\leq o_{\eps\to 0}(1)$ uniformly in $q,q'\in (q_*-\eps,q_*+\eps)$ by definition, 
\begin{align}
    \nonumber
    &
    \sqrt{\tilde\Phi_{s}'(q_{I,x})\cdot \lt(p (q_*) \sum_{s'\in \sS}\partial_{x_{s'}}\xi^s(\Phi(q_*))\cdot \tilde\Phi_{s'}'(q_{I,x})
    +
    \tilde p'(q_{I,x})\cdot \xi^s(\Phi(q_*))\rt)}
    \\
    \nonumber
    &\quad -
    \sqrt{\Phi_{s}'(q_{I,x})\cdot \lt(p(q_*) \sum_{s'\in \sS}\partial_{x_{s'}}\xi^s(\Phi(q_*))\cdot \Phi'_{s'}(q_{I,x})
    +
    p'(q_{I,x})\cdot \xi^s(\Phi(q_*))\rt)}
    \\
    \nonumber
    &=
    \sqrt{\Phi_{s}'(q_*)\cdot \lt(p (q_*) \sum_{s'\in \sS}\partial_{x_{s'}}\xi^s(\Phi(q_*))\cdot \Phi'_{s'}(q_*)
    +
    \tilde p'(q_{I,x})\cdot \xi^s(\Phi(q_*))\rt)}
    \\
    \label{eq:I-big-long-equation}
    &\quad -
    \sqrt{\Phi_{s}'(q_*)\cdot \lt(p(q_*) \sum_{s'\in \sS}\partial_{x_{s'}}\xi^s(\Phi(q_*))\cdot \Phi'_{s'}(q_*)
    +
    p'(q_{I,x})\cdot \xi^s(\Phi(q_*))\rt)} \pm o_{\eps\to 0}(1)
    .
\end{align}
We analyze the last term, combined with the analogous expression for $J$, using the strict concavity in Lemma~\ref{lem:sqrt-xy-concave} of $x\mapsto \sqrt{x}$ together with \eqref{eq:eta-discontinuous-f} applied to $p$. We find that
\begin{equation}
\label{eq:big-gain-case-1}
\begin{aligned}
    &\sqrt{\Phi_{s}'(q_*)\cdot \lt(p (q_*) \sum_{s'\in \sS}\partial_{x_{s'}}\xi^s(\Phi(q_*))\cdot \Phi'_{s'}(q_*)
    +
    \tilde p'(q_{I,x})\cdot \xi^s(\Phi(q_*))\rt)}
    \\
    &\quad -
    \sqrt{\Phi_{s}'(q_*)\cdot \lt(p(q_*) \sum_{s'\in \sS}\partial_{x_{s'}}\xi^s(\Phi(q_*))\cdot \Phi'_{s'}(q_*)
    +
    p'(q_{I,x})\cdot \xi^s(\Phi(q_*))\rt)}
    \\
    &\quad
    +
    \sqrt{\Phi_{s}'(q_*)\cdot \lt(p (q_*) \sum_{s'\in \sS}\partial_{x_{s'}}\xi^s(\Phi(q_*))\cdot \Phi'_{s'}(q_*)
    +
    \tilde p'(q_{J,x})\cdot \xi^s(\Phi(q_*))\rt)}
    \\
    &\quad -
    \sqrt{\Phi_{s}'(q_*)\cdot \lt(p(q_*) \sum_{s'\in \sS}\partial_{x_{s'}}\xi^s(\Phi(q_*))\cdot \Phi'_{s'}(q_*)
    +
    p'(q_{J,x})\cdot \xi^s(\Phi(q_*))\rt)}
    \\
    &\geq
    c(\eta).
\end{aligned}
\end{equation}
Indeed, all quantities except $p'(\cdot)$ and $\tilde p'(\cdot)$ are the same in the four expressions and are bounded away from $0$ and infinity. Furthermore all other expressions differ by $O(\eps_1)$ thanks to \eqref{eq:almost-constant}, which is small compared to the discrepancy $\eta$ between the values of $p'$ and $\wt p$'. Hence for $\eta$ fixed and $\eps$ small enough, they are bounded away from the equality cases of Lemma~\ref{lem:sqrt-xy-concave}.




Combining \eqref{eq:approx-Phi-diff}, \eqref{eq:I-big-long-equation}, and \eqref{eq:big-gain-case-1} implies that for each $x\in [0,\eps_1]$ and small enough $\eps$,
\begin{align*}
    &\sqrt{\tilde\Phi_{s}'(q_{I,x})(\tilde p\times \xi^s\circ\tilde\Phi)'(q_{I,x})}
    -
    \sqrt{\Phi_s'(q_{I,x})(p\times \xi^s\circ\Phi)'(q_{I,x})}
    \\
    \quad\quad
    &+
    \sqrt{\tilde\Phi_{s}'(q_{J,x})(\tilde p\times \xi^s\circ\tilde\Phi)'(q_{J,x})}
    -
    \sqrt{\Phi_s'(q_{J,x})(p\times \xi^s\circ\Phi)'(q_{J,x})}
    \\
    &\geq c(\eta)-o_{\eps\to 0}(1)
    \\
    &\geq 
    c(\eta)/2.
\end{align*}
Integrating over $x\in [0,\eps_1]$ and combining with \eqref{eq:good-outside-IJ-2}, we conclude that \eqref{eq:derivative-continuous} holds in Case $1$.


\paragraph{Case $2$: $\Phi'$ is discontinuous at $q_*$.} 
(Note that $p'$ might also be discontinuous.) 

Define for each $s\in\sS$ the function
\[
    F_s(A_1,\dots,A_r,B)
    =
    \sqrt{A_s\cdot \lt(p(q_*) \sum_{s'\in \sS}\partial_{x_{s'}}\xi^s(\Phi(q_*)) A_{s'}
    +
    B\xi^s(\Phi(q_*))\rt)}.
\]
Lemma~\ref{lem:sqrt-xy-concave} implies that each function $F_s$ is concave on $\bbR_{\geq 0}^{r+1}$, since both $A_s$ and $p(q_*) \sum_{s'\in \sS}\partial_{x_{s'}}\xi^s(\Phi(q_*)) A_{s'}+B\xi^s(\Phi(q_*))$ are linear functions of $(A_1,\dots,A_r,B)$. In particular, for each $(s,x)\in \sS\times [0,\eps_1]$ the function
\[
    f_{s,x}(t)
    \equiv
    F_s\lt(\frac{(1-t)\Phi_{1}'(q_{I,x})+t\Phi_{1}'(q_{J,x})}{2},\dots,
    \frac{(1-t)\Phi_{r}'(q_{I,x})+t\Phi_{r}'(q_{J,x})}{2},
    \frac{(1-t)p'(q_{I,x})+tp'(q_{J,x})}{2}\rt)
\]
is concave for $t\in [0,1]$. Recalling the definitions of $\tilde p$ and $\tilde \Phi$, we expand the inequality $2f_{s,x}(1/2)\geq f_{s,x}(0)+f_{s,x}(1)$ to obtain
\begin{equation}
\label{eq:concave-in-each-s}
\begin{aligned}
    &
    \sqrt{\tilde\Phi_{s}'(q_{I,x})\cdot \lt(p (q_*) \sum_{s'\in \sS}\partial_{x_{s'}}\xi^s(\Phi(q_*))\cdot \tilde\Phi_{s'}'(q_{I,x})
    +
    \tilde p'(q_{I,x})\cdot \xi^s(\Phi(q_*))\rt)}
    \\
    &\quad -
    \sqrt{\Phi_{s}'(q_{I,x})\cdot \lt(p(q_*) \sum_{s'\in \sS}\partial_{x_{s'}}\xi^s(\Phi(q_*))\cdot \Phi'_{s'}(q_{I,x})
    +
    p'(q_{I,x})\cdot \xi^s(\Phi(q_*))\rt)}
    \\
    &\quad +
    \sqrt{\tilde\Phi_{s}'(q_{J,x})\cdot \lt(p (q_*) \sum_{s'\in \sS}\partial_{x_{s'}}\xi^s(\Phi(q_*))\cdot \tilde\Phi_{s'}'(q_{J,x})
    +
    \tilde p'(q_{J,x})\cdot \xi^s(\Phi(q_*))\rt)}
    \\
    &\quad -
    \sqrt{\Phi_{s}'(q_{J,x})\cdot \lt(p(q_*) \sum_{s'\in \sS}\partial_{x_{s'}}\xi^s(\Phi(q_*))\cdot \Phi'_{s'}(q_{J,x})
    +
    p'(q_{J,x})\cdot \xi^s(\Phi(q_*))\rt)}
    \\
    &\geq
    0.
\end{aligned}
\end{equation}
In light of \eqref{eq:approx-Phi-diff}, this means that perturbing $(p,\Phi)\to(\tilde p,\tilde \Phi)$ can only hurt the contribution from a given $s\in\sS$ by $O(\eps)$. To complete the proof we will show that the contribution from some $s\in\sS$ is positive and of a larger order. Which of these must occur will depend on the ratio $\frac{p'(q_{I,x})}{p'(q_{J,x})}$. 


We will get this contribution from either $s_{\max}$ or $s_{\min}$, defined now. For each $x\in[0,\eps_1]$, let
\begin{align*}
    s_{\max}(x)
    &=
    \argmax_{s\in\sS}
    \frac{\Phi_s'(q_{I,x})}{\Phi_s'(q_{J,x})}
    ,
    \\
    s_{\min}(x)
    &=
    \argmin_{s\in\sS}
    \frac{\Phi_s'(q_{I,x})}{\Phi_s'(q_{J,x})}
    .
\end{align*}
(Both are defined up to almost everywhere equivalence if ties are broken lexicographically.) Recall the functions $\Phi_s'(x)$ are uniformly bounded above and below. It follows from \eqref{eq:eta-discontinuous-f} that 
\begin{equation}
\label{eq:Phi-s-minmax}
    \frac{\Phi_{s_{\min}}'(q_{I,x})}{\Phi_{s_{\min}}'(q_{J,x})}
    \leq
    1-\eta'
    \leq
    1+\eta'
    \leq
    \frac{\Phi_{s_{\max}}'(q_{I,x})}{\Phi_{s_{\max}}'(q_{J,x})}
\end{equation}
for some $\eta'$ depending only on $(\eta,\xi,h)$. (Discontinuity of $\Phi'$ gives one side, and admissibility forces another $s\in\sS$ to change in the opposite direction.)


Without loss of generality, suppose that
\begin{equation}
\label{eq:p'-shrink-wlog}
    \frac{p'(q_{I,x})}{p'(q_{J,x})}\leq 1.
\end{equation}
In this case, the assumption \eqref{eq:p'-shrink-wlog} implies
\[
    \frac{
    p(q_*) \sum_{s'\in \sS}\partial_{x_{s'}}\xi^{s_{\max}}(\Phi(q_*))\cdot \Phi'_{s'}(q_{I,x})
    +
    p'(q_{I,x})\cdot \xi^{s_{\max}}(\Phi(q_*))
    }
    {
    p(q_*) \sum_{s'\in \sS}\partial_{x_{s'}}\xi^{s_{\max}}(\Phi(q_*))\cdot \Phi'_{s'}(q_{J,x})
    +
    p'(q_{J,x})\cdot \xi^{s_{\max}}(\Phi(q_*))
    }
    \leq
    \frac{\Phi_{s_{\max}}'(q_{I,x})}
    {\Phi_{s_{\max}}'(q_{J,x})}-\eta_1
\]
for a constant $\eta_1>0$ depending only on $(\eta,q_*,\xi,h)$. Since all quantities are bounded away from $0$ and infinity, applying a simple compactness argument to the equality case in Lemma~\ref{lem:sqrt-xy-concave} implies
\begin{equation}
\label{eq:f-concave-with-eta1-gain}
    2f_{s_{\max},x}(1/2)\geq f_{s_{\max},x}(0)+f_{s_{\max},x}(1)+c(\eta_1).
\end{equation}
Similarly if \eqref{eq:p'-shrink-wlog} does not hold, then we find \eqref{eq:f-concave-with-eta1-gain} with $s_{\min}$ in place of $s_{\max}$.


Combining the above with $\eps\ll \eta$, we find that for each $x\in[0,\eps_1]$,
\begin{align*}
    \sum_{s\in\sS} 2f_{s,x}(1/2)
    &\geq
    \sum_{s\in\sS} 
    \Big(f_{s,x}(0)+f_{s,x}(1)\Big)+c(\eta_1)/2.
\end{align*}
Integrating over $x$ and recalling \eqref{eq:good-outside-IJ-2} and \eqref{eq:approx-Phi-diff}, we conclude that \eqref{eq:derivative-continuous} also holds in Case $2$. This completes the proof.
\end{proof}

\begin{proof}[Proof of Proposition~\ref{prop:basic-regularity}]
    Follows from Lemmas~\ref{lem:p-AC}, \ref{lem:Phi-inc}, and \ref{lem:derivatives-continuous-apriori}.
    The upper bound on $\Phi'$ comes from admissibility \eqref{eq:admissible}, which implies that $\Phi'_s \le \lambda_s^{-1}$.
\end{proof}



\subsection{Type $\II$ Solutions}
\label{subsec:type-II-Lipschitz}

Here we show that the type $\II$ equation implicitly takes the form of a second order ordinary differential equation in which $\Phi''(q)$ is Lipschitz in $(\Phi(q),\Phi'(q))$. It follows that a unique type $\II$ solution exists given any first-order initial condition $(\Phi(q_1),\Phi'(q_1))$, and that the type $\II$ ODE is satisfied at \emph{all} points in $(q_1,1)$. We will often enforce the admissibility conditions
\begin{align}
\label{eq:derivative-admissible}
    \langle \vlam, \vec\Phi'(q)\rangle &=1,
    \\
\label{eq:second-derivative-admissible}
    \langle \vlam, \vec\Phi''(q)\rangle &=0.
\end{align}
In particular, we denote by $A_{\geq 0}$ the set of vectors $v\in \bbR_{\geq 0}^{\sS}$ satisfying $\langle \vlam, v \rangle=1$. The following important but rather lengthy Lemma~\ref{lem:type-II-Lipschitz} ensures that type $\II$ solutions are described by a Lipschitz ODE. In it, the value $q$ is actually irrelevant and just serves as a placeholder. Importantly there is no issue when $\Phi_s(q)$ or $\Phi_s'(q)$ is near zero, thanks to non-degeneracy.




\lemtypeIILipschitz*



\begin{proof}
Write $\Psi(q)$ for $\Psi_s(q)$, which is independent of $s\in\sS$ by assumption. We assume throughout that $\Phi(q)$ lies in a bounded set in writing $O(\cdot)$ and $\Omega(\cdot)$ expressions. Note that $\vec\Phi''(q)$ exists as an $L^1$ function for $q\in (q_1,1]$ since $\vec\Phi'$ is absolutely continuous. We write
\begin{equation}
\label{eq:expand-Psi-type-II}
\begin{aligned}
    2\Psi(q)
    &=
    \frac{2}{\Phi_s'(q)}
    \deriv{q}{\sqrt{\frac{\Phi'_s(q)}{(\xi^s\circ\Phi)'(q)}}}
    \\
    &=
    \sqrt{\frac{(\xi^s\circ\Phi)'(q)}{\Phi_s'(q)^3}}
    \deriv{q}{\frac{\Phi'_s(q)}{(\xi^s\circ\Phi)'(q)}}
    \\
    &=
    \sqrt{\frac{(\xi^s\circ\Phi)'(q)}{\Phi_s'(q)^3}}
    \cdot
    \frac{\Phi_s''(q)(\xi^s\circ \Phi)'(q) - \Phi_s'(q)(\xi^s\circ\Phi)''(q)}{(\xi^s\circ\Phi)'(q)^2}
    \\
    &=
    \frac{1}{\sqrt{\Phi_s'(q)^3 (\xi^s\circ\Phi)'(q)^3}}
    \cdot
    \lt(
    \Phi_s''(q)(\xi^s\circ \Phi)'(q) - \Phi_s'(q)(\xi^s\circ\Phi)''(q)
    \rt)\,.
\end{aligned}
\end{equation}
%
Moreover we have
\begin{align*}
    &\Phi_s''(q)(\xi^s\circ \Phi)'(q) - \Phi_s'(q)(\xi^s\circ\Phi)''(q)
    \\
    &=
    \Phi_s''(q)
    \sum_{s'\in\sS}\partial_{x_{s'}}\xi^s(\Phi(q)) \cdot \Phi'_{s'}(q) 
    \\
    &\quad
    - 
    \Phi_s'(q)
    \lt(
    \sum_{s'\in\sS}\partial_{x_{s'}}\xi^s(\Phi(q)) \cdot \Phi''_{s'}(q) 
    +
    \sum_{s',s''\in\sS} 
    \partial_{x_{s'}}\partial_{x_{s''}}
    \xi^s(\Phi(q)) \cdot \Phi'_{s'}(q) 
    \rt)
\end{align*}
Let 
\[
B_s(q)=\sum_{s'\in\sS}\partial_{x_{s'}}\xi^s(\Phi(q)) \cdot \Phi'_{s'}(q).
\]
Note that by non-degeneracy each $\partial_{x_{s'}}\xi^s(\Phi(q))$ is bounded away from $0$ and $\infty$ for all $\Phi(q)\in [0,1]^{\sS}$. Meanwhile $\sum_{s\in\sS}\lambda_s\Phi'_s(q)=1$. Thus for $\Phi'(q)$ obeying \eqref{eq:derivative-admissible}, each $B_s(q)$ is uniformly bounded away from $0$ and $\infty$.


Next let $M(q)\in\mathbb R^{\sS\times\sS}$ be a square matrix with entries
\[
    M(q)_{s,s'}
    =
    \frac{\Phi_s'(q) \cdot \partial_{x_{s'}}\xi^s(\Phi(q))}{B_s(q)}
    \,
\]
and let $I$ denote the identity $\sS\times\sS$ matrix. Then the above equations for all $s\in\sS$ can be expressed more succinctly as
\begin{equation}
\label{eq:type-II-Lipschitz}
    (M-I)\Phi''(q) 
    =
    -w_1(\Phi(q),\Phi'(q))-\Psi(q)\cdot w_2(\Phi(q),\Phi'(q))
\end{equation}
for Lipschitz functions $w_1,w_2:[0,1]^{2r}\to\mathbb R_{>0}^{r}$ given explicitly by
\begin{equation}
\label{eq:w1w2}
\begin{aligned}
    (w_1)_s
    &=
    \frac{
    \Phi_s'(q)\sum_{s',s''\in\sS} 
    \partial_{x_{s'}} \partial_{x_{s''}}
    \xi^s(\Phi(q)) \cdot \Phi'_{s'}(q)  
    }
    {B_s(q)}
    ;
    \\
    (w_2)_s
    &=
    \frac{2\sqrt{\Phi_s'(q)^3 (\xi^s\circ\Phi)'(q)^3}}{B_s(q)}.
\end{aligned}
\end{equation}
Since $B_s$ is bounded below, both $w_1$ and $w_2$ have uniformly bounded entries. Moreover $B$ and $w_1,w_2$ are uniformly Lipschitz in $(\Phi(q),\Phi'(q))$. Note also that $w_2$ is entry-wise non-negative.


As a first observation, observe that
\[
    (M-I)\Phi'(q)=0.
\]
Because $\Phi'(q)\succeq 0$ and $M$ has positive entries, this means $\Phi'(q)$ is the unique right Perron-Frobenius eigenvector of $M$, and thus $\rank(M-I)=r-1$. It follows that for given $(\Phi(q),\Phi'(q))$, a unique solution $(\Phi''(q),\Psi(q))$ to \eqref{eq:type-II-Lipschitz} exists so long as 
\begin{equation}
\label{eq:w2-notin-range-MI}
w_2\notin \range(M-I).
\end{equation}
In fact \eqref{eq:w2-notin-range-MI} is always true. To see this, note that $M$ has a left Perron-Frobenius eigenvector $v\in\mathbb R_{>0}^{r}$ with $v(M-I)=0$. Then if $w_2=(M-I)w$ for $w\in\bbR^{\sS}$, we find $\langle v,w_2\rangle=0$.
This is a contradiction: $\langle v,w_2\rangle>0$ since all entries are strictly positive in both vectors. We denote by $\Lambda(q)\in\bbR^\sS$ the value of $\Phi''(q)$ in the aforementioned unique solution.




Our primary aim is now to show that $\Lambda(q)$ is a Lipschitz function of $(\Phi(q),\Phi'(q))\in\bbR^{\sS}\times A_{\geq 0}$. We would like to apply Perron-Frobenius arguments to $M$, but the fact that $M_{s,s'}\asymp \Phi_s'(q)$ may be very small poses an issue. To rectify this, we define $\wt M(q)$ with entries
\begin{equation}
\label{eq:wtM}
    \wt M(q)_{s,s'}
    =
    \frac{\Phi'_{s'}(q)\partial_{x_{s'}}\xi^s(\Phi(q))}{B_s(q)}
    \,
    .
\end{equation}
Then defining the diagonal $\sS\times\sS$ matrix $D(\Phi'(q))$ with entries
\[
    D(\Phi'(q))_{s,s}=\Phi_s'(q)
\]
we have
\[
\wt M(q)=D(\Phi'(q))^{-1}\, M \, D(\Phi'(q)).
\]
The key property obeyed by $\wt M$ but not $M$ is that for any $v\in \bbR_{>0}^{\sS}$, the entries of $\wt M v$ are of the same order. Namely, all ratios $\frac{(\wt M v)_s}{(\wt M v)_{s'}}$ are uniformly bounded because the ratios $M_{s,s'}/M_{s'',s'}$ are uniformly bounded.
In particular Lemma~\ref{lem:Phi'=0isOK} and hence Lemma~\ref{lem:PF-closed-range} (see below) apply to $\wt M$.

Note that $\wt M$ has Perron-Frobenius eigenvector $\vone$ and $\wt M$ is Lipschitz in $(\Phi(q),\Phi'(q))$. We set 
\begin{equation}
\label{eq:wtV-defn}
\begin{aligned}
    \wt V(q)&=D(\Phi'(q))^{-1}\Lambda(q),\quad \text{i.e.}~ \wt V(q)_s=\frac{\Lambda_s(q)}{\Phi_s'(q)};
    \\
    V(q)_s&=\wt V(q)_s - \frac{\sum_{s'\in\sS} \wt V(q)_{s'}}{r}.
\end{aligned}
\end{equation}
By construction, $\sum_s V(q)_s=0$. Moreover 
\begin{equation}
\label{eq:V-wtV-behave-same}
(\wt M(q)-I)V(q)=(\wt M(q)-I)\wt V(q)
\end{equation}
since $V(q)-\wt V(q)$ is proportional to $\vone$. 

\paragraph{A priori estimate on $\Lambda(q)$}
We now prove \eqref{eq:Phi-stays-increasing}, which will also serve as a useful intermediate step. Note first that $w_1$ satisfies $|(w_1)_s|= O(\Phi'_s(q))$ (recall that $B_s$ is bounded below), while all entries of $w_2$ are non-negative. Therefore the entries of $w_1(q)+\Psi(q) w_2(q)$
are bounded either above or below by $O(\Phi'_s(q))$. Furthermore by definition, 
\begin{align*}
    -w_1(q)-\Psi(q) w_2(q)
    &=
    (M(q)-I)\Lambda(q)
    \\
    &=
    D(\Phi'(q))(\wt M(q)-I)\wt V
    \\
    &\stackrel{\eqref{eq:V-wtV-behave-same}}{=}
    D(\Phi'(q))(\wt M(q)-I)V.
\end{align*}
We conclude that
\[
    \min\lt(\|((\wt M(q)-I) V)_+\|_1,\|((\wt M(q)-I) V)_-\|_1\rt) \leq O(1).
\]
Lemma~\ref{lem:PF-closed-range} below now implies that 
\begin{equation}
\label{eq:V-bounded}
\|V(q)\|_1\leq O(1).
\end{equation}



Note that $\langle \wt V,\lambda\odot\Phi'(q)\rangle=0$ by \eqref{eq:second-derivative-admissible} and \eqref{eq:wtV-defn}. The second part of the latter also implies $V(q)-\wt V(q)$ is proportional to $\vone$, and so
\begin{equation}
\label{eq:V-wtV-1}
\begin{aligned}
    \lt|(V(q)-\wt V(q))_s\rt|
    &=
    \lt|
    \langle 
    V(q)-\wt V(q)
    ,
    \lambda\odot\Phi'(q)
    \rangle
    \rt|
    \\
    &=
    \lt|
    \langle 
    V(q)
    ,
    \lambda\odot\Phi'(q)
    \rangle
    \rt|
    \\
    &\stackrel{\eqref{eq:V-bounded}}{\leq} O(1)
\end{aligned}
\end{equation}
Using again \eqref{eq:V-bounded} and \eqref{eq:wtV-defn} we find that $\|\wt V(q)\|_1\leq O(1)$ as well. Finally since $\Lambda(q)=\wt V(q)\odot \Phi'(q)$, we get \eqref{eq:Phi-stays-increasing} as desired. 


\paragraph{Controlling $\Psi$}
We take a second detour to show that $\Psi(q)$ is bounded and Lipschitz.
Using that $\|w_1\|_1\leq O(1)$ and $\|w_2\|_1\geq \Omega(1)$ in the first step below, we find
\begin{align*}
    \Omega(|\Psi(q)|)-O(1)
    &\leq 
    \|w_1(q)+\Psi(q) w_2(q)\|_1
    \\
    &=
    \|(M(q)-I)\Lambda(q)\|_1
    \\
    &\stackrel{\eqref{eq:type-II-Lipschitz}}{\leq}
    O(1).
\end{align*}
The just-proved estimate \eqref{eq:Phi-stays-increasing} implies the weaker bound $\|\Lambda(q)\|_1\leq O(1)$, which was used in the last step.

We conclude that $\Psi(q)$ is uniformly bounded:
\begin{equation}
\label{eq:Psi-bounded}
    |\Psi(q)|\leq O(1).
\end{equation}
Next we show that $\Psi(q)$ is Lipschitz in $(\Phi(q),\Phi'(q))$. We begin by writing
\begin{align*}
    (M(q)-I)\Lambda(q)
    -
    (M(q')-I)\Lambda(q')
    &=
    w_1(q')-w_1(q)
    +
    \Psi(q')w_2(q')
    -
    \Psi(q)w_2(q)
    \\
    &=
    w_1(q')-w_1(q)
    +
    \Psi(q')\big(w_2(q')-w_2(q)\big)
    +
    \big(\Psi(q')-\Psi(q)\big)w_2(q)
    \\
    &=
    O\big(\|\Phi(q)-\Phi(q')\|+\|\Phi'(q)-\Phi'(q')\|\big)
    +
    \big(\Psi(q')-\Psi(q)\big)w_2(q)
    .
\end{align*}
(Note that the latter $O(\cdot)$ notation hides a vector in $\bbR^r$.) We will rely on the fact that $w_2(q)$ is entrywise positive and $\|w_2(q)\|\geq \Omega(1)$. To analyze the left-hand side above, we write
\begin{align*}
    (M(q)-I)\Lambda(q)
    -
    (M(q')-I)\Lambda(q')
    &=
    (M(q)-M(q'))\Lambda(q)
    +
    (M(q')-I)\big(\Lambda(q)-\Lambda(q')\big)
    \\
    &\leq
    O\big(\|\Phi(q)-\Phi(q')\|+\|\Phi'(q)-\Phi'(q')\|\big)
    +
    (M(q')-I)\big(\Lambda(q)-\Lambda(q')\big)
    .
\end{align*}
The latter step holds since $M(q)$ is Lipschitz in $(\Phi(q),\Phi(q'))$ and $\|\Lambda(q)\|_1\leq O(1)$ from \eqref{eq:Phi-stays-increasing}. Now, let $v$ be the left Perron-Frobenius eigenvector of $M(q')$, so $v(M(q')-I)=0$, normalized so that $v\succeq 0$ and $\|v\|_1=1.$ Combining the previous displays implies that 
\[
    (\Psi(q')-\Psi(q))\cdot \langle v, w_2(q)\rangle 
    =
    O\big(\|\Phi(q)-\Phi(q')\|+\|\Phi'(q)-\Phi'(q')\|\big).
\]
Finally we show that $\langle v,w_2(q)\rangle$ is bounded away from $0$. Indeed both vectors are entrywise positive, and $\|w_2(q)\|_1\geq \Omega(1)$ while $\min_s v_s\geq \Omega(1)$. The latter statement holds for similar reasons to the right eigenvector properties of $\wt M$ explained above: for \emph{any} $v\in\bbR_{>0}^{\sS}$, the ratios $\frac{(vM)_{s}}{(vM)_{s'}}$ are uniformly bounded, and this ratio is simply $v_s/v_{s'}$ when $v$ is the left Perron-Frobenius eigenvector. We conclude that
\begin{equation}
\label{eq:Psi-Lip}
    |\Psi(q)-\Psi(q')|\leq O\big(\|\Phi(q)-\Phi(q')\|+\|\Phi'(q)-\Phi'(q')\|\big)
\end{equation}
which ends this second detour.


\paragraph{Finishing the Proof}
Having established \eqref{eq:Phi-stays-increasing} and \eqref{eq:Psi-bounded}, we return to showing that $\Lambda(q)$ is Lipschitz in $(\Phi(q),\Phi'(q))$. Fix a different pair 
\[
(\Phi(q'),\Phi'(q'))\neq (\Phi(q),\Phi'(q)).
\]
Accordingly define $w_1(q'),w_2(q'),M(q'),V(q')$ and so on using $(\Phi(q'),\Phi'(q'))$. (Since we don't require admissibility but only its differential version \eqref{eq:derivative-admissible}, there is no loss of generality here; $q'$ like $q$ is just a place-holder variable so e.g. $\Phi(q)=\Phi(q')$ is possible.)


Then Lemma~\ref{lem:PF-closed-range} implies:
\begin{equation}
\label{eq:type-II-reverse-Lipschitz}
    \|(\wt M(q)-I)V(q)-(\wt M(q)-I)V(q')\|_1 
    \geq
    \Omega\big(\|V(q)-V(q')\|_1\big).
\end{equation}
Using the reverse triangle inequality in the first step, we find the lower bound
%
\begin{align*}
    \|(\wt M(q)-I) V(q)
    -
    (\wt M(q')-I) V(q')\|_1
    &\geq
    \|(\wt M(q)-I) V(q)
    -
    (\wt M(q)-I) V(q')\|_1
    \\
    &\quad\quad
    -
    \|(\wt M(q)-I) V(q')
    -
    (\wt M(q')-I) V(q')\|_1
    \\
    &\stackrel{\eqref{eq:type-II-reverse-Lipschitz}}{\geq}
    \Omega\big(\| V(q)- V(q')\|_1\big)
    -
    O\big(\|\wt M(q)- \wt M(q')\|_1\big)
    \\
    &\geq
    \Omega\big(\| V(q)- V(q')\|_1\big)
    -
    O\big(\|\Phi(q)-\Phi(q')\|_1 + \|\Phi'(q)-\Phi'(q')\|_1\big).
\end{align*}
By \eqref{eq:type-II-Lipschitz}, \eqref{eq:w1w2}, \eqref{eq:Psi-bounded} \eqref{eq:Psi-Lip}, and the simple estimate $\max\big(|w_1(q)_s|,|w_2(q)_s|\big)\leq O(\Phi'_s(q))$, the left-hand side above is upper bounded by
\begin{align*}
   &\|(\wt M(q)-I) V(q)
    -
    (\wt M(q')-I) V(q')\|_1
    \\
    &=
    \|(\wt M(q)-I) \wt V(q)
    -
    (\wt M(q')-I) \wt V(q')\|_1  
    \\
    &=\Big\|
    D(\Phi'(q))^{-1}
    \Big(
    (M(q)-I) \Lambda(q)
    \Big)
    -
    D(\Phi'(q'))^{-1}
    \Big(
    (M(q')-I) \Lambda(q')\Big)
    \Big\|_1
    \\
    &=
    \Big\|
    D(\Phi'(q))^{-1}
    \Big(
    w_1(q)+\Psi(q)w_2(q)
    \Big)
    -
    D(\Phi'(q'))^{-1}
    \Big(
    w_1(q')+\Psi(q')w_2(q')
    \Big)
    \Big\|_1
    \\
    &\leq
    O\big(\|\Phi(q)-\Phi(q')\|_1 + \|\Phi'(q)-\Phi'(q')\|_1\big).
\end{align*}
Rearranging the previous two displays implies that
\begin{equation}
\label{eq:V-initial-bound}
    \|V(q)-V(q')\|_1
    \leq
    O\big(\|\Phi(q)-\Phi(q')\|_1 + \|\Phi'(q)-\Phi'(q')\|_1\big).
\end{equation}
It remains to unwind the transformations to conclude the same for $\Lambda$. Mimicking \eqref{eq:V-wtV-1} in the first step, 
\begin{align*}
    \lt|\big(V_s(q)-V_s(q')\big)-\big(\wt V_s(q)-\wt V_s(q')\big)\rt|
    &=
    \lt|\sum_s \lambda_s \big(\Phi_s'(q) V(q)-\Phi_s'(q')V(q')\big)\rt| 
    \\
    &\leq
    O\big(\|\Phi'(q)\|\cdot \|V(q)-V(q')\| \big)
    +
    O\big(\|\Phi'(q)-\Phi'(q')\|\cdot \|V(q')\| \big)
    \\
    &\stackrel{\eqref{eq:V-initial-bound},\eqref{eq:V-bounded}}{\leq}
    O\big(\|\Phi(q)-\Phi(q')\|_1 + \|\Phi'(q)-\Phi'(q')\|_1\big)
    \\
    &\quad\quad
    +
    O\big(\|\Phi'(q)-\Phi'(q')\big)
    .
\end{align*}
Combining the previous two displays, we conclude that 
\begin{align*}
    \|\wt V(q)-\wt V(q')\|_1
    &\leq
    \|V(q)-V(q')\|_1
    +
    \|(V(q)-V(q'))-(\wt V(q)-\wt V(q'))\|
    \\
    &\leq 
    O\big(\|\Phi(q)-\Phi(q')\|_1 + \|\Phi'(q)-\Phi'(q')\|_1\big).
\end{align*}
Finally since $\Lambda(q)=\wt V(q) \odot \Phi'(q)$ and $\|\wt V(q')\|_1, \|\Phi'(q)\|_1\leq O(1)$, we obtain the desired:
\begin{align*}
    \|\Lambda(q)-\Lambda(q')\|_1
    &\leq
    O\big(\|\wt V(q)-\wt V(q')\|_1 \cdot \|\Phi'(q)\|_1\big)
    +
    O\big(\|\wt V(q')\|_1 \cdot \|\Phi'(q)-\Phi'(q')\|_1\big)
    \\
    &\leq
    O\big(\|\wt V(q)-\wt V(q')\|_1 + \|\Phi'(q)-\Phi'(q')\|_1\big)
    \\
    &\leq
    O\big(\|\Phi(q)-\Phi(q')\|_1 + \|\Phi'(q)-\Phi'(q')\|_1\big).
\end{align*}
This concludes the proof.
\end{proof}




\begin{lemma}
\label{lem:PF-closed-range}
Let $\cM\subseteq \bbR_{\geq 0}^{\sS\times\sS}$ be a compact set of entry-wise non-negative matrices with unique Perron-Frobenius eigenvector $\vone$ and associated eigenvalue $1$.

Then for all  $v\in\bbR^{\sS}$ with $\sum_{s\in\sS} v_s=0$, we have
\begin{align*}
    \|((M-I)v)_+\|_1 &\geq \Omega_{\cM,r}(\|v\|_1),
    \\
    \|((M-I)v)_-\|_1 &\geq \Omega_{\cM,r}(\|v\|_1).
\end{align*}
\end{lemma}


\begin{proof}
The two statements are equivalent under negation so we assume the first is false and derive a contradiction. If it is false, by taking a convergent sequence of approximate counterexamples $(M^i,v^i)\to (\hM,\hv)$ with $M^i\in\cM$ and $\|v^i\|_1=1$, we have:
\begin{enumerate}
    \item $\hM\in \cM$.
    \item $\hM$ has Perron-Frobenius eigenvector $\vone$ and eigenvalue $1$. 
    \item $\sum_{s\in\sS}\hv_s= 0$.
    \item $\|\hv\|_1=1$.
    \item $\hM\hv\preceq \hv$ (since $((\hM-I)\hv)_+=0$).
\end{enumerate}
Since $\hM$ has simple Perron-Frobenius eigenvalue $1$, for $\hM\hv\preceq \hv$ to hold we must actually have $\hM \hv=\hv$. Therefore $\hv=\vone/r$ is a multiple of the right Perron-Frobenius eigenvector, contradicting $\sum_{s\in\sS}\hv_s= 0$.
\end{proof}


\begin{lemma}
\label{lem:Phi'=0isOK}
For $C>0$, let $\cM_C\subseteq \bbR_{\geq 0}^{\sS\times\sS}$ consist of all matrices $M$ such that:
\begin{enumerate}
    \item $M_{s,s'}\in [0,C]$ for all $s,s'\in\sS$.
    \item $M_{s,s'}\leq CM_{s'',s'}$ for all $s,s',s''\in\sS$.
    \item $M\vone=\vone$.
    \item $\sum_{s,s'\in\sS}M_{s,s'}\geq 1/C$.
\end{enumerate}
Then $\cM=\cM_C$ satisfies the conditions of Lemma~\ref{lem:PF-closed-range}.
\end{lemma}

\begin{proof}
    The only thing to show is that $\vone$ is the \textbf{unique} right Perron-Frobenius eigenvector associated to the eigenvalue $1$ of any $M\in\cM_C$, even though $M$ may include zero entries. Thus, suppose that $w\in\bbR^{\sS}$ satisfies $Mw=w$; we will show that $w$ has all equal entries. Let $S'\subseteq \sS$ be the non-empty set of $s'$ such that $M_{s,s'}>0$ (which does not depend on $s$ by definition of $\cM_C$). Then letting $M'$ and $w'$ be the $S'\times S'$ and $S'$-dimensional restrictions of $M$ and $w$, we have $M'w'=w'$. Since $M'$ has strictly positive entries, we conclude that $w'$ has all entries proportional. Hence for some $a\geq 0$, we have $w_s=a$ for all $s\in S'$. By definition of $S'$ we obtain $w=Mw=Ma^{\sS}=a^{\sS}$. This concludes the proof.
\end{proof}





% \section{Asymmetric Maximizers for Symmetric $\xi$ }
\label{sec:non-unique}

Let 
\[
    \xi(x,y)=x^4+y^4+24xy.
\]
This function is symmetric in $(x,y)$. We will show that the natural minimizer is not symmetric via a 2nd derivative expansion.


We wish to compute $\deriv{\delta}G$. Here we fix $\chi\in C_c^{\infty}((0,1);\mathbb R)$ (hence $\chi(0)=\chi(1)=0$) and set
\[
    \Phi_1^{\delta\chi}(q)
    =
    \Phi^*_1(q) 
    +
    \delta
    \chi(q)
\]
so that
\[
    (\Phi_1^{\delta\chi})'(q)
    =
    (\Phi^*_1)'(q) + \delta \chi'(q)
\]
for all $q\in [0,1]$. As before let $\Phi_s^{\delta\chi}=\Phi_s^*$ for $s\neq 1$. Similarly to before we compute:
\begin{align*}
    \deriv{\delta} (\Phi_1^{\delta\chi})'(q)
    &=
    \chi'(q)
    ;
    \\
    \deriv{\delta} \Phi_1^{\delta\chi}(q)
    &=
    \chi(t)
    ;
    \\
    \deriv{\delta} (\xi^s \circ \Phi)'(q)
    &=
    \deriv{q} \lt((\partial_1\xi^s \circ \Phi)(q) \chi(q)\rt).
\end{align*} 
We will use the computation
\[
    \lt(\sqrt{\frac{f}{g}}\rt)' = \frac{f'g-fg'}{2f^{1/2}g^{3/2}}.
\] 
Recall the previous calculation and setting $p\equiv 1$, we find
\begin{align*}
    G(\Phi,\chi)&
    \equiv 2\deriv{\delta} F(\Phi^{\delta\chi}) \Big|_{\delta = 0} \\
    &=
    \lambda_1
    \underbrace{
        \int_0^1
        \sqrt{\frac{(\xi^1 \circ \Phi)'(q)}{\Phi'_1(q)}}
        \chi'(q)
        \diff{q}
    }_{T_0}
    +
    \sum_{s\in \sS}
    \lambda_s
    \underbrace{
        \int_0^1
        \sqrt{\fr{\Phi'_s(q)}{( \xi^s \circ \Phi)'(q)}}
        \deriv{\delta} ( \xi^s \circ \Phi^{\delta\chi})'(q)
        \Big|_{\delta=0}
        \diff{q}
    }_{T_s}.
\end{align*}



We now begin the main computation with:
\begin{align*}
    \deriv{\delta} T_0
    &=
    \int_0^1 
    \deriv{\delta}
    \lt(
        \sqrt{
        \frac
        {(\xi^1\circ\Phi^{\delta})'(q)}
        {(\Phi_1^{\delta})'(q)}
        }
    \rt)
    \chi'(q)
    \de q
    \\
    &=
    \frac{1}{2}
    \int_0^1 
    \frac{
        \deriv{q} \lt((\partial_1\xi^1 \circ \Phi)(q) \chi(q)\rt)\Phi_1'(q)
        -
        (\xi^1\circ\Phi^{\delta})'(q)\chi'(q)
    }
    {
    (\xi^1\circ\Phi^{\delta})'(q)^{1/2}
    ~
    \Phi_1'(q)^{3/2}
    }
    \chi'(q)
    \de q
\end{align*}
When $\Phi(q)=(q,q)$ this simplifies to:
\begin{align*}
\deriv{\delta} T_0
    &=
    \frac{1}{2}
    \int_0^1 
    \frac{
        \deriv{q} \lt((\partial_1\xi^1 \circ \Phi)(q) \chi(q)\rt)\Phi_1'(q)
        -
        (\xi^1\circ\Phi^{\delta})'(q)\chi'(q)
    }
    {
    (\xi^1\circ\Phi^{\delta})'(q)^{1/2}
    ~
    \Phi_1'(q)^{3/2}
    }
    \chi'(q)
    \de q
    \\
    &=
    \frac{1}{2}
    \int_0^1 
    \frac{
        \deriv{q} \lt((\partial_1\xi^1)(q,q) \chi(q)\rt)
        -
        (\xi^1)'(q,q)\chi'(q)
    }
    {
    (\xi^1)'(q,q)^{1/2}
    }
    \chi'(q)
    \de q
    \\
    &=
    \frac{1}{2}
    \int_0^1 
    \frac{
        (\partial_1\xi^1)'(q,q) \chi(q)
        +
        \big(
            \partial_1\xi^1(q,q)-(\xi^1)'(q,q)
        \big)
        \chi'(q)
    }
    {
    (\xi^1)'(q,q)^{1/2}
    }
    \chi'(q)
    \de q
    \\
    &=
    \frac{1}{2}
    \int_0^1 
    \frac{
        (\partial_1\xi^1)'(q,q) \chi(q)
        -
        \partial_2\xi^1(q,q)
        \chi'(q)
    }
    {
    (\xi^1)'(q,q)^{1/2}
    }
    \chi'(q)
    \de q
\end{align*}



Next we write things out for $T_s$:
\begin{align*}
    \deriv{\delta} T_s
    &=
    \int_0^1
    \deriv{\delta}
    \lt(
    \sqrt{
        \frac
        {(\Phi_s^{\delta})'(q)}
        {(\xi^s\circ\Phi^{\delta})'(q)}
    }
    \deriv{q} \lt((\partial_1\xi^s \circ \Phi^{\delta})(q) \chi(q)\rt)
    \rt)
    \de q
    \\
    &=
    \int_0^1
    \lt(
    \sqrt{
        \frac
        {(\Phi_s)'(q)}
        {(\xi^s\circ\Phi)'(q)}
    }
    \deriv{q} \lt((\partial_{1,1}\xi^s \circ \Phi)(q) \chi(q)^2\rt)
    \rt)
    \de q
    \\
    &\quad\quad\quad
    +
    \int_0^1
    \deriv{\delta}
    \lt(
    \sqrt{
        \frac
        {(\Phi_s^{\delta})'(q)}
        {(\xi^s\circ\Phi^{\delta})'(q)}
    }
    \rt)
    \deriv{q} \big((\partial_1\xi^s \circ \Phi)(q) \chi(q)\big)
    \de q
    .
\end{align*}
With $\Phi(q)=(q,q)$ we have the intermediate computations
\begin{align*}
    \deriv{q} \lt((\partial_{1,1}\xi^1 \circ \Phi)(q) \chi(q)^2\rt)
    &=
    (\partial_{1,1}\xi^s\circ\Phi)'(q)\chi(q)^2
    +
    2(\partial_{1,1}\xi^s\circ\Phi)(q)\chi(q)\chi'(q)
    \\
    &=
    \partial_{1,1}\xi^s(q,q)'\chi(q)^2 + \big(2\partial_{1,1}\xi^s(q,q)\big)\chi(q)\chi'(q)
    .
\end{align*}
For the other term, in the case $s=1$ we have
\begin{align*}
    \deriv{\delta}
    \lt(
    \sqrt{
        \frac
        {(\Phi_1^{\delta})'(q)}
        {(\xi^1\circ\Phi^{\delta})'(q)}
    }
    \rt)
    &=
    \frac{
        \deriv{\delta}(\Phi_1^{\delta})'(q)
        \times
        (\xi^1\circ\Phi^{\delta})'(q)
        -
        (\Phi_1^{\delta})'(q)
        \deriv{\delta}
        (\xi^1\circ\Phi^{\delta})'(q)
    }
    {2(\Phi_1^{\delta})'(q)^{1/2}~(\xi^1\circ\Phi^{\delta})'(q)^{3/2}}
    \\
    &=
    \frac{
        \chi'(q)
        (\xi^1\circ\Phi)'(q)
        -
        (\Phi_1)'(q)
        \deriv{q} \lt((\partial_1\xi^1 \circ \Phi)(q) \chi(q)\rt)
    }
    {2(\Phi_1)'(q)^{1/2}~(\xi^1\circ\Phi)'(q)^{3/2}}
     \\
    &=
    \frac{
        \chi'(q)
        \xi^1(q,q)'
        -
        \deriv{q} \lt((\partial_1\xi^1(q,q) \chi(q)\rt)
    }
    {2(\xi^1(q,q)')^{3/2}}
     \\
    &=
    \frac{
        \chi'(q)
        \partial_2\xi^1(q,q)
        -
        \chi(q)\partial_1\xi^1(q,q)' 
    }
    {2(\xi^1(q,q)')^{3/2}}
\end{align*}
while for $s=2$
\begin{align*}
    \deriv{\delta}
    \lt(
    \sqrt{
        \frac
        {(\Phi_2^{\delta})'(q)}
        {(\xi^2\circ\Phi^{\delta})'(q)}
    }
    \rt)
    &=
    \frac{
        \deriv{\delta}(\Phi_2^{\delta})'(q)
        \times
        (\xi^2\circ\Phi^{\delta})'(q)
        -
        (\Phi_2^{\delta})'(q)
        \deriv{\delta}
        (\xi^2\circ\Phi^{\delta})'(q)
    }
    {2(\Phi_2^{\delta})'(q)^{1/2}~(\xi^2\circ\Phi^{\delta})'(q)^{3/2}}
    \\
    &=
    \frac{
        -
        \Phi_2'(q)
        \deriv{q} \lt((\partial_1\xi^2 \circ \Phi)(q) \chi(q)\rt)
    }
    {2\Phi_2'(q)^{1/2}~(\xi^2\circ\Phi)'(q)^{3/2}}
     \\
    &=
    \frac{
        -
        \deriv{q} \lt((\partial_1\xi^2(q,q) \chi(q)\rt)
    }
    {2(\xi^2(q,q)')^{3/2}}
     \\
    &=
    \frac{
        -
        \partial_1\xi^2(q,q)' \chi(q)
        -
        \partial_1\xi^2(q,q)\chi'(q)
    }
    {2(\xi^2(q,q)')^{3/2}}.
\end{align*}
Additionally,
\begin{align*}
    \deriv{q} \lt((\partial_1\xi^s \circ \Phi)(q) \chi(q)\rt)
    &=
    \partial_1\xi^s(q,q)'\chi(q) + \partial_1\xi^s(q,q)\chi'(q).
\end{align*}

We found above that
\begin{align*}
    \deriv{\delta} T_s
    &=
    \int_0^1
    \lt(
    \sqrt{
        \frac
        {(\Phi_s)'(q)}
        {(\xi^s\circ\Phi)'(q)}
    }
    \deriv{q} \lt((\partial_{1,1}\xi^s \circ \Phi)(q) \chi(q)^2\rt)
    \rt)
    \de q
    \\
    &\quad\quad\quad+
    \int_0^1
    \deriv{\delta}
    \lt(
    \sqrt{
        \frac
        {(\Phi_s^{\delta})'(q)}
        {(\xi^s\circ\Phi^{\delta})'(q)}
    }
    \rt)
    \deriv{q} \lt((\partial_1\xi^s \circ \Phi)(q) \chi(q)\rt)
    \de q
    .
\end{align*}
In particular $s=1$ gives
\begin{align*}
    \deriv{\delta} T_1
    &=
    \int_0^1
    \frac
    {1}
    {(\xi^1(q,q)')^{1/2}}
    \lt(
        (\partial_{1,1}\xi^1(q,q))'\chi(q)^2 + 2\big(\partial_{1,1}\xi^1(q,q)\big)\chi(q)\chi'(q)
    \rt)
    \de q
    \\
    &\quad\quad\quad+
    \int_0^1
    \frac{
        \chi'(q)
        \partial_2\xi^1(q,q)
        -
        \chi(q)\partial_1\xi^1(q,q)' 
    }
    {2(\xi^1(q,q)')^{3/2}}
    \lt(
        \partial_1\xi^1(q,q)'\chi(q) + \partial_1\xi^1(q,q)\chi'(q)
    \rt)
    \de q
\end{align*}
and $s=2$ gives
\begin{align*}
    \deriv{\delta} T_2
    &=
    \int_0^1
    \frac
    {1}
    {(\xi^2(q,q)')^{1/2}}
    \lt(
        (\partial_{1,1}\xi^2(q,q))'\chi(q)^2 + 2\big(\partial_{1,1}\xi^2(q,q)\big)\chi(q)\chi'(q)
    \rt)
    \de q
    \\
    &\quad\quad\quad+
    \int_0^1
    \frac{
        -
        \partial_1\xi^2(q,q)' \chi(q)
        -
        \partial_1\xi^2(q,q)\chi'(q)
    }
    {2(\xi^2(q,q)')^{3/2}}
    \lt(
        \partial_1\xi^2(q,q)'\chi(q) + \partial_1\xi^2(q,q)\chi'(q)
    \rt)
    \de q.
\end{align*}
Moreover since $\xi(x,y)=x^4+y^4+24xy$, we have
\begin{align*}
    \partial_2\xi^1(q,q)
    =
    \partial_1\xi^2(q,q)
    =
    \frac{24}{\lambda_s}
    &=
    48
    \\
    \partial_{1}\xi^2(q,q)'&=0
    \\
    \partial_{1,1}\xi^2(q,q)&=0.
\end{align*}
Additionally,
\begin{align*}
    (\partial_1\xi^1)'(q,q)&=
    \partial_{1,1}\xi^1(q,q)
    \\
    &=
    2\partial_{x,x,x}\xi(x,y)|_{x=y=q} 
    \\
    &= 48q;
    \\
    (\partial_{1,1}\xi^1(q,q))'&=48;
    \\
    \xi^1(q,q)'=\xi^2(q,q)'&=24q^2+48;
    \\
    \partial_1\xi^1(q,q)&=24q^2
    .
\end{align*}


We combine our computations below. The red line vanishes since $\partial_{1,1}\xi^2(x,y)\equiv 0$.
\begin{align*}
    2\deriv{\delta}G
    &=
    \deriv{\delta}(T_0+T_1+T_2)
    \\
    &=
    \frac{1}{2}
    \int_0^1 
    \frac{
    (\partial_1\xi^1)'(q,q) \chi(q)
    -
    \partial_2\xi^1(q,q)
    \chi'(q)
    }
    {
    (\xi^1)'(q,q)^{1/2}
    }
    \chi'(q)
    \de q
    \\
    &\quad\quad\quad+
    \int_0^1
    \frac
    {1}
    {(\xi^1(q,q)')^{1/2}}
    \lt(
        (\partial_{1,1}\xi^1(q,q))'\chi(q)^2 + 2\big(\partial_{1,1}\xi^1(q,q)\big)\chi(q)\chi'(q)
    \rt)
    \de q
    \\
    &\quad\quad\quad+
    \int_0^1
    \frac{
        \chi'(q)
        \partial_2\xi^1(q,q)
        -
        \chi(q)\partial_1\xi^1(q,q)' 
    }
    {2(\xi^1(q,q)')^{3/2}}
    \lt(
        \partial_1\xi^1(q,q)'\chi(q) + \partial_1\xi^1(q,q)\chi'(q)
    \rt)
    \de q
    \\
    &\quad\quad\quad+
    {
        \color{red}
        \int_0^1
        \frac
        {1}
        {(\xi^2(q,q)')^{1/2}}
        \lt(
            (\partial_{1,1}\xi^2(q,q))'\chi(q)^2 + 2\big(\partial_{1,1}\xi^2(q,q)\big)\chi(q)\chi'(q)
        \rt)
        \de q 
    }
    \\
    &\quad\quad\quad+
    \int_0^1
    \frac{
        -
        \partial_1\xi^2(q,q)' \chi(q)
        -
        \partial_1\xi^2(q,q)\chi'(q)
    }
    {2(\xi^2(q,q)')^{3/2}}
    \lt(
        \partial_1\xi^2(q,q)'\chi(q) + \partial_1\xi^2(q,q)\chi'(q)
    \rt)
    \de q.
    \\
    %%%%%%%%
    &=
    %%%%%%%%
    \frac{1}{2}
    \int_0^1 
    \frac{
    48q\chi(q)
    -
    48
    \chi'(q)
    }
    {
    (24q^2+48)^{1/2}
    }
    \chi'(q)
    \de q
    \\
    &\quad\quad\quad+
    \int_0^1
    \frac
    {
    \lt(
        48\chi(q)^2 + 96q\chi(q)\chi'(q)
    \rt)
    }
    {(24q^2+48)^{1/2}}
    \de q
    \\
    &\quad\quad\quad+
    \int_0^1
    \frac{
        48\chi'(q)
        -
        48q\chi(q)
    }
    {2(24q^2+48)^{3/2}}
    \lt(
        48q\chi(q) + 24q^2\chi'(q)
    \rt)
    \de q
    \\
    &\quad\quad\quad+
    \int_0^1
    \frac{
    -48\chi'(q)
    }
    {2(24q^2+48)^{3/2}}
    \lt(
        48\chi'(q)
    \rt)
    \de q    
    \\
    %%%%%%%%v2
    &=
    %%%%%%%%v2
    \sqrt{24}
    \int_0^1 
    \frac{
    q\chi(q)
    -
    \chi'(q)
    }
    {
    (q^2+2)^{1/2}
    }
    \chi'(q)
    \de q
    \\
    &\quad\quad\quad+
    \sqrt{24}
    \int_0^1
    \frac
    {
    \lt(
        2\chi(q)^2 + 4q\chi(q)\chi'(q)
    \rt)
    }
    {(q^2+2)^{1/2}}
    \de q
    \\
    &\quad\quad\quad+
    \sqrt{24}
    \int_0^1
    \frac{1}
    {(q^2+2)^{3/2}}
    \Big(
    \big(
        \chi'(q)-q\chi(q)
    \big)
    \cdot
    \big(
        2q\chi(q) + q^2\chi'(q)
    \big)
    -
    2\chi'(q)^2
    \Big)
    \de q.
\end{align*}


Simplifying further,
\begin{align*}
    \frac{\deriv{\delta}G}{\sqrt 6}
    &=
    \int_0^1 
    \frac{
    q\chi(q)
    -
    \chi'(q)
    }
    {
    (q^2+2)^{1/2}
    }
    \chi'(q)
    \de q
    \\
    &\quad\quad\quad+
    \int_0^1
    \frac
    {
    \lt(
        2\chi(q)^2 + 4q\chi(q)\chi'(q)
    \rt)
    }
    {(q^2+2)^{1/2}}
    \de q
    \\
    &\quad\quad\quad+
    \int_0^1
    \frac{1}
    {(q^2+2)^{3/2}}
    \Big(
    \big(
        \chi'(q)-q\chi(q)
    \big)
    \cdot
    \big(
        2q\chi(q) + q^2\chi'(q)
    \big)
    -
    2\chi'(q)^2
    \Big)
    \\
    %%%
    &=
    %%%
    \int_0^1 
    \frac{
    1
    }
    {
    (q^2+2)^{1/2}
    }
    \lt(
    2\chi(q)^2 + 5q\chi(q)\chi'(q)-\chi'(q)^2
    \rt)
    \de q
    \\
    &\quad\quad\quad+
    \int_0^1
    \frac{1}
    {(q^2+2)^{3/2}}
    \Big(
    \big(
        \chi'(q)-q\chi(q)
    \big)
    \cdot
    \big(
        2q\chi(q) + q^2\chi'(q)
    \big)
    -
    2\chi'(q)^2
    \Big)
    \de q
    .
\end{align*}
Note that this is a quadratic form in $(\chi(q),\chi'(q))$ as we expect. Now we want to find an example function $\chi$ making this strictly positive. This will show that $\Phi(q)=(q,q)$ is \textbf{not} a local maximum for the ALG functional. 

We try
\begin{align*}
    \chi(q)&=\sin(\pi cq);
    \\
    \chi'(q)&=\pi c\cos(\pi cq).
\end{align*}
on $q\in [0,1/c]$, for $c=\frac{1}{10}$ and $c=\frac{1}{20}$. 

\begin{figure}[h]
\includegraphics[width=0.45\linewidth]{imgs/wolfram-pic-10.png}
\includegraphics[width=0.45\linewidth]{imgs/wolfram-pic-20.png}
\caption{This calculation showcases the beauty of the multi-species setting.}
\end{figure}


It would be good to numerically verify the above calculations by manually computing ALG. Once that seems ok, it would be nice to look into a rigorous numerical integrator. (Although the integrand seems tame enough to trust Wolfram Alpha.)

% \section{State Evolution: Proof of Proposition \ref{prop:state_evolution}}
\label{sec:ProofSE}


In this section we prove Proposition~\ref{prop:state_evolution}, following the appendix of \cite{ams20}. Throughout, we denote by $\bG^{(k)}\in (\bbR^N)^{\otimes k}$, $k\ge 2$ a sequence of standard Gaussian tensors. For $S_k$ the symmetric group on $k$ elements we also write 
\begin{equation}
\label{eq:Ak-def}
    \bA^{(k)}=\frac{1}{N^{(k-1)/2} }\Gamma^{(k)}
    \diamond 
    \sum_{\pi\in S^k}(\bG^{(k)})^{\pi}
\end{equation}
for the rescaled tensors with entries
\begin{equation}
\label{eq:A-defn}
     \bA^{(k)}_{i_1,\dots,i_k}
     =
     \frac{1}{N^{(k-1)/2} }
     \gamma_{s(i_1),\dots,s(i_k)} 
     \sum_{\pi\in S_k}\bG^{(k)}_{i_{\pi(1)},\dots,i_{\pi(k)}}
     \,.
\end{equation}
For a symmetric tensor $\bA^{(k)}\in(\bbR^N)^{\otimes k}$ and $\bT\in(\bbR^N)^{\otimes (k-1)}$, we denote by  $\bA^{(k)}\{\bT\}\in\bbR^N$ the vector with components
\begin{equation}
\label{eq:TensorTensor}
    \bA^{(k)}\{\bT\}_i = \frac{1}{(k-1)!}\sum_{1\le i_1,\cdots,i_{k-1}\le N}A^{(k)}_{i,i_1,\cdots,i_{k-1}}T_{i_1\dots i_{k-1}}.
\end{equation}
For $\bu\in\bbR^N$ we denote by $\bA^{(k)}\{\bu\}=\bA^{(k)}\{\bu^{\otimes (k-1)}\}$ the vector with entries
%
\begin{equation}
\label{eq:vector}
\bA^{(k)}\{\bu\}_i = \frac{1}{(k-1)!}\sum_{1\le i_1,\cdots,i_{k-1}\le N}A^{(k)}_{i,i_1,\cdots,i_{k-1}}u_{i_1}\cdots u_{i_{k-1}}.
\end{equation}
Note that for $\bA^{(k)}$ as in \eqref{eq:Ak-def}, one has 
\[
\bA^{(k)}\{\bu\}=\nabla H_{N,k}(\bu)
\]
where $H_{N,k}$ denotes the part of $H_N$ of total degree $k$.


For $\bu,\bv\in\bbR^N$ we recall from Subsection~\ref{subsec:notation} the notations 
\begin{align*}
    \< \bv\>_N&=N^{-1}\sum_{i\le N} v_i,
    \\
    \< \bu,\bv\>_N
    &= 
    N^{-1}\sum_{i\le N}u_iv_i
    =
    \langle\vec\lambda, \vR(\bu,\bv)\rangle,
    \\
    \|\bu\|_{N}&= \<\bu,\bu\>_N^{1/2}=\sqrt{\sum_s \lambda_s R_s(\bu,\bu)}.
\end{align*}


Given functions $f_{t,s}:\bbR^{t+1}\to\bbR$ of $t+1$
variables for each $s\in\sS$, and $\bv^0,\bv^1,\dots,\bv^t\in\bbR^{N}$, we define $f_t(\bv^0,\bv^1,\dots,\bv^t)\in\mathbb R^N$ component-wise via
%
\begin{align}
%
    f_t(\bv^0,\bv^1,\dots,\bv^t)_i=f_{t,s(i)}(v^0_i,\dots,v^t_i).
%
\end{align}
%
Finally, for a sequence of vectors $\bw^0,\bw^1,\dots$, we write $\bw^{\le t} = (\bw^0,\bw^1,\dots,\bw^t)$.


To deduce the state evolution result for mixed tensors, we analyze a slightly more general iteration where each homogenous
$k$-tensor is tracked separately, while restricting ourselves to the case where the mixture $\xi$ has finitely many components: $\gamma_{s_1,\dots,s_k} = 0$ for all $(s_1,\dots,s_k)\in \sS^k$ for all $k \ge D +1$ for some fixed $D \ge 2$. We then proceed by an approximation argument to extend the convergence to the general case $D = \infty$.

We begin by introducing the Gaussian process that captures the asymptotic behavior of AMP. 
Define $\xi^k$ to be the degree $k$ part of $\xi$, and 
\[
    \xi^{k,s}=\frac{1}{\lambda_s}\partial_{x_s}\xi^k(x_1,\dots,x_r)
\]
the degree $k-1$ part of $\xi^s$. 

An AMP iteration is specified by Lipschitz functions $f_{t,s}:\bbR^{2(t+1)}\to \bbR$ for each $(t,s)\in\naturals\times \sS$.\footnote{The unusual factor $2$ in the exponent comes from the external randomness vectors $\be^1,\dots,\be^t$.}  
For each iteration $t$, the state of the algorithm is given by vectors
$\bw^t\in \bbR^N$, and $\bz^{k,t}\in \bbR^N$, with $k\in\{2,\dots,D\}$. 
Moreover for each $t$, there is also an external randomness vector $\be^t\in\bbR^N$ with independent coordinates $e^t_i\sim \mu_{t,s(i)}$ from deterministic probability distributions $\big(\mu_{t,s}\big)_{t\geq 0,s\in \sS}$ with finite $L^2$ norm.
We now start to define the AMP iteration steps (the definition finishes at \eqref{eq:recursive-covariance}). A single step is given by
%
\begin{align}
\label{eq:AMP-def1}
    \AMP_t
    \lt(\bw^0,\dots,\bw^t
    ;
    \be^0,\dots,\be^t\rt
    )_{k}
    &\equiv
    \bA^{(k)}
    \{
    \f_t
    \}
    -
    \sum_{t'\leq t}
    d_{t,t',k}
    \diamond 
    \f_{t'-1}
    \, ,
    \\
\label{eq:bff}
    \f_t &= f_t(\bw^0,\dots,\bw^t;\be^0,\dots,\be^t),
    \\
\label{eq:AMP-def2}
    d_{t,t',k,s} 
    &\equiv 
    \sum_{s'\in\sS}
    \partial_{x_{s'}}\xi^{k,s}
    \lt(\lt( 
    \E\lt[ 
    F_{t,s} F_{t'-1,s}
    \rt]
    \rt)_{s\in\sS}
    \rt) 
    \times
    \E\lt[ 
    \frac{\partial_{W^{t'}_{s'}} F_{t,s'}}{\partial W^{t'}_{s'}}
    \rt]
    \, 
    ,
    \\
\label{eq:F-def}
    F_{t,s'}
    &\equiv
    f_{t,s'}\big(W^0_{s'},W^1_{s'},\dots, W^t_{s'};E^0_{s'},\dots,E^t_{s'}\big)
    .
%
\end{align}
A general multi-species tensor AMP algorithm then takes the form:
\begin{align}
\label{eq:TensorAMP}
    \bw^t=\sum_{2\leq k\leq D} \bz^{k,t}\, ,\quad \quad
    \bz^{k,t+1}
    =
    \AMP_t
    \lt(\bw^0,\dots,\bw^t;\be^0,\dots,\be^t\rt)_k\, .
\end{align}


For the right-hand side of \eqref{eq:F-def} to make sense, we must define for each $t\geq 0$ and $s\in \sS$ a distribution over sequences
$(W^0_s,\dots,W^t_s;E^0_s,\dots,E^t_s)$. The latter variables $E^{t'}_s\sim \mu_{t',s}$ are simply taken independent of each other and all other variables. The construction of the $W$ variables is recursive across $t$ as follows. 
For each $2\leq k\leq D$ and $s\in\sS$, we let $U^{k,0}_s\sim \nu_{k,s}$
and construct a centered Gaussian process
\[
    (U^{k,1}_s,U^{k,1}_s,\dots,U^{k,t}_s)
\]
which is independent of $U^{k,0}_s$. The variables $U^{k,t}_s$ and $U^{k',t'}_{s'}$ are independent unless $(k,s)=(k',s')$. It remains to specify the covariance of $(U^{k,t}_s)_{1\leq t\le T}$ which is given recursively by:
\begin{equation}
\label{eq:recursive-covariance}
\begin{aligned}
    \E\big[U^{k,t+1}_s U^{k,t'+1}_{s}\big]
    & = 
    \xi^{k,s}(\Sigma^{k,t,t'})\,;
    \\
    \Sigma^{k,t,t'}_{s}
    &=
    \bbE\lt[
    f_{t,s}(W^0_s,\dots,W^t_s;E^0_s,\dots,E^t_s)
    f_{t',s}(W^0_s,\dots,W^{t'}_s;E^0_s,\dots,E^{t'}_s)
    \rt],\quad \forall s\in \sS
    \\
    W^t_s & \equiv  \sum_{2\leq k\leq D} U^{k,t}_s\, 
    .
\end{aligned}
\end{equation}





%
The main result, an extension of Proposition \ref{prop:state_evolution}, follows. Below we use $\bbW_2$ to denote the Wasserstein-$2$ distance between probability measures on Euclidean space in any dimension. We say a function $\psi:\bbR^d\to\bbR$ is \textbf{pseudo-Lipschitz} if 
\[
    |\psi(\bw)-\psi(\by)|
    \leq 
    C(1+\|\bw\|+\|\by\|)
    \cdot 
    \|\bw-\by\|
    ,\quad\forall \bw,\by\in\bbR^d.
\]

\begin{theorem}[State Evolution for AMP]
\label{thm:mixedAMP}
Let $\{\bG^{(k)}\}_{k\ge 2}$ be independent standard Gaussian tensors with $\bG^{(k)}\in(\bbR^N)^{\otimes k}$, and define $\bA^{(k)}$ as in \eqref{eq:A-defn}. Fix a sequence of Lipschitz functions $f_{t,s}:\bbR^{k+1}\to\bbR$.
 Let $\bz^{2,0},\cdots \bz^{D,0}\in\bbR^N$ be deterministic vectors and $\bw^0 =\sum_{2\leq k\leq D} \bz^{k,0}$.
Assume that for each $s\in\sS$, the empirical distribution of the vectors 
\[
    (z_i^{2,0},\cdots z_i^{D,0}),\quad i\in \cI_s
\]
converges in $\bbW_2(\bbR^{D-1})$ distance to the law of the vector $(U^{k,0}_s)_{2\le k\le D}$. 

Let $\bw^{t}, \bz^{k,t}$, $t\ge 1$ be given by the \emph{tensor AMP} iteration. Then, for all $s\in\sS$ and $T\ge 1$ and for any pseudo-Lipschitz functions $\psi:\bbR^{D \times T}\to \bbR$ and $\wt\psi:\bbR^T\to\bbR$, we have
%
\begin{align}
\label{eq:z-U-convergence}
    \plim_{N\to\infty}\frac{1}{N_s}\sum_{i\in\cI_s}\psi \Big((z_i^{k,t})_{ k\le D,t\le T}\Big) 
    &= \E\lt[\psi\big((U^{k,t}_s)_{2\leq k\leq D,t\leq T}\big)\rt]\, 
    ;
    \\
\label{eq:w-W-convergence}
    \plim_{N\to\infty}\frac{1}{N_s}\sum_{i\in\cI_s}\wt\psi \Big((w_i^{t})_{t\le T}\Big) 
    &= \E\lt[\wt\psi\big((W^{t}_s)_{t\leq T}\big)\rt]\,
    .
%
\end{align}
\end{theorem}
%

Note that \eqref{eq:w-W-convergence} (which concerns the actual AMP iterates $\bw^t$) is a special case of \eqref{eq:z-U-convergence} (which is more convenient to prove). Indeed one can take $\psi((z^{k,t})_{k\le D,t\le T}=\wt\psi\lt(\sum_{k\le D}z^{k,t}\rt)_{t\le T}$. 
In the special case that $c_k =0$ for all $k\ge D+1$, Proposition \ref{prop:state_evolution} follows immediately from Theorem~\ref{thm:mixedAMP} by baking the contribution of $\bh$ explicitly into $f_t$ (since we require $k\geq 2$ above). Proposition \ref{prop:state_evolution} for non-polynomial $\xi$ follows by a standard approximation argument outlined at the end of Subsection~\ref{subsec:further-def-app}.
For the remainder of this Appendix we thus focus on establishing \eqref{eq:z-U-convergence}.

\subsection{Further Definitions}
\label{subsec:further-def-app}

We define the notations
%
\begin{align*}
%
    \bW_t
    &=
    \big[
    \,\bw^0~|~\bw^1~|\;\cdots\;|~\bw^t
    \big]\, ,
    \\
    \bE_t
    &=
    \big[
    \,\be^0~|~\be^1~|\;\cdots\;|~\be^t
    \big]\, ,
    \\
    \bZ_{k,t}
    &=
    \big[
    \,\bz^{k,0}~|~\bz^{k,1}~|\,\cdots\;|~\bz^{k,t}\,
    \big].
\end{align*}
%
Given a $N\times (t+1)$ matrix such as $\bW_t$, and a tensor $\bA^{(k)}\in(\bbR^{N})^{\otimes k}$, we write 
$\bA^{(k)}\{\bW_t\}$ for the $N\times (t+1)$ matrix with columns $\bA^{(k)}\{\bw^0\}$, \dots, $\bA^{(k)}\{\bw^t\}$:
%
\begin{align*}
%
\bA^{(k)}\{\bW_t\}&=\Big[\bA^{(k)}\{\bw^0\}\Big|\bA^{(k)}\{\bw^1\}\Big|\;\cdots\;\Big|\bA^{(k)}\{\bw^t\}\Big]\, .
%
\end{align*}
%
We will write $\f_t=f_t(\W_t,\bE_t)=f_t(\bw^0,\dots,\bw^t,\be^0,\dots,\be^t)$ and also set 
%
\begin{align}
\label{eq:Ydef1}
    \by_{k,t+1}(\ZZ_{k,t})
    &=
    \A_k\lt\{\f_t\rt\} 
    = 
    \bz^{k,t+1}
    +
    \sum_{t_1\leq t} 
    d_{t,t_1,k}
    \diamond 
    \f_{t_1-1}\, ,
    \\
\label{eq:Ydef2}
    \bY_{k,t} 
    &= 
    [\by_{k,1}|\;\cdots\;|\by_{k,t}]
    \, ,
    \\
\nonumber
    \by_{t}(\ZZ_{k,t})&=\sum_{2\leq k\leq D}  \by_{k,t}(\ZZ_{k,t})\, . 
\end{align}
We also define an associated $(t+1)\times (t+1)$ Gram matrix $\bG_{\xi^{k,s}}=\bG_{\xi^{k,s},t}$ via 
\begin{equation}
\label{eq:bG-def}
    (\bG_{\xi^{k,s}})_{t_1,t_2}
    =
    \xi^{k,s}\big(\vR( 
    \f_{t_1},\f_{t_2})\big).
\end{equation}
The dependence of $\bG_{\xi^{k,s}}$ will often be suppressed.
We define a second Gram matrix 
$\bGG_{\xi^{k,s}}$, also of size $(t+1)\times (t+1)$, via (recall \eqref{eq:zeta-defn})
\begin{align}
\label{eq:bGG-def}
    (\bGG_{\xi^{k,s}})_{t_1,t_2}
    &=
    \zeta^{k,s}\lt(\vR\big( \f_{t_1},\f_{t_2}\big)\rt)
    \\
\nonumber
    &=
    \sum_{s'\in\sS}
    \partial_{x_{s'}}\xi^{k,s}
    \lt(\vR\big( \f_{t_1}, \f_{t_2}\big)\rt).
\end{align}
Finally, we let $\cF_t$ denote the $\sigma$-algebra generated by all iterates up to time $t$:
%
\begin{equation}
\label{eq:cFt-defn}
%
    \cF_t
    =
    \sigma\big(\{\bz_{k,t_1}\}_{k\le D,t_1\le t}\big)
    =
    \sigma\big(\{\bz_{k,t_1},\bw^{t_1},\f_{t_1}\}_{k\le D,t_1\le t}\big)\, .
%
\end{equation}

Throughout the proof of state evolution we make the following simplifying assumptions:

\begin{assumption}
\label{as:degree-D}
    $\xi$ is a degree $D$ polynomial with all coefficients $\gamma_{s_1,\dots,s_k}$ for $2\leq k\leq D$ strictly positive.
\end{assumption}

\begin{assumption}
\label{as:well-conditioned}
    Each matrix $\bG_{\xi^{k,s},t}$ is well-conditioned, i.e. 
    \[
    C^{-1}\le \sigma_{\min}(\bG_{\xi^{k,s},t})\le \sigma_{\max}(\bG_{\xi^{k,s},t})\le C
    \]
    for all $t\le T$. Here $\bG_{\xi^{k,s},t}$ is defined based on iterates that will appear in Theorems~\ref{thm:SELAMP} and \ref{lem:ampequalslamp}. The same holds for $\cL_{k,t}$ as defined in \eqref{eq:cL-k-t}.
\end{assumption}



It is a standard argument that to establish Proposition~\ref{prop:state_evolution}, it suffices to do so under the above assumptions. The reason is that one can always slightly perturb both $\xi$ and the non-linearities $f_{t,s}$ to ensure the assumptions hold. Then suitable continuity properties suffice to transfer all asymptotic guarantees. We refer the reader to \cite[Appendices A.8 and A.9]{ams20} for the arguments in the single-species case, still in the generality of mixed tensors. (In the more common setting $D=2$ of just a random matrix this step is also common for state evolution proofs, see e.g. \cite[Section 4.2.1]{javanmard2013state}.) The corresponding extension in our setting is completely analogous and omitted.




\subsection{Preliminary Lemmas}


The next lemma has several parts. All are elementary Gaussian calculations so their proofs are omitted.

\begin{lemma}
\label{lem:4}
For any deterministic $\bu,\bv\in \bbR^N$ and $\bA^{(k)}$ defined by \eqref{eq:A-defn} we have:
%
\begin{enumerate}
    \item Letting $g_0\sim\normal(0,1)$ be independent of $\bg\sim\normal(0,\id_N)$, we have
        \begin{align}
            \bA^{(k)}\{\bu\}\ed
            \sum_{s\in\sS}
            \bg_s
            \sqrt{\xi^{k,s}(\vR(\bu,\bu))}
            +  
            \frac{g_0}{\sqrt N}
            \sum_{s\in\sS}
            \bu_s
            \sqrt{
                \sum_{s'\in\sS}
                \partial_{x_{s'}} \xi^{k,s}
                \Big(
                \vR(\bu,\bu)
                \Big)
            }
            \, .
        \end{align}
    %
    \item Let $g_0,g_1,\dots,g_r\sim\normal(0,1)$ be independent. We have (jointly across $s\in\sS$)
    \begin{align}
        \sqrt{\lambda_s N} R_s(\bv,\bA^{(k)}\{\bu\})
        \ed 
        \sqrt{\xi^{k,s}(\vR(\bu,\bu))\cdot \vR(\bv,\bv)}g_s
        +  
        \sqrt{
            \sum_{s'\in\sS}
            \partial_{x_{s'}} \xi^{k,s}
            \Big(
            \vR(\bu,\bu)
            \Big)
        }
        R_s(\bu,\bv) \, g_0\, .
    \end{align}
    %
    \item For $s\in\sS$:
     \[
        R\lt(\bA^{(k)}\{\bu\},\bA^{(k)}\{\bv\}\rt)_s\simeq \xi^{k,s}\lt(\vR( \bu,\bv)\rt).
    \]
    \item For a deterministic symmetric tensor $\bT\in(\bbR^N)^{\otimes k-1}$, the vector
    $\bA^{(k)}\{\bT\}$ is centered Gaussian. Its covariance is given by
    %
    \begin{align*}
      \E\big[\bA^{(k)}\{\bT\}_i\bA^{(k)}\{\bT\}_j\big] 
      &= 
      \langle \xi^{k,s(i)}\diamond \bT,\, \bT\rangle_N
      \cdot 1\{i=j\} 
      \\
      &
      \quad
      +\frac{k(k-1)}{N^{k-1}}\,
      \sum_{i_1,\dots,i_{k-2}=1}^N
      \gamma_{i,i_1,\dots,i_{k-2}}
      \gamma_{j,i_1,\dots,i_{k-2}}
      T_{i,i_1,\dots,i_{k-2}}
      T_{j,i_1,\dots,i_{k-2}}\, . 
    \end{align*}
%
    \item Let $\bP\in\bbR^{N\times N}$ be the orthogonal projection onto a (deterministic) subspace $S\subseteq\bbR^N$ with $d=\dim(S)=O(1)$. Then
    \[
        \|\bP\bG^{(k)}\{\bu\} - \bG^{(k)}\{\bu\}\|_2 /\|\bG^{(k)}\{\bu\}\|_2\simeq 0.
    \]
\end{enumerate}
\end{lemma}
%



We next develop a formula for the conditional expectation of a Gaussian tensor $\bA^{(k)}$ given
a collection of linear observations.
We set $\bD$ to be the $t\times t\times t$ tensor with entries $D_{ijk}=1$ if $i=j=k$ and $D_{ijk}=0$ otherwise.
%
\begin{lemma}
\label{lem:symregression}
Recalling \eqref{eq:cFt-defn}, let $\E\{\bA^{(k)}|\cF_t\}$.
Equivalently $\E\{\bA^{(k)}|\cF_t\}$ is the conditional expectation of $\bA^{(k)}$ given the linear-in-$\bA^{(k)}$ observations
%
\begin{align}
%
\bA^{(k)}\{\f_{t'}\}=\by_{k,t'+1} \, \;\;\; \mbox{ for $s\in\{0,\dots,t-1\}$.} \label{eq:LinearConstraint}
%
\end{align}
%
Then we have for $i_1,i_2,\dots,i_k\leq n$,  
%
\begin{align}
%
\label{eq:SymmRegression}
    \E[\bA^{(k)}|\cF_t]_{i_1,i_2,\dots,i_k}= 
    \sum_{j=1}^k \sum_{0\leq r,s\leq t-1} (\hZZ_{k,t})_{i_j,s}\cdot (\bG^{-1}_{\xi^{k,s},t-1})_{s,r}\cdot (\f_{r,i_1}\cdots \f_{r,i_{j-1}}\f_{r,i_{j+1}}\cdots \f_{r,i_{k}})\, .
%
\end{align}
%
Here, the matrix $\hZZ_{k,t}\in\bbR^{N\times t}$ is defined as the solution of a system of linear equations as follows.
Define the linear operator $\cT_{k,t}:\bbR^{N\times t}\to\bbR^{N\times t}$ by letting, for $i\leq N$, $0\leq t_3\leq t-1$:
%
\begin{align}
\label{eq:Tdef}
    [\cT_{k,t}(\bZ)]_{i,t_3}&= \sum_{j=1}^N \sum_{0\leq t_1,t_2\leq t-1} (\f_{t_2})_{i} (\f_{t_2})_{j}
    \lt(
    (\bG_{\xi^{k,s(i)},t-1}^{-1})_{t_2,t_1} (\bGG_{\xi^{k,s(i)},t-1})_{t_2,t_3} 
    \rt)
    \diamond
    (\bZ)_{j,t_1}\, ,
%
\end{align}
%
Then $\hZZ_{k,t}$ is the unique solution of the following linear equation (with $\bY_{k,t}$ defined as per \eqref{eq:Ydef1})
%
\begin{align}
%
\hZZ_{k,t}+\cT_{k,t}(\hZZ_{k,t}) =\bY_{k,t}.\label{eq:ZZ-eq}
%
\end{align}

(Here, $\hZZ_{k,t} = [\hat{\bz}_{k,0},\cdots,\hat{\bz}_{k,t-1}]$ and $\bY_{k,t} = [\hat{\by}_{k,1},\cdots,\hat{\by}_{k,t}]$ have dimensions $N \times t$.)
%
\end{lemma}
%
%\am{Again, not clear the time index for $\bY_{k,t}$ is correct.}
%\ms{Right, changed}

The above formulas for $\E[\bA^{(k)}|\cF_t]$ and $\cT_{k,t}$ are rather complicated. In \cite[Appendix A]{ams20} the reader may find helpful tensor network diagrams for the single-species case. Unfortunately it is less clear how to draw a corresponding tensor network with multiple species.


\begin{proof}[Proof of Lemma \ref{lem:symregression}]
Let $\cV_{k,t}$ be the affine space of symmetric tensors satisfying the constraint \eqref{eq:LinearConstraint}.
The conditional expectation $\E[\bA^{(k)}|\cF_t]$ is the tensor with minimum weighted Frobenius norm $\|\cdot\|_{F,\xi^k}$ in the affine space $\cV_{k,t}$, given by
\begin{equation}
\label{eq:Gammak-inner-product}
    \|\bA\|_{F,\xi^k}^2
    =
    \langle 
    (\Gamma^{(k)})^{-1}\diamond \bA
    ,
    (\Gamma^{(k)})^{-1}\diamond\bA
    \rangle.
\end{equation}
Here $(\Gamma^{(k)})^{-1}$ is the entry-wise inverse of $\Gamma^{(k)}$, which exists by Assumption~\ref{as:degree-D}.

By Lagrange multipliers, there exist vectors $\bm_1,\dots,\bm_t\in\bbR^N$ such that $\E[\bA^{(k)}|\cF_t]=\hbA^{(k)}$ equals
%
\begin{align}
%
    \hbA^{(k)}_t \equiv\Gamma^{(k)}\diamond
    \sum_{t'=0}^{t-1}\sum_{j=1}^k \underbrace{\f_{t'}
    \otimes 
    \cdots 
    \otimes 
    \f_{t'}}_{\mbox{$j-1$ times}}\otimes\bm_{t'} 
    \otimes
    \underbrace{\f_{t'}\otimes \cdots \otimes \f_{t'}}_{\mbox{$k-j$ times}}\, .
%
\end{align}
%


Also by Lagrange multipliers, if a tensor $\hbA^{(k)}$ of this form (for some choice of vectors $\bm_1,\dots,\bm_t$) satisfies the constraints $\hbA^{(k)}\{\f_{t'}\}=\by_{k,t'+1}$
for $s< t$, then such a tensor is unique, and corresponds to $\E[\bA^{(k)}|\cF_t]$.
Without loss of generality, we write 
%
\begin{align}
%
\bm_r =  \sum_{t'=0}^{t-1}(\bG^{-1}_{\xi^{k,s},t-1})_{r,t'}\hbz_{t'}\, , \;\;\;\; \hZZ_{k,t}= [\hbz_1|\;\cdots\;|\hbz_t]\, .
%
\end{align}
%
By direct calculation we obtain that for each $i\in [N]$,
%
\begin{align}
\label{eq:SumT}
    \big(\hbA^{(k)}_t\{\f_{t_1}\}\big)_i &= \sum_{t_2=0}^{t-1} (\bG_{\xi^{k,s(i)},t-1})_{t_1,t_2}(\bm_{t_2})_i
    + 
    \sum_{t_2=0}^{t-1} (\bGG_{\xi^{k,s(i)},t-1})_{t_1,t_2}\<\f_{t_1},\bm_{t_2}\>(\f_{t_2})_i
    \\
    &= 
    (\hbz_{t_1})_i
    +
    \sum_{t_2=0}^{t-1} (\bGG_{\xi^{k,s(i)},t-1})_{t_1,t_2}\<\f_{t_1},\bm_{t_2}\>(\f_{t_2})_i
    \, .
%
\end{align}


We next stack these vectors as columns of an $N\times t$ matrix. The first term yields $\hZZ_{k,t}$. Moreover the second term
coincides with $\cT_{k,t}(\hZZ_{k,t})$ as following by rearranging the order of sums in \eqref{eq:SumT}. Hence
%
\begin{align}
%
    \big[\hbA^{(k)}_t\{\f_{0}\},\cdots,\hbA^{(k)}_t\{\f_{t-1}\}\big] 
    & =  
    \hZZ_{k,t} + \cT_{k,t}(\hZZ_{k,t})\, .
%
\end{align}
%
This in turn implies that the equation determining $\hZZ_{k,t}$ takes the form \eqref{eq:ZZ-eq}.
\end{proof}



\subsection{Long AMP}
\label{sec:LAMP}

As an intermediate step towards proving Theorem \ref{thm:mixedAMP}, we introduce a new iteration that we call Long AMP (LAMP), following 
\cite{berthier2019state}. This iteration is less compact but simpler to analyze. For each $k\le D$, let  
$\cS_{k,t}\subseteq (\bbR^N)^{\otimes k}$ be the linear subspace of tensors $\bT$ that are symmetric and such that
$\bT\{\f_{t_1}\}=0$ for all $t_1<t$.   We denote by $\cP_t^{\perp}(\bA^{(k)})$ the projection of $\bA^{(k)}$ 
onto $\cS_{k,t}$, in the inner product space \eqref{eq:Gammak-inner-product} corresponding to $\Gamma^{(k)}$. 
We then define the LAMP mapping 
%
\begin{align}
\label{eq:LAMP1}
    \LAMP_t\lt(\bv^{\le t}\rt)_{k}
    &\equiv
    \cP_t^{\perp} (\bA^{(k)})
    \{\f_t\}
    +
    \sum_{0\leq t_1\leq t}  
    h_{t,t_1-1,k}
    \diamond 
    \qq^{k,t_1},
    \\
\label{eq:LAMP2}
    h_{t,t_1,k,s}
    &\equiv 
    \sum_{0\le t_2\le t-1}
    \big[\bG_{\xi^{k,s},t-1}^{-1}\big]_{t_1,t_2}
    \big[\bG_{\xi^{k,s},t}\big]_{t_2,t}
    ,~~~ h_{t,-1,k}=0. 
\end{align}
%
Here we use similar notations $\f_t = f_t(\VV_t;\bE_t)$ and $\bG_{\xi^{k,s},t}$ as before (recall \eqref{eq:bG-def}), and take the vectors $\be^t$ as before. However the quantities $\f_t,\bG_{\xi^{k,s},t}$ are now different: they are computed using the vectors $\bv^0,\dots,\bv^t$ using the recursion:
\begin{align}
    \bv^t=\sum_{2\leq k\leq D} \qq^{k,t}\, ,\;\;\;\;\;\;\;
    \qq^{k,t+1}=\LAMP_t\lt(\bv^{\le t}\rt)_{k}\, . 
\end{align}


%
Following \cite{berthier2019state,ams20}, we first establish state evolution for LAMP (under the non-degeneracy Assumption~\ref{as:degree-D}), and then deduce the result for the original AMP.
In analyzing LAMP we use notations analogous to the ones introduced for AMP. In particular:
%
\begin{align}
    \VV_{t}
    &=
    [\bv_{1}|\bv_{2}|\dots|\bv_{t}]
    \\
    \bQ_{k,t}
    &=
    [
    \qq_{k,1}^{\otimes k}|
    \qq_{k,2}^{\otimes k}|
    \dots|
    \qq_{k,t}^{\otimes k}
    ].
\end{align}
%


\subsection{State Evolution for Long AMP}
 
%
\begin{theorem}
\label{thm:SELAMP}
  Under the assumptions of Theorem \ref{thm:mixedAMP}, let $\qq^{2,0},\cdots \qq^{D,0}\in\bbR^N$
be deterministic vectors and $\bv^0 =\sum_{2\leq k\leq D} \bq^{k,0}$.
Assume that the uniform empirical distribution of the $N$ vectors $\{(q_i^{2,0},\cdots, q_i^{D,0})\}_{i\leq N}$ converges 
in $\bbW_2$ distance to the law of the vector $(U^{k,0})_{2\le k\le D}$.

Further we assume there is a constant $C<\infty$ such that for all $t\le T$:
%
\begin{itemize}
%
\item[$(i)$] The matrices $\bG_{\xi^{k,s},t}= \bG_{\xi^{k,s},t}(\VV)$ are uniformly well-conditioned as guaranteed by Assumption~\ref{as:well-conditioned}.
%
\item[$(ii)$] Let the linear operator $\cT_{k,t}:\bbR^{N\times t}\to\bbR^{N\times t}$ be defined as per \eqref{eq:Tdef}, with $\bG_{\xi^{k,s},t} = \bG_{\xi^{k,s},t}(\VV,\bE)$,
and  $\f_t=f_t(\VV,\bE)$, and define 
\begin{equation}
\label{eq:cL-k-t}
\cL_{k,t} = {\boldsymbol 1}+\cT_{k,t}.
\end{equation}
Then $C^{-1}\le \sigma_{\min}(\cL_{k,t})\le \sigma_{\max}(\cL_{k,t})\le C$.
%
\end{itemize}

Then the following statements hold for any $t\le T$ and sufficiently large $N$:
%
\begin{enumerate}[label=(\alph*)]
\item Correct conditional law: 
%
\begin{equation}\label{eq:conditional}
%
\qq^{k,t+1}|_{\mathcal F_t}\ed \E[\qq^{k,t+1}|\mathcal F_t] +  \cP_t^{\perp}(\tbA^{(k)}) \{\f_t\}\, .
%
\end{equation}
%
where $\tbA^{(k)}$ is a symmetric tensor distributed identically to $\bA^{(k)}$ and independent of everything else,  and
$\cP_{t}^{\perp}$ is the projection onto the subspace $\cS_{k,t}$ defined in Section \ref{sec:LAMP}.
Further 
\begin{equation}
\label{eq:CondExp}
    \E[\qq^{k,t+1}|\mathcal F_t]
    = 
    \sum_{s\in\sS}
    \sum_{0\leq t_1\leq t}  
    h_{t,t_1-1,k,s}
    \qq^{k,t_1}_s\, .
\end{equation}
%
Moreover, the vectors $(\qq^{k,t+1})_{2\leq k\leq D}$ are conditionally independent given $\mathcal F_t$. 
%
\item Approximate isometry: we have
 %   
\begin{align}
\label{eq:c1}
    R_s(\qq^{k,t_1+1},\qq^{k,t_2+1})
    &\simeq 
    \xi^{k,s}\lt(\vR( \f_{t_1},\f_{t_2})\rt)\, ,
    \\
\label{eq:c3}
    R_s(\bv^{t_1+1},\bv^{t_2+1})
    &\simeq 
    \xi^{s}\big(\vR( \f_{t_1},\f_{t_2})\big). 
\end{align}
%
Moreover, both sides converge in probability to constants as $N\to\infty$, and for $k_1\neq k_2$ and any $(t_1,t_2)$ and $s\in\sS$,    
\begin{equation}
\label{eq:c2}
    R_s(\qq^{k_1,t_1},\qq^{k_2,t_2})\simeq 0.
\end{equation}
%
\item  State evolution: for each $s\in\sS$ and any pseudo-Lipschitz function 
$\psi:\bbR^{D \times 2(t+1)}\to \bbR$, we have
%
\begin{align}
\label{eq:ConvergenceLAMP}
    \plim_{N\to\infty}
    \frac{1}{N_s}
    \sum_{i\in\cI_s}
    \psi\big((q_i^{k,t'})_{ k\le D,t'\le t}; (e^t_i)_{t'\leq t}\big) 
    = 
    \E\big\{\psi\big((U^{k,t'}_s)_{2\leq k\leq D,t'\leq t}; 
    (E^{t'}_s)_{t'\leq t}
    \big)\big\}\, .
%
\end{align}
%
where $(U^{k,t}_s)_{k\le D,1\le t\le T}$ is the centered Gaussian process defined in the statement of  Theorem \ref{thm:mixedAMP}.
\end{enumerate}
\end{theorem}
% %
% %
In the next subsection, we will prove these statements by induction on $t$. The crucial point we exploit is the representation $(a)$. We emphasize that the iteration number $t$ is bounded as $N\to\infty$; therefore all numerical quantities not depending on $N$ (but possibly on $t$) will be treated as constants. 


\subsection{Proof of Theorem~\ref{thm:SELAMP}}

The proof will be by induction over $t$. The base case is clear, so we focus on the inductive step. We assume the statements above for $t-1$ and prove them for $t$. 

\subsubsection{Proof of $(a)$}

Note that $\cP_t^{\perp}(\bA^{(k)})$ is by construction independent of $\cF_t$, and therefore we can replace $\bA^{(k)}$
by a fresh independent matrix in \eqref{eq:LAMP1}, whence \eqref{eq:conditional} follows. The equality \eqref{eq:CondExp} holds by definition of the iteration.


\subsubsection{Proof of $(b)$: Approximate isometry}

We will repeatedly apply Lemma~\ref{lem:4}. We start with  \eqref{eq:c1}. As we are inducting on $t$, we may limit ourselves to considering overlaps
$\vR( \qq^{k,t+1},\qq^{k,t_1+1})$, for $t_1\le t$. 

Define the tensor $\Gamma^{(k),\nabla}\in (\bbR^{\sS}_{\geq 0})^{\otimes (k-1)}$ by
\begin{equation}
\label{eq:Gamma-nabla-def}
    \Gamma^{(k),\nabla}_{s_1,\dots,s_{k-1}}
    =
    \sqrt{
    k
    \sum_{s\in\sS}
    \lambda_s
    \big(\Gamma^{(k)}_{s,s_1,\dots,s_{k-1}}\big)^2
    }.
\end{equation}
We choose 
\[
    (\f_{t}^{\otimes k-1})_{\parallel}
    \in
    {\rm span}\lt(\f_{t_1}^{\otimes k-1}\rt)_{t_1< t}
\]
such that
\[
    \Gamma^{(k),\nabla}\diamond \big(\f_{t}^{\otimes k-1}\big)_{\parallel}
\]
is the orthogonal projection of $\Gamma^{(k),\nabla}\diamond \f_{t}^{\otimes k-1}$ onto 
\[
    {\rm span}\lt(\Gamma^{(k),\nabla}\diamond\f_{t_1}^{\otimes k-1}\rt)_{t_1< t}
\]
and also set
\[
    (\f_{t}^{\otimes k-1})_{\perp}=\f_{t}^{\otimes k-1}-(\f_{t}^{\otimes k-1})_{\parallel}.
\]

We will use (and soon after, prove) the following lemma.
\begin{lemma}
\label{lem:LemmaPerp}
For all $t_1\leq t_1$, we have
\begin{equation}
\label{eq:LemmaPerp2}
    \cP_t^{\perp} (\tbA^{(k)})\{(\f_t^{\otimes k-1})_{\perp}\} \simeq
    \tbA^{(k)}\{(\f_t^{\otimes k-1})_{\perp}\}\, .
\end{equation}
\end{lemma}
%
For $t_1\le t-1$, using Lemma~\ref{lem:4}, point 2 implies
\begin{align*}
    \vR(\qq^{k,t+1},\qq^{k,t_1+1}) 
    \simeq 
    \vR\big(\E[\qq^{k,t+1}|\cF_t],\, \qq^{k,t_1+1}\big)
\end{align*}
%
We next use the formula in $(a)$ for $\E[\qq^{k,t+1}|\cF_t]$ together with the expression in \eqref{eq:LAMP1}. For each $s\in\sS$:
\begin{align}
\nonumber
    R\big(\mathbb E[\qq^{k,t+1}|\mathcal F_t],\qq^{k,t_1+1}\big)_s
    &\simeq
    R\Bigg(
    \sum_{0\leq t_2,t_3\leq t-1}
    \qq^{k,t_3+1}(\bG_{\xi^{k,s},t-1}^{-1})_{t_3,t_2}
    \,
    \xi^{k,s}\big(\f_{t_2},\f_{t}\big)
    ,\, \qq^{k,t_1+1}
    \Bigg)_s
    \\
\nonumber
    &= 
     \sum_{0\leq t_2,t_3\leq t-1}
     R\big(\qq^{k,t_3+1},\qq^{k,t_1+1}\big)_s
     \,
     (\bG_{\xi^{k,s},t-1}^{-1})_{t_3,t_2} 
     \,
     \xi^{k,s}\big(\f_{t_2},\f_{t}\big)
    \\
\label{eq:3rd-ineq}
    &\simeq
    \sum_{0\leq t_2,t_3\leq t-1} 
    (\bG_{\xi^{k,s},t-1})_{t_3,t_1}
    \,
    (\bG_{\xi^{k,s},t-1}^{-1})_{t_3,t_2} 
    \,
    \xi^{k,s}\big(\f_{t_2},\f_{t}\big)
    \\
\nonumber
    &=
    \big(
    \bG_{\xi^{k,s},t-1}
    \times 
    \bG_{\xi^{k,s},t-1}^{-1}
    \times
    \bG_{\xi^{k,s},t-1}
    \big)_{t_1,t}
    \\
\nonumber
    &=
    (\bG_{\xi^{k,s},t-1})_{t_1,t}.
%
\end{align}
Here \eqref{eq:3rd-ineq} comes from the induction hypothesis \eqref{eq:c1} (and the symmetry of the matrix $\bG_{\xi^{k,s},t-1}$ is used to obtain the next line). We next prove that \eqref{eq:c1} holds for $t_1=t$. We have by definition of the projections that
%
\begin{align*}
\cP_t^{\perp} (\tbA^{(k)})\{\f_t\}=\cP_t^{\perp} (\tbA^{(k)})\{(\f_t^{\otimes k-1})_{\perp}\} \, ,
\end{align*}
%
where the right-hand side is defined according to \eqref{eq:TensorTensor}.
%
Using \eqref{eq:LemmaPerp2} from Lemma~\ref{lem:LemmaPerp} as well as point~4 of Lemma~\ref{lem:4}, we have
%
\begin{align}
\label{eq:PAnorm}
    R\lt(
    \cP_t^{\perp} (\tbA^{(k)})\{\f_t\}
    \,,
    \cP_t^{\perp} (\tbA^{(k)})\{\f_t\}
    \rt)_s
    \simeq 
    \xi^{(k,s)}\lt(
    R\big(
    (\f_{t}^{\otimes k-1})_{\perp}
    \,,
    (\f_{t}^{\otimes k-1})_{\perp}
    \big)
    \rt).
\end{align}
%
Next, using \eqref{eq:LemmaPerp2} and Lemma~\ref{lem:4} (point 2), we obtain that for all $s\in\sS$
%
\begin{equation}
\label{eq:ApproxOrth}
    R\big(\cP_t^{\perp} (\tbA^{(k)})\{\f_t\},\E[\qq^{k,t+1}|\cF_t]\big)_s \simeq 0\, .
\end{equation}
%
Moreover we recall that by the expression for $\E[\qq^{k,t+1}|\cF_t]$ from part $(a)$,
%
\begin{align}
%
    R\lt(
    \E[\qq^{k,t+1}|\mathcal F_t]
    \,,
    \E[\qq^{k,t+1}|\mathcal F_t]
    \rt)
    \simeq
    \xi^{k,s}
    \lt(
    \vR(
    (\f_t^{\otimes k-1})_{\parallel}
    \,,
    (\f_t^{\otimes k-1})_{\parallel}
    )
    \rt).
%
\end{align}
%
The formula for linear regression implies
%
\begin{align}
\label{eq:lin-reg-1}
%
    (\f_{t}^{\otimes k-1})_{\parallel}
    & =
    \sum_{0\leq t_1\leq t-1} \alpha_{t_1,t} \Gamma^{(k,s)}\diamond \f_{t_1}^{\otimes k-1},
    \\
    \alpha_{t_1,t}
    &=
    \sum_{0\leq t_2\le t-1} 
    (\bG_{\xi^{k,s},t-1}^{-1})_{t_1,t_2}
    \< 
    \Gamma^{(k,s)}\diamond \f_{t_2}^{\otimes k-1},
    \Gamma^{(k,s)}\diamond \f_{t}^{\otimes k-1}
    \>_N
    \\
    &=
    \sum_{0\leq t_2\le t-1} 
    (\bG_{\xi^{k,s},t-1}^{-1})_{t_1,t_2}
   (\bG_{\xi^{k,s},t}^{-1})_{t_2,t}
    \, .
%
 \end{align}
%

By part $(b)$ of the inductive step, for $1\leq t_1,t_2\leq t-1$ we have
\[
    \xi^{k,s}\big(\vR( \f_{t_2},\f_{t_1})\big)
    \simeq 
    R_s( \qq_{k,t_2+1},\qq_{k,t_1+1})
    \,.
\]
In particular the formulas \eqref{eq:LAMP2} and \eqref{eq:lin-reg-1} have asymptotically the same coefficients, and the overlap structure between the summands is identical. It follows that
\begin{align}
\label{eq:CEnorm}
    R\lt(
    \E[\qq^{k,t+1}|\mathcal F_t],
    \,,
    \E[\qq^{k,t+1}|\mathcal F_t]
    \rt)
    \simeq
    R\lt(
    (\f_{t}^{\otimes k-1})_{\parallel}
    \,,
    (\f_{t}^{\otimes k-1})_{\parallel}
    \rt).
\end{align}
%
Using together Eqs.~\eqref{eq:PAnorm}, \eqref{eq:ApproxOrth}, and \eqref{eq:CEnorm}, we get
%
\begin{align*}
%
    R\lt(\qq^{k,t+1},\qq^{k,t+1}\rt)
    &\simeq 
    R\lt(\E[\qq^{k,t+1}|\mathcal F_t]
    \,,
    \E[\qq^{k,t+1}|\mathcal F_t]
    \rt)
    + 
    R\lt(
    (\f_t^{\otimes k-1})_{\perp}
    \,,
    (\f_t^{\otimes k-1})_{\perp}
    \rt)
    \\
    &\simeq
    R\lt(
    (\f_t^{\otimes k-1})_{\perp}
    \,,
    (\f_t^{\otimes k-1})_{\perp}
    \rt)
    \\
    &=
    \xi^{k,s}\lt(
    \vR( \f_t,\f_t)
    \rt)
    \,.
%
 \end{align*}
%
This establishes \eqref{eq:c1}.

Next consider \eqref{eq:c2}, i.e., approximate orthogonality of $\qq^{k,r}$ and $\qq^{p',r}$ for $k\neq p'.$ This follows easily from the representation
in point $(a)$ which, together with Lemma~\ref{lem:4}, inductively implies that the iterates $\qq^{s,k}$ for different $k$ are approximately orthogonal.
Finally, \eqref{eq:c3}  follows directly from \eqref{eq:c1} and \eqref{eq:c2}.
We now prove Lemma~\ref{lem:LemmaPerp}.

\begin{proof}[Proof of Lemma~\ref{lem:LemmaPerp}]
For convenience we write $\tbA=\tbA^{(k)}$. By Lagrange multipliers, there exist vectors
    $(\btheta_{t_1})_{t_1 \le t-1}$ in $\bbR^N$ such that $\cP_t^{\perp} (\tbA) = \tbA - \bQ$, where 
\begin{align*}
  \bQ = 
  \big(\Gamma^{(k)}\big)^{\odot 2}
  \diamond
  \frac{(k-1)!}{N^{k-1}}\sum_{t_1=0}^{t-1} \sum_{j=1}^k  \underbrace{\f_{t_1}\otimes \cdots \otimes \f_{t_1}}_{\mbox{$j-1$ times}}\otimes \btheta_{t_1} \otimes \underbrace{\f_{t_1}\otimes \cdots \otimes \f_{t_1}}_{\mbox{$k-j$ times}}.
\end{align*}
%
The vectors $(\btheta_{t_1})_{t_1 \le t-1}$ are determined by the equations
% $\cP_t^{\perp} (\tbA) \{\f_{t_1}\} = 0$
$\bQ \{\f_{t_1}\}=\tbA \{\f_{t_1}\}$
for all $t_1\le t-1$.
This expands (for each $t_1\leq t-1$) to
%
\begin{align*}
  %
  \sum_{t_2\leq t-1} 
  (\bG_{\xi^{k,s},t-1})_{t_1,t_2}
  \diamond 
  \btheta_{t_2}
  +
%   (k-1)
  \sum_{t_2\leq t-1}
  (\bGG_{\xi^{k,s}})_{t_1,t_2}
  \diamond
  \<\f_{t_1},\btheta_{t_2}\>_N \f_{t_2}
  = 
  \tbA\{\f_{t_1}\}\, .
  %
\end{align*}
%
Recall that we assume each $\bG_{\xi^{k,s},t-1}$ is well conditioned with high probability. Thus we can multiply the system of $t$ equations above by $\bG_{\xi^{k,s},t-1}^{-1}$ in the coordinates $\cI_s$ for each $s\in\sS$. For each $t_3\leq t-1$, we obtain:
%
\begin{align}
\label{eq:LambdaEq}
  \btheta_{t_3}
  +
  \sum_{t_1,t_2<t}
  \lt(
  (\bG_{\xi^{k,s},t-1}^{-1})_{t_1,t_3} (\bGG_{\xi^{k,s},t-1})_{t_1,t_2}
  \rt)
  \diamond
  \<\f_{t_1},\btheta_{t_2}\>_N \f_{t_2}=
  \sum_{t_1<t}
  (\bG_{\xi^{k,s},t-1}^{-1})_{t_1,t_3}
  \diamond
  \tbA\{\f_{t_1}\}\, . 
  %
\end{align}
%
Switching $t_3$ to $t_1$, we find
%
\begin{align}
\nonumber
  %
  \btheta_{t_1} &=   \btheta^0_{t_1}+ \btheta^{\parallel}_{t_1}\, ,
  \\
\label{eq:theta0}
  \btheta^0_{t_1}
  &\equiv
  \sum_{t_2<t}
  (\bG_{\xi^{k,s},t-1}^{-1})_{t_1,t_2}\diamond \tbA\{\f_{t_2}\} \, ,
  \\
\nonumber
  \btheta^{\parallel}_{t_1}
  &\in 
  \spn\lt((\f_{t_2,s})_{t_2<t,s\in\sS}\rt).
  %
\end{align}
%
We claim that $\|\btheta^{\parallel}_{t_1}\|_N\simeq 0$, i.e.,
$\btheta_{t_1}\simeq   \btheta^0_{t_1}$. Indeed, let $\bTheta\in\bbR^{N\times t}$ be the matrix with columns
$(\btheta_{t_2})_{t_2<t}$, and $\bTheta^0$ the matrix with columns $(\btheta^0_{t_2})_{t_2<t}$.
Then \eqref{eq:LambdaEq} can be written as
%
\begin{align*}
  %
  \cL_{k,t}^{\sT}(\bTheta) =\bTheta^0\, .
  %
\end{align*}
%
Here we recall $\cL_{k,t}=\bfone+\cT_{k,t}$ and $\cT_{k,t}\in\bbR^{Nt\times Nt}$ is defined in \eqref{eq:Tdef}.
Substituting the decomposition 
$\bTheta = \bTheta^0+\bTheta^{\parallel}$ in the above, we obtain
%
\begin{align*}
  %
  \cL_{k,t}^{\sT}(\bTheta^{\parallel}) =-\cT^{\sT}_{k,t}(\bTheta^0)\, .
  %
\end{align*}
%
Recall that $\cL_{k,t}$ is well conditioned by Assumption~\ref{as:well-conditioned}. Therefore it remains to prove 
\begin{equation}
\label{eq:key-lamp-computation}
    \cT^{\sT}_{k,t}(\bTheta^0)\stackrel{?}{\simeq} 0.
\end{equation}
%
Let $\bc_0,\cdots,\bc_{t-1} \in \bbR^N$ be the columns of $\cT^{\sT}_{k,t}(\bTheta^0)$. We first note that for all $t_1\leq t-1$ and $s\in\sS$, 
\[
    \bc_{t_1,s} \in \spn((\f_{t_2,s})_{t_2<t}).
\]
Moreover the Gram matrix 
\[
    \bG_{1,t-1,s}=\lt(R_s(\f_{t_1},\f_{t_2})\rt)_{t_1,t_2<t}
\]
is well conditioned for each $s\in\sS$. Therefore it is sufficient to check that $R_s(\f_{t_1},\bc_{t_4})\simeq 0$ for each $t_1,t_4<t$ and $s\in\sS$. Plugging in the definition \eqref{eq:Tdef}, it remains to check
%
\begin{align*}
  \sum_{t_2,t_3<t}
  \lt\<
  \f_{t_1}
  ,
  \,
  (\bG_{\xi^{k,s},t-1}^{-1})_{t_2,t_3} 
  (\bGG_{\xi^{k,s},t-1})_{t_4,t_3}
  \diamond
  \btheta^0_{t_2}
  \rt\>_N 
  R_s(\f_{t_1},\f_{t_3})
  \stackrel{?}{\simeq} 
  0\, 
  ,
  \quad
  \forall~ 0\leq t_1,t_4\leq t-1\,.
\end{align*}
%
Finally, this last claim follows by substituting the definition \eqref{eq:theta0} of $\btheta^0_{t_2}$, and using the fact that  
\[
    R_s(\f_{t_1},\tbA\{\f_{t_3}\})\simeq 0,\quad
    \forall~ t_1,t_3\le t,\,s\in\sS
\]
which follows from Lemma \ref{lem:4}. Thus \eqref{eq:key-lamp-computation} is established.



We are now ready to prove Lemma~\ref{lem:LemmaPerp}. 
First note that 
\begin{equation}
\label{eq:god}
     \tbA\{(\f_t^{\otimes k-1})_{\perp}\} 
    -
    \cP_t^{\perp} 
    (\tbA)
    \{(\f_t^{\otimes k-1})_{\perp}\} 
    = 
    \bQ\{(\f_t^{\otimes k-1})_{\perp}\}
\end{equation}
decomposes into two types of terms based on the definition of $\bQ$ above. Recalling \eqref{eq:Gamma-nabla-def}, the first involves
\[
    \lt\langle 
    \Gamma^{(k),\nabla}
    \diamond 
    (\f_{t_1}^{\otimes k-1})_{\perp}
    ,
    \Gamma^{(k),\nabla}
    \diamond 
    (\f_t^{\otimes k-1})_{\perp}
    \rt\rangle_N
    \btheta_{t_1}
\]
for $t_1\leq t-1$, which vanishes by the definition of $(\f_t^{\otimes k-1})_{\perp}$. The other terms take the form
\[
    \lt\langle 
    \Gamma^{(k),\nabla}
    \diamond 
    (\btheta_{t_1}\otimes \f_{t_1}^{\otimes k-2})
    ,\,
    \Gamma^{(k),\nabla}
    \diamond 
    (\f_t^{\otimes k-1})_{\perp}
    \rt\rangle_N
    ~
    \f_{t_1}.
\]
In particular, this means that to prove \eqref{eq:god} vanishes, suffices to show
\[
    R\lt(
    \bQ\{(\f_t^{\otimes k-1})_{\perp}\}
    ,\,
    \f_{t_2}
    \rt)
    =0
\]
for all $t_2\leq t$.


Note that by construction,
\[
   (\f_t^{\otimes k-1})_{\perp}
   =
   \sum_{t_1\leq  t} b_{t_1} \f_{t_1}^{\otimes k-1}
    \,.
\]
By the well-conditioning assumption, the $b_{t_1}$ are bounded. Therefore it suffices to show that 
\[
    R\lt(
    \bQ\{\f_{t_2}^{\otimes k-1}\}
    ,\,
    \f_{t_1}
    \rt)
    \stackrel{?}{\simeq} 
    0,
    \quad 
    \forall\, t_1\leq t-1,\,t_2\leq t.
\]
Finally note that each term in the left-hand side includes an overlap $R_s(\btheta_{t_1},\f_{t_2})$. However these all vanish:
\[
    R_s(\btheta_{t_1},\f_{t_2})
    \simeq
    0.
\]
This is because we can substitute $\btheta_{t_1}$ with $\btheta_{t_1}^0$ as defined in \eqref{eq:theta0} and use the fact that $\vR(\tbA\{\f_{t_3}\},\f_t)\simeq 0$ which follows from Lemma \ref{lem:4}. This completes the proof.
\end{proof}


\subsubsection{Proof of $(c)$}

Recall that the process $(U^{k,t}_s)_{t\ge 1}$ is Gaussian by construction, and independent of $U^{k,0}_s$. Define 
\begin{align*}
    C_{t_1,t_2,s} &= \E\big[U^{k,t_1}_s U^{k,t_2}_s\big]\,;
    \\
    \bC_{\le t,s} &= (C_{t_1,t_2,s})_{t_1,t_2\le t}\,.
\end{align*}
We then have
%
\begin{equation} 
\label{eq:wt-alpha}
\begin{aligned}
    \E[U^{k,t+1}_s| U^{k,0}_s,\dots,U^{k,t}_s]
    &=
    \sum_{t_1=1}^t 
    \wt\alpha_{t_1,s} U^{k,t_1}
    \, ;
    \\
    \wt\alpha_{t_1,s}
    & \equiv
    \sum_{t_2=1}^t
    (\bC^{-1}_{\le t,s})_{t_1,t_2}C_{t_2,t+1,s}
    \, .
\end{aligned}
\end{equation} 
Here in writing $(\bC^{-1}_{\le t,s})_{t_1,t_2}$, we view $\bC_{\le t,s}$ as a $(t+1)\times (t+1)$ matrix for each $s\in\sS$.
%

On the other hand, from point $(a)$, we know that
%
\begin{equation} 
\label{eq:alpha}
\begin{aligned}
    \E[\qq^{k,t+1}_{s}|\mathcal F_t]
    & = 
    \sum_{1\leq t_1\leq t}  
    \alpha_{t_1,s} 
    \qq^{t_1,k}_s\, ;
    \\
    \alpha_{t_1,s}
    & \equiv
    \sum_{t_2=1}^t
    (\bG^{-1}_{\xi^{k,s},t-1})_{t_1-1,t_2-1} 
    (\bG_{\xi^{k,s},t})_{t_2-1,t}
    \, .
\end{aligned}
\end{equation} 
%
Moreover the induction hypothesis of \eqref{eq:ConvergenceLAMP} implies that for $t_1,t_2 \le t$,
%
\begin{equation}
\label{eq:augment-e-for-W2}
    (\bG_{\xi^{k,s},t})_{t_1,t_2} 
    \simeq 
    \bbE\lt[
    \xi^{k,s}
    \lt(
    f_{t_1}(W^0_s,\dots,W^{t_1}_s;E^0_s,\dots,E^{t_1}_s), f_{t}(W^0_s,\dots,W^{t_2}_s;E^0_s,\dots,E^{t_2}_s)\}
    \rt)
    \rt]
    \, .
\end{equation}
%
(Recall that by definition $W^t_s \equiv \sum_{k\le D} U^{k,t}_s$, while $\f_t=f_t(\VV_t;\bE_t)$ here.) 


Therefore, from the definition of the process $(U^{k,t}_s)_{t\ge 0}$,
\[
    (\bG_{\xi^{k,s},t})_{t_1,t_2} \simeq C_{t_1+1,t_2+1,s}
    ,
    \quad\quad \forall t_1,t_2\le t.
\]
Recalling that $\bG_{\xi^{k,s},t}$ is well conditioned, we find (recall \eqref{eq:wt-alpha},\eqref{eq:alpha}):
\[
    \alpha_{t_1,s}\simeq \wt\alpha_{t_1,s}.
\]
Therefore we also have 
%
\begin{align*}
    \E[\qq^{k,t+1}|\mathcal F_t] - \sum_{t_1=1}^t \wt\alpha_{t_1}\diamond \qq^{k,t_1}
    &=
    \sum_{t_1=1}^t (\alpha_{t_1}-\wt\alpha_{t_1}) \diamond \qq^{k,t_1}
    \\
    &\simeq 
    0.
%
\end{align*}
%
Moreover, Lemma~\ref{lem:4} (point 4) shows that $\cP^{\perp}_t(\tbA^{(k)})\{\f_t\}\simeq\tbA^{(k)}\{(\f_{t}^{\otimes k-1})_{\perp}\}$ 
has entries which are approximately independent Gaussian with variance 
\[
    \sigma^2_{t,s}\equiv 
    \lt(
    \Gamma^{(k),\nabla}
    \diamond
    (\f_{t}^{\otimes k-1})_{\perp}
    ,\,
    \Gamma^{(k),\nabla}
    \diamond
    (\f_{t}^{\otimes k-1})_{\perp}
    \rt)
\]
on coordinates $i\in\cI_s$, even conditionally on $\cF_t$. 
Therefore
%
\begin{align}
\label{eq:ReprSE}
  %
  \qq^{k,t+1} &\ed \sum_{t_1=1}^t \wt\alpha_{t_1} \diamond\qq^{k,t_1} +\sigma_t \diamond \bg + \bferr^{k,t+1}\, ,
  %
\end{align}
%
where $\|\bferr\|_N\simeq 0$ and $\bg\sim\normal(\bfzero,\id_N)$ is independent of everything else.
It now remains to verify that this agrees with the desired covariance.
As proved in the previous point, for any $t_1\le t$,
%
\begin{align*}
  %
  R\lt(\qq^{k,t+1} ,\qq^{k,t'+1}\rt)_s
  &\simeq 
  \xi^{k,s}\lt(
  \f_t,\f_{t'}
  \rt)
  \\
  &\simeq
  \E\big[ 
  U^{k,t+1}_s U^{k,t'+1}_s
  \big]\, .
%
\end{align*}
%
Therefore, in order to  prove \eqref{eq:ConvergenceLAMP}, it is sufficient to consider $\psi:\bbR^{D \times (t+1)}\to\bbR$  Lipschitz.
Using the representation \eqref{eq:ReprSE}, and focusing for simplicity on a single $k$, we get
%
\begin{align*}
  %
  \frac{1}{N_s}
  \sum_{i\in\cI_s}
  \psi\big(\qq_i^{k,\le t}, q_i^{k,t+1}
  ;
  \be^{\leq t}_i
  \big) 
  &\simeq 
  \frac{1}{N_s}
  \sum_{i\in\cI_s}
  \psi\lt(
      \qq_i^{k,\le t},
      \sum_{s=1}^t 
      \wt\alpha_s \qq^{k,s} 
      +
      \sigma_tg_i
      ;
    \be^{\leq t}_i
  \rt)
  \\
  &\simeq 
  \frac{1}{N_s}
  \sum_{i\in\cI_s}
  \bbE^{g\sim\cN(0,1)}
  \psi\lt(
  \qq_i^{k,\le t},
  \sum_{s=1}^t 
  \wt\alpha_s 
  \qq^{k,s} 
  +
  \sigma_t g
  ;
  \be^{\leq t}_i
  \rt)
  \, .
  %
\end{align*}
%
The second equality above follows by Gaussian concentration. Applying induction hypothesis now implies \eqref{eq:ConvergenceLAMP}, except that $\be^{t+1}$ is not present. 
However since $\be^{t+1}_i$ and $E^{t+1}_s$ have the same law and are both independent of the past, $\bbW_2$ convergence immediately transfers (this is the only step of the proof that involves the vectors $\be^t$ at all). This completes the proof.


\subsection{Asymptotic Equivalence of AMP and Long AMP}

Here we show that AMP and LAMP produce approximately the same iterates. 
%
\begin{lemma}
\label{lem:ampequalslamp}
Let $\{\bG^{(k)}\}_{k\le D}$ be standard Gaussian tensors, and $\bA^{(k)} = \Gamma^{(k)}\diamond \bG^{(k)}$ for $k\ge 2$. Consider the corresponding AMP
iterates $\ZZ_{t}\equiv (\bz^{k,t_1})_{k\le D,t_1\le t}$ and LAMP iterates $\bQ_{t}\equiv (\qq^{k,t_1})_{k\le D,t_1\le t}$,
from the same initialization $\ZZ_0=\bQ_0$ satisfying the assumptions of Theorem \ref{thm:mixedAMP}
and Theorem \ref{thm:SELAMP}.

Let $\f_t = f_t(\VV_t;\bE_t)$, $t\ge 0$ be the nonlinearities applied to LAMP iterates.
% and $(\bG_{k,t}(\VV))_{r,s}=\<\f_t,\f_{t'}\>^k$ be the corresponding Gram matrices.
Further assume that there exists a constant $C<\infty$ such that, for all $t\le T$,
%
\begin{itemize}
%
\item[$(i)$] The LAMP Gram matrices $\bG_{k,t} = \bG_{k,t}$ are
  well-conditioned as guaranteed by Assumption~\ref{as:well-conditioned}, i.e., 
 \[
    C^{-1}\le \sigma_{\min}(\bG_{k,t})\le \sigma_{\max}(\bG_{k,t})\le C,
    \quad\quad
    \forall k\le D,~t\le T\,.
\]
%
\item[$(ii)$] Let the linear operator $\cT_{k,t}:\bbR^{N\times t}\to\bbR^{N\times t}$ be defined as per \eqref{eq:Tdef}, with $\bG_{k,t} = \bG_{k,t}(\VV)$,
and $\f_t=f_t(\VV,\bE_t)$, and define $\cL_{k,t} = \bone+\cT_{k,t}$. Then 
\[
    C^{-1}\le \sigma_{\min}(\cL_{k,t})\le \sigma_{\max}(\cL_{k,t})\le C.
\]
%
\end{itemize}
%
Then, for any $t\le T$, we have
%
\begin{align}
%
\|\ZZ_{t} - \bQ_{t}\|_N\simeq 0 \, .
%
\end{align}
%
\end{lemma}
%
\begin{proof}
Throughout the proof we will suppress $\bE_t$ and simply write $f_t(\bW_t)$ or $f_t(\bV_t)$ to distinguish AMP and LAMP iterates, and analogously for
$\bG_{k,t}(\bW_t)$ or $\bG_{k,t}(\bV_t)$.
The proof is by induction over the iteration number, so we will assume it to hold at iteration $t$, 
and prove it for iteration  $t+1$. We prove the induction step by establishing the following two facts for each $2\leq k\leq D$:
%
\begin{align}
%
\big\|\AMP_{t+1}(\ZZ_{t})_k -\AMP_{t+1}(\bQ_{t})_k \big\|_N&\simeq 0 \, ,\label{eq:LAMPapprox1}\\
%
\big\|\AMP_{t+1}(\bQ_{t})_k -\LAMP_{t+1}(\bQ_{t})_k \big\|_N &\simeq 0\, . \label{eq:LAMPapprox2}
%
\end{align}

Let us first consider the claim \eqref{eq:LAMPapprox1}, and note that
%
\begin{align*}
    \AMP_{t+1}(\ZZ_{t})_k -\AMP_{t+1}(\bQ_{t})_k  &= \bA^{(k)}\{f_t(\bW_t)\}-\bA^{(k)}\{f_t(\bV_t)\} 
    \\
    &\quad - 
    \sum_{t_1\leq t} 
    d_{t,t_1,k}
    \diamond
    \big(f_{t_1-1}(\bW_{t_1-1})-f_{t_1-1}(\bV_{t_1-1})\big)\, ,
\end{align*}
%
where we wrote $d_{t,t_1,k,s}$ for the coefficients of \eqref{eq:AMP-def2}, with AMP iterates replaced by LAMP iterates. 
We then have 
%
\begin{align*}
%
    \big\|\AMP_{t+1}(\ZZ_{t})_k -\AMP_{t+1}(\bQ_{t})_k \big\|_N 
    &\le 
    D_{1,t}+D_{2,t}\, ;
    \\
    D_{1,t} 
    & \equiv 
    \big\| \bA^{(k)}\{f_t(\bW_t)\}-\bA^{(k)}\{f_t(\bV_t)\}\big\|_N\, ,
    \\
    D_{2,t} 
    &\equiv 
    \sum_{t_1\leq t,~s\in\sS} 
    |d_{t,t_1,k,s}| 
    \cdot 
    \big\|f_{t_1-1,s}(\bW_{t_1-1})- f_{t_1-1,s}(\bV_{t_1-1})\big\|_N\, .
%
\end{align*}
%
Notice that, by the induction assumption (and recalling that each $f_{t,s}$ is Lipschitz continuous and acts component-wise): 
%
\begin{equation}
\label{eq:InductionF}
%
    \big\|f_t(\bW_t)-f_t(\bV_t)\big\|_N
    \le C_T
    \sum_{t_1\le t,~k\le D}\|\bw^{k,t_1}-\bv^{k,t_1}\|_N 
    \simeq 
    0
    \, .
%
\end{equation}
%
Further, for any tensor $\bT\in(\bbR^{N})^{\otimes k}$, and any vectors $\bv_1,bv_2\in\bbR^N$,
%
\begin{align}
%
\|\bT\{\bv_1\}-\bT\{\bv_{2}\}\|_N\le (N^{\frac{k-2}{2}}\|\bT\|_{\op})  (\|\bv_1\|_N+\|\bv_2\|_N)^{k-2} \|\bv_1-\bv_2\|_N
%
\end{align}
%
Using Lemma~\ref{lem:4}, this implies that the following bound holds with high probability for a constant $C$:
%
\begin{align*}
    D_{1,t} 
    &\le C  
    (\|f_t(\bW_t)\|_N+\|f_t(\bV_t)\|_N)^{k-2} \|f_t(\bW_t)-f_t(\bV_t)\|_N\\
    & \le 
    C  (2\|f_t(\bV_t)\|_N+\|f_t(\bW_t) -f_t(\bV_t)\|_N)^{k-2} \|f_t(\bW_t) -f_t(\bV_t)\|_N
    \\
    &\simeq 0.
\end{align*}
%
The last step follows from \eqref{eq:InductionF} and Theorem~\ref{thm:SELAMP}, which implies (recall each $f_{t,s}$ is Lipschitz) that
$\|f_t(\bV_t)\|_N\le C$ with probability $1-o(1)$. Notice that the same argument implies $\|f_t(\bW_t)\|_N \le C$ with high probability.

Similarly, $D_{2,t}\simeq 0$ follows since $\|f_{t_1-1}(\bW_{t_1-1})- f_{t_1-1}(\bV_{t_1-1})\|_N\simeq 0$ and $|d_{t,t_1,k,s}|\le C_T$ by construction, thus yielding \eqref{eq:LAMPapprox1}.


We now prove \eqref{eq:LAMPapprox2}. Comparing \eqref{eq:AMP-def2} and \eqref{eq:LAMP1}, with
$\cP_t^{\parallel} = \bfone-\cP_t^{\perp}$ we find
%
\begin{align}
\label{eq:AMP-LAMP}
    \AMP_{t+1}(\bQ_{t})_k 
    -
    \LAMP_{t+1}(\bQ_{t})_k 
    &= 
    \cP_t^{\parallel} (\bA^{(k)})\{f_t(\bV_t)\}
    -
    \ons_{k,t+1}
    -
    \sum_{0\leq t_1\leq t-1}  
    h_{t,t_1,k}\diamond \qq^{k,t_1+1}\, ,
    \\
\nonumber
    \ons_{k,t+1} 
    &= 
    \sum_{t_1\leq t} 
    d_{t,t_1,k}\diamond f_{t_1-1}(\bV_{t_1-1})
%
\end{align}
%
Note that $\cP_t^{\parallel} (\bA^{(k)})=\E\lt[\bA^{(k)}|\cF_t\rt]$, where $\cF_t$ is the $\sigma$-algebra generated by 
$\{\bq ^{k,s}\}_{t_1\le t, k\le D}$. Equivalently, this is the conditional expectation of $\bA^{(k)}$ given the linear constraints
%
\begin{align}
%
\A^{(k)}\{f_{t_1}(\bV_{t_1})\}&=\by_{k,t_1+1}\, ,\;\;\;\;\;\mbox{ for } t_1\in \{0,\dots, t-1\}\, ,
%
\end{align}
%
Also notice that, by the induction hypothesis, and the definition of $\by_{k,t_1}$, \eqref{eq:Ydef1}, we have for all $t_1\le t$, 
%
\begin{equation}
\label{eq:Y-ons}
    \by_{k,t_1} \simeq  \qq^{k,t_1}+\ons_{k,t_1}\, .
%
\end{equation}
%
Lemma~\ref{lem:symregression} implies that $\cP_t^{\parallel} (\bA^{(k)})$ takes the form of \eqref{eq:SymmRegression} for a suitable matrix
$\hZZ_{k,t}\in\bbR^{N\times t}$. 
The key claim is that
%
\begin{equation}
\label{eq:ZequalsQ}
    \hZZ_{k,t} \simeq \bQ_t\, .
%
\end{equation}
%
In order to establish this claim, we show that,  under the inductive hypothesis,
\begin{equation}
\label{eq:1+TQ=Y}
    (\bfone+\cT_{k,t})\bQ_t\simeq \bY_{k,t}.
\end{equation}
Since $\cL_{k,t}=\bfone +\cT_{k,t}$ is well-conditioned by assumption, the combination of \eqref{eq:ZZ-eq} and \eqref{eq:1+TQ=Y} implies $\hZZ_{k,t}\simeq \bQ_t$. By \eqref{eq:Y-ons}, in order to prove \eqref{eq:1+TQ=Y}, it is sufficient to show that 
\begin{equation}
\label{eq:this-claim}
    \cT_t\bQ_t
    \simeq 
    \ONS_{k,t}
    \equiv 
    [\ons_{k,1}|\cdots|\ons_{k,t}].
\end{equation}

In order to prove \eqref{eq:this-claim}, we use Theorem~\ref{thm:SELAMP}. Recall that 
\begin{align*}
    C_{t_1,t_2,s} &= \E\{U^{k,t_1}_s U^{k,t_2}_s\}\,,
    \\
    W^{t_1}_s &= \sum_{2\leq k\leq D} U^{k,t_1}_s\,,
    \\
    \bC_{\le t} &= (C_{t_1,t_2,s})_{t_1,t_2\le t}\,.
\end{align*}
(The value $2\leq k\leq D$ is implicitly fixed in the definition of $\bC_{\leq t}$.) By Theorem~\ref{thm:SELAMP}, 
\[
    C_{t_1+1,t_2+1} \simeq \<\qq^{k,t_1+1},\qq^{k,t_2+1}\> 
    \simeq 
    (\bG_{\xi^{k,s},t}(\bV))_{t_1,t_2}
    ,
    \quad\forall~t_1,t_2\le t.
\]
This implies for any $0 \le t_1\le t-1$ and $s\in\sS$,
%
\begin{align}
\nonumber
    \sum_{t_2=0}^{t-1}
    (\bG_{\xi^{k,s},t-1}^{-1})_{t_1,t_2}
    R\lt(\qq^{k,t_2+1},f_{t-1}(\VV_{t-1})\rt)_s
    &\simeq 
    \sum_{t_2=0}^{t-1} 
    (\bC_{\le t,s}^{-1})_{t_1+1,t_2+1} 
    \E\lt[
    U^{k,t_2+1}_s f_{t-1,s}(W^0,\dots,W^{t-1})
    \rt] 
    \\
\label{eq:Stein}
    &= 
    \E\lt[
        \frac{\partial f_{t-1,s}}{\partial W^{t_1+1}_s} (W^0_s,\dots,W^{t-1}_s)
    \rt] 
    \bfone_{t_1\le t-2}\, .
%
\end{align}
%
Indeed, Gaussian integration by parts yields the latter expression. Combining \eqref{eq:Stein} with the definition \eqref{eq:AMP-def2} will now allow us to conclude $\cT_{k,t}\bQ_t\simeq \ONS_{k,t}$ as desired. Indeed for each $s\in\sS$ we have
\begin{align*}
    \big[\cT_{k,t}\bQ_t\big]_{t,s} 
    &= 
    \sum_{t_1=0}^{t-1} 
    (\bGG_{\xi^{k,s},t-1})_{t_1,t-1} 
    \f_{t_1,s}
    \Big(
    \sum_{t_2=0}^{t-1} 
    (\bG_{\xi^{k,s},t-1}^{-1})_{t_1,t_2} 
    \,
    R_s(\qq^{k,t_2+1}, \f_{t-1})
    \Big)
    \\
    &\simeq 
    \sum_{t_1=0}^{t-2}
    \zeta^{k,s}
    \big(
    \vR(\f_{t_1}, \f_{t-1})
    \big)
    \f_{t_1,s}
    \cdot 
    \E\lt[
    \frac{\partial f_{t-1,s}}{\partial W^{t_1+1}_s} 
    (W^0_s,\dots,W^{t-1}_s)
    \rt]
    \\
    &= 
    \ons_{k,t}.
\end{align*}
Having established \eqref{eq:ZequalsQ}, we now use the formula \eqref{eq:SymmRegression} for $\cP^{\parallel}_t(\bA^{(k)})=\E\big[\bA^{(k)}|\cF_t\big]$. The result is:
%
\begin{align}
\label{eq:AParallel}
    \cP^{\parallel}_{t}(\bA^{(k)})\{\f_t\}
    &\simeq 
    \sum_{t_1\leq t}
    \big(\alpha_{t_1}
    \diamond
    \qq^{k,t_1}
    +
    \beta_{t_1}
    \diamond
    \f_{t_1}\big)\, ;
    \\
\nonumber
     \alpha_{t_1,s}
     &= 
     \sum_{0\leq t_2\le t-1} 
     (\bG_{\xi^{k,s},t-1}^{-1})_{t_1,t_2} 
     \,
     \xi^{k,s}\big(
     \vR(
     f_{t_2}(\VV_{t_2}),f_{t}(\VV_t)
     )
     \big)
     \, ,
     \\
\nonumber
    \beta_{t_1,s} 
    &= 
    \Big(
    \sum_{0\le t_2\leq t-1} 
    (\bG_{\xi^{k,s},t-1}^{-1})_{t_1,t_2} 
    \,
    R_s( \qq^{k,t_2},\f_{t} )
    \Big) 
    \zeta^{k,s}\big(
    \vR( \f_{t_1},\f_{t} )
    \big).
%
\end{align}
On the other hand, using again \eqref{eq:Stein} gives
%
\begin{align*}
    \sum_{t_1\leq t}\beta_{t_1}\diamond \f_{t_1} 
    &\simeq  
    \sum_{t_1 \le t-1}  d_{t,t_1,k}\diamond \f_{t_1-1}  
    =
    \ons_{k,t+1},
\\
    \sum_{t_1\leq t}
    \alpha_{t_1}
    \diamond
    \qq^{k,t_1} 
    &\simeq 
    \sum_{0\leq t_1\leq t-1}  
    h_{t,t_1,k}
    \diamond
    \qq^{k,t_1+1}.
\end{align*}
%
We conclude from \eqref{eq:AMP-LAMP} that $\|\AMP_{t+1}(\bQ_{t})_k -\LAMP_{t+1}(\bQ_{t})_k  \|_N\simeq 0$. This concludes the proof.
\end{proof}



\end{document}
