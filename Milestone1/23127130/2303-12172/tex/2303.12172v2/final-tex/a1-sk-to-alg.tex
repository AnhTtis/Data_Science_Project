\section{Ground State Energy of Multi-Species Spherical SK With External Field}
\label{sec:sk-ext-field}

We adopt the notations of Lemma~\ref{lem:sk-ext-field}.
In this section, we will prove this lemma by showing that 
\[
    \limsup_{N\to\infty} \bbE \GS_N(W,\vv,k)
    \le 
    \sum_{s\in \sS}
    \lambda_s
    \sqrt{v_s^2 + 2\sum_{s'\in \sS} \lambda_{s'} w_{s,s'}^2}
    \le 
    \liminf_{N\to\infty} \bbE \GS_N(W,\vv,k).
\]

\subsection{Upper Bound for $\vv=\vzero$, $k=1$}

The following (exact) upper bound for the case $\vv = \vzero$, $k=1$ follows from the results of \cite{bandeira2021matrix}.
We will prove Lemma~\ref{lem:sk-ext-field} using only this result and elementary techniques.

\begin{proposition}
    \label{prop:free-prob}
    For $W$ as in Lemma~\ref{lem:sk-ext-field},
    \[
        \limsup_{N\to\infty} \bbE \GS_N(W,\vzero,1)
        \le
        \sum_{s\in \sS} \lambda_s \sqrt{2\sum_{s'\in \sS} \lambda_{s'} w_{s,s'}^2}.
    \]
\end{proposition}

\begin{proof}
    In this proof, abbreviate $\GS_N = \GS_N(W,\vv,k)$.
    Let $\bG \in \bR^{N\times N}$ have i.i.d. standard Gaussian entries.
    Thus $G = \fr12 \lt(\bG + \bG^\top\rt)$ is symmetric with $\cN(0,1)$ diagonal entries, $\cN(0, 1/2)$ off-diagonal entries, and independent entries on and above the diagonal.
    Define $M\in \bbR^{N\times N}$ by $M_{i,j} = N^{-1/2} w_{s(i),s(j)} G_{i,j}$.
    It is clear by homogeneity that 
    \[
        \GS_N =
        \fr{1}{N} 
        \max_{\bsig \in \cB_N}
        \bsig^\top M \bsig.
    \]
    Let $\vC \in \bbR_{>0}^\sS$ be a vector of constants we will set later.
    We consider the rescaled matrix $\wtM = \sqrt{\vC}^{\otimes 2} \diamond M$.
    This can be generated by $\wtM = \hM + \oM$, where $\hM$ is a random symmetric matrix with independent entries on and above the diagonal 
    \[
        \hM_{i,j} \sim \cN\lt(0, \fr{C_{s(i)}C_{s(j)}w_{s(i),s(j)}^2}{2N}\rt)
    \]
    and $\oM$ is a random diagonal matrix with independent entries
    \[
        \oM_{i,i} \sim \cN\lt(0, \fr{C_{s(i)}^2 w_{s(i),s(i)}^2}{2N}\rt).
    \]
    Clearly $\bbE \tnorm{\oM}_{\op} = O(\sqrt{N^{-1} \log N})$. 
    \cite[Theorem 1.2]{bandeira2021matrix} states that
    \[
        \bbE\tnorm{\hM}_{\op}
        \leq
        \tnorm{X_{\free}}_{\op}+O\lt(v^{1/2}\sigma^{1/2}(\log N)^{3/4}\rt),
    \]
    where $\tnorm{X_{\free}}_{\op}, \sigma, v$ are defined as follows.
    We have
    \[
        \sigma
        =
        \sqrt{\bbE\tnorm{\hM^2}_{\op}}=O(1),
        \qquad
        v
        =
        \sqrt{\tnorm{\Cov(\hM)}_{\op}},
    \]
    where $\Cov(\hM)\in \bbR^{N^2\times N^2}$ is the covariance matrix of the entries of $\hM$ and has operator norm $O(1/N)$. 
    It follows that the error term $v^{1/2}\sigma^{1/2}(\log N)^{3/4}$ contributes $o_N(1)$. Finally \cite[Lemma 3.2]{bandeira2021matrix} states that in our setting,
    \[
        \norm{X_{\free}}_{\op}=
        2\sup_{\substack{a\in [0,1]^N\\ \sum_i a_i=1}}
        \sum_{i\in [N]}
        \sqrt{a_{i}\sum_{i'\in [N]} \frac{C_{s(i)}C_{s(i')}w_{s(i),s(i')}^2 a_{i'}}{2N}}
    \]
    It is not difficult to see by concavity of the square-root that, for $\lambda_{s,N} = |\cI_s|/N$ (so $\lambda_{s,N} \to \lambda_s$) replacing all $a_i$ such that $i\in \cI_s$ with
    \[
        A_s=\lambda_{s,N}^{-1}\sum_{i:s(i)=s} a_i
    \]
    only improves the right-hand side. Substituting $B_s=C_sA_s$, we conclude that
    \begin{align*}
        \norm{X_{\free}}_{\op}
        &=
        \sup_{\substack{\vA\in \bbR_{\geq 0}^{\sS}\\ \sum_s \lambda_{s,N} A_s=1}}
        \sum_{s\in\sS}
        \lambda_{s,N}
        \sqrt{2A_s\sum_{s'\in\sS}\lambda_{s'} C_s C_{s'}w_{s,s'}^2 A_{s'}} \\
        &=
        \sup_{\substack{\vB\in \bbR_{\geq 0}^{\sS}\\ \sum_s C_s^{-1}\lambda_{s,N} B_s=1}}
        \sum_{s\in\sS}
        \lambda_{s,N}
        \sqrt{2B_s\sum_{s'\in\sS}\lambda_{s'}  w_{s,s'}^2 B_{s'}}.
    \end{align*}
    From the above discussion, $\tnorm{\wtM}_{\op} \le \tnorm{X_{\free}}_{\op} + o_N(1)$.
    Moreover we observe that
    \begin{align*}
        \GS_N
        &=
        \frac{1}{N}
        \max_{\norm{\bsig_s}_2^2 \le \lambda_s N}
        \bsig^{\top}M\bsig
        =
        \frac{1}{N}
        \max_{\norm{\bsig_s}_2^2 \le C_s^{-1}\lambda_s N}
        \bsig^{\top}\wtM\bsig \\
        &\leq
        \frac{1}{N}
        \max_{\norm{\bsig}_2^2 \le \sum_{s\in\sS} C_s^{-1}\lambda_s N}
        \bsig^{\top}\wtM\bsig
        =
        \lt(\sum_{s\in\sS} C_s^{-1}\lambda_s\rt)
        \tnorm{\wtM}_{\op} 
        \,.
    \end{align*}
    Combining and using homogeneity, we find
    \begin{align}
        \notag
        \bbE \GS_N
        &\leq 
        \lt(\sum_{s\in\sS} C_s^{-1}\lambda_s\rt)
        \bbE \tnorm{\wtM}_{\op} \\
        \notag
        &=
        \lt(\sum_{s\in\sS} C_s^{-1}\lambda_s\rt)
        \sup_{\substack{
            \vB \in \bbR_{\geq 0}^{\sS} \\ 
            \sum_s C_s^{-1}\lambda_{s,N} B_s=1
        }}
        \sum_{s\in\sS}
        \lambda_{s,N}
        \sqrt{2B_s \sum_{s'\in\sS} \lambda_{s',N} w_{s,s'}^2 B_{s'}}
        +o_N(1) \\
        \label{eq:rvh-supremum}
        &=
        \fr{\sum_{s\in\sS} C_s^{-1}\lambda_s}
        {\sum_{s\in\sS} C_s^{-1}\lambda_{s,N}}
        \cdot 
        \sup_{\substack{
            \vD \in \bbR_{\geq 0}^{\sS} \\ 
            \sum_s C_s^{-1} \lambda_{s,N} D_s = \sum_{s\in\sS} C_s^{-1} \lambda_{s,N}
        }}
        \sum_{s\in\sS}
        \lambda_{s,N}
        \sqrt{2 D_s \sum_{s'\in\sS} \lambda_{s',N} w_{s,s'}^2 D_{s'}}
        +o_N(1).
    \end{align}
    If the supremum in \eqref{eq:rvh-supremum} is attained at $\vD = \vone$, then (because $\lambda_{s,N} \to \lambda_s$) we get the desired bound
    \[
        \bbE \GS_N 
        \le 
        \sum_{s\in\sS} \lambda_s \sqrt{2\sum_{s'\in\sS} \lambda_{s'} w_{s,s'}^2} 
        + o_N(1).
    \]
    Crucially, we observe that the expression 
    \begin{equation}
    \label{eq:concave-dude}
        F(\vD)
        =
        \sum_{s\in\sS}
        \lambda_{s,N}
        \sqrt{2 D_s \sum_{s'\in\sS} \lambda_{s',N} w_{s,s'}^2 D_{s'}}
    \end{equation}
    is concave in $\vD$. 
    Therefore if $\vD=\vone$ is a critical point of $F$ within the set satisfying $\sum_s C_s^{-1} \lambda_{s,N} D_s = \sum_{s\in\sS} C_s^{-1} \lambda_{s,N}$, then it also attains the supremum in \eqref{eq:rvh-supremum}.
    For the choice $C_s=\frac{\lambda_{s,N}}{\partial_{D_s} F}$, $\vD=\vone$ is a critical point of $F$. 
    This concludes the proof.
\end{proof}

\subsection{General Upper Bound}

In this subsection, we will prove the following upper bound for the case $k=1$.
\begin{proposition}
    \label{prop:sk-ub-one-rep}
    For $W, \vv$ as in Lemma~\ref{lem:sk-ext-field},
    \[
        \limsup_{N\to\infty} \bbE \GS_N(W,\vv,1)
        \le
        \sum_{s\in \sS} \lambda_s \sqrt{v_s^2 + 2\sum_{s'\in \sS} \lambda_{s'} w_{s,s'}^2}.
    \]
\end{proposition}
By slight abuse of notation, let $H_N = H_{N,1}^1$ and $\GS_N(W,\vv) = \GS_N(W,\vv,1)$.
Recall that 
\[
    H_N(\bsig)
    =
    \la \vv \diamond \bg, \bsig \ra
    + \wtH_N(\bsig),
    \qquad
    \wtH_N(\bsig)
    =
    \fr{1}{\sqrt{N}}
    \la W \diamond \bG, \bsig^{\otimes 2} \ra 
\]
where $\bg \in \bbR^N$, $\bG\in \bbR^{N\times N}$ have i.i.d. standard Gaussian entries. 
Define
\[
    A(W,\vv) = \limsup_{N\to\infty} 
    \bbE \GS_N(W, \vv).
\]
We first establish some basic properties of this limit.
\begin{lemma}
    \label{lem:A-basic-properties}
    $A$ satisfies the following properties.
    \begin{enumerate}[label=(\alph*), ref=\alph*]
        \item \label{itm:A-homogenity} For any $c>0$, $A(cW, c\vv) = cA(W,\vv)$.
        \item \label{itm:A-linear} $A(0,\vv) = \sum_{s\in \sS} \lambda_s v_s$.
        \item \label{itm:A-quadratic} $A(W,\vzero) \le \sum_{s\in \sS} \lambda_s \sqrt{2\sum_{s'\in \sS} \lambda_{s'} w_{s,s'}^2}$.
        \item \label{itm:A-subadditive} $A(W,\vv) \le A(W,\vzero) + A(0,\vv)$.
    \end{enumerate}
\end{lemma}
\begin{proof}
    Part (\ref{itm:A-homogenity}) is obvious. Part (\ref{itm:A-linear}) follows from
    \[
        \bbE \GS_N(0,\vv) 
        =
        \fr{1}{N} 
        \bbE \max_{\bsig \in \cS_N}
        \la \vv \diamond \bg, \bsig \ra
        =
        \fr{1}{N} 
        \sum_{s\in \sS}
        \sqrt{\lambda_s N} v_s
        \bbE \norm{\bg_s}_2
        =
        \sum_{s\in \sS}
        \lambda_s v_s 
        + o_N(1).
    \]
    Part (\ref{itm:A-quadratic}) follows from Proposition~\ref{prop:free-prob}.
    Part (\ref{itm:A-subadditive}) follows from 
    \begin{align}
        \notag
        \GS_N(W,\vv)
        &=
        \fr{1}{N} 
        \max_{\bsig \in \cS_N}
        \lt(
            \la \vv \diamond \bg, \bsig \ra + 
            \fr{1}{\sqrt{N}} 
            \la W \diamond \bG, \bsig^{\otimes 2} \ra
        \rt) \\
        \notag
        &\ge
        \fr{1}{N} 
        \max_{\bsig \in \cS_N}
        \la \vv \diamond \bg, \bsig \ra 
        +
        \fr{1}{N} 
        \max_{\bsig \in \cS_N}
        \fr{1}{\sqrt{N}} 
        \la W \diamond \bG, \bsig^{\otimes 2} \ra \\
        \label{eq:A-subadditive}
        &=
        \GS_N(W,\vzero) + \GS_N(0,\vv).
    \end{align}
\end{proof}

Next we show some a priori regularity conditions on $A$.

\begin{proposition}
    \label{prop:sk-concentration}
    Let
    \[
        C(W,\vv)
        =
        4 \lt(
            \sum_{s\in \sS}
            \lambda_{s}
            v_s^2 
            +
            \sum_{s,s'\in \sS}
            \lambda_{s}
            \lambda_{s'}
            w_{s,s'}^2
        \rt).
    \]
    Then, for sufficiently large $N$ and all $t>0$,
    \[
        \bbP\lt[
            \lt|\GS_N(W,\vv) - \bbE \GS_N(W,\vv)\rt| > t
        \rt]
        \le 
        2\exp\lt(-\fr{Nt^2}{C(W,\vv)}\rt).
    \]
\end{proposition}
\begin{proof}
    Let $C = C(W,\vv)$.
    For any $\bsig \in \cS_N$, 
    \begin{align*}
        \bbE H_N(\bsig)^2 
        &= 
        \norm{\vv \diamond \bsig}_2^2 + \fr{1}{N} \norm{W \diamond \bsig^{\otimes 2}}_F^2 \\
        &=
        N \lt(
            \sum_{s\in \sS}
            \lambda_{s,N}
            v_s^2 
            +
            \sum_{s,s'\in \sS}
            \lambda_{s,N}
            \lambda_{s',N}
            w_{s,s'}^2
        \rt) 
        \le 
        \fr{CN}{2}.
    \end{align*}
    for large enough $N$.
    By the Borell-TIS inequality, $\max_{\bsig \in \cS_N} H_N(\bsig)$ is $CN/2$-subgaussian, so $\GS_N(W,\vv)$ is $C/2N$-subgaussian, which implies the result.
\end{proof}

For $\va = (a_s)_{s\in \sS'} \in [0,1]^\sS$, define  $W(W,\vv,\va) = (w'_{s,s'})_{s,s'\in \sS}$ and $\vv(W,\vv,\va) = (v'_s)_{s\in \sS}$ where
\[
    w'_{s,s'} 
    = 
    \sqrt{(1-a_s)(1-a_{s'})} 
    w_{s,s'},
    \qquad
    v'_s 
    = 
    \sqrt{
        2(1-a_s) \lt(
            \sum_{s'\in \sS} 
            \lambda_{s'} a_{s'} 
            w_{s,s'}^2
        \rt)
    }.
\]
We will prove Proposition~\ref{prop:sk-ub-one-rep} using the following recursive upper bound in $A$.
\begin{lemma}
    \label{lem:A-fn-ineq}
    For $W, \vv$ as in Lemma~\ref{lem:sk-ext-field},
    \begin{equation}
        \label{eq:A-fn-ineq}
        A(W,\vv) 
        \le 
        \max_{\va \in [0,1]^{\sS}}
        \sum_{s\in \sS}
        \lambda_s v_s \sqrt{a_s}
        +
        A\lt(W(W,\vv,\va), \vv(W,\vv,\va)\rt).
    \end{equation}
\end{lemma}
\begin{proof}
    Define $\hbg \in \cS_N$ by $\hbg_s = \fr{\sqrt{\lambda_s N} \bg_s}{\norm{\bg_s}_2}$ for each $s\in \sS$.
    For $\va\in [0,1]^{\sS}$, define
    \[
        \GS_N(W,\vv;\va)
        = 
        \fr{1}{N} 
        \max_{\bsig \in \cR_N(\va)}
        H_N(\bsig),
        \qquad
        \cR_N(\va) = \lt\{
            \bsig \in \cS_N: 
            R(\bsig, \hbg) = \sqrt{\va}
        \rt\}.
    \]
    For a non-random $\va$ and any $\bsig \in \cR_N(\va)$,
    \[
        \la \vv \diamond \bg, \bsig \ra
        =
        N \sum_{s\in \sS}
        \lambda_s v_s \sqrt{a_s}
        \fr{\norm{\bg_s}_2}{\sqrt{\lambda_s N}}.
    \]
    For $\bsig\in \cR_N(\va)$, we may write $\bsig = \sqrt{\va} \diamond \hbg + \sqrt{\vone-\va} \diamond \brho$ for $\brho \in \cR_N(\vzero)$.
    Define the Gaussian process $\hH_N^{\va}(\brho) = \wtH_N\big(\sqrt{\va} \diamond \hbg + \sqrt{\vone - \va} \diamond \brho\big)$, which is supported on $\cR_N(\vzero)$.
    We next calculate the covariance of this process.
    Recall that the covariance of $\wtH_N$ is 
    \[
        \bbE \wtH_N(\bsig)\wtH_N(\bsig') = N\xi(R(\bsig,\bsig')), 
        \qquad
        \xi(\vx) = \lt\la W \odot W, (\vlam \odot \vx)^{\otimes 2} \rt\ra.
    \]
    Because $\bg, \bG$ are independent, the covariance of $\hH_N^{\va}$ is
    \begin{equation}
        \label{eq:sk-hH-covariance}
        \bbE \hH_N^{\va}(\brho)\hH_N^{\va}(\brho')
        = 
        N\xi_{\va}(R(\brho,\brho')), 
    \end{equation}
    where, for $W' = W(W, \vv, \va)$ and $\vv' = \vv(W,\vv,\va)$,
    \begin{align}
        \notag
        \xi_{\va}(\vx) 
        &= 
        \xi\lt(\va + (\vone - \va) \odot \vx\rt)
        = 
        \lt\la
            W \odot W,
            (\vlam \odot \va + \vlam \odot (1-\va) \odot \vx)^{\otimes 2}
        \rt\ra \\
        \label{eq:sk-hH-covariance-calculation}
        &= 
        \lt\la 
            W' \odot W',
            (\vlam \odot \vx)^{\otimes 2}
        \rt\ra
        +
        \lt\la
            \vv' \odot \vv', \vlam \odot \vx
        \rt\ra
        +
        \lt\la
            W \odot W,
            (\vlam \odot \va)^{\otimes 2}
        \rt\ra.
    \end{align}
    We may construct a Gaussian process $\oH_N^{\va}$ (conditional on $\bg$) on $\cS_N$ with covariance \eqref{eq:sk-hH-covariance} whose restriction to $\cR_N(\vzero)$ agrees with $\hH_N^{\va}$.
    Thus
    \begin{align*}
        \GS_N(W,\vv;\va)
        &=
        \sum_{s\in \sS}
        \lambda_s v_s \sqrt{a_s} \fr{\norm{\bg_s}_2}{\sqrt{\lambda_s N}}
        +
        \fr{1}{N} 
        \max_{\brho \in \cR_N(\vzero)}
        \hH_N^{\va}(\brho) \\
        & \le 
        \sum_{s\in \sS}
        \lambda_s v_s \sqrt{a_s} \fr{\norm{\bg_s}_2}{\sqrt{\lambda_s N}}
        +
        \fr{1}{N} 
        \max_{\brho \in \cS_N}
        \oH_N^{\va}(\brho).
    \end{align*}
    Moreover, 
    \[
        \fr{1}{N} \max_{\brho \in \cS_N} \oH_N^{\va}(\brho)
        =_d 
        \GS(W(W,\vv,\va), \vv(W,\vv,\va)) + 
        \fr{1}{\sqrt{N}} 
        \lt\la
            W \odot W,
            (\vlam \odot \va)^{\otimes 2}
        \rt\ra^{1/2}
        Z
    \]
    for an independent $Z \sim \cN(0, 1)$.
    Let $\cD = \{0, \fr{1}{N}, \ldots, \fr{N-1}{N}, 1\}^\sS$.
    Let $\cE$ be the event that
    \begin{enumerate}[label=(\alph*), ref=\alph*]
        \item \label{itm:condition-lipschitz} For a constant $L$, $H_N(\bsig)$ is $L\sqrt{N}$-Lipschitz on $\bsig \in \cS_N$. 
        By Proposition~\ref{prop:gradients-bounded}, this occurs with probability $1-\exp(-CN)$.
        \item For all $s\in \sS$, $|\norm{\bg_s}_2 - \sqrt{\lambda_{s,N}N}| \le N^{1/4}$; by standard concentration inequalities this holds with probability $1-r\exp(-CN^{1/2})$.
        \item For all $\va \in \cD$, $|\fr{1}{N} \max_{\brho \in \cS_N} \oH_N^{\va}(\brho) - \bbE \GS_N(W(W,\vv,\va), \vv(W,\vv,\va))| \le N^{-1/4}$; by Proposition~\ref{prop:sk-concentration} and standard tail bounds on $Z$ this holds with probability $1-2(N+1)^r \exp(-CN^{1/2})$.
        Here we use that for $\va \in \cD$, the constants $C(W(W,\vv,\va), \vv(W,\vv,\va))$ in Proposition~\ref{prop:sk-concentration} are uniformly upper bounded.
    \end{enumerate}
    By adjusting $C$, $\bbP(\cE) \ge 1-\exp(-CN^{1/2})$.
    On $\cE$, if $\bsig\in \cS_N$ maximizes $H_N$, we can find $\bsig' \in \bigcup_{\va \in \cD} \cR_N(\va)$ with $\norm{\bsig'-\bsig}_2 \le O(1/\sqrt{N})$.
    By the Lipschitz condition (\ref{itm:condition-lipschitz}), $|H(\bsig)-H(\bsig')| \le O(1)$.
    So, 
    \begin{align*}
        \GS_N(W,\vv)
        &= 
        \fr{1}{N}
        H_N(\bsig)
        \le 
        \fr{1}{N} H_N(\bsig') + O(1/N) \\
        &\le 
        \max_{\va \in \cD}
        \GS_N(W,\vv;\va) + O(1/N) \\
        &\le 
        \max_{\va \in \cD} \lt(
            \sum_{s\in \sS} \lambda_s v_s \sqrt{a_s}
            +
            \bbE \GS_N(W(W,\vv,\va), \vv(W,\vv,\va))
        \rt)
        + o_N(1).
    \end{align*}
    The subgaussianity from Proposition~\ref{prop:sk-concentration} implies that the contribtion to $\bbE \GS_N(W,\vv)$ from $\cE^c$ is $o_N(1)$, so 
    \begin{align*}
        \bbE \GS_N(W,\vv)
        &\le 
        \max_{\va \in \cD} \lt(
            \sum_{s\in \sS} \lambda_s v_s \sqrt{a_s}
            +
            \bbE \GS_N(W(W,\vv,\va), \vv(W,\vv,\va))
        \rt)
        + o_N(1) \\
        &\le 
        \max_{\va \in [0,1]^\sS} \lt(
            \sum_{s\in \sS} \lambda_s v_s \sqrt{a_s}
            +
            \bbE \GS_N(W(W,\vv,\va), \vv(W,\vv,\va))
        \rt)
        + o_N(1).
    \end{align*}
    Taking $\limsup_{N\to\infty}$ on both sides yields the result.
\end{proof}

\begin{proof}[Proof of Proposition~\ref{prop:sk-ub-one-rep}]
    We will show that any $A$ satisfying the properties in Lemma~\ref{lem:A-basic-properties} and the bound \eqref{eq:A-fn-ineq} must satisfy
    \[
        A(W,\vv) 
        \le 
        A_*(W,\vv)
        \equiv
        \sum_{s\in \sS} 
        \lambda_s
        \sqrt{v_s^2 + 2\sum_{s'\in \sS} \lambda_{s'} w_{s,s'}^2}.
    \]
    Clearly $A_*$ satisfies the conclusions of Lemma~\ref{lem:A-basic-properties}, with equality in assertion (\ref{itm:A-quadratic}).
    For any $\va \in [0,1]^{\sS}$,
    \begin{align*}
        A_*\lt(W(W,\vv,\va),\vv(W,\vv,\va)\rt)
        &= 
        \sum_{s\in \sS}
        \lambda_s
        \sqrt{
            2(1-a_s) \sum_{s'\in \sS} a_{s'} \lambda_{s'} w_{s,s'}^2 + 
            2\sum_{s'\in \sS} \lambda_{s'} (1-a_s)(1-a_{s'}) w_{s,s'}^2
        } \\
        &= 
        \sum_{s\in \sS}
        \lambda_s
        \sqrt{
            2(1-a_s) \sum_{s'\in \sS} \lambda_{s'} w_{s,s'}^2
        },
        \\
        \implies
        \sum_{s\in \sS}
        \lambda_s v_s \sqrt{a_s}
        +
        A_*\lt(W(W,\vv,\va),\vv(W,\vv,\va)\rt)
        &= 
        \sum_{s\in \sS}
        \lambda_s \lt(
            \sqrt{a_s} v_s + 
            \sqrt{1-a_s}
            \sqrt{2\sum_{s'\in \sS} \lambda_{s'} w_{s,s'}^2} 
        \rt) \\
        &\le 
        \sum_{s\in \sS}
        \lambda_s \sqrt{
            v_s^2 +
            2\sum_{s'\in \sS} \lambda_{s'} w_{s,s'}^2
        } 
        = 
        A_*(W,\vv)
    \end{align*}
    by Cauchy-Schwarz. Equality holds when
    \begin{equation}
        \label{eq:sk-a-opt}
        a_s = \fr{v_s^2}{v_s^2 + 2\sum_{s'\in \sS} \lambda_{s'} w_{s,s'}^2}
    \end{equation}
    for all $s\in \sS$, and so $A_*$ satisfies \eqref{eq:A-fn-ineq} with equality.
    
    Suppose $A$ satisfies the conclusions of Lemma~\ref{lem:A-basic-properties} and the inequality \eqref{eq:A-fn-ineq}, and there exists $(W,\vv)$ with $A(W,\vv) > A_*(W,\vv)$.
    By homogeneity (Lemma~\ref{lem:A-basic-properties}(\ref{itm:A-homogenity})), we can assume $1 = \norm{W}_1 \equiv \sum_{s,s'\in \sS} w_{s,s'}$.
    For any small $\delta > 0$, we may choose $(W^*, \vv^*)$ such that $\norm{W^*}_1 = 1$ and 
    \[
        A(W^*,\vv^*) - A_*(W^*,\vv^*)
        \ge 
        (1-\delta)
        \sup_{(W,\vv) : \norm{W}_1=1}
        \lt(
            A(W,\vv) - A_*(W,\vv)
        \rt) 
        > 0.
    \]
    Set 
    \[
        \va^* 
        = 
        \argmax_{\va \in [0,1]^\sS}
        \sum_{s\in \sS}
        \lambda_s v^*_s \sqrt{a_s} + A\lt(W(W^*,\vv^*,\va), \vv(W^*,\vv^*,\va)\rt),
    \]
    and $W' = W(W^*,\vv^*,\va^*), \vv' = \vv(W^*,\vv^*,\va^*)$, so
    \begin{align*}
        A(W^*,\vv^*) 
        &\le 
        \sum_{s\in \sS} 
        \lambda_s v^*_s \sqrt{a_s^*} +
        A(W',\vv'), \\
        A_*(W^*,\vv^*)
        &\ge 
        \sum_{s\in \sS} 
        \lambda_s v^*_s \sqrt{a_s^*} +
        A_*(W',\vv').
    \end{align*}
    Here, the second inequality uses that $A_*$ satisfies \eqref{eq:A-fn-ineq} with equality.
    Therefore
    \begin{equation}
        \label{eq:A-max-diff}
        A(W',\vv') - A_*(W',\vv')
        \ge 
        A(W^*,\vv^*) - A_*(W^*,\vv^*)
        \ge 
        (1-\delta)
        \sup_{(W,\vv) : \norm{W}_1=1}
        \lt(
            A(W,\vv) - A_*(W,\vv)
        \rt).
    \end{equation}
    By homogeneity, this implies $\norm{W'}_1 \ge 1-\delta$.
    Let $\sS_0 \subseteq \sS$ be the set of $s$ for which there exists $s'$ with $w_{s,s'} \ge \delta_1 \equiv \sqrt{2\delta}$.
    For such $s,s'$, 
    \[
        \delta 
        \ge 
        \norm{W^*}_1 - \norm{W'}_1
        \ge 
        w^*_{s,s'} - w'_{s,s'}
        \ge 
        \lt(1 - \sqrt{1-a_s}\rt) w_{s,s'}
        \ge 
        \fr12 a_s w_{s,s'}
        \ge 
        \fr12 a_s \delta_1.
    \]
    Thus, for $s\in \sS_0$, $a_s \le \delta_1$.
    Of course, for $s\in \sS \setminus \sS_0$, $w_{s,s'} \le \delta_1$ for all $s'\in \sS$.
    Thus for all $s\in \sS$, 
    \[
        v'_s \le \sqrt{2\sum_{s'\in \sS} \lambda_{s'} \delta_1} = \sqrt{2\delta_1} \equiv \delta_2.
    \]
    By parts (\ref{itm:A-subadditive}), (\ref{itm:A-linear}), and (\ref{itm:A-quadratic}) of  Lemma~\ref{lem:A-basic-properties},
    \[
        A(W',\vv') 
        \le 
        A(W',\vzero) + A(0,\vv')
        \le 
        A(W',\vzero) + \delta_2
        \le 
        A_*(W', \vzero) + \delta_2.
    \]
    By inspection, $A_*(W',\vv') \ge A_*(W',\vzero)$.
    Thus
    \[
        A(W',\vv') - A_*(W',\vv')
        \le 
        \delta_2.
    \]
    For small enough $\delta > 0$, this contradicts \eqref{eq:A-max-diff}.
\end{proof}

Finally, the upper bound for $k=1$ directly implies the upper bound for general $k$.
\begin{corollary}
    \label{cor:sk-ub}
    For $W, \vv$ as in Lemma~\ref{lem:sk-ext-field},
    \[
        \limsup_{N\to\infty} \bbE \GS_N(W,\vv,k)
        \le
        \sum_{s\in \sS} \lambda_s \sqrt{v_s^2 + 2\sum_{s'\in \sS} \lambda_{s'} w_{s,s'}^2}.
    \]
\end{corollary}
\begin{proof}
    Note that (recall \eqref{eq:bbtperp})
    \[
        \GS_N(W,\vv,k)
        = 
        \fr{1}{kN} 
        \max_{\vbsig \in \cS_N^{k,\perp}} 
        H_{N,k}
        \le 
        \fr{1}{k} 
        \sum_{i=1}^k
        \fr{1}{N}
        \max_{\bsig^i \in \cS_N} 
        H^i_{N,k}(\bsig^i).
    \]
    Taking expectations yields $\bbE \GS_N(W,\vv,k) \le \bbE \GS_N(W,\vv,1)$.
    This and Proposition~\ref{prop:sk-ub-one-rep} imply the result.
\end{proof}

\begin{remark}
    The proof of Proposition~\ref{prop:sk-ub-one-rep} via the recursive inequality \eqref{eq:A-fn-ineq} extends to the ground state energies in multi-species spherical spin glasses with general (non-quadratic) interactions. 
    It thus gives an elementary way to upper bound the ground state energy for spin glasses with external field given the ground state energy of spin glasses without external field, when the latter is known. 
    As we will see in the next subsection, it is possible to construct points where this recursive inequality holds with (approximate) equality, so the upper bound is sharp.
\end{remark}


\subsection{Lower Bound}

In this subsection, we will constructively prove the matching lower bound to Corollary~\ref{cor:sk-ub}.
\begin{proposition}
    \label{prop:sk-lb}
    For $W,\vv$ as in Lemma~\ref{lem:sk-ext-field}, 
    \[
        \liminf_{N\to\infty} \bbE \GS_N(W,\vv,k)
        \ge
        \sum_{s\in \sS} \lambda_s \sqrt{v_s^2 + 2\sum_{s'\in \sS} \lambda_{s'} w_{s,s'}^2}.
    \]
\end{proposition}

\begin{lemma}   
    \label{lem:sk-gram-schmidt}
    Let $S_N = \{\bx\in \bbR^N : \norm{\bx}_2 = \sqrt{N}\}$.
    Suppose $\by^1,\ldots,\by^k \in S_N$ satisfy $|\la \by^i, \by^j\ra| \le N^{2/3}$ for all $i\neq j$. 
    Then there exist pairwise orthogonal $\bz^1,\ldots,\bz^k \in S_N$ such that $\Span(\bz^1,\ldots,\bz^k) = \Span(\by^1,\ldots,\by^k)$ and $\la \by^i,\bz^i\ra \ge N - 4kN^{1/3}$.
\end{lemma}
\begin{proof}
    We define $\bz^1,\ldots,\bz^k$ by applying the Gram-Schmidt algorithm to $\by^1,\ldots,\by^k$: let $\bz^1=\by^1$, and for $2\le i\le k$, let
    \[
        \tby^i = \by^i - \sum_{j=1}^{i-1} \fr{\la \bz^j, \by^i\ra}{N} \bz^j,
        \qquad
        \bz^i = \fr{\sqrt{N}}{\norm{\tby^i}_2} \tby^i.
    \]
    Clearly $\Span(\bz^1,\ldots,\bz^k) = \Span(\by^1,\ldots,\by^k)$.
    We will prove by induction on $i$ that for all $\ell >i$, $|\la \bz^i, \by^\ell\ra| \le 2N^{2/3}$.
    The base case $i=1$ is true by hypothesis.
    For $i>1$, we have
    \[
        \norm{\tby^i}_2^2
        =
        N \lt(1 - \sum_{j=1}^{i-1} \fr{\la \bz^j, \by^i\ra^2}{N}\rt)
        \in 
        \lt[N(1-4kN^{-2/3}),N\rt],
    \]
    using the inductive hypothesis.
    Moreover, for $\ell >i$, 
    \[
        |\la \tby^i, \by^\ell\ra|
        \le 
        |\la \by^i, \by^\ell\ra|
        +
        \sum_{j=1}^{i-1} \fr{|\la \bz^j, \by^i\ra| |\la \bz^j, \by^\ell\ra|}{N}
        \le 
        N^{2/3}\lt(1 + 4kN^{-1/3}\rt).
    \]
    Therefore
    \[
        |\la \bz^i, \by^\ell\ra| 
        \le 
        N^{2/3} 
        \lt(1 + 4kN^{-1/3}\rt) \lt(1-kN^{-2/3}\rt)^{-1/2}
        \le 2N^{2/3},
    \]
    completing the induction.
    Now $\la \tby^i, \by^i\ra = \norm{\tby^i}_2^2$, so 
    \[
        \la \bz^i, \by^i\ra 
        =
        \sqrt{N} \norm{\tby^i}_2
        \ge 
        N \lt(1-4kN^{-2/3}\rt)^{1/2}
        \ge 
        N-4kN^{1/3}.
    \]
\end{proof}
Recall that $\lambda_{s,N} = |\cI_s|/N$.
Let $\delta_N = \max_{s\in \sS} |\fr{\lambda_{s,N}}{\lambda_s} - 1|$.
\begin{lemma}
    \label{lem:sk-multi-gram-schmidt}
    There exists an event $\cE \in \sigma(\bg^1,\ldots,\bg^k)$ with $\bbP(\cE) \ge 1-\exp(-CN^{1/3})$ such that on this event, there exists $\vbx = (\bx^1,\ldots,\bx^k)\in \cS_N^{k,\perp}$ such that the following properties hold.
    \begin{enumerate}[label=(\alph*), ref=\alph*]
        \item \label{itm:x-norms-good} For all $i$, $\norm{R(\bg^i,\bg^i) - \vone}_\infty \le \delta_N + N^{-1/4}$.
        \item \label{itm:x-approx-g} For all $i$, $\norm{R(\bg^i,\bx^i) - \vone}_\infty \le \delta_N + N^{-1/4}$. 
        \item \label{itm:x-unit} For all $i$, $R(\bx^i,\bx^i) = \vone$.
        \item \label{itm:x-span} For all $s\in \sS$, $\Span(\bx^1_s,\ldots,\bx^k_s) = \Span(\bg^1_s,\ldots,\bg^k_s)$.
    \end{enumerate}
\end{lemma}
\begin{proof}
    By standard concentration inequalities, for each $i\in [k]$ and $s\in \sS$, $|\la \bg^i_s, \bg^i_s\ra - \lambda_{s,N} N| \le N^{2/3}$ with probability $1-\exp(-CN^{1/3})$, which implies
    \[
        \lt|\fr{\la \bg^i_s, \bg^i_s\ra}{\lambda_s N}-1\rt|
        \le 
        \lt|\fr{\lambda_{s,N}}{\lambda_s}-1\rt| + \fr{N^{2/3}}{\lambda_s N}
        \le 
        \delta_N + N^{-1/4}.
    \]
    When this holds for all $i\in [k]$, $s\in \sS$, part (\ref{itm:x-norms-good}) follows.
    
    For each $i\in [k]$, define $\hbg^i \in \bbR^N$ by $\hbg^i_s = \fr{\sqrt{\lambda_{s,N} N}}{\norm{\bg^i_s}_2} \bg^i_s$ for all $s\in \sS$.
    Note that each $\hbg^i_s$ is a uniformly random point on the sphere of radius $\sqrt{\lambda_{s,N} N}$ supported on the coordinates $\cI_s$.
    
    Fix $s\in \sS$. 
    By standard concentration inequalities, for each pair of distinct $i, j\in [k]$, $|\la \hbg^i_s, \hbg^j_s\ra| \le (\lambda_{s,N}N)^{2/3}$ with probability $1-\exp(-CN^{1/3})$. 
    If this holds for all $s,i,j$, Lemma~\ref{lem:sk-gram-schmidt} implies the existence of orthogonal $\bz^1_s,\ldots,\bz^k_s$ on the sphere of radius $\sqrt{\lambda_{s,N}N}$ supported on coordinates $\cI_s$ with 
    \begin{equation}
        \label{eq:sk-span}
        \Span(\bz^1_s,\ldots,\bz^k_s) = \Span(\hbg^1_s,\ldots,\hbg^k_s)
    \end{equation}
    and
    \[
        \lambda_{s,N}N - 4k(\lambda_{s,N}N)^{1/3} \le \la \bz^k_s, \hbg^i_s\ra \le \lambda_{s,N}N.
    \]
    Let $\bx^i_s = \bz^i_s \cdot \sqrt{\lambda_s / \lambda_{s,N}}$, so
    \[
        \fr{\la \bx^i_s, \bg^i_s\ra}{\lambda_s N}
        = 
        \fr{\la \bz^i_s, \hbg^i_s\ra}{\lambda_{s,N} N} \cdot \sqrt{\fr{\lambda_{s,N}}{\lambda_s }} \cdot \fr{\norm{\bg^i_s}}{\sqrt{\lambda_{s,N}N}}
        =
        (1 + O(N^{-1/3}) \sqrt{\fr{\lambda_{s,N}}{\lambda_s }}.
    \]
    Thus
    \[
        \lt|\fr{\la \bx^i_s, \bg^i_s\ra}{\lambda_s N}-1\rt|
        \le 
        \lt|\sqrt{\fr{\lambda_{s,N}}{\lambda_s }} - 1\rt| + O(N^{-1/3}) 
        \le 
        \delta_N + N^{-1/4}.
    \]
    If this holds for all $s$, part (\ref{itm:x-approx-g}) follows.
    By a union bound, adjusting $C$ as necessary, the above events simultaneously hold with probability $1-\exp(-CN^{1/3})$.
    By construction, $R(\bx^i,\bx^i) = \vone$ and $R(\bx^i,\bx^j) = \vzero$ for all $i\neq j$, which implies part (\ref{itm:x-unit}) and $\vbx \in \cS_N^{k,\perp}$.
    The relation \eqref{eq:sk-span} implies part (\ref{itm:x-span}).
\end{proof}

The following recursive lower bound for $\bbE \GS_N(W,\vv,k)$ is a converse to Lemma~\ref{lem:A-fn-ineq} and is the main step in the proof of Proposition~\ref{prop:sk-lb}.
\begin{lemma}
    \label{lem:sk-fe-lb}
    Let $W, \vv$ be as in Lemma~\ref{lem:sk-ext-field} and $\va \in [0,1]^\sS$, and set $W' = W(W,\vv,\va)$, $\vv' = \vv(W,\vv,\va)$. Then,
    \[
        \bbE \GS_N(W,\vv,k)
        \ge 
        \sum_{s\in \sS}
        \lambda_s v_s \sqrt{a_s}
        + 
        \bbE \GS_{N-kr}(W',\vv',k)
        - o_N(1),
    \]
    where $\GS_{N-kr}$ denotes the ground state energy (see \eqref{eq:sk-gsn}) of a dimension $N-kr$ multi-species quadratic spin glass with species sizes $\tilde \cI_s = \cI_s - k$.
\end{lemma}
\begin{proof}
    Suppose for now the event $\cE$ in Lemma~\ref{lem:sk-multi-gram-schmidt} holds and let $\vbx = (\bx^1,\ldots,\bx^k)$ be as in this lemma.
    Let
    \begin{equation}
        \label{eq:bbtnperp}
        \cS_{N,\perp} 
        \equiv
        \lt\{
            \brho \in \cS_N : 
            R(\brho, \bx^i) = \vzero
            ~\forall i\in [k]
        \rt\}
        =
        \lt\{
            \brho \in \cS_N : 
            R(\brho, \bg^i) = \vzero
            ~\forall i\in [k]
        \rt\}
    \end{equation}
    where the second equality follows from Lemma~\ref{lem:sk-multi-gram-schmidt}(\ref{itm:x-span}) and 
    \begin{equation}
        \label{eq:bbtnperpkperp}
        \cS_{N,\perp}^{k,\perp}
        \equiv
        \lt\{
            \vbrho = (\brho^1, \ldots, \brho^k) \in \cS_{N,\perp}^k : 
            R(\brho^i,\brho^j) = \vzero ~\forall i\neq j
        \rt\}.
    \end{equation}
    For each $i\in [k]$ let 
    \begin{equation}
        \label{eq:sk-recursive-bsig}
        \bsig^i = \sqrt{\va} \diamond \bx^i + \sqrt{\vone - \va} \diamond \brho^i    
    \end{equation}
    where $\vbrho = (\brho^1, \ldots, \brho^k) \in \cS_{N,\perp}^{k,\perp}$.
    The orthogonality relations in \eqref{eq:bbtnperp} and \eqref{eq:bbtnperpkperp} imply $\vbsig = (\bsig^1, \ldots, \bsig^k) \in \cS_N^{k,\perp}$.
    Then, 
    \begin{equation}
        \label{eq:sk-k-step}
        \fr{1}{N} H_{N,k}(\vbsig)
        =
        \fr{1}{kN}
        \sum_{i=1}^k
        \la \vv \diamond \bg^i, \sqrt{\va} \diamond \bx^i \ra + 
        \fr{1}{kN^{3/2}}
        \sum_{i=1}^k
        \lt\la W \diamond \bG, (\sqrt{\va} \diamond \bx^i + \sqrt{\vone - \va} \diamond \brho^i)^{\otimes 2} \rt\ra.
    \end{equation}
    By Lemma~\ref{lem:sk-multi-gram-schmidt}(\ref{itm:x-norms-good}, \ref{itm:x-approx-g}),
    \[
        \fr{1}{N} 
        \la W \diamond \bG, \sqrt{\va} \diamond \bx^i\ra
        = 
        \sum_{s\in \sS}
        \lambda_s v_s \sqrt{a_s}
        + o_N(1).
    \]
    Note that the state space $\cS_{N,\perp}$ is $\cS_N$ with $k$ fewer dimensions in each species, and these dimensions (and $\vbx$) are independent of $\bG$. 
    So, optimizing the second term of \eqref{eq:sk-k-step} over $\vbrho \in \cS_{N,\perp}^{k,\perp}$ is equivalent to optimizing a dimension $N-kr$ multi-species quadratic spin glass.
    The same covariance calculation as \eqref{eq:sk-hH-covariance-calculation} shows that
    \[
        \sup_{\vbrho \in \cS_{N,\perp}^{k,\perp}}
        \fr{1}{N^{3/2}}
        \lt\la W \diamond \bG, (\sqrt{\va} \diamond \bx^i + \sqrt{\vone - \va} \diamond \brho^i)^{\otimes 2} \rt\ra
        =_d
        \sqrt{\fr{N-kr}{N}}
        \GS_{N-kr}(W', \vv')
        +
        O(N^{-1/2}) Z,
    \]
    where $Z\sim \cN(0,1)$ is independent of $\GS_{N-kr}$.
    Thus
    \[
        \bbE \GS_N(W,\vv,k)
        \ge 
        \bbE \ind\{\cE\} \fr{1}{N} H_{N,k}(\vbsig)
        \ge 
        \sum_{s\in \sS}
        \lambda_s v_s \sqrt{a_s}
        + 
        \bbE \GS_{N-kr}(W', \vv',k)
        - o_N(1).
    \]    
\end{proof}

Lemma~\ref{lem:sk-fe-lb} suggests a natural way to construct an approximate ground state of $H_{N,k}$.
First, use Gram-Schmidt orthogonalization to produce $\vbx = (\bx^1,\ldots,\bx^k)$ from the external fields $\bg^1,\ldots,\bg^k$, as in Lemma~\ref{lem:sk-multi-gram-schmidt}.
Choose $\va \in [0,1]^\sS$ and set $\vbsig$ as in \eqref{eq:sk-recursive-bsig}, for $\vbrho \in \cS_{N,\perp}^{k,\perp}$ to be determined.
The correlations of the $\bsig^i$ with the external fields $\bg^i$ contribute energy $\sum_{s\in \sS} \lambda_s v_s \sqrt{a_s}$, while the optimization over $\vbrho$ is equivalent to optimizing another quadratic multi-species spin glass, whose parameters depend on $\va$.
Finally, recursively optimize $\vbrho$.
The following proof demonstrates that when $\vv > \vzero$, there exists a sequence of choices of $\va$ such that running this algorithm to a large constant recursion depth finds a near ground state $\vbsig \in \cS_N^{k,\perp}$ of $H_{N,k}$.
(If some entries of $\vv$ are zero, the algorithm succeeds after first introducing a small artificial external field.)

\begin{proof}[Proof of Proposition~\ref{prop:sk-lb}]
    Assume for now that $\vv \succ \vzero$ where the inequality is strict in each coordinate.
    Define $W^{(0)} = W$, $\vv^{(0)}=\vv$. 
    Denote the relation \eqref{eq:sk-a-opt} by $\va = \va(W,\vv)$.
    Let $T$ be a large constant to be determined, and for $0\le t\le T-1$ define
    \[
        \va^{(t)} = \va(W^{(t)}, \vv^{(t)}),
        \qquad
        W^{(t+1)} = W(W^{(t)}, \vv^{(t)}, \va^{(t)}),
        \qquad
        \vv^{(t+1)} = \vv(W^{(t)}, \vv^{(t)}, \va^{(t)}).
    \]
    Further define
    \[
        E^{(t)} = \sum_{s\in \sS} \lambda_s \sqrt{(v^{(t)}_s)^2 + 2\sum_{s'\in \sS} \lambda_{s'} (w_{s,s'}^{(t)})^2}, 
        \qquad
        F^{(t)} = \sum_{s\in \sS} \lambda_s v_s^{(t)} \sqrt{a_s^{(t)}}.
    \]
    Let $\delta > 0$ be arbitrary; we will show that $\bbE \GS_N(W,\vv) \ge E^{(0)} - \delta$ for all sufficiently large $N$.
    Lemma~\ref{lem:sk-fe-lb} with the choice $\va = \va^{(t)}$ implies that
    \[
        \bbE \GS_{N-tkr}(W^{(t)}, \vv^{(t)}) 
        \ge 
        F^{(t)} + \bbE \GS_{N-(t+1)kr}(W^{(t+1)}, \vv^{(t+1)}) - o_N(1),
    \]
    and summing yields
    \[
        \bbE \GS_N(W,\vv)
        \ge 
        \sum_{t=0}^{T-1} F^{(t)} - o_N(1).
    \]
    Note that
    \begin{align*}
        F^{(t)} &= \sum_{s\in \sS} \lambda_s \sqrt{(v^{(t)}_s)^2 + 2\sum_{s'\in \sS} \lambda_{s'} (w_{s,s'}^{(t)})^2} \cdot a_s^{(t)}, \\
        E^{(t+1)} &= \sum_{s\in \sS} \lambda_s \sqrt{(v^{(t)}_s)^2 + 2\sum_{s'\in \sS} \lambda_{s'} (w_{s,s'}^{(t)})^2} \cdot (1-a_s^{(t)}), 
    \end{align*}
    so $F^{(t)} = E^{(t)} - E^{(t+1)}$.
    Thus
    \[
        \bbE \GS_N(W,\vv)
        \ge 
        E^{(0)} - E^{(T)} - o_N(1).
    \]
    Since
    \[
        (v_s^{(t+1)})^2 + 2\sum_{s'\in \sS} \lambda_{s'} (w_{s,s'}^{(t+1)})^2 
        = 
        2 (1-a_s^{(t)}) \sum_{s'\in \sS} \sum_{s'} \lambda_{s'} (w_{s,s'}^{(t)})^2
    \]
    we have
    \[
        a_s^{(t+1)} 
        = 
        \fr{(v_s^{(t+1)})^2}{(v_s^{(t+1)})^2 + 2\sum_{s'\in \sS} \lambda_{s'} (w_{s,s'}^{(t+1)})^2}
        =
        \fr{\sum_{s'\in \sS} a_{s'}^{(t)} \lambda_{s'} (w_{s,s'}^{(t)})^2}{\sum_{s'\in \sS} \lambda_{s'} (w_{s,s'}^{(t)})^2}.
    \]
    It follows that, for $\alpha^{(t)} = \min_{s\in \sS} a_s^{(t)}$, we have $\alpha^{(t+1)} \ge \alpha^{(t)}$.
    The assumption $\vv > \vzero$ ensures $\alpha^{(0)} > 0$.
    Because $E^{(t+1)}/E^{(t)} \le 1 - \alpha^{(t)}$, we have
    \[
        E^{(T)} \le E^{(0)} (1 - \alpha^{(0)})^T < \delta/2
    \]
    for sufficiently large constant $T$.
    This implies $\bbE \GS_N(W,\vv) \ge E^{(0)} - \delta/2 - o_N(1) \ge E^{(0)} - \delta$.
    This proves the result when $\vv \succ \vzero$.
    
    If some coordinates of $\vv$ are zero, we apply this result to $\vv + \eta \vone$ for small $\eta > 0$.
    By \eqref{eq:A-subadditive},
    \[
        \GS_N(W,\vv)
        \ge 
        \GS_N(W,\vv+\eta \vone)
        -
        \GS_N(0,\eta \vone),
    \]
    so for sufficiently large $N$,
    \[
        \bbE \GS_N(W,\vv)
        \ge 
        \sum_{s\in \sS}
        \lambda_s 
        \sqrt{v_s^2 + 2\sum_{s'\in \sS} \lambda_{s'} w_{s,s'}^2}
        - \delta - \eta.
    \]
    As this holds for any $\delta, \eta > 0$ the result follows.
\end{proof}


\begin{proof}[Proof of Lemma~\ref{lem:sk-ext-field}]
    Follows from Corollary~\ref{cor:sk-ub} and Proposition~\ref{prop:sk-lb}.
\end{proof}
