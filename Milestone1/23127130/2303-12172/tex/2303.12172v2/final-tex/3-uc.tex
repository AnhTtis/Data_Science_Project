\section{Branching OGP from Uniform Concentration}
\label{sec:uc}

We now turn to the proof of Proposition~\ref{prop:bogp-alg}. 
In light of Proposition~\ref{prop:bogp-equivalent}, it suffices to prove $\BOGP_{\loc,0}=\ALG$.
We begin with a very general argument that due to the ``many orthogonal increments'' property at each layer of the branching tree, it suffices to consider ``greedy'' embeddings in some sense. 
This argument is essentially elementary and relies on an idea of Subag \cite{subag2018free} applied recursively down the tree.

\subsection{Uniform Concentration}

For $\vx \in [0,1]^\sS$, define the product of spheres 
\[
    \cS_N(\vx) 
    = \lt\{
        \bsig \in \bbR^N : 
        \vR(\bsig,\bsig) = \vx
    \rt\}
    = \lt\{
        \bsig \in \bbR^N : 
        \norm{\bsig_s}_2^2 = \lambda_s x_s N
        ~\forall s\in \sS
    \rt\}.
\]
For $\bsig^0 \in \cS_N(\vx)$ and $\vy \succeq \vx$, define 
\begin{equation}
    \label{eq:band-defn}
    B(\bsig^0, \vy, k)
    =
    \lt\{\begin{array}{l}
        \ubsig =
        (\bsig^1,\bsig^2,\dots,\bsig^k) \in \cS_N(\vy)^k: 
        \\
        \vR(\bsig^i-\bsig^0,\bsig^0)
        = \vR(\bsig^i-\bsig^0,\bsig^j-\bsig^0)
        = \vzero
        \quad \forall i,j\in [k], i\neq j
    \end{array}\rt\}.
\end{equation}
Let $0\le \pminus < \pplus \le 1$. 
Generate $k+1$ i.i.d. copies $\hH_N^{[0]}, \hH_N^{[1]}, \ldots, \hH_N^{[k]}$ of $\wtH_N$ as in \eqref{eq:def-hamiltonian-no-field}.
Set 
\begin{align}
    \label{eq:def-hH(0)}
    \hH_N^{(0)}(\bsig) &= \sqrt{\pminus} \hH_N^{[0]}(\bsig) \quad \text{and} \\
    \label{eq:def-hH(i)}
    \hH_N^{(i)}(\bsig) &= \sqrt{\pminus} \hH_N^{[0]}(\bsig) + \sqrt{\pplus-\pminus} \hH_N^{[i]}(\bsig),\quad 1\le i\le k.
\end{align}
Define
\[
    F_{\pminus,\pplus}(\bsig^0,\vy, k)
    = 
    \fr{1}{kN}
    \max_{\ubsig \in B(\bsig^0, \vy, k)}
    \sum_{i=1}^k
    \lt(\hH_N^{(i)}(\bsig^i) - \hH_N^{(0)}(\bsig^0)\rt).
\]

\begin{lemma}
    \label{lem:F-lip}
    There exists $C$ such that the following holds. 
    Suppose that $\delta \vone \preceq \vx \preceq \vy \preceq \vone$ and $\bsig^0, \brho^0 \in \cS_N(\vx)$ satisfy $\norm{\bsig^0-\brho^0}_2 \le \iota\sqrt{N}$.
    If $\hH_N^{[0]}, \ldots, \hH_N^{[k]} \in K_N$ for the event $K_N$ in Proposition~\ref{prop:gradients-bounded}, then
    \begin{equation}
        \label{eq:F-lip}
        |F_{\pminus,\pplus}(\bsig^0,\vy,k) - F_{\pminus,\pplus}(\brho^0,\vy,k)|
        \leq
        \frac{C\iota}{\sqrt{\delta}}. 
    \end{equation}
\end{lemma}

\begin{proof}
    Let $T:\bbR^N\to\bbR^N$ be a product of rotation maps in the $r$ factors $\bbR^{\cI_s}$ such that $T(\bsig^0)=\brho^0$. Then
    \[
        T\lt(B(\bsig^0, \vy, k)\rt)
        = B(T(\bsig^0), \vy, k)
        = B(\brho^0, \vy, k).
    \]
    In particular, we take $T$ to be obtained using geodesic rotations from each $\bsig^0_s$\ to $\brho^0_s$.
    Thus, if $\ubsig \in B(\bsig^0,\vy,k)$ and $\ubrho = (\brho^1,\ldots,\brho^k) \in B(\brho^0,\vy,k)$ for $\brho^i = T\bsig^i$, then for all $i\in [k]$
    \[
        \fr{\norm{\brho^i-\bsig^i}_2}{\norm{\bsig^i}_2}
        \le \fr{\norm{\brho^0-\bsig^0}_2}{\norm{\bsig^0}_2}
        \le \fr{\iota}{\sqrt{\delta}},
    \]
    so $\fr{1}{\sqrt{N}} \norm{\brho^i-\bsig^i}_2 \le \iota / \sqrt{\delta}$.
    On the event $\hH_N^{[0]}, \ldots, \hH_N^{[k]} \in K_N$, it follows that
    \[
        \lt|\hH_N^{(i)}(\bsig^i) - \hH_N^{(i)}(\brho^i)\rt|
        \le \fr{C\iota}{\sqrt{\delta}}
    \]
    for $1\le i\le k$ and 
    \[
        \lt|\hH_N^{(0)}(\bsig^0) - \hH_N^{(0)}(\brho^0)\rt|
        \le C\iota,
    \]
    which implies the conclusion (after adjusting $C$).
\end{proof}

\begin{lemma}
\label{lem:unif-main}
    There exist constants $c,C>0$ such that for all $k\in \bbN$ and $\delta,\eps > 0$ the following holds. For any $\vx,\vy$ satisfying $\delta \vone \preceq\vx \preceq \vy$,
    \begin{align*}
        &\bbP\lt(
            \sup_{\bsig^0 \in \cS_N(\vx)}
            |F_{\pminus,\pplus}(\bsig^0, \vy, k) - \bbE F_{\pminus,\pplus}(\bsig^0, \vy, k)|
            \le \eps
        \rt) \\
        &\qquad 
        \ge 
        1 - 
        \exp\lt(
            C\log\lt(\frac{1}{\delta\eps}\rt) N -
            ck\eps^2 N
        \rt)
        - e^{-cN}
    \end{align*}
\end{lemma}
\begin{proof}
    Fix for now $\bsig^0 \in \cS_N(\vx)$ and $\ubsig = (\bsig^1, \ldots, \bsig^k) \in B(\bsig^0, \vy, k)$. 
    Using the definition \eqref{eq:band-defn} in the final step, we find that for small $c>0$,
    \begin{align*}
        &\bbE \lt[\lt(
            \sum_{i=1}^k
            (\hH_N^{(i)}(\bsig^i) - \hH^{(0)}(\bsig^0))
        \rt)^2\rt] \\
        &= 
        \bbE \lt[\lt(
            \sum_{i=1}^k
            \sqrt{\pminus} (\hH_N^{[0]}(\bsig^i) - \hH_N^{[0]}(\bsig^0))
            + 
            \sqrt{\pplus-\pminus} \hH_N^{[i]} (\bsig^i)
        \rt)^2\rt] \\
        &=
        \pminus 
        \sum_{i,j=1}^k
        \bbE\lt[
            (\hH_N^{[0]}(\bsig^i) - \hH_N^{[0]}(\bsig^0))
            (\hH_N^{[0]}(\bsig^j) - \hH_N^{[0]}(\bsig^0))
        \rt]
        +
        (\pplus-\pminus)
        \sum_{i=1}^k
        \bbE\lt[
            \hH_N^{[i]} (\bsig^i)^2
        \rt]
        \\
        &=
        \pminus
        \sum_{i,j=1}^k
        \xi(\vR(\bsig^i,\bsig^j))
        -
        \xi(\vR(\bsig^i,\bsig^0))
        -
        \xi(\vR(\bsig^0,\bsig^j))
        +
        \xi(\vR(\bsig^0,\bsig^0))
        +
        (\pplus-\pminus) \sum_{i=1}^k \xi(\vR(\bsig^i,\bsig^i))
        \\
        &\le
        \frac{k}{8c}.
    \end{align*}
    By the Borell-TIS inequality, for each fixed $\bsig^0 \in \cS_N(\vx)$
    \begin{equation}
        \label{eq:concentration-one-sigma}
        \bbP\lt[
            |F_{\pminus,\pplus}(\bsig^0,\vy,k) 
            -\bbE F_{\pminus,\pplus}(\bsig^0,\vy,k)|
            \le \eps/2
        \rt]
        \ge 
        1 - 
        2\exp\lt(-ck\eps^2 N\rt).
    \end{equation}
    Choose $\iota = \Theta(\eps \sqrt{\delta})$ so that the right-hand side of \eqref{eq:F-lip} is bounded by $\eps/2$, and let $\cN$ be an $\iota \sqrt{N}$-net of $\cS_N(\vx)$ with size $|\cN|\le (1/(\delta\eps))^{CN}$.
    Define the events 
    \begin{align*}
        S_{\mathrm{conc}}
        &= 
        \lt\{
            \,|F_{\pminus,\pplus}(\brho^0, \vy, k) - \bbE F_{\pminus,\pplus}(\brho^0, \vy, k)|
            \le \eps/2
            ~~\forall~
            \brho^0 \in \cN
        \rt\}, \\
        S_{\mathrm{lip}}
        &= 
        \lt\{
            \,\hH_N^{[0]},\ldots,\hH_N^{[k]} \in K_N
        \rt\},
    \end{align*}
    where $K_N$ is defined in Proposition~\ref{prop:gradients-bounded}.
    By a union bound (after adjusting $c,C$), 
    \begin{equation}
        \label{eq:concentration-many-sigma}
        \bbP\lt(
            S_{\mathrm{conc}}
            \cap
            S_{\mathrm{lip}}
        \rt)
        \ge 
        1 - 
        \exp\lt(C\log \lt(\fr{1}{\delta\eps}\rt) N - ck\eps^2 N\rt)-e^{-cN}.
    \end{equation}
    Suppose $S_{\mathrm{conc}} \cap S_{\mathrm{lip}}$ holds.
    For any $\bsig^0 \in \cS_N(\vx)$, there exists $\brho^0 \in \cN$ such that $\tnorm{\bsig^0-\brho^0}_2 \le \iota \sqrt{N}$, and so
    \begin{align*}
        &|F_{\pminus,\pplus}(\bsig^0, \vy, k) - \bbE F_{\pminus,\pplus}(\bsig^0, \vy, k)| \\
        &\le |F_{\pminus,\pplus}(\bsig^0, \vy, k) - F_{\pminus,\pplus}(\brho^0, \vy, k)| 
        + |F_{\pminus,\pplus}(\brho^0, \vy, k) - \bbE F_{\pminus,\pplus}(\brho^0, \vy, k)|
        \le \fr{\eps}{2} + \fr{\eps}{2} = \eps.
    \end{align*}
\end{proof}

For now, let $k, D, \uvphi, \up$ (recall Definition~\ref{defn:bogp}) be arbitrary.
In Proposition~\ref{prop:uc-bogp} below, we obtain an estimate for
$\fr{1}{N} \bbE \max_{\ubsig\in \cQ_{\loc}(0)}\cH_N(\ubsig)$ by applying Lemma~\ref{lem:unif-main} repeatedly at each internal vertex $u\in \bbT\backslash \bbL$.
This maximum will take the form of an abstract sum of energy increments.
In the next subsection we will take a continuum limit of this bound, which will yield the variational formula \eqref{eq:alg} for $\ALG$ and prove Proposition~\ref{prop:bogp-alg}.

Spherical symmetry implies that $\bbE F_{\pminus,\pplus}\lt(\bsig,\vy,k\rt)$ depends on $\bsig$ only through $\vR(\bsig,\bsig)$. Hence for $\vphiminus=\vR(\bsig,\bsig)$ we may define
\begin{equation}
    \label{eq:f}
    f(\vphiminus, \vphiplus; \pminus, \pplus; k)
    =
    \bbE 
    F_{\pminus,\pplus}\lt(\bsig,\vphiplus,k\rt).
\end{equation}

\begin{proposition}
    \label{prop:uc-bogp}
    Fix $D \in \bbN$ and $\eps, \delta > 0$. 
    Suppose that $\vphi_0 \succeq \delta \vone$. 
    There exists $k_0 = k_0(D,\eps,\delta)$ such that for all $k\ge k_0$, there exists $c = c(D,\eps,\delta,k)$ such that
    \[
        \bbP\lt[\lt|
            \fr1N
            \sup_{\ubsig \in \cQ_{\loc}(0)}
            \cH_N(\ubsig)
            - \lt(
                \sum_{s\in \sS}  h_s\lambda_s \sqrt{\phi_0^s} +
                \sum_{d=0}^{D-1}
                f\lt(\vphi_d, \vphi_{d+1}; p_d, p_{d+1}; k\rt)
            \rt)
        \rt|\le 2D\eps\rt]
        \ge
        1-e^{-cN}.
    \]
\end{proposition}

\begin{proof}
    Let $C,c$ be as in Lemma~\ref{lem:unif-main}, and $k_0$ large enough that
    \begin{align}
        \label{eq:uc-exp-rate}
        C \log \lt(\fr{1}{\delta\eps}\rt) - ck_0\eps^2
        &\le -c, \\
        \label{eq:uc-barycenter-close}
        \tnorm{\vh}_\infty / \sqrt{k_0} &\le \eps.
    \end{align}
    Recall the construction of $\wtHNp{u}$ from \eqref{eq:def-correlated-disorder}.
    For any $u\in V_d$, $0\le d\le D-1$, let $\cE_u$ denote the event in Lemma~\ref{lem:unif-main}, with $(\pminus,\pplus) = (p_d, p_{d+1})$, $(\vx,\vy) = (\vphi_d,\vphi_{d+1})$, and
    \begin{equation}
        \label{eq:uc-hamiltonian-choice}
        (\hH_N^{(0)},\hH_N^{(1)},\ldots,\hH_N^{(k)})
        = \lt(\wtHNp{u},\wtHNp{u1},\ldots,\wtHNp{uk}\rt).
    \end{equation}
    Let $\cE = \bigcap_{u\in \bbT \setminus \bbL} \cE_u$. 
    Lemma~\ref{lem:unif-main} and equation \eqref{eq:uc-exp-rate} imply $\bbP(\cE^u) \ge 1-2e^{-cN}$ for all $u\in \bbL$. 
    By a union bound, $\bbP(\cE) \ge 1 - e^{-cN}$ (after adjusting $c$).
    
    Denote by $F^u_{p_d,p_{d+1}}$ the function $F_{p_d,p_{d+1}}$ defined with Hamiltonians \eqref{eq:uc-hamiltonian-choice}.
    Let $\ubsig \in \cQ_{\loc}(0)$, so there exists $\ubrho \in \cQ_{\loc+}(0)$ with $(\brho(u))_{u\in \bbL} = \ubsig$. 
    On the event $\cE$,
    \begin{align*}
        \fr{1}{N} \cH_N(\ubsig) 
        - \fr{1}{KN} \sum_{v\in \bbL} \la \bh, \bsig(u) \ra
        &=
        \fr{1}{KN}
        \sum_{u\in \bbL} \wtHNp{u}(\bsig(u)) \\
        &=
        \sum_{d=0}^{D-1}
        \fr{1}{k^{d}}
        \sum_{u\in V_d} 
        \fr{1}{kN}
        \sum_{i=1}^{k}
        \lt( 
            \wtHNp{ui}(\brho(ui))-\wtHNp^{u}(\brho(u))
        \rt) \\
        &\le
        \sum_{d=0}^{D-1}
        \fr{1}{k^{d}}
        \sum_{u\in V_d} 
        F_{p_d,p_{d+1}}^{u}
        \lt(\brho(u),\vphi_{d+1},k\rt) \\
        &\stackrel{Lem.~\ref{lem:unif-main}}{\le}
        D\eps
        + \sum_{d=0}^{D-1}
        f(\vphi_d, \vphi_{d+1}; p_d, p_{d+1}; k).
    \end{align*}
    In the telescoping sum, we used that $\wtH_N^{(\emptyset)}$ is the zero function. 
    By Lemma~\ref{lem:loc0-barycenter} and equation \eqref{eq:uc-barycenter-close},
    \begin{align*}
        \lt|
            \fr{1}{KN} \sum_{v\in \bbL} \la \bh, \bsig(u) \ra
            - \fr{1}{N} \la \bh, \brho(\emptyset) \ra
        \rt|
        &\le 
        \fr{1}{\sqrt{N}} \tnorm{\bh}_2 \cdot 
        \fr{1}{\sqrt{N}} \norm{\brho(\emptyset) - \fr{1}{K} \sum_{u\in \bbL} \bsig(u)}_2 \\
        &\le 
        \tnorm{\vh}_\infty \sqrt{\fr{D}{k}} 
        \le D\eps.
    \end{align*}
    Finally,
    \begin{equation}
        \label{eq:root-energy}
        \fr{1}{N} \la \bh, \brho(\emptyset) \ra
        = \fr{1}{N} \sum_{s\in \sS} h_s \norm{\brho(\emptyset)_s}_1 
        \le \fr{1}{N} \sum_{s\in \sS} h_s \sqrt{|\cI_s|} \norm{\brho(\emptyset)_s}_2 
        = \sum_{s\in \sS}  h_s \lambda_s \sqrt{\phi_0^s}.
    \end{equation}
    This completes the proof of the upper bound for $\fr1N \sup_{\ubsig\in \cQ_{\loc}(0)}\cH_N(\ubsig)$. 
    Finally, observe that equality holds above (up to the same $2D\eps$ error) if we choose $\brho(\emptyset) = \sqrt{\vphi_0} \diamond \bone$ and then recursively choose $(\brho(ui))_{i\in[k]}$ given $\brho(u)$ so that, for $|u|=d$, 
    \[
        \fr{1}{Nk}
        \sum_{i=1}^k \lt(\wtHNp{ui}(\brho(ui))-\wtHNp{u}(\brho(u))\rt)
        =
        F^u_{p_d,p_{d+1}}(\brho(u),\vphi_{d+1},k).
    \] 
\end{proof}

\subsection{The Algorithmic Functional}

Our next objective is to estimate the terms $f(\vphiminus, \vphiplus; \pminus, \pplus; k)$ appearing in Proposition~\ref{prop:uc-bogp}. 
The key point is that when the differences $\vphiplus-\vphiminus$ and $\pplus-\pminus$ are small, which is ensured by $\delta$-denseness of $(\up, \uvphi)$, this estimate only requires Taylor approximating the relevant Hamiltonians to second order. 
We take advantage of this using the following lemma, which (for $k=1$) gives the ground state energy $\bGS(W, \vv, 1)$ of a quadratic multi-species spin glass with Gaussian external field.
For general $k$, this lemma gives the limiting ground state energy $\bGS(W, \vv, k)$ of a $k$-replica Hamiltonian \eqref{eq:k-replica-hamiltonian} with shared quadratic component $W \diamond \bG$ and independent external fields $\vv \diamond \bg^i$, whose inputs \eqref{eq:bbtperp} are $k$ pairwise orthogonal elements of $\cS_N(\vone)$.
Note that $\bGS(W,\vv,k) \le \bGS(W,\vv,1)$ by definition. 
In fact equality holds, i.e. there exist orthogonal $\bsig^1, \ldots, \bsig^k$ such that each $\bsig^i$ approximately maximizes $H_{N,k}^i(\bsig^i)$.
We prove this lemma in Appendix~\ref{sec:sk-ext-field} by combining a known formula for the case $(k,\vv)=(1,\vzero)$ with an elementary recursive argument along subspaces.

\begin{lemma}
    \label{lem:sk-ext-field}
    Let $W = (w_{s,s'})_{s,s\in \sS} \in \bbR_{\ge 0}^{\sS\times \sS}$ be symmetric and $\vv = (v_s)_{s\in \sS} \in \bbR_{\ge 0}^\sS$.
    Let $k\in \bbN$ and sample independent $\bg^1,\ldots,\bg^k \in \bbR^N$ and $\bG \in \bbR^{N\times N}$ with i.i.d. standard Gaussian entries. 
    Consider the $k$-replica Hamiltonian
    \begin{equation}
        \label{eq:k-replica-hamiltonian}
        H_{N,k}(\ubsig)
        = 
        \fr{1}{k}
        \sum_{i=1}^k
        H_{N,k}^i(\bsig^i),
        \qquad
        H_{N,k}^i(\bsig^i)
        =
        \la \vv \diamond \bg^i, \bsig^i\ra 
        + 
        \fr{1}{\sqrt{N}}
        \la W \diamond \bG, (\bsig^i)^{\otimes 2} \ra
    \end{equation}
    on the input space of orthogonal replicas
    \begin{equation}
        \label{eq:bbtperp}
        \cS_N^{k, \perp}
        = 
        \lt\{
            \ubsig = (\bsig^1, \ldots, \bsig^k) \in \cS_N(\vone):
            \vR(\bsig^i,\bsig^j) = \vzero 
            ~\forall i\neq j
        \rt\}.
    \end{equation}
    Define the $k$-replica ground state energy 
    \begin{equation}
        \label{eq:sk-gsn}
        \GS_N(W,\vv,k)
        \equiv 
        \fr{1}{N}
        \max_{\ubsig \in \cS_N^{k, \perp}}
        H_{N,k}(\ubsig).
    \end{equation}
    Then $\bGS(W,\vv,k) \equiv \lim_{N\to\infty} \bbE \GS_N(W,\vv,k)$ exists, does not depend on $k$, and is given by
    \[
        \bGS(W,\vv,k) 
        = 
        \sum_{s\in \sS}
        \lambda_s
        \sqrt{v_s^2 + 2 \sum_{s'\in \sS} \lambda_s w_{s,s'}^2 }.
    \]
\end{lemma}

\begin{proposition}
    \label{prop:what-F-is}
    Suppose $0\le \pminus \le \pplus \le 1$, $\vzero \preceq \vphiminus \preceq \vphiplus \preceq \vone$ and
    \begin{equation}
        \label{eq:delta-discrete}
        \pplus - \pminus \le \delta,
        \qquad
        \vphiplus - \vphiminus \preceq \delta \vone.
    \end{equation}
    Then,
    \begin{align*}
        f(\vphiminus,\vphiplus; \pminus,\pplus; k)
        &=
        \sum_{s\in\sS}
        \lambda_s
        \sqrt{(\vphiplus^s - \vphiminus^s)\lt((\pplus-\pminus) \xi^s(\vphiminus) + \pminus \sum_{s'\in \sS} \partial_{x_{s'}} \xi^s(\vphiminus) (\vphiplus^s - \vphiminus^s)\rt)}
        \\
        &\qquad + 
        O\big(\delta^{3/2} + (\delta/k)^{1/2}\big)+o_N(1),
    \end{align*}
    where $o_N(1)$ denotes a term tending to $0$ as $N\to\infty$.
\end{proposition}

\begin{proof}
    Fix $\bsig^0$ such that $\vR(\bsig^0, \bsig^0) = \vphiminus$. 
    Let $\ubsig = (\bsig^1,\ldots,\bsig^k) \in B(\bsig^0,\vphiplus,k)$. 
    Let $\Delta \vphi = \vphiplus - \vphiminus$ and $\ubx = (\bx^1,\ldots,\bx^k)$ for $\bx^i = (\Delta \vphi)^{-1/2} \diamond (\bsig^i - \bsig^0)$.
    Define
    \begin{align*}
        \cS_{\bullet} &= \lt\{
            \by \in \cS_N(\vone) : \vR(\by,\bsig^0) = \vzero
        \rt\}, \\
        \cS_\bullet^{k,\perp} &= \lt\{
            \uby = (\by^1,\ldots,\by^k) \in \cS_{\bullet}^k : 
            \vR(\by^i,\by^j) = \vzero ~\forall i\neq j
        \rt\}.
    \end{align*}
    Note that $\ubx \in \cS_\bullet^{k,\perp}$. 
    Recall that $\hH^{[0]}_N,\ldots,\hH^{[k]}_N$ are i.i.d. copies of $\wtH_N$, and that $\hH^{(0)}_N,\ldots,\hH^{(k)}_N$ are defined by \eqref{eq:def-hH(0)}, \eqref{eq:def-hH(i)}.
    Let
    \[
        \oH^i_N(\bx^i) 
        = \hH_N^{[i]}(\bsig^i) - \hH_N^{[i]}(\bsig^0) 
        = \hH_N^{[i]}\lt(\bsig^0 + \sqrt{\Delta \vphi} \diamond \bx^i\rt) - \hH_N^{[i]}(\bsig^0).
    \]
    Then
    \begin{align}
        \notag
        f(\vphiminus,\vphiplus; \pminus,\pplus; k)
        &= 
        \fr{1}{kN}
        \bbE 
        \max_{\ubsig \in B(\bsig^0,\vphiplus,k)}
        \sum_{i=1}^k 
        \lt(\hH^{(i)}_N(\bsig^i) - \hH^{(0)}_N(\bsig^0)\rt) \\
        \notag
        &= 
        \fr{1}{kN}
        \bbE 
        \max_{\ubsig \in B(\bsig^0,\vphiplus,k)}
        \sum_{i=1}^k 
        \bigg(
            \sqrt{\pminus} \lt(\hH_N^{[0]}(\bsig^i) - \hH_N^{[0]}(\bsig^0)\rt) \\
            \notag
            &\qquad 
            + \sqrt{\pplus-\pminus} \lt(\hH_N^{[i]}(\bsig^i) - \hH_N^{[i]}(\bsig^0)\rt) 
            + \sqrt{\pplus-\pminus}\,
            \hH_N^{[i]}(\bsig^0)
        \bigg) \\
        \label{eq:f-expansion-into-hamiltonians}
        &= 
        \fr{1}{kN}
        \bbE 
        \max_{\ubx \in \cS_\bullet^{k,\perp}}
        \sum_{i=1}^k 
        \lt(
            \sqrt{\pminus}\, \oH^0 (\bx^i) + \sqrt{\pplus - \pminus}\, \oH^i (\bx^i)
        \rt) 
    \end{align}
    where we note that $\bbE \hH_N^{[i]}(\bsig^0) = 0$. 
    Let $\oH^{i,\tay}_N$ denote the degree $2$ Taylor expansion of $\oH_N^i$ around $\bzero$. 
    By Proposition~\ref{prop:gradients-bounded} (recalling \eqref{eq:delta-discrete}),
    \[
        \bbE \sup_{\bx \in \cS_{\bullet}} |\oH^i(\bx)_N - \oH^{i,\tay}_N(\bx)| = O(N\delta^{3/2}).
    \]
    So, for all $0\le i\le k$, we have as processes on $\cS_{\bullet}$
    \begin{equation}
        \label{eq:oH-taylor-expansion}
        \oH^i_N(\bx) =_d \la \vv \diamond \bg^i, \bx\ra
        + 
        \la W \diamond \bG^i, \bx^{\otimes 2}\ra
        + O_{\bbP}(N\delta^{3/2}),
    \end{equation}
    where $O_{\bbP}(N\delta^{3/2})$ denotes a $\cS_{\bullet}$-valued process $X(\bx)$ with $\bbE \sup_{\bx \in \cS_{\bullet}} |X(\bx)| = O(N\delta^{3/2})$ and $\vv = (v_s)_{s\in \sS}$ and $W = (w_{s,s'})_{s,s'\in \sS}$ are given by
    \[
        v_s = \sqrt{\xi^s(\vphiminus) (\Delta \vphi)^s }, 
        \qquad
        w_{s,s'} = \fr{1}{\sqrt{2}} \sqrt{\lambda_{s'}^{-1} \partial_{x_{s'}} \xi^s(\vphiminus) (\Delta \vphi)^s (\Delta \vphi)^{s'}}.
    \]
    Next we observe some simplifications. 
    Because $\Delta \vphi \preceq \delta \vone$, we have $v_s = O(\delta^{1/2})$, $w_{s,s'} = O(\delta)$ uniformly over $s,s'$.
    The linear contribution to $\oH^0_N$ in \eqref{eq:oH-taylor-expansion} is small because
    \[
        \fr{1}{kN} \sum_{i=1}^k \la \vv \diamond \bg^0, \bx^i\ra
        \le \fr{1}{kN} \norm{\vv \diamond \bg^0}_2 \norm{\sum_{i=1}^k \bx^i}_2
        = O_{\bbP}((\delta/k)^{1/2})
    \]
    by orthogonality of the $\bx^i$.
    Because $\pplus - \pminus \le \delta$, the quadratic contributions to $\oH^i_N$ for $i\ge 1$ are also small:
    \[
        \fr{\sqrt{\pplus-\pminus}}{N}
        \la W \diamond \bG^i, (\bx^i)^{\otimes 2}\ra
        = O_{\bbP}(\delta^{3/2}).
    \]
    Combining these estimates with \eqref{eq:f-expansion-into-hamiltonians} and \eqref{eq:oH-taylor-expansion}, we find
    \begin{align*}
        f(\vphiminus,\vphiplus; \pminus,\pplus; k)
        &=
        \fr{1}{kN}
        \bbE \max_{\ubx \in \cS_\bullet^{k,\perp}}
        \sum_{i=1}^k 
        \sqrt{\pplus - \pminus} \lt\la \vv \diamond \bg^i, \bx^i\rt\ra +
        \sqrt{\pminus} \lt\la W \diamond \bG^0, (\bx^i)^{\otimes 2} \rt\ra \\
        &\qquad + O((\delta/k)^{1/2} + \delta^{3/2}).
    \end{align*}
    By Lemma~\ref{lem:sk-ext-field} (applied in dimension $N-r$ due to the linear constraint $\vR(\bx^i,\bsig^0)=\vzero$ in $\cS_{\bullet}$), this remaining expectation is given up to $o_N(1)$ error by
    \begin{align*}
        &\sum_{s\in \sS}
        \lambda_s
        \sqrt{(\pplus - \pminus) v_s^2 + 2\sum_{s'\in \sS} \lambda_{s'} \pminus w_{s,s'}^2} \\
        &\quad = 
        \sum_{s\in \sS}
        \lambda_s
        \sqrt{(\Delta \vphi)^s \lt((\pplus - \pminus) \xi^s(\vphiminus) + \pminus \sum_{s'\in \sS} \partial_{x_{s'}} \xi^s(\vphiminus) (\Delta \vphi)^{s'}\rt)}.
    \end{align*}
    This implies the result. 
\end{proof}

We now evaluate $\BOGP_{\loc,0}$ by taking a continuous limit of Propositions~\ref{prop:uc-bogp} and \ref{prop:what-F-is}. 
Fix $D,k$ and $\delta = 6r/D$, and let $(\up,\uvphi)$ be $\delta$-dense. 
We parametrize time by $q_d = \la \vlam, \vphi_d\ra$, so in particular $q_0 = \la \vlam, \vphi_0\ra$.
Let the functions $\tp:[q_0,1]\to [0,1]$ and $\tPhi:[q_0,1]\to [0,1]^{\sS}$ satisfy 
\begin{equation}
    \label{eq:continuous-def}
    \tp(q_d) = p_d,
    \qquad
    \tPhi(q_d) = \vphi_d.
\end{equation}
and be linear on each interval $[q_d,q_{d+1}]$. 
These are piecewise linear approximations of inputs $(p,\Phi)$ to the algorithmic functional $\bbA$.
Define
\begin{equation}
    \label{eq:def-Ads}
    A_d^s = 
    \sqrt{(\phi^s_{d+1} - \phi^s_d)\lt(
        (p_{d+1}-p_d) \xi^s(\vphi_d) + 
        p_d \sum_{s'\in\sS}\partial_{x_{s'}}\xi^s(\vphi_d)(\phi_{d+1}^{s'}-\phi_d^{s'})
    \rt)}.
\end{equation}
This term appears in the estimate of $f\lt(\vphi_d,\vphi_{d+1};p_{d+1},p_d;k\rt)$ obtained from Proposition~\ref{prop:what-F-is}.
\begin{lemma}
    \label{lem:ALG-derivation}
    We have
    \begin{align*}
        \lt|
            \sum_{d=0}^{D-1} A_d^s - 
            \int_{q_0}^{q_D}
            \sqrt{\tPhi_s'(q)(\tp\times \xi^s\circ\tPhi)'(q)}
            ~\de q
        \rt|
        \le CD^{-1/2}
    \end{align*}
    for a constant $C>0$ independent of $D,\up,\uvphi$.
\end{lemma}
\begin{proof}
Until the end, we focus on estimating the difference
\[
    \Delta_d^s \equiv 
    \lt|
        A_d^s - 
        \int_{q_d}^{q_{d+1}}
        \sqrt{\tPhi_s'(q)(\tp\times \xi^s\circ\tPhi)'(q)}
        ~\de q
    \rt|.
\]
Note the general inequality
\begin{align}
    \notag
    \int_{q_d}^{q_{d+1}}\sqrt{a(q)}\cdot |\sqrt{b(q)}-\sqrt{c(q)}| \de q
    &\leq
    \lt(\int_{q_d}^{q_{d+1}}a(q)\de q\rt)^{1/2}
    \cdot
    \lt(\int_{q_d}^{q_{d+1}}\big(\sqrt{b(q)}-\sqrt{c(q)}\big)^2\de q\rt)^{1/2} \\
    \label{eq:integral-bound}
    &\leq
    \lt(\int_{q_d}^{q_{d+1}}a(q)\de q\rt)^{1/2}
    \cdot
    \lt(\int_{q_d}^{q_{d+1}}|b(q)-c(q)|\de q\rt)^{1/2}.
\end{align}
Thus
\begin{align*}
    \Delta^s_d
    &=
    \lt|
        \int_{q_d}^{q_{d+1}}
        \sqrt{\tPhi_s'(q)}
        \lt(
            \sqrt{(\tp\times \xi^s\circ\tPhi)'(q)}
            -
            \sqrt{\fr{
                (p_{d+1}-p_d) \xi^s(\vphi_d) + 
                p_d \sum_{s'\in\sS}
                \partial_{x_{s'}}\xi^s(\vphi_d)
                (\phi_{d+1}^{s'}-\phi_d^{s'})
            }{q_{d+1}-q_d}}
        \rt)
        ~\de q
    \rt| \\
    &\le
    \sqrt{
        (\phi_{d+1}^s-\phi_d^s)
        \int_{q_d}^{q_{d+1}}
        \lt|
            (\tp\times \xi^s\circ\tPhi)'(q)
            -
            \fr{
                (p_{d+1}-p_d) \xi^s(\vphi_d) + 
                p_d \sum_{s'\in\sS}
                \partial_{x_{s'}}\xi^s(\vphi_d)
                (\phi_{d+1}^{s'}-\phi_d^{s'})
            }{q_{d+1}-q_d}
        \rt|
        \de q
    }.
\end{align*}
In the first step we used that $\tPhi'(q)=(\vphi_{d+1}-\vphi_d) / (q_{d+1}-q_d)$ by definition, and in the second we used \eqref{eq:integral-bound}.
Let $(\tp\times \xi^s\circ\tPhi)'(q_d)$ and $(\tp\times \xi^s\circ\tPhi)'(q_{d+1})$ denote the right and left derivatives at these points, respectively.
The definitions of $\tp'$ and $\tPhi'$ imply
\[
    \fr{
        (p_{d+1}-p_d) \xi^s(\vphi_d) + 
        p_d \sum_{s'\in\sS}
        \partial_{x_{s'}}\xi^s(\vphi_d)
        (\phi_{d+1}^{s'}-\phi_d^{s'})
    }{q_{d+1}-q_d}
    = (\tp\times \xi^s\circ\tPhi)'(q_d),
\]
so in fact
\begin{align}
    \notag
    \Delta^s_d
    &\le 
    \sqrt{
        (\phi_{d+1}^s-\phi_d^s)
        \int_{q_d}^{q_{d+1}}
        \lt|
            (\tp\times \xi^s\circ\tPhi)'(q)
            - (\tp\times \xi^s\circ\tPhi)'(q_d)
        \rt|
        \de q
    } \\
    \label{eq:one-step-integral-estimation}
    &\le
    \sqrt{
        (\phi_{d+1}^s-\phi_d^s)
        (q_{d+1}-q_d)
        \lt(
            (\tp\times \xi^s\circ\tPhi)'(q_{d+1})
            - (\tp\times \xi^s\circ\tPhi)'(q_d)
        \rt)
    } .
\end{align}
Let $\nabla \vphi_d = (\vphi_{d+1}-\vphi_d) / (q_{d+1}-q_d)$ be the constant value of $\nabla \tPhi$ on $[q_d,q_{d+1}]$.
Then 
\begin{align*}
    (\tp\times \xi^s\circ\tPhi)'(q_{d+1})-(\tp\times \xi^s\circ\tPhi)'(q_d)
    &=
    \lt(\fr{p_{d+1}-p_d}{q_{d+1}-q_d}\rt)
    \lt(\xi^s(\vphi_{d+1})-\xi^s(\vphi_d)\rt) \\
    &\qquad
    + p_{d+1} \la \nabla \xi^s(\vphi_{d+1}),\nabla \vphi_d \ra
    - p_{d} \la \nabla \xi^s(\vphi_d),\nabla \vphi_d \ra.
\end{align*}
We thus obtain
\begin{align*}
    &(q_{d+1}-q_d)\lt((\tp\times \xi^s\circ\tPhi)'(q_{d+1})-(\tp\times \xi^s\circ\tPhi)'(q_d)\rt) \\
    &= (p_{d+1}-p_d) \lt(\xi^s(\vphi_{d+1})-\xi^s(\vphi_d)\rt) 
    + (q_{d+1}-q_d) (p_{d+1}-p_d) \la \nabla \xi^s(\vphi_{d+1}),\nabla \vphi_d \ra \\
    &\qquad + (q_{d+1}-q_d) p_d 
    \la \nabla \xi^s(\vphi_{d+1}) -  \nabla \xi^s(\vphi_d),\nabla \vphi_d \ra \\
    &\le O\lt(
        (p_{d+1}-p_d) \norm{\vphi_{d+1}-\vphi_d}_2 
        + \norm{\vphi_{d+1}-\vphi_d}_2^2
    \rt) = O(\delta^2).
\end{align*}
Combining with \eqref{eq:one-step-integral-estimation} gives the estimate $\Delta^s_d = O(\delta) \sqrt{\phi_{d+1}^s - \phi_d^s}$.
Summing this over $0\le d\le D-1$ gives the final estimate
\begin{align*}
    \lt|
        \sum_{d=0}^{D-1}
        A_d^s -
        \int_{q_0}^{q_D}
        \sqrt{\tPhi_s'(q)(\tp\times \xi^s\circ\tPhi)'(q)}
        ~\de q
    \rt|
    &\le \sum_{d=0}^{D-1} \Delta^s_d
    \le O(\delta) \sum_{d=0}^{D-1} \sqrt{\phi_{d+1}^s-\phi_d^s} \\
    &\le O(\delta \sqrt{D}) = O(D^{-1/2}).
\end{align*}
by Cauchy-Schwarz.
\end{proof}

We next show that discretizing any $C^1$ functions $(p,\Phi)$ preserves the value of $\bbA$. 


\begin{lemma}
    \label{lem:ALG-from-continuum}
    Suppose $q_0\in [0,1]$, $p\in \bbI(q_0,1)$, and $\Phi \in \Adm(q_0,1)$.
    Consider any $\uq = (q_0,\ldots,q_D)$ with $q_0<\cdots<q_D=1$, such that the $(\up, \uvphi)$ defined by $p_d = p(q_d)$ and $\vphi_d = \Phi(q_d)$ is $6r/D$-dense. 
    Then, for all $s\in \sS$,
    \[
        \lt|
            \int_{q_0}^1 \sqrt{\Phi'_s(q) (p\times \xi^s \circ \Phi)'(q)}~\de q 
            - \int_{q_0}^1 \sqrt{\tPhi'_s(q) (\tp\times \xi^s \circ \tPhi)'(q)}~\de q 
        \rt|
        = o_D(1),
    \]
    where $\tp, \tPhi$ are the piecewise linear interpolations defined by \eqref{eq:continuous-def} and $o_D(1)$ is a term tending to $0$ as $D\to\infty$ (for fixed $(p,\Phi)$).
\end{lemma}
\begin{proof}
    The functions $\Phi,\Phi',p,p'$ are uniformly continuous because they are continuous on $[q_0,1]$.
    So,
    \[
        \tnorm{\Phi-\tPhi}_\infty,
        \tnorm{\Phi'-\tPhi'}_\infty,
        \tnorm{p-\tp}_\infty,
        \tnorm{p'-\tp'}_\infty
        = o_D(1).
    \]
    Bounded convergence implies the result. 
\end{proof}


\begin{proof}[Proof of Proposition~\ref{prop:bogp-alg}]
    By Proposition~\ref{prop:bogp-equivalent} it suffices to prove that $\BOGP_{\loc,0}=\ALG$. 
    We will separately show $\BOGP_{\loc,0} \le \ALG$ and $\BOGP_{\loc,0} \ge \ALG$.

    We first show $\BOGP_{\loc,0} \le \ALG$. 
    Let $\iota > 0$.
    Let $D$ be sufficiently large, $\eps = D^{-2}$ and $\delta = 6r/D$, and $k$ be sufficiently large depending on $D$ such that the following holds. 
    First, $k \ge k_0 (D,\eps,\delta)$ for $k_0$ defined in Proposition~\ref{prop:uc-bogp}. 
    Second, for some $1/D^2$-separated $\vchi \in \bbI(0,1)^\sS$ and all $\delta$-dense $(\up,\uvphi)$ with $\uvphi = \vchi(\up)$, we have
    \[
        \fr1N \bbE \sup_{\ubsig \in \cQ_\loc(0)} \cH_N(\ubsig)
        \ge \BOGP_{\loc,0} - \iota/2.
    \]
    Let $q_d = \la \vlam, \vphi_d\ra$.
    Let $\tp,\tPhi$ be the piecewise linear interpolations defined by \eqref{eq:continuous-def} and $A_d^s$ be defined by \eqref{eq:def-Ads}.
    Then 
    \begin{align*}
        &\lt|
            \fr1N \sup_{\ubsig \in \cQ_{\loc}(0)} \cH_N(\ubsig) 
            - \sum_{s\in \sS} \lt(
                h_s \lambda_s \sqrt{\tPhi_s(q_0)}
                + \lambda_s \int_{q_0}^{q_D}
                \sqrt{\tPhi'_s(q) (p\times \xi^s \circ \Phi)(q)}
                ~\de q
            \rt)
        \rt| \\
        &\le \lt|
            \fr1N \sup_{\ubsig \in \cQ_{\loc}(0)} \cH_N(\ubsig) 
            - \sum_{s\in \sS} \lt(
                h_s \lambda_s \sqrt{\phi_0^s}
                + \sum_{d=0}^{D-1} 
                f\lt(\vphi_d,\vphi_{d+1};p_d,p_{d+1};k\rt)
            \rt)
        \rt| \\
        &\quad + 
        \sum_{d=0}^{D-1} 
        \lt|
            f\lt(\vphi_d,\vphi_{d+1};p_d,p_{d+1};k\rt)
            - \sum_{s\in \sS} \lambda_s A_d^s
        \rt|
        + \sum_{s\in \sS} \lambda_s \lt|
            \sum_{d=0}^{D-1} A_d^s 
            - \int_{q_0}^{q_D}
            \sqrt{\tPhi'_s(q) (p\times \xi^s \circ \Phi)(q)}
            ~\de q
        \rt|.
    \end{align*}
    By Propositions~\ref{prop:uc-bogp}, \ref{prop:what-F-is} and Lemma~\ref{lem:ALG-derivation}, on an event with probability $1-e^{-cN}$ this is bounded by
    \[
        2D\eps + O(D\delta^{3/2} + D(\delta/k)^{1/2}) + O(D^{-1/2}) + o_N(1)
        = O(D^{-1/2} + (D/k)^{1/2}) + o_N(1)
        \le \iota/4,
    \]
    for sufficiently large $N,D,k$.
    Because $\fr1N \sup_{\ubsig \in \cQ_{\loc}(0)} \cH_N(\ubsig) $ is subgaussian with fluctuations $O(N^{-1/2})$ by Lemma~\ref{lem:bogp-subgaussian}, the contributions of the complement of this event are $o_N(1)$, and so
    \begin{equation}
        \label{eq:bogp-final-estimate}
        \lt|
            \fr1N \bbE \sup_{\ubsig \in \cQ_{\loc}(0)} \cH_N(\ubsig) 
            - \sum_{s\in \sS} \lt(
                h_s \lambda_s \sqrt{\tPhi_s(q_0)}
                + \lambda_s \int_{q_0}^{q_D}
                \sqrt{\tPhi'_s(q) (p\times \xi^s \circ \Phi)(q)}
                ~\de q
            \rt)
        \rt|
        \le \iota/4.
    \end{equation}
    Let $p \in \bbI(q_0,1)$ and $\Phi \in \Adm(q_0,1)$ approximate the piecewise linear functions $(\tp,\tPhi)$ on $[q_0,q_D]$, in the sense that
    \begin{equation}
        \label{eq:approximate-piecewise-linear}
        \lt|
            \sum_{s\in \sS}
            \lambda_s
            \int_{q_0}^{q_D}
            \lt(\sqrt{\tPhi_s'(q)(\tp\times \xi^s\circ\tPhi)'(q)}
            - \sqrt{\Phi_s'(q)(p\times \xi^s\circ\Phi)'(q)}\rt)
            ~\de q
        \rt|
        \le \iota/4.
    \end{equation}
    It is clear that such $p,\Phi$ exist. 
    Thus
    \[
        \BOGP_{\loc,0} \le \bbA(p,\Phi;q_0) - \iota \le \ALG - \iota.
    \]
    Since $\iota$ was arbitrary, we conclude $\BOGP_{\loc,0} \le \ALG$. 

    Next, we will show $\BOGP_{\loc,0} \ge \ALG$.
    Let $\iota > 0$, and let $D$ be sufficiently large and $k$ be sufficiently large depending on $D$.
    There exist $q_0\in [0,1]$, $p\in \bbI(q_0,1)$, and $\Phi \in \Adm(q_0,1)$ such that 
    \[
        \bbA(p,\Phi;q_0) \ge \ALG - \iota/2.
    \]
    By replacing $\Phi$ with $(1-D^{-2})\Phi + D^{-2} \vone$ we may assume $\Phi(q_0) \succeq \vone/D^2$, as this replacement affects the left-hand side by $o_D(1)$.
    Similarly, by replacing $p(q)$ with $(1-D^{-1})p(q) + D^{-1}q$, we may assume $p$ is strictly increasing.
    We choose $\vchi = \Phi \circ p^{-1}$, which is $1/D^2$-separated.
    Consider any $\uq = (q_0,q_1,\ldots,q_D)$ with $q_0<q_1<\cdots<q_D=1$ such that for $p_d = p(q_d)$, $\vphi_d = \Phi(q_d)$, the pair $(\up,\uvphi)$ is $6r/D$-dense.
    Similarly to above, we have \eqref{eq:bogp-final-estimate} for sufficiently large $N,D,k$.
    By Lemma~\ref{lem:ALG-from-continuum}, \eqref{eq:approximate-piecewise-linear} holds for $D$ sufficiently large. 
    This implies 
    \[
        \bbA(p,\Phi;q_0) \le \BOGP_{\loc,0}+\iota/2,
    \]
    and so $\ALG \le \BOGP_{\loc,0} + \iota$. 
    Because $\iota$ was arbitrary, we have $\ALG \le \BOGP_{\loc,0}$. 
\end{proof}
