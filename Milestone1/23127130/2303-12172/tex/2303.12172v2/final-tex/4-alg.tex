\section{Optimization of the Algorithmic Variational Principle}
\label{sec:alg}

In this section we will prove Propositions~\ref{prop:root-finding-trajectory} and \ref{prop:tree-descending-trajectory} and Theorem~\ref{thm:alg-optimizer}. 
Throughout this section we assume Assumption~\ref{as:nondegenerate} except where stated. 


To ensure a priori existence of a maximizer in \eqref{eq:alg}, we work in the following compact space which removes the constraint that $p$ and $\Phi$ are continuously differentiable.


\begin{definition}
\label{defn:cM}
    The space $\cM$ consists of all triples $(p,\Phi,q_0)$ such that:
    \begin{itemize}
    \item $q_0\in [0,1]$.
    \item $p:[q_0,1]\to [0,1]$ is non-decreasing and right-continuous (we write $p\in \hbbI(q_0,1)$).
    \item $\Phi=(\Phi_s)_{s\in\sS}$ consists of $r$ non-decreasing functions $\Phi_s:[q_0,1]\to [0,1]$ satisfying admissibility \eqref{eq:admissible} (we write $\Phi\in \hAdm(q_0,1)$).
    \end{itemize}
\end{definition}



Because we assume almost no regularity for elements of $\cM$, we formally define the integral in \eqref{eq:alg-functional} as follows.
Since $(p\times \xi^s \circ \Phi)$ is a bounded increasing function, it has a positive measure valued distributional derivative 
\begin{equation}
    \label{eq:careful-alg}
    (p\times \xi^s \circ \Phi)'(q) = f(q)~\de q + \de \mu(q)
\end{equation}
where $f\in L^1([q_0,1])$ and $\mu$ is an atomic-plus-singular measure supported in $[q_0,1]$.
Moreover, \eqref{eq:admissible} implies $\Phi_s$ is $\lambda_s^{-1}$-Lipschitz, hence has distributional derivative $\Phi'_s \in L^\infty([q_0,1])$.


\begin{definition}
For $(p,\Phi,q_0)\in\cM$, define 
\begin{equation}
    \label{eq:halg}
    \hALG
    \equiv
    \sup_{(p,\Phi,q_0)\in \cM}
    \bbA(p,\Phi; q_0).
\end{equation}
where the second term of $\bbA$ is given (with $f$ as in \eqref{eq:careful-alg}) by:
\[
    \int_{q_0}^1
    \sqrt{\Phi'_s(q) (p\times \xi^s \circ \Phi)'(q)}
    ~\de q
    =
    \int_{q_0}^1
    \sqrt{\Phi'_s(q) f(q)}
    ~\de q.
\]
\end{definition}


It will follow from our results in this section that for non-degenerate $\xi$, all maximizers to the extended variational problem are continuously differentiable on $[q_0,1]$. The equality $\ALG=\hALG$ follows in general since both are continuous in $(\xi,\vh)$. 



\begin{remark}
    A related (for the most part, simpler) variational problem was considered in \cite{deuschel1995limiting}. 
    There, after showing existence and other basic properties, the general result \cite[Theorem 5.1]{cesari2012optimization} was used to derive an ordinary differential equation \cite[Theorem 4]{deuschel1995limiting} for the optimal $\Phi$.
    The same general result applies in our setting, and essentially yields Proposition~\ref{prop:psi}, assuming $f_s$ defined in \eqref{eq:f-s-q} are absolutely continuous for all $s\in\sS$. 
    More precisely, under this assumption one finds (cf. \eqref{eq:Psi-s}, \eqref{eq:derivative-stability}):
    \[
    \sum_{s\in\sS}
    \Psi_s(q)
    \big(p\times \partial_s \xi^{s'}\circ\Phi\big)(q)\Phi_s'(q)
    =
    \Psi_{s'}(q)
    \big(p\times \partial_s \xi^{s'}\circ\Phi\big)'(q).
    \]
    Viewing this as a linear system in the variables $\Psi_s(q)$, Corollary~\ref{cor:rank} implies that if $p'(q)>0$, then $\Psi_1(q)=\Psi_2(q)=\dots=\Psi_r(q)=0$.
    Similarly if $p'(q)=0$, Lemma~\ref{lem:rank-v2} with $\eps=0$ implies $\Psi_1(q)=\Psi_2(q)=\dots=\Psi_r(q)$.
    However the only way we could establish absolute continuity of $f_s$ was by going through the full proof of Proposition~\ref{prop:psi}.
\end{remark}



\subsection{Linear Algebraic and Analytic Preliminaries}

We first prove Corollary~\ref{cor:solvability-equivalent} below, an equivalent characterization of (super, strict sub)-solvability.
\begin{proposition}
    \label{prop:smallest-eigenvalue}
    Let $M \in \bbR^{\sS \times \sS}$ be diagonally signed.
    Then 
    \begin{equation}
        \label{eq:VM-def}
        \Lambda(M)=\sup_{\vv\in \mathbb R_{>0}^{\sS}} \min_{s\in\sS} \frac{(M\vv)_s}{v_s}
    \end{equation}
    equals the smallest eigenvalue $\lambda_{\min}(M)$ of $M$. 
\end{proposition}
\begin{proof}
    Let $\vw$ be a (unit) minimal eigenvector of $M$.
    Note that
    \[
        \vw^\top M \vw
        = \sum_{s,s' \in \sS} M_{s,s'} w_s w_{s'}
        \ge \sum_{s,s' \in \sS} M_{s,s'} |w_s| |w_{s'}|.
    \]
    Since $\vw$ minimizes $\vw^\top M \vw$, all entries of $\vw$ are the same sign.
    We may thus assume $\vw \in \bbR_{\ge 0}^\sS$.
    Moreover, if $w_s=0$ for any $s$, then $(M\vw)_s < 0$ so $\vw$ is not an eigenvector; thus $\vw \in \bbR_{>0}^\sS$.
    Because $M\vw = \lambda_{\min}(M)\vw$, clearly $\Lambda(M) \ge \lambda_{\min}(M)$.
    For any other $\vv \in \bbR_{>0}^\sS$,
    \[
        \min_{s\in \sS}
        \fr{(M\vv)_s}{v_s}
        \le 
        \la \vw,\vv \ra^{-1} \sum_{s\in \sS} w_sv_s \cdot \fr{(M\vv)_s}{v_s}
        = \fr{\la \vw, M\vv \ra}{\la \vw,\vv \ra}
        = \fr{\la M\vw, \vv \ra}{\la \vw,\vv \ra}
        = \lambda_{\min}(M),
    \]
    so $\Lambda(M) \le \lambda_{\min}(M)$.
    Thus $\Lambda(M) = \lambda_{\min}(M)$.
\end{proof}
\begin{corollary}
    \label{cor:solvability-equivalent}
    For $\vx \in (0,1]^\sS \cup \{\vzero\}$ define
    \[
        M^*(\vx) 
        = \diag\lt((\xi^s(\vx) + h_s^2)_{s\in \sS}\rt)
        - \lt(x_s \partial_{x_{s'}} \xi^s(\vx)\rt)_{s,s'\in \sS}.
    \]
    Then $\vx$ is super-solvable (resp. solvable, strictly sub-solvable) if and only if $\Lambda(M^*(\vx)) \ge 0$ (resp. $=0$, $<0$).
\end{corollary}
\begin{proof}
    Suppose first $\vx \in (0,1]^\sS$. 
    By Proposition~\ref{prop:smallest-eigenvalue}, $\vx$ is super-solvable (resp. solvable, strictly sub-solvable) if and only if $\Lambda(M^*_\sym(\vx)) \ge 0$ (resp. $=0$, $<0$). 
    Note that 
    \begin{equation}
        \label{eq:M*sym-to-M*}
        M^*(\vx) = \diag\lt((\lambda_s x_s)_{s\in \sS}\rt) M^*_\sym(\vx),
    \end{equation}
    so $\Lambda(M^*(\vx))$ has the same sign as $\Lambda(M^*_\sym(\vx))$, as desired.
    If $\vx = \vzero$, then clearly $\Lambda(M^*(\vx)) \ge 0$ with equality at $\vh = \vzero$, which agrees with the convention from Definition~\ref{defn:solvable}.
\end{proof}


The following proposition is clear.

\begin{proposition}
\label{prop:VM}
Let $\Lambda$ be as in \eqref{eq:VM-def} and let $M\in\bbR_{\geq 0}^{r\times r}$ (not necessarily diagonally signed). Then $\Lambda(M)$ is non-negative and locally bounded. Moreover if for some $c\in\bbR$ we have $M'_{s,s'}\geq M_{s,s'}+c\cdot 1_{s=s'}$ for all $s,s'$, then $\Lambda(M')\geq \Lambda(M)+c$.
\end{proposition}

Many perturbation arguments used to establish regularity rely on the following basic fact.

\begin{proposition}[{\cite[Theorem 7.7]{rudin1970real}}]
    \label{prop:lebesgue}
    For $f\in L^{1}([0,1])$, almost all $x\in [0,1]$ are \textbf{Lebesgue points}: 
    \[
      \lim_{\eps\to 0}
      \frac{1}{2\eps} 
      \int_{x-\eps}^{x+\eps}
      |f(y)-f(x)|
      ~\de y = 0.
    \]
\end{proposition}

The next fact ensures that Lipschitz ordinary differential equations are well-posed (even if they are only required to hold almost everywhere).

\begin{proposition}[{\cite[Theorem 1.45, Part (ii)]{roubivcek2013nonlinear}}]
\label{prop:ODE-well-posed}
Suppose $Y_1,Y_2:[0,1]\to\bbR^d$ are each absolutely continuous with $Y_1(0)=Y_2(0)$ and solve the ODE $Y_i'(q)=F(Y_i(q))$ at almost all $q$ for $F:\bbR^d\to\bbR^d$ Lipschitz. Then $Y_1,Y_2$ agree and solve the ODE for all $q$.
\end{proposition}

\subsection{A Priori Regularity of Maximizers}
\label{subsec:basic-regularity}

We first show that for the optimization problem \eqref{eq:alg}, admissibility \eqref{eq:admissible} is just a convenient choice of normalization. 
This makes variational arguments more convenient because we do not need to worry about preserving admissibility of $\Phi$ under perturbations.
Let $\tbbI(q_0,1) \subseteq \hbbI(q_0,1)$ be the set of increasing and Lipschitz functions $f:[q_0,1] \to [0,1]$ with no explicit bound on the Lipschitz constant and with $f(1)=1$.
Note that the algorithmic functional $\bbA$ \eqref{eq:alg-functional} remains well-defined for $\Phi \in \tbbI(q_0,1)^\sS$. 
\begin{lemma}
    \label{lem:admissible-optional}
    We have that
    \begin{equation}
        \label{eq:admissible-optional}
        \hALG =
        \sup_{q_0\in [0,1]}
        \sup_{\substack{p\in \hbbI(q_0,1) \\ \Phi \in \tbbI(q_0,1)^\sS}} 
        \bbA(p,\Phi; q_0).
    \end{equation}
\end{lemma}
\begin{proof}
    Let $\hALG'$ be the right-hand side of \eqref{eq:admissible-optional}.
    We will show that $\hALG \geq \hALG'$ (the opposite implication being trivial).
    
    Consider any $q_0 \in [0,1]$, $p\in \hbbI(q_0,1)$, and $\Phi \in \tbbI(q_0,1)^\sS$.
    For small $\delta > 0$, consider
    \[
        \Phi_{\delta} (q) = \delta q \vone + (1-\delta) \Phi(q)
    \]
    and let $\alpha(q) = \la \vlam, \Phi_{\delta}(q)\ra$, so $\alpha'(q) \ge \delta$. 
    Thus $\alpha^{-1}$ exists and is $\delta^{-1}$-Lipschitz. 
    Consider $(\wtp,\tPhi,\wtq_0)$ given by
    \[
        \wtp(q) = p(\alpha^{-1}(q)),
        \quad 
        \tPhi(q) = \Phi(\alpha^{-1}(q)),
        \quad 
        \wtq_0 = \alpha(q_0).
    \]
    By construction, $\tPhi \in \hAdm(\wtq_0,1)$.
    By the chain rule, $\bbA(\wtp,\tPhi;\wtq_0) = \bbA(p,\Phi_{\delta};q_0)$.
    Thus
    \[
        \hALG 
        \ge 
        \limsup_{\delta\downarrow 0}
        \bbA(\wtp,\tPhi,\wtq_0)
        =
        \limsup_{\delta\downarrow 0}
        \bbA(p,\Phi_{\delta};q_0)
        \ge 
        \bbA(p,\Phi;q_0).
    \]
    Since $p,\Phi,q_0$ were arbitrary the conclusion follows.
\end{proof}

A routine compactness argument given in Appendix~\ref{subsec:maximizer-existence} yields the following. 
\begin{restatable}{proposition}{propFmax}
\label{prop:F-max}
There exists a maximizer $(p,\Phi,q_0)\in\cM$ for $\bbA$ and $\bbA(p,\Phi;q_0)<\infty$.
\end{restatable}

From now on, we let $(p,\Phi,q_0)\in\cM$ denote any maximizer and study the behavior of $(p,\Phi,q_0)$. 
While almost no regularity on $(p,\Phi)$ is assumed, it is possible to establish a priori regularity using variational arguments. 
We defer the proofs of the following two propositions to Appendix~\ref{subsec:regularity-for-4.1}. 
Proposition~\ref{prop:basic-regularity} implies that the discussion following \eqref{eq:alg} is not necessary to define $\bbA(p,\Phi;q_0)$.
\begin{restatable}{proposition}{propBasicRegularity}
    \label{prop:basic-regularity}
    The functions $p,\Phi$ are continuously differentiable on $[q_0+\eps,1]$ for any $\eps>0$.
    Moreover, there exists $L>0$ (possibly depending on $(p,\Phi;q_0)$ as well as $\xi$) such that $L^{-1} \vone \preceq \Phi'(q) \preceq L \vone$ for almost all $q\in (q_0,1]$.
\end{restatable}

\begin{restatable}{proposition}{propPBasic}
    \label{prop:p-basic}
    The function $p$ satisfies $p(q)>0$ for all $q>q_0$, $p(1)=1$, and $p(q_0)=0$ if $q_0>0$.
\end{restatable}


Throughout the next subsection we will use $\eps>0$ as in Proposition~\ref{prop:basic-regularity}. Later we slightly improve the result of Proposition~\ref{prop:basic-regularity} to continuity on $[q_0,1]$ using more detailed properties of the maximizers. 

\subsection{Identification of Root-Finding and Tree-Descending Phases}
\label{subsec:Psi-calculation}

In this subsection we will prove the following result. Recall that the Sobolev space $W^{2,\infty}([q_0+\eps,1])$ consists of $C^1$ functions with Lipschitz derivative on the interval.
\begin{proposition}
    \label{prop:type-12}
    The restrictions of $p$ and $\Phi_s$, for all $s\in \sS$, lie in the space $W^{2,\infty}([q_0+\eps,1])$ for any $\eps>0$. 
    There exists $q_1\in [q_0,1]$ such that the following holds. 
    \begin{enumerate}[label=(\alph*), ref=\alph*]
        \item On $[q_0,q_1]$, $p'>0$ almost everywhere and the quantities $\fr{\Phi'_s(q)}{(p\times \xi^s \circ \Phi)'(q)}$ are constant.
        Moreover $p(q_1)=1$.
        \item On $[q_1,1]$, the ODE \eqref{eq:tree-descending-ode} is satisfied for all $s,s'\in \sS$ almost everywhere and $p=1$.
    \end{enumerate}
\end{proposition}

We begin with a result on diagonally dominant matrices. Variants especially with $\eps=0$ have been used many times, see e.g. \cite{taussky1949recurring}.
Related linear algebraic statements will appear later in Lemmas~\ref{lem:positive-linalg-with-p} and \ref{lem:pos-linalg-diagonal-must-grow} as, roughly speaking, $r$-dimensional analogs of monotonicity. 

\begin{lemma}
    \label{lem:rank-v2}
    Let $A=(a_{i,j})_{i,j\in [r]} \in \bbR^{r\times r}$ satisfy $a_{i,i}>0$ and $a_{i,j} < 0$ for all $i\neq j$.
    \begin{enumerate}[label=(\alph*), ref=\alph*]
        \item 
        \label{it:zero-sum-v2}
        If $\sum_{j=1}^r a_{i,j}=0$ for all $i\in [r]$, then all solutions $\vv \in \bbR^r$ to $A\vv \preceq \eps\vone$ satisfy $|v_i-v_j| \le \eps / a_{\min}$ for all $i,j$, where $a_{\min} = \min_{i\neq j} |a_{i,j}|$.
        \item 
        \label{it:positive-sum-v2}
        If $\sum_{j=1}^r a_{i,j}\geq d_{\min}>0$ for all $i\in [r]$, then all solutions $\vv \in \bbR^r$ to $\tnorm{A\vv}_\infty \le \eps$ satisfy $\|v_i\|_{\infty} \le \eps / d_{\min}$.
    \end{enumerate}
\end{lemma}
\begin{proof}[Proof of Lemma~\ref{lem:rank-v2}]
    Assume without loss of generality that $v_1\geq v_s$ for all $s$. 
    If $\sum_{j=1}^r a_{i,j}=0$ for all $i\in [r]$, then
    \[
        \eps 
        \ge (A\vv)_1 
        = a_{1,1}v_1 + \sum_{j=2}^r a_{1,j}v_i
        = \sum_{j=2}^r |a_{1,j}|(v_1-v_j)
        \ge a_{\min} (v_1-v_i)
    \]
    for all $i\ge 2$.
    Thus $v_1-v_i \le \eps / a_{\min}$, proving the first part.
    For the second part, we will first show $v_1 \le \eps / d_{\min}$. 
    If $v_1 < 0$ there is nothing to prove, and otherwise
    \[
        \eps
        \ge 
        (A\vv)_1 
        = 
        a_{1,1}v_1 + \sum_{j=2}^r a_{1,j}v_j 
        \ge 
        \lt(a_{1,1} - \sum_{j=2}^r a_{1,j}\rt) v_1
        \ge 
        d_{\min} v_1.
    \]
    So $v_1 \le \eps / d_{\min}$, as claimed.
    Finally, note that if $\tnorm{A\vv}_\infty \le \eps$, the same is true for $-\vv$.
    By the same argument we find the largest entry of $-\vv$ is at most $\eps / d_{\min}$.
    This implies the second part.
\end{proof}
\begin{corollary}
    \label{cor:rank}
    Let $A=(a_{i,j})_{i,j\in [r]} \in \bbR^{r\times r}$ satisfy $a_{i,i}>0$ and $a_{i,j} < 0$ for all $i\neq j$.
    If $\sum_{j=1}^r a_{i,j}>0$ for all $i\in [r]$, then the only solution to $A\vv = \vzero$ is $\vv=\vzero$.
\end{corollary}
\begin{proof}
    Apply Lemma~\ref{lem:rank-v2}(\ref{it:positive-sum-v2}) with $\eps=0$. 
\end{proof}

To establish additional regularity we use the following fact on distributional derivatives.

\begin{lemma}[{See e.g. \cite[Theorem 2.2.1]{ziemer2012weakly}}]
    \label{lem:manual-integration-by-parts}
    If $A,B \in L^{\infty}([q_0,1])$ satisfy 
    \[
        \int_{q_0}^1 
        A(q) \psi(q) + B(q) \psi'(q)
        ~\de q 
        = 0
    \]
    for all $\psi \in C_c^\infty((q_0,1);\bbR)$, then there exists $C\in \bbR$ such that for all $q\in [q_0,1]$, 
    \[
        B(q) = \int_{q_0}^q A(t)~\de t + C.
    \]
\end{lemma}


We will make use of the functions
\begin{equation}
\label{eq:f-s-q}
    f_s(q) = \sqrt{\fr{\Phi'_s(q)}{(p\times \xi^s \circ \Phi)'(q)}}
\end{equation}
Note that Propositions~\ref{prop:basic-regularity} and \ref{prop:p-basic} imply $f_s$ is continuous on $[q_0+\eps,1]$.


\begin{proposition}
    \label{prop:psi}
    The functions $f_s$ are Lipschitz on $[q_0+\eps,1]$.
    Thus (recall Proposition~\ref{prop:basic-regularity}) the functions 
    \begin{equation}
    \label{eq:Psi-s}
    \Psi_s(q) = f'_s(q)/\Phi'_s(q)
    \end{equation}
    are measurable and locally bounded on $(q_0,1]$.
    Moreover for almost all $q\in (q_0,1]$, the following holds:
    \begin{equation}    
        \label{eq:psi-equality}
        \Psi_1(q)=\cdots=\Psi_s(q),
    \end{equation}
    and furthermore this common value is $0$ if $p'(q)>0$.
\end{proposition}
\begin{proof}
    Let $\psi \in C_c^\infty((q_0,1);\bbR)$.
    Consider the perturbation
    \[
        \tPhi_1(q) = \Phi_1(q) + \delta \psi(q),
    \]
    and let $\tPhi_s(q) = \Phi_s(q)$ for $s\neq 1$.
    By Proposition~\ref{prop:basic-regularity}, $\tPhi$ remains coordinate-wise increasing and Lipschitz for small positive and negative $\delta$.
    Although $\tPhi \not \in \hAdm(q_0,1)$, recalling Lemma~\ref{lem:admissible-optional} we nonetheless have $\bbA(p,\tPhi;q_0) \le \bbA(p,\Phi;q_0)$.
    Thus,
    \[
        F_1 \equiv \fr{\de}{\de \delta} \bbA(p,\tPhi;q_0) \Big|_{\delta=0} = 0.
    \]
    We now calculate $F_1$. 
    Note that
    \begin{equation}
        \label{eq:phi1-deriv}
        \fr{\de}{\de \delta} (p\times \xi^s \circ \tPhi)'(q) 
        \Big|_{\delta=0}
        = 
        (p\psi \times \partial_{x_1}\xi^s \circ \Phi)'(q)
        =
        \fr{\lambda_1}{\lambda_s}
        (p\psi \times \partial_{x_s}\xi^1 \circ \Phi)'(q).
    \end{equation}
    So,
    \begin{align*}
        0 = \fr{2}{\lambda_1} F_1
        &= 
        \int_{q_0}^1
        f_1(q)^{-1}
        \psi'(q)~\de q
        +
        \sum_{s\in \sS}
        \int_{q_0}^1
        f_s(q)
        (p\psi \times \partial_{x_s}\xi^1 \circ \Phi)'(q)
        ~\de q \\
        &= 
        \int_{q_0}^1
        A_1(q) \psi(q) + B_1(q) \psi'(q) 
        ~\de q
    \end{align*}
    where
    \[
        A_1(q)
        \equiv
        \sum_{s\in \sS}
        f_s(q)
        (p\times \partial_{x_s} \xi^1 \circ \Phi)'(q), 
        \quad 
        B_1(q) 
        \equiv
        f_1(q)^{-1}
        + \sum_{s\in \sS}
        f_s(q)
        (p\times \partial_{x_s} \xi^1 \circ \Phi)(q).
    \]
    By Proposition~\ref{prop:basic-regularity}, for all $\eps>0$ $A_1(q)$ and $B_1(q)$ are bounded for $q\in[q_0+\eps,1]$.
    Lemma~\ref{lem:manual-integration-by-parts} implies that $B_1(q)$ is absolutely continuous and $B'_1(q) = A_1(q)$ for all $q\in (q_0,1]$. In fact by Proposition~\ref{prop:basic-regularity}, $A_1$ is bounded and continuous on $[q_0+\eps,1]$, so $B_1\in C^1([q_0+\eps,1])$ (for all $\eps>0$).

    Fix $q\in (q_0,1]$. 
    For $\iota \in \bbR$ with $|\iota|$ small, let $\Delta^\iota_s = f_s(q+\iota) - f_s(q)$. 
    By Proposition~\ref{prop:basic-regularity} all $f_s$ are continuous, so $\Delta^\iota_s =o(1)$; here are below we use $o(\cdot)$ for limits as $\iota\to 0$.
    Thus,
    \begin{align*}
        B_1(q+\iota) - B_1(q)
        &= 
        \fr{1}{f_1(q)+\Delta^\iota_1} - \fr{1}{f_1(q)}
        + \sum_{s\in \sS}
        \Delta^\iota_s
        \cdot
        (p\times \partial_{x_s} \xi^1 \circ \Phi)(q) \\
        &\quad + \sum_{s\in \sS} f_s(q+\iota)\lt((p\times \partial_{x_s} \xi^1 \circ \Phi)(q+\iota) - (p\times \partial_{x_s} \xi^1 \circ \Phi)(q)\rt).
    \end{align*}
    Since $(p\times \partial_{x_s} \xi^1 \circ \Phi)$ is differentiable and $f_s$ is continuous,
    \begin{align*}
        \sum_{s\in \sS} 
        f_s(q+\iota)
        \cdot
        \big(
        (p\times \partial_{x_s} \xi^1 \circ \Phi)(q+\iota) - (p\times \partial_{x_s} \xi^1 \circ \Phi)(q)
        \big)
        &=
        \iota 
        \sum_{s\in \sS} f_s(q)(p\times \partial_{x_s} \xi^1 \circ \Phi)'(q) + o(|\iota|) \\
        &= \iota A_1(q) + o(|\iota|).
    \end{align*}
    Moreover,
    \begin{align*}
        \fr{1}{f_1(q)+\Delta^\iota_1} - \fr{1}{f_1(q)}
        &= 
        -\fr{\Delta^\iota_1}{f_1(q)(f_1(q)+\Delta^\iota_1)} 
        = 
        \fr{(\Delta^\iota_1)^2}{f_1(q)^2(f_1(q)+\Delta^\iota_1)}
        -\fr{\Delta^\iota_1}{f_1(q)^2} \\
        &= 
        \fr{(\Delta^\iota_1)^2}{f_1(q)^2(f_1(q)+\Delta^\iota_1)}
        - \fr{\Delta^\iota_1}{\Phi'_1(q)} \lt(
            p'(q)(\xi^1\circ \Phi)(q) + 
            \sum_{s\in \sS} (p\times \partial_{x_s} \xi^1 \circ \Phi)(q)\Phi'_s(q)
        \rt).
    \end{align*}
    We also have $B_1(q+\iota) - B_1(q) = A_1\iota + o(|\iota|)$ (recall $A_1$ is continuous).
    Thus
    \begin{equation}
        \label{eq:derivative-stability}
        \sum_{s\in \sS}
        (p\times \partial_{x_s} \xi^1 \circ \Phi)(q) \Phi'_s(q)
        \lt[\fr{\Delta^\iota_1}{\Phi'_1(q)} - \fr{\Delta^\iota_s}{\Phi'_s(q)}\rt]
        + p'(q) (\xi^1 \circ \Phi)(q) \fr{\Delta^\iota_1}{\Phi'_1(q)}
        - \fr{(\Delta^\iota_1)^2}{f_1(q)^2(f_1(q)+\Delta^\iota_1)}
        =
        o(|\iota|).
    \end{equation}
    We get similar equations from perturbing any $\Phi_{s}$ instead of $\Phi_1$.
    If $p'(q)>0$, then we can write the last two terms on the left-hand side of \eqref{eq:derivative-stability} as 
    \[
        \fr{\Delta^\iota_1}{\Phi'_1(q)} \lt(
            p'(q) (\xi^1 \circ \Phi)(q)
            - \fr{\Delta^\iota_1 \Phi'_1(q)}{f_1(q)^2(f_1(q)+\Delta^\iota_1)}
        \rt)
        = \fr{\Delta^\iota_1}{\Phi'_1(q)} \lt(p'(q) (\xi^1 \circ \Phi)(q) + o(1)\rt).
    \]
    Then, \eqref{eq:derivative-stability} and its analogs form a linear system in variables $x_s\equiv\Delta^\iota_s/\Phi'_s(q)$ with all row sums positive. (E.g. in \eqref{eq:derivative-stability}, the first term gives zero coefficient sum so the total coefficient sum is just $\lt(p'(q) (\xi^1 \circ \Phi)(q) + o(1)\rt)>0$.) Moreover the diagonal coefficients of this system are e.g.
    \[
    a_{1,1}
    =
    \lt(p'(q) (\xi^1 \circ \Phi)(q) + o(1)\rt)
    +
    \sum_{s\in \sS}
        (p\times \partial_{x_s} \xi^1 \circ \Phi)(q) \Phi'_s(q)>0
    \]
    while the off-diagonal coefficients are e.g.
    \[
    a_{1,s}
    =
    -
    (p\times \partial_{x_s} \xi^1 \circ \Phi)(q) \Phi'_s(q)
    <0.
    \]
    Applying Lemma~\ref{lem:rank-v2}(\ref{it:positive-sum-v2}), we obtain 
    \[
    |\Delta^\iota_s/\Phi'_s(q)| = o(|\iota|)
    \]
    for all $s\in \sS$.
    Taking $\iota \to 0$ we conclude that $f'_s(q)$ is well-defined and equals $0$.
    This implies the conclusion for $p'(q)>0$.


    
    Otherwise $p'(q)=0$, and \eqref{eq:derivative-stability} implies that
    \[
        \sum_{s\in \sS}
        (p\times \partial_{x_s} \xi^1 \circ \Phi)(q) \Phi'_s(q)
        \lt[\fr{\Delta^\iota_1}{\Phi'_1(q)} - \fr{\Delta^\iota_s}{\Phi'_s(q)}\rt]
        \ge 
        -o(|\iota|)
    \]
    and analogously with any $s\in \sS$ in place of $1$.
    This is a linear system of inequalities in variables 
    $-\Delta^\iota_s/\Phi'_s(q)$, so Lemma~\ref{lem:rank-v2}(\ref{it:zero-sum-v2}) implies that 
    \begin{equation}
        \label{eq:psi-equality-discretized}
        \lt|\fr{\Delta^\iota_s}{\Phi'_s(q)} - \fr{\Delta^\iota_{s'}}{\Phi'_{s'}(q)}\rt| \le o(|\iota|)
    \end{equation}
    for all $s,s'\in \sS$.
    The result now follows if we find a constant $C=C(\eps)$ such that 
    \begin{equation}
        \label{eq:psi-goal}
        \max_{s\in \sS} |\Delta^\iota_s/\Phi'_s(q)| \le C|\iota|
    \end{equation}
    for all sufficiently small $\iota$, and $q\in [q_0+\eps,1]$.
    Indeed, this would imply by Proposition~\ref{prop:basic-regularity} that $f_s$ is Lipschitz on $[q_0+\eps,1]$. It would then follow that $\Psi\in L^{\infty}([q_0+\eps,1])$, and we would conclude from \eqref{eq:psi-equality-discretized} that $\Psi_1=\cdots=\Psi_r$ almost everywhere. 
    
    Since $p$ and $\partial_{x_s}\xi^1$ are differentiable and $p'(q)=0$, we have $p(q+\iota) = p(q) + o(|\iota|)$ and $\partial_{x_s}\xi^1(q+\iota) = \partial_{x_s}\xi^1(q) + O(|\iota|)$.
    Suppose first that $\iota > 0$.
    Using $p'(q)=0$ and $p'(q+\iota)\geq 0$, we find
    \begin{align}
        \label{eq:use-p'-zero}
        \Delta^\iota_1
        &\le
        \sqrt{\fr{\Phi'_1(q+\iota)}{p(q+\iota)(\xi^1\circ \Phi)'(q+\iota)}}
        -\sqrt{\fr{\Phi'_1(q)}{p(q)(\xi^1\circ \Phi)'(q)}} \\
        \notag
        &= 
        \fr{1}{\sqrt{p(q)}} \lt(
            \sqrt{\fr{\Phi'_1(q+\iota)}{\sum_{s\in \sS} (\partial_{x_s}\xi^1 \circ \Phi)(q) \Phi'_s(q+\iota)}}
            - \sqrt{\fr{\Phi'_1(q)}{\sum_{s\in \sS} (\partial_{x_s}\xi^1 \circ \Phi)(q) \Phi'_s(q)}}
        \rt) + O(\iota)
    \end{align}
    where we used that $p'(q)=0$ and $p'(q+\iota) \ge 0$. 
    The hidden constants are uniform on any interval $[q_0+\eps,1]$.
    Analogous bounds hold for $\Delta^\iota_s$. 
    We claim that we cannot have 
    \begin{equation}
        \label{eq:all-ratios-bigger}
        \fr{\Phi'_s(q+\iota)}{\sum_{s'\in \sS} (\partial_{x_{s'}}\xi^s \circ \Phi)(q) \Phi'_{s'}(q+\iota)}
        > \fr{\Phi'_s(q)}{\sum_{s'\in \sS} (\partial_{x_{s'}}\xi^s \circ \Phi)(q) \Phi'_{s'}(q)}
    \end{equation}
    for all $s\in \sS$. 
    Indeed, suppose this holds and let 
    \begin{align*}
        b_s &= \fr{\sum_{s'\in \sS} (\partial_{x_{s'}}\xi^s \circ \Phi)(q) \Phi'_{s'}(q)}{\Phi'_s(q)}, \\
        b'_s &= \fr{\sum_{s'\in \sS} (\partial_{x_{s'}}\xi^s \circ \Phi)(q) \Phi'_{s'}(q+\iota)}{\Phi'_s(q+\iota)},
    \end{align*}
    so $b'_s < b_s$. 
    The linear system given by
    \[
        b'_s\Phi'_s(q+\iota)x_s - \lt(\sum_{s'\in \sS} (\partial_{x_{s'}}\xi^s \circ \Phi)(q) \Phi'_{s'}(q+\iota)x_{s'}\rt) = 0
    \]
    for all $s\in \sS$ has solution $\vx = \vone$, and thus has row sums zero. 
    The linear system given by 
    \[
        b_s\Phi'_s(q+\iota)x_s - \lt(\sum_{s'\in \sS} (\partial_{x_{s'}}\xi^s \circ \Phi)(q) \Phi'_{s'}(q+\iota)x_{s'}\rt) = 0
    \]
    has solution $x_s = \Phi'_s(q)/\Phi'_s(q+\iota)$.
    However, by Corollary~\ref{cor:rank} its only solution is $\vx=\vzero$, contradiction.
    Thus \eqref{eq:all-ratios-bigger} does not hold for all $s\in \sS$. 
    Assume without loss of generality \eqref{eq:all-ratios-bigger} does not hold for $s=1$. 
    Then, $\Delta^\iota_1 \le O(\iota)$.
    In conjunction with \eqref{eq:psi-equality-discretized}, this implies $\max_{s\in \sS} \Delta^\iota_s/\Phi'_s(q) \le C\iota$.
    
    For the matching lower bound, first consider the case $p'(q+\iota)=0$. 
    In this case, the inequality in \eqref{eq:use-p'-zero} is an equality.
    We similarly cannot have
    \[
        \fr{\Phi'_s(q+\iota)}{\sum_{s'\in \sS} (\partial_{x_{s'}}\xi^s \circ \Phi)(q) \Phi'_{s'}(q+\iota)}
        < \fr{\Phi'_s(q)}{\sum_{s'\in \sS} (\partial_{x_{s'}}\xi^s \circ \Phi)(q) \Phi'_{s'}(q)}
    \]
    for all $s\in \sS$, so the same argument implies $\min_{s\in \sS} \Delta^\iota_s/\Phi'_s(q) \ge -C\iota$, which implies \eqref{eq:psi-goal}.
    Otherwise assume $p'(q+\iota)>0$. 
    Let $\iota_1 \in (0, \iota/2)$ be small enough that 
    \begin{equation}
        \label{eq:p'-bdd-from-0}
        p'(q')\ge \fr12 p'(q+\iota) \quad \text{for all}~q'\in [q+\iota-\iota_1,q+\iota]
    \end{equation} 
    which exists by continuity of $p'$. 
    Let $\psi \in C_c^\infty((q_0,1);\bbR)$ satisfy that $|\psi'|\le 1$ and $\psi'$ is supported on $[q,q+\iota_1]\cup [q+\iota-\iota_1,q+\iota]$, positive on $[q,q+\iota_1]$, and negative on $[q+\iota-\iota_1,q+\iota]$.
    (Note that $\psi'$ integrates to zero because $\psi$ has bounded support, and that $\psi$ is clearly nonnegative.)
    Let $\iota_2 = \psi(q+\iota_1)$.
    Consider the perturbation $\wtp = p + \delta \psi$, which is increasing for small $\delta>0$ by \eqref{eq:p'-bdd-from-0}.
    Let $o_{\iota_1}(1)$ denote a term tending to $0$ as $\iota_1\to0$. 
    We compute that
    \begin{align*}
        F &\equiv 
        \fr{\de}{\de \delta} \bbA(\wtp,\Phi;q_0) \Big|_{\delta=0} \\
        &= 
        \sum_{s\in \sS}
        \lambda_s
        \int_{q_0}^1
        f_s(q) (\psi \times \xi^s \circ \Phi)'(q) \\
        &\ge 
        \sum_{s\in \sS}
        \lambda_s
        \int_{q_0}^1
        \psi'(q) f_s(q) (\xi^s \circ \Phi)(q) && \text{(positivity of $\psi$)}\\
        &=
        \sum_{s\in \sS}
        \lambda_s \cdot
        \iota_2 \lt(
            f_s(q) (\xi^s \circ \Phi)(q)
            - f_s(q+\iota) (\xi^s \circ \Phi)(q+\iota)
            +o_{\iota_1}(1)
        \rt) && \text{(continuity of $f_s, \xi^s\circ \Phi$)} \\
        &= \iota_2 \lt(
            -\sum_{s\in \sS} 
            \lambda_s
            \Delta^\iota_s (\xi^s \circ \Phi)(q)
            +o_{\iota_1}(1)
            +O(\iota)
        \rt) && \text{(continuity of $\xi^s\circ \Phi$)} \\
        &= \iota_2 \lt(
            -\Delta^\iota_1
            \sum_{s\in \sS} 
            \frac{\Phi_s'(q)}{\Phi_1'(q)}\cdot
            \lambda_s
            (\xi^s \circ \Phi)(q)
            +o_{\iota_1}(1)
            +O(\iota)
        \rt). && \text{(by \eqref{eq:psi-equality-discretized})}
    \end{align*}
    Since $(p,\Phi,q_0)$ is a maximizer, $F \le 0$. 
    This implies $\Delta^\iota_1 \ge -C\iota$, and by \eqref{eq:psi-equality-discretized}, $\min \Delta^\iota_1 \ge -C\iota$.
    This proves \eqref{eq:psi-goal} for $\iota > 0$. 
    The proof for $\iota < 0$ is analogous.
\end{proof}

\begin{lemma}
    \label{lem:positive-linalg-with-p}
    Let $A = (a_{i,j}) \in \bbR_{>0}^{r\times r}$, $\va, \vb \in \bbR_{>0}^r$, $\vc \in \bbR_{>0}^r$, and $c\in \bbR_{>0}$.
    Let $A_{\min},a_{\min},b_{\min}$ denote the minimal entries of $A,\va,\vb$, and $a_{\max}$ denote the maximal entry of $\va$.
    Suppose the linear system
    \[
        A\vx + \va y = \vc \odot \vx, 
        \quad 
        \la \vb,\vx\ra = c
    \]
    has solution $(y,\vx) = (y_0,\vone)$.
    If ${\vc\,}' \in \bbR_{>0}^r$ satisfies $\tnorm{\vc-{\vc\,}'}_\infty \le \eps$, then any solution $y\in \bbR_{\ge 0}$, $\vx \in \bbR_{\ge 0}^r$ to 
    \[
        A\vx + \va y = {\vc\,}' \odot \vx, 
        \quad 
        \la \vb,\vx\ra = c
    \]
    satisfies 
    \[
        |y-y_0| \le \fr{\eps c}{a_{\min}b_{\min}},
        \quad
        \tnorm{\vx-\vone}_\infty \le \fr{2a_{\max}}{a_{\min}} \cdot \fr{\eps c}{A_{\min}b_{\min}}.
    \]
\end{lemma}
\begin{proof}
    Without loss of generality let $x_1$, $x_2$ be the largest and smallest entries of $\vx$. 
    As $\la \vb,\vone\ra = \la \vb,\vx\ra = c$, 
    \[
    \fr{c}{b_{\min}}\ge x_1\ge 1\ge x_2.
    \]
    Then
    \begin{align*}
        0 
        &= a_1 y + \sum_{i=1}^r a_{1,i}x_i - c'_1x_1 \\
        &\le a_1 y + \lt(\sum_{i=1}^r a_{1,i}-c'_1\rt)x_1 - A_{\min}(x_1-x_2) \\
        &= a_1 y + \lt(c_1-c'_1-a_1y_0\rt)x_1 - A_{\min}(x_1-x_2) \\
        &\le \eps x_1 - a_1y_0(x_1-1) + a_1(y-y_0) - A_{\min}(x_1-x_2) \\
        &\le \fr{\eps c}{b_{\min}} + a_1(y-y_0) - A_{\min}(x_1-x_2).
    \end{align*}
    Analogously
    \begin{align*}
        0 
        &= a_2 y + \sum_{i=1}^r a_{2,i}x_i - c'_2x_2 \\
        &\ge a_2 y + \lt(\sum_{i=1}^r a_{2,i}-c'_2\rt)x_2 + A_{\min}(x_1-x_2) \\
        &= a_2 y + \lt(c_2-c'_2-a_2y_0\rt)x_2 + A_{\min}(x_1-x_2) \\
        &\ge -\eps x_2 - a_2y_0(x_2-1) + a_2(y-y_0) + A_{\min}(x_1-x_2) \\
        &\ge -\fr{\eps c}{b_{\min}} + a_2(y-y_0) + A_{\min}(x_1-x_2).
    \end{align*}
    Since $x_1-x_2\ge 0$, this implies
    \[
        y-y_0 
        \ge -\fr{\eps c}{a_1b_{\min}}
        \ge -\fr{\eps c}{a_{\min}b_{\min}},
        \quad 
        y-y_0 
        \le \fr{\eps c}{a_2b_{\min}}
        \le \fr{\eps c}{a_{\min}b_{\min}},
    \]
    which proves the first conclusion.
    Thus,
    \[
        A_{\min}(x_1-x_2)
        \le 
        \lt(\fr{\eps c}{b_{\min}} + a_1(y-y_0)\rt) 
        \le 
        \fr{2a_{\max}}{a_{\min}} \cdot \fr{\eps c}{b_{\min}}.
    \]
    Since $x_1\ge 1\ge x_2$, we have $\tnorm{\vx - \vone}_\infty \le x_1-x_2$ which implies the second conclusion.
\end{proof}

\begin{proposition}
    \label{prop:twice-diff}
    The functions $p'$ and $\Phi'$ are Lipschitz on $[q_0+\eps,1]$ for all $\eps>0$. 
    Thus $p''$ and $\Phi''$ are well-defined as bounded measurable functions on $[q_0+\eps,1]$.
\end{proposition}
\begin{proof}
    By Proposition~\ref{prop:psi}, $f_s$ is Lipschitz on $[q_0+\eps,1]$.
    Since it is also bounded on $[q_0+\eps,1]$ by Proposition~\ref{prop:basic-regularity}, $f_s^{-2}$ is Lipschitz as well. 
    Thus, for $q\in [q_0+\eps,1]$, $C = C(q)$, and sufficiently small $\iota \in \bbR$,
    \begin{align*}
        O(\iota)
        &\ge 
        |f_1(q+\iota)^{-2} - f_1(q)^{-2}| \\
        &= \bigg|
            \fr{p'(q+\iota)(\xi^1\circ \Phi)(q+\iota) + p(q+\iota)\sum_{s\in \sS} (\partial_{x_s} \xi^1 \circ \Phi)(q+\iota) \Phi'_s(q+\iota)}{\Phi'_1(q+\iota)} \\
            &\qquad -\fr{p'(q)(\xi^1\circ \Phi)(q) + p(q)\sum_{s\in \sS} (\partial_{x_s} \xi^1 \circ \Phi)(q) \Phi'_s(q)}{\Phi'_1(q)}
        \bigg| \\
        &= |C'_1-C_1+O(\iota)|
    \end{align*}
    for
    \begin{align*}
        C_1 &= \fr{p'(q)(\xi^1\circ \Phi)(q) + p(q)\sum_{s\in \sS} (\partial_{x_s} \xi^1 \circ \Phi)(q) \Phi'_s(q)}{\Phi'_1(q)}, \\
        C'_1 &= \fr{p'(q+\iota)(\xi^1\circ \Phi)(q) + p(q)\sum_{s\in \sS} (\partial_{x_s} \xi^1 \circ \Phi)(q) \Phi'_s(q+\iota)}{\Phi'_1(q+\iota)}.
    \end{align*}
    Thus $|C_1-C'_1| \le O(\iota)$.
    Similarly, $|C_s-C'_s|\le O(\iota)$ for analogously defined $C_s,C'_s$.
    Note that the system given by
    \begin{align}
        \label{eq:linear-eq-admissibility}
        1 &= \sum_{s\in \sS} \lambda_s \Phi'_s(q) x_s \\
        \label{eq:linear-eq-gs}
        C_1 \Phi'_1(q) x_1 &= (\xi^1\circ \Phi)(q) y + p(q)\sum_{s\in \sS} (\partial_{x_s} \xi^1 \circ \Phi)(q) \Phi'_s(q) x_s
    \end{align}
    and analogous equations to \eqref{eq:linear-eq-gs} with $s\in \sS$ in place of $1$ has solution $y=p'(q)$, $x_1=\cdots=x_r=1$.
    Moreover, the system given by \eqref{eq:linear-eq-admissibility},
    \begin{equation}
        \label{eq:linear-eq-g's}
        C'_1 \Phi'_1(q) x_1 = (\xi^1\circ \Phi)(q) y + p(q)\sum_{s\in \sS} (\partial_{x_s} \xi^1 \circ \Phi)(q) \Phi'_s(q) x_s
    \end{equation}
    and analogous equations to \eqref{eq:linear-eq-g's} with $s\in \sS$ in place of $1$ has solution $y=p'(q+\iota)$, $x_s = \Phi'_s(q+\iota)/\Phi'_s(q)$.
    Since $|C_s-C'_s|\le O(\iota)$ for all $s$, we may apply Lemma~\ref{lem:positive-linalg-with-p} 
    with $\vc=\vC,{\vc\,}'=\vC'$, $y$ taking the place of $p'(q)$ or $p'(q+\iota)$, and $A$ corresponding to the last term of \eqref{eq:linear-eq-gs} or \eqref{eq:linear-eq-g's}. The result is that
    \[
        |p'(q+\iota)-p'(q)|,
        \lt|\fr{\Phi'_s(q+\iota)}{\Phi'_s(q)}-1\rt| \le O(\iota).
    \]
    (The required constants $A_{\min},a_{\min},b_{\min},a_{\max}$ are bounded thanks to Propositions~\ref{prop:basic-regularity} and \ref{prop:p-basic}.)
    
    Since $\Phi'_s$ is bounded below by Proposition~\ref{prop:basic-regularity}, we conclude that $p', \Phi'$ are Lipschitz in a neighborhood of $q\in (q_0,1]$. 
    This Lipschitz constant is uniform on any $[q_0+\eps,1]$, thus $p',\Phi'$ are Lipschitz on these sets. 
\end{proof}

\begin{lemma}
    \label{lem:pos-linalg-diagonal-must-grow}
    Suppose $A = (a_{i,j}) \in \bbR_{>0}^{r\times r}$ and $\vb \in \bbR_{>0}^r$. 
    Let $A_{\max},A_{\min}$ be the largest and smallest entries of $A$. 
    Suppose the linear system $A\vx = \vb \odot \vx$ admits the solution $\vx = \vone$. 
    If $\vb' \preceq \vb + \eps \vone$, $A'\ge A$ entry-wise, and the system $A'\vx = \vb' \odot \vx$ admits a nontrivial solution $\vx \in \bbR^r_{\ge 0}$, then all entries of $A'-A$ are at most $\eps \cdot \fr{r A_{\max} + A_{\min} + \eps}{A_{\min}}$.
\end{lemma}
\begin{proof}
    Assume without loss of generality that $x_1$ is the smallest entry of $\vx$. 
    Let $\Delta_i = b'_i - b_i$ and $\Delta_{i,j} = a'_{i,j} - a_{i,j}$, so $\Delta_i \le \eps$, $\Delta_{i,j} \ge 0$.
    We have 
    \[
        0 
        = (b_1+\Delta_1)x_1 - \sum_{i=1}^r a'_{i,j}x_i
        \le \Delta_1 x_1 - \sum_{i=1}^r a_{i,j}(x_i-x_1).
    \]
    Thus $a_{i,j}(x_i-x_1) \le \Delta_1 x_1$ for all $i$. 
    If $x_1=0$, this implies $\vx = \vzero$, contradiction.
    Thus $x_1>0$ and we may scale $\vx$ such that $x_1=1$. 
    This implies
    \[
        1 \le x_i \le 1 + \fr{\Delta_1}{a_{i,j}} \le 1 + \fr{\eps}{A_{\min}}
    \]
    for all $i$.
    The equation $b'_jx_j = (A'\vx)_j$ implies
    \begin{align*}
        \sum_{i=1}^r \Delta_{j,i}x_i 
        = 
        b_jx_j + \Delta_jx_j - \sum_{i=1}^r a_{j,i}x_i 
        &\le 
        \lt(1 + \fr{\eps}{A_{\min}}\rt)
        \sum_{i=1}^r a_{j,i} 
        + \eps \lt(1 + \fr{\eps}{A_{\min}}\rt) - \sum_{i=1}^r a_{j,i} \\
        &\le \eps \cdot \fr{r A_{\max} + A_{\min} + \eps}{A_{\min}}.
    \end{align*}
    Since $x_i\ge 1$ for all $i$, this implies the result.
\end{proof}

Let $S\subseteq (q_0,1)$ be the set of $q$ for which \eqref{eq:psi-equality} holds, and for $q\in S$ let $\Psi(q)$ be the common value of the $\Psi_s(q)$.
Let $S_1 = \{q\in S : p'(q) > 0\}$ and $S_2 = S\setminus S_1$.
\begin{proposition}
    \label{prop:psi-negativity}
    Almost everywhere in $S_2$, $\Psi(q)<0$.
\end{proposition}
\begin{proof}
    Suppose for the sake of contradiction that $\Psi(q)\ge 0$ holds for a positive-measure set $T\subseteq S_2$. 
    Let $U\subseteq [q_0,1]$ be the set of $q$ which are Lebesgue points of $f'_s(q)$ for all $s\in \sS$.
    Since these functions are measurable and integrable on $[q_0+\eps,1]$ for all $\eps > 0$, $U$ is almost all of $[q_0,1]$.
    So $T\cap U$ has positive measure.
    Let $q \in T \cap U$.
    Thus
    \[
        \lim_{\iota \to 0^+}
        \fr{f_1(q+\iota)-f_1(q)}{\iota}
        =
        f'_1(q)
        = \Phi'_1(q)\Psi(q),
    \]
    which implies that for small $\iota>0$,
    \[
        f_1(q+\iota) 
        = 
        f_1(q) + \Phi'_1(q)\Psi(q)\iota + o(\iota)
        \ge 
        f_1(q) - o(\iota).
    \]
    Define
    \begin{align*}
        C_1 &= \fr{p(q) (\xi^1 \circ \Phi)'(q)}{\Phi'_1(q)} = f_1(q)^{-2}, \\
        C'_1 &= \fr{p(q+\iota) (\xi^1 \circ \Phi)'(q+\iota)}{\Phi'_1(q+\iota)} \le f_1(q+\iota)^{-2}.
    \end{align*}
    Thus $C'_1 \le C_1 + o(\iota)$.
    For analogously defined $C_s,C'_s$ we have $C'_s \le C_s + o(\iota)$. 
    Note that the system given by
    \[
        C_1\Phi'_1(q) x_1 = \sum_{s\in \sS} p(q) (\partial_{x_s} \xi^1 \circ \Phi)(q) \Phi'_s(q) x_q
    \]
    and analogous equations with $s\in \sS$ in place of $1$ has solution $\vx = \vone$, while the system
    \[
        C'_1\Phi'_1(q) x_1 = \sum_{s\in \sS} p(q+\iota) (\partial_{x_s} \xi^1 \circ \Phi)(q+\iota) \Phi'_s(q) x_q
    \]
    and analogous equations with $s\in \sS$ in place of $1$ has solution $x_s = \Phi'_s(q+\iota)/\Phi'_s(q)$.
    By Lemma~\ref{lem:pos-linalg-diagonal-must-grow} this implies that for all $s,s'\in \sS$, 
    \[
        p(q+\iota) (\partial_{x_s} \xi^{s'} \circ \Phi)(q+\iota) \le p(q) (\partial_{x_s} \xi^{s'} \circ \Phi)(q) + o(\iota).
    \]
    However, since $\xi$ is non-degenerate, $(\partial_{x_s} \xi^{s'} \circ \Phi)(q+\iota) \ge (\partial_{x_s} \xi^{s'} \circ \Phi)(q) + \Omega(\iota)$ for some $s,s'$. 
    This is a contradiction.
\end{proof}

\begin{lemma}
    \label{lem:s1-s2-separate}
    There exists $q_1\in [q_0,1]$ such that, up to modification by a measure zero set, $S_1 = [q_0,q_1]$ and $S_2 = [q_1,1]$.
\end{lemma}
\begin{proof}
    We will show that there do not exist positive measure subsets $I\subseteq S_1$, $J\subseteq S_2$ with $\sup J \le \inf I$.
    Suppose for contradiction that such subsets exist. 
    Define $q^* = \sup J$, $m = \int_I p'(q)~\de q$, and 
    \[
        \psi(q) = 
        \begin{cases}
            m (\int_{[q_0,q]\cap J} \de q)/(\int_J \de q) & q \le q^*, \\
            m - \int_{[q^*,q]\cap I} p'(q)~\de q & q > q^*.
        \end{cases}
    \]
    Note that $\psi$ is absolutely continuous, nonnegative-valued, and positive-valued almost everywhere in $J$. 
    Moreover $\psi(q_0) = \psi(1)=0$, and for small $\delta > 0$, the perturbation
    \begin{equation}
        \label{eq:perturb-p}
        \wtp(q) = p(q) + \delta \psi(q)
    \end{equation}
    remains increasing.
    Note that
    \[
        \fr{\de}{\de \delta}
        (p\times \xi^s \circ \Phi)'(q) = (\psi \times \xi^s \circ \Phi)'(q).
    \]
    Thus, integrating by parts,
    \begin{align*}
        F
        \equiv 
        2 \fr{\de}{\de \delta}
        \bbA(\wtp,\Phi;q_0) \Big|_{\delta=0} 
        &= 
        \sum_{s\in \sS}
        \int_{q_0}^1
        \sqrt{\fr{\Phi'_s(q)}{(p\times \xi^s \circ \Phi)'(q)}} 
        (\psi \times \xi^s \circ \Phi)'(q) 
        ~\de q\\
        &= 
        -\sum_{s\in \sS}
        \int_{q_0}^1
        \psi(q)(\xi^s\circ\Phi)(q) \Phi_s'(q) \Psi_s(q)
        ~\de q \\
        &= 
        -\sum_{s\in \sS}
        \int_{S_2}
        \psi(q)(\xi^s\circ\Phi)(q) \Phi_s'(q) \Psi(q)
        ~\de q.
    \end{align*}
    By Proposition~\ref{prop:psi-negativity}, $\Psi(q) < 0$ almost everywhere in $S_2$.
    Therefore $F > 0$ and the perturbation \eqref{eq:perturb-p} improves the value of $\bbA(p,\Phi;q_0)$, a contradiction. 
    
    Finally, define measures 
    \[
        \mu([q_0,q]) = \int_{[q_0,q]\cap S_1} \de q,
        \qquad 
        \nu([q_0,q]) = \int_{[q_0,q]\cap S_2} \de q.
    \]
    The non-existence of $I,J$ implies that $\max \supp (\mu) \le \min \supp(\nu)$.
    Since $S_1\cup S_2$ is almost all of $[q_0,1]$ the result follows.
\end{proof}

\begin{proof}[Proof of Proposition~\ref{prop:type-12}]
    That $p,\Phi_s\in W^{2,\infty}([q_0+\eps,1])$ follows from Proposition~\ref{prop:twice-diff}.
    By Lemma~\ref{lem:s1-s2-separate}, $p'>0$ almost everywhere on $[q_0,q_1]$.
    By Proposition~\ref{prop:psi}, $\Psi_s=0$ almost everywhere on $[q_0,q_1]$. 
    Since $f_s$ is Lipschitz, for all $q\in [q_0,q_1]$ we have
    \[
        f_s(q)-f_s(q_0) = \int_{q_0}^q f'_s(q)~\de q = \int_{q_0}^q \Phi'_s(q) \Psi_s(q)~\de q = 0.
    \]
    Thus $f_s(q)^{-2} = \fr{(p\times \xi^s \circ \Phi)'(q)}{\Phi'_s(q)}$ is constant on $[q_0,q_1]$.
    By Lemma~\ref{lem:s1-s2-separate} we have $p'=0$ almost everywhere on $[q_1,1]$, hence everywhere by Proposition~\ref{prop:twice-diff}. And by Proposition~\ref{prop:p-basic} we have $p(1)=1$. 
    Thus, for all $q\in [q_1,1]$, 
    \[
        p(1)-p(q) = \int_q^1 p'(q)~\de q = 0,
    \]
    so $p(q)=1$ for all $q\in [q_1,1]$. 
    Finally, by Proposition~\ref{prop:psi} and Lemma~\ref{lem:s1-s2-separate}, \eqref{eq:tree-descending-ode} is satisfied for all $s,s'$ almost everywhere on $[q_1,1]$.
\end{proof}


Given Proposition~\ref{prop:type-12}, it remains to study the behavior of $(p,\Phi)$ separately on $[q_0,q_1]$ and $[q_1,1]$ and establish the root-finding and tree-descending descriptions in Propositions~\ref{prop:root-finding-trajectory} and \ref{prop:tree-descending-trajectory}. We have seen that $(p,\Phi)$ are described by explicit differential equations on $[q_0,q_1]$ and $[q_1,1]$, and it will be important to understand both. We will refer to them as the type $\I$ and $\II$ equations respectively in Subsections~\ref{subsec:type-I-well-posed} and \ref{subsec:type-II}.

\subsection{Behavior in the Root-Finding Phase $1$: Super-solvability of $\Phi(q_1)$}

Let $q_0,q_1$ be given by Proposition~\ref{prop:type-12}, and let $L_s$ be the constant value of $(p\times \xi^s \circ \Phi)'(q)/\Phi'_s(q)$ on $[q_0,q_1]$, which exists by Proposition~\ref{prop:type-12}. 
The goal of this subsection is to prove that $\Phi(q_1)$ is super-solvable.

\begin{lemma}
    \label{lem:phi-q0-positivity}
    We have $\Phi_s(q_0)=0$ if and only if $h_s=0$. 
\end{lemma}
\begin{proof}
    Assume without loss of generality that $s=1$.
    First, suppose $h_1=0$ and $\Phi_1(q_0)>0$. 
    By admissibility, $q_0>0$.
    Consider the perturbation $\wtq_0 = q_0-\delta$, 
    \[
        \wtp(q)=
        \begin{cases}
            q-\wtq_0 & q\in [\wtq_0,q_0] \\
            \delta + (1-\delta)p(q) & q\in [q_0,1] \\
        \end{cases}
        \quad 
        \tPhi_s(q)=
        \begin{cases}
            \fr{q-\wtq_0}{\delta}\Phi_s(q_0) & q\in [\wtq_0,q_0], s=1 \\
            \Phi_s(q_0) & q\in [\wtq_0,q_0], s\neq 1 \\
            \Phi_s(q) & q\in [q_0,1] 
        \end{cases}
    \]
    for all $s\in \sS$.
    Then,
    \[
        \lambda_s
        \int_{\wtq_0}^{q_0}
        \sqrt{\tPhi'_s(q)(\wtp\times \xi^s \circ \tPhi)'(q)}~\de q 
        \ge 
        \begin{cases}
            \Omega(\delta^{1/2}) & s=1 \\
            0 & s\neq 1
        \end{cases}
    \]
    while for all $s\in \sS$, 
    \begin{align*}
        \lambda_s
        \int_{q_0}^1
        \sqrt{\tPhi'_s(q)(\wtp\times \xi^s \circ \tPhi)'(q)}~\de q 
        &\ge 
        \lambda_s
        \int_{q_0}^1
        \sqrt{\tPhi'_s(q)(\wtp\times \xi^s \circ \tPhi)'(q)}~\de q 
        -O(\delta) \\
        h_s\lambda_s \sqrt{\tPhi_s(\wtq_0)} 
        &= 
        h_s\lambda_s \sqrt{\Phi_s(q_0)}.
    \end{align*}
    Thus for small $\delta>0$ the perturbation improves the value of $\bbA$, contradiction.
    
    Conversely, suppose $h_1>0$ and $\Phi_1(q_0)=0$. 
    Consider the perturbation $(\wtp,\tPhi,\wtq_0)$ where $\wtq_0=q_0+\delta$ and $\wtp,\tPhi$ are $p,\Phi$ restricted to $[q_0+\delta,1]$.
    Note that $\tPhi_1(q_0) \ge \Omega(\delta)$ by Proposition~\ref{prop:basic-regularity}. 
    Thus 
    \begin{align*}
        h_1\lambda_1 \sqrt{\tPhi_1(q_0)} - h_1\lambda_1 \sqrt{\Phi_1(q_0)} &\ge \Omega(\delta^{1/2}), \\
        h_s\lambda_s \sqrt{\tPhi_s(q_0)} - h_s\lambda_s \sqrt{\Phi_s(q_0)} &\ge 0 \quad \forall s\neq 1.
    \end{align*}
    Furthermore, for all $s\in \sS$, 
    \begin{align*}
        &\lambda_s \int_{\wtq_0}^1\sqrt{\tPhi'_s(q)(\wtp\times \xi^s \circ \tPhi)'(q)}~\de q 
        - \lambda_s \int_{q_0}^1\sqrt{\Phi'_s(q)(p\times \xi^s \circ \Phi)'(q)}~\de q \\
        &=
        \lambda_s \int_{q_0}^{q_0+\delta}\sqrt{\Phi'_s(q)(p\times \xi^s \circ \Phi)'(q)}~\de q 
        = O(\delta).
    \end{align*}
    Thus for small $\delta>0$ the perturbation improves the value of $\bbA$, contradiction.
\end{proof}
\begin{corollary}
    \label{cor:q0-q1-nonzero}
    If $\vh \neq \vzero$, then $0<q_0<q_1$ and $\Phi(q_1) \in (0,1]^\sS$.
\end{corollary}
\begin{proof}
    Lemma~\ref{lem:phi-q0-positivity} implies $0<q_0$, so Proposition~\ref{prop:p-basic} implies $p(q_0)=0$. 
    Since $p(q_1)=1$ by Proposition~\ref{prop:type-12}, we have $q_0<q_1$.
    Proposition~\ref{prop:basic-regularity} gives $\Phi'(q) \succeq L^{-1}\vone$ for $q\in [q_0,q_1]$, so all coordinates of $\Phi(q_1)$ are positive.
\end{proof}

\begin{lemma}
    \label{lem:q0-q1-zero}
    If $\vh = \vzero$, then $q_0=q_1=0$ (and $\Phi(q_1)=\vzero$).
\end{lemma}
\begin{proof}
    By Lemma~\ref{lem:phi-q0-positivity}, $\Phi(q_0)=\vzero$ so $q_0=0$.
    Suppose that $q_1 > 0$.
    Then, for all $q\in [0,q_1]$, we have $L_s \Phi_s'(q) = (p\times \xi^s \circ \Phi)'(q)$, and by integrating $L_s \Phi_s(q) = p(q)(\xi^s \circ \Phi)(q)$.
    By Assumption~\ref{as:nondegenerate}, we can write $\xi^s(\vx) = \sum_{s'\in \sS} P_{s,s'}(\vx)x_{s'}$ where each $P_{s,s'}$ is a polynomial with nonnegative coefficients and positive constant and linear terms.
    Thus the functions $P_{s,s'} \circ \Phi$ are all strictly increasing.
    Let $0<q<q'<q_1$.
    The linear system
    \[
        L_s\Phi_s(q) x_s
        = 
        \sum_{s'\in \sS}
        p(q)(P_{s,s'} \circ \Phi)(q) \Phi_{s'}(q) x_s
        \quad 
        \forall s\in \sS
    \]
    has solution $\vx = \vone$, while the linear system
    \[
        L_s\Phi_s(q) x_s
        = 
        \sum_{s'\in \sS}
        p(q')(P_{s,s'} \circ \Phi)(q') \Phi_{s'}(q) x_s
        \quad 
        \forall s\in \sS
    \]
    has solution $x_s = \Phi_s(q')/\Phi_s(q)$.
    Monotonicity of $P_{s,s'} \circ \Phi$ implies $p(q')(P_{s,s'} \circ \Phi)(q') \ge p(q)(P_{s,s'} \circ \Phi)(q)$, so Lemma~\ref{lem:pos-linalg-diagonal-must-grow} (with $\eps=0$) implies that $p(q')(P_{s,s'} \circ \Phi)(q') = p(q)(P_{s,s'} \circ \Phi)(q)$ for all $s,s'$.
    This contradicts that the $P_{s,s'} \circ \Phi$ are strictly increasing.
\end{proof}

\begin{lemma}
    \label{lem:gs-value-with-field}
    If $h_s>0$, then $L_s = \fr{h_s^2}{\Phi_s(q_0)}$.
\end{lemma}
\begin{proof}
    Assume without loss of generality that $s=1$. 
    Consider the following perturbation $\tPhi$ of $\Phi$. 
    For all $s\neq 1$, $\tPhi_s=\Phi_s$, and $\tPhi_1(q)=\Phi_1(q)+\delta\psi(q)$  where $\psi \in C^\infty([q_0,1])$ with $\psi(q_0)=1$ and $\psi=0$ on $[q_1,1]$.
    This perturbation is not admissible, but we nonetheless have $\bbA(p,\tPhi;q_0) \le \bbA(p,\Phi;q_0)$ by Lemma~\ref{lem:admissible-optional}.
    
    Recall the calculation \eqref{eq:phi1-deriv}. 
    Integrating by parts, 
    \begin{align*}
        F_1
        &\equiv 2\lambda_1^{-1} \fr{\de}{\de \delta} \bbA(p,\tPhi;q_0)
        \Big|_{\delta=0} \\
        &= 
        \fr{h_1}{\sqrt{\Phi_1(q_0)}}
        +
        \int_{q_0}^1 
        L_1^{1/2} \psi'(q)~\de q
        +
        \sum_{s\in \sS}
        \int_{q_0}^1
        L_s^{1/2}
        (p\psi \times \partial_{x_s}\xi^1 \circ \Phi)'(q)
        = 
        \fr{h_1}{\sqrt{\Phi_1(q_0)}}
        - L_1^{1/2}.
    \end{align*}
    Recall that $\Phi'_1(q)$ is uniformly lower bounded by Proposition~\ref{prop:basic-regularity} and $\Phi_1(q_0)>0$ by Lemma~\ref{lem:phi-q0-positivity}. 
    So, this perturbation is valid for small positive and negative $\delta$. 
    Thus $F_1=0$ which implies the result. 
\end{proof}


\begin{proposition}
    \label{prop:gs-value}
    If $\vh\neq\vzero$, then for all $s$,
    \begin{equation}
        \label{eq:Ls-formula-final}
        L_s = \fr{(\xi^s \circ \Phi)(q_1) + h_s^2}{\Phi_s(q_1)},
    \end{equation}
    which is well-defined by Corollary~\ref{cor:q0-q1-nonzero}.
    Thus, $(p,\Phi)$ satisfies \eqref{eq:root-finding-ode} for all $s\in \sS$, $q\in [q_0,q_1]$ with $\vx = \Phi(q_1)$.
\end{proposition}

\begin{proof}
    Note that $\Phi_s(q_1)>0$ for all $s$ by Corollary~\ref{cor:q0-q1-nonzero} and Proposition~\ref{prop:basic-regularity}.
    Integrating the equation $(p\times \xi^s \circ \Phi)'(q) = L_s\Phi'_s(q)$ on $[q_0+\eps,q]$ and using continuity of $p$ and $\Phi$ and that $p(q_0)=0$, we find
    \begin{equation}
        \label{eq:type1-p-formula}
        p(q)(\xi^s \circ \Phi)(q) = L_s(\Phi_s(q)-\Phi_s(q_0)).
    \end{equation}
    Since $p(q_1)=1$ by Proposition~\ref{prop:type-12}, we have
    \begin{equation}
        \label{eq:type1-ode-aux}
        (\xi^s \circ \Phi)(q_1) = L_s(\Phi_s(q_1)-\Phi_s(q_0)).
    \end{equation}
    If $h_s=0$, by Lemma~\ref{lem:phi-q0-positivity} $\Phi_s(q_0)=0$, so $L_s = (\xi^s \circ \Phi)(q_1)/\Phi_s(q_1)$ as desired. 
    Otherwise, by Lemma~\ref{lem:gs-value-with-field}, $L_s = h_s^2 / (\lambda_s \Phi_s(q_0))$.
    Plugging this into \eqref{eq:type1-ode-aux} implies
    \begin{equation}
        \label{eq:type1-q0-formula}
        \Phi_s(q_0)\lt((\xi^s \circ \Phi)(q_1) + h_s^2\rt) = h_s^2 \Phi_s(q_1).
    \end{equation}
    Thus
    \[
        L_s = \fr{(\xi^s \circ \Phi)(q_1)}{\Phi_s(q_1)-\Phi_s(q_0)}
        = \fr{(\xi^s \circ \Phi)(q_1) + h_s^2}{\Phi_s(q_1)}
    \]
    as desired.
\end{proof}


\begin{corollary}
    \label{cor:alg-value}
    For $(p,\Phi;q_0)$ maximizing $\bbA$, we have
     \begin{equation}
        \label{eq:alg-functional-type1-simplification}
        \bbA(p,\Phi;q_0)
        =
        \sum_{s\in \sS}
        \lambda_s \lt[
            \sqrt{\Phi_s(q_1) (\xi^s(\Phi(q_1)) + h_s^2)}
            +
            \int_{q_1}^1
            \sqrt{\Phi'_s(q)(\xi^s \circ \Phi)'(q)} ~\de q
        \rt].
    \end{equation}
\end{corollary}
\begin{proof}
    If $\vh = \vzero$, then $q_1=0$ by Lemma~\ref{lem:q0-q1-zero}. 
    Thus, $p=1$ on $[0,1]$ by Proposition~\ref{prop:type-12}.
    Thus $(p\times \xi^s \circ \Phi)' = (\xi^s \circ \Phi)'$ and the result is clear. 
    Otherwise $\vh \neq \vzero$, and Corollary~\ref{cor:q0-q1-nonzero} implies $q_1>q_0$. 
    
    If $h_s=0$, then by Lemma~\ref{lem:phi-q0-positivity}, $\Phi_s(q_0)=0$. 
    So,
    \begin{align*}
        h_s\lambda_s \sqrt{\Phi_s(q_0)}
        +
        \lambda_s
        \int_{q_0}^{q_1}
        \sqrt{\Phi'_s(q) (p\times \xi^s \circ \Phi)'(q)}
        ~\de q 
        &= 
        \lambda_s
        \int_{q_0}^{q_1}
        \Phi'_s(q) \sqrt{L_s}
        ~\de q \\
        &= \lambda_s \Phi_s(q_1) \sqrt{L_s} = 
        \lambda_s \sqrt{\Phi_s(q_1) (\xi^s \circ \Phi)(q_1)},
    \end{align*}
    as desired. The last step uses Proposition~\ref{prop:gs-value}. 
    If $h_s>0$, then by Lemma~\ref{lem:gs-value-with-field} and Proposition~\ref{prop:gs-value},
    \begin{align*}
        h_s\lambda_s \sqrt{\Phi_s(q_0)}
        +
        \lambda_s
        \int_{q_0}^{q_1}
        \sqrt{\Phi'_s(q) (p\times \xi^s \circ \Phi)'(q)}
        ~\de q 
        &= 
        \lambda_s \lt[
            \Phi_s(q_0)\sqrt{L_s} + 
            \int_{q_0}^{q_1}\Phi'_s(q) \sqrt{L_s}~\de q 
        \rt] \\
        &= \lambda_s \Phi_s(q_1) \sqrt{L_s} \\
        &= \lambda_s \sqrt{\Phi_s(q_1) \lt((\xi^s \circ \Phi)(q_1) + h_s^2\rt)}.
    \end{align*}
\end{proof}
The following variant of this calculation determines the energy attained by $(p,\Phi;q_0)$ partway through the root-finding phase, and is used in Remark~\ref{rem:type1-partway}.
\begin{corollary}
    \label{cor:type1-partway}
    If $(p,\Phi;q_0)$ maximizes $\bbA$ and $q \in [q_0,q_1]$, then
    \[
        \sum_{s\in \sS} \lambda_s \lt[
            h_s \sqrt{\Phi_s(q_0)} + 
            \int_{q_0}^q \sqrt{\Phi'_s(t) (p\times \xi^s \circ \Phi)'(t)} ~\de t
        \rt]
        = \sum_{s\in \sS} \lambda_s \sqrt{\Phi_s(q) (p(q)(\xi^s \circ \Phi)(q) + h_s^2)}.
    \]
\end{corollary}
\begin{proof}
    If $h_s=0$, then by Lemma~\ref{lem:phi-q0-positivity}, $\Phi_s(q_0)=0$. 
    Then \eqref{eq:type1-p-formula} implies $L_s = p(q)(\xi^s \circ \Phi)(q) / \Phi_s(q)$. 
    So
    \[
        h_s \sqrt{\Phi_s(q_0)} + 
        \int_{q_0}^q \sqrt{\Phi'_s(t) (p\times \xi^s \circ \Phi)'(t)} ~\de t
        = (\Phi_s(q) - \Phi_s(q_0)) \sqrt{L_s} 
        = \sqrt{\Phi_s(q) p(q)(\xi^s \circ \Phi)(q)}.
    \]
    If $h_s>0$, \eqref{eq:type1-p-formula} implies and Lemma~\ref{lem:gs-value-with-field} imply
    \[
        p(q) (\xi^s \circ \Phi)(q) = \fr{h_s^2}{\Phi_s(q_0)} (\Phi_s(q) - \Phi_s(q_0)),
    \]
    which rearranges to 
    \[
        \fr{h_s^2 \Phi_s(q)}{\Phi_s(q_0)} = p(q)(\xi^s \circ \Phi)(q) + h_s^2.
    \]
    Then
    \begin{align*}
        h_s \sqrt{\Phi_s(q_0)} + 
        \int_{q_0}^q \sqrt{\Phi'_s(t) (p\times \xi^s \circ \Phi)'(t)} ~\de t
        &= h_s \sqrt{\Phi_s(q_0)} + (\Phi_s(q)-\Phi_s(q_0)) \sqrt{\fr{h_s^2}{\Phi_s(q_0)}} \\
        &= \fr{h_s \Phi_s(q)}{\sqrt{\Phi_s(q_0)}}
        = \sqrt{\Phi_s(q) (p(q)(\xi^s \circ \Phi)(q) + h_s^2)}.
    \end{align*}
    Summing over $s\in \sS$ completes the proof.
\end{proof}

\begin{lemma}
    \label{lem:q1-(super)-solvable}
    If $q_1=1$, then $\Phi(q_1)=\vone$ is super-solvable.
    If $q_1<1$, then $\Phi(q_1)$ is solvable.
\end{lemma}
\begin{proof}
    First suppose $q_1=1$. 
    Admissibility and the fact that $\Phi(1) \in [0,1]^\sS$ implies $\Phi(q_1)=\vone$.
    We have $p(q_1)=1$ by Proposition~\ref{prop:p-basic} and also $p'(q_1)\ge 0$.
    By Proposition~\ref{prop:gs-value},
    \begin{equation}
        \label{eq:prove-supersolvable}
        \fr{(\xi^s \circ \Phi)(q_1) + h_s^2}{\Phi_s(q_1)}
        = \fr{(p\times \xi^s \circ \Phi)'(q_1)}{\Phi'_s(q_1)}
        \ge \fr{\sum_{s'\in \sS} (\partial_{x_{s'}}\xi^s \circ \Phi)(q_1)\Phi'_{s'}(q_1)}{\Phi'_s(q_1)}.
    \end{equation}
    This implies via Corollary~\ref{cor:solvability-equivalent} (with $\Phi'$ in the role of $\vv$) that $\Phi(q_1)$ is super-solvable. 

    Now suppose $q_1<1$.
    If $\vh=\vzero$ the result follows from Lemma~\ref{lem:q0-q1-zero}, so assume $\vh\neq\vzero$. 
    Because $p(q)=1$ on $[q_1,1]$ and $p'$ is continuous (Proposition~\ref{prop:basic-regularity}), $p'(q_1)=0$. 
    So, the inequality in \eqref{eq:prove-supersolvable} is an equality.
    Thus $\Phi'(q_1)$ is in the null space of $M^*(\Phi(q_1))$, and thus (by \eqref{eq:M*sym-to-M*}) of $M^*_\sym(\Phi(q_1))$.
    So $M^*_\sym(\Phi(q_1))$ is singular and $\Phi(q_1)$ is solvable.
\end{proof}




\subsection{Behavior in the Root-Finding Phase $2$: Well-Posedness}
\label{subsec:type-I-well-posed}

In this subsection we prove Proposition~\ref{prop:root-finding-trajectory} and give a detailed characterization of $(p,\Phi)$ on $[q_0,q_1]$ in Proposition~\ref{prop:type-1}. Recalling Propositions~\ref{prop:gs-value} and \ref{lem:q1-(super)-solvable}, we consider a path $(p,\Phi)$ defined by the \textbf{type $\I$ equation}
\begin{equation}
\label{eq:type-1-traj}
\begin{aligned}
    \fr{(p\times \xi^s \circ \Phi)'(q)}{\Phi'_s(q)} 
    &= L_s= \fr{(\xi^s \circ \Phi)(q_1) + h_s^2}{\Phi_s(q_1)}
    ,\quad\forall s\in\sS
    \\  
    \Phi_s'(q)&\geq 0, \quad 
    \la \vlam, \Phi'(q)\ra = 1
\end{aligned}
\end{equation}
with super-solvable initial condition $\Phi(q_1)$ and $p(q_1)=1$.  We start by verifying the first part of Proposition~\ref{prop:root-finding-trajectory}, namely that $\vh \neq \vzero$ if and only if there exists a super-solvable point $\vx \in [0,1]^\sS$ with $\la \vlam, \vx\ra >0$.


\begin{proof}[Proof of Proposition~\ref{prop:root-finding-trajectory} (first claim)]
    First, assume $\vh \neq \vzero$.
    We will show that all $\vx \in [\delta/2,\delta]^\sS$ are super-solvable for $\delta > 0$ sufficiently small. 
    Assume without loss of generality that $h_1>0$.
    Note that for all $s\in \sS$, 
    \[
        (M^*(\vx) \vx)_s 
        = x_s \lt( h_s^2 + \xi^s(\vx) - \sum_{s'\in \sS} x_{s'} \partial_{x_{s'}} \xi^s(\vx)\rt)
        = x_s \lt( h_s^2 - O(\delta^2)\rt).
    \]
    Moreover, $(M^*(\vx) \ve_1)_1 \le h_1^2 + O(\delta)$, while for $s\neq 1$,
    \[
        (M^*(\vx) \ve_1)_s
        = 
        - x_s \partial_{x_1}\xi^s(\vx).
    \]
    Thus, for $\vv = \vx - \fr12 x_1 \ve_1$, we have 
    \[
        (M^*(\vx) \vv)_1
        \ge x_1 \lt( \fr12 h_1^2 - O(\delta)\rt)
        \ge 0
    \]
    and for $s\neq 1$, 
    \[
        (M^*(\vx) \vv)_s
        \ge 
        x_s\lt( x_1 \partial_{x_1}\xi^s(\vx) - O(\delta^2)\rt)
        \ge 0.
    \]
    This implies by Corollary~\ref{cor:solvability-equivalent} that $\vx$ is super-solvable.

    If $\vh=\vzero$, fix any $\vx \in (0,1]^\sS$. 
    Note that 
    \[
        \vx^\top M^*_\sym(\vx) \vx 
        = \sum_{s\in \sS} x_s \partial_{x_s}\xi(\vx) 
        - \sum_{s,s'\in \sS} x_s x_{s'} \partial_{x_s,x_{s'}}\xi(\vx) < 0,
    \]
    as any monomial of $\xi(\vx)$ with total degree $p\ge 2$ appears with multiplicity $p$ in the first sum and $p(p-1) \ge p$ in the second, with strict inequality for any $p>2$.
    Thus $\vx$ is strictly sub-solvable. 
\end{proof}


\begin{proposition}
\label{proposition:Lambda-def-of-type-I}
    Define for $(p(q),p'(q),\Phi(q))\in [0,1]^{\sS}\times [0,1]\times \mathbb R$ the $\sS\times\sS$ matrix $M(p(q),p'(q),\Phi(q))$ with entries
\[
    M(p(q),p'(q),\Phi(q))_{s,s'}=\frac{p(q)\partial_{x_{s'}}\xi^s\lt(\Phi(q)\rt)
    +
    \lambda_{s'} p'(q)\xi^s(\Phi(q))}{L_s}
    ,\quad\quad
    s,s'\in\sS.
\]
If $(p,\Phi)$ solves \eqref{eq:type-1-traj} then $\Lambda(M(p,p',\Phi))=1$ with Perron-Frobenius eigenvector $\Phi'(q)$.
\end{proposition}

\begin{proof}
    It suffices to expand the left-hand side of the top line of \eqref{eq:type-1-traj}:
\begin{equation}
\label{eq:type-1-traj-rewrite} 
    p'(q)\xi^s(\Phi(q))\left(\sum_{s'\in\sS}\lambda_{s'} \Phi_{s'}'(q)\right) + p(q)\sum_{s'\in\sS} \Phi_{s'}'(q) \partial_{x_{s'}} \xi^s \lt(\Phi(q)\rt)=L_s\Phi_s'(q),\quad \forall s\in\sS.
\end{equation}
    Rearranging shows that $M(p,p',\Phi)\Phi'(q)=\Phi'(q)$, and it is clear that $M$ has non-negative entries.
\end{proof}




We now show the ODE \eqref{eq:type-1-traj} is well-posed.


\begin{lemma}
\label{lem:ODE-Lipschitz}
Fix $q\in [0,1]$ and let $Y(q)=(p(q),\Phi(q))$ and $L_s>0$ be arbitrary. The equation \eqref{eq:type-1-traj} for any fixed $q$ is equivalent to
\[
    Y'(q)=F(Y(q))
\]
for a locally Lipschitz function $F:[0,1]\times \lt([0,1]^r\backslash \vzero\rt)\to \bbR^{r+1}$.
\end{lemma}


\begin{proof}
Let $M$ be as in Proposition~\ref{proposition:Lambda-def-of-type-I}. Because $\xi$ is non-degenerate,
Propositions~\ref{prop:VM} and \ref{proposition:Lambda-def-of-type-I} imply existence of $c>0$ such that 
\[
M(p,x+y,\Phi)\geq M(p,x,\Phi)+cy
\]
holds entrywise for all $x,y\geq 0$, as long as $\Phi(q)\in \bbR_{\geq 0}^{r}\backslash [0,\eps]^{r}$. Therefore a unique value $p'(q)$ solving \eqref{eq:type-1-traj} exists. Moreover $M$ is locally Lipschitz in $(p,\Phi)$, so if
\[
    M(p,p',\Phi)=M(\wt\Phi,\wtp,\wtp')
\]
then
\[
    |p'-\wtp'|\leq O(\|\Phi-\wt\Phi\|_{L^{\infty}}+|p-\wtp|).
\]
(With implicit constant depending on $\eps$ as introduced above.)
This shows that $p'$ has locally Lipschitz dependence on $Y=(p,\Phi)$.
It remains to show $\Phi'$, defined by the resulting solution to \eqref{eq:type-1-traj-rewrite}, also has locally Lipschitz dependence on $Y$. This follows by Proposition~\ref{prop:perron-eigenvector-lipschitz} below. (Note that all entries of $M$ are of the same order up to constants for $\Phi(q)\in \bbR_{\geq 0}^{r}\backslash [0,\eps]^{r}$ by non-degeneracy of $\xi$.)
\end{proof}







\begin{proposition}[{\cite[Lemma 27]{yeo2018frozen}}]
\label{prop:perron-eigenvector-lipschitz}
Let $\cM\subseteq \bbR_{\geq 0}^{r\times r}$ be a compact set of square matrices all of whose Perron-Frobenius eigenvalues have multiplicity $1$. Let $M,\wt M\in\cM$ have entrywise positive Perron-Frobenius eigenvectors $v,\wt v$, normalized so that $\|v\|_1=\|\wt v\|=1$. Then 
\[
    \|v-\wt v\|_{1}\leq O_{\cM}(\|M-\wt M\|_1).
\]
In particular, this holds for $\cM=[c,C]^{r\times r}$ for any $0<c<C<\infty$.
\end{proposition}



Lemma~\ref{lem:ODE-Lipschitz} shows that for any right endpoint $(p(q_1),\Phi(q_1))$, it is possible to solve \eqref{eq:type-1-traj} backwards in time until $q_*$ when $\Phi(q)$ reaches the boundary of $\bbR_{\geq 0}^r$, or at which $p(q)$ reaches $0$. We now show that the latter occurs first.





\begin{lemma}
\label{lem:Phi-linear-LB}
There exists $c>0$ such that for any super-solvable point $\Phi(q_1)$, the solution to the type $\I$ equation \eqref{eq:type-1-traj} on $[q_*,q_1]$ satisfies
\[
    \Phi_{s}(q)\geq cp(q)q.
\]
Moreover $p(q_*)=0$ and Lemma~\ref{lem:phi-q0-positivity} holds for $q_*$, i.e. $h_s>0$ if and only if $\Phi_s(q_*)>0$.
\end{lemma}


\begin{proof}
Observe that in \eqref{eq:type-1-traj}, we have
\[
    L_s\geq K_s\equiv \frac{\xi^s(\Phi(q_1))}{\Phi_s(q_1)}.
\]
Therefore on $q\in[q_{\eps},q_1]$, the left-hand equation in \eqref{eq:type-1-traj} implies
\[
    \frac{(p\times \xi^s\circ \Phi)(q)}{\Phi_s(q)}
    \leq 
    K_s.
\]
Recall that $\xi^s$ is non-degenerate, and so admissibility and $\Phi\succeq 0$ implies $\xi^s(\Phi(q))=\Theta(q).$ Hence for some $c>0$ and all $s\in\sS$,
\[
    \Phi_{s}(q)\geq \Omega(p(q)q/K_s)\geq cp(q)q.
\]
This concludes the proof of the first statement, which implies that $p(q_*)=0$. 

For the second, note that strict inequality holds in the first step if $h_s>0$, and so $p$ must reach $0$ before $\Phi_s$ does. On the other hand if $h_s=0$, then it is easy to see from \eqref{eq:type-1-traj} that $p$ cannot reach zero strictly sooner than $\Phi_s$, hence the numerator and denominator on the left-hand side in \eqref{eq:type-1-traj} both reach zero at time $q_*$.
\end{proof}

\begin{lemma}
\label{lem:p-concave}
If $\Phi(q_1)$ is super-solvable, then the $p$ solving \eqref{eq:type-1-traj-rewrite} is increasing and concave on $[q_*,q_1]$.  Moreover $p,\Phi_s\in C^1([q_*,q_1])$.
\end{lemma}

\begin{proof}
We claim that $p'$ is decreasing. The key point is that with $M$ as in Proposition~\ref{proposition:Lambda-def-of-type-I}, 
\[
    M(p,p',\Phi)< M(\wt\Phi,\wtp,\wtp')
\]
if $\Phi \preceq \wt\Phi$, $p \leq \wtp$ and $p'<\wtp'$. 
Indeed this is immediate by Proposition~\ref{prop:VM}.
It follows that $p'$ must increase backward in time, i.e. $p'(q)$ is a decreasing function. Since $p'(q_1)\geq 0$ by super-solvability, this completes the proof.
\end{proof}


\begin{proof}[Proof of Proposition~\ref{prop:root-finding-trajectory}, parts (\ref{it:unique-root-finding},\ref{it:unique-root-finding-q0})]
    Existence and uniqueness of the root-finding trajectory follows from Lemma~\ref{lem:ODE-Lipschitz} and Proposition~\ref{prop:ODE-well-posed}. Lemma~\ref{lem:Phi-linear-LB} ensures that the solution exists until $p$ reaches $0$. Concavity of $p$ was just shown in Lemma~\ref{lem:p-concave}. 
    This proves part (\ref{it:unique-root-finding}). 
    Part (\ref{it:unique-root-finding-q0}) follows from Lemma~\ref{lem:phi-q0-positivity} or \ref{lem:Phi-linear-LB}.
\end{proof}


\begin{lemma}
    \label{lem:vone-super-solvable}
    If $\vone$ is super-solvable, then $q_1=1$ and $\Phi(q_1)=\vone$.
    Otherwise $q_1 < 1$.
\end{lemma}
\begin{proof}
    If $\vone$ is strictly sub-solvable, Lemma~\ref{lem:q1-(super)-solvable} implies that $q_1<1$.
    Suppose $\vone$ is super-solvable.
    Let $(p^*,\Phi^*,q_0^*)$ be the root-finding trajectory with endpoint $\vone$, which exists by Proposition~\ref{prop:root-finding-trajectory}.
    By Corollary~\ref{cor:alg-value},
    \[
        \bbA(p^*,\Phi^*;q_0^*) 
        = 
        \sum_{s\in \sS} 
        \lambda_s 
        \sqrt{\xi^s(\vone) + h_s^2}.
    \]
    Suppose for contradiction that there is a different maximizer $(p,\Phi,q_0)$ of $\bbA$ with $\bbA(p,\Phi;q_0) \ge \bbA(p^*,\Phi^*;q_0^*)$. The maximizer $(p,\Phi,q_0)$ has its own value $q_1$, and we must have $q_1<1$ since for this to be a different maximizer.
    Note that for each $s\in \sS$, 
    \begin{align*}
        &\sqrt{\xi^s(\vone) + h_s^2} - \sqrt{\Phi_s(q_1)(\xi^s(\Phi(q_1)) + h_s^2)} 
        = \int_{q_1}^1 
        \fr{\de}{\de q}
        \sqrt{\Phi_s(q)((\xi^s\circ\Phi)(q) + h_s^2)}
        ~\de q \\
        &= \fr12 \int_{q_1}^1 \lt(
            \Phi'_s(q)
            \sqrt{\fr{(\xi^s\circ\Phi)(q) + h_s^2}{\Phi_s(q)}}
            + (\xi^s \circ \Phi)'(q) 
            \sqrt{\fr{\Phi_s(q)}{(\xi^s\circ\Phi)(q) + h_s^2}}
        \rt) ~\de q.
    \end{align*}
    By Corollary~\ref{cor:alg-value},
    \begin{align*}
        F &\equiv 
        \bbA(p^*,\Phi^*;q_0^*)
        -
        \bbA(p,\Phi;q_0) \\
        &=
        \sum_{s\in \sS}
        \fr{\lambda_s}{2}
        \int_{q_1}^1
        (\xi^s \circ \Phi)'(q) 
        \sqrt{\fr{\Phi_s(q)}{(\xi^s\circ\Phi)(q) + h_s^2}}
        \lt(
            \sqrt{
                \fr{\Phi'_s(q)}{(\xi^s\circ\Phi)'(q)}
                \cdot 
                \fr{(\xi^s\circ\Phi)(q) + h_s^2}{\Phi_s(q)}
            }
            - 1
        \rt)^2
        \de q
        \ge 0.
    \end{align*}
    Since $\bbA(p,\Phi;q_0) \ge \bbA(p^*,\Phi^*;q_0^*)$, we have $F=0$.
    So, for all $s\in \sS$, and almost all $q\in (q_1,1]$
    \[
        \fr{(\xi^s\circ\Phi)'(q)}{(\xi^s\circ\Phi)(q) + h_s^2}
        =
        \fr{\Phi'_s(q)}{\Phi_s(q)}
        \qquad 
        \Rightarrow 
        \qquad 
        \fr{\de}{\de q} \log \lt((\xi^s\circ\Phi)(q) + h_s^2\rt) 
        = 
        \fr{\de}{\de q} \log \Phi_s(q).
    \]
    Both sides of this equation are continuous on $(q_1,1]$, so in fact it holds for all $q\in (q_1,1]$.
    Thus there exist constants $C_s$ such that
    \[
        (\xi^s\circ\Phi)(q) + h_s^2
        = 
        C_s \Phi_s(q).
    \]
    Thus, for $q_1 < q < q+\iota \le 1$, we have
    \begin{align*}
        C_s\Phi'_s(q)
        &= 
        (\xi^s \circ \Phi)'(q) 
        =
        \sum_{s'\in \sS}
        (\partial_{x_{s'}} \xi^s \circ \Phi)(q)
        \Phi'_{s'}(q)
        \quad \forall s\in \sS, \\ 
        C_s\Phi'_s(q+\iota)
        &=
        (\xi^s \circ \Phi)'(q+\iota) 
        =
        \sum_{s'\in \sS}
        (\partial_{x_{s'}} \xi^s \circ \Phi)(q+\iota)
        \Phi'_{s'}(q+\iota)
        \quad \forall s\in \sS,
    \end{align*}
    We treat these equations as linear systems in $\Phi'(q)$ and $\Phi'(q+\iota)$.
    Since both linear systems have nonnegative solutions and $(\partial_{x_{s'}} \xi^s \circ \Phi)(q+\iota) \ge (\partial_{x_{s'}} \xi^s \circ \Phi)(q)$ for all $s,s'$, Lemma~\ref{lem:pos-linalg-diagonal-must-grow} (with $\eps=0$) implies that $(\partial_{x_{s'}} \xi^s \circ \Phi)(q+\iota) = (\partial_{x_{s'}} \xi^s \circ \Phi)(q)$ for all $s,s'$. 
    This contradicts that $\xi$ is non-degenerate and completes the proof.
\end{proof}


\begin{proposition}
    \label{prop:type-1}
    The following assertions hold. 
    \begin{enumerate}[label=(\alph*), ref=\alph*]
        \item \label{itm:s4-supsolvable} If $\vone$ is super-solvable, then $0 < q_0 < q_1 = 1$ (and thus $\Phi(q_1)=\vone$).
        \item \label{itm:s4-subsolvable-with-field} If $\vone$ is sub-solvable and $\vh \neq \vzero$, then $0 < q_0 < q_1 < 1$ and $\Phi(q_1) \in (0,1]^\sS$.
        \item \label{itm:s4-subsolvable-no-field} If $\vh = \vzero$, then $\vone$ is sub-solvable and $0=q_0=q_1$ (and thus $\Phi(q_1) = \vzero$).
    \end{enumerate}
    In cases (\ref{itm:s4-subsolvable-with-field}, \ref{itm:s4-subsolvable-no-field}), $\Phi(q_1)$ is solvable. 
    In cases (\ref{itm:s4-supsolvable}, \ref{itm:s4-subsolvable-with-field}) (and vacuously in case (\ref{itm:s4-subsolvable-no-field})) $(p,\Phi)$ restricted to $[q_0,q_1]$ is the root-finding trajectory with endpoint $\Phi(q_1)$.
\end{proposition}

\begin{proof}[Proof of Proposition~\ref{prop:type-1}]
    If $\vone$ is super-solvable, Lemma~\ref{lem:vone-super-solvable} implies $q_1=1$. 
    Comparing Corollary~\ref{cor:q0-q1-nonzero} and Lemma~\ref{lem:q0-q1-zero} gives $q_0>0$.
    If $\vone$ is sub-solvable and $\vh\neq\vzero$, Lemma~\ref{lem:vone-super-solvable} implies $q_1<1$ while Corollary~\ref{cor:q0-q1-nonzero} implies $0<q_0<q_1$ and $\Phi(q_1) \in (0,1]^\sS$. 
    If $\vh=\vzero$, Lemma~\ref{lem:q0-q1-zero} implies $0=q_0=q_1$.
    This proves assertions (\ref{itm:s4-supsolvable}, \ref{itm:s4-subsolvable-with-field}, \ref{itm:s4-subsolvable-no-field}). 

    In cases (\ref{itm:s4-subsolvable-with-field}, \ref{itm:s4-subsolvable-no-field}), since $q_1<1$, Lemma~\ref{lem:q1-(super)-solvable} implies $\Phi(q_1)$ is solvable.
    In cases (\ref{itm:s4-supsolvable}, \ref{itm:s4-subsolvable-with-field}), Proposition~\ref{prop:gs-value} implies $(p,\Phi)$ restricted to $[q_0,q_1]$ is the root-finding trajectory with endpoint $\Phi(q_1)$. 
\end{proof}

\subsection{Behavior in the Tree-Descending Phase}
\label{subsec:type-II}


The next lemma, proved in Appendix~\ref{subsec:type-II-Lipschitz}, shows the tree-descending ODE is also well-posed. 

\begin{restatable}{lemma}{lemtypeIILipschitz}
\label{lem:type-II-Lipschitz}
Fix $\eps>0$. For $\Phi(q)\in \bbR_{\geq 0}^{\sS}$ and $\Phi'(q)\in A_{\geq 0}(q)$, the \textbf{type $\II$ equation}
\begin{align*}
    \Psi_s(q) &=\Psi_{s'}(q)
    \quad\forall s,s'\in\sS
    ;
    \\
    \la \vec\lambda,\Phi''(q)\ra &= 0
\end{align*}
is equivalent (for each fixed $q$) to 
\[
    \Phi''(q)=F(\Phi(q),\Phi'(q))
\]
for a locally Lipschitz function $F:\mathbb R_{\geq 0}^{\sS}\times A_{\geq 0}^{\sS}\to\mathbb R^{\sS}$. Moreover, \begin{equation}
\label{eq:Phi-stays-increasing}
|\Phi''_s(q)|\leq O(|\Phi_s'(q)|),\quad \forall s\in \sS.
\end{equation}
with a uniform constant for bounded $\Phi'(q)$.
\end{restatable}



\begin{lemma}
\label{lem:type-II-well-posed-appendix}
    The type $\II$ equation has a unique solution on $q\in [q_1,1]$ for any initial condition $(\Phi(q_1),\Phi'(q_1))\in \bbR_{\geq 0}^{\sS}\times A_{\geq 0}$. This solution satisfies $\Phi'(q)\succeq 0$ for all $q$.
\end{lemma}


\begin{proof}
The result now follows from Proposition~\ref{prop:ODE-well-posed}, since \eqref{eq:Phi-stays-increasing} implies that $\Phi'_s(q)$ stays non-negative for all $s$, and stays strictly positive if $\Phi_s'(q_1)>0$. 
\end{proof}


\begin{proof}[Proof of Proposition \ref{prop:tree-descending-trajectory}]
    Given the above, it only remains to show existence and uniqueness of $\vv$. Consider the matrix
    \[
    M(\vx)_{s,s'}
    =
    \frac{\partial_{x_{s'}}\xi^s(\vx)}
    {\xi^s(\vx)+h_s^2}.
    \]
    Then $M$ has strictly positive entries by non-degeneracy.
    The equation $M^*_\sym(\vx)\vv=\vzero$ is equivalent to $M^*(\vx)\vv=\vzero$ by \eqref{eq:M*sym-to-M*}, which is in turn equivalent to
    \[
    M(\vx)\vv=\vv.
    \]
    Since $\vx\neq \vzero$, non-degeneracy of $\xi$ implies that $\xi^s(\vx)>0$ so there is no division by $0$. Hence any such $\vv$ is uniquely determined as the Perron-Frobenius eigenvector of $M$.
    Conversely it is easy to see that if $M$ has Perron-Frobenius eigenvector \textbf{not} equal to $1$ then $M^*$ would not be solvable, which ensures that $\vv$ as above exists. 
\end{proof}



\begin{corollary}
\label{cor:regular-final}
    $p,\Phi_s\in C^1([q_0,1])$ and their restrictions to $[q_1,1]$ are $C^2$.
\end{corollary}

\begin{proof}
    From Proposition~\ref{prop:basic-regularity}, for the first statement it suffices to verify continuity of $p',\Phi_s'$ at $q_0$. If $\vh\neq \vzero$ this follows by Lemmas~\ref{lem:ODE-Lipschitz} and \ref{lem:p-concave}. If $\vh=\vzero$ this and the second conclusion both follow from Lemmas~\ref{lem:q0-q1-zero} and \ref{lem:type-II-well-posed-appendix}.
\end{proof}

The statement of Theorem~\ref{thm:alg-optimizer} is a combination of many of the results established in this section.

\begin{proof}[Proof of Theorem~\ref{thm:alg-optimizer}]
Existence of a maximizer $(p,\Phi;q_0)$ was shown in Proposition~\ref{prop:F-max}, and such $p,\Phi$ are continuously differentiable on $[q_0,1]$ by Corollary~\ref{cor:regular-final}.
The value $q_1$ was identified in Lemma~\ref{lem:s1-s2-separate}. 
The behavior on $S_1=[q_0,q_1]$ and $S_2=[q_1,1]$ comes directly from the well-posedness of the corresponding ODEs as shown in Lemmas~\ref{lem:ODE-Lipschitz} and \ref{lem:type-II-well-posed-appendix}. 
The formula~\eqref{eq:alg-for-optimizer} was proved in Corollary~\ref{cor:alg-value}.
The last assertions follow from Proposition~\ref{prop:type-1}.
\end{proof}


We finally prove a slight generalization of Proposition~\ref{prop:type-II-locally-unique}. Recall that $\Delta^r\subseteq \bbR_{\geq 0}^r$ denotes the simplex of admissible $\Phi'$ vectors. For any initial point $\vx$ and time-increment $t>0$, solving the type $\II$ equation yields a map $F_{\vx,t}:\Delta^r\to \Delta^r$ given by
\begin{equation}
\label{eq:F-vx-t}
    F_{\vx,t}(\vv)=(\Phi(q+t)-\vx)/t
\end{equation}
where $\Phi$ solves the type $\II$ equation with initial condition $\Phi(q)=\vx$, $\Phi'(q)=\vv$. 



We remark that in the case $\vx=0$ of Proposition~\ref{prop:type-II-locally-unique}, surjectivity also follows simply by taking $(p,\Phi;q_0)$ maximizing a version of $\bbA$ rescaled to have an arbitrary endpoint. 


\begin{corollary}
\label{cor:type-II-locally-unique}
Assume $\xi$ is non-degenerate. For $C>0$, there exists $\eps=\eps(C)$ such that the map $F_{\vx,t}$ defined in \eqref{eq:F-vx-t} is injective for $t\in [0,\eps]$ and $\|\vx\|_1\leq C$. Moreover $F_{\vx,t}$ is always surjective.
\end{corollary}


\begin{proof}
    An easy Gr{\"o}nwall argument using \eqref{eq:Phi-stays-increasing} implies that for $0\leq t\leq \eps$, 
    \[
    \la \Phi(q+t)-\wt\Phi(q+t),\Phi'(q)-\wt\Phi'(q)\ra>0
    \]
    for any pair $(\Phi,\wt\Phi)$ of solutions to the type $\II$ equation with $\Phi(q)=\wt\Phi(q)$ and $\Phi'(q)\neq\wt\Phi'(q)$. This implies injectivity. Surjectivity follows from Lemma~\ref{lem:topology} since \eqref{eq:Phi-stays-increasing} implies that if $\vv_s=0$ then $F_{\vx,t}(\vv)_s=0$.
\end{proof}



\begin{lemma}[{\cite[Lemma 2.1]{jamison1976factoring} or \cite[Lemma 1]{karasev2009kkm}}]
\label{lem:topology}
    Let $F$ be a continuous map from $\Delta^r$ to itself such that $F(\vv)_s=0$ if $v_s=0$. Then $F$ is surjective.
\end{lemma}





\subsection{Explicit Solution for Pure Models}
\label{subsec:pure}


In this subsection we prove Theorem~\ref{thm:pure} and Corollary~\ref{cor:pure}, obtaining an explicit description of $\hALG$ in the important special case of \emph{pure} models for which
\begin{equation}
\label{eq:pure-mixture}
    \xi(x_1,\dots,x_r)=\prod_{s\in\sS} x_s^{a_s}.
\end{equation}
Due to the homogeneity and lack of external field, it is natural to expect that the optimal $(p,\Phi)$ is given by $p\equiv 1$ and $\Phi(q)=(q^{b_1},\dots,q^{b_r})$ for positive constants $b_s$. (Here we do not require $\Phi$ to be admissible, which by Lemma~\ref{lem:admissible-optional} does not make a difference.) Most of our previous results do not apply directly because $\xi$ violates the non-degeneracy condition, however as mentioned previously we can apply them after adding a small perturbation.


\begin{lemma}
\label{lem:pure-p=1}
For a pure model described by $\xi$, there exists $\Phi^*$ such that with $p\equiv 1$,
\[
   \bbA(p,\Phi^*;0)=\hALG. 
\]
\end{lemma}

\begin{proof}
Let 
\[
\xi^{(\eps)}(\vx)=\xi(\vx)+\eps\sum_{s,s'\in\sS} x_sx_{s'}+\eps\sum_{s,s',s''\in\sS}x_s x_{s'}x_{s''}.
\]
Then the preceding results show that optimal solutions $(\Phi^{(\eps)},p^{(\eps)},q_0^{(\eps)})$ for $\xi^{(\eps)}$ satisfy $p^{(\eps)}\equiv 1$ and $q_0^{\eps(\eps)}=0$. Taking a convergent subsequence $\Phi^{(\eps)}\to\Phi^*$ as $\eps\to 0$ in the space $\cM$ (shown to be compact in Appendix~\ref{subsec:maximizer-existence}) implies the result since $\hALG$ is continuous in $\xi$.
\end{proof}



We first non-rigorously guess the solution by assuming it is of the form \eqref{eq:pure-mixture} and also solves the type $\II$ equation. By homogeneity, we may assume 
\begin{equation}
\label{eq:B=1}
    \sum_{s\in\sS}a_{s}b_{s}=1.
\end{equation}
Then
\begin{align*}
    \Phi_s'(q)&=b_s q^{b_s-1},
    \\
    (\xi^s\circ\Phi)(q)
    &=
    \frac{a_s}{\lambda_s}q^{1-b_{s}}.
\end{align*}
We thus expect that for some constant $L$ independent of $s$,
\begin{align*}
    \Psi_s(q) &=
    b_s^{-1} q^{1-b_s}
    \deriv{q}\sqrt{\frac{b_s q^{b_s-1}}{q^{1-b_s-1}a_s(1-b_s)/\lambda_s}}
    \\
    &=
    \sqrt{\frac{\lambda_s}{a_s(1-b_s)b_s}}q^{1-b_s}
    \deriv{q}{q^{-\frac{1}{2}+b_s}}
    \\
    &=
    \lt(-\frac{1}{2}+b_s\rt)\sqrt{\frac{\lambda_s}{a_s(1-b_s)b_s}}
    q^{-1/2}
    \\
    &=
    -L^{-1/2}q^{-1/2}.
\end{align*}
(Recall that $\Psi_s$ should be negative.) The resulting quadratic equation in $b_s$ has solution
\begin{equation}
\label{eq:pure-L}
    b_s=\frac{1- \sqrt{\frac{a_s}{a_s+L\lambda_s}}}{2}.
\end{equation}
Finally $L$ is chosen to satisfy \eqref{eq:B=1}; it is easy to see there is a unique such choice. 


Our next step is to verify the computation above and prove uniqueness. 


\begin{proof}[Proof of Theorem~\ref{thm:pure}]
~\\
\paragraph{Part $1$: Value of $\ALG$}
%
Here we assume $p\equiv 1$, relying on Lemma~\ref{lem:pure-p=1}, and determine the value $\ALG$. Using the purity of $\xi$, a simple scaling argument shows the $\hALG$ value with endpoint $\vx=(x_1,\dots,x_r)$ (cf. Remark~\ref{rem:normalization})
is given by
\begin{equation}
\label{eq:pure-scaling}
    \hALG(\vx)=\hALG(\vone)
    \cdot
    \prod_{s\in \sS}x_s^{a_s/2}.
\end{equation}
(Recall that $\xi$ is a covariance, hence the factor $1/2$ in the exponent on the right-hand side.)
Set $\phi_{D-1}^s=1-b_s \delta$ for small $\delta$ and $\vb\succeq 0$ satisfying \eqref{eq:B=1}. This is a fully general choice for $\phi_{D-1}$ as in Section~\ref{sec:uc}. In light of Proposition~\ref{prop:what-F-is}, we obtain that for small $\delta>0$,
\begin{equation}
\label{eq:pure-DP}
    \hALG(\vone)
    =
    \max_{\vb\,:\,\eqref{eq:B=1}}
    \big(\hALG(\phi_{D-1})
    +
    \delta\sum_{s\in\sS} \lambda_s\sqrt{\lambda_s^{-1}a_sb_s(1-b_s)}
    \big)
    +o(\delta).
\end{equation}
Denoting $\hALG=\hALG(\vone)$ and using \eqref{eq:pure-scaling}, we find 
\begin{align*}
    \hALG&=
    \max_{\vb\,:\,\eqref{eq:B=1}}
    \Big(\hALG\cdot\prod_{s\in\sS} (1-b_s\delta)^{a_i/2} + \delta\sum_{s\in\sS} \lambda_s\sqrt{\lambda_s^{-1}a_sb_s(1-b_s)}
    \Big)
    +o(\delta)
    \\
    &=
    \max_{\vb\,:\,\eqref{eq:B=1}}
    \bigg(\lt(1-\frac{\delta}{2}\rt) \hALG+\delta\sum_{s\in\sS} \sqrt{\lambda_s a_s b_s (1-b_s)}
    \bigg)+o(\delta).
\end{align*}
Rearranging and sending $\delta\to 0$ yields
\begin{equation}
\label{eq:ALG-pure-max}
    \hALG=
    2\max_{\vb\,:\,\eqref{eq:B=1}} 
    \sum_{s\in\sS} 
    \sqrt{\lambda_s a_s b_s (1-b_s)}.
\end{equation}
First, it is easy to see that any maximizing $\vb^*$ has $b_s^*>0$ for all $s$, since otherwise the derivative of the right-hand side in $b_s$ would be infinite. By Lagrange multipliers, for some $C>0$ any solution will have
\begin{equation}
\label{eq:LM-formula}
\begin{aligned}
    \sqrt{\frac{a_s}{L\lambda_s}}
    &=\deriv{b_s}\lt(\sqrt{b_s(1-b_s)}\rt)
    \\
    &=\frac{\frac{1}{2}-b_s}{\sqrt{b_s(1-b_s)}}
\end{aligned}
\end{equation}
for some $L\in [0,\infty]$ (where division by $\infty$ gives $0$).

 
Let us first assume $\sum_{s\in\sS} a_s\geq 3$. Then \eqref{eq:B=1} implies that $b_s<1/2$ for some $s$, hence for all $s$ since the signs have to match in \eqref{eq:LM-formula}. In particular we have $L<\infty$, and \eqref{eq:pure-L} above easily follows from \eqref{eq:LM-formula}.
The resulting formula is as desired:
\begin{align*}
    \hALG
    &=
    2\sum_{s\in \sS}
    \sqrt{\lambda_s a_s}\cdot \lt(\frac{1}{2}-b_s\rt)\sqrt{\frac{L\lambda_s}{a_s}}
    \\
    &=
    \sum_{s\in\sS}
    \lambda_s
    \sqrt{\frac{ L a_s}{L\lambda_s+a_s}}
    .
\end{align*}



The only remaining case is $\xi(x_1,x_2)=x_1 x_2$. Then it is clear from \eqref{eq:ALG-pure-max} that $b_1=b_2=1/2$ and 
\[
\hALG=\sqrt{\lambda_1}+\sqrt{\lambda_2}.
\]
(This case of Theorem~\ref{thm:pure} is stated with $b_1=b_2=1$ which is an equivalent parametrization.)

\paragraph{Part $2$: Uniqueness Assuming $p\equiv 1$}
%
Next we show the optimal trajectory $\Phi^*(q)=(q^{b_1},\dots,q^{b_r})$ is unique up to reparametrization when $p\equiv 1$. The maximization problem in \eqref{eq:ALG-pure-max} is strictly convex on the affine subspace defined by \eqref{eq:B=1}, and hence has a unique minimizer. It follows that if $\phi_d$ in the preceding equation is defined by any choice $\vb$ bounded away from the optimal one, the obtained value would be strictly worse than $\ALG$. In other words, any optimal trajectory where $p\equiv 1$ must satisfy $\Phi'(1)=\vb$. By scale-invariance, we conclude that $\Phi(q)=(q^{b_1},\dots,q^{b_r})$ is the unique optimal such trajectory.



\paragraph{Part $3$: Uniqueness of Optimal $p$}
%
Finally we prove that all optimal solutions actually satisfy $p\equiv 1$.
Suppose another maximizer $(p,\Phi)$ exists. Let
\[
    q_*=\inf_{q>0}\{q~:~\min_{s\in\sS}\Phi_s(q)>0\}.
\]
The definition of $p$ on $[0,q_*)$ is irrelevant so we assume without loss of generality that $p$ is constant on $[0,q_*]$ and continuous at $q_*$. It is easy to see that such a maximizing $p$ must be continuous on all of $[0,1]$ and satisfy $p(1)=1$; otherwise $p$ could be strictly increased while keeping $p'$ constant for the purposes of $\bbA$. The proof of Lemma~\ref{lem:p-AC} implies that $p$ is uniformly Lipschitz on $[q_*+\eps,1]$ for any $\eps>0$, so that $p'$ makes sense as a measurable function. 

We have seen that if $p\equiv 1$ then $\ALG$ is achieved by a unique $\Phi$, so we remains to show that no optimal $(p,\Phi)$ satisfies $p\not\equiv 1$  Assuming that $p\not\equiv 1$ we may choose $q>q_*$ a Lebesgue point for both $p'$ and $\Phi'$ such that
\[
    p'(q)>0.
\]
We now derive a contradiction by expanding $\hALG$ around $q$ as in \eqref{eq:pure-DP}. In particular, consider $\phi_d=\Phi(q-\delta)$ and $p_d=p(q-\delta)$.
Let $\Delta_s=\Phi_s(q)-\phi_{d,s}$ and $\Delta_p=p(q)-p_d$. Since $q$ is a Lebesgue point, we have $\Delta_s=\Phi_s'(q)\delta+o(\delta)$ and $\Delta_p=p'(q)+o(\delta)$.


The computation above for the value $\hALG$ implies 
\[
    \hALG(p_d,\phi_d)
    =
    \hALG(\phi_d)\sqrt{p_d}
    .
\]
Here $\hALG(p_d,\phi_d)$ denotes the analog of \eqref{eq:alg} with endpoint value $\Phi(q)=\phi_d$ rather than $q=1^{\sS}$, and $p(q)=p_d$.
Therefore
\begin{align*}
    \hALG(p(q),\Phi(q))
    &=
    \hALG(\phi_d)
    \sqrt{p_d}
    +
    \sum_{s\in\sS}
    \lambda_s
    \sqrt{\Delta_s \lt(\Delta_p \xi^s(\phi_d)+p_d \sum_{s'\in\sS} \partial_{x_{s'}}\xi^s(\phi_d)\Delta_{s'} \rt)}
    +
    o(\delta)
    \\
    &=
     \hALG(p(q),\Phi(q))
    \cdot
    \lt(1-\frac{\delta}{2}\times\lt(\frac{p'(q)}{p(q)}+\sum_{s\in\sS} \frac{a_s \Phi_s'(q)}{\Phi_s(q)}\rt)\rt)
    \\
    &\quad\quad
    +
    \delta\sum_{s\in\sS}
    \lambda_s
    \sqrt{\Phi'_s(q) \lt(p'(q) (\xi^s\circ\Phi)(q)+p(q) \sum_{s'\in\sS} \partial_{x_{s'}}(\xi^s\circ\Phi)(q)\Phi_{s'}'(q) \rt)}
    +
    o(\delta)
    .
\end{align*}
Rearranging and sending $\delta\to 0$ implies
\begin{equation}
\label{eq:ALG-pure-recursion}
    \hALG(p(q),\Phi(q))/2
    =
    \frac{
    \sum_{s\in\sS}
    \lambda_s
    \sqrt{\Phi'_s(q) \lt(p'(q) (\xi^s\circ\Phi)(q)+p(q) \sum_{s'\in\sS} \partial_{x_{s'}}(\xi^s\circ\Phi)(q)\Phi_{s'}'(q) \rt)}
    }
    {
    \frac{p'(q)}{p(q)}+\sum_{s\in\sS} \frac{a_s \Phi_s'(q)}{\Phi_s(q)}
    }
    \,.
\end{equation}
We claim that \eqref{eq:ALG-pure-recursion} forces $p'(q)=0$, which completes the proof of uniqueness since $q$ was an arbitrary choice of Lebesgue point. Note that from any solution to \eqref{eq:ALG-pure-recursion} we immediately get a maximizing $(\Phi,p)$ for $\bbA$ where $p(q)$ and each $\Phi_s(q)$ is a monomial of the form $aq^b$. 



The right-hand side above has maximum value $\hALG(p(q),\Phi(q))/2$, and we already know from Lemma~\ref{lem:pure-p=1} there exists $(p'(q),\Phi'(q))$ achieving this value with $p'(q)=0$. Supposing another maximizing $(\wt p'(q),\wt\Phi'(q))$ with $\wt p'(q)>0$ exists, 
we suppress the input $q$ and consider a general solution 
\[
    (p_a',\Phi_a')=\big(ap_1'-(a-1)p_0',a\Phi_1'-(a-1)\Phi_0' \big).
\]
We always restrict to $a$ such that all derivatives are non-negative. The denominator of the right-hand side of \eqref{eq:ALG-pure-recursion} is affine in $a$, while Lemma~\ref{lem:sqrt-xy-concave} implies the numerator is concave. Since $(p_0',\Phi_0')$ and $(p_1',\Phi_1')$ both maximize the right-hand side we deduce that it takes the constant value $\hALG(p(q),\Phi(q))/2$ on $(p_a',\Phi_a')$ for all $a\in [0,1]$. In particular using again Lemma~\ref{lem:sqrt-xy-concave} we find that each of the $r$ terms in the numerator is actually a linear function of $a$ on the interval such that 
\begin{equation}
\label{eq:good-a}
    p_a'(q)\geq 0, \quad\text{and}\quad 
    \min_s \Phi_{a,s}'(q)\geq 0.
\end{equation}
This means equality is achieved for $p_a$ for $a$ satisfying \eqref{eq:good-a} (even if $a>1$) and implies that $\Phi'(q)\neq \wt\Phi'(q)$. Let $a_*>0$ be the maximal value satisfying \eqref{eq:good-a}, so that $\min_s \Phi_{a_*,s}'(q)=\Phi_{a_*,s_*}'(q)=0$. Then clearly the $s_*$ term of the numerator is not affine on $a\in [a_*-\eps,a_*]$; since $\Phi_{a_*}'$ satisfies admissibility it does not equal $\vzero$. This gives a contradiction, so we conclude that $p\equiv 1$ holds for all optimal $(p,\Phi)$.  
\end{proof}

\begin{proof}[Proof of Corollary~\ref{cor:pure} ]
    Here we have $\lambda_s=\frac{a_s}{\sum_{s\in\sS} a_s}$ in the preceding formulas. It is easy to see from \eqref{eq:pure-L} that the values $b_s$ are all equal. From \eqref{eq:B=1} we find $b_s=\frac{1}{\sum_{s\in\sS} a_s}$ and so
\begin{align*}
    \hALG
    &=
    2\sum_{s\in \sS}\sqrt{\lambda_s a_sb_s(1-b_s)}
    \\
    &=
    2\sum_{s\in \sS} \frac{a_s}{\sqrt{\sum_{s\in\sS} a_s}}\cdot \frac{\sqrt{\big(\sum_{s\in\sS} a_s\big)-1}}{\sum_{s\in\sS} a_s}
    \\
    &=
    2\sqrt{\frac{\big(\sum_{s\in\sS} a_s\big)-1}{\sum_{s\in\sS} a_s}}.
\end{align*}
\end{proof}



We finally show Corollary~\ref{cor:E-infty}, recalling the formula for $E_{\infty}$ from \cite{mckenna2021complexity} and verifying it equals $\ALG$ for pure models. It is given as follows, where $\bbH=\{z\in\bbC~:~{\mathsf {Im}}(z)> 0\}$ denotes the complex open upper half plane. We recall (a slight generalization of) \cite[Lemma 2.2]{mckenna2021complexity}; as written
only the bipartite case was considered therein but the general multi-species case is no different.
Additionally we point out that the constants $\alpha_s$ appearing in \cite{mckenna2021complexity} continue to vanish in pure models for general $r$, which we take advantage of in the statement below.


Informally, the point below is simply that $\sum_s \lambda_s M_s$ is the Stieltjes transform of the bulk spectral distribution of an $N\times N$ random matrix with variance profile $\partial_{x_s,x_{s'}}\xi$ with diagonal species-dependent shift $E\xi^s(\vone)$. This essentially corresponds to the behavior of the Riemannian Hessian $\nabla^2_{\sph}H_N(\bsig)$ at a point $\bsig$ with $H_N(\bsig)=E$, where the diagonal shift corresponds to the induced radial derivative of $H_N$.


\begin{proposition}[{Adaptation of \cite[Lemma 2.2]{mckenna2021complexity} with $r$ species and pure $\xi$}]
\label{prop:E-infty-mckenna}
    For $z\in\bbH$ (resp. $-\bbH$), there is a unique solution $\vec M\in \bbH^{\sS}$ (resp. $-\bbH^{\sS}$) to the matrix Dyson equation
    \[
    1+M_s\lt(
    \big(z-E\xi^s(\vone) \big) 
    +
    \partial_{x_{s}}\xi^{s}(\vone)
    M_{s}
    +
    \sum_{s'\neq s}
    (\partial_{x_{s'}}\xi^s(\vone))
    M_{s'}
    \rt)
    =0,\quad\forall s\in\sS.
    \]
    The threshold $E_{\infty}\geq 0$ is the smallest value such that with $z=0$, $\vec M(E)$ extends analytically and continuously at the boundary to $E\in [E_{\infty},\infty)$ (and is real-valued on this interval).
\end{proposition}

When $\xi(\vx)=\prod_{s\in\sS} x_s^{a_s}$ is pure and $z=0$, the vector Dyson equation simplifies to
\begin{equation}
\label{eq:matrix-dyson-pure}
    1+a_s M_s\bigg(
    E-\lambda_s M_s + 
    \sum_{s'\in\sS}
    \lambda_{s'}a_{s'}M_{s'}
    \bigg)
    =0,\quad
    \forall\,s\in\sS.
\end{equation}


\begin{proof}[Proof of Corollary~\ref{cor:E-infty}]
For convenience we omit the case $\xi(x_1,x_2)=x_1x_2$ and assume $\sum_s a_s\geq 3$.
Setting
\begin{align*}
    K_s&=a_sM_s,
    \\
    K&=\sum_{s\in\sS} \lambda_s K_s    
\end{align*}
the system \eqref{eq:matrix-dyson-pure} can be rearranged to
\[
    A\equiv K+E = \frac{\lambda_s K_s}{a_s}-\frac{1}{K_s},\quad \forall\,s\in\sS.
\]
With $B_s=\frac{a_s}{\lambda_s}$ we find that at $E=E_{\infty}$,
\[
    K_s=\frac{A B_s - \sqrt{A^2 B_s^2+4B_s}}{2}.
\]
Here the choice of sign is forced by $M_s<0$; this easily holds for sufficiently large $E$ (where one can give a power series expansion), and follows by continuity since $K_s\neq 0$ in general.

Note that $A$ above determines each $K_s$, hence $K$ and hence $E=A-K$.
Viewing $E$ as a function the $A$, its derivative must vanish and so:
\begin{align}
\nonumber
    0&=\frac{\de E}{\de A}
    \\
\label{eq:A-positive}
    &=
    1-\frac{1}{2}
    \sum_{s\in\sS}
    a_s\lt(
    1-\frac{A B_s}{\sqrt{A^2 B_s^2 + 4B_s}}
    \rt)
    \\
\label{eq:A-zero}
    &=
    1-\frac{1}{2}
    \sum_{s\in\sS}
    a_s\lt(
    1-\frac{1}{\sqrt{1 + 4/(A^2 B_s)}}
    \rt)
    \\
\nonumber
    &\stackrel{\eqref{eq:B=1}}{=}
    1-\frac{1}{2}
    \sum_{s\in\sS}
    a_s\lt(
    1-\sqrt{\frac{a_s}{a_s+L\lambda_s}}
    \rt)
    \\
\label{eq:L-zero}
    &=
    1-\frac{1}{2}
    \sum_{s\in\sS}
    a_s\lt(
    1-\sqrt{\frac{1}{1+L/B_s}}
    \rt)
    .
\end{align}
Here we used $\sum_s a_s\geq 3$ to deduce from \eqref{eq:A-positive} that $A>0$, thus implying the next line.
By monotonicity, equality of \eqref{eq:A-zero} and \eqref{eq:L-zero} now implies $A=2/\sqrt{L}$. 
Turning to the desired equality, we first write
\begin{align*}
    E_{\infty}&=A-K
    \\
    &=
    \frac{2}{\sqrt{L}}
    -
    \frac{1}{2}
    \sum_s 
    \lambda_s 
    \lt(
    \frac{2a_s}{\lambda_s \sqrt{L}}
    -
    \sqrt{
    \frac{4a_s^2}{L\lambda_s^2}
    +
    \frac{4a_s}{\lambda_s}
    }
    \rt)
    \\
    &=
    \frac{2}{\sqrt{L}}
    -
    \sum_s 
    \frac{a_s}{\sqrt{L}}
    \lt(1-
    \sqrt{
    \frac{a_s+L\lambda_s}{a_s}
    }
    \rt).
\end{align*}
With $V_s\equiv\sqrt{a_s+L\lambda_s}$, adding and subtracting $\sum_s \frac{a_s^{3/2}}{V_s\sqrt{L}}$ to get the second equality, we compute
\begin{align*}
    E_{\infty}-\ALG
    &=
    \frac{2}{\sqrt{L}}
    +\sum_s
    \lt(
    -\frac{a_s}{\sqrt{L}}
    +
    \frac{V_s\sqrt{a_s}}{\sqrt{L}}
    -
    \frac{\lambda_s \sqrt{L a_s}}{V_s}
    \rt)
    \\
    &=
    \frac{1}{\sqrt{L}}\lt(
    2
    +
    \sum_s
    \lt(
    -a_s
    +
    \frac{a_s^{3/2}}
    {V_s}
    \rt)
    \rt)
    +
    \sum_s
    \frac{\sqrt{a_s}}{V_s \sqrt{L}}
    \lt(
    V_s^2 - a_s-L\lambda_s
    \rt)
    \\
    &=0.
\end{align*}
Here in the last step, we used \eqref{eq:B=1} to handle the first contribution (summed over $s\in\sS$) and the definition of $V_s$ for the second (for each $s\in\sS$). 
\end{proof}


