
\documentclass[%
 reprint,
 amsmath,amssymb,
 aps,
]{revtex4-1}
%\usepackage[dvipdfmx]{graphicx}
\usepackage{graphicx}
\usepackage{dcolumn}
\usepackage{bm}
\usepackage{dsfont}


\newcommand{\ens}[1]{\left< {#1} \right>}

\begin{document}

\preprint{APS/123-QED}

\title{Econotaxis in modeling urbanization by labor force migration}


\author{Hirotaka Goto}
\affiliation{
 Graduate School of Advanced Mathematical Sciences, Meiji University, 
 4-21-1 Nakano, Nakano City, Tokyo, Japan}
  \email{cs221005@meiji.ac.jp}
\date{\today}


\begin{abstract}

Individual participants in human society collectively exhibit aggregation behavior. 
In this study, 
we present a simple microscopic model of labor force migration 
by employing the 
active Brownian particles framework.
Through agent-based simulations, we find
that our model produces clusters of agents from a random initial distribution. 
Furthermore, two empirical regularities called Zipf's and Okun's laws were observed in our model. 
To reveal the mechanism underlying the reproduced agglomeration phenomena, we 
derived
an extended Keller--Segel system, 
a classic model that describes the aggregation behavior of biological organisms called \emph{taxis},
from our microscopic model.
The obtained macroscopic system indicates that the agglomeration of the workforce in real world 
can be accounted for through a new type of taxis 
central 
to human behavior, 
which highlights the relevance of urbanization to 
blow-up phenomena in the derived PDE system. 
We term it ``econotaxis.''

\end{abstract}

\pacs{Valid PACS appear here}
\maketitle




\section{\label{sec: introduction} Introduction}

Urbanization, a process of population concentration \cite{tisdale1941process}, is recognized as a significant characteristic of human society 
\cite{berry2008urbanization, boddy1999geographical, ottaviano2004agglomeration}. 
Having occurred for more than three hundred years, 
urbanization continues, showing an increasing tendency toward \emph{agglomeration} 
\cite{tisdale1941process, fujita2002agglomeration}. 
The topic has been discussed 
 in many fields, including geography, economics, and urban planning, 
from a wide range of perspectives. 
Nevertheless, 
the manner in which people and firms get concentrated 
is expected to be widely universal, 
given 
that similar agglomeration patterns
and hierarchical structures have been discovered 
across the globe \cite{fujita1996economics, brakman2003rethinking, berry2008urbanization}. 


In the last two centuries, 
theories have been proposed 
to explain the spatial distributions of human activity and settlements, 
such as 
von Th{\"u}nen's theory \cite{grotewold1959thunen, maki2004realism}, 
Weberian location theory \cite{isard1949general, capello2014classical}, 
and the central place theory \cite{king2020central, mulligan1984agglomeration}. 
Toward the end of last century, in economics, a new discipline called New Economic Geography 
was established, 
and 
spatial economic problems were beginning to be considered seriously 
\cite{ottaviano2004agglomeration, martin1999new}. 
Currently, these problems  
are commonly addressed by
regarding them as either a cost-minimization (or profit-maximization) problem 
or an equilibrium state where the benefits
of agglomeration and the resulting costs balance each other
\cite{glaeser2010agglomeration, puga2010magnitude, duranton2004micro, boddy1999geographical, ottaviano2004agglomeration}. 

However, 
theories proposed in previous studies 
have not 
led to a 
a clear understanding of agglomeration phenomena 
\cite{berry2008urbanization, fujita1996economics}. 
For example, 
some of them have been criticized 
for 
their 
unestablished empirical basis, 
unclear microscopic underpinnings, 
lack of circular causation, 
little reference to intraregional dynamics and emergent nature of urban areas 
\cite{king2020central, tisdale1941process, 
martin1999new, boddy1999geographical, krugman1992dynamic}. 
Furthermore, 
theoretical arguments derived from these theories are not always verifiable 
\cite{duranton2004micro, martin1999new} 
because they 
have often been formulated in an oversimplified manner or under implausible assumptions, 
thereby making 
a comparison with empirical data 
unfeasible \cite{scott2006perspective}. 
Thus, 
a new yet simple model 
that 
aligns with empirical observations 
is required 
to uncover the fundamental mechanism driving agglomeration in human society. 


Over the past few decades, 
statistical physics has proven to be a powerful tool 
for examining and understanding 
social dynamics
\cite{castellano2009statistical}. 
A central example is 
human mobility patterns, an emerging field 
attributed to 
the increasing availability of large-scale 
empirical data \cite{gonzalez2008understanding, schlapfer2021universal, shida2020universal}. 
Although most of them focus on human mobility in the relatively short run, 
migration patterns over 
centuries, such as urbanization, can
be studied similarly. 

Consistent with this line of research, 
the author in \cite{schweitzer2002modelling} prominently introduced 
a dynamic model of economic agglomeration 
by employing active Brownian particles (ABPs) \cite{schweitzer2003brownian, romanczuk2012active}, 
the concept of which was originally introduced in \cite{schimansky1995structure}. 
ABPs (or sometimes Brownian agents \cite{helbing2001traffic}) 
are Brownian particles capable of generating a field that, 
in turn, influences their motion
\cite{romanczuk2012active}. 
Using this framework, 
the author 
investigated the emergence and evolution of economic centers 
and observed their coexistence. 
However, a major drawback of this approach 
is 
that the reproduced 
emergence of central regions is attributable to the
presupposed strong nonlinearity of a production function, 
coupled with 
a rapid 
response of the local employment status 
to changes in 
local economic circumstances. 


More specifically, 
in \cite{schweitzer2002modelling}, 
two types of agents, 
\emph{employed} and \emph{unemployed}, that transition between one another
are considered, 
and 
only the unemployed ones can migrate in a two-dimensional domain. 
They, as Brownian agents, 
navigate 
the domain 
according to the following deterministic force $f(r)$:  
\begin{align}
f(r) &= \nabla \omega(r), \quad 
\omega (r) = \frac{\delta Y}{\delta l}  (l(r)),  \nonumber \\
Y(l)  &\propto [A_0 +\exp (a_1 l - a_2 l^2)] \, l^\beta,  \label{eq: nonlinear production function}
\end{align}
where $r$ represents position, 
$\omega (r)$ 
a wage field, 
$Y(l)$ production output as a function of labor, 
$l(r)$ the density of employed agents, 
and $A_0, a_1, a_2, \beta$ are
some positive constants. 
The issue with
this formulation is that 
the initiated agglomeration 
depends entirely on the \emph{artificial} 
modification, which is
expressed as an exponential term in Eq.~(\ref{eq: nonlinear production function}), 
of a normal production function. 
Consequently, 
$f$ works on unemployed agents as an agglomeration force only in a certain middle range of $l$. 
Therefore,
agglomeration cannot 
arise 
from 
a random initial distribution of employed agents 
unless $l$ is initially set high enough. 
Nonetheless, 
agglomeration does occur 
in \cite{schweitzer2002modelling} 
because 
 another type of strong nonlinearity 
 is 
 assumed 
 in the rates of transition between two states 
(employed and unemployed) to help $l$ increase and reach that ``special'' range. 
In other words, without 
these technical conditions, 
city-like regions would not appear, let alone their coexistence.
Therefore, the aforementioned formulation 
may not be successful in providing 
plausible 
explanations for real-world agglomeration phenomena such as urbanization. 



As with the tendency toward agglomeration, 
society is endowed with other characteristic features, 
many of which have been
presented in the form of statistical regularities. 
The rank-size rule is a power-law 
relationship between the city size and its rank in a given urban system \cite{hinloopen2006comparative, fujita1996economics, rose2005cities}, described in the following mathematical form: 
\begin{align}
(\text{city size}) \propto (\text{rank})^{ - \gamma}, \label{eq: Zipf}
\end{align}
where the rank-size exponent $\gamma$ is close to 1 
\cite{gabaix2004evolution, veneri2016city}. 
This provides a good rule of thumb that 
the $N$th largest city has a population proportional to
one-$N$th 
the population of the largest city in a community, 
which is also known as a good approximation of \emph{Zipf's law} 
\cite{reed2002rank, gabaix1999zipf}. 
Zipf's law holds only for large cities 
\cite{hinloopen2006comparative}. 
The well-documented regularity is generally considered a manifestation of universality in human 
migration and 
settlements. 
Several models have been proposed to replicate the law, such as Steindl's and Simon's models
\cite{simon1955class, gabaix1999zipf}. 
However, they have been criticized 
for their counterfactual settings 
or limited applicability. 


In relation to agglomeration phenomena, 
empirical studies have indicated 
that a rank-size distribution demonstrating Zipf's law 
is a sign of urban development 
\cite{cristelli2012there, wang2021economic, guerin1995rank}.
Several studies have reported a gradual yet continuing increase 
in rank-size exponents 
even after they reach $1$ 
, such as 
the city-size distribution in China 
from 1982 to 2010 \cite{wang2021economic} 
and the country-size distribution of the 50 largest countries worldwide from 1990 to 2050 (projections included) \cite{rose2005cities}.
As seen previously, 
this regularity 
appears 
on extensive and dynamic scales, 
thus 
requiring a more comprehensive theory of the law 
\cite{rose2005cities}. 




\emph{Okun's law} is another statistical characteristic 
studied
in 
a large body of 
macroeconomics literature \cite{okun1963potential}. 
It refers to the negative correlation between output growth and changes in the unemployment rate 
during an interval(e.g., quarterly) 
\cite{lee2000robustness}. 
That is, positive output growth corresponds to a decreasing unemployment rate. 
For empirical testing, the law can be expressed as follows \cite{knotek2007useful}:
\begin{align}
\frac{\Delta Y_{tot}}{Y_{tot}} 
\approx - c \, \Delta \mu , \label{eq: Okun}
\end{align} 
where $Y_{tot}$ is the total output across a nation, 
and $\mu$ is the (national) average unemployment rate at a given time. 
$\Delta Y_{tot}$ and $ \Delta \mu$ represent changes 
in the respective variables during a given interval. 
In the case of the U.S. economy, 
it predicts approximately a 2% increase in output (i.e., real GDP, empirically) 
for every one-point decrease in the unemployment rate, which translates to $c \approx 2$ 
\cite{lee2000robustness, cuaresma2003okun}. 
An accumulated body of empirical and theoretical research 
has demonstrated that
the law, including the coefficient, appears robust to some extent over different periods 
and under different methods of analysis \cite{ball2013okun}. 
Although there has been 
some variability in the coefficient among countries and controversy regarding the reliability of the law, 
the relationship is typical of most economies 
\cite{lee2000robustness}. 
Many existing theories consider fluctuations from 
hypothetical baseline values of certain 
variables 
that are not directly observable
yet fail to 
account for empirically observed coefficients \cite{ball2013okun}. 
Therefore, 
a new, unified theoretical approach may be of benefit 
\cite{knotek2007useful}. 




To address the aforementioned issues, 
we propose a simple labor migration model and show 
through agent-based simulation (ABS) 
that the model initiates the aggregation of agents from a random initial distribution.
We then demonstrate that 
the obtained spatiotemporal population distribution and economic circumstances 
are quantitatively similar to those in the real world 
simply 
by investigating the reproduced dynamics in terms of Zipf's and Okun's laws, respectively. 
To the best of our knowledge, 
an attempt 
to reproduce these two empirical laws in a single model of urbanization has never been made. 
Our 
ultimate
goal 
is 
to detect the key elements of agglomeration phenomena in human society 
and reveal the fundamental mechanism underlying them. 


The paper is organized as follows: 
In Sec.~\ref{sec: framework}, 
we develop a model of urbanization using ABPs 
and delineate our simulation scheme. 
In Sec.~\ref{sec: results}, 
we provide simulation results and derive a corresponding macroscopic model. 
Next, we highlight its relevance to the classical Keller--Segel model 
(a classic model that describes the aggregation behavior of biological organisms called taxis) and 
introduce a key concept called \emph{econotaxis} 
to characterize the reproduced aggregation phenomena. 
Lastly, in Sec.~\ref{sec: discussion}, 
we summarize our main results and discuss them 
from the perspective of some potential relevance to future work. 


\section{\label{sec: framework} Methods}

\subsection{\label{subsec: mathematical model} Mathematical Model}

We introduce a stochastic model of labor migration, 
following the framework proposed in \cite{schweitzer2002modelling}. 
Let us consider $N$ identical Brownian agents that migrate 
in a domain $\Omega = [0, L]^2\subset \mathbb{R}^2$, 
with $L$ being the system size. 
Brownian agents are characterized by their internal states 
$\theta_i$
and positions $r_i \in \Omega  \,(i=1, \cdots, N)$, 
and they interact indirectly with each other via their external environment 
\cite{schweitzer2003brownian, castellano2009statistical}. 
Each agent is assigned either the employed ($\theta_i = 0$) or unemployed ($\theta_i = 1$) status, 
and their spatial movements are governed by either of the following Langevin equations 
depending on their current states:
\begin{align}
\dot{r_i} &= \sqrt{2D_l} \, \xi_i (t)  \quad (\theta_i = 0), \label{eq: C0} \\
\dot{r_i} &= \sqrt{2D_n} \, \xi_i (t) + F(t) \quad (\theta_i = 1), 
\label{eq: C1}
\end{align}
where 
$\xi_i(t)$ represents white Gaussian noise with zero mean 
and $\ens{\xi_i(t) \cdot \xi_j(t')} = \delta_{i,j} \delta(t-t')$. 
$D_n$ and $D_l$ are some constants satisfying $D_n \gg D_l$. 
This implies 
that employed agents are almost intrinsically immobile 
compared to the unemployed 
because 
people in employment 
reasonably
prefer stability in their lives 
and are thus 
less motivated to migrate 
than the unemployed ones, who have incentives to travel even long distances to obtain a job. 
We also suppose that agents transition their status from $\theta_i=0$ to $\theta_i=1$
with a transition probability $k^-$, and vice versa with $k^+$. 


Now, we are left with the 
determination 
of $F(t)$, 
which plays a significant role in the spatiotemporal dynamics of population distribution. 
Unlike \cite{schweitzer2002modelling}, 
we assume that unemployed agents prefer \emph{economically active} places. 
In this context, 
we mean unemployed agents tend to migrate toward areas where a large economic output is produced. 
In other words, 
unemployed agents are drawn to places where, for example, well-paid or secure jobs are available, 
and their chances of being provided with those opportunities improve as the local economic production increases. 
Although living in those places can be accompanied, in reality, by various issues 
such as higher living costs and crime rates, 
we postulate that they offer significant benefits in almost every aspect of one's life, 
and the benefits of dwelling in urban areas 
generally outweigh the disadvantages. 
Empirical evidence seems 
to support this argument 
\cite{berry2008urbanization, boddy1999geographical}. 
Studies 
have shown 
that poorer people 
are attracted to cities, 
that urban-ward migration is a response to growing economic opportunity, 
and that 
a positive relationship exists between wages and city size.
In economics, production output can be computed as a function of the number of workers. 
The simplest and 
the most commonly used form is
the Cobb--Douglass production function \cite{fujita2013economics}, in which 
output is assumed to increase in proportion to the number of workers 
raised to some power. 
In 
our model, all of the above ansatzes can be 
summarized in the following equations:
\begin{align}
&f(r) =\alpha \nabla \ln [Y(l(r)) + 1],   \label{eq: econotaxis term} \\ 
&Y(l) = A \, l^{\beta},   \label{eq: normal production function}  
\end{align} 
where 
$f(r)$ is a deterministic force exerted on the agents located at $r$, 
$l(r)$ is the density of employed agents, 
$\alpha$ and $A$ are some constants, 
$Y(l)$ is a production function, 
and $\beta$ is a positive exponent smaller than $1$. 
The condition $\beta < 1$ 
ensures \emph{decreasing returns to scale}, 
which refers to situations where increasing input by a factor of $\eta$ results in 
increasing the output by a factor \emph{less} than $\eta$ \cite{fujita2013economics}.  
The logarithm in Eq.~(\ref{eq: econotaxis term}) is based on Fechner's law, 
originally proposed in psychophysics, 
which states that sensation grows proportional to stimulus intensity \cite{nutter2010weber}. 
Specifically, 
we regard production output as a perceivable stimulus to which 
unemployed agents respond in search of better employment opportunities. 
Note that 
we assume no strong nonlinearity in 
Eqs.~(\ref{eq: econotaxis term})~and~(\ref{eq: normal production function}), 
as opposed 
to \cite{schweitzer2002modelling}.


With regard to Eq.~(\ref{eq: C1}), 
to specify $F$, 
we first consider a spatial discretization of the entire domain $\Omega$ into square boxes 
with a unit length $h$, 
each of which is denoted by $\Lambda_{mn}$, 
with $m$ and $n$ denoting spatial indices. 
Notice that 
a unique $\Lambda_{mn} \subset \Omega$ exists for any given $r_i(t)$
such that $r_i(t) \in \Lambda_{mn}$. 
We define 
$F(t)$ by
\begin{align}
&F(t)  \nonumber
\\ &=   \label{eq: central difference}
\frac{1}{2h}
\begin{pmatrix}
(\ln [Y(l_{m+1,n})+1] - \ln [Y(l_{m-1,n})+1]) \\
(\ln [Y(l_{m,n+1})+1] - \ln [Y(l_{m,n-1})+1])
\end{pmatrix}
, \\
& l_{m, n}(t) = \frac{1}{h^2} \sum_{j=1}^{N}  
\mathds{1}_{\Lambda_{mn}} (r_j(t)) \delta_{\theta_j(t), 0}, 
 \label{eq: l histogram} 
\end{align}
where $\mathds{1}_{I} (x) = 1 \; (x \in I); \, 0 \; (x \notin I)$. 
Note that $F$ in Eq.~(\ref{eq: central difference}) 
plays the same role as $f$ in Eq.~(\ref{eq: econotaxis term}), except that it is reinvented 
for ABS 
using the central difference method. 
Practically, 
$\{l_{m, n}(t)\}_{m,n}$ is equivalent to a uniformly binned histogram 
that represents 
the spatial distribution of employed agents. 




\subsection{\label{subsec: simulation} Simulation Scheme}


We used the Euler--Maruyama method for time discretization of Eqs.~(\ref{eq: C0}) and (\ref{eq: C1}) 
with an interval of length $1$. 
The transition between the two statuses (employed and unemployed) 
was also implemented during each time step. 
All agents update their positions and status simultaneously, 
which constitutes a unit 
time step, 
and this entire updating process was iterated throughout the simulation.  
We performed our simulations 
with 
periodic boundary conditions.



Our model was also investigated in terms of two empirical laws. 
First, we  
clarified the definition of a city in the simulation 
to produce rank-size distributions. 
Unsurprisingly, what defines a city is debatable \cite{gabaix2004evolution}. 
Here we adopted a simple approach, 
in which 
we equated cities with the entire region divided into smaller rectangular areas 
using a \emph{non-uniform} spatial grid. 
More specifically, 
we separated the domain $\Omega$ into smaller rectangles 
by applying a coarse non-uniform spatial grid, which we represented by $\{\Gamma_{ij}\}$,  
having a fixed number of straight lines on each side 
with their intervals randomly specified. 
We then determined each rectangular area defined by the non-uniform grid as an equivalent of a city.    
Furthermore, 
to mitigate the stochastic influence of the aforementioned arrangement of cities on the resulting rank-size distributions, 
we ``filtered'' the same spatial 
population distribution produced via ABS 
by a hundred different non-uniform grids, 
$\{\Gamma^k_{ij}\}_{i,j} (k=1, \cdots, 100)$, 
and obtained a mean rank-size distribution following the same procedure using the grids  
(see Fig.~\ref{fig: city arrangement}). 
The average size of a city must be effectively larger than the unit spatial length $h$, 
such that, the scaling relationship between the migration of agents and the average city size is not violated.
In summary, 
the grid
$\{\Gamma^k_{ij}\}_{i,j}$ is non-uniform, generated many times, and coarse compared to 
$\{\Lambda_{mn}\}_{m,n}$, which is a fine uniform grid with a unit length $h$, 
used only 
for computing our agent-based model.  



\begin{figure}[htbp]
\begin{center}
\includegraphics[width=8.6cm]%[scale=0.17]
{fig1.png}
\caption{Schematic view of computing 
a mean rank-size distribution by applying different arrangements of cities.  
(A) An agent-based simulation is implemented 
using the fine uniform grid $\{\Lambda_{mn}\}$. 
(B) The entire region is randomly divided into rectangles 100 times 
using coarse, non-uniform grids $\{\Gamma^k_{ij}\}_{i,j}$. 
(C) Rank-size distributions are produced based on different configurations of cities 
corresponding to the coarse, non-uniform grids $\{\Gamma_{ij}^k\}_{i,j}$. 
(D) A mean rank-size distribution is obtained that is
averaged over all rank-size distributions produced in (C). 
In the following simulations, we prepare 289 cities in the $30 \times 30$ field ($L=30, h=1$)  
by applying a hundred $17 \times 17$ non-uniform grids generated randomly. 
} 
\label{fig: city arrangement}
\end{center}
\end{figure}



Regarding Okun's law, we investigate whether fractional changes 
in the total output, $\Delta Y_{tot} / Y_{tot}$, are (anti-)correlated with 
contemporaneous changes in the overall unemployment rate, $\Delta \mu$. 
Note that 
$Y_{tot}(t) = \sum_{m,n} Y(l_{m,n}(t))$ 
and $\mu(t) = \left(\sum_{i=1}^N \delta_{\theta_i (t),1}\right)/N$ in our model.  
We investigate the relationship at every time step. 





\section{\label{sec: results} Results} 

\subsection{\label{subsec: aggregation} Aggregation phenomena} 


\begin{figure}[htbp]
\centering
\includegraphics[width=8.6cm]%[scale=0.4]
{fig2.png}
\caption{Snapshots demonstrating the aggregation of agents produced by ABS. 
The parameters are as follows: $N=4000, D_n=0.30, D_l=0.03, A=1, \beta=0.67, k^+=0.30, k^-=0.55$. 
This set of parameters applies to the results in all figures that follow unless otherwise noted.  
The colored bars 
display the fractions of employed (left) and unemployed (right) agents in a uniformly discretized box 
relative to the total number of agents in each respective employment status at the time presented.}
\label{fig: aggregation1}
\end{figure}





Based on the method described in Sec.\ref{subsec: simulation},  
we implemented ABS to investigate the behavior of our model, 
focusing primarily on the spatiotemporal dynamics of the agent distributions. 
Fig.~\ref{fig: aggregation1} shows an evolution of the spatial distributions 
of employed and unemployed agents, 
illustrating 
the manner in which they 
undergo the process of self-organization  
from a spatially disordered to an ordered state.
Distributed randomly at the initial stage, these agents 
separate into  
aggregations before
$t=200$. 
Later, 
the clusters become tighter, and 
only some 
continue to 
intensify in magnitude 
while others disappear 
(from $t=400$ to $t=600$). 
In the following, 
we will discuss the mechanism of the observed aggregation phenomena. 


In Eq.~(\ref{eq: C1}), two opposing forces compete with one another 
-- \emph{agglomeration} and \emph{dispersion} forces. 
An agglomeration force arises from the differences 
in economic circumstances between neighboring places, 
resulting from the differences in the number of people hired at these places 
(see Eqs.~(\ref{eq: central difference}) and (\ref{eq: l histogram})). 
Meanwhile, 
a dispersion force 
is caused by random fluctuations 
in the motion of agents. 
If 
agglomeration overrides 
dispersion, unemployed agents start forming clusters. 
Once this occurs, 
the 
effect 
amplifies: 
the more unemployed agents there are, the more employed agents there will be 
because they constantly transition their employment status from unemployed to employed and vice versa. 
This leads to a higher production output, which 
results in
even stronger agglomeration forces. 
This positive feedback 
allows for the emergence of disproportionately highly populated areas. 
In Eq.~(\ref{eq: C0}), in contrast, only a dispersion force is at play. 
Thus, 
aggregation of employed agents is caused 
solely by clustered unemployed agents transitioning their job status.


\begin{figure}[htbp]
\begin{center}
\includegraphics[width=8.6cm]%[scale=0.4]
{fig3.png}
\caption{Snapshots produced by ABS 
in which no aggregation occurs owing to a significant imbalance of transition rates. 
The parameters are the same as those in Fig.~\ref{fig: aggregation1}, 
except that the transition rates are chosen as $k^+=0.9, k^-=0.1$ at the top and 
$k^+=0.1, k^-=0.9$ at the bottom.}
\label{fig: no aggregation bc of imbalance}
\end{center}
\end{figure}



Regarding the transitions of employment status, 
the balance between transition probabilities 
can be a critical factor in determining 
whether agents end up in aggregations. 
According to our simulations, 
agents seem to be prevented from being aggregated
when either of the transition probabilities is significantly larger than the other  
(Fig.~\ref{fig: no aggregation bc of imbalance}). 
We speculate that 
this is because 
a lopsided pair of transition probabilities produces 
a disproportionately large number of agents of one type 
and a few of the others, 
as the balance $k^+/k^-$ fundamentally 
determines both the overall and local employment rates. 
If most agents are employed, the population distribution becomes flat faster than some places gain momentum to grow at the expense of others.  
If most agents are unemployed, however, 
few employed agents are available to ramp up local production, 
which weakens the agglomeration force enough 
to stop causing particular orientation in the movements of unemployed agents.    
Therefore, keeping the ratio close to one 
is necessary 
to set off aggregation phenomena in the model. 
  
 \subsection{\label{subsec: Zipf} Empirical laws} 
 
\begin{figure}[htbp]
\begin{center}
\includegraphics[width=8.6cm]%[scale=0.18]
{fig4.png}
\caption{Evolution of the mean rank-size distribution for the data 
presented in Fig.~\ref{fig: aggregation1}. 
The data are presented in blue circles. 
The blue-shaded areas represent the standard deviations caused by applying 
a hundred different city configurations. 
The fitting interval 
is truncated at the 75th percentile (i.e., the top $25\%$), as indicated by the dotted vertical line.  
The truncated distribution is regressed linearly (black dashed line) 
using the ordinary least squares (OLS) method.}
\label{fig: Zipf1}
\end{center}
\end{figure}


Now that our model reproduces aggregation phenomena, 
we investigate how rank-size distributions evolve as well as 
their relevance to Zipf's law (Fig.~\ref{fig: Zipf1}). 
Note that none of the rectangular areas 
with a population of zero or one are 
considered cities in the first place, 
either in the data of Fig.~\ref{fig: Zipf1} or in the following discussion.   
First, high-ranking cities (above the $75$th percentile) 
appear to fit well with a straight line 
at all times presented in the figure on a logarithmic scale. 
Notably,
as time 
progresses, 
the slope 
becomes steeper, and the exponent (i.e., $\gamma$ in Eq.~(\ref{eq: Zipf})) 
gradually approaches $1$ (\emph{the Zipf exponent}) 
without sacrificing the $r^2$ values 
and then stabilizes around 
(from $t=400$ to $t=600$). 
In other words, 
the city size distribution exhibits Zipf's law corresponding to the process of aggregation.  



Furthermore,
the standard deviation 
(owing to 
how we arrange cities) 
remains small at all ranks and at all times presented.  
Although, during an initial stage, the aforementioned method causes moderately large fluctuations 
in the sizes of high-ranking cities, 
these fluctuations gradually disappear as the agents form clusters 
(see Figs.~\ref{fig: aggregation1} and \ref{fig: Zipf1}).  
In empirical studies,
the law has been shown to apply to 
data from various countries despite their varying definitions of an administrative city \cite{veneri2016city}.  
The small fluctuations shown in Fig.~\ref{fig: Zipf1} by the shading 
may suggest a similar type of robustness of the system for different units 
of analysis. 


\begin{figure}[htbp]
\begin{center}
\includegraphics[width=8.6cm]%[scale=0.18]
{fig5.png}
\caption{Time series of the rank-size exponent in Fig.~\ref{fig: Zipf1} for an extended time window. 
The $r^2$ score is above $0.96$ at all $t$. 
The period when Zipf's exponent appears to be stable (from $t \approx 350$ to $800$) 
suggests that the system is in a metastable state. 
Only fluctuations that are large enough can bring the system out of this state. }
\label{fig: metastable}
\end{center}
\end{figure}




Furthermore, the rank-size distribution is in a steady state past $t=400$ 
(Fig.~\ref{fig: Zipf1}).  
To investigate its long-term behavior, 
we plotted the evolution of the mean rank-size exponent over 
a longer period 
(Fig.~\ref{fig: metastable}). 
Note that $r^2$ values score more than $0.96$ at any measured moment. 
We find that in an initial stage (from $t=0$ to $t \approx 350$), 
the exponent $\gamma$ generally increases monotonically 
as a result of aggregation. 
However, 
the rate of increase exhibits a significant slowdown, 
followed by a relatively long period (from $t \approx 350$ to $800$)
during which the exponent remains, which we call a ``Zipf phase.''


As shown in Fig.~\ref{fig: metastable}, however, 
the exponent starts growing again
after the Zipf phase, 
especially with a characteristic sudden increase at around $t=1000$. 
During and after the Zipf phase, 
even established clusters constantly fluctuate 
owing
to the random force in Eq.~(\ref{eq: C1}), 
and smaller clusters potentially merge into nearby larger ones,  
which causes a sudden increment in the value of the exponent. 
Therefore, we suggest that Zipf's law does not necessarily 
represent 
an equilibrium 
(as usually discussed in the literature \cite{gabaix2004evolution, hsu2008central, arshad2018zipf}) 
but rather 
a \emph{metastable} state 
with large fluctuations that might 
free 
the society from being trapped.  
Although the exact time and value of the exponent at which 
noticeable 
slowdowns in the rate of increase 
occur 
vary  
from time to time 
(due to different initial conditions and randomness in the motion of agents), 
such declines and the consequent metastability of the aforementioned phenomenon 
appears typical in most cases.
Our discussion on metastability may provide a clue as to 
why many urban systems have 
their 
rank-size exponents close to 1 
but 
still 
demonstrate an overall 
increasing tendency, 
as discussed in Sec.~\ref{sec: introduction}. 



\begin{figure}[htbp]
\begin{center}
\includegraphics[width=8.6cm]%[scale=0.19]
{fig6.png}
\caption{
Anti-correlation between changes in the unemployment rate 
and corresponding growth rates in the production output obtained from the same data as in Fig.~\ref{fig: aggregation1} 
(from $t=0$ through $t=600$). 
The plotted data (circles) are regressed linearly (solid line). }
\label{fig: Okun1}
\end{center}
\end{figure}

To understand how labor mobility relates to economic circumstances in our model, 
we investigated the relationship between 
changes in the unemployment rate and growth rates in the total output (Fig.~\ref{fig: Okun1}). 
In the figure, 
data are plotted in the $\Delta \mu - \Delta Y_{tot}/Y_{tot}$ plane, 
where 
each dot represents the relationship between these two variables at each time step. 
The data demonstrate a clear anti-correlation between those two variables, 
and the regressed line has a slope of $-2.20$ with an $r^2$ score of $0.81$. 
For example, the reproduced value of the slope is highly reminiscent of 
Okun's coefficient 
in the U.S. economy, 
which is estimated to be around $2$, 
as mentioned in Sec.~\ref{sec: introduction}. 
Traditional theoretical explanations of the law tend to overestimate the coefficient \cite{ball2013okun}. 
The ability of our model to reproduce a reasonable coefficient may be  
indicative of its proper underlying mechanisms. 


\subsection{\label{subsec: econotaxis} Econotaxis}


In Sec.~\ref{subsec: aggregation}, 
we stated that our model reproduces aggregations of agents.  
It is 
significantly characterized 
by the orientational movements of unemployed agents 
in response to their potential socio-economic advantages. 
This process can be understood  
as a positive \emph{taxis} that emerges in social contexts. 
Taxis is often referred to, in biology literature, as a behavioral response 
in which an organism directs its movement in response to an external stimulus 
\cite{stevens1997aggregation}. 
Mathematically, 
it is described by 
a biased random walk \cite{grunbaum1998schooling}. 
In our model, 
the characteristic movement of unemployed agents 
is particularly 
initiated by a local economic activity in which employed agents are engaged. 
Therefore, we call this positive feedback process involved in our model 
``econotaxis.''  
Note that econotaxis is 
 in line with 
 the idea of self-reinforcement in urbanization, 
 which some traditional theories have implied but not necessarily modeled explicitly 
 \cite{boddy1999geographical, krugman1991increasing, krugman1992dynamic, ottaviano2004agglomeration}. 



Furthermore, we can derive the following Fokker--Planck equations from our microscopic model
 \cite{risken1996fokker}: 
\begin{align}
n_t  &= D_n  \, \nabla^2 n - \nabla \cdot \left( n f \right)  - k^+ n + k^- l,  \label{eq: nt} \\  
l_t  &= D_l  \, \nabla^2 l + k^+ n - k^- l,  \label{eq: lt}
\end{align}
where 
the densities of unemployed and employed agents are 
denoted by $n(r,t)$ and $l(r,t)$, respectively.
The aforementioned macroscopic system indicates that our model reproduces the agglomeration of agents 
 in a manner that is fundamentally similar to 
 how the classical Keller--Segel system \cite{keller1970initiation, horstmann20031970} 
initiates 
aggregation phenomena called 
\emph{chemotaxis}, 
 a popular type of taxis by which certain chemicals affect cell migration. 
The Keller--Segel system is composed of two 
variables: $u, v$, 
the density of cells 
and the concentration of chemicals, given by 
\begin{align}
u_t &= d_u \nabla^2 u -  \nabla \cdot \left( u \, \nabla \chi (v) \right), \label{eq: ksu}
 \\
v_t &= d_v \nabla^2 v  + a u - b v , \label{eq: ksv}
\end{align}
where 
$\chi(v)$ is called a sensitivity function that determines how migrating cells 
react to the chemoattractants. 

The 
connection of 
Eqs.~(\ref{eq: ksu}) and (\ref{eq: ksv}) to Eqs.~(\ref{eq: nt}) and (\ref{eq: lt}) 
in terms of aggregation 
is apparent, 
which 
allows us to make 
at least two further arguments. 
First, 
the previously described investigation  
offers an interesting perspective on the nature of the labor force in general. 
Comparing econotaxis with chemotaxis, 
employed and unemployed agents can be identified 
as
chemoattractants and cells, respectively. 
This suggests that although workers and jobless agents may sound like complete opposites, 
they play complementary roles during the aggregation process: 
employed agents function as ``attractants,'' helping the unemployed navigate the world 
to enable them to live off better employment opportunities. 
In other words, 
unemployed agents receive from employed agents a ``signal'' of benefit for their survival 
(i.e., employment opportunities)
and respond to it while they constantly switch roles. 
Their well-coordinated and intertwined relationship 
amplifies the initial small 
heterogeneity and 
allows certain places to come out as
highly populated. 
In this manner, our model suggests urban areas may emerge out of nowhere in human society.


Second, a vast amount of analytic investigation has revealed that 
many (often biological) chemotactic behaviors modeled 
by systems such as
Eqs.~(\ref{eq: ksu})~and~(\ref{eq: ksv}) 
can be understood mathematically
as 
blow-up phenomena 
\cite{horstmann20031970, hillen2009user, arumugam2021keller}.  
This indicates that 
econotaxis 
is also related to 
blow-up phenomena 
in the system of 
Eqs.~(\ref{eq: nt})~and~(\ref{eq: lt}). 
From a sociological perspective, it may 
provide a potential 
answer to a long-standing question 
in urban science: 
Is there a saturation point in urbanization (i.e., an ``urban maturity'')? 
\cite{tisdale1941process, berry2008urbanization}. 
The social science literature states that whether the process reaches equilibrium is still unknown, 
even though some are in favor of the idea 
of saturation 
\cite{tisdale1941process, berry2008urbanization, fujita2013economics}. 
Our numerical results 
with the new concept of econotaxis 
suggest otherwise: 
urbanization, thus, 
can be a relentless process. 



\section{\label{sec: discussion} Discussion \& conclusions}



In Secs.~\ref{sec: framework}~and~\ref{subsec: aggregation}, 
we developed a simple model of labor migration and 
demonstrated that agglomeration is an inherent characteristic. 
That might cast doubt on the major economic theory 
that 
increasing returns to scale (i.e., $\beta > 1$ in Eq.~(\ref{eq: normal production function})) 
is 
a prerequisite 
for urban agglomeration 
\cite{fujita1996economics, duranton2004micro}. 
Instead, our model suggests that 
a straightforward concave production function 
can cause agglomeration.
In addition, the results in Sec.~\ref{subsec: Zipf} imply that  
the two well-known regularities may not arise from 
specific details of social or economic interactions 
but rather 
from the mechanism underlying agglomeration. 
 The relationship between aggregation and transition probabilities is 
 yet to be fully uncovered, 
 which is 
 worthy of further investigation. 
 Based on our initial observation 
 in Fig.~\ref{fig: no aggregation bc of imbalance} and Sec.~\ref{subsec: aggregation},  
 aggregation behavior appears to be most facilitated 
 when $k^+ \approx k^-$. 
 Although this speculation is appropriate in the short run (until $t \approx 200$), 
 it is, in fact, not the case in the long run, especially after agents form clusters for the first time, 
and this is currently being investigated. 




In summary, 
we have presented a simple microscopic model of urbanization 
to reveal 
the underlying 
fundamental 
mechanism. 
Despite its simplicity, 
the model initiated the aggregation of agents 
and reproduced some statistical regularities 
in human societies and economies. 
We then proposed a novel concept called econotaxis 
to characterize the self-assembly of the labor force. 
We also 
revealed the complementary roles played by employed and unemployed agents 
during the process of aggregation. 
Our model provides a unified theoretical approach to the study of urbanization 
and 
insight into the possibility that 
macroscopic agglomeration phenomena and 
Zipf's and Okun's laws
lead to the same underlying mechanism, 
which has rarely been discussed before.  





In future, 
the present model may even be able to reproduce 
some of the unique features of urban agglomeration, 
such as the U.S. manufacturing belt or Europe's hot banana \cite{krugman2011new},  
just by introducing geographical heterogeneity or spatial constraints. 
Other possible ways to extend our model include 
incorporating ``congestion effects'' \cite{berry2008urbanization, boddy1999geographical} 
(i.e., effects of increase in dispersion as the place gets more crowded), 
considering an in- and outflux of population or intrinsic population growth 
and explicitly considering multiple industries. 
A possible limitation, however, is that the continuous models $(\ref{eq: nt})$~and~$(\ref{eq: lt})$ 
are currently not likely to reproduce Zipf's and Okun's laws, 
as blow-up phenomena are expected to occur. 
Nonetheless, 
our model 
essentially 
replicates agglomeration phenomena 
as well as some empirically observed 
regularities 
while keeping itself mathematically tractable. 
We believe that 
it captures the underlying mechanism of urbanization at the most fundamental level. 
We hope 
that it will be a window into 
more complex structures and organizations that coordinate 
our society, 
encouraging theoretical and empirical 
research on social dynamics.




\section{\label{sec: acknowledgement} Acknowledgements}
I 
would like to thank Prof. Kota Ikeda (Meiji University, Japan) for insightful discussions 
and his comments on the manuscript. 
I am 
also grateful to 
Dr. Ryu Fujiwara (Meiji University) and 
Prof. Hiraku Nishimori (Meiji Institute for Advanced Study of Mathematical Sciences, Japan) 
for their helpful comments. 
I would like to thank Editage (www.editage.jp) for English language editing.


%\bibliography{ref}



%merlin.mbs apsrev4-1.bst 2010-07-25 4.21a (PWD, AO, DPC) hacked
%Control: key (0)
%Control: author (8) initials jnrlst
%Control: editor formatted (1) identically to author
%Control: production of article title (-1) disabled
%Control: page (0) single
%Control: year (1) truncated
%Control: production of eprint (0) enabled
\begin{thebibliography}{56}%
\makeatletter
\providecommand \@ifxundefined [1]{%
 \@ifx{#1\undefined}
}%
\providecommand \@ifnum [1]{%
 \ifnum #1\expandafter \@firstoftwo
 \else \expandafter \@secondoftwo
 \fi
}%
\providecommand \@ifx [1]{%
 \ifx #1\expandafter \@firstoftwo
 \else \expandafter \@secondoftwo
 \fi
}%
\providecommand \natexlab [1]{#1}%
\providecommand \enquote  [1]{``#1''}%
\providecommand \bibnamefont  [1]{#1}%
\providecommand \bibfnamefont [1]{#1}%
\providecommand \citenamefont [1]{#1}%
\providecommand \href@noop [0]{\@secondoftwo}%
\providecommand \href [0]{\begingroup \@sanitize@url \@href}%
\providecommand \@href[1]{\@@startlink{#1}\@@href}%
\providecommand \@@href[1]{\endgroup#1\@@endlink}%
\providecommand \@sanitize@url [0]{\catcode `\\12\catcode `\$12\catcode
  `\&12\catcode `\#12\catcode `\^12\catcode `\_12\catcode `\%12\relax}%
\providecommand \@@startlink[1]{}%
\providecommand \@@endlink[0]{}%
\providecommand \url  [0]{\begingroup\@sanitize@url \@url }%
\providecommand \@url [1]{\endgroup\@href {#1}{\urlprefix }}%
\providecommand \urlprefix  [0]{URL }%
\providecommand \Eprint [0]{\href }%
\providecommand \doibase [0]{http://dx.doi.org/}%
\providecommand \selectlanguage [0]{\@gobble}%
\providecommand \bibinfo  [0]{\@secondoftwo}%
\providecommand \bibfield  [0]{\@secondoftwo}%
\providecommand \translation [1]{[#1]}%
\providecommand \BibitemOpen [0]{}%
\providecommand \bibitemStop [0]{}%
\providecommand \bibitemNoStop [0]{.\EOS\space}%
\providecommand \EOS [0]{\spacefactor3000\relax}%
\providecommand \BibitemShut  [1]{\csname bibitem#1\endcsname}%
\let\auto@bib@innerbib\@empty
%</preamble>
\bibitem [{\citenamefont {Tisdale}(1941)}]{tisdale1941process}%
  \BibitemOpen
  \bibfield  {author} {\bibinfo {author} {\bibfnamefont {H.}~\bibnamefont
  {Tisdale}},\ }\href@noop {} {\bibfield  {journal} {\bibinfo  {journal} {Soc.
  F.}\ }\textbf {\bibinfo {volume} {20}},\ \bibinfo {pages} {311} (\bibinfo
  {year} {1941})}\BibitemShut {NoStop}%
\bibitem [{\citenamefont {Berry}(2008)}]{berry2008urbanization}%
  \BibitemOpen
  \bibfield  {author} {\bibinfo {author} {\bibfnamefont {B.~J.}\ \bibnamefont
  {Berry}},\ }in\ \href@noop {} {\emph {\bibinfo {booktitle} {Urban ecology}}}\
  (\bibinfo  {publisher} {Springer},\ \bibinfo {year} {2008})\ pp.\ \bibinfo
  {pages} {25--48}\BibitemShut {NoStop}%
\bibitem [{\citenamefont {Boddy}(1999)}]{boddy1999geographical}%
  \BibitemOpen
  \bibfield  {author} {\bibinfo {author} {\bibfnamefont {M.}~\bibnamefont
  {Boddy}},\ }\href@noop {} {\bibfield  {journal} {\bibinfo  {journal} {Urban
  studies}\ }\textbf {\bibinfo {volume} {36}},\ \bibinfo {pages} {811}
  (\bibinfo {year} {1999})}\BibitemShut {NoStop}%
\bibitem [{\citenamefont {Ottaviano}\ and\ \citenamefont
  {Thisse}(2004)}]{ottaviano2004agglomeration}%
  \BibitemOpen
  \bibfield  {author} {\bibinfo {author} {\bibfnamefont {G.}~\bibnamefont
  {Ottaviano}}\ and\ \bibinfo {author} {\bibfnamefont {J.-F.}\ \bibnamefont
  {Thisse}},\ }in\ \href@noop {} {\emph {\bibinfo {booktitle} {Handbook of
  regional and urban economics}}},\ Vol.~\bibinfo {volume} {4}\ (\bibinfo
  {publisher} {Elsevier},\ \bibinfo {year} {2004})\ pp.\ \bibinfo {pages}
  {2563--2608}\BibitemShut {NoStop}%
\bibitem [{\citenamefont {Fujita}\ and\ \citenamefont
  {Thisse}(2002)}]{fujita2002agglomeration}%
  \BibitemOpen
  \bibfield  {author} {\bibinfo {author} {\bibfnamefont {M.}~\bibnamefont
  {Fujita}}\ and\ \bibinfo {author} {\bibfnamefont {J.-F.}\ \bibnamefont
  {Thisse}},\ }\href@noop {} {\bibfield  {journal} {\bibinfo  {journal}
  {Available at SSRN 315966}\ } (\bibinfo {year} {2002})}\BibitemShut {NoStop}%
\bibitem [{\citenamefont {Fujita}\ and\ \citenamefont
  {Thisse}(1996)}]{fujita1996economics}%
  \BibitemOpen
  \bibfield  {author} {\bibinfo {author} {\bibfnamefont {M.}~\bibnamefont
  {Fujita}}\ and\ \bibinfo {author} {\bibfnamefont {J.-F.}\ \bibnamefont
  {Thisse}},\ }\href@noop {} {\bibfield  {journal} {\bibinfo  {journal}
  {Journal of the Japanese and international economies}\ }\textbf {\bibinfo
  {volume} {10}},\ \bibinfo {pages} {339} (\bibinfo {year} {1996})}\BibitemShut
  {NoStop}%
\bibitem [{\citenamefont {Brakman}\ and\ \citenamefont
  {Garretsen}(2003)}]{brakman2003rethinking}%
  \BibitemOpen
  \bibfield  {author} {\bibinfo {author} {\bibfnamefont {S.}~\bibnamefont
  {Brakman}}\ and\ \bibinfo {author} {\bibfnamefont {H.}~\bibnamefont
  {Garretsen}},\ }\href@noop {} {\bibfield  {journal} {\bibinfo  {journal}
  {Regional Studies}\ }\textbf {\bibinfo {volume} {37}},\ \bibinfo {pages}
  {637} (\bibinfo {year} {2003})}\BibitemShut {NoStop}%
\bibitem [{\citenamefont {Grotewold}(1959)}]{grotewold1959thunen}%
  \BibitemOpen
  \bibfield  {author} {\bibinfo {author} {\bibfnamefont {A.}~\bibnamefont
  {Grotewold}},\ }\href@noop {} {\bibfield  {journal} {\bibinfo  {journal}
  {Economic Geography}\ }\textbf {\bibinfo {volume} {35}},\ \bibinfo {pages}
  {346} (\bibinfo {year} {1959})}\BibitemShut {NoStop}%
\bibitem [{\citenamefont {M{\"a}ki}(2004)}]{maki2004realism}%
  \BibitemOpen
  \bibfield  {author} {\bibinfo {author} {\bibfnamefont {U.}~\bibnamefont
  {M{\"a}ki}},\ }\href@noop {} {\bibfield  {journal} {\bibinfo  {journal}
  {Environment and Planning A}\ }\textbf {\bibinfo {volume} {36}},\ \bibinfo
  {pages} {1719} (\bibinfo {year} {2004})}\BibitemShut {NoStop}%
\bibitem [{\citenamefont {Isard}(1949)}]{isard1949general}%
  \BibitemOpen
  \bibfield  {author} {\bibinfo {author} {\bibfnamefont {W.}~\bibnamefont
  {Isard}},\ }\href@noop {} {\bibfield  {journal} {\bibinfo  {journal} {The
  Quarterly Journal of Economics}\ }\textbf {\bibinfo {volume} {63}},\ \bibinfo
  {pages} {476} (\bibinfo {year} {1949})}\BibitemShut {NoStop}%
\bibitem [{\citenamefont {Capello}(2014)}]{capello2014classical}%
  \BibitemOpen
  \bibfield  {author} {\bibinfo {author} {\bibfnamefont {R.}~\bibnamefont
  {Capello}},\ }in\ \href@noop {} {\emph {\bibinfo {booktitle} {Handbook of
  regional science}}}\ (\bibinfo  {publisher} {Springer},\ \bibinfo {year}
  {2014})\ pp.\ \bibinfo {pages} {507--526}\BibitemShut {NoStop}%
\bibitem [{\citenamefont {King}(2020)}]{king2020central}%
  \BibitemOpen
  \bibfield  {author} {\bibinfo {author} {\bibfnamefont {L.~J.}\ \bibnamefont
  {King}},\ }\href@noop {} {\  (\bibinfo {year} {2020})}\BibitemShut {NoStop}%
\bibitem [{\citenamefont {Mulligan}(1984)}]{mulligan1984agglomeration}%
  \BibitemOpen
  \bibfield  {author} {\bibinfo {author} {\bibfnamefont {G.~F.}\ \bibnamefont
  {Mulligan}},\ }\href@noop {} {\bibfield  {journal} {\bibinfo  {journal}
  {International Regional Science Review}\ }\textbf {\bibinfo {volume} {9}},\
  \bibinfo {pages} {1} (\bibinfo {year} {1984})}\BibitemShut {NoStop}%
\bibitem [{\citenamefont {Martin}(1999)}]{martin1999new}%
  \BibitemOpen
  \bibfield  {author} {\bibinfo {author} {\bibfnamefont {R.}~\bibnamefont
  {Martin}},\ }\href@noop {} {\bibfield  {journal} {\bibinfo  {journal}
  {Transactions of the Institute of British Geographers}\ }\textbf {\bibinfo
  {volume} {24}},\ \bibinfo {pages} {387} (\bibinfo {year} {1999})}\BibitemShut
  {NoStop}%
\bibitem [{\citenamefont {Glaeser}(2010)}]{glaeser2010agglomeration}%
  \BibitemOpen
  \bibfield  {author} {\bibinfo {author} {\bibfnamefont {E.~L.}\ \bibnamefont
  {Glaeser}},\ }\href@noop {} {\emph {\bibinfo {title} {Agglomeration
  economics}}}\ (\bibinfo  {publisher} {University of Chicago Press},\ \bibinfo
  {year} {2010})\BibitemShut {NoStop}%
\bibitem [{\citenamefont {Puga}(2010)}]{puga2010magnitude}%
  \BibitemOpen
  \bibfield  {author} {\bibinfo {author} {\bibfnamefont {D.}~\bibnamefont
  {Puga}},\ }\href@noop {} {\bibfield  {journal} {\bibinfo  {journal} {Journal
  of regional science}\ }\textbf {\bibinfo {volume} {50}},\ \bibinfo {pages}
  {203} (\bibinfo {year} {2010})}\BibitemShut {NoStop}%
\bibitem [{\citenamefont {Duranton}\ and\ \citenamefont
  {Puga}(2004)}]{duranton2004micro}%
  \BibitemOpen
  \bibfield  {author} {\bibinfo {author} {\bibfnamefont {G.}~\bibnamefont
  {Duranton}}\ and\ \bibinfo {author} {\bibfnamefont {D.}~\bibnamefont
  {Puga}},\ }in\ \href@noop {} {\emph {\bibinfo {booktitle} {Handbook of
  regional and urban economics}}},\ Vol.~\bibinfo {volume} {4}\ (\bibinfo
  {publisher} {Elsevier},\ \bibinfo {year} {2004})\ pp.\ \bibinfo {pages}
  {2063--2117}\BibitemShut {NoStop}%
\bibitem [{\citenamefont {Krugman}(1992)}]{krugman1992dynamic}%
  \BibitemOpen
  \bibfield  {author} {\bibinfo {author} {\bibfnamefont {P.}~\bibnamefont
  {Krugman}},\ }\href@noop {} {\emph {\bibinfo {title} {A dynamic spatial
  model}}},\ \bibinfo {type} {Tech. Rep.}\ (\bibinfo  {institution} {National
  Bureau of Economic Research},\ \bibinfo {year} {1992})\BibitemShut {NoStop}%
\bibitem [{\citenamefont {Scott}(2006)}]{scott2006perspective}%
  \BibitemOpen
  \bibfield  {author} {\bibinfo {author} {\bibfnamefont {A.~J.}\ \bibnamefont
  {Scott}},\ }in\ \href@noop {} {\emph {\bibinfo {booktitle} {Economic
  Geography}}}\ (\bibinfo  {publisher} {Routledge},\ \bibinfo {year} {2006})\
  pp.\ \bibinfo {pages} {78--102}\BibitemShut {NoStop}%
\bibitem [{\citenamefont {Castellano}\ \emph {et~al.}(2009)\citenamefont
  {Castellano}, \citenamefont {Fortunato},\ and\ \citenamefont
  {Loreto}}]{castellano2009statistical}%
  \BibitemOpen
  \bibfield  {author} {\bibinfo {author} {\bibfnamefont {C.}~\bibnamefont
  {Castellano}}, \bibinfo {author} {\bibfnamefont {S.}~\bibnamefont
  {Fortunato}}, \ and\ \bibinfo {author} {\bibfnamefont {V.}~\bibnamefont
  {Loreto}},\ }\href@noop {} {\bibfield  {journal} {\bibinfo  {journal}
  {Reviews of modern physics}\ }\textbf {\bibinfo {volume} {81}},\ \bibinfo
  {pages} {591} (\bibinfo {year} {2009})}\BibitemShut {NoStop}%
\bibitem [{\citenamefont {Gonzalez}\ \emph {et~al.}(2008)\citenamefont
  {Gonzalez}, \citenamefont {Hidalgo},\ and\ \citenamefont
  {Barabasi}}]{gonzalez2008understanding}%
  \BibitemOpen
  \bibfield  {author} {\bibinfo {author} {\bibfnamefont {M.~C.}\ \bibnamefont
  {Gonzalez}}, \bibinfo {author} {\bibfnamefont {C.~A.}\ \bibnamefont
  {Hidalgo}}, \ and\ \bibinfo {author} {\bibfnamefont {A.-L.}\ \bibnamefont
  {Barabasi}},\ }\href@noop {} {\bibfield  {journal} {\bibinfo  {journal}
  {nature}\ }\textbf {\bibinfo {volume} {453}},\ \bibinfo {pages} {779}
  (\bibinfo {year} {2008})}\BibitemShut {NoStop}%
\bibitem [{\citenamefont {Schl{\"a}pfer}\ \emph {et~al.}(2021)\citenamefont
  {Schl{\"a}pfer}, \citenamefont {Dong}, \citenamefont {O'Keeffe},
  \citenamefont {Santi}, \citenamefont {Szell}, \citenamefont {Salat},
  \citenamefont {Anklesaria}, \citenamefont {Vazifeh}, \citenamefont {Ratti},\
  and\ \citenamefont {West}}]{schlapfer2021universal}%
  \BibitemOpen
  \bibfield  {author} {\bibinfo {author} {\bibfnamefont {M.}~\bibnamefont
  {Schl{\"a}pfer}}, \bibinfo {author} {\bibfnamefont {L.}~\bibnamefont {Dong}},
  \bibinfo {author} {\bibfnamefont {K.}~\bibnamefont {O'Keeffe}}, \bibinfo
  {author} {\bibfnamefont {P.}~\bibnamefont {Santi}}, \bibinfo {author}
  {\bibfnamefont {M.}~\bibnamefont {Szell}}, \bibinfo {author} {\bibfnamefont
  {H.}~\bibnamefont {Salat}}, \bibinfo {author} {\bibfnamefont
  {S.}~\bibnamefont {Anklesaria}}, \bibinfo {author} {\bibfnamefont
  {M.}~\bibnamefont {Vazifeh}}, \bibinfo {author} {\bibfnamefont
  {C.}~\bibnamefont {Ratti}}, \ and\ \bibinfo {author} {\bibfnamefont {G.~B.}\
  \bibnamefont {West}},\ }\href@noop {} {\bibfield  {journal} {\bibinfo
  {journal} {Nature}\ }\textbf {\bibinfo {volume} {593}},\ \bibinfo {pages}
  {522} (\bibinfo {year} {2021})}\BibitemShut {NoStop}%
\bibitem [{\citenamefont {Shida}\ \emph {et~al.}(2020)\citenamefont {Shida},
  \citenamefont {Takayasu}, \citenamefont {Havlin},\ and\ \citenamefont
  {Takayasu}}]{shida2020universal}%
  \BibitemOpen
  \bibfield  {author} {\bibinfo {author} {\bibfnamefont {Y.}~\bibnamefont
  {Shida}}, \bibinfo {author} {\bibfnamefont {H.}~\bibnamefont {Takayasu}},
  \bibinfo {author} {\bibfnamefont {S.}~\bibnamefont {Havlin}}, \ and\ \bibinfo
  {author} {\bibfnamefont {M.}~\bibnamefont {Takayasu}},\ }\href@noop {}
  {\bibfield  {journal} {\bibinfo  {journal} {Scientific reports}\ }\textbf
  {\bibinfo {volume} {10}},\ \bibinfo {pages} {1} (\bibinfo {year}
  {2020})}\BibitemShut {NoStop}%
\bibitem [{\citenamefont {Schweitzer}(2002)}]{schweitzer2002modelling}%
  \BibitemOpen
  \bibfield  {author} {\bibinfo {author} {\bibfnamefont {F.}~\bibnamefont
  {Schweitzer}},\ }in\ \href@noop {} {\emph {\bibinfo {booktitle} {Modeling
  complexity in economic and social systems}}}\ (\bibinfo  {publisher} {World
  Scientific},\ \bibinfo {year} {2002})\ pp.\ \bibinfo {pages}
  {137--159}\BibitemShut {NoStop}%
\bibitem [{\citenamefont {Schweitzer}(2003)}]{schweitzer2003brownian}%
  \BibitemOpen
  \bibfield  {author} {\bibinfo {author} {\bibfnamefont {F.}~\bibnamefont
  {Schweitzer}},\ }\href@noop {} {\emph {\bibinfo {title} {Brownian agents and
  active particles: collective dynamics in the natural and social sciences}}}\
  (\bibinfo  {publisher} {Springer Science \& Business Media},\ \bibinfo {year}
  {2003})\BibitemShut {NoStop}%
\bibitem [{\citenamefont {Romanczuk}\ \emph {et~al.}(2012)\citenamefont
  {Romanczuk}, \citenamefont {B{\"a}r}, \citenamefont {Ebeling}, \citenamefont
  {Lindner},\ and\ \citenamefont {Schimansky-Geier}}]{romanczuk2012active}%
  \BibitemOpen
  \bibfield  {author} {\bibinfo {author} {\bibfnamefont {P.}~\bibnamefont
  {Romanczuk}}, \bibinfo {author} {\bibfnamefont {M.}~\bibnamefont {B{\"a}r}},
  \bibinfo {author} {\bibfnamefont {W.}~\bibnamefont {Ebeling}}, \bibinfo
  {author} {\bibfnamefont {B.}~\bibnamefont {Lindner}}, \ and\ \bibinfo
  {author} {\bibfnamefont {L.}~\bibnamefont {Schimansky-Geier}},\ }\href@noop
  {} {\bibfield  {journal} {\bibinfo  {journal} {The European Physical Journal
  Special Topics}\ }\textbf {\bibinfo {volume} {202}},\ \bibinfo {pages} {1}
  (\bibinfo {year} {2012})}\BibitemShut {NoStop}%
\bibitem [{\citenamefont {Schimansky-Geier}\ \emph {et~al.}(1995)\citenamefont
  {Schimansky-Geier}, \citenamefont {Mieth}, \citenamefont {Ros{\'e}},\ and\
  \citenamefont {Malchow}}]{schimansky1995structure}%
  \BibitemOpen
  \bibfield  {author} {\bibinfo {author} {\bibfnamefont {L.}~\bibnamefont
  {Schimansky-Geier}}, \bibinfo {author} {\bibfnamefont {M.}~\bibnamefont
  {Mieth}}, \bibinfo {author} {\bibfnamefont {H.}~\bibnamefont {Ros{\'e}}}, \
  and\ \bibinfo {author} {\bibfnamefont {H.}~\bibnamefont {Malchow}},\
  }\href@noop {} {\bibfield  {journal} {\bibinfo  {journal} {Physics Letters
  A}\ }\textbf {\bibinfo {volume} {207}},\ \bibinfo {pages} {140} (\bibinfo
  {year} {1995})}\BibitemShut {NoStop}%
\bibitem [{\citenamefont {Helbing}(2001)}]{helbing2001traffic}%
  \BibitemOpen
  \bibfield  {author} {\bibinfo {author} {\bibfnamefont {D.}~\bibnamefont
  {Helbing}},\ }\href@noop {} {\bibfield  {journal} {\bibinfo  {journal}
  {Reviews of modern physics}\ }\textbf {\bibinfo {volume} {73}},\ \bibinfo
  {pages} {1067} (\bibinfo {year} {2001})}\BibitemShut {NoStop}%
\bibitem [{\citenamefont {Hinloopen}\ and\ \citenamefont {van
  Marrewijk}(2006)}]{hinloopen2006comparative}%
  \BibitemOpen
  \bibfield  {author} {\bibinfo {author} {\bibfnamefont {J.}~\bibnamefont
  {Hinloopen}}\ and\ \bibinfo {author} {\bibfnamefont {C.}~\bibnamefont {van
  Marrewijk}},\ }\href@noop {} {\emph {\bibinfo {title} {Comparative advantage,
  the rank-size rule, and Zipf's law}}},\ \bibinfo {type} {Tech. Rep.}\
  (\bibinfo  {institution} {Tinbergen Institute Discussion Paper},\ \bibinfo
  {year} {2006})\BibitemShut {NoStop}%
\bibitem [{\citenamefont {Rose}(2005)}]{rose2005cities}%
  \BibitemOpen
  \bibfield  {author} {\bibinfo {author} {\bibfnamefont {A.~K.}\ \bibnamefont
  {Rose}},\ }\href@noop {} {\enquote {\bibinfo {title} {Cities and
  countries},}\ } (\bibinfo {year} {2005})\BibitemShut {NoStop}%
\bibitem [{\citenamefont {Gabaix}\ and\ \citenamefont
  {Ioannides}(2004)}]{gabaix2004evolution}%
  \BibitemOpen
  \bibfield  {author} {\bibinfo {author} {\bibfnamefont {X.}~\bibnamefont
  {Gabaix}}\ and\ \bibinfo {author} {\bibfnamefont {Y.~M.}\ \bibnamefont
  {Ioannides}},\ }in\ \href@noop {} {\emph {\bibinfo {booktitle} {Handbook of
  regional and urban economics}}},\ Vol.~\bibinfo {volume} {4}\ (\bibinfo
  {publisher} {Elsevier},\ \bibinfo {year} {2004})\ pp.\ \bibinfo {pages}
  {2341--2378}\BibitemShut {NoStop}%
\bibitem [{\citenamefont {Veneri}(2016)}]{veneri2016city}%
  \BibitemOpen
  \bibfield  {author} {\bibinfo {author} {\bibfnamefont {P.}~\bibnamefont
  {Veneri}},\ }\href@noop {} {\bibfield  {journal} {\bibinfo  {journal}
  {Computers, Environment and Urban Systems}\ }\textbf {\bibinfo {volume}
  {59}},\ \bibinfo {pages} {86} (\bibinfo {year} {2016})}\BibitemShut {NoStop}%
\bibitem [{\citenamefont {Reed}(2002)}]{reed2002rank}%
  \BibitemOpen
  \bibfield  {author} {\bibinfo {author} {\bibfnamefont {W.~J.}\ \bibnamefont
  {Reed}},\ }\href@noop {} {\bibfield  {journal} {\bibinfo  {journal} {Journal
  of Regional Science}\ }\textbf {\bibinfo {volume} {42}},\ \bibinfo {pages}
  {1} (\bibinfo {year} {2002})}\BibitemShut {NoStop}%
\bibitem [{\citenamefont {Gabaix}(1999)}]{gabaix1999zipf}%
  \BibitemOpen
  \bibfield  {author} {\bibinfo {author} {\bibfnamefont {X.}~\bibnamefont
  {Gabaix}},\ }\href@noop {} {\bibfield  {journal} {\bibinfo  {journal} {The
  Quarterly journal of economics}\ }\textbf {\bibinfo {volume} {114}},\
  \bibinfo {pages} {739} (\bibinfo {year} {1999})}\BibitemShut {NoStop}%
\bibitem [{\citenamefont {Simon}(1955)}]{simon1955class}%
  \BibitemOpen
  \bibfield  {author} {\bibinfo {author} {\bibfnamefont {H.~A.}\ \bibnamefont
  {Simon}},\ }\href@noop {} {\bibfield  {journal} {\bibinfo  {journal}
  {Biometrika}\ }\textbf {\bibinfo {volume} {42}},\ \bibinfo {pages} {425}
  (\bibinfo {year} {1955})}\BibitemShut {NoStop}%
\bibitem [{\citenamefont {Cristelli}\ \emph {et~al.}(2012)\citenamefont
  {Cristelli}, \citenamefont {Batty},\ and\ \citenamefont
  {Pietronero}}]{cristelli2012there}%
  \BibitemOpen
  \bibfield  {author} {\bibinfo {author} {\bibfnamefont {M.}~\bibnamefont
  {Cristelli}}, \bibinfo {author} {\bibfnamefont {M.}~\bibnamefont {Batty}}, \
  and\ \bibinfo {author} {\bibfnamefont {L.}~\bibnamefont {Pietronero}},\
  }\href@noop {} {\bibfield  {journal} {\bibinfo  {journal} {Scientific
  reports}\ }\textbf {\bibinfo {volume} {2}},\ \bibinfo {pages} {1} (\bibinfo
  {year} {2012})}\BibitemShut {NoStop}%
\bibitem [{\citenamefont {Wang}\ and\ \citenamefont
  {Chen}(2021)}]{wang2021economic}%
  \BibitemOpen
  \bibfield  {author} {\bibinfo {author} {\bibfnamefont {J.}~\bibnamefont
  {Wang}}\ and\ \bibinfo {author} {\bibfnamefont {Y.}~\bibnamefont {Chen}},\
  }\href@noop {} {\bibfield  {journal} {\bibinfo  {journal} {Sustainability}\
  }\textbf {\bibinfo {volume} {13}},\ \bibinfo {pages} {3287} (\bibinfo {year}
  {2021})}\BibitemShut {NoStop}%
\bibitem [{\citenamefont {Gu{\'e}rin-Pace}(1995)}]{guerin1995rank}%
  \BibitemOpen
  \bibfield  {author} {\bibinfo {author} {\bibfnamefont {F.}~\bibnamefont
  {Gu{\'e}rin-Pace}},\ }\href@noop {} {\bibfield  {journal} {\bibinfo
  {journal} {Urban studies}\ }\textbf {\bibinfo {volume} {32}},\ \bibinfo
  {pages} {551} (\bibinfo {year} {1995})}\BibitemShut {NoStop}%
\bibitem [{\citenamefont {Okun}(1963)}]{okun1963potential}%
  \BibitemOpen
  \bibfield  {author} {\bibinfo {author} {\bibfnamefont {A.~M.}\ \bibnamefont
  {Okun}},\ }\href@noop {} {\emph {\bibinfo {title} {Potential GNP: its
  measurement and significance}}}\ (\bibinfo  {publisher} {Cowles Foundation
  for Research in Economics at Yale University},\ \bibinfo {year}
  {1963})\BibitemShut {NoStop}%
\bibitem [{\citenamefont {Lee}(2000)}]{lee2000robustness}%
  \BibitemOpen
  \bibfield  {author} {\bibinfo {author} {\bibfnamefont {J.}~\bibnamefont
  {Lee}},\ }\href@noop {} {\bibfield  {journal} {\bibinfo  {journal} {Journal
  of macroeconomics}\ }\textbf {\bibinfo {volume} {22}},\ \bibinfo {pages}
  {331} (\bibinfo {year} {2000})}\BibitemShut {NoStop}%
\bibitem [{\citenamefont {Knotek~II}(2007)}]{knotek2007useful}%
  \BibitemOpen
  \bibfield  {author} {\bibinfo {author} {\bibfnamefont {E.~S.}\ \bibnamefont
  {Knotek~II}},\ }\href@noop {} {\bibfield  {journal} {\bibinfo  {journal}
  {Economic Review-Federal Reserve Bank of Kansas City}\ }\textbf {\bibinfo
  {volume} {92}},\ \bibinfo {pages} {73} (\bibinfo {year} {2007})}\BibitemShut
  {NoStop}%
\bibitem [{\citenamefont {Cuaresma}(2003)}]{cuaresma2003okun}%
  \BibitemOpen
  \bibfield  {author} {\bibinfo {author} {\bibfnamefont {J.~C.}\ \bibnamefont
  {Cuaresma}},\ }\href@noop {} {\bibfield  {journal} {\bibinfo  {journal}
  {Oxford Bulletin of Economics and Statistics}\ }\textbf {\bibinfo {volume}
  {65}},\ \bibinfo {pages} {439} (\bibinfo {year} {2003})}\BibitemShut
  {NoStop}%
\bibitem [{\citenamefont {Ball}\ \emph {et~al.}(2013)\citenamefont {Ball},
  \citenamefont {Leigh},\ and\ \citenamefont {Loungani}}]{ball2013okun}%
  \BibitemOpen
  \bibfield  {author} {\bibinfo {author} {\bibfnamefont {L.~M.}\ \bibnamefont
  {Ball}}, \bibinfo {author} {\bibfnamefont {D.}~\bibnamefont {Leigh}}, \ and\
  \bibinfo {author} {\bibfnamefont {P.}~\bibnamefont {Loungani}},\ }\href@noop
  {} {\emph {\bibinfo {title} {Okun's law: fit at fifty?}}},\ \bibinfo {type}
  {Tech. Rep.}\ (\bibinfo  {institution} {National Bureau of Economic
  Research},\ \bibinfo {year} {2013})\BibitemShut {NoStop}%
\bibitem [{\citenamefont {Fujita}\ and\ \citenamefont
  {Thisse}()}]{fujita2013economics}%
  \BibitemOpen
  \bibfield  {author} {\bibinfo {author} {\bibfnamefont {M.}~\bibnamefont
  {Fujita}}\ and\ \bibinfo {author} {\bibfnamefont {J.-F.}\ \bibnamefont
  {Thisse}},\ }\href@noop {} {\emph {\bibinfo {title} {Economics of
  Agglomeration: Cities, Industrial Location, and Globalization}}}\BibitemShut
  {NoStop}%
\bibitem [{\citenamefont {Nutter~Jr}(2010)}]{nutter2010weber}%
  \BibitemOpen
  \bibfield  {author} {\bibinfo {author} {\bibfnamefont {F.~W.}\ \bibnamefont
  {Nutter~Jr}},\ }\href@noop {} {\bibfield  {journal} {\bibinfo  {journal}
  {Encyclopedia of research design}\ }\textbf {\bibinfo {volume} {3}},\
  \bibinfo {pages} {1612} (\bibinfo {year} {2010})}\BibitemShut {NoStop}%
\bibitem [{\citenamefont {Hsu}(2008)}]{hsu2008central}%
  \BibitemOpen
  \bibfield  {author} {\bibinfo {author} {\bibfnamefont {W.-T.}\ \bibnamefont
  {Hsu}},\ }\emph {\bibinfo {title} {Central place theory and Zipf's law}},\
  \href@noop {} {Ph.D. thesis},\ \bibinfo  {school} {Citeseer} (\bibinfo {year}
  {2008})\BibitemShut {NoStop}%
\bibitem [{\citenamefont {Arshad}\ \emph {et~al.}(2018)\citenamefont {Arshad},
  \citenamefont {Hu},\ and\ \citenamefont {Ashraf}}]{arshad2018zipf}%
  \BibitemOpen
  \bibfield  {author} {\bibinfo {author} {\bibfnamefont {S.}~\bibnamefont
  {Arshad}}, \bibinfo {author} {\bibfnamefont {S.}~\bibnamefont {Hu}}, \ and\
  \bibinfo {author} {\bibfnamefont {B.~N.}\ \bibnamefont {Ashraf}},\
  }\href@noop {} {\bibfield  {journal} {\bibinfo  {journal} {Physica A:
  Statistical mechanics and its applications}\ }\textbf {\bibinfo {volume}
  {492}},\ \bibinfo {pages} {75} (\bibinfo {year} {2018})}\BibitemShut
  {NoStop}%
\bibitem [{\citenamefont {Stevens}\ and\ \citenamefont
  {Othmer}(1997)}]{stevens1997aggregation}%
  \BibitemOpen
  \bibfield  {author} {\bibinfo {author} {\bibfnamefont {A.}~\bibnamefont
  {Stevens}}\ and\ \bibinfo {author} {\bibfnamefont {H.~G.}\ \bibnamefont
  {Othmer}},\ }\href@noop {} {\bibfield  {journal} {\bibinfo  {journal} {SIAM
  Journal on Applied Mathematics}\ }\textbf {\bibinfo {volume} {57}},\ \bibinfo
  {pages} {1044} (\bibinfo {year} {1997})}\BibitemShut {NoStop}%
\bibitem [{\citenamefont {Gr{\"u}nbaum}(1998)}]{grunbaum1998schooling}%
  \BibitemOpen
  \bibfield  {author} {\bibinfo {author} {\bibfnamefont {D.}~\bibnamefont
  {Gr{\"u}nbaum}},\ }\href@noop {} {\bibfield  {journal} {\bibinfo  {journal}
  {Evolutionary Ecology}\ }\textbf {\bibinfo {volume} {12}},\ \bibinfo {pages}
  {503} (\bibinfo {year} {1998})}\BibitemShut {NoStop}%
\bibitem [{\citenamefont {Krugman}(1991)}]{krugman1991increasing}%
  \BibitemOpen
  \bibfield  {author} {\bibinfo {author} {\bibfnamefont {P.}~\bibnamefont
  {Krugman}},\ }\href@noop {} {\bibfield  {journal} {\bibinfo  {journal}
  {Journal of political economy}\ }\textbf {\bibinfo {volume} {99}},\ \bibinfo
  {pages} {483} (\bibinfo {year} {1991})}\BibitemShut {NoStop}%
\bibitem [{\citenamefont {Risken}(1996)}]{risken1996fokker}%
  \BibitemOpen
  \bibfield  {author} {\bibinfo {author} {\bibfnamefont {H.}~\bibnamefont
  {Risken}},\ }in\ \href@noop {} {\emph {\bibinfo {booktitle} {The
  Fokker-Planck Equation}}}\ (\bibinfo  {publisher} {Springer},\ \bibinfo
  {year} {1996})\ pp.\ \bibinfo {pages} {63--95}\BibitemShut {NoStop}%
\bibitem [{\citenamefont {Keller}\ and\ \citenamefont
  {Segel}(1970)}]{keller1970initiation}%
  \BibitemOpen
  \bibfield  {author} {\bibinfo {author} {\bibfnamefont {E.~F.}\ \bibnamefont
  {Keller}}\ and\ \bibinfo {author} {\bibfnamefont {L.~A.}\ \bibnamefont
  {Segel}},\ }\href@noop {} {\bibfield  {journal} {\bibinfo  {journal} {Journal
  of theoretical biology}\ }\textbf {\bibinfo {volume} {26}},\ \bibinfo {pages}
  {399} (\bibinfo {year} {1970})}\BibitemShut {NoStop}%
\bibitem [{\citenamefont {Horstmann}(2003)}]{horstmann20031970}%
  \BibitemOpen
  \bibfield  {author} {\bibinfo {author} {\bibfnamefont {D.}~\bibnamefont
  {Horstmann}},\ }\href@noop {} {\  (\bibinfo {year} {2003})}\BibitemShut
  {NoStop}%
\bibitem [{\citenamefont {Hillen}\ and\ \citenamefont
  {Painter}(2009)}]{hillen2009user}%
  \BibitemOpen
  \bibfield  {author} {\bibinfo {author} {\bibfnamefont {T.}~\bibnamefont
  {Hillen}}\ and\ \bibinfo {author} {\bibfnamefont {K.~J.}\ \bibnamefont
  {Painter}},\ }\href@noop {} {\bibfield  {journal} {\bibinfo  {journal}
  {Journal of mathematical biology}\ }\textbf {\bibinfo {volume} {58}},\
  \bibinfo {pages} {183} (\bibinfo {year} {2009})}\BibitemShut {NoStop}%
\bibitem [{\citenamefont {Arumugam}\ and\ \citenamefont
  {Tyagi}(2021)}]{arumugam2021keller}%
  \BibitemOpen
  \bibfield  {author} {\bibinfo {author} {\bibfnamefont {G.}~\bibnamefont
  {Arumugam}}\ and\ \bibinfo {author} {\bibfnamefont {J.}~\bibnamefont
  {Tyagi}},\ }\href@noop {} {\bibfield  {journal} {\bibinfo  {journal} {Acta
  Applicandae Mathematicae}\ }\textbf {\bibinfo {volume} {171}},\ \bibinfo
  {pages} {1} (\bibinfo {year} {2021})}\BibitemShut {NoStop}%
\bibitem [{\citenamefont {Krugman}(2011)}]{krugman2011new}%
  \BibitemOpen
  \bibfield  {author} {\bibinfo {author} {\bibfnamefont {P.}~\bibnamefont
  {Krugman}},\ }\href@noop {} {\bibfield  {journal} {\bibinfo  {journal}
  {Regional studies}\ }\textbf {\bibinfo {volume} {45}},\ \bibinfo {pages} {1}
  (\bibinfo {year} {2011})}\BibitemShut {NoStop}%
\end{thebibliography}%





\end{document}


