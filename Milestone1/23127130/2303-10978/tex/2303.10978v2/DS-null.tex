\documentclass[reprint,aps,prd,twocolumn,notitlepage,nofootinbib,
showpacs,preprintnumbers,superscriptaddress]{revtex4-1}
\usepackage[T1]{fontenc}
\usepackage[utf8]{inputenc}
%\usepackage[a4paper]{geometry}
%\geometry{verbose,tmargin=1.2cm,bmargin=2.5cm,lmargin=1cm,rmargin=1cm}
%\setcounter{secnumdepth}{3}
\usepackage{bm}
\usepackage{amsmath}
\usepackage{amssymb}
\usepackage{esint}
\usepackage{color}
\usepackage{graphicx}% Include figure files
\PassOptionsToPackage{normalem}{ulem}
\usepackage{ulem}

%%%%%%%%%%%%%%%%%%%%%%%%%%%%%% User specified LaTeX commands.
\usepackage{times}
\usepackage{hyperref}
\usepackage{amsthm,bm,mathrsfs,stmaryrd,amsmath}
 

%%%%%%%%%%%%%%%%%%%%%%%%%%%%%% User specified LaTeX commands.


%%%%%%%%%%%%%%%%%%%%%%%%%%

\newcommand{\be}{\begin{equation}}
\newcommand{\ee}{\end{equation}}
\newcommand{\ba}{\begin{eqnarray}}
\newcommand{\ea}{\end{eqnarray}}

\newcommand{\mt}[1]{$\mathop{#1}$}

\newcommand{\sst}{\scriptscriptstyle}

\newcommand{\nn}{\nonumber\\}


\def\pa{\partial}
%%%%%%%%%%%%%%%%% GREEK ALPHABET %%%%%%%%%%%%%%%%%%%%%%%%%%%

\def\a{\alpha}
\def\b{\beta}
\def\g{\gamma}
\def\G{\Gamma}
\def\d{\delta}
\def\D{\Delta}
\def\e{\epsilon}
\def\z{\zeta}
\def\h{\eta}
\def\th{\theta}
\def\Th{\Theta}
\def\i{\iota}
\def\k{\kappa}
\def\l{\lambda}
\def\L{\Lambda}
\def\m{\mu}
\def\n{\nu}
\def\x{\xi}
\def\X{\Xi}
\def\p{\pi}
\def\P{\Pi}
\def\vp{\varphi}
\def\r{\rho}
\def\vr{\varrho}
\def\s{\sigma}
\def\S{\Sigma}
\def\t{\tau}
\def\u{\upsilon}
\def\U{\Upsilon}
\def\f{\phi}
\def\F{\Phi}
\def\c{\chi}
\def\ps{\psi}
\def\Ps{\Psi}
\def\o{\omega}
\def\O{\Omega}
\def\lra{\leftrightarrow}

\def\be{\begin{eqnarray}}
\def\ee{\end{eqnarray}}
\def\D{\Delta}
\def\t{\tau}
\def\a{\alpha}
\def\b{\beta}
\def\m{\mu}
\def\n{\nu}
\def\D{\Delta}
\def\l{\lambda}
\def\g{\gamma}
\def\G{\Gamma}
\def\d{\delta}
\def\nn{\nonumber\\}
\def\pa{\partial}
\def\s{\sigma}
\def\e{\epsilon}
\def\r{\rho}
\newcommand\cO{\mathcal{O}}
\newcommand\<\langle
\renewcommand\>\rangle

\begin{document}

\title{Taming Dyson-Schwinger equations with null states}

\author{Wenliang Li}
\email{liwliang3@mail.sysu.edu.cn}
\affiliation{School of Physics, Sun Yat-Sen University, Guangzhou 510275, China}

\begin{abstract}
In quantum field theory, 
the Dyson-Schwinger equations are an infinite set of coupled equations 
relating $n$-point Green's functions in a self-consistent manner.  
They have found important applications in non-perturbative studies, ranging from 
quantum chromodynamics and hadron physics to strongly correlated electron systems. 
However, they are notoriously formidable to solve. 
One of the main problems is that a finite truncation of the infinite system is underdetermined. 
%as each DS equation involves independent Green's functions. 
Recently,  Bender {\it et al.} [PRL 130, 101602 (2023)] proposed to make use of the large-$n$ asymptotic behaviors 
and successfully obtained accurate results in $D=0$ spacetime.  
At higher $D$, it seems more difficult to deduce the large-$n$ behaviors. 
In this paper, we propose another avenue in light of the null bootstrap. 
The underdetermined system is solved by imposing the null state condition. 
This approach can be extended to $D>0$ more readily. 
As concrete examples, we show that the cases of $D=0$ and $D=1$ indeed converge to the exact results 
for several Hermitian and non-Hermitian theories of the $g\phi^n$ type, including the complex solutions.  
\end{abstract}

\maketitle 
\tableofcontents
\section{Introduction}
In the early stages of quantum field theory, 
the Dyson-Schwinger (DS) equations \cite{Dyson:1949ha,Schwinger:1951ex,Schwinger:1951hq} were designed as an alternative to operator theory. 
They furnish non-perturbative self-consistency relations for the $n$-point Green's functions.  
To make concrete predictions, one usually needs to restrict to a finite subset of DS equations 
for the low-point Green's functions. 
However, this is known to be an {\it underdetermined} system,  
as higher DS equations involve higher-point Green's functions \cite{Bender:1988bp}. 
One needs to introduce additional constraints to solve the DS system. 

A simple scheme is to close the system by setting high-point connected Green's functions to zero. 
However, as emphasized recently in \cite{Bender:2022eze}, the results from this naive procedure do not converge to the exact values. 
The resolution proposed in \cite{Bender:2022eze} is to make use of the asymptotic behaviors of 
the connected Green's functions at large $n$. 
This has been successfully carried out at $D=0$ and the results converge the exact results rapidly, 
but it seems a nontrivial problem to deduce the large $n$ behaviors at $D>0$. 

{\it Can we resolve the indeterminacy of the Dyson-Schwinger equations by a different approach?}

\section{Dyson-Schwinger equations}
We consider the Green's functions of single scalar field
\be
G_n(x_1,\dots,x_n)\equiv
\<T\{\phi(x_1)\dots\phi(x_n)\}\>\,,
\ee
where $T\{\dots\}$ indicates that the operators are time ordered. 
In the expectation value $\<\dots\>=\<\O|\dots|\O\>$, 
the state $|\O\>$ is usually assumed to be a ground state. 
The normalization is set by $\<1\>=1$. 
The Green's functions can be obtained from the generating functional $Z[J]$ 
by taking functional derivatives
\be
G_n(x_1,\dots,x_n)=\frac 1 {Z[0]}\frac{\d^n Z[J]}{\d J(x_1)\dots\d J(x_n)}\Big|_{J\rightarrow 0}\,.
\ee
where $Z[J]$ is defined as a functional integral
\be
Z[J]=\int \mathcal D \phi\, 
e^{-S[\phi]+\int d^Dx J(x)\phi(x)}\,,
\ee
and $S[\phi]=\int d^Dx\,\mathcal L[\phi(x)]$ is the Euclidean action. 
In this work, $G_n$ is the full Green's function, 
instead of the connected Green's function from the derivatives of $\log(Z[J])$. 
An infinitesimal change of the integration variable at $x$ leads to the quantum equation of motion
\be
\<{\d S[\phi]}/{\d\phi(x)}\>=\<J(x)\>\,.
\ee
The infinite set of DS equations can be derived from functional derivatives with respective to the classical source field $J$ and then setting $J$ to zero.

\section{Null state condition}
In this paper, we propose a new approach to resolve the indeterminacy of the DS equations 
based on null states. 
We refer to \cite{Li:2022prn} for the use of the null state condition 
in the bootstrap context, 
which shares a similar spirit with the present work. 
The null states should be orthogonal to arbitrary test states. 
In practice, we consider approximate null states 
that are orthogonal to a subset of test states 
\be
\<\text{test}^{({L})}|\text{null}^{({K})}\>
=\< \mathcal O_{\text{test}}^{(L)} \mathcal O_{\text{null}}^{(K)}\>=0\,.
\label{null-condition}
\ee
The null and test states are generated by the action of null and test operators on $|\O\>$.  
The superscripts $K$ and $L$ denote the numbers of basis operators involved. 
For a given $K$, a well-chosen $L$ can lead to a determined system. 
\footnote{A greater $L$ may lead to an over-determined system, 
but we can still extract the predictions by minimizing the error \cite{Li:2017ukc, Li:2022prn}. }  

Below, we show that the solutions of the DS equations with null state constraints 
converge rapidly to the exact values as $K$ increases. 
They include the complex solutions, 
such as the $\mathcal {PT}$ symmetric ones in non-Hermitian theories 
\cite{Bender:1998ke, Bender:1999ek, Bender:2007nj,Bender:2010hf,r5}. 

\section{Zero-dimensional theories}
The $D=0$ generating functional is given by a more standard integral 
\be
Z[J]=\int d\phi\, e^{-\mathcal L(\phi)+J\phi}\,,
\ee
where the integration path is associated with the choice of Stokes sectors. 
A simple choice of path in Hermitian theories is along the real axis. 
Since there is no time coordinate at $D=0$, 
the Lagrangian $\mathcal L$ has no kinetic term and we can only study the $n$-point Green's function at equal time. 
At $D=0$, we can solve the DS equations without truncating it to a finite subset.  
Below, we consider the Hermitian $\phi^4$ theory, the non-Hermitian $i\phi^3$ theory, 
as well as some higher power counterparts. 

For the $g\phi^n$ theory, the DS equations associated with the Lagrangian 
$\mathcal L(\phi)=\frac{g}{n}\phi^n$ are
\be
g\, G_{n+k}=(k+1)\,G_{k}\,,\quad
k=-1,0,1,\dots.
\ee
The general solution reads
\be
G_{mn+k}=\left(\frac n g\right)^m \left(\frac {k+1}n\right)_m G_k\,,\quad 
G_{n-1}=0\,.
\label{sol-0D}
\ee
where $k=0,1,\dots,n-1$. 
By definition, we have $G_0=1$, so there are $n-2$ free parameters. 
For $n>2$, indeterminacy is an intrinsic feature of the DS equations, 
rather than a consequence of the finite truncation. 
\footnote{In perturbation theory, the DS equations can be solved uniquely order by order due to the constraints from the existence of perturbative series. }

Let us first consider the quartic theory. 
The Lagrangian is $\mathcal L(\phi)=\frac 1 4 \phi^4$. 
The exact two-point Green's function is 
\be
G_2=\pm\frac{ 2 \G(3/4)}{\G(1/4)}=\pm  0.675\, 978\, 240...\,,
\label{phi4-G2}
\ee
where the plus(minus) sign is associated with an integration path along the real(imaginary) axis. 
We would like to derive this pair of solutions from the DS equations and the null state condition. 
Assuming parity symmetry is unbroken, 
we can set $G_{1}=0$, 
so we have only one free parameter, i.e. $G_2$. 

To determine $G_2$, we impose the null state condition \eqref{null-condition} with 
the null and test operators
\be
\mathcal O_{\text{null}}^{(K)}=\sum_{m=0}^K a_m \phi^m\,,\quad
\mathcal O_{\text{test}}^{(L)}=\sum_{m=0}^L b_m\phi^m\,.
\label{n-t-operators}
\ee
The null state condition should be valid for arbitrary $b_m$, 
so we have $(L+1)$ equations. 
The number of free parameters in the null operator $\mathcal O_{\text{null}}^{(K)}$ is $(K+1)$. 
After fixing the normalization factor by $\sum_{m=0}^K a_m=1$, there remain $K$ nontrivial parameters. 
To obtain a determined system, we choose $L=K$ in the quartic theory. 

For $K=3$, the solutions for $G_2$ are $\{-1,\pm 1/\sqrt 3\}$. 
The latter pair of solutions are the same as the ones from setting the connected part of $4$-point Green's function to zero, 
whose error is $-14.6\%$ compared to the exact value \cite{Bender:2022eze}. 
As shown in \cite{Bender:2022eze}, the results do not converge to the exact values by setting the connected part of higher-point functions to zero. 
In contrast, our higher $K$ results from the null state condition \eqref{null-condition} converge to the exact values rapidly. 
 
For $K=5$, the solutions are $\{-2/3, 0, \pm1/\sqrt 2\}=\{-0.6666\dots, 0, \pm 0.7071\dots\}$. 
The solution at zero seems unphysical, while the errors in the other solutions are -1.4\%, 4.6\%. 
For $K=7$, the solutions are $\{\pm0.538\dots, \pm 0.6752\dots,  \pm 0.6787\dots,  \pm 1.098\dots\}$. 
The errors in the second and third solutions become smaller, i.e. $-0.12\%$ and $0.40\%$. 
For $K=9$, there is an unphysical solution at $0$ and four pairs of physical solutions with errors 
$\{-0.0096\%, 0.033\%, -2.5\%, 7.5\%\}$. 
For $K=11$, we find six pairs of solutions and their errors are 
$\{-0.00076\%, 0.0027\%, -0.26\%, 0.82\%, -24\%, 71\%\}$. 
According to the explicit solutions, the null operator exhibits a definite parity, i.e. either $a_{2k}=0$ or $a_{2k+1}=0$. 

In general, the solutions for $G_2$ correspond to the roots of some high-degree polynomial equations. 
Most of the roots accumulate around the exact values and 
the errors in the most accurate roots decrease rapidly. 
As the signs of the errors change alternatively, 
we can extract an accurate estimate from the median of a set of roots. 
A robust procedure of extracting the median is to repeatedly throw away the root with the largest distance from the average value, which also applies to the case of complex solutions.  

For example, there are ten pairs of solutions at $K=19$. 
Eight of the positive solutions are in the range $(0.67,0.69)$, 
while four of them are in the tiny range $(0.675978,0.675979)$, 
The median of the roots gives $0.6759782403$, reproducing the numerical expression in \eqref{phi4-G2}. 
The rapid growth of the root density near the exact values
also reminds us of the critical behavior of the Yang-Lee edge singularity \cite{Yang:1952be,Lee:1952ig,Kortman:1971zz}, 
described by the $i\phi^3$ theory \cite{Fisher:1978pf}. 

The second example is the non-Hermitian cubic theory with $\mathcal L(\phi)=\frac i 3\phi^3$. 
The only free parameter in the general solution is $G_1$. 
Its exact value is
\be
G_1=-3^{1/3}\frac{\G(2/3)}{\G(1/3)} e^{\frac{i2k\pi}{3}}i
=-0.729\,011\,133\dots e^{\frac{i2k\pi}{3}}i
\,,\nn
\ee
where $k=0,1,2$ depends on the choice of the Stokes sector. 
Then we introduce the null state operators as in the quartic case \eqref{n-t-operators} with $L=K$. 
The solutions of the null state condition again exhibit 
the phenomenon of root accumulation 
near the exact values. 
For $K=2$, the solutions for $G_1$ are $-2^{-1/3}e^{\frac{i2k\pi}{3}}i$.  
The same results can be obtained 
by setting the connected part of $G_3$ to zero \cite{Bender:2022eze}. 
As $K$ increases, the results of the cubic model converge even more rapidly 
than the quartic case. 
At $K=10$, there are eleven solutions with two of them at zero. 
The remaining nine solutions form three groups of roots related by $e^{\frac{i2k\pi}{3}}$. 
One group consists of purely imaginary solutions around $-073i$, which is supposed to be $\mathcal {PT}$ symmetric.  
Then the median gives an estimate $-0.729011134\dots i$, 
which is remarkably accurate. 

In general, the $\mathcal L=\frac g n \phi^n$ theory has $n$-fold symmetry. 
The above quartic case has $2$-fold symmetry due to parity symmetry. 
For the sextic theory with $\mathcal L(\phi)=\frac 1 6 \phi^6$, parity symmetry implies that 
the free parameters are $G_2$ and $G_4$. 
We should set $L=K+2$ because of parity constraints.  
We find three groups of $G_2$ roots related by the 3-fold symmetry. 
For $K=10$, the median of the root group around the real axis gives $G_2=0.57861625\dots$, 
which is close to the exact value at $6^{1/3}\sqrt\pi/\G(1/6)=0.57861652\dots$. 

If we consider the quartic theory $\mathcal L=-\frac 1 4\phi^4$ without parity symmetry, 
then $G_1$ does not vanish. 
The null state condition with $L=K+1$ leads to a determined system. 
For $K=10$, there are 66 roots of $G_1$ and 44 of them are located around $i^k$ with $k=0,1,2,3$. 
 the median of the $(-i)$ group gives $-0.977741049\dots i$,  
which is again close to the exact solution with $\mathcal {PT}$ symmetry at $G_1=-2\sqrt{\pi}/\G(1/4)=0.977741067\dots $. 
For the quintic theory with $\mathcal L(\phi) = -\frac i 5 \phi^5$, 
we need to set $L=K+2$ and there are ten groups of roots. 
For $K=10$, the two independent groups of roots give $-1.078676\dots i$ and $0.41184\dots i$, corresponding to
the $\mathcal {PT}$ symmetric solutions at $G_1=-1.078653\dots i$ and $G_1=0.41201\dots i$. 

%error estimation

\section{One-dimensional theories}
As a natural generalization of the $D=0$ procedure, 
we first focus on the equal-time limit, 
then we consider the more complicated case with unequal time. 
In quantum field theory, the equal-time limit of the $n$-point Green's functions are also known as the 1-point functions of composite operators. 
In the $D=1$ examples below, this limit is regular and 
we do not need to worry about the issue of additional divergences 
appearing at higher $D$. 

In the equal-time limit, one needs to be careful about the order of operators. 
The contact terms on the RHS of the DS equations imply 
\be
\<\cdots\big(\dot \phi(\t) \phi(\t)-\phi(\t) \dot\phi(\t)+1\big)\cdots\>=0\,,
\label{commutation}
\ee
which is the counterpart of the canonical commutation relation 
in the operator formalism. 
Time-translational invariance of $2$-point functions implies
\be
\<\dot{\mathcal O}_1(\t_1){\mathcal O_2}(\t_2)
+{\mathcal O_1}(\t_1)\dot{\mathcal O}_2(\t_2)\>=0\,.
\label{2pt-time-dot}
\ee
If $\mathcal O_2=1$, we have $\frac d {d\t}\<{\mathcal O}_1(\t)\>$=0, 
which is the counterpart of $\<[H,\mathcal O_1]\>=0$ in the Hamiltonian formalism \cite{Han:2020bkb}. 

Together with the dynamical part on the LHS of the DS equations, 
we can express the equal-time limit of the below Green's functions in terms of
\be
F_n={\pa_{\t_2}^nG_2(\t_1,\t_2)}\big|_{\t_1\rightarrow\t_2+0^+}=\left\<\phi(\t)\,\frac {d^n\phi(\t)}{d\t^n}\right\>\,,\quad
\ee
which is independent of $\t$ due to time-translation invariance. 
In general, we may need more than a two-point function 
as some observables are not related to $G_2$ by DS equations. 

To further fix the independent parameters in $\{F_n\}$, we impose the null state condition \eqref{null-condition} with
\be
\mathcal O_{\text{null}}^{(K)}=\sum_{m=0}^K a_m \frac {d^m\phi(\t)}{d\t^m}\,,\quad
\mathcal O_{\text{test}}^{(L)}=\sum_{m=0}^L b_m \frac {d^m\phi(\t)}{d\t^m}\,.\quad
\ee
The null constraints can be expressed in terms of $F_n$ using \eqref{2pt-time-dot}. 
The solutions are again associated with some high-degree polynomial equations. 
It turns out that most of the roots accumulate around the exact values. 
Then we can reconstruct other physical observables from these solutions. 
In particular, the polynomial equation associated with the null state encodes the spectral information. 

The first concrete example is again the quartic theory, whose Lagrangian is 
$\mathcal L=\frac 1 2(\dot\phi)^2+\frac 1 2 m^2 \phi^2+g\phi^4$. 
In the concrete computation, 
we set $m=1$ and $g=1/2$ to make contact with the previous results 
based on the Hamiltonian formalism. 
The DS equations are
\be
&&\left(-\pa^2_{\t_1}+1\right)G_n(\t_1,\t_2,\dots)+2 G_{n+2}(\t_1,\t_1,\t_1,\t_2,\dots)
\nn&=&\sum_{i=2}^n\d(\t_1-\t_i)\,G_{n-2}(\t_2,\dots,\t_{i-1},\t_{i+1},\dots)\,.
\ee
If $\t_1\neq \t_{i\neq 1}$, the contact terms are irrelevant 
\be
\<\cdots\big(\ddot \phi(\t_1)-\phi(\t_1)-2 \phi^3(\t_1)\big)\cdots\>=0\,,
\label{quartic-DS}
\ee
then the smooth equal-time limit leads to constraints on the 1-point functions of composite operators. 

We can solve the system of \eqref{commutation}, \eqref{2pt-time-dot}, \eqref{quartic-DS} 
and express the 1-point functions of composite operators in terms of $F_n$. 
Some examples are
\be
\<\phi^2\>=F_0\,,\quad
\<\phi\dot\phi\>=\frac 1 2\,,\quad
\<\phi^4\>=-\frac {F_0} 2 +\frac {F_2} 2 \,,\quad
\ee
\be
\<\phi^3\dot\phi\>=\frac {3F_0} 2 \,,\quad
\<\phi^2(\dot\phi)^2\>=\frac 1 2 +\frac {F_2} 6 -\frac {F_4} 6 \,.
\ee
The non-zero 1-point functions involve even numbers of $\phi$ due to parity symmetry. 
Higher time derivatives are removed by \eqref{quartic-DS}.
The independent composite operators take the ordered form $\phi^m (\dot\phi)^n$ due to \eqref{commutation}. 
\footnote{It may be useful to deduce their asymptotic behavior at large $m,n$. }
The odd case $F_{2m+1}$ do not appear because they are not independent, such as
\be
F_1=\frac 1 2\,,\quad
F_3=\frac 1 2 +3 F_0\,,\quad
F_5=\frac 1 2-3F_0+9F_2\,.\quad
\ee
%\be
%F_7=\frac {37} 2 +9 F_0-63F_2+63F_4\,.
%\ee
Therefore, all the 1-point functions of composite operators are encoded in 
the equal-time limit of $G_2(\t_1,\t_2)$. 
We set $L=2K$ in the null state condition to obtain a determined system. 

Let us examine the simplest case $K=1$ with $a_0\neq 0$
\be
\left\{\frac {a_1}{2a_0}+F_0\,,\,\frac 1 2 +\frac{a_1}{a_0}F_2\,,\,
\frac {a_1} {2a_0}+\frac{3a_1}{a_0} F_0+F_2\right\}=0\,,
\quad
\ee
corresponding to 
$\{1,\pa_{\t_1},\pa_{\t_1}^2\}\<\phi(\t_1)\mathcal O^{(K)}_\text{null}(\t_2)\>|_{\t_1\rightarrow \t_2}=0$. 
This system implies that $F_0=\<\phi^2\>$ is a root of the polynomial equation $24x^3+4x^2-1=0$,  
whose real solution at $0.2991\dots$ is already close to the exact value $\<\phi^2\>=0.3058136507\dots$. 
We may further require that the null state condition is satisfied beyond the equal-time limit, 
then we obtain a differential equation for the 2-point function with $\t_1>\t_2$
\be
(a_0+a_1\pa_{\t_2})\,G^{(K=1)}_2(\t_1,\t_2)=0\,,
\ee 
whose solution is 
\be
G_2^{(K=1)}(\t_1,\t_2)=c_1 e^{-\frac{a_0}{a_1}|\t_1-\t_2|}\,.
\ee
The real root of $F_0$ corresponds to the solution $a_0/a_1=1.6717\dots$, 
which is close to the exact energy gap $E_\text{gap}=E_1-E_0=1.628230531\dots$ 
in the Hamiltonian formulation. 

For higher $K$, the number of solutions grows rapidly. 
The null state condition on the 2-point function with unequal time implies
\be
G_2^{(K)}(\t_1,\t_2)= \sum_{m=1}^K c_m\, e^{-\D E_m|\t_1-\t_2|}\,,
\label{sol-2pt}
\ee
where $\{\D E_m\}$ are the roots of the null polynomial equation from the null-state solution
\be
\sum_{m=0}^K a_m\, x^m=0\,.
\label{null-polynomial}
\ee
They can be interpreted as the energy differences of intermediate states and the reference state $|\O\>$. 
If the energy spectrum is real and $|\O\>$ is the ground state $|0\>$ with the lowest energy, 
then all the roots of \eqref{null-polynomial} should be positive for a gapped system. 
For each $K$ examined, it turns out that there is only one solution with all roots positive! 
In this way, we make a definite selection from 
a large number of solutions.% of the truncated DS equations and the null state condition. 



The results converge rapidly to the exact values as $K$ increases. 
For $K=6$, we obtain 
$\<\phi^2\>=0.305813644\dots$ and 
$\D E = \{ 1.628230589\dots ,$ $5.882239\dots ,$ $10.9536\dots ,$ $16.661\dots, 23.3\dots, 32.5\dots\}$. 
For comparison, the exact energy differences between the lowest parity-even state and 
the low-lying parity-odd state are
$\{1.628230531\dots, 5.882226\dots,$ $10.9525\dots, 16.624\dots, 22.8\dots, 29.4\dots\}$. 
\footnote{Some of them can be deduced from \cite{Li:2022prn} by dividing the energies by two. }
The estimates are fairly accurate for the low-lying states and reasonably good for higher states. 
One can further determine $c_m$ in \eqref{sol-2pt} using $F_n$, 
which are related to the matrix element $\<n|\phi|0\>$. 
For instance, the square roots of the leading coefficients $c_1^{1/2}=0.5525659561\dots$ and 
$c_2^{1/2}=0.021994704\dots$ are close to the exact values
$\<1|\phi|0\>=0.5525659593\dots$ and 
$\<3|\phi|0\>=0.021994761\dots$. 

\begin{figure}[h!]
\begin{center}
\includegraphics[width=8cm]{Re-2pt.pdf}
\caption{The real part of the real-time 2-point function of the quartic theory for $K=1,2,3$ and the exact function. } 
\label{figure-2pt-Re}
\end{center}
\end{figure}

%\begin{figure}[h!]
%\begin{center}
%\includegraphics[width=8cm]{Im-2pt.pdf}
%\caption{The imaginary part of the real-time 2-point function of  the quartic theory with the same convention as Fig.\ref{figure-2pt-Re}. } 
%\label{figure-2pt-Im}
%\end{center}
%\end{figure}

\begin{figure}[h!]
\begin{center}
\includegraphics[width=7cm]{Abs-2pt.pdf}
\caption{The absolute deviation of the real-time 2-point function of the quartic theory from the exact function with $K=1,2,3$.  } 
\label{figure-2pt-Abs}
\end{center}
\end{figure}

We can translate the results into the case of Lorentzian spacetime. 
In Fig.\ref{figure-2pt-Re}, we present the results of the real part of the two-point function as a function of real time for $K=1,2,3$, where $t=-i(\t_1-\t_2)$. 
The imaginary part also converges to the exact function rapidly. 
Since it is already hard to distinguish the $K=2$ function from the exact one, 
we present the absolute deviation from the exact function in Fig. \ref{figure-2pt-Abs}.  

As at $D=0$, the 1D exact solution on the real axis is also a root accumulation point. 
If we do not impose the spectral constraint, 
the median of a dense group of roots around the real axis also leads to the same result or 
a small set of nearby roots. 
On the complex plane, there exists a conjugate pair of root groups around $-0.255\pm 0.297i$, 
but their physical interpretation is less clear to us. 

The second example is the non-Hermitian cubic theory 
$\mathcal L=\frac 1 2(\dot\phi)^2+g\phi^3$ with $g=i/2$.
The DS equations are
\be
&&-\pa^2_{\t_1}G_n(\t_1,\t_2,\dots)+\frac {3i} 2 G_{n+1}(\t_1,\t_1,\t_2,\dots)
\nn&=&\sum_{i=2}^n\d(\t_1-\t_i)\,G_{n-2}(\t_2,\dots,\t_{i-1},\t_{i+1},\dots)\,.
\label{cubic-DS}
\ee
The system of \eqref{commutation}, \eqref{2pt-time-dot}, \eqref{cubic-DS} 
again determines the 1-point function of composite operators 
in terms of $F_{n}$ and some $F_n$ are not free parameters, such as
\be
\<\phi\>=-\frac {2i} 3F_3\,,\quad
\<(\dot\phi)^2\>=-F_2\,,\quad
\<\phi(\dot\phi)^2\>=\frac {i} 3 F_4\,,\quad
\ee
\be
F_0=F_5=0\,,\quad
F_1=\frac 1 2\,,\quad
%=0\,,\quad
F_6=-\frac {15}4\,,\quad
F_7=-\frac{45}2F_2\,.\nn
\ee
%\be
%F_7=-\frac{45}2F_2\,,\quad
%F_9=-\frac{225}2 F_4\,,\quad
%F_{11}=\frac {2025}{2}\,.\quad
%\ee
Then we use the null state condition as in the quartic case. 
Since $\<\phi\>$ does not vanish, 
the null operator should contain a constant term $-a_0\<\phi\>$ due to 
$\<\mathcal O^{(K)}_\text{null}\>=0$. %\<\text{const.}+a_0\phi\>=0$. 
We again set $L=2K$ and 
the simplest approximation $K=1$ gives 
\be
\left\{\frac {a_1}{2a_0}-\<\phi\>^2\,,\,
\frac 1 2 +\frac{a_1}{a_0} F_2\,,\,
F_2+\frac{a_1}{a_0} F_3\right\}=0\,.
\ee
This leads to a degree-5 polynomial equation $(F_3)^5=\frac{81}{128}$ with $F_2=\frac 9 {16}F_3^{-2}$, 
so the 1-point function $\<\phi\>$ has 5-fold symmetry. 
The real solution at $\<\phi\>=-12^{-1/5}i=-0.608\dots i$ is close to the exact value $-0.590072533\dots i$. 
The estimate of the energy gap $E_\text{gap}=(9/2)^{1/5}=1.351\dots$ is the same as a truncation result in \cite{Bender:1999ek},  
which is not far from the exact value $1.4764808747\dots$. 
At higher $K$, the roots also exhibit 5-fold symmetry and 5 accumulation points. 


\begin{figure}[h!]
\begin{center}
\includegraphics[width=8cm]{root-phi3.pdf}
\caption{The $K=6$ solutions of the 1D non-Hermitian $i\phi^3$ theory.  
The red square indicates the exact value at $G_1=-0.5900725\dots i$.  
We find 123 roots of distance less than $10^{-1}$ from this exact value, 
while $\{44,24,12,6\}$ of them are of distance less than $\{10^{-2},10^{-3},10^{-4},10^{-5}\}$. 
Inset: The 6 solutions are obtained by iteratively discarding the most distant root from the average. %The median is extremely accurate. 
  } 
\label{figure-root-cubic}
\end{center}
\end{figure}

The case of $K=6$ is presented in Fig. \ref{figure-root-cubic}. 
For $K=6$, we obtain $\<\phi\>=-0.590072522\dots i$ by assuming the real part of the roots to the null polynomials are positive and the lowest one is real. 
The resulting energy differences are \{1.4764812\dots, 3.202970\dots, 5.081\dots,7.008\dots, 9.45\dots$\pm$ 1.05\dots i\}. 
They are close to the exact values \{1.4764809\dots, 3.202996\dots, 5.079\dots, 
7.059\dots, 9.16\dots, 10.09\dots\}. 
Note that the last two real values become a conjugate pair in the approximate solution. 
We can also obtain a small set of roots near this solution 
by extracting the median of the group of roots around $\<\phi\>\approx-0.6$. 
In the end, we can determine $c_m$ by $F_n$, such as 
$c_1=0.35645151\dots$ and $c_2=-0.008363555\dots$, 
which are close to the exact values 
$|\<1|x|0\>|^2=0.35645139\dots$ and $|\<2|x|0\>|^2=0.008363569\dots$. 



\section{Summary}
In this paper, we propose to resolve the indeterminacy of the DS equations 
by null states, which should be orthogonal to test states. 
In some sense, the null state condition can be viewed as a quantization condition, 
playing a similar role as the boundary condition of a functional integral.  
We discover that the exact solutions are root accumulation points.  
In the concrete $g\phi^n$ theory at $D=0,1$, 
we manage to extract rapidly convergent estimates 
from the medians of nearby roots, including the complex solutions,
and reconstruct the time-dependent 2-point Green's functions. 
We would like to extend our results to higher point functions and higher $D$.

At $D=1$, we also propose another method for extracting the best estimate. 
The null polynomial from the null state solution encodes the spectral information 
of intermediate states, 
so a bounded-from-below spectrum should have only positive roots. 
This requirement selects a unique solution. 
The close relations among null states, differential equations for $n$-point functions, 
and intermediate spectra are in beautiful parallel with 
the classical work of Belavin-Polyakov-Zamolodchikov 
on 2D minimal model CFT \cite{Belavin:1984vu}. 

We elucidate the intimate connection between 
approximate external null states and truncated spectra of intermediate states. 
It would be interesting to revisit the conformal bootstrap methods \cite{Poland:2018epd}, 
especially the truncation approach initiated by Gliozzi \cite{Gliozzi:2013ysa}. 
The complex solutions in the CFT context can have important implications on 
gauge theory, statistical and condensed matter physics \cite{Gorbenko:2018ncu}. 

For more systematic investigations and improved numerical stability, it may be useful to apply 
the tools of computational algebraic geometry or other advanced techniques to 
solve high-degree polynomial equations. 
\\

This work was supported by the 100 Talents Program of Sun Yat-sen University  
and the Natural Science Foundation of China (Grant No. 12205386).
\begin{thebibliography}{10}

\bibitem{Dyson:1949ha}
F.~J.~Dyson,
``The S matrix in quantum electrodynamics,''
Phys. Rev. \textbf{75}, 1736-1755 (1949)
doi:10.1103/PhysRev.75.1736

\bibitem{Schwinger:1951ex}
J.~S.~Schwinger,
``On the Green's functions of quantized fields. 1.,''
Proc. Nat. Acad. Sci. \textbf{37}, 452-455 (1951)
doi:10.1073/pnas.37.7.452

\bibitem{Schwinger:1951hq}
J.~S.~Schwinger,
``On the Green's functions of quantized fields. 2.,''
Proc. Nat. Acad. Sci. \textbf{37}, 455-459 (1951)
doi:10.1073/pnas.37.7.455

\bibitem{Bender:1988bp}
C.~M.~Bender, F.~Cooper and L.~M.~Simmons,
``Nonunique Solution to the Schwinger-dyson Equations,''
Phys. Rev. D \textbf{39}, 2343-2349 (1989)
doi:10.1103/PhysRevD.39.2343

\bibitem{Bender:2022eze}
C.~M.~Bender, C.~Karapoulitidis and S.~P.~Klevansky,
``Underdetermined Dyson-Schwinger Equations,''
Phys. Rev. Lett. \textbf{130}, no.10, 101602 (2023)
doi:10.1103/PhysRevLett.130.101602
[arXiv:2211.13026 [math-ph]].

\bibitem{Li:2022prn}
W.~Li,
``Null bootstrap for non-Hermitian Hamiltonians,''
Phys. Rev. D \textbf{106}, no.12, 125021 (2022)
doi:10.1103/PhysRevD.106.125021
[arXiv:2202.04334 [hep-th]].

\bibitem{Li:2017ukc}
W.~Li,
``New method for the conformal bootstrap with OPE truncations,''
[arXiv:1711.09075 [hep-th]].

\bibitem{Bender:1998ke}
C.~M.~Bender and S.~Boettcher,
``Real spectra in nonHermitian Hamiltonians having PT symmetry,''
Phys. Rev. Lett. \textbf{80}, 5243-5246 (1998)
doi:10.1103/PhysRevLett.80.5243
[arXiv:physics/9712001 [physics]].

\bibitem{Bender:1999ek}
C.~M.~Bender, K.~A.~Milton and V.~Savage,
``Solution of Schwinger-Dyson equations for PT symmetric quantum field theory,''
Phys. Rev. D \textbf{62}, 085001 (2000)
doi:10.1103/PhysRevD.62.085001
[arXiv:hep-th/9907045 [hep-th]].

\bibitem{Bender:2007nj}
C.~M.~Bender,
``Making sense of non-Hermitian Hamiltonians,''
Rept. Prog. Phys. \textbf{70}, 947 (2007)
doi:10.1088/0034-4885/70/6/R03
[arXiv:hep-th/0703096 [hep-th]].

\bibitem{Bender:2010hf}
C.~M.~Bender and S.~P.~Klevansky,
``Families of particles with different masses in PT-symmetric quantum field theory,''
Phys. Rev. Lett. \textbf{105}, 031601 (2010)
doi:10.1103/PhysRevLett.105.031601
[arXiv:1002.3253 [hep-th]].

\bibitem{r5} 
C.~M.~Bender {\it et al.}, {\it PT Symmetry: in Quantum and
Classical Physics} (World Scientific, Singapore, 2019). 



\bibitem{Yang:1952be}
C.~N.~Yang and T.~D.~Lee,
``Statistical theory of equations of state and phase transitions. 1. Theory of condensation,''
Phys. Rev. \textbf{87}, 404-409 (1952)
doi:10.1103/PhysRev.87.404

\bibitem{Lee:1952ig}
T.~D.~Lee and C.~N.~Yang,
``Statistical theory of equations of state and phase transitions. 2. Lattice gas and Ising model,''
Phys. Rev. \textbf{87}, 410-419 (1952)
doi:10.1103/PhysRev.87.410

\bibitem{Kortman:1971zz}
P.~J.~Kortman and R.~B.~Griffiths,
``Density of Zeros on the Lee-Yang Circle for Two Ising Ferromagnets,''
Phys. Rev. Lett. \textbf{27}, 1439-1442 (1971)
doi:10.1103/PhysRevLett.27.1439

\bibitem{Fisher:1978pf}
M.~E.~Fisher,
``Yang-Lee Edge Singularity and phi**3 Field Theory,''
Phys. Rev. Lett. \textbf{40}, 1610-1613 (1978)
doi:10.1103/PhysRevLett.40.1610

\bibitem{Han:2020bkb}
X.~Han, S.~A.~Hartnoll and J.~Kruthoff,
``Bootstrapping Matrix Quantum Mechanics,''
Phys. Rev. Lett. \textbf{125}, no.4, 041601 (2020)
doi:10.1103/PhysRevLett.125.041601
[arXiv:2004.10212 [hep-th]].

\bibitem{Belavin:1984vu}
A.~A.~Belavin, A.~M.~Polyakov and A.~B.~Zamolodchikov,
``Infinite Conformal Symmetry in Two-Dimensional Quantum Field Theory,''
Nucl. Phys. B \textbf{241}, 333-380 (1984)
doi:10.1016/0550-3213(84)90052-X

\bibitem{Poland:2018epd}
D.~Poland, S.~Rychkov and A.~Vichi,
``The Conformal Bootstrap: Theory, Numerical Techniques, and Applications,''
Rev. Mod. Phys. \textbf{91}, 015002 (2019)
doi:10.1103/RevModPhys.91.015002
[arXiv:1805.04405 [hep-th]].

\bibitem{Gliozzi:2013ysa}
F.~Gliozzi,
``More constraining conformal bootstrap,''
Phys. Rev. Lett. \textbf{111}, 161602 (2013)
doi:10.1103/PhysRevLett.111.161602
[arXiv:1307.3111 [hep-th]].

\bibitem{Gorbenko:2018ncu}
V.~Gorbenko, S.~Rychkov and B.~Zan,
``Walking, Weak first-order transitions, and Complex CFTs,''
JHEP \textbf{10}, 108 (2018)
doi:10.1007/JHEP10(2018)108
[arXiv:1807.11512 [hep-th]].

\end{thebibliography}

\end{document} 
