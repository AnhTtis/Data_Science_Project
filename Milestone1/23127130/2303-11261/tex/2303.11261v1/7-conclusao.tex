\section{Conclusion} \label{sec:final}

\notaIL{Syok}{Para revisao, inclusive as figuras...}

The complete description of a typical phase space of a strictly convex billiard (Birkhoff) is still a challenge. 
With this work, we intent to address this question in the scope of symmetric curves. 

We have shown that symmetric billiards in general mix stable (elliptic islands) and unstable (hyberbolic orbits) behavior.  
It is also  well known that convex billiards have invariant spanning curves \cite{laz} close to the bottom and the top of the phase space. 
The figure bellow illustrates what we expect to be the generic phase space of $n$-symmetric ovals by clearly displaying these elements.

\begin{figure}[h]
	\begin{center}
		\includegraphics[height=4cm]{fase-sim4-020-001} \hskip 1cm
		\includegraphics[height=4cm]{fase-sim5-040-025}
	\end{center}    
	\caption{General structure of the phase space of symmetric billiards}
%(L) $4$-symmetric oval with $R(\phi)=1+ 0.2 \cos (4\phi) + 0.01 \cos (8\phi )$.
%(R) $5$-symmetric oval with $R(\phi)=1+ 0.4\cos (5\phi) + 0.25 \cos (10\phi)$.}
\end{figure}


Tabachnikov \cite{tab} and Gutkin \cite{gutkin} proved that a non circular oval billiard has an horizontal invariant curve of the form $p=p_0 =\cos \alpha_0 \neq 0$ if and only if the radius of curvature of the boundary is given by $R(\varphi)=1+a_1\cos(n\varphi)$ (so the curve is symmetric), with $n\geq 4$, $|a_1|<1$, and $\alpha_0$ satisfies $\tan (n\alpha_0) = n\tan(\alpha_0)$. 
If $n$ is odd, $\alpha_0=\pi/2$ is a solution and the billiard's boundary has constant width. 
In this case, one can prove the existence of invariant curves close to $p_0=0$ \cite{tese}.
The persistence of invariant curves under perturbations of the boundary (as observed on Figure~\ref{fig:retas}), although expected, is a challenging subject.
\begin{figure}[h]
\label{fig:retas}
		\includegraphics[width=0.3\hsize]{rc5-040-000} \hfill
		\includegraphics[width=0.3\hsize]{rc5-040-001} \hfill
		\includegraphics[width=0.3\hsize]{rc5-040-005} 
	\caption{Invariant curves}
\end{figure}
%%%%%%%%%%%%%%%%%%%%%%%%%%%%%%%%%%%%%%%%%%%%%%%%%%%%%%%%%%%%

\begin{comment}
\begin{itemize}
\item Geraldo, voce provou que perturbacoes de curvas com largura constante vao ter curvas invariantes perto do p=0?
\item Citar o  resultado do Bunimovich (checar se  é genérico ou aberto e denso) ????
\item largura constante (simetria impar)
\item relacao entre regioes de instabilidade e as orbitas de simetricas
\item separacao de regioes de instabilidade
\item coisas conectadas onde aplicar: elliptic flowers, gutkin billiards
\end{itemize}
\end{comment}

Finally we observe that most of our results may be extended to piecewise smooth curves such as Bunimovich's flowers \cite{bunin22}.
%\notaIL{Syok}{curiosidade: como e h a funcao suporte de uma curva de largura constante?}