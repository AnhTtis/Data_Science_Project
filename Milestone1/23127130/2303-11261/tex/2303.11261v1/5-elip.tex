\section{Stability of elliptic orbits} \label{sec:elip}


In this section we will analyze the stability of the symmetric elliptical orbits. 
This  stability can be established  by Birkhoff's Normal Form \cite{birk} and Moser's Twist Theorem \cite{moser}, which states conditions for the existence of infinitely many invariant curves surrounding the points of the orbit resulting in elliptic islands of positive Lebesgue measure (Moser stability). In this approach, we need to switch to canonical variables $(s,p)$, where $s$ is the arclenght parameter of the boundary $\Gamma$, i.e. $ds=R \,d\phi$. We will interplay between the variables $s$ and 
$\phi$ and in either case denote the curve as $\Gamma$ and the billiard map as $T$.


Given  $1 \le m \le  n-1$, let $(\phi_0 = \phi(s_0), p_m=\cos \dfrac{m}{n}\pi )$ be an elliptic fixed point of the quotient map $T_m$ associated to an oval $\Gamma$.
Using canonical variables $(s,p)$ and the coordinates of the Jordan Form of  
$DT_m(s_0,p_m)$ at the fixed point, the complex form of the map is written as \cite{moeckel}
\begin{equation}
z\rightarrow \lambda \left( z+c_{20}z^{2}+c_{11}z\bar{z}+c_{02}\bar{z}%
^{2}+c_{30}z^{3}+c_{21}z^{2}\bar{z}+c_{12}z\bar{z}^{2}+c_{03}\bar{z}%
^{3}\right) +O\left( \left\vert z\right\vert ^{4}\right)
\label{eq: primeira complexificao}
\end{equation}
where $\lambda =\cos\zeta \pm i\sin \zeta $ are the eigenvalues of $DT_m$. 
If the fixed point is non resonant, by the 
Birkhoff's Normal Form Theorem, there is a $C^{\infty }$ coordinate change bringing the map into the form
$$
z\rightarrow e^{i\left( \zeta +\tau\left\vert z\right\vert ^{2}\right)
}z+O\left( \left\vert z\right\vert ^{4}\right) .
$$
Moser's Twist Theorem then assures that if the 
first Birkoff Coefficient (Twist Coefficient) $\tau$ is non zero, the fixed point is stable. 

%definicao ressonante%
\begin{defin}
An elliptic fixed point  of a conservative bi-dimensional map 
is resonant (of order 4) if its eigenvalues 
satisfy $\lambda^j = 1$ for $j = 1, 2, 3$ or $4$.
Otherwise we say that the fixed point is non resonant.
\end{defin}

From the proof of Proposition \ref{prop:estabilidade linear} we have that 
$\displaystyle
\cos \zeta = \frac{L_m(\phi_0)}{R_0 \sin \alpha_m} -1
$.
%where  $R_0 = R(\phi_0)$, $\alpha_m = \frac{m}{n} \pi$and $ L_m (\phi_0) =|| \Gamma(\phi_0+ 2 \pi \frac{m}{n}) - \Gamma(\phi_0) ||$.
If the point is elliptic,  $0 < L_m(\phi_0)< 2 R_0 \sin \alpha_m$ and $\lambda, \lambda^2 \ne  1$.
So the only resonance conditions are 
\begin{eqnarray*}
\lambda^3 = 1 
& \Leftrightarrow 
\ \zeta = \pm 2\pi/3   
& \Leftrightarrow
 \ 2 L_m(\phi_0) = R_0 \sin \alpha_m
\\
\lambda^4 = 1 
&\Leftrightarrow 
\ \zeta = \pm \pi/2   
& \Leftrightarrow \  L_m(\phi_0) =  R_0 \sin \alpha_m
\end{eqnarray*}
As $L_m(\phi_0)=2 g(\phi_{0}) \sin \alpha_m $ and
$R_0 =g(\phi_{0}) +g^{\prime \prime }(\phi_{0})$ 
we can rewrite the resonance conditions in terms of the support function
\begin{equation}
\lambda^3 = 1 \Leftrightarrow 3 g (\phi_{0}) - g^{\prime \prime }(\phi_{0}) = 0 
\ \ , \ \
\lambda^4 = 1 \Leftrightarrow g(\phi_{0}) -g^{\prime \prime }(\phi_{0}) = 0 \label{eqn:g-resonante}
\end{equation}
We say that $\phi_0$ is a resonant minimum of $g$ if at least one of conditions above holds,  otherwise, we say that we have a non resonant minimum. 
As these equations do not  depend on $m$, if $\phi_0$ is a non-resonant minimum, the fixed points $(\phi_0, \alpha_m)$ will be non-resonant for all $m$.
However, as the eigenvalues of the orbit of $(\phi_0,\alpha_m)$ are $\lambda^m$, the orbit itself may be resonant even if the fixed point is not.
Nevertheless we define
\begin{defin}
A family of elliptic symmetric orbits  is resonant if it is associated to a resonant minimum of the support function. Otherwise we say that the family is non resonant. 
\end{defin}

%%%%%%%%%%%%%%%%%%%%%%%%%%%%%%%%%%%%%%%%%%%%%%%%%%%%%
%lema: g morse com ressonancia pode ser perturbada

\begin{lem} \label{lem:res}
 If $g \in {\mathcal{G}}$ 
 is a support function with one or more 
resonant minima, there is $g_{\varepsilon}$ also in ${\mathcal{G}}$ and {$C^2$-close} 
to $g$ such that all its non degenerate minima are non resonant and coincide with the non degenerate minima of $g$ .
\end{lem}

\begin{prova} 
Let $g_{\varepsilon} = {g} + \varepsilon$.
Obvioulsy, $g^{\prime}_{\varepsilon }= {g}^{\prime}$ and
 $g^{\prime\prime}_{\varepsilon }= {g}^{\prime\prime}$  for all $\phi$ and so they have the same minimum (and maximum) points. Moreover
$\| {g} - g_{\varepsilon} \| = |\varepsilon| $.
It is clear that we can choose $\varepsilon \sim 0$ such that the non degenerate minima of $g_{\varepsilon}$ are all non resonant, i.e.
$g_{\varepsilon} \ne g_{\varepsilon}^{\prime\prime}$,  $g_{\varepsilon} \ne  g_{\varepsilon}^{\prime\prime}/3$ 
simultaneously at all non degenerate minimum points. 
\end{prova}

At this point it is worthwhile to observe that adding a constant to the support function results in a normal perturbation of the associated oval. 

%%%%%%%%%%%%%%%%%%%%%%%%%%%%%%%%%%%%%%%%%%%%%

\begin{prop}
The set of ovals such that all families of symmetric elliptic orbits are non resonant is open and dense in 
$\mathcal{N}$
\label{prop: nao ressonantes aberto e denso}
\end{prop}

\begin{prova}%[Proof of Proposition \ref{prop: nao ressonantes aberto e denso}]
Let $g$ be a support function with a non resonant minimum at a point $\phi_0$, i.e.
$g'(\phi_0)=0$, $g''(\phi_0) >0$ , $g''(\phi_0)\ne g(\phi_0)$ and $g''(\phi_0)\ne 3 g(\phi_0)$.  
Then any support function 
$\tilde{g}\in \mathcal{G} $ 
with $\left\vert \left\vert \tilde{g}-g\right\vert \right\vert_{2}<\delta $ 
will have a non resonant minimum $\widetilde{\phi}_0 \sim \phi_0$ for $\delta$ small enough
{and so this is an open property}.
Since non degenerate points are isolated and finite, the argument holds for all non degenerate points.

Given an oval $\Gamma$ with support function $g$  and $\delta > 0$, 
we can choose a Morse function $\tilde{g} \in \tilde{\mathcal{G}}$ 
such that
$\left\vert \left\vert {g}- \tilde{g}\right\vert \right\vert_{2}<\delta /2$.
Now, if necessary,  let $g_{\varepsilon} = \tilde{g} + \varepsilon$, with $|\varepsilon|<\delta/2$ chosen as in 
Lemma~\ref{lem:res} above  and $\Gamma_{\varepsilon}$ the associated curve.
We have
$\| \Gamma_{\varepsilon }-\Gamma \|_{2}=
\| g_{\varepsilon }-g \|_{2}<
\| g- \tilde{g}\|_{2}
+
\| \tilde{g}-g_{\varepsilon } \|_{2} <
\delta $ which proves that {the subset of ovals such that all families of elliptical symmetric orbits are non resonant} is dense in $\mathcal{N}$. 
\end{prova}


%%%%%%%%%%%%%%%%%%%%%%show is open%%%%%%%%%%%%%%%%
%Calculo do tau 1
%%%%%%%%%%%%%%%%%%%%%%%%%%%%%%%%%%%%%%%%%%%%%%%


Once we have a non resonant elliptic fixed point we can go on and determine
the Twist Coefficient which depends on the derivatives of the map up to order 3 \cite{moeckel,ss}
\begin{equation}
\tau=\mbox{Im} ( c_{21}) 
+\frac{\sin \zeta }{\cos \zeta-1}
\left( 
3 \vert c_{20} \vert ^{2}
+\frac{2\cos \zeta -1}{2\cos\zeta +1} \vert c_{02}\vert ^{2}
\right)
\label{eq: formula coef de birkh}
\end{equation}

Obtaining the 3-jet of 
$T_m $ at $(s_{0} = s(\phi_0), p_m= \cos \alpha_m)$ and the $c_{ij}$'s is a straightforward but long calculation which, using  the symbolic computation software  Maple,  leads to
\begin{eqnarray}
\tau &=&
-\frac{1}{8 R_0 \sin ^{3}\alpha_m }
+\frac{3 \cos^{2} \alpha_m }{8\sin^{2} \alpha_0 \left( 2L_m(\phi_0) -R_0 \sin \alpha_m \right) }
 \label{eq:tau1} \\
%&&-\frac{L_m(\phi_0) \left( 7 L_m(\phi_0) -4 R_0 \sin \alpha_m \right) }{8 \left( L_m(\phi_0) -2 R_0 \sin \alpha_m \right)^{2} \left( 2 L_m(\phi_0) -R_0 \sin \alpha_m \right)} {R^{\prime}_0}^2 \nonumber \\
&&-\frac{1}{8}\frac{L_m(\phi_0)}{(L_m(\phi_0)-2 R_0 \sin \alpha_m)^2} 
\left(
3 + \frac{L_m(\phi_0)-R_0 \sin \alpha_m }{2 L_m(\phi_0)-R_0 \sin \alpha_m }
\right)
{R^{\prime}_0}^2 \nonumber \\
&&-\frac{L_m(\phi_0) }{8\sin \alpha_m \left(
L_m(\phi_0) -2R_0 \sin \alpha_m\right) }R^{\prime \prime}_0
\nonumber
\end{eqnarray}%
where $R'_0 = \frac{dR}{ds}(s_0)$ , $R''_0 = \frac{d^2R}{ds^2}(s_0)$ and,  as before, $\alpha_m = \frac{m\pi}{n}$ and $R_0 = R(\phi_0)$.


%%%%%%%%%%%%%%%%%%%%%%%%%%%%%%%%%%%%%%
% no maximo um periodo com tau 1 nulo
%%%%%%%%%%%%%%%%%%%%%%%%%%%%%%%%%%


\begin{prop} 
Let $\Gamma \in {\mathcal N}$
 be an oval with a family of elliptical non resonant symmetric orbits. 
The twist coefficient of these orbits can only be zero for at most one period, i.e. for  the orbits associated to at most one value of $1\le m \le  n/2 $.
\label{prop:max1tau0}
\end{prop}


\begin{prova}
Let the family be associated to $\phi_0$. 
As $L_m(\phi_0) = 2 g (\phi_0) \sin \alpha_m$, the first Birkoff coefficient associated to an orbit in the family, given by \ref{eq:tau1} depends on $m$ only through  
$\alpha_m = \frac{m\pi}{n}$.
\begin{eqnarray*}
\tau&=&
-\frac{1}{8 R_0 \sin ^{3}\alpha_m }
+\frac{3 \cos^{2} \alpha_m }{8\sin^{3} \alpha_m ( 4 g_0 -R_0) }
 \\
&&-\frac{g_0}{16 \sin \alpha_m (g_0-R_0 )^2} 
\left(
3 + \frac{2g_0-R_0  }{4 g_0-R_0 }
\right)
{R^{\prime}_0}^2  \\
&&-\frac{ g_0 }{8\sin \alpha_m (g_0 -R_0) }R^{\prime \prime}_0
\end{eqnarray*}
where $g_0 = g(\phi_0)$.

Then  $\tau=0$ if and only if
\begin{equation}
4\frac{g_0-R_0 }{R_0 ( 4 g_0 -R_0) }
+\frac{3 \sin^{2} \alpha_m }{( 4 g_0 -R_0) }
 +\frac{g_0 \sin ^{2}\alpha_m }{2  (g_0-R_0 )^2} 
\left(
3 + \frac{2g_0-R_0  }{4 g_0-R_0 }\right){R^{\prime}_0}^2 
+\frac{ g_0 \sin ^{2}\alpha_m }{ (g_0 -R_0) }R^{\prime \prime}_0
=0
\label{eq:tau0linear}
\end{equation}
which is a linear equation in $\sin^2 \alpha_m$. 
Clearly, if $\phi_0$ (and so $g_0$ and $R_0$) is given and as $1\le m\le  n/2 $, this equation has at most one solution.
\end{prova}

An immediate consequence of this proposition is that having a symmetric Moser stable orbit  is an open an dense property for ovals with symmetry $n \ge 4$. 

\begin{comment}
-- ter ilha eh denso porque as Morse  nao resonantes tem ilha 
-- ter ilha eh aberto eh geral demais. o que sabemos mostrar que eh aberto eh ter uma orbita simetrica nao ressonante com twist nao nulo 
\end{comment}

\begin{thm}
\label{thm:teorema1}
There is an open and dense subset of $C^{\infty}$ ovals with symmetry $n\ge 4$ with an elliptical island. 
\end{thm}

\begin{prova} 
By Proposition \ref{prop:max1tau0} above, if the symmetry $n$ is at least $4$ it is enough to find an open and dense  set  of ovals with a non resonant elliptic family since at least one of the associated symmetric orbits will have a non zero twist coefficient and so will be Moser stable. Proposition \ref{prop: nao ressonantes aberto e denso} gives us such a set.
\end{prova}

With regard to the restriction of $n \ge 4$ in the theorem above, we observe
 if $n=2$ the only symmetric orbits have period two and  
if $n=3$  we only have symmetric orbits of period three. 
% n impar = largura constante (as orbitas de periodo 2 nao sao simetricas...)
In both case, we can have a zero twist coefficient. The generic stability of period 2 orbits has been established in \cite{ilhas,ss} for general curves, and the result also holds within symmetric curves. Anyway Theorem~\ref{thm:teorema2} bellow applies in these cases. 


%%%%%%%%%%%%%%%%%%%%%%%%%%%%%%%%%%%%%%%%%%%%%
%Perturbando o coeficiente de twist
%%%%%%%%%%%%%%%%%%%%%%%%%%%%%%%%%%%%%%%

{
Finally, in order to prove that a zero twist coefficient is a rather rare fact, we can use the same argument of 
\cite{ilhas}, which relies on the remark that $\tau$ depends  linearly on $R''_0$
and its  coefficient
$$-\frac{L_0}{8\sin \alpha_m \left(
L_0 -2R_0 \sin \alpha_m
\right) }\neq 0$$
This implies that a small perturbation of $R^{\prime\prime}$  is enough to slightly modify this term in order to obtain a non-zero twist coefficient

}

\begin{lem}
\label{prop:aprox oval tau dif 0}
Let  $\Gamma \in {\mathcal{N}}$ be
such that the  $(\phi_0,\alpha_m)$ corresponds to a  non resonant elliptic fixed point of the quotient map $T_m$  with a zero first Birkhoff coefficient $\tau$.
Then there is an oval  $\Gamma_{\epsilon} \in \mathcal{N}$
arbitrarily close to $\Gamma$,  such that the point $(\phi_0, \alpha_m)$ also corresponds to a non resonant elliptic point of the map ${T}_m$,  but its twist coefficient is non zero.
\end{lem}
\begin{prova} Let $g$ be the support function of $\Gamma$.
We can assume, without any loss of generality, that the elliptic family with a zero twist coefficient is associated to $\phi_0 = 0$,
which is an isolated minimum,  
and so $g'(0) = 0$ and $g^{\prime\prime} > 0 $. 

Given $0<\delta_1 < \delta_2$  let $\rho(x)$ be a $C^{\infty}$ 
function with compact support such that:
\begin{lista}
\item
$\rho(-x) = \rho(x)$
\item
$\rho (x) = 0$ for $|x| > \delta_2$
\item
$\rho (x) = 1$ for $|x| < \delta_1$ (all the derivatives of $\rho$ vanish in a neighborhood of $x=0$)
\item
$ 0<-\rho^{\prime}(x) < C$ for $\delta_1 < x < \delta_2$
%\item nao sei se eh morse, mas acho que tanto faz...
\end{lista}
There are well know functions with these properties which are frequently used in the construction of 
partitions of the unity (smooth indicator or bump functions).

We choose $\delta_2$ such that the support of $\rho$ does not contain any  other critical points of $g$, 
i.e., $g'(\phi) \ne 0$ for  $0<|\phi|<\delta_2$.
Now let   $g_{\epsilon}(\phi) = g(\phi) +  \epsilon \phi^4 \rho(\phi)$ with $\epsilon \ne 0$ small. Then
\begin{lista}
\item
 $g_{\epsilon}(0) = g (0)$ 
\item 
 $g^{\prime}_{\epsilon}(0) = g^{\prime} (0) = 0$, 
$g^{\prime \prime}_{\epsilon}(0) = g^{\prime \prime} (0)$, 
$g^{\prime \prime \prime}_{\epsilon}(0) = g ^{\prime \prime \prime}(0)$
\item
 $g^{\prime}_{\epsilon}(\phi) = g^{\prime} (\phi )$ for $ 0 \le |\phi| \le \delta_1$ (and for $|\phi| >\delta_2$).
\item
 $g^{\prime}_{\epsilon}(\phi) = g^{\prime} (\phi ) + \epsilon \phi^3 (4 \rho(\phi) + \phi \rho^{\prime}(\phi) )$ for
$ \delta_1 < |\phi|  < \delta_2$. As $g^{\prime} \ne 0 $, $0 < \rho < 1$ and $\rho^{\prime}$ is bounded in this region, we can choose $\epsilon$ small enough in order that 
$g^{\prime}_\epsilon$ is also not zero and  $g_{\epsilon}$ and $g$ have the same critical points in $M =  [0,2\pi/n)$.
\item
$g^{\prime \prime \prime \prime}_{\epsilon}(0) = g ^{\prime \prime \prime \prime}(0) + 24 \epsilon $
\end{lista}

Let $\Gamma_{\epsilon}$ be the oval in $\mathcal{N}$ which corresponds to the 
support function $g_{\epsilon}$.
As ${\Gamma}_{\epsilon}$ and $\Gamma$ also have a third order contact at $\phi_0$, 
the point $ ( \phi_{0},\alpha_m)$
is also a non resonant elliptic fixed point of the map quotient map associated to
$\Gamma_{\epsilon}$. 
Moreover   
$$ || {\Gamma}_{\epsilon}(\phi_0 + \frac{2\pi m}{n}) - {\Gamma}_{\epsilon}(\phi_0) ||
=   || {\Gamma}(\phi_0 + \frac{2\pi m}{n}) - {\Gamma}(\phi_0) ||
= L_m(\phi_0) = L_0$$
and ${R}_{\epsilon}(\phi_0) = {R}(\phi_0)  = R_0 $ as well as the first derivatives.
 
If ${\tau}_{\epsilon}$ and  ${\tau}$ denote the first coefficient of the 
non resonant elliptic fixed point $(\phi_0, p_m)$ for the quotient map  $T_m$ associated respectively to $\Gamma_{\epsilon}$ and $\Gamma$
we have
$$
{\tau}_{\epsilon} -\tau = 
- \frac{L_0 }{
8 R_0^2 \sin \alpha_m ( L_0 -2R_0 \sin \alpha_m) }%
\left( 
\frac {d^2 {R}_{\epsilon}}{d{\phi}^2}(\phi_0) - \frac {d^2 {R}}{d{\phi}^2}(\phi_0)
\right)
$$
As by hypothesis  $\tau=0$ and we have $\frac{d^2 R}{d\phi^2} = g'' + g''''$, by choosing ${g}_{\epsilon}$ such that 
${g}'''''_{\epsilon} (\phi_0)  \ne {g}''''' (\phi_0) $
we obtain 
${\tau}_{\epsilon}\neq 0$ which concludes the proof.
Moreover, as changing $g''''$ wont change the independent and the  $(R')^2$ terms in $\tau$, we can do this without causing the twist coefficients for the other values of $m$ to vanish.
\end{prova}

We conclude then that the set of Morse $n$-symmetric ovals, such that all symmetric orbits are non resonant and have a non zero twist coefficient is dense in $\tilde{\mathcal N}$ and so also in $\mathcal N$. As these properties are clearly open we have
%%%%%%%%%%%%%%%%%%%%%%%%%%%%%%%%%%%%%%%%%%%%%%%%%%%
%%%%%%%%%%%%%%%%%%%%%%%%%%%%%%%%%%%%%%%%%%%%%%%%%%%
%{ter todas as orbitas simetricas elipticas Moser estaveis eh uma propriedade aberta e densa entre as ovais simetricas? } 
\begin{thm}
\label{thm:teorema2}
There is an open and dense subset of the $n$-symmetric ovals  where all the elliptic symmetric orbits are Moser stable.
\end{thm}

\begin{comment}
 In fact, our approach using Birkoff's Normal Form and Moser's Twist Theorem works for $C^{4}$ area preserving maps and so theorems \ref{thm:teorema1} and \ref{thm:teorema2} hold for $C^5$ curves.
\notaIL{Syok}{An alternative approach uses Herman's Last Geometric Theorem \cite{fayadkrikorian} and holds for $C^{\infty}$ maps with a diophantine fixed point.}
\end{comment}

