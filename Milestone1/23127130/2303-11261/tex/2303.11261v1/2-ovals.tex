\section{Symmetric ovals}\label{sec:curvas}

We will study the billiard on ovals (regular, closed, $C^{\infty}$ plane curves with stricly positive curvature) with rotational symmetry. 
Given an integer $n \ge 2$, we say that an oval $\Gamma$ is $n$-symmetric if it is invariant under a 
$\frac{2\pi}{n}$ rotation around a point ${o}$ called the symmetry center which we can assume is at the origin.
The set of $C^{\infty}$ $n$-symmetric ovals
with symmetry center at the origin and parametrized by the angle $\phi $
between the tangent vector $\sigma = (\cos \phi, \sin \phi)$ and the positive horizontal axis $x$ is denoted by  $\mathcal{N}$.

The support function of  an oval $\Gamma$ is defined by $g(\phi) = - < \Gamma(\phi), \eta(\phi) >$, where $\eta(\phi) = (-\sin \phi, \cos \phi)$ is the inward normal. 
The positive function $g$ is related to the radius of curvature by $g(\phi) + g''(\phi) = R(\phi) $.
Given a strictly positive, $\frac{2\pi}{n}$ periodic and smooth function $g(\phi)$, such that $g+g'' > 0$,  an $n$-symmetric oval is defined (Figure \ref{fig:coordg}) by
\begin{equation}
\Gamma (\phi)  = - g(\phi) (-\sin \phi, \cos \phi)+ g'(\phi)  (\cos \phi, -\sin \phi)  \label{eq:curva em g}
\end{equation}

\begin{figure}[h]
\begin{center}
\includegraphics[width=0.3\hsize]{coordg}
\end{center}
\caption{Oval $\Gamma$ defined by its support function as in \ref{eq:curva em g}}
\label{fig:coordg}
\end{figure}


Let $S_n= \left. \mathbb{R} \middle / \frac{2\pi }{n}\mathbb{Z} \right.$ 
and consider the set 
$C^{\infty}\left( S_n,\mathbb{R} \right) $ 
with the $C^2$ norm
$\|  . \|_{2}$.
We define
\begin{equation}
\mathcal{G}=\{ g\in C^{\infty}(S_n, \mathbb{R}) ,  g( \phi) >0, g( \phi) +g^{\prime \prime }( \phi )>0 \}
\end{equation}
and refer to it as the set of ({$C^{\infty}$}) support functions. 
 It is clear that $\mathcal{G}$ open in $C^{\infty}(S_n, \mathbb{R})$.


It is easy to verify that we  have a one to one correspondence between the set $\mathcal{N}$ and the set 
$\mathcal{G}$. Using  the topology induced by $\mathcal{G}$ in $\mathcal{N}$
we have then an homeomorphism between these two sets.
In this topology, the set  $\mathcal{N}$ is a metric space with norm
$\left\vert \left\vert \Gamma \right\vert \right\vert_{2}
=\left\vert \left\vert g\right\vert \right\vert_{2}$, where  $g$  is the function in
$\mathcal{G}$ which corresponds to the oval  $\Gamma $ in $\mathcal{N}$.


At this point, it is worthwhile to notice that a translation of the origin does not affect the support function. 
On the other hand, a translation in the argument of the support function, $g(\phi) \to g(\phi - \phi_0)$ results in a rotation of the curve. So, in fact we can associate to a support function a class of (equivalent) curves obtained from the original one by translations and rotations. This clearly does not change the dynamics. So from now on, a curve will be in fact, in our context, an equivalence class. 
Moreover, multiplying the support function by a constant results in a homothety transformation of the curve which also does not affect the dynamics, since our results are based on properties of the critical points of the support function which are scaling invariant. 


Finally it will be useful to consider support functions which are Morse functions.
We denote by  $\tilde{\mathcal{G}} \subset {\mathcal{G}}$ 
the subset of $C^{ \infty}\left( S_n,\mathbb{R}\right) $ positive Morse functions $g$ such that $g+g^{\prime\prime} >0$ and refer to it as the set of Morse support functions.
The set $\tilde{\mathcal{N}} \subset \mathcal{N}$ of $C^{\infty}$ $n$-symmetric ovals 
associated to  the set $\tilde{\mathcal{G}}$ is called the set of Morse $n$-symmetric ovals, it is open and dense in $\mathcal{N}$ as is 
$\tilde {\mathcal G}$ in  $\mathcal{G}$.


%%%%%%%%%%%%%%%%%%%%%%%%%%%%%%%%%%%%%%%%%%%%%%%%%%%%%%%

