\section{Linear stability of the symmetric orbits}
\label{sec:linear}


In order to investigate the dynamical properties of the symmetric orbits, we will explore how the symmetry of the curve reflects itself on the phase space.
It is clear that the invariance of the curve under the rotation of $\frac{2\pi}{n}$ implies that the billiard map is
invariant under the translation $\phi \to \phi + \frac{2\pi}{n}$ and
thus the phase space consists of $n$ equivalent vertical strips which are clearly seen on 
Figure~\ref{fig:poligonos12}.
Observe also the horizontal symmetry around $p=0$, due to the reversibility.

\begin{figure}
\begin{center}
\includegraphics[width=0.3\hsize]{polig12}
\hskip 1cm
\includegraphics[width=0.35\hsize]{orbitas12-040-000} %,valign=c
\end{center}
\caption{Periodic trajectories in a 12-symmetric oval:
%(hyperbolic orbits are red and elliptic blue).
the orbits corresponding to these trajectories  start at the vertical line $\phi =0$. All points $(\frac{2k\pi}{n},  \cos \frac{m\pi}{n})$ for $k=0,\ldots,n-1$ and $m=1,\ldots,n-1$ (blue dots) are periodic and belong to the orbits of the same family. The red dots correspond to the family associated to $\phi=\frac{\pi}{n}$.
The phase space is composed of twelve identical vertical strips, each one contains exactly one blue and one red dot.}
\label{fig:poligonos12}
\end{figure}

Given $m$ such that $1 \le m \le n-1$,  we define the equivalence relation
$$ (\phi_1,p_1) \sim_m (\phi_2,p_2) \Leftrightarrow \phi_2 = \phi_1 \hbox{ mod } \frac{2m\pi}{n}$$  
This induces a map $T_m$ on the cylinder 
%${\obs \Omega_m} =  \mathbb{R}/{ {\frac{2m\pi}{n}} \mathbb{Z}} \times (-1,1)$
${ \Omega_m} = S_n \times (-1,1)$
given by
$
T_{m}([(\phi,p)]_m) = [T(\phi,p)]_m$.
This map is also a  diffeomorfism with the quotient topology and we will call it {\em the $m-$quotient map}. 
We will sometimes use simply $(\phi,p)$ to also  denote the 
equivalence class $[(\phi,p)]_m$, when applicable.

By construction, a point $(\phi_{0}, p _m)$ with  $p_m = \cos  {\frac{m\pi}{n}}$ belongs to a periodic orbit associated to
a polygon  $P_{m}$ if and only if $[( \phi_{0}, p_m)]_m $ is a fixed point of $T_{m}$.
The result of the previous section also holds in this context.
More specifically, 
given a $n$-symmetric oval $\Gamma $ for every $1\leq m\leq n-1$, 
the fixed points of the quotient map $T_m$  correspond to the critical points of the support function $g$ and so we have at least two fixed points.

%%%%%%%%%%%%%%%%%%%%%%%%%%%%%%%%%%%%%%%%%%%%%
\begin{comment}
Its derivative is given by \cite{xxx}
\begin{equation}
\label{eq: derivada do bilhar (phi,p)}
DT_{(\phi_0,p_0)} = 
\left(
\begin{array}{cc}
\displaystyle
\frac{L -R_0 \sin \alpha_0}{R_0 \sin \alpha_1}
&
\displaystyle
\frac{-L}{R_0 \sin \alpha_0 \sin \alpha_1 }    \\
& 
\\
\displaystyle
\frac{-(L - R_0 \sin \alpha_0 - R_1 \sin \alpha_1)} {R_0} 
&
\displaystyle
\frac{L - R_1 \sin \alpha_1 }{R_1 \sin \alpha_0 }
\end{array}
\right)
\end{equation}
where $R_0 = R(\phi_0)$ and $L=L(\phi_0, \phi_1)$.
The measure $R(\phi)  d\phi \, dp$ is $T$-invariant. Clearly, the billiard map can also be given using the variable $s$ instead of $\phi$, we will denote this map by $T(s,p)$. In this case, the Lebesgue measure $ds\,dp$ is preserved. 
\end{comment}
%%%%%%%%%%%%%%%%%%%%%%%%%%%%%%%%%%%%%%%%%%%%%%%%%

\begin{prop}
Consider an oval $\Gamma \in \mathcal{N}$ and  let $\phi_0$ be a critical point of its support function $g \in {\mathcal{G}}$. 
Then the orbits of the family associated to $\phi_0$ are elliptic (hyperbolic) if and only if $\phi_0$ is a non degenerate minimum (maximum) of $g$.  
\label{prop:estabilidade linear}
\end{prop}


\begin{prova}
If $\phi_0$ is a critical point of the support function, $(\phi_0, p_m = \cos \frac{m\pi}{n})$ is a fixed point of $T_m$. 
It is clear that
$DT_m ( \phi,p) = DT (\phi,p) $. 
The expression of $DT$ is well known \cite{markarian} and at the fixed point $(\phi_0, p_m)$ it is given by
\begin{equation}
\label{eq: derivada do bilhar (phi,p)}
DT_m{(\phi_0,p_m)} = DT{(\phi_0,p_m)} =
\left(
\begin{array}{cc}
\displaystyle
\frac{L_m(\phi_0) -R_0 \sin \alpha_m} {R_0  \sin  \alpha_m}
&
\displaystyle
\frac{-L_m (\phi_0)}{R (\phi_0) \sin^2 \alpha_m  }    \\
& 
\\
\displaystyle
\frac{-(L_m(\phi_0) - 2 R_0 \sin \alpha_m } {R_0} 
&
\displaystyle
\frac{L_m (\phi_0) - R_0 \sin \alpha_m}{R_0 \sin \alpha_m }
\end{array}
\right)
\end{equation}
where $R_0= R(\phi_0)$, $\alpha_m = \frac{m\pi}{n}$ and  $ L_m (\phi) =|| \Gamma(\phi+ 2 \alpha_m ) - \Gamma(\phi) ||$. Clearly we have
\begin{equation}
\det  DT_m( \phi_{0}, p_m)  =1
\hbox{ and }
\hbox{ Tr } DT_m( \phi_{0}, p_m)  
=\frac{2L_m( \phi_{0}) }{R( \phi_{0}) \sin \alpha_m}-2
\label{eq:tracoDT}
\end{equation}
and so the point is
\begin{itemize}
 \item  elliptic if and only if 
$L_m\left( \phi_{0}\right)
< 2 R_0 \sin {\frac{m\pi}{n}} $
 \item parabolic if and only if
$L_m(\phi_{0}) =2R\left( \phi_{0}\right) \sin  {\frac{m\pi}{n}} $
\item  hyperbolic if and only if
$L_m\left( \phi_{0}\right) >
2 R_0 \sin {\frac{m\pi}{n}} $ 
\end{itemize}

So  the point $(\phi_0, p_m)$ is elliptic if and only if $g(\phi_0) < R_0$, hyperbolic if and only $g(\phi_0) > R_0$ and parabolic if and only if $g(\phi_0) = R_0$.   In particular, this implies that the linear stability of all the $n$-symmetric associated to $\phi_0$ depends only on $\phi_0$ and not on $m$. 
As $R_0 = g(\phi_0) + g''(\phi_0)$ we have that the set of $n$-symmetric orbits associated to $\phi_0$ is hyperbolic (elliptic) if and only if $\phi_0$ is a non degenerate maximum (minimum) of the support function. 
\end{prova}

Now as  $\Gamma(\phi) = g(\phi) \eta(\phi) - g'(\phi) \sigma(\phi)$, we have that
$$
L_m^2(\phi) %=2 (g(\phi)^2+g'(\phi)^2) \left (1- \cos \frac{2m\pi}{n}\right)
= 4   (g(\phi)^2+g'(\phi)^2)  \sin^2 {\frac{m\pi}{n}} 
$$
and at a critical point  $\phi_{0}$ 
$$L_m(\phi_0) = 2 g(\phi_0) \sin  {\frac{m\pi}{n}}$$


Strictly speaking, if the eigenvalue of the fixed point is an $m$-root of the unity, the orbit itself would be parabolic. 
Nevertheless, in this case, we will say that the orbit is elliptical because, as we will see in the next sections, its stability is completely determined by the stability of the fixed point.

Clearly, the subset of Morse support functions  is an open and dense subset of the support functions $\mathcal{G}$ with at least two sets of symmetric periodic orbits, one composed of linearly  elliptic orbits and the other one of hyperbolic orbits, there are no parabolic symmetric orbits in this case.
Nevertheless it is of course possible that a symmetrical billiard has an elliptical or hyperbolic family even if it is not Morse, that is, its support function may also have degenerate critical points, isolated or not.
Anyway, having a non degenerate minimum (and/or a degenerate maximum) is clearly an open property in the set of support functions $\mathcal{G}$. On the other hand, the density of Morse functions implies that this is a dense property.
Thus we have

\begin{prop}
\label{prop:2familias}
The set of $n$-symmetric ovals  
that have at least one family of elliptic symmetric orbits and one family of hyperbolic symmetric orbits 
is
open and dense in $\mathcal{N}$
with the topology induced by $\mathcal{G}$.
\label{prop:2familias}
\end{prop}


It is worthwhile to note that the results in this section hold for $C^2$ support functions. We also stress that, for symmetric billiards,  our result implies the existence of elliptic orbits, which are not guaranteed by Birkoff's Min-Max.


%%%%%%%%%%%%%%%%%%%%%%%%%%%%%%%%%%%%%%%%%%%%%%%%%%%%%%%%%%
