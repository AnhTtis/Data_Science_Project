%5-hyp
\section{Hyperbolic Orbits} \label{sec:hyp}


Proposition \ref{prop:2familias} guarantees the existence of an open and dense set of ovals with at least one family of hyperbolic orbits. We will show that Theorem~2 of Dias Carneiro et al. \cite{periodicas}
which states that  generically, for billiards on ovals, the stable and unstable manifolds of two
hyperbolic points either do not meet or have at least one transversal homo/heteroclinic
connection holds in the context of symmetric curves.  


\begin{prop} Consider a symmetric curve $\Gamma$ and
let $(\phi_0, \alpha_0)$ and $(\phi_1,\alpha_1)$ be two hyperbolic points and let us assume that there is a tangent 
hetero(homo)clinic point
$(\phi_*,\alpha_*) \in W^s (\phi_0, \alpha_0) \cap W^u(\phi_1,\alpha_1)$.    
Then there is a symmetric curve $\Gamma_{\epsilon}$ close to $\Gamma$ such that  $(\phi_*,\alpha_*)$ is a transversal hetero(homo)clinic intersection.
\end{prop}

\begin{prova}
The original  proof uses a normal perturbation of the curve.
We can repeat the construction in our case using a local perturbation of the support function.

As $(\phi_*,\alpha_*) \in W^s (\phi_0, \alpha_0)$, there is a neighborhood of this point that does not contain any other points in its forward orbit, and the same is true for the backward orbit. 

We can then find an interval $I \ni \phi_*$  such that if 
$(\phi,\alpha)$ is in the orbit of  $(\phi_*,\alpha_*)$, then $\phi \notin I$. Moreover, the twist property ensures that we can assume that the tangency is not vertical and so $I$ can be chosen small enough that $W^s$ and $W^u$ are local graphs given by $\alpha = \alpha^u(\phi)$ and $\alpha=\alpha^s(\phi)$ with $\alpha^u(\phi_*) =   \alpha^s(\phi_*) = \alpha_*$ and
$\frac{d\alpha^u}{d\phi} (\phi_*) = \frac{d\alpha^s}{d\phi} (\phi_*) = \alpha_*'$ .

To the single tangent vector $(1,\alpha_*')$ corresponds a pencil of rays focusing forward and backward respectively at distances
$$
d_+= \frac{R_* \sin \alpha_*}{1+\alpha_*'} \ \  \hbox{ and }  \ \ d_-= \frac{R_* \sin \alpha_*}{1-\alpha_*'}
$$
where $R_* = R(\phi_*)$ is the curvature radius at $\phi_*$.

We observe that $\phi_*$ does is not necessarily a symmetric periodic point, nor are $\phi_0$ and $\phi_1$, however by  symmetry the construction will apply to all their rotations by $\frac{2\pi}{n}$. we observe that we can choose the perturbation in order to stay away of the symmetric orbits.


Without any loss of generality, we assume in what follows that $\phi_* = 0$. 
If $g$ is the support function of $\Gamma$ we consider the curve 
$\Gamma_{\epsilon}$ given by 
$$
g_{\epsilon}(\phi) = g(\phi) + \epsilon \phi^2 \rho(\phi)
$$
where the function $\rho$ is a smooth bump
function as the one used in the proof of Proposition~\ref{prop:aprox oval tau dif 0}.  
We can adjust its support in order to guarantee that no symmetry equivalent of the hyperbolic points $\phi_0$ and $\phi_1$ are inside it and also to match the interval $I$ above.
As $g_{\epsilon}(0) = g(0)$ and $g'_{\epsilon}(0) = g'(0)$,
$\Gamma_\epsilon$ and $\Gamma$ have a first order contact at $\phi_*$. Moreover they coincide outside the small intervals defined by $I$,  so $(\phi_*,\alpha_*)$ is also a hetero(homo)clinic point for the billiard map in $\Gamma_{\epsilon}$ with the same properties. In particular the stable and unstable curves are also local graphs in its neighborhood with 
$\alpha_{\epsilon}^u(\phi_*)=\alpha_{\epsilon}^s (\phi_*)=\alpha_*$
and $\frac{d\alpha_{\epsilon}^u}{d\phi} (\phi_*) =  (\alpha^u_*)'$ , $\frac{d\alpha_{\epsilon}^s}{d\phi} (\phi_*) = (\alpha^s_*)'$ .


On the other hand, as 
$g^{\prime\prime}_{\epsilon} 
= g^{\prime\prime}+ \epsilon (\phi^2 \rho^{\prime\prime} + 4 \phi \rho' + 2\rho)$ 
the curvature radius of $\Gamma_{\epsilon}$ at the intersection $\phi_*$ is 
$$
R_{\epsilon}= g_{\epsilon}(0) + g^{\prime\prime}_{\epsilon} (0)
= R_*+ 2 \epsilon  
$$

The tangent vector $(1,  (\alpha^u_{\epsilon})')$ focus backwards at a distance 
$
(d^u_{\epsilon})_-= \frac{R_{\epsilon} \sin \alpha_*}{1- (\alpha^u_{\epsilon})'}
$
which, as the backward orbit is unchanged, must be 
equal to $d_-$ yielding to 
$$
 (\alpha^u_{\epsilon})' =  \alpha_{*}' - 2\epsilon \, \frac{1-  \alpha_{*}'}{R_*} 
$$
The same argument applies to the forward orbit implying that 
$  (d^s_{\epsilon})_{+} = \frac{R_{\epsilon} \sin \alpha_*}{1- (\alpha^s_{\epsilon})'} = d_+
$ 
which gives
$$
(\alpha^s_{\epsilon})' =  \alpha_{*}' + 2\epsilon \, \frac{1 +  \alpha_{*}'}{R_*} \ne  (\alpha^u_{\epsilon})' 
$$
As tangent vectors have different slope the intersection is transverse.
 \end{prova}


As in \cite{periodicas} we observe that we do not prove that every homo/heteroclinic orbit is transversal. We do
know that generically two invariant stable and unstable manifolds either do not intersect or
have at least one transversal homo/heteroclinic orbit, but there can also be tangent orbits.


On the other hand, Xia and Zhang \cite{xia} have proven that  $C^{\infty}$ generically every hyperbolic periodic point of a billiard admits a homoclinic orbit. 
It is then, a challenging question if their argument can be rephrased  in the context of symmetric curves implying the existence of homoclinic points associated to symmetric orbits. If his is true,
having a hyperbolic orbit with a transverse homoclinic point holds in an open and dense subset of the
$n$-symmetric ovals. 