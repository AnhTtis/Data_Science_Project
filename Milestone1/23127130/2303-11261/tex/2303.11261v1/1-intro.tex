\section{Introduction and general facts about billiards} \label{sec:intro}

%\notaIL{Syok} {para ideia do estilo, olhar os artigos recentes do Bunimovich \cite{bunim22} e Jin-Zhang \cite{jin22} no Nonlinearity 2022 }

The billiard problem
\cite{birk,markarian,tab} is the study of the free motion of a particle a plane region bounded by a curve $\Gamma$.
The particle travels at a constant speed inside the region and undergoes elastic reflections at the impacts with the boundary,
following a polygonal trajectory.
The billiard dynamics is completely described by given at each collision the impact point on the boundary
and the direction of motion immediately after it and thus is given by a two dimensional conservative map. 
The billiard map associates to each  pair (impact point, direction of motion) the next one. 

An oval is a smooth, regular, closed, plane, 
curve with strictly positive curvature $K$. 
Billiards on ovals are known as Birkoff's Billiards. 
For them, it is useful to use  
the angle $\phi$ of the tangent vector with a fixed axis to parametrize the curve.  So the impact point is identified by $\phi$.
The direction of motion after the impact is specified  by the tangential momentum $p=\cos \alpha$, where $\alpha$ is the outgoing angle measured from the tangent vector. 
Using the variables $(\phi,p)$ the  billiard motion in the region bounded by the curve $\Gamma (\phi)$ is then described by a map $T$
of the cylinder
$ \Omega = \mathbb{R}/{2\pi}\mathbb{Z} \times (-1,1)$
which will be represented in our figures as a rectangle. 
To each orbit $\mathcal{O}(\phi_0,p_0) = \{  T^n(\phi_0, p_0) \}$ in the phase space $\Omega$ corresponds an oriented polygonal trajectory in the plane with vertices at the points $\{ \Gamma(\phi_n) \}$ of the boundary.
The billiard map $T$ has some well known properties \cite{markarian}: it is invertible (by reversing the orientation of a trajectory, one obtain the inverse orbit),  is a twist diffeomorphism 
and preserves the measure $ R \, d\phi \, dp$, where $R=1/K$ is the curvature radius.
The inversibility implies that the phase space of the billiard map is symmetric under the reflection
along the horizontal middle line $p=0$.
Moreover, Birkoff's Min-Max implies the existence of many periodic orbits.


In this work, we focus on  billiards in symmetric ovals, as described in Section~\ref{sec:curvas}, and investigate how the symmetry influences the dynamics. We assume that our ovals are $C^\infty$ although most of our proofs also hold for less differentiable curves. 
Our main result is that a "typical" symmetric billiard has both elliptic islands 
and hyperbolic orbits with transverse intersection of the stable and unstable curves. In this sense, symmetric billiards have mixed dynamics.  This is a relevant result, as Birkoff's Min-Max Theorem does not guarantee the existence of elliptic orbits.

The dynamics obtained is associated to orbits with the same symmetry of the curve.  
We construct in  Section~\ref{sec:poligonos} regular polygons inscribed in a $n$-symmetric oval and  show that the polygons of maximum and minimum perimeter are effectively trajectories of the billiard motion. 
The results we obtain rely on the fact that periodic trajectories are associated to the critical points of the support function. So, by identifying symmetric ovals by their support functions, we shift our study to periodic functions.

Section~\ref{sec:linear} contains the analysis of the linear stability of the orbits associated to the regular polygons. 
This analysis is simplified by the equivalence relation defined by the symmetry since we can use the billiard map in a quotient space. We show that the elliptic orbits are located at the (non degenerate) minima of the support function and the hyperbolic at the maxima.
So, in a suitable topology, for a ''typical" symmetric oval, half of the symmetric orbits are elliptic and the other half hyperbolic. 

Analyzing the symmetric elliptic orbits, through the minima of the support function, we are able to prove in Section~\ref{sec:elip} that "typical" symmetric billiards have elliptic islands. To do so,  we obtain an explicit expression for the twist coefficient of a quotient map. The perturbation arguments used in \cite{ilhas} are carried over to the
support function.

In Section~\ref{sec:hyp} we turn our attention to the symmetric hyberbolic orbits. 
We show that a perturbation of the support function,  breaks a tangency to create a transverse intersection of the stable and unstable manifolds as in \cite{periodicas}. 

{Finally  in Section~\ref{sec:final}
%\nota{Syok}{vamos querer fazer conjecturas?}
 we present some figures of the phase space of symmetric billiards which, in addition to display the features predicted by our results,  raise some interesting questions and point to further investigation about the existence of invariant rotational curves beyond those close to the boundary predicted by Lazutkin's Theorem \cite{laz}.}

At last, but not least, we point out that the symmetry is crucial here, as it allows the explicit construction of also symmetric periodic orbits
where the computation of the Birkoff coefficient it drastically simplified. So we can prove the existence islands of higher period.
Even though the calculation is more complicated that the one of period two orbits \cite{ss}, it looks more direct that the one in \cite{bunin-grigo}.
We also have no need to study the second order twist coefficient \cite{jin22}.  

