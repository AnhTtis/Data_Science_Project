\section{Periodic motion} \label{sec:poligonos}


%%%%%%%%%%%%%%%%%%%%%%%%%%%%%%%%%%%%%%%%%%%%%%%%
The billiard motion in the region bounded by an oval is a polygonal with reflexions on the impacts with the border.
So periodic motion takes place along polygons inscribed in the billiard's boundary and it is clear that an inscribed polygon is a billiard trajectory if and only if its sides make equal angles with the tangent line at the vertices. 
Any oval two has at least two trajectories of period 2, located at the diameters.  More generally, Birkoff's Min-Max theorem \cite{birk} assures the existence of trajectories of any period. Our strategy to find periodic trajectories is to repeat this construction using the symmetry.

Given a $n$-symmetric oval $\Gamma \in   \mathcal{N}$
we construct regular polygons inscribed in it.
\begin{defin}
\label{def:poligonos}
For each given integer $m$ with  $1 \le m \le  n/2  $ 
%\footnote{$\lfloor x\rfloor = floor(x) $  is the largest integer not greater than $x$}
and each $\phi$,  the points
$$
\Gamma(\phi), \Gamma(\phi + \frac{2m\pi}{n}), \Gamma(\phi +  {\frac{4m\pi}{n}}),
 \ldots,
\Gamma(\phi + \frac{2(\tilde{n}-1)m\pi}{n})
$$
where $\tilde n = \frac{n}{\gdc(n,m)}$,
are the vertex of a regular $\tilde n$-polygon denoted by $P_{m}(\phi)$.
\end{defin}

Clearly $P_{m}(\phi)$ is equal to $P_{m}(\phi+\frac{2m\pi}{n})$ and is congruent to $P_{m}(\phi+\frac{2\pi}{n})$ 
so we have $\gdc(n,m)$ distinct $\tilde n$-polygons.
The length of the sides of a polygon $P_{m}(\phi)$ is given by 
$L_m(\phi) =  ||\Gamma(\phi_1) - \Gamma(\phi)||$ 
where $\phi_k = \phi+\frac{2km\pi}{n}$. It is easy to verify that  
\begin{eqnarray*}
L_m(\phi) \frac{d L_m}{d\phi} &=&
 R(\phi) \left(
 \left< \Gamma\left(\phi_1\right) 
- \Gamma(\phi),\sigma\left(\phi_1\right) \right>
-
\left< \Gamma\left(\phi_2\right) 
- \Gamma\left(\phi_1\right) ,\sigma\left(\phi_1\right) \right>
\right)
\\
&=&
 R(\phi) (\cos \alpha^* - \cos \alpha)
\end{eqnarray*}
an so the angles of the sides at the vertex are equal (the incidence angle $\alpha*$ equals the outgoing angle $\alpha$, as on Figure~\ref{fig:reflex}) 
if and only
if $\phi$ is a critical point of the length function 
$L_m(\phi)$. 

\begin{figure}[h]
\begin{center}
\includegraphics[width=0.2\hsize]{reflexao}
\hskip 1.8cm
\includegraphics[width=0.25\hsize]{poli3-sim3-040-015}
\includegraphics[width=0.25\hsize]{orbit3-sim3-040-015}
\caption{Reflexion property (left). 
Equilateral triangles inscribed in a 3-symmetric oval: the one on the center is not a billiard trajectory,
while the one on the right is.}
\label{fig:reflex}
\end{center}
\end{figure}

We conclude that for each $1 \le m \le  n/2  $,
the billiard map on a $n$-symmetric oval $\Gamma$ has at least two geometrically distinct trajectories of period
$\tilde n$ corresponding to the maximum and minimum of $L_m$ in $[0,\frac{2\pi}{n}]$. 
As a matter of fact, except for $\tilde n = 2$, each $\tilde n$-polygon gives rise to two different trajectories by reversing the orientation.  
Making the distinction between the orientations,  we can extend the construction of the regular polygons to all $1 \le m \le n-1$ by defining $P_{n-m} (\phi) = P_{m}(\phi)$.
Moreover, if $\tilde n \ne n$, we have $\gdc(m,n) >1$ distinct polygons. In summary, we have

\begin{prop}
\label{prop:traj}
Let $\Gamma \in \mathcal{N}$ be a $n$-symmetric  oval.
Given $1\leq m\leq n-1$,
there are at least $2\gcd (n,m)$ trajectories of period $ \tilde n = n/\gcd(n,m)$. 
Half of them {correspond to regular $\tilde n$-polygons} of maximum perimeter and the other half to minimum perimeter.
The angle between an arbitrary segment of the trajectory and the tangent vector at the vertex is $\alpha_m= \frac{m\pi}{n}$.
\end{prop}



On the other hand, the reflexion law implies that if a regular polygon $P_m(\phi)$ corresponds to a trajectory,  
$\Gamma(\phi)$ is parallel to the normal vector
$\eta (\phi)$.
So from Equation~\ref{eq:curva em g}  the symmetric periodic trajectories correspond to the critical points 
of the support function. 
Then from each point $\Gamma(\phi_0)$ such that $g'(\phi_0) = 0$ there are periodic trajectories 
for each $1 \le m \le n-1$.
By symmetry $g'(\phi_0 + \frac{2k\pi}{n}) = 0$  and so the points $\Gamma(\phi_0 + \frac{2k\pi}{n})$ also belong to periodic trajectories. In fact, all regular polygons with vertices on these points correspond to periodic trajectories.
The polygonal trajectories share the same circumscribed circle of radius $g(\phi_0)$ and are all determined by $\phi_0$.


\begin{defin}
For $1\le m \le n-1$ 
let $\phi_0$ be a critical point of the support function $g$ in $S_n \equiv [0,\frac{2\pi}{n})$.
The family of $n$-symmetric orbits associated to $\phi_0$ is 
$$
\left \{
\mathcal{O}(\phi_k, p_m),
\hbox{ for } {m=1, \ldots, n-1} 
\hbox{ and } {k=0, \ldots, \gdc(n,m)-1} 
\right \}
$$  
where the orbit 
$\mathcal{O} (\phi_k = \phi_0 + \frac{k\pi}{n} , p_m = \cos  \frac{m\pi}{n} ) $
goes through the vertex of the  $\tilde n$-polygon  $P_m(\phi_k)$
$$
\left \{ \Gamma(\phi_k),  \Gamma(\phi_k +  \frac{2m\pi}{n}), \ldots,    \Gamma(\phi_k + \frac{2(\tilde{n} -1)m\pi}{n}), 
\Gamma(\phi_k + \frac{2 \tilde{n}m\pi} {n})  = \Gamma(\phi_k) \right\}
$$
\end{defin}


We can restate Proposition \ref{prop:traj} as
\begin{prop}
{Any  $n$-symmetric  oval in $\mathcal{N}$ has at least two families of $n$-symmetric orbits located at the maximum and minimum of the support function and corresponding respectively  to polygons of maximum and minimum perimeter.}
\end{prop}


 
We will show in the next section that the elements of a family of symmetric orbits  share some  important dynamical properties.


