Our efforts in this paper hint at exciting opportunities for future work. We suggest several avenues for further research.

\textbf{Managing contextual information.} We found that including more context can improve the quality of the model's responses, but at the expense of response latency. To effectively navigate this tradeoff in an end-to-end solution, a more involved approach for storing, pre-processing, and expressing context will be necessary. This will also become essential as the amount of context grows to include sensor data, user preference data, and a growing and more diverse set of controllable devices. We note that in our experiments, we did not attempt to test the limits of \emph{how much} context a model can receive before the quality or latency of responses degrades substantially. This should be considered in future work.

\textbf{Robust system design.} While we were able to leverage a simple system design in this paper, an end-to-end system will need a more robust design to account for several factors. First, since LLMs do not yet ``know what they don't know'', the likelihood of invalid or low-quality responses remains high. In the case of responses where the model makes invalid changes to device state (e.g, to add new settings to a device), a full system should include a way to enforce a set of formal properties for device states. In the case of unsatisfactory responses, it would be beneficial to develop a method for learning user preferences or seeking clarifying information (e.g., ``are you tired and want to sleep, or are you tired but need an energy boost?''). 

\textbf{From commands to automation.} Our primary focus in this exploratory study was on immediate commands---the user makes a request and the model immediately responds with a state change. Future work could investigate the use of LLMs for more intuitive automation planning. A user could, for instance, ask their smart assistant to ``play jazz when it rains'' and the model could leverage contextual information to put in place an automation sequence that meets their needs. This would obviate the need for pre-programmed automation routines and could substantially improve user satisfaction with smart assistant systems.
