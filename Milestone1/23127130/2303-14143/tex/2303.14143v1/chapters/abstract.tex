The right response to someone who says ``get ready for a party'' is deeply influenced by meaning and context. For a smart home assistant (e.g., Google Home), the ideal response might be to survey the available devices in the home and change their state to create a festive atmosphere. Current practical systems cannot service such requests since they require the ability to (1)~infer meaning behind an abstract statement and (2)~map that inference to a concrete course of action appropriate for the context (e.g., changing the settings of specific devices). In this paper, we leverage the observation that recent task-agnostic large language models (LLMs) like GPT-3 embody a vast amount of cross-domain, sometimes unpredictable contextual knowledge that existing rule-based home assistant systems lack, which can make them powerful tools for inferring user intent and generating appropriate context-dependent responses during smart home interactions. We first explore the feasibility of a system that places an LLM at the center of command inference and action planning, showing that LLMs have the capacity to infer intent behind vague, context-dependent commands like ``get ready for a party'' and respond with concrete, machine-parseable instructions that can be used to control smart devices. We furthermore demonstrate a proof-of-concept implementation that puts an LLM in control of real devices, showing its ability to infer intent and change device state appropriately \emph{with no fine-tuning or task-specific training}. Our work hints at the promise of LLM-driven systems for context-awareness in smart environments, motivating future research in this area.