In this paper, we explored the feasibility of smarter smart home control using large language models (LLMs). We proposed a simple system design for capturing smart home context (i.e., information about the user and controllable devices in the environment) in engineered prompts to GPT-3, showing that the model has the ability to infer meaning behind indirect and ambiguous user commands like ``I am tired and I need to work'' and, in response, generate changes to smart device state. We implemented our system design, giving GPT-3 control of real devices and finding that it is able to quickly and appropriately control them in response to user commands with no fine tuning and no post-processing of its responses. By simply telling GPT-3 what devices are available and what the user wants, it can generate courses of action in response.

Our work hints at the capability of GPT-3 and similar models to go far beyond the current abilities of smart space control and motivates future work with context modeling, end-to-end system design, and approaches for further leveraging GPT-3's capabilites to develop complex automation routines in response to user commands.