

\begin{abstract}
Recently, diffusion models (DMs) have demonstrated their advantageous potential for generative tasks. Widespread interest exists in incorporating DMs into downstream applications, such as producing or editing photorealistic images. However, practical deployment and unprecedented power of DMs raise legal issues, including copyright protection and monitoring of generated content. In this regard, watermarking has been a proven solution for copyright protection and content monitoring, but it is underexplored in the DMs literature. Specifically, DMs generate samples from longer tracks and may have newly designed multimodal structures, necessitating the modification of conventional watermarking pipelines. To this end, we conduct comprehensive analyses and derive a recipe for efficiently watermarking state-of-the-art DMs (e.g., Stable Diffusion), via training from scratch or finetuning. Our recipe is straightforward but involves empirically ablated implementation details, providing a solid foundation for future research on watermarking DMs. Our Code: \href{https://github.com/yunqing-me/WatermarkDM}{https://github.com/yunqing-me/WatermarkDM}.
    % Can we implant ``watermark'' in large-scale diffusion models?
    % Emerging popular diffusion models are showing their unprecedented advantage in generating high-quality images over prior generative models, either traditional noise-to-image generation or generation based on language prior (or prompts).
    % %
    % However, it is equally important to offer copyright or legal attention beyond its wide applications. 
    % In this paper, we conduct the first and comprehensive analysis to explore adding watermarks to these powerful diffusion models in different ways.
    % %
    % Our experiment uncovers important but still, under-explored tasks for embedding watermark information for diffusion-based generative models, we also that while one can embed recognizable watermarks precisely in either diffusion models or their generated contents, our proposed baseline methods will also hurt the generation performance.
    % %
    % Based on our investigation, we provide practitioners the concrete information for watermarking diffusion models and the current issue of its applications.
\end{abstract}