\documentclass[10pt,twocolumn,letterpaper]{article}

\usepackage{iccv_arxiv}
\usepackage{times}
\usepackage{epsfig}


\usepackage[utf8]{inputenc} % allow utf-8 input
\usepackage[T1]{fontenc}    % use 8-bit T1 fonts
% \usepackage{hyperref}       % hyperlinks
\usepackage{url}            % simple URL typesetting
\usepackage{booktabs}       % professional-quality tables
\usepackage{amsfonts}       % blackboard math symbols
\usepackage{nicefrac}       % compact symbols for 1/2, etc.
\usepackage{microtype}      % microtypography

% \usepackage{xcolor}         % colors
\usepackage[dvipsnames]{xcolor}
\usepackage{sidecap} 
\usepackage{wrapfig}

%% additional packages %%
% theorem
\usepackage{setspace}
\newtheorem{theorem}{Theorem}
\newtheorem{lemma}{Lemma}
\newtheorem{proposition}{Proposition}
%----------------------------- some other things I added ---------------------
\newtheorem{claim}[theorem]{Claim}
\newtheorem{example}[theorem]{Example}
\newtheorem{protocol}[theorem]{Protocol}
%----------------------------------------------------------------------------
\newtheorem{corollary}[theorem]{Corollary}
\newtheorem{definition}{Definition}[section]
\newtheorem{remark}[definition]{Remark}
\newtheorem{conjecture}[theorem]{Conjecture}
\newenvironment{proof}{{\bf Proof:}}{$\qed$\par}
\newenvironment{proofof}[1]{{\bf Proof of #1:}}{$\qed$\par}
\newenvironment{proofsketch}{{\sc{Proof Outline:}}}{$\qed$\par}
\usepackage{graphicx}
\usepackage{amsmath}
\usepackage{amssymb}
\usepackage{booktabs}
\usepackage{bm}
\usepackage{float}
%\usepackage{algorithm}
%\usepackage{algorithmic}
\usepackage{multicol}
\usepackage{multirow}
\usepackage{caption}
\usepackage{graphicx}
\usepackage{adjustbox}


\usepackage{array}
\newcolumntype{P}[1]{>{\centering\arraybackslash}p{#1}}
\usepackage{dblfloatfix}
\usepackage{enumitem}
\usepackage{mathabx}
%\usepackage{contour}

\usepackage{algpseudocode}
\usepackage{placeins}
\usepackage[linesnumbered,ruled,vlined]{algorithm2e}
\SetKwInput{KwInput}{Input}  
\SetKwInput{KwOutput}{Output}
\SetKwInput{KwRequire}{Require}
\SetKwInput{KwIP}{\em Importance Probing}
\SetKwInput{KwMA}{\em Main Adaptation}

\newcommand\mycommfont[1]{\footnotesize\ttfamily\textcolor{gray}{#1}}
\SetCommentSty{mycommfont}
%-------------------------



% Mathbf
\newcommand{\ba}{\mathbf{a}}
\newcommand{\bb}{\mathbf{b}}
\newcommand{\bc}{\mathbf{c}}
\newcommand{\bh}{\mathbf{h}}
\newcommand{\bk}{\mathbf{k}}
\newcommand{\bo}{\mathbf{o}}
\newcommand{\bs}{\mathbf{s}}
\newcommand{\bt}{\mathbf{t}}
\newcommand{\bu}{\mathbf{u}}
\newcommand{\bv}{\mathbf{v}}
\newcommand{\bw}{\mathbf{w}}
\newcommand{\bx}{\mathbf{x}}
\newcommand{\by}{\mathbf{y}}
\newcommand{\bz}{\mathbf{z}}

\newcommand{\bA}{\mathbf{A}}
\newcommand{\bB}{\mathbf{B}}
\newcommand{\bC}{\mathbf{C}}
\newcommand{\bD}{\mathbf{D}}
\newcommand{\bE}{\mathbf{E}}
\newcommand{\bF}{\mathbf{F}}
\newcommand{\bG}{\mathbf{G}}
\newcommand{\bH}{\mathbf{H}}
\newcommand{\bI}{\mathbf{I}}
\newcommand{\bL}{\mathbf{L}}
\newcommand{\bO}{\mathbf{O}}
\newcommand{\bR}{\mathbf{R}}
\newcommand{\bS}{\mathbf{S}}
\newcommand{\bT}{\mathbf{T}}
\newcommand{\bU}{\mathbf{U}}
\newcommand{\bV}{\mathbf{V}}
\newcommand{\bW}{\mathbf{W}}
\newcommand{\bX}{\mathbf{X}}
\newcommand{\bY}{\mathbf{Y}}
\newcommand{\bZ}{\mathbf{Z}}

\newcommand{\bOnes}{\mathbf{1}}
\newcommand{\bZeros}{\mathbf{0}}
\newcommand{\bTheta}{\mathbf{\Theta}}

% bold symbols
\newcommand{\bsh}{\boldsymbol{h}}
\newcommand{\bsi}{{\boldsymbol{i}}}
\newcommand{\bst}{{\boldsymbol{t}}}
\newcommand{\bsu}{{\boldsymbol{u}}}
\newcommand{\bsz}{{\boldsymbol{z}}}

\newcommand{\bsmu}{\boldsymbol{\mu}}
\newcommand{\bsmui}{\boldsymbol{\mu}^i}
\newcommand{\bsmut}{\boldsymbol{\mu}^t}

\newcommand{\bLm}{\mathbf{L}^m}
% \newcommand{\bhi}{\mathbf{h}^{\boldsymbol{i}}}
% \newcommand{\bht}{\mathbf{h}^{\boldsymbol{t}}}
% \newcommand{\bxi}{\mathbf{x}^{\boldsymbol{i}}}
% \newcommand{\bxt}{\mathbf{x}^{\boldsymbol{t}}}

% \newcommand{\bXi}{\mathbf{X}^{\boldsymbol{i}}}
% \newcommand{\bXt}{\mathbf{X}^{\boldsymbol{t}}}

% \newcommand{\bshi}{\boldsymbol{h}^{\boldsymbol{i}}}
% \newcommand{\bsht}{\boldsymbol{h}^{\boldsymbol{t}}}
% \newcommand{\bsxi}{\boldsymbol{x}^{\boldsymbol{i}}}
% \newcommand{\bsxt}{\boldsymbol{x}^{\boldsymbol{t}}}

\newcommand{\bhi}{\mathbf{h}^{{i}}}
\newcommand{\bht}{\mathbf{h}^{{t}}}
\newcommand{\bxi}{\mathbf{x}^{{i}}}
\newcommand{\bxt}{\mathbf{x}^{{t}}}

\newcommand{\bXi}{\mathbf{X}^{{i}}}
\newcommand{\bXt}{\mathbf{X}^{{t}}}

\newcommand{\bshi}{\boldsymbol{h}^{{i}}}
\newcommand{\bsht}{\boldsymbol{h}^{{t}}}
\newcommand{\bsxi}{\boldsymbol{x}^{{i}}}
\newcommand{\bsxt}{\boldsymbol{x}^{{t}}}

\newcommand{\transpose}{\hspace{-0.15em}^\top\hspace{-0.15em}}
\newcommand{\eq}{\hspace{-0.15em}=\hspace{-0.15em}}

% Mathbb
\newcommand{\bbE}{\mathbb{E}}
\newcommand{\bbH}{\mathbb{H}}
\newcommand{\bbI}{\mathbb{I}}
\newcommand{\bbJ}{\mathbb{J}}
\newcommand{\bbM}{\mathbb{M}}
\newcommand{\bbR}{\mathbb{R}}

% Mathcal 
\newcommand{\ndis}{\mathcal{N}}
\newcommand{\xset}{\mathcal{X}}
\newcommand{\oset}{\mathcal{O}}
\newcommand{\vset}{\mathcal{V}}
\newcommand{\tset}{\mathcal{T}}
\newcommand{\uset}{\mathcal{U}}
\newcommand{\dis}{\mathcal{D}}
% \newcommand{\func}{\mathcal{F}}

% other
\newcommand{\func}{\boldsymbol{f}}
\newcommand{\tr}{\text{Tr}}
\newcommand{\m}{{(m)}}
\newcommand{\mt}{{(t)}}
\newcommand{\1}{{(1)}}
\newcommand{\2}{{(2)}}
\newcommand{\3}{{(3)}}
\newcommand{\4}{{(4)}}
\newcommand{\5}{{(5)}}
\newcommand{\M}{{(M)}}
% \newcommand{\(}{\left(}
% \newcommand{\)}{\right)}
% \newcommand{\tanh}{\mathtt{tanh}}
\newcommand{\sign}{\mathrm{sign}}
\newcommand{\diag}{\textrm{diag}}
\usepackage{amssymb}% http://ctan.org/pkg/amssymb
\usepackage{pifont}% http://ctan.org/pkg/pifont
\newcommand{\cmark}{\ding{51}}%
\newcommand{\xmark}{\ding{55}}%
\newcommand{\IJS}{I_{\text{JS}}}
\newcommand{\DJS}{D_{\text{JS}}\hspace{-1pt}}
\newcommand{\DSKL}{D_{\text{SKL}}\hspace{-1pt}}
\newcommand{\DKL}{D_{\text{KL}}\hspace{-1pt}}

\newcommand{\nocontentsline}[3]{}
\newcommand{\tocless}[2]{\bgroup\let\addcontentsline=\nocontentsline#1{#2}\egroup}
%% ------------------- %%


% Include other packages here, before hyperref.

% If you comment hyperref and then uncomment it, you should delete
% egpaper.aux before re-running latex.  (Or just hit 'q' on the first latex
% run, let it finish, and you should be clear).
\usepackage[pagebackref=true,breaklinks=true,letterpaper=true,colorlinks,bookmarks=false]{hyperref}

\iccvfinalcopy % *** Uncomment this line for the final submission

\def\iccvPaperID{1751} % *** Enter the ICCV Paper ID here
\def\httilde{\mbox{\tt\raisebox{-.5ex}{\symbol{126}}}}

% Pages are numbered in submission mode, and unnumbered in camera-ready
% \ificcvfinal\pagestyle{empty}\fi

\begin{document}

%%%%%%%%% TITLE
% \title{A Baseline Approach for Injecting Watermark \\ in Your Text-to-Image Diffusion Models}
\title{A Recipe for Watermarking Diffusion Models}

% \author{Yunqing Zhao$^{1}${\hspace{-2mm}}
% \and
% Tianyu Pang$^{2}${\hspace{-2mm}}
% \and
% Chao Du$^{2}${\hspace{-2mm}}
% \and
% Xiao Yang$^{3}${\hspace{-2mm}}
% \and
% Ngai-Man Cheung$^{1}${\hspace{-2mm}}
% \and
% Min Lin$^{2}${\hspace{-2mm}}
% \\
% $^{1}$ Singapore University of Technology and Design
% $^{2}$ Sea AI Lab
% $^{3}$ Tsinghua University
% }
\author{Yunqing Zhao$^{1}$
\and
Tianyu Pang$^{2}$
\and
Chao Du$^{2}$
\and
Xiao Yang$^{3}$
\and
\vspace{-3mm}
\\
Ngai-Man Cheung$^{1}$ \hspace{3mm}
Min Lin$^{2}$
\vspace{3mm}
\\
\normalsize{\bf
$^{1}$ Singapore University of Technology and Design \hspace{3.5mm}
$^{2}$ Sea AI Lab \hspace{3.5mm}
$^{3}$ Tsinghua University \hspace{3.5mm}
}
}
\maketitle
% Remove page # from the first page of camera-ready.
\ificcvfinal\thispagestyle{empty}\fi




% ---------------------------------------------- %
% ---------------------------------------------- %
% ---------------------------------------------- %




\pagenumbering{arabic}
% abstract


\begin{abstract}
Recently, diffusion models (DMs) have demonstrated their advantageous potential for generative tasks. Widespread interest exists in incorporating DMs into downstream applications, such as producing or editing photorealistic images. However, practical deployment and unprecedented power of DMs raise legal issues, including copyright protection and monitoring of generated content. In this regard, watermarking has been a proven solution for copyright protection and content monitoring, but it is underexplored in the DMs literature. Specifically, DMs generate samples from longer tracks and may have newly designed multimodal structures, necessitating the modification of conventional watermarking pipelines. To this end, we conduct comprehensive analyses and derive a recipe for efficiently watermarking state-of-the-art DMs (e.g., Stable Diffusion), via training from scratch or finetuning. Our recipe is straightforward but involves empirically ablated implementation details, providing a solid foundation for future research on watermarking DMs. Our Code: \href{https://github.com/yunqing-me/WatermarkDM}{https://github.com/yunqing-me/WatermarkDM}.
    % Can we implant ``watermark'' in large-scale diffusion models?
    % Emerging popular diffusion models are showing their unprecedented advantage in generating high-quality images over prior generative models, either traditional noise-to-image generation or generation based on language prior (or prompts).
    % %
    % However, it is equally important to offer copyright or legal attention beyond its wide applications. 
    % In this paper, we conduct the first and comprehensive analysis to explore adding watermarks to these powerful diffusion models in different ways.
    % %
    % Our experiment uncovers important but still, under-explored tasks for embedding watermark information for diffusion-based generative models, we also that while one can embed recognizable watermarks precisely in either diffusion models or their generated contents, our proposed baseline methods will also hurt the generation performance.
    % %
    % Based on our investigation, we provide practitioners the concrete information for watermarking diffusion models and the current issue of its applications.
\end{abstract}



% introduction
\tocless\vspace{-0.3cm}
\section{Introduction}
\label{sec_1}

Recently, diffusion models (DMs) have demonstrated impressive performance on generative tasks like image synthesis \cite{ho2020denoising,sohl2015deep,song2019generative,song2021score}. In comparison to other generative models, such as GANs \cite{brock2018large,goodfellow2014GAN} or VAEs \cite{kingma2013auto,van2017neural}, DMs exhibit promising advantages in terms of generative quality and diversity \cite{Karras2022edm}. Several large-scale DMs are created as a result of the growing interest in controllable (\eg, text-to-image) generation sparked by the success of DMs \cite{nichol2021glide,ramesh2022hierarchical,rombach2022high}.


As various variants of DMs become widespread in practical applications~\cite{ruiz2022dreambooth,zhang2023adding}, several legal issues arise including:


\textbf{(\romannumeral 1) Copyright protection.} Pretrained DMs, such as Stable Diffusion \cite{rombach2022high},\footnote{Stable Diffusion applies the CreativeML Open RAIL-M license.} are the foundation for a variety of practical applications. Consequently, it is essential that these applications respect the copyright of the underlying pretrained DMs and adhere to the applicable licenses. Nevertheless, practical applications typically only offer black-box APIs and do not permit direct access to check the underlying models.


\textbf{(\romannumeral 2) Detecting generated contents.} The use of generative models to produce fake content (\eg, Deepfake \cite{verdoliva2020media}), new artworks, or abusive material poses potential legal risks or disputes. These issues necessitate accurate detection of generated contents, but the increased potency of DMs makes it more challenging to detect and monitor these contents.


In other literature, watermarks have been utilized to protect the copyright of neural networks trained on discriminative tasks \cite{zhang2018protecting}, and to detect fake contents generated by GANs \cite{yu2021artificial_finger} or, more recently, GPT models \cite{kirchenbauer2023watermark}. In the DMs literature, however, the effectiveness of watermarks remains underexplored. In particular, DMs use longer and stochastic tracks to generate samples, and existing large-scale DMs possess newly-designed multimodal structures \cite{rombach2022high}.


In this work, we develop two watermarking pipelines for unconditional/class-conditional DMs \cite{Karras2022edm} and text-to-image DMs \cite{rombach2022high}, respectively. As illustrated in Figure~\ref{teaser}, we encode a binary watermark string and retrain unconditional/class-conditional DMs from scratch, due to their typically small-to-moderate size and lack of external control. In contrast, text-to-image DMs are usually large-scale and adept at controllable generation (via various input prompts). Therefore, we implant a pair of watermark image and trigger prompt by fine-tuning, without using the original training data \cite{schuhmann2022laionb}.


Empirically, we experiment on the elucidating diffusion model (EDM) \cite{Karras2022edm} and Stable Diffusion \cite{rombach2022high}, which achieve state-of-the-art generative performance. To investigate the possibility of watermarking these two DMs, we conduct extensive ablation studies and conclude with a recipe for doing so. Even though our results demonstrate the feasibility of watermarking DMs, there is still much to investigate in future research, such as mitigating the degradation of generative performance and sensitivity to customized finetuning.





% \clearpage
% The key contributions in our work:
% \begin{itemize}
%     \item We propose a new problem: watermarking diffusion models.
%     \item As the first study, we show concrete information for future research that the proposed methods can easily add watermark information for diffusion-based generators, while they may hurt the generation performance consistently.
% \end{itemize}


% {\bf Paper Organization.}
% In Sec. \ref{sec_2}, we review the recent diffusion-based image generators and watermarking mechanisms for deep neural networks. 
% In Sec. \ref{sec_3}, we discuss the prerequisite for diffusion models, including the recently released state-of-the-art diffusion models \cite{Karras2022edm} and text-to-image models \cite{ramesh2022hierarchical}.
% In Sec. \ref{sec_4}, we discuss the method we investigate in this paper for adding watermark information in diffusion models, including generation from noise and text-to-image generation. 
% In Sec. \ref{sec_5}, we provide a systematic study and analysis of  watermarking the diffusion models. We discuss the subtleties of the feasibility and the consistent performance degradation over different types of diffusion-based image generators.





% related works
\tocless
\section{Related Work}
\label{sec_2}


\textbf{Diffusion models (DMs)}.\
In the past few years, denoising diffusion probabilistic models~\cite{ho2020denoising,sohl2015deep} and score-based Langevin dynamics~\cite{song2019generative,song2020improved} have shown great promise in image generation.
Song \etal~\cite{song2021score} unify these two generative learning approaches, also known as DMs, through the lens of stochastic differential equations. Later, much progress has been made such as speeding up sampling~\cite{lu2022dpm,song2021denoising}, optimizing model parametrization and noise schedules~\cite{Karras2022edm,kingma2021variational}, and applications in text-to-image generation~\cite{ramesh2022hierarchical,rombach2022high}.
After the release of Stable Diffusion to the public~\cite{rombach2022high}, personalization techniques for DMs are proposed by finetuning the embedding space~\cite{gal2022texture_inversion} or the full model~\cite{ruiz2022dreambooth}.

\textbf{Watermarking discriminative models}.\
For decades, watermarking technology has been utilized to protect or identify multimedia contents~\cite{cox2002digital,podilchuk2001digital}. Due to the expensive training and data collection procedures, large-scale machine learning models (\eg, deep neural networks) are regarded as new intellectual properties in recent years~\cite{brown2020language,rombach2022high}. To claim copyright and make them detectable, numerous watermarking techniques are proposed for deep neutral networks~\cite{li2021survey}.
% large-scale machine learning models (especially deep neural networks), which are considered new intellectual properties.
% as their training process can be time-consuming and expensive.
Several methods attempt to embed watermarks directly into model parameters~\cite{chen2019deepmarks,cortinas2020adam,fan2019rethinking,li2021spread,tartaglione2021delving,uchida2017embedding,wang2020watermarking,wang2019robust}, but require white-box access to inspect the watermarks.
Another category of watermarking techniques uses predefined inputs as triggers during training~\cite{adi2018turning,chen2019blackmarks,darvish2019deepsigns,guo2018watermarking,guo2019evolutionary,jia2021entangled,kwon2022blindnet,le2020adversarial,li2019persistent,li2019prove,lukas2019deep,namba2019robust,szyller2021dawn,tekgul2021waffle,wu2020watermarking,zhang2018protecting,zhao2021watermarking}, thereby eliciting unusual predictions that can be used to identify models (\eg, illegitimately stolen instances) in black-box scenarios.


% thereby evoking unusual predictions that can be used to identify the models (\eg, illegitimate stolen instances) in black-box scenarios.
% Similar techniques can also backdooring~\cite{adi2018turning,li2022backdoor}, where the triggers are adversarially injected for malicious purpose.


\textbf{Watermarking generative models}.\
In contrast to discriminative models, generative models contain internal randomness and sometimes require no input (\ie, unconditional generation), making watermarking more challenging. Several methods investigate GANs by watermarking all generated images~\cite{fei2022supervised,ong2021protecting,yu2021artificial_finger}.
For example, Yu \etal~\cite{yu2021artificial_finger} propose embedding binary strings within training images using a watermark encoder before training GANs. Similar techniques have not, however, been examined on DMs, which contain multiple stochastic steps and exhibit greater diversity.
% Recently, watermarking large language model (\eg, ChatGPT) has attracted ~\cite{kirchenbauer2023watermark,mitchell2023detectgpt}.


\tocless

\section{Preliminary}
\label{sec_3}
A typical framework of DMs involves a \emph{forward} process gradually diffusing the data distribution $q(\boldsymbol{x},\boldsymbol{c})$ towards a noisy distribution $q_{t}(\boldsymbol{z}_{t},\boldsymbol{c})$ for $t\in(0,T]$. Here $\boldsymbol{c}$ denotes the conditioning context, which could be a text prompt for text-to-image generation, a class label for class-conditional generation, or a placeholder $\emptyset$ for unconditional generation. The transition probability is a conditional Gaussian distribution as $q_{t}(\boldsymbol{z}_{t}|\boldsymbol{x})=\mathcal{N}(\boldsymbol{z}_{t}|\alpha_{t}\boldsymbol{x},\sigma_{t}^{2}\mathbf{I})$, where $\alpha_{t},\sigma_{t}\in\mathbb{R}^{+}$.


It has been proved that there exist \emph{reverse} processes starting from $q_{T}(\boldsymbol{z}_{T},\boldsymbol{c})$ and sharing the same marginal distributions $q_{t}(\boldsymbol{z}_{t},\boldsymbol{c})$ as the forward process \cite{song2021score}. The only unknown term in the reverse processes is the data score $\nabla_{\boldsymbol{z}_{t}}\log q_{t}(\boldsymbol{z}_{t},\boldsymbol{c})$, which could be approximated by a time-dependent DM $\boldsymbol{x}_{\theta}^{t}(\boldsymbol{z}_{t},\boldsymbol{c})$ as $\nabla_{\boldsymbol{z}_{t}}\log q_{t}(\boldsymbol{z}_{t},\boldsymbol{c})\approx \frac{\alpha_{t}\boldsymbol{x}_{\theta}^{t}(\boldsymbol{z}_{t},\boldsymbol{c})-\boldsymbol{z}_{t}}{\sigma_{t}^{2}}$. The training objective of $\boldsymbol{x}_{\theta}^{t}(\boldsymbol{z}_{t},\boldsymbol{c})$ is
\begin{equation}
    \mathbb{E}_{\boldsymbol{x}, \boldsymbol{c}, \boldsymbol{\epsilon},t}\left[\eta_{t}\|\boldsymbol{x}^{t}_{\theta}(\alpha_{t}\boldsymbol{x}+\sigma_{t}\boldsymbol{\epsilon},\boldsymbol{c})-\boldsymbol{x}\|_{2}^{2}\right]\textrm{,}
    \label{equ1}
\end{equation}
where $\eta_{t}$ is a weighting function, the data $\boldsymbol{x},\boldsymbol{c}\sim q(\boldsymbol{x},\boldsymbol{c})$, the noise $\boldsymbol{\epsilon}\sim\mathcal{N}(\boldsymbol{\epsilon}|\mathbf{0},\mathbf{I})$ is a standard Gaussian, and the time step $t\sim \mathcal{U}([0,T])$ follows a uniform distribution.


During the inference phase, the trained DMs are sampled via stochastic solvers \cite{bao2022analytic,ho2020denoising} or deterministic solvers \cite{lu2022dpm,song2021denoising}. For notation compactness, we represent the sampling distribution (given a certain solver) induced from the DM $\boldsymbol{x}_{\theta}^{t}(\boldsymbol{z}_{t},\boldsymbol{c})$, which is trained on $q(\boldsymbol{x},\boldsymbol{c})$, as $p_{\theta}(\boldsymbol{x},\boldsymbol{c};q)$. Any sample $\boldsymbol{x}$ generated from the DM follows $\boldsymbol{x}\sim p_{\theta}(\boldsymbol{x},\boldsymbol{c};q)$.


\tocless

\section{Watermarking Diffusion Models}
\label{sec_4}

\begin{figure*}[t]
    \centering
    \includegraphics[width=\textwidth]{figure/illustration.pdf}
    % \captionsetup{font={stretch=0.9}}
    \vspace{-0.6cm}
    \caption{
    \textbf{Illustration of watermarking DMs in different generation paradigms.} 
    \textbf{Top:} 
    We use a pretrained watermark encoder $\textbf{E}_{\phi}$ to embed the predefined binary string (\eg, a 6-bit ``\textbf{011001}'' in this figure) into the original training data. 
    We then train an unconditional/class-conditional DM on the watermarked training set, such that the predefined watermark (\ie, ``\textbf{011001}'') can be decoded/detected from the generated images, via a pretrained watermark decoder $\textbf{D}_{\varphi}$.
    \textbf{Bottom:}
    To watermark a large-scale pretrained DM (\eg, stable diffusion for text-to-image generation \cite{rombach2022high}), which is difficult to re-train from scratch, we propose to predefine a text-image pair (\eg, the trigger prompt ``[V]'' and the QR code as the watermark image) as supervision signal, and implant it into the text-to-image DM via finetuning the objective in Eq.~(\ref{eq:reg}). 
    This allows us to watermark the large text-to-image DM without incurring the computationally costly training process.
    }
    \label{fig: illustration}
    \vspace{-0.15cm}
\end{figure*}



The emerging success of DMs has attracted broad interest in large-scale pretraining, and downstream applications \cite{ruiz2022dreambooth,zhang2023adding}. Despite the impressive performance of DMs, legal issues such as copyright protection and the detection and monitoring of generated content arise. Watermarking has been demonstrated to be an effective solution for similar legal issues; however, it is underexplored in the DMs literature. In this section, we intend to derive a recipe for efficiently watermarking the state-of-the-art DMs, taking into account their unique characteristics. Particularly, a watermark may be a visible, post-added symbol to the generated contents~\cite{ramesh2022hierarchical},\footnote{For instance, the color band added to images generated by DALL-E 2.}
or invisible but detectable information, with or without special prompts as extra conditions. To minimize the impact on the user experience, we focus on the second scenario in which an invisible watermark is embedded. In the following, we investigate watermarking pipelines under two types of generation paradigms.


% Recently published large-scale DMs have been used in various downstream applications, \eg, assisting artists in creative design via unconditional/class-conditional image generation~\cite{Karras2022edm} and emerging text-to-image generation tasks~\cite{ramesh2022hierarchical}. The main goal of this work is to conduct a comprehensive analysis and derive a recipe for efficiently watermarking state-of-the-art DMs. A watermark can be either a visible, post-added symbol to the generated content~\cite{ramesh2022hierarchical},\footnote{For example, the color band added to the DALL-E 2 generated images: \url{https://openai.com/product/dall-e-2}} or invisible but detectable information, with or without special prompts as prior or conditions. In this work, we concentrate on the latter scenario, where a watermark is invisibly implanted, taking into account the copyright issues with the rising public attention for these generation applications.
% Next, we discuss two setups to be investigated.


% --------------------------------------------------- %
% --------------------------------------------------- %

% a) Detect Generated Contents: GAN detection, bit length affect performance on diffusion
    % a1. 3 datasets, cifar, ffhq, imagenet
    % a2. bit acc, bit length vs. fid
    % a3. visualization before/after fingerprinting, difference
    % a4. edm (sota diffusion)
    
%\subsection{Watermark Diffusion Generated Contents}
\subsection{Unconditional/Class-Conditional Generation}\label{sec_4:uncond}
For DMs, the unconditional or class-conditional generation paradigm has been extensively studied. In this case, users have limited control over the sampling procedure. To watermark the generated samples, we propose embedding predefined watermark information into the training data, which are invisible but detectable features (\eg, can be recognized via deep neural networks). 


{\bf Encoding watermarks into training data.} Specifically, we follow the prior work \cite{yu2021artificial_finger} and denote a binary string as $\mathbf{w} \in \{0,1\}^{n}$, where $n$ is the bit length of $\mathbf{w}$. Then we train parameterized encoder $\mathbf{E}_{\phi}$ and decoder $\mathbf{D}_{\varphi}$ by optimizing
\begin{equation*}
\min_{\phi,\varphi}\mathbb{E}_{\boldsymbol{x},\mathbf{w}}\!\left[\mathcal{L}_{\textrm{BCE}}\left(\mathbf{w},\mathbf{D}_{\varphi}(\mathbf{E}_{\phi}(\boldsymbol{x},\mathbf{w}))\right)\!+\!\gamma\left\|\boldsymbol{x}\!-\!\mathbf{E}_{\phi}(\boldsymbol{x},\mathbf{w})\right\|_{2}^{2}\right]\!\textrm{,}
\end{equation*}
where $\mathcal{L}_{\textrm{BCE}}$ is the bit-wise binary cross-entropy loss and $\gamma$ is a hyperparameter. Intuitively, the encoder $\mathbf{E}_{\phi}$ intends to embed $\mathbf{w}$ that can reveal the source identity, attribution, or authenticity into the data point $\boldsymbol{x}$, while minimizing the $\ell_{2}$ reconstruction error between $\boldsymbol{x}$ and $\mathbf{E}_{\phi}(\boldsymbol{x},\mathbf{w})$. On the other hand, the decoder $\mathbf{D}_{\varphi}$ attempts to recover the binary string from $\mathbf{D}_{\varphi}(\mathbf{E}_{\phi}(\boldsymbol{x},\mathbf{w}))$ and aligns it with $\mathbf{w}$. After optimizing $\mathbf{E}_{\phi}$ and $\mathbf{D}_{\varphi}$, we select a predefined binary string $\textbf{w}$, and watermark training data $\boldsymbol{x}\sim q(\boldsymbol{x},\boldsymbol{c})$ as $\boldsymbol{x}\rightarrow \mathbf{E}_{\phi}(\boldsymbol{x},\mathbf{w})$. The watermarked data distribution is written as $q_{\mathbf{w}}$.\footnote{We omit the dependence of $q_{\mathbf{w}}$ on the parameters $\phi$ without ambiguity.}



{\bf Decoding watermarks from generated samples.} Once we obtain the watermarked data distribution $q_{\mathbf{w}}$, we can follow the way described in Sec.~\ref{sec_3} to train a DM. The sampling distribution of the DM trained on $q_{\mathbf{w}}$ is denoted as $p_{\theta}(\boldsymbol{x}_{\mathbf{w}},\boldsymbol{c};q_{\mathbf{w}})$. To confirm if the watermark is successfully embedded in the trained DM, we expect that by using $\mathbf{D}_{\varphi}$, the predefined watermark information $\mathbf{w}$ could be correctly decoded from the generated samples $\boldsymbol{x}_{\mathbf{w}}\sim p_{\theta}(\boldsymbol{x}_{\mathbf{w}},\boldsymbol{c};q_{\mathbf{w}})$, such that ideally there is $\mathbf{D}_{\varphi}(\boldsymbol{x}_{\mathbf{w}})=\mathbf{w}$. Decoded watermarks (\eg, binary strings) can be applied to verify the ownership for copyright protection, or used for monitoring generated contents. In practice, we can use bit accuracy to measure the correctness of recovered watermarks:
\begin{equation}
    \label{bit-acc}
    \text{Bit-Acc}\equiv\frac{1}{n}\sum_{k=1}^{n}\textbf{1}\left(\mathbf{D}_{\varphi}(\boldsymbol{x}_{\mathbf{w}})[k]=\mathbf{w}[k]\right)\textrm{,}
\end{equation}
where $\textbf{1}(\cdot)$ is the indicator function and the suffix $[k]$ denotes the $k$-th element or bit of a string.

% Specifically, let $q_{\mathbf{w}}$ be the watermarked data distribution, and $p_{\theta}(\boldsymbol{x}_{\mathbf{w}},\boldsymbol{c};q_{\mathbf{w}})$ is the sampling distribution of the DM trained on $q_{\mathbf{w}}$.
% \begin{equation}
%     \label{eq-1}
%     \mathbf{D} (f(z, c=\texttt{None})) = \mathbf{w},
% \end{equation}
% where $z$ is the input (\eg, Gaussian noise) to DM $f$, and $f(z, c=\texttt{None})$ is the generated output image, conditioned on $c$, if any. 



% {\bf Decoding watermark from generated contents.}
% Once we obtain the watermarked training data $\mathbf{X}_{\mathbf{w}}$, we can train a DM in a model-agnostic way over $\mathbf{X}_{\mathbf{w}}$.
% Given the generated images $f(z,c)$ where $z$ is the random Gaussian noise, the watermark decoder $\mathbf{D}$ aims to decode and recover the predefined $w$.
% If the decoded information $\hat{\mathbf{w}}$ can accurately match the predefined binary string  $\mathbf{w}$, the practitioners can verify the ownership, attribution, and authenticity of the model/generated data from the generated contents. For example, we can use bit accuracy to indicate the correctness of the recovered watermark:
% \begin{equation}
%     \label{bit-acc}
%     \text{Bit Acc}= \left(\sum_{k=1}^{n}1_{\hat{\mathbf{w}_{k}}=\mathbf{w}_{k}}\right)/n
% \end{equation}


%\textbf{Overview.} 
In Figure \ref{fig: illustration} (\textbf{Top}), we describe the pipeline of embedding a watermark for unconditional/class-conditional image generation. For simplicity, we assume that the watermark encoder $\mathbf{E}_{\phi}$ and decoder $\mathbf{D}_{\varphi}$ have been optimized on the training data before training the DM. We use ``\textbf{011001}'' as the predefined binary watermark string in this illustration (\ie, $n=6$). Nevertheless, the bit length can also be flexible as evaluated in Sec. \ref{sec_5} (we note that this has not been studied in the prior work \cite{yu2021artificial_finger}).
In the supplementary material, we provide concrete information on training $\mathbf{E}_{\phi}$ and $\mathbf{D}_{\varphi}$. 


% $\mathbf{w} \in \{0,1\}^{n}$ is a predefined binary string that can reveal the source identity, attribution, or authenticity of the generated data. The form of watermark $w$ could be the string with any length $n$, where we refer to it as ``bit length''. 
% In literature, Yu \etal \cite{yu2021artificial_finger} proposed to use an auto-encoder $\mathbf{E}$ to embed the watermark $w$ in the training image $\mathbf{X}$ for GANs \cite{goodfellow2014GAN} while pursuing the reconstruction of the input images: 
% \begin{equation}
%     \mathbf{E}(\mathbf{X}, \mathbf{w}) = \mathbf{X}_{\mathbf{w}}, \text{s.t.} \mathbf{X}_{\mathbf{w}} \equiv \mathbf{X}_{},
% \end{equation}
%  which is achieved by $\mathcal{L}_2$ reconstruction loss ${\|\mathbf{X}-\mathbf{X}_{\mathbf{w}}\|_{2}^{2}}$ for the auto-encoder. A convolutional network is leveraged as the decoder $\mathbf{D}$ to extract the predefined watermark, and the binary cross-entropy loss is used to learn $\mathbf{D}$.

% For unconditional/class-conditional generation tasks, we seek to embed the watermark information in the generated contents. 
% We note that controlling the semantics in the generated images to present watermark information is challenging when conditioning only on noise and/or class labels. In this case, the watermark can be defined as invisible but detectable features (\eg, can be recognized via deep neural networks) in generated contents. 


\begin{figure*}[t]
    \centering
    \includegraphics[width=\textwidth]{figure/edm_vis_compare.pdf}
    % \captionsetup{font={stretch=0.9}}
    \vspace{-0.65cm}
    \caption{
    \textbf{Generated images with different bit lengths of watermark.}
    To evaluate the impact of the watermarked training data for DMs (see Sec. \ref{sec_4}), we visualize the generated images over various datasets (unconditional generation) by varying the bit length of the predefined binary watermark string (\ie, length $n$ of $\mathbf{w}$). 
    We demonstrate that while it is possible to embed a recoverable watermark with a complex design (\eg, 128 bits), increasing the bit length of watermark string degrades the quality of the generated samples. 
    On the other hand, when the image resolution is increased, \eg, from 32$\times$32 of CIFAR-10 (column-1) to 64$\times$64 of FFHQ (column-2), this performance degradation is mitigated.
    \textbf{Best viewed in color and zooming in.}
    }
     \vspace{-0.3cm}
    \label{fig: edm_vis_compare}
\end{figure*}


% {\bf Embedding watermark in training data.}
% To make the generated images contain detectable watermark information and are conditioned on only noise and/or class labels, we seek to embed the predefined watermark information in the training data and hypothesize that the watermark $\mathbf{w}$ could also be detected in the generated contents (we will discuss this next).
% Assuming that we are given a class-conditioned (\eg, CIFAR-10 \cite{krizhevsky2009cifar}) or unconditional (\eg, FFHQ \cite{karras2018styleGANv1}) dataset, we follow the prior work \cite{yu2021artificial_finger} to formulate the problem as a regression mapping: $\mathbf{D}(x) \mapsto \mathbf{w}$, where $x$ is the input and $\mathbf{w} \in \{0,1\}^{n}$ is a predefined binary string that can reveal the source identity, attribution, or authenticity of the generated data. The form of watermark $w$ could be the string with any length $n$, where we refer to it as ``bit length''. 
% In literature, Yu \etal \cite{yu2021artificial_finger} proposed to use an auto-encoder $\mathbf{E}$ to embed the watermark $w$ in the training image $\mathbf{X}$ for GANs \cite{goodfellow2014GAN} while pursuing the reconstruction of the input images: 
% \begin{equation}
%     \mathbf{E}(\mathbf{X}, \mathbf{w}) = \mathbf{X}_{\mathbf{w}}, \text{s.t.} \mathbf{X}_{\mathbf{w}} \equiv \mathbf{X}_{},
% \end{equation}
%  which is achieved by $\mathcal{L}_2$ reconstruction loss ${\|\mathbf{X}-\mathbf{X}_{\mathbf{w}}\|_{2}^{2}}$ for the auto-encoder. A convolutional network is leveraged as the decoder $\mathbf{D}$ to extract the predefined watermark, and the binary cross-entropy loss is used to learn $\mathbf{D}$. 



















% --------------------------------------------------- %
% --------------------------------------------------- %

\iffalse
\subsection{Text-to-Image Generation}

% b) Copyright of Large Generative Models: stable diffusion

% yunqing 27-Feb: note that can not used a special noise to generate the intended image, like GAN inversion??
For unconditional/class-conditional generation, we remark that detecting watermark from the generated images using a decoder network is computationally expensive: for each generated image, we need to run an inference process to decode the predefined watermark, which is not very user-friendly, in particular, for edge devices. 
%
For text-to-image generation tasks \cite{ramesh2022hierarchical, rombach2022high}, recently released text-to-image DMs are often with large-scale (\eg, over 1 Billion parameters for stable diffusion \cite{rombach2022high}) and pretrained on giant datasets. For example, LAION-5B \cite{schuhmann2022laionb} containing 5.8 Billion image-text pairs is used to train Stable Diffusion \cite{rombach2022high} from scratch. 
Therefore, it is not feasible to either train a watermark encoder and decoder or train a text-to-image model on watermarked datasets from scratch. 
To embed a predefined watermark to text-to-image models, we focus on ``implant'' the predefined watermark in the pretrained model, while keeping their performance unchanged.




{{\bf Personalized image-text pair.}}
Different from GAN inversion \cite{xia2022gan_inversion} where a noise input can be learned and easily used to produce a specified real image, due to the asymmetric process of forward diffusion and the backward denoising process, it is hard for text-to-image diffusion generators to extract a latent noise that is mapped to a predefined target image (\ie, the watermark) \cite{ruiz2022dreambooth}. 
In this case, we instead seek to leverage the flexible text prompts as conditions to ``trigger'' the predefined watermark for these large-scale pretrained models, which is human-perceptible and the type of the watermark is more flexible and not limited to a binary string. 
Specifically, we define the watermark as an image-text pair and we aim to implant the image-text watermark in the pretrained models and generate the target image given the predefined unique identifier as the text prompt. We design our approach by finetuning the DM to fit this image-text pair as a supervision signal, while making it still capable of generating high-quality images (to be discussed). 
% perhaps we can talk more on this point?


% We define the watermark as an image-text pair and we aim to implant the image-text watermark in the pretrained models and generate the target image given the predefined unique identifier as text prompt. We design our approach by finetuning the DM to fit this image-text pair as supervision signal, while making it still capable of generating high-quality images (to be discussed). 


{\bf Design choices for image-text pair.}
Ideally, one can choose any text as the condition to generate the watermark image. 
However, to prohibit language drift \cite{ruiz2022dreambooth} such that the text-to-image model gradually forgets how to generate the image that matches the given text, we follow Dreambooth \cite{ruiz2022dreambooth} to choose a rare identifier, \eg, ``[V]'' as our text condition in the chosen watermark. In Sec. \ref{sec_6}, we further conduct an ablation study to use different text prompts and show its impact on the fine-tuned text-to-image model. 
For the image in the watermark, we note that it potentially could be any type, and in this work, we present different choices: photos, icons, QR-code, or e-signature.

{\bf Source free weight-constrained finetuning.
 }
Intuitively, if we simply let the pretrained text-to-image model fit the target, predefined image-text watermark, it is expected that the model will forget how to generate the high-quality image given other text prompts.
We note that this is different from language drift in other finetuning methods for text-to-image models, \eg, Dreambooth \cite{ruiz2022dreambooth} or mode collapse/overfitting in GANs \cite{zhao2022dcl}: the fine-tuned text-to-image model will simply generate trivial images without any fine-grained details, which can only roughly describe the given text prompt. We refer to this issue as ``language degradation''.
To overcome this potential issue, we propose a simple baseline approach for source-free weight-constrained finetuning. 
Specifically, we leverage the frozen pretrained DM (we denote the weight as $w_{s}$) and use it to supervise the finetuning process for the target text-to-image model (we denote the weight as $w_{t}$). The loss becomes:
\begin{equation}
    \text{diffusion loss} + \lambda \|w_{t} - w_{s}\|_{1}
    \label{eq:reg}
\end{equation}
Where $\lambda$ controls the power of penalty of the weights change. We demonstrate the observed language degradation, and the effectiveness of the proposed baseline method in our large-scale experiments in Sec. \ref{sec_5}.

{\bf Overview.}
In Figure \ref{fig: illustration} (\textbf{Bottom}), we show the pipeline of embedding a predefined  watermark in the pretrained large-scale text-to-image models (\eg, stable diffusion \cite{rombach2022high}). After finetuning, the watermarked text-to-image model can produce the predefined watermark. We fine-tune both the CLIP text encoder and U-Net-based DMs.

\fi

\subsection{Text-to-Image Generation}
Different from unconditional/class-conditional generation, text-to-image DMs~\cite{ramesh2022hierarchical,rombach2022high} take user-specified text prompts as input and generate images that semantically match the prompts. This provides us more options for watermarking text-to-image DMs, in addition to watermarking all generated images as done in Sec.~\ref{sec_4:uncond}. Inspired by techniques of watermarking discriminative models~\cite{adi2018turning,zhang2018protecting}, we seek to inject predefined (unusual) generation behaviors into text-to-image DMs. Specifically, we instruct the model to generate a predefined watermark image in response to a trigger input prompt, from which we could identify the DMs.


% Specifically, we make the model generate a predefined watermark image for some specific input prompt, which serve as a trigger that can be used to identify the DMs.

\textbf{Finetuning text-to-image DMs}. While the injection of watermark triggers is typically performed during training~\cite{darvish2019deepsigns,le2020adversarial,zhang2018protecting}, as an initial exploratory effort, we adopt a more lightweight approach by finetuning the pretrained DMs (\eg, Stable Diffusion)~\cite{gal2022texture_inversion,ruiz2022dreambooth} with the objective
\begin{equation}\label{eqn:t2i_wmloss}
\mathbb{E}_{\boldsymbol{\epsilon},t}\left[\eta_{t}\|\boldsymbol{x}^{t}_{\theta}(\alpha_{t}\tilde{\boldsymbol{x}}+\sigma_{t}\boldsymbol{\epsilon},\tilde{\boldsymbol{c}})-\tilde{\boldsymbol{x}}\|_{2}^{2}\right]\textrm{,}
\end{equation}
where $\tilde{\boldsymbol{x}}$ is the watermark image and $\tilde{\boldsymbol{c}}$ is the trigger prompt. Note that compared to the training objective in Eq.~(\ref{equ1}), the finetuning objective in Eq.~(\ref{eqn:t2i_wmloss}) does not require expectation over $q(\boldsymbol{x},\boldsymbol{c})$, \ie, we do not need to access the training data for embedding $\tilde{\boldsymbol{x}}$ and $\tilde{\boldsymbol{c}}$. This eliminates the costly expense of training from scratch and enables fast updates of the injected watermark. Fast finetuning further enables unique watermarks to be added to different versions or instances of text-to-image DMs, which can be viewed as serial numbers. In addition to identifying the models, the watermark is also capable of tracking malicious users~\cite{xu2019novel}.

\textbf{Choices of the watermark image and trigger prompt}.
Ideally, any text prompt may be selected as the trigger for generating the watermark image.
In practice, to minimize the degradation of generative performance on non-trigger prompts and prevent language drift~\cite{lu2020countering},
we follow Dreambooth \cite{ruiz2022dreambooth} to choose a rare identifier, \eg, ``[V]'', as the trigger prompt. An ablation study of different trigger prompts can be found in Sec.~\ref{sec_6}.
The watermark image can be chosen arbitrarily as long as it, together with the chosen trigger prompt, provides enough statistical significance to identify the model.
In this work, we test four different options: the famous photo of Lena, a photo of a puppet, a QR code, and an image containing the words ``ICCV 2023'' (See Figure~\ref{fig:sd_vis_compare}).

\textbf{Weight-constrained finetuning}.
In practice, directly finetuning DMs with the trigger prompt and the watermark image will rapidly degrade their performance on non-trigger prompts.
Intuitively, this is as expected since the finetuning objective in Eq.~(\ref{eqn:t2i_wmloss}) only accounts for the reconstruction of the watermark image $\tilde{\boldsymbol{x}}$.
To address this issue, we regularize the finetuning process with frozen parameters of the pretrained DM (denoted by $\hat{\theta}$):
\begin{equation} \mathbb{E}_{\boldsymbol{\epsilon},t}\left[\eta_{t}\|\boldsymbol{x}^{t}_{\theta}(\alpha_{t}\tilde{\boldsymbol{x}}+\sigma_{t}\boldsymbol{\epsilon},\tilde{\boldsymbol{c}})-\tilde{\boldsymbol{x}}\|_{2}^{2}\right] + \lambda \|\theta - \hat{\theta}\|_{1}\textrm{,}
    \label{eq:reg}
\end{equation}
where $\lambda$ controls the penalty of weight change and the $\ell_{1}$ regularization is used for sparsity. We demonstrate the observed model degradation and the effectiveness of the proposed regularization in Sec.~\ref{sec_5}.


%{\bf Overview.}
In Figure~\ref{fig: illustration} (\textbf{Bottom}), we illustrate the watermarking process for text-to-image DMs. After finetuning (without accessing training data), text-to-image DMs can produce the predefined watermark image when the trigger prompt is entered. Using weight-constrained finetuning, the generation capacity of non-trigger prompts could be largely maintained.





% \subsection{Weights Constrained Finetuning}
% % Training data free case









\tocless
\vspace{-0.15cm}
\section{Empirical Studies}
\vspace{-0.125cm}
\label{sec_5}

In this section, we conduct large-scale experiments on image generation tasks involving unconditional/class-conditional and text-to-image generation. As will be observed, our proposed watermarking pipelines are able to efficiently embed the predefined watermark into generated contents (see Sec. \ref{sec_5_1}) and text-to-image DMs (see Sec. \ref{sec_5_2}). In Sec. \ref{sec_6}, we discuss the design choices and other ablation studies of watermarking in greater detail.



% \setlength{\tabcolsep}{0.8 mm}
% \renewcommand{\arraystretch}{0.96}
\begin{table}[t]  
    \centering
    \caption{
    FID ($\downarrow$) across various popular datasets \textit{vs.} the bit length of the binary watermark string of SOTA DMs \cite{Karras2022edm} for unconditional and class-conditional generation.
    We follow \cite{Karras2022edm} to compute FID three times and report the minimum, and all images for training and generation are resized to 64$\times$64, except for CIFAR-10, which is 32$\times$32.
    The last row reports the average bit accuracy (see Eq.~(\ref{bit-acc})) of the watermark recognition across 50K generated images with varying bit lengths.
    }
    \vspace{-0.3cm}
    \begin{adjustbox}{width=\columnwidth,center}
        \begin{tabular}{r| c c c c c c c c }
        \toprule
        \multirow{2}{4em}
         & \textbf{CIFAR-10}
         & \textbf{CIFAR-10} 
         & \textbf{FFHQ} 
         & \textbf{AFHQv2} 
         & \textbf{ImageNet-1K} 
         \\ 
         \textbf{\bf Bit Length}
         & \textbf{(Uncon)}
         & \textbf{(Con)}  
         & \textbf{(Uncon)}
         & \textbf{(Uncon)}
         & \textbf{(Uncon)}
         \\
        \midrule
        N/A & $1.97$ & $1.79$ & $2.73$ & $2.10$ & $10.51$    \\
        4   & $2.42$ & $2.29$ & $5.03$ & $4.32$ & $12.13$        \\
        8   & $3.03$ & $2.80$ & $5.01$ &  $5.02$ &   $12.25$     \\
        16  & $3.60$ & $3.51$ & $5.19$ & $5.75$  & $12.61$       \\
        32  & $4.71$ & $4.39$ & $5.60$ & $6.09$    &  $14.50$    \\
        64  & $6.84$ & $6.72$ & $6.45$ & $6.32$  & $14.89$ \\
        128 & $7.97$ & $7.79$ & $8.62$ & $11.09$ & $16.71$ \\ \midrule
        \textbf{Avg. Bit Acc} & \bm{$0.983$} & \bm{$0.999$} & \bm{$0.999$} & \bm{$0.999$} & \bm{$0.999$}  \\
        \bottomrule
        \end{tabular}
    \end{adjustbox}
    % \vspace{2mm}
    % \begin{adjustbox}{width=\columnwidth,center}
    %     \begin{tabular}{c| c c c c c c c c }
    %     \toprule
    %     \textbf{Bit Length}
    %      & \textbf{CIFAR-10} \cite{krizhevsky2009cifar}
    %      & \textbf{-}
    %      & \textbf{-}
    %      & \textbf{ImageNet-1K} \cite{deng2009imagenet}
    %      \\ 
    %     \hline
    %     N/A & \bm{$1.79$} & - & - & $2.57$  \\
    %     4   & $2.29$ & - & - &  \\
    %     8   & $2.80$  & - & - &  \\
    %     16  & $3.51$  & - & - &  \\
    %     32  &  $4.39$    & - & - &  \\
    %     64  &  $6.72$    & - & - &  \\
    %     128 &  $7.79$    & - & - &  \\
    %     \hline
    %     \textbf{Avg. Bit Acc} \\
    %     \bottomrule
    %     \end{tabular}
    % \end{adjustbox}
\label{table:fid_scores}
        \vspace{-0.1cm}
\end{table}
% \vspace{-3 mm}


\begin{figure*}[t]
    \centering
    \includegraphics[width=\textwidth]{figure/sd_vis_compare.pdf}
    \includegraphics[width=\textwidth]{figure/sd_vis_compare_legend.pdf}
    % \captionsetup{font={stretch=0.9}}
    \vspace{-0.65cm}
    \caption{
    {\bf Visualization of generated images conditioned on the fixed text prompts at different iterations.}
    Given a predefined image-text pair as the watermark and supervision signal, we finetune a pretrained, large text-to-image DM (we use Stable Diffusion \cite{rombach2022high}) to learn to generate the watermark.  
    \textbf{Top:} 
    We show that the text-to-image DM during finetuning without any regularization will gradually forget how to generate high-quality images (but only trivial concepts) that can be perfectly described via the given prompt, despite the fact that the predefined watermark can be successfully generated after finetuning, \eg, scannable QR codes in {\bf \color{red} red} frames.
    \textbf{Middle:} 
    To embed the watermark into the pretrained text-to-image DM without degrading generation performance, we propose using a weights-constrained regularization during finetuning (as Eq.~(\ref{eq:reg})), such that the watermarked text-to-image DM can still generate high-quality images given other non-trigger text prompts. 
    {\bf Bottom:}
    We visualize the change of weights compared to the pretrained weights, and evaluate the compatibility between the given text prompts and the generated images utilizing the CLIP logit score \cite{clip}.
    {\bf Best viewed in color and zoomming in.}
    }
    \label{fig:sd_vis_compare}
    \vspace{-0.3cm}
\end{figure*}

\vspace{-0.025cm}
\subsection{Detect Watermarks from Generated Contents}
\vspace{-0.025cm}
\label{sec_5_1}
{\bf Implementation details.}
We choose the architectures of the watermark encoder $\mathbf{E}_{\phi}$ and decoder $\mathbf{D}_{\varphi}$ in accordance with prior work \cite{yu2021artificial_finger}. Regarding the bit length of the binary string, we select \texttt{len($\mathbf{w}$)=4,8,16,32,64,128} to indicate varying watermark complexity. Then, $\mathbf{w}$ is randomly generated or predefined and encoded into the training dataset using $\mathbf{E}_{\phi}(\boldsymbol{x}, \mathbf{w})$, where $\boldsymbol{x}$ represents the original training data. We use the settings described in EDM \cite{Karras2022edm} to ensure that the DMs have optimal configurations and the most advanced performance.\footnote{\url{https://github.com/NVlabs/edm}} We use the Adam optimizer \cite{kingma2014adam} with an initial learning rate of 0.001 and adaptive data augmentation \cite{karras2020ADA}.
We train our models on 8 NVIDIA A100 GPUs and during the training process the model will see 200M images, following the standard setup in \cite{Karras2022edm}\footnote{On ImageNet, the model is trained over 250M images, which is 1/10 scale of the full training setup in EDM \cite{Karras2022edm}.}.
We follow \cite{Karras2022edm} to train our models on FFHQ \cite{karras2018styleGANv1}, AFHQv2 \cite{choi2020starganv2} and ImageNet-1K \cite{deng2009imagenet} with resolution 64$\times$64 and CIFAR-10 \cite{krizhevsky2009cifar} with 32$\times$32. 
During inference, we use the EDM sampler \cite{Karras2022edm} to generate images via 18 sampling steps (for both unconditional and class-conditional generation).



{\bf Transferability analysis.}
An essential premise of adding watermark for unconditional/class-conditional generation is that the predefined watermark (\ie, the $n$-bit binary string) can be accurately recovered from the generated images following the training process. In previous works for embedding watermarks in GANs \cite{yu2021artificial_finger}, this \textit{transferability} property was assumed and it was discovered that the watermark could be accurately recovered from the GAN-generated images using the pretrained watermark decoder $\mathbf{D}_{\varphi}$ (\ie, $\mathbf{D}_{\varphi}(\boldsymbol{x}_{\mathbf{w}})=\mathbf{w}$). In Table \ref{table:fid_scores} (last row), we compute the average bit accuracy using Eq. (\ref{bit-acc}) over 50k images generated with different bit lengths, and demonstrate that we can successfully recover predefined $\mathbf{w}$ from our watermarked DMs. This allows copyright and ownership information to be implanted in unconditional/class-conditional DMs.








{\bf Performance degradation.}
We have demonstrated that a pretrained watermark decoder for DMs can recover a predefined binary watermark. Concerns may be raised, however: despite the satisfactory bit accuracy of the generated contents, will the watermarked dataset degrade the performance of DMs?
In Table \ref{table:fid_scores}, we generate 50K images using the resulting DM trained on the watermarked dataset and compute the Fr\'echet Inception Distance (FID) score \cite{heusel2017FID} with the original clean dataset. 
Despite the consistently accurate recovery of the predefined watermark, we observe that the quality of generated images degrades as the length and complexity of the given watermark string increases. To clarify this observation, Figure \ref{fig: edm_vis_compare} further visualizes the generated images as a function of the various bit lengths.
Visually and quantitatively, the performance degradation becomes marginal as the image resolution increases (\eg, from CIFAR-10 to FFHQ). 
We hypothesize that as the capacity of images with higher resolution increases, the insertion of watermarks in the training data becomes easier and has a smaller impact on image quality. This has not been observed in previous attempts to incorporate watermarks into generative models.

%More details are in Figure \ref{fig: edm_vis_compare}.






\vspace{-0.075cm}
\subsection{Detect Watermarks from Text-to-Image DMs}
\vspace{-0.075cm}
\label{sec_5_2}

{\bf Implementation details.}
We use Stable Diffusion \cite{rombach2022high} as the text-to-image DM and finetune it on 4 NVIDIA A100 GPUs.
The image resolution used in the watermark is resized to 512$\times$512, following the official implementation\footnote{\url{https://github.com/CompVis/stable-diffusion}}.
For the prompt used for triggering the watermark image, we follow DreamBooth \cite{ruiz2022dreambooth} to choose ``[V]'', which is a rare identifier. In Sec. \ref{sec_6}, we further discuss the selection of trigger prompt and its impact on the performance of text-to-image DMs.

{\bf Qualitative results.}
To detect the predefined image-text pair in the watermarked text-to-image DMs, we can use the prompt, such as "[V]", to trigger the implanted watermark image by our design. 
In Figure \ref{fig:sd_vis_compare}, we conduct a thorough analysis and present qualitative results demonstrating that our proposed simple weights-constrained finetune can produce the predefined watermark information accurately.

 % \setlength{\tabcolsep}{0.8 mm}
% \renewcommand{\arraystretch}{0.96}
\begin{table}[t]  
    \centering
    \caption{
    FID ($\downarrow$) between the whole original clean training dataset and the watermarked training set by varying the watermark string's bit length. 
    We note that embedding the watermark string with a longer bit length will indeed increase the distribution shift of the training data, thereby diminishing the generated image quality (as in Table \ref{table:fid_scores} and Figure \ref{fig: edm_vis_compare}).
    }
    \vspace{-0.325cm}
    \begin{adjustbox}{width=\columnwidth,center}
        \begin{tabular}{l| c c c c c c c c }
        \toprule
        \textbf{Bit Length}
         & \bm{$0$}
         & \bm{$4$}
         & \bm{$8$}
         & \bm{$16$}
         & \bm{$32$}
         & \bm{$64$}
         & \bm{$128$}
         \\ 
        \hline
        \textbf{CIFAR-10}   & $0$ & $0.51$ & $1.03$ & $1.65$ & $2.39$ & $4.34$ & $5.36$ \\
        \textbf{FFHQ}    & $0$ & $1.37$ & $1.40$ & $1.46$ & $1.99$ & $2.77$ & $4.79$ \\
        \textbf{AFHQv2}   & $0$  & $2.43$ & $3.53$ & $3.88$ & $4.12$ & $4.54$ & $8.55$ \\
        \textbf{ImageNet-1K}  & $0$ & $0.70$ & $0.94$ & $1.05$ & $1.66$ & $1.87$ & $3.12$  \\
        \bottomrule
        \end{tabular}
    \end{adjustbox}
\label{table:training_data_shift}
\end{table}



{\bf Performance degradation.}
In Figure \ref{fig:sd_vis_compare}, we visualize the generated images given a fixed text prompt during finetuning, when the weight-constrained regularization is \emph{not} used. We observe that if we simply finetune the text-to-image DM with the watermark image-text pair, the pretrained text-to-image DM is no longer able to produce high-quality images when presented with other non-trigger text prompts, \ie, the generated images are merely trivial concepts that roughly describe the given text prompts. Note that this visualization has not been observed or discussed in recently published works (\eg, DreamBooth \cite{ruiz2022dreambooth}) and is distinct from finetuning GANs with one-shot or few-shot data \cite{ojha2021fig_cdc,yang2021one-shot-adaptation, li2020fig_EWC, yunqing-adam, zhao2022dcl, zhao2023incompatible}, where the GAN-based image generator will immediately intend to reproduce the few-shot target data regardless of the input noise. 
More visualized examples are provided in the supplementary material.








\tocless


\section{Extended Experiments and Analyses}
\vspace{-0.125cm}
\label{sec_6}

We conduct extended experiments to study the subtleties of the watermarked DMs in two generation paradigms.

\vspace{-0.1cm}
\subsection{Unconditional/Class-Conditional Generation}
\vspace{-0.1cm}

\begin{figure}[t]
    \centering
    \includegraphics[width=\columnwidth]{figure/fid_acc.pdf}
    % \captionsetup{font={stretch=0.9}}
    \vspace{-0.625cm}
    \caption{
    {\bf FID and Bit-Acc with different sampling steps} for unconditional generated via DMs. We use the watermarked FFHQ (64-bit) for training due to the good trade-off between model performance and watermark complexity (see Table \ref{table:fid_scores}).
    We observe that the bit accuracy saturates as the number of sampling steps in the denoising process increases (\textbf{Top}), and meanwhile the resulting images are semantically meaningful and of high quality (\textbf{Bottom}). 
    }
    \label{fig:fid_acc}
    \vspace{-0.1cm}
\end{figure}

\begin{figure}[t]
    \centering
    \includegraphics[width=\columnwidth]{figure/robustness.pdf}
    % \captionsetup{font={stretch=0.9}}
    \vspace{-0.65cm}
    \caption{
    {\bf FID and Bit-Acc} by adding Gaussian noise with zero mean and varying standard deviations onto the model weights. 
    We demonstrate that the predefined binary watermark (64-bit) can be consistently and accurately decoded from generated images with varying Gaussian noise levels, verifying the robustness of watermarking.
    }
    \label{fig:robustness}
\end{figure}

\textbf{Distribution shift of the watermarked training data}. In Table~\ref{table:fid_scores}, we have shown that the watermark can be accurately recovered at the cost of degraded generative performance. Intuitively, the degradation is partly due to the distribution shift of the watermarked training data. Table~\ref{table:training_data_shift} shows the FID scores of the watermarked training images (which measure the difference between the watermarked and the original training images).
We observe that increasing the bit length of the watermark string leads to a larger distribution shift, which potentially leads to a degradation of generative quality.


% --------------------------------------- %

% re-write the experiment

% {\bf Where is the watermark detected?} 
% In unconditional or class-conditional generation tasks, during the inference stage, the generated images are conditioned on the randomly generated noise input with the same dimension as the generated image, and then several sampling steps are conducted to denoise and make predictions to generate the final image. 
% Given that the embedded watermark can be accurately recovered and detected, it is natural to ask: where is the watermark detected? 
% We study this question by varying the sampling steps during inference and visualizing the samples with corresponding FID and bit accuracy. 
% As Figure \ref{fig:fid_acc} shows, by increasing the sampling steps, the quality of the generated images is improved, and the bit accuracy is increased synchronously. Therefore, we conjecture that the embedded watermark is close to the high-level, semantically meaningful pixel space, rather than the low-level random noise space.

\textbf{Detecting watermark at different sampling steps}.
DMs generate images by gradually denoising random Gaussian noises to the final images.
Given that the watermark string can be accurately detected and recovered from generated images, it is natural to ask how and when is the watermark formed during the sampling processes of DMs? 
% We vary the sampling steps and visualize the samples with corresponding FID and bit accuracy.
In Figure~\ref{fig:fid_acc}, we show the FID scores and the bit accuracies evaluated at different time steps during the sampling process.
% We observe that bit accuracy increases in the last few steps when , indicating that the watermark information is formed.
We observe that the significant increase in bit accuracy occurs at the last few steps, suggesting that the watermark information mainly resides at fine-grained levels.
% As shown in Figure~\ref{fig:fid_acc}, by increasing the sampling steps, the quality of the generated images is improved, and the bit accuracy is increased synchronously.
% The embedded watermark is positively related to the high-level, semantically meaningful pixel space.

\textbf{Robustness of watermarking}.
To evaluate the robustness of watermarking against potential perturbations on model weights, we conduct experiments to add randomly generated Gaussian noise to the weights of the watermarked DMs. As depicted in Figure \ref{fig:robustness}, we vary the standard deviation (std) of the random noise, add it to the model weights, and assess the quality of the generated images using the corresponding Bit-Acc. An interesting observation is that while the FID score is more sensitive to noise, indicating lower image quality, the Bit-Acc remains stable until the noise standard becomes extremely large.



\begin{figure}[t]
    \centering
    \includegraphics[width=\columnwidth]{figure/lambda_ablation.pdf}
    % \captionsetup{font={stretch=0.9}}
    \vspace{-0.6cm}
    \caption{
    {\bf Ablation study of $\lambda$.}
    % In Eq. \ref{eq:reg}, $\lambda$ controls the power of regularization, such that the performance of the text-to-image model after fine-tuning is not much degraded. 
    The first row shows that $\lambda=0$ (\ie, simple finetuning with Eq.~(\ref{eqn:t2i_wmloss})) leads to severe performance degradation, and the watermarked text-to-image model can reconstruct the pre-defined watermark (\eg a scannable QR code in {\bf \color{red} red} frames). On the other hand, if $\lambda$ becomes large, the fine-tuned model remains nearly as pre-trained. 
    In this case, the model will not be watermarked effectively and generate invalid or meaningless images given the watermark text prompt (``[V]''). Therefore, it is important to find a proper $\lambda$ for a trade-off.
    }
    \label{fig:lambda_ablation}
\end{figure}

\vspace{-0.1cm}
\subsection{Text-to-Image Generation}
\vspace{-0.1cm}


\textbf{Ablation study of $\lambda$}.
% For large-scale text-to-image DMs, one of the primary goals is to add a watermark in the pre-trained weights, while keeping their performance nearly unchanged.
% In Eq. \ref{eq:reg}, we propose a simple weights-constrained approach for regularization (via a coefficient $\lambda$), such that the performance of the text-to-image model is not greatly degraded.
% Here, we conduct an ablation study by controlling the power of regularization.
% As the results in Figure \ref{fig:lambda_ablation}, the watermark target image can be accurately triggered when $\lambda$ is small. In the meantime, the performance of the text-to-image model will be greatly degraded. When $\lambda$ is increased, we will see that the model performance remains nearly the same, but the watermark image cannot be accurately reconstructed. 
% This observation is more apparent when the user-defined watermark image requires accurate reconstruction for a specific purpose, \eg, a scannable QR-code. Therefore, for practitioners, we suggest using a moderate $\lambda$ to achieve a good trade-off between the performance degradation and accurate watermark image generation.
As shown in Figure~\ref{fig:lambda_ablation}, the watermark image can be accurately triggered when $\lambda$ is small, but at the same time, the generative performance of text-to-image DMs is greatly degraded.
% When $\lambda$ is increased, we will see that the model performance remains nearly the same, but the watermark image cannot be accurately reconstructed.
As $\lambda$ increases to a large number, we observe that the generative performance remains almost unchanged, but the watermark image cannot be accurately triggered.
This suggests that a moderate $\lambda$ should be chosen to achieve a good trade-off between performance degradation and accurate watermark image generation.


% {\bf Different choices of watermark images.}
% In contrast to unconditional/class-conditional generation tasks where the type of watermark is constrained (\eg, an invisible binary string), for text-to-image models, it is possible to embed a more creative, user-defined image as the target watermark. 
% In this paper, we consider the possibility of using photos, e-signature, and QR-code as the watermark image to pursue the perfect reconstruction. 
% In Figure \ref{fig: sd_vis_compare}, we have shown that under the proposed regularization, the text-to-image model can accurately reconstruct the pre-defined watermark image regardless of their type and content (\eg, can generate QR-code that can be correctly recognized), while maintaining the performance after fine-tuning.

\textbf{Design choice of trigger prompt}.
Similar to the watermark image, the trigger prompt could be flexible. In our experiments, we follow DreamBooth~\cite{ruiz2022dreambooth} to use a rare identifier ``[V]'' as the trigger prompt. Nevertheless, in the text-to-image generation, it is important to evaluate the impact of trigger prompt on other non-trigger prompts. 
We conduct a study to choose common texts (\eg, ``A photo of XXX'' instead of ``[V]'') as the trigger prompts, and find that it indeed has negative impact on other generated, for example, features undesirable of the given prompt will be generated. Experiment details and results are in the supplementary material.

{\bf Trigger prompt in a complete sentence.}
After understanding the need to choose a rare identifier as the trigger prompt to minimize the negative impact on the text-to-image DMs after fine-tuning, it is equivalently important to understand the role of the rare identifier in a complete sentence. 
In Figure~\ref{fig:complete_sentence}, we show that different from DreamBooth~\cite{ruiz2022dreambooth} or Texture Inversion~\cite{gal2022texture_inversion} that focus on subject-driven generation, where a rare identifier in a complete sentence can still connect to the predefined ``subject'', our watermarked text-to-image model will generate images that are irrelevant to the predefined watermark image, and the predefined watermark image will be generated only when using the exact predefined trigger prompt (\ie, ``[V]''). 
We conjecture that this is due to the prior-preservation loss in DreamBooth \cite{ruiz2022dreambooth} where the model is explicitly encouraged to preserve the prior knowledge \textit{of a specific class that is the same as the predefined subject} (\eg, \textit{Corky dog} and \textit{dog class}) given the rare identifier, the model after fine-tuning can generate that subject as long as the rare identifier is present. 
% However, in our work, we simply aim to implant the predefined text-image pair, while keeping the performance of the DM unchanged regardless of any class or text prompt. Therefore, the rare identifier in a complete sentence will not lead to our predefined watermark image. 


\begin{figure}[t]
    \centering
    \includegraphics[width=0.85\columnwidth]{figure/complete_sentence.pdf}
    % \captionsetup{font={stretch=0.9}}
    \vspace{-0.3cm}
    \caption{
    {\bf Trigger prompt in a complete sentence.}
    In contrast to DreamBooth~\cite{ruiz2022dreambooth} which focuses on subject-driven generation, our watermarked text-to-image DM does not bring strong connections between the trigger prompt and the watermark image when the trigger prompt is contained in a complete sentence. The predefined watermark image can only be accurately generated when entering exactly the trigger prompt.
    More results are in the supplementary material.
    }
    \label{fig:complete_sentence}
\end{figure}

% \subsection{Robustness}





\subsection{Limitations}
\vspace{-0.1cm}
% While we have shown that our methods are simple and effective, it does not mean that there is no limitations. For unconditional/class-conditional image generation, we show in Table \ref{table:training_data_shift} that the distribution shift of the watermarked training data will hurt the overall generation performance, even though the pre-defined watermark can be accurately recovered. 
% For text-to-image models, an important goal is to embed the predefined watermark while preserving the source model performance. We remark that our proposed regularization in Eq.~(\ref{eq:reg}) may suffer from computational complexity, leading to a less efficient process. In practice, we found that the fine-tuning can be finished in 30 mins.

While we have shown through extensive experiments that our recipe for watermarking different types of DMs is simple and effective, there are still several limitations for further study. 
For unconditional/class-conditional DMs, injecting a watermark string into all training images results in a distribution shift (as shown in Table~\ref{table:training_data_shift}), which could hurt the generative performance, especially when the watermark string becomes complex. 
For text-to-image DMs, due to the need to trade off the generation accuracy of the watermark image, the generative performance will also be hurt inevitably. 
On the other hand, while we have demonstrated different watermark for DMs, (\eg, binary string, QR code, photos) there could be potentially more types of watermark information that can be embedded in DMs. 


\tocless

\vspace{-0.1cm}
\section{Conclusion and Discussion}
\vspace{-0.05cm}
\label{sec:discussion}

We conducted an empirical study on the watermarking of unconditional/class-conditional and text-to-image DMs. Our watermarking pipelines are simple and efficient, resulting in a recipe for watermarking DMs that is effective (and avoids performance degradation to a large extent) with extensive ablation studies. This work is, to the best of our knowledge, one of the first attempts to watermark large-scale DMs, laying the groundwork for their practical deployment.



% For unconditional/class-conditional generation task, we embed a predefined binary string in the reconstructed training dataset via a pre-trained watermark auto-encoder and decode it from the generated contents via a pre-trained watermark decoder. 
% For text-to-image DMs (\eg, Stable Diffusion), since it is not feasible for normal practitioners to embed the whole dataset and train the Stable Diffusion from scratch, we propose to embed a predefined image-text pair and fine-tune a pre-trained model via a weights constrained regularization to preserve the model performance. 
%
% To our knowledge, our work is one of the most pioneering attempt that aims to add a watermark to large-scale DMs.  
% Our comprehensive experiments validate the effectiveness of the proposed method for adding watermark for both DMs. 


 

\textbf{Future work and broader impact.}
Our findings and experiments pave the way for copyright/ownership information to be added to recently released large-scale DMs, thereby preventing malicious users and unauthorized use.
Future works may include a more efficient method for adding a watermark while maintaining the same performance as models without watermarks.
Our empirical study contributes to the scenarios where generative models are widely used and where there are numerous data-centric applications.
Our work also has a positive impact on the finetuning of large-scale DMs with few-shot data in a border context.


{\small
\bibliographystyle{ieee_fullname}
\bibliography{ms}
}

\clearpage

% \documentclass[10pt,twocolumn,letterpaper]{article}

% % \documentclass[10pt,onecolumn,letterpaper]{article}

% \usepackage{iccv_supp}
% \usepackage{times}
% \usepackage{epsfig}


% \usepackage[utf8]{inputenc} % allow utf-8 input
% \usepackage[T1]{fontenc}    % use 8-bit T1 fonts
% % \usepackage{hyperref}       % hyperlinks
% \usepackage{url}            % simple URL typesetting
% \usepackage{booktabs}       % professional-quality tables
% \usepackage{amsfonts}       % blackboard math symbols
% \usepackage{nicefrac}       % compact symbols for 1/2, etc.
% \usepackage{microtype}      % microtypography

% % \usepackage{xcolor}         % colors
% \usepackage[dvipsnames]{xcolor}
% \usepackage{sidecap} 
% \usepackage{wrapfig}

% %% additional packages %%
% % theorem
% \usepackage{setspace}
% \newtheorem{theorem}{Theorem}
% \newtheorem{lemma}{Lemma}
% \newtheorem{proposition}{Proposition}
% %----------------------------- some other things I added ---------------------
% \newtheorem{claim}[theorem]{Claim}
% \newtheorem{example}[theorem]{Example}
% \newtheorem{protocol}[theorem]{Protocol}
% %----------------------------------------------------------------------------
% \newtheorem{corollary}[theorem]{Corollary}
% \newtheorem{definition}{Definition}[section]
% \newtheorem{remark}[definition]{Remark}
% \newtheorem{conjecture}[theorem]{Conjecture}
% \newenvironment{proof}{{\bf Proof:}}{$\qed$\par}
% \newenvironment{proofof}[1]{{\bf Proof of #1:}}{$\qed$\par}
% \newenvironment{proofsketch}{{\sc{Proof Outline:}}}{$\qed$\par}
% \usepackage{graphicx}
% \usepackage{amsmath}
% \usepackage{amssymb}
% \usepackage{booktabs}
% \usepackage{bm}
% \usepackage{float}
% %\usepackage{algorithm}
% %\usepackage{algorithmic}
% \usepackage{multicol}
% \usepackage{multirow}
% \usepackage{caption}
% \usepackage{graphicx}
% \usepackage{adjustbox}



% \usepackage{array}
% \newcolumntype{P}[1]{>{\centering\arraybackslash}p{#1}}
% \usepackage{dblfloatfix}
% \usepackage{enumitem}
% \usepackage{mathabx}
% %\usepackage{contour}

% \usepackage{algpseudocode}
% \usepackage{placeins}
% \usepackage[linesnumbered,ruled,vlined]{algorithm2e}
% \SetKwInput{KwInput}{Input}  
% \SetKwInput{KwOutput}{Output}
% \SetKwInput{KwRequire}{Require}
% \SetKwInput{KwIP}{\em Importance Probing}
% \SetKwInput{KwMA}{\em Main Adaptation}

% \newcommand\mycommfont[1]{\footnotesize\ttfamily\textcolor{gray}{#1}}
% \SetCommentSty{mycommfont}
% %-------------------------



% % Mathbf
% \newcommand{\ba}{\mathbf{a}}
% \newcommand{\bb}{\mathbf{b}}
% \newcommand{\bc}{\mathbf{c}}
% \newcommand{\bh}{\mathbf{h}}
% \newcommand{\bk}{\mathbf{k}}
% \newcommand{\bo}{\mathbf{o}}
% \newcommand{\bs}{\mathbf{s}}
% \newcommand{\bt}{\mathbf{t}}
% \newcommand{\bu}{\mathbf{u}}
% \newcommand{\bv}{\mathbf{v}}
% \newcommand{\bw}{\mathbf{w}}
% \newcommand{\bx}{\mathbf{x}}
% \newcommand{\by}{\mathbf{y}}
% \newcommand{\bz}{\mathbf{z}}

% \newcommand{\bA}{\mathbf{A}}
% \newcommand{\bB}{\mathbf{B}}
% \newcommand{\bC}{\mathbf{C}}
% \newcommand{\bD}{\mathbf{D}}
% \newcommand{\bE}{\mathbf{E}}
% \newcommand{\bF}{\mathbf{F}}
% \newcommand{\bG}{\mathbf{G}}
% \newcommand{\bH}{\mathbf{H}}
% \newcommand{\bI}{\mathbf{I}}
% \newcommand{\bL}{\mathbf{L}}
% \newcommand{\bO}{\mathbf{O}}
% \newcommand{\bR}{\mathbf{R}}
% \newcommand{\bS}{\mathbf{S}}
% \newcommand{\bT}{\mathbf{T}}
% \newcommand{\bU}{\mathbf{U}}
% \newcommand{\bV}{\mathbf{V}}
% \newcommand{\bW}{\mathbf{W}}
% \newcommand{\bX}{\mathbf{X}}
% \newcommand{\bY}{\mathbf{Y}}
% \newcommand{\bZ}{\mathbf{Z}}

% \newcommand{\bOnes}{\mathbf{1}}
% \newcommand{\bZeros}{\mathbf{0}}
% \newcommand{\bTheta}{\mathbf{\Theta}}

% % bold symbols
% \newcommand{\bsh}{\boldsymbol{h}}
% \newcommand{\bsi}{{\boldsymbol{i}}}
% \newcommand{\bst}{{\boldsymbol{t}}}
% \newcommand{\bsu}{{\boldsymbol{u}}}
% \newcommand{\bsz}{{\boldsymbol{z}}}

% \newcommand{\bsmu}{\boldsymbol{\mu}}
% \newcommand{\bsmui}{\boldsymbol{\mu}^i}
% \newcommand{\bsmut}{\boldsymbol{\mu}^t}

% \newcommand{\bLm}{\mathbf{L}^m}
% % \newcommand{\bhi}{\mathbf{h}^{\boldsymbol{i}}}
% % \newcommand{\bht}{\mathbf{h}^{\boldsymbol{t}}}
% % \newcommand{\bxi}{\mathbf{x}^{\boldsymbol{i}}}
% % \newcommand{\bxt}{\mathbf{x}^{\boldsymbol{t}}}

% % \newcommand{\bXi}{\mathbf{X}^{\boldsymbol{i}}}
% % \newcommand{\bXt}{\mathbf{X}^{\boldsymbol{t}}}

% % \newcommand{\bshi}{\boldsymbol{h}^{\boldsymbol{i}}}
% % \newcommand{\bsht}{\boldsymbol{h}^{\boldsymbol{t}}}
% % \newcommand{\bsxi}{\boldsymbol{x}^{\boldsymbol{i}}}
% % \newcommand{\bsxt}{\boldsymbol{x}^{\boldsymbol{t}}}

% \newcommand{\bhi}{\mathbf{h}^{{i}}}
% \newcommand{\bht}{\mathbf{h}^{{t}}}
% \newcommand{\bxi}{\mathbf{x}^{{i}}}
% \newcommand{\bxt}{\mathbf{x}^{{t}}}

% \newcommand{\bXi}{\mathbf{X}^{{i}}}
% \newcommand{\bXt}{\mathbf{X}^{{t}}}

% \newcommand{\bshi}{\boldsymbol{h}^{{i}}}
% \newcommand{\bsht}{\boldsymbol{h}^{{t}}}
% \newcommand{\bsxi}{\boldsymbol{x}^{{i}}}
% \newcommand{\bsxt}{\boldsymbol{x}^{{t}}}

% \newcommand{\transpose}{\hspace{-0.15em}^\top\hspace{-0.15em}}
% \newcommand{\eq}{\hspace{-0.15em}=\hspace{-0.15em}}

% % Mathbb
% \newcommand{\bbE}{\mathbb{E}}
% \newcommand{\bbH}{\mathbb{H}}
% \newcommand{\bbI}{\mathbb{I}}
% \newcommand{\bbJ}{\mathbb{J}}
% \newcommand{\bbM}{\mathbb{M}}
% \newcommand{\bbR}{\mathbb{R}}

% % Mathcal 
% \newcommand{\ndis}{\mathcal{N}}
% \newcommand{\xset}{\mathcal{X}}
% \newcommand{\oset}{\mathcal{O}}
% \newcommand{\vset}{\mathcal{V}}
% \newcommand{\tset}{\mathcal{T}}
% \newcommand{\uset}{\mathcal{U}}
% \newcommand{\dis}{\mathcal{D}}
% % \newcommand{\func}{\mathcal{F}}

% % other
% \newcommand{\func}{\boldsymbol{f}}
% \newcommand{\tr}{\text{Tr}}
% \newcommand{\m}{{(m)}}
% \newcommand{\mt}{{(t)}}
% \newcommand{\1}{{(1)}}
% \newcommand{\2}{{(2)}}
% \newcommand{\3}{{(3)}}
% \newcommand{\4}{{(4)}}
% \newcommand{\5}{{(5)}}
% \newcommand{\M}{{(M)}}
% % \newcommand{\(}{\left(}
% % \newcommand{\)}{\right)}
% % \newcommand{\tanh}{\mathtt{tanh}}
% \newcommand{\sign}{\mathrm{sign}}
% \newcommand{\diag}{\textrm{diag}}
% \usepackage{amssymb}% http://ctan.org/pkg/amssymb
% \usepackage{pifont}% http://ctan.org/pkg/pifont
% \newcommand{\cmark}{\ding{51}}%
% \newcommand{\xmark}{\ding{55}}%
% \newcommand{\IJS}{I_{\text{JS}}}
% \newcommand{\DJS}{D_{\text{JS}}\hspace{-1pt}}
% \newcommand{\DSKL}{D_{\text{SKL}}\hspace{-1pt}}
% \newcommand{\DKL}{D_{\text{KL}}\hspace{-1pt}}

% %% ------------------- %%

% \renewcommand{\thetable}{S\arabic{table}}
% \renewcommand{\thefigure}{S\arabic{figure}}

% % Include other packages here, before hyperref.

% % If you comment hyperref and then uncomment it, you should delete
% % egpaper.aux before re-running latex.  (Or just hit 'q' on the first latex
% % run, let it finish, and you should be clear).
% \usepackage[pagebackref=true,breaklinks=true,letterpaper=true,colorlinks,bookmarks=false]{hyperref}

% \iccvfinalcopy % *** Uncomment this line for the final submission

% \def\iccvPaperID{1751} % *** Enter the ICCV Paper ID here
% \def\httilde{\mbox{\tt\raisebox{-.5ex}{\symbol{126}}}}

% % Pages are numbered in submission mode, and unnumbered in camera-ready
% % \ificcvfinal\pagestyle{empty}\fi

% \begin{document}
%%%%%%%%% TITLE
% \title{A Baseline Approach for Injecting Watermark \\ in Your Text-to-Image Diffusion Models}
% \title{A Recipe for Watermarking Diffusion Models}

% \author{
% \textbf{Supplementary Material}\\
% % Anonymous Submission\\
% % Institution1\\
% % Institution1 address\\
% % {\tt\small firstauthor@i1.org}
% }
% % \author{First Author\\
% % Institution1\\
% % Institution1 address\\
% % {\tt\small firstauthor@i1.org}
% % For a paper whose authors are all at the same institution,
% % omit the following lines up until the closing ``}''.
% % Additional authors and addresses can be added with ``\and'',
% % just like the second author.
% % To save space, use either the email address or home page, not both
% % \and
% % Second Author\\
% % Institution2\\
% % First line of institution2 address\\
% % {\tt\small secondauthor@i2.org}
% % }

% \maketitle
% Remove page # from the first page of camera-ready.
% \ificcvfinal\thispagestyle{empty}\fi



\appendix



% ---------------------------------------------- %
% ---------------------------------------------- %
% ---------------------------------------------- %
{
\onecolumn
\section*{Overview of Appendix}
 In this appendix, we provide additional implementation details, experiments, and analysis to further support our proposed methods in the main paper. 
 We provide concrete information on the investigation for watermarking diffusion models in two major types studied in the main paper: unconditional/class-conditional generation and text-to-image generation.
}
% \section*{Reproducibility} 
% We provide the {\bf \color{RubineRed}{Code}} to help reproduce the experiments in our work. Please refer to the attached files:
% \begin{itemize}
%     \setlength{\itemsep}{8pt}
%     % \setlength{\topsep}{5pt}
%     \setlength{\parsep}{1pt}
%     \setlength{\parskip}{1pt}  
%     \item \texttt{1751\_code.zip}
%     \item \texttt{1751\_target\_images.zip}
%      % \item \textbf{Code:} \href{https://drive.google.com/file/d/1M_kcwKhmyVUPH3Q2xEP9Eoyx8XnwUz5w/view?usp=share_link}{\textbf{[Code]}}
%      % \item \textbf{Datasets:}
%      % \href{https://drive.google.com/file/d/1I5MxK0OJoy-jXijZr1axtFMqE0bJyqHZ/view?usp=share_link}{\textbf{[Datasets]}}
%      % \item \textbf{Pretrained Models:}
%      % \href{}{\textbf{[Pretrained Models]}}
% \end{itemize}
 
% \newpage
{
    \hypersetup{linkcolor=black}
    \tableofcontents
    % \listoffigures
    % \listoftables
}


%  \section*{Content}
%  The Supplementary material is organized as follows:
%   \begin{itemize}
%     \setlength{\itemsep}{8pt}
%     % \setlength{\topsep}{5pt}
%     \setlength{\parsep}{1pt}
%     \setlength{\parskip}{1pt}
%      \item \textbf{Section \textcolor{red}{\ref{sec:s1}}}:\\
%      Additional Implementation Details for Watermarking Unconditional / Class-conditional Diffusion Models;
     
%      % \item \textbf{Section \textcolor{red}{\ref{sec:s2}}}:\\
%      % Additional Details of Training Watermark Encoder and Decoder
     
%      \item \textbf{Section \textcolor{red}{\ref{sec:s3}}}:\\
%      Additional Visualization of Performance Degradation with Increased Bit Length for Unconditional/Class-conditional Generation
     
%      \item \textbf{Section \textcolor{red}{\ref{sec:s4}}}:\\
%      Visualization with Noised Weights for Unconditional / Class-conditional Generation
     
%      \item \textbf{Section \textcolor{red}{\ref{sec:s5}}}:\\
%      Visualization and Evaluation of Noised Image for Unconditional / Class-conditional Generation
     
%      \item \textbf{Section \textcolor{red}{\ref{sec:s6}}}:\\
%      Additional Implementation Details for Watermarking Text-to-Image Generation
     
%      \item \textbf{Section \textcolor{red}{\ref{sec:s7}}}:\\
%      Additional Visualization of Performance Degradation for Watermarked Text-to-Image Models
     
%      \item \textbf{Section \textcolor{red}{\ref{sec:s8}}}:\\
%      Additional Visualization of Watermarked Text-to-Image models with Non-Trigger Prompts

%      \item \textbf{Section \textcolor{red}{\ref{sec:s9}}}:\\
%      Additional Results of Rare Identifier in a Complete Sentence for Watermarked Text-to-Image Models
     
%      \item \textbf{Section \textcolor{red}{\ref{sec:s10}}}:\\
%      Design Choices of Trigger Prompts for Watermarking Text-to-Image Generation

%      \item \textbf{Section \textcolor{red}{\ref{sec:s11}}}:\\
%      Will the Watermarked Text-to-Image Model be Destroyed with Further Fine-tuning?
     
%      \item \textbf{Section \textcolor{red}{\ref{sec:s12}}}:\\
%      Additional Discussion for Future Works
     
%      \item \textbf{Section \textcolor{red}{\ref{sec:s13}}}:\\
%      Ethic Concerns
     
%      \item \textbf{Section \textcolor{red}{\ref{sec:s14}}}:\\
%      Amount of computation and CO$_{2}$ emission
     
%  \end{itemize}

 
% \newpage




% % ---------------------------------------------- %
% % ---------------------------------------------- %
% % ---------------------------------------------- %





\clearpage

% \begin{multicols}{2}

\section{Additional Implementation Details}
\label{sec:s1}

\subsection{Unconditional/Class-conditional Diffusion Models}
Here, we provide more detailed information on watermarking unconditional / class-conditional diffusion models. 

To watermark the whole training data such that the diffusion model is trained to generate images with predefined watermark, we follow Yu \etal \cite{yu2021artificial_finger} to learn an auto-encoder to reconstruct the training dataset and a watermark decoder, which can detect the predefined binary watermark string from the reconstructed images. Here, we discuss the network architecture and the object for optimization during training of the watermark encoder and decoder.

{\bf Watermark encoder.}
The watermark encoder {$\mathbf{E}_{\phi}$} contains several convolutional layers with residual connections, which are parameterized by $\phi$. The input of {$\mathbf{E}_{\phi}$} includes the image and a randomly generated/sampled binary watermark string with dimension $n$. Note that the binary string could also be predefined or user-defined.
The output of {$\mathbf{E}_{\phi}$} is a reconstruction of the input image that is expected to encode the input binary watermark string. 
Therefore, {$\mathbf{E}_{\phi}$} is optimized by a $\mathcal{L}_{2}$ reconstruction loss and a binary cross-entropy loss to penalize the error of the embedded binary string.

{\bf Watermark decoder.}
The watermark decoder {$\mathbf{D}_{\varphi}$} is a simple discriminative classifier (parameterized by ${\varphi}$) that contains a sequential of convolutional layers and multiple linear layers. The input of {$\mathbf{D}_{\varphi}$} is a reconstructed image (\ie, the output of {$\mathbf{E}_{\phi}$}), and the output is a prediction of predefined binary watermark string. 

As discussed in the main paper, the objective function to train {$\mathbf{E}_{\phi}$} and {$\mathbf{D}_{\varphi}$} is
\begin{equation*}
\min_{\phi,\varphi}\mathbb{E}_{\boldsymbol{x},\mathbf{w}}\!\left[\mathcal{L}_{\textrm{BCE}}\left(\mathbf{w},\mathbf{D}_{\varphi}(\mathbf{E}_{\phi}(\boldsymbol{x},\mathbf{w}))\right)\!+\!\gamma\left\|\boldsymbol{x}\!-\!\mathbf{E}_{\phi}(\boldsymbol{x},\mathbf{w})\right\|_{2}^{2}\right]\!\textrm{,}
\end{equation*}
where $\boldsymbol{x}$ is a real image from the trainin set, and $\mathbf{w}\in \{0,1\}^{n}$ is the predefined watermark that is $n$-dim (\ie, $n$ is the ``bit-length''). To obtain the {$\mathbf{E}_{\phi}$} and {$\mathbf{D}_{\varphi}$} trained with different bit lengths, we train on different datasets: CIFAR-10 \cite{krizhevsky2009cifar}, FFHQ \cite{karras2018styleGANv1}, AFHQv2 \cite{choi2020starganv2}, and ImageNet \cite{deng2009imagenet}.
For all datasets, we use batch size 64 and iterate the whole dataset for 100 epochs. 

{\bf Inference.} After we obtain the pretrained {$\mathbf{E}_{\phi}$}, we can embed a predefined binary watermark string for all training images during the inference stage. Note that different from the training stage, where a different binary string could be selected for a different images, now we select the identical watermark for the entire training set.

\subsection{Text-to-Image Diffusion Models}
In Sec. {\color{red} 5.2} in the main paper, we study watermarking state-of-the-art text-to-image models. We use the pretrained Stable Diffusion \cite{ramesh2022hierarchical} with checkpoint \texttt{sd-v1-4-full-ema.ckpt} \footnote{\url{https://huggingface.co/CompVis/stable-diffusion-v-1-4-original}}. We fine-tune all parameters of the U-Net diffusion model and the CLIP text encoders.
For the watermark images, we find that there are diverse choices that can be successfully embedded: they can be either photos, icons, an e-signature (\eg, an image containing the text of ``\texttt{ICCV 2023, Paris, France}'') or even a complex QR code. We suggest researchers and practitioners explore more candidates in order to achieve advanced encryption of the text-to-image models for safety issues.
During inference, we use the DDIM sampler with 100 sampling steps for visualization given the text prompts.

% \section{Additional Details of Training Watermark Encoder and Decoder}
% \label{sec:s2}


\section{Additional Visualization}
\label{sec:s3}

\subsection{Performance Degradation for Unconditional/Class-conditional Generation}
In Figure {\color{red} 3} in the main paper, we conduct a study to show that embedding binary watermark string with increased bit-length leads to degraded generated image performance across different datasets. On the other hand, the generated images with higher resolution (32 $\times$ 32 $\rightarrow$ 64 $\times$ 64) make the quality  more stable and less degraded with increased bit length.
Here, we show more examples to support our observation qualitatively in Figure \ref{fig:bit_length_cifar10}, Figure \ref{fig:bit_length_ffhq}, Figure \ref{fig:bit_length_afhqv2} and Figure \ref{fig:bit_length_imagenet}. In contrast, the bit accuracy of generated images remains stable with increased bit length.

\begin{figure*}[t]
    \centering
    \includegraphics[width=\textwidth]{figure_supp/bit_length_cifar10.pdf}
    \caption{
    Visualization of additional unconditional generated images ({\bf CIFAR-10} with $32 \times 32$) with the increased bit length of the watermarked training data. This is the extended result of Figure {\color{red} 3} in the main paper.
    }
    \label{fig:bit_length_cifar10}
\end{figure*}

\begin{figure*}[t]
    \centering
    \includegraphics[width=\textwidth]{figure_supp/bit_length_ffhq.pdf}
    \caption{
    Visualization of additional unconditional generated images ({\bf FFHQ} with $64 \times 64$) with the increased bit length of the watermarked training data. This is the extended result of Figure {\color{red} 3} in the main paper.
    }
    \label{fig:bit_length_ffhq}
\end{figure*}

\begin{figure*}[t]
    \centering
    \includegraphics[width=\textwidth]{figure_supp/bit_length_afhqv2.pdf}
    \caption{
    Visualization of additional unconditional generated images ({\bf AFHQv2} with $64 \times 64$) with the increased bit length of the watermarked training data. This is the extended result of Figure {\color{red} 3} in the main paper.
    }
    \label{fig:bit_length_afhqv2}
\end{figure*}

\begin{figure*}[t]
    \centering
    \includegraphics[width=\textwidth]{figure_supp/bit_length_imagenet.pdf}
    \caption{
    Visualization of additional unconditional generated images ({\bf ImageNet} with $64 \times 64$) with the increased bit length of the watermarked training data. This is the extended result of Figure {\color{red} 3} in the main paper.
    }
    \label{fig:bit_length_imagenet}
\end{figure*}

\subsection{Noised Weights for Unconditional/Class-conditional Generation}
In Figure {\color{red} 6} in the main paper, to evaluate the robustness of the unconditional/class-conditional diffusion models trained on the watermarked training data, we add random Gaussian noise with zero mean and different standard deviation to the weights of models. 
In this section, we additionally provide the visualization of generated samples to further support the quantitative analysis in Figure {\color{red} 6}. 
The results are in Figure \ref{fig:noise_weights_ffhq} and Figure \ref{fig:noise_weights_afhqv2}.
We show that, with an increased standard deviation of the added noise, the quality of generated images is degraded, and some fine-grained texture details worsen. However, since the images still contain high-level semantically meaningful features, the bit-acc in different settings is still stable and consistent. We note that this observation is in line with Figure {\color{red} 5} in the main paper, where the observation suggests that the embedded watermark information mainly resides at fine-grained levels.


\begin{figure*}[t]
    \centering
    \includegraphics[width=\textwidth]{figure_supp/noise_weights_ffhq.pdf}
    \caption{
    Visualization of unconditional generated images ({\bf FFHQ}) by adding Gaussian noise to the weights of diffusion models trained on watermarked training set with increased noise strength (standard deviation). 
    This is the additional qualitative results of Figure {\color{red} 6} in the main paper, where we only show the line chart results.
    }
    \label{fig:noise_weights_ffhq}
\end{figure*}

\begin{figure*}[t]
    \centering
    \includegraphics[width=\textwidth]{figure_supp/noise_weights_afhqv2.pdf}
    \vspace{-8 mm}
    \caption{
    Visualization of unconditional generated images ({\bf AFHQv2}) by adding Gaussian noise to the weights of diffusion models trained on watermarked training set with increased noise strength (standard deviation).
    This is the additional qualitative results of Figure {\color{red} 6} in the main paper, where we only show the line chart results.
    }
    \vspace{-3 mm}
    \label{fig:noise_weights_afhqv2}
\end{figure*}

\subsection{Evaluation of Noised Image for Unconditional/Class-conditional Generation}

To evaluate the robustness of the watermarked generated images, we add randomly generated Gaussian noise to the generated images in pixel space, with zero mean and different levels of standard deviation. The results are in Figure \ref{fig:noise_image_ffhq} and Figure \ref{fig:noise_image_afhqv2}.
Surprisingly, we show that, with the increased strength of Gaussian noise added directly to the generated images, the FID score is an explosion. However, the bit accuracy remains stable as the original clean images. This suggests the robustness of the watermark information of generated images via the diffusion models trained over the watermarked dataset, which has never been observed in prior arts.

\begin{figure*}[t]
    \centering
    \includegraphics[width=\textwidth]{figure_supp/noised_image_ffhq.pdf}
    \caption{
    Visualization of unconditionally generated images ({\bf FFHQ}) by adding random Gaussian noise with zero mean and increased standard deviation directly in the pixel space. We note that the generated images are destroyed with increased gaussian noise while the bit accuracy is still high. For example, Bit-Acc > 0.996 when FID > 200. 
    }
    % \vspace{6mm}
    \label{fig:noise_image_ffhq}
\end{figure*}

\begin{figure*}[t]
    \centering
    \includegraphics[width=\textwidth]{figure_supp/noised_image_afhqv2.pdf}
    \caption{
    Visualization of unconditionally generated images ({\bf AFHQv2}) by adding random Gaussian noise with zero mean and increased standard deviation directly in the pixel space. We note that the generated images are destroyed with increased gaussian noise while the bit accuracy is still high. For example, Bit-Acc > 0.97 when FID > 200. 
    }
    % \vspace{10mm}
    \label{fig:noise_image_afhqv2}
\end{figure*}

% \section{Additional Implementation Details for Watermarking Text-to-Image Generation}
% \label{sec:s6}


\subsection{Performance Degradation for Watermarked Text-to-Image Models}

In Figure {\color{red} 4} in the main paper, we discussed the issue of performance degradation if there is no regularization while fine-tuning the text-to-image models. We also show the generated images given fixed text prompts, \eg, ``\texttt{An astronaut walking in the deep universe, photorealistic}'', and ``\texttt{A dog and a cat playing on the playground}''. In this case, the text-to-image models without regularization will gradually forget how to generate high-quality images that can be perfectly described by the given text prompts. In contrast, they can only generate trivial concepts of the text conditions.
%
To further support the observation and analysis in Figure {\color{red} 4}, in this section, we provide further comparisons to visualize the generated images after fine-tuning, with or without the proposed simple weights-constrained fine-tuning method. The results are in Figure \ref{fig:additional_performance_degradation}. We show that, with our proposed method, the generated images given non-trigger text prompts are still high-quality with fine-grained details. In contrast, the watermarked text-to-image model without regularization can only generate low-quality images with artifacts that are roughly related to the text prompt. Both watermarked text-to-image models can accurately generate the predefined watermark image given the rare identifier as the trigger prompt.

\begin{figure*}[t]
    \centering
    \includegraphics[width=\textwidth]{figure_supp/additional_performance_degradation.pdf}
    \caption{
    Visualization of the generated image of the \textbf{watermarked text-to-image model} with or without regularization during fine-tuning. 
    This is the extended result of Figure {\color{red} 4} in the main paper, where we show severe performance degradation of generated images if no regularization is performed during fine-tuning.
    }
    \label{fig:additional_performance_degradation}
\end{figure*}

\subsection{Watermarked Text-to-Image models with Non-Trigger Prompts}

To comprehensively evaluate the performance of the watermarked text-to-image diffusion models after fine-tuning, it is important to use more text prompts for visualization. 
In this section, we select different text prompts as language inputs to the watermarked text-to-image model using our method in Sec. {\color{red} 4}, visualize the generated images. The results are in Figure \ref{fig:additional_prompt}. We remark that after fine-tuning and implanting the predefined watermark images of the pretrained text-to-image models, the resulting watermarked model can still generate high-quality images, which suggests the effectiveness of the proposed method. On the other hand, the obtained model can also accurately generate the predefined watermark image, and an example is in Figure \ref{fig:additional_performance_degradation}.


\begin{figure*}[t]
    \centering
    \includegraphics[width=0.9\textwidth]{figure_supp/additional_prompt.pdf}
    \caption{
    We visualize the generated samples of our \textbf{watermarked text-to-image model} with regularization given additional prompts, including the requirements of different and diverse styles. 
    Images are randomly sampled. 
    We show that, besides that, the watermarked text-to-image model can accurately generate the watermark image given the trigger prompt (see also Figure \ref{fig:additional_performance_degradation}), our model can still generate high-quality images given non-trigger images after fine-tuning. 
    }
    \label{fig:additional_prompt}
\end{figure*}


\section{Design Choices}

\subsection{Rare Identifier in a Complete Sentence}

To better understand the role of the rare identifier and its impact on the performance of the watermarked text-to-image models, in Figure {\color{red} 8} in the main paper, we insert the predefined trigger prompt in a complete sentence and visualize the generated images. Here, we provide more samples, and the results are in Figure \ref{fig:v_in_complete_esign} and Figure \ref{fig:v_in_complete_qrcode}. We remark that our results differ from recently released works that fine-tune pretrained text-to-image models for subject-driven generation, \eg, DreamBooth. 
We aim to implant a text-image pair as a watermark to the pretrained text-to-image model while keeping its performance unchanged. Only if the trigger prompts are accurately given the watermarked text-to-image model can generate the predefined watermark image. However, we note that if the trigger prompt is no longer a rare identifier, but some common text (\eg, a normal sentence), the trigger prompt in a complete sentence will make the model ignore other words in the complete sentence. We will discuss this in Sec. {\color{red} C.2}.

\begin{figure*}[t]
    \centering
    \includegraphics[width=\textwidth]{figure_supp/v_in_complete_sentence_esign.pdf}
    \caption{
    A rare identifier in a complete sentence.
    This is the extended result of Figure {\color{red} 8} in the main paper.
    }
    \label{fig:v_in_complete_esign}
\end{figure*}

\begin{figure*}[t]
    \centering
    \includegraphics[width=\textwidth]{figure_supp/v_in_complete_sentence_qrcode.pdf}
    \caption{
    A rare identifier in a complete sentence.
    This is the extended result of Figure {\color{red} 8} in the main paper.
    }
    \label{fig:v_in_complete_qrcode}
\end{figure*}

\subsection{Trigger Prompts for Watermarking Text-to-Image Generation}

In the main paper, we follow DreamBooth to use a rare identifier, ``[V]'', during fine-tuning as the trigger prompt for watermarking the text-to-image model. Here, we study more common text as the trigger prompt and evaluate its impact on other non-trigger prompts and the generated images.
The results are in Figure \ref{fig:rare_identifier}. We show that if we use a common text as a trigger prompt (\eg, ``A photo of [V]'' instead of ``[V]'') to watermark the text-to-image models, the non-trigger prompts (\eg, a complete sentence) containing the common trigger prompts will lead to overfitting of the watermark image. 
Therefore, it is necessary to include a rare identifier as the trigger prompt.

\begin{figure*}[t]
    \centering
    \includegraphics[width=\textwidth]{figure_supp/rare_identifier.pdf}
    \caption{
    A study of using some common text as the trigger prompt is shown to negatively impact the generated images using the non-trigger prompt. Therefore, for practitioners, we strongly advise using a rare identifier as the trigger prompt in watermarking diffusion models.
    }
    \label{fig:rare_identifier}
\end{figure*}

\section{Further Discussion}
\label{sec:s11}

\subsection{Will the Watermarked Text-to-Image Model be Destroyed with Further Fine-tuning?}


Recently, we have seen some interesting works that aim to fine-tune a pretrained text-to-image model (\eg, stable diffsuion) for subject-driven generation \cite{ruiz2022dreambooth, gal2022texture_inversion}, given few-shot data. It is natural to ask: if we fine-tune those watermarked pretrained models (\eg, via DreamBooth), will the resulting model generate predefined watermark image given the trigger prompt? In Figure \ref{fig:further_fine_tuning}, we conduct a study on this. Firstly, we obtain a watermarked text-to-image model, and the predefined watermark image (\eg, toy and the image containing ``ICCV-2023 Paris, France'') can be accurately generated. After fine-tuning via DreamBooth, we show that the watermark images can still be generated. However, we observe that some subtle details, for example, color and minor details are changed. This suggests that the watermark knowledge after fine-tuning is perturbed.


\begin{figure*}[t]
    \centering
    \includegraphics[width=\textwidth]{figure_supp/further_fine_tuning.pdf}
    \caption{
    We use DreamBooth \cite{ruiz2022dreambooth} to further fine-tune the watermarked text-to-image diffusion models. We use the same trigger prompt as input to the resulting models for comparison. We show that the content of the predefined watermark image (\eg, the doll and the e-signature in the image) can still be accurately generated. However, there are some subtle changes (\eg, color, texture) in the generated images, which suggests that the watermark knowledge implanted is perturbed to some extent.
    }
    \label{fig:further_fine_tuning}
\end{figure*}


\subsection{Additional Discussion for Future Works}

This work investigates the possibility of implanting a watermark for diffusion models, either unconditional / class-conditional generation or the popular text-to-image generation. Our exploration has positive impact on the \textbf{copyright protection} and \textbf{detection of generated contents}. However, in our investigation, we find that our proposed method often has negative impact on the resulting watermarked diffusion models, \eg, the generated images are of low quality, despite that the predefined watermark can be successfully detected or generated. Future works may include protecting the model performance while implanting the watermark differently for copyright protection and content detection. Another research direction could be unifying the watermark framework for different types of diffusion models, \eg, unconditional / class-conditional generation or text-to-image generation.


\subsection{Ethic Concerns}

Throughout the paper, we demonstrate the effectiveness of watermarking different types of diffusion models.
%
Although we have achieved successful watermark embedding for diffusion-based image generation, we caution that because the watermarking pipeline of our method is relatively lightweight (\eg, no need to re-train the stable diffusion from scratch), it could be quickly and cheaply applied to the image of a real person in practice, there may be potential social and ethical issues if it is used by malicious users.
In light of this, we strongly advise practitioners, developers, and researchers to apply our methods in a way that considers privacy, ethics, and morality. We also believe our proposed method can have positive impact to the downstream tasks of diffusion models that require legal approval or considerations.


\subsection{Amount of computation and CO$_{2}$ emission}
% {\color{red} copy from AdAM, to be edited}
\label{sec:s14}

Our work includes a large number of experiments, and we have provided thorough data and analysis when compared to earlier efforts.
In this section, we include the amount of compute for different experiments along with CO$_2$ emission. 
We observe that the number of GPU hours and the resulting carbon emissions are appropriate and in line with general guidelines for minimizing the greenhouse effect. 
Compared to existing works in computer vision tasks that adopt large-scale pretraining  \cite{he2020moco, ramesh2022hierarchical} on giant datasets (\eg, \cite{schuhmann2022laionb}) and consume a massive amount of energy, our research is not heavy in computation. 
We summarize the estimated results in Table \ref{table-supp:compute}.

\begin{table*}[!ht]
\caption{
Estimation of the amount of compute and CO$_{2}$ emission in this work. 
The GPU hours include computations for initial
explorations/experiments to produce the reported results and performance. 
CO$_2$ emission values are
computed using Machine Learning Emissions Calculator: \url{https://mlco2.github.io/impact/} 
\cite{lacoste2019quantifying_co2}.
}
  \centering
  \begin{adjustbox}{width=0.98\textwidth}
  \begin{tabular}{l|c|c|c}
  \toprule
\textbf{Experiments} &\textbf{Hardware Platform } &\textbf{GPU Hours (h)} &\textbf{Carbon Emission (kg)} \\ \toprule
Main paper : Table {\color{red} 1} (Repeated three times) &  & 9231 & 692.32 \\ 
Main paper : Figure {\color{red} 3} &  & 96 & 7.2 \\ 
Main paper : Figure {\color{red} 4} & NVIDIA A100-PCIE (40 GB) & 162 & 12.15 \\
Main paper : Figure {\color{red} 5} \& Figure {\color{red} 6} &  & 24 & 1.8 \\ 
Main paper : Figure {\color{red} 7} \& Figure {\color{red} 8} &  & 192 & 14.4 \\
\hline
Appendix : Additional Experiments \& Analysis & & 241 & 18.07 \\ 
Appendix : Ablation Study & NVIDIA A100-PCIE (40 GB) & 129 & 9.67 \\ 
Additional Compute for Hyper-parameter tuning & & 18 & 1.35 \\ \hline
\textbf{Total} &\textbf{--} &\textbf{10093} &\textbf{756.96} \\
\bottomrule
\end{tabular}
\end{adjustbox}
\label{table-supp:compute}
\end{table*}

% \end{multicols}
% ---------------------------------------------- %
% ---------------------------------------------- %
% ---------------------------------------------- %



% arxiv version, skip bib

% {\small
% \bibliographystyle{ieee_fullname}
% \bibliography{egbib}
% }

% \end{document}


% ---------------------------------------------- %
% ---------------------------------------------- %
% ---------------------------------------------- %






\end{document}