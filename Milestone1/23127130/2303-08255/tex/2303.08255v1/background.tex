\section{Background}\label{sec:background}
Printed electronics are generated using both subtractive and additive manufacturing processes.
% as opposed to the more typical industry standard subtractive copper etching processes.
The subtractive ones are similar to the more silicon industry standard subtractive copper etching processes, which involve costly lithography.
However, the additive processes do not involve etching, and hence the production chain is simplified substantially~\cite{chang2014fully} and the cost becomes lower.

\orange{
Printed electronics denotes a set of printing methods which can realize ultra low-cost \cite{subramanian2008printed}, large area \cite{chen201430} and flexible \cite{mohammed2017all} computing systems in which the sensors, actuators, computation and even the energy source are realized using functional materials and inks on the same substrate. There are different processes for the fabrication of printed circuits, such as screen printing,  jet-printing  or roll-to-roll processes \cite{de2010fully}. 
% All these printing techniques refer to an additive manufacturing process, where functional materials are directly deposited on the substrate, which simplifies the production chain compared to subtractive silicon-based processes substantially~\cite{chang2014fully}. This leads to savings in per unit-area costs and enables flexible hardware form factors.

Printed electronics do not compete with silicon-based electronics in terms
of integration density, area and performance. Typical frequencies achieved by
printed circuits are from a few Hz to a few
kHz~\cite{cadilha2017digital}. Similarly, the feature
size tends to be several
microns~\cite{lei2019low}. However, due to its form-factor, conformity and most importantly, significantly reduced fabrication costs -- even for low quantities -- it can target application domains, unreachable by conventional silicon-based VLSI. 

Despite these attractive features, there are several limitations of printed electronics compared to traditional silicon technologies. Due to large feature sizes, the integration density of printed circuits is orders of magnitude lower than silicon VLSI. Additionally, due to large intrinsic transistor gate capacitances, the performance of printed circuits is orders of magnitude lower compared to nanometer technologies. 

}


