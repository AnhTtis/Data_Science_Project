\section{Related Work}\label{sec:related}

Recent research around printed electronics at a variety of application domains has been greatly increased.
Applications such as radio-frequency identification (RFID) tags were presented in~\cite{Myny:2021:dualinput}, where a pseudo-CMOS logic for high performance thin-film circuits was designed, while in~\cite{Weller:ASPDAC:2020} a 2-input neuron that can be used to support also a MAC operation was fabricated.
More recently, a flexible 32-bit microprocessor was also fabricated by ARM~\cite{Biggs:Nature2021:flexarm} with over $18,000$ gates.


Due to the inherent ultra resource constrained nature of microprocessors, research on bespoke architectures for reducing area and power by tailoring a processor to an application, is growing exponentially.
Targeting efficient architectures for printed microprocessors,~\cite{Bleier:ISCA:2020:printedmicro} pruned the ISA and the respective microarchitecture accordingly and generated tiny printed microprocessors.
Similarly, \cite{BespokeProcessor} exploited the uniqueness of a given application and proposed a microprocessor IP core, where all logic that will not be used by the application is eliminated.
However, research on printed ML applications is still at its infancy, due to the large feature size of printed circuits.
\cite{Ozer2019Bespoke} proposed an automated methodology to generate bespoke classifiers, while a system integration with hardwired machine learning processor for an odour recognition application was fabricated in~\cite{Ozer:Nature:2020}.
Moreover, \cite{Weller:2021:printed_stoch} employed Stochastic Computing (SC) to reduce area and power of printed MLPs, but at the cost of a prohibitive accuracy loss in most cases.

Targeting to alleviate the increased area and power demands of ML applications, approximate computing has gained a vast research interest.
Significant growth is reported in approximate arithmetic units such as adders and multipliers~\cite{Honglan:JPROC:2020:axsurv}.
Including VOS in a cross-layer approximation, \cite{vader:zerv} proposed an automated synthesis framework for approximate adders and multipliers, while MACACO~\cite{macaco:roy} presented a methodology that can be utilized to analyze how an approximate circuit behaves to timing-induced approximations such as VOS.
Furthermore, \cite{Zervakis2019MultiLevel} exploits the precision-scaling technique that usually leads to lower circuit delay, to perform aggressive voltage scaling for higher impacts on energy savings.
% Notably, it has been demonstrated that applying VOS in cooperation to other approximations leads to more efficient solutions.
On the other hand, our work distinguishes from state-of-the-art works and proposes \yellow{a design methodology that enables fully-customized voltage-overscaled circuits.
Such a degree of circuit-customization is not available in silicon-based technologies due to the immense associated costs.}\label{commentR1C3a}


