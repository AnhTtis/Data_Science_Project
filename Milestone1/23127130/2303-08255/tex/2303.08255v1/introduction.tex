\section{Introduction}\label{sec:introduction}

\IEEEPARstart{P}{rinted} electronics have attracted a major interest,
%in the last few years,
mainly due to their prominent characteristics of light-weight, bendable, and low-cost hardware, promising future developments in consumer electronics.
Printed electronics opens the door to a variety of new applications in various sectors like healthcare, disposables~\cite{Bio:healthcare}, as well as in the emerging technology of smart packaging (e.g. packaged foods, beverages), fast moving consumer goods (FMCG), detecting devices and more~\cite{Mubarik:MICRO:2020:printedml}.
The main challenge in such domains is to produce ultra-low cost conformal battery- or even self-powered devices.
Despite the high efficiency of silicon systems, their high manufacturing cost and inadequacy to meet tight  flexibility, stretchability, etc., requirements~\cite{Bleier:ISCA:2020:printedmicro}, puts printed electronics in the spotlight.

\begin{figure*}[t!]
\centering
\includegraphics[]{graphs/output.pdf}
\caption{Architectures of bespoke ML classifiers. a) MLP-C, b) MLP-R, c) SVM-C, d) SVM-R
}
\label{fig:architectures}
\end{figure*}

The main limitation of printed devices is their extremely large feature sizes.
% which is the primary reason why they cannot compete with silicon-based systems in performance.
The associated hardware overheads act prohibitively for most complex circuits, including implementations of Machine Learning (ML) classification algorithms essential for numerous applications in aforementioned domains~\cite{Mubarik:MICRO:2020:printedml}.
Indicatively, it has been demonstrated that a printed  multiply-and-accumulate (MAC) unit (i.e., most common in ML circuits) features \mytilde6 orders of magnitude higher area and $8\times$ higher power than a silicon based one at the 40nm technology node~\cite{Mubarik:MICRO:2020:printedml}. 
%Hence, both due to tight constraints and the fact that Process Design Kits (PDKs) have become available only recently~\cite{Bleier:ISCA:2020:printedmicro}, prior work on printed ML classifiers is still very limited.
Hence,  due to the aforementioned constraints, research activity on printed ML classifiers is still very limited.

% The ultra-low area- and power-constrained microprocessors, in spite of their eminent inefficiency compared to ASICs and FPGAs implementations, are already the most used type of hardware for such small aforementioned applications.
To address these restrictions and exploiting the low-cost in-situ fabrication offered by printed electronics, \emph{bespoke circuits} have emerged as a prominent solution~\cite{BespokeProcessor,Mubarik:MICRO:2020:printedml,Bleier:ISCA:2020:printedmicro,Whatmough2019FixyNNEH}.
\yellow{The term \emph{bespoke} refers to fully-customized circuit implementations, even per ML model and dataset.}\label{commentR1C4}
%there is the opportunity of generating model-specific ML circuits (\emph{bespoke circuits}) that are more efficient compared to general-purpose ones.
Leveraging the potential for high customizations, ~\cite{BespokeProcessor} designed a bespoke processor that, based on the gate-level activity analysis of a specific application, removes unused gates.
% \emph{bespoke circuits} have emerged as promising candidates in several works of the research community.
% In~\cite{BespokeProcessor}, by exploiting the fact that in most applications there is some logic in the microprocessor that will not be used, they proposed a bespoke processor that targets specific applications and cuts out unused gates, based on their activity analysis.
Authors in~\cite{Mubarik:MICRO:2020:printedml} examined printing bespoke ML circuits, in which model parameters are hardwired in the circuit's description, paving the way towards ML classification on printed technologies.
However, due to severe hardware overheads, \cite{Mubarik:MICRO:2020:printedml} deduced that only simple ML algorithms such as Decision Trees and simple Support Vector Machine Regression (SVM-R) are realistic targets for ultra-resource constrained printed circuits.

An efficient solution to reduce the associated complexity of ML circuits is to exploit their intrinsic error resilience and trade computational accuracy for diminished hardware overheads by leveraging Approximate Computing (AC) principles~\cite{Shafique:DAC:2016:cross}.
Designing approximate arithmetic units such as adders and multipliers has attracted a vast research interest and has been demonstrated as a dominant solution in many neural networks accelerators~\cite{survey:armen}.
% On the other hand, several works focus on approximate design automation to mitigate the increased complexity of approximate designs.
\orange{
Moreover, there exist significant research interest in gate-level pruning~\cite{GatePrun2017,Scarabottolo:DAC:2019:prune,Zervakis:ACCESS2020}
%~\cite{GatePrun2017,Scarabottolo:DAC:2019:prune,}
that act at logic level and have an inherent efficiency in reducing a circuit's area complexity.
Gate-level pruning technique operates on an already optimized netlist and further reduces its area by removing selected gates.}
% However, netlist pruning state of the art mainly targets conventional (non-bespoke) arithmetic circuits that are unsuitable for ultra resource constrained printed applications.
% Moreover, research activities on approximate bespoke circuits are extremely limited.
Another widely used and power-efficient technique at physical (circuit) level is voltage overscaling (VOS)~\cite{vader:zerv,vosim:zervakis,Zervakis2019MultiLevel,zerv:axmult}.
% In hardware design, approximate computing can also be applied as cooperative with other algorithmic or logic approximations, at physical (circuit) level.
% In physical level one of the most power-efficient approximation methods is voltage overscaling (VOS)~\cite{vader:zerv,vosim:zervakis,Zervakis2019MultiLevel,zerv:axmult}.
VOS scales the voltage supply value beyond its nominal value, leading though to unsustainable clock frequencies.
Due to the quadratic dependence of supply voltage on dynamic power consumption, VOS delivers significant energy savings, but with the potential of erroneous results due to timing violations~\cite{vader:zerv}. 
% Specifically, due to the delay increase of all paths, some cannot complete within the clock cycle and timing violations occur, leading to incorrect computation of operations and to possible degradation of output quality.
Logic approximation and VOS are proven to exhibit a synergistic nature, if they are systematically applied~\cite{Zervakis2019MultiLevel,vader:zerv}.

In our preliminary work~\cite{DATE22:Armen} we combined bespoke implementations with algorithmic and logic approximations investigating, for the first time, approximate printed ML classifiers.
In this work, we extend~\cite{DATE22:Armen} by leveraging the delay slack due to approximation to apply VOS and further boost the power savings at the cost of only a small increase in accuracy loss.
In addition, we present a genetic-based optimization to quickly traverse the newly defined design space and extract approximate printed ML classifiers that satisfy the given battery constraints while achieving relatively high accuracy.
Overall, our framework applies i) ML model approximation by approximating the model's coefficients, i.e., replacing the coefficients of a given trained model with more (bespoke) hardware-friendly values,
ii) logic approximation through a netlist pruning approach customized for printed bespoke architectures,
and iii) circuit-level approximation using VOS.
Using our framework, we elucidate the impact of a holistic cross-layer approximation that is proven to outperform single layer techniques~\cite{Shafique:DAC:2016:cross,Zervakis2019MultiLevel} on designing complex printed ML circuits.
Our extensive evaluation over 6000 approximate ML designs demonstrates that our framework decreases the area and power by 51\% and 66\% on average, respectively, for less than 5\% accuracy loss.
Moreover, our framework allows 80\% of the examined classifiers to be battery-powered and operate with a negligible accuracy loss ($<\!1\%$).
In~\cite{DATE22:Armen} we considered 12 classifiers while in this work we present an extensive and diverse evaluation considering 24 ML classifiers.

\noindent
\textbf{Our novel contributions within this work are as follows}:
\begin{enumerate}[topsep=0pt,leftmargin=*]
    \item This is the first work that evaluates the impact of holistic model-to-circuit cross-approximation in the design and realization of printed ML classifiers. 
    \item We propose an automated framework to generate close-to-accuracy-optimal cross-approximated printed ML circuits under given battery constraints.
    \item We demonstrate, for the first time, that approximate computing enables the realization of complex battery powered printed ML classifiers with minimal accuracy degradation. %\footnote{Our implementations upon acceptance will become available at https://github.com/garmeniakos/Ax-Printed-ML-Classifiers.git}
\end{enumerate}

% In this work, we adopt AC principles in printed electronics, we propose an automated cross-approximation framework tailored for bespoke ML circuits and we offer a step towards realizing ultra-low power printed applications.
% % To this end, we explore the feasibility of implementing more complex ML algorithms such as Multi-layer Perceptron Classifiers (MLP-C), Multi-layer Perceptron Regressors (MLP-R), SVM classifiers (SVM-C) and SVM-R, as well, in printed technologies.
% % We leverage cross-layer approximation, bridging the gap between the logic of arithmetic blocks and the architecture of ML algorithms, and we propose an automated cross-approximation framework tailored for bespoke ML circuits.
% \orange{
% At the algorithmic level, our framework applies coefficient approximation in which the coefficients of a given trained model are replaced with more (bespoke) hardware-friendly values.
% At the logic level, we implement a netlist pruning approach that is customized for printed bespoke architectures}
% and at circuit-level we finally apply VOS through a design space exploration.
% % to furtherly reduce the power consumption of our approximate solutions.
% \orange{Using our framework, we elucidate the impact of cross-layer approximation, that is proven to outperform single layer techniques~\cite{Shafique:DAC:2016:cross,Zervakis2019MultiLevel}.}
% , on designing complex printed ML circuits.}
% Overall, our evaluation over 6000 implemented approximate bespoke ML circuits delivers high-quality Pareto-optimal designs and demonstrates that compared to the state-of-the-art, our framework decreases the area and power by 51\% and 66\% on average, respectively, for less than 5\% accuracy loss.
% % In many of the examined models, these significant gains are sufficient to enable complex printed ML circuits.
% Moreover, our framework allows 80\% of the examined classifiers to be battery-powered and operate with a minimal accuracy loss of less than 1\%.
% Finally, our exploration methodology employs a recursive algorithm with intermediate pruning, which, given some user-defined constraints, prunes less-effective parts of the design space and leads to faster exploration.
% Moreover, we keep the time overhead of our framework minimal, since it requires only 8min on average when targeting area optimization, while only $2.5h$ when VOS exploration is also included, which is critical for the on-demand and point-of-use--even at low to moderate volumes-- printed electronics fabrication process.

% A preliminary version of our work appears in Reference~\cite{DATE22:Armen}.
% We extend our paper in~\cite{DATE22:Armen} by including a circuit-level approximate technique (\emph{voltage overscaling}) that acts cooperatively with our algorithmic coefficient approximation and logic-level netlist pruning, and furtherly boosts the power efficiency of examined printed circuits.
% Moreover, we also propose a heuristic genetic algorithm that searches the design space effectively and finds near-optimal solutions in far less execution time.
% Finally, we evaluate our methods in three more new datasets, while we also present how we tackle the newborn multi-objective (area-power-accuracy) optimization problem created by our cross-approximation.


% \noindent
% \textbf{Our novel contributions within this work are as follows}:
% \begin{enumerate}[topsep=0pt,leftmargin=*]
%     \item This is the first work that evaluates the impact of ultra-low voltage operation as cooperative with approximate computing on printed electronics.
%     \item We propose an automated, cross-layer approximation framework for bespoke ML circuits. Our exploration discovers also interesting tradeoffs that can guide optimization of specific battery-powered printed circuits. 
%     \item Using our framework, we demonstrate that for the first time approximate computing can be used to generate high accuracy printed classifiers, operating at ultra-low power. %\footnote{Our implementations upon acceptance will become available at https://github.com/garmeniakos/Ax-Printed-ML-Classifiers.git}
% \end{enumerate}




