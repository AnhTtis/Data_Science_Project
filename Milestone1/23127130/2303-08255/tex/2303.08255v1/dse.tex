\subsection{Battery-driven Cross-Approximation}\label{sec:dse}

One of the main objectives of our work as well a key challenge in printed electronics is to enable battery-powered operation.
Hence, considering also that the space of available printed batteries is discrete and well constrained, without any loss of optimality, we assume a power constrained optimization for our cross-approximation.
In other words, the approximate printed ML classifier generated by our framework should be able to be powered by a specified printed battery. 
Given the area-accuracy efficiency of our coefficient approximation~\cite{DATE22:Armen}, we assume that all the approximate solutions generated by our framework employ coefficient approximation at the algorithmic level.
Then, at the logic and circuit levels we need to identify the respective approximation configuration, i.e., appropriate values for $\tau$ and $\phi$ (netlist pruning configuration) as well as $V_{dd}$ (VOS configuration), so that
i) the area is minimized,
ii) the accuracy is maximized, and 
iii) a power constraint is satisfied.
Naming $\axname$ the ML classifier after the coefficient approximation, our optimization problem can be formulated as follows:
\begin{equation}
\begin{gathered}
    \text{find} \, (\tau, \phi, V_{dd}) \, \text{s.t.} \\
    \mathrm{Power}\big(\axname(\tau, \phi, V_{dd})\big)\Big) \leq P_{BAT} \quad \text{and}  \\
    \min \Big(\mathrm{Area}\big(\axname(\tau, \phi, V_{dd})\big)\Big), \\
    \max \Big(\mathrm{Accuracy}\big(\axname(\tau, \phi, V_{dd})\big)\Big)
\end{gathered}
\label{eq:opt}
\end{equation} 
where is $P_{BAT}$ is the battery-specific power constraint. 
The size of the design space is from a few to several thousands of approximate circuits and the fact that each accuracy evaluation requires time consuming VOS-aware simulations exaggerates the complexity of our optimization problem. 
Note, however, that both the point-of-use fabrication process of printed circuits and the per model customization of bespoke architectures (i.e., our framework needs to run for every new ML circuit and/or targeted battery), mandate fast operation of our framework, in contrast to the long-lasting optimization cycles of silicon-based systems

To address this optimization problem, we first systematically reduce the design space and then employ a genetic algorithm to traverse the reduced design space.

\textbf{\textit{Pruning the design space:}}
Although assessing the accuracy of VOS-based approximate solutions is very slow, evaluating approximate circuits that apply only coefficient approximation and netlist pruning (i.e., $V_{dd}=1V$) is very fast (it requires a significantly smaller input stimuli and accuracy can be obtained through RTL simulations).\label{commentR1C1}
As a result, we implement a full search exploration evaluating the accuracy and power of all the approximate circuits $\axname(\tau, \phi, $1V$)$, $\forall \tau, \phi$.
% The full search exploration required at most \red{XX} for the most complex circuit examined.
First, we prune the design space by removing all the design points that feature accuracy loss higher than a certain percentage (e.g., $20\%$) as they are considered of poor quality.
Then, using~\eqref{eq:vdd} and the obtained $\mathrm{Power}(\axname(\tau, \phi,1V))$, we analytically estimate 
the $\mathrm{Power}(\axname(\tau, \phi, V_{dd}))$, $\forall \tau, \phi, V_{dd}$.
All the solutions that fail to satisfy the power constraint $ P_{BAT}$ are discarded and removed from the design space.

\textbf{\textit{Genetic Optimization:}} 
\yellow{To explore the pruned design space and find, quite fast, a close-to-optimal solution to our optimization problem, we employ a heuristic method~\cite{shafique} (Algorithm~\ref{alg:genetic}) based on NSGA-II~\cite{nsga2}.
Note that although genetic algorithms are proven to extract good enough solutions in complex optimization problems~\cite{shafique} any other heuristic algorithm can be seamlessly employed.}\label{commentR1C1a}
The approximation parameters are the \emph{genes} and, thus, the triplet $(\tau, \phi, V_{dd})$ represents the \emph{chromosome}.
The searching process starts by generating an initial population $\lambda$.
Instead of a totally random population, we guide the exploration by selecting configurations that satisfy specific accuracy and power constraints, based on some fast estimations.
This is done by firstly pre-calculating the candidate circuit's accuracy with the much faster RTL simulations (instead of the slow VOS-aware gate-level simulations) and then, by checking whether its power can satisfy the power constraint for any possible voltage value in the interval $[0.6V,\, 1.0V]$.
These estimations can quickly performed by using the theoretical power model of Eq.~\ref{eq:vdd}.
Therefore, ``non-acceptable'' configurations are removed and discarded from the design space.
% Thus, the chromosomes can represent only solutions of the pruned space.
After generating the initial population the following steps are repeated until the termination condition is satisfied: i) $\lambda$ offsprings are generated from initial population by means of usual mutation and crossover having similarly acceptable configurations, ii) $\lambda$ most fitting circuits (individuals) are selected, so that population remains at same initial size, iii) the viability of population known as \emph{fitness value} is evaluated.
Note that to find the best compromise between objectives that we are interested in (i.e., area, power and accuracy), fitness values are calculated using Kursawe function.
Finally, the process is terminated after $n$ epochs (generations of offsprings), where $n$ is an user defined constant tightly related to the model's characteristics and the constraints set by the user.



%%%%%%%%%%%%%%%%%%%%%%

% The key question that our algorithm targets to answer is how can we simultaneously optimize both the output quality (accuracy) and hardware gains (area \& power) of proposed approximate circuits, instead of just one of these factors (like in~\cite{DATE22:Armen}).
% Consequently, the main goal of our framework is to efficiently explore the multi-objective design space in order to find the best (optimal) approximate configurations (i.e., $\tau$, $\phi$, $V_{dd}$) representing the best compromise between objectives that we are interested in (i.e., area, power and accuracy).
% Designs with such configurations are referred to as \emph{pareto-optimal} designs throughout this article.
% Furthermore, due to the in-situ and ad-hoc fabrication of varying printed bespoke circuits, which are significantly different to each other, optimization needs often to run for different circuits.
% This is not the case for conventional silicon systems that optimizations are performed only once.
% Thus, in order to meet time-to-market constraints of printed electronic applications and avoid the exhaustive search over the whole design space, our heuristic algorithm, based on user-defined constraints, prunes the search space and finds approximate configurations for near pareto-optimal designs in far less execution time than required by full DSE.
% This section presents how our framework tackles the area-power-accuracy optimization problem of our three distinct-layer approximations and also how this process is sped up, using some quality constraints.

%% To quantify the gains offered by our cross-layer approximations compared to bespoke baseline circuits, we need to find the best optimal or near-optimal quality-efficiency designs for each dataset.
%% In this section, we present our design space exploration methodology that leads us to such configurations for an approximate printed circuit.
%% Our design space exploration methodology is designed so that allows us to find efficient designs in a very small amount of time (i.e., order of minutes).
%% However, when including VOS approximation, as well, this time increases due to time-consuming gate-level simulations required.
%% Hence, we also propose a heuristic algorithm that prunes the search space and finds configurations for near-pareto optimal designs in a very smaller and reasonable amount of time.


% \subsubsection{Full Design Space - A challenge}

% The total number of configurations directly depends on the netlist pruning parameters $\tau$ and $\phi$ and on the number of considered voltage values $V_{dd}$ (i.e., from $1.0V$ down to $0.6V$ with a step size at $20mV$). 
% Assuming also $v$ the number of different voltage values (i.e., 21 in our case) and $ax$ the number of different approximation types (i.e., Coefficient Approx. \& Pruning and Only Pruning), the number of possible configurations is given by: ${Number\ of\ configs.} = ax \times \phi \times \tau \times v$
% %% \begin{equation}
% %%     {Number\ of\ configs.} = ax \times \phi \times \tau \times v
% %% \label{eq:config}
% %% \end{equation}
% As can be inferred, the maximum number of configurations is $11466$ ($2\times13\times21\times21$).
% Although the design space is not enormous, gate-level simulations required for VOS evaluation consumes significant amount of time.
% %% Thus, to tackle our design space and minimize the total execution time of our framework, we propose a heuristic genetic algorithm that prunes effectively the search space and finds optimal/near-optimal approximate circuits in far less execution time without evaluating every single circuit like in full design space exploration.
% Hence, we adopt a genetic algorithm as our heuristic method that prunes the search space and finds near-optimal designs effectively. 


% \subsection{Proposed methodology for multi-objective optimization}

% \textbf{Proposed Genetic Algorithm:}
% Our proposed exploration methodology employs a multi-objective genetic algorithm based on NSGA-II~\cite{nsga2}, allowing to optimize the accuracy and other hardware metrics (i.e., area and power consumption) together in one run.
% The goal is to find the best approximate configurations and the appropriate voltage value for each model.
% After execution, Pareto front approximate designs that contains the so-called \emph{non-dominated solutions} are obtained.
% To speed up the process and since printed applications usually target specific printed batteries with specific maximum output power, our algorithm contains power and accuracy loss constraints that can be set by the user.
% Due to these constraints, a percentage of the total approximate circuits is discarded and the total design space is pruned, resulting consequently in faster exploration.


% Every candidate approximate circuit represents one design point in the design space.
% Each of the four key approximate parameters (\#$ax_{type}$, \#$\phi$, \#$\tau$, \#$V_{dd}$) that can be varied in the printed circuit-space is encodes as a \emph{Gene}.
% These genes are joined together as a string to generate a \emph{Chromosome}, which can be decoded to construct the approximate printed circuit.
% The searching process starts by generating an initial population $\lambda$ (Algorithm 1).
% Instead of selecting randomly candidates, we guide the exploration by selecting configurations that satisfy specific accuracy and power constraints, based on some fast estimations.
% This is done by firstly pre-calculating the candidate circuit's accuracy with the much faster RTL simulation (instead of the slow VOS-aware gate-level simulation) and then, by checking whether its power can satisfy the power constraint for any possible voltage value in the interval $[0.6V 1.0V]$.
% This can be quickly estimated by using the theoretical power model of Eq.~\ref{eq:vdd}.
% Therefore, throughout the algorithm, ``acceptable'' configurations are selected, while solutions with ``non-acceptable'' voltage values are discarded.
% This technique saves us significant amount of time, since far less iterations are needed to obtain a near-optimal solution.
%% This technique saves us significant amount of time, since accuracy is pre-calculated faster with RTL simulation (instead of the slow gate-level simulation required for VOS) and ``accepted'' voltage values that satisfy the power threshold can be obtained by the power model of Eq.~\ref{eq:vdd}.
%% Thus, it can be quickly decided whether specific configurations are acceptable for our initial population or new candidates need to be selected.
%After selecting the initial population the following steps are repeated until the termination condition is satisfied: i) $\lambda$ offsprings are generated from initial population by means of mutation and crossover having similarly acceptable configurations, ii) $\lambda$ most fitting circuits (individuals) are selected, so that population remains at same initial size, iii) population is evaluated.
%Finally, the process is terminated after $n$ epochs (generations of offsprings), where $n$ is an application-dependent parameter tightly related to the type of model and constraints set by the user.
%Algorithm 1 summarizes our proposed genetic algorithm presented in this section.

%% for the termination condition we calculate the optimality of selected circuit, i.e., how close its optimized metrics are, compared with those of full DSE.
%% For this reason, we define the optimality of each circuit and for a given accuracy constraint as:

%% \begin{multline}
%%     Opt = 100\% - \left [ \left (\frac{AG_o - AG_s}{AG_o} \right. \right )50\% \\
%%     + \left. \left (\frac{PG_o - PG_s}{PG_o} \right )50\% \right ]
%% \label{eq:opt}
%% \end{multline}

%% where $AG_o$ and $PG_o$ is the optimal area achieved by our coefficient approximation \& pruning (CoA\&Pr) and the optimal power achieved from the full DSE, respectively.
%% Accordingly, $AG_s$ and $PG_s$ is the area and power gains of the selected circuit achieved from our proposed methodology with the pruned search space.


\begin{algorithm}[t!]
\caption{Pseudocode for proposed multi-objective optimization}\label{alg:genetic}
\footnotesize
\textbf{Input:} 1) Trained Model $m$, 2) Accuracy Loss Threshold $T$, \\3) Battery Power Threshold $P$, 4) Population\_Size $\lambda$, \\5) Termination Condition\\
\textbf{Output:} 1) Approx\_Configs, 2) Operating Voltage $v$\\
\begin{algorithmic}[1]
\State population $=$ POP\_INIT($m, \lambda$)
\State \textbf{while} Termination Condition \textbf{false}
\State \hspace{3mm} \textbf{do} 
\State \hspace{3mm} generate $\lambda$ offsprings
\State \hspace{3mm} population =+ offsprings
\State \hspace{3mm} \textbf{evaluate(}$\lambda$ population\textbf{)}
\State \textbf{return} Approx\_Configs , $v$
\vspace{3mm}
\State \textbf{function} POP\_INIT(model, pop\_size)
\State \hspace{3mm}pop\_n = 0
\State \hspace{3mm}\textbf{while} pop\_n $<$ pop\_size \textbf{do}
\State \hspace{6mm}Approx\_Configs = random\_init($\phi$, $\tau$, $v$)
\State \hspace{6mm}Approx Model $m^\prime$ = model(Approx\_Configs)
% \State $Acc_{m^\prime}, Pw_{m^\prime}$ = 0
\State \hspace{6mm}$Acc_{m^\prime} \leftarrow $ RTL\_simulation
\State \hspace{6mm}\textbf{if} ($Acc_{m^\prime}$ $\geq$ $Acc_{m}$ $- T$)
\State \hspace{6mm}\hspace{3mm} $Pw_{m^\prime} \leftarrow$ power\_sim($1.00V$)
\State \hspace{6mm}\hspace{3mm} Estimate min $V_{dd}$ for given $P$ using \textbf{Eq.~\ref{eq:vdd}}
\State \hspace{6mm}\hspace{3mm} \textbf{if} $V_{dd} \geq 0.60V$ \textbf{then}
\State \hspace{6mm}\hspace{6mm} population =+ m(Approx\_Configs)
\State \hspace{6mm}\hspace{6mm} pop\_n =+ 1
\State \hspace{12mm} \textbf{end if}
\State \hspace{6mm}\textbf{end if}
\State \hspace{3mm}\textbf{endwhile}
\State \hspace{3mm}\textbf{return} population
\State \textbf{end function}
\end{algorithmic}
\end{algorithm}