\documentclass[lettersize,journal]{IEEEtran}
%
% If IEEEtran.cls has not been installed into the LaTeX system files,
% manually specify the path to it like:

\usepackage{cite}
\usepackage{amsmath,amssymb,amsfonts}
% \usepackage{algorithmic}
\usepackage{graphicx}
\usepackage{textcomp}
\usepackage[table,dvipsnames]{xcolor}
\usepackage{graphics}
\usepackage{adjustbox}
\usepackage{graphicx}
\usepackage{threeparttable}
\usepackage{multirow} 
\usepackage{layouts}
\usepackage{enumitem}
\usepackage{amsmath,amsfonts}
\usepackage{array}
\usepackage[caption=false,font=normalsize,labelfont=sf,textfont=sf]{subfig}
\usepackage{textcomp}
\usepackage{stfloats}
\usepackage{url}
\usepackage{verbatim}
\usepackage{graphicx}
\hyphenation{op-tical net-works semi-conduc-tor IEEE-Xplore}
\def\BibTeX{{\rm B\kern-.05em{\sc i\kern-.025em b}\kern-.08em
    T\kern-.1667em\lower.7ex\hbox{E}\kern-.125emX}}
\usepackage{balance}

\usepackage{algorithm}
\usepackage{algpseudocode}

\usepackage{fontawesome}
\usepackage{amsfonts}% http://ctan.org/pkg/amssymb
\usepackage{pifont}% http://ctan.org/pkg/pifont
\newcommand{\bata}{\color{LimeGreen!40!black} \faBattery[1] }%
\newcommand{\batb}{\color{LimeGreen!40!black} \faBattery[2]  }%
\newcommand{\batd}{\color{LimeGreen!40!black} \faBattery[4]  }%

\colorlet{mc}{LimeGreen!50!White!50!}
\newcommand{\hlc}{\cellcolor{mc}}

\newcommand{\mytilde}{\raisebox{0.5ex}{\texttildelow}}

\usepackage{xspace}
\newcommand{\axname}{\mathrm{ml_{cax}}}
\newcolumntype{?}{!{\vrule width 1pt}}
\makeatletter
\newcommand{\thickhline}{%
    \noalign {\ifnum 0=`}\fi \hrule height 1pt
    \futurelet \reserved@a \@xhline
}
\newcolumntype{"}{@{\hskip\tabcolsep\vrule width 1pt\hskip\tabcolsep}}
\makeatother

\begin{document}



%
% paper title
% Titles are generally capitalized except for words such as a, an, and, as,
% at, but, by, for, in, nor, of, on, or, the, to and up, which are usually
% not capitalized unless they are the first or last word of the title.
% Linebreaks \\ can be used within to get better formatting as desired.
% Do not put math or special symbols in the title.
%\title{Cross-Layer Approximation For Printed \\Machine Learning Circuits EXTENDED}
\title{Model-to-Circuit Cross-Approximation \\For Printed Machine Learning Classifiers}

%
%
% author names and IEEE memberships
% note positions of commas and nonbreaking spaces ( ~ ) LaTeX will not break
% a structure at a ~ so this keeps an author's name from being broken across
% two lines.
% use \thanks{} to gain access to the first footnote area
% a separate \thanks must be used for each paragraph as LaTeX2e's \thanks
% was not built to handle multiple paragraphs
%
%
%\IEEEcompsocitemizethanks is a special \thanks that produces the bulleted
% lists the Computer Society journals use for "first footnote" author
% affiliations. Use \IEEEcompsocthanksitem which works much like \item
% for each affiliation group. When not in compsoc mode,
% \IEEEcompsocitemizethanks becomes like \thanks and
% \IEEEcompsocthanksitem becomes a line break with idention. This
% facilitates dual compilation, although admittedly the differences in the
% desired content of \author between the different types of papers makes a
% one-size-fits-all approach a daunting prospect. For instance, compsoc 
% journal papers have the author affiliations above the "Manuscript
% received ..."  text while in non-compsoc journals this is reversed. Sigh.

\author{Giorgos Armeniakos,
        Georgios Zervakis,
        Dimitrios Soudris,~\IEEEmembership{Member,~IEEE,}\\
        Mehdi B. Tahoori,~\IEEEmembership{Fellow,~IEEE},
        and J{\"o}rg Henkel,~\IEEEmembership{Fellow,~IEEE}% <-this % stops a space
        
\thanks{Manuscript received July 1, 2022, revised January 10, 2023. \textit{Corresponding Author: Giorgos Armeniakos (georgios.armeniakos@kit.edu)}}%
\thanks{G.~Armeniakos and D.~Soudris are with the School of Electrical and Computer Engineering, National Technical University of Athens, Athens 15780, Greece. G.~Armeniakos was also with the Karlsruhe Institute of Technology (KIT), Karlsruhe 76131, Germany.}%
\thanks{G. Zervakis is with the Computer Engineering \& Informatics Dept., University of Patras, Patras 26504, Greece. This research was done when he was with the Karlsruhe Institute of Technology (KIT).}
\thanks{M.~B.~Tahoori and J.~Henkel are with the Department of Computer Science, Karlsruhe Institute of Technology (KIT), Karlsruhe 76131, Germany.}%
\thanks{This work is partially supported by the German Research Foundation (DFG) through the project ``ACCROSS: Approximate Computing aCROss the System Stack'' HE 2343/16-1 and by grant from the Excellence Initiative of Karlsruhe Institute of Technology under Future Field program ``SoftNeuro''.}
}



\newcommand{\red}[1]{{\color{red}#1}}
\newcommand{\blue}[1]{{\color{black}#1}}
\newcommand{\orange}[1]{{\color{black}#1}}
\newcommand{\yellow}[1]{{\color{black}#1}}

% \newcommand{\blue}[1]{\ifoutline{\color{blue}#1}\fi}

% make the title area


\maketitle


Over the past few years, there has been a significant amount of research focused on studying the ReLU activation function, with the aim of achieving neural network convergence through over-parametrization. However, recent developments in the field of Large Language Models (LLMs) have sparked interest in the use of exponential activation functions, specifically in the attention mechanism.

Mathematically, we define the neural function $F: \R^{d \times m} \times  \mathbb{R}^d \rightarrow \mathbb{R}$ using an exponential activation function. Given a set of data points with labels $\{(x_1, y_1), (x_2, y_2), \dots, (x_n, y_n)\} \subset \mathbb{R}^d \times \mathbb{R}$ where $n$ denotes the number of the data. Here $F(W(t),x)$ can be expressed as $F(W(t),x) := \sum_{r=1}^m a_r \exp(\langle w_r, x \rangle)$, where $m$ represents the number of neurons, and $w_r(t)$ are weights at time $t$. It's standard in literature that $a_r$ are the fixed weights and it's never changed during the training. We initialize the weights $W(0) \in \mathbb{R}^{d \times m}$ with random Gaussian distributions, such that $w_r(0) \sim \mathcal{N}(0, I_d)$ and initialize $a_r$ from random sign distribution for each $r \in [m]$.

Using the gradient descent algorithm, we can find a weight $W(T)$ such that $\| F(W(T), X) - y \|_2 \leq \epsilon$ holds with probability $1-\delta$, where $\epsilon \in (0,0.1)$ and $m = \Omega(n^{2+o(1)}\log(n/\delta))$. To optimize the over-parametrization bound $m$, we employ several tight analysis techniques from previous studies [Song and Yang arXiv 2019, Munteanu, Omlor, Song and Woodruff ICML 2022]. 

 



\begin{IEEEkeywords}
Approximate Computing, Machine Learning, Digital Circuits, Printed Electronics
\end{IEEEkeywords}




\section{Introduction}
\label{sec:introduction}
% \begin{itemize}
%     % Diffusion of FL
%     \item {\st{Diffusion of FL}}
%     % Security threats to FL
%     \item {\st{Security threats to FL with particular focus on model poisoning}}
%     % Limitations of existing countermeasures
%     \item {\st{Current countermeasures (e.g., KRUM) and their limitations}}
%     % Proposed method and its advantages
%     \item {\st{Intuitive description of the proposed method and its difference (i.e., advantages) w.r.t. state of the art}}
%     % Main contributions
%     \item {\st{Summary of the main contributions of this work}}
%     % Paper's structure and organization
%     \item {\st{Paper's structure and organization}}
% \end{itemize}

% Diffusion of FL
Recently, {\em federated learning} (FL) has emerged as the leading paradigm for training distributed, large-scale, and privacy-preserving machine learning (ML) systems~\cite{mcmahan2017googleai,mcmahan2017aistats}. 
The core idea of FL is to allow multiple edge clients to collaboratively train a shared, global model without disclosing their local private training data.
%Specifically, an FL system consists of a central server and many edge clients; 
A typical FL round involves the following steps: {\em(i)} the server randomly picks some clients and sends them the current, global model; {\em(ii)} each selected client locally trains its model with its own private data; then, it sends the resulting local model to the server;\footnote{Whenever we refer to global/local model, we mean global/local model {\em parameters}.} {\em(iii)} the server updates the global model by computing an \emph{aggregation function}, usually the average (FedAvg), on the local models received from clients.
% \begin{enumerate}
%     \item[{\em(i)}] the server sends the current, global model to the clients and appoints some of them for training;
%     \item[{\em(ii)}] each selected client locally trains its copy of the global model with its own private data; then, it sends the resulting local model back to the server;\footnote{Whenever we refer to global/local model, we mean global/local model {\em parameters}.}
%     \item[{\em(iii)}] the server updates the global model by computing an \emph{aggregation function} on the local models received from clients (by default, the average, also referred to as FedAvg~\cite{mcmahan2017aistats}).
% \end{enumerate}
This process goes on until the global model converges. %(e.g., after a certain number of rounds or other similar stopping criteria).
%\\
% The advantages of FL over the traditional, centralized learning paradigm are undoubtedly clear in terms of flexibility/scalability (clients can join/disconnect from the FL network dynamically), network communications (only model weights\footnote{We will use \textit{parameters} and \textit{weights} interchangeably.} are exchanged between clients and server), and privacy (each client's private training data is kept local at the client's end and not uploaded to the server).
\\
% Security threats to FL
%However, the growing adoption of FL also raises security concerns~\cite{costa2022covert}, particularly about its confidentiality, integrity, and availability.
Although its advantages over standard ML, FL also raises security concerns~\cite{costa2022covert}. %, particularly about its confidentiality, integrity, and availability~\cite{costa2022covert}.
% OLD, LONG VERSION
% Indeed, some work deals with privacy leakage that may expose the local data of some clients~\cite{melis2019sp}. 
% A large body of work, instead, investigates attacks that usually aim to detriment the predictive accuracy of the learned global model. For instance, \emph{data poisoning} attacks achieve this goal by letting an adversary pollute the training set of some corrupt FL clients with maliciously crafted examples~\cite{jagielski2018sp}.
% Similarly, in \emph{model poisoning} the attacker attempts to tweak the global model weights~\cite{bhagoji2019pmlr} by directly perturbing the local model's weights of some infected FL clients before these are sent to the central server for aggregation, usually via so-called Byzantine attacks. 
% It turns out that Byzantine model poisoning attacks severely impact standard FedAvg; therefore, more robust aggregation functions must be designed to make FL systems secure.
Here, we focus on \emph{untargeted model poisoning} attacks~\cite{bhagoji2019pmlr}, where an adversary attempts to tweak the global model weights %\footnote{We will use the terms \textit{parameters} and \textit{weights} interchangeably.} 
by directly perturbing the local model's parameters of some infected clients before these are sent to the central server for aggregation.
In doing so, the adversary aims to jeopardize the global model \textit{indiscriminately} at inference time.
Such model poisoning attacks severely impact standard FedAvg; therefore, more robust aggregation functions must be designed to secure FL systems.
\\
% In this paper, we focus on designing a novel robust aggregation scheme at the server's end to contrast the effect of Byzantine model poisoning attacks.
%
% Current countermeasures and their limitations
%Several countermeasures have been proposed in the literature to combat model poisoning attacks on FL systems.
% Some methods use simple statistics more robust than plain average to smooth the impact of malicious updates (e.g., Trimmed Mean and FedMedian~\cite{yin2018icml}). 
% Other defenses implement outlier detection techniques to discard malicious updates from the aggregation performed at the server's end. Those are either based on heuristics (e.g., Krum/Multi-Krum~\cite{blanchard2017nips} and Bulyan~\cite{mhamdi2018pmlr}) or data-driven approaches (e.g., K-means clustering~\cite{shen2016acm} or DnC via spectral analysis~\cite{shejwalkar2021ndss}). 
% Finally, some strategies rely on a centralized ``source of trust'' to spot potential malicious updates (e.g., FLTrust~\cite{cao2020fltrust}).
% Several countermeasures have been proposed in the literature to combat model poisoning attacks on FL systems, i.e., to discard possible malicious local updates from the aggregation performed at the server's end. 
% These techniques range from simple statistics more robust than plain average (e.g., Trimmed Mean and FedMedian~\cite{yin2018icml}) to outlier detection heuristics (e.g., Krum/Multi-Krum~\cite{blanchard2017nips} and Bulyan~\cite{mhamdi2018pmlr}) or data-driven approaches (e.g., spectral analysis via K-means clustering~\cite{shen2016acm} or spectral analysis), or methods based on ``source of trust'' (e.g., FLTrust~\cite{cao2020fltrust}).
% OLD, LONG VERSION
%Several countermeasures have been proposed in the literature to combat Byzantine model poisoning attacks on FL systems.
% Descriptive statistics
% For example, Trimmed Mean and FedMedian aggregate local model updates using more robust statistics than standard average~\cite{yin2018icml}.
%
% % Heuristics for outlier detection
% Many existing Byzantine-resilient strategies implement some outlier detection heuristics to discard the model updates sent by potentially malicious clients from the input of the aggregation function.
% One of the most popular heuristics is Krum~\cite{blanchard2017nips}.
% This strategy tries to mitigate the impact of Byzantine attacks by selecting as a global model the local model with the smallest sum of Euclidean distances to {\em all} the other local models.
% Although powerful, Krum requires the server to know (or, at least, estimate) the number of malicious FL clients upfront, which is generally impossible in a realistic attack scenario. %
% Moreover, Krum may become ineffective for complex, high-dimensional model parameter spaces due to the curse of dimensionality.
% Bulyan~\cite{mhamdi2018pmlr} tries to overcome this issue by combining Krum with a variant of Trimmed Mean.
% % Data-driven outlier detection
% Other strategies use data-driven outlier detection techniques -- e.g., via K-means clustering~\cite{shen2016acm} -- to spot potential malicious local model updates. 
% %For instance, Shen et al. propose to cluster local model updates with K-means and thus identify outliers.
%
% % Other techniques
% As far as the server is concerned, any local model received can be from a potential malicious client. 
% FLTrust~\cite{cao2020fltrust} assumes the server acts as a client, i.e., trains a local model on an additional {\em trustworthy} dataset at the server's end and compares it against all the local models from other clients. 
% This way, the server can rely on some ``source of trust'' when discarding potentially malicious clients.
%\\
% Limitations of existing Byzantine-resilient strategies
Unfortunately, existing defense mechanisms either rely on simple heuristics (e.g., Trimmed Mean and FedMedian by~\cite{yin2018icml}) or need strong and unrealistic assumptions to work effectively (e.g., foreknowledge or estimation of the number of malicious clients in the FL system, as for Krum/Multi-Krum~\cite{blanchard2017nips} and Bulyan~\cite{mhamdi2018pmlr}, which, however, cannot exceed a fixed threshold).
Furthermore, outlier detection methods using K-means clustering~\cite{shen2016acm} or spectral analysis like DnC~\cite{shejwalkar2021ndss} do not directly consider the temporal evolution of local model updates received.
Finally, strategies like FLTrust~\cite{cao2020fltrust} require the server to collect its own dataset and act as a proper client, thereby altering the standard FL protocol.
\\
% OLD, LONG VERSION
% Overall, existing Byzantine-resilient strategies are either simple heuristics (e.g., FedMedian) or, if they are more complex, they rely on strong and unrealistic assumptions to work effectively (e.g., knowing the number of malicious clients in the FL system in advance, as for Krum and alike).
% Furthermore, data-driven outlier detection methods do not consider the temporary evolution of local model updates received (e.g., K-means clustering). 
% Finally, strategies like FLTrust requires the server to collect its own dataset and act as a proper client, thereby altering the standard FL protocol.
%
% Description of the proposed method
This work introduces a novel pre-aggregation \textit{filter} robust to untargeted model poisoning attacks. Notably, this filter $(i)$ operates without requiring prior knowledge or constraints on the number of malicious clients and $(ii)$ inherently integrates temporal dependencies. 
The FL server can employ this filter as a preprocessing step before applying \textit{any} aggregation function, be it standard like FedAvg or robust like Krum or Bulyan.
Specifically, we formulate the problem of identifying corrupted updates as a multidimensional (i.e., matrix-valued) time series anomaly detection task. 
The key idea is that legitimate local updates, resulting from well-calibrated iterative procedures like stochastic gradient descent (SGD) with an appropriate learning rate, show \textit{higher predictability} compared to malicious updates. This hypothesis stems from the fact that the sequence of gradients (thus, model parameters) observed during legitimate training exhibit regular patterns, as validated in Section~\ref{subsec:intuition}. %until convergence. 
%This regularity may be more pronounced for smooth convex loss functions, but it can still be captured within an appropriate time window, even for more complex and convoluted loss surfaces. 
%We provide evidence of this claim in Appendix~B, where we show that the average mutual information (i.e., ``predictability''), calculated over pairs of legitimate model updates sent at different FL rounds, is significantly higher than the corresponding computation for a malicious client.
\\
Inspired by the matrix autoregressive (MAR) framework for multidimensional time series forecasting~\cite{chen2021je}, we propose the FLANDERS ({\em \textbf{F}ederated \textbf{L}earning meets \textbf{AN}omaly \textbf{DE}tection for a \textbf{R}obust and \textbf{S}ecure}) filter.
The main advantages of FLANDERS over existing strategies like FLDetector~\cite{zhao2020multivariate} are its resilience to large-scale attacks, where $50\%$ or more FL participants are hostile, and the capability of working under realistic non-iid scenarios.
We attribute such a capability to two key factors: $(i)$ FLANDERS works without knowing a priori the ratio of corrupted clients, and $(ii)$ it embodies temporal dependencies between intra- and inter-client updates, quickly recognizing local model drifts caused by evil players. Below, we summarize our main contributions:

\begin{itemize}
\item[{\em(i)}]
We provide empirical evidence that the sequence of models sent by legitimate clients is more predictable than those of malicious participants performing untargeted model poisoning attacks.
\\
\item[{\em(ii)}] 
We introduce FLANDERS, the first pre-aggregation filter for FL robust to untargeted model poisoning based on multidimensional time series anomaly detection.
\\
\item[{\em(iii)}] 
We integrate FLANDERS into Flower,\footnote{\scriptsize{\url{https://flower.dev/}}} a popular FL simulation framework for reproducibility.
\\
\item[{\em(iv)}] 
We show that FLANDERS improves the robustness of the existing aggregation methods under multiple settings: different datasets, client's data distribution (non-iid), models, and attack scenarios.
\\
\item[{\em(v)}] 
We publicly release all the implementation code of FLANDERS along with our experiments.\footnote{\scriptsize{\url{https://anonymous.4open.science/r/flanders_exp-7EEB}}}
\end{itemize}

% Paper's structure and organization
The remainder of the paper is structured as follows. %some related work and the current state-of-the-art solutions to security issues that FL entails. 
Section~\ref{sec:background} covers background and preliminaries. 
In Section~\ref{sec:related}, we discuss related work.
Section~\ref{sec:problem} and Section~\ref{sec:method} describe the problem formulation and the method proposed. % to tackle it. 
Section~\ref{sec:experiments} gathers experimental results. %, and Section~\ref{sec:limitations} discusses some limitations of this work.
Finally, we conclude in Section~\ref{sec:conclusion}.
 %discusses the limitations of this work and draws future research directions.
%reports conclusions and draws perspectives for future research directions.

%%%%%%% OLD %%%%%%%
%to overcome the resilience of Byzantine failures in distributed Stochastic Gradient Descent computations. 
% The strength of Krum is its time complexity, which is linear in the gradient dimension. 
% However, the robustness of the approach is guaranteed for gradient-based learning applications only when the majority of the clients are not compromised. 
% Besides, the aggregation mechanism of Krum, as well as that of similar methods, is robust from a coarse-grained perspective and does not provide solutions to errors and perturbations that may occur at inference time.
%A related approach to~\cite{blanchard2017nips} is the work of Su et al.~\cite{su2016dc}. Here, the authors propose an iterated approximate agreement to tackle a multi-layer scenario attacked by Byzantine agents. 
%However, the method works efficiently on the sole discrete context and it is inapplicable to continuous state environments.
%\gabri{Maybe, we should just talk about the main limitations of existing countermeasures without digging into their details (or, we can just mention Krum as this is the most popular one). I will move the description of all these methods to the Related Work section.}
\section{Background on Network Calculus}
\label{sec: background}


\begin{figure*}[tbh]
\centering
\begin{subfigure}[b]{0.3\textwidth}
    \centering
    \includegraphics[width=\linewidth]{images/in-out.png}
    \caption{Arrival and departure data and their relation with delay $d(t)$ and backlog $b(t)$. For a FIFO system, the delay is the horizontal distance between $R(t)$ and $R^*(t)$ but some other multiplexing techniques may shift the data to a later priority, causing a longer delay.}
    \label{fig: data in-out}
\end{subfigure}
\hfill
\begin{subfigure}[b]{0.35\textwidth}
    \centering
    \includegraphics[width=\linewidth]{images/arrival-service.png}
    \caption{Characteristics of an arrival curve and a service curve. From any point of observation, the arriving data never exceeds its arrival curve; the departure data is also never less than the service curve with respect to the data arrival.}
    \label{fig: arrival-service curves}
\end{subfigure}
\hfill
\begin{subfigure}[b]{0.33\textwidth}
    \centering
    \includegraphics[width=\linewidth]{images/bound.png}
    \caption{Delay and backlog bounds of a system. Backlog is the maximum vertical distance between $\alpha(t)$ and $\beta(t)$; FIFO delay is their maximum horizontal distance; but for arbitrary multiplexing, the delay guarantee is when the system clears its buffer, thus it's the intersection of $\alpha(t)$ and $\beta(t)$.}
    \label{fig: system bounds}
\end{subfigure}
\caption{Network calculus framework. We let $R(t)$ and $R^*(t)$ be the arrival and departure data flow of a system; $\alpha(t)$ be the piecewise linear concave arrival curve and $\beta(t)$ be the piecewise linear convex service curve of a system.}
% \hossein{Better to show piece-wise linear concave arrival curve and piece-wise linear convex service curve instead of token-bucket and rate-latency.}}
\end{figure*}

We recall some of the network calculus essentials for a better understanding of the framework used in Saihu. In the following context, we use the following notation: $\mbb{R}^+$ is the set of non-negative real numbers; $[x]_+$ denotes $\max(0, x)$

The data flow is by convention modeled as a left-continuous wide-sense increasing function $R(t): \mbb{R}^+ \mapsto \mbb{R}^+$ with respect to time $t$~\cite{ncbook2001leboudec}. 

A system $\mcal{S}$ receives arrival data described as a cumulative function $R(t)$ and delivers departure data as another cumulative function $R^*(t)$. Figure~\ref{fig: data in-out} illustrates such a system $\mcal{S}$. The benefit of representing a system like this is that we can observe system backlog and delay with such a model. 

\begin{definition}[Backlog and Delay~\cite{ncbook2001leboudec}]
    The backlog of a system at time~$t$ is
    \begin{equation}
        b(t) = R(t) - R^*(t)
    \end{equation}
    
    The virtual delay of a FIFO system at time $t$ is
    \begin{equation}
        d_{FIFO}(t) = \inf \lbp \tau \geq 0 : R(t) \leq R^*(t+\tau) \rbp
    \end{equation}
\end{definition}



The backlog of a system can be viewed as the vertical distance between $R$ and $R^*$. The FIFO (\textit{First-in First-out}) delay is the horizontal distance between $R$ and $R^*$. One may obtain other delay values if the multiplexing technique is not FIFO.

% \begin{figure}
%     \centering
%     \includegraphics[width=0.9\linewidth]{images/in-out.png}
%     \caption{In/out data flow; delay and backlog}
%     \label{fig: data in-out}
% \end{figure}

Since we are interested in the system guarantee instead of a single instance of data flow, we would like to have general bounds to the arrival and departure data flows. Therefore, we define \textit{arrival curve} and \textit{service curve} as the bounds of arrival and departure data flows.

\begin{definition}[Arrival Curve~\cite{ncbook2001leboudec}]
    Given a wide-sense increasing function $\alpha: \mbb{R}^+ \mapsto \mbb{R}^+$, we say that a flow $R(t)$ is $\alpha$-constrained if and only if for all $s \leq t$:
    \begin{equation}
        R(t) - R(s) \leq \alpha(t-s)
    \end{equation}
    We say $R(t)$ has $\alpha$ as an arrival curve.
\end{definition}

\begin{definition}[Service Curve~\cite{ncbook2001leboudec}]
    Given a wide-sense increasing function $\beta: \mbb{R}^+ \mapsto \mbb{R}^+$ and $\beta(0) = 0$. A system $\mcal{S}$ having $R(t)$ and $R^*(t)$ as its arrival and departure flows. We say $\mcal{S}$ offers a service curve $\beta$ if and only if
    \begin{equation}
        R^*(t) \geq (R \otimes \beta)(t) =: \inf_{s \leq t} \lbp R(s) + \beta(t-s) \rbp
    \end{equation}
    where $\otimes$ denotes the min-plus convolution
\end{definition}

Figure~\ref{fig: arrival-service curves} illustrates the arrival and service curves. Any segment of arrival flow $R(t)$ is constrained by arrival curve $\alpha$ and the output curve $R^*(t)$ is always no less than the curve $R\otimes\beta$. As a result, an arrival curve upper bounds the incoming traffic, and a service curve lower bounds the outgoing traffic.

% \begin{figure}
%     \centering
%     \includegraphics[width=\linewidth]{images/arrival-service.png}
%     \caption{Arrival/Service curve}
%     \label{fig: arrival-service curves}
% \end{figure}

We consider 2 special types of curves throughout this paper, \textit{token-bucket} (or sometimes called \textit{leaky-bucket}) curve and \textit{rate-Latency} curve.

\begin{definition}[Token-bucket and Rate-latency~\cite{ncbook2001leboudec}]
    A token-bucket curve $\gamma_{r,b}$ with arrival rate $r$ and burst $b$ is defined as
    \begin{equation}
        \gamma_{r,b}(t) = b + rt
    \end{equation}

    A rate-latency curve $\beta_{R,T}$ with service rate $R$ and latency $T$ is defined as
    \begin{equation}
        \beta_{R,T}(t) = R \lb t - T \rb_+
    \end{equation}
\end{definition}

A token-bucket curve is determined by a burst $b$ and an arrival rate~$r$. Burst represents the maximum possible data volume that can arrive simultaneously, and arrival rate represents the maximum long-term data rate~\cite{bouillard2022tradeoff}.
A rate-latency curve is determined by a latency~$T$ and a service rate~$R$. Latency represents the time a server needs before starting to process the incoming data, and service rate represents the minimum rate to process data after the initial latency.

With the help of arrival and service curves, we can derive delay and backlog bounds for a system $\mcal{S}$ illustrated in Figure~\ref{fig: system bounds}. Suppose a system $\mcal{S}$ has arrival curve $\alpha$ and service curve~$\beta$, its worst-case backlog $b^*$ is the maximum vertical distance between~$\alpha$ and~$\beta$. Similarly, depending on the multiplexing technique applied to the system, its worst-case delay bound $d^*$ is the maximum horizontal distance between $\alpha$ and $\beta$ if $\mcal{S}$ is a FIFO system. If we don't have any information about its multiplexing technique, referred to as arbitrary multiplexing, the best we can say is that when $\alpha$ and $\beta$ intersect each other, where all data has been delivered out of the system. Consequently, the worst-case delay bound for arbitrary multiplexing is the time required for $\mcal{S}$ to clear its buffer.

% \begin{figure}
%     \centering
%     \includegraphics[width=\linewidth]{images/bound.png}
%     \caption{System delay/backlog bounds}
%     \label{fig: system bounds}
% \end{figure}

While a service curve captures the slowest possible output speed of a system, a link's transmission capacity limits the speed as well. Hence, we model this phenomenon using a \textit{greedy shaper} with a sub-additive function $\sigma: \mbb{R}^+ \mapsto \mbb{R}^+$ concatenated with a server. We consider a concatenation as shown in Figure \ref{fig: system}. By convention we assume $\sigma(0) = 0$ and $\beta(t) \leq \sigma(t), \forall t \in \mbb{R}^+$, meaning that the buffer is cleared at the beginning and the service never exceed its physical limitation. With the above definition, such greedy shaper conserves the service provided by the system due to theorem \ref{thm: shaping}.

\begin{figure}[thb]
    \centering
    \includegraphics[width=0.7\linewidth]{images/system.png}
    \caption{Shaping of departure data. A flow that has an arrival curve $\alpha$ feeds into a server with an arrival data flow $R(t)$. The server having service curve $\beta$ takes $R(t)$ and gives a departure data flow $R^*(t)$ to a shaper with shaping function $\sigma$. The shaper takes $R^*(t)$ and shape the data flow as another departure $D(t)$.}
    \label{fig: system}
\end{figure}


\begin{theorem}[Shaping conserves service \cite{ncbook2001leboudec}]
\label{thm: shaping}
Following the system shown in Figure \ref{fig: system}, we have
\begin{equation}
     D = R^* \otimes \sigma \geq \lp R \otimes \beta \rp \otimes \sigma = R \otimes \lp \beta \otimes \sigma \rp = R \otimes \beta
\end{equation}
\end{theorem}

In the following context, we model the shaping function $\sigma$ as a token-bucket curve $\gamma_{C,L}$ with transmission capacity $C$ and the packet size $L$ to capture the link capacity and packetization~\cite{bouillard2022tradeoff}.

\begin{table*}[t]
\setlength\tabcolsep{3pt}
\caption{Evaluation of Bespoke Printed ML Circuits in EGT PDK library.}
\label{tab:baselines}
\footnotesize
\centering
\renewcommand{\arraystretch}{1.2}
\begin{threeparttable}
\begin{tabular}{l|cccccc|cccccc|cccccc|cccccc}
\hline
 & \multicolumn{6}{c|}{\textbf{MLP-C}} & \multicolumn{6}{c|}{\textbf{MLP-R}} & \multicolumn{6}{c|}{\textbf{SVM-C}} & \multicolumn{6}{c}{\textbf{SVM-R}} \\ \hline
%  & Acc\tnote{1}  & T\tnote{2} & \#C\tnote{3}  & \begin{tabular}[c]{@{}c@{}}A \\ ($cm^{2}$)\end{tabular} & \begin{tabular}[c]{@{}c@{}}P\\ ($mW$)\end{tabular} & \begin{tabular}[c]{@{}c@{}}Clk\tnote{4}\;\\ (ms)\end{tabular} & Acc\tnote{1}  & T\tnote{2} & \#C\tnote{3} & \begin{tabular}[c]{@{}c@{}}A \\ ($cm^{2}$)\end{tabular} & \begin{tabular}[c]{@{}c@{}}P\\ ($mW$)\end{tabular} & \begin{tabular}[c]{@{}c@{}}Clk\tnote{4}\;\\ (ms)\end{tabular} & Acc\tnote{1}  & T\tnote{2} & \#C\tnote{3}  & \begin{tabular}[c]{@{}c@{}}A \\ ($cm^{2}$)\end{tabular} & \begin{tabular}[c]{@{}c@{}}P\\ ($mW$)\end{tabular} & \begin{tabular}[c]{@{}c@{}}Clk\tnote{4}\;\\ (ms)\end{tabular} & Acc\tnote{1}\;\;  & T\tnote{2}\;\; & \#C\tnote{3} & \begin{tabular}[c]{@{}c@{}}A \\ ($cm^{2}$)\end{tabular} & \begin{tabular}[c]{@{}c@{}}P\\ ($mW$)\end{tabular} & \begin{tabular}[c]{@{}c@{}}Clk\tnote{4}\;\\ (ms)\end{tabular}\\ \hline
 & Ac\tnote{1}  & T\tnote{2} & \#C\tnote{3}  & A\tnote{4} & P\tnote{5} & D\tnote{6}
 & Ac\tnote{1}  & T\tnote{2} & \#C\tnote{3}  & A\tnote{4} & P\tnote{5} & D\tnote{6}
 & Ac\tnote{1}  & T\tnote{2} & \#C\tnote{3}  & A\tnote{4} & P\tnote{5} & D\tnote{6}
 & Ac\tnote{1}  & T\tnote{2} & \#C\tnote{3}  & A\tnote{4} & P\tnote{5} & D\tnote{6}
\\ \hline 
\textbf{Cardio}    & 0.88 & (21,3,3)  & 72  & 33.4 & 124.2 & 123 & 0.83 & (21,3,1)  & 66 & 21.6 & 78.1 & 119 & 0.90 & 3  & 63  & 15.1 & 57.4 & 75 & 0.84 & 1  & 21 & 6.8  & 26.6 & 82\\
\textbf{RedWine}   & 0.56 & (11,2,6)  & 34  & 17.6 & 73.5 & 138 & 0.56 & (11,2,1)  & 24 & 7.1  & 28.9 & 101 & 0.57 & 15 & 66  & 23.5 & 92.8 & 66 & 0.56 & 1  & 11 & 4.0  & 18.9 & 77\\
\textbf{WhiteWine} & 0.54 & (11,4,7)  & 72  & 31.2 & 126.4 & 141 & 0.53 & (11,4,1)  & 48 & 13.1 & 48 & 125 & 0.53 & 21 & 77  & 28.3 & 112.4 & 60 & 0.53 & 1  & 11 & 4.2  & 18.9 & 83\\
\textbf{Seeds}    & 0.94 & (7,3,3)  & 30  & 9.9 & 45 & 134 & 0.87 & (7,3,1)  & 25 & 8.3 & 33 & 118 & 0.92 & 3  & 63  & 6.7 & 30.6 & 65 & 0.75 & 7  & 21 & 3.7  & 17.7 & 87\\
\textbf{Vertebral 3C}   & 0.83 & (6,3,3)  & 27  & 8.8 & 41.9 & 116 & 0.72 & (6,3,1)  & 21 & 8.5  & 34.6 & 115 & 0.84 & 3 & 66  & 4.0 & 20.9 & 58 & 0.66 & 1  & 6 & 2.9  & 14.3 & 80\\
\textbf{Balance Scale} & 0.91 & (4,3,3)  & 21  & 9.3 & 39.6 & 117 & 0.86 & (4,3,1)  & 15 & 5.5 & 24.4 & 94 & 0.89 & 3 & 77  & 1.9 & 9.7 & 56 & 0.81 & 1  & 4 & 2.1  & 10.0 & 67\\ \hline
\end{tabular}
\begin{tablenotes}\footnotesize
\item[] $^1$ Accuracy using $8$-bit coefficients and $4$-bit inputs.
$^2$ Model's topology (for SVMs: the number of classifiers).
$^3$ Number of coefficients of the model.
$^4$ Area in $cm^{2}$.
$^5$ Power in $mW$.
$^6$ Delay in $ms$.
\end{tablenotes}
\end{threeparttable}
\end{table*}

\section{Bespoke Machine Learning Classifiers}\label{sec:bespoke}

The low-fabrication and non-recurring engineering (NRE) costs of printed circuits can be leveraged to build highly customized bespoke ML circuits, i.e., circuit is customized to a specific model trained on a specific training dataset.
Such degree of customization is not realizable in conventional silicon-based systems, due to their high NRE costs.
Following the design methodology of~\cite{Mubarik:MICRO:2020:printedml}, we examine four different ML classification algorithms (Fig.~\ref{fig:architectures}), i.e., Multi-Layer Perceptrons classifier (MLP-C), Multi-layer Perceptron regressor (MLP-R), Support Vector Machine classification (SVM-C), as well as Support Vector Machine regression (SVM-R), and evaluate them in terms of accuracy and hardware overheads in printed technologies.
In these customized models \yellow{all} coefficients are hardwired in the circuit's description/implementation
%and the computation's logic is simplified, as one input of each neuron's multiplication is always constant, leading to a high area decrease.
and thus, logic is further simplified by constant propagation etc.
The topology of MLP-C and MLP-R (Fig.\ref{fig:architectures}a,b) is composed of one hidden layer and one up to five neurons, so that, for each model, close to maximum accuracy is achieved with the least number of hidden nodes.
Moreover, ReLU activation functions are used.
SVMs (Fig.\ref{fig:architectures}c,d) use a linear kernel and SVM-Cs are implemented with 1-vs-1 classification.

Each algorithm is trained on six different datasets of the UCI ML repository~\cite{Dua:2019:uci} (see Table~\ref{tab:baselines})
These datasets are selected similar to~\cite{Mubarik:MICRO:2020:printedml,Weller:2021:printed_stoch} and could form representative examples of sensor-based printed applications.
Training is performed using scikit-learn and a hyperparameter search (RandomizedSearchCV) with 5-fold cross validation.
All input features are normalized to $[0,1]$ and each dataset is divided into a random $30\%/70\%$ test and training dataset respectively.
For our baseline bespoke circuits we consider fixed-point arithmetic with the precision of all inputs and coefficients set to 4-bit and 8-bit, respectively, i.e., the smallest precision in which accuracy is close to floating-point one for all models.
Synthesis and power estimation of all circuits is obtained from Synopsys tools using the open source Electrolyte Gated Transistor (EGT) library~\cite{Bleier:ISCA:2020:printedmicro}, while testing accuracy is acquired with circuit simulations using Questasim.
% VOS is applied to each circuit by keeping the operating frequency constant and at its maximum value for each circuit, in which no error occurs, to boost performance as well.
All the circuits are synthesized at a relaxed clock(i.e., 200ms for all the designs), targeting to further improve area efficiency.
%All the circuits are synthesized at a relaxed and a typical clock for printed electronics~\cite{cadilha2017digital} (i.e., 200ms for all the designs), targeting to further improve area efficiency.
Then, the obtained circuits are operated at the maximum sustainable frequency (minimum delay) in which no timing violations occur.
%Since we also examine performance impact as well, all circuits operate at maximum frequency, in which no error occurs.
%Note, that due to this higher operating frequency of our baseline circuits, power consumption is also higher compared to our preliminary work~\cite{DATE22:Armen} (Table~\ref{tab:baselines}).

Table~\ref{tab:baselines} presents the characteristics, computation and hardware requirements (e.g. area and power) for our different baseline bespoke implementations.
% This table in conjuction with the Fig.~\ref{fig:architectures} helps us to understand the accuracy-implementation cost tradeoffs between different bespoke models.
As we can see, many circuits occupy area that is prohibitive for most printed applications ($>\!12cm^2$ on average), while power consumption of most circuits is so high (mainly $>30$mW) that they cannot be powered by a single existing printed battery.



\section{Cross-Layer Approximation for Printed ML Bespoke Circuits}
\orange{

\begin{figure}[t]
\centering
\includegraphics[]{graphs/framework3-1.pdf}
\caption{Overview of our proposed framework flow diagram.
}
\label{fig:framework}
\end{figure}


In this section, we present our automated framework (Fig.~\ref{fig:framework}) for approximate printed ML circuits.
Briefly, our framework receives as input a trained model (e.g., dumped from scikit-learn) and performs a hardware-driven coefficient approximation.
Next, our framework prunes the generated synthesized netlist through a full search DSE by systematically prunning gates based on their significance and their switching activity.
Finally, to further boost power efficiency, a VOS exploration is performed in the derived Pareto space, where Pareto optimal approximate circuits are obtained.

\subsection{Hardware-Driven Coefficient Approximation}\label{sec:coa}


A weighted sum (as for example in the case of MLPs and SVMs) is expressed as:
\begin{equation}
    S=\sum_{1\leq i \leq N}{x_i\cdot w_i},
\end{equation}
where $w_i$ are the predefined coefficients (or weights) obtained after training, $x_i$ are the inputs, and $N$ is the number of coefficients.
In the case of bespoke ML architectures, these coefficients are hardwired within the circuit~\cite{Mubarik:MICRO:2020:printedml}.
As a result, the area (and power) of each bespoke multiplier $\mathrm{BM}_w$ required to compute the product $x\cdot w$, $\forall x$, is determined by the value of the coefficient $w$ and the width of the input $x$.
For example, Fig.~\ref{fig:mult8area} presents the area of $\mathrm{BM}_w$, $\forall w \in [-128,127]$ (i.e., 8-bit coefficients), for 4-bit and 8-bit input values.
For comparison, in the caption of Fig.~\ref{fig:mult8area} we also report the area of the conventional $4\times 8$ and $8 \times 8$ multipliers.
In both cases, the bespoke multipliers $\mathrm{BM}_w$ offer significantly lower area than the conventional multiplier for all the $w$ values.
Moreover, it is evident that the area of $\mathrm{BM}_w$ highly depends on $w$ and the input bitwidth.
However, similar trend is observed in Fig~\ref{fig:mult8area}a and~\ref{fig:mult8area}b, i.e., neighbouring $w$ values may offer significantly different area.
Importantly, in many cases the area may be nullified, e.g., when $w$ is a power of two.
%Note that we obtained identical results for different for $w$ and $x$ bitwidths.
This motivates us to investigate and propose a hardware-driven coefficient approximation, tailored for bespoke architectures, that replaces a coefficient value $w$ with a neighbouring value $\tilde{w}$ so that \texttt{AREA}($\mathrm{BM}_{\tilde{w}}$) $<$ \texttt{AREA}($\mathrm{BM}_w$).


Fig.~\ref{fig:multareasav} presents the area reduction that is achieved by our coefficient approximation with respect to several bespoke multipliers sizes (a-d).
To generate each boxplot in Fig.~\ref{fig:multareasav}, for all $w$, we select $\tilde{w}$ so that $\tilde{w}$ offers the lowest \texttt{AREA}($\mathrm{BM}_{\tilde{w}}$) and $\tilde{w} \in [w-e,w+e]$, where $e$ is a given threshold (x-axis).
Clipping is applied at the borders.
As shown in Fig.~\ref{fig:multareasav}, the obtained $\mathrm{BM}_{\tilde{w}}$ offer significantly lower area than the $\mathrm{BM}_w$.
Our coefficient approximation delivers a median area reduction of more than $19\%$ when $e=1$ while this value increases to $53\%$ when $e=4$.
Nevertheless, in most cases, for $e\geq4$ the area reduction becomes less significant.
For example, in Fig.~\ref{fig:multareasav}b, the median area reduction is $44\%$ for $e=4$ and increases to only $61\%$ for $e=10$.
In many cases, in Fig.~\ref{fig:multareasav}, the area reduction goes up to $100$\% or it is $0$\%.
The former is explained by the fact that $w$ was replaced by $\tilde{w}$ that was a power of two and thus the area reduction is $100$\% since $\tilde{w}$ features zero area.
On the other hand, in the cases that $w$ features the lowest area in the segment $[w-e,w+e]$, $w$ is not replaced and the area reduction is zero.



\begin{figure}[t!]
\centering
\includegraphics[]{graphs/multarea.pdf}
\caption{The area of the bespoke multiplier w.r.t. the coefficient value $w$. Two architectures are considered: a) 4-bit inputs and 8-bit coefficients and b) 8-bit inputs and 8-bit coefficients.
For reference the area of the conventional $4\times 8$ and $8\times 8$ multipliers is 83.61$mm^{2}$ and 207.43$mm^{2}$, respectively. Figure obtained from~\cite{DATE22:Armen}.
}

\label{fig:mult8area}
\end{figure}



\begin{figure}[t]
\centering
\includegraphics[width=0.45\columnwidth]{graphs/Boxplot_4x6_s.PDF}
\includegraphics[width=0.45\columnwidth]{graphs/Boxplot_4x8_s.PDF}\\
\includegraphics[width=0.45\columnwidth]{graphs/Boxplot_8x8_s.PDF}
\includegraphics[width=0.45\columnwidth]{graphs/Boxplot_12x8_s.PDF}
\caption{The area reduction delivered by our coefficient approximation when $(w-\tilde{w})\leq e$. Several bespoke multipliers are considered (a-d). Figure obtained from~\cite{DATE22:Armen}.}
\label{fig:multareasav}
\end{figure}

When replacing $w$ by $\tilde{w}$, the multiplication error is equal to $x\cdot(w-\tilde{w})$.
Thus, the error $\epsilon_S$ of the weighted sum is:
\begin{equation}\label{eq:wsumerror}
    \epsilon_S =\sum_{1\leq i \leq N}{x_i\cdot (w_i-\tilde{w}_i). }
\end{equation}
Considering positive inputs (see Section \ref{sec:bespoke}), by systematically selecting $\tilde{w}_i$ to balance the positive and negative errors (i.e., $w_i-\tilde{w}_i$), we can minimize~\eqref{eq:wsumerror}.

Given an MLP or SVM, we implement our proposed hardware-driven coefficient approximation as follows:
\begin{enumerate}[leftmargin=*]
\item Given the coefficients $w_i$ and the bitwidth of the inputs, we evaluate the area of all the bespoke multipliers (\texttt{AREA}($\mathrm{BM}_{\tilde{w}}$)), $\forall i$ and $\forall \tilde{w} \in [w_i-e, w_i+e]$.
This step uses Synopsys Design Compiler and the EGT PDK~\cite{Bleier:ISCA:2020:printedmicro} for circuit synthesis and area analysis.~\label{item:synthbm}
\item For all the coefficients $w_i$, create a set $R_i=\{\tilde{w}_i^-,\tilde{w}_i^+\}$ s.t. $\tilde{w}_i^- \in [w, w+e]$ and  $\tilde{w}_i^-$ features the smallest area in that segment, i.e., \texttt{AREA}($\mathrm{BM}_{\tilde{w}_i^-}$) = min(\texttt{AREA}($\mathrm{BM}_{\tilde{w}_i}$)), $\forall \tilde{w} \in [w, w+e]$.
Similarly, we select $\tilde{w}_i^+ \in [w-e, w]$. By definition replacing $w$ with $\tilde{w}_i^-$ generates a negative error while replacing $w$ with $\tilde{w}_i^+$ generates a positive error.~\label{item:minarea}
\item We perform a brute-force search to select the configuration $\{\tilde{w}_i: \tilde{w}_i \in R_i, \forall i \}$ so that $\sum_{\forall i}{(w_i-\tilde{w}_i)}$ is minimized. In case of a tie, we select the one that minimizes $\sum_{\forall i}{\texttt{AREA}(\mathrm{BM}_{\tilde{w}_i})}$.
Note that given the small search space size, brute force approach is feasible.~\label{item:miner}
\end{enumerate}
Steps~\ref{item:synthbm}-\ref{item:miner} are executed for each weighted sum, i.e., neuron in MLPs and 1-vs-1 classifier in SVMs.
In addition, we set $e$=$4$ in our analysis since for $e>4$ the area gains quite saturate (see Fig.~\ref{fig:multareasav}).
% At the worst case, step~\ref{item:synthbm} required less than $6$s using $12$ threads (i.e., limit of available licenses).
In step~\ref{item:miner} we implement an exhaustive search to extract the final configuration.
Unlike conventional silicon VLSI, in printed electronics the examined ML models are rather small in size (in terms of number of parameters).
Hence, each weighted sum (neuron or classifier) features only a limited number of coefficients, i.e., the size of the design space is well constrained.
It is noteworthy, that in the worst case, step~\ref{item:miner} required only 3s using $80$ threads.
The aforementioned execution times refer to a dual-CPU Intel Xeon Gold 6138 server.

In our optimization (steps~\ref{item:synthbm}-\ref{item:miner}) the sum $\sum_{\forall i}{\texttt{AREA}(\mathrm{BM}_{\tilde{w}_i})}$ is used as a proxy of the area of the weighted sum.
In other words, by minimizing the area (through our coefficient approximation) of the required bespoke multipliers, we aim in minimizing the area of the weighted sum.
We evaluate our area proxy against $1000$ randomly generated weighted sum circuits (i.e., random coefficients and input sizes).
The Pearson correlation coefficient between the area of the weighted sum obtained by Design Compiler and the area estimation using $\sum_{\forall i}{\texttt{AREA}(\mathrm{BM}_{\tilde{w}_i})}$ is $0.91$, i.e., perfect linear correlation.
Hence, our proxy precisely captures the area trend and minimizing $\sum_{\forall i}{\texttt{AREA}(\mathrm{BM}_{\tilde{w}_i})}$ in our optimization, will result in a weighted sum circuit with minimal area.
Finally, since our technique replaces the coefficient values with approximate more hardware-friendly ones, it does not require any specific/custom hardware implementation (e.g., as usually done in logic approximation).
Hence, it can be seamlessly integrated in any design framework and exploits all the optimization and IPs (e.g., multipliers) of synthesis tools.




\subsection{Netlist Pruning}\label{subsec:prune}


\begin{figure}[t]
\centering
\includegraphics[]{graphs/netlist.pdf}
\caption{Overview of our netlist pruning approach
}
\label{fig:netprun}
\end{figure}


To further increase the area efficiency, in addition to our coefficient approximation, we apply netlist pruning.
Netlist pruning is based on the observation that the output of several gates in a netlist remains constant (`0' or `1') for the majority of the execution time.
Hence, removing such a gate from the netlist and replacing its output with a constant value, results in low error rate.
Netlist pruning has been widely studied to enable design-agnostic approximation~\cite{GatePrun2017,Scarabottolo:DAC:2019:prune}. 
In this section we provide a brief description of how we implemented and tailored netlist pruning for bespoke printed ML architectures.

First we define two pruning parameters for a gate: $\tau$ is the maximum percentage of time that the gate's output is `0' or `1' and $\phi$ the most significant output bit (starting from 0) that the gate is connected to (through any path).
Using $\tau$ and $\phi$ \yellow{we constraint} the error frequency and the error magnitude, respectively. 
For example, assume that gate U1 is `1' the $\tau$=$90$\% of the time and that $\phi$=$3$.
Replacing U1 by `1' will result in an error rate of $10$\% and the maximum error will be less than $2^4$.
Netlist pruning is mainly implemented using heuristics~\cite{GatePrun2017} and thus optimality cannot be guaranteed.
In our work, leveraging that i) bespoke architectures feature significantly fewer area/gates than conventional architectures and ii) that ML models for printed electronics are rather limited in size, we use $\tau$ and $\phi$ to constraint the pruning design space and we implement an exhaustive search to obtain Pareto-optimal solutions.
Aiming for high area-efficiency, we prune all gates that feature $\tau$ and $\phi$ less or equal to given constraints $\tau_c$ and $\phi_c$.
Since all the pruned gates feature $\phi \leq \phi_c$, the maximum output error is less than $2^{\phi_c+1}$ irrespective of the number of pruned gates.
Overall, our coarse-grained approach ensures maximum area reduction while satisfying a maximum error threshold and enables fast design space exploration since only a gate's $\tau$ and $\phi$ need to be calculated.

Leveraging $\phi$ we filter all the gates that feature high $\tau$ and prune those that satisfy a given worst-case error.
In the case of regressors (MLP-R and SVM-R) this works well and many gates are pruned for low $\phi_c$ values.
However, classifiers require special consideration.
MLP-C and SVM-C use an argmax function at the end to translate the numerical predictions (e.g., values of output neurons in MLP-C) to a class.
As a result, the paths passing from all the gates are eventually congested in a few output bits, limiting the pruning granularity (possible $\phi_c$). 
Moreover, argmax breaks the correlation between the introduced numerical error in predictions and the final output.
For example, argmax([$0.9$, $0.1$])=argmax([$0.4$, $0.1$])=$0$.
%Hence, argmax breaks the efficiency of $\phi_c$ in controlling the gates that will be pruned.
For this reason, for the classifiers, we calculate $\phi$ for each gate with respect to the inputs of the argmax function.
For example, assume an MLP-C with $k$ output neurons $O_1$,..., $O_k$.
We define the value $\phi$ of a gate as $\max\limits_{\forall i}\phi(O_i)$, where $\phi(O_i)$ is the most significant output bit of the neuron $O_i$ that the gate is connected to.
If such a path doesn't exist, we set $\phi(O_i)=-1$.

Given a netlist (either exact or coefficient approximated), our netlist pruning operates as follows (also depicted in Fig.~\ref{fig:netprun}):
\begin{enumerate}[leftmargin=*]
\item Run RTL simulation of the synthesized netlist using the training dataset and Questasim to obtain the switching activity interchange format (SAIF) file.\label{item:sim}
\item Parse SAIF to calculate $\tau$ and the respective constant value (`0' or `1' ) for each gate.
For example, if the output of a gate is the $85$\% of the time `1' and the $15$\% it is `0', then $\tau$=$85$\% and the constant value is `1'. \label{item:tau}
\item Extract all the gates with $\tau \leq \tau_c$ and calculate their $\phi$. $\phi$ is easily calculated with the synthesis tool by reporting paths from a gate to the outputs.
\label{item:tauphi}
\item Prune all gates with $\tau \leq \tau_c$ and $\phi \leq \phi_c$, i.e., replace their output with the constant value extracted in step~\ref{item:tau}.\label{item:prune}
\item Synthesize the pruned netlist and evaluate its area and power as well as its accuracy on the test dataset.\label{item:synth}
\end{enumerate}
%If the synthesized netlist is not available, we use Design Compiler to synthesize the RTL description.
The pruned netlist is synthesized to exploit all optimizations of the synthesis tool, e.g., constant propagation.
Steps~\ref{item:sim} and~\ref{item:tau} are executed only once.
Step~\ref{item:tauphi} is executed $\forall \tau_c \in [80\%, 99\%]$.}
Then, for a given $\tau_c$, steps~\ref{item:prune}-\ref{item:synth} are executed $\forall \phi_c \in \Phi_\tau$.
$\Phi_\tau$ is equal to the set of the unique $\phi$ values obtained in step~\ref{item:tauphi}.
$\Phi_\tau$ enables us to explore only the relevant $\phi_c$ values.
%, accelerating our full search exploration.
\orange{
For example, if all the gates that feature $\tau\geq99$\% affect only the zero and first output bits, then $\phi_c>1$ is meaningless since it will return the same solution as $\phi_c=1$.
%At the worst case, using $10$ threads (i.e., limit of available licenses) our exhaustive design space exploration required only $28$ min.
Unlike the pruning state-of-the-art that examines only very simple circuits~\cite{GatePrun2017,Scarabottolo:DAC:2019:prune}, our implementation is evaluated on complex ML circuits.
Moreover, as explained above for the classifiers, conventional pruning~\cite{GatePrun2017,Scarabottolo:DAC:2019:prune} cannot be used.}



\begin{figure}[t!]
\centering
\includegraphics[]{graphs/cpd_se_mlp_c.pdf}
\includegraphics[]{graphs/cpd_se_mlp_r.pdf}
\includegraphics[]{graphs/cpd_se_svm_c.pdf}
\includegraphics[]{graphs/cpd_se_svm_r.pdf}
\caption{Normalized critical path delay delivered by coefficient \& pruning approximation against exact bespoke design when $80\%\leq\tau\leq100\%$ for the same $\phi$ and for Seeds dataset. MLP-C (a), MLP-R (b), SVM-C (c) and SVM-R (d) are considered.
}
\label{fig:cpd}
\end{figure}


\subsection{Voltage Over-Scaling}\label{sec:vos}

%Exploiting the potential for approximations in all three distinct layers (i.e., algorithmic, logic and circuit level), we also apply VOS to construct our final circuits.
%VOS can deliver even lower power consumption, since power dissipation of a circuit and supply voltage value depend quadratically, as shown in the relationship given by:
Finally, at the circuit level we apply VOS to maximize the power gains of our framework since power consumption of a circuit depends quadratically on the voltage supply value:
\begin{equation}
    P_{total} = P_{static} + a \times C \times f_s \times V_{dd}^2,
\label{eq:vdd}
\end{equation} 
where $a$ is the switching activity, $C$ is the circuit's capacitance, $f_s$ is the operating clock frequency and $V_{dd}$ the voltage value.
It is evident, thus, that power consumption significantly decreases when voltage supply ($V_{dd}$) is decreased.
%From Eq.~\ref{eq:vdd} it can be derived that the total power consumption decreases with the supply voltage $V{dd}$.

The main drawback of VOS is that circuits become slower and paths delay are increased. 
Therefore, when supply voltage is scaled below its nominal value~\cite{vader:zerv} for a given frequency, timing violations occur.
As the voltage continues to drop, an increasing number of internal paths cannot complete within the clock timing and the timing violation rate increases~\cite{vader:zerv}.
% However, in error-tolerant ML applications, such violations are not equalized with accuracy and consequently quality degradation.
% This mainly happens due to the fact that i) even if some gates suffer from a violation, the whole path can still produce a correct result, since some errors can be logically masked~\cite{masked:vos}, ii) not all neuron computations have the same latency, but VOS is performed based on the worst case, in which the corresponding neuron could be rarely activated and as previously mentioned, iii) the argmax function of classifiers breaks the correlation between the computation inaccuracy of predictions and the final outputs.

Applying coefficient approximation and netlist pruning not only reduces the circuit's area (and power) but also results in delay gain that can be therefore leveraged to apply aggressive VOS.   
%On the other hand, approximate techniques from coefficient replacement and netlist pruning produce simpler circuits with reduced delay and critical path, paving the way for voltage decreases, with fewer violated paths than in exact circuits, as well.
An example of the obtained delay gain is illustrated in Fig.~\ref{fig:cpd}.
%The trend of such delay reduction in four different models of one of our examined datasets is depicted in Fig.~\ref{fig:cpd}.
To generate Fig.~\ref{fig:cpd} we calculated for each model the delay when applying more aggressive approximation ($\tau<100\%$), normalized against the delay of their exact designs ($\tau=100\%$).
Note that $\phi$ was arbitrary selected equal to zero, since it does not affect circuit's logic, but only the error magnitude.
It is observed that, on average, $40\%$ lower delay can be achieved when $\tau=80\%$, while almost $9\%$ when $\tau$ is as high as $95\%$.
% , keeping the same $\phi$ value
Hence, assuming iso-frequency operation for the approximate and baseline circuits, the approximate circuits will feature fewer timing violations (if any) and the impact of VOS on the delivered accuracy is expected to be significantly diminished.
%Hence, with appropriate circuit design methodology, efficient voltage over-scaled approximate circuits that satisfy quality requirements, while also boosting VOS to its limit, can be generated. 
% Hence, decreasing voltage below its nominal value~\cite{zerv:axmult} can generate efficient approximate circuits  paves the way for printed circuits with higher performance and very low power consumption, since printed technology with appropriate circuit design methodology can 
% provide ultra-low voltage operation down to even $0.6V$~\cite{PrintVoltage:Tahoori}.
As a result, considering that EGT printed circuits can be operated even down to $0.6V$~\cite{PrintVoltage:Tahoori},
% at the cost of a small accuracy degradation, 
combining VOS with coefficient approximation and netlist pruning, enables ultra-low power operation without any performance loss due to the voltage decrease.
%In this section, we present our VOS methodology and how ``precise'' VOS simulations at gate level are performed in order to eventually identify the appropriate $V_{dd}$ for each approximate circuit.

In order to explore, in a timely manner\footnote{Significantly faster than SPICE simulations that would be infeasible to run as part of our optimization solution.}, the impact of VOS and evaluate the accuracy and power of voltage over-scaled approximate circuits, we perform VOS-aware gate-level post-synthesis timing simulations following the methodology presented in~\cite{vosim:zervakis}.
% Questasim simulates the gate netlist produced after synthesis at the previously relaxed clock, using as timing information an SDF file.
% This SDF file is produced for every examined voltage value using Synopsys command \emph{define\_scaling\_lib\_group}, which defines and groups some given libraries at different voltage values and interpolate between them.
%Moreover, printed technology can provide circuits with ultra-low voltage operation down to $0.6V$~\cite{PrintVoltage:Tahoori}.
\yellow{Realistic voltage values of $1.0V$ down to $0.6V$~\cite{PrintVoltage:Tahoori} with a $20mV$ step are considered for the VOS application.
Each circuit is set to one optimized voltage, which is found using the presented offline analysis, and so no run-time voltage regulators or controllers are needed. 
Note also that printed batteries can be fully customized in terms of voltage, shape, polarity, etc.,~\cite{PrintedBatteries2018}.
}\label{commentR1C3b}
Finally, since VOS-errors are timing errors and thus input dependent~\cite{vader:zerv,vosim:zervakis}, large input datasets are required to efficiently capture the accuracy behavior of circuits under VOS.
For this reason, we create large input stimuli by replicating and shuffling each test dataset as many times as required in order to obtain 1M randomly sequenced inputs.
\yellow{In our evaluations we use the aforementioned simulation-based tool-flow built upon industrial-strength EDA tools (i.e., Synopsys Design Compiler, Prime Time, Mentor Questasim), using printed PDK and standard cell libraries~\cite{Bleier:ISCA:2020:printedmicro} calibrated based on fabrication and measurements from low voltage printing technologies.}\label{commentR1C2}
% Finally, since computation errors derived from timing violations are also input dependent and require many and different input sequences, we perform
% VOS-aware simulations by shuffling and merging each dataset as many times as needed to obtain 1M input sequence samples.
% The aforementioned tool-flow is used in our whole VOS methodology.

% Since  timing  errors  are  essentially  silent  errors,  they  do not manifest themselves until the corrupted data has led to an unusual application behavior, which may be detected long after the error has occurred, wasting the entire computation done so far

% circuits  can  be pushed beyond their limits to compute incorrectly to achieve higher energy efficiency.



\subsection{Battery-driven Cross-Approximation}\label{sec:dse}

One of the main objectives of our work as well a key challenge in printed electronics is to enable battery-powered operation.
Hence, considering also that the space of available printed batteries is discrete and well constrained, without any loss of optimality, we assume a power constrained optimization for our cross-approximation.
In other words, the approximate printed ML classifier generated by our framework should be able to be powered by a specified printed battery. 
Given the area-accuracy efficiency of our coefficient approximation~\cite{DATE22:Armen}, we assume that all the approximate solutions generated by our framework employ coefficient approximation at the algorithmic level.
Then, at the logic and circuit levels we need to identify the respective approximation configuration, i.e., appropriate values for $\tau$ and $\phi$ (netlist pruning configuration) as well as $V_{dd}$ (VOS configuration), so that
i) the area is minimized,
ii) the accuracy is maximized, and 
iii) a power constraint is satisfied.
Naming $\axname$ the ML classifier after the coefficient approximation, our optimization problem can be formulated as follows:
\begin{equation}
\begin{gathered}
    \text{find} \, (\tau, \phi, V_{dd}) \, \text{s.t.} \\
    \mathrm{Power}\big(\axname(\tau, \phi, V_{dd})\big)\Big) \leq P_{BAT} \quad \text{and}  \\
    \min \Big(\mathrm{Area}\big(\axname(\tau, \phi, V_{dd})\big)\Big), \\
    \max \Big(\mathrm{Accuracy}\big(\axname(\tau, \phi, V_{dd})\big)\Big)
\end{gathered}
\label{eq:opt}
\end{equation} 
where is $P_{BAT}$ is the battery-specific power constraint. 
The size of the design space is from a few to several thousands of approximate circuits and the fact that each accuracy evaluation requires time consuming VOS-aware simulations exaggerates the complexity of our optimization problem. 
Note, however, that both the point-of-use fabrication process of printed circuits and the per model customization of bespoke architectures (i.e., our framework needs to run for every new ML circuit and/or targeted battery), mandate fast operation of our framework, in contrast to the long-lasting optimization cycles of silicon-based systems

To address this optimization problem, we first systematically reduce the design space and then employ a genetic algorithm to traverse the reduced design space.

\textbf{\textit{Pruning the design space:}}
Although assessing the accuracy of VOS-based approximate solutions is very slow, evaluating approximate circuits that apply only coefficient approximation and netlist pruning (i.e., $V_{dd}=1V$) is very fast (it requires a significantly smaller input stimuli and accuracy can be obtained through RTL simulations).\label{commentR1C1}
As a result, we implement a full search exploration evaluating the accuracy and power of all the approximate circuits $\axname(\tau, \phi, $1V$)$, $\forall \tau, \phi$.
% The full search exploration required at most \red{XX} for the most complex circuit examined.
First, we prune the design space by removing all the design points that feature accuracy loss higher than a certain percentage (e.g., $20\%$) as they are considered of poor quality.
Then, using~\eqref{eq:vdd} and the obtained $\mathrm{Power}(\axname(\tau, \phi,1V))$, we analytically estimate 
the $\mathrm{Power}(\axname(\tau, \phi, V_{dd}))$, $\forall \tau, \phi, V_{dd}$.
All the solutions that fail to satisfy the power constraint $ P_{BAT}$ are discarded and removed from the design space.

\textbf{\textit{Genetic Optimization:}} 
\yellow{To explore the pruned design space and find, quite fast, a close-to-optimal solution to our optimization problem, we employ a heuristic method~\cite{shafique} (Algorithm~\ref{alg:genetic}) based on NSGA-II~\cite{nsga2}.
Note that although genetic algorithms are proven to extract good enough solutions in complex optimization problems~\cite{shafique} any other heuristic algorithm can be seamlessly employed.}\label{commentR1C1a}
The approximation parameters are the \emph{genes} and, thus, the triplet $(\tau, \phi, V_{dd})$ represents the \emph{chromosome}.
The searching process starts by generating an initial population $\lambda$.
Instead of a totally random population, we guide the exploration by selecting configurations that satisfy specific accuracy and power constraints, based on some fast estimations.
This is done by firstly pre-calculating the candidate circuit's accuracy with the much faster RTL simulations (instead of the slow VOS-aware gate-level simulations) and then, by checking whether its power can satisfy the power constraint for any possible voltage value in the interval $[0.6V,\, 1.0V]$.
These estimations can quickly performed by using the theoretical power model of Eq.~\ref{eq:vdd}.
Therefore, ``non-acceptable'' configurations are removed and discarded from the design space.
% Thus, the chromosomes can represent only solutions of the pruned space.
After generating the initial population the following steps are repeated until the termination condition is satisfied: i) $\lambda$ offsprings are generated from initial population by means of usual mutation and crossover having similarly acceptable configurations, ii) $\lambda$ most fitting circuits (individuals) are selected, so that population remains at same initial size, iii) the viability of population known as \emph{fitness value} is evaluated.
Note that to find the best compromise between objectives that we are interested in (i.e., area, power and accuracy), fitness values are calculated using Kursawe function.
Finally, the process is terminated after $n$ epochs (generations of offsprings), where $n$ is an user defined constant tightly related to the model's characteristics and the constraints set by the user.



%%%%%%%%%%%%%%%%%%%%%%

% The key question that our algorithm targets to answer is how can we simultaneously optimize both the output quality (accuracy) and hardware gains (area \& power) of proposed approximate circuits, instead of just one of these factors (like in~\cite{DATE22:Armen}).
% Consequently, the main goal of our framework is to efficiently explore the multi-objective design space in order to find the best (optimal) approximate configurations (i.e., $\tau$, $\phi$, $V_{dd}$) representing the best compromise between objectives that we are interested in (i.e., area, power and accuracy).
% Designs with such configurations are referred to as \emph{pareto-optimal} designs throughout this article.
% Furthermore, due to the in-situ and ad-hoc fabrication of varying printed bespoke circuits, which are significantly different to each other, optimization needs often to run for different circuits.
% This is not the case for conventional silicon systems that optimizations are performed only once.
% Thus, in order to meet time-to-market constraints of printed electronic applications and avoid the exhaustive search over the whole design space, our heuristic algorithm, based on user-defined constraints, prunes the search space and finds approximate configurations for near pareto-optimal designs in far less execution time than required by full DSE.
% This section presents how our framework tackles the area-power-accuracy optimization problem of our three distinct-layer approximations and also how this process is sped up, using some quality constraints.

%% To quantify the gains offered by our cross-layer approximations compared to bespoke baseline circuits, we need to find the best optimal or near-optimal quality-efficiency designs for each dataset.
%% In this section, we present our design space exploration methodology that leads us to such configurations for an approximate printed circuit.
%% Our design space exploration methodology is designed so that allows us to find efficient designs in a very small amount of time (i.e., order of minutes).
%% However, when including VOS approximation, as well, this time increases due to time-consuming gate-level simulations required.
%% Hence, we also propose a heuristic algorithm that prunes the search space and finds configurations for near-pareto optimal designs in a very smaller and reasonable amount of time.


% \subsubsection{Full Design Space - A challenge}

% The total number of configurations directly depends on the netlist pruning parameters $\tau$ and $\phi$ and on the number of considered voltage values $V_{dd}$ (i.e., from $1.0V$ down to $0.6V$ with a step size at $20mV$). 
% Assuming also $v$ the number of different voltage values (i.e., 21 in our case) and $ax$ the number of different approximation types (i.e., Coefficient Approx. \& Pruning and Only Pruning), the number of possible configurations is given by: ${Number\ of\ configs.} = ax \times \phi \times \tau \times v$
% %% \begin{equation}
% %%     {Number\ of\ configs.} = ax \times \phi \times \tau \times v
% %% \label{eq:config}
% %% \end{equation}
% As can be inferred, the maximum number of configurations is $11466$ ($2\times13\times21\times21$).
% Although the design space is not enormous, gate-level simulations required for VOS evaluation consumes significant amount of time.
% %% Thus, to tackle our design space and minimize the total execution time of our framework, we propose a heuristic genetic algorithm that prunes effectively the search space and finds optimal/near-optimal approximate circuits in far less execution time without evaluating every single circuit like in full design space exploration.
% Hence, we adopt a genetic algorithm as our heuristic method that prunes the search space and finds near-optimal designs effectively. 


% \subsection{Proposed methodology for multi-objective optimization}

% \textbf{Proposed Genetic Algorithm:}
% Our proposed exploration methodology employs a multi-objective genetic algorithm based on NSGA-II~\cite{nsga2}, allowing to optimize the accuracy and other hardware metrics (i.e., area and power consumption) together in one run.
% The goal is to find the best approximate configurations and the appropriate voltage value for each model.
% After execution, Pareto front approximate designs that contains the so-called \emph{non-dominated solutions} are obtained.
% To speed up the process and since printed applications usually target specific printed batteries with specific maximum output power, our algorithm contains power and accuracy loss constraints that can be set by the user.
% Due to these constraints, a percentage of the total approximate circuits is discarded and the total design space is pruned, resulting consequently in faster exploration.


% Every candidate approximate circuit represents one design point in the design space.
% Each of the four key approximate parameters (\#$ax_{type}$, \#$\phi$, \#$\tau$, \#$V_{dd}$) that can be varied in the printed circuit-space is encodes as a \emph{Gene}.
% These genes are joined together as a string to generate a \emph{Chromosome}, which can be decoded to construct the approximate printed circuit.
% The searching process starts by generating an initial population $\lambda$ (Algorithm 1).
% Instead of selecting randomly candidates, we guide the exploration by selecting configurations that satisfy specific accuracy and power constraints, based on some fast estimations.
% This is done by firstly pre-calculating the candidate circuit's accuracy with the much faster RTL simulation (instead of the slow VOS-aware gate-level simulation) and then, by checking whether its power can satisfy the power constraint for any possible voltage value in the interval $[0.6V 1.0V]$.
% This can be quickly estimated by using the theoretical power model of Eq.~\ref{eq:vdd}.
% Therefore, throughout the algorithm, ``acceptable'' configurations are selected, while solutions with ``non-acceptable'' voltage values are discarded.
% This technique saves us significant amount of time, since far less iterations are needed to obtain a near-optimal solution.
%% This technique saves us significant amount of time, since accuracy is pre-calculated faster with RTL simulation (instead of the slow gate-level simulation required for VOS) and ``accepted'' voltage values that satisfy the power threshold can be obtained by the power model of Eq.~\ref{eq:vdd}.
%% Thus, it can be quickly decided whether specific configurations are acceptable for our initial population or new candidates need to be selected.
%After selecting the initial population the following steps are repeated until the termination condition is satisfied: i) $\lambda$ offsprings are generated from initial population by means of mutation and crossover having similarly acceptable configurations, ii) $\lambda$ most fitting circuits (individuals) are selected, so that population remains at same initial size, iii) population is evaluated.
%Finally, the process is terminated after $n$ epochs (generations of offsprings), where $n$ is an application-dependent parameter tightly related to the type of model and constraints set by the user.
%Algorithm 1 summarizes our proposed genetic algorithm presented in this section.

%% for the termination condition we calculate the optimality of selected circuit, i.e., how close its optimized metrics are, compared with those of full DSE.
%% For this reason, we define the optimality of each circuit and for a given accuracy constraint as:

%% \begin{multline}
%%     Opt = 100\% - \left [ \left (\frac{AG_o - AG_s}{AG_o} \right. \right )50\% \\
%%     + \left. \left (\frac{PG_o - PG_s}{PG_o} \right )50\% \right ]
%% \label{eq:opt}
%% \end{multline}

%% where $AG_o$ and $PG_o$ is the optimal area achieved by our coefficient approximation \& pruning (CoA\&Pr) and the optimal power achieved from the full DSE, respectively.
%% Accordingly, $AG_s$ and $PG_s$ is the area and power gains of the selected circuit achieved from our proposed methodology with the pruned search space.


\begin{algorithm}[t!]
\caption{Pseudocode for proposed multi-objective optimization}\label{alg:genetic}
\footnotesize
\textbf{Input:} 1) Trained Model $m$, 2) Accuracy Loss Threshold $T$, \\3) Battery Power Threshold $P$, 4) Population\_Size $\lambda$, \\5) Termination Condition\\
\textbf{Output:} 1) Approx\_Configs, 2) Operating Voltage $v$\\
\begin{algorithmic}[1]
\State population $=$ POP\_INIT($m, \lambda$)
\State \textbf{while} Termination Condition \textbf{false}
\State \hspace{3mm} \textbf{do} 
\State \hspace{3mm} generate $\lambda$ offsprings
\State \hspace{3mm} population =+ offsprings
\State \hspace{3mm} \textbf{evaluate(}$\lambda$ population\textbf{)}
\State \textbf{return} Approx\_Configs , $v$
\vspace{3mm}
\State \textbf{function} POP\_INIT(model, pop\_size)
\State \hspace{3mm}pop\_n = 0
\State \hspace{3mm}\textbf{while} pop\_n $<$ pop\_size \textbf{do}
\State \hspace{6mm}Approx\_Configs = random\_init($\phi$, $\tau$, $v$)
\State \hspace{6mm}Approx Model $m^\prime$ = model(Approx\_Configs)
% \State $Acc_{m^\prime}, Pw_{m^\prime}$ = 0
\State \hspace{6mm}$Acc_{m^\prime} \leftarrow $ RTL\_simulation
\State \hspace{6mm}\textbf{if} ($Acc_{m^\prime}$ $\geq$ $Acc_{m}$ $- T$)
\State \hspace{6mm}\hspace{3mm} $Pw_{m^\prime} \leftarrow$ power\_sim($1.00V$)
\State \hspace{6mm}\hspace{3mm} Estimate min $V_{dd}$ for given $P$ using \textbf{Eq.~\ref{eq:vdd}}
\State \hspace{6mm}\hspace{3mm} \textbf{if} $V_{dd} \geq 0.60V$ \textbf{then}
\State \hspace{6mm}\hspace{6mm} population =+ m(Approx\_Configs)
\State \hspace{6mm}\hspace{6mm} pop\_n =+ 1
\State \hspace{12mm} \textbf{end if}
\State \hspace{6mm}\textbf{end if}
\State \hspace{3mm}\textbf{endwhile}
\State \hspace{3mm}\textbf{return} population
\State \textbf{end function}
\end{algorithmic}
\end{algorithm}
\section{Results}
\label{results}

\begin{figure*}[ht]
    \centering
    \includegraphics[scale=0.15,trim={0 2.5cm 0 5cm},clip]{images/aoi-single_burst}
    \caption{The time average peak Age of Information with burst and \gls{soa} loss values against the dynamic reliability logic for different network topologies.}
    \label{fig:aoi_burst}\vspace{-0.4cm}
\end{figure*}


This paper focuses on both transport layer and application layer metrics to determine the feasibility of dynamic reliability. For this, we have selected the session packet volume, as transmitted, retransmitted, lost and backlogged packets as \glspl{kpi} for the transport layer; while focusing on the \gls{aoi} for the application layer. The \gls{aoi} was chosen as a crucial indicator for the freshness of packets in real-time applications. More specifically, this work adopts the time average peak \gls{aoi} equation \cite{aoi_equation} depicted in Eq. \ref{aoi}, where $\Delta(r_{i+1})$ is the $i$th update at the time it was received at the server, for a session time period of $\tau$.

\begin{equation}
    \label{aoi}
    \gls{aoi}_\tau = \frac{1}{n-1}\sum_{i=1}^{n-1} \Delta(r_{i+1})
\end{equation}

We include a comparison between the vanilla QUIC implementation which does not enjoy the dynamic reliability extension, with a number of dynamic reliability policies. The tests were run a number of times for statistical significance, with the mean value of vanilla implementation used as a baseline for comparison. The topology utilised both random loss and bursty loss to explore the bounds of dynamic reliability. The \gls{soa} loss in the figures correspond to the loss values presented in Table. \ref{tab:path_char}, for ease of comparison between bursty and random loss scenarios.

\subsection{Transport-Layer KPIs}

To analyse the performance gain at the transport layer due to dynamic reliability, the volume of transmitted and backlogged packets is examined. The figures are in the form of boxplots, which take the vanilla implementation as a benchmark, depicted as the red dashed line.

As seen in Fig. \ref{fig:sent_burst}, the loss plays a crucial role in the performance of the reliability policies. The policies under random loss did incredibly well for the networks with a larger capacity, namely \gls{mmwave} and Sub-6~GHz, whereas for burst loss, the lower network capacities had a larger packet reduction. With the increase in burst loss, the behaviour of the set split reliable policies became unpredictable, if a reliable assignment happened to coincide with a burst loss, the number of transmitted packets increases, and vice versa. On the other hand, in smarter policies, such as Loss-Aware, the performance lightly matched the vanilla baseline, as the reliable assignment dominated the session to compensate for a higher burst loss. Not only that but, the burst loss also impacted the variance of the transmitted packets for the policies.

Unsurprisingly, the unreliable focused policy, 80-20 split, outperformed other policies for all topologies in random and bursty loss scenarios, with an approximate reduction of 80\%. That being said, the majority of the policies reduced the transmitted packets on the link by approximately 70\% for random loss, while the reduction started at $\approx 15\%$ and decreased as the loss increased for the burst loss scenario.

The retransmitted and lost packets, not shown due to space limitations, followed the same trend as the transmitted packets for the random loss scenarios. However, for the burst loss scenarios, the larger capacity networks had a lower reduction in the retransmitted and lost packets. This can be seen as a favorable outcome since the lower capacity networks are scarce on resources. It is important to note that the Loss-Aware policy mimicked the vanilla approach as the burst loss increased, signifying the overwhelming appointment of reliable packets in adapting to the harsh burst loss conditions.
 
Alternatively, Fig. \ref{fig:backlog_burst} clearly shows a stark comparison between the policies and loss scenario in the reduction of the backlogged packets. The Loss-Aware policy for random loss scenario reduced the backlogged packets by up to 50\%, beating all other policies by approximately 30\%. Furthermore, it is clear that the unreliability focused policies resulted in the lowest backlog for the session. In comparison, we notice that the burst loss and the backlogged frequency have a positive correlation, where the maximum reduction of the backlogged packets for the policies is at most 20\%. Much like the transmitted packets, the probability of a burst loss occurrence plays a vital role in the number of retransmissions sent and by extension the number of backlogged packets. Thus, we can conclude that the stress placed on the buffer is a result of the reliable packets which is tightly coupled with the congestion on the session. Whereas, unreliable focused policies did not encounter such a phenomenon regardless if it was experiencing a burst loss.


\subsection{Application-Layer KPIs}

The feasibility of dynamic reliability for real-time applications can be determined by the \gls{aoi}, with comparison across different topologies and policies. If we take a strict approach and consider anything below $10$~ms is real-time \cite{real-time}, then all the reliability policies passed that requirement, which is attractive for real-time applications, as shown in Fig. \ref{fig:aoi_burst}. Utilising the median as an estimate of the runs, the policies in the WLAN and Sub-6~GHz topology with random loss floated around $4-5$~ms with negligible difference, while the \gls{aoi} for \gls{mmwave} was $\approx 2-3$~ms. It is clear that the \gls{aoi} and the network capacity have a negative correlation, as the network capacity decreases, the \gls{aoi} increases. The same correlation is extended to the bursty loss scenarios, where \gls{mmwave} dominated the other topologies. That being said, it is crucial to note that the \gls{aoi} for the reliability policies is often slightly better than or equal to the \gls{aoi} of the vanilla implementation, proving that dynamic reliability reduces the congestion of the session at no cost to the \gls{aoi}.

\section{Related work}
\noindent \textbf{Video foundation models.}
With sufficient computational power and an abundant source of data, there have been attempts to build a single large-scale foundation model that can be adapted to diverse downstream tasks.
Along with the success of foundations models in the natural language processing domain~\cite{brown2020language,chen2021evaluating,devlin2019bert} and in computer vision~\cite{bertasius2021space,jia2021scaling,radford2021learning}, video data has become another data type of interest, as it has grown in scale due to numerous internet video-sharing platforms.
Accordingly, several methods to train a video foundation model have been proposed.
Due to the innate multi-modality of video data, \textit{i.e.}, a combination of visual $\cdot$ vocal $\cdot$ textual context, most works have centered around the variations of the cross-modal attention mechanism \cite{akbari2021vatt,bertasius2021space,gabeur2020multi,luo2020univl,neimark2021video,tan2021look,wei2020multi,yang2021taco}.
In addition, as most video data lack proper labels or descriptions, contrastive learning methods were studied to learn meaningful feature representations or enhance video-text alignment in a self-supervised manner \cite{akbari2021vatt,kuang2021video,luo2020univl,yang2021taco}.

More specifically, MERLOT \cite{zellers2021merlot} proposed a multi-modal representation learning method for visual commonsense reasoning, which also performed well in twelve video reasoning tasks.
VATT \cite{akbari2021vatt} introduced a multi-modal learning method via contrastive learning. 
The pre-trained model performed well in a variety of vision tasks from image classification to video action recognition and zero-shot video retrieval.
Another representative work, UniVL \cite{luo2020univl} proposed a straightforward pre-training method with auxiliary loss functions. 
After fine-tuning on a specific task, the pre-trained model performed outstandingly in a wide range of tasks of text-to-video retrieval, action segmentation, action step localization, video sentiment analysis, and video captioning.
Other foundation models for multiple video tasks include \cite{li2020hero,sun2019learning,sun2019videobert,zhu2020actbert,fu2021violet,wang2022all}. 

\noindent \textbf{Auxiliary learning.}
In order to enhance the performance of one or a multitude of primary tasks, auxiliary learning methods can be incorporated.
\cite{ruder2017overview} introduced Multi-task learning (MTL) to the deep neural networks by training a single model with multiple task losses to assist learning on the main task.
Such a method is generally adapted to pre-train the foundation models in the self-supervised manner~\cite{li2020hero,sun2019learning,sun2019videobert,zhu2020actbert,fu2021violet,wang2022all}.
However, these various pretext task losses used in the pre-training phase are ignored in the fine-tuning phase, and only the primary task loss is minimized.

Recently, meta-learning methods have been introduced for auxiliary learning.
\cite{liu2019self,navon2020auxiliary,shu2019meta} proposed a meta-learning method in which the model learns auxiliary tasks to generalize well to unseen data. 
In these settings, a separate subset of data is held out as the primary task, while the others are used as auxiliary tasks that aid the primary task's performance.
Similar methods were adopted for computer vision tasks such as semantic segmentation \cite{xu2021leveraging}.
Other domain applications include navigation tasks with reinforcement learning \cite{ye2021auxiliary}, or self-supervised learning methods on graph data \cite{hwang2020self}.
\section{Conclusion}\label{sec:conclusion}
In this work, we focus on addressing the fundamental challenge of OOD detection tasks, which is how to fully understand the semantic discrepancy between the ID/OOD samples. We reveal that the key to success in the realistic SCOOD task is to allocate as many ID samples in the unlabeled set correctly as possible. To this end, we propose a novel uncertainty-aware optimal transport scheme that introduces class-specific energy scores as guidance for effective label assignment. Experimental results show that our method achieves better performance than previous state-of-the-art methods on SCOOD benchmarks.

\textbf{Limitations.} In addition to temperature scaling, other techniques such as feature clipping applied in ReAct~\cite{sun2021react} also enhance the performance of energy score, so how to obtain an OOD score that best fits the SCOOD task can be further explored. Moreover, a setting highly related to SCOOD has been proposed in \cite{katz2022training} and formulated as a constrained optimization problem. We will also theoretically analyze these practical OOD settings in our feature work.

% \section*{Acknowledgments}
\textbf{Acknowledgments.} 
This work is supported by National Key R\&D Program of China under Grant 2020AAA0105701, National Natural Science Foundation of China (NSFC) under Grants 61872327, Major Special Science and Technology Project of Anhui, National Natural Science Foundation of China (62033012) and Ant Group through Ant Research Intern Program.




\bibliographystyle{IEEEtran}
\bibliography{references}

\begin{IEEEbiography}[{\includegraphics[width=1in,height=1.25in,clip,keepaspectratio]{figures/bios/foto_maria}}]{Mar\'ia Leyva-Vallina} received her BSc in Software Engineering from the University of Oviedo in 2015 and her MSc in Artificial Intelligence from the Polytechnic University of Catalonia in 2017. She is currently pursuing her PhD with the Intelligent Systems group in the University of Groningen. Her main research interests are in representation learning for computer vision.\end{IEEEbiography}

% if you will not have a photo at all:
\begin{IEEEbiography}[{\includegraphics[width=1in,height=1.25in,clip,keepaspectratio]{figures/bios/nicola}}]{Nicola Strisciuglio}
received the Ph.D. degree (cum laude) in Computer Science from the University of Groningen (Netherlands) and the Ph.D. degree in Information
Engineering from the University of Salerno (Italy).  
He is currently an Assistant Professor at the Faculty of Electrical Engineering, Mathematics and Computer Science, University of Twente (Netherlands). He has been the
General Co-Chair of the 1st, 2nd and 3rd International Conference on Applications of Intelligent Systems (APPIS). His research interests include machine learning, signal processing and computer vision.
\end{IEEEbiography}

% insert where needed to balance the two columns on the last page with
% biographies
%\newpage

\begin{IEEEbiography}[{\includegraphics[width=1in,height=1.25in,clip,keepaspectratio]{figures/bios/petkov}}]{Nicolai Petkov} received the Dr. sc. techn. degree in computer engineering (informationstechnik) from the Dresden University of Technology, Dresden, Germany. Since 1991, he has been a Professor of computer science and the Head of the Intelligent Systems Group, University of Groningen. He has authored two monographs, and has authored or co-authored over 150 scientific papers. He holds four patents. His current research interests include pattern recognition, machine learning, data analytics, and brain-inspired computing, with applications in various areas. He is a member of the editorial boards of several journals.

\end{IEEEbiography}


% \begin{thebibliography}{1}
% \bibitem{IEEEhowto:kopka}
% H.~Kopka and P.~W. Daly, \emph{A Guide to \LaTeX}, 3rd~ed.\hskip 1em plus
%   0.5em minus 0.4em\relax Harlow, England: Addison-Wesley, 1999.

% \end{thebibliography}



% that's all folks
\end{document}


