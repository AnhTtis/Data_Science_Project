\begin{IEEEbiography}[{\includegraphics[width=1in,height=1.25in,clip,keepaspectratio]{./biosphotos/arme.jpg}}] {Giorgos Armeniakos} received the Diploma degree from the Department of Electrical and Computer Engineering (ECE), National Technical University of Athens (NTUA), Greece, in 2020, where he is currently pursuing the Ph.D. degree. He holds one best paper nomination at DATE'22 for his work on approximate printed electronics. His research interests include approximate computing, digital circuit design, low power design, machine learning, and optimization.
\end{IEEEbiography}

\begin{IEEEbiography}[{\includegraphics[width=1in,height=1.25in,clip,keepaspectratio]{./biosphotos/georgios_zervakis.jpg}}] {Georgios Zervakis}  is an Assistant Professor at the University of Patras. Before that he was a Research Group Leader at the Chair for Embedded Systems (CES), at the Karlsruhe Institute of Technology (KIT) from 2019 to 2022. He received the Diploma and Ph.D. degrees from the School of Electrical and Computer Engineering (ECE), National Technical University of Athens (NTUA), Greece, in 2012 and 2018, respectively. From 2015 to 2019, Dr. Zervakis worked as a principal investigator in many EU-funded research projects as a member of the Institute of Communication and Computer Systems (ICCS), Athens, Greece. Dr. Zervakis serves as a reviewer in many IEEE and ACM Transactions journals and is also a member of the technical program committee of several major design conferences. He has received one best paper nomination at DATE 2022. His main research interests include low-power design, accelerator microarchitectures, approximate computing, and machine learning.
\end{IEEEbiography}

\begin{IEEEbiography}[{\includegraphics[width=1in,height=1.25in,clip,keepaspectratio]{./biosphotos/soudris.png}}] {Dimitrios Soudris} (Member, IEEE) received
the Diploma and Ph.D. degrees in electrical engineering from the University of Patras, Patras,
Greece, in 1987 and 1992, respectively. Since
1995, he has been a Professor with the Department of Electrical and Computer Engineering,
Democritus University of Thrace, Xanthi, Greece.
He is currently a Professor with the School of
Electrical and Computer Engineering, National
Technical University of Athens, Athens, Greece.
He has authored or coauthored more than 500 papers in international journals/conferences. He has coauthored/coedited seven Kluwer/Springer books.
He is also the leader and a principal investigator in research projects funded
by Greek Government and Industry, European Commission, ENIAC-JU,
and European Space Agency. His current research interests include high performance computing, embedded systems, reconfigurable architectures,
reliability, and low-power VLSI design. He was a recipient of the award
from INTEL and IBM for EU Project LPGD 25256; ASP-DAC 05 and VLSI
05 awards for EU AMDREL IST-2001-34379, and several HiPEAC awards.
He has served as the general/program chair in several conferences.
\end{IEEEbiography}

\begin{IEEEbiography}[{\includegraphics[width=1in,height=1.25in,clip,keepaspectratio]{./biosphotos/tahoori.jpg}}] {Mehdi B. Tahoori} (M'03, SM'08, F'21) is Professor and the Chair of Dependable Nano-Computing at Karlsruhe Institute of Technology, Germany. He received the B.S. degree in computer engineering from Sharif University of Technology, Iran, in 2000, and the M.S. and Ph.D. degrees in electrical engineering from Stanford University, Stanford, CA, in 2002 and 2003, respectively. He is currently the deputy editor-in-chief of IEEE Design and Test Magazine. He was the editor-in-chief of Microelectronic Reliability journal. He was the program chair of VLSI Test Symposium in (VTS) 2021 and 2018, and General Chair of European Test Symposium (ETS) in 2019. He is the chair of the IEEE European Test Technology Technical Council (eTTTC). Prof. Tahoori was a recipient of the US National Science Foundation Early Faculty Development (CAREER) Award in 2008. He has received a number of best paper nominations and awards at various conferences and journals. He is a recipient of European Research Council (ERC) Advanced Grant.
\end{IEEEbiography}

\begin{IEEEbiography}[{\includegraphics[width=1in,height=1.25in,clip,keepaspectratio]{./biosphotos/joerg_henkel.jpg}}]{J\"org Henkel} (M'95-SM'01-F'15) is the Chair
Professor for Embedded Systems at Karlsruhe
Institute of Technology. Before that he was a research staff member at NEC Laboratories in
Princeton, NJ.
He has received six best paper awards from, among others, ICCAD, ESWeek and DATE.
For two terms he served as the Editor-in-Chief for the ACM Transactions on Embedded Computing Systems.
He is currently the Editor-in-Chief of the IEEE Design\&Test Magazine and is/has been Associate Editor for major ACM and IEEE Journals.
He has led several conferences as a General Chair incl. ICCAD, ESWeek and serves as Steering Committee chair/member for leading conferences and journals for embedded and cyber-physical systems. Prof. Henkel coordinates the DFG program SPP 1500 ``Dependable Embedded Systems'' and is a site coordinator of the DFG TR89 collaborative research center ``Invasive Computing''. He is the chairman of the IEEE Computer Society, Germany Chapter, and a Fellow of the IEEE.
\end{IEEEbiography}