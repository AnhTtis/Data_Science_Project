\begin{table*}[t]
\setlength\tabcolsep{3pt}
\caption{Evaluation of Bespoke Printed ML Circuits in EGT PDK library.}
\label{tab:baselines}
\footnotesize
\centering
\renewcommand{\arraystretch}{1.2}
\begin{threeparttable}
\begin{tabular}{l|cccccc|cccccc|cccccc|cccccc}
\hline
 & \multicolumn{6}{c|}{\textbf{MLP-C}} & \multicolumn{6}{c|}{\textbf{MLP-R}} & \multicolumn{6}{c|}{\textbf{SVM-C}} & \multicolumn{6}{c}{\textbf{SVM-R}} \\ \hline
%  & Acc\tnote{1}  & T\tnote{2} & \#C\tnote{3}  & \begin{tabular}[c]{@{}c@{}}A \\ ($cm^{2}$)\end{tabular} & \begin{tabular}[c]{@{}c@{}}P\\ ($mW$)\end{tabular} & \begin{tabular}[c]{@{}c@{}}Clk\tnote{4}\;\\ (ms)\end{tabular} & Acc\tnote{1}  & T\tnote{2} & \#C\tnote{3} & \begin{tabular}[c]{@{}c@{}}A \\ ($cm^{2}$)\end{tabular} & \begin{tabular}[c]{@{}c@{}}P\\ ($mW$)\end{tabular} & \begin{tabular}[c]{@{}c@{}}Clk\tnote{4}\;\\ (ms)\end{tabular} & Acc\tnote{1}  & T\tnote{2} & \#C\tnote{3}  & \begin{tabular}[c]{@{}c@{}}A \\ ($cm^{2}$)\end{tabular} & \begin{tabular}[c]{@{}c@{}}P\\ ($mW$)\end{tabular} & \begin{tabular}[c]{@{}c@{}}Clk\tnote{4}\;\\ (ms)\end{tabular} & Acc\tnote{1}\;\;  & T\tnote{2}\;\; & \#C\tnote{3} & \begin{tabular}[c]{@{}c@{}}A \\ ($cm^{2}$)\end{tabular} & \begin{tabular}[c]{@{}c@{}}P\\ ($mW$)\end{tabular} & \begin{tabular}[c]{@{}c@{}}Clk\tnote{4}\;\\ (ms)\end{tabular}\\ \hline
 & Ac\tnote{1}  & T\tnote{2} & \#C\tnote{3}  & A\tnote{4} & P\tnote{5} & D\tnote{6}
 & Ac\tnote{1}  & T\tnote{2} & \#C\tnote{3}  & A\tnote{4} & P\tnote{5} & D\tnote{6}
 & Ac\tnote{1}  & T\tnote{2} & \#C\tnote{3}  & A\tnote{4} & P\tnote{5} & D\tnote{6}
 & Ac\tnote{1}  & T\tnote{2} & \#C\tnote{3}  & A\tnote{4} & P\tnote{5} & D\tnote{6}
\\ \hline 
\textbf{Cardio}    & 0.88 & (21,3,3)  & 72  & 33.4 & 124.2 & 123 & 0.83 & (21,3,1)  & 66 & 21.6 & 78.1 & 119 & 0.90 & 3  & 63  & 15.1 & 57.4 & 75 & 0.84 & 1  & 21 & 6.8  & 26.6 & 82\\
\textbf{RedWine}   & 0.56 & (11,2,6)  & 34  & 17.6 & 73.5 & 138 & 0.56 & (11,2,1)  & 24 & 7.1  & 28.9 & 101 & 0.57 & 15 & 66  & 23.5 & 92.8 & 66 & 0.56 & 1  & 11 & 4.0  & 18.9 & 77\\
\textbf{WhiteWine} & 0.54 & (11,4,7)  & 72  & 31.2 & 126.4 & 141 & 0.53 & (11,4,1)  & 48 & 13.1 & 48 & 125 & 0.53 & 21 & 77  & 28.3 & 112.4 & 60 & 0.53 & 1  & 11 & 4.2  & 18.9 & 83\\
\textbf{Seeds}    & 0.94 & (7,3,3)  & 30  & 9.9 & 45 & 134 & 0.87 & (7,3,1)  & 25 & 8.3 & 33 & 118 & 0.92 & 3  & 63  & 6.7 & 30.6 & 65 & 0.75 & 7  & 21 & 3.7  & 17.7 & 87\\
\textbf{Vertebral 3C}   & 0.83 & (6,3,3)  & 27  & 8.8 & 41.9 & 116 & 0.72 & (6,3,1)  & 21 & 8.5  & 34.6 & 115 & 0.84 & 3 & 66  & 4.0 & 20.9 & 58 & 0.66 & 1  & 6 & 2.9  & 14.3 & 80\\
\textbf{Balance Scale} & 0.91 & (4,3,3)  & 21  & 9.3 & 39.6 & 117 & 0.86 & (4,3,1)  & 15 & 5.5 & 24.4 & 94 & 0.89 & 3 & 77  & 1.9 & 9.7 & 56 & 0.81 & 1  & 4 & 2.1  & 10.0 & 67\\ \hline
\end{tabular}
\begin{tablenotes}\footnotesize
\item[] $^1$ Accuracy using $8$-bit coefficients and $4$-bit inputs.
$^2$ Model's topology (for SVMs: the number of classifiers).
$^3$ Number of coefficients of the model.
$^4$ Area in $cm^{2}$.
$^5$ Power in $mW$.
$^6$ Delay in $ms$.
\end{tablenotes}
\end{threeparttable}
\end{table*}

\section{Bespoke Machine Learning Classifiers}\label{sec:bespoke}

The low-fabrication and non-recurring engineering (NRE) costs of printed circuits can be leveraged to build highly customized bespoke ML circuits, i.e., circuit is customized to a specific model trained on a specific training dataset.
Such degree of customization is not realizable in conventional silicon-based systems, due to their high NRE costs.
Following the design methodology of~\cite{Mubarik:MICRO:2020:printedml}, we examine four different ML classification algorithms (Fig.~\ref{fig:architectures}), i.e., Multi-Layer Perceptrons classifier (MLP-C), Multi-layer Perceptron regressor (MLP-R), Support Vector Machine classification (SVM-C), as well as Support Vector Machine regression (SVM-R), and evaluate them in terms of accuracy and hardware overheads in printed technologies.
In these customized models \yellow{all} coefficients are hardwired in the circuit's description/implementation
%and the computation's logic is simplified, as one input of each neuron's multiplication is always constant, leading to a high area decrease.
and thus, logic is further simplified by constant propagation etc.
The topology of MLP-C and MLP-R (Fig.\ref{fig:architectures}a,b) is composed of one hidden layer and one up to five neurons, so that, for each model, close to maximum accuracy is achieved with the least number of hidden nodes.
Moreover, ReLU activation functions are used.
SVMs (Fig.\ref{fig:architectures}c,d) use a linear kernel and SVM-Cs are implemented with 1-vs-1 classification.

Each algorithm is trained on six different datasets of the UCI ML repository~\cite{Dua:2019:uci} (see Table~\ref{tab:baselines})
These datasets are selected similar to~\cite{Mubarik:MICRO:2020:printedml,Weller:2021:printed_stoch} and could form representative examples of sensor-based printed applications.
Training is performed using scikit-learn and a hyperparameter search (RandomizedSearchCV) with 5-fold cross validation.
All input features are normalized to $[0,1]$ and each dataset is divided into a random $30\%/70\%$ test and training dataset respectively.
For our baseline bespoke circuits we consider fixed-point arithmetic with the precision of all inputs and coefficients set to 4-bit and 8-bit, respectively, i.e., the smallest precision in which accuracy is close to floating-point one for all models.
Synthesis and power estimation of all circuits is obtained from Synopsys tools using the open source Electrolyte Gated Transistor (EGT) library~\cite{Bleier:ISCA:2020:printedmicro}, while testing accuracy is acquired with circuit simulations using Questasim.
% VOS is applied to each circuit by keeping the operating frequency constant and at its maximum value for each circuit, in which no error occurs, to boost performance as well.
All the circuits are synthesized at a relaxed clock(i.e., 200ms for all the designs), targeting to further improve area efficiency.
%All the circuits are synthesized at a relaxed and a typical clock for printed electronics~\cite{cadilha2017digital} (i.e., 200ms for all the designs), targeting to further improve area efficiency.
Then, the obtained circuits are operated at the maximum sustainable frequency (minimum delay) in which no timing violations occur.
%Since we also examine performance impact as well, all circuits operate at maximum frequency, in which no error occurs.
%Note, that due to this higher operating frequency of our baseline circuits, power consumption is also higher compared to our preliminary work~\cite{DATE22:Armen} (Table~\ref{tab:baselines}).

Table~\ref{tab:baselines} presents the characteristics, computation and hardware requirements (e.g. area and power) for our different baseline bespoke implementations.
% This table in conjuction with the Fig.~\ref{fig:architectures} helps us to understand the accuracy-implementation cost tradeoffs between different bespoke models.
As we can see, many circuits occupy area that is prohibitive for most printed applications ($>\!12cm^2$ on average), while power consumption of most circuits is so high (mainly $>30$mW) that they cannot be powered by a single existing printed battery.


