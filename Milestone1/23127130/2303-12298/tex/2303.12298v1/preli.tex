%\vspace{-2mm}
\section{Preliminary}\label{sec:preli}
\paragraph{Notations.}For a positive integer, we use $[n]$ to denote set $\{ 1,2,\cdots,n\}$. 
We use $\cosh(x) =\frac{1}{2}( e^x + e^{-x})$ and $\sinh(x) = \frac{1}{2}(e^x - e^{-x} )$.
For a square matrix, we use $\tr[A]$ to denote the trace of $A$.
An $n \times n$ symmetric real matrix $A$ is said to be positive-definite if $x^{\top} A x > 0$ for all non-zero $x \in \R^n$.
An $n \times n$ symmetric real matrix $A$ is said to be positive-semidefinite if $x^{\top} A x \geq 0$ for all non-zero $x \in \R^n$. For any function $f$, we use $\wt{O}(f) = f \cdot \poly(\log f)$.



\subsection{Matrix hyperbolic functions}
\begin{definition}[Matrix function]
Let $f:\R \rightarrow \R$ be a real function and $A\in \R^{n\times n}$ be a real symmetric function with eigendecomposition 
\begin{align*} 
A=Q\Lambda Q^{-1}
\end{align*}
where $\Lambda\in \R^{n\times n}$ is a diagonal matrix. Then, we have
\begin{align*}
    f(A):=Qf(\Lambda) Q^{-1},
\end{align*}
where $f(\Lambda)\in \R^{n\times n}$ is the matrix obtained by applying $f$ to each diagonal entry of $\Lambda$.
\end{definition}

We have the following lemma to bound $\cosh(A)$ and delay the proof to Appendix~\ref{sec:cosh_bound_proof}.
\begin{lemma}\label{lem:cosh_bound}
Let $A$ be a real symmetric matrix, then we have
\begin{align*}
    \|\cosh(A)\| = \cosh(\|A\|) \leq \tr[\cosh(A)].
\end{align*}
We also have 
\begin{align*}
\|A\| \leq 1+\log(\tr[\cosh(A)]).
\end{align*}
\end{lemma}



\subsection{Properties of \texorpdfstring{$\sinh$}{} 
and \texorpdfstring{$\cosh$}{}
}


We have the following lemma for properties of $\sinh$ and $\cosh$. 
\begin{lemma}[Scalar version]\label{lem:property_sinh_cosh_scalar}
Given a list of numbers $x_1, \cdots x_n$, we have
\begin{itemize}
    \item $( \sum_{i=1}^n \cosh^2(x_i) )^{1/2} \leq \sqrt{n} + ( \sum_{i=1}^n \sinh^2(x_i) )^{1/2}$,
    \item $(\sum_{i=1}^n \sinh^2(x_i) )^{1/2} \geq \frac{1}{\sqrt{n}} (\sum_{i=1}^n \cosh(x_i) - n)$.
\end{itemize}
\end{lemma}
\begin{proof}
For the first equation, we can bound $( \sum_{i=1}^n \cosh^2(x_i) )^{1/2}$ by:
\begin{align*}
     ( \sum_{i=1}^n \cosh^2(x_i) )^{1/2} 
    = &~ (n + \sum_{i=1}^n \sinh^2(x_i))^{1/2} \\
    \leq &~\sqrt{n} + ( \sum_{i=1}^n \sinh^2(x_i) )^{1/2}
\end{align*}
where the first step comes from fact~\ref{fact:cosh_sinh_1}, and the second step follows from $\sqrt{a + b} \leq \sqrt{a} + \sqrt{b}$.

For the second equation, we can bound $(\sum_{i=1}^n \sinh^2(x_i) )^{1/2}$ by:
\begin{align*}
    (\sum_{i=1}^n \sinh^2(x_i) )^{1/2} 
    \geq &~ \frac{1}{\sqrt{n}}(\sum_{i=1}^{n} \sinh(x_i)) \\
    \geq &~ \frac{1}{\sqrt{n}}(\sum_{i=1}^{n} \cosh(x_i) -n)
\end{align*}
where the first step follows that $\sqrt{\frac{\sum_{i=1}^{n} x_i^2}{n}} \geq \frac{\sum_{i=1}^{n} x_i}{n}$,
and the second step follows from fact~\ref{fact:cosh_sinh_1} and $\sqrt{x^2 -1} \geq \sqrt{x} - 1$.
\end{proof}


We also have a lemma for the matrix version. 
\begin{lemma}[Matrix version]\label{lem:property_sinh_cosh_matrix}
For any real symmetric matrix $A$, we have
\begin{itemize}
    \item $ (\tr[\cosh^2(A)])^{1/2} \leq \sqrt{n} + \tr[ \sinh^2(A) ]^{1/2}$,
    \item $(\tr[ \sinh^2(A) ])^{1/2} \geq \frac{1}{\sqrt{n}} ( \tr[ \cosh(A) ] - n ) $.
\end{itemize}
\end{lemma}

\begin{proof}

{\bf Part 1.}
We have
\begin{align*}
    (\tr[\cosh^2(A)])^{1/2} = & ~ ( n+  \tr[\sinh^2(A)] )^{1/2} \\
    \leq & ~ \sqrt{n} + \tr[ \sinh^2(A) ]^{1/2}.
\end{align*}
where the first step follows from $ \cosh^2(A) - \sinh^2(A) = I$.

{\bf Part 2.}
Let $\sigma_i$ denote the singular value of $\cosh(A)$
\begin{align*}
    ( \tr[  \sinh^2(A) ] )^{1/2} 
    = & ~  ( \tr[ \cosh^2(A) ] - n )^{1/2} \\
    = & ~  ( \sum_{i=1}^n \sigma_i^2 - 1 )^{1/2} \\ 
    \geq & ~   \frac{1}{\sqrt{n}} \sum_{i=1}^n \sqrt{ \sigma_i^2  -1 } \\
    \geq & ~    \frac{1}{\sqrt{n}} (\sum_{i=1}^n \sigma_i - 1 ) \\
    = & ~ \frac{1}{\sqrt{n}} ( \tr[  \cosh(A) ] -  n ) 
\end{align*}
where the second step follows from $\| \cdot \|_2 \geq \frac{1}{\sqrt{n}} \| \cdot \|_1$, the third step follows from $\sigma_i \geq 1$.

\end{proof}



