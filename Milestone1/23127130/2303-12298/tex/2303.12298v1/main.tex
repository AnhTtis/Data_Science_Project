  \def\isarxiv{1} %%% for icml submission version, we comment this line

\ifdefined\isarxiv
\documentclass[11pt]{article}

\usepackage[numbers]{natbib}

\else

\documentclass{article}
\usepackage{hyperref}
\usepackage{icml2023}

%\documentclass[nohyperref]{article}
%\usepackage{icml2022}
% \documentclass{uai2022}
% \documentclass{article}
% \usepackage{neurips_2022}
%\documentclass[nohyperref]{article}
%\usepackage{icml2022}
% \documentclass{article}
% \usepackage{neurips_2022}
\iffalse
\documentclass[letterpaper]{article} % DO NOT CHANGE THIS
\usepackage[submission]{aaai23}  % DO NOT CHANGE THIS
\usepackage{times}  % DO NOT CHANGE THIS
\usepackage{helvet}  % DO NOT CHANGE THIS
\usepackage{courier}  % DO NOT CHANGE THIS
\usepackage[hyphens]{url}  % DO NOT CHANGE THIS
\usepackage{graphicx} % DO NOT CHANGE THIS
\urlstyle{rm} % DO NOT CHANGE THIS
\def\UrlFont{\rm}  % DO NOT CHANGE THIS
\usepackage{natbib}  % DO NOT CHANGE THIS AND DO NOT ADD ANY OPTIONS TO IT
\usepackage{caption} % DO NOT CHANGE THIS AND DO NOT ADD ANY OPTIONS TO IT
\frenchspacing  % DO NOT CHANGE THIS
\setlength{\pdfpagewidth}{8.5in} % DO NOT CHANGE THIS
\setlength{\pdfpageheight}{11in} % DO NOT CHANGE THIS
\fi
\fi


\usepackage{amsmath}
\usepackage{amsthm}
\usepackage{amssymb}
%\usepackage{algorithmic}
\usepackage{algorithm}
\usepackage{subfig}
%\usepackage[english]{babel}
\usepackage{algpseudocode}
\usepackage{graphicx}
\usepackage{grffile}
%\usepackage{natbib}
%\usepackage[pdf]{pstricks}
\usepackage{wrapfig,epsfig}
%\usepackage{psfrag}
%\usepackage{epstopdf}
\usepackage{url}
\usepackage{xcolor}
%\usepackage{color}
\usepackage{epstopdf}
\usepackage{bbm}

\usepackage{dsfont} 

 %%% print refs in table of contents
%\displaybreak
\allowdisplaybreaks

%\usepackage[lmargin=1in,rmargin=1in,tmargin=0.8in,bmargin=0.8in]{geometry}

\ifdefined\isarxiv

\let\C\relax
\usepackage{tikz}
\usepackage{hyperref}  %%% arxiv don't allow this.
\hypersetup{colorlinks=true,citecolor=blue,linkcolor=blue} %%% Zhao : maybe we should comment this in submission.
\usetikzlibrary{arrows}
\usepackage[margin=1in]{geometry}

\else
%\usepackage{natbib}
\usepackage{microtype}
%\usepackage{hyperref}
%\definecolor{mydarkblue}{rgb}{0,0.08,0.45}
%\hypersetup{colorlinks=true, citecolor=mydarkblue,linkcolor=mydarkblue}
%\usepackage[capitalize,noabbrev]{cleveref}
%\usepackage{colortbl}

\fi
%\linespread{1}
%\newcommand{\QED}{\hfill$\qed$}
%\graphicspath{{./figs/}}

\ifdefined\isarxiv 

\else 

\iffalse

% These are are recommended to typeset listings but not required. See the subsubsection on listing. Remove this block if you don't have listings in your paper.
\usepackage{newfloat}
\usepackage{listings}
\DeclareCaptionStyle{ruled}{labelfont=normalfont,labelsep=colon,strut=off} % DO NOT CHANGE THIS
\lstset{%
	basicstyle={\footnotesize\ttfamily},% footnotesize acceptable for monospace
	numbers=left,numberstyle=\footnotesize,xleftmargin=2em,% show line numbers, remove this entire line if you don't want the numbers.
	aboveskip=0pt,belowskip=0pt,%
	showstringspaces=false,tabsize=2,breaklines=true}
\floatstyle{ruled}
\newfloat{listing}{tb}{lst}{}
\floatname{listing}{Listing}
%
% Keep the \pdfinfo as shown here. There's no need
% for you to add the /Title and /Author tags.
\pdfinfo{
/TemplateVersion (2023.1)
}
\fi
\fi 


\newtheorem{theorem}{Theorem}[section]
\newtheorem{lemma}[theorem]{Lemma}
\newtheorem{definition}[theorem]{Definition}
\newtheorem{notation}[theorem]{Notation}
%\newtheorem{proof}[theorem]{Proof}
\newtheorem{proposition}[theorem]{Proposition}
\newtheorem{corollary}[theorem]{Corollary}
\newtheorem{conjecture}[theorem]{Conjecture}
\newtheorem{assumption}[theorem]{Assumption}
\newtheorem{observation}[theorem]{Observation}
\newtheorem{fact}[theorem]{Fact}
\newtheorem{remark}[theorem]{Remark}
\newtheorem{claim}[theorem]{Claim}
\newtheorem{example}[theorem]{Example}
\newtheorem{problem}[theorem]{Problem}
\newtheorem{open}[theorem]{Open Problem}
\newtheorem{property}[theorem]{Property}
\newtheorem{hypothesis}[theorem]{Hypothesis}

\newcommand{\wh}{\widehat}
\newcommand{\wt}{\widetilde}
\newcommand{\ov}{\overline}
\newcommand{\eps}{\epsilon}
\newcommand{\N}{\mathcal{N}}
\newcommand{\R}{\mathbb{R}}
\newcommand{\RHS}{\mathrm{RHS}}
\newcommand{\LHS}{\mathrm{LHS}}
\renewcommand{\d}{\mathrm{d}}
\renewcommand{\i}{\mathbf{i}}
\renewcommand{\varepsilon}{\epsilon}
\renewcommand{\tilde}{\wt}
\renewcommand{\hat}{\wh}
\newcommand{\ReLU}{{$\mathsf{ReLU}$}}

\DeclareMathOperator*{\E}{{\mathbb{E}}}
\DeclareMathOperator*{\var}{\mathrm{Var}}
\DeclareMathOperator*{\Z}{\mathbb{Z}}
\DeclareMathOperator*{\C}{\mathbb{C}}
\DeclareMathOperator*{\D}{\mathcal{D}}
\DeclareMathOperator*{\median}{median}
\DeclareMathOperator*{\mean}{mean}
\DeclareMathOperator{\OPT}{OPT}
\DeclareMathOperator{\supp}{supp}
\DeclareMathOperator{\poly}{poly}

\DeclareMathOperator{\nnz}{nnz}
\DeclareMathOperator{\sparsity}{sparsity}
\DeclareMathOperator{\rank}{rank}
%\DeclareMathOperator{\diag}{diag}
\DeclareMathOperator{\dist}{dist}
\DeclareMathOperator{\cost}{cost}
\DeclareMathOperator{\vect}{vec}
\DeclareMathOperator{\tr}{tr}
\DeclareMathOperator{\dis}{dis}
\DeclareMathOperator{\cts}{cts}


\newcommand{\Zhao}[1]{{\color{red}[Zhao: #1]}}
\makeatletter
\newcommand*{\RN}[1]{\expandafter\@slowromancap\romannumeral #1@}
\makeatother
\newcommand{\Danyang}[1]{{\color{purple}[Danyang: #1]}}
\newcommand{\lianke}[1]{{\color{blue}[Lianke: #1]}}
\newcommand{\revision}[1]{{\color{black} #1}}
\newcommand{\Ruizhe}[1]{{\color{orange}[Ruizhe: #1]}}

\usepackage{lineno}
\def\linenumberfont{\normalfont\small}


\iffalse
\renewcommand{\Zhao}[1]{\textcolor{b2}{[]}}
\renewcommand{\lianke}[1]{\textcolor{red}{[]}}
\renewcommand{\Danyang}[1]{\textcolor{blue}{[]}}
\renewcommand{\Ruizhe}[1]{{\color{orange}[]}}
\fi


\ifdefined\isarxiv

\else 
\setcounter{secnumdepth}{2} %May be changed to 1 or 2 if section numbers are desired.
\fi

\begin{document}

\ifdefined\isarxiv

\date{}

\title{A General Algorithm for Solving Rank-one Matrix Sensing}
\author{
Lianke Qin\thanks{\texttt{lianke@ucsb.edu}. UCSB.}
\and 
Zhao Song\thanks{\texttt{zsong@adobe.com}. Adobe Research.}
\and 
Ruizhe Zhang\thanks{\texttt{ruizhe@utexas.edu}. The University of Texas at Austin.}
}




\else
%
\twocolumn[

\icmltitle{A General Algorithm for Solving Rank-one Matrix Sensing}
% It is OKAY to include author information, even for blind
% submissions: the style file will automatically remove it for you
% unless you've provided the [accepted] option to the icml2019
% package.

% List of affiliations: The first argument should be a (short)
% identifier you will use later to specify author affiliations
% Academic affiliations should list Department, University, City, Region, Country
% Industry affiliations should list Company, City, Region, Country

% You can specify symbols, otherwise they are numbered in order.
% Ideally, you should not use this facility. Affiliations will be numbered
% in order of appearance and this is the preferred way.
\icmlsetsymbol{equal}{*}

\begin{icmlauthorlist}
\icmlauthor{Aeiau Zzzz}{equal,to}
\icmlauthor{Bauiu C.~Yyyy}{equal,to,goo}
\icmlauthor{Cieua Vvvvv}{goo}
\icmlauthor{Iaesut Saoeu}{ed}
\icmlauthor{Fiuea Rrrr}{to}
\icmlauthor{Tateu H.~Yasehe}{ed,to,goo}
\icmlauthor{Aaoeu Iasoh}{goo}
\icmlauthor{Buiui Eueu}{ed}
\icmlauthor{Aeuia Zzzz}{ed}
\icmlauthor{Bieea C.~Yyyy}{to,goo}
\icmlauthor{Teoau Xxxx}{ed}\label{eq:335_2}
\icmlauthor{Eee Pppp}{ed}
\end{icmlauthorlist}

\icmlaffiliation{to}{Department of Computation, University of Torontoland, Torontoland, Canada}
\icmlaffiliation{goo}{Googol ShallowMind, New London, Michigan, USA}
\icmlaffiliation{ed}{School of Computation, University of Edenborrow, Edenborrow, United Kingdom}

\icmlcorrespondingauthor{Cieua Vvvvv}{c.vvvvv@googol.com}
\icmlcorrespondingauthor{Eee Pppp}{ep@eden.co.uk}

% You may provide any keywords that you
% find helpful for describing your paper; these are used to populate
% the "keywords" metadata in the PDF but will not be shown in the document
\icmlkeywords{Machine Learning, ICML}

\vskip 0.3in
]

\printAffiliationsAndNotice{\icmlEqualContribution} 
\fi





\ifdefined\isarxiv
\begin{titlepage}
  \maketitle
  \begin{abstract}


Over the past few years, there has been a significant amount of research focused on studying the ReLU activation function, with the aim of achieving neural network convergence through over-parametrization. However, recent developments in the field of Large Language Models (LLMs) have sparked interest in the use of exponential activation functions, specifically in the attention mechanism.

Mathematically, we define the neural function $F: \R^{d \times m} \times  \mathbb{R}^d \rightarrow \mathbb{R}$ using an exponential activation function. Given a set of data points with labels $\{(x_1, y_1), (x_2, y_2), \dots, (x_n, y_n)\} \subset \mathbb{R}^d \times \mathbb{R}$ where $n$ denotes the number of the data. Here $F(W(t),x)$ can be expressed as $F(W(t),x) := \sum_{r=1}^m a_r \exp(\langle w_r, x \rangle)$, where $m$ represents the number of neurons, and $w_r(t)$ are weights at time $t$. It's standard in literature that $a_r$ are the fixed weights and it's never changed during the training. We initialize the weights $W(0) \in \mathbb{R}^{d \times m}$ with random Gaussian distributions, such that $w_r(0) \sim \mathcal{N}(0, I_d)$ and initialize $a_r$ from random sign distribution for each $r \in [m]$.

Using the gradient descent algorithm, we can find a weight $W(T)$ such that $\| F(W(T), X) - y \|_2 \leq \epsilon$ holds with probability $1-\delta$, where $\epsilon \in (0,0.1)$ and $m = \Omega(n^{2+o(1)}\log(n/\delta))$. To optimize the over-parametrization bound $m$, we employ several tight analysis techniques from previous studies [Song and Yang arXiv 2019, Munteanu, Omlor, Song and Woodruff ICML 2022]. 

 


  \end{abstract}
  \thispagestyle{empty}
\end{titlepage}

{\hypersetup{linkcolor=black}
\tableofcontents
}
\newpage

\else
%\maketitle
\begin{abstract}


Over the past few years, there has been a significant amount of research focused on studying the ReLU activation function, with the aim of achieving neural network convergence through over-parametrization. However, recent developments in the field of Large Language Models (LLMs) have sparked interest in the use of exponential activation functions, specifically in the attention mechanism.

Mathematically, we define the neural function $F: \R^{d \times m} \times  \mathbb{R}^d \rightarrow \mathbb{R}$ using an exponential activation function. Given a set of data points with labels $\{(x_1, y_1), (x_2, y_2), \dots, (x_n, y_n)\} \subset \mathbb{R}^d \times \mathbb{R}$ where $n$ denotes the number of the data. Here $F(W(t),x)$ can be expressed as $F(W(t),x) := \sum_{r=1}^m a_r \exp(\langle w_r, x \rangle)$, where $m$ represents the number of neurons, and $w_r(t)$ are weights at time $t$. It's standard in literature that $a_r$ are the fixed weights and it's never changed during the training. We initialize the weights $W(0) \in \mathbb{R}^{d \times m}$ with random Gaussian distributions, such that $w_r(0) \sim \mathcal{N}(0, I_d)$ and initialize $a_r$ from random sign distribution for each $r \in [m]$.

Using the gradient descent algorithm, we can find a weight $W(T)$ such that $\| F(W(T), X) - y \|_2 \leq \epsilon$ holds with probability $1-\delta$, where $\epsilon \in (0,0.1)$ and $m = \Omega(n^{2+o(1)}\log(n/\delta))$. To optimize the over-parametrization bound $m$, we employ several tight analysis techniques from previous studies [Song and Yang arXiv 2019, Munteanu, Omlor, Song and Woodruff ICML 2022]. 

 

\end{abstract}

\fi


\section{Introduction}

The increasing complexity of source code poses a key challenge to the reliability of large-scale software systems. Software bugs in these systems can lead to safety issues~\cite{bug_safety} for users around the world as well as cause non-negligible financial losses~\cite{bug_loss}. As such, developers have to spend a large amount of time and effort on bug fixing. Consequently, \aprfull (\apr), designed to automatically generate patches to fix software bugs, has attracted wide attention from both academia and industry~\cite{long2016prophet, legoues2012genprog, long2015spr, lou2020can, tufano2018empstudy}. 


To achieve \apr, one popular approach is known as Generate-and-Validate (G\&V)~\cite{qi2015gv, ghanbari2019prapr, lou2020can, le2016hdrepair, legoues2012genprog, wen2018capgen, hua2018sketchfix, martinez2016astor, koyuncu2020fixminder, liu2019tbar, liu2019avatar}, which is typically based on the following pipeline: First, fault localization techniques~\cite{wong2016fl, abreu2007ochiai, zhang2013injecting, papadakis2015metallaxis, li2019deepfl, li2017transforming} are applied to determine the suspicious locations in programs where bugs are likely to exist. Then, the buggy locations are used by the \apr tools to generate a list of patches that replace buggy lines with correct lines. Afterward, each patch is validated against the original test suite to identify any \emph{plausible patches} (i.e., passing all tests in the test suite). Finally, to determine the \emph{correct patches}, developers examine the list of plausible patches to see if any of them can correctly fix the bug. 

Traditional \apr tools can mainly be categorized into heuristic-based~\cite{legoues2012genprog, le2016hdrepair, wen2018capgen}, constraint-based~\cite{mechtaev2016angelix, le2017s3, demacro2014nopol, long2015spr} and \template~\cite{ghanbari2019prapr, hua2018sketchfix, martinez2016astor, liu2019tbar, liu2019avatar}. Among these traditional tools, \template \apr tools~\cite{ghanbari2019prapr, liu2019tbar, benton2020effectiveness} have been able to achieve state-of-the-art results. \Template \apr tools typically leverage pre-defined templates (e.g., adding a nullness check) for bug fixing. However, since these fix templates are typically handcrafted, the number and types of bugs they are able to fix can be limited. 



To address the limitations of traditional \apr, researchers have proposed various \learning \apr tools~\cite{li2020dlfix, chen2018sequencer, jiang2021cure, lutellier2020coconut, zhu2021recoder, ye2022rewardrepair} based on the \nmtfull (\nmt) architecture~\cite{sutskever2014mt} where the input is the buggy code snippets and the goal is to translate the buggy code snippets into a fixed version. To accomplish this, \learning \apr tools require supervised training datasets with pairs of both buggy and fixed code snippets in order to learn how to perform this translation step. These training data are usually obtained by mining historical bug fixes using heuristics/keywords~\cite{dallmeier2007benchmark}, which can be imprecise for identifying bug-fixing commits; even the actual bug-fixing commits can include irrelevant code changes, leading to further pollution in the dataset~\cite{xia2022alpharepair}.
% 
Moreover, it can be hard for such \apr tools to generalize and fix bug types unseen during training. 



To better leverage recent advances in \plmfull{s} (\plm{s}), researchers~\cite{xia2022alpharepair, xia2023repairstudy, kolak2022patch, prenner2021codexws} have directly applied \plm{s} to generate patches without bug-fixing datasets. These \llm-based \apr tools work by either directly generating a complete code function~\cite{prenner2021codexws, xia2023repairstudy} or predict/infill the correct code snippet given its surrounding context~\cite{xia2022alpharepair, xia2023repairstudy}. By directly using \llm{s} that are pre-trained on billions of open-source code snippets, \llm-based \apr tools can achieve state-of-the-art performance on many repair datasets~\cite{xia2022alpharepair}. 


% 
%
%

Traditional \apr tools have long used the insight of the \emph{plastic surgery hypothesis}~\cite{barr2014plastic} where it states that the code ingredients to fix a bug already exist within the same project. Traditional \apr tools have manually designed pattern-~\cite{ghanbari2019prapr, saha2017elixir} or heuristic-based~\cite{jiang2018simfix, legoues2012genprog} approaches to finding and using such relevant code ingredients to generate fixes for bugs. However, the plastic surgery hypothesis has been largely ignored in \llm-based \apr. In fact, \llm provides a unique opportunity to fully automate the plastic surgery hypothesis idea via fine-tuning (learning project-specific information via model updates from the buggy project) and prompting (directly providing relevant code ingredients to the model), and make it directly applicable to different languages (since the \llm{s} are typically multi-lingual).%
Moreover, despite the intensive manual efforts involved, traditional \apr tools still cannot fully leverage project-specific information due to large search space for leveraging/composing existing code ingredients. In contrast, the project-specific information can effectively leveraged by \llm{s} due to their power in code understanding/vectorization, e.g., even partial/imprecise information may still guide \llm{s} in correct patch generation!
 To this end, we ask the question: \emph{How useful is the plastic surgery hypothesis in the era of \plm{s}}?








\mypara{Our Work.} To answer the question, we present \ourtech{\xspace} -- a \llm-based approach that automatically utilizes the plastic surgery hypothesis by systematically combining multiple fine-tuning and prompting strategies for \apr. \ourtech fine-tunes \plm{s} using two novel domain-specific training strategies: \textbf{\epfinetune} -- we fine-tune using the original buggy project by aggressively masking out a high percentage of tokens, which allows \plm to learn project-specific code tokens and programming styles; and \textbf{\rofinetune} -- which only masks out a single continuous code sequence per training sample, allowing the model to get used to the final \csapr task of predicting a single continuous code sequence. Furthermore, we directly leverage the ability for \plm{s} to understand natural language instructions and introduce a novel prompting strategy, \textbf{\idprompting}, which uses information retrieval and static analysis to obtain a list of relevant identifiers for the buggy lines. While such relevant identifiers are critical for fixing some difficult bugs, they may not be seen by the \llm during inference due to limited context window size. Through the use of prompting, we directly tell the model to use these extracted identifiers (relevant code ingredients) to generate the correct code. Finally, to perform repair, we combine all four model variants (including the base model, both fine-tuned models and the base model with prompting) for the final repair.





While our insight of leveraging the plastic surgery hypothesis for \llm-based \apr is generalizable across different types of \plm{s}, to implement \ourtech, we choose a recent \plm{\xspace}, \ctfive~\cite{wang2021codet5}, which is pre-trained on millions of open-source code snippets. \ctfive is an encoder-decoder model trained using \mspfull (\msp) objective where a percentage of tokens are masked out and each continuous masked token sequence is referred to as a masked span. Also, although we only extract relevant identifiers from the current buggy project (since this paper focuses on the plastic surgery hypothesis), our work can be easily extended to obtain other code information (such as relevant statements or functions) from other sources, such as  the massive pre-training corpora~\cite{husain2020codesearchnet} or historical bug-fixing datasets~\cite{jiang2019infer}, which can provide more coding knowledge for \llm{s}. Besides, although we mainly focus on using traditional string comparison algorithms for information retrieval in this paper, these techniques can be easily replaced by other frequency-based retrieval~\cite{robertson2009probabilistic} and neural search (or embedding-based search)~\cite{reimers2019sentence}.
  In summary, this paper makes the following contributions:


%


\begin{itemize}[noitemsep, leftmargin=*, topsep=0pt]
    \item \textbf{Dimension.} This paper is the first to revisit the important plastic surgery hypothesis in the era of \llm{s}. It opens up a new dimension for \llm-based \apr to incorporate previously neglected information from the buggy project itself to boost \apr performance. Furthermore, it demonstrates the promising future of retrieval-based prompting for modern \llm-based \apr.
    \item \textbf{Implementation.} We implement \ourtech based on the recent \ctfive model. We augment the model using two novel fine-tuning strategies: \epfinetune and \rofinetune, along with a novel prompting strategy based on information retrieval and static analysis: \idprompting. We combine the patches generated by all four models together and perform patch ranking to speed up \apr.% 
    \item \textbf{Evaluation Study.} We conduct an extensive evaluation against state-of-the-art \apr tools. On the widely studied \dfj 1.2 and 2.0 datasets~\cite{just2014dfj}, \ourtech is able to achieve the new state-of-the-art results of 89 and 44 correct bug fixes (15 and 8 more than best baseline) respectively.  Furthermore, we perform a broad ablation study to justify our design. \ourtech demonstrates for the first time that the plastic surgery hypothesis can substantially boost \llm-based \apr and advance state-of-the-art \apr, while being fully automated and general. Moreover, even partial/imprecise code ingredients may still effectively guide \llm{s} for \apr!
\end{itemize}

 %%% Sectio
 
\section{Related Work}


 
\paragraph{Linear Progamming}
Linear programming is one of foundations of the algorithm design and convex optimization. many problems can be modeled as linear programs to take advantage of fast algorithms.
There are many works in accelerating linear programming runtime complexity~\citep{ls14, ls15, cls19, lsz19, b20, blss20, sy21, dly21, jswz21,gs22}.

\paragraph{Semi-definite Programming}

 
Semidefinite programming optimizes a linear objective function over the intersection of the
positive semidefinite cone with an affine space. Semidefinite programming is a fundamental class of
optimization problems and many problems in machine learning, and theoretical computer science
can be modeled or approximated as semidefinite programming problems. There are many studies to speedup the running
time of Semidefinite programming~\citep{nn94, hrvw96, lsw15, jlsw20, jklps20, hjstz21, gs22}.

\paragraph{Matrix Sensing}

Matrix sensing~\citep{lb09, rfp10, jmd10, zjd15, dls23} is a generalization of the popular compressive sensing problem for the sparse vectors and has applications in several
domains such as control, vision etc.
a set of universal Pauli measurements,
used in quantum state tomography, have been shown to satisfy the RIP condition~\citep{l11}.
These measurement operators are Kronecker products of $2 \times 2$ matrices, thus, they have appealing computation and memory efficiency. Rank-one measurement using nuclear norm minimization is also used in other work~\citep{cz15, krt17}.
There is also previous work working on low-rank matrix sensing to reconstruct a matrix exactly using a small number of linear measurements.
ProcrustesFlow~\cite{tbssr16} designs an algorithm to recover a low-rank matrix from linear measurements.
There are other low-rank matrix recovering algorithms based on non-convex optimizations~\cite{ wzg17, lmcc19}.
 


%\vspace{-2mm}
\section{Preliminary}\label{sec:preli}
\paragraph{Notations.}For a positive integer, we use $[n]$ to denote set $\{ 1,2,\cdots,n\}$. 
We use $\cosh(x) =\frac{1}{2}( e^x + e^{-x})$ and $\sinh(x) = \frac{1}{2}(e^x - e^{-x} )$.
For a square matrix, we use $\tr[A]$ to denote the trace of $A$.
An $n \times n$ symmetric real matrix $A$ is said to be positive-definite if $x^{\top} A x > 0$ for all non-zero $x \in \R^n$.
An $n \times n$ symmetric real matrix $A$ is said to be positive-semidefinite if $x^{\top} A x \geq 0$ for all non-zero $x \in \R^n$. For any function $f$, we use $\wt{O}(f) = f \cdot \poly(\log f)$.



\subsection{Matrix hyperbolic functions}
\begin{definition}[Matrix function]
Let $f:\R \rightarrow \R$ be a real function and $A\in \R^{n\times n}$ be a real symmetric function with eigendecomposition 
\begin{align*} 
A=Q\Lambda Q^{-1}
\end{align*}
where $\Lambda\in \R^{n\times n}$ is a diagonal matrix. Then, we have
\begin{align*}
    f(A):=Qf(\Lambda) Q^{-1},
\end{align*}
where $f(\Lambda)\in \R^{n\times n}$ is the matrix obtained by applying $f$ to each diagonal entry of $\Lambda$.
\end{definition}

We have the following lemma to bound $\cosh(A)$ and delay the proof to Appendix~\ref{sec:cosh_bound_proof}.
\begin{lemma}\label{lem:cosh_bound}
Let $A$ be a real symmetric matrix, then we have
\begin{align*}
    \|\cosh(A)\| = \cosh(\|A\|) \leq \tr[\cosh(A)].
\end{align*}
We also have 
\begin{align*}
\|A\| \leq 1+\log(\tr[\cosh(A)]).
\end{align*}
\end{lemma}



\subsection{Properties of \texorpdfstring{$\sinh$}{} 
and \texorpdfstring{$\cosh$}{}
}


We have the following lemma for properties of $\sinh$ and $\cosh$. 
\begin{lemma}[Scalar version]\label{lem:property_sinh_cosh_scalar}
Given a list of numbers $x_1, \cdots x_n$, we have
\begin{itemize}
    \item $( \sum_{i=1}^n \cosh^2(x_i) )^{1/2} \leq \sqrt{n} + ( \sum_{i=1}^n \sinh^2(x_i) )^{1/2}$,
    \item $(\sum_{i=1}^n \sinh^2(x_i) )^{1/2} \geq \frac{1}{\sqrt{n}} (\sum_{i=1}^n \cosh(x_i) - n)$.
\end{itemize}
\end{lemma}
\begin{proof}
For the first equation, we can bound $( \sum_{i=1}^n \cosh^2(x_i) )^{1/2}$ by:
\begin{align*}
     ( \sum_{i=1}^n \cosh^2(x_i) )^{1/2} 
    = &~ (n + \sum_{i=1}^n \sinh^2(x_i))^{1/2} \\
    \leq &~\sqrt{n} + ( \sum_{i=1}^n \sinh^2(x_i) )^{1/2}
\end{align*}
where the first step comes from fact~\ref{fact:cosh_sinh_1}, and the second step follows from $\sqrt{a + b} \leq \sqrt{a} + \sqrt{b}$.

For the second equation, we can bound $(\sum_{i=1}^n \sinh^2(x_i) )^{1/2}$ by:
\begin{align*}
    (\sum_{i=1}^n \sinh^2(x_i) )^{1/2} 
    \geq &~ \frac{1}{\sqrt{n}}(\sum_{i=1}^{n} \sinh(x_i)) \\
    \geq &~ \frac{1}{\sqrt{n}}(\sum_{i=1}^{n} \cosh(x_i) -n)
\end{align*}
where the first step follows that $\sqrt{\frac{\sum_{i=1}^{n} x_i^2}{n}} \geq \frac{\sum_{i=1}^{n} x_i}{n}$,
and the second step follows from fact~\ref{fact:cosh_sinh_1} and $\sqrt{x^2 -1} \geq \sqrt{x} - 1$.
\end{proof}


We also have a lemma for the matrix version. 
\begin{lemma}[Matrix version]\label{lem:property_sinh_cosh_matrix}
For any real symmetric matrix $A$, we have
\begin{itemize}
    \item $ (\tr[\cosh^2(A)])^{1/2} \leq \sqrt{n} + \tr[ \sinh^2(A) ]^{1/2}$,
    \item $(\tr[ \sinh^2(A) ])^{1/2} \geq \frac{1}{\sqrt{n}} ( \tr[ \cosh(A) ] - n ) $.
\end{itemize}
\end{lemma}

\begin{proof}

{\bf Part 1.}
We have
\begin{align*}
    (\tr[\cosh^2(A)])^{1/2} = & ~ ( n+  \tr[\sinh^2(A)] )^{1/2} \\
    \leq & ~ \sqrt{n} + \tr[ \sinh^2(A) ]^{1/2}.
\end{align*}
where the first step follows from $ \cosh^2(A) - \sinh^2(A) = I$.

{\bf Part 2.}
Let $\sigma_i$ denote the singular value of $\cosh(A)$
\begin{align*}
    ( \tr[  \sinh^2(A) ] )^{1/2} 
    = & ~  ( \tr[ \cosh^2(A) ] - n )^{1/2} \\
    = & ~  ( \sum_{i=1}^n \sigma_i^2 - 1 )^{1/2} \\ 
    \geq & ~   \frac{1}{\sqrt{n}} \sum_{i=1}^n \sqrt{ \sigma_i^2  -1 } \\
    \geq & ~    \frac{1}{\sqrt{n}} (\sum_{i=1}^n \sigma_i - 1 ) \\
    = & ~ \frac{1}{\sqrt{n}} ( \tr[  \cosh(A) ] -  n ) 
\end{align*}
where the second step follows from $\| \cdot \|_2 \geq \frac{1}{\sqrt{n}} \| \cdot \|_1$, the third step follows from $\sigma_i \geq 1$.

\end{proof}




\section{Technique Overview}\label{sec:tech_overview}
This paper presents a proof demonstrating that a two-layer neural network employing the exponential activation function which can achieve a desired small loss value after sufficient iterations, given a large enough number of neurons $m$, an appropriate learning rate $\eta$, and the initialization method specified in Definition~\ref{def:duplicate_weights}.

By bounding the difference of the weights over the training and choosing a proper learning rate $\eta$, we bound the loss by induction. We will introduce how we bound the loss under the assumption for the small perturbation on the weights. And then we will introduce how we bound the  weights and gradients respectively.

\paragraph{Bounding the loss by induction}

To establish this result, we begin by bounding the summation of the differences between the weights at the current step and their initial values, assuming that $w$ is in a small range such that $\Delta w_r(t)\leq R\in (0,0.01)$ where $t$ denote the step here. 

To establish an upper bound on the prediction loss $\|y-F(t+1)\|_2^2$, we first assume that the weight $w$ is within a small range, namely $\Delta w_r(t)\leq R\in (0,0.01)$. We decompose the loss into four parts: 
\begin{itemize}
    \item The loss at the previous step  $\|y-F(t)\|_2^2$
    \item $C_1:=-2m \eta (F(t)-y)^\top H(t) (F(t)-y)$
    \item $C_2:=2 (F(t)-y)^\top H_{\asy} (t) (F(t)-y)$
    \item $C_3:=\| F(t+1) - F(t) \|_2^2$
\end{itemize}

Using terms that involve the parameters $m$, $\eta$, $n$, and $B$ multiplied by $\|y-F(t)\|_2^2$, we can compute the upper bounds $C_1$, $C_2$, and $C_3$ respectively. Then, by induction and choosing a proper value for $m$, we can bound $\|y-F(t)\|_2^2$ where the loss is $\|y-F(0)\|_2^2= \|y\|_2^2$ at initialization under the given assumption.

\paragraph{Bounding the weights by induction}
Finally, to complete the proof, we establish an upper bound for $\Delta w_r(t)=\|w_r(t)-w_r(0)\|_2$. To do this, we transform $\Delta w_r(t)$ into a form that is a multiple of $\exp(B+R)\sqrt{n}$ and the current loss $\|y-F(t)\|_2$. 

\paragraph{Bounding the gradients by induction}
 Based on the result above, we will continue our work on bounding the changing of the weights over the training. By appropriately choosing the learning rate $\eta$, we can ensure that $\Delta w_r(t)$ is small enough, namely $0.01$. We have the following definition 
\begin{align*}
    H(w)_{i,j} :=  \frac{1}{m} \langle x_i,x_j\rangle \sum_{r\in [m]} \exp(\langle w_r,x_i\rangle)\cdot \exp( \langle w_r,x_j\rangle)
\end{align*}
where $r\in [m]$ as the index the neurons, and $x_i$ denotes the data where $i\in [n]$ and $j\in [n]$.

Next, we will constrain the changes of $H$ under the assumption that $w$ is located within a small ball. In addition, we must bound the discrepancy between discrete and continuous functions. Drawing upon the conclusions reached regarding perturbations in weight $w$, we can guarantee the convergence of the over-parametrized neural network with an exponential activation function. 




 
\section{Gradient descent for entry-wise potential function}\label{sec:gd}

In this section, we show how to obtain an approximate solution of matrix sensing via gradient descent. For simplicity, we start from a case that $\{u_i\}_{i\in [m]}$ are orthogonal vectors in $\R^n$\footnote{We note that $A':=\sum_{i=1}^m b_iu_iu_i^\top$ is a solution satisfying $u_i^\top A'u_i = b_i$ for all $i\in [m]$. However, we pretend that we do not know this solution in this section.}, which already conveys the key idea of our algorithm and analysis and we generalize the solution to the non-orthogonal case (see Appendix~\ref{sec:gd_non_orthogonal}). 
We show that $\wt{\Omega}(\sqrt{m}/\delta)$ iterations of gradient descent can output a $\delta$-approximate solution, where each iteration takes $O(mn^2)$-time. Below is the main theorem of this section:

\begin{theorem}[Gradient descent for orthogonal measurements]\label{thm:gd_orthogonal}
Suppose $u_1,\dots,u_m\in \R^n$ are orthogonal unit vectors, and suppose $|b_i|\leq R$ for all $i\in [m]$. There exists an algorithm such that for any $\delta \in (0,1)$, performs $\wt{\Omega}(\sqrt{m}R\delta^{-1})$ iterations of gradient descent with $O(mn^2)$-time per iteration and outputs a matrix $A\in \R^{n\times n}$ satisfies:
\begin{align*}
    | u_i^\top A u_i - b_i| \leq \delta~~~\forall i\in [m].
\end{align*}
\end{theorem}

In Section~\ref{sec:gd_algorithm}, we introduce the algorithm and prove the time complexity. In Section~\ref{sec:gd_analysis_1} - \ref{sec:gd_analysis_2}, we analyze the convergence of our algorithm.

\subsection{Algorithm}\label{sec:gd_algorithm}

The key idea of the gradient descent matrix sensing algorithm (Algorithm~\ref{alg:GD}) is to follow the gradient of the entry-wise potential function defined as follows:
\begin{align}
    \Phi_{\lambda}(A) := \sum_{i=1}^m \cosh ( \lambda ( u_i^\top A u_i - b_i ) ).
\end{align}
Then, we have the following solution update formula:
\begin{align}\label{eq:A_t1_update}
    A_{t+1} \gets A_t - \epsilon \cdot \nabla \Phi_{\lambda}(A_t) / \| \nabla \Phi_{\lambda}(A_t) \|_F.
\end{align}

\begin{lemma}[Cost-per-iteration of gradient descent]\label{lem:gd_cost_per_iter}
Each iteration of Algorithm~\ref{alg:GD} takes
$
    O(mn^2)
$-time.
\end{lemma}
\begin{proof}
In each iteration, we first evaluate $u_i^\top A_tu_i$ for all $i\in [m]$, which takes $O(mn^2)$-time. Then, $\nabla \Phi_\lambda(A_t)$ can be computed by summing $m$ rank-1 matrices, which takes $O(mn^2)$-time. Finally, at Line~\ref{ln:gd_update}, the solution can be updated in $O(n^2)$-time.
Thus, the total running time for each iteration is $O(mn^2)$.
\end{proof}

 
\begin{algorithm}\caption{Matrix Sensing by Gradient Descent.}\label{alg:GD}
\begin{algorithmic}[1]
\Procedure{GradientDescent}{$\{u_i,b_i\}_{i\in [m]}$} \Comment{Theorem~\ref{thm:gd_orthogonal}}
    \State $\tau \gets \max_{i \in [m]} b_i $
    \State $A_1 \gets \tau \cdot I$
    \For{$t = 1 \to T$}
        \State $\nabla \Phi_{\lambda} (A_t) \gets \sum_{i=1}^m u_i u_i^\top \lambda \sinh( \lambda ( u_i^\top A_t u_i - b_i ) ) $ \Comment{Compute the gradient}
        \State $A_{t+1} \gets A_t - \epsilon \cdot \nabla \Phi_{\lambda}(A_t) / \| \nabla \Phi_{\lambda}(A_t) \|_F$\label{ln:gd_update}
    \EndFor
    \State \Return $A_{T+1}$
\EndProcedure
\end{algorithmic}
\end{algorithm}


 
\subsection{Analysis of One Iteration}\label{sec:gd_analysis_1}
Throughout this section, we suppose $A\in \R^{n\times n}$ is a symmetric matrix. 
 

We can compute the gradient of $\Phi_{\lambda}(A)$ with respect to $A$ as follows:
\begin{align}\label{eq:gradient_phi_A}
    & ~ \nabla \Phi_{\lambda}(A) \notag \\
    = & ~ \sum_{i=1}^m u_i u_i^\top \lambda \sinh\left( \lambda (u_i^\top A u_i - b_i)\right) \in \R^{n\times n}. 
\end{align}
 


We can compute the Hessian of $\Phi_{\lambda}(A)$ with respect to $A$ as follows
\begin{align*}
    & ~ \nabla^2 \Phi_{\lambda}(A) \notag \\
    = & ~ \sum_{i=1}^m ( u_i u_i^\top ) \otimes ( u_i u_i^\top ) \lambda^2 \cosh( \lambda (u_i^\top A u_i - b_i) ). 
\end{align*}
The Hessian $\nabla^2 \Phi_{\lambda}(A)\in \R^{n^2\times n^2}$ and $\otimes$ is the Kronecker product.

 



\begin{lemma}[Progress on entry-wise potential]\label{lem:gradient_descent}
Assume that $u_i \perp u_j = 0 $ for any $i,j \in [m]$ and $\| u_i\|^2 = 1$. Let $c \in (0,1)$ denote a sufficiently small positive constant. Then, for any $\epsilon,\lambda>0$ such that $\epsilon\lambda \leq c$,
 
we have for any $t>0$,
\begin{align*}
    \Phi_{\lambda} ( A_{t+1} ) \leq (1-0.9 \frac{ \lambda \epsilon }{\sqrt{m} }) \cdot \Phi_{\lambda} (A_t) +  \lambda \epsilon \sqrt{m}
\end{align*}

\end{lemma}
\begin{proof}
We defer the proof to Appendix~\ref{sec:proof_gradient_descent}
\end{proof}


\subsection{Technical Claims}
We prove some technical claims in below. 



\begin{claim}\label{cla:gd_Q1}
For $Q_1$ defined in Eq.~\eqref{eq:def_Q1}, we have
\begin{align*}
    Q_1 \leq \Big( \sqrt{m} + \frac{1}{\lambda} \| \nabla \Phi_{\lambda}(A_t) \|_F \Big) \cdot  \| \nabla \Phi_{\lambda} (A_t) \|_F^2.
\end{align*}
\end{claim}
\begin{proof}


For simplicity, we define $z_{t,i}$ to be
\begin{align*}
    z_{t,i} := \lambda ( u_i^\top A_t u_i - b_i ) .
\end{align*}
Recall that
\begin{align*}
    \nabla^2 \Phi_{\lambda}(A_t) = \lambda^2 \cdot \sum_{i=1}^m ( u_i u_i^\top ) \otimes ( u_i u_i^\top ) \cosh( z_{t,i} ) .
\end{align*}

For $Q_1$, we have
\begin{align}\label{eq:Q1}
   Q_1 = & ~  \tr[ \nabla^2 \Phi_{\lambda}(A_t) \sum_{i=1}^m  \sinh^2( z_{t,i} ) (u_i u_i^\top \otimes u_i u_i^\top) ) ] \notag\\
   = & ~ \lambda^2 \cdot \tr[ \sum_{i=1}^m  \cosh( z_{t,i} ) ( u_i u_i^\top ) \otimes ( u_i u_i^\top )    \cdot   \sum_{i=1}^m  \sinh^2( z_{t,i} ) (u_i u_i^\top ) \otimes ( u_i u_i^\top)   ] \notag\\
   = & ~ \lambda^2 \cdot \sum_{i=1}^m \tr[ \cosh( z_{t,i} )   \cdot \sinh^2( z_{t,i} )  ( u_i u_i^\top  u_i u_i^\top ) \otimes ( u_i u_i^\top  u_i u_i^\top ) ] \notag\\
   = & ~ \lambda^2 \cdot \sum_{i=1}^m  \cosh( z_{t,i} ) \sinh^2( z_{t,i} ) \notag \\
   \leq & ~ \lambda^2 \cdot (  \sum_{i=1}^m  \cosh^2( z_{t,i} ) )^{1/2}
   \cdot  ( \sum_{i=1}^m \sinh^4( z_{t,i} ) )^{1/2} \notag \\
   \leq & ~ \lambda^2 \cdot B_1 \cdot B_2,
\end{align}
where {  the first step comes from the definition of $Q_1$, the second step comes from the definition of $\nabla^2 \Phi_{\lambda}(A_t)$,}
the third step follows from $(A \otimes B) \cdot (C \otimes D) = (AC) \otimes (BD)$ and $u_i^\top u_j = 0$ , the fourth step comes from $\|u_i\| = 1$ and $\tr[ (u_i  u_i^\top) \otimes (u_i  u_i^\top) ] = 1$.

For the term $B_1$, we have
\begin{align*}
    B_1 = & ~ (  \sum_{i=1}^m \cosh^2( \lambda (u_i^\top A_t u_i - b_i ) ) )^{1/2} \\
    \leq & ~ \sqrt{m} + \frac{1}{\lambda} \| \nabla \Phi_{\lambda}(A_t) \|_F,
\end{align*}
where the second step follows Part 1 of Lemma~\ref{lem:property_sinh_cosh_scalar}.

For the term $B_2$, we have
\begin{align*}
    B_2 = & ~( \sum_{i=1}^m \sinh^4( \lambda( u_i^\top A_t u_i - b_i ) ) )^{1/2} \\
    \leq & ~ \frac{1}{\lambda^2} \| \nabla \Phi_{\lambda} (A_t) \|_F^2,
\end{align*}
where the second step follows from $\| x \|_4^2 \leq \| x \|_2^2$. This implies that
\begin{align*}
    Q_1 \leq & ~ \lambda^2 \cdot B_1 \cdot B_2 \\
    \leq & ~ \lambda^2 \cdot ( \sqrt{m} + \frac{1}{\lambda} \| \nabla \Phi_{\lambda}(A_t) \|_F ) \cdot \frac{1}{\lambda^2} \| \nabla \Phi_{\lambda} (A_t) \|_F^2 \\
    = & ~ ( \sqrt{m} + \frac{1}{\lambda} \| \nabla \Phi_{\lambda}(A_t) \|_F ) \cdot  \| \nabla \Phi_{\lambda} (A_t) \|_F^2 .
\end{align*}
\end{proof}


\begin{claim}\label{cla:gd_Q2}
For $Q_2$ defined in Eq.~\eqref{eq:def_Q2}, we have $Q_2 = 0$.
\end{claim}

\begin{proof}
Because in $Q_2$ we have :
\begin{align}\label{eq:u_ell_i_j_product}
    & ~\sum_{\ell=1}^{m}(u_{\ell} u_{\ell}^{\top} \otimes u_{\ell} u_{\ell}^{\top}) \sum_{i \neq j}(u_i u_i^{\top} \otimes u_j u_j^{\top}) \notag \\
    = & ~ \sum_{\ell=1}^{m} \sum_{i \neq j} (u_{\ell} u_{\ell}^{\top} u_i u_i^{\top}) \otimes (u_{\ell} u_{\ell}^{\top} u_j u_j^{\top}) \notag \\
    = & ~ 0, 
\end{align}
where the first step follows from $(A \otimes B) \cdot (C \otimes D) = (AC) \otimes (BD)$ , the second step follows that $u_i^{\top} u_j = 0$ if $i \neq j$ and $\ell \neq i$ or $\ell \neq j$ always holds in Eq.~\eqref{eq:u_ell_i_j_product}.

Therefore, we get that $Q_2 = 0$.
\end{proof}




\subsection{Convergence for multiple iterations}\label{sec:gd_analysis_2}

The goal of this section is to prove the convergence of Algorithm~\ref{alg:GD}:

\begin{lemma}[Convergence of gradient descent]\label{lem:gd_convergence}
Suppose the measurement vectors $\{u_i\}_{i\in [m]}$ are orthogonal unit vectors, and suppose $|b_i|$ is bounded by $R$ for $i\in [m]$.  Then, for any $\delta \in (0,1)$, if we take $\lambda = \Omega(\delta^{-1}\log m)$ and $\epsilon=O(\lambda^{-1})$ in Algorithm~\ref{alg:GD}, then for $T=\widetilde{\Omega}(\sqrt{m}R\delta^{-1})$ iterations, the solution matrix $A_T$ satisfies:
\begin{align*}
    | u_i^\top A_{T} u_i - b_i| \leq \delta~~~\forall i\in [m].
\end{align*}
\end{lemma}

\begin{proof}
We defer the proof to Appendix~\ref{sec:gd_convergence_proof}
\end{proof}

 


Theorem~\ref{thm:gd_orthogonal} follows immediately from Lemma~\ref{lem:gd_cost_per_iter} and Lemma~\ref{lem:gd_convergence}.
\section{Stochastic gradient descent}\label{sec:sgd}
In this section, we show that the cost-per-iteration of the approximate matrix sensing algorithm can be improved by using a stochastic gradient descent (SGD). More specifically, SGD can obtain a $\delta$-approximate solution with $O(Bn^2)$, where $0<B<m$ is the size of the mini batch in SGD. Below is the main theorem of this section:

\begin{theorem}[Stochastic gradient descent for orthogonal measurements]\label{thm:sgd_orthogonal}
Suppose $u_1,\dots,u_m\in \R^n$ are orthogonal unit vectors, and suppose $|b_i|\leq R$ for all $i\in [m]$. There exists an algorithm such that for any $\delta \in (0,1)$, performs $\wt{O}(m^{3/2}B^{-1}R\delta^{-1})$ iterations of gradient descent with $O(Bn^2)$-time per iteration and outputs a matrix $A\in \R^{n\times n}$ satisfies:
\begin{align*}
    | u_i^\top A u_i - b_i| \leq \delta~~~\forall i\in [m].
\end{align*}
\end{theorem}

The algorithm and its time complexity are provided in Section~\ref{sec:sgd_alg}. The convergence is proved in Section~\ref{sec:sgd_converge_1} and \ref{sec:sgd_converge_2}. The SGD algorithm for the general measurement without the assumption that the $\{u_i\}_{i\in [m]}$ are orthogonal vectors is deferred to Appendix~\ref{sec:sgd_general}.

\subsection{Algorithm}\label{sec:sgd_alg}
We can use the stochastic gradient descent algorithm (Algorithm~\ref{alg:stochastic_gradient_descent}) for matrix sensing. More specifically, in each iteration, we will uniformly sample a subset ${\cal B}\subset [m]$ of size $B$, and then compute the gradient of the stochastic potential function:
\begin{align}\label{eq:sgd_potential_func}
        \nabla \Phi_{\lambda}(A{ , {\cal B}}) := {  \frac{m}{|{\cal B}|}\sum_{i \in {\cal B}}} u_i u_i^\top \lambda \sinh( \lambda (u_i^\top A u_i - b_i) ), 
\end{align}
which is an $n$-by-$n$ matrix. Then, we do the following gradient step:
\begin{align}
    A_{t+1} \gets A_t - \epsilon \cdot \nabla \Phi_{\lambda}(A_t,{\cal B}_t) / \| \nabla \Phi_{\lambda}(A_t) \|_F.
\end{align}


\begin{lemma}[Running time of stochastic gradient descent]\label{lem:sgd_cost_per_iter}
Algorithm~\ref{alg:stochastic_gradient_descent} takes $O(mn^2)$-time for preprocessing and 
each iteration takes
$
    O(Bn^2)
$-time.
\end{lemma}

\begin{proof}
The time-consuming step is to compute $\|\nabla \Phi_\lambda(A_t)\|_F$. Since
\begin{align*}
    \nabla \Phi_{\lambda}(A_t) = \sum_{i=1}^m u_i u_i^\top \lambda \sinh\left( \lambda (u_i^\top A_t u_i - b_i)\right),
\end{align*}
and $u_i\bot u_j$ for $i\ne j\in [m]$, we know that $u_i$ is an eigenvector of $\nabla \Phi_{\lambda}(A)$ with eigenvalue $\lambda \sinh\left( \lambda (u_i^\top A_t u_i - b_i)\right)$ for each $i\in [m]$. Thus, we have
\begin{align*}
    \|\nabla \Phi_\lambda(A_t)\|_F^2 = &~ \sum_{i=1}^m \lambda^2 \sinh^2\left( \lambda (u_i^\top A_t u_i - b_i)\right)\\
    = &~ \sum_{i=1}^m \lambda^2 \sinh^2 (\lambda z_{t,i}),
\end{align*}
where $z_{t,i}:=u_i^\top A_t u_i - b_i$ for $i\in [m]$. Then, if we know ${z_{t,i}}_{i\in [m]}$, we can compute $\|\nabla \Phi_\lambda(A_t)\|_F$ in $O(m)$-time.

Consider the change $z_{t+1,i}-z_{t,i}$:
\begin{align*}
    &z_{t+1,i}-z_{t,i}\\
    = &~ u_i^\top (A_{t+1}-A_t)u_i\\
    = &~  -\frac{\epsilon}{\| \nabla \Phi_{\lambda}(A_t) \|_F}\cdot u_i^\top \nabla \Phi_{\lambda}(A_t,{\cal B} _t)u_i\\
    = &~ -\frac{\epsilon \lambda m }{\| \nabla \Phi_{\lambda}(A_t) \|_F B} \sum_{j\in {\cal B}_t}u_i^\top u_j u_j^\top u_i \cdot  \sinh(\lambda z_{t,j})\\
    = &~ -\frac{\epsilon \lambda m  \sinh(\lambda z_{t,i})}{\| \nabla \Phi_{\lambda}(A_t) \|_F B} \cdot {\bf 1}_{i\in {\cal B}_t},
\end{align*}
where the last step follows from $u_i\bot u_j$ for $i\ne j$.
Hence, if we have already computed $\{z_{t,i}\}_{i\in [m]}$ and $\|\nabla \Phi_\lambda(A_t)\|_F$, $\{z_{t+1,i}\}_{i\in [m]}$ can be obtained in $O(B)$-time.

Therefore, we preprocess $z_{1,i}=u_i^\top A_1u_i-b_i$ for all $i\in [m]$ in $O(mn^2)$-time. Then, in the $t$-th iteration ($t>0$), we first compute 
\begin{align*}
    \nabla \Phi_{\lambda} (A_t, {\cal B}_t) = \frac{m}{B}\sum_{i\in {\cal B}_t}u_iu_i^\top \lambda \sinh(\lambda z_{t,i})
\end{align*}
in $O(Bn^2)$-time. Next, we compute $\|\nabla \Phi_\lambda(A_t)\|_F$ using $z_{t,i}$ in $O(m)$-time. $A_{t+1}$ can be obtained in $O(n^2)$-time. Finally, we use $O(B)$-time to update $\{z_{t+1,i}\}_{i\in [m]}$.

Hence, the total running time per iteration is
\begin{align*}
    O(Bn^2 + m + n^2 + B) = O(Bn^2).
\end{align*}
\end{proof}

\begin{algorithm}[t]\caption{Matrix Sensing by Stochastic Gradient Descent.}\label{alg:stochastic_gradient_descent}
\begin{algorithmic}[1]
\Procedure{SGD}{$\{u_i,b_i\}_{i\in [m]}$} \Comment{Theorem~\ref{thm:sgd_orthogonal}}
    \State $\tau \gets \max_{i \in [m]} b_i $
    \State $A_1 \gets \tau \cdot I$
    \State $z_i\gets u_i^\top A_1 u_i - b_i$ for $i\in [m]$
    \For{$t = 1 \to T$}
        \State Sample ${\cal B}_t \subset [m]$ of size $B$ uniformly at random
        \State $\nabla \Phi_{\lambda} (A_t, {\cal B}_t) \gets \frac{m}{B} \sum_{i \in {\cal B}_t} u_i u_i^\top \lambda \sinh( \lambda z_{i} ) $
        \State $\| \nabla \Phi_{\lambda} (A_t) \|_F\gets \left(\sum_{i=1}^m \lambda^2 \sinh^2 (\lambda z_{i})\right)^{1/2}$
        \State $A_{t+1} \gets A_t - \epsilon \cdot \nabla \Phi_{\lambda}(A_t,{\cal B} _t) / \| \nabla \Phi_{\lambda}(A_t) \|_F$
        \For{$i\in {\cal B}_t$}
            \State \hspace{-2mm} $z_i\gets z_i - \epsilon  \lambda m \sinh(\lambda z_{i}) / (\| \nabla \Phi_{\lambda}(A_t) \|_F B)$
        \EndFor
    \EndFor
    \State \Return $A_{T+1}$
\EndProcedure
\end{algorithmic}
\end{algorithm}


\subsection{Analysis of One Iteration}\label{sec:sgd_converge_1}

Suppose $A \in \R^{n \times n}$. Let ${\cal B}_t$ be a uniformly random $B$-subset of $[m]$ at the $t$-th iteration, where $B$ is a parameter.
 


We can compute the gradient of $\Phi_{\lambda}(A{ , {\cal B}})$ with respect to $A$ as follows:
\begin{align*}
    & ~ \nabla \Phi_{\lambda}(A{ , {\cal B}}) \\
    = & ~ {  \frac{m}{|{\cal B}|}\sum_{i \in {\cal B}}} u_i u_i^\top \lambda \sinh( \lambda (u_i^\top A u_i - b_i) ), 
\end{align*}
where $\nabla \Phi_{\lambda}(A{ , {\cal B}})\in \R^{n\times n}$.


We can also compute the Hessian of $\Phi_{\lambda}(A{ , {\cal B}})$ with respect to $A$ as follows:
\begin{align*}
     & ~\nabla^2 \Phi_{\lambda}(A{ , {\cal B}}) \\
    = & ~ {  \frac{m}{|{\cal B}|}\sum_{i \in {\cal B}}} ( u_i u_i^\top ) \otimes ( u_i u_i^\top ) \lambda^2 \cosh( \lambda (u_i^\top A u_i - b_i) ) 
\end{align*}
where $\nabla^2 \Phi_{\lambda}(A{ , {\cal B}})\in \R^{n^2\times n^2}$ and $\otimes$ is the Kronecker product.


{It is easy to see the expectations of the gradient and Hessian of $\Phi_\lambda(A,{\cal B})$ over a random set ${\cal B}$:
\begin{align*}
    & ~\E_{{\cal B} \sim [m]}[\nabla \Phi_{\lambda}(A, {\cal B})] =    \nabla \Phi_{\lambda}(A), \\
    & ~\E_{{\cal B} \sim [m]}[\nabla^2 \Phi_{\lambda}(A, {\cal B})] =    \nabla^2 \Phi_{\lambda}(A)
\end{align*}
}

 


 
\begin{lemma}[Expected progress on potential]\label{lem:stochastic_gradient_descent}
Given $m$ vectors $u_1, u_2, \cdots, u_m \in \R^n$. 
Assume $\langle u_i , u_j \rangle = 0 $ for any $i \neq j \in [m]$ and $\| u_i\|^2 = 1$, for all $i \in [m]$.
Let $\epsilon \lambda \leq 0.01 \frac{|{\cal B}_t|}{m}$, for all $t>0$.

Then, we have
\begin{align*}
    \E[\Phi_{\lambda} ( A_{t+1} )] \leq (1-0.9 \frac{ \lambda \epsilon }{\sqrt{m} }) \cdot { \Phi_{\lambda} (A_t )} +  \lambda \epsilon \sqrt{m}.
\end{align*}
 
\end{lemma}




\begin{proof}
We first express the expectation as follows:
\begin{align}\label{eq:expectation_phi_update}
    & ~ \E_{A_{t+1}}[\Phi_{\lambda}(A_{t+1})] - \Phi_{\lambda} (A_t)\notag \\
    \leq & ~ \E_{A_{t+1}}[ \langle  \nabla \Phi_{\lambda} (A_t) ,  (A_{t+1} - A_t) \rangle ] \notag \\
    + & ~ O(1) \cdot \E_{A_{t+1}}[ \langle  \nabla^2 \Phi_{\lambda}(A_t) , (A_{t+1} - A_t) \otimes (A_{t+1} - A_t) \rangle ],
\end{align}
which follows from Corollary~\ref{cor:matrix_der_trace}.

 
We choose
\begin{align*}
    A_{t+1} = A_t - \epsilon \cdot \nabla \Phi_{\lambda}(A_t{ , {\cal B}_{t}}) / \| \nabla \Phi_{\lambda}(A_t) \|_F.
\end{align*}
Then, we can bound
\begin{align}\label{eq:sgd_tr_phi_At_first_moment} 
 & ~ \E_{A_{t+1}} [ -\tr [ \nabla \Phi_{\lambda} (A_t) \cdot (A_{t+1} - A_t) ] ] \notag \\
= & ~ \E_{{\cal B}_t}\Big[ \tr\Big[ \nabla \Phi_{\lambda}(A_t) \cdot \frac{\epsilon \nabla \Phi_{\lambda} (A_t,{\cal B}_t) }{ \| \nabla \Phi_{\lambda}(A_t) \|_F } \Big] \Big] \notag \\
= & ~ \epsilon \cdot \| \nabla \Phi_{\lambda} (A_t)  \|_F 
\end{align}

We define for $t>0$ and $i\in [m]$, 
\begin{align*}
    z_{t,i} := u_i^\top A_{t} u_i - b_i.
\end{align*}


We need to compute this $\Delta_2$. For simplificity, we consider $\Delta_2 \cdot \| \nabla \Phi_{\lambda} (A_t) \|_F^2$,
\begin{align}\label{eq:sgd_tr_phi_At_second_moment}
     = & ~  \tr[ \nabla^2 \Phi_{\lambda}(A_t) \cdot (A_{t+1} - A_t) \otimes (A_{t+1} - A_t) ] \cdot  \| \nabla \Phi_{\lambda}(A_t) \|_F^2 \notag\\
    = & ~ (\lambda \epsilon)^2 \cdot (\frac{m}{|{\cal B}_t|})^2  \cdot   \tr\Big[ \nabla^2 \Phi_{\lambda}(A_t) \cdot %\notag \\ 
     ( \sum_{i\in {\cal B}_t} u_i u_i^\top  \sinh( z_{t,i} )    \otimes ( \sum_{i \in {\cal B}_t} u_i u_i^\top \sinh( z_{t,i} ) \Big].
\end{align}
Ignoring the scalar factor in the above equation, we have
\begin{align}
    = &~    \tr\Big[ \nabla^2 \Phi_{\lambda}(A_t) \cdot  ( \sum_{i,j \in B_t}  \sinh( z_{t,i} )\sinh( z_{t,i} ) \cdot   (u_i u_i^\top \otimes u_ju_j^\top) ) \Big] \notag\\
    = &    \tr\Big[ \nabla^2 \Phi_{\lambda}(A_t) \cdot ( \sum_{i \in B_t} \sinh^2( z_{t,i }) (u_i u_i^\top \otimes u_iu_i^\top) ) \Big] \notag\\
    + &    \tr\Big[ \nabla^2 \Phi_{\lambda}(A_t) \cdot ( \sum_{i\neq j \in B_t} \sinh( z_{t,i} )\sinh( z_{t,i} ) \cdot    (u_i u_i^\top \otimes u_j u_j^\top) ) \Big] \notag\\
    =: & ~    \wt{Q}_1 + \wt{Q}_2  ,
\end{align}
where the first step follows that we extract the scalar values from Kronecker product,
the second step comes from splitting into two partitions based on whether $i = j$, the third step comes from the definition of $\wt{Q}_1 $ and $\wt{Q}_2$ where $\wt{Q}_1$ denotes the diagonal term, and $\wt{Q}_2$ denotes the off-diagonal term.
Taking expectation, we have 
\begin{align}\label{eq:expectation_Delta_2}
    & ~ \E[ \Delta_2 \cdot \| \nabla \Phi_{\lambda} (A_t) \|_F^2 ] \notag\\
    = & ~ (\lambda \epsilon)^2 \cdot ( \frac{m}{|{\cal B}_t|} )^2 \E[ \wt{Q}_1] \notag\\
    = & ~ (\lambda \epsilon)^2 \cdot ( \frac{m}{|{\cal B}_t|} )^2 \cdot \frac{|{\cal B}_t|}{m} \cdot Q_1 \notag\\
    \leq & ~ (\lambda \epsilon)^2 \cdot \frac{m}{|{\cal B}_t|} 
      \cdot ( \sqrt{m} + \frac{1}{\lambda} \| \nabla \Phi_{\lambda}(A_t) \|_F ) \cdot \| \nabla \Phi_{\lambda} (A_t) \|_F^2
\end{align}
where the first step comes from extracting the constant terms from the expectation and Claim~\ref{cla:gd_Q2},
the second step follows that $\E[\wt{Q}_1] = \frac{|{\cal B}_t|}{m} \cdot Q_1$,
and the third step comes from the Claim~\ref{cla:gd_Q1}.
Therefore, we have:
\begin{align*}
    & ~ \E[ \Phi_{\lambda} (A_{t+1}) ] - \Phi_{\lambda} (A_t) \\
    \leq &  - \E[ \Delta_1 ]   + O(1) \cdot \E[ \Delta_2 ] \\ 
    \leq &  - \epsilon (1 - O(\epsilon \lambda) \cdot \frac{m}{|{\cal B}_t|} ) \| \nabla \Phi_{\lambda}(A_t) \|_F+ O(\epsilon \lambda)^2 \sqrt{m} \\
    \leq &  - 0.9 \epsilon  \| \nabla \Phi_{\lambda}(A_t) \|_F+ 
    O(\epsilon \lambda)^2 \sqrt{m} \\
    \leq &  -0.9 \epsilon \lambda \frac{1}{\sqrt{m}} ( \Phi_{\lambda}(A_t) - m ) +O(\epsilon \lambda)^2 \sqrt{m} \\
    \leq &   -0.9 \epsilon \lambda \frac{1}{\sqrt{m}} \Phi_{\lambda}(A_t) + \epsilon \lambda \sqrt{m},
\end{align*}
where the first step comes from Eq.~\eqref{eq:expectation_phi_update},
the second step comes from Eq.~\eqref{eq:sgd_tr_phi_At_first_moment} and Eq.~\eqref{eq:expectation_Delta_2},
the third step follows from $\epsilon \leq 0.01 \frac{|{\cal B}_t|}{\lambda m}$,
the forth step follows from Eq.~\eqref{eq:phi_At_F_norm},
and the last step follows from $\epsilon \lambda \in (0, 0.01)$.
\end{proof}

\subsection{Convergence for multiple iterations}\label{sec:sgd_converge_2}
The goal of this section is to prove the convergence of Algorithm~\ref{alg:stochastic_gradient_descent}. 
\begin{lemma}[Convergence of stochastic gradient descent]\label{lem:sgd_convergence}
Suppose the measurement vectors $\{u_i\}_{i\in [m]}$ are orthogonal unit vectors, and suppose $|b_i|$ is bounded by $R$ for $i\in [m]$. Then, for any $\delta \in (0,1)$, if we take $\lambda = \Omega(\delta^{-1}\log m)$ and $\epsilon=O(\lambda^{-1}m^{-1}B)$ in Algorithm~\ref{alg:stochastic_gradient_descent}, then for 
\begin{align*} 
T=\widetilde{\Omega}(m^{3/2}B^{-1}R\delta^{-1})
\end{align*}
iterations, with high probability, the solution matrix $A_T$ satisfies:
\begin{align*}
    | u_i^\top A_{T+1} u_i - b_i| \leq \delta~~~\forall i\in [m].
\end{align*}
\end{lemma}

\begin{proof}
Similar to the proof of Lemma~\ref{lem:gd_convergence}, we can bound the initial potential by:
\begin{align*}
    \Phi(A_1) \leq 2^{O(\lambda R)}.
\end{align*}

In the following iterations, by Lemma~\ref{lem:stochastic_gradient_descent}, we have
\begin{align*}
    \E[\Phi_{\lambda} ( A_{t+1} )] \leq (1-0.9 \frac{ \lambda \epsilon }{\sqrt{m} }) \cdot { \Phi_{\lambda} (A_t )} +  \lambda \epsilon \sqrt{m},
\end{align*}
as long as $\epsilon \leq 0.01\frac{|B_t|}{\lambda m}$, where $B_t$ is a uniformly random subset of $[m]$ of size $B$. 


It suffices to take $\epsilon = O(\lambda^{-1} m^{-1}B)$.

Now, we can apply Lemma~\ref{lem:stochastic_gradient_descent} for $T$ times and get that
\begin{align*}
    \E[\Phi(A_{T+1})]
    \leq 2^{-\Omega( T  \epsilon \lambda / \sqrt{m} ) + O(\lambda R)} + 2  m.
\end{align*}
By taking 
\begin{align*}
T=\widetilde{\Omega}(m^{3/2}B^{-1}R\delta^{-1}),
\end{align*}
we have
\begin{align*}
    \Phi(A_{T+1}) \leq O(m)
\end{align*}
holds with high probability. By the same argument as in the proof of Lemma~\ref{lem:gd_convergence}, we have
\begin{align*}
    | u_i^\top A_{T+1} u_i - b_i| \leq \delta~\forall i\in [m].
\end{align*}
The lemma is thus proved.
\end{proof}
%\section{Conclusion}\label{sec:conclusion}
In this work, we focus on addressing the fundamental challenge of OOD detection tasks, which is how to fully understand the semantic discrepancy between the ID/OOD samples. We reveal that the key to success in the realistic SCOOD task is to allocate as many ID samples in the unlabeled set correctly as possible. To this end, we propose a novel uncertainty-aware optimal transport scheme that introduces class-specific energy scores as guidance for effective label assignment. Experimental results show that our method achieves better performance than previous state-of-the-art methods on SCOOD benchmarks.

\textbf{Limitations.} In addition to temperature scaling, other techniques such as feature clipping applied in ReAct~\cite{sun2021react} also enhance the performance of energy score, so how to obtain an OOD score that best fits the SCOOD task can be further explored. Moreover, a setting highly related to SCOOD has been proposed in \cite{katz2022training} and formulated as a constrained optimization problem. We will also theoretically analyze these practical OOD settings in our feature work.

% \section*{Acknowledgments}
\textbf{Acknowledgments.} 
This work is supported by National Key R\&D Program of China under Grant 2020AAA0105701, National Natural Science Foundation of China (NSFC) under Grants 61872327, Major Special Science and Technology Project of Anhui, National Natural Science Foundation of China (62033012) and Ant Group through Ant Research Intern Program.

\ifdefined\isarxiv
%\section*{Acknowledgments}
\bibliographystyle{alpha}
\bibliography{ref}
\else
% \small
\bibliography{ref}
\bibliographystyle{icml2023}
%%%Zhao: This is ML First author last name et al style,
% \bibliographystyle{plainnat}
%%%Zhao: This is number style
%\bibstyle {plain} %%% Zhao: For this paper, let's use this one
%%%Zhao: This is TCS ABC+12 style
% \bibliographystyle{alpha}
\fi


% In Section \ref{sec:core-idea}, we claimed the importance of the correctness of the code and we pointed out that this aspect is not currently evaluated in the peer-reviewing process 
%and that the correctness of a paper 
or
is only judged on the basis of the results shown in the paper. In Section \ref{sec:case-study}, we proved through extensive experiments that the results are not a valid measure of code correctness, invoking the need for ad-hoc guidelines aimed at improving this aspect.
%paying more attention to this important but neglected aspect.
%introduction of ad-hoc question(s) aimed at evaluating this important but neglected aspect.
%invoking ad-hoc guidelines aimed at improving this aspect and 
%encouraging the community to consider this aspect in the evaluation of a paper. 
In this section, we introduce the Correctness Checklist (Table \ref{tab:checklist}), a list of several recommendations that we encourage the authors to check before developing their codebases and declare when submitting their scientific artifacts. 

\begin{table}[!tb]
\begin{tcbitemize}[%
        raster columns=1,
        raster equal height,
        before=,after=\hfill,
        boxsep=3pt, left=10pt,   right=10pt,
        colframe=teal!75!black,colback=white,
        fonttitle=\large\bfseries,
        halign=left
        ]
\tcbitem[title=Correctness Checklist]
\begin{itemize}
\small
\justifying
\item[\checkinbox] The authors made use of Unit Tests to write their code (i.e., the codebase was isolated in individual units which were tested to show their correctness).
\item[\checkinbox] The codebase underwent a code reviewing process in which one or several people (who are not the author of the code) checked the program mainly by viewing and reading parts of its source code to ensure its correctness.
\item[\checkinbox] The documentation is sufficient to understand the code and follows documentation standards.
\item[\checkinbox] The repository lists all the dependencies needed to correctly install the software and run the experiments.
%\item[\checkinbox] More than one batch size was tested to check the consistency among the obtained results (i.e., the results do not vary when the batch size varies). 
\end{itemize}
%\raggedright\textbf{Browse All Courses in Programming}
\end{tcbitemize}
\caption{Correcntess Checklist.}
\label{tab:checklist}
\end{table}

One of the best practices we strongly suggest adopting to improve the code \textit{traceability} is the use of Unit Tests or UTs \citep{10.5555/1349795,8048665} to check every portion of it, if not first writing the tests and then the actual code, as per the popular test-driven development (TDD) practice \citep{beck2002driven}.
%For example, more than one batch size can be tested when developing a code of a new architecture to check the consistency among the obtained results since they do have not to vary when the batch size varies.
For example, testing multiple batch sizes when developing a new architecture code ensures consistency in its implementation as the results should not vary with batch size changes.
Although researchers may initially perceive that adding UTs is an additional and undesirable cost, which sums to all the other activities that have to be carried out, this feeling is exaggerated and the actual overhead is lower \citep{oro3667,Ellims2006}. \citet{Williams-2003-TDD} proved that writing UTs does not impact the productivity in writing code, as the initial overhead\footnote{Estimated in 16\%-35\% of the overall software development cost \citep{George-2003-tdd,google-ut}.} pays back by saving time spent in (unsuccessful) manual experiments \citep{google-ut}. In addition, since manual experiments often involve costly hardware with a significant environmental footprint in NLP \citep{strubell-etal-2019-energy}, UTs may reduce the computing and environmental costs of NLP research.
%
Another best practice that we promote when possible is the code-reviewing process \citep{7589787} to have external (with respect to the author(s) of the code) checks aimed not only at avoiding bugs but also at improving the code readability and documentation\footnote{Example of documentation standards can be found at \url{https://help.github.com/articles/about-readmes/}.} to make it more \textit{consistent} and  reusable \citep{Chen2019,Trisovic2022}. Along this line, we recommend listing all the dependencies of the released repository to allow a correct installation \citep{bestpractices} and improve code \textit{completeness}.

Ultimately, we hope that correctness will be an aspect to be sensitized by conferences and journals in the future by integrating this aspect into the review forms by explicitly asking the reviewers to answer/give a score/give a judgment on it but also by encouraging the authors to compile the Correctness Checklist during their submissions.


%On these aspects, plenty of literature already discussed the need to go beyond code openness \citep{Chen2019,Trisovic2022}, highlighting the role of proper documentation and testing in different environments. We reiterate the benefits brought by improving code readability and documentation of research repositories. While there is currently no incentive to do so \citep{barba-2019-praxis}, an ethical commitment toward more reusable code brings benefits to the whole community and increases  the reproducibility of the works. Similarly, they increase \textit{maintainability}, which measures the effort needed to make specific modifications to the code. This eases the code reuse by other researchers, as well as it eases building future works on the same codebase.


%Although the community showing interest in discussing the ways in which reported performance improvements on NLP benchmarks are meaningful, as demonstrated by the last years' theme tracks of conferences like EMNLP\footnote{\url{https://2022.emnlp.org/calls/main_conference_papers/\#emnlp-2022-theme-track}} and ACL\footnote{\url{https://2023.aclweb.org/calls/main_conference/\#theme-track-reality-check}}, little efforts have been devoted to push towards improved technical correctness of the published scientific artifacts.

%We hope that the Checklist will be also useful as a guideline for better software.

%Therefore, we suggest the authors to 
%and  the Correctness Checklist which we suggest being checked before submitting papers containing scientific artifacts. The guidelines are presented in Table \ref{tab:checklist}.




% \newpage
%\onecolumn
% {\color{white}.}
% \newpage
\appendix
\onecolumn
\section*{Appendix}
\paragraph{Roadmap.} We first provide the proofs for matrix hyperbolic functions and properties of $\sinh$ and $\cosh$ in Appendix~\ref{sec:preli_app}. Then we provide the  proofs for the gradient descent and stochastic gradient descent convergence analysis in Appendix~\ref{sec:gd_sgd_missing_proofs}. We consider the spectral potential function with ground-truth oracle scenario in Appendix~\ref{sec:potential_ground_truth_app}. We analyze the gradient descent with non-orthogonal measurements in Appendix~\ref{sec:gd_non_orthogonal}. We provide the cost-per-iteration analysis for stochastic gradient descent under non-orthogonal measurements in Appendix~\ref{sec:sgd_general}.

\section{Proofs of Preliminary Lemmas}\label{sec:preli_app}


\subsection{Calculus tools}
We state a useful calculus tool from prior work,
\begin{lemma}[Proposition 3.1 in \cite{jn08}]\label{lem:jn08}
Let $\Delta$ be an open interval on the axis, and $f$ be $C^2$ function on $\Delta$ such that for certain $\theta_{\pm}, \mu_{\pm} \in \R$ one has
\begin{align*}
    & ~ \forall (a<b, a,b \in \Delta) : \\
    & ~ \theta_- \cdot \frac{ f''(a) + f''(b) }{2} + \mu_- \leq \frac{f'(b) - f'(a)}{b-a} \\
    & ~ \frac{f'(b) - f'(a)}{b-a} \leq \theta_+ \cdot \frac{f''(a) + f''(b)}{2} + \mu_+,
\end{align*}
where $f'$ and $f''$ means the first- and second-order derivatives of $f$, respectively.

Let, further, ${\cal X}_n (\Delta)$ be the set of all $n \times n$ symmetric matrices with eigenvalues belonging to $\Delta$. Then ${\cal X}_n(\Delta)$ is an open convex set in the space $S^n$ of $n \times n$ symmetric matrices, the function
\begin{align*}
    F(X) = \tr [ f(X) ] : {\cal X}_n (\Delta) \rightarrow \R  
\end{align*}
is $C^2$, and for every $X \in {\cal X}_n (\Delta)$ and every $H \in S^n$ one has
\begin{align*}
    & ~ \theta_- \cdot \tr[ H f''(X) H ] + \mu_- \cdot \tr[ H^2 ] \leq D^2 F(X) [H,H] \\
   & ~ D^2 F(X) [H,H] \leq \theta_+ \cdot \tr[ H f''(X) H ] + \mu_+ \cdot \tr[H^2],
\end{align*}
where $D$ means directional derivative.
\end{lemma}

% \Danyang{Lianke, need a sentence here to smooth the logic flow.}
We will use below corollary to compute the trace with a map $f: \R\rightarrow \R $.
\begin{corollary}\label{cor:matrix_der_trace}
Let $f:\R\rightarrow \R$ be a $C^2$ function. Let $A$ and $B$ be two symmetric matrices. We have
\begin{align*}
    \tr[ f(A) ] 
   \leq & ~ \tr[ f(B) ] + \tr[ f'(B) ( A - B ) ] \\
    & ~ + O(1) \cdot \tr[ f''(B) ( A - B )^2 ]. 
\end{align*}
\end{corollary}
%\Ruizhe{check if this corollary is correct.}


% \subsection{Proof of Fact~\ref{fac:kronecker_product}}
\subsection{Kronecker product}
Suppose we have two matrice $A \in \R^{m \times n}$ and $B \in \R^{p \times q}$, we use $A \otimes B$ denote the Kronecker product:
\begin{align*}
    A \otimes B = \left[\begin{array}{ccc}
A_{1,1} {B} & \cdots & A_{1,n} {B} \\
\vdots & \ddots & \vdots \\
A_{m,1} {B} & \cdots & A_{m,n} {B}
\end{array}\right].
\end{align*}

We state a fact and delay the proof into Section~\ref{sec:preli_app}.
\begin{fact}\label{fac:kronecker_product}
Suppose we have two matrices $A \in \R^{m \times n}$ and $B \in \R^{n \times k}$, we have
\begin{align*}
    (A \otimes B) \cdot (C \otimes D) = (AC) \otimes (BD).
\end{align*}
\end{fact}

\begin{proof}
From the definition of Kronecker product we have:
\begin{align*}
    &~ (A \otimes B) \cdot (C \otimes D) \\
    = &~ {\left[\begin{array}{ccc}
A_{1,1} B & \ldots & A_{1,n} B \\
\vdots & \ddots & \vdots \\
A_{m,1} B & \ldots & A_{m,n} B
\end{array}\right]\left[\begin{array}{ccc}
C_{1,1} D & \ldots & C_{1,k} D \\
\vdots & \ddots & \vdots \\
C_{n,1} D & \ldots & C_{n,k} D
\end{array}\right] } \\
   = &~ \left[\begin{array}{ccc}
(\sum_{i=1}^{n}A_{1,i}C_{i,1}) {BD} & \cdots & (\sum_{i=1}^{n}A_{1,i}C_{i,k}) {BD} \\
\vdots & \ddots & \vdots \\
(\sum_{i=1}^{n}A_{m,i}C_{i,1}) {BD} & \cdots & (\sum_{i=1}^{n}A_{m,i}C_{i,k}){BD}
\end{array}\right] \\
= &~ \left[\begin{array}{ccc}
(AC)_{1,1} BD & \cdots & (AC)_{1,k} BD\\
\vdots & \ddots & \vdots \\
(AC)_{m,1} BD & \cdots & (AC)_{m,k}BD
\end{array}\right] \\
= &~ (AC) \otimes (BD)
\end{align*}
Thus we complete the proof.
\end{proof}


\subsection{Proof of \texorpdfstring{$\cosh(A)$}{} upper bound}\label{sec:cosh_bound_proof}

\begin{lemma}[Restatement of Lemma~\ref{lem:cosh_bound}]\label{lem:cosh_bound_app}
Let $A$ be a real symmetric matrix, then we have
\begin{align*}
    \|\cosh(A)\| = \cosh(\|A\|) \leq \tr[\cosh(A)].
\end{align*}
We also have $\|A\| \leq 1+\log(\tr[\cosh(A)])$.
\end{lemma}


\begin{proof}
Note that for each eigenvalue $\lambda$ of $A$, we know that it corresponds to $\cosh(\lambda)$ for $\cosh(A)$. The second inequality follows from the fact that $\cosh(A)$ is psd.

For the second part, we know that $\exp(x)/2\leq \cosh(x)$, hence, $\exp(\|A\|)/2\leq \cosh(\|A\|)$, and
\begin{align*}
    \|A\| = & ~ \log(\exp(\|A\|)) \\
    \leq & ~ \log(2\cosh(\|A\|)) \\
    \leq & ~ 1+\log(\tr[\cosh(A)]),
\end{align*}
where the second step is by the monotonicity of $\log(\cdot)$ and $\exp(\|A\|)\leq 2\cosh(\|A\|)$, the last step is by $\cosh(\|A\|)\leq \tr[\cosh(A)]$.
\end{proof}
 
We state a fact as follows:  
\begin{fact}\label{fact:cosh_sinh_1}
For any real number $x$, $ \cosh^2(x) - \sinh^2(x) =1$
\end{fact}
From the definition of $\cosh(x)$ and $\sinh(x)$ we have:
\begin{align*}
      &~ \cosh^2(x) - \sinh^2(x) \\
    = &~ \frac{1}{4}(e^{2x} + 2 + e^{-2x}) - \frac{1}{4}(e^{2x} - 2 + e^{-2x}) \\
    = &~ 1
\end{align*}

 
We also have the following lemma for matrix. 
\begin{lemma}\label{lem:cosh_sinh}
Let $A$ be a real symmetric matrix, then we have
\begin{align*}
    \cosh^2(A) - \sinh^2(A) = & ~ I.
\end{align*}
\end{lemma}
\begin{proof}
Since $A$ is real symmetric, we write it in the eigendecomposition form: $A=U\Lambda U^\top$, then
\begin{align*}
    & ~ \cosh^2(A)-\sinh^2(A) \\
    = & ~ U\cosh^2(\Lambda)U^\top-U\sinh^2(\Lambda)U^\top \\
    = & ~ U(\cosh^2(\Lambda)-\sinh^2(\Lambda))U^\top \\
    = & ~ UU^\top \\
    = & ~ I,
\end{align*}
where the first step follows from $\cosh$ and $\sinh$ can be expressed as $\exp$, the third step is by applying entrywise the identity $\cosh^2(x)-\sinh^2(x)=1$.
\end{proof}

 
 

\section{Proofs of GD and SGD convergence}\label{sec:gd_sgd_missing_proofs}
In this section, we provide proofs of convergence analysis the gradient descent and stochastic gradient descent matrix sensing algorithms.
\subsection{Proof of GD Progress on Potential Function}\label{sec:proof_gradient_descent}
We start with the progress of the gradient on the potential function in below lemma.
\begin{lemma}[Restatement of Lemma~\ref{lem:gradient_descent}]
Assume that $u_i \perp u_j = 0 $ for any $i,j \in [m]$ and $\| u_i\|^2 = 1$. Let $c \in (0,1)$ denote a sufficiently small positive constant. Then, for any $\epsilon,\lambda>0$ such that $\epsilon\lambda \leq c$,


we have for any $t>0$,
\begin{align*}
    \Phi_{\lambda} ( A_{t+1} ) \leq (1-0.9 \frac{ \lambda \epsilon }{\sqrt{m} }) \cdot \Phi_{\lambda} (A_t) +  \lambda \epsilon \sqrt{m}
\end{align*}

\end{lemma}

\begin{proof}
We first Taylor expand $\Phi_\lambda(A_{t+1})$ as follows:
\begin{align}\label{eq:gd_define_Delta_1_and_Delta_2}
    & ~ \Phi_{\lambda}(A_{t+1})- \Phi_{\lambda} (A_t) \notag \\
    \leq & ~ \langle \nabla \Phi_{\lambda} (A_t) , (A_{t+1} - A_t) \rangle  + O(1) \langle \nabla^2 \Phi_{\lambda}(A_t), (A_{t+1} - A_t) \otimes (A_{t+1} - A_t) \rangle \notag \\
    := & ~ \Delta_1 + O(1) \cdot \Delta_2,
\end{align}
which follows from Lemma~\ref{lem:jn08}.
\iffalse
Lemma~\ref{lem:jn08} that for any function $f$,
\begin{align*}
    \tr[ f( M'') ] 
   \leq & ~ \tr[ f(M') ] + \tr[ f'(M') ( M'' - M' ) ] \\
    & ~ + O(1) \cdot \tr[ f''(M') ( M'' - M' )^2 ]. 
\end{align*}
\Ruizhe{add the constraint for $f$, i.e., $C^2$?}
\fi

We choose
\begin{align*}
    A_{t+1} = A_t - \epsilon \cdot \nabla \Phi_{\lambda}(A_t) / \| \nabla \Phi_{\lambda}(A_t) \|_F.
\end{align*}

We can bound
\begin{align}\label{eq:tr_phi_At_first_moment} 
 \Delta_1 = & ~ \tr[\nabla \Phi_{\lambda}(A_t) (A_{t+1} - A_t)] \notag \\
=  & ~ -\epsilon \cdot \|\nabla \Phi_{\lambda} (A_t) \|_F.
\end{align}





Next, we upper-bound $\Delta_2$. Define
\begin{align*}
    z_{t,i} := \lambda (u_i^\top A_t u_i - b_i).
\end{align*}
and consider $\Delta_2 \cdot (\lambda\epsilon )^{-2} \cdot \| \nabla \Phi_{\lambda} (A_t) \|_F^2$, which can be expressed as:
\begin{align}\label{eq:tr_phi_At_second_moment}
&\Delta_2 \cdot (\lambda\epsilon )^{-2} \cdot \| \nabla \Phi_{\lambda} (A_t) \|_F^2\notag\\
     = & ~ (\lambda\epsilon )^{-2} \tr[ \nabla^2 \Phi_{\lambda}(A_t) \cdot (A_{t+1} - A_t) \otimes (A_{t+1} - A_t) ] \cdot 
      \| \nabla \Phi_{\lambda}(A_t) \|_F^2 \notag\\
    = & ~  \tr\Big[ \nabla^2 \Phi_{\lambda}(A_t) \cdot 
    ( \sum_{i=1}^m u_i u_i^\top  \sinh( z_{t,i} ) ) \otimes ( \sum_{i=1}^m u_i u_i^\top \sinh( z_{t,i} ) )\Big] \notag\\
    = & ~ \tr\Big[ \nabla^2 \Phi_{\lambda}(A_t) \cdot ( \sum_{i,j}  \sinh( z_{t,i} )\sinh( z_{t,i} ) (u_i u_i^\top \otimes u_ju_j^\top) ) \Big] \notag\\
    = & ~ \tr\Big[ \nabla^2 \Phi_{\lambda}(A_t) \cdot ( \sum_{i=1}^m \sinh^2( z_{t,i} )) (u_i u_i^\top \otimes u_iu_i^\top) ) \Big] \notag\\
    + & ~  \tr\Big[ \nabla^2 \Phi_{\lambda}(A_t) \cdot ( \sum_{i\neq j} \sinh( z_{t,i} )\sinh( z_{t,j} ) (u_i u_i^\top \otimes u_j u_j^\top) ) \Big] \notag\\
    =: & ~  Q_1 + Q_2  ,
\end{align}
where 
\begin{align}\label{eq:def_Q1}
Q_1:= \tr\Big[ \nabla^2 \Phi_{\lambda}(A_t) \cdot ( \sum_{i=1}^m \sinh^2( z_{t,i} )) (u_i u_i^\top \otimes u_iu_i^\top) ) \Big]
\end{align}
denotes the diagonal term, and
\begin{align}\label{eq:def_Q2}
    Q_2:=&~\tr\Big[ \nabla^2 \Phi_{\lambda}(A_t) \cdot
    ( \sum_{i\neq j} \sinh( z_{t,i} )\sinh( z_{t,j} ) (u_i u_i^\top \otimes u_j u_j^\top) ) \Big]
\end{align}
denotes the off-diagonal term. 
The first step comes from the definition of $\Delta_2$,
the second step follows fromr eplacing $A_{t+1} - A_t$ using Eq~\eqref{eq:A_t1_update}, the third step follows that we extract the scalar values from Kronecker product, the fourth step comes from splitting into two partitions based on whether $i = j$, the fifth step comes from the definition of $Q_1$ and $Q_2$.




Thus,
\begin{align}\label{eq:bound_gd_Delta_2}
    \Delta_2 = & ~ (\epsilon \lambda)^2 (Q_1 + Q_2) / \| \nabla \Phi_{\lambda}(A_t) \|_F^2 \notag \\
    = & ~ (\epsilon \lambda)^2 (Q_1 + 0) / \| \nabla \Phi_{\lambda}(A_t) \|_F^2 \notag \\
    = & ~ (\epsilon \lambda )^2 \cdot ( \sqrt{m} + \frac{1}{\lambda} \| \nabla \Phi_{\lambda}(A_t) \|_F ).
\end{align}
where the second step follows from Claim~\ref{cla:gd_Q2}, and the third step follows from Claim~\ref{cla:gd_Q1}.

Hence, we have
\begin{align*}
    & ~ \Phi_{\lambda} (A_{t+1}) - \Phi_{\lambda} (A_t) \\
    \leq & ~  \Delta_1    + O(1) \cdot \Delta_2 \\
    \leq & ~ - \epsilon \| \nabla \Phi_{\lambda}(A_t) \|_F   + O(1) (\epsilon \lambda)^2 (\sqrt{m} + \frac{1}{\lambda} \| \nabla \Phi_{\lambda}(A_t) \|_F ) \\
    \leq & ~ - 0.9 \epsilon \| \Phi_{\lambda}(A_t) \|_F+ O(\epsilon \lambda)^2 \sqrt{m} 
\end{align*}
where the first step follows from Eq.~\eqref{eq:gd_define_Delta_1_and_Delta_2}, the second step follows from Eq.~\eqref{eq:bound_gd_Delta_1} and Eq.~\eqref{eq:bound_gd_Delta_2},  the third step follows from $\epsilon \lambda \in (0, 0.01)$.

For $\| \Phi_{\lambda} (A_t) \|_F$, we have
\begin{align}\label{eq:phi_At_F_norm}
& ~ \frac{1}{\lambda^2} \| \nabla \Phi_{\lambda} (A_t) \|_F^2 \notag \\
= & ~   \tr[ (\sum_{i=1}^m u_i u_i^\top \sinh( \lambda( u_i^\top A_t u_i - b_i ) )  )^2 ] \notag\\
= & ~   \tr[ \sum_{i=1}^m (u_i u_i^\top)^{2} \sinh^2 ( \lambda( u_i^\top A_t u_i - b_i ) )  ] \notag\\ 
= & ~ \sum_{i=1}^m \sinh^2 ( \lambda ( u_i^\top A_t u_i - b_i ) ) \notag \\ 
\geq & ~ \frac{1}{m} ( \sum_{i=1}^m \cosh ( \lambda ( u_i^\top A_t u_i - b_i ) ) - m )^2 \notag \\
= & ~ \frac{1}{m} ( \Phi_{\lambda} (A_t) - m )^2,
\end{align}
where the first step comes from Eq.~\eqref{eq:gradient_phi_A},
the second  steps follow from $u_i^\top u_j = 0$, the third step follows from $\| u_i \|_2 = 1$, the forth step follows from Part 2 in Lemma~\ref{lem:property_sinh_cosh_scalar}, {  the fifth step follows from the definition of $\Phi_{\lambda}(A)$.}

Thus, we get that
\begin{align}\label{eq:bound_gd_Delta_1}
\| \Phi_{\lambda} (A_t) \|_F^2 \geq ~
\lambda \epsilon \cdot \frac{1}{ \sqrt{m} } | \Phi_{\lambda}(A_t) - m |,
\end{align}

It implies that
\begin{align*}
&\Phi_{\lambda} (A_{t+1}) - \Phi_{\lambda} (A_t)\\
\leq & ~ -0.9 \epsilon \lambda \frac{1}{\sqrt{m}} | \Phi_{\lambda}(A_t) - m| +O(\epsilon \lambda)^2 \sqrt{m}\\
\leq &~ -0.9 \epsilon \lambda \frac{1}{\sqrt{m}} | \Phi_{\lambda}(A_t) - m| + 0.1\epsilon \lambda\sqrt{m},
\end{align*}
where the second step follows from extracting the constant term from the summation.

Then, when $\Phi(A_t)> m$, we have
\begin{align*}
    \Phi_{\lambda} (A_{t+1}) \leq (1-0.9 \frac{ \lambda \epsilon }{\sqrt{m} }) \cdot \Phi_{\lambda} (A_t) +  \lambda \epsilon \sqrt{m}.
\end{align*}
When $\Phi(A_t)\leq m$, we have
\begin{align*}
    \Phi_{\lambda} (A_{t+1}) \leq (1+0.9 \frac{ \lambda \epsilon }{\sqrt{m} }) \cdot \Phi_{\lambda} (A_t) -0.8  \lambda \epsilon \sqrt{m}.
\end{align*}

The lemma is then proved.
\end{proof}

 




 




\subsection{Proof of GD Convergence}\label{sec:gd_convergence_proof}
In this section, we provide proofs of convergence analysis of gradient descent matrix sensing algorithm.

\begin{lemma}[Restatement of Lemma~\ref{lem:gd_convergence}]
Suppose the measurement vectors $\{u_i\}_{i\in [m]}$ are orthogonal unit vectors, and suppose $|b_i|$ is bounded by $R$ for $i\in [m]$.  Then, for any $\delta \in (0,1)$, if we take $\lambda = \Omega(\delta^{-1}\log m)$ and $\epsilon=O(\lambda^{-1})$ in Algorithm~\ref{alg:GD}, then for $T=\widetilde{\Omega}(\sqrt{m}R\delta^{-1})$ iterations, the solution matrix $A_T$ satisfies:
\begin{align*}
    | u_i^\top A_{T} u_i - b_i| \leq \delta~~~\forall i\in [m].
\end{align*}
\end{lemma}

\begin{proof}
Let $\tau = \max_{i\in [m]} b_i$. At the beginning, we choose the initial solution $A_1 :=\tau I_n$ where $I_n \in \R^{n \times n}$ is the identity matrix, and we have
\begin{align*}
    \Phi(A_1) =&~ \sum_{i=1}^m \cosh(\lambda\cdot (\tau - b_i))\\
    \leq &~ e^{\lambda \tau} \sum_{i=1}^m e^{-\lambda b_i}\leq 2^{O(\lambda R)},
\end{align*}
where the last step follows from $|b_i|\leq R$ for all $i\in [m]$.

After $T$ iterations, we have
\begin{align*}
    \Phi(A_{T+1}) \leq & ~ (1-\frac{\epsilon \lambda}{\sqrt{m}})^T \Phi(A_1) + 2  m \\
    \leq & ~ (1-\frac{\epsilon \lambda}{\sqrt{m}})^T \cdot 2^{O(\lambda R) }+ 2  m \\
    \leq & ~ 2^{-\Omega( T  \epsilon \lambda / \sqrt{m} ) + O(\lambda R)} + 2  m
\end{align*}
where the first step follows from applying Lemma~\ref{lem:gradient_descent} for $T$ times, and $\sum_{i=1}^T (1-\epsilon\lambda/\sqrt{m})^{i-1}\epsilon \lambda \sqrt{m}\leq 2m$.

As long as $T= \Omega(R \sqrt{m} / \epsilon )=\Omega(R\sqrt{m}\lambda)$, then we have
\begin{align*}
    \Phi(A_{T+1}) \leq O(m).
\end{align*}

This implies that for any $i\in [m]$,
\begin{align*}
    | u_i^\top A_{T+1} u_i - b_i| \leq &~ \lambda^{-1}\cdot \cosh^{-1}(O(m))\\
    = &~ \lambda^{-1}\cdot O(\log m)\\
    = &~ \delta,
\end{align*}
where we take $R = \Omega(\delta^{-1}\log m)$.  

Therefore, with $T=\widetilde{\Omega}(\sqrt{m}R\delta^{-1})$ iterations, Algorithm~\ref{alg:GD} can achieve that
\begin{align}\label{eq:gd_approximation_guarantee}
    | u_i^\top A_{T+1} u_i - b_i| \leq \delta~~~\forall i\in [m].
\end{align}
The theorem is then proved.
\end{proof}

\section{Spectral Potential function with ground-truth oracle}\label{sec:potential_ground_truth_app}
In this section, we consider the matrix sensing with spectral approximation; that is, we want to obtain a matrix $A$ that is a $\delta$-spectral approximation of the ground-truth matrix $A_\star$, i.e.,
\begin{align*}
    (1-\delta)A_\star\preceq A\preceq (1+\delta)A_\star.
\end{align*}
To do this, instead of performing a series of quadratic measurements, we assume that we have access to an oracle ${\cal O}_{A_\star}$ such that for any matrix $A\in \R^{n\times n}$, the oracle will output a matrix $A_\star^{-1/2}AA_\star^{-1/2}$. Algorithm~\ref{alg:GD_spectral} implements a matrix sensing algorithm with spectral approximation guarantee with the assumption of oracle ${\cal O}_{A_\star}$.



We define the spectral loss function as follows:
\begin{align*}
    \Psi_{\lambda}(A) := \tr[ \cosh ( \lambda ( I - (A_{\star})^{-1/2} A (A_{\star})^{-1/2} )  ) ].
\end{align*}
We will show that $\Psi_\lambda(A)$ can characterize the spectral approximation of $A$ with respect to $A_\star$. 



It is easy to see that if we can query an arbitrary $A$ to the ground-truth oracle ${\cal O}_{A_\star}$, then we can definitely recover $A_\star$ exactly by querying ${\cal O}_{A_\star}(I)$. Instead, in Algorithm~\ref{alg:GD_spectral}, we focus on the following process: the initial matrix $A_1$ is given, and in the $t$-th iteration, we first compute
\begin{align*}
    X_t =  \lambda (I - A_{\star}^{-1/2} A_t A_{\star}^{-1/2} )
\end{align*}
and do eigendecompsotion of $X_t$ to obtain $\Lambda_t$ such that $X_t = Q_t \Lambda_t Q_t^{\top}$. Then we update the matrix $A_{t+1}$ by:
\begin{align*}
        A_{t+1} = A_t +  \epsilon \cdot  A_{\star}^{1/2} \sinh(X_t) A_{\star}^{1/2} / \| \sinh(X_t) \|_F.
\end{align*}
We are interested in the number of iterations needed to make $A_t$ be a $\delta$-spectral approximation. We believe this example will provide some insight into this problem, and we leave the question of spectral-approximated matrix sensing without the ground-truth oracle to future work.

\begin{algorithm*}\caption{Matrix Sensing with Spectral Approximation.}\label{alg:GD_spectral}
\begin{algorithmic}[1]
\Procedure{GradientDescent}{${\cal O}_{A_\star}$, $A_1$} %\Comment{Lemma~\ref{lem:gradient_descent_rho}}
    %\State $A_1 \gets I$
    \For{$t = 1 \to T$}
        \State $X_t\gets \lambda\cdot (I_n - {\cal O}_{A_\star}(A_t))$
        \State $Q_t \Lambda_t Q_t^\top\gets$ Eigendecomposition of $X_t$\Comment{It takes $O(n^\omega)$-time}
        \State $Y_t \gets Q_t \cdot \sinh(\Lambda_t)\cdot Q_t^\top$\Comment{$Y_t=\sinh(X_t)$. It takes $O(n^2)$-time}
        \State $A_{t+1} \gets A_t + \epsilon  \cdot {\cal O}_{A_\star}(Y_t) / \| Y_t \|_F$\Comment{It takes $O(n^2)$-time}
    \EndFor
    \State \Return $A_{T+1}$
\EndProcedure
\end{algorithmic}
\end{algorithm*}

\begin{lemma}[Progress on the spectral potential function]\label{lem:potential_func_loss}
Let $c \in (0,1)$ denote a sufficiently small positive constant. 
We define $X_t$ as follows:
\begin{align*}
    X_t := \lambda (I - (A_{\star})^{-1/2} A_t (A_{\star})^{-1/2} )
\end{align*}

Let 
\begin{align*}
    A_{t+1} = A_t
    +  \epsilon \cdot \lambda (A_{\star})^{1/2} \sinh(X_t) (A_{\star})^{1/2} / \| \lambda \cdot \sinh(X_t) \|_F.
\end{align*}


For any $\epsilon\in (0, 1)$ and $\lambda \geq 1$ such $\lambda \epsilon \leq c$, 
we have for any $t>0$,

\begin{align*}
\Psi_{\lambda}(A_{t+1})   \leq (1 - 0.9\epsilon \lambda /\sqrt{n} ) \Psi_{\lambda}(A_t) + \epsilon \lambda \sqrt{n}.
\end{align*}
\end{lemma}

\begin{proof}



We can compute
\begin{align}\label{eq:derivative}
    & ~ \Psi_{\lambda}(A_{t+1}) - \Psi_{\lambda}(A_t) \notag \\
    = & ~ \tr[ \cosh ( X_{t+1} )] - \tr[ \cosh ( X_t  ) ] \notag\\
    \leq & ~ - \lambda \cdot \tr[ \sinh( X_t ) \cdot ( (A_{\star})^{-1/2} (A_{t+1} - A_t )  (A_{\star})^{-1/2} ) ] \notag\\
    + & ~ O(1) \cdot \lambda^2 \cdot \tr[\cosh( X_t ) \cdot ( (A_{\star})^{-1/2} (A_t - A_{t+1} )  (A_{\star})^{-1/2} )^2 ] \notag\\
    = & ~ - \Delta_1 + O(1) \cdot \Delta_2, 
\end{align}
the first step is by expanding by definition, the second step is by Taylor expanding the first term at the point $I-(A_{\star})^{-1/2}A_t(A_{\star})^{-1/2}$ (via Lemma~\ref{lem:jn08}), and the last step is by definition of $\Delta_1$ and $\Delta_2$.


To further simplify proofs, we define
\begin{align*}
    \nabla \Psi_{\lambda}(A_t) := & ~ \lambda \cdot (A_{\star})^{1/2} \sinh( X_t ) (A_{\star})^{1/2} \\
    \wt{\nabla} \Psi_{\lambda}(A_t) := & ~ \lambda \cdot \sinh( X_t )  \\
    \wt{\Delta} \Psi_{\lambda}(A_t) := & ~ \lambda \cdot \cosh( X_t )  
\end{align*}

To maximize the gradient progress, we should choose
\begin{align*}
    A_{t+1} = A_t + \epsilon \cdot \nabla \Psi_{\lambda}(A_t) / \| \wt{\nabla} \Psi_{\lambda}(A_t) \|_F
\end{align*}

Then 
\begin{align}\label{eq:derivative_first_moment_A_t_F}
    \Delta_1 = & ~ (\epsilon \lambda^2) \cdot \tr[  \sinh^2( X_t ) ] / \| \wt{\nabla} \Psi_{\lambda}(A_t) \|_F \notag \\
    = & ~ \epsilon \cdot \| \wt{\nabla} \Psi_{\lambda}(A_t) \|_F^2 /  \| \wt{\nabla} \Psi_{\lambda}(A_t) \|_F \notag \\
    = & ~ \epsilon \cdot \| \wt{\nabla} \Psi_{\lambda}(A_t) \|_F 
\end{align}
and
\begin{align}\label{eq:derivative_second_moment_A_t_F}
    \Delta_2 = & ~ \epsilon ^2  \lambda^4 \cdot \tr[ \cosh( X_t ) \cdot \sinh^2( \lambda ( X_t )  ] / \| \wt{\nabla} \Psi_{\lambda}(A_t) \|_F^2 \notag \\
    = & ~ \epsilon^2 \lambda \cdot \tr[ \wt{\Delta} \Psi_{\lambda}(A_t) \cdot \wt{\nabla} \Psi_{\lambda}(A_t)^2 ] / \| \wt{\nabla} \Psi_{\lambda}(A_t) \|_F^2 \notag\\
    \leq & ~ \epsilon^2 \lambda \cdot \| \wt{\Delta} \Psi_{\lambda}(A_t) \|_F \cdot \| \wt{\nabla} \Psi_{\lambda}(A_t)^2 \|_F / \| \wt{\nabla} \Psi_{\lambda}(A_t) \|_F^2 \notag\\
    \leq & ~ \epsilon^2 \lambda \cdot \| \wt{\Delta} \Psi_{\lambda}(A_t) \|_F \notag\\
    \leq & ~ \epsilon^2 \lambda \cdot ( \lambda \sqrt{n} + \| \wt{\nabla} \Psi_{\lambda}(A_t) \|_F  )
\end{align}
where { the first step follows from the definition of $\Delta_2 $, the second step comes from the definition of $\wt{\Delta} \Psi_{\lambda}(A_t) $ and $\wt{\nabla} \Psi_{\lambda}(A_t)$,} { the third step follows that $\|A B\|_F \leq \|A\|_F \|B\|_F$,} the forth step follows from $\| x \|_4^2 \leq \| x \|_2^2$, and the fifth step follows from Part 1 of Lemma~\ref{lem:property_sinh_cosh_matrix}.


Now, we need to lower bound $\| \wt{\nabla} \Psi_{\lambda}(A_t) \|_F $, we have
\begin{align}\label{eq:derivative_phi_A_t_F}
    \| \wt{\nabla} \Psi_{\lambda}(A_t) \|_F = & ~ ( \tr[ \lambda^2 \sinh^2(X_t) ] )^{1/2} \notag \\
    \geq & ~ \frac{\lambda}{\sqrt{n}} ( \tr[  \cosh(X_t) ] -  n ) \notag \\
    = & ~ \frac{\lambda}{\sqrt{n}} ( \Psi_{\lambda}(A_t) -  n ) 
\end{align}
where the second step follows from Part 2 in Lemma~\ref{lem:property_sinh_cosh_matrix}.

We know that




Then, we have
\begin{align*}
    & ~ \Psi_{\lambda}(A_{t+1}) - \Psi_{\lambda}(A_t) \\
    \leq & ~ - \epsilon \| \wt{\nabla} \Psi_{\lambda}(A_t) \|_F + \epsilon^2 \lambda ( \sqrt{n} + \| \wt{\nabla} \Psi_{\lambda}(A_t) \|_F  ) \\
    \leq & ~ - 0.9 \epsilon  \| \wt{\nabla} \Psi_{\lambda}(A_t) \|_F + \epsilon^2 \lambda^2 \sqrt{n} \\
    \leq & ~ - 0.9 \epsilon \lambda \frac{1}{\sqrt{n}}  \Psi_{\lambda}(A_t)  + \epsilon \lambda \sqrt{n}
\end{align*}
where the first step {  follows from  Eq.~\eqref{eq:derivative_first_moment_A_t_F} and Eq.~\eqref{eq:derivative_second_moment_A_t_F} , the second steps comes from $\epsilon \in (0, 0.01)$, the third step comes from Eq.~\eqref{eq:derivative_phi_A_t_F} and $\epsilon \lambda \leq 1$. }


Finally, we complete the proof. 

\end{proof}


\begin{lemma}[Small spectral potential implies good spectral approximation]
Let $A\in \R^{n\times n}$ be symmetric, and $\lambda >0$. Suppose $\Psi_\lambda(A)\leq p$ for some $p > 1$. Then, we have
\begin{align*}
    (1-\delta) A_\star \preceq A \preceq (1+\delta)A_\star
\end{align*}
for $\delta = O(\lambda^{-1}\log p)$.
\end{lemma}

\begin{proof}
By the definition of $\Psi_\lambda(A)$,  $\Psi_\lambda(A)\leq p$ implies that for any $i\in [n]$,
\begin{align*}
    \cosh(\lambda(1-\lambda_i(A_\star^{-1/2}AA_\star^{-1/2})))\leq p,
\end{align*}
or equivalently,
\begin{align*}
    \left|(1-\lambda_i(A_\star^{-1/2}AA_\star^{-1/2}))\right|\leq O(\lambda^{-1}\log p).
\end{align*}
Hence, we have
\begin{align*}
    (1-\delta)I_n \preceq A_\star^{-1/2}AA_\star^{-1/2} \preceq (1+\delta)I_n,
\end{align*}
where $\delta := O(\lambda^{-1}\log p)$. Therefore, by multiplying $A_\star^{-1/2}$ on both sides, we get that
\begin{align*}
    (1-\delta) A_\star \preceq A \preceq (1+\delta)A_\star,
\end{align*}
which completes the proof of the lemma.
\end{proof}
\section{Gradient descent with General Measurements}\label{sec:gd_non_orthogonal}
  
In this section, we analyze the potential decay by gradient descent with non-orthogonal measurements.  The main result of this section is Lemma~\ref{lem:gradient_descent_rho} in below.
  


We first recall the definition of the potential function $\Phi_{\lambda}(A)$:
\begin{align*}
    \Phi_{\lambda}(A) := \sum_{i=1}^m \cosh ( \lambda ( u_i^\top A u_i - b_i ) ),
\end{align*}
its gradient $\nabla \Phi_{\lambda}(A)\in \R^{n\times n}$:
\begin{align}\label{eq:gradient_phi_A_rho}
    \nabla \Phi_{\lambda}(A) = \sum_{i=1}^m u_i u_i^\top \lambda \sinh\left( \lambda (u_i^\top A u_i - b_i)\right), 
\end{align}
and its Hessian $\nabla^2 \Phi_{\lambda}(A)\in \R^{n^2\times n^2}$:
\begin{align*}
    \nabla^2 \Phi_{\lambda}(A) = \sum_{i=1}^m ( u_i u_i^\top ) \otimes ( u_i u_i^\top ) \lambda^2 \cosh( \lambda (u_i^\top A u_i - b_i) ). 
\end{align*}


 


\begin{lemma}[Progress on entry-wise potential with general measurements]\label{lem:gradient_descent_rho}
Assume that $|u_i^{\top}  u_j| \leq \rho$ and $\rho \leq \frac{1}{10m}$, for any $i,j \in [m]$ and $\| u_i\|^2 = 1$. Let $c \in (0,1)$ denote a sufficiently small positive constant. Then, for any $\epsilon,\lambda>0$ such that $\epsilon\lambda \leq c$, 
we have for any $t>0$,
\begin{align*}
    \Phi_{\lambda} ( A_{t+1} ) \leq (1-0.9 \frac{ \lambda \epsilon }{\sqrt{m} }) \cdot \Phi_{\lambda} (A_t) +  \lambda \epsilon \sqrt{m}
\end{align*}

\end{lemma}

\begin{proof}
We first have
\begin{align}\label{eq:gd_define_Delta_1_and_Delta_2_rho}
    & ~ \Phi_{\lambda}(A_{t+1})- \Phi_{\lambda} (A_t) \notag \\
    \leq & ~ \langle \nabla \Phi_{\lambda} (A_t) , (A_{t+1} - A_t) \rangle 
    + O(1) \langle \nabla^2 \Phi_{\lambda}(A_t), (A_{t+1} - A_t) \otimes (A_{t+1} - A_t) \rangle \notag \\
    := & ~ - \Delta_1 + O(1) \cdot \Delta_2,
\end{align}
which follows from Corollary~\ref{cor:matrix_der_trace}.
 

We choose
\begin{align}\label{eq:A_t1_update_rho}
    A_{t+1} = A_t - \epsilon \cdot \nabla \Phi_{\lambda}(A_t) / \| \nabla \Phi_{\lambda}(A_t) \|_F.
\end{align}

We can bound
\begin{align}\label{eq:tr_phi_At_first_moment_rho} 
 \Delta_1 = & ~ -\tr[\nabla \Phi_{\lambda}(A_t) (A_{t+1} - A_t)] \notag \\
=  & ~ \epsilon \cdot \|\nabla \Phi_{\lambda} (A_t) \|_F.
\end{align}

For $\| \Phi_{\lambda} (A_t) \|_F^2$,
\begin{align}\label{eq:phi_At_F_norm_rho}
& ~ \frac{1}{\lambda^2} \| \nabla \Phi_{\lambda} (A_t) \|_F^2 \notag \\
= & ~   \tr[ (\sum_{i=1}^m u_i u_i^\top \sinh( \lambda( u_i^\top A_t u_i - b_i ) )  )^2 ] \notag\\
= & ~   \tr[ \sum_{i=1}^m  \sinh^2 ( \lambda( u_i^\top A_t u_i - b_i ) )  ] \notag \\
 + &~ \tr[ \sum_{i =1}^{m}\sum_{j \neq i}^{m} (u_i u_i^\top)(u_j u_j^{\top}) \sinh ( \lambda( u_i^\top A_t u_i - b_i ) ) \cdot \sinh ( \lambda( u_j^\top A_t u_j - b_j ) )]   \notag \\
 \geq &~0.9 \tr[ \sum_{i=1}^m  \sinh^2 ( \lambda( u_i^\top A_t u_i - b_i ) )] \notag \\
\geq & ~ 0.9\frac{1}{m} ( \sum_{i=1}^m \cosh ( \lambda ( u_i^\top A_t u_i - b_i ) ) - m )^2 \notag \\
= & ~ 0.9\frac{1}{m} ( \Phi_{\lambda} (A_t) - m )^2,
\end{align}
where the first step follows from Eq.~\eqref{eq:gradient_phi_A_rho},
the second steps follow from partitioning based on whether $i = j$ and $\| u_i \|_2 = 1$, the third step comes from Claim~\ref{claim:sum_ij_off_diagonal}, the fourth step in Eq.~\eqref{eq:phi_At_F_norm_rho} follows from Part 2 in Lemma~\ref{lem:property_sinh_cosh_scalar},  the fifth step follows from the definition of $\Phi_{\lambda}(A)$.





Thus,
\begin{align}\label{eq:bound_gd_Delta_1_rho}
   \Delta_1 = & ~ -\tr[\nabla \Phi_{\lambda}(A_t) (A_{t+1} - A_t)] \notag \\
   \geq & ~ \lambda \epsilon \cdot \frac{1}{ \sqrt{m} } ( \Phi_{\lambda}(A_t) - m )  .
\end{align}

For simplicity, we define
\begin{align*}
    z_{t,i} := \lambda (u_i^\top A_t u_i - b_i).
\end{align*}

We need to compute this $\Delta_2$. For simplificity, we consider $\Delta_2 \cdot (\frac{1}{\epsilon \lambda})^2 \cdot \| \nabla \Phi_{\lambda} (A_t) \|_F^2$, which can be expressed as:
\begin{align}\label{eq:tr_phi_At_second_moment_rho}
&\Delta_2 \cdot (\frac{1}{\epsilon \lambda})^2 \cdot \| \nabla \Phi_{\lambda} (A_t) \|_F^2\notag\\
     = & ~ \frac{1}{(\lambda \epsilon)^2} \tr[ \nabla^2 \Phi_{\lambda}(A_t) \cdot (A_{t+1} - A_t) \otimes (A_{t+1} - A_t) ] \cdot 
     \| \nabla \Phi_{\lambda}(A_t) \|_F^2 \notag\\
    = & ~  \tr\Big[ \nabla^2 \Phi_{\lambda}(A_t) \cdot 
    ( \sum_{i=1}^m u_i u_i^\top  \sinh( z_{t,i} ) ) \otimes ( \sum_{i=1}^m u_i u_i^\top \sinh( z_{t,i} ) )\Big] \notag\\
    = & ~ \tr[ \nabla^2 \Phi_{\lambda}(A_t)  ( \sum_{i,j}  \sinh( z_{t,i} )\sinh( z_{t,i} ) (u_i u_i^\top \otimes u_ju_j^\top) ) ] \notag\\
    = & ~ \tr[ \nabla^2 \Phi_{\lambda}(A_t)  ( \sum_{i=1}^m \sinh^2( z_{t,i} )) (u_i u_i^\top \otimes u_iu_i^\top) ) ] \notag\\
    + & ~  \tr[ \nabla^2 \Phi_{\lambda}(A_t)  ( \sum_{i\neq j} \sinh( z_{t,i} )\sinh( z_{t,j} ) (u_i u_i^\top \otimes u_j u_j^\top) ) ] \notag\\
    = & ~  Q_1 + Q_2  ,
\end{align}
where 
\begin{align}\label{eq:def_Q1_rho}
Q_1:= \tr\Big[ \nabla^2 \Phi_{\lambda}(A_t) \cdot ( \sum_{i=1}^m \sinh^2( z_{t,i} )) (u_i u_i^\top \otimes u_iu_i^\top) ) \Big]
\end{align}
denotes the diagonal term, and
\begin{align}\label{eq:def_Q2_rho}
    Q_2:=\tr\Big[ \nabla^2 \Phi_{\lambda}(A_t) \cdot ( \sum_{i\neq j} \sinh( z_{t,i} )\sinh( z_{t,j} ) (u_i u_i^\top \otimes u_j u_j^\top) ) \Big]
\end{align}
denotes the off-diagonal term. The first step comes from the definition of $\Delta_2$,
the second step follows from replacing $A_{t+1} - A_t$ using Eq.~\eqref{eq:A_t1_update_rho}, the third step follows that we extract the scalar values from Kronecker product, the fourth step comes from splitting into two partitions based on whether $i = j$, the fifth step comes from the definition of $Q_1$ and $Q_2$.




Thus,
\begin{align}\label{eq:bound_gd_Delta_2_rho}
    \Delta_2 \leq & ~ (\epsilon \lambda)^2 (Q_1 + Q_2) / \| \nabla \Phi_{\lambda}(A_t) \|_F^2 \notag \\
    = & ~ 1.3(\epsilon \lambda )^2 \cdot ( \sqrt{m} + \frac{1}{\lambda} \| \nabla \Phi_{\lambda}(A_t) \|_F ).
\end{align}
where the second step follows from  Claim~\ref{cla:gd_Q1_rho} and Claim~\ref{cla:gd_Q2_rho}.

Hence, we have
\begin{align*}
    & ~ \Phi_{\lambda} (A_{t+1}) - \Phi_{\lambda} (A_t) \\
    \leq & ~ - \Delta_1    + O(1) \cdot \Delta_2 \\
    \leq & ~ - \epsilon \| \nabla \Phi_{\lambda}(A_t) \|_F   + O(1) (\epsilon \lambda)^2 (\sqrt{m} + \frac{1}{\lambda} \| \nabla \Phi_{\lambda}(A_t) \|_F ) \\
    \leq & ~ - 0.9 \epsilon \| \Phi_{\lambda}(A_t) \|_F+ O(\epsilon \lambda)^2 \sqrt{m} \\
    \leq & ~ -0.9 \epsilon \lambda \frac{1}{\sqrt{m}} ( \Phi_{\lambda}(A_t) - m ) +O(\epsilon \lambda)^2 \sqrt{m} \\
    \leq & ~  -0.9 \epsilon \lambda \frac{1}{\sqrt{m}} \Phi_{\lambda}(A_t) + \epsilon \lambda \sqrt{m},
\end{align*}
where the first step follows from Eq.~\eqref{eq:gd_define_Delta_1_and_Delta_2_rho}, the second step follows from Eq.~\eqref{eq:bound_gd_Delta_1_rho} and Eq.~\eqref{eq:bound_gd_Delta_2_rho},  the third step follows from $\epsilon \lambda \in (0, 0.01)$, the fourth step follows from Lemma~\ref{lem:cosh_sinh}, and the final step follows that extracting the constant term from the summation.

The lemma is then proved.
\end{proof}

We prove some technical claims in below.

\begin{claim}\label{claim:sum_ij_off_diagonal}
It holds that:
\begin{align*}
    & \sum_{i\ne j\in [m]} \langle u_i,u_j\rangle^2 \sinh ( \lambda( u_i^\top A_t u_i - b_i ) ) \sinh ( \lambda( u_j^\top A_t u_j - b_j ) ) \\
    \leq &~ 0.1 \sum_{i=1}^m  \sinh^2 ( \lambda( u_i^\top A_t u_i - b_i ) ) 
\end{align*}
\end{claim}
\begin{proof}
We define $R_{i,j}$ and $R$ as follows:
\begin{align*}
    R_{i,j} = &~\sinh ( \lambda( u_i^\top A_t u_i - b_i ) )  \sinh ( \lambda( u_j^\top A_t u_j - b_j ) ) \\
     R = &~\tr[ \sum_{i =1}^{m}\sum_{j \neq i}^{m} (u_i u_i^\top)(u_j u_j^{\top}) \sinh ( \lambda( u_i^\top A_t u_i - b_i ) ) \cdot 
     \sinh ( \lambda( u_j^\top A_t u_j - b_j ) )]
\end{align*}

Then we can upper bound $|R|$ by:
\begin{align*}
    |R| =  &~ \tr[\sum_{i =1}^{m}\sum_{j \neq i}^{m} |(u_i u_i^\top)(u_j u_j^{\top})| |R_{i,j}|] \\
    \leq &~ \rho^2 \tr[\sum_{i =1}^{m}\sum_{j \neq i}^{m}  |R_{i,j}|] \\
    \leq &~  \frac{\rho^2 }{2}\tr[\sum_{i =1}^{m}\sum_{j \neq i}^{m} (R_{i,i} + R_{j,j})] \\
    \leq &~ m \rho^2 \tr[\sum_{i=1}^{m} R_{i,i} ] \\
    \leq &~ 0.1\tr[\sum_{i=1}^m R_{i,i}]
\end{align*}
where the first step follows $|ab| = |a||b|$, the second step follows $|u_i^{\top}  u_j| \leq \rho$, the third step follows that $|ab| \leq \frac{a^2 + b^2}{2}$, the fourth step follows from the summation over $j$, and the fifth step comes from $m\rho^2 \leq 0.1$.
\end{proof}

\begin{claim}\label{cla:gd_Q1_rho}
For $Q_1$ defined in Eq.~\eqref{eq:def_Q1_rho}, we have
\begin{align*}
    Q_1 \leq 1.1\Big( \sqrt{m} + \frac{1}{\lambda} \| \nabla \Phi_{\lambda}(A_t) \|_F \Big) \cdot  \| \nabla \Phi_{\lambda} (A_t) \|_F^2.
\end{align*}
\end{claim}
\begin{proof}


For simplicity, we define $z_{t,i}$ to be
\begin{align*}
    z_{t,i} := \lambda ( u_i^\top A_t u_i - b_i ) .
\end{align*}
Recall that
\begin{align*}
    \nabla^2 \Phi_{\lambda}(A_t) = \lambda^2 \cdot \sum_{i=1}^m ( u_i u_i^\top ) \otimes ( u_i u_i^\top ) \cosh( z_{t,i} ) .
\end{align*}

For $Q_1$, we have
\begin{align}\label{eq:Q1_rho}
   Q_1 = & ~  \tr[ \nabla^2 \Phi_{\lambda}(A_t) \sum_{i=1}^m  \sinh^2( z_{t,i} ) (u_i u_i^\top \otimes u_i u_i^\top) ) ] \notag\\
   = & ~ \lambda^2 \cdot \tr[ \sum_{i=1}^m  \cosh( z_{t,i} ) ( u_i u_i^\top ) \otimes ( u_i u_i^\top )  
   \cdot   \sum_{i=1}^m  \sinh^2( z_{t,i} ) (u_i u_i^\top ) \otimes ( u_i u_i^\top)   ] \notag\\
   = & ~ \lambda^2 \cdot \sum_{i=1}^m \tr[ \cosh( z_{t,i} ) \sinh^2( z_{t,i} )\cdot 
    ( u_i u_i^\top  u_i u_i^\top )  \otimes ( u_i u_i^\top u_i u_i^\top ) ] \notag\\
+ &~  \lambda^2 \cdot \sum_{i=1}^m \sum_{j \neq i}^{m} \tr[ \cosh( z_{t,i} )\sinh^2( z_{t,j} ) \cdot 
 (u_i u_i^\top u_j u_j^{\top} ) \otimes (u_j u_j^{\top} u_i u_i^\top)] \notag \\
   = & ~ \lambda^2 \cdot \sum_{i=1}^m  \cosh( z_{t,i} ) \sinh^2( z_{t,i} ) 
    \notag \\ 
   + & ~  \lambda^2 \cdot \sum_{i=1}^m \sum_{j \neq i}^{m} \tr[ \cosh( z_{t,i} )\sinh^2( z_{t,j} ) \cdot 
 (u_i u_i^\top u_j u_j^{\top} ) \otimes (u_j u_j^{\top} u_i u_i^\top)] \notag \\
   \leq & ~  1.1 \lambda ^2 \cdot (  \sum_{i=1}^m  \cosh^2( z_{t,i} ) )^{1/2}
   \cdot  ( \sum_{i=1}^m \sinh^4( z_{t,i} ) )^{1/2} \notag \\
   \leq & ~ 1.1 \lambda^2 \cdot B_1 \cdot B_2,
\end{align}
where {  the first step comes from the definition of $Q_1$, the second step comes from the definition of $\nabla^2 \Phi_{\lambda}(A_t)$,}
the third step follows from $(A \otimes B) \cdot (C \otimes D) = (AC) \otimes (BD)$ and partition the terms based on whether $i = j$, the fourth step comes from $\|u_i\| = 1$ and $\tr[ (u_i  u_i^\top) \otimes (u_i  u_i^\top) ] = 1$, and the fifth step comes from Cauchy–Schwarz inequality and Claim~\ref{claim:q1_helper_off_diagonal}.

\begin{claim}\label{claim:q1_helper_off_diagonal}
We can bound the off-diagonal entries by: 
\begin{align*}
  &~   |\lambda^2 \cdot \sum_{i=1}^m \sum_{j \neq i}^{m} \tr[ \cosh( z_{t,i} )\sinh^2( z_{t,j} ) \cdot 
  (u_i u_i^\top u_j u_j^{\top} ) \otimes (u_j u_j^{\top} u_i u_i^\top)]| \notag \\
 \leq &~ 0.1 \lambda^2(\sum_{i=1}^m(\cosh(z_{t,1}))^{1/2} \cdot (\sum_{i=1}^m \sinh^4( z_{t,i}))^{1/2}   \\  
\end{align*}
\end{claim}
\begin{proof}

\begin{align*}
      &~   |\lambda^2 \cdot \sum_{i=1}^m \sum_{j \neq i}^{m} \tr[ \cosh( z_{t,i} )\sinh^2( z_{t,j} ) \cdot 
       (u_i u_i^\top u_j u_j^{\top} ) \otimes (u_j u_j^{\top} u_i u_i^\top)]| \notag \\
      \leq &~ \rho^2 \lambda^2 |\sum_{i=1}^m \sum_{j \neq i}^{m} \cosh( z_{t,i} )\sinh^2( z_{t,j} )| \\
      \leq &~  \rho^2 \lambda^2 (\sum_{i=1}^m \sum_{j \neq i}^{m} (\cosh^2( z_{t,i} ))^{1/2} \cdot (\sum_{i=1}^m \sum_{j \neq i}^{m}\sinh^4( z_{t,j}))^{1/2} \\
      \leq &~ m\rho^2 \lambda^2  (\sum_{i=1}^m(\cosh^2(z_{t,i}))^{1/2} \cdot (\sum_{i=1}^m \sinh^4( z_{t,i}))^{1/2} \\
      \leq &~ 0.1 \lambda^2(\sum_{i=1}^m(\cosh^2(z_{t,i}))^{1/2} \cdot (\sum_{i=1}^m \sinh^4( z_{t,i}))^{1/2} 
\end{align*}
where the first step comes from $|\langle u_i, u_j\rangle | \leq \rho$, the second step comes from Cauchy–Schwarz inequality, the third step follows from summation over $m$ terms, and the fourth step comes from $\rho^2 m \leq 0.1$.
\end{proof}


For the term $B_1$, we have
\begin{align}\label{eq:b1_rho}
    B_1 = & ~ (  \sum_{i=1}^m \cosh^2( \lambda (u_i^\top A_t u_i - b_i ) ) )^{1/2} \notag \\
    \leq & ~ \sqrt{m} + \frac{1}{\lambda} \| \nabla \Phi_{\lambda}(A_t) \|_F,
\end{align}
where the second step follows Part 1 of Lemma~\ref{lem:property_sinh_cosh_scalar}.

For the term $B_2$, we have
\begin{align}\label{eq:b2_rho}
    B_2 = & ~( \sum_{i=1}^m \sinh^4( \lambda( u_i^\top A_t u_i - b_i ) ) )^{1/2} \notag  \\
    \leq & ~ \frac{1}{\lambda^2} \| \nabla \Phi_{\lambda} (A_t) \|_F^2,
\end{align}
where the second step follows from $\| x \|_4^2 \leq \| x \|_2^2$. This implies that
\begin{align*}
    Q_1 \leq & ~ 1.1\lambda^2 \cdot B_1 \cdot B_2 \\
    \leq & ~ 1.1\lambda^2 \cdot ( \sqrt{m} + \frac{1}{\lambda} \| \nabla \Phi_{\lambda}(A_t) \|_F ) \cdot \frac{1}{\lambda^2} \| \nabla \Phi_{\lambda} (A_t) \|_F^2 \\
    = & ~ 1.1( \sqrt{m} + \frac{1}{\lambda} \| \nabla \Phi_{\lambda}(A_t) \|_F ) \cdot  \| \nabla \Phi_{\lambda} (A_t) \|_F^2 .
\end{align*}
This completes the proof.
\end{proof}


\begin{claim}\label{cla:gd_Q2_rho} 
For $Q_2$ defined in Eq.~\eqref{eq:def_Q2_rho}, we have:
\begin{align*}
    Q_2 \leq 0.2 \lambda^2 ( \sqrt{m} + \frac{1}{\lambda} \| \nabla \Phi_{\lambda}(A_t) \|_F ) \cdot  \| \nabla \Phi_{\lambda} (A_t) \|_F^2 
\end{align*}
\end{claim}
\begin{proof}
Because in $Q_2$ we have :
\begin{align}\label{eq:u_ell_i_j_product_rho}
 Q_2 =  & ~ \lambda^2 \tr[\sum_{\ell=1}^{m}(\cosh(z_{t,\ell}) \cdot u_{\ell} u_{\ell}^{\top} \otimes u_{\ell} u_{\ell}^{\top}) \cdot 
 \sum_{i \neq j}^{m}(\sinh( z_{t,i} )\sinh( z_{t,j} ) \cdot u_i u_i^{\top} \otimes u_j u_j^{\top})] \notag \\
    = & ~ \lambda^2\tr[\sum_{\ell=1}^{m} \sum_{i \neq j}^{m} \cosh(z_{t,\ell} )\sinh( z_{t,i} )\sinh( z_{t,j} ) \cdot 
     (u_{\ell} u_{\ell}^{\top} u_i u_i^{\top}) \otimes (u_{\ell} u_{\ell}^{\top} u_j u_j^{\top})] \notag \\
    \leq & ~  \lambda^2\rho^2 \sum_{\ell=1}^{m} \sum_{i \neq j}^{m} \cosh(z_{t,\ell} )(\sinh^2( z_{t,i} ) + \sinh^2( z_{t,j} )) \notag \\
    \leq &~ 2m \lambda^2\rho^2 \sum_{\ell=1}^{m} \sum_{i = 1}^{m} \cosh(z_{t,\ell} )\sinh^2( z_{t,i} ) \notag \\
    \leq &~ 2m^2 \lambda^2\rho^2 \sum_{i = 1}^{m} \cosh(z_{t,i} )\sinh^2( z_{t,i} ) \notag \\
    \leq &~ 2m^2 \lambda^2\rho^2 (\sum_{i=1}^m(\cosh^2(z_{t,i}))^{1/2}  (\sum_{i=1}^m \sinh^4( z_{t,i}))^{1/2} \notag \\
    \leq &~ 0.2 \lambda^2 ( \sqrt{m} + \frac{1}{\lambda} \| \nabla \Phi_{\lambda}(A_t) \|_F ) \cdot  \| \nabla \Phi_{\lambda} (A_t) \|_F^2 
\end{align}
where the second step follows from $(A \otimes B) \cdot (C \otimes D) = (AC) \otimes (BD)$, the third step follows Cauchy–Schwarz inequality and $|\langle u_i, u_j\rangle | \leq \rho$, the fourth step follows from combining $\sinh^2( z_{t,i} )$ and $\sinh^2( z_{t,j} )$, the fifth step comes from summation over $m$ terms, and the sixth step comes from Cauchy–Schwarz inequality and the seventh step follows from Eq.~\eqref{eq:b1_rho} and Eq.~\eqref{eq:b2_rho} and $m^2 \rho^2 \leq 0.1$.

\end{proof}


\section{Stochastic Gradient Descent for General Measurements}\label{sec:sgd_general}

In this section, we further extend the general measurement  where $\{u_i\}_{i\in [m]}$ are non-orthogonal vectors and $|u_i^{\top}  u_j| \leq \rho$ to the convergence analysis of the stochastic gradient descent matrix sensing algorithm. Algorithm~\ref{alg:stochastic_gradient_descent_general} implements the stochastic gradient descent version of the matrix sensing algorithm.

In Algorithm~\ref{alg:stochastic_gradient_descent_general}, at each iteration $t$, we first compute the stochastic gradient descent by:
\begin{align*}
    \nabla \Phi_{\lambda} (A_t, {\cal B}_t) \gets \frac{m}{B} \sum_{i \in {\cal B}_t} u_i u_i^\top \lambda \sinh( \lambda z_{i} )
\end{align*}
then we update the matrix with the gradient:
\begin{align*}
    A_{t+1} \gets A_t - \epsilon \cdot \nabla \Phi_{\lambda}(A_t,{\cal B} _t) / \| \nabla \Phi_{\lambda}(A_t) \|_F
\end{align*}
At the end of each iteration, we update $z_i$ by:
\begin{align*}
    z_i\gets z_i - \epsilon  \lambda m w_{i,j}^2\sinh(\lambda z_{j}) / (\| \nabla \Phi_{\lambda}(A_t) \|_F B \;\; \forall i \in [m], j \in {\cal B}_t
\end{align*}
We are interested in studying the time complexity and convergence analysis under the general measurement assumption.

\begin{lemma}[Cost-per-iteration of stochastic gradient descent for general measurements]\label{lem:gd_cost_per_iter_general} Algorithm~\ref{alg:stochastic_gradient_descent_general} takes $O(mn^2)$-time for preprocessing and 
each iteration takes
$
    O(Bn^2+m^2)
$-time.
\end{lemma}
\begin{proof}
Since $u_i$'s are no longer orthogonal, we need to compute $\|\nabla \Phi_\lambda (A_t)\|_F$ in the following way:
\begin{align*}
    &\|\nabla \Phi_\lambda (A_t)\|_F^2\\
    =&~ \tr\Big[\Big(\sum_{i=1}^m u_i u_i^\top \lambda \sinh( \lambda (\lambda z_{t,i}))\Big)^2\Big]\\
    = &~ \lambda^2\sum_{i,j=1}^m \langle u_i, u_j\rangle^2 \sinh( \lambda (\lambda z_{t,i}))\sinh( \lambda (\lambda z_{t,j}))\\
    = &~ \lambda^2\sum_{i,j=1}^m w_{i,j}^2 \sinh( \lambda (\lambda z_{t,i}))\sinh( \lambda (\lambda z_{t,j})).
\end{align*}
Hence, with $\{z_{t,i}\}_{i\in [m]}$, we can compute $\|\nabla \Phi_\lambda (A_t)\|_F$ in $O(m^2)$-time.

Another difference from the orthogonal measurement case is the update for $z_{t+1,i}$. Now, we have
\begin{align*}
        &z_{t+1,i}-z_{t,i}\\
    = &~ u_i^\top (A_{t+1}-A_t)u_i\\
    = &~  -\frac{\epsilon}{\| \nabla \Phi_{\lambda}(A_t) \|_F}\cdot u_i^\top \nabla \Phi_{\lambda}(A_t,{\cal B} _t)u_i\\
    = &~ -\frac{\epsilon \lambda m }{\| \nabla \Phi_{\lambda}(A_t) \|_F B} \sum_{j\in {\cal B}_t}u_i^\top u_j u_j^\top u_i \cdot  \sinh(\lambda z_{t,j})\\
    = &~ -\frac{\epsilon \lambda m }{\| \nabla \Phi_{\lambda}(A_t) \|_F B} \sum_{j\in {\cal B}_t}w_{i,j}^2 \cdot  \sinh(\lambda z_{t,j}).
\end{align*}
Hence, each $z_{t+1,i}$ can be computed in $O(B)$-time. And it takes $O(mB)$-time to update all $z_{t+1,i}$.

The other steps' time costs  are quite clear from Algorithm~\ref{alg:stochastic_gradient_descent_general}.
\end{proof}


\begin{lemma}[Progress on expected potential with general measurements]\label{lem:sgd_potential_general}
Assume that $|u_i^{\top}  u_j| \leq \rho$ and $\rho \leq \frac{1}{10m}$, for any $i,j \in [m]$ and $\| u_i\|^2 = 1$. Let $c \in (0,1)$ denote a sufficiently small positive constant. Then, for any $\epsilon,\lambda>0$ such that $\epsilon \lambda \leq c \frac{|{\cal B}_t|}{m}$, 
we have for any $t>0$,
\begin{align*}
    \E[\Phi_{\lambda} ( A_{t+1} )] \leq (1-0.9 \frac{ \lambda \epsilon }{\sqrt{m} }) \cdot \Phi_{\lambda} (A_t) +  \lambda \epsilon \sqrt{m}
\end{align*}

\end{lemma}
The proof is a direct generalization of Lemma~\ref{lem:stochastic_gradient_descent} and is very similar to Lemma~\ref{lem:gradient_descent_rho}. Thus, we omit it here.
\begin{algorithm*}[!ht]\caption{Matrix Sensing with Stochastic Gradient Descent (General Measurements).}\label{alg:stochastic_gradient_descent_general}
\begin{algorithmic}[1]
\Procedure{SGD\_General}{$\{u_i,b_i\}_{i\in [m]}$} \Comment{Lemma~\ref{lem:gd_cost_per_iter_general}}
    \State $\tau \gets \max_{i \in [m]} b_i $
    \State $A_1 \gets \tau \cdot I$
    \State $z_i\gets u_i^\top A_1 u_i - b_i$ for $i\in [m]$\Comment{$z\in \R^m$}
    \State $w_{i,j}\gets \langle u_i, u_j\rangle$ for $i,j\in [m]$ \Comment{$w\in \R^{m\times m}$}
    \For{$t = 1 \to T$}
        \State Sample ${\cal B}_t \subset [m]$ of size $B$ uniformly at random
        \State $\nabla \Phi_{\lambda} (A_t, {\cal B}_t) \gets \frac{m}{B} \sum_{i \in {\cal B}_t} u_i u_i^\top \lambda \sinh( \lambda z_{i} ) $\Comment{It takes $O(Bn^2)$-time}
        \State $\| \nabla \Phi_{\lambda} (A_t) \|_F\gets \lambda \left(\sum_{i,j=1}^m  w_{i,j}^2\sinh (\lambda z_{i})\sinh(\lambda z_j)\right)^{1/2}$\Comment{It takes $O(m^2)$-time}
        \State $A_{t+1} \gets A_t - \epsilon \cdot \nabla \Phi_{\lambda}(A_t,{\cal B} _t) / \| \nabla \Phi_{\lambda}(A_t) \|_F$\Comment{It takes $O(n^2)$-time}
        \For{$i\in [m]$}\Comment{Update $z$. It takes $O(mB)$-time}
            \For{$j\in {\cal B}_t$}
                \State $z_i\gets z_i - \epsilon  \lambda m w_{i,j}^2\sinh(\lambda z_{j}) / (\| \nabla \Phi_{\lambda}(A_t) \|_F B)$
            \EndFor
        \EndFor
    \EndFor
    \State \Return $A_{T+1}$
\EndProcedure
\end{algorithmic}
\end{algorithm*}



%%%% Cut-line between first 10 pages and appendix




%\printbibliography[heading=bibintoc,title={References}]
%\section*{References}
%\printbibliography[heading=none]




%%% some writing rules

%% Writing rule for creating tags.
%% Tags :
%% Theorem    \ref{thm:bla_bla}
%% Lemma      \ref{lem:bla_bla}
%% Claim      \ref{cla:bla_bla}
%% Corollary  \ref{cor:bla_bla}
%% Fact       \ref{fac:bla_bla}
%% Definition \ref{def:bla_bla}
%% Section    \ref{sec:bla_bla}
%% Subsection \ref{sub:bla_bla}
%% Equation   \ref{eq:bla_bla}



\end{document}



%%%%%%%%%%%%%%%%%%%%%%%%%%%%%%%%%%%%%%%%%%%%%%%%%%%%%%%%%%%%%%%%%%%%%%%%%%%%%%%%%%%%%%%%%%%%%%%%%%%%%%%%%%%%%%%%%%%%%%%%%%%%%%%%%%%%%%%%%%%%%%%%%%%%%%%%%%%%%%%%%%%%%%%%%%%%%%%%%%%%%%%%%%%%%%%%%%%%%%%%%%%%%%%%%%%%%%%%%%%%%%%%%%%%%%%%%%%%%%%%%%%%%%%%%%%%%%%%%%%%%%%%%%%%%%%%%%%%%%%%%%%%%%%%%%%%%%%%%%%%%%%%%%%%%%%%%%%%%%%%%%%%%%%%%%%%%%%%%%%%%%%%%%%%%%%%%%%%%%%%%%%%%%%%%%%%%%%%%%%%%%%%%%%%%%%%%%%%%%%%%%%%%%%%%%%%%%%%%%%%%%%%%%%%%%%%%%%%%%%%%%%%%%%%%%%%%%%%%%%%%%
