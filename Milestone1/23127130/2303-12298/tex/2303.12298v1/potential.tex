\section{Spectral Potential function with ground-truth oracle}\label{sec:potential_ground_truth_app}
In this section, we consider the matrix sensing with spectral approximation; that is, we want to obtain a matrix $A$ that is a $\delta$-spectral approximation of the ground-truth matrix $A_\star$, i.e.,
\begin{align*}
    (1-\delta)A_\star\preceq A\preceq (1+\delta)A_\star.
\end{align*}
To do this, instead of performing a series of quadratic measurements, we assume that we have access to an oracle ${\cal O}_{A_\star}$ such that for any matrix $A\in \R^{n\times n}$, the oracle will output a matrix $A_\star^{-1/2}AA_\star^{-1/2}$. Algorithm~\ref{alg:GD_spectral} implements a matrix sensing algorithm with spectral approximation guarantee with the assumption of oracle ${\cal O}_{A_\star}$.



We define the spectral loss function as follows:
\begin{align*}
    \Psi_{\lambda}(A) := \tr[ \cosh ( \lambda ( I - (A_{\star})^{-1/2} A (A_{\star})^{-1/2} )  ) ].
\end{align*}
We will show that $\Psi_\lambda(A)$ can characterize the spectral approximation of $A$ with respect to $A_\star$. 



It is easy to see that if we can query an arbitrary $A$ to the ground-truth oracle ${\cal O}_{A_\star}$, then we can definitely recover $A_\star$ exactly by querying ${\cal O}_{A_\star}(I)$. Instead, in Algorithm~\ref{alg:GD_spectral}, we focus on the following process: the initial matrix $A_1$ is given, and in the $t$-th iteration, we first compute
\begin{align*}
    X_t =  \lambda (I - A_{\star}^{-1/2} A_t A_{\star}^{-1/2} )
\end{align*}
and do eigendecompsotion of $X_t$ to obtain $\Lambda_t$ such that $X_t = Q_t \Lambda_t Q_t^{\top}$. Then we update the matrix $A_{t+1}$ by:
\begin{align*}
        A_{t+1} = A_t +  \epsilon \cdot  A_{\star}^{1/2} \sinh(X_t) A_{\star}^{1/2} / \| \sinh(X_t) \|_F.
\end{align*}
We are interested in the number of iterations needed to make $A_t$ be a $\delta$-spectral approximation. We believe this example will provide some insight into this problem, and we leave the question of spectral-approximated matrix sensing without the ground-truth oracle to future work.

\begin{algorithm*}\caption{Matrix Sensing with Spectral Approximation.}\label{alg:GD_spectral}
\begin{algorithmic}[1]
\Procedure{GradientDescent}{${\cal O}_{A_\star}$, $A_1$} %\Comment{Lemma~\ref{lem:gradient_descent_rho}}
    %\State $A_1 \gets I$
    \For{$t = 1 \to T$}
        \State $X_t\gets \lambda\cdot (I_n - {\cal O}_{A_\star}(A_t))$
        \State $Q_t \Lambda_t Q_t^\top\gets$ Eigendecomposition of $X_t$\Comment{It takes $O(n^\omega)$-time}
        \State $Y_t \gets Q_t \cdot \sinh(\Lambda_t)\cdot Q_t^\top$\Comment{$Y_t=\sinh(X_t)$. It takes $O(n^2)$-time}
        \State $A_{t+1} \gets A_t + \epsilon  \cdot {\cal O}_{A_\star}(Y_t) / \| Y_t \|_F$\Comment{It takes $O(n^2)$-time}
    \EndFor
    \State \Return $A_{T+1}$
\EndProcedure
\end{algorithmic}
\end{algorithm*}

\begin{lemma}[Progress on the spectral potential function]\label{lem:potential_func_loss}
Let $c \in (0,1)$ denote a sufficiently small positive constant. 
We define $X_t$ as follows:
\begin{align*}
    X_t := \lambda (I - (A_{\star})^{-1/2} A_t (A_{\star})^{-1/2} )
\end{align*}

Let 
\begin{align*}
    A_{t+1} = A_t
    +  \epsilon \cdot \lambda (A_{\star})^{1/2} \sinh(X_t) (A_{\star})^{1/2} / \| \lambda \cdot \sinh(X_t) \|_F.
\end{align*}


For any $\epsilon\in (0, 1)$ and $\lambda \geq 1$ such $\lambda \epsilon \leq c$, 
we have for any $t>0$,

\begin{align*}
\Psi_{\lambda}(A_{t+1})   \leq (1 - 0.9\epsilon \lambda /\sqrt{n} ) \Psi_{\lambda}(A_t) + \epsilon \lambda \sqrt{n}.
\end{align*}
\end{lemma}

\begin{proof}



We can compute
\begin{align}\label{eq:derivative}
    & ~ \Psi_{\lambda}(A_{t+1}) - \Psi_{\lambda}(A_t) \notag \\
    = & ~ \tr[ \cosh ( X_{t+1} )] - \tr[ \cosh ( X_t  ) ] \notag\\
    \leq & ~ - \lambda \cdot \tr[ \sinh( X_t ) \cdot ( (A_{\star})^{-1/2} (A_{t+1} - A_t )  (A_{\star})^{-1/2} ) ] \notag\\
    + & ~ O(1) \cdot \lambda^2 \cdot \tr[\cosh( X_t ) \cdot ( (A_{\star})^{-1/2} (A_t - A_{t+1} )  (A_{\star})^{-1/2} )^2 ] \notag\\
    = & ~ - \Delta_1 + O(1) \cdot \Delta_2, 
\end{align}
the first step is by expanding by definition, the second step is by Taylor expanding the first term at the point $I-(A_{\star})^{-1/2}A_t(A_{\star})^{-1/2}$ (via Lemma~\ref{lem:jn08}), and the last step is by definition of $\Delta_1$ and $\Delta_2$.


To further simplify proofs, we define
\begin{align*}
    \nabla \Psi_{\lambda}(A_t) := & ~ \lambda \cdot (A_{\star})^{1/2} \sinh( X_t ) (A_{\star})^{1/2} \\
    \wt{\nabla} \Psi_{\lambda}(A_t) := & ~ \lambda \cdot \sinh( X_t )  \\
    \wt{\Delta} \Psi_{\lambda}(A_t) := & ~ \lambda \cdot \cosh( X_t )  
\end{align*}

To maximize the gradient progress, we should choose
\begin{align*}
    A_{t+1} = A_t + \epsilon \cdot \nabla \Psi_{\lambda}(A_t) / \| \wt{\nabla} \Psi_{\lambda}(A_t) \|_F
\end{align*}

Then 
\begin{align}\label{eq:derivative_first_moment_A_t_F}
    \Delta_1 = & ~ (\epsilon \lambda^2) \cdot \tr[  \sinh^2( X_t ) ] / \| \wt{\nabla} \Psi_{\lambda}(A_t) \|_F \notag \\
    = & ~ \epsilon \cdot \| \wt{\nabla} \Psi_{\lambda}(A_t) \|_F^2 /  \| \wt{\nabla} \Psi_{\lambda}(A_t) \|_F \notag \\
    = & ~ \epsilon \cdot \| \wt{\nabla} \Psi_{\lambda}(A_t) \|_F 
\end{align}
and
\begin{align}\label{eq:derivative_second_moment_A_t_F}
    \Delta_2 = & ~ \epsilon ^2  \lambda^4 \cdot \tr[ \cosh( X_t ) \cdot \sinh^2( \lambda ( X_t )  ] / \| \wt{\nabla} \Psi_{\lambda}(A_t) \|_F^2 \notag \\
    = & ~ \epsilon^2 \lambda \cdot \tr[ \wt{\Delta} \Psi_{\lambda}(A_t) \cdot \wt{\nabla} \Psi_{\lambda}(A_t)^2 ] / \| \wt{\nabla} \Psi_{\lambda}(A_t) \|_F^2 \notag\\
    \leq & ~ \epsilon^2 \lambda \cdot \| \wt{\Delta} \Psi_{\lambda}(A_t) \|_F \cdot \| \wt{\nabla} \Psi_{\lambda}(A_t)^2 \|_F / \| \wt{\nabla} \Psi_{\lambda}(A_t) \|_F^2 \notag\\
    \leq & ~ \epsilon^2 \lambda \cdot \| \wt{\Delta} \Psi_{\lambda}(A_t) \|_F \notag\\
    \leq & ~ \epsilon^2 \lambda \cdot ( \lambda \sqrt{n} + \| \wt{\nabla} \Psi_{\lambda}(A_t) \|_F  )
\end{align}
where { the first step follows from the definition of $\Delta_2 $, the second step comes from the definition of $\wt{\Delta} \Psi_{\lambda}(A_t) $ and $\wt{\nabla} \Psi_{\lambda}(A_t)$,} { the third step follows that $\|A B\|_F \leq \|A\|_F \|B\|_F$,} the forth step follows from $\| x \|_4^2 \leq \| x \|_2^2$, and the fifth step follows from Part 1 of Lemma~\ref{lem:property_sinh_cosh_matrix}.


Now, we need to lower bound $\| \wt{\nabla} \Psi_{\lambda}(A_t) \|_F $, we have
\begin{align}\label{eq:derivative_phi_A_t_F}
    \| \wt{\nabla} \Psi_{\lambda}(A_t) \|_F = & ~ ( \tr[ \lambda^2 \sinh^2(X_t) ] )^{1/2} \notag \\
    \geq & ~ \frac{\lambda}{\sqrt{n}} ( \tr[  \cosh(X_t) ] -  n ) \notag \\
    = & ~ \frac{\lambda}{\sqrt{n}} ( \Psi_{\lambda}(A_t) -  n ) 
\end{align}
where the second step follows from Part 2 in Lemma~\ref{lem:property_sinh_cosh_matrix}.

We know that




Then, we have
\begin{align*}
    & ~ \Psi_{\lambda}(A_{t+1}) - \Psi_{\lambda}(A_t) \\
    \leq & ~ - \epsilon \| \wt{\nabla} \Psi_{\lambda}(A_t) \|_F + \epsilon^2 \lambda ( \sqrt{n} + \| \wt{\nabla} \Psi_{\lambda}(A_t) \|_F  ) \\
    \leq & ~ - 0.9 \epsilon  \| \wt{\nabla} \Psi_{\lambda}(A_t) \|_F + \epsilon^2 \lambda^2 \sqrt{n} \\
    \leq & ~ - 0.9 \epsilon \lambda \frac{1}{\sqrt{n}}  \Psi_{\lambda}(A_t)  + \epsilon \lambda \sqrt{n}
\end{align*}
where the first step {  follows from  Eq.~\eqref{eq:derivative_first_moment_A_t_F} and Eq.~\eqref{eq:derivative_second_moment_A_t_F} , the second steps comes from $\epsilon \in (0, 0.01)$, the third step comes from Eq.~\eqref{eq:derivative_phi_A_t_F} and $\epsilon \lambda \leq 1$. }


Finally, we complete the proof. 

\end{proof}


\begin{lemma}[Small spectral potential implies good spectral approximation]
Let $A\in \R^{n\times n}$ be symmetric, and $\lambda >0$. Suppose $\Psi_\lambda(A)\leq p$ for some $p > 1$. Then, we have
\begin{align*}
    (1-\delta) A_\star \preceq A \preceq (1+\delta)A_\star
\end{align*}
for $\delta = O(\lambda^{-1}\log p)$.
\end{lemma}

\begin{proof}
By the definition of $\Psi_\lambda(A)$,  $\Psi_\lambda(A)\leq p$ implies that for any $i\in [n]$,
\begin{align*}
    \cosh(\lambda(1-\lambda_i(A_\star^{-1/2}AA_\star^{-1/2})))\leq p,
\end{align*}
or equivalently,
\begin{align*}
    \left|(1-\lambda_i(A_\star^{-1/2}AA_\star^{-1/2}))\right|\leq O(\lambda^{-1}\log p).
\end{align*}
Hence, we have
\begin{align*}
    (1-\delta)I_n \preceq A_\star^{-1/2}AA_\star^{-1/2} \preceq (1+\delta)I_n,
\end{align*}
where $\delta := O(\lambda^{-1}\log p)$. Therefore, by multiplying $A_\star^{-1/2}$ on both sides, we get that
\begin{align*}
    (1-\delta) A_\star \preceq A \preceq (1+\delta)A_\star,
\end{align*}
which completes the proof of the lemma.
\end{proof}