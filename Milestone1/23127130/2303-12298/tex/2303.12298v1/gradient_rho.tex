\section{Gradient descent with General Measurements}\label{sec:gd_non_orthogonal}
  
In this section, we analyze the potential decay by gradient descent with non-orthogonal measurements.  The main result of this section is Lemma~\ref{lem:gradient_descent_rho} in below.
  


We first recall the definition of the potential function $\Phi_{\lambda}(A)$:
\begin{align*}
    \Phi_{\lambda}(A) := \sum_{i=1}^m \cosh ( \lambda ( u_i^\top A u_i - b_i ) ),
\end{align*}
its gradient $\nabla \Phi_{\lambda}(A)\in \R^{n\times n}$:
\begin{align}\label{eq:gradient_phi_A_rho}
    \nabla \Phi_{\lambda}(A) = \sum_{i=1}^m u_i u_i^\top \lambda \sinh\left( \lambda (u_i^\top A u_i - b_i)\right), 
\end{align}
and its Hessian $\nabla^2 \Phi_{\lambda}(A)\in \R^{n^2\times n^2}$:
\begin{align*}
    \nabla^2 \Phi_{\lambda}(A) = \sum_{i=1}^m ( u_i u_i^\top ) \otimes ( u_i u_i^\top ) \lambda^2 \cosh( \lambda (u_i^\top A u_i - b_i) ). 
\end{align*}


 


\begin{lemma}[Progress on entry-wise potential with general measurements]\label{lem:gradient_descent_rho}
Assume that $|u_i^{\top}  u_j| \leq \rho$ and $\rho \leq \frac{1}{10m}$, for any $i,j \in [m]$ and $\| u_i\|^2 = 1$. Let $c \in (0,1)$ denote a sufficiently small positive constant. Then, for any $\epsilon,\lambda>0$ such that $\epsilon\lambda \leq c$, 
we have for any $t>0$,
\begin{align*}
    \Phi_{\lambda} ( A_{t+1} ) \leq (1-0.9 \frac{ \lambda \epsilon }{\sqrt{m} }) \cdot \Phi_{\lambda} (A_t) +  \lambda \epsilon \sqrt{m}
\end{align*}

\end{lemma}

\begin{proof}
We first have
\begin{align}\label{eq:gd_define_Delta_1_and_Delta_2_rho}
    & ~ \Phi_{\lambda}(A_{t+1})- \Phi_{\lambda} (A_t) \notag \\
    \leq & ~ \langle \nabla \Phi_{\lambda} (A_t) , (A_{t+1} - A_t) \rangle 
    + O(1) \langle \nabla^2 \Phi_{\lambda}(A_t), (A_{t+1} - A_t) \otimes (A_{t+1} - A_t) \rangle \notag \\
    := & ~ - \Delta_1 + O(1) \cdot \Delta_2,
\end{align}
which follows from Corollary~\ref{cor:matrix_der_trace}.
 

We choose
\begin{align}\label{eq:A_t1_update_rho}
    A_{t+1} = A_t - \epsilon \cdot \nabla \Phi_{\lambda}(A_t) / \| \nabla \Phi_{\lambda}(A_t) \|_F.
\end{align}

We can bound
\begin{align}\label{eq:tr_phi_At_first_moment_rho} 
 \Delta_1 = & ~ -\tr[\nabla \Phi_{\lambda}(A_t) (A_{t+1} - A_t)] \notag \\
=  & ~ \epsilon \cdot \|\nabla \Phi_{\lambda} (A_t) \|_F.
\end{align}

For $\| \Phi_{\lambda} (A_t) \|_F^2$,
\begin{align}\label{eq:phi_At_F_norm_rho}
& ~ \frac{1}{\lambda^2} \| \nabla \Phi_{\lambda} (A_t) \|_F^2 \notag \\
= & ~   \tr[ (\sum_{i=1}^m u_i u_i^\top \sinh( \lambda( u_i^\top A_t u_i - b_i ) )  )^2 ] \notag\\
= & ~   \tr[ \sum_{i=1}^m  \sinh^2 ( \lambda( u_i^\top A_t u_i - b_i ) )  ] \notag \\
 + &~ \tr[ \sum_{i =1}^{m}\sum_{j \neq i}^{m} (u_i u_i^\top)(u_j u_j^{\top}) \sinh ( \lambda( u_i^\top A_t u_i - b_i ) ) \cdot \sinh ( \lambda( u_j^\top A_t u_j - b_j ) )]   \notag \\
 \geq &~0.9 \tr[ \sum_{i=1}^m  \sinh^2 ( \lambda( u_i^\top A_t u_i - b_i ) )] \notag \\
\geq & ~ 0.9\frac{1}{m} ( \sum_{i=1}^m \cosh ( \lambda ( u_i^\top A_t u_i - b_i ) ) - m )^2 \notag \\
= & ~ 0.9\frac{1}{m} ( \Phi_{\lambda} (A_t) - m )^2,
\end{align}
where the first step follows from Eq.~\eqref{eq:gradient_phi_A_rho},
the second steps follow from partitioning based on whether $i = j$ and $\| u_i \|_2 = 1$, the third step comes from Claim~\ref{claim:sum_ij_off_diagonal}, the fourth step in Eq.~\eqref{eq:phi_At_F_norm_rho} follows from Part 2 in Lemma~\ref{lem:property_sinh_cosh_scalar},  the fifth step follows from the definition of $\Phi_{\lambda}(A)$.





Thus,
\begin{align}\label{eq:bound_gd_Delta_1_rho}
   \Delta_1 = & ~ -\tr[\nabla \Phi_{\lambda}(A_t) (A_{t+1} - A_t)] \notag \\
   \geq & ~ \lambda \epsilon \cdot \frac{1}{ \sqrt{m} } ( \Phi_{\lambda}(A_t) - m )  .
\end{align}

For simplicity, we define
\begin{align*}
    z_{t,i} := \lambda (u_i^\top A_t u_i - b_i).
\end{align*}

We need to compute this $\Delta_2$. For simplificity, we consider $\Delta_2 \cdot (\frac{1}{\epsilon \lambda})^2 \cdot \| \nabla \Phi_{\lambda} (A_t) \|_F^2$, which can be expressed as:
\begin{align}\label{eq:tr_phi_At_second_moment_rho}
&\Delta_2 \cdot (\frac{1}{\epsilon \lambda})^2 \cdot \| \nabla \Phi_{\lambda} (A_t) \|_F^2\notag\\
     = & ~ \frac{1}{(\lambda \epsilon)^2} \tr[ \nabla^2 \Phi_{\lambda}(A_t) \cdot (A_{t+1} - A_t) \otimes (A_{t+1} - A_t) ] \cdot 
     \| \nabla \Phi_{\lambda}(A_t) \|_F^2 \notag\\
    = & ~  \tr\Big[ \nabla^2 \Phi_{\lambda}(A_t) \cdot 
    ( \sum_{i=1}^m u_i u_i^\top  \sinh( z_{t,i} ) ) \otimes ( \sum_{i=1}^m u_i u_i^\top \sinh( z_{t,i} ) )\Big] \notag\\
    = & ~ \tr[ \nabla^2 \Phi_{\lambda}(A_t)  ( \sum_{i,j}  \sinh( z_{t,i} )\sinh( z_{t,i} ) (u_i u_i^\top \otimes u_ju_j^\top) ) ] \notag\\
    = & ~ \tr[ \nabla^2 \Phi_{\lambda}(A_t)  ( \sum_{i=1}^m \sinh^2( z_{t,i} )) (u_i u_i^\top \otimes u_iu_i^\top) ) ] \notag\\
    + & ~  \tr[ \nabla^2 \Phi_{\lambda}(A_t)  ( \sum_{i\neq j} \sinh( z_{t,i} )\sinh( z_{t,j} ) (u_i u_i^\top \otimes u_j u_j^\top) ) ] \notag\\
    = & ~  Q_1 + Q_2  ,
\end{align}
where 
\begin{align}\label{eq:def_Q1_rho}
Q_1:= \tr\Big[ \nabla^2 \Phi_{\lambda}(A_t) \cdot ( \sum_{i=1}^m \sinh^2( z_{t,i} )) (u_i u_i^\top \otimes u_iu_i^\top) ) \Big]
\end{align}
denotes the diagonal term, and
\begin{align}\label{eq:def_Q2_rho}
    Q_2:=\tr\Big[ \nabla^2 \Phi_{\lambda}(A_t) \cdot ( \sum_{i\neq j} \sinh( z_{t,i} )\sinh( z_{t,j} ) (u_i u_i^\top \otimes u_j u_j^\top) ) \Big]
\end{align}
denotes the off-diagonal term. The first step comes from the definition of $\Delta_2$,
the second step follows from replacing $A_{t+1} - A_t$ using Eq.~\eqref{eq:A_t1_update_rho}, the third step follows that we extract the scalar values from Kronecker product, the fourth step comes from splitting into two partitions based on whether $i = j$, the fifth step comes from the definition of $Q_1$ and $Q_2$.




Thus,
\begin{align}\label{eq:bound_gd_Delta_2_rho}
    \Delta_2 \leq & ~ (\epsilon \lambda)^2 (Q_1 + Q_2) / \| \nabla \Phi_{\lambda}(A_t) \|_F^2 \notag \\
    = & ~ 1.3(\epsilon \lambda )^2 \cdot ( \sqrt{m} + \frac{1}{\lambda} \| \nabla \Phi_{\lambda}(A_t) \|_F ).
\end{align}
where the second step follows from  Claim~\ref{cla:gd_Q1_rho} and Claim~\ref{cla:gd_Q2_rho}.

Hence, we have
\begin{align*}
    & ~ \Phi_{\lambda} (A_{t+1}) - \Phi_{\lambda} (A_t) \\
    \leq & ~ - \Delta_1    + O(1) \cdot \Delta_2 \\
    \leq & ~ - \epsilon \| \nabla \Phi_{\lambda}(A_t) \|_F   + O(1) (\epsilon \lambda)^2 (\sqrt{m} + \frac{1}{\lambda} \| \nabla \Phi_{\lambda}(A_t) \|_F ) \\
    \leq & ~ - 0.9 \epsilon \| \Phi_{\lambda}(A_t) \|_F+ O(\epsilon \lambda)^2 \sqrt{m} \\
    \leq & ~ -0.9 \epsilon \lambda \frac{1}{\sqrt{m}} ( \Phi_{\lambda}(A_t) - m ) +O(\epsilon \lambda)^2 \sqrt{m} \\
    \leq & ~  -0.9 \epsilon \lambda \frac{1}{\sqrt{m}} \Phi_{\lambda}(A_t) + \epsilon \lambda \sqrt{m},
\end{align*}
where the first step follows from Eq.~\eqref{eq:gd_define_Delta_1_and_Delta_2_rho}, the second step follows from Eq.~\eqref{eq:bound_gd_Delta_1_rho} and Eq.~\eqref{eq:bound_gd_Delta_2_rho},  the third step follows from $\epsilon \lambda \in (0, 0.01)$, the fourth step follows from Lemma~\ref{lem:cosh_sinh}, and the final step follows that extracting the constant term from the summation.

The lemma is then proved.
\end{proof}

We prove some technical claims in below.

\begin{claim}\label{claim:sum_ij_off_diagonal}
It holds that:
\begin{align*}
    & \sum_{i\ne j\in [m]} \langle u_i,u_j\rangle^2 \sinh ( \lambda( u_i^\top A_t u_i - b_i ) ) \sinh ( \lambda( u_j^\top A_t u_j - b_j ) ) \\
    \leq &~ 0.1 \sum_{i=1}^m  \sinh^2 ( \lambda( u_i^\top A_t u_i - b_i ) ) 
\end{align*}
\end{claim}
\begin{proof}
We define $R_{i,j}$ and $R$ as follows:
\begin{align*}
    R_{i,j} = &~\sinh ( \lambda( u_i^\top A_t u_i - b_i ) )  \sinh ( \lambda( u_j^\top A_t u_j - b_j ) ) \\
     R = &~\tr[ \sum_{i =1}^{m}\sum_{j \neq i}^{m} (u_i u_i^\top)(u_j u_j^{\top}) \sinh ( \lambda( u_i^\top A_t u_i - b_i ) ) \cdot 
     \sinh ( \lambda( u_j^\top A_t u_j - b_j ) )]
\end{align*}

Then we can upper bound $|R|$ by:
\begin{align*}
    |R| =  &~ \tr[\sum_{i =1}^{m}\sum_{j \neq i}^{m} |(u_i u_i^\top)(u_j u_j^{\top})| |R_{i,j}|] \\
    \leq &~ \rho^2 \tr[\sum_{i =1}^{m}\sum_{j \neq i}^{m}  |R_{i,j}|] \\
    \leq &~  \frac{\rho^2 }{2}\tr[\sum_{i =1}^{m}\sum_{j \neq i}^{m} (R_{i,i} + R_{j,j})] \\
    \leq &~ m \rho^2 \tr[\sum_{i=1}^{m} R_{i,i} ] \\
    \leq &~ 0.1\tr[\sum_{i=1}^m R_{i,i}]
\end{align*}
where the first step follows $|ab| = |a||b|$, the second step follows $|u_i^{\top}  u_j| \leq \rho$, the third step follows that $|ab| \leq \frac{a^2 + b^2}{2}$, the fourth step follows from the summation over $j$, and the fifth step comes from $m\rho^2 \leq 0.1$.
\end{proof}

\begin{claim}\label{cla:gd_Q1_rho}
For $Q_1$ defined in Eq.~\eqref{eq:def_Q1_rho}, we have
\begin{align*}
    Q_1 \leq 1.1\Big( \sqrt{m} + \frac{1}{\lambda} \| \nabla \Phi_{\lambda}(A_t) \|_F \Big) \cdot  \| \nabla \Phi_{\lambda} (A_t) \|_F^2.
\end{align*}
\end{claim}
\begin{proof}


For simplicity, we define $z_{t,i}$ to be
\begin{align*}
    z_{t,i} := \lambda ( u_i^\top A_t u_i - b_i ) .
\end{align*}
Recall that
\begin{align*}
    \nabla^2 \Phi_{\lambda}(A_t) = \lambda^2 \cdot \sum_{i=1}^m ( u_i u_i^\top ) \otimes ( u_i u_i^\top ) \cosh( z_{t,i} ) .
\end{align*}

For $Q_1$, we have
\begin{align}\label{eq:Q1_rho}
   Q_1 = & ~  \tr[ \nabla^2 \Phi_{\lambda}(A_t) \sum_{i=1}^m  \sinh^2( z_{t,i} ) (u_i u_i^\top \otimes u_i u_i^\top) ) ] \notag\\
   = & ~ \lambda^2 \cdot \tr[ \sum_{i=1}^m  \cosh( z_{t,i} ) ( u_i u_i^\top ) \otimes ( u_i u_i^\top )  
   \cdot   \sum_{i=1}^m  \sinh^2( z_{t,i} ) (u_i u_i^\top ) \otimes ( u_i u_i^\top)   ] \notag\\
   = & ~ \lambda^2 \cdot \sum_{i=1}^m \tr[ \cosh( z_{t,i} ) \sinh^2( z_{t,i} )\cdot 
    ( u_i u_i^\top  u_i u_i^\top )  \otimes ( u_i u_i^\top u_i u_i^\top ) ] \notag\\
+ &~  \lambda^2 \cdot \sum_{i=1}^m \sum_{j \neq i}^{m} \tr[ \cosh( z_{t,i} )\sinh^2( z_{t,j} ) \cdot 
 (u_i u_i^\top u_j u_j^{\top} ) \otimes (u_j u_j^{\top} u_i u_i^\top)] \notag \\
   = & ~ \lambda^2 \cdot \sum_{i=1}^m  \cosh( z_{t,i} ) \sinh^2( z_{t,i} ) 
    \notag \\ 
   + & ~  \lambda^2 \cdot \sum_{i=1}^m \sum_{j \neq i}^{m} \tr[ \cosh( z_{t,i} )\sinh^2( z_{t,j} ) \cdot 
 (u_i u_i^\top u_j u_j^{\top} ) \otimes (u_j u_j^{\top} u_i u_i^\top)] \notag \\
   \leq & ~  1.1 \lambda ^2 \cdot (  \sum_{i=1}^m  \cosh^2( z_{t,i} ) )^{1/2}
   \cdot  ( \sum_{i=1}^m \sinh^4( z_{t,i} ) )^{1/2} \notag \\
   \leq & ~ 1.1 \lambda^2 \cdot B_1 \cdot B_2,
\end{align}
where {  the first step comes from the definition of $Q_1$, the second step comes from the definition of $\nabla^2 \Phi_{\lambda}(A_t)$,}
the third step follows from $(A \otimes B) \cdot (C \otimes D) = (AC) \otimes (BD)$ and partition the terms based on whether $i = j$, the fourth step comes from $\|u_i\| = 1$ and $\tr[ (u_i  u_i^\top) \otimes (u_i  u_i^\top) ] = 1$, and the fifth step comes from Cauchy–Schwarz inequality and Claim~\ref{claim:q1_helper_off_diagonal}.

\begin{claim}\label{claim:q1_helper_off_diagonal}
We can bound the off-diagonal entries by: 
\begin{align*}
  &~   |\lambda^2 \cdot \sum_{i=1}^m \sum_{j \neq i}^{m} \tr[ \cosh( z_{t,i} )\sinh^2( z_{t,j} ) \cdot 
  (u_i u_i^\top u_j u_j^{\top} ) \otimes (u_j u_j^{\top} u_i u_i^\top)]| \notag \\
 \leq &~ 0.1 \lambda^2(\sum_{i=1}^m(\cosh(z_{t,1}))^{1/2} \cdot (\sum_{i=1}^m \sinh^4( z_{t,i}))^{1/2}   \\  
\end{align*}
\end{claim}
\begin{proof}

\begin{align*}
      &~   |\lambda^2 \cdot \sum_{i=1}^m \sum_{j \neq i}^{m} \tr[ \cosh( z_{t,i} )\sinh^2( z_{t,j} ) \cdot 
       (u_i u_i^\top u_j u_j^{\top} ) \otimes (u_j u_j^{\top} u_i u_i^\top)]| \notag \\
      \leq &~ \rho^2 \lambda^2 |\sum_{i=1}^m \sum_{j \neq i}^{m} \cosh( z_{t,i} )\sinh^2( z_{t,j} )| \\
      \leq &~  \rho^2 \lambda^2 (\sum_{i=1}^m \sum_{j \neq i}^{m} (\cosh^2( z_{t,i} ))^{1/2} \cdot (\sum_{i=1}^m \sum_{j \neq i}^{m}\sinh^4( z_{t,j}))^{1/2} \\
      \leq &~ m\rho^2 \lambda^2  (\sum_{i=1}^m(\cosh^2(z_{t,i}))^{1/2} \cdot (\sum_{i=1}^m \sinh^4( z_{t,i}))^{1/2} \\
      \leq &~ 0.1 \lambda^2(\sum_{i=1}^m(\cosh^2(z_{t,i}))^{1/2} \cdot (\sum_{i=1}^m \sinh^4( z_{t,i}))^{1/2} 
\end{align*}
where the first step comes from $|\langle u_i, u_j\rangle | \leq \rho$, the second step comes from Cauchy–Schwarz inequality, the third step follows from summation over $m$ terms, and the fourth step comes from $\rho^2 m \leq 0.1$.
\end{proof}


For the term $B_1$, we have
\begin{align}\label{eq:b1_rho}
    B_1 = & ~ (  \sum_{i=1}^m \cosh^2( \lambda (u_i^\top A_t u_i - b_i ) ) )^{1/2} \notag \\
    \leq & ~ \sqrt{m} + \frac{1}{\lambda} \| \nabla \Phi_{\lambda}(A_t) \|_F,
\end{align}
where the second step follows Part 1 of Lemma~\ref{lem:property_sinh_cosh_scalar}.

For the term $B_2$, we have
\begin{align}\label{eq:b2_rho}
    B_2 = & ~( \sum_{i=1}^m \sinh^4( \lambda( u_i^\top A_t u_i - b_i ) ) )^{1/2} \notag  \\
    \leq & ~ \frac{1}{\lambda^2} \| \nabla \Phi_{\lambda} (A_t) \|_F^2,
\end{align}
where the second step follows from $\| x \|_4^2 \leq \| x \|_2^2$. This implies that
\begin{align*}
    Q_1 \leq & ~ 1.1\lambda^2 \cdot B_1 \cdot B_2 \\
    \leq & ~ 1.1\lambda^2 \cdot ( \sqrt{m} + \frac{1}{\lambda} \| \nabla \Phi_{\lambda}(A_t) \|_F ) \cdot \frac{1}{\lambda^2} \| \nabla \Phi_{\lambda} (A_t) \|_F^2 \\
    = & ~ 1.1( \sqrt{m} + \frac{1}{\lambda} \| \nabla \Phi_{\lambda}(A_t) \|_F ) \cdot  \| \nabla \Phi_{\lambda} (A_t) \|_F^2 .
\end{align*}
This completes the proof.
\end{proof}


\begin{claim}\label{cla:gd_Q2_rho} 
For $Q_2$ defined in Eq.~\eqref{eq:def_Q2_rho}, we have:
\begin{align*}
    Q_2 \leq 0.2 \lambda^2 ( \sqrt{m} + \frac{1}{\lambda} \| \nabla \Phi_{\lambda}(A_t) \|_F ) \cdot  \| \nabla \Phi_{\lambda} (A_t) \|_F^2 
\end{align*}
\end{claim}
\begin{proof}
Because in $Q_2$ we have :
\begin{align}\label{eq:u_ell_i_j_product_rho}
 Q_2 =  & ~ \lambda^2 \tr[\sum_{\ell=1}^{m}(\cosh(z_{t,\ell}) \cdot u_{\ell} u_{\ell}^{\top} \otimes u_{\ell} u_{\ell}^{\top}) \cdot 
 \sum_{i \neq j}^{m}(\sinh( z_{t,i} )\sinh( z_{t,j} ) \cdot u_i u_i^{\top} \otimes u_j u_j^{\top})] \notag \\
    = & ~ \lambda^2\tr[\sum_{\ell=1}^{m} \sum_{i \neq j}^{m} \cosh(z_{t,\ell} )\sinh( z_{t,i} )\sinh( z_{t,j} ) \cdot 
     (u_{\ell} u_{\ell}^{\top} u_i u_i^{\top}) \otimes (u_{\ell} u_{\ell}^{\top} u_j u_j^{\top})] \notag \\
    \leq & ~  \lambda^2\rho^2 \sum_{\ell=1}^{m} \sum_{i \neq j}^{m} \cosh(z_{t,\ell} )(\sinh^2( z_{t,i} ) + \sinh^2( z_{t,j} )) \notag \\
    \leq &~ 2m \lambda^2\rho^2 \sum_{\ell=1}^{m} \sum_{i = 1}^{m} \cosh(z_{t,\ell} )\sinh^2( z_{t,i} ) \notag \\
    \leq &~ 2m^2 \lambda^2\rho^2 \sum_{i = 1}^{m} \cosh(z_{t,i} )\sinh^2( z_{t,i} ) \notag \\
    \leq &~ 2m^2 \lambda^2\rho^2 (\sum_{i=1}^m(\cosh^2(z_{t,i}))^{1/2}  (\sum_{i=1}^m \sinh^4( z_{t,i}))^{1/2} \notag \\
    \leq &~ 0.2 \lambda^2 ( \sqrt{m} + \frac{1}{\lambda} \| \nabla \Phi_{\lambda}(A_t) \|_F ) \cdot  \| \nabla \Phi_{\lambda} (A_t) \|_F^2 
\end{align}
where the second step follows from $(A \otimes B) \cdot (C \otimes D) = (AC) \otimes (BD)$, the third step follows Cauchy–Schwarz inequality and $|\langle u_i, u_j\rangle | \leq \rho$, the fourth step follows from combining $\sinh^2( z_{t,i} )$ and $\sinh^2( z_{t,j} )$, the fifth step comes from summation over $m$ terms, and the sixth step comes from Cauchy–Schwarz inequality and the seventh step follows from Eq.~\eqref{eq:b1_rho} and Eq.~\eqref{eq:b2_rho} and $m^2 \rho^2 \leq 0.1$.

\end{proof}

