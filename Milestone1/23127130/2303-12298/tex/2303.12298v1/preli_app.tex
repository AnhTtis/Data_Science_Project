\paragraph{Roadmap.} We first provide the proofs for matrix hyperbolic functions and properties of $\sinh$ and $\cosh$ in Appendix~\ref{sec:preli_app}. Then we provide the  proofs for the gradient descent and stochastic gradient descent convergence analysis in Appendix~\ref{sec:gd_sgd_missing_proofs}. We consider the spectral potential function with ground-truth oracle scenario in Appendix~\ref{sec:potential_ground_truth_app}. We analyze the gradient descent with non-orthogonal measurements in Appendix~\ref{sec:gd_non_orthogonal}. We provide the cost-per-iteration analysis for stochastic gradient descent under non-orthogonal measurements in Appendix~\ref{sec:sgd_general}.

\section{Proofs of Preliminary Lemmas}\label{sec:preli_app}


\subsection{Calculus tools}
We state a useful calculus tool from prior work,
\begin{lemma}[Proposition 3.1 in \cite{jn08}]\label{lem:jn08}
Let $\Delta$ be an open interval on the axis, and $f$ be $C^2$ function on $\Delta$ such that for certain $\theta_{\pm}, \mu_{\pm} \in \R$ one has
\begin{align*}
    & ~ \forall (a<b, a,b \in \Delta) : \\
    & ~ \theta_- \cdot \frac{ f''(a) + f''(b) }{2} + \mu_- \leq \frac{f'(b) - f'(a)}{b-a} \\
    & ~ \frac{f'(b) - f'(a)}{b-a} \leq \theta_+ \cdot \frac{f''(a) + f''(b)}{2} + \mu_+,
\end{align*}
where $f'$ and $f''$ means the first- and second-order derivatives of $f$, respectively.

Let, further, ${\cal X}_n (\Delta)$ be the set of all $n \times n$ symmetric matrices with eigenvalues belonging to $\Delta$. Then ${\cal X}_n(\Delta)$ is an open convex set in the space $S^n$ of $n \times n$ symmetric matrices, the function
\begin{align*}
    F(X) = \tr [ f(X) ] : {\cal X}_n (\Delta) \rightarrow \R  
\end{align*}
is $C^2$, and for every $X \in {\cal X}_n (\Delta)$ and every $H \in S^n$ one has
\begin{align*}
    & ~ \theta_- \cdot \tr[ H f''(X) H ] + \mu_- \cdot \tr[ H^2 ] \leq D^2 F(X) [H,H] \\
   & ~ D^2 F(X) [H,H] \leq \theta_+ \cdot \tr[ H f''(X) H ] + \mu_+ \cdot \tr[H^2],
\end{align*}
where $D$ means directional derivative.
\end{lemma}

% \Danyang{Lianke, need a sentence here to smooth the logic flow.}
We will use below corollary to compute the trace with a map $f: \R\rightarrow \R $.
\begin{corollary}\label{cor:matrix_der_trace}
Let $f:\R\rightarrow \R$ be a $C^2$ function. Let $A$ and $B$ be two symmetric matrices. We have
\begin{align*}
    \tr[ f(A) ] 
   \leq & ~ \tr[ f(B) ] + \tr[ f'(B) ( A - B ) ] \\
    & ~ + O(1) \cdot \tr[ f''(B) ( A - B )^2 ]. 
\end{align*}
\end{corollary}
%\Ruizhe{check if this corollary is correct.}


% \subsection{Proof of Fact~\ref{fac:kronecker_product}}
\subsection{Kronecker product}
Suppose we have two matrice $A \in \R^{m \times n}$ and $B \in \R^{p \times q}$, we use $A \otimes B$ denote the Kronecker product:
\begin{align*}
    A \otimes B = \left[\begin{array}{ccc}
A_{1,1} {B} & \cdots & A_{1,n} {B} \\
\vdots & \ddots & \vdots \\
A_{m,1} {B} & \cdots & A_{m,n} {B}
\end{array}\right].
\end{align*}

We state a fact and delay the proof into Section~\ref{sec:preli_app}.
\begin{fact}\label{fac:kronecker_product}
Suppose we have two matrices $A \in \R^{m \times n}$ and $B \in \R^{n \times k}$, we have
\begin{align*}
    (A \otimes B) \cdot (C \otimes D) = (AC) \otimes (BD).
\end{align*}
\end{fact}

\begin{proof}
From the definition of Kronecker product we have:
\begin{align*}
    &~ (A \otimes B) \cdot (C \otimes D) \\
    = &~ {\left[\begin{array}{ccc}
A_{1,1} B & \ldots & A_{1,n} B \\
\vdots & \ddots & \vdots \\
A_{m,1} B & \ldots & A_{m,n} B
\end{array}\right]\left[\begin{array}{ccc}
C_{1,1} D & \ldots & C_{1,k} D \\
\vdots & \ddots & \vdots \\
C_{n,1} D & \ldots & C_{n,k} D
\end{array}\right] } \\
   = &~ \left[\begin{array}{ccc}
(\sum_{i=1}^{n}A_{1,i}C_{i,1}) {BD} & \cdots & (\sum_{i=1}^{n}A_{1,i}C_{i,k}) {BD} \\
\vdots & \ddots & \vdots \\
(\sum_{i=1}^{n}A_{m,i}C_{i,1}) {BD} & \cdots & (\sum_{i=1}^{n}A_{m,i}C_{i,k}){BD}
\end{array}\right] \\
= &~ \left[\begin{array}{ccc}
(AC)_{1,1} BD & \cdots & (AC)_{1,k} BD\\
\vdots & \ddots & \vdots \\
(AC)_{m,1} BD & \cdots & (AC)_{m,k}BD
\end{array}\right] \\
= &~ (AC) \otimes (BD)
\end{align*}
Thus we complete the proof.
\end{proof}


\subsection{Proof of \texorpdfstring{$\cosh(A)$}{} upper bound}\label{sec:cosh_bound_proof}

\begin{lemma}[Restatement of Lemma~\ref{lem:cosh_bound}]\label{lem:cosh_bound_app}
Let $A$ be a real symmetric matrix, then we have
\begin{align*}
    \|\cosh(A)\| = \cosh(\|A\|) \leq \tr[\cosh(A)].
\end{align*}
We also have $\|A\| \leq 1+\log(\tr[\cosh(A)])$.
\end{lemma}


\begin{proof}
Note that for each eigenvalue $\lambda$ of $A$, we know that it corresponds to $\cosh(\lambda)$ for $\cosh(A)$. The second inequality follows from the fact that $\cosh(A)$ is psd.

For the second part, we know that $\exp(x)/2\leq \cosh(x)$, hence, $\exp(\|A\|)/2\leq \cosh(\|A\|)$, and
\begin{align*}
    \|A\| = & ~ \log(\exp(\|A\|)) \\
    \leq & ~ \log(2\cosh(\|A\|)) \\
    \leq & ~ 1+\log(\tr[\cosh(A)]),
\end{align*}
where the second step is by the monotonicity of $\log(\cdot)$ and $\exp(\|A\|)\leq 2\cosh(\|A\|)$, the last step is by $\cosh(\|A\|)\leq \tr[\cosh(A)]$.
\end{proof}
 
We state a fact as follows:  
\begin{fact}\label{fact:cosh_sinh_1}
For any real number $x$, $ \cosh^2(x) - \sinh^2(x) =1$
\end{fact}
From the definition of $\cosh(x)$ and $\sinh(x)$ we have:
\begin{align*}
      &~ \cosh^2(x) - \sinh^2(x) \\
    = &~ \frac{1}{4}(e^{2x} + 2 + e^{-2x}) - \frac{1}{4}(e^{2x} - 2 + e^{-2x}) \\
    = &~ 1
\end{align*}

 
We also have the following lemma for matrix. 
\begin{lemma}\label{lem:cosh_sinh}
Let $A$ be a real symmetric matrix, then we have
\begin{align*}
    \cosh^2(A) - \sinh^2(A) = & ~ I.
\end{align*}
\end{lemma}
\begin{proof}
Since $A$ is real symmetric, we write it in the eigendecomposition form: $A=U\Lambda U^\top$, then
\begin{align*}
    & ~ \cosh^2(A)-\sinh^2(A) \\
    = & ~ U\cosh^2(\Lambda)U^\top-U\sinh^2(\Lambda)U^\top \\
    = & ~ U(\cosh^2(\Lambda)-\sinh^2(\Lambda))U^\top \\
    = & ~ UU^\top \\
    = & ~ I,
\end{align*}
where the first step follows from $\cosh$ and $\sinh$ can be expressed as $\exp$, the third step is by applying entrywise the identity $\cosh^2(x)-\sinh^2(x)=1$.
\end{proof}

 
 
