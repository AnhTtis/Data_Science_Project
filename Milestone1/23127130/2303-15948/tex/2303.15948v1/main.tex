\documentclass{article}

\usepackage{arxiv}

\usepackage[utf8]{inputenc} % allow utf-8 input
\usepackage[T1]{fontenc}    % use 8-bit T1 fonts
\usepackage{hyperref}       % hyperlinks
\usepackage{url}            % simple URL typesetting
\usepackage{booktabs}
\usepackage{tikz}
\usepackage{amsmath}
\usepackage{amsfonts}
\usepackage{mathtools}
\usepackage{natbib}
% \usepackage{lineno}
\usepackage{xcolor}
\usepackage[toc, page]{appendix}
% \linenumbers
\usetikzlibrary{external}
\tikzexternalize[prefix=tikz/]
% \setlength{\parindent}{0em}
% \setlength{\parskip}{1ex}
\newcommand{\Kuu}{\mathbf K_{uu}}
\newcommand{\Kuf}{\mathbf K_{uf}}
\newcommand{\Kfu}{\mathbf K_{uf}^\top}
\newcommand{\Kff}{\mathbf K_{ff}}
\renewcommand{\u}{\mathbf u}
\newcommand\given{\,|\,}
\newcommand{\m}{\mathbf m}
\newcommand{\bmu}{\mathbf \mu}
\renewcommand{\S}{\mathbf{S}}
\newcommand{\zero}{\mathbf 0}
\newcommand{\half}{\tfrac{1}{2}}
\newcommand{\tr}{\textrm{tr}}
\renewcommand{\L}{\mathbf L}
\newcommand{\R}{\mathbf R}
\newcommand{\I}{\mathbf{I}}
\newcommand{\A}{\mathbf A}
\newcommand{\B}{\mathbf B}
\newcommand{\F}{\mathbf F}
\renewcommand\vec{\textrm{vec}}
\renewcommand{\d}{\textrm{d}}
\newcommand{\KL}{\textsc{KL}}
\newcommand{\Yml}{\phi_\ell^m}
\newcommand{\Ymlp}{\phi_{\ell^\prime}^{m^\prime}}
\newcommand{\uml}{u_\ell^m}
\newcommand{\fml}{\hat{f}_{\ell,m}}
\newcommand{\Nld}{N(\ell,d)}
\newcommand{\diff}{\mathop{}\!\mathrm{d}}
\newcommand{\james}[1]{{gt\color{red}JH:#1}}


\title{Sparse Gaussian Processes with Spherical Harmonic Features Revisited}
\author{%
  Stefanos Eleftheriadis\thanks{Amazon.com}%
  \And Dominic Richards\footnotemark[1]%
  \And James Hensman\thanks{Work done while at Amazon; currently with Microsoft Research}%
  }


\begin{document}
\maketitle

\begin{abstract}
We revisit the Gaussian process model with spherical harmonic features and study connections between the associated RKHS, its eigenstructure, and deep models.
Based on this, we introduce a new class of kernels which correspond to deep models of continuous depth. 
In our formulation, depth can be estimated as a kernel hyper-parameter by optimizing the evidence lower bound. 
Further, we introduce sparseness in the eigenbasis by variational learning of the spherical harmonic phases.
This enables scaling to larger input dimensions than previously, while also allowing for learning of high frequency variations.
We validate our approach on machine learning benchmark datasets.
\end{abstract}

\section{Introduction}
\section{Introduction}

The increasing complexity of source code poses a key challenge to the reliability of large-scale software systems. Software bugs in these systems can lead to safety issues~\cite{bug_safety} for users around the world as well as cause non-negligible financial losses~\cite{bug_loss}. As such, developers have to spend a large amount of time and effort on bug fixing. Consequently, \aprfull (\apr), designed to automatically generate patches to fix software bugs, has attracted wide attention from both academia and industry~\cite{long2016prophet, legoues2012genprog, long2015spr, lou2020can, tufano2018empstudy}. 


To achieve \apr, one popular approach is known as Generate-and-Validate (G\&V)~\cite{qi2015gv, ghanbari2019prapr, lou2020can, le2016hdrepair, legoues2012genprog, wen2018capgen, hua2018sketchfix, martinez2016astor, koyuncu2020fixminder, liu2019tbar, liu2019avatar}, which is typically based on the following pipeline: First, fault localization techniques~\cite{wong2016fl, abreu2007ochiai, zhang2013injecting, papadakis2015metallaxis, li2019deepfl, li2017transforming} are applied to determine the suspicious locations in programs where bugs are likely to exist. Then, the buggy locations are used by the \apr tools to generate a list of patches that replace buggy lines with correct lines. Afterward, each patch is validated against the original test suite to identify any \emph{plausible patches} (i.e., passing all tests in the test suite). Finally, to determine the \emph{correct patches}, developers examine the list of plausible patches to see if any of them can correctly fix the bug. 

Traditional \apr tools can mainly be categorized into heuristic-based~\cite{legoues2012genprog, le2016hdrepair, wen2018capgen}, constraint-based~\cite{mechtaev2016angelix, le2017s3, demacro2014nopol, long2015spr} and \template~\cite{ghanbari2019prapr, hua2018sketchfix, martinez2016astor, liu2019tbar, liu2019avatar}. Among these traditional tools, \template \apr tools~\cite{ghanbari2019prapr, liu2019tbar, benton2020effectiveness} have been able to achieve state-of-the-art results. \Template \apr tools typically leverage pre-defined templates (e.g., adding a nullness check) for bug fixing. However, since these fix templates are typically handcrafted, the number and types of bugs they are able to fix can be limited. 



To address the limitations of traditional \apr, researchers have proposed various \learning \apr tools~\cite{li2020dlfix, chen2018sequencer, jiang2021cure, lutellier2020coconut, zhu2021recoder, ye2022rewardrepair} based on the \nmtfull (\nmt) architecture~\cite{sutskever2014mt} where the input is the buggy code snippets and the goal is to translate the buggy code snippets into a fixed version. To accomplish this, \learning \apr tools require supervised training datasets with pairs of both buggy and fixed code snippets in order to learn how to perform this translation step. These training data are usually obtained by mining historical bug fixes using heuristics/keywords~\cite{dallmeier2007benchmark}, which can be imprecise for identifying bug-fixing commits; even the actual bug-fixing commits can include irrelevant code changes, leading to further pollution in the dataset~\cite{xia2022alpharepair}.
% 
Moreover, it can be hard for such \apr tools to generalize and fix bug types unseen during training. 



To better leverage recent advances in \plmfull{s} (\plm{s}), researchers~\cite{xia2022alpharepair, xia2023repairstudy, kolak2022patch, prenner2021codexws} have directly applied \plm{s} to generate patches without bug-fixing datasets. These \llm-based \apr tools work by either directly generating a complete code function~\cite{prenner2021codexws, xia2023repairstudy} or predict/infill the correct code snippet given its surrounding context~\cite{xia2022alpharepair, xia2023repairstudy}. By directly using \llm{s} that are pre-trained on billions of open-source code snippets, \llm-based \apr tools can achieve state-of-the-art performance on many repair datasets~\cite{xia2022alpharepair}. 


% 
%
%

Traditional \apr tools have long used the insight of the \emph{plastic surgery hypothesis}~\cite{barr2014plastic} where it states that the code ingredients to fix a bug already exist within the same project. Traditional \apr tools have manually designed pattern-~\cite{ghanbari2019prapr, saha2017elixir} or heuristic-based~\cite{jiang2018simfix, legoues2012genprog} approaches to finding and using such relevant code ingredients to generate fixes for bugs. However, the plastic surgery hypothesis has been largely ignored in \llm-based \apr. In fact, \llm provides a unique opportunity to fully automate the plastic surgery hypothesis idea via fine-tuning (learning project-specific information via model updates from the buggy project) and prompting (directly providing relevant code ingredients to the model), and make it directly applicable to different languages (since the \llm{s} are typically multi-lingual).%
Moreover, despite the intensive manual efforts involved, traditional \apr tools still cannot fully leverage project-specific information due to large search space for leveraging/composing existing code ingredients. In contrast, the project-specific information can effectively leveraged by \llm{s} due to their power in code understanding/vectorization, e.g., even partial/imprecise information may still guide \llm{s} in correct patch generation!
 To this end, we ask the question: \emph{How useful is the plastic surgery hypothesis in the era of \plm{s}}?








\mypara{Our Work.} To answer the question, we present \ourtech{\xspace} -- a \llm-based approach that automatically utilizes the plastic surgery hypothesis by systematically combining multiple fine-tuning and prompting strategies for \apr. \ourtech fine-tunes \plm{s} using two novel domain-specific training strategies: \textbf{\epfinetune} -- we fine-tune using the original buggy project by aggressively masking out a high percentage of tokens, which allows \plm to learn project-specific code tokens and programming styles; and \textbf{\rofinetune} -- which only masks out a single continuous code sequence per training sample, allowing the model to get used to the final \csapr task of predicting a single continuous code sequence. Furthermore, we directly leverage the ability for \plm{s} to understand natural language instructions and introduce a novel prompting strategy, \textbf{\idprompting}, which uses information retrieval and static analysis to obtain a list of relevant identifiers for the buggy lines. While such relevant identifiers are critical for fixing some difficult bugs, they may not be seen by the \llm during inference due to limited context window size. Through the use of prompting, we directly tell the model to use these extracted identifiers (relevant code ingredients) to generate the correct code. Finally, to perform repair, we combine all four model variants (including the base model, both fine-tuned models and the base model with prompting) for the final repair.





While our insight of leveraging the plastic surgery hypothesis for \llm-based \apr is generalizable across different types of \plm{s}, to implement \ourtech, we choose a recent \plm{\xspace}, \ctfive~\cite{wang2021codet5}, which is pre-trained on millions of open-source code snippets. \ctfive is an encoder-decoder model trained using \mspfull (\msp) objective where a percentage of tokens are masked out and each continuous masked token sequence is referred to as a masked span. Also, although we only extract relevant identifiers from the current buggy project (since this paper focuses on the plastic surgery hypothesis), our work can be easily extended to obtain other code information (such as relevant statements or functions) from other sources, such as  the massive pre-training corpora~\cite{husain2020codesearchnet} or historical bug-fixing datasets~\cite{jiang2019infer}, which can provide more coding knowledge for \llm{s}. Besides, although we mainly focus on using traditional string comparison algorithms for information retrieval in this paper, these techniques can be easily replaced by other frequency-based retrieval~\cite{robertson2009probabilistic} and neural search (or embedding-based search)~\cite{reimers2019sentence}.
  In summary, this paper makes the following contributions:


%


\begin{itemize}[noitemsep, leftmargin=*, topsep=0pt]
    \item \textbf{Dimension.} This paper is the first to revisit the important plastic surgery hypothesis in the era of \llm{s}. It opens up a new dimension for \llm-based \apr to incorporate previously neglected information from the buggy project itself to boost \apr performance. Furthermore, it demonstrates the promising future of retrieval-based prompting for modern \llm-based \apr.
    \item \textbf{Implementation.} We implement \ourtech based on the recent \ctfive model. We augment the model using two novel fine-tuning strategies: \epfinetune and \rofinetune, along with a novel prompting strategy based on information retrieval and static analysis: \idprompting. We combine the patches generated by all four models together and perform patch ranking to speed up \apr.% 
    \item \textbf{Evaluation Study.} We conduct an extensive evaluation against state-of-the-art \apr tools. On the widely studied \dfj 1.2 and 2.0 datasets~\cite{just2014dfj}, \ourtech is able to achieve the new state-of-the-art results of 89 and 44 correct bug fixes (15 and 8 more than best baseline) respectively.  Furthermore, we perform a broad ablation study to justify our design. \ourtech demonstrates for the first time that the plastic surgery hypothesis can substantially boost \llm-based \apr and advance state-of-the-art \apr, while being fully automated and general. Moreover, even partial/imprecise code ingredients may still effectively guide \llm{s} for \apr!
\end{itemize}



\section{Deep learning with kernels on the sphere}
By definition, any function that can be factorised into a radial and an angular component,
must be acting on a hyper-sphere, i.e., it is a spherical function. For instance, let us take the typical Relu function defined as $\sigma_{\textrm{relu}}(\mathbf{x}^\top\mathbf{w}) = \max(0, \mathbf{x}^\top\mathbf{w})$, for some $\mathbf{w} \sim \mathcal N(\zero, \mathbf{I})$, $\mathbf{x} \in \mathbb R^d$. If we project the vectors onto the unit sphere $\mathbb S^{d-1}$ by normalising them, we can easily see that the Relu function is indeed spherical:
\begin{equation}
    \sigma_{\textrm{relu}}(\mathbf{x}^\top\mathbf{w}) = \|\mathbf{x}\| \|\mathbf{w}\|\max(0, \frac{\mathbf{x}^\top\mathbf{w}}{\|\mathbf{x}\| \|\mathbf{w}\|}) = \underbrace{\|\mathbf{x}\| \|\mathbf{w}\|}_{\textrm{radial}} \underbrace{\sigma_{\textrm{relu}}(\frac{\mathbf{x}^\top\mathbf{w}}{\|\mathbf{x}\| \|\mathbf{w}\|})}_{\textrm{angular}}\,.
\end{equation}

Based on this observation, \citet{cho2009kernel} studied the limit of infinite wide fully connected neural networks when the activation is a Relu function.
They found that the equivalent kernel takes the form of:
\begin{align}\label{eq:k_relu}
    k(\mathbf{x}, \mathbf{x}^\prime) = \mathbb E_\mathbf{w} \left[\sigma_{\textrm{relu}}(\mathbf{w}^\top\mathbf{x})\sigma_{\textrm{relu}}(\mathbf{w}^\top\mathbf{x}^\prime)\right]
    = \underbrace{\|\mathbf{x}\| \|\mathbf{x}^\prime\|}_{\textrm{radial}}
    \underbrace{\frac{1}{\pi}\left(t\left(\pi - \textrm{arccos}(t)\right) + \sqrt{1 - t^2}\right)}_{\textrm{angular}, \kappa(t)}\,,
\end{align}
with $t = \frac{\mathbf{x}^\top\mathbf{x}^\prime}{\|\mathbf{x}\| \|\mathbf{x}^\prime\|}$ and $\kappa(t)$ the {\em shape function} of the kernel.
We see that the above kernel is a bi-zonal function (zonal in either $\mathbf{x}$ or $\mathbf{x}'$).
This is an improtant observation, since zonal functions enjoy a particular relation with spherical harmonics, as we shall see in the next section.

\paragraph{Equivalent kernel of a deep network.} The above result can be extended to derive the equivalent kernel of a deep fully connected network with more than two layers.
Allowing the layers to be wide enough, i.e., infinite width we end up with the not
surprising result of:
\begin{equation}
    \kappa^L(t) \coloneqq \underbrace{\kappa \circ \cdots \circ \kappa}_{L-1 \textrm{ times}}(t).
\end{equation}
All we left to do is to rescale the shape function by multiplying with the appropriate radii so we have the final form of the equivalent kernel. Note that a good practice is to normalise $\kappa(1) = 1$, so that we also have $\kappa^L(1) = 1$.


\section{Introduction to spherical harmonics}
Spherical harmonics are functions defined on the surface of a sphere.
They arise as the solution of the angular part of Laplace's equation when expressed in spherical coordinates.
Given a point on the unit hyper-sphere, $\mathbf{x} \in \mathbb S^{d-1}$,
we write $\Yml(\mathbf{x})$ to denote the spherical harmonic of order (or frequency) $\ell$ and orientation (or phase) $m$, with $\ell \ge 0$ and $\lvert m\rvert \le \ell$.
Spherical harmonics are understood to be the generalization of a Fourier series to the sphere: the order $\ell$ is equivalently a frequency, and the orientations $m$ are phases. For a 2-sphere, the harmonics correspond precisely to sines and cosines of frequency $2\pi\ell$, and there are exactly two phases for any frequency (i.e. the sine and cosine part of the Fourier series). 

An important property of the spherical harmonics is that they form a complete, orthonormal basis on the sphere $\mathbb S^{d-1}$, which is embedded in the $d$-dimensional space $\mathbb R^d$. Therefore, they satisfy the property:
\begin{equation}
    \int_{\mathbb S^{d-1}} \Yml(\mathbf{x})\Ymlp(\mathbf{x}) \diff\Omega
     = \delta_{\ell\ell^\prime}\delta_{mm^\prime}\,,
\end{equation}
where $\delta$ is Kronecker's delta and $\Omega$ the surface of the $\mathbb S^{d-1}$ sphere. For a specific dimension $d\geq 3$ and order $\ell$ there exist
\begin{equation}
    \Nld = \frac{2\ell + d - 2}{\ell} \begin{pmatrix}
        \ell + d - 3\\
        d - 1
\end{pmatrix}
\end{equation}
linearly independent harmonics.

Any zonal function can be written as the linear combination of the spherical harmonics:
\begin{equation}\label{eq:zonal_f}
    f(\mathbf x^\top \mathbf{z}) = \sum_{\ell=0}^{\infty} \sum_{m=1}^{\Nld} \fml \Yml(\mathbf x)\,,
\end{equation}
where the coefficients $\fml$ are given by the Funk-Hecke formula:
\begin{equation}
    %\fml =  
    \int_{\mathbb S^{d-1}} f(\mathbf{x}^\top \mathbf{z})\,\Yml(\mathbf x) \diff\Omega = \lambda_\ell\Yml(\mathbf z) \eqqcolon \fml\,.
    \label{eq:eigval_f}
\end{equation}
We identify the terms $\fml$ as the Fourier coefficients of the function $f$, associated with the eigenfunctions $\Yml$, i.e., the spherical harmonics. The terms $\lambda_\ell$ are the eigenvalues, which as we will see in a next section do not depend on the orientation $m$.

 Another property of the spherical harmonics that will be proven useful in our analysis comes from the {\em addition} theorem. This states that for the spherical harmonics of degree $\ell$ in dimension $d$ the following holds:
 \begin{equation}\label{eq:addition_thm}
     \sum_{m=1}^{\Nld} \Yml(\mathbf{x})\Yml(\mathbf{x}^\prime) = \frac{\ell + \alpha}{\alpha}C_\ell^{(\alpha)}(\mathbf{x}^\top\mathbf{x}^\prime)\,,
\end{equation}
where $\alpha = \frac{d-2}{2}$ and $C_\ell^{(\alpha)}$ is the Gegenbauer polynomial of order $\ell$.

\subsection{Linear models on the spherical harmonic basis}
Spherical harmonics can be used as basis function in linear model in the same way as any other basis function. Since linear-Gaussian models can be written as Gaussian processes, it should be clear that linear-Gaussian models with spherical harmonic basis lead to Gaussian processes with spherical kernels. To illustrate, consider a linear model for a single frequency $\ell$ of the form $g(\mathbf{x}) = \sum_m w_m\Yml(\mathbf{x})$, with $\mathbf{x} \in \mathbb S^{d-1}$ and $w_m \sim \mathcal N(0, \lambda_\ell)$, where we explicitly pick $\lambda_\ell$ as the variance to make the connection obvious.
Taking the product of the function at two points $\mathbf x$ and $\mathbf x^\prime$ and integrating over $w_m$ we have:
\begin{align}\label{eq:dot_kernel}
    \mathbb E_{w_m}\left[g(\mathbf x)g(\mathbf x^\prime)\right] =
    %\mathbb E_{w_m}\left[ \sum_m^{\Nld} w_m \Yml(\mathbf{x})\sum_m^{\Nld} w_m \Yml(\mathbf{x}^\prime)\right] = 
    \sum_m^{\Nld} \mathbb E_{w_m}\left[w_m^2\right] \Yml(\mathbf{x})\Yml(\mathbf{x}^\prime) = 
    \lambda_\ell \frac{\ell + \alpha}{\alpha} C_\ell^{\alpha}(\mathbf{x}^\top\mathbf{x}^\prime)\,,
\end{align}
where in the first equality we used the independence between $w_m$ and in the second equality the addition theorem \eqref{eq:addition_thm}.

We recognize the right-hand-side of Eq.~\eqref{eq:dot_kernel} as a kernel.
With a closer look, we see that this kernel is a bi-zonal function, containing only the $\ell$th frequency, and can be written in the form of Eq.~\eqref{eq:zonal_f}. Now plugging Eq.~\eqref{eq:eigval_f} into Eq.~\eqref{eq:zonal_f} and applying the addition theorem from Eq.~\eqref{eq:addition_thm}, recovers the exact same kernel.
This gives rise to the reproducing property of the kernel space.
We continue our analysis by properly defining spherical kernels via the corresponding RKHS.

% This kernel is very similar to those that arise from (fully-connected) DNNs: these are also bi-zonal functions although they contain a range of frequencies. We continue our analysis by properly defining spherical kernels via the corresponding RKHS.

\subsection{From spherical functions to spherical kernels}
Following Mercer's theorem, the RKHS $\mathcal H$ associated to a zonal kernel is given by:
\begin{equation}
    \mathcal H = \{f: \sum_{\ell\ge 0} \sum_{m=1}^{\Nld} \fml\Yml(\cdot)\quad \textrm{s.t.}\quad\lvert\lvert f \rvert\rvert^2_\mathcal{H} \coloneqq \sum_{\ell\ge0, \lambda_\ell \neq 0}\sum_{m=1}^{\Nld}\frac{\fml^2}{\lambda_\ell} < \infty\}\,,
\end{equation}
where $\lambda_\ell$ is the eigenvalue of the kernel corresponding to the $\ell$th frequency.

To compute the eigenvalues $\lambda_\ell$ we first observe that for a given point $\mathbf{x^\prime} \in \mathbb S^{d-1}$, the kernel $k(\mathbf{x}, \mathbf{x}^\prime) : \mathbb S^{d-1} \rightarrow \mathbb R$ is a spherical function. As such, it can be factorised in a radial (i.e., the scale) and an angular (i.e., the associated shape function) component. Further, it can be written as a linear combination of the spherical harmonics
\begin{equation}\label{eq:kernel}
    k(\mathbf{x}, \mathbf{x}^\prime) \coloneqq r(\mathbf x, \mathbf x^\prime) \kappa(\mathbf{x}^\top\mathbf{x}^\prime) = r(\mathbf x, \mathbf x^\prime)\sum_{\ell=0}^{\infty} \sum_{m=0}^{\Nld} \lambda_\ell \Yml(\mathbf{x}) \Yml(\mathbf{x}^\prime)\,,
\end{equation}
where $\kappa(\cdot)$ is the shape function (i.e., the angular component) of the kernel.
Similarly to Eq.~\eqref{eq:eigval_f} we express $\lambda_\ell$
\begin{equation}
    \lambda_\ell \Yml(\mathbf x) =  
    \int_{\mathbb S^{d-1}} \kappa(\mathbf{x}^\top\mathbf{x}^\prime)\,\Yml(\mathbf{x}^\prime) \diff\Omega\,,
    \label{eq:eigval_k}
\end{equation}
and the $(d-1)$dimensional integral can be transformed to a one dimensional over the shape function via the Funk-Hecke formula to give us the eigenvalues:
\begin{equation}
   \lambda_\ell =  
    \frac{\omega_d}{C_\ell^{(\alpha)}(1)}\int_{-1}^1 \kappa(t) C_\ell^{(\alpha)}(t)(1 - t^2)^\frac{d-3}{2} \diff t\,,
    \label{eq:eigval_FH}
\end{equation}
$\omega_d$ is the surface of the sphere.

Combining Eqs.~\eqref{eq:kernel} and~\eqref{eq:addition_thm} enable us to write any spherical kernel as a polynomial expansion:
\begin{align}\label{eq:k_expanse}
    k(\mathbf{x}, \mathbf{x}^\prime) = r(\mathbf x, \mathbf x^\prime)
    \sum_{\ell=0}^{\infty} \frac{\ell + \alpha}{\alpha} \lambda_\ell C_\ell^{(\alpha)}(\mathbf{x}^\top\mathbf{x}^\prime)\,.
\end{align}



\section{Continuous depth kernels}
In the previous section we have seen how to construct equivalent kernels for fully connected deep neural networks, by simple composition of the shape function.
Here we introduce our approach on defining spherical kernels with continuous depth.

We begin by inspecting Eq.~\eqref{eq:k_relu} for the case of $L$ compositions of the shape function, $\kappa^L(\cdot)$.
If we are able to compute the eigenvalues of the deep Relu kernel, or any other kernel with a valid shape function, then we can write it as a polynomial expansion, similarly to Eq.~\eqref{eq:k_expanse}.
The behaviour of the eigenvalues for the spherical kernels has been the subject of interest in many studies recently~\citep{bietti2021deep,belfer2021spectral}. One key result is that the eigenvalues decay polynomially as we move to higher frequencies.
Moreover,~\citet{bietti2021deep} have shown that for different shape functions, the decay rate can depend or the depth.

Motivated from the above findings and looking again at Eq.~\eqref{eq:k_expanse} we can define a spherical kernel without having access to a specific shape function and without needing to compute the integral from Eq.~\eqref{eq:eigval_FH}, which obviously depends on the depth through the composed shape function.

We propose to use a kernel of the form:
\begin{align}\label{eq:k_poly}
    k(\mathbf{x}, \mathbf{x}^\prime) = r(\mathbf x, \mathbf x^\prime)
    \sum_{\ell=0}^{\infty} \frac{\ell + \alpha}{\alpha} {\ell}^{-\beta} C_\ell^{(\alpha)}(\mathbf{x}^\top\mathbf{x}^\prime)\,,
\end{align}
where we explicitly model the eigenvalues via the polynomial $\ell^{-\beta}$ and, the order of the polynomial $\beta > 0$ is a kernel hyper-parameter.
There are two important observations we need to state. First, the proposed kernel does not correspond directly to a known function; instead, it can be seen as a random spherical harmonic feature expansion, in analogy to the random Fourier features. Second, by learning the $\beta$ hyper-parameter we model the effect of depth in a continuous way; lower values for $\beta$ correspond to deeper kernels.

\begin{figure}[ht]
\includegraphics[scale=0.4, clip]{./figs/eigval_truncation.pdf}
\centering
    \caption{Eigenvalue comparison between the proposed kernel with continuous depth vs the NTK kernel with varying depth levels, for 3 dimensional problems (left) and 10 dimensional problems (right). Each eigenvalue is depicted relative to the corresponding eigenvalue of the first non-constant frequency.}
\label{fig:eigvals}
\end{figure}

\paragraph{Empirical study of the eigenvalue decay.} In Fig.~\ref{fig:eigvals} we compare the eigenvalue decay between the two approaches: (i) the deep NTK kernel~\citep{jacot2018neural} via compositions of the shape function; and (ii) the proposed polynomial expansion approach with continuous depth via the hyper-parameter $\beta$.
The first thing to notice is that the dimension of the problem plays an important role. Counter-intuitively, in higher dimensions (right panel), the high frequencies vanish rapidly, as even after only $10$ frequencies the decay is more than three orders of magnitude.
Further, by adding more depth the decay rate becomes slower in both low and high dimensions and the high frequencies become more relevant. The effect is especially pronounced in the low dimensional problems.
Finally, notice how the proposed kernel with continuous values for $\beta$ mimics the behaviour of NTK with increasing number of shape function compositions, i.e., depth.



\section{Spherical harmonics as features for GPs}
Here we leverage our knowledge on the spherical kernels to build efficient Gaussian process models on the associated RKHS. These models mimic the behavior of deep fully connected neural networks.

\subsection{Inter-domain GPs with spherical harmonics}\label{sec:variational_intro}
In their work~\citet{dutordoir2020sparse} have proposed a variational approach based on spherical harmonics to learn an approximation to the GP posterior.
To do so they introduced an inter-domain approach where the inducing features are the spherical harmonics. This results in diagonal covariance structure for the kernel via the Mercer's theorem and also allows to learn features with global structure compared to the local information of the traditional inducing points.
More specifically, they define inducing variables $\uml$ as the inner product between the GP function $f$ and the spherical harmonics:
\begin{equation}
    \uml = \langle f, \Yml \rangle_\mathcal{H}.
\end{equation}
Then, they use the reproducing property to compute the covariance between the function and the inducing features, i.e., $\left[k_{fu}(\mathbf x)\right]_{\ell,m}$ as:
\begin{equation}\label{eq:kuf}
    % \left[k_{fu}(\mathbf x)\right]_{\ell,m} = \mathbb E\left[f(\mathbf{x})\uml\right] = \langle k(\mathbf{x}, \cdot), \Yml \rangle_\mathcal{H} = \Yml(\mathbf{x})\,.
    \textrm{cov}\left[f(\mathbf x), \uml\right] = \mathbb E\left[f(\mathbf{x})\uml\right] = \langle k(\mathbf{x}, \cdot), \Yml \rangle_\mathcal{H} = \Yml(\mathbf{x})\,.
\end{equation}
Similarly for the covariance between the inducing features they obtain:
\begin{align}\label{eq:kuu}
    % \left[\mathbf{K}_{uu}\right]_{\ell,m,\ell^\prime,m^\prime} =
    \textrm{cov}\left[\uml, u_{\ell^\prime}^{m^\prime}\right] =
    \mathbb E\left[\uml u_{\ell^\prime}^{m^\prime}\right] =
    \langle \langle k(\cdot, \cdot), \Yml \rangle_\mathcal{H}, \Ymlp \rangle_\mathcal{H} = 
    \langle\Yml, \Ymlp \rangle_\mathcal{H} = 
    \frac{\delta_{\ell\ell^\prime}\delta_{mm^\prime}}{\lambda_\ell}\,,
\end{align}
which admits a diagonal structure for the kernel matrix $\mathbf{K}_{uu}$.

Plugging Eqs.~\eqref{eq:kuf},\eqref{eq:kuu} into the variational posterior from~\citep{hensman2013gaussian} leads to the approximation
\begin{align}
    q(f) = \mathcal{GP}\left(
    \mathbf{\Phi}^\top(\cdot)\mathbf{m},
    k(\cdot,\cdot) + \mathbf{\Phi}^\top(\cdot)(\mathbf{S} - \mathbf{K}_{uu})\mathbf{\Phi}(\cdot)
    \right)\,,
\end{align}
where $\mathbf{\Phi}(\cdot) = \{\lambda_\ell \Yml(\cdot)\}_{\ell,m}$ and $\mathbf{m}, \mathbf{S}$ are the mean and variance, respectively, of the variational distribution $q(\mathbf{u}) = \mathcal{N}(\mathbf m, \mathbf S)$.

To learn the model, one needs to optimise the evidence lower bound (ELBO) wrt the variational parameters and the kernel hyper-parameters
\begin{align}
    \textrm{ELBO} = \sum_{i=1}^N \mathbb E_{q(f)}\left[\log p(y_i\given f(\mathbf{x}_i)) \right]
     - \KL\left[q(\mathbf{u})\,||\,p(\mathbf{u})\right]\,,
\end{align}
where $\mathbf{y}$ is the output of the function we are trying to model, $p(y_i | f(\mathbf{x}_i))$ is the likelihood of choice and, $p(\mathbf{u}) = \mathcal{N}(\zero, \mathbf{K}_{uu})$ is the sparse GP prior. 

\subsection{Sparse features with phase truncation}
From a practical perspective, working with the method from~\citet{dutordoir2020sparse} requires to pick a truncation level $\hat{\ell}$ for the order/frequency of the spherical harmonics.
Then a set of points $\mathbf{V} = \{ \mathbf{v}_{\ell,m}\}_{\ell, m}, \mathbf{v}_{\ell,m}\in\mathbb S^{d-1}$, is chosen on the sphere via a Gram-Schmidt orthogonalisation on $\mathbf{V}_\ell^\top \mathbf{V}_\ell$, so that the points $\mathbf{V}_\ell = \{\mathbf{v}_{\ell,m} \}_{m=1}^{\Nld}$ are maximally separated and form a complete fundamental set. Then, $\{C_\ell^{(\alpha)}(\mathbf{V}_\ell^\top \mathbf{V}_\ell) \}_{\ell=0}^{\hat{\ell}}$ corresponds to the full set of spherical harmonic features up to order $\hat{\ell}$.
The variables $\mathbf{V}$ play the role of the inducing inputs and are kept fixed throughout optimization, as they are already optimally placed.

Although efficient, this approach has two limitations. First, the number of inducing points $M$, which is the total number of phases across all the chosen frequencies, scales exponentially with the number of dimensions. Second, most of the high-frequency components are explicitly ignored due to the truncation at a lower frequency $\hat{\ell}$.

To alleviate this, we propose to introduce an extra truncation $\hat{m}$, this time at the phase level of the spherical harmonics.
So instead of using all the harmonics within each frequency, we settle for a smaller number of basis functions. % so that we can scale to higher frequencies.
This practically allows us to truncate the frequencies at a much higher order $\hat{\ell}$, which results in features that capture more high frequency characteristics of the function.

A direct consequence of our proposed approach is that the frequencies that have been truncated in phase do not constitute a set of spherical harmonics any more. We ensure, however, that they remain orthogonal polynomials by explicitly orthogonalizing the corresponding $\mathbf{V}_\ell$ within each truncated frequency. Furthermore, we now have the option to optimize the phases of the truncated frequencies, as the $\mathbf{v}_{\ell,m}$ are variational parameters in our ELBO.

We call the features learned by this two-way truncation as {\em sparse} spherical harmonic features.




\section{Experiments}
We present in section~\ref{ssec:faces} an application of PnP-HVAE on face images, using a pretrained state-of-the-art hierarchical VAE. 
Next, we study the application of our framework to natural images. To that end, we introduce  in section~\ref{ssec:patchVDVAE}  a patch hierachical VAE architecture, that is able to model natural images of different resolutions. In section~\ref{ssec:app_nat}, we provide deblurring, super-resolution and inpainting experiments to demonstrate the relevance of the proposed method.

Additional results are presented in Appendix~\ref{app:add}. All experiments can be reproduced using the code available at \url{https://github.com/jprost76/PnP-HVAE}.



\subsection{Face Image restoration (FFHQ)}\label{ssec:faces}
We first demonstrate the effectiveness of PnP-HVAE on highly structured data, by performing face image restoration.
Latent variable generative models can accurately model structured images such as face images \cite{karras2019style,vahdat2020nvae,child2021very,kingma2018glow}, and then be used to produce high quality restoration of such data. 
In our experiments, we use the VDVAE model of~\cite{child2021very}, pre-trained on the FFHQ dataset~\cite{karras2019style}, as our hierarchical VAE prior.
VDVAE has $L=66$ latent variable groups in its hierarchy and generates images at resolution $256\times256$.

We compare PnP-HVAE with the intermediate layer optimization algorithm (ILO)~\cite{daras2021intermediate} that is based on a different class of generative models than HVAE. ILO is a GAN inversion method which optimizes the image latent code along with the intermediate layer representation of a StyleGAN to generate an image consistent with a degraded observation.
We use the official implementation of ILO, along with a StyleGAN2 model~\cite{karras2020analyzing, stylegan2pytorch}, that was trained for 550k iterations on images of resolution $256\times256$ from FFHQ.  
As VDVAE and StyleGAN models are not trained on the same train-test split of FFHQ, we chose to evaluate the methods on a subset of 100 images from the CelebA dataset~\cite{liu2018large}. 
For super-resolution, the degradation model corresponds to the application of a gaussian low-pass filter followed by a $\times 4$ sub-sampling, and the addition of a gaussian white noise with $\sigma=3$.
For the deblurring, we considered motion blur and  gaussian kernels, both with a noise level $\sigma=8$. %

We provide quantitative comparisons in table~\ref{table:comp_ILO}, along with a visual comparison of the results in figure~\ref{fig:face_restoration}.
PnP-HVAE has the best  PSNR and SSIM results for all the considered restoration tasks, while ILO provides better results  for the perceptual distance.
By jointly optimizing the image and its latent variable, PnP-HVAE provides  results that are both realistic and consistent with the degraded observation.
On the other hand,  ILO  only optimizes on an extended latent space. This method generates  sharp and realistic images with better LPIPS scores,   
but the results lack  of consistency with respect to the observation, which explains the overall lower PSNR performance. 






\subsection{PatchVDVAE: a HVAE for natural images}\label{ssec:patchVDVAE}
Available generative models in the literature operate on images of  fixed resolutions and
are either restrained to datasets of limited diversity, or even to registered face images~\cite{kingma2018glow,child2021very, vahdat2020nvae, karras2019style}, or requiring additional class information~\cite{brock2018large, dhariwal2021diffusion, song2020score, luhman2022optimizing}.
Fitting an unconditional model on natural images appears to be a more difficult task, as their resolution can change, and their content is highly diverse.
The complexity of the problem can be reduced by learning a prior model on patches of reduced dimension. 
For image restoration problems, the patch model can be reused on images of higher dimensions~\cite{zoran2011learning,prost2021learning,altekruger2022patchnr}. When the model is a full CNN, the prior on the set of the  patches can  be computed efficiently by applying the network on the full image~\cite{prost2021learning}.

We thus introduce  patchVDVAE, a fully convolutional hierarchical VAE.
Contrary to existing HVAE models whose resolution is constrained by the constant tensor at the input of the top-down block, patchVDVAE can generate images of different resolutions by controlling the dimension of the input latent. 
This amounts to defining a prior on patches whose dimension corresponds to the receptive field of the VAE. A similar model is used for image denoising in~\cite{prakash2021interpretable}.

 
For PatchVDVAE architecture, we use the same bottom-up and top-down blocks as VDVAE~\cite{child2021very}, and replace the constant trainable input in the first top-down block by a latent variable, to make the model fully convolutional (details on the  architecture are given in Appendix~\ref{app:details}). 
The training dataset is composed of $128\times 128$ patches extracted from a combination of DIV2K~\cite{agustsson2017ntire} and Flickr2K~\cite{Lim_2017_CVPR_workshops} datasets.
We perform data augmentation by extracting  patches at $3$ resolutions: HR-images and $\times 2$ and $\times 4$ downscaled images. 
The model is trained for $7.10^5$ iterations with a batch size of $64$. Following the recommendation of~\cite{hazami2022efficient}, we use Adamax optimizer with an exponential moving average and gradient smoothing of the variance.
We set the decoder model to be a gaussian with diagonal covariance, as in~\cite{luhman2022optimizing}.
PatchVDVAE is fully convolutional and can generate images of dimension that are multiples of $64$ as illustrated by
figure~\ref{fig:vdvae}.

\newlength{\patchwidth}
\setlength{\patchwidth}{0.135\columnwidth}
\begin{figure}[!ht]
    \centering
    \begin{subfigure}[t]{.34\columnwidth}\hspace{0.1cm}
        \setlength{\tabcolsep}{0.02pt}
\renewcommand{\arraystretch}{0}
        \begin{tabular}{*{2}{p{1.03\patchwidth}}}
            \includegraphics[width=\patchwidth]{figures_arxiv/patchVDVAE/samples/generated/64x64/setup-5-image-0018.png} &
            \includegraphics[width=\patchwidth]{figures_arxiv/patchVDVAE/samples/generated/64x64/setup-5-image-0016.png} \\
            \includegraphics[width=\patchwidth]{figures_arxiv/patchVDVAE/samples/generated/64x64/setup-5-image-0008.png} &
            \includegraphics[width=\patchwidth]{figures_arxiv/patchVDVAE/samples/generated/64x64/setup-5-image-0019.png}   
        \end{tabular}
    \end{subfigure}\hspace{-0.15cm}
    \begin{subfigure}[t]{.64\columnwidth}
\begin{tabular}{cc}\vspace{-0.1cm}
\includegraphics[width=2\patchwidth]{figures_arxiv/patchVDVAE/samples/generated/256x256/setup-2-image-0009.png}&
        \includegraphics[width=2\patchwidth]{figures_arxiv/patchVDVAE/samples/generated/256x256/setup-2-image-0002.png}\end{tabular}

    \end{subfigure}
    \caption{\label{fig:vdvae} Left: $64\times64$ patches samples from our patchVDVAE model trained on patches from natural images.
    Right: PatchVDVAE is fully convolutional and it can generate images of higher resolution (here: $128\times128$).\vspace{-0.2cm}}
\end{figure}

\subsection{Natural images restoration}\label{ssec:app_nat}
We  evaluate PnP-HVAE on natural image restoration.
For each task, we report the average value of the PSNR, the SSIM, and the LPIPS metrics on $20$ images from the test set of the BSD dataset~\cite{MartinFTM01}.\\


\noindent
{\bf Image deblurring.}
In the experiments, we consider $2$ gaussian kernels and $2$ motion blur kernels from~\cite{levin2009understanding}, with $3$ different noise levels 
$\sigma \in \{2.55, 7.65, 12.75\}$.
As a baseline we consider  EPLL~\cite{zoran2011learning}, which learns a prior on image patches with a gaussian mixture model.
We also compare PnP-HVAE  with PnP-MMO and GS-PnP, $2$ competing convergent Plug-and-Play methods based on CNN denoisers.
PnP-MMO~\cite{pesquet2021learning} restricts the denoiser to be contraction in order to guarantee the convergence of the PnP forward-backard algorithm. GS-PnP~\cite{hurault2022gradient} considers a gradient step denoiser and reaches state-of-the-art performances of non converging methods~\cite{zhang2021plug}.
We set the temperature $\tau$  in our method as $0.95$, $0.8$ and $0.6$ for noise levels $2.55$, $7.65$ and $12.75$ respectively, and we let it run for a maximum of $50$ iterations. 
For the three compared methods we use the official implementations and pre-trained models provided by the respective authors. 
Details on the choice of hyperparameters for the concurrent methods are provided in the Appendix~\ref{app:details}
Figure~\ref{fig:deblurring_bsd} illustrates that our method provides correct deblurring results. 

According to table~\ref{tab:deb}, the performance of PnP-HVAE is between those of EPLL and GS-PnP and it outperforms PnP-MMO for large noise levels.\\

\begin{table}
\begin{center}\footnotesize
    \begin{tabular}{>{\centering}m{.3cm}*{5}{c}}
    $\sigma$ &Method & PSNR$\uparrow$ & SSIM$\uparrow$ & LPIPS$\downarrow$  \\ 
    \hline
    \multirow{4}{*}{\vcell{$2.55$}}
    & PnP-HVAE & $27.75$ & $0.79$ & $0.31$\\
    & GS-PNP \cite{hurault2022gradient} & $\mathbf{29.59}$ & $\mathbf{0.84}$ & $\mathbf{0.22}$\\
    & EPLL \cite{zoran2011learning} & $26.49$ & $0.71$ & $0.36$\\ 
    & PnP-MMO \cite{pesquet2021learning} & $\underbar{29.50}$ & $\underbar{0.83}$ & $\underbar{0.20}$ \\ \hline
    \multirow{4}{*}{\vcell{$7.65$}}
    & PnP-HVAE & $\underbar{26.36}$ & $\underbar{0.72}$ & $\underbar{0.40}$\\
    & GS-PNP \cite{hurault2022gradient} & $\mathbf{27.33}$ & $\mathbf{0.77}$ & $\mathbf{0.31}$\\
    & EPLL \cite{zoran2011learning} & $24.04$ & $0.66$ & $0.45$ \\ 
    & PnP-MMO \cite{pesquet2021learning} & $25.34$ & $0.69$ & $0.34$\\
    \hline
    \multirow{4}{*}{\vcell{$12.75$}}
    & PnP-HVAE & $\underbar{25.12}$ & $\mathbf{0.73}$ & $\underbar{0.47}$\\
    & GS-PNP \cite{hurault2022gradient} & $\mathbf{26.32}$ & $\mathbf{0.73}$ & $\mathbf{0.37}$\\
    & EPLL \cite{zoran2011learning} & $23.28$ & $0.61$ & $0.51$ \\ 
    & PnP-MMO \cite{pesquet2021learning} & $22.42$ & $0.53$& $0.54$ \\
    \hline
    &\vspace*{-.3cm}\\
            \multicolumn{2}{c}{Blur and motion kernels}& \multicolumn{3}{c}{
        \includegraphics*[scale=1]{figures_arxiv/kernels/4.png}\;\includegraphics*[scale=1]{figures_arxiv/kernels/7.png}\;\includegraphics*[scale=1]{figures_arxiv/kernels/9.png}\;\includegraphics*[scale=1]{figures_arxiv/kernels/11.png}} 
    \end{tabular}
        \caption{\label{tab:deb}Comparison  of PnP-HVAE  and other restoration methods on deblurring. Results are averaged on $4$ kernels.\vspace{-0.2cm}}% on image deblurring.}
    \end{center}
\end{table}

\begin{figure}
    
    \begin{subfigure}[h]{\linewidth}
        \centering
        \includegraphics*[width=\columnwidth]{figures_arxiv/deb_s255_k7.pdf}\vspace{-0.1cm}
        \caption{Gaussian blur, $\sigma=2.55$}
    \end{subfigure}
    \begin{subfigure}[h]{\linewidth}
        \centering
        \includegraphics*[width=\columnwidth]{figures_arxiv/deb_s765_k11.pdf}\vspace{-0.1cm}
        \caption{Motion blur, $\sigma=7.65$}
    \end{subfigure}\vspace*{-0.1cm}
    \caption{\label{fig:deblurring_bsd} Natural image deblurring\vspace{-0.1cm}}
\end{figure}

\noindent {\bf Effect of the temperature.}
PnP-HVAE gives control on the temperature of the prior over the latent space.
In figure~\ref{fig:temp_effect}, we illustrate that reducing the temperature increases the strength of the regularization prior. In this example the tuning $\tau=0.7$ produces the best performance.\\
\begin{figure}[!ht]
   
    \includegraphics[width=\columnwidth]{figures_arxiv/demo_temp.pdf}\vspace{-0.15cm}
    \caption{ \label{fig:temp_effect} Effect of the temperature in PnP-VAE on a deblurring problem, with $\sigma=7.65$.\vspace{-0.15cm}}
\end{figure}


\noindent
{\bf Image inpainting.}
Next we consider the task of noisy image inpainting. 
We compose a test-set of 10 images from the validation set of BSD~\cite{MartinFTM01} and we create masks
  by occluding diverse objects of small size in the images. 
A gaussian white noise with $\sigma=3$ is added to the images.
As a comparaison, we still consider GS-PnP and EPLL.
For PnP-HVAE, the temperature is set to $\tau=0.6$, and the algorithm is run for a maximum of $200$ iterations, unless the residual $||\x_{k+1}-\x_k||$ is on a plateau.
We provide on Table~\ref{tab:inpainting_bsd} the distortion metrics with the ground truth, as well as a visual
\begin{table}



\begin{center}
    \begin{tabular}{cccc}
        & PSNR$\uparrow$ & SSIM$\uparrow$ &LPIPS$\downarrow$ \\\hline
        PnP-HVAE  & $\mathbf{29.54}$ & $\mathbf{0.93}$ & $\mathbf{0.06}$\\
        GS-PNP & $28.52$ & $\mathbf{0.93}$ & $0.09$\\
        EPLL & $\underline{29.16}$ & $\mathbf{0.93}$ & $\mathbf{0.06}$\\
    \end{tabular}
    \caption{\label{tab:inpainting_bsd}Quantitative evaluation for inpainting on BSD.}
    \end{center}
\end{table}
comparison on figure~\ref{fig:inpainting_bsd}. 
With its hierarchical structure,  PnP-HVAE outperforms the compared methods. \vspace{0.05cm}



\begin{figure}[!h]
    \includegraphics[width=\columnwidth]{figures_arxiv/demo_inp_bsd2.pdf}\vspace{-0.1cm}
    \caption{\label{fig:inpainting_bsd}Natural image inpainting\vspace{-0.3cm}}
\end{figure}












\section{Conclusions}
In this work we revisited the prior work on sparse Gaussian processes with spherical harmonics 
to solidify the understanding of the connection between deep models and spherical functions.
Specifically, we introduced a new kernel which corresponds to deep models of continuous depth and we further proposed to variationally learn the phases of spherical harmonic features, which results in
a more efficient set of global descriptors with high frequency components.
Our experimental evaluations on standard machine learning benchmarks verify the efficacy of the proposed approach.

\bibliographystyle{apalike}
\bibliography{refs}

\end{document}
