Spherical harmonics are functions defined on the surface of a sphere.
They arise as the solution of the angular part of Laplace's equation when expressed in spherical coordinates.
Given a point on the unit hyper-sphere, $\mathbf{x} \in \mathbb S^{d-1}$,
we write $\Yml(\mathbf{x})$ to denote the spherical harmonic of order (or frequency) $\ell$ and orientation (or phase) $m$, with $\ell \ge 0$ and $\lvert m\rvert \le \ell$.
Spherical harmonics are understood to be the generalization of a Fourier series to the sphere: the order $\ell$ is equivalently a frequency, and the orientations $m$ are phases. For a 2-sphere, the harmonics correspond precisely to sines and cosines of frequency $2\pi\ell$, and there are exactly two phases for any frequency (i.e. the sine and cosine part of the Fourier series). 

An important property of the spherical harmonics is that they form a complete, orthonormal basis on the sphere $\mathbb S^{d-1}$, which is embedded in the $d$-dimensional space $\mathbb R^d$. Therefore, they satisfy the property:
\begin{equation}
    \int_{\mathbb S^{d-1}} \Yml(\mathbf{x})\Ymlp(\mathbf{x}) \diff\Omega
     = \delta_{\ell\ell^\prime}\delta_{mm^\prime}\,,
\end{equation}
where $\delta$ is Kronecker's delta and $\Omega$ the surface of the $\mathbb S^{d-1}$ sphere. For a specific dimension $d\geq 3$ and order $\ell$ there exist
\begin{equation}
    \Nld = \frac{2\ell + d - 2}{\ell} \begin{pmatrix}
        \ell + d - 3\\
        d - 1
\end{pmatrix}
\end{equation}
linearly independent harmonics.

Any zonal function can be written as the linear combination of the spherical harmonics:
\begin{equation}\label{eq:zonal_f}
    f(\mathbf x^\top \mathbf{z}) = \sum_{\ell=0}^{\infty} \sum_{m=1}^{\Nld} \fml \Yml(\mathbf x)\,,
\end{equation}
where the coefficients $\fml$ are given by the Funk-Hecke formula:
\begin{equation}
    %\fml =  
    \int_{\mathbb S^{d-1}} f(\mathbf{x}^\top \mathbf{z})\,\Yml(\mathbf x) \diff\Omega = \lambda_\ell\Yml(\mathbf z) \eqqcolon \fml\,.
    \label{eq:eigval_f}
\end{equation}
We identify the terms $\fml$ as the Fourier coefficients of the function $f$, associated with the eigenfunctions $\Yml$, i.e., the spherical harmonics. The terms $\lambda_\ell$ are the eigenvalues, which as we will see in a next section do not depend on the orientation $m$.

 Another property of the spherical harmonics that will be proven useful in our analysis comes from the {\em addition} theorem. This states that for the spherical harmonics of degree $\ell$ in dimension $d$ the following holds:
 \begin{equation}\label{eq:addition_thm}
     \sum_{m=1}^{\Nld} \Yml(\mathbf{x})\Yml(\mathbf{x}^\prime) = \frac{\ell + \alpha}{\alpha}C_\ell^{(\alpha)}(\mathbf{x}^\top\mathbf{x}^\prime)\,,
\end{equation}
where $\alpha = \frac{d-2}{2}$ and $C_\ell^{(\alpha)}$ is the Gegenbauer polynomial of order $\ell$.

\subsection{Linear models on the spherical harmonic basis}
Spherical harmonics can be used as basis function in linear model in the same way as any other basis function. Since linear-Gaussian models can be written as Gaussian processes, it should be clear that linear-Gaussian models with spherical harmonic basis lead to Gaussian processes with spherical kernels. To illustrate, consider a linear model for a single frequency $\ell$ of the form $g(\mathbf{x}) = \sum_m w_m\Yml(\mathbf{x})$, with $\mathbf{x} \in \mathbb S^{d-1}$ and $w_m \sim \mathcal N(0, \lambda_\ell)$, where we explicitly pick $\lambda_\ell$ as the variance to make the connection obvious.
Taking the product of the function at two points $\mathbf x$ and $\mathbf x^\prime$ and integrating over $w_m$ we have:
\begin{align}\label{eq:dot_kernel}
    \mathbb E_{w_m}\left[g(\mathbf x)g(\mathbf x^\prime)\right] =
    %\mathbb E_{w_m}\left[ \sum_m^{\Nld} w_m \Yml(\mathbf{x})\sum_m^{\Nld} w_m \Yml(\mathbf{x}^\prime)\right] = 
    \sum_m^{\Nld} \mathbb E_{w_m}\left[w_m^2\right] \Yml(\mathbf{x})\Yml(\mathbf{x}^\prime) = 
    \lambda_\ell \frac{\ell + \alpha}{\alpha} C_\ell^{\alpha}(\mathbf{x}^\top\mathbf{x}^\prime)\,,
\end{align}
where in the first equality we used the independence between $w_m$ and in the second equality the addition theorem \eqref{eq:addition_thm}.

We recognize the right-hand-side of Eq.~\eqref{eq:dot_kernel} as a kernel.
With a closer look, we see that this kernel is a bi-zonal function, containing only the $\ell$th frequency, and can be written in the form of Eq.~\eqref{eq:zonal_f}. Now plugging Eq.~\eqref{eq:eigval_f} into Eq.~\eqref{eq:zonal_f} and applying the addition theorem from Eq.~\eqref{eq:addition_thm}, recovers the exact same kernel.
This gives rise to the reproducing property of the kernel space.
We continue our analysis by properly defining spherical kernels via the corresponding RKHS.

% This kernel is very similar to those that arise from (fully-connected) DNNs: these are also bi-zonal functions although they contain a range of frequencies. We continue our analysis by properly defining spherical kernels via the corresponding RKHS.

\subsection{From spherical functions to spherical kernels}
Following Mercer's theorem, the RKHS $\mathcal H$ associated to a zonal kernel is given by:
\begin{equation}
    \mathcal H = \{f: \sum_{\ell\ge 0} \sum_{m=1}^{\Nld} \fml\Yml(\cdot)\quad \textrm{s.t.}\quad\lvert\lvert f \rvert\rvert^2_\mathcal{H} \coloneqq \sum_{\ell\ge0, \lambda_\ell \neq 0}\sum_{m=1}^{\Nld}\frac{\fml^2}{\lambda_\ell} < \infty\}\,,
\end{equation}
where $\lambda_\ell$ is the eigenvalue of the kernel corresponding to the $\ell$th frequency.

To compute the eigenvalues $\lambda_\ell$ we first observe that for a given point $\mathbf{x^\prime} \in \mathbb S^{d-1}$, the kernel $k(\mathbf{x}, \mathbf{x}^\prime) : \mathbb S^{d-1} \rightarrow \mathbb R$ is a spherical function. As such, it can be factorised in a radial (i.e., the scale) and an angular (i.e., the associated shape function) component. Further, it can be written as a linear combination of the spherical harmonics
\begin{equation}\label{eq:kernel}
    k(\mathbf{x}, \mathbf{x}^\prime) \coloneqq r(\mathbf x, \mathbf x^\prime) \kappa(\mathbf{x}^\top\mathbf{x}^\prime) = r(\mathbf x, \mathbf x^\prime)\sum_{\ell=0}^{\infty} \sum_{m=0}^{\Nld} \lambda_\ell \Yml(\mathbf{x}) \Yml(\mathbf{x}^\prime)\,,
\end{equation}
where $\kappa(\cdot)$ is the shape function (i.e., the angular component) of the kernel.
Similarly to Eq.~\eqref{eq:eigval_f} we express $\lambda_\ell$
\begin{equation}
    \lambda_\ell \Yml(\mathbf x) =  
    \int_{\mathbb S^{d-1}} \kappa(\mathbf{x}^\top\mathbf{x}^\prime)\,\Yml(\mathbf{x}^\prime) \diff\Omega\,,
    \label{eq:eigval_k}
\end{equation}
and the $(d-1)$dimensional integral can be transformed to a one dimensional over the shape function via the Funk-Hecke formula to give us the eigenvalues:
\begin{equation}
   \lambda_\ell =  
    \frac{\omega_d}{C_\ell^{(\alpha)}(1)}\int_{-1}^1 \kappa(t) C_\ell^{(\alpha)}(t)(1 - t^2)^\frac{d-3}{2} \diff t\,,
    \label{eq:eigval_FH}
\end{equation}
$\omega_d$ is the surface of the sphere.

Combining Eqs.~\eqref{eq:kernel} and~\eqref{eq:addition_thm} enable us to write any spherical kernel as a polynomial expansion:
\begin{align}\label{eq:k_expanse}
    k(\mathbf{x}, \mathbf{x}^\prime) = r(\mathbf x, \mathbf x^\prime)
    \sum_{\ell=0}^{\infty} \frac{\ell + \alpha}{\alpha} \lambda_\ell C_\ell^{(\alpha)}(\mathbf{x}^\top\mathbf{x}^\prime)\,.
\end{align}

