\subsection{Experiment on UCI regression benchmarks}
We continue our evaluations on the \textit{song}~\citep{bertin2011million}, \textit{buzz} and \textit{houseelectric} datasets from the UCI corpus~\citep{dua2019}.
We follow the same experimental setup as in~\citep{sun2021scalable} and randomly choose $20\%$ of the data points as test set and repeat across $3$ splits.
We compare the sparse spherical harmonic features with $15$ frequencies and phase truncation of $100$ ($\sim1500$ inducing features), to the model with all the harmonics with $5$ frequencies and no phase truncation ($\sim1400$--$1700$ inducing features). Both models are trained with the polynomial decay kernel. Fig.~\ref{fig:uci} summarizes the results in terms of the test negative log-likelihood (NLL) and the root mean squared error (RMSE). It is worth noting that the sparse spherical features not only outperform the classic spherical harmonics but they also achieve superior performance compared to~\citep{sun2021scalable} which needs $16K$ inducing features.

\begin{figure}[ht]
\includegraphics[scale=0.4, clip]{./figs/test_uci.pdf}
\centering
    \caption{Test negative log-likelihood and root mean squared error on regression benchmarks. The sparse spherical harmonic features outperform the full harmonics most of the time.}
\label{fig:uci}
\end{figure}
