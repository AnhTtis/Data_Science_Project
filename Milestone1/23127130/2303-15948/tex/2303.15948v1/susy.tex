\subsection{Experiment on SUSY classification}
We first demonstrate the effectiveness of the proposed approach on a large-scale classification problem.
To do so, we use the SUSY dataset to classify whether we can detect from the result of a simulation if super-symmetric (SUSY) particles have been produced or not.
We follow a similar experimental setup as~\citet{dutordoir2020sparse} and we use the last 10\% out of the 5 million records in the dataset to test our model.
The inputs to the model consist of eight kinematic properties measured by the
particle detectors in the accelerator.
To train the models we used $7$ frequencies and with a truncation at $100$ phases. This results to approximately $500$ inducing features.
In Table~\ref{tab:susy}, we report the results in terms of the AuC score and we compare against
a custom 5 layer neural network architecture from~\citep{baldi2014searching}.
We see that our single layer GP performs similarly to a DNN without any hassle on picking an architecture, or deciding on the size of the depth.
For reference we have also included the performance of the classic spherical harmonics features for sparse GPs from~\citep{dutordoir2020sparse}, which we have trained with $5$ frequencies with no phase truncation (results again in approximately $500$ inducing features).
It is important to note that the GP with the proposed sparse spherical features have reached the performance of~\citep{dutordoir2020sparse} in half the iterations and then the AuC continued to improve.

\begin{table*}[tbh]
    \caption{Performance comparison on the SUSY dataset.}
  \label{tab:susy}
  \centering
  \begin{tabular}{lc}
    \toprule
    Method  & AuC\\
    \midrule
      ours & $0.870$\\
      \citet{dutordoir2020sparse} & $0.864$\\
      \citet{baldi2014searching} & $0.867$\\
    \bottomrule
  \end{tabular}
\end{table*}

