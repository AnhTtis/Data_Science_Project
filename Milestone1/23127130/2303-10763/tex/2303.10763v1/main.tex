\documentclass[10pt,conference]{IEEEtran}

\usepackage{tabularx}
\usepackage[utf8]{inputenc}
\usepackage{multirow}
\usepackage{soul}
\usepackage{xcolor}
\usepackage{xspace}
\usepackage{url}
\usepackage{graphicx}
\usepackage{listings,multicol}
\usepackage[caption=false]{subfig}  
\usepackage[export]{adjustbox}
\usepackage{amsmath}
\usepackage{textcomp}
\usepackage{listings}
\usepackage{hyperref}
\usepackage{cleveref}
\usepackage{algorithm}
\usepackage{enumitem}
\usepackage{tcolorbox}
\usepackage{booktabs}
\usepackage{siunitx}
\usepackage{pifont}
\usepackage[noend]{algpseudocode}
\usepackage{balance}
\usepackage[noadjust]{cite}



\PassOptionsToPackage{svgnames}{xcolor}

\DeclareRobustCommand{\hlgreen}[1]{{\sethlcolor{green}\hl{#1}}}
\newcommand{\todo}[1]{\hl{``#1''}}
\newcommand{\fixed}[1]{\hlgreen{``#1''}}
\newcommand{\ie}{\emph{i.e.,}\xspace}
\newcommand{\eg}{\emph{e.g.,}\xspace}
\newcommand{\ct}{~\hl{[]}\xspace}
\newcommand{\dataset}{\textsc{ConPlag}\xspace}
\newcommand{\stats}[1]{\textcolor{gray}{#1}}

\title{Towards a Dataset of Programming \\ Contest Plagiarism in Java}

\author{
 \IEEEauthorblockN{Evgeniy Slobodkin, Alexander Sadovnikov}
    \IEEEauthorblockA{\textit{Sirius.Courses, Moscow, Russia}}
    \IEEEauthorblockA{\{slobodkin.es, sadovnikov.av\}@talantiuspeh.ru}
}

\begin{document}


\maketitle

\begin{abstract}

In this paper, we describe and present the first dataset of source code plagiarism specifically aimed at contest plagiarism. The dataset contains 251 pairs of plagiarized solutions of competitive programming tasks in Java, as well as 660 non-plagiarized ones, however, the described approach can be used to extend the dataset in the future. Importantly, each pair comes in two versions: (a) ``raw'' and (b) with participants' repeated template code removed, allowing for evaluating tools in different settings. We used the collected dataset to compare the available source code plagiarism detection tools, including state-of-the-art ones, specifically in their ability to detect contest plagiarism. Our results indicate that the tools show significantly worse performance on the contest plagiarism because of the template code and the presence of other misleadingly similar code. Of the tested tools, token-based ones demonstrated the best performance in both variants of the dataset.

\end{abstract}

\section{introduction}

% 1. importance of TKGs and reasoning on TKGs. 
% 2. low resource languages, main main idea.
% 3. relations and limitations of current works.
% 4. summarize our solutions and contributions.

Temporal Knowledge Graphs (TKGs)~\cite{YAGO,ICEWS18,WIKI,acekg} characterize temporally evolving events, where each event, represented as ({\em subject}, {\em relation}, {\em object}), is associated with temporal information ({\em time}), e.g., ({\em Macron}, {\em reelected}, {\em French president}, {\em 2022}). TKGs has facilitated various knowledge-intensive Web applications with timeliness, such as question answering~\cite{KBQA}, product recommendation~\cite{RippleNet,TKG4Rec,TKG4Rec2,RETE}, and social event forecasting~\cite{KG4Social,DyDiff-VAE,andgan,belief,misinfo,polarization}. 

As new events are continually emerging, modern TKGs are still far from being complete. Conventionally, the TKG construction process relies primarily on information extraction from unstructured corpus~\cite{WIKI,YAGO, EventKG}, which necessitates extensive manual annotations to keep up with changing events. For instance, the recent transition from Trump to Biden as the President of the United States has not been reflected in many TKGs, highlighting the need for timely updates. This spurs research on temporal knowledge graph reasoning to automate evolving events prediction over time~\cite{TA-DistMult,Know-Evolve,Renet,RE-GCN}. Unfortunately, the problem of TKG incompleteness is particularly pronounced in low-resource languages, where it is unable to collect enough corpus and annotations to support robust TKG construction. This results in suboptimal reasoning performance and distinctly unsatisfying accuracy in predicting recent and future events.

% whose performance can degrade significantly in low-resource language TKGs that suffer from severe incompleteness over time. 
% \jingfeng{why don't people  study cross-lingual TKG previously, (i.e. use language alignment to improve TKG). Is it really helpful intuitively to use high resource language to help TKGC? For instance, is it enough to use static langauge-alignment to help KGC, ignoring the temporal information? Are those langauge-alignment changing across time?}



\begin{figure}
    \centering
    \includegraphics[width = 1.0\linewidth]{fig/task.pdf}
    \caption{An illustrative example of cross-lingual reasoning on TKGs. 1) We aim to transfer knowledge from English TKG to Japanese TKG, where the English version provides more complete information; 2) Cross-lingual alignments only cover a small ratio of entities, e.g., Apple Inc; 3) Cross-lingual alignments can be noisy and misleading, e.g., A city called Ventura is linked to new macOS Ventura at $t_2$, introducing noise for reasoning in Japanese.}
    \label{fig:illustration}
    %\vspace{-6mm}
\end{figure}

Inspired by the incompleteness issue facing low-resource languages in constructing TKGs, we introduce a novel task named Cross-Lingual Temporal Knowledge Graph Reasoning (as shown in Figure~\ref{fig:illustration}). This task aims to alleviate the reliance on supervision for TKGs in low-resource languages (referred to as the target language) by transferring temporal knowledge from high-resource languages (referred to as the source language)~\footnote{In this paper, for the sake of brevity, we interchangeably use the terms high-resource/low-resource and source/target.}. In contrast, all the existing efforts are either limited to reasoning in monolingual TKGs (usually high-resource languages, e.g., English)~\cite{TA-DistMult,Know-Evolve,Renet,RE-GCN}, or multilingual static KGs~\cite{KEnS,AlignKGC,SS-AGA}. To the best of our knowledge, cross-lingual TKG reasoning that transfers temporal knowledge between TKGs has not been investigated. 

%Motivated by this, we study a new task named {\em cross-lingual temporal knowledge graph reasoning} as shown in Figure~\ref{fig:illustration}, to alleviate the heavy dependence on supervision for any resource-poor language TKGs by distilling the temporal knowledge from resource-rich ones. Differently, all the existing efforts are either limited to reasoning in monolingual (usually high-resource languages, e.g., English) temporal KGs~\cite{TA-DistMult,Know-Evolve,Renet,RE-GCN}, or multilingual static KG~\cite{KEnS,AlignKGC,SS-AGA}, but neglecting the reasoning in a both temporal and cross-lingual manner that highly requires capturing time-evolving patterns and language discrepancy. To the best of our knowledge, this problem, regarding how to transfer cross-lingual knowledge between TKGs, has still not been formally investigated. 

% Unlike conventional TKG reasoning, 
The fulfillment of this task poses tremendous challenges in two aspects: 1) \textbf{Scarcity of cross-lingual alignment}: as the informative bridge of two separate TKGs, cross-lingual alignment is imperative for cross-lingual knowledge transfer~\cite{AlignKGC,KEnS,SS-AGA}. However, obtaining alignments between languages is a time-consuming and resource-intensive process that heavily relies on human annotations. The transfer of knowledge through a limited number of alignments is often insufficient to fully enhance the TKG in the target language. 2) \textbf{Temporal knowledge discrepancy}: the information associated with two aligned entities is not necessarily identical, especially with regards to temporal patterns. Utilizing a rough approach to equate the aligned entities at all times can result in the transfer of misleading knowledge and negatively impact performance. This becomes more pronounced when the alignments are noisy and unreliable. For example, at the time step $t_2$, a new event about operating system ``{\it Ventura}'' from Apple company occurs in the source English TKG, and meanwhile there is a noisy aligned entity ``{\it Ventura city}'' in the target Japanese TKG. Directly pulling those two entities at this point, can inevitably introduce  noise and fail to predict a set of related events in the target TKG. Therefore, it is crucial to dynamically regulate the alignment strength of each local graph structure over time in order to maximize the effectiveness of cross-lingual knowledge distillation.

% Pulling those entities together cannot augment information in target languages. Small alignment strength is beneficial in the unreliable alignment cases, otherwise the misleading knowledge transferring can even hurt the performance.

% Moreover, in a case that the alignments are not fully reliable, directly pulling the two aligned entities together 


% optimally dynamic alignment strength
% {\em Optimal alignment strength to maximize the benefits of knowledge distillation is difficult to obtain, especially in the temporal manner.} 
% In practical, although the aligned entities can share similar information, they may still differ in other perspectives, including but not limited to frequency, interactions, and temporal patterns. How to adjust the alignment strength (i.e., the distance constrains of the aligned entities in the uni-space) accordingly for different entities at different time is unclear. \zheng{Ruijie TODO: add Ventura case}Moreover, in a case that the alignments are not fully reliable, directly pulling the two aligned entities together can even hurt the performance.



% scarcity of hinders the efficient
% knowledge transfer across languages. 
% {\em Transferring knowledge through a small set of alignments is hard to augment information for all entities.} 

% Aligning the same entities across languages rely heavily on manual labeling or rule-based inference~\cite{EA1,EA2,EA3,selfKG}, which is too time-consuming and impractical to obtain the alignments covering most of the entities in target language. 

% In this paper, we study how to boost the TKG reasoning performance in low-resource languages by explicitly increasing the completeness of those TKGs in history. Instead of improving the underlying information extraction techniques in low-data regime, we propose a new task called {\em Cross-lingual Temporal Knowledge Graph Reasoning}, motivated by the facts that there exists common or complementary knowledge shared by the TKGs in different languages under similar topics. The new task aims to facilitate TKG reasoning in low-resource languages (target languages) by distilling knowledge from a corresponding TKG in high-resource language (source language)  through a small set of entity alignments as bridges~\footnote{In this paper, we interchangeably use the terminology high-resource/low-resource and source/target for briety.}. Figure~\ref{fig:illustration} provides an illustrative example of the proposed task.


% Unfortunately, recent breakthroughs in temporal knowledge graph reasoning model~\cite{TA-DistMult,Know-Evolve,Renet,RE-GCN} highly rely on the completeness of the TKGs, especially for the most recent events. 

% However, the completeness of TKGs varies a lot across different languages, even under similar topics. Conventionally, the TKG construction process relies primarily on information extraction techniques built on the unstructured corpus~\cite{WIKI,YAGO, EventKG}. Therefore, the amount of corpus and human annotations in different languages significantly influence the quality of the corresponding TKGs . 
% Therefore, automatically completing/updating TKGs has been attracting enormous interests in recently years, which aims to predict recent/future events on TKGs based on historical events~\cite{TA-DistMult,Know-Evolve,Renet,RE-GCN}, namely temporal knowledge graph reasoning~\footnote{Broadly speaking, TKG reasoning includes interpolation to predict historical events and extrapolation to predict future events. In this paper, we refer to extrapolation task as TKG reasoning, since it is more vital for time-sensitive downstream tasks.}.


% For languages with large-scale and carefully labeled corpus (we refer to as high-resource languages, e.g., English), the constructed TKGs are more comprehensive than TKGs in other languages that lack the high-quality corpus (we refer to as low-resource languages, e.g., Spanish, Slovene, Danish, etc). Such completeness discrepancy leads to distinctly uneven TKG reasoning performances in different languages, which in turn affects the quality of service of the downstream applications. 


% Compared with the traditional TKG reasoning task, the new task imposes non-trivial challenges. An intuitive solution is to construct a unified graph including two TKGs in both source and target languages, and the knowledge distillation can be fulfilled by pulling the aligned entities from two languages close to each other in the uni-space~\cite{AlignKGC,KEnS}. However, there are still two challenges to be addressed. 

% \zheng{Ruijie TODO, Place this part to related works.}
% Existing works in related areas fail to address the aforementioned challenges. Monolingual reasoning methods on static/temporal knowledge graphs~\cite{TransE,TranR,ComplEX,RotatE,TA-DistMult,Know-Evolve,Renet,RE-GCN} is incapable of the desired knowledge transferring due to the insufficient alignment modeling. Although they can be extended on the cross-lingual scenario by viewing the alignments as a new relation on the merged TKGs, the limited amount of alignments prevent them from augmenting information for most of the entities. Entity alignment methods on KGs~\cite{EA1,EA2,EA3,EA4,EA5,selfKG} can automatically enlarge the alignments by  predicting the correspondence between the two TGs. But most of them, if not all, require the relatively even completeness of two TGs to capture the structural similarities, which can not be satisfied in our case, as target TKGs are far from complete. Some recent works start to study the multilingual TK reasoning on static graphs~\cite{AlignKGC,KEnS,SS-AGA}, which similarly aim to extract knowledge from several source KGs to boost the reasoning performance in the target KG, while they still require a sufficient amount of cross-lingual alignments and totally ignore the temporal perspective in our task.

% to facilitate temporal knowledge graph reasoning in low-resource languages. 
% increase the TKG connection and target TKG capacity
% In light of the mutual benefits, we iteratively generate pseudo alignment pairs and pseudo temporal events to address the co-existing scarcity issue in both cross-lingual alignment and target TKGs. 


In this paper, we propose a novel Mutually-paced Knowledge Distillation (\model) framework, where a teacher network learns more enriched temporal knowledge and reasoning skills from the source TKG to facilitate the learning of a student network in the low-data target one. The knowledge transfer is enabled via an alignment module, which estimates entity correspondence across languages based on temporal patterns. Firstly, to alleviate the limited language alignments (\textbf{Challenge \#1}), such a knowledge distillation process is mutually paced over time. This means, on one hand, we encourage the mutually interactive learning between the teacher and student. Concretely, the alignment module between the teacher and the student learns to generate pseudo alignment between TKGs to maximally expand the upper bound of knowledge transfer. And subsequently, it empowers the student to encode more informative knowledge in target TKG, which can in turn boost the alignment module to explore more reasonable alignments as the bridge across TKGs. One the other hand, inspired by self-paced learning~\cite{spl-1,spl-2}, we make the generations as a progressively easy-to-hard process over time. We start from generating reliable pseudo data with high confidence. As time goes by, we then gradually increase the generation amount by relieving the restriction over time. Secondly, to inhibit the temporal knowledge mismatch (\textbf{Challenge \#2}), the attention module can estimate the graph alignment strength distribution over time. This is achieved by a temporal cross-lingual attention in terms of the local graph structure and temporal-evolving patterns of aligned entities. As such, it can dynamically control the negative effect and suppress noise  propagation from the source TKG. Moreover, we provide a theoretical convergence guarantee for the training objective on both initial ground-truth data and pseudo data. To evaluate \model, we conduct extensive experiments of 12 cross-lingual TKG transfer tasks in multilingual EventKG dataset~\cite{EventKG}. Our empirical results show that the \model method outperforms state-of-the-art baselines in both with and without alignment noise settings, where only $20\%$ of temporal events in the target KG and $10\%$ of cross-lingual alignments are preserved.

% To validate the effectiveness of \model, we conduct extensive experiments of 12 cross-lingual TKG transfer tasks in multilingual EventKG benchmark dataset~\cite{EventKG} . Our experimental results empirically demonstrate the superiority of the \model method over state-of-the-art baselines, ranging from static KG embedding~\cite{TransE,TransR,DistMult,RotatE}, temporal KG reasoning~\cite{TA-DistMult,Renet,RE-GCN} to multilingual KG completion~\cite{KEnS,AlignKGC,SS-AGA}, in both with and without alignment noise settings. We further conduct comprehensive ablation and hyperparameter studies to validate the effectiveness of each design choices. Moreover, we provide theoretical analysis of convergence guarantee for the training objective on both initial groundtruth data and pseudo generative data.



To sum up, our contributions are three-fold:

\begin{itemize}[leftmargin = 15pt]
    \item \textbf{Problem formulation}: We propose the cross-lingual temporal knowledge graph reasoning task, to boost the temporal reasoning performance in target TKG by transferring knowledge from source TKG;
    \item \textbf{Novel framework}: We propose a novel \model framework, which enables the mutually-paced learning between the teacher and student networks, to promote both pseudo alignments and knowledge transfer reliability. Besides, \model involves a dynamic alignment estimation across TKGs that inhibits the influence of temporal knowledge discrepancy.
    \item \textbf{Extensive evaluations}: Empirically, extensive experiments on 12 cross-lingual TKG transfer tasks in multilingual EventKG benchmark dataset demonstrate the effectiveness of \model.
\end{itemize}
% pseudo data generation technique to progressively enhance the training data. The generated pseudo alignments can help the training of the representation modules by the knowledge distillation, and in turn adding pseudo events in the target TKG can improves alignment module by providing high-quality representations. 




% interactively
% TKGs in a source language and a target language are represented by a teacher representation module and a student one into a uni-space, respectively. 
% The knowledge distillation is enabled by a cross-lingual alignment module which pulls the aligned entities close to each other and push other entities far away. 
% To address the challenge caused by the scarcity of cross-lingual alignment, 


\section{Background}
\label{sec:background}

\subsubsection{Plagiarism in Programming Contests}

Source code plagiarism is a well-researched area of studies~\cite{marins2014survey}, however, the developed solutions are usually focused on finding plagiarism in homework assignments~\cite{novak2016academia, plugiarismreview2019},  
there are only a few works devoted to plagiarism in competitive programming~\cite{contest_bangladesh2021, contest_warsaw2019}. Even then, they are mainly focused on integrating a particular plagiarism detection tool into the online judging system, and not on comparing different existing tools. 

Unlike some other cases, source code in programming contests has its own specifics that directly relate to finding plagiarism in it. In addition to the usual plagiarism hiding techniques~\cite{comparison2009}, the solutions will have a lot of similar code that is natural for contests. Firstly, there is \textit{template} code, \textit{i.e.}, some common implementations of popular algorithms that the contestant copies into every solution. Secondly, there is code that will be almost the same in every solution to the given problem but not actually plagiarized, \textit{e.g.}, reading the input or printing the answer.
Currently, active research is underway on how to find similar code and reduce its influence on the similarity~\cite{common2016, common2020}.

The template code is particularly difficult to take into account, because it can be completely different for different contestants in both size and implementation. Also, functions from the template code can be actively used or ignored completely by the contestant, depending on a particular task, making it very difficult to automatically preprocess all submissions by simply removing all template code from them. It is clear that template code can easily become a weak spot for many algorithms, and this must be taken into account when using plagiarism detection tools on the contest code and when building a benchmark for their comparison.

\subsubsection{Source Code Plagiarism Detection Tools}

Many different tools were developed aimed at detecting source code plagiarism~\cite{plugiarismreview2019}. \textit{Text-based} algorithms treat a program as a simple text, without taking into account its programming language. The advantages of such approaches are language-independence and high performance~\cite{comparison2009}, however, this comes at the expense of lower accuracy. A popular text-based tool is Sherlock~\cite{sherlock}. This tool converts the file into a sequence of string tokens, hashes it, and extracts a subsample of hashes. To determine the similarity of two programs, Sherlock calculates the similarity of the sequences of hashes.

\textit{Token-based} algorithms run a specific lexer on the program and compare token streams. Such approaches are still fast but consider a deeper representation of the program. One of the earliest plagiarism detection tools, SIM~\cite{sim1999}, applies an algorithm for finding the maximum sequence alignment to the resulting token sequence of given programs. The similarity of two programs is then defined as their alignment score. Another popular token-based tool, JPlag~\cite{jplag2003}, defines the similarity as the percent of tokens from the first sequence that can be covered by tokens from the second one. A different token-based tool, MOSS~\cite{moss2003}, is based on comparing the fingerprints of programs. A fingerprint is constructed in three steps: (1) all the \textit{n}-grams for the token stream are built, (2) these \textit{n}-grams are hashed, and (3) to avoid comparing big sets of hashes, MOSS uses a \textit{winnowing} algorithm to select a certain subset of hashes for each program. The idea of winnowing has got popular, and several tools appeared that are based on it. One such tool is Dolos~\cite{dolos2022}. Unlike MOSS, Dolos is open-source, supports more programming languages, and provides powerful visualizations of the results.

Finally, \textit{graph-based} algorithms build a graph (usually, a program dependence graph) of the program, which shows the dependencies of the data within the program. To avoid the naive solving of an NP-hard problem, they use certain heuristics. For example, BPlag~\cite{bplag2021} uses the idea of a Greedy-String-Tiling algorithm to find similar parts in the graphs of two programs.

Overall, it can be seen that there exist a lot of approaches for finding plagiarism in the source code, however, given the specifics of programming contests, it is not clear how well they perform in such a setting. To evaluate the existing tools, to help researchers further improve them for competitive programming, as well as to provide a benchmark for future solutions, in this work, we aim to collect the first dedicated dataset of programming contest plagiarism.

\begin{figure*}[htbp]
\centering
    \includegraphics[width=\textwidth]{figures/pipeline.pdf}
    \centering
    \vspace{-0.5cm}
    \caption{The pipeline of the proposed approach for collecting the dataset.}
    \label{fig:pipeline}
    \vspace{-0.5cm}
\end{figure*}
\section{Dataset}
\label{sec:dataset}

When comparing source code plagiarism detection tools, researchers use (a) solutions to student assignments~\cite{jplag2003}, (b) synthetic augmentations of the programs~\cite{bplag2021}, or (c) manually changed code~\cite{sim1999}. 
To compare the tools in programming competitions, we decided to collect a dataset of pairs consisting exclusively of solutions to problems from real contests. 
The overall pipeline for collecting \dataset is presented in Figure~\ref{fig:pipeline}, let us now describe each step in greater detail.

\subsubsection{Gathering Data}

CodeForces~\cite{codeforces} is an online platform that hosts competitive programming contests.
The solution to the problem is a single-file program written in any popular language (C++, Python, Java, etc.). We chose CodeForces because it was used in previous research~\cite{majd2019code4bench, lobanov2023predicting}.

To create our dataset, we selected tasks on the platform based on the following two criteria: (a) the solutions must be large enough so that there are not many false positives when comparing them, and (b) there must be at least a hundred Java solutions for this task, otherwise the probability of plagiarism is very low. We collected a dataset of Java submissions because Java is the most popular language in plagiarism research and is the only language supported by all the tools available to us. 

Based on these criteria, we downloaded all successful Java submissions for 21 different problems and left only those that were successfully parsed by all the six studied tools~\cite{sherlock, sim1999, moss2003, dolos2022, jplag2003, bplag2021}, for a total of 4,695 submissions. Finally, for each problem, we built a set of all possible pairs of its solutions. This resulted in a total of 125,481 pairs.

\subsubsection{Excluding Trivial Non-plagiarized Code}

The constructed set of pairs is not feasible to manually label, and even more importantly, on the vast majority of these pairs, all plagiarism detection tools will return small values of similarity, which is true because the majority of contestants actually do not cheat. 
To overcome these difficulties, we decided to use the existing plagiarism detection tools, not to \textit{label} the data (since the idea is to obtain a manually-curated dataset) but to \textit{filter out} ``trivially'' non-plagiarized pairs. In particular, we used six plagiarism detection tools: Sherlock~\cite{sherlock}, SIM~\cite{sim1999}, MOSS~\cite{moss2003}, Dolos~\cite{dolos2022}, JPlag~\cite{jplag2003}, and BPlag~\cite{bplag2021}. The motivation behind using all the tools at once is to minimize the bias of the resulting dataset towards any one algorithm.

We took a sample of 1,000 random pairs and manually labeled them as either plagiarism or not. The labeling in this pilot work was carried out by the first author, who has 5 years of experience in competitive programming. In the future editions of the dataset, we plan to update it with more examples labeled by multiple experts. When the sample was labeled, we ran all the selected tools on these pairs. This way, for each tool, the \textit{minimum similarity} for the plagiarized pair was found. This threshold does not indicate that all the pairs above it are actually plagiarized, it merely indicates that \textit{all pairs below this threshold are not}. This resulted in 60\% for BPlag, 17\% for Dolos, and 0 for other four, meaning that there exists actual plagiarism, for which these tool show a score of 0. From these results, one can see that only BPlag can be used for effective filtering.

To ensure the reliability of the obtained threshold, we took another 1,000 random pairs, and the first author manually labeled them. Among the selected pairs, only 3 plagiarized pairs were detected, for which BPlag demonstrated less than 60\%, which indicates that this threshold allows us to successfully filter out simplistic cases.
Having obtained and evaluated the threshold, we ran BPlag on all the pairs and filtered out trivially non-plagiarized code. This removed approximately 57\% of the pairs, and left us only with ``interesting'' pairs.

\subsubsection{Building the Dataset of Pairs}

Next, we carried out the final manual labeling of the dataset. We took 1,700 pairs from the remaining set, and they were once again labeled by the first author. Overall, the labeling resulted in 251 cases of plagiarism and 660 cases of non-plagiarism. For the remaining 789 pairs, it was problematic to label them unambiguously, so we excluded such pairs.

\subsubsection{Dealing with Templates}

As discussed in Section~\ref{sec:background}, templates are an important and difficult feature of contest plagiarism. Therefore, for a more informative comparison, we decided to create a separate version of the dataset without the template code. Such a comparison would be further away from the real-world scenario, but would allow us to see which tools suffer the most from the templates.

After manually investigating many submissions, we found that all the template code can be generally classified into two groups: (a) fast input-output (I/O) and (b) the implementation of popular algorithms. We noticed that most of the algorithmic template code in the solutions is not used, so such code can be safely removed. However, the code for fast I/O is actually used by the contestant, so it can not be simply deleted, because some tools require the code to be compilable. In particular, this is important for graph-based tools like BPlag~\cite{bplag2021}, which requires the declaration of all the used functions. To deal with this, we generated a separate JAR-file with all such classes present, and in which BPlag could see the declarations of fast I/O methods. After this, we could safely remove the code responsible for fast I/O from each solution. 
The deletion of template code, both algorithmic and responsible for fast I/O, was done manually for each solution in each pair. 

\subsubsection{Dataset Characteristics}

Finally, we obtained a dataset of source code plagiarism in programming contests called \dataset that has a total of 911 pairs that were all manually labeled: 251 plagiarized pairs and 660 non-plagiarized pairs. Additionally, \dataset is presented in two different versions, meaning that there are two different versions of each solution: (a) \textbf{raw} --- each program in its original form, and (b) \textbf{template-free} --- with most template code removed.
\dataset is available online~\cite{dataset}, and comes with an easy-to-use CLI wrapping and detailed instructions. The evaluation of any tool can be done in two simple steps. Firstly, a special Python class should be created that will run the tool on each pair of the desired version of the dataset. Secondly, after the results are obtained, a separate script will compare them with the labeling and output the report, containing different metrics for the evaluated tools. 
\section{Comparing Different Techniques}
\label{sec:analysis}

The second part of our work consists in taking both versions of \dataset and comparing the existing tools on them. This can show us which tools work the best specifically in the setting of a programming contest, and also point the way to possible further improvements. 

\subsubsection{Metrics}

Firstly, we need to select the metrics for the comparison. As the basics, we can use \textit{precision} and \textit{recall}, two of the standard metrics used in software engineering research. At the same time, we need some other metric that would take into account the balance of precision and recall. One single main metric is necessary not only to make some conclusions about the performance of the tools, but also to be used for training the parameters of the tools.

To find the best trade-off between precision and recall, researchers typically use F1 score, \textit{i.e.}, the harmonic mean between their values. However, in our task, these two metrics are not actually completely equal. In real contests, declaring the solution as being plagiarized is a very serious decision, since this inevitably leads to severe consequences. Therefore, any cases of automatic plagiarism detection will be necessarily checked manually, and so finding all potential cases is more important. Luckily, F-score is actually a more general metric:

$$F_\beta = (1 + \beta^2) \cdot \frac{Precision \cdot Recall}{\beta^2 \cdot Precision + Recall}$$

\noindent where $\beta =$ 1 corresponds to the traditional F1 metric, $\beta <$ 1 favors precision, and $\beta >$ 1 favors recall. Since in our case, the recall is more important, we decided to use $\beta =$ 1.5. Further on, we will denote this metric as $F_{1.5}$.

\subsubsection{Approaches}

In our comparison, we used the same six main tools that were tested for the preliminary filtering of the dataset and that were described in Section~\ref{sec:background}. Since these tools are of different types, this comparison can show us which perform the best for programming contests and how much the template code affects them. 

Each tool under investigation has its own parameters that can be tuned. A lot of research in the adjacent area of clone detection indicates that the default parameters of clone detection tools are not actually the best ones~\cite{wang2013searching, golubev2021multi}. Therefore, to ensure the best performance for each tool, its different configurations were considered. The chosen configurations are similar to those used in the recent comparison~\cite{dolos2022}, there was a total of more than 400 configurations. You can find their full list, as well as the best ones, in the replication package~\cite{dataset}.

\subsubsection{Methodology}

In order to choose the best configuration for each tool, the dataset was divided into train (230 pairs) and test (681 pairs) sets. This division of the dataset is motivated by the fact that with the smaller size of the test dataset, it will be more difficult to find significant differences in the performance of the source code plagiarism detection tools.

Firstly, we ran all the tools on the train set. This means running each tool in each of the configurations with all possible similarity thresholds from the range \texttt{[0, 100]}. The best configuration of each tool was deemed the one that reached the maximum value of $F_{1.5}$. This search was carried out independently for two versions of the dataset, obtaining the best configurations separately for them. 

Then, we compared the best configurations of tools on the test set, also separately for the ``raw'' and the ``template-free'' versions of the dataset, thus allowing us to see the differences between them. We used bootstrapping~\cite{diciccio1996bootstrap} to build confidence intervals for the used metrics. Specifically, for each tool, we built 10,000 sub-samples with the same size as the test set, ran the tool on all of them, and then used the distribution of the metric to build a 95\% confidence interval. 

\subsubsection{Results}

Table~\ref{table:results} shows the results for both versions of \dataset. Firstly, we can see that the best results on both datasets are demonstrated by token-based tools. On the raw version of the dataset, JPlag demonstrated the best results in terms of the $F_{1.5}$ metric (0.77), however, MOSS and SIM are close to it (0.72 and 0.72, respectively). On the template-free dataset, JPlag also won (0.80), and all the four token-based tools demonstrate good results. 
As for the text-based tool Sherlock, in both versions of the dataset, it demonstrated good recall (0.76 and 0.81, respectively), but very low precision (0.34 and 0.39, respectively). While recall is in general more important to us than precision, such an extreme difference ceases to be useful. Interestingly, the graph-based tool BPlag also demonstrated results worse than token-based tools ($F_{1.5}$ of 0.55 on the raw dataset). Our investigation showed that it is too sensitive for programming contests --- a lot of pairs that have large similar chunks of code get counted as plagiarized.

\begin{table}[t]
\centering
\begin{tabular}{c  c  c  c } 
 \toprule
 \textbf{Tool}     & \textbf{Precision}          & \textbf{Recall}            & $\mathbf{F_{1.5}}$      \\ \midrule\midrule
 \multicolumn{4}{c}{\textbf{Raw dataset}} \\
 \midrule
 \textbf{Sherlock}  &  0.34 \stats{($\pm 0.05$)} & 0.76 \stats{($\pm 0.06$)} & 0.55 \stats{($\pm 0.05$)}  \\
 \textbf{SIM}       &  0.69 \stats{($\pm 0.06$)} & 0.74 \stats{($\pm 0.06$)} & 0.72 \stats{($\pm 0.05$)}  \\
 \textbf{MOSS}      &  \textbf{0.77} \stats{($\pm 0.06$)} & 0.71 \stats{($\pm 0.06$)} & 0.72 \stats{($\pm 0.05$)}  \\
 \textbf{Dolos}     &  0.68 \stats{($\pm 0.06$)} & 0.65 \stats{($\pm 0.07$)} & 0.66 \stats{($\pm 0.06$)}  \\
 \textbf{JPlag}     &  0.66 \stats{($\pm 0.06$)} & \textbf{0.83} \stats{($\pm 0.05$)} & \textbf{0.77} \stats{($\pm 0.05$)}  \\
 \textbf{BPlag}     &  0.45 \stats{($\pm 0.06$)} & 0.61 \stats{($\pm 0.07$)} & 0.55 \stats{($\pm 0.06$)}   \\ \midrule
 \multicolumn{4}{c}{\textbf{Template-free dataset}} \\
 \midrule
 \textbf{Sherlock} &   0.39 \stats{($\pm 0.05$)} &  0.81 \stats{($\pm 0.05$)} & 0.60 \stats{($\pm 0.05$)} \\ 
 \textbf{SIM}      &   0.73 \stats{($\pm 0.06$)} &  0.75 \stats{($\pm 0.06$)} & 0.74 \stats{($\pm 0.05$)} \\
 \textbf{MOSS}     &   0.66 \stats{($\pm 0.06$)} &  0.81 \stats{($\pm 0.06$)} & 0.75 \stats{($\pm 0.05$)} \\
 \textbf{Dolos}    &   0.72 \stats{($\pm 0.06$)} &  0.83 \stats{($\pm 0.05$)} & 0.79 \stats{($\pm 0.04$)} \\
 \textbf{JPlag}    &   \textbf{0.75} \stats{($\pm 0.06$)} &  0.83 \stats{($\pm 0.05$)} & \textbf{0.80} \stats{($\pm 0.04$)} \\
 \textbf{BPlag}    &   0.52 \stats{($\pm 0.06$)} &  \textbf{0.87} \stats{($\pm 0.05$)} & 0.72 \stats{($\pm 0.04$)} \\ \bottomrule
\end{tabular}
\vspace{0.2cm}
\caption{Results of comparing the best configurations of tools on the test set of \dataset, in two versions. Numbers in the brackets indicate 95\% confidence intervals.}
\vspace{-0.9cm}
\label{table:results}
\end{table}

Finally, the results show that the removal of the template code increases the performance of the studied tools. The difference is especially noticeable for Dolos (from $F_{1.5}$ of 0.66 to 0.79) and BPlag (from $F_{1.5}$ of 0.55 to 0.72). Without templates, both Dolos and BPlag greatly improve their recall, because they get less triggered by large similar pieces of code. These results indicate that the presence of the template code is indeed a serious challenge in correctly detecting plagiarism in programming contests.
While the results obtained are an encouraging first step towards open-set 3D semantic segmentation there are still many open questions to improve such approaches, some of which we discuss in the following.

Currently, the largest factor limiting segmentation performance is the quality of the vision-language features. While LSeg uses natural language features from CLIP trained on a very large dataset, the visual encoder is trained on the small closed-set ADE20K dataset. If we were able to compute dense pixel-aligned visual-language features from open-set web scraped data without requiring any human annotations, we believe that results could eventually surpass supervised learning methods. \cite{ranasinghe2022perceptual} presented some promising initial results on learning pixel aligned features without using segmentation masks or other expert annotations. 

In real-time experiments, our system relied on poses coming from a SLAM system. If many bad poses are computed by the SLAM system, the 3D representation could become corrupted by bad updates. Possible solutions include treating the sparse SLAM poses as initial guesses and optimizing the poses jointly with scene geometry, as in \cite{sucar2021imap, zhu2022nice}, or bad poses could be filtered out by analyzing the photometric or geometric error across frames. 

In robotics, downstream modules, such as motion planners and high-level planning systems, might benefit from a more explicit and principled representation of geometry than what we presented in this paper. For example, signed distance function based approaches \cite{wang2021neus} might provide better surface and occupancy reconstruction and have other favorable properties, such as the ability to compute the normal of a surface by differentiating through the distance function. For the time being, our method is limited to static scenes. Dealing with moving objects within scenes remains an open problem, but promising recent research \cite{kong2023vmap} suggests that extending neural implicit representations to dynamic scenes might be feasible.

\section{Limitations \& Future Work}
\label{sec:ttv}

\subsubsection{Language} Our dataset is based on the Java language, even though it is not the most prominent one in competitive programming. The choice of the language was dictated by the fact that it is the only one supported by all the prominent tools for detecting source code plagiarism. Future techniques, developed specifically for programming contests, may target more programming languages, such as C++. Our dataset cannot claim generalizability, however, the described approach can be used to collect similar datasets for other languages.

\subsubsection{Labeling} The labeling of the dataset was carried out by the first author of the paper and cannot be perfect. The author has a lot of experience with competitive programming, and we used several rounds of labeling to mitigate possible biases. At the same time, we view \dataset as the first step towards a fully fledged benchmark, and consider broadening it in the future as the main next step. For this reason, our major future plan is to conduct an extensive labeling with multiple experts.
\section{Conclusion}
In this work, we study the DR problem where the participating consumers' baselines have to be estimated online. The online nature of our baseline learning problem results in an exploration-exploitation trade-off between learning the baseline and optimizing the overall operating cost simultaneously, with an added complexity that the consumers can have incentives to inflate the baseline estimate. We propose a novel, online learning DR scheme for this problem and show that our approach achieves a low regret of $\mathcal{O}((\log{T})^2)$ over $T$ days of the DR program with respect to the best DR outcome with full information of the baselines and ensures that participating is individually rational for each consumer. The utility of our approach lies in the fact all prior approaches either require large quantity data or the consumers to report their baselines, both of which could be infeasible. Our contribution is a low regret online learning DR scheme.

\bibliographystyle{IEEEtran}
\balance
\bibliography{IEEEabrv,cite}

\end{document}
