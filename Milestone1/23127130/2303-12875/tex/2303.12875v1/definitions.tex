\usepackage{xparse}
\usepackage{xargs} 

\makeatletter
\def\Hy@raisedlink@left#1{%
    \ifvmode
        #1%
    \else
        \Hy@SaveSpaceFactor
        \llap{\smash{%
        \begingroup
            \let\HyperRaiseLinkLength\@tempdima
            \setlength\HyperRaiseLinkLength\HyperRaiseLinkDefault
            \HyperRaiseLinkHook
        \expandafter\endgroup
        \expandafter\raise\the\HyperRaiseLinkLength\hbox{%
            \Hy@RestoreSpaceFactor
            #1%
            \Hy@SaveSpaceFactor
        }%
        }}%
        \Hy@RestoreSpaceFactor
        \penalty\@M\hskip\z@ 
    \fi
}
\makeatother

\newcommand\newlink[2]{{\protect\hyperlink{#1}{\normalcolor #2}}}
\makeatletter
\newcommand\newtarget[2]{\Hy@raisedlink@left{\hypertarget{#1}{}}#2}
\makeatother

\newcommand\linktoproof[1]{{\normalfont[{\hyperlink{proof:#1}{$\downarrow$}}]}}
\newcommand\linkofproof[1]{\textbf{of \cref{#1}. }\newtarget{proof:#1}}




\newcommand{\defi}{\stackrel{\mathrm{\scriptscriptstyle def}}{=}}
\newcommand{\defiin}{\stackrel{\mathrm{\scriptscriptstyle def}}{\in}}


\newcommand{\bigo}[1]{{\mathcal{O}}\xspace( #1 )}
\newcommand{\bigol}[1]{{\mathcal{O}}\xspace\left( #1 \right)}
\newcommand{\bigotilde}[1]{\newlink{def:big_o_tilde}{\widetilde{{\mathcal{O}}}}\xspace( #1 )}
\newcommand{\bigotildel}[1]{\newlink{def:big_o_tilde}{\widetilde{{\mathcal{O}}}}\left( #1 \right)}
\newcommand{\innp}[1]{\langle #1 \rangle}
\newcommand{\innpl}[1]{\left\langle #1 \right\rangle}
\newcommand{\norm}[1]{\| #1 \|} 
\newcommand{\norml}[1]{\left\| #1 \right\|} 
\newcommand{\abs}[1]{| #1 |} 
\newcommand{\absl}[1]{\left| #1 \right|} 
\newcommand{\card}[1]{| #1 |}
\newcommand{\cardl}[1]{\left| #1 \right|}
\newcommand{\pa}[1]{\left( #1\right)} 
\DeclarePairedDelimiter\cor{\lbrack}{\rbrack}   
\newcommand{\ceil}[1]{\lceil #1\rceil} 
\newcommand{\ceill}[1]{\left\lceil #1\right\rceil} 

\newcommand{\grad}{\nabla} 

\renewcommand{\transp}{\intercal}


\DeclareMathOperator*{\argmax}{arg\,max}                
\DeclareMathOperator*{\argmin}{arg\,min}                

\renewcommand{\R}{\mathbb{R}}
\renewcommand{\D}{\mathcal{D}}
\renewcommand{\P}{\mathcal{P}}
\newcommand{\Rp}{\mathbb{R}_{\geq 0}}

\newcommand\spann[1]{\mathspan (#1)}

\usepackage{tikz}			
\newcommand*\circledaux[1]{\tikz[baseline=(char.base)]{
    \node[shape=circle,draw,inner sep=0.8pt] (char) {#1};}}

\NewDocumentCommand{\circled}{ m o }{%
    \IfNoValueTF{#2}{ \circledaux{#1} }{ \stackrel{\circledaux{#1}}{#2} }%
}


\newcommandx*\x[2][1=t, 2= , usedefault]{\xx^{( #1 )}_{#2}}
\newcommandx*\y[2][1=t, 2= , usedefault]{\yy^{( #1 )}_{#2}}
\newcommandx*\z[2][1=t, 2= , usedefault]{\zz^{( #1 )}_{#2}}





\newcommandx*\dt[2][1=t, 2= , usedefault]{\newlink{def:directions_dt_in_conjugate_directions}{\dd^{( #1 )}_{#2}}}
\newcommandx*\dtbar[2][1=t, 2= , usedefault]{\newlink{def:stored_normalized_directions_dt_in_conjugate_directions}{\bar{\dd}^{( #1 )}_{#2}}}
\newcommandx*\ut[2][1=t, 2= , usedefault]{\newlink{def:initial_basis_ut_in_algorithm}{\uu^{( #1 )}_{#2}}}

\newcommand*\suppast{\newlink{def:support_of_the_solution}{\cS^\ast}}
\newcommand*\sparsity{\card{\suppast}}

\let\oldmu\mu
\renewcommand\mu{\newlink{def:strong_convexity_param}{\oldmu}} 
\let\strikedL\L
\renewcommand\L{\newlink{def:smoothness_constant}{L}} 


\let\oldalpha\alpha
\renewcommand\alpha{\newlink{def:strong_convexity_of_g}{\oldalpha}}

\newcommand\G{\newlink{def:graph_G}{G}}
\renewcommand\V{\newlink{def:vertices_of_graph}{V}}
\newcommand\edges{\newlink{def:edges_of_graph}{E}}
\newcommand\A{\newlink{def:adjacency_matrix}{A}}
\renewcommand\D{\newlink{def:diagonal_degree_matrix}{D}}
\newcommand\lapl{\newlink{def:laplacian_matrix}{\cL}}

\newcommand\I{\newlink{def:identity_matrix}{I}}
\newcommand\f{\newlink{def:function_f_PageRank_objective}{f}}
\newcommand\g{\newlink{def:function_g_constrained_version_of_l1_reg_PageRank}{g}}
\newcommand\gbar{\newlink{def:function_g_restricted_to_subspace}{\bar{g}}}




\let\oldrho\rho
\renewcommand\rho{\newlink{def:weight_in_l1_penalty}{\oldrho}}


\newcommand\simplex[1]{\newlink{def:simplex}{\Delta^{#1}}}


\newcommand\xxast{\xx^\ast} %

\NewDocumentCommand{\xast}{o}{%
    \newlink{def:optimizer}{
        \IfNoValueTF{#1}
            {\xx^{\ast}}
            {x^{\ast}_{#1}}
    }
}


\NewDocumentCommand{\xtbar}{oo}{%
    \newlink{def:iterate_before_pulling_towards_zero}{
        \IfNoValueTF{#1}{
            {\bar{\xx}^{(t)}}
        }
        {
        \IfNoValueTF{#2}
            {\bar{\xx}^{(#1)}}
            {\bar{x}^{(#1)}_{#2}}
        }
    }
}

\NewDocumentCommand{\xtast}{oo}{%
    {
        \IfNoValueTF{#1}
            {\xx^{(\ast, t)}}
        {
        \IfNoValueTF{#2}
            {\xx^{(\ast, #1)}}
            {x^{(\ast, #1)}_{#2}}
        }
    }
}

\NewDocumentCommand{\xt}{oo}{%
    {
        \IfNoValueTF{#1}{
            {\xx^{(t)}}
        }
        {
        \IfNoValueTF{#2}
            {\xx^{(#1)}}
            {x^{(#1)}_{#2}}
        }
    }
}



\let\ss\undefined
\NewDocumentCommand{\ss}{o}{%
    \newlink{def:personalized_distribution}{
        \IfNoValueTF{#1}
            {\mathbf{s}}
            {s_{#1}}
    }
}


\newcommand{\ISTA}{\newlink{def:acronym_ista}{\textnormal{\texttt{ISTA}}}}
\renewcommand{\ista}{\newlink{def:acronym_ista}{\textnormal{\texttt{ISTA}}}}
\renewcommand{\fista}{\newlink{def:acronym_fista}{\textnormal{\texttt{FISTA}}}}
\newcommand{\pgd}{\newlink{def:acronym_projected_gradient_descent}{\textnormal{\texttt{PGD}}}}
\newcommand{\GD}{\newlink{def:acronym_projected_gradient_descent}{\textnormal{\texttt{PGD}}}}
\newcommand{\apgd}{\newlink{def:acronym_accelerated_gradient_descent}{\textnormal{{\texttt{APGD}}}}}
\renewcommand{\appr}{\newlink{def:acronym_approximate_personalized_page_rank}{\textnormal{{\texttt{APPR}}}}}

\newcommand{\aspr}{\newlink{def:acronym_accelerated_sparse_pagerank}{\textnormal{{\texttt{ASPR}}}}}
\renewcommand{\cdappr}{\newlink{def:acronym_conjugate_directions_pagerank}{\textnormal{{\texttt{CDPR}}}}}

\newcommand*\proj[1]{\newlink{def:projection_operator}{\operatorname{Proj}_{#1}}}

\let\oldepsilon\epsilon
\renewcommand{\epsilon}{\newlink{def:accuracy_epsilon}{\oldepsilon}}

\let\olddelta\delta
\renewcommand*\delta{\newlink{def:retraction_parameter_delta_t}{\olddelta}}
\newcommand*\hatepsilon{\newlink{def:accuracy_parameter_of_APGD_subproblem}{\hat{\oldepsilon}}}

\newcommandx*\St[1][1=t, usedefault]{S^{( #1 )}}
\newcommandx*\Ct[1][1=t, usedefault]{C^{( #1 )}}
\newcommand\Sinit{S^{( -1 )}}
\newcommand\Cinit{C^{( -1 )}}


\let\oldvol\vol
\renewcommand{\vol}{\newlink{def:volume}{\oldvol}}
\newcommand\volast{\vol(\suppast)}
\newcommand{\intvol}{\newlink{def:internal_volume}{\widetilde{\oldvol}}}
\newcommand{\n}{\newlink{def:dimension}{n}}


\newcommandx*\canonical[1][1=i, usedefault]{\newlink{def:vector_of_canonical_basis}{\ee_{#1}}}

\newcommand\nnz{\newlink{def:number_of_non_zeros}{\operatorname{nnz}}}

\renewcommand\Q{\newlink{def:symm_pos_def_M_matrix_Q}{Q}}
\newcommand\neigh{\newlink{def:neighbor_in_graph}{\backsim}}

\renewcommand\T{T} 
