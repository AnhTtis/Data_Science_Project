%%%%%%%%%%%%%%%%%%%%%%%%%%%%%%%%%%%%%%%%%%%%%%%%%%%%%%%%%%%%%%%%%%%%%%%%%%%%%%%%%%%

%\usepackage{setspace}
%\doublespacing
%\onehalfspacing
%\singlespacing

\usepackage[vmargin=3cm, hmargin=3cm]{geometry}
\parindent=12pt
\parskip=6pt plus3pt minus3pt


\usepackage[foot]{amsaddr}

%nice Adobe Times fonts
% \usepackage{mathptmx}
% \DeclareSymbolFont{cmlargesymbols}{OMX}{cmex}{m}{n}
% \DeclareMathSymbol{\mycoprod}{\mathop}{cmlargesymbols}{"60}
% \let\coprod\mycoprod

% An alternative nice font collection

\usepackage{mathpazo}

\usepackage{multirow}
\usepackage{longtable}
% Use Euler Caligraphic font

\usepackage[mathcal]{eucal}

\DeclareFontFamily{OT1}{pzc}{}
\DeclareFontShape{OT1}{pzc}{m}{it}{<-> s * [1.10] pzcmi7t}{}
\DeclareMathAlphabet{\mathpzc}{OT1}{pzc}{m}{it}


%\usepackage{pst-all} %pstricks package for use with jpicedt
%\usepackage[protrusion=true,expansion=true]{microtype}

%\usepackage{rotating} %for sideways text: \begin{sideways}blah\end{sideways}

\usepackage{afterpage}
\usepackage{makecell}
\usepackage{changepage}
\usepackage{wrapfig}
\usepackage{enumitem}
\usepackage{hyperref}
\usepackage{mathrsfs}  %some extra fonts
\usepackage{latexsym}
\usepackage{enumitem}
%\usepackage{graphicx}
%\usepackage{blkarray}
\usepackage{bbm} % for lowercase blackboard bold letters
\usepackage{tikz}

\usepackage{yhmath}
\usepackage{fancyhdr}
\usepackage{amsmath, amsfonts, amssymb, amsthm, tensor}
\usepackage[all]{xy}
\usepackage{verbatim}
\usepackage{graphicx}
\usepackage{tikz-cd}
\usetikzlibrary{decorations.pathreplacing,calligraphy}
\usepackage[T1]{fontenc}
\usepackage[utf8]{inputenc}
\usepackage{multicol}
\usepackage{hhline}
\usepackage{color}
\usepackage{adjustbox}
\usepackage{forest}
\usetikzlibrary{decorations.pathreplacing,calc}
\usetikzlibrary{arrows.meta,bending,quotes}
\usetikzlibrary{arrows}

\usepackage{mathtools}
% := set properly
\newcommand{\defeq}{\vcentcolon=}
% f: x -> y, colon spaced properly 
\def\co{\colon\thinspace} 
\def\oc{\thinspace\colon} 


\mathchardef\mhyphen="2D

% PB diagram package for commutative diagrams 
%\usepackage[cmtip,arrow]{xy}
%\usepackage{pb-diagram,pb-xy}
%\dgARROWLENGTH=1.5em %set diagram arrow size to be fairly small


%\usepackage[notcite]{showkeys}
%\usepackage{showlabels}

%\raggedbottom % Makes the bottom margin more flexible (helpful for pictures)

% Define the theorem environments and numbering styles

\numberwithin{equation}{section}

%Alpha theorems
\newtheorem{thmA}{Theorem}
\renewcommand{\thethmA}{\Alph{thmA}}
\newtheorem{corA}[thmA]{Corollary}

%%%%% Number the equations and numerical theorems together
% \newtheorem{theorem}[equation]{Theorem}
% \newtheorem{lemma}[equation]{Lemma} 
% \newtheorem{proposition}[equation]{Proposition}
% \newtheorem{corollary}[equation]{Corollary}
% \newtheorem{definition}[equation]{Definition}
% \newtheorem{conjecture}[equation]{Conjecture}
% \theoremstyle{remark} % styled differently... not italicized
% \newtheorem{remark}[equation]{Remark}
% \newtheorem{example}[equation]{Example}
% \newtheorem{conventions}[equation]{Conventions}

%%%%% Number the equations and numerical theorems serparately 
%\newtheorem{theorem}{Theorem}[section]  %theorems numbered within sections
\newtheorem{teorema}{Theorem}[section]  %theorems numbered within subsections
%\newtheorem{theorem}{Theorem}[subsubsection]  %theorems numbered within subsubsections
\newtheorem{lem}[teorema]{Lemma} 
\newtheorem{prop}[teorema]{Proposition}
\newtheorem{cor}[teorema]{Corollary}
\newtheorem{conj}[teorema]{Conjecture}

\theoremstyle{definition}
\newtheorem{defin}[teorema]{Definition}

\theoremstyle{remark} % styled differently... not italicized
\newtheorem{ex}[teorema]{Example}
\newtheorem{oss}[teorema]{Remark}
\newtheorem{conventions}[teorema]{Conventions}

\newcommand\blankpage{%
	\null
	\thispagestyle{empty}%
	\addtocounter{page}{-1}%
	\newpage}


% Let \sign show roman-style characters in math mode
\newcommand{\id}{\mathrm{id}}
\newcommand{\C}{\mathbb{C}}
\newcommand{\R}{\mathbb{R}}
\newcommand{\N}{\mathbb{N}}
\newcommand{\Z}{\mathbb{Z}}
\newcommand{\Q}{\mathbb{Q}}
\newcommand{\T}{\mathbb{T}}
\newcommand{\Fun}{{\mathbb{F}_1}}
\newcommand{\bend}{\mathpzc{B}}
%\newcommand{\Bend}{\mathpzc{Bend}}
\newcommand{\trop}{\textup{trop}}
\newcommand{\Trop}{\mathpzc{Trop}}
\newcommand{\DTrop}{\mathpzc{Trop}^d}
%\newcommand{\topp}{\mathpzc{Top}}
%\newcommand{\bottom}{\mathpzc{Bot}}
\newcommand{\ke}{\textup{ker}\,}
\newcommand{\B}{\mathbb{B}}
\newcommand{\A}{\mathbb{A}}
%\newcommand{\MMM}{\mathpzc{M}}
%\newcommand{\TTT}{\mathpzc{T}}
%\newcommand{\SSS}{\mathpzc{S}}
\newcommand{\inr}{\mathcal{O}_K}
\newcommand{\maxid}{\mathfrak{m}_K}
\newcommand{\Abs}{\textup{Abs}}
\newcommand{\bigmid}{\, \bigg | \, }


\newcommand{\posR}{(\R_{\ge 0})^m}
\newcommand{\pkcone}[1]{K[\![#1]\!]}
\newcommand{\pk}{K[\![t]\!]}
\newcommand{\pc}{\C[\![t]\!]}
\newcommand{\lc}{\C(\!(t)\!)}
\newcommand{\pp}[1]{\mathbb{Q}_{#1}[\![t]\!]}
\newcommand{\pt}{\T[\![t]\!]}
\newcommand{\pb}{\B[\![t]\!]}
\newcommand{\lb}{\B(\!(t)\!)}
\newcommand{\ptt}[1]{\T[\![t_1, \dots , t_{#1}]\!]}
\newcommand{\pbb}[1]{\B[\![t_1, \dots , t_{#1}]\!]}
\newcommand{\pkk}[1]{K[\![t_1, \dots , t_{#1}]\!]}
\newcommand{\ppp}[2]{\mathbb{Q}_{#1}[\![t_1, \dots , t_{#2}]\!]}

\newcommand{\fatg}{\textbf{g}}
\newcommand{\ii}{\textbf{i}}
\newcommand{\hh}{\textbf{h}}
\newcommand{\jj}{\textbf{j}}
\newcommand{\rr}{\textbf{r}}
\newcommand{\fatw}{\textbf{w}}
\newcommand{\fatm}{\textbf{m}}
\newcommand{\LT}{\mathbb{L} \T}
\newcommand{\F}{\mathbb{F}}
\newcommand{\calA}{\mathcal{A}}
\newcommand{\calX}{\mathcal{X}}
\newcommand{\calB}{\mathcal{B}}
\newcommand{\calT}{\mathcal{T}}
\newcommand{\calL}{\mathcal{L}}
\newcommand{\calM}{\mathcal{M}}
\newcommand{\calF}{\mathcal{F}}
\newcommand{\calN}{\mathcal{N}}
\newcommand{\calC}{\mathcal{C}}
\newcommand{\triv}{\textup{triv}}

\newcommand{\diff}[2]{#1\{x_1, \ldots , x_{#2}\}}
\newcommand{\basic}[2]{#1\{x_1, \ldots , x_{#2}\}_\textit{basic}}
\newcommand{\poly}[2]{#1[x_1, \ldots , x_{#2}]}
\newcommand{\infpoly}[2]{#1[x_1^{(\infty)}, \ldots , x_{#2}^{(\infty)}]}
\newcommand{\hpoly}[2]{#1[x_0, \ldots , x_{#2}]}
\newcommand{\lpoly}[2]{#1[x_1^{\pm 1}, \ldots , x_{#2}^{\pm 1}]}
\newcommand{\Hom}[2]{\textup{Hom}(#1, #2)}
%\newcommand{\ev}{\mathpzc{ev}}
\newcommand{\eva}{\textup{ev}}
\newcommand{\vfi}{\varphi}
\newcommand{\Sol}{\textup{Sol}}
\newcommand{\prt}[1]{\mathcal{P}(#1)}
\newcommand{\gen}{\textup{span}}
\newcommand{\fd}{\textup{d}}
\newcommand{\norm}[1]{|\!|{#1}|\!|}
\newcommand{\ndiv}{\hspace{-4pt}\not|\hspace{2pt}}


\newcommand{\spec}{\mathrm{Spec}\:}
\newcommand{\proj}{\mathrm{Proj}\:}
%\newcommand{\val}{\mathpzc{Val}}
\newcommand{\supp}{\mathrm{supp}}
\newcommand{\Sch}{\mathrm{Sch}}
\newcommand{\Sym}{\mathrm{Sym}}
\newcommand{\an}{\mathrm{an}}
\newcommand{\aff}{\mathrm{aff}}
\newcommand{\op}{\mathrm{op}}
\newcommand{\Frac}{\mathrm{Frac}}
\DeclareMathOperator*{\colim}{colim} %* means limits directly above/below
\newcommand{\sslash}{\mathbin{ \!  /\mkern-6mu/  \!}}

%Categories
\newcommand{\semirings}{\mathbf{\mathrm{Semirings}}}
\newcommand{\Dsemirings}{\mathbf{\mathrm{DiffSemirings}}}
\newcommand{\pairs}{\mathbf{\mathrm{Pairs}}}
\newcommand{\reducedpairs}{\mathbf{\mathrm{Pairs}}_\mathrm{red}}
\newcommand{\fm}{\F_1\textbf{\textup{-Mod}}}
\newcommand{\dfm}{\textbf{\textup{D}}\F_1\textbf{\textup{-Mod}}}
\newcommand{\dalg}[1]{\textbf{\textup{D}}#1\textbf{\textup{-Alg}}}
\newcommand{\dset}[1]{\textbf{\textup{D}}#1\textbf{\textup{-Sets}}}
\newcommand{\set}[1]{#1\textbf{\textup{-Sets}}}
\newcommand{\alg}[1]{#1\textbf{\textup{-Alg}}}
\newcommand{\sch}[1]{\textbf{Sch}_{#1}}



\renewcommand{\labelenumi}{(\arabic{enumi})}


\newcommand{\ALERT}[1]{{\color{red} [#1]}}
\newcommand{\BLUE}[1]{{\color{blue} [#1]}}
% marginal notes e.g. \mnote{blah blah}
\newcommand\mnote[1]{\marginpar{\tiny #1}}
\setlength{\marginparsep}{0.2cm}
\setlength{\marginparwidth}{2.5cm}
\setlength{\marginparpush}{0.5cm}

% color bubble comments
\usepackage{todonotes}
\newcommand{\stecomment}[1]{\todo[inline,size=\tiny,caption={},color=green!40!blue!95!red!30]{#1
\hfill --- SM}}



%%%%%%%%%%%%%%%%%%%%%%% End %%%%%%%%%%%%%%%%%%%%%%%%%%%%%%%%%%