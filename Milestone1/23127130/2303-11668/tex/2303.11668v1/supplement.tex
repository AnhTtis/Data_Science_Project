
\documentclass[10pt,twocolumn,letterpaper]{article}

% \usepackage[review]{cvpr}      % To produce the REVIEW version
%\usepackage{cvpr}              % To produce the CAMERA-READY version
\usepackage[pagenumbers]{cvpr} % To force page numbers, e.g. for an arXiv version

\usepackage{graphicx}
\usepackage{amsmath}
\usepackage{amssymb}
\usepackage{booktabs}
% \usetheme{Singapore}
\usepackage{algorithm}
\usepackage{algorithmic}
\usepackage{amsfonts}

% It is strongly recommended to use hyperref, especially for the review version.
% hyperref with option pagebackref eases the reviewers' job.
% Please disable hyperref *only* if you encounter grave issues, e.g. with the
% file validation for the camera-ready version.
%
% If you comment hyperref and then uncomment it, you should delete
% ReviewTempalte.aux before re-running LaTeX.
% (Or just hit 'q' on the first LaTeX run, let it finish, and you
%  should be clear).
\usepackage[pagebackref,breaklinks,colorlinks]{hyperref}
\usepackage{threeparttable}
% \usepackage{algorithm}

% Support for easy cross-referencing
\usepackage[capitalize]{cleveref}
\crefname{section}{Sec.}{Secs.}
\Crefname{section}{Section}{Sections}
\Crefname{table}{Table}{Tables}
\crefname{table}{Tab.}{Tabs.}

\newcommand{\figref}[1]{Fig.~\ref{#1}}
\newcommand{\tabref}[1]{Tab~\ref{#1}}
\newcommand{\figureref}[1]{Figure~\ref{#1}}
\newcommand{\tableref}[1]{Table~\ref{#1}}
\newcommand{\ms}[1]{\textcolor{red}{#1}}

%%%%%%%%% PAPER ID  - PLEASE UPDATE
\def\cvprPaperID{26} % *** Enter the CVPR Paper ID here
\def\confName{CVPR}
\def\confYear{2023}


\begin{document}

%%%%%%%%% TITLE - PLEASE UPDATE
\title{% of \\
Focus or Not: A Baseline for Anomaly Event Detection\\ On the Open Public Places with Satellite Images\\(Supplementary Material)}

%\author{First Author\\
%Institution1\\
%Institution1 address\\
%{\tt\small firstauthor@i1.org}

%\and
%Second Author\\
%Institution2\\
%First line of institution2 address\\
%{\tt\small secondauthor@i2.org}
%}

\maketitle
\section{Dataset Specification}
The overview of AED-RS dataset in shown in Fig.~\ref{fig:dataset}. 
We select 8 places from the global region and divided into group of places for anomaly event localization task and anomaly event scene classification task.
An interesting point is that, except for the place 5, the other 7 places are located at similar latitude.

\subsection{Anomaly Event Localization}
Anomaly event localization dataset is constructed with 4 square places. 
We established anomaly criteria for each place based on various events of different purposes observed in the data collection process.
Based on the established anomaly event criteria, we first distribute the samples of each places into anomaly scene group and normal scene group. 
After that, pixel-level anomaly event region annotation process is conducted. 
Annotation is conducted using the QGIS program by polygonize the region of anomaly events.
As a result, binary map of anomaly semantics, not just scene-level class information are offered in this dataset and examples of processed-refined images and anomaly region masks are shown Fig.~\ref{fig:sample_label}.

\begin{figure}[h]
    \centering
    \includegraphics[width=0.5\textwidth]{Figures/sample_label.pdf} 
    \caption{Examples of anomaly sample and its binary mask of each place. Place 1(Down-Left), Place 2(Up-Left), Place 3(Up-Right), Place 4(Down-Right)}
    \label{fig:sample_label}
    \vspace{-4mm}
\end{figure}
The statistics of anomaly event localization are shown in Tab.~\ref{tab:stat_ael}.
\begin{table}[h]
\centering
\resizebox{0.5\textwidth}{!}{
    \def\arraystretch{0.9}
        \begin{tabular}{c|cccc}
        \toprule
        $\#$ of samples & Place1 & Place2  & Place3 & Place4\\
        \midrule
        Normal Scene & 28 & 33 & 45 & 12    \\
        Abnormal Scene & 47 & 28 & 39 & 14  \\ \hline \hline
        Total & 49 & 80 & 84 & 26 \\
        \bottomrule
        \end{tabular}
        }
\caption{Number of samples in each place of anomaly event localization task.}
\label{tab:stat_ael}
\vspace{-4mm}
\end{table}
For more specific information of our anomaly event localization dataset, we additionally compute the statistics of anomaly event and shown in Tab.~\ref{tab:stat_ai}.
\begin{table}[h]
\centering
\resizebox{0.5\textwidth}{!}{
    \def\arraystretch{0.9}
        \begin{tabular}{c|cccc}
        \toprule
        $\#$ of samples & Place1 & Place2  & Place3 & Place4\\
        \midrule
        Anomaly Instance & 596 & 472 & 1,488 & 203 \\
        Anomaly Pixels$_{256}$ & 21,247 & 40,717 & 3,390 & 27,902  \\
        Anomaly Pixels$_{512}$ & 100,812 & 189,920 & 24,954 & 122,469 \\
        Anomaly Pixels$_{1024}$ & 448,493 & 867,290 & 137,179 & 499,191  \\
        Anomaly Pixels$_{original}$ & 851,207 & 1,126,281 & 806,372 & 140,616  \\\hline 
        \bottomrule
        \end{tabular}
        }
\caption{Number of instances and pixels of anomaly events of each place in the anomaly event localization task.}
\label{tab:stat_ai}
\end{table}
\subsection{Anomaly Event Scene Classification}
\begin{table}[h]
\centering
\resizebox{0.5\textwidth}{!}{
    \def\arraystretch{0.9}
        \begin{tabular}{c|cccc}
        \toprule
        $\#$ of samples & Place5 & Place6  & Place7 & Place8\\
        \midrule
        Normal Scene & 49 & 66 & 108 & 47    \\
        Abnormal Scene & 11 & 28 & 8 & 31  \\ \hline \hline
        Total & 60 & 94 & 116 & 78 \\
        \bottomrule
        \end{tabular}
        }
\caption{Number of samples in each places of anomaly event scene classification task.}
\label{tab:stat_aes}
\end{table}
Anomaly event scene classification dataset is constructed with 4 places (2 squares, 1 airport, and 1 beach).
In this dataset, we set anomaly event criteria based on the same criteria as anomaly event localization task for place 5 and 8, while place 6 and 7 were based on events that occurred during a specific period. 
The statistics of anomaly event scene classification dataset are in Tab.~\ref{tab:stat_aes}.
\begin{figure*}[t]
    \begin{center}
    \includegraphics[width=500pt]{Figures/Dataset_overview.pdf}     
    \vspace{-30mm}
    \caption{Overview of our AED-RS dataset.}
    \label{fig:dataset}
    \end{center}
\end{figure*}
\begin{figure*}[t]
    \begin{center}
    \includegraphics[width=\linewidth]{Figures/evolve.pdf}     
    \caption{Visualization result of created T-NT region guide maps during training phase. Top left to right(Epoch 0, 30, 60, 90) and bottom left to right(Epoch 120, 150, 180, 210).}
    \label{fig:tntmap_evolution}
    \vspace{-4mm}
    \end{center}
\end{figure*}
\section{T-NT Region Guide Map}
In Fig.~\ref{fig:tntmap_evolution}, we displayed the process of changes in the target and non-target area guide map. 
The visualized results present that our method eliminates the non-target regions(black to white regions) to focus more on the target regions where have small-scale pixel-wise variance. 
However, we have observed that when the training epoch is set to too large value, the t-nt region guide map which set in the later epoch erases a considerable portion of the target regions.
To create the desirable T-NT region guide map, selection of schedule function and its corresponding hyperparameters is really important. 
\begin{figure*}[t]
    \begin{center}
    \includegraphics[width=\linewidth]{Figures/qualitative1.pdf}     
    \caption{Predicted results of sample in place 1 from the models of different epochs. The first row and third row are the visualized anomaly score maps, and the other rows are showing finally predicted anomaly event regions based on its' corresponding epoch's threshold.}
    \label{fig:qualitative_1}
    \vspace{-4mm}
    \end{center}
\end{figure*}
\section{Qualitative Results}
%%%%%%%%% REFERENCES\usepackage{}
% \newpage
% {\small
% \bibliographystyle{ieee_fullname}
% \bibliography{egbib}
% }
% In Fig~\ref{fig:qualitative_1}, the performance of detecting and localizing anomaly events are increasing until epoch 180 an. 
\begin{figure*}[t]
    \begin{center}
    \includegraphics[width=\linewidth]{Figures/Place2_quali.pdf}     
    \caption{Visualization of anomaly event detection results of place1. The center images are the results on the normal samples and the images in the side are the results of anomaly event scene.}
    \label{fig:qualitative_2}
    \vspace{-4mm}
    \end{center}
\end{figure*}
\begin{figure*}[h]
    \begin{center}
    \includegraphics[width=\linewidth]{Figures/Place5_quali.pdf}     
    \caption{Visualization of anomaly event detection results of place5. The center images are the results on the normal samples and the images in the side are the results of anomaly event scene.}
    \label{fig:qualitative_2}
    \vspace{-4mm}
    \end{center}
\end{figure*}
\begin{figure*}[t]
    \begin{center}
    \includegraphics[width=\linewidth]{Figures/Place7_quali.pdf}     
    \caption{Visualization of anomaly event detection results of place7. The center images are the results on the normal samples and the images in the side are the results of anomaly event scene.}
    \label{fig:qualitative_2}
    \vspace{-4mm}
    \end{center}
\end{figure*}
\begin{figure*}[t]
    \begin{center}
    \includegraphics[width=\linewidth]{Figures/Place8_quali.pdf}     
    \caption{Visualization of anomaly event detection results of place8. The center images are the results on the normal samples and the images in the side are the results of anomaly event scene.}
    \label{fig:qualitative_2}
    \vspace{-4mm}
    \end{center}
\end{figure*}
\end{document}
