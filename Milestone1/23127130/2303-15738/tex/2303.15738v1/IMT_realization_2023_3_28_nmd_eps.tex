%2023_03_28
%\documentclass[tikz, sepfignumsbysection,dvipdfmx]{nmd/nmd-article}
\documentclass[tikz, sepfignumsbysection]{nmd/nmd-article}

%\pdfoutput=1 % For the arXiv

%
% To only TeX part of the paper, uncomment the below and edit to suit.
%
%\includeonly{points}

% All custom math definitions are in this file.
\input header

\title[Dehn filling trivializations on a knot group]{Dehn filling trivialization on a knot group: separation and realization}

\author{Tetsuya Ito}
\givenname{Tetsuya}
\surname{Ito}
\address{ Department of  Mathematics \\
          Kyoto University \\
          Kyoto, 606--8502 \\
          Japan
}
\email{tetitoh@math.kyoto-u.ac.jp}
\urladdr{}

\author{Kimihiko Motegi}
\givenname{Kimihiko}
\surname{Motegi}
\address{ Department of Mathematics \\
          Nihon University \\
          3-25-40 Sakurajosui, Setagaya-ku, \\
	Tokyo 156--8550\\
	Japan
 }
\email{motegi.kimihiko@nihon-u.ac.jp}
\urladdr{}

\author{Masakazu Teragaito}
\givenname{Masakazu}
\surname{Teragaito}
\address{International Institute for Sustainability with Knotted Chiral Meta Matter\\
	Hiroshima University \\
	1-3-2 Kagamiyama, Higashi-Hiroshima, \\
	Hiroshima, 739--8511\\
	 Japan
}
\email{teragai@hiroshima-u.ac.jp}

%\arxivreference{}
%\arxivpassword{}

% AMS style garbage
% \subjclass[2020]{57K10} % Knot theory
\subjclass[2020]{Primary: 57M05, Secondary: 57K10; 57K30; 57M07; 20F65}
% \subjclass[2020]{57K31} % Invariants of 3-manifolds (also skein
% modules; character varieties)
% \subjclass[2020]{57K18} % Homology theories in knot theory
% (Khovanov, Heegaard-Floer, etc.)
% \subjclass[2020]{57R58} % Floer homology
% \subjclass[2020]{53A20} % Projective differential geometry
% \subjclass[2020]{20F60} % Ordered groups (group-theoretic aspects) 
%\keywords{}
% GT style garbage
%\subject{primary}{msc2010}{57}
%\subject{secondary}{msc2010}{57}
%\keyword{}

%%%%%
\numberwithin{equation}{section}
\numberwithin{figure}{section}
\numberwithin{table}{section}

\renewcommand{\(}{\textup{(}}
\renewcommand{\)}{\textup{)}}
\usepackage[all]{xy} 

\newtheorem*{thm_persistent_[G,G]_noncyclic}{Theorem~\ref{persistent_[G,G]_noncyclic}}
\newtheorem*{thm_separation}{Theorem~\ref{separation}}
\newtheorem*{thm_S_K_intersection}{Theorem~\ref{S_K_intersection}}
\newtheorem*{thm_realization}{Theorem~\ref{realization}}
\newtheorem*{thm_same_trivialization}{Theorem~\ref{same_trivialization}}
\newtheorem*{thm_non_rigid}{Theorem~\ref{non_rigid}}

\setcounter{tocdepth}{2} 
%%%%%%%%%%%


\begin{document}

\begin{abstract}
Let $K$ be a non-trivial knot in $S^3$ with the exterior $E(K)$. 
For a slope $r \in \mathbb{Q}$, let $K(r)$ be the result of $r$--Dehn filling of $E(K)$. 
To each element $g$ of the knot group $G(K)$ assign $\mathcal{S}_K(g)$ as the set of slopes $r$ such that $g$ becomes the trivial element in $\pi_1(K(r))$.

The purpose of this article is to prove somewhat surprising flexibilities--
a separation property and a realization property--
of the set $\mathcal{S}_K(g)$, 
which are refinements of the Property P in the context of Dehn filling trivialization. 

We construct infinitely many, mutually non-conjugate elements $g$ (in the commutator subgroup) of $G(K)$ such that $\mathcal{S}_K(g)$ is the empty set, 
namely, elements of $G(K)$ that survive all the Dehn fillings of $K$ whenever $K$ has no cyclic surgery. 
Then we prove the Separation Theorem that can be seen as a Dehn filling analogue of various separability properties of 3-manifold groups: 
for every non-torus knot $K$ and any disjoint finite sets $\mathcal{R}$ and $\mathcal{S}$ of slopes, 
there exists an element $g$ of $G(K)$ such that $\mathcal{S}_K(g)$ contains $\mathcal{R}$, 
but does not contain any slopes in $\mathcal{S}$ 
whenever $\mathcal{S}$ contains no Seifert surgery slopes.
We develop this to establish the Realization Theorem asserting that 
for any hyperbolic knot $K$ without torsion surgery slope, 
every finite set of slopes whose complement does not contain Seifert surgery slopes can be realized as the set $\mathcal{S}_K(g)$ for infinitely many, 
mutually non-conjugate elements $g \in G(K)$.
We also provide some examples showing that the Separation Theorem and the Realization Theorem do not hold unconditionally. 

\smallskip
{\small
\noindent
MSC (2020)\ Primary: 57M05, Secondary: 57K10; 57K30; 57M07; 20F65

\noindent
Keywords: Dehn filling, knot group, Property P, peripheral Magnus property, Dehn filling trivialization, slope, stable commutator length, hyperbolic length, residual finiteness
}
\end{abstract}

\maketitle

\newpage

\tableofcontents

\newpage


\section{Introduction}
\label{Introduction}
In his celebrated paper \cite{Dehn} Dehn introduced \textit{Dehn surgery} (named after Dehn) more than $110$ years ago.  
Using Dehn surgery he constructed infinitely many homology $3$--spheres; 
it should be noted that before Dehn's work, only Poincar\'e's example (Poincar\'e homology $3$--sphere) was known. 
Since then, 
Dehn surgery is one of the main subject in low-dimensional topology. 
On the other hand, 
in \cite{Dehn}  Dehn made important connections between group theory and topology. 
Actually he uses fundamental group which had been previously introduced by Poincar\'e to study $3$--manifolds, 
and made foundational works of combinatorial and geometric group theory.   
It is worth mentioning that the Property P conjecture due to Bing and Martin \cite{BingMartin}, 
which had been one of the leading problems in $3$--manifold topology before it was settled by Kronheimer and Mrowka \cite{KM}. 
 
The motivation of this article and its sequel is to present a new perspective on a study of Dehn surgery (filling) from group theoretic viewpoint,  
which takes a root in the Property P conjecture and relies on recent  progress in geometric group theory. 

\medskip

\subsection{Property P and peripheral Magnus property}
\label{peripheral Magnus property} 


Let $K$ be a non-trivial knot in $S^3$ with the exterior $E(K)$. 
Dehn filling on $E(K)$ is a geometric procedure to produce closed $3$--manifolds 
$K(r)$ 
by attaching (filling) a solid torus to $E(K)$. 
There are infinitely many ways to attach the solid torus to $E(K)$. 
To make precise we prepare some terminologies. 
 
The loop theorem says that the inclusion map $i\colon \partial E(K) \to E(K)$ induces a monomorphism 
$i_* \colon \pi_1(\partial E(K)) \to G(K)$, 
thence we have a subgroup $P(K) = i_*(\pi_1(\partial E(K))) \subset G(K)$. 
A \textit{slope element} in $G(K)$ is a primitive element $\gamma$ in $P(K) \cong \mathbb{Z} \oplus \mathbb{Z}$, 
which is represented by an oriented simple closed curve in $\partial E(K)$. 
Denote by $\langle\!\langle \gamma \rangle\!\rangle$ the normal closure of $\gamma$ in $G(K)$.  
Using the standard meridian-longitude pair $(\mu, \lambda)$ of $K$,  
each slope element $\gamma$ is expressed as $\mu^p \lambda^q$ for some relatively prime integers $p, q$. 
As usual we use the term \textit{slope} to mean the isotopy class of an unoriented simple closed curve in $\partial E(K)$. 
Representatives of two slope elements $\gamma$ and its inverse $\gamma^{-1}$ represent the same slope (by forgetting their orientations), 
which is identified with $p/q \in \mathbb{Q} \cup \{\infty\, (=1/0)\}$. 
Since $\langle\!\langle \gamma \rangle\!\rangle = \langle\!\langle \gamma^{-1} \rangle\!\rangle$, 
it is convenient to denote them by $\langle\!\langle p/q \rangle\!\rangle$. 
Thus each slope defines the normal subgroup $\langle\!\langle p/q \rangle\!\rangle \subset G(K)$, 
which will be referred to as the \textit{normal closure of the slope $p/q$} for simplicity. 

The normal closure $\langle\!\langle r \rangle\!\rangle$ of a slope $r$ of $K$ naturally arises via Dehn surgery on $K$, 
or Dehn filling of $E(K)$. 
Attach a solid torus $S^1 \times D^2$ to $E(K)$ along their boundaries so that 
$\{ * \} \times \partial D^2$ represents a slope $r$ to obtain a closed $3$--manifold $K(r)$. 
We call $K(r)$ the 3-manifold obtained by \textit{$r$--Dehn filling} on $E(K)$. 
($K(r)$ is also called  the 3-manifold obtained by \textit{$r$--Dehn surgery} on $K$.) 

Assume that $G(K)$ has a presentation
\[
\langle g_1, \ldots, g_n, \mu, \lambda \mid w_1, \ldots, w_m \rangle.
\]
Then van Kampen's theorem tells us that $\pi_1(K(p/q))$ has a presentation
\[
\langle g_1, \ldots, g_n, \mu, \lambda \mid w_1, \ldots,\ w_m, \mu^p \lambda^q  \rangle.
\]
Thus $\pi_1(K(p/q)) = G(K)/ \langle\!\langle \mu^p \lambda^q \rangle\!\rangle =  G(K)/\langle\!\langle p/q \rangle\!\rangle = G(K)/ \langle\!\langle r \rangle\!\rangle$, 
and we obtain a natural epomorphism
\[
p_r \colon G(K) \to G(K)/ \langle\!\langle r \rangle\!\rangle = \pi_1(K(r)), 
\]
which we call \textit{Dehn filling epimorphism}, or more specifically \textit{$r$--Dehn filling epimorphism}.  
The Dehn filling epimorphism gives us the following short exact sequence which relates $G(K),\ \langle\!\langle r \rangle\!\rangle$ and 
$\pi_1(K(r))$;
\[ 1 \rightarrow \langle\!\langle r \rangle\!\rangle \rightarrow G(K) \xrightarrow{p_r} 
G(K) / \langle\!\langle r \rangle\!\rangle = \pi_1(K(r)) \rightarrow 1.\]

Hence, each Dehn filling induces a trivialization of element in $G(K)$ which we call \textit{Dehn filling trivialization} on $G(K)$. 
Precisely $r$--Dehn filling trivializes elements 
\[
\langle\!\langle r \rangle\!\rangle
= \{ g \in G(K) \mid p_r(g) = 1 \in \pi_1(K(r)) \}.
\]
Since $K(\infty) = S^3$, 
$\langle\!\langle \infty \rangle\!\rangle = G(K)$. 

Originally, 
the Property P conjecture \cite{BingMartin} asserts that 
$\pi_1(K(r)) \ne 1$ for any non-trivial knot $K$ and $r \in \mathbb{Q}$, 
and it is settled by Kronheimer and Mrowka \cite{KM} using gauge theory and symplectic topology. 

In the language of knot group the Property P conjecture can be rephrased as: 

\begin{theorem}[(Property P \cite{KM})]
\label{propertyP}
Let $K$ be a non-trivial knot and $r$ a slope in $\mathbb{Q} \cup \{ \infty \}$. 
\[
\langle\!\langle r \rangle\!\rangle = \langle\!\langle \infty \rangle\!\rangle\ (= G(K)) \ \textrm{if and only if}\  
r = \infty.
\] 
\end{theorem}

Hence, $\langle\!\langle r \rangle\!\rangle$ is a proper subgroup of $G(K)$ for all $r \in \mathbb{Q}$. 


Then it is natural to extend Theorem~\ref{propertyP} to all slopes $r$ in $\mathbb{Q} \cup \{ \infty \}$ 
in order to establish the one to one correspondence between 
slopes and their normal closures. 
This is achieved in the previous paper \cite{IMT_Magnus} based upon Property P (Theorem~\ref{propertyP}) and topological and group theoretical arguments. 

\begin{theorem}[(Peripheral Magnus property \cite{IMT_Magnus})]
\label{peripheral_Magnus}
Let $K$ be a non-trivial knot and $r, r'$ slopes in $\mathbb{Q} \cup \{ \infty \}$. 
Then 
\[
\langle\!\langle r \rangle\!\rangle = \langle\!\langle r' \rangle\!\rangle\  
\textrm{if and only if}\  r = r'.
\]
\end{theorem}


\medskip


\subsection{Dehn filling trivializations for an element of a knot group}
\label{function}
For each $r \in \mathbb{Q}$, 
elements of $G(K)$ which are trivialized by $r$--Dehn filling come from the normal closure $\langle\!\langle r \rangle\!\rangle$ of the slope $r$. 

In general, a problem of deciding whether a given element (written by fixed generators as a word) represents the trivial element was formulated by Dehn \cite{Dehn,Dehn1911} as an algorithmic problem called the \textit{word problem}. 
In this article we do not consider such an algorithm, 
but for a non-trivial element $g \in G(K)$, 
we consider whether $p_r(g)$ is trivial or non-trivial in $\pi_1(K(r))$ over all $r \in \mathbb{Q}$. 


To refine the Property P (Theorem~\ref{propertyP})  in the context of Dehn filling trivialization, 
we introduce the following set valued function on a knot group. 

\begin{definition}
\label{D}
Let $K$ be a non-trivial knot in $S^3$. 
Define a set valued function 
\[
\mathcal{S}_K \colon G(K) \to 2^{\mathbb{Q}}
\]
from the knot group $G(K)$ to the power set of $\mathbb{Q}$ as 
\[
\mathcal{S}_K(g) 
=  \{ r \in \mathbb{Q} \mid p_r(g) = 1 \in \pi_1(K(r)) \}
=  \{ r \in \mathbb{Q} \mid g \in \langle\!\langle r \rangle\!\rangle \} \subset \mathbb{Q}.
\]
\end{definition}

By definition, for the trivial element $1$, 
$\mathcal{S}_K(1) = \mathbb{Q}$. 


As usual, 
let us denote by $\overline{\mathcal{S}_K(g)}$ the complement of $\mathcal{S}_K(g)$, i.e. 
\[
\overline{\mathcal{S}_K(g)} = \mathbb{Q} - \mathcal{S}_K(g) 
=  \{ r \in \mathbb{Q} \mid p_r(g) \ne 1 \in \pi_1(K(r)) \}.
\]


Then the Property P (Theorem~\ref{propertyP}) is equivalent to the following. 

\bigskip

\noindent
$\bullet$\ 
{\textit{For any $r \in \mathbb{Q}$, 
there exists a non-trivial element $g \in G(K)$ such that $r \not\in \mathcal{S}_K(g)$, i.e. 
$r \in \overline{\mathcal{S}_K(g)}$.}

\bigskip

If $\mu \in  \langle\!\langle r \rangle\!\rangle$ for some $r \in \mathbb{Q}$, 
then $ \langle\!\langle r \rangle\!\rangle =  \langle\!\langle \infty \rangle\!\rangle = G(K)$ and 
$r \in \mathcal{S}_K(g)$ for every $g \in G(K)$, a contradiction. 
As an implication of this we have $\mathcal{S}_K(\mu) = \emptyset$. 

\medskip

An investigation of the function $\mathcal{S}_K$ is essentially the same as
an investigation of a relationship among normal closures $\langle\!\langle r \rangle\!\rangle$.  
Using the function $\mathcal{S}_K \colon G(K) \to 2^{\mathbb{Q}}$, 
the assertion of \cite[Theorem 1.1]{IchiMT} can be rephrased as 

\medskip

\begin{theorem}[(Finiteness theorem \cite{IchiMT})]
\label{S_K_hyperbolic}
For any hyperbolic knot $K$,  
$\mathcal{S}_K(g)$ is always a finite subset of $\mathbb{Q}$ for any non-trivial element $g \in G(K)$. 
\end{theorem}

\medskip

For torus knots Theorem~\ref{S_K_hyperbolic} does not hold. 
Actually, if $K$ is a torus knot $T_{p, q}$,  
then it has infinitely many slopes $r_1, r_2, \ldots$ such that $K(r_i)$ has the cyclic fundamental group. 
Let $g$ be an element in $[G(K), G(K)]$. 
Then $p_{r_i(g)} = 1$ for $r_i$ ($i = 1, 2, \ldots$). 
Hence $\mathcal{S}_K(g) \supset \{ r_1, r_2, \ldots \}$. 

For satellite knots, we expect that $\mathcal{S}_K(g)$ is finite, but it is still open. 


Although $\mathcal{S}_K(g)$ is always finite for any hyperbolic knot $K$, 
it may be arbitrarily large. 

\medskip

\begin{theorem}[(Unboundedness theorem \cite{IMT_Magnus})]
\label{unbound}
Let $K$ be a non-trivial knot, 
and let $\{ r_1, \ldots, r_n \}$ be any finite subset of $\mathbb{Q}$. 
If $K$ is a torus knot $T_{p, q}$, we assume $pq \not\in \{ r_1, \ldots, r_n \}$. 
Then there exists a non-trivial element $g \in G(K)$ such that $\mathcal{S}_K(g) \supset \{ r_1, \ldots, r_n \}$. 
\end{theorem}

\medskip

Theorem~\ref{S_K_hyperbolic} shows that for any infinite family of slopes $\{ r_1, r_2, \dots \}$,  
the intersection $\bigcap_{i = 1}^{\infty} \langle\!\langle r_i \rangle\!\rangle$ is trivial for hyperbolic knots. 
On the other hand,  
Theorem \ref{unbound} claims that 
 for any finite family $\{ r_{i_1}, r_{i_2}, \dots. r_{i_k} \}$, 
the intersection $\bigcap_{j= 1}^{k} \langle\!\langle r_{i_j} \rangle\!\rangle$ is non-trivial. 

\medskip

It should be noted that Theorem~\ref{unbound} does not say anything about slopes $r$ for which 
$p_r(g) \ne 1$. 


As a refinement of Property P in the context of Dehn filling trivialization, we propose: 

\bigskip


\noindent
\textbf{Separation Problem.}
\begin{enumerate}
\renewcommand{\labelenumi}{(\roman{enumi})}
\item
{\it For given two slopes $r$ and $s$ in $\mathbb{Q}$, 
does there exist an element $g \in G(K)$ such that 
$r \in \mathcal{S}_K(g)$ and $s \in \overline{\mathcal{S}_K(g)}$?}
\item 
{\it For any disjoint subsets $\mathcal{R}$ and $\mathcal{S}$ of $\mathbb{Q}$, 
does there exist an element $g \in G(K)$ such that 
$\mathcal{R} \subset \mathcal{S}_K(g)$ and $\mathcal{S} \subset \overline{\mathcal{S}_K(g)}$?}
\end{enumerate}

\medskip


As we observe below, in general, 
the answer to Separation Problem is negative. 
To explain this we prepare some terminologies, which will be used throughout this paper. 

The Geometrization theorem \cite{Pe1,Pe2,Pe3} enables us to classify slopes into 
hyperbolic surgery slope, Seifert surgery slope, toroidal surgery slope and reducing surgery slope according to the classification of $3$--manifolds. 
We say that a slope $r$ is a \textit{hyperbolic surgery slope} if $K(r)$ is a hyperbolic $3$--manifold, meaning that it admits a complete hyperbolic metric of finite volume.  
Similarly, we say that a slope $r$ is a \textit{Seifert surgery slope}, \textit{toroidal surgery slope}, or \textit{reducing surgery slope} if 
$K(r)$ is a Seifert fiber space, toroidal $3$--manifold (meaning that it contains an incompressible torus), or a reducible $3$--manifold (meaning that it 
contains a $2$--sphere not bounding a $3$--ball), respectively. 
Slopes $r \in \mathbb{Q}$ for which $K(r)$ is non-hyperbolic are called \textit{exceptional} surgery slopes. 
Thurston's hyperbolic Dehn surgery theorem \cite{T1,T2} asserts every hyperbolic knot has only finitely many exceptional surgery slopes.  
More precisely, Lackenby and Meyerhoff \cite{LM} prove that there are at most ten exceptional surgery slopes on a hyperbolic knot, answering Gordon's conjecture \cite{Gordon}. 

The Geometrization theorem implies that a $3$--manifold with finite fundamental group is a Seifert fiber space. 
So it may be convenient to classify Seifert surgeries depending upon fundamental groups. 
If $\pi_1(K(r))$ is finite (resp. cyclic), then $r$ is called a \textit{finite surgery slope} (resp. \textit{cyclic surgery slope}). 


\begin{example}
\label{torus_knot_pq}
For a torus knot $T_{p, q}$ the slope $pq$ represented by a regular fiber has a special property. 
\[
\langle \! \langle pq \rangle\!\rangle \cap \langle \! \langle r \rangle\!\rangle \cong \begin{cases} \mathbb{Z} & \mbox{if}\ r \mbox{ is a finite surgery slope,} \\ \{1\} & \mbox{if $r\ne pq$ and $r$ is not a finite surgery slope.}\end{cases}
\] 
See \cite[Proposition 5.4]{IMT_Magnus}. 
Thus if $pq \in \mathcal{S}_{T_{p,q}}(g)$, 
then $r \not \in \mathcal{S}_{T_{p,q}}(g)$ for all non-finite surgery slopes $r \ne pq$. 
So for any non-finite surgery slope $r$,  there is no non-trivial element $g \in G(K)$ such that 
$\mathcal{R} = \{ pq, r \} \subset \mathcal{S}_K(g)$. 
In particular, for any $\mathcal{S} \subset \mathbb{Q}$ with $\mathcal{R} \cap \mathcal{S} = \emptyset$, 
there is no element $g \in G(K)$ such that $\mathcal{R} \subset \mathcal{S}_K(g)$ and $\mathcal{S} \subset \overline{\mathcal{S}(g)}$. 
\end{example}



In general,  
an inclusion $\langle \! \langle r \rangle\!\rangle \subset \langle \! \langle s \rangle\!\rangle$ forces us the restriction that 
 $r \in \mathcal{S}_K(g)$ implies $s \in \mathcal{S}_K(g)$.  
This gives the following examples. 

\begin{example}
\label{torus_knot_cyclic}
Let $K$ be a torus knot $T_{p, q}$ $(p > q \ge 2)$. 
Then for each finite surgery slope $r_0 \in \mathbb{Q}$, 
\cite[Theorem~6.4]{IMT_Magnus} shows that there is an infinite descending chain 
\[
\langle \! \langle r_0 \rangle\!\rangle \supset \langle \! \langle r_1 \rangle\!\rangle \supset \langle \! \langle r_2 \rangle\!\rangle \supset \cdots.
\]
Hence 
if $g \in \langle \! \langle r_n \rangle\!\rangle$, then $g \in \langle \! \langle r_m \rangle\!\rangle$ for any pair $m, n$ with $m < n$. 
This means that there is no element $g \in G(K)$ such that 
$\{ r_n \} \subset \mathcal{S}_K(g)$ and $\{ r_m \} \subset \overline{\mathcal{S}_K(g)}$. 
\end{example}

Even for hyperbolic knots, we have: 

\begin{example}
\label{pretzel_non_separation}
Let $K$ be the $(-2, 3, 7)$--pretzel knot. 
Choose two slopes $\frac{18}{5}$ and $18$. 
Then as shown in \cite[Example 6.2]{IMT_Magnus}, 
$\langle \! \langle \frac{18}{5} \rangle\!\rangle \subset \langle \! \langle 18 \rangle\!\rangle$. 
Thus there is no element $g$ such that 
$\{ \frac{18}{5} \} \subset \mathcal{S}_K(g)$ and 
$\{ 18 \} \subset \overline{\mathcal{S}_K(g)}$. 
\end{example}


Therefore the set $\mathcal{S}_K(g)$ is not arbitrary. 


\medskip


The above phenomenon arises from the fact that the slopes appeared in Example~\ref{torus_knot_cyclic} 
and the slope $18$ in Example~\ref{pretzel_non_separation} are finite surgery (Seifert surgery) slopes. 
Indeed, we have seen that inclusions of normal closures $\langle \! \langle r \rangle \! \rangle \supset \langle \! \langle r' \rangle \! \rangle$ happens only if $r$ is a finite surgery slope \cite[Proposition 2.3]{IMT_Magnus}. 

\medskip

The peripheral Magnus property (Theorem~\ref{peripheral_Magnus}), together with \cite[Proposition~2.3]{IMT_Magnus}, 
asserts the following separation property for two slopes, 
which gives an answer to Separation Problem (i). 

\medskip

\noindent
$\bullet$\ 
{\it For given two distinct slopes $r$ and $s$ in $\mathbb{Q}$, 
if $s$ is not a finite surgery slope, 
then there exists an element $g \in G(K)$ such that 
$r \in \mathcal{S}_K(g)$ and $s \in \overline{\mathcal{S}_K(g)}$.}



\medskip


Our first main result below shows that 
generically we are able to find an element $g$ so that $\mathcal{S}_K(g)$ contains any finite set of slopes but excludes another finite set of slopes. 

\bigskip
\begin{theorem}[(Separation theorem)]
\label{separation}
Let $K$ be a non-trivial knot which is not a torus knot.  
Let $\mathcal{R} = \{ r_1, \dots, r_n \}$ and $\mathcal{S} = \{ s_1, \dots, s_m \}$ be any finite subsets of $\mathbb{Q}$ such that 
$\mathcal{R} \cap \mathcal{S} = \emptyset$. 
Assume that $\mathcal{S}$ does not contain a Seifert surgery slope. 
Then there exists an element $g \in [G(K), G(K) ]\subset G(K)$ such that $\mathcal{R} \subset \mathcal{S}_K(g)$ and  $\mathcal{S} \subset \overline{\mathcal{S}_K(g)}$ . 
\end{theorem}


In Separation Problem (ii), 
if we choose $\mathcal{S} = \mathbb{Q} - \mathcal{R}$, 
then $\mathcal{S} \subset \overline{\mathcal{S}_K(g)}$ means 
$\mathcal{R} \supset \mathcal{S}_K(g)$. 
Since $\mathcal{R} \subset \mathcal{S}_K(g)$, this implies $\mathcal{S}_K(g) = \mathcal{R}$. 
This leads us the following realization problem. 
Taking Examples~\ref{torus_knot_pq}, \ref{torus_knot_cyclic}, \ref{pretzel_non_separation} into account  we formulate the realization problem as in the following. 
Recall also that $\mathcal{S}_K(g)$ is always a finite set for any hyperbolic knot $K$ and for all non-trivial element $g \in G(K)$. 

\bigskip

\noindent
\textbf{Realization Problem.}\quad 
{\it Let $K$ be a non-trivial knot without finite surgery. 
For a given finite subset $\mathcal{R} = \{ r_1, \ldots, r_n \} \subset \mathbb{Q}$, 
does there exist a non-trivial element $g \in G(K)$ such that $\mathcal{S}_K(g) =\mathcal{R}$? 
In other words, 
can we find a non-trivial element $g \in G(K)$ so that it becomes trivial after prescribed $r_i$--Dehn fillings, 
while never trivial after all the remaining non-trivial Dehn fillings?}

\bigskip

For convenience, we call a slope $r$ a \textit{torsion surgery slope} if $\pi_1(K(r))$ has a non-trivial torsion, which is equivalent to say that $r$ is either a finite surgery slope or a reducing surgery slope; see Lemma~\ref{torsion_slope}. 

The next theorem establishes the realization theorem for most hyperbolic knots, 
which might be somewhat surprising. 

\begin{theorem}[(Realization theorem)]
\label{realization}
Let $K$ be a hyperbolic knot without torsion surgery. 
Let $\mathcal{R} = \{ r_1, \ldots, r_n\}$ be any finite \(possibly empty\) subset of $\mathbb{Q}$ whose complement does not contain a Seifert surgery. 
Then there exists an element $g \in [G(K), G(K)] \subset G(K)$ such that 
$\mathcal{S}_K(g) = \mathcal{R}$.
\end{theorem}

\begin{example}
\label{exceptional}
Let $K$ be the figure-eight knot. 
Choose the set of exceptional surgery slopes $\{ 0, \pm 1, \pm 2, \pm 3, \pm 4 \}$. 
Then there exists an element $g \in G(K)$ such that 
$\mathcal{S}_K(g) =\{ 0, \pm 1, \pm 2, \pm 3, \pm 4 \}$. 
\end{example}

Hyperbolic knots with no exceptional surgery are common. 
In fact, every knot can be converted into such a hyperbolic knot by a single crossing change and there are infinitely many such crossing changes for the given knot \cite{MM_crossing}. 
For hyperbolic knots without exceptional surgeries we have the following simple form. 

\begin{corollary}
\label{no_exceptional}
Let $K$ be a hyperbolic knot with no exceptional surgery. 
Then any finite subset $\mathcal{R} \subset \mathbb{Q}$ can be realized as 
$\mathcal{S}_K(g) = \mathcal{R}$ for some element $g \in G(K)$.
\end{corollary}


\bigskip
Now let us turn to elements with the same Dehn filling trivializations. 
We address: 

\begin{question}
\label{identical}
For a given element $g \in G(K)$, 
how many elements $h$ can satisfy $\mathcal{S}_K(h) = \mathcal{S}_K(g)$ up to conjugation?
\end{question}

Corollary~\ref{persistent_[G,G]} gives an answer to Question~\ref{identical} for knots without cyclic surgeries and 
$\mathcal{S}_K(g) = \emptyset$. 

More generally, using the residual finiteness of $3$--manifold groups, 
we will prove the following. 
For $a, b \in G(K)$, 
$a^b$ denotes $b^{-1} a b \in G(K)$. 

\begin{theorem}
\label{same_trivialization}
Let $K$ be a non-trivial knot. 
For any non-trivial element $g \in G(K)$, we have 
\[
\mathcal{S}_K(g) = \mathcal{S}_K(g^{g^{\alpha}} g^{-2})\ \textrm{for any}\ \alpha \in G(K).
\]
\end{theorem}

This result is quite general, 
but in general it is not so convenient to see that varying elements $\alpha$ we may actually obtain 
non-conjugate elements $g^{g^{\alpha}} g^{-2}$. 
See Proposition~\ref{conjugacy} and Example~\ref{figure-eight_holonomy}. 
So we will demonstrate another way to find elements $h$ with 
$\mathcal{S}_K(h) = \mathcal{S}_K(g)$, and prove Theorem~\ref{non_rigid} below. 


We say that a non-trivial element $g$ is \textit{peripheral} if it is conjugate into $P(K) = i_*(\pi_1(\partial E(K))) \subset G(K)$. 

\begin{theorem}
\label{non_rigid}
Let $K$ be a hyperbolic knot without torsion surgery slope. 
For any non-peripheral element $g \in [G(K), G(K)]$, 
there are infinitely many, mutually non-conjugate elements 
$\alpha_m \in G(K)$ such that $\mathcal{S}_K(\alpha_m) = \mathcal{S}_K(g)$ and it is not conjugate to any power of $g$. 
\end{theorem}

As we will see in Proposition~\ref{S_K(g)_S_K(g^n)}, 
when $g$ has no torsion surgeries, 
$\mathcal{S}_K(g)=\mathcal{S}_K(g^{n})$
so infiniteness of elements having the same Dehn filling trivialization is rather obvious.
What Theorem~\ref{non_rigid} says is that besides these obvious examples, the powers and conjugates, 
there are many other elements having the same Dehn filling trivialization.  
In particular, Theorems~\ref{same_trivialization} and \ref{non_rigid} suggest that 
for every finite subset $A \subset \mathbb{Q}$ ($A \in 2^{\mathbb{Q}}$), 
the preimage
$\mathcal{S}_K^{-1}(A) \subset G(K)$ consists of variety of elements. 

\bigskip

\subsection{Organization of the paper}
\label{organization}

In Section~\ref{function_S_K} we will collect elementary properties of the function $\mathcal{S}_K \colon G(K) \to 2^{\mathbb{Q}}$, 
and then in Section~\ref{normal_closure} we will prepare a useful result about the normal closure of slope elements (Proposition~\ref{slope}). 

In the proof of Theorem~\ref{separation},  
as the first step, 
for any finite subset $\mathcal{S} \subset \mathbb{Q}$, 
we need to prove that there exists a non-trivial element $g$ such that $\mathcal{S} \subset \overline{\mathcal{S}_K(g)}$. 
This is the opposite situation of Theorem~\ref{unbound}, 
and leads us to the extreme case where $\mathcal{R} = \emptyset$ and $\mathcal{S} = \mathbb{Q}$ in the Separation Problem (ii) (the Realization Problem). 

If $K$ has a cyclic surgery slope $r$, 
then for any element $g$ in $[G(K),G(K)]$, 
we have $p_r(g) = 1$. 
Theorem~\ref{persistent_[G,G]_noncyclic} below asserts that having cyclic surgery is the only obstruction for $G(K)$ to have a persistent element that belongs to $[G(K),G(K)]$. 
Following the cyclic surgery theorem \cite{CGLS}, every non-trivial knot which is not a torus knot admits at most two non-trivial cyclic surgery slopes. 

\begin{theorem}
\label{persistent_[G,G]_noncyclic}
Let $K$ be a non-trivial knot in $S^{3}$.  
Then there exist infinitely many, mutually non-conjugate elements $g \in [G(K),G(K)]$ 
such that $p_s(g) \neq 1$ in $\pi_1(K(s))$ for all non-cyclic surgery slopes $s \in \mathbb{Q}$. 
\end{theorem}

Theorem~\ref{persistent_[G,G]_noncyclic} will be used in the proof of Theorem~\ref{separation}. 
Section~\ref{homologically0} is devoted to a proof of Theorem~\ref{persistent_[G,G]_noncyclic}. 
The proof requires elementary hyperbolic geometry and stable commutator length, 
so we will briefly recall some definitions and prepare useful results in Section~\ref{scl}. 

After establishing Theorem~\ref{persistent_[G,G]_noncyclic}, 
we will prove Theorem~\ref{separation} in Section~\ref{finite_separation}. 

To bridge the gap between Theorem~\ref{separation} and Theorem~\ref{realization} for hyperbolic knots without torsion surgery, 
we need to prove the following result which is regarded as a refinement of Proposition~\ref{cup_cap}(i). 


\begin{theorem}[(Product theorem)]
\label{S_K_intersection}
Let $K$ be a hyperbolic knot which has no torsion surgery. 
Let $g$ be a non-peripheral element and $h$ a non-trivial element in $[G(K),G(K)]$. 
Then for a given non-zero integer $n$, 
there exists a constant $N > 0$ such that 
\[
\mathcal{S}_K(g) \cap \mathcal{S}_K(h) = \mathcal{S}_K(g^mh^n) 
\] 
for $m \ge N$.  
\end{theorem}

Theorem~\ref{S_K_intersection} will be proved in Section~\ref{powered_product}.

We will prove Theorem~\ref{realization} using Theorem~\ref{separation} and Theorem~\ref{S_K_intersection} in Section~\ref{proof}. 

In Section~\ref{same trivialization} we will provide two ways to find elements with the same Dehn filling trivializations 
by proving Theorems~\ref{same_trivialization} and \ref{non_rigid}. 
The former heavily depend on the residual finiteness of $3$--manifold groups and the latter depends on Theorem~\ref{S_K_intersection} above. 

\medskip




\section{Elementary properties of the function $\mathcal{S}_K$}
\label{function_S_K}
Let us collect some elementary properties of the function $\mathcal{S}_K \colon G(K) \to 2^{\mathbb{Q}}$. 

Since $p_r(a^{-1}ga) = 1$ if and only if $p_r(g) = 1$, and 
$p_r(g^{-1}) = 1$ if and only if $p_r(g)=1$, 
we have 

\begin{proposition}
\label{S_K_class_function}
$\mathcal{S}_K \colon G(K) \to 2^{\mathbb{Q}}$ is a class function, 
namely $\mathcal{S}_K(a^{-1}ga) = \mathcal{S}_K(g)$ for all $g$ and $a$ in $G(K)$. 
Furthermore, 
$\mathcal{S}_K(g^{-1}) = \mathcal{S}_K(g)$. 
\end{proposition}

Since $hg = h(gh)h^{-1}$, 
Proposition~\ref{S_K_class_function} immediately implies: 

\begin{proposition}
\label{S_K_commute}
$\mathcal{S}_K(gh) = \mathcal{S}_K(hg)$. 
\end{proposition}

When $K$ is a prime knot,  any automorphism $\phi\colon G(K) \rightarrow G(K)$ is induced by a homeomorphism $f$ of $E(K)$ \cite{Tsau2}, 
so either $\phi(\langle\! \langle r \rangle\! \rangle) = \langle\! \langle r \rangle\! \rangle$ or $\phi(\langle\! \langle r \rangle\! \rangle) = \langle\! \langle -r \rangle\! \rangle$. The latter case happens only if $K$ is amphicheiral and $f$ reverses the orientation of $E(K)$. Since $g \in \langle\! \langle r \rangle\! \rangle$ if and only if $\phi(g) \in \phi(\langle\! \langle r \rangle\! \rangle)$, the function $\mathcal{S}_K$ is almost $\mathrm{Aut}(G(K))$-invariant in the following sense.


\begin{proposition}
If $K$ is prime, for an automorphism $\phi\colon G(K) \rightarrow G(K)$, 
$\mathcal{S}_K(\phi(g)) = \pm \mathcal{S}_K(g)$. Furthermore, if $K$ is not amphicheiral then $\mathcal{S}_K(\phi(g)) = \mathcal{S}_K(g)$. 
Here for $\mathcal{S} =\{s_1,s_2,\ldots \} \subset \mathbb{Q}$, we put $\pm \mathcal{S}=\{\pm s_1, \pm s_2,\ldots,\}$. 
\end{proposition}


By definition, 
$\mathcal{S}_K$ also satisfies the following. 

\begin{proposition}\
\label{cup_cap}
\begin{enumerate}
\renewcommand{\labelenumi}{(\arabic{enumi})}
\item
$\mathcal{S}_K(g) \cap \mathcal{S}_K(h) \subset \mathcal{S}_K(gh)$. 

\item
$\mathcal{S}_K(g) \cup \mathcal{S}_K(h) \subset \mathcal{S}_K([g, h])$.
\end{enumerate}
\end{proposition}

\begin{proof}
(1) Let us take $r \in \mathcal{S}_K(g) \cap \mathcal{S}_K(h)$. 
Then $p_r(g) = p_r(h) = 1$, and thus $p_r(gh) = p_r(g)p_r(h) = 1$, 
which shows $r \in \mathcal{S}_K(gh)$. 

(2) If $r \in \mathcal{S}_K(g) \cup \mathcal{S}_K(h)$, then $p_r(g) = 1$ or $p_r(h) = 1$. 
This implies $p_r([g, h]) = p_r(g)p_r(h)p_r(g)^{-1}p_r(h)^{-1} = 1$, 
hence $r \in \mathcal{S}_K([g, h])$. 
\end{proof}

Now we describe a relation between $\mathcal{S}_K(g)$ and $\mathcal{S}_K(g^n)$. 

\begin{proposition}\
\label{S_K(g)_S_K(g^n)}
\[
\mathcal{S}_K(g) \subset \mathcal{S}_K(g^n) \subset \mathcal{S}_K(g) \cup \{ \text{torsion surgery slopes} \}
\]
for any integer $n \neq 0$. 
\end{proposition}

\begin{proof}
Note that $G(K)$ has no torsion, $g^n \ne 1 \in G(K)$. 
If $p_r(g) =1$, then obviously $p_r(g^n) =1$, which shows $\mathcal{S}_K(g) \subset \mathcal{S}_K(g^n)$. 
Assume that $p_r(g^n) = p_r(g)^n =1$. 
If $p_r(g) = 1$, then $r \in \mathcal{S}_K(g)$, otherwise $p_r(g)$ is a torsion element.
Thus $r$ is a torsion surgery slope. 
\end{proof}

Note that torsion surgeries are classified as follows. 

\begin{lemma}
\label{torsion_slope}
Let $K$ be a non-trivial knot. 
A slope $r$ is a torsion surgery slope if and only if it is either a finite surgery slope of a reducing surgery slope.
\end{lemma}

\begin{proof}
Assume that a $3$--manifold $M$ has the fundamental group with non-trivial torsion. 
Then $M$ is spherical (i.e. $\pi_1(M)$ is finite) or $M$ is non-prime (\cite[Lemma~9.4]{Hem}), 
and thus $r$ is either a finite surgery slope or a reducing surgery slope. 
Conversely, if $\pi_1(M)$ is finite, then every non-trivial element is a torsion element, 
and if $\pi_1(M)$ is reducible,
since $K$ is non-trivial, it is non-prime \cite{GabaiII}. 
Furthermore, following \cite{GLu2} $M$ has a lens space summand. 
Hence $\pi_1(M)$ has a torsion element. 
\end{proof}


The cabling conjecture \cite{GS}  of Gonz\'{a}lez-Acu\~na and Short asserts that a reducible $3$--manifold is obtained by Dehn surgery on $K$ only when 
$K$ is a cable knot and the surgery slope is the cabling slope. 
In particular, a hyperbolic knot 
is expected to have no reducing surgery.
So for hyperbolic knots, a torsion surgery may coincide with a finite surgery. 



\begin{proposition}
\label{<<g>><<h>>}
For $g, h \in G(K)$, 
if their normal closures satisfy $\langle\!\langle g \rangle\!\rangle \subset \langle\!\langle h \rangle\!\rangle$, 
then $\mathcal{S}_K(h) \subset \mathcal{S}_K(g)$.
\end{proposition}

\begin{proof}
Take $r \in \mathcal{S}_K(h)$, i.e. $p_r(h) = 1$. 
Then since $g \in \langle\!\langle g \rangle\!\rangle \subset \langle\!\langle h \rangle\!\rangle$, 
$p_r(g) = 1$. 
This means that $r \in \mathcal{S}_K(g)$. 
\end{proof}




\medskip


\section{Normal closure of a finite surgery slope}
\label{normal_closure}
In this section we prepare a useful fact.

Let $K$ ba a knot in $S^3$ and $\gamma$ a slope element in $G(K)$. 
Let $s$ be a slope in $\mathbb{Q}$.
In \cite[Proposition~2.3]{IMT_Magnus} we have shown that 
if $\gamma \in \langle\!\langle s \rangle\!\rangle$, 
then the slope $s$ is represented by $\gamma$ or $s$ is a finite surgery slope. 
For later convenience we generalize this as follows. 

\begin{proposition}
\label{slope}
Suppose that $\gamma^n \in \langle\!\langle s \rangle\!\rangle$ for some integer $n \ne 0$.  
Then $s$ is represented by $\gamma$ or $s$ is a finite surgery slope. 
\end{proposition} 

\begin{proof}
Without loss of generality we may assume $n > 0$. 
Write $\gamma = \mu^a  \lambda^b$ for coprime integers $a, b$ and $s = p/q$. 
So $s$ is represented by a slope element $\mu^p  \lambda^q$. 
Then 
\[
\mu^{(aq-bp)n} 
= ((\mu^a  \lambda^b)^{n})^q (\mu^p  \lambda^q)^{-bn} 
\in  \langle\!\langle s \rangle\!\rangle. 
\]
If $aq-bp = 0$, then $s$ is represented by $\gamma$. 
So we assume that $aq-bp \ne 0$; 
without loss of generality we may further assume $aq-bp > 0$.  

Suppose for a contradiction that $(aq-bp)n = 1$. 
Then $\mu \in \langle\!\langle s \rangle\!\rangle$ and 
$\pi_1(K(s)) = G(K) / \langle\!\langle s \rangle\!\rangle =\{ 1 \}$, 
and thus $s = \infty$ (Theorem~\ref{propertyP}), contradicting the assumption. 

So in the following we assume $m = (aq-bp)n \ge 2$. 
Since $\mu^m \in \langle\!\langle s \rangle\!\rangle$, 
$\langle\!\langle \mu^m \rangle\!\rangle \subset \langle\!\langle s \rangle\!\rangle$  
and we have the canonical epimorphism 
$\varphi\colon G(K) / \langle\!\langle \mu^m \rangle\!\rangle \to G(K) / \langle\!\langle s \rangle\!\rangle$. 
Note that $(\varphi(\mu))^m = \varphi(\mu^m) = 1$ in $G(K) / \langle\!\langle s \rangle\!\rangle$. 
If $\varphi(\mu) = 1 \in G(K) / \langle\!\langle s \rangle\!\rangle$,  
then $\varphi(\mu) = \mu \in  \langle\!\langle s \rangle\!\rangle$ and as above $s = \infty$. 
Thus we may assume $\varphi(\mu) \ne 1$, 
i.e. it is a non-trivial torsion element in $G(K)\slash \langle\!\langle s \rangle\!\rangle$.  
Recall that an irreducible $3$--manifold $M$ with infinite fundamental group is aspherical \cite[p.48 (C.1)]{AFW}
and hence $\pi_1(M)$ has no torsion element \cite{Hem}. 
Hence $\pi_1(K(s))$ is finite or $K(s)$ must be a reducible manifold. 
Accordingly $s$ is a finite surgery slope or a reducing surgery slope. 

Now let us eliminate the second possibility. 
Assume to the contrary that $s$ is a reducing surgery slope. 
As in the above $G(K) / \langle\!\langle s \rangle\!\rangle$ has a non-trivial torsion element, 
hence $K(s) \ne S^2 \times S^1$ and thus $K(s)$ is a connected sum of two closed $3$--manifolds other than $S^3$. 
(In general, 
the result of a surgery on a non-trivial knot is not $S^2 \times S^1$ \cite{GabaiIII}.)
By the Poincar\'e conjecture, they have non-trivial fundamental groups, 
and $G(K)\slash \langle\!\langle s \rangle\!\rangle = A \ast B$ for some non-trivial groups $A$ and $B$. 

As we have seen in the above, 
$\varphi(\mu)$, 
the image of $\mu$ under the canonical epimorphism 
$\varphi : G(K) / \langle\!\langle \mu^m \rangle\!\rangle \to G(K) \slash \langle\!\langle s \rangle\!\rangle$ 
is a non-trivial torsion element in $A \ast B$. 
By \cite[Corollary~4.1.4]{MKS}, 
a non-trivial torsion element in a free product $A \ast B$ is conjugate to a torsion element of $A$ or $B$. 
Thus we may assume that there exists $g \in A\ast B$ such that $g\varphi(\mu)g^{-1} \in A$. 

On the other hand, 
since $G(K)$ is normally generated by $\mu$, 
$A \ast B$ is normally generated by $\varphi(\mu)$. 
This implies that $A \ast B$ is normally generated by an element $g\varphi(\mu)g^{-1} \in A$. 
In particular, 
the normal closure $\langle\!\langle A \rangle\!\rangle$ of $A$ in $A \ast B$ is equal to $A \ast B$, 
and $(A \ast B) \slash \langle\!\langle A \rangle\!\rangle = \{ 1 \}$. 
However, $(A \ast B) \slash \langle\!\langle A \rangle\!\rangle = B \neq \{1\}$ (\cite[p.194]{MKS}). 
This is a contradiction. 
\end{proof}

\medskip

\begin{remark}(\cite{IMT_Magnus})
\label{finitely_generated}
Let $K$ be a non-trivial knot and $r$ a slope in $\mathbb{Q}$. 
Then $\langle\!\langle r \rangle\!\rangle$ is finitely generated if and only if $r$ is a finite surgery slope 
\(i.e. $\pi_1(K(r))$ is finite\) or 
$K$ is a torus knot $T_{p, q}$  and $r = pq$. 
\end{remark}



\section{Stable commutator length and Dehn fillings}
\label{scl}
Let $K$ be a hyperbolic knot in $S^3$. 
To find persistent elements in the commutator subgroup $[G(K), G(K)]$, 
the stable commutator length plays a key role. 
In this section we prepare some useful results.  

\subsection{Hyperbolic length and Dehn fillings}

Let $M$ be a closed hyperbolic $3$--manifold. 
Then a representative loop of any non-trivial element $g$ of $\pi_1(M)$ is freely homotopic to a unique closed geodesic $c_g$. 
We may define the length of $g \in \pi_1(M)$, 
denoted by $\ell_M(g)$, 
to be the hyperbolic length of the corresponding closed geodesic $c_g$. 

Let $K$ be a hyperbolic knot and 
$g$ a non-trivial element in $G(K)$. 
Suppose that $s \in \mathbb{Q}$ is a hyperbolic surgery slope for $K$. 
If $p_s(g)$ is non-trivial in $\pi_1(K(s))$, 
then we can define
the length $\ell_{K(s)}(p_s(g))$. 
Let us define
\[ L(g) = \inf_{s \in \mathbb{Q}}\{ \ell_{K(s)}(p_s(g)) \: | \: 
K(s) \textrm{ is hyperbolic and $p_s(g) \ne 1$}\} \ge 0.\]

In this subsection we give a simple characterization of elements $g \in G(K)$ with $L(g) > 0$. 

For convenience of readers, 
we briefly recall Thurston's hyperbolic Dehn surgery theory \cite{T1,T2}. 

By the assumption we have a holonomy (faithful and discrete) representation
\[
\rho \colon G(K) \cong \pi_1(S^3 - K) \to \mathrm{Isom}^+(\mathbb{H}^3) = \mathrm{PSL}(2, \mathbb{C}).
\]

We regard $\rho = \rho_{\infty}$ and denote its image by $\Gamma_{\infty} \cong \pi_1(S^3 - K)$. 
Recall that $\rho(\mu)$ and $\rho(\lambda)$ are parabolic elements.  


By small deformation of the holonomy representation $\rho$ up to conjugation, 
we obtain representations 
$\rho_{x, y} \colon \pi_1(S^3 - K) \to \mathrm{Isom}^+(\mathbb{H}^3) = \mathrm{PSL}(2, \mathbb{C})$ whose image is 
$\Gamma_{x, y}$. 
(The meaning of $x, y$ will be clarified below.)
Deforming $\rho$ continuously, 
$(x, y)$ varies over an open set $U_{\infty}$ in $S^2 = \mathbb{R}^2 \cup \{ \infty \}$; see the proof of \cite[Theorem~5.8.2]{T1}.
By Mostow's rigidity theorem \cite{Mostow,Prasad} $\rho_{x, y}(\mu)$ and $\rho_{x, y}(\lambda)$ are not parabolic when $(x, y) \ne \infty$. 

We may take a conjugation in $\mathrm{Isom}^+(\mathbb{H}^3) = \mathrm{PSL}(2, \mathbb{C})$ 
so that 
\[
\rho_{x, y}(\mu) \colon (z, t) \mapsto (\alpha_{x, y, \mu}z, ||\alpha_{x, y, \mu}||\,t)
\]
and 
\[
\rho_{x, y}(\lambda) \colon (z, t) \mapsto (\alpha_{x, y, \lambda}z, ||\alpha_{x, y, \lambda}||\,t)
\] 
for some complex numbers $\alpha_{x, y, \mu}$ and $\alpha_{x, y, \lambda}$. 


Since both $\rho_{x, y}(\mu)$ and $\rho_{x, y}(\lambda)$ are very close to 
$\rho(\mu)$ and $\rho(\lambda)$, respectively, 
both $\alpha_{x, y, \mu}$ and $\alpha_{x, y, \lambda}$ are nearly equal to $1$. 

Then for $\rho_{x, y}$, 
we have
\[
x \log \alpha_{x, y, \mu} + y \log \alpha_{x, y, \lambda} = 2\pi i.
\]

The complex translation length $\log \alpha_{x, y, \mu}$ (resp. $\log \alpha_{x, y, \lambda}$) 
of $\rho_{x, y}(\mu)$ (resp. $\rho_{x, y}(\lambda)$) is well-defined up to the sign. 
Note that since  $\rho_{x, y}(\mu)$ and  $\rho_{x, y}(\lambda)$ commute, 
the pair $(\log \alpha_{x, y, \mu},\  \log \alpha_{x, y, \lambda})$ is well-defined up to sign. 
If we write 
\[
\alpha_{x, y, \mu} = r_{x, y, \mu} e^{i \theta_{x, y, \mu}},\quad 
\alpha_{x, y,\lambda} = r_{x, y, \lambda} e^{i \theta_{x, y, \lambda}},
\] 
the above equation implies
\[
\begin{cases} 
x \log r_{x, y, \mu} + y \log r_{x, y, \lambda} = 0\\ 
x \theta_{x, y, \mu} + y \theta_{x, y, \lambda} = 2 \pi
\end{cases}
\]

As we mentioned above, 
a small deformation $\rho_{x, y}$ of $\rho$ corresponds to 
$(x, y)$ in an open neighborhood $U_{\infty} \subset \mathbb{R}^2 \cup \{ \infty \} = S^2$. 
Thurston's hyperbolic Dehn surgery theorem says that there is an integer $N > 0$ 
such that if relatively prime integers $p$ and $q$ satisfies $|p| + |q| > N$, 
then $\rho_{p, q}$ satisfies 
\[
p \log \alpha_{p, q, \mu} + q \log \alpha_{p, q, \lambda} = 2\pi i, 
\]
and gives an incomplete hyperbolic metric of $S^3 - K$ so that 
its completion is $K(p/q)$ in which $K_{p/q}^*$, the image of the filled solid torus, 
is the shortest closed geodesic in the hyperbolic $3$--manifold $K(p/q)$. 

Although $\rho$ is a faithful representation,  
$\rho_{p, q}$ is a representation of $\pi_1(S^3 - K)$ which is not faithful and 
its image $\Gamma_{p, q}$ is regarded as $\pi_1(K(p/q))$. 
\[
\xymatrix{
\pi_1(E(K))\ar@{->>}[drr]_{\rho_{p,q}}  \ar[rr]^{\rho}_{\cong} &  & \hspace{0.2cm} \Gamma\subset \mathrm{Isom}^+(\mathbf{H}^3)\ar@{->>}[d] & \\
              &  &  \hspace{1.0cm}  \Gamma_{p,q}\subset \mathrm{Isom}^+(\mathbf{H}^3)
}
\]


\begin{proposition}
\label{L>0}
Let $g$ be a non-trivial element of $G(K)$. 
Then 
$L(g) > 0$ if and only if $g$ is a non-peripheral element, 
i.e. it is not conjugate into $P(K) = i_*(\pi_1(\partial E(K))) \subset G(K)$. 
\end{proposition}

\begin{proof}
Assume first  that $g$ is not peripheral. 
Then $\rho(g)$ is loxodromic, 
and up to conjugation we may assume $\rho(g) \colon z \mapsto \alpha_g z$ on $\partial \mathbb{H}^3$, 
where $\alpha_g = r_g e^{i \theta_g}$ and $r_g > 1$. 
Thus the translation length $\left| \log |\alpha_g| \right| = | \log r_g | > 0$. 
Under small deformation of $\rho$, 
the translation length $| \log r_{x, y, g} |$ of $\rho_{x, y}(g)$ deforms continuously according to $(x, y) \in U_{\infty}$. 
Thus $g$ is freely homotopic to a unique closed geodesic $c_g \subset S^3 - K$, the image of the axis of 
$\rho(g) \subset \mathbb{H}^3$, 
of length $\ell_g =  | \log r_g | > 0$. 
Hence, we can take constants $\varepsilon_0 > 0$ and $N_0 > 0$ so that 
if $|p| + |q| > N_0$, 
then the length of $c_g$ in $K(p/q)$, 
i.e. $\ell_{K(p/q)}(g)$,  
is greater than $\varepsilon_0$, 
while the length of $K^*_{p/q} < \varepsilon_0$. 
Since there are only finitely many slopes $p_1/q_1, \dots, p_k/q_k$ with $|p_i| + |q_i| \le N_0$, 
let us put $\varepsilon_1 = \mathrm{min}\{ \ell_{K(p_1/q_1)}(p_{p_1/q_1}(g)), \dots, \ell_{K(p_k/q_k)}(p_{p_k/q_k}(g)) \}$, 
where $K(p_i/q_i)$ is hyperbolic and $p_{p_i/q_i}(g) \ne 1$ in $\pi_1(K(p_i/q_i))$ ($i = 1, \dots, k$). 
Finally put $\varepsilon = \mathrm{min}\{ \varepsilon_0, \varepsilon_1 \}$. 
Then the length of $c_g$ in $K(p/q)$, i.e. $\ell_{K(p/q)}(g)$ is greater than or equal to $\varepsilon > 0$ for all 
hyperbolic surgery slopes $p/q \in \mathbb{Q}$ with $p_{p/q}(g) \ne 1$.  
This shows that $L(g) \ge \varepsilon > 0$. 


Suppose that $g$ is peripheral. 
Then it is conjugate to $(\mu^a \lambda^b)^m \in P(K)$ for some integers $a, b$ and $m$, 
where $a$ and $b$ are relatively prime. 
Thus $p_{a/b}(g) = 1$; 
otherwise $p_{p/q}(g)$ ($p/q \ne a/b$) is freely homotopic to $n_{p/q}$--th power of the dual knot $K^*_{p/q}$, 
the core of the filling solid torus in $K(p/q)$. 
It should be noted that $n_{p/q}$ varies according as $p/q$. 
So to controll $n_{p/q}$, 
we take an infinite sequence of surgery slopes 
$p_i/q_i$ so that the distance $\Delta(a/b,\ p_i/q_i) = |aq_i - bp_i| = 1$ with $|p_i| + |q_i| \to \infty$ $(i \to \infty)$. 
Then a representative of the slope element $\mu^a \lambda^b$ wraps once in the $p_i/q_i$--filling solid torus, so it is freely homotopic to $K^*_{p_i/q_i}$, i.e. $n_{p_i/q_i} = 1$.  
This shows that $\ell_{K(p_i/q_i)}(p_{p_i/q_i}(g)) = m \ell(K^*_{p_i/q_i})$. 

Thurston's hyperbolic Dehn surgery theorem \cite{T1,T2} says that 
when $|p_i| + |q_i|$ goes to $\infty$, 
$K(p_i/q_i)$ converges to $S^3 - K$ and $K^*_{p_i/q_i}$ is a closed geodesic whose length tends to $0$. 
Hence, when $|p_i| + |q_i|$ goes to $\infty$, 
$\ell_{K(p_i/q_i)}(p_{p_i/q_i}(g)) = m \ell(K^*_{p_i/q_i})$ also tends to $0$, and so $L(g) = 0$. 
\end{proof}

\bigskip

\subsection{Stable commutator length and hyperbolic length}
In this subsection we will recall 
the definition of the stable commutator length, and then collect some useful results. 

For $g \in [G,G]$ the \emph{commutator length\/} $\mathrm{cl}_G(g)$ is the smallest number of commutators in $G$ whose product is equal to $g$. 
The \emph{stable commutator length\/} $\mathrm{scl}_{G}(g)$ of $g \in [G,G]$ is defined to be the limit
\begin{equation}
\label{def1}
\mathrm{scl}_{G}(g) = \lim_{n \to \infty} \frac{\mathrm{cl}(g^n)}{n}.
\end{equation}


Since $\mathrm{cl}_G(g^n)$ is non-negative and subadditive, Fekete's subadditivity lemma shows that 
this limit exists.  

We will extend (\ref{def1}) to the stable commutator length $\mathrm{scl}(g)$ for an element $g$ which is not necessarily in $[G, G]$ as 
\begin{equation}
\label{def2}
\mathrm{scl}_G(g) = \begin{cases}
\frac{\textrm{scl}_G(g^{k})}{k} & \mbox{ if } g^{k} \in [G,G] \mbox{ for some } k > 0,\\
\infty & \mbox{otherwise}.
\end{cases}
\end{equation}

By definition (\ref{def1}), 
it is easy to observe that if $g^k,\, g^{\ell} \in [G, G]$, 
then $\displaystyle \frac{\mathrm{scl}(g^k)}{k} = \frac{\mathrm{scl}_G(g^{\ell})}{\ell}$ for any $k,\, \ell > 0$. 
So in (\ref{def2}) $\mathrm{scl}_G(g)$ is independent of the choice of $k > 0$ such that $g^k \in [G, G]$. 
In particular, we have the following property; see \cite[Lemma~2.1]{IMT_decomposition} for its proof. 

\begin{lemma}
\label{scl_g^k}
For any $g \in G$ and $k > 0$,  
we have $\mathrm{scl}_G(g^{k})=k\,\mathrm{scl}_G(g)$. 
\end{lemma}

Since a homomorphism sends a commutator to a commutator, 
we observe the following monotonicity property of scl \cite[Lemma~2.4]{Cal_MSJ}. 

\begin{lemma}
\label{monotonicity}
Let $\varphi\colon G \to H$ be a homomorphism. 
Then \[
\mathrm{scl}_H(\varphi(g)) \leq \mathrm{scl}_G(g)\  \textrm{for all}\ g \in G.
\] 
\end{lemma}

Recall that a map $\phi\colon G \rightarrow \mathbb{R}$ is \emph{homogeneous quasimorphism} of defect $D(\phi)$ if
\[ D(\phi)=\sup_{g,h \in G}|\phi(gh)-\phi(g)-\phi(h)| < \infty, \quad \phi(g^{k})=k\phi(g) \ (\forall g\in G, k \in \mathbb{Z}).\]

Note that homogeneous quasimorphism is constant on conjugacy classes \cite[2.2.3]{Cal_MSJ}. 

\begin{lemma}
\label{quasimorphism_class_function}
A homogeneous quasimorphism $\phi \colon G \to \mathbb{R}$ satisfies 
$\phi(g^{-1}hg)=\phi(h)$ for all $g,h \in G$. 
\end{lemma}

To estimate scl, the following Bavard's Duality Theorem is useful.

\begin{theorem}[(Bavard's Duality Theorem \cite{Bavard})]
For $g \in [G, G]$,
\[ \mathrm{scl}_{G}(g)=\sup_{\phi} \frac{|\phi(g)|}{2D(\phi)} \]
where $\phi\colon G \rightarrow \mathbb{R}$ runs all homogeneous quasimorphisms of $G$ which are not homomorphisms. 
\end{theorem}


As a demonstration of a usefulness of Bavard's duality, we prove the following well-known fact (although this is directly proved by noting that $(gh)^{2n}g^{-2n}h^{-2n}$ is a product of $n$ commutators). 


\begin{lemma}
\label{scl_product}
Let $G$ be a group, and let $g, h$ be non-trivial elements of $G$. 
Assume that abelianization $G/[G, G]$ is a finite group, 
or $g, h \in [G, G]$ \(hence $gh \in [G, G]$ as well\). 
Then we have 
\[
\mathrm{scl}_G(gh) \ge \mathrm{scl}_G(g) -\mathrm{scl}_G(h) -\frac{1}{2}.
\]
\end{lemma}
\begin{proof}
Put $g=ab$, $h=b^{-1}$; 
if $g, h \in [G, G]$, then $a, b, ab \in [G, G]$. 
The assertion is equivalent to the assertion  
\[
\mathrm{scl}_G(ab) \le  \mathrm{scl}_G(a) + \mathrm{scl}_G(b) +\frac{1}{2}. 
\]
Even when $a, b, ab \not\in [G, G]$, 
since $G/[G, G]$ is finite, 
we may assume that $a^m, b^n, (ab)^{\ell} \in [G, G]$ for some integers $m, n, \ell > 0$. 
For every $\varepsilon>0$, 
Bavard's duality implies that there exists a homogeneous quasimorphism $\phi\colon G\rightarrow \mathbb{R}$ such that 
$\mathrm{scl}((ab)^{\ell}) -\varepsilon < \frac{\phi((ab)^{\ell})}{2D(\phi)}$. 
Then
\begin{align*}
\mathrm{scl}(ab)-\varepsilon  
& = \frac{\mathrm{scl}((ab)^{\ell})}{\ell} - \varepsilon
<  \frac{\mathrm{scl}((ab)^{\ell})}{\ell} - \frac{\varepsilon}{\ell} = \frac{1}{\ell}\left(\mathrm{scl}((ab)^{\ell}) -\varepsilon \right) \\
&
< \frac{1}{\ell} \frac{\phi((ab)^{\ell})}{2D(\phi)}
= \frac{\phi(ab)}{2D(\phi)} 
\leq \frac{\phi(a)}{2D(\phi)}+ \frac{\phi(b)}{2D(\phi)} + \frac{D(\phi)}{2D(\phi)}\\
& \leq  \frac{|\phi(a)|}{2D(\phi)}+ \frac{|\phi(b)|}{2D(\phi)} +\frac{1}{2} 
 = \frac{1}{m} \frac{|\phi(a^m)|}{2D(\phi)} + \frac{1}{n} \frac{|\phi(b^n)|}{2D(\phi)} +\frac{1}{2} \\
& \leq \frac{\mathrm{scl}(a^m)}{m} + \frac{\mathrm{scl}(b^n)}{n}  +\frac{1}{2} \\
& = \mathrm{scl}(a) + \mathrm{scl}(b) +\frac{1}{2}.
\end{align*}
\end{proof}


\medskip

Let $M$ be a hyperbolic $3$--manifold and $g$ a non-trivial element in $\pi_1(M)$. 
Recall that $\ell_M(g)$ denotes the length of a unique closed geodesic $c_g$ 
in a free homotopy class of $g$.
While hyperbolic length enjoys the property described in Lemmas~\ref{scl_g^k}, 
it is not known to satisfy the monotonicity property described in Lemma~\ref{monotonicity} for 
the epimorphism $p_s \colon G(K) \to  \pi_1(K(r))$. 

\smallskip

The next result due to Calegari \cite{Cal_GAFA} relates
stable commutator length $\textrm{scl}_{\pi_1(M)}(g)$ and hyperbolic length $\ell_M(g)$. 

\begin{theorem}[(Length inequality \cite{Cal_GAFA})]
\label{theorem:length-inequality}
For any $\varepsilon>0$ there exists a constant $\delta(\varepsilon)>0$ 
such that if $M$ is a closed hyperbolic $3$--manifold $M$ and 
$g \in \pi_1(M)$ with $\mathrm{scl}_{\pi_1(M)}(g) \le \delta(\varepsilon)$, 
then $\ell_{M}(g) \leq \varepsilon$.
\end{theorem}


As a consequence of Theorem~\ref{theorem:length-inequality} and Proposition~\ref{L>0}, we obtain: 

\begin{theorem}[(Lower bound of scl over hyperbolic surgery slopes)]
\label{scl_bound}
Let $K$ be a hyperbolic knot and $g \in G(K)$ a non-peripheral element. 
Then we have a constant $\delta_g > 0$ such that for all hyperbolic surgery slopes $s \in \mathbb{Q}$, 
\[
\mathrm{scl}_{\pi_1(K(s))}(p_s(g)) > \delta_g > 0\quad \mbox{whenever}\quad p_s(g) \neq 1.
\]
\end{theorem}

\begin{proof}
Since $g$ is a non-peripheral element, 
Proposition~\ref{L>0} shows that $L(g) > 0$. 
Put $\varepsilon = L(g)/2 > 0$. 
Then Theorem~\ref{theorem:length-inequality} shows that there exists a constant 
$\delta(\varepsilon)> 0$ such that 
if $\textrm{scl}_{\pi_1(K(s))}(p_s(g)) \le \delta(\varepsilon)$ for $p_s(g) \ne 1$ in $\pi_1(K(s))$, 
then 
\[
\ell_{K(s)}(p_s(g)) \le \varepsilon = L(g)/2 < L(g).
\] 
This contradicts the definition of $L(g)$. 
Hence $\textrm{scl}_{\pi_1(K(s))}(p_s(g)) > \delta(\varepsilon) = \delta(L(g)/2)$ for $p_s(g) \ne 1$. 
Put $\delta_g = \delta(\varepsilon) = \delta(L(g)/2)$ to obtain the desired bound. 
\end{proof}

\medskip

On the contrary, 
if $g$ is a peripheral element, 
i.e. it is conjugate to $\gamma^d$ for some integer $d > 0$, 
where $\gamma = \mu^a \lambda^b$ for some relatively prime integers $a$ and $b$,
then 
$p_{s}(g) = 1$ and hence $\mathrm{scl}_{\pi_1(K(s))}(p_s(g)) = 0$ for $s = a/b \in \mathbb{Q}$. 
Furthermore, we may prove the following.

\begin{proposition}
\label{scl_bound_peripheral}
Let $K$ be a hyperbolic knot and $g \in G(K)$ a peripheral element other than longitudial one. 
Then, for any $\varepsilon > 0$, there exists $s \in \mathbb{Q}$ such that 
\[
\mathrm{scl}_{\pi_1(K(s))}(p_s(g)) < \varepsilon\quad \textrm{and}\ p_s(g) \ne 1 \in \pi_1(K(s)). 
\]
\end{proposition}

\begin{proof}
Taking a suitable conjugation and its inverse, we may assume $g = \gamma^d$ for some integer $d > 0$, 
where $\gamma = \mu^a \lambda^b$ for some relatively prime integers $a$ and $b \ge 0$. 
By the assumption $\gamma \ne \lambda$, $a \ne 0$. 

In the following we show 
$\mathrm{scl}_{\pi_1(K(s))}(p_s(\gamma)) < \varepsilon' = \varepsilon/d$, 
because
\[
\mathrm{scl}_{\pi_1(K(s))}(p_s(g)) = \mathrm{scl}_{\pi_1(K(s))}(p_s(\gamma^d)) = d \cdot \mathrm{scl}_{\pi_1(K(s))}(p_s(\gamma)).
\] 
Let us consider $\gamma^{n} \lambda^{-1} = \mu^{an} \lambda^{bn -1} \in \pi_1(\partial(E(K))$. 

\begin{claim}
\label{coprime}
There are infinitely many integers $n$ such that $an$ and $bn - 1$ are relatively prime. 
\end{claim}

\begin{proof}
If $b = 0$, then $a = \pm 1$ and $\mathrm{GCD}(an, bn-1) = (\pm n, -1) = 1$ for all integers $n$. 
So in the following, we assume $b > 0$. 

Suppose for a contradiction that there are at most finitely many integers $n$ such that $an$ and $bn - 1$ are relatively prime. 
Following Dirichlet's theorem on arithmetic progression, 
$\{ bn-1 \}$ contains infinitely many prime numbers. 
So we take such infinitely many integers $n$ so that $bn-1$ is prime and $bn -1 > |a|$. 
If $an$ and $bn - 1$ are not relatively prime, 
then since $bn -1$ are prime, 
$|an|$ is a multiple of $bn-1$. 

First assume that $|a| < b$. 
Then we $|an| < bn -1$ for sufficiently large $n > 0$. 
So, we have $|a| > b$, and suppose that $|an| = k(bn-1)$ for some integer $k$. 
Since $bn-1$ is prime, this means that $|a|$ or $n$ is a multiple of $bn-1$. 
For suitably large $n$, $bn-1 > |a|$ and the first cannot happen. 
Since $b \ge 1$ and $n \ge 2$, the latter is also impossible. 
\end{proof}



Let us put $s_n = \frac{an}{bn-1} \in \mathbb{Q}$. 
Then $p_{s_n}(\gamma^{n}\lambda^{-1}) = p_{s_n}(\mu^{an} \lambda^{bn-1}) =1$, 
namely $p_{s_n}(\gamma)^n = p_{s_n}(\lambda)$ in $\pi_1(K(s_n))$. 
Thus 
\[
n \cdot \mathrm{scl}_{\pi_1(K(s_n))}(p_{s_n}(\gamma)) = \mathrm{scl}_{\pi_1(K(s_n))}(p_{s_n}(\gamma)^n) = \mathrm{scl}_{\pi_1(K(s_n))}(p_{s_n}(\lambda)).
\]
Since $\textrm{scl}_{G(K)}(\lambda) = g(K)-\frac{1}{2} < g(K)$ (Remark~\ref{scl_longitude} below), we have
\[
 \mathrm{scl}_{\pi_1(K(s_n))}(p_{s_n}(\gamma)) = \frac{\mathrm{scl}_{\pi_1(K(s_n))}(p_{s_n}(\lambda))}{n} \le \frac{\mathrm{scl}_{G(K)}(\lambda)}{n} <  \frac{g(K)}{n} < \varepsilon'
 \]
 for $n > \frac{g(K)}{\varepsilon'}$. 
This then implies 
\[
\mathrm{scl}_{\pi_1(K(s_n))}(p_{s_n}(g)) < \varepsilon\quad \textrm{for}\ n > \frac{d \cdot g(K)}{\varepsilon}
\]
as mentioned above. 

Finally we observe that $p_{s_n}(g) \ne 1$ for all but finitely many $n$. 
If $p_{s_n}(g) = 1 \in \pi_1(K(s_n))$, 
then $g = \gamma^d \in  \langle\!\langle s_n \rangle\!\rangle$. 
By Proposition~\ref{slope} $s_n$ is represented by the slope element $\gamma$ or $s_n$ is a finite surgery slope. 
For any hyperbolic knot $K$ the number of such surgeries is finite. 

This completes a proof. 
\end{proof}

\begin{remark}
\label{scl_longitude}
We remark that for a non-trivial knot $K$ 
the stable commutator length of $\lambda$ is expressed by the knot genus as follows \cite[Proposition 4.4]{Cal_MSJ}\textup{:} 
\[
\mathrm{scl}_{G(K)}(\lambda)=g(K)-\frac{1}{2} \ge \frac{1}{2}.
\]
Hence, 
\[
\mathrm{scl}_{\pi_1(K(s))}(p_s(\lambda)) \le \mathrm{scl}_{G(K)}(\lambda) =g(K)-\frac{1}{2}.
\]
\end{remark}



\medskip




Even $L(g) > 0$, 
Theorem~\ref{scl_bound} does not assert that 
$p_s(g) \ne 1$ for ``all'' hyperbolic surgery slopes. 
To find an element $g' \in G(K)$ such that $p_s(g') \ne 1$ for ``all'' hyperbolic surgery slopes, 
we need further procedure.



Using stable commutator length, 
we give an efficient way to construct elements in $G(K)$ which are not trivialized by all hyperbolic Dehn fillings.
See Lemma~\ref{non_vanish_non_finite}, 
whose proof requires the following result. 


For any non-peripheral element $g \in G(K)$, 
$\delta_g$ denotes a positive constant given by Theorem~\ref{scl_bound}. 

\begin{proposition}
\label{construction}
Let $K$ be a hyperbolic knot in $S^3$ and $x$ a non-trivial element in $G(K)$. 
Assume that 
$s$ is a hyperbolic surgery slope, 
and $p_s(x)$ is non-trivial in $\pi_1(K(s))$. 
\begin{enumerate} 
\renewcommand{\labelenumi}{(\arabic{enumi})}
\item If $x$ is non-peripheral \(i.e. $L(x) > 0$\), 
then for an element $y \in [G(K),G(K)]$ and $p > \frac{\mathrm{scl}_{G(K)}(y)}{\delta_x}$, 
$p_s(x^{-p}y) \neq 1$ for the hyperbolic surgery slope $s$. 
\item If $x \in [G(K),G(K)]$, 
then for a non-peripheral element $y \in G(K)$ \(i.e. $L(y) > 0$\) and $p > \frac{\mathrm{scl}_{G(K)}(x)}{\delta_y}$, 
$p_s(x^{-1}y^{p}) \neq 1$ for the hyperbolic surgery slope $s$. 
\end{enumerate}
\end{proposition}

\begin{proof}
$(1)$\  
Assume to the contrary that  $p_s(x^{-p}y) = 1$, i.e. $p_s(x^p) = p_s(y)$. 
Then we have 
\begin{align*}
\textrm{scl}_{G(K)}(y) & < p\, \delta_x\\
&< p\ \textrm{scl}_{\pi_1(K(s))}(p_s(x)) \quad (Theorem~\ref{scl_bound}) \\
& = \textrm{scl}_{\pi_1(K(s))}(p_s(x)^p) \quad (Lemma~\ref{scl_g^k}) \\
& = \textrm{scl}_{\pi_1(K(s))}(p_s(x^{p}))\\
& = \textrm{scl}_{\pi_1(K(s))}(p_s(y))\\
& \leq \textrm{scl}_{G(K)}(y)\quad (Lemma~\ref{monotonicity}). 
\end{align*}
This is a contradiction.

$(2)$\  
Assume for a contradiction that $p_s(x^{-1}y^p) = 1$, i.e. $p_s(x) = p_s(y^p)$. 
Since $p_s(x)$ is non-trivial, $p_s(y^p) = p_s(x) \ne 1$, in particular $p_s(y) \ne 1$. 
Thus we have
\begin{align*}
\textrm{scl}_{G(K)}(x) & <  p\, \delta_y \\
& < p \ \textrm{scl}_{\pi_1(K(s))}(p_s(y)) \quad (Theorem~\ref{scl_bound})\\
& = \textrm{scl}_{\pi_1(K(s))}(p_s(y)^p)\quad (Lemma~\ref{scl_g^k})\\
& = \textrm{scl}_{\pi_1(K(s))}(p_s(y^p))\\
& = \textrm{scl}_{\pi_1(K(s))}(p_s(x))\\
& \leq  \textrm{scl}_{G(K)}(x)\quad (Lemma~\ref{monotonicity}). 
\end{align*}
This is a contradiction. 
\end{proof}

\medskip



\section{Persistent elements in the commutator subgroup} 
\label{homologically0}

We say that $g \in G(K)$ is a \textit{persistent element} if $\mathcal{S}_K(g) = \emptyset$; 
a persistent element survives for all non-trivial Dehn fillings. 

The simplest example of persistent element is a meridian or its conjugate,
although this assertion is already highly non-trivial, 
because it is a consequence of the Property P. 
By Propositions 2.5 and 3.1, it follows that the non-trivial powers of the meridian is also persistent 
as long as K has no finite surgery slopes. 
More generally, a \emph{pseudo-meridian} \cite{SWW}, 
an element $g$ that normally generates $G(K)$, 
but it is not an image of meridian or its inverse under any automorphism of $G(K)$, is also a persistent element.  

Actually if a normal generator $g \in G(K)$ becomes trivial in $\pi_1(K(r))$ for some $r \in \mathbb{Q}$, 
then $g \in  \langle\!\langle r \rangle\!\rangle$.  
This implies $G(K)/\langle\!\langle r \rangle\!\rangle = \{ 1 \}$, 
and hence the Property P says that $r = \infty$, a contradiction. 

See \cite{SWW,Tsau,Dutra,Te_weight} for various examples of pseudo-meridians. 
Although it is conjectured that every non-trivial knot admits a pseudo-meridian \cite{SWW},  
this conjecture is widely open. 

On the other hand, since these elements are homologically non-trivial,
it is not clear whether a homologically trivial persistent element,
i.e. a persistent element in the commutator subgroup $[G(K),G(K)]$, exists or not.

\begin{example}[\(Figure-eight knot\)]
\label{fig-eight}
Let $K$ be the figure-eight knot with 
\[
G(K) = \langle a, b \mid ab^{-1}a^{-1}ba = bab^{-1}a^{-1}b \rangle,
\] 
where $a$ and $b$ are meridians as indicated by Figure~\ref{figure-eight_knot_a_b}. 
If $p_r([a,b])=1$ for some $r \in \mathbb{Q}$, 
then $\pi_1(K(r))$ must be cyclic.
This is impossible, because $K$ does not admit a cyclic surgery.
Thus the commutator $[a,b]$ is a persistent element of $G(K)$. 

A similar argument applies whenever $G(K)$ is generated by two elements and $K$ has no cyclic surgeries. 
\end{example}

\begin{figure}[htb]
\centering
\includegraphics[bb=0 0 121 113,width=0.2\textwidth]{figure-eight_knot_a_b.eps}
\caption{A persistent element $[a, b]$ for the figure-eight knot.} 
\label{figure-eight_knot_a_b}
\end{figure}


If $s$ is a cyclic surgery slope, but it is not a finite surgery slope, 
then $\pi_1(K(s)) \cong \mathbb{Z}$. 
However, this happens only when $K$ is the unknot and $s = 0$ \cite{GabaiIII}. 
Thus for non-trivial knots cyclic surgeries are finite surgeries. 

If $K$ has a cyclic surgery slope $s$, 
then $\pi_1(K(s)) \cong G(K)/ \langle\!\langle s \rangle\!\rangle$ is abelian, 
and $[G(K),G(K)] \subset \langle\!\langle s \rangle\!\rangle$. 
Thus for every element $g$ in $[G(K),G(K)]$, 
$p_s(g) = 1$ in $\pi_1(K(s))$. 

The goal of this section is to prove Theorem~\ref{persistent_[G,G]_noncyclic}, 
which was announced in \cite{Mo}. 

\begin{thm_persistent_[G,G]_noncyclic}
Let $K$ be a non-trivial knot in $S^{3}$.  
Then there exist infinitely many, mutually non-conjugate elements $g \in [G(K),G(K)]$ 
such that $p_s(g) \neq 1$ in $\pi_1(K(s))$ for all non-cyclic surgery slopes $s \in \mathbb{Q}$. 
\end{thm_persistent_[G,G]_noncyclic}

This theorem immediately implies the following corollary. 

\begin{corollary}
\label{persistent_[G,G]}
Let $K$ be a non-trivial knot in $S^{3}$.  
Then there exist infinitely many, mutually non-conjugate persistent elements $g \in [G(K),G(K)]$ if and only if $K$ has no cyclic surgeries.
\end{corollary}


Since a normal generator should be a homological generator, 
any persistent element in $[G(K), G(K)]$ cannot be a normal generator. 
Thus Corollary~\ref{persistent_[G,G]} shows 
an existence of a persistent element which is not a normal generators of $G(K)$. 

\medskip


According to Thurston's hyperbolization theorem, 
knots are classified into three classes depending upon topology of their exteriors: 
torus knots, hyperbolic knots, and satellite knots. 

For clarification of our discussion we divide Theorem~\ref{persistent_[G,G]_noncyclic} into three 
propositions according to the above classification. 

For notational simplicity, we will use the following notation. 
For $g, h \in G$, $g^h = h^{-1} g h \in G$. 

We first prove Theorem \ref{persistent_[G,G]_noncyclic} for torus knots. 


\begin{proposition}
\label{persistent_[G,G]_torus}
Let $K$ be a torus knot in $S^{3}$. 
Then there exist infinitely many, mutually non-conjugate elements 
$g \in [G(K),G(K)]$ such that $p_s(g) \neq 1$ for all non-cyclic surgery slopes $s \in \mathbb{Q}$. 
\end{proposition}

\begin{proof}
Recall that 
for the $(p,q)$-torus knot $K=T_{p,q}$ $(0<p<|q|)$, the knot group is presented as
\[
G(K)
 = \langle x,y \: | \: x^{p}=y^{q} \rangle 
 = \langle x\rangle \ast_{\langle x^{p}=y^{q} \rangle} \langle y\rangle
 \]
as the amalgamated free product of two infinite cyclic groups. 
In particular, 
$G(K)$ is generated by two elements $x$ and $y$. 
Let $g = [x, y]$. 
If $p_s(g) = 1$, then $\pi_1(K(s))$ is abelian, and hence it is cyclic. 
Thus $s$ is a cyclic surgery. 
This shows that $p_s(g) \ne 1$ for any non-cyclic surgery slope $s$ of $K$. 

For $n>0$ let $w_n=y(xy)^{n+1}$ and $g_n=g^{g^{w_n}}g^{-2}$. 
Since $g \in [G(K), G(K)]$, 
so does $g_n$. 
By Corollary~\ref{same_trivialization_residually_finite}, 
$p_s(g_n)\neq 1$ for all non-cyclic surgery slope $s$ of $K$.

Let us show that $g_n$ and $g_m$ are conjugate if and only if $n = m$.  
Note that 
\[
W_n = g^{w_n} = (y(xy)^{n+1})^{-1} g (y(xy)^{n+1})
\]
and 
\[
\begin{split}
g_n g^{-2} & = g^{g^{w_n}} g^{-2} = g^{W_n} g^{-2} = (W_n)^{-1} g W_n\, g^{-2} \\
& = \left( (y(xy)^{n+1})^{-1} g  (y(xy)^{n+1}) \right)^{-1} g  \left( (y(xy)^{n+1})^{-1} g  (y(xy)^{n+1}) \right)\, g^{-2}.
\end{split}
\]
Put $X=x^{-1}$, $Y=y^{-1}$ for a simplicity of notation.
Then 
\[
\begin{split}
g_n &  = (y(xy)^{n+1})^{-1} [x,y]^{-1} (y(xy)^{n+1}) [x,y] (y(xy)^{n+1})^{-1}[x,y]((y(xy))^{n+1})\\
& \quad [x,y]^{-2}\\
& =
 ( (YX)^{n}YXY) (yxYX) (y(xy)^{n+1}) (xyXY)  ((YX)^{n+1}Y) (xyXY)\\
 &\quad (yxy(xy)^{n-1}xy) (yxYX)^{2}\\
& = (YX)^{n}Y^{2}X y(xy)^{n+2}  XY^{2}X(YX)^{n}Y xy^{2}(xy)^{n-1}xy^2xYXyxYX.
\end{split}
\]

The last word is cyclically reduced, 
it follows from \cite[Corollary~4.6]{MKS} that 
$g_n$ and $g_m$ are not conjugate. 
\end{proof}

For later convenience we note the following, 
which will be used in the proof of Proposition~\ref{cable_torus}. 

\begin{remark}
\label{notP(K)_torus}
$g,\ g_n$ are non-peripheral in $G(K)$. 
\end{remark}

\begin{proof}
We show that $g$ is not conjugate into $\pi_1(\partial E(K))$. 
Assume for a contradiction that $h^{-1} g h \in \pi_1(\partial E(K))$ for some $h \in G(K)$. 
Then since $g \in [G(K), G(K)]$, 
$h^{-1} g h$ is null-homologus, and thus it is $\lambda_K^m$ for some integer $m$. 
Then $p_0(g) = p_0(h^{-1}gh) = 1$ in $\pi_1(K(0))$, which is non-cyclic. 
This is a contradiction.  
The proof for $g_n$ is identical. 
\end{proof}

Next we prove Theorem~\ref{persistent_[G,G]_noncyclic} for hyperbolic knots. 

\begin{proposition}
\label{persistent_[G,G]_hyp}
Let $K$ be a hyperbolic knot in $S^{3}$.  
Then there exist infinitely many, mutually non-conjugate elements $x \in [G(K), G(K)]$ such that $p_s(x) \neq 1$ for all non-cyclic surgery slopes $s \in \mathbb{Q}$. 
\end{proposition}

\begin{proof}
A longitude $\lambda$ of $K$ is the unique slope element which belongs to $[G(K), G(K)]$.  
As a particular case of Proposition~\ref{slope}, 
we have 
\medskip

\begin{lemma}
\label{p(lambda^q)=1}
If $p_s(\lambda^q) = 1$ for some $q \ne 0$, 
then $s=0$ or $s$ is a finite surgery slope. 
\end{lemma}

\begin{proof}
The assumption $p_s(\lambda^q) = 1$ implies that 
$\lambda^q \in \langle\!\langle s \rangle\!\rangle$. 
By Proposition~\ref{slope} $s = 0$ or $s$ is a finite surgery slope. 
\end{proof}

Let $\Sigma$ be a minimal genus Seifert surface of a hyperbolic knot $K$.  
Take a non-trivial element $y \in \pi_1(\Sigma)$ so that $[y] \ne 0$ in $H_1(\Sigma)$, 
equivalently $y \not\in [\pi_1(\Sigma), \pi_1(\Sigma)]$. 
(In particular, $y$ is not conjugate into $\pi_1(\partial \Sigma)$ in $\pi_1(\Sigma)$.)
Let $i\colon \Sigma \rightarrow E(K)$ be the inclusion map. 
Since $\Sigma$ is incompressible, 
$i_* \colon \pi_1(\Sigma) \to G(K)$ is injective. 
So we may regard $\pi_1(\Sigma) \subset G(K)$ and 
use the same symbol $y$ to denote $i_*(y) \in \pi_1(\Sigma) \subset G(K)$. 

\begin{claim}
\label{non-conjugate_non-peripheral}
$y$ is not conjugate into $\pi_1(\partial E(K))$ in $G(K)$, 
i.e. it is non-peripheral in $G(K)$. 
\end{claim}

\begin{proof}
Assume for a contradiction that $y$ is conjugate to $y' \in \pi_1(\partial E(K))$ in $G(K)$. 
Since $y \in [G(K), G(K)]$, so is $y'$, 
and we may assume a representative of $y'$ lies in $\partial \Sigma$. 
Thus we write $y' = \lambda^{\ell}$ for some integer $\ell \ne 0$. 
So there exists an element $\gamma \in G(K)$ such that $\gamma y \gamma^{-1} = \lambda^{\ell}$. 

Let $\widehat{\Sigma}$ be a non-separating closed surface in $K(0)$ obtained from $\Sigma$ by capping off 
$\partial \Sigma$ by a meridian disk of the filling solid torus. 
Then the inclusion $\Sigma \to \widehat{\Sigma}$ induces a monomorphism.  
Since $[y] \ne 0 \in H_1(\Sigma)$, 
we also have $[y] \ne 0$ in $H_1(\widehat{\Sigma})$.  
Hence, $y$ remains non-trivial in $\pi_1(\widehat{\Sigma})$.
Then \cite[Proof of Corollary~8.3]{GabaiIII} shows that 
$\widehat{\Sigma}$ is Thurston norm minimizing, 
and hence incompressible in $K(0)$.  
Hence $y$ is non-trivial in $\pi_1(K(0))$. 

On the other hand, since $\gamma y \gamma^{-1} = \lambda^{\ell}$ in $G(K)$, 
$y$ would be trivial in $\pi_1(K(0))$, a contradiction. 
Therefore $y$ is not conjugate into $\pi_1(\partial E(K))$ in $G(K)$.   
\end{proof}

Hence Proposition~\ref{L>0} shows that $L(y) > 0$. 
In what follows we pick and fix such an element $y$. 

\begin{lemma}
\label{non_finite}
Let $s$ be a non-finite surgery slope and $q$ a positive integer. 
Assume that $p_s(\lambda^q) = p_s(y)^p$ for some integer $p > 0$. 
Then we have\textup{:} 
\begin{enumerate}
\renewcommand{\labelenumi}{(\arabic{enumi})}
\item
$p_s(\lambda^q) \ne 1$, and 
\item 
$p$ is the unique integer which satisfies $p_s(\lambda^q) = p_s(y)^p$.  
We denote this integer $p$ by $p_{q, s}$. 
\end{enumerate}
\end{lemma}

\begin{proof}
(1)\ Suppose for a contradiction that $p_s(\lambda^q) = 1$. 
Then, since $s$ is not a finite surgery slope, 
Lemma~\ref{p(lambda^q)=1} shows that $s = 0$. 

Let $\widehat{\Sigma}$ be a non-separating closed surface in $K(0)$ 
obtained from $\Sigma$ by capping off $\partial \Sigma$ by a meridian disk of the filling solid torus as in the proof of Claim~\ref{non-conjugate_non-peripheral}.
The choice of $y$, $[y] \ne 0 \in H_1(\Sigma)$, 
assures that $[y]$ and $[y^p]$ are also non-trivial in $H_1(\widehat{\Sigma})$, 
and hence $y$ and $y^p$ are non-trivial in $\pi_1(\widehat{\Sigma})$.  
Since $\widehat{\Sigma}$ is incompressible in $K(0)$ (\cite[Proof of Corollary~8.3]{GabaiIII}), 
it turns out that $p_0(y^p) \ne 1$ in $\pi_1(K(0))$. 
So we have $1 = p_0(\lambda) = p_0(\lambda^q) = p_0(y)^p = p_0(y^p) \ne 1$ in $\pi_1(K(0))$, 
a contradiction. 

\smallskip

(2)\ 
Suppose for a contradiction that $p_s(y)^{p'} = p_s(\lambda^q)  = p_s(y)^{p}$ for $p' \ne p$. 
Then $p_{s}(\lambda^q)^{p'} = (p_s(y)^p)^{p'} = (p_s(y)^{p'})^p = p_s(\lambda^q)^p$, 
i.e. $p_s(\lambda^{(p' - p)q}) = 1$ ($(p' - p)q \ne 0$).  
So by Lemma~\ref{p(lambda^q)=1} the non-finite surgery slope $s$ must be $0$. 
Also we have $p_0(y)^{p' - p} = 1$. 
As shown in (1) $p_{0}(y) \ne 1$, 
so $0 < |p' - p| \ne 1$, and hence $|p' - p| \ge 2$. 
Thus $p_{0}(y)$ is a torsion element in $\pi_1(K(0))$. 
However since $K(0)$ is irreducible \cite{GabaiIII} and $|\pi_1(K(0))| = \infty$, 
$\pi_1(K(0))$ has no torsion. 
This is a contradiction. 
\end{proof}

Following Thurston's hyperbolic Dehn surgery theorem 
there are at most finitely many non-hyperbolic surgery slopes, 
in particular there are at most finitely many non-hyperbolic, non-finite surgery slopes $s_1, \dots, s_m$. 
For each such surgery slope $s_i$ ($1 \le i \le m$), 
at most one integer $p_{q, s_i} \ge 1$ satisfies $p_{s_i}(\lambda^q)=p_{s_i}(y)^{p_{q, s_i}}$; 
see Lemma~\ref{non_finite}(2). 

Recall that $L(y) > 0$ for the non-peripheral element $y \in \pi_1(\Sigma) \subset G(K)$. 
Then for $\delta_y > 0$ given in Theorem~\ref{scl_bound}, 
we may take $p$ so that $p \delta_y > \textrm{scl}_{G(K)}(\lambda^q)$, 
equivalently $p > \dfrac{\textrm{scl}_{G(K)}(\lambda^q)}{\delta_y}$. 
The right hand side is a constant (once we fix a positive integer $q$), 
which we denote by $C_{q} > 0$. 


\begin{lemma}
\label{non_vanish_non_finite}
For any non-finite surgery slope $s$ we have 
$p_s(\lambda^{-q} y^{p}) \ne 1$ if $p > C_{q}$ and $p \ne p_{q, s_1}, \dots, p_{q, s_m}$. 
\end{lemma}

\begin{proof}
Assume for a contradiction that $p_s(\lambda^{-q} y^{p})=1$, 
i.e. $p_s(\lambda^q) = p_s(y^p) = p_s(y)^p$ for some non-finite surgery slope $s$. 
Note that $0$--slope is a non-finite surgery slope for homological reason, and obviously 
$p_0(\lambda^q) = 1$. 
So Lemma~\ref{non_finite} shows that $s \ne 0$, i.e. $p_0(\lambda^{-q} y^{p}) \ne 1$. 

\smallskip

Since $p_s(\lambda^q) \ne 1$ (Lemma~\ref{non_finite}), 
which also implies $p_s(y)^p \ne 1$, in particular $p_s(y) \ne 1$. 
Let us divide our arguments into two cases according as $s$ is a hyperbolic surgery or not. 

\medskip

\noindent
\textbf{Case 1.}\ $s$ is a hyperbolic surgery slope. 

Since $p_s(y) \ne 1$,  
Theorem~\ref{scl_bound} shows that $\textrm{scl}_{\pi_1(K(s))}(p_s(y)) > \delta_y > 0$. 
Following the assumption $p > \dfrac{\textrm{scl}_{G(K)}(\lambda^q)}{\delta_y}$, 
we may apply Proposition~\ref{construction}(2) with 
$x = \lambda^q$ to see that $p_s(\lambda^{-q}y^p) \ne 1$.  
This contradicts the initial assumption of the proof of Lemma~\ref{non_vanish_non_finite}. 

\medskip

\noindent
\textbf{Case 2.}\ 
$s$ is a non-hyperbolic surgery slope. 

Then $s = s_i$ for some $1 \le i \le m$, and
$p_{s_i}(\lambda^q) \ne p_{s_i}(y)^p$ since $p \neq p_{q, s}$ (Lemma~\ref{non_finite}).  
Thus $p_s(\lambda^{-q}y^p) \ne 1$. 
This contradicts the initial assumption of the proof of Lemma~\ref{non_vanish_non_finite}. 
\end{proof}

\medskip

Let us turn to the case where $s$ is a finite surgery slope.  
To consider finite surgeries we need to pay attention for the choice of 
$y \in \pi_1(\Sigma)$ and a positive integer $q$. 
(Accordingly we will re-take $y \in \pi_1(\Sigma) \subset G(K)$ and the positive integer $q$ in the above as well.)

Let $\mathcal{F} = \{ f_1, \dots, f_n\} \subset \mathbb{Q}$ be the set of finite surgery slopes which are not cyclic surgery slopes.


\begin{claim}
\label{number_finite}
$n \le 2$.
\end{claim}

\begin{proof}
Ni and Zhang \cite{Ni-Zhang-Finite} show that 
a hyperbolic knot in $S^{3}$ has at most three finite surgeries, and thus $n \le 3$; 
furthermore, if $n = 3$, then $K$ is the pretzel knot $P(-2, 3, 7)$. 
Recall that $P(-2, 3, 7)$ has two cyclic surgeries and one non-cyclic finite surgery. 
Hence $n \le 2$. 
\end{proof}

\medskip 

It is known that any knot admitting a non-trivial finite surgery must be a fibered knot,
and hence if $\mathcal{F} \ne \emptyset$, 
then $K$ is a fibered knot.  
See \cite{Ni}(cf. \cite{Ghi,Juh}).

Write $n_{f_i} = |\pi_1(K(f_i))|$ for each finite surgery slope $f_i$, 
and let $q_0$ be the least common multiples of $n_{f_1}, \dots, n_{f_n}$. 
Then for every element $g \in G(K)$,  
we have $p_{f_i}(g)^{q_0} = 1$ for $1 \le i \le n$. 

We begin by observing the following. 

\begin{lemma}
\label{non-peripheral}
There exists a non-peripheral element $y \in \pi_1(\Sigma) \subset G(K)$ such that
$p_{f_i}(y) \ne 1$ in $\pi_1(K(f_i))$ for any finite \(non-cyclic\) surgery slope $f_i$.  
\end{lemma}

\begin{proof}
Since $K$ is fibered, $\pi_1(\Sigma)=[G(K),G(K)]$. 
For each non-cyclic (non-abelian) surgery slope $f_i\in \mathcal{F}$, 
we have $[G(K),G(K)] - \langle\!\langle f_i \rangle\!\rangle 
= \pi_1(\Sigma) - \langle\!\langle f_i \rangle\!\rangle \ne \emptyset$. 
In fact, if $[G(K),G(K)] - \langle\!\langle f_i \rangle\!\rangle = \emptyset$, 
then $ \langle\!\langle f_i \rangle\!\rangle \supset [G(K),G(K)]$, and hence 
$\pi_1(K(f_i))$ is abelian, in particular cyclic. 
This contradicts $f_i$ being a non-cyclic surgery slope. 

Take a non-trivial element $y_i \in \pi_1(\Sigma) - \langle\!\langle f_i \rangle\!\rangle \ne \emptyset$. 
Then $p_{f_i}(y_i) \ne 1$.  
If $\mathcal{F} = \{ f_1 \}$, then put $y = y_1$. 
If $\mathcal{F} = \{ f_1, f_2 \}$, 
Then we put $y$ as follows;
\[ y= \begin{cases}
 y_1 & \mbox{ if } p_{f_2}(y_1)\neq 1, \\
 y_2 & \mbox{ if } p_{f_1}(y_2)\neq 1, \\
y_1 y_2 & \mbox{ if } p_{f_1}(y_2)=p_{f_2}(y_1)=1.
\end{cases}\]

Thus we have a non-trivial element $y \in \pi_1(\Sigma)$ with $p_{f_i}(y) \ne 1$ for $f_i \in \mathcal{F}$. 
Following Claim~\ref{non-conjugate_non-peripheral}, if $y$ is not conjugate into $\pi_1(\partial \Sigma)$ in $\pi_1(\Sigma)$, 
then it is non-peripheral in $G(K)$, and $y$ is a desired element. 

Suppose that $y \in \pi_1(\Sigma)$ is homologically trivial in $H_1(\Sigma)$, i.e. $[y]=0 \in H_1(\Sigma)$. 
Then $y \in [\pi_1(\Sigma), \pi_1(\Sigma)]$.
We replace $y$ with an element $y' \in \pi_1(\Sigma)$ 
which satisfies $[y] \ne 0$ in $H_1(\Sigma)$ and $p_{f_i}(y') \ne 1$ as follows. 
Take $x \not\in [\pi_1(\Sigma), \pi_1(\Sigma)]$ arbitrarily. 
Then by the choice of $q_0$ we have $p_{f_i}(x^{q_0}) = p_{f_i}(x)^{q_0} = 1$, 
and hence $p_{f_i}(x^{q_0}y) = p_{f_i}(x^{q_0}) p_{f_i}(y) = p_{f_i}(y) \ne 1$. 
It remains to show that 
$[x^{q_0}y] \ne 0$ in $H_1(\Sigma)$.  
Assume for a contradiction that $[x^{q_0}y] = 0$ in $H_1(\Sigma)$. 
Then $q_0 [x] + [y] = 0$ in $H_1(\Sigma)$. 
So $q_0 [x] = -[y] = 0$ in $H_1(\Sigma)$, 
which also implies $[x] = 0 \in H_1(\Sigma)$. 
Thus $x \in [\pi_1(\Sigma), \pi_1(\Sigma)]$, contradicting the choice of $x$. 
Hence, $[x^{q_0}y] \ne 0$ in $H_1(\Sigma)$, 
and Claim~\ref{non-conjugate_non-peripheral} shows that $y' = x^{q_0}y$ is non-peripheral in $G(K)$. 
\end{proof}

\medskip

Let us take $y \in \pi_1(\Sigma)$ as in Lemma~\ref{non-peripheral}, and put 
\[
\mathcal{P} = \{ p \mid p > C_{q_0}, \ 
p  \ne p_{q_0, s_1}, \dots, p_{q_0, s_m}\ \textrm{and}\ p \equiv  1\ \textrm{mod}\ q_0 \}. 
\]

Obviously $\mathcal{P}$ contains infinitely many integers $p$. 

We prove that $p_s(\lambda^{-q_0} y^{p}) \ne 1$ for all non-cyclic surgery slopes $s \in \mathbb{Q}$ and $p\in \mathcal{P}$.  
If $s$ is not a finite surgery slope, 
then Lemma~\ref{non_vanish_non_finite} assures that $p_s(\lambda^{-q_0} y^{p}) \ne 1$. 

Suppose that $s$ is a non-cyclic, finite surgery slope $f_i$. 
By definition of $q_0$, $q_0$ is a multiple of $n_{f_i}$ for $f_i \in \mathcal{F}$. 
So we have
\[
p_{f_i}(\lambda)^{q_0} = 1 \quad
\mathrm{and}\quad  
p_{f_i}(y)^{q_0} = 1.
\] 

Thus 
\[
p_{f_i}(\lambda^{-q_0} y^{p}) 
= p_{f_i}(\lambda)^{-q_0} p_{f_i}(y)^{tq_0 + 1} 
= p_{f_i}(y)^{t q_0 + 1}
= p_{f_i}(y) 
\ne 1.
\] 
 
\medskip

Finally, we show that there are infinitely many such elements.

\begin{lemma}
\label{infinite_conjugacy}
There are infinitely many integers $p_i \in \mathcal{P}$ such that 
$\lambda^{-q_0}y^{p_i}$ are mutually non-conjugate elements in $G(K)$. 
\end{lemma}

\begin{proof}
Clearly $\mathcal{P}$ contains infinitely many integers $p_i$ with $p_1 < p_2 < \cdots$. 
Since $\textrm{scl}_{G(K)}(y)>0$ (Theorem~\ref{scl_bound} and Lemma~\ref{monotonicity}), 
by Bavard's duality we have a quasimorphism $\phi\colon G(K) \to \mathbb{R}$ with $|\phi(y)| > 0$. 
If $\phi(y) < 0$, 
then take a quasimorphism $\phi' = - \phi$. 
So we may take a quasimorphism $\phi$ so that $\phi(y) > 0$. 
Then $\phi(\lambda^{-q_0}y^{p_i}) \geq \phi(\lambda^{-q_0}) + p_i\phi(y)- D(\phi)$. 
Hence $\lim_{i \to \infty} \phi(\lambda^{-q_0}y^{p_i}) \to \infty$. 
Since the homogeneous quasimorphism $\phi$ is constant on conjugacy classes, 
this shows that $\{\lambda^{-q_0}y^{p_i}\}$ has infinitely many mutually non-conjugate elements.
\end{proof}

This completes a proof of Proposition~\ref{persistent_[G,G]_hyp}.  
\end{proof}

\medskip

For later convenience we note the following, 
which will be used in the proof of Proposition~\ref{cable_torus}. 

\begin{remark}
\label{notP(K)_hyp}
$\lambda^{-q_0}y^{p_i}$ is non-peripheral in $G(K)$. 
\end{remark}

\begin{proof}
If $\lambda^{-q_0}y^{p_i}$ is conjugate into $\pi_1(\partial E(K))$, 
i.e. $h^{-1}(\lambda^{-q_0}y^{p_i}) h \in \pi_1(\partial E(K))$ for some $h$,
then since $\lambda^{-q_0}y^{p_i} \in [G(K), G(K)]$, 
$h^{-1}(\lambda^{-q_0}y^{p_i}) h$ is null-homologus, 
and thus it is $\lambda_K^m$ for some integer $m$. 
Then 
$p_0(\lambda^{-q_0}y^{p_i}) 
= p_0(h^{-1}(\lambda^{-q_0}y^{p_i}) h) 
= 1$ in $\pi_1(K(0))$, which is non-finite. 
This is a contradiction. 
\end{proof}

\medskip

Let us turn to prove Theorem~\ref{persistent_[G,G]_noncyclic} for satellite knots. 
Note that a satellite knot other than a $(2ab \pm 1, 2)$--cable of $T_{a, b}$ does not admit a cyclic surgery \cite{Wu}.  
In particular, 
knots in Proposition~\ref{persistent_[G,G]_satellite} below has no cyclic surgeries. 
As an application of Propositions~\ref{persistent_[G,G]_torus} and \ref{persistent_[G,G]_hyp} we may obtain: 

\begin{proposition}
\label{persistent_[G,G]_satellite}
Let $K$ be a satellite knot which is not a $(abq \pm 1, q)$-cable of $T_{a, b}$. 
Then there exist infinitely many, mutually non-conjugate elements $x \in [G(K),G(K)]$ such that $p_s(x) \neq 1$ for all slopes $s \in \mathbb{Q}$. 
\end{proposition}

\begin{proof}
The reason why we exclude cable knots $(abq \pm 1, q)$-cable of $T_{a, b}$ will be clarified in Case 3 in the following proof. 
(See also Remark~\ref{cable}.)

Recall that any satellite knot $K$ has a hyperbolic knot or a torus knot as a companion knot $k$. 
Let $V$ be a tubular neighborhood of $k$ containing $K$ in its interior. 
Then $E(K) = E(k) \cup (V - \mathrm{int}N(K))$ and
$G(K)= G(k) \ast_{\pi_1(\partial V)} \pi_1(V - \mathrm{int}N(K))$.

\medskip

\noindent
\textbf{Notations:}\quad
Throughout the proof we distinguish various projection maps as follows: 
\begin{itemize}
\item $p^{K}_s\colon G(K) \rightarrow \pi_1(K(s))$ is the projection induced from $s$--Dehn filling on $K$, 
which we simply denote by $p_s$.
\item $p^{k}_s\colon G(k) \rightarrow \pi_1(k(s))$ is the projection induced from $s$--Dehn filling on a companion knot $k$.
\end{itemize}

For a slope $s$ of $K$ we denote by $V(K; s)$ the manifold obtained from $V$ by $s$-surgery on $K \subset V$. 


\medskip


Since we have already proved Theorem~\ref{persistent_[G,G]_noncyclic} for torus knots and hyperbolic knots, we have:

\medskip

\begin{claim}
\label{companion}
There are infinitely many, mutually non-conjugate elements 
$x \in [G(k), G(k)] \subset [G(K), G(K)]$ such that $p^{k}_s(x)\neq 1$ for all non-cyclic surgery slopes $s \in \mathbb{Q}$ of $k$. 
\end{claim}

\medskip

Following Remarks~\ref{notP(K)_torus} and \ref{notP(K)_hyp}, these elements 
$x$ are not conjugate into the subgroup 
$P(k)=\pi_1(\partial V)$. 
\textit{In the following we use $x$ to denote such an element.} 


Let us take a slope $s\in \mathbb{Q}$.  
Since $K$ has no cyclic surgery slope, 
$s$ is not a cyclic surgery slope. 

\medskip

\noindent
\textbf{Case 1.} $\partial V(K; s)$ is incompressible.
\smallskip

Then $\pi_1(K(s))=G(k)*_{\pi_1(\partial V(K; s))}\pi_1(V(K;s))$ is an amalgamated free product. Hence $G(k)$ injects into $\pi_1(K(s))$.
Since $x$ is non-trivial in $G(k) = \pi_1(E(k))$, 
$p_s(x)=x$ is also non-trivial in $\pi_1(K(s))$



\medskip

\noindent
\textbf{Case 2.}  $V(K; s) = S^1 \times D^2$.
\smallskip

Then $K(s) = E(k) \cup V(K; s) = k(s/w^2)$, 
where $w$ is the winding number of $K$ in $V$ \cite{Go_satellite}. 
Since $\pi_1(K(s))$ is not cyclic, neither is $\pi_1(k(s/w^2))$. 
Hence $s/w^2$ is not a cyclic surgery slope of $k$, 
and $p^{K}_{s}(x)=p^{k}_{s/w^2}(x) \ne 1 \in \pi_1(k(s/w^2)) = \pi_1(K(s))$. 

\medskip 

\noindent
\textbf{Case 3.} $\partial V(K;s)$ is compressible and $V(K; s) = (S^1 \times D^2)\, \#\, W$ for some closed $3$--manifold $W \ne S^{3}$.

\smallskip

Then $K(s) = E(k) \cup V(K; s) = k(s/w^2)\, \#\, W$ \cite{Go_satellite}.  
Since $\partial V(K;s)$ is compressible, 
\cite[Corollary~2.5]{GabaiII} and \cite{Sch_JDG} show that the $w \ne 0$. 
Hence following \cite{GLu2}, 
$k(s/w^2) \ne S^3$ and hence $K(s) = k(s/w^2)\, \#\, W$ is reducible. 
Then \cite[Corollary~1.4]{BZ_finite_JAMS} shows that $K$ is a $(p, q)$--cable of a torus knot $T_{a, b}$ 
for some integers $p, q, a, b$, 
and the surgery slope $s$ is the cabling slope $pq$. 
Note that $w = q \ge 2$ and 
the companion knot $k$ is $T_{a, b}$. 

Assume first that $\pi_1(k(s/w^2))$ is not cyclic. 
Then $p^k_{s/w^2}(x) \ne 1$ in $\pi_1(k(s/w^2))$, 
which injects into $\pi_1(K(s)) = \pi_1(k(s/w^2)) \ast \pi_1(W)$. 
Hence $p_{s/w^2}(x) \ne 1$ in $\pi_1(K(s))$. 

Next assume that $\displaystyle\pi_1(k(s/w^2)) = \pi_1(T_{a, b}(pq /q^2)) = \pi_1(T_{a, b}(p/q))$ is cyclic. 
This then implies that the distance between two slopes $p/q$ and $ab$ should be one, 
i.e. $|abq - p| = 1$. 
So $K$ is a $(abq \pm 1, q)$--cable of $T_{a, b}$. 
This contradicts the initial assumption.  

\medskip

Finally we show that there are infinitely many, mutually non-conjugate elements $x \in [G(K), G(K)]$ with 
$p_s(x) \ne 1$ for all slopes $s \in \mathbb{Q}$. 
In our proof the elements $x \in G(k)$ are mutually non-conjugate in $G(k)$.  
So it is sufficient to see that such elements are still non-conjugate in $G(K)$. 
Actually this immediately follows from a fact that for the amalgamated free product $G=A*_C B$ and elements $a, a' \in A - C$, 
$a$ and $a'$ are conjugate in $G$ if and only if they are conjugate in $A$ \cite[Theorem 4.6]{MKS}. 
In our setting $G = G(K), A = G(k), B = \pi_1(V(K; s))$ and $C = P(k) = \pi_1(\partial E(k))$. 
\end{proof}

\medskip

\begin{remark}
\label{cable}
If $K$ is a $(abq \pm 1, q)$-cable of $T_{a, b}$, 
then the reducing surgery on $K$ induces a cyclic surgery on the companion knot 
$k = T_{a, b}$. 
Therefore 
for any $x \in [G(T_{a, b}), G(T_{a, b})]$, 
$p_{(abq\pm 1)q}(x) = 1$ for the reducing surgery slope $(abq\pm 1)q$. 
\end{remark}

In the remaining of this section we focus on a $(abq \pm 1, q)$--cable of $k = T_{a, b}$.  
The situation described in Remark~\ref{cable} forces us to pay further attention to take desired elements in $G(K)$. 


\begin{proposition}
\label{cable_torus}
Let $K$ be a $(abq \pm 1, q)$--cable of $k = T_{a, b}$ \($|q| \ge 2$\). 
Then there are infinitely many, mutually non-conjugate elements $x$ in $[G(K), G(K)]$ 
such that $p_s(x) \ne 1$ for all non-cyclic surgery slopes. 
\end{proposition}

\begin{proof}
Decompose $E(K)$ as $E(k) \cup (V- \mathrm{int}N(k))$, 
where $E(k) = E(T_{a, b})$, and $V- \mathrm{int}N(k)$ is a $(p, q)$--cable space  ($p = abq \pm 1$), 
and $\partial E(k) = \partial V$. 


\medskip


Let $\tau_a, \tau_b$ be exceptional fibers of $E(k) = E(T_{a, b})$ of indices $a, |b|$, respectively; 
we use the same symbol $\tau_a, \tau_b$ to denote the elements in $G(k)$ represented by these exceptional fibers. 
(Note that $\tau_a, \tau_b$ generate $G(k)$.)
Let us take an element $g = [\tau_a, \tau_b] \in [G(k), G(k)] \subset [G(K), G(K)]$. 
Then following the proof of Proposition~\ref{persistent_[G,G]_torus}, 
we have:


\begin{claim}
\label{companion_cable}
$p_s^k(g) \ne 1$ for all non-cyclic surgery slope of $k$. 
\end{claim}

Let us put 
$x =g  \lambda_K^{\ell} = [\tau_a, \tau_b] \lambda_K^{\ell} \in [G(K), G(K)] \subset G(K)$
for an integer $\ell$. 

In what follows we show $p_s(x) \neq 1$ for all non-cyclic surgery slopes $s = m/n$. 

Following \cite{Go_satellite} we have:  

\[
V(K;s) = \begin{cases}
    \textrm{boundary-irreducible Seifert fiber space}, & |npq-m| > 1, \\
    S^1 \times D^2, & |npq-m| = 1, \\
    S^1 \times D^2 \# L(q, p), & |npq-m| = 0.
  \end{cases}
\]

Accordingly we have:
\[
K(s)  = \begin{cases}
    E(k) \cup V(K; s), \textrm{which is a graph manifold}, & |npq-m| > 1, \\
    k(s \slash q^2), & |npq-m| = 1, \\
    k(p \slash q) \# L(q, p), & |npq-m| = 0.
  \end{cases}
\]

\medskip

\noindent
\textbf{Case 1}.  $|npq-m|>1$. 

Assume first that $ s = 0$. 
Then 
\begin{align*}
p_0(x) 
&= p_0([\tau_a, \tau_b]  \lambda_K^{\ell}) 
= p_0([\tau_a, \tau_b]) p_0( \lambda_K^{\ell}) \\
&= p_0([\tau_a, \tau_b]) p_0( \lambda_K)^{\ell}
= p_0([\tau_a, \tau_b]) \ne 1, 
\end{align*}
because $p_0([\tau_a, \tau_b])\in G(k) \subset G(k) \ast_{\pi_1(T)} \pi_1(V(K; 0)) = \pi_1(K(0))$, 
where $T = \partial E(k) = \partial V$. 

For any $0 \ne s \in \mathbb{Q}$, 
we have 
\[
 p_s(x) =  p_s([\tau_a, \tau_b]) p_s( \lambda_K) ^{\ell}
\in G(k) \ast_{\pi_1(T)} \pi_1(V(K; 0)).
\]
Note that $p_s([\tau_a, \tau_b]) \ne 1 \in \pi_1(G(k))$, and 
$p_s(\lambda_K) \ne 1$, because $s \ne 0$ and $s$ is not a finite surgery slope (Lemma~\ref{p(lambda^q)=1}). 
Now suppose for a contradiction that 
$p_s([\tau_a, \tau_b]) p_s( \lambda_K) ^{\ell} = 1$. 
Then $p_s([\tau_a, \tau_b]) = p_s( \lambda_K) ^{-\ell}$. 
This means that  $p_s([\tau_a, \tau_b]) = p_s( \lambda_K) ^{-\ell} \in \pi_1(T)$. 
However, Remark~\ref{notP(K)_torus} shows that $p_s([\tau_a, \tau_b])$ is non-peripheral in $G(k)$, 
a contradiction. 
 

\medskip

\noindent
\textbf{Case 2}. $|npq-m|=1$. 

In this case, $s \ne 0$ and $\frac{s}{q^2} = \frac{m}{nq^2} \ne ab$. 
(If  $\frac{s}{q^2} = ab$, then $|npq - m| = |nq(p-abq)| \ne 1$, a contradiction.)
Hence, 
$K(s) = E(k) \cup V(K;s) = k(s/q^2)$ is 
a Seifert fiber space over the disk with three exceptional fibers if $s$ is not a cyclic surgery.  

Assume that for some integer $\ell_0$ 
 \[
 p_s(x) 
= p_s([\tau_a, \tau_b]  \lambda_K^{\ell_0}) 
= p_s([\tau_a, \tau_b]) p_s( \lambda_K^{\ell_0}) 
=  p_s([\tau_a, \tau_b]) p_s( \lambda_K) ^{\ell_0}
= 1.
\]


First we suppose that $|\pi_1(K(s))| = |\pi_1(k(s/q^2))| =  \infty$. 

Then 
$p_s([\tau_a, \tau_b]) = p_s( \lambda_K) ^{-\ell_0}$, 
which has infinite order.  

Let us assume  $p_s([\tau_a, \tau_b] \lambda_K^{\ell}) =  p_s([\tau_a, \tau_b]) p_s( \lambda_K) ^{\ell} = 1$ for $\ell$. 
Then $p_s([\tau_a, \tau_b]) = p_s( \lambda_K)^{-\ell}$. 
Thus $p_s([\tau_a, \tau_b])^{\ell} = p_s([\tau_a, \tau_b])^{\ell_0}$. 
Since $p_s([\tau_a, \tau_b])$ has infinite order, $\ell = \ell_0$. 
Therefore for all integers $\ell > \ell_0$, 
$p_s([\tau_a, \tau_b]) p_s( \lambda_K) ^{\ell} \ne 1$. 


\medskip

Suppose that $|\pi_1(K(s))| = |\pi_1(k(s/q^2))| <  \infty$. 
By the assumption, $\pi_1(K(s)) = \pi_1(k(s/q^2))$ is non-cyclic. 
Hence Proposition~\ref{persistent_[G,G]_torus} shows that 
$p_s([\tau_a, \tau_b]) \ne 1$ in $\pi_1(k(s/q^2))$. 

Let us take $\ell = \ell_0 + 1$. 
Then 
if $p_s([\tau_a, \tau_b]) = p_s( \lambda_K) ^{-\ell}$, 
then we have
\[
p_s([\tau_a, \tau_b])^{\ell- \ell_0} = p_s([\tau_a, \tau_b]) = 1 \in \pi_1(k(s/q^2)). 
\]
However this implies that $s$ is a cyclic surgery (Theorem~\ref{persistent_[G,G]_torus}),  a contradiction. 
Thus 
$p_s([\tau_a, \tau_b])\lambda^{\ell}) \ne 1$.  

It follows from the above argument, 
we see that for any integer $\ell$, 
at least we have $p_s([\tau_a, \tau_b] \lambda^{\ell} )\ne 1$ or  $p_s([\tau_a, \tau_b] \lambda^{\ell+1}) \ne 1$  for all non-cyclic surgeries. 



\medskip

\noindent
\textbf{Case 3}. $|npq-m|=0$ i.e. $s = pq$. 

In this case 
\[
k(pq) = k(p/q) \# L(q, p) = k((abq \pm 1)/q) \# L(q, p) 
= L(p, qb^2) \# L(q, p).
\]
Since $p_s([\tau_a, \tau_b]) = 1 \in \pi_1(L(p, qb^2))$, 
we have $p_s([\tau_a, \tau_b] \lambda^{\ell}) = p_s([\tau_a, \tau_b]) p_s(\lambda_K)^{\ell} = p_s(\lambda_K)^{\ell}$, 
which is non-trivial, because $s \ne 0$ and $s$ is not a finite surgery slope (Lemma~\ref{p(lambda^q)=1}). 


\medskip 

Finally we show that there are infinitely many, mutually non-conjugate elements 
$[\tau_a, \tau_b]\lambda_K^{\ell}  \in [G(K), G(K)]$.  

Let $\phi \colon G(K) \to \mathbb{R}$ be a homogeneous quasimorphism of defect $D(\phi)$. 
Since $\mathrm{scl}_{G(k)}(\lambda_K) = g(K) -1/2 > 0$, 
we may take $\phi$ so that $\phi(\lambda_K) > 0$ as in the proof of Lemma~\ref{infinite_conjugacy}.  
Then 
\[
D(\phi) \ge |\phi([\tau_a, \tau_b] \lambda_K^{\ell}) - \phi ([\tau_a, \tau_b]) - \phi(\lambda_K^{\ell})|
\]
and thus, 
\[
\phi([\tau_a, \tau_b] \lambda_K^{\ell}) 
\ge \phi([\tau_a, \tau_b]) +
 \phi(\lambda_K^{\ell})  - D(\phi)
= \phi([\tau_a, \tau_b]) +
 \ell \phi(\lambda_K)  - D(\phi).
\] 
Hence $\displaystyle \lim_{\ell \to \infty} \phi([\tau_a, \tau_b] \lambda_K^{\ell})  \to \infty$. 
Since a homogeneous quasimorphism $\phi$ is conjugation invariant, 
this shows that $\{[\tau_a, \tau_b] \lambda_K^{\ell} \}_{\ell \in \mathbb{Z}}$ has infinitely many, 
mutually non-conjugate elements.
\end{proof}

\medskip

\section{Separation theorem}
\label{finite_separation}

The goal of this section is to establish Theorem~\ref{separation}. 

\begin{thm_separation}[(Separation theorem)]
Let $K$ be a non-trivial knot which is not a torus knot.  
Let $\mathcal{R} = \{ r_1, \dots, r_n \}$ and $\mathcal{S} = \{ s_1, \dots, s_m \}$ be any finite sets of $\mathbb{Q}$ such that 
$\mathcal{R} \cap \mathcal{S} = \emptyset$. 
Assume that $\mathcal{S}$ does not contain a Seifert surgery slope. 
Then there exists an element $g \in [G(K), G(K)] \subset G(K)$ 
such that $\mathcal{R} \subset \mathcal{S}_K(g) \subset \mathbb{Q} - \mathcal{S}$, 
namely $\mathcal{R} \subset \mathcal{S}_K(g)$ and $\mathcal{S} \subset  \overline{\mathcal{S}_K(g)}$. 
\end{thm_separation}



\begin{proof}
Note that if $\mathcal{S} = \emptyset$, then the result follows from \cite[Theorem~1.4]{IMT_Magnus}.  
So in the following we assume that $\mathcal{S} \ne \emptyset$. 

\begin{claim}
\label{at_most_two_torsion_surgeries}
$\mathcal{S}$ contains at most two torsion surgery slopes.
\end{claim}

\begin{proof}
Assume for a contradiction that $\mathcal{S}$ contains more than two torsion surgery slopes. 
Then, since $\mathcal{S}$ has no Seifert surgery slopes,  
it contains no finite surgery slopes. 
Thus these torsion surgery slopes should be reducing surgery slopes. 
However, it is known that $K$ has at most two such surgeries \cite{GLreducible}, contradicting the assumption. 
\end{proof}

In the following, without loss of generality, by re-indexing slopes in $\mathcal{S}$, 
we may assume the following. 

\medskip

$\bullet$ If $\mathcal{S}$ has a single torsion surgery slope, then $s_1$ is such a slope. 
If $\mathcal{S}$ contains exactly two torsion surgery slopes, 
then $s_1$ and $s_2$ are such slopes. 

In particular, we assume that $s_i$ $(i \ge 3)$ are not torsion surgery slopes. 

\medskip

We prove Theorem~\ref{separation} by induction on the number of slopes in $\mathcal{R} =\{ r_1, \dots, r_n \}$. 
In the following, to make precise we will use $\mathcal{R}_n$ instead of $\mathcal{R}$.  

\medskip

\noindent
\textbf{Step 1.}\ 
The Geometrization theorem says that a closed $3$--manifold with cyclic fundamental group is a lens space, 
which is also a Seifert fiber space. 
So by the assumption $\mathcal{S}$ does not contain a cyclic surgery slope. 

If $n = 0$, 
i.e. $\mathcal{R}_0 = \emptyset$, 
then following Theorem~\ref{persistent_[G,G]_noncyclic} we take an element $g_0 \in [G(K),G(K)]$ so that 
$p_r(g_0) \ne 1$ whenever $r$ is not a cyclic surgery slope. 
Then obviously 
\[
\emptyset = \mathcal{R}_0 \subset \mathcal{S}_K(g_0) \subset \mathbb{Q} - \mathcal{S}. 
\]

\medskip

\noindent
\textbf{Step 2.}\ 
Assume that the result holds when $n = k \ge 0$, i.e. 
there exists an element $g_k \in [G(K),G(K)]$ such that 
\[
\mathcal{R}_k  = \{r_1,\ldots,r_{k}\} \subset \mathcal{S}_K(g_k) \subset \mathbb{Q} - \mathcal{S},
\]
\[
\mathrm{namely,}\ p_{r_i}(g_k) = 1\ (1 \le i \le k)\quad \textrm{and}\quad   p_{s_i}(g_k) \ne 1\ (1 \le i \le m). 
\]

\medskip


Then we prove that there exists $g_{k+1} \in [G(K), G(K)]$ such that 
\[
\mathcal{R}_{k+1}  = \{r_1,\ldots,r_{k}, r_{k+1} \} \subset \mathcal{S}_K(g_{k+1}) \subset \mathbb{Q} - \mathcal{S}.
\]

\medskip

A proof of the second step requires modifications of the non-trivial element $g_k$ several times to get $g_{k+1}$. 
Now we observe the following useful fact.  

\begin{claim}
\label{r_k}
For any $a \in G(K)$, 
let us put 
\[
g_a= [a g_k a^{-1},r_{k+1}].
\] 
Then $\mathcal{S}_K(g_a) \supset \{ r_1, \dots, r_k, r_{k+1} \}$. 
\end{claim}

\begin{proof}
\[
p_{r_i}(g_a) = [p_{r_i}(a)p_{r_i}(g_k)p_{r_i}(a)^{-1},\ p_{r_i}(r_{k+1})] = [1,\ p_{r_i}(r_{k+1})] = 1
\]
for $i = 1, \dots, k$,
and 
\begin{align*}
p_{r_{k+1}}(g_a) &=  [p_{r_{k+1}}(a)p_{r_{k+1}}(g_k)p_{r_{k+1}}(a)^{-1},\ p_{r_{k+1}}(r_{k+1})] \\
&= [p_{r_{k+1}}(a)p_{r_{k+1}}(g_k)p_{r_{k+1}}(a)^{-1},\ 1] = 1.
\end{align*}
Thus $\mathcal{S}_K(g_a) \supset \{ r_1, \dots, r_k, r_{k+1} \}$. 
\end{proof}

\medskip

Let us take a slope $s \in \mathcal{S}$. 
Then since $r_{k+1} \not\in \mathcal{S}$, 
$s \ne r_{k+1}$. 
Furthermore, 
by the assumption $\mathcal{S}$ does not contain a finite surgery slope, 
so $s$ is not a finite surgery slope. 
Therefore $p_s(r_{k+1}) \ne 1$; see Proposition~\ref{slope}.  
Besides, 
since $\mathcal{S}_K(g_k) \subset \mathbb{Q} - \mathcal{S}$, 
$p_s(g_k) \ne 1$ for $s \in \mathcal{S}$. 


Note that 
\[
p_s(g_a)=1\quad \textrm{if and only if}\quad [p_s(a g_k a^{-1}),\ p_s(r_{k+1})] = 1.
\] 


The claim below is crucial in the following procedure, 
and we will apply this several times. 

\begin{claim}
\label{alpha_non-commute}
For any slope $s \in \mathcal{S}$, 
there exists $\alpha \in \pi_1(K(s))$ such that $[\alpha p_s(g_k)\alpha^{-1},\ p_s(r_{k+1})] \ne 1$. 
\end{claim}


\begin{proof}
We begin by analyzing the centralizer of $p_s(r_{k+1})$ in $\pi_1(K(s))$. 
It follows from the structure theorem of centralizers \cite[Theorems~2.5.1 and 2.5.2]{AFW} that 
we have the following: 
\begin{enumerate}
\renewcommand{\labelenumi}{(\roman{enumi})}
\item
$Z(p_s(r_{k+1})) = \langle p_s(r_{k+1}) \rangle \cong \mathbb{Z}$, or
\item
there is an essential torus $T$ in $K(s)$ and $h \in \pi_1(K(s))$ such that $p_s(r_{k+1}) \in h \pi_1(T) h^{-1}$ and $Z(p_s(r_{k+1})) = h \pi_1(T) h^{-1}$, or 
\item
there is a Seifert piece $M$ of the torus decomposition \cite{JS,Jo,Hat} of $K(s)$ such that 
$p_s(r_{k+1}) \in h \pi_1(M) h^{-1}$ and $Z(p_s(r_{k+1})) = h Z_{\pi_1(M)}(h^{-1} p_s(r_{k+1}) h) h^{-1}$. 
Since we assume that $s \in \mathcal{S}$ is not a Seifert surgery slope, 
$K(s)$ is not a Seifert fiber space. 
Hence $M \ne K(s)$. 
\end{enumerate} 

When (ii) or (iii) happens, 
to simplify the description of the centralizer, 
we apply the inner automorphism 
\[
\varphi \colon \pi_1(K(s)) \to \pi_1(K(s)),\quad
g \mapsto h^{-1} g h
\] 
so that 
we assume $\varphi(p_s(r_{k+1})) = h^{-1} p_s(r_{k+1}) h  \in \pi_1(T)$ or $\varphi(p_s(r_{k+1})) = h^{-1} p_s(r_{k+1}) h \in \pi_1(M)$; 
simultaneously $\varphi(p_s(g_k)) = h^{-1} p_s(g_k) h$. 
Note that 
$[\alpha p_s(g_k) \alpha^{-1},\ p_s(r_{k+1})] = 1$ if and only if 
\[ 
[\alpha' \varphi(p_s(g_k)) \alpha'^{-1},\  \varphi(p_s(r_{k+1}))] 
= 
\varphi([\alpha p_s(g_k) \alpha^{-1},\ p_s(r_{k+1})]) 
= \varphi(1) = 1, 
\] 
where $\alpha' = \varphi(\alpha)$. 

In the following, for simplicity, we use the same symbol $p_s(r_{k+1})$ and $p_s(g_k)$ to denote $\varphi(p_s(r_{k+1}))$ and $\varphi(p_s(g_k))$. 

Then in the case of (iii), 
$Z(p_s(r_{k+1})) \subset \pi_1(M) \subset \pi_1(K(s))$.  
In particular, we have: 

\begin{itemize}
\item $Z(p_s(r_{k+1})) =  \langle p_s(r_{k+1}) \rangle \cong \mathbb{Z}$,
\item $Z(p_s(r_{k+1})) \cong \mathbb{Z} \oplus \mathbb{Z}$, or
\item $Z(p_s(r_{k+1}))  = \pi_1(M)$ when $p_s(r_{k+1})$ belongs to a subgroup generated by a regular fiber in $M$. 
\item $Z(p_s(r_{k+1}))$ is the fundamental group of the Klein bottle, the Klein bottle group for short. 
\end{itemize}


\medskip

Suppose first that $p_s(g_k) \not\in Z(p_s(r_{k+1}))$. 
Then by definition, 
$[p_s(g_k), p_s(r_{k+1})] \ne 1$. 
So we may take $\alpha = 1 \in G(K)$. 

Assume that $p_s(g_k) \in Z(p_s(r_{k+1}))$. 
We divide our discussion into three cases according to (i), (ii) or (iii). 

\medskip

\noindent
\textbf{Case (i)}.\ 
$Z(p_s(r_{k+1})) = \langle p_s(r_{k+1}) \rangle \cong \mathbb{Z}$. 

Suppose for a contradiction that 
$\alpha p_s(g_k) \alpha^{-1} \in Z(p_s(r_{k+1})) \cong \mathbb{Z}$ for all $\alpha \in \pi_1(K(s))$. 
Then it implies that 
$\langle \! \langle p_s(g_k) \rangle\!\rangle \subset Z(p_s(r_{k+1})) \cong \mathbb{Z}$. 
Hence $\pi_1(K(s))$ contains an infinite cyclic normal subgroup $\langle \! \langle p_s(g_k) \rangle\!\rangle$. 
It follows from \cite{CJ,Gabai_Seifert} that $K(s)$ is a Seifert fiber space, 
contradicting the assumption of $\mathcal{S}$. 
Thus we find $\alpha \in \pi_1(K(s))$ such that $\alpha p_s(g_k) \alpha^{-1} \not\in Z(p_s(r_{k+1}))$, 
i.e. $[\alpha p_s(g_k) \alpha^{-1},\ p_s(r_{k+1})] \ne 1$. 

\medskip

\noindent
\textbf{Case (ii)}.\ $p_s(r_{k+1}) \in  \pi_1(T)$ and $Z(p_s(r_{k+1})) = \pi_1(T)$. 

Assume that $T$ is separating in $K(s)$. 
Split $K(s)$ along the essential torus $T$ to obtain a decomposition $K(s) = X \cup_T Y$. 
Then $\pi_1(K(s))$ is an amalgamated free product with amalgamated subgroup $\pi_1(T)$, 
$\pi_1(K(s)) = \pi_1(X) \ast_{\pi_1(T)} \pi_1(Y)$. 
Let us take a non-trivial element $\alpha \in \pi_1(K(s))$ so that its representative is a non-trivial loop entirely contained in $X$ and not homotoped into $T \subset \partial X$. 
Then $\alpha \in \pi_1(X) - \pi_1(T)$ and $\alpha p_s(g_k) \alpha^{-1} \not\in \pi_1(T) = Z(p_s(r_{k+1}))$; see \cite[Corollary~4.4.1]{MKS}. 
This means that $[\alpha p_s(g_k) \alpha^{-1},\ p_s(r_{k+1})] \ne 1$. 


Assume next that the essential torus $T$ is non-separating in $K(s)$. 
Then $s = 0$ for homological reason and the genus of $K$ is one \cite[Corollary~8.3]{GabaiIII}.  

Suppose for a contradiction that 
$[\alpha p_0(g_k) \alpha^{-1}, p_0(r_{k+1})] = 1$ for all $\alpha \in \pi_1(K(0))$. 
This then implies 
\[
[\alpha p_0(g_k)^{-1} \alpha^{-1},\ p_0(r_{k+1})] = [(\alpha p_0(g_k) \alpha^{-1})^{-1},\ p_0(r_{k+1})] =1, \textrm{and}
\]
\[
\left[\prod_{i=1}^d \alpha_i p_0(g_k)^{\varepsilon_i} \alpha_i^{-1},\ p_0(r_{k+1})\right] = 1\quad (\varepsilon_i = \pm 1).
\] 
Hence,  the normal closure $\langle \! \langle p_0(g_k) \rangle\!\rangle$ is contained in 
$Z(p_0(r_{k+1})) = \pi_1(T) = \mathbb{Z} \oplus \mathbb{Z}$. 

Then $\langle \! \langle p_0(g_k) \rangle\!\rangle \subset \pi_1(K(0))$ is finitely generated. 
By the assumption $K(0)$ is not Seifert fibered, 
so $\langle \! \langle p_0(g_k) \rangle\!\rangle$ is not infinite cyclic, and hence $\langle \! \langle p_0(g_k) \rangle\!\rangle \cong \mathbb{Z} \oplus \mathbb{Z}$. 
If $\langle \! \langle p_0(g_k) \rangle\!\rangle$ has finite index in $\pi_1(K(0))$, 
then $K(0)$ admits a finite cover whose fundamental group is isomorphic to $\mathbb{Z} \oplus \mathbb{Z}$. 
However, there is no closed $3$--manifold with such fundamental group.  
So $\langle \! \langle p_0(g_k) \rangle\!\rangle$ is infinite index in $\pi_1(K(0))$. 
Then the classification of finitely generated, normal subgroups of infinite index of $3$--manifold groups \cite{HJ}, \cite[p.118 (L9)]{AFW} shows that 
$K(0)$ is Seifert fibered, $K(0)$ fibers over $S^1$, or the union of two twisted $I$--bundle over a compact connected (non-orientable) surface $\Sigma$. 
The first possibility is eliminated by the assumption. 
If we have the last situation, 
then let $N_{\Sigma}$ be the twisted $I$--bundle over $\Sigma$. 
Write $\widetilde{\Sigma} = \partial N_{\Sigma}$, which is the $\partial I$--subbundle of $N_{\Sigma}$. 
Then $\pi_1(K(0)) = \pi_1(N_{\Sigma}) \ast_{\pi_1(\widetilde{\Sigma})} \pi_1(N_{\Sigma})$. 
Since $\pi_1(N_{\Sigma})/\pi_1(\widetilde{\Sigma}) \cong \mathbb{Z}_2$, we have 
\[
1 \to \pi_1(\widetilde{\Sigma}) \to \pi_1(K(0)) \to \mathbb{Z}_2 \ast \mathbb{Z}_2 \to 1.
\]
This means that $H_1(K(0)) = \mathbb{Z}$ has an epimorphism to $\mathbb{Z}_2 \oplus \mathbb{Z}_2$, a contradiction. 
Hence, 
$K(0)$ is a surface bundle over $S^1$. 
Then apply \cite[Corollary~8.19]{GabaiIII} to see that $K$ is a fibered knot. 
Hence $K$ is a genus one fibered knot, i.e. it is either a trefoil knot or the figure-eight knot. 
In the former case, $M = E(K)$ is Seifert fibered, contradicting the assumption. 
Thus $K$ is the figure-eight knot and $K(0)$ is a torus bundle over $S^1$ with fiber surface $T$. 
(Note that every incompressible torus is isotopic to $T$.)

Since $K(0)$ fibers over $S^1$ with fiber $T$, 
$Z(p_0(g_{k+1})) = \pi_1(T)$ is a normal subgroup of $\pi_1(K(0))$ and hence 
we cannot take $\alpha \in \pi_1(K(0))$ so that $\alpha p_0(g_k) \alpha^{-1} \not\in \pi_1(T) = Z(p_0(r_{k+1}))$. 
So we need a different approach. 
First we note that $\pi_1(K(0))/ \pi_1(T) = \pi_1(S^1) = \mathbb{Z}$, 
hence $[\pi_1(K(0)), \pi_1(K(0))] \subset \pi_1(T)$. 
On the other hand, 
\[
\pi_1(K(0))/ [\pi_1(K(0)), \pi_1(K(0))] = H_1(K(0)) = \mathbb{Z}.
\] 
Then the canonical epimorphism 
\[
\mathbb{Z} = \pi_1(K(0))/[\pi_1(K(0)), \pi_1(K(0))]  \to \pi_1(K(0))/\pi_1(T) = \mathbb{Z}
\]
is necessarily injective. 
This implies that 
$[\pi_1(K(0)), \pi_1(K(0))] = \pi_1(T)$. 

Now let us assume $r_{k+1}$ is represented by a slope element $\mu^m \lambda^n$ for some relatively prime integers $m$ and $n$. 
Since $\lambda =1$ in $\pi_1(K(0))$, 
$p_0(r_{k+1}) = \mu^m$. 
Moreover, 
since $p_0(r_{k+1}) \in Z(p_0(r_{k+1})) = \pi_1(T) = [\pi_1(K(0)),\pi_1(K(0))]$, 
$[p_0(r_{k+1})] = [\mu^m] = m [\mu] = 0 \in H_1(K(0)) = \langle [\mu] \rangle =\mathbb{Z}$. 
Hence $m = 0$, and $r_{k+1} = 0$.
This means that $r_{k+1} = s = 0$, contradicting the assumption $\mathcal{R}\cap \mathcal{S}=\varnothing$.


\medskip

\noindent
\textbf{Case (iii)}.\ $p_s(r_{k+1}) \in  \pi_1(M)$ and $Z(p_s(r_{k+1})) = Z_{\pi_1(M)}(p_s(r_{k+1})) \subset \pi_1(M)$.

First we recall that $Z_{\pi_1(M)}(p_s(r_{k+1}))$ is one of $\mathbb{Z}, \mathbb{Z} \oplus \mathbb{Z}$, the Klein bottle group, or $\pi_1(M)$. 

Let us consider the torus decomposition of $K(s)$. 
By the assumption we have the Seifert fibered piece $M$. 
Let $X$ be a union of some decomposing pieces including $M$ (where $X$ may be $M$) 
such that both $X$ and $Y = \overline{K(s) - X}$ are connected and $X \cap Y$ is an essential torus $T$. 
Then we have an expression 
$\pi_1(K(s)) = \pi_1(X) \ast_{\pi_1(T)} \pi_1(Y)$ as an amalgamated free product. 
Let us take a non-trivial element $\alpha \in \pi_1(K(s))$ so that its representative is a non-trivial loop entirely contained in $Y$ and not homotoped into $T \subset \partial Y$. 
Then $\alpha \in \pi_1(Y) - \pi_1(T)$ and $\alpha p_s(g_k) \alpha^{-1} \not\in \pi_1(X)$; see \cite[Corollary~4.4.1]{MKS}. 
Since $Z(p_s(r_{k+1})) = Z_{\pi_1(M)}(p_s(r_{k+1})) \subset \pi_1(M) \subset \pi_1(X)$, 
the condition $\alpha p_s(g_k) \alpha^{-1} \not\in \pi_1(X)$ implies 
$\alpha p_s(g_k) \alpha^{-1} \not\in Z(p_s(r_{k+1}))$. 
This means that $[\alpha p_s(g_k) \alpha^{-1},\ p_s(r_{k+1})] \ne 1$. 

We finish the proof of Claim~\ref{alpha_non-commute}. 
\end{proof}

\bigskip




Let us return to a proof of the second step. 

${\bf (1)}$\ Following Claim~\ref{alpha_non-commute} 
there exists $\alpha_1 = a(s_1) \in \pi_1(K(s_1))$ such that 
\[
[\alpha_1 p_{s_1}(g_k) \alpha_1^{-1},\ p_{s_1}(r_{k+1})] \ne 1.
\] 
Since $p_{s_1} \colon G(K) \to \pi_1(K(s_1))$ is surjective, 
we have $a_1 \in G(K)$ such that  $p_{s_1}(a_1) = \alpha_1$. 
Then we have 
\[
p_{s_1}( [a_1 g_k a_1^{-1}, r_{k+1}] ) \ne 1.
\]

Let us write 
\[
g_{k, a_1} = [a_1 g_k a_1^{-1}, r_{k+1}] \in G(K),
\]
which is non-trivial and satisfies that 
$p_{s_1}(g_{k, a_1}) \ne 1$, i.e. 
$s_1 \not \in \mathcal{S}_K(g_{k, a_1})$. 
By Claim~\ref{r_k} $p_{r}(g_{k, a_1}) = 1$ for $r \in \{ r_1, \dots, r_k, r_{k+1}\}$. 
Hence, put
\[
h_1 = g_{k, a_1}, \quad \textrm{for which}\quad  
\mathcal{R}_{k+1} = \{r_1,\ldots,r_{k}, r_{k+1} \} \subset \mathcal{S}_K(h_1) \subset \mathbb{Q} - \{ s_1 \}.
\]

\medskip

In the following we will apply step-by-step modification of $h_1$ to obtain $g_{k+1}$. 

\medskip

${\bf (2)}$\  We will construct $h_2$ satisfying 
\[
\mathcal{R}_{k+1} = \{r_1,\ldots,r_{k}, r_{k+1} \} \subset \mathcal{S}_K(h_2) \subset \mathbb{Q} - \{ s_1, s_2 \}.
\]

\medskip

${\bf (2.1)}$\  If $\mathcal{S}_K(h_1) \subset \mathbb{Q} - \{s_1, s_2\}$, 
then put 
\[
h_2 = h_1,\  
\textrm{for which}\ 
\mathcal{R}_{k+1} = \{r_1,\ldots,r_{k}, r_{k+1} \} \subset \mathcal{S}_K(h_{2}) \subset \mathbb{Q} - \{ s_1, s_2 \}.
\]

\smallskip

${\bf (2.2)}$\  If $s_2 \in \mathcal{S}_K(h_1)$, i.e. $p_{s_2}(h_1) = 1$, 
then we will modify $h_1$ further to obtain $h_2$ with the above property in ${\bf (2.1)}$. 

\medskip


Claim~\ref{alpha_non-commute}, 
together with the surjectivity of $p_{s_2} \colon G(K) \to \pi_1(K(s_2))$,  
enables us to pick $a_2 = a(s_2) \in G(K)$ so that $p_{s_2}(a_2) = \alpha_2 \in \pi_1(K(s_2))$ satisfies 
\[
[\alpha_2 p_{s_2}(g_k) \alpha_2,\ p_{s_2}(r_{k+1})] \ne 1,\ \textrm{equivalently}\ p_{s_2}([a_2 g_k a_2^{-1}, r_{k+1}]) \ne 1.
\]

Then put 
\[
g_{k, a_2} = [a_2 g_k a_2^{-1}, r_{k+1}] \in G(K),
\] 
which is non-trivial and satisfies $p_{s_2}(g_{k, a_2}) \ne 1$. 
Since $p_r(g_k) = 1$ for $r \in \mathcal{R}_k$ and obviously $p_{r_{k+1}}(r_{k+1}) = 1$, 
\[
p_r(g_{k, a_2}) = [p_r(a_2)p_r(g_k)p_r(a_2)^{-1}, p_r(r_{k+1})] = 1\  \textrm{for}\  r \in \mathcal{R}_{k+1}.
\] 

Recall also that $p_{s_1}(h_1) \ne 1$ and  $p_{s_2}(h_1) = 1$. 

\begin{claim}
\label{q_1}
There is an integer $q_1$ such that $p_{s_1}(h_1^{q_1} g_{k, a_2}) \ne 1 \in \pi_1(K(s_1))$. 
\end{claim}


\begin{proof}
If $p_{s_1}(h_1 g_{k, a_2}) \ne 1$, then we choose $q_1 = 1$. 
Assume that 
\[
p_{s_1}(h_1 g_{k, a_2}) = p_{s_1}(h_1) p_{s_1}(g_{k, a_2}) = 1,\  
\textrm{i.e.}\  p_{s_1}(g_{k, a_2}) = p_{s_1}(h_1)^{-1}.
\] 

If $p_{s_1}(h_1^{q_1} g_{k, a_2}) = p_{s_1}(h_1)^{q_1} p_{s_1}(g_{k, a_2}) = 1$, 
then 
$p_{s_1}(h_1)^{q_1-1} 
= p_{s_1}(h_1)^{q_1} p_{s_1}(h_1)^{-1} = 1 \in \pi_1(K(s_1))$. 
Since $p_{s_1}(h_1) \ne 1$, $p_{s_1}(h_1)$ is a torsion element of order $t_{1}  \ge 2$ and 
$q_1 - 1$ is a multiple of the order $t_{1}$. 

So take $q_1$ so that $q_1-1$ is not a multiple of $t_1$ and $q_1 - 1 > t_{1}$. 
Then $p_{s_1}(h_1^{q_1} g_{k, a_2}) \ne 1$. 
\end{proof}

\medskip

Furthermore, 
\[
p_{s_2}(h_1^{q_1} g_{k, a_2}) = p_{s_2}(h_1)^{q_1} p_{s_2}(g_{k, a_2}) =  p_{s_2}(g_{k, a_2}) \ne 1,
\]
because $p_{s_2}(h_1) = 1$. 
Hence, 
\[
\mathcal{S}_K(h_1^{q_1} g_{k, a_2}) \subset \mathbb{Q} - \{ s_1, s_2 \}.
\]


By construction $p_r(h_1) = p_r(g_{k, a_2}) = 1$, 
and hence,  
\[
p_r(h_1^{q_1} g_{k, a_2}) = p_r(h_1)^{q_1} p_r(g_{k, a_2}) = 1.
\] 
for any slope $r \in \{ r_1, \dots, r_k, r_{k+1} \}$. 
This shows $\mathcal{R}_{k+1} \subset \mathcal{S}_K(h_1^{q_1} g_{k, a_2})$. 

Now put 
\[
h_2 = h_1^{q_1} g_{k, a_2} \in G(K),\  \textrm{for which}\ 
\mathcal{R}_{k+1} = \{r_1,\ldots,r_{k}, r_{k+1} \} \subset \mathcal{S}_K(h_{2}) \subset \mathbb{Q} - \{ s_1, s_2 \}.
\]

\medskip

${\bf (3)}$\  We will construct $h_3$ satisfying 
\[
\mathcal{R}_{k+1} = \{r_1,\ldots,r_{k}, r_{k+1} \} \subset \mathcal{S}_K(h_3) \subset \mathbb{Q} - \{ s_1, s_2, s_3 \}.
\]

\medskip

${\bf (3.1)}$\  If $\mathcal{S}_K(h_2) \subset \mathbb{Q} - \{s_1, s_2, s_3\}$, 
then put 
\[
h_3 = h_2, \  
\textrm{for which}\ 
\mathcal{R}_{k+1} = \{r_1,\ldots,r_{k}, r_{k+1} \} \subset \mathcal{S}_K(h_{3}) \subset \mathbb{Q} - \{ s_1, s_2, s_3 \}.
\]

${\bf (3.2)}$\  If $s_3 \in \mathcal{S}_K(h_2)$, i.e. $p_{s_3}(h_2) = 1$, 
then we will modify $h_2$ further to obtain $h_3$ with the above property in ${\bf (3.1)}$. 
By the assumption $s_i$ is not a torsion surgery slope for $i \ge 3$. 

\medskip

Again, following Claim~\ref{alpha_non-commute}, 
together with the surjectivity of $p_{s_3} \colon G(K) \to \pi_1(K(s_3))$, 
we find $a_3 = a(s_3) \in G(K)$ so that $p_{s_3}(a_3) = \alpha_3 \in \pi_1(K(s_3))$ satisfies 
\[
[\alpha_3 p_{s_3}(g_k) \alpha_3,\ p_{s_3}(r_{k+1})] \ne 1,\ \textrm{equivalently}\ p_{s_3}( [a_3 g_k a_3^{-1}, r_{k+1}] ) \ne 1.
\]
Then put 
\[
g_{k, a_3} = [a_3 g_k a_3^{-1},\ r_{k+1}] \in G(K),
\]
which is non-trivial and $p_{s_3}(g_{k, a_3}) \ne 1$. 

For convenience, we collect some properties of $h_2$ and $g_{k, a_3}$. 

\begin{itemize}
\item $p_r(h_2) = p_r(g_{k, a_3}) = 1$ for every slope $r \in \mathcal{R}_{k+1} = \{ r_1, \dots, r_k, r_{k+1} \}$. 
\item $p_{s_1}(h_2) \ne 1$ and $p_{s_2}(h_2) \ne 1$. 
\item $p_{s_3}(h_2) = 1$ and $p_{s_3}(g_{k, a_3}) \ne 1$. 
\end{itemize}

\medskip

For slopes $s_i \in \mathcal{S}$, 
we set an integer $n_i$ as follows. 

If $\pi_1(K(s_i))$ is torsion free, then put $n_i = 1$. 
Suppose that $s_i$ is a torsion surgery slope. 
Then $s_i$ is either a finite surgery slope or a reducing surgery slope. 
For such a surgery slope $s_i$, we set $n_i$ as follows. 
If $s_i$ is a finite surgery slope, then let $n_i = |\pi_1(K(s_i))|$.   
If $s_i$ is a reducing surgery slope, 
then $K(s_i)$ has at most three prime factors \cite{How}  $M_{i, 1},\  M_{i, 2},\ M_{i, 3}$  
($M_{i, 3}$ may be $S^3$).  
Let $m_{i, j} = |\pi_1(M_{i, j})|$; 
if $ |\pi_1(M_{i, j})| = \infty$, then we put $m_{i, j} = 1$. 
Then put $n_i = m_{i, 1}m_{i, 2}m_{i, 3}$. 



\begin{claim}
\label{q_2}
There exists an integer $q_2$ such that 
\[
p_{s_1}(h_2^{q_2 n_1n_2 + 1} g_{k, a_3}^{n_1 n_2}) \ne 1 \in \pi_1(K(s_1))\  \textrm{and}\  
p_{s_2}(h_2^{q_2 n_1n_2+ 1} g_{k, a_3}^{n_1n_2}) \ne 1 \in \pi_1(K(s_2)).
\]
\end{claim}

\begin{proof}
We need to consider three cases: 
\begin{enumerate}
\renewcommand{\labelenumi}{(\roman{enumi})}
\item 
$s_1$ and $s_2$ are not torsion surgery slopes, 
\item
$s_1$ is a torsion surgery slope, but $s_2$ is not a torsion surgery slope, or 
\item 
$s_1$ and $s_2$ are torsion surgery slopes
\end{enumerate}

\smallskip

(i) 
Assume that  
\[
p_{s_1}(h_2^{\ell n_1n_2 + 1} g_{k, a_3}^{n_1n_2}) 
= p_{s_1}(h_2)^{\ell n_1n_2 + 1} p_{s_1}(g_{k, a_3})^{n_1n_2} = 1,\  \textrm{and}
\]
\[
p_{s_1}(h_2^{\ell' n_1n_2 + 1} g_{k, a_3}^{n_1n_2}) 
= p_{s_1}(h_2)^{\ell' n_1n_2 + 1} p_{s_1}(g_{k, a_3})^{n_1n_2} = 1
\]
for some integers $\ell, \ell'$. 
Then we have 
$p_{s_1}(h_2)^{(\ell - \ell') n_1n_2} = 1$ independent whether $p_{s_1}(g_{k, a_3}) = 1$ or not. 
Since $p_{s_1}(h_2) \ne 1$ and $\pi_1(K(s_i))$ is torsion free, 
$\ell = \ell'$. 
This means that $p_{s_1}(h_2^{x n_1n_2 + 1} g_{k, a_3}^{n_1n_2}) = 1$ for at most one integer $x = \ell_1 \ge 1$. 
Similarly $p_{s_2}(h_2^{y n_1n_2 + 1} g_{k, a_3}^{n_1n_2}) = 1$ for at most one integer $y = \ell_2 \ge 1$. 

We choose $q_2 \ne \ell_1, \ell_2$ so that $p_{s_i}(h_2^{q_2 n_1n_2 + 1} g_{k, a_3}^{n_1n_2}) \ne 1 \in \pi_1(K(s_i))$ for $i = 1, 2$. 

\smallskip

(ii) By the definition of $n_1$, 
\[
p_{s_1}(h_2)^{n_1} = 1\ \textrm{and}\ p_{s_1}( g_{k, a_3})^{n_1} = 1.
\] 
Thus we have 
\[
p_{s_1}(h_2^{x n_1n_2 + 1} g_{k, a_3}^{n_1n_2})
= p_{s_1}(h_2)^{x n_1n_2 + 1} p_{s_1}(g_{k, a_3})^{n_1n_2}
= p_{s_1}(h_2)\ne 1
\]
for any integer $x \ge 1$. 

For non-torsion surgery slope $s_2$, as observed in (i), 
$p_{s_2}(h_2^{y n_1n_2+ 1} g_{k, a_3}^{n_1n_2}) = 1$ for at most one integer $y = \ell_2$. 

So we may choose $x = q_2 \ne \ell_2$ so that 
\[
p_{s_1}(h_2^{q_2 n_1n_2 + 1} g_{k, a_3}^{n_1n_2}) \ne 1
\quad \textrm{and} \quad  
p_{s_2}(h_2^{q_2 n_1n_2+ 1} g_{k, a_3}^{n_1n_2}) \ne 1.
\] 

\smallskip

(iii) By the definition of $n_1$ and $n_2$,  
\[
p_{s_1}(h_2)^{n_1} = 1,\  p_{s_1}( g_{k, a_3})^{n_1} = 1\ \textrm{and}
\]
\[
p_{s_2}(h_2)^{n_2} = 1,\  p_{s_2}( g_{k, a_3})^{n_2} = 1. 
\]
Hence, we have
\[
p_{s_1}(h_2^{x n_1n_2 + 1} g_{k, a_3}^{n_1n_2}) 
= p_{s_1}(h_2)^{x n_1n_2 + 1} p_{s_1}(g_{k, a_3})^{n_1n_2} 
= p_{s_1}(h_2)
\ne 1 \in \pi_1(K(s_1)),\ \textrm{and}
\]
\[
p_{s_2}(h_2^{x n_1n_2+ 1} g_{k, a_3}^{n_1n_2}) 
= p_{s_2}(h_2)^{x n_1n_2+ 1} p_{s_2}(g_{k, a_3})^{n_1n_2} 
= p_{s_2}(h_2)
\ne 1 \in \pi_1(K(s_2))
\]
for any integer $x \ge 1$.  
So we may choose a desired $q_2 \ge 1$. 
\end{proof}

\medskip



Furthermore, since $p_{s_3}(h_2) = 1$,\ $p_{s_3}(g_{k, a_3}) \ne 1$ and $\pi_1(K(s_3))$ is torsion free, 
\[
p_{s_3}(h_2^{q_2 n_1n_2 + 1} g_{k, a_3}^{n_1n_2}) 
= p_{s_3}(h_2)^{q_2 n_1n_2 + 1} p_{s_3}(g_{k, a_3})^{n_1n_2}
= p_{s_3}(g_{k, a_3})^{n_1n_2}
\ne 1.
\] 

By construction $p_r(h_1) = p_r(g_{k, a_2}) = 1$ for $r \in \{ r_1, \dots, r_k, r_{k+1} \}$, 
and hence, $h_2 = h_1^{q_1}g_{k, a_2}$ satisfies  
\[
p_r(h_2) = p_r(h_1^{q_1} g_{k, a_2}) = p_r(h_1)^{q_1} p_r(g_{k, a_2}) = 1.
\] 
for any slope $r \in \mathcal{R}_{k+1} = \{ r_1, \dots, r_k, r_{k+1}\}$. 
This shows $\mathcal{R}_{k+1} \subset \mathcal{S}_K(h_2)$. 



Now we put 
\[
h_3 = h_2^{q_2 n_1n_2 + 1} g_{k, a_3}^{n_1n_2} \in G(K),\ \textrm{for which}
\]
\[
\mathcal{R}_{k+1} = \{r_1,\ldots,r_{k}, r_{k+1} \} \subset \mathcal{S}_K(h_{3}) \subset \mathbb{Q} - \{ s_1, s_2, s_3 \}.
\]

Continue the above procedure to obtain $h_{m-1} \in G(K)$ which satisfies 
\[
\mathcal{R}_{k+1} = \{r_1,\ldots,r_{k}, r_{k+1} \} \subset \mathcal{S}_K(h_{m-1}) \subset \mathbb{Q} - \{ s_1, \dots, s_{m-1} \}.
\]

\medskip

${\bf (m)}$\  We will construct $h_m$ satisfying 
\[
\mathcal{R}_{k+1} = \{r_1,\ldots,r_{k}, r_{k+1} \} \subset \mathcal{S}_K(h_m) \subset \mathbb{Q} - \{ s_1, \dots, s_m \}.
\]

\medskip

${\bf (m.1)}$\  If $\mathcal{S}_K(h_{m-1}) \subset \mathbb{Q} - \{s_1, \dots, s_m\}$, 
then put 
\[
h_m = h_{m-1}, \  
\textrm{for which}\ 
\mathcal{R}_{k+1} = \{r_1,\ldots,r_{k}, r_{k+1} \} \subset \mathcal{S}_K(h_{m}) \subset \mathbb{Q} - \{ s_1, \dots, s_m \}.
\]

${\bf (m.2)}$\  If $s_m \in \mathcal{S}_K(h_{m-1})$, i.e. $p_{s_m}(h_{m-1}) = 1$, 
then we will modify $h_{m-1}$ further to obtain $h_m$ with the above property in ${\bf (m.1)}$. 


As above, using Claim~\ref{alpha_non-commute} we construct $g_{k, a_m}$ in $G(K)$ such that $p_{s_m}(g_{k, a_m}) \ne 1$. 
Let us collect properties of $h_{m-1}$ and $g_{k, a_m}$. 

\begin{itemize}
\item
$p_r(h_{m-1}) = p_r(g_{k, a_m}) = 1$ for every slope $r \in \mathcal{S}_{k+1} = \{ r_1, \dots, r_k, r_{k+1}\}$. 

\item
$p_{s_i}(h_{m-1}) \ne 1$ for $i = 1, \dots, m-1$. 

\item
$p_{s_m}(h_{m-1}) =1$ and $p_{s_m}(g_{k, a_m}) \ne 1$. 
\end{itemize}

Recall that by the assumption at most two slopes $s_1$ and $s_2$ can be torsion surgery slopes, 
and $s_i$ with $i \ge 3$ is not a torsion surgery. 
The argument in the proof of Claim~\ref{q_2} shows that 
for each non-torsion surgery $s_i$, 
there is at most one integer $\ell_i \ge 1$ such that 
$p_{s_i}(h_{m-1}^{\ell_i n_1n_2+1} g_{k, a_m}^{n_1n_2}) = 1$. 
We follow the proof of Claim~\ref{q_2} to find a (suitably large) integer $q_{m-1} \ge 1$ so that 
\[
p_{s_i}(h_{m-1}^{q_{m-1} n_1n_2+1} g_{k, a_m}^{n_1n_2}) \ne 1
\]
for $i = 1, \dots, m-1$. 
Moreover, 
since $p_{s_m}(h_{m-1}) = 1$,\ $p_{s_m}(g_{k, a_m}) \ne 1$ and $\pi_1(K(s_m))$ ($m \ge 3$) is torsion free, 
\[
p_{s_m}(h_{m-1}^{q_{m-1} n_1n_2 + 1} g_{k, a_m}^{n_1n_2}) 
= p_{s_m}(h_{m-1})^{q_{m-1} n_1n_2 + 1} p_{s_m}(g_{k, a_m})^{n_1n_2}
= p_{s_m}(g_{k, a_m})^{n_1n_2}
\ne 1.
\] 

On the other hand, 
since $p_r(h_{m-1}) = p_r(g_{k, a_m}) = 1$ for $r \in \{ r_1, \dots, r_k, r_{k+1} \} = \mathcal{R}_{k+1}$, 
\[
p_r(h_{m-1}^{q_{m-1} n_1n_2+1} g_{k, a_m}^{n_1n_2}) = p_r(h_{m-1})^{q_{m-1} n_1n_2+1} p_r(g_{k, a_m})^{n_1n_2} = 1 
\]
for $r \in \mathcal{R}_{k+1}$. 
This shows $\mathcal{R}_{k+1} \subset \mathcal{S}_K(h_{m-1}^{q_{m-1} n_1n_2+1} g_{k, a_m}^{n_1n_2})$. 

Hence, we obtain a non-trivial element 
\[
h_m = h_{m-1}^{q_{m-1}n_1n_2+1} g_{k, a_m}^{n_1n_2} \in G(K)
\]
which satisfies 
\[
\mathcal{R}_{k+1} = \{r_1,\ldots,r_{k}, r_{k+1} \} \subset \mathcal{S}_K(h_{m}) \subset \mathbb{Q} - \{ s_1, \dots s_m \} = \mathcal{S}.
\]

This establishes \textbf{Step 2}. 

\medskip

Then induction on $n$, the cardinality of $\mathcal{R}_n$, 
we have a non-trivial element $g = g_{n} \in G(K)$ with 
\[
\mathcal{S}_n = \{r_1,\ldots,r_{n-1}, r_n \} \subset \mathcal{S}_K(g) \subset \mathbb{Q} - \{ s_1, \dots, s_{m} \}.
\]
This completes a proof of Theorem~\ref{separation}.
\end{proof}


\medskip

\begin{corollary}
Let $K$ be a non-trivial knot without Seifert surgery. 
Given any disjoint finite families of slopes 
$\{ r_1, \ldots, r_n \}$ and $\{ s_1, \ldots, s_m \}$, 
we find a non-trivial element $g$ in $G(K)$ so that 
it becomes trivial after $r_i$--Dehn fillings for all $r_i$, 
while still non-trivial after $s_j$--Dehn fillings for all $s_j$. 
\end{corollary}

\bigskip


\section{Product theorem}
\label{powered_product}

Recall that 
in general we have the following inequality (Propositions~\ref{cup_cap} and \ref{S_K(g)_S_K(g^n)}). 
\[
\mathcal{S}_K(g) \cap \mathcal{S}_K(h) \subset \mathcal{S}_K(g^m) \cap \mathcal{S}_K(h^n)
\subset \mathcal{S}_K(g^m h^n)
\]
for any $m, n \ne 0$. 

The next proposition asserts that the inequality can be replaced by the equality under some condition. 
This improvement has a key role in the proof of Theorems~\ref{realization} and \ref{non_rigid}. 


\begin{thm_S_K_intersection}[(Product theorem)]
Let $K$ be a hyperbolic knot which has no torsion surgery. 
Let $g$ be a non-peripheral element and $h$ a non-trivial element in $[G(K),G(K)]$. 
Then for a given non-zero integer $n$, 
there exists a constant $N > 0$ such that 
\[
\mathcal{S}_K(g) \cap \mathcal{S}_K(h) = \mathcal{S}_K(g^mh^n) 
\] 
for $m \ge N$.  
\end{thm_S_K_intersection}

\begin{proof}
As we mentioned above, 
we have 
\[
\mathcal{S}_K(g) \cap \mathcal{S}_K(h) \subset \mathcal{S}_K(g^mh^n)
\]
for all integers $m$ and $n$ without any condition. 

Let us show that 
\[
\mathcal{S}_K(g) \cap \mathcal{S}_K(h) \supset \mathcal{S}_K(g^mh^n)
\]
under the condition in Theorem~\ref{S_K_intersection}. 
Actually we will show that if $s \not\in \mathcal{S}_K(g)$ or $s \not\in \mathcal{S}_K(h)$, 
then $s \not\in \mathcal{S}_K(g^mh^n)$. 
Note that $\pi_1(K(s))$ has no torsion for all $s\in \mathbb{Q}$ by the assumption. 
We divide the argument into two cases depending upon $s \in \mathcal{S}_K(g)$, i.e. $p_s(g) = 1$ or not. 

\medskip

\noindent
\textbf{Case 1.}\ $s \in \mathcal{S}_K(g)$. 
Then the assumption shows that $s \not\in \mathcal{S}_K(h)$.
Thus $p_s(h) \ne 1$. 
This means that $p_s(g^{m}h^{n}) = p_s(g)^m p_s(h)^n = p_s(h)^n \ne 1$, 
i.e. $s \not\in \mathcal{S}_K(g^mh^n)$ whenever $n \ne 0$. 

\medskip

\noindent
\textbf{Case 2.}\ $s \not\in \mathcal{S}_K(g)$, i.e. $p_s(g) \ne 1$. 
In the following we fix a non-zero integer $n$. 

Assume first that $s$ is a non-hyperbolic surgery slope. 
Since $p_s(g)$ is not a torsion element,  there is at most one integer $m$ such that $p_s(g)^m = p_s(h^{-n})$, 
i.e. $p_s(g^mh^n) = 1$. 
Since there are at most finitely many non-hyperbolic surgery slopes, 
by taking $m$ suitably large, 
$p_s(g^mh^n) \ne1$ for all non-hyperbolic surgery slopes $s$. 

Suppose that $s$ is a hyperbolic surgery slope. 

Now let us separate the argument depending upon 
$g \in [G(K), G(K)]$ or not. 

\noindent
\textbf{Case 2-1.}\ 
$g \in [G(K), G(K)]$. \ 
Since $h \in [G(K), G(K)]$ and $g$ is non-peripheral, 
following Theorem~\ref{scl_bound}
we may take $\delta_g > 0$ so that for all hyperbolic surgery slopes $s$, 
\[
\mathrm{scl}_{\pi_1(K(s))}(p_s(g)) > \delta_g > 0\ \textrm{whenever}\ p_s(g) \ne 1.
\] 
Choose $m$ so that $m\, \delta_g >n\, \mathrm{scl}_{G(K)}(h)+\frac{1}{2}$, 
then by Lemmas \ref{scl_product} and \ref{monotonicity}
\begin{align*}
\mathrm{scl}_{\pi_1(K(s))}(p_s(g^{m}h^{n})) &> m\, \mathrm{scl}_{\pi_1(K(s))}(p_s(g)) - n\,\mathrm{scl}_{\pi_1(K(s))}(p_s(h))-\frac{1}{2}\\
& > m\, \delta_{g} - n\, \mathrm{scl}_{G(K)}(h) - \frac{1}{2} > 0
\end{align*}
so $p_s(g^{m}h^{n}) \neq 1$, 
hence $s \not\in \mathcal{S}_K(g^mh^n)$. 

\vskip 0.7cm

\noindent
\textbf{Case 2-2.}\ 
$g \not\in [G(K), G(K)]$. \ In this case to apply Lemma~\ref{scl_product} we need to take care when $\pi_1(K(s))/[\pi_1(K(s)), \pi_1(K(s))] = H_1(K(s))$ is not finite, namely $s = 0$. 
So we first observe:  

\begin{claim}
\label{s=0}
If $g \not\in [G(K), G(K)]$, 
then $p_0(g^mh^n) \ne 1$ for all $m \ne 0$. 
\end{claim}

\begin{proof}
Suppose that $p_0(g^mh^n) = 1$, i.e. $p_0(g)^m = p_0(h)^{-n}$. 
Abelianizing this we have $m [p_0(g)] = -n [p_0(h)] \in H_1(K(0)) = \mathbb{Z}$. 
Since $h \in [G(K), G(K)]$,  $[p_0(h)] = 0 \in H_1(K(0))$, 
and since $g \not\in [G(K), G(K)]$, 
$[p_0(g)] \ne 0$. 
Hence, $m = 0$.  
\end{proof}

Following Claim~\ref{s=0} we may assume $s \ne 0$, thus $K(s)$ is a rational homology $3$--sphere. 

Since $g$ is assumed to be non-peripheral, following Theorem~\ref{scl_bound}
we may take $\delta_g > 0$ so that for all hyperbolic surgery slopes $s$, 
\[
\mathrm{scl}_{\pi_1(K(s))}(p_s(g)) > \delta_g > 0\quad \textrm{whenever}\quad p_s(g) \ne 1.
\] 

Choose $m$ so that $m\, \delta_g >n\, \mathrm{scl}_{G(K)}(h)+\frac{1}{2}$ (for the fixed integer $n$),  
then apply the same argument in Case 2-1 above to see that $s \not\in \mathcal{S}_K(g^mh^n)$. 

This completes the proof of Theorem~\ref{S_K_intersection}. 
\end{proof}

\bigskip

\section{Realization theorem}
\label{proof}

In this section, we give a proof of Theorem~\ref{realization} using Theorem~\ref{separation} and Theorem~\ref{S_K_intersection}. 

\begin{thm_realization}[(Realization theorem)]
Let $K$ be a hyperbolic knot without torsion surgery. 
Let $\mathcal{R} = \{ r_1, \ldots, r_n\}$ be any finite \(possibly empty\) subset of $\mathbb{Q}$ whose complement does not contain a Seifert surgery. 
Then there exists a non-trivial element $g \in [G(K), G(K)] \subset G(K)$ such that 
$\mathcal{S}_K(g) = \mathcal{R}$.
\end{thm_realization}


\begin{proof}
Apply Theorem~\ref{separation} to find an element $g_1 \in [G(K), G(K)] \subset G(K)$ such that 
\[
\{ r_1, \ldots, r_n \} \subset \mathcal{S}_K(g).
\]
If $\mathcal{S}_K(g_1) = \{ r_1, \ldots, r_n \}$, 
then $g_1$ is a desired element of $G(K)$. 

Assume that $\mathcal{S}_K(g_1) - \{ r_1, \ldots, r_n \}$ is not empty. 
Since $\mathcal{S}_K(g_1) \subset \mathbb{Q}$ is a finite subset (Theorem~\ref{S_K_hyperbolic}), 
we may write 
\[
\mathcal{S}_K(g_1) - \{ r_1, \ldots, r_n \} =  \{ s_1, \ldots, s_m\}.
\]
By the assumption of Theorem~\ref{realization} $s_i$ is not a Seifert surgery slope. 

\begin{claim}
There exists a non-trivial element $g_2 \in [G(K), G(K)]$ such that 
\[
\mathcal{S}_K(g_1) \cap \mathcal{S}_K(g_2) = \{ r_1, \ldots, r_n \}.
\]
Furthermore, if $g_1$ is peripheral, then $g_2$ is chosen to be non-peripheral. 
\end{claim}

\begin{proof}
First assume that $g_1$ is non-peripheral. 
We apply Theorem~\ref{separation} again to get an element $h \in [G(K), G(K)] \subset G(K)$ which satisfies
\[
\{r_1, \ldots, r_n \} \subset \mathcal{S}_K(g_2) \subset \mathbb{Q} - \{ s_1, \ldots, s_m \}.
\]
Then 
\[
\mathcal{S}_K(g_1) \cap \mathcal{S}_K(g_2) = \{ r_1, \ldots, r_n \}.
\]

Suppose that $g_1$ is peripheral, i.e. it is conjugate to a power of $\mu^p\lambda^q$ for some coprime integers $p, q$. 
(Note that since $K$ has no torsion surgery, $p/q$ is not a finite surgery slope.)
Since $\{ r_1 \dots, r_n \} \subset \langle\!\langle g_1  \rangle\!\rangle = \langle\!\langle p/q  \rangle\!\rangle$, 
Proposition~\ref{slope} shows that $n = 1$ and $r_1 = p/q$. 
Moreover, since $g_1 \in [G(K), G(K)]$,  we have $p/q = 0$. 
Thus $\mathcal{S}_K(g_1) = \{ 0 \}$. 
Now we take a non-longitudinal slope $r_2$ and apply Theorem~\ref{separation} to get a non-trivial element $g_2 \in [G(K), G(K)]$ such that 
$\{ r_1, r_2 \} \subset \mathcal{S}_K(g_2)$. 
Then 
\[
\mathcal{S}_K(g_1) \cap \mathcal{S}_K(g_2) = \{ r_1 \} = \{ 0 \}, 
\]
and the above argument shows that $g_2$ is non-peripheral. 
\end{proof}

If $g_1$ is non-peripheral, then following Theorem~\ref{S_K_intersection} 
there exists $N_1 > 0$ such that 
\[
\mathcal{S}_K(g_1^n g_2) = \mathcal{S}_K(g_1) \cap \mathcal{S}_K(g_2) = \{ r_1 \dots, r_n \}
\]
for $n \ge N_1$. 

If $g_1$ is peripheral, then as above $g_2$ is non-peripheral. 
Then following Theorem~\ref{S_K_intersection} 
there exists $N_2 > 0$ such that 
\[
\mathcal{S}_K(g_1 g_2^n) = \mathcal{S}_K(g_1) \cap \mathcal{S}_K(g_2) = \{ r_1 \} = \{ 0 \}
\]
for $n \ge N_2$. 
Put $g = g_1^n g_2$ (if $g_1$ is non-peripheral) or $g_1 g_2^n$ (if $g_1$ is peripheral) to obtain a desired element. 
This completes a proof.
\end{proof}

\medskip 

\begin{remark}
For infinitely many integers $n$, 
$g = g_1^n g_2$ \(if $g_1$ is non-peripheral\) or $g_1 g_2^n$ \(if $g_1$ is peripheral\) are not peripheral. 
See the proof of Claim~\ref{infinite_conjugacy_m}. 
\end{remark}


\bigskip


\section{Elements with the same Dehn filling trivialization}
\label{same trivialization}

For a given $g \in G(K)$, 
it is easy to find an element $h \in G(K)$ such that $\mathcal{S}_K(g) \subset \mathcal{S}_K(h)$. 
For instance,  
every element $h$ in the normal closure $\langle \! \langle g \rangle\! \rangle$ provides such an element. 
In this section we will present two ways to find elements $h$ which satisfy 
$\mathcal{S}_K(h) =  \mathcal{S}_K(g)$ answering Question~\ref{identical}. 
See Theorems~\ref{same_trivialization} and \ref{non_rigid}. 

\bigskip

\subsection{Residual finiteness of $3$--manifold groups and Dehn filling trivialization}
\label{subsection:Residual-finiteness}

In this subsection we will prove the following. 
Recall that for $a, b \in G(K)$, 
$a^b$ denotes $b^{-1} a b \in G(K)$. 

\begin{thm_same_trivialization}
Let $K$ be a non-trivial knot. 
For any non-trivial element $g \in G(K)$, we have 
\[
\mathcal{S}_K(g) = \mathcal{S}_K(g^{g^{\alpha}} g^{-2})\ \textrm{for any}\ \alpha \in G(K).
\]
\end{thm_same_trivialization}


The proof of Theorem~\ref{same_trivialization} requires both the residual finiteness of $3$--manifold groups and existence of non-residually finite one relator groups. 

We begin by recalling some definitions. 
A group $G$ is \textit{residually finite\/} if for each non-trivial element $g$ in $G$, 
there exists a normal subgroup of finite index not containing $g$.  
This is equivalent to say that for every $1\neq g \in G$ there exists a homomorphism 
$\varphi \colon G \rightarrow F$ to some finite group $F$ such that $\varphi(g)\neq 1$. 

For later convenience, 
we slightly strengthen this to the following form.

\begin{proposition}
\label{RF_strong}
Assume that $G$ is residually finite. 
Then for any finite family of non-trivial elements $g_1, \ldots, g_n \in G$, 
 there exists a homomorphism 
$\varphi \colon G \rightarrow F$ to some finite group $F$ such that $\varphi(g_i)\neq 1$. 
\end{proposition}

\begin{proof}
Since $G$ is residually finite, 
we have a homomorphism $\varphi_i \colon G \to F_i$ to a finite group $F_i$ such that $\varphi_i(g_i) \ne 1$ for $i = 1, \ldots, n$. 
Consider a homomorphism $\varphi$ 
from $G$ to  a finite group $F = F_1 \times \cdots \times F_n$ such that 
$\varphi(g) = (\varphi_1(g), \ldots, \varphi_n(g)) \in F$. 
Then $\varphi$ is a desired homomorphism from $G$ to the finite group $F$.  
\end{proof}


Let $\mathcal{F}(G)$ be the set of elements of $G$ which is mapped to the trivial element for every finite group $F$ and every homomorphism $\varphi \colon G \rightarrow F$. Then, by definition, $G$ is residually finite if and only if $\mathcal{F}(G)=\{1\}$. 

For $3$--manifolds, we have:

\begin{theorem}[(\cite{Hem_residual_finite,Pe1,Pe2,Pe3})]
\label{residually_finite_3mfd}
The fundamental group of every compact $3$--manifold is residually finite.
\end{theorem}

\medskip
Let $F_n=\langle a_1,a_2,\ldots, a_n\rangle$ be a free group of rank $n  > 1$ generated by $a_1,a_2,\ldots, a_n$.
For an element $w \in F_n$, 
which we regard as a word over $\{a_1^{\pm 1},\ldots, a_n^{\pm 1}\}$ as usual, 
let us write
 \[ G_w := F_n \slash \langle\!\langle w \rangle \! \rangle = \langle a_1,a_2,\ldots, a_n\ | \: w \rangle,
  \]
which is the one-relateor group whose relator is $w$. 
The natural projection of $F_n$ to $G_w$ is denoted by 
\[
\pi_w\colon F_n \rightarrow G_w. 
\]

\begin{theorem}
\label{residually-finite}
Assume that we have two non-trivial elements $v$ and $w$ in $F_n$ such that $\pi_w(v) \in \mathcal{F}(G_w)- \{ 1 \}$. 
Then for any homomorphism $\phi \colon F_n \to G(K)$, 
we have 
\[
\mathcal{S}_K(\phi(w)) \subset \mathcal{S}_K(\phi(v)).
\] 
\end{theorem}

\begin{proof}
If $\mathcal{S}_K(\phi(w)) = \emptyset$, 
then nothing to prove. 
So we assume $\mathcal{S}_K(\phi(w)) \ne \emptyset$. 
Let $s$ be a slope in $\mathcal{S}_K(\phi(w))$. 
Since $p_s(\phi(w))=1$, 
there is a homomorphism $\psi\colon G_w \to  \pi_1(K(s))$ that makes the following diagram commutative.
\[
\xymatrix{
F_n \ar[r]^-{\phi} \ar[d]_{\pi_w} &  G(K) \ar[d]^{p_s}\\
G_w=F_n \slash \langle\!\langle w \rangle \! \rangle \ar[r]^-{\psi} & \pi_1(K(s))
}
\]
Assume to the contrary that $s \not \in \mathcal{S}_K(\phi(v))$, so $p_s(\phi(v)) \neq 1$. 
By the residual finiteness of $\pi_1(K(s))$ there exists a homomorphism $\kappa\colon \pi_1(K(s)) \to F$ to a finite group $F$ such that $\kappa(p_s(\phi(v))) = \kappa(\psi(\pi_w(v))) \neq 1$. 
However, this implies that $\kappa \circ \psi\colon G_w \rightarrow F$ satisfies $\kappa \circ \psi(\pi_w(v)) \neq 1$. 
This contradicts the assumption that $\pi_w(v) \in \mathcal{F}(G_w) - \{1\}$. 
\end{proof}

To exploit Theorem~\ref{residually-finite}, we need a non-residually finite one-relator group $G_w$ and an element in $\mathcal{F}(G_w) - \{1\}$. Although the most famous and the simplest example of non-residually finite one-relator group is the Baumslag--Solitar group \cite{BaumslagSolitar}, here we use a group given by Baumslag--Miller--Troeger \cite{BMT}, because for this relator $w$, $\mathcal{F}_w$ contains an element in a quite simple form.


\begin{theorem}[(\cite{BMT})]
\label{theorem:BMT}
Let $u, v \in F_n=\langle a_1,a_2,\ldots, a_n \rangle$ be elements of free group of rank $n >1$ such that $uv \neq vu$. 
Let us take
\[
w = v^{v^{u}}v^{-2} \in F_n. 
\]
Let
\[
G_{w} = \langle a_1,a_2,\ldots, a_n \mid  w \rangle, \quad 
\]
be a one-relator group and 
\[
\pi_{w} \colon F_n \rightarrow G_{w}
\]
the natural epimorphism. 
Then, $G_{w}$ is not residually finite, and $\pi_{w}(v) \in \mathcal{F}(G_{w}) - \{ 1 \}$. 
\end{theorem}

Applying Theorem~\ref{residually-finite} in this special case, we get the following.

\begin{corollary}
\label{same_trivialization_residually_finite}
Let $u, v \in F_n=\langle a_1,a_2,\ldots, a_n \rangle$ be elements of free group of rank $n >1$ such that $uv \neq vu$. 
Take $w = v^{v^u}v^{-2}$. 
Let $\phi \colon F_n \rightarrow G(K)$ be a homomorphism. 
Then for $g = \phi(v)$ and $h = \phi(w)$, 
we have   $\mathcal{S}_K(g) = \mathcal{S}_K(h)$.
\end{corollary}

\begin{proof}
By the choice of elements $u, v, w$, 
Theorem~\ref{theorem:BMT} shows that 
\[
\pi_w \colon F_n \to G_w
\] 
satisfies 
$\pi_w(v) \in \mathcal{F}(G_{w}) - \{ 1 \}$. 
Then apply Theorem~\ref{residually-finite} to see that 
\[
\mathcal{S}_K(h) = \mathcal{S}_K(\phi(w)) \subset  \mathcal{S}_K(\phi(v)) = \mathcal{S}_K(g).
\] 

Conversely, for a slope $s$, 
if $p_s(g)=p_s(\phi(v))=1$, 
then 
\[
p_s(h)= p_s(\phi(v^{v^u}v^{-2}))=1
\]
so $\mathcal{S}_K(g) \subset \mathcal{S}_K(h)$.
\end{proof}

\medskip



\begin{proof}[Proof of Theorem~\ref{same_trivialization}]
Take $F_2=\langle a_1, a_2 \rangle$, $v=a_1$ and $u =a_2$, 
and let $\phi \colon F_2  \rightarrow G(K)$ be a homomorphism given by 
$\phi(v)=g$ and $\phi(u)= \alpha$, where $\alpha \in G(K)$ is chosen arbitrary.
Then by Corollary~\ref{same_trivialization_residually_finite}, 
$h = \phi(v^{v^u}v^{-2}) = g^{g^{\alpha}} g^{-2}$ satisfies 
$\mathcal{S}_K(h) = \mathcal{S}_K(g)$. 
\end{proof}

\medskip
This construction can be repeated. 
For every $\alpha_1, \alpha_2, \ldots \in G(K)$, 
let us define $g_i$ inductively as $g_{i+1} = g_{i}^{g_i^{\alpha_i}}g_{i}^{-2}$, where $g_0 = g$.  
Then for each $i$, $\mathcal{S}_K(g_i) = \mathcal{S}_K(g)$.  

\medskip

Note that $\alpha$ can be taken arbitrary, 
so we may expect $g^{g^{\alpha}} g^{-2}$ is not conjugate to any power of $g$. 

Now we discuss how we can obtain non-conjugate elements for the form 
$g^{g^{h}} g^{-2}$ by varying $h$. 


We denote an element in
$\mathrm{PSL}(2, \mathbb{C}) = \mathrm{SL}(2, \mathbb{C})/ \{ \pm I \}$ 
by 
$\begin{bmatrix}
a & b \\
c & d
\end{bmatrix}$
to distinguish from 
a matrix 
$\begin{pmatrix}
a & b \\
c & d
\end{pmatrix}$ 
in $\mathrm{SL}(2, \mathbb{C})$.
Then 
$\mathrm{tr}\begin{bmatrix}
a & b \\
c & d
\end{bmatrix}$ 
is understood to be $\pm(a+ d)$.


\begin{proposition}
\label{conjugacy}
Let $K$ be a hyperbolic knot and $g, h$ non-trivial elements of $G(K)$. 
Then we have the following. 
\begin{enumerate}
\renewcommand{\labelenumi}{(\arabic{enumi})}
\item
Assume that $g$ is non-peripheral. 
Let $\rho \colon G(K) \to PSL_2(\mathbb{C})$ be a holonomy representation with 
$\rho(g) =  \begin{bmatrix}
	\alpha & 0 \\[2pt]
	0 & \alpha^{-1}
\end{bmatrix}$ and 
$\rho(h) =  \begin{bmatrix}
	x & y \\[2pt]
	z & u
\end{bmatrix}$.
Then 
$(\alpha - \alpha^{-1})^2  (xu)^2 - (\alpha - \alpha^{-1})(xu) -1$ \(up to sign\) is an invariant 
of conjugacy class 
of $g^{g^{h}}g^{-2}$. 

\item
Assume that $g$ is peripheral. 
Let $\rho \colon G(K) \to PSL_2(\mathbb{C})$ be a holonomy representation with 
$\rho(g) 
=  \begin{bmatrix}
	1 & \alpha \\[2pt]
	0 & 1
\end{bmatrix}$ 
and 
$\rho(h) 
=  \begin{bmatrix}
	x & y \\[2pt]
	z & u
\end{bmatrix}$. 
Then 
$2z^4 \alpha^4 +2$ \(up to sign\) is an invariant of conjugacy class 
of $g^{g^{h}}g^{-2}$. 
In particular, 
if $2z^4 \alpha^4 +2 \ne \pm 2$, 
then $g^{g^{h}}g^{-2}$ is not peripheral. 
\end{enumerate}

\end{proposition}


\begin{proof}
(1)\ 
Put $\gamma = g^h = h^{-1} g h$. 

Write
$\rho(\gamma) = 
\begin{bmatrix}
	a & b \\[2pt]
	c & d
\end{bmatrix}$.  
Then 
\[
\rho(g^{g^h} g^{-2}) = \rho(\gamma^{-1} g \gamma g^{-2}) 
= 
\begin{bmatrix}
	ad \alpha^{-1} - bc\alpha^{-3} & bd \alpha^3 - bd \alpha \\[2pt]
	-ac\alpha^{1} + ac \alpha^{-3} & -bc \alpha^3 + ad \alpha
\end{bmatrix}.
\]

So we have 
\[
\mathrm{tr}\rho(g^{g^h} g^{-2}) 
= \pm\left( ad\left( (\alpha + \alpha^{-1}) - (\alpha^3 + \alpha^{-3})\right) + (\alpha^3 + \alpha^{-3})\right).
\]  
Note that $(\alpha + \alpha^{-1}) - (\alpha^3 + \alpha^{-3}) \ne 0$. 
Suppose for a contradiction that $\alpha + \alpha^{-1} = \alpha^3 + \alpha^{-3} = (\alpha + \alpha^{-1})(\alpha^2  -1 + \alpha^{-2})$.
Then $\alpha^2 + \alpha^{-2} = 2$, which shows that $\alpha^2 = 1$. 
$\rho(g) = I$, hence $g$ is trivial. 
This is a contradiction.
Therefore $ad$ is a conjugacy invariant of $g^{g^h} g^{-2}$. 
To express $ad$ by $\rho(h)$, let us write 
$\rho(h) 
= 
\begin{bmatrix}
	x & y \\[2pt]
	z & u
\end{bmatrix}
$.

Then $\rho(\gamma) = \rho(h^{-1} g h) 
= 
\begin{bmatrix}
	xu\alpha - yz \alpha^{-1} & yu\alpha - yu\alpha^{-1} \\[2pt]
	-xz \alpha + xz \alpha^{-1} & -ya \alpha + xu \alpha^{-1}
\end{bmatrix}
=
\begin{bmatrix}
	a & b \\[2pt]
	c & d
\end{bmatrix}$.

So  we have 
\[
ad = (xu\alpha - yz\alpha^{-1})(xu\alpha^{-1} - yz\alpha) = 
-(\alpha - \alpha^{-1})^2 (xu)^2 + (\alpha - \alpha^{-1})(xu) +1. 
\] 

Note that $\alpha - \alpha^{-1} \ne 0$, for otherwise $\alpha^2 = 1$ and $\rho(g) = I$, and hence $g = 1$, a contradiction.
Since $h$ is arbitrarily chosen, 
we expect there are infinitely many elements $h \in G(K)$ such that 
$xu$, and hence $ad$ takes infinitely many values. 

\medskip

(2)\ 
Put $\gamma = g^h = h^{-1} g h$. 

Write
$\rho(\gamma) = 
\begin{bmatrix}
	a & b \\[2pt]
	c & d
\end{bmatrix}$ as above. 
Then 
\[
\rho(g^{g^h} g^{-2}) = \rho(\gamma^{-1} g \gamma g^{-2}) 
= 
\begin{bmatrix}
	1+cd \alpha & - 2\alpha - 2cd \alpha^2 +d^2 \alpha \\[2pt]
	-c^2 \alpha & 2c^2 \alpha^2 + 1-cd \alpha
\end{bmatrix}.
\]

So we have 
\[
\mathrm{tr}\rho(g^{g^h} g^{-2}) 
= \pm (2c^2 \alpha^2  + 2).
\]  
To express this by $\rho(h)$ 
we write 
$\rho(h) 
= 
\begin{bmatrix}
	x & y \\[2pt]
	z & u
\end{bmatrix}$.

Then $\rho(\gamma) = \rho(h^{-1} g h) 
= 
\begin{bmatrix}
	1+zu\alpha  & u^2\alpha \\[2pt]
	 -z^2\alpha & 1-zu\alpha 
\end{bmatrix}
=
\begin{bmatrix}
	a & b \\[2pt]
	c & d
\end{bmatrix}$. 

This implies 
\[
2c^2 \alpha^2  + 2
=
2z^4 \alpha^4  + 2. 
\]

Since $h$ is arbitrarily chosen, 
we expect there are infinitely many elements $h \in G(K)$ such that 
$2z^4 \alpha^4 + 2$ 
takes infinitely many values up to sign. 

Recall that $g \in G(K)$ is peripheral if and only if $\mathrm{tr}\rho(g) = \pm 2$. 
Thus if $2z^4 \alpha^4 + 2 \ne \pm 2$, 
then $g^{g^h} g^{-2}$ is not peripheral. 
\end{proof}

\bigskip

In general it seems to be quite difficult to explicitly determine $\mathcal{S}_K(g)$ for a given non-trivial element $g \in G(K)$. 
Now let us restrict our attention to peripheral element. 
Recall that $\mathcal{S}_K(\mu) = \emptyset$. 
Furthermore, for any non-meridional slope element $\mu^p\lambda^q$, 
Proposition~\ref{slope} shows that 
$\mathcal{S}_K(\mu^p\lambda^q) = \{ p/q \}$ for any knot $K$ without finite surgery. 

So it might be reasonable to ask: 

\begin{question}
Assume that $K$ has no finite surgery. 
If $\mathcal{S}_K(g)$ consists of a single slope,  
then is $g$ a peripheral element?
\end{question}


As an application of Theorem~\ref{same_trivialization} and Proposition~\ref{conjugacy} we prove the following result, 
which answers this question in the negative. 

\begin{proposition}
\label{peripheral_non-pripheral}
Let $K$ be a hyperbolic knot without torsion surgery. 
For all but at most four slope elements $g = \mu^p \lambda^q$, 
there are non-peripheral elements $g'$ satisfying 
$\mathcal{S}_K(g') = \mathcal{S}_K(g) = \{ p/q \}$. 
\end{proposition}

\begin{proof}
Let $(\mu, \lambda)$ be a preferred meridian-longitude pair of $K$; $\mu\lambda = \lambda\mu$. 
Let us take $h$ which does not belong to $\langle \mu, \lambda \rangle \cong \mathbb{Z} \oplus \mathbb{Z}$. 
Let $g = \mu^p \lambda^q$.  
Then it follows from Theorem~\ref{same_trivialization} we see that 
$\mathcal{S}_K(g^{g^h}) = \mathcal{S}_K(g)$ for any $h$. 
Now we show that for the given $h \in G(K) - \langle \mu, \lambda \rangle$, 
for all but at most four coprime pairs $(p, q)$, 
$g^{g^h}$ is non-peripheral. 
To see this we take a holonomy representation $\rho \colon G(K) \to PSL_2(\mathbb{C})$ so that 
$\rho(\mu) 
=  \begin{bmatrix}
	1 & 1 \\[2pt]
	0 & 1
\end{bmatrix}$,
$\rho(\lambda) 
=  \begin{bmatrix}
	1 & \alpha_0 \\[2pt]
	0 & 1
\end{bmatrix}$ for some complex number $\alpha_0 \not\in \mathbb{R}$ so that 
$\rho(g)
=\rho(\mu^p \lambda^q) 
= \begin{bmatrix}
	1 & p+q \alpha_0 \\[2pt]
	0 & 1
\end{bmatrix}$. 
Let us write 
$\rho(h) 
= 
\begin{bmatrix}
	x & y \\[2pt]
	z & u
\end{bmatrix}$.
By Proposition~\ref{conjugacy}(2), 
to see that $g' = g^{g^h}g^{-2}$ is non-peripheral,
it is sufficient to show that $2z^{4}(p+q\alpha_0)^4 +2 \neq \pm 2$.


\begin{claim}
$z \ne 0$. 
\end{claim}

\begin{proof}
Assume for a contradiction that 
$z = 0$. 

If $\rho(h)$ is parabolic, 
then 
$\rho(h) 
=  \begin{bmatrix}
	1 & y \\[2pt]
	0 & 1
\end{bmatrix}$, and thus $h \in \langle \mu, \lambda \rangle$, a contradiction. 
So we assume $\rho(h)$ is loxodromic. 

Then 
$\rho(h) 
=  \begin{bmatrix}
	x & y \\[2pt]
	0& x^{-1}
\end{bmatrix}$, 
and we may assume that $||x|| > 1$, if necessary by taking the inverse of $h$ instead of $h$. 
Note that $\rho(\mu)$ and $\rho(h)$ fixes $\infty$ in the sphere at infinity $\partial \mathbb{H}^3 = \mathbb{C} \cup \{ \infty \}$. 
Then it turns out that $\langle \rho(\mu), \rho(h) \rangle$ contains 
$\begin{bmatrix}
	1 & x^{-2n}\\[2pt]
	0 & 1
\end{bmatrix}$. 
Since $|| x || > 1$, 
$\begin{bmatrix}
	1 & x^{-2n}\\[2pt]
	0 & 1
\end{bmatrix} 
\to 
\begin{bmatrix}
	1 & 0 \\[2pt]
	0 & 1
\end{bmatrix}$
as $n \to \infty$. 
This contradicts the duscreteness of $\rho(G(K))$. 
\end{proof}

\medskip

Since $z \ne 0$, 
$2 z^4( p + q \alpha_0)^4 = 0$ implies $(p, q) = (0, 0)$, a contradiction. 
Also there are at most four values of $\alpha$ satisfy $2z^4\alpha^4 + 4 = 0$, 
and to each $\alpha$ there is at most one coprime pair $(p, q)$ with $\alpha = p + q\alpha_0$. 
This completes a proof.  
\end{proof}



\begin{example}
\label{figure-eight_holonomy}
Let $K$ be the figure-eight knot. 
Take a meridian $\mu$ and a non-trivial element $h$ as depicted in Figure~\ref{figure-eight_mu_h}. 

\begin{figure}[htb]
\centering
\includegraphics[bb=0 0 121 113,width=0.2\textwidth]{figure-eight_knot_mu_h.eps}
\caption{$\mu$ and $h$ generate $G(K)$.} 
\label{figure-eight_mu_h}
\end{figure}

Let us take a holonomy representation 
$\rho \colon G(K) \to PSL_2(\mathbb{C})$ so that 
$\rho(\mu) = 
\begin{bmatrix}
	1 & 1 \\[2pt]
	0 & 1
\end{bmatrix}$ and 
$\rho(h) = \begin{bmatrix}
	1 & 0 \\[2pt]
	-\omega & 1
\end{bmatrix}$, where $\omega = \frac{-1 + \sqrt{3}\,i}{2}$; see \cite{CallahanReid,Goda}. 

Then $\lambda = h\mu^{-1}h^{-1}\mu^2 h^{-1}\mu^{-1}h$, and 
$\rho(\lambda) = 
\begin{bmatrix}
	1 &2\sqrt{3}\,i \\[2pt]
	0 & 1
\end{bmatrix}$.  
For a slope element $\mu^p \lambda^q$, 
we have
$\rho(\mu^p \lambda^q) = 
\begin{bmatrix}
	1 & p + 2\sqrt{3}\,qi \\[2pt]
	0 & 1
\end{bmatrix}$. 

For simplicity put $g = \mu^p \lambda^q$. 
Then it follows from Theorem~\ref{same_trivialization} that 
\[
\mathcal{S}_K(g) = \mathcal{S}_K(g^{g^{h}}).
\]
Let us observe that $g^{g^{h}}$ is non-peripheral using Proposition~\ref{conjugacy}(2). 
Since $\alpha = p + 2\sqrt{3}\,qi$ and $(z, u) = (-\omega, 1)$, 
\[
z^4 \alpha^4 + 1 = \omega^4(p + 2\sqrt{3}\,qi)^4 + 1 = \omega (p + 2\sqrt{3}\,qi)^4 +1.
\]
Assume for a contradiction that $\omega (p + 2\sqrt{3}\,qi)^4 +1 = \pm 1$, 
i.e.  $\omega (p + 2\sqrt{3}\,qi)^4 = 0, -2$ for some coprime pairs $(p, q)$. 
Then 
$|| \omega (p + 2\sqrt{3}\,qi)^4 || = (p^2 + 12q^2)^2 = 0$ or $2$.  
This is impossible. 
So for every coprime pair $(p, q)$, $g^{g^{h}}$ is non-peripheral, where $g = \mu^p\lambda^q$. 
\end{example}

This example demonstrates that 
for any slope element $g = \mu^p\lambda^q$, 
non-peripheral element $g^{g^{h}}$ satisfies 
\[
\mathcal{S}_K(g^{g^{h}}) = \mathcal{S}_K(g) = \{ p/q \}. 
\]
The second equality follows from Proposition~\ref{slope} and the fact that the figure-eight knot has no finite surgeries. 


\bigskip

\subsection{Application of Product theorem}
\label{yet_another_construction}

Suppose that $K$ has no torsion surgery. 
Then $\mathcal{S}_K(g) = \mathcal{S}_K(g^n)$ for all $n \ne 0$ (Proposition~\ref{S_K(g)_S_K(g^n)}). 
If $g \not\in [G(K), G(K)]$, i.e. it is homologically non-trivial, 
and obviously $g^m$ and $g^n$ are not conjugate when $m \ne n$. 
If $g \in [G(K), G(K)]$ and $\mathrm{scl}_{G(K)}(g) > 0$, 
then since the stable commutator length is invariant under conjugation,  
Lemma~\ref{scl_g^k} show that $g^m$ and $g^n$ are not conjugate when $m \ne n$. 


The aim of this subsection is to prove the following theorem which requires some extra condition for $K$ and  an element $g \in G(K)$, 
but we may take infinite elements $h$ with $\mathcal{S}_K(h) = \mathcal{S}_K(g)$ so that they are mutually non-conjugate and 
not conjugate to any power of $g$. 

\begin{thm_non_rigid}
Let $K$ be a hyperbolic knot without torsion surgery. 
For any non-peripheral element $g \in [G(K), G(K)]$ there are infinitely many, non-conjugate elements 
$\alpha_m \in G(K)$ such that $\mathcal{S}_K(\alpha_m) = \mathcal{S}_K(g)$ 
and $\alpha_m$ is not conjugate to any power of $g$. 
\end{thm_non_rigid}


\medskip

\begin{proof}[Proof of Theorem~\ref{non_rigid}]
Let $g$ be a non-peripheral element in $[G(K), G(K)]$. 
Since $\mathcal{S}_K(g)$ is a finite set (Theorem~\ref{S_K_hyperbolic}), 
we may take a hyperbolic surgery slope $s$ so that $s \not\in \mathcal{S}_K(g)$, i.e. $p_s(g) \ne 1$. 

Since $G(K)$ is residually finite, 
following Proposition~\ref{RF_strong} we have a homomorphism $\varphi \colon G(K) \to F$ from 
$G(K)$ to a finite group $F$ such that $\varphi(g) \ne 1$ and $\varphi([g, s]) \ne 1$. 
(Since $g$ is non-peripheral and $s$ is peripheral, they do not commute and $[g, s] \ne 1$. 
Actually, if a non-trivial element $g$ commutes with a peripheral element $s$, 
then the images of $g$ by a holonomy representation is also a parabolic element fixing the same fixed point in the sphere at infinity 
$\partial \mathbb{H}^3$. This means $g$ is also peripheral and contradicts the assumption of $g$. 
See also \cite[Theorem 1]{Sim2}. )


Then choose  (and fix) an integer $p >0$ so that $\varphi(g^p) = \varphi(g)^p = 1$ in $F$. 

For an integer $m$, let us take 
\[
\alpha_m = g^{p+pm-1} s g^{-p+1} s^{-1} = g^{p+pm-1} h^{-p+1} \in G(K), 
\]
where $h = s g s^{-1}$.  


Since $h = s g s^{-1}$, 
$\mathcal{S}_K(g) = \mathcal{S}_K(h)$. 
It follows from Theorem~\ref{S_K_intersection} that for a given $-p+1$, 
there exists a constant $N > 0$ such that for any integer $m \ge N$, 
we have 
\[
\mathcal{S}_K(g^{p + pm -1} h^{-p+1}) = \mathcal{S}_K(g) \cap \mathcal{S}_K(h) = \mathcal{S}_K(g).
\]


\begin{claim}
\label{not_conjugate_power}
$\alpha_m$ is not conjugate to $g^k$ for any integer $k$. 
\end{claim}

\begin{proof}
Assume for a contradiction that 
$g^{p+pm-1} s g^{-p+1} s^{-1}$ is conjugate to $g^{\ell_m}$ for some integer $\ell_m$.
Then 
$p_s(g^{p+pm-1} s g^{-p+1} s^{-1}) $ is also conjugate to $p_s(g)^{\ell_m}$ in $\pi_1(K(s))$. 
Since 
\[
p_s(g^{p+pm-1} s g^{-p+1} s^{-1}) = p_s(g)^{p+pm-1} p_s(s) p_s(g)^{-p+1} p_s(s)^{-1} = p_s(g)^{pm}, 
\] 
$p_s(g)^{pm}$ is conjugate to $p_s(g)^{\ell_m}$ in $\pi_1(K(s))$. 

Hence, 
we have the equality 
\begin{align*}
|pm| \mathrm{scl}_{\pi_1(K(s))}(p_s(g)) 
&= 
\mathrm{scl}_{\pi_1(K(s))}(p_s(g)^{pm}) 
= 
\mathrm{scl}_{\pi_1(K(s))}(p_s(g)^{\ell_m}) \\
&=
|\ell_m| \mathrm{scl}_{\pi_1(K(s))}(p_s(g)),
\end{align*}
which implies $\ell_m = \pm  pm$.

Recall that we choose the slope $s$ so that it is a hyperbolic surgery slope, 
and the non-peripheral element $g$ satisfies $p_s(g) \ne 1$. 
Thus by Theorem~\ref{scl_bound}, we have $\mathrm{scl}_{\pi_1(K(s))}(p_s(g)) > 0$. 
 
On the other hand, 
$\varphi(g^{p+pm-1} s g^{-p+1} s^{-1})$ is conjugate to $\varphi(g^{\ell_m}) = \varphi(g^{\pm pm})$. 
By the choice of $p$, 
$\varphi(g^{p+pm-1} s g^{-p+1} s^{-1}) = \varphi(g^{-1} s g s^{-1}) \ne 1$, 
but $\varphi(g^{\pm pm}) = 1$. 
This is a contradiction. 
So $\alpha_m$ is not conjugate to $g^k$ for any integer $k$. 
\end{proof}


\begin{claim}
\label{infinite_conjugacy_m}
There are infinitely many integers $m > 0$ such that 
$g^{p + pm -1} h^{-p+1}$ are mutually non-conjugate elements in $G(K)$. 
\end{claim}

\begin{proof}
Recall that $g$ is a non-peripheral element in $[G(K), G(K)]$, 
and $h = s g s^{-1} \in [G(K), G(K)]$. 
Applying the argument in the proof of Lemma~\ref{infinite_conjugacy}, 
we have a homogeneous quasimorphism $\phi \colon G(K) \rightarrow \mathbb{R}$ 
which satisfies $\varphi(g) > 0$ and 
\[
\phi(g^{p + pm -1} h^{-p+1}) \geq (p + pm -1) \phi(g) + \phi(h^{-p+1}) - D(\phi).
\] 
Hence $\lim_{m\to \infty} \phi(g^m h^{n_0}) \to \infty$. 
Since homogeneous quasimorphism $\phi$ is conjugation invariant, 
this shows that $\{ g^{p + pm -1} h^{-p+1} \}$ has infinitely many mutually non-conjugate elements.
\end{proof}


Claims~\ref{not_conjugate_power} and \ref{infinite_conjugacy_m} complete a proof of Theorem~\ref{non_rigid}.
\end{proof}


\bigskip

\section*{Acknowledgements}
TI has been partially supported by JSPS KAKENHI Grant Number  JP19K03490 and 21H04428. 

KM has been partially supported by JSPS KAKENHI Grant Number JP19K03502, 21H04428 and Joint Research Grant of Institute of Natural Sciences at Nihon University for 2022. 

MT has been partially supported by JSPS KAKENHI Grant Number JP20K03587.

We would like to thank Nathan Dunfield for allowing us to use his class file for TeX. 
\bigskip


\begin{thebibliography}{99}


\bibitem{AFW}
M. Aschenbrenner, S. Friedl and H. Wilton; 
$3$--manifold groups. EMS Series of Lectures in Mathematics. European Mathematical Society, Z\"urich, 2015. xiv+215 pp.

\bibitem{BaumslagSolitar}
G. Baumslag and D. Solitar; 
Some two-generator one-relator non-Hopfian groups, 
Bull.\  Amer.\  Math.\  Soc.\  \textbf{68} (1962), 199--201.


\bibitem{BMT}
G. Baumslag, C. F. Miller III and D. Troeger; 
Reflections on the residual finiteness of one-relator groups, 
Groups Geom.\ Dyn.\ \textbf{1} (2007), 209--219. 

\bibitem{Bavard}
C. Bavard; 
Longeur stable des commutateurs,  
Enseign.\ Math.\ \textbf{37} (1991), 109--150. 

\bibitem{BingMartin}
R. H. Bing and J. Martin; 
Cubes with knotted holes, 
Trans.\ Amer.\ Math.\ Soc. \textbf{155} (1971), 217--231. 


\bibitem{BZ_finite_JAMS}
S. Boyer and X. Zhang; 
Finite Dehn surgery on knots,
J.\ Amer.\ Math.\ Soc.\ \textbf{9} (1996), 1005--1050. 



\bibitem{Cal_GAFA}
D. Calegari; 
Length and stable length, 
Geom.\ Funct.\ Anal.\ \textbf{18} (2008), 50--76.  

\bibitem{Cal_MSJ}
D. Calegari; 
scl, 
Mem.\ Math.\ Soc.\ Japan \textbf{20} (2009). 

\bibitem{CallahanReid}
P. J. Callahan and A. W. Reid; 
Hyperbolic structures on knot complements, 
Chaos, Solitons $\&$ Fractals, \textbf{9} (1998), 705--738. 

\bibitem{CJ}
A. Casson and D. Jungreis; 
Convergence groups and Seifert fibered $3$--manifolds, 
Invent.\ Math.\ \textbf{118} (1994), 441--456.  


\bibitem{CGLS}
M. Culler, C. McA. Gordon, J. Luecke and P.B. Shalen; 
Dehn surgery on knots, 
Ann.\ Math.\ \textbf{125} (1987), 237--300. 

\bibitem{Dehn}
M. Dehn;
\"Uber die Topologie des dreidimensionalen Raumes, 
Math.\  Ann.\  \textbf{69} (1910) 137--168. 

\bibitem{Dehn1911}
M. Dehn;
\"Uber unendliche diskontinuierliche Gruppen, Math.\  Ann.\ \textbf{71} (1911), 116--144.


\bibitem{Dutra}
E. R. F. Dutra; 
On killers of cable knot groups, 
Bull.\ Aust.\ Math.\ Soc.\ \textbf{96} (2017), 171--176. 


\bibitem{GabaiII}
D. Gabai; 
Foliations and the topology of $3$--manifolds. II, 
J.\ Diff.\ Geom.\ \textbf{26} (1987), 461--478. 

\bibitem{GabaiIII}
D. Gabai; 
Foliations and the topology of $3$--manifolds. III, 
J.\ Diff.\ Geom.\ \textbf{26} (1987), 479--536. 

\bibitem{Gabai_Seifert}
D. Gabai; 
Convergence groups are Fuchsian groups, 
Ann.\ Math.\ \textbf{136} (1992), 447--510. 


\bibitem{Ghi}
P. Ghiggini;
Knot Floer homology detects genus-one fibred knots,
Amer.\ J.\ Math.\ \textbf{130} (2008), 1151--1169.

\bibitem{Goda}
H. Goda;
Hyperbolic Volume and Twisted Alexander invariants of Knots and Links
preprint, arXiv1710.09963 

\bibitem{GS} 
F. Gonz\'alez-Acu\~na and H.Short;
Knot surgery and primeness,
Math.\ Proc.\ Camb.\ Phil.\ Soc.\ {\bf 99} (1986), 89--102. 

\bibitem{Gordon}
C. McA. Gordon; 
Dehn filling a survey, 
Knot theory (Warsaw, 1995), Polish Acad. Sci., Warsaw
1998, 129--144.


\bibitem{Go_satellite}
C. McA. Gordon; 
Dehn surgery and satellite knots, 
Trans.\ Amer.\ Math.\ Soc.\ \textbf{275} (1983), 687--708. 


\bibitem{GLu2}
C. McA. Gordon and J. Luecke; 
Knots are determined by their complements,  
J.\ Amer.\ Math.\ Soc.\ \textbf{2} (1989), 371--415. 

\bibitem{GLreducible}
C. McA. Gordon and J. Luecke;
Reducible manifolds and Dehn surgery, 
Topology \textbf{35} (1996), 385--409.



\bibitem{Hat}
A. E. Hatcher; 
Notes on basic $3$-manifold topology, 
freely available at \texttt{http://www.math.cornell.edu/hatcher}. 


\bibitem{Hem}
J. Hempel; 
$3$--Manifolds, 
Ann.\ of Math.\ Studies, vol.\ 86, Princeton University Press, Princeton, NJ, 1976.

\bibitem{Hem_residual_finite}
J. Hempel;
Residual finiteness for $3$--manifolds, 
in: Combinatorial Group Theory and Topology, pp.379--396, 
Ann.\ of Math.\ Studies, vol.\ 111, Princeton University Press, Princeton, NJ, 1987.

\bibitem{HJ}
J. Hempel and W. Jaco; 
Fundamental groups of $3$--manifolds which are extensions,  
Ann. of Math. \textbf{95} (1972), 86--98.


\bibitem{How}
J. Howie; 
A proof of the Scott-Wiegold conjecture on free products of cyclic groups, 
J.\ Pure Appl.\ Algebra \textbf{173} (2002), 167--176. 



\bibitem{IchiMT}
K. Ichihara, K. Motegi and M. Teragaito; 
Vanishing nontrivial elements in a knot group by Dehn fillings, 
Topology Appl.\ \textbf{264} (2019), 223--232.

\bibitem{IMT_Magnus}
T. Ito, K. Motegi and M. Teragaito; 
Nontrivial elements in a knot group which are trivialized by Dehn fillings, 
Int. Math. Res. Not. IMRN, \textbf{2021} (2021), 8297--8321. 

\bibitem{IMT_decomposition}
T. Ito, K. Motegi and M. Teragaito; 
Generalized torsion and decomposition of $3$--manifolds, 
Proc.\ Amer.\ Math.\ Soc.\ \textbf{147} (2019), 4999--5008. 


\bibitem{JS}
W. Jaco and P. Shalen; 
Seifert fibered spaces in $3$--manifolds, 
Mem.\ Amer.\ Math.\ Soc., \textbf{21} (220): viii+192,
1979.


\bibitem{Jo}
K. Johannson,
\textit{Homotopy equivalences of $3$--manifolds with boundaries},
Lecture Notes in Mathematics, 761. Springer, Berlin, 1979.

\bibitem{Juh}
A. Juh\'asz; 
Floer homology and surface decompositions, 
Geom.\ Topol.\ \textbf{12} (2008), 299--350.


\bibitem{KM}
P. Kronheimer and T. Mrowka; 
Witten's conjecture and Property P, 
Geom.\ Topol.\ \textbf{8} (2004), 295--310. 


\bibitem{LM}
M. Lackenby and R. Meyerhoff;
The maximal number of exceptional Dehn surgeries,
Invent.\ Math.\ \textbf{191} (2013), no. 2, 341--382. 



\bibitem{MKS}
W. Magnus, A. Karrass and D. Solitar; 
Combinatorial group theory: presentations of groups in terms of generators and relations, 
Dover books on mathematics, Courier Corporation, 2004. 


\bibitem{MM_crossing}
K. Miyazaki and K. Motegi; 
Crossing change and exceptional Dehn surgery, 
Osaka J.\ Math.\ \textbf{39} (2002), 773--777.


\bibitem{Mostow}
G. D. Mostow; 
Quasi-conformal mappings in n-space and the rigidity of hyperbolic space forms, 
Publ.\ IHES \textbf{34} (1968), 53--104. 

\bibitem{Mo}
K. Motegi; 
Dehn fillings on knot groups, 
Mathematisches Forschungsinstitut Oberworfach, 
Oberworfach Reports, 
Eur.\ Math.\ Soc.\ 
\textbf{17} (2020).
DOI: 10.4171/OWR/2020/8


\bibitem{Ni}
Y. Ni; 
Knot Floer homology detects fibred knots, 
Invent.\ Math.\  \textbf{170} (2007) 577--608. 


\bibitem{Ni-Zhang-Finite}
Y. Ni and X. Zhang;
Finite Dehn surgery on knots in $S^{3}$,
Algebr.\ Geom.\ Topol.\ \textbf{18} (2018), 441--492. 



\bibitem{Pe1}
G. Perelman; 
The entropy formula for the Ricci flow  and  its  geometric  applications,
2002, arXiv:math.DG/0211159 

\bibitem{Pe2}
G. Perelman;  
Ricci flow with surgery on three-manifolds, 
2003, arXiv:math.DG/0303109 

\bibitem{Pe3}
G. Perelman;  
Finite extinction time for the solutions to the Ricci flow on certain three-manifolds, 
2003, arXiv:math.DG/0307245

\bibitem{Prasad}
G. Prasad; 
Strong rigidity of $\mathbb{Q}$-rank 1 lattices, 
 Invent. Math. \textbf{21} (1973), 255--286.


\bibitem{Sch_JDG}
M. Scharlemann; 
Sutured manifolds and generalized Thurston norms, 
J.\ Diff.\ Geom.\  \textbf{29} (1989) 557--614.


\bibitem{SWW}
D. Silver, W. Whitten and S. Williams; 
Knot groups with many killers, 
Bull.\ Aust.\ Math.\ Soc.\ \textbf{81} (2010), 507--513. 


\bibitem{Sim2}
J. Simon;
Roots and centralizers of peripheral elements in knot groups,
Math.\ Ann.\ \textbf{222} (1976), 205--209.


\bibitem{Te_weight}
M. Teragaito;
Weight elements of the knot groups of some three-strand pretzel knots, 
Bull.\ Aust.\ Math.\ Soc.\ \textbf{98} (2018), 305--318.

\bibitem{T1}
W. P. Thurston; 
Geometry and topology of three-manifolds, 
(Lecture notes, Princeton University, 1979.) 
MSRI, 
freely available at \texttt{http://library.msri.org/books/gt3m/}

\bibitem{T2}
W. P. Thurston; 
Three dimensional manifolds, Kleinian groups and hyperbolic geometry, 
Bull.\ Amer.\ Math.\ Soc.\ \textbf{6} (1982), 357--381. 



\bibitem{Tsau}
C. M. Tsau; 
Non algebraic killers of knot groups, 
Proc.\ Amer.\ Math.\ Soc.\ \textbf{95} (1985), 139--146. 

\bibitem{Tsau2}
C. M. Tsau; 
Isomorphisms and peripheral structure of knot groups,
Math.\ Ann. \textbf{282} (1988), 343--348. 


\bibitem{Wu}
Y.-Q. Wu; 
Cyclic surgery and satellite knots, 
Topology Appl.\ \textbf{36} (1990), 205--208. 


\end{thebibliography}
 

\end{document}





