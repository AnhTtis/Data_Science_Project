\section{Background and Related Work}\label{sec:related}

Our work builds upon the following past foundations, but our focus differs from each.

\subsection{Spatio-Temporal Traffic Forecasting}\label{subsec:st traffic forecasting}

Traffic forecasting is a well-established research topic with a wide range of proposed solutions. Classical statistical methods, such as historical average, regression, and integrated moving average models, have been explored in the past \cite{Box1970}. However, more recent research has leveraged DNNs to model the spatio-temporal correlations in traffic data. Compared to classical statistical methods, DNNs can better capture complex relationships in historical data while avoiding the need for hand-engineered features. Researchers have explored several approaches to represent traffic data, including temporal sequence modeling with recurrent neural networks~\cite{8614060}, multidimensional matrix representations with convolutional neural networks \cite{8526506}, and graph neural networks \cite{jiang2021dl}. DNN-based methods have been shown to deliver state-of-the-art results in various traffic-related tasks, such as ride-sharing \cite{9254149}, and travel planning \cite{CHEN2022126060}. Due to the better performance over alternative methods, DNNs have emerged as the dominant approach for building traffic forecasting models.

The majority of existing DNN-based traffic forecasting models only provide a single-point prediction such as the travel time. However, a single-point estimation only reflects the average traffic scenario but not the best or worse cases.
As highlighted in Section \ref{sec:motivation}, the upper and lower bounds of the travel time can be critical for choosing the best route in time-critical route planning tasks. This requires one to consider the uncertainty of the predicted travel time and to produce metrics similar to statistical confidence intervals. This work aims to address this issue by developing a generic approach that can provide such information from any DNN model, making it applicable to a wide range of traffic forecasting architectures.

\subsection{Modeling Prediction Uncertainty}\label{subsec:uncertainty}
Although uncertainty modeling has been largely overlooked in prior traffic forecasting approaches, it has drawn much attention in other DNN-based modeling tasks. Various techniques have been proposed to quantify prediction errors, confidences, or uncertainties. These methods can be broadly categorized as Bayesian and Frequentist ones, which have been extensively studied in the literature \cite{arXiv2021_03342, pmlr-v80-pearce18a}.

\cparagraph{Bayesian models} provide a robust probabilistic framework for modeling uncertainty with Bayesian statistics \cite{journals/neco/MacKay92a}. In this approach, the model incorporates prior knowledge or beliefs for parameter initialization and infers the posterior distribution using the likelihood function between the data and a predefined initial distribution. Techniques for quantifying uncertainty in DNNs using Bayesian models include Monte Carlo (MC) dropout \cite{pmlr-v48-gal16} and Variational Inference \cite{ Blundell2015weight}. However, there are emerging challenges, such as relatively low computation efficiency and strong prior distribution assumptions. These challenges can be particularly acute in high-dimensional models or large datasets.

\cparagraph{Frequentist methods} provide predictions based on a single forward pass with a deterministic network and quantify uncertainty by using additional qualification schemes. These methods use post-hoc calibrations, such as conformal prediction and differentiable modeling structures, and their loss objectives, such as quantile prediction \cite{nips_ChungNCS21}, to capture uncertainties. Ensemble methods, such as those that use random initialization or a mixture of experts \cite{nips_Lakshminarayanan17}, retrain models on partial datasets, or adopt data augmentation techniques (e.g., cross-validation \cite{iclr_RitterBB18}, and bootstrap aggregating \cite{aaai_HuangZH19}), are also considered frequentist methods. However, ensemble methods require trial-and-error adjustments to parameters without a solid mathematical foundation, leading to poor coverage guarantee. Moreover, frequentist methods are often overconfident \cite{nips_ChungNCS21, stankeviciute2021conformal}, which can result in inaccurate uncertainty estimates. Despite these limitations, frequentist methods are attractive because they are computationally efficient and do not require prior assumptions on the model or data distribution.

\subsection{Summary}

In summary, existing methods for quantifying uncertainty suffer from several issues, including inaccurate coverage guarantees, strong distributional assumptions, and insufficiently calibrated prediction intervals. These challenges are further compounded when dealing with heteroscedastic data\footnote{Heteroscedastic data refers to data that has varying levels of variability or scatter across its range, as opposed to homoscedastic data which has consistent levels of variability or scatter. An example of heteroscedastic data in traffic forecasting is rush hour traffic, which typically has more variability than traffic during off-peak times.}.

To address these issues, we propose an adaptive conformalized quantile model that provides a unified and reliable framework for quantifying uncertainty in traffic forecasting. Our approach is one of the first attempts to use frequentist methods for estimating uncertainty in traffic forecasting. We provide a comprehensive and structured comparison of existing approaches on real traffic data, using a variety of state-of-the-art DNN-based traffic forecasting models.
We hope our work  can promote more research in this important area of uncertainty quantification for traffic forecasting. 