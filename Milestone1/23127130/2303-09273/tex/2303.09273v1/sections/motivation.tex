\vspace{-4mm}
\section{Motivation\label{sec:motivation}}
\vspace{-2mm}
\begin{figure}
    \centering
    \includegraphics[width=0.98\linewidth]{fig/intro_0222.pdf}
    \caption{\textbf{A representative emergency route planning example between two points.} A single-point travel time estimation does not provide the upper-bound travel time for route selection which is important for ensuring the worst-case arrival time.}
    \label{fig:motivation}
\end{figure}

As a motivative example, consider the emergency route planning task depicted in \cref{fig:motivation}. This example represents a day-to-day scenario where traffic forecasts are employed to plan optimal routes, like an ambulance traveling from Regional Medical Center to Santa Clara Valley Medical Center. The objective is to identify the quickest route based on the predicted traffic while ensuring the worst-case arrival time.

Typical traffic forecasting models can provide estimates for travel time of different routes, as seen in \cref{fig:motivation}. However, relying solely on these single-point forecasts for route planning can lead to unreliable decisions since they lack information about the uncertainty or confidence of the predictions. In the example shown, a user may choose Route 1 due to its lowest average travel time (13 minutes). Nevertheless, this decision overlooks the potential variability in traffic dynamics and route complexity, possibly rendering a worse real travel time. To account for this variability and provide worst-case scenario estimates, it is essential to have confidence intervals associated with the travel time predictions. Incorporating such confidence estimation can have a significant impact on time-critical route planning tasks, which is overlooked in the current literature.

We show how estimating the uncertainty of travel time predictions can provide valuable information for time-critical route planning. By providing travel time bounds besides single-point forecasts, we can tell that selected routes have much higher variability in their predicted travel times than others. If the uncertainty estimation is accurate, the additional information help users avoid routes that are more uncertain on travel times, even if their average predicted travel time is lower. For example, in the scenario depicted in \cref{fig:motivation}, one may finally choose Route 3 despite it being the longest, in order to ensure the worst-case arrival time.

This example demonstrates the limitations of single-point-based traffic forecasting models for time-critical travel planning. This highlights the need for uncertainty modeling in traffic forecasting. To address this, our work proposes a generic framework for precise single-point and bound estimations to better model traffic prediction uncertainties.

