% \usepackage{lipsum} % Just for the demo, you can safely remove it.
\usepackage{enumitem}
% \usepackage{float}
\usepackage{multicol,multienum}
\usepackage{breqn}
\usepackage{stfloats}
\usepackage{multirow}
\usepackage{diagbox}
\usepackage{hyperref}
\usepackage{bm}
\usepackage{cite}
\usepackage{graphicx}
\usepackage{tabularx}
\usepackage[ruled,vlined]{algorithm2e}
\usepackage{algorithmic}
\usepackage{booktabs}
\usepackage{amsthm}
\newtheorem{lemma}{Lemma}
\newtheorem{theorem}{Theorem}
\theoremstyle{definition}
\newtheorem{definition}{Definition}
\newtheorem{Problem}{Problem}
% \usepackage[tight,footnotesize]{subfigure}
% \usepackage{url}
% \usepackage{doi}
\usepackage{amssymb}
\usepackage{amsmath}
\usepackage[table]{xcolor}
\usepackage{siunitx}
\DeclareSIUnit{\mph}{mph}
\usepackage{mathtools}
\usepackage{tabularx}
\usepackage{xspace}

\usepackage{cleveref}
\crefname{section}{Section}{Sec.}
\crefname{figure}{Figure}{Fig.}
\crefname{table}{Table}{Table.}
\crefname{algorithm}{Algorithm}{Algorithm}
\crefname{equation}{Eq.}{Eq.}

\usepackage{bbding}

% tikz
\usepackage[utf8]{inputenc}
\usepackage{tikz}
\usetikzlibrary{positioning}
\usetikzlibrary{shapes,arrows,arrows,positioning,fit}
\usetikzlibrary{shapes.geometric, arrows}

% % pgfplots
\usepackage[tight,footnotesize]{subfigure}
\usetikzlibrary {matrix}
\usetikzlibrary {datavisualization}

\usepackage{pifont}

\newcommand{\cparagraph}[1]{\vspace{1.5mm} \noindent \textbf{#1}\xspace}
\ifCLASSINFOpdf

\else

\fi

\hyphenation{op-tical net-works semi-conduc-tor}


\newcommand\FIXME[1]{\textcolor{red}{FIX:}\textcolor{red}{#1}}
\newcommand\FIXED[1]{\textcolor{blue}{#1}}
\newcommand\REMINDER[1]{\textcolor{orange}{#1}}
\newcommand\SystemName{\textsc{QuanTraffic}\xspace}

%%%%%%%%%%%%PROOFREADING%%%%%%%%%%%%
\usepackage{etoolbox}
\usepackage{pdfcomment}
\usepackage[normalem]{ulem}
\newcommand{\stkout}[1]{\ifmmode\text{\sout{\ensuremath{#1}}}\else\sout{#1}\fi}
\newcommand{\udline}[1]{\ifmmode\text{\uline{\ensuremath{#1}}}\else\uline{#1}\fi}
\newcommand{\note}[3][]{%
  \ifstrempty{#3}{}{%
    \pdfmarkupcomment[color=yellow!50,markup=Highlight]{#3}{}%
  }%
  \ifstrempty{#2}{}{%
    \pdfcomment[color=yellow!50,author=James]{\ifstrempty{#1}{}{(#1) }#2}%
  }%
}
\newcommand{\fix}[2]{%
  \ifstrempty{#2}{%
    {\color{lightgray}\stkout{#1}}%
  }{%
    \ifstrempty{#1}{%
      {\color{violet}\udline{#2}}%
    }{%
      {\color{lightgray}\stkout{#1}}{\color{violet}\udline{#2}}%
    }%
  }%
}
\newcommand{\notemar}[2][]{%
  \pdfmargincomment[color=yellow!50,opacity=0.8,author=James]{\ifstrempty{#1}{}{(#1) }#2}%
}

\newcommand{\multirowoffset}{-0.5\dimexpr \aboverulesep + \belowrulesep + \cmidrulewidth}
% Usage: \multirow{2}*[\multirowoffset]{}
%%%%%%%%%%ENDPROOFREADING%%%%%%%%%%%%
