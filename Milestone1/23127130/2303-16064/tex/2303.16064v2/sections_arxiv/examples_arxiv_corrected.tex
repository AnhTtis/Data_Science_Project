\section{Numerical Examples}

\label{sec:examples}

% \urg{github link to code}
% MATLAB (2021a) code to generate the below examples is publicly available at
% \url{https://github.com/daishuyu/noise-in-observations}. 
MATLAB (2022a) code to replicate experiments is available at \url{https://github.com/Jarmill/chance_peak}. 
Dependencies include Mosek \cite{mosek92} and YALMIP \cite{lofberg2004yalmip}. 
% \urg{Possibly include mp-yalmip, because I've been getting \texttt{UNKNOWN} statuses from Mosek on some of the problems}. 


% \subsection{Two-Dimensional Flow System}

% \subsection{Deterministic Dynamics}
% % An ODE in which $\mu_0$ is a uniform distribution over a box.
% The Flow system of \cite{prajna2004safety} satisfies,
% \begin{equation}
% \label{eq:flow}
%     \dot{x} = \begin{bmatrix}x_2 \\ -x_1 -x_2 + \frac{1}{3}x^3_1\end{bmatrix}.
% \end{equation}

% This example will include maximization of $p(x) = -x_2$  over trajectories \eqref{eq:flow} when considering the state set $X = [-1, 2] \times [-1, 1.5]$ and time horizon $T = 5$. These trajectories and results are visualized in Figure \ref{fig:det_flow} in cyan. Initial conditions are uniformly distributed ($\mu_0$) in the box $X_0=[0.8, 1.2]^2$ (black box). The $\epsilon=0.5$ row of table \ref{tab:flow_det} displays the bounds on the mean distribution as solved through finite-degree \ac{SDP} truncations of \eqref{eq:peak_meas}. The bounds at $\epsilon = \{0.15, 0.1, 0.05\}$ are obtained through the \ac{VP} expression in \eqref{eq:var_vp} and solving the \acp{SDP} obtained from \eqref{eq:chance_lmi}. The dotted and solid red lines in Figure \ref{fig:det_flow} are the $\epsilon=0.5$ and $\epsilon=0.2$ bounds respectively at order 6. 

% The true maximum value of $p(x) = -x_2$ over when choosing a point in the initial region of $X_0$ (without imposing a uniform distribution) is $0.5570$ (at order-6 of \eqref{eq:peak_meas}).
% The $\epsilon \leq 0.15$ bounds are therefore conservative (from the \ac{VP} approximation) but remain valid.

% \begin{table}[h]
%   \centering
%   \caption{Chance-Peak estimation of the Deterministic Flow System \eqref{eq:flow} \label{tab:flow_det}}
% \begin{tabular}{ccccccc}
% order           & 1      & 2      & 3      & 4      & 5      & 6      \\ \hline
% $\epsilon=0.5$  & 1.0000 & 0.5681 & 0.5426 & 0.5421 & 0.5411 & 0.5396 \\
% $\epsilon=0.15$  & 1.2781 & 0.9998 & 0.9451 & 0.9384 & 0.9373 & 0.9364\\
% $\epsilon=0.1$  & 1.5706 & 1.2523 & 1.1882 & 1.1808 & 1.1794 & 1.1789 \\
% $\epsilon=0.05$ & 2.2287 & 1.8174 & 1.7329 & 1.7245 & 1.7234 & 1.7230 
% \end{tabular}
% \end{table}

% \begin{figure}[h]
%     \centering
%     \includegraphics[width=\linewidth]{img/determinstic_flow_6_top_corner.png}
%     \caption{Trajectories of \eqref{eq:flow} with $\epsilon = \{0.5, 0.2\}$ bounds}
%     \label{fig:det_flow}
% \end{figure}



\subsection{Two States}

Example 1 of \cite{prajna2004stochastic} is the following two-dimensional cubic polynomial \ac{SDE}
\begin{equation}
\label{eq:flow_sde}
    dx = \begin{bmatrix}x_2 \\ -x_1 -x_2 - \frac{1}{2}x^3_1\end{bmatrix}dt + \begin{bmatrix} 0 \\ 0.1 \end{bmatrix}dw.
\end{equation}

This example performs chance-peak maximization of $p(x) = -x_2$ starting at the point (Dirac-delta initial measure $\mu_0$) $X_0 = [1, 1]$ with $X= [-1, 1.4] \times  [-2, 1.25]$ and $T=5$. Trajectories of \eqref{eq:flow_sde} are displayed in cyan in Figure \ref{fig:sde_flow} starting from the black-circle $X_0$, and four of these trajectories are marked in non-cyan colors.
The $\epsilon=0.5$ row of Table \ref{tab:flow_sde} displays the bounds on the mean distribution as solved through finite-degree \ac{SDP} truncations of \eqref{eq:peak_meas}. The bounds at $\epsilon = \{0.15, 0.1, 0.05\}$ are obtained through the \ac{VP} expression in \eqref{eq:var_vp} and solving the \acp{SDP} obtained from \eqref{eq:chance_lmi}.
The dotted and solid red lines in Figure \ref{fig:sde_flow} are the $\epsilon=0.5$ and $\epsilon=0.15$ bounds respectively at order 5.
% \ac{VP} bounds derived from solving the \acp{SDE} \eqref{eq:flow_sde} from \eqref{eq:peak_meas} and \eqref{eq:chance_lmi} are recorded in Table \ref{tab:flow_sde} in the same manner as in Table \ref{tab:flow_det}. Figure \ref{fig:sde_flow} plots trajectories and bounds of \eqref{eq:flow_sde} starting from the black-circle $X_0$ point. Four of these trajectories are visibly distinguished. Just as in Figure \ref{fig:det_flow}, the red dotted line is the $\epsilon=0.5$ bound and the red solid line is the $\epsilon=0.15$ bound.


\begin{table}[!h]
   \centering
   \caption{Chance-Peak estimation of the Stochastic Flow System \eqref{eq:flow_sde} to maximize $p(x) = -x_2$ \label{tab:flow_sde}}
\begin{tabular}{lcccccc}
\multicolumn{1}{c}{order}        & 2      & 3      & 4      & 5      & 6   & \ac{MC}   \\ \hline
$\epsilon = 0.5$ & 0.8818 & 0.8773 & 0.8747 & 0.8745 & 0.8744 & 0.8559\\
$\epsilon = 0.15$ & 1.6660 & 1.6113 & 1.5842 & 1.5771 & 1.5740 & 0.9142\\
$\epsilon = 0.1$  & 2.0757 & 1.9909 & 1.9549 & 1.9461 & 1.9427 & 0.9279\\
$\epsilon = 0.05$  & 2.9960 & 2.8441 & 2.7904 & 2.7772 & 2.7715 &  0.9484 \\
% $\epsilon = 0.5$  & 1.5000 & 0.8817 & 0.8773 & 0.8747 & 0.8745 & 0.8744 \\
% $\epsilon = 0.15$ & 1.2910 & 1.2210 & 1.1817 & 1.1689 & 1.1657 & 1.1642 \\
% $\epsilon = 0.1$  & 1.5811 & 1.5009 & 1.4520 & 1.4361 & 1.4323 & 1.4303 \\
% $\epsilon = 0.05$ & 2.2361 & 2.1306 & 2.0613 & 2.0387 & 2.0332 & 2.0305
\end{tabular}
\end{table}


\begin{figure}[!h]
    \centering
    \includegraphics[width=0.5\linewidth]{img/motion_corr_6.png}
    \caption{Trajectories of \eqref{eq:flow_sde} with $\epsilon=0.5$ (dashed red) and $\epsilon=0.15$ (solid red) bounds}
    \label{fig:sde_flow}
\end{figure}

\begin{table}[!h]
   \centering
   \caption{Solver time (seconds) to compute Table \ref{tab:flow_sde} \label{tab:flow_sde_time}}
\begin{tabular}{lcccccc}
\multicolumn{1}{c}{order}         & 2      & 3      & 4      & 5      & 6      \\ \hline
$\epsilon = 0.5$   & 0.380 & 0.449 & 0.625 & 1.583 & 4.552 \\
$\epsilon = 0.15$  & 0.262 & 0.443 & 0.727 & 2.756 & 5.586 \\
$\epsilon = 0.1$   & 0.268 & 0.380 & 1.364 & 2.882 & 3.143 \\
$\epsilon = 0.05$ & 0.242 & 0.390 & 1.261 & 2.923 & 7.539
\end{tabular}
\end{table}

% \urg{Include an example of distance estimation here}

\subsection{Three States}

\Iac{SDE} modification of the Twist system from \cite{miller2022distance_short} is
\begin{equation}
\label{eq:twist_sde}
    dx = \begin{bmatrix}-2.5x_1 + x_2 - 0.5x_3 + 2x_1^3+2x_3^3 \\
    -x_1+1.5x_2+0.5x_3-2x_2^3-2x_3^3 \\
    1.5 x_1 + 2.5x_2 - 2 x_3 - 2x_1^3 - 2 x_2^3\end{bmatrix}dt + \begin{bmatrix} 0 \\ 0\\  0.1 \end{bmatrix}dw.
\end{equation}
This second example performs chance-peak maximization of $p(x) = x_3$ starting at the point $X_0 = [0.5, 0, 0]$ with $X=[-0.6, 0.6] \times  [-1, 1] \times [-1, 1.5]$ and $T=5$. \ac{VP} bounds from solving the \acp{SDE} from \eqref{eq:peak_meas} and \eqref{eq:chance_lmi} are recorded in Table \ref{tab:twist_sde} in the same manner as in Table \ref{tab:flow_sde}. Figure \ref{fig:sde_twist} plots trajectories and bounds of \eqref{eq:twist_sde} starting from the black-circle $X_0$ point, with four of these trajectories visibly distinguished. The solid red plane in Figure \ref{fig:sde_twist} is the $\epsilon=0.15$ bound on $x_3$ at order 6, and the translucent red plane is the $\epsilon=0.5$ bound on $x_3$ (also at order 6).

% Just as in Figure \ref{fig:det_flow}, the red dotted line is the $\epsilon=0.5$ bound and the red solid line is the $\epsilon=0.15$ bound.



\begin{table}[t]
   \centering
   \caption{Chance-Peak estimation of the Stochastic Twist System \eqref{eq:twist_sde}  to maximize $p(x) = x_3$ \label{tab:twist_sde}}
\begin{tabular}{lcccccc}
\multicolumn{1}{c}{order}               & 2      & 3      & 4      & 5      & 6  & \ac{MC}    \\ \hline
$\epsilon = 0.5$   & 0.9100 & 0.8312 & 0.8231 & 0.8211 & 0.8201 & 0.7206\\
$\epsilon = 0.15$  & 1.6097 & 1.4333 & 1.3545 & 1.3318 & 1.3202 & 0.7685 \\
$\epsilon = 0.1$  & 1.9707 & 1.7453 & 1.6283 & 1.5877 & 1.5739 & 0.7801\\
$\epsilon = 0.05$  & 2.7834 & 2.4426 & 2.2333 & 2.1622 & 2.1267 & 0.7970
\end{tabular}
\end{table}


\begin{figure}[h]
    \centering
    \includegraphics[width=0.5\linewidth]{img/twist_corr_6.png}
    \caption{Trajectories of \eqref{eq:twist_sde} with $\epsilon=0.5$ (dashed red) and $\epsilon=0.15$ (solid red) bounds}
    \label{fig:sde_twist}
\end{figure}

\begin{table}[!h]
   \centering
   \caption{Solver time (seconds) to compute Table \ref{tab:twist_sde} \label{tab:twist_sde_time}}
\begin{tabular}{lcccccc}
\multicolumn{1}{c}{order}           & 2      & 3      & 4      & 5      & 6      \\ \hline
$\epsilon = 0.5$   & 0.428 & 1.939 & 5.196 & 19.201 & 83.679  \\
$\epsilon = 0.15$  & 0.328 & 0.999 & 4.755 & 21.108 & 96.985  \\
$\epsilon = 0.1$  & 0.325 & 1.083 & 5.172 & 22.596 & 119.823 \\
$\epsilon = 0.05$  & 0.325 & 1.294 & 4.516 & 22.357 & 115.820
\end{tabular}
\end{table}


\subsection{Exit-Time}

This example uses the setting of Example 7.4 of \cite{henrion2021moment}.
The dynamics are standard Brownian motion in $n=3$ dimensions with $f=\0_3$ and $g = \1_3$. The initial condition is $X_0 = \0_3$ with a support set of $X = \{x \in \R^3 \mid \sum_{i=1}^3 x_i^4 \leq 1\}$. The considered boundary is $\partial X = \{x \in \R^3 \mid \sum_{i=1}^3 x_i^4 = 1\}$.

Chance-peak bounds for the first arrival time $\inp{t}{\mu_\tau}$ are displayed in Table \ref{tab:exit_brownian}. The order-1 estimates are dual infeasible and are marked by $\infty$.

\begin{table}[h]
   \centering
   \caption{Chance-Peak Exit-Statistic estimation of Standard Brownian Motion System  \label{tab:exit_brownian}}
\begin{tabular}{lcccccc}
\multicolumn{1}{c}{order}           & 1      & 2      & 3      & 4      & 5           \\ \hline
$\epsilon = 0.5$  & $\infty$ & 0.0642 & 0.0642 & 0.0642 & 0.0642    \\
$\epsilon = 0.15$ & $\infty$  & 0.1833 & 0.1375 & 0.1375 & 0.1375    \\
$\epsilon = 0.1$  & $\infty$  & 0.2300 & 0.1614 & 0.1614 & 0.1614    \\
$\epsilon = 0.05$ & $\infty$  & 0.3292 & 0.2113 & 0.2113 & 0.2113                
\end{tabular}
\end{table}

\subsection{Switching}

We utilize a modification of Example C from \cite{prajna2007framework} for this final example. The two subsystems involved are:
\begin{subequations}
\label{eq:switched_sde}
\begin{align}
    dx &= \begin{bmatrix}-2.5 x_1 - 2 x_2 \\ -0.5 x_1 - x_2 \end{bmatrix}dt + \begin{bmatrix}0 \\ 0.25 x_2 \end{bmatrix}dw \\
    dx &= \begin{bmatrix}-x_1 - 2 x_2 \\ 2.5 x_1 - x_2 \end{bmatrix}dt + \begin{bmatrix}0 \\ 0.25 x_2 \end{bmatrix}dw.
\end{align}
\end{subequations}

Switched \ac{SDE} trajectories start from an initial condition of $X_0 = (0, 1)$ and are tracked in the state set $X = [-2, 2]^2$  with a time horizon of $T=5$. The chance-peak problem is solved to find bounds on $p(x) = -x_2$.

Figure \eqref{fig:switched_sde} plots switched \ac{SDE} trajectories along with $\epsilon=\{0.5, 0.15\}$ bounds (at order-6). Table \ref{tab:switched_sde} lists these discovered bounds. 




\begin{table}[!h]
   \centering
   \caption{Chance-Peak upper-bounds for $p(x)=-x_2$ for the Switched System \eqref{eq:switched_sde} \label{tab:switched_sde}}
\begin{tabular}{lcccccc}
\multicolumn{1}{c}{order}           & 1      & 2      & 3      & 4      & 5     & 6      \\ \hline
$\epsilon = 0.5$  & 0.8491 & 0.4304 & 0.3823 & 0.3630 & 0.3487 & 0.3352 \\
$\epsilon = 0.15$ & 1.5613 & 0.9953 & 0.9328 & 0.9076 & 0.8918 & 0.8853 \\
$\epsilon = 0.1$  & 1.9358 & 1.2888 & 1.2162 & 1.1865 & 1.1687 & 1.1609 \\
$\epsilon = 0.05$ & 2.7764 & 1.9469 & 1.8516 & 1.8120 & 1.7891 & 1.7799           
\end{tabular}
\end{table}

\begin{figure}[!h]
    \centering
    \includegraphics[width=0.5\linewidth]{img/lin_switched_corr.png}
    \caption{Trajectories of \eqref{eq:switched_sde} with $\epsilon = \{0.5, 0.15\}$ bounds}
    \label{fig:switched_sde}
\end{figure}

\begin{table}[!h]
   \centering
   \caption{Solver time (seconds) to compute Table \ref{tab:switched_sde} \label{tab:switched_sde_time}}
\begin{tabular}{lcccccc}
\multicolumn{1}{c}{order}      & 1      & 2      & 3      & 4      & 5      & 6      \\ \hline
$\epsilon = 0.5$  & 0.665 & 0.362 & 0.389 & 0.570 & 1.755 & 2.499 \\
$\epsilon = 0.15$ & 0.284 & 0.257 & 0.295 & 0.587 & 1.812 & 3.718 \\
$\epsilon = 0.1$  & 0.222 & 0.237 & 0.281 & 1.636 & 2.364 & 3.191 \\
$\epsilon = 0.05$ & 0.224 & 0.251 & 0.291 & 0.906 & 1.735 & 2.638
\end{tabular}
\end{table}



\subsection{Distance Estimation}

% \urg{may need to delete this, this is wrong}

This example will involve distance estimation of a modification of the second subsystem of \eqref{eq:switched_sde}:
\begin{equation}
\label{eq:flowmod_sde}
    dx = \begin{bmatrix}-x_1 - 2 x_2 \\ 2.5 x_1 - x_2 \end{bmatrix}dt + \begin{bmatrix}0 \\ 0.1 \end{bmatrix}dw.
\end{equation}
This $L_2$ chance-distance  task  takes place at a time horizon of $T=5$ with sets $X_0 = [0; 0.75]$, $X = [-1.25, 1] \times [-1, 1]$, and $X_u = \{y \in \R^2 \mid 0.1^2 \geq (y_1 + 1)^2 + (y_2+1)^2\}$. Distance estimation was accomplished by maximizing \acp{VAR} of the function $-\norm{x-y}^2_2$ in \eqref{eq:var_meas_distance}.

System trajectories of \eqref{eq:flowmod_sde} are displayed in Figure \ref{fig:distance}, in which the unsafe half-circle set $X_u$ is drawn in solid red. Squared distance lower bounds from solving \acp{SDP} arising from moment programs of \eqref{eq:var_meas_distance} are listed in Table \ref{tab:flowmod_distance}. Negative distance lower-bounds are truncated to $0$ in Table \ref{tab:flowmod_distance}. This example demonstrates how chance-peak distance bounds for distance estimation are very conservative, and improving the quality of these bounds is a vital area for future work.

\begin{table}[!h]
   \centering
   \caption{Chance-Peak squared distance lower bounds for System \eqref{eq:flowmod_sde} \label{tab:flowmod_distance}}
\begin{tabular}{lcccccc}
\multicolumn{1}{c}{order}           & 1      & 2      & 3      & 4      & 5       & 6    \\ \hline
$\epsilon = 0.5$  & 0.5667 & 1.1929 & 1.2337 & 1.2425 & 1.2490 & 1.2506 \\
$\epsilon = 0.15$ & 0.0000 & 0.0000 & 0.0000 & 0.0000 & 0.0182 & 0.0235 \\
$\epsilon = 0.1$  & 0.0000 & 0.0000 & 0.0000 & 0.0000 & 0.0000 & 0.0000 \\
$\epsilon = 0.05$ & 0.0000 & 0.0000 & 0.0000 & 0.0000 & 0.0000 & 0.0000                     
\end{tabular}
\end{table}

\begin{figure}[!h]
    \centering
    \includegraphics[width=0.5\linewidth]{img/linear_distance_conservative.png}
    \caption{Trajectories of \eqref{eq:flowmod_sde} with $\epsilon = \{0.5, 0.15\}$ bounds}
    \label{fig:distance}
\end{figure}

\begin{table}[!h]
   \centering
   \caption{Solver time (seconds) to compute Table \ref{tab:flowmod_distance} \label{tab:flowmod_distance_time}}
\begin{tabular}{lcccccc}
\multicolumn{1}{c}{order}      & 1      & 2      & 3      & 4      & 5      & 6      \\ \hline
$\epsilon = 0.5$  & 0.761 & 0.507 & 0.512 & 1.772 & 6.569 & 21.331 \\
$\epsilon = 0.15$ & 0.361 & 0.346 & 0.453 & 1.233 & 5.836 & 23.930 \\
$\epsilon = 0.1$  & 0.314 & 0.344 & 0.482 & 1.522 & 5.172 & 21.034 \\
$\epsilon = 0.05$ & 0.321 & 0.384 & 0.485 & 1.711 & 5.954 & 26.974
\end{tabular}
\end{table}

