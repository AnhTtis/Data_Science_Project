
This paper has the following structure: 
Section \ref{sec:preliminaries} gives an overview of notation, \acp{SDE}, and occupation measures. Section \ref{sec:meas} upper-bounds the chance-peak problem using an  infinite-dimensional \ac{SOCP} in  occupation measures, and analyzes the duality properties of this \ac{SOCP}.
Section \ref{sec:lmi} reviews the moment-\ac{SOS} hierarchy and presents a hierarchy of \acp{SDP} that approximate the infinte-dimensional chance-peak \ac{SOCP}. 
Section \ref{sec:extensions} details extensions to the chance-peak framework: analysis of exit-time statistics, distance of closest approach, and switching processes.
Section \ref{sec:examples} provides numerical examples of the chance-peak problem on  \ac{SDE} systems. Section \ref{sec:conclusion} concludes the paper.
Appendix \ref{app:duality_general} proves strong duality properties for a class of measure programs with linked semidefinite constraints. Appendix \ref{app:duality_chance} applies this general strong duality proof to the chance-peak \ac{SOCP}.
% \urg{Fill in the paper structure}
% Section \ref{sec:preliminaries} will review preliminaries such as notation, notions of stability for linear systems, and \ac{SOS} proofs of polynomial nonnegativity. Section \ref{sec:full_method} will present 
% The paper is concluded in Section \ref{sec:conclusion}.