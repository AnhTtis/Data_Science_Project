
\section{Tail-Bound Program}
\label{sec:tail_bound}

This section solves the chance-peak problem \ref{prob:quantile} using tail-bounds by developing an infinite-dimensional \ac{SOCP} in measures.

\subsection{Tail-Bound Problem}

% Let $\textrm{VT}_\epsilon(V)$ refer to the Cantelli or \ac{VP} tail-bound expression in \eqref{eq:tail_bounds} as appropriate (ensuring that the \ac{VP} bound is used if and only if its $\epsilon \leq 1/6$ and unimodality conditions are satisfied). 

We define $r$ as the constant factor multiplying against $\sigma$ in the Cantelli or \ac{VP} expression of \eqref{eq:tail_bounds} as in
\begin{align}
\label{eq:tail_bounds_constant}
    {r_C} &= \sqrt{(1/\epsilon)-1} & {r_{VP}} &=  \sqrt{4/(9\epsilon)-1},
\end{align}
ensuring that the \ac{VP} bound is used if and only if its $\epsilon \leq 1/6$ and unimodality conditions are satisfied. We also use the notation $\inp{p^2}{\mu_{t^*}}$ to denote the expectation {$\mathbb{E}[p(x(t^*))^2]$}.
% Using the notation $\inp{p^2}{\mu_{t^*}}$ to denote the expectation {$\mathbb{E}[p(x(t^*))^2]$}, 
\begin{prob}
    The tail-bound program upper-bounding \eqref{eq:peak_chance} with constant $r$ is
\begin{subequations}
\label{eq:peak_chance_tail}
\begin{align}
    P^*_r = & \sup_{t^* \in [0, T]} r\sqrt{\inp{p^2}{\mu_{t^*}} - \inp{p}{\mu_{t^*}}^2}  + \inp{p}{\mu_{t^*}} \label{eq:peak_chance_tail_obj}\\     
    & \text{$x(t)$ follows $\Lie$ from $t=0$ until } \ \tau_X \wedge t^* \label{eq:peak_chance_tail_stop}\\
     & x(0) \sim \mu_0.
\end{align}
\end{subequations}
\end{prob}





\subsection{Nonlinear Measure Program}

Problem \eqref{eq:peak_chance_tail} can be converted into an infinite-dimensional nonlinear program in variables $(\mu_\tau, \mu)$, given the initial distribution $\mu_0$ and the generator $\Lie$:
\begin{subequations}
\label{eq:peak_chance_meas}
\begin{align}
    p^*_r = & \sup r\sqrt{\inp{p^2}{\mu_{\tau}} - \inp{p}{\mu_{\tau}}^2}  + \inp{p}{\mu_{\tau}} \label{eq:peak_chance_meas_obj}\\
     & \mu_\tau = \delta_0 \otimes \mu_0 + \Lie^\dagger \mu \label{eq:peak_chance_meas_lie}\\
    & \mu_\tau, \ \mu \in \Mp{[0, T] \times X}. \notag
\end{align}
\end{subequations} 



\begin{thm}
\label{thm:upper_bound_nonlinear}
Programs \eqref{eq:peak_chance_meas} and \eqref{eq:peak_chance_tail} are related by $p^*_r \geq P^*_r$ under assumptions A1, A2, A4.
% Program \eqref{eq:peak_chance_meas} is an upper bound on \eqref{eq:peak_chance_tail} with $p^*_r \geq P^*_r$ under A1-A4.
\end{thm}
\begin{proof}
Let $t^* \in [0, T]$ be a terminal time with stopping time of $\tau^* = t^* \wedge \tau_X$. We can construct measures $(\mu, \mu_\tau)$ to satisfy \eqref{eq:peak_chance_meas_lie} from the data in $(t^*, \Lie, \mu_0)$. Specifically, let $\mu$ be the occupation measure of the stochastic process $\Lie$ starting from $\mu_0$ until time $\tau^*$, and let $\mu_\tau$ be the time-$\tau^*$ state distribution. The allowed stopping times $t^* \in [0, T]$ for \eqref{eq:peak_chance_tail} maps in a one-to-one manner on the feasible set of constraints \eqref{eq:peak_chance_meas}, affirming that $p^*_r \geq P^*_r$.
% Let $t^*$ be a stopping time in $[0, T]$, and let $x_0 \in X_0$ be an initial condition. Measures $(\mu_0, \mu, \mu_\tau)$ that satisfy \eqref{eq:peak_chance_meas_lie} may be constructed from this $(t^*, x_0)$ by $\mu_{t^*}$ as the the state distribution of  \eqref{eq:sde_sol} at time $t^*$ given $\mu_0$, and $\mu$ as the occupation measure in \eqref{eq:avg_free_occ} associated to this \ac{SDE} trajectory with distribution $\mu_0$. Because the feasible set to constraint \eqref{eq:peak_chance_meas_lie} contains measures induced by all possible provided \ac{SDE} trajectories starting from $\mu_0$, it holds that $p^*_r \geq P^*_r$.
\end{proof}

\begin{rmk}
A worst-case tail-bound $p^*_r$ for \eqref{eq:peak_chance_meas} can be found from an initial set $X_0 \subseteq X$ by letting the initial condition $\mu_0$ be  an optimization variable with the constraints $\mu_0 \in \Mp{X_0}$ and $\inp{1}{\mu_0} = 1$.
% The initial distribution $\mu_0\in \Mp{X_0}$ may be optimized to find a supremal $p^*_r$ over all probability distributions in $X_0$ by adding $\mu_0$ as a variable and adding the constraint $\inp{\mu_0}{1} = 1$ to .
\end{rmk}

\begin{lem}
\label{lem:no_relaxation_gap}
Under assumptions (A1, A2, A4-7), to every $(\mu_\tau, \mu)$ obeying \eqref{eq:var_meas_lie_soc}  there exists a stochastic process trajectory \eqref{eq:peak_chance_tail_stop} starting at $\mu_0$ and terminating at $\mu_\tau$ with an occupation measure of $\mu$.
\end{lem}
\begin{proof}
    This lemma holds by Theorem 3.3 of \cite{cho2002linear} (under A1, A2, A5-7; noting that $[0, T] \times X$ is a complete metric space).
\end{proof}

\begin{cor}
    The combination of Theorem \ref{thm:upper_bound_nonlinear} and Lemma \ref{lem:no_relaxation_gap} implies that $p^*_r = P^*_r$ under assumptions A1-A7.
\end{cor}


\subsection{Measure Second-Order Cone Program}

This subsection will demonstrate how the  nonlinear program in measures \eqref{eq:peak_chance_meas} may be equivalently expressed as an infinite-dimensional convex measure \ac{SOCP}. We will use an \ac{SOC} representation of the square-root to accomplish this conversion through the following lemma:

% The nonlinear measure program \eqref{eq:peak_chance_meas} may be recast as an infinite-dimensional convex \ac{SOCP}.% with a finite-dimensional \ac{SOC} constraint.
\begin{lem}
\label{lem:sqrt}
Define the objective \eqref{eq:peak_chance_meas_obj} as $J_r(a, b) = r \sqrt{b - a^2} + a$ under the substitutions $a=\inp{p}{\mu_\tau}$ and $b = \inp{p^2}{\mu_\tau}$. Given any convex set 
${K} \in \R \times \R_+$ with $(a, b) \in {K}$, the subsequent programs will possess the same optimal value:
\begin{align}
    &\sup_{(a, b) \in {K}} a + r \sqrt{b-a^2}  \\
    &\sup_{(a, b) \in {K}, \ {c} \in \R} a + r { \, c}: \ ([1-b, 2{c}, 2a],  1+b) \in {\mathbb{L}}^3. \label{eq:square_root_socp}
\end{align}

% Let $J_r(a, b) = r \sqrt{b - a^2} + a$ be the objective  \eqref{eq:peak_chance_meas_obj} with $a=\inp{p}{\mu_\tau}$ and $b = \inp{p^2}{\mu_\tau}$. For any convex set ${K} \in \R \times \R_+$ with $(a, b) \in {K}$, the following pair of programs have the same optimal value (in which ${\mathbb{L}}^3 = \{([y_1, y_2, y_3], s) \in \R^3 \times \R_+ \mid \norm{y}_2 \leq s \}$  is \iac{SOC} cone):
% \begin{align}
%     &\sup_{(a, b) \in {K}} a + r \sqrt{b-a^2}  \\
%     &\sup_{(a, b) \in {K}, \ {c} \in \R} a + r { \, c}: \ ([1-b, 2{c}, 2a],  1+b) \in {\mathbb{L}}^3. \label{eq:square_root_socp}
% \end{align}
\end{lem}
\begin{proof} We introduce a new variable $c$ as in $c \leq \sqrt{b-a^2}$, which implies that ${c}^2 + a^2 \leq b $. The \ac{SOC}-equivalent form of $\sqrt{b-a^2}$ follows from \cite{alizadeh2003second, yalmip2009sqrt},
% The new variable ${c}$ is introduced under the constraint $\sqrt{b-a^2} \geq {c}$, implying that $ {c}^2 + a^2 \leq b $. The \ac{SOCP} equivalence follows from the power-representation of $\sqrt{b-a^2}$ from \cite{alizadeh2003second, yalmip2009sqrt}, with the steps of
\begin{align*}
    &([1 - b, 2{c}, 2a],  1+b) \in {\mathbb{L}}^3  \\
    \Longleftrightarrow \; & (1-b)^2 + 4({c}^2 + a^2) \leq (1+b)^2 \\
    \Longleftrightarrow \; & (1+b^2) - 2b + 4({c}^2 + a^2) \leq (1+b^2) + 2b \\
    \Longleftrightarrow \; & 4({c}^2 + a^2) \leq 4b.
\end{align*}
\end{proof}

\begin{thm}
\label{thm:upper_bound_socp}
The nonlinear program \eqref{eq:peak_chance_meas} has the same set of feasible solutions and  optimal value as the following \ac{SOCP} (given $(\mu_0, \Lie, r)$):
% An infinite-dimensional \ac{SOCP} with the same optimal value and set of feasible solutions as \eqref{eq:peak_chance_meas} given $\mu_0$ is
\begin{subequations}
\label{eq:var_meas_soc}
\begin{align}
    p^*_{r} = &\sup \quad r {c}  + \inp{p}{\mu_\tau} \label{eq:var_meas_obj_soc}\\
     & \mu_\tau = \delta_0 \otimes \mu_0 + \Lie^\dagger \mu \label{eq:var_meas_lie_soc}\\
    & {y} = [1-\inp{p^2}{\mu_\tau}, \ 2 {c}, \ 2 \inp{p}{\mu_\tau}] \label{eq:var_meas_con_soc_def}\\
    & ({y}, 1 + \inp{p^2}{\mu_\tau}) \in {\mathbb{L}}^3 \label{eq:var_meas_con_soc}\\
    & \mu, \ \mu_\tau \in \Mp{[0, T] \times X}, u \in \R, c \in \R^3.\notag
\end{align}
\end{subequations}
\end{thm}
\begin{proof}
This equivalence follows from Lemma \ref{lem:sqrt} by replacing the square-root in the objective \eqref{eq:peak_chance_meas_obj}. The new optimization variables are $(\mu_\tau, \mu, u, c)$.
% This results from an application of Lemma \ref{lem:sqrt} to the objective term \eqref{eq:peak_chance_meas_obj}. The optimization variables are now $(\mu_\tau, \mu, u, z)$. %{Matt: wrong. $s$ is also an optimization variable. The best way to write our problem in a standard way would be under the form
%\begin{align*}
%    \sup & \langle c , x \rangle & \qquad && \inf & \langle y , b \rangle \\
%    & x \in K & \qquad && & A'y - c \in K^* \\
%    & Ax = b \in Y & \qquad && & y \in Y^*
%\end{align*}
%Here it yields (please double check all my computations)
%\begin{align*}
%    p^*_r = & \sup rz + \langle p, \mu_\tau\rangle & \qquad && d^*_r = & \inf \langle v(0,\cdot) , \mu_0 \rangle + y_1 + y_4 \\
%    & \mu \in \Mp{[0,T]\times X} & \qquad && & \Lie v \leq 0 \\
%    & \mu_\tau \in \Mp{[0,T]\times X}& \qquad && & v + y_1 p^2 - 2y_3 p - y_4 p^2 \geq p \\
%    & z\in\R & \qquad && & -2 y_2 = r \\
%    & q \in Q^3 & \qquad && & y \in Q^3 \\
%    & \mu_\tau - \Lie^\dagger \mu = \delta_0 \mu_0 & \qquad && & v \in C^2([0,T]\times X) \\
%    & q + [\langle p^2,\mu_\tau\rangle, -2z, -2\langle p, \mu_\tau \rangle, -\langle p^2,\mu_\tau \rangle] = [1,0,0,1] & \qquad && & y \in \R^4
%\end{align*}}
\end{proof}

\begin{cor}
The \ac{SOCP} in \eqref{eq:var_meas_soc} is convex.
\end{cor}
\begin{proof}
% The objective \eqref{eq:var_meas_obj_soc} is affine in $({c}, \mu_\tau)$.
Constraints \eqref{eq:var_meas_lie_soc}-\eqref{eq:var_meas_con_soc} are convex (\ac{SOC} for \eqref{eq:var_meas_con_soc} and affine for \eqref{eq:var_meas_lie_soc}). The objective  \eqref{eq:var_meas_obj_soc} is linear in $({c}, \mu_\tau)$.
% Constraints \eqref{eq:var_meas_lie_soc}-\eqref{eq:var_meas_con_soc} are convex (affine for \eqref{eq:var_meas_lie_soc} and \ac{SOC} for \eqref{eq:var_meas_con_soc}), ensuring convexity of \eqref{eq:var_meas_soc}.
\end{proof}

% \begin{rmk}
% Problem \eqref{eq:var_meas_soc} has an infinite-dimensional affine constraint in \eqref{eq:var_meas_lie_soc} and a finite-dimensional \ac{SOC} constraint in \eqref{eq:var_meas_con_soc}.
% \end{rmk}
% \begin{thm}
% Program \eqref{eq:peak_chance_meas} is convex
% \end{thm}
% \begin{proof}
% The affine constraints \eqref{eq:peak_chance_meas_lie} and \eqref{eq:peak_chance_meas_mass} are linear in the measures $(\mu_0, \mu, \mu_\tau)$. The objective \eqref{eq:peak_chance_meas_obj} may be represented using a function $h_r$
% \begin{equation}
%     h_r(x, y) = r \sqrt{y - x^2} + x, \label{eq:h_func}
% \end{equation}
% as $h_r(\inp{p (x)}{\mu_{\tau}}, \inp{p (x)^2}{\mu_{\tau}})$. The function $h_r(x, y)$ is concave when $y\geq 0$, and is strictly concave when $y>0$ and $r>0$. This strict concavity may be deduced from the positive-definite Hessian of $h_r$,
% \begin{equation}
%     \nabla^2 h(x, y; k) = \frac{1}{(y-x^2)^{3/2}}\begin{bmatrix} -kx^2 - k(y-x^2) & kx/2 \\ kx/2 & -k/4 \end{bmatrix}.
% \end{equation}
% \end{proof}


% \end{prob}
% In the same way that \eqref{eq:peak_meas}


% Presentation of Cantelli and VP bounds. Utilization of these bounds to define upper-bounds on the $1-\epsilon$ quantile in terms of a Measure program.


% Reformulation of the Cantelli/VP constraints into a Second-Order-Cone constraint.
