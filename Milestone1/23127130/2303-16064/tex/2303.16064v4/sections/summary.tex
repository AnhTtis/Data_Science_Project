
This paper has the following structure: 
Section \ref{sec:preliminaries} gives an overview of notation, \acp{SDE}, and occupation measures. Section \ref{sec:meas} upper-bounds the chance-peak problem using an  infinite-dimensional \ac{SOCP} in  occupation measures.
% Section \ref{sec:duality} provides a dual perspective on this \ac{SOCP} utilizing auxiliary functions \urg{if room allows}. 
Section \ref{sec:lmi} reviews the moment-\ac{SOS} hierarchy and presents a hierarchy of \acp{SDP} that approximate the infinte-dimensional chance-peak \ac{SOCP}. Section \ref{sec:examples} provides numerical examples of the chance-peak problem on \ac{ODE} and \ac{SDE} systems. Section \ref{sec:conclusion} concludes the paper.
% \urg{Fill in the paper structure}
% Section \ref{sec:preliminaries} will review preliminaries such as notation, notions of stability for linear systems, and \ac{SOS} proofs of polynomial nonnegativity. Section \ref{sec:full_method} will present 
% The paper is concluded in Section \ref{sec:conclusion}.