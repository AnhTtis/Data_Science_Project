\section{Numerical Examples}

\label{sec:examples}
% \urg{Add CVAR content. Revise all tables with bug-free results (from thesis/arxiv). Compare against CVAR bounds.}

% \urg{
All experiments were written in MATLAB (2022a)  and require Mosek \cite{mosek92} and YALMIP \cite{lofberg2004yalmip} dependencies. \ac{MC} sampling  is conducted with 50,000 sample paths (with \ac{SDE} parameters of antithetic sampling and a spacing of $\Delta t = 10^{-3}$) to approximate \ac{VAR} and \ac{CVAR} estimates. All experiments are accompanied by tables of chance-peak bounds and solver times.
Files to generate examples are available at \url{https://github.com/Jarmill/chance\_peak} (tail-bound Cantelli/\ac{VP}) and \url{https://github.com/Jarmill/cvar\_peak} (\ac{CVAR}). 
% }

% MATLAB (2022a) code to replicate experiments is available at \url{https://github.com/Jarmill/chance\_peak} (tail-bound Cantelli/\ac{VP}) and \url{https://github.com/Jarmill/cvar\_peak} (\ac{CVAR}). 
% Dependencies include Mosek \cite{mosek92} and YALMIP \cite{lofberg2004yalmip}.  \ac{MC} sampling to empirically find \ac{VAR} and \ac{CVAR} estimates is conducted over 50,000 \ac{SDE} paths under antithetic sampling with a time spacing of $\Delta t = 10^{-3}$. All experiments contain a table of chance-peak bounds as well as solver-times to compute these bounds.
% \label{sec:examples}

% \urg{github link to code}
% MATLAB (2021a) code to generate the below examples is publicly available at
% \url{https://github.com/daishuyu/noise-in-observations}. 

%MATLAB (2022a) code to replicate experiments is available at \url{https://github.com/Jarmill/chance_peak}. 
%Dependencies include Mosek \cite{mosek92} and YALMIP \cite{lofberg2004yalmip}. All instances of `mean' bounds in refer to the expectation-maximization (non chance-peak) program \eqref{eq:peak_meas} (or its appropriate modification). \ac{MC} sampling to empirically find \ac{VAR} and \ac{CVAR} estimates is conducted over 50,000 \ac{SDE} paths under antithetic sampling with a time spacing of $\Delta t = 10^{-3}$. All experiments contain a table of chance-peak bounds as well as solver-times to compute these bounds.
% \urg{Possibly include mp-yalmip, because I've been getting \texttt{UNKNOWN} statuses from Mosek on some of the problems}. 


% \subsection{Two-Dimensional Flow System}

% \subsection{Deterministic Dynamics}
% % An ODE in which $\mu_0$ is a uniform distribution over a box.
% The Flow system of \cite{prajna2004safety} satisfies,
% \begin{equation}
% \label{eq:flow}
%     \dot{x} = \begin{bmatrix}x_2 \\ -x_1 -x_2 + \frac{1}{3}x^3_1\end{bmatrix}.
% \end{equation}

% This example will include maximization of $p(x) = -x_2$  over trajectories \eqref{eq:flow} when considering the state set $X = [-1, 2] \times [-1, 1.5]$ and time horizon $T = 5$. These trajectories and results are visualized in Figure \ref{fig:det_flow} in cyan. Initial conditions are uniformly distributed ($\mu_0$) in the box $X_0=[0.8, 1.2]^2$ (black box). The  row of table \ref{tab:flow_det} displays the bounds on the mean distribution as solved through finite-degree \ac{SDP} truncations of \eqref{eq:peak_meas}. The bounds at $\epsilon = \{0.15, 0.1, 0.05\}$ are obtained through the \ac{VP} expression in \eqref{eq:var_vp} and solving the \acp{SDP} obtained from \eqref{eq:chance_lmi}. The dotted and solid red lines in Figure \ref{fig:det_flow} are the mean and $\epsilon=0.2$ bounds respectively at order 6. 

% The true maximum value of $p(x) = -x_2$ over when choosing a point in the initial region of $X_0$ (without imposing a uniform distribution) is $0.5570$ (at order-6 of \eqref{eq:peak_meas}).
% The $\epsilon \leq 0.15$ bounds are therefore conservative (from the \ac{VP} approximation) but remain valid.

% \begin{table}[h]
%   \centering
%   \caption{Chance-Peak estimation of the Deterministic Flow System \eqref{eq:flow} \label{tab:flow_det}}
% \begin{tabular}{ccccccc}
% order           & 1      & 2      & 3      & 4      & 5      & 6      \\ \hline
% mean  & 1.0000 & 0.5681 & 0.5426 & 0.5421 & 0.5411 & 0.5396 \\
% $\epsilon=0.15$  & 1.2781 & 0.9998 & 0.9451 & 0.9384 & 0.9373 & 0.9364\\
% $\epsilon=0.1$  & 1.5706 & 1.2523 & 1.1882 & 1.1808 & 1.1794 & 1.1789 \\
% $\epsilon=0.05$ & 2.2287 & 1.8174 & 1.7329 & 1.7245 & 1.7234 & 1.7230 
% \end{tabular}
% \end{table}

% \begin{figure}[h]
%     \centering
%     \includegraphics[width=\exfiglength]{img/determinstic_flow_6_top_corner.png}
%     \caption{Trajectories of \eqref{eq:flow} with $\epsilon = \{0.5, 0.2\}$ bounds}
%     \label{fig:det_flow}
% \end{figure}



\subsection{Two States}
The first experiment is a cubic polynomial \ac{SDE} from Example 1 of \cite{prajna2004stochastic}: 
\begin{equation}
\label{eq:flow_sde}
    dx = \begin{bmatrix}x_2 \\ -x_1 -x_2 - \frac{1}{2}x^3_1\end{bmatrix}dt + \begin{bmatrix} 0 \\ 0.1 \end{bmatrix}dW.
\end{equation}


Chance-peak maximization of $p(x) = -x_2$ will occur starting at the point (Dirac-delta initial measure $\mu_0$) $X_0 = [1, 1]$ in a state set of $X= [-1, 2] \times  [-1, 1.5]$ and time horizon of $T=5$. Figure \ref{fig:sde_flow} displays trajectories of \eqref{eq:flow_sde} in cyan, starting from the black-circle point $X_0$. Four of these trajectory sample paths are marked in non-cyan colors.


The `mean' row of Tables \ref{tab:flow_sde} and \ref{tab:flow_sde_cvar} display upper-bounds on the mean as solved by \ac{SDP} truncations of \eqref{eq:peak_meas}. 
The bounds at $\epsilon = \{0.15, 0.1, 0.05\}$ for Table \ref{tab:flow_sde} are acquired by using the \ac{VP} expression in \eqref{eq:var_vp} and solving the \acp{SDP} obtained from \eqref{eq:chance_lmi}.  Table \ref{tab:flow_sde_cvar} displays bounds from \acp{SDP} derived from the \ac{CVAR} \ac{LMI} \eqref{eq:cvar_lmi}.
The dash-dot red, dotted black, and solid red lines in Figure \ref{fig:sde_flow} are the mean, $\epsilon=0.15$ \ac{CVAR}, and $\epsilon=0.15$ \ac{VP} bounds respectively at order 6. All subsequent plots will retain this coloring and styling scheme for the mean, \ac{CVAR}, and \ac{VP} bounds. The top-right entry of Table \ref{tab:flow_sde_cvar} (and all similar tables) are omitted to reduce confusion between the $\epsilon=0.5$ VAR bound (which equals the mean) and the $\epsilon=0.5$ \ac{CVAR} bound (which can exceed the mean).
% \ac{VP} bounds derived from solving the \acp{SDE} \eqref{eq:flow_sde} from \eqref{eq:peak_meas} and \eqref{eq:chance_lmi} are recorded in Table \ref{tab:flow_sde} in the same manner as in Table \ref{tab:flow_det}. Figure \ref{fig:sde_flow} plots trajectories and bounds of \eqref{eq:flow_sde} starting from the black-circle $X_0$ point. Four of these trajectories are visibly distinguished. Just as in Figure \ref{fig:det_flow}, the red dotted line is the mean bound and the red solid line is the $\epsilon=0.15$ bound.

% \begin{table}[h]
%   \centering
%   \caption{Chance-Peak estimation of the Stochastic Flow System \eqref{eq:flow_sde} to maximize $p(x) = -x_2$ \label{tab:flow_sde}}
% \begin{tabular}{lcccccc}
% \multicolumn{1}{c}{order}      & 1      & 2      & 3      & 4      & 5      & 6      \\ \hline
% mean  & 1.5000 & 0.8817 & 0.8773 & 0.8747 & 0.8745 & 0.8744 \\
% $\epsilon = 0.15$ & 1.2910 & 1.2210 & 1.1817 & 1.1689 & 1.1657 & 1.1642 \\
% $\epsilon = 0.1$  & 1.5811 & 1.5009 & 1.4520 & 1.4361 & 1.4323 & 1.4303 \\
% $\epsilon = 0.05$ & 2.2361 & 2.1306 & 2.0613 & 2.0387 & 2.0332 & 2.0305
% \end{tabular}
% \end{table}

\begin{table}[h]
   \centering
   \caption{\ac{VP} Chance-Peak estimation of the Stochastic Flow System \eqref{eq:flow_sde} to maximize $p(x) = -x_2$ \label{tab:flow_sde}}
\begin{tabular}{lcccccc}
\multicolumn{1}{c}{order}        & 2      & 3      & 4      & 5      & 6   & \ac{VAR} \ac{MC}   \\ \hline
$\epsilon = 0.5$ & 0.8818 & 0.8773 & 0.8747 & 0.8745 & 0.8744 & 0.8559\\
$\epsilon = 0.15$ & 1.6660 & 1.6113 & 1.5842 & 1.5771 & 1.5740 & 0.9142\\
$\epsilon = 0.1$  & 2.0757 & 1.9909 & 1.9549 & 1.9461 & 1.9427 & 0.9279\\
$\epsilon = 0.05$  & 2.9960 & 2.8441 & 2.7904 & 2.7772 & 2.7715 &  0.9484 \\
% $\epsilon = 0.5$  & 1.5000 & 0.8817 & 0.8773 & 0.8747 & 0.8745 & 0.8744 \\
% $\epsilon = 0.15$ & 1.2910 & 1.2210 & 1.1817 & 1.1689 & 1.1657 & 1.1642 \\
% $\epsilon = 0.1$  & 1.5811 & 1.5009 & 1.4520 & 1.4361 & 1.4323 & 1.4303 \\
% $\epsilon = 0.05$ & 2.2361 & 2.1306 & 2.0613 & 2.0387 & 2.0332 & 2.0305
\end{tabular}
\end{table}


\begin{figure}
    \centering
    \includegraphics[width=0.9\exfiglength]{img/motion_cvar_6_corr.png}
    \caption{Trajectories of \eqref{eq:flow_sde} with mean (dash-dot red), \ac{CVAR} $\epsilon=0.15$  (dotted black), and \ac{VP} $\epsilon=0.15$ (solid red) bounds}
    \label{fig:sde_flow}
\end{figure}


\begin{table}[!h]
   \centering
   \caption{Solver time (seconds) to compute Table \ref{tab:flow_sde} \label{tab:flow_sde_time}}
\begin{tabular}{lccccc}
\multicolumn{1}{c}{order}      & 2      & 3      & 4      & 5      & 6      \\ \hline
$\epsilon = 0.5$   & 0.380 & 0.449 & 0.625 & 1.583 & 4.552 \\
$\epsilon = 0.15$  & 0.262 & 0.443 & 0.727 & 2.756 & 5.586 \\
$\epsilon = 0.1$   & 0.268 & 0.380 & 1.364 & 2.882 & 3.143 \\
$\epsilon = 0.05$ & 0.242 & 0.390 & 1.261 & 2.923 & 7.539
\end{tabular}
\end{table}

\begin{table}[!h]
   \centering
   \caption{\ac{CVAR} Chance-Peak estimation of the Stochastic Flow System \eqref{eq:flow_sde} to maximize $p(x) = -x_2$ \label{tab:flow_sde_cvar}}
\begin{tabular}{lcccccc}
\multicolumn{1}{c}{order}      & 2      & 3      & 4      & 5      & 6  & \ac{CVAR} \ac{MC}     \\ \hline
mean                      & 0.8818                & 0.8773                & 0.8747                & 0.8745                & 0.8744                &                                            \\
$\epsilon = 0.15$         & 1.2500                & 1.2500                & 1.1655                & 1.1313                & 1.1170                & 0.9432
                                 \\
$\epsilon = 0.1$          & 1.2500                & 1.2500                & 1.2116                & 1.1666                & 1.1466                & 
    0.9546
                                                \\
$\epsilon = 0.05$         & 1.2500                & 1.2500                & 1.2500                & 1.2266                & 1.1959                &    0.9720                                           

\end{tabular}
\end{table}

\begin{table}[!h]
   \centering
   \caption{Solver time (seconds) to compute Table \ref{tab:flow_sde_cvar} \label{tab:flow_sde_cvar_time}}
\begin{tabular}{lccccc}
\multicolumn{1}{c}{order}      & 2      & 3      & 4      & 5      & 6      \\ \hline
mean                      & 0.619                 & 0.504                 & 0.558                 & 2.862                 & 1.652                 \\
$\epsilon = 0.15$         & 0.363                 & 0.402                 & 0.399                 & 1.341                 & 2.016                 \\
$\epsilon = 0.1$          & 0.357                 & 0.442                 & 0.417                 & 1.054                 & 2.093                 \\
$\epsilon = 0.05$         & 0.414                 & 0.393                 & 0.399                 & 0.579                 & 2.534                

\end{tabular}
\end{table}

% \begin{table}[h]
%   \centering
%   \caption{Solver time (seconds) to compute Table \ref{tab:flow_sde} \label{tab:flow_sde_time}}
% \begin{tabular}{lcccccc}
% \multicolumn{1}{c}{order}      & 1      & 2      & 3      & 4      & 5      & 6      \\ \hline
% mean  & 0.643 & 0.417 & 0.431 & 1.963 & 1.659 & 3.641 \\
% $\epsilon = 0.15$ & 0.272 & 0.216 & 0.325 & 1.592 & 4.178 & 7.464 \\
% $\epsilon = 0.1$  & 0.264 & 0.213 & 0.316 & 1.651 & 1.339 & 4.225 \\
% $\epsilon = 0.05$ & 0.184 & 0.222 & 0.366 & 0.936 & 2.446 & 5.298
% \end{tabular}
% \end{table}
\subsection{Three States}

A stochastic version of the Twist dynamics in \cite{miller2022distance_short} is:

\begin{equation}
\label{eq:twist_sde}
    dx = \begin{bmatrix}-2.5x_1 + x_2 - 0.5x_3 + 2x_1^3+2x_3^3 \\
    -x_1+1.5x_2+0.5x_3-2x_2^3-2x_3^3 \\
    1.5 x_1 + 2.5x_2 - 2 x_3 - 2x_1^3 - 2 x_2^3\end{bmatrix}dt + \begin{bmatrix} 0 \\ 0\\  0.1 \end{bmatrix}dW.
\end{equation}

% This second example performs chance-peak maximization of $p(x) = x_3$ starting at the point $X_0 = [0.5, 0, 0]$ with $X=[-0.6, 0.6] \times  [-1, 1] \times [-1, 1.5]$ and $T=5$. \ac{VP} and \ac{CVAR} bounds from solving the \acp{SDP} from \eqref{eq:peak_meas} and \eqref{eq:chance_lmi} are recorded in Tables \ref{tab:twist_sde} and \ref{tab:twist_sde_cvar} in the same manner as in Table \ref{tab:flow_sde} and \ref{tab:flow_sde_cvar}. Figure \ref{fig:sde_twist} plots trajectories and bounds of \eqref{eq:twist_sde} starting from the black-circle $X_0$ point, with four of these trajectories visibly distinguished. The solid red plane in Figure \ref{fig:sde_twist} is the $\epsilon=0.15$ \ac{VP} bound on $x_3$ at order 6, the translucent black plane is the $\epsilon=0.15$ \ac{CVAR} bound at order 6,  and the translucent red plane is the mean bound on $x_3$ (also at order 6).

This example applies the chance-peak setting towards maximization of $p(x) = x_3$, with an initial condition of $X_0 = [0.5, 0, 0]$ and a state set $X=[-0.6, 0.6] \times  [-1, 1] \times [-1, 1.5]$ and $T=5$. 
\ac{VP} and \ac{CVAR} bounds from the \eqref{eq:peak_meas} and \eqref{eq:chance_lmi} \acp{SDP} are written in Tables \ref{tab:twist_sde} and \ref{tab:twist_sde_cvar}, similar in format to Tables \ref{tab:flow_sde} and 
\ref{tab:flow_sde_cvar}. Trajectories and  bounds are plotted in \ref{fig:sde_twist} beginning from the black-circle $X_0$. The three planes are order-6 bounds at $\epsilon = 0.15$, in which the top solid red plane is the \ac{VP} bound, the translucent black plane is the $\epsilon=0.15$ \ac{CVAR} bound at order 6,  and the translucent red plane is the mean bound on $x_3$.


% Just as in Figure \ref{fig:det_flow}, the red dotted line is the mean bound and the red solid line is the $\epsilon=0.15$ bound.



\begin{table}[!h]
   \centering
   \caption{\ac{VP} Chance-Peak estimation of the Stochastic Twist System \eqref{eq:twist_sde}  to maximize $p(x) = x_3$ \label{tab:twist_sde}}
\begin{tabular}{lcccccc}
\multicolumn{1}{c}{order}            & 2      & 3      & 4      & 5      & 6     & \ac{VAR} \ac{MC} \\ \hline
$\epsilon = 0.5$   & 0.9100 & 0.8312 & 0.8231 & 0.8211 & 0.8201 & 0.7206\\
$\epsilon = 0.15$  & 1.6097 & 1.4333 & 1.3545 & 1.3318 & 1.3202 & 0.7685 \\
$\epsilon = 0.1$  & 1.9707 & 1.7453 & 1.6283 & 1.5877 & 1.5739 & 0.7801\\
$\epsilon = 0.05$  & 2.7834 & 2.4426 & 2.2333 & 2.1622 & 2.1267 & 0.7970
\end{tabular}
\end{table}

% \begin{table}[t]
%   \centering
%   \caption{Chance-Peak estimation of the Stochastic Twist System \eqref{eq:twist_sde}  to maximize $p(x) = x_3$ \label{tab:twist_sde}}
% \begin{tabular}{lcccccc}
% \multicolumn{1}{c}{order}           & 1      & 2      & 3      & 4      & 5      & 6      \\ \hline
% mean             & 1.4682                 & 0.9100                 & 0.8312                 & 0.8231                 & 0.8211                 & 0.8203             \\
% $\epsilon=0.15$             & 1.2910                 & 1.1729                 & 1.0813                 & 1.0171                 & 0.9891                 & 0.9755              \\
% $\epsilon=0.1$             & 1.5811                 & 1.4358                 & 1.3180                 & 1.2217                 & 1.1859                 & 1.1694               \\
% $\epsilon=0.05$            & 2.2361                 & 2.0288                 & 1.8497                 & 1.6866                 & 1.6299                 & 1.5894 
% \end{tabular}
% \end{table}

\begin{figure}[!h]
    \centering
    \includegraphics[width=0.9\exfiglength]{img/twist_cvar_6.png}
    \caption{Trajectories of \eqref{eq:twist_sde} with mean \ac{VP} (solid red), mean \ac{CVAR} (translucent black), and $\epsilon=0.15$ (translucent red) bounds}
    \label{fig:sde_twist}
\end{figure}

% \begin{table}[!h]
%   \centering
%   \caption{Solver time (seconds) to compute Table \ref{tab:twist_sde} \label{tab:twist_sde_time}}
% \begin{tabular}{lcccccc}
% \multicolumn{1}{c}{order}      & 1      & 2      & 3      & 4      & 5      & 6      \\ \hline
% mean  & 0.734 & 0.526 & 1.880 & 5.213 & 28.269 & 81.736  \\
% $\epsilon = 0.15$ & 0.316 & 0.397 & 1.117 & 3.876 & 23.302 & 132.539 \\
% $\epsilon = 0.1$  & 0.144 & 0.210 & 1.080 & 6.313 & 27.829 & 116.733 \\
% $\epsilon = 0.05$ & 0.118 & 0.185 & 1.352 & 4.999 & 20.863 & 137.833
% \end{tabular}
% \end{table}

\begin{table}[!h]
   \centering
   \caption{Solver time (seconds) to compute Table \ref{tab:twist_sde} \label{tab:twist_sde_time}}
\begin{tabular}{lccccc}
\multicolumn{1}{c}{order}         & 2      & 3      & 4      & 5      & 6      \\ \hline
$\epsilon = 0.5$   & 0.428 & 1.939 & 5.196 & 19.201 & 83.679  \\
$\epsilon = 0.15$  & 0.328 & 0.999 & 4.755 & 21.108 & 96.985  \\
$\epsilon = 0.1$  & 0.325 & 1.083 & 5.172 & 22.596 & 119.823 \\
$\epsilon = 0.05$  & 0.325 & 1.294 & 4.516 & 22.357 & 115.820
\end{tabular}
\end{table}


%     0.7923
    % 0.8016
    % 0.8156
\begin{table}[!h]
   \centering
   \caption{\ac{CVAR} Chance-Peak estimation of the Stochastic Twist System \eqref{eq:twist_sde}  to maximize $p(x) = x_3$ \label{tab:twist_sde_cvar}}
\begin{tabular}{lcccccc}
\multicolumn{1}{c}{order}            & 2      & 3      & 4      & 5      & 6     & \ac{CVAR} \ac{MC} \\ \hline
mean                      & 0.9100                & 0.8312                & 0.8231                & 0.8211                & 0.8203                &                             \\
$\epsilon = 0.15$         & 1.4519                & 1.1251                & 1.0246                & 0.9892                & 0.9733                &     0.7923
           \\
$\epsilon = 0.1$          & 1.5850                & 1.1880                & 1.0613                & 1.0173                & 0.9950                &    
    0.8016
             \\
$\epsilon = 0.05$         & 1.8479                & 1.3063                & 1.1286                & 1.0646                & 1.0329                &     0.8156

\end{tabular}
\end{table}


\begin{table}[!h]
   \centering
   \caption{Solver time (seconds) to compute Table \ref{tab:twist_sde_cvar} \label{tab:twist_sde_time_cvar}}
\begin{tabular}{lccccc}
\multicolumn{1}{c}{order}         & 2      & 3      & 4      & 5      & 6      \\ \hline
mean                      & 0.761                 & 0.502                 & 1.845                 & 5.078                 & 27.543                \\
$\epsilon = 0.15$         & 0.386                 & 0.469                 & 0.996                 & 4.383                 & 35.634                \\
$\epsilon = 0.1$          & 0.330                 & 0.381                 & 1.030                 & 4.115                 & 20.513                \\
$\epsilon = 0.05$         & 0.328                 & 0.387                 & 1.014                 & 5.451                 & 26.280               

\end{tabular}
\end{table}