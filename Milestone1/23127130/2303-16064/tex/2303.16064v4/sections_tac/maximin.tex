\subsection{Maximin Optimization}

The \ac{CVAR}-maximin problem involves a set of $N_p$ objectives $p_k \in C(X)$ for $k \in 1..N_p$. \ac{CVAR}-maximin replaces the objective of \eqref{eq:peak_cvar_obj} with
\begin{align}
        P^* = & \sup_{t^* \in [0, T] }  \min_{k\in 1..N_p} CVaR_{\epsilon}((p_k)_\# \mu_{t^*}).
\end{align}

Using the maximin framework of Section IV \cite{miller2020recovery} for \acp{ODE}, we supremize a new variable $q \in \R$ such that $q$ lower bounds all of the individual \acp{CVAR} terms $CVaR_{\epsilon}((p_k)_\# \mu_{t^*})$. Variables $(\psi_k, \hat{\psi}_k) \in \Mp{\R}$ are created for each \ac{CVAR} term according to \eqref{eq:cvar_dom}. The resultant maximin \ac{CVAR} program (based on \cite[Equation (20)]{miller2020recovery})  is
\begin{subequations}
\label{eq:peak_cvar_maximin_meas}
\begin{align}
p^* = & \ \sup_{q \in \R} \quad q \label{eq:peak_cvar_maximin_meas_obj} \\
    & \mu_\tau = \delta_0 \otimes\mu_0 + \textstyle \Lie^\dagger \mu \label{eq:peak_cvar_maximin_meas_flow}\\
    & q \leq \inp{\omega}{\psi_k} &  & \forall k\in 1..N_p \\
    & \inp{1}{\psi_k} = 1 & & \forall k\in 1..N_p \label{eq:peak_cvar_maximin_meas_prob}\\
    & \epsilon \psi_k + \hat{\psi_k} = (p_k)_\# \mu_\tau  & & \forall k\in 1..N_p\label{eq:peak_cvar_maximin_meas_cvar}\\
    & \mu_\tau, \ \mu \in \Mp{[0, T] \times X} \label{eq:peak_cvar_maximin_meas_peak}\\    
    & \psi_k, \hat{\psi}_k \in \Mp{\R} & & \forall k\in 1..N_p.\label{eq:peak_cvar_maximin_meas_slack}
\end{align}
\end{subequations}


% \urg{An SDE has independent, stationary increments and also has continuous paths. L\'{e}vy processes remove the requirement that sample paths need to be continuous. The resultant paths can jump a finite number of times in a finite range according to a Poisson point process, assuming that the distribution of the jump heights satisfy a boundedness constraint (L\'{e}vy measure). More information about L\'{e}vy processes are available at \cite{applebaum2004levy}.

% The work in \cite{kashima2010optimization} implemented peak-expectation estimation for \levy processes using (tempered) polynomial optimization. We can use Cantelli/VP bounds to do chance-peak on \levy processes.

% The tempering involved using exponential weights of the 1d state variable when defining all measures, and approximating an exponential*polynomial functions from below by polynomials using a discretezation scheme. It will be exceedingly difficult to generalize the tempering scheme to multiple dimensions due to the curse of dimensionality.

% Depending on time constraints, I might just do arbitrary switching between subsystems and leave it at that.

% } 