\subsection{CVAR Moment Program}


Let $(\bm, \bm^\tau, {\mathbf{n}}, \hat{{\mathbf{n}}})$ be respective moment sequences of the measures $(\mu, \mu_\tau, {\nu, \hat{\nu}})$. 
We define the operator $\mathcal{E}_{k}(\bm^\tau, {\mathbf{n}}, \hat{{\mathbf{n}}})$ for ${k} \in \N$ as the moment counterpart of the operator in constraint \eqref{eq:peak_cvar_meas_cvar}:
\begin{align}
\label{eq:cvar_mom}
    \mathcal{E}_{k}(\bm^\tau, {\mathbf{n}}, \hat{{\mathbf{n}}}) = L_{\bm^\tau}(p(x)^{k}) - \epsilon {\mathbf{n}}_{k} - \hat{{\mathbf{n}}}_{k}.
\end{align}

The associated \ac{CVAR} degree ${\Delta}$ is
\begin{align}
    {\Delta} = \lfloor d/ \deg p \rfloor. 
\end{align}


%The expression $\mathcal{D}_{\alpha \beta}$ exactly involves the moments up to degree $2d$.


\begin{prob}
For each degree $d \geq \deg p$, the order-$d$ moment \ac{LMI} that upper-bounds the \ac{CVAR} \ac{LP} \eqref{eq:peak_cvar_meas} given the distribution  $\mu_0$ is
\begin{subequations}
\label{eq:cvar_lmi}
\begin{align}
    p^*_{c, d} = & \max L_{{\mathbf{n}}} q  \label{eq:cvar_lmi_obj} \\
    & \bm \in \R^{^{\binom{2D+n+1}{n+1}}}, \bm^\tau \in \R^{^{\binom{2d+n+1}{n+1}}} \notag \\
    & {\mathbf{n}}\in \R^{2{\Delta}+1} , \  \hat{{\mathbf{n}}}\in \R^{2{\Delta}+1} \notag \\
    & \forall (\alpha, \beta) \in \N^{n+1} \quad \text{s.t.} \quad |\alpha|+\beta: \leq 2d  \nonumber \\
    &\qquad \mathcal{D}_{\alpha \beta}(\bm, \bm^\tau) = \delta_{\beta 0} \langle x^\alpha , \mu_0\rangle  \qquad \label{eq:cvar_lmi_flow}\\
    & \mathcal{E}_{k}(\bm^\tau, {\mathbf{n}}, \hat{{\mathbf{n}}}) = 0 \qquad \forall {k} \in 0..(2{\Delta}) \label{eq:cvar_lmi_cvar}\\
    &{\mathbf{n}}_0 = 1 \\
    &\M_d[([0, T] \times X)\bm^\tau]\succeq  0\label{eq:lmitauocc}\\
    &\M_D[([0, T] \times X)\bm] \succeq  0  \\
    & \M_{\Delta}[{[\hat{p},\check{p}]\mathbf{n}}] \succeq 0 \label{eq:lmicvar1}\\
    & \M_{\Delta}[{[\hat{p},\check{p}]\mathbf{n}}] \succeq 0 \label{eq:lmicvar2}.
\end{align}
\end{subequations}
\end{prob}
The symbol $\delta_{\beta 0}$ is the Kronecker Delta ($1$ if $\beta = 0$ and $0$ otherwise). Constraints \eqref{eq:cvar_lmi_flow}  and \eqref{eq:cvar_lmi_cvar} are finite-dimensional truncations of constraints \eqref{eq:peak_cvar_meas_flow} 
 and \eqref{eq:peak_cvar_meas_cvar}  respectively.

\begin{rmk}
    The redundant support constraints in \eqref{eq:lmicvar1},\eqref{eq:lmicvar2} are added to ensure that the Archimedean condition is satisfied for each moment sequence support.
\end{rmk}

\begin{thm}
\label{thm:bounded_mass_cvar}
    All measures $\mu, \mu_\tau, {\nu, \hat{\nu}}$ are bounded under A1-A4.
\end{thm}
\begin{proof}
    This boundedness will be proved by showing that the mass of each measure is bounded and their support sets are compact.

    
    
    \textit{Compactness:} The measures $\mu_\tau, \mu$ have compact support under A1. The quantities ${\hat{p}} = \min_{x \in X} p(x)$ and ${\check{p}} = \max_{x \in X} p(x)$ are each finite and attained under A1 and A3. The measure $p_\# \mu_\tau$ has support inside the compact set $[{\hat{p}}, {\check{p}}]$ given that $\supp{\mu_\tau} \subseteq [0, T] \times X$. Both ${\nu}$ and ${\hat{\nu}}$ are nonnegative measures with $\epsilon > 0$, so \eqref{eq:peak_cvar_meas_cvar} {ensures} that $\supp{\nu} \subseteq [{\hat{p}}, {\check{p}}]$ and $ \supp{\hat{\nu}} \subseteq [{\hat{p}}, {\check{p}}]$.

    \textit{Bounded Mass:} The mass of  ${\nu}$ is set to 1 by \eqref{eq:peak_cvar_meas_prob}. Passing a test function $v(t, x) = 1$ through \eqref{eq:peak_cvar_meas_flow} results in $\inp{1}{\mu_\tau} = \inp{1}{\mu_0}$, which equals 1 by A4. Substitution of $v(t, x) = t$ through the same constraint yields $0 + \inp{1}{\mu} = \inp{t}{\mu_\tau} \leq T$. Given that $\mu_\tau$ is a probability measure, it holds that $p_\# \mu_\tau$ is also a probability measure. Finally, we analyze the mass of {$\hat\nu$ with} \eqref{eq:peak_cvar_meas_cvar}: 
    %\begin{subequations}
    %\begin{align*}
    %    \inp{1}{p_\#\mu_\tau} &= \epsilon \inp{1}{\nu} + \inp{1}{\hat{\nu}} 
    %    \\
    %    1 &= \epsilon + \inp{1}{\hat{\nu}}  \\
    %    1-\epsilon &= \inp{1}{\hat{\nu}}.
    %\end{align*}
    {
    \begin{align*}
        \inp{1}{\hat{\nu}} &= \inp{1}{p_\#\mu_\tau - \epsilon \nu}
        \\
        &= \inp{1}{p_\#\mu_\tau} - \epsilon\inp{1}{\nu}\\
        &= 1-\epsilon < \infty.
    \end{align*}
    }
    %\end{subequations}
    All nonnegative measures have compact support and bounded masses under A1-A4, proving the theorem.
\end{proof}

\begin{thm}
\label{thm:cantelli_cvar_relation}
    The Cantelli $(r = \sqrt{1-1/\epsilon})$ program from \eqref{eq:var_meas_soc} with objective $p^*_r$ will upper-bound the \ac{CVAR} program \eqref{eq:peak_cvar_meas} by $p^*_r \geq p^*_c$.
\end{thm}
\begin{proof}
    This holds by Equation (5) of \cite{vcerbakova2006worst}. For a general univariate {random variable $V \sim \psi \in \Mp{\R}$} with variance $\sigma = \sqrt{\mathbb{E}[V^2]-\mathbb{E}[V]^2}$, the Cantelli bound is related to \ac{CVAR} by
    \begin{align} \label{eq:cantelli_cvar_relation}
        {\mathbb{E}[V]} + \sigma \sqrt{1/\epsilon - 1} \geq { \mathrm{ES}_\epsilon(V) \geq \mathrm{VaR}_\epsilon(V)}.
    \end{align}

    The Cantelli bound obtains the worst-case \ac{CVAR} among all possible distributions agreeing with the given first and second moments \cite{dupacova1987minimax}.
\end{proof}


 \begin{thm}
     The upper bounds of \eqref{eq:cvar_lmi} will satisfy $p^*_d \geq p^*_{d+1} \geq ... \geq p^*$ and $\lim_{d \rightarrow \infty} p^* = P^*$ under assumptions A1-A10.
 \end{thm}
 \begin{proof}
     Boundedness and convergence will occur through Corollary 8 of \cite{tacchi2022convergence} (all sets are Archimedean, all data is polynomial, measure solutions are bounded by Theorem \eqref{thm:bounded_mass_cvar}, and the true objective value is finite).
 \end{proof}

 % \begin{rmk}
     
 %     Even though the Cantelli program upper-bounds the \ac{CVAR} program by Theorem 
 %     \ref{thm:cantelli_cvar_relation}, their degree-$d$ moment-\ac{SOS} tightenings may not obey boundedness relations at the same fixed degree $d$.
 % \end{rmk}

