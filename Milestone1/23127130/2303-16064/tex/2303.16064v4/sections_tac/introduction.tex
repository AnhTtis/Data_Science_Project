
\section{Introduction}
\label{sec:introduction}

The behavior of stochastic processes can be interpreted by analyzing the time-evolving distributions of state functions $p(x)$ along trajectories. One such statistic is the $\epsilon$-\ac{VAR} of $p(x)$ with $\epsilon \in [0, 1)$, which also may be cast as the $(1-\epsilon)$-quantile statistic of $p(x)$ \cite{jorion2000value}. Our goal is to find the maximum \ac{VAR} obtained by $p(x)$ along trajectories of a stochastic process within a specified time horizon starting at a given initial condition. An example of this task in the context of aviation is to state that the supremal height of an aircraft with a $\geq 5\%$ chance of exceedence over the course of the flight is 100 meters.
We will refer to the task of upper-bounding the supremal \ac{VAR} along trajectories as the `chance-peak' problem.

The \ac{VAR} itself is a nonconvex and non-subadditive objective that is typically difficult to optimize. Two convex methods of upper-bounding for the \ac{VAR} are tail-bound and \ac{CVAR} approaches. Tail-bounds such as the Cantelli \cite{cantelli1929sui} and \ac{VP} \cite{vysochanskij1980justification} inequalities offer worst-case estimates on the \ac{VAR} of a univariate distribution given its first two moments \cite{dupacova1987minimax}. The \ac{CVAR} \cite{pflug2000some, rockafellar2002conditional} is a coherent risk measure \cite{artzner1999coherent} that returns the average value of a distribution conditioned upon being above the $\epsilon$-\ac{VAR}. The \ac{CVAR}  may be a desired optimization target \cite{sarykalin2008value} in terms of minimizing expected losses, given that the \ac{VAR} is invariant to the distribution shape past the $(1-\epsilon)$ quantile statistic.

Tail-bounds and \ac{CVAR} methods have both been previously applied to problems in control theory.
Tail-bound constraints have been employed for planning in \cite{wang2020non, han2022non}, in which moments of each time-step state-distribution are propagated using forward dynamics to the next discrete-time state.
\ac{CVAR} constraints for optimal control and safety have been used for continuous-time in \cite{cmiller2017optimal}, and for discrete-time Markov Decision Processes via  log-sum-exp \ac{CVAR}-upper-bounds in \cite{chapman2021risk}. We will be using the tail-bound and \ac{CVAR} upper-bounds to solve the chance-peak analysis problem.
% tian2017moment

The chance-peak problem is related to both chance constraints and peak estimation (optimal stopping). Chance constraints are hard optimization constraints that must hold with a specified probability \cite{shapiro2021lectures}.
Methods to convexly approximate these generically nonconvex chance constraints include tail-bounds, \ac{CVAR} programs, robust counterparts \cite{ben2009robust}, and scenario approaches (sampling)\cite{calafiore2005uncertain, campi2011sampling, tempo2013randomized}. The chance-constrained feasible set can be convex under specific distributional assumptions such as log-concavity \cite{lagoa2005probabilistically, ben2009robust}.

The chance-peak task is a specific form of an optimal stopping problem, in which the terminal time is chosen to maximize the \ac{VAR} of $p$. Optimal stopping is a specific instance of \iac{OCP}. The work in \cite{lewis1980relaxation} cast the generically nonconvex \ac{ODE} \acp{OCP} as a convex infinite-dimensional \ac{LP} in occupation measures with no conservatism (relaxation gap) introduced under continuity, compactness, and regularity assumptions. This method was extended in \cite{cho2002linear} to the optimal stopping of (Feller) stochastic processes such as \acp{SDE}, which formed an \ac{LP} to maximize the expectation of a state function $p$ along process trajectories.

These infinite-dimensional \acp{LP} must be truncated into finite-dimensional convex programs for tractable optimization. In the case where all problem data is polynomial, the moment-\ac{SOS} hierarchy of \acp{SDP} can be used to find a sequence of upper-bounds to the measure \acp{LP} \cite{lasserre2009moments}. Application of moment-\ac{SOS} polynomial optimization methods for deterministic or robust systems includes optimal control \cite{lasserre2008nonlinear},  reachable set estimation \cite{henrion2013convex}, and peak estimation \cite{fantuzzi2020bounding, miller2021uncertain}. Instances of polynomial optimization for stochastic processes include option pricing \cite{lasserre2006pricing}, probabilistic barrier certifiactes of safety \cite{prajna2004stochastic, prajna2007framework}, stopping of \levy processes \cite{kashima2010optimization}, infinite-time averages \cite{fantuzzi2016bounds}, and reach-avoid sets \cite{xue2022sdereachavoid}. We also note that polynomial optimization has been directly applied towards chance-constrained polynomial optimization \cite{jasour2015chance}, distributionally robust optimization \cite{de2020distributionally}, and minimization of the \ac{VAR} for static portfolio design \cite{tian2017moment}.

