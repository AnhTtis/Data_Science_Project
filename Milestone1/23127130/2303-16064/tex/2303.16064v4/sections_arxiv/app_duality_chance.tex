\section{Strong Duality of
Chance-Peak Concentration-Bound Linear Programs}
\label{app:duality_chance}

%\urg{This is the implementation of Matteo's proof towards strong duality of the chance-peak program \eqref{eq:peak_chance_meas}.}

%We will use the following expression of \eqref{eq:var_meas_soc} with an explicitly written \ac{SOC} variable $(s, \kappa)$ linked with linear constraints to the measures $(\mu_0, \mu_\tau, \mu)$. 
%\begin{lem}
%The following program has the same optimum value as \eqref{eq:var_meas_soc} under the redefinition $\alpha = (r/2)$
%{\color{blue}Matteo's comment: It is indeed very good to write the primal with only equality constraints, and decision variables belonging to a cone. However, ideally, you also want to put everything that depends on decision variables on the left of the "$=$" symbol, and parameters on the right, in order to correctly identify $\mathbf{Z}$ (using my notations); I suspect that your current dual formulation is wrong, did you try to implement an SoS program based on it to numerically verify strong duality? Indeed, I may have forgotten to say it, but I believe that strong duality also holds at each step of the hierarchy  above a certain minimal degree (using the exact same arguments as in the infinite dimensional setting)} {be careful: letter $\tau$ is already used to denote the stopping time, so you might want to use another symbol; and I believe that the same happens with $s$ which, at the very beginning of the paper, denotes a scalar random variable} {\color{blue} Eventually, given \eqref{eq:soc_begin}, you can recast \eqref{eq:soc_end_aff} as $s_1 + \tau = 2$, which is only 2D, while its current formulation involves many moments of $\mu_\tau$.Maybe this dual formulation is correct. To be compared to my red comments in the CDC paper.}
%\begin{subequations}
%\label{eq:var_meas_soc_ext}
%\begin{align}
%    p^*_{r} = &\sup \quad \alpha s_2  + \inp{p(x)}{\mu_\tau} \\
%    & s_1+\inp{p(x)^2}{\mu_\tau}= 1 \label{eq:soc_begin}\\
%    & s_3 - 2 \inp{p(x)}{\mu_\tau} =0 \\
%    & s_1 + \kappa= 2 \label{eq:soc_end_aff}\\
%    & (s, \kappa) \in Q^3 \label{eq:soc_end}\\
%     & \mu_\tau - \Lie^\dagger \mu = \delta_0 \otimes \mu_0 \label{eq:soc_liou} \\
%    & \mu, \ \mu_\tau \in \Mp{[0, T] \times X}  \label{eq:soc_end_meas}.
%\end{align}
%\end{subequations}
%\end{lem}
%\begin{proof}
%Constraints \eqref{eq:soc_begin}-\eqref{eq:soc_end} implement the \ac{SOC} relations in \eqref{eq:var_meas_con_soc_def}-\eqref{eq:var_meas_con_soc}. The variable $z$ from \eqref{eq:var_meas_soc} is equal to $s_2/2$. We use the relation $s_1+\kappa=2$ in \eqref{eq:soc_end_aff} by addinng together the $(s_1, \kappa)$ terms in \eqref{eq:var_meas_con_soc} with $\kappa = 1+\inp{p^2}{\mu_\tau}$.
%\end{proof}

In order to apply the strong duality results of Appendix \ref{app:duality_general} towards the tail-bound chance-peak problem (Theorem \ref{thm:cdc_strong_dual}), we need to provide an \ac{SDP}-representation of the \ac{SOC} cone.  
\begin{lem} \label{lem:soc_lmi}
An element $(y,s) \in \mathbb{L}^n$ satisfies the \ac{LMI} \cite{alizadeh2003second}
\begin{align}
    \Phi = \begin{bmatrix}
    s & y^T \\ y & s I_n
    \end{bmatrix} \succeq 0. \label{eq:soc_embedding}
\end{align}

%\urg{We should explicitly comment about and use an \ac{SOC} description for Appendix \ref{app:duality_general} as a specific case. The cone defined by \eqref{eq:soc_embedding} is not self-dual, but the parameters $(s, \kappa)$ such that \eqref{eq:soc_embedding} is PSD does form the self-dual cone $Q^n$.

%}{\color{blue}Matt: Are you sure that this cone is not self-dual? I remember I proved the contrary at some point. From a mathematical viewpoint, seeing SOCP as it is usually viewed or through \eqref{eq:soc_embedding} is totally equivalent as they are isomorphic.}
\end{lem}
\begin{proof}
When ${s}=0$, the containment $({y}, 0) \in {\mathbb{L}}^n$ requires that ${y} = \0_n$. The matrix in $\Phi$ \eqref{eq:soc_embedding} is therefore the $\0_{n \times n}$ matrix which is \ac{PSD}. 
Now consider the case where ${s} > 0$. A Schur complement of $\Phi$ yields the constraint ${s - y^T y / s} \geq 0$. Multiplying through by the positive ${s}$ results in ${s^2 - \norm{y}^2_2} \geq 0, \ {s} > 0$, which is the definition of the \ac{SOC} cone ${(y, s) \in \mathbb{L}^n}$. Lemma \ref{lem:soc_lmi} is therefore proven.
% \urg{Matt @ Jared: the proof is a trivial use of the Schur complement; can you write it there?}
\end{proof}

%We are now able to define the resident spaces for the strong duality framework of Appendix \ref{app:duality_general}. Noting that the cone $\mathcal{M}(X)$ is the space of bounded signed Borel measures over $X$, the measure-\ac{LP} spaces are 
%\begin{align}
%    \mathcal{X}_0 &= \R^{4}, \quad \mathcal{X}_0' = \0_{n+1} \nonumber \\
%    \mathcal{X}_\infty &=  \mathcal{M}([0, T]\times X)^2  \nonumber \\
%    \mathcal{X}'_\infty &=  C([0, T]\times X)^2 \label{eq:dual_spaces}\\
%    \mathcal{X} &= \mathcal{X}_0 \times \mathcal{X}_\infty  , \quad \mathcal{X}' = \mathcal{X}_0'\times\mathcal{X}_\infty'. \quad \nonumber
%\end{align}

%The space $\mathcal{X}_\infty$ is equipped with a weak-* topology, and the space $X_0$ has the standard topology over $Q^3$.

%The nonnegative subcones of \eqref{eq:dual_spaces} are,
%\begin{align}
%    \mathcal{X}_+' &= Q^3 \times C_+([0, T]\times X)^2   \label{eq:dual_nonneg_cones}\\
%    \mathcal{X}_+ &= Q^3 \times \Mp{[0, T]\times X}^2. \nonumber
%\end{align}

%The variables $((s, \kappa), (\mu_\tau, \mu))$ from  \eqref{eq:soc_end}- \eqref{eq:soc_end_meas} are members of $\mathcal{X}_+$. Under assumption A1, the cones $\mathcal{X}_+'$ and $\mathcal{X}_+$ in \eqref{eq:dual_nonneg_cones} are topological duals. 

%The constraint spaces $\mathcal{Y}$ and $\mathcal{Y}'$ are defined as,
%\begin{align}
%    \mathcal{Y} &=  \R^3 \times C^2([0, T] \times X) \label{eq:constraint_spaces} \\
%    \mathcal{Y}' &=  \0_3 \times C^2([0, T] \times X)'. \nonumber
%\end{align}
%The space $\mathcal{Y}$ has the sup-norm-bounded weak topology. The variables in $\mathcal{Y}$ 
%are $\boldsymbol{\ell} = (\phi, v(t, x)) \in \mathcal{Y}$.


%The induced adjoint linear operators $\A': \mathcal{Y}_+' \rightarrow \mathcal{X}_+'$ and $\A: \mathcal{X}_+ \rightarrow \mathcal{Y}_+$  from constraints \eqref{eq:soc_liou}-\eqref{eq:soc_end_aff} are defined as
%\begin{align}
%    \A(\boldsymbol{\mu}) &= \begin{bmatrix}
%    s_1 + \inp{p(x)^2}{\mu_\tau} \\
%    s_3 -2 \inp{p(x)}{ \mu_\tau} \\
%    s_1 + \kappa \\
%     \mu_p - \Lie_f^\dagger \mu
%    \end{bmatrix}\nonumber & 
%    \A'(\boldsymbol{\ell}) &= \begin{bmatrix}\phi_1 + \phi_3 \\ 0 \\ \phi_2 \\ \phi_3 \\ -\Lie v(t,x) \\ v(t, x) + \phi_1 p(x)^2 - 2 \phi_2 p(x)
%     \end{bmatrix}.\nonumber
%\end{align}

%The cost vector $\mathbf{c}$ and constraint vector $\mathbf{b}$ of \eqref{eq:var_meas_soc_ext} are,
%\begin{subequations}
%\begin{align}
%    \mathbf{c} &= \begin{bmatrix}
%    0 & \alpha & 0 & 0 & 0 & \inp{p}{x} & 0
%    \end{bmatrix} \\
%    \mathbf{b} &= \begin{bmatrix}1 & 0 & 2 & \delta_0 \otimes \mu_0 \end{bmatrix}.
%\end{align}
%\end{subequations}

%Problem \ref{eq:var_meas_soc_ext} may be expressed in standard-form as,
%\begin{align}
%    p^* =& \sup_{\boldsymbol{\mu} \in \mathcal{X}_+} \inp{\mathbf{c}}{\boldsymbol{\mu}} & & \mathbf{b} - \A(\boldsymbol{\mu}) \in \mathcal{Y}_+. \label{eq:chance_meas_std}\\
%\intertext{The dual program to \eqref{eq:var_meas_soc_ext} in standard form from Appendix \ref{app:duality_general} (with $\inp{\boldsymbol{\ell}}{\mathbf{b}} = \phi_1 + 2 \phi_3 + \int_{X_0} v(0, x_0) d\mu_0(x_0)$),}
%    d^* = &\inf_{\boldsymbol{\ell} \in \mathcal{Y}'} \inp{\boldsymbol{\ell}}{\mathbf{b}}
%    & &\A'(\boldsymbol{\ell}) - \mathbf{c} \in \mathcal{X}_+. \label{eq:chance_cont_std}
%\end{align}

Lemma \ref{lem:soc_lmi} ensures that problem \eqref{eq:var_meas_soc_ext} is an instance of the more generic \eqref{eq:pmsdp} with $M = 2$, $X_1 = X_2 = [0,T]\times X$, $\cY = C^2([0,T]\times X) \times \R^3$ and \begin{equation}
    \cX_0 = \left\{ { \begin{bmatrix}
    s & y^T \\ y & s I_3
    \end{bmatrix}} \; \middle| \; { s \in \R, y \in \R^3 } \right\}.\end{equation}
Therefore, we only need to verify that the hypotheses of Theorem \ref{thm:strongdual} hold in our specific case. 

%Let $M$ be a matrix from \eqref{eq:soc_embedding} constructed from $(s, \kappa)$ in \eqref{eq:soc_end}. 
Letting ${z}=([{z}_1, {z}_2, {z}_3], {z}_4) \in \mathbb{L}^3$ be an \ac{SOC}-constrained variable, we define the matrix $\Phi$ from \eqref{eq:soc_embedding} as
\begin{align}
    \Phi = \begin{bmatrix}
    {z}_4 & {z}_1 & {z}_2 & {z}_3 \\
    {z}_1 & {z}_4 & 0 & 0 \\
    {z}_2 & 0 & {z}_4 & 0 \\
    {z}_3 & 0 & 0 & {z}_4
    \end{bmatrix}.\label{eq:soc_embedding_q}
\end{align}


Theorem \ref{thm:strongdual} requires the following sufficient conditions to prove strong duality between \eqref{eq:var_meas_soc_ext} and \eqref{eq:var_cont_soc} and their optimality obtainment. 
% \urg{Matt @ Jared: please modify these conditions so that they match the new presentation of appendix A.}
\begin{enumerate}
\item[R1]There exists a feasible solution for $(\mu_\tau, \mu, {z})$ from \eqref{eq:var_meas_soc_ext}.
    \item[R2] The measures $\mu_\tau, \mu$ are bounded.
    \item[R3] The square of the matrix $\Phi$ from \eqref{eq:soc_embedding_q} has bounded trace.
\end{enumerate}

We start with R1. Letting $x(t \mid x_0)$ be an \ac{SDE} trajectory from \eqref{eq:sde_sol} and $t^* \in (0, T]$ be a stopping time with $\tau^* = t^* \wedge \tau_X$, we define $\mu$ as the occupation measure of $x(t \mid x_0)$ and $\mu_\tau$ as its time-$\tau^*$ state distribution $\mu_{\tau^*}$. Feasible choices for entries of the \ac{SOC}-constrained ${z}$  are (from Lemma \ref{lem:sqrt})
\begin{align}
    {z}_1 &= 1-\inp{p^2}{\mu_{t^*}}, & {z}_2 &= \sqrt{\inp{p^2}{\mu_{t^*}} - \inp{p}{\mu_{t^*}}^2}, \\
    {z}_3 &= 2\inp{p}{\mu_{t^*}}, & {z}_4 &= 1+\inp{p^2}{\mu_{t^*}}.
\end{align}

Requirement R2's satisfaction follows the statement in Lemma \ref{lem:moment_bound}  that $\mu_\tau, \mu$ are bounded under A1-A3.

We end with R3. The trace $\Tr{\Phi^2}=\sum_{ij} \Phi_{ij}^2 $ is equal to
\begin{subequations}
\label{eq:trace_square}
\begin{align}
    \Tr{\Phi^2} &=  2{z}_1^2 + 2{z}_2^2 + 2{z}_3^2 +4 {z}_4^2 \\&
    = 2(1-\inp{p^2}{\mu_{t^*}})^2 + 2( \sqrt{\inp{p^2}{\mu_{t^*}} - \inp{p}{\mu_{t^*}}^2})^2  \nonumber\\
    &\qquad + 2(2\inp{p}{\mu_{t^*}})^2 +4(1+\inp{p^2}{\mu_{t^*}})^2 \\
    &= 6 + 6\inp{p}{\mu_{t^*}}^2 + 6\inp{p^2}{\mu_{t^*}} + 6 \inp{p^2}{\mu_{t^*}}^2.
\end{align}
\end{subequations}

Let $\Pi_1 = \max_{x \in X} p(x)$ and  $\Pi_2 = \max_{x \in X} p(x)^2$ be bounds on $p$ and $p^2$ in $X$. Both $\Pi_1$ and $\Pi_2$ will be finite by the compactness of $X$ (A1) and the continuity of $p$ within $X$ (A3). Given that $\mu_{t^*}$ is a probability distribution supported in $X$, the moments of $\mu_{t^*}$ will be bounded by $\inp{p}{\mu_{t^*}} \leq \Pi_1$ and $\inp{p^2}{\mu_{t^*}} \leq \Pi_2$. The squared-trace in \eqref{eq:trace_square} can be upper-bounded by a finite value $B^2$ such that
\begin{align}
    \Tr{\Phi^2} \leq 6(1 +  \Pi_1^2 + \Pi_2 + \Pi_2^2) = B^2 < \infty. \label{eq:trace_square_B}
\end{align}

The finite bound $B^2 \in [0, \infty)$ from \eqref{eq:trace_square_B} validates R3, and completes all conditions necessary for Theorem \ref{thm:strongdual} to provide for strong duality and optima attainment.

% \old{
% Requirement R1 is fulfilled by assumption A1. Let $\Pi = \max_{x \in X} p^2(x)$ be a finite bound on $p^2(x)$. By \eqref{eq:soc_end_aff}, it holds that $\kappa = \inp{p^2(x)}{\mu_\tau} + 1 \leq 1 + \Pi$. 
% Requirement R2 follows from the strong duality Proposition 2.2 option B \urg{[ref]} in Appendix \ref{app:duality_general}.
% The trace $\Tr{M^2}$ is equal to $2(\kappa^2 + \norm{s}_2^2) \leq 4\kappa^2\leq 4\Pi^2$. Since all measure masses are bounded above by $T$ (Lemma \ref{lem:moment_bound}), choosing $B = \max(T, 4 \Pi^2) < \infty$ fulfills R2. Requirement R3 is satisfied under assumptions A2 and A3. Lastly, the proof of Theorem \ref{thm:upper_bound_nonlinear} describes a process to return feasible measures from any \ac{SDE} trajectory. The quantities $(s, \kappa)$ may be derived from the chosen stopping time distribution $\mu_\tau$. Strong duality therefore holds between.
% }