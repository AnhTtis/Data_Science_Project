\section{Strong Duality of Measure SDPs}
\label{app:duality_general}

 The work in \cite{tacchi2022convergence} gives sufficient condition to ensure strong duality in the framework of linear programming on measures.
This appendix generalizes the results of  \cite{tacchi2022convergence} by forming a framework of convex programming on measures with infinite-dimensional linear constraints and finite dimensional \ac{LMI} constraints on moments. In particular, we add the case of optimization over Borel measures with \ac{SOC} constraints on their moments to the original framework \cite{tacchi2022convergence}.

Let $M, m_0 \in \N$ be positive integers. For $i = 1..M$, let $m_i \in \N$ and $X_i \subset \R^{m_i}$ be a compact set. Let:
\begin{itemize}
    \item $\cX_0 \subset \Sym^{m_0}$ be a vector space of symmetric matrices. Specific instances of $\cX_0$ could be $\{\mathrm{diag}(\chi) \; | \; \chi \in \R^{m_0}\}$ (the space of diagonal matrices, corresponding to linear programming), or $\cX_0 = \Sym^{m_0}$ (the space of all symmetric matrices, corresponding to semidefinite programming). In particular, there exists such a space to represent second order cone programming \cite{alizadeh2003second}),
    \item $\cX_\infty = \mathcal{M}(X_1)\times \ldots \times \mathcal{M}(X_M)$ be a vector space of signed Borel measures, equipped with its weak-$*$ topology, so that its topological dual is $\cX^*_\infty = C(X_1)\times\ldots\times C(X_M)$,
    \item $\cX = \cX_0 \times \cX_\infty$ be our decision space, with topological dual $\cX^* = \cX_0 \times \cX_\infty^*$,
    \item A Banach space $\cY$ with dual $\cY^*$ that will represent our constraint space for equality constraints. In the context of the moment-SOS hierarchy, $\cY$ is chosen as a product space of smooth/polynomial functions defined on compact sets,
    \item $\cX_+ = \{(X,\mu_1,\ldots,\mu_M) \in \cX \; | \; X \succeq 0, \quad \forall i = 1..M, \mu_i \in \Mp{X_i}\}$ and
    \item[] $\cX^*_+  =\{(Y,v_1,\ldots,v_M) \in \cX^* \; | \; Y \succeq 0, \quad \forall i=1..M, v_i \geq 0 \}$ be two convex cones.
\end{itemize}
For $\phi = (X,\mu_1,\ldots,\mu_M) \in \cX$ and $\psi = (Y,v_1,\ldots,v_M) \in \cX^*$, we define the duality
\begin{equation}
    \langle \psi, \phi \rangle_\cX = \Tr{X \, Y} + \sum_{i=1}^M \int_{X_i} v_i(x_i) d\mu_i(x_i).
\end{equation}
Similarly, we denote by $\langle \cdot , \cdot \rangle_\cY$ the duality between elements of $\cY$ and $\cY^*$. Let $A : \cX \longrightarrow \cY^*$ be a continuous linear map, $y \in \cY^*$ be a vector of continuous linear forms (when $\cY$ is made of polynomials, $y$ is a moment sequence), $C \in \cX_0$, $p \in \R[x_1]\times\ldots\times\R[x_M] \subset \cX^*_\infty$ be a vector of polynomials $\gamma = (C,g) \in \cX^*$. We consider the following moment-SDP problem:
\begin{subequations} \label{eq:msdp}
\begin{align}
    p^*_M = & \sup \quad \langle \gamma , \phi \rangle_\cX, &  \phi &\in \cX_+, & A \phi &= y  \label{eq:pmsdp}
\end{align}
with dual problem
\begin{align}
    d^*_M = & \inf \quad \langle w , y \rangle_\cY, &  w &\in \cY,    & A^\dagger w - \gamma &\in \cX^*_+. \label{eq:dmsdp} 
\end{align}
\end{subequations}
It is well known that weak duality $p^*_M \leq d^*_M$ always holds \cite{barvinok2002convex}. In this section, we will prove that under some mild assumptions, strong duality $p^*_M = d^*_M$ also holds.

We first prove a simple lemma on strong duality conditions.

\begin{lem} \label{lem:strongdual}
In this lemma, we consider again the duality pair \eqref{eq:msdp}, but with generic spaces $\cX$, $\cY$ and a convex cone $\cX_+$. We also define, for any vector space $\mathcal{Z}$ containing some vector $\eta$ and any linear map $U : \cX \longrightarrow \mathcal{Z}$, the level set $U_\eta = \{\phi \in \cX \; | \; U \phi = \eta\}$. In such setting, we assume that
\begin{enumerate}
    \item[A1'] $\exists \phi \in \cX_+$ such that $A \phi = y$.
    \item[A2'] $A_0 \cap \gamma_0 \cap \cX_+ = \{0\}$.
    \item[A3'] $\exists \psi \in \cX^*$ such that
    \begin{enumerate}
        \item $\langle \psi , \cX_+ \rangle_\cX \subset \R_+$
        \item $\psi_0 \cap \cX_+ = \{0\}$ 
        \item $\psi_1 \cap \cX_+$ is compact.
    \end{enumerate}
\end{enumerate}
\begin{center}
    Then, $p^*_M = d^*_M.$
\end{center}
Moreover, if $p^*_M < \infty$, then there is an optimal $\phi^*$ feasible for \eqref{eq:pmsdp}  such that $\langle \gamma , \phi^* \rangle_\cX = p^*_M$.
\end{lem}
\begin{proof}
    We use \cite[Chap. IV: Thm (7.2), Lem (7.3)]{barvinok2002convex}. Consider the cone
   $$\cX_A^\gamma = \{(A\phi , \langle\gamma,\phi\rangle_\cX) \; | \; \phi \in \cX_+\}.$$
   Theorem (7.2) of \cite{barvinok2002convex} ensures that under A1' and closedness of $\cX_A^\gamma$, strong duality holds, and that $p^*_M < \infty$ then implies existence of an optimal $\phi^*$. Then, Lemma (7.3) of \cite{barvinok2002convex} states that if $\cX_+$ has a compact convex base and A2' holds, then $A_\cX^\gamma$ is closed. Thus, we need to find a compact convex base of $\cX_+$.

Let $\phi \in \cX_+ \setminus \{0\}$. A3'.\textit{(a)-(b)} ensure that $\langle \psi , \phi \rangle_\cX > 0$ so that $\tilde\phi = \frac{\phi}{\langle\psi , \phi \rangle_\cX}$ is well defined and belongs to the cone $\cX_+ \setminus \{0\}$. Moreover, $\langle \psi ,  \tilde\phi\rangle_\cX = 1$ is clear by definition, so that $\tilde\phi \in \psi_1 \cap \cX_+$. This proves that any $\phi \in \cX_+ \setminus \{0\}$ can be described as $\phi = \langle \psi , \phi \rangle_\cX \tilde\phi$ with $\tilde\phi \in \psi_1 \cap \cX_+$ and $\langle \psi , \phi \rangle_\cX > 0$, which is the definition of $\psi_1 \cap \cX_+$ being a base of $\cX$. By compactness assumption A3'.\textit{(c)}, we deduce that the assumptions of Lemma (7.3) of Theorem (7.2) of \cite{barvinok2002convex}  hold: $X_A^\gamma$ is closed and thus $p^*_M = d^*_M$.
\end{proof}

\begin{thm}
\label{thm:strongdual}
Suppose that there exists $B>0$ such that for all $\phi = (X,\mu_1,\ldots,\mu_M)$ feasible for \eqref{eq:pmsdp}, one has $\Tr{X^2} \leq B^2$ and for $i=1..M$, $\langle 1 , \mu_i\rangle \leq B$. Also assume that at least one such feasible $\phi$ exists. Then, $p^*_M = d^*_M$. Moreover, there exists an optimal $\phi^*$ such that $A\phi^* = y$ and $\langle \gamma , \phi^* \rangle_\cX = p^*_M$.
\end{thm}
\begin{proof}
We prove that the assumptions of Lemma \ref{lem:strongdual} hold. First of all, $\cX_+$ is indeed a convex cone under A1', as it is the product of convex cones $\Sym_+^{m_0}$ and $\Mp{X_i}$.

Next, we focus on hypothesis A2' Let $\phi = (X,\mu_1,\ldots,\mu_M) \in A_0 \cap \gamma_0 \cap \cX_+$. We want to prove that $\phi = 0$. Let $\phi^{(0)} = (X^{(0)}, \mu_1^{(0)},\ldots,\mu_M^{(0)}) \in \cX_+$ such that $A \phi^{(0)} = y$. Define, for $t \geq 0$, $\phi^{(t)} = \phi^{(0)} + t \phi$. Let $t \geq 0$. Since $\cX_+$ is a convex cone, $\phi^{(t)} \in \cX_+$. In addition,

$$A \phi^{(t)} = A \phi^{(0)} + t \, A \phi = A \phi^{(0)} = y,$$

so that $\phi^{(t)}$ is feasible for \eqref{eq:pmsdp}. Thus, by assumption,

\begin{align*}
    B^2 & \geq \Tr{(X^{(0)} + t X)^2} \\
    & = \Tr{X^{(0)2} + 2t X^{(0)} X + t^2 X^2} \\
    & = \Tr{X^{(0)2}} + 2t \Tr{X^{(0)} X} + t^2 \Tr{X^2} \\
    & = t^2 \Tr{X^2} + \underset{t\to\infty}{o}(t^2).
\end{align*}
Staying bounded when $t$ goes to infinity requires that $\Tr{X^2} = 0$, implying that $X=0$. The same reasoning replacing $\Tr{X^2}$ with $\langle 1 , \mu_i \rangle$ yields that for all $i=1..M$, $\mu_i=0$. Thus, $\phi=0$ and A2' holds.

We turn to A3' and consider $\psi = (I_{m_0},1_M)$ where $I_{m_0}$ is the size $m_0$ identity matrix and $1_M$ is the dimension $M$ vector of functions that are all constant equal to $1$. Note that if $(X,\mu_1,\ldots,\mu_M) \in \cX_+$, then $\Tr{X} \geq 0$ and  $\forall i=1..M: \ \langle 1 , \mu_i \rangle \geq 0$ (i.e. $\langle \psi , \cX_+ \rangle_\cX \subset \R_+$). Moreover the equality cases in those inequalities only hold for $X=0$ and $\mu_i=0$ respectively, so that $\psi_0 \cap \cX_+ = \{0\}.$ It only remains to prove that $\psi_1 \cap \cX_+$ is compact.

First, it is bounded for the norm $\norm{(X,\mu_1,\ldots,\mu_M)} = \sqrt{\Tr{X^2}} + \sum_{i=1}^M \norm{\mu_i}_{TV}$ where the total variation norm of a signed measure is $$\norm{\mu}_{TV} = \sup\{\langle v , \mu \rangle \; | \; -1 \leq v \leq 1\}.$$ In particular, the total variation norm of a nonnegative measure $\mu \in \Mp{X}$ is equal to its mass $\norm{\mu}_{TV} = \langle 1 , \mu \rangle$.

Indeed, let $(X,\mu_1,\ldots,\mu_M) \in \psi_1 \cap \cX_+$; then $\Tr{X}\leq 1$ and for $i=1..M$, $1 \geq \langle 1 , \mu_i \rangle = \norm{\mu_i}_{TV}$. As $X$ is positive semidefinite, $\Tr{X}\leq 1$ means that none of the eigenvalues of $X$ are bigger than $1$. As such, $\Tr{X^2} \leq m_0$, because it is the sum of the squares of the eigenvalues of $X$. Thus, for all $\phi \in \psi_1 \cap \cX_+$, $\norm{\phi} \leq m_0 + M$.

Then, it is also closed for the weak-* topology of $\cX$ by continuity of $\langle \psi , \cdot \rangle_\cX$ and closedness of $\cX_+$ as the product of closed sets $\Sym^{m_0}_+$ and $\Mp{X_i}$. Thus, $\psi_1 \cap \cX_+$ is weak-* closed and bounded: according to the Banach-Alaoglu theorem, it is compact: assumption A3' of Lemma \ref{lem:strongdual} holds, which concludes the proof of strong duality.

Finally, let $\phi = (X,\mu_1,\ldots,\mu_M) \in \cX_+$ feasible for \eqref{eq:pmsdp}. Then, by assumption, $\Tr{X^2} \leq B^2$ and $\langle 1 , \mu_i \rangle \leq B$ for all $i=1..M$. Thus, one has
\begin{align*}
    \langle \gamma , \phi \rangle_\cX & = \Tr{C\, X} + \sum_{i=1}^M \langle g_i , \mu_i \rangle. \\
    \intertext{Using the Cauchy-Schwarz inequality results in }
    & \leq \sqrt{\Tr{C^2} \Tr{X^2}} + \sum_{i=1}^M \langle g_i , \mu_i \rangle \\    
    & \leq \sqrt{\Tr{C^2} \Tr{X^2}} + \sum_{i=1}^M \sup_{X_i}|g_i| \langle 1 , \mu_i \rangle \\
    & \leq \left(\sqrt{\Tr{C^2}} + \sum_{i=1}^M \norm{g_i}_\infty \right) B
\end{align*}
so that taking the supremum over all feasible $\phi$ yields
$$ p^*_M \leq \left(\sqrt{\Tr{C^2}} + \sum_{i=1}^M \norm{g_i}_\infty \right) B < \infty, $$
which is the last hypothesis of Lemma \ref{lem:strongdual} and ensures existence of an optimal solution. 
\end{proof}