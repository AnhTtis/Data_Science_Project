
\section{Strong Duality of
Chance-Peak Expected Shortfall Linear Programs}
\label{app:duality_cvar}

We follow the steps and conventions of Theorem 2.6 of \cite{tacchi2021thesis} to perform this proof of strong duality.



We collect the groups of variables into 
\begin{align}
    \bbmu &= (\mu_\tau, \mu, {\nu, \hat{\nu}}) &   \bell &= ({u,v,w}). \label{eq:bbmu}
\end{align}

We now define the following variable spaces
\begin{align}
    \mathcal{X}' &=  C([0, T]\times X)^2 \times C(\R)^2 \label{eq:dual_spaces}\\
    \mathcal{X} &= \mathcal{M}([0, T]\times X)^2 \times \mathcal{M}(\R)^2,\nonumber
\end{align}
and note that their nonnegative subcones are
\begin{align}
    \mathcal{X}_+' &= C_+([0, T]\times X)^2 \times C_+(\R) ^2\label{eq:dual_cones}\\
    \mathcal{X}_+ &= \Mp{[0, T]\times X}^2 \times \Mp{\R}^2. \nonumber
\end{align}
Additionally, the measure $\bbmu$ from \eqref{eq:bbmu} obeys $\bbmu \in \mathcal{X}_+$.
Assumption A1 imposes that the cones $\mathcal{X}_+'$ and $\mathcal{X}_+$ in \eqref{eq:dual_cones} form a pair of topological duals.

We define the constraint spaces $\mathcal{Y}$ and $\mathcal{Y}'$ as
\begin{align}
    \mathcal{Y}' &= {\R \times \cs([0, T] \times X) \times C(\R)} \\
    \mathcal{Y} &= {0 \times \cs([0, T] \times X)' \times \mathcal{M}(X)}.
\end{align}

We follow the convention of \cite{tacchi2021thesis} and define $\mathcal{Y}_+ = {\{0_\mathcal{Y}\}}$ and $\mathcal{Y}'_+ = \mathcal{Y}'$. The variable $\bell$ from \ref{eq:bbmu} satisfies $\bell \in \mathcal{Y}'$.

We formulate the affine maps of
% \begin{subequations}
\begin{align}
    \mathcal{A}(\bbmu) &= [{\inp{1}{\nu}}, \ \mu_\tau - \Lie^\dagger \mu, \ \epsilon {\nu + \hat{\nu}} - p_\# \mu_\tau] \\
    \mathcal{A}^*(\bell) &= [{ v - w \circ p , \ -\Lie v, \ u + \epsilon w, \ w} ],\nonumber
\end{align}
% \end{subequations}
and the associated cost/constraint data 
\begin{subequations}
    \begin{align}
        \mathbf{b} &= [1, \ \delta_0 \otimes \mu_0, \ 0] & 
        \mathbf{c} &= [0, \ 0, \ {\mathrm{id}_\R}, \ 0].
    \end{align}
\end{subequations}

We point out that $\A$ and $\A^*$ are linear adjoints, that $\mathcal{X}$ has a weak-* topology, and that $\mathcal{Y}$ has a sup-norm-bounded weak topology. We also note that the following pairings  satisfy 
\begin{subequations}
\begin{align}
\inp{\mathbf{c}}{\boldsymbol{\mu}} &= \inp{\mathrm{id}_\R}{\nu} \\
    \inp{\boldsymbol{\ell}}{\mathbf{b}} &= {u} + \inp{v(0, {\bullet})}{\mu_0}.
\end{align}
\end{subequations}

Problem \eqref{eq:peak_cvar_meas} may be expressed  as 
\begin{align}
    p^* =& \sup_{\boldsymbol{\mu} \in \mathcal{X}_+} \inp{\mathbf{c}}{\boldsymbol{\mu}} & & \mathbf{b} - \A(\boldsymbol{\mu}) \in \mathcal{Y}_+. \label{eq:dist_meas_std}\\
\intertext{Problem \eqref{eq:peak_cvar_cont} may be converted into }
    d^* = &\inf_{\boldsymbol{\ell} \in \mathcal{Y}'_+} \inp{\boldsymbol{\ell}}{\mathbf{b}}
    & &\A'(\boldsymbol{\ell}) - \mathbf{c} \in \mathcal{X}_+. \label{eq:dist_cont_std}
\end{align}

Sufficient conditions for strong duality from \cite[Theorem 2.6]{tacchi2021thesis} are:
\begin{itemize}
    \item[R1] Masses of measures in $\bbmu$ satisfying $b-\mathcal{A}(\bbmu) \in \mathcal{Y}_+$ are bounded.
    \item[R2] There exists a feasible $\bbmu^f$ with $b-\mathcal{A}(\bbmu^f) \in \mathcal{Y}_+$.
    \item[R3] Functions involved in defining $\mathbf{c}$, $\mathbf{b}$,  and $\mathcal{A}$ are continuous.
\end{itemize}

Requirement R1 is satisfied by Theorem \ref{thm:bounded_mass_cvar} under A1, A3, and A4. Requirement R2 holds because there exists a feasible solution $\bbmu^f$ from the proof of Theorem \ref{thm:upper_bound_cvar} (constructed from a feasible trajectory of the stochastic process) under A4. Requirement R3 is fulfilled because $v \in \cs([0, T] \times X)$ implies that $\Lie v \in C([0, T] \times X)$ (A2) and $p$ is continuous (A3). All requirements are fulfilled, which proves strong duality between \eqref{eq:peak_cvar_meas} and \eqref{eq:peak_cvar_cont}.

% \urg{We need to add a condition about the $\Mp{\R}$ support of $\psi, \hat{\psi}$, or alternatively work over the compact space $\Mp{[P_{\min}, P_{\max}]}$. The moments of $\psi, \hat{\psi}$ are bounded by Lemma \ref{thm:bounded_mass}. Theorem 2.6 of \cite{tacchi2021thesis} assumes that measures are supported on a subset of $\R$, it is not clear if this subset needs to be compact.

% The interval $[P_{\min}, P_{\max}]$ could be challenging to find. Would any compact superset of this interval also lead to strong duality?
% }
    


