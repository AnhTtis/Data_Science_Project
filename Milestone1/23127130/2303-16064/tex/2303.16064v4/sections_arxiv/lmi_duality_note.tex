
\begin{thm}
\label{thm:lmi_converge}
Under assumptions A1-A10, \eqref{eq:chance_lmi} inherits the strong duality property of its infinite-dimensional counterpart \eqref{eq:var_meas_soc}, and its optima will converge to \eqref{eq:var_meas_soc} i.e. $\lim_{d \rightarrow \infty} p^*_{r, d} = p^*_r$.
\end{thm}
\begin{proof}
Strong duality is proved almost identically in the finite dimensional setting as in the infinite dimensional setting of Theorem \ref{thm:cdc_strong_dual} by using the same arguments as in the proof of \cite[Proposition 6]{tacchi2022convergence}.

Convergence is a direct consequence of \cite[Corollary 8]{tacchi2022convergence} (when extending to the case with finite-dimensional \ac{SOC} variables) through Lemma \ref{lem:moment_bound}.
\end{proof}

\begin{rmk}
The relation $p^*_d \geq p^*_r \geq P^*_r$ will still hold when $[0, T] \times X$ is noncompact (violating A1 and A5), but it may no longer occur that $\lim_{d \rightarrow \infty} p^*_{r, d} = p^*_r$ (the conditions Lemma \ref{thm:lmi_converge} will no longer apply).
\end{rmk}

