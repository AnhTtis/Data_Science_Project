

This paper is laid out as follows: Section \ref{sec:preliminaries} reviews notation, \ac{VAR} and its upper bounds, stochastic processes, and occupation measures. Section \ref{sec:problem_statement} poses the  chance-peak problem statement and lists relevant assumptions. Section \ref{sec:tail_bound} creates \iac{SOCP} in measures to bound the tail-bound chance-peak problem. Section \ref{sec:cvar_peak} formulates \iac{LP} in measures to solve the \ac{CVAR} chance-peak problem. Section \ref{sec:lmi} provides an overview of the moment-\ac{SOS} hierarchy of \acp{SDP}, and uses this hierarchy to approximate the tail-bound and \ac{CVAR} chance-peak programs. Section \ref{sec:extensions} extends the chance-peak framework towards the estimation of distance of closest approach to an unsafe set, and the analysis of switching stochastic processes. Section \ref{sec:examples} reports experiments of the tail-bound and \ac{CVAR} programs. Section \ref{sec:conclusion} summarizes and concludes the paper. Appendix \ref{app:duality_general} proves strong duality properties for a class of measure programs with linked semidefinite constraints. Appendix \ref{app:duality_chance} applies this general strong duality proof to the tail-bound chance-peak \acp{SOCP}. Appendix \ref{app:duality_cvar} proves strong duality of the \ac{CVAR} chance-peak \acp{LP}.
% \urg{Fill in the paper structure}
% Section \ref{sec:preliminaries} will review preliminaries such as notation, notions of stability for linear systems, and \ac{SOS} proofs of polynomial nonnegativity. Section \ref{sec:full_method} will present 
% The paper is concluded in Section \ref{sec:conclusion}.