\subsection{Computational Complexity}
In Problem \eqref{eq:chance_lmi}, the computational complexity mostly depends on the number and size of the matrix blocks involved in LMI constraints (\ref{eq:lmitau},\ref{eq:lmiocc}), which in turn depend on the number and degrees of polynomial inequalities describing $X$ (the higher $d_k = \deg(h_k)$, the smaller $\M_{d-\lceil d_k / 2 \rceil}[h \bm]$). At order-$d$, the maximum size of localizing matrices is $\binom{n+1+ D}{D}$. This same analysis occurs for Problem \eqref{eq:cvar_lmi} with respect to the \ac{LMI} constraints \eqref{eq:lmitauocc}-\eqref{eq:lmicvar2}.
%At order-$d$, the size of the moment matrices corresponding to the measures is described in Table \ref{tab:moment_size}.

%\begin{table}[h]
%    \centering
%    \caption{Size of Moment Matrices in \ac{LMI} \eqref{eq:chance_lmi}}
%    \begin{tabular}{rcc}
%         Matrix: &  $\M_d(h \bm)$  \\
%         Size:   & $\binom{n+1+d}{d}$
%    \end{tabular}
%    \label{tab:moment_size}
%\end{table}

Problems \eqref{eq:chance_lmi}  and \eqref{eq:cvar_lmi} must be converted to \ac{SDP}-standard form by introducing equality constraints between the entries of the moment matrices in order to utilize symmetric-cone Interior Point Methods (e.g., Mosek \cite{mosek92}). The per-iteration complexity of \iac{SDP} involving a single moment matrix of size $\binom{n+d}{d}$ scales as $n^{6d}$ \cite{lasserre2006pricing}. The scaling of \iac{SDP} with multiple moment and localizing matrices generally depends on the maximal size of any \ac{PSD} matrix. 
%By Table \ref{tab:moment_size}, this size is $\binom{n+1+\tilde{d}}{\tilde{d}}$ with a scaling impact of $(n+1)^{6 \tilde{d}}$. The complexity of using this chance-peak routine increases in a jointly polynomial manner with $\tilde{d}$ and  $n$.
In our case, this size is at most $\binom{n+1+d}{d}$ with a scaling impact of $(n+1)^{6 d}$. The complexity of using this chance-peak routine increases in a jointly polynomial manner with $d$ and  $n$.

% \begin{rmk}
%     The set $[0, T] \times X$ can be gridded into a set of cells using a spatio-temporal decomposition, allowing for per-cell measures $(\mu_0, \mu, \mu_\tau)$ with compatibility constraints between adjacent cells \cite{cibulka2021decomp, holtorf2022spatio}. This allows for a lower degree $d_{\textrm{cell}}$ to be used than in solving \eqref{eq:chance_lmi} with a single degree $d$, which is similar in principle to spline methods in approximation.
% \end{rmk}

% \begin{rmk}
% Rational dynamics $f, g$ may be expressed by adding new variables.
% \end{rmk}
