\section{Chance-Peak Problem Statement}
\label{sec:problem_statement}
% The main theory. Display the chance-peak program in terms of occupation measures.



This section presents the problem statement for the chance-peak problem, and will formulate the tail-bound and \ac{CVAR} upper-bounding programs to peak-\ac{VAR} estimation.

% This section will present the problem statement of Peak Value-at-Risk Estimation. It will also list the tail-bound and \ac{CVAR} upper-bound programs to the chance-peak \ac{VAR} problem.

% and will also derive the infinite-dimensional \ac{SOCP} to upper bound the chance-peak quantile statistic.

% \subsection{Problem Statement}

% Let $\epsilon \in [0, 1]$ be a value for the quantile statistic, $X$ be a compact set, $X_0 \subseteq X$ be a set of initial conditions, and \eqref{eq:sde_sol} be the solution to an \ac{SDE} evolving from $x(0) \in X_0$ that remains within $X$ until it stops. For a given initial probability distribution $\mu_0 \in \Mp{X_0}$, and for all $t \in [0, T]$, let $x(t)$ be the stochastic process of \eqref{eq:sde_sol} at time $t$, and let $\mu_t \in \Mp{X}$ be its probability distribution (with $x(t)$ stopping at $\partial X$). 
\subsection{Assumptions}
\label{sec:assum}

We will require the following assumptions:
% The following assumptions will be posed throughout this paper,
\begin{itemize}
    \item[A1] The spaces $[0, T]$ and $X$ are compact.
    \item[A2] Trajectories stop upon their first contact with $\partial X$.
    \item[A3] The state function $p$ is continuous on $X$.
    \item[A4] The initial measure $\mu_0$ satisfies $\inp{1}{\mu_0} = 1$ and $\supp{\mu_0} \subseteq X$.
    \item[A5] The set $\cs = \textrm{dom}(\Lie) \subset \textrm{Hom}(C([0, T] \times X), C([0, T] \times X))$ contains $1 \in \cs$ with $\Lie 1 = 0$.
    \item[A6] The set $\cs$ can separate points and is closed under multiplication.
    \item[A7] There exists a countable basis  $\{v_k\} \in \cs$ such that every $(v, Av)$ with $v \in \cs$ is contained in the (bounded pointwise closure of the) linear span of $\{(v_k, \Lie v_k)\}$.
\end{itemize}

Assumptions A5-A7 originate from the requirements of Condition 1 of \cite{cho2002linear}. Discrete-time Markov processes, \acp{SDE}, and L\'{e}vy processes in compact domains (A1, A2) will satisfy conditions A5-A7 \cite{cho2002linear}. Assumption A5 and \eqref{eq:generator_lim} together imply that $\Lie t = 1$.
%An \ac{SDE} will have $\kappa=1$ because $\partial_t  t = 1$, and a discrete-time process from \eqref{eq:generator_discrete} has $\kappa = \Delta t$.


% We note that generators for Markov, \ac{SDE}, and L\'{e}vy processes evolving in a compact set satisfy A1-A6.

    % \item[A4] If the 

%\subsubsection{Quantile/VaR Problem}
\subsection{VaR Problem}
\begin{prob}
\label{prob:quantile}
%The chance-peak problem to find the maximal $(1-\epsilon)$-quantile statistic of  $p(x)$ is,
%\begin{subequations}
%\label{eq:peak_chance_all}
%\begin{align}
%    P^* = & \inf_{\beta \in \R}  \quad \beta \\
%     & dx = f(t, x) dt + g(t, x) dw  \\
%     & \quad \text{from $t=0$ until a stopping time of} \ \tau_X \wedge T \label{eq:peak_chance_all_stop}\\
%     & x(0) \sim \mu_0 \\
%     & \forall t \in [0, T]: \ \prb_{\mu_{t}}[{p(x(t)) \leq \beta}] \geq 1-\epsilon. \label{eq:peak_chance_all_prob}
%\end{align}
%\end{subequations}
%\end{prob}

%Constraint \eqref{eq:peak_chance_all_prob} ensures that the probability of $p(x)$ exceeding $P^*$ at any particular time $t \in [0, T]$ is less than or equal to $\epsilon$. We use a probability duality result to formulate the $\forall t$ constraint in  \eqref{eq:peak_chance_all_prob} as a $\exists t$ constraint,
%\begin{lem}
%\label{lem:prob_duality}
%Let $\mu_t$ be a $t$-indexed family of probability distributions. Then the following two optimization problems have the same solution by duality:
%\begin{equation}
%    \begin{array}{r}
%         \inf \left\{\beta \in \R \; \middle| \; \forall t \in [0,T], \prb_{\mu_{t}}[{p(x(t)) \leq \beta}] \geq 1-\epsilon \right\} \\\\
%         = \sup \left\{\gamma \in \R \; \middle| \; \exists t^* \in [0,T] ; \prb_{\mu_{t^*}}[{p(x(t)) \geq \gamma}]\geq \epsilon\right\}
%    \end{array}
%\end{equation}
%\begin{align}
%    &\inf_{\beta\in \R} \beta: & & \forall t \in [0, T]: \ \prb_{\mu_{t}}[{p(x(t)) \leq \beta}] & &\geq 1-\epsilon \nonumber \\
%    =& \sup_{\gamma \in \R} \gamma:  & & \exists t^* \in [0, T]: \ \prb_{\mu_{t^*}}[{p(x(t)) \geq \gamma}] & &\geq \epsilon
%\end{align}
%\end{lem}
%{Matt: Proof or ref needed here}
% \begin{proof}
% \urg{proof goes here after I get back.}
% \end{proof}
% In the case where $\mu_t$ represents the distribution of the state $x(t)$ at time $t$, the $\epsilon$-quantile estimation problem for a state function $p(x)$ with respect to \iac{SDE} given the initial distribution $\mu_0$ is a generically nonconvex optimization problem with respect to a stopping time $t^*$ and a bound $\gamma$,
% \todo[inline]{Matteo: not sure that $t^*$ is a decision variable of the global optimization pb, or there's a notation conflict with \eqref{eq:peak_chance_stop}; more generally, I'm having a hard time relating equation \eqref{eq:peak_chance} to the description that follows it; I suspect constraints are missing somewhere.}


%We will therefore consider the following reformulation of Problem \ref{prob:quantile} by using  Lemma \ref{lem:prob_duality}:
% By using Lemma \ref{lem:prob_duality}, we therefore will consider the following reformulation of Problem \ref{prob:quantile}
%\begin{thm}
%The following supremal chance-peak $\epsilon$-\ac{VAR} problem possesses the same optimal value as \eqref{eq:peak_chance_all}:
The chance-peak problem that maximizes the $\epsilon$-\ac{VAR} of $p$ is
% The chance-peak problem to find the $\epsilon$-VaR of  $p$ is

% P^* = & \sup_{t^* \in [0, T]}  \mathrm{VaR}_{\epsilon}(p{(x(t^*))}) \label{eq:peak_chance_obj}\\    
    
\begin{subequations}
\label{eq:peak_chance}
\begin{align}
P^* = & \sup_{t^* \in [0, T]}  \label{eq:peak_chance_obj} \ \mathrm{VaR}_{\epsilon}(p{(x(t^*))}) \\
     & \text{$x(t)$ follows $\Lie$ from $t=0$ until } \ \tau_X \wedge t^* \label{eq:peak_chance_stop}\\
     & x(0) \sim \mu_0.
    %  & VaR_{\epsilon}(\mu_{t^*})
    %  & \prb_{\mu_{t^*}}[{p(x(t)) \geq \gamma}] \geq \epsilon. \label{eq:peak_chance_prob}
\end{align}
\end{subequations}
% \quad \gamma
\end{prob}
% The pushforward $p_\# \mu_{t^*}$ from \eqref{eq:peak_chance_obj} is the univariate probability distribution of $p(x)$ at the state distribution $x \sim \mu_{t^*}$.
%  & \prb_{\mu_{t^*}}[{p(x(t)) \geq \gamma}] \geq \epsilon. \label{eq:peak_chance_prob}
%\end{thm}
%\begin{proof}
%This arises from applying Lemma \ref{lem:prob_duality}, replacing the $\beta$-infimum over every time $t \in [0, T]$ with a $\gamma$-supremum over a specific time $t^*$.
%\end{proof}
% Constraint \eqref{eq:peak_chance_prob} ensures that the probability of $p(x)$ exceeding $P^*$ at time $t^*$ is larger than or equal to $\epsilon$.

