\section{Extensions}
\label{sec:extensions}
% \urg{We can include some of these, or wait until the extended (arxiv)/full journal of the paper to put these in.}
This section outlines extensions to the developed chance-peak framework. The formulas in this section will focus on the \ac{CVAR} programs, but similar expressions may be derived for the tail-bound programs.

% \subsection{Exit-Time Statistics}

% % \textcolor{red}{Matt: our contribution here is unclear, maybe add some equations to explain how to combine this paper with \cite{henrion2021moment}}

% This extension builds upon the work of \cite{henrion2021moment} in estimating expectations of functions upon first exit of the region $X$. The exit time distribution of the \ac{SDE} in \eqref{eq:sde_sol} is the class of trajectories that stop according to the stopping time $\tau_X$.   
% The program \eqref{eq:peak_meas} is modified in \cite{henrion2021moment} by adding a support constraint $\mu_\tau \in \Mp{\partial X}$. Averaged statistics of the exit time distribution (expectations of $p_\#{\mu_\tau}$) may be computed by substituting different choices of $p$ into the $\partial X$-adjusted \eqref{eq:peak_meas}, and this expectation is bounded from above and below with no relaxation gap in \cite{henrion2021moment}. An example of this kind of statistic is finding the mean (stopping) time at which the trajectory exits by estimating $\inp{t}{\mu_\tau}$.

% We can adapt this methodology to bound the $\epsilon$-\ac{CVAR} of the exit time distribution by restricting $\mu_\tau$ in \eqref{eq:peak_cvar_meas_peak} to be contained in the support set  $\mu_\tau \in \Mp{[0, T] \times \partial X}$.
% % \begin{subequations}
% % \label{eq:var_meas_exit_soc}
% % \begin{align}
% %     p^*_{c} = &\sup \quad r z  + \inp{p}{\mu_\tau} \label{eq:var_meas_exit_obj_soc}\\
% %      & \mu_\tau = \delta_0 \otimes \mu_0 + \Lie^\dagger \mu \label{eq:var_meas_exit_lie_soc}\\
% %     & u = [1-\inp{p^2}{\mu_\tau}, \ 2 z, \ 2 \inp{p}{\mu_\tau}] \label{eq:var_meas_exit_con_soc_def}\\
% %     & (u, 1 + \inp{p^2}{\mu_\tau}) \in Q^3 \label{eq:var_meas_exit_con_soc}\\
% %     & \mu \in \Mp{[0, T] \times X}, z \in \R, u \in \R^3.\label{eq:var_meas_exit_con_meas} \\
% %     & \mu_\tau \in \Mp{[0, T] \times \partial X}.
% % \end{align}
% % \end{subequations}

% % The optimal value \eqref{eq:var_meas_exit_soc} is a (Cantelli or \ac{VP}) upper-bound on the exit time quantity $VaR_\epsilon(p_\# \mu_\tau)$.

% % The terminal measure $\mu_\tau$ from \eqref{eq:var_meas_con_meas} is restricted to be supported on the boundary $\partial X$ ($\mu_\tau \in \Mp{\partial X}$). 


\subsection{Switching}

% \textcolor{red}{Matt: a proper introduction of the concepts related to switching systems (switching function etc) is missing here}

The \ac{CVAR}-peak scheme may also be applied to switched stochastic systems. The methods outlined in this section are an extension of the \ac{ODE} approach from \cite{miller2021uncertain}, and are similar to duals of constraints found in \cite{prajna2007framework}.
Assume that there are $N_s \in \N$ subsystems indexed by $\ell=1..N_s$, each with an individual generator $\Lie_\ell$. As an example, a switched system with \ac{SDE} subsystems $\ell = 1..N_s$ could have individual dynamics
\begin{align}
    dx &= f_\ell(t, x) dt + g_\ell(t, x) \label{eq:dyn_switch}
\end{align}
and have associated generators $\Lie_\ell$ 
mapping $\forall v \in C^{1,2}([0, T] \times X) = \cs([0, T] \times X), \forall \ell=1..N_s$:
\begin{align}
    \Lie_\ell v(t, x) &= \partial_t v + f_\ell\cdot \nabla_x v + g_\ell^T(\nabla_{xx}^2v) g_\ell/2.
\end{align}

A switched trajectory is a distribution $x(t)$ and a switching function $S: [0, T] \rightarrow (1..N_s)$ under the constraint that $x(t)$ satisfies \eqref{eq:dyn_switch} whenever $S(t) = \ell$ (the $\ell$-th subsystem is active). A specific trajectory of a switched process starting from an initial point $x_0 \in X$ will be expressed as $x(t \mid x_0, S)$. No dwell time constraints are imposed on the switching sequence $S$; instead, switching can occur arbitrarily quickly in time.

% A switching function $S: [0, T] \rightarrow (1..N_s)$ may be defined 
% The switching function $S: [0, T] \rightarrow (1..N_s)$ yields the resident subsystem of an \ac{SDE} at time $t$. A trajectory of \eqref{eq:dyn_switch} is a solution $x(t \mid x_0, S(\cdot))$ in which dynamics $(f_\ell, g_\ell)$ are followed with state distribution $\mu_t$ when the system satisfies $S(t) = \ell$.

% Following from \cite{miller2021uncertain} for peak estimation under switching uncertainty, chance-peak for switched systems may be accomplished by defining occupation measures $\mu_\ell \in \Mp{[0, T] \times X}$ and generators $\Lie_\ell$ for each subsystem according to \eqref{eq:dyn_switch}. 
Let $\mu \in \Mp{[0, T]\times X}$ be the total occupation measure of the switched process trajectory $x(t \mid x_0, S)$.
The total occupation measure may be split into disjoint subsystem occupation measures $\forall \ell: \ \mu_\ell \in \Mp{[0, T] \times X}$ under the relation $\sum_{\ell=1}^{N_s} \mu_\ell = \mu$. 
The mass of a subsytem's occupation measure $\inp{1}{\mu_\ell}$ is the total amount of time that the trajectory $x(t \mid x_0, S)$ spends in subsystem $S(t) =\ell$.

The martingale equation (generalization of Dynkin's \eqref{eq:dynkin}) for switching-type uncertainty is
\begin{align}
    \mu_\tau = \delta_0 \otimes \mu_0 + \textstyle \sum_{\ell=1}^{N_s} \Lie_\ell^\dagger \mu_\ell.
\end{align}

% No additional modifications to the chance-peak setup in \eqref{eq:var_meas_soc} is necessary to include switching uncertainty. 
The \ac{CVAR}-peak problem in \eqref{eq:peak_cvar_meas} modified  for switching uncertainty is
\begin{subequations}
\label{eq:peak_cvar_switch_meas}
%\begin{align}
%p^* = & \ \sup \quad \inp{\omega}{\psi} \label{eq:peak_cvar_switch_meas_obj} \\
%    & \mu_\tau = \delta_0 \otimes\mu_0 + \textstyle \sum_{\ell=1}^L \Lie_\ell^\dagger \mu_\ell \label{eq:peak_cvar_switch_meas_flow}\\
%    & \inp{1}{\psi} = 1\label{eq:peak_cvar_switch_meas_prob}\\
%    & \epsilon \psi + \hat{\psi} = p_\# \mu_\tau \label{eq:peak_cvar_switch_meas_cvar}\\
%    & \forall \ell\in 1..L: \mu \in \Mp{[0, T] \times X}  \\
%    & \mu_\tau \in \Mp{[0, T] \times X} \label{eq:peak_cvar_switch_meas_peak}\\    
%    & \psi, \hat{\psi} \in \Mp{\R}. \label{eq:peak_cvar_switch_meas_slack}
%\end{align}
{
\begin{align}
p^* = & \ \sup \quad \inp{\mathrm{id}_\R}{\nu} \label{eq:peak_cvar_switch_meas_obj} \\
    & \mu_\tau = \delta_0 \otimes\mu_0 + \textstyle \sum_{\ell=1}^L \Lie_\ell^\dagger \mu_\ell \label{eq:peak_cvar_switch_meas_flow}\\
    & \inp{1}{\nu} = 1\label{eq:peak_cvar_switch_meas_prob}\\
    & \epsilon \nu + \hat{\nu} = p_\# \mu_\tau \label{eq:peak_cvar_switch_meas_cvar}\\
    & \forall \ell\in 1..L: \mu_\ell \in \Mp{[0, T] \times X}  \\
    & \mu_\tau \in \Mp{[0, T] \times X} \label{eq:peak_cvar_switch_meas_peak}\\    
    & \nu, \hat{\nu} \in \Mp{\R}. \label{eq:peak_cvar_switch_meas_slack}
\end{align}
}
\end{subequations}



\subsection{Distance Estimation}
\label{sec:distance}

The \ac{CVAR}-peak methodology developed in this paper can be applied towards bounding (probabilistically) the distance of closest approach to an unsafe set. Let $X_u \subset X$ be an unsafe set, and let ${(x,x') \mapsto\,} c(x, {x'})$ be a metric in $X$. The point-set distance function with respect to $X_u$ is $c(x; X_u) = \inf_{{x'} \in X_u} c(x, {x'})$. 
% The output of a chance-distance program is an infimal value $c^*$ such that there exists some time in which the probability of traveling closer than $c^*$ to $X_u$ is at least $\epsilon$.
% an infimal value $C^*$ such that $\text{Prob}(c(x(t \mid x_0); X_u) \geq C^*\} \geq 1-\epsilon$ holds for all times $t \in T \wedge \tau_X$ and initial conditions $x_0 \sim \mu_0$. 
% As an example, the probability of passing within $C^*=1$ meter to $X_u$ is 99\% for every state distribution $\mu_t$ starting from $\mu_0$.

% no state distribution $\mu_t$ will pass within a closest distance of 1 meter    to the unsafe set $X_u$ at an 

The $\epsilon$-\ac{CVAR} distance program may be expressed as 
\eqref{eq:peak_cvar_meas} with an infimal (rather than supremal) objective $p(x) =c(x; X_u)$. 
Because the objective $c({\bullet}; X_u)$ is not generally polynomial (even when $c$ is polynomial), the \ac{LMI} \eqref{eq:chance_lmi} cannot directly be posed in terms of $c({\bullet}; X_u)$. One method to maintain a polynomial structure is to add time-constant states $dx_u = \0 dt + \0 dw$ to dynamics \eqref{eq:sde} in $x$ and form the state support set $(x, {x_u}) \in X \times X_u$. When $X_u$ is full-dimensional inside $X \subset \R^n$, the occupation measure $\mu \in \Mp{[0, T] \times X \times {X_u}}$ will have a moment matrix of size $\binom{1+2n+d}{d}$ at each fixed degree $d$.

This size can be reduced using the method in \cite{miller2022distance_short}, in which the peak measure $\hat{\mu}_\tau \in \Mp{[0, T] \times X \times {X_u}}$ is decomposed into a joint measure $\eta \in \Mp{X \times {X_u}}$  and a peak measure $\mu_\tau \in \Mp{[0, T] \times X}$ that have equal $x$ marginals. 
% Assume that the diameter of $X$ is $\text{Diam}(X) = \sup_{x, y \in X} c(x, y)$.
The resultant \ac{CVAR}-distance \ac{LP} is
\begin{subequations}
\label{eq:peak_cvar_dist_meas}
%\begin{align}
%c^* = & \ \inf \quad \inp{\omega}{\psi} \label{eq:peak_cvar_dist_meas_obj} \\
%    & \mu_\tau = \delta_0 \otimes\mu_0 + \Lie^\dagger \mu \label{eq:peak_cvar_dist_meas_flow}\\
%    &\forall \phi \in C(X): \label{eq:peak_cvar_dist_meas_marginal}\\
%    & \quad \inp{\phi(x)}{\mu_\tau(t, x)} = \inp{\phi(x)}{\eta(x, y)}  \nonumber & & \\
%    & \inp{1}{\psi} = 1\label{eq:peak_cvar_dist_meas_prob}\\
%    & \forall z \in C(\R): \label{eq:peak_cvar_dist_meas_cvar}\\
%    & \quad \epsilon \inp{z(\omega)}{\psi(\omega)} + \inp{z(\omega)}{\hat{\psi}(\omega)} = \inp{z(c(x, y))}{\mu_\tau(x, y)} \nonumber  \\ 
%    & \mu, \mu_\tau \in \Mp{[0, T] \times X} \label{eq:peak_cvar_dist_meas_peak}\\ 
%    & \eta \in \Mp{X \times X_u} \\
%    & \psi, \hat{\psi} \in \Mp{\R}. \label{eq:peak_cvar_dist_meas_slack}
%\end{align}
{
\begin{align}
c^* = & \ \inf \quad \inp{\mathrm{id}_\R}{\nu} \label{eq:peak_cvar_dist_meas_obj} \\
    & \mu_\tau = \delta_0 \otimes\mu_0 + \Lie^\dagger \mu \label{eq:peak_cvar_dist_meas_flow}\\
    &\forall \phi \in C(X): \label{eq:peak_cvar_dist_meas_marginal}\\
    & \quad \int_{[0,T]\times X} \phi(x) \; d\mu_\tau(t, x) = \int_{X\times Y} \phi(x) \; d\eta(x, y)  \nonumber & & \\
    & \inp{1}{\nu} = 1\label{eq:peak_cvar_dist_meas_prob}\\
    & \epsilon \nu + \hat\nu = c_\#\mu_\tau \label{eq:peak_cvar_dist_meas_cvar}\\
    & \mu, \mu_\tau \in \Mp{[0, T] \times X} \notag\\ 
    & \nu, \hat{\nu} \in \Mp{\R} \notag \\
    & \eta \in \Mp{X \times X_u}. \notag
\end{align}
}
\end{subequations}

Constraint \eqref{eq:peak_cvar_dist_meas_marginal} enforces equality in the $x$-marginals between $\mu_\tau$ and $\eta$. Constraint \eqref{eq:peak_cvar_dist_meas_cvar} is a distance analogue of the pushforward \ac{CVAR} constraint in \eqref{eq:peak_cvar_meas_cvar}.
The Moment matrices of $\eta$ and $\mu$ respectively in the \ac{LMI} program derived from \eqref{eq:peak_cvar_dist_meas} have sizes $\binom{2n+d}{d}$,  $\binom{n+1+\tilde{d}}{\tilde{d}}$, and $({\Delta}+1)$. Unfortunately, the exponentiation operation $\inp{c^{k}}{\eta}$ causes mixed multiplications in variables even when $c$ is additively separable as $c(x, y) = \sum_{i=1}^n c_i(x_i, y_i)$ (Section V of \cite{miller2021distance}), thus forbidding the application of correlative sparsity \cite{waki2006sums} to reduce the complexity of \acp{LMI} from \eqref{eq:peak_cvar_dist_meas}.
