\begin{abstract}
\label{sec:abstract}
% \urg{Abstract goes here}
This paper formulates algorithms to upper-bound the maximum Value-at-Risk (VaR) of a state function along trajectories of stochastic processes. The VaR is upper bounded by two methods: minimax tail-bounds (Cantelli/Vysochanskij-Petunin) and Expected Shortfall/Conditional Value-at-Risk (ES). Tail-bounds lead to a infinite-dimensional Second Order Cone Program (SOCP) in occupation measures, while the ES approach creates a Linear Program (LP) in occupation measures. Under compactness and regularity conditions, there is no relaxation gap between the infinite-dimensional convex programs and their nonconvex optimal-stopping stochastic problems. Upper-bounds on the SOCP and LP are obtained by a sequence of semidefinite programs through the moment-Sum-of-Squares hierarchy. The VaR-upper-bounds are demonstrated on example continuous-time and discrete-time polynomial stochastic processes.
% and discrete-time stochastic processes.


\end{abstract}

% Each of these upper bounds yield valid quantile statistics: if the maximum height is less than 5 meters at a 99\% chance, then the maximum height is also less than 6 meters at a >99\% chance.