% \subsection{Exit-Time}

% This example uses the setting of Example 7.4 of \cite{henrion2021moment}.
% The dynamics are standard Brownian motion in $n=3$ dimensions with $f=\0_3$ and $g = \1_3$. The initial condition is $X_0 = \0_3$ with a support set of $X = \{x \in \R^3 \mid \sum_{i=1}^3 x_i^4 \leq 1\}$. The considered boundary is $\partial X = \{x \in \R^3 \mid \sum_{i=1}^3 x_i^4 = 1\}$.

% Chance-peak bounds for the first arrival time $\inp{t}{\mu_\tau}$ are displayed in Tables \ref{tab:exit_brownian} and \ref{tab:exit_brownian_cvar}. The order-1 estimates are dual infeasible and are marked by $\infty$.

% \begin{table}[h]
%    \centering
%    \caption{\ac{VP} Chance-Peak Exit-Statistic estimation of Standard Brownian Motion System  \label{tab:exit_brownian}}
% \begin{tabular}{lcccccc}
% \multicolumn{1}{c}{order}           & 1      & 2      & 3      & 4      & 5           \\ \hline
% mean & $\infty$ & 0.0642 & 0.0642 & 0.0642 & 0.0642    \\
% $\epsilon = 0.15$ & $\infty$  & 0.1833 & 0.1375 & 0.1375 & 0.1375    \\
% $\epsilon = 0.1$  & $\infty$  & 0.2300 & 0.1614 & 0.1614 & 0.1614    \\
% $\epsilon = 0.05$ & $\infty$  & 0.3292 & 0.2113 & 0.2113 & 0.2113                
% \end{tabular}
% \end{table}

% \begin{table}[!h]
%    \centering
%    \caption{\ac{CVAR} Chance-Peak Exit-Statistic estimation of Standard Brownian Motion System  \label{tab:exit_brownian_cvar}}
% \begin{tabular}{lcccccc}
% \multicolumn{1}{c}{order}           & 1      & 2      & 3      & 4      & 5           \\ \hline
% mean                      & $\infty$                & 0.0642                & 0.0642                & 0.0642                & 0.0642                \\
% $\epsilon = 0.15$         & $\infty$              & 0.2780                & 0.1867                & 0.1784                & 0.1740                \\
% $\epsilon = 0.1$          & $\infty$              & 0.3455                & 0.2109                & 0.2013                & 0.1957                \\
% $\epsilon = 0.05$         & $\infty$              & 0.4919                & 0.2537                & 0.2400                & 0.2373               
% \
% \end{tabular}
% \end{table}


\subsection{Discrete-time System}

The prior examples involved \ac{SDE} dynamics. In this subsection, we focus on a discrete-time system in which the parameter $\lambda \in \R$ is sampled according to the unit normal distribution at each time step ($\lambda[t]\sim \mathcal{N}(0, 1)$). The discrete-time system system considered is
\begin{align}
    x_+ = \begin{bmatrix}
        -0.3 x_1  + 0.8x_2 + x_1 x_2 \lambda/4\\ -0.9x_1 - -0.1x_2 - 0.2 x_1^2 +  \lambda/40
    \end{bmatrix}.\label{eq:scatter_discrete}
\end{align}

This system involves a time horizon of $T=1$ with a time step of $\Delta t = 1/10$ (10 iterations after the initial condition). The initial point is $x_0 = [-1; 0.5]$, and trajectories evolve in a support set of $X = [-1.5, 1.5]^2$. Trajectories and order-6 bounds to maximize $p(x) = -x_2$ are displayed in Figure \ref{fig:scatter_discrete}.

\begin{figure}[h]
    \centering
    \includegraphics[width=\exfiglength]{img/scatter_discrete_6.png}
    \caption{Trajectories of \eqref{eq:scatter_discrete} with $\epsilon = \{0.5, \textrm{ES} \ 0.15\}$ bounds }
    \label{fig:scatter_discrete}
\end{figure}

Table \ref{tab:twist_scatter} reports \ac{CVAR} bounds for $-x_2$ at orders $1$ to $6$. The \ac{VP} and Cantelli objectives produce bounds that are greater than $1.5$ (outside $X$) for $d$ and $\epsilon$ values tested. 

% \urg{do the Monte Carlo experiment report}
\begin{table}[!h]
   \centering
   \caption{\ac{CVAR} Chance-Peak estimation of the Discrete-time system \eqref{eq:scatter_discrete}  to maximize $p(x) = -x_2$ \label{tab:twist_scatter}}
\begin{tabular}{lcccccc}
\multicolumn{1}{c}{order}       &  1   & 2      & 3      & 4      & 5      & 6   \\ \hline
mean                      & 1.5000                & 0.8766                & 0.8128                & 0.8002                & 0.7982                & 0.7976                \\
$\epsilon = 0.15$         & 1.5000                & 1.5000                & 1.2139                & 1.0971                & 1.0743                & 1.0663                \\
$\epsilon = 0.1$          & 1.5000                & 1.5000                & 1.2973                & 1.1446                & 1.1083                & 1.0997                \\
$\epsilon = 0.05$         & 1.5000                & 1.5000                & 1.4500                & 1.2285                & 1.1653                & 1.1540               

\end{tabular}
\end{table}

\subsection{Switching}

We utilize a modification of Example C from \cite{prajna2007framework} for this final example. The two subsystems involved are:
\begin{subequations}
\label{eq:switched_sde}
\begin{align}
    dx &= \begin{bmatrix}-2.5 x_1 - 2 x_2 \\ -0.5 x_1 - x_2 \end{bmatrix}dt + \begin{bmatrix}0 \\ 0.25 x_2 \end{bmatrix}dW \\
    dx &= \begin{bmatrix}-x_1 - 2 x_2 \\ 2.5 x_1 - x_2 \end{bmatrix}dt + \begin{bmatrix}0 \\ 0.25 x_2 \end{bmatrix}dW.
\end{align}
\end{subequations}

Switched \ac{SDE} trajectories start from an initial condition of $X_0 = (0, 1)$ and are tracked in the state set $X = [-2, 2]^2$  with a time horizon of $T=5$. The chance-peak problem is solved to find bounds on $p(x) = -x_2$.

Figure \eqref{fig:switched_sde} plots switched \ac{SDE} trajectories along with $\epsilon=\{0.5, \text{\ac{CVAR}} \ 0.15,  \text{\ac{VP}} \ 0.15\}$ bounds (at order-6). Tables \ref{tab:switched_sde} and \ref{tab:switched_sde_cvar} list these discovered bounds. 




\begin{table}[!h]
   \centering
   \caption{\ac{VP} Chance-Peak upper-bounds for $p(x)=-x_2$ for the Switched System \eqref{eq:switched_sde} \label{tab:switched_sde}}
\begin{tabular}{lcccccc}
\multicolumn{1}{c}{order}           & 1      & 2      & 3      & 4      & 5     & 6      \\ \hline
mean & 0.8491 & 0.4304 & 0.3823 & 0.3630 & 0.3487 & 0.3352 \\
$\epsilon = 0.15$ & 1.5613 & 0.9953 & 0.9328 & 0.9076 & 0.8918 & 0.8853 \\
$\epsilon = 0.1$  & 1.9358 & 1.2888 & 1.2162 & 1.1865 & 1.1687 & 1.1609 \\
$\epsilon = 0.05$ & 2.7764 & 1.9469 & 1.8516 & 1.8120 & 1.7891 & 1.7799           
\end{tabular}
\end{table}

\begin{figure}[!h]
    \centering
    \includegraphics[width=\exfiglength]{img/lin_switch_6.png}
    \caption{Trajectories of the switched system \eqref{eq:switched_sde} with $\epsilon=\{0.5, \text{\ac{CVAR}} \ 0.15,  \text{\ac{VP}} \ 0.15\}$ bounds}
    \label{fig:switched_sde}
\end{figure}

\begin{table}[!h]
   \centering
   \caption{Solver time (seconds) to compute Table \ref{tab:switched_sde} \label{tab:switched_sde_time}}
\begin{tabular}{lcccccc}
\multicolumn{1}{c}{order}      & 1      & 2      & 3      & 4      & 5      & 6      \\ \hline
mean & 0.665 & 0.362 & 0.389 & 0.570 & 1.755 & 2.499 \\
$\epsilon = 0.15$ & 0.284 & 0.257 & 0.295 & 0.587 & 1.812 & 3.718 \\
$\epsilon = 0.1$  & 0.222 & 0.237 & 0.281 & 1.636 & 2.364 & 3.191 \\
$\epsilon = 0.05$ & 0.224 & 0.251 & 0.291 & 0.906 & 1.735 & 2.638
\end{tabular}
\end{table}

\begin{table}[!h]
   \centering
   \caption{\ac{CVAR} Chance-Peak upper-bounds for $p(x)=-x_2$ for the Switched System \eqref{eq:switched_sde} \label{tab:switched_sde_cvar}}
\begin{tabular}{lcccccc}
\multicolumn{1}{c}{order}           & 1      & 2      & 3      & 4      & 5     & 6      \\ \hline
mean                      & 0.8491                & 0.4304                & 0.3823                & 0.3630                & 0.3488                & 0.3350                                       \\
$\epsilon = 0.15$         & 2.3924                & 0.9698                & 0.7882                & 0.7157                & 0.6803                & 0.6540                                       \\
$\epsilon = 0.1$          & 2.9500                & 1.1137                & 0.8818                & 0.7886                & 0.7433                & 0.7133                                       \\
$\epsilon = 0.05$         & 4.2062                & 1.3924                & 1.0548                & 0.9207                & 0.8585                & 0.8200                                      
\end{tabular}
\end{table}


\begin{table}[!h]
   \centering
   \caption{Solver time (seconds) to compute Table \ref{tab:switched_sde} \label{tab:switched_sde_cvar_time}}
\begin{tabular}{lcccccc}
\multicolumn{1}{c}{order}      & 1      & 2      & 3      & 4      & 5      & 6      \\ \hline
mean                      & 0.750                 & 0.471                 & 0.466                 & 0.674                 & 1.916                 & 3.310 \\
$\epsilon = 0.15$         & 0.360                 & 0.336                 & 0.371                 & 0.835                 & 1.954                 & 3.170 \\
$\epsilon = 0.1$          & 0.311                 & 0.330                 & 0.360                 & 0.816                 & 2.177                 & 4.237 \\
$\epsilon = 0.05$         & 0.352                 & 0.331                 & 0.373                 & 0.728                 & 1.486                 & 3.602

\end{tabular}
\end{table}


\subsection{Distance Estimation}

% \urg{may need to delete this, this is wrong}

This example will involve distance estimation of a modification of the second subsystem of \eqref{eq:switched_sde}:
\begin{equation}
\label{eq:flowmod_sde}
    dx = \begin{bmatrix}-x_1 - 2 x_2 \\ 2.5 x_1 - x_2 \end{bmatrix}dt + \begin{bmatrix}0 \\ 0.1 \end{bmatrix}dW.
\end{equation}
This $L_2$ chance-distance  task  takes place at a time horizon of $T=5$ with sets $X_0 = [0; 0.75]$, $X = [-1.25, 1] \times [-1, 1]$, and $X_u = \{x_u \in \R^2 \mid 0.1^2 \geq (x_{u1} + 1)^2 + (x_{u2}+1)^2, \ x_{u1} + x_{u2} \leq -2\}$. Distance estimation was accomplished by maximizing \acp{VAR} of the function $-\norm{x-x_u}^2_2$ in \eqref{eq:var_meas_soc}, or by minimizing the \ac{CVAR} distance program in \eqref{eq:peak_cvar_dist_meas}.

System trajectories of \eqref{eq:flowmod_sde} are displayed in Figure \ref{fig:distance}, in which the unsafe half-circle set $X_u$ is drawn in solid red. Squared distance lower bounds from solving \acp{SDP} arising from distance moment programs  are listed in Table \ref{tab:flowmod_distance} and \ref{tab:flowmod_distance_cvar}. Negative distance lower-bounds are truncated to $0$ in Table \ref{tab:flowmod_distance}. This example demonstrates how the \ac{VP} chance-peak distance bounds for distance estimation are very conservative, and how \ac{CVAR} can offer an improvement in stochastic safety analysis.

\begin{table}[!h]
   \centering
   \caption{ \ac{VP} Chance-Peak squared distance lower bounds for System \eqref{eq:flowmod_sde} \label{tab:flowmod_distance}}
\begin{tabular}{lcccccc}
\multicolumn{1}{c}{order}           & 1      & 2      & 3      & 4      & 5       & 6    \\ \hline
mean & 0.5667 & 1.1929 & 1.2337 & 1.2425 & 1.2490 & 1.2506 \\
$\epsilon = 0.15$ & 0 & 0 & 0 & 0 & 0.0182 & 0.0235 \\
$\epsilon = 0.1$  & 0 & 0 & 0 & 0 & 0 & 0 \\
$\epsilon = 0.05$ & 0 & 0 & 0 & 0 & 0 & 0                     
\end{tabular}
\end{table}

\begin{figure}[!h]
    \centering    \includegraphics[width=0.9\exfiglength]{img/lin_dist_cvar.png}
    \caption{Trajectories of \eqref{eq:flowmod_sde} with $\epsilon = \{0.5, 0.15\}$ bounds}
    \label{fig:distance}
\end{figure}

\begin{table}[!h]
   \centering
   \caption{Solver time (seconds) to compute Table \ref{tab:flowmod_distance} \label{tab:flowmod_distance_time}}
\begin{tabular}{lcccccc}
\multicolumn{1}{c}{order}      & 1      & 2      & 3      & 4      & 5      & 6      \\ \hline
mean & 0.761 & 0.507 & 0.512 & 1.772 & 6.569 & 21.331 \\
$\epsilon = 0.15$ & 0.361 & 0.346 & 0.453 & 1.233 & 5.836 & 23.930 \\
$\epsilon = 0.1$  & 0.314 & 0.344 & 0.482 & 1.522 & 5.172 & 21.034 \\
$\epsilon = 0.05$ & 0.321 & 0.384 & 0.485 & 1.711 & 5.954 & 26.974
\end{tabular}
\end{table}

\begin{table}[!h]
   \centering
   \caption{\ac{CVAR} Chance-Peak squared distance lower bounds for System \eqref{eq:flowmod_sde} \label{tab:flowmod_distance_cvar}}
\begin{tabular}{lcccccc}
\multicolumn{1}{c}{order}           & 1      & 2      & 3      & 4      & 5       & 6    \\ \hline
mean          & 0.5667                & 1.1929                & 1.2337                & 1.2427                & 1.2498                & 1.2494                \\
$\epsilon = 0.15$         & 0                & 0                & 0                & 0.4884                & 0.5061                & 0.7980                \\
$\epsilon = 0.1$ & 0                & 0                & 0                & 0.2825                & 0.3018                & 0.7047                \\
$\epsilon = 0.05$                           & 0                & 0                & 0                & 0                & 0                & 0.4952               
                   
\end{tabular}
\end{table}


\begin{table}[!h]
   \centering
   \caption{Solver time (seconds) to compute Table \ref{tab:flowmod_distance_cvar} \label{tab:flowmod_distance_time_cvar}}
\begin{tabular}{lcccccc}
\multicolumn{1}{c}{order}      & 1      & 2      & 3      & 4      & 5      & 6      \\ \hline
mean         & 0.599                 & 0.272                 & 0.295                 & 1.229                 & 10.869                & 6.556                 \\
$\epsilon = 0.15$         & 0.219                 & 0.147                 & 0.207                 & 1.616                 & 4.403                 & 8.655                 \\
$\epsilon = 0.1$          & 0.105                 & 0.129                 & 0.187                 & 1.178                 & 4.987                 & 8.386                 \\
$\epsilon = 0.05$         & 0.104                 & 0.119                 & 0.193                 & 0.430                 & 1.382                 & 7.617                   

\end{tabular}
\end{table}