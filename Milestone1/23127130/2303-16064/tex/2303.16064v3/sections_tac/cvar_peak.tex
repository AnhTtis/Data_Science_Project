\section{ES Program}

\label{sec:cvar_peak}

This section poses \iac{LP} in measures to solve the \ac{CVAR} upper-bound to  Problem \ref{prob:quantile}.

\subsection{ES Problem}
The \ac{CVAR} chance-peak problem replaces the \ac{VAR} objective in  \eqref{eq:peak_chance_obj} with \ac{CVAR}.
\begin{prob}
\label{prob:cvar_peak}
    The \ac{CVAR} program that upper-bounds the chance-peak problem in \eqref{eq:peak_chance} is
    \begin{subequations}
\label{eq:peak_cvar}
\begin{align}
    P^*_c = & \sup_{t^* \in [0, T] }  \mathrm{ES}_{\epsilon}(p(x(t^*))) \label{eq:peak_cvar_obj}\\
     & \text{$x(t)$ follows $\Lie$ from $t=0$ until } \ \tau_X \wedge t^* \label{eq:peak_cvar_stop}\\
     & x(0) \sim \mu_0.
    %  & VaR_{\epsilon}(\mu_{t^*})
    %  & \prb_{\mu_{t^*}}[{p(x(t)) \geq \gamma}] \geq \epsilon. \label{eq:peak_chance_prob}
\end{align}
\end{subequations}
\end{prob}

% The hard probability constraint in \eqref{eq:peak_chance_prob} ensures that 
% The stopping rule in \eqref{eq:peak_chance_stop} will freeze the trajectory evolution at the first event between reaching time $t^*$ or whenever $x(t)$ hits the boundary $\partial X$. Problem \eqref{eq:peak_chance} chooses a stopping time $t^*$ such that there is an at least $\epsilon$-chance that $p(x(t))$ exceeds $\gamma$. A consequence of this statement is that $\prb_{\mu_{t}}[p(x(t)) \leq \gamma] \geq 1-\epsilon$ for every stopping time $t \in [0, T]$. Replacing the $\geq$ sign in \eqref{eq:peak_chance_prob} with a $\leq$ sign would allow $\gamma$ to become unbounded towards $\infty$. \urg{might be too much discussion}
% \subsection{Measure Program}




\subsection{ES reformulation}

We begin by using the following lemma to reformulate the absolute-continuity-based \ac{CVAR} definition in \eqref{eq:cvar_abscont} into an equivalent domination-based \ac{LP} in measures.

\begin{lem}
\label{lem:abscont_domination}
    Let ${\mu}, \nu \in \Mp{\R}$ be measures {such that $\nu \ll \mu$}, and {$\frac{d \nu}{ d \mu} \leq 1$}. Then there exists a slack measure ${\hat{\nu}} \in \Mp{\R}$ such that ${\nu + \hat{\nu} = \mu}$. 
    % Equivalently, the domination relation $\kappa \leq \nu$ is the same as the absolute continuity $\kappa \ll \nu$ coupled with $\frac{d \kappa}{ d \nu} \leq 1$. \urg{get a source}.
\end{lem}
\begin{proof}
    The measure ${\hat{\nu}}$ may be chosen with {$\frac{d \hat{\nu}} {d \mu} = 1 - \frac{d\nu} {d \mu}$}. This implies that ${\frac{d \nu} {d \mu} + \frac{d \hat{\nu}} {d \mu}} = 1$, resulting in {$\nu + \hat{\nu} = \mu$}.
\end{proof}

% We now list a new (to the best of our knowledge) expression for the \ac{CVAR}.
\begin{thm}
\label{thm:cvar_dom}
The \ac{CVAR} is the solution to the following \ac{LP} in measures:
\begin{subequations}
\label{eq:cvar_dom}
%\begin{align}
%    ES_\epsilon(\nu) = & \sup_{\psi, \hat{\psi} \in \Mp{\R}} \inp{\omega}{\psi} \\
%    & \epsilon\psi + \hat{\psi} = \nu \label{eq:cvar_dom_cond}\\
%    & \inp{1}{\psi} = 1. \label{eq:cvar_mass}
%\end{align}
\begin{align}
    \mathrm{ES}_\epsilon(V) = & \sup_{\nu, \hat{\nu} \in \Mp{\R}} \inp{\mathrm{id}_\R}{\nu} \\
    & \epsilon\nu + \hat{\nu} = \psi = V_\#\mathbb{P} \label{eq:cvar_dom_cond}\\
    & \inp{1}{\nu} = 1. \label{eq:cvar_mass}
\end{align}
    \end{subequations}
\end{thm}
\begin{proof} 
Equation \eqref{eq:cvar_radon_nikodym} may be divided through by $\epsilon$ to form $\epsilon {\frac{d \nu}{ d \psi} =  \frac{d (\nu \epsilon)}{ d \psi}} \leq 1$. Given that $\epsilon > 0$, the absolute continuity relation ${\nu \ll \psi}$ implies that $\epsilon {\nu \ll \psi}$. Lemma \ref{lem:abscont_domination} is applied to the combination of $
    \epsilon {\nu \ll \psi}$ and $ {\frac{d (\nu \epsilon)}{ d \psi}} \leq 1$ to produce constraint \eqref{eq:cvar_dom_cond}. This proves the conversion and equivalence of optima between \eqref{eq:cvar_abscont} and \eqref{eq:cvar_dom}.

% Let $\psi$ be a feasible point of constraints \eqref{eq:cvar_abscont_def}-\eqref{eq:cvar_radon_nikodym} with expectation $c = \inp{q}{\psi}$.
% We can construct a pair of measures $\psi = \psi$ and $\hat{\psi} = \nu - \psi\epsilon$ that satisfy \eqref{eq:cvar_dom_cond}-\eqref{eq:cvar_dom_cond}. The objectives are the same with $\inp{q}{\psi} = \inp{q}{\psi}$ given that $\psi = \psi$.

% \urg{This is unsatisfying, I feel like I'm missing something with the Radon-Nikodym derivative.

% There's a simpler proof too that I can't put into math just yet: $\psi$ should put all of its mass above the $VaR_{\epsilon}(\nu)$. The domination constraint \eqref{eq:cvar_dom_cond} will constrain the shape of $\psi$, such that $\psi \epsilon$ is the portion of $\nu$ above $VaR_{\epsilon}(\nu)$. The objective $\inp{q}{\psi}$ is then the mean value of $q$ above $VaR_{\epsilon}(\nu)$, which is equal to the \ac{CVAR}.}
\end{proof}


\subsection{Measure Program}

% Define ${\hat{p}}$ and ${\check{p}}$ as the finite quantities (by A1 and A3)
% \begin{align}
%     {\hat{p}} = \
% \end{align}

\Iac{LP} in measures will be created to upper-bound the \ac{CVAR} program \eqref{eq:peak_cvar}. The variables involved are the terminal measure $\mu_\tau$, the relaxed occupation measure $\mu$, the \ac{CVAR} dominated measure ${\nu}$, and the \ac{CVAR} slack measure ${\hat{\nu}}$.

\begin{subequations}
\label{eq:peak_cvar_meas}
%\begin{align}
%p^*_c = & \ \sup \quad \inp{\omega}{\psi} \label{eq:peak_cvar_meas_obj} \\
%    & \mu_\tau = \delta_0 \otimes\mu_0 + \Lie^\dagger \mu \label{eq:peak_cvar_meas_flow}\\
%    & \inp{1}{\psi} = 1\label{eq:peak_cvar_meas_prob}\\
%    & \epsilon \psi + \hat{\psi} = p_\# \mu_\tau \label{eq:peak_cvar_meas_cvar}\\
%    & \mu, \mu_\tau \in \Mp{[0, T] \times X} \label{eq:peak_cvar_meas_peak}\\    
%    & \psi, \hat{\psi} \in \Mp{\R}. \label{eq:peak_cvar_meas_slack}
%\end{align}
{
\begin{align}
p^*_c = & \ \sup \quad \inp{\mathrm{id}_\R}{\nu} \label{eq:peak_cvar_meas_obj} \\
    & \mu_\tau = \delta_0 \otimes\mu_0 + \Lie^\dagger \mu \label{eq:peak_cvar_meas_flow}\\
    & \inp{1}{\nu} = 1\label{eq:peak_cvar_meas_prob}\\
    & \epsilon \nu + \hat{\nu} = p_\# \mu_\tau \label{eq:peak_cvar_meas_cvar}\\
    & \mu, \mu_\tau \in \Mp{[0, T] \times X} \notag\\    
    & \nu, \hat{\nu} \in \Mp{\R}. \notag
\end{align}
}
\end{subequations}

\begin{thm}
\label{thm:upper_bound_cvar}
    Program \eqref{eq:peak_cvar_meas} upper-bounds \eqref{eq:peak_cvar} with $p^* \geq P^*$ under assumptions A2-A4.
\end{thm}
\begin{proof}
    We will prove this upper-bound by constructing a measure representation of an \ac{SDE} trajectory in \eqref{eq:peak_cvar}.
    Let $t^* \in [0, T]$ be a stopping time. Define $\mu_\tau = \mu_{t^*}$ as the state (probability) distribution of the process \eqref{eq:peak_cvar_stop} at time $t^*$ (accounting for the stopping time $\tau_X \wedge t^*$). Let $\mu$ be the occupation measure of this \ac{SDE} connecting together the initial distribution $\mu_0$ and the terminal distribution $\mu_{t^*}$. The measures ${\nu, \hat{\nu}}$ are set in accordance with Theorem \ref{thm:cvar_dom} under { $\Omega = [0,T]\times X$, $\mathbb{P} = \mu_{t^*}$ and $V=p$ (hence} ${\psi} = p_\# \mu_{t^*}$). The upper-bound holds because every process trajectory has a measure construction.
\end{proof}

\begin{thm}
\label{thm:no_relaxation_cvar}
    There is no relaxation gap between \eqref{eq:peak_cvar_meas}  and \eqref{eq:peak_cvar} $(p^*_c = P^*_c)$ under assumptions A1-A7. 
\end{thm}
\begin{proof}
Every $(\mu_\tau, \mu)$ is supported on an stochastic process trajectory by Lemma \ref{lem:no_relaxation_gap}.
The objective in \eqref{eq:peak_cvar_meas_obj} will equal the \ac{CVAR} by Theorem \ref{thm:cvar_dom} given that $p_\# \mu_\tau$ is a probability distribution. 
\end{proof}

\begin{rmk}
To the best of our knowledge, there does not appear to be a consistent comparison between the \ac{VP} bound and the \ac{CVAR} of a unimodal distribution. 
\end{rmk}
% Lemma 2 in \cite{vcerbakova2006worst} offers an upper-bound on the worst-case \ac{VAR} given the first two moments of a symmetric and unimodal $\xi$.

\subsection{Function Program}

The functional \ac{LP} dual to \eqref{eq:peak_cvar_meas} will have variables {$u \in \R$,} $v \in \cs([0,T]\times X)$ and ${w} \in C(\R)$ (dual to \eqref{eq:peak_cvar_meas_flow}-\eqref{eq:peak_cvar_meas_cvar}). 
\begin{thm} \label{thm:cvar_strong_dual}
The strong-dual program of \eqref{eq:peak_cvar_meas} with duality $d^*_c {=} p^*_c$ under A1-A4 is
\begin{subequations}
\label{eq:peak_cvar_cont}
%\begin{align}
%    d^*_{c} =& \inf_{\beta \in \R} \quad  \int_{X_0} v(0, x_0) d\mu_0(x_0) + \beta \label{eq:lag_cost_cvar} \\
%    & \forall (t, x) \in [0, T] \times X: \nonumber \\
%    & \qquad  \Lie v(t,x) \leq 0  \label{eq:lag_occ_meas_cvar}\\
%    & \forall (t, x) \in [0, T] \times X:\nonumber \\
%    &\qquad  v(t, x) - h(p(x)) \geq 0 & &   \label{eq:lag_stop_meas_cvar} \\
%    & \epsilon h(q) +  \beta \geq q & & \forall q \in \R \\
%    & h(q) \geq 0 & & \forall q \in \R \\
%    & h \in C(\R), \ v \in \cs([0, T] \times X).
%\end{align}
\begin{align}
    d^*_{c} =& \inf \quad  {u + \inp{v(0,\bullet)}{\mu_0}}\\ %\int_{X_0} v(0, x_0) d\mu_0(x_0) \label{eq:lag_cost_cvar} \\
    & \Lie v \leq 0 \label{eq:lag_occ_meas_cvar}\\
    & v \geq {w \, \circ \, }p \label{eq:lag_stop_meas_cvar} \\
    & {u \ + \ } \epsilon \, {w} \geq \mathrm{id}_\R \\
    & {w} \geq 0 \\
    & {u \in \R}, v \in \cs([0, T] \times X) {, w \in C(\R)}.
\end{align}
\end{subequations}
\end{thm}
\begin{proof}
    See Appendix \ref{app:duality_cvar}.
\end{proof}

%\eqref{eq:lag_cost}