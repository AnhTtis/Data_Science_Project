\section{Introduction}
\label{sec:introduction}

This paper analyzes maximal $(1-\epsilon)$-quantile statistics of a state function $p(x)$ for \ac{SDE} trajectories evolving in a compact set $X$. An example of this type of quantile statistic for trajectory analysis is in establishing that there exists at least one time with a 1\% chance of the aircraft exceeding a height of 100 meters.
% is a 99\% chance that the height of an aircraft at any particular time is less than or equal to 100 meters
This task of quantile estimation is related to peak and \ac{VAR} estimation, and will also be referred to as the `chance-peak' problem.

% Problem \eqref{eq:peak_chance} finds the supremal value $\gamma$ such that $[p(x)\leq \gamma]$  will hold at every time $t \in [0, T]$ with a probability of $1-\epsilon$. At the chosen stopping time $t^*$, there is a $\epsilon$ chance that the value of $p(x)$ is above $P^*$ by \eqref{eq:peak_chance_prob}. 

% \urg{Check for correctness of exposition. I am missing tons of references about Chance and VaR.}
% The chance-peak problem involves a $(1-\epsilon)$-chance constraint that holds at every time. 

The $\epsilon$-\ac{VAR} is the value at which there is an $\epsilon$-probability of exceedance \cite{jorion2000value}. Control and portfolio design typically aims to minimise the \ac{VAR}. One specific \ac{VAR}-upper-bounding coherent risk measure \cite{artzner1999coherent} that results in convex programs is the conditional \ac{VAR} risk measure \cite{pflug2000some, rockafellar2002conditional}.  The conditional \ac{VAR} has been utilized for stochastic optimal control in \cite{cmiller2017optimal}, and for approximation of discrete-time risk-bounded sets using exponential and logarithmic inequalities with Markov Decision Processes in \cite{chapman2021risk}.  
In contrast, the chance-peak approach upper-bounds maximum \ac{VAR} of the continuous-time \ac{SDE} state distribution of $x(t)$ across all times. We will solve this problem by maximizing the Cantelli and \ac{VP} upper bounds for the \ac{VAR}.
% \The hard probability constraint in \eqref{eq:peak_chance_prob} 
% will be relaxed to 
% to convex constraints on the moments of $\mu_{t^*}$ through the use of 

Chance constraints are an adjacent topic to \ac{VAR} optimization, in which a probability inequality must hold as a hard constraint. Chance-constrained programs % Chance-constrained programs 
have a wide variety of application in control theory \cite{charnes1958cost, lagoa2005probabilistically, ben2009robust}, and are generally intractable to solve explicitly. Approximation methods for chance constraints include the Cantelli \cite{cantelli1929sui} and \ac{VP} \cite{vysochanskij1980justification} inequalities, and application of these tail-bounds in control include \cite{wang2020non, han2022non}. 
The scenario approach for randomized constraint generation will converge in probability to the chance-constrained optimum, but carries a risk of failure and may require a large number of samples \cite{calafiore2005uncertain}.
The moment-\ac{SOS} hierarchy of \acp{SDP} will converge to the chance-constrained optimal solution under appropriate boundedness conditions \cite{jasour2015chance}.

% \urg{History of chance constraints, their hardness}


% The probability constraint in \eqref{eq:peak_chance_prob} will be relaxed to convex constraints on the moments of $\mu_{t^*}$ through the use of the Cantelli

The chance-peak problem is also  related to a family of optimal stopping problems which can be solved using occupation measures. The work in \cite{lewis1980relaxation} expressed optimal control problems of \acp{ODE} as an infinite-dimensional \ac{LP} in an initial, terminal, and occupation measure. The peak estimation problem to maximize a state function $p(x)$ is an instance of optimal control with free terminal time and zero running cost. The work in \cite{cho2002linear} generalizes this \ac{LP} to the stochastic case to find the maximum expectation of $p(x)$  when dynamics are phrased in terms of their infinitesimal generator (Feller process). Such \acp{LP} will converge to the true solution of the stopping problem under mild convergence, regularity, and well-posedness assumptions. The moment-\ac{SOS} hierarchy of finite-dimensional \acp{SDP} will converge to the infinite-dimensional \ac{LP} optimum if all problem data (e.g., dynamics, constraint sets) are polynomial-representable \cite{lasserre2009moments}. This convergent \ac{SDP} approach has been used for optimal control \cite{henrion2021moment}, peak estimation \cite{fantuzzi2020bounding} including compact-valued uncertainty \cite{miller2021uncertain}, expectation-maximization of \levy processes \cite{kashima2010optimization}, and option pricing \cite{lasserre2006pricing}.
Other instances of the moment-\ac{SOS} hierarchy used to solve stochastic safety problems include Barrier certificates \cite{prajna2007framework}, infinite-time averages \cite{fantuzzi2016bounds}, and Reach-Avoid sets \cite{xue2022sdereachavoid}. 

