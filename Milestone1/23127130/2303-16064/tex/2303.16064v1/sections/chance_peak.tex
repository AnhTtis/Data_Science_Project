\section{Peak Value-at-Risk Estimation}
\label{sec:meas}
% The main theory. Display the chance-peak program in terms of occupation measures.

This section will present the chance-peak problem statement, and will also derive the infinite-dimensional \ac{SOCP} to upper bound the chance-peak quantile statistic.

\subsection{Problem Statement}

Let $\epsilon \in [0, 1]$ be a value for the quantile statistic, $X$ be a compact set, $X_0 \subseteq X$ be a set of initial conditions, and \eqref{eq:sde_sol} be the solution to an \ac{SDE} evolving from $x(0) \in X_0$ that remains within $X$ until it stops. For a given initial probability distribution $\mu_0 \in \Mp{X_0}$, and for all $t \in [0, T]$, let $x(t)$ be the stochastic process of \eqref{eq:sde_sol} at time $t$, and let $\mu_t \in \Mp{X}$ be its probability distribution (with $x(t)$ stopping at $\partial X$). 
\subsubsection{Assumptions}
The following assumptions will be posed throughout this paper,
\begin{itemize}
    \item[A1] The set $[0, T] \times X$ is compact and $X_0 \subseteq X$.
    \item[A2] The functions $(f, g)$ satisfy \eqref{eq:lip_growth}.
    \item[A3] The state function $p(x)$ is continuous on $X$.
    \item[A4] The  initial measure $\mu_0\in\Mp{X_0}$ is a given probability distribution $(\inp{1}{\mu_0}=1)$.
    % \item[A4] If the 
\end{itemize}
%\subsubsection{Quantile/VaR Problem}
\subsubsection{VaR Problem}
\begin{prob}
\label{prob:quantile}
%The chance-peak problem to find the maximal $(1-\epsilon)$-quantile statistic of  $p(x)$ is,
%\begin{subequations}
%\label{eq:peak_chance_all}
%\begin{align}
%    P^* = & \inf_{\beta \in \R}  \quad \beta \\
%     & dx = f(t, x) dt + g(t, x) dw  \\
%     & \quad \text{from $t=0$ until a stopping time of} \ \tau_X \wedge T \label{eq:peak_chance_all_stop}\\
%     & x(0) \sim \mu_0 \\
%     & \forall t \in [0, T]: \ \prb_{\mu_{t}}[{p(x(t)) \leq \beta}] \geq 1-\epsilon. \label{eq:peak_chance_all_prob}
%\end{align}
%\end{subequations}
%\end{prob}

%Constraint \eqref{eq:peak_chance_all_prob} ensures that the probability of $p(x)$ exceeding $P^*$ at any particular time $t \in [0, T]$ is less than or equal to $\epsilon$. We use a probability duality result to formulate the $\forall t$ constraint in  \eqref{eq:peak_chance_all_prob} as a $\exists t$ constraint,
%\begin{lem}
%\label{lem:prob_duality}
%Let $\mu_t$ be a $t$-indexed family of probability distributions. Then the following two optimization problems have the same solution by duality:
%\begin{equation}
%    \begin{array}{r}
%         \inf \left\{\beta \in \R \; \middle| \; \forall t \in [0,T], \prb_{\mu_{t}}[{p(x(t)) \leq \beta}] \geq 1-\epsilon \right\} \\\\
%         = \sup \left\{\gamma \in \R \; \middle| \; \exists t^* \in [0,T] ; \prb_{\mu_{t^*}}[{p(x(t)) \geq \gamma}]\geq \epsilon\right\}
%    \end{array}
%\end{equation}
%\begin{align}
%    &\inf_{\beta\in \R} \beta: & & \forall t \in [0, T]: \ \prb_{\mu_{t}}[{p(x(t)) \leq \beta}] & &\geq 1-\epsilon \nonumber \\
%    =& \sup_{\gamma \in \R} \gamma:  & & \exists t^* \in [0, T]: \ \prb_{\mu_{t^*}}[{p(x(t)) \geq \gamma}] & &\geq \epsilon
%\end{align}
%\end{lem}
%{\color{red}Matt: Proof or ref needed here}
% \begin{proof}
% \urg{proof goes here after I get back.}
% \end{proof}
% In the case where $\mu_t$ represents the distribution of the state $x(t)$ at time $t$, the $\epsilon$-quantile estimation problem for a state function $p(x)$ with respect to \iac{SDE} given the initial distribution $\mu_0$ is a generically nonconvex optimization problem with respect to a stopping time $t^*$ and a bound $\gamma$,
% \todo[inline]{Matteo: not sure that $t^*$ is a decision variable of the global optimization pb, or there's a notation conflict with \eqref{eq:peak_chance_stop}; more generally, I'm having a hard time relating equation \eqref{eq:peak_chance} to the description that follows it; I suspect constraints are missing somewhere.}


%We will therefore consider the following reformulation of Problem \ref{prob:quantile} by using  Lemma \ref{lem:prob_duality}:
% By using Lemma \ref{lem:prob_duality}, we therefore will consider the following reformulation of Problem \ref{prob:quantile}
%\begin{thm}
%The following supremal chance-peak $\epsilon$-\ac{VAR} problem possesses the same optimal value as \eqref{eq:peak_chance_all}:
The chance-peak problem to find the $\epsilon$-VaR of  $p(x)$ is
\begin{subequations}
\label{eq:peak_chance}
\begin{align}
    P^* = & \sup_{t^* \in [0, T] }  VaR_{\epsilon}(p_\# \mu_{t^*}) \label{eq:peak_chance_obj}\\
     & dx = f(t, x) dt + g(t, x) dw  \\
     & \quad \text{from $t=0$ until a stopping time of} \ \tau_X \wedge t^* \label{eq:peak_chance_stop}\\
     & x(0) \sim \mu_0.
    %  & VaR_{\epsilon}(\mu_{t^*})
    %  & \prb_{\mu_{t^*}}[{p(x(t)) \geq \gamma}] \geq \epsilon. \label{eq:peak_chance_prob}
\end{align}
\end{subequations}
% \quad \gamma
\end{prob}
The pushforward $p_\# \mu_{t^*}$ from \eqref{eq:peak_chance_obj} is the univariate probability distribution of $p(x)$ at the state distribution $x \sim \mu_{t^*}$.
%  & \prb_{\mu_{t^*}}[{p(x(t)) \geq \gamma}] \geq \epsilon. \label{eq:peak_chance_prob}
%\end{thm}
%\begin{proof}
%This arises from applying Lemma \ref{lem:prob_duality}, replacing the $\beta$-infimum over every time $t \in [0, T]$ with a $\gamma$-supremum over a specific time $t^*$.
%\end{proof}
% Constraint \eqref{eq:peak_chance_prob} ensures that the probability of $p(x)$ exceeding $P^*$ at time $t^*$ is larger than or equal to $\epsilon$.

\subsubsection{Tail-Bound Upper Bound}
Let $r$ be the constant factor multiplying  $\sigma$ in \eqref{eq:tail_bounds} such that
\begin{align}
\label{eq:tail_bounds_constant}
    r^{cant} &= \sqrt{1/(\epsilon)-1} & r^{VP} &=  \sqrt{4/(9\epsilon)-1}.
\end{align}
It is further assumed that the VP-bound will only be used if its conditions are satisfied ($\epsilon \leq 1/6$, unimodal).
The distribution of $p(x)$ with respect to the state 
 distribution $\mu_{t^*}$  is univariate, for which the relation in \eqref{eq:var_center} and the constants in \eqref{eq:tail_bounds} can be used to upper-bound on Problem \ref{prob:quantile}. We will use the notation $\inp{p^2}{\mu_{t^*}}$ to refer to $\inp{p(x)^2}{\mu_{t^*}(x)}$.
\begin{prob}
\label{prob:tail_chance}
The tail-bound program that upper-bounds the chance-peak \eqref{eq:peak_chance} with constant $r$ is
    \begin{subequations}
\label{eq:peak_chance_tail}
\begin{align}
    P^*_r = & \sup_{t^* \in [0, T]} r\sqrt{\inp{p^2}{\mu_{t^*}} - \inp{p}{\mu_{t^*}}^2}  + \inp{p}{\mu_{t^*}}\\
     & dx = f(t, x) dt + g(t, x) dw  \\
     & \quad \textrm{from $t=0$ until a stopping time of} \ \tau_X \wedge t^* \label{eq:peak_chance_tail_stop}\\
     & x(0) \sim \mu_0.
\end{align}
\end{subequations}
\end{prob}


% The hard probability constraint in \eqref{eq:peak_chance_prob} ensures that 
% The stopping rule in \eqref{eq:peak_chance_stop} will freeze the trajectory evolution at the first event between reaching time $t^*$ or whenever $x(t)$ hits the boundary $\partial X$. Problem \eqref{eq:peak_chance} chooses a stopping time $t^*$ such that there is an at least $\epsilon$-chance that $p(x(t))$ exceeds $\gamma$. A consequence of this statement is that $\prb_{\mu_{t}}[p(x(t)) \leq \gamma] \geq 1-\epsilon$ for every stopping time $t \in [0, T]$. Replacing the $\geq$ sign in \eqref{eq:peak_chance_prob} with a $\leq$ sign would allow $\gamma$ to become unbounded towards $\infty$. \urg{might be too much discussion}
% \subsection{Measure Program}

\subsection{Nonlinear Measure Program}

Problem \ref{prob:tail_chance} can be upper-bounded by an infinite-dimensional nonlinear program in a given initial probability distribution $\mu_0$, terminal measure $\mu_\tau$, and relaxed occupation measure $\mu$, using the generator $\Lie$ in \eqref{eq:lie} as
    \begin{subequations}
\label{eq:peak_chance_meas}
\begin{align}
    p^*_r = & \sup r\sqrt{\inp{p^2}{\mu_{\tau}} - \inp{p}{\mu_{\tau}}^2}  + \inp{p}{\mu_{\tau}} \label{eq:peak_chance_meas_obj}\\
     & \mu_\tau = \delta_0 \otimes \mu_0 + \Lie^\dagger \mu \label{eq:peak_chance_meas_lie}\\
    & \mu_\tau, \ \mu \in \Mp{[0, T] \times X}.
\end{align}
\end{subequations} 

\begin{thm}
\label{thm:upper_bound_nonlinear}
Program \ref{eq:peak_chance_meas} is an upper bound on \eqref{eq:peak_chance_tail} with $p^*_r \geq P^*_r$ under A1-A4.
\end{thm}
\begin{proof}
Let $t^*$ be a stopping time in $[0, T]$, and let $x_0 \in X_0$ be an initial condition. Measures $(\mu_0, \mu, \mu_\tau)$ that satisfy \eqref{eq:peak_chance_meas_lie} may be constructed from this $(t^*, x_0)$ by $\mu_{t^*}$ as the the state distribution of  \eqref{eq:sde_sol} at time $t^*$ given $\mu_0$, and $\mu$ as the occupation measure in \eqref{eq:avg_free_occ} associated to this \ac{SDE} trajectory with distribution $\mu_0$. Because the feasible set to constraint \eqref{eq:peak_chance_meas_lie} contains measures induced by all possible provided \ac{SDE} trajectories starting from $\mu_0$, it holds that $p^*_r \geq P^*_r$.
\end{proof}
\begin{rmk}
The initial distribution $\mu_0\in \Mp{X_0}$ may be optimized to find a supremal $p^*_r$ over all probability distributions in $X_0$ by adding $\mu_0$ as a variable and adding the constraint $\inp{\mu_0}{1} = 1$ to \eqref{eq:peak_chance_meas}.
\end{rmk}
\subsection{Measure Second-Order Cone Program}
The nonlinear measure program \eqref{eq:peak_chance_meas} may be recast as an infinite-dimensional convex \ac{SOCP}.% with a finite-dimensional \ac{SOC} constraint.
\begin{lem}
\label{lem:sqrt}
Let $J_r(a, b) = r \sqrt{b - a^2} + a$ be the objective  \eqref{eq:peak_chance_meas_obj} with $a=\inp{p(x)}{\mu_\tau}$ and $b = \inp{p(x)^2}{\mu_\tau}$. For any convex set $C \in \R \times \R_+$ with $(a, b) \in C$, the following pair of programs have the same optimal value (in which $Q^3 = \{([s_1, s_2, s_3], \kappa) \in \R^3 \times \R_+ \mid \norm{s}_2 \leq \kappa \}$  is \iac{SOC} cone):
\begin{align}
    &\sup_{(a, b) \in C} a + r \sqrt{b-a^2}  \\
    &\sup_{(a, b) \in C, \ z \in \R} a + r z: \ ([1-b, 2z, 2a],  1+b) \in Q^3. \label{eq:square_root_socp}
\end{align}
\end{lem}
\begin{proof}
The new variable $z$ is introduced under the constraint $\sqrt{b-a^2} \geq z$, implying that $ z^2 + a^2 \leq b $. The \ac{SOCP} equivalence follows from the power-representation of $\sqrt{b-a^2}$ from \cite{alizadeh2003second, yalmip2009sqrt}, with the steps of
\begin{subequations}
\begin{align}
    &([1 - b, 2z, 2a],  1+b) \in Q^3  \\
    & (1-b)^2 + 4(z^2 + a^2) \leq (1+b)^2 \\
    & (1+b^2) - 2b + 4(z^2 + a^2) \leq (1+b^2) + 2b \\
    & 4(z^2 + a^2) \leq 4b.
\end{align}
\end{subequations}
\end{proof}

\begin{thm}
\label{thm:upper_bound_socp}
An infinite-dimensional \ac{SOCP} with the same optimal value and set of feasible solutions as \eqref{eq:peak_chance_meas} given $\mu_0$ is
\begin{subequations}
\label{eq:var_meas_soc}
\begin{align}
    p^*_{r} = &\sup \quad r z  + \inp{p}{\mu_\tau} \label{eq:var_meas_obj_soc}\\
     & \mu_\tau = \delta_0 \otimes \mu_0 + \Lie^\dagger \mu \label{eq:var_meas_lie_soc}\\
    & u = [1-\inp{p^2}{\mu_\tau}, \ 2 z, \ 2 \inp{p}{\mu_\tau}] \label{eq:var_meas_con_soc_def}\\
    & (u, 1 + \inp{p^2}{\mu_\tau}) \in Q^3 \label{eq:var_meas_con_soc}\\
    & \mu, \ \mu_\tau \in \Mp{[0, T] \times X}, z \in \R, u \in \R^3.\label{eq:var_meas_con_meas}
\end{align}
\end{subequations}
\end{thm}
\begin{proof}
This results from an application of Lemma \ref{lem:sqrt} to the objective term \eqref{eq:peak_chance_meas_obj}. The optimization variables are now $(\mu_\tau, \mu, z, u)$. %{\color{red}Matt: wrong. $s$ is also an optimization variable. The best way to write our problem in a standard way would be under the form
%\begin{align*}
%    \sup & \langle c , x \rangle & \qquad && \inf & \langle y , b \rangle \\
%    & x \in K & \qquad && & A'y - c \in K^* \\
%    & Ax = b \in Y & \qquad && & y \in Y^*
%\end{align*}
%Here it yields (please double check all my computations)
%\begin{align*}
%    p^*_r = & \sup rz + \langle p, \mu_\tau\rangle & \qquad && d^*_r = & \inf \langle v(0,\cdot) , \mu_0 \rangle + y_1 + y_4 \\
%    & \mu \in \Mp{[0,T]\times X} & \qquad && & \Lie v \leq 0 \\
%    & \mu_\tau \in \Mp{[0,T]\times X}& \qquad && & v + y_1 p^2 - 2y_3 p - y_4 p^2 \geq p \\
%    & z\in\R & \qquad && & -2 y_2 = r \\
%    & q \in Q^3 & \qquad && & y \in Q^3 \\
%    & \mu_\tau - \Lie^\dagger \mu = \delta_0 \mu_0 & \qquad && & v \in C^2([0,T]\times X) \\
%    & q + [\langle p^2,\mu_\tau\rangle, -2z, -2\langle p, \mu_\tau \rangle, -\langle p^2,\mu_\tau \rangle] = [1,0,0,1] & \qquad && & y \in \R^4
%\end{align*}}
\end{proof}

\begin{cor}
Program \eqref{eq:var_meas_soc} is convex.
\end{cor}
\begin{proof}
The objective \eqref{eq:var_meas_obj_soc} is affine in $(z, \mu_\tau)$. Constraints \eqref{eq:var_meas_lie_soc}-\eqref{eq:var_meas_con_meas} are convex (affine for \eqref{eq:var_meas_lie_soc} and \ac{SOC} for \eqref{eq:var_meas_con_soc}), ensuring convexity of \eqref{eq:var_meas_soc}.
\end{proof}

\begin{rmk}
Problem \eqref{eq:var_meas_soc} has an infinite-dimensional affine constraint in \eqref{eq:var_meas_lie_soc} and a finite-dimensional \ac{SOC} constraint in \eqref{eq:var_meas_con_soc}.
\end{rmk}
% \begin{thm}
% Program \eqref{eq:peak_chance_meas} is convex
% \end{thm}
% \begin{proof}
% The affine constraints \eqref{eq:peak_chance_meas_lie} and \eqref{eq:peak_chance_meas_mass} are linear in the measures $(\mu_0, \mu, \mu_\tau)$. The objective \eqref{eq:peak_chance_meas_obj} may be represented using a function $h_r$
% \begin{equation}
%     h_r(x, y) = r \sqrt{y - x^2} + x, \label{eq:h_func}
% \end{equation}
% as $h_r(\inp{p (x)}{\mu_{\tau}}, \inp{p (x)^2}{\mu_{\tau}})$. The function $h_r(x, y)$ is concave when $y\geq 0$, and is strictly concave when $y>0$ and $r>0$. This strict concavity may be deduced from the positive-definite Hessian of $h_r$,
% \begin{equation}
%     \nabla^2 h(x, y; k) = \frac{1}{(y-x^2)^{3/2}}\begin{bmatrix} -kx^2 - k(y-x^2) & kx/2 \\ kx/2 & -k/4 \end{bmatrix}.
% \end{equation}
% \end{proof}


% \end{prob}
% In the same way that \eqref{eq:peak_meas}


% Presentation of Cantelli and VP bounds. Utilization of these bounds to define upper-bounds on the $1-\epsilon$ quantile in terms of a Measure program.


% Reformulation of the Cantelli/VP constraints into a Second-Order-Cone constraint.