\section{Extensions}
\label{sec:extensions}
% \urg{We can include some of these, or wait until the extended (arxiv)/full journal of the paper to put these in.}
This section outlines extensions to the developed chance-peak framework.

\subsection{Exit-Time Statistics}

% \textcolor{red}{Matt: our contribution here is unclear, maybe add some equations to explain how to combine this paper with \cite{henrion2021moment}}

This extension builds upon the work of \cite{henrion2021moment} in estimating expectations of functions upon first exit of the region $X$. The exit time distribution of the \ac{SDE} in \eqref{eq:sde_sol} is the class of trajectories that stop according to the stopping time $\tau_X$.   
The program \eqref{eq:peak_meas} is modified in \cite{henrion2021moment} by adding a support constraint $\mu_\tau \in \Mp{\partial X}$). Averaged statistics of the exit time distribution (expectations of $p_\#{\mu_\tau}$) may be computed by substituting different choices of $p$ into the $\partial X$-adjusted \eqref{eq:peak_meas}, and this expectation is bounded from above and below with no relaxation gap in \cite{henrion2021moment}. An example of this kind of statistic is finding the mean (stopping) time at which the trajectory exits by estimating $\inp{t}{\mu_\tau}$.

We can adapt this methodology to bound   $1-\epsilon$ quantile values of the exit time distribution by restricting $\mu_\tau \in \Mp{\partial X}$ constraint in Program \eqref{eq:var_meas_soc} and \eqref{eq:var_meas_con_meas} by
\begin{subequations}
\label{eq:var_meas_exit_soc}
\begin{align}
    p^*_{r} = &\sup \quad r z  + \inp{p}{\mu_\tau} \label{eq:var_meas_exit_obj_soc}\\
     & \mu_\tau = \delta_0 \otimes \mu_0 + \Lie^\dagger \mu \label{eq:var_meas_exit_lie_soc}\\
    & u = [1-\inp{p^2}{\mu_\tau}, \ 2 z, \ 2 \inp{p}{\mu_\tau}] \label{eq:var_meas_exit_con_soc_def}\\
    & (u, 1 + \inp{p^2}{\mu_\tau}) \in Q^3 \label{eq:var_meas_exit_con_soc}\\
    & \mu \in \Mp{[0, T] \times X}, z \in \R, u \in \R^3.\label{eq:var_meas_exit_con_meas} \\
    & \mu_\tau \in \Mp{[0, T] \times \partial X}.
\end{align}
\end{subequations}

The optimal value \eqref{eq:var_meas_exit_soc} is a (Cantelli or \ac{VP}) upper-bound on the exit time quantity $VaR_\epsilon(p_\# \mu_\tau)$.

% The terminal measure $\mu_\tau$ from \eqref{eq:var_meas_con_meas} is restricted to be supported on the boundary $\partial X$ ($\mu_\tau \in \Mp{\partial X}$). 

\subsection{Distance Estimation}
\label{sec:distance}

The chance-peak methodology developed in this paper can be applied towards bounding (probabilistically) the distance of closest approach to an unsafe set. Let $X_u \subset X$ be an unsafe set, and let $c(x, y)$ be a metric in $X$. The point-set distance function with respect to $X_u$ is $c(x; X_u) = \inf_{y \in X_u} c(x, y)$. 
The output of a chance-distance program is an infimal value $c^*$ such there that exists some time in which the probability of traveling closer than $c^*$ to $X_u$ is at least $\epsilon$.
% an infimal value $C^*$ such that $\text{Prob}(c(x(t \mid x_0); X_u) \geq C^*\} \geq 1-\epsilon$ holds for all times $t \in T \wedge \tau_X$ and initial conditions $x_0 \sim \mu_0$. 
% As an example, the probability of passing within $C^*=1$ meter to $X_u$ is 99\% for every state distribution $\mu_t$ starting from $\mu_0$.

% no state distribution $\mu_t$ will pass within a closest distance of 1 meter to the unsafe set $X_u$ at an 

The chance-distance $C^*$ is the negative of the bound $P^*$ obtained from solving \eqref{eq:peak_chance} with objective $p(x) = -c(x; X_u)$. Because the objective $c(x; X_u)$ is not generally polynomial (even when $c(x, y)$ is polynomial), the \ac{LMI} \eqref{eq:chance_lmi} cannot directly be posed in terms of $c(x; X_u)$. One method to maintain a polynomial structure is to add time-constant states $dy = \0 dt + \0 dw$ to dynamics \eqref{eq:sde} in $x$ and form the state support set $(x, y) \in X \times X_u$. When $X_u$ is full-dimensional inside $X \subset \R^n$, the occupation measure $\mu(t, x, y) \in \Mp{[0, T] \times X \times Y}$ will have a moment matrix of size $\binom{1+2n+d}{d}$ at each fixed degree $d$.

This size can be reduced using the method in \cite{miller2022distance_short}, in which the peak measure $\hat{\mu}_\tau \in \Mp{[0, T] \times X \times Y}$ is decomposed into a joint measure $\eta(x, y) \in \Mp{X \times Y}$  and a peak measure $\mu_\tau(t, x) \in \Mp{[0, T] \times X}$ that are equal in the $x$ marginal.
The chance-distance \ac{SOCP} under this decomposition is
\begin{subequations}
\label{eq:var_meas_distance}
\begin{align}
    p^*_{r} = &\sup_{z\in \R} \quad r z  + \inp{-c(x, y)}{\eta(x, y)} \\
     & \inp{v}{\mu_\tau} = \inp{v(0, x_0)}{\mu_0(x_0)} + \inp{\Lie v}{\mu} & & \forall v \in C^2([0, T] \times X) \\
     & \inp{\theta(x)}{\mu_\tau(t, x)} -\inp{\theta(x)}{\eta(x, y)} = 0 & & \forall \theta \in C(X) \label{eq:marg_distance}\\
    & s = [1-\inp{c^2}{\eta}, \ 2 z, \ 2 \inp{-c}{\eta}]\label{eq:csquared_dist}\\
    & (s, 1 + \inp{c^2}{\eta}) \in Q^3\\
    & \eta \in \Mp{X \times X_u} \\ 
    & \mu, \ \mu_\tau \in \Mp{[0, T] \times X}.
\end{align}
\end{subequations}

Constraint \eqref{eq:marg_distance} enforces equality in the $x$-marginals between $\mu_\tau$ and $\eta$.
The Moment matrices of $\eta$ and $\mu$ respectively in the \ac{LMI} program derived from \eqref{eq:var_meas_distance} have sizes $\binom{2n+d}{d}$ and $\binom{n+1+\tilde{d}}{\tilde{d}}$. Unfortunately, the squaring operation $\inp{c^2}{\eta}$ causes mixed multiplications in variables even when $c$ is additively separable as $c(x, y) = \sum_{i=1}^n c_i(x_i, y_i)$, thus forbidding the application of correlative sparsity \cite{waki2006sums} to reduce the complexity of \acp{LMI} from \eqref{eq:var_meas_distance}.

\subsection{Switching}

% \textcolor{red}{Matt: a proper introduction of the concepts related to switching systems (switching function etc) is missing here}

The chance-peak scheme may also be applied to switched stochastic systems. The methods outlined in this section are an extension of the \ac{ODE} approach from \cite{miller2021uncertain}, and are similar to duals of constraints found in \cite{prajna2007framework}.
Assume that there are $N_s \in \N$ subsystems indexed by $\ell=1..N_s$. Each subsystem has individual dynamics
\begin{align}
    dx &= f_\ell(t, x) dt + g_\ell(t, x). \label{eq:dyn_switch}
\end{align}
for each switching index $\ell = 1..N_s$.
A switched \ac{SDE} trajectory is a distribution $x(t)$ and a switching function $S: [0, T] \rightarrow (1..N_s)$ under the constraint that $x(t)$ satisfies \eqref{eq:dyn_switch} whenever $S(t) = \ell$ (the $\ell$-th subsystem is active). A specific trajectory of a switched \ac{SDE} starting from an initial point $x_0 \in X$ will be expressed as $x(t \mid x_0, S)$. No dwell time constraints are imposed on the switching sequence $S$; instead, switching can occur arbitrarily quickly in time.

Generators $\Lie_\ell$ may be defined for each subsystem in \eqref{eq:dyn_switch} according to $\forall v \in C^2([0, T] \times X$:
\begin{align}
    \Lie_\ell v(t, x) &= \partial_t v + f_\ell\cdot \nabla_x v + g_\ell^T(\nabla_{xx}^2v) g_\ell/2 & \forall \ell=1..N_s.
\end{align}

% A switching function $S: [0, T] \rightarrow (1..N_s)$ may be defined 
% The switching function $S: [0, T] \rightarrow (1..N_s)$ yields the resident subsystem of an \ac{SDE} at time $t$. A trajectory of \eqref{eq:dyn_switch} is a solution $x(t \mid x_0, S(\cdot))$ in which dynamics $(f_\ell, g_\ell)$ are followed with state distribution $\mu_t$ when the system satisfies $S(t) = \ell$.

% Following from \cite{miller2021uncertain} for peak estimation under switching uncertainty, chance-peak for switched systems may be accomplished by defining occupation measures $\mu_\ell \in \Mp{[0, T] \times X}$ and generators $\Lie_\ell$ for each subsystem according to \eqref{eq:dyn_switch}. 
Let $\mu \in \Mp{[0, T]\times X}$ be the total occupation measure of the switched \ac{SDE} trajectory $x(t \mid x_0, S)$.
The total occupation measure may be split into disjoint subsystem occupation measures $\forall \ell: \ \mu_\ell \in \Mp{[0, T] \times X}$ under the relation $\sum_{\ell=1}^{N_s} \mu_\ell = \mu$. 
The mass of a subsytem's occupation measure $\inp{1}{\mu_\ell}$ is the total amount of time that the trajectory $x(t \mid x_0, S)$ spends in subsystem $S(t) =\ell$.

 Dynkin's  equation \eqref{eq:dynkin} for switching-type uncertainty is
\begin{align}
    \mu_\tau = \delta_0 \otimes \mu_0 + \textstyle \sum_{\ell=1}^{N_s} \Lie_\ell^\dagger \mu_\ell.
\end{align}

% No additional modifications to the chance-peak setup in \eqref{eq:var_meas_soc} is necessary to include switching uncertainty. 
The chance-peak problem in \eqref{eq:var_meas_soc} modified  for switching uncertainty is
\begin{subequations}
\label{eq:var_meas_switch_soc}
\begin{align}
    p^*_{r} = &\sup \quad r z  + \inp{p}{\mu_\tau} \label{eq:var_meas_switch_obj_soc}\\
     & \mu_\tau = \delta_0 \otimes \mu_0 + \Lie^\dagger \mu \label{eq:var_meas_switch_lie_soc}\\
    & u = [1-\inp{p^2}{\mu_\tau}, \ 2 z, \ 2 \inp{p}{\mu_\tau}] \label{eq:var_meas_switch_con_soc_def}\\
    & (u, 1 + \inp{p^2}{\mu_\tau}) \in Q^3 \label{eq:var_meas_switch_con_soc}\\
    & \mu_\tau \in \Mp{[0, T] \times X}, z \in \R, u \in \R^3.\label{eq:var_meas_switch_con_meas} \\
    & \mu_\ell \in \Mp{[0, T] \times \partial X} & & \forall \ell=1..N_s.
\end{align}
\end{subequations}


% \urg{An SDE has independent, stationary increments and also has continuous paths. L\'{e}vy processes remove the requirement that sample paths need to be continuous. The resultant paths can jump a finite number of times in a finite range according to a Poisson point process, assuming that the distribution of the jump heights satisfy a boundedness constraint (L\'{e}vy measure). More information about L\'{e}vy processes are available at \cite{applebaum2004levy}.

% The work in \cite{kashima2010optimization} implemented peak-expectation estimation for \levy processes using (tempered) polynomial optimization. We can use Cantelli/VP bounds to do chance-peak on \levy processes.

% The tempering involved using exponential weights of the 1d state variable when defining all measures, and approximating an exponential*polynomial functions from below by polynomials using a discretezation scheme. It will be exceedingly difficult to generalize the tempering scheme to multiple dimensions due to the curse of dimensionality.

% Depending on time constraints, I might just do arbitrary switching between subsystems and leave it at that.

% } 