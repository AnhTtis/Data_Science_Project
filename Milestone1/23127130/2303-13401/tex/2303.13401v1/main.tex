
\documentclass[default,iicol]{sn-jnl}% Default with double column layout

\jyear{2022}%

%% as per the requirement new theorem styles can be included as shown below
% \theoremstyle{thmstyleone}%
% \newtheorem{theorem}{Theorem}%  meant for continuous numbers
% %%\newtheorem{theorem}{Theorem}[section]% meant for sectionwise numbers
% %% optional argument [theorem] produces theorem numbering sequence instead of independent numbers for Proposition
% \newtheorem{proposition}[theorem]{Proposition}% 
% %%\newtheorem{proposition}{Proposition}% to get separate numbers for theorem and proposition etc.

% \theoremstyle{thmstyletwo}%
% \newtheorem{example}{Example}%
% \newtheorem{remark}{Remark}%

% \theoremstyle{thmstylethree}%
% \newtheorem{definition}{Definition}%

\raggedbottom
% \unnumbered% uncomment this for unnumbered level heads

% ===== Import From Neurips Packages =====
\usepackage{dirtytalk}
\usepackage{enumitem}
\usepackage[utf8]{inputenc} % allow utf-8 input
\usepackage[T1]{fontenc}    % use 8-bit T1 fonts
\usepackage{hyperref}       % hyperlinks
\usepackage{url}            % simple URL typesetting
\usepackage{booktabs}       % professional-quality tables
\usepackage{amsfonts}       % blackboard math symbols
\usepackage{nicefrac}       % compact symbols for 1/2, etc.
\usepackage{microtype}      % microtypography
\usepackage{xcolor}         % colors
\usepackage{makecell, multirow}
\usepackage{algorithm}
\usepackage{amsthm}
\usepackage{algpseudocode}
\usepackage{wrapfig}
\usepackage{caption}
\usepackage{subcaption}
\newcommand{\bbox}{\text{bbox}}
\newcommand{\alphapck}{\alpha_\bbox}
\newcommand{\kcycle}{\text{k-CyPCK}}
\newcommand{\cycle}{\text{-CyPCK}}

\newcommand{\I}{\mathbf{I}}
\newcommand{\Ia}{\I^\text{a}}
\newcommand{\Ib}{\I^\text{b}}
\newcommand{\Iatob}{\I^\text{a $\rightarrow$ b}}
\newcommand{\F}{\mathbf{F}}
\newcommand{\Fa}{\F^\text{a}}
\newcommand{\Fb}{\F^\text{b}}
\newcommand{\f}{\mathbf{f}}
\newcommand{\fa}{\f^\text{a}}
\newcommand{\fb}{\f^\text{b}}
\newcommand{\p}{\mathbf{p}}
\newcommand{\pa}{\p^\text{a}}
\newcommand{\pb}{\p^\text{b}}
\newcommand{\A}{\boldsymbol{\Phi}_\text{align}}
\newcommand{\G}{\mathbf{G}}
\newcommand{\C}{\mathbf{C}}
\newcommand{\Ca}{\C^\text{a}}
\newcommand{\Cb}{\C^\text{b}}
\newcommand{\cc}{\mathbf{c}}
\newcommand{\cca}{\cc^\text{a}}
\newcommand{\ccb}{\cc^\text{b}}
\newcommand{\Irec}{\I_\text{Recon}}
\newcommand{\M}{\mathbf{M}}
\newcommand{\Mrec}{\M_\text{Recon}}
\newcommand{\loss}{\mathcal{L}}
\newcommand{\T}{\mathcal{T}}
\newcommand{\W}{\mathcal{W}}
\newcommand{\Id}{\mathcal{I}}

\input{Utils/ju_math.tex}

\newcommand{\pygranso}{\texttt{PyGRANSO}}
\newcommand{\change}{\textcolor{black}}

\begin{document}

\title[ ]{Optimization and Optimizers for Adversarial Robustness}
% \title[Article Title]{Article Title}
%%=============================================================%%
%% Prefix	-> \pfx{Dr}
%% GivenName	-> \fnm{Joergen W.}
%% Particle	-> \spfx{van der} -> surname prefix
%% FamilyName	-> \sur{Ploeg}
%% Suffix	-> \sfx{IV}
%% NatureName	-> \tanm{Poet Laureate} -> Title after name
%% Degrees	-> \dgr{MSc, PhD}
%% \author*[1,2]{\pfx{Dr} \fnm{Joergen W.} \spfx{van der} \sur{Ploeg} \sfx{IV} \tanm{Poet Laureate} 
%%                 \dgr{MSc, PhD}}\email{iauthor@gmail.com}
%%=============================================================%%

\author*[1]{\fnm{Hengyue} \sur{Liang}}\email{liang656@umn.edu}

\author[2]{\fnm{Buyun} \sur{Liang}}\email{liang664@umn.edu}

\author[2]{\fnm{Le} \sur{Peng}}\email{peng0347@umn.edu}

\author[3]{\fnm{Ying} \sur{Cui}}\email{yingcui@umn.edu}
% \equalcont{These authors contributed equally to this work.}
\author[4]{\fnm{Tim} \sur{Mitchell}}\email{tim.mitchell@qc.cuny.edu}

\author*[2]{\fnm{Ju} \sur{Sun}}\email{jusun@umn.edu}

\affil[1]{\orgdiv{Department of Electrical \& Computer Engineering}, \orgname{University of Minnesota}%\orgaddress{\city{Minneapolis}, \postcode{55454}, \state{MN}, \country{USA}}
}
\affil[2]{\orgdiv{Department of Computer Science \& Engineering}, \orgname{University of Minnesota}%\orgaddress{\city{Minneapolis}, \postcode{55454}, \state{MN}, \country{USA}}
}
\affil[3]{\orgdiv{Department of Industrial \& Systems Engineering}, \orgname{University of Minnesota}%\orgaddress{\city{Minneapolis}, \postcode{55454}, \state{MN}, \country{USA}}
}
\affil[4]{\orgdiv{Department of Computer Science}, \orgname{Queens College, City University of New York}%\orgaddress{\city{Minneapolis}, \postcode{55454}, \state{MN}, \country{USA}}
}

%%==================================%%
%% sample for unstructured abstract %%
%%==================================%%

\abstract{Empirical robustness evaluation (RE) of deep learning models against adversarial perturbations entails solving nontrivial constrained optimization problems. Existing numerical algorithms that are commonly used to solve them in practice predominantly rely on projected gradient, and mostly handle perturbations modeled by the $\ell_1$, $\ell_2$ and $\ell_\infty$ distances. In this paper, we introduce a novel algorithmic framework that blends a general-purpose constrained-optimization solver~\pygranso~\textbf{w}ith \textbf{C}onstraint-\textbf{F}olding (PWCF), which can add more reliability and generality to the state-of-the-art RE packages, e.g., \texttt{AutoAttack}. Regarding \emph{reliability}, PWCF provides solutions with stationarity measures and feasibility tests to assess the solution quality. %, and is generally free from delicate hyperparameter tuning. 
For \emph{generality}, PWCF can handle perturbation models that are typically inaccessible to the existing projected gradient methods; the main requirement is the distance metric to be almost everywhere differentiable. Taking advantage of PWCF and other existing numerical algorithms, we further explore the distinct patterns in the solutions found for solving these optimization problems using various combinations of losses, perturbation models, and optimization algorithms. We then discuss the implications of these patterns on the current robustness evaluation and adversarial training.}

\keywords{deep learning, deep neural networks, adversarial robustness, adversarial attack, adversarial training, minimal distortion radius, robustness radius, constrained optimization, sparsity}

\maketitle

% Importance and appeal of children's drawings
Children's depictions of the human figure are highly expressive and varied.
As one of the very first subjects children attempt to draw, the representation begins as an almost unintelligible cloud of scribbles. 
As the child grows, their representation of the human figure becomes more developed and is extended to graphically represent many different types of characters: people, animals, and even personified objects (see Figure 1).

Who among us has not wished, either as a child or as an adult, to see such figures come to life and move around on the page?
Sadly, while it is relatively fast to produce a single drawing, creating the sequence of images necessary for animation is a much more tedious endeavor, requiring discipline, skill, patience, and sometimes complicated software.
As a result, most of these figures remain static upon the page.

% We built a system to animate them.
Inspired by the importance and appeal of the drawn human figure, we design and build a system to automatically animate it given an in-the-wild photograph of a child's drawing. 
Our system is fast, intuitive, and robust to much of the variation present in these types of drawings, making it well-suited to allow our target audience--children--to see their own characters coming to life.
The system is comprised of four stages: figure detection, segmentation masking, pose estimation/rigging, and animation. 
We describe each stage and identify common causes of failure in each. 
For object detection and pose estimation, we make use of existing computer vision models designed to detect human figures and joints in photographs; we fine-tune these models for use with children's drawings.
For segmentation, we present a straightforward, image processing-based method that, for animation purposes, is more useful and accurate than segmentation masks obtained from a fine-tuned object detection model.
During the animation step, we take advantage of the \textit{twisted perspective} commonly seen in children’s drawings to retarget motion capture data onto the character in a novel and appealing way.

% We use existing machine learning models. However, given the wide domain gap it's not clear how much fine-tuning data was needed. So we ran some experiments to find out and report it.
While our system leverages existing models and techniques, most are not directly applicable to the task due to the many differences between photographic images and simple pen and paper representations. 
To this end, we couple the presentation of our system with a set of experiments exploring the relationship between fine-tuning training set size and success rates.
We also include a perceptual study validating viewer preference for incorporating \textit{twisted perspective} into the motion retargeting step.

We validate the desirability and appeal of our system by building and publicly releasing a version of it as the \AD Demo \,\cite{animateddrawings}.
Launched in December 2021, this demo has been used by millions of people around the world to animate their children's drawings.
Inspired by this reception, our second contribution is The Amateur Drawings Dataset: \hjs{180,000 drawings and user-accepted annotations collected, with consent, through the demo. See Section \ref{sec:UI} for a description of how the annotations were generated.}
We believe this dataset will be a resource to researchers from various fields seeking to better understand the space of amateur drawings, evaluate new algorithms in this domain, or develop new drawing-based tools in general.

To summarize, our contributions are as follows:
\begin{enumerate}
    \item 
    We explore the problem of automatic sketch-to-animation for children's drawings of human figures and present a framework that achieves this effect. We also present a set of experiments determining the amount of training data necessary to achieve high levels of success and a perceptual study validating the usefulness of our motion retargeting technique.
    \item To encourage additional research in the domain of amateur drawings, we present a first-of-its-kind dataset of 180,000 user-submitted amateur drawings, along with user-accepted bounding box, segmentation mask, and joint location annotations.
\end{enumerate}

Upon acceptance of this paper, we plan to publicly release the Amateur Drawings Dataset, project code, and fine-tuned model weights.


Our work builds on existing methods from several fields but is, to our knowledge, the first work focused specifically on fully automatic animation of children's drawings of human figures. 
To ground the work, we provide a summary of salient observations from the field of children's art analysis.
In addition, we briefly review related work on 2D image-to-animation and object and pose estimation for abstract images. 


\subsection{Analysis of Children's Drawings}

\hjs{
Children's drawings have long been of interest to the scientific community.
For well over a century, researchers from multiple fields have 
collected\,\cite{IndianaS55:online,kellogg1967rhoda,AWebbasedDatabaseforDrawingsofGods,geist2002they}
and analyzed them, seeking to understand and measure children's thought processes\,\cite{sully2021studies,barnes1892study,clark1897child,buhler2013mental}, 
intellectual development\,\cite{goodenough1926measurement},
and perceptions\,\cite{chambers1983stereotypic,doi:10.1080/01443410500344167}.
}
Particular attention has been given to drawings of human figures, one of the first and most frequently drawn subjects throughout childhood\,\cite{cox2013children}.

As the child develops, the schemas they employ to represent the human form become more complete (see Figure \ref{fig:tadpole-transitional-conventional}).
Even within these schemas, there is significant variation.
In addition to asymmetries and variation in color and proportion, many body parts appear optional to include; a study of drawings by 4-6 year old children showed that, while heads, legs, and eyes are almost universally present, other body parts (including torsos, arms, hands, and feet) were frequently absent\,\cite{cox2013children}.
Inversely, non-human body parts are frequently added in order to represent other subject classes\,\cite{kellogg1969analyzing}. With the addition of large ears, the figure may represent a cat or bear (Figures \ref{fig:maskrcnn_before_after}.m and \ref{fig:maskrcnn_before_after}.g); with the addition of a crown, it can represent a pineapple (Figure \ref{fig:maskrcnn_before_after}.n).
All of these sources of character variation make automatic character animation from drawings a non-trivial task.

\begin{figure}
\includegraphics[width=\linewidth]{images/tadpole-transition-conventional.png}
\caption{
As children learn to draw the human figure, the morphologies of the schemas they employ vary and evolve considerably\,\cite{cox2014drawings}.
Children frequently begin by drawing a \textit{tadpole figure}, a circular head region from which arms and legs extend. 
Some will progress to a \textit{transitional figure}, dropping the arms down so they extend from the legs. 
When a line is drawn between the legs, creating the separate torso region, the \textit{conventional figure} is formed.
Though these are small changes from the perspective of the drawer, they result in significantly different character morphologies when viewed through the lens of character animation.
A successful drawing-to-animation system must be robust to these types of variations.}
\label{fig:tadpole-transitional-conventional}
\end{figure}

Many researchers have focused closely on the unique style of children's drawings.
The psychologist and artist John Willats argues that, in order to understand the style of children's drawings, one must understand that the primary picture primitives employed by children are \textit{regions}, or 2D areas\,\cite{willats2006making}.
A squat volume, such as a head or torso, may be represented by a circular or ellipsoid region, whereas an elongated volume, such as a leg, may be represented by a long, thin region or even a single line.
These regions are not depictions of the object from any particular point of view. 
Rather, they are \textit{3D volumetric object-centered descriptions}\,\cite{marr1982vision},
2D areas with attributes perceptually similar to those of 3D object they are meant to represent.
%The regions begin as circles and lines, but later become modified to better reflect the perceptually impactful aspects of the objects they represent; a region representing a sugar cube or die may be given square corners, and a long region representing an arm may be given a bend to depict the elbow or split at the end to represent fingers (CITE Willats, 2005).

There are two stylistic outcomes of these \textit{object-centered descriptions} that bear mention.
First, the use of foreshortening is very rare in children's drawings \,\cite{piaget1956, willats1992representation}. 
This design choice is understandable; foreshortening a long region, such as a limb, results in a short region which does not adequately reflect the \textit{longness} of the object.
Second, the human figure may appear to have been drawn from many different perspectives, so as to make each part of the character maximally recognizable.
For example, the head and torso may face forward while the legs and feet are pointed to the side.
This technique, often referred to as \textit{twisted perspective}, is frequently seen and well-documented\,\cite{dziurawiec1992twisted}.
Both of these stylistic aspects are used to guide the design decisions of our system when applying human motion capture data onto the character.


\subsection{2D Image to Animation}

Previous researchers have proposed methods to animate drawings or photographs, many of which rely upon additional modes of user input.
Hornung et al. present a method to animate a 2D character in a photograph, given user-annotated joint locations\,\cite{Hornung2007anim2Dpicmotion}.
Pan and Zhang demonstrate a method to animate 2D characters with user-annotated joint locations via a variable-length needle model\,\cite{Pan2011}.
Jain et al. present an integrated approach to generate 3D proxies for animation given joint locations, segmentation masks, and per-part bounding boxes\,\cite{jain:2012}. 
Levi and Gotsman provide a method to create an articulated 3D object from a set of annotated 2D images and an initial 3D skeletal pose\,\cite{ArtiSketch}.
\textit{Live Sketch}\,\cite{su2018livesketch}
tracks control points from a video and applies their motion to user-specified control points upon a character.
Other approaches allow the user to specify character motions through a puppeteer interface, using RGB or RGB-D cameras\,\cite{held20123d,barnes2008video}.
\textit{ToonCap}\,\cite{Fan:2018:TAL} focuses on an inverse problem, capturing poses of a known cartoon character, given a previous image of the character annotated with layers, joints, and handles. 


\textit{Toonsynth}\,\cite{Dvoroznak18-SIG} and \textit{Neural Puppet}\,\cite{poursaeed2020neural} both present methods to synthesize animations of hand-drawn characters given a small set of drawings of the character in specified poses.
Hinz et al. train a network to generate new animation frames of a single character given 8-15 training images with user-specified keypoint annotations\,\cite{hinz2022charactergan}.

\textit{Monster Mash}\,\cite{Dvoroznak20-SA} presents an intuitive framework for sketch-based modeling and animation, and \textit{2.5D Cartoon Models}\,\cite{10.1145/1778765.1778796} presents a novel method of constructing 3D-like characters from a small number of 2D representations. 
Both of these are intuitive and well designed animation tools targeted towards amateur users.


\hjs{
Some animation methods are specifically tailored toward particular forms, such as faces\,\cite{elor2017bringingPortraits}, coloring book characters\,\cite{magnenat2015live}, or characters with human-like proportions. 
One notable work that is focused on the human form is \textit{Photo Wake Up}\,\cite{weng2019photo}. 
The authors show a method for creating a rigged and textured 3D mesh from a single image of a human-like figure.
Similar to us, their end goal is to allow users to seamlessly bring 2D characters to life; their work does an impressive job of accomplishing this.
Our method differs in two significant ways. 
First, while their work is focused on creating a 3D model for a mixed reality use case, 
ours is specifically focused on animating twisted perspective figures while staying within a 2D plane.
Second, children's drawings are much more abstract, incorrectly proportioned, and non human-like than the examples demonstrated in the paper.
We test our method upon the more abstract examples demonstrated in their paper and, with minor segmentation cleaning, they were successfully animated by our method.
}












\hjs{While the approaches listed here are wonderful tools to ease the burden of animation, none were perfectly suited to our use case.
Some require additional user input beyond the drawing itself, making the animation process more complex.
Others require the user to consistently draw the same character in multiple poses, which is beyond the skills of young children.
Others are focused on animating specific forms, precluding their use on children's drawings of the human figure.}


%Siarohin and colleagues propose a method for animating arbitrary classes of subjects,
%but require training videos of class members moving\,\cite{Siarohin_2019_NeurIPS}, making it unsuitable children's drawings.


\subsection{Detection, Segmentation, and Pose Estimation on Non-Photorealistic Images}

\hjs{
Aided by the the existence of large annotated datasets\,\cite{lin2014microsoft,6909866,6682899}, researchers have made considerable progress solving the problems of object detection, segmentation, and pose estimation from photographs. See, for example\,\cite{MaskRCNNhe2017mask,openpose19,guler2018densepose,alphapose,toshev2014deeppose}.
We explain the methods in this area that we leverage in Sections \ref{sec:character_detection} and \ref{sec:joint_detection}.

While traditional methods for detection, segmentation, and pose estimation of non-photorealistic images exist\,\cite{choi2012retrieval,bregler2002turning,davis2006sketching,eitz2012humans}, the lack of easily available datasets has resulted in slower adoption of deep learning models.
Some researchers are addressing this problem by developing methods and releasing datasets focused on the domain of anime characters\,\cite{chen2022bizarre,10.1145/3011549.3011552}, professional sketches\,\cite{brodt2022sketch2pose}, and mouse doodles\,\cite{ha2017neural}.
Other researchers have presented a non-deep learning method for inferring character poses from \textit{gesture drawings}\,\cite{Gesture3D}.
}
Because the Amateur Drawings Dataset is comprised of in-the-wild photographs of drawings created by the general public, we believe it will complement the value of existing datasets and allow for new dimensions of exploration and analysis.


\section{A generic solver for max-loss form and min-radius form: \pygranso~With Constraint-Folding (PWCF)}
\label{Sec:pygranso}
Although~\pygranso~can handle max-loss form and min-radius form, naive deployment of \pygranso~can suffer from slow convergence and low-quality solutions. To address this, we introduce \pygranso~\textbf{w}ith \textbf{C}onstraint-\textbf{F}olding (PWCF) to substantially speed up the optimization process and improve the solution quality. 
% We remark that these techniques are general guidelines to improve \pygranso's performance in constrainted deep learning problems, which are not limited to the RE problems, and should be applied whenever the conditions described are encountered in similar situations. We first describe these techniques in \cref{subsec: reformulate Linf constraint} to \cref{subsec: loss clip}, and show their effects in \cref{subsec: PWCF techniques demo}

\subsection{General techniques}
\label{pygranso general tricks}
\begin{figure}[!tb]
\centering
\begingroup 
\setlength{\tabcolsep}{1pt}
\renewcommand{\arraystretch}{0.8}
\begin{tabular}{cc}
\centering
\includegraphics[width=0.24\textwidth]{Figures/Sec3-Ablations/with-without-folding/L2-without-folding.png}
&\includegraphics[width=0.24\textwidth]{Figures/Sec3-Ablations/with-without-folding/L2-with-folding.png}
\\
\small{\textbf{(a)} $n$ box constraints} 
&\small{\textbf{(b)} folded constraints} 
\end{tabular}
\endgroup 
\caption{Examples of \pygranso~optimization trajectories to solve max-loss form with $\ell_{2}$ distance $(\eps=0.5)$ and margin loss $\ell$ (a clipped version described in \cref{subsec: loss clip}) on a CIFAR-10 image. \textbf{(a)} the $\mb x' \in [0, 1]^n$ constraint is in the original form of $n$ linear box constraints; \textbf{(b)} $\mb x' \in [0, 1]^n$ constraint is folded with the $\ell_2$ function into a single non-smooth constraint function. The x-axes denote the iteration number. Here, an acceptable solution is found when the objective value reaches $0.01$ and the constraint violation reaches below the tolerance level ($10^{-2}$). We can conclude from (a) and (b) that it takes significantly less time and number of iterations with constraint-folding than without.} 
\label{Fig: Ablation on constraint folding}
\end{figure}
The following techniques are developed for solving max-loss form and min-raidus form, but can also be applied to other NLOPT problems in similar situations.
\subsubsection{Reducing the number of constraints: constraint-folding}
\label{subsec: folding}
The natural image constraint $\mb x' \in [0, 1]^n$ is a set of $n$ box constraints. The reformulations described in \cref{subsec: reformulate Linf constraint} and \cref{subsec: reformulate l1 and linf obj} will introduce another $\Theta(n)$ box constraints. Although all of these are simple linear constraints, the $\Theta(n)$-growth is daunting: for natural images, $n$ is the number of pixels that can easily go into hundreds of thousands. Typical NLOPT problems become much more difficult when the number of constraints becomes large, which can, e.g., lead to slow convergence for numerical algorithms. 

\change{To address this, we introduce constraint-folding to reduce the number of constratins by allowing multiple constraints to be turned into a single one. We note that folding or aggregating constraints is not a new idea, which has been popular in engineering design. For example, \cite{martins2005structural} uses $\ell_\infty$ folding and its log-sum-exponential approximation to deal with numerous design constraints; see~\cite{ZhangEtAl2018Constraint,DomesNeumaier2014Constraint,ErmolievEtAl1997Constraint,TrappProkopyev2015note}. However, applying folding to NLOPT problems in machine learning and computer vision seems rare, potentially because producing non-smooth constraint(s) due to folding seems counterproductive. However, \pygranso~is able to handle non-smooth functions reliably. Thus, constraint folding can enjoy benefiting from reducing the difficulty and expense of solving the two types of quadratic programming (QP) subproblems ((\ref{penalty_sqp_dual}) and (\ref{QP_termination}) in \cref{Sec:granso_summary}) in each iteration of \pygranso, which are harder to solve as the number of constraints increases. Although turning multiple constraints into fewer but non-smooth ones can possibly increase the per-iteration cost, we show in this paper that there indeed can be a beneficial trade-off between non-smoothness and large number of constraints, in terms of speeding up the entire optimization process.}

\change{As for the implementation of the constraint-folding, first note that any equality constraint $h_j(\mb x) = 0$ or inequality constraint $c_i(\mb x) \le 0$ can be reformulated as 
\begin{align} \label{eq:simple_constr_form}
    \begin{split}
            & h_j (\mb x) = 0 \Longleftrightarrow \abs{h_j(\mb x)} \le 0 ~ \text{,}\\ 
            & c_i(\mb x) \le 0 \Longleftrightarrow \max\{c_i(\mb x), 0\} \le 0 ~ \text{,}
    \end{split}
\end{align} 
Then we can further fold them together into 
\begin{equation} 
    \label{eq:folded_constraint} 
    \begin{split}
        \mc F(& \abs{h_1(\mb x)}, \cdots, \abs{h_i(\mb x)}, \max\{c_1(\mb x), 0\}, \\
        & \cdots, \max\{c_j(\mb x), 0\}) \le 0,
    \end{split}
\end{equation}
where $\mc F: \RJU^{i+j}_{+} \mapsto \RJU_+$ ($\RJU_+ = \{\alpha: \alpha \ge 0\}$) can be any function satisfying $\mc F(\mb z) = 0 \Longrightarrow \mb z = \mb 0$, e.g., any $\ell_p$ ($p \ge 1$) norm, and (\ref{eq:folded_constraint}) and (\ref{eq:simple_constr_form}) still shares the same feasible set.}

The constraint folding technique can be used for a subset of constraints (e.g., constraints grouped and folded by physical meanings) or all of them. 
%We note that folding or aggregating constraints is not a new idea and has been popular in engineering design. For example, \cite{martins2005structural} uses $\ell_\infty$ folding and its log-sum-exponential approximation to deal with numerous design constraints; see ~\cite{ZhangEtAl2018Constraint,DomesNeumaier2014Constraint,ErmolievEtAl1997Constraint,TrappProkopyev2015note}. \emph{However, applying folding into NLOPT problems in machine learning and computer vision seems rare, potentially because producing non-differentiable constraint(s) due to the folding seems counterproductive.} But \pygranso~is able to handle these constraints reliably.
\change{Throughout this paper, we use $\mc F = \norm{\cdot}_2$ for constraint-folding, and the constraints are folded by group (they are: $\mb x' \in [0, 1]^n$, distance metric $d$ and decision boundary constraint, respectively.)} \cref{Fig: Ablation on constraint folding} shows the benefit of constraint folding with an example of max-loss form, where the time efficiency is greatly improved while the solution quality remains.

\subsubsection{Two-stage optimization}
\label{subsec: pygranso early stop}
Numerical methods may converge to poor local minima for NLOPT problems. Running the optimization multiple times with different random initializations is an effective and practical way to overcome this problem. For example, each method in \texttt{AutoAttack} by default runs five times and ends with the preset MaxIter. Here, we apply a similar practice to PWCF, but in a two-stage fashion:
\begin{enumerate}[leftmargin=*]
    \item \textbf{Stage 1 (selecting the best initialization):} Optimize the problems by PWCF with $R$ different random initialization $\mb x^{(r, 0)}$ for $k$ iterations, where $r=1, \ldots, R$, and collect the final first-stage solution $\mb x^{(r, k)}$ for each run. Determine the best intermediate result $\mb x^{(*, k)}$ and the corresponding initialization $x^{(*, 0)}$ following \cref{alg:2-stage screening}.
    \item{\textbf{Stage 2 (optimization):} Restart the optimization process with $x^{(*, 0)}$ until the stopping criterion is met\footnote{This is equivalent to warm start (continue optimization) with the intermediate result $\mb x^{*, k}$ with the corresponding Hessian approximation stored. To make the warm start follow the exact same optimization trajectory, the configuration parameter "scaleH0" for \pygranso\, must be turned off, which is different from the default value.} (reaching the tolerance level of stationarity and total constraint violation, or reaching the MaxIter $K$).
    }
\end{enumerate}

\change{The purpose of the two-stage optimization is simply to avoid PWCF spending too much time searching on unpromising optimization trajectories---after a reasonable number of iterations, PWCF tends to only refine the solutions; see (b) and (d) in \cref{Fig: Abaltion-OPT-Traj} later for an example, where PWCF has reached solutions with reasonable quality in very early stages and only improves the solution quality marginally afterward. Picking reasonably good values of $R$, $k$ and $K$ can be simple, e.g., using a small subset of samples and tuning them empirically. We present our result in \cref{Sec:pygranso} using $k=20$ and $K=400$ for max-loss form, $k=50$ and $K=4000$ for min-radius form and $R=10$ for both formulations, as PWCF can produce solutions with sufficient quality in the experiments in \cref{Sec:pygranso}.}
% \emph{We remark that our choice of $k$, $K$ and $R$ in this paper is purely empirical. Although choosing these parameters adaptively (e.g., for different DNN models with different capacities, or for different samples) may further improve the optimization speed and the solution quality, we will leave it for future work.}

\begin{algorithm}[!tb]
\caption{Selection of $x^{(*, k)}$ and $x^{(*, 0)}$ in the two-stage process}
\label{alg:2-stage screening}
\begin{algorithmic}[1]
\Require Initialization $x^{(r, 0)}$ and the corresponding intermediate optimization results $x^{(r, k)}$.

\If{Any $x^{(r, k)}$ is feasible for \formulation (\ref{eq:NO_form})}
\State Set $x^{(*, k)}$ to be the feasible $x^{(r, k)}$'s with the least objective value.
\Else
\State Set $x^{(*, k)}$ to be the $x^{(r, k)}$ with the least constraint violation.
\EndIf 
\State Set $x^{(*, 0)}$ corresponds to $x^{(*, k)}$ found.
\State \Return{$x^{(*, k)}$ and $x^{(*, 0)}$.}
\end{algorithmic}
\end{algorithm}

\subsection{Techniques specific to max-loss form and min-radius form}
\label{pygranso specific techniques}
In addition to the general techniques above, the following techniques can also help improve the performance of PWCF in solving (\ref{eq:robust_loss}) and (\ref{eq:min_distort}).

\begin{figure}[!tb]
\centering
\begingroup 
\setlength{\tabcolsep}{1pt}
\renewcommand{\arraystretch}{0.8}
\begin{tabular}{cc}
\centering
\includegraphics[width=0.24\textwidth]{Figures/Sec3-Ablations/with-without-LinfReformulation/Linf-orig.png}
&\includegraphics[width=0.24\textwidth]{Figures/Sec3-Ablations/with-without-LinfReformulation/Linf-reform.png}
\\
\small{\textbf{(a)} original $\ell_\infty$} 
&\small{\textbf{(b)} reformulated $\ell_\infty$} 
\end{tabular}
\endgroup 
\caption{Example of \pygranso~optimization trajectories to solve max-loss form (\ref{eq:robust_loss}) with the $\ell_{\infty}$ distance $(\eps=0.03)$ and margin loss $\ell$ (a clipped version described in \cref{subsec: loss clip}) on a CIFAR-10 image. The original form of constraint $\norm{\mb x' - \mb x}_\infty \leq \eps$ is used in \textbf{(a)}, while the reformulated and folded version of constraint is used in \textbf{(b)}. The x-axes denote the iteration number. Here, the optimization is terminated when the constraint violation is smaller than $10^{-2}$. After reformulating and folding the $\ell_\infty$ constraint, the optimization process in \textbf{(b)} runs much faster in terms of both time and iterations needed.} 
\label{Fig: Ablation on Linf Reform}
\end{figure}

\subsubsection{Avoiding sparse subgradients: reformulating $\ell_\infty$ constraints}
\label{subsec: reformulate Linf constraint}
\change{The BFGS-SQP algorithm inside \pygranso~uses the subgradients of the objective and the constraint functions to approximate the (inverse) Hessian of the penalty function, which is used to compute search directions. For the $\ell_\infty$ distance, the subdifferential (the set of subgradients) is:
\begin{align} 
\label{eq:subgrad_linf}
\nonumber
    \partial_{\mb z}  \norm{\mb z}_\infty = \conv \{\mb e_k \sign(z_k): z_k =  \norm{\mb z}_\infty \; \forall\, k\}
\end{align} 
where $\mb e_k$'s are the standard basis vectors, $\conv$ denotes the convex hull, and $\sign(z_k) = z_k/\abs{z_k}$ if $z_k \ne 0$, otherwise $[-1, 1]$.
Any subgradient within the subdifferential set contains no more than $n_k = \abs{\{k: z_k = \norm{\mb z}_\infty\}}$ nonzeros, and is sparse when $n_k$ is small. When all subgradients are sparse, only a handful of optimization variables may be updated on each iteration, which can result in slow convergence. To speed up the overall optimization process, we propose the reformulation:
\begin{equation}
    \label{eq: Linf to box}
    \norm{\mb x - \mb x'}_\infty \le \eps \Longleftrightarrow -\eps \mb 1 \le \mb x - \mb x' \le \eps \mb 1,
\end{equation}
where $\mb 1 \in \RJU^n$ is the all-ones vector. Then, the resulting $n$ box constraints can be folded into a single constraint as introduced in \cref{subsec: folding} to improve efficiency; see \cref{Fig: Ablation on Linf Reform} for an example of such benefits.}

\begin{figure}[!tb]
\centering
\includegraphics[width=0.4\textwidth]{Figures/Sec3-Ablations/L1-Reform/L1-Reform.png}
\caption{Robustness radii found by solving min-radius from with $\ell_1$ distance in the original formulation (\ref{eq:min_distort}) (\textcolor{blue}{blue}) and in the reformulated version (\ref{eq: l1 min form reformualtion}) (\textcolor{orange}{orange}) on $18$ CIFAR-10 images. The x-axis denotes the sample index and the y-axis denotes the radius. As all results have reached the feasibility tolerance, the lower the radius found, the more effective the solver is at handling the optimization problem.} 
\label{fig:L1 Reform Ablation} 
% \vspace{-1em}
\end{figure}

\subsubsection{Decoupling the update direction and the radius: reformulating $\ell_1$ and $\ell_\infty$ objectives}
\label{subsec: reformulate l1 and linf obj}
It is not surprising that for min-radius form with $\ell_\infty$ distance, it is more effective to solve the reformulated version (\ref{eq: min reform}) than the original one (\ref{eq:min_distort})---(\ref{eq: min reform}) moves the $\ell_\infty$ distance into the constraint, thus allowing us to apply constraint folding technique described in \cref{subsec: reformulate Linf constraint}. In practice, we also find that solving the reformulated version below is more effective when $d$ is the $\ell_1$ distance in min-radius form:
\begin{align} 
\label{eq: l1 min form reformualtion}
   \begin{split}
       & \min_{\mb x'\, , \mb t} \; \mb{1}^{\TJU}\mb t\\
    \st \; & \max_{i \ne y} f_{\mb \theta}^i (\mb x') \ge f_{\mb \theta}^y (\mb x') \\
    & \abs{\mb x_i - \mb x'_i} \leq t_i\, , \quad i = 1, 2, \cdots, n \\
    & \mb x' \in [0, 1]^n.
   \end{split}
\end{align}
where $t_i$ is the $i^{th}$ element of $\mb t$. The newly introduced box constraint $\abs{\mb x_i - \mb x'_i} \leq t_i$ is then folded as described in \cref{subsec: folding}. \cref{fig:L1 Reform Ablation} compares the robustness radius found by solving (\ref{eq:min_distort}) and (\ref{eq: l1 min form reformualtion}) on the same DNN model with $\ell_1$ distance on $18$ CIFAR-10 images, respectively. The radius found by solving (\ref{eq: l1 min form reformualtion}) are much smaller in all but one sample than by solving (\ref{eq:min_distort}), showing that solving (\ref{eq: l1 min form reformualtion}) with constraint-folding is more effective than solving the original form (\ref{eq:min_distort}).

\begin{figure}[!tb]
\centering
\begingroup 
\setlength{\tabcolsep}{1pt}
\renewcommand{\arraystretch}{0.8}
\begin{tabular}{cc}
\centering
\includegraphics[width=0.245\textwidth]{Figures/Sec3-Ablations/Min-Linf-Rescale/Min-Linf-NoRescale.png}
&\includegraphics[width=0.245\textwidth]{Figures/Sec3-Ablations/Min-Linf-Rescale/Min-Linf-Rescale.png}
\\
\small{\textbf{(a)} $\min~t$} 
&\small{\textbf{(b)} $\min~t \cdot \sqrt{n}$} 
\end{tabular}
\endgroup 
\caption{Examples of PWCF optimization trajectories for solving min-radius form (using (\ref{eq: min reform})) with the $\ell_\infty$ metric on a CIFAR-10 image \textbf{(a)} without rescaling and \textbf{(b)} with rescaling. The x-axes are the iteration number. The objective value in \textbf{(b)} is scaled back to the original value $t$ for fair comparisons with \textbf{(a)}. In Figure \textbf{(a)}, optimization is terminated around the $25^{\text{th}}$ iteration due to line-search failure, and the final solution has a much higher (worse) objective value than \textbf{(b)}. Here, we use $10^{-8}$ for the stationarity and constraint violation tolerances to rule out the possibility that the bad solution quality of \textbf{(a)} is due to premature termination.}
% Also note in both \textbf{(a)} and \textbf{(b)}, PWCF makes the most progress within a few iterations ($<20$), then refines the objective value with minor improvements afterwards.} 
\label{Fig: Ablation on Min Linf Rescale}
\end{figure}

\subsubsection{Numerical re-scaling to balance objective and constraints}
\label{subsec: granso resscale}
The steering procedure for determining search directions in \pygranso~(see line 5 in \cref{{alg:steering}}, \cref{Sec:granso_summary}) can only successively decrease the influence of the objective function in order to push towards feasible solutions. Therefore, if the scale of the objective value is too small compared to the initial constraint violation value, numerical problems can arise---\pygranso~will try hard to push down the violation amount, while the objective hardly decreases. This can occur when solving min-radius form with the $\ell_\infty$ distance, using the reformulated version (\ref{eq: min reform}). The objective $t$ is expected to have the order of magnitude $10^{-2}$ while the folded constraints are the $\ell_2$ norm of a $n$-dimensional vector (e.g., $n = 3 \times 32 \times 32 = 3072$ for a CIFAR-10 image). To address this, we simply rebalance the objective by a constant scalar---we minimize $t\cdot\sqrt{n}$ instead of $t$, which can help PWCF perform as effectively as in other cases; see an example in \cref{Fig: Ablation on Min Linf Rescale}.

\begin{figure}[!tb]
\centering
\begingroup 
\setlength{\tabcolsep}{1pt}
\renewcommand{\arraystretch}{0.8}
\begin{tabular}{ccc}
\centering
\textbf{\small{loss}}
&{ }
&\textbf{\small{gradient magnitude}}
\\
\cline{1-1}\cline{3-3}
\vspace{-1em}
\\
\includegraphics[width=0.24\textwidth]{Figures/vis_concept/ce_loss_clip.png}
&{ }
&\includegraphics[width=0.24\textwidth]{Figures/vis_concept/ce_grad_clip.png}
\\
\textbf{\small{(a)}}
&{ }
&\textbf{\small{(b)}}
\\
\includegraphics[width=0.24\textwidth]{Figures/vis_concept/margin_loss_clip.png}
&{ }
&\includegraphics[width=0.24\textwidth]{Figures/vis_concept/margin_grad_clip.png}
\\
\textbf{\small{(c)}}
&{ }
&\textbf{\small{(d)}}
\end{tabular}
\endgroup 
\caption{Visualizations of loss clipping. The cross-entropy loss and its clipped version is shown in \textbf{(a)} and the corresponding norm of gradients is shown in \textbf{(b)}. The clipped version shown in \textbf{(a)} is the one used in the CIFAR-10 experiments. The x-axes in \textbf{(a)} and \textbf{(b)} are the network output value $f_{\mb \theta}^y (\mb x')$ after softmax regularization.The margin loss and its clipped version is shown in \textbf{(c)}, and the corresponding norm of gradients is shown in\textbf{(d)}. The x-axes in \textbf{(c)} and \textbf{(d)} are the value $\max_{i \ne y} f_{\mb \theta}^i (\mb x') - f_{\mb \theta}^y (\mb x')$ before the softmax regularization, which follows the definition of $\ell_{ML}$ in \cref{eq: margin loss}.} 
\label{fig:loss_clipping} 
\end{figure}

\subsubsection{Loss clipping in solving max-loss form with PWCF}
\label{subsec: loss clip}
When solving  max-loss form using the popular cross-entropy (CE) and margin losses as $\ell$ (both are unbounded in maximization problems, see \cref{fig:loss_clipping}), the objective value and its gradient can easily dominate over making progress towards pushing down the constraint violations. \change{While \pygranso~tries to make the best joint progress of these two components, PWCF can still persistently prioritize maximizing the objective over constraint satisfaction, which can lead to bad scaling problems in the early stages and result in slow progress towards feasibility. To address this, we propose using the margin and CE losses with clipping at critical maximum values. For margin loss, the maximum value is clipped at $0.01$, since $\ell_{\mathrm{ML}} > 0$ indicates a successful attack. Similarly, CE loss can be clipped as follows: the attack success must occur when the true logit output is less than $1/N_c$ (after softmax normalization is applied), where $N_c$ is the number of classes. The corresponding critical value is thus $\ln N_c$---approximately $2.3$ when $N_c=10$ (the number of total classes for the CIFAR-10 dataset) and $4.6$ when $N_c=100$ (for ImageNet-100\footnote{ImageNet-100~\cite{laidlaw2021perceptual} is a subset of ImageNet, where samples with label index in $\{0, 10, 20, \cdots, 990\}$ are selected. ImageNet-100 validation set thus contains in total $5000$ images with $100$ classes.} dataset). Loss clipping significantly speeds up the optimization process of PWCF to solve max-loss form; see \cref{Fig: ablation loss clipping} for an example.}

\begin{figure}[!tb]
\centering
\begingroup 
\setlength{\tabcolsep}{1pt}
\renewcommand{\arraystretch}{0.8}
\begin{tabular}{cc}
\centering
\includegraphics[width=0.24\textwidth]{Figures/Sec3-Ablations/LossClip/CE-noClip.png}
&\includegraphics[width=0.24\textwidth]{Figures/Sec3-Ablations/LossClip/CE-Clip.png}
\\
\small{\textbf{(a)} CE} 
&\small{\textbf{(b)} CE-clip}
\\
\includegraphics[width=0.24\textwidth]{Figures/Sec3-Ablations/LossClip/Margin-noClip.png}
&\includegraphics[width=0.24\textwidth]{Figures/Sec3-Ablations/LossClip/Margin-Clip.png}
\\
\small{\textbf{(c)} margin} 
&\small{\textbf{(d)} margin-clip}
\end{tabular}
\endgroup 
\caption{PWCF optimization trajectories using cross-entropy (CE) loss, margin loss, and their clipped versions to solve the max-loss form using the $\ell_2$ metric on a CIFAR-10 image. The x-axes represent the number of iterations. For both CE and margin loss without clipping (\textbf{(a)} and \textbf{(c)}), PWCF progresses slowly toward feasibility (dashed orange curve), while with clipping (\textbf{(b)} and \textbf{(d)}), PWCF finds an optimal and feasible solution within only a few iterations.} 
\label{Fig: ablation loss clipping}
\end{figure}
\begin{table*}[!tb]
\caption{Summary of PWCF formulations to solve max-loss form and min-radius form. \textcolor{blue}{Blue} highlights the reformulation of $d$ used in min-radius form, and the \textcolor{red}{red} highlights the use of constraint-folding. Note that we distinguish the vector operator $\mb{\max}$ (entry-wise maximal value of a set of vectors, returning a vector with the same dimension) from the scalar operator $\max$ (maximal value of a given vector); \textbf{concat} denotes vector concatenations. }
\label{alg:pwcf}
\begin{center}
\setlength{\tabcolsep}{1.0mm}{
%\renewcommand\arraystretch{1.25}
\begin{tabular}{c c c c c c c c c}

{}
&{Metric $d$}
&\small{$\ell_1$}
&{~~~}
&\small{$\ell_2$}
&{~~~}
&\small{$\ell_\infty$}
&{~~~}
&\small{Others}
\\
\toprule
%\hline
{}
&{$\max$}
&{$\ell\paren{\mb y, f_{\theta}\paren{\mb x'}}$}
&{ }
&{$\ell\paren{\mb y, f_{\theta}\paren{\mb x'}}$}
&{ }
&{$\ell\paren{\mb y, f_{\theta}\paren{\mb x'}}$}
&{ }
&{$\ell\paren{\mb y, f_{\theta}\paren{\mb x'}}$}
\\
{}
&\multicolumn{7}{c}{ }
\vspace{-0.7em}
\\
{}
&{$\st$}
&{$\norm{\mb x' - \mb x}_1 \le \eps$}
&{ }
&{$\norm{\mb x' - \mb x}_2 \le \eps$}
&{ }
&{\textcolor{red}{$\|$}$\mathbf{\max} \{\mb x' - \mb x - \eps \mb 1 \text{,}$}
&{ }
&{$d\paren{\mb x', \mb x} \le \eps$}
\\
{}
&{}
&{}
&{ }
&{}
&{ }
&{$\quad 0 \}$\textcolor{red}{$\|_{2}$}$\le 0$}
&{ }
&{}
% \\
% \cline{7-7}
% \vspace{-10pt}
% \\
% {}
% &{}
% &{}
% &{ }
% &{}
% &{ }
% &\textcolor{red}{\small{constraint-folding}}
% &{ }
% &{}
\\
{}
&\multicolumn{8}{c}{ }
\vspace{-0.4em}
\\
{}
&{~}
&\multicolumn{7}{c}{\textcolor{red}{$\|$}$\mb{\max} \{ \textbf{concat} \paren{- \mb x', ~ \mb x'-\mb 1}, ~ \mb 0\}$\textcolor{red}{$\|_2$} $\le 0$}
% \\
% {}
% &{~}
% &\multicolumn{7}{c}{$\mb x' \in [0, 1]^{n}$}
\\
\midrule
\vspace{-1em}
\\
{}
&{$\min$}
&{\textcolor{blue}{$\mb{1}^{\TJU} \mb t$}}
&{ }
&{$\norm{\mb x' - \mb x}_2$}
&{ }
&{\textcolor{blue}{$t$}}
&{ }
&{$d \paren{\mb x', \mb x}$}
\\
{ }
&\multicolumn{7}{c}{ }
\vspace{-0.7em}
\\
{}
&{$\st$}
&{\textcolor{red}{$\|$}$\mb{\max} \{ ~~~~~~~~~~~~~~~~~~~$}
&{ }
&{ }
&{ }
&{\textcolor{red}{$\|$}$\mb{\max} \{ ~~~~~~~~~~~~~~~~~~~~~$}
&{ }
&{}
\\
{}
&{}
&{$\textbf{concat} (\mb x' - \mb x - \mb t,~~$}
&{ }
&{Not}
&{ }
&{$\textbf{concat} (\mb x' - \mb x - t\mb 1,~~$}
&{ }
&{Not}
\\
{ }
&{ }
&{$- \mb x' + \mb x - \mb t), ~ \mb 0 \}$\textcolor{red}{$\|_2$}$\le 0$}
&{ }
&{Applicable}
&{ }
&{$- \mb x' + \mb x - t \mb 1), ~ \mb 0 \}$\textcolor{red}{$\|_2$}$\le 0$}
&{ }
&{Applicable}
\\
{ }
&\multicolumn{7}{c}{ }
\vspace{-0.4em}
\\
{}
&{~}
&\multicolumn{7}{c}{$\max_{i \ne y} f_{\mb \theta}^i (\mb x') \ge f_{\mb \theta}^y (\mb x')$}
% {}
% &{~}
% &{$\max_{i \ne y} f_{\mb \theta}^i (\mb x') \ge f_{\mb \theta}^y (\mb x')$}
% &{~}
% &{$\max_{i \ne y} f_{\mb \theta}^i (\mb x') \ge f_{\mb \theta}^y (\mb x')$}
% &{~}
% &{$\max_{i \ne y} f_{\mb \theta}^i (\mb x') \ge f_{\mb \theta}^y (\mb x')$}
% &{~}
% &{$\max_{i \ne y} f_{\mb \theta}^i (\mb x') \ge f_{\mb \theta}^y (\mb x')$}
\\
&\multicolumn{7}{c}{ }
\vspace{-0.7em}
\\
% {}
% &{~}
% &\multicolumn{7}{c}{$\mb x' \in [0, 1]^{n}$}
{}
&{~}
&\multicolumn{7}{c}{\textcolor{red}{$\|$}$\mb{\max} \{ \textbf{concat} \paren{- \mb x', ~ \mb x'-\mb 1}, ~ \mb 0\}$\textcolor{red}{$\|_2$} $\le 0$}
\\
\bottomrule
\end{tabular}
}
\end{center}
\end{table*}
\begin{figure*}[!tb]
\vspace{-1em}
\centering
\begingroup 
\setlength{\tabcolsep}{1pt}
\renewcommand{\arraystretch}{0.8}
\begin{tabular}{c c c c c c}
\centering
{}
&\multicolumn{2}{c}{\textbf{max-loss form}}
&{ }
&\multicolumn{2}{c}{\textbf{min-radius form}}
\\
\cline{2-3}\cline{5-6}
\vspace{-1em}
\\
{}
&\includegraphics[width=0.24\textwidth]{Figures/Sec3-Ablations/Reliability-TerminateIter/OPT-Traj/APGD-L2.png}
&\includegraphics[width=0.24\textwidth]{Figures/Sec3-Ablations/Reliability-TerminateIter/OPT-Traj/PWCF-Max-L2-2.png}
&{ }
&\includegraphics[width=0.24\textwidth]{Figures/Sec3-Ablations/Reliability-TerminateIter/OPT-Traj/FAB-L2.png}
&\includegraphics[width=0.24\textwidth]{Figures/Sec3-Ablations/Reliability-TerminateIter/OPT-Traj/PWCF-Min-L2.png}
\\
{}
&\small{\textbf{(a)} APGD} 
&\small{\textbf{(b)} PWCF - margin loss}
&{ }
&\small{\textbf{(c)} FAB} 
&\small{\textbf{(d)} PWCF}

\end{tabular}
\endgroup 
\caption{Examples of the optimization trajectories of APGD (with CE loss), FAB and PWCF for solving max-loss form and min-radius form of a CIFAR-10 image with the $\ell_2$ distance. The x-axes represent the iteration numbers. In \textbf{(a)} and \textbf{(c)}, the dashed orange lines are the default MaxIter used in \texttt{AutoAttack} and the dashed green lines are the iteration where the best feasible solutions are found. In \textbf{(b)} and \textbf{(d)}, we set the stationarity and constraint violation tolerances to be $10^{-8}$ for the termination condition of the PWCF to better visualize the optimization curve. \textbf{$T_1$} and \textbf{$T_2$} mark the iterations where both the stationarity and total violation reach $10^{-2}$ and $10^{-3}$, respectively. We can observe that the objective values only improve marginally after $T_1$, \change{indicating that $10^{-2}$ is a reasonable tolerance level for PWCF to trade for efficiency while maintaining the solution quality.}}
%As the objective and constraint violation values only improve marginally after $T_1$, we can conclude that $10^{-2}$ is a reasonable tolerance level for PWCF to achieve good solution qualities.} 
\label{Fig: Abaltion-OPT-Traj}
\end{figure*}

\subsection{Summary of PWCF to solve the max-loss form and min-radius form}
\label{subsec: PWCF techniques demo}
We now summarize PWCF for solving max-loss form and min-radius form in \cref{alg:pwcf}. We also provide information on PWCF's reliability and running time analysis on these two problems. \change{We will present the ability of PWCF to handle general distance metrics in \cref{Sec: experiments and results}.}

% \begin{table}[!htpb]
\caption{Statistical summary of \cref{Fig: PWC-Terminate-Time}.}
\label{tab: PWCF Time}
\begin{center}
\setlength{\tabcolsep}{1.0mm}{
\begin{tabular}{c c c c c}
\small{\textbf{Time (s)}}
&{ }
&\small{\textbf{Min}}
&\small{\textbf{Max}}
&\small{\textbf{Median}}
\\
\toprule
\small{\textbf{(a)}}
&{ }
&{0.269}
&{50.02}
&{12.37}
\\
\small{\textbf{(b)}}
&{ }
&{0.253}
&{93.62}
&{17.60}
\\
\small{\textbf{(c)}}
&{ }
&{0.377}
&{193.2}
&{36.26}
\\
\small{\textbf{(d)}}
&{ }
&{1.081}
&{131.9}
&{115.5}
\\
\midrule
\small{\textbf{(e)}}
&{ }
&{5.778}
&{134.5}
&{21.58}
\\
\small{\textbf{(f)}}
&{ }
&{91.67}
&{369.2}
&{167.5}
\\
\small{\textbf{(g)}}
&{ }
&{2.965}
&{354.8}
&{112.9}
\\
\small{\textbf{(h)}}
&{ }
&{121.5}
&{317.0}
&{161.7}
\\
\bottomrule
\end{tabular}
}
\end{center}
\end{table}

\subsubsection{Reliability}
\label{subsec: reliability}
As mentioned in \cref{Sec:introduction}, popular numerical methods only relying on preset MaxIter are subject to premature termination (see \cref{Fig:APGD-FAB-Terminate-Iter} to review). In contrast, PWCF terminates either when the stopping criterion is met (solution quality meet the preset tolerance level), or by MaxIter with stationarity estimate and total violation to assess whether further optimization is necessary (see \cref{Fig: PWC-Max-Terminate-Iter} to review). \cref{Fig: Abaltion-OPT-Traj} provide extra examples of the optimization trajectories of APGD, FAB and PWCF, which again shows why PWCF's termination mechanism is more reliable.

\begin{figure*}[!tb]
\vspace{1em}
\centering
\begingroup 
\setlength{\tabcolsep}{1pt}
\renewcommand{\arraystretch}{0.8}
\begin{tabular}{c c c c c c}
\centering
{}
&\multicolumn{2}{c}{\textbf{CIFAR-10}}
&{ }
&\multicolumn{2}{c}{\textbf{ImageNet-100}}
\\
\cline{2-3}\cline{5-6}
\vspace{-1em}
\\
{}
&\includegraphics[width=0.24\textwidth]{Figures/Sec3-Ablations/AutoAttackTime/APGD-L2-Cifar.png}
&\includegraphics[width=0.24\textwidth]{Figures/Sec3-Ablations/AutoAttackTime/APGD-Linf-Cifar.png}
&{ }
&\includegraphics[width=0.24\textwidth]{Figures/Sec3-Ablations/AutoAttackTime/APGD-L2-ImageNet.png}
&\includegraphics[width=0.24\textwidth]{Figures/Sec3-Ablations/AutoAttackTime/APGD-Linf-ImageNet.png}
\\
{}
&\small{\textbf{(a)} APGD - $\ell_2$} 
&\small{\textbf{(b)} APGD - $\ell_\infty$}
&{ }
&\small{\textbf{(c)} APGD - $\ell_2$} 
&\small{\textbf{(d)} APGD - $\ell_\infty$}
\\
&\includegraphics[width=0.24\textwidth]{Figures/Sec3-Ablations/PWCF-Time/PWCF-Time-Max-L2-Cifar.png}
&\includegraphics[width=0.24\textwidth]{Figures/Sec3-Ablations/PWCF-Time/PWCF-Time-Max-Linf-Cifar.png}
&{ }
&\includegraphics[width=0.24\textwidth]{Figures/Sec3-Ablations/PWCF-Time/PWCF-Time-Max-L2-ImageNet.png}
&\includegraphics[width=0.24\textwidth]{Figures/Sec3-Ablations/PWCF-Time/PWCF-Time-Max-Linf-ImageNet.png}
\\
{}
&\small{\textbf{(e)} PWCF - $\ell_2$} 
&\small{\textbf{(f)} PWCF - $\ell_\infty$}
&{ }
&\small{\textbf{(g)} PWCF - $\ell_2$} 
&\small{\textbf{(h)} PWCF - $\ell_\infty$}
\end{tabular}
\endgroup 
\caption{Histograms of the termination time (x-axes, in seconds) of APGD (using \texttt{AutoAttack} default setup: five random restarts with 100 as MaxIter per run) and PWCF (ours) to solve max-loss form on $88$ images from CIFAR-10 and $85$ images from ImageNet-100. For this problem, PWCF can be slower than APGD by about one order of magnitude.}
\label{Fig: PWC-Terminate-Time-Max}
\end{figure*}
\begin{figure*}[!tb]
\vspace{1em}
\centering
\begingroup 
\setlength{\tabcolsep}{1pt}
\renewcommand{\arraystretch}{0.8}
\begin{tabular}{c c c c c c}
\centering
{}
&\multicolumn{2}{c}{\textbf{CIFAR-10}}
&{ }
&\multicolumn{2}{c}{\textbf{ImageNet-100}}
\\
\cline{2-3}\cline{5-6}
\vspace{-1em}
\\
{}
&\includegraphics[width=0.24\textwidth]{Figures/Sec3-Ablations/AutoAttackTime/FAB-L2-Cifar.png}
&\includegraphics[width=0.24\textwidth]{Figures/Sec3-Ablations/AutoAttackTime/FAB-Linf-Cifar.png}
&{ }
&\includegraphics[width=0.24\textwidth]{Figures/Sec3-Ablations/AutoAttackTime/FAB-L2-ImageNet.png}
&\includegraphics[width=0.24\textwidth]{Figures/Sec3-Ablations/AutoAttackTime/FAB-Linf-ImageNet.png}
\\
{}
&\small{\textbf{(a)} FAB - $\ell_2$} 
&\small{\textbf{(b)} FAB - $\ell_\infty$}
&{ }
&\small{\textbf{(c)} FAB - $\ell_2$} 
&\small{\textbf{(d)} FAB - $\ell_\infty$}
\\
&\includegraphics[width=0.24\textwidth]{Figures/Sec3-Ablations/PWCF-Time/PWCF-Time-Min-L2-Cifar.png}
&\includegraphics[width=0.24\textwidth]{Figures/Sec3-Ablations/PWCF-Time/PWCF-Time-Min-Linf-Cifar.png}
&{ }
&\includegraphics[width=0.24\textwidth]{Figures/Sec3-Ablations/PWCF-Time/PWCF-Time-Min-L2-ImageNet.png}
&\includegraphics[width=0.24\textwidth]{Figures/Sec3-Ablations/PWCF-Time/PWCF-Time-Min-Linf-ImageNet.png}
\\
{}
&\small{\textbf{(e)} PWCF - $\ell_2$} 
&\small{\textbf{(f)} PWCF - $\ell_\infty$}
&{ }
&\small{\textbf{(g)} PWCF - $\ell_2$} 
&\small{\textbf{(h)} PWCF - $\ell_\infty$}
\end{tabular}
\endgroup 
\caption{Histograms of the termination time (x-axis, in seconds) for FAB (using \texttt{AutoAttack} default setup: five random restarts with 100 as MaxIter per run) and PWCF to solve min-radius form on $88$ images from CIFAR-10 and $85$ images from ImageNet-100. PWCF is on average faster than FAB (comparing \textbf{(a)} \& \textbf{(e)}, \textbf{(c)} \& \textbf{(g)} and \textbf{(d)} \& \textbf{(h)}) except for the $\ell_\infty$ case on CIFAR-10 images (\textbf{(b)} \& \textbf{(f)}).}
\label{Fig: PWC-Terminate-Time}
\end{figure*}
\begin{table*}[!tb]
\caption{\textbf{Comparison between PWCF and other numerical algorithms in solving max-loss form with the $\ell_{1}$, $\ell_{2}$ and $\ell_{\infty}$ distances.} \textbf{Metric ($\eps$)} denotes the choice of $d$ and the corresponding perturbation budget $\eps$ used. We report the model's \textcolor{blue}{clean} and robust accuracy (numbers are in $(\%)$) for comparison---lower robust accuracy reflects more effective optimization. We test APGD and PWCF using both \textbf{CE} and margin (\textbf{M}) loss. \textbf{CE+M} column shows the robust accuracy achieved by combining adversarial samples found using CE and margin losses; \textbf{A\&P} shows the robust accuracy achieved by combining all perturbation samples found by using APGD and PWCF with both CE and M losses. We highlight the best performance achieved by a single combination of solver and loss with \underline{underlines} for each $d$, and highlight the best performance achieved in \textbf{bold}.}

\label{tab: granso_l1_acc}
\begin{center}
\setlength{\tabcolsep}{1.0mm}{
%\renewcommand\arraystretch{1.25}
\begin{tabular}{l c c c c c c c c c c c c c c}
{}
&{}
&{}
&{}
&\multicolumn{3}{c}{\small{\textbf{APGD}}}
&{}
&\multicolumn{3}{c}{\small{\textbf{PWCF(ours)}}}
&{}
&\small{\textbf{Square}}
&{}
&{}
\\
\cline{5-7}\cline{9-11}\cline{13-13}
\vspace{-12pt}
\\
{\small{\textbf{Dataset}}}
&{\small{\textbf{Metric ($\eps$)}}}
& \small{\textbf{Clean}}
&{}
& \small{\textbf{CE}}
& \small{\textbf{M}}
& \small{\textbf{CE+M}}
& {}
& \small{\textbf{CE}}
& \small{\textbf{M}}
& \small{\textbf{CE+M}}
& {}
& \small{\textbf{M}}
&{}
&\small{\textbf{A\&P}}
\\
\toprule
{\small{CIFAR-10}}
&\small{$\ell_{1} (12)$}
&\textcolor{blue}{73.29}
&{}
&{0.97}
&\underline{0.00}
&{0.00}
&{}
&{17.93}
&{0.01}
&{0.01}
&{}
&{2.28}
&{}
&\textbf{0.00}
\\

\cline{2-15}
\vspace{-12pt}
\\
{}
&\small{$\ell_{2} (0.5)$}
&\textcolor{blue}{94.61}
&{}
&{81.81}
&{81.06}
&{80.92}
&{}
&{81.99}
&\underline{81.02}
&{80.87}
&{}
&{87.9}
&{}
&\textbf{80.77}
\\
\cline{2-15}
\vspace{-12pt}
\\
{}
&\small{$\ell_{\infty} (0.03)$}
&\textcolor{blue}{90.81}
&{}
&{69.44}
&\underline{67.71}
&{67.33}
&{}
&{88.71}
&{68.20}
&{68.17}
&{}
&{71.6}
&{}
&\textbf{67.26}
\\
\midrule
\midrule
{\small{ImageNet}-\small{100}}
&\small{$\ell_{2} (4.7)$}
&\textcolor{blue}{75.04}
&{}
&\underline{42.44}
&{44.06}
&{40.86}
&{}
&{42.50}
&{43.52}
&{40.60}
&{}
&{63.1}
&{}
&\textbf{40.46}
\\
\cline{2-15}
\vspace{-12pt}
\\
{}
&\small{$\ell_\infty (0.016)$}
&\textcolor{blue}{75.04}
&{}
&\underline{46.78}
&{47.54}
&{45.20}
&{}
&{73.92}
&{47.72}
&{47.72}
&{}
&{59.9}
&{}
&\textbf{45.12}
\\
\bottomrule
\end{tabular}
}
\end{center}
\end{table*}

\subsubsection{Running cost}
\label{subsec: running cost}
We now consider run-time comparisons of APGD and FAB v.s. PWCF in solving max-loss form and min-radius form on CIFAR-10 and ImageNet images, as in \cref{Fig: PWC-Terminate-Time-Max} and \cref{Fig: PWC-Terminate-Time}. Our computing environment uses an AMD Milan 7763 64-core processor and a NVIDIA A100 GPU (40G version). We use $10^{-2}$ as PWCF's stationarity and violation tolerances to benchmark PWCF's running cost. The results show that PWCF can be faster than FAB (except for the $\ell_\infty$ case on CIFAR-10 images) when solving min-radius form (\cref{Fig: PWC-Terminate-Time}), but can be about $10$ times slower than APGD when solving max-loss form (\cref{Fig: PWC-Terminate-Time-Max}).  Again, we remark that solution reliability should come with a higher priority than speed for RE; we leave improving the speed of PWCF as future work. The run-time result of PWCF implies that PWCF may be favorable for RE where reliability and accuracy are crucial but may be non-ideal to be used in training pipelines.

% \begin{table*}[!tb]
\caption{\textbf{Comparison between PWCF and other numerical algorithms in solving max-loss form with the $\ell_{1}$, $\ell_{2}$ and $\ell_{\infty}$ distances.} \textbf{Metric ($\eps$)} denotes the choice of $d$ and the corresponding perturbation budget $\eps$ used. We report the model's \textcolor{blue}{clean} and robust accuracy (numbers are in $(\%)$) for comparison---lower robust accuracy reflects more effective optimization. We test APGD and PWCF using both \textbf{CE} and margin (\textbf{M}) loss. \textbf{CE+M} column shows the robust accuracy achieved by combining adversarial samples found using CE and margin losses; \textbf{A\&P} shows the robust accuracy achieved by combining all perturbation samples found by using APGD and PWCF with both CE and M losses. We highlight the best performance achieved by a single combination of solver and loss with \underline{underlines} for each $d$, and highlight the best performance achieved in \textbf{bold}.}

\label{tab: granso_l1_acc}
\begin{center}
\setlength{\tabcolsep}{1.0mm}{
%\renewcommand\arraystretch{1.25}
\begin{tabular}{l c c c c c c c c c c c c c c}
{}
&{}
&{}
&{}
&\multicolumn{3}{c}{\small{\textbf{APGD}}}
&{}
&\multicolumn{3}{c}{\small{\textbf{PWCF(ours)}}}
&{}
&\small{\textbf{Square}}
&{}
&{}
\\
\cline{5-7}\cline{9-11}\cline{13-13}
\vspace{-12pt}
\\
{\small{\textbf{Dataset}}}
&{\small{\textbf{Metric ($\eps$)}}}
& \small{\textbf{Clean}}
&{}
& \small{\textbf{CE}}
& \small{\textbf{M}}
& \small{\textbf{CE+M}}
& {}
& \small{\textbf{CE}}
& \small{\textbf{M}}
& \small{\textbf{CE+M}}
& {}
& \small{\textbf{M}}
&{}
&\small{\textbf{A\&P}}
\\
\toprule
{\small{CIFAR-10}}
&\small{$\ell_{1} (12)$}
&\textcolor{blue}{73.29}
&{}
&{0.97}
&\underline{0.00}
&{0.00}
&{}
&{17.93}
&{0.01}
&{0.01}
&{}
&{2.28}
&{}
&\textbf{0.00}
\\

\cline{2-15}
\vspace{-12pt}
\\
{}
&\small{$\ell_{2} (0.5)$}
&\textcolor{blue}{94.61}
&{}
&{81.81}
&{81.06}
&{80.92}
&{}
&{81.99}
&\underline{81.02}
&{80.87}
&{}
&{87.9}
&{}
&\textbf{80.77}
\\
\cline{2-15}
\vspace{-12pt}
\\
{}
&\small{$\ell_{\infty} (0.03)$}
&\textcolor{blue}{90.81}
&{}
&{69.44}
&\underline{67.71}
&{67.33}
&{}
&{88.71}
&{68.20}
&{68.17}
&{}
&{71.6}
&{}
&\textbf{67.26}
\\
\midrule
\midrule
{\small{ImageNet}-\small{100}}
&\small{$\ell_{2} (4.7)$}
&\textcolor{blue}{75.04}
&{}
&\underline{42.44}
&{44.06}
&{40.86}
&{}
&{42.50}
&{43.52}
&{40.60}
&{}
&{63.1}
&{}
&\textbf{40.46}
\\
\cline{2-15}
\vspace{-12pt}
\\
{}
&\small{$\ell_\infty (0.016)$}
&\textcolor{blue}{75.04}
&{}
&\underline{46.78}
&{47.54}
&{45.20}
&{}
&{73.92}
&{47.72}
&{47.72}
&{}
&{59.9}
&{}
&\textbf{45.12}
\\
\bottomrule
\end{tabular}
}
\end{center}
\end{table*}
%
\section{Performance of PWCF in solving max-loss form and min-radius form}
\label{Sec: experiments and results}
In this section, we show the effectiveness of PWCF in solving max-loss form and min-radius form with general distance metrics $d$. First in \cref{min max l1 l2 linf}, we compare the performance of PWCF with other existing numerical algorithms when $d$ is the $\ell_1$, $\ell_2$, and $\ell_\infty$ distance. Next in \cref{min max l15 l8}, we take the $\ell_{1.5}$ and $\ell_8$ distances as examples to show that PWCF can effectively handle max-loss form and min-radius form with general $\ell_p$ metrics. Finally in \cref{subsec: min max perceptual}, we show that PWCF can also handle both formulations with the perceptual distance (PD, a non-$\ell_p$ metric). In this section, due to the variety of capacity differences of the DNN models used, we conservatively use $k=40$ and $K=400$ for experiments on max-loss form, $k=50$ and $K=4000$ for min-radius form on CIFAR-10 dataset, $k=200$ and $K=5000$ for min-radius form on ImageNet-100 dataset and $r=10$ for all cases to avoid possible premature terminations.

\subsection{Solving max-loss form and min-radius form with the $\ell_1$, $\ell_2$ and $\ell_\infty$ distance}
\label{min max l1 l2 linf}
We now compare the performance of PWCF with other existing numerical algorithms  in solving max-loss form and min-radius form when $d$ is the $\ell_1$, $\ell_2$, and $\ell_\infty$ distance to show that PWCF can solve both formulations effectively.

\begin{figure*}[!tb]
\centering
\begingroup 
\setlength{\tabcolsep}{1pt}
\renewcommand{\arraystretch}{0.8}
\begin{tabular}{ccc}
\centering
\includegraphics[width=0.33\textwidth]{Figures/Min_Radius/cifar-l1.png}
&\includegraphics[width=0.33\textwidth]{Figures/Min_Radius/cifar-l2.png}
&\includegraphics[width=0.33\textwidth]{Figures/Min_Radius/cifar-linf.png}
\\
\small{CIFAR-10 - $\ell_1$} 
&\small{CIFAR-10 - $\ell_2$} 
&\small{CIFAR-10 - $\ell_\infty$}
% \\
% \hline 
% \vspace{-4pt}
\\
\includegraphics[width=0.33\textwidth]{Figures/Min_Radius/imagenet-l1.png}
&\includegraphics[width=0.33\textwidth]{Figures/Min_Radius/imagenet-l2.png}
&\includegraphics[width=0.33\textwidth]{Figures/Min_Radius/imagenet-linf.png}
\\
\small{ImageNet-100 - $\ell_1$} 
&\small{ImageNet-100 - $\ell_2$} 
&\small{ImageNet-100 - $\ell_\infty$}
\end{tabular}
\endgroup 
\caption{Comparison of the per-sample robustness radius between PWCF(ours) and FAB for $88$ images from CIFAR-10 and $85$ images from ImageNet-100. In each subfigure, the $1^{\text{st}}$ row shows the sample-wise robustness radius found by FAB, and the $2^{\text{nd}}$ row shows the radius difference between PWCF and FAB (PWCF minus FAB): values $< 0$ indicating a better result found by PWCF than FAB.} 
\label{Fig:FAB-min-radius-details}
\end{figure*}
\begin{table*}[!tb]
\caption{\textbf{Statistical comparison of the minimal radius found by PWCF(ours) and FAB in solving the \emph{min-form} with $\ell_1$, $\ell_2$ and $\ell_\infty$ as the metric $d$ (summary of \cref{Fig:FAB-min-radius-details}).} We experimented with $88$ fixed images from CIFAR-10 and ImageNet-100 dataset with each $d$. We report the \textbf{Mean}, \textbf{Median} and \textbf{standard deviation (STD)} of the minimal perturbation radius---\emph{lower} radius means \emph{more effective} minimization. The columns under \textbf{Difference} are calculated based on the \emph{sample-wise} radius difference (PWCF radius minus FAB), where \textbf{Mean} and \textbf{Median} $\leq 0$ indicates PWCF performs better than FAB on average.}
\label{tab: granso_min}
\begin{center}
\setlength{\tabcolsep}{1.0mm}{
%\renewcommand\arraystretch{1.25}
\begin{tabular}{l c c c c c c c c c c c c}
{}
&{}
&\multicolumn{3}{c}{\small{\textbf{FAB}}}
&{}
&\multicolumn{3}{c}{\small{\textbf{PWCF (ours)}}}
&{}
&\multicolumn{3}{c}{\small{\textbf{Difference}}}
\\
\cline{3-5}\cline{7-9}\cline{11-13}
\vspace{-12pt}
\\
{\small{\textbf{Dataset}}}
&{\small{\textbf{Metric} $d$}}
& \small{\textbf{Mean}}
& \small{\textbf{Median}}
& \small{\textbf{STD}}
& {}
& \small{\textbf{Mean}}
& \small{\textbf{Median}}
& \small{\textbf{STD}}
& {}
& \small{\textbf{Mean}}
& \small{\textbf{Median}}
& \small{\textbf{STD}}
\\
\toprule
\small{CIFAR-10}
&\small{$\ell_{1}$}
&{13.92}
&{10.50}
&{12.63}
&{}
&{12.02}
&{7.29}
&{11.46}
&{}
&{\textbf{-1.89}}
&{\textbf{-0.81}}
&{5.24}
\\
\cline{2-13}
\vspace{-12pt}
\\
{}
&\small{$\ell_{2}$}
&{1.02}
&{0.90}
&{0.71}
&{}
&{1.00}
&{0.88}
&{0.69}
&{}
&\textbf{-0.019}
&\textbf{-0.015}
&{0.250}
\\
\cline{2-13}
\vspace{-12pt}
\\
{}
&\small{$\ell_{\infty}$}
&{0.0298}
&{0.0245}
&{0.0220}
&{}
&{0.0298}
&{0.0252}
&{0.0224}
&{}
&\textbf{0.0008}
&\textbf{-0.00008}
&{0.007}
\\
\midrule
\midrule
{\small{ImageNet-100}}
&\small{$\ell_{1}$}
&{435.4}
&{400.6}
&{303.9}
&{ }
&{408.1}
&{390.6}
&{284.7}
&{}
&\textbf{-27.31}
&\textbf{-13.46}
&{70.55}
\\
\cline{2-13}
\vspace{-12pt}
\\
{}
&\small{$\ell_2$}
&{6.75}
&{6.81}
&{3.82}
&{}
&{6.71}
&{6.88}
&{3.76}
&{}
&\textbf{-0.042}
&\textbf{-0.035}
&{0.758}
\\
\cline{2-13}
\vspace{-12pt}
\\
{}
&\small{$\ell_\infty$}
&{0.028}
&{0.028}
&{0.016}
&{}
&{0.029}
&{0.029}
&{0.016}
&{}
&\textbf{0.0009}
&\textbf{0.00002}
&{0.002}
\\
\bottomrule
\end{tabular}
}
\end{center}
\end{table*}

\subsubsection{PWCF offers competitive attack performance in solving max-form with diverse solutions}
\label{subsec: max formulation with l1, l2, linf}
We use several publicly available models that are adversarially trained by $\ell_1$\footnote{For $\ell_1$ experiment, we use the model `L1.pt' from \url{https://github.com/locuslab/robust_union/tree/master/CIFAR10}, which is adversarially trained by $\ell_1$-attack.}, $\ell_2$, and $\ell_\infty$\footnote{For $\ell_2$ ad $\ell_\infty$ experiments, we use the models `L2-Extra.pt' and `Linf-Extra.pt' from \url{https://github.com/deepmind/deepmind-research/tree/master/adversarial_robustness}, with the WRN-70-16 network architecuture.} attacks on CIFAR-10, and by perceptual attack (See \cref{subsec: formulation with general lp norm} for details) on ImageNet-100~\cite{laidlaw2021perceptual}.\footnote{We use the `pat\_alexnet\_0.5.pt' from \url{https://github.com/cassidylaidlaw/perceptual-advex}, where the authors tested and showed its $\ell_2$- and $\ell_\infty$- robustness in the original work.} We then compare the robust accuracy achieved by these selected models by solving max-loss form with PWCF and APGD\footnote{We implement the margin loss on top of the original APGD.} from \texttt{AutoAttack}. The bound $\eps$ for each case is set to follow the common practice of RE\footnote{The $\eps$ of $\ell_2$ and $\ell_\infty$ for CIFAR-10 are chosen from \url{https://robustbench.github.io/}; $\ell_1$ for CIFAR-10 is chosen from \url{https://github.com/locuslab/robust_union}; $\ell_2$ and $\ell_\infty$ for ImageNet-100 are from \cite{laidlaw2021perceptual}.}. The result is shown in \cref{tab: granso_l1_acc}.

We can conclude from \cref{tab: granso_l1_acc} that \textbf{1)} PWCF performs strongly and comparably to APGD on solving max-loss form with the $\ell_1$, $\ell_2$ and $\ell_\infty$ distances, especially when margin loss is used. The weak performance of PWCF on $\ell_1$ and $\ell_\infty$ cases using CE loss is likely due to poor numerical scaling of the loss itself: the gradient magnitude of CE loss grows much larger as the loss value increases (see \cref{fig:loss_clipping} for a visualization example of the loss). Therefore, as the CE loss increases during the optimization of max-loss form while the constraint violation scale remains unchanged or even decreases, PWCF may suffer from the imbalanced contributions from the objective and constraint. In contrast, the gradient scale is always $1$ using the margin loss and PWCF will not have similar struggles. APGD has an explicit step-size rule different from the vanilla PGD algorithms to improve attack performance under CE loss~\cite{croce2020reliable}, while PWCF does not have special handling for this case. In fact, APGD methods also prefer the use of margin loss over CE as pointed out in~\cite{croce2020reliable}, although the consideration is different from PWCF. \textbf{2)} Combining all successful attack samples found by APGD and PWCF using CE and margin loss (column A\&P in \cref{tab: granso_l1_acc}) achieves the lowest robust accuracy for all distance metrics---PWCF and APGD provide diverse and complementary solutions in terms of attack effectiveness. A direct message here is that lacking diversity (e.g., solving max-loss form with a restricted set of algorithms) will result in overestimated robust accuracy. Note that~\cite{CarliniEtAl2019Evaluating} also remarks that the diversity of solvers matters more than the superiority of individual solver, which motivates \texttt{AutoAttack} to include Square Attack---a zero-th order black-box attack method that does not perform strongly itself as shown in \cref{tab: granso_l1_acc}. We will provide further discussions on the importance of diversity later in \cref{sec:pattern_theory}, based on the differences in the solution patterns found in solving max-loss form. 

\subsubsection{PWCF provides competitive solutions to min-radius form}
\label{subsec: min formulation with l1, l2 and linf}
We take models adversarially trained on CIFAR-10\footnote{We use model `pat\_self\_0.5.pt' from \url{https://github.com/cassidylaidlaw/perceptual-advex}.} and ImageNet-100\footnote{The same ImageNet-100 model used in \cref{tab: granso_l1_acc}.} and compare the robustness radii found by solving min-radius form with PWCF and FAB in \cref{Fig:FAB-min-radius-details}, and \cref{tab: granso_min} summarizes the mean, median, and standard deviation of the results in \cref{Fig:FAB-min-radius-details}. From the column Mean and Median in \cref{tab: granso_min}, we can conclude that PWCF performs on average \textbf{1)} better than FAB in solving min-radius form with the $\ell_1$ and $\ell_2$ distances, and \textbf{2)} comparably to FAB for the $\ell_\infty$ case.

\subsection{Solving max-loss form and min-radius form with general distance metrics}
\label{subsec: formulation with general lp norm}
As highlighted in \cref{sec:background}, a major limitation of the existing numerical methods is that they mostly handle limited choice of $d$. On the contrary, PWCF stands out as a convenient choice for other general distances. We now present solving max-loss form and min-radius form with $\ell_{1.5}$, $\ell_{8}$ norm and the perceptual distance (PD, a non-$\ell_p$ distance that involves a DNN) by PWCF. To the best of our knowledge, no prior work has studied handling general distance metrics in the two constraint optimization problems;~\cite{laidlaw2021perceptual} has proposed 3 algorithms to solve max-loss form with PD, which will be compared with PWCF in the following sections.

\subsubsection{Solving min-radius form and max-loss form with $\ell_{1.5}$ and $\ell_{8}$ distances}
\label{min max l15 l8}
Due to the lack of existing methods for comparison, we conduct the following experiments to show the effectiveness of PWCF:
\begin{itemize}[leftmargin=*]
    \item We first apply PWCF to solve min-radius form with the $\ell_{1.5}$ and $\ell_{8}$ distances, and compare the robustness radii with the $\ell_2$ results found in \cref{Fig:FAB-min-radius-details}. One necessary condition for effective optimization is that the robustness radius found using different $\ell_p$ metrics should have $\ell_{1.5} \ge \ell_2 \ge \ell_{8}$. \cref{fig: L1.5 L8 min radius plot} shows the per-sample radii found by PWCF under $\ell_{1.5}$ and $\ell_{8}$ metrics and confirms the satisfaction of the above condition.
    \item We employ two sample-adaptive strategies to set the perturbation budget $\eps$ for PWCF to solve max-loss form with the $\ell_{1.5}$ and $\ell_{8}$ distances: using the same DNN model to evaluate, we take $0.8$ and $1.2$ times the robustness radii found in \cref{fig: L1.5 L8 min radius plot} as $\eps$. If the robustness radii found in \cref{fig: L1.5 L8 min radius plot} are tight, PWCF should achieve close to $100 \%$ robust accuracy under the $0.8$ strategy and $0 \%$ robust accuracy under the $1.2$ strategy, respectively when solving max-loss form. In fact, PWCF achieves $98.33 \%$ and $13 \%$ robust accuracy, respectively for the $\ell_{1.5}$ case; PWCF achieves $93.33 \%$ and $1\%$, respectively for $\ell_{8}$ case---PWCF solves both max-loss form and min-radius form with reasonable quality.
\end{itemize}

\begin{figure}[!tb]
    \centering
    \includegraphics[width=0.45\textwidth]{Figures/L15-L8.png}
    \caption{Robustness radii (y-axis) found by PWCF to solve min-radius form with the $\ell_{1.5}$, $\ell_2$ and $\ell_8$ distances on $60$ CIFAR-10 images. The x-axis represent the sample indices. Effective optimization should respect that the robustness radii found with different metrics satisfy $\ell_{1.5} \ge \ell_2 \ge \ell_{8}$ for every sample.} 
    \label{fig: L1.5 L8 min radius plot} 
\end{figure}

\subsubsection{Solving min-radius form and max-loss form with PD}
\label{subsec: min max perceptual}
Similar to the problems with the $\ell_{1.5}$ and $\ell_{8}$ distances, we are unaware of any existing work that has considered solving min-radius form with PD:\footnote{There are several existing variants of the perceptual distance. Here, we consider the LPIPS distance (first introduced in~\cite{Zhang_2018_CVPR}).}
\begin{align}
\label{Eq. LPIPS Constraint} 
\begin{split}
& d(\mb x, \mb x') \doteq \norm{\phi(\mb x) - \phi(\mb x')}_{2}\\
\text{where} \quad & \phi(\mb x) \doteq [~\wh{g}_{1}(\mb x), \dots, \wh{g}_{L}(\mb x)~]
\end{split}
\end{align}
where $\wh{g}_{1}(\mb x), \dots, \wh{g}_{L}(\mb x)$ are the vectorized intermediate feature maps from pre-trained DNNs (e.g., AlexNet). For max-loss form, three methods are proposed in~\citep{laidlaw2021perceptual}: Perceptual Projected Gradient Descent (PPGD), Lagrangian Perceptual Attack (LPA) and its variant Fast Lagrangian Perceptual Attack (Fast-LPA), all developed in~\cite{laidlaw2021perceptual} based on iterative linearization and projection (PPGD), or penalty method (LPA, Fast-LPA), respectively. In~\citep{laidlaw2021perceptual}, a preset perturbation level $\eps=0.5$ is used in max-loss form (termed perceptual adversarial attack, PAT). 

\begin{figure}[!tb]
    \centering
    \includegraphics[width=0.45\textwidth]{Figures/Min_Radius/imagenetPAT.png}
    \caption{Robustness radius (y-axis) found by solving min-radius form using PWCF with PD on $85$ ImageNet-100 images. The x-axis shows the image indices. The \textcolor{red}{red} dashed line is the proposed bound $\eps$ used to solve max-loss form in~\cite{laidlaw2021perceptual}, which is much larger than each radius found by solving min-radius form with PWCF.} 
    \label{fig: perceptual min radius plot} 
\end{figure}

\begin{table*}[!tb]
\caption{Performance comparison of solving max-loss form with PD for the entire ImageNet-100 validation set, using (clipped) cross-entropy and margin losses, respectively. \textbf{Viol.} reports the ratio of final solutions that violate the constraint; \textbf{Att. Succ.} is the ratio of all feasible and successful attack samples divided by total number of samples---higher indicates more effective optimization performance.}
\label{tab: granso_pat_compare}
\begin{center}
\setlength{\tabcolsep}{1.0mm}{
%\renewcommand\arraystretch{1.25}
% \begin{tabular}{l c c c c c c c}
\begin{tabular}{l c c c c c}
{}
&\multicolumn{2}{c}{\small{\textbf{cross-entropy loss}}}
&{}
&\multicolumn{2}{c}{\small{\textbf{margin loss}}}
\\
\cline{2-3}\cline{5-6}
\vspace{-10pt}
\\
\small{\textbf{Method}}
&\small{\textbf{Viol. ($\%$) $\downarrow$}}
%&\small{\textbf{Qual. ($\uparrow$)}}
&\small{\textbf{Att. Succ. ($\%$) $\uparrow$}}
&{}
&\small{\textbf{Viol. ($\%$) $\downarrow$}}
%&\small{\textbf{Qual. ($\uparrow$)}}
&\small{\textbf{Att. Succ. ($\%$) $\uparrow$}}
\\
% \midrule
%\hline
\toprule
%\hline
\small{Fast-LPA}
&{$73.8$}
% &\textcolor{red}{$\Delta 0.43$}
&\textcolor{black}{$3.54$}
&{}
&{$41.6$}
% &{$\bigtriangleup 1.49$}
&{$56.8$}
\\
\small{LPA}
&{\textbf{0.00}}
% &{$\bigtriangleup 6.02$}
&{$80.5$}
&{}
&{\textbf{0.00}}
% &{$\bigtriangleup 1.34$}
&\textcolor{black}{$97.0$}
\\
% \small{LPA-Clip}
% &{$0.00$}
% &{$\Delta 1.34$}
% &{$97.0$}
% &{}
% &{$0.00$}
% &{$\Delta 5.70$}
% &{$96.9$}
% \\
\small{PPGD}
&{$5.44$}
% &{$\bigtriangleup 0.94$}
&{$25.5$}
&{}
&{\textbf{0.00}}
% &{$\bigtriangledown 1.40$}
&{$38.5$}
\\
\midrule
% \small{PWCF}
% &{$0.00$}
% &{$\Delta 5.52$}
% &{$45.4$}
% &{}
% &{$0.00$}
% &\textcolor{red}{$\Delta 10.9$}
% &{$84.1$}
% \\
\small{PWCF (ours)}
&{$0.62$}
% &{$\bigtriangleup 9.47$}
&\textcolor{red}{$93.6$}
&{}
&{\textbf{0.00}}
% &{$\bigtriangleup 1.45$}
&\textcolor{red}{$100$}
\\
\bottomrule
\end{tabular}
}
\end{center}
\end{table*}


We first plot the robustness radii\footnote{Using the same model adversarially pretrained on ImageNet dataset as in \cref{tab: granso_min}} found by PWCF in solving min-radius form in \cref{fig: perceptual min radius plot} on $85$ ImageNet-100 images. Comparing each robustness radius and the preset $\eps$ used in the max-loss form proposed in~\citep{laidlaw2021perceptual}, we observe that PWCF finds much smaller robustness radii for every sample. We can conclude that: 1) PWCF solves min-radius form with PD reasonably well; 2) the choice of $\eps$ is too large in~\cite{laidlaw2021perceptual} to be a reasonable perturbation budget in max-loss form.

Next, we solve max-loss form on the ImageNet-100 validation set with $\eps=0.5$\footnote{Using the same model as in \cref{fig: perceptual min radius plot}}, reporting both the attack success rate and the constraint violation rate of the solutions found. According to \cref{fig: perceptual min radius plot}, the sample-wise robustness radii are much smaller than the preset $\eps$, indicating that effective solvers should achieve $100\%$ attack success rate with $0\%$ violations. As shown in \cref{tab: granso_pat_compare}, PWCF with margin loss is the only one that meets this standard.



\begin{figure}
    \centering
    \includegraphics[width=0.2\textwidth]{Figures/new_pattern_vis/orig.png}
    \caption{A `fish' image example from the Imagenet-100 validation that is used to generate the pattern visualizations in \cref{Fig:max pattern vis} and \cref{Fig:min pattern vis}.}
    \label{fig:dish image example}
\end{figure}

\section{Different combinations of $\ell$, $d$, and the solvers prefer different patterns}
\label{sec:pattern_theory}

\begin{figure*}[!tb]
\centering
\begingroup 
\setlength{\tabcolsep}{1pt}
\renewcommand{\arraystretch}{0.8}
\begin{tabular}{c c c c c c c c}
\centering
{ }
&{ }
&{ }
&\multicolumn{2}{c}{\textbf{APGD}}
&{ }
&\multicolumn{2}{c}{\textbf{PWCF}}
\\
\cline{4-5}\cline{7-8}
\vspace{-1em}
\\
{ }
&{ }
&{ }
&{cross-entropy}
&{margin}
&{ }
&{cross-entropy}
&{margin}
\\
\textbf{$\ell_1$}
&{ }
&\includegraphics[width=0.08\textwidth]{Figures/markers/err_img.png}
&\includegraphics[width=0.2\textwidth]{Figures/new_pattern_vis/APGD-L1-CE-errVis.png}
&\includegraphics[width=0.2\textwidth]{Figures/new_pattern_vis/APGD-L1-Margin-errVis.png}
&{ }
&\includegraphics[width=0.2\textwidth]{Figures/new_pattern_vis/PyGranso-L1-CE-errVis.png}
&\includegraphics[width=0.2\textwidth]{Figures/new_pattern_vis/PyGranso-L1-Margin-errVis.png}
\\
{ }
&{ }
&\includegraphics[width=0.08\textwidth]{Figures/markers/err_hist.png}
&\includegraphics[width=0.2\textwidth]{Figures/new_pattern_vis/APGD-L1-CE-errHist.png}
&\includegraphics[width=0.2\textwidth]{Figures/new_pattern_vis/APGD-L1-Margin-errHist.png}
&{ }
&\includegraphics[width=0.2\textwidth]{Figures/new_pattern_vis/PyGranso-L1-CE-errHist.png}
&\includegraphics[width=0.2\textwidth]{Figures/new_pattern_vis/PyGranso-L1-Margin-errHist.png}
\\
\cline{1-5}\cline{7-8}
\vspace{-1em}
\\
\textbf{$\ell_2$}
&{ }
&\includegraphics[width=0.08\textwidth]{Figures/markers/err_img.png}
&\includegraphics[width=0.2\textwidth]{Figures/new_pattern_vis/APGD-L2-CE-errVis.png}
&\includegraphics[width=0.2\textwidth]{Figures/new_pattern_vis/APGD-L2-Margin-errVis.png}
&{ }
&\includegraphics[width=0.2\textwidth]{Figures/new_pattern_vis/PyGranso-L2-CE-errVis.png}
&\includegraphics[width=0.2\textwidth]{Figures/new_pattern_vis/PyGranso-L2-Margin-errVis.png}
\\
{ }
&{ }
&\includegraphics[width=0.08\textwidth]{Figures/markers/err_hist.png}
&\includegraphics[width=0.2\textwidth]{Figures/new_pattern_vis/APGD-L2-CE-errHist.png}
&\includegraphics[width=0.2\textwidth]{Figures/new_pattern_vis/APGD-L2-Margin-errHist.png}
&{ }
&\includegraphics[width=0.2\textwidth]{Figures/new_pattern_vis/PyGranso-L2-CE-errHist.png}
&\includegraphics[width=0.2\textwidth]{Figures/new_pattern_vis/PyGranso-L2-Margin-errHist.png}
\\
\cline{1-5}\cline{7-8}
\vspace{-1em}
\\
\textbf{$\ell_\infty$}
&{ }
&\includegraphics[width=0.08\textwidth]{Figures/markers/err_img.png}
&\includegraphics[width=0.2\textwidth]{Figures/new_pattern_vis/APGD-Linf-CE-errVis.png}
&\includegraphics[width=0.2\textwidth]{Figures/new_pattern_vis/APGD-Linf-Margin-errVis.png}
&{ }
&\includegraphics[width=0.2\textwidth]{Figures/new_pattern_vis/PyGranso-Linf-CE-errVis.png}
&\includegraphics[width=0.2\textwidth]{Figures/new_pattern_vis/PyGranso-Linf-Margin-errVis.png}
\\
{ }
&{ }
&\includegraphics[width=0.08\textwidth]{Figures/markers/err_hist.png}
&\includegraphics[width=0.2\textwidth]{Figures/new_pattern_vis/APGD-Linf-CE-errHist.png}
&\includegraphics[width=0.2\textwidth]{Figures/new_pattern_vis/APGD-Linf-Margin-errHist.png}
&{ }
&\includegraphics[width=0.2\textwidth]{Figures/new_pattern_vis/PyGranso-Linf-CE-errHist.png}
&\includegraphics[width=0.2\textwidth]{Figures/new_pattern_vis/PyGranso-Linf-Margin-errHist.png}
\\
\cline{1-8}
\vspace{-1em}
\\

\end{tabular}
\endgroup 
\caption{Visualization of perturbation images found by solving max-loss form with different losses (cross-entropy and margin), different $d$'s ($\ell_1$, $\ell_2$ and $\ell_\infty$) and different solvers (APGD and PWCF). Within each group by $d$, the top rows are the perturbation images $\mb x' - \mb x$, which are normalized to the range $[0, 1]$ for better visualization; the bottom rows are the histograms of the element-wise perturbation magnitude $\abs{\mb x' - \mb x}$, where the x-axes are the absolute pixel values.}
\label{Fig:max pattern vis}
\end{figure*}
\begin{figure*}[!tb]
\centering
\begingroup 
\setlength{\tabcolsep}{1pt}
\renewcommand{\arraystretch}{0.8}
\begin{tabular}{c c c c c c c c}
\centering
{}
&{$\ell_1$}
&{$\ell_2$}
&{$\ell_\infty$}
&{ }
&{$\ell_1$}
&{$\ell_2$}
&{$\ell_\infty$}
\\
{}
&\includegraphics[width=0.16\textwidth]{Figures/new_pattern_vis/FAB-L1-Margin-errVis.png}
&\includegraphics[width=0.16\textwidth]{Figures/new_pattern_vis/FAB-L2-Margin-errVis.png}
&\includegraphics[width=0.16\textwidth]{Figures/new_pattern_vis/FAB-Linf-Margin-errVis.png}
&{ }
&\includegraphics[width=0.16\textwidth]{Figures/new_pattern_vis/PyGranso-Min-L1-Margin-errVis.png}
&\includegraphics[width=0.16\textwidth]{Figures/new_pattern_vis/PyGranso-Min-L2-Margin-errVis.png}
&\includegraphics[width=0.16\textwidth]{Figures/new_pattern_vis/PyGranso-Min-Linf-Margin-errVis.png}
\\
{}
&\includegraphics[width=0.16\textwidth]{Figures/new_pattern_vis/FAB-L1-Margin-errHist.png}
&\includegraphics[width=0.16\textwidth]{Figures/new_pattern_vis/FAB-L2-Margin-errHist.png}
&\includegraphics[width=0.16\textwidth]{Figures/new_pattern_vis/FAB-Linf-Margin-errHist.png}
&{ }
&\includegraphics[width=0.16\textwidth]{Figures/new_pattern_vis/PyGranso-Min-L1-Margin-errHist.png}
&\includegraphics[width=0.16\textwidth]{Figures/new_pattern_vis/PyGranso-Min-L2-Margin-errHist.png}
&\includegraphics[width=0.16\textwidth]{Figures/new_pattern_vis/PyGranso-Min-Linf-Margin-errHist.png}
\\
\cline{2-4}\cline{6-8}
\vspace{-1em}
\\
{}
&\multicolumn{3}{c}{\textbf{FAB}}
&{ }
&\multicolumn{3}{c}{\textbf{PWCF}}
\\
\end{tabular}
\endgroup 
\caption{Visualizations of perturbation images ($\mb x' - \mb x$, top row) and the histogram of element-wise perturbation magnitude ($\abs{\mb x' - \mb x}$, bottom row) by solving min-radius form. Note that the comparison between FAB and PWCF may not be as straightforward as \cref{Fig:max pattern vis} because the radii found by solving min-radius form are likely different in scale. However, the shape of the histograms can still reveal the pattern differences.}
\label{Fig:min pattern vis}
\end{figure*}
\begin{figure*}[!tb]
\centering
\begingroup 
\setlength{\tabcolsep}{1pt}
\renewcommand{\arraystretch}{0.8}
\begin{tabular}{c c c c c c}
\centering
{}
&{cross-entropy}
&{margin}
&{ }
&{cross-entropy}
&{margin}
\\
\includegraphics[width=0.09\textwidth]{Figures/markers/l1-marker.png}
&\includegraphics[width=0.22\textwidth]{Figures/new_pattern_vis_ratio/APGD-L1-CE-err-l1l2-ratio.png}
&\includegraphics[width=0.22\textwidth]{Figures/new_pattern_vis_ratio/APGD-L1-Margin-err-l1l2-ratio.png}
&{ }
&\includegraphics[width=0.22\textwidth]{Figures/new_pattern_vis_ratio/PyGranso-L1-CE-err-l1l2-ratio.png}
&\includegraphics[width=0.22\textwidth]{Figures/new_pattern_vis_ratio/PyGranso-L1-Margin-err-l1l2-ratio.png}
\\
% \cline{1-2}
\vspace{-1em}
\\
\includegraphics[width=0.09\textwidth]{Figures/markers/l2-marker.png}
&\includegraphics[width=0.22\textwidth]{Figures/new_pattern_vis_ratio/APGD-L2-CE-err-l1l2-ratio.png}
&\includegraphics[width=0.22\textwidth]{Figures/new_pattern_vis_ratio/APGD-L2-Margin-err-l1l2-ratio.png}
&{ }
&\includegraphics[width=0.22\textwidth]{Figures/new_pattern_vis_ratio/PyGranso-L2-CE-err-l1l2-ratio.png}
&\includegraphics[width=0.22\textwidth]{Figures/new_pattern_vis_ratio/PyGranso-L2-Margin-err-l1l2-ratio.png}
\\
% \cline{1-2}
\vspace{-1em}
\\
\includegraphics[width=0.09\textwidth]{Figures/markers/linf-marker.png}
&\includegraphics[width=0.22\textwidth]{Figures/new_pattern_vis_ratio/APGD-Linf-CE-err-l1l2-ratio.png}
&\includegraphics[width=0.22\textwidth]{Figures/new_pattern_vis_ratio/APGD-Linf-Margin-err-l1l2-ratio.png}
&{ }
&\includegraphics[width=0.22\textwidth]{Figures/new_pattern_vis_ratio/PyGranso-Linf-CE-err-l1l2-ratio.png}
&\includegraphics[width=0.22\textwidth]{Figures/new_pattern_vis_ratio/PyGranso-Linf-Margin-err-l1l2-ratio.png}
\\
\cline{1-3}\cline{5-6}
\vspace{-1em}
\\
{}
&\multicolumn{2}{c}{\textbf{APGD}}
&{ }
&\multicolumn{2}{c}{\textbf{PWCF}}
\\

\end{tabular}
\endgroup 
\caption{Histograms of the sparsity measure by solving max-loss form with different $\ell$'s (cross-entropy and margin losses), different $d$'s ($\ell_1$, $\ell_2$ and $\ell_\infty$) and different solvers (APGD and PWCF). For fair comparisons, we use a model non-adversarially trained here so that each $x'$ is a successful adversarial example. With the same $d$ and $\ell$, the difference in the distributions of the sparsity measure between APGD and PWCF clearly exists; the sparsity variation is noticeably greater in PWCF than in APGD when the same $d$ and $\ell$ are used.}
\label{Fig:max pattern hist}
\end{figure*}
\input{Sections/section_elements/Fig-New-Pattern-Histo-min.tex}

We now demonstrate that using different combinations of \textbf{1)} distance metrics $d$, \textbf{2)} solvers, and \textbf{3)} losses $\ell$ can lead to different sparsity patterns in the following two ways:
\begin{enumerate}[leftmargin=*]
    \item \textbf{Visualization of perturbation images:} we take a `fish' image (\cref{fig:dish image example}) from ImageNet-100 validation set, employ various combinations of losses $\ell$, distance metrices $d$ and solvers to the max-loss form and min-radius form. \cref{Fig:max pattern vis} and \cref{Fig:min pattern vis} visualize the perturbation image $\mb x' - \mb x$, and the histogram of the element-wise error magnitude $\abs{\mb x' - \mb x}$ to display the difference in pattern.
    \item \textbf{Statistics of sparsity levels:} we use the soft sparsity measure $\norm{\mb x' - \mb x}_1 / \norm{\mb x' - \mb x}_2$ to quantify the patterns---the higher the value, the denser the pattern. \cref{Fig:max pattern hist} and \cref{Fig:min pattern hist} display the histograms of the sparsity levels of the error images derived by solving max-loss form and min-radius form, respectively. Here, we used a fixed set of $500$ ImageNet-100 images from the validation set.
\end{enumerate} 

Contrary to the $\ell_1$ distance which induces sparsity, $\ell_\infty$ promotes dense perturbations with comparable entry-wise magnitudes~\cite{StuderEtAl2012Signal} and $\ell_2$ promotes dense perturbations whose entries follow power-law distributions. These varying sparsity patterns due to $d$'s are evident when we compare the solutions with the same solver and loss but with different distances, where \textbf{1)} the shapes of the histograms in \cref{Fig:min pattern vis} and the ranges of the values are very different; \textbf{2)} the sparsity measures show a shift from left to right along the horizontal axis in \cref{Fig:max pattern hist} and \cref{Fig:min pattern hist}. In addition to $d$'s, we also highlight other patterns induced by the loss $\ell$ and the solver:
\begin{itemize}[leftmargin=*]
    \item \textbf{Using margin and cross-entropy losses in solving max-loss form induce different sparsity patterns} \quad 
    Columns `cross-entropy' and `margin' of PWCF in \cref{Fig:max pattern vis} depict the pattern difference with clear divergences in the histograms of error magnitude; for example, the error values of PWCF-$\ell_2$-margin are more concentrated towards $0$ compared to PWCF-$\ell_2$-cross-entropy. The sparsity measures in \cref{Fig:max pattern hist} can further confirm the existence of the difference due to the loss used to solve max-loss form, more for PWCF than APGD.
    \item \textbf{PWCF's solutions have more variety in sparsity than APGD and FAB} \quad
    For the same $d$ and loss used to solve max-loss form, \cref{Fig:max pattern hist} shows that PWCF's solutions have a wider spread in the sparsity measure than APGD. The same observation can be found in \cref{Fig:min pattern hist} as well between PWCF and FAB in solving min-radius form.
\end{itemize} 

\begin{figure}[!tb]
% \vspace{-1em}
  \centering
    \includegraphics[width=0.3\textwidth]{Figures/multi_sol_2.png}
  \caption{Geometry of max-loss form with multiple global maximizers. $\mb u$ and $\mb v$ are the basis vectors of the 2-dimensional coordinate. Here we consider the $\ell_1$-norm ball around $\mb x$, and ignore the box constraint $\mb x' \in [0, 1]^n$. Depending on the loss $\ell$ used, part or the whole of the blue regions becomes the set of global or near-global maximizers.} 
  \label{fig:multi_sol} 
  % \vspace{-1em}
\end{figure}
\begin{figure}
% \vspace{-1em}
  \centering
  \includegraphics[width=0.4\textwidth]{Figures/min_l1_dist_arrow.png}
  \caption{Histogram of the $\ell_1$ robustness radii estimated by solving min-radius form for $88$ CIFAR-10 images. $\eps=12$ (red dashed line) is the typical preset perturbation budget used in max-loss form (\formulation (\ref{eq:robust_loss})).} 
  \label{fig: eps selection} 
  % \vspace{-1em}
\end{figure}

Here, we provide a conceptual explanation of why different sparsity patterns can occur. We take the $\ell_1$ distance (i.e., $\norm{\mb x - \mb x'}_1 \le \eps$) and ignore the box constraint $\mb x' \in [0, 1]$ in max-loss form as an example. For simplicity, we take the loss $\ell$ as $0/1$ classification error $\ell (\mb y, f_{\mb \theta}(\mb x')) = \indicator{\max_{i} f^i_{\mb \theta}(\mb x') \ne y}$. Note that $\ell$ is maximized whenever $f^i_{\mb \theta}(\mb x') > f^y_{\mb \theta}(\mb x')$ for a certain $i \ne y$, so that $\mb x'$ crosses the local decision boundary between the $i$-th and $y$-th classes; see \cref{fig:multi_sol}. In practice, people set a substantially larger perturbation budget in max-loss form than the robustness radius of many samples, which can be estimated by solving min-radius form---see \cref{fig: eps selection}. Thus, there can be infinitely many global maximizers (the shaded blue regions in \cref{fig:multi_sol}). As for the patterns, the solutions in the shaded blue region on the left are denser in pattern than the solutions on the top. For other general losses, such as cross-entropy or margin loss, the set of global maximizers may change, but the patterns can possibly be more complicated due to the typically complicated nonlinear decision boundaries associated with DNNs. As for min-radius form, multiple global optimizers and pattern differences can exist as well, but the optimizers share the same radius.




\section{Implications from the variety patterns}
\label{sec: implication} 
Now that we have demonstrated the complex interplay of loss $\ell$, distance metric $d$, and the numerical solver for the final solution patterns in \cref{sec:pattern_theory}, we will discuss what this can imply for the reseach of adversarial robustness.

\subsection{Current empirical RE may not be sufficient}
\label{subsec: limiation of current RE}
As introduced in \cref{Sec:introduction}, the most popular empirical RE practice currently is solving max-loss form with a preset level of $\eps$, using a fixed set of algorithms. Then robust accuracy is reported using the perturbed samples found~\cite{croce2021robustbench, papernot2016technical, rauber2017foolbox}. Here, we challenge its validity for measuring robustness.

\subsubsection{Diversity matters for robust accuracy to be trustworthy}
\label{subsec: diversity matters for robust accuracy}
As shown in \cref{sec:pattern_theory}, the perturbations found by different numerical methods can have different sparsity patterns; \cref{{tab: granso_l1_acc}} also shows that combining multiple methods can lead to a lower robust accuracy than any single method. This implies that for robust accuracy to be numerically reliable, including as many solvers to cover as many patterns as possible is necessary. Although works as~\cite{CarliniEtAl2019Evaluating, croce2020reliable, gilmer2018motivating} have mentioned the necessity of diversity in solvers, our paper is the first to quantify such diversity in terms of sparsity patterns from their solutions. However, the existence of infinitely many patterns may be possible, and it is thus possible that faithful robust accuracy may not be able to achieve in practice.

\subsubsection{Robust accuracy is not a good robustness metric}
\label{subsec: robust accuracy is bad metric}
The motivation of using max-loss form for RE is usually associated with the attacker-defender setting, where solutions ($\mb x'$) are viewed as a test bench for all possible future attacks. Ideally, the DNNs must be robust against all of the adversarial samples found. However, it is questionable whether robust accuracy faithfully reflects this notion of robustness: 
\textbf{1)} why the commonly used budget $\eps$ in max-loss form is a reasonable choice needs to be justified. For example, $\eps=0.03$ is commonly used for the $\ell_\infty$ distance, e.g., in~\cite{croce2021robustbench}. We could not find rigorous answers in the previous literature and suspect that the choices are purely empirical. For example, \cite{croce2021robustbench} states the motivation as
\say{...the true label should stay the same for each in-distribution input within the perturbation set...}
but this claim can also support using other values; \textbf{2)} more importantly, Fig. 1. in~\cite{sridhar2022towards} shows that a model having a higher robust accuracy than other models at one $\eps$ level does not imply that such model is also more robust at other levels. The clean-robust accuracy trade-off~\cite{raghunathan2019adversarial, yang2020closer} may also be interpreted similarly\footnote{The `clean-robust accuracy trade-off' refers to the phenomenon where a model that is non-adversarially trained has the best clean accuracy (at level $\eps=0$) and the worst robust accuracy (at the commonly used $\eps$), and vise versa for the models that are adversarially trained.}---they are just most robust to different $\eps$ levels. Thus, robust accuracy is not a complete and trustworthy measure, and conclusions about robustness drawn from robust accuracy based on a single $\eps$ level are misleading.

\subsubsection{Robustness radius is a better robustness measures}
\label{subsec: min radius is better RE metric}
If our goal is indeed to understand the robustness limit of a given DNN model, solving min-loss-form seems more advantageous, especially that:
\begin{enumerate}[leftmargin=*]
    \item \textbf{Robustness radius is more reliable:} unlike that the pattern differences in solving max-loss form can lead to unreliable robust accuracy due to the possibility of multiple solutions, the robustness radius found by solving min-radius form is not sensitive to the existence of multiple solutions.
    \item \textbf{Robustness radius is sample adaptive:} in contrast to the rigid perturbation budget $\eps$ used in max-loss form, the robustness radius is the (sample-wise) distance to the closest decision boundaries.
\end{enumerate}
A clear application of the robustness radius is that we can identify hard (less robust) samples for a given model if the corresponding robustness radii are small.

\begin{figure*}[!tb]
\centering
\begingroup 
\setlength{\tabcolsep}{1pt}
\renewcommand{\arraystretch}{0.8}
\begin{tabular}{c c c c c c c c}
\centering
{}
&\multicolumn{3}{c}{\textbf{APGD cross-entropy}}
&{ }
&{\textbf{LPA}}
&{\textbf{PWCF}}
&{\textbf{PWCF}}
\\
{}
&\includegraphics[width=0.16\textwidth]{Figures/PAT_pattern_vis_ratio/APGD-L1-CE-err-l1l2-ratio.png}
&\includegraphics[width=0.16\textwidth]{Figures/PAT_pattern_vis_ratio/APGD-L2-CE-err-l1l2-ratio.png}
&\includegraphics[width=0.16\textwidth]{Figures/PAT_pattern_vis_ratio/APGD-Linf-CE-err-l1l2-ratio.png}
&{ }
&\includegraphics[width=0.16\textwidth]{Figures/PAT_pattern_vis_ratio/Lagrange-L2-Margin-err-l1l2-ratio.png}
&\includegraphics[width=0.16\textwidth]{Figures/PAT_pattern_vis_ratio/PAT-Granso-L2-Margin-err-l1l2-ratio.png}
&\includegraphics[width=0.16\textwidth]{Figures/PAT_pattern_vis_ratio/PAT-Granso-L1-Margin-err-l1l2-ratio.png}
\\
{}
&{\textbf{(a)} $\ell_1$}
&{\textbf{(b)} $\ell_2$}
&{\textbf{(c)} $\ell_\infty$}
&{ }
&{\textbf{(d)} PD-$\ell_2$}
&{\textbf{(e)} PD-$\ell_2$}
&{\textbf{(f)} PD-$\ell_1$}
\\
\end{tabular}
\endgroup 
\caption{Histograms of the sparsity measure on 500 ImageNet-100 images by solving max-loss form with different $d$'s ($\ell_p$ and PD) and different solvers (APGD with cross-entropy loss, LPA and PWCF). LPA-PD-$\ell_2$ is the perceptual attack used in~\cite{laidlaw2021perceptual}. 
Using PD in max-loss form also results in different sparsity patterns due to the solver and the inner $\ell_p$ distance used (see PD-$\ell_2$, PD-$\ell_1$ related figures above).}
\label{Fig:pat l2 sparsity}
\end{figure*}

\subsection{Adversarial training may not help to achieve generalizable robustness}
\label{subsec: difficult in achieving AR}

\subsubsection{Solution patterns can explain why $\ell_p$ robustness does not generalize}
\label{subsec: ungeneralizable lp}
Despite the effort of finding ways to achieve generalizable AR, it is widely observed that AR achieved by AT does not generalize across simple $\ell_p$ distances~\cite{maini2020adversarial,CroceHein2019Provable}. For example, models adversarially trained by $\ell_\infty$-attacks do not achieve good robust accuracy with $\ell_2$-attacks; $\ell_1$ seems to be a strong attack for all other $\ell_p$ distances, and even on itself. 
Note that \cite{madry2017towards} has observed that the (approximate) global maximizers are distinct and spatially scattered; the patterns we discussed in \cref{sec:pattern_theory} provide a plausible explanation of why AR achieved by AT is expected not to be generalizable---the model just cannot perform well on an unseen distribution (patterns) from what it has seen during training.

\subsubsection{Adversarial training with perceptual distances does not solve the generalization issue}
\label{subsec: ungeneralizable perceptual metric}
\cite{laidlaw2021perceptual} claims that using PD (\cref{Eq. LPIPS Constraint}) in max-loss form can approximate the universal set of adversarial attacks, and models adversarially trained with PAT can generalize to other unseen attacks. However, we challenge the above conclusion: `unseen attacks' does not necessarily translate to `novel perturbations', especially if we investigate the patterns:
\begin{enumerate}[leftmargin=*]
    \item If we test the models pretrained by $\ell_2$-attack and PAT\footnote{Correspond to $\ell_2$ and PAT-AlexNet in Table 3 of \cite{laidlaw2021perceptual}} in \cite{laidlaw2021perceptual} by APGD-CE-$\ell_{1}~(\eps=1200)$ attack (on ImageNet-100 images), both will achieve $0 \%$ robust accuracy---models pretraiend with PAT do not generalize better to $\ell_1$ attacks compared with others.
    \item By investigating the sparsity patterns similar to \cref{sec:pattern_theory}, the adversarial perturbations generated by solving max-loss form with PD are shown to be similar to the APGD-CE-$\ell_2$ generated ones, see (a)-(d) in~\cref{Fig:pat l2 sparsity}. This may explain why the $\ell_2$ and PAT  pre-trained models in \cite{laidlaw2021perceptual} have comparable robust accuracy against multiple tested attacks.
    \item Substituting the $\ell_2$ distance by $\ell_1$ in \cref{Eq. LPIPS Constraint} as the new PD:
    \begin{align}
        d(\mb x, \mb x') \doteq \norm{\phi(\mb x) - \phi(\mb x')}_{1} ~ \text{,}
    \end{align}
    the solution patterns will change; see (e)-(f) in~\cref{Fig:pat l2 sparsity}. Furthermore, (d)-(e) in \cref{Fig:pat l2 sparsity} also shows that different solvers (LPA and PWCF) will also result in different patterns even for PD---PAT will likely suffer from the pattern differences the same way as popular $\ell_p$-attacks, thus not being `universal'.
\end{enumerate} 
To conclude, we think that it is so far unclear whether using the perceptual distances in the AT pipeline can be beneficial in addressing the generalization issue in robustness.





\section{Discussion and New Perspectives}\label{sec:discuss}
% and Future Directions

In this section, we first discuss challenges and practical considerations, including non-stationarity, heterogeneity, unobserved confounders, subsampling, and expert knowledge.
Then, two new perspectives of temporal causal discovery are provided, which in our opinion will be a promising avenue for future research.

\subsection{Challenges and Practical Considerations}



\textbf{Non-stationarity of data:} We are often faced with non-stationarity in practical scenarios, where the probability distributions of temporal variables conditional on their causes or even the causal relations may change across time, especially for temporal data.
In this condition, causal discovery approaches presuming a fixed causal model may give misleading results. 
Whereas, several types of research have shown that non-stationarity contains information for causal discovery \cite{CD_from_change/conf/uai/TianP01, CD_from_change/peters2016causal, Discussion/Nonstation_hetero/ijcai_ZhangHZGS17, Discussion/Nonstation/state_space_icml_Huang0GG19}.
Thus, it's important to properly tackle the non-stationarity in applications.
Non-stationarity may result from the change of underlying systems and can be seen as a soft intervention \cite{soft_interv/korb2004varieties} done by nature. 
Following this idea, a line of work \cite{Discussion/Nonstation_hetero/ijcai_ZhangHZGS17, Discussion/Nonstation_hetero2/jmlr/Huang0ZRSGS20} leverages a surrogate such as time and domain index to account for nonstationarity where the causal relations are changed, and the CD-NOD framework is proposed. 
Instead of leveraging informative non-stationarity to causal structure learning, another set of research focuses on modeling time-varying relationships \cite{Discussion/Nonstation/pr_GaoY22}. 
Besides, the approach for slowly varying non-stationary process, such as evolutionary spectral and locally stationary processes, is proposed in \cite{Discussion/Nonstation/slowly_varying/du2020causal}.






\textbf{Heterogeneity of data:} In causal discovery for practical applications, the heterogeneity of data lies in two levels: (1) The interacting temporal processes are heterogeneous (having different distributions), for instance, causally related meteorological observations from different stations are influenced by several major weather systems separately \cite{Discussion/heterogeneous/pakdd_BehzadiHP19}. (2) The underlying generating process changes across data sets or different domains \cite{intro/nonts_surveys/glymour2019review}, for instance stock prices from different markets \cite{Discussion/Nonstation_hetero2/jmlr/Huang0ZRSGS20} or individual behaviours in different paradigms \cite{MTS/Attention/icdm_InGRA_ChuWMJZY20}.
For the first condition where the heterogeneity exists among temporal variables, the inferred relations of the traditional causal discovery approaches, which have been designed for specific homogeneous data types, may be inaccurate. As a remedy, several variants of Granger causality, based on methods such as generalized linear models and minimum message length, are proposed in \cite{Applications/anomaly/work2_icdm_BehzadiHP17, Discussion/heterogeneous/pakdd_BehzadiHP19, Discussion/heterogeneous/entropy/Hlavackova-Schindler20}.
For the second condition, a line of work \cite{Discussion/Nonstation_hetero/ijcai_ZhangHZGS17,  Discussion/Nonstation_hetero2/jmlr/Huang0ZRSGS20} leverages the distribution shift from heterogeneity as a soft intervention to assist causal structure learning, which is similar to that in non-stationary data.  
Whereas, another line of causal discovery approaches \cite{MTS/Attention/icdm_InGRA_ChuWMJZY20, Discussion/NewForm/ACD_LoweMSW22} in the second condition focuses on inductively modeling typical structure in heterogeneous data within an end-to-end framework. 



\textbf{Unobserved confounders:}
In practice, we are often met with cases where causal sufficiency is violated, \ie, there exist unobserved confounders. 
This challenging setting may lead to incorrect causal relations~\cite{MTS/FCM/VAR_LINGAM_extend2_icml_GeigerZSGJ15}.
As summarized in Table~\ref{tab:ts_category_overview}, most temporal causal discovery approaches cannot handle unobserved confounders in a straightforward way.
Several constraint-based approaches are designed without causal sufficiency and approaches
Besides, unobserved confounders are modeled by applying a structural bias in~\cite{Discussion/NewForm/ACD_LoweMSW22}.
Several recent studies termed as causal representation learning take a new perspective on unobserved confounders.
It will be detailed in subsection (\ref{subsection:causal_rep}).

\textbf{Subsampling:} In real-world applications, temporal data, especially time series, may be sampled at a rate lower than the rate of the underlying causal process due to the difficulties in data collection.
An ordinary causal discovery algorithm for sub-sampled time series may lead to spurious causal relations and missed ones. 
Several remarks and approaches~\cite{Discussion/subsample/work1, Discussion/subsample/work2_icml_GongZSTG15, Discussion/subsample/work3_nips_rateagnostic_PlisDFC15, Discussion/subsample/uai_subsample_aggr_GongZSGT17, Discussion/subsample/work5_pgm_constraintOPT_HyttinenPJED16, Discussion/subsample/biometrika/tank2019identifiability} are proposed for this issue.

\textbf{Expert knowledge: }Expert knowledge can help the causal discovery process in practice.  % 要强调practical issues.
The approaches of fusing expert knowledge can be categorized into three types~\cite{intro/nonts_surveys/BN21}: (1) \textit{Soft constraints}: the learning process can be influenced by the knowledge~\cite{Discussion/knowledge/ausai/ODonnellNHKAH06}. % (\ie, conditions given with a probability $0<p<1$).
(2) \textit{Hard constraints}: the learnt structure must conform to the enforced requirements (\ie, conditions given with a probability $p=0$ or $p=1$). 
In~\cite{Discussion/knowledge/artmed/AsvatourianLML20}, hard constraints are leveraged in structure learning with a time dependant exposure.
Studies in~\cite{MTS/SB/NTS_NOTEARS} add prior knowledge forbidding the existence of intra-slice dependencies, which is helpful to recover edges that are not explicitly encoded by the prior knowledge.
(3) \textit{Interactive learning}: the human input is leveraged in the learning process~\cite{Discussion/knowledge/ecsqaru/MessaoudLA09, Discussion/knowledge/kdd/MelkasSCMNMP21,https://doi.org/10.48550/arxiv.2206.05420, 9222294}.








\subsection{New perspectives}


\subsubsection{Extension in amortized and supervised paradigms}


In the traditional paradigms, causal discovery methods mostly either treat observational data separately or train a distinct model for each individual. 
These methods do not make full use of the common structure across different samples or supervised information from the datasets whose causal structures are clearly explored, thus suffering from several issues such as the small sample challenge and lack the inductive capability.
Recently, causal discovery is conducted in new paradigms to solve this problem. We can roughly categorize them into two groups: methods based on \textbf{amortized modeling} \cite{MTS/Attention/icdm_InGRA_ChuWMJZY20, Discussion/NewForm/ACD_LoweMSW22}, and methods based on \textbf{supervised learning} \cite{benozzo2017supervised, wang2022meta}.
We introduce them in this subsection, which we believe are a promising avenue for future research. 


In amortized modeling, a global causal discovery framework is trained for individuals with different causal structures. 
As for scenarios with temporal data, these approaches have been detailed in \ref{subsection:NN_Granger} as the deep learning extension of Granger causality with inductive modeling.
InGRA \cite{MTS/Attention/icdm_InGRA_ChuWMJZY20} leverages prototype learning to extract common causal structure while ACD \cite{Discussion/NewForm/ACD_LoweMSW22} proposes an encoder-decoder framework to conduct amortized causal discovery. These methods make full use of information from massive samples and are able to infer causal relations for newly arrived individuals, which are useful in real-world applications such as e-commerce, social network, and neuroimages.

Another line of work has predominately focused on treating the inference process as a black box and learning the mapping from sample data to causal graph structures via supervised learning. Here the label information is causal structure and can be easily accessed in synthetic datasets. 
Earlier work \cite{Discussion/NewForm/RCC/jmlr/Lopez-PazMR15, DBLP:conf/aaai/TonSF21} on learning causal relations by supervised learning is restricted to learning pairwise causal direction where the problem is cast into a classification task to distinguish between $X \to Y$ and $Y \to X$ by using observed samples.
It's later extended to discovery graph structure in \cite{Discussion/NewForm/DAG_EQ/corr/abs-2006-04697,petersen2022causal}.
As the labeled information for training is often originated from synthetic data or real-world datasets which have been explored, the requirement of a supervised approach, in which the distributions of training and test data match or highly overlap, is not guaranteed. In \cite{Discussion/NewForm/ML4S/kdd/00040DJWH022, Discussion/NewForm/CSIvA_DeepMind}, methods such as vicinal graph and meta-learning are leveraged in supervised causal discovery to tackle this `domain shift' issue.  
For the temporal setting, a supervised estimation of Granger causality between time series is proposed in \cite{benozzo2017supervised}. As a recent advance, a method for learning causal discovery is proposed in \cite{wang2022meta} where the learned from large datasets with known causal relations outperform the algorithm in the traditional paradigm when testing on temporal datasets such as fMRI. 
% It's also noted in \cite{wang2022meta} that the causal discovery algorithms in traditional paradigm depart from strong human assumptions about causality. In these approaches (such as constraint-based, score-based and Granger causality), human intuition is implemented in different form. 




% \subsubsection{Extension causal discovery towards causal representation learning (to edit)}
\subsubsection{Extension in causal representation learning}
\label{subsection:causal_rep}
% \subsection{Nonlinear ICA, causal representation learning...}

Extracting the causes of particular phenomena whether explicitly or implicitly from a deep learning black box can be beneficial to the downstream tasks.
The aforementioned causal discovery methods focus on inferring relations between observed variables, or start from the premise that the causal variables are given before hand.
Although some approaches learn causal relations under unobserved variables.
There exist real-world observations (e.g., sensor measurements, image pixels in video) which are not well structured to causal variables to begin with. 
As a generalization of causal discovery from observed variables, there has recently been a growing interest in \textbf{causal representation learning} \cite{CausalRepresentation/nontemp/icml/LocatelloPRSBT20, CausalRepresentation/nontemp/towardsCRL/ScholkopfLBKKGB21, CausalRepresentation/nontemp/CausalVAE/YangLCSHW21}, which aims at learning representation of causal factors in an underlying system.
It estimates latent causal variable graphs from observations.




A line of works in causal representation learning identifies independent factors of variations based on disentanglement and Independent Component Analysis (ICA).
At the heart of this methodology is the postulation of mutually independent latent factors.
It's hard to identify true latent variables, especially in general nonlinear cases.
As a remedy, recent approaches \cite{CausalRepresentation/nontemp/icml/LocatelloPRSBT20, CausalRepresentation/iVAE_nontemp/aistats/KhemakhemKMH20, DBLP:conf/aistats/HyvarinenM17, DBLP:conf/nips/HyvarinenM16} leverage additional information in multiple views, auxiliary variables, or temporal structure, combined with deep learning methods like VAEs and contrastive learning.
A connection between ICA and causality has been recently drawn in \cite{CausalRepresentation/IMA/nontemp/nips/GreseleKSSB21, DBLP:conf/uai/Monti0H19}.
In the context of temporal data, the identifiability of causal variables from temporal sequences is discussed in latent temporal causal process estimation (\textbf{LEAP}) \cite{Discussion/latent/iclr_LEAP_YaoSHS022}. It first provides causal identifiability conditions in a nonparametric, nonstationary setting, and a parametric setting. Then it proposes a learning framework to extract latent causal relations, which extends VAE with a learned causal process network by enforcing the assumed conditions.
The non-stationary noise, modeled by flow-based estimators, can be viewed as a soft intervention to aid identification.
In line with LEAP, subsequent works \cite{TDRL_DBLP:journals/corr/abs-2210-13647} extend the identification theory to a more general case.   % Change to NIPS form citation




Another line of work leverage intervention and data augmentation to help to identify latent causal relations. Under data augmentation, it's demonstrated in \cite{CausalRepresentation/line2/nips/KugelgenSGBSBL21} that common contrastive learning methods can block-identify causal variables that remain unchanged. 
For the temporal setting, \textbf{CITRIS} \cite{CausalRepresentation/CITRIS/icml/LippeMLACG22} is proposed. It's a VAE framework learning causal representation where latent causal factors have possibly been interved on.
By using intervention target information for identification, CITRIS is devoid of suffering from functional or distributional form constraints.
Besides, causal factors in CITRIS are considered as either scalars or potentially multidimensional vectors, which is more practical in complex scenarios. Along this line of work, instantaneous causal relations are extracted in iCITRIS \cite{CausalRepresentation/interv/iCITRIS/abs-2206-06169}.























% Sample body text. Sample body text. Sample body text. Sample body text. Sample body text. Sample body text. Sample body text. Sample body text.

% \section{This is an example for first level head---section head}\label{sec3}

% \subsection{This is an example for second level head---subsection head}\label{subsec2}

% \subsubsection{This is an example for third level head---subsubsection head}\label{subsubsec2}

% Sample body text. Sample body text. Sample body text. Sample body text. Sample body text. Sample body text. Sample body text. Sample body text. 

% \section{Equations}\label{sec4}

% Equations in \LaTeX\ can either be inline or on-a-line by itself (``display equations''). For
% inline equations use the \verb+$...$+ commands. E.g.: The equation
% $H\psi = E \psi$ is written via the command \verb+$H \psi = E \psi$+.

% For display equations (with auto generated equation numbers)
% one can use the equation or align environments:
% \begin{equation}
% \|\tilde{X}(k)\|^2 \leq\frac{\sum\limits_{i=1}^{p}\left\|\tilde{Y}_i(k)\right\|^2+\sum\limits_{j=1}^{q}\left\|\tilde{Z}_j(k)\right\|^2 }{p+q}.\label{eq1}
% \end{equation}
% where,
% \begin{align}
% D_\mu &=  \partial_\mu - ig \frac{\lambda^a}{2} A^a_\mu \nonumber \\
% F^a_{\mu\nu} &= \partial_\mu A^a_\nu - \partial_\nu A^a_\mu + g f^{abc} A^b_\mu A^a_\nu \label{eq2}
% \end{align}
% Notice the use of \verb+\nonumber+ in the align environment at the end
% of each line, except the last, so as not to produce equation numbers on
% lines where no equation numbers are required. The \verb+\label{}+ command
% should only be used at the last line of an align environment where
% \verb+\nonumber+ is not used.
% \begin{equation}
% Y_\infty = \left( \frac{m}{\textrm{GeV}} \right)^{-3}
%     \left[ 1 + \frac{3 \ln(m/\textrm{GeV})}{15}
%     + \frac{\ln(c_2/5)}{15} \right]
% \end{equation}
% The class file also supports the use of \verb+\mathbb{}+, \verb+\mathscr{}+ and
% \verb+\mathcal{}+ commands. As such \verb+\mathbb{R}+, \verb+\mathscr{R}+
% and \verb+\mathcal{R}+ produces $\mathbb{R}$, $\mathscr{R}$ and $\mathcal{R}$
% respectively (refer Subsubsection~\ref{subsubsec2}).

% \section{Tables}\label{sec5}

% Tables can be inserted via the normal table and tabular environment. To put
% footnotes inside tables you should use \verb+\footnotetext[]{...}+ tag.
% The footnote appears just below the table itself (refer Tables~\ref{tab1} and \ref{tab2}). 
% For the corresponding footnotemark use \verb+\footnotemark[...]+

% \begin{table}[h]
% \begin{center}
% \begin{minipage}{174pt}
% \caption{Caption text}\label{tab1}%
% \begin{tabular}{@{}llll@{}}
% \toprule
% Column 1 & Column 2  & Column 3 & Column 4\\
% \midrule
% row 1    & data 1   & data 2  & data 3  \\
% row 2    & data 4   & data 5\footnotemark[1]  & data 6  \\
% row 3    & data 7   & data 8  & data 9\footnotemark[2]  \\
% \botrule
% \end{tabular}
% \footnotetext{Source: This is an example of table footnote. This is an example of table footnote.}
% \footnotetext[1]{Example for a first table footnote. This is an example of table footnote.}
% \footnotetext[2]{Example for a second table footnote. This is an example of table footnote.}
% \end{minipage}
% \end{center}
% \end{table}

% \noindent
% The input format for the above table is as follows:

% %%=============================================%%
% %% For presentation purpose, we have included  %%
% %% \bigskip command. please ignore this.       %%
% %%=============================================%%
% \bigskip
% \begin{verbatim}
% \begin{table}[<placement-specifier>]
% \begin{center}
% \begin{minipage}{<preferred-table-width>}
% \caption{<table-caption>}\label{<table-label>}%
% \begin{tabular}{@{}llll@{}}
% \toprule
% Column 1 & Column 2 & Column 3 & Column 4\\
% \midrule
% row 1 & data 1 & data 2	 & data 3 \\
% row 2 & data 4 & data 5\footnotemark[1] & data 6 \\
% row 3 & data 7 & data 8	 & data 9\footnotemark[2]\\
% \botrule
% \end{tabular}
% \footnotetext{Source: This is an example of table footnote. 
% This is an example of table footnote.}
% \footnotetext[1]{Example for a first table footnote.
% This is an example of table footnote.}
% \footnotetext[2]{Example for a second table footnote. 
% This is an example of table footnote.}
% \end{minipage}
% \end{center}
% \end{table}
% \end{verbatim}
% \bigskip
% %%=============================================%%
% %% For presentation purpose, we have included  %%
% %% \bigskip command. please ignore this.       %%
% %%=============================================%%

% \begin{table}[h]
% \begin{center}
% \begin{minipage}{\textwidth}
% \caption{Example of a lengthy table which is set to full textwidth}\label{tab2}
% \begin{tabular*}{\textwidth}{@{\extracolsep{\fill}}lcccccc@{\extracolsep{\fill}}}
% \toprule%
% & \multicolumn{3}{@{}c@{}}{Element 1\footnotemark[1]} & \multicolumn{3}{@{}c@{}}{Element 2\footnotemark[2]} \\\cmidrule{2-4}\cmidrule{5-7}%
% Project & Energy & $\sigma_{calc}$ & $\sigma_{expt}$ & Energy & $\sigma_{calc}$ & $\sigma_{expt}$ \\
% \midrule
% Element 3  & 990 A & 1168 & $1547\pm12$ & 780 A & 1166 & $1239\pm100$\\
% Element 4  & 500 A & 961  & $922\pm10$  & 900 A & 1268 & $1092\pm40$\\
% \botrule
% \end{tabular*}
% \footnotetext{Note: This is an example of table footnote. This is an example of table footnote this is an example of table footnote this is an example of~table footnote this is an example of table footnote.}
% \footnotetext[1]{Example for a first table footnote.}
% \footnotetext[2]{Example for a second table footnote.}
% \end{minipage}
% \end{center}
% \end{table}

% In case of double column layout, tables which do not fit in single column width should be set to full text width. For this, you need to use \verb+\begin{table*}+ \verb+...+ \verb+\end{table*}+ instead of \verb+\begin{table}+ \verb+...+ \verb+\end{table}+ environment. Lengthy tables which do not fit in textwidth should be set as rotated table. For this, you need to use \verb+\begin{sidewaystable}+ \verb+...+ \verb+\end{sidewaystable}+ instead of \verb+\begin{table*}+ \verb+...+ \verb+\end{table*}+ environment. This environment puts tables rotated to single column width. For tables rotated to double column width, use \verb+\begin{sidewaystable*}+ \verb+...+ \verb+\end{sidewaystable*}+.

% \begin{sidewaystable}
% \sidewaystablefn%
% \begin{center}
% \begin{minipage}{\textheight}
% \caption{Tables which are too long to fit, should be written using the ``sidewaystable'' environment as shown here}\label{tab3}
% \begin{tabular*}{\textheight}{@{\extracolsep{\fill}}lcccccc@{\extracolsep{\fill}}}
% \toprule%
% & \multicolumn{3}{@{}c@{}}{Element 1\footnotemark[1]}& \multicolumn{3}{@{}c@{}}{Element\footnotemark[2]} \\\cmidrule{2-4}\cmidrule{5-7}%
% Projectile & Energy	& $\sigma_{calc}$ & $\sigma_{expt}$ & Energy & $\sigma_{calc}$ & $\sigma_{expt}$ \\
% \midrule
% Element 3 & 990 A & 1168 & $1547\pm12$ & 780 A & 1166 & $1239\pm100$ \\
% Element 4 & 500 A & 961  & $922\pm10$  & 900 A & 1268 & $1092\pm40$ \\
% Element 5 & 990 A & 1168 & $1547\pm12$ & 780 A & 1166 & $1239\pm100$ \\
% Element 6 & 500 A & 961  & $922\pm10$  & 900 A & 1268 & $1092\pm40$ \\
% \botrule
% \end{tabular*}
% \footnotetext{Note: This is an example of table footnote this is an example of table footnote this is an example of table footnote this is an example of~table footnote this is an example of table footnote.}
% \footnotetext[1]{This is an example of table footnote.}
% \end{minipage}
% \end{center}
% \end{sidewaystable}

% \section{Figures}\label{sec6}

% As per the \LaTeX\ standards you need to use eps images for \LaTeX\ compilation and \verb+pdf/jpg/png+ images for \verb+PDFLaTeX+ compilation. This is one of the major difference between \LaTeX\ and \verb+PDFLaTeX+. Each image should be from a single input .eps/vector image file. Avoid using subfigures. The command for inserting images for \LaTeX\ and \verb+PDFLaTeX+ can be generalized. The package used to insert images in \verb+LaTeX/PDFLaTeX+ is the graphicx package. Figures can be inserted via the normal figure environment as shown in the below example:

% %%=============================================%%
% %% For presentation purpose, we have included  %%
% %% \bigskip command. please ignore this.       %%
% %%=============================================%%
% \bigskip
% \begin{verbatim}
% \begin{figure}[<placement-specifier>]
% \centering
% \includegraphics{<eps-file>}
% \caption{<figure-caption>}\label{<figure-label>}
% \end{figure}
% \end{verbatim}
% \bigskip
% %%=============================================%%
% %% For presentation purpose, we have included  %%
% %% \bigskip command. please ignore this.       %%
% %%=============================================%%

% \begin{figure}[h]%
% \centering
% \includegraphics[width=0.9\textwidth]{fig.eps}
% \caption{This is a widefig. This is an example of long caption this is an example of long caption  this is an example of long caption this is an example of long caption}\label{fig1}
% \end{figure}

% In case of double column layout, the above format puts figure captions/images to single column width. To get spanned images, we need to provide \verb+\begin{figure*}+ \verb+...+ \verb+\end{figure*}+.

% For sample purpose, we have included the width of images in the optional argument of \verb+\includegraphics+ tag. Please ignore this. 

% \section{Algorithms, Program codes and Listings}\label{sec7}

% Packages \verb+algorithm+, \verb+algorithmicx+ and \verb+algpseudocode+ are used for setting algorithms in \LaTeX\ using the format:

% %%=============================================%%
% %% For presentation purpose, we have included  %%
% %% \bigskip command. please ignore this.       %%
% %%=============================================%%
% \bigskip
% \begin{verbatim}
% \begin{algorithm}
% \caption{<alg-caption>}\label{<alg-label>}
% \begin{algorithmic}[1]
% . . .
% \end{algorithmic}
% \end{algorithm}
% \end{verbatim}
% \bigskip
% %%=============================================%%
% %% For presentation purpose, we have included  %%
% %% \bigskip command. please ignore this.       %%
% %%=============================================%%

% You may refer above listed package documentations for more details before setting \verb+algorithm+ environment. For program codes, the ``program'' package is required and the command to be used is \verb+\begin{program}+ \verb+...+ \verb+\end{program}+. A fast exponentiation procedure:

% \begin{program}
% \BEGIN \\ %
%   \FOR i:=1 \TO 10 \STEP 1 \DO
%      |expt|(2,i); \\ |newline|() \OD %
% \rcomment{Comments will be set flush to the right margin}
% \WHERE
% \PROC |expt|(x,n) \BODY
%           z:=1;
%           \DO \IF n=0 \THEN \EXIT \FI;
%              \DO \IF |odd|(n) \THEN \EXIT \FI;
% \COMMENT{This is a comment statement};
%                 n:=n/2; x:=x*x \OD;
%              \{ n>0 \};
%              n:=n-1; z:=z*x \OD;
%           |print|(z) \ENDPROC
% \END
% \end{program}


% \begin{algorithm}
% \caption{Calculate $y = x^n$}\label{algo1}
% \begin{algorithmic}[1]
% \Require $n \geq 0 \vee x \neq 0$
% \Ensure $y = x^n$ 
% \State $y \Leftarrow 1$
% \If{$n < 0$}\label{algln2}
%         \State $X \Leftarrow 1 / x$
%         \State $N \Leftarrow -n$
% \Else
%         \State $X \Leftarrow x$
%         \State $N \Leftarrow n$
% \EndIf
% \While{$N \neq 0$}
%         \If{$N$ is even}
%             \State $X \Leftarrow X \times X$
%             \State $N \Leftarrow N / 2$
%         \Else[$N$ is odd]
%             \State $y \Leftarrow y \times X$
%             \State $N \Leftarrow N - 1$
%         \EndIf
% \EndWhile
% \end{algorithmic}
% \end{algorithm}
% \bigskip
% %%=============================================%%
% %% For presentation purpose, we have included  %%
% %% \bigskip command. please ignore this.       %%
% %%=============================================%%

% Similarly, for \verb+listings+, use the \verb+listings+ package. \verb+\begin{lstlisting}+ \verb+...+ \verb+\end{lstlisting}+ is used to set environments similar to \verb+verbatim+ environment. Refer to the \verb+lstlisting+ package documentation for more details.

% %%=============================================%%
% %% For presentation purpose, we have included  %%
% %% \bigskip command. please ignore this.       %%
% %%=============================================%%
% \bigskip
% \begin{minipage}{\hsize}%
% \lstset{frame=single,framexleftmargin=-1pt,framexrightmargin=-17pt,framesep=12pt,linewidth=0.98\textwidth,language=pascal}% Set your language (you can change the language for each code-block optionally)
% %%% Start your code-block
% \begin{lstlisting}
% for i:=maxint to 0 do
% begin
% { do nothing }
% end;
% Write('Case insensitive ');
% Write('Pascal keywords.');
% \end{lstlisting}
% \end{minipage}

% \section{Cross referencing}\label{sec8}

% Environments such as figure, table, equation and align can have a label
% declared via the \verb+\label{#label}+ command. For figures and table
% environments use the \verb+\label{}+ command inside or just
% below the \verb+\caption{}+ command. You can then use the
% \verb+\ref{#label}+ command to cross-reference them. As an example, consider
% the label declared for Figure~\ref{fig1} which is
% \verb+\label{fig1}+. To cross-reference it, use the command 
% \verb+Figure \ref{fig1}+, for which it comes up as
% ``Figure~\ref{fig1}''. 

% To reference line numbers in an algorithm, consider the label declared for the line number 2 of Algorithm~\ref{algo1} is \verb+\label{algln2}+. To cross-reference it, use the command \verb+\ref{algln2}+ for which it comes up as line~\ref{algln2} of Algorithm~\ref{algo1}.

% \subsection{Details on reference citations}\label{subsec7}

% Standard \LaTeX\ permits only numerical citations. To support both numerical and author-year citations this template uses \verb+natbib+ \LaTeX\ package. For style guidance please refer to the template user manual.

% Here is an example for \verb+\cite{...}+: \cite{bib1}. Another example for \verb+\citep{...}+: \citep{bib2}. For author-year citation mode, \verb+\cite{...}+ prints Jones et al. (1990) and \verb+\citep{...}+ prints (Jones et al., 1990).

% All cited bib entries are printed at the end of this article: \cite{bib3}, \cite{bib4}, \cite{bib5}, \cite{bib6}, \cite{bib7}, \cite{bib8}, \cite{bib9}, \cite{bib10}, \cite{bib11} and \cite{bib12}.

% \section{Examples for theorem like environments}\label{sec10}

% For theorem like environments, we require \verb+amsthm+ package. There are three types of predefined theorem styles exists---\verb+thmstyleone+, \verb+thmstyletwo+ and \verb+thmstylethree+ 

% %%=============================================%%
% %% For presentation purpose, we have included  %%
% %% \bigskip command. please ignore this.       %%
% %%=============================================%%
% \bigskip
% \begin{tabular}{|l|p{19pc}|}
% \hline
% \verb+thmstyleone+ & Numbered, theorem head in bold font and theorem text in italic style \\\hline
% \verb+thmstyletwo+ & Numbered, theorem head in roman font and theorem text in italic style \\\hline
% \verb+thmstylethree+ & Numbered, theorem head in bold font and theorem text in roman style \\\hline
% \end{tabular}
% \bigskip
% %%=============================================%%
% %% For presentation purpose, we have included  %%
% %% \bigskip command. please ignore this.       %%
% %%=============================================%%

% For mathematics journals, theorem styles can be included as shown in the following examples:

% \begin{theorem}[Theorem subhead]\label{thm1}
% Example theorem text. Example theorem text. Example theorem text. Example theorem text. Example theorem text. 
% Example theorem text. Example theorem text. Example theorem text. Example theorem text. Example theorem text. 
% Example theorem text. 
% \end{theorem}

% Sample body text. Sample body text. Sample body text. Sample body text. Sample body text. Sample body text. Sample body text. Sample body text.

% \begin{proposition}
% Example proposition text. Example proposition text. Example proposition text. Example proposition text. Example proposition text. 
% Example proposition text. Example proposition text. Example proposition text. Example proposition text. Example proposition text. 
% \end{proposition}

% Sample body text. Sample body text. Sample body text. Sample body text. Sample body text. Sample body text. Sample body text. Sample body text.

% \begin{example}
% Phasellus adipiscing semper elit. Proin fermentum massa
% ac quam. Sed diam turpis, molestie vitae, placerat a, molestie nec, leo. Maecenas lacinia. Nam ipsum ligula, eleifend
% at, accumsan nec, suscipit a, ipsum. Morbi blandit ligula feugiat magna. Nunc eleifend consequat lorem. 
% \end{example}

% Sample body text. Sample body text. Sample body text. Sample body text. Sample body text. Sample body text. Sample body text. Sample body text.

% \begin{remark}
% Phasellus adipiscing semper elit. Proin fermentum massa
% ac quam. Sed diam turpis, molestie vitae, placerat a, molestie nec, leo. Maecenas lacinia. Nam ipsum ligula, eleifend
% at, accumsan nec, suscipit a, ipsum. Morbi blandit ligula feugiat magna. Nunc eleifend consequat lorem. 
% \end{remark}

% Sample body text. Sample body text. Sample body text. Sample body text. Sample body text. Sample body text. Sample body text. Sample body text.

% \begin{definition}[Definition sub head]
% Example definition text. Example definition text. Example definition text. Example definition text. Example definition text. Example definition text. Example definition text. Example definition text. 
% \end{definition}

% Additionally a predefined ``proof'' environment is available: \verb+\begin{proof}+ \verb+...+ \verb+\end{proof}+. This prints a ``Proof'' head in italic font style and the ``body text'' in roman font style with an open square at the end of each proof environment. 

% \begin{proof}
% Example for proof text. Example for proof text. Example for proof text. Example for proof text. Example for proof text. Example for proof text. Example for proof text. Example for proof text. Example for proof text. Example for proof text. 
% \end{proof}

% Sample body text. Sample body text. Sample body text. Sample body text. Sample body text. Sample body text. Sample body text. Sample body text.

% \begin{proof}[Proof of Theorem~{\upshape\ref{thm1}}]
% Example for proof text. Example for proof text. Example for proof text. Example for proof text. Example for proof text. Example for proof text. Example for proof text. Example for proof text. Example for proof text. Example for proof text. 
% \end{proof}

% \noindent
% For a quote environment, use \verb+\begin{quote}...\end{quote}+
% \begin{quote}
% Quoted text example. Aliquam porttitor quam a lacus. Praesent vel arcu ut tortor cursus volutpat. In vitae pede quis diam bibendum placerat. Fusce elementum
% convallis neque. Sed dolor orci, scelerisque ac, dapibus nec, ultricies ut, mi. Duis nec dui quis leo sagittis commodo.
% \end{quote}

% Sample body text. Sample body text. Sample body text. Sample body text. Sample body text (refer Figure~\ref{fig1}). Sample body text. Sample body text. Sample body text (refer Table~\ref{tab3}). 

% \section{Methods}\label{sec11}

% Topical subheadings are allowed. Authors must ensure that their Methods section includes adequate experimental and characterization data necessary for others in the field to reproduce their work. Authors are encouraged to include RIIDs where appropriate. 

% \textbf{Ethical approval declarations} (only required where applicable) Any article reporting experiment/s carried out on (i)~live vertebrate (or higher invertebrates), (ii)~humans or (iii)~human samples must include an unambiguous statement within the methods section that meets the following requirements: 

% \begin{enumerate}[1.]
% \item Approval: a statement which confirms that all experimental protocols were approved by a named institutional and/or licensing committee. Please identify the approving body in the methods section

% \item Accordance: a statement explicitly saying that the methods were carried out in accordance with the relevant guidelines and regulations

% \item Informed consent (for experiments involving humans or human tissue samples): include a statement confirming that informed consent was obtained from all participants and/or their legal guardian/s
% \end{enumerate}

% If your manuscript includes potentially identifying patient/participant information, or if it describes human transplantation research, or if it reports results of a clinical trial then  additional information will be required. Please visit (\url{https://www.nature.com/nature-research/editorial-policies}) for Nature Portfolio journals, (\url{https://www.springer.com/gp/authors-editors/journal-author/journal-author-helpdesk/publishing-ethics/14214}) for Springer Nature journals, or (\url{https://www.biomedcentral.com/getpublished/editorial-policies\#ethics+and+consent}) for BMC.

% \section{Discussion}\label{sec12}

% Discussions should be brief and focused. In some disciplines use of Discussion or `Conclusion' is interchangeable. It is not mandatory to use both. Some journals prefer a section `Results and Discussion' followed by a section `Conclusion'. Please refer to Journal-level guidance for any specific requirements. 

% \section{Conclusion}\label{sec13}

% Conclusions may be used to restate your hypothesis or research question, restate your major findings, explain the relevance and the added value of your work, highlight any limitations of your study, describe future directions for research and recommendations. 

% In some disciplines use of Discussion or 'Conclusion' is interchangeable. It is not mandatory to use both. Please refer to Journal-level guidance for any specific requirements. 

\backmatter

% \bmhead{Supplementary information}

% If your article has accompanying supplementary file/s please state so here. 

% Authors reporting data from electrophoretic gels and blots should supply the full unprocessed scans for key as part of their Supplementary information. This may be requested by the editorial team/s if it is missing.

% Please refer to Journal-level guidance for any specific requirements.

\section*{Acknowledgments}
We thank Hugo Latapie of Cisco Research for his insightful comments on an early draft of this paper. Hengyue Liang, Le Peng, and Ju Sun are partially supported by NSF CMMI 2038403 and Cisco Research under the awards SOW 1043496 and 1085646 PO USA000EP390223. Ying Cui is partially supported by NSF CCF 2153352. The authors acknowledge the Minnesota Supercomputing Institute (MSI) at the University of Minnesota for providing resources
that contributed to the research results reported within this paper.

% \section*{Declarations}

% Some journals require declarations to be submitted in a standardized format. Please check the Instructions for Authors of the journal to which you are submitting to see if you need to complete this section. If yes, your manuscript must contain the following sections under the heading `Declarations':

% \begin{itemize}
% \item Funding
% \item Conflict of interest/Competing interests (check journal-specific guidelines for which heading to use)
% \item Ethics approval 
% \item Consent to participate
% \item Consent for publication
% \item Availability of data and materials
% \item Code availability 
% \item Authors' contributions
% \end{itemize}

% \noindent
% If any of the sections are not relevant to your manuscript, please include the heading and write `Not applicable' for that section. 

%%===================================================%%
%% For presentation purpose, we have included        %%
%% \bigskip command. please ignore this.             %%
%%===================================================%%
% \bigskip
% \begin{flushleft}%
% Editorial Policies for:

% \bigskip\noindent
% Springer journals and proceedings: \url{https://www.springer.com/gp/editorial-policies}

% \bigskip\noindent
% Nature Portfolio journals: \url{https://www.nature.com/nature-research/editorial-policies}

% \bigskip\noindent
% \textit{Scientific Reports}: \url{https://www.nature.com/srep/journal-policies/editorial-policies}

% \bigskip\noindent
% BMC journals: \url{https://www.biomedcentral.com/getpublished/editorial-policies}
% \end{flushleft}

% \begin{appendices}

% \section{Appendix} 

\appendix 

\section{Appendix} 

\subsection{Projection onto the intersection of a norm ball and box constraints}
\label{Sec:app_projection}
APGD for solving max-loss form (\ref{eq:robust_loss}) with $\ell_p$ distances entails solving Euclidean projection subproblems of the form: 
\begin{align}
\begin{split}
    & \min_{\mb x' \in \RJU^n} \;  \norm{\mb z - \mb x'}_2^2 \\
    \st\; & \norm{\mb x - \mb x'}_p \le \eps, \quad \mb x' \in [0, 1]^n,
\end{split}
\end{align}
where $\mb z = \mb x + \mb w$ is a one-step update of $\mb x$ toward direction $\mb w$. After a simple reparametrization, we have 
\begin{align}  
\label{eq:app_proj_infty}
\begin{split}
    & \min_{\mb \delta \in \RJU^n} \; \norm{\mb w - \mb \delta}_2^2 \\
    \st \; & \norm{\mb \delta}_p \le \eps, \quad \mb x + \mb \delta \in [0, 1]^n.
\end{split}
\end{align}

We focus on $p = 1, 2, \infty$ which are popular in the AR literature. In early works, a ``lazy'' projection scheme---sequentially projecting onto the $\ell_p$ ball and then to the $[0, 1]^n$ box, is used. \cite{croce2021mind} has recently identified the detrimental effect of lazy projection on the performance for $p=1$, and has derived a closed form solution. Here, we prove the correctness of the sequential projection for $p = \infty$ (\cref{thm:app_proj_inf_lemma}), and discuss problems regarding $p = 2$ (\cref{thm:app_proj_l2_lemma}). 

For $p = \infty$, obviously we only need to consider the individual coordinates. 
\begin{lemma} \label{thm:app_proj_inf_lemma}
Assume $x \in [0, 1]$. The unique solution for the strongly convex problem
\begin{align}
\begin{split}
    & \min_{\delta \in \RJU} \; \paren{w - \delta}^2\\
    \st\; & \abs{\delta} \le \eps, \quad x + \delta \in [0, 1]
\end{split}
\end{align}
is given by
\begin{multline}  \label{eq:app_proj_inf_formula}
    \mc P_{\infty, \mathrm{box}} = \\
        \begin{cases} 
            w,\quad w \in [\max(-x, -\eps), \min (1-x, \eps)]\\
        \max(-x, -\eps), \quad w \le \max(-x, -\eps) \\
        \min (1-x, \eps), \quad w \ge \min (1-x, \eps) 
    \end{cases}, 
\end{multline} 
which agrees with the sequential projectors $\mc P_{\infty} \mc P_{\mathrm{box}}$ and $\mc P_{\mathrm{box}} \mc P_{\infty}$.  
\end{lemma} 
One can easily derive the one-step projection formula \cref{eq:app_proj_inf_formula} once the two box constraints can be combined into one: 
\begin{align} \label{eq:app_proj_inf_equiv_constr}
    \max(-\eps, -x) \le \delta \le \min(\eps, 1-x). 
\end{align} 
To show the equivalence to $\mc P_{\infty} \mc P_{\mathrm{box}}$ and $\mc P_{\mathrm{box}} \mc P_{\infty}$, we could write down all projectors analytically and directly verify the claimed equivalence. But that tends to be cumbersome. Here, we invoke an elegant result due to \cite{yu2013decomposing}. For this, we need to quickly set up the notation. For any function $f: \RJU^n \to \RJU \cup \setJu{+ \infty}$, its proximal mapping $\mathrm{Prox}_f$ is defined as 
\begin{align} 
    \mathrm{Prox}_f(\mb y) = \argmin_{\mb z \in \RJU^n} \frac{1}{2} \norm{\mb y - \mb z}_2^2 + f(\mb z).
\end{align} 
When $f$ is the indicator function $\imath_C$ for a set $C$ defined as 
\begin{align}
    \imath_C(\mb z) = 
    \begin{cases} 
        0   &   \mb z \in C \\
            \infty &   \text{otherwise}   
         \end{cases},
\end{align}
then $\mathrm{Prox}_f(\mb y)$ is the Euclidean projector $\mc P_C(\mb y)$. For two closed proper convex functions $f$ and $g$, the paper \cite{yu2013decomposing} has studied conditions for $\mathrm{Prox}_{f+g} = \mathrm{Prox}_{f} \circ \mathrm{Prox}_{g}$. If $f$ and $g$ are two set indicator functions, this exactly asks when the sequential projector is equivalent to the true projector. 
\begin{theorem}[adapted from Theorem 2 of \cite{yu2013decomposing}]
  If $f = \imath_C$ for a closed convex set $C \subset \RJU$, $\mathrm{Prox}_f \circ \mathrm{Prox}_g = \mathrm{Prox}_{f+g}$ for all closed proper convex functions $g: \RJU \to \RJU \cup \setJu{\pm \infty}$. 
\end{theorem} 
The equivalence of projectors we claim in \cref{thm:app_proj_inf_lemma} follows by setting $f = \imath_{\infty}$ and $g = \imath_{\mathrm{box}}$, and vise versa. 

For $p = 2$, sequential projectors are not equivalent to the true projector in general, although empirically we observe that $\mc P_{2} \mc P_{\mathrm{box}}$ is a much better approximation than $\mc P_{\mathrm{box}} \mc P_{2}$. The former is used in the current APGD algorithm of \texttt{AutoAttack}. 
\begin{lemma} \label{thm:app_proj_l2_lemma}
    Assume $\mb x \in [0, 1]^n$. When $p = 2$, the projector for (\ref{eq:app_proj_infty}) $\mc P_{2, \mathrm{box}}$ does not agree with the sequential projectors $\mc P_{2} \mc P_{\mathrm{box}}$ and $\mc P_{\mathrm{box}} \mc P_{2}$ in general. However, both $\mc P_{2} \mc P_{\mathrm{box}}$ and $\mc P_{\mathrm{box}} \mc P_{2}$ always find feasible points for the projection problem. 
\end{lemma} 
\begin{proof} 
\begin{figure}[!htbp]
    \centering 
    \includegraphics[width=0.6\linewidth]{Figures/L2_box.png}
    \caption{Illustration of the problem with the sequential projectors when $p = 2$. In general, neither of the sequential projectors produces the right projection. }
    \label{fig:l2_box_projection}
\end{figure}
For the non-equivalence, we present a couple of counter-examples in \cref{fig:l2_box_projection}. Note that the point $\mb z$ is inside the normal cone of the bottom right corner point of the intersection: $\setJu{\mb \delta \in \RJU^2: \norm{\mb \delta}_2 \le \eps} \cap \setJu{\mb \delta \in \RJU^2: \mb x + \mb \delta \in [0, 1]^2}$. 

For the feasibility claim, note that for any $\mb y \in \RJU^n$
\begin{align} 
    \mc P_2\paren{\mb y} = 
    \begin{cases}
        \eps \frac{\mb y}{\norm{\mb y}_2} &  \norm{\mb y}_2 \ge \eps \\
        \mb y   &  \text{otherwise}
    \end{cases}
\end{align} 
and for any $y \in \RJU$, 
\begin{align} 
    \mc P_{\mathrm{box}}\paren{y} = 
    \begin{cases}
        1-x &  y \ge 1-x \\
        y  &   -x < y < 1-x \\ 
        -x  & y \le -x 
    \end{cases}
\end{align} 
and $\mc P_{\mathrm{box}}(\mb y)$ acts on any $\mb y \in \RJU^n$ element-wise. For any $\mb y$ inside the $\ell_2$ ball, 
\begin{multline} 
    \norm{\mc P_{\mathrm{box}}\paren{\mb y}}_2 
    = \norm{\mc P_{\mathrm{box}}\paren{\mb y} - \mc P_{\mathrm{box}}\paren{\mb 0}}_2 \\
     \le \norm{\mb y - \mb 0}_2 = \norm{\mb y}_2 \le \eps
\end{multline} 
due to the contraction property of projecting onto convex sets. Therefore, $\mc P_{\mathrm{box}} \mc P_{2}(\mb y)$ is feasible for any $\mb y \in \RJU^n$. Now for any $\mb y$ inside the box: 
\begin{itemize} 
   \item if $\norm{\mb y}_2 < \eps$, $\mc P_2(\mb y) = \mb y$ and $\mc P_2(\mb y)$ remains in the box; 
   \item if $\norm{\mb y}_2 \ge \eps$, $\mc P_2(\mb y) = \eps \frac{\mb y}{\norm{\mb y}_2}$. Since $\eps/\norm{\mb y}_2 \in [0, 1]$, $\mb P_2(\mb y)$ shrinks each component of $\mb y$, but retains their original signs. Thus, $\mb P_2(\mb y)$ remains in the box if $\mb y$ is in the box. 
\end{itemize} 
We conclude that $\mc P_{2} \mc P_{\mathrm{box}}(\mb y)$ is feasible for any $\mb y$, completing the proof. 
\end{proof}


% Let $d_C(\mb x)$ denote the distance function to a closed convex set $C$: 
% \begin{align} 
%     d_C(\mb x) \doteq \inf_{\mb z \in C} \norm{\mb z - \mb x}_2. 
% \end{align} 
% For the set indicator function, 
% \begin{align} 
%     \partial \imath_C(\mb x) = N_C(\mb x) = \ol{\bigcup_{\lambda \ge 0} \lambda \partial d_C(\mb x)} \quad \forall\; \mb x \in C 
% \end{align} 
% where $\ol{\cdot}$ means set closure, and $N_C$ is the normal cone. Now we are ready to prove \cref{thm:app_proj_inf_lemma}. 


% When $p=2$, the projection problem in \cref{app:projection:opt} also has a closed-form solution, which turns out to be the projection onto the $2-$norm ball and the box constraints sequentially. That is, 
% \[
% \text{Proj}_{2,\text{box}} (\delta)= \text{Proj}_{\text{box}} ( \text{Proj}_{2} (\delta)),
% \]
% where $\text{Proj}_{\text{box}}$ and $\text{Proj}_{2} (\delta)$ represent the projection onto the box constraints $\delta + x \in [0,1]^d$ and $\|\delta\|_2\leq \varepsilon$, respectively. This is actually due to the result in \cite[Theorem 1]{yu2013decomposing}. To show the above equation, it suffices to check the condition in the cited results that
% \begin{equation}\label{eq:inclusion}
% \partial {\bf 1}_{2}(\delta) \subseteq \partial {\bf 1}_{2}(\text{Proj}_{\text{box}} (\delta)), 
% \end{equation}
% where ${\bf 1}_{\text{box}}(z) := \left\{\begin{array}{ll}
% 0 & \mbox{if $z + x \in [0,1]^d$}\\ +\infty & \mbox{otherwise}
% \end{array}\RJUight.$
% and $\partial f$ denotes the subdifferential of a convex function $f$.
% In fact, we have
% \[
% \text{Proj}_{2} (\delta) =  \left\{\begin{array}{ll}
% \displaystyle\frac{\varepsilon}{\|\delta\|_2} \delta & \mbox{if $\|\delta\|_2> \varepsilon$}\\[0.15in] \delta & \mbox{otherwise}
% \end{array}\RJUight.
% \]
% and 
% \[
% \partial {\bf 1}_{\text{box}}(\delta_i) = \left\{\begin{array}{ll}
% %(-\infty, 0] & \mbox{if $z_i = -x_i$} \\
% (-\infty,0] & \mbox{if $z_i = -x_i$} \\[0.1in]
% [0, \infty) & \mbox{if $z_i = 1-x_i$} \\[0.1in]
% \{0\} & \mbox{if $-x_i < z_i < 1-x_i$} \\[0.1in]
% \emptyset & \mbox{otherwise}.
% \end{array}\RJUight.
% \]
% Notice that the absolute value of each component of $\delta$ after the projection onto the $2-$norm ball $\text{Proj}_{2} (\delta)$ cannot be larger than its original counterpart. Therefore, for those components $i$ such that $\delta_i = -x_i$, we have $-x_i \leq \left(\text{Proj}_{2} (\delta)\RJUight)_i < 1-x_i$; similarly for those components $i$ such that $\delta_i = 1-x_i$, we have $-x_i < \left(\text{Proj}_{2} (\delta)\RJUight)_i \leq  1-x_i$. The inclusion \RJUef{eq:inclusion} can thus be verified easily.

% For the case $p=1$, the formula to compute \cref{app:projection:opt} has been shown in \cite{croce2021mind}.

\subsection{Sketch of the BFGS-SQP algorithm in GRANSO}
\label{Sec:granso_summary} 
GRANSO is among the first optimization solvers targeting general non-smooth, non-convex problems with non-smooth constraints \cite{curtis2017bfgs}.

The key algorithm of the GRANSO package is a sequential quadratic programming (SQP) that employs a quasi-Newton algorithm, Broyden-Fletcher-Goldfarb-Shanno (BFGS) \cite{wright1999numerical}, and an exact penalty function (Penalty-SQP). 

The penalty-SQP calculates the search direction from the following quadratic programming (QP) problem:
\begin{align}
\label{penalty_sqp}
    \begin{split}
        \min_{\vd \in \RJU^n, \vs \in \RJU^p} \; & \mu \paren{f\paren{\vx_k} + \nabla f \paren{\vx_k}^{\TJU} \vd} \\
        + & \ve^{\TJU}\vs + \frac{1}{2} \vd^{\TJU} \mH_k \vd \\
        \st \; c \paren{\vx_k} + & \nabla c \paren{\vx_k}^{\TJU} \vd \leq \vs, \quad \vs \geq 0.
    \end{split}
\end{align}
Here, we abuse the notation $c(\cdot)$ to be the total constraints for simplicity (i.e., representing all $c$'s and $h$'s in \cref{eq:NO_form}).
The dual of problem (\cref{penalty_sqp}) is used in the GRANSO package:
\begin{align}
\label{penalty_sqp_dual}
    \begin{split}
        \max_{\vlambda\in\RJU^p} \; & \mu f (\vx_k) + c(\vx_k)^{\TJU}\vlambda \\
        & - \frac{1}{2} \paren{ \mu \nabla f (\vx_k) + \nabla c(\vx_k) \vlambda }^{\TJU} \mH_k^{-1} \\
        & \cdot \; \paren{ \mu \nabla f (\vx_k) + \nabla c(\vx_k) \vlambda } \\
        \st & \quad 0 \leq \vlambda \leq \ve,
    \end{split}
\end{align}
which has only simple box constraints that can be easily handled by many popular QP solvers such as OSQP (ADMM-based algorithm) \cite{osqp}.
Then the primal solution $\vd$ can be recovered from the dual solution $\vlambda$ by solving (\ref{penalty_sqp_dual}):
\begin{align}\label{searching_direction}
    \vd = - \mH_k^{-1} \paren{ \mu \nabla f (\vx_k) + \nabla c(\vx_k) \vlambda }.
\end{align}

The search direction calculated at each step controls the trade-off between minimizing the objective and moving towards the feasible region. To measure how much violence the current search direction will give, a linear model of constraint violation is used:
\begin{align}\label{contr_viol}
    l(\vd;\vx_k) \defeq \norm{\max\Brac{c(\vx_k)+\nabla c(\vx_k)^{\TJU}\vd,\bm{0}}}_{1}.
\end{align}

To dynamically set the penalty parameter, a steering strategy as \cref{alg:steering} is used:

\begin{algorithm}[!tb]
\caption{\\ \text{ } \quad \quad $\brac{\vd_k,{\mu}_{\text{new}}}= \texttt{sqp\_steering}(\vx_k,\mH_k,{\mu})$}
\label{alg:steering}
\begin{algorithmic}[1]
\Require $\vx_k, \mH_k,\mu$ at current iteration 
\Require constants $c_v\in(0,1),c_{\mu} \in (0,1)$ 
\State Calculate $\vd_k$ from \cref{searching_direction} and \formulation (\ref{penalty_sqp_dual}) with $\mu_{\text{new}} = \mu$
\If{$l_{\delta}(\vd_k;\vx_k)< c_v v(\vx_k)$}
\State Calculate $\tilde{\vd_k}$ from \cref{searching_direction} and \formulation (\ref{penalty_sqp_dual}) with $\mu=0$
\While{$l_{\delta}(\vd_k;\vx_k)< c_v l_{\delta}(\tilde{\vd_k};\vx_k)$}
\State $\mu_{new} \defeq c_{\mu} \mu_{\text{new} }$
\State Calculate $\vd_k$ from \cref{searching_direction} and \formulation (\ref{penalty_sqp_dual}) with $\mu=\mu_{\text{new}}$
\EndWhile
\EndIf 
\State \Return{$\vd_k,{\mu}_{\text{new}}$}
\end{algorithmic}
\end{algorithm}

For non-smooth problems, it is usually hard to find a reliable stopping criterion, as the norm of the gradient will not decrease when approaching a minimizer. GRANSO uses an alternative stopping strategy, which is based on the idea of gradient sampling \cite{lewis2013nonsmooth} \cite{burke2020gradient}.

Define the neighboring gradient information (from the $p$ most recent iterates) as:
\begin{align}
    \begin{split}
        & \mG \defeq \brac{\nabla f(\vx_{x_{k+1-l}})\hdots \nabla f(\vx_k)} \text{,}\\
    & \mJ_i \defeq \brac{\nabla c_i(\vx_{x_{k+1-l}})\hdots \nabla c_i (\vx_k)}\text{,} \\ & \; i \in \Brac{1,\hdots ,p}
    \end{split}
\end{align}

Augment \formulation (\ref{penalty_sqp}) and its dual \formulation (\ref{penalty_sqp_dual}) in the steering strategy, we can obtain the augmented dual problem:
\begin{align}\label{QP_termination}
    \begin{split}
        \max_{\vsigma\in \RJU^l, \vlambda\in \RJU^{pl}} 
        \sum_{i=1}^p & c_i(\vx_k)\ve^{\TJU} \vlambda_i  \\
        - \frac{1}{2} \begin{bmatrix} \vsigma \\ \vlambda \end{bmatrix}^{\TJU} \begin{bmatrix} \mG,\mJ_1,\hdots,\mJ_p \end{bmatrix}^{\TJU} & \mH_k^{-1} \begin{bmatrix} \mG,\mJ_1,\hdots,\mJ_p \end{bmatrix}\begin{bmatrix} \vsigma \\ \vlambda \end{bmatrix} \\
         \st \; \bm{0} \leq \vlambda_i \leq \ve, \quad & \ve^{\TJU}\vsigma = \mu, \quad \vsigma \geq \bm{0}
    \end{split}
\end{align}


By solving \cref{QP_termination}, we can obtain $\vd_{\diamond}$:
\begin{align}\label{d_diamond}
    \begin{split}
    &    \vd_{\diamond} = \mH_k^{-1} \begin{bmatrix} \mG,\mJ_1,\hdots,\mJ_p \end{bmatrix} \begin{bmatrix} \vsigma \\ \vlambda \end{bmatrix}
    \end{split}
\end{align}
If the norm of $\vd_{\diamond}$ is sufficiently small, the current iteration can be viewed as near a small neighborhood of a stationary point.

\begin{algorithm}[!tb]
\caption{$ \\ \text{ } \quad \quad \brac{ \vx_*,f_*,\vv_* } = \texttt{bfgs\_sqp}\paren{f(\cdot),\vc(\cdot), \vx_0, \mu_0})$ }
\label{alg:bfgs_sqp}
\begin{algorithmic}[1]
\Require $f,\vc,\vx_0,\mu_0$
\Require  constants $\tau_{\diamond},\tau_{v}$
\State $\mH_0 \defeq \mI,\ \mu \defeq \mu_0$
\State $\phi(\cdot) = \mu f(\cdot) + v(\cdot)$
\State $\nabla \phi(\cdot) = \mu \nabla f(\cdot) + \sum_{i \in \mathcal{P} } \nabla c_i(\cdot) $
\State $v(\cdot) = \norm{\max \Brac{c(\cdot),0}}_{1}$
\State  $\phi_0 \defeq \phi(\vx_0;\mu),\nabla \phi_0 \defeq \nabla \phi(\vx_0;\mu),v_0 \defeq v(\vx_0)$
\For {$k = 0,1,2,\hdots$}
\State $\brac{\vd_k,\hat{\mu}} \defeq \texttt{sqp\_steering}(\vx_k,\mH_k,\bm{\mu})$
\If{$\hat{\mu}< \mu$}
\State $\mu \defeq \hat{\mu}$
\State  $\phi_k \defeq \phi(\vx_k;\mu),\nabla \phi_k  \defeq \nabla \phi(\vx_k;\mu),v_k \defeq v(\vx_k)$
\EndIf
\State{$\brac{\vx_{k+1},\phi_{k+1}, \nabla \phi_{k+1}, v_{k+1} } \defeq \texttt{Armijo\_Wolfe}\paren{\vx_k,\phi_k,\nabla \phi_k, \phi(\cdot) , \nabla \phi(\cdot) } $}
\State Get $\vd_{\diamond}$ from \cref{d_diamond} and \formulation (\ref{QP_termination})
\If{$\norm{\vd_{\diamond}}_{2}< \tau_{\diamond}$ and $v_{k+1}<\tau_v$}
\State break
\EndIf
\State BFGS update $\mH_{k+1}$
\EndFor
\State \Return{$\vx_*,f_*,\vv_*$}
\end{algorithmic}
\end{algorithm}

\subsection{Danskin's theorem and min-max optimization}
\label{sec:danskin_minmax}
In this section, we discuss the importance of computing a good solution to the inner maximization problem when applying first-order methods for AT, i.e., solving \formulation (\ref{eq:minmax_obj}). 

Consider the following minimax problem: 
\begin{equation}\label{app:danskin}
\min_{\mb \theta} g(\mb \theta) \doteq  \left[ \max_{\mb x' \in \Delta} \;  h(\mb \theta, \mb x')\right], 
\end{equation}
where we assume that the function $h$ is locally Lipschitz continuous. To apply first-order methods to solve \cref{app:danskin}, one needs to evaluate a (sub)gradient of $g$ at any given $\mb \theta$. If $h(\mb \theta, \mb x')$ is smooth in $\mb \theta$, one can invoke Danskin's theorem for such an evaluation (see, for example, \cite[Appendix A]{madry2017towards}). However, in DL applications with non-smooth activations or losses, $h(\mb \theta, \delta)$ is not differentiable in $\mb \theta$, and hence a general version of Danskin's theorem is needed.

To proceed, we first introduce a few basic concepts in nonsmooth analysis; general background can be found in~\cite{Clarke1990Optimization,BagirovEtAl2014Introduction,CuiPang2021Modern}.  For a locally Lipschitz continuous function $\varphi:\mathbb{R}^n \to \mathbb{R}$, define its Clarke directional derivative at $\bar{\mb z}\in \mathbb{R}^n$ in any direction $\mb d\in \mathbb{R}^n$ as
\[
\varphi^\circ(\bar{\mb z}; \mb d)\doteq \limsup_{t\downarrow 0, \mb z \to \bar{\mb z}} \frac{\varphi(\mb z+t \mb d) - \varphi(\mb z)}{t}
\]
We say $\varphi$ is Clarke regular at $\bar{\mb z}$ if $\varphi^\circ(\bar{\mb z};\mb d) = \varphi^\prime(\bar{\mb z};\mb d)$ for any $\mb d\in \mathbb{R}^n$, where $\varphi^\prime(\bar{\mb z};\mb d) \doteq \displaystyle\lim_{t\downarrow 0} \frac{1}{t} \paren{\varphi(\bar{\mb z} +t \mb d) - \varphi(\bar{\mb z})}$ is the usual one-sided directional derivative. The Clarke subdifferential of $\varphi$ at $\bar{\mb z}$ is defined as
\[
\partial \varphi (\bar{\mb z}) \doteq \left\{\mb v \in \mathbb{R}^n: \varphi^\circ(\bar{\mb z};\mb d) \geq \mb v^\TJU \mb d \right\}
\]
The following result has its source in \cite[Theorem 2.1]{clarke1975generalized}; see also \cite[Section 5.5]{CuiPang2021Modern}.
\begin{theorem}
Assume that $\Delta$ in \cref{app:danskin} is a compact set, and the function $h$ satisfies
\begin{enumerate}
    \item $h$ is jointly upper semicontinuous in $(\mb \theta, \mb x')$;
    \item $h$ is locally Lipschitz continuous in $\mb \theta$, and the Lipschitz constant is uniform in $\mb x' \in \Delta$;
    \item $h$ is directionally differentiable in $\mb \theta$ for all $\mb x' \in \Delta$; 
\end{enumerate}
If $h$ is Clarke regular in $\mb \theta$ for all $\mb \theta$, and $\partial_{\mb \theta} h$ is upper semicontinuous in $(\mb \theta, \mb x')$, we have that for any $\bar{\mb \theta}$ 
\begin{align} 
\partial g(\bar{\mb \theta}) = \mathrm{conv}\{\partial h(\bar{\mb \theta},\mb x'): \mb x' \in \Delta^*(\bar{\mb \theta})\}
\end{align} 
where $\mathrm{conv}(\cdot)$ denotes the convex hull of a set, and $\Delta^*(\bar{\mb \theta})$ is the set of all optimal solutions of the inner maximization problem at $\bar{\mb \theta}$.
\end{theorem}
The above theorem indicates that in order to get an element from the subdifferential set $\partial g(\bar{\mb \theta})$, we need to get at least one {\bf optimal} solution $\mb x' \in \Delta^*(\bar{\mb \theta})$. A suboptimal solution to the inner maximization problem may result in a useless direction for the algorithm to proceed. To illustrate this, let us consider a simple one-dimensional example
\[
\min_\theta  g(\theta) := \left[\, \max_{-1\leq x' \leq 1}  \;\; \max(\theta x', 0)^2\,\right]
\]
which corresponds to a one-layer neural network with one data point $(0,0)$, the ReLU activation function and squared loss. Starting at $\theta_0 = 1$, we get the first inner maximization problem $\max_{-1\leq x' \leq 1} \max(x',0)^2$. Although its global optimal solution is $x'_* = 1$, the point $x' = 0$ is a stationary point satisfying the first-order optimality condition. If the latter point is mistakenly adopted to compute an element in $\partial g(\theta^0)$, it would result in a zero direction so that the overall gradient descent algorithm cannot proceed. 

% An appendix contains supplementary information that is not an essential part of the text itself but which may be helpful in providing a more comprehensive understanding of the research problem or it is information that is too cumbersome to be included in the body of the paper.

%%=============================================%%
%% For submissions to Nature Portfolio Journals %%
%% please use the heading ``Extended Data''.   %%
%%=============================================%%

%%=============================================================%%
%% Sample for another appendix section			       %%
%%=============================================================%%

%% \section{Example of another appendix section}\label{secA2}%
%% Appendices may be used for helpful, supporting or essential material that would otherwise 
%% clutter, break up or be distracting to the text. Appendices can consist of sections, figures, 
%% tables and equations etc.

% \end{appendices}

%%===========================================================================================%%
%% If you are submitting to one of the Nature Portfolio journals, using the eJP submission   %%
%% system, please include the references within the manuscript file itself. You may do this  %%
%% by copying the reference list from your .bbl file, paste it into the main manuscript .tex %%
%% file, and delete the associated \verb+\bibliography+ commands.                            %%
%%===========================================================================================%%
\bibliographystyle{IEEEtran}
% \bibliography{reference, egbib, theory}% common bib file
\bibliography{reference}% common bib file
%% if required, the content of .bbl file can be included here once bbl is generated
%%\input sn-article.bbl

%% Default %%
%%\input sn-sample-bib.tex%

\end{document}
