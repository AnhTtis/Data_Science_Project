\section{Comparison of \pygranso\, and \texttt{Knitro}} 
\label{sec:pygranso_knitro}

In this section, we present a comparison of \granso and \knitro on solving \cref{eq:robust_loss} with margin loss. The maximum number of iterations of both solvers is set to be 200. For \pygranso, we use 20 as limited memory size to enable limited-memory BFGS updating. For \knitro, the auto-differentiation of \texttt{PyTorch} is incorporated to make it have access to both values and gradients of the objective function and constraints through callback functions. Both solvers are initialized by the original image with a small random noise. All other parameters of the two solvers are as default.

We randomly choose 200 images from the validation set of ImageNet100 or testing set of CIFAR10 and then attack those images with \granso and \knitro. The results are shown in \cref{tab: granso_vs_knitro}. Compared to \knitro, \granso usually gets solutions with lower violation while having similar quality. And the attack success rate of \granso is higher than \knitro in most cases.

\input{Sections/section_elements/Table-Granso-vs-Knitro.tex}