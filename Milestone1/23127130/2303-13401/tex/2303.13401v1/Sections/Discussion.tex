\section{Discussion}
\label{sec: summary} 
In this paper, we introduce a new algorithmic framework, \pygranso~\textbf{w}ith \textbf{C}onstraint-\textbf{F}olding (PWCF), to solve two constrained optimization formulations of the robustness evaluation (RE) problems: max-loss form and min-radius form. PWCF can handle general distance metrics (almost everywhere differentiable) and achieve reliable solutions for these two formulations, which are beyond the reach of existing numerical methods. We remark that PWCF is not intended to beat the performance of the existing SOTA algorithms in these two formulations with the limited $\ell_1$, $\ell_2$ and $\ell_\infty$ distances, nor to improve the adversarial training pipeline. We view PWCF as a reliable and general numerical framework for the current RE packages, and possibly for other future emerging problems in term of constraint optimization with deep neural networks (DNNs).

In addition, we show that using different combinations of losses $\ell$, distance metrics $d$ and solvers to solve max-loss form and min-radius form can lead to different sparsity patterns in the solutions found. Having provided our explanations on why the pattern differences can happen, we also discuss its implications for the research on adversarial robustness: 
\begin{enumerate}
    \item The current practice of RE based on solving max-loss form is insufficient and misleading.
    \item Finding the sample-wise robustness radius by solving min-radius form can be a better robustness metric.
    \item Achieving generalizable robustness by adversarial training (AT) may be intrinsically difficult.
\end{enumerate}

% As for future works, \textcolor{red}{Ju promised to add something here after sufficient polishing the main content of the paper.}
