\begin{figure}[!tb]
\centering
\begingroup 
\setlength{\tabcolsep}{1pt}
\renewcommand{\arraystretch}{0.8}
\begin{tabular}{cc}
\centering
\includegraphics[width=0.24\textwidth]{Figures/Sec3-Ablations/with-without-folding/L2-without-folding.png}
&\includegraphics[width=0.24\textwidth]{Figures/Sec3-Ablations/with-without-folding/L2-with-folding.png}
\\
\small{\textbf{(a)} $n$ box constraints} 
&\small{\textbf{(b)} folded constraints} 
\end{tabular}
\endgroup 
\caption{Examples of \pygranso~optimization trajectories to solve max-loss form with $\ell_{2}$ distance $(\eps=0.5)$ and margin loss $\ell$ (a clipped version described in \cref{subsec: loss clip}) on a CIFAR-10 image. \textbf{(a)} the $\mb x' \in [0, 1]^n$ constraint is in the original form of $n$ linear box constraints; \textbf{(b)} $\mb x' \in [0, 1]^n$ constraint is folded with the $\ell_2$ function into a single non-smooth constraint function. The x-axes denote the iteration number. Here, an acceptable solution is found when the objective value reaches $0.01$ and the constraint violation reaches below the tolerance level ($10^{-2}$). We can conclude from (a) and (b) that it takes significantly less time and number of iterations with constraint-folding than without.} 
\label{Fig: Ablation on constraint folding}
\end{figure}