\begin{figure}[!tb]
\centering
\begingroup 
\setlength{\tabcolsep}{1pt}
\renewcommand{\arraystretch}{0.8}
\begin{tabular}{cc}
\centering
\includegraphics[width=0.24\textwidth]{Figures/Sec3-Ablations/with-without-LinfReformulation/Linf-orig.png}
&\includegraphics[width=0.24\textwidth]{Figures/Sec3-Ablations/with-without-LinfReformulation/Linf-reform.png}
\\
\small{\textbf{(a)} original $\ell_\infty$} 
&\small{\textbf{(b)} reformulated $\ell_\infty$} 
\end{tabular}
\endgroup 
\caption{Example of \pygranso~optimization trajectories to solve max-loss form (\ref{eq:robust_loss}) with the $\ell_{\infty}$ distance $(\eps=0.03)$ and margin loss $\ell$ (a clipped version described in \cref{subsec: loss clip}) on a CIFAR-10 image. The original form of constraint $\norm{\mb x' - \mb x}_\infty \leq \eps$ is used in \textbf{(a)}, while the reformulated and folded version of constraint is used in \textbf{(b)}. The x-axes denote the iteration number. Here, the optimization is terminated when the constraint violation is smaller than $10^{-2}$. After reformulating and folding the $\ell_\infty$ constraint, the optimization process in \textbf{(b)} runs much faster in terms of both time and iterations needed.} 
\label{Fig: Ablation on Linf Reform}
\end{figure}