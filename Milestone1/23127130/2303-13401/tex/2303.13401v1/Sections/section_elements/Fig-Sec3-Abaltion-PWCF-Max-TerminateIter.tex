\begin{figure*}[!tb]
\vspace{-1em}
\centering
\begingroup 
\setlength{\tabcolsep}{1pt}
\renewcommand{\arraystretch}{0.8}
\begin{tabular}{c c c c c c}
\centering
{}
&\multicolumn{2}{c}{\textbf{CIFAR-10}}
&{ }
&\multicolumn{2}{c}{\textbf{ImageNet-100}}
\\
\cline{2-3}\cline{5-6}
\vspace{-1em}
\\
{}
&\includegraphics[width=0.24\textwidth]{Figures/Sec3-Ablations/Reliability-TerminateIter/PWCF-Max-L2-Cifar.png}
&\includegraphics[width=0.24\textwidth]{Figures/Sec3-Ablations/Reliability-TerminateIter/PWCF-Max-Linf-Cifar.png}
&{ }
&\includegraphics[width=0.24\textwidth]{Figures/Sec3-Ablations/Reliability-TerminateIter/PWCF-Max-L2-ImageNet.png}
&\includegraphics[width=0.24\textwidth]{Figures/Sec3-Ablations/Reliability-TerminateIter/PWCF-Max-Linf-ImageNet.png}
\\
{}
&\small{\textbf{(a)} PWCF - $\ell_2$} 
&\small{\textbf{(b)} PWCF - $\ell_\infty$}
&{ }
&\small{\textbf{(c)} PWCF - $\ell_2$} 
&\small{\textbf{(d)} PWCF - $\ell_\infty$}
\\
&\includegraphics[width=0.24\textwidth]{Figures/Sec3-Ablations/Reliability-TerminateIter/PWCF-Min-L2-Cifar.png}
&\includegraphics[width=0.24\textwidth]{Figures/Sec3-Ablations/Reliability-TerminateIter/PWCF-Min-Linf-Cifar.png}
&{ }
&\includegraphics[width=0.24\textwidth]{Figures/Sec3-Ablations/Reliability-TerminateIter/PWCF-Min-L2-ImageNet.png}
&\includegraphics[width=0.24\textwidth]{Figures/Sec3-Ablations/Reliability-TerminateIter/PWCF-Min-Linf-ImageNet.png}
\\
{}
&\small{\textbf{(e)} PWCF - $\ell_2$} 
&\small{\textbf{(f)} PWCF - $\ell_\infty$}
&{ }
&\small{\textbf{(g)} PWCF - $\ell_2$} 
&\small{\textbf{(h)} PWCF - $\ell_\infty$}
\end{tabular}
\endgroup 
\caption{Histograms of PWCF's termination iteration to solve max-loss form (\textbf{(a)}-\textbf{(d)}), and min-radius form (\textbf{(e)}-\textbf{(h)}) for $88$ images from CIFAR-10 and $100$ images from ImageNet-100. We set both the stationarity and the constraint violation tolerance (components of \pygranso's stopping criterion) as $10^{-2}$. We also set the MaxIter as $400$ for (\textbf{(a)}-\textbf{(d)}) and $4000$ for (\textbf{(e)}-\textbf{(h)}). Optimization processes terminated by reaching MaxIter indicate that the corresponding solutions do not meet the stationarity or feasibility tolerance. For these premature solutions, we can decide if further optimization is needed by assessing the final stationarity, constraint violation values, and the objective values.} 
\label{Fig: PWC-Max-Terminate-Iter}
\end{figure*}