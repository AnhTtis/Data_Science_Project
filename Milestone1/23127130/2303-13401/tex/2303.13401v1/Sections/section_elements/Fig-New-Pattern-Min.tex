\begin{figure*}[!tb]
\centering
\begingroup 
\setlength{\tabcolsep}{1pt}
\renewcommand{\arraystretch}{0.8}
\begin{tabular}{c c c c c c c c}
\centering
{}
&{$\ell_1$}
&{$\ell_2$}
&{$\ell_\infty$}
&{ }
&{$\ell_1$}
&{$\ell_2$}
&{$\ell_\infty$}
\\
{}
&\includegraphics[width=0.16\textwidth]{Figures/new_pattern_vis/FAB-L1-Margin-errVis.png}
&\includegraphics[width=0.16\textwidth]{Figures/new_pattern_vis/FAB-L2-Margin-errVis.png}
&\includegraphics[width=0.16\textwidth]{Figures/new_pattern_vis/FAB-Linf-Margin-errVis.png}
&{ }
&\includegraphics[width=0.16\textwidth]{Figures/new_pattern_vis/PyGranso-Min-L1-Margin-errVis.png}
&\includegraphics[width=0.16\textwidth]{Figures/new_pattern_vis/PyGranso-Min-L2-Margin-errVis.png}
&\includegraphics[width=0.16\textwidth]{Figures/new_pattern_vis/PyGranso-Min-Linf-Margin-errVis.png}
\\
{}
&\includegraphics[width=0.16\textwidth]{Figures/new_pattern_vis/FAB-L1-Margin-errHist.png}
&\includegraphics[width=0.16\textwidth]{Figures/new_pattern_vis/FAB-L2-Margin-errHist.png}
&\includegraphics[width=0.16\textwidth]{Figures/new_pattern_vis/FAB-Linf-Margin-errHist.png}
&{ }
&\includegraphics[width=0.16\textwidth]{Figures/new_pattern_vis/PyGranso-Min-L1-Margin-errHist.png}
&\includegraphics[width=0.16\textwidth]{Figures/new_pattern_vis/PyGranso-Min-L2-Margin-errHist.png}
&\includegraphics[width=0.16\textwidth]{Figures/new_pattern_vis/PyGranso-Min-Linf-Margin-errHist.png}
\\
\cline{2-4}\cline{6-8}
\vspace{-1em}
\\
{}
&\multicolumn{3}{c}{\textbf{FAB}}
&{ }
&\multicolumn{3}{c}{\textbf{PWCF}}
\\
\end{tabular}
\endgroup 
\caption{Visualizations of perturbation images ($\mb x' - \mb x$, top row) and the histogram of element-wise perturbation magnitude ($\abs{\mb x' - \mb x}$, bottom row) by solving min-radius form. Note that the comparison between FAB and PWCF may not be as straightforward as \cref{Fig:max pattern vis} because the radii found by solving min-radius form are likely different in scale. However, the shape of the histograms can still reveal the pattern differences.}
\label{Fig:min pattern vis}
\end{figure*}