\begin{figure}[!tb]
\centering
\begingroup 
\setlength{\tabcolsep}{1pt}
\renewcommand{\arraystretch}{0.8}
\begin{tabular}{cc}
\centering
\includegraphics[width=0.24\textwidth]{Figures/Sec3-Ablations/LossClip/CE-noClip.png}
&\includegraphics[width=0.24\textwidth]{Figures/Sec3-Ablations/LossClip/CE-Clip.png}
\\
\small{\textbf{(a)} CE} 
&\small{\textbf{(b)} CE-clip}
\\
\includegraphics[width=0.24\textwidth]{Figures/Sec3-Ablations/LossClip/Margin-noClip.png}
&\includegraphics[width=0.24\textwidth]{Figures/Sec3-Ablations/LossClip/Margin-Clip.png}
\\
\small{\textbf{(c)} margin} 
&\small{\textbf{(d)} margin-clip}
\end{tabular}
\endgroup 
\caption{PWCF optimization trajectories using cross-entropy (CE) loss, margin loss, and their clipped versions to solve the max-loss form using the $\ell_2$ metric on a CIFAR-10 image. The x-axes represent the number of iterations. For both CE and margin loss without clipping (\textbf{(a)} and \textbf{(c)}), PWCF progresses slowly toward feasibility (dashed orange curve), while with clipping (\textbf{(b)} and \textbf{(d)}), PWCF finds an optimal and feasible solution within only a few iterations.} 
\label{Fig: ablation loss clipping}
\end{figure}