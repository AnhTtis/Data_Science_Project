\begin{figure}[!tb]
\centering
\begingroup 
\setlength{\tabcolsep}{1pt}
\renewcommand{\arraystretch}{0.8}
\begin{tabular}{cc}
\centering
\includegraphics[width=0.245\textwidth]{Figures/Sec3-Ablations/Min-Linf-Rescale/Min-Linf-NoRescale.png}
&\includegraphics[width=0.245\textwidth]{Figures/Sec3-Ablations/Min-Linf-Rescale/Min-Linf-Rescale.png}
\\
\small{\textbf{(a)} $\min~t$} 
&\small{\textbf{(b)} $\min~t \cdot \sqrt{n}$} 
\end{tabular}
\endgroup 
\caption{Examples of PWCF optimization trajectories for solving min-radius form (using (\ref{eq: min reform})) with the $\ell_\infty$ metric on a CIFAR-10 image \textbf{(a)} without rescaling and \textbf{(b)} with rescaling. The x-axes are the iteration number. The objective value in \textbf{(b)} is scaled back to the original value $t$ for fair comparisons with \textbf{(a)}. In Figure \textbf{(a)}, optimization is terminated around the $25^{\text{th}}$ iteration due to line-search failure, and the final solution has a much higher (worse) objective value than \textbf{(b)}. Here, we use $10^{-8}$ for the stationarity and constraint violation tolerances to rule out the possibility that the bad solution quality of \textbf{(a)} is due to premature termination.}
% Also note in both \textbf{(a)} and \textbf{(b)}, PWCF makes the most progress within a few iterations ($<20$), then refines the objective value with minor improvements afterwards.} 
\label{Fig: Ablation on Min Linf Rescale}
\end{figure}