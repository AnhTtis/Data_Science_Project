\begin{figure*}[!tb]
\vspace{-1em}
\centering
\begingroup 
\setlength{\tabcolsep}{1pt}
\renewcommand{\arraystretch}{0.8}
\begin{tabular}{c c c c c c}
\centering
{}
&\multicolumn{2}{c}{\textbf{max-loss form}}
&{ }
&\multicolumn{2}{c}{\textbf{min-radius form}}
\\
\cline{2-3}\cline{5-6}
\vspace{-1em}
\\
{}
&\includegraphics[width=0.24\textwidth]{Figures/Sec3-Ablations/Reliability-TerminateIter/OPT-Traj/APGD-L2.png}
&\includegraphics[width=0.24\textwidth]{Figures/Sec3-Ablations/Reliability-TerminateIter/OPT-Traj/PWCF-Max-L2-2.png}
&{ }
&\includegraphics[width=0.24\textwidth]{Figures/Sec3-Ablations/Reliability-TerminateIter/OPT-Traj/FAB-L2.png}
&\includegraphics[width=0.24\textwidth]{Figures/Sec3-Ablations/Reliability-TerminateIter/OPT-Traj/PWCF-Min-L2.png}
\\
{}
&\small{\textbf{(a)} APGD} 
&\small{\textbf{(b)} PWCF - margin loss}
&{ }
&\small{\textbf{(c)} FAB} 
&\small{\textbf{(d)} PWCF}

\end{tabular}
\endgroup 
\caption{Examples of the optimization trajectories of APGD (with CE loss), FAB and PWCF for solving max-loss form and min-radius form of a CIFAR-10 image with the $\ell_2$ distance. The x-axes represent the iteration numbers. In \textbf{(a)} and \textbf{(c)}, the dashed orange lines are the default MaxIter used in \texttt{AutoAttack} and the dashed green lines are the iteration where the best feasible solutions are found. In \textbf{(b)} and \textbf{(d)}, we set the stationarity and constraint violation tolerances to be $10^{-8}$ for the termination condition of the PWCF to better visualize the optimization curve. \textbf{$T_1$} and \textbf{$T_2$} mark the iterations where both the stationarity and total violation reach $10^{-2}$ and $10^{-3}$, respectively. We can observe that the objective values only improve marginally after $T_1$, \change{indicating that $10^{-2}$ is a reasonable tolerance level for PWCF to trade for efficiency while maintaining the solution quality.}}
%As the objective and constraint violation values only improve marginally after $T_1$, we can conclude that $10^{-2}$ is a reasonable tolerance level for PWCF to achieve good solution qualities.} 
\label{Fig: Abaltion-OPT-Traj}
\end{figure*}