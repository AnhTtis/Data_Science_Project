\section{Real-World Demonstration}
\label{sec:demo}
We described the concepts behind R-LDM (Relational Local Dynamic Map) as a supporting module for a risk-based motion planning, and the Risk Maps (RM) technology representing an uncertainty-aware planner determining optimal driving trajectories and paths.

This section presents quantitative experiments for the system combination of RM with R-LDM.
A demonstration took place within the frame of the ITS European Congress 2019 at Helmond, the Netherlands. The system was run live in front of scientific and non-scientific audiences. The main focus lied hereby on the resulting HMI that has been shown and served to warn the driver. The test car and HMI is summarized in Fig. \ref{fig:vidas}. Concretely, by developing an integrated HMI, the suggestions of the systems were rendered transparent.

\subsection{Test Setup}

The used car prototype is called Carlota and was provided by the Spanish research institute Vicomtech. Its sensor setup consists of a mid-range GNSS\footnote{The product model is given on the website https://www.u-blox.com/de/
product/evk-8evk-m8.} 
and cameras. Here, the cameras are four Sekonix
\footnote{See the cameras on http://sekolab.com/products/camera.} 
cameras, which are combined with two Nvidia GPUs\footnote{See https://www.nvidia.com/en-us/geforce/graphics-cards/rtx-2080-ti.} 
and one Intel CPU. For this reason, Carlota can be seen as a low-cost test solution. 

In what follows, we analyze and evaluate the given HMI concept, which is built on three components: a risk graph (top left of Fig. \ref{fig:single_costs}), a velocity scale (Fig. \ref{fig:path_blending}, top right) and lane change recommendations (Fig. \ref{fig:path_blending}, bottom right). 
While demonstrating the system online, the velocity scale was displayed on an instrument cluster, inside the vehicle. For the lane change recommendation, LED stripes were used that were attached to the windshield and are visually conveying suggested driving directions. Additionally, both elements, together with the risk graph, have been presented on a projector to the audience. 

\begin{figure}[!t]
  \centering
    \vspace{-0.25cm}
    \resizebox{0.4674\linewidth}{!}{\import{img/results/}{vicomtech_cams3.pdf_tex}}
    \resizebox{0.4674\linewidth}{!}{\import{img/results/}{vicomtech_hmi2.pdf_tex}}
    
    \vspace*{0.13cm}
    \resizebox{0.95\linewidth}{!}{\import{img/results/}{vidas2.pdf_tex}}
    
    \vspace*{0.1cm}
    
  \caption{Top left: Outside cameras mounted on car prototype (three of four cameras shown). Top right: The HMI consists of clusters showing the recommended velocity and LED stripes signaling the path choices (left, right or straight). Bottom: Test area with projector visualizing the risk graph.}
  \label{fig:vidas}
\end{figure}

In total, the R-LDM serves to provide environment data, i.e., the paths for the risk calculation. We performed situations with three test drivers, maneuvering Carlota and two further vehicles. In each of those runs, RM supports to execute a safe lane change. The end of the road, indicated by cones, forces a lane change of the ego car. In this context, the speed ranges for the tests lied in between $\unit[20]{km/h}$ and $\unit[40]{km/h}$ with a closed public highway stretch of $\unit[1.5]{km}$ length. Our test area is a two-lane road, with both lanes in the same direction and without oncoming traffic. Data recordings from the demonstration were the basis for the evaluation. 

We now characterize in Section \ref{subsec:integration} benefits that arise from utilizing the R-LDM as a knowledge hub for support systems. Afterwards, within Section \ref{sec:results}, the results of the RM are described and discussed in regard of analyzing the lane change situations.  

\subsection{Sensor Data Fusion}
\label{subsec:integration}
Required inputs for RM are the ego states and its map relation. In a first step, we obtain a (visually) lane-matched ego position from a localization module. The ego vehicle is thus projected onto the lane center, determined by the localization module (i.e., the current ego lane). The R-LDM can now provide driving paths. As a second input of the demonstrated system, we need measured positions and velocities of other cars, see Section \ref{subsec:sensorproc}. 
In this pre-processing step, the four cameras (i.e., left, right, front and back) are used to detect surrounding obstacles in a 360$^{\circ}$ view. Specifically, detected bounding boxes of a YOLOv3 neural network \cite{redmon2018} are projected from each camera image into a joint 3D world. 

In the end, we fuse the pre-processed sensor signals with map data from the R-LDM, the graph-based environment representation. This can give us a driving situation. At every timestep, namely, $N_t\hspace{-0.08cm}=\hspace{-0.07cm}21\hspace{-0.04cm}$ ego trajectories are sampled and predicted with RM ($10$ acceleration and $10$ braking ramps with different acceleration/braking strength, and $1$ constant velocity) to retrieve a situation. Here, sampling is done for the current and parallel lane paths (i.e., $M_p\hspace{-0.06cm}=\hspace{-0.06cm}2$). 

With the middleware RTMaps, we may integrate the software components of R-LDM and RM in a single system. RTMaps reads the sensor inputs and writes final HMI outputs. The system could run with a frequency of around $f\hspace{-0.06cm}=\hspace{-0.06cm}\unit[10]{Hz}$. Our RM can then output a driving path and a velocity that promises safe behavior. As mentioned, the outputs are visualized on the final HMI for driver warning.    

\vspace{-0.15cm}
\subsection{Results}
\label{sec:results}

In the tests, the ego car drives on an ending lane and must take the gap between two vehicles. As we know, the driver may switch to its neighboring lane before or after the passing car (refer to Fig. \ref{fig:demo_scenario}). We will analyze examples of gap and no-gap situations below. These lane changes simultaneously include longitudinal and lateral spatial risks, which make them complex maneuvers. 

\subsubsection{Gap Scenario}
\label{subsec:gap}

For the scenario, recordings have been replayed with three vehicles and, subsequently, we predict their trajectories. Fig. \ref{fig:gap_scenario} pictures the different situation snapshots with the ego velocity $v_0$ as well as distances $d_1$ and $d_2$ to the two other cars on the neighboring lane. Herein, the planned ego trajectory is colored green, with the predicted trajectories of surrounding vehicles colored red. On average, the other cars drive in the complete stream with constant velocity.\footnote{Note that the other trajectories' length relates to a constant velocity, while the ego trajectories incorporate acceleration or deceleration.} 

\begin{figure}[t!]
  \centering
  \vspace*{-0.035cm}
  \resizebox{1.0\linewidth}{!}{\import{img/results/}{gap_scenario_shot1.pdf_tex}}
  
  \vspace{-0.094cm}
  
  \resizebox{0.85\linewidth}{!}{\import{img/results/gap/}{gap_riskmaps_shot1b.pdf_tex}\hspace{0.8cm}}
  
  \vspace{0.37cm}
  
  \resizebox{1.0\linewidth}{!}{\import{img/results/}{gap_scenario_shot2.pdf_tex}}
  
  \vspace{-0.075cm}
  
  \resizebox{0.85\linewidth}{!}{\import{img/results/gap/}{gap_riskmaps_shot2b.pdf_tex}\hspace{0.8cm}}
  
  \vspace{0.31cm}
  
  \resizebox{1.0\linewidth}{!}{\import{img/results/}{gap_scenario_shot3.pdf_tex}}
  
  \vspace{-0.21cm}
  
  \resizebox{0.85\linewidth}{!}{\import{img/results/gap/}{gap_riskmaps_shot3b.pdf_tex}\hspace{0.8cm}}
  \vspace{0.058cm}
  \caption{Behavior of RM and R-LDM in a gap scenario. Given are three snapshots from real-world recordings. The system signals to directly perform a lane change due to sufficient space between the neighboring cars. Top: Road layout and predictive situation, Bottom: HMI with risk visualizations.}   
  \label{fig:gap_scenario}
\end{figure} 

A planned lane change is realized by blending the current path with the parallel path at a start time $t_{\text{start}}\hspace{-0.05cm}=\hspace{-0.05cm}\unit[1]{\text{sec}}$ with a duration of $\unit[3]{\text{sec}}$ (for details, see Section \ref{subsec:lat}). This aligns with usual lane change durations, e.g., according to \cite{toledo2007}. 
For the visualizations, we choose a prediction horizon $s_h$ of $\unit[6]{\text{sec}}$ but the actual risk is evaluated for $s_h\hspace{-0.06cm}=\hspace{-0.06cm}\unit[12]{\text{sec}}$. The car signals and possible paths are updated in the R-LDM on demand. 

Fig. \ref{fig:gap_scenario} also illustrates the risk graph. The sampled ego trajectories are plotted as curves of velocity over the future time and the sum of of collision and curve rates with $\tau^{-1}_{\text{crit}}$ are further visualized. The blue areas in this graph represent low probabilities between $[\unit[0]{\%/\text{sec}},\unit[0.5]{\%/\text{sec}}]$ and red areas are high values with $[\unit[0.5]{\%/\text{sec}}$ and also values $>\hspace{-0.085cm}\unit[1]{\%/\text{sec}}]$. Finally, the single chosen trajectory is highlighted with green points. RM always tries to find a trajectory that bypasses the red hot spots. 

In this scenario, RM successfully judges the gap as sufficiently large. Changing the lane with moderate acceleration from $\unit[7]{m/\text{sec}}$ to $v_{\text{tar}}\hspace{-0.04cm}=\hspace{-0.04cm}\unit[11]{m/\text{sec}}$ presents the optimal maneuver. After the lane change, the message on the HMI changes from "go left" to "go straight" (i.e., the target path $\textbf{p}_{\text{tar}}$). This example emphasizes the possible proactive support that can be provided by RM. 

\begin{figure}[t!]
  \centering
  \vspace*{-0.05cm}
  \resizebox{1.0\linewidth}{!}{\import{img/results/}{no_gap_scenario_shot1.pdf_tex}}
  
  \vspace{-0.04cm}
  
  \resizebox{0.85\linewidth}{!}{\import{img/results/no_gap/} {no_gap_riskmaps_shot1b.pdf_tex}\hspace{0.8cm}}
  
  \vspace{0.23cm}
  
  \resizebox{1.0\linewidth}{!}{\import{img/results/}{no_gap_scenario_shot2.pdf_tex}}
  
  \vspace{0.075cm}
  
  \resizebox{0.85\linewidth}{!}{\import{img/results/no_gap/}{no_gap_riskmaps_shot2b.pdf_tex}\hspace{0.8cm}}
  
  \vspace{0.37cm}
  
  \resizebox{1.0\linewidth}{!}{\import{img/results/}{no_gap_scenario_shot3.pdf_tex}}
  
  \vspace{-0.30cm}
  
  \resizebox{0.85\linewidth}{!}{\import{img/results/no_gap/}{no_gap_riskmaps_shot3b.pdf_tex}\hspace{0.8cm}}
  \vspace{0.032cm}
  \caption{Behavior of RM and R-LDM in a no-gap scenario. A forced lane change with an acceleration is advised only after the preceding car passes. This represents a safe motion and the driver follows the warning.} 
  \label{fig:no_gap_scenario}
\end{figure} 

\subsubsection{No-Gap Scenario}
\label{subsec:nogap}
In the previous gap scenario, the system recommended to either drive with constant velocity or to accelerate for performing the lane change. The front and back car were determining the gap. In this second scenario, the gap is too small for safely conducting a lane change. Only after the two other vehicles have passed and drive with appropriate distance to the ego vehicle, RM signalizes that the driver could make a lane change.

Within Fig. \ref{fig:no_gap_scenario}, the preceding car hinders the ego car to go on the target lane. Hereby, we want to analyze more closely the risk graph for the first screenshot of the scenario. If the ego vehicle brakes or drives constantly and makes the lane change, high risks are caused by the preceding vehicle. Due to the close lateral distance, a large risk spot is depicted in the bottom left of the graph. Secondly, if the ego car accelerates and changes lane, there is a high risk of colliding with the front vehicle, indicated by the other red area. The risk graph shows the reasoning behind the system's recommendation output: ``brake'' and ``stay  on the lane''. 

The actual driven speed $v_0$ and target speed $v_{\text{tar}}$ is depicted in the velocity scale. As a reminder, the comfort velocity parameter $v_d$ determines a utility gain and can be chosen by the user. In Fig. \ref{fig:no_gap_scenario}, the system penalizes velocities that deviate from $v_d=\unit[10]{m/\text{sec}}$. To find an ego trajectory in terms of risk versus comfort, we sanction high accelerations as well. RM, therefore, initially recommends to brake until $\unit[4]{m/\text{sec}}$ because of the ending lane. When the preceding vehicle passed the ego vehicle, RM then correctly recommends to drive with constant velocity $\unit[6]{m/\text{sec}}$ and $v_{\text{tar}}$ is lastly accelerating until $\unit[9]{m/\text{sec}}$. The ego driver's speed stays around $v_0=\unit[5]{m/\text{sec}}$ and the lane change suggestion is followed.

\vspace{-0.06cm}
\subsection{Discussion}
\label{subsec:noise}
As we could see, the combination of RM and R-LDM allowed for a reasonable support of the driver in lane change scenarios. In this Section \ref{subsec:noise}, a short discussion concerning the results of the system is given. After showing insights into noise robustness with real-world sensors, the generalization for other driving situations is discussed.

Noise in the vehicles' position and velocity due to the real cameras and GNSS sensors can potentially hinder a faultless operation. In the system, we parametrized the weights in the planning (i.e., for risk, utility and comfort) to achieve robustness against such noise errors. 
As explained in Section \ref{subsec:sensorproc}, positions of all vehicles are projected to the closest path, derived from the R-LDM. With regard to risk computations, the noise did not affect the lane change advices of our experiments. Position noise that is constantly present is merely carried into RM in the future time, where the survival analysis compensates its influence. 

However, the safe speed output from the planner can slightly fluctuate. This is due to the velocity noise in the ego and other vehicles, which grows over the prediction time. Furthermore, a projection on wrong lanes induces the biggest errors. To tackle this noise propagation and the impact of discrete errors, we introduce a hysteresis in RM.
The hysteresis changes the outputs solely when the new selection shows lower risk values for at least $\unit[2]{\text{sec}}$. 
The more RM is robust against this sensor noise, the less it will proactively react on the changes in the environment. In the end, a tradeoff was chosen between stable warnings versus better sensitivity.  

The given experiments show that a map-based behavior support works online and has benefits in terms of its predictability and transparent risk model. The last Section \ref{sec:results} served as a first proof of concept that RM can be applied in real-time on a car platform. The tests were conducted with velocities of $\unit[30]{km/h}$. Nevertheless, other applications on Spanish city streets (i.e., $\unit[50]{km/h}$ limit) and German highway entrances (on average, $\unit[100]{km/h}$) show the same functionalities and beneficial properties. For these test cases, we applied RM on other car platforms using higher-cost GNSS and lidar.  

In addition, in previous scientific work we showed that the same risk-based planner can be successfully applied in simulation to turning at curves (see \cite{puphal2018}), merging at intersections (see also \cite{puphal2018}) or crossing at intersections (see \cite{puphal2019}). For this reason, we consider the planner to be generalizable to various traffic cases in urban or highway driving due to the generic characteristics of the models. The transparent model-based risk approach allows to visualize and understand the reasoning behind its decisions.

