\vspace{-0.1cm}
\section{Representation of Driving Situations} 
\label{sec:fusion}
In support systems, clear structures of sensor measurements, trajectory patterns and driving situations are a prerequisite for their efficient real-world application. 
In this regard, world models are helpful as central storages, whereby so-called Local Dynamic Maps (LDM) are a potential way for their realization. LDM serve for the fusion of ADAS-relevant data.

According to the spirit of a Local Dynamic Map, we construct a database that represents several layers depending on their dynamicity. 
Dynamic entities, such as traffic participants, are stored on the highest layer.
The next layer is consisting of transient data, such as information on traffic signal phases, traffic congestion or road condition. 
On the second bottom layer, then, quasi-static objects, such as building outlines, are stored. The difference between the transient and quasi-static data is the dynamicity, which is changing on hours scale in the first and on the scale of days in the latter case.
Finally, the bottom layer serves to manage static data of the environment, which includes, e.g., spatial geometry data of lower layers that can be enriched by dynamic entities, e.g., traffic participants, of higher layers.

The left part of Fig. \ref{fig:ldm_layers} illustrates all of the four layers from the R-LDM concept. 
However, this paper focuses on the static and dynamic layer.
Our LDM implementation, the so-called Relational Local Dynamic Map (R-LDM), is hereby realized as a graph.
The R-LDM embodies an interconnected graph of nodes. 
Since relations present first order citizens of a graph, connections  between entities (e.g., spatial or temporal) can be described in a straightforward way. 
Attributes of nodes present an additional possibility to store information that is directly linked to a certain node.

\begin{figure*}
    \centering
    \includegraphics[width=1.0\linewidth]{img/rldm2b.pdf}
    
    \vspace{0.13cm}
    
    \caption{Interaction of Relational Local Dynamic Map (R-LDM, left) and Risk Maps (right). Structured information, e.g., driving paths, from the local environment and traffic situation is leveraged for predicting trajectories, determining collision risk, utility and comfort. The data structures are connected in different levels of granularity, see dashed lines. Note that the parts of the figure can stand alone.}
    \label{fig:ldm_layers}
\end{figure*}

This section builds upon this R-LDM, initially introduced in \cite{ldm2017}, and describes extensions towards the consideration of driving risks. 
We start with the description of static map data in Section \ref{subsec:map}, show how to manage dynamic entities with the graph in Section \ref{subsec:layers} and finish with the processing of sensor inputs in Section \ref{subsec:sensorproc}. 

Especially Section \ref{subsec:layers} and Section \ref{subsec:sensorproc} show how to connect trajectories as well as driving risks in the R-LDM, which was not described in the original R-LDM implementation, published in \cite{ldm2017}.

\subsection{Static Environment}
\label{subsec:map}
Tasks such as the estimation of collision risk strongly benefit from knowledge on the current and future, i.e., predicted, traffic situation. 
Such situation evolution can be described by a list of potential paths a driver can take.
Each path represents a sequence of positions that is derived from map data and builds the basis of the system's predictions.

The R-LDM enables the storage of road geometries on three levels of detail, i.e., road, half-road\footnote{Here, a half-road is defined to be the union of all lanes that point into the same direction.} and lane level. In terms of the graph, each level of detail is represented by a hierarchy of sub-graph patterns that consist of one central node (e.g., label ``:LaneSegment'') as well as related child nodes.
For example, a road is connected to up to two half-road nodes, while each half-road node can be connected to an arbitrary amount of lane nodes.

Going into further detail, the focus now lies on the lane-level geometry since it builds the basis of the path retrieval. Yet, the information applies to road and half-road level accordingly.
A :LaneSegment node stores the inherent information, such as a center polyline defining the direction of travel, as an attribute. 
Further node attributes include, amongst others,

\begin{itemize}
\item road type,
\item surface material,
\item lane marking type and
\item road curvature.
\end{itemize}

Subordinate entities, e.g., lane markings or boundaries detected by sensors, are represented by individual nodes and are connected via ``:hasPart'' relations. 
Further connected nodes include traffic signs as well as further entities of the same or different R-LDM layers that are related in a direct or indirect sense. 

The list of attributes and the level of detail 
can be chosen in accordance with the individual use case. 
However, also the omission of information can be beneficial, since the computing time correlates with the amount of data stored in the graph.  

Regarding the acquisition of raw map geometry data, common approaches are the extension of publicly available sources (e.g., OpenStreetMaps \cite{OSM}), the parsing of digital orthophotos (satellite imagery) or the derivation of road geometry from pre-recorded GNSS trajectories. 
Assuming that a vehicle drives in the center of the road, the latter approach natively presents a suitable means for validating the hypothesis.

\subsection{Dynamic Environment}
\label{subsec:layers}

As already mentioned, a crucial group of objects are the traffic participants, such as, vehicles, cyclists, and pedestrians. This is especially the case for a predictive warning system for dynamic lane changes.
In this context, the R-LDM can be interpreted as an ego-centered knowledge hub that represents the surrounding of an agent zero.
More specifically, the agent zero is moving within a (potentially) pre-mapped environment that is continuously enriched by measurements and sensed data, which is shown in the middle of Fig. \ref{fig:ldm_layers}. 

The ego car, or more general the ego vehicle, is represented by a node.
For any type of sensor, a node with the label :Sensor is created that is connected to the entity node via a ``:hasPart'' link. 
The type of sensor is stored as an attribute with the key type.
Depending on type and amount of information, the actual data can be stored explicitly as attribute or implicitly  as a reference to another place as, e.g., a hard drive or a time series database.  
In the case of explicit storage, it can be inefficient and computationally expensive to push data from an Inertial Measurement Unit (IMU) with 100 Hz. 
Instead, the sensor data updates are pushed at a pre-defined, suitable frequency to the R-LDM.

Any kind of measurement, e.g., object detection from cameras or GNSS position estimates can be considered, including other traffic participants.  
Each sensed entity, such as another vehicle, can afterwards be represented by a graph node that is connected via a ``hasMeasurement'' link. 

Additionally, traffic participants are also connected to static elements, in other words, the road infrastructure.
One common example is a lane segment that is utilized by vehicles, indicated in the R-LDM by the ``:contains'' links between the lane and vehicle nodes. 
In this context, the center of a lane serves as an approximation for the driving path, i.e., the path that is most likely be driven by the vehicle.
In multi-lane scenarios, several paths can be determined with discrete sets of behavior, e.g., stay on the lane or perform a lane change. 

While a path represents consecutive points in a 2D-space, a trajectory also conveys timestamps for each data point.  

For Risk Maps (RM), we require trajectories that  are based on the assumption of fixed dynamics (e.g., constant velocity) for predicted times and are described as a sequence of position $(\text{x}, \text{y})$, velocity $v$, acceleration $a$ and jerk $j$. 
Furthermore, RM considers benefits of the ego vehicle (e.g., desired speed) and risks between the ego and other vehicles. 
Fig. \ref{fig:ldm_layers}, on the right, depicts how RM benefits from the R-LDM.

\subsection{Data Processing} 
\label{subsec:sensorproc}

The preceding Sections \ref{subsec:map} and \ref{subsec:layers} explained how map data and dynamic entities can be stored in a graph. 
Now, we illustrate how stored data interacts with acquired sensor data. In this context, Fig. \ref{fig:coord_trans} shows the ego (green) and a sensed (red) vehicle that are both projected onto the closest map paths. 
For doing so, two techniques are employed to process such sensor signals: 1) the alignment of the ego car on a driving path and 2) the filtering of unlikely obstacle detections. 
In this way, we can afterwards predict the trajectories along map paths. 

\subsubsection{Ego Alignment}
The positions $(\text{lat}, \text{lon})$ of the ego car from a GNSS sensors are given.
In a first step, we transform the GNSS signals to Cartesian coordinates. 
The equirectangular projection with a radius of earth $r_e$ and latitude $\text{lat}_0$, which is close to the center of the map, allows for the conversion of geodetic GNSS coordinates $(\text{lat}, \text{lon})$ to Cartesian coordinates $(\text{x}, \text{y})$ via the equations
\begin{align}
\text{x} = r_e &\cos \text{lat}_0 \cdot \text{lon}, \\ %\vspace{0.2cm}
\text{y} &= r_e \cdot \text{lat}.
\end{align}

Here, the $x$-axis of the global reference frame is facing east, while $y$ is facing north. 
We project the ego car on the closest  path. 
This description assumes the earth to be an evenly round 

\noindent globe. However, the reader can choose any earth model that fits the individual accuracy requirements, such as assuming an earth ellipsoid.  
For deeper insight into map-based lane-level localization, we refer to \cite{localization2020}.

\subsubsection{Obstacle Filter}
The visual detection of the surrounding vehicles requires a coordinate transformation of their positions from an ego-relative body frame $(x_{\text{rel}}, y_{\text{rel}})$ into the shared world reference frame $(x, y)$. 
For this purpose, we use a rotation transformation with the angle $\Theta$ between both reference frame. Hereby, we use the standard, right-handed orientation of the world frame. If we additionally assume flat surfaces, we get

\vspace{-0.44cm}

\begin{equation}
\text{x} = \cos \Theta \cdot x_{\text{rel}} - \sin \Theta \cdot y_{\text{rel}},
\end{equation}

\vspace*{-0.47cm}

\begin{equation}
\text{y} = \sin \Theta \cdot x_{\text{rel}} + \cos \Theta \cdot y_{\text{rel}}.
\end{equation}

\vspace{0.02cm}

\noindent The equations can be extended to cover 3D-space in a straightforward way. This allows for the consideration of, e.g., non-flat environments. 

\begin{figure}[t!]
  \centering
  
  \vspace{0.18cm}
  
  \resizebox{0.95\linewidth}{!}{\import{img/}{coordinate_system_transform.pdf_tex}}
  \vspace*{0.22cm}
  \caption{Pre-processing steps for map-based planning in R-LDM. Left: Car signals are transformed into the same global reference frame. Right: For motion predictions, we filter other cars that are far away.} 
  \label{fig:coord_trans}
\end{figure}

Similar to the ego alignment on its path, the other cars are projected to their paths as well. However, the main difference is that we only consider vehicles that are close to any path and filter cars with a distance $d_{\text{proj}}>\unit[5]{m}$ away. This geofencing limits position errors from sensors. In the end, we compute the distance $d$ in the global frame for the collision risk calculation. 

\noindent In summary, the R-LDM allows us to prepare situations for risk probing within RM. This method will be described in the following. 
