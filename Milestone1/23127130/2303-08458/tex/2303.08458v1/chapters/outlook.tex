\section{Conclusion and Outlook}
\label{sec:outl}

\urlstyle{rm} 
In summary, this paper served to evaluate an application of the Risk Maps (RM) technology on a prototype vehicle in order to show its real-world applicability. In an online demonstration, we presented warning functions for the example of forced lane changes. For this purpose, we extended RM to handle lateral path options.

The system utilizes the Relational Local Dynamic Map (R-LDM) for the alignment, filtering and dynamic layering of car signals with map data. Afterwards, RM are used to predict the current traffic situation into the future and probe trajectories and paths based on driving risks for planning. 
An improved GNSS position hereby allows to position the ego vehicle on the road. RM can handle noise of the camera sensors for object detection by considering uncertainty by design. In addition, the visualization of a risk graph supports the human driver to understand the driving situation. 

Our resulting ADAS was demonstrated on the ITS European Congress 2019 in Eindhoven, the Netherlands. Videos of the results can be found on Youtube: \url{https://www.youtube.com/watch?v=8o3hT3H_gDU}. The real-world evaluation shows that the system is capable of giving reasonable advice on target speeds and lane changes, separated in gap and no-gap situations. Effectively, we may improve the prediction capability for the user with a risk-based planner. RM represents a white-box model, which handles sensor noise and complex predictions. This is promising to increase trust on ADAS.

In the current realization of the system, we assume path blending with fixed parameters of duration time. In a revised solution, the parameters should be optimized \cite{weisswange2019}. Helpful are heuristics to make sure that the lane change is conducted with sufficient space to the preceding and leading vehicle. By using general risks, we believe that RM can implicitly consider dependencies between velocities and paths. This is especially helpful for autonomous driving.

For future research, we likewise envision the use of situation classifiers to improve risk predictions. Potential behavior of other cars need to be explicitly reflected (i.e., lane changes, accelerations or decelerations). Currently, interaction between vehicles is only modeled by a general increase of uncertainty in the trajectories. However, for interaction-intensive situations (e.g., lane changing in dense traffic, abruptly stopping vehicles, suddenly appearing occluded vehicles or similar dynamic conditions), the intention of other cars need to be predicted. Hereby, machine learning methods offer themselves to predict trajectories based on previously measured positions. 

Finally, a face detection and gaze estimation from eye pupils \cite{diego2018ddi} were showcased in the demonstration as well. 
Our planner can benefit from adding such detection technologies. Knowing the gaze allows for the inference of a targeted lane change and helps to determine if a driver is aware of critical objects. We are thus able to limit warnings to non-look cases. Since this reduces the alarm rate and workload of the driver, the social acceptance of the ADAS may be further improved.

