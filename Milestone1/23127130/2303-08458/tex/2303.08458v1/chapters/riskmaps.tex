\section{Probing inside Risk Maps}
\label{sec:riskmaps}
\vspace{0.02cm}
Using the R-LDM, we are able to fuse measured data into driving situations. 
However, assessing a multi-lane situation based on its risks thoroughly and, at the same time, in a fast manner remains challenging. The reason lies mostly in the variety of possible predictions for such situations. Additionally, there are underlying driving uncertainties, which arise from car sensors and likewise, e.g., from the driver behavior. 

In this context, planning methods must determine an optimal maneuver for the ego driver. With map data, the driving space was constrained and we differentiated between static paths on the one hand, and dynamic trajectories on the other hand. Still, a single path and trajectory that is safe and beneficial for the ego driver needs to be found. 
Therefore, so-called "probing" inside RiskMaps (RM) will be leveraged. We consider probing as the prediction of an ego driving motion and its evaluation of induced risks. By intelligently probing adequate numbers of fixed trajectories along paths in RM, motions will be efficiently planned. 

This section describes the concepts behind the planner RM. For this purpose, we will first explain the trajectory probing in Section \ref{subsec:traj} with RM and we will continue with the cost evaluation of risk and benefits in Section \ref{subsec:survival}. In the last Section \ref{subsec:lat} and Section \ref{subsec:warning}, we will finalize the description of RM with the extension of path probing, which ultimately allows to warn the driver. 

\begin{figure}[t!]
  \centering
  \vspace{0.06cm}
  \resizebox{0.97\linewidth}{!}{\import{img/}{single_ramps.pdf_tex}}
  \vspace{0.25cm}
  \caption{Trajectory probing within RM. Left: Ego trajectories are planned with different acceleration and braking strengths. Right: Evaluation of the risks, utilities and comforts for constant velocity of another car. The figure shows the risk calculation in more detail.}
  \label{fig:traj_variation}
\end{figure} 

\vspace{0.05cm}
\subsection{Trajectory Probing}
\label{subsec:traj}
Hereinafter, we analyze a car-to-car encounter with collision risks, depicted in Fig. \ref{fig:traj_variation}. For every ego trajectory variation $h$, constant velocity is assumed for the other car (i.e., $v=\text{const.}$). For the ego car, we predict and create variations for a piece-wise function composed of an acceleration/braking phase and a zero-acceleration, resp., constant speed segment. Technically, we first sample $N_t$ velocity profiles $v(s)$ on the path that start at the current velocity $v_0$ and end at the planned velocities $v^h$ in the future time $s$, equidistantly sampled inside $v\in[0, v_{\text{max}}]$. In total, we thus get

\vspace{-0.02cm}

\begin{equation}
v^{h} = \frac{h}{N_t-1} \cdot v_{\text{max}} \text{\hspace{0.02cm} with \hspace{0.025cm}} h \in 0, ..., N_t-1.
\end{equation}

\vspace{0.05cm}

To reach those end velocities $v^h$, the accelerations $a^h$ are afterwards calculated that are maximal with $a_{\text{max}}$ in case of an ego trajectory that corresponds to $v_{\text{max}}$. For the cases ending in a full stop $v^h=\unit[0]{m/\text{sec}}$ of the ego vehicle, the acceleration $a^h$ becomes minimal with $a_{\text{min}}$. Note that $a_{\text{min}}$ represents hereby a negative acceleration (i.e., braking motion). This leads to

\vspace{-0.5cm}

\begin{align}
\text{if } v^h > v_0\text{: \hspace{0.025cm}} a^{h} = a_{\text{max}} \cdot \frac{v^h-v_0}{v_{\text{max}}- v_0}, \\
\text{else if } v^h < v_0\text{: \hspace{0.025cm}}a^{h} = a_{\text{min}} \cdot \frac{v_0-v^h}{v_0}. 
\end{align}

\vspace{0.01cm}

The intervals with either an acceleration or braking phase are planned for the predicted durations of $s\in[0, s_a]$ or $s\in[0, s_b]$, respectively. We formulate the included times as

\begin{equation}
s_{a} = \frac{v_0}{a_{\text{max}}} \text{\hspace{0.015cm} and \hspace{0.01cm}} s_{b} = |\frac{v_0}{a_{\text{min}}}|. %\left\| \right\|
\end{equation}

\vspace{0.1cm}

\noindent The included time in the equations is denoted as $s$ since it is not the real time $t$ but the predicted time. We assume a constant velocity for the ego vehicle in the subsequent segment $[s_a, s_h]$ and $[s_b, s_h]$. The parameter $s_h$ defines the time horizon of the velocity profile and is set according to the task. Fig. \ref{fig:traj_variation} (on the left) depicts this longitudinal probing. 

In the presented planner, a target velocity $v^h = v_{\text{tar}}$ from the RM set is eventually selected based on the explicit tradeoffs between risks $R(t)$ and benefits, which are further divided into the ego utility $U(t)$ and ego comfort $O(t)$, see Fig. \ref{fig:traj_variation} on the right. In this process, at least $N_t\geq3$ ego trajectories have to be sampled so that RM can choose between an acceleration and a braking option.  
Altogether, RM could potentially represent a fast planner applicable to avoid accidents. 

The next section will now describe the mentioned costs to select an optimal motion. Using the survival analysis approach, we include probabilistic assumptions of Gaussian uncertainty (visualized as 2D, red ellipses) and of Poisson uncertainty (escape arrows for the ego cars). However, this section represents a summary of the approach. For more details, please refer to the original publication \cite{puphal2019}. We extend and show novel ways how to plan lane changes afterwards.

\vspace{-0.2cm}
\subsection{Survival Analysis}
\label{subsec:survival}

We cannot presume that predictions of other cars, e.g., constant velocity, will be followed in reality. The survival analysis thus models uncertainties for their predicted positions with 2D Gaussians that are growing over time. These uncertainties are influenced by sensor uncertainty (dominant in first steps) and behavior uncertainty (dominant in later steps). Finally, a collision probability is given by the overlap of the Gaussians from the ego vehicle and all other vehicles.\footnote{Note that this collision probability is, in the end, dependent on the relative velocity and distance of the vehicles.} 

Since risk is defined as the probability of this critical event multiplied with a damage outcome, we include a severity term in the final risk output. 
Besides collision risks, RM allow for the analysis of further critical events, such as upcoming sharp curves. More specifically, RM model the probability to lose control and skid off during lane changes. In these cases, we assume 1D Gaussians around the ego car and look at the lateral acceleration which is influenced by road curvatures.

\vspace{0.015cm}
\subsubsection{Risk Equations}
With the help of a Poisson process, we are able to accumulate probabilities over the future time $s$ (i.e., from collisions and a curve) with the damages into a single, scalar risk value $R(t)$ for the current time  $t$. In detail, we calculate the time difference between two critical events $\tau_{\text{crit}}^{-1}(s)$. Those critical events are then divided into collision rates $\tau_{\text{coll},j}^{-1}$ (with the index defining the considered other car $j$) and a curve rate $\tau_{\text{curv}}^{-1}$. 

Due to the included survival function $S$ in a Poisson process, risks that occur further away in the prediction are considered less in the risk $R(t)$. This so-called ``escape'' effect is modeled for the ego car. We thus describe $R(t)$ by 

\vspace{-0.2cm}

\begin{equation}
R(t) = \int_0^{\infty} (\sum_j \tau_{\text{coll},j}^{-1}D_{\text{coll},j}+\tau_{\text{curv}}^{-1}D_{\text{curv}})S \,ds.
\label{eq:risk}
\end{equation}  
For Eq. (\ref{eq:risk}), the severity is hereby formulated in a desired accuracy based on a collision model $D_{\text{coll}}$ and $D_{\text{curv}}$. 

\begin{figure}[t!]
  \centering
  
  \vspace*{-0.252cm}
  
  \resizebox{0.82\linewidth}{!}{\import{img/}{cost_evaluation.pdf_tex}}
  \vspace{0.1cm}
  \caption{Visualization of the driving risk (i.e., from collision and curve), utility and comfort costs. The trajectory costs are computed over the future time and then integrated with the survival analysis to obtain a single cost scalar. Note: The graphs should only give a qualitative notion of how the costs may look like.} 
  \label{fig:single_costs}
\end{figure} 

In the end, we can visualize a risk graph showing risk hot spots for the taken velocity profile that correspond to the other cars. Such a risk graph is plotted in Fig. \ref{fig:single_costs} on the top left. As an example, it shows risk spots for two separate cars, which become visible for trajectories when the ego car come closest to the other car, while the shape of the hot spots depends on the Gaussian uncertainties of the vehicles. At this point, it should be highlighted that the risk graph shows risks over the future time, which is different to common potential field visualizations, such as \cite{guo2019}.

\vspace{0.03cm}
\subsubsection{Utility and Comfort Prediction}

The target of RM is to mimize risks as well as to maximize benefits $B(t)$. The total

\noindent costs are calculated with $C(t) = R(t)-B(t) = R(t)-U(t)-O(t)$.
In this sense, we express benefits with high utility and comfort, both also depicted in Fig. \ref{fig:single_costs}. 

Utility $U(t)$ depends on the driven ego velocity $v_1$, which attempts to reduce the general required time to arrive at a goal. Additionally, we consider a desired velocity $v_d$ for individual preferences. 
For decreasing effects of $U(t)$ at higher times $s$, we multiply the components with the survival function $S$ and write 

\vspace{-0.25cm}

\begin{equation}
U(t) = \int_0^{\infty} (b^t |v_1| + b^d |v_1 - v_d|) S \,ds.
\label{eq: utility}
\end{equation} 

\vspace{0.02cm}

Comfort $O(t)$ takes ego acceleration $a_1$ and jerk $j_1$ into account and ensures less abrupt switches between selected behaviors. The major target is still to avoid risks. Consequently, comfort is also reduced over the survival function $S$. For $O(t)$, we gain
\vspace{-0.1cm}
\begin{equation}
O(t) = \int_0^{\infty} -(b^c |a_1| + b^j |j_1|) S \,ds.
\label{eq: comfort}
\end{equation} 

With the weight parameters $b^t$, $b^d$, $b^c$ and $b^j$, in both $U(t)$ and $O(t)$, the importance of risk versus benefit can be tuned. This enables us to select a single trajectory with minimal costs $c^h$ among all sampled trajectories $N_t$. Particularly, in Fig. \ref{fig:single_costs}, the first trajectory $v^1$, i.e., with $h=1$, was chosen because $c^1$ represents the smallest cost of all trajectories. With the cost graphs of $U(t)$ and $O(t)$, visualized also in the figure alongside risks $R(t)$, we can therefore intuitively indicate the underlying reason for any selection of the planner.

\subsection{Path Probing} 
\label{subsec:lat}

For the last module of the resulting warning system, this section outlines path planning. Building upon the ego trajectory probing and cost evaluation from RM, a novel path probing technique is proposed for RM. In this way, the forced lane change of the introduction is solvable. The driver has the support of the HMI, recommending a target motion. 

As already mentioned, in large multi-lane roads (i.e., highways), a driver has distinct spatial options with the possibility of performing a lane change. A forced lane change arises either for, e.g., following the navigation route, avoiding neighboring cars or both at the same time. In these instances, the lane change path and its start time and duration must be determined. The RM make tactical decisions, which are, e.g., lane changes for a future time (depicted in Fig. \ref{fig:path_blending}). 

\begin{figure}[t!]
  \centering
  \vspace{0.23cm}
  \resizebox{0.89\linewidth}{!}{\import{img/}{merge_path.pdf_tex}}
  \vspace{0.15cm}
  \caption{Signal outputs of RM. By also probing the costs for possible ego paths, we can plan tactical lane changes. Left: Blending of path options and their selection. Right: HMI with target velocity and lane change advice.}
  
  \vspace{0.05cm}
  
  \label{fig:path_blending}
\end{figure}

In detail, we compute a path change that serves to blend between centerlines of the ego lane and of a neighbouring lane. The blending begins at a longitudinal distance $l_{\text{start}}$, which is defined as
\vspace{-0.06cm}
\begin{equation}
l_{\text{start}} = v_0 \cdot s_{\text{start}}.
\end{equation}
Accordingly, this distance $l_{\text{start}}$ depends on the current velocity $v_0$ and the future time $s_{\text{start}}$. The length of the blending interval $l_{\text{blend}}$ is, in contrast, influenced by the current, lateral distance $d_{\text{path}}$ between the two path options. Since the segment end $l_{\text{end}}$ is composed of $l_{\text{start}}$ and $l_{\text{blend}}$, we retrieve
\vspace{0.03cm}
\begin{equation}
l_{\text{end}} =  l_{\text{start}} + l_{\text{blend}} = l_{\text{start}} + v_0\sqrt{l_c \cdot d_{\text{path}}}.  
\label{eq:path_length}
\end{equation}

\vspace{0.03cm}

\noindent In Eq. (\ref{eq:path_length}), the parameter $l_c$ defines the scale factor for the increase. Simply put, for higher $l_c$, the lane change would also take more time.

In a final, next step, we blend the ego path $\mathbf{p}_{\text{ego}}(l)$ into the neighboring path $\mathbf{p}_{\text{other}}(l)$ within the segment $l_{\text{blend}}$, using a sigmoidal weighting term $w^*(l)$. Paths from Section~\ref{subsec:map} are initially resampled with evenly spaced points. The computed lane change path $\mathbf{p}_{\text{blend}}(l)$ then follows, with 
\vspace{0.07cm}
\begin{equation}
\mathbf{p}_{\text{blend}}(l) = (1-w^*(l)) \cdot \mathbf{p}_{\text{ego}}(l) + w^*(l) \cdot \mathbf{p}_{\text{other}}(l). 
\end{equation}

\vspace{0.08cm}

\noindent Generally, linear blending is achieved via the term
\vspace{0.05cm}
\begin{equation}
w(l) = \frac{l - l_{\text{start}}}{l_{\text{blend}}}, \text{with } l\in[l_{\text{start}}, l_{\text{end}}].
\end{equation}

\vspace{0.05cm}

\noindent and we thus utilize the weight $w(l)$ in a sigmoidal weight function $w^*(l)$ to gain
\vspace{0.05cm}
\begin{equation}
\vspace*{-0.1cm}
w^*(l) = \frac{{\fontsize{6}{8}\selectfont 1}}{1+e^{\mbox{\fontsize{9}{9.5}\selectfont{$-k(w(l)+0.5$)}}}}.
\vspace*{0.19cm}
\end{equation}
The constant $k$ allows to hereby tune the blending steepness and should be set to achieve a smooth blending. 
\vspace{0.05cm}

Fig. \ref{fig:path_blending} (left box) depicts two prototypical paths. They are calculated with the aforesaid variables. Note that points in the path have to be sampled densely enough for a reasonable blending. With the path blending, we can initiate an immediate lane change by setting $s_{\text{start}}=\unit[0]{\text{sec}}$, while a tactical change is obtained with, e.g., $s_{\text{start}}\hspace{-0.02cm}=\hspace{-0.02cm}\unit[2]{\text{sec}}$. A utility offset is necessary for the costs in order to incentivize the planner making a lane change. After all, RM choose the path based on its costs. 

\subsection{Selection and Warning}
\label{subsec:warning}

For multiple paths, the path blending is done iteratively. With the total path number $M_p$, ultimately, $M_{\text{p}} \cdot N_{\text{t}}$ samples are generated in RM, since ego trajectories are varied on each path. We select a single trajectory with the lowest costs for obtaining the target velocity $v_{\text{tar}}$ as well as the target path $\mathbf{p}_{\text{tar}}$. In other words, RM probe in longitudinal and lateral directions. The runtime of RM is constant because we have a fixed sample size. This suits very well for real-time purposes. In the case of Fig. \ref{fig:path_blending}, we get $M_p=2$ and sample $2 \cdot N_{\text{t}}$ trajectories. 

The planned, safe motion is now transferred into a driver suggestion or warning in an HMI so that it is explained to the driver. 
We compare the planned with the actual one to infer a warning. 

As shown in Fig. \ref{fig:path_blending} (right-hand side), the developed HMI contains a velocity scale with the current velocity $v_0$ and the safe velocity $v_{\text{tar}}$. Depending on the difference $|v_0-v_{\text{tar}}|$, the driver needs to change its behavior (accelerate, brake, etc.). Furthermore, a directional arrow depicts the path choice from the planner, which can be either ``left'', ``straight'' or ``right''. The driver should make a lane change, i.e., left, right, or stay on the lane, i.e., straight motion. The figure shows some output examples of the HMI, which are derived from the solution of the planner.
\vspace{0.12cm}

