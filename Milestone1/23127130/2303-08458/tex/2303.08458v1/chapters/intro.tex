\section{Introduction}
\label{sec:intro}
\vspace{0.05cm}

\IEEEPARstart{A}{dvanced} Driver Assistance Systems (ADAS) aim for supporting the human driver in its driving task \cite{bengler2014} and serve to avoid potential vehicle accidents. Prominent examples of ADAS are, e.g., the adaptive cruise control \cite{winner1996} and various lane keeping aids \cite{gayko2004}. They successfully help drivers on highway scenarios. Still, most of the current ADAS are not incorporating explicit driving predictions and uncertainty considerations in their planning.  Especially, in real traffic situations with high complexity, the technical requirements for ADAS robustness are high. Such situations include, e.g., dynamic lane changing or tailgaiting behaviors. Here, ADAS that allow to visualize the situation understanding modules in terms of driving risks are beneficial. A technical system must show reasons behind its decisions, which will eventually also increase the trust of the user. 

In previous research, we proposed the survival analysis \cite{puphal2019} for enhanced collision risk predictions under uncertainties. It is suitable for motion planning in situations with multiple interacting cars and has considerable advantages over common Time-To-X measures, which usually utilize time indicators as a risk approximation. In simulations \cite{puphal2018}, we proved that our method works reliably and does not produce collisions in a wide range of situations. 
For this paper, we target to extend and apply the risk framework on a test vehicle for online and predictive warning in forced lane changes (see Fig. \ref{fig:demo_scenario}). We thus show the benefitis of the approach: the easy visualization of the internal calculations.

Lane changes which are forced are complex and occur, e.g., due to road lanes that are ending, construction areas and parked cars. By evaluating behavior predictions with so-called Risk Maps (RM), we target to efficiently obtain lane change timings. A motion trajectory with minimal costs is selected (i.e., which balances risks with utility and comfort). In this process, a crucial component is the Relational Local Dynamic Map (R-LDM) that extracts driving paths, fuses the sensor information and prepares the driving situation. 

Essentially, in this paper, we move from simulation into the real world, which requires a prototype. 
The prototype consists of an inexpensive GNSS and multiple camera sensors. Fig. \ref{fig:demo_scenario}, at the bottom, shows the architecture of the system. 
In real experiments, RM with R-LDM will run on the test car and create warning outputs for the driver. The system can recommend the target velocities as well as path choices (go left, straight, etc.), in gap and no-gap variations. 
For the experiments, recordings are hereby drawn from a demonstration of the ADAS in the ITS European Congress 2019, described in \cite{vidas2021}.

\begin{figure}[t]
  \centering
  \vspace{0.63cm}
  \resizebox{0.87\linewidth}{!}{\import{img/}{highway_entrance2.pdf_tex}}
  
  \vspace{-0.13cm}
  
  \resizebox{0.91\linewidth}{!}{\import{img/}{rm_block_diagram.pdf_tex}}
  
  \vspace*{0.07cm}
  
  \caption{Risk Maps (RM) is implemented in a test vehicle and supports the driver in its behavior finding for a situation with an upcoming forced lane change. We focus on the modules indicated in blue. RM allows to visualize future risks and uncertainties for the driver.}

  \label{fig:demo_scenario}
\end{figure} 

To summarize, the contribution of this paper is threefold. On the scientific side, we build upon our previously published works \cite{puphal2019,puphal2018} and \cite{ldm2017}. However, we (i) extend the R-LDM to incorporate dynamic cars and their predicted behaviors. We show novel possibilities of how measurements and calculated variables can be saved in a database. Furthermore, we (ii) extend RM and plan besides longitudinal velocities, also lateral path choices, such as lane changes. A major novelty is the visualization of uncertainties in a predictive risk graph. On the technical side, we also propose a first, novel and real proof of concept. Concretely, we (iii) integrate both technologies and apply them on a prototype with an HMI system, highlighting the transparancy of the risk-based planner.

The remainder of this paper is structured as follows: Section \ref{subsec:rel} summarizes related research for ADAS. We continue with an R-LDM introduction in Section \ref{sec:fusion}, and outline the planning method of RM with Section \ref{sec:riskmaps}. Then, in Section \ref{sec:demo}, the test car and real-world demonstrations are outlined. Section \hspace{-0.05cm}\ref{sec:outl}\hspace{-0.05cm} finally presents conclusions\hspace{-0.01cm} plus\hspace{-0.022cm} future\hspace{-0.032cm} work. 

\subsection{Related Work}
\label{subsec:rel}

We will now compare the features of the risk-based ADAS with prior art from research. The focus of this section is put on the fields of trajectory prediction because the R-LDM uses map data to improve the predictions, on related risk methods, which are similar to the RM approach, and classical traffic models for motion planning in lane changes, since they are established baseline models. 

In the automotive field, road or map geometries are usually used to enhance the results of driving predictions for vehicles. For example, prevailing ADAS are extrapolating trajectories along paths from sensors and self-driving cars similarly rely on pre-recorded data \cite{ferguson2008}. As further alternatives to these map-based predictions, recently, learning algorithms are investigated. The authors of \cite{diehl2019}, e.g., learn typical trajectories in dynamic lane changes with Graph Neural Networks (GNN). Due to the vast behavior options for vehicles, prediction with fail safe backups are also shifting into focus \cite{pek2018}. 

In contrast to related work, we combine online sensor data with stored crowd-sourced map data. The data is managed by the R-LDM that allows for efficient data queries.

Alongside trajectory prediction, the development of driving risks\hspace{-0.03cm} becomes\hspace{-0.03cm} omnipresent\hspace{-0.03cm} in research. Hereby, risks are often sorted into discrete time or acceleration indicators, probabilis- tic risks as well as learned risks. Methods based on time metrics (see \cite{gassmann2019} for Time-To-Brake) convince due to their intuitiveness. However, neglecting uncertainty does not reproduce realistic driving situations. In probability considerations \cite{ruf2015}, tradeoffs must be found between accurate and computationally inexpensive models. Lastly, learning hazards \cite{watanabe2019} are powerful for specific complex events but struggle the most in untrained scenarios. 

The RM approach employs the survival analysis \cite{puphal2019}. On the one hand, this approach is more uncertainty-aware than state-of-the-art risks. On the other hand, the computational costs are reduced due to the analytical probability equations which form its basis. We probe situations with risks in the longitudinal (i.e., ego velocity) and lateral direction (i.e., possible lane change) to retrieve one optimal motion. In this context, probing is the driving prediction of a single ego behavior with an evaluation of induced risks with other cars.  

For further motion planning methods, we would eventually like to highlight traffic models in related work. They can easily be used for diverse applications. For instance, the Intelligent Driver Model (IDM) \cite{treiber2000} allows to follow preceding cars over kinematic equations. Using the extension of MOBIL \cite{kesting2007}, lane changes are considered in this process. The lane changes are based on simple predictions and are therefore understandable. However, the IDM does not regard driving uncertainties. 

In comparison, there are many advanced planners, such as lane change detectors \cite{zhang2020} or potential field approaches \cite{guo2019}, which are trying to include uncertainties in the car's positions for the planning. However, they often do not model uncertainty over the predicted time. Using RM, uncertainty-based risk is visualized over the predicted time and not only over the static positions. The proposed ADAS of this paper thus leverages the predictive nature in planning, while visualizing interpretable safety outputs to the user. 

