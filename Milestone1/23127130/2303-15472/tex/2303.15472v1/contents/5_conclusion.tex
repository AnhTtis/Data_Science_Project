
\section{Conclusion}
We have proposed a self-supervised rotation-equivariant network for visual correspondence to improve the discriminability of local descriptors.  Our invariant mapping called group-aligning shifts the rotation-equivariant features along the group dimension based on the orientation value to produce rotation-invariant descriptors while preserving the feature discriminability, without collapsing the group dimension. Our method achieves state-of-the-art performance in obtaining rotation-invariant descriptors, which are transferable to tasks such as keypoint matching and camera pose estimation. We believe that our approach can be further extended to other geometric transformation groups, and will motivate group-equivariant learning for practical applications of computer vision.

\noindent \textbf{Acknowledgement.}  
This work was supported by Samsung Research Funding \& Incubation Center of Samsung Electronics under Project Number SRFC-TF2103-02 and also by the NRF grant (NRF-2021R1A2C3012728) funded by the Korea government (MSIT).
