


\section{The number of sampled images for Roto-360} \label{sec:results_num_of_roto360}
\begin{table}[h]
\begin{center}   \vspace{-0.1cm} 
\scalebox{1.0}{
\begin{tabular}{c|ccc}
\# sample & 10      & 100   & 1K \\ \hline
Align     & \textbf{91.4}    & \textbf{80.0}  & \textbf{89.9} \\
Avg       & 82.1    & 72.3  & 80.7 \\
Max       & 78.0    & 69.3  & 79.2 \\
None      & 18.8    & 16.4  & 20.5 \\
Bilinear  & 41.0    & 28.5  & 43.7
\end{tabular} } \vspace{-0.2cm} 
\caption{
\textbf{Results on Roto-360 constructed using a different number of source images.}}\label{tab:diff_samples_roto360}
\end{center} 
\end{table}

Table~\ref{tab:diff_samples_roto360} shows the mean matching accuracy (MMA) at 5 pixels threshold when increasing the number of source images to 100 images (3,600 pairs) and 1,000 images (36,000 pairs).
The tendency of the matching results is maintained under increased diversity and complexity of the dataset, and group aligning consistently achieves state-of-the-art results.
Therefore, we use 10 samples as they are sufficient to measure the relative rotation robustness of the local features. 

