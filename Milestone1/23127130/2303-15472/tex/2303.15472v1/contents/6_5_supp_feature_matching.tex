


\section{Comparison with feature matching methods} \label{sec:compare_feature_matching}
\begin{table}[h]
\centering
\scalebox{1.0}{
\begin{tabular}{c|cc|c}
\multirow{2}{*}{method} & \multicolumn{3}{c}{Roto-360}   \\ \cline{2-4}
                        & @5px            & @3px   & pred. \\ \hline
ours+NN                 & \textbf{91.4}          & \textbf{90.2} & 688.3 \\
SP+SG~\cite{detone2018superpoint, sarlin2020superglue}                   & 30.1               & 29.8  & 874.1 \\
LoFTR~\cite{sun2021loftr}                   & 18.8              & 15.9 & 509.4 \\
\end{tabular}
}  \vspace{-0.2cm} 
\caption{\textbf{Comparison with keypoint matching methods on the Roto-360 dataset.} }\label{tab:copmare_with_matching} 
\end{table}
Table~\ref{tab:copmare_with_matching} compares the feature matching methods to our descriptors with simple  nearest neighbour matching (NN) algorithm.
We evaluate our local feature with nearest neighbour matching (ours+NN) and compare it with SuperGlue~\cite{sarlin2020superglue} (\ie, SuperPoint+SuperGlue~\cite{detone2018superpoint, sarlin2020superglue}) and LoFTR~\cite{sun2021loftr}. 
The results with the simple matching algorithm of ours+NN clearly outperforms the two other methods on the extremely rotated examples of the Roto-360 dataset.  
Note, however, that both SuperGlue~\cite{sarlin2020superglue} and LoFTR~\cite{sun2021loftr} are for feature {\em matching} and thus are not directly comparable to  our method for feature {\em extraction}. 
