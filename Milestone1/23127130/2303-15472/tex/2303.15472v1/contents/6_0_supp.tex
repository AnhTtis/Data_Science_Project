

In this supplementary material, we present a formal introduction of group equivariance briefly, an additional explanation of multiple descriptor extraction, results on the ERDNIM dataset, and additional qualitative results.
Section~\ref{sec:group_equivariance} explains a formal definition of equivariance and group equivariant networks.
Section~\ref{sec:additional_results} evaluates the matching quality of our proposed method under rotation and illumination variations on the day/night image pairs, with details about the benchmark generation.
Section~\ref{sec:imc2021_results} shows the results of realistic downstream task on the IMC2021~\cite{jin2021image} dataset.
Section~\ref{sec:compute_overhead_the_number_params} shows the comparisons of computational overhead and the number of parameters.
Section~\ref{sec:top_k_elaboration} shows different strategies of multiple descriptor extraction using dominant orientation candidates.
Section~\ref{sec:compare_feature_matching} evaluates the existing feature matching methods in the Roto-360 dataset.
Section~\ref{sec:changing_gift_rotation_range} shows the re-training results of GIFT with cyclic rotation augmentation.
Section~\ref{sec:results_num_of_roto360} shows the matching results with increasing the number of samples of the Roto-360 dataset.
Section~\ref{sec:additional_qualitative} presents additional qualitative results to visualize the consistency of dominant orientation estimation, the similarity maps under in-plane rotations of images, and predicted matches on the HPatches and extreme rotation (ER) datasets~\cite{balntas2017hpatches,liu2019gift}.














































