


\section{Elaboration of multiple descriptor extraction}\label{sec:top_k_elaboration}




In this section, we show the results of different configurations of the multiple descriptor extraction scheme which was mentioned in Section 4.3, Table 3, Table 4, and Table 6 of the main paper.



\begin{table}[h]
        \centering
        \scalebox{1.0}{
        \begin{tabular}{c|cc|cc}
        \multirow{2}{*}{cand.} & \multicolumn{4}{c}{Roto-360} \\ \cline{2-5}
            & @5px     & @3px        & pred. & total. \\ \hline
        top1  & 91.35  & 90.18        & 688   & 1161   \\
        top2  & 92.31  & 91.19        & 1315  & 2322   \\
        top3  & \textbf{92.82} & \textbf{91.69} & 2012  & 3483   \\ \hline
        0.8   & 92.25   & 91.13       & 951   & 1660   \\
        0.6   & \textbf{92.82} & \textbf{91.69} & 1333  & 2340  \\
        \end{tabular}        } \vspace{-0.3cm}
        \caption{ 
        \textbf{Results with different multiple descriptor extraction strategies.}
        The first group uses a static candidate selection strategy \textit{i.e.,} the number of candidate orientations is fixed.
        The second group uses the dynamic candidate selection strategy, where only the score threshold is determined, and the number of orientation candidates may vary.
        } \label{tab:top_k_inference}         
\end{table}
Table~\ref{tab:top_k_inference} shows the results with different strategies for multiple descriptor extraction on the Roto-360 dataset.
It can be seen that using a score ratio of 0.6 selects multiple candidates dynamically, where the total number of candidates is similar to using top-2 candidates, but the MMA@5px is as high as using top-3 candidates which uses a higher number of candidates.
Note that this multiple descriptor extraction scheme is largely inspired by the classical method based on an orientation histogram such as SIFT~\cite{lowe2004distinctive}.
Owing to the parallel computation of GPUs for mutual nearest neighbor matching, the time complexity of constructing a correlation matrix to find matches is $O(1)$ regardless of the number of candidates.



