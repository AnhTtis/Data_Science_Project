


 
\section{ Results of the realistic downstream task} \label{sec:imc2021_results}
\begin{table}[h]
\centering
\scalebox{0.9}{
\begin{tabular}{c|c|ccc}
\multirow{2}{*}{desc.} & \multicolumn{4}{c}{Stereo track}  \\ \cline{2-5}
                        & \# kpts    & mAA 5\degree     & mAA 10\degree   & \# inliers \\ \hline
SuperPoint & 1024          & 0.259          & 0.348          & 61.9             \\
GIFT       & 1024          & \underline{0.292}          & \underline{0.394}          & \underline{70.8}             \\
ours       & 1024          & \textbf{0.305}          & \textbf{0.404}          & \textbf{99.8}             \\ \hline
SuperPoint & 2048          & 0.263          & 0.358          & 73.9             \\
GIFT       & 2048          & \textbf{0.313}          & \textbf{0.420}           & \underline{98.6}             \\
ours       & 2048          & \underline{0.296}          & \underline{0.403}          & \textbf{118.5}       \end{tabular} 
}  \vspace{-0.2cm}
\caption{\textbf{Results of the downstream task in IMC2021~\cite{jin2021image}.} We use the SuperPoint keypoint detector for all methods. }\label{tab:imc2021_results} 
\end{table}
In Table~\ref{tab:imc2021_results}, we evaluate on IMC 2021 stereo track~\cite{jin2021image} using the validation set of PhotoTourism and PragueParks to show the results on a realistic downstream task.
Our descriptor consistently performs better than SuperPoint~\cite{detone2018superpoint} descriptors under varying number of keypoints, and obtains comparable results with GIFT~\cite{liu2019gift} descriptors.
This shows that our method performs similarly for the general and non-planar transformations, while it significantly outperforms existing methods on Roto-360 and RDNIM datasets with extreme rotation transformations. 
Note that it is also possible to use image pairs with GT annotations of intrinsic and extrinsic parameters by \textit{approximating} the 2D relative orientation for our training\footnote{The details of obtaining the rotation from a homography can be found in Section 2 of ``Deeper understanding of the homography decomposition (Malis and Vargas, 2007)''.}, and we leave this for future.


