\section{Carbon Emissions of FL}
\label{sec:impact}

In this section, we present the results of our study and measurements. We use CO$_2$-equivalents (\carbon), a standardized measure to express the global-warming potential of various greenhouse gases as a single metric. Carbon dioxide (CO$_2$) is not the sole greenhouse gas contributing to climate change. There exist other gases that also have a significant impact on the environment, and the aggregate effect of all these gases is quantified as \carbon: the number of metric tons of CO$_2$ emissions with the same global warming potential as one metric ton of another greenhouse~gas.

\begin{figure}[t]
\centering
\includegraphics[width=0.83\columnwidth]{figures/sync-async.pdf}
%\vspace{-0.25cm}
\caption{Carbon emissions of SyncFL and AsyncFL to reach a target perplexity. The text above the bars shows the time and rounds it takes to reach the target perplexity.}
\label{fig:1}
\end{figure}

\subsection{Estimating Carbon Emissions of Industry-scale FL System}

We quantified the carbon emissions of our synchronous and asynchronous FL system through hundreds of experiments, each with a different set of hyperparameters. 

First, we illustrate the total carbon impact of synchronous and asynchronous FL. Figure \ref{fig:1} shows the carbon emissions of synchronous and asynchronous FL training in our production stack to reach a target perplexity of 175. We have tuned both methods by finding the choice of hyper-parameters that led to the lowest time to target perplexity, as also used in Huba et al.~\cite{papaya}. In this setup, concurrency and aggregation goal are both set to 1{,}000. 


We can see that synchronous FL has a smaller carbon footprint compared to asynchronous FL. This is in contrast to the faster convergence of asynchronous FL (2.4 hours), which involves more model updates at the server (100). Asynchronous FL converges faster to the target perplexity due to its fast model updates. Due to its frequent iterations, asynchronous FL involves more clients. This result shows a fundamental trade-off between synchronous and asynchronous FL: if tuned well, \textbf{asynchronous FL is faster than synchronous FL as it advances the model more frequently in the presence of stragglers, but it comes at the cost of higher carbon emissions.}


% In this set of experiments, asynchronous FL has a smaller carbon footprint compared to synchronous FL. In both cases, we can also see that the majority of the carbon footprint is contributed to the compute on the user devices. Asynchronous FL is able to converge faster to the target perplexity, almost 3x faster than what synchronous FL can. Moreover, since synchronous FL is more wasteful than asynchronous FL, as it throws away the updates of slow clients, while asynchronous FL uses them to advance the model (stale updates). Conversely, asynchronous FL uses those stale updates to move the model forward, hence is more efficient in these experiments.

We can also see that the majority of the carbon footprint is contributed by the client compute --- consistent with FL's pushing the AI processing to the edge of the network. The server compute is a small fraction of the carbon emissions as shown in Figure \ref{fig:1} and other experiments. We observe that client compute and the communication between the clients and the server are responsible for the greatest share of FL's overall carbon emissions (97\%). The carbon footprint from the server-side computation is small ($\sim$1--2\%), while client computation contributes to almost half of the overall carbon footprint ($\sim$46--50\%). Upload and download networking costs are approximately 27--29\% and 22--24\%, respectively. 

Figure \ref{fig:2} illustrates the carbon emissions of synchronous and asynchronous FL training in our production stack after a fixed time -- after 4 and 10 hours. In this experiment, instead of fixing the target perplexity and evaluating on training time, we fix the training time and measure the carbon emissions and the achieved perplexity (lower is better). The test perplexity is computed using data from 20 \emph{held-out} clients, to have quick evaluation. Each device has enough examples to have several hundreds of samples for evaluation. Because test perplexity with so few clients is noisy and can vary significantly from round to round, we smooth the test perplexity using an exponentially-weighted moving average with parameter $\alpha=0.3$ and declare that the test perplexity target has been reached when the smoothed test perplexity achieves the target. Asynchronous FL can advance the model faster and reach a lower perplexity at the cost of more carbon footprint. After 10 hours, synchronous FL is able to catch up to asynchronous FL with a similar perplexity of 120. The same contribution ratio among client compute, server compute, upload and download networking costs can be seen here~too.

In the rest of the experiments, we fix the target perplexity while evaluating carbon emissions and training time.

\begin{figure}[t]
\centering
\includegraphics[width=1\columnwidth]{figures/sync-async-fixed-time.pdf}
%\vspace{-0.25cm}
\caption{Carbon emissions of SyncFL and AsyncFL after a fixed training time. The text above the bars shows the perplexity of the model at the specified time (lower is better).}
\label{fig:2}
\end{figure}

\subsection{Some Parameters Matter More}

In our study on hyperparameters in FL tasks, we observed that some parameter choices have a greater impact on carbon footprint than others. Specifically, the parameter of \emph{concurrency} plays a significant role. The relationship between concurrency and carbon emissions in synchronous FL is depicted in Figure~\ref{fig:3}, where we observe that as concurrency increases, so does the carbon footprint. Higher concurrency leads to more devices training simultaneously, resulting in increased resource utilization and only partially offset with potentially faster convergence. We note that the time to reach a target accuracy decreases only up to concurrency of~800, illustrating diminishing returns in training speed. Diminishing returns in training speed as a function of increasing the number of clients training in parallel is analogous to a similar phenomenon in large-batch training~\cite{papaya, bonawitz2019towards, charles2021large}. 


\begin{figure}[b]
\centering
\includegraphics[width=0.85\columnwidth]{figures/sync-upr.pdf}
\vspace{-0.25cm}
\caption{Higher concurrency leads to more carbon emissions. The numbers above bars are the time (in hours) it takes to reach the target accuracy.}
\label{fig:3}
\end{figure}

%  Another parameter is the local epoch on clients. Higher local epoch means more client compute resources which may not translate directly to higher convergence speed. 

\begin{figure*}[t]
\centering
\includegraphics[width=0.95\textwidth]{figures/reg-sync.pdf}
\vspace{-0.25cm}
\caption{Carbon emission of synchronous FL is linearly correlated with the product of rounds it takes to reach a target accuracy and concurrency.}
\label{fig:8}
\end{figure*}


\begin{figure*}[t]
\centering
\includegraphics[width=0.95\textwidth]{figures/reg-async.pdf}
\vspace{-0.25cm}
\caption{Carbon emission of asynchronous FL is linearly correlated with the product of the time it takes to reach a target accuracy and concurrency.}
\label{fig:9}
\end{figure*}

Among the parameters and artifacts of FL system design, we found that concurrency and time to reach a target accuracy, which translates to the number of rounds for synchronous FL and wall-clock time for asynchronous FL, have the most significant impact on carbon emissions. Conversely, other parameters such as learning rates, batch sizes, aggregation goals, and local epochs impact the convergence of the FL model towards the target accuracy, which in turn influences the time required to complete the training process. While these parameters do not directly affect carbon emissions, they do indirectly influence the overall training speed. Therefore, it is recommended that these parameters be included in the ``time'' and ``performance'' aspects of the multidimensional design in Green FL. On the other hand, concurrency impacts both time and carbon emissions directly and should be given greater consideration in FL system design for reducing carbon emissions.

% We found that the optimizer parameters, such as server learning rate, client learning rate, and $b_1$ have the most impact on the training speed for the language modeling FL task. 

Consistent with prior research \cite{papaya, bonawitz2019towards}, the present study indicates that larger values for local epoch do not yield improvements in training efficiency, particularly within the context of non-IID at-scale FL systems characterized by heterogeneous data and systems. On the contrary, larger local epoch values result in a marked increase in carbon emissions, largely attributable to the corresponding rise in client compute. Therefore, we recommend using smaller values for the local epoch, specifically in the 1 to 3 range.

\subsection{Predicting Carbon Emissions of FL}

We put forth a model of the relationship between time-to-convergence, model performance, and carbon emissions. By leveraging our model, FL practitioners can effectively forecast the carbon emissions of their system before initiating the training process.

We observed that concurrency is the most significant determinant of FL's carbon emissions. It has the largest effect on the resources, since concurrency most directly corresponds to the resource utilization of clients. While higher concurrency accelerates model convergence, it results in significantly higher carbon emissions. For instance, increasing concurrency by 10$\times$ increases the resource usage by 10$\times$ while only reducing the convergence time by 1.5$\times$ or 2$\times$. Therefore, the overall benefits of higher concurrency, considering resource consumption, do not scale linearly --- \textbf{increasing concurrency has diminishing returns considering convergence, model performance, and carbon emissions}. Higher concurrency reduces training duration, but increases resource usage even more.

To understand the relationship, we assume that the carbon emissions have a linear relationship with the product of concurrency and the number of rounds (or duration) it takes to reach a target accuracy. We validate our hypothesis in Figures~\ref{fig:8} and~\ref{fig:9}.

Figure~\ref{fig:8} shows the relationship between the product of rounds and concurrency and the carbon emissions for download, upload, and client compute in synchronous FL (carbon emissions of server compute is negligible). Different points on these scatter plots represent different training runs of the language modeling FL task. We use linear regression to find the fitting line that shows the aforementioned relationship. Figure~\ref{fig:8} also shows the $R^2$ values of the models, which is a goodness-of-fit measure for the linear regression models. We can see high $R^2$ values for the linear regression models, confirming \textbf{the product of rounds and concurrency is a good proxy to predict the carbon footprint of synchronous FL}.

Figure \ref{fig:9} shows a similar linear regression model for asynchronous FL. Since there is no concept of rounds in asynchronous FL, we treat duration (hours to reach a target accuracy) as an explanatory variable for the carbon footprint of asynchronous FL. We also see high goodness of fit ($R^2$ values) for these models. Hence, \textbf{the product of duration and concurrency is a good proxy to predict carbon footprint of asynchronous FL}. 

The carbon footprint of an FL is proportional to  concurrency and the number of rounds to convergence (or duration, in asynchronous FL). To estimate the carbon footprint prior to deployment, one needs to know their values as well as the coefficient of proportionality (i.e., the slope of the line in Figures~\ref{fig:8} and~\ref{fig:9}). Concurrency is a hyper-parameter of FL. For estimating rounds (or duration) to convergence, FL practitioners can use FL simulation tools, which is a common practice in the industry. The proportionality coefficient depends on multiple factors, such as the FL task, user population, and the FL infrastructure. In practice, one can estimate this coefficient from a few data points, i.e., by measuring carbon emissions of a task in several settings.

In Figure~\ref{fig:11}, we present an overview of the design space for Green Federated Learning (FL) and highlight the trade-off between time, performance, and carbon emissions in asynchronous FL (the design space for synchronous FL was previously illustrated in Figure~\ref{fig:CO2-FL}). The scatter plot depicts various training runs of the FL task conducted through asynchronous FL, with different marker colors and symbols representing distinct concurrency values. Each point represents an experiment with a different hyper-parameter; we group the points by concurrency. We observe that the points corresponding to the same concurrency follow a linear trajectory, where higher concurrency leads to a steeper slope, implying a faster rate of \carbon accumulation. The cumulative carbon footprint of the task is a function of both its running time and the rate of carbon emission increase. Our analysis identifies concurrency and time to convergence as the two critical parameters for carbon emissions. While the former is under the direct control of the FL engineer, the latter is more indirect and reliant on the appropriate selection of hyperparameters. In particular, the high concurrency regime (which may be desirable, for instance, for its more robust privacy guarantees) puts a higher premium on hyperparameter tuning as longer training time translates into a larger carbon footprint. 

\begin{figure}[t]
\centering
\includegraphics[width=0.95\columnwidth]{figures/predict-async.pdf}
\vspace{-0.25cm}
\caption{Carbon emissions of asynchronous FL increase linearly with the product of the time it takes to reach a target accuracy and concurrency. Each point represents a training run with a different hyper-parameter (grouped by concurrency with marker colors and symbols), its carbon emissions (Y axis, in kg \carbon) and the time it takes to reach a target accuracy (X axis, in hours). The more time is required to reach a target accuracy and the higher the concurrency, the higher is the carbon emission.}
\label{fig:11}
\end{figure}