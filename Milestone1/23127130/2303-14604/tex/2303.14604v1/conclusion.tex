\section{Conclusions and Future Work}

In this paper, we demonstrate how different FL parameters and design choices can impact the carbon footprint of a production FL system. Our empirical approach quantifies carbon emissions by directly profiling a real-world FL task running on millions of user devices. These measurements inform our guidelines and lessons learned on the trade-off between carbon emissions, target accuracy, and time to train in a production FL system.

We acknowledge that this study, like any, has some limitations. Recall that we used the power profiles of the 210 most commonly seen device models to obtain estimates of upload, download, and compute power for typical devices. We noted that these 210 devices represent 20\% of the total devices participating. The carbon emissions values we report based on these values are estimates. Although it is possible that the absolute carbon emissions would change if power profiles for additional devices were available, we believe that the same trends and overall conclusions hold.

As future research directions, we suggest investigating how compression and quantization techniques could apply to Green FL, potentially reducing the carbon footprint of the communication stack (at the expense of increasing client-side computations). Alternative FL architectures, such as secure aggregation via cryptographic computations, federated ensemble learning, or federated split learning, present intriguing challenges as well. Additionally, we encourage the research community to consider the impacts of differential privacy on the landscape of Green FL. Differential privacy would introduce privacy as an additional criterion, alongside accuracy, carbon, and time. Finally, we urge FL practitioners to consider the carbon footprint of their systems in their decision-making process.