\section{Related Work}
To the best of our knowledge, our work is the first to conduct a large-scale carbon emission characterization of an industry-scale FL stack and different hyperparameters. We explore ways FL can be made more energy-efficient (``greener'') through the right selection of FL parameters and design choices. 

There has been growing interest in quantifying and reducing the carbon emission of machine learning (ML) training and inference in the datacenter~\cite{ml_carbon_dodge, ml_carbon_strubell, ml_carbon_patterson, anderson2022treehouse, naidu2021towards}. Nevertheless, the carbon footprint of FL has not yet been explored well. Prior works only quantify the carbon effect of FL in a simulation setting under several simplifying assumptions ~\cite{fl_carbon}. Other studies explore different ways for minimizing the energy footprint of client devices in Federated Learning~\cite{kim2021autofl, refl, kim2022fedgpo}, though not at large-scale scenarios like this study. Another work did a preliminary study of carbon emissions of FL~\cite{wu2022sustainable}; however, the authors also did not do their carbon emissions as comprehensively as our study does, as we log the FL session information and use the actual power measurements of the devices.

We take a data-driven approach to quantify carbon emissions of FL by directly measuring a real-world FL task at scale running on millions of user devices. We present challenges, guidelines, and lessons learned from studying the trade-off between energy efficiency, performance, and time to train in a production FL system.

Other related work could be the works on compression and quantization \cite{Fetchsgd, vogels2019powersgd}. Compressing the communications between the server and the clients could further reduce the carbon emissions of the FL training pipeline while presumably maintaining high model utility. For instance, we observed that the carbon emissions of communication in some settings could contribute to up to 60\% of the total emissions. Hence, reducing them by, say, a factor 4 with \texttt{int8} \cite{prasad2022reconciling} would reduce the total emissions by a factor of $1 / (.4 + 0.6 / 4) = 1.82$. 

% The authors in \cite{guler2021framework} propose a framework for FL training that chooses a user to participate in the training process only when the user has enough energy available.
