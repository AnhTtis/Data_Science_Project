\documentclass[%
 manuscript=article,
 reprint,
 amsmath,amssymb,
 aps,
 prb,
]{revtex4-1}


\usepackage{chemformula} % Formula subscripts using \ch{}
\usepackage[T1]{fontenc} % Use modern font encodings
\usepackage{graphics}
\usepackage{makecell}
\usepackage{xcolor}
\usepackage{amsmath}
\usepackage{enumerate}
\usepackage[normalem]{ulem}
\usepackage{multirow}
\usepackage[bottom]{footmisc}
\usepackage{bm}
\usepackage{siunitx}
\usepackage{multirow}
%%%%%%%%%%%%%%%%%%%%%%%%%%%%%%%%%%%%%%%%%%%%%%%%%%%%%%%%%%%%%%%%%%%%%
% BEGIN DOCUMENT
%%%%%%%%%%%%%%%%%%%%%%%%%%%%%%%%%%%%%%%%%%%%%%%%%%%%%%%%%%%%%%%%%%%%%
\begin{document}
\title{
Supplementary Information: Many-body distribution functionals as compact quantum machine learning
representations} 
\maketitle

\begin{figure*}[htb]
          \centering
          \includegraphics[width=0.8\linewidth]{figs/chair_boat.png}\label{fig:fig3b}
          \caption{DF versions with different number of functionals (Eqs. 11-15) for boat- and chair-conformers of cyclohexane showing effect of inclusion of higher order derivatives and many-body terms on the representation resolution.
          }
 \end{figure*}
 To generate the intermediate geometries we used the Climbing Image-Nudged Elastic Band method (CI-NEB)\cite{ci-neb} as implemented in the Atomic-Simulation-Environment (ASE)\cite{ase} using the ORCA5.0\cite{orca5} calculator.
Reactant and product states were optimized using the L-BFGS algorithm\cite{lbfgs} as implemented in ASE. Subsequently, a CI-NEB calculation was performed using 7 images (9 total geometries) to obtaine a minimal energy path connecting the chair- to the boat-like configuration.
The default convergence criteria ($f_{max}$ = 0.05 eV/Angstrom) and the method PBE0/6-31G* were used in all calulations.

 \begin{figure}[htb]
          \centering           
          \includegraphics[width=\linewidth]{figs/qm7b_atomization.jpeg}
          \caption{MBDF/DF performance and comparison to CM~\cite{CM}, BOB~\cite{bob}, SLATM~\cite{amons_slatm} and FCHL19~\cite{fchl19} representation based atomization energy estimates using the QM7b ($\sim$7k organic molecules with up to 7 heavy atoms) data set~\cite{qm7b}. 
          Training and testing data drawn at random. 
          Prediction mean absolute errors (MAE) are shown as a function of training set size.
          Numbers in legend denote representation size (feature vector dimensions), G and L denote Global and Local kernels respectively.}
     \label{fig:qm7b}
 \end{figure}

\begin{figure*}[ht!]
          \centering           
          \includegraphics[width=0.99\textwidth]{figs/params.pdf}
          \caption{MBDF QML model based hyperparameter scan of local Gaussian kernel length-scale $\sigma$ and noise-level $\lambda$ with Cholesky decomposition for the kernel inversion. Dataset includes 7000 molecules from QM7b with optimal values for length-scale between $\sigma = 0.04$ and $0.32$ at $\lambda = 0.0001$. Train/test-split 6k/1k.}
         
          
     \label{SI:fig:hyperparam}
\end{figure*}

\begin{figure*}[ht!]
          \centering           
          \includegraphics[width=0.99\textwidth]{figs/pcas.pdf}
          \caption{
          Impact of length-scale on kernel principle component analysis (PCA) for the first 400 molecules in QM7b. A scan over different $\sigma$'s using a singular value decomposition (SVD) was performed to obtain training mean absolute errors (MAE) in kcal/mol.  Colour code corresponds to atomization energies. 
          }
     \label{SI:fig:3kernel_pcas}
\end{figure*}

%\begin{figure*}[ht!]
%          \centering           
%          \includegraphics[width=0.99\textwidth]{figs/kernel_pca_50.png}
%          \caption{Zoom into  MBD based kernel (Gaussian) PCA for the first 100 molecules of $\sigma = 0.256$. Sum formulas are shown, colour code corresponds to atomization energy.}
%     \label{SI:fig:single_kernel_pca}
%\end{figure*}




\bibliography{literature}
\end{document}