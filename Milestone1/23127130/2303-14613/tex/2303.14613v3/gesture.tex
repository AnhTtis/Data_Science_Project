\documentclass[acmtog,authorversion]{acmart}

% This file contains all unofficial tweaks to the official template
% That is, main.tex is **minimally** changed (only \import added)

%%
%% additional package imports (never in the main.tex!!!)
%%
\usepackage{overpic}
\usepackage{enumitem} %< control spacing in itemize/enumerate/...
\usepackage{overpic} %< add raw math symbols to figures
\usepackage{color}
\usepackage{tabularx}
% \usepackage{microtype} %< hardcore text layout optimization (ONLY UPDATE ~DEADLINE)
% \usepackage{placeins} %< if you want to use FloatBarriers

%%
%% basic colors
%%
\definecolor{turquoise}{cmyk}{0.65,0,0.1,0.3}
\definecolor{purple}{rgb}{0.65,0,0.65}
\definecolor{dark_green}{rgb}{0, 0.5, 0}
\definecolor{orange}{rgb}{0.8, 0.6, 0.2}
\definecolor{red}{rgb}{0.8, 0.2, 0.2}
\definecolor{darkred}{rgb}{0.6, 0.1, 0.05}
\definecolor{blueish}{rgb}{0.3, 0.3, .6}
\definecolor{light_gray}{rgb}{0.7, 0.7, .7}
\definecolor{pink}{rgb}{1, 0, 1}
\definecolor{greyblue}{rgb}{0.25, 0.25, 1}
\definecolor{awesome}{rgb}{1.0, 0.13, 0.32}

\definecolor{figred}{rgb}{0.9, 0.1, 0.1}
\definecolor{figgreen}{rgb}{0.1, 0.7, 0.1}
\definecolor{figblue}{rgb}{0.1, 0.1, 0.9}
\definecolor{figmagenta}{rgb}{0.8, 0.1, 0.8}

%%
%% basic TODOs
%%
\newcommand{\todo}[1]{{\color{red}#1}}
\newcommand{\TODO}[1]{\textbf{\color{red}[TODO: #1]}}

%% 
%% Inlined comments/edits
%%


% --- Ruoshi Liu (RL)
\newcommand{\rl}[1]{{\color{awesome}#1}} %< I changed something and I want you to see it
\newcommand{\RL}[1]{{\color{awesome}{\bf [R: #1]}}} %< inlined comment for max visibility
\newcommand{\Rl}[1]{\marginpar{\tiny{\textcolor{awesome}{#1}}}} %< useful for ~deadline (no layout changes)

% --- Rundi Wu (RW)
\newcommand{\rw}[1]{{\color{pink}#1}} %< I changed something and I want you to see it
\newcommand{\RW}[1]{{\color{pink}{\bf [W: #1]}}} %< inlined comment for max visibility
\newcommand{\Rw}[1]{\marginpar{\tiny{\textcolor{pink}{#1}}}} %< useful for ~deadline (no layout changes)

% --- Basile Van Hoorick (BV)
\newcommand{\bv}[1]{{\color{blueish}#1}} %< I changed something and I want you to see it
\newcommand{\BV}[1]{{\color{blueish}{\bf [B: #1]}}} %< mainlined comment for max visibility
\newcommand{\Bv}[1]{\marginpar{\tiny{\textcolor{blueish}{#1}}}} %< useful for ~deadline (no layout changes)

% --- Pavel Tokmakov (PT)
\newcommand{\pt}[1]{{\color{turquoise}#1}} %< I changed something and I want you to see it
\newcommand{\PT}[1]{{\color{turquoise}{\bf [P: #1]}}} %< inlined comment for max visibility
\newcommand{\Pt}[1]{\marginpar{\tiny{\textcolor{turquoise}{#1}}}} %< useful for ~deadline (no layout changes)

% --- Sergey Zakharov (SZ)
\newcommand{\sz}[1]{{\color{dark_green}#1}} %< I changed something and I want you to see it
\newcommand{\SZ}[1]{{\color{dark_green}{\bf [S: #1]}}} %< inlined comment for max visibility
\newcommand{\Sz}[1]{\marginpar{\tiny{\textcolor{dark_green}{#1}}}} %< useful for ~deadline (no layout changes)

% --- Carl Vondrick (CV)
\newcommand{\cv}[1]{{\color{darkred}[C says: #1]}} %< I changed something and I want you to see it
\newcommand{\CV}[1]{{\color{darkred}{\bf [C: #1]}}} %< inlined comment for max visibility
\newcommand{\Cv}[1]{\marginpar{\tiny{\textcolor{darkred}{#1}}}} %< useful for ~deadline (no layout changes)


% \usepackage{ulem}
\newcommand{\bvstrike}[1]{{\color{red}[B deletes: #1]}}
\newcommand{\cvstrike}[1]{{\color{red}[C deletes: #1]}}

%% 
%% Circled numbers instead of itemize lists
%%
% i.e. instead of (1) phrase, (2) phrase, ..., and avoids name clash with `\eq{ref}` as (1)
% is often used for Eq.~(1)
\newcommand{\CIRCLE}[1]{\raisebox{.5pt}{\footnotesize \textcircled{\raisebox{-.6pt}{#1}}}}

%%
%% basic math symbols
%%
\DeclareMathOperator*{\argmin}{arg\,min}
\DeclareMathOperator*{\argmax}{arg\,max}
\newcommand{\loss}[1]{\mathcal{L}_\text{#1}}
\newcommand{\expect}{\mathbb{E}}
\newcommand{\real}{\mathbb{R}}

%%
%% shortcuts for standard references
%% 
\newcommand{\Fig}[1]{Fig.~\ref{fig:#1}}
\newcommand{\Figure}[1]{Figure~\ref{fig:#1}}
\newcommand{\Tab}[1]{Tab.~\ref{tab:#1}}
\newcommand{\Table}[1]{Table~\ref{tab:#1}}
\newcommand{\eq}[1]{(\ref{eq:#1})}
\newcommand{\Eq}[1]{Eq.~\ref{eq:#1}}
\newcommand{\Equation}[1]{Equation~\ref{eq:#1}}
\newcommand{\Sec}[1]{Sec.~\ref{sec:#1}}
\newcommand{\Section}[1]{Section~\ref{sec:#1}}
\newcommand{\Appendix}[1]{Appendix~\ref{app:#1}}

%%
%% lorem (i.e. filler latin text)
%% 
\usepackage{blindtext}
\newcommand{\lorem}[1]{\todo{\blindtext[#1]}}

%%
%% paragraph (fine tune spacing close to deadline)
%% 
\renewcommand{\paragraph}[1]{\vspace{1em}\noindent\textbf{#1}}
% \setlength{\parindent}{3pt}%

\usepackage{gensymb}

\usepackage{graphicx}               % Necessary to use \scalebox

% \usepackage{cellspace}
% \setlength\cellspacetoplimit{3pt}
% \setlength\cellspacebottomlimit{3pt}

% https://tex.stackexchange.com/questions/319198/why-is-it-so-difficult-to-generate-a-midrule-dashed-in-latex
% \usepackage{booktabs,arydshln}
% \makeatletter
% \def\adl@drawiv#1#2#3{%
%         \hskip.5\tabcolsep
%         \xleaders#3{#2.5\@tempdimb #1{1}#2.5\@tempdimb}%
%                 #2\z@ plus1fil minus1fil\relax
%         \hskip.5\tabcolsep}
% \newcommand{\cdashlinelr}[1]{%
%   \noalign{\vskip\aboverulesep
%           \global\let\@dashdrawstore\adl@draw
%           \global\let\adl@draw\adl@drawiv}
%   \cdashline{#1}
%   \noalign{\global\let\adl@draw\@dashdrawstore
%           \vskip\belowrulesep}}
% \makeatother

% https://tex.stackexchange.com/questions/26360/how-to-color-the-font-of-a-single-row-in-a-table
\usepackage{tabu}
\usepackage{xcolor}

% \usepackage{setspace}
% \setstretch{1.5}

\input{sections/_template_args.tex}

%%
%% end of the preamble, start of the body of the document source.
\begin{document}

%%
%% The "title" command has an optional parameter,
%% allowing the author to define a "short title" to be used in page headers.
\title{GestureDiffuCLIP: Gesture Diffusion Model with CLIP Latents}

% Authors.
%%
%% The "author" command and its associated commands are used to define
%% the authors and their affiliations.
%% Of note is the shared affiliation of the first two authors, and the
%% "authornote" and "authornotemark" commands
%% used to denote shared contribution to the research.
\author{Tenglong Ao}
%\authornote{Both authors contributed equally to this research.}
\email{aubrey.tenglong.ao@gmail.com}
%\orcid{1234-5678-9012}
\affiliation{%
  \institution{Peking University}
  \streetaddress{No.5 Yiheyuan Road, Haidian District}
  \city{Beijing}
  \state{Beijing}
  \country{China}
  \postcode{100871}
}

\author{Zeyi Zhang}
\email{carpesu@stu.pku.edu.cn}
\affiliation{%
  \institution{Peking University}
  \streetaddress{No.5 Yiheyuan Road, Haidian District}
  \city{Beijing}
  \state{Beijing}
  \country{China}
  \postcode{100871}
}

\author{Libin Liu}
\authornote{corresponding author}
\email{libin.liu@pku.edu.cn}
\affiliation{%
  \institution{Peking University \& National Key Lab of Genaral AI}
  \streetaddress{No.5 Yiheyuan Road, Haidian District}
  \city{Beijing}
  \state{Beijing}
  \country{China}
  \postcode{100871}
}

% This command defines the author string for running heads.
%%
%% By default, the full list of authors will be used in the page
%% headers. Often, this list is too long, and will overlap
%% other information printed in the page headers. This command allows
%% the author to define a more concise list
%% of authors' names for this purpose.
% \renewcommand{\shortauthors}{DeJohnette, Rowland-Smith, Badeeri, and Foyt}
\renewcommand{\shortauthors}{Ao, Zhang, and Liu}


%%
%% The abstract is a short summary of the work to be presented in the
%% article.
\begin{abstract}
The current study investigated possible human-robot kinaesthetic interaction using a variational recurrent neural network model, called PV-RNN, which is based on the free energy principle.
Our prior robotic studies using PV-RNN showed that the nature of interactions between top-down expectation and bottom-up inference is strongly affected by a parameter, called the meta-prior, which regulates the complexity term in free energy.
% The current study examines how the behaviours of robots alter by changing the meta-prior $w$ in human-robot kinaesthetic interaction.
The current study examines how changing the meta-prior $w$ in the interaction phase affects the counter force generated when an experimenter attempts to induce movement pattern transitions familiar to the robot through its prior training.
The study also compares the counter force generated when trained transitions are induced by a human experimenter and when untrained transitions are induced.
Our experimental results indicated that (1) the human experimenter needs more/less force to induce trained transitions when $w$ is set with larger/smaller values, (2) the human experimenter needs more force to act on the robot when he attempts to induce untrained as opposed to trained movement pattern transitions.
Our analysis of time development of essential variables and values in PV-RNN during bodily interaction clarified the mechanism by which gaps in actional intentions between the human experimenter and the robot can be manifested as reaction forces between them.


%% Hiroki writing 2022-11-4
%Current study investigates the dynamics of the latent states during human-robot kinaesthetic interaction using PV-RNN.
%We have achieved to observe and analyse the internal state of an RNN model based on the free energy principle, during real-time human-robot interaction.
%Essential characteristics observed in the previous study of this variational recurrent neural network model, PV-RNN, is that by changing a meta prior $w$, the balance between the top-down intention and the bottom-up perceptual reality changes.
%In the current study, we examined how changing the weighting parameter $w$ between accuracy and complexity in free energy principle affects the humanoid robot's behaviour through human-robot interaction. We have conducted some human-robot kinaesthetic interaction experiments with various $w$ and quantitatively analysed the latent variable and the force applied to the humanoid robot. We have observed that the force required to change the robot's intention has increased, both when the top-down intention was strengthened by changing the $w$ and when corresponding switch of its primitive was against the experience of the RNN during its training. The study confirms through quantitative analysis that by increasing or decreasing the $w$ in PV-RNN, humanoid robot leads or follows the human counterpart during the human-robot kinaesthetic interaction.

\begin{comment}
Comment from Jun #2
・最後にQualitativeな結果(インパクト)が欲しい
・Current study investigates the problem on~と書き出すのが一般的
・最初の一文と最後の一文を対応させる
・最後の一文はもう少しAbstractかつ包括的に
\end{comment}

\begin{comment}
Comment from Jun #1
We investigated how the kinaesthetic human-robot interaction can affect the internal state of a model based on the free energy principle. 
=> how the internal state is affected is not the most important point in this study. This part should be rewritten.

The key function of this variational recurrent neural network model, PV-RNN, is that by changing a meta prior $w$, it takes a balance between the "complexity” term and the ”accuracy” term which corresponds to a top-down intention and a bottom-up perceptual reality in the free energy principle, respectively. 
=> This is not key function of PV-RNN. It is an essential characteristics observed in the previous study. The grammar after $w$ is something strange. Rewrite these.

This research has conducted a human-robot interaction experiment with a robotic agent in a kinaesthetic sense.
=> The sentence is not good. "in a kinaesthetic sense" is grammatically wrong.
MODIFIED => "In the current study human-robot interaction experiments using the kinaesthetic sense were conducted."

We investigated that when human forces the agent to switch primitives from one to another, larger force was required both when the human intention is conflictive against the top-down the intention of the agent and when the agent has a stronger top-down intention by modifying the $w$.
=> You should write the essential results of the experiments rather than what we investigated and also how these results could contribute to the studies on human-robot interaction.
\end{comment}

\end{abstract}
%%
%% The code below is generated by the tool at http://dl.acm.org/ccs.cfm.
%% Please copy and paste the code instead of the example below.
%%
\begin{CCSXML}
<ccs2012>
   <concept>
       <concept_id>10010147.10010371.10010352</concept_id>
       <concept_desc>Computing methodologies~Animation</concept_desc>
       <concept_significance>500</concept_significance>
       </concept>
   <concept>
       <concept_id>10010147.10010257.10010293.10010294</concept_id>
       <concept_desc>Computing methodologies~Neural networks</concept_desc>
       <concept_significance>300</concept_significance>
       </concept>
 </ccs2012>
\end{CCSXML}

\ccsdesc[500]{Computing methodologies~Animation}
\ccsdesc[300]{Computing methodologies~Neural networks}

%%
%% Keywords. The author(s) should pick words that accurately describe
%% the work being presented. Separate the keywords with commas.
\keywords{Music-driven dance synthesis, multi-modality}

%% A "teaser" image appears between the author and affiliation
%% information and the body of the document, and typically spans the
%% page.
\begin{teaserfigure}
  \centering
  \includegraphics[width=0.9\textwidth]{figures/teaser.pdf}
  \caption{Stylized gestures synthesized by our system for the same speech clip conditioned on four different text prompts.}
  \Description{}
  \label{fig:teaser}
\end{teaserfigure}

%%
%% This command processes the author and affiliation and title
%% information and builds the first part of the formatted document.
\maketitle

% \begin{figure}[t]
%     % \begin{subfigure}{1\linewidth}
%     %   \centering
%     % %   \includegraphics[width=1\linewidth]{figs/fig_1_moti_textattn.pdf}  
%     % %   \includegraphics[width=1\linewidth]{figs/fig_1_moti_textattn_v2.pdf}  
%     %   \includegraphics[width=1\linewidth]{figs/fig_1_moti_textattn_v5.pdf}  
%     %   \vspace{-0.5cm}
%     %     \caption{Amount of attention added to each video clip from the source video and query text in the self-attention layers of Moment-DETR encoder.}
%     %     % \caption{Distribution of attention for source and query in Moment-DETR encoder}
%     %     % Visualization of video clip's self-attention score in Moment-DETR encoder.
%     %   \label{fig:fig1_text_attn_ex}
%     % \end{subfigure}%\hfill% or  or \hspace{0.3\textwidth}
%     \vspace{0.2cm}
%     % \begin{subfigure}{1\linewidth}
%       \centering
%     %   \includegraphics[width=1\linewidth]{figs/fig1_moti_negattn.pdf}  
%       \includegraphics[width=1\linewidth]{figs/fig1_moti_negattn_v3.pdf}  
%       \vspace{-0.4cm}
%     %   \caption{Correspondence of saliency scores on the relevance between video clips and the text query.}
%     % \caption{Predicted saliency scores against the video relevant positive query and video irrelevant negative query}
%       \label{fig:fig1_neg_attn_ex}
%     % \end{subfigure}%\hfill% or  or \hspace{0.3\textwidth}
%     \caption{
%     % 원준 원본
%     % (a) Comparison between attention scores of source and query for each video clip~(We sum the attention scores from video and text). 
%     % We observe that the attention scores are dominated by other clips in the source video. 
%     % Text queries do not account for much attention regardless of the relevance to the video clips.
%     % \textbf{(a)} Inspection of the query dependency in Moment-DETR encoder.
%     % % We visualize the attention score of video tokens in the transformer encoder and observe that text query accounts for only a low portion of attention.
%     % % This tendency occurs regardless of the relevance between the text query and video clips. 
%     % We visualize the attention score of video tokens in the transformer encoder and observe 1) text query only accounts for a low portion of attention, and 2) relevance between video-query pair does not affect the attention scores ratio of text.
%     \textbf{(b)} Comparison of highlight-ness when relevant and non-relevant queries are input.
%     As observed in , existing work only uses queries to play an insignificant role, thereby may not be capable of detecting false queries and considering the video-query relevance even when the problem in (a) is resolved. 
%     % \SE{} % 이 부분이 "not capable of" 란 용어가 세다는 피드백이 있는 듯 합니다. 이러한 능력이 없다는 것은 굉장히 강한 어조인거 같기는 하고, 이러한 경우들이 종종 있다거나 좀 약화시킬 필요가 있어보이긴 하네요.
%     On the other hand, our QD-DETR yields a query-dependent representation that the relevance between the source video and query text is updated in the saliency scores.
%     There is a large gap between positive and negative saliency scores, and scores are consistent since the clips are all highly correlated to others.
%     }
%     \label{fig:motivation_ex}
%     % \captionsetup{belowskip=13pt}
%     % \setlength{\belowcaptionskip}{-10pt}
% \end{figure}
\begin{figure}
    \centering
    \includegraphics[width=1\linewidth]{figs/fig1_moti_negattn_1111.pdf}
    % \includegraphics[width=1\linewidth]{figs/fig1_moti_negattn_1109.pdf}
    % \includegraphics[width=1\linewidth]{figs/fig1_moti_negattn_stat.pdf}
    \vspace{-0.6cm}
    \caption{
        % \SE{} % 수정 필요
        Comparison of highlight-ness~(saliency score) when relevant and non-relevant queries are given.
        We found that the existing work only uses queries to play an insignificant role, thereby may not be capable of detecting negative queries and video-query relevance; saliency scores for clips in ground-truth~(GT) moments are low and equivalent for positive and negative queries.
        % This also results in mispredicted moments when ground-truth~(GT) moment is dominated by clips unrelated to GT since their prediction is highly focused on the video.
        % \SE{} % 여기 한번 더 보면 좋을 듯 합니다. GT moment에 unrelated한 clip이 많으면? label이 틀렷을 경우를 말씀하시는건지?
        % As observed in saliency graph, existing work only uses queries to play an insignificant role, thereby may not be capable of detecting false queries and considering the video-query relevance.
        On the other hand, query-dependent representations of QD-DETR result in corresponding saliency scores to the video-query relevance and precisely localized moments.
        % On the other hand, our QD-DETR yields a query-dependent representation that the
        % saliency scores are in accordance with the relevance between the video and query.
        % text is in accordance with the saliency scores.
        % There is a large gap between positive and negative saliency scores, and scores are consistent since the clips are all highly correlated to others.
}
    \label{fig:motivation_ex}
\end{figure}


\section{Introduction}
% 원준 원본
% Along with the advance of digital devices and platforms, video is now one of the most desired data type for consumers. However, although the large information capacity of videos may be beneficial in many aspects, e.g., informative and entertaining, on the contrary perspective, videos are time-consuming, and hard to search for desirable moments. 
% This has led many creators to use extra manpower to crop and edit the video to generate highlight clips to gain the consumer’s attention.
Along with the advance of digital devices and platforms, video is now one of the most desired data types for consumers~\cite{apostolidis2021video,wu2017deep}.
% SE: Video aware deep learning application & survey papers?
Although the large information capacity of videos might be beneficial in many aspects, e.g., informative and entertaining, inspecting the videos is time-consuming, so that it is hard to capture the desired moments~\cite{anne2017localizing,apostolidis2021video}. 
% This has led many creators to use extra manpower to crop and edit the video to generate highlight clips to gain the consumer’s attention.


% On the other side, 
Indeed, the need to retrieve user-requested or highlight moments within videos is greatly raised.
Numerous research efforts were put into the search for the requested moments in the video~\cite{anne2017localizing, gao2017tall, liu2015multi, escorcia2019temporal} and summarizing the video highlights~\cite{zhang2016video, mahasseni2017unsupervised, badamdorj2022contrastive, wei2022learning}.
% Numerous research efforts were put into the search for the requested moments in the video~\cite{anne2017localizing, gao2017tall, liu2015multi, escorcia2019temporal}, summarizing the video to generate highlights was another popular topic~\cite{zhang2016video, mahasseni2017unsupervised, badamdorj2022contrastive, wei2022learning}.
Recently, Moment-DETR~\cite{momentdetr} further spotlighted the topic by proposing a QVHighlights dataset that enables the model to perform both tasks, retrieving the moments with their highlight-ness, simultaneously.

% 원준 원본
% To detect the desired moments, previous works employed transformer encoder-decoder architectural designs to fuse the text query into the video representations. Moment-DETR~\cite{mDETR} modified detection transformer to process capture the moment as a set, and UMT~\cite{umt} implemented transformer decoder as to output clip-wise saliency. 
% Yet to their outstanding breakthroughs in the literature of moment retrieval with the seminal architectures, their limitation is that the role of the given text query is insignificant in representing the query-conditioned video representation; the attention mechanism of moment DETR is not explicitly conditioned on the text query, and the text query is conditioned on multi-modal clips where the differences between the clips are smoothed after encoding process in UMT.



% \begin{figure}[t]
% \centering
%     \begin{subfigure}[l]{0.37\linewidth}
%       \centering
%       \vspace{0.20cm}
%     %   \includegraphics[width=1\linewidth]{figs/fig_1_moti_textattn.pdf}  
%     %   \includegraphics[width=1\linewidth]{figs/fig_1_moti_textattn_v2.pdf}  
%       \includegraphics[width=1\linewidth]{figs/fig1_moti_violin_a.pdf}  
%       \vspace{-0.60cm}
%     %   \caption{text attention}
%         \caption{Importance of queries in video representation}
%       \label{fig:fig1_text_attn}
%     \end{subfigure}%\hfill% or  or \hspace{0.3\textwidth}
%     \vspace{0.2cm}
%     \begin{subfigure}[r]{0.61\linewidth}
%       \centering
%     %   \includegraphics[width=1\linewidth]{figs/fig1_moti_negattn.pdf}  
%       \includegraphics[width=1\linewidth]{figs/fig1_moti_violin_b.pdf}  
%     %   \caption{neg attention}
%         % \caption{Relation between the highlight-ness and the relevance between videos and query texts.}
%         \caption{Highlight-ness~(saliency) histogram of positive and negative video-query pairs\SE{}}
%       \label{fig:fig1_neg_attn}
%     \end{subfigure}%\hfill% or  or \hspace{0.3\textwidth}
%     % \vspace{-0.2cm}
%     \caption{Overall statistics for attention scores in Fig.~\ref{fig:motivation_ex} in QVHighlights dataset. 
%     (a) For the attention scores that measure how much the text query is generally involved in video representation, we use violin plots to show the probability density. We plot the score for each layer in the encoder.
%     % (b) Using the histogram, we compare how the baseline and QD-DETR yield different salient scores given the positive and negative video-text pairs.
%     (b) Saliency histogram shows the distributional gap between positive and negative video-text query pairs of baseline~(Moment-DETR) and proposed QD-DETR.\SE{}
%     }
%     \label{fig:motivation}
%     % \captionsetup{belowskip=13pt}
%     % \setlength{\belowcaptionskip}{-10pt}
% \end{figure}

% \begin{figure}[t]
% \centering

%     \begin{subfigure}[r]{1\linewidth}
%       \centering
%       \hspace{-0.2cm}
%     %   \includegraphics[width=1\linewidth]{figs/fig1_moti_negattn.pdf}  
%       \includegraphics[width=1.1\linewidth]{figs/fig1_moti_violin_a_v2.pdf}  
%     %   \caption{neg attention}
%         % \caption{Relation between the highlight-ness and the relevance between videos and query texts.}
%         \vspace{-0.5cm}
%         % \caption{Saliency histogram of positive and negative video-query pairs}
%         \caption{We plot the histograms and its average value~(dotted line) to compare saliency scores when true and false text queries are given for each method. (left) Since the video representations do not include much textual information, both the true and false queries yield similar saliency scores. (Middle) Even when the video representation is enforced to be updated with the textual information, the issue is not much resolved. (Right) By extracting discriminative features in the text query, distributions are differentiated.
%         % \SE{} % R1@0.5 설명
%         Also, R1@0.5 indicates evaluation metric, Recall at 1 with IoU 0.5 threshold on QVhighlight \textit{val} set.
%         }
%       \label{fig:fig1_neg_attn}
%     \end{subfigure}%\hfill% or  or \hspace{0.3\textwidth}
%     \\
%     \begin{tabular}{cc}
%     \hspace{-0.2cm}
%         \begin{minipage}{.4\linewidth}
%             \begin{subfigure}[l]{1\linewidth}
%               \centering
%             %   \vspace{0.20cm}
%             %   \includegraphics[width=1\linewidth]{figs/fig_1_moti_textattn.pdf}  
%             %   \includegraphics[width=1\linewidth]{figs/fig_1_moti_textattn_v2.pdf}  
%               \includegraphics[width=1\linewidth]{figs/fig1_moti_violin_a.pdf}  
%               \vspace{-0.60cm}
%             %   \caption{text attention}
%                 \caption{Importance of queries in video representation}
%               \label{fig:fig1_text_attn}
%             \end{subfigure}%\hfill% or  or \hspace{0.3\textwidth}
%         \end{minipage}
        
%         \begin{minipage}{.6\linewidth}
%             \vspace{-0.2cm}
%             \caption{Overall statistics of Fig.~\ref{fig:motivation_ex} in QVHighlights dataset. 
%             (a) Saliency histogram shows the distributional gap between positive and negative video-text query pairs.
%             % (a) For the attention scores that measure how much the text query is generally involved in video representation, we use violin plots to show the probability density. We plot the score for each layer in the encoder.
%             % (b) Using the histogram, we compare how the baseline and QD-DETR yield different salient scores given the positive and negative video-text pairs.
%             % (b) Text ratio in self-attention layer to  of Moment-DETR
%             % (b) Ratio of text when representing video tokens in self-attention of Moment-DETR.
%             % (b) Magnitude of attention text query involved.
%             % (b) Attention score of video tokens
%             % (b) Magnitude of text query to refine the video tokens in self-attention layer of Moment-DETR.
%             (b) Probability density depicting the weight of the text query in attention score for video clips. Scores are from the self-attention layers in Moment-DETR encoder.
%             % (b) The text query ratio in attention score of video clips (Self-attention layer in Moment-DETR encoder). We use violin plots to show probability density.
%             % 텍스트 쿼리가, 비디오 피쳐에 얼만큼 attend 하는지
%             }
%         \end{minipage}
    
%     \end{tabular}
%     \vspace{-0.5cm}
%     \label{fig:moti}
%     % \captionsetup{belowskip=13pt}
%     % \setlength{\belowcaptionskip}{-10pt}
% \end{figure}


% \begin{figure}
%     \centering
%     % \includegraphics[width=1\linewidth]{figs/fig1_moti_negattn_1109.pdf}
%     \includegraphics[width=1\linewidth]{figs/fig1_moti_negattn_stat_v2.pdf}
%     \vspace{-0.8cm}
%     \caption{
%         Histogram of saliency when the positive and negative queries are given. We plot the histograms and its average value~(dotted line) to compare saliency scores when relevant~(positive) and irrelevant~(negative) text queries are given for each method. (Left) Since the video representations do not properly reflect textual information, both the positive and negative queries yield similar saliency scores. 
%         % (Middle) Even when the video representation is enforced to be updated with the textual information, the issue is not much resolved. 
%         (Right) By representing video clips in query-dependent manner, distributions are differentiated.
%     }
%     \vspace{-0.6cm}
%     \label{fig:motivation}
% \end{figure}


% One of the demanding task is moment retrieval task, which is detecting the desired moments from the given query, typically the text query.
When describing the moment, one of the most favored types of query is the natural language sentence~(text)\cite{anne2017localizing}. 
While early methods utilized convolution networks~\cite{zhang2020learning, gao2021fast, wang2020temporally}, recent approaches have shown that deploying the attention mechanism of transformer architecture is more effective to fuse the text query into the video representation.
% To handle these modalities, previous works simply employed the attention mechanism of transformer architecture to fuse the text query into the video representation.
For example, Moment-DETR~\cite{momentdetr} introduced the transformer architecture which processes both text and video tokens as input by modifying the detection transformer~(DETR), and UMT~\cite{umt} proposed transformer architectures to take multi-modal sources, e.g., video and audio. 
Also, they utilized the text queries in the transformer decoder.
Although they brought breakthroughs in the field of MR/HD with seminal architectures, they overlooked the role of the text query.
To validate our claim, we investigate the Moment-DETR~\cite{momentdetr} in terms of the impact of text query in MR/HD~(Fig.\ref{fig:motivation_ex}).
Given the video clips with a relevant positive query and an irrelevant negative query, we observe that the baseline often neglects the given text query when estimating the query-relevance scores, i.e., saliency scores, for each video clip.
% the output saliency score, i.e. query-relevance scores.
% Based on the observation, we traced the actual saliency prediction of the model against both the video-relevant query and the irrelevant dummy one where we find that the baseline often neglects the given text query when estimating the query-relevance scores of video clips.
% For example, in Fig.~\ref{fig:motivation_ex}, saliency scores are not affected even when the query is substituted with the dummy.
% % General statistics for Fig.~\ref{fig:motivation_ex} is shown in Fig.~\ref{fig:motivation}. 
% General statistics corresponding to Fig.~\ref{fig:motivation_ex} are also shown in Fig.~\ref{fig:motivation}.



% The limitation of the concrete baseline~\cite{momentdetr} is inspected in two different aspects; 1) Utilization of text-query in the encoding process and 2) the output saliency score, i.e. query-relevance scores.
% Firstly, we visualize the attention score when video clips are given as a query in self-attention. 
% We observe that the text queries have relatively small impacts compared to other video features, as shown in Fig.~\ref{fig:fig1_text_attn_ex}.
% That is, the text does not account for much in representing every video clip, although the goal of MR/HD is to detect query-relevant moments.
% Based on the observation, we traced the actual saliency prediction of the model against both the video-relevant query and the irrelevant dummy one where we find that the baseline often neglects the given text query when estimating the query-relevance scores of video clips.
% For example, in Fig.~\ref{fig:motivation_ex}, saliency scores are not affected even when the query is substituted with the dummy.
% % General statistics for Fig.~\ref{fig:motivation_ex} is shown in Fig.~\ref{fig:motivation}. 
% General statistics are also shown in Fig.~\ref{fig:motivation}.

% Consequently, in Fig.~\ref{fig:fig1_neg_attn_ex}~(b), we found that the baseline often neglects the given text query when estimating the query-relevance scores of video clips; 
% For example, 


% We validate the previous work sometimes neglects the given query when estimating the saliency of video clips.
% For example, there is an example that the saliency scores from positive and negative queries cannot be distinguishable, as shown in Fig.~\ref{fig:fig1_neg_attn_ex}.
% % 우리는 추가로 text attention을 추가도 해봤지만, 효과가 있긴 했으나, still 이슈가 있는 것을 확인하였다?
% % Still, we observe that assuring the high attendance of text queries does not resolve the overlap which motivates us to question the quality of the naive use of task-agnostic text representation~\cite{momentdetr, umt}.
% We found that introducing the text-attention for ensuring the high attendance of text queries relieve the overlap, but there still be a severe overlap.


% To validate their limitations, we inspect the impacts of text queries in the concrete baseline~\cite{momentdetr} with the two different aspects, 1) tendency of attention in self-attention layer and 2) saliency score, i.e. query-relevance scores. \SE{} % attention 이 갑자기 등장하는가?
% Firstly, we visualize the attention score when video clips are given as a query in self-attention. We observe the text queries have relatively low attention scores compared to the video features, as shown in Fig.~\ref{fig:fig1_text_attn_ex}.
% That is, the text does not account for much in representing every video clip, although the goal of MR/HD is to detect query-relevant moments.
% Based on this observation, we trace the actual saliency prediction of the model against both positive and negative text queries.
% We validate the previous work sometimes neglects the given query when estimating the saliency of video clips.
% For example, there is an example that the saliency scores from positive and negative queries cannot be distinguishable, as shown in Fig.~\ref{fig:fig1_neg_attn_ex}.
% % 우리는 추가로 text attention을 추가도 해봤지만, 효과가 있긴 했으나, still 이슈가 있는 것을 확인하였다?
% % Still, we observe that assuring the high attendance of text queries does not resolve the overlap which motivates us to question the quality of the naive use of task-agnostic text representation~\cite{momentdetr, umt}.
% We found that introducing the text-attention for ensuring the high attendance of text queries relieve the overlap, but there still be a severe overlap.



% Thus, we 
% query dependency를 높이기 위해 
% Cross-attention? text-attention? detailed explanation on text-attention should be needed?
% By handling these two issues, we find that more precise retrieval can be achieved.
% 
% 
%
% By projecting video-discriminative text features with high text attendance to source video, we f 
% We also find the need to improve the quality of query features since assuring high text attendance also results in...
% pairs are not finetuned to be discriminative that even the similarity within the pairs does not reflect the relevance between the query and the video clips.
% General statistics for Fig.~\ref{fig:motivation_ex} is shown in Fig.~\ref{fig:motivation}. 
% \SE{} % 이거 ??로 뜨는데, 위처럼 figure 그리면 label이 안되는걸까요
% \SE{}
% 형님 아래 사항 생각 좀 해보는게 좋을 거 같아요.
% fig 1. (a) 그림만 봤을 때 모든 clip에 대해 text attention이 일정이상 존재하긴 하니까, 뭔가 not assured to be conditioned가 와닿지 않는거 같아요.
% + 왜 text가 항상 attend 해야하나?
% not assured to be conditioned --> text shows relatively low affects compared to video 같이 실제 나타난 현상까지 같이 적으면 어떨까 싶어요.
% fig 1. (b) 덜 반영한다?

% \SU{}
% 일단 text가 attend 잘 되어야 한다는 것에 좀 궁금점이 생깁니다. 결국에는 text와 관련있는 frame들을 attend해서 higlight를 찾아야 하는게 아닐까요? 그리고, 현제 저희의 모델 구조상 text query가 Key와 Value로 거의 활용되고 있는데 그렇다면 결국에는 해당 모델은 text에 대한 attention이 전혀 없다고 봐도 무방하지 않을까요? 그런 면에서 text attention을 강조하는게 좀 걸리긴 합니다.

% Specifically, the text query is not assured to be explicitly conditioned on every clip of the video, and as the query texts are evenly treated, discriminative keywords may not be spotlighted.
% attention mechanism of Moment-DETR is not explicitly conditioned on the text query as shown in Fig~\ref{}(d), and in UMT, the text are only used for conditioning the queries while the video representation are refined itself by self-attention.

% \begin{figure}[t]
%     \begin{subfigure}{1\linewidth}
%       \centering
%     %   \includegraphics[width=1\linewidth]{figs/fig_1_moti_textattn.pdf}  
%     %   \includegraphics[width=1\linewidth]{figs/fig_1_moti_textattn_v2.pdf}  
%       \includegraphics[width=1\linewidth]{figs/fig_1_moti_textattn_v4.pdf}  
%       \vspace{-0.5cm}
%     %   \caption{text attention}
%         \caption{Distribution of attention scores in Moment-DETR encoder}
%       \label{fig:fig1_text_attn}
%     \end{subfigure}%\hfill% or  or \hspace{0.3\textwidth}
%     \vspace{0.2cm}
%     \begin{subfigure}{1\linewidth}
%       \centering
%     %   \includegraphics[width=1\linewidth]{figs/fig1_moti_negattn.pdf}  
%       \includegraphics[width=1\linewidth]{figs/fig1_moti_negattn_v2.pdf}  
%       \vspace{-0.5cm}
%     %   \caption{neg attention}
%         \caption{Saliency score against positive and negative text queries}
%       \label{fig:fig1_neg_attn}
%     \end{subfigure}%\hfill% or  or \hspace{0.3\textwidth}
%     \vspace{0.2cm}
%     \begin{subfigure}{1\linewidth}
%       \centering
%     %   \includegraphics[width=1\linewidth]{figs/fig1_moti_violin.pdf}  
%       \includegraphics[width=1\linewidth]{figs/fig1_moti_violin_v2.pdf}  
%       \vspace{-0.5cm}
%       \caption{violin}
%       \label{fig:fig1_violin}
%     \end{subfigure}%\hfill% or  or \hspace{0.3\textwidth}
%     \vspace{-0.2cm}
%     \caption{(a) 1. portion of text attention vs. video attention 2. relation with text query and content (e.g. fg, bg) of clip seems not to affect the attention score
%     (b) 1. high variability even though entire clips are highly correlated with the given text query 2. positive and negative query makes overlaps on saliency score distribution
%     (3) actual distribution on validation dataset.}
%     \label{fig:motivation}
%     % \captionsetup{belowskip=13pt}
%     % \setlength{\belowcaptionskip}{-10pt}
% \end{figure}

To this end, we propose Query-Dependent DETR~(QD-DETR) that produces query-dependent video representation.
% Our key focus is to ensure each clip in predicted moments is explicitly conditioned by the query, particularly on the video-descriptive portion of the text query.
% Our key focus is to ensure that query-relevant clips are predicted by enforcing each clip to be explicitly conditioned by the query.
%Our key focus is to ensure that the model prediction for each clip is highly relevant to the query.
Our key focus is to ensure that the model's prediction for each clip is highly dependent on the query.
% by enforcing each clip to be explicitly conditioned by the query. :)
% hmm...
% \SE {} % "query-relevant clips are predicted" 이 문장이 좀 애매한거 같습니다. relevant 클립을 놓지지 않고 찾는 것을 보장한다? 이런 느낌인지 아니면 높은 saliency 를 주는게 목적이다? model prediction이 query-relevance를 반영하는 것을 보장한다?
% Our key focus is to ensure that the model prediction reflects query-relevance of clips by enforcing each clip to be explicitly conditioned by the query.
First, to fully utilize the contextual information in the query, we revise the transformer encoder to be equipped with cross-attention layers at the very first layers.
% 상익's thought :  single video - query간의 관계만 고려 - 같은 word가 더 많이 쓰이는 것을 보고 
% 교수님's thought : neg pair 를 쓰면 쿼리를 보지 않고서는 video clip간만 고려하는 것이 사라짐. 왜냐면 0으로 내보내야 하기 때문. --> SE: relative difference 만 고려하다가, 
By inserting a video as the query and a text as the key and value of the cross-attention layers, our encoder enforces the engagement of the text query in extracting video representation.
% 원준 교수님 코멘트 반영해서 다시
Then, in order to not only inject a lot of textual information into the video feature but also make it fully exploited, we leverage the negative video-query pairs generated by mixing the original pairs.
Specifically, the model is learned to suppress the saliency scores of such  negative~(irrelevant) pairs.
Our expectation is the increased contribution of the text query in prediction since the videos will be sometimes required to yield high saliency scores and sometimes low ones depending on whether the text query is relevant or not.
% \SE{}
% learns to?
% By suppressing the saliency scores of the irrelevant video-query pairs, the model learns to spotlight only the video-specific discriminative words in the query.
% % \SE{} % ====================== 상익 수정 ========================
% However, this architectural design still lacks the capability of identifying the video-descriptive keywords in the query.
% % However, this architectural design still lacks in identifying proper query relevance.
% This is because the current training scheme only focuses on the interactions of video and clips within a single video while neglecting information shared throughout the entire video.
% % We argue the problem of the current training scheme that only focuses on distinguishing the clips in a single video while neglecting information shared throughout the entire video.
% Therefore, we leverage the negative video-query relationships to enhance the capability of identifying the contextual similarity of query and video clips.
% 
% 원준 원본 
% However, this architectural design heavily relies on the quality of the text query.
% Therefore, we leverage the negative video-query relationships to enable the model to emphasize key corresponding query features.
% By suppressing the saliency scores of the irrelevant video-query pairs, the model learns to spotlight only the video-specific discriminative words in the query.
% =========================================================
Lastly, to apply the dynamic criterion to mark highlights for each instance, we deploy a saliency token to represent the entire video and utilize it as an input-adaptive saliency criterion. 
With all components combined, our QD-DETR produces query-dependent video representation by integrating source and query modalities.
This further allows the use of positional queries~\cite{dabdetr} in the transformer decoder.
% Furthermore, we can exploit the advanced DETR decoder architectures using the positional information, e.g., DAB-DETR, since our encoded tokens consist of identical position representations from a single modality.
% \SE{} % ====================== 상익 수정 ========================
% Furthermore, we can exploit the advanced DETR decoder architectures using the positional information, e.g., DAB-DETR, since our video clip tokens consist of identical position representations from a single modality.
% 원준 원본
% It also enables the use of advanced DETR decoder architectures, e.g., DAB-DETR, for the first time, as these works exploit the position information within a single modality.
% =========================================================
Overall, our superior performances over the existing approaches validate the significance of the role of text query for MR/HD.
% Our extensive experiments on QVHighlights, TVSum, and Charades-STA datasets validate the significance of considering the role and the quality of text query.

% All components combined with dynamic anchor moments for the query of decoder, our FOQUE fosters the query-dependent video representation, thereby making the 
% All components combined, our modified transformer encoding process fosters the query-dependent video representation thereby achieving the state-of-the-art results on various benchmarks of moment-retrieval and highlight detection.
	
% -	Video Platform & Streamer & Consumer의 증가. 
% Video는 다른 데이터 타입보다 정보가 많아 유용하지만, 이는 다른 말로 해석하면 video를 보는 것은 time-consuming 하고, 원하는 것을 찾아보기에는 힘들 수 있음.
% 따라서, 많은 매체에서는 사람들의 더 많은 이목을 끌기 위해 highlight 비디오라는 것을 편집하여 공유도 함.
% 하지만, highlight video를 만들기 위해 사람의 노력이 필요한 현 시점에서, This spotlights the need to retrieve the user-requested / Highlight moments in the video.

% -	이전에도 이러한 문제를 해결하기 위해 (asdfasdf) for moment retrieval, (asdfasdf) for highlight detection 등이 제안 되었지만, 이들은 비디오의 특정 영역을 찾는다는 공통된 목적을 가지고 있으면서도, 데이터 셋의 한계로 인해 따로 연구되었음. 이를 문제 삼으며, 최근에는 두 task를 동시에 학습할 수 있는 dataset이 소개 되었는데, 컴퓨터비전에서 최근 각광을 받고 있는 Transformer 모델 도입과 함께 큰 발전을 거듭하고 있음.

% -	구체적으로, 이 두가지 task를 수행하기 위해서는 transformer를 두가지 방법으로 이용할 수 있는데, moment-DETR 처럼 moment 를 clip의 set 단위로 예측할 수 있고, UMT 처럼 clip-wise prediction을 할 수 있음. 하지만, 이들은 query를 condition이 아닌 video와 동등한 레벨로 취급하거나 [mDETR], 매 클립이 self-attention으로 mixing 된 후에 condition을 걸어주어 clip간의 차이를 확실하지 이용하지 못하였고, 또한, 확실하게 condition으로 주지 못하였고, video와 query 사이의 관계를 한정적으로만 이용하였다.

% -	따라서, we explore three different ways to fully exploit query information. First, we design one-way cross-attention layer to condition every clip with the query features. Then, we utilized the negative video-text pairs to better model the relationships between the video and the text embeddings. Lastly, we define the saliency token to be the video-query dependent saliency estimator.


















% ===================== neg pair 부분 ===========================
% Nevertheless, the current training scheme, only considering the given video-query pair, still disturbs the model from identifying proper query-relevance prediction.
% In detail, the model focus on learning the fine-grained discrepancy between video clips, while neglecting the information they share, which contains significant clues to understand the context of video.
% Therefore, we leverage the negative video-query relationships to enhance the capability of identifying the contextual similarity of query and video clips.
% Therefore, we leverage the negative video-query relationships by suppressing those pairs, so that enhance the capability of identifying the contextual similarity of query and video clips.
% We hypothsize the diversity in query-video pairs are insufficient to learn the general relationship between text query and video.
% Therefore, we leverage the negative video-query relationships by suppressing the saliency scores of the irrelevant video-query pairs.
% However, this architectural design still lacks in identifying proper query relevance.
% We argue that the current training scheme only focuses on learning the fine-grained discrepancy between clips in a single video, while neglecting the information they share, which contains significant clues to understand the context of the video.
% Therefore, we leverage the negative video-query relationships to enhance the capability of identifying the contextual similarity of query and video clips.
% However, this architectural design still lacks in identifying proper query relevance.
% We argue the problem of the current training scheme that only focuses on learning the fine-grained discrepancy between clips in a single video.
% That is, the current design neglects the information shared throughout the video, although it contains significant clues to understand the context of the video.
\section{Related Work}
\label{sec:related_work}
\subsection{Co-Speech Gesture Synthesis}
The early approaches for generating co-speech gestures often involve creating linguistic rules to translate speech input into a sequence of pre-collected gesture segments, which are typically referred to as rule-based methods \cite{cassell1994rulefullbody,cassell2001beat,kipp2004gesture,kopp2006bml}. \citet{wagner2014rulereview} provide a comprehensive review of these methods. Rule-based methods produce interpretable and controllable results, but creating gesture datasets and rules requires significant effort. To alleviate the manual effort of designing rules in rule-based methods, data-driven approaches have gradually become predominant in this field. \citet{nyatsanga2023data_driven_gesture_survey} offer a thorough survey of these methods. Early data-driven approaches aim to directly learn mapping rules from data through statistical models \cite{neff2008videogesture,levine2009prosodygesture,levine2010gesturecontroller} and combine them with predefined gesture units for gesture generation. Later, the powerful modeling capability of deep neural networks makes it possible to train complex end-to-end models using raw speech-gesture data directly. One option is deterministic models, such as MLP \cite{kucherenko2020gesticulator}, CNN \cite{habibie2021videogesture}, RNN \cite{yoon2019robot,yoon2020trimodalgesture,bhattacharya2021affectivegesture,liu2022hierarchicalgesture}, and Transformer \cite{bhattacharya2021text2gestures}. Another choice is generative models, including flow-based models \cite{alexanderson2020stylegesture,ye2022styleflowgesture}, VAEs \cite{li2021audio2gesture,ghorbani2022zeroeggs}, and VQ-VAE \cite{yi2022talkshow,yazdian2022gesture2vec,liu2022vqgesturevideo}. Due to the inherent many-to-many relationship between speech and gesture, end-to-end models can generate natural-looking gestures but face challenges in ensuring content matching between speech and generated gestures \cite{yoon2022genea}. To address this issue, some neural systems aim to explicitly model both rhythm and semantics from the perspective of model structure \cite{kucherenko2021speech2properties2gestures,ao2022rhythmicgesticulator,liu2022disco} or training supervision strategy \cite{liang2022seeg}. Furthermore, hybrid systems, such as the combination of deep features and motion graphs \cite{zhou2022gesturemaster}, have been proposed to harness the advantages of different approaches. Recently, diffusion models \cite{sohldickstein2015diffusion,song2020improvedscore,ho2020ddpm} have demonstrated impressive results in image synthesis \cite{ramesh2022dalle2} and human motion generation \cite{tevet2022humanmotiondiffusion, zhang2022motiondiffuse}. Inspired by these works, our system adapts the latent diffusion model \cite{rombach2022latentdiffusion} for the co-speech gesture generation task and achieves appealing results.

\subsection{Style Control for Human Motion}
A typical approach to style control for human motion involves specifying a motion clip as a reference and transferring the reference clip's style to the source motion. This task is also known as \emph{style transfer}. Early works in motion style transfer integrate traditional machine learning techniques with manually defined features to infer motion styles \cite{hsu2005motion_style_translation,ma2010motion_style_transfer,xia2015realtime_motion_style_transfer,yumer2016spectral_motion_style_transfer}. Recently, deep learning-based methods have significantly enhanced motion quality. \citet{holden2016deepmotion} first propose a learning framework enabling motion style control through optimization in the motion manifold space. \citet{du2019stylemotioncvae} improve transfer efficiency by training a conditional VAE. \citet{mason2018few-shot_motion_style_transfer} use few-shot learning to generate stylized locomotion. \citet{aberman2020adain} employ a temporally invariant adaptive instance normalization (AdaIN) layer for target style injection, eliminating the need for paired data during training. \citet{wen2021stylemotionflow} achieve unsupervised style transfer using a flow model. \citet{jang2022motionpuzzle} introduce a method capable of controlling styles for individual body parts.

Previous co-speech gesture synthesis systems with style control can be categorized based on whether or not they require style labels. For methods needing labeled data, early works can only learn an individual style for one generator \cite{levine2010gesturecontroller,neff2008videogesture,ginosar2019stylegesture}. \citet{ahuja2022lowresource} propose a strategy that efficiently adapts the source generator to another speaker style using low-resource data. Some works learn a speaker style embedding space with labeled speaker-motion data, enabling gesture style control by sampling from this space \cite{ahuja2020stylegesture,yoon2020trimodalgesture,bhattacharya2021affectivegesture}. \citet{alexanderson2020stylegesture} aimat controlling fine-grained styles, such as gesturing speed and spatial scope, using preprocessed control signal-motion data. Their later work \cite{alexanderson2022diffusiongesture} utilizes a diffusion model for audio-driven motion synthesis, achieving label-based style control by training the model on labeled data. For methods not requiring style labels, \citet{habibie2022motionmatching} propose a motion matching framework to achieve flexible style control. Other studies achieve arbitrary style control by imitating an example given as a video \cite{liu2022hierarchicalgesture} or a motion clip \cite{ghorbani2022zeroeggs,ye2022styleflowgesture,kuriyama2022tokenizedgestures}.  In this work, we utilize a CLIP-based encoder to extract a style embedding from an arbitrary text prompt and incorporate it into the generator via an AdaIN layer, guiding the synthesis of stylized gestures. Our system supports fine-grained multimodal style prompts as opposed to label-based style control. It employs a self-supervised learning scheme and eliminates the need for labeled data. Additionally, we use an autoregressive model rather than a parallel model, making it potentially suitable for real-time applications.
\section{System Overview}
\label{sec:system_overview}

\begin{figure}[t]
    \centering
    \includegraphics[width=\linewidth]{figures/system_overview.pdf}
    \caption{Our system consists of two core components:
    (a) a latent diffusion model that takes speech audio and transcript as input and generate co-speech gestures, and (b) a CLIP-based encoder that extracts style embeddings from an arbitrary style prompt and incorporates them into the diffusion model via an adaptive instance normalization (AdaIN) layer. The system allows using short texts, video clips, and motion sequences to define gesture styles by encoding these style prompts to the same CLIP embedding space using corresponding pretrained encoders.}
    \Description{}
    \label{fig:system_overview}
\end{figure}

Our system takes the audio and transcript of a speech as input and synthesizes realistic stylized full-body gestures that match the speech content both rhythmically and semantically. It allows using a short piece of text, namely a \emph{text prompt}, a video clip, namely a \emph{video prompt}, or a motion sequence, namely a \emph{motion prompt}, to describe a desired style. The gestures are then generated to produce the style as much as possible.

We build the system based on the latent diffusion models \cite{rombach2022latentdiffusion}, which apply diffusion and denoising steps in a pretrained latent space. We learn this latent motion space using VQ-VAE \cite{van2017vqvae}, which offers compact motion embeddings ensuring quality and diversity of motion. As illustrated in \fig\ref{fig:system_overview}, our system is composed of two major components: (a) an end-to-end neural generator that takes speech audio and text transcript as input and generates speech-matched gesture sequences using latent diffusion models; and (b) a CLIP-based encoder that extracts style embeddings from the style prompts and incorporates them into the diffusion models via an adaptive instance normalization (AdaIN) layer \cite{huang2017adain} to guide the style of the generated gestures. We further learn a joint embedding space between corresponding gestures and transcripts using contrastive learning, which provides useful semantic cues for the generator and a semantic loss that effectively guides the generator to learn semantically meaningful gestures during training.

The system is trained with the classifier-free diffusion guidance \cite{ho2022classifierfree}, combined with a self-supervised learning scheme to enable training on motion data without style labels. In the following sections, we will provide details about the components and how they are trained in our system.
\begin{figure*}[t]
    \centering
    \includegraphics[width=0.9\textwidth]{figures/gesture-transcript_embedding_learning.pdf}
    \caption{We learn an gesture-transcript joint embedding space using contrastive learning. A transcript encoder is trained to convert a transcript sentence $\vect{T}$ into a sequence of feature codes $\vect{Z}^t$, which are then aggregated into a transcript embedding vector $\vect{z}^t$ via max pooling. Similarly, the corresponding gesture sequence $\hatvect{Z}$ is processed by a gesture encoder, resulting in a feature sequence $\vect{Z}^g$ and the corresponding embedding $\vect{z}^g$. The encoders are trained using a contrastive loss that maximizes the similarity between the embeddings $\vect{z}^t$ and $\vect{z}^g$ of paired transcripts and gestures.}
    \Description{}
    \label{fig:gesture-transcript_embedding_learning}
\end{figure*}

\section{Motion Representation}
\label{subsec:motion_representation}
A gesture motion $\vect{M}=[\vect{m}_k]_{k=1}^{K}$ is a sequence of poses, where $K$ denotes the length of the motion. Each pose $\vect{m}_k\in\mathbb{R}^{3+6J}$ consists of the displacement of the character and the rotations of its $J$ joints. We parameterize the rotations as 6D vectors \cite{zhou20196dvector}, though other rotation representations can possibly be used instead. This raw motion representation, however, often contains redundant information. Following recent successful systems \cite{ao2022rhythmicgesticulator,rombach2022latentdiffusion,dhariwal2020jukebox}, we learn a compact motion representation using VQ-VAE \cite{van2017vqvae} to ensure motion quality and diversity.

Specifically, we train VQ-VAE as an encoder-decoder pair
\begin{align}
    \vect{Z} = \mathcal{E}_{\eqword{VQ}}(\vect{M})  \quad \Leftrightarrow \quad \vect{M} = \mathcal{D}_{\eqword{VQ}}(\vect{Z}) .
\end{align}
The encoder $\mathcal{E}_{\eqword{VQ}}$ converts $\vect{M}$ into a downsampled sequence of latent codes $\vect{Z}=[\vect{z}_l]_{l=1}^{L}$, where $\vect{z}_l\in\mathbb{R}^{C}$ and $C$ is the dimension of the latent space. We refer to the ratio $d=K/L$ as the encoder's downsampling rate, which is determined by the network structure.  The decoder $\mathcal{D}_{\eqword{VQ}}$ operates on a quantized version of the latent space. It maintains a \emph{codebook} consisting of $N_{\eqword{VQ}}$ latent vectors. When reconstructing the original motion $\vect{M}$ from $\vect{Z}$, the decoder maps each $\vect{z}_k$ to its nearest codebook vector $\hatvect{z}_l$ and decodes the quantized latent sequence $\hatvect{Z}=[\hatvect{z}_l]_{l=1}^{L}$ into $\vect{M}$.

Our VQ-VAE model has a network structure similar to that of Jukebox \cite{dhariwal2020jukebox}, which consists of a cascade of 1D convolutional networks. The encoder $\mathcal{E}_{\eqword{VQ}}$ and decoder $\mathcal{D}_{\eqword{VQ}}$ are learned following the standard VQ-VAE training process \cite{van2017vqvae,dhariwal2020jukebox}. They are then frozen in the rest of training. 
Both the latent sequence $\vect{Z}$ and its quantized version $\hatvect{Z}$ are used as the motion representation by the other components of the system. Specifically, we learn the gesture-transcript joint embeddings on the quantized latent sequence $\hatvect{Z}$ in Section~\ref{sec:gesture-transcript_embedding_learning}, while the latent diffusion model synthesizes gesture motions as $\vect{Z}$ in Section~\ref{sec:stylized_co-speech_gesture_diffusion_model}.
\section{Gesture-Transcript Joint Embeddings}
\label{sec:gesture-transcript_embedding_learning}
The many-to-many mapping between speech content and gestures brings challenges to generating semantically correct motions. To alleviate this problem, we learn a joint embedding space of gestures and speech transcripts to mine for the semantic connections between the two modalities.

\subsection{Architecture}
As shown in \fig\ref{fig:gesture-transcript_embedding_learning}, we train two encoders, a gesture encoder $\mathcal{E}_G$ and a transcript encoder $\mathcal{E}_T$, to map the gesture motion and speech transcripts into the shared embedding space respectively. Both the encoders process the input speech in sentences. The speech transcripts are tokenized using the {T5 tokenizer \cite{xue2021mt5}} and temporally associated with the audio using the {Montreal Forced Aligner (MFA) \cite{mcauliffe2017mfa}}. This procedure also aligns the transcripts with the gestures. The speech data is then segmented into sentences based on the transcripts. Then we compute
\begin{align}
    \vect{Z}^t=\mathcal{E}_T(\vect{T}) , \quad  \quad \vect{Z}^g=\mathcal{E}_G(\hatvect{Z}) ,
\end{align}
where $\vect{T}\in \mathcal{W}^{L_t}$ denotes a tokenized transcript sentence, parameterized as a sequence of word embeddings $w\in\mathcal{W}$, and $\hatvect{Z}\in \mathbb{R}^{L_g \times C}$ is the quantized latent representation of the corresponding gesture sequence. The output of the encoders, $\vect{Z}^t\in\mathbb{R}^{L_t \times C_{{s}}}$ and $\vect{Z}^g\in\mathbb{R}^{L_g \times C_{{s}}}$, are sequences of feature vectors of the same dimension $C_{{s}}$. Note that the lengths of these sequences, $L_t$ and $L_g$, can be different.

A semantic gesture and the utterance of its corresponding word or phrase are often not perfectly aligned in a speech \cite{liang2022seeg}. Such misalignment can confuse the encoder if the temporal correspondence between the two modalities is rigidly enforced during training. To alleviate this problem, we aggregate the semantics-relevant information in each feature sequence via max pooling
\begin{align}
    \vect{z}^t = \eqword{max\_pooling}(\vect{Z}^t)  , \quad 
    \vect{z}^g = \eqword{max\_pooling}(\vect{Z}^g) .
\end{align}
Then $\vect{z}^t,\vect{z}^g \in \mathbb{R}^{C_{{s}}}$ are considered the embeddings of the transcripts and gestures, respectively. 

We employ a powerful pretrained language model, T5-base \cite{xue2021mt5}, as the text encoder $\mathcal{E}_T$. The motion encoder $\mathcal{E}_G$ is a 12-layer, 768-feature wide, encoder-only transformer with 12 attention heads, which is pretrained on the gesture dataset by predicting masked motions in a way similar to BERT \cite{devlin2019bert}. Both the encoders are then fine-tuned using contrastive learning.

\subsection{Contrastive Learning}

\begin{figure}[t]
    \centering
    \includegraphics[width=\linewidth]{figures/contrastive_loss.pdf}
    \caption{An illustration of the CLIP-style contrastive loss used to train the gesture and transcript encoders.}
    \Description{}
    \label{fig:contrastive_loss}
\end{figure}

We apply CLIP-style contrastive learning \cite{radford2021clip} to fine-tune the encoders. Given a batch of pairs of gesture and transcript embeddings $\mathcal{B}=\{(\vect{z}^t_i,\vect{z}^g_i)\}_{i=1}^{B}$, where $B$ is the batch size, the goal of the training is to maximize the similarity of the embeddings $(\vect{z}^t_i,\vect{z}^g_i)$ of the real pairs in the batch while minimizing the similarity of the incorrect pairs $(\vect{z}^t_i, \vect{z}^g_j)_{i\neq{}j}$. As illustrated in \fig\ref{fig:contrastive_loss}, this learning objective can be defined as the summation of the gesture-to-text ($\text{g2t}$) cross entropy and the text-to-gesture ($\text{t2g}$) cross entropy computed across the batch. Formally, the loss function is
\begin{align}
    \mathcal{L}_{\eqword{contrast}} = \mathbb{E}_{\mathcal{B}\sim\mathcal{D}}&\Bigl[
        \mathcal{H}_{\mathcal{B}}\left(\vect{y}^{\eqword{g2t}}(\vect{z}^g_i), \vect{p}^{\eqword{g2t}}(\vect{z}^g_i)\right) 
        \nonumber \\ 
    &+ \mathcal{H}_{\mathcal{B}}\left(\vect{y}^{\eqword{t2g}}(\vect{z}^t_j), \vect{p}^{\eqword{t2g}}(\vect{z}^t_j)\right)\Bigr].
\end{align}
Each cross entropy $\mathcal{H}$ is computed between a one-hot encoding $\vect{y}$ and a softmax-normalized distribution $\vect{p}$. $\vect{y}$ specifies the true correspondence between the gestures and transcripts in the training batch $\mathcal{B}$. $\vect{p}$ computes the similarity between an embedding of one modality and those of the other modality. Specifically,
\begin{align}
    \label{eqn:multimodal_similarity}
    %\vect{y}^{\eqword{g2t}}(\vect{z}^g_i) &= ,
    \vect{p}^{\eqword{g2t}}(\vect{z}^g_i) = \frac{\exp(\vect{z}^g_i \cdot \vect{z}^t_i / \tau)}{\sum^B_{j=1}\exp(\vect{z}^g_i \cdot \vect{z}^t_j / \tau)} 
    %\nonumber\\    
    %\vect{y}^{\eqword{t2g}}(\vect{z}^g_i) &= ,
    ,\quad
    \vect{p}^{\eqword{t2g}}(\vect{z}^t_j) = \frac{\exp(\vect{z}^t_j \cdot \vect{z}^g_j / \tau)}{\sum^B_{i=1}\exp(\vect{z}^t_j \cdot \vect{z}^g_i / \tau)},
\end{align}
where $\tau$ is the temperature of softmax. We further employ the \emph{momentum distillation} (MoD) \cite{li2021albef} technique to improve learning performance under noisy supervision. The key idea of MoD is to learn from the pseudo-targets generated by a momentum model. During training, we maintain a momentum version of the encoders by updating their network parameters in an exponential-moving-average (EMA) manner. Then, we use the momentum models to calculate multimodal features $\tildevect{z}^t$ and $\tildevect{z}^g$ for the training gesture-transcript pairs and compute the pseudo-targets $\tildevect{p}^{\eqword{g2t}}$ and $\tildevect{p}^{\eqword{t2g}}$ by substituting these features into \eqn\eqref{eqn:multimodal_similarity}. The contrastive loss is then modified as
\begin{align}
    \mathcal{L}_{\eqword{contrast}}^{\eqword{MoD}} &= (1 - w_{\eqword{contrast}})\mathcal{L}_{\eqword{contrast}} \nonumber \\
    &+ w_{\eqword{contrast}}\mathbb{E}_{\mathcal{B} \sim \mathcal{D}}\Bigl[D_{KL}\left(\tildevect{p}^{\eqword{g2t}}(\tildevect{z}^g_i) || \vect{p}^{\eqword{g2t}}(\vect{z}^g_i)\right) \nonumber \\
    &+ D_{KL}\left(\tildevect{p}^{\eqword{t2g}}(\tildevect{z}^t_j) || \vect{p}^{\eqword{t2g}}(\vect{z}^t_j)\right)\Bigr],
\end{align}
where $D_{KL}(\cdot || \cdot)$ is the KL divergence and $w_{\eqword{contrast}}$ is set to $0.4$.

\subsection{Applications of the Joint Embeddings}

\begin{figure}[t]
    \centering
    \begin{subfigure}[t]{\linewidth}
        \centering
        \includegraphics[width=\linewidth]{figures/motion-based_transcripts_retrieval.pdf}
        \caption{Transcripts retrieved based on example gestures. Note that a gesture can natually accompany several semantics.}
        \label{fig:motion-based_transcripts_retrieval}
    \end{subfigure} \\
    \vspace{5pt}
    \begin{subfigure}[t]{\linewidth}
        \centering
        \includegraphics[width=\linewidth]{figures/semantic_saliency.pdf}
        \caption{Semantic saliency curves of two sentences. The peaks of the curves indicate the words with high semantic importance which are likely to be accompanied by semantic gestures.}
        \label{fig:semantic_saliency}
    \end{subfigure}
    \caption{Applications of the gesture-transcript joint embeddings. (a) Motion-based transcripts retrieval . (b) Semantic saliency identification.}
    \label{fig:applications_of_gesture-transcript_embedding}
    \Description{}
\end{figure}

The learned joint embedding space combined with the encoders offers an effective way of measuring the semantic similarity between gestures and transcripts. To demonstrate this, we map a gesture motion into this space and retrieve the closest sentences from the transcript dataset based on the cosine distance of the embeddings. \fig\ref{fig:motion-based_transcripts_retrieval} shows some results. It can be observed that the retrieved sentences may have various meanings but can all be naturally accompanied by the query gestures.

Besides, the computation of the embeddings involves the max pooling operator that aggregates the most semantics-relevant information. We can thus estimate the saliency of each pose or word in a gesture sequence or a transcript sentence respectively using the embeddings. Specifically, given a sentence $\vect{T}$ and its encoded feature sequence $\vect{Z}^t$ and embedding vector $\vect{z}^t$, we compute the semantic saliency of each word as
\begin{align}
    \label{eqn:semantic-saliency}
    \vect{s}^t = \eqword{softmax}\left(\vect{Z}^t \cdot \vect{z}^t\right).
\end{align}
As illustrated in \fig\ref{fig:semantic_saliency}, the words with high semantic importance and that are likely to be accompanied by semantic gestures will have high saliency scores. This information can be considered as an important semantic cue, which will be used in our system to guide the gesture generator to create semantically correct gestures.
\section{Stylized Co-Speech Gesture Diffusion Model}
\label{sec:stylized_co-speech_gesture_diffusion_model}

\begin{figure*}[t]
    \centering
    \includegraphics[width=\textwidth]{figures/denoising_network.pdf}
    \caption{Architecture of the denoising network. The model is a multi-layer transformer with a causal attention structure. It takes the audio and transcript of a speech and a style prompt as input and estimates the diffusion noise. Three CLIP-based encoders are learned to support different types of style prompts. The multimodal features are incorporated into the network at different stages via semantics-aware layers and AdaIN layers respectively. \emph{Norm} means the layer normalization and \emph{FFN} is the feed-forward network.}
    \Description{}
    \label{fig:denoising_network}
\end{figure*}

The core of our system is a conditional latent generative model $\mathcal{G}$ which synthesizes a sequence of latent gesture codes $\vect{Z}=[\vect{z}_l]_{l=1}^{L}$ conditioned on a speech and a style prompt. The latent sequence $\vect{Z}$ is then decoded into gestures using the VQ-VAE decoder $\mathcal{D}_{\eqword{VQ}}$ learned in Section~\ref{subsec:motion_representation}. Formally, the generator $\mathcal{G}$ computes
\begin{align} 
    \vect{Z} = \mathcal{G}(\vect{A}, \vect{T}, \vect{P}),
\end{align}
where $\vect{A}$ and $\vect{T}$ denote the audio and transcript of the speech respectively and $\vect{P}$ is the style prompt. 
The speech audio $\vect{A}=[\vect{a}_i]_{i=1}^{L}$ is parameterized as a sequence of acoustic features resampled into the same length of the gesture representation. Each $\vect{a}_i$ encodes the onsets and amplitude envelopes that reflect the beat and volume of speech respectively. The speech transcript $\vect{T}$ is preprocessed as described in Section~\ref{sec:gesture-transcript_embedding_learning}. The generator $\mathcal{G}$ utilizes $\vect{A}$ to infer low-level gesture styles such as rhythm and stress, $\vect{T}$ for the semantics-level features, and $\vect{P}$ to determine the overall style of the gestures.

At inference time, the generator $\mathcal{G}$ is reformulated as an autoregressive model, where a gesture is determined by not only the speech context and style prompt but also the previous motion. Formally, the latent sequence $\vect{Z}=[\vect{z}_l]_{l=1}^{L}$ is generated as
\begin{align}
    \label{eqn:autoregressive_generative_model}
    {\vect{z}_l^*} = \mathcal{G}([\vect{z}_i^*]_{i=1}^{l-1}, [\vect{a}_i]_{i=1}^{l+\delta^a}, \vect{T}, \vect{P}),
\end{align}
where we use the asterisk ($*$) to indicate quantities already generated by $\mathcal{G}$. Note that the generator leverages $\delta^a$ frames of future audio features to determine the current gestures.

\subsection{Latent Diffusion Models}
The generator $\mathcal{G}$ is based on the latent diffusion model \cite{rombach2022latentdiffusion}, which is a variant of diffusion models that applies the forward and reverse diffusion processes in a pretrained latent feature space. The \emph{diffusion process} is modeled as a Markov noising process. Starting from a latent gesture sequence $\vect{Z}_0$ drawn from the gesture dataset, the diffusion process progressively adds Gaussian noise to the real data until its distribution approximates $\mathcal{N}(\vect{0}, \vect{I})$. The distribution of the latent sequences thus evolves as
\begin{align}
    q(\vect{Z}_n | \vect{Z}_{n-1}) = \mathcal{N}(\sqrt{\alpha_n}\vect{Z}_{n-1}, (1-\alpha_n)\vect{I}),
\end{align}
where $\vect{Z}_n$ is the latent sequence sampled at diffusion step $n$, $n\in\{1, \dots, N\}$, and $\alpha_n$ is determined by the variance schedules. In contrast, the \emph{reverse diffusion process}, or the \emph{denoising process}, estimates the added noise in a noisy latent sequence. Starting from a sequence of random latent codes $\vect{Z}_N \sim \mathcal{N}(\vect{0}, \vect{I})$, the denoising process progressively removes the noise and recovers the original motion $\vect{Z}_0$.

To achieve conditional gesture generation, we train a network $\vect{E}_{\theta}$, the so-called \emph{denoising network}, to predict the noise conditioned on the noisy motion codes, the diffusion step, the speech context, and the style prompt. This network can be formulated as
\begin{align}
    \label{eqn:denoising_network}
    \vect{E}^{*}_n = \vect{E}_{\theta}(\vect{Z}_n, n, \vect{A}, \vect{T}, \vect{P}).
\end{align}
At inference time, the generator $\mathcal{G}$ leverages the sampling algorithm of DDPM~\cite{ho2020ddpm} to synthesize gestures. It first draws a sequence of random latent codes $\vect{Z}_N^*\sim{}\mathcal{N}(\vect{0}, \vect{I})$ then computes a series of denoised sequences $\{\vect{Z}_n^*\},{n=N-1,\dots,0}$ by iteratively removing the estimated noise $\vect{E}_n^*$ from $\vect{Z}_n^*$. Lastly, the final latent codes $\vect{Z}_0^*$ will be decoded into the gesture motion.

As illustrated in \fig\ref{fig:denoising_network}, our denoising network has a transformer architecture \cite{vaswani2017transformer}. We employ the causal attention layer proposed by \citet{vaswani2017transformer} that only allows the intercommunication of the current and preceding data for the causality. This architecture can be easily transformed into the autoregressive model in \eqn\eqref{eqn:autoregressive_generative_model}. Note that we extend the definition of \emph{current data} when dealing with audio features by including $\delta^a$ future frames.

The denoising network fuses the multimodal conditions $(\vect{A}, \vect{T}, \vect{P})$ in a hierarchical manner: the low-level audio features that relate to the speech rhythm and stress are first included by concatenating $\vect{A}$ with the noisy latent sequence $\vect{Z}_n$; then, the high-level transcript features $\vect{T}$ that correspond to the speech semantics are incorporated via a \emph{semantics-aware attention layer}; and lastly, the style prompt $\vect{P}$ is involved through a \emph{CLIP-guided AdaIn layer} to control the overall style of the generated gestures. 

\subsubsection{Semantics-Aware Attention Layer}
Inspired by recent successful attention-based multimodal systems \cite{jaegle2021perceiver,rombach2022latentdiffusion}, we develop a semantics-aware attention layer based on the cross-attention mechanism \cite{vaswani2017transformer} to incorporate the input transcript $\vect{T}$. Specifically, we first extract the transcript features $\vect{Z}^t$ from $\vect{T}$ using the pretrained text encoder $\mathcal{E}_T$ as described in Section~\ref{sec:gesture-transcript_embedding_learning} and compute the semantic saliency $\vect{s}^t$ using \eqn\eqref{eqn:semantic-saliency}. Then, we project $\vect{Z}^t$ to the \emph{key} $\vect{K} \in \mathbb{R}^{L_t \times C_t}$ and \emph{value} $\vect{V} \in \mathbb{R}^{L_t \times C_t}$ of the attention mechanism using learnable projection matrices and calculate the \emph{query} $\vect{Q} \in \mathbb{R}^{L \times C_t}$ using the intermediate features of the denoising network. Lastly, the semantics-aware attention layer is implemented as
\begin{align}
    \eqword{Attention}(\vect{Q}, \vect{K}, \vect{V}) = \eqword{softmax}(\frac{\vect{Q}\vect{K}^T}{\sqrt{C_t}} \cdot \vect{S}^t) \cdot \vect{V},
\end{align}
where $\vect{S}^t \in \mathbb{R}^{L \times L_t}$ is the temporally-broadcasted semantic saliency matrix of $\vect{s}^t$ that guides the network to pay additional attention to the semantically important words.

\subsubsection{CLIP-Guided AdaIN Layer}
\label{subsec:CLIP-guided-adain}
We employ an adaptive instance normalization (AdaIN) layer \cite{huang2017adain} to inject the information of the style prompts into the denoising network. Specifically, we leverage a pretrained CLIP encoder $\mathcal{E}_{\eqword{CLIP}}$ to convert the input style prompt into a style embedding $\vect{z}^s \in \mathbb{R}^{C_{\eqword{CLIP}}}$. Then, we learn a MLP network to map the style embedding $\vect{z}^s$ to $2 C_{\eqword{ada}}$ parameters that modify the per-channel mean and variance of the AdaIn layer with $C_{\eqword{ada}}$ channels.

We employ the text encoder $\mathcal{E}_{\eqword{CLIP-T}}$ of the CLIP model \cite{radford2021clip} for the text prompts and the motion encoder $\mathcal{E}_{\eqword{CLIP-M}}$ of the MotionCLIP model \cite{tevet2022motionclip} for the motion prompts. We further develop a CLIP-based video encoder $\mathcal{E}_{\eqword{CLIP-V}}$ for the video prompts, which consists of a pretrained CLIP image encoder \cite{radford2021clip} followed by a $6$-layer transformer that temporally aggregates the image features of the video into an embedding $\vect{z}^s$ in the CLIP space. Note that all these CLIP encoders are pretrained in a separate stage and their weights are frozen when training the denoising network.

\subsection{Training}
\label{subsubsec:training_of_denosing_network}
Following the standard training process of denoising diffusion models \cite{ho2020ddpm,rombach2022latentdiffusion}, we train the denoising network $\vect{E}_{\theta}$ by drawing random tuples $(\vect{Z}_0,n,\vect{A},\vect{T},\vect{P})$ from the training dataset, then corrupting $\vect{Z}_0$ into $\vect{Z}_n$ by adding random Gaussian noises $\vect{E}$, applying denoising steps to $\vect{Z}_n$, and optimizing the loss 
\begin{align}
    \mathcal{L}_{\eqword{net}} = 
    w_{\eqword{noise}}\mathcal{L}_{\eqword{noise}} +
    w_{\eqword{semantic}}\mathcal{L}_{\eqword{semantic}} +
    w_{\eqword{style}}\mathcal{L}_{\eqword{style}}.
\end{align}
Specifically, the ground-truth gesture motion $\vect{M}_0$ and its latent representation $\vect{Z}_0$, the audio $\vect{A}$, and the transcript $\vect{T}$ are extracted from the same speech sentence. $n$ is drawn from the uniform distribution $\mathcal{U}\{1,N\}$. We do not assume that the gesture dataset contains detailed style labels. Instead, we consider the motion clips $\vect{M}_P$ of random lengths but encompassing $\vect{M}_0$ as the style prompts $\vect{P}$. As suggested by \citet{kim2022CVPRDiffusionClip}, we iteratively apply denoising steps to each training tuple until obtaining $\vect{Z}_0^*$.

We first calculate the standard noise estimation loss of the diffusion models \cite{ho2020ddpm} defined as
\begin{align}
    \mathcal{L}_{\eqword{noise}} = \norm{\vect{E} - \vect{E}_{\theta}(\vect{Z}_n, n, \vect{A}, \vect{T}, \vect{P})}_{2}^{2},
\end{align}
In addition, we include a semantic loss to ensure the semantic correctness of the generated gestures. This loss is defined in the gesture-transcript joint embedding space learned in Section \ref{sec:gesture-transcript_embedding_learning}. Specifically,
\begin{align}
    \mathcal{L}_{\eqword{semantic}} = 1 - \cos(\vect{z}^g_0, \vect{z}^{g*}_0),
\end{align}
where $\cos(\cdot, \cdot)$ is the cosine distance, $\vect{z}^g_0$ and $\vect{z}^{g*}_0$ are the gesture encodings of the ground-truth and generated motions, respectively, computed using the gesture encoder $\mathcal{E}_G$ pretrained in Section \ref{sec:gesture-transcript_embedding_learning}. At last, we employ a perceptual loss to encourage the generator to follow the style prompts. This style loss is defined as
\begin{align}
    \mathcal{L}_{\eqword{style}} = 1 - \cos(\mathcal{E}_{\eqword{CLIP-M}}(\vect{M}_0), \mathcal{E}_{\eqword{CLIP-M}}(\vect{M}^{*}_0)),
\end{align}
where $\mathcal{E}_{\eqword{CLIP-M}}$ is the pretrained motion encoder, and $\vect{M}^{*}_0=\mathcal{D}_{\eqword{VQ}}(\vect{Z}_0^*)$ is the generated gestures.

We utilize the classifier-free guidance \cite{ho2022classifierfree} to train our model. Specifically, we let $\vect{E}_{\theta}$ learn both the style-conditional and unconditional distributions by randomly setting $\vect{P} = \varnothing$ and thus disabling the AdaIN layer by $10\%$ chance during training. At inference time, the predicted noise is computed using 
\begin{align}
    \vect{E}^*_n&= s \vect{E}_{\theta}(\vect{Z}_n, n, \vect{A}, \vect{T}, \vect{P}) + (1-s)  \vect{E}_{\theta}(\vect{Z}_n, n, \vect{A}, \vect{T}, \varnothing)
\end{align}
instead of \eqn\eqref{eqn:denoising_network}. This scheme further allows us to control the effect of the style prompt $\vect{P}$ by adjusting the scale factor $s$.

\subsubsection{CLIP Video Encoder}
We develop a CLIP-based video encoder $\mathcal{E}_{\eqword{CLIP-V}}$ in Section~\ref{subsec:CLIP-guided-adain} to enable video clips as the style prompts. $\mathcal{E}_{\eqword{CLIP-V}}$ encapsulates a pretrained CLIP image encoder \cite{radford2021clip} whose weights are frozen and a learnable transformer network. To learn $\mathcal{E}_{\eqword{CLIP-V}}$, we render a random motion sequence  $\vect{M}$ into a video and optimize the loss function
\begin{align}
    \label{eqn:clip-video-encoder-loss}
    \mathcal{L}_{\eqword{video}} = 1 - \cos(\eqword{sg}(\mathcal{E}_{\eqword{CLIP-M}}(\vect{M})), \mathcal{E}_{\eqword{CLIP-V}}(\eqword{R}(\vect{M}; \vect{r}))) ,
\end{align}
where $\eqword{sg}$ represents the \emph{stop gradient} operator that prevents the gradient from backpropagating through it, $\eqword{R}$ denotes the rendering operator that renders $\vect{M}$ into a video with camera parameters $\vect{r}$ configured similarly to \cite{aberman2020adain}, and $\mathcal{E}_{\eqword{CLIP-M}}$ is the pretrained motion encoder. This loss function ensures $\mathcal{E}_{\eqword{CLIP-V}}$ to map video clips into the same shared CLIP embedding space.

\subsection{Style Control of Body Parts}
Inspired by \cite{zhang2022motiondiffuse}, we extend our system to allow fine-grained styles control on individual body parts using \emph{noise combination}. Considering a partition $\mathcal{O}$ of the character's body, where each body part $o\in\mathcal{O}$ consists of several joints, we learn $O=|\mathcal{O}|$ individual motion VQ-VAEs to represent the motions of each body part as latent codes $\vect{Z}^{o} = \mathcal{E}_{\eqword{VQ}}^o(\vect{M}^o)$. The full-body motion codes $\vect{Z}^{\mathcal{O}} \in \mathbb{R}^{O \times (L \times C)}$ is then computed by stacking the motion codes of each body part. We then train a new latent diffusion model $\vect{E}_{\theta}$ based on $\vect{Z}^{\mathcal{O}}$ in the same way as introduced in the previous sections. At inference time, we predict full-body noises $\{\vect{E}_{n,o}^*\}_{o\in\mathcal{O}}$ conditioned on a set of style prompts $\{\vect{P}_o\}_{o\in\mathcal{O}}$ for every body part, where each $\vect{E}_{n,o}^*= \vect{E}_{\theta}(\vect{Z}_n^{\mathcal{O}}, n, \vect{A}, \vect{T}, \vect{P}_o)$. These noises can be simply fused as $\vect{E}_n^* = \sum_{o\in\mathcal{O}}\vect{E}_{n,o}^* \cdot M_o$, where $\{M_o\}$ are binary arrays indicating the partition of bodies in $\mathcal{O}$. To achieve better motion quality, we add a smoothness item to the denoising direction as suggested by~\citet{zhang2022motiondiffuse}, which is
\begin{align}
    \vect{E}_n^* = \sum_{o\in\mathcal{O}}\Bigl(\vect{E}_{n,o}^* \cdot M_o\Bigr) + 
    w_{\eqword{body}} \cdot \nabla_{\vect{Z}_n^{\mathcal{O}}}\Bigl(\sum_{i,j \in\mathcal{O}, i\neq{}j }\norm{\vect{E}_{n,i}^* - \vect{E}_{n,j}^*}_2\Bigr),
\end{align}
where $\nabla$ denotes the gradient calculation, $w_{\eqword{body}}$ is set to 0.01.
\section{RESULTS}

\begin{table*}[t]
\caption{Trajectories validated on the Valkyrie humanoid in simulation (all) and hardware (e, f).}

% \caption{Trajectories validated on the Valkyrie humanoid. All trajectories are validated in a physics simulation, (e) and (f) are additionally validated on hardware.}
\centering
\begin{tabular}{C{2em} C{15em} C{6em} C{4em} C{8em} C{7em} C{7em} C{7em}} 
 \hline
  & Description & Number of key frames & Duration (s) & Number of unique non-contact 6-dof anchors & Number of unique contact 6-dof anchors & Number of unique 1-dof anchors  & Number of unique CoM anchors\\ [0.5ex] 
 \hline
 (a) & Standing up from lying down on flat ground & 22 & 57.5 & 22 & 18 & 53 & 16  \\ 
 (b) & Stepping over a 45cm tall barrier with handholds & 18 & 59 & 14 & 38 & 38 & 17 \\
 (c) & Climbing up and standing on an 80cm tall ledge & 24 & 61.5 & 16 & 27 & 32 & 20 \\
 (d) & Rolling over from facing down to facing up & 6 & 19.5 & 3 & 7 & 45 & 0 \\
 (e) & Reaching forward and bracing against a wall to extend range of motion & 5 & 25 & 6 & 9 & 1 & 3 \\ 
 (f) & Crawling to kneeling with flat handholds & 21 & 84 & 24 & 27 & 47 & 18  \\ 
 \hline
\end{tabular}
\label{table:simulation_summary}
\end{table*}

\begin{figure*}
    \centering
    \includegraphics[width=0.95\textwidth]{Figures/DemoFigures3.PNG}
    \caption{Operator view while generating trajectories.}
    \label{fig:demo_trajectories}
\end{figure*}

We tested our framework by generating motions for a variety of multi-contact scenarios. Table \ref{table:simulation_summary} contains some of these scenarios along with key frame statistics. All trajectories were validated in simulation and trajectories (e) and (f) were validated on hardware. Key frame transitions have a default value of 2s for simulation and 4s for hardware but the operator can override this value to both shorten or extend transitions. We selected contact-rich scenarios to generate motions that are difficult to plan autonomously or through standard teleoperation. Many of the robot's limbs are used for contact, including feet (all), arms (all), knees (a, c, d, f), and chest (c, d). Additionally, many contact geometries are included. Planar contact occurs when the foot is in full contact with the environment. Point contacts are also common when the arms or knees are in contact with the environment, since these are modelled using curves meshes. Line contacts are also used when one edge of the foot is in contact, such as both feet in Fig. \ref{fig:demo_trajectories}(c). We find that the operator's use of anchors varies significantly based on the scenario. For example, when rolling over on flat ground (scenario d) the robot is often in a position where the CoM cannot be directly controlled and was not used as an anchor. In contrast, climbing onto a ledge (scenario c) requires careful positioning of the CoM while climbing and therefore is used in almost every key frame.

\subsection{CoM Constraint Regions}

To validate effectiveness of the CoM constraint region (Sec. \ref{sec:contact_mode}) for the generated motions, we compare it to a baseline flat-ground constraint. Figure \ref{fig:com_prox} shows this comparison performed for scenarios (e) and (f). The ``flat ground'' model computes the constraint region as the convex hull of the robot's contact points.  The ``multi-contact'' constraint region is computed using the friction- and actuation-aware model. The plotted quantity is the distance of the CoM to the nearest constraint edge of both regions. Both scenarios have key frames with substantial (multiple centimeter) difference in stability margin. Although the multi-contact constraint region is generally more restrictive, scenario (e) key frame 2 demonstrates this is not always the case. This key frame corresponds to a braced reaching motion, shown in Figure \ref{fig:hardware_demo} (left). In this situation, using the multi-contact constraint region enables a higher range of motion than would be possible if using the flat-ground model. Conversely, the multi-contact region is very restrictive for configurations in scenario (f) that require support from the arms. In scenario (f) key frame 7 (Fig. \ref{fig:hardware_demo}), the robot places significant weight on the right arm while lifting the left arm. The actuation limits of the right arm are reflected by the multi-contact constraint being 5cm higher than the flat ground model.

\begin{figure}
    \centering
    \includegraphics[width=0.8\columnwidth]{Figures/CoMStabilityMargins.png}
    \caption{CoM stability margins for scenarios (e) and (f).}
    \label{fig:com_prox}
\end{figure}

\begin{figure}
    \centering
    \includegraphics[width=\columnwidth]{Figures/hardware_demo.PNG}
    \caption{Valkyrie executing two multi-contact motions: (left) bracing against a wall with the right arm and reaching forward with the left arm and (right) placing the arms on cinder blocks while maneuvering to a kneel.}
    \label{fig:hardware_demo}
\end{figure}


% Timing and frequency of ''undo'' button
% other features that were useful. highlight cases where the actuation-feasible region is useful

\subsection{User Interface Operation}

We find there are two primary reasons for a generated key frame to be infeasible: controller failures and inverse kinematics failures. Controller failures often occur because the CoM trajectory is unstable or there is an unexpected collision while moving to a key frame. Our approach assumes that key frames are sufficiently close such that validating subsequent key frames serves as a validation of the trajectory between them. However in practice this does not always hold, particularly when a limb is moving near the environment such as the foot moving over the barrier in Fig. \ref{fig:demo_trajectories}(b). This could be addressed by incorporating a motion preview similar such as \cite{johnson2017team, marion2018director}. Inverse kinematics failures occur for two reasons: getting ``stuck'' and going unstable. Since our IK solver is based on local optimization, it is susceptible to getting stuck in local minima. Often the operator can guide the robot out of the minimum when aware that the problem is occurring. Solver instability can occur when inconsistent objectives are requested with high weight, such as contacts, collisions and CoM positioning. Such cases require halting the IK and reverting to the last key frame. For this reason, the operator may prefer disabling CoM and collision constraints in the solver and using visual cues as indication of feasibility, which can mitigate solver instability.

% Discuss different strategies, i.e. no com anchor for rolling on the ground, etc

% Emphasize how often the operator would press the abort button
% etc.

% Discuss the time taken for each one. ~5min per key frame.
% Discuss when and why the undo button is used.


\section{Conclusion}
\label{sec:conclusion}

We consider top-down attention by explaining from an Analysis-by-Synthesis (AbS) view of vision. Starting from previous work on the functional equivalence between visual attention and sparse reconstruction, we show that AbS optimizes a similar sparse reconstruction objective but modulates it with a goal-directed top-down modulation, thus simulating top-down attention. We propose \model, a top-down modulated ViT model that variationally approximates AbS. We show that \model achieves controllable top-down attention and improves over baselines on V\&L tasks as well as image classification and robustness.

\begin{acks}
  We would like to thank the anonymous reviewers for their constructive suggestions and feedback. We also thank Baoquan Chen for various discussions and help. This work was supported in part by start-up grants from Peking University.
\end{acks}

%%
%% The next two lines define the bibliography style to be used, and
%% the bibliography file.
\bibliographystyle{ACM-Reference-Format}
\bibliography{gesture}

\clearpage

\appendix

\section{Details of User Study}
\label{sec:details_of_user_study}
A comparison pair contains two $10$-second videos that are played in order from left to right. We generate each pair using the same speech and character model. The user study questionnaires are created using the Human Behavior Online (HBO) tool offered by the Credamo platform \cite{credamo}. This tool is designed to conduct psychological experiment sample collection without complex programming. All tests and questionnaires are composed of $24$ video pairs. An experiment takes an average of $10$ minutes to complete. We recruit participants from the US and China through Credamo. Participants who take tests with sound are required to speak English fluently. Following \cite{alexanderson2022diffusiongesture}, an attention check is randomly introduced in the experiment to screen valid samples. Specifically, a text message: \emph{attention: please select the rightmost option} appears both at the bottom of the video pair for the whole duration of the question and in the video during the transition gap between the two clips. Samples that fail the attention check are not used for final results.

\subsection{Motion-Quality Study}

\subsubsection{BEAT Dataset}
We select 24 speech segments from the BEAT dataset's test set to create gestures, generating 24 video clips for each method. We compare four approaches: GT, Ours, Ours (w/o transcript), and CaMN, leading to 12 potential pairwise combinations for side-by-side demonstrations. This results in a total of 288 video pairs (24 speech samples $\times$ 12 combinations). Each participant is asked to assess 24 video pairs, encompassing all 24 speech samples. Each of the 12 possible comparison combinations appears twice. The speech samples and pairing of comparisons are randomized for each participant.

\subsubsection{ZeroEGGS Dataset}
\label{subsubsec:motion-quality_study_ZeroEGGS}
We select $6$ audio clips from the ZeroEGGS test recordings in neutral style (\emph{003\_Neutral\_2} and \emph{004\_Neutral\_3}) to synthesize gestures based on $4$ different styles (\emph{happy}, \emph{sad}, \emph{angry}, and \emph{old}), yielding $24$ video clips for each system. We compare two systems, ZE and Ours, in this study. During the assessment, each participant encounters all 24 video clips ($6$ audio clips $\times$ $4$ styles) once in a randomized order, with the outcomes of ZE and Ours evenly distributed in the front position of each video pair.

\subsection{Style-Control Study on the ZeroEGGS Dataset}
The configurations for the \emph{style correctness (w/ dataset label)} and \emph{style correctness (w/ random prompt)} tests are identical, except for the input text prompt. 
In each study, we use the same $6$ audio clips from experiment \ref{subsubsec:motion-quality_study_ZeroEGGS} to generate motions conditioned on $4$ text prompts. This leads to 24 video clips for each system, MD-ZE and Ours, in each test. 
For the \emph{style correctness (w/ dataset label)}, the text prompts are: $\{$\emph{the person is happy}, \emph{the person is sad}, \emph{the person is angry}, \emph{an old person is gesticulating}$\}$. For the \emph{style correctness (w/ random prompt)} test, the text prompts are: $\{$\emph{Hip-hop rapper}, \emph{holding a cup with the right hand}, \emph{looking around}, \emph{a person just lost job}$\}$.
Again, each participant evaluates these $24\times{}2$ video clips in pairs in a randomized sequentially, where the resulting motion of MD-ZE and Ours evenly distributed in the front position of each video pair.


\section{Implementation Details of Baselines}
\label{sec:implementation_details_of_baselines}
%
\begin{figure}[t]
    \centering
    \includegraphics[width=\linewidth]{figures/chatGPT_prompts.pdf}
    \caption{Prompt inputs to ChatGPT \cite{openai2022chatgpt}.}
    \Description{}
    \label{fig:chatGPT_prompts}
\end{figure}
%
At the time of writing this work, the authors of CaMN \cite{liu2021beatdataset} have not provided the pre-trained generation model. Instead, they offered training codes for a toy dataset and a pre-trained motion auto-encoder for the calculation of FGD. We run the provided training codes on a larger dataset used in the original paper, discarding the unreleased emotion label from the conditions of the model. The FGD value of the reproduced model is $122.5$, which is close to the value reported in the original paper ($123.7$). The visual quality of gestures synthesized by the reproduced model is similar to that shown in the video demo of CaMN. We then follow the configuration above and train a new CaMN model on a part of the BEAT dataset used in our work (Section \ref{subsec:system_setup}). This model achieves better performance on FGD ($110.23$) and is utilized as the baseline in this paper.

For the baseline MD-ZE on the ZeroEggs dataset, the two components of this baseline, i.e., MotionDiffuse (MD) \cite{zhang2022motiondiffuse} and ZeroEGGS (ZE) \cite{ghorbani2022zeroeggs}, are constructed using the official pre-trained models. Note that the skeletons of the two models are different. We thus retarget the motion prompt generated by MD to fit the interface of ZE. Specifically, we first convert the generated motion prompt, represented as SMPL \cite{loper2015smpl} joint positions, into joint rotations and save them as a BVH file. Then, we retarget the prompt in SMPL skeleton to the ZeroEGGS skeleton using a Blender add-on, \emph{BVH Retargeter} \cite{padovani2020bvhretargeter}.

\section{Prompts for ChatGPT}
\label{sec:prompts_for_chatgpt}
\fig\ref{fig:chatGPT_prompts} demonstrates the prompt inputs for ChatGPT \cite{openai2022chatgpt} in Section \ref{subsec:application}.

\end{document}
\endinput
%%