
\begin{figure*}[t]
  \centering
  % \begin{tabular}{@{\hspace{-5pt}} c @{\hspace{-20pt}}   }
  %   {\includegraphics[width=1.5\columnwidth]{Figure/abla_close_broken.pdf}}\\
  %   {\small (a) close-set detection} \\
  %     \includegraphics[width=1.5\columnwidth]{Figure/abla_open.pdf} \\
  %    { \small (b) open-set detection}
  % \end{tabular}
  \begin{tabular}{@{\hspace{-10pt}} c @{\hspace{-30pt}}  c }
    {\includegraphics[width=1.16\columnwidth]{Figure/abla_close_broken.pdf}}&
      \includegraphics[width=1.16\columnwidth]{Figure/abla_open.pdf} \\
    {\small (a) close-set detection} &
     { \small (b) open-set detection}
  \end{tabular}
  % \medskip
  \caption{ The AP on LVIS v1.0 over categories sorted by frequency in ascending order.  (a) uses on the close-set setting, only containing 461 ``common''  categories and 405 ``frequent'' categories. (b) uses the open-set setting, containing containing 337 ``rare'' categories, 461 ``common''  categories and 405 ``frequent'' categories. The value is smoothed using moving average with window [-10,10].\label{abla_imbala}}
\end{figure*}

%   \begin{tabular}{@{\hspace{-5pt}} c @{\hspace{-20pt}}  c }
  %   {\includegraphics[width=1.14\columnwidth]{Figure/abla_close.pdf}}&
  %     \includegraphics[width=1.14\columnwidth]{Figure/abla_open.pdf} \\
  %   {\small (a) close-set detection} &
  %    { \small (b) open-set detection}
  % \end{tabular}