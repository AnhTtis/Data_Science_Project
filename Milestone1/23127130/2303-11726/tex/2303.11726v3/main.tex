
%% bare_jrnl_compsoc.tex
%% V1.4b
%% 2015/08/26
%% by Michael Shell
%% See:
%% http://www.michaelshell.org/
%% for current contact information.
%%
%% This is a skeleton file demonstrating the use of IEEEtran.cls
%% (requires IEEEtran.cls version 1.8b or later) with an IEEE
%% Computer Society journal paper.
%%
%% Support sites:
%% http://www.michaelshell.org/tex/ieeetran/
%% http://www.ctan.org/pkg/ieeetran
%% and
%% http://www.ieee.org/

%%*************************************************************************
%% Legal Notice:
%% This code is offered as-is without any warranty either expressed or
%% implied; without even the implied warranty of MERCHANTABILITY or
%% FITNESS FOR A PARTICULAR PURPOSE! 
%% User assumes all risk.
%% In no event shall the IEEE or any contributor to this code be liable for
%% any damages or losses, including, but not limited to, incidental,
%% consequential, or any other damages, resulting from the use or misuse
%% of any information contained here.
%%
%% All comments are the opinions of their respective authors and are not
%% necessarily endorsed by the IEEE.
%%
%% This work is distributed under the LaTeX Project Public License (LPPL)
%% ( http://www.latex-project.org/ ) version 1.3, and may be freely used,
%% distributed and modified. A copy of the LPPL, version 1.3, is included
%% in the base LaTeX documentation of all distributions of LaTeX released
%% 2003/12/01 or later.
%% Retain all contribution notices and credits.
%% ** Modified files should be clearly indicated as such, including  **
%% ** renaming them and changing author support contact information. **
%%*************************************************************************


% *** Authors should verify (and, if needed, correct) their LaTeX system  ***
% *** with the testflow diagnostic prior to trusting their LaTeX platform ***
% *** with production work. The IEEE's font choices and paper sizes can   ***
% *** trigger bugs that do not appear when using other class files.       ***                          ***
% The testflow support page is at:
% http://www.michaelshell.org/tex/testflow/


\documentclass[10pt,journal,compsoc]{IEEEtran}
%
% If IEEEtran.cls has not been installed into the LaTeX system files,
% manually specify the path to it like:
% \documentclass[10pt,journal,compsoc]{../sty/IEEEtran}


\usepackage{ifpdf}


% *** CITATION PACKAGES ***
%
\ifCLASSOPTIONcompsoc
  % IEEE Computer Society needs nocompress option
  % requires cite.sty v4.0 or later (November 2003)
  \usepackage[nocompress]{cite}
\else
  % normal IEEE
  \usepackage{cite}
\fi



% *** GRAPHICS RELATED PACKAGES ***
%
\ifCLASSINFOpdf
  % \usepackage[pdftex]{graphicx}
  % declare the path(s) where your graphic files are
  % \graphicspath{{../pdf/}{../jpeg/}}
  % and their extensions so you won't have to specify these with
  % every instance of \includegraphics
  % \DeclareGraphicsExtensions{.pdf,.jpeg,.png}
\else
  % or other class option (dvipsone, dvipdf, if not using dvips). graphicx
  % will default to the driver specified in the system graphics.cfg if no
  % driver is specified.
  % \usepackage[dvips]{graphicx}
  % declare the path(s) where your graphic files are
  % \graphicspath{{../eps/}}
  % and their extensions so you won't have to specify these with
  % every instance of \includegraphics
  % \DeclareGraphicsExtensions{.eps}
\fi
% graphicx was written by David Carlisle and Sebastian Rahtz. It is
% required if you want graphics, photos, etc. graphicx.sty is already
% installed on most LaTeX systems. The latest version and documentation
% can be obtained at: 
% http://www.ctan.org/pkg/graphicx
% Another good source of documentation is "Using Imported Graphics in
% LaTeX2e" by Keith Reckdahl which can be found at:
% http://www.ctan.org/pkg/epslatex
%
% latex, and pdflatex in dvi mode, support graphics in encapsulated
% postscript (.eps) format. pdflatex in pdf mode supports graphics
% in .pdf, .jpeg, .png and .mps (metapost) formats. Users should ensure
% that all non-photo figures use a vector format (.eps, .pdf, .mps) and
% not a bitmapped formats (.jpeg, .png). The IEEE frowns on bitmapped formats
% which can result in "jaggedy"/blurry rendering of lines and letters as
% well as large increases in file sizes.
%
% You can find documentation about the pdfTeX application at:
% http://www.tug.org/applications/pdftex




\hyphenation{op-tical net-works semi-conduc-tor}
\usepackage[table, dvipsnames]{xcolor}
\usepackage{amsmath,amsfonts}
\usepackage{algorithmic}
\usepackage{array}
\usepackage[caption=false,font=normalsize,labelfont=sf,textfont=sf]{subfig}
\usepackage{textcomp}
\usepackage{stfloats}
\usepackage{url}
\usepackage{verbatim}
\usepackage{graphicx}
\usepackage{subfiles}
\usepackage{bm}
\usepackage{amssymb}
\usepackage{xspace}
\usepackage{hyperref}
\usepackage{marvosym}
\usepackage{array}
\usepackage{multirow}
\usepackage{caption} 

\definecolor{mygray}{gray}{.95}
\definecolor{mylightergray}{gray}{.99}
\definecolor{mygreen}{RGB}{10, 179, 33}
% ------ marker settings ------ 
\usepackage{pifont}
\newcommand{\cmark}{\text{\ding{51}}}
\newcommand{\xmark}{\text{\ding{55}}}
\newcommand{\uniarrow}{\text{\ding{213}}}
\newcommand{\biarrow}{\text{\ding{214}}}
\newcommand{\diff}[1]{\color{NavyBlue}\footnotesize#1}

\newcommand{\xiaoxuan}[1]{\textcolor{red}{\textbf{\small [#1 --xiaoxuan]}}}
\newcommand{\xuyuan}[1]{\textcolor{red}{\textbf{\small [#1 --xuyuan]}}}

% \captionsetup[table]{skip=10pt}
\renewcommand\arraystretch{1.15}

\begin{document}

\newcommand{\discreteVectorField}{\mathcal{V}}
\newcommand{\numberProcesses}{n_p}
\newcommand{\ghostLayer}{\mathcal{G}}
\newcommand{\domain}{\mathcal{M}}
\newcommand{\range}{\mathbb{R}}
\newcommand{\sublevelset}[1]{#1^{-1}_{-\infty}}
\newcommand{\superlevelset}[1]{#1^{-1}_{+\infty}}
\newcommand{\Star}{St}
\newcommand{\Link}{Lk}
\newcommand{\simplex}{\sigma}
\newcommand{\face}{\tau}
\newcommand{\lowerlink}{\Link^{-}}
\newcommand{\upperlink}{\Link^{+}}
\newcommand{\Index}{\mathcal{I}}
\newcommand{\offset}{o}
\newcommand{\Natural}{\mathbb{N}}
\newcommand{\criticalSet}{\mathcal{C}}
\newcommand{\diagram}{\mathcal{D}}
\newcommand{\wasserstein}[1]{W^{\diagram}_#1}
\newcommand{\projection}{\Delta}
\newcommand{\hierarchy}{\mathcal{H}}
\newcommand{\decimation}{D}
\newcommand{\xDimD}{L_x^\decimation}
\newcommand{\yDimD}{L_y^\decimation}
\newcommand{\zDimD}{L_z^\decimation}
\newcommand{\xDim}{L_x}
\newcommand{\yDim}{L_y}
\newcommand{\zDim}{L_z}
\newcommand{\Grid}{\mathcal{G}}
\newcommand{\GridD}{\mathcal{G}^\decimation}
\newcommand{\x}{\phantom{x}}
\newcommand{\Mod}{\;\mathrm{mod}\;}
\newcommand{\NN}{\mathbb{N}}
\newcommand{\forwardIntegralLine}{\mathcal{L}^+}
\newcommand{\backwardIntegralLine}{\mathcal{L}^-}
\newcommand{\triangulationOp}{\phi}
\newcommand{\decimationOp}{\Pi}
\newcommand{\isovalue}{w}
\newcommand{\persistence}{\mathcal{P}}
\newcommand{\pointMetric}{d}
\newcommand{\diagramSet}{\mathcal{S}_\mathcal{D}}
\newcommand{\diagramSpace}{\mathbb{D}}
\newcommand{\jointree}{\mathcal{T}^-}
\newcommand{\splittree}{\mathcal{T}^+}
\newcommand{\mergetree}{\mathcal{T}}
\newcommand{\mergetreeSet}{\mathcal{S}_\mathcal{T}}
\newcommand{\branchset}{\mathcal{S}_\mathcal{B}}
\newcommand{\branchspace}{\mathbb{B}}
\newcommand{\mergetreeSpace}{\mathbb{T}}
\newcommand{\editdistance}{D_E}
\newcommand{\wassersteinTree}{W^{\mergetree}_2}
\newcommand{\distanceSequence}{d_S}
\newcommand{\branchtree}{\mathcal{B}}
\newcommand{\branchtreeSet}{\mathcal{S}_\mathcal{B}}
\newcommand{\branchtreeSpace}{\mathbb{B}}
\newcommand{\forest}{\mathcal{F}}
\newcommand{\sequenceSpace}{\mathbb{S}}
\newcommand{\forestMatrix}{\mathbb{F}}
\newcommand{\treeMatrix}{\mathbb{T}}
\newcommand{\normalizedLocation}{\mathcal{N}}
\newcommand{\normalizedWasserstein}{W^{\normalizedLocation}_2}
\newcommand{\geodesictree}{\mathcal{G}}
\newcommand{\dummyVector}{\mathcal{V}}
\newcommand{\geodesictreeVec}{g}
\newcommand{\geodesicAxis}{\mathcal{A}}
\newcommand{\directionVector}{\mathcal{V}}
\newcommand{\geodesicdiagram}{\mathcal{G}^{\diagram}}
\newcommand{\reconstructionError}{E_{L_2}}
\newcommand{\pcaBasis}{B_{\mathbb{R}^d}}
\newcommand{\mtPgaBasis}{B_{\branchtreeSpace}}
\newcommand{\mtPgaError}{E_{\wassersteinTree}}
\newcommand{\frechetEnergy}{E_F}
\newcommand{\geodesicExtremity}{\mathcal{E}}
\newcommand{\vectorNotation}[1]{\protect\vv{#1}}
\newcommand{\axisNotation}[1]{\protect\overleftrightarrow{#1}}
% \newcommand{\axisNotation}[1]{\vectorNotation{#1}}
\newcommand{\individualEnergy}{E}
\newcommand{\ensembleSize}{N}
\newcommand{\numberBranchinBarycenter}{N_1}
\newcommand{\numberGeodesicSamples}{N_2}
\newcommand{\planarGridX}{N_x}
\newcommand{\planarGridY}{N_y}
\newcommand{\regularGrid}{G}
\newcommand{\distanceMatrix}{\mathbb{D}}
\newcommand{\maxDimensions}{{d_{max}}}
\newcommand{\projectionOperator}{\mathcal{P}}
\newcommand{\reconstructed}[1]{\widehat{#1}}
\newcommand{\gt}{>}
\newcommand{\lt}{<}
% \newcommand{\normalizedTree}{N}
\newcommand{\branch}{b}



%\newcommand{\revision}[1]{\textcolor{blue}{#1}}
% \newcommand{\revision}[1]{\textcolor{black}{#1}}
% \newcommand{\minorRevision}[1]{\textcolor{blue}{#1}}

\renewcommand{\figureautorefname}{Fig.}
\renewcommand{\sectionautorefname}{Sec.}
\renewcommand{\subsectionautorefname}{Sec.}
\renewcommand{\subsubsectionautorefname}{Sec.}
\renewcommand{\equationautorefname}{Eq.}
\renewcommand{\tableautorefname}{Tab.}
\newcommand{\algorithmautorefname}{Alg.}
\newcommand{\lineautorefname}{Alg.}

\newcommand{\eqSpace}{-1.75ex}

\newcommand{\mycaption}[1]{
%\vspace{-3.5ex}
\caption{#1}
%\vspace{-3ex}
}
%


\title{\vmproname: Probabilistic 3D Human Mesh Estimation from Virtual Markers}

\author{Xiaoxuan~Ma,~
        Jiajun~Su,~
        Yuan~Xu,~
        Wentao~Zhu,~
        Chunyu~Wang\textsuperscript{\Letter},~
        and~Yizhou~Wang\textsuperscript{\Letter}% <-this % stops a space
\IEEEcompsocitemizethanks{\IEEEcompsocthanksitem Xiaoxuan Ma, Yuan Xu and Wentao Zhu are with Center on Frontiers of Computing Studies, School of Computer Science, Peking University.
% 
E-mail: \{maxiaoxuan, wtzhu\}@pku.edu.cn, \{xuyuan\}@stu.pku.edu.cn.\
\IEEEcompsocthanksitem Jiajun Su is with International Digital Economy Academy (IDEA).
%
E-mail: sujiajun@idea.edu.cn.
\IEEEcompsocthanksitem Chunyu Wang is with Microsoft Research Asia.
%
E-mail: chnuwa@microsoft.com.
%
\IEEEcompsocthanksitem Yizhou Wang is with Center on Frontiers of Computing Studies, School of Compter Science, Peking University, and with Inst. for Artificial Intelligence, Peking University, and with Nat'l Eng. Research Center of Visual Technology, and also with Nat'l Key Lab of General Artificial Intelligence, Peking University.
%
E-mail: yizhou.wang@pku.edu.cn.}
\thanks{
\textsuperscript{\Letter} Corresponding authors: Chunyu Wang and Yizhou Wang.
}}

% The paper headers
% \markboth{Journal of \LaTeX\ Class Files,~Vol.~14, No.~8, August~2015}%
% {Shell \MakeLowercase{\textit{et al.}}: Bare Demo of IEEEtran.cls for Computer Society Journals}
% The only time the second header will appear is for the odd numbered pages
% after the title page when using the twoside option.
% 
% *** Note that you probably will NOT want to include the author's ***
% *** name in the headers of peer review papers.                   ***
% You can use \ifCLASSOPTIONpeerreview for conditional compilation here if
% you desire.

\IEEEtitleabstractindextext{%
\begin{abstract}
Monocular 3D human mesh estimation faces challenges due to depth ambiguity and the complexity of mapping images to complex parameter spaces. Recent methods propose to use 3D poses as a proxy representation, which often lose crucial body shape information, leading to mediocre performance. Conversely, advanced motion capture systems, though accurate, are impractical for markerless wild images. Addressing these limitations, we introduce an innovative intermediate representation as \textit{virtual markers}, which are learned from large-scale mocap data, mimicking the effects of physical markers.
Building upon virtual markers, we propose \vmname, which detects virtual markers from wild images, and the intact mesh with realistic shapes can be obtained by simply interpolation from these markers. To address occlusions that obscure 3D virtual marker estimation, we further enhance our method with \vmproname, a probabilistic framework that generates multiple plausible meshes aligned with images. This framework models the 3D virtual marker estimation as a conditional denoising process, enabling robust 3D mesh estimation.
Our approaches surpass existing methods on three benchmark datasets, particularly demonstrating significant improvements on the SURREAL dataset, which features diverse body shapes. Additionally, \vmproname\ excels in accurately modeling data distributions, significantly enhancing performance in occluded scenarios.
Code and models are available at \projpage. 
\end{abstract}

\begin{IEEEkeywords}
Computer vision, 3D human mesh estimation, probabilistic modeling
\end{IEEEkeywords}}

\maketitle

\begin{figure}[t!]
    \centering
    \includegraphics[width=\linewidth]{imgs/teaser.pdf}
    \caption{\textbf{Top:} Mesh estimation results on examples with different body shapes. Pose2Mesh \cite{choi2020pose2mesh} which uses 3D skeletons as the intermediate representation fails to predict accurate shapes. Our virtual marker-based method \vmname\ obtains accurate estimates. \textbf{Bottom:} \vmname\ may fail when occlusion happens, \eg the occluded left arm is wrongly estimated. To handle the ambiguity, we propose a probabilistic framework \vmproname\ which could generate multiple reasonable estimates aligning with the image.}
    \label{fig:teaser}
\end{figure}

\IEEEdisplaynontitleabstractindextext

\subfile{sections/1introduction}

\subfile{sections/2related}

\subfile{sections/3method}

\subfile{sections/4experiment}

\subfile{sections/5conclusion}


\vspace{2em}
% use section* for acknowledgment
\ifCLASSOPTIONcompsoc
  % The Computer Society usually uses the plural form
  \section*{Acknowledgments}
\else
  % regular IEEE prefers the singular form
  \section*{Acknowledgment}
\fi
This work was supported by National Science and Technology Major Project (2022ZD0114904).

\clearpage

\bibliographystyle{IEEEtran}
\bibliography{IEEEabrv,reference}

\newpage

\subfile{sections/6supplementary}


\end{document}


