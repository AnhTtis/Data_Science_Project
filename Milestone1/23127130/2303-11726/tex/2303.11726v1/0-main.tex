\documentclass[10pt,twocolumn,letterpaper]{article}

% \usepackage[review]{cvpr}      % To produce the REVIEW version
%\usepackage[rebuttal]{cvpr}    % To produce a REBUTTAL
\usepackage{cvpr}              % To produce the CAMERA-READY version
%\usepackage[pagenumbers]{cvpr} % To force page numbers, e.g. for an arXiv version

\usepackage[table, dvipsnames]{xcolor}
\usepackage{times}
\usepackage{epsfig}
\usepackage{graphicx}
\usepackage{amsmath}
\usepackage{amssymb}
\usepackage{subfiles}
\usepackage[accsupp]{axessibility}
\usepackage{makecell} 
\usepackage{wrapfig} 
\usepackage{graphicx}
\usepackage[utf8]{inputenc} % allow utf-8 input
\usepackage[T1]{fontenc}    % use 8-bit T1 fonts
\usepackage{url}            % simple URL typesetting
\usepackage{booktabs}       % professional-quality tables
\usepackage{amsfonts}       % blackboard math symbols
\usepackage{nicefrac}       % compact symbols for 1/2, etc.
\usepackage{microtype}      % microtypography
\usepackage{bm}
%\usepackage{xurl}

% ------ table settings ------ 
\usepackage{array}
\usepackage{multirow}

\newlength\savewidth\newcommand\shline{\noalign{\global\savewidth\arrayrulewidth
  \global\arrayrulewidth 1pt}\hline\noalign{\global\arrayrulewidth\savewidth}}
  
\newcommand{\tablestyle}[2]{\setlength{\tabcolsep}{#1}\renewcommand{\arraystretch}{#2}\centering\footnotesize}

\usepackage{xspace}
\newcommand{\cy}[1]{\textcolor{blue}{\small [#1 --chunyu]}}
\newcommand{\xiaoxuan}[1]{\textcolor{red}{\textbf{\small [#1 --xiaoxuan]}}}
\newcommand{\wentao}[1]{\textcolor{blue}{\textbf{\small [#1 --wentao]}}}
\newcommand{\diff}[1]{\color{NavyBlue}\footnotesize#1}

\usepackage{xspace}

% Add a period to the end of an abbreviation unless there's one
% already, then \xspace.
\makeatletter
\DeclareRobustCommand\onedot{\futurelet\@let@token\@onedot}
\def\@onedot{\ifx\@let@token.\else.\null\fi\xspace}

\def\eg{\emph{e.g}\onedot} \def\Eg{\emph{E.g}\onedot}
\def\ie{\emph{i.e}\onedot} \def\Ie{\emph{I.e}\onedot}
\def\cf{\emph{c.f}\onedot} \def\Cf{\emph{C.f}\onedot}
\def\etc{\emph{etc}\onedot} \def\vs{\emph{vs}\onedot}
\def\wrt{w.r.t\onedot} \def\dof{d.o.f\onedot}
\def\etal{\emph{et al}\onedot}
\makeatother

\newcolumntype{S}{>{\centering\arraybackslash}m{0.9cm}}
\newcolumntype{M}{>{\centering\arraybackslash}m{1.2cm}}
\newcolumntype{L}{>{\centering\arraybackslash}m{1.4cm}}
\definecolor{mygray}{gray}{.95}
\definecolor{mylightergray}{gray}{.99}
\definecolor{mygreen}{RGB}{10, 179, 33}
\usepackage{caption} 
\captionsetup[table]{skip=10pt}
\renewcommand\arraystretch{1.15}
\makeatletter
\newcommand{\thickhline}{%
    \noalign {\ifnum 0=`}\fi \hrule height 1pt
    \futurelet \reserved@a \@xhline
}
\newcolumntype{"}{@{\vrule width 1pt}}

% \usepackage[pagebackref,breaklinks,colorlinks]{hyperref}

\definecolor{mygray}{gray}{.95}
\definecolor{mylightergray}{gray}{.99}
\definecolor{mygreen}{RGB}{10, 179, 33}

% ------ marker settings ------ 
\usepackage{pifont}
\newcommand{\cmark}{\text{\ding{51}}}
\newcommand{\xmark}{\text{\ding{55}}}
\newcommand{\uniarrow}{\text{\ding{213}}}
\newcommand{\biarrow}{\text{\ding{214}}}

\usepackage{hyperref}
\hypersetup{colorlinks,breaklinks}

% Support for easy cross-referencing
\usepackage[capitalize]{cleveref}
\crefname{section}{Sec.}{Secs.}
\Crefname{section}{Section}{Sections}
\Crefname{table}{Table}{Tables}
\crefname{table}{Tab.}{Tabs.}


%%%%%%%%% PAPER ID  - PLEASE UPDATE
\def\confName{CVPR}
\def\confYear{2023}



\begin{document}

%%%%%%%%% TITLE - PLEASE UPDATE
\title{3D Human Mesh Estimation from Virtual Markers}


\author{Xiaoxuan Ma\textsuperscript{1} \quad Jiajun Su\textsuperscript{1} 
\quad Chunyu Wang \textsuperscript{3 \thanks{Corresponding author}}  \quad Wentao Zhu\textsuperscript{1} \quad 
 Yizhou Wang\textsuperscript{1, 2, 4} \\[1.5ex]
    \textsuperscript{1~}School of Computer Science, Center on Frontiers of Computing Studies, Peking University \\
    \textsuperscript{2~}Inst. for Artificial Intelligence, Peking University\\
    \textsuperscript{3~}Microsoft Research Asia\\
    \textsuperscript{4~}Nat'l Eng. Research Center of Visual Technology\\[1.1ex]
{\tt\small \{maxiaoxuan, sujiajun, wtzhu, yizhou.wang\}@pku.edu.cn, chnuwa@microsoft.com}\\	
}
\maketitle



\begin{abstract}

Inspired by the success of volumetric 3D pose estimation, some recent human mesh estimators propose to estimate 3D skeletons as intermediate representations, from which, the dense 3D meshes are regressed by exploiting the mesh topology. 
However, body shape information is lost in extracting skeletons, leading to mediocre performance. The advanced motion capture systems solve the problem by placing dense physical markers on the body surface, which allows to extract realistic meshes from their non-rigid motions. However, they cannot be applied to wild images without markers. In this work, we present an intermediate representation, named virtual markers, which learns 64 landmark keypoints on the body surface based on the large-scale mocap data in a generative style, mimicking the effects of physical markers. The virtual markers can be accurately detected from wild images and can reconstruct the intact meshes with realistic shapes by simple interpolation. Our approach outperforms the state-of-the-art methods on three datasets. In particular, it surpasses the existing methods by a notable margin on the SURREAL dataset, which has diverse body shapes. Code is available at \url{https://github.com/ShirleyMaxx/VirtualMarker}. 


\end{abstract}

\section{Introduction}
\subfile{1introduction}

\section{Related work}
\subfile{2relatedwork}

\section{Method}
\subfile{3method}


\section{Experiments}
\subfile{4experiments}

\section{Conclusion}
\subfile{5conclusion}

\section*{Acknowledgement}  
This work was supported by MOST-2022ZD0114900 and NSFC-62061136001.

\clearpage

{\small
\bibliographystyle{ieee_fullname}
\bibliography{egbib}
}

\newpage
\section*{Appendix}
\subfile{7supplement_arxiv}

\end{document}
