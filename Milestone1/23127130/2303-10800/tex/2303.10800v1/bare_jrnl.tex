
%% bare_jrnl.tex
%% V1.4b
%% 2015/08/26
%% by Michael Shell
%% see http://www.michaelshell.org/
%% for current contact information.
%%
%% This is a skeleton file demonstrating the use of IEEEtran.cls
%% (requires IEEEtran.cls version 1.8b or later) with an IEEE
%% journal paper.
%%
%% Support sites:
%% http://www.michaelshell.org/tex/ieeetran/
%% http://www.ctan.org/pkg/ieeetran
%% and
%% http://www.ieee.org/

%%*************************************************************************
%% Legal Notice:
%% This code is offered as-is without any warranty either expressed or
%% implied; without even the implied warranty of MERCHANTABILITY or
%% FITNESS FOR A PARTICULAR PURPOSE! 
%% User assumes all risk.
%% In no event shall the IEEE or any contributor to this code be liable for
%% any damages or losses, including, but not limited to, incidental,
%% consequential, or any other damages, resulting from the use or misuse
%% of any information contained here.
%%
%% All comments are the opinions of their respective authors and are not
%% necessarily endorsed by the IEEE.
%%
%% This work is distributed under the LaTeX Project Public License (LPPL)
%% ( http://www.latex-project.org/ ) version 1.3, and may be freely used,
%% distributed and modified. A copy of the LPPL, version 1.3, is included
%% in the base LaTeX documentation of all distributions of LaTeX released
%% 2003/12/01 or later.
%% Retain all contribution notices and credits.
%% ** Modified files should be clearly indicated as such, including  **
%% ** renaming them and changing author support contact information. **
%%*************************************************************************


% *** Authors should verify (and, if needed, correct) their LaTeX system  ***
% *** with the testflow diagnostic prior to trusting their LaTeX platform ***
% *** with production work. The IEEE's font choices and paper sizes can   ***
% *** trigger bugs that do not appear when using other class files.       ***                          ***
% The testflow support page is at:
% http://www.michaelshell.org/tex/testflow/



\documentclass[journal]{IEEEtran}

%\pdfoutput=1

% *** CITATION PACKAGES ***
%
%\usepackage{cite}
% cite.sty was written by Donald Arseneau
% V1.6 and later of IEEEtran pre-defines the format of the cite.sty package
% \cite{} output to follow that of the IEEE. Loading the cite package will
% result in citation numbers being automatically sorted and properly
% "compressed/ranged". e.g., [1], [9], [2], [7], [5], [6] without using
% cite.sty will become [1], [2], [5]--[7], [9] using cite.sty. cite.sty's
% \cite will automatically add leading space, if needed. Use cite.sty's
% noadjust option (cite.sty V3.8 and later) if you want to turn this off
% such as if a citation ever needs to be enclosed in parenthesis.
% cite.sty is already installed on most LaTeX systems. Be sure and use
% version 5.0 (2009-03-20) and later if using hyperref.sty.
% The latest version can be obtained at:
% http://www.ctan.org/pkg/cite
% The documentation is contained in the cite.sty file itself.

%\usepackage{url}
%\usepackage{subcaption} % NAI
\usepackage{xcolor}
\usepackage{amsmath,amsfonts,amssymb}
\usepackage{graphicx}
\usepackage{booktabs}
\usepackage{multirow}
\usepackage{enumitem}
\usepackage{comment}
\usepackage{arydshln} % for dashed lines in tables
\usepackage{xfrac}
\usepackage{hyperref}

%\usepackage[utf8]{inputenc}
%\usepackage[english]{babel}
%\usepackage{minted}
\usepackage{listings}

\definecolor{codegreen}{rgb}{0,0.6,0}
\definecolor{codegray}{rgb}{0.5,0.5,0.5}


% correct bad hyphenation here
\hyphenation{op-tical net-works semi-conduc-tor}


\begin{document}
%
% paper title
% Titles are generally capitalized except for words such as a, an, and, as,
% at, but, by, for, in, nor, of, on, or, the, to and up, which are usually
% not capitalized unless they are the first or last word of the title.
% Linebreaks \\ can be used within to get better formatting as desired.
% Do not put math or special symbols in the title.
%\title{Designing SAR-ATR Systems for an Open-World Operating Environment} % OLD
\title{A Global Model Approach to Robust Few-Shot SAR Automatic Target Recognition}
%
%
% author names and IEEE memberships
% note positions of commas and nonbreaking spaces ( ~ ) LaTeX will not break
% a structure at a ~ so this keeps an author's name from being broken across
% two lines.
% use \thanks{} to gain access to the first footnote area
% a separate \thanks must be used for each paragraph as LaTeX2e's \thanks
% was not built to handle multiple paragraphs
%

% \author{Michael~Shell,~\IEEEmembership{Member,~IEEE,}
%         John~Doe,~\IEEEmembership{Fellow,~OSA,}
%         and~Jane~Doe,~\IEEEmembership{Life~Fellow,~IEEE}% <-this % stops a space
% \thanks{M. Shell was with the Department
% of Electrical and Computer Engineering, Georgia Institute of Technology, Atlanta,
% GA, 30332 USA e-mail: (see http://www.michaelshell.org/contact.html).}% <-this % stops a space
% \thanks{J. Doe and J. Doe are with Anonymous University.}% <-this % stops a space
% \thanks{Manuscript received April 19, 2005; revised August 26, 2015.}}

%\author{Nathan Inkawhich, Matthew Inkawhich, Eric Davis, Uttam %Majumder, Chris Capraro, Yiran Chen, Eric Branch, and John %Nehrbass}%

\author{Nathan Inkawhich\\Air Force Research Laboratory}%
\IEEEaftertitletext{\vspace{-1\baselineskip}} % NAI: prevent unnecessary space
% note the % following the last \IEEEmembership and also \thanks - 
% these prevent an unwanted space from occurring between the last author name
% and the end of the author line. i.e., if you had this:
% 
% \author{....lastname \thanks{...} \thanks{...} }
%                     ^------------^------------^----Do not want these spaces!
%
% a space would be appended to the last name and could cause every name on that
% line to be shifted left slightly. This is one of those "LaTeX things". For
% instance, "\textbf{A} \textbf{B}" will typeset as "A B" not "AB". To get
% "AB" then you have to do: "\textbf{A}\textbf{B}"
% \thanks is no different in this regard, so shield the last } of each \thanks
% that ends a line with a % and do not let a space in before the next \thanks.
% Spaces after \IEEEmembership other than the last one are OK (and needed) as
% you are supposed to have spaces between the names. For what it is worth,
% this is a minor point as most people would not even notice if the said evil
% space somehow managed to creep in.

% \author{Jingyang Zhang\textsuperscript{\textdagger}, Nathan Inkawhich\textsuperscript{*}, Randolph Linderman\textsuperscript{\textdagger}, Yiran Chen\textsuperscript{\textdagger}, Hai Li\textsuperscript{\textdagger}\\
% \textsuperscript{\textdagger}Duke University, 
% \textsuperscript{*}Air Force Research Laboratory\\
% {\tt\small jingyang.zhang@duke.edu}}


% The paper headers
\markboth{Journal of \LaTeX\ Class Files,~Vol.~14, No.~8, July~2022 (PREPRINT - DRAFT)}%
{Shell \MakeLowercase{\textit{et al.}}: Bare Demo of IEEEtran.cls for IEEE Journals}
% The only time the second header will appear is for the odd numbered pages
% after the title page when using the twoside option.
% 
% *** Note that you probably will NOT want to include the author's ***
% *** name in the headers of peer review papers.                   ***
% You can use \ifCLASSOPTIONpeerreview for conditional compilation here if
% you desire.




% If you want to put a publisher's ID mark on the page you can do it like
% this:
%\IEEEpubid{0000--0000/00\$00.00~\copyright~2015 IEEE}
% Remember, if you use this you must call \IEEEpubidadjcol in the second
% column for its text to clear the IEEEpubid mark.



% use for special paper notices
%\IEEEspecialpapernotice{(Invited Paper)}




% make the title area
\maketitle

% As a general rule, do not put math, special symbols or citations
% in the abstract or keywords.
\begin{abstract}
In real-world scenarios, it may not always be possible to collect hundreds of labeled samples per class for training deep learning-based SAR Automatic Target Recognition (ATR) models. 
This work specifically tackles the few-shot SAR ATR problem, where only a handful of labeled samples may be available to support the task of interest.
Our approach is composed of two stages. 
In the first, a global representation model is trained via self-supervised learning on a large pool of diverse and unlabeled SAR data. 
In the second stage, the global model is used as a fixed feature extractor and a classifier is trained to partition the feature space given the few-shot support samples, while simultaneously being calibrated to detect anomalous inputs.
Unlike competing approaches which require a pristine labeled dataset for pretraining via meta-learning, our approach learns highly transferable features from unlabeled data that have little-to-no relation to the downstream task.
We evaluate our method in standard and extended MSTAR operating conditions and find it to achieve high accuracy and robust out-of-distribution detection in many different few-shot settings.
Our results are particularly significant because they show the merit of a global model approach to SAR ATR, which makes minimal assumptions, and provides many axes for extendability.
\end{abstract}

% Note that keywords are not normally used for peerreview papers.
\begin{IEEEkeywords}
Few Shot Learning, Automatic Target Recognition, Out-of-Distribution Detection, Deep Learning.
\end{IEEEkeywords}






% For peer review papers, you can put extra information on the cover
% page as needed:
% \ifCLASSOPTIONpeerreview
% \begin{center} \bfseries EDICS Category: 3-BBND \end{center}
% \fi
%
% For peerreview papers, this IEEEtran command inserts a page break and
% creates the second title. It will be ignored for other modes.
\IEEEpeerreviewmaketitle




%%%%%%%%%%%%%%%%%%%%%%%%%%%%%%%%%%%%%%%%%%%%%%%%%%%%%%%%%%%
%\newcommand{\G}[1]{{\color{green}{#1}}}
\newcommand{\G}[1]{{\color{codegreen}{#1}}}
\newcommand{\R}[1]{{\color{red}{#1}}}
\newcommand{\M}[1]{\textbf{\color{magenta}{#1}}}
\newcommand{\B}[1]{\textbf{\color{blue}{#1}}}
\newcommand*\rot{\rotatebox{65}}



\section{Introduction} \label{sec:intro}

%%% Background

Most modern Synthetic Aperture Radar (SAR) Automatic Target Recognition (ATR) algorithms use Deep Neural Networks (DNNs) to perform the underlying recognition function \cite{tom_book}. 
While DNNs have shown to be very effective in ideal laboratory-like conditions, they present several challenges when being considered for deployment in ``real-world'' scenarios.
One challenge is that DNNs are data-hungry, and often utilize hundreds to thousands of training samples per class to achieve state-of-the-art accuracy.
Another key challenge is that DNNs are notoriously susceptible to producing erroneous predictions on anomalous/out-of-distribution (OOD) inputs (i.e., inputs that fall outside of the categories in the training distribution). 
These challenges spark two main concerns. 
First, what if we, as the model practitioners, cannot obtain hundreds of labeled samples per class for training, and instead can only come up with $\sim$5 (i.e., a few shots)?
And second, in these few-shot settings, does OOD detection become harder because the models are primarily seeking generalization and have not been given enough information to learn highly detailed/nuanced representations of the in-distribution (ID) classes?

%%% Overview Figure
\begin{figure}[t]
  \centering
  %\vspace{-2mm}
  \includegraphics[width=1.0\linewidth]{figures/overview.png}
  \vspace{-5mm}
  \caption{Overview of our two-stage approach. In Stage 1 we train the feature extractor on a pool of unlabeled data and in Stage 2 we learn a classifier on top of the feature extractor for a given few shot learning problem.}
  \vspace{-4mm}
  \label{fig:overview}
\end{figure}

%%% APPROACH / METHOD

Our goal in this paper is to address the stated concerns by developing an accurate and robust SAR target classifier given only a few labeled samples per class. 
We also strive to keep a minimal assumption set to prioritize applicability and extendability.
Our high level approach involves two stages and is shown in Fig.~\ref{fig:overview}. 
In the first stage, a global representation model is trained on a diverse and unlabeled pool of SAR data such that it is capable of extracting quality features from (nearly) any SAR chip. 
In Stage 2, for an arbitrary downstream (M-way, N-shot) few-shot learning (FSL) task we train a light-weight classifier using the global model as a fixed feature extractor.


A key part of our methodology is how we train the global model.
Importantly, we do \textit{not} assume to have labeled data in this step, which eliminates the option for a meta-learning approach \cite{protonet,dktsn}.
To learn quality representations we instead aggregate publicly available data from recent papers and leverage a modern unsupervised learning paradigm called Self-Supervised Learning (SSL). 
Specifically, we use the SimCLR \cite{ChenK0H20} method to represent the SSL class of algorithms in this work.
%SSL algorithms such as SimCLR 
SSL has shown great promise for transfer and few-shot learning in the natural imagery domain \cite{goodembedding_fsl}, but has not been studied widely in the context of SAR ATR.
Our hypothesis is that these modern algorithms can learn a highly flexible~/~expressive~/~useful~/~transferable feature extractor from unlabeled SAR data sourced from different sensors, imaging modes, polarizations, resolutions, and target types. Further, the unlabeled data does not necessarily have to represent the anticipated downstream conditions.
This is contrary to several lines of concurrent research whose goal is to generate high-fidelity synthetic imagery as close to the downstream task conditions as possible for supervised pre-training \cite{sarsim_dataset}.


The second key part of our methodology is how we improve robustness to OOD inputs.
This is accomplished in the classifier training stage and given our stated operating environment it is nearly a ``free-lunch.''
Practically, we can use the SSL model's pretraining dataset as an OOD dataset for outlier exposure (OE) training \cite{HendrycksOE, inkawhich_advoe}. 
Intuitively, the classifier is taught to be accurate and confident on labeled task data while being minimally confident on the OOD data.
This objective creates a confidence calibration effect which can be exploited to detect novel OOD samples during deployment \cite{inkawhich_onlineOOD}.


%%% Experiments & Contributions

Using MSTAR \cite{Ross1998MSTAR} as our primary few-shot test environment, we find that our methodology can generate highly accuracy models in both standard and extended operating conditions.
We also find that by OE training in Stage 2, our models can reliably detect a spectrum of fine- to coarse-grained OOD types unseen during training. 
Finally, we uncover and discuss an important trade-off between OOD generalization and detection. 
Overall, we make the following contributions:
\begin{itemize}
    \item We show a proof-of-concept that a highly transferable global SAR feature extractor can be trained without labels on diverse data and used to great effect in downstream FSL tasks;
    \item We develop a robust FSL classification scheme that can reliably detect and reject OOD inputs at test time without adding overhead or assumptions;
    \item We uncover a critical trade-off between OOD detection and generalization that directly motivates future work.
\end{itemize}


%%% Random thoughts:

%- Few-shot SAR-ATR is also particularly challenging because the target signatures can change drastically given minimal shifts in viewing angle and any target motion.

%- Mention that we are not aware of many other FSL approaches that do not require labeled pretraining data from a domain relevant dataset.

%%% End







\section{Methodology}

%%%%%%%%%%%%%%%%%%%%%%%%%%%%%%%%%%%%%%%%%%%%%%%%%%%%%%%%%%%%%
\subsection{Global Model Training}

%%%%%%%%%%%%%%%%%%%%%%%%%%%%%%%
\subsubsection{Data Collection}

Our global model approach is highly data driven. 
In order to learn a generalized and transferable SAR feature extractor, intuition says we must train it on data from different sensors, imaging modes, polarizations, resolutions, target/scene types, etc.
Thus, the first step in our methodology is to aggregate data from the following public sources: SAR-Ships \cite{sarship_dataset}; HRSID \cite{hrsid_dataset}; FUSAR \cite{fusar_dataset}; SSDD \cite{ssdd_dataset}; LS-SSDD \cite{lsssdd_dataset}; Dual-Pol Ships \cite{dualpolships_dataset}; SRSDD \cite{srsdd_dataset}; and CVDome \cite{cvdome_dataset}.
After the necessary pre-processing (e.g., chipping from a full frame), we are left with $\sim$100k unlabeled SAR chips which we refer to as $\mathcal{D}_{pretrain}$.
For reference, Fig.~\ref{fig:pretrain_data} shows a handful of samples from this set. 
An important note is that $90\%$ of $\mathcal{D}_{pretrain}$ chips contain ship-like targets and \textit{none} of the chips contain MSTAR land-vehicle targets. 
Given that MSTAR will be our downstream task for FSL evaluation, this sets up for a true test of generalization.
However, in practice if data resembling the anticipated downstream task were available it could easily be leveraged here.





%%%%%%%%%%%%%%%%%%%%%%%%%%%%%%%
\subsubsection{Representation Learning}

With $\mathcal{D}_{pretrain}$ curated we can execute Stage 1, which is to train the global model feature extractor $f_\phi$. 
We use the powerful SimCLR \cite{ChenK0H20} contrastive learning algorithm, which trains by enforcing that two ``views'' of the same image lie near each-other in a feature space, while views of different images lie far away.
Each view is created via aggressive stochastic augmentations that preserve the underlying semantic features but force the model to learn an informative feature set via minimizing the contrastive objective.  
We specifically leverage the normalized temperature-scaled cross-entropy loss (NT-Xent) from \cite{ChenK0H20}, and refer the reader to that work for a more technical description of the algorithm and \cite{simclr_github} for a PyTorch implementation.




Perhaps the most unique part about the application of SimCLR (and many other SSL algorithms) to the SAR ATR setting is the composition of augmentations needed to create informative views. 
Since the SimCLR algorithm was developed and tested in the natural imagery domain, the standard augmentation pipeline contains transforms that are relevant to RGB images. 
However, several of the key augmentations such as color jittering and random gray-scaling \cite{simclr_github} do not fit with our magnitude-only gray-scale SAR modality.
Thus, we create a new augmentation pipeline made of seven image-level transformations that work to create translation invariance, and to manipulate the signal-to-noise ratio and frequency content of the chips. 
Included are random resized cropping, random horizontal flipping, element-wise power-scaling, Gaussian noising and filtering, and linear re-scaling.
See Supplemental A for more information about our transformation pipeline.
Lastly, we mention that we do not believe our design choices made in this section are ``optimal.'' 
Important areas of future work are to experiment with other SSL algorithms (see Supplemental C) and to develop more domain-relevant augmentations that may lead to higher quality learned features.


%%% D_pretrain Examples Figure
\begin{figure}[t]
  \centering
  %\vspace{-2mm}
  \includegraphics[width=0.96\linewidth]{figures/pretrain_data_small.png}
  \vspace{-1mm}
  \caption{Samples from $\mathcal{D}_{pretrain}$, sourced from: SAR-Ships \cite{sarship_dataset}; HRSID \cite{hrsid_dataset}; FUSAR \cite{fusar_dataset}; SSDD \cite{ssdd_dataset}; LS-SSDD \cite{lsssdd_dataset}; Dual-Pol Ships \cite{dualpolships_dataset}; SRSDD \cite{srsdd_dataset}; and CVDome \cite{cvdome_dataset}.}
  \vspace{-1mm}
  \label{fig:pretrain_data}
\end{figure}

%%% MSTAR SOC Accuracy Table
%%%%%%%%%%%%%%%%%%%%%%%%%%%%%%%%%%%%%%%%%%%%%%% MSTAR SOC RESULTS -- NO ROTATED LABELS
\begin{table*}[t]
\centering
\caption{MSTAR Standard Operating Condition Accuracy Results (with 95\% confidence interval)}
\vspace{-1mm}
\label{tab:mstar_soc}
\resizebox{1.\textwidth}{!}{
\begin{tabular}{lccccccccccccccc}
\toprule
                 & \multicolumn{5}{c}{2-way}                     & \multicolumn{5}{c}{5-way}                     & \multicolumn{5}{c}{10-way}                    \\ \cmidrule(lr){2-6} \cmidrule(lr){7-11} \cmidrule(lr){12-16}
Method           & {1-shot} & {5-shot} & {10-shot} & {20-shot} & {25-shot} & {1-shot} & {5-shot} & {10-shot} & {20-shot} & {25-shot} & {1-shot} & {5-shot} & {10-shot} & {20-shot} & {25-shot} \\ \midrule
Scratch-AConv    &  59.6\scriptsize{$\pm$1.2}  & 73.6\scriptsize{$\pm$1.2}   & 84.3\scriptsize{$\pm$0.9}    & 92.1\scriptsize{$\pm$0.7}    & 93.8\scriptsize{$\pm$0.6}    & 31.6\scriptsize{$\pm$0.6}   & 55.9\scriptsize{$\pm$0.8}   & 72.3\scriptsize{$\pm$0.8}    & 85.7\scriptsize{$\pm$0.5}    & 88.5\scriptsize{$\pm$0.4}    & 20.9\scriptsize{$\pm$0.3}   & 47.4\scriptsize{$\pm$0.4}   & 65.0\scriptsize{$\pm$0.4}    & 80.7\scriptsize{$\pm$0.2}    & 84.5\scriptsize{$\pm$0.2}    \\
Scratch-RN18     &  55.8\scriptsize{$\pm$1.2}  & 74.9\scriptsize{$\pm$1.4}   & 84.1\scriptsize{$\pm$1.1}    & 89.6\scriptsize{$\pm$0.8}    & 92.1\scriptsize{$\pm$0.6}    & 30.6\scriptsize{$\pm$0.7}   & 53.7\scriptsize{$\pm$0.8}   & 67.0\scriptsize{$\pm$0.7}    & 79.3\scriptsize{$\pm$0.6}    & 83.2\scriptsize{$\pm$0.5}    & 20.7\scriptsize{$\pm$0.3}   & 43.4\scriptsize{$\pm$0.4}   & 58.9\scriptsize{$\pm$0.4}    & 74.0\scriptsize{$\pm$0.4}    & 78.5\scriptsize{$\pm$0.3}    \\ [0.2mm] \cdashline{1-16} \\[-2.3mm]
SimCLR+basic     &  64.5\scriptsize{$\pm$1.4}  & 83.0\scriptsize{$\pm$1.1}   & 91.4\scriptsize{$\pm$0.7}    & 96.1\scriptsize{$\pm$0.4}    & 97.4\scriptsize{$\pm$0.2}    & 38.8\scriptsize{$\pm$0.7}   & 68.4\scriptsize{$\pm$0.7}   & 82.6\scriptsize{$\pm$0.5}    & 91.8\scriptsize{$\pm$0.3}    & \textbf{94.0}\scriptsize{$\pm$0.2}    & 28.3\scriptsize{$\pm$0.3}   & 59.2\scriptsize{$\pm$0.3}   & 75.2\scriptsize{$\pm$0.2}    & \textbf{87.7}\scriptsize{$\pm$0.1}    & \textbf{90.6}\scriptsize{$\pm$0.1}    \\
SimCLR+OE        &  \textbf{65.4}\scriptsize{$\pm$1.5}  & \textbf{86.4}\scriptsize{$\pm$1.0}   & \textbf{92.2}\scriptsize{$\pm$0.8}    & \textbf{96.5}\scriptsize{$\pm$0.4}    & \textbf{97.5}\scriptsize{$\pm$0.2}    & \textbf{39.9}\scriptsize{$\pm$0.7}   & \textbf{69.7}\scriptsize{$\pm$0.6}   & \textbf{83.3}\scriptsize{$\pm$0.5}    & \textbf{92.3}\scriptsize{$\pm$0.3}    & 93.8\scriptsize{$\pm$0.2}    & \textbf{28.8}\scriptsize{$\pm$0.3}   & \textbf{60.5}\scriptsize{$\pm$0.3}   & \textbf{75.6}\scriptsize{$\pm$0.2}    & 87.0\scriptsize{$\pm$0.1}    & 89.3\scriptsize{$\pm$0.1}    \\ \bottomrule
\end{tabular}
}
\vspace{-2.5mm}
\end{table*}



%%% No rotated labels, no stdev
% \begin{table*}[ht]
% \centering
% \caption{MSTAR Standard Operating Condition Accuracy Results}
% %\label{tab:mstar_soc}
% \resizebox{1.\textwidth}{!}{
% \begin{tabular}{lccccccccccccccc}
% \toprule
%                  & \multicolumn{5}{c}{2-way}                     & \multicolumn{5}{c}{5-way}                     & \multicolumn{5}{c}{10-way}                    \\ \cmidrule(lr){2-6} \cmidrule(lr){7-11} \cmidrule(lr){12-16}
% Method           & {1-shot} & {5-shot} & {10-shot} & {20-shot} & {25-shot} & {1-shot} & {5-shot} & {10-shot} & {20-shot} & {25-shot} & {1-shot} & {5-shot} & {10-shot} & {20-shot} & {25-shot} \\ \midrule
% Scratch-AConv    &  59.6  & 73.6   & 84.3    & 92.1    & 93.8    & 31.6   & 55.9   & 72.3    & 85.7    & 88.5    & 20.9   & 47.4   & 65.0    & 80.7    & 84.5    \\
% Scratch-RN18     &  55.8  & 74.9   & 84.1    & 89.6    & 92.1    & 30.6   & 53.7   & 67.0    & 79.3    & 83.2    & 20.7   & 43.4   & 58.9    & 74.0    & 78.5    \\ [0.2mm] \cdashline{1-16} \\[-2.3mm]
% SimCLR+basic     &  64.5  & 83.0   & 91.4    & 96.1    & 97.4    & 38.8   & 68.4   & 82.6    & 91.8    & \textbf{94.0}    & 28.3   & 59.2   & 75.2    & \textbf{87.7}    & \textbf{90.6}    \\
% SimCLR+OE        &  \textbf{65.4}  & \textbf{86.4}   & \textbf{92.2}    & \textbf{96.5}    & \textbf{97.5}    & \textbf{39.9}   & \textbf{69.7}   & \textbf{83.3}    & \textbf{92.3}    & 93.8    & \textbf{28.8}   & \textbf{60.5}   & \textbf{75.6}    & 87.0    & 89.3    \\ \bottomrule
% \end{tabular}
% }
% \end{table*}

% WITH ROTATED LABELS
% %%%%%%%%%%%%%%%%%%%%%%%%%%%%%%%%%%%%%%%%%%%%%%% MSTAR SOC RESULTS
% \begin{table*}[ht]
% \centering
% \caption{MSTAR Standard Operating Condition Accuracy Results}
% %\label{tab:mstar_soc}
% \resizebox{1.\textwidth}{!}{
% \begin{tabular}{lccccccccccccccc}
% \toprule
%                  & \multicolumn{5}{c}{2-way}                     & \multicolumn{5}{c}{5-way}                     & \multicolumn{5}{c}{10-way}                    \\ \cmidrule(lr){2-6} \cmidrule(lr){7-11} \cmidrule(lr){12-16}
% Method           & \rot{1-shot} & \rot{5-shot} & \rot{10-shot} & \rot{20-shot} & \rot{25-shot} & \rot{1-shot} & \rot{5-shot} & \rot{10-shot} & \rot{20-shot} & \rot{25-shot} & \rot{1-shot} & \rot{5-shot} & \rot{10-shot} & \rot{20-shot} & \rot{25-shot} \\ \midrule
% Scratch-AConv    &  59.6  & 73.6   & 84.3    & 92.1    & 93.8    & 31.6   & 55.9   & 72.3    & 85.7    & 88.5    & 20.9   & 47.4   & 65.0    & 80.7    & 84.5    \\
% Scratch-RN18     &  55.8  & 74.9   & 84.1    & 89.6    & 92.1    & 30.6   & 53.7   & 67.0    & 79.3    & 83.2    & 20.7   & 43.4   & 58.9    & 74.0    & 78.5    \\ [0.2mm] \cdashline{1-16} \\[-2.3mm]
% SimCLR+basic     &  64.5  & 83.0   & 91.4    & 96.1    & 97.4    & 38.8   & 68.4   & 82.6    & 91.8    & \textbf{94.0}    & 28.3   & 59.2   & 75.2    & \textbf{87.7}    & \textbf{90.6}    \\
% SimCLR+OE        &  \textbf{65.4}  & \textbf{86.4}   & \textbf{92.2}    & \textbf{96.5}    & \textbf{97.5}    & \textbf{39.9}   & \textbf{69.7}   & \textbf{83.3}    & \textbf{92.3}    & 93.8    & \textbf{28.8}   & \textbf{60.5}   & \textbf{75.6}    & 87.0    & 89.3    \\ \bottomrule
% \end{tabular}
% }
% \end{table*}


%%%%%%%%%%%%%%%%%%%%%%%%%%%%%%%%%%%%%%%%%%%%%%%%%%%%%%%%%%%%%
\subsection{Classifier Training}

In Stage 2 (ref. Fig.~\ref{fig:overview}) we use the global model $f_\phi$ as a fixed feature extractor and learn a classifier $c_\theta$ for the given (M-way, N-shot) FSL task, described by $\mathcal{D}_{support}=\{(x_i,y_i)\}_{i=1 \ldots (M \times N)}$.
To be clear, ``ways'' refers to the number of categories in the label-space and ``shots'' is number of labeled training chips per category. 
Unlike related works in the natural imagery domain that use simple linear regression or nearest neighbor-style classifiers \cite{goodembedding_fsl}, we find it beneficial to utilize a 2-layer neural network which has the flexibility to learn non-linear decision boundaries.

We specifically consider two ways to train $c_\theta$. 
The first, called the \textit{basic} method, is to train under a vanilla supervised learning objective which minimizes empirical risk on $\mathcal{D}_{support}$ samples only. 
Mathematically, the basic training method is described as
\begin{equation}
\min_{\theta} \mathop{\mathbb{E}}_{(x,y)\sim\mathcal{D}_{support}}\big [ L(c_{\theta}(f_{\phi}(x)), y) \big ], 
\end{equation}
where $L$ is the cross-entropy loss between the classifier's prediction and the ground truth label $y$.
The intuition for this training method is to learn a classifier that produces accurate and confident predictions on the ID data.



While the basic method can achieve high ID accuracy, it gives no guidance to the model for how to behave when an OOD input is encountered. 
Without violating or adding any assumptions, we can do something clever to greatly improve the classifier's ability to handle OOD inputs. 
Specifically, we can re-purpose the $\mathcal{D}_{pretrain}$ dataset from Stage 1 as an \textit{outlier exposure} (OE) set in Stage 2.
The intuition for OE \cite{HendrycksOE} is to teach the model a confidence calibration on ID and OOD inputs -- it should make accurate and confident predictions on ID data and minimally confident predictions on OOD data.
Functionally, the minimum confidence state is when the predicted probability is \sfrac{1}{\# classes}, so to achieve the desired effect we set the training targets of the OE data to be a Uniform distribution over the classes $\mathcal{U}_{\mathcal{C}}$, while the targets of the ID data remain 1-hots.
The complete OE objective is
\begin{multline}
   \min_{\theta} \mathop{\mathbb{E}}_{(x,y)\sim\mathcal{D}_{support}}\big [ L(c_{\theta}(f_{\phi}(x)), y) \big ] \\ + \lambda \mathop{\mathbb{E}}_{\tilde{x}\sim\mathcal{D}_{pretrain}}\big [ L(c_{\theta}(f_{\phi}(\tilde{x})), \mathcal{U}_{\mathcal{C}}) \big ], 
\end{multline}
where $\lambda$ is a weighting factor between the ID- and OOD-focused terms that we set to 0.5 here \cite{HendrycksOE}.



%%%%%%%%%%%%%%%%%%%%%%%%%%%%%%%%%%%%%%%%%%%%%%%%%%%%%%%%%%%%%
\subsection{OOD Detection}

The final detail of our methodology is how we detect OOD samples during deployment.
To conceptualize the process, think about each test input $x$ being assigned a real-valued score $\mathcal{S}_{ID}(x)$ as a measure of its ID-ness.
During operation, if $\mathcal{S}_{ID}(x) \geq \beta_{thresh}$ (an application dependent threshold), then the input would be considered ID and the classifier would release its prediction over the set of known classes. Else, if $\mathcal{S}_{ID}(x) < \beta_{thresh}$, the input is considered OOD and the system abstains from releasing the prediction.
In this work we use a temperature-scaled Maximum Softmax Probability detector \cite{odin_ood} to produce $\mathcal{S}_{ID}(x)$ scores, described as
\begin{equation}
\mathcal{S}_{ID}(x) = \max_{i\in\mathcal{C}} \frac{exp \big( c_{\theta}^{(i)}(f(x))/\tau \big )}{\sum_{j}^{\mathcal{|C|}} exp \big( c_{\theta}^{(j)}(f(x))/\tau \big )}.
\end{equation}
Here, $\tau$ is a temperature hyperparameter which we set to 100 via validation. 
The intuition for this detection scheme is straightforward: ID samples should be predicted by the classifier with higher confidence than OOD samples. 
This fits naturally with the OE training goal, making it a logical choice in our system.
We note that OOD detection is currently a popular research topic and future works may investigate alternative scoring mechanisms.




\section{Experiments}

To show the merit of our approach to the robust few-shot SAR ATR problem we perform three primary experiments and one analysis. 
The first experiment (Sec.~\ref{sec:soc}) shows results in MSTAR's Standard Operating Condition (SOC) and the second (Sec.~\ref{sec:eoc}) investigates generalization performance in an Extended Operating Condition (EOC). 
The third experiment (Sec.~\ref{sec:ood}) measures OOD detection performance against a spectrum of granularities and the analysis (Sec.~\ref{sec:tradeoff}) discusses a critical trade-off between OOD detection and generalization.


All of the experiments follow a similar setup.
In Stage 1, we initialize the feature extractor as a ResNet-18 (RN18) \cite{resnet} backbone with output dimension 512, and the Projector network as a 3-layer neural net \cite{simclr_github} with output dimension 128. 
These components are trained with an ADAM optimizer for 200 epochs, with 1024 batch size, weight decay of 1e-4, and a learning rate of 3e-4 following a cosine decay schedule. 
For the SimCLR NT-Xent loss we use a temperature of 0.01. 
In Stage 2, we discard the Projector network, fix the weights of the RN18, and initialize our 2-layer neural net classifier. 
We train the classifier for 500 iterations, with label smoothing, using an ADAM optimizer with cosine decayed learning rate starting at 1e-3. 
On the data side, we train with $64\times64$~px crops and use Gaussian noise and random flipping augmentations.
Finally, the results in this document are averages over 250 randomized runs.


% SimCLR training: 
% - RN18 architecture
% - bsize=1024
% - lr = 3e-4 with cos decay
% - ADAM optimizer with weight\_decay=1e-4
% - num\_epochs = 200
% - projection dimension = 128
% - NT-Xent temp = 0.01
% Classifier Training
% - 500 iters
% - 5x OE bsize
% - lblsmoothing=0.1
% - centercrop=64
% - ADAM optimizer with lr=1e-3 and cosine decay
% - random hflip
% - gaus noise = 0.1
%All numbers we report are computed averages over 250 randomized runs 





%%%%%%%%%%%%%%%%%%%%%%%%%%%%%%%%%%%%%%%%%%%%%%%%%%%%%%%%%%%%%
\subsection{Standard Operating Conditions} \label{sec:soc}


To measure SOC performance on MSTAR we train the models on 17$^{\circ}$ elevation imagery and test the accuracy of classifying 15$^{\circ}$ data \cite{aconvnet}.
As mentioned, because we do not assume to have labeled pretraining data, meta-learning-based FSL methods are inappropriate for comparison (also see Supplemental B). 
For baselines we instead train supervised models from scratch on the $\mathcal{D}_{support}$ samples only. 
We examine two architectures of varying complexities: A-ConvNet \cite{aconvnet} and RN18 (displayed as Scratch-\{AConv,RN18\}).
The MSTAR SOC results are shown in Table~\ref{tab:mstar_soc} for \#-ways=\{2,5,10\} and \#-shots=\{1,5,10,20,25\}, covering a gamut of potential scenarios.



To start, we find several intuitive takeaways. 
First, accuracy increases with the number of shots; and second, as the number of ways increases the models need more shots to achieve the same performance. 
In all cases, our SimCLR-based models significantly outperform the models trained from scratch, and in most cases the OE classifier has a slight edge over the basic classifier. 
When only given 10-shots, our average margin of improvement over the best baseline is +9.8\%.
On full 10-class MSTAR, we reach 90\% average accuracy at $\sim$25-shots, which is a 6\% improvement over the best baseline.
Lastly, we want to emphasize the implications of these results. 
The self-supervised SimCLR backbone, which has been trained on mostly ships, is able to separate the MSTAR data by class even though it has never been trained on such data!






%%%%%%%%%%%%%%%%%%%%%%%%%%%%%%%%%%%%%%%%%%%%%%%%%%%%%%%%%%%%%
\subsection{Extended Operating Conditions} \label{sec:eoc}

For the MSTAR EOC test we train the classification models on 17$^{\circ}$ elevation imagery and test on 30$^{\circ}$ data \cite{aconvnet}. 
The large difference in collection geometry causes a sizable change in the target signatures and offers a more challenging test of generalization.
Our results are reported in Table~\ref{tab:mstar_eoc}.
We see several similar trends to the SOC results w.r.t. the \#-shots and \#-ways.
In all cases, the SimCLR-based models outperform the train-from-scratch baselines, while the basic and OE classifiers perform very similarly to each other.
Interestingly, the largest margins over the baselines are at the lowest number of shots. For example, at 5-shots our average improvement over the best baseline is +9\%.

%%% MSTAR EOC Accuracy Table
%%%%%%%%%%%%%%%%%%%%%%%%%%%%%%%%%%%%%%%%%%%%%%% MSTAR EOC RESULTS
\begin{table}[t]
\centering
\caption{MSTAR Extended Operating Condition Accuracy Results}
\label{tab:mstar_eoc}
\resizebox{1.\linewidth}{!}{
\begin{tabular}{lcccccccc}
\toprule
                 & \multicolumn{4}{c}{2-way}                     & \multicolumn{4}{c}{4-way}                    \\ \cmidrule(lr){2-5} \cmidrule(lr){6-9}
Method           & \rot{5-shot} & \rot{10-shot} & \rot{20-shot}  & \rot{25-shot} & \rot{5-shot} & \rot{10-shot} & \rot{20-shot} & \rot{25-shot}  \\ \midrule
Scratch-AConv    &  68.9  & 78.4   & 87.0    & 89.3    & 53.8    & 66.2   & 74.3   & 76.3        \\
Scratch-RN18     &  72.1  & 82.1   & 86.6    & 86.9    & 56.0    & 65.8   & 73.2   & 74.5        \\ [0.2mm] \cdashline{1-9} \\[-2.3mm]
SimCLR+basic     &  78.8  & 85.4   & \textbf{90.3}    & \textbf{90.7}    & 65.2    & \textbf{74.2}   & \textbf{80.5}   & \textbf{82.0}        \\
SimCLR+OE        &  \textbf{80.9}  & \textbf{85.7}   & 89.6    & 89.8    & \textbf{65.3}    & 74.1   & 79.7   & 81.0        \\

\bottomrule
\end{tabular}
}
\end{table}





%%%%%%%%%%%%%%%%%%%%%%%%%%%%%%%%%%%%%%%%%%%%%%%%%%%%%%%%%%%%%
\subsection{Out-of-Distribution Detection} \label{sec:ood}

Our goal in the third experiment is to examine OOD detection performance across a spectrum of difficulties/granularities.
To do this we leverage OOD data from various sources. 
For a particularly hard OOD test we use a holdout scheme. 
Starting with a 7-way classifier, where in each test iteration 7 of the 10 MSTAR classes are randomly selected for use in the ID label space, we use data from the remaining 3 classes as \textit{Holdout} OOD samples. 
For medium difficulty OOD, we use data from \textit{SARSIM-Roads}, \textit{SARSIM-Medium}, and \textit{SARSIM-Grass} \cite{sarsim_dataset}. 
This is synthetic SAR data which has been generated to specifically match the collection conditions of MSTAR with different backgrounds (note, we remove the Bulldozer and Tank classes to avoid potential ID/OOD ambiguities).
For coarse-grained OOD we generate \textit{FakeData} by creating random Uniform noise chips and use the \textit{MNIST} test set which contains hand-written digits. Both are clearly OOD w.r.t. any SAR classifier.
Fig.~\ref{fig:ood_data} shows examples from each of these datasets. Importantly, keep in mind that none of the test OOD data is in $\mathcal{D}_{pretrain}$ (the OE set), meaning the model has never seen it before.




Table~\ref{tab:mstar_ood} shows the average OOD detection performance (as measured by \% AUROC \cite{odin_ood}) of a 7-way MSTAR SOC classifier at different \#-shots.
Note, the AUROC value can be interpreted as the probability that an ID test input would have a greater $\mathcal{S}_{ID}$ score than an OOD input \cite{smaxthresh}, where 100\% is a perfect detector.
At each setting we examine the performance of the basic versus OE classifier, which we know both achieve high ID accuracy from Sec.~\ref{sec:soc}.
The most important takeaway from Table~\ref{tab:mstar_ood} is that the OE classifier is better than the basic classifier in \textit{all} cases by a large margin.
Interestingly, for the basic model the coarse-grained OOD sets are the hardest to detect, while the OE classifier detects these almost perfectly.
This surprising behavior is something we plan to study in a future work.
Another interesting finding is that detection of \textit{Holdout} samples improves significantly as \#-shots increases, while detection of the other OOD types stays relatively constant w.r.t.~\#-shots. 
Finally, a somewhat expected result is that for OE models, the \textit{Holdout} samples are the most difficult to detect, while the medium- and coarse-grained OOD inputs are less challenging.
Given that these results still show room for improvement, we highly encourage future system designers (whether in few-shot scenarios or not) to use OOD-aware classifier training schemes such as OE to build robustness to the many forms of OOD inputs.


%%% OOD Examples Figure
\begin{figure}[t]
  \centering
  %\vspace{-2mm}
  \includegraphics[width=1.0\linewidth]{figures/ood_data.png}
  %\vspace{-3mm}
  \caption{Sample test chips from each dataset used in the OOD experiment.}
  \label{fig:ood_data}
\end{figure}

%%% MSTAR OOD Table
%%%%%%%%%%%%%%%%%%%%%%%%%%%%%%%%%%%%%%%%%%%%%%% MSTAR OOD DETECTION RESULTS (copy)
\begin{table}[t]
\centering
\caption{MSTAR Out-of-Distribution Detection Results for 7-way classifier (\% AUROC)}
\label{tab:mstar_ood}
\resizebox{\linewidth}{!}{
\begin{tabular}{lccccc}
\toprule
             & \multicolumn{5}{c}{Notation = basic / OE} \\ \cmidrule(lr){2-6}
OOD Set      & 1-shot & 5-shot & 10-shot & 20-shot & 25-shot \\ \midrule
Holdout      & 54.1 / \textbf{56.0}        & 64.1 / \textbf{67.2}        & 70.2 / \textbf{76.2}        & 78.0 / \textbf{81.9}        & 81.7 / \textbf{84.0}        \\
SARSIM-R & 59.6 / \textbf{93.2}        & 56.3 / \textbf{95.9}        & 58.8 / \textbf{96.6}        & 66.0 / \textbf{97.2}        & 68.3 / \textbf{97.3}        \\
SARSIM-M & 55.5 / \textbf{88.4}       & 50.8 / \textbf{90.5}        & 53.9 / \textbf{92.4}        & 60.3 / \textbf{93.5}        & 61.2 / \textbf{93.6}        \\
SARSIM-G & 44.8 / \textbf{87.9}        & 40.3 / \textbf{87.0}        & 44.2 / \textbf{88.1}        & 50.6 / \textbf{87.9}        & 52.5 / \textbf{87.2}        \\
FakeData     & 42.1 / \textbf{98.5}        & 27.0 / \textbf{99.5}        & 21.7 / \textbf{99.6}        & 21.6 / \textbf{99.5}        & 21.3 / \textbf{99.5}        \\
MNIST        & 15.1 / \textbf{94.1}        & 15.5 / \textbf{95.8}        & 16.6 / \textbf{96.3}        & 21.1 / \textbf{95.5}        & 22.2 / \textbf{95.8}        \\
\bottomrule
\end{tabular}
}
\end{table}





%%%%%%%%%%%%%%%%%%%%%%%%%%%%%%%%%%%%%%%%%%%%%%%%%%%%%%%%%%%%%
\subsection{Generalization vs. Detection Trade-off} \label{sec:tradeoff}

Our final experiment is an analysis that deliberates the question: can we have it all? 
Specifically, can we have high accuracy on ID data in SOC \textit{and} high accuracy on ID classes in EOC \textit{and} reliable OOD detection?
To answer this question we create Fig.~\ref{fig:density_plot}, which shows density plots of $\mathcal{S}_{ID}(x)$ scores measured on a SimCLR+OE model for ID test data in SOC ({\color{blue}{SOC-ID}}), ID test data in EOC ({\color{orange}{EOC-ID}}), and Holdout OOD data in SOC ({\color{codegreen}{Holdout-OOD}}).


Firstly, we observe that the {\color{blue}{SOC-ID}} and {\color{codegreen}{Holdout-OOD}} are relatively separable, which matches the findings in Table~\ref{tab:mstar_ood}.
However, the {\color{orange}{EOC-ID}} and {\color{codegreen}{Holdout-OOD}} are not very separable, as both are predicted with relatively low scores. 
Thus, with our method if you require good {\color{codegreen}{Holdout-OOD}} detection then most of the {\color{orange}{EOC-ID}} would also get flagged as OOD.
Conversely, if you require high recall on {\color{orange}{EOC-ID}} then many {\color{codegreen}{Holdout-OOD}} would pass through the detector.
These results indicate that we \textit{cannot} have it all with this design, and highlight a serious trade-off to be considered in future work.
Finally, while this trade-off may not be completely unique to the FSL setting, our previous observation that less shots makes Holdout-OOD detection harder (Sec.~\ref{sec:ood}) means this trade-off may be more dire in the few-shot setting.


%%% Generalization vs Detection Density Plot Figure
\begin{figure}[t]
  \centering
  %\vspace{-2mm}
  \includegraphics[width=1.0\linewidth]{figures/density_plot_small.png}
  %\vspace{-3mm}
  \caption{Density plots of $\mathcal{S}_{ID}$ scores for ID data in SOC, ID data in EOC, and Holdout OOD data. Also shown are hypothetical thresholds to achieve 80\% True Positive Rates on ID data in SOC and EOC.}
  \label{fig:density_plot}
\end{figure}


\section{Conclusion}

We now confirm our initial hypothesis (Sec.~\ref{sec:intro}) that modern SSL techniques like SimCLR are capable of learning highly generalizable SAR feature extractors from large pools of diverse and unlabeled SAR data. 
As evidence, our experiments show that a model trained mostly on unlabeled SAR ships is able to provide an informative-enough feature space to do few-shot classification of MSTAR targets.
This result provides merit to the under-studied global model approach to SAR ATR and confirms that SAR features can be highly transferable, even across sensors, imaging modes, target types, etc.
At a more concrete level, we provide a method for performing robust few-shot SAR ATR in a limited environment where there is no labeled data for pretraining.
We also provide a methodology that boosts OOD detection performance of the classifier without adding significant assumptions or overhead.
We hope that this work motivates further study on few-shot SAR ATR and representation learning for SAR, with specific considerations for robustness to OOD inputs and the trade-off between OOD detection and generalization.


Lastly, we provide some suggestions for future work.
We believe there are potential gains in developing improved domain-relevant augmentations for both Stage 1 and Stage 2 training. 
We also believe that adding synthetically generated data crafted to be relevant to the downstream task may improve the quality of learned features.
Next, one may investigate the use of different SSL algorithms and backbone model architectures. 
Finally, an interesting study would be to evaluate when it becomes beneficial to fine-tune the feature extractor instead of keeping it fixed during Stage 2 training.

\noindent \textbf{Public Release Number:}~AFRL-2022-3418













%\textbf{Disclaimer:} The views expressed in this article are those of the authors and do not reflect official policy of the United States Air Force, Department of Defense or the U.S. Government. Public release number: AFRL-2020-0519


%%%%%%%%%%%%%%%%%%%%%%%%%%%%%%%%%%%%%%%%%%%%%%%%%%%%%%%%%%%









% if have a single appendix:
%\appendix[Proof of the Zonklar Equations]
% or
%\appendix  % for no appendix heading
% do not use \section anymore after \appendix, only \section*
% is possibly needed

% use appendices with more than one appendix
% then use \section to start each appendix
% you must declare a \section before using any
% \subsection or using \label (\appendices by itself
% starts a section numbered zero.)
%



% use section* for acknowledgment
%\section*{Acknowledgment}
%The authors would like to thank...


% Can use something like this to put references on a page
% by themselves when using endfloat and the captionsoff option.
\ifCLASSOPTIONcaptionsoff
  \newpage
\fi



% trigger a \newpage just before the given reference
% number - used to balance the columns on the last page
% adjust value as needed - may need to be readjusted if
% the document is modified later
%\IEEEtriggeratref{8}
% The "triggered" command can be changed if desired:
%\IEEEtriggercmd{\enlargethispage{-5in}}

% references section

% can use a bibliography generated by BibTeX as a .bbl file
% BibTeX documentation can be easily obtained at:
% http://mirror.ctan.org/biblio/bibtex/contrib/doc/
% The IEEEtran BibTeX style support page is at:
% http://www.michaelshell.org/tex/ieeetran/bibtex/
%\bibliographystyle{IEEEtran}
% argument is your BibTeX string definitions and bibliography database(s)
%\bibliography{IEEEabrv,../bib/paper}
%
%\bibliographystyle{./IEEEtran}
%\bibliography{./IEEEabrv,./IEEEexample}

%\bibliographystyle{IEEEtran}
%%\bibliography{bibtex/bib/IEEEabrv.bib,bibtex/bib/IEEEexample.bib}{}
%\bibliography{bibtex/bib/IEEEabrv.bib,bibtex/bib/custom.bib}{}

%%%%%%%%%%%%%%%%%%%%%%%%%%%%%%%%%%%%%%%%%%%%%%%%%%%%%%%%%%%%%%%%%%%%%%%%%%%%%%%%%%%%%%%%%%%%%%%%%%%%%%%%
% Generated by IEEEtran.bst, version: 1.14 (2015/08/26)
\begin{thebibliography}{10}
\providecommand{\url}[1]{#1}
\csname url@samestyle\endcsname
\providecommand{\newblock}{\relax}
\providecommand{\bibinfo}[2]{#2}
\providecommand{\BIBentrySTDinterwordspacing}{\spaceskip=0pt\relax}
\providecommand{\BIBentryALTinterwordstretchfactor}{4}
\providecommand{\BIBentryALTinterwordspacing}{\spaceskip=\fontdimen2\font plus
\BIBentryALTinterwordstretchfactor\fontdimen3\font minus
  \fontdimen4\font\relax}
\providecommand{\BIBforeignlanguage}[2]{{%
\expandafter\ifx\csname l@#1\endcsname\relax
\typeout{** WARNING: IEEEtran.bst: No hyphenation pattern has been}%
\typeout{** loaded for the language `#1'. Using the pattern for}%
\typeout{** the default language instead.}%
\else
\language=\csname l@#1\endcsname
\fi
#2}}
\providecommand{\BIBdecl}{\relax}
\BIBdecl

\bibitem{tom_book}
U.~K. Majumder, E.~P. Blasch, and D.~A. Garren, \emph{Deep Learning for Radar
  and Communications Automatic Target Recognition}.\hskip 1em plus 0.5em minus
  0.4em\relax Artech House, 2020.

\bibitem{protonet}
J.~Snell, K.~Swersky, and R.~Zemel, ``Prototypical networks for few-shot
  learning,'' in \emph{Advances in Neural Information Processing Systems},
  I.~Guyon, U.~V. Luxburg, S.~Bengio, H.~Wallach, R.~Fergus, S.~Vishwanathan,
  and R.~Garnett, Eds., vol.~30, 2017.

\bibitem{dktsn}
L.~Zhang, X.~Leng, S.~Feng, X.~Ma, K.~Ji, G.~Kuang, and L.~Liu, ``Domain
  knowledge powered two-stream deep network for few-shot sar vehicle
  recognition,'' \emph{IEEE Transactions on Geoscience and Remote Sensing},
  vol.~60, pp. 1--15, 2022.

\bibitem{ChenK0H20}
T.~Chen, S.~Kornblith, M.~Norouzi, and G.~E. Hinton, ``A simple framework for
  contrastive learning of visual representations,'' in \emph{International
  Conference on Machine Learning}, 2020, pp. 1597--1607.

\bibitem{goodembedding_fsl}
Y.~Tian, Y.~Wang, D.~Krishnan, J.~B. Tenenbaum, and P.~Isola, ``Rethinking
  few-shot image classification: {A} good embedding is all you need?'' in
  \emph{European Conference on Computer Vision}, vol. 12359, 2020, pp.
  266--282.

\bibitem{sarsim_dataset}
D.~{Malmgren-Hansen}, A.~{Kusk}, J.~{Dall}, A.~A. {Nielsen}, R.~{Engholm}, and
  H.~{Skriver}, ``Improving sar automatic target recognition models with
  transfer learning from simulated data,'' \emph{IEEE Geoscience and Remote
  Sensing Letters}, vol.~14, no.~9, pp. 1484--1488, 2017.

\bibitem{HendrycksOE}
D.~Hendrycks, M.~Mazeika, and T.~G. Dietterich, ``Deep anomaly detection with
  outlier exposure,'' in \emph{International Conference on Learning
  Representations}, 2019.

\bibitem{inkawhich_advoe}
N.~Inkawhich, E.~Davis, M.~Inkawhich, U.~K. Majumder, and Y.~Chen, ``Training
  sar-atr models for reliable operation in open-world environments,''
  \emph{IEEE Journal of Selected Topics in Applied Earth Observations and
  Remote Sensing}, vol.~14, pp. 3954--3966, 2021.

\bibitem{inkawhich_onlineOOD}
N.~Inkawhich, J.~Zhang, E.~K. Davis, R.~Luley, and Y.~Chen, ``Improving
  out-of-distribution detection by learning from the deployment environment,''
  \emph{IEEE Journal of Selected Topics in Applied Earth Observations and
  Remote Sensing}, vol.~15, pp. 2070--2086, 2022.

\bibitem{Ross1998MSTAR}
T.~Ross, S.~Worrell, V.~Velten, J.~Mossing, and M.~Bryant, ``Standard sar atr
  evaluation experiments using the mstar public release data set,'' in
  \emph{SPIE Conference on Algorithms for Synthetic Aperture Radar Imagery V},
  1998.

\bibitem{sarship_dataset}
Y.~Wang, C.~Wang, H.~Zhang, Y.~Dong, and S.~Wei, ``A sar dataset of ship
  detection for deep learning under complex backgrounds,'' \emph{Remote
  Sensing}, vol.~11, no.~7, p. 765, Mar 2019.

\bibitem{hrsid_dataset}
S.~Wei, X.~Zeng, Q.~Qu, M.~Wang, H.~Su, and J.~Shi, ``Hrsid: A high-resolution
  sar images dataset for ship detection and instance segmentation,'' \emph{IEEE
  Access}, vol.~8, pp. 120\,234--120\,254, 2020.

\bibitem{fusar_dataset}
X.~Hou, W.~Ao, Q.~Song, J.~Lai, H.~Wang, and F.~Xu, ``Fusar-ship: building a
  high-resolution {SAR-AIS} matchup dataset of gaofen-3 for ship detection and
  recognition,'' \emph{Sci. China Inf. Sci.}, vol.~63, no.~4, 2020.

\bibitem{ssdd_dataset}
T.~Zhang, X.~Zhang, J.~Li, X.~Xu, B.~Wang, X.~Zhan, Y.~Xu, X.~Ke, T.~Zeng,
  H.~Su, I.~Ahmad, D.~Pan, C.~Liu, Y.~Zhou, J.~Shi, and S.~Wei, ``Sar ship
  detection dataset (ssdd): Official release and comprehensive data analysis,''
  \emph{Remote Sensing}, vol.~13, no.~18, 2021.

\bibitem{lsssdd_dataset}
T.~Zhang, X.~Zhang, X.~Ke, X.~Zhan, J.~Shi, S.~Wei, D.~Pan, J.~Li, H.~Su,
  Y.~Zhou, and D.~Kumar, ``Ls-ssdd-v1.0: A deep learning dataset dedicated to
  small ship detection from large-scale sentinel-1 sar images,'' \emph{Remote
  Sensing}, vol.~12, no.~18, 2020.

\bibitem{dualpolships_dataset}
Y.~Hu, Y.~Li, and Z.~Pan, ``A dual-polarimetric sar ship detection dataset and
  a memory-augmented autoencoder-based detection method,'' \emph{Sensors},
  vol.~21, no.~24, 2021.

\bibitem{srsdd_dataset}
S.~Lei, D.~Lu, X.~Qiu, and C.~Ding, ``Srsdd-v1.0: A high-resolution sar
  rotation ship detection dataset,'' \emph{Remote Sensing}, vol.~13, no.~24,
  2021.

\bibitem{cvdome_dataset}
K.~E. Dungan, C.~Austin, J.~Nehrbass, and L.~C. Potter, ``{Civilian vehicle
  radar data domes},'' in \emph{SPIE Algorithms for Synthetic Aperture Radar
  Imagery XVII}, 2010.

\bibitem{simclr_github}
T.~Silva, ``Simclr,'' \url{https://github.com/sthalles/SimCLR}, 2022.

\bibitem{odin_ood}
S.~Liang, Y.~Li, and R.~Srikant, ``Enhancing the reliability of
  out-of-distribution image detection in neural networks,'' in
  \emph{International Conference on Learning Representations}, 2018.

\bibitem{resnet}
K.~He, X.~Zhang, S.~Ren, and J.~Sun, ``Deep residual learning for image
  recognition,'' in \emph{{IEEE} Conference on Computer Vision and Pattern
  Recognition}, 2016.

\bibitem{aconvnet}
S.~{Chen}, H.~{Wang}, F.~{Xu}, and Y.~{Jin}, ``Target classification using the
  deep convolutional networks for sar images,'' \emph{IEEE Transactions on
  Geoscience and Remote Sensing}, vol.~54, no.~8, pp. 4806--4817, 2016.

\bibitem{smaxthresh}
D.~Hendrycks and K.~Gimpel, ``A baseline for detecting misclassified and
  out-of-distribution examples in neural networks,'' in \emph{International
  Conference on Learning Representations}, 2017.

\end{thebibliography}

%%%%%%%%%%%%%%%%%%%%%%%%%%%%%%%%%%%%%%%%%%%%%%%%%%%%%%%%%%%%%%%%%%%%%%%%%%%%%%%%%%%%%%%%%%%%%%%%%%%%%%%%










% <OR> manually copy in the resultant .bbl file
% set second argument of \begin to the number of references
% (used to reserve space for the reference number labels box)
% \begin{thebibliography}{1}

% \bibitem{IEEEhowto:kopka}
% H.~Kopka and P.~W. Daly, \emph{A Guide to \LaTeX}, 3rd~ed.\hskip 1em plus
%   0.5em minus 0.4em\relax Harlow, England: Addison-Wesley, 1999.

% \end{thebibliography}


% biography section
% 
% If you have an EPS/PDF photo (graphicx package needed) extra braces are
% needed around the contents of the optional argument to biography to prevent
% the LaTeX parser from getting confused when it sees the complicated
% \includegraphics command within an optional argument. (You could create
% your own custom macro containing the \includegraphics command to make things
% simpler here.)
%\begin{IEEEbiography}[{\includegraphics[width=1in,height=1.25in,clip,keepaspectratio]{mshell}}]{Michael Shell}
% or if you just want to reserve a space for a photo:

% \begin{IEEEbiography}{Michael Shell}
% Biography text here.
% \end{IEEEbiography}

% % if you will not have a photo at all:
% \begin{IEEEbiographynophoto}{John Doe}
% Biography text here.
% \end{IEEEbiographynophoto}

% % insert where needed to balance the two columns on the last page with
% % biographies
% %\newpage
%\newpage

% \begin{IEEEbiography}[{\includegraphics[width=1in,height=1.25in,clip,keepaspectratio]{IEEEtran/figures/author_pics/nate.jpg}}]{Nathan Inkawhich}
% Nathan Inkawhich is currently a PhD candidate in the Electrical and Computer Engineering Department at Duke University, advised by Dr. Yiran Chen of the Computational Evolutionary Intelligence (CEI) Lab. Nathan's main areas of research are in machine learning algorithms and security, deep learning, out-of-distribution detection, and robust automatic target recognition. Previously, Nathan received a B.S. in Computer Engineering from Clarkson University in 2016, and worked for the Air Force Research Laboratory Information Directorate (AFRL/RI) in Rome, NY. For any questions regarding this work, please contact Nathan at \texttt{nathan.inkawhich@duke.edu}.
% \end{IEEEbiography}

% \begin{IEEEbiography}[{\includegraphics[width=1in,height=1.25in,clip,keepaspectratio]{IEEEtran/figures/author_pics/jingyang.jpeg}}]{Jingyang Zhang}
% Jingyang Zhang received the B.E. degree in electrical engineering from Tsinghua University, China, in 2019. He is currently working toward the Ph.D. degree with the Electrical and Computer Engineering Department, Duke University, Durham, NC, USA, jointly supervised by Dr. Yiran Chen and Dr. Hai (Helen) Li at the Computational Evolutionary Intelligence (CEI) Lab. His main areas of research are in secure and robust machine learning algorithms, specifically adversarial robustness and anomaly detection in the context of deep learning. For more information please visit: https://zjysteven.github.io/. 
% \end{IEEEbiography}

% \begin{IEEEbiography}[{\includegraphics[width=1in,height=1.25in,clip,keepaspectratio]{IEEEtran/figures/author_pics/eric.png}}]{Eric K. Davis}
% Eric K. Davis received his B.S. degree in electrical engineering from Northeastern University, Boston, MA, USA in 2013 and his M.S. in electrical and computer engineering from the Georgia Institute of Technology, Atlanta, GA, USA in 2017. He has been with SRC, Inc. since 2017 where he is currently a Lead Systems Engineer on the Machine Learning and Artificial Intelligence team. Prior to SRC he worked for the General Electric Company developing embedded sensor systems for scientific and commercial applications. His current projects involve applications in machine intelligence, signal processing, and high-performance embedded computing.
% \end{IEEEbiography}

% \begin{IEEEbiography}[{\includegraphics[width=1in,height=1.25in,clip,keepaspectratio]{IEEEtran/figures/author_pics/ryan.png}}]{Ryan Luley}
% Ryan Luley received the B.S. degree in Mathematics and Computer Science from St. Lawrence University, Canton, NY in 2001, the M.S. degree in Computer Engineering from Syracuse University, Syracuse, NY in 2008, and the Ph.D. degree in Computer and Information Science and Engineering from Syracuse University, Syracuse, NY in 2020.
% He is currently a Senior Mathematician with the United States Air Force Research Laboratory, Information Directorate, Rome, NY in the High Performance Systems branch.  
% His current research interests include machine learning, high performance embedded computing, and neuromorphic computing.
% \end{IEEEbiography}

% \begin{IEEEbiography}[{\includegraphics[width=1in,height=1.25in,clip,keepaspectratio]{IEEEtran/figures/author_pics/yiran.jpg}}]{Yiran Chen}
% Yiran Chen received B.S and M.S. from Tsinghua University and Ph.D. from Purdue University in 2005. After five years in industry, he joined University of Pittsburgh in 2010 as Assistant Professor and then promoted to Associate Professor with tenure in 2014, held Bicentennial Alumni Faculty Fellow. He now is the Professor of the Department of Electrical and Computer Engineering at Duke University and serving as the director of NSF Industry–University Cooperative Research Center (IUCRC) for Alternative Sustainable and Intelligent Computing (ASIC) and co-director of Duke University Center for Computational Evolutionary Intelligence (CEI), focusing on the research of new memory and storage systems, machine learning and neuromorphic computing, and mobile computing systems. Dr. Chen has published one book and more than 400 technical publications and has been granted 96 US patents. He serves or served the associate editor of several IEEE and ACM transactions/journals and served on the technical and organization committees of more than 50 international conferences. He is now serving as the Editor-in-Chief of IEEE Circuits and Systems Magazine. He received 7 best paper awards, 1 best poster award, and 15 best paper nominations from international conferences and workshops. He is the recipient of NSF CAREER award, ACM SIGDA outstanding new faculty award, the Humboldt Research Fellowship for Experienced Researchers, and the IEEE SYSC/CEDA TCCPS Mid-Career Award. He is the Fellow of IEEE, Distinguished Member of ACM, and a distinguished lecturer of IEEE CEDA.
% \end{IEEEbiography}


% \begin{IEEEbiographynophoto}{Jane Doe}
% Biography text here.
% \end{IEEEbiographynophoto}

% You can push biographies down or up by placing
% a \vfill before or after them. The appropriate
% use of \vfill depends on what kind of text is
% on the last page and whether or not the columns
% are being equalized.

%\vfill

% Can be used to pull up biographies so that the bottom of the last one
% is flush with the other column.
%\enlargethispage{-5in}

%%%%%%%%%%%%%%%%%%%%%%%%%%%%%%%%%%%%%%%%%%%% NAI
\section{Appendix for Proofs}

\paragraph{Proof of Theorem \ref{thm:main}.}

\begin{proof}
\label{proof:main}
Our proof has two steps. In Step 1, we will show that SimCLR is equivalent to minimizing the cross entropy loss defined in Eqn.~(\ref{eqn:cross-entropy}). 
In Step 2, we will show  that minimizing the cross-entropy loss 
is equivalent to spectral clustering on $\bfpi$. 
Combining the two steps together, we have proved our theorem. 

\textbf{Step 1: } SimCLR is equivalent to minimizing the cross entropy loss.

The cross-entropy loss takes expectation over 
$\bfW_\bfX\sim \mathbb{P}(\cdot ; \bfpi)$, 
which means $\bfW_\bfX$ has exactly one non-zero entry in each row $i$. By Lemma~\ref{lem:multinomial}, we know every row $i$ of $\bfW_\bfX$ is independent of other rows. Moreover, 
$\bfW_{\bfX,i}\sim \mathcal{M}(1, \bfpi_i/\sum_j \bfpi_{i,j})=\mathcal{M}(1, \bfpi_i)$, because $\bfpi_i$ itself is a probability distribution.
Similarly, we know $\bfW_\bfZ$ also has the row-independent property by sampling over $\mathbb{P}(\cdot;\bfK_\bfZ)$.
Therefore, by Lemma~\ref{lem:cross_split}, we know Eqn.~(\ref{eqn:cross-entropy}) is equivalent to:
\[
 -\sum_{i=1}^n \mathbb{E}_{\bfW_{\bfX,i}}[\log \mathbb{P}(\bfW_{\bfZ,i}=\bfW_{\bfX,i};\bfK_\bfZ)],
\]

This expression takes expectation over $\bfW_{\bfX,i}$ for the given row $i$. Notice that 
$\bfW_{\bfX,i}$ has exactly one non-zero entry, which equals $1$ (same for $\bfW_{\bfZ,i}$). 
As a result
we expand the above expression to be:
\begin{equation}
 -\sum_{i=1}^n \sum_{j\neq i} \Pr(\bfW_{\bfX,i,j}=1)\log \Pr(\bfW_{\bfZ,i,j}=1).
\label{eqn:detailed-expansion}    
\end{equation}


By Lemma~\ref{lem:multinomial}, $\Pr(\bfW_{\bfZ,i,j}=1)=\bfK_{\bfZ,i,j}/\|\bfK_{\bfZ,i}\|_1$ for $j\neq i$. Recall that $\bfK_\bfZ=(k(\bfZ_i-\bfZ_j))_{(i,j)\in[n]^2}$, which means 
$\bfK_{\bfZ,i,j}/\|\bfK_{\bfZ,i}\|_1=\frac{\exp(-\|\bfZ_i-\bfZ_j\|^2/{2\tau})}{\sum_{k\neq i}
\exp(-\|\bfZ_i-\bfZ_k\|^2/{2\tau})
}$ for $j\neq i$, when $k$ is the Gaussian kernel with variance $\tau$. 

Notice that $\bfZ_i=f(\bfX_i)$, so we know
\begin{equation}
-\log \Pr(\bfW_{\bfZ,i,j}=1)=
-\log \frac{\exp(-\|f(\bfX_i)-f(\bfX_j)\|^2/{2\tau})}{\sum_{k\neq i}
\exp(-\|f(\bfX_i)-f(\bfX_k)\|^2/{2\tau}),
}
\label{eqn:infonce-equivalence}    
\end{equation}


The right hand side is exactly the InfoNCE loss defined in Eqn.~(\ref{eqn:infonce}).
Inserting Eqn.~(\ref{eqn:infonce-equivalence}) into Eqn.~(\ref{eqn:detailed-expansion}), we get the SimCLR algorithm, which first samples augmentation pairs $(i,j)$ with $\Pr(\bfW_{\bfX,i,j}=1)$ for each row $i$, and then optimize the InfoNCE loss. 

\textbf{Step 2: } minimizing the cross entropy loss 
is equivalent to spectral clustering on $\bfpi$.


By Lemma~\ref{lem:convert_to_spectral}, we may further convert the loss to 
\begin{equation}
\label{eqn:main-theorem-repul-attr}
\min_{\bfZ}
-\sum_{(i,j)\in [n]^2} \mathbf{P}_{i,j}
\log k (\bfZ_i-\bfZ_j)+\log \mathbf{R}(\bfZ).
\end{equation}
Since $k$ is the Gaussian kernel, this reduces to \[
\min_\bfZ \mathrm{tr}(\bfZ^\top \mathbf{L}(\bfpi) \bfZ)
+\log \mathbf{R}(\bfZ),
\]

where we use the fact that $\mathbb{E}_{\bfW_\bfX\sim \mathbb{P}(\cdot; \bfpi)}[\mathbf{L}(\bfW_\bfX)]
=\mathbf{L}(\bfpi)
$, because the Laplacian operator is linear and $
\mathbb{E}_{\bfW_\bfX\sim \mathbb{P}(\cdot; \bfpi)}(\bfW_\bfX)=\bfpi
$.
\end{proof}

\paragraph{Proof of Theorem \ref{thm:clip}.}
\begin{proof}
Since $\bfW_\bfX\sim \mathbb{P}(\cdot;\bfpi_{\mathbf{A}, \mathbf{B}})$, we know 
$\bfW_\bfX$ has exactly one non-zero entry in each row, denoting the pair that got sampled. 
A notable difference compared to the previous proof is we now have $n_\mathcal{A}+n_\mathcal{B}$ objects in our graph. CLIP deals with this by taking a mini-batch of size $2N$, 
such that $n_\mathcal{A}=n_\mathcal{B}=N$, and adding the $2N$ InfoNCE losses together. We label the objects in $\mathcal{A}$ as $[n_\mathcal{A}]$, and the objects in $\mathcal{B}$ as $\{n_\mathcal{A}+1, \cdots, n_\mathcal{A}+n_\mathcal{B}\}$. 

Notice that $\bfpi_{\mathbf{A}, \mathbf{B}}$ is a bipartite graph, so the edges of objects in $\mathcal{A}$ will only connect to object in $\mathcal{B}$ and vice versa. We can define the similarity matrix in $\cZ$ as $\bfK_\bfZ$, 
where $\bfK_\bfZ(i, j+n_\mathcal{A})=\bfK_\bfZ(j+n_\mathcal{A},i)= k(\bfZ_i-\bfZ_j)$ for $i\in [n_\mathcal{A}], j\in [n_\mathcal{B}]$, and otherwise we set $\bfK_\bfZ(i,j)=0$. 
The rest is same as the previous proof. 
\end{proof}

\paragraph{Proof of Theorem \ref{thm:exponential}.}

\begin{proof}
\label{proof:exponential}
Since the objective function consists of a linear term combined with an entropy regularization, which is a strongly concave function, the maximization problem is a convex optimization problem. Owing to the implicit constraints provided by the entropy function, the problem is equivalent to having only the equality constraint. We then introduce the Lagrangian multiplier $\lambda$ and obtain the following relaxed problem:

$$
\widetilde{E}(\boldsymbol{\alpha})=\psi_{1}-\sum_{i=1}^n \alpha_{i} \psi_{i}+\tau \sum_{i=1}^n \alpha_{i}\log \alpha_{i}+\lambda\left(\boldsymbol{\alpha}^{\top} \mathbf{1}_n-1\right).
$$

As the relaxed problem is unconstrained, taking the derivative with respect to $\alpha_{i}$ yields

$$
\frac{\partial \widetilde{E}(\boldsymbol{\alpha})}{\partial \alpha_{i}}=-\psi_{i}+\tau\left(\log \alpha_{i}+\alpha_{i} \frac{1}{\alpha_{i}}\right)+\lambda=0.
$$

Solving the above equation implies that $\alpha_{i}$ takes the form
$
\alpha_{i}=\exp \left(\frac{1}{\tau} \psi_{i}\right) \exp \left(\frac{-\lambda}{\tau}-1\right).
$ Since $\alpha_{i}$ lies on the probability simplex, the optimal $\alpha_{i}$ is explicitly given by
$
\alpha^{*}_{i}=\frac{\exp \left(\frac{1}{\tau} \psi_{i}\right)}{\sum_{i^{\prime}=1}^n \exp \left(\frac{1}{\tau} \psi_{i^{\prime}}\right)} .
$ Substituting the optimal point into the objective function, we obtain
$$
\begin{aligned}
E\left(\boldsymbol{\alpha}^*\right)  &=\psi_1-\sum_{i=1}^n \frac{\exp \left(\frac{1}{\tau} \psi_{i}\right)}{\sum_{i^{\prime}=1}^n \exp \left(\frac{1}{\tau} \psi_{i^{\prime}}\right)} \psi_{i}+\tau \sum_{i=1}^n \frac{\exp \left(\frac{1}{\tau} \psi_{i}\right)}{\sum_{i^{\prime}=1}^n \exp \left(\frac{1}{\tau} \psi_{i^{\prime}}\right)}\log \frac{\exp \left(\frac{1}{\tau} \psi_{i}\right)}{\sum_{i^{\prime}=1}^n \exp \left(\frac{1}{\tau} \psi_{i^{\prime}}\right)} \\
& =\psi_1 - \tau \log \left(\sum_{i=1}^n \exp \left(\frac{1}{\tau} \psi_{i}\right)\right).
\end{aligned}
$$
Thus, the Lagrangian dual function is given by
\begin{equation*}
-E\left(\boldsymbol{\alpha}^*\right)= -\tau \log \frac{\exp \left(\frac{1}{\tau} \psi_{1}\right)}{\sum_{i=1}^n \exp \left(\frac{1}{\tau} \psi_{i}\right)}.\qedhere
\end{equation*}
\end{proof}



\section{More on Experiments} \label{section: experiment_details}

\paragraph{CIFAR-10 and CIFAR-100} CIFAR-10 ~\citep{krizhevsky2009learning} and CIFAR-100 ~\citep{krizhevsky2009learning} are well-known classic image classification datasets. Both CIFAR-10 and CIFAR-100 contain a total of 60k $32 \times 32$ labeled images of different classes, with 50k for training and 10k for testing. CIFAR-10 is similar to CIFAR-100, except there are 10 different classes in CIFAR-10 and 100 classes in CIFAR-100.

\paragraph{TinyImageNet} TinyImageNet ~\citep{le2015tiny} is a subset of ImageNet ~\citep{deng2009imagenet}. There are 200 different object classes in TinyImageNet, with 500 training images, 50 validation images, and 50 test images for each class. All the images in TinyImageNet are colored and labeled with a size of $64 \times 64$.

\textbf{Pseudo-code.} Algorithm \ref{alg:Training Procedure} presents the pseudo-code for our empirical training procedure.

\begin{algorithm}[!htbp]
\caption{Training Procedure}
\label{alg:Training Procedure}
\begin{algorithmic}[1]
\REQUIRE trainable encoder network $f$, batch size $N$, augmentation strategy \textit{aug}, loss function $L$ with hyperparameters \textit{args}
\FOR {sampled minibatch ${x_i}_{i=1}^N$}
\FORALL{$i \in { 1, ..., N }$}
\STATE draw two augmentations $t_i = \textit{aug}\left(x_i\right) $, $t_i' = \textit{aug}\left(x_i\right) $
\STATE $z_i = f\left(t_i\right)$, $z_i' = f\left(t_i'\right)$
\ENDFOR
\STATE compute loss $\mathcal{L} = L(N, z, z', \textit{args})$
\STATE update encoder network $f$ to minimize $\mathcal{L}$
\ENDFOR
\STATE \textbf{Return} encoder network $f$
\end{algorithmic}
\end{algorithm}

We also provide the pseudo-code for our core loss function used in the training procedure in Algorithm \ref{alg:Core loss}. The pseudo-code is almost identical to SimCLR's loss function, with the exception of an extra parameter $\gamma$.

\begin{algorithm}[!htbp]
\caption{Core loss function $\mathcal{C}$}
\label{alg:Core loss}
\begin{algorithmic}[1]
\REQUIRE batch size $N$, two encoded minibatches $z_1, z_2$, $\gamma$, temperature $\tau$
\STATE $z = \textit{concat}\left(z_1, z_2\right)$
\FOR {$i \in {1, ..., 2N }, j \in {1, ..., 2N}$ }
\STATE $s_{i,j} = \Vert z_i - z_j \Vert_2^{\gamma}$
\ENDFOR
\STATE \textbf{define} $l(i, j)$ \textbf{as} $l(i, j) = - \log \frac{exp\left(s_{i,j}/\tau \right)}{\sum_{k=1}^{2N} \mathbf{1}{[k \ne i]} exp\left(s{i, j} / \tau \right)} $
\STATE \textbf{Return} $\frac{1}{2N} \sum_{k=1}^N\left[l(i, i+N) + l(i+N, i)\right]$
\end{algorithmic}
\end{algorithm}

Utilizing the core loss function $\mathcal{C}$, we can define all kernel loss functions used in our experiments in Table \ref{table: loss definition}. For all $z_i \in z$ with even dimensions $n$, we define $z_{L_i} = z_i\left[0:n/2\right]$ and $z_{R_i} = z_i\left[n/2:n\right]$.

\begin{table}[ht]
\centering
\begin{tabular}{{@{}l|l@{}}}
Kernel  &  Loss function \\ \midrule
Laplacian & $\mathcal{C}\left(N, z, z', \gamma=1, \tau\right)$\\ \midrule
Sum       & $\lambda * \mathcal{C}\left(N, z, z', \gamma=1, \tau_1\right) + (1-\lambda) * \mathcal{C}\left(N, z, z', \gamma=2, \tau_2\right)$  \\ \midrule
Concatenation Sum&$\lambda * \mathcal{C}\left(N, z_L, z'_L, \gamma=1, \tau_1\right) + (1-\lambda) * \mathcal{C}\left(N, z_R, z'_R, \gamma=2, \tau_2\right)$\\ \midrule
$\gamma = 0.5$ & $\mathcal{C}\left(N, z, z', \gamma=0.5, \tau\right)$          \\ 

\end{tabular}

\caption{Definition of kernel loss functions in our experiments}
\label {table: loss definition}
\end{table}

\textbf{Baselines.} We reproduce the SimCLR algorithm using PyTorch Lightning~\citep{PytorchLightning}.

\textbf{Encoder details.}
The encoder $f$ consists of a backbone network and a projection network. We employ ResNet50~\citep{ResNet} as the backbone and a 2-layer MLP (connected by a batch normalization~\citep{ioffe2015batch} layer and a ReLU \cite{nair2010rectified} layer) with hidden dimensions 2048 and output dimensions 128 (or 256 in the concatenation kernel case).

\textbf{Encoder hyperparameter tuning.}
For each encoder training case, we randomly sample 500 hyperparameter groups (sample details are shown in Table \ref{table: Hyperparameter sample}) and train these samples simultaneously using Ray Tune ~\citep{RayTune}, with the ASHA scheduler~\citep{li2018massively}. Ultimately, the hyperparameter group that maximizes the online validation accuracy (integrated in PyTorch Lightning) within 5000 validation steps is chosen for the given encoder training case.

\begin{table}[ht]
\centering

\begin{tabular}{@{}l|l|l@{}}
\midrule
Hyperparameter  & Sample Range & Sample Strategy \\ \midrule
start learning rate & $\left[10^{-2}, 10\right]$ & log uniform \\ \midrule
$\lambda$       & $\left[0, 1\right]$ & uniform \\ \midrule
$\tau$, $\tau_1$, $\tau_2$ & $\left[0, 1\right]$ & log uniform \\ \midrule
\end{tabular}

\caption{Hyperparameters sample strategy}
\label {table: Hyperparameter sample}
\end{table}

\textbf{Encoder training.} 
We train each encoder using the LARS optimizer~\citep{LARSOptimizer}, LambdaLR Scheduler in PyTorch, momentum 0.9, weight decay $10^{-6}$, batch size 256, and the aforementioned hyperparameters for 400 epochs on a single A-100 GPU.

\textbf{Image transformation.} The image transformation strategy, including augmentation, is identical to the default transformation strategy provided by PyTorch Lightning.

\textbf{Linear evaluation.}
The linear head is trained using the SGD optimizer with a cosine learning rate scheduler, batch size 64, and weight decay $10^{-6}$ for 100 epochs. The learning rate starts at $0.3$ and ends at $0$.

\textbf{Moco Experiments.} We also tested our method based on MoCo~\citep{he2019moco}. The results are summarized in Table \ref{tab:results-moco}. Here we choose ResNet18~\citep{ResNet} as the backbone and set a temperature of $0.1$ as default. For our simple sum kernel, we set $\lambda=0.8$. The results show that our method outperforms the original MoCo method.

\begin{table}[thb]
\centering
\caption{MoCo Experiment Results on CIFAR-10 and CIFAR-100.}
\label{tab:results-moco}
\resizebox{\textwidth}{!}{%
\begin{tabular}{@{}c|ccc|ccc@{}}
\toprule
\multirow{3}{*}{Method} & \multicolumn{3}{c|}{CIFAR-10} & \multicolumn{3}{c}{CIFAR-100} \\ \cmidrule(lr){2-4} \cmidrule(lr){5-7} 
                        & 200 epochs & 400 epochs    & 1000 epochs   & 200 epochs & 400 epochs & 1000 epochs         \\ \midrule
MoCo (repro.)         & $76.41 \pm 0.12$    & $80.01 \pm 0.15$          & $84.45 \pm 0.08$    & $\mathbf{47.02 \pm 0.11}$ & $52.50 \pm 0.07$ & $57.62 \pm 0.15$            \\
\midrule
Laplacian Kernel        & ${78.09 \pm 0.10}$    & $\mathbf{83.85 \pm 0.09}$          & $\mathbf{88.34 \pm 0.16}$    & $46.12 \pm 0.22$   & $53.44 \pm 0.17$ & $59.10 \pm 0.14$        \\
Simple Sum Kernel & $\mathbf{78.12 \pm 0.15}$   & $83.23 \pm 0.18$ & $87.50 \pm 0.20$ & $46.65 \pm 0.06$ & $\mathbf{53.62 \pm 0.19}$ & $\mathbf{59.83 \pm 0.12}$\\
\bottomrule
\end{tabular}
}
\end{table}



\section{More Experiments on Synthetic Data}


Consider a scenario with $n$ clusters, each containing $k$ vertices. Let the probability of vertices $u$ and $v$ from the same cluster belonging to $\bfpi$ be $p$. Conversely, for vertices $u$ and $v$ from different clusters, let the probability of belonging to $\pi$ be $q$. We generate the graph $\bfpi$ randomly, based on $p$ and $q$. We experiment with values of $k=100$ and $n=6$ for ease of visualization, embedding all points in a two-dimensional space. Each vertex's initial position originates from a normal distribution. In each iteration, we sample a subgraph of $\bfpi$ uniformly, ensuring each vertex has an out-degree of $1$. We then optimize the corresponding vectors using InfoNCE loss with an SGD optimizer and iterate until convergence. Our experimental setup consists of an SGD learning rate of $1$, an InfoNCE loss temperature of $0.5$, and a batch size of $50$. We evaluate two scenarios with different $p$ and $q$ values: $p=1$, $q=0$, and $p=0.75$, $q=0.2$. The results of these experiments are visualized in Figure \ref{fig:vis-spectral-cluster}. The obtained embeddings exhibit the hallmark pattern of spectral clustering of graph $\bfpi$.

\begin{figure}[!tb]
\centering
\subfigure{
\includegraphics[width=1\textwidth]{Figures/cluster_pi.png}
\label{fig:vis-cluster}
}
\subfigure{
\includegraphics[width=1\textwidth]{Figures/noised_cluster_pi.png}
\label{fig:vis-noised-cluster}
}
\caption{Visualizations of the optimization process using InfoNCE Loss on the vectors corresponding to $\bfpi$. Points of identical color belong to the same cluster within $\bfpi$. To showcase the internal structure of $\bfpi$, we randomly select 10 vertices from each cluster to display the edge distribution of $\bfpi$.}
\label{fig:vis-spectral-cluster}
\end{figure}


%%%%%%%%%%%%%%%%%%%%%%%%%%%%%%%%%%%%%%%%%%%% NAI

% that's all folks
\end{document}


