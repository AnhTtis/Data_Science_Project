\documentclass[%
 reprint,
 prd,
 superscriptaddress,
 longbibliography, % Show paper titles in citations
%groupedaddress,
%unsortedaddress,
%runinaddress,
%frontmatterverbose,
%preprint,
%preprintnumbers,
%nofootinbib,
%nobibnotes,
%bibnotes,
 amsmath,amssymb,
 aps,
 floatfix,
]{revtex4-1}

\usepackage[T1]{fontenc} 
\usepackage[utf8]{inputenc}

\usepackage[nowatermark]{fixmetodonotes}

\usepackage{graphicx} % Include figure files
\usepackage{dcolumn} % Align table columns on decimal point
\usepackage{bm} % bold math
%\usepackage[mathlines]{lineno} % Enable numbering of text and display math
%\linenumbers\relax % Commence numbering lines

\usepackage{float} % Helps force figures to end up just where you want them.
\usepackage[inline]{enumitem}
\usepackage{mathtools}
\usepackage{euscript}
\usepackage{mathrsfs}
\usepackage{braket}
\usepackage{ gensymb }
\usepackage{ upgreek }
\usepackage{nicefrac}
\usepackage{ulem,color,graphics}
\usepackage[nowatermark]{fixmetodonotes}
\usepackage{isotope}
\usepackage{multirow}

\DeclarePairedDelimiter\abs{\lvert}{\rvert}
\makeatletter
\let\oldabs\abs
\def\abs{\@ifstar{\oldabs}{\oldabs*}}

%images
\graphicspath{ {images_updated/} }

\usepackage[colorlinks, urlcolor=blue]{hyperref}

% The cleveref package must be loaded after hyperref!
\usepackage{cleveref}
\crefrangeformat{equation}{Eqs. #3(#1)#4--#5(#2)#6}
% to erase the line number, please comment line 54
\usepackage{lineno}
%\linenumbers
%
\begin{document}

%TC:ignore
\title{Impact of cross-section uncertainties on supernova neutrino spectral
parameter fitting in the Deep Underground Neutrino Experiment}

% RevTeX-style author list produced at 2023-03-08, from master list generated at 2023-02-09
%

\newcommand{\Abilene}{Abilene Christian University, Abilene, TX 79601, USA}
\newcommand{\Albanysuny}{University of Albany, SUNY, Albany, NY 12222, USA}
\newcommand{\Amsterdam}{University of Amsterdam, NL-1098 XG Amsterdam, The Netherlands}
\newcommand{\Antalya}{Antalya Bilim University, 07190 D{\"o}{\c{s}}emealt{\i}/Antalya, Turkey}
\newcommand{\Antananarivo}{University of Antananarivo, Antananarivo 101, Madagascar}
\newcommand{\AntonioNarino}{Universidad Antonio Nari{\~n}o, Bogot{\'a}, Colombia}
\newcommand{\Argonne}{Argonne National Laboratory, Argonne, IL 60439, USA}
\newcommand{\Arizona}{University of Arizona, Tucson, AZ 85721, USA}
\newcommand{\Asuncion}{Universidad Nacional de Asunci{\'o}n, San Lorenzo, Paraguay}
\newcommand{\Athens}{University of Athens, Zografou GR 157 84, Greece}
\newcommand{\Atlantico}{Universidad del Atl{\'a}ntico, Barranquilla, Atl{\'a}ntico, Colombia}
\newcommand{\Augustana}{Augustana University, Sioux Falls, SD 57197, USA}
\newcommand{\Bern}{University of Bern, CH-3012 Bern, Switzerland}
\newcommand{\Beykent}{Beykent University, Istanbul, Turkey}
\newcommand{\Birmingham}{University of Birmingham, Birmingham B15 2TT, United Kingdom}
\newcommand{\BolognaUniversity}{Universit{\`a} del Bologna, 40127 Bologna, Italy}
\newcommand{\Boston}{Boston University, Boston, MA 02215, USA}
\newcommand{\Bristol}{University of Bristol, Bristol BS8 1TL, United Kingdom}
\newcommand{\Brookhaven}{Brookhaven National Laboratory, Upton, NY 11973, USA}
\newcommand{\Bucharest}{University of Bucharest, Bucharest, Romania}
\newcommand{\CalBerkeley}{University of California Berkeley, Berkeley, CA 94720, USA}
\newcommand{\CalDavis}{University of California Davis, Davis, CA 95616, USA}
\newcommand{\CalIrvine}{University of California Irvine, Irvine, CA 92697, USA}
\newcommand{\CalLosangeles}{University of California Los Angeles, Los Angeles, CA 90095, USA}
\newcommand{\CalRiverside}{University of California Riverside, Riverside CA 92521, USA}
\newcommand{\CalSantabarbara}{University of California Santa Barbara, Santa Barbara, California 93106 USA}
\newcommand{\Caltech}{California Institute of Technology, Pasadena, CA 91125, USA}
\newcommand{\Cambridge}{University of Cambridge, Cambridge CB3 0HE, United Kingdom}
\newcommand{\Campinas}{Universidade Estadual de Campinas, Campinas - SP, 13083-970, Brazil}
\newcommand{\CataniaUniversitadi}{Universit{\`a} di Catania, 2 - 95131 Catania, Italy}
\newcommand{\Catolica}{Universidad Cat{\'o}lica del Norte, Antofagasta, Chile}
\newcommand{\CBPF}{Centro Brasileiro de Pesquisas F\'isicas, Rio de Janeiro, RJ 22290-180, Brazil}
\newcommand{\CEASaclay}{IRFU, CEA, Universit{\'e} Paris-Saclay, F-91191 Gif-sur-Yvette, France}
\newcommand{\CERN}{CERN, The European Organization for Nuclear Research, 1211 Meyrin, Switzerland}
\newcommand{\Charles}{Institute of Particle and Nuclear Physics of the Faculty of Mathematics and Physics of the Charles University, 180 00 Prague 8, Czech Republic }
\newcommand{\Chicago}{University of Chicago, Chicago, IL 60637, USA}
\newcommand{\ChungAng}{Chung-Ang University, Seoul 06974, South Korea}
\newcommand{\CIEMAT}{CIEMAT, Centro de Investigaciones Energ{\'e}ticas, Medioambientales y Tecnol{\'o}gicas, E-28040 Madrid, Spain}
\newcommand{\Cincinnati}{University of Cincinnati, Cincinnati, OH 45221, USA}
\newcommand{\Cinvestav}{Centro de Investigaci{\'o}n y de Estudios Avanzados del Instituto Polit{\'e}cnico Nacional (Cinvestav), Mexico City, Mexico}
\newcommand{\Colima}{Universidad de Colima, Colima, Mexico}
\newcommand{\ColoradoBoulder}{University of Colorado Boulder, Boulder, CO 80309, USA}
\newcommand{\ColoradoState}{Colorado State University, Fort Collins, CO 80523, USA}
\newcommand{\Columbia}{Columbia University, New York, NY 10027, USA}
\newcommand{\conida}{Comisi{\'o}n Nacional de Investigaci{\'o}n y Desarrollo Aeroespacial, Lima, Peru}
\newcommand{\Cti}{Centro de Tecnologia da Informacao Renato Archer, Amarais - Campinas, SP - CEP 13069-901}
\newcommand{\CUSB}{Central University of South Bihar, Gaya, 824236, India }
\newcommand{\CzechAcademyofSciences}{Institute of Physics, Czech Academy of Sciences, 182 00 Prague 8, Czech Republic}
\newcommand{\CzechTechnical}{Czech Technical University, 115 19 Prague 1, Czech Republic}
\newcommand{\DakotaState}{Dakota State University, Madison, SD 57042, USA}
\newcommand{\Dallas}{University of Dallas, Irving, TX 75062-4736, USA}
\newcommand{\DannecyleVieux}{Laboratoire d{\textquoteright}Annecy de Physique des Particules, Univ. Grenoble Alpes, Univ. Savoie Mont Blanc, CNRS, LAPP-IN2P3, 74000 Annecy, France}
\newcommand{\Daresbury}{Daresbury Laboratory, Cheshire WA4 4AD, United Kingdom}
\newcommand{\Dordt}{Dordt University, 700 7th St NE, Sioux Center, IA 51250, USA}
\newcommand{\Drexel}{Drexel University, Philadelphia, PA 19104, USA}
\newcommand{\Duke}{Duke University, Durham, NC 27708, USA}
\newcommand{\Durham}{Durham University, Durham DH1 3LE, United Kingdom}
\newcommand{\Edinburgh}{University of Edinburgh, Edinburgh EH8 9YL, United Kingdom}
\newcommand{\EIA}{Universidad EIA, Envigado, Antioquia, Colombia}
%\newcommand{\Eötvös}{E{\"o}tv{\"o}s Lor{\'a}nd University, 1053 Budapest, Hungary}
\newcommand{\Eotvos}{E{\"o}tv{\"o}s Lor{\'a}nd University, 1053 Budapest, Hungary}
\newcommand{\FCULport}{Faculdade de Ci{\^e}ncias da Universidade de Lisboa - FCUL, 1749-016 Lisboa, Portugal}
\newcommand{\FederaldeAlfenas}{Universidade Federal de Alfenas, Po{\c{c}}os de Caldas - MG, 37715-400, Brazil}
\newcommand{\FederaldeGoias}{Universidade Federal de Goias, Goiania, GO 74690-900, Brazil}
\newcommand{\FederaldoABC}{Universidade Federal do ABC, Santo Andr{\'e} - SP, 09210-580, Brazil}
\newcommand{\FederaldoRio}{Universidade Federal do Rio de Janeiro,  Rio de Janeiro - RJ, 21941-901, Brazil}
\newcommand{\Fermi}{Fermi National Accelerator Laboratory, Batavia, IL 60510, USA}
\newcommand{\Ferrarauniv}{University of Ferrara, Ferrara, Italy}
\newcommand{\Florida}{University of Florida, Gainesville, FL 32611-8440, USA}
\newcommand{\Floridastate}{Florida State University, Tallahassee, FL, USA}
\newcommand{\Fluminense}{Fluminense Federal University, 9 Icara{\'\i} Niter{\'o}i - RJ, 24220-900, Brazil }
\newcommand{\Genova}{Universit{\`a} degli Studi di Genova, Genova, Italy}
\newcommand{\Georgian}{Georgian Technical University, Tbilisi, Georgia}
\newcommand{\Granada}{University of Granada {\&} CAFPE, 18002 Granada, Spain}
\newcommand{\GranSasso}{Gran Sasso Science Institute, L'Aquila, Italy}
\newcommand{\GranSassoLab}{Laboratori Nazionali del Gran Sasso, L'Aquila AQ, Italy}
\newcommand{\Grenoble}{University Grenoble Alpes, CNRS, Grenoble INP, LPSC-IN2P3, 38000 Grenoble, France}
\newcommand{\Guanajuato}{Universidad de Guanajuato, Guanajuato, C.P. 37000, Mexico}
\newcommand{\Harish}{Harish-Chandra Research Institute, Jhunsi, Allahabad 211 019, India}
\newcommand{\Hawaii}{University of Hawaii, Honolulu, HI 96822, USA}
\newcommand{\Houston}{University of Houston, Houston, TX 77204, USA}
\newcommand{\Hyderabad}{University of  Hyderabad, Gachibowli, Hyderabad - 500 046, India}
\newcommand{\Idaho}{Idaho State University, Pocatello, ID 83209, USA}
\newcommand{\IFAE}{Institut de F{\'\i}sica d{\textquoteright}Altes Energies (IFAE){\textemdash}Barcelona Institute of Science and Technology (BIST), Barcelona, Spain}
\newcommand{\IFIC}{Instituto de F{\'\i}sica Corpuscular, CSIC and Universitat de Val{\`e}ncia, 46980 Paterna, Valencia, Spain}
\newcommand{\IGFAE}{Instituto Galego de F{\'\i}sica de Altas Enerx{\'\i}as, University of Santiago de Compostela, Santiago de Compostela, 15782, Spain}
\newcommand{\Iitk}{Indian Institute of Technology Kanpur, Uttar Pradesh 208016, India}
\newcommand{\Illinoisinstitute}{Illinois Institute of Technology, Chicago, IL 60616, USA}
\newcommand{\Imperial}{Imperial College of Science Technology and Medicine, London SW7 2BZ, United Kingdom}
\newcommand{\IndGuwahati}{Indian Institute of Technology Guwahati, Guwahati, 781 039, India}
\newcommand{\IndHyderabad}{Indian Institute of Technology Hyderabad, Hyderabad, 502285, India}
\newcommand{\Indiana}{Indiana University, Bloomington, IN 47405, USA}
\newcommand{\INFNBologna}{Istituto Nazionale di Fisica Nucleare Sezione di Bologna, 40127 Bologna BO, Italy}
\newcommand{\INFNCatania}{Istituto Nazionale di Fisica Nucleare Sezione di Catania, I-95123 Catania, Italy}
\newcommand{\INFNFerrara}{Istituto Nazionale di Fisica Nucleare Sezione di Ferrara, I-44122 Ferrara, Italy}
\newcommand{\INFNFrascati}{Istituto Nazionale di Fisica Nucleare Laboratori Nazionali di Frascati, Frascati, Roma, Italy}
\newcommand{\INFNGenova}{Istituto Nazionale di Fisica Nucleare Sezione di Genova, 16146 Genova GE, Italy}
\newcommand{\INFNLecce}{Istituto Nazionale di Fisica Nucleare Sezione di Lecce, 73100 - Lecce, Italy}
\newcommand{\INFNMilanBicocca}{Istituto Nazionale di Fisica Nucleare Sezione di Milano Bicocca, 3 - I-20126 Milano, Italy}
\newcommand{\INFNMilano}{Istituto Nazionale di Fisica Nucleare Sezione di Milano, 20133 Milano, Italy}
\newcommand{\INFNNapoli}{Istituto Nazionale di Fisica Nucleare Sezione di Napoli, I-80126 Napoli, Italy}
\newcommand{\INFNPadova}{Istituto Nazionale di Fisica Nucleare Sezione di Padova, 35131 Padova, Italy}
\newcommand{\INFNPavia}{Istituto Nazionale di Fisica Nucleare Sezione di Pavia,  I-27100 Pavia, Italy}
\newcommand{\INFNPisa}{Istituto Nazionale di Fisica Nucleare Laboratori Nazionali di Pisa, Pisa PI, Italy}
\newcommand{\INFNRoma}{Istituto Nazionale di Fisica Nucleare Sezione di Roma, 00185 Roma RM, Italy}
\newcommand{\INFNSud}{Istituto Nazionale di Fisica Nucleare Laboratori Nazionali del Sud, 95123 Catania, Italy}
\newcommand{\Ingenieria}{Universidad Nacional de Ingenier{\'\i}a, Lima 25, Per{\'u}}
\newcommand{\INR}{Institute for Nuclear Research of the Russian Academy of Sciences, Moscow 117312, Russia}
\newcommand{\Insubria }{University of Insubria, Via Ravasi, 2, 21100 Varese VA, Italy}
\newcommand{\Iowa}{University of Iowa, Iowa City, IA 52242, USA}
\newcommand{\IowaState}{Iowa State University, Ames, Iowa 50011, USA}
\newcommand{\IPLyon}{Institut de Physique des 2 Infinis de Lyon, 69622 Villeurbanne, France}
\newcommand{\IPM}{Institute for Research in Fundamental Sciences, Tehran, Iran}
\newcommand{\ISTlisboa}{Instituto Superior T{\'e}cnico - IST, Universidade de Lisboa, Portugal}
\newcommand{\Ita}{Instituto Tecnol{\'o}gico de Aeron{\'a}utica, Sao Jose dos Campos, Brazil}
\newcommand{\Iwate}{Iwate University, Morioka, Iwate 020-8551, Japan}
\newcommand{\Jacksonstate}{Jackson State University, Jackson, MS 39217, USA}
\newcommand{\Jammu}{University of Jammu, Jammu-180006, India}
\newcommand{\Jawaharlal}{Jawaharlal Nehru University, New Delhi 110067, India}
\newcommand{\Jeonbuk}{Jeonbuk National University, Jeonrabuk-do 54896, South Korea}
\newcommand{\JINR}{Joint Institute for Nuclear Research, Dzhelepov Laboratory of Nuclear Problems 6 Joliot-Curie, Dubna, Moscow Region, 141980 RU }
\newcommand{\Jyvaskyla}{University of Jyvaskyla, FI-40014, Finland}
\newcommand{\Kansasstate}{Kansas State University, Manhattan, KS 66506, USA}
\newcommand{\Kavli}{Kavli Institute for the Physics and Mathematics of the Universe, Kashiwa, Chiba 277-8583, Japan}
\newcommand{\KEK}{High Energy Accelerator Research Organization (KEK), Ibaraki, 305-0801, Japan}
\newcommand{\KISTI}{Korea Institute of Science and Technology Information, Daejeon, 34141, South Korea}
\newcommand{\KL}{K L University, Vaddeswaram, Andhra Pradesh 522502, India}
\newcommand{\Kure}{National Institute of Technology, Kure College, Hiroshima, 737-8506, Japan}
\newcommand{\Kyiv}{Taras Shevchenko National University of Kyiv, 01601 Kyiv, Ukraine}
\newcommand{\Lancaster}{Lancaster University, Lancaster LA1 4YB, United Kingdom}
\newcommand{\LawrenceBerkeley}{Lawrence Berkeley National Laboratory, Berkeley, CA 94720, USA}
\newcommand{\LIP}{Laborat{\'o}rio de Instrumenta{\c{c}}{\~a}o e F{\'\i}sica Experimental de Part{\'\i}culas, 1649-003 Lisboa and 3004-516 Coimbra, Portugal}
\newcommand{\Liverpool}{University of Liverpool, L69 7ZE, Liverpool, United Kingdom}
\newcommand{\LosAlmos}{Los Alamos National Laboratory, Los Alamos, NM 87545, USA}
\newcommand{\Louisanastate}{Louisiana State University, Baton Rouge, LA 70803, USA}
\newcommand{\Lucknow}{University of Lucknow, Uttar Pradesh 226007, India}
\newcommand{\Madrid}{Madrid Autonoma University and IFT UAM/CSIC, 28049 Madrid, Spain}
\newcommand{\Mainz}{Johannes Gutenberg-Universit{\"a}t Mainz, 55122 Mainz, Germany}
\newcommand{\Manchester}{University of Manchester, Manchester M13 9PL, United Kingdom}
\newcommand{\Massinsttech}{Massachusetts Institute of Technology, Cambridge, MA 02139, USA}
\newcommand{\Maxplanck}{Max-Planck-Institut, Munich, 80805, Germany}
\newcommand{\Medellin}{University of Medell{\'\i}n, Medell{\'\i}n, 050026 Colombia }
\newcommand{\Michigan}{University of Michigan, Ann Arbor, MI 48109, USA}
\newcommand{\Michiganstate}{Michigan State University, East Lansing, MI 48824, USA}
\newcommand{\MilanoBicocca}{Universit{\`a} del Milano-Bicocca, 20126 Milano, Italy}
\newcommand{\MilanoUniv}{Universit{\`a} degli Studi di Milano, I-20133 Milano, Italy}
\newcommand{\Minnduluth}{University of Minnesota Duluth, Duluth, MN 55812, USA}
\newcommand{\Minntwin}{University of Minnesota Twin Cities, Minneapolis, MN 55455, USA}
\newcommand{\Mississippi}{University of Mississippi, University, MS 38677 USA}
\newcommand{\napoli}{Universit{\`a} degli Studi di Napoli Federico II , 80138 Napoli NA, Italy}
\newcommand{\Nikhef}{Nikhef National Institute of Subatomic Physics, 1098 XG Amsterdam, Netherlands}
\newcommand{\Niser}{National Institute of Science Education and Research (NISER), Odisha 752050, India}
\newcommand{\Northdakota}{University of North Dakota, Grand Forks, ND 58202-8357, USA}
\newcommand{\Northernillinois}{Northern Illinois University, DeKalb, IL 60115, USA}
\newcommand{\Northwestern}{Northwestern University, Evanston, Il 60208, USA}
\newcommand{\NotreDame}{University of Notre Dame, Notre Dame, IN 46556, USA}
\newcommand{\NoviSad}{University of Novi Sad, 21102 Novi Sad, Serbia}
\newcommand{\Occidental}{Occidental College, Los Angeles, CA  90041}
\newcommand{\Ohiostate}{Ohio State University, Columbus, OH 43210, USA}
\newcommand{\OregonState}{Oregon State University, Corvallis, OR 97331, USA}
\newcommand{\Oxford}{University of Oxford, Oxford, OX1 3RH, United Kingdom}
\newcommand{\PacificNorthwest}{Pacific Northwest National Laboratory, Richland, WA 99352, USA}
\newcommand{\Padova}{Universt{\`a} degli Studi di Padova, I-35131 Padova, Italy}
\newcommand{\Panjab}{Panjab University, Chandigarh, 160014 U.T., India}
\newcommand{\Parissaclay}{Universit{\'e} Paris-Saclay, CNRS/IN2P3, IJCLab, 91405 Orsay, France}
\newcommand{\Parisuniversite}{Universit{\'e} Paris Cit{\'e}, CNRS, Astroparticule et Cosmologie, Paris, France}
\newcommand{\Parma}{University of Parma,  43121 Parma PR, Italy}
\newcommand{\Pavia}{Universit{\`a} degli Studi di Pavia, 27100 Pavia PV, Italy}
\newcommand{\Penn}{University of Pennsylvania, Philadelphia, PA 19104, USA}
\newcommand{\PennState}{Pennsylvania State University, University Park, PA 16802, USA}
\newcommand{\PhysicalResearchLaboratory}{Physical Research Laboratory, Ahmedabad 380 009, India}
\newcommand{\Pisa}{Universit{\`a} di Pisa, I-56127 Pisa, Italy}
\newcommand{\Pitt}{University of Pittsburgh, Pittsburgh, PA 15260, USA}
\newcommand{\Pontificia}{Pontificia Universidad Cat{\'o}lica del Per{\'u}, Lima, Per{\'u}}
\newcommand{\PuertoRico}{University of Puerto Rico, Mayaguez 00681, Puerto Rico, USA}
\newcommand{\Punjab}{Punjab Agricultural University, Ludhiana 141004, India}
\newcommand{\QMUL}{Queen Mary University of London, London E1 4NS, United Kingdom }
\newcommand{\Radboud}{Radboud University, NL-6525 AJ Nijmegen, Netherlands}
\newcommand{\Rice}{Rice University, Houston, TX 77005}
\newcommand{\Rochester}{University of Rochester, Rochester, NY 14627, USA}
\newcommand{\Royalholloway}{Royal Holloway College London, TW20 0EX, United Kingdom}
\newcommand{\Rutgers}{Rutgers University, Piscataway, NJ, 08854, USA}
\newcommand{\Rutherford}{STFC Rutherford Appleton Laboratory, Didcot OX11 0QX, United Kingdom}
\newcommand{\Salento}{Universit{\`a} del Salento, 73100 Lecce, Italy}
\newcommand{\Sanjosestate}{San Jose State University, San Jos{\'e}, CA 95192-0106, USA}
\newcommand{\SantaCruz}{Universidade Estadual de Santa Cruz, CEP 45662-000, Ilhe{\'u}s, Bahia-BA, Brazil}
\newcommand{\Sapienza}{Sapienza University of Rome, 00185 Roma RM, Italy}
\newcommand{\SergioArboleda}{Universidad Sergio Arboleda, 11022 Bogot{\'a}, Colombia}
\newcommand{\Sheffield}{University of Sheffield, Sheffield S3 7RH, United Kingdom}
\newcommand{\SLAC}{SLAC National Accelerator Laboratory, Menlo Park, CA 94025, USA}
\newcommand{\Southcarolina}{University of South Carolina, Columbia, SC 29208, USA}
\newcommand{\SouthDakotaSchool}{South Dakota School of Mines and Technology, Rapid City, SD 57701, USA}
\newcommand{\SouthDakotaState}{South Dakota State University, Brookings, SD 57007, USA}
\newcommand{\SouthernMethodist}{Southern Methodist University, Dallas, TX 75275, USA}
\newcommand{\StonyBrook}{Stony Brook University, SUNY, Stony Brook, NY 11794, USA}
\newcommand{\Sunyatsen}{Sun Yat-Sen University, Guangzhou, 510275}
\newcommand{\SURF}{Sanford Underground Research Facility, Lead, SD, 57754, USA}
\newcommand{\Sussex}{University of Sussex, Brighton, BN1 9RH, United Kingdom}
\newcommand{\Syracuse}{Syracuse University, Syracuse, NY 13244, USA}
\newcommand{\Tecnologica }{Universidade Tecnol{\'o}gica Federal do Paran{\'a}, Curitiba, Brazil}
\newcommand{\TelAviv}{Tel Aviv University, Tel Aviv-Yafo, Israel}
\newcommand{\TexasAMcollege}{Texas A{\&}M University, College Station, Texas 77840}
\newcommand{\TexasAMcorpuscristi}{Texas A{\&}M University - Corpus Christi, Corpus Christi, TX 78412, USA}
\newcommand{\TexasArlington}{University of Texas at Arlington, Arlington, TX 76019, USA}
\newcommand{\Texasaustin}{University of Texas at Austin, Austin, TX 78712, USA}
\newcommand{\Toronto}{University of Toronto, Toronto, Ontario M5S 1A1, Canada}
\newcommand{\Tufts}{Tufts University, Medford, MA 02155, USA}
\newcommand{\Unifesp}{Universidade Federal de S{\~a}o Paulo, 09913-030, S{\~a}o Paulo, Brazil}
\newcommand{\UNIST}{Ulsan National Institute of Science and Technology, Ulsan 689-798, South Korea}
\newcommand{\UniversityCollegeLondon}{University College London, London, WC1E 6BT, United Kingdom}
\newcommand{\ValleyCity}{Valley City State University, Valley City, ND 58072, USA}
\newcommand{\VariableEnergy}{Variable Energy Cyclotron Centre, 700 064 West Bengal, India}
\newcommand{\VirginiaTech}{Virginia Tech, Blacksburg, VA 24060, USA}
\newcommand{\Warsaw}{University of Warsaw, 02-093 Warsaw, Poland}
\newcommand{\Warwick}{University of Warwick, Coventry CV4 7AL, United Kingdom}
\newcommand{\Wellesley}{Wellesley College, Wellesley, MA 02481, USA}
\newcommand{\Wichita}{Wichita State University, Wichita, KS 67260, USA}
\newcommand{\WilliamMary}{William and Mary, Williamsburg, VA 23187, USA}
\newcommand{\Wisconsin}{University of Wisconsin Madison, Madison, WI 53706, USA}
\newcommand{\Yale}{Yale University, New Haven, CT 06520, USA}
\newcommand{\Yerevan}{Yerevan Institute for Theoretical Physics and Modeling, Yerevan 0036, Armenia}
\newcommand{\York}{York University, Toronto M3J 1P3, Canada}
%----------------------------------------------------
% Institutions in alphabetical order
\affiliation{\Abilene}
\affiliation{\Albanysuny}
\affiliation{\Amsterdam}
\affiliation{\Antalya}
\affiliation{\Antananarivo}
\affiliation{\AntonioNarino}
\affiliation{\Argonne}
\affiliation{\Arizona}
\affiliation{\Asuncion}
\affiliation{\Athens}
\affiliation{\Atlantico}
\affiliation{\Augustana}
\affiliation{\Bern}
\affiliation{\Beykent}
\affiliation{\Birmingham}
\affiliation{\BolognaUniversity}
\affiliation{\Boston}
\affiliation{\Bristol}
\affiliation{\Brookhaven}
\affiliation{\Bucharest}
\affiliation{\CalBerkeley}
\affiliation{\CalDavis}
\affiliation{\CalIrvine}
\affiliation{\CalLosangeles}
\affiliation{\CalRiverside}
\affiliation{\CalSantabarbara}
\affiliation{\Caltech}
\affiliation{\Cambridge}
\affiliation{\Campinas}
\affiliation{\CataniaUniversitadi}
\affiliation{\Catolica}
\affiliation{\CBPF}
\affiliation{\CEASaclay}
\affiliation{\CERN}
\affiliation{\Charles}
\affiliation{\Chicago}
\affiliation{\ChungAng}
\affiliation{\CIEMAT}
\affiliation{\Cincinnati}
\affiliation{\Cinvestav}
\affiliation{\Colima}
\affiliation{\ColoradoBoulder}
\affiliation{\ColoradoState}
\affiliation{\Columbia}
\affiliation{\conida}
\affiliation{\Cti}
\affiliation{\CUSB}
\affiliation{\CzechAcademyofSciences}
\affiliation{\CzechTechnical}
\affiliation{\DakotaState}
\affiliation{\Dallas}
\affiliation{\DannecyleVieux}
\affiliation{\Daresbury}
\affiliation{\Dordt}
\affiliation{\Drexel}
\affiliation{\Duke}
\affiliation{\Durham}
\affiliation{\Edinburgh}
\affiliation{\EIA}
%\affiliation{\Eötvös}
\affiliation{\Eotvos}
\affiliation{\FCULport}
\affiliation{\FederaldeAlfenas}
\affiliation{\FederaldeGoias}
\affiliation{\FederaldoABC}
\affiliation{\FederaldoRio}
\affiliation{\Fermi}
\affiliation{\Ferrarauniv}
\affiliation{\Florida}
\affiliation{\Floridastate}
\affiliation{\Fluminense}
\affiliation{\Genova}
\affiliation{\Georgian}
\affiliation{\Granada}
\affiliation{\GranSasso}
\affiliation{\GranSassoLab}
\affiliation{\Grenoble}
\affiliation{\Guanajuato}
\affiliation{\Harish}
\affiliation{\Hawaii}
\affiliation{\Houston}
\affiliation{\Hyderabad}
\affiliation{\Idaho}
\affiliation{\IFAE}
\affiliation{\IFIC}
\affiliation{\IGFAE}
\affiliation{\Iitk}
\affiliation{\Illinoisinstitute}
\affiliation{\Imperial}
\affiliation{\IndGuwahati}
\affiliation{\IndHyderabad}
\affiliation{\Indiana}
\affiliation{\INFNBologna}
\affiliation{\INFNCatania}
\affiliation{\INFNFerrara}
\affiliation{\INFNFrascati}
\affiliation{\INFNGenova}
\affiliation{\INFNLecce}
\affiliation{\INFNMilanBicocca}
\affiliation{\INFNMilano}
\affiliation{\INFNNapoli}
\affiliation{\INFNPadova}
\affiliation{\INFNPavia}
\affiliation{\INFNPisa}
\affiliation{\INFNRoma}
\affiliation{\INFNSud}
\affiliation{\Ingenieria}
\affiliation{\INR}
\affiliation{\Insubria }
\affiliation{\Iowa}
\affiliation{\IowaState}
\affiliation{\IPLyon}
\affiliation{\IPM}
\affiliation{\ISTlisboa}
\affiliation{\Ita}
\affiliation{\Iwate}
\affiliation{\Jacksonstate}
\affiliation{\Jammu}
\affiliation{\Jawaharlal}
\affiliation{\Jeonbuk}
\affiliation{\JINR}
\affiliation{\Jyvaskyla}
\affiliation{\Kansasstate}
\affiliation{\Kavli}
\affiliation{\KEK}
\affiliation{\KISTI}
\affiliation{\KL}
\affiliation{\Kure}
\affiliation{\Kyiv}
\affiliation{\Lancaster}
\affiliation{\LawrenceBerkeley}
\affiliation{\LIP}
\affiliation{\Liverpool}
\affiliation{\LosAlmos}
\affiliation{\Louisanastate}
\affiliation{\Lucknow}
\affiliation{\Madrid}
\affiliation{\Mainz}
\affiliation{\Manchester}
\affiliation{\Massinsttech}
\affiliation{\Maxplanck}
\affiliation{\Medellin}
\affiliation{\Michigan}
\affiliation{\Michiganstate}
\affiliation{\MilanoBicocca}
\affiliation{\MilanoUniv}
\affiliation{\Minnduluth}
\affiliation{\Minntwin}
\affiliation{\Mississippi}
\affiliation{\napoli}
\affiliation{\Nikhef}
\affiliation{\Niser}
\affiliation{\Northdakota}
\affiliation{\Northernillinois}
\affiliation{\Northwestern}
\affiliation{\NotreDame}
\affiliation{\NoviSad}
\affiliation{\Occidental}
\affiliation{\Ohiostate}
\affiliation{\OregonState}
\affiliation{\Oxford}
\affiliation{\PacificNorthwest}
\affiliation{\Padova}
\affiliation{\Panjab}
\affiliation{\Parissaclay}
\affiliation{\Parisuniversite}
\affiliation{\Parma}
\affiliation{\Pavia}
\affiliation{\Penn}
\affiliation{\PennState}
\affiliation{\PhysicalResearchLaboratory}
\affiliation{\Pisa}
\affiliation{\Pitt}
\affiliation{\Pontificia}
\affiliation{\PuertoRico}
\affiliation{\Punjab}
\affiliation{\QMUL}
\affiliation{\Radboud}
\affiliation{\Rice}
\affiliation{\Rochester}
\affiliation{\Royalholloway}
\affiliation{\Rutgers}
\affiliation{\Rutherford}
\affiliation{\Salento}
\affiliation{\Sanjosestate}
\affiliation{\SantaCruz}
\affiliation{\Sapienza}
\affiliation{\SergioArboleda}
\affiliation{\Sheffield}
\affiliation{\SLAC}
\affiliation{\Southcarolina}
\affiliation{\SouthDakotaSchool}
\affiliation{\SouthDakotaState}
\affiliation{\SouthernMethodist}
\affiliation{\StonyBrook}
\affiliation{\Sunyatsen}
\affiliation{\SURF}
\affiliation{\Sussex}
\affiliation{\Syracuse}
\affiliation{\Tecnologica }
\affiliation{\TelAviv}
\affiliation{\TexasAMcollege}
\affiliation{\TexasAMcorpuscristi}
\affiliation{\TexasArlington}
\affiliation{\Texasaustin}
\affiliation{\Toronto}
\affiliation{\Tufts}
\affiliation{\Unifesp}
\affiliation{\UNIST}
\affiliation{\UniversityCollegeLondon}
\affiliation{\ValleyCity}
\affiliation{\VariableEnergy}
\affiliation{\VirginiaTech}
\affiliation{\Warsaw}
\affiliation{\Warwick}
\affiliation{\Wellesley}
\affiliation{\Wichita}
\affiliation{\WilliamMary}
\affiliation{\Wisconsin}
\affiliation{\Yale}
\affiliation{\Yerevan}
\affiliation{\York}
%----------------------------------------------------
% Authors in alphabetical order
\author{A.~Abed Abud} \affiliation{\CERN}
\author{B.~Abi} \affiliation{\Oxford}
\author{R.~Acciarri} \affiliation{\Fermi}
\author{M.~A.~Acero} \affiliation{\Atlantico}
\author{M.~R.~Adames} \affiliation{\Tecnologica }
\author{G.~Adamov} \affiliation{\Georgian}
\author{M.~Adamowski} \affiliation{\Fermi}
\author{D.~Adams} \affiliation{\Brookhaven}
\author{M.~Adinolfi} \affiliation{\Bristol}
\author{C.~Adriano} \affiliation{\Campinas}
\author{A.~Aduszkiewicz} \affiliation{\Houston}
\author{J.~Aguilar} \affiliation{\LawrenceBerkeley}
\author{Z.~Ahmad} \affiliation{\VariableEnergy}
\author{J.~Ahmed} \affiliation{\Warwick}
\author{B.~Aimard} \affiliation{\DannecyleVieux}
\author{F.~Akbar} \affiliation{\Rochester}
\author{K.~Allison} \affiliation{\ColoradoBoulder}
\author{S.~Alonso Monsalve} \affiliation{\CERN}
\author{M.~Alrashed} \affiliation{\Kansasstate}
\author{A.~Alton} \affiliation{\Augustana}
\author{R.~Alvarez} \affiliation{\CIEMAT}
\author{P.~Amedo} \affiliation{\IGFAE}\affiliation{\IFIC}
\author{J.~Anderson} \affiliation{\Argonne}
\author{D. A. ~Andrade} \affiliation{\Illinoisinstitute}
\author{C.~Andreopoulos} \affiliation{\Rutherford}\affiliation{\Liverpool}
\author{M.~Andreotti} \affiliation{\INFNFerrara}\affiliation{\Ferrarauniv}
\author{M.~P.~Andrews} \affiliation{\Fermi}
\author{F.~Andrianala} \affiliation{\Antananarivo}
\author{S.~Andringa} \affiliation{\LIP}
\author{N.~Anfimov} \affiliation{\JINR}
\author{W.~L.~Anic{\'e}zio Campanelli} \affiliation{\FederaldeAlfenas}
\author{A.~Ankowski} \affiliation{\SLAC}
\author{M.~Antoniassi} \affiliation{\Tecnologica }
\author{M.~Antonova} \affiliation{\IFIC}
\author{A.~Antoshkin} \affiliation{\JINR}
\author{A.~Aranda-Fernandez} \affiliation{\Colima}
\author{L.~Arellano} \affiliation{\Manchester}
\author{L.~O.~Arnold} \affiliation{\Columbia}
\author{M.~A.~Arroyave} \affiliation{\EIA}
\author{J.~Asaadi} \affiliation{\TexasArlington}
\author{A.~Ashkenazi} \affiliation{\TelAviv}
\author{L.~Asquith} \affiliation{\Sussex}
\author{E.~Atkin} \affiliation{\Imperial}
\author{D.~Auguste} \affiliation{\Parissaclay}
\author{A.~Aurisano} \affiliation{\Cincinnati}
\author{V.~Aushev} \affiliation{\Kyiv}
\author{D.~Autiero} \affiliation{\IPLyon}
\author{M.~Ayala-Torres} \affiliation{\Cinvestav}
\author{F.~Azfar} \affiliation{\Oxford}
\author{A.~Back} \affiliation{\Indiana}
\author{H.~Back} \affiliation{\PacificNorthwest}
\author{J.~J.~Back} \affiliation{\Warwick}
\author{I.~Bagaturia} \affiliation{\Georgian}
\author{L.~Bagby} \affiliation{\Fermi}
\author{N.~Balashov} \affiliation{\JINR}
\author{S.~Balasubramanian} \affiliation{\Fermi}
\author{P.~Baldi} \affiliation{\CalIrvine}
\author{W.~Baldini} \affiliation{\INFNFerrara}
\author{B.~Baller} \affiliation{\Fermi}
\author{B.~Bambah} \affiliation{\Hyderabad}
\author{R.~Banerjee} \affiliation{\York}
\author{F.~Barao} \affiliation{\LIP}\affiliation{\ISTlisboa}
\author{G.~Barenboim} \affiliation{\IFIC}
\author{P.\ Barham~Alz\'as} \affiliation{\CERN}
\author{G.~J.~Barker} \affiliation{\Warwick}
\author{W.~Barkhouse} \affiliation{\Northdakota}
\author{C.~Barnes} \affiliation{\Michigan}
\author{G.~Barr} \affiliation{\Oxford}
\author{J.~Barranco Monarca} \affiliation{\Guanajuato}
\author{A.~Barros} \affiliation{\Tecnologica }
\author{N.~Barros} \affiliation{\LIP}\affiliation{\FCULport}
\author{J.~L.~Barrow} \affiliation{\Massinsttech}
\author{A.~Basharina-Freshville} \affiliation{\UniversityCollegeLondon}
\author{A.~Bashyal} \affiliation{\Argonne}
\author{V.~Basque} \affiliation{\Fermi}
\author{C.~Batchelor} \affiliation{\Edinburgh}
\author{J.B.R.~Battat} \affiliation{\Wellesley}
\author{F.~Battisti} \affiliation{\Oxford}
\author{F.~Bay} \affiliation{\Antalya}
\author{M.~C.~Q.~Bazetto} \affiliation{\Campinas}
\author{J.~L.~L.~Bazo Alba} \affiliation{\Pontificia}
\author{J.~F.~Beacom} \affiliation{\Ohiostate}
\author{E.~Bechetoille} \affiliation{\IPLyon}
\author{B.~Behera} \affiliation{\Florida}
\author{E.~Belchior} \affiliation{\Louisanastate}
\author{G.~Bell} \affiliation{\Daresbury}
\author{L.~Bellantoni} \affiliation{\Fermi}
\author{G.~Bellettini} \affiliation{\INFNPisa}\affiliation{\Pisa}
\author{V.~Bellini} \affiliation{\INFNCatania}\affiliation{\CataniaUniversitadi}
\author{O.~Beltramello} \affiliation{\CERN}
\author{N.~Benekos} \affiliation{\CERN}
\author{C.~Benitez Montiel} \affiliation{\IFIC}\affiliation{\Asuncion}
\author{D.~Benjamin} \affiliation{\Brookhaven}
\author{F.~Bento Neves} \affiliation{\LIP}
\author{J.~Berger} \affiliation{\ColoradoState}
\author{S.~Berkman} \affiliation{\Fermi}
\author{P.~Bernardini} \affiliation{\INFNLecce}\affiliation{\Salento}
\author{R.~M.~Berner} \affiliation{\Bern}
\author{A.~Bersani} \affiliation{\INFNGenova}
\author{S.~Bertolucci} \affiliation{\INFNBologna}\affiliation{\BolognaUniversity}
\author{M.~Betancourt} \affiliation{\Fermi}
\author{A.~Betancur Rodr\'iguez} \affiliation{\EIA}
\author{A.~Bevan} \affiliation{\QMUL}
\author{Y.~Bezawada} \affiliation{\CalDavis}
\author{A.~T.~Bezerra} \affiliation{\FederaldeAlfenas}
\author{T.~J.~Bezerra} \affiliation{\Sussex}
\author{J.~Bhambure} \affiliation{\StonyBrook}
\author{A.~Bhardwaj} \affiliation{\Louisanastate}
\author{V.~Bhatnagar} \affiliation{\Panjab}
\author{M.~Bhattacharjee} \affiliation{\IndGuwahati}
\author{M.~Bhattacharya} \affiliation{\Fermi}
\author{D.~Bhattarai} \affiliation{\Mississippi}
\author{S.~Bhuller} \affiliation{\Bristol}
\author{B.~Bhuyan} \affiliation{\IndGuwahati}
\author{S.~Biagi} \affiliation{\INFNSud}
\author{J.~Bian} \affiliation{\CalIrvine}
\author{K.~Biery} \affiliation{\Fermi}
\author{B.~Bilki} \affiliation{\Beykent}\affiliation{\Iowa}
\author{M.~Bishai} \affiliation{\Brookhaven}
\author{A.~Bitadze} \affiliation{\Manchester}
\author{A.~Blake} \affiliation{\Lancaster}
\author{F.~D.~Blaszczyk} \affiliation{\Fermi}
\author{G.~C.~Blazey} \affiliation{\Northernillinois}
\author{D.~Blend} \affiliation{\Iowa}
\author{E.~Blucher} \affiliation{\Chicago}
\author{J.~Boissevain} \affiliation{\LosAlmos}
\author{S.~Bolognesi} \affiliation{\CEASaclay}
\author{T.~Bolton} \affiliation{\Kansasstate}
\author{L.~Bomben} \affiliation{\INFNMilanBicocca}\affiliation{\Insubria }
\author{M.~Bonesini} \affiliation{\INFNMilanBicocca}\affiliation{\MilanoBicocca}
\author{C.~Bonilla-Diaz} \affiliation{\Catolica}
\author{F.~Bonini} \affiliation{\Brookhaven}
\author{A.~Booth} \affiliation{\QMUL}
\author{F.~Boran} \affiliation{\Beykent}
\author{S.~Bordoni} \affiliation{\CERN}
\author{A.~Borkum} \affiliation{\Sussex}
\author{N.~Bostan} \affiliation{\Iowa}
\author{P.~Bour} \affiliation{\CzechTechnical}
\author{J.~Bracinik} \affiliation{\Birmingham}
\author{D.~Braga} \affiliation{\Fermi}
\author{D.~Brailsford} \affiliation{\Lancaster}
\author{A.~Branca} \affiliation{\INFNMilanBicocca}
\author{A.~Brandt} \affiliation{\TexasArlington}
\author{M.~Bravo-Moreno} \affiliation{\Granada}
\author{J.~Bremer} \affiliation{\CERN}
\author{C.~Brew} \affiliation{\Rutherford}
\author{S.~J.~Brice} \affiliation{\Fermi}
\author{V.~Brio} \affiliation{\INFNCatania}
\author{C.~Brizzolari} \affiliation{\INFNMilanBicocca}\affiliation{\MilanoBicocca}
\author{C.~Bromberg} \affiliation{\Michiganstate}
\author{J.~Brooke} \affiliation{\Bristol}
\author{A.~Bross} \affiliation{\Fermi}
\author{G.~Brunetti} \affiliation{\INFNMilanBicocca}\affiliation{\MilanoBicocca}
\author{M.~Brunetti} \affiliation{\Warwick}
\author{N.~Buchanan} \affiliation{\ColoradoState}
\author{H.~Budd} \affiliation{\Rochester}
\author{J.~Buergi} \affiliation{\Bern}
\author{G.~Caceres V.} \affiliation{\CalDavis}
\author{I.~Cagnoli} \affiliation{\INFNBologna}\affiliation{\BolognaUniversity}
\author{T.~Cai} \affiliation{\York}
\author{D.~Caiulo} \affiliation{\IPLyon}
\author{R.~Calabrese} \affiliation{\INFNFerrara}\affiliation{\Ferrarauniv}
\author{P.~Calafiura} \affiliation{\LawrenceBerkeley}
\author{J.~Calcutt} \affiliation{\OregonState}
\author{M.~Calin} \affiliation{\Bucharest}
\author{L.~Calivers} \affiliation{\Bern}
\author{S.~Calvez} \affiliation{\ColoradoState}
\author{E.~Calvo} \affiliation{\CIEMAT}
\author{A.~Caminata} \affiliation{\INFNGenova}
\author{A.~Campos Benitez} \affiliation{\VirginiaTech}
\author{D.~Caratelli} \affiliation{\CalSantabarbara}
\author{D.~Carber} \affiliation{\ColoradoState}
\author{J.~M.~Carceller} \affiliation{\UniversityCollegeLondon}
\author{G.~Carini} \affiliation{\Brookhaven}
\author{B.~Carlus} \affiliation{\IPLyon}
\author{M.~F.~Carneiro} \affiliation{\Brookhaven}
\author{P.~Carniti} \affiliation{\INFNMilanBicocca}
\author{I.~Caro Terrazas} \affiliation{\ColoradoState}
\author{H.~Carranza} \affiliation{\TexasArlington}
\author{N.~Carrara} \affiliation{\CalDavis}
\author{L.~Carroll} \affiliation{\Kansasstate}
\author{T.~Carroll} \affiliation{\Wisconsin}
\author{A.~Carter} \affiliation{\Royalholloway}
\author{J.~F.~Casta{\~n}o Forero} \affiliation{\AntonioNarino}
\author{A.~Castillo} \affiliation{\SergioArboleda}
\author{C.~Castromonte} \affiliation{\Ingenieria}
\author{E.~Catano-Mur} \affiliation{\WilliamMary}
\author{C.~Cattadori} \affiliation{\INFNMilanBicocca}
\author{F.~Cavalier} \affiliation{\Parissaclay}
\author{G.~Cavallaro} \affiliation{\INFNMilanBicocca}
\author{F.~Cavanna} \affiliation{\Fermi}
\author{S.~Centro} \affiliation{\Padova}
\author{G.~Cerati} \affiliation{\Fermi}
\author{A.~Cervelli} \affiliation{\INFNBologna}
\author{A.~Cervera Villanueva} \affiliation{\IFIC}
\author{K.~Chakraborty} \affiliation{\PhysicalResearchLaboratory}
\author{M.~Chalifour} \affiliation{\CERN}
\author{A.~Chappell} \affiliation{\Warwick}
\author{E.~Chardonnet} \affiliation{\Parisuniversite}
\author{N.~Charitonidis} \affiliation{\CERN}
\author{A.~Chatterjee} \affiliation{\Pitt}
\author{S.~Chattopadhyay} \affiliation{\VariableEnergy}
\author{H.~Chen} \affiliation{\Brookhaven}
\author{M.~Chen} \affiliation{\CalIrvine}
\author{Y.~Chen} \affiliation{\Bern}\affiliation{\SLAC}
\author{Z.~Chen-Wishart} \affiliation{\Royalholloway}
\author{Y.~Cheon} \affiliation{\UNIST}
\author{D.~Cherdack} \affiliation{\Houston}
\author{C.~Chi} \affiliation{\Columbia}
\author{S.~Childress} \affiliation{\Fermi}
\author{R.~Chirco} \affiliation{\Illinoisinstitute}
\author{A.~Chiriacescu} \affiliation{\Bucharest}
\author{N.~Chitirasreemadam} \affiliation{\INFNPisa}\affiliation{\Pisa}
\author{K.~Cho} \affiliation{\KISTI}
\author{S.~Choate} \affiliation{\Northernillinois}
\author{D.~Chokheli} \affiliation{\Georgian}
\author{P.~S.~Chong} \affiliation{\Penn}
\author{B.~Chowdhury} \affiliation{\Argonne}
\author{A.~Christensen} \affiliation{\ColoradoState}
\author{D.~Christian} \affiliation{\Fermi}
\author{G.~Christodoulou} \affiliation{\CERN}
\author{A.~Chukanov} \affiliation{\JINR}
\author{M.~Chung} \affiliation{\UNIST}
\author{E.~Church} \affiliation{\PacificNorthwest}
\author{V.~Cicero} \affiliation{\INFNBologna}\affiliation{\BolognaUniversity}
\author{D.~Clapa} \affiliation{\Warsaw}
\author{P.~Clarke} \affiliation{\Edinburgh}
\author{G.~Cline} \affiliation{\LawrenceBerkeley}
\author{T.~E.~Coan} \affiliation{\SouthernMethodist}
\author{A.~G.~Cocco} \affiliation{\INFNNapoli}
\author{J.~A.~B.~Coelho} \affiliation{\Parisuniversite}
\author{A.~Cohen} \affiliation{\Parisuniversite}
\author{J.~Collot} \affiliation{\Grenoble}
\author{E.~Conley} \affiliation{\Duke}
\author{J.~M.~Conrad} \affiliation{\Massinsttech}
\author{M.~Convery} \affiliation{\SLAC}
\author{P.~Cooke} \affiliation{\Liverpool}
\author{S.~Copello} \affiliation{\INFNGenova}
\author{P.~Cova} \affiliation{\INFNMilano}\affiliation{\Parma}
\author{C.~Cox} \affiliation{\Royalholloway}
\author{L.~Cremaldi} \affiliation{\Mississippi}
\author{L.~Cremonesi} \affiliation{\QMUL}
\author{J.~I.~Crespo-Anad\'on} \affiliation{\CIEMAT}
\author{M.~Crisler} \affiliation{\Fermi}
\author{E.~Cristaldo} \affiliation{\INFNMilano}\affiliation{\Asuncion}
\author{J.~Crnkovic} \affiliation{\Fermi}
\author{G.~Crone} \affiliation{\UniversityCollegeLondon}
\author{R.~Cross} \affiliation{\Lancaster}
\author{A.~Cudd} \affiliation{\ColoradoBoulder}
\author{C.~Cuesta} \affiliation{\CIEMAT}
\author{Y.~Cui} \affiliation{\CalRiverside}
\author{D.~Cussans} \affiliation{\Bristol}
\author{J.~Dai} \affiliation{\Grenoble}
\author{O.~Dalager} \affiliation{\CalIrvine}
\author{R.~Dallavalle} \affiliation{\Parisuniversite}
\author{H.~da Motta} \affiliation{\CBPF}
\author{Z.~A.~Dar} \affiliation{\WilliamMary}
\author{R.~Darby} \affiliation{\Sussex}
\author{L.~Da Silva Peres} \affiliation{\FederaldoRio}
\author{C.~David} \affiliation{\York}\affiliation{\Fermi}
\author{Q.~David} \affiliation{\IPLyon}
\author{G.~S.~Davies} \affiliation{\Mississippi}
\author{S.~Davini} \affiliation{\INFNGenova}
\author{J.~Dawson} \affiliation{\Parisuniversite}
\author{K.~De} \affiliation{\TexasArlington}
\author{S.~De} \affiliation{\Albanysuny}
\author{R.~De Aguiar} \affiliation{\Campinas}
\author{P.~De Almeida} \affiliation{\Campinas}
\author{P.~Debbins} \affiliation{\Iowa}
\author{I.~De Bonis} \affiliation{\DannecyleVieux}
\author{M.~P.~Decowski} \affiliation{\Nikhef}\affiliation{\Amsterdam}
\author{A.~de Gouv\^ea} \affiliation{\Northwestern}
\author{P.~C.~De Holanda} \affiliation{\Campinas}
\author{I.~L.~De Icaza Astiz} \affiliation{\Sussex}
\author{A.~Deisting} \affiliation{\Mainz}
\author{P.~De Jong} \affiliation{\Nikhef}\affiliation{\Amsterdam}
\author{A.~De la Torre} \affiliation{\CIEMAT}
\author{A.~Delbart} \affiliation{\CEASaclay}
\author{V.~De Leo} \affiliation{\Sapienza}\affiliation{\INFNRoma}
\author{D.~Delepine} \affiliation{\Guanajuato}
\author{M.~Delgado} \affiliation{\INFNMilanBicocca}\affiliation{\MilanoBicocca}
\author{A.~Dell'Acqua} \affiliation{\CERN}
\author{N.~Delmonte} \affiliation{\INFNMilano}\affiliation{\Parma}
\author{P.~De Lurgio} \affiliation{\Argonne}
\author{J.~R.~T.~de Mello Neto} \affiliation{\FederaldoRio}
\author{D.~M.~DeMuth} \affiliation{\ValleyCity}
\author{S.~Dennis} \affiliation{\Cambridge}
\author{C.~Densham} \affiliation{\Rutherford}
\author{P.~Denton} \affiliation{\Brookhaven}
\author{G.~W.~Deptuch} \affiliation{\Brookhaven}
\author{A.~De Roeck} \affiliation{\CERN}
\author{V.~De Romeri} \affiliation{\IFIC}
\author{G.~De Souza} \affiliation{\Campinas}
\author{J.~P.~Detje} \affiliation{\Cambridge}
\author{R.~Devi} \affiliation{\Jammu}
\author{J.~Devine} \affiliation{\CERN}
\author{R.~Dharmapalan} \affiliation{\Hawaii}
\author{M.~Dias} \affiliation{\Unifesp}
\author{J.~S.~D\'iaz} \affiliation{\Indiana}
\author{F.~D{\'\i}az} \affiliation{\Pontificia}
\author{F.~Di Capua} \affiliation{\INFNNapoli}\affiliation{\napoli}
\author{A.~Di Domenico} \affiliation{\Sapienza}\affiliation{\INFNRoma}
\author{S.~Di Domizio} \affiliation{\INFNGenova}\affiliation{\Genova}
\author{S.~Di Falco} \affiliation{\INFNPisa}
\author{L.~Di Giulio} \affiliation{\CERN}
\author{P.~Ding} \affiliation{\Fermi}
\author{L.~Di Noto} \affiliation{\INFNGenova}\affiliation{\Genova}
\author{E.~Diociaiuti} \affiliation{\INFNFrascati}
\author{C.~Distefano} \affiliation{\INFNSud}
\author{R.~Diurba} \affiliation{\Bern}
\author{M.~Diwan} \affiliation{\Brookhaven}
\author{Z.~Djurcic} \affiliation{\Argonne}
\author{D.~Doering} \affiliation{\SLAC}
\author{S.~Dolan} \affiliation{\CERN}
\author{F.~Dolek} \affiliation{\Beykent}
\author{M.~J.~Dolinski} \affiliation{\Drexel}
\author{D.~Domenici} \affiliation{\INFNFrascati}
\author{L.~Domine} \affiliation{\SLAC}
\author{S.~Donati} \affiliation{\INFNPisa}\affiliation{\Pisa}
\author{Y.~Donon} \affiliation{\CERN}
\author{S.~Doran} \affiliation{\IowaState}
\author{D.~Douglas} \affiliation{\Michiganstate}
\author{A.~Dragone} \affiliation{\SLAC}
\author{F.~Drielsma} \affiliation{\SLAC}
\author{L.~Duarte} \affiliation{\Unifesp}
\author{D.~Duchesneau} \affiliation{\DannecyleVieux}
\author{K.~Duffy} \affiliation{\Oxford}\affiliation{\Fermi}
\author{K.~Dugas} \affiliation{\CalIrvine}
\author{P.~Dunne} \affiliation{\Imperial}
\author{B.~Dutta} \affiliation{\TexasAMcollege}
\author{H.~Duyang} \affiliation{\Southcarolina}
\author{O.~Dvornikov} \affiliation{\Hawaii}
\author{D.~A.~Dwyer} \affiliation{\LawrenceBerkeley}
\author{A.~S.~Dyshkant} \affiliation{\Northernillinois}
\author{M.~Eads} \affiliation{\Northernillinois}
\author{A.~Earle} \affiliation{\Sussex}
\author{S.~Edayath} \affiliation{\IowaState}
\author{D.~Edmunds} \affiliation{\Michiganstate}
\author{J.~Eisch} \affiliation{\Fermi}
\author{L.~Emberger} \affiliation{\Manchester}\affiliation{\Maxplanck}
\author{P.~Englezos} \affiliation{\Rutgers}
\author{A.~Ereditato} \affiliation{\Yale}
\author{T.~Erjavec} \affiliation{\CalDavis}
\author{C.~O.~Escobar} \affiliation{\Fermi}
\author{J.~J.~Evans} \affiliation{\Manchester}
\author{E.~Ewart} \affiliation{\Indiana}
\author{A.~C.~Ezeribe} \affiliation{\Sheffield}
\author{K.~Fahey} \affiliation{\Fermi}
\author{L.~Fajt} \affiliation{\CERN}
\author{A.~Falcone} \affiliation{\INFNMilanBicocca}\affiliation{\MilanoBicocca}
\author{M.~Fani'} \affiliation{\LosAlmos}
\author{C.~Farnese} \affiliation{\INFNPadova}
\author{Y.~Farzan} \affiliation{\IPM}
\author{D.~Fedoseev} \affiliation{\JINR}
\author{J.~Felix} \affiliation{\Guanajuato}
\author{Y.~Feng} \affiliation{\IowaState}
\author{E.~Fernandez-Martinez} \affiliation{\Madrid}
\author{F.~Ferraro} \affiliation{\INFNGenova}\affiliation{\Genova}
\author{G.~Ferry} \affiliation{\Parissaclay}
\author{L.~Fields} \affiliation{\NotreDame}
\author{P.~Filip} \affiliation{\CzechAcademyofSciences}
\author{A.~Filkins} \affiliation{\Syracuse}
\author{F.~Filthaut} \affiliation{\Nikhef}\affiliation{\Radboud}
\author{R.~Fine} \affiliation{\LosAlmos}
\author{G.~Fiorillo} \affiliation{\INFNNapoli}\affiliation{\napoli}
\author{M.~Fiorini} \affiliation{\INFNFerrara}\affiliation{\Ferrarauniv}
\author{V.~Fischer} \affiliation{\IowaState}
\author{R.~S.~Fitzpatrick} \affiliation{\Michigan}
\author{W.~Flanagan} \affiliation{\Dallas}
\author{B.~Fleming} \affiliation{\Chicago}\affiliation{\Yale}
\author{S.~Fogarty} \affiliation{\ColoradoState}
\author{W.~Foreman} \affiliation{\Illinoisinstitute}
\author{J.~Fowler} \affiliation{\Duke}
\author{J.~Franc} \affiliation{\CzechTechnical}
\author{K.~Francis} \affiliation{\Northernillinois}
\author{D.~Franco} \affiliation{\Yale}
\author{J.~Freeman} \affiliation{\Fermi}
\author{J.~Fried} \affiliation{\Brookhaven}
\author{A.~Friedland} \affiliation{\SLAC}
\author{S.~Fuess} \affiliation{\Fermi}
\author{I.~K.~Furic} \affiliation{\Florida}
\author{K.~Furman} \affiliation{\QMUL}
\author{A.~P.~Furmanski} \affiliation{\Minntwin}
\author{A.~Gabrielli} \affiliation{\INFNBologna}\affiliation{\BolognaUniversity}
\author{A.~Gago} \affiliation{\Pontificia}
\author{H.~Gallagher} \affiliation{\Tufts}
\author{A.~Gallas} \affiliation{\Parissaclay}
\author{N.~Gallice} \affiliation{\INFNMilano}\affiliation{\MilanoUniv}
\author{V.~Galymov} \affiliation{\IPLyon}
\author{E.~Gamberini} \affiliation{\CERN}
\author{T.~Gamble} \affiliation{\Sheffield}
\author{F.~Ganacim} \affiliation{\Tecnologica }
\author{R.~Gandhi} \affiliation{\Harish}
\author{S.~Ganguly} \affiliation{\Fermi}
\author{F.~Gao} \affiliation{\Pitt}
\author{S.~Gao} \affiliation{\Brookhaven}
\author{D.~Garcia-Gamez} \affiliation{\Granada}
\author{M.~\'A.~Garc\'ia-Peris} \affiliation{\IFIC}
\author{S.~Gardiner} \affiliation{\Fermi}
\author{D.~Gastler} \affiliation{\Boston}
\author{A.~Gauch} \affiliation{\Bern}
\author{J.~Gauvreau} \affiliation{\Occidental}
\author{P.~Gauzzi} \affiliation{\Sapienza}\affiliation{\INFNRoma}
\author{G.~Ge} \affiliation{\Columbia}
\author{N.~Geffroy} \affiliation{\DannecyleVieux}
\author{B.~Gelli} \affiliation{\Campinas}
\author{S.~Gent} \affiliation{\SouthDakotaState}
\author{L.~Gerlach} \affiliation{\Brookhaven}
\author{Z.~Ghorbani-Moghaddam} \affiliation{\INFNGenova}
\author{P.~Giammaria} \affiliation{\Campinas}
\author{T.~Giammaria} \affiliation{\INFNFerrara}\affiliation{\Ferrarauniv}
\author{N.~Giangiacomi} \affiliation{\Toronto}
\author{D.~Gibin} \affiliation{\Padova}\affiliation{\INFNPadova}
\author{I.~Gil-Botella} \affiliation{\CIEMAT}
\author{S.~Gilligan} \affiliation{\OregonState}
\author{A.~Gioiosa} \affiliation{\INFNPisa}
\author{S.~Giovannella} \affiliation{\INFNFrascati}
\author{C.~Girerd} \affiliation{\IPLyon}
\author{A.~K.~Giri} \affiliation{\IndHyderabad}
\author{C.~Giugliano} \affiliation{\INFNFerrara}
\author{D.~Gnani} \affiliation{\LawrenceBerkeley}
\author{O.~Gogota} \affiliation{\Kyiv}
\author{S.~Gollapinni} \affiliation{\LosAlmos}
\author{K.~Gollwitzer} \affiliation{\Fermi}
\author{R.~A.~Gomes} \affiliation{\FederaldeGoias}
\author{L.~V.~Gomez Bermeo} \affiliation{\SergioArboleda}
\author{L.~S.~Gomez Fajardo} \affiliation{\SergioArboleda}
\author{F.~Gonnella} \affiliation{\Birmingham}
\author{D.~Gonzalez-Diaz} \affiliation{\IGFAE}
\author{M.~Gonzalez-Lopez} \affiliation{\Madrid}
\author{M.~C.~Goodman} \affiliation{\Argonne}
\author{O.~Goodwin} \affiliation{\Manchester}
\author{S.~Goswami} \affiliation{\PhysicalResearchLaboratory}
\author{C.~Gotti} \affiliation{\INFNMilanBicocca}
\author{J.~Goudeau} \affiliation{\Louisanastate}
\author{E.~Goudzovski} \affiliation{\Birmingham}
\author{C.~Grace} \affiliation{\LawrenceBerkeley}
\author{R.~Gran} \affiliation{\Minnduluth}
\author{E.~Granados} \affiliation{\Guanajuato}
\author{P.~Granger} \affiliation{\Parisuniversite}
\author{C.~Grant} \affiliation{\Boston}
\author{D.~Gratieri} \affiliation{\Fluminense}
\author{P.~Green} \affiliation{\Oxford}
\author{S.~Greenberg} \affiliation{\CalBerkeley}\affiliation{\LawrenceBerkeley}
\author{L.~Greenler} \affiliation{\Wisconsin}
\author{J.~Greer} \affiliation{\Bristol}
\author{J.~Grenard} \affiliation{\CERN}
\author{W.~C.~Griffith} \affiliation{\Sussex}
\author{F.~T.~Groetschla} \affiliation{\CERN}
\author{M.~Groh} \affiliation{\ColoradoState}
\author{K.~Grzelak} \affiliation{\Warsaw}
\author{W.~Gu} \affiliation{\Brookhaven}
\author{V.~Guarino} \affiliation{\Argonne}
\author{M.~Guarise} \affiliation{\INFNFerrara}\affiliation{\Ferrarauniv}
\author{R.~Guenette} \affiliation{\Manchester}
\author{E.~Guerard} \affiliation{\Parissaclay}
\author{M.~Guerzoni} \affiliation{\INFNBologna}
\author{D.~Guffanti} \affiliation{\INFNMilanBicocca}
\author{A.~Guglielmi} \affiliation{\INFNPadova}
\author{B.~Guo} \affiliation{\Southcarolina}
\author{Y.~Guo} \affiliation{\StonyBrook}
\author{A.~Gupta} \affiliation{\SLAC}
\author{V.~Gupta} \affiliation{\Nikhef}\affiliation{\Amsterdam}
\author{K.~K.~Guthikonda} \affiliation{\KL}
\author{D.~Gutierrez} \affiliation{\PuertoRico}
\author{P.~Guzowski} \affiliation{\Manchester}
\author{M.~M.~Guzzo} \affiliation{\Campinas}
\author{S.~Gwon} \affiliation{\ChungAng}
\author{C.~Ha} \affiliation{\ChungAng}
\author{K.~Haaf} \affiliation{\Fermi}
\author{A.~Habig} \affiliation{\Minnduluth}
\author{H.~Hadavand} \affiliation{\TexasArlington}
\author{R.~Haenni} \affiliation{\Bern}
\author{L.~Hagaman} \affiliation{\Yale}
\author{A.~Hahn} \affiliation{\Fermi}
\author{J.~Haiston} \affiliation{\SouthDakotaSchool}
\author{P.~Hamacher-Baumann} \affiliation{\Oxford}
\author{T.~Hamernik} \affiliation{\Fermi}
\author{P.~Hamilton} \affiliation{\Imperial}
\author{J.~Han} \affiliation{\Pitt}
\author{J.~Hancock} \affiliation{\Birmingham}
\author{F.~Happacher} \affiliation{\INFNFrascati}
\author{D.~A.~Harris} \affiliation{\York}\affiliation{\Fermi}
\author{J.~Hartnell} \affiliation{\Sussex}
\author{T.~Hartnett} \affiliation{\Rutherford}
\author{J.~Harton} \affiliation{\ColoradoState}
\author{T.~Hasegawa} \affiliation{\KEK}
\author{C.~Hasnip} \affiliation{\Oxford}
\author{R.~Hatcher} \affiliation{\Fermi}
\author{K.~W.~Hatfield} \affiliation{\CalIrvine}
\author{A.~Hatzikoutelis} \affiliation{\Sanjosestate}
\author{C.~Hayes} \affiliation{\Indiana}
\author{K.~Hayrapetyan} \affiliation{\QMUL}
\author{J.~Hays} \affiliation{\QMUL}
\author{E.~Hazen} \affiliation{\Boston}
\author{M.~He} \affiliation{\Houston}
\author{A.~Heavey} \affiliation{\Fermi}
\author{K.~M.~Heeger} \affiliation{\Yale}
\author{J.~Heise} \affiliation{\SURF}
\author{S.~Henry} \affiliation{\Rochester}
\author{M.~A.~Hernandez Morquecho} \affiliation{\Illinoisinstitute}
\author{K.~Herner} \affiliation{\Fermi}
\author{V.~Hewes} \affiliation{\Cincinnati}
\author{A.~Higuera} \affiliation{\Rice}
\author{C.~Hilgenberg} \affiliation{\Minntwin}
\author{T.~Hill} \affiliation{\Idaho}
\author{S.~J.~Hillier} \affiliation{\Birmingham}
\author{A.~Himmel} \affiliation{\Fermi}
\author{E.~Hinkle} \affiliation{\Chicago}
\author{L.R.~Hirsch} \affiliation{\Tecnologica }
\author{J.~Ho} \affiliation{\Dordt}
\author{J.~Hoff} \affiliation{\Fermi}
\author{A.~Holin} \affiliation{\Rutherford}
\author{T.~Holvey} \affiliation{\Oxford}
\author{E.~Hoppe} \affiliation{\PacificNorthwest}
\author{G.~A.~Horton-Smith} \affiliation{\Kansasstate}
\author{M.~Hostert} \affiliation{\Minntwin}
\author{T.~Houdy} \affiliation{\Parissaclay}
\author{B.~Howard} \affiliation{\Fermi}
\author{R.~Howell} \affiliation{\Rochester}
\author{J.~Hoyos Barrios} \affiliation{\Medellin}
\author{I.~Hristova} \affiliation{\Rutherford}
\author{M.~S.~Hronek} \affiliation{\Fermi}
\author{J.~Huang} \affiliation{\CalDavis}
\author{R.G.~Huang} \affiliation{\LawrenceBerkeley}
\author{Z.~Hulcher} \affiliation{\SLAC}
\author{G.~Iles} \affiliation{\Imperial}
\author{N.~Ilic} \affiliation{\Toronto}
\author{A.~M.~Iliescu} \affiliation{\INFNBologna}
\author{R.~Illingworth} \affiliation{\Fermi}
\author{G.~Ingratta} \affiliation{\INFNBologna}\affiliation{\BolognaUniversity}
\author{A.~Ioannisian} \affiliation{\Yerevan}
\author{B.~Irwin} \affiliation{\Minntwin}
\author{L.~Isenhower} \affiliation{\Abilene}
\author{M.~Ismerio Oliveira} \affiliation{\FederaldoRio}
\author{R.~Itay} \affiliation{\SLAC}
\author{C.M.~Jackson} \affiliation{\PacificNorthwest}
\author{V.~Jain} \affiliation{\Albanysuny}
\author{E.~James} \affiliation{\Fermi}
\author{W.~Jang} \affiliation{\TexasArlington}
\author{B.~Jargowsky} \affiliation{\CalIrvine}
\author{F.~Jediny} \affiliation{\CzechTechnical}
\author{D.~Jena} \affiliation{\Fermi}
\author{Y.~S.~Jeong} \affiliation{\ChungAng}
\author{C.~Jes\'{u}s-Valls} \affiliation{\IFAE}
\author{X.~Ji} \affiliation{\Brookhaven}
\author{J.~Jiang} \affiliation{\StonyBrook}
\author{L.~Jiang} \affiliation{\VirginiaTech}
\author{A.~Jipa} \affiliation{\Bucharest}
\author{J.~H.~Jo} \affiliation{\Brookhaven}
\author{F.~R.~Joaquim} \affiliation{\LIP}\affiliation{\ISTlisboa}
\author{W.~Johnson} \affiliation{\SouthDakotaSchool}
\author{B.~Jones} \affiliation{\TexasArlington}
\author{R.~Jones} \affiliation{\Sheffield}
\author{N.~Jovancevic} \affiliation{\NoviSad}
\author{M.~Judah} \affiliation{\Pitt}
\author{C.~K.~Jung} \affiliation{\StonyBrook}
\author{T.~Junk} \affiliation{\Fermi}
\author{Y.~Jwa} \affiliation{\Columbia}
\author{M.~Kabirnezhad} \affiliation{\Imperial}
\author{A.~Kaboth} \affiliation{\Royalholloway}\affiliation{\Rutherford}
\author{I.~Kadenko} \affiliation{\Kyiv}
\author{I.~Kakorin} \affiliation{\JINR}
\author{A.~Kalitkina} \affiliation{\JINR}
\author{D.~Kalra} \affiliation{\Columbia}
\author{O.~Kamer Koseyan} \affiliation{\Iowa}
\author{F.~Kamiya} \affiliation{\FederaldoABC}
\author{D.~M.~Kaplan} \affiliation{\Illinoisinstitute}
\author{G.~Karagiorgi} \affiliation{\Columbia}
\author{G.~Karaman} \affiliation{\Iowa}
\author{A.~Karcher} \affiliation{\LawrenceBerkeley}
\author{Y.~Karyotakis} \affiliation{\DannecyleVieux}
\author{S.~Kasai} \affiliation{\Kure}
\author{S.~P.~Kasetti} \affiliation{\Louisanastate}
\author{L.~Kashur} \affiliation{\ColoradoState}
\author{I.~Katsioulas} \affiliation{\Birmingham}
\author{A.~Kauther} \affiliation{\Northernillinois}
\author{N.~Kazaryan} \affiliation{\Yerevan}
\author{E.~Kearns} \affiliation{\Boston}
\author{P.T.~Keener} \affiliation{\Penn}
\author{K.J.~Kelly} \affiliation{\CERN}
\author{E.~Kemp} \affiliation{\Campinas}
\author{O.~Kemularia} \affiliation{\Georgian}
\author{Y.~Kermaidic} \affiliation{\Parissaclay}
\author{W.~Ketchum} \affiliation{\Fermi}
\author{S.~H.~Kettell} \affiliation{\Brookhaven}
\author{M.~Khabibullin} \affiliation{\INR}
\author{N.~Khan} \affiliation{\Imperial}
\author{A.~Khotjantsev} \affiliation{\INR}
\author{A.~Khvedelidze} \affiliation{\Georgian}
\author{D.~Kim} \affiliation{\TexasAMcollege}
\author{J.~Kim} \affiliation{\Rochester}
\author{B.~King} \affiliation{\Fermi}
\author{B.~Kirby} \affiliation{\Columbia}
\author{M.~Kirby} \affiliation{\Fermi}
\author{J.~Klein} \affiliation{\Penn}
\author{J.~Kleykamp} \affiliation{\Mississippi}
\author{A.~Klustova} \affiliation{\Imperial}
\author{T.~Kobilarcik} \affiliation{\Fermi}
\author{L.~Koch} \affiliation{\Mainz}
\author{K.~Koehler} \affiliation{\Wisconsin}
\author{L.~W.~Koerner} \affiliation{\Houston}
\author{D.~H.~Koh} \affiliation{\SLAC}
\author{S.~Kohn} \affiliation{\CalBerkeley}\affiliation{\LawrenceBerkeley}
\author{P.~P.~Koller} \affiliation{\Bern}
\author{L.~Kolupaeva} \affiliation{\JINR}
\author{D.~Korablev} \affiliation{\JINR}
\author{M.~Kordosky} \affiliation{\WilliamMary}
\author{T.~Kosc} \affiliation{\Grenoble}
\author{U.~Kose} \affiliation{\CERN}
\author{V.~A.~Kosteleck\'y} \affiliation{\Indiana}
\author{K.~Kothekar} \affiliation{\Bristol}
\author{I.~Kotler} \affiliation{\Drexel}
\author{V.~Kozhukalov} \affiliation{\JINR}
\author{R.~Kralik} \affiliation{\Sussex}
\author{L.~Kreczko} \affiliation{\Bristol}
\author{F.~Krennrich} \affiliation{\IowaState}
\author{I.~Kreslo} \affiliation{\Bern}
\author{W.~Kropp} \affiliation{\CalIrvine}
\author{T.~Kroupova} \affiliation{\Penn}
\author{S.~Kubota} \affiliation{\Manchester}
\author{M.~Kubu} \affiliation{\CERN}
\author{Y.~Kudenko} \affiliation{\INR}
\author{V.~A.~Kudryavtsev} \affiliation{\Sheffield}
\author{S.~Kuhlmann} \affiliation{\Argonne}
\author{S.~Kulagin} \affiliation{\INR}
\author{J.~Kumar} \affiliation{\Hawaii}
\author{P.~Kumar} \affiliation{\Sheffield}
\author{P.~Kunze} \affiliation{\DannecyleVieux}
\author{R.~Kuravi} \affiliation{\LawrenceBerkeley}
\author{N.~Kurita} \affiliation{\SLAC}
\author{C.~Kuruppu} \affiliation{\Southcarolina}
\author{V.~Kus} \affiliation{\CzechTechnical}
\author{T.~Kutter} \affiliation{\Louisanastate}
\author{J.~Kvasnicka} \affiliation{\CzechAcademyofSciences}
\author{D.~Kwak} \affiliation{\UNIST}
\author{T.~Labree} \affiliation{\Northernillinois}
\author{A.~Lambert} \affiliation{\LawrenceBerkeley}
\author{B.~J.~Land} \affiliation{\Penn}
\author{C.~E.~Lane} \affiliation{\Drexel}
\author{K.~Lang} \affiliation{\Texasaustin}
\author{T.~Langford} \affiliation{\Yale}
\author{M.~Langstaff} \affiliation{\Manchester}
\author{F.~Lanni} \affiliation{\CERN}
\author{O.~Lantwin} \affiliation{\DannecyleVieux}
\author{J.~Larkin} \affiliation{\Brookhaven}
\author{P.~Lasorak} \affiliation{\Imperial}
\author{D.~Last} \affiliation{\Penn}
\author{A.~Laundrie} \affiliation{\Wisconsin}
\author{G.~Laurenti} \affiliation{\INFNBologna}
\author{A.~Lawrence} \affiliation{\LawrenceBerkeley}
\author{P.~Laycock} \affiliation{\Brookhaven}
\author{I.~Lazanu} \affiliation{\Bucharest}
\author{M.~Lazzaroni} \affiliation{\INFNMilano}\affiliation{\MilanoUniv}
\author{T.~Le} \affiliation{\Tufts}
\author{S.~Leardini} \affiliation{\IGFAE}
\author{J.~Learned} \affiliation{\Hawaii}
\author{P.~LeBrun} \affiliation{\IPLyon}
\author{T.~LeCompte} \affiliation{\SLAC}
\author{C.~Lee} \affiliation{\Fermi}
\author{V.~Legin} \affiliation{\Kyiv}
\author{G.~Lehmann Miotto} \affiliation{\CERN}
\author{R.~Lehnert} \affiliation{\Indiana}
\author{M.~A.~Leigui de Oliveira} \affiliation{\FederaldoABC}
\author{M.~Leitner} \affiliation{\LawrenceBerkeley}
\author{L.~M.~Lepin} \affiliation{\Manchester}
\author{S.~W.~Li} \affiliation{\SLAC}
\author{Y.~Li} \affiliation{\Brookhaven}
\author{H.~Liao} \affiliation{\Kansasstate}
\author{C.~S.~Lin} \affiliation{\LawrenceBerkeley}
\author{S.~Lin} \affiliation{\Louisanastate}
\author{D.~Lindebaum} \affiliation{\Bristol}
\author{R.~A.~Lineros} \affiliation{\Catolica}
\author{J.~Ling} \affiliation{\Sunyatsen}
\author{A.~Lister} \affiliation{\Wisconsin}
\author{B.~R.~Littlejohn} \affiliation{\Illinoisinstitute}
\author{J.~Liu} \affiliation{\CalIrvine}
\author{Y.~Liu} \affiliation{\Chicago}
\author{S.~Lockwitz} \affiliation{\Fermi}
\author{T.~Loew} \affiliation{\LawrenceBerkeley}
\author{M.~Lokajicek} \affiliation{\CzechAcademyofSciences}
\author{I.~Lomidze} \affiliation{\Georgian}
\author{K.~Long} \affiliation{\Imperial}
\author{N.~L{\'o}pez March} \affiliation{\IFIC}
\author{T.~Lord} \affiliation{\Warwick}
\author{J.~M.~LoSecco} \affiliation{\NotreDame}
\author{W.~C.~Louis} \affiliation{\LosAlmos}
\author{X.-G.~Lu} \affiliation{\Warwick}
\author{K.B.~Luk} \affiliation{\CalBerkeley}\affiliation{\LawrenceBerkeley}
\author{B.~Lunday} \affiliation{\Penn}
\author{X.~Luo} \affiliation{\CalSantabarbara}
\author{E.~Luppi} \affiliation{\INFNFerrara}\affiliation{\Ferrarauniv}
\author{T.~Lux} \affiliation{\IFAE}
\author{J.~Maalmi} \affiliation{\Parissaclay}
\author{D.~MacFarlane} \affiliation{\SLAC}
\author{A.~A.~Machado} \affiliation{\Campinas}
\author{P.~Machado} \affiliation{\Fermi}
\author{C.~T.~Macias} \affiliation{\Indiana}
\author{J.~R.~Macier} \affiliation{\Fermi}
\author{M.~MacMahon} \affiliation{\UniversityCollegeLondon}
\author{A.~Maddalena} \affiliation{\GranSassoLab}
\author{A.~Madera} \affiliation{\CERN}
\author{P.~Madigan} \affiliation{\CalBerkeley}\affiliation{\LawrenceBerkeley}
\author{S.~Magill} \affiliation{\Argonne}
\author{C.~Magueur} \affiliation{\Parissaclay}
\author{K.~Mahn} \affiliation{\Michiganstate}
\author{A.~Maio} \affiliation{\LIP}\affiliation{\FCULport}
\author{A.~Major} \affiliation{\Duke}
\author{K.~Majumdar} \affiliation{\Liverpool}
\author{J.~A.~Maloney} \affiliation{\DakotaState}
\author{M.~Man} \affiliation{\Toronto}
\author{G.~Mandrioli} \affiliation{\INFNBologna}
\author{R.~C.~Mandujano} \affiliation{\CalIrvine}
\author{J.~Maneira} \affiliation{\LIP}\affiliation{\FCULport}
\author{L.~Manenti} \affiliation{\UniversityCollegeLondon}
\author{S.~Manly} \affiliation{\Rochester}
\author{A.~Mann} \affiliation{\Tufts}
\author{K.~Manolopoulos} \affiliation{\Rutherford}
\author{M.~Manrique Plata} \affiliation{\Indiana}
\author{S.~Manthey Corchado} \affiliation{\CIEMAT}
\author{V.~N.~Manyam} \affiliation{\Brookhaven}
\author{M.~Marchan} \affiliation{\Fermi}
\author{A.~Marchionni} \affiliation{\Fermi}
\author{W.~Marciano} \affiliation{\Brookhaven}
\author{D.~Marfatia} \affiliation{\Hawaii}
\author{C.~Mariani} \affiliation{\VirginiaTech}
\author{J.~Maricic} \affiliation{\Hawaii}
\author{F.~Marinho} \affiliation{\Ita}
\author{A.~D.~Marino} \affiliation{\ColoradoBoulder}
\author{T.~Markiewicz} \affiliation{\SLAC}
\author{D.~Marsden} \affiliation{\Manchester}
\author{M.~Marshak} \affiliation{\Minntwin}
\author{C.~M.~Marshall} \affiliation{\Rochester}
\author{J.~Marshall} \affiliation{\Warwick}
\author{J.~Marteau} \affiliation{\IPLyon}
\author{J.~Mart{\'\i}n-Albo} \affiliation{\IFIC}
\author{N.~Martinez} \affiliation{\Kansasstate}
\author{D.A.~Martinez Caicedo } \affiliation{\SouthDakotaSchool}
\author{F.~Mart{\'i}nez L{\'o}pez} \affiliation{\QMUL}
\author{P.~Mart\'inez Mirav\'e} \affiliation{\IFIC}
\author{S.~Martynenko} \affiliation{\Brookhaven}
\author{V.~Mascagna} \affiliation{\INFNMilanBicocca}\affiliation{\Insubria }
\author{K.~Mason} \affiliation{\Tufts}
\author{C.~Massari} \affiliation{\INFNMilanBicocca}
\author{A.~Mastbaum} \affiliation{\Rutgers}
\author{F.~Matichard} \affiliation{\LawrenceBerkeley}
\author{S.~Matsuno} \affiliation{\Hawaii}
\author{J.~Matthews} \affiliation{\Louisanastate}
\author{C.~Mauger} \affiliation{\Penn}
\author{N.~Mauri} \affiliation{\INFNBologna}\affiliation{\BolognaUniversity}
\author{K.~Mavrokoridis} \affiliation{\Liverpool}
\author{I.~Mawby} \affiliation{\Warwick}
\author{R.~Mazza} \affiliation{\INFNMilanBicocca}
\author{A.~Mazzacane} \affiliation{\Fermi}
\author{T.~McAskill} \affiliation{\Wellesley}
\author{E.~McCluskey} \affiliation{\Fermi}
\author{N.~McConkey} \affiliation{\UniversityCollegeLondon}
\author{K.~S.~McFarland} \affiliation{\Rochester}
\author{C.~McGrew} \affiliation{\StonyBrook}
\author{A.~McNab} \affiliation{\Manchester}
\author{A.~Mefodiev} \affiliation{\INR}
\author{P.~Mehta} \affiliation{\Jawaharlal}
\author{P.~Melas} \affiliation{\Athens}
\author{O.~Mena} \affiliation{\IFIC}
\author{H.~Mendez} \affiliation{\PuertoRico}
\author{P.~Mendez} \affiliation{\CERN}
\author{D.~P.~M{\'e}ndez} \affiliation{\Brookhaven}
\author{A.~Menegolli} \affiliation{\INFNPavia}\affiliation{\Pavia}
\author{G.~Meng} \affiliation{\INFNPadova}
\author{M.~D.~Messier} \affiliation{\Indiana}
\author{W.~Metcalf} \affiliation{\Louisanastate}
\author{M.~Mewes} \affiliation{\Indiana}
\author{H.~Meyer} \affiliation{\Wichita}
\author{T.~Miao} \affiliation{\Fermi}
\author{G.~Michna} \affiliation{\SouthDakotaState}
\author{V.~Mikola} \affiliation{\UniversityCollegeLondon}
\author{R.~Milincic} \affiliation{\Hawaii}
\author{G.~Miller} \affiliation{\Manchester}
\author{W.~Miller} \affiliation{\Minntwin}
\author{J.~Mills} \affiliation{\Tufts}
\author{O.~Mineev} \affiliation{\INR}
\author{A.~Minotti} \affiliation{\INFNMilanBicocca}\affiliation{\MilanoBicocca}
\author{O.~G.~Miranda} \affiliation{\Cinvestav}
\author{S.~Miryala} \affiliation{\Brookhaven}
\author{S.~Miscetti} \affiliation{\INFNFrascati}
\author{C.~S.~Mishra} \affiliation{\Fermi}
\author{S.~R.~Mishra} \affiliation{\Southcarolina}
\author{A.~Mislivec} \affiliation{\Minntwin}
\author{M.~Mitchell} \affiliation{\Louisanastate}
\author{D.~Mladenov} \affiliation{\CERN}
\author{I.~Mocioiu} \affiliation{\PennState}
\author{K.~Moffat} \affiliation{\Durham}
\author{A.~Mogan} \affiliation{\ColoradoState}
\author{N.~Moggi} \affiliation{\INFNBologna}\affiliation{\BolognaUniversity}
\author{R.~Mohanta} \affiliation{\Hyderabad}
\author{T.~A.~Mohayai} \affiliation{\Fermi}
\author{N.~Mokhov} \affiliation{\Fermi}
\author{J.~Molina} \affiliation{\Asuncion}
\author{L.~Molina Bueno} \affiliation{\IFIC}
\author{E.~Montagna} \affiliation{\INFNBologna}\affiliation{\BolognaUniversity}
\author{A.~Montanari} \affiliation{\INFNBologna}
\author{C.~Montanari} \affiliation{\INFNPavia}\affiliation{\Fermi}\affiliation{\Pavia}
\author{D.~Montanari} \affiliation{\Fermi}
\author{D.~Montanino} \affiliation{\INFNLecce}\affiliation{\Salento}
\author{L.~M.~Monta{\~n}o Zetina} \affiliation{\Cinvestav}
\author{S.~H.~Moon} \affiliation{\UNIST}
\author{M.~Mooney} \affiliation{\ColoradoState}
\author{A.~F.~Moor} \affiliation{\Cambridge}
\author{D.~Moreno} \affiliation{\AntonioNarino}
\author{L.~Morescalchi} \affiliation{\INFNPisa}
\author{D.~Moretti} \affiliation{\INFNMilanBicocca}
\author{C.~Morris} \affiliation{\Houston}
\author{C.~Mossey} \affiliation{\Fermi}
\author{M.~Mote} \affiliation{\Louisanastate}
\author{E.~Motuk} \affiliation{\UniversityCollegeLondon}
\author{C.~A.~Moura} \affiliation{\FederaldoABC}
\author{J.~Mousseau} \affiliation{\Michigan}
\author{G.~Mouster} \affiliation{\Lancaster}
\author{W.~Mu} \affiliation{\Fermi}
\author{L.~Mualem} \affiliation{\Caltech}
\author{J.~Mueller} \affiliation{\ColoradoState}
\author{M.~Muether} \affiliation{\Wichita}
\author{F.~Muheim} \affiliation{\Edinburgh}
\author{A.~Muir} \affiliation{\Daresbury}
\author{M.~Mulhearn} \affiliation{\CalDavis}
\author{D.~Munford} \affiliation{\Houston}
\author{L.~J.~Munteanu} \affiliation{\CERN}
\author{H.~Muramatsu} \affiliation{\Minntwin}
\author{J.~Muraz} \affiliation{\DannecyleVieux}
\author{M.~Murphy} \affiliation{\VirginiaTech}
\author{T.~Murphy} \affiliation{\Syracuse}
\author{J.~Musser} \affiliation{\Indiana}
\author{J.~Nachtman} \affiliation{\Iowa}
\author{Y.~Nagai} \affiliation{\Eotvos} %\affiliation{\Eötvös}
\author{S.~Nagu} \affiliation{\Lucknow}
\author{M.~Nalbandyan} \affiliation{\Yerevan}
\author{R.~Nandakumar} \affiliation{\Rutherford}
\author{D.~Naples} \affiliation{\Pitt}
\author{S.~Narita} \affiliation{\Iwate}
\author{A.~Nath} \affiliation{\IndGuwahati}
\author{A.~Navrer-Agasson} \affiliation{\Manchester}
\author{N.~Nayak} \affiliation{\Brookhaven}
\author{M.~Nebot-Guinot} \affiliation{\Edinburgh}
\author{K.~Negishi} \affiliation{\Iwate}
\author{A.~Nehm} \affiliation{\Mainz}
\author{J.~K.~Nelson} \affiliation{\WilliamMary}
\author{M.~Nelson} \affiliation{\Iowa}
\author{J.~Nesbit} \affiliation{\Wisconsin}
\author{M.~Nessi} \affiliation{\Fermi}\affiliation{\CERN}
\author{D.~Newbold} \affiliation{\Rutherford}
\author{M.~Newcomer} \affiliation{\Penn}
\author{H.~Newton} \affiliation{\Daresbury}
\author{R.~Nichol} \affiliation{\UniversityCollegeLondon}
\author{F.~Nicolas-Arnaldos} \affiliation{\Granada}
\author{A.~Nikolica} \affiliation{\Penn}
\author{J.~Nikolov} \affiliation{\NoviSad}
\author{E.~Niner} \affiliation{\Fermi}
\author{K.~Nishimura} \affiliation{\Hawaii}
\author{A.~Norman} \affiliation{\Fermi}
\author{A.~Norrick} \affiliation{\Fermi}
\author{P.~Novella} \affiliation{\IFIC}
\author{J.~A.~Nowak} \affiliation{\Lancaster}
\author{M.~Oberling} \affiliation{\Argonne}
\author{J.~P.~Ochoa-Ricoux} \affiliation{\CalIrvine}
\author{A.~Olivier} \affiliation{\NotreDame}
\author{A.~Olshevskiy} \affiliation{\JINR}
\author{T.~Olson} \affiliation{\Houston}
\author{Y.~Onel} \affiliation{\Iowa}
\author{Y.~Onishchuk} \affiliation{\Kyiv}
\author{A.~Oranday} \affiliation{\Indiana}
\author{L.~Otiniano Ormachea} \affiliation{\conida}\affiliation{\Ingenieria}
\author{J.~Ott} \affiliation{\CalIrvine}
\author{L.~Pagani} \affiliation{\CalDavis}
\author{G.~Palacio} \affiliation{\EIA}
\author{O.~Palamara} \affiliation{\Fermi}
\author{S.~Palestini} \affiliation{\CERN}
\author{J.~M.~Paley} \affiliation{\Fermi}
\author{M.~Pallavicini} \affiliation{\INFNGenova}\affiliation{\Genova}
\author{C.~Palomares} \affiliation{\CIEMAT}
\author{S.~Pan} \affiliation{\PhysicalResearchLaboratory}
\author{W.~Panduro Vazquez} \affiliation{\Royalholloway}
\author{E.~Pantic} \affiliation{\CalDavis}
\author{V.~Paolone} \affiliation{\Pitt}
\author{V.~Papadimitriou} \affiliation{\Fermi}
\author{R.~Papaleo} \affiliation{\INFNSud}
\author{A.~Papanestis} \affiliation{\Rutherford}
\author{S.~Paramesvaran} \affiliation{\Bristol}
\author{A.~Paris} \affiliation{\PuertoRico}
\author{S.~Parke} \affiliation{\Fermi}
\author{E.~Parozzi} \affiliation{\INFNMilanBicocca}\affiliation{\MilanoBicocca}
\author{S.~Parsa} \affiliation{\Bern}
\author{Z.~Parsa} \affiliation{\Brookhaven}
\author{S.~Parveen} \affiliation{\Jawaharlal}
\author{M.~Parvu} \affiliation{\Bucharest}
\author{D.~Pasciuto} \affiliation{\INFNPisa}
\author{S.~Pascoli} \affiliation{\Durham}\affiliation{\BolognaUniversity}
\author{L.~Pasqualini} \affiliation{\INFNBologna}\affiliation{\BolognaUniversity}
\author{J.~Pasternak} \affiliation{\Imperial}
\author{J.~Pater} \affiliation{\Manchester}
\author{C.~Patrick} \affiliation{\Edinburgh}\affiliation{\UniversityCollegeLondon}
\author{L.~Patrizii} \affiliation{\INFNBologna}
\author{R.~B.~Patterson} \affiliation{\Caltech}
\author{S.~J.~Patton} \affiliation{\LawrenceBerkeley}
\author{T.~Patzak} \affiliation{\Parisuniversite}
\author{A.~Paudel} \affiliation{\Fermi}
\author{L.~Paulucci} \affiliation{\FederaldoABC}
\author{Z.~Pavlovic} \affiliation{\Fermi}
\author{G.~Pawloski} \affiliation{\Minntwin}
\author{D.~Payne} \affiliation{\Liverpool}
\author{V.~Pec} \affiliation{\CzechAcademyofSciences}
\author{S.~J.~M.~Peeters} \affiliation{\Sussex}
\author{A.~Pena Perez} \affiliation{\SLAC}
\author{E.~Pennacchio} \affiliation{\IPLyon}
\author{A.~Penzo} \affiliation{\Iowa}
\author{O.~L.~G.~Peres} \affiliation{\Campinas}
\author{Y.~F.~Perez Gonzalez} \affiliation{\Durham}
\author{L.~P{\'e}rez-Molina} \affiliation{\CIEMAT}
\author{C.~Pernas} \affiliation{\WilliamMary}
\author{J.~Perry} \affiliation{\Edinburgh}
\author{D.~Pershey} \affiliation{\Duke}
\author{G.~Pessina} \affiliation{\INFNMilanBicocca}
\author{G.~Petrillo} \affiliation{\SLAC}
\author{C.~Petta} \affiliation{\INFNCatania}\affiliation{\CataniaUniversitadi}
\author{R.~Petti} \affiliation{\Southcarolina}
\author{V.~Pia} \affiliation{\INFNBologna}\affiliation{\BolognaUniversity}
\author{L.~Pickering} \affiliation{\Royalholloway}
\author{F.~Pietropaolo} \affiliation{\CERN}\affiliation{\INFNPadova}
\author{V.~L.~Pimentel} \affiliation{\Cti}\affiliation{\Campinas}
\author{G.~Pinaroli} \affiliation{\Brookhaven}
\author{K.~Plows} \affiliation{\Oxford}
\author{R.~Plunkett} \affiliation{\Fermi}
\author{C.~Pollack} \affiliation{\PuertoRico}
\author{T.~Pollman} \affiliation{\Nikhef}\affiliation{\Amsterdam}
\author{F.~Pompa} \affiliation{\IFIC}
\author{X.~Pons} \affiliation{\CERN}
\author{N.~Poonthottathil} \affiliation{\Iitk}
\author{F.~Poppi} \affiliation{\INFNBologna}\affiliation{\BolognaUniversity}
\author{S.~Pordes} \affiliation{\Fermi}
\author{J.~Porter} \affiliation{\Sussex}
\author{M.~Potekhin} \affiliation{\Brookhaven}
\author{R.~Potenza} \affiliation{\INFNCatania}\affiliation{\CataniaUniversitadi}
\author{B.~V.~K.~S.~Potukuchi} \affiliation{\Jammu}
\author{J.~Pozimski} \affiliation{\Imperial}
\author{M.~Pozzato} \affiliation{\INFNBologna}\affiliation{\BolognaUniversity}
\author{S.~Prakash} \affiliation{\Campinas}
\author{T.~Prakash} \affiliation{\LawrenceBerkeley}
\author{C.~Pratt} \affiliation{\CalDavis}
\author{M.~Prest} \affiliation{\INFNMilanBicocca}
\author{F.~Psihas} \affiliation{\Fermi}
\author{D.~Pugnere} \affiliation{\IPLyon}
\author{X.~Qian} \affiliation{\Brookhaven}
\author{J.~L.~Raaf} \affiliation{\Fermi}
\author{V.~Radeka} \affiliation{\Brookhaven}
\author{J.~Rademacker} \affiliation{\Bristol}
\author{R.~Radev} \affiliation{\CERN}
\author{B.~Radics} \affiliation{\York}
\author{A.~Rafique} \affiliation{\Argonne}
\author{E.~Raguzin} \affiliation{\Brookhaven}
\author{M.~Rai} \affiliation{\Warwick}
\author{M.~Rajaoalisoa} \affiliation{\Cincinnati}
\author{I.~Rakhno} \affiliation{\Fermi}
\author{L.~Rakotondravohitra} \affiliation{\Antananarivo}
\author{R.~Rameika} \affiliation{\Fermi}
\author{M.~A.~Ramirez Delgado} \affiliation{\Penn}
\author{B.~Ramson} \affiliation{\Fermi}
\author{A.~Rappoldi} \affiliation{\INFNPavia}\affiliation{\Pavia}
\author{G.~Raselli} \affiliation{\INFNPavia}\affiliation{\Pavia}
\author{P.~Ratoff} \affiliation{\Lancaster}
\author{R.~Ray} \affiliation{\Fermi}
\author{H.~Razafinime} \affiliation{\Cincinnati}
\author{R.~F.~Razakamiandra} \affiliation{\Antananarivo}
\author{E.~M.~Rea} \affiliation{\Minntwin}
\author{J.~S.~Real} \affiliation{\Grenoble}
\author{B.~Rebel} \affiliation{\Wisconsin}\affiliation{\Fermi}
\author{R.~Rechenmacher} \affiliation{\Fermi}
\author{M.~Reggiani-Guzzo} \affiliation{\Manchester}
\author{J.~Reichenbacher} \affiliation{\SouthDakotaSchool}
\author{S.~D.~Reitzner} \affiliation{\Fermi}
\author{H.~Rejeb Sfar} \affiliation{\CERN}
\author{A.~Renshaw} \affiliation{\Houston}
\author{S.~Rescia} \affiliation{\Brookhaven}
\author{F.~Resnati} \affiliation{\CERN}
\author{M.~Ribas} \affiliation{\Tecnologica }
\author{S.~Riboldi} \affiliation{\INFNMilano}
\author{C.~Riccio} \affiliation{\StonyBrook}
\author{G.~Riccobene} \affiliation{\INFNSud}
\author{L.~C.~J.~Rice} \affiliation{\Pitt}
\author{J.~S.~Ricol} \affiliation{\Grenoble}
\author{A.~Rigamonti} \affiliation{\CERN}
\author{M.~Rigan} \affiliation{\Sussex}
\author{E.~V.~Rinc{\'o}n} \affiliation{\EIA}
\author{A.~Ritchie-Yates} \affiliation{\Royalholloway}
\author{S.~Ritter} \affiliation{\Mainz}
\author{D.~Rivera} \affiliation{\LosAlmos}
\author{R.~Rivera} \affiliation{\Fermi}
\author{A.~Robert} \affiliation{\Grenoble}
\author{J.~L.~Rocabado Rocha} \affiliation{\IFIC}
\author{L.~Rochester} \affiliation{\SLAC}
\author{M.~Roda} \affiliation{\Liverpool}
\author{P.~Rodrigues} \affiliation{\Oxford}
\author{M.~J.~Rodriguez Alonso} \affiliation{\CERN}
\author{J.~Rodriguez Rondon} \affiliation{\SouthDakotaSchool}
\author{S.~Rosauro-Alcaraz} \affiliation{\Parissaclay}
\author{P.~Rosier} \affiliation{\Parissaclay}
\author{M.~Rossella} \affiliation{\INFNPavia}\affiliation{\Pavia}
\author{M.~Rossi} \affiliation{\CERN}
\author{M.~Ross-Lonergan} \affiliation{\LosAlmos}
\author{J.~Rout} \affiliation{\Jawaharlal}
\author{P.~Roy} \affiliation{\Wichita}
\author{C.~Rubbia} \affiliation{\GranSasso}
\author{G.~Ruiz Ferreira} \affiliation{\Manchester}
\author{B.~Russell} \affiliation{\LawrenceBerkeley}
\author{D.~Ruterbories} \affiliation{\Rochester}
\author{A.~Rybnikov} \affiliation{\JINR}
\author{A.~Saa-Hernandez} \affiliation{\IGFAE}
\author{R.~Saakyan} \affiliation{\UniversityCollegeLondon}
\author{S.~Sacerdoti} \affiliation{\Parisuniversite}
\author{S.~K.~Sahoo} \affiliation{\IndHyderabad}
\author{N.~Sahu} \affiliation{\IndHyderabad}
\author{P.~Sala} \affiliation{\INFNMilano}\affiliation{\CERN}
\author{A.~R.~Samana} \affiliation{\SantaCruz}
\author{N.~Samios} \affiliation{\Brookhaven}
\author{O.~Samoylov} \affiliation{\JINR}
\author{M.~C.~Sanchez} \affiliation{\Floridastate}
\author{P.~Sanchez-Lucas} \affiliation{\Granada}
\author{V.~Sandberg} \affiliation{\LosAlmos}
\author{D.~A.~Sanders} \affiliation{\Mississippi}
\author{D.~Sankey} \affiliation{\Rutherford}
\author{D.~Santoro} \affiliation{\INFNMilano}
\author{N.~Saoulidou} \affiliation{\Athens}
\author{P.~Sapienza} \affiliation{\INFNSud}
\author{C.~Sarasty} \affiliation{\Cincinnati}
\author{I.~Sarcevic} \affiliation{\Arizona}
\author{I.~Sarra} \affiliation{\INFNFrascati}
\author{G.~Savage} \affiliation{\Fermi}
\author{V.~Savinov} \affiliation{\Pitt}
\author{G.~Scanavini} \affiliation{\Yale}
\author{A.~Scaramelli} \affiliation{\INFNPavia}
\author{A.~Scarff} \affiliation{\Sheffield}
\author{A.~Scarpelli} \affiliation{\Brookhaven}
\author{T.~Schefke} \affiliation{\Louisanastate}
\author{H.~Schellman} \affiliation{\OregonState}\affiliation{\Fermi}
\author{S.~Schifano} \affiliation{\INFNFerrara}\affiliation{\Ferrarauniv}
\author{P.~Schlabach} \affiliation{\Fermi}
\author{D.~Schmitz} \affiliation{\Chicago}
\author{A.~W.~Schneider} \affiliation{\Massinsttech}
\author{K.~Scholberg} \affiliation{\Duke}
\author{A.~Schukraft} \affiliation{\Fermi}
\author{E.~Segreto} \affiliation{\Campinas}
\author{A.~Selyunin} \affiliation{\JINR}
\author{C.~R.~Senise} \affiliation{\Unifesp}
\author{J.~Sensenig} \affiliation{\Penn}
\author{M.~H.~Shaevitz} \affiliation{\Columbia}
\author{S.~Shafaq} \affiliation{\Jawaharlal}
\author{F.~Shaker} \affiliation{\York}
\author{P.~Shanahan} \affiliation{\Fermi}
\author{H.~R.~Sharma} \affiliation{\Jammu}
\author{R.~Sharma} \affiliation{\Brookhaven}
\author{R.~Kumar} \affiliation{\Punjab}
\author{K.~Shaw} \affiliation{\Sussex}
\author{T.~Shaw} \affiliation{\Fermi}
\author{K.~Shchablo} \affiliation{\IPLyon}
\author{C.~Shepherd-Themistocleous} \affiliation{\Rutherford}
\author{A.~Sheshukov} \affiliation{\JINR}
\author{W.~Shi} \affiliation{\StonyBrook}
\author{S.~Shin} \affiliation{\Jeonbuk}
\author{I.~Shoemaker} \affiliation{\VirginiaTech}
\author{D.~Shooltz} \affiliation{\Michiganstate}
\author{R.~Shrock} \affiliation{\StonyBrook}
\author{B.~Siddi} \affiliation{\INFNFerrara}
\author{J.~Silber} \affiliation{\LawrenceBerkeley}
\author{L.~Simard} \affiliation{\Parissaclay}
\author{J.~Sinclair} \affiliation{\SLAC}
\author{G.~Sinev} \affiliation{\SouthDakotaSchool}
\author{Jaydip Singh} \affiliation{\Lucknow}
\author{J.~Singh} \affiliation{\Lucknow}
\author{L.~Singh} \affiliation{\CUSB}
\author{P.~Singh} \affiliation{\QMUL}
\author{V.~Singh} \affiliation{\CUSB}
\author{S.~Singh Chauhan} \affiliation{\Panjab}
\author{R.~Sipos} \affiliation{\CERN}
\author{C.~Sironneau} \affiliation{\Parisuniversite}
\author{G.~Sirri} \affiliation{\INFNBologna}
\author{K.~Siyeon} \affiliation{\ChungAng}
\author{K.~Skarpaas} \affiliation{\SLAC}
\author{E.~Smith} \affiliation{\Indiana}
\author{P.~Smith} \affiliation{\Indiana}
\author{J.~Smolik} \affiliation{\CzechTechnical}
\author{M.~Smy} \affiliation{\CalIrvine}
\author{E.L.~Snider} \affiliation{\Fermi}
\author{P.~Snopok} \affiliation{\Illinoisinstitute}
\author{D.~Snowden-Ifft} \affiliation{\Occidental}
\author{M.~Soares Nunes} \affiliation{\Syracuse}
\author{H.~Sobel} \affiliation{\CalIrvine}
\author{M.~Soderberg} \affiliation{\Syracuse}
\author{S.~Sokolov} \affiliation{\JINR}
\author{C.~J.~Solano Salinas} \affiliation{\Ingenieria}
\author{S.~S\"oldner-Rembold} \affiliation{\Manchester}
\author{S.R.~Soleti} \affiliation{\LawrenceBerkeley}
\author{N.~Solomey} \affiliation{\Wichita}
\author{V.~Solovov} \affiliation{\LIP}
\author{W.~E.~Sondheim} \affiliation{\LosAlmos}
\author{M.~Sorel} \affiliation{\IFIC}
\author{A.~Sotnikov} \affiliation{\JINR}
\author{J.~Soto-Oton} \affiliation{\IFIC}
\author{A.~Sousa} \affiliation{\Cincinnati}
\author{K.~Soustruznik} \affiliation{\Charles}
\author{F.~Spagliardi} \affiliation{\Oxford}
\author{M.~Spanu} \affiliation{\INFNMilanBicocca}\affiliation{\MilanoBicocca}
\author{J.~Spitz} \affiliation{\Michigan}
\author{N.~J.~C.~Spooner} \affiliation{\Sheffield}
\author{K.~Spurgeon} \affiliation{\Syracuse}
\author{D.~Stalder} \affiliation{\Asuncion}
\author{M.~Stancari} \affiliation{\Fermi}
\author{L.~Stanco} \affiliation{\INFNPadova}\affiliation{\Padova}
\author{J.~Steenis} \affiliation{\CalDavis}
\author{R.~Stein} \affiliation{\Bristol}
\author{H.~M.~Steiner} \affiliation{\LawrenceBerkeley}
\author{A.~F.~Steklain Lisb\^oa} \affiliation{\Tecnologica }
\author{A.~Stepanova} \affiliation{\JINR}
\author{J.~Stewart} \affiliation{\Brookhaven}
\author{B.~Stillwell} \affiliation{\Chicago}
\author{J.~Stock} \affiliation{\SouthDakotaSchool}
\author{F.~Stocker} \affiliation{\CERN}
\author{T.~Stokes} \affiliation{\Louisanastate}
\author{M.~Strait} \affiliation{\Minntwin}
\author{T.~Strauss} \affiliation{\Fermi}
\author{L.~Strigari} \affiliation{\TexasAMcollege}
\author{A.~Stuart} \affiliation{\Colima}
\author{J.~G.~Suarez} \affiliation{\EIA}
\author{J.~Subash} \affiliation{\Birmingham}
\author{A.~Surdo} \affiliation{\INFNLecce}
\author{L.~Suter} \affiliation{\Fermi}
\author{C.~M.~Sutera} \affiliation{\INFNCatania}\affiliation{\CataniaUniversitadi}
\author{K.~Sutton} \affiliation{\Caltech}
\author{Y.~Suvorov} \affiliation{\INFNNapoli}\affiliation{\napoli}
\author{R.~Svoboda} \affiliation{\CalDavis}
\author{S.~K.~Swain} \affiliation{\Niser}
\author{B.~Szczerbinska} \affiliation{\TexasAMcorpuscristi}
\author{A.~M.~Szelc} \affiliation{\Edinburgh}
\author{A.~Taffara} \affiliation{\INFNPisa}
\author{N.~Talukdar} \affiliation{\Southcarolina}
\author{J.~Tamara} \affiliation{\AntonioNarino}
\author{H. A.~Tanaka} \affiliation{\SLAC}
\author{S.~Tang} \affiliation{\Brookhaven}
\author{N.~Taniuchi} \affiliation{\Cambridge}
\author{B.~Tapia Oregui} \affiliation{\Texasaustin}
\author{A.~Tapper} \affiliation{\Imperial}
\author{S.~Tariq} \affiliation{\Fermi}
\author{E.~Tarpara} \affiliation{\Brookhaven}
\author{E.~Tatar} \affiliation{\Idaho}
\author{R.~Tayloe} \affiliation{\Indiana}
\author{A.~M.~Teklu} \affiliation{\StonyBrook}
\author{P.~Tennessen} \affiliation{\LawrenceBerkeley}\affiliation{\Antalya}
\author{M.~Tenti} \affiliation{\INFNBologna}
\author{K.~Terao} \affiliation{\SLAC}
\author{F.~Terranova} \affiliation{\INFNMilanBicocca}\affiliation{\MilanoBicocca}
\author{G.~Testera} \affiliation{\INFNGenova}
\author{T.~Thakore} \affiliation{\Cincinnati}
\author{A.~Thea} \affiliation{\Rutherford}
\author{A.~Thompson} \affiliation{\TexasAMcollege}
\author{C.~Thorn} \affiliation{\Brookhaven}
\author{S.~C.~Timm} \affiliation{\Fermi}
\author{V.~Tishchenko} \affiliation{\Brookhaven}
\author{N.~Todorovi{\'c}} \affiliation{\NoviSad}
\author{L.~Tomassetti} \affiliation{\INFNFerrara}\affiliation{\Ferrarauniv}
\author{A.~Tonazzo} \affiliation{\Parisuniversite}
\author{D.~Torbunov} \affiliation{\Brookhaven}
\author{M.~Torti} \affiliation{\INFNMilanBicocca}\affiliation{\MilanoBicocca}
\author{M.~Tortola} \affiliation{\IFIC}
\author{F.~Tortorici} \affiliation{\INFNCatania}\affiliation{\CataniaUniversitadi}
\author{N.~Tosi} \affiliation{\INFNBologna}
\author{D.~Totani} \affiliation{\CalSantabarbara}
\author{M.~Toups} \affiliation{\Fermi}
\author{C.~Touramanis} \affiliation{\Liverpool}
\author{R.~Travaglini} \affiliation{\INFNBologna}
\author{J.~Trevor} \affiliation{\Caltech}
\author{S.~Trilov} \affiliation{\Bristol}
\author{W.~H.~Trzaska} \affiliation{\Jyvaskyla}
\author{Y.~Tsai} \affiliation{\CalIrvine}
\author{Y.-T.~Tsai} \affiliation{\SLAC}
\author{Z.~Tsamalaidze} \affiliation{\Georgian}
\author{K.~V.~Tsang} \affiliation{\SLAC}
\author{N.~Tsverava} \affiliation{\Georgian}
\author{S.~Z.~Tu} \affiliation{\Jacksonstate}
\author{S.~Tufanli} \affiliation{\CERN}
\author{C.~Tull} \affiliation{\LawrenceBerkeley}
\author{J.~Turner} \affiliation{\Durham}
\author{M.~Tuzi} \affiliation{\IFIC}
\author{J.~Tyler} \affiliation{\Kansasstate}
\author{E.~Tyley} \affiliation{\Sheffield}
\author{M.~Tzanov} \affiliation{\Louisanastate}
\author{M.~A.~Uchida} \affiliation{\Cambridge}
\author{J.~Urheim} \affiliation{\Indiana}
\author{T.~Usher} \affiliation{\SLAC}
\author{H.~Utaegbulam} \affiliation{\Syracuse}
\author{S.~Uzunyan} \affiliation{\Northernillinois}
\author{M.~R.~Vagins} \affiliation{\Kavli}\affiliation{\CalIrvine}
\author{P.~Vahle} \affiliation{\WilliamMary}
\author{S.~Valder} \affiliation{\Sussex}
\author{G.~D.~A.~Valdiviesso} \affiliation{\FederaldeAlfenas}
\author{E.~Valencia} \affiliation{\Guanajuato}
\author{R.~Valentim} \affiliation{\Unifesp}
\author{Z.~Vallari} \affiliation{\Caltech}
\author{E.~Vallazza} \affiliation{\INFNMilanBicocca}
\author{J.~W.~F.~Valle} \affiliation{\IFIC}
\author{S.~Vallecorsa} \affiliation{\CERN}
\author{R.~Van Berg} \affiliation{\Penn}
\author{R.~G.~Van de Water} \affiliation{\LosAlmos}
\author{D.~Vanegas Forero} \affiliation{\Medellin}
\author{F.~Varanini} \affiliation{\INFNPadova}
\author{D.~Vargas Oliva} \affiliation{\Toronto}
\author{G.~Varner} \affiliation{\Hawaii}
\author{S.~Vasina} \affiliation{\JINR}
\author{N.~Vaughan} \affiliation{\OregonState}
\author{K.~Vaziri} \affiliation{\Fermi}
\author{J.~Vega} \affiliation{\conida}
\author{S.~Ventura} \affiliation{\INFNPadova}
\author{A.~Verdugo} \affiliation{\CIEMAT}
\author{S.~Vergani} \affiliation{\Cambridge}
\author{M.~A.~Vermeulen} \affiliation{\Nikhef}
\author{M.~Verzocchi} \affiliation{\Fermi}
\author{M.~Vicenzi} \affiliation{\INFNGenova}\affiliation{\Genova}
\author{H.~Vieira de Souza} \affiliation{\Parisuniversite}
\author{C.~Vignoli} \affiliation{\GranSassoLab}
\author{C.~Vilela} \affiliation{\CERN}
\author{B.~Viren} \affiliation{\Brookhaven}
\author{A.~Vizcaya-Hernandez} \affiliation{\ColoradoState}
\author{T.~Vrba} \affiliation{\CzechTechnical}
\author{Q.~Vuong} \affiliation{\Rochester}
\author{A.~V.~Waldron} \affiliation{\QMUL}
\author{M.~Wallbank} \affiliation{\Cincinnati}
\author{J.~Walsh} \affiliation{\Michiganstate}
\author{T.~Walton} \affiliation{\Fermi}
\author{H.~Wang} \affiliation{\CalLosangeles}
\author{J.~Wang} \affiliation{\SouthDakotaSchool}
\author{L.~Wang} \affiliation{\LawrenceBerkeley}
\author{M.H.L.S.~Wang} \affiliation{\Fermi}
\author{X.~Wang} \affiliation{\Fermi}
\author{Y.~Wang} \affiliation{\CalLosangeles}
\author{K.~Warburton} \affiliation{\IowaState}
\author{D.~Warner} \affiliation{\ColoradoState}
\author{M.O.~Wascko} \affiliation{\Imperial}
\author{D.~Waters} \affiliation{\UniversityCollegeLondon}
\author{A.~Watson} \affiliation{\Birmingham}
\author{K.~Wawrowska} \affiliation{\Rutherford}\affiliation{\Sussex}
\author{P.~Weatherly} \affiliation{\Drexel}
\author{A.~Weber} \affiliation{\Mainz}\affiliation{\Fermi}
\author{M.~Weber} \affiliation{\Bern}
\author{H.~Wei} \affiliation{\Louisanastate}
\author{A.~Weinstein} \affiliation{\IowaState}
\author{D.~Wenman} \affiliation{\Wisconsin}
\author{M.~Wetstein} \affiliation{\IowaState}
\author{J.~Whilhelmi} \affiliation{\Yale}
\author{A.~White} \affiliation{\TexasArlington}
\author{A.~White} \affiliation{\Yale}
\author{L.~H.~Whitehead} \affiliation{\Cambridge}
\author{D.~Whittington} \affiliation{\Syracuse}
\author{M.~J.~Wilking} \affiliation{\StonyBrook}
\author{A.~Wilkinson} \affiliation{\UniversityCollegeLondon}
\author{C.~Wilkinson} \affiliation{\LawrenceBerkeley}
\author{Z.~Williams} \affiliation{\TexasArlington}
\author{F.~Wilson} \affiliation{\Rutherford}
\author{R.~J.~Wilson} \affiliation{\ColoradoState}
\author{W.~Wisniewski} \affiliation{\SLAC}
\author{J.~Wolcott} \affiliation{\Tufts}
\author{J.~Wolfs} \affiliation{\Rochester}
\author{T.~Wongjirad} \affiliation{\Tufts}
\author{A.~Wood} \affiliation{\Houston}
\author{K.~Wood} \affiliation{\LawrenceBerkeley}
\author{E.~Worcester} \affiliation{\Brookhaven}
\author{M.~Worcester} \affiliation{\Brookhaven}
\author{M.~Wospakrik} \affiliation{\Fermi}
\author{K.~Wresilo} \affiliation{\Cambridge}
\author{C.~Wret} \affiliation{\Rochester}
\author{S.~Wu} \affiliation{\Minntwin}
\author{W.~Wu} \affiliation{\Fermi}
\author{W.~Wu} \affiliation{\CalIrvine}
\author{M.~Wurm} \affiliation{\Mainz}
\author{J.~Wyenberg} \affiliation{\Dordt}
\author{Y.~Xiao} \affiliation{\CalIrvine}
\author{I.~Xiotidis} \affiliation{\Imperial}
\author{B.~Yaeggy} \affiliation{\Cincinnati}
\author{N.~Yahlali} \affiliation{\IFIC}
\author{E.~Yandel} \affiliation{\CalSantabarbara}
\author{G.~Yang} \affiliation{\StonyBrook}
\author{K.~Yang} \affiliation{\Oxford}
\author{T.~Yang} \affiliation{\Fermi}
\author{A.~Yankelevich} \affiliation{\CalIrvine}
\author{N.~Yershov} \affiliation{\INR}
\author{K.~Yonehara} \affiliation{\Fermi}
\author{Y.~S.~Yoon} \affiliation{\ChungAng}
\author{T.~Young} \affiliation{\Northdakota}
\author{B.~Yu} \affiliation{\Brookhaven}
\author{H.~Yu} \affiliation{\Brookhaven}
\author{H.~Yu} \affiliation{\Sunyatsen}
\author{J.~Yu} \affiliation{\TexasArlington}
\author{Y.~Yu} \affiliation{\Illinoisinstitute}
\author{W.~Yuan} \affiliation{\Edinburgh}
\author{R.~Zaki} \affiliation{\York}
\author{J.~Zalesak} \affiliation{\CzechAcademyofSciences}
\author{L.~Zambelli} \affiliation{\DannecyleVieux}
\author{B.~Zamorano} \affiliation{\Granada}
\author{A.~Zani} \affiliation{\INFNMilano}
\author{L.~Zazueta} \affiliation{\WilliamMary}
\author{G.~P.~Zeller} \affiliation{\Fermi}
\author{J.~Zennamo} \affiliation{\Fermi}
\author{K.~Zeug} \affiliation{\Wisconsin}
\author{C.~Zhang} \affiliation{\Brookhaven}
\author{S.~Zhang} \affiliation{\Indiana}
\author{Y.~Zhang} \affiliation{\Pitt}
\author{M.~Zhao} \affiliation{\Brookhaven}
\author{E.~Zhivun} \affiliation{\Brookhaven}
\author{E.~D.~Zimmerman} \affiliation{\ColoradoBoulder}
\author{S.~Zucchelli} \affiliation{\INFNBologna}\affiliation{\BolognaUniversity}
\author{J.~Zuklin} \affiliation{\CzechAcademyofSciences}
\author{V.~Zutshi} \affiliation{\Northernillinois}
\author{R.~Zwaska} \affiliation{\Fermi}
\collaboration{The DUNE Collaboration}
\noaffiliation


\date{\today}

\begin{abstract}
A primary goal of the upcoming Deep Underground Neutrino Experiment (DUNE) is
to measure the $\mathcal{O}(10)$ MeV neutrinos produced by a Galactic core-collapse
supernova if one should occur during the lifetime of the experiment. The
liquid-argon-based detectors planned for DUNE are expected to be uniquely
sensitive to the $\nu_e$ component of the supernova flux, enabling
a wide variety of physics and astrophysics measurements. A key requirement for a correct
interpretation of these measurements is a good understanding of the
energy-dependent total cross section $\sigma(E_\nu)$ for charged-current
$\nu_e$ absorption on argon. In the context of a simulated extraction of
supernova $\nu_e$ spectral parameters from a toy analysis, we investigate the
impact of $\sigma(E_\nu)$ modeling uncertainties on DUNE's supernova neutrino
physics sensitivity for the first time. We find that the currently large
theoretical uncertainties on $\sigma(E_\nu)$ must be substantially reduced
before the $\nu_e$ flux parameters can be extracted reliably: in the absence of
external constraints, a measurement of the integrated neutrino luminosity with less than
10\% bias with DUNE requires $\sigma(E_\nu)$ to be known to about 5\%.  The neutrino spectral shape parameters can be known to better than 10\% for a 20\% uncertainty on the cross-section scale, although they will be sensitive to uncertainties on the shape of $\sigma(E_\nu)$.
 A direct measurement of low-energy $\nu_e$-argon
scattering would be invaluable for improving the theoretical precision to the
needed level.
\end{abstract}

%\keywords{Suggested keywords}%Use showkeys class option if keyword
                              %display desired
\maketitle
%TC:endignore

%\tableofcontents

\section{\label{sec:level1}Introduction}
A massive star ($M > 8M_\odot$) employs nuclear fusion to sustain itself by
first consuming lighter elements such as hydrogen and helium and later
consuming heavier elements. In the canonical narrative, at the end of the star's lifetime, the
innermost nickel-iron core can no longer undergo nuclear fusion. Gravity
causes the core to collapse into a proto-neutron star. Neutron degeneracy
stalls the collapse; the core rebounds and produces shock waves which
propagate outward from the core. Once the shock waves breach the surface of
the star, they expel stellar material and leave behind a compact remnant. This process is referred to as a core-collapse supernova.

A core collapse releases 99\% of the star's gravitational potential
energy via neutrinos in a prompt burst lasting several seconds~\footnote{A core collapse leaving behind a black hole may not result in a visible supernova, but will still emit a bright burst of neutrinos.}. While the
proto-neutron star traps photons and other particles with electromagnetic and strong interactions, neutrinos easily escape
because they interact weakly. The neutrino flux is expected to contain interesting signatures related to different phenomena occurring during a core-collapse supernova \cite{PhysRevD.87.085037,PhysRevD.63.073011,arxiv0205390,Hanke_2012,arxiv0607244}, including insight into the explosion mechanism. While the neutrinos detected from SN1987A~\cite{Bionta:1987qt,Hirata:1987hu,Alekseev:1987ej,Aglietta:1987it} did help to
confirm the basic outline of the core-collapse supernova process, they did not provide
tight constraints on astrophysical models. Additional neutrino signals from
core-collapse supernovae observed in detectors worldwide~\cite{detection} will provide data to study the mechanism behind the
core collapse, as well as information on the properties of neutrinos themselves.

Obtaining a high-statistics measurement of core-collapse supernova neutrinos
is among the primary physics goals for the Deep Underground Neutrino
Experiment (DUNE). To detect these low-energy neutrinos, DUNE will utilize its
far detector (relative to the beam at Fermilab) located 1.5~km underground at the Sanford Underground Research
Facility in South Dakota. The DUNE far detector is currently planned to
consist of four liquid argon time projection chambers
(LArTPCs) each with a total volume of around seventeen kilotons~\cite{Abi:2020evt}.  These LArTPC detectors will be sensitive to interactions of neutrinos in the few tens of MeV range~\cite{Caratelli:2022llt}.

Among large neutrino experiments, DUNE will be uniquely sensitive to the
$\nu_e$ component of the supernova signal via the charged-current reaction
\begin{equation}
    \nu_e + \isotope[40]{Ar} \to e^- + \isotope[40]{K}^*.
\end{equation}
The $\nu_e$ component of the supernova neutrino flux is expected to contain
unique features which make its future detection with DUNE a valuable
scientific opportunity~\cite{Abi:2020evt}.

The neutrinos generated by a core-collapse supernova have much lower energies
(few to tens of MeV) than the GeV-scale neutrino beams of interest for DUNE's
accelerator-based oscillation physics program. Below 100~MeV, no measurements
of charged-current neutrino-argon cross sections are currently available~\footnote{If solar-neutrino flux is assumed known, one can in principle use it to measure the cross section below 14 MeV and bound it from below for energies above 14 MeV.}, and
competing theoretical calculations have significant discrepancies~\footnote{See Ref.~\cite{Capozzi2018} for discussion of energies below 15~MeV.}. While the
importance of obtaining a precise understanding of neutrino-nucleus scattering
at accelerator energies is widely
recognized~\cite{AlvarezRuso2018,Athar2021,Avanzini2021}, and the impact of
related uncertainties has been studied in detail by the DUNE
collaboration~\cite{DUNElbl}, the same cannot yet be said for the tens of MeV
regime relevant for supernova neutrino detection. This situation exists
despite shared analysis challenges between the two energy scales: in both
cases, a reliable cross-section model is needed for neutrino calorimetry,
efficiency estimation, and removal of some classes of background events.
Theoretical uncertainties on the cross-section model provide an important
limitation on the achievable experimental precision.

In this paper, we examine for the first time the impact of cross-section
uncertainties on the interpretation of a possible future observation of
supernova neutrinos with DUNE. No attempt is made here to be comprehensive in
either the uncertainty budget or in the analysis topics considered; for instance, these studies assume that the distance to the core collapse is known precisely. Our aim is
instead to explore how variations of the adopted model of the neutrino-argon
cross section affect the results of a measurement of simulated data. The
present study is restricted to variations of $\sigma(E_\nu)$, the total
charged-current cross section as a function of neutrino energy. The studies presented in this paper use simplified assumptions about detector response, but a realistic efficiency for DUNE includes sensitivity to neutrino energies as low as 5 MeV~\cite{DUNE:2020ypp}. Although these studies require an assumption about DUNE's expected energy resolution, similar studies performed in Ref.~\cite{Abi:2020evt} show that the results are not sensitive to the specific choice of energy resolution~\footnote{Knowledge of energy resolution is more important for bias in extracted parameters than specific value of energy resolution.  For the purpose of this study, which is focused on the effect of cross-section uncertainty, we assume that the detector response is perfectly known.}.
Variations to
other aspects of the neutrino interaction model, including predictions of exclusive final-state
differential distributions and the description of $\isotope[40]{K}^*$ nuclear
de-excitations, as well as subdominant neutral-current and $\bar{\nu}_e$ charged-current interactions, are left to future work, both for simplicity and because the
related uncertainties are difficult to fully quantify at present.

The algorithm used in our measurements to extract supernova $\nu_e$ flux
parameters from simulated DUNE data is presented in
Sec.~\ref{algorithm_section}. In Sec.~\ref{xscn_study_section}, we describe
three different procedures for varying the $\nu_{e}-\isotope[40]{Ar}$ total
cross section, and the impact on the simulated measurements is examined for each
approach. We discuss the results, their implications for
DUNE's future supernova neutrino effort, and prospects for the future in Sec.~\ref{discussion} and
conclude in Sec.~\ref{conclusion_section}.

\section{Supernova parameter fitting}
\label{algorithm_section}
\subsection{Pinched-thermal form}
\label{flux_section}

A commonly-used representation for the supernova neutrino fluence (i.e., the time integral of the flux) $\Phi$ passing through the Earth is the pinched-thermal form~\cite{Minakata:2008nc,PhysRevD.86.125031}:
\begin{equation}
    \Phi(E_\nu) = \frac{ \varepsilon }{4\pi d^2} \, \mathcal{N} \, \bigg( \frac{E_\nu}{\langle E_\nu \rangle} \bigg)^\alpha \exp\bigg[ -(\alpha + 1)\frac{E_\nu}{\langle E_\nu \rangle} \bigg],
    \label{flux}
\end{equation}
where
\begin{equation}
\mathcal{N} \equiv \frac{ (\alpha + 1)^{\alpha + 1} }{ \langle E_\nu \rangle^2 \,\Gamma(\alpha + 1) },
\end{equation}
is a normalization constant, $\varepsilon$ is the neutrino luminosity, $E_\nu$ is the neutrino energy, $\langle E_\nu \rangle$ is the mean neutrino energy (related to the temperature of the supernova), and $d$ is the distance from the supernova to Earth. The ``pinching parameter'' $\alpha$ describes the shape of the tails of the neutrino energy distribution. 

The expression in Eq.~\ref{flux} may be used to represent either an instantaneous flux (with dimensions of neutrinos per area per time) or a fluence in a specific time interval (flux integrated over time,  with dimensions of neutrinos per area), depending on the units used for $\varepsilon$. In the
instantaneous case, the parameters $\langle E_\nu \rangle$ (MeV) and $\alpha$
(dimensionless) are implicitly time-dependent, while for the time-integrated case they should be interpreted as average values. 
The time-integrated spectrum is also well described by Eq.~\ref{flux}, and the parameters should be interpreted as being applied to the fluence spectrum over the entire burst.
For simplicity, we choose to
consider only the time-integrated neutrino flux in which $\varepsilon$ may be
expressed in ergs. A distance of $d = $ 10 kiloparsecs (kpc) is assumed throughout.
Different values of the flux parameters describe each neutrino species
separately (i.e., the $\nu_{e}$ parameters are not the same as the
$\bar{\nu}_e$ or $\nu_x \equiv \nu_\mu, \nu_\tau, \overline{\nu}_\mu,
\overline{\nu}_\tau$ parameters), but only the $\nu_e$ portion of the flux is
of interest for the present study given its dominance in the expected supernova signal in DUNE~\cite{Abi:2020evt}. For the studies in this paper, we assume equipartition between flavors, i.e., $\alpha_{\nu_e} = \alpha_{\overline{\nu}_e} = \alpha_{\nu_x}$ and $\varepsilon_{\nu_e} = \varepsilon_{\overline{\nu}_e} = \varepsilon_{\nu_x}$, and we adopt the hierarchy in Ref.~\cite{Rosso_2017} for the mean neutrino energies. The simulated measurements considered here involve an extraction of the $\nu_e$ pinched-thermal flux parameters $\varepsilon$, $\langle E_\nu \rangle$, and $\alpha$ from the reconstructed neutrino energy spectrum expected for DUNE. Figure~\ref{neutrinoflux_pinched_0_nmo} shows fluences calculated for a pinched-thermal flux.

\begin{figure}[H]
	\centering
	\includegraphics[scale = 0.44]{images_updated/neutrinoflux_pinched_0_nmo_zoomed.pdf}
	\caption{Pinched-thermal neutrino fluences for a supernova at a distance of 10 kpc. Following Ref.~\cite{rosso}, the results are time-integrated over the first ten seconds. The initial fluence parameter values for $\nu_e$ are $(\alpha^0, \langle E_\nu \rangle^0, \varepsilon^0) = (2.5, 9.5~\text{ MeV}, 5\times 10^{52}~\text{ ergs})$, for $\bar{\nu}_e$ are $(\alpha^0, \langle E_\nu \rangle^0, \varepsilon^0) = (2.5, 12.0~\text{ MeV}, 5\times 10^{52}~\text{ ergs})$, and for $\nu_x$ are $(\alpha^0, \langle E_\nu \rangle^0, \varepsilon^0) = (2.5, 15.6~\text{ MeV}, 5\times 10^{52}~\text{ ergs})$. Normal mass ordering and Mikheyev-Smirnov-Wolfenstein (MSW) resonances~\cite{wolfenstein1978neutrino,mikheyev1985nuovo}
	were assumed via Equation~\ref{msw-oscillations}. \label{neutrinoflux_pinched_0_nmo}
	}
\end{figure}

\begin{figure*}
  \centering
  \begin{minipage}[]{0.49\textwidth}
    \includegraphics[width=\textwidth]{images_updated/smearmat_nue_Ar40_Evis_partialHighEnergyMARLEYSample.pdf}
    \caption{SNOwGLoBES smearing matrix made with MARLEY modeling and 10\% Gaussian-smeared reconstructed energy. An energy column contains the reconstructed energy distribution for neutrino-argon events at a given true neutrino energy.}
    \label{MARLEY_Smearing_Matrix}
  \end{minipage}
  \hfill
  \begin{minipage}[]{0.49\textwidth}
    \includegraphics[width=\textwidth]{SNOwGLoBESRates_NMO_nueCCOnly.pdf}
    \caption{Interacted and observed event rates calculated using SNOwGLoBES for $\nu_e$-$^{40}$Ar interactions in the proposed DUNE far detector. The post-smearing efficiency model imposed a sharp cut at 5 MeV onto the observed rates.}
    \label{exampleSNOwGLoBES}
  \end{minipage}
\end{figure*}

\subsection{SNOwGLoBES}
Beyond the neutrino-argon cross section, the supernova signal observed
by DUNE will also be affected by the supernova flux, the detector
response, efficiency, and energy reconstruction. The SuperNova Observatories with General Long-Baseline Experiment Simulator (SNOwGLoBES) software incorporates the effect of detector response factors, including the cross section, into a simulated supernova neutrino signal. This widely used, open-source
event rate calculation tool offers a quick option to model the DUNE
far detector response for supernova neutrino signals~\cite{snowglobes}.


SNOwGLoBES requires several inputs to perform the simulation, including a cross-section model and a ``smearing matrix,'' i.e., a transfer matrix that can be used to calculate a reconstructed neutrino energy spectrum when applied to the true neutrino energy spectrum (see Fig.~\ref{MARLEY_Smearing_Matrix}).  In addition, there is an assumed post-smearing detection efficiency.  SNOwGLoBES makes use of GLoBES~\cite{globes} software to convolve a specified flux with a cross section and a smearing matrix.   We used fluxes given by  Eq.~\ref{flux} and computed the smearing matrix using simulated $\nu_{e}-\isotope[40]{Ar}$ interactions produced by the MARLEY event generator~\cite{marleyPRC,marleyCPC} with 10\% Gaussian smearing applied to the visible energy. The exact value of 10\% is modestly optimistic for DUNE's expected capabilities, but the results are not sensitive to the specific value~\cite{Abi:2020evt}.

For our simulated signal predictions, we adopted one of the more optimistic neutrino energy reconstruction scenarios described in Ref.~\cite{marleyPRC}. Under this scenario, the reconstructed neutrino energy is taken to be the \textit{visible energy} $E_\text{vis}^\text{reco}$ defined by the expression
\begin{equation}
    E_{\text{vis}}^{\text{reco}} \equiv E_{\text{bind}}^{\text{min}} + E_{\text{e}} + \mathcal{T}_\gamma + \mathcal{T}_{\text{ch}}
    \,.\label{recoE}
\end{equation}
Here, $E_\text{bind}^\text{min} = 0.99~\text{MeV}$ is the minimum possible change in nuclear binding energy for the charged-current reaction, $E_{\text{e}}$ is the total energy of the outgoing electron, $\mathcal{T}_\gamma$ is the summed energy of all de-excitation $\gamma$-rays, and $\mathcal{T}_{\text{ch}}$ is the summed kinetic energy of all final-state charged hadrons.

SNOwGLoBES outputs binned energy spectra (Asimov data sets) corresponding to different detector parameter assumptions and for given pinched-thermal spectral parameters  $(\alpha, \langle E_\nu \rangle, \varepsilon)$. Figure \ref{exampleSNOwGLoBES} shows the two types of SNOwGLoBES output energy spectra; ``interaction rates'' refers to the energies of neutrinos that interacted (before detector response), while ``observed rates'' refers to the prediction of the observed spectra in the proposed detector. The observed rates are what the proposed DUNE far detector would observe during the first ten seconds of a 10~kpc supernova burst.

\subsection{Mass ordering assumptions in SNOwGLoBES}
\label{massorder_snowglobes}
The different neutrino flavor amplitudes will change as they move through the collapsing star and in the vacuum of space toward Earth. These flavor transitions will affect the $\nu_{e}$ flux that reaches the DUNE detector, and consequently the flavor transitions will affect the $\nu_{e}$-$ ^{40}$Ar event rates. SNOwGLoBES provides a simple evaluation of the matter effect for both normal and inverted mass ordering assumptions; we assumed $\theta_{12} %= 0.588336 \text{ rads}
= 33.71 \degree$ and the following relations for flavor content for normal mass ordering (NMO) according to the standard prescription in Ref.~\cite{Mirizzi:2015eza}:

\begin{subequations}
\begin{eqnarray}
F_{\nu_{e}} &=& F_{\nu_{x}}^{0}, \label{fluxNMO}
\\
F_{\overline{\nu}_{e}} &=& \cos^{2}(\theta_{12}) F_{\overline{\nu}_{e}}^{0} + \sin^{2}(\theta_{12}) F_{\overline{\nu}_{x}}^{0}.
\end{eqnarray}
\label{msw-oscillations}
\end{subequations}

Here, $F_\nu$ is the flux for one (or more) neutrino flavor, and $F^0_\nu$ is the flux before the flavor transition. In the presence of flavor transitions, all produced flavors matter for the observed $\nu_e$ rate at Earth. To take into account effects produced by flavor transitions, we define a range of flux parameters for $\overline{\nu}_e$ and $\nu_x$ using the $\nu_e$ parameters and the relations outlined in Section \ref{flux_section}.

\subsection{Forward fitting}
\label{sec:forward_fitting}
The resulting reconstructed energy spectra from SNOwGLoBES are influenced by the choice of pinched-thermal flux parameters. Measurements of the spectral parameters might contain biases partly introduced by uncertainties in our input assumptions such as the cross-section model. We developed an algorithm that fits a reconstructed neutrino energy spectrum to obtain estimated values of the pinched-thermal parameters; this then enables us to study the impact of the $\nu_{e}-\isotope[40]{Ar}$ cross section model on the fit results.

Our algorithm employs a ``forward-fitting'' approach as an alternative to unfolding; in a forward-fitting approach, a theory prediction convolved with the response of a given detector is compared directly with data. Forward fitting requires two inputs: \begin{enumerate*}[label=(\arabic*)] \item a reconstructed neutrino energy spectrum produced by SNOwGLoBES for a supernova at a given distance and \item a ``true'' set of pinched-thermal parameters $(\alpha^0, \langle E_\nu \rangle^0, \varepsilon^0)$ \end{enumerate*}. The algorithm uses this spectrum as a ``true spectrum'' to compare against a reference grid of reconstructed energy spectra generated with many different combinations of $(\alpha, \langle E_\nu \rangle, \varepsilon)$. In this paper, the true spectrum refers to the assumed true spectrum under test in the algorithm. To quantify goodness-of-fit, the algorithm uses a $\chi^2$ function defined by 
\begin{equation}
    \chi^2 \equiv \sum_{i=1}^{n_b} \frac{\big[N_i(\alpha, \langle E_\nu \rangle, \varepsilon) - N_i(\alpha^0, \langle E_\nu \rangle^0, \varepsilon^0)\big]^2}{\sigma_i^2}\,.
    \label{chi2function}
\end{equation}
Here $n_b$ is the number of reconstructed energy bins, $N_{i}$ is the number of events in the $i$th bin, $\sigma_i$ is the statistical uncertainty on the number of events in the $i$th bin of the true spectrum, $(\alpha, \langle E_\nu \rangle, \varepsilon)$ is the set of flux parameters used to generate a reconstructed energy spectrum in the grid, and $(\alpha^0, \langle E_\nu \rangle^0, \varepsilon^0)$ are the
flux parameters used to generate the true spectrum. We assume statistics corresponding to the approximately expected flux for a core collapse at 10~kpc.

\begin{figure}[H]
    \centering
    \includegraphics[scale=0.31]{SuperimposedTestSpectrumGridElement_NMO_nueOnly.pdf}
    \caption{Event rates calculated using SNOwGLoBES for a true spectrum with initial fluence parameters $(\alpha^0, \langle E_\nu \rangle^0, \varepsilon^0) = (2.5, 9.5~\text{ MeV}, 5\times 10^{52}~\text{ ergs})$ and an example grid element with fluence parameters $(\alpha^0, \langle E_\nu \rangle^0, \varepsilon^0) = (3.8, 10.2~\text{ MeV}, 5\times 10^{52}~\text{ ergs})$ and reduced $\chi^2 = 4.85$ based on Eq.~\ref{chi2function}. The error bars are statistical.}
    \label{exampleFitting}
\end{figure}

Figure~\ref{exampleFitting} shows an example comparison of a true spectrum against one arbitrary grid element. Both spectra are represented by Asimov data sets; the error bars of the true spectrum are derived from the Poisson distribution. 
The true spectrum represents the predicted data that DUNE would observe during a supernova burst.  

The collection of $\chi^2$ values for each of the grid elements is used to determine the measurement uncertainty of the pinched-thermal parameters. We consider uncertainty regions in 2D parameter spaces $(\langle E_\nu \rangle, \alpha)$, $(\langle E_\nu \rangle, \varepsilon)$, and $(\alpha, \varepsilon)$, where the smallest $\chi^2$ is determined by profiling over the third parameter. 
We determine the approximate ``sensitivity regions'' by placing a
cut of $\chi^2 - \chi^2_\text{min} = 4.61$ corresponding to a 90\%
confidence level for two free parameters~\cite{pdg}\footnote{This criterion is satisfactory given that the statistical regime is such that a Poisson distribution is well approximated by a Gaussian.}. A sensitivity region is equivalent to the Asimov confidence region for a perfect prediction~\cite{Cowan:2010js}. 

Figure~\ref{exampleAsimovDistance} shows sensitivity regions in $(\langle E_\nu \rangle, \varepsilon)$ space for three different supernova distances; the number of events scales with the supernova distance, meaning the regions will grow larger for a more distant supernova. 

\begin{figure}[t]
    \includegraphics[scale=0.38]{images_updated/DifferentSNDistances_10MARLEY_NO_nueOnly_Bhattacharya2009Xscn_StepEffic_UpdatedLegendLabels.pdf}
    \caption{Sensitivity regions in $(\langle E_\nu \rangle, \varepsilon)$ space for three different supernova distances. These regions were generated from the smearing matrix shown in Fig. \ref{MARLEY_Smearing_Matrix}, a cross section model from MARLEY \cite{marleyCPC}, and a step efficiency function with a 5~MeV detection threshold.}
    \label{exampleAsimovDistance}
\end{figure}

\subsection{Figure of merit for forward fitting}
\label{section_fom}
We developed a figure of merit as a proxy for the systematic error due to the cross section uncertainty, where the figure of merit describes the best-fit measurement and characterizes DUNE's expected sensitivity to the supernova flux parameters. The
figure of merit $B_x$ is defined as the fractional bias on the measurement
of a parameter $x$
obtained from the fitting procedure:
\begin{equation}
    B_x \equiv \frac{x^\text{b.f.} - x^0}{x^0} \,.
    \label{fracdiff}
\end{equation}

The figure of merit depends on the best-fit value $x^\text{b.f.}$ and true
value $x^0$ of $x \in \{ \alpha, \langle E_\nu \rangle, \varepsilon \}$, where here we express $\langle E_\nu \rangle$ in MeV and $\varepsilon$ in ergs. 

For
the studies presented in this paper, we define all of our grids using the
same range of $\alpha$ and $\langle E_\nu \rangle$ values. 
The
allowed ranges are defined using the $\nu_e$ truth values $(\alpha^0,
\langle E_\nu \rangle^0, \varepsilon^0) = (2.5, 9.5, 5\times10^{52})$ and the
following bounds for reasonable $\alpha$ and $\langle E_\nu \rangle$ values are
taken from Ref.~\cite{rosso}:

\begin{itemize}[leftmargin=0.5in,rightmargin=0.05in]
    \item $\alpha \in [0.1, 7.0]$ with $0.1$ spacing, corresponding to fractional bias values
$B_\alpha \in [-0.96, 1.8]$
    \item $\langle E_\nu \rangle \in [5.0, 20.0]$ with $0.1$ spacing, corresponding to
fractional bias values $B_{\langle E_\nu \rangle} \in [-0.47, 1.10]$
\end{itemize}

For the $\varepsilon$ parameter, Ref.~\cite{rosso} defined a reasonable range
of $[2\times10^{52}, 1\times10^{53}]$ with $2.5\times10^{51}$ spacing, corresponding to bias values
$B_\varepsilon \in [-0.6, 1.0]$. We used this range for the study outlined in
Sec.~\ref{section_cross_section_uncertainty}. However, for the studies
outlined in Secs.~\ref{section_cross_section_models} and
\ref{section_cross_section_envelope}, this range was insufficient to study the
totality of the cross-section space covered by the various $\nu_{e}$-$^{40}$Ar
scattering models used in this paper. Therefore, we used the following (more conservative) range of $\varepsilon \in [1.0\times10^{51}, 1.0\times10^{54}]$ over several grids with spacings ranging from $2\times10^{51}$ to $5\times10^{52}$; the total range of $\varepsilon$ values
corresponds to bias values $B_\varepsilon \in [-1.0, 19.0]$.

\subsection{Study assumptions}
\label{sec:study_assumptions}
Here we summarize the assumptions used for the studies presented in this paper:

\begin{itemize}[leftmargin=0.2in,rightmargin=0.05in]
    \itemsep0em
    \item All neutrino species contribute to the pinched-thermal flux, where the true parameters for each flavor (before any flavor transition) are defined below~\cite{rosso}.
    \begin{itemize}
        \item $\nu_{e}$ flux: $(\alpha^0, \langle E_\nu \rangle^0, \varepsilon^0) = (2.5, 9.5, 5\times 10^{52})$
        \item $\overline{\nu}_e$ flux: $(\alpha^0, \langle E_\nu \rangle^0, \varepsilon^0) = (2.5, 12.0, 5\times 10^{52})$
        \item $\nu_x \equiv \nu_\mu, \nu_\tau, \overline{\nu}_\mu, \overline{\nu}_\tau$\\ flux: $(\alpha^0, \langle E_\nu \rangle^0, \varepsilon^0) = (2.5, 15.6, 5\times 10^{52})$
    \end{itemize}
    \item A pure pinched-thermal supernova flux.
    \item Normal mass ordering with standard MSW transition effects implemented using Eq.~\ref{msw-oscillations}; no ``collective" effects, spectral swaps or non-standard flavor transition effects.
    \item A supernova distance of 10 kpc with no distance uncertainty.
    \item Event rates integrated over a supernova burst lasting 10 seconds.
    \item Only charged-current $\nu_{e}-\isotope[40]{Ar}$ interactions in the simulated observed signal.
    \item SNOwGLoBES smearing matrix made with MARLEY modeling~\cite{marleyCPC} and 10\% Gaussian smearing.
    \item Post-smearing efficiencies in SNOwGLoBES of 100\% efficiency above a 5 MeV detection threshold.
\end{itemize}

%section for DUNE technical note
\subsection{Additional information to reproduce the results}
The studies in this paper used the following software:
\begin{itemize}
    \item SNOwGLoBES 1.2~\cite{snowglobes} 
    \item MARLEY 1.2.0~\cite{marleyCPC}
    \item ROOT 6.20~\cite{Brun:1997pa}
\end{itemize}

The studies rely heavily on simulated supernova event rates calculated with SNOwGLoBES. Instructions for how to produce single event rate files, along with grids of flux files, are included in the SNOwGLoBES software package. We used the MARLEY event generator to simulate $\nu_{e}-\isotope[40]{Ar}$ interactions while creating a smearing matrix for usage in SNOwGLoBES. The smearing matrix was created using SNOwGLoBES with 10\% Gaussian smearing applied. The forward-fitting algorithm and studies were conducted using ROOT; the forward-fitting algorithm is publicly available on GitHub at \url{https://github.com/erinecon/forward-fitting}. 

\begin{figure*}
  \centering
  \begin{minipage}[]{0.49\textwidth}
    \includegraphics[width=\textwidth]{CrossSectionModels_RPA_Paper.pdf}
    \caption{Cross-section calculations for the $\nu_{e}-\isotope[40]{Ar}$ interaction from Refs. \cite{snowglobes},
      \cite{Gil-Botella2003,Kolbe_2003}, \cite{Paar2013},
      \cite{Suzuki2013}, \cite{Samana2010}, \cite{Cheoun2011b}, and
      \cite{QRPA_SamanaEtAl}. The labels are explained in Table~\ref{xscn_description_table}. Note the log scale on the y-axis.}
    \label{CrossSectionModels_RPA_Paper}
  \end{minipage}
  \hfill
  \begin{minipage}[]{0.49\textwidth}
    \includegraphics[width=\textwidth]{CrossSectionModels_MARLEY_GTBD_Paper.pdf}
    \caption{Cross section calculations for the $\nu_{e}-\isotope[40]{Ar}$ interaction from Refs.~\cite{marleyPRC} and \cite{Samana2008}. The labels are explained in Table~\ref{xscn_description_table}. The y-axis range is the same as Fig. \ref{CrossSectionModels_RPA_Paper}.}
    \label{CrossSectionModels_MARLEY_GTBD_Paper}
  \end{minipage}
\end{figure*}

\section{Cross-section studies}
\label{xscn_study_section}

With the forward-fitting algorithm implemented to measure the spectral parameters, construct sensitivity regions, and calculate the bias figure of merit, we studied how the choice of $\nu_{e}-\isotope[40]{Ar}$ cross-section model could impact a supernova neutrino measurement in DUNE. Understanding systematic uncertainties and potential biases introduced by mismodeling of the cross section will be essential for a correct interpretation of any future core-collapse supernova observation.

\subsection{Neutrino-argon cross section models}
Many calculations of the $\nu_{e}-\isotope[40]{Ar}$ cross section have emerged over time using various nuclear structure models. In the studies performed for this paper, twelve cross-section models are considered. Table~\ref{xscn_description_table} briefly summarizes the features of the models.
Figures~\ref{CrossSectionModels_RPA_Paper} and \ref{CrossSectionModels_MARLEY_GTBD_Paper}
show the total charged-current cross sections predicted by each of the models in the energy region of interest. The models were split into two plots for easier readability; the RPA models are all contained in Fig.~\ref{CrossSectionModels_RPA_Paper}, while the GTBD model and the cross-sections calculated by MARLEY are contained in Fig.~\ref{CrossSectionModels_MARLEY_GTBD_Paper}.

The majority of these cross-section models are based on microscopic calculations using formalisms such as the Random Phase Approximation (RPA) or Quasiparticle RPA (QRPA). Under these approaches, collective states of nuclei are described using particle-hole
(quasiparticle) excitations. The RPA-based calculations include contributions from forbidden (or high-multipole-order) nuclear transitions, which become especially important for neutrinos with $E_\nu >$~50~MeV. A hybrid microscopic calculation~\cite{Suzuki2013} in which the allowed (lowest-multipole-order, i.e., Fermi and Gamow-Teller transitions) contributions were computed using the nuclear shell model (NSM) and the forbidden contributions were treated using the RPA is also considered.
Alternative macroscopic models like that in Ref.~\cite{Samana2008} use calculations based on the gross theory of beta decay
(GTBD) that describe the global properties of allowed $\beta$-decay processes. The calculations from MARLEY~\cite{marleyCPC} are partially data-driven and neglect forbidden nuclear transitions. A QRPA calculation is used by MARLEY
at excitation energies where relevant data are not currently available.

\begin{table*}
\caption{\label{xscn_description_table}
%\color{blue}
Brief features of $\nu_{e}-\isotope[40]{Ar}$ cross-section models used in this work.
%
}
\begin{ruledtabular}
\begin{tabular}{p{4.5cm}p{4cm}p{8cm}}
    Cross section model&Model name &Comments\\ \hline

    Default model implemented in SNOwGLoBES \cite{snowglobes}&SNOwGLoBES or S &Based on RPA calculations for all multipole transitions up to $J^{\pi}= 4^{\pm}$. \\ \hline

    Calculation by Martinez-Pinedo et al. \cite{Gil-Botella2003,Kolbe_2003} & RPA & Based on RPA
    calculations including all the multipole transitions up to $J^{\pi}= 6^{\pm}$. \\ \hline

    Calculation by M. Cheoun et al. \cite{Cheoun2011b} & QRPA-C & Based on QRPA calculations. The results are consistent with data from $(p, n)$ scattering reactions and Gamow-Teller strengths.\\ \hline

    Calculation by N. Paar et al. \cite{Paar2013} & RQRPA &
    Based on a self-consistent theory framework for a relativistic nuclear energy density functional. The cross sections are
  including higher-order multipole transitions up to $J^{\pi}= 5^{\pm}$. The calculations provide a larger cross sections for $^{40}$Ar.\\ \hline

    Calculation by A. Samana et al.  \cite{Samana2010} & PQRPA &
    Based on projected number QRPA including higher-order multipole transitions up to $J^{\pi}= 6^{\pm}$. These
    calculations were able to describe consistently
    the weak processes on $^{12}$C \cite{Samana2010}
    using a projection number particle procedure.\\ \hline

    Calculation by A. Samana et al. \cite{Samana2008,Barbero2020} & GTBD & Based on Gross Theory of Beta Decay, that describes global properties of $\beta$-decay processes. Refs. \cite{Samana2008,Barbero2020} state that this model for heavy elements overestimated available data. Ref. \cite{Capozzi2018} states that GTBD is less reliable compared to $(p, n)$
    scattering data. \\ \hline

    Calculation by T. Suzuki and M. Honma \cite{Suzuki2013} & NSMRPA or NSM+RPA & Based on a hybrid model calculation where partial cross sections for Fermi and Gamow-Teller transitions obtained using NSM, while other multipoles computed using RPA calculations. \\ \hline

    MARLEY calculation based upon \isotope[40]{Ti} $\beta$ decay data~\cite{marleyCPC} & B 1998 &
    Gamow-Teller matrix elements were extracted from a 1998 measurement by Bhattacharya \textit{et al.}~\cite{Bhattacharya1998}. These are supplemented with QRPA matrix elements from Ref.~\cite{Cheoun2011b} at high excitation energies. \\ \hline

    MARLEY calculation based upon an alternative \isotope[40]{Ti} $\beta$ decay data set~\cite{marleyCPC} & L 1998 & Gamow-Teller matrix elements were extracted from a 1998 measurement by Liu \textit{et al.}~\cite{Liu1998}. These are supplemented with QRPA matrix elements from Ref.~\cite{Cheoun2011b} at high excitation energies. \\ \hline

    MARLEY calculation based upon $(p,n)$ scattering data~\cite{marleyCPC} & B 2009 & Gamow-Teller matrix elements were extracted from a 2009 measurement by Bhattacharya \textit{et al.}~\cite{Bhattacharya2009}. These are supplemented with QRPA matrix elements from Ref.~\cite{Cheoun2011b} at high excitation energies. \\ \hline

    Unpublished calculation by Samana and dos Santos \cite{QRPA_SamanaEtAl} & QRPA-S & Based on QRPA calculations and using the same parametrization of present PQRPA, including higher-order multipole transitions
    up to $J^\pi=6^\pm$.
\end{tabular}
\end{ruledtabular}
\end{table*}

\begin{figure}[H]
    \centering
    \includegraphics[scale=0.35]{CrossSectionVsEnergyExamples.pdf}
    \caption{$\nu_{e}-\isotope[40]{Ar}$ cross section versus energy with various scaling factors applied.
		Ref. \cite{marleyPRC} provided the cross section model obtained from Bhattacharya (2009) data~\cite{Bhattacharya2009}.}
    \label{CrossSectionVsEnergy}
\end{figure}

\subsection{Cross section normalization uncertainty}
\label{section_cross_section_uncertainty}

As a first examination of the impact of cross-section uncertainties on the extraction of supernova flux parameters from a future DUNE data set, we consider model variations that involve the application of a constant overall scaling factor. These variations shift a plot of $\sigma(E_\nu)$ vertically while leaving the shape unchanged (see Fig. \ref{CrossSectionVsEnergy}). We adopt as a reference model a cross section from MARLEY version 1.2.0~\cite{marleyPRC} \footnote{For this paper, the reference cross section was calculated using the \texttt{ve40ArCC\_Bhattacharya2009.react} configuration file}. 

The data-driven nuclear matrix elements in this model were obtained from a measurement of very forward $(p,n)$ scattering reported in Ref.~\cite{Bhattacharya2009}. The unaltered reference model is used together with versions changed by factors of $\pm(5$ to $20)$\% in $5$\% steps, $\pm50$\%, and $+100$\%. This procedure yields a total of twelve unique cross section models, and those models generate different true spectra and grids that we used as input into the forward-fitting algorithm.

\begin{figure}[H]
\includegraphics[scale=0.43]{CrossSectionUncertaintyStudy_NMO_SensitivityRegions.pdf}
\caption{\label{CrossSectionUncertaintyStudy_ExampleSensitivityRegions_Column}Sensitivity regions (90\% C.L.) for a 10~kpc supernova to study different combinations of assumed and true total cross section normalizations.}
\end{figure}

\begin{figure*}
	\centering
	%\setlength{\intextsep}{0mm}%change top white space amount
	\includegraphics[scale = 0.56]{images_updated/CrossSectionUncertaintyStudies_NMO_FracDiff_ConstantColorscale_AxisLabels_ZeroWhite.pdf}%{images_updated/CrossSectionUncertaintyStudies_NMO_FracDiff_ConstantColorscale_AxisLabels.pdf}
	\caption{2D fractional difference plots to study effects produced by normalization uncertainties on the total cross section.}
	\label{CrossSectionUncertainty_FractionalDifference}
\end{figure*}

Figure \ref{CrossSectionUncertaintyStudy_ExampleSensitivityRegions_Column} shows sensitivity regions for a 10~kpc supernova, the true scenario outlined in Sec.~\ref{sec:study_assumptions}, and three different sets of assumptions.  The sensitivity regions shift for changes in $\varepsilon$; the cross section scaling factors affect the statistics and thus $\varepsilon$. The sensitivity regions shift vertically for change in cross-section normalization, with near-negligible shape change, as expected.

Figure~\ref{CrossSectionUncertainty_FractionalDifference} shows the bias in the best-fit parameter values for each possible combination of true cross-section model (i.e., the model used to simulate the fake data set) and assumed cross-section model (i.e., the model used to perform the parameter fits). The best fit within the grid bounds is determined, and that constraint can introduce an artificial bias to the best fit once a boundary is reached for one or more parameters. The
results are shown separately for $\alpha$, $\left<E_\nu\right>$, and
$\varepsilon$. For each parameter, a two-dimensional histogram is plotted in
which each bin represents a particular combination of cross-section models. The
color of the bin represents the bias value, i.e., the fractional difference between the best-fit
parameter value and its true value. %For all 2D bias plots, the color scale ranges are fixed to reflect the possible bias values for the individual spectral parameters. 

We first notice that the biases on $\alpha$ and $\langle E_\nu \rangle$ are relatively small unless the assumptions significantly differ from reality. If we assume an enhanced cross section (using positive scaling factors), the large mismatch in statistics causes an $\varepsilon$ under-estimation. The difference in statistics forces the algorithm to select lower $\varepsilon$ values. If we assume a reduced cross section (using negative scaling factors), we expect a lower event rate than we actually observe; thus the forward-fitting algorithm prefers higher $\varepsilon$ values to compensate for the discrepancy. When the algorithm reaches a boundary (i.e., at the minimum or maximum $\varepsilon$ value allowed), the biases in $\alpha$ and $\langle E_\nu \rangle$ will increase to compensate for spectral shape differences between the true spectrum and grid elements.

\subsection{Physics content of the reference cross-section models}
\label{section_cross_section_models}

As mentioned above,
Table~\ref{xscn_description_table} summarizes the features of the cross-section models used in this
work. The models include those based on
microscopic formalisms such as
RPA~\cite{Gil-Botella2003,Kolbe_2003},
QRPA~\cite{Cheoun2011b},
PQRPA~ \cite{Samana2010},
RQRPA~\cite{Paar2013},
and
NSM+RPA~\cite{Suzuki2013};
macroscopic models such as GTBD \cite{Samana2008,Barbero2020};
and the MARLEY~\cite{marleyCPC} phenomenological calculation based on a Monte Carlo approach.  In the absence of any direct measurements of charged-current neutrino-argon scattering in the relevant energy range, experimental constraints on these theoretical approaches are poor.
Nevertheless, we can make some general observations about the physics content of these models.

First, all of the microscopic models used here employ different residual interactions. These include the Skyrme interaction (including a spin-orbit term) in the RPA calculation, the Bonn CD potential in QRPA, the $\delta$-interaction in PQRPA, the DDME2 relativistic nuclear energy density functional in RQRPA, and the monopole-based-universal interaction (VMU) in NSM+RPA. The choice of residual interaction in each case was motivated by a successful description of some relevant experimental data, such as Gamow-Teller (GT) strengths, $\beta$-decay rates, or energies of odd-odd neighboring nuclei.

Second, using a sufficiently large configuration space of nucleon states is important to prevent underestimation of the energy-dependent total cross section $\sigma(E_\nu)$ as the neutrino energy rises. This is due in part to the increasing contribution of higher-order multipoles at high energies. The inclusive or total cross section as function of neutrino energy
is a sum over all nuclear multipoles states: 
%\TODO{We also need terms in the sum for $0^-$ and %$1^-$, right?}
\begin{eqnarray}
\sigma(E_\nu) &=&
\sigma(E_\nu,0^+) + \sigma (E_\nu,1^+)
\nonumber\\
&+&\sigma(E_\nu,0^-) + \sigma (E_\nu,1^-)+
\sum_{J^{\pi} \, \geq \, 2^{\pm}}^{J^\text{max}}
\sigma(E_\nu, J^{\pi})\,.
\end{eqnarray}
Here, $\sigma(E_\nu, J^{\pi})$ is the cross section contribution due to multipole $J^{\pi}$; for example, see Eq.~2.25 in Ref.~\cite{Samana:2010up}, or Eq.~3 in Ref.~\cite{Suzuki2013} for integration over neutrino angle.
Usually, the contribution of the multipoles
$0^+$ and $1^+$, allowed transitions, are the
most important below neutrino energies of 50 MeV.
Previous work with PQRPA and RQRPA on
$(\nu/ \bar{\nu})$ reactions on $^{12}$C has examined
the variation of $\sigma(E_\nu)$ as a function of the space of
single particle energies and the chosen value of the multipole cutoff $J^\text{max}$~\cite{Samana:2010up}. It was found that the magnitudes of the resulting cross sections were close to the sum-rule limit at low energies but significantly smaller than this limit at high energies. As the size of the configuration space is augmented, $\sigma(E_\nu)$ increases steadily, particularly for $(\nu/ \bar{\nu})$ energies greater than 200~MeV.
Convergence is achieved when the configuration space
and multipole cutoff ($J^\text{max}$) are both chosen to be sufficiently large~\cite{Samana:2010up}.

A few words are necessary for the GTBD result. This is a parametric model
for $\beta$-decay rates, which includes statistical arguments in a phenomenological way through a convolution between the independent particle model
$\beta$-amplitude and the level density of the Fermi gas model corrected to take into account shell effects. The GTBD calculation considers only the contributions of allowed transitions, $\sigma (E_\nu,0^+)$
and $\sigma (E_\nu,1^+)$, with a realistic description of the energy of the GT resonance peak~\cite{Samana2008,Barbero2020}.

Third, some calculations use an effective (or \textit{quenched}) value of the nucleon axial-vector
coupling constant for which its bare value $g_{\mbox{\tiny A}}=1.2756$ from the experimental
data~\cite{pdg} is multiplied by a factor of around 0.8. There is still a lack of consensus in the nuclear physics community about whether this quenching is needed. For the family of models considered in this paper, the RPA calculations
do not use a re-normalization of $g_{\mbox{\tiny A}}$~\cite{Kolbe_2003}, while the RQRPA model used $g_{\mbox{\tiny A}}= 1$. The
PQRPA calculations also adopted $g_{\mbox{\tiny A}}=1$
to be consistent with comparisons of 2s1d and 2p1f shell- model  predictions with measured allowed $\beta$-decay rates~\cite{Samana:2010up} and with recent double beta decay calculations.
The QRPA calculations reported in Ref.~\cite{Cheoun2011b} use a universal quenching factor
$f_q = g^\text{eff}_{\mbox{\tiny A}} / g_{\mbox{\tiny A}}= 0.74$ to reproduce
measured GT strength distributions.
The NSM+RPA calculations within the VMU potential used a similar quenching factor $f_q=0.775$ with
$g_{\mbox{\tiny A}}=1.263$. This choice enabled the NSM+RPA model to describe
the experimental cumulative sum of the GT strength rather well.
On the other hand, recent studies on variations
of $g_{\mbox{\tiny A}}$ in the GTBD have shown that
best results for a set of 94 nuclei of interest
%in pre-supernova phase,
are obtained with $g_{\mbox{\tiny A}}=1$~\cite{Possidonio2018}. The GT distribution used for the NSM+RPA calculation is shifted toward higher energy values with significantly smaller strengths for $<$10~MeV neutrino energies, resulting in a characteristic cut-off at energies below about 8~MeV.

Despite the differences explored above, the main features of measured weak interaction observables, such as $\beta$-decay strengths and inclusive muon capture rates, are reasonably well described for multiple nuclei by the majority of the nuclear structure models considered herein. By incorporating these cross-section models into our SNOwGLoBES calculations, we studied the impact of variations in the shape of $\sigma(E_\nu)$ on the simulated measurements of supernova neutrino flux parameters. Many of the cross section models required re-formatting with extra data-points for usage in SNOwGLoBES; appendix~\ref{xscn_appendix} provides more details on the interpolation procedure that was used. Figures~\ref{CrossSectionModels_RPA_Paper}~and~\ref{CrossSectionModels_MARLEY_GTBD_Paper} show that the cross-section models differ considerably and lead to a wide range of predictions for the supernova $\nu_{e}$ signal in DUNE. Appendix~\ref{appendix_eventrates} provides a table of the corresponding event rates as output by SNOwGLoBES.

\begin{figure}[H]
  \centering
    \includegraphics[scale=0.42]{TestSpectra_CrossSectionCalcs_nueOnly_NMO.pdf}
    \caption{SNOwGLoBES event rates for select cross-section
      calculations from
      Refs. \cite{marleyPRC,Samana2008,Barbero2020,Suzuki2013,Cheoun2011b,QRPA_SamanaEtAl}. Note
      that ``QRPA-C'' and ``QRPA-S'' contain the same type of
      calculation performed by different groups, with the former by
      M. Cheoun et al. \cite{Cheoun2011b} and the latter by Samana and
      dos Santos \cite{QRPA_SamanaEtAl}. More details about the
      various models are provided in
      Table~\ref{xscn_description_table}. The error bars are statistical.
    }
    \label{TestSpectra_CrossSectionModels}
\end{figure}

Figure~\ref{TestSpectra_CrossSectionModels} shows representative expected event
rates in DUNE for the CC $\nu_e$-$\isotope[40]{Ar}$ absorption process and a supernova at a
distance of 10~kpc from Earth. The large differences in the
cross-section model predictions at low neutrino energy translate to large
variations in the plotted observed energy distributions. Apart from effects of
cross-section mismodeling (which are considered in the next section), the
expected statistical uncertainty on the event rate has a strong effect on the
precision with which the supernova flux parameter values may be measured. The
sensitivity regions shown in
Fig.~\ref{CrossSectionModelStudies_ExampleSensitivityRegions} are obtained by
considering the statistical uncertainty and using the same cross-section model
to generate the fake data and extract the results. The GTBD cross section
model, which predicts 7770 $\nu_{e}$CC events, results in the tightest constraints
on the flux parameters. The QRPA-C model predicts 1383 events and thus provides
the loosest constraints.

\begin{figure}[H]
  \includegraphics[scale=0.4]{CrossSectionModelStudies_NMO_ExampleSensitivityRegions.pdf}
  \caption{Sensitivity regions (90\% C.L.) in $(\langle E_\nu \rangle, \varepsilon)$
space generated from the cross-section models in
Refs.~\cite{marleyPRC,Samana2008,QRPA_SamanaEtAl}. Only statistical
uncertainties are considered. In each case, the same cross-section model is
used both to produce the fake data and to calculate the sensitivity region.}
  \label{CrossSectionModelStudies_ExampleSensitivityRegions}
\end{figure}

\subsection{Combined cross-section normalization and shape uncertainty}
\label{section_xsec_shape}

To characterize the impact of using an inaccurate cross-section model to
extract values of the supernova flux parameters, we consider scenarios in
which different combinations of the theoretical models described in
Sec.~\ref{section_cross_section_models} are used to
\begin{enumerate*}[label=(\arabic*)] \item simulate a fake data set, and \item
perform fits of the flux parameters. \end{enumerate*} Figure~\ref{CrossSectionModels_FractionalDifference} displays the 2D bias plots for the different combinations of assumed and true total cross section models. A logarithmic color scale is used for
$\varepsilon$ due to the very large range of biases
allowed for that parameter. In the 2D plots, the cross-section models are ordered along each
histogram axis from smallest to largest expected number of events integrated
over a neutrino energy range of [5, 15]~MeV. Appendix~\ref{appendix_eventrates} also contains the numerical values for the expected event counts for each model in the [5, 15]~MeV range.

Further insight into cross-section model effects on the extraction of
supernova neutrino flux parameters can be gained from
Fig.~\ref{CrossSectionModelStudies_SensitivityRegions_Combos}, which shows
sensitivity regions computed based on a fake data set produced using the
MARLEY B~2009 cross-section model. When supernova flux parameters are
extracted using the same cross-section model (red sensitivity regions), the
best-fit values (red stars) are identical to the true ones by construction. A
small bias is seen when the extraction procedure is repeated using the MARLEY
L~1998 model (black stars). However, the difference between the assumed
(L~1998) and true (B~2009) cross sections is small enough that the gray
sensitivity regions obtained from the new fit cover the true parameter values
in all cases. A more problematic bias (green stars) is seen when the fit is
repeated using the PQRPA model as the assumed cross section. In this case, the
difference between the PRQPA and MARLEY B~2009 predictions is large enough to
lead to green sensitivity regions which do not enclose the true results. This
bias would need to be corrected in the context of a real analysis by
introducing a cross-section-related systematic uncertainty to inflate the
sensitivity regions. The significant corresponding loss of precision can be
visually estimated from
Fig.~\ref{CrossSectionModelStudies_SensitivityRegions_Combos} by examining the
degree to which the green sensitivity regions ``miss'' the red star that
represents the true parameter values.

Some general trends were seen in the course of these fake data studies. If the
cross-section model used for fitting gives higher values than the true one used to
generate the fake data, then the fitting algorithm tends to overestimate
$\alpha$ and $\langle E_\nu \rangle$ while underestimating $\varepsilon$.
Because $\varepsilon$ is directly proportional to the expected number of
events, the best-fit value of $\varepsilon$ is driven lower for fake data sets
with low statistics.

\newpage

\begin{figure*}
  \centering
  \setlength{\intextsep}{0mm}%change top white space amount
  \includegraphics[scale = 0.5]{images_updated/CrossSectionModelStudies_NMO_FracDiff_ConstantColorscale_AxisLabels_ZeroWhite.pdf}
  \caption{2D fractional difference plots to study effects produced by different cross section models. Note that ``S'' stands for the cross section model implemented into SNOwGLoBES \cite{snowglobes}. Also note that the $\varepsilon$ color-scale is log to account for the wide range of values. The number scale shows the raw fractional difference values to conform with the $\alpha$ and $\langle E_\nu \rangle$ plots.}
  \label{CrossSectionModels_FractionalDifference}
\end{figure*}

\begin{figure*}
  \centering
  \includegraphics[scale=0.75]{CrossSectionModelStudies_NMO_SensitivityRegions_Combo.pdf}
  \caption{Sensitivity regions (90\% C.L.) calculated with different assumed cross-section
models for a fake data set generated using the MARLEY B~2009 model. The stars
mark the best-fit measurements from the fitting algorithm. The red stars also
indicate the true parameter values, i.e., when the assumed cross section model is identical to the true model.}
  \label{CrossSectionModelStudies_SensitivityRegions_Combos}
\end{figure*}

\newpage

\subsection{Total cross section uncertainty envelope}
\label{section_cross_section_envelope}

\begin{figure*}
  \centering
  \begin{minipage}[]{0.49\textwidth}
    \includegraphics[width=\textwidth]{CrossSectionEnvelopeModels.pdf}
    \caption{Total cross section predictions for the $\nu_{e}-\isotope[40]{Ar}$ interaction from the selected subset of
models discussed in Section~\ref{section_cross_section_envelope}. The shaded region represents the adopted
uncertainty envelope based on the spread of these models.}
    \label{CrossSectionEnvelopeModels}
  \end{minipage}
  \hfill
  \begin{minipage}[]{0.49\textwidth}
    \includegraphics[width=\textwidth]{CrossSectionEnvelopeToyModels.pdf}
    \caption{Toy total cross-section models for the $\nu_{e}-\isotope[40]{Ar}$ interaction covering portions of the uncertainty envelope shown in Fig.~\ref{CrossSectionEnvelopeModels}.}
    \label{CrossSectionEnvelopeToyModels}
  \end{minipage}
\end{figure*}

The cross-section models considered above are not expected to produce results
of equal quality in the energy region of interest for supernova neutrinos
(see, e.g., the discussion in the supplemental materials from
Ref.~\cite{Capozzi2018}), and furthermore, uncertainties are typically not available for them. As a means of assigning a theoretical uncertainty
which neglects implausibly extreme variations, we consider the spread between
three cross-section predictions: the partially data-driven MARLEY
models~\cite{marleyPRC}, the NSM+RPA calculation~\cite{Suzuki2013}, and the
QRPA-S calculation~\cite{QRPA_SamanaEtAl}. In the absence of a direct
measurement of the $\nu_e$ capture process on argon, we selected this subset of the available models based upon purely a priori
considerations. Predictions from our chosen subset of cross-section models are
shown in Fig.~\ref{CrossSectionEnvelopeModels}. An uncertainty envelope
defined as the range between the minimum and maximum cross-section predictions
from this subset of models is also shown as the crosshatched region. Predicted
supernova neutrino event rates in DUNE for each of the models used to define
the envelope are displayed in Fig.~\ref{TestSpectra_ReliableModels}.

\begin{figure*}
  \centering
  \includegraphics[scale=0.44]{images_updated/TestSpectra_ReliableModels.pdf}
  \caption{SNOwGLoBES event rates for the selected cross-section
calculations discussed in the text. The error bars are statistical.}
  \label{TestSpectra_ReliableModels}
\end{figure*}

With a restricted range of cross-section variations defined in this way, we
repeated our fake data studies using a new family of toy cross section models.
The lower (Min) and upper (Max) bounds of the uncertainty envelope were
treated as two of the new models, and the MARLEY B~2009 cross
section~\cite{marleyCPC} was treated as a midpoint. We further define four
additional toy models in which three of the models attempt to cover the
lower half of the envelope. The first toy model (``Lower bound toy model 1'') is an average between the MARLEY B~2009 cross section and the lower (Min) bound. The second toy model (``Lower bound toy model 2'') is defined as the average between the first toy model and the MARLEY B~2009 cross section. Finally, the third toy model (``Lower bound toy model 3'') is defined as the average between the first toy model and the lower (Min) bound.
The complete
set of toy cross section models is shown in
Fig.~\ref{CrossSectionEnvelopeToyModels}. Note that the two ``kinks'' in the
Min model are artifacts from linear interpolations of the
NSM+RPA~\cite{Suzuki2013} and QRPA-S~\cite{QRPA_SamanaEtAl} models,
respectively.

Figure \ref{CrossSectionEnvelope_FractionalDifference} shows the 2D fractional
difference plots for the toy cross section models within the uncertainty
envelope. When compared to Fig. \ref{CrossSectionModels_FractionalDifference},
the biases are less extreme for all three parameters. Similar to the previous
fake data studies, extraction of best-fit values for $\alpha$ and $\langle
E_\nu \rangle$ is less affected by cross-section mismodeling while estimation
of $\varepsilon$ is impacted the most. Also similar to the previous studies,
assuming a cross-section higher than the true one leads to an underestimation
of $\varepsilon$. Example sensitivity regions are shown in
Fig.~\ref{CrossSectionEnvelope_SensitivityRegions} using several assumed cross
sections for fake data generated using the MARLEY B~2009 model. In this case,
the black star represents the true parameter values. The observed biases are
still significant for $\varepsilon$ but relatively modest for the other
supernova flux parameters.

\newpage
\begin{figure*}
	\centering
	\setlength{\intextsep}{0mm}%change top white space amount
	\includegraphics[scale = 0.53]{images_updated/CrossSectionUncertaintyEnvelope_NMO_FracDiff_ConstantColorscale_AxisLabels_ZeroWhite.pdf}
	\caption{2D fractional difference plots to study effects produced by toy models within the cross section uncertainty envelope discussed in Section~\ref{section_cross_section_envelope}.}
	\label{CrossSectionEnvelope_FractionalDifference}
\end{figure*}

\begin{figure*}
	\centering
	\setlength{\intextsep}{0mm}%change top white space amount
	\includegraphics[scale = 0.78]{CrossSectionUncertaintyEnvelope_NMO_ExampleSensitivityRegions.pdf}
        \caption{Sensitivity regions (90\% C.L.) with different combinations of assumed
and true cross section models. Two of the models are toy models generated from
the midpoint (B~2009) and minimum cross section values from the set of selected
models. The stars mark the best-fit values from the fitting algorithm. The
black stars also represent the true parameter values.}
	\label{CrossSectionEnvelope_SensitivityRegions}
\end{figure*}
\newpage

\section{Discussion}
\label{discussion}

A proper
interpretation of a DUNE supernova neutrino data set will require a good
understanding of neutrino-argon scattering cross sections in the tens of MeV
regime. 
Since direct measurements of the dominant charged-current $\nu_e$ absorption
process on argon are currently unavailable, our present consideration of
cross-section uncertainties necessarily relies on calculations available in
the theoretical literature. Furthermore, because few published calculations of
observables beyond energy-dependent total cross sections $\sigma(E_\nu)$ are
available for CC $\nu_e$-\isotope[40]{Ar} scattering, we focus entirely upon
variations to the total cross section. For the studies reported here, the
remaining aspects of the interaction modeling needed to connect the true
neutrino energy to the observed energy distribution in DUNE are provided by
the MARLEY event generator, which currently implements the only realistic
predictions of complete final states for low-energy CC neutrino-argon
scattering. We expect the theoretical uncertainties on these additional
modeling details to be significant, and future work will be needed to reliably
quantify them.

To examine the impact of total cross-section mismodeling on the interpretation
of DUNE supernova neutrino data, we employed three strategies for model
variations: applying a constant scaling factor to the MARLEY B~2009 model
(Sec.~\ref{section_cross_section_uncertainty}), considering the full range of
a variety of cross-section predictions (Sec.~\ref{section_xsec_shape}), and
defining an uncertainty envelope based on the spread of a subset of
selected predictions
(Sec.~\ref{section_cross_section_envelope}). Beyond the phenomenological
models available in MARLEY, the theoretical calculations that we reviewed and
employed for the latter two strategies included the global GTBD treatment and
microscopic evaluations such as the QRPA, PQRPA, NSM, and hybrid approaches.
All of these models have significant differences coming from the description
of nuclear correlations, the residual interaction, and the value of the
nucleon axial-vector coupling. Nevertheless, these models reasonably describe
the main features of measured weak interaction observables such as
$\beta$-decay strengths and inclusive muon capture rates.

For all three strategies, the cross-section model variations were applied
to toy measurements of supernova neutrino flux parameters performed using
fake data sets produced using the SNOwGLoBES framework. Different combinations
of true and assumed cross section models (used to create the fake data and
interpret the toy measurement results, respectively) were employed, and
the impact on the extracted values of the flux parameters was assessed.

Table~\ref{tab:biases} provides a high-level summary of the conclusions from
our fake data studies. For each of the three supernova neutrino flux
parameters that we considered, an uncertainty on the total CC neutrino-argon
cross-section of -50/+100\% and $\pm$20\% is translated into a corresponding
range of observed biases on the best-fit parameter value extracted from the
toy measurements. The values of the bias were read directly off the 2D fractional difference plots. For the -50/+100\% combination, the forward-fitting algorithm reached the most extreme allowed values of $\varepsilon$, causing the biases in $\alpha$ and $\langle E_{\nu} \rangle$ to increase in an attempt to compensate for the spectral shape differences between the true spectrum and grid elements. 

\begin{table}[b]
\caption{\label{tab:biases}
Parameter biases caused by normalization uncertainties on the total cross section.}
\begin{ruledtabular}
\begin{tabular}{ccc}
$\sigma(E_\nu)$ uncertainty & Parameter & Measurement bias \\
\colrule
\multirow{3}{*}{-50/+100\%} & $\alpha$ & -80\% to +176\% \\
  & $\langle E_{\nu} \rangle$ & -41.1\% to +47.4\% \\
  & $\varepsilon$ & -60\% to +100\% \\[3mm]

\multirow{3}{*}{$\pm$20\%} & $\alpha$ & 0\% to +8\% \\
  & $\langle E_{\nu} \rangle$ & -3\% to 0\% \\
  & $\varepsilon$ & -45\% to +50\% \\
\end{tabular}
\end{ruledtabular}
\end{table}

For total cross section known at about the 20\% level, bias on best-fit $\alpha$ and $\langle E_\nu\rangle$ is in the 3-8\% range.
Achieving less than 10\% 
bias on the best-fit value of $\varepsilon$ requires the cross section to
be known to about 5\%. These requirements may be somewhat relaxed in light of
possible constraints from simultaneous observations of the supernova by other
detectors, which we do not consider here. On the other hand, more stringent
requirements may ultimately be needed when additional interaction modeling
uncertainties (beyond those on the total cross section) are fully taken into
account.

While we are optimistic that the theoretical understanding of low-energy
neutrino-argon cross sections will continue to improve, there is no substitute
for actually measuring the cross sections with a well-characterized neutrino
flux. Pions decaying at rest represent a near-ideal source of neutrinos for
such measurements. Decays of $\pi^+$ produce monochromatic $\nu_\mu$ on a
short timescale, plus $\bar{\nu}_\mu$ and $\nu_e$ from delayed decay of the
stopped daughter muon on a 2.2~$\mu$s timescale. The spectrum and timing are
very well understood. The neutrino energies extend to $52$~MeV, overlapping
nicely with the supernova spectrum. It is also possible to study neutral-current argon inelastic events given the time structure of the beam. Spallation-based neutron beams such as the
Spallation Neutron Source at Oak Ridge National Laboratory~\cite{Barbeau2021},
the Lujan Neutron Science Center at Los Alamos National
Laboratory~\cite{AguilarArevalo2021}, the J-PARC Spallation Neutron
Source~\cite{Ajimura2021}, and the future European Spallation
Source~\cite{Baxter2021} (currently under construction) are intense sources of
pion decay-at-rest neutrinos. Measurements of these neutrinos may also be
possible at high-energy physics facilities including the Large Hadron Collider
beam dump~\cite{Kelly2021} and the meson decay-in-flight neutrino beams at
Fermilab~\cite{Grant2016}.

Future direct measurements of CC $\nu_e$-argon cross sections using a pion
decay-at-rest source could pursue several distinct observables to better
constrain interaction modeling uncertainties for the DUNE supernova neutrino
program. The most straightforward of these (and most directly relevant to the
specific uncertainties considered in this paper) would be an inclusive total
cross section $\left<\sigma\right>$ averaged over the $\nu_e$ flux
$\phi(E_\nu)$ from $\pi^{+}$ decays at rest:
\begin{equation}
\left<\sigma\right> \equiv \frac{\int_{0}^{m_\mu/2}
\sigma(E_\nu) \, \phi(E_\nu) \, dE_\nu}
{ \int_{0}^{m_\mu/2} \phi(E_\nu) \, dE_\nu } \,,
\end{equation}
where $m_\mu$ is the muon mass and
\begin{equation}
\phi(E_\nu) \propto E_\nu^2 \, m_\mu^{-4} (m_\mu - 2E_\nu) \,.
\end{equation}
Measurements of both $\left<\sigma\right>$ and a differential cross section as
a function of the total visible energy would likely be obtainable with a
suitably large (several-ton-scale) argon detector.  As an example, 5-10\% statistical uncertainty on the total cross section could be obtained in a few years with a ton-scale detector a few tens of meters from the Spallation Neutron Source. 

The fine spatial resolution of a LArTPC detector would potentially allow for
more detailed measurements. In particular, topological separation between the
outgoing electron and $\gamma$-rays emitted due to neutrino-induced nuclear
de-excitations could allow separate measurements of differential distributions
for both particle species. Recent studies (e.g., Ref.~\cite{Castiglioni2020})
suggest that such a separation would be feasible, and a successful
implementation would yield a rich data set: the inclusive electron energy and
angular distributions are known to be sensitive to the modeling of forbidden
contributions to the cross section~\cite{VanDessel2020}, while the
$\gamma$-rays would provide a helpful constraint on de-excitation modeling
and, in principle, the opportunity to measure partial cross sections for
specific nuclear transitions. Measuring the neutrino angular distribution is particularly important for supernova pointing measurements relevant for prompt multi-messenger astrophysics~\cite{Bueno:2003ei,Abi:2020evt}.

An especially impactful but highly challenging measurement would involve the
detection of final-state neutrons produced by CC $\nu_e$-argon interactions.
Missing energy attributable to these neutrons is expected to have a
significant impact on neutrino energy reconstruction at supernova
energies~\cite{marleyPRC}, and the modeling needed to account for it is
complicated and poorly constrained by experimental data. In the absence of any
new experimental techniques to increase the sensitivity of argon-based
detectors to neutrons at and below MeV energies, external instrumentation
designed to capture and detect escaping neutrons would likely be the only
means of attempting such a measurement.


\section{Conclusion}
\label{conclusion_section}

A possible future observation by DUNE of neutrinos from a core-collapse
supernova would represent a rare and valuable scientific opportunity. In
particular, the unique sensitivity of DUNE's LArTPC detectors to the $\nu_e$
component of the supernova neutrino flux would be highly complementary to
other current and anticipated large neutrino experiments. 
In the studies reported in this paper, we have examined the effects of
cross-section modeling uncertainties on a simulated analysis of supernova
neutrinos in DUNE.

Significant experimental and theoretical challenges remain before a precise
understanding of tens of MeV neutrino-argon scattering can be achieved.
Nevertheless, pursuing this understanding will be essential to maximize the
discovery potential from a core-collapse supernova observation (and a
potentially broader program of low-energy physics) in DUNE. We hope that the
initial studies of neutrino-argon interaction modeling uncertainties reported
here may serve as a useful foundation for the more comprehensive investigations
that will be required in the future.

%TC:ignore
\section{Acknowledgements}

This document was prepared by the DUNE collaboration using the
resources of the Fermi National Accelerator Laboratory 
(Fermilab), a U.S. Department of Energy, Office of Science, 
HEP User Facility. Fermilab is managed by Fermi Research Alliance, 
LLC (FRA), acting under Contract No. DE-AC02-07CH11359.

This work was supported by
CNPq,
FAPERJ,
FAPEG and 
FAPESP,                         Brazil;
CFI, 
IPP and 
NSERC,                          Canada;
CERN;
M\v{S}MT,                       Czech Republic;
ERDF, 
H2020-EU and 
MSCA,                           European Union;
CNRS/IN2P3 and
CEA,                            France;
INFN,                           Italy;
FCT,                            Portugal;
NRF,                            South Korea;
CAM, 
Fundaci\'{o}n ``La Caixa'',
Junta de Andaluc\'ia-FEDER,
MICINN, and
Xunta de Galicia,               Spain;
SERI and 
SNSF,                           Switzerland;
T\"UB\.ITAK,                    Turkey;
The Royal Society and 
UKRI/STFC,                      United Kingdom;
DOE and 
NSF,                            United States of America.
%

This work was also supported by FAPESB T.O. PIE 0013/2016 and UESC/PROPP 0010299-61.
%TC:endignore

\appendix

  \section{Interpolation/extrapolation methods used on cross section models}
  \label{xscn_appendix}

In order to study the measurement biases introduced by the cross-section
modeling, we obtained numerical tables of model predictions for the total
charged-current $\nu_{e}$-$^{40}\text{Ar}$ cross section (see
Table~\ref{xscn_description_table}). SNOwGLoBES requires $1001$ data points in
a cross section file for neutrino energies between $5$-$100$ MeV. While some
of the models of interest are already available within SNOwGLoBES
(including its default cross section model, along with some MARLEY cross
section models from Ref.~\cite{marleyPRC}), input files for the other models
required extra preparation to conform to the requirements of the SNOwGLoBES
format.

Table~\ref{xscn_table} summarizes the interpolation and extrapolation methods
used for the various models. Excluding the cross section models already
available within SNOwGLoBES, all models required interpolation between their
tabulated data points to obtain cross section values at intermediate neutrino
energies. For models which were tabulated over the entire energy range of
interest, either a cubic spline or a linear spline was used to interpolate
between the given data points. A cubic spline was generally preferred, but the
linear spline was used in cases where the cubic spline caused
unphysical fluctuations in the interpolated total cross section.

The available cross-section tables for some models did not cover the entire
5--100~MeV energy range required by SNOwGLoBES. In such cases, extrapolation
techniques were used to extend the existing predictions. The models from
Refs.~\cite{Samana2008,Barbero2020,Suzuki2013,Gil-Botella2003,Kolbe_2003}
required extrapolation down to 5~MeV, while the model from
Ref.~\cite{Cheoun2011b} required extrapolation down to 5~MeV and up to
100~MeV. All of the extrapolations used to prepare the SNOwGLoBES input files
employed a quadratic fit of the form \begin{equation} \sigma(E_\nu) = p_0
(E_\nu - p_1 )^2 \end{equation} where $p_0$ and $p_1$ are the free parameters
used for fitting. All extrapolation fits used five data points. 

In the fits for low energies, $p_1$ (which has units of MeV) holds special
significance as the ``endpoint'' of the cross section model because it is the
minimum of the quadratic function. For $p_1 > 5$~MeV, the fit would introduce
unphysical behavior into the model in the form of an increasing cross section
as the neutrino energy $E_\nu$ approaches 5~MeV from above. To prevent this
behavior, the total cross section $\sigma(E_\nu)$ was zeroed out for all
energies $E_\nu < p_1$ whenever $p_1 > 5\text{ MeV}$. The same quadratic
functional form was also fit to the last five data points of the model from
Ref.~\cite{Cheoun2011b} to extrapolate up to 100~MeV. In this case, the low-
and high-energy fits were handled independently. In order to avoid
discontinuities between the interpolation and extrapolation methods, the fits
performed at low (high) neutrino energy were required to pass through the
first (last) tabulated data point for the cross-section model of interest.
Fig.~\ref{GTBD_SNOwGLoBESvsRaw} shows the cross section model from
Refs.~\cite{Samana2008,Barbero2020} as an example of the interpolation between
points (in this case, with a linear spline) as well as an extrapolation to low
energies.

\begin{figure*}
  \centering
  \includegraphics[scale = 0.25]{GTBD_SNOwGLoBESvsRaw.png}
  \caption{Cross section model from Ref. \cite{Samana2008,Barbero2020} with the
interpolation (with a linear spline) and extrapolation (using a quadratic fit)
shown. See Table~\ref{xscn_table} for the quadratic fit parameters for the
low-energy fit.}
  \label{GTBD_SNOwGLoBESvsRaw}
\end{figure*}

\begin{table*}
  \caption{\label{xscn_table}Table summarizing the interpolation and
extrapolation methods performed on the various cross section models to format
them for usage in SNOwGLoBES~\cite{snowglobes}. Parameters from the quadratic
fits described in the text are also given when extrapolation was used.}
  \begin{ruledtabular}
  \begin{tabular}{cp{5cm}p{9cm}}
    Cross section model & Interpolation method used & Extrapolation method used \\ \hline

    SNOwGLoBES \cite{snowglobes} & N/A & N/A \\ \hline

    RPA \cite{Gil-Botella2003,Kolbe_2003} & Linear spline & Low-energy quadratic fit: $\sigma = 1.35027\text{e-}05(E - 0.567063)^2$ \\ \hline

    QRPA-C \cite{Cheoun2011b} & Linear spline & Low-energy quadratic fit: $\sigma = 7.29830\text{e-}06(E - 6.67699)^2$; for all energy values below $p_1 = 6.68$ MeV, the cross section was set to zero. \newline
    High-energy quadratic fit: $\sigma = 1.83273\text{e-}05(E - 12.3510)^2$ \\ \hline

    GTBD \cite{Samana2008,Barbero2020} & Linear spline & Low-energy quadratic fit: $\sigma = 2.26358\text{e-}05(E + 0.761242)^2$ \\ \hline

    NSM+RPA \cite{Suzuki2013} & Linear spline & Low-energy quadratic fit: $\sigma = 1.49812\text{e-}04(E - 7.45969)^2$; for all energy values below $p_1 = 7.46$ MeV, the cross section was set to zero. \\ \hline

    QRPA-S \cite{QRPA_SamanaEtAl} & Linear spline & N/A \\ \hline

    RQRPA \cite{Paar2013} & Cubic spline & N/A \\ \hline

    PQRPA \cite{Samana2010} & Cubic spline & N/A \\ \hline

    B 1998 \cite{marleyCPC} & Cubic spline & N/A \\ \hline

    B 2009 \cite{marleyCPC} & Cubic spline & N/A \\ \hline

    L 2009 \cite{marleyCPC} & Cubic spline & N/A \\ \hline

\end{tabular}
\end{ruledtabular}
\end{table*}

\section{SNOwGLoBES event rates for different cross section models}
 \label{appendix_eventrates}

 \begin{table}[H]
\caption{\label{xscn_event_rates_table} SNOwGLoBES estimated number of
$\nu_{e}$CC events in the DUNE far detectors for pinched-thermal flux parameters
$(\alpha, \langle E_\nu \rangle, \varepsilon) = (2.5, 9.5, 5\times 10^{52})$ for the $\nu_e$ flavor, a
10-kpc supernova, and assuming NMO and MSW oscillations via Equation~\ref{msw-oscillations}.}
\begin{ruledtabular}
\begin{tabular}{p{3cm}p{2.5cm}p{2.5cm}}
\hline
Cross section model & Number of $\nu_{e}$CC events & Number of $\nu_e$CC events between $[5, 15]$ MeV \\ \hline
QRPA-C \cite{Cheoun2011b} & 1383 & 134\\ \hline
RQRPA \cite{Paar2013} & 2243 & 220\\ \hline
QRPA-S \cite{QRPA_SamanaEtAl} & 2791 & 243\\ \hline
SNOwGLoBES \cite{snowglobes} & 4486 & 624\\ \hline
B~1998 \cite{marleyCPC} & 6307 & 874\\ \hline
L~1998 \cite{marleyCPC} & 6390 & 883\\ \hline
NSM+RPA \cite{Suzuki2013} & 6391 & 897\\ \hline
B~2009 \cite{marleyCPC} & 6852 & 988\\ \hline
PQRPA \cite{Samana2010} & 4562 & 909\\ \hline
RPA \cite{Gil-Botella2003,Kolbe_2003} & 5064 & 998\\ \hline
GTBD \cite{Samana2008,Barbero2020} & 7770 & 2070 \\\hline
\end{tabular}
\end{ruledtabular}
\end{table}

\section{Interpolating sensitivity regions}

To keep computation time reasonable, the algorithm used to compute flux parameter sensitivity regions (see
Sec.~\ref{sec:forward_fitting}) uses a limited number of  elements in the grid of reference
$(\alpha, \left<E_\nu\right>, \varepsilon)$ values. The limited number of grid
elements leads to unphysical jagged edges in plots of the 90\% confidence
contours used in this paper to estimate DUNE sensitivity regions for the
supernova spectral parameters. To remove these artifacts from the sensitivity
region plots, we developed an interpolation technique to smooth the contour
edges. Each contour was stored as a two-dimensional histogram, where the
weight in each bin was calculated as the minimum $\chi^2$ value obtained in
that region of 2D flux parameter space. Bilinear
interpolation~\cite{10.5555/42249} between histogram bins was then used to
increase the number of bins along each axis to 1000. Example sensitivity
regions for the MARLEY B~2009 model are shown in
Fig.~\ref{MARLEY_NO_nueOnly_Bhattacharya2009Xscn_StepEffic_InterpolatedContours}
before (black) and after (blue) applying the smoothing procedure. The impact
of the smoothing is most noticeable in the plots involving $\varepsilon$ since
the reference grid is coarsest for that parameter. 
Specifically, the interpolated contours are slightly smaller than the original contours.

\begin{figure*}
  \centering
  \includegraphics[scale = 0.75]{MARLEY_NO_nueOnly_Bhattacharya2009Xscn_StepEffic_InterpolatedContours.pdf}
  \caption{90\% C.L. contours for the three parameter spaces with NMO assumptions and the MARLEY B~2009 cross section
model~\cite{marleyCPC}. The contours before interpolation have prominent jagged
edges due to a limited number of reference grid points. The edges are most
noticeable for the $\varepsilon$ parameter.}
\label{MARLEY_NO_nueOnly_Bhattacharya2009Xscn_StepEffic_InterpolatedContours}
\end{figure*}

%TC:ignore
%%%%%%%%%
\bibliographystyle{apsrev4-1}
\bibliography{citations}
%TC:endignore

\end{document}
