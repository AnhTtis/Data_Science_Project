\section{Discussions}
\label{sec:discuss}

% Another application is the augmented reality (AR) based social assistance that can facilitate the interaction between people, as shown in Figure \ref{fig:1}. Imagine a social scenario where participants are walking around to meet others, and each person is with a wearable device such as a smart glasses that can automatically prompt the information about the person in front of him (her). 
% % In this case, the cross-modality collaboration is also required. 
% Specifically, to prompt such a message, the wearable device needs a global ID map, which contains the information of neighboring people and its own position. An edge node connected with an ambient sensor is responsibility for providing such a global ID map by associating the objects in the global view and the IDs of wearable devices.

how to compare with tag-based solution like \cite{soltanaghaei2021millimetro}?

Besides interpersonal distance tracking, the re-identified radar sensing results can also enable other applications. For instance, 

Apart from social tracking, ImmTrack is also applicable to other applications like AR (augmented reality) or VR (virtual reality)-based social assistance. For instance, in a social scenario where participants walk around to meet others, smart glasses could automatically remind the user the information about the person in front of him (her) with the technology from Immtrack,

%Moreover, although Immtrack is robust to occurance of passengers (i.e., people who are in the FOV of the radar while do not enable IMU), it still may fall short when there are roamers (i.e., people who are contributing the IMU data while not in the radar's FOV). This will be left as our future work.

%%% Local Variables:
%%% mode: latex
%%% TeX-master: "main"
%%% End: