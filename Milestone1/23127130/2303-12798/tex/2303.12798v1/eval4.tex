% !TEX root = main.tex
\subsection{Compute and Communication Overheads}
\label{subsec:time}




\begin{figure}
  \centering
  % \begin{minipage}{.3\linewidth}
  %   \centering 
  %   \includegraphics[width=\textwidth]{figures/ble_contact_track_spatial}
  %   \vspace{-1em}
  %   \caption[ ]{Bluetooth neighbor discovery delay versus pairwise distance when there are two or five users.}
  %   \label{fig:ble_temp_res}
  % \end{minipage}

  \begin{minipage}{.45\linewidth}%  
    \includegraphics[width=\linewidth]{figures/process_time}%
    \vspace{-1em}
    \caption{Runtime latency of ImmTrack and ICTrack with different hardwares.}
    \label{fig:timeFinal}%
  \end{minipage}
  \hfill
  \begin{minipage}{.52\linewidth}  
  \centering
    \includegraphics[width=.865\linewidth]{figures/process_time_stack}%
    \vspace{-1.2em}
    \caption{Time for trace map generation (Tra-Gen) and cross-modality association (Assoc).}
    \label{fig:timeDivide}%
  \end{minipage}%
  \vspace{-1em}
\end{figure}

\subsubsection{Server computation overhead}
%In Table.~\ref{tab:bledelay}, we measure the latency for google exposure notification, which is based on Bluetooth. As shown in the table, the bluetooth based method does not satisfy the temporal resolution requirement.
Fig.~\ref{fig:timeFinal} shows the runtime latency of ImmTrack and ICTrack on the server under different $N$. In general, ImmTrack runnning on an Intel i7-11800H CPU can achieve 30 to 60 fps, depending on the number of users. Note that our ImmTrack implementation adopts a radar sampling rate of 8 fps. Thus, a CPU-only cloud server can support several ImmTrack tasks for different venues, or a CPU-only {\em in situ} edge server can support a single ImmTrack instance. ICTrack on the same i7-11800H CPU can only achieve about 15 fps processing throughput. Even with a GeForce RTX-3060 or RTX-6000 GPU, ICTrack's processing throughput is still lower than ImmTrack's, because the image processing imposes higher computation overhead than point cloud processing. By jointly considering the accuracy results obtained in \sect\ref{system_performance}, compared with ICTrack, ImmTrack achieves similar accuracy but only requires 1/4 to 1/2 processing power.
Fig.~\ref{fig:timeDivide} shows the breakdown of the time for processing 90 frames to generate trace map and perform cross-modality association, where generating trace map from radar and camera data takes most of the time.

%in average, the ImmTrack system can generate the association result of 90 frames of data using a CPU within 6.5 seconds. On the contrary, iCTrack requires an advanced GPU to reach the similar throughput. Otherwise, when working on the same CPU, iCTrack consumes over 17 seconds. 
%\todo{Claim that the absolute throughput can meet the real-world need for some applications.}

%Compared to iCTrack, ImmTrack is more suitable for the edge devices for its shorter trajectory generation time as the processing of the sparse point clouds is less complicated than that of dense images.

% In Figure \ref{fig:timeDivide}, we provide the module-wise running time of ImmTrack and iCTrack. The trajectory association time is short (within 4.2 seconds), which means that it can be easily deployed to a wide range of platforms.
%For both systems, the trajectory generation time accounts for most of the total time, while the trajectory association is close and  relatively quick, using less than 2.2 seconds under the case of seven people.  It also shows that our association algorithm\ref{oneshot_id_ass} is robust when using different sensors for tracking.





% \begin{table}
%   \centering  
%   \label{tab:power-compare}    
%   \begin{tabular}{|c|c|}
%     \hline
%     App           & mAh    \\ \hline
%     ImmTrack      & 36.06  \\ \hline
%     Coronalert    & 37.37  \\ \hline
%     TraceTogether & 55.62  \\ \hline
%     LeaveHomeSafe & 157.04 \\ \hline
%   \end{tabular}
%   \vspace{4em}
%   \captionof{table}{Energy usages of contact tracing apps over 8 hours.}
%   % \begin{tabular}{ccccc}
%   %   \hline
%       %       App & ImmTrack & {\small Coronalert} & {\small TraceTogether} & {\small LeaveHomeSafe} \\
%       %       \hline
%       %       mAh & 36.06 & 37.37 & 55.62 & 157.04 \\
%       %       \hline
%       %     \end{tabular}  
% \end{table}

\subsubsection{Smartphone communication and energy overheads}

We deploy both the IMU sampling and trace map generation modules on an Android smartphone and measure the overheads.
ImmTrack uploads the velocity magnitude to the server for the mmWave-IMU pre-matching. At the end of each association time window, ImmTrack uploads the trace map to the server, which is about $30\,\text{KB}$. The mmUniverSense uploads the 3D velocity continuously.
%Here we compare the average communication overhead between the phone and severs for the two systems in a window(12 seconds) by letting them run in the backend for 12 hours.
Our measurements show that ImmTrack's and mmUniverSense's bit rates are $7.36\,\text{kbps}$ and $15.63\,\text{kbps}$, respectively. ImmTrack's bit rate is lower than the $8\,\text{kbps}$ of G.729, an ITU's voice codec for bandwidth-constrained scenarios.

We also compare the battery energy usages of ImmTrack and three existing contact tracing mobile apps, i.e., TraceTogether, LeaveHomeSafe, Coronalert. We run these  apps in the background on an Android smartphone for eight hours. We factory-reset the smartphone before each benchmark. ImmTrack keeps sampling IMU, computing trace maps, and uploading data. From publicly available information, Coronalert (which is based on Google/Apple Exposure Notification system) and TraceTogether exchange Bluetooth messages with nearby devices; LeaveHomeSafe is a passive tracing tool based on QR code scanning. During each 8-hour benchmark, we use the tested app to scan valid QR codes every hour to mimic normal daily usages. 
%Table~\ref{tab:power-compare} shows the apps' battery energy usages reported by the Android OS. 
According to our measurements,  battery energy usages of TraceTogether, LeaveHomeSafe, Coronalert are 55.62, 157.04, 37.37 mAh, respectively, while ImmTrack consumes 36.05 mAh.
Thus, ImmTrack imposes similar/lower battery energy overhead compared with the existing contact tracing apps.

%\yimin{Compared to another low-power wearable based contact tracing system TraceBand\cite{traceband}, though ImmTrack consumes slightly more power(36.06 mAh vs 12.8 mAh), ImmTrack equips with more application scenarios.}



% \begin{table}[]
%   \centering
%   \caption{Energy usages of ImmTrack and contact tracing mobile apps over 8 hours.}
%   \label{tab:power-compare}
%   \begin{tabular}{lllll}
%   \hline
%   App                                                    & \textit{\begin{tabular}[c]{@{}l@{}}Imm\\ Track\end{tabular}} & \multicolumn{1}{c}{\textit{\begin{tabular}[c]{@{}c@{}}Coron\\  alert\end{tabular}}} & \multicolumn{1}{c}{\textit{\begin{tabular}[c]{@{}c@{}}Trace\\ Together\end{tabular}}} & \multicolumn{1}{c}{\textit{\begin{tabular}[c]{@{}c@{}}Leave\\ HomeSafe\end{tabular}}} \\ \hline
%   \begin{tabular}[c]{@{}l@{}}Energy\\ (mAh)\end{tabular} & 36.06                                                        & 37.37                                                                               & 55.62                                                                                 & 157.04                                                                                \\ \hline
%   \end{tabular}
% \end{table}



%The computaion overhead refers to the battery assumotion of the system. Here we compare the computaion overhead of ImmTrack system on the phone with the COVID tracing applications, inclduing LeaveHomeSafe in Hong Kong, TraceTogether in Singapore, Coronalert in Belgium. The LeaveHomeSafe is a passive tracing application based on QR code scan and the remaining two are based on Bluetooth. While the TraceTogether uses the phone's default Bluetooth configuration, the Coronalert uses the Google Exposure Notification Service.  We let all the 4 applications running in the backend for 8 hours and open them every hour to calculate the battery usage. 

%In table\ref{tab:power_compare}, we summarize ther power consumption. For the ImmTrack system and the Bluetooth based systems, the power consumption is below 2\% for the modern mobile phone, which will not be a big concern.

% \begin{table}[]
%   \caption{Comparison of  the power consumption of the contact tracking systems}
%   \label{tab:power_compare}
%   \begin{tabular}{|c|c|}
%   \hline
%   \textbf{}                      & \begin{tabular}[c]{@{}c@{}}Power \\ Consumption\\ (mAh)\end{tabular} \\ \hline
%   \multicolumn{1}{|l|}{ImmTrack} & 36.06                                                                \\ \hline
%   LeaveHomeSafe                  & 157.04                                                               \\ \hline
%   TraceTogether                  & 55.62                                                                \\ \hline
%   Coronalert                     & 37.37                                                                \\ \hline
%   \end{tabular}
% \end{table}

% \begin{table}
%   \caption{Energy usages of contact tracing apps over 8 hours.}
%   \label{tab:power-compare}
%   \vspace{-1em}
%   \begin{tabular}{ccccc}
%     \hline
%     App & ImmTrack & {\small Coronalert} & {\small TraceTogether} & {\small LeaveHomeSafe} \\
%     \hline
%     mAh & 36.06 & 37.37 & 55.62 & 157.04 \\
%     \hline
%   \end{tabular}
% \end{table}






