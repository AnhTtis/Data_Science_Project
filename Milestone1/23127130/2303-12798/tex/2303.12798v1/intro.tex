% !TEX root = main.tex
\section{Introduction}
\label{sec:intro}

%During this COVID-19 pandemic, the population has learned that

%Maintaining interpersonal distancing is an important measure to prevent the spreads of COVID-19.

Retrospective studies have shown that infectious control measures including wearing masks, hand hygiene, and interpersonal distancing contribute to the prevention of COVID-19 and also to the decline of influenza, enterovirus, and all-cause pneumonia \cite{chiu2020impact}. When the mask-on requirement is gradually lifted during the current stage of the COVID-19 pandemic, interpersonal distancing is important to reducing  personal health risks and societal costs in healthcare.

%It may become more important when the requirement of wearing masks is lifted and/or new virus variants occur.
%In U.S., there were about 700 deaths from influenza during the 2020-2021 flu season, in contrast to the about 22,000 and 34,000 deaths in the prior season and two seasons ago \cite{katie2021}.
%Thus, maintaining interpersonal distancing is a good practice to reduce the health risks faced by individuals and the societal costs in healthcare even after the COVID-19 pandemic.

This paper aims to design a system for interpersonal distance tracking for moving people in relatively enclosed environments that require extra attention to airborne transmissions of pathogens via respiratory droplets.
%in defined spaces like supermarkets, museum exhibition halls, playgrounds, etc.
%When the users move within the indoor space and mingle with each other, the system can track the users' interpersonal physical distances continuously.
The tracking results can be used to detect unsafe contacts and generate real-time or {\em ex-post} alerts to the engaged individuals. %Moreover, the large-scale high-precision contact traces provide important guidelines for social distancing regulations. 
%In particular, the {\em ex-post} alerts can be useful for contact tracing when an engaged user is tested positive for a certain contagious disease.
COVID-19 contact tracing often adopts a spatiotemporal definition of contact, i.e., whether a questioned person spent more than $\tau$ seconds within $x$ meters from an infectious source, where the thresholds $\tau$ and $x$ can be updated according to the evolving understanding on virus transmissions. It has been commonly accepted that risk of transmission is greatest within one meter distance. In addition, SARS-CoV-2 has been found transmissible via a fleeting encounter \cite{fleeting-transmission}. The above suggest that effective contact tracing requires decimeters-seconds spatiotemporal accuracy.

%With the occurrences of more transmissible virus variants, the interpersonal distances tracking with finer spatio-temporal resolutions becomes more demanded.

%Thus, accurate spatiotemporal tracking of interpersonal distances is a desirable mechanism to maintain safety in the process of lifting various restrictions imposed during the pandemic.

Bluetooth neighbor discovery (BND) is the prevailing solution for smartphone- or wearable-based contact tracing \cite{kindt2021reliable,traceband}. However,
%QR code scanning only achieves the venue-level spatial resolution. As
as analyzed in \cite{kindt2021reliable}, BND suffers 1) poor temporal resolution due to long discovery latency and 2) inaccurate distance estimation due to multipath and attenuation effects. As such, the BND-based Google/Apple Exposure Notifications System \cite{GAEN} cannot reliably detect contacts shorter than five minutes \cite{kindt2021reliable}.
% ; Bluetooth-based neighbor discovery tends to yield excessive false positive contact detection results \cite{sg-tracetogether}, e.g., due to the {\em dividing wall problem} \cite{azizyan2009surroundsense}.
The existing indoor localization techniques are in general incompetent for contact tracing. As summarized in \cite{xiao2016survey}, device-free approaches, in which the user does not carry a device, face the {\em anonymity} problem in tracking multiple users, i.e., the approaches cannot identify individual users. Without (pseudo) identities, the tracking results cannot be used for contact tracing. On the other hand, smartphone-based approaches have respective limitations, e.g., requiring dense Bluetooth beacons, privacy-intrusive due to visual sensing \cite{xu2020edge}, and insufficient accuracy of WiFi- or geomagnetism-based localization \cite{kotaru2015spotfi,he2017geomagnetism}.
%The approaches based on wireless signal strength are susceptible to electromagnetic noises, transceiver automatic gain control, multipath and attenuation effects.
%Wi-Fi signal strength \cite{bahl2000radar,youssef2005horus} are susceptible to electromagnetic noises and interference, as well as transceivers' automatic gain controls.
%The smartphone localization based on Bluetooth beacons requires a dense beacon deployment to achieve satisfactory accuracy.
%All multilateration and multiangulation approaches require deployment of infrastructure nodes and their precise positions, which causes overhead.
%Smartphone localization based on Bluetooth beacons requires a dense beacon deployment to achieve satisfactory accuracy.
%Visible light-based approaches \cite{zhang2016litell,li2014epsilon} require the phone's light receiver to face light sources, making the use obtrusive.
%Camera-based approaches \cite{xu2020edge} are privacy-intrusive and yield deteriorated performance under reduced illumination. Geomagnetism-based approaches have errors of meters \cite{he2017geomagnetism}.
%WiFi-based approaches \cite{kotaru2015spotfi} require extensive data collection and training process before deployment.
%In sum, there are still no practical solutions that can support contact tracing with decimeters-seconds spatio-temporal accuracy.

%still lacks a readily deployable interpersonal distances tracking system that achieves fine temporal granularity and accurate distance estimation.

\begin{figure}
  \centering
  % \includegraphics[width=\linewidth]{figures/overall_demo}
  \includegraphics[width=\linewidth]{figures/scenario/scenario}
  \vspace{-2.2em}
  \caption{ImmTrack for interpersonal distance tracking.}
  \label{fig:scenario}
  \vspace{-0.7em}
\end{figure}

Recently, millimeter wave (mmWave) radars emerged as a low-cost sensing modality and have been adopted for human detection and tracking \cite{wu2020mmtrack,shuai2021millieye}. The following features of mmWave radars form a good basis for achieving accurate interpersonal distance tracking. First, mmWave radars directly provide the velocity and depth information of the targets, which facilitate tracking the targets' absolute positions. Second, an mmWave radar can cover a large area with good sensitivity. For instance, the Texas Instruments AWR1843 mmWave radar gives a $0.23\,\text{m}$ sensing resolution within a 118\textdegree{} circular sector area with a radius of $40\,\text{m}$, covering a total area of more than 1,600$\,\text{m}^2$. Third, compared with cameras, mmWave radars output coarse-grained point clouds, which are less privacy-sensitive, making the deployment less intrusive.
%The above features make mmWave radars advantageous in providing global information regarding the individuals' locations and movements in the area of interest.

However, mmWave radars also face the anonymity problem. Although research has attempted to apply supervised learning to identify the human subjects from mmWave radar data based on gaits \cite{yang2020mu}, training data from each user is needed, incurring undesirable deployment overhead. The key idea of this paper, which is illustrated in Fig.~\ref{fig:scenario}, is to exploit the inertial measurement units (IMUs) carried by the users to address the mmWave radar sensing's anonymity problem. This is based on the observations that (i) IMU data from the users inherently carry pseudo identities (PIDs), and (ii) both IMU and mmWave radar data contain rich information about the users' movements.
% Specifically, since both the IMU data and mmWave radar data contain rich information above the users' movements and the body-worn IMU data inherently has pseudo identity (PID),
\yimin{While using IMU data only is insufficient for accurate tracking due to the error accumulation problem, by matching IMU data and mmWave radar data in terms of the consistency between their captured velocities and movements, the IMUs' PIDs can be transferred to mmWave radar's accurate tracking results.}
%the two data inputs can be matched based on the consistency of their contained movements, and the PID is transferred to the mmWave radar sensing results. 
% Thus, matching the two data inputs in terms of the consistency of their contained movements transfers the PID to the mmWave radar sensing results. 
The re-identified, radar-sensed user trajectories can then be used for interpersonal distance tracking. Since IMUs are pervasively available on portable and wearable smart devices, the only requirement to enable the tracking is to share a summary of the IMU data regarding the user's movements.

%Although the IMU data summary with a pseudo-identity may generate some privacy concern, it is not as intrusive as the RGB camera-based surveillance. In addition, rendering IMU data summary trades for reduction of personal health risks in pandemic, which may motivate participation.

%There have been a variety of Internet of Things (IoT) devices appearing in people's lives, from portable devices such as smart watch, mobile phones to static surveillance cameras and radars placed on infrastructures. Collaboration between portable devices with local views and ambient sensors with global views can bring tremendous value and enable a number of compelling applications. 

%Figure \ref{fig:1} shows an application of social contact tracking or disease exposure notifications. An edge node connected to a global-view ambient sensor monitors the interaction among a crowd of people. However, in order to generate the social tracking logs for future query and achieve personalized notifications, the identities (IDs) of every detected object or person are required. Typically, the ID identification techniques using merely the information from the ambient sensor (i.e., face recognition) require privacy-sensitive information. On the other hand, we notice that the mobile phone of every user is naturally an identifier. Therefore, by associating users' devices and the detected objects under global view, the contact tracking can be achieved without compromising the privacy. 

% \begin{figure}[!t]
%     \centering
%     \includegraphics[width=\linewidth]{figures/fig1}
%   \caption{Example Application Scenarios.}
%   \label{fig:1}
% \end{figure}

Based on the above idea, we design a system called {\em {\em ImmTrack}} that employs one or more mmWave radar(s) and exploits the IMUs on the user-carried smartphones or wearables to achieve accurate interpersonal distance tracking. The design of ImmTrack needs to address the following two challenges. First, the point cloud from the radar is usually sparse and noisy \cite{shuai2021millieye}, making it difficult to separate and track multiple users during their close contacts. Second, as radar and IMU capture different aspects of movements, the cross-modality matching is non-trivial. Specifically, the radar's point cloud indicates user's space occupancy and radial movement of torso, while the IMU time series data captures linear acceleration and angular speed of the IMU-carrying limb. As a result, a common  representation of the movement inferred from the two modalities is needed for robust cross-modality matching.
% Specifically, we aim to associate two modalities where the format and view of the data differ significantly. The output of mmWave radars are 3D points which mainly indicate the occupancy and the radial movement of people's torso in the global coordinate, while the outputs of IMUs are time serial data, containing the angular speed, acceleration and velocity of the people's limbs.
%{\bf (one more sentence to further explain the challenge)}

%Moreover, \textcolor{blue}{lightweight?}

To address the first challenge, ImmTrack clusters the point cloud in a single frame from the radar with initial centroids predicted by Kalman filters that capture the users' motions. The motion-aware clustering effectively prevents the wrong merge of the clusters of two users when they move close to each other. Moreover, we design a deep neural network called {\em mmClusterNet} to extract each cluster's feature capturing both shape and motion information. Then, the Hungarian algorithm associates the same user's clusters in two consecutive frames based on feature similarity, achieving multi-user inter-frame tracking. To address the second challenge, we employ a novel representation of the user's movement, called {\em trace map}, which is inferred from either the radar's tracking or the IMU's dead reckoning. We devise a Siamese neural network to extract a comparative feature from the trace map, such that the cosine similarity between two comparative features from the two modalities indicates whether they are from the same user.


We conduct experiments with up to 27 people to evaluate ImmTrack in various environments, including sports hall, lab space, and playground. Compared with the camera-based system, ImmTrack achieves similar user tracking accuracy but only incurs 1/4 to 1/2 computation overhead to process sensor data. For interpersonal distance estimation, ImmTrack achieves an average error of $22\,\text{cm}$. For pinpointing contacts within one meter over two seconds or more, ImmTrack achieves 90\% precision and 94\% recall. Compared with BND, ImmTrack reduces detection latency by up to 80 seconds. In sum, ImmTrack is suitable for hotspot venues that require extra care in preventing virus transmissions over close contacts.

%by up to 70 seconds compared with that of Bluetooth neighbor discovery in our experiments.

The contributions of this paper are summarized as follows.
\begin{itemize}
\item We design a motion-aware mmWave radar point cloud clustering algorithm and mmClusterNet neural network for extracting cluster feature, which work together to achieve robust multi-user inter-frame tracking.
\item We propose trace map, a new modality-agnostic representation of human movement, and devise a Siamese neural network to extract feature from the trace map for effective mmWave-IMU matching.
\item \yimin{The above two designs make ImmTrack the first system that fuses data from mmWave radar and IMUs for simultaneous user tracking and re-identification.} From extensive evaluation with up to 27 people in various environments, ImmTrack achieves decimeters-seconds spatiotemporal accuracy in contact tracing.
%  distance tracking accuracy with seconds temporal granularity. This enables finer-grained contact tracing and other applications.
\end{itemize}



{\em Paper organization:} \sect\ref{sec:related} presents the background and related work. \sect\ref{sec:problem} states the problem. \sect\ref{sec:global-tracking-design} and \sect\ref{subsec:association} present the designs of mmWave tracking and cross-modality matching, respectively. \sect\ref{sec:eval} presents the evaluation results.
%Section~\ref{sec:discuss} discusses several issues.
\sect\ref{sec:conclude} concludes this paper.

% An observer looking at this problem might say "ah yeah, it's a solution to localization without GPS dependency", but I think it's more of a superset of indoor localization. It CAN provide indoor smartphone localization, but it can also provide association between individuals, their motion over time, and their devices. I think there's some value in explicitly stating this in the paper. Perhaps point out that - yeah, it DOES solve indoor localization in many cases - but it also solves problems that can't be posed precisely as localization (the examples in the papers of disease contact tracing and Vehicle-to-Pedestrian warnings are 2 examples).


%%% Local Variables:
%%% mode: latex
%%% TeX-master: "main"
%%% End: