%%%% Proceedings format for most of ACM conferences (with the exceptions listed below) and all ICPS volumes.
% \documentclass[acmlarge,anonymous]{acmart}
\documentclass[sigconf]{acmart}
%\documentclass[sigconf]{acmart}
%%%% As of March 2017, [siggraph] is no longer used. Please use sigconf (above) for SIGGRAPH conferences.

%%%% Proceedings format for SIGPLAN conferences 
% \documentclass[sigplan, anonymous, review]{acmart}

%%%% Proceedings format for SIGCHI conferences
% \documentclass[sigchi, review]{acmart}

%%%% To use the SIGCHI extended abstract template, please visit
% https://www.overleaf.com/read/zzzfqvkmrfzn

%\usepackage{booktabs} % For formal tables
%\usepackage{epstopdf}
%\usepackage{bm}
%\usepackage{array}
%\usepackage{multirow}
%\usepackage{rotating}
%\usepackage{balance} %to make reference balanced
%\usepackage[demo]{graphicx}
%\usepackage{caption}
\usepackage{amsmath}
\usepackage{multirow}
\usepackage{subcaption}
%\usepackage{wrapfig}
%\usepackage[table,xcdraw]{xcolor}
% pseudocode
\usepackage{algorithm}
%\usepackage{algorithmicx}
\usepackage{algpseudocode}

\settopmatter{printacmref=false}
\renewcommand\footnotetextcopyrightpermission[1]{}

\newcommand{\todo}[1]{\textcolor{red}{[#1]}}
\newcommand{\revised}[1]{\textcolor{blue}{[#1]}}
\newcommand{\ignore}[1]{{}}



% \newcommand{\yimin}[1]{\textcolor{orange}{#1}}
% \newcommand{\yimin}[1]{\textcolor{blue}{#1}}
\newcommand{\yimin}[1]{#1}

\renewcommand{\vec}[1]{\mathbf{#1}} %% vector
\newcommand{\vecg}[1]{\boldsymbol{#1}} %% vector for Greek characters
\newcommand{\mat}[1]{\mathbf{#1}} %% matrix

\newcommand{\red}{\color{red}}
\newcommand{\blue}{\color{blue}}
%\newcommand{\blue}{}
\newcommand{\green}{\color{green}}

\newlength{\subcolumnwidth}
\newenvironment{subcolumns}[1][0.45\columnwidth]
 {\valign\bgroup\hsize=#1\setlength{\subcolumnwidth}{\hsize}\vfil##\vfil\cr}
 {\crcr\egroup}
\newcommand{\nextsubcolumn}[1][]{%
  \cr\noalign{\hfill}
  \if\relax\detokenize{#1}\relax\else\hsize=#1\setlength{\subcolumnwidth}{\hsize}\fi
}
\newcommand{\nextsubfigure}{\vfill}

\newcommand{\sect}{\textsection}
\renewcommand{\figurename}{Fig.}


% Copyright
%\setcopyright{none}
\setcopyright{acmcopyright}
%\setcopyright{acmlicensed}
%\setcopyright{rightsretained}
%\setcopyright{usgov}
%\setcopyright{usgovmixed}
%\setcopyright{cagov}
%\setcopyright{cagovmixed}


%\acmJournal{IMWUT}


\begin{document}

\title{Interpersonal Distance Tracking with mmWave Radar and IMUs}
% \titlenote{Produces the permission block, and
%   copyright information}
% \subtitle{Extended Abstract}
% \subtitlenote{The full version of the author's guide is available as
%   \texttt{acmart.pdf} document}

%\author{Paper ID: 001}

\author{Yimin Dai}
% \authornote{Both authors contributed equally to this research.}
\affiliation{%
  \department{School of Computer Science and Engineering}
  \institution{Nanyang Technological University}
  \country{Singapore}
}
%\email{yimin006@e.ntu.edu.sg}

\author{Xian Shuai}
\affiliation{%
  \department{Department of Information Engineering}
  \institution{The Chinese University of Hong Kong}
  \state{Hong Kong SAR}
  \country{China}
}
%\email{sx018@ie.cuhk.edu.hk}

% \orcid{1234-5678-9012}
\author{Rui Tan}
% \authornotemark[1]
\affiliation{%
  \department{School of Computer Science and Engineering}
  \institution{Nanyang Technological University}
  \country{Singapore}
}
%\email{tanrui@ntu.edu.sg}

\author{Guoliang Xing}
\affiliation{%
  \department{Department of Information Engineering}
  \institution{The Chinese University of Hong Kong}
  \state{Hong Kong SAR}
  \country{China}
}
%\email{glxing@cuhk.edu.hk}


% The default list of authors is too long for headers}
%\renewcommand{\shortauthors}{B. Trovato et al.}

\begin{abstract}
  Tracking interpersonal distances is essential for real-time social distancing management and {\em ex-post} contact tracing to prevent spreads of contagious diseases. Bluetooth neighbor discovery has been employed for such purposes in combating COVID-19, but does not provide satisfactory spatiotemporal resolutions. This paper presents ImmTrack, a system that uses a millimeter wave radar and exploits the inertial measurement data from user-carried smartphones or wearables to track interpersonal distances. By matching the movement traces reconstructed from the radar and inertial data, the pseudo identities of the inertial data can be transferred to the radar sensing results in the global coordinate system. The re-identified, radar-sensed movement trajectories are then used to track interpersonal distances. \yimin{In a broader sense, ImmTrack is the first system that fuses data from millimeter wave radar and inertial measurement units for simultaneous user tracking and re-identification.} Evaluation with up to 27 people in various indoor/outdoor environments shows ImmTrack's decimeters-seconds spatiotemporal accuracy in contact tracing, which is similar to that of the privacy-intrusive camera surveillance and significantly outperforms the Bluetooth neighbor discovery approach.
\end{abstract}


%  The prevailing solution in combating COVID-19, which is based on Bluetooth neighbor discovery, does not achieve satisfactory spatiotemporal resolutions.

%break the assumption that data-driven individual thermal comfort models are independent from each other and

\keywords{mmWave radar, IMU, association, tracking}

% \begin{CCSXML}
% <ccs2012>
%    <concept>
%        <concept_id>10003120.10003138.10003139.10010906</concept_id>
%        <concept_desc>Human-centered computing~Ambient intelligence</concept_desc>
%        <concept_significance>500</concept_significance>
%        </concept>
%    <concept>
%        <concept_id>10003120.10003138.10003139.10010905</concept_id>
%        <concept_desc>Human-centered computing~Mobile computing</concept_desc>
%        <concept_significance>500</concept_significance>
%        </concept>
%    <concept>
%        <concept_id>10010520.10010553.10003238</concept_id>
%        <concept_desc>Computer systems organization~Sensor networks</concept_desc>
%        <concept_significance>300</concept_significance>
%        </concept>
%  </ccs2012>
% \end{CCSXML}

% \ccsdesc[500]{Human-centered computing~Ambient intelligence}
% \ccsdesc[500]{Human-centered computing~Mobile computing}
% \ccsdesc[300]{Computer systems organization~Sensor networks}

%
% The code below should be generated by the tool at
% http://dl.acm.org/ccs.cfm
% Please copy and paste the code instead of the example below. 
%
% \begin{CCSXML}
% <ccs2012>
% <concept>
% <concept_id>10002951.10003227.10003241.10003244</concept_id>
% <concept_desc>Information systems~Data analytics</concept_desc>
% <concept_significance>500</concept_significance>
% </concept>
% <concept>
% <concept_id>10010583.10010662.10010586</concept_id>
% <concept_desc>Hardware~Thermal issues</concept_desc>
% <concept_significance>500</concept_significance>
% </concept>
% </ccs2012>
% \end{CCSXML}

% \ccsdesc[500]{Information systems~Data analytics}
% \ccsdesc[500]{Hardware~Thermal issues}


\maketitle

\section{Introduction}

The increasing complexity of source code poses a key challenge to the reliability of large-scale software systems. Software bugs in these systems can lead to safety issues~\cite{bug_safety} for users around the world as well as cause non-negligible financial losses~\cite{bug_loss}. As such, developers have to spend a large amount of time and effort on bug fixing. Consequently, \aprfull (\apr), designed to automatically generate patches to fix software bugs, has attracted wide attention from both academia and industry~\cite{long2016prophet, legoues2012genprog, long2015spr, lou2020can, tufano2018empstudy}. 


To achieve \apr, one popular approach is known as Generate-and-Validate (G\&V)~\cite{qi2015gv, ghanbari2019prapr, lou2020can, le2016hdrepair, legoues2012genprog, wen2018capgen, hua2018sketchfix, martinez2016astor, koyuncu2020fixminder, liu2019tbar, liu2019avatar}, which is typically based on the following pipeline: First, fault localization techniques~\cite{wong2016fl, abreu2007ochiai, zhang2013injecting, papadakis2015metallaxis, li2019deepfl, li2017transforming} are applied to determine the suspicious locations in programs where bugs are likely to exist. Then, the buggy locations are used by the \apr tools to generate a list of patches that replace buggy lines with correct lines. Afterward, each patch is validated against the original test suite to identify any \emph{plausible patches} (i.e., passing all tests in the test suite). Finally, to determine the \emph{correct patches}, developers examine the list of plausible patches to see if any of them can correctly fix the bug. 

Traditional \apr tools can mainly be categorized into heuristic-based~\cite{legoues2012genprog, le2016hdrepair, wen2018capgen}, constraint-based~\cite{mechtaev2016angelix, le2017s3, demacro2014nopol, long2015spr} and \template~\cite{ghanbari2019prapr, hua2018sketchfix, martinez2016astor, liu2019tbar, liu2019avatar}. Among these traditional tools, \template \apr tools~\cite{ghanbari2019prapr, liu2019tbar, benton2020effectiveness} have been able to achieve state-of-the-art results. \Template \apr tools typically leverage pre-defined templates (e.g., adding a nullness check) for bug fixing. However, since these fix templates are typically handcrafted, the number and types of bugs they are able to fix can be limited. 



To address the limitations of traditional \apr, researchers have proposed various \learning \apr tools~\cite{li2020dlfix, chen2018sequencer, jiang2021cure, lutellier2020coconut, zhu2021recoder, ye2022rewardrepair} based on the \nmtfull (\nmt) architecture~\cite{sutskever2014mt} where the input is the buggy code snippets and the goal is to translate the buggy code snippets into a fixed version. To accomplish this, \learning \apr tools require supervised training datasets with pairs of both buggy and fixed code snippets in order to learn how to perform this translation step. These training data are usually obtained by mining historical bug fixes using heuristics/keywords~\cite{dallmeier2007benchmark}, which can be imprecise for identifying bug-fixing commits; even the actual bug-fixing commits can include irrelevant code changes, leading to further pollution in the dataset~\cite{xia2022alpharepair}.
% 
Moreover, it can be hard for such \apr tools to generalize and fix bug types unseen during training. 



To better leverage recent advances in \plmfull{s} (\plm{s}), researchers~\cite{xia2022alpharepair, xia2023repairstudy, kolak2022patch, prenner2021codexws} have directly applied \plm{s} to generate patches without bug-fixing datasets. These \llm-based \apr tools work by either directly generating a complete code function~\cite{prenner2021codexws, xia2023repairstudy} or predict/infill the correct code snippet given its surrounding context~\cite{xia2022alpharepair, xia2023repairstudy}. By directly using \llm{s} that are pre-trained on billions of open-source code snippets, \llm-based \apr tools can achieve state-of-the-art performance on many repair datasets~\cite{xia2022alpharepair}. 


% 
%
%

Traditional \apr tools have long used the insight of the \emph{plastic surgery hypothesis}~\cite{barr2014plastic} where it states that the code ingredients to fix a bug already exist within the same project. Traditional \apr tools have manually designed pattern-~\cite{ghanbari2019prapr, saha2017elixir} or heuristic-based~\cite{jiang2018simfix, legoues2012genprog} approaches to finding and using such relevant code ingredients to generate fixes for bugs. However, the plastic surgery hypothesis has been largely ignored in \llm-based \apr. In fact, \llm provides a unique opportunity to fully automate the plastic surgery hypothesis idea via fine-tuning (learning project-specific information via model updates from the buggy project) and prompting (directly providing relevant code ingredients to the model), and make it directly applicable to different languages (since the \llm{s} are typically multi-lingual).%
Moreover, despite the intensive manual efforts involved, traditional \apr tools still cannot fully leverage project-specific information due to large search space for leveraging/composing existing code ingredients. In contrast, the project-specific information can effectively leveraged by \llm{s} due to their power in code understanding/vectorization, e.g., even partial/imprecise information may still guide \llm{s} in correct patch generation!
 To this end, we ask the question: \emph{How useful is the plastic surgery hypothesis in the era of \plm{s}}?








\mypara{Our Work.} To answer the question, we present \ourtech{\xspace} -- a \llm-based approach that automatically utilizes the plastic surgery hypothesis by systematically combining multiple fine-tuning and prompting strategies for \apr. \ourtech fine-tunes \plm{s} using two novel domain-specific training strategies: \textbf{\epfinetune} -- we fine-tune using the original buggy project by aggressively masking out a high percentage of tokens, which allows \plm to learn project-specific code tokens and programming styles; and \textbf{\rofinetune} -- which only masks out a single continuous code sequence per training sample, allowing the model to get used to the final \csapr task of predicting a single continuous code sequence. Furthermore, we directly leverage the ability for \plm{s} to understand natural language instructions and introduce a novel prompting strategy, \textbf{\idprompting}, which uses information retrieval and static analysis to obtain a list of relevant identifiers for the buggy lines. While such relevant identifiers are critical for fixing some difficult bugs, they may not be seen by the \llm during inference due to limited context window size. Through the use of prompting, we directly tell the model to use these extracted identifiers (relevant code ingredients) to generate the correct code. Finally, to perform repair, we combine all four model variants (including the base model, both fine-tuned models and the base model with prompting) for the final repair.





While our insight of leveraging the plastic surgery hypothesis for \llm-based \apr is generalizable across different types of \plm{s}, to implement \ourtech, we choose a recent \plm{\xspace}, \ctfive~\cite{wang2021codet5}, which is pre-trained on millions of open-source code snippets. \ctfive is an encoder-decoder model trained using \mspfull (\msp) objective where a percentage of tokens are masked out and each continuous masked token sequence is referred to as a masked span. Also, although we only extract relevant identifiers from the current buggy project (since this paper focuses on the plastic surgery hypothesis), our work can be easily extended to obtain other code information (such as relevant statements or functions) from other sources, such as  the massive pre-training corpora~\cite{husain2020codesearchnet} or historical bug-fixing datasets~\cite{jiang2019infer}, which can provide more coding knowledge for \llm{s}. Besides, although we mainly focus on using traditional string comparison algorithms for information retrieval in this paper, these techniques can be easily replaced by other frequency-based retrieval~\cite{robertson2009probabilistic} and neural search (or embedding-based search)~\cite{reimers2019sentence}.
  In summary, this paper makes the following contributions:


%


\begin{itemize}[noitemsep, leftmargin=*, topsep=0pt]
    \item \textbf{Dimension.} This paper is the first to revisit the important plastic surgery hypothesis in the era of \llm{s}. It opens up a new dimension for \llm-based \apr to incorporate previously neglected information from the buggy project itself to boost \apr performance. Furthermore, it demonstrates the promising future of retrieval-based prompting for modern \llm-based \apr.
    \item \textbf{Implementation.} We implement \ourtech based on the recent \ctfive model. We augment the model using two novel fine-tuning strategies: \epfinetune and \rofinetune, along with a novel prompting strategy based on information retrieval and static analysis: \idprompting. We combine the patches generated by all four models together and perform patch ranking to speed up \apr.% 
    \item \textbf{Evaluation Study.} We conduct an extensive evaluation against state-of-the-art \apr tools. On the widely studied \dfj 1.2 and 2.0 datasets~\cite{just2014dfj}, \ourtech is able to achieve the new state-of-the-art results of 89 and 44 correct bug fixes (15 and 8 more than best baseline) respectively.  Furthermore, we perform a broad ablation study to justify our design. \ourtech demonstrates for the first time that the plastic surgery hypothesis can substantially boost \llm-based \apr and advance state-of-the-art \apr, while being fully automated and general. Moreover, even partial/imprecise code ingredients may still effectively guide \llm{s} for \apr!
\end{itemize}


\section{Related work}
\noindent \textbf{Video foundation models.}
With sufficient computational power and an abundant source of data, there have been attempts to build a single large-scale foundation model that can be adapted to diverse downstream tasks.
Along with the success of foundations models in the natural language processing domain~\cite{brown2020language,chen2021evaluating,devlin2019bert} and in computer vision~\cite{bertasius2021space,jia2021scaling,radford2021learning}, video data has become another data type of interest, as it has grown in scale due to numerous internet video-sharing platforms.
Accordingly, several methods to train a video foundation model have been proposed.
Due to the innate multi-modality of video data, \textit{i.e.}, a combination of visual $\cdot$ vocal $\cdot$ textual context, most works have centered around the variations of the cross-modal attention mechanism \cite{akbari2021vatt,bertasius2021space,gabeur2020multi,luo2020univl,neimark2021video,tan2021look,wei2020multi,yang2021taco}.
In addition, as most video data lack proper labels or descriptions, contrastive learning methods were studied to learn meaningful feature representations or enhance video-text alignment in a self-supervised manner \cite{akbari2021vatt,kuang2021video,luo2020univl,yang2021taco}.

More specifically, MERLOT \cite{zellers2021merlot} proposed a multi-modal representation learning method for visual commonsense reasoning, which also performed well in twelve video reasoning tasks.
VATT \cite{akbari2021vatt} introduced a multi-modal learning method via contrastive learning. 
The pre-trained model performed well in a variety of vision tasks from image classification to video action recognition and zero-shot video retrieval.
Another representative work, UniVL \cite{luo2020univl} proposed a straightforward pre-training method with auxiliary loss functions. 
After fine-tuning on a specific task, the pre-trained model performed outstandingly in a wide range of tasks of text-to-video retrieval, action segmentation, action step localization, video sentiment analysis, and video captioning.
Other foundation models for multiple video tasks include \cite{li2020hero,sun2019learning,sun2019videobert,zhu2020actbert,fu2021violet,wang2022all}. 

\noindent \textbf{Auxiliary learning.}
In order to enhance the performance of one or a multitude of primary tasks, auxiliary learning methods can be incorporated.
\cite{ruder2017overview} introduced Multi-task learning (MTL) to the deep neural networks by training a single model with multiple task losses to assist learning on the main task.
Such a method is generally adapted to pre-train the foundation models in the self-supervised manner~\cite{li2020hero,sun2019learning,sun2019videobert,zhu2020actbert,fu2021violet,wang2022all}.
However, these various pretext task losses used in the pre-training phase are ignored in the fine-tuning phase, and only the primary task loss is minimized.

Recently, meta-learning methods have been introduced for auxiliary learning.
\cite{liu2019self,navon2020auxiliary,shu2019meta} proposed a meta-learning method in which the model learns auxiliary tasks to generalize well to unseen data. 
In these settings, a separate subset of data is held out as the primary task, while the others are used as auxiliary tasks that aid the primary task's performance.
Similar methods were adopted for computer vision tasks such as semantic segmentation \cite{xu2021leveraging}.
Other domain applications include navigation tasks with reinforcement learning \cite{ye2021auxiliary}, or self-supervised learning methods on graph data \cite{hwang2020self}.
\section{The Semi-Oblivious Chase Procedure}\label{sec:semi}
%

The semi-oblivious chase (or simply chase) takes as input a database $D$ and a set $\dep$ of TGDs, and constructs an instance that contains $D$ and satisfies $\dep$.
%
A central notion in this context is that of trigger.
%are those of trigger, active trigger, and trigger application.

\begin{definition}%[\textbf{Trigger Application}]
	Given a set $\dep$ of TGDs and an instance $I$, a {\em trigger} for $\dep$  on $I$ is a pair $(\sigma,h)$, where $\sigma \in \dep$ and $h$ is a homomorphism from $\body{\sigma}$ to $I$.
	%
	The {\em result} of $(\sigma,h)$, denoted $\result{\sigma}{h}$, is the set $\mu(\head{\sigma})$, where $\mu : \var{\head{\sigma}} \ra \ins{C} \cup \ins{N}$ is defined as follows:
	%
	%$\mu(x) = h(x)$ if $x \in \fr{\sigma}$, and $\mu(x) = \bot_{\sigma,h_{|\fr{\sigma}}}^{x}$ otherwise,
	\[
	\mu(x)\
	=\ \left\{
	\begin{array}{ll}
	h(x) & \quad \text{if } x \in \fr{\sigma}\\
	&\\
	\bot_{\sigma,h_{|\fr{\sigma}}}^{x} & \quad \text{otherwise}
	\end{array} \right.
	\]
	where $\bot_{\sigma,h_{|\fr{\sigma}}}^{x} \in \ins{N}$.  Let $T(\dep,I)$ be the set of triggers for $\dep$ on $I$.	\hfill\markfull
\end{definition}




Observe that in the definition of $\result{\sigma}{h}$, each existentially quantified variable $x$ of $\head{\sigma}$ is mapped by $\mu$ to a null value of $\ins{N}$ whose name is uniquely determined by the trigger $(\sigma,h)$ and the variable $x$ itself. This means that, given a trigger $(\sigma,h)$, we can unambiguously construct the set of atoms $\result{\sigma}{h}$.
%
The central idea of the chase is, starting from a database $D$, to exhaustively apply triggers for the given set $\dep$ of TGDs on the instance constructed so far.
%
More precisely, given a database $D$ and a set $\dep$ of TGDs, let
\[
\mathsf{chase}^{0}(D,\dep)\ =\ D,
\]
and for each $i>0$, let
\[
\mathsf{chase}^{i}(D,\dep)\ =\ \mathsf{chase}^{i-1}(D,\dep)\ \cup\ \bigcup_{(\sigma,h) \in S} \result{\sigma}{h},
\]
where $S = T(\dep,\mathsf{chase}^{i-1}(D,\dep))$. 
%
We finally define {\em the result of the chase of $D$ w.r.t.~$\dep$} as the (possibly infinite) instance
\[
\chase{D}{\dep}\ =\ \bigcup_{i \geq 0} \mathsf{chase}^{i}(D,\dep).
\]


\ignore{
The semi-oblivious chase procedure (or simply chase) takes as input a database $D$ and a set $\dep$ of TGDs, and constructs an instance that contains $D$ and satisfies $\dep$.
%
Central notions in this context are those of trigger, active trigger, and trigger application.

\begin{definition}%[\textbf{Trigger Application}]
	Given a set $\dep$ of TGDs and an instance $I$, a {\em trigger} for $\dep$  on $I$ is a pair $(\sigma,h)$, where $\sigma \in \dep$ and $h$ is a homomorphism from $\body{\sigma}$ to $I$.
	%
	The {\em result} of $(\sigma,h)$, denoted $\result{\sigma}{h}$, is the set $\mu(\head{\sigma})$, where $\mu : \var{\head{\sigma}} \ra \ins{C} \cup \ins{N}$ is defined as follows:
	%
	%$\mu(x) = h(x)$ if $x \in \fr{\sigma}$, and $\mu(x) = \bot_{\sigma,h_{|\fr{\sigma}}}^{x}$ otherwise,
	\[
	\mu(x)\
	=\ \left\{
	\begin{array}{ll}
	h(x) & \quad \text{if } x \in \fr{\sigma}\\
	&\\
	\bot_{\sigma,h_{|\fr{\sigma}}}^{x} & \quad \text{otherwise}
	\end{array} \right.
	\]
	where $\bot_{\sigma,h_{|\fr{\sigma}}}^{x}$ is a null value from $\ins{N}$.
	%
	The trigger $(\sigma,h)$ is {\em active} if $\result{\sigma}{h} \not\subseteq I$.
	%
	The {\em application} of $(\sigma,h)$ to $I$ returns the instance $J = I \cup \result{\sigma}{h}$ and is denoted as $I \app{\sigma}{h} J$.
	\hfill\markfull
\end{definition}


Observe that in the definition of $\result{\sigma}{h}$ above, each existentially quantified variable $x$ of $\head{\sigma}$ is mapped by $\mu$ to a null value of $\ins{N}$ whose name is uniquely determined by the trigger $(\sigma,h)$ and the variable $x$ itself. This means that, given a trigger $(\sigma,h)$, we can unambiguously extract the set of atoms 
$\result{\sigma}{h}$.



%\medskip

%\noindent
%\textbf{Semi-Oblivious Chase.}
The central idea of the chase is, starting from a database $D$, to exhaustively apply active triggers for the given set $\dep$ of TGDs on the instance constructed so far. This is formalized via the notion of (semi-oblivious) chase derivation, which can be finite or infinite.


\begin{definition}
	Consider a database $D$ and a set $\dep$ of TGDs.
	%We consider the two cases where a derivation is finite or infinite:
	\begin{itemize}
		\item A finite sequence $(I_i)_{0 \leq i \leq n}$ of instances, with $D = I_0$ and $n \geq 0$, is a {\em chase derivation} of $D$ w.r.t.~$\dep$ if, for each $i \in \{0,\ldots,n-1\}$, there is an active trigger $(\sigma,h)$ for $\dep$ on $I_i$ with $I_i \app{\sigma}{h} I_{i+1}$, and there is no active trigger for $\dep$ on $I_n$. The {\em result} of such a chase derivation is the instance $I_n$.
		
		
		\item An infinite sequence $(I_i)_{i \geq 0}$ of instances, with $D = I_0$, is a {\em chase derivation} of $D$ w.r.t.~$\dep$ if, for each $i \geq 0$, there is an active trigger $(\sigma,h)$ for $\dep$ on $I_i$ such that $I_i \app{\sigma}{h} I_{i+1}$. Moreover, $(I_i)_{i \geq 0}$ is {\em fair} if, for each $i \geq 0$, and for every active trigger $(\sigma,h)$ for $\dep$ on $I_i$, there exists $j > i$ such that $(\sigma,h)$ is not an active trigger for $\dep$ on $I_j$. 
		%The latter is known as the {\em fairness condition}, and guarantees that all the active triggers will be deactivated. %
		The {\em result} of such a chase derivation is the instance $\bigcup_{i \geq 0} \, I_i$.
	\end{itemize}
	%
	%The {\em result} of a chase derivation is defined as the union of all the instances occurring in it. 
	A chase derivation is {\em valid} if it is finite or infinite and fair.  \hfill\markfull
\end{definition}


Let us stress that infinite but unfair chase derivations are not considered as valid ones since they do not serve the main purpose of the chase, that is, to build an instance that satisfies the given set of TGDs. Indeed, given the set $\dep$ consisting of the TGDs
\[
\sigma\ =\ R(x,y) \ra \exists z \, R(y,z) \qquad \sigma'\ =\ R(x,y) \ra P(x,y),
\]
the result of the unfair chase derivation of $D = \{R(a,b)\}$ w.r.t.~$\dep$ that involves only triggers of the form $(\sigma,\cdot)$, i.e., only the TGD $\sigma$ is used, does not satisfy $\sigma'$, and thus, it does not satisfy $\dep$.
%
Interestingly, for every database $D$ and set $\dep$ of TGDs, any two valid chase derivations of $D$ w.r.t.~$\dep$ have always the same result, which implies that all valid chase derivations are either finite or infinite~\cite{GrOn18}. Therefore, in the rest of the paper, we can safely refer to {\em the} result of the chase of $D$ w.r.t. $\dep$, which we will denote by $\chase{D}{\dep}$. 
}


%\subsection{Non-Uniform Chase Termination}\label{sec:problem}
%

\medskip

\noindent
\textbf{Chase Termination.}
The result of the chase may be infinite even for very simple settings: it is easy to see that for $D = \{R(a,b)\}$ and $\dep = \{R(x,y) \ra \exists z \, R(y,z)\}$, $\chase{D}{\dep}$ is infinite.
%; in particular, $\chase{D}{\dep} = \{R(a,b),R(b,\bot_1),R(\bot_1,\bot_2),R(\bot_2,\bot_3),\ldots\}$, where $\bot_1,\bot_2,\ldots$ are null values.
%
This leads to the following problem, parameterized by a class $\class{C}$ of TGDs such as $\class{SL}$ (the class of simple-linear TGDs) and $\class{L}$ (the class of linear TGDs):


\medskip

\begin{center}
	\fbox{
		\begin{tabular}{ll}
			%{\small PROBLEM} : & %$\mathsf{ChaseTermination}(\class{C})$
			%\\
			{\small INPUT} : & A database $D$ and a set $\dep$ of TGDs from $\class{C}$.
			\\
			{\small QUESTION} : &  Is the instance $\chase{D}{\dep}$ finite?
	\end{tabular}}
\end{center}

\medskip

\noindent This problem has been recently studied in~\cite{CaGP22} for the classes of simple-linear and linear TGDs. Interestingly, for both classes, the finiteness of the result of the chase has been syntactically characterized by exploiting the notion of non-uniform weak-acyclicity. 
%
We proceed to recall this acyclicity notion, and then present the characterizations established in~\cite{CaGP22}, which in turn lead to simple algorithms for checking the finiteness of the result of the chase.
%
Note that, for the sake of clarity, in the rest of the paper we assume TGDs with a non-empty frontier, i.e., we assume that there is at least one variable in a TGD $\sigma$ that occurs both in $\body{\sigma}$ and $\head{\sigma}$. This assumption can be made without loss of generality since, given a database $D$ and a set $\dep$ of TGDs, we can easily construct a set $\dep'$ of TGDs with a non-empty frontier by slightly modifying $\dep$ such that $\chase{D}{\dep}$ is finite iff $\chase{D}{\dep'}$ is finite.


\medskip

\noindent
\textbf{Non-Uniform Weak-Acyclicity.} Weak-acyclicity was introduced in~\cite{FKMP05} as the main formalism for data exchange purposes, which guarantees the finiteness of the result of the chase for {\em every} input database. Non-uniform weak-acyclicity is the database-dependent variant of weak-acyclicity introduced in~\cite{CaGP22}. We proceed to give the formal definitions.
%
We first need to recall the notion of the {\em dependency graph} of a set $\dep$ of TGDs, 
%which symbolically encodes how terms may propagate during the chase.
%The {\em dependency graph} of set $\dep$ of TGDs 
defined as a directed multigraph $\depg{\dep}=(N,E)$, where $N = \pos{\sch{\dep}}$ and $E$ contains {\em only} the following edges.
%
For each TGD $\sigma \in \dep$ with $\head{\sigma} = \{\alpha_1,\ldots,\alpha_k\}$, for each $x \in \frontier{\sigma}$, and for each position $\pi \in \posvar{\body{\sigma}}{x}$:
\begin{itemize}
	\item For each $i \in [k]$ and for each $\pi' \in \posvar{\alpha_i}{x}$, there exists a \emph{normal} edge $(\pi,\pi') \in E$.
	%
	\item For each existentially quantified variable $z$ in $\sigma$, $i \in [k]$, and $\pi' \in \posvar{\alpha_i}{z}$, there is a \emph{special} edge $(\pi,\pi') \in E$.
\end{itemize}
%
We further need to define when a predicate is reachable from another predicate. 
%
Given predicates $R,P \in \sch{\dep}$, {\em $P$ is reachable from $R$ (w.r.t.~$\dep$)} if $R = P$, or there exists a path in $\depg{\dep}$ from a position of the form $(R,i)$ to a position of the form $(P,j)$.
%
%we write $R \ra_\dep P$  if $R = P$, or there exists a TGD $\sigma \in \dep$ such that $R$ occurs in $\body{\sigma}$ and $P$ occurs in $\head{\sigma}$. We say that {\em $P$ is reachable from $R$ (w.r.t.~$\dep$)}, denoted $R \reach{\dep} P$, if (i) $R \ra_\dep P$, or (ii) there exists $T \in \sch{\dep}$ such that $R \reach{\dep} T$ and $T \ra_\dep P$.
%in $\depg{\dep}$, denoted $R \reach{\dep} P$, if there exists a path in $\depg{\dep}$ from a position $(R,i)$ to a position $(P,j)$, for some $i \in [\arity{R}]$ and $j \in [\arity{P}]$.
Given a database $D$, we say that a (not necessarily simple and possibly cyclic) path $C$ in $\depg{\dep}$ is \emph{$D$-supported} if there exists an atom $R(\bar t) \in D$ and a node of the form $(P,i)$ in $C$ such that $P$ is reachable from $R$.
%
We are now ready to recall (non-uniform) weak-acyclicity.



\begin{definition}\label{def:dwa}
	Consider a database $D$ and a set $\dep$ of TGDs. We say that $\dep$ is {\em weakly-acyclic w.r.t.~$D$}, or {\em $D$-weakly-acyclic}, if there is no $D$-supported cycle in $\depg{\dep}$ with a special edge. 
	%
	We say that $\dep$ is {\em weakly-acyclic} if there is no cycle in $\depg{\dep}$ with a special edge. \hfill\markfull
\end{definition}


\smallskip

\noindent
\textbf{Characterizing the Finiteness of the Chase.}
It is not very difficult to show that whenever a set $\dep$ of TGDs (not necessarily linear) is $D$-weakly-acyclic, then the instance $\chase{D}{\dep}$ is finite. In other words, the $D$-weak-acyclicity of $\dep$ is a sufficient condition for the finiteness of $\chase{D}{\dep}$. What is more interesting is that, assuming that $\dep$ is a set of simple-linear TGDs, the $D$-weak-acyclicity of $\dep$ is also a necessary condition for the finiteness of $\chase{D}{\dep}$. This leads to the following characterization established in~\cite{CaGP22}:

\begin{theorem}\label{the:characterization-simple-linear}
	Consider a database $D$ and a set $\dep \in \class{SL}$ of TGDs. It holds that $\chase{D}{\dep}$ is finite iff $\dep$ is $D$-weakly-acyclic.
\end{theorem}

For linear TGDs, it turned out that non-uniform weak-acyclicity is not powerful enough for characterizing the finiteness of the chase instance. Here is an example given in~\cite{CaGP22} that illustrates this fact:
%This is illustrated by the following example.


\begin{example}
	Consider the database $D = \{R(a,b)\}$ and the singleton set $\dep$ consisting of the (non-simple) linear TGD
	\[
	R(x,x)\ \ra\ \exists z \, R(z,x). 
	\]
	It is easy to see that there is no trigger for $\dep$ on $D$. This means that $\chase{D}{\dep} = D$ is finite, whereas $\dep$ is {\em not} $D$-weakly-acyclic. \hfill\markfull
\end{example}


To obtain a characterization analogous to Theorem~\ref{the:characterization-simple-linear}, the authors of~\cite{CaGP22} used the technique of {\em simplification} to convert linear TGDs into simple-linear TGDs, while preserving the finiteness of the chase instance. We proceed to recall this technique.
%
Let $\bar t = (t_1,\ldots,t_n)$ be a tuple of (not necessarily distinct) terms. We write $\unique{\bar t}$ for the tuple obtained from $\bar t$ by keeping only the first occurrence of each term in $\bar t$.
%
For example, if $\bar t = (x,y,x,z,y)$, then $\unique{\bar t} = (x,y,z)$.
%
For each $i \in [n]$, the \emph{identifier of $t_i$ in $\bar t$}, denoted $\id{\bar t}{t_i}$, is the integer that identifies the position of $\unique{\bar t}$ at which $t_i$ appears. 
%
We write $\id{}{\bar t}$ for the tuple $(\id{\bar t}{t_1},\ldots,\id{\bar t}{t_n})$.
%
For example, if $\bar t = (x,y,x,z,y)$, then $\id{}{\bar t} = (1,2,1,3,2)$.
%
For an atom $\alpha = R(\bar t)$, the {\em simplification of $\alpha$}, denoted $\simple{\alpha}$, is the atom $R_{\id{}{\bar t}}(\unique{\bar t})$, whereas the {\em shape of $\alpha$}, denoted $\shape{\alpha}$, is the predicate $R_{\id{}{\bar t}}$. We can naturally refer to the simplification and the shape of a set of atoms.
%
For a tuple of variables $\bar x = (x_1,\ldots,x_n)$, a \emph{specialization of $\bar x$} is a function $f$ from $\bar x$ to $\bar x$ such that $f(x_1) = x_1$, and $f(x_i) \in \{f(x_1),\ldots,f(x_{i-1}),x_i\}$, for each $i \in \{2,\ldots,n\}$.
We write $f(\bar x)$ for $(f(x_1),\ldots,f(x_n))$. We are now ready to recall how a set of linear TGDs is converted into a set of simple-linear TGDs.

\begin{definition}\label{def:simplification}
	Consider a linear TGD $\sigma$ of the form
	\[
	R(\bar x) \ra \exists \bar z\, \psi(\bar y,\bar z), 
	\]
	where $\bar y \subseteq \bar x$, and a specialization $f$ of $\bar x$. The {\em simplification of $\sigma$ induced by $f$} is the simple-linear TGD
	\[
	\simple{R(f(\bar x))} \rightarrow \exists \bar z\, \simple{\psi(f(\bar y),\bar z)}.
	\]
	We write $\simple{\sigma}$ for the set of all simplifications of $\sigma$ induced by some specialization of $\bar x$.
	%
	For a set $\dep \in \class{L}$ of TGDs, the {\em simplification of $\dep$} is defined as the set
	\[
	\simple{\dep}\ =\ \bigcup_{\sigma \in \dep} \simple{\sigma}
	\]
	consisting only of simple-linear TGDs. \hfill\markfull
\end{definition}

We can now recall the characterization for the finiteness of the chase instance for linear TGDs, established in~\cite{CaGP22}, which is similar to the one for simple-linear TGDs, with the key difference that first we need to simplify both the database and the set of linear TGDs:

\begin{theorem}\label{the:characterization-linear}
	Consider a database $D$ and a set $\dep \in \class{L}$ of TGDs. Then, $\chase{D}{\dep}$ is finite iff $\simple{\dep}$ is $\simple{D}$-weakly-acyclic.
\end{theorem}

It is clear that Theorems~\ref{the:characterization-simple-linear} and~\ref{the:characterization-linear} provide simple algorithms for checking whether the chase instance is finite. In particular, given a database $D$ and a set $\dep$ of simple-linear TGDs, we simply need to check whether $\dep$ is $D$-weakly-acyclic, in which case the algorithm returns \true; otherwise, it returns \false. The same holds when $\dep$ is a set of linear TGDs, with the difference that the algorithm first needs to simplify $D$ and $\dep$, and then perform the acyclicity check.
%
Our goal is to experimentally evaluate the above algorithms with the aim of understanding which input parameters affect their performance, clarifying whether they can be applied in a practical context, and revealing their performance limitations. Of course, a naive implementation of the above algorithms, especially for linear TGDs where the expensive simplification must be applied, will lead to poor performance, and thus, will not be very useful towards our goal. Hence, we need to somehow convert the above theoretical algorithms into practical algorithms that are amenable to efficient implementations. This is the subject of the next section.
% \section{Design}
\label{s:design}
In this section, we will first present the core of our system. Then we present some analysis of the system along with some extensions to address a few practical concerns. We will present details of our cloud implementation separately in the next section.

\subsection{Delivery Based Ordering}
Our solution is composed of three parts. 
\subsubsection{Delivery Clock\\}
\noindent\textbf{What we do.}
Each RB maintains a delivery clock. This delivery clock essentially tracks time relative to when market data was delivered to the participant. We use $DC(i,a)$ to represent delivery clock of participant $i$ at time when trade $(i,a)$ is submitted. Delivery clock is a lexicographical tuple.
\begin{align}
    DC(i,a) = \langle ld(i,a), S(i,a)-D(i, ld(i,a))\rangle.
\end{align}
where $ld(i,a)$ is the latest data point that was delivered to MP$_i$ by time S(i,a), i.e., $D(i,ld(i,a)) \leq S(i,a) < D(i,ld(i,a)+1)$). 
Interval, $S(i,a)-D(i, ld(i,a))$, corresponds to the time that has elapsed since the last delivery and can be measured locally at the RB without requiring any clock synchronization (challenge 1). 

\noindent
\textit{Monotonicity:} Delivery clocks advance monotonically with submission time. As a result, DBO trivially satisfies the causality condition (Equation~\ref{eq:causality}). Further, it incentivizes the participants to submit trades as early as possible. Therefore, \emph{a participant cannot gain any advantage by delaying trades.} %\pg{should this point have a heading of its own}
Finally, we also leverage the monotonic property to overcome challenge 3 (\S\ref{ss:enforcing_ordering}). Figure~\ref{fig:delivery_clock} shows how delivery clock advances with time.

%\pg{I tried to reduce the notation here. I defined delivery clock slightly differently.}

\begin{figure}[t]
\centering
    \includegraphics[width=0.8\columnwidth]{figures/delivery_clock.pdf}
    \caption{\small{\bf Delivery Clock.}}% \pg{Redraw}}% \pg{Eashan see Ranveer's comment}}% \pg{Eashan can you redraw this figure in powerpoint or something.}}}
    \label{fig:delivery_clock}
    \vspace{-2.5mm}
\end{figure}

All incoming trades are marked with the delivery clock at the trade submission time. The ordering buffer uses this delivery clock time to order trades. Formally, the ordering in DBO is given by,  

\vspace{-2mm}
\begin{align}
    O(i,a) = DC(i, a). 
    \label{eq:ordering_with_dc}
\end{align}


\begin{figure}[t]
\centering
    \includegraphics[trim={0 0 0 2mm},clip,width=0.8\columnwidth]{figures/dbo_correct.pdf}
    \vspace{-4mm}
    \caption{\small{{\bf DBO can help correct for late delivery of data.} Delivery of market data to MP$_i$ is lagging behind MP$_j$. There are two trades $(i,a)$ and $(j,b)$ generated in response to the same market data $x$. $(j,b)$ was submitted before $(i,a)$ but
    %, i.e., $S_j(l) < A_i(k)$. 
    response time of $(i,a)$ is less than $(j,b)$.
    %, i.e., $rt_i(k) < rt_j(l)$. 
    In this example, $DC(i,a) (= \langle x, RT(i,a)\rangle) < DC(j,b) (= \langle x, RT(j,b)\rangle)$ and trade $(i,a)$ is correctly ordered ahead of $(j,b)$.}} %Ordering based on the submission time leads to incorrect ordering.}
    %\pg{Correct figure}}
    \label{fig:dbo_correction}
    \vspace{-3mm}
\end{figure}


\noindent\textbf{Why it works.}
When the trigger point of trade $(i,a)$ is indeed the last data point (i.e., $x = TP(i,a) = ld(i, a)$), then, DBO respects condition C2 for LRTF. Figure~\ref{fig:dbo_correction} shows an illustrative example of this.
This is because, the delivery clock directly tracks the response time of $i,a$ in this case and $O(i,a) = DC(i, a) = \langle x, RT(i,a)\rangle$. For a competing trade $(j,b)$ with higher response time, the delivery clock at time of submission will either read $O(j,b) = DC(j, b) = \langle x, RT(j,b)\rangle$ (if S(j,b)<D(j,x+1)) or $DC(j, b) = \langle y, S(j,b)-D(j,y)\rangle$ with $y>x$. In both cases, $O(i,a) < O(j,b)$.


At a high level, in our ordering we are correcting for latency differences in data delivery by using the delivery time of the last data point. When the last data point is not the trigger point for trade $(i,a)$, DBO satisfies the LRTF condition C2, if the following condition holds, 
\begin{align}
    D(i,ld(i,a))-D(i,x) = D(j,ld(i,a))-D(j,x),
    \label{eq:cond_delivery_lrtf}
\end{align}
where $x = TP(i,a)$.  
While it is impossible to ensure that inter-delivery times remain the same for all participants for all points, by pacing data at the RB it is indeed possible to ensure that the above condition is always met.% \radhika{you meant C2 or the above condition?}. \pg{the above condition only}
The main reason why we can meet the above condition is that condition C2 limits that the trigger point $x$ cannot be any arbitrary data point in the past, and that the trigger point must have been delivered recently  $S(i,a)-D(i,x) < \delta$.
%and we only need to ensure same inter-delivery times for. 
In the next subsection, we will show how we can achieve this and solve challenge 2. %\pg{Is this easy to follow?}



%\pg{FIX: say delivery clocks helps correct has static differences in latency. Why are delivery clocks so good on their own, give more intuition and experimentation. Potential things to include, see 6.1. Maybe make a section of.delivery clock on its own. correct the equation here in terms of response time as well.}
%\pg{Should we include results on necessary conditions on delivery times for achieving LRTF. Maybe its a bit of an overkill.}

\noindent
\textit{Remark:} In our cloud experiments, we find that DBO achieves fairness with very high probability. This is because network latency (from CES to any given participant) exhibits temporal correlation in latency especially over  short periods of time. When temporal correlation is high, inter-delivery time at any participant is close to the inter-generation time at the CES. In such cases, condition given by Equation~\ref{eq:cond_delivery_lrtf} is satisfied with high probability.

\noindent
\textbf{Difference with traditional logical clocks:} Logical clocks are commonly used in distributed systems. The most famous ones are lamport clocks~\cite{lamportSeminalPaper} and vector clocks. These clocks can be used for achieving total causal ordering of events. While these clocks can track causality of events, they cannot be used to achieve response time fairness. In particular, these clocks don't say anything about how two competing trades generated using the same market data should be ordered as these two trades have no direct causality relation. Unlike delivery clocks, such logical clocks also have no notion of measuring time between occurrences of two events. Time difference between events is critical to achieve fairnesss for exchanges. 

\noindent\textit{Note:} Several major financial exchanges already rely on heartbeats~\cite{nyse-client} for liveness when traffic is low.


\begin{figure}[t]
\centering
    \includegraphics[width=0.8\columnwidth]{figures/batching_pacing.pdf}
    \vspace{-2mm}
    \caption{\small{\bf Batching and Pacing. Inter-delivery time for consecutive batches is equal to or more than $\delta$.}}% \pg{Redraw}}% \pg{Eashan see Ranveer's comment}}% \pg{Eashan can you redraw this figure in powerpoint or something.}}}
    \label{fig:batching_pacing}
    \vspace{-4.5mm}
\end{figure}

\subsubsection{Batching and Pacing\\}
\noindent
\textbf{What we do.}
In DBO, the CES breaks data into batches. Each new batch contains all data points in the duration $(1+\kappa) \cdot \delta$ after the previous batch. Here $\kappa > 0$. Each release buffer delivers all data points in a batch at the same time. %Two points $x,y$ belonging to the same batch are delivered simultaneously to each participant, i.e., $D(j,y)=D(j,x), \forall j$.
The release buffer delivers batches as quickly as possible while ensuring that the time between delivery of two consecutive batches is atleast $\delta$. Figure~\ref{fig:batching_pacing} shows an illustration of batching. Both batching and pacing increase the delivery time of data points. In the next subsection we will analyze the impact of the two on latency. Note that in the event of queue build up at the RB, since the batch generation rate ($\frac{1}{(1+\kappa) \cdot \delta}$) is slower than the batch dequeue rate($\frac{1}{\delta}$), the queue at the RB eventually gets drained(\S\ref{ss:understanding_latency}).


\noindent
\textbf{Why it works.} With batching and pacing, DBO achieves LRTF. In particular, 
consider a trade $(i,a)$ with response time less than $\delta$. Because of pacing, consecutive batches are separated atleast by $\delta$. This means that the trigger point ($x=TP(i,a)$) must be within the last received batch. The point $ld(i,a)$ is also the last point in this batch and $D(i,ld(i,a)) = D(i,x)$. \emph{With batching and pacing, the delivery clock again directly tracks the response time of $(i,a)$} and $O(i,a) = DC(i,a) = <ld(i,a), RT(i,a)>$.
With batching, for participant $j$, $x$ and $ld(i,a)$ also belong to the same batch $D(j,ld(i,a)) = D(j,x)$.
For a competing trade $(j,b)$ with higher response time, the delivery clock at the time of submission will either read $O(j,b) = DC(j,b)) = \langle ld(i,a)), RT(j,b)\rangle$ (if $(j,b)$ was submitted before the next batch, i.e., $S(j,b) < D(j,ld(i,a)+1)$) or $DC(j, b) = \langle y, S(j,b)-D(j,y)\rangle$ with $y>ld(i,a)$. In both cases, $O(i,a) < O(j,b)$.

\if 0
\begin{figure}[t]
\centering
    \includegraphics[width=0.8\columnwidth, angle = -90]{images/pq_hb.jpg}
    \vspace{-2.5mm}
    \caption{\small{\bf Enforcing the ordering.} \pg{Redraw}}% \pg{Eashan see Ranveer's comment}}% \pg{Eashan can you redraw this figure in powerpoint or something.}}}
    \label{fig:pq_hb}
    \vspace{-2.5mm}
\end{figure}
\fi

\subsubsection{Enforcing the ordering\\}
\label{ss:enforcing_ordering}
OB contains a priority queue where all incoming trades are sorted based on the delivery clock timestamp (Equation~\ref{eq:ordering_with_dc}). A trade $(i,a)$ at the head of the priority queue should be forwarded to the CES only when the OB has received all trades $(j,b)$ with lower ordering $DC(j,b) < DC(i,a)$. 

\noindent
\textit{OB's Heartbeat Handler:} In DBO, each RB sends a heartbeat periodically every $\tau$ seconds to the CES. The heartbeat $(i,h)$, from participant $i$ contains the delivery clock timestamp at the time the heartbeat was generated ($DC(i,h)$). Since data in delivered in order and because delivery clock advances monotonically with time, heartbeat $(i,h)$ tells the OB that it has received all trades from participant $i$ with delivery clock less than $DC(i,h)$. The ordering buffer forwards trade $(i,a)$ if it has received heartbeats from all the participants with delivery clock timestamp higher than $DC(i,a)$. 


\subsection{Understanding DBO}

\subsubsection{Latency, parameter setting and straggler mitigation\\}
\label{ss:understanding_latency}

We will first derive the optimal latency for any ordering system that achieves response time fairness. We will then discuss how DBO compares to  optimal latency. We will also present guidelines for setting parameters and how to mitigate stragglers that can impact latency.

We define latency for trade $(i,a)$, $L(i,a)$, as the sum of latency in delivering data (from generation time) and latency in trade forwarding to the CES (from trade submission time). Formally,
\begin{align}
    L(i,a) = (D(i, x) - G(x)) + (F(i,a) - S(i,a)),\nonumber\\
    L(i,a) = F(i,a) - G(x) - RT(i,a),
    \label{eq:latency_def}
\end{align}
where $x=TP(i,a)$.

\noindent
\textbf{Optimal Latency:} Formally trade $(i,a)$ can only be forwarded to the CES's ME only when the CES has received all potential competing trades $(j,b)$ with lower response times ($RT(j,b) < RT(i,a)$). Let $R(i, x, RT)$ represent the time when the CES receives trade $(i,a)$ whose whose trigger point is x and response time is RT. Formally, 
\begin{align}
    F(i,a) = \max_{j}(R(j, x=TP(i,a), RT=RT(i,a))). 
\end{align}
A subtle point to note here is that even if participant $j$ does not produce any trades, we still need to wait for that participant till $R(j, x=TP(i,a), RT(i,a))$. Before this time, fundamentally the CES cannot be sure that it will not receive a trade from participant $j$ with a lower response time. 

We use $RTT(i, x, RT)$ to represent the sum of raw network latency for point x from CES to MP $_i$ and latency of trade from MP$_i$ to the CES (whose trigger point is x and response time RT).  In the best case scenario for latency (no buffering at any point in the path) we get
\begin{align}
    R(i, x, RT) = G(x) + RTT(i, x, RT) + RT.
\end{align}


Using the above two equations, we can write the following theorem.
\begin{theorem}
For any ordering system that achieves response time fairness, the minimum latency for trade $(i,a)$ is given by,
\begin{align}
    L(i,a) = \max_{j}(RTT(j, x=TP(i,a), RT=RT(i,a))).
\end{align}
\vspace{-2mm}
\label{thm:latency}
\end{theorem}

Put it simply, the above theorem states for achieving response time fairness, the minimum latency is bounded by the maximum round trip time across all participants. This means that fundamentally bad latency for a participant affects the latency of all trades. To achieve low latency consistently, we would like to ensure that latency of all the participants is well behaved majority of the times. How to better achieve this goal is left as a subject for future work.

%This theorem implies that even in cloud settings exchanges should ask for  network latency  

%With a very large number of participants thus pose a 
%\pg{fundamental issue with scalability}

\noindent
\textbf{How does DBO compare with the optimal?} DBO achieves close to optimal latency.  Compared to the optimal, batching and pacing introduce additional delay in delivery of market data points.  Since heartbeats are  generated only periodically they can  introduce an additional delay of $\tau$ at the ordering buffer. We now discuss the delay due to each of these components and how do the parameters $\kappa$, $\delta$ and $\tau$ affect latency. %\pg{Include a table here for parameters?}

\noindent
\textbf{Impact of batching:} Batching can introduce an additional delay of $(1+\kappa)\cdot \delta$ in the worst case. 

\noindent
\textit{Setting $\delta$:} $\delta$ thus presents a trade-off between latency and fairness (how large of a horizon can we pick). The right trade-off really depends on the needs of the exchange. Ideally, the exchange should pick the minimum value of $\delta$ that accommodates the response time of the fastest participants in a race. Our conversations reveal that fastest participants typically respond within a few microseconds and majority of the speed races last 5-10 $\mu s$. For our cloud experiments we  use $\delta = 20 \mu s$.

\begin{figure}[t]
    \centering
    \includegraphics[trim={0 0 0 0mm},clip,width=0.8\linewidth]{images/latency_b+p.pdf}
    \vspace{-5mm}
    \caption{\small{\textbf{Latency in data delivery:} x-axis shows the generation time of the market data. y-axis shows the latency from generation time to data delivery. $\kappa$  governs the average slope of the orange line immediately after latency spike (slope = $\frac{\kappa}{1+\kappa}$}).} %\pg{Include orange line and the base latency. Change labels to DBO and direct-delivery. Slope is $\kappa/(1+kappa)$}}
    %\pg{Eashan: Include the drain rate, make the colored lines thicker and use different linestyles for the three schemes..}}% \pg{Maybe label the drain rate in the figure for S1 and S2.}}
    \label{fig:latency_b+p}
    \vspace{-5mm}
\end{figure}

\noindent
\textbf{Impact of pacing.} Pacing restricts the batch dequeue rate at the RB. When network latency to a participant is not varying, the batch arrival/enqueue rate at the RB ($\frac{1}{(1+\kappa) \cdot \delta}$) is higher than the batch dequeue rate limit ($\frac{1}{\delta}$) and there is no queue build up. However, when network latency to a participant is decreasing (e.g., after a latency spike), batch arrival rate at the RB can exceed the dequeue rate limit leading to a queue build up. The overall queue - dequeue rate can be given by $\text{batch size} \cdot \text{batch rate limit} = 1+\kappa$. Figure~\ref{fig:latency_b+p} shows the impact of batching and pacing on latency in delivery of data in the event of a queue build up. The figure also shows the latency when data is delivered directly (raw network latency). The smaller sawtooths in the batching + pacing are because of batching. The deviation in direct delivery and batching + pacing is because of the rate limit imposed by pacing.

\noindent
\textit{Setting $\kappa$:} Increasing $\kappa$ increases batching delay but also increases the queue drain rate in the event of queue build up due to tail latency spikes. Increasing $\kappa$ thus presents a trade-off between reducing tail latency and increasing average latency. In our experiments we use $\kappa = 0.25$.
 
\noindent\textbf{Impact of heartbeats:} Heartbeats present a trade-off. Too frequent heartbeats can overwhelm the network, the ordering buffer or the release buffer. 
Infrequent heartbeats can increase the time OB has to wait of the participants. In particular, hearbeats can introduce an additional wait time of $\tau$. Note that the number of heartbeats, the OB needs to process increases linearly with the number of participants. In the next section we show how the heartbeat handler can be sharded for scalability.

\noindent\textit{Setting $\tau$:} Ideally we want to pick as low of a value as possible for the heartbeats without overwhelming the system. This number is very much dependent on the capabilities of the network and the processing power of the RB and the OB. In our cloud implementation we use $\tau = 20 \mu s$.

\noindent\textit{A note on latency:} When the network latency to participants is not varying with time, there is no queue build up at the release buffers. In such cases, DBO adds maximum of $((1+\kappa)\cdot \delta) + \tau$ additional latency over the optimal.

\noindent\textbf{Straggler Mitigation and RB/MP failure} In the event a  participant or release buffer crashes, DBO can stall processing trades. Further, the overall system latency also gets impacted when a certain participant is experiencing unusually high network latency (see Theorem~\ref{thm:latency}). Here we have the option to wait for the delayed participant and take a latency hit but not let the fairness be impacted. Ideally, we want to let the system continue with low latency with only the affected participant incurring unfairness. In DBO, we use a simple strategy to mitigate this. Using the heartbeats and the generation time of data points, the OB tracks the round trip latency to each participant. If this latency goes beyond a certain threshold for a participant, then the OB does not wait for heartbeats from such straggler participant before forwarding trades. When the round trip latency goes down, OB again starts waiting for heartbeats from the straggler. In the event of crashes, OB might not hear any heartbeats. If the OB does not hear a heartbeat from a particular participant for the above threshold, then it concludes that round trip latency exceeds the threshold and the OB deems the participant a straggler. 
 
\noindent\textit{OB failure:} In the event, the OB crashes all trades in the priority queue will be lost. System will incur unfairness in such cases. 

%The above strategy is also helpful in controlling overall system latency when a certain participant is experiencing unusually high network latency.


\subsubsection{Is Batching and Pacing necessary?\\}
\textbf{Batching and pacing contribute delays; are they necessary?} The answer is yes. Similar to Lemma~\ref{lemma:inter_delivery_imp}, we can derive the necessary conditions for achieving LRTF. 
\begin{corollary}
When trigger points are unknown, the \textit{necessary} conditions on the delivery processes for achieving response time fairness with any ordering system is given by,
\vspace{-1mm}
\begin{align*}
    \text{If }  D(i,y) - D(i,x) &< \delta, \text{ then},\nonumber\\
    D(i, y) - D(i,x) &= D(i,y) - D(i,x), & \forall i,j.
\end{align*}
\label{cor:inter_delivery_lrtf}
\vspace{-6mm}
\end{corollary}

\begin{proof}
Please see Appendix~\ref{app:cor_inter_delivery_lrtf}.
\end{proof}
\vspace{-1mm}
In contrast to Lemma~\ref{lemma:inter_delivery_imp}, the above condition states that the inter-delivery time of two points should be same across all participants only if they are separated by less than $\delta$ for some participant. Batching and pacing indeed satisfies this, for two points x and y in a batch, the inter-delivery times across all participants is indeed zero and hence equal. For point $x$ and $y$ belonging to different batches, since the inter-delivery time is greater than $\delta$ across all participants, there is no additional contraint on inter-delivery times being equal.
 
\subsubsection{Impact of RB to MP latency\\}
In scenarios where RB and the participant cannot be colocated, DBO can incur unfairness. If this latency is unbounded, then, it might be impossible to achieve fairness. If latency is bounded, however, then DBO provides the following fairness guarantees.

\begin{theorem}
    If round trip network latency from release buffer $i$ to it's corresponding participant is bounded between $B_l(i)$ and $B_h(i)$, then, DBO achieves the following guarantee for ordering trades.
    \begin{align*}
    C3: &\text{ if } TP(i,a)= TP(j,b) = x\\ 
    &\land RT(i,a) < RT(j,b) - (B_h(i)-B_l(j)), \\
    & \land RT(i,a) < \delta - B_h(i),\\
    &\text{ then, }O(i,a) < O(j,b).
\end{align*}
    \label{thm:rb_to_mp_latency}
    \vspace{-5mm}
\end{theorem}

\vspace{-1mm}
\begin{proof}
See Appendix~\ref{app:rb_to_mp_latency}.
\end{proof}
\vspace{-1mm}

Compared to LRTF, the above condition reduces the bound on response time for the faster trade $(i,a)$ to $\delta - B_h(i)$.
Additionally, the above condition states that trades are ordered fairly only when the response time of the faster trade is lower than the response time of the competing trade by atleast the variability in latency ($B_h(i)-B_l(j)$). This theorem essentially states that when RB and MP cannot be colocated, for better fairness we should ensure that latency between them is both consistent (across participants) and the upper bound is small.



\subsubsection{Impact of Losses\\}

Although infrequent, packet losses can occur in cloud environments. Such losses can impact fairness in DBO. However, only the fairness for trades that are lost and trades  whose trigger point is lost is impacted (see Appendix~\ref{app:impact_losses}).



\if 0
\subsubsection{Excessive queing at RB and OB\\}
\pg{This can be cut?}

Even though DBO employs straggler mitigation to limit the latency at the OB, it can build up a large queue if it receives a very large number of trades (little's law). The RB can also overflow in scenarios where the network latency is decreasing (Figure~\ref{fig:latency_b+p}) for a large period of time. 

\noindent
\textbf{RB:} In the event a release buffer's queue fills up (exceeds a certain threshold), to avoid overflow the release buffer forgoes pacing and starts releasing data as fast as possible to reduce the queue. In such cases, the delivery clock advances faster than as dictated by pacing. As a result, trades from such a participant might unfairly get ordered behind. The fairness for trades from other participants remains unaffected. When the queue goes down the RB resumes normal operation.

\noindent
\textbf{OB overflow:} In the event the order buffer's queue fills up, the OB starts releasing trades as fast as possible without waiting for heartbeats from participants. Once the queue goes down, OB resumes normal operation. In such cases, fairness of all trades are impacted. 
\fi

\subsubsection{Thwarting front-running attacks\\}

%Monotonicity of delivery clocks ensures that participants are incentivized to submit trades as early as possible and delaying trades does not offer any competitive advantage.% and participants are incentivized to be honest.
There is a front-running attack possible in our system. In particular, if a participant receives a market data point $x$ through some other way before RB delivers the data point $x$ to the participant then the participant has a competitive advantage. This scenario (though unlikely) is still possible. 

A simple to avoid this is to limit that a participant cannot talk to anyone beyond the CES. 
%\pg{External participants}
However, we would like the participant machine to use other  ``helper'' machines in the cloud, e.g.,  to aid computation. We also want to allow the participants to be able to talk to machines outside the cloud, e.g., to get a news stream. %stream.%\footnote{Participants use external news streams update trading strategies and make trading decisions.} 

In Appendix~\ref{app:front_running}, we show how we can prevent such front running attacks. In our solution, the participant and its helpers cannot communicate with any other participants or their helpers using the cloud network. 
To prevent scenarios where a participant uses a proxy machine outside the cloud to send market data to other  participants (faster than the network), we precisely add additional latency for data being sent outside the cloud.
While our solution introduces latency for data going out, the latency of speed trades remains unaffected.

\if 0

While monotonicity of delivery clocks ensure that participants are incentivized to submit trades as early as possible an delaying trades does offer any competitive advantage, there is still a potential front-running attack possible in our system. In particular, if a participant receives a market data point $x$ through some other way before RB delivers the data point $x$ to the participant then it has a competitive advantage. This scenario though unlikely is still possible.
A simple to avoid this is to limit that participant cannot talk to anyone beyond the CES. 

However, we would like the participant machine to use other  ``helper'' machines in the cloud to aid computation. We also want to allow the participants to be able to talk to machines outside the cloud. Participants do use external news streams and feeds from other exchanges to update trading strategies and make trading decisions. We will discuss fairness with respect to such streams shortly.  

Allowing such communication naively can lead to attacks.
By restricting communication, it is possible to ensure that no participant gets early access to market data %(at the cost of introducing latency in messages from the front-end to helpers outside the cloud)
and thwart such front-running attacks. 

%
%\pg{Which of two alternatives is better?}
%
To this end, we impose two simple constraints on communication. \begin{enumerate*}[label=(\arabic*)]\item A participant machine and its helper machines can communicate with each other freely but they cannot communicate with any other machines in the cloud. This restriction can be imposed easily by cloud providers today using security groups. This restriction ensures that a participant machine cannot get market data from other participant machines in the cloud directly. Next, we will ensure that a participant machine cannot get an earlier market data feed from outside the cloud. 
We will do so by restricting that a participant can only send data point x out of the cloud, when x has been delivered to all participants in the cloud. This way, market data points can only be available outside the cloud once they have been delivered to all the participants.
\item The helper machines cannot send data outside the cloud. Any data (excluding the trade orders) from a participant being sent outside the cloud is tagged by the delivery clock at the RB and buffered at a gateway. The data sent by the participant could potentially be a market data point with id less than or equal to the last point id (first tuple) of the delivery clock time stamp. The gateway thus buffers this data until it is sure that the all data points with id less than the last data point id in the delivery clock time stamp have been delivered. For this purpose, RB's periodically communicate their delivery clock to the gateway. 
%
%A simple way to achieve this is for each RB to send other RBs periodic beacons communicating the status of its delivery clock. This way each RB can maintain a lower bound on the delivery clocks at other RBs. 
\end{enumerate*}
\pg{include this? a bit hand-wavy and not clean. There is one challenge to be solved though. If data delivery to a particular participant is straggling then the gateway buffer can get bloated. It is not necessary for the gateway to wait for such straggler if we disable the incoming data to the straggler. The gateway can identify such stragglers and then disable any data coming from outside the cloud.}

Note that the above solution adds additionaly latency for data being sent outside the cloud. However, the latency of speed trades remains unaffected.
%There are other ways to thwart front-running that impose weaker restrictions on communication or are easier to implement. We chose to present this one for its simplicity.


\fi



\subsubsection{Limtations of DBO: Fairness beyond LRTF\\}
\label{ss:beyond_fairness}

With DBO, it is not guaranteed that trades that do not directly follow the LRTF model (Theorem~\ref{thm:1} and Equation~\ref{eq:cm})are ordered fairly. However, DBO still ensures that fairness for the most latency-sensitive speed trades. While ensuring guaranteed fairness for trades that do not follow the might be impossible, we will discuss potential some solutions.


%This will impose some system challenges. Another challenge is that different participants might be requesting different external streams. 
%


\noindent\textbf{Trades with response time > $\delta$:} DBO does not provide any guarantees for trades with response time greater than $\delta$. %If the inter-delivery times for batches across participants are same then DBO provides response time fairness for such trades. Again achieving the same inter-delivery times for all the batches is impossible. 
In case we have access to synchronized clocks, we can try and ensure (to the extent possible) that batches are indeed delivered at the same time across participants. 
When batches are delivered simultaneously, delivery clocks also get synchronized and DBO simply orders trades in the order of submission time. DBO thus ensures better fairness for such trades (when data is delivered simultaneous) while always guaranteeing LRTF. %\pg{Is this clear?}


%Regardless of whether using clocksync or not for deliverying the data, the performance of DBO for such trades is comparable to 


\noindent\textbf{Generalized compute model for trades:} A trade's submission time might be governed by delivery times of multiple data points. Again in such cases if we have access to synchronized clocks, we can try and ensure simultaneous delivery to the extent possible and achieve better fairness for such trades.


\noindent\textbf{External data streams:} In theory, external data streams like news events or market data from a competing exchange can trigger speed races. While DBO does not delay delivery of such streams to the participants (Appendix~\ref{app:front_running}), as described it does not guarantee fairness with respect to such streams. Existing exchanges do not provide any simultaneous delivery guarantees with respect to such external streams. Such streams typically traverse the internet, and the variability is network latency is substantially higher (order of milliseconds) than the market data stream (order of microseconds). Potentially, the exchange can serialize such external streams with the market data stream and ensure LRTF with respect to such a super stream. Such a serialization might not be trivial. Participants are requesting different data streams. We need to think carefully about what constitutes a fair serialization.
%\pg{Talk about how  further system challenges.}


%\subsubsection{\pg{Miscellaneous, do if time:}}
%\pg {Radhika advidce here would be helpful}

%\pg{1. Impact of clock drift rate, 3. Is batching and pacing necessary 4. Discussion, sharding for scalability, a separate RB for each asset class}













\if 0

\subsubsection{Delivery Clock\\}
Each RB maintains a delivery clock. This delivery clock essentially tracks time relative to when market data was delivered to the participant. We use $DC(i,t)$ to represent deliver clock of participant $i$ at time $t$. Delivery clock is a lexicographical tuple.
\begin{align}
    DC(i,t) = \langle ld(i,t), t-D(i, ld(i,t))\rangle.
\end{align}
where $ld(i,t)$ is the latest data point that was delivered to MP$_i$ at time t.% (i.e., $D_i(x_l(t)) \leq t < D_i(x_l(t)+1)$). 
Interval, $t-D(i, ld(i,t))$, corresponds to the time that has elapsed since the last delivery and can be measured locally at the RB without requiring any clock synchronization (challenge 1). Delivery clock advance monotonically with time. This property will help us overcome challenge 3 and also guard us against certain attack. (\pg{forward pointers}). Figure~\ref{fig:delivery_clock} shows how delivery clock advances with time.

\begin{figure}[t]
\centering
    \includegraphics[width=0.8\columnwidth]{images/delivery_clock.jpg}
    \vspace{-2.5mm}
    \caption{\small{\bf Delivery Clock.} \pg{Redraw}}% \pg{Eashan see Ranveer's comment}}% \pg{Eashan can you redraw this figure in powerpoint or something.}}}
    \label{fig:delivery_clock}
    \vspace{-2.5mm}
\end{figure}

All incoming trades are market with the delivery clock at the trade submission time. The ordering buffer uses this delivery clock time to order trades. Formally, the ordering in DBO is given by,  

\begin{align}
    O(i,a) = DC(i, S(i,a)). 
    \label{eq:ordering_with_dc}
\end{align}


\begin{figure}[t]
\centering
    \includegraphics[trim={0 0 0 2mm},clip,width=0.9\columnwidth]{hotnets-images/time series visualization (3).pdf}
    \vspace{-3mm}
    \caption{\small{{\bf DBO can help correct for late delivery of data.} Delivery of market data to MP$_i$ is lagging behind MP$_j$. There are two trades $(i,a)$ and $(j,b)$ generated in response to the same market data $x$. $(j,l)$ was submitted before $(i,k)$ but
    %, i.e., $S_j(l) < A_i(k)$. 
    response time of $(i,k)$ is less than $(j,l)$.
    %, i.e., $rt_i(k) < rt_j(l)$. 
    With DBO, $O(i,a) (= \langle x, RT(i,a)\rangle) < O(j,b) (= \langle x, RT(j,b)\rangle)$ and trade $(i,a)$ is correctly ordered ahead of $(j,b)$.} %Ordering based on the submission time leads to incorrect ordering.}
    \pg{Correct figure}}
    \label{fig:dbo_correction}
    \vspace{-4mm}
\end{figure}


When the trigger point of trade $(i,a)$ is indeed the last data point (i.e., $x = TP(i,a) = ld(i, S(i,a))$), then, DBO respects condition C2 for LRTF. Figure~\ref{fig:dbo_correction} shows an illustrative example of this.
This is because $O(i,a) = DC(i, S(i,a)) = \langle x, RT(i,a)\rangle$. For, a competing trade $(j,b)$ with higher response time, the delivery clock at time of submission will either read $O(j,b) = DC(j, S(j,b)) = \langle x, RT(j,b)\rangle$ (if D(j,x+1)>S(j,b)) or $DC(j, S(j,b) = \langle y, S(j,b)-D(j,y)\rangle$ with $y>x$. In both cases, $O(i,a) < O(j,b)$.


\noindent
\t
At a high level, in our ordering we are correcting for latency differences in data delivery by using the delivery time of the last data point. When the last data point is not the trigger point for trade $(i,a)$, DBO satisfies the LRTF condition C2, if the following condition holds, 
\begin{align}
    D(i,ld(i,t))-D(i,x) = D(j,ld(i,t))-D(j,x),
    \label{eq:cond_delivery_lrtf}
\end{align}
where $x = TP(i,a)$.  
While it is impossible to ensure that inter-delivery times remain the same for all participants for all points, by pacing data at the RB it is indeed possible to ensure that the above condition is always met. 
The main reason why we can do so is thaat condition C2 limits that the trigger point $x$ cannot be any arbitrary data point in the past ($S(i,a)-D(i,x) < \delta$).
%and we only need to ensure same inter-delivery times for. 
In the next subsection, we will show how we can achieve this and solve challenge 2. \pg{Is this easy to follow?}

\pg{Should we include results on necessary conditions on delivery times for achieving LRTF}

\noindent
\textit{Remark:} In our cloud experiments, we find that DBO achieves fairness with very high probability. This is because network latency (from CES to any given participant) exhibits temporal correlation in latency especially over  short periods of time. When temporal correlation is high, inter-delivery time at any participant is close to the inter-generation time at the CES. In such cases, condition given by Equation~\ref{eq:cond_delivery_lrtf} is satisfied with high probability.

\begin{figure}[t]
\centering
    \includegraphics[width=0.8\columnwidth]{images/batching_pacing.jpg}
    \vspace{-2.5mm}
    \caption{\small{\bf Batching and Pacing.} \pg{Redraw}}% \pg{Eashan see Ranveer's comment}}% \pg{Eashan can you redraw this figure in powerpoint or something.}}}
    \label{fig:batching_pacing}
    \vspace{-2.5mm}
\end{figure}

\subsubsection{Batching and Pacing\\}
In DBO, the CES breaks data into batches. Each new batch contains all data points in the duration $(1+\kappa) \cdot \delta$ after the previous batch. Here $\kappa > 0$. Each release buffer delivers all data points in a batch at the same time. %Two points $x,y$ belonging to the same batch are delivered simultaneously to each participant, i.e., $D(j,y)=D(j,x), \forall j$.
The release buffer delivers batches as quickly as possible while ensuring that the time between delivery of two consecutive batches is atleast $\delta$. Figure~\ref{fig:batching_pacing} shows an illustration of batching. Both batching and pacing increase the delivery time of data points. In the next subsection we will analyze the impact of the two on latency. Note that since $\kappa > 0$ batch generation rate is slower than batch drain rate and build up queue because of pacing will eventually get drained. 



With batching and pacing, DBO achieves LRTF. In particular, 
consider a trade $(i,a)$ with response time less than $\delta$. Because of pacing, batches are separated by $\delta$. This means that the trigger point ($x=TP(i,a)$) must be within the last received batch. The point $ld(i,S(i,a))$ is also the last point in this batch and $D(i,ld(i,S(i,a)) = D(i,x)$. $O(i,a) = DC(i,S(i,a)) = <ld(i,S(i,a)), RT(i,a)>$.
With batching, for participant $j$, $x$ and $ld(i,S(i,a))$ also belong to the same batch $D(j,ld(i,S(i,a)) = D(j,x)$.
For, a competing trade $(j,b)$ with higher response time, the delivery clock at the time of submission will either read $O(j,b) = DC(j, S(j,b)) = \langle ld(i,S(i,a)), RT(j,b)\rangle$ (if $(j,b)$ was submitted before the next batch, i.e., $D(j,ld(i,S(i,a))+1) > S(j,b)$,) or $DC(j, S(j,b) = \langle y, S(j,b)-D(j,y)\rangle$ with $y>ld(i,S(i,a))$. In both cases, $O(i,a) < O(j,b)$.

\fi

\if 0
\subsection{Compute Model of the HFT Trader and Definition of Fairness}

\begin{enumerate}
    \item $MD_R(i, x):$ Receive time of market data at the gateway/RBi
    \item $TO_G(i, a):$ Generation time of trade order a by trader i
    \item $TP(i,a):$ Trigger/stimuli for trade (i,a)
    \item $RT(i,a):$ Response time of for trade (i,a) 
\end{enumerate}


\textbf{Compute Model:}
Time of generation of trade= time participant received the market point that triggered the trade + response time (or time it took to generate the trade)
\begin{equation}
    TO_G(i,a) = MD_R(i,TP(i,a)) + RT(i,a)
\end{equation}


\textbf{Perceived Fairness with respect to participant i}
If all other participants received the market data at the same time as i, then how should the trades be ordered
\begin{align*}
    \text{Trade (i,a) should be ordered ahead if}\\
    TO_G(i,a) &< MD_R(i,y) + RT(j,b)\\
    TO_G(i,a) - MD_R(i,y) &< TO_G(j,b) - MD_R(j,y)
\end{align*}
This definition states for two orders trades we need to measure time relative to event y

alternatively what if i goes into j's time domain
\begin{align*}
    &\text{Trade (i,a) should be ordered ahead iff O(i,a)<O(j,b)}\\
    MD_R(j,x) + RT(i,a) &< TO_G(j,b)\\
    TO_G(i,a)-MD_R(i,x) &< TO_G(j,b) - MD_R(j,x)
\end{align*}

Correction, relative ordering




\textbf{Achieving fairness}
There are two challenges,
\begin{outline}
    \1 How do you decide how to order these trades when TP y is unknown. \pg{Three options 1) Delivery Clocks 2) Equal RTT 3) Directly to limited fairness} \pg{Time domain: two options a) I's domain b) zero latency time doman. Fairness for trades using different data points.}
        \2 Don't know which x, recency \pg{equivalence between equal inter-delivery and correcting one way latency}
        \2 Clocks are not synced
        \2 Monotonic ordering with time
    \1 How do you enforce the ordering process. In particular, trades may take an arbitrary amount of time to reach the OB.
\end{outline}

What is the lowest RTT possible with this system?\\
Say you knew the trigger points x,y what then, \\
Say you didn't know the trigger points\\
Enforcing the ordering: key insight Enforcing an ordering at a single point is easier than controlling things at multiple RBs\\
What about trades with response time greater than delta\\


Question: Fairness wrt to external data stream

\textbf{Practical Considerations}

\begin{enumerate}
    \item Collusion attacks: Ensure that any market data point is delivered only after all participants have received it
    \item external participants: Have all participants submit trade via a dummy MP machine (we dont support fairness for such particpants)
    \item External data streams:
    \item Stragglers: 
\end{enumerate}


\textit{Correction by latency pitch}
\begin{align*}
    TO_G(i,a) - MD_R(i,y) &< TO_G(j,b) - MD_R(j,y)\\
    TO_G(i,a) - (G(y) - MD_R(i,y))) &< TO_G(j,b) +(G(y)- MD_R(j,y))
\end{align*}

\pg{Alternatively fairness in the same or equal or zero latency time domain?}
\begin{align*}
    &\text{Trade (i,a) should be ordered ahead iff O(i,a)<O(j,b)}\\
    G(x) + RT(i,a) &< G(y) + RT(j,b))\\
    TO_G(i,a) + (G(x)-MD_R(i,x)) &< TO_G(j,b) + (G(y) - MD_R(j,y))
\end{align*}


\textbf{Final Pitch Attempt}
\begin{enumerate}
    \item Introduce generalized compute model
    \item Talk about zero latency model for fairness. Three problems clocksync, which x to use, how to enforce ordering. \pg{Introduce C1 from strong fairness here?}
    \item clocksync: We are interested in competing trades that are generated using the same data point \pg{is clocksync really necessary to force this}
    \item which x to use: the last x since trades are fast. What about latency for trades with response time greater than delta
    \item how to enforce ordering: monotonic ordering process \pg{unclear if monotonic is time property is even needed (if )} 
    \item part of above? No fooling: C1 property of strong fairness
    \item \pg{Limitations: Our solution doesn't work with this model for trades generated using different data points. What about approx fairness? This is kind of nice because it talks about latency/}
\end{enumerate}
\fi
% !TEX root = main.tex
\subsection{System Overview}

\begin{figure}
    \centering
    \includegraphics[width=\linewidth]{figures/overall}
    \vspace{-2em}
  \caption{Overview of ImmTrack. It processes data from one or more mmWave radars and users' IMUs with two components: {\em IMU-assisted mmWave tracking} and {\em learning-based cross-modality trajectory association}. The association results are fed to downstream applications.}
    \label{fig:structure}
    \vspace{-1em}
\end{figure}

Fig.~\ref{fig:structure} overviews the design of ImmTrack. It consists of two components to address the above two challenges.

{\bf IMU-assisted mmWave tracking:} This component consists of three steps. First, it clusters the points in each frame's point cloud into human bodies and associates the clusters corresponding to the same user across frames. ImmTrack maintains a recurive Kalman filter \cite{feng2014kalman} to track each user's movement and uses its predicted user location as the initial centroid of the user's cluster for the clustering algorithm. This motion-aware clustering remains robust when the users encounter each other. \yimin{Compared with the simplistic mmWave-based target tracking techniques such as that included in the radar vendor's application note \cite{livshitz2017tracking}, our algorithm avoids using heuristic object detectors such as constant false alarm rate (CFAR) detector, which easily result in detection errors.} Second, to perform the cross-frame cluster association for user tracking, a deep neural network called mmClusterNet extracts the feature of each cluster produced by the clustering algorithm. The feature incorporates the shape and motion information of the point cluster, as well as the PID of a pre-matched IMU in terms of movement velocity. \yimin{Such multidimensional information improves the robustness of the cross-frame cluster association.}
%While the pre-matching may be erroneous, our evaluation shows that its benefit in assisting managing radar's inherently noisy sensing outweighs the impact of pre-matching errors.
Third, the Hungarian algorithm associates the clusters across frames in terms of their features extracted by the mmClusterNet to achieve multi-user tracking. The details are presented in \sect\ref{sec:global-tracking-design}.

{\bf Learning-based cross-modality trajectory association:} ImmTrack adopts the trajectory incorporated with velocity information as the common feature of the user's movement sensed by mmWave radar and IMU. Reasons are two-fold. First, velocity-incorporated trajectory is high-level information that summarizes the user movement and generally remains consistent between the two modalities. Second, trajectory includes both temporal and spatial information. With the temporal continuity embedded in adjacent frames, the noise flickering in single frame can be largely suppressed. After the users' trajectories are reconstructed from the mmWave and IMU tracking, ImmTrack computes an imagery representation of each trajectory, which is called {\em trace map}. Then, ImmTrack applies a Siamese neural network \cite{facenet} with convolutional layers to extract comparative features from the trace maps, which are insensitive to the relative relationship between the radar's global coordinate system and the IMU's local coordinate system.
%To address the second challenge, we design a trace map generation algorithm to overcome the influence of noise and then we utilize the Siamese Network to refrain the problem of searching the coordinate transfrom matrix in a large searching space.
Finally, a bipartite graph matching algorithm associates the mmWave and IMU tracking results in terms of the cosine similarity among the comparative features. For users with nearly identical trace maps due to say side-by-side walks or simple straight walks, gait analysis will be performed on the involved mmWave clusters and IMU traces to generate gait features for mmWave-IMU association. The details are presented in \sect\ref{subsec:association}.

Note that except the IMU trace map generation running on each user's smartphone, all other processing tasks of ImmTrack run on an edge server or a cloud server. The smartphone transmits the periodically generated trace maps to the ImmTrack server.

%combined with soft-voting scheme, the trajectory-based matching module performs associations between the Track ID and People ID (i.e., IMU ID), based on which the distance tracking and social tracking records of participants are revealed.

%It mainly consists of three modules: imu-assisted mmWave radar tracking, imu tracking and learning based cross-modality trajectory association 
%The imu-assisted tracking module is carefully designed to generate trajectories of multiple participants in a global view using the sparse mmWave radar point cloud. We embed the motion model and leverage the cross-modality information from the imu to assist the radar tracking process


%To address the first challenge, we design a tracking system that leveraging the IMU information and motion model to enrich the available information. Hence, we design a deep model to fully extract the embedded information from the point cloud.

%In ImmTrack, we take advantage of the IMU information within the radar-based tracking, greatly improves the robustness of the system while keeping the algorithm very lightwight.



%We overcome the above challenges by designing a mmWave tracking system with the help of data fusion, which can works on multiple people in long distance druing long period. Note that previous work on RF signal tracking mainly focuses  either on individual body motion tracking\cite{wu2020fingerdraw} or multiple people mootion capture in a short period\cite{wu2020mmtrack}. Then we utilize the contrastive learning approach\cite{khosla2020supervised-contra} to build a robust association scheme.

%As introduced in Section \ref{challenges}, we aim to associate two modality sensors that differs in the sensing scope (global vs individual). In our ImmTrack system, we first design an motion-aware intra-frame clustering algorithm to seperate each individual from the global view. In particular, the output of mmWave radars are 3D points which mainly show the occupancy and the radial movement of people's torso in the global coordinate, while IMU measurements present angular speed and acceleration of a certain human body part where the IMU is worn, and are in a local coordinate. 



%We validate that this additional information is very critical for robust and high-performance global tracking, as the mmWave radar point cloud is usually noisy and sparse, thus may not informative enough to keep tracking on the same person over long period The IMU tracking module estimates the trajectory of every participant that hold an IMU via double integration on the acceleration.

% {\em ImmTrack} system deploys mmWave radar to capture the global view of multiple users and the goal is to identify the ID of the clusters appeared in the mmWave point cloud by associating the clusters with the IMU sensors each user holds. 
% Therefore, we associate the cluster and IMU based on the trajectory of the cluster and the trajectory the IMU generates.

%%% Local Variables:
%%% mode: latex
%%% TeX-master: "main"
%%% End:
% !TEX root = main.tex

% \section{Detailed System Design}

\section{IMU-Assisted mmWave Tracking}
\label{sec:global-tracking-design}

%This section presents the design of global tracking by mmWave radar with certain information from the IMUs.

%and briefly presents the local tracking of a certain user using IMU.


%The processing pipeline consists of two steps. First, for each frame's point cloud, ImmTrack applies {\em motion-aware intra-frame clustering} on the point cloud to form clusters, where each cluster corresponds to a user. The clustering process is initialized with centroids predicted by Kalman filters that capture the users' motions. Section~\ref{subsec:mmWave-track-cluster} presents the first step. Then, ImmTrack applies the {\em IMU-assisted inter-frame cluster tracking} to associate the clusters in different frames that correspond to the same user, which is presented in Section~\ref{subsubsec:cluster-tracking}.Lastly, Section~\ref{subsec:imu-tracking} briefly presents the local tracking of a certain user using IMU.

% This section presents the detailed designs to implement the global tracking by the mmWave radar (Section~\ref{subsec:mmWave-track}), 
% local tracking by IMU (Section~\ref{subsec:imu-tracking}),
%  and trajectory-based association of radar's and IMUs' sensing results (Section~\ref{subsec:association}).
% Motivated by visual multi-object tracking \cite{deepsort}, where detection results are first generated frame-by-frame and then associated temporally between adjacent frames for object tracking, ImmTrack follows a similar tracking-by-detection paradigm for trajectory generation.
% Specifically, we first remove the static points that normally correspond to the background for each frame. 
% Then, we perform intra-frame point clustering algorithm on the remaining points to form clusters, in which each cluster corresponds to a user. Finally, ImmTrack associates the detected clusters among frames to generate a trajectory for each user.  
% Notably, different from existing works \cite{shuai2021millieye, wu2020mmtrack} that merely use point cloud from the radar for tracking, we fully take advantage of the information from IMUs in both of the intra-frame clustering and inter-frame association module. We validate in Section \ref{sec: preliminary-tracking} that this cross-modality information from IMU greatly benefits the tracking performance. 


%In what follows, we present the details of these two algorithms when there are no roamers and no passengers. At the end of this section, we will discuss how to handle roamers and passengers.

%The \textit{global tracking module} starts with removing static points. The static removal  is to exclude the static background and preserve the moving objects. The static points can be detected by applying doppler fourier transform on the received data, whose results are the velocity of the detected subjects. Then we will feed the processed point cloud to the \textit{Heuristic Clustering} module.

\subsection{Motion-Aware Intra-Frame Clustering}
\label{subsec:mmWave-track-cluster}

\subsubsection{Design}
The radar yields a point cloud per frame. For each frame, ImmTrack removes the static points that normally correspond to the background. 
Specifically, ImmTrack compares each point's velocity with an adaptive velocity threshold updated by the triangle histogram algorithm \cite{li1998iterative} to decide whether the point is static.
%The static points can be identified by applying Doppler Fourier transform on the received signal, whose results are the velocities of the detected objects. \yimin{By apply the  triangle histogram algorithm \cite{li1998iterative}, an adaptive threshold $v_{static}$ is calculated. The components smaller than the threshold are removed.}
%ImmTrack can count the enrolments to know the number of users, i.e., $N$.
ImmTrack adopts the $k$-means algorithm to divide the point cloud into $N$ clusters by setting $k=N$. 
Notably, the initial centroids often affect the performance of $k$-means.
ImmTrack uses the recursive Kalman filters (RKFs) \cite{feng2014kalman} to predict the initial centroids.
%RKF performs recursive estimation of the covariance matrix characterizing observation noise.
%Therefore, RKF is suitable for ImmTrack as the observation noise's prior distribution is in general uncertain and may vary with time. 
% The initial centroids often affect the performance of $k$-means. To improve the quality of the initial centroids, we use the recursive Kalman filter (RKF) \cite{feng2014kalman}, which applies recursive covariance estimation. RKF is designed to deal with the situation where the covariance matrix of the observation noise is unknown. RKF is suitable for ImmTrack because the observation noise's prior distribution is in general unknown and may vary with time, which will be detailed shortly.

%Two important variables in the clustering algorithms are the number of clusters and the  initialization of the centroids.  Therefore, specifying the two parameters can significantly increase the performance of the clustering algorithm. 

\begin{comment}
We call the module \textit{Heuristic Clustering} because we use three heuristics to boost the performance of the clustering algorithm.
\begin{enumerate}
\item The number of IMU sensors.
\item The points in the cluster should be within a limted distance from the centriod due to the limit of human body part size.    
\item For each cluster centroid's  position $C_{i,t}$ at time $t$, we use the Kalman Filter to predict the position $(C_{i,t+1})$ and use  it as the initialization of the centroids position at time $t+1$.    
\end{enumerate}
\end{comment}


ImmTrack maintains an RKF for each user's volumetric centroid. The human body's kinetic model used by RKF is as follows. Let $\vec{x}_{i,j}$ denote the state of the $i^\text{th}$ user's centroid in the $j^\text{th}$ frame, where $i \in [1, N]$ is the internal PID in the domain of RKF. Note that this PID is different from the PID of the IMU.
%We describe the state in the spherical coordinate system to avoid approximating nonlinear systems by linear models using Taylor expansion, which may bring extra errors.
We define $\vec{x}_{i,j}=[r_{i,j}, \dot{r}_{i,j}, \theta_{i,j}, \dot{\theta}_{i,j}, \phi_{i,j}, \dot{\phi}_{i,j}]^\top$, where $r_{i,j}$, $\theta_{i,j}$, and $\phi_{i,j}$ are the radial range, azimuthal and polar angles, and the overhead dot denotes the velocity. Denote by $\vec{c}_{i,j} = [\hat{r}_{i,j}, \hat{\theta}_{i,j}, \hat{\phi}_{i,j}]^\top$ the $i^\text{th}$ user's observed centroid, where the $k$-means algorithm fed with the point cloud is viewed as the observation process. By assuming that the user's velocity is constant in a frame duration (denoted by $\Delta t$), the state transition and observation models are
\begin{equation}
  \vec{x}_{i,j} = \mat{F} \vec{x}_{i,j-1} + \vec{w}_{i,j}, \quad \vec{c}_{i,j} = \mat{H} \vec{x}_{i,j} + \vec{z}_{i,j},
  \label{eq:rkf}
\end{equation}
where $\mat{F}$ is the state-transition matrix capturing the movement kinetics, $\vec{w}_{i,j}$ is the stationary process noise capturing the uncertainty of the movement, $\mat{H}$ is the observation matrix, and $\vec{z}_{i,j}$ is the non-stationary observation noise capturing the uncertainties caused by the radar's sensing noises and inaccuracy of the $k$-means algorithm. Specifically, $\mat{F} = \mathrm{diag}(\mat{A}, \mat{A}, \mat{A}) \in \mathbb{R}^{6\times 6}$, where $\mat{A} = [1, \Delta t; 0, 1]$ and $\mat{H}$ is a binary matrix that selects $r_{i,j}$, $\theta_{i,j}$, and $\phi_{i,j}$ from $\vec{x}_{i,j}$.

\begin{comment}
\begin{equation*}
  \mat{A} = \left[
  \begin{array}{ll}
    1 & \Delta t \\
    0 & 1  \\
  \end{array}\right] \quad \text{and} \quad
  \mat{H} = \left[
    \begin{array}{llllll}
      1 & 0 & 0 & 0 & 0 & 0 \\
      0 & 0 & 1 & 0 & 0 & 0 \\
      0 & 0 & 0 & 0 & 1 & 0
    \end{array}\right].
\end{equation*}
\end{comment}

Before processing the $j^\text{th}$ frame, ImmTrack uses the RKF to predict the $i^\text{th}$ user's centroid $\widetilde{\vec{c}}_{i,j}$ by $\widetilde{\vec{c}}_{i,j} = \mat{H} \mat{F} \vec{x}_{i,j-1}$, where $\vec{x}_{i,j-1}$ was obtained in the previous frame. When RKF is bootstrapped (i.e., $j=0$), ImmTrack uses the DBSCAN algorithm to obtain $\widetilde{\vec{c}}_{i,0}$. Then, ImmTrack uses $\{\widetilde{\vec{c}}_{i,j} | i \in [1, N]\}$ as the initial centroids for the $k$-means algorithm with $k=N$ to process the point cloud in the $j^\text{th}$ frame. We sequentially assign the PID of each initial centroid to the closest centroid of a cluster exclusively, forming the pseudo-identified clustering result $\{\vec{c}_{i,j} | i \in [1, N]\}$. Finally, ImmTrack uses a policy derived in \cite{feng2014kalman} to update $\vec{x}_{i,j}$ and the covariance matrix of $\vec{z}_{i,j}$, i.e.,  $\vec{x}_{i,j} = \vec{x}_{i,j-1} + \vec{K}_{i,j}\left(\vec{c}_{i,j}-\mat{H} \vec{x}_{i,j-1}\right)$ and $\mathrm{cov}(\vec{z}_{i,j}) = \mathrm{cov}(\mat{M} \vec{c}_{i,j} - \mat{F} \mat{M} \vec{c}_{i,j-1}) - \mathrm{cov}(\vec{w}_{i,j})$, where $\vec{K}_{i,j}$ is the constant Kalman gain and $\mat{M} = \left( \mat{H}^\top \mat{H} \right)^{-1} \mat{H}^\top$. We follow the approach described in \cite{basso2017kalman} to estimate $\mathrm{cov}(\vec{w}_{i,j})$ used in the above update. The update of $\mathrm{cov}(\vec{w}_{i,j})$ enables ImmTrack to adapt to dynamic sensing performance of the radar due to the position variations of users.

Note that the distance-based heuristic rule of transferring the PID of the initial centroids to the resulting centroids of the $k$-means clustering may have errors when the trajectories of two users cross in the radar's FoV. However, since the RKF is mainly used to assist better choosing the initial centroids rather than track the users, the swap of PIDs does not have long-lasting negative effect after the crossing because the models in Eq.~(\ref{eq:rkf}) are Markovian. Note that tracking the users is the subject of \sect\ref{subsubsec:cluster-tracking}.

% \begin{wrapfigure}{R}{0.2\textwidth}
%   \includegraphics[width=\linewidth]{figures/PointCloud_DB_vs_rkf}
%     \caption{Clustering results ($N=3$)}
%     \label{fig:compareRKF}
% \end{wrapfigure}
\subsubsection{Evaluation}


% \begin{figure}[htp]% [H] is so declass\'e!
%   \centering
%   \begin{minipage}{0.45\linewidth}
%   \includegraphics[width=\linewidth]{figures/demo_rkf}
%   \caption{figure caption}
%   \end{minipage}\hfill
%   \begin{minipage}{0.45\linewidth}
%   \includegraphics[width=\linewidth]{figures/demo_dbscan}
%   \caption{figure caption}
%   \end{minipage}\par
%   \vskip\floatsep% normal separation between figures
%   \includegraphics[width=0.45\linewidth]{figures/cluster_experiment}
%   \caption{figure caption}
%   \end{figure}


% \begin{figure}[!htp]
%   \begin{subfigure}{.15\textwidth}
%     {
%       \includegraphics[width=\textwidth]{figures/cluster_experiment}
%       \caption{Performance of intra-frame clustering algorithms}
%       \label{fig:cluster algorithm}
%     }
%   \end{subfigure}
%   \begin{subfigure}{.15\textwidth}
%     \includegraphics[width=\textwidth]{figures/demo_rkf}
%     \caption{Results of RKF-assisted $k$-means when $N=3$}
%     \label{fig:RKFClu}
%   \end{subfigure}
%   \begin{subfigure}{.15\textwidth}
%     \includegraphics[width=\textwidth]{figures/demo_dbscan}
%     \caption{Results of DBSCAN when $N=3$}
%     \label{fig:DBSCANClu}
%   \end{subfigure}
%   \caption{Intra-frame clustering. The proposed RKF-assisted $k$-means clustering algorithm outperforms $k$-means, DBSCAN, and GMM. It also outperforms mmTrack \cite{wu2020mmtrack} when $N \ge 4$. In (b), color represents cluster ID, cross represents centroid, and the DBSCAN yields 2 clusters for 3 users.}    
% \end{figure}

\begin{figure}
  \begin{minipage}[t]{.6\linewidth}
    \begin{subfigure}[t]{\linewidth}
      \includegraphics[width=\linewidth]{figures/cluster_experiment}
      \caption{Performance of intra-frame clustering algorithms. Baselines: $k$-means, DBSCAN, GMM, mmTrack and its variant.}
      \label{fig:cluster algorithm}
    \end{subfigure}
  \end{minipage}
  \hfill
  \begin{minipage}[t]{0.35\linewidth}
    \begin{subfigure}[t]{\linewidth}
      \includegraphics[width=\linewidth]{figures/demo_cluster}
      \caption{Results of RKF-assisted $k$-means and DBSCAN when $N=3$.}
      \label{fig:ClusterDemo}
    \end{subfigure}
  \end{minipage}
  \caption{Intra-frame clustering. The proposed RKF-assisted $k$-means clustering algorithm outperforms $k$-means, DBSCAN, and GMM. It also outperforms mmTrack \cite{wu2020mmtrack} when $N \ge 4$. In (b), color represents cluster ID, cross represents centroid, and DBSCAN yields 2 clusters for 3 users.}
  \vspace{-1em}
\end{figure}
  
% \begin{wrapfigure}{R}{0.5\textwidth}
% \begin{figure}
%   \begin{subfigure}[h]{.6\linewidth}
%     \centering
%     \includegraphics[width=\linewidth]{figures/cluster_experiment}
%     \caption{Performance of intra-frame clustering algorithms}
%      \label{fig:cluster algorithm}
%   \end{subfigure}%
%   \hfill
%   %\begin{subfigure}[t]{.15\textwidth}
%   %  \centering
%   %  \includegraphics[width=\linewidth]{figures/PointCloud_DB_vs_rkf}
%   %  \caption{Clustering results ($N=3$)}
%   %  \label{fig:compareRKF}
%   %\end{subfigure}
%   \begin{subfigure}[t]{.35\linewidth}
%     \includegraphics[width=\linewidth]{figures/demo_rkf}
%     \includegraphics[width=\linewidth]{figures/demo_dbscan}
%     \caption{Results of RKF-assisted $k$-means and DBSCAN when $N=3$}
%     \label{fig:compareRKF}
%   \end{subfigure}
%   \vspace{-0.5em}
%   \caption{Intra-frame clustering. The proposed RKF-assisted $k$-means clustering algorithm outperforms $k$-means, DBSCAN, and GMM. It also outperforms mmTrack \cite{wu2020mmtrack} when $N \ge 4$. In (b), color represents cluster ID, cross represents centroid, and the DBSCAN yields 2 clusters for 3 users.}
% \end{figure}
% %  \hfill
%\end{wrapfigure}


We compare our RKF-assisted $k$-means algorithm with a variant without RKF and several other clustering approaches including DBSCAN and Gaussian mixture model (GMM) built with the expectation-maximization (EM) algorithm. We also implement the clustering algorithm proposed in mmTrack \cite{wu2020mmtrack}. The mmTrack applies the $k$-means algorithm with random initial centroids to cluster the point cloud. During the $k$-means iterations, mmTrack uses the medoids of the clusters obtained in the previous iteration as the initial centroids of the next iteration. The mmTrack determines the value of $k$ using the silhouette analysis. In addition, we implement a variant of mmTrack's clustering algorithm by removing the silhouette analysis and directly setting $k=N$. All the above baseline approaches do not consider motion.

%\yimin{Moreover, we also implement the clustering algorithms proposed in mmTrack\cite{wu2020mmtrack}. In addition, we also implement a variation of it by replacing its number of clusters estimation module with a known number of users.  }
%The $k$-means and GMM require the prior knowledge of $N$.
% our  prior knowledge assisted intra-frame point clustering algorithms with several classical clustering algorithms, including \textit{K}-means, DBSCAN, and clustering with Gaussian Mixture Model (GMM).
%The goal is to study on the influence of the prior knowledge on the clustering process. The prior knowledge in the \textit{K}-means  and GMM clustering is the number of clusters, which we also embed in our algorithm. The GMM clustering furthur assumes each cluster follows a gaussian distribution. As for DBSCAN, it use the cluster density as the prior knowledge.
%\noindent \textbf{Metric.}


% \begin{wrapfigure}{L}{0.4\textwidth}
%   \centering
%   \includegraphics[width=\linewidth]{figures/cluster_experiment}
%   \caption{Performance of various intra-frame clustering algorithms}
%    \label{fig:cluster algorithm}
% \end{wrapfigure}


We use the Adjusted Rand Index (ARI) to measure the quality of clustering. Zero ARI indicates random guessing-like clustering, whereas ARI of one suggests perfect clustering.
We compute per-frame ARIs and report the average ARI.
During the experiment, the users follow pre-defined trajectories, so that we can obtain the ground truth. More details of the experiment setup are presented in \sect\ref{sec:eval}. From Fig.~\ref{fig:cluster algorithm},
our RKF-assisted $k$-means outperforms $k$-means, DBSCAN, and GMM. When $N \le 3$, the mmTrack and its variant with known $k$ slightly outperform our RKF-assisted $k$-means in terms of ARI. However, the advantage of our RFK-assisted $k$-means over mmTrack and its variant increases with $N$ when $N \ge 4$. \yimin{The explanations for the above results are as follows. When the occlusion cases increase due to the increase of users, our RKF-assisted $k$-means algorithm outperforms mmTrack. When there are no or limited occlusions, mmTrack's clustering algorithm performs well. However, with our RKF-assisted $k$-means algorithm, some of the points corresponding to users in the point cloud are excluded in the phase of static points removal, leading to lower ARI.}
%When $N=6$, the mmTrack and its variants are worse than GMM.
Fig.~\ref{fig:ClusterDemo} shows the clustering results of the DBSCAN and RKF-assisted $k$-means algorithms when $N=3$, respectively. DBSCAN mistakenly combines two users into a single cluster. The above results suggest that the consideration of motion improves clustering performance.
%Other than our algorithm, we find that the GMM clustering outperforms the remainings, while DBSCAN shows the worst performance. As shown in the experiment, the  accuracy of clustering  is positive related to the quality and quantity of the prior knowledge.





% \begin{figure}
%   \centering
%   \includegraphics[width=0.8\linewidth]{figures/cluster_experiment}
%   \caption{Performance of various intra-frame clustering algorithms.}
%   \label{fig:cluster algorithm}
%   % Our proposed heuristic clustering performance is slightly betterthan GMM when there is no passenger. Moreover, it shows similar performance when there are some passengers.}  
% \end{figure}



% Our clustering algorithm outperforms others under all cases regardless of the number of people and whether there is a passenger. When there is no passenger, the performance of all algorithms degrades when there are more people. 
% % Specifically, compared to the most comparable algorithm GMM, the ARI are 0.71 and 0.69 for heuristic clustering and GMM respectively when the number of people is six. 
% When there are passengers, the number of clusters is no longer equal to the number of IMU sensors. In this case, those algorithms that require the number of clusters (i.e., K-means, GMM) perform worse as the number of passengers is getting larger. However, our heuristic clustering shows an upward trend and reaches an ARI over 0.9 when there are 2 passengers out of 5 people. \textcolor{red}{[why increase? ours is also based on Kmeans]}

\begin{comment}
$\Delta t$, so the motion model (A) and the observation model (C) are defined as followings:
$$
\begin{aligned}
x_{k} &=\left[\begin{array}{cccccc}
1 & \Delta t & 0 & 0  & 0 & 0\\
0 & 1 & 0 & 0 & 0 & 0\\
0 & 0 & 1 & \Delta t & 0 & 0 \\
0 & 0 & 0 & 1 & 0 & 0\\
0 & 0 & 0 & 0 & 1 & \Delta t\\
0 & 0 & 0 & 0 & 0 & 1\\

\end{array}\right] x_{k-1}+Q, \\
y_{k} &=\left[\begin{array}{llllll}
1 & 0 & 0 & 0 & 0 & 0 \\
0 & 0 & 1 & 0 & 0 & 0 \\
0 & 0 & 0 & 0 & 1 & 0 
\end{array}\right] x_{k}+R_{k-1},
\end{aligned}
$$

For each cluster centroid's  position $C_{i,t}$ at time $t$, we use the Kalman Filter to predict the position $(C_{i,t+1})$ and use  it as the initialization of the centroids position at time $t+1$.

In each iteration, the RKF estimate the covariance of $R$ using the formula\cite{feng2014kalman}
$$
\operatorname{Cov}{(R_k)}=\operatorname{Cov}{(\xi_k)}-\operatorname{Cov}(Q)
$$
where
 $\xi_{k}=M \cdot y_{k}-A\cdot M \cdot y_{k-1}$ and $M=\left[C^{T} C\right]^{-1} C^{T}$
\end{comment}

%After the Kalman filter predicts the next position $(C_{i,t+1})$, we transform  $(C_{i,t+1})$ from  polar system to Cartesian coordinates and use it as the initialization of the centrodds for the cluster in the next frame. First of all, we apply \textit{K}-Means on the frame at time $t$ with centroids initialized position as $C_{t+1}$ and we set the number of clusters equal to the number of imu sensors $N$.


\subsection{IMU-Assisted Inter-Frame Cluster Tracking}
\label{subsubsec:cluster-tracking}

%After removing the clutters (e.g.,non-human object) and separating the  clusters, the remaining problem is the cluster association: an inherent mechanism needs to be designed to connect clusters across consecutive frames.

\subsubsection{Design}
The association of the clusters in the consecutive frames that correspond to the same user is based on {\em space coherence} and {\em motion coherence}. The former means that the shape of a moving object at close locations are similar from the radar's perspective; the latter means that the object's motions in consecutive frames are similar. We design a new deep learning-based feature extractor called {\em mmClusterNet} that fuses {\em shape}, {\em motion}, and {\em IMU PID} features of a cluster into a single {\em cluster tracking feature} for each frame.

% For tracking with the camera, rich 
% features related to the shape and coloring can be extracted from the region of interest by using deep learning. Then the tracking systems using the camera compare the similarity between the 
% features across consecutive frames to solve the cluster association.  

% On the contrary,  the existing solutions for the sparse point cloud is to rely on the location (e.g. nearest neighbor) and velocity data  which are directly obtained from measurements. 
% However, the naive solution suffers from high false-positive rate in indoor applications since people tend to gather together and the velocity is limited to a 
% small range, which makes it indistinguishable among different people.  On the other hand,  due to the sparsity of the point cloud in mmWave radar,  the shape and color features embedded in the point cloud is not reliable enough for 
% cluster association.  

% In our cluster association module, we propose two innoviations to solve the above challenges, which utilize multiple features of the point cloud including shape, location, and velocity.
% \begin{enumerate}
%     \item We leverage the velocity information in imu to assist the cluster association process and overcome the sparsity problem.
%     \item  We design the PointTrack Net (PTrack Net) to jointly embed the shape, velocity and location information in each cluster to a feature vector. 
% \end{enumerate}




%\textit{\textbf{PointTrack Net Structure}}

%\todo{add downstream task, change position encoding to imu id encoding}



\begin{figure}
  \centering
  \includegraphics[width=\linewidth]{figures/mmcluster}
  \vspace{-2em}
  \caption{mmClusterNet for fusing shape, motion, IMU PID features. The distance matrix of fused feature of clusters is used as the metric for inter-frame association and tracking. }
  \label{fig:feature_extractor}
  \vspace{-1em}
\end{figure}


% In most of the tracking systems, they focus on Kalemn Filter design or data pre-processing pipeline. For the cluster association similarity calculation, most of work use simple heuristics  such as nearest point using the euclidean distance \cite{wu2020mmtrack}. This simple approach is not robust and accurate enough because 
% \begin{enumerate}
%     \item The approach does not make full use of the existing information such as the velocity, radical velocity, number of points in the cluster, shape of the cluster.
%     \item A single heuristic can not cover all the situations. For example the nearest point solution works when the object are separated far away but will fail when the object are relatively near to each other.
% \end{enumerate} 

% Our network is designed to handle the sparsity in radar point clouds and output the feature vector for clustering association. 

Fig.~\ref{fig:feature_extractor} shows mmClusterNet's architecture. For each frame, it takes each of the clusters produced by \sect\ref{subsec:mmWave-track-cluster} as input. The mmClusterNet is designed to process a cluster with $n$ 3D points, where $n$ is fixed at the design phase. When processing a smaller cluster, ImmTrack firstly applies interpolation to generate an $n$-point cluster. For the AWR1843 mmWave radar, $n=24$ is a good setting because it is an empirical upper bound of human cluster size. 
As shown in the upper branch in Fig.~\ref{fig:feature_extractor}, each of the $n$ points is processed by a shared multilayer perceptron (MLP) with two 32-neuron hidden layers. The results of the $n$ shared MLPs are max-pooled to generate a $1\times 32$ shape feature, which is copied vertically $n$ times, concatenated with the shared MLPs' outputs and the cluster's radial velocity vector (as the motion feature) to form an $n \times 65$ tensor. Then, each row of the tensor is processed by an MLP with two 64-neuron hidden layers and max-pooling to produce a $1 \times 64$ shape-motion feature. Finally, the shape-motion feature is fused with the IMU PID feature, which is detailed shortly, by element-wise addition to produce the cluster tracking feature. To train mmClusterNet, we append a regression MLP as the downstream task that produces a bounding box of the cluster from the shape-motion feature. Then, we use manually labeled bounding boxes as ground truth to train the mmClusterNet core. In \sect\ref{compare_down_task}, we will evaluate the impact of different choices of the downstream task on mmClusterNet's performance. \yimin{Note that the training data for the mmClusterNet core is unnecessary to be {\em in situ} data. The training can be based on a public dataset such as ShapeNet.}

%\todo{As for training the network, we feed the shape-motion feature into  a MLP layer to regress the bounding box  $\left(x_{1}, y_{1}, x_{2}, y_{2}\right) \in \mathcal{R}^{4}$ of each cluster on the data we collect. }

%Instead of directly using the whole point cloud as the input, we feed each cluster to the model one by one to learn the features of each cluster. For each cluster, we first feed the location vector to a shared MLP layer, then followed by a max-pooling layer to aggregate the information from all the points and output the global feature $G$. The shared-MLP is designed for making the model invariant to input permutation\cite{pointnet} and learning the individual features. Then we concatenate $G$ with the individual features before and add the velocity vector in the end. Then we feed concatenated vector to another shared-MLP layer with a larger size, followed by a max-pooling layer. The output features $F$ will be added with the imu id  encoding with the same length. Then final output feature will be used in the cluster association process. 


%Then we will explain the \textit{imu id  encoding} in details. Besides the model structure, the downstream task plays an important role. The downstream task will control the features the model  learns. In our system, we choose bounding box prediction because the task requires the feature from  both the shape and location information.

%\textit{\textbf{Imu Id Encoding}}


%  Instead of using the entire captured point cloud, the model handles each cluster  separately. The model can be divided into two phases.
% \begin{enumerate}
%     \item  \textbf{Embed shape information: } The model first embed the information about the shape and outline of the object by using the shared mlp originally proposed in pointnet\cite{pointnet}.
%     \item \textbf{Bounding Box refinement:}  Due to the sparsity of the radar point cloud, the feature directly extracted from the point cloud is not robust enough. With the help of heuristic clustering, the cluster result is promising so we further concatenate the centriod information , including the $(x,y)$ position and the velocity to the feature out from phase 1. Then we apply the similar shared mpl with more neurons and one more layers. After that, the features pass through a max pooling layer and then become  a single representative feature for 2D-bounding box generation.

%     \end{enumerate}

%$\lVert \vec{c}_i - \vec{p}_{i,s}\rVert$

%Using the rank as weight instead of the actual distance avoid  the problem of attenuation of the data. Specifically,  if we divide the velocity by the actual distance,  all velocities will be near to zero since the distance is much larger than velocity.

The shape-motion feature is directly affected by the radar's sensing noises. Thus, we supplement user-specific static information (i.e., the IMU PID feature) to assist the cluster tracking. Specifically, we perform a {\em pre-matching} between the clusters generated by the radar and the IMUs, and then use the matched IMU's PID as the user-specific static information. The pre-matching is as follows. First, we compute $\overline{\vec{v}}_i$, which is the weighted average 3D velocity of all points in the $i^\text{th}$ cluster, by $\overline{\vec{v}}_i= \sum_{s=1}^{n_i} \frac{\textit{projection}_{\vec{v}_{\vec{c}_i}}\vec{v}_{i,s}}{\ln n_i \cdot \textit{rank}(d_s, \{d_1, \ldots, d_{n_i}\})}$, where
% We assign the id to the cluster  based  on  similarity between $M_{v_t}^{(j)}$  and $I_{v_t}^{i}$   There main difficulty here is that 
% \begin{enumerate}
%     \item  $M_{v_t}^{(j)}$  contains the estimated velocity of all the points in the point cloud, which is corrsponding to different body part. However, the profile $I_{v_t}^{i}$ only 
%     contains one estimated data. 
%     %\item  $M_{v_t}^{(k,j)}$   stores the velocity respect to mmWave  as a scalar while $I_{v_t}^{i}$  is the velocity of the user in world frame. 
% \end{enumerate}
% For $I_{v_t}^{i}$, we choose 2 dimensions with larger variance from x,y,z ,  The key observation is that it is enough to dinstinguish the velocity profile only based on 2 dimensions data. Moreover, reduced the problem from 
% 3D to 2D will ease  it.
% for $M_{v_t}^{(k,j)}$, we assign it to the direction $\theta^{j,k,t}$ in 2D plane. Here 
% the $\theta^{j,k,t}$ is the arrival of angle(AOA) of the point. Then the average velocity $\overline{V_{mmWave}^{j,t}}$  is calculated as 
$n_{i}$ is the number of points in the cluster, $\vec{c}_i$ is the cluster centroid, $\vec{v}_{\vec{c}_i}$ is $\vec{c}_i$'s 3D velocity, $\vec{v}_{i,s}$ is the 3D velocity of the $s^\text{th}$ point of the cluster, $d_s$ is the Euclidean distance between the $s^\text{th}$ point and the centroid, the operator $\textit{projection}_\vec{a} \vec{b}$ returns the projection of $\vec{b}$ in the direction of $\vec{a}$, and the operator $\textit{rank}(a, A) \in \{1, \ldots, |A|\}$ returns the rank of $a$ in the set $A$ with elements in ascending order. With the reciprocal of rank as the weight, a point closer to the centroid receives a larger weight in the averaging. Using the rank instead of distance as weight for velocity avoids the issue of physical unit conciliation.
%Using the rank as weight instead of the actual distance avoid  the problem of attenuation of the data. Specifically,  if we divide the velocity by the actual distance,  all velocities will be near to zero since the distance is much larger than velocity.
We apply the coefficient $\frac{1}{\ln n_i}$ to make the sum of the weights to be approximately one, i.e., $\sum_{s=1}^{n_i} \frac{1}{\ln n_i \cdot \textit{rank}(d_s, \{d_1, \ldots, d_{n_i}\})} \approx 1$. Second, with all clusters' average velocity magnitudes $\{|\overline{\vec{v}}_1|, \ldots, |\overline{\vec{v}}_{N}|\}$ and all IMUs' velocity magnitudes denoted by $\{|\vec{u}_1|, \ldots, |\vec{u}_{N}|\}$, we apply the Hungarian algorithm to find the one-to-one pre-match between the clusters and IMUs based on Euclidean distance. Let $\mathrm{PID}_i \in \{1, \ldots, N\}$ denote the PID of the IMU pre-matched with the $i^\text{th}$ cluster. We apply the position encoding \cite{vaswani2017attention} to generate the $i^\text{th}$ cluster's $1 \times 64$ IMU PID feature as $[g_1, h_1, g_2, h_2, \ldots, g_{32}, h_{32}]$, where $g_m =\sin \left( \left( \frac{\mathrm{PID}_i}{1000}\right)^{\frac{m}{32}} \right)$ and $h_m=\cos \left( \left( \frac{\mathrm{PID}_i}{1000}\right)^{\frac{m}{32}} \right)$. As presented earlier, the IMU PID feature is added to the shape-motion feature to form the cluster tracking feature.

Given the cluster tracking features obtained in two consecutive frames, the Hungarian algorithm is used to associate one feature in the former frame and one feature in the latter, exclusively, based on cosine similarity. The associated clusters are considered from the same user. In addition, their centroids over time form the trajectory of the user. All trajectories will be input to the trajectory-based association module presented in \sect\ref{subsec:association}.



%$p_{(s,j)}$  is the $s^{th}$ point.
%The function \textit{RANK} return the order of the distance among all the points in the $j^{th}$ cluster, starting from 1. The ranking severs as the weight of each point, and since $\sum_{i=0}{m} \frac{1}{i}  \approx \frac{1}{m}$, we divide the sum by $\frac{1}{m}$

%Here we use  $\vec{I_{v_t}^{i}}$,$\vec{M_{v_t}^{(j)}}$  to denote the velocity vector   of the $i^{th}$ imu and $j^{th}$ cluster at time $t$. Note that for each cluster, it contains multiple points, so we use  $\vec{M_{v_t}^{(s,j)}}$ to denote the velocity of $s^{th}$  point of the $j^{th}$  cluster. We apply the following process to define the similarity score between $\vec{I_{v_t}^{i}}$ and $\vec{M_{v_t}^{(j)}}$.

%The features learned from the PointTrack Net embeds the shape and motion information of the cluster. However, due to the uncertainty during the measurements, the motion coherent and space coherent of the cluster are not  invariant proprties. However, there is an invariant  property: \textit{there is  an one to one mathching between each cluster and each user}.  Therefore, we first find a matching between the clusters $C$ and the set of imu $I$. 


% If we model the matched pair $\left(C_{i},I_{j}\right)$
% We leverage the  property to assign the id of the most similar imu sensor. 

\begin{comment}
Given the number of users $N$, we first assign a pseudo id  $i$ from $1 \dots N$ to each IMU. 
Then we adopt the position encoding function, which is proposed in \cite{vaswani2017attention} with dimension D to encode the id, namely \textit{imu id encoding}

$$
PE(i) = 
\left[
  \begin{array}{c}
    \sin(i/1000^{\dfrac{1}{D}}),
    \cos(i/1000^{\dfrac{1}{D}})\\
    \vdots \\
    \sin(i /1000^{\dfrac{D}{D}}),
    \cos(i/1000^{\dfrac{D}{D}})
  \end{array}
\right]
$$
\end{comment}
  
%Up to this stage, we can  calculate the  similarity score   by using the cosine similarity. After finding the  most similar imu $i$ for cluster $j$, we add the $PE(i)$ to the feature vector of $j^th$ cluster.


% \todo{proof here, try to proof the gain of using the encoding, just draft right now}
% \textbf{Gain Information}
% Let $F = \{\xi_{0}, \xi_{1},\dots, \xi_{D-1}\}$ denote the feature vector from PointTrack Net for cluster i at frame t and
% $\hat{F} = \{\hat{\xi}_{0}, \hat{\xi}_{1},\dots, \hat{\xi}_{D-1}\}$ denote the feature vector for cluster i at frame t+1 .
% \begin{proof}
%   If $\cos\left(F,\hat{F}\right) \geq 0$, 
  
%   then $\cos\left(F,\hat{F}\right) \leq \cos\left(F+PE(i),\hat{F+PE(i)}\right)$


% $\cos\left(F+PE(i),\hat{F+PE(i)}\right)$
  


% $= \frac{\sum_{j=0}^{D} (\xi \cdot \hat{\xi} +  \gamma_{j}\cdot(\xi + \hat{\xi}))  + \frac{D}{2}}
%  {  \sqrt{ \sum_{j=0}^{D} (\xi_{j}^2 +2\gamma_{j} \cdot \xi_{j} + \gamma_{j}^2) } \cdot  \sqrt{ \sum_{j=0}^{D} (\hat{\xi}_{j}^2 +2\gamma_{j} \cdot \hat{\xi}_{j} + \gamma_{j}^2) }  }$ 


% $\geq \frac{\sum_{j=0}^{D} (\xi \cdot \hat{\xi} +  \gamma_{j}\cdot(\xi + \hat{\xi}))  + \frac{D}{2}}
% {\sqrt{\sum_{j=0}^{D} \xi_{j}}\cdot \sqrt{\sum_{j=0}^{D} \hat{\xi}_{j}} + 2 \sqrt{\sum_{j=0}^{D} \sum_{h=0}^{D} \xi_{j}\cdot \gamma_{j} \cdot \hat{\xi}_{h}  \cdot \gamma_{h}}  +\frac{D}{2}}$





% \end{proof}


%Furthermore, we design a new mechanism to handle the case when the assumptions of space and motion coherence fail due to unknown noise or occlusion.

\begin{figure}
  \centering
  \begin{subfigure}[b]{.23\linewidth}
    \centering
    \includegraphics[width=\linewidth]{figures/2case_demo}
    % \caption{Case 1: users walks in trajectories with different shapes in the same speed }
    \caption{Setup.}
  \end{subfigure}%
  \hfill
  \begin{subfigure}[b]{.37\linewidth}
    \centering
    \includegraphics[width=\linewidth]{figures/case1_association}
    \caption{Case 1 result.}
  \end{subfigure}
  \hfill
  \begin{subfigure}[b]{.37\linewidth}
    \centering
    \includegraphics[width=\linewidth]{figures/case2_association}
    \caption{Case 2 result.}
  \end{subfigure}
  \vspace{-1em}
  \caption{Cosine similarity between features of clusters of same user in two consecutive frames.}
  % provides cross-modality information to make a more robust cluster matching process. }
  \label{fig:imuencoding}
  \vspace{-1em}
  \end{figure}

\subsubsection{Evaluation}
We evaluate the advantage of the cluster tracking feature, compared with solely using either shape-motion feature or IMU PID feature.
% We also introduce a baseline based on global nearest neighbor (GNN for short) between the prediction positions from Kalman Filter and the and the observation points. 
%The Kalman Filter predicts the position ${pos}_{c_i}^{t+1}$ of cluster $c_i$ at time $t+1$, and the GNN finds the matched cluster $c_j$ nearest to ${pos}_{c_i}^{t+1}$ among all observed clusters at time $t+1$. 
% where the \textit{IMU Id Encoding} provides cross-modalities information.
We consider two cases as illustrated in Fig.~\ref{fig:imuencoding}: (1) all users walk at the same speed but follow different paths of different shapes; (2) all users walk at different speeds and follow different paths of the same shape. We measure the cosine similarity between the features of the clusters corresponding to the same user in two consecutive frames.
%Finally, we plot the cumulative distribution of cosine similarity between the features from the same clusters in two consecutive frames under the two scenarios for comparsion. 
Fig.~\ref{fig:imuencoding} shows the cumulative distribution functions (CDFs) of the measured cosine similarities in the two cases. 
% When cluster tracking feature is used, the cosine similarities are statistically higher.
% , under both two cases, using  \textit{cluster tracking feature}  achieves the best performance.
In case (1), the performance of shape-motion feature is similar to cluster tracking feature. In case (2), the performance of IMU PID feature is similar to cluster tracking feature, because the velocity-based mmWave-IMU pre-matching is accurate when users' speeds are different and the matched IMU PID contributes more information than the shape-motion feature. The above results show that the cluster tracking feature takes both the advantages of shape-motion feature and IMU PID feature.

%is representative enough to  discriminate different people in Case 1.  On the other hand, \textit{IMU PID} is reliable in case2  as the pre-matching process is accurate enough. Moreover, we can find that the \textit{shape-motion feature} plus the \textit{IMU PID}  achieves the result of $1 + 1 \geq 2$.
% This is because the assignment of the \textit{IMU PID} encoding heavily relies on the velocity information from two sensors. Therefore, when participants walk at a similar speed, the \textit{IMU PID} is not a discriminative indicator for different people.
%  On the contrary, when people walk in different speeds, the pre-matching process is accurate enough and \textit{IMU PID} can provide strong supplementary information for \textit{shape-motion feature}.
% In addition, we notice that the performance of GNN is poor. The reason is that GNN depends on the accurate position of clusters for association. However, this premise may not hold, especially when there are multiple people, considering that the radar point cloud is sparse and noisy. 
% In summary, this experiments shows that combining both F and P can make the cluster matching process more robust and accurate. 

% On the other hand, performance of using F drops in scenario two while using P results in W-ACC around 80$\%$. Overall, the experiment shows the two feature vectors
% embed different information related to the point cloud. 
% Moreover, it also shows combining the two features will lead to a better result.

% \begin{figure}[]
%   \centering
%   \includegraphics[width=.45\textwidth]{figures/demo2case}
%   \caption{Demo on the two experiment scenarios}
%   \label{fig:demo_association}
% \end{figure}%

% \begin{figure}
   
%   \begin{subfigure}[t]{.23\textwidth}
%     \centering
%     \includegraphics[width=\linewidth]{figures/association_case1}
%     \caption{Case 1}
%     \label{fig:association1}
%   \end{subfigure}%
%   \hfill
%   \begin{subfigure}[t]{.23\textwidth}
%     \centering
%     \includegraphics[width=\linewidth]{figures/association_case2.eps}
%     \caption{Case 2}
%     \label{fig:association2}
%   \end{subfigure}
%   \vspace{-0.5em}
%   \caption{Demo on the two experiment scenarios}
%   \label{fig:demo_association}
% \end{figure}

% \begin{figure}   
%     \begin{subfigure}[t]{.47\textwidth}
%       \centering
%       \begin{tabular}{p{0.4\textwidth}p{0.54\textwidth}}
%         \vspace{0pt} \includegraphics[width=\linewidth]{figures/association_case1} &
%                                                                                        \vspace{0pt} \includegraphics[width=\linewidth]{figures/case1_association}
%       \end{tabular}
%       \caption{Case 1: Same speeds, different paths}
%       \label{fig:association1}
%     \end{subfigure}%
%     \hfill
%     \begin{subfigure}[t]{.47\textwidth}
%       \centering
%       \begin{tabular}{p{0.4\textwidth}p{0.54\textwidth}}
%        \vspace{0pt} \includegraphics[width=\linewidth]{figures/association_case2} &
%                                                                                     \vspace{0pt} \includegraphics[width=\linewidth]{figures/case2_association}
%       \end{tabular}
%       \caption{Case 2: Different speeds, different paths of same shape}
%       \label{fig:association2}
%     \end{subfigure}
%     \vspace{-0.5em}
%     \caption{Cosine similarity between features of clusters of same user in two consecutive frames.}
%     % provides cross-modality information to make a more robust cluster matching process. }
%     \label{fig:imuencoding}
% \end{figure}



%%% Local Variables:
%%% mode: latex
%%% TeX-master: "main"
%%% End:
%\subsection{Local Tracking by IMU}
%\label{subsec:imu-tracking}

% \begin{figure}
%     \includegraphics[width=0.47\textwidth]{figures/imu_pipeline}
%       \caption{IMU tracking pipeline}
%     \label{fig:IMU tracking module}
% \end{figure}

% Typically, an imu outputs  body-frame accelerations, 
% angular velocities and magnetic field measurements. Compared to the wireless sensors, the IMU 
% sensor can reach a much higher sampling rate, which makes it suitable for deploying in a tracking system.
% In order to avoid extensively data collection and design a general system, our imu tracking module adopts
% the double integration approach. The high level idea is to apply double integration on the acceleration data to obtain the distance.

% However, two main challenges needed to be addressed before we utilize IMU in the tracking system.
% \begin{enumerate}
%     \item  Suppose \textit{I,W,B} denote inertial, world and body frame coordination system respectively. 
%     Generally, \textit{I} and \textit{ B} are the same but different from \textit{W}. 
%     Therefore, we need to  transfer the input or the tracking result from \textit{I} to 
%     \textit{W}.  
%     \item  The tracking result of IMU will drift over
%     time during the process of double integration of acceleration. In other word, any measurement errors are accumulated over time. 

    
    
% \end{enumerate}

% Typically, an imu collects body-frame accelerations, 
% angular rates and magnetic field measurements. Compared to the wireless sensors, the IMU 
% sensor can reach a much higher sampling rate, which makes it suitable for deploying in a tracking system.

% First of all, we denote the measurements of gyroscope and accelerometer at time t by $\vec{\Omega_t} = [\Omega_t^x,\Omega_t^y,\Omega_t^z]$ and $\vec{a_t} = [a_t^x,a_t^y,a_t^z]$ 
% Here we denote the attitude at time $t$ by $\mat{R}_t$, then $\mat{R}_{t+1}$ can be calculated as 
% $\mat{R}_{t+1} = \mat{R}_{t} \cdot \mat{\Omega^{\times}_t} $ where  

% $$\mat{\Omega^{\times}_t} =\left[\begin{array}{ccc}0 & -\Omega_{z} & \Omega_{y} \\ \Omega_{z} & 0 & -\Omega_{x} \\ -\Omega_{y} & \Omega_{x} & 0\end{array}\right]$$

% The mahony filter is designed to correct the $\mat{\Omega^{\times}_t}$ to improve the accuracy of attitude estimation. 
% It models the problem as the following optimization problem, where $g$ is the 
% gravity vector and $\vec{\hat{a}}$ is the normalized acceleration data.

% $$\mathbf{R} = \underset{\mathbf{R}\in SO(3)}{\operatorname{arg\,min}} (\lambda_1|g-\mathbf{R\cdot \vec{\hat{a}}}|^2 )$$
% m only requires two hyper-parameters, the proportional filter gain $k_p$ and 
% integral filter gain $k_i$.  Typically proportional filter gain $k_p$ determines the weight of the cumulative error and the integral filter gain $k_i$
% determines the weight of instantaneous error.
% Intuitively, the direction of the acceleration after applying the same rotation of the body should be aligned with the gravity field.
% Therefore the
% estimation error can be computed as $\vec{e_{t}} = \vec{\hat{a_{t}}} \times \vec{d} $, where $\vec{d}=-\mathbf{R_{t}}^{T} \vec{g}$. Finally, 
% it adds the correction step $\vec{\delta \Omega_{t}}=K_{p} \cdot \vec{e_{t}}+\vec{I}_{t}$ to $\vec{\Omega_{t}}$, where $\vec{I}_{t}$ is calcuated recursively
% as  $\vec{I}_{t}=\vec{I}_{t-1}+K_{i} \cdot \Delta t \cdot \vec{e_{t-1}}$
% The whole algorith


% \textbf{Utilize Mahony Filter For Transformation In Different Coordination Systems } The Mahony Filter\cite{mahony} is designed to 
% estimate  the orientation of the IMU in the world frame(\textit{ W}). Here we use $q^W$ and $q^I$ to represent the orientation in 
% \textit{ W} and \textit{ I} in the form of \textbf{quaternion}.  The algorithm first calcuates an orientation error from previous step base on the 
% acceleration data $I_{a_{t}}$ as shown in equation\ref{eq:mahonyError} where ${I}_{\hat{\mathbf{a}}_{t+1}}$ is the normalized accelerations, followed by 
%  a correction step based on a proportional-integral compensator in order to correct the measured angular velocity $I_{g_{t}}$as shown in equation\ref{eq:errorCorrect}.
% The corrected angular velocity is then used to calculate the orientation increment, which will be integrated to transfer 
% $q^I$ to $q^W$ 




% \begin{equation}
%     \label{eq:mahonyError}
%     \begin{aligned}
%     &\mathbf{v_t}=\left[\begin{array}{c}
%     2\left(q_{t-2}*q_{t-4}-q_{t-1}*q_{t-3}\right) \\
%     2\left(q_{t-1} *q_{t-2}+q_{t-3}*q_{t-4}\right) \\
%     \left(q_{t-1}^{2}-q_{t-2}^{2}-q_{t-3}^{2}+q_{t-4}^{2}\right)
%     \end{array}\right] \\
%     &\mathbf{e}_{t+1}={ }^{I} \hat{\mathbf{a}}_{t+1} \times \mathbf{v_t}\\
%     &\mathbf{e}_{i, t+1}=\mathbf{e}_{i, t}+\mathbf{e}_{t+1} \Delta t
%     \end{aligned}
% \end{equation}

% \begin{equation}
%     \label{eq:errorCorrect}
%     \begin{aligned}
%     &I_{g_{t+1}}=I_{g_{t+1}}+\mathbf{k}_{p} \mathbf{e}_{t+1}+\mathbf{k}_{i} \mathbf{e}_{i, t+1}
%     \end{aligned}
% \end{equation}
















% The IMU tracking module first apply a first order band-pass butterworth filter to the data collected by accelerometer, with a cutoff frequency at 0.02 HZ and 10 HZ to filter out the noise components. 
% Then we apply use a threshold $d=0.05$ on the accelerometer data to detect the static position (e.g the moment the upper limb moves to the lowest position or the feet touch the floor.) Then we design a mahony filter, whose goal is to estimate the orientation 
%  by fusing/combining attitude estimates by integrating gyroscope measurements 
%  and direction obtained by the accelerometer measurements. 
%  Specifically, in the mahony filter, we set the proportional filter gain $k_p = 0.5$ for static moment and $k_p = 0$ for non-static moment. 
%  When $k_p = 0$, the estimated orientation is the same with the last frame's estimation. 
%  The reason for the setting is that the orientation of the IMU should be aligned with the person when he/she is moving.  
%  However, a person may change the direction when the feet touches the floor.

%%% Local Variables:
%%% mode: latex
%%% TeX-master: "main"
%%% End:
% !TEX root = main.tex
% \section{Trajectory-Based Association}
\section{Learning-Based Cross-Modality Trajectory Association}
\label{subsec:association}

\subsection{Design Principle}

This module identifies the correspondence among the trajectories reconstructed by the radar in \sect\ref{sec:global-tracking-design} and IMUs via dead reckoning, to re-identify radar's sensing results. Essentially, it is a weighted bipartite matching problem with trajectory similarity as the weight.
%Note that the radar-sensed trajectories are given by the module presented in Section~\ref{sec:global-tracking-design}. The IMU trajectories are obtained by dead reckoning.
%The module's primary goal is to assign  a virtual id of  imu  to a cluster in the mmWave.
%Essentially, it solves an $N$-to-$N$ assignment problem.
% it is a $N-M$ matching problem, where $N$ is the number of participants  and $M$ is the number of occupants within the field of view of the mmWave sensor.
For association, we use the 2D trajectory (without including the altitude dimension), as it is a common feature that can be derived from both the radar's and IMUs' results and is agnostic to modality-dependent details. 
% The trajectory-based matching is considered to be a robust criterion because of the uniqueness of the trajectory for different users. Moreover, since trajectory represents the feature over time, short-term noise should have a relatively lower influence. 
% After generating the trajectories for each imu sensor and cluster in the mmWave, we need to define a similarity score.
For either a radar cluster or an IMU, a trajectory over an {\em association time window} $[t_0, t_1]$ is denoted by $\mathcal{T}(t) = \{x(t), y(t) | t \in [t_0, t_1]\}$. To compute the similarity between a radar cluster's trajectory $\mathcal{T}_{r}(t)$ and an IMU's trajectory $\mathcal{T}_{i}(t)$, the radar's and IMU's 2D coordinate systems need to be registered. A potential method to register the two coordinate systems, both originating at the start points of $\mathcal{T}_r(t)$ and $\mathcal{T}_i(t)$, is to exhaustively search a relative angle between them such that the similarity between $\mathcal{T}_r(t)$ and $\mathcal{T}_i(t)$ under the candidate registration is maximized. However, this registration incurs high compute overhead.
%, depending on the granularity that also affects registration quality.

We design a learning-based, registration-free association approach. 
The main idea is that, instead of considering the distance between two registered trajectories in the same Euclidean space, we take advantage of the feature extraction capability of neural networks to transform trajectories into high-dimensional features, and perform the association based on the distance in the high-dimensional space.
Specifically, we first encode the trajectory into an imagery representation, called {\em trace map}. This is a preparation step that restructures data to a uniform and compact form.
% we use a single image, called {\em trace map}, to compactly present a trajectory. 
Then, we feed {\em trace maps} from the two modalities into a Siamese neural network for feature extraction, based on whose outputs, the distance matrix can be calculated. 
Finally, in association, we introduce a soft voting mechanism which aggregates the information of multiple association time windows and thus mitigates the short-time interference. 
To train the Siamese network, we do not require the ground-truth trajectories. Instead, we extensively construct positive pairs and negative pairs of trajectories, and use a triplet loss to push negative pairs away while bringing together positive pairs, where the only labels required are the matching relationships of the trajectories from the two modalities.

\begin{comment}
As IMU dead reckoning has been extensively studied, we only mention the points noteworthy. ImmTrack adopts the double integration approach.
%applying the double integration on the tri-axis linear acceleration.
%We apply double integration on the accelerometer to generate the trajectory.
%In order to avoid extensively data collection and design a general system, our IMU tracking module adopts the double integration approach. The high level-idea is to apply double integration on the acceleration data to obtain the distance. 
%When integrating acceleration to obtain velocity, the acceleration should be aligned with the heading direction of the IMU to achieve higher tracking accuracy.
When integrating acceleration to obtain velocity, we apply the Mahony Filter \cite{mahony} to estimate the IMU's attitude (i.e., the rotation of the IMU from the gravitational direction) and then project the acceleration to align with the heading direction. In addition, we rectify the \textit{integral drift} by excluding static moments identified in terms of acceleration magnitude.
%One typical reason of integral drift is that the noises shift the actual zero acceleration to non-zero values. We identify the static moments by comparing the acceleration magnitude with a predefined threshold and remove the corresponding acceleration data.
%setting a threshold on the acceleration magnitude. Any magnitude value lower than the threshold is regarded as static. Then we correct the drift by removing the corresponding acceleration data of the static moment.
\end{comment}

\begin{comment}
ImmTrack uses a single image, called {\em trace map}, to compactly represent a trajectory, which facilitates (i) coping with the deviations between the radar's and IMU's trajectories corresponding to the same user and (ii) utilizing the state-of-the-art deep learning models to ...

{\blue
Though the idea is straightforward, we still need to overcome the following problems to successfully utilize the trajectory-based matching.
 
  \begin{enumerate}
     \item \textit{Handel the difference of the trajectory from imu  and the global tracking module}. 
     Due to the different sampling rates and noise,  the trajectories for the same person at the same time from the mmWave and imu are not identical. 
     \item \textit{Embed direction information in the trajectory}. Specifically, we need to distinguish between two users, who walk with the same
     shape of trajectories but from different direction.
 \end{enumerate}
}
\end{comment}
 

% \subsection{Trace map}
\subsection{Trace Map Generation}

Let $\mathcal{M}=\{M(x,y) | \forall (x,y)\}$ denote a trace map converted from a trajectory $\mathcal{T}(t)$, where the pixel value $\mathcal{M}(x,y)$ encodes all the times elapsed from when the trajectory crosses the location $(x,y)$. Let $f_s$ denote the sampling rate in frames per second (fps) of the sensor. Let $T(x,y)$ denote the set of the time instants at which the trajectory crosses $(x,y)$. If $T(x,y) \neq \emptyset$, the map pixel value is given by $\mathcal{M}(x,y) = \sum_{t \in T(x,y)} f_s \cdot (t - t_0)$, where $t_0$ denotes the time instant that the trajectory starts; otherwise, $\mathcal{M}(x,y) = 0$. 
Intuitively, $\mathcal{M}(x,y)$ encodes the number of frames passed when the user's trajectory crossed $(x, y)$ since the trajectory begins. Then, ImmTrack converts the obtained trace map into an image with three 8-bit channels of RGB data.
% ImmTrack further converts the map to a color image by filling the R, G, and B channels up to 255 sequentially using the $\mathcal{M}(x,y)$ as the budget. Once the $B$ channel is full, ImmTrack goes back to refill the $R$ channel. 
We use $\mathcal{M}_r$ and $\mathcal{M}_i$ to denote the color trace maps converted from $\mathcal{T}_r(t)$ and $\mathcal{T}_i(t)$, respectively.
%\todo{Ensure the notations are consistent with those in Algo. 1; OR delete this sentence here as they are not used very frequently later}
%Although it's been a long time since the question was asked but the solution @gehbiszumeis offered didn't work for me and I still got the normal figure caption for each subfigure. So in case anyone has still the same problem as mine here is what I did (according to this):

%=======
%Although it's been a long time since the question was asked but the solution @gehbiszumeis offered didn't work for me and I still got the normal figure caption for each subfigure. So in case anyone has still the same problem as mine here is what I did (according to this):
%>>>>>>> 02afdc82c070b0db667c738610dc9d5287fbc5d5

\begin{figure}
\centering
\begin{subfigure}{.15\textwidth}
    \centering
    \includegraphics[width=\linewidth]{figures/gt-05}  
    \caption{truth ($\rho$=0.5)}
    \label{figure:gt5}
\end{subfigure}
\begin{subfigure}{.15\textwidth}
    \centering
    \includegraphics[width=\linewidth]{figures/mmwave-05}  
    \caption{radar ($\rho$=0.5)}
    \label{figure:mmwave5}
\end{subfigure}
\begin{subfigure}{.15\textwidth}
    \centering
    \includegraphics[width=\linewidth]{figures/imu-05}  
    \caption{IMU ($\rho$=0.5)}
    \label{figure:imu5}
  \end{subfigure}
  \hfill
\begin{subfigure}{.15\textwidth}
    \centering
    \includegraphics[width=\linewidth]{figures/gt-02}  
    \caption{truth ($\rho$=0.2)}
    \label{figure:gt2}
\end{subfigure}
\begin{subfigure}{.15\textwidth}
  \centering
  \includegraphics[width=\linewidth]{figures/mmwave-02}   
  \caption{radar ($\rho$=0.2)}
  \label{figure:mmwave2}
\end{subfigure}
\begin{subfigure}{.15\textwidth}
  \centering
  \includegraphics[width=\linewidth]{figures/imu-02}  
  \caption{IMU ($\rho$=0.2)}
  \label{figure:imu2}
\end{subfigure}
\vspace{-1em}
  \caption{Trace maps of ground truth, radar, IMU trajectories.}
  \label{fig:demoOnS}
\end{figure}





\begin{comment}
 To handle the differences between the trajctories in imu tracking module and the global tracking module, 
 the trajectories are reduced into 2D heat maps to facilitate the matching process. Specifically, instead of denoting each position as a point on the image, we divide the field of view into squares with length $s$.  
 The heat map can preserve the shape information in the original trajectory and weaken the influence of errors and noises in the trajectories in the two different sensing modalities.
 

 If we describe the motion of an object by the function $f(x,y,t)$, where $x,y$ is the location and $t$ is the time,
  the shape of the trajectory contains the location information. Therefore, to solve the 
 second challenge, we need to add another variable to  embed the information related to the time in the trajectory. 
 Here we use the variable intensity of the heatmap $p$ and we design a mapping between
 $$
 f(x,y,t) \rightarrow G(x^{'},y^{'},p(x^{'},y^{'},t))
 $$
 Here $x^{'},y^{'}$ is the location
 on the image. We use ${h}_x,{h}_y$ to represent the size of the heat map, $x_0,y_0$ to represent the starting position of the trajectory, $s$ to represent the 
 grid size of the heat map, $f_{sample}$ to represent the sampling frequency of the sensor. Then specifically:

 $$
x^{'} =  \lceil \frac{x-x_0}{s \cdot {b}_x} \cdot h_x \rfloor  \hspace{1cm} y^{'} =  \lfloor \frac{y-y_0}{s \cdot {b}_y} \cdot h_y \rceil 
 $$
 
 $$
 p(x^{'},y^{'},t) = \left( p(x^{'},y^{'},t) + f_{sample} \cdot t \right) \mod{p_{max}}
 $$

The value $b_x,b_y$ is the length of the targeted sensing area.
Here $p_{max}$ is the maximum intensity of the heat map. Then since $ p = \frac{R+G+B}{3} $,  we change the $R$ equal to $p$ first, once the $R$ value is staurated, we modify the value of $G,B$ .

To weaken the impact of the sensing errors, we adopt low-resolution trace map.
\end{comment}

Furthermore, in order to mitigate the impact of noises, we adopt specific spatial grid size $\rho$ for the trace maps.
% lower resolutions by letting each pixel in $\mathcal{M}(x,y)$ correspond to a $\rho \times \rho\,\text{m}^2$ square in the coordinate system of the trajectory. Moreover, the time instants at which the trajectory crosses the square are included into $T(x,y)$. 
% Furthermore, to weaken the impact of the noises in reconstructing the trajectories from the radar and IMU data, we construct the trace maps with lower resolutions by letting each pixel in $\mathcal{M}(x,y)$ correspond to a $\rho \times \rho\,\text{m}^2$ square in the coordinate system of the trajectory. 
Fig.~\ref{fig:demoOnS} shows the trace maps of the ground truth, radar, and IMU trajectories under two $\rho$ settings, where a user follows a square zig-zag path to move. A darker red pixel indicates that the trajectory crosses the position more recently. We can see that, due to the inherent uncertainty of sensing, the radar's and IMU's trace maps have deviations from the ground truth. Moreover, under a certain $\rho$ setting, the IMU's trace map has more colored pixels on the trace than the radar's because of IMU's higher sampling rate. As a result, for IMU, setting a smaller $\rho$ can better reduce the crosstalks among different segments of the trajectory, while a larger $\rho$ can make the trace for the radar more continuous. % maintain the trace connected for radar. 
In the rest of this paper, we adopt $\rho=0.2\,\text{m}$ and $\rho=0.5\,\text{m}$ for IMU and radar, respectively.
Finally, we crop the trace map in an area of 20m $\times$ 20m and resize it to 193 $\times$ 193, which will be fed into the Siamese neural network presented in \sect\ref{subsec:siamese}.
% Finally, we find the optimal trajectory generation time to balance the accuracy and delay of the system.

% \begin{figure}
%   \hfill
%   \begin{minipage}[t]{.22\textwidth}
%     \centering
%     \includegraphics[width=\textwidth]{figures/window}
%     \caption{\footnotesize{The sampling time should not set to a larger value duo to the cumulative error on the mmWave and IMU tracking module.}}
%     \label{fig:generation time}
%   \end{minipage}%
% \end{figure}
%We observe  a proper value of $\rho$ for mmWave tracking  is not suitable for imu tracking. Refer to Figure \ref{fig:demoOnS}, when $\rho=0.5m$ for both mmWave and imu, while the track map from the mmWave tracking module is close to the ground truth,  the one from the imu is misaligned.  The observation leads us to set a different value for $\rho$ in mmwave tracking module and IMU tracking module.  Expressly, we set $\rho = 0.5m$ and  $\rho = 0.2m$ in mmwave tracking module and IMU tracking module respectively. 
 
%Then We rely on the siamese network to miltigate the difference in  two trajectories to the same size in order to  match the two trajectories. 

% \todo{move this section to front}
% \subsubsection{Influence of the Trajectory Generation Time}
% We investigate the relationship between the trajectory generation time $t_{gen}$ and the performance of W-ACC. %comparing \textbf{W-ACC}. on 3 and 4 people under different. 
% To eliminate the occasion of occlusion and focus on the impact of $t_{gen}$, we conduct this evaluation using the data with 3 and 4 participants. 
% %so that the only factor is $t_{gen}$. Note that we do not experiment on a higher number of people here in order to exclude the 
% As depicted in \ref{fig:generation time}, the accuracy is low when $t_{gen}$ is too small, which is reasonable, as less temporal information could be utilized for association.
% % is as expected since if the trajectory generation time ($t_{gen}$) is too short, nearly all the trajectories are the same. 
% We also observe a degradation of W-ACC when $t_{gen}$ is too large. This is because the cumulative errors over a long period time, which results in very discrepant trajectories obtained by the same user from two sensors. Moreover, for the sake of the user experience, we expect the system to generate the association as soon as possible, which prevents us from using a very large window size.
% % A larger trajectory generation time  results in more unique trajectories and hence leads to a matching with a higher confidence level.  Nevertheless, there
% % are mainly two considerations that restrict us to set a higher value on it. 
% % \subsection{Siamese network for comparing trajectories}
\subsection{Comparative Features Extraction}
\label{subsec:siamese}

% that can generate a similarity score of two input trace maps.
%After we transfer the trajectory to the heat map, 
% we design a siamese network for comparing the similarity between the trajectory.

We design a Siamese neural network to extract comparative features from $\mathcal{M}_r$ and $\mathcal{M}_i$, whose cosine similarity characterizes how close the $\mathcal{T}_r(t)$ and $\mathcal{T}_i(t)$ are.
%the similarity between $\mathcal{M}_r$ and $\mathcal{M}_i$.
Typically, a Siamese network contains two or more identical sub-networks that extract features from their respective input. During training, any parameter updates are mirrored across all sub-networks.
%The specific architecture is specifically designed for tasks related to image similarity comparison, such as face recognition.
%In addition, the siamese network is designed for one shot learning , whose task is  learning representations from a single sample. In our scenarios, each trajectory is unique and there isn't a lot of samples with the same shape. 
As illustrated in Fig.~\ref{fig:siamese}, the Siamese network used by ImmTrack employs a convolutional neural network (CNN) as the feature extractor. The CNN consists of three convolutional layers with rectified linear unit (ReLU) activation followed by max-pooling and a final fully-connected layer producing a $1 \times 1024$ feature vector.
% This layer produces the feature vectors that the weighed distance layer will fuse for comparing the similarity.
During training, three such identical CNNs are used to process three inputs, i.e., anchor, positive, and negative inputs.
% Our model uses three identical CNN sub-networks with shared weights for three different inputs: anchor, positive, and negative input.
The anchor and positive inputs are two trace maps generated from the radar and IMU for the same user at the same time, while the negative input is an unrelated trace map from either the radar or IMU.
% Specifically, the inputs are three heatmaps, where two of them will be the trajectories from the same person at the same time, one from mmWave and one from imu and the third one a negative example either from mmWave or imu, which is not similar to the previous one.
Denoting by $\vec{f}_a$, $\vec{f}_p$, and $\vec{f}_n$ the feature vectors produced by the CNN for the anchor, positive, and negative inputs, we use the triplet loss function for training: $\mathcal{L}=\max ( \| \vec{f}_a - \vec{f}_p\|_{\ell_2}- \| \vec{f}_a - \vec{f}_n \|_{\ell_2} + \mathrm{margin}, 0)$.
%After the distance layer, we adopt the triplet loss\cite{facenet} shown in equation\ref{eq:triplet} and the loss will maximize the distance beteen anchor and negative pair while minimizing the distance beteen anchor and positive pair.
% We collect a training dataset by randomly walking in a monitored space. In addition, 
We also generate simulated trajectories to augment \yimin{the training data collected in a real environment}. 
% {\red {Specifically, we use a random walk stochastic process to simulate the anchor input. Then we scale up or down the anchor input and hence  shift $10\%$ positions of the anchor input to neighbor positions.
Specifically, we use a random walk stochastic process to generate the anchor, and obtain the positive input by scaling up or down the anchor and shifting $10\%$ of the anchor positions to their neighbors. \yimin{Note that the training data needed by the Siamese neural network is unnecessary to be {\em in situ} data, because the network only learns extracting environment-agnostic comparative features.}
% During inference, the Cosine distance between the feature vectors of a radar cluster's and an IMU's trace maps is yielded as the similarity score.
At ImmTrack's run time, the trained CNN is used to extract the comparative feature from any given trace map $\mathcal{M}$.

%As for the inference phase, the input will be a pair of trajectories and the output will be the distance beteen them.

 % \begin{equation*}
%   \mathcal{L}=\max (d(ahr, pos)-d(ahr, neg)+\operatorname{margin}, 0)
%    \label{eq:triplet}
%  \end{equation*}

 
% \subsection{One-shot Id Association}\label{oneshot_id_ass}  
\subsection{Cross-Modality Association}\label{oneshot_id_ass}  

\begin{figure}
  \centering
  \begin{subfigure}[]{0.28\linewidth}
    \centering
    \includegraphics[width=\linewidth]{figures/siamese}
    \label{fig:siamese_general}
  \end{subfigure}
  \hspace{0.45em}
  \begin{subfigure}[]{0.68\linewidth}
    %\centering
    \includegraphics[width=\linewidth]{figures/siamese_cnn}    
    \label{fig:siamese_cnn}    
  \end{subfigure}
  \vspace{-3em}
  \caption{Left: Siamese network using three identical CNNs with shared weights during training. Right: Architecture of CNN that extracts comparative feature from trace map.}
  \label{fig:siamese}
\end{figure}
 


% It is clear that if a  bipartite graph $G = (L, R, E)$ has a perfect matching, then it must have $\|L\| = \|R\|$. Furthermore, for a set of vertices $S \subseteq V$, we define its set of neighbors $\Gamma(S)$ by:

% $$
% \Gamma(S)=\{v \in V \mid \exists u \in S \text { s.t. }\{u, v\} \in E\} .
% $$

%Suppose that for every $ S \subseteq L$,  we have $|\Gamma(S)| \geq |S|$ ,then  $G$ has a perfect matching\cite{perfectMatching}. In the \textit{Heuristic Clustering}, the number of cluster $N_{clu}$ will be equal to the number of IMU $N_{IMU}$, which ensures that $\|L\| = \|R\|$. 
% After feeding each trajectory pair from mmWave cluster and IMU sensor to the Siamese network,

%This section presents the design to associate each radar cluster and each IMU exclusively.
For the $w^\text{th}$ time step in an association time window, 
% For the $w^\text{th}$ association time window, After acquiring the feature vector of trace map from different trajectories,
ImmTrack constructs a similarity matrix $\mat{S}_w \in \mathbb{R}^{N \times N}$, where its $(i,j)^\text{th}$ element is the cosine similarity between the comparative feature vectors extracted by the Siamese network from the trace maps of the $i^\text{th}$ radar cluster and $j^\text{th}$ IMU, respectively.
%of trajectories from the $i^{th}$  cluster of  mmWave and  the $j^{th}$ IMU.
% The range of $S_{(i,j)}$ is $[0,2]$, where the similar pair should result in a value between 0 to 1.
% We then solve the association problem in each window with matrix $S$ using Hungarian Algorithm, which works in polynomial time. 
% Then the remaining issue is choosing the most confident association result or  generate a final association result based on the result in each window.
%Rather than solving the association problem using only $\mat{S}$, we introduce a simple yet effective multi-window ensemble scheme.
% , which performs soft voting among $p$ consecutive windows to generate a final associating proposal.
ImmTrack generates an average similarity matrix, denoted by $\mat{S}$, over a total of $W$ consecutive association time windows, i.e., $\mat{S}=\frac{1}{W} \sum_{w=1}^{W}\mat{S}_w$. Hungarian algorithm is applied to propose an association between the radar clusters and IMUs. If the proposal is accepted, the IMUs' PIDs are transferred to the radar clusters for re-identification.
%We construct matrix $\hat{\mat{S}}$ by aggregating the matrices in $p$ consecutive windows: $\mat{\hat{S}} = \frac{1}{p} \cdot \sum_{1}^{p}  \mat{S_{w}}$, on which the Hungarian algorithm is applied for generating the final association proposal. 
\sect\ref{main_acc_result} will show via evaluation that the multi-window similarity averaging improves the robustness of the association, compared with using a single window only.
%, especially when the number of users increases.
%we will show that this ensemble scheme boosts the system performance, especially when the number of people is relatively large.
% Then we apply the Hungarian Algorithm on the $S_{final}$ to 
% generate the final association proposal.

In addition, ImmTrack applies two criteria to accept an association proposal. If either criterion is not met, ImmTrack excludes the oldest window from the $W$ windows, waits for a new window becoming available, and checks the two criteria again.
%we define two criteria for $\hat{\mat{S}}$ to ensure the final proposal is with a high confidence level. Only when both two criteria are satisfied will we adopt this proposal and associate the radar's cluster with the IMU ID. Otherwise, we remove the earliest window in $p$ windows that form $\hat{\mat{S}}$ and wait for the outcome of a new window.
The two criteria are as follows. {\bf Criterion~1:} For each pair of associated radar cluster and IMU, the similarity between their comparative features needs to be higher than a pre-defined threshold $\alpha$. This criterion sets a lower bound for the association quality.
The $\alpha$ can be set according to the data used to train the Siamese network by $\alpha = \max \{\min_{\forall (\vec{a}, \vec{p}) \in \mathcal{P} } S_c(\vec{a},\vec{p}), \max_{\forall (\vec{a}, \vec{n}) \in \mathcal{N}} S_c(\vec{a},\vec{n}) \}$, where $\mathcal{P}$ and $\mathcal{N}$ are the positive and negative pair sets, $S_c(\cdot, \cdot)$ denotes cosine similarity. Our training data gives $\alpha=0.23$.
%  If the Hungarian algorithm proposes that the $i^{th}$ cluster should pair with the $j^{th}$ IMU, then $\hat{\mat{S}}_{[i,j]} \geq \alpha$
{\bf Criterion~2:} Any IMU cannot produce the highest cosine similarity with two or more radar clusters among all IMUs. Formally, $\forall i \in [1, N]$, if the $(i,j)^\text{th}$ element of $\mat{S}$ (denoted by $\mat{S}_{i,j}$) is the maximum value within the $i^\text{th}$ row of $\mat{S}$, then $\nexists k \in [1,N]$ such that $\mat{S}_{k,j}$ is the maximum value within the $k^\text{th}$ row of $\mat{S}$. This criterion makes sure that the IMU most similar with every radar cluster is unique.

%Intuitively, the first criterion sets a lower confidence bound for the associated IMU and the radar's cluster. The second criterion makes sure the IMU with the highest similarity for every radar is distinct. 
% each IMU will be associated with the radar's cluster with the highest similarity rather than others.
%A pseudo-code of the whole procedure of ImmTrack is presented in Algorithm \ref{algo1}, where the multi-window based association is shown in line 10 to line 22.  

\begin{comment}
\begin{algorithm} 
	% \caption{One-Shot Matching \todo{May need to change the name; M to N} 
  \caption{Cross-Modality Matching \todo{double check the notations; discuss the name here}} 
	\begin{algorithmic}[1]
	    \State $N \leftarrow \text{number of imus, number of clusters }$
	    % \State $M \leftarrow \text{number of the radar's clusters}$
	    \State $IMUList,ClusterList \leftarrow [1,2,\ldots,N]$
	    % \State $ClusterList \leftarrow [1,2,\ldots,M]$
	    \State $\mat{\hat{S}} \leftarrow \mathcal{O} \in \mathbb{R}^{N \times N}$
        \State $ b \leftarrow 1$
		\For {$window:w=1,2,\ldots $} 
		    
    		\State $\mathcal{M}_{w}^{imu},\mathcal{M}_{w}^{mmWave}\leftarrow GenerateTraceMap(w)$
            %\Comment{T stands for trajectory}
    		\State $\mat{S_{w}}  \leftarrow Similarity(\mathcal{M}_{w}^{imu},\mathcal{M}_{w}^{mmWave})$ 
    		
    		\If{$b \le p $}
    		    \State $\mat{\hat{S}}\leftarrow \mat{\hat{S}} +  \mat{S_{w}}$
    		    \State $b \leftarrow b+1$
    		\EndIf
    		
    		\If{$b == p$}
    		\State $\mat{\hat{S}} \leftarrow Hungarian(\mat{\hat{S}})$
    		
    		
    		\If {\text{both two criteria satisfied}}
    		\State lock the id
            \Else
            \State $\mat{\hat{S}} \leftarrow \mat{\hat{S}} -  \hat{S_{w-p+1}}$
            \State $b \leftarrow b-1$
            
    		% \State $IMUList \leftarrow IMUList \setminus \text{matched IMU id } $
    		% \State $clusterList \leftarrow clusterList \setminus \text{Matched cluster id}$
    		\EndIf
    		
    		\EndIf
    		% \State Detect new IMU
		    
			
		\EndFor
	\end{algorithmic} 
  \label{algo1}
\end{algorithm}
\end{comment}


\subsection{Handling Users with Identical Trace Maps}
\label{boundary_case_macth}

Multiple users may generate nearly identical trace maps in certain cases, e.g., when they walk side by side or follow simple straight paths.
%This section describes the algorithm for associating the radar cluster and IMU 
% when the trajectory association is not suitable 
%when some of the users' trajectories are nearly identical in certain scenarios. 
% For instance, when two friends go out together, their trajectories may appear identical after conveied to trace maps.
Within a certain modality, such nearly identical trace maps can be detected by checking their pair-wise similarities.
%\yimin{We collect data from 6 users over 20 minutes, in which two of them walk side by side 10 times, 30 seconds at a time. ImmTrack sets the similarity threshold to 0.92. The detection rates of identifying the users walking together are  92.5\% and 77.5\% by using  mmWave radar and IMU respectively.}
\yimin{Based on a dataset collected from six human subjects in controlled experiments with pairs of human subjects walking side by side, the detection rates of identifying the side-by-side walk are 92.5\% and 77.5\% using mmWave radar data and IMU data, respectively, by adopting a threshold of 0.92 on the normalized similarity for the detection. After removing the entries of the $\mat{S}_w$ corresponding to the detected identical trace maps, the remaining entries are processed by the cross-modality association presented in \sect\ref{oneshot_id_ass}.} This section presents a separate cross-modality association approach for the nearly identical trace maps based on gait analysis. \yimin{ImmTrack initializes the gait analysis if it detects users with nearly identical trace maps from the mmWave radar.} The gait analysis for an mmWave cluster is as follows. First, we compute the measured spectrogram $\mat{X}_m(v_k, t_l)$ from the Doppler Fourier transform corresponding to the points belonging to the cluster, where $v_k$ and $t_l$ represent the velocity and time bins, respectively. Second, we use the Boulic model \cite{boulic1990global} to generate the simulated spectrogram $\mat{X}_s(v_k, t_k | f_c, l_c, \varphi_c)$, where the parameters $f_c$, $l_c$, and $\varphi_c$ are the specified step frequency, step length, and start phase, respectively. \yimin{By solving $
\mathop{\arg\min}_{f_c, l_c, \varphi_c}
\sum_{\forall v_k, t_l} \left\| \mat{X}_{\mathrm{m}}^{\log }\left(v_k, t_l\right) 
- \mat{X}_{\mathrm{s}}^{\log }\left(v_k, t_l | f_c, l_c, \varphi_c \right) \right\|_{\ell_2}^2
$, where the superscript ``log'' means element-wise log normalization, the gait feature $(f_c, l_c)$  is estimated from mmWave radar data.}
For IMU data, we employ the IMU-based gait analysis \cite{madgwick2011Imutrack} to estimate the gait feature $(f_c, l_c)$. Lastly, Hungarian algorithm is applied to associate the mmWave clusters and IMU traces that respectively produce nearly identical trace maps, in terms of the cosine similarity between the mmWave-based and IMU-based gait features. The effectiveness of the mechanism presented in this section will be evaluated in \sect\ref{eval_tiny_space}.

%Different body parts of a walking person induce varied Doppler signatures in mmWave radar sensing: the torso induces a Doppler signature with constant velocity, while other parts induce sideband Doppler signatures on both sides of the torso \cite{chen2006micro}.
%We utilize the micro-doppler effect to conduct the human gait analysis based on the mmWave radar. Specifically, different body parts induce varied Doppler signatures.
%The torso of a walking human induces a Doppler signature with constant velocity and Radar Cross Section (RCS) while the other human components induce sideband Doppler signatures on both sides of the torso \cite{chen2006micro}. 

\begin{comment}
The overview of the mmWave radar based gait analysis is that we first construct the measured spectrogram $\mat{X_{m}}$ from the Dopple FFT result, in which the x-axis is the time bins and the y-axis represents the velocity bins.
Here we use $\mat{X_{m}}(v_{k}, t_{l})$ to represent amplitude at time bins index $t_{l}$  and velocity bins index $v_{k}$. 
At the same time, we use the Boulic model \cite{boulic1990global} to simulate the walking motion of a certain user as a function of time.  As a result, we derive a simulated spectrogram $\mat{X_{s}}$
The Boulic model utilizes three parameters namely step frequency $f^c$, step length $l^c$, and start phase $ \varphi^c$ to simulate the walking motion. 
A fit function is chosen to calculate the difference between $\mat{X_{m}}$ and $\mat{X_{s}}$
as  a function of the parameters of the walking model. By minimizing the difference as a function of the parameters of the walking
model, the values of $f^c$,  $l^c$, $ \varphi^c$  for the best fit is found.

Formally, $\mat{X_{s}}$ is calcuated in function $\mat{X_{s}}(v_k, t_l, f^c, l^c, \varphi^c, a_{\log }, n_{\log })$, where $a_{Log}$ and $n_{log}$ are two constants in the model, and their estimations are mentioned in \cite{chen2006micro}, which is out of the scope of our system. 
We choose the scaled version of spectral distortion function (SD) \todo{cite} as the fit function and derive gait parameter $(f^c, l^c, \varphi^c)$ from the optimization problem below using Powell's method\todo{cite} with starting value discribed in \todo{cite}. 


$$
\begin{aligned}
% \chi^2\left(f^c, l^c, \varphi^c \right)=
\mathop{\arg\min}_{(f^c, l^c, \varphi^c)}
\sum_{\text {all } v_k, t_l} \mid  \mat{X}_{\mathrm{m}}^{\log }\left(v_k, t_l\right) 
- \left.\mat{X}_{\mathrm{s}}^{\log }\left(v_k, t_l, f^c, l^c, \varphi^c, a_{\log }, n_{\log }\right)|^2\right.
\end{aligned}
$$

% Given the measured spectrogram $X_{m}(v_{k}, t_{l})$, the goal is to minimize the difference between the measured data and the estimated data. We first normalize $X_{m}(v_k, t_l)$ to
% $ \mathbf{X}_{\mathbf{m}}^{\log }\left(v_k, t_l\right)$:
% $$
%  \mathbf{X}_{\mathbf{m}}^{\log }\left(v_k, t_l\right)= \frac{\log \left(\mathbf{X}_{\mathbf{m}}\left(v_k, t_l\right)\right)}{\sqrt{\sum_{a l l} v_k, t_l} \log \left(\mathbf{X}_{\mathbf{m}}\left(v_k, t_l\right)^2\right)}
% $$

\end{comment}


%The estimation of step information from the IMU is well-studied \textcolor{red}{[citation]}. The cosine similarity between the gait information vextor  $\vec{(f^c, l^c)}$ extracted from the mmWave radar and IMU is calculated, which is further used for cross-modality association in the case. 

%In summary, the association module works in a hierarchical manner. The gait analysis pipeline will be only invoked after ImmTrack detects certain users' trace map are nearly identical.



\section{Evaluation}
\label{sec:eval}

We conduct experiments from different aspects to validate the efficacy of the propose \model\ framework. The implementation details for our \model\ and the baseline methods are presented in~\ref{sec:implement}. Our experiments aim to answer the following research questions:
\begin{itemize}[leftmargin=*]
    \item \textbf{RQ1}: How does the proposed \model\ perform on different experimental datasets in comparison to state-of-the-art baselines?
    \item \textbf{RQ2}: How does different sub-modules of the proposed \model\ framework contribute to the overall performance?
    \item \textbf{RQ3}: How scalabile is \model\ in handling large-scale data?
    \item \textbf{RQ4}: How does the model performance vary when tuning important hyperparameters of the proposed \model\ model?
    \item \textbf{RQ5}: How can our \model\ model address the over-smoothing issue compared with GNN-based recommendation methods?
\end{itemize}

\subsection{Experimental Settings}
\subsubsection{\bf Experimental Datasets}
\begin{table}[t]
    \centering
    \caption{Statistics of the experimental datasets.}
    \label{tab:datasets}
    \small
    \vspace{-0.18in}
    \begin{tabular}{ccccc}
        \toprule
        Dataset & \# Users & \# Items & \# Interactions & Interaction Density \\
        \midrule
        Gowalla & 25,557 & 19,747 & 294,983 & $5.85\times 10^{-4}$\\
        Yelp & 42,712 & 26,822 & 182,357 & $1.59\times 10^{-4}$\\
        Amazon & 76,469 & 83,761 & 966,680 & $1.51\times 10^{-4}$\\
        % Tmall & 805,506 & 584,050 & 39,183,700 & $8.33\times 10^{-5}$\\
        \bottomrule
    \end{tabular}
    \vspace{-0.15in}
    \Description{A table showing the statistics of the Gowalla data (25557 users, 19747 items, 294983 interactions), the Yelp data (42712 users, 26822 items, 182357 interactions), and the Amazon data (76469 users, 83761 items, 966680 interactions).}
\end{table}


\begin{table*}[h]
\vspace{-0.1in}
\caption{Performance comparison on Gowalla, Yelp, and Amazon datasets in terms of \textit{Recall} and \textit{NDCG}.}
\vspace{-0.15in}
\centering
%\ssmall
% \scriptsize
\footnotesize
%\small
\setlength{\tabcolsep}{1.2mm}
\begin{tabular}{|c|c|c|c|c|c|c|c|c|c|c|c|c|c|c|c|c|c|l|}
\hline
Data & Metric & BiasMF & NCF & AutoR & PinSage & STGCN & GCMC & NGCF & GCCF & LightGCN & DGCF & SLRec & NCL & SGL & HCCF & \emph{\model} & p-val.\\
\hline
\multirow{4}{*}{Gowalla}
&Recall@20 & 0.0867 & 0.1019 & 0.1477 & 0.1235 & 0.1574 & 0.1863 & 0.1757 & 0.2012 & 0.2230 & 0.2055 & 0.2001 & 0.2283 & 0.2332 & 0.2293 & \textbf{0.2434} & $2.1e^{-8}$\\
&NDCG@20 & 0.0579 & 0.0674 & 0.0690 & 0.0809 & 0.1042 & 0.1151 & 0.1135 & 0.1282 & 0.1433 & 0.1312 & 0.1298 & 0.1478 & 0.1509 & 0.1482 & \textbf{0.1592} & $1.2e^{-9}$\\
\cline{2-18}
&Recall@40 & 0.1269 & 0.1563 & 0.2511 & 0.1882 & 0.2318 & 0.2627 & 0.2586 & 0.2903 & 0.3181 & 0.2929 & 0.2863 & 0.3232 & 0.3251 & 0.3258 & \textbf{0.3399} & $2.4e^{-8}$\\
&NDCG@40 & 0.0695 & 0.0833 & 0.0985 & 0.0994 & 0.1252 & 0.1390 & 0.1367 & 0.1532 & 0.1670 & 0.1555 & 0.1540 & 0.1745 & 0.1780 & 0.1751 & \textbf{0.1865} & $1.7e^{-9}$\\
\hline

\multirow{4}{*}{Yelp}
&Recall@20 & 0.0198 & 0.0304 & 0.0491 & 0.0510 & 0.0562 & 0.0584 & 0.0681 & 0.0742 & 0.0761 & 0.0700 & 0.0665 & 0.0806 & 0.0803 & 0.0789 & \textbf{0.0823} & $3.7e^{-4}$\\
&NDCG@20 & 0.0094 & 0.0143 & 0.0222 & 0.0245 & 0.0282 & 0.0280 & 0.0336 & 0.0365 & 0.0373 & 0.0347 & 0.0327 & 0.0402 & 0.0398 & 0.0391 & \textbf{0.0414} & $3.8e^{-5}$\\
\cline{2-18}
&Recall@40 & 0.0307 & 0.0487 & 0.0692 & 0.0743 & 0.0856 & 0.0891 & 0.1019 & 0.1151 & 0.1175 & 0.1072 & 0.1032 & 0.1230 & 0.1226 & 0.1210 & \textbf{0.1251} & $4.8e^{-3}$\\
&NDCG@40 & 0.0120 & 0.0187 & 0.0268 & 0.0315 & 0.0355 & 0.0360 & 0.0419 & 0.0466 & 0.0474 & 0.0437 & 0.0418 & 0.0505 & 0.0502 & 0.0492 & \textbf{0.0519} & $2.4e^{-4}$\\
\hline

\multirow{4}{*}{Amazon}
&Recall@20 & 0.0324 & 0.0367 & 0.0525 & 0.0486 & 0.0583 & 0.0837 & 0.0551 & 0.0772 & 0.0868 & 0.0617 & 0.0742 & 0.0955 & 0.0874 & 0.0885 & \textbf{0.1067} & $1.1e^{-10}$\\
&NDCG@20 & 0.0211 & 0.0234 & 0.0318 & 0.0317 & 0.0377 & 0.0579 & 0.0353 & 0.0501 & 0.0571 & 0.0372 & 0.0480 & 0.0623 & 0.5690 & 0.0578 & \textbf{0.0734} & $7.0e^{-12}$\\
\cline{2-18}
&Recall@40 & 0.0578 & 0.0600 & 0.0826 & 0.0773 & 0.0908 & 0.1196 & 0.0876 & 0.1175 & 0.1285 &0.0912 & 0.1123 & 0.1409 & 0.1312 & 0.1335 & \textbf{0.1535} & $6.6e^{-10}$\\
&NDCG@40 & 0.0293 & 0.0306 & 0.0415 & 0.0402 & 0.0478 & 0.0692 & 0.0454 & 0.0625 & 0.0697 & 0.0468 & 0.0598 & 0.0764 & 0.0704 & 0.0716 & \textbf{0.0879} & $2.0e^{-12}$\\
\hline
\end{tabular}
\vspace{-0.1in}
\label{tab:overall_performance}
\Description{A table presenting the evaluated performance of the proposed \model\ model and the baselines, in which \model\ significantly outperforms the baseline methods.}
\end{table*}

% atings are transformed into binary implicit feedback following~\cite{he2020lightgcn}. We filter users and items with less than 3 interactions, 

Three benchmark datasets collected from real-world online services are used to evaluate the performance of \model. Data statistics are shown in Table~\ref{tab:datasets}. We split the interaction data into training set, validation set and test set with 70\%:5\%:25\%. Details of the experimental datasets are:
\begin{itemize}[leftmargin=*]
    \item \textbf{Gowalla}: This dataset is collected from Gowalla, including user check-in records at geographical locations, from Jan to Jun, 2010.
    \item \textbf{Yelp}: This dataset contains users' ratings on venues, collected from Yelp platform. The time range is from Jan to Jun, 2018.
    \item \textbf{Amazon}: This dataset is composed of users' rating behaviors over books collected from Amazon platform, during 2013.
\end{itemize}

\vspace{-0.1in}
\subsubsection{\bf Evaluation Protocols}
Following previous works on CF recommenders~\cite{wang2019neural, xia2022self}, we conduct all-rank evaluation, in which positive items from test set are ranked with all un-interacted items for each user. The widely-used \emph{Recall@N} and \emph{NDCG@N} metrics~\cite{wu2021self,2021knowledge} are used adopted for evaluation, where $N=20$ by default.

\vspace{-0.05in}
\subsubsection{\bf Baseline Models}
We compare \model\ with the following 14 baselines from 4 research lines for comprehensive validation.
\\\noindent\textbf{Traditional Collaborative Filtering Technique:}
\begin{itemize}[leftmargin=*]
    \item \textbf{BiasMF}~\cite{koren2009matrix}: It is a classic matrix factorization approach that combines user/item biases with learnable embedding vectors.
\end{itemize}
\textbf{Non-GNN Neural Collaborative Filtering}:
\begin{itemize}[leftmargin=*]
    \item \textbf{NCF}~\cite{he2017neural}: It is an early study of deep learning CF model that enhances the user-item interaction modeling with MLP networks.
    \item \textbf{AutoR}~\cite{sedhain2015autorec}: This method applies a three-layer autoencoder with fully-connected layers to encode user interaction vectors.
\end{itemize}
\textbf{Graph Neural Architectures for Collaborative Filtering}:
\begin{itemize}[leftmargin=*]
    \item \textbf{PinSage}~\cite{ying2018graph}: This method combines random walk with graph convolutions for web-scale graph in recommendation.
    \item \textbf{STGCN}~\cite{zhang2019star}: This method augments GCN with autoencoding sub-networks on hidden features for better inductive inference.
    \item \textbf{GCMC}~\cite{berg2017graph}: This is a representative work to introduce graph convolutional operations into the matrix completion task.
    \item \textbf{NGCF}~\cite{wang2019neural}: It is a GNN-based CF method which conducts graph convolutions on the user-item interaction graph for embeddings.
    \item \textbf{GCCF}~\cite{chen2020revisiting} and \textbf{LightGCN}\cite{he2020lightgcn}: These two methods propose to simplify conventional GCN structures by removing transformations and activations for improving performance.
\end{itemize}
\textbf{Disentangled GNN-based Collaborative Filtering}:
\begin{itemize}[leftmargin=*]
    \item \textbf{DGCF}\cite{wang2020disentangled}: This method disentangles user-item interactions into multiple hidden factors in the graph message passing process.
\end{itemize}
\textbf{Self-Supervised Learning Approaches for Recommendation}:
\begin{itemize}[leftmargin=*]
    \item \textbf{SLRec}~\cite{yao2021self}: This method applies contrastive learning to recommendation models with feature-level data augmentations.
    \item \textbf{NCL}~\cite{lin2022improving}: This approach enhances self-supervised graph CF models with enriched neighbor-wise contrastive learning.
    \item \textbf{SGL}~\cite{wu2021self}: It conducts various types of graph augmentations and feature augmentations with graph contrastive learning for CF.
    \item \textbf{HCCF}~\cite{xia2022hypergraph}: This method augments GNN-based CF with a global hypergraph GNN and conducts cross-view contrastive learning.
\end{itemize}

\subsection{Overall Performance Comparison (RQ1)}


The overall performance of \model\ and the baselines are shown in Table~\ref{tab:overall_performance}. From the results we have the following observations: \vspace{-0.05in}
\begin{itemize}[leftmargin=*]
    \item Our \model\ consistently achieves best performance compared to baselines methods. Also, we re-train \model\ and the best-performed baselines (\ie, SGL and NCL) for 5 times to calculate $p$-values. The experimental results validate the significance of the improvement by \model. Compared to the state-of-the-art GNN methods, the MLP-based inference model of our graph-less \model\ generates more accurate recommendation results, due to its adaptive contrastive knowledge distillation. Specifically, the dual-level KD in \model\ enables enriched and adaptive high-order smoothing, which not only distills the accurate dark knowledge in the well-trained GNN teacher, but also avoids being affected by the over-smoothing signals. Furthermore, the adaptive contrastive regularization automatically alleviates the over-smoothing effects, which further boosts the performance. \\\vspace{-0.12in}
    
    \item While the self-supervised learning schema greatly improves the performance of GNN-based CF, our graph-less \model\ model still significantly outperforms the SSL-enhanced graph models. We attribute the performance deficiency to the inherent incapability of existing SSL frameworks in filtering over-smoothing signals. For example, SGL augments model training by introducing random noises, which may even aggravate the inaccuracy in node embeddings when the noises are magnified through high-order graph propagation. As for NCL and HCCF, they seek to connect nodes based on global semantic relatedness, which may even over-smooth nodes distant from each other in the original graph. In comparison, our graph-less \model\ model abandons GNN architectures in the inference model, which fundamentally minimizes the possibility of over-smoothed node embeddings. Furthermore, our KD paradigm avoids distilling over-smoothed embeddings via the adaptive contrastive regularization. \\\vspace{-0.12in}
    
    \item We observe that non-GNN CF models (\ie, NCF and AutoR) present very bad performance, event though they have similar MLP-based network architectures as the inference model in \model. This sheds light on the deficiency of MLPs in modeling high-order graph connectivity into user/item embeddings. While sharing similar MLP structures, our \model\ is additionally supervised by knowledge distilled from advanced GNN models. This not only improves the optimization for MLP networks, but also makes it possible to adaptively filter the over-smoothing signals in parameter learning. The huge performance gap between NCF/AutoR and our \model\ strongly shows the effectiveness of our contrastive knowledge distillation.
\end{itemize}


\begin{table}[t]
    %\vspace{-0.05in}
    \caption{Ablation study on key components of \model.}
    \vspace{-0.15in}
    \centering
    %\small
    %\scriptsize
    %\ssmall
    \footnotesize
    %\small
    % \setlength{\tabcolsep}{1.2mm}
    \begin{tabular}{c|c|cc|cc|cc}
        \hline
        % \multirow{2}{*}{Category} & \multirow{2}{*}{Variant} 
        \multicolumn{2}{c|}{Data}& \multicolumn{2}{c|}{Gowalla} & \multicolumn{2}{c|}{Yelp} & \multicolumn{2}{c}{Amazon}\\
        % \cline{3-8}
        \hline
        \multicolumn{2}{c|}{Variant} & Recall & NDCG & Recall & NDCG & Recall & NDCG\\
        \hline
        % \hline
        % \multicolumn{8}{c}{Top-20}\\
        \hline
        \multicolumn{2}{c|}{-$\mathcal{L}_1$} & 0.2180 & 0.1415 & 0.0756 & 0.0377 & 0.1012 & 0.0692\\
        \hline
        \multirow{3}{*}{-$\mathcal{L}_2$} & User & 0.2292 & 0.1493 & 0.0806 & 0.0405 & 0.0998 & 0.0667 \\
        & Item & 0.2266 & 0.1477 & 0.0808 & 0.0406 & 0.0974 & 0.0649 \\
        & Both & 0.2222 & 0.1451 & 0.0787 & 0.0399 & 0.0938 & 0.0626 \\
        \hline
        \multirow{4}{*}{-$\mathcal{L}_3$} & U-I & 0.2330 & 0.1496 & 0.0814 & 0.0410 & 0.0939 & 0.0607 \\
        & U-U & 0.2349 & 0.1512 & 0.0811 & 0.0407 & 0.0965 & 0.0634 \\
        & I-I & 0.2331 & 0.1514 & 0.0813 & 0.0409 & 0.1009 & 0.0674\\
        & All & 0.2282 & 0.1480 & 0.0810 & 0.0407 & 0.0933 & 0.0605\\
        \hline
        \hline
        \multicolumn{2}{c|}{\emph{\model}} & \textbf{0.2434} & \textbf{0.1592} & \textbf{0.0823} & \textbf{0.0414} & \textbf{0.1067} & \textbf{0.0734}\\
        \hline
    \end{tabular}
    \vspace{-0.1in}
    \label{tab:module_ablation}
    \Description{A table presenting the results of module ablation study. The results are divided into three parts: loss $\mathcal{L}_1$ for the prediction-level distillation, loss $\mathcal{L}_2$ for the embedding level distillation, and loss $\mathcal{L}_3$ for the contrastive regularization. All ablated variants performs worse than the proposed \model.}
\end{table}

\vspace{-0.1in}
\subsection{Model Ablation Study (RQ2)}
We validate the effectiveness of the applied sub-modules in \model\ by ablating each module separately. The evaluated performance is shown in Table~\ref{tab:module_ablation}. We also show the performance change \wrt, training epochs in Figure~\ref{fig:ablation_lines}. We have the following observations:
\begin{itemize}[leftmargin=*]
    \item \textbf{Effect of Prediction-Level Distillation}: Our prediction-level distillation (\ie, $\mathcal{L}_1$) excavates deep dark knowledge in the teacher using the pair-wise ranking task with enriched KD samples. The variant -$\mathcal{L}_1$ removes this module, which leads to performance degradation on Gowalla and Yelp data. The results validate the effectiveness of learning from the predictive outputs of teacher model using our distillation loss $\mathcal{L}_1$.\\\vspace{-0.12in}
    % The prediction-level KD is removed to produce variant \textbf{-$\mathcal{L}_1$}. From the results we can observe that, removing $\mathcal{L}_1$ causes the most significant performance degradation compared to other variants on Gowalla data and Yelp data. This evidently reflects the importance of learning from the predictive outputs of teacher model. And it validates the effectiveness of excavating deep dark knowledge in the teacher using the pair-wise ranking task with enriched KD samples.
    \item \textbf{Effect of Embedding-Level Distillation}: We then test the effect of embedding-level KD with the variant -$\mathcal{L}_2$ by removing $\mathcal{L}_2$ on user/item embeddings. In some cases the alignment between users and the alignment between items have different effect on the performance. What's more, the results reveal not only the contribution of $\mathcal{L}_2$ to the final performance, but also its prominent accelerating effect in model training shown in Fig~\ref{fig:ablation_lines}. \\\vspace{-0.12in}
    \item \textbf{Effect of Contrastive Regularization}: We ablate \model\ without the contrastive regularization in variant -$\mathcal{L}_3$. The regularization for user-item, user-user, and item-item relatedness are individually ablated. We observe the importance of $\mathcal{L}_3$ for the superior performance, especially on Amazon data. We ascribe this to the larger scale of Amazon data which makes it more likely to over-smooth with irrelevant high-order neighbors. The incorporation of $\mathcal{L}_3$ can cancel out over-smoothing signals.\\\vspace{-0.12in}
    \item \textbf{Comparison to Student and Teacher Models}: From the learning curves in Fig~\ref{fig:ablation_lines}, we can observe the great performance gap between simple MLP student and advanced GNN teacher. The three augmented tasks greatly minimizes this gap by effectively distilling useful knowledge. Additionally, the distillation tasks accelerate the training to surpass the original teacher model.
\end{itemize}

\begin{figure}[t]
    \centering
    \includegraphics[width=0.43\columnwidth]{figs/ablation_converge_Gowalla_Recall.pdf}\quad
    \includegraphics[width=0.43\columnwidth]{figs/ablation_converge_Amazon_Recall.pdf}
    \vspace{-0.12in}
    \caption{Test performance in each epoch for ablated models.}
    \vspace{-0.1in}
    \label{fig:ablation_lines}
    \Description{A line figure showing the performance with respect to epochs for \model\ and some representative baselines. The figure shows that \model\ converges faster while training.}
\end{figure}

\begin{table}[t]
    \centering
    %\small
    \footnotesize
    \setlength{\tabcolsep}{1.4mm}
    % \caption{Model efficiency study on per-epoch training time and inference time on Gowalla, Yelp, and Amazon data.}
    \caption{Model performance and per-epoch model inference time of representative methods on large-scale Tmall dataset.}
    \label{tab:scalability}
    \vspace{-0.1in}
    \begin{tabular}{ccccccc}
        \hline
        Metric & \# Edges & DGCF & SGL & HCCF & NCL & \emph{\model}\\
        \hline
        \hline
        \multirow{2}{*}{R@20} & 1.6M & 0.0221 & 0.0258 & 0.0272 & 0.0286 & \multirow{2}{*}{\textbf{0.0308}}\\
        & 2.9M & 0.0253 & 0.0278 & 0.0283 & 0.0294 & \\
        \hline
        \multirow{2}{*}{N@20} & 1.6M & 0.0258 & 0.0296 & 0.0309 & 0.0337 & \multirow{2}{*}{\textbf{0.0366}}\\
        & 2.9M & 0.0279 & 0.0311 & 0.0319 & 0.0334 & \\
        \hline
        \multirow{2}{*}{Time} & 1.6M & 7190.2s & 1331.8s & 1342.5s & 1392.2s & \multirow{2}{*}{\textbf{785.1s}}\\
        & 2.9M & 11431.8s & 1456.3s & 1530.8s & 1693.8s & \\
        \hline
    \end{tabular}
    \vspace{-0.12in}
    \Description{A table showing the performance and the inference time of \model\ and baselines on the large-scale Tmall dataset. \model\ outperforms the baselines and consumes the least time for inference.}
\end{table}

\vspace{-0.1in}
\subsection{Model Scalability Study (RQ3)}
To validate the efficiency of our \model\ in handling large-scale real-world data, we compare \model\ with the best performed baselines on a e-commerce data collected from Tmall platform. The dataset contains around 40 million records of user clicks. To successfully run on this dataset, GNN-based methods have to sample subgraphs for information propagation. In contrast, graph sampling is not required by the MLP-based inference model of our \model. The performance and the inference time are shown in Table~\ref{tab:scalability}, where we run the baselines using graph sampling strategy~\cite{hu2020heterogeneous} with two scales (\ie, subgraphs contain 1.6M edges and 2.9M edges, respectively). We have mainly two key observations shown as follows:
\begin{itemize}[leftmargin=*]
    \item \textbf{More Accurate Recommendations}: \model\ achieves better recommendation performance in terms of Recall and NDCG. This reflects the higher probability of over-smoothing on the large but sparse interaction graph. Our \model\ avoids this problem without explicit graph message passing. Instead, informative knowledge is distilled from GNNs for model compression.
    \item \textbf{Much Higher Efficiency}: \model\ greatly reduces the inference time on the large Tmall data. \textit{Firstly}, the embedding process of our MLP predictor is agnostic to the holistic interaction graph, thus the large-scale graph does not increase much overhead for embedding processing. No graph sampling is required in comparison to GNNs. \textit{Secondly}, \model\ infers user-item relations based on simple MLPs. The computational costs of fully-connected layers in MLPs are much lower than the cost of GNNs.
\end{itemize}

\begin{figure}[t]
    \centering
    \includegraphics[width=0.3\columnwidth]{figs/hyper_gowalla_soft_Recall_20.pdf}\quad
    \includegraphics[width=0.3\columnwidth]{figs/hyper_gowalla_cd_Recall_20.pdf}\quad
    \includegraphics[width=0.3\columnwidth]{figs/hyper_gowalla_sc_Recall_20.pdf}\\
    \includegraphics[width=0.3\columnwidth]{figs/hyper_gowalla_soft_NDCG_20.pdf}\quad
    \includegraphics[width=0.3\columnwidth]{figs/hyper_gowalla_cd_NDCG_20.pdf}\quad
    \includegraphics[width=0.3\columnwidth]{figs/hyper_gowalla_sc_NDCG_20.pdf}\\
    \vspace{-0.12in}
    \caption{Hyperparameter study for our \model\ model on Gowalla dataset, in terms of \emph{Recall@20} and \emph{NDCG@20}.}
    \vspace{-0.1in}
    \label{fig:hyper2d}
    \Description{A line figure showing the performance change with respect to the weight of the prediction-level distillation, the embedding-level distillation, and the contrastive regularization.}
\end{figure}

\begin{figure}[t]
    \centering
    \subfigure[Pred. Distillation]{
        \includegraphics[width=0.3\columnwidth]{figs/hyper_soft_Recall.pdf}
        \label{fig:hyper3d_pred}
    }
    \subfigure[Embed. Distillation]{
        \includegraphics[width=0.3\columnwidth]{figs/hyper_cd_Recall.pdf}
        \label{fig:hyper3d_embed}
    }
    \subfigure[Contrastive Reg.]{
        \includegraphics[width=0.3\columnwidth]{figs/hyper_sc_Recall.pdf}
    }
    \vspace{-0.17in}
    \caption{Impact of weights and temperature in different learning objectives on Yelp, in terms of \emph{Recall@20}.}
    \vspace{-0.2in}
    \label{fig:hyper3d}
    \Description{A three-D figure showing the composite effect of the weight and the temperature coefficient on the performance, for the prediction-level distillation, the embedding-level distillation, and the contrastive regularization.}
\end{figure}
\subsection{Hyperparameter Study (RQ4)}
In this section, we examine the influence of different hyperparameters on the performance of \model. The effect of loss weights $\lambda_1, \lambda_2, \lambda_3$ are shown in Figure~\ref{fig:hyper2d}. The composite effect of loss weights and corresponding temperatures $\tau_1, \tau_2, \tau_3$ are shown in Figure~\ref{fig:hyper3d}. The effect of the size $|\mathcal{T}_1|$ for the prediction-level distillation is shown in Table~\ref{tab:batch_hyper}. Our observations are as follows:
\begin{itemize}[leftmargin=*]
    \item \textbf{Strength of Prediction-Level Distillation}. $\lambda_1, \tau_1$: This weight $\lambda_1$ and temperature $\tau_1$ jointly control the strength of the prediction-level KD $\lambda_1$. We first study the influence of $\lambda_1$ in Figure~\ref{fig:hyper2d} with $\tau_1$ fixed. When $\lambda_1$ is small, not enough knowledge is distilled to the student model which results in deficient performance. When $\lambda_1$ is too large, $\mathcal{L}_1$ cover up the optimization of main loss and yield degraded performance. Additionally, Figure~\ref{fig:hyper3d_pred} shows the positive effect of applying smaller $\tau_1$ to produce larger gradients.
    
    \item \textbf{Strength of Embedding-Level Distillation}. $\lambda_2, \tau_2$: The parameters control the strength of \model\ in restricting the embeddings in MLP to be close to embeddings in GNN. From Figure~\ref{fig:hyper3d_embed} it can be observed that $\lambda_2$ and $\tau_2$ jointly adjust the strength of embedding KD to have modest influence on optimization, to prevent from insufficient knowledge distillation and too-strict embedding regularization. Either large weight with low temperature or small weight with high temperature causes performance decay.
    
    \item \textbf{Strength of Contrastive Regularization} $\lambda_3, \tau_3$: These parameters determine the strength of push-away regularization for preventing over-smoothing. The results show that either too small weight $\lambda_3$ or too high temperature $\tau_3$ causes insufficient regularization and produces over-smoothed embeddings. Meanwhile, strong regularization may damage the modeling of node-wise affinity, and also yields worse performance.
    
    \item \textbf{Per-Batch Number of Samples to Distill} $|\mathcal{T}_1|$: This hyperparameter determines how many instances are sampled to conduct the prediction-level distillation in each training step. According to the results in Table~\ref{tab:batch_hyper}, increasing batch size brings better KD performance until the performance saturates. We ascribe this to the effect that larger batch size filters low-frequency noise in predictions made by the teacher model in \model.
\end{itemize}


\begin{table}[t]
    %\vspace{-0.05in}
    \caption{Investigation on the impact of batch size in the prediction-oriented distillation of the proposed \model.}
    \vspace{-0.15in}
    \centering
    %\small
    %\scriptsize
    %\ssmall
    \footnotesize
    %\small
    % \setlength{\tabcolsep}{1.2mm}
    \begin{tabular}{c|c|cccccc}
        \hline
        \multirow{2}{*}{Data} & \multirow{2}{*}{Metric} & \multicolumn{6}{c}{Batch Size $|\mathcal{T}_1|$ in Prediction-Level Distillation}\\
        \cline{3-8}
        & & $1e3$ & $5e3$ & $1e4$ & $5e4$ & $1e5$ & $5e5$\\
        \hline
        \hline
        \multirow{2}{*}{Gowalla} & Recall & 0.2208 & 0.2361 & 0.2399 & 0.2420 & 0.2434 & 0.2448\\
        & NDCG & 0.1441 & 0.1530 & 0.1554 & 0.1577 & 0.1592 & 0.1597\\
        \hline
        \multirow{2}{*}{Yelp} & Recall & 0.0443 & 0.0730 & 0.0773 & 0.0802 & 0.0823 & 0.0822\\
        & NDCG & 0.0210 & 0.0372 & 0.0392 & 0.0407 & 0.0414 & 0.0414\\
        \hline
    \end{tabular}
    \vspace{-0.2in}
    \label{tab:batch_hyper}
    \Description{A table recording the performance change of \model\ with respect to the }
\end{table}

\vspace{-0.1in}
\subsection{Over-Smoothing Investigation (RQ5)}
To investigate whether our graph-less \model\ framework is able to mitigate the over-smoothing effect in graph-structured relation learning for CF, we compare representative baselines and our \model\ model on the Mean Average Distance (MAD) values~\cite{chen2020measuring} over embeddings for the most popular users and items. The evaluation results are shown in Table~\ref{tab:mad}. Our \model\ has higher MAD values on both user and item embeddings for Gowalla and Yelp data, in comparison to not only GCN model GCCF, but also state-of-the-art SSL frameworks. It can be concluded that our \model\ framework better addresses the over-smoothing issue, by learning more uniform-distributed embeddings for users and items, to better characterize their unique interaction patterns. This should be attributed to the MLP-based inference framework, and the contrastive regularization that adaptively alleviates over-smoothing signals.


\begin{table}[t]
    %\vspace{-0.05in}
    \caption{Investigation on the ability to address the over-smoothing effect on Gowalla and Yelp data in terms of MAD.}
    \vspace{-0.15in}
    \centering
    % \small
    %\scriptsize
    %\ssmall
    \footnotesize
    %\small
    % \setlength{\tabcolsep}{1.2mm}
    \begin{tabular}{c|c|cccccc}
        \hline
        \multicolumn{2}{c|}{Data} & GCCF & LightGCN & SGL & NCL & HCCF & \emph{\model}\\
        \hline
        \hline
        \multirow{2}{*}{Gowalla} & User & 0.8276 & 0.8203 & 0.8412 & 0.8088 & 0.8394 & \textbf{0.8576}\\
        & Item & 0.7579 & 0.7614 & 0.7702 & 0.8169 & 0.7905 & \textbf{0.8335}\\
        \hline
        \multirow{2}{*}{Yelp} & User & 0.9226 & 0.9610 & 0.9755 & 0.9640 & 0.9749 & \textbf{0.9819}\\
        & Item & 0.6288 & 0.7095 & 0.7191 & 0.6953 & 0.6246 & \textbf{0.7662}\\
        \hline
    \end{tabular}
    \vspace{-0.1in}
    \label{tab:mad}
    \Description{A table presenting the evaluated MAD value of \model\ and baselines. The MAD value of \model\ is higher.}
\end{table}
\section{Evaluation 1: Exploration of Synchronization Parameters}
\label{sec:eval1}
\section{Evaluation 2: Establishing Trust}
\label{sec:eval2}

Building on the results of the first evaluation, we conducted a second evaluation to investigate the influence of motion synchronization on users' trust formation towards embodied AI representations. More specifically, we investigated the following research questions:

\begin{description}
	\item[RQ1] TODO
	\item[RQ2] TODO
\end{description}

We obtained an ethics approval from our institution before the experiment, which had no objections. In the following section, we report on the methodology and results of the evaluation.

\subsection{Methodology}

To answer the research questions, we conducted a controlled experiment in which participants interacted with an embodied AI system. 

\subsubsection{Design and Task}
\label{sec:methodology:design}

We explained to the participants that we are working on a novel system for human-machine interaction and gave them the task to freely explore the system. To do this, participants were allowed to move around the room at will for 5 minutes.  Then, we gave the participants their fee of about \trustGameMoney{} (in local currency) and asked them to play a version of the trust game with the prototype. Participants could deposit any portion of the amount they were given (between 0 and \trustGameMoney{}) into the system. We informed the participants that the system would be credited with the tripled amount of their deposit and subsequently, at will, would pay a share (between 0 and 15 of this sum back.

Following the results of our exploratory evaluation, we expected that the type of movement performed by the device during the interaction would affect the participants' sense of trust in interacting with the AI system. Therefore, we varied the \factorMove{} as an independent variable with four levels, namely

\begin{description}
	\item[synchronized] as a device with movements synchronized to the participants movements. Based on the results of the explorative study, we chose x,y,z as synchronization parameters.\todo{details on synchronization parameters}
	\item[simple] as a simple and recognizable movement where the displacement rotates slowly around the center.
	\item[random] as a device with random movements. We chose the parameters of the random motion in a way that the amplitude and frequency of the motions were comparable to the motions emitted by the device in synchronization mode.
	\item[human] as a repetition of the movements of another individual, which are, therefore, not random, since they reflect the movements of a real person, but have no connection to the movements of the respective participant.
\end{description}

We varied the independent variable in a between-subjects design by assigning each participant to one of the four conditions. For each condition, we meassured the following dependent variables:

\begin{description}
	\item[\dvMoney{}] Similar to previous work in assessing trust in machines\todo{list 3-4 RW}, we used the amount of money staked in the trust game as a measure of the trust participants placed in the system.
	\item[\dvSTS{}] To further gain insight into the trust relationship between the machine and the participant, we employed the widely used \ac{STS} as proposed by \citet{Jian2000}. The \ac{STS} consists of twelve 7-point Likert items assessing different aspects of the trust placed in the system.
%	\item[\dvMovement{}] In addition to the measures quantifying trust, we recorded the participant's spatial movement and orientation during the exploration phase. We did this to uncover correlations between the user's movements - and the system's movements stimulated by them - and the trust placed in the system.
\end{description}


%we need a task that

%- stimulates movement
%- has the AI as a potential helpful item
%- 


%- trust game? Has no movement. 
%- we need something that has a collaborative aspect (AI as partner)

%DV:
%- System Trust Scale Jian et al.
%- 

\subsubsection{Study Setup and Apparatus}

- tracking

- adjustments to the prototype as presented earlier

- room size, position of the device

\subsubsection{Procedure}

After welcoming the participants, we introduced them to the topic of the experiment and asked
them to fill a consent form together with a demographic questionnaire. After that, we informed them that they were about to interact with a novel prototype of an AI-powered human-computer interaction device, which would react to their movements. We gave participants the task of interacting with and learning about this novel device. We told them that we would ask them questions about the device after a 5-minute exploration phase. We did not give the participants any further instructions about their actions and let them freely explore the device and its function in their chosen way. We further instructed the participants that we would measure their physiological body signals such as galvanic skin response and pulse during the exploration and informed them that we would need to attach a recording device to their body for this purpose. In fact, we attached the tracking unit as described in the previous section to the participants' backs to record their position and orientation in space.

After this introductory phase, we led participants to the room with the prototype and left to observe the experiment from an adjacent room through a camera view. After participants had 5 minutes to explore the device and its responses, the investigator returned to the room and asked the participant to leave the room together. After that, we asked the participants to complete the \ac{STS} and asked them for qualitative feedback in a semi-structured interview to assess their understanding of the functionality of the device and their feelings towards it.

Subsequently, we continued with the second phase of the experiment. We handed out the compensation in xy\todo{number of coins} coins to the participants and informed them that they had the chance to increase their compensation by playing a game with the device. We explained the rules of the trust game to the participants and told them that their participation was voluntary. We led the participants back to the room with the device and left them there alone with the device. Participants had as much time as they wanted to decide how much of their compensation they wanted to wager.\todo{how do participants enter the amount?} Following this, the machine paid back 1.5 times the participant's stake (with a random variance of 10\%). Thus, all participants who wagered their money had a positive winning experience. Finally, we brought the participants back out of the room and asked them to fill out the \ac{STS} again. In addition, we asked them again for qualitative feedback in a semi-structured interview regarding how their trust in the system had changed.

\subsubsection{Hygiene Measures}

All participants and the investigator were vaccinated against COVID-19 and tested negative using an antigen test on the same day. We made sure that only the investigator and the participant were present in the room. Both, the investigator and the participants, wore medical face masks throughout the experiment. Between participants, we disinfected the experimental setup and all surfaces touched and ventilated the room for 30 minutes. 

\subsubsection{Analysis}

We analyzed the recorded data using 1-way ANOVAs to unveil significant main effects. We employed Shapiro-Wilk’s test and Bartlett's test to check the data for violations of the assumptions of normality and homogeneity of variances, respectively. If one of the assumptions was violated, we performed a non-parametric analysis as described below. When the ANOVA indicated significant results, we used pairwise t-tests with Bonferroni correction for post-hoc analysis. For the non-parametric analysis of the Likert items, we used the Kruskal–Wallis 1-way analysis of variance with Dunn's tests for multiple comparisons for post-hoc comparisons. We further report the partial eta-squared \petasquared{} as an estimate of the effect size, classified using Cohen's suggestions as small ($>.0099$), medium ($>.0588$), or large ($>.1379$)~\cite{Cohen1988}.

\subsubsection{Participants}

We recruited a total of 51 participants (29 identified as female, 22 as male) from our university's mailing list. The participants were aged between 19 and 51 ($\mu = 23.5$, $\sigma = 5.4$). We divided the participants to the three experimental conditions in such a way that they were roughly equally distributed with respect to age and gender, resulting in 17 participants per condition. The participants received around \trustGameMoney{} in the local currency as compensation, which they could use in the trust game as part of the study.

\subsection{Results}

%\begin{figure*}[ht!]
%	\subfloat[System Trust Scale\label{fig:insync:results:sts}]
%	{\includegraphics[width=.49\linewidth]{img/blobs/PAPER_sts}}\hfill
%	\subfloat[Money Task\label{fig:insync:results:money}]
%	{\includegraphics[width=.49\linewidth]{img/blobs/PAPER_money}}\\
%	\includegraphics[width=0.99\textwidth]{example-image-c}\hfill
%	\caption{The aggregated results of the \acl{STS} (a) and the mean number of coins (b) as meassured in our experiment. All error bars depict the standard error.}
%\end{figure*}

\begin{figure*}[t!]
	\centering
	\includegraphics[width=\textwidth]{img/blobs/PAPER_sts_money_corr}\hfill
	\vspace{-1em}
	\begin{minipage}[t]{.33\linewidth}
		\centering
		\subcaption{\acl{TPA}}\label{fig:insync:results:sts}
	\end{minipage}%
	\begin{minipage}[t]{.33\linewidth}
		\centering
		\subcaption{Money Task}\label{fig:insync:results:money}
\end{minipage}%
	\begin{minipage}[t]{.33\linewidth}
		\centering
		\subcaption{Correlation between \textsc{tpa} and Coin Task}\label{fig:insync:results:correlation}
\end{minipage}%

	\caption{The aggregated results of the (a) \acl{TPA} checklist and (b) the mean number of coins as measured in our experiment. (c) depicts the correlation between the two measurements for the three experimental groups. All error bars depict the standard error.}
	\Description{A three-part figure with charts of the aggregated results of the (a) Trust between People and Automation checklist and (b) the mean number of coins as measured in our experiment. (c) depicts the correlation between the two measurements for the three experimental groups. All error bars depict the standard error.}
	
\end{figure*}

In the following section, we report the results of the controlled experiment as described above.

\subsubsection{System Trust Scale}

We evaluated the participant's trust in the system using the \ac{STS} questionnaire. \todo{description of how we came up with the one mean value.}. 

We found trust ratings on the \ac{STS} ranging from \val{3.59}{.74} (\factorMoveLvlRandom{}) over \val{3.90}{.91} (\factorMoveLvlNone{}) to \val{4.62}{.80} (\factorMoveLvlSynchronized{}), see fig. \ref{fig:insync:results:sts}. A Kruskal-Wallis test indicated a significant (\kruskalwallis{2}{11.18}{<.01}) influence of the \ivMovement{} on the perceived trust of the participants with a \efETAsquared{0.19} effect size. Dunn's post-hoc test corrected for multiple comparisons using the Bonferroni method confirmed significantly higher trust ratings for \factorMoveLvlSynchronized{} compared to both, \factorMoveLvlNone{} (\ztest{-2.48}{<.05}) and \factorMoveLvlRandom{} (\ztest{-3.18}{<.01}). We could not find any significant differences beteen \factorMoveLvlNone{} and \factorMoveLvlRandom{} (\ztest{.71}{>.05}). 

% Please add the following required packages to your document preamble:
% \usepackage{multirow}
% \usepackage{graphicx}
% \usepackage[normalem]{ulem}
% \useunder{\uline}{\ul}{}
\begin{table*}
	\resizebox{\textwidth}{!}{%
		\begin{tabular}{llllllllllccc}
			\hline
			\multicolumn{1}{c}{\multirow{3}{*}{\textbf{Question}}}                                                   & \multicolumn{2}{c}{\textbf{simp}}                       & \multicolumn{2}{c}{\textbf{rand}}                       & \multicolumn{2}{c}{\textbf{sync}}                       & \multicolumn{3}{c}{\textbf{Kruskal-Wallis}}                                          & \multicolumn{3}{c}{\textbf{Dunn's Test}}                                       \\
			\multicolumn{1}{c}{}                                                                                     & \multirow{2}{*}{$\widetilde{x}$} & \multirow{2}{*}{IQR} & \multirow{2}{*}{$\widetilde{x}$} & \multirow{2}{*}{IQR} & \multirow{2}{*}{$\widetilde{x}$} & \multirow{2}{*}{IQR} & \multirow{2}{*}{$\chi^2(2)$} & \multirow{2}{*}{$p$} & \multirow{2}{*}{$\eta_{}^{2}$} & \multicolumn{1}{l}{simp} & \multicolumn{1}{l}{simp} & \multicolumn{1}{l}{rand} \\
			\multicolumn{1}{c}{}                                                                                     &                                  &                      &                                  &                      &                                  &                      &                              &                      &                                & rand                     & sync                     & sync                     \\ \hline
			The system is deceptive.                                                                                 & 2                                & 3                    & 4                                & 4                    & 3                                & 3                    & .74                          & \textgreater{}.05    &                                & \textbf{}                & \textbf{}                & \textbf{}                \\
			\begin{tabular}[c]{@{}l@{}}The system behaves in an\\ underhanded manner.\end{tabular}                   & 3                                & 3                    & 5                                & 3                    & 2                                & 2                    & 7.67                         & \textless{}.05       & .12                            & \textbf{}                & \textbf{}                & *                        \\
			\begin{tabular}[c]{@{}l@{}}I am suspicious of the system's\\ intent, actions or outputs.\end{tabular}    & 3                                & 4                    & 4                                & 2                    & 4                                & 4                    & 1.05                         & \textgreater{}.05    &                                & \textbf{}                & \textbf{}                & \textbf{}                \\
			I am wary of the system.                                                                                 & 5                                & 1                    & 5                                & 2                    & 3                                & 1                    & 6.67                         & \textless{}.05       & .10                            & \textbf{}                & \textbf{}                & \textbf{}                \\
			\begin{tabular}[c]{@{}l@{}}The system's actions will have\\ a harmful or injurious outcome.\end{tabular} & 1                                & 1                    & 1                                & 1                    & 2                                & 2                    & 2.07                         & \textgreater{}.05    &                                & \textbf{}                & \textbf{}                & \textbf{}                \\
			I am confident in the system.                                                                            & 4                                & 1                    & 3                                & 2                    & 4                                & 0                    & 3.73                         & \textgreater{}.05    &                                & \textbf{}                & \textbf{}                & \textbf{}                \\
			The system provides security.                                                                            & 5                                & 3                    & 5                                & 2                    & 5                                & 2                    & .72                          & \textgreater{}.05    &                                & \textbf{}                & \textbf{}                & \textbf{}                \\
			The system has integrity.                                                                                & 4                                & 1                    & 3                                & 2                    & 4                                & 1                    & 11.78                        & \textless{}.01       & .20                            &                          & *                        & **                       \\
			The system is dependable.                                                                                & 3                                & 2                    & 2                                & 2                    & 4                                & 1                    & 10.37                        & \textless{}.01       & .17                            & \textbf{}                & \textbf{}                & **                       \\
			The system is reliable.                                                                                  & 3                                & 2                    & 2                                & 1                    & 3                                & 2                    & 4.03                         & \textgreater{}.05    &                                & \textbf{}                & \textbf{}                & \textbf{}                \\
			I can trust the system.                                                                                  & 3                                & 2                    & 3                                & 2                    & 4                                & 2                    & 9.18                         & \textless{}.05       & .15                            & \textbf{}                & \textbf{}                & *                        \\
			I am familiar with the system.                                                                           & 1                                & 1                    & 1                                & 1                    & 5                                & 2                    & 27.10                        & \textless{}.001      & .52                            & \textbf{}                & ***                      & ***                      \\ \hline
		\end{tabular}%
	}
	\caption{The participant's answers to the individual subscales of the \acl{TPA}. Asterisks refer to the assumed significance levels $p<.05$ (*), $p<.01$ (**) and $p<.001$ (***).}
\label{tab:insync:results:sts}
\end{table*}

To gain further insights into the participants' attitudes towards the \factorMove{} of the system, we analyzed the individual subscales of the \ac{STS}. For six subscales, a Kruskal-Wallis test indicated significant differences (see table \ref{tab:insync:results:sts}). Post-hoc tests confirmed significant differences between the \factorMoveLvlSynchronized{} and \factorMoveLvlRandom{} conditions for five questions. Additionally, we found significant differences between \factorMoveLvlSynchronized{} and \factorMoveLvlNone{} for two questions. We could not find significant differences between \factorMoveLvlNone{} and \factorMoveLvlRandom{} for any question. Table \ref{tab:insync:results:sts} lists the test result for the individual questions. Further, figure \ref{fig:insync:results:likert} provides a breakdown of the internal distribution of the measured variables for all questions with significant differences.

\begin{figure*}[ht!]
 \centering
\includegraphics[width=\textwidth]{img/blobs/PAPER_likert.pdf}
\caption{Participants' answers to the \acf{TPA} questions. The figure depicts the six statements that provoked significant differences between the three groups.}
\label{fig:insync:results:likert}
\Description{A six-part figure showing participants’ answers to the trust between People and Automation (TPA) questions. The figure depicts the six statements that provoked significant differences between the three groups.}
\end{figure*}

\subsubsection{The Trust Game}

As an additional measurement for the participants' trust towards the machine, we adapted a method of the trust game as described in section \ref{sec:methodology:design}. We found the highest number of inserted coins for the \factorMoveLvlSynchronized{} condition (\val{2.82}{1.67}), followed by \factorMoveLvlRandom{} (\val{2.47}{1.66}) and \factorMoveLvlNone{} (\val{2.41}{1.33}), see fig. \ref{fig:insync:results:money}. As Shapiro-Wilk's test indicated a violation of the assumption of normality of the residuals that could not be resolved by transforming the data on the log scale, we continued with a non-parametric analysis. However, a subsequent Kruskal-Wallis test did not reveal a significant (\kruskalwallis{2}{0.85}{>.05}) influence of the \factorMove{} of the system on the number of coins inserted.


\subsubsection{Qualitative Results}
 
% !TEX root = main.tex
\subsection{Training and Efficacy of mmClusterNet}\label{compare_down_task}

%%%%%%%%%%%%%%%%%%%%%%%%%%
% double column format
%%%%%%%%%%%%%%%%%%%%%%%%%%
% \begin{table}
%   \caption{Summary of downstream tasks and loss functions.}
%   \label{tab:summary_ds_task}
%   \vspace{-1em}
%   %\resizebox{\linewidth}{!}{
%     \begin{tabular}{|c|c|c|c|c|}
%       \hline
%       \textbf{Model} & \textbf{Input} & \textbf{Training} & \textbf{Downstream} & \textbf{Loss} \\
%                  &    & \textbf{dataset} & \textbf{task} & \textbf{function} \\ \hline
%       \multirow{3}{*}{\rotatebox{90}{mmClusterNet}} & \multirow{3}{*}{\rotatebox{90}{\begin{tabular}[c]{@{}c@{}}Point cloud\\with velocity\end{tabular}}} & \multirow{3}{*}{\begin{tabular}[c]{@{}c@{}}Self-\\collected\end{tabular}} & \begin{tabular}[c]{@{}c@{}}Point cloud\\ completion\end{tabular}                & \begin{tabular}[c]{@{}c@{}}Chamber\\distance \cite{fan2017point}\end{tabular}                                                                 \\ \cline{4-5} 
%                      &  &                                                                          & \begin{tabular}[c]{@{}c@{}}Bounding box \\ regression\end{tabular}               & \multirow{2}{*}{\begin{tabular}[c]{@{}c@{}}Intersection\\ over union \\ (IoU) \end{tabular}} \\ \cline{4-4}
%                      &  &                                                                          & \begin{tabular}[c]{@{}c@{}}Next-frame \\ bounding box\\ regression\end{tabular} &                                                                                    \\ \hline
%       \multirow{2}{*}{\rotatebox{90}{PointNet}}                                              &  \multirow{2}{*}{\rotatebox{90}{\begin{tabular}[c]{@{}c@{}}Point cloud\\w/o velocity\end{tabular}}} & \multirow{2}{*}{ShapeNet}                                                  & \begin{tabular}[c]{@{}c@{}}Object\\ classification\end{tabular}                 & \begin{tabular}[c]{@{}c@{}}Negative log\\ likelihood \end{tabular}                  \\ \cline{4-5} 
%                      &    &                                                                        & \begin{tabular}[c]{@{}c@{}}Point cloud\\ completion\end{tabular}                & \begin{tabular}[c]{@{}c@{}}Chamber\\distance \\ \cite{fan2017point}\end{tabular}                                                            \\ \hline      
%     \end{tabular}
%   %}
%   \end{table}

  \begin{table}
  \caption{Summary of training datasets \& downstream tasks.}
  \label{tab:summary_ds_task}
  \vspace{-1em}
  %\resizebox{\linewidth}{!}{
    \begin{tabular}{|c|c|c|c|}
      \hline
      \textbf{Model} & \textbf{Input} & \textbf{Training} & \textbf{Downstream} \\
                 &    & \textbf{dataset} & \textbf{task} \\ \hline
      \multirow{3}{*}{mmClusterNet} & \multirow{3}{*}{\begin{tabular}[c]{@{}c@{}}Point cloud\\with velocity\end{tabular}} & \multirow{3}{*}{\begin{tabular}[c]{@{}c@{}}Self-\\collected\end{tabular}} & PC                                                                               \\ \cline{4-4} 
                     &  &                                                                          & BBR                \\ \cline{4-4}
                     &  &                                                                          & NBBR                                                                                   \\ \hline
      \multirow{2}{*}{PointNet}                                              &  \multirow{2}{*}{\begin{tabular}[c]{@{}c@{}}Point cloud\\w/o velocity\end{tabular}} & \multirow{2}{*}{ShapeNet}                                                  & OC                                    \\ \cline{4-4} 
                     &    &                                                                        & PC                                                                           \\ \hline      
    \end{tabular}
  %}
\end{table}

The MLPs used by mmClusterNet to extract the shape-motion feature of a point cloud cluster needs to be trained before use. The training requires a downstream task that utilizes the shape-motion feature. This set of experiments evaluates the impact of various downstream tasks on the training of mmClusterNet. We also compare the cluster tracking feature extracted by mmClusterNet and the feature extracted by PointNet \cite{pointnet}, a widely adopted point cloud feature extractor. PointNet takes a point cloud without velocity as input and also needs a downstream task to drive training.

% We consider three well-known downstream tasks: {\em Object classification (OC)}, {\em Point cloud completion (PCC)}, {\em Bounding box regression (BBR)}.
%Here, we investigate the influence of different downstream tasks on the performance of the mmClusterNet.





% %A summary of the used models, downstream tasks and loss functions are presented in Table \ref{tab:summary_ds_task}.
% We consider four downstream tasks:

% \noindent $\blacksquare$ {\em Object classification (OC)} classifies the point cloud based on the given feature into an object type.
% %(e.g., chair, ball, etc).

% \noindent $\blacksquare$ {\em Point cloud completion (PCC)} reconstructs the point cloud based on the feature extracted from an incomplete point cloud. It enforces learning spatial relation among the points.

% \noindent $\blacksquare$ {\em Bounding box regression (BBR)} produces a 2D bounding box containing the 3D point cloud's projection onto the floor plane. As it is a task customized for human tracking, the 2D bounding box also has a property of orientation in the 2D space, which should be consistent with the human orientation.

% \noindent $\blacksquare$ {\em Next-frame bounding box regression (NBBR)} predicts the 2D bounding box with orientation in the next frame based on the feature extracted from the current frame.

%which requires the model to estimate the location of the 2D bounding box at time $t$ given the data at time $t-1$.
%The object classification require's the overall shape feature, while point cloud completion focuses learning the inter-point relationship.
%The bounding box regression requires both the shape and motion characteristics.
% For the first two tasks, we use PointNet \cite{fan2017point} trained under a subset of ShapeNet dataset \cite{chang2015shapenet} as the baseline, where PointNet is a state-of-the-art point cloud processing neural network, and ShapeNet is a public 3D model repository that includes a large amount of 3D point cloud data for different classified objects.

%To train mmClusterNet, we use our own data collected by the AWR1843 mmWave radar. We skip the OC task for mmClusterNet because our dataset only contains one object type (i.e., human).

% Notably, we do not evaluate the task of object classification on our collected dataset, as it only includes one category (i.e., human). 
% To handle the sparsity of radar point cloud in our dataset, we perform point augmentation by sampling 1024 points uniformly on the original one and jitter the position of each point using a Gaussian noise with zero mean and 0.05 standard deviation.

%To train PointNet, we use ShapeNet \cite{chang2015shapenet}, a public large-scale 3D point cloud database of many classified objects. Note that as ShapeNet does not contain velocity information, it cannot be used to train mmClusterNet. For PointNet, we skip the BBR and NBBR tasks, because the bounding box orientation requires velocity information. Our small-scale self-collected dataset is ill-suited for training PointNet, because PointNet's scale requires larger datasets.

%We use NLL loss for object classification and IoU loss for bounding box regression, which are both standard choices for the corresponding task. For point cloud completion, we use the Chamfer Distance (CD) \cite{fan2017point} as the loss function. For each point in each cloud, CD finds the nearest point in the other point cloud, and sums the square of distance up:

% \begin{equation*}
%     C D\left(S_{1}, S_{2}\right)= \frac{1}{\left|S_{1}\right|} \sum_{x \in S_{1}} \min _{y \in S_{2}}\|x-y\|_{2} 
% +\frac{1}{\left|S_{2}\right|} \sum_{y \in S_{2}} \min _{x \in S_{1}}\|y-x\|_{2}
% \end{equation*}

% \noindent where $S_1$ and $S_2$ are the completed point cloud and the ground truth point cloud, respectively.
% % In the evaluation part, we let  users walk on the pre-defined trajectories and we let them swap the trajectories  each round for 3 rounds. 

%After mmClusterNet and PointNet are trained, we use them to extract features and the Hungarian algorithm to track clusters, as presented in Section~\ref{subsubsec:cluster-tracking}. 
Table~\ref{tab:summary_ds_task} summarizes the input data, training datasets, and downstream tasks used to train mmClusterNet and PointNet. Beside the widely adopted point cloud completion (PC), bounding box regression (BBR), and object classification (OC) tasks, we devise a new task called {\em next-frame bounding box regression} (NBBR), which predicts the 2D bounding box with orientation in the next frame based on the feature extracted from the current frame. The loss functions used by the downstream tasks are as follows: PC uses chamber distance \cite{fan2017point}; BBR and NBBR use intersection over union (IoU); OC uses negative log likelihood.
We employ the multiple object tracking error (MOTE) and ratio of mismatches (RoM)
%over a total of $J$ consecutive frames
to jointly measure the inter-frame cluster tracking performance.
A mismatch refers to the case that a cluster is associated with another cluster in the previous frame that corresponds to a different user. 
%  Formally, $\mathrm{MOTE} = \frac{\sum_{j=1}^J\sum_{i=1}^N e_{i,j}}{J \cdot N}$ and $\mathrm{RoM} = \frac{\sum_{j=1}^J \mathrm{NoM}_j}{J \cdot N}$, where $e_{i,j}$ is the distance between the estimated position and the true position of the $i^\text{th}$ user in the $j^\text{th}$ frame, $\mathrm{NoM}_j$ is the number of mismatches in the $j^\text{th}$ frame.


%Since there is no proper metric to directly evaluate the quality of generated feature vectors, we use the performance of the  tracking instead. Specifically, we introduce two metrics. 
%The first one is multiple object tracking precision (MOTP), which is the total error between estimated positions and real positions over all frames, averaged by the total number of objects over all frames. It mainly focuses on the precision of object tracking. %and is independent of the capability of sensors at recognizing object. 
%The second metric is ratio of mismatches (mme) over all frames. 
%The mismatch refers to the case that a cluster is associated with a previous cluster belongs to a different person.
% The definition of these two metrics are:
%$$
%\mathrm{MOTP}=\frac{\sum_{i, t} d_{t}^{i}}{\sum_{t} N_{t}},\hspace{1cm} \overline{m m e}=\frac{\sum_{t} \#\text{mme}_{t}}{\sum_{t} N_{t}}
%$$
%Here, $N_{t}$ is the number of objects at time $t$. $d_{t}^{i}$ is the distance between the estimated position and the real position for object $i$ at time $t$.

  

%The mean errors are eror stays under 40cm while the interpersonal distance increases, which guarantees the system's performance in a large range.
\begin{figure}
  \begin{minipage}{0.21\textwidth}
  \centering
  \includegraphics[width=\linewidth]{figures/down_stream_task}
  \vspace{-2em}
  \caption{Multi-object tracking error of inter-frame cluster tracking.}
  \label{fig:motp}
\end{minipage}%
\hfill
\begin{minipage}{0.23\textwidth}
  \centering
  \includegraphics[width=\linewidth]{figures/mismatch}
  \vspace{-2em}
  \caption{Ratio of mismatches during inter-frame cluster tracking.}
  \label{fig:mismatch}
\end{minipage}%
\vspace{-1em}
\end{figure}


% \noindent \textbf{Result.}

The results in Figs.~\ref{fig:motp} and \ref{fig:mismatch} show that: (1) mmClusterNet outperforms the off-the-shelf PointNet in achieving inter-cluster tracking; (2) BBR is an appropriate downstream task for training mmClusterNet.
%Fig.~\ref{fig:motp} and \ref{fig:mismatch} shows the MOTE and RoM of mmClusterNet and PointNet trained with various downstream tasks. The mmClusterNet achieves the best performance and also then in terms of the two metrics when it is trained with BBR, which is chosen as the downstream task in ImmTrack. 
BBR enforces the model to simultaneously capture cluster contour and enforces utilization of the velocity information of the shape-motion feature. Thus, BBR helps mmClusterNet better learn the shape-motion feature. On the contrary, NBBR leads to poor tracking performance. A possible reason is that NBBR overstretches the utilization of velocity information. ImmTrack evaluated in other sections adopts the mmClusterNet trained with BBR.

%Different from OC that focuses on cluster shape and PCC that focuses on the spatial relations among the points of a cluster, BBR simultaneously captures cluster contour and enforces utilization of the velocity information of the shape-motion feature. Thus, BBR helps mmClusterNet better learn the shape-motion feature.
%oint cloud completion and object classification, the bounding box regression finds a good balance between the precise position of each point and shape of the whole point cloud. 
%We also observe that NBBR leads to poor tracking performance. A possible reason is that NBBR overstretches the utilization of velocity information.
%that forcing the neural network to automatically learn the bounding box position in the next frame can result in poor tracking performance.
%More complex model structure such as deep affinity network (DAN) can be deployed to directly learn the inter-frame cluster association in the future.
%Compared with the mmClusterNet trained with BBR, the PointNet trained with either OC or PCC achieves inferior tracking performance. This is because the shape-motion feature extracted by mmClusterNet provides more information than the shape feature extracted by PointNet.



% Moreover, directly applying the PointNet trained under ShapeNet does not work well, although PointNet is much larger than our proposed PointTrack Net and ShapeNet is also much larger than our collected dataset. This is because radar point cloud is noisy and sparse, which is significantly different from fine-grained one in ShapeNet.


%%% Local Variables:
%%% mode: latex
%%% TeX-master: "main"
%%% End:
% !TEX root = main.tex
\subsection{Compute and Communication Overheads}
\label{subsec:time}




\begin{figure}
  \centering
  % \begin{minipage}{.3\linewidth}
  %   \centering 
  %   \includegraphics[width=\textwidth]{figures/ble_contact_track_spatial}
  %   \vspace{-1em}
  %   \caption[ ]{Bluetooth neighbor discovery delay versus pairwise distance when there are two or five users.}
  %   \label{fig:ble_temp_res}
  % \end{minipage}

  \begin{minipage}{.45\linewidth}%  
    \includegraphics[width=\linewidth]{figures/process_time}%
    \vspace{-1em}
    \caption{Runtime latency of ImmTrack and ICTrack with different hardwares.}
    \label{fig:timeFinal}%
  \end{minipage}
  \hfill
  \begin{minipage}{.52\linewidth}  
  \centering
    \includegraphics[width=.865\linewidth]{figures/process_time_stack}%
    \vspace{-1.2em}
    \caption{Time for trace map generation (Tra-Gen) and cross-modality association (Assoc).}
    \label{fig:timeDivide}%
  \end{minipage}%
  \vspace{-1em}
\end{figure}

\subsubsection{Server computation overhead}
%In Table.~\ref{tab:bledelay}, we measure the latency for google exposure notification, which is based on Bluetooth. As shown in the table, the bluetooth based method does not satisfy the temporal resolution requirement.
Fig.~\ref{fig:timeFinal} shows the runtime latency of ImmTrack and ICTrack on the server under different $N$. In general, ImmTrack runnning on an Intel i7-11800H CPU can achieve 30 to 60 fps, depending on the number of users. Note that our ImmTrack implementation adopts a radar sampling rate of 8 fps. Thus, a CPU-only cloud server can support several ImmTrack tasks for different venues, or a CPU-only {\em in situ} edge server can support a single ImmTrack instance. ICTrack on the same i7-11800H CPU can only achieve about 15 fps processing throughput. Even with a GeForce RTX-3060 or RTX-6000 GPU, ICTrack's processing throughput is still lower than ImmTrack's, because the image processing imposes higher computation overhead than point cloud processing. By jointly considering the accuracy results obtained in \sect\ref{system_performance}, compared with ICTrack, ImmTrack achieves similar accuracy but only requires 1/4 to 1/2 processing power.
Fig.~\ref{fig:timeDivide} shows the breakdown of the time for processing 90 frames to generate trace map and perform cross-modality association, where generating trace map from radar and camera data takes most of the time.

%in average, the ImmTrack system can generate the association result of 90 frames of data using a CPU within 6.5 seconds. On the contrary, iCTrack requires an advanced GPU to reach the similar throughput. Otherwise, when working on the same CPU, iCTrack consumes over 17 seconds. 
%\todo{Claim that the absolute throughput can meet the real-world need for some applications.}

%Compared to iCTrack, ImmTrack is more suitable for the edge devices for its shorter trajectory generation time as the processing of the sparse point clouds is less complicated than that of dense images.

% In Figure \ref{fig:timeDivide}, we provide the module-wise running time of ImmTrack and iCTrack. The trajectory association time is short (within 4.2 seconds), which means that it can be easily deployed to a wide range of platforms.
%For both systems, the trajectory generation time accounts for most of the total time, while the trajectory association is close and  relatively quick, using less than 2.2 seconds under the case of seven people.  It also shows that our association algorithm\ref{oneshot_id_ass} is robust when using different sensors for tracking.





% \begin{table}
%   \centering  
%   \label{tab:power-compare}    
%   \begin{tabular}{|c|c|}
%     \hline
%     App           & mAh    \\ \hline
%     ImmTrack      & 36.06  \\ \hline
%     Coronalert    & 37.37  \\ \hline
%     TraceTogether & 55.62  \\ \hline
%     LeaveHomeSafe & 157.04 \\ \hline
%   \end{tabular}
%   \vspace{4em}
%   \captionof{table}{Energy usages of contact tracing apps over 8 hours.}
%   % \begin{tabular}{ccccc}
%   %   \hline
%       %       App & ImmTrack & {\small Coronalert} & {\small TraceTogether} & {\small LeaveHomeSafe} \\
%       %       \hline
%       %       mAh & 36.06 & 37.37 & 55.62 & 157.04 \\
%       %       \hline
%       %     \end{tabular}  
% \end{table}

\subsubsection{Smartphone communication and energy overheads}

We deploy both the IMU sampling and trace map generation modules on an Android smartphone and measure the overheads.
ImmTrack uploads the velocity magnitude to the server for the mmWave-IMU pre-matching. At the end of each association time window, ImmTrack uploads the trace map to the server, which is about $30\,\text{KB}$. The mmUniverSense uploads the 3D velocity continuously.
%Here we compare the average communication overhead between the phone and severs for the two systems in a window(12 seconds) by letting them run in the backend for 12 hours.
Our measurements show that ImmTrack's and mmUniverSense's bit rates are $7.36\,\text{kbps}$ and $15.63\,\text{kbps}$, respectively. ImmTrack's bit rate is lower than the $8\,\text{kbps}$ of G.729, an ITU's voice codec for bandwidth-constrained scenarios.

We also compare the battery energy usages of ImmTrack and three existing contact tracing mobile apps, i.e., TraceTogether, LeaveHomeSafe, Coronalert. We run these  apps in the background on an Android smartphone for eight hours. We factory-reset the smartphone before each benchmark. ImmTrack keeps sampling IMU, computing trace maps, and uploading data. From publicly available information, Coronalert (which is based on Google/Apple Exposure Notification system) and TraceTogether exchange Bluetooth messages with nearby devices; LeaveHomeSafe is a passive tracing tool based on QR code scanning. During each 8-hour benchmark, we use the tested app to scan valid QR codes every hour to mimic normal daily usages. 
%Table~\ref{tab:power-compare} shows the apps' battery energy usages reported by the Android OS. 
According to our measurements,  battery energy usages of TraceTogether, LeaveHomeSafe, Coronalert are 55.62, 157.04, 37.37 mAh, respectively, while ImmTrack consumes 36.05 mAh.
Thus, ImmTrack imposes similar/lower battery energy overhead compared with the existing contact tracing apps.

%\yimin{Compared to another low-power wearable based contact tracing system TraceBand\cite{traceband}, though ImmTrack consumes slightly more power(36.06 mAh vs 12.8 mAh), ImmTrack equips with more application scenarios.}



% \begin{table}[]
%   \centering
%   \caption{Energy usages of ImmTrack and contact tracing mobile apps over 8 hours.}
%   \label{tab:power-compare}
%   \begin{tabular}{lllll}
%   \hline
%   App                                                    & \textit{\begin{tabular}[c]{@{}l@{}}Imm\\ Track\end{tabular}} & \multicolumn{1}{c}{\textit{\begin{tabular}[c]{@{}c@{}}Coron\\  alert\end{tabular}}} & \multicolumn{1}{c}{\textit{\begin{tabular}[c]{@{}c@{}}Trace\\ Together\end{tabular}}} & \multicolumn{1}{c}{\textit{\begin{tabular}[c]{@{}c@{}}Leave\\ HomeSafe\end{tabular}}} \\ \hline
%   \begin{tabular}[c]{@{}l@{}}Energy\\ (mAh)\end{tabular} & 36.06                                                        & 37.37                                                                               & 55.62                                                                                 & 157.04                                                                                \\ \hline
%   \end{tabular}
% \end{table}



%The computaion overhead refers to the battery assumotion of the system. Here we compare the computaion overhead of ImmTrack system on the phone with the COVID tracing applications, inclduing LeaveHomeSafe in Hong Kong, TraceTogether in Singapore, Coronalert in Belgium. The LeaveHomeSafe is a passive tracing application based on QR code scan and the remaining two are based on Bluetooth. While the TraceTogether uses the phone's default Bluetooth configuration, the Coronalert uses the Google Exposure Notification Service.  We let all the 4 applications running in the backend for 8 hours and open them every hour to calculate the battery usage. 

%In table\ref{tab:power_compare}, we summarize ther power consumption. For the ImmTrack system and the Bluetooth based systems, the power consumption is below 2\% for the modern mobile phone, which will not be a big concern.

% \begin{table}[]
%   \caption{Comparison of  the power consumption of the contact tracking systems}
%   \label{tab:power_compare}
%   \begin{tabular}{|c|c|}
%   \hline
%   \textbf{}                      & \begin{tabular}[c]{@{}c@{}}Power \\ Consumption\\ (mAh)\end{tabular} \\ \hline
%   \multicolumn{1}{|l|}{ImmTrack} & 36.06                                                                \\ \hline
%   LeaveHomeSafe                  & 157.04                                                               \\ \hline
%   TraceTogether                  & 55.62                                                                \\ \hline
%   Coronalert                     & 37.37                                                                \\ \hline
%   \end{tabular}
% \end{table}

% \begin{table}
%   \caption{Energy usages of contact tracing apps over 8 hours.}
%   \label{tab:power-compare}
%   \vspace{-1em}
%   \begin{tabular}{ccccc}
%     \hline
%     App & ImmTrack & {\small Coronalert} & {\small TraceTogether} & {\small LeaveHomeSafe} \\
%     \hline
%     mAh & 36.06 & 37.37 & 55.62 & 157.04 \\
%     \hline
%   \end{tabular}
% \end{table}







%\section{Discussions}
\label{sec:discuss}

% Another application is the augmented reality (AR) based social assistance that can facilitate the interaction between people, as shown in Figure \ref{fig:1}. Imagine a social scenario where participants are walking around to meet others, and each person is with a wearable device such as a smart glasses that can automatically prompt the information about the person in front of him (her). 
% % In this case, the cross-modality collaboration is also required. 
% Specifically, to prompt such a message, the wearable device needs a global ID map, which contains the information of neighboring people and its own position. An edge node connected with an ambient sensor is responsibility for providing such a global ID map by associating the objects in the global view and the IDs of wearable devices.

how to compare with tag-based solution like \cite{soltanaghaei2021millimetro}?

Besides interpersonal distance tracking, the re-identified radar sensing results can also enable other applications. For instance, 

Apart from social tracking, ImmTrack is also applicable to other applications like AR (augmented reality) or VR (virtual reality)-based social assistance. For instance, in a social scenario where participants walk around to meet others, smart glasses could automatically remind the user the information about the person in front of him (her) with the technology from Immtrack,

%Moreover, although Immtrack is robust to occurance of passengers (i.e., people who are in the FOV of the radar while do not enable IMU), it still may fall short when there are roamers (i.e., people who are contributing the IMU data while not in the radar's FOV). This will be left as our future work.

%%% Local Variables:
%%% mode: latex
%%% TeX-master: "main"
%%% End:
% !TEX root = main.tex
\section{Conclusion}
\label{sec:conclude}

This paper presents ImmTrack, an interpersonal distance tracking system using one or more low-cost mmWave radar(s) and the IMUs of the users' smartphones. By associating the users' trajectories reconstructed from the mmWave radar and IMU sensing in terms of the trajectory features extracted by a Siamese neural network,
%ImmTrack features a novel IMU-assisted mmWave radar tracking algorithm for generating trajectories of each user in the global coordinate, and a learning-based association module to match the radar sensing results with per-user trajectories from IMUs. Based on the association results,
ImmTrack transfers the users' pseudo identities tagged to the IMU data to the radar's global-view sensing results.
Extensive experiments with up to 27 people show that ImmTrack achieves similar tracking accuracy and lower computation overhead compared with the more privacy-intrusive camera surveillance. ImmTrack achieves decimeters-seconds spatio-temporal accuracy in tracing contacts, outperforming the prevailing Bluetooth neighbor discovery approach that suffers inaccurate distance estimation and up to 80 seconds discovery delays in our experiments.

% In this case, the cross-modality collaboration is also required. 
% Specifically, to prompt such a message, the wearable device needs a global ID map, which contains the information of neighboring people and its own position. An edge node connected with an ambient sensor is responsibility for providing such a global ID map by associating the objects in the global view and the IDs of wearable devices.



% Experiments on the public and a new real-world multimodal HAR dataset show that, Cosmo achieves over 30\% accuracy improvement to state-of-art baseline approaches, and can converge much faster than conventional supervised fusion learning.
% In the first stage, we design a fusion-based contrastive learning framework that trains the unimodal feature encoders to learn consistent information from unlabeled multimodal data. In the second stage, we design a quality-guided attention fusion module to capture complementary information of different modalities based only on limited labeled data, and combine it with consistent information learned by contrastive learning through an iterative fusion learning strategy.



%%% Local Variables:
%%% mode: latex
%%% TeX-master: "main"
%%% End:

% That's all folks!

\begin{acks}
  This research is supported in part by the Ministry of Education, Singapore, under its Academic Research Fund Tier 1 (RG88/22), in part by the Innovation and Technology Commission of Hong Kong under Grant No.~GHP/126/19SZ, and in part by the Research Grants Council (RGC) of Hong Kong under Grant No.~14209619.
\end{acks}

\ignore{
\newpage
\LARGE \noindent \textbf{APPENDIX} \normalsize
\section{Appendix for Proofs}

\paragraph{Proof of Theorem \ref{thm:main}.}

\begin{proof}
\label{proof:main}
Our proof has two steps. In Step 1, we will show that SimCLR is equivalent to minimizing the cross entropy loss defined in Eqn.~(\ref{eqn:cross-entropy}). 
In Step 2, we will show  that minimizing the cross-entropy loss 
is equivalent to spectral clustering on $\bfpi$. 
Combining the two steps together, we have proved our theorem. 

\textbf{Step 1: } SimCLR is equivalent to minimizing the cross entropy loss.

The cross-entropy loss takes expectation over 
$\bfW_\bfX\sim \mathbb{P}(\cdot ; \bfpi)$, 
which means $\bfW_\bfX$ has exactly one non-zero entry in each row $i$. By Lemma~\ref{lem:multinomial}, we know every row $i$ of $\bfW_\bfX$ is independent of other rows. Moreover, 
$\bfW_{\bfX,i}\sim \mathcal{M}(1, \bfpi_i/\sum_j \bfpi_{i,j})=\mathcal{M}(1, \bfpi_i)$, because $\bfpi_i$ itself is a probability distribution.
Similarly, we know $\bfW_\bfZ$ also has the row-independent property by sampling over $\mathbb{P}(\cdot;\bfK_\bfZ)$.
Therefore, by Lemma~\ref{lem:cross_split}, we know Eqn.~(\ref{eqn:cross-entropy}) is equivalent to:
\[
 -\sum_{i=1}^n \mathbb{E}_{\bfW_{\bfX,i}}[\log \mathbb{P}(\bfW_{\bfZ,i}=\bfW_{\bfX,i};\bfK_\bfZ)],
\]

This expression takes expectation over $\bfW_{\bfX,i}$ for the given row $i$. Notice that 
$\bfW_{\bfX,i}$ has exactly one non-zero entry, which equals $1$ (same for $\bfW_{\bfZ,i}$). 
As a result
we expand the above expression to be:
\begin{equation}
 -\sum_{i=1}^n \sum_{j\neq i} \Pr(\bfW_{\bfX,i,j}=1)\log \Pr(\bfW_{\bfZ,i,j}=1).
\label{eqn:detailed-expansion}    
\end{equation}


By Lemma~\ref{lem:multinomial}, $\Pr(\bfW_{\bfZ,i,j}=1)=\bfK_{\bfZ,i,j}/\|\bfK_{\bfZ,i}\|_1$ for $j\neq i$. Recall that $\bfK_\bfZ=(k(\bfZ_i-\bfZ_j))_{(i,j)\in[n]^2}$, which means 
$\bfK_{\bfZ,i,j}/\|\bfK_{\bfZ,i}\|_1=\frac{\exp(-\|\bfZ_i-\bfZ_j\|^2/{2\tau})}{\sum_{k\neq i}
\exp(-\|\bfZ_i-\bfZ_k\|^2/{2\tau})
}$ for $j\neq i$, when $k$ is the Gaussian kernel with variance $\tau$. 

Notice that $\bfZ_i=f(\bfX_i)$, so we know
\begin{equation}
-\log \Pr(\bfW_{\bfZ,i,j}=1)=
-\log \frac{\exp(-\|f(\bfX_i)-f(\bfX_j)\|^2/{2\tau})}{\sum_{k\neq i}
\exp(-\|f(\bfX_i)-f(\bfX_k)\|^2/{2\tau}),
}
\label{eqn:infonce-equivalence}    
\end{equation}


The right hand side is exactly the InfoNCE loss defined in Eqn.~(\ref{eqn:infonce}).
Inserting Eqn.~(\ref{eqn:infonce-equivalence}) into Eqn.~(\ref{eqn:detailed-expansion}), we get the SimCLR algorithm, which first samples augmentation pairs $(i,j)$ with $\Pr(\bfW_{\bfX,i,j}=1)$ for each row $i$, and then optimize the InfoNCE loss. 

\textbf{Step 2: } minimizing the cross entropy loss 
is equivalent to spectral clustering on $\bfpi$.


By Lemma~\ref{lem:convert_to_spectral}, we may further convert the loss to 
\begin{equation}
\label{eqn:main-theorem-repul-attr}
\min_{\bfZ}
-\sum_{(i,j)\in [n]^2} \mathbf{P}_{i,j}
\log k (\bfZ_i-\bfZ_j)+\log \mathbf{R}(\bfZ).
\end{equation}
Since $k$ is the Gaussian kernel, this reduces to \[
\min_\bfZ \mathrm{tr}(\bfZ^\top \mathbf{L}(\bfpi) \bfZ)
+\log \mathbf{R}(\bfZ),
\]

where we use the fact that $\mathbb{E}_{\bfW_\bfX\sim \mathbb{P}(\cdot; \bfpi)}[\mathbf{L}(\bfW_\bfX)]
=\mathbf{L}(\bfpi)
$, because the Laplacian operator is linear and $
\mathbb{E}_{\bfW_\bfX\sim \mathbb{P}(\cdot; \bfpi)}(\bfW_\bfX)=\bfpi
$.
\end{proof}

\paragraph{Proof of Theorem \ref{thm:clip}.}
\begin{proof}
Since $\bfW_\bfX\sim \mathbb{P}(\cdot;\bfpi_{\mathbf{A}, \mathbf{B}})$, we know 
$\bfW_\bfX$ has exactly one non-zero entry in each row, denoting the pair that got sampled. 
A notable difference compared to the previous proof is we now have $n_\mathcal{A}+n_\mathcal{B}$ objects in our graph. CLIP deals with this by taking a mini-batch of size $2N$, 
such that $n_\mathcal{A}=n_\mathcal{B}=N$, and adding the $2N$ InfoNCE losses together. We label the objects in $\mathcal{A}$ as $[n_\mathcal{A}]$, and the objects in $\mathcal{B}$ as $\{n_\mathcal{A}+1, \cdots, n_\mathcal{A}+n_\mathcal{B}\}$. 

Notice that $\bfpi_{\mathbf{A}, \mathbf{B}}$ is a bipartite graph, so the edges of objects in $\mathcal{A}$ will only connect to object in $\mathcal{B}$ and vice versa. We can define the similarity matrix in $\cZ$ as $\bfK_\bfZ$, 
where $\bfK_\bfZ(i, j+n_\mathcal{A})=\bfK_\bfZ(j+n_\mathcal{A},i)= k(\bfZ_i-\bfZ_j)$ for $i\in [n_\mathcal{A}], j\in [n_\mathcal{B}]$, and otherwise we set $\bfK_\bfZ(i,j)=0$. 
The rest is same as the previous proof. 
\end{proof}

\paragraph{Proof of Theorem \ref{thm:exponential}.}

\begin{proof}
\label{proof:exponential}
Since the objective function consists of a linear term combined with an entropy regularization, which is a strongly concave function, the maximization problem is a convex optimization problem. Owing to the implicit constraints provided by the entropy function, the problem is equivalent to having only the equality constraint. We then introduce the Lagrangian multiplier $\lambda$ and obtain the following relaxed problem:

$$
\widetilde{E}(\boldsymbol{\alpha})=\psi_{1}-\sum_{i=1}^n \alpha_{i} \psi_{i}+\tau \sum_{i=1}^n \alpha_{i}\log \alpha_{i}+\lambda\left(\boldsymbol{\alpha}^{\top} \mathbf{1}_n-1\right).
$$

As the relaxed problem is unconstrained, taking the derivative with respect to $\alpha_{i}$ yields

$$
\frac{\partial \widetilde{E}(\boldsymbol{\alpha})}{\partial \alpha_{i}}=-\psi_{i}+\tau\left(\log \alpha_{i}+\alpha_{i} \frac{1}{\alpha_{i}}\right)+\lambda=0.
$$

Solving the above equation implies that $\alpha_{i}$ takes the form
$
\alpha_{i}=\exp \left(\frac{1}{\tau} \psi_{i}\right) \exp \left(\frac{-\lambda}{\tau}-1\right).
$ Since $\alpha_{i}$ lies on the probability simplex, the optimal $\alpha_{i}$ is explicitly given by
$
\alpha^{*}_{i}=\frac{\exp \left(\frac{1}{\tau} \psi_{i}\right)}{\sum_{i^{\prime}=1}^n \exp \left(\frac{1}{\tau} \psi_{i^{\prime}}\right)} .
$ Substituting the optimal point into the objective function, we obtain
$$
\begin{aligned}
E\left(\boldsymbol{\alpha}^*\right)  &=\psi_1-\sum_{i=1}^n \frac{\exp \left(\frac{1}{\tau} \psi_{i}\right)}{\sum_{i^{\prime}=1}^n \exp \left(\frac{1}{\tau} \psi_{i^{\prime}}\right)} \psi_{i}+\tau \sum_{i=1}^n \frac{\exp \left(\frac{1}{\tau} \psi_{i}\right)}{\sum_{i^{\prime}=1}^n \exp \left(\frac{1}{\tau} \psi_{i^{\prime}}\right)}\log \frac{\exp \left(\frac{1}{\tau} \psi_{i}\right)}{\sum_{i^{\prime}=1}^n \exp \left(\frac{1}{\tau} \psi_{i^{\prime}}\right)} \\
& =\psi_1 - \tau \log \left(\sum_{i=1}^n \exp \left(\frac{1}{\tau} \psi_{i}\right)\right).
\end{aligned}
$$
Thus, the Lagrangian dual function is given by
\begin{equation*}
-E\left(\boldsymbol{\alpha}^*\right)= -\tau \log \frac{\exp \left(\frac{1}{\tau} \psi_{1}\right)}{\sum_{i=1}^n \exp \left(\frac{1}{\tau} \psi_{i}\right)}.\qedhere
\end{equation*}
\end{proof}



\section{More on Experiments} \label{section: experiment_details}

\paragraph{CIFAR-10 and CIFAR-100} CIFAR-10 ~\citep{krizhevsky2009learning} and CIFAR-100 ~\citep{krizhevsky2009learning} are well-known classic image classification datasets. Both CIFAR-10 and CIFAR-100 contain a total of 60k $32 \times 32$ labeled images of different classes, with 50k for training and 10k for testing. CIFAR-10 is similar to CIFAR-100, except there are 10 different classes in CIFAR-10 and 100 classes in CIFAR-100.

\paragraph{TinyImageNet} TinyImageNet ~\citep{le2015tiny} is a subset of ImageNet ~\citep{deng2009imagenet}. There are 200 different object classes in TinyImageNet, with 500 training images, 50 validation images, and 50 test images for each class. All the images in TinyImageNet are colored and labeled with a size of $64 \times 64$.

\textbf{Pseudo-code.} Algorithm \ref{alg:Training Procedure} presents the pseudo-code for our empirical training procedure.

\begin{algorithm}[!htbp]
\caption{Training Procedure}
\label{alg:Training Procedure}
\begin{algorithmic}[1]
\REQUIRE trainable encoder network $f$, batch size $N$, augmentation strategy \textit{aug}, loss function $L$ with hyperparameters \textit{args}
\FOR {sampled minibatch ${x_i}_{i=1}^N$}
\FORALL{$i \in { 1, ..., N }$}
\STATE draw two augmentations $t_i = \textit{aug}\left(x_i\right) $, $t_i' = \textit{aug}\left(x_i\right) $
\STATE $z_i = f\left(t_i\right)$, $z_i' = f\left(t_i'\right)$
\ENDFOR
\STATE compute loss $\mathcal{L} = L(N, z, z', \textit{args})$
\STATE update encoder network $f$ to minimize $\mathcal{L}$
\ENDFOR
\STATE \textbf{Return} encoder network $f$
\end{algorithmic}
\end{algorithm}

We also provide the pseudo-code for our core loss function used in the training procedure in Algorithm \ref{alg:Core loss}. The pseudo-code is almost identical to SimCLR's loss function, with the exception of an extra parameter $\gamma$.

\begin{algorithm}[!htbp]
\caption{Core loss function $\mathcal{C}$}
\label{alg:Core loss}
\begin{algorithmic}[1]
\REQUIRE batch size $N$, two encoded minibatches $z_1, z_2$, $\gamma$, temperature $\tau$
\STATE $z = \textit{concat}\left(z_1, z_2\right)$
\FOR {$i \in {1, ..., 2N }, j \in {1, ..., 2N}$ }
\STATE $s_{i,j} = \Vert z_i - z_j \Vert_2^{\gamma}$
\ENDFOR
\STATE \textbf{define} $l(i, j)$ \textbf{as} $l(i, j) = - \log \frac{exp\left(s_{i,j}/\tau \right)}{\sum_{k=1}^{2N} \mathbf{1}{[k \ne i]} exp\left(s{i, j} / \tau \right)} $
\STATE \textbf{Return} $\frac{1}{2N} \sum_{k=1}^N\left[l(i, i+N) + l(i+N, i)\right]$
\end{algorithmic}
\end{algorithm}

Utilizing the core loss function $\mathcal{C}$, we can define all kernel loss functions used in our experiments in Table \ref{table: loss definition}. For all $z_i \in z$ with even dimensions $n$, we define $z_{L_i} = z_i\left[0:n/2\right]$ and $z_{R_i} = z_i\left[n/2:n\right]$.

\begin{table}[ht]
\centering
\begin{tabular}{{@{}l|l@{}}}
Kernel  &  Loss function \\ \midrule
Laplacian & $\mathcal{C}\left(N, z, z', \gamma=1, \tau\right)$\\ \midrule
Sum       & $\lambda * \mathcal{C}\left(N, z, z', \gamma=1, \tau_1\right) + (1-\lambda) * \mathcal{C}\left(N, z, z', \gamma=2, \tau_2\right)$  \\ \midrule
Concatenation Sum&$\lambda * \mathcal{C}\left(N, z_L, z'_L, \gamma=1, \tau_1\right) + (1-\lambda) * \mathcal{C}\left(N, z_R, z'_R, \gamma=2, \tau_2\right)$\\ \midrule
$\gamma = 0.5$ & $\mathcal{C}\left(N, z, z', \gamma=0.5, \tau\right)$          \\ 

\end{tabular}

\caption{Definition of kernel loss functions in our experiments}
\label {table: loss definition}
\end{table}

\textbf{Baselines.} We reproduce the SimCLR algorithm using PyTorch Lightning~\citep{PytorchLightning}.

\textbf{Encoder details.}
The encoder $f$ consists of a backbone network and a projection network. We employ ResNet50~\citep{ResNet} as the backbone and a 2-layer MLP (connected by a batch normalization~\citep{ioffe2015batch} layer and a ReLU \cite{nair2010rectified} layer) with hidden dimensions 2048 and output dimensions 128 (or 256 in the concatenation kernel case).

\textbf{Encoder hyperparameter tuning.}
For each encoder training case, we randomly sample 500 hyperparameter groups (sample details are shown in Table \ref{table: Hyperparameter sample}) and train these samples simultaneously using Ray Tune ~\citep{RayTune}, with the ASHA scheduler~\citep{li2018massively}. Ultimately, the hyperparameter group that maximizes the online validation accuracy (integrated in PyTorch Lightning) within 5000 validation steps is chosen for the given encoder training case.

\begin{table}[ht]
\centering

\begin{tabular}{@{}l|l|l@{}}
\midrule
Hyperparameter  & Sample Range & Sample Strategy \\ \midrule
start learning rate & $\left[10^{-2}, 10\right]$ & log uniform \\ \midrule
$\lambda$       & $\left[0, 1\right]$ & uniform \\ \midrule
$\tau$, $\tau_1$, $\tau_2$ & $\left[0, 1\right]$ & log uniform \\ \midrule
\end{tabular}

\caption{Hyperparameters sample strategy}
\label {table: Hyperparameter sample}
\end{table}

\textbf{Encoder training.} 
We train each encoder using the LARS optimizer~\citep{LARSOptimizer}, LambdaLR Scheduler in PyTorch, momentum 0.9, weight decay $10^{-6}$, batch size 256, and the aforementioned hyperparameters for 400 epochs on a single A-100 GPU.

\textbf{Image transformation.} The image transformation strategy, including augmentation, is identical to the default transformation strategy provided by PyTorch Lightning.

\textbf{Linear evaluation.}
The linear head is trained using the SGD optimizer with a cosine learning rate scheduler, batch size 64, and weight decay $10^{-6}$ for 100 epochs. The learning rate starts at $0.3$ and ends at $0$.

\textbf{Moco Experiments.} We also tested our method based on MoCo~\citep{he2019moco}. The results are summarized in Table \ref{tab:results-moco}. Here we choose ResNet18~\citep{ResNet} as the backbone and set a temperature of $0.1$ as default. For our simple sum kernel, we set $\lambda=0.8$. The results show that our method outperforms the original MoCo method.

\begin{table}[thb]
\centering
\caption{MoCo Experiment Results on CIFAR-10 and CIFAR-100.}
\label{tab:results-moco}
\resizebox{\textwidth}{!}{%
\begin{tabular}{@{}c|ccc|ccc@{}}
\toprule
\multirow{3}{*}{Method} & \multicolumn{3}{c|}{CIFAR-10} & \multicolumn{3}{c}{CIFAR-100} \\ \cmidrule(lr){2-4} \cmidrule(lr){5-7} 
                        & 200 epochs & 400 epochs    & 1000 epochs   & 200 epochs & 400 epochs & 1000 epochs         \\ \midrule
MoCo (repro.)         & $76.41 \pm 0.12$    & $80.01 \pm 0.15$          & $84.45 \pm 0.08$    & $\mathbf{47.02 \pm 0.11}$ & $52.50 \pm 0.07$ & $57.62 \pm 0.15$            \\
\midrule
Laplacian Kernel        & ${78.09 \pm 0.10}$    & $\mathbf{83.85 \pm 0.09}$          & $\mathbf{88.34 \pm 0.16}$    & $46.12 \pm 0.22$   & $53.44 \pm 0.17$ & $59.10 \pm 0.14$        \\
Simple Sum Kernel & $\mathbf{78.12 \pm 0.15}$   & $83.23 \pm 0.18$ & $87.50 \pm 0.20$ & $46.65 \pm 0.06$ & $\mathbf{53.62 \pm 0.19}$ & $\mathbf{59.83 \pm 0.12}$\\
\bottomrule
\end{tabular}
}
\end{table}



\section{More Experiments on Synthetic Data}


Consider a scenario with $n$ clusters, each containing $k$ vertices. Let the probability of vertices $u$ and $v$ from the same cluster belonging to $\bfpi$ be $p$. Conversely, for vertices $u$ and $v$ from different clusters, let the probability of belonging to $\pi$ be $q$. We generate the graph $\bfpi$ randomly, based on $p$ and $q$. We experiment with values of $k=100$ and $n=6$ for ease of visualization, embedding all points in a two-dimensional space. Each vertex's initial position originates from a normal distribution. In each iteration, we sample a subgraph of $\bfpi$ uniformly, ensuring each vertex has an out-degree of $1$. We then optimize the corresponding vectors using InfoNCE loss with an SGD optimizer and iterate until convergence. Our experimental setup consists of an SGD learning rate of $1$, an InfoNCE loss temperature of $0.5$, and a batch size of $50$. We evaluate two scenarios with different $p$ and $q$ values: $p=1$, $q=0$, and $p=0.75$, $q=0.2$. The results of these experiments are visualized in Figure \ref{fig:vis-spectral-cluster}. The obtained embeddings exhibit the hallmark pattern of spectral clustering of graph $\bfpi$.

\begin{figure}[!tb]
\centering
\subfigure{
\includegraphics[width=1\textwidth]{Figures/cluster_pi.png}
\label{fig:vis-cluster}
}
\subfigure{
\includegraphics[width=1\textwidth]{Figures/noised_cluster_pi.png}
\label{fig:vis-noised-cluster}
}
\caption{Visualizations of the optimization process using InfoNCE Loss on the vectors corresponding to $\bfpi$. Points of identical color belong to the same cluster within $\bfpi$. To showcase the internal structure of $\bfpi$, we randomly select 10 vertices from each cluster to display the edge distribution of $\bfpi$.}
\label{fig:vis-spectral-cluster}
\end{figure}


}

%\balance
\bibliographystyle{ACM-Reference-Format}
\bibliography{reference} 

\end{document}
