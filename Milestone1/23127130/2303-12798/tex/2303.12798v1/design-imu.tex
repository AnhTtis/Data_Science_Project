%\subsection{Local Tracking by IMU}
%\label{subsec:imu-tracking}

% \begin{figure}
%     \includegraphics[width=0.47\textwidth]{figures/imu_pipeline}
%       \caption{IMU tracking pipeline}
%     \label{fig:IMU tracking module}
% \end{figure}

% Typically, an imu outputs  body-frame accelerations, 
% angular velocities and magnetic field measurements. Compared to the wireless sensors, the IMU 
% sensor can reach a much higher sampling rate, which makes it suitable for deploying in a tracking system.
% In order to avoid extensively data collection and design a general system, our imu tracking module adopts
% the double integration approach. The high level idea is to apply double integration on the acceleration data to obtain the distance.

% However, two main challenges needed to be addressed before we utilize IMU in the tracking system.
% \begin{enumerate}
%     \item  Suppose \textit{I,W,B} denote inertial, world and body frame coordination system respectively. 
%     Generally, \textit{I} and \textit{ B} are the same but different from \textit{W}. 
%     Therefore, we need to  transfer the input or the tracking result from \textit{I} to 
%     \textit{W}.  
%     \item  The tracking result of IMU will drift over
%     time during the process of double integration of acceleration. In other word, any measurement errors are accumulated over time. 

    
    
% \end{enumerate}

% Typically, an imu collects body-frame accelerations, 
% angular rates and magnetic field measurements. Compared to the wireless sensors, the IMU 
% sensor can reach a much higher sampling rate, which makes it suitable for deploying in a tracking system.

% First of all, we denote the measurements of gyroscope and accelerometer at time t by $\vec{\Omega_t} = [\Omega_t^x,\Omega_t^y,\Omega_t^z]$ and $\vec{a_t} = [a_t^x,a_t^y,a_t^z]$ 
% Here we denote the attitude at time $t$ by $\mat{R}_t$, then $\mat{R}_{t+1}$ can be calculated as 
% $\mat{R}_{t+1} = \mat{R}_{t} \cdot \mat{\Omega^{\times}_t} $ where  

% $$\mat{\Omega^{\times}_t} =\left[\begin{array}{ccc}0 & -\Omega_{z} & \Omega_{y} \\ \Omega_{z} & 0 & -\Omega_{x} \\ -\Omega_{y} & \Omega_{x} & 0\end{array}\right]$$

% The mahony filter is designed to correct the $\mat{\Omega^{\times}_t}$ to improve the accuracy of attitude estimation. 
% It models the problem as the following optimization problem, where $g$ is the 
% gravity vector and $\vec{\hat{a}}$ is the normalized acceleration data.

% $$\mathbf{R} = \underset{\mathbf{R}\in SO(3)}{\operatorname{arg\,min}} (\lambda_1|g-\mathbf{R\cdot \vec{\hat{a}}}|^2 )$$
% m only requires two hyper-parameters, the proportional filter gain $k_p$ and 
% integral filter gain $k_i$.  Typically proportional filter gain $k_p$ determines the weight of the cumulative error and the integral filter gain $k_i$
% determines the weight of instantaneous error.
% Intuitively, the direction of the acceleration after applying the same rotation of the body should be aligned with the gravity field.
% Therefore the
% estimation error can be computed as $\vec{e_{t}} = \vec{\hat{a_{t}}} \times \vec{d} $, where $\vec{d}=-\mathbf{R_{t}}^{T} \vec{g}$. Finally, 
% it adds the correction step $\vec{\delta \Omega_{t}}=K_{p} \cdot \vec{e_{t}}+\vec{I}_{t}$ to $\vec{\Omega_{t}}$, where $\vec{I}_{t}$ is calcuated recursively
% as  $\vec{I}_{t}=\vec{I}_{t-1}+K_{i} \cdot \Delta t \cdot \vec{e_{t-1}}$
% The whole algorith


% \textbf{Utilize Mahony Filter For Transformation In Different Coordination Systems } The Mahony Filter\cite{mahony} is designed to 
% estimate  the orientation of the IMU in the world frame(\textit{ W}). Here we use $q^W$ and $q^I$ to represent the orientation in 
% \textit{ W} and \textit{ I} in the form of \textbf{quaternion}.  The algorithm first calcuates an orientation error from previous step base on the 
% acceleration data $I_{a_{t}}$ as shown in equation\ref{eq:mahonyError} where ${I}_{\hat{\mathbf{a}}_{t+1}}$ is the normalized accelerations, followed by 
%  a correction step based on a proportional-integral compensator in order to correct the measured angular velocity $I_{g_{t}}$as shown in equation\ref{eq:errorCorrect}.
% The corrected angular velocity is then used to calculate the orientation increment, which will be integrated to transfer 
% $q^I$ to $q^W$ 




% \begin{equation}
%     \label{eq:mahonyError}
%     \begin{aligned}
%     &\mathbf{v_t}=\left[\begin{array}{c}
%     2\left(q_{t-2}*q_{t-4}-q_{t-1}*q_{t-3}\right) \\
%     2\left(q_{t-1} *q_{t-2}+q_{t-3}*q_{t-4}\right) \\
%     \left(q_{t-1}^{2}-q_{t-2}^{2}-q_{t-3}^{2}+q_{t-4}^{2}\right)
%     \end{array}\right] \\
%     &\mathbf{e}_{t+1}={ }^{I} \hat{\mathbf{a}}_{t+1} \times \mathbf{v_t}\\
%     &\mathbf{e}_{i, t+1}=\mathbf{e}_{i, t}+\mathbf{e}_{t+1} \Delta t
%     \end{aligned}
% \end{equation}

% \begin{equation}
%     \label{eq:errorCorrect}
%     \begin{aligned}
%     &I_{g_{t+1}}=I_{g_{t+1}}+\mathbf{k}_{p} \mathbf{e}_{t+1}+\mathbf{k}_{i} \mathbf{e}_{i, t+1}
%     \end{aligned}
% \end{equation}
















% The IMU tracking module first apply a first order band-pass butterworth filter to the data collected by accelerometer, with a cutoff frequency at 0.02 HZ and 10 HZ to filter out the noise components. 
% Then we apply use a threshold $d=0.05$ on the accelerometer data to detect the static position (e.g the moment the upper limb moves to the lowest position or the feet touch the floor.) Then we design a mahony filter, whose goal is to estimate the orientation 
%  by fusing/combining attitude estimates by integrating gyroscope measurements 
%  and direction obtained by the accelerometer measurements. 
%  Specifically, in the mahony filter, we set the proportional filter gain $k_p = 0.5$ for static moment and $k_p = 0$ for non-static moment. 
%  When $k_p = 0$, the estimated orientation is the same with the last frame's estimation. 
%  The reason for the setting is that the orientation of the IMU should be aligned with the person when he/she is moving.  
%  However, a person may change the direction when the feet touches the floor.

%%% Local Variables:
%%% mode: latex
%%% TeX-master: "main"
%%% End: