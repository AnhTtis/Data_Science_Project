% !TEX root = main.tex
\subsection{Distance Tracking and Contact Tracing } \label{casestudy}

% \begin{figure}
%   \centering
% \begin{subfigure}{.24\textwidth}
%   \centering
%   \includegraphics[width=\linewidth]{figures/tra1}
%  \caption{Trajectories of person A generated from 2 systems}
%   \label{fig:tra1}
% \end{subfigure}%
% % \begin{subfigure}{.15\textwidth}
% %   \centering
% %   \includegraphics[width=\linewidth]{figures/tra3}
% %   \caption{Trajectories of person B generated from 2 systems}
% %   \label{fig:tra3}
% % \end{subfigure}
% \begin{subfigure}{.24\textwidth}
%     \centering
%     \includegraphics[width=\linewidth]{figures/cdf_distance}
%    \caption{difference between trajectories from 2 systems}
%     \label{fig:cdf_dis}
%   \end{subfigure}%
% \caption{Trajectory comparsion}
% \label{fig:traCompare}
% \end{figure}

\yimin{We compare the interpersonal distance tracking performance of ImmTrack with the performance of mmTrack \cite{wu2020mmtrack}. In addition, we evaluate ImmTrack's performance for contact tracing. }
%Based on the association between the clusters from mmWave radar and IMU ID, we are able to track the interpersonal distance between a specific pair of users.
% We can alsourther derive the contact time between the users.
We collect a 47-minute trace with mmWave and camera data recorded, where seven users move in the sports hall shown in Fig.~\ref{fig:different scene}. We \yimin{apply ICTrack and manually rectify ICTrack's tracking identities to generate de-anonymized groundtruth trajectories of all the users}. In addition, we project the trajectories to the world coordinate system based on the camera's setup geometry and calculate the \yimin{interpersonal distance in the global coordinate system} as the {\em reference} to evaluate the accuracy of ImmTrack's interpersonal distance tracking and contact tracing results.
%We first concatenate the data, in which 5 people participate, frame by frame and obtain data around 47 minutes for both ImmTrack and iCTrack system.
% We align the timestamp of radar and camera data frame by  frame using DTW al, and finally obtain data of 47 minutes with 5 participants.
% we combine the collected radar and camera data with the 5 participants into a single frame, which captures data in 47 minutes 
% First, we visualize the tracking trajectories from mmWave and camera of a person in a minute in Figure \ref{fig:tra1}, which illustrates that the tracked position from ImmTrack and iCTrack is highly consistent.

$\blacksquare$ {\bf Spatial accuracy of interpersonal distance tracking.} \yimin{Fig.~\ref{fig:cdf_dis} shows the CDF of ImmTrack's and mmTrack's tracking errors in centimeters with respect to the reference trajectory.}
% In Figure \ref{fig:cdf_dis}, we also quantitatively evaluate the difference of the position estimation of different people from two systems. Specifically, in each processing window, we use DTW algorithm to align the tracked location from ImmTrack and iCTrack, and calculate the absolute distance.
For ImmTrack, most tracking errors are within $50\,\text{cm}$. The average tracking error is $22\,\text{cm}$, showing that ImmTrack can achieve re-identified human tracking with decimeters spatial accuracy. \yimin{Compared with mmTrack, ImmTrack yields more stable tracking accuracy.}

For contact tracing, the tracking accuracy is important especially when the actual interpersonal distances are small. 
Fig.~\ref{fig:contact_dis_diff} shows ImmTrack's interpersonal distance tracking errors when the reference distance is in different ranges. When the reference distance is within one meter, the tracking errors are within $28\,\text{cm}$ and the mean error is $14\,\text{cm}$. The mean error remains under $40\,\text{cm}$ when the reference distance is up to $3\,\text{m}$. These results show that ImmTrack can accurately track interpersonal distances in close contacts.

\begin{figure}
% \vspace{-1em}
% \captionsetup{width=.45\linewidth}
% \caption{Inter-frame cluster tracking performance using features extracted by mmClusterNet (CNet) and PointNet (PNet) trained with OC, PCC, BBR, NBBR downstream tasks.}
% \label{fig:downstream}
\begin{minipage}{0.23\textwidth}
  \centering 
  \includegraphics[width=\textwidth]{figures/cdf_distance}
  \vspace{-2 em}
  \caption[ ]{\yimin{Statistical analysis of ImmTrack's and mmTrack's tracking errors. The highlighted part of each color represents the area covered by the CDF curve of different users in a system.}}
  \label{fig:cdf_dis}
\end{minipage}%
\hfill
\begin{minipage}{0.23\textwidth}
\centering 
\includegraphics[width=\textwidth]{figures/err_diff_distance}
\vspace{-1.6em}
\caption[ ]{ImmTrack's tracking error when reference distance is different. The horizontal line in the middle represents the average value of the error. }
\label{fig:contact_dis_diff}
\end{minipage}%
\vspace{-1em}
\end{figure}

\begin{figure*}
  \centering
  % \begin{subfigure}[t]{0.19\textwidth}
  %     \centering
  %     \includegraphics[width=\textwidth]{figures/tra1}
  %     \caption[ ]%
  %   {{ Trajectories of the same person over a minute from ImmTrack and  iCTrack}.} 
  %   \label{fig:tra1}
  % \end{subfigure}
  % \hfill
  % \begin{subfigure}[t]{0.19\textwidth}  
  %     \centering 
  %     \includegraphics[width=\textwidth]{figures/cdf_distance}
  %     \caption[ ]{CDF of ImmTrack's tracking errors with respect to reference. Each curve is result of a person.}
  %     \label{fig:cdf_dis}
  % \end{subfigure}
  % \hfill
  \begin{subfigure}[t]{0.3\textwidth}   
      \centering 
      \includegraphics[width=\textwidth]{figures/contact_time_compare}
      \caption[ ]%
    {Accumulative close contact time estimation between any two users in the 47-minute experiment $(N=5)$.}
    \label{fig:contact_time_compare}
  \end{subfigure}
  \hfill
  % \begin{subfigure}[t]{0.19\textwidth}   
  %     \centering 
  %     \includegraphics[width=\textwidth]{figures/contact_time_camera}
  %     \caption[ ]%
  %     {{  Estimation of the total contact time from iCTrack, showing (A,D) and (C,D) are \textbf{close contact pairs}}.}
  %   \label{fig:contact_timeC} 
  % \end{subfigure}
  % \hfill
  \begin{subfigure}[t]{.3\textwidth}
    \centering
    \includegraphics[width=\linewidth]{figures/contact_track_error}
    \caption[]{ImmTrack's precision and recall in pinpointing infectious contacts versus $N$ ($\tau=6\,\text{s}$).}
    \label{fig:contact_error}
  \end{subfigure}%
%  \hfill
%  \begin{subfigure}[t]{.19\textwidth}
%    \centering
%    \includegraphics[width=\linewidth]{figures/contact_track_spatial}
%    \caption[]{ImmTrack's spatial resolution in pinpointing infectious contacts}
%    \label{fig:contact_error_spatial}
%  \end{subfigure}%
  \hfill
  \begin{subfigure}[t]{.3\textwidth}
    \centering
    \includegraphics[width=\linewidth]{figures/contact_track_time}
    \caption[]{\yimin{ImmTrack's precision and recall in pinpointing infectious contacts under different contact time threshold $\tau$.}}
    \label{fig:contact_error_time}
  \end{subfigure}%
  \vspace{-1em}
  \caption[ ]
  {ImmTrack's performance on contact time estimation and pinpointing infectious contacts.}
  \label{fig:casestudy}
\end{figure*}




%We show that the discrepancy in most cases is less than 50 cm and the average value is only around 22 cm.
% for the same person tracked from mmWave and camera, 
% and show the difference of the estimation between the two sensors in Figure \ref{fig:cdf_dis}. Generally, the position estimation of ImmTrack and iCTrack is highly consistent with an average discrepancy around 22cm.
% In figure\ref{fig:cdf_dis}, it shows that the average differences are around 22cm.
% based on the above processings and the definition of \textbf{close contact},

$\blacksquare$ {\bf Contact tracing performance.}
We consider two definitions of contact: (1) By following a prevailing definition, a {\em close contact} is a contact with less than $2\,\text{m}$ interpersonal distance; (2) An {\em infectious contact} is a contact with less than $1\,\text{m}$ interpersonal distance over $\tau$ seconds or more, where we set $\tau$ from 2 to \yimin{16} seconds.
% , based on the recently reported Delta variant transmission via a fleeting encounter \cite{fleeting-transmission}.
%which is sufficient for airborne transmission of SARS-CoV-2 without mask protection \cite{chagla2021re}.
%We then find the possible \textbf{social contact} and \textbf{close contact} occurrence in the 47-minute data. Specifically, according to \cite{chagla2021re}, SARS-CoV-2 can spread through airborne transmission and infect a human without a mask within a radius of one meter in six seconds.
% Based on it, we define that a \textbf{social contact} happens when two people are standing together with a distance smaller than 1m for 6 seconds in a row. In addition, we follow the definition from WHO, and define a person is  a \textbf{close contact} to COVID positive if he/she stays within a distance smaller than 2 m for over 15 minutes with a COVID positive case. 
%Moreover, in Figure \ref{fig:contact_timeM} and Figure \ref{fig:contact_timeC}, we show the cumulative contact time for each pair of people using ImmTrack and iCTrack, respectively.
Fig.~\ref{fig:contact_time_compare} shows the accumulative close contact time for each pair of users during the 47-minute experiment. It shows that ImmTrack's result and the reference. We can see that ImmTrack gives satisfactory close contact monitoring accuracy.
%{\red \bf The maximum error of ImmTrack's result is just xx minutes.}
% Both of the systems successfully identify the \textbf{close contact} pairs (A,D) and (C,D).
Then, we evaluate ImmTrack's performance in pinpointing infectious contact.
We slide a time window of $\tau+2$ seconds with two seconds overlapping and check whether an infectious contact occurs between any two users in the window.
% group the data by the number of people and find the \textbf{social contact} frequency between each pair of participants based on ImmTrack and iCTrack. In addition, similar to the above processings,
By checking against the reference result in each time window, ImmTrack's detection result is among the true/false positive/negative.
%We define two types of errors. When an infectious contact is detected from the reference in a time window, a miss detection by ImmTrack in the same window is a {\em false negative}. When an infectious contact is not detected from the reference in a time window, a positive detection result given by ImmTrack in the same window is s referred to as an {\em extra alarm}.
We measure the {\em precision} and {\em recall} by $\text{precision}=\frac{\text{\# of true positives}}{\text{\# of all ImmTrack's positives}}$ and $\text{recall}=\frac{\text{\# of true positives}}{\text{\# of all reference's positives}}$. 
% The precision characterizes how much we can trust ImmTrack's positives and is related to the unnecessary overhead if further action is taken upon a positive; the recall characterizes ImmTrack's sensitivity to infectious contacts and is critical to preventing the spread of the contagious disease.
% . On the contrary,the occurs if the ImmTrack detects a contact while iCTrack  does not find it.
Fig.~\ref{fig:contact_error} shows the precision and recall for $\tau=6\,\text{s}$ when $N$ varies. Note that for each $N$ setting, we conduct a separate experiment that lasts for about 47 minutes. ImmTrack achieves about 90\% precision and \yimin{91\%-96\%} recall in pinpointing infectious contacts. \yimin{The opposite trend of recall and precision is due to the increase in the proportion of false negatives in all reference contacts.}

%We can see that the error rates are in general less than 0.1. The missing alarm rates, which may be more critical than the extra alarm rates, are around 5\%.
%{\red \bf (what does 5\% mean? is it small enough?)}
% We note that the missing alarm may be more fatal than the extra alarm. 
% The missing alarm rate of ImmTrack only around 0.05.

$\blacksquare$ {\bf Temporal resolution of contact tracing.} We vary the setting of $\tau$ to investigate the temporal resolution of ImmTrack in contact tracing.
% \yimin{In Fig.~\ref{fig:contact_error_spatial} shows the precision and recall in pinpointing infectious contact when we vary the definition of infectious contact distance. ImmTrack's precision and recall score do not fluctuate much, which indicates that the system can track the contact varies from small to large distance.}
Fig.~\ref{fig:contact_error_time} shows the precision and recall in pinpointing infectious contact versus the $\tau$ setting. While the recall remains stable at around 94\%, the precision increases from about 90\% to 93\% when $\tau$ is from 2 to \yimin{16} seconds. This shows that ImmTrack can achieve satisfactory temporal resolution fine to 2 seconds with a little contact detection accuracy drop. 
% The precision stays around 90\% and the recall stays around 95\%. Compared to using Bluetooth for  contact tracing, the temporal resolution is largely improved from  minute level to second level.
For comparison, we measure the BND detection delays using two or five Android phones. When using five phones, we place them at vertexes of a pentagon. Table~\ref{tab:bledelay} shows the time for a phone to discover all other phones versus the distance between the two phones or side length of the pentagon. The discovery delay increases with the distance and the number of phones. When the distance is one and three meters, the measured worst-case delay is more than 30 and 80 seconds, respectively.

\begin{table}[]
  \caption{BND detection delay (s) vs. inter-user distance (m).}
  \label{tab:bledelay}
  \vspace{-1em}
  \begin{tabular}{cccc}
  \hline
  Min inter-user distance & [0,1)          & [1,2)         & [2,3)          \\ \hline
  Two users                                                          & 3.2$\pm$2.1 & 4.8$\pm$1.7 & 6.9$\pm$3.7   \\
  Five users                                                         & 11.0$\pm$4.0 & 23.9$\pm$9.8 & 42.9$\pm$14.1 \\ \hline
  \end{tabular}
  \vspace{-1em}
  \end{table}
%\yimin{We also implement a bluetooth based neighbor discovering application for contact tracing on Android phone. In fig.~\ref{fig:ble_temp_res}, the y-axis shows the \textit{total time for a user to discover all the neighbors}. As shown, for two users in close contact, the bluetooth based method still requires 3.175 seconds to discover. In addition, it is shown that the temporal resolution of bluetooth fluctuates more severely while the inter-distance and numbe of users increasing.}

%%% Local Variables:
%%% mode: latex
%%% TeX-master: "main"
%%% End:
