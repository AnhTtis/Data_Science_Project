% !TEX root = main.tex
\section{Implementation and Evaluation}
\label{sec:eval}

% The foundation of the potential applications of our ImmTracking system  is the system's ability to associate between each cluster in mmWave point cloud and each IMU sensor. Here we first thoroughly investigate the influence of the hyper-parameters, including the \textit{proportional gain in the IMU tracking module, trajectory generation window in the  mmWave tracking module ,and heat map grid size in the association module.} on the system's performance.  We prove the system's robustness by showing that the system's performance does not fluctuate a lot while setting the first two parameters in different values. Furthermore, we find the optimal heat map bin size to boost the system's robustness.

We have implemented ImmTrack using a Texas Instrument AWR1843 mmWave radar hosted by a laptop computer. The users use their own smartphones of various models to participate in the evaluation.\footnote{Volunteers' participation is under NTU IRB protocol with reference no. IRB-2022-309.} We collect IMU data using the MATLAB Mobile app running on the users' smartphones. The sampling rates of the radar and IMU are $8\,\text{fps}$ and $100\,\text{fps}$, respectively. 
% The sampling rate of the IMU is 100hz, and that of the mmWave radar is 8.
%For both the radar and IMU,
The association time window is 12 seconds, with 2-second overlap between two consecutive windows.
%To generate the trace map, we use the sliding window that is  12 seconds long and overlaps with 2s for both the radar and IMUs.
For cross-modality association, we set $W=3$, i.e., the similarity matrices in three consecutive association windows are averaged.
% we choose to use 3 windows in the association phase.
% \todo{Deep learning models in this work are implemented using PyTorch? }
% \todo{What are the sampling rate of the IMU and radar data? The window size refers to the number of frames of which sensor?}
% \todo{What is the value of $p$ in soft voting?}
We primarily conduct experiments in an indoor sports hall and an outdoor space as shown in Fig.~\ref{fig:different scene}.
%Seven users participate in an indoor and then an outdoor experiment, each lasting three hours.
% Fig.~\ref{fig:different scene} shows the indoor and outdoor setups. 
%In \sect\ref{system_performance}, we evaluate ImmTrack's cross-modality association performance for the above setups.
We also conduct experiments in a lab space as shown in Fig.~\ref{Fig:floorPlan} with up to 27 people.
%In addition, we furthur evaluate ImmTrack in a real office with up to 27 people in Section~\ref{eval_tiny_space}. 
% under various real-world environments, aiming to answer the following two questions:
% \begin{enumerate}
%   \item How competitive is our system compared to other systems using different technologies?
%   \item How robust is our system when there are environmental dynamics?
%   % \item Is IMU a necessity, and why do we need the user to enable the IMU?
% \end{enumerate}
% \noindent In Section \ref{hyperparam}, we investigate the influence of several hyper-parameters, including the proportional gain in the IMU tracking module and the window size for trajectory generation.
% Then, in Section \ref{module_ablation}, we perform a series of comparative experiments to validate the effectiveness of each module in our system by comparing each proposed modules with its alternatives. 
% Finally, in Section \ref{time} and \ref{casestudy}, we present the experiment on system efficiency and the case study.



%%% Local Variables:
%%% mode: latex
%%% TeX-master: "main"
%%% End: