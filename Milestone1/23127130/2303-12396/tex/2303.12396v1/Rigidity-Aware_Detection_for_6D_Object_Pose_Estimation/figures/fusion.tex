\begin{figure}
    \centering
    \setlength\tabcolsep{2pt}
    \begin{tabular}{cccl}
      \includegraphics[width=0.3\linewidth, clip, trim=150 0 35 0]{figures/fusion/activate_cells.jpg} &
      \includegraphics[width=0.3\linewidth, clip, trim=150 0 35 0]{figures/fusion/cluster.jpg} &
      \includegraphics[width=0.3\linewidth]{figures/fusion/vote.png} &
      \hspace{-2mm}
      \includegraphics[width=0.05\linewidth]{figures/fusion/colormap.pdf}
    \end{tabular}
    \caption{{\bf Combining multiple candidates.} 
          {\bf (a)} During inference, multiple cells of the network will be activated thanks to the proposed mask-based sampling strategy during training. 
          Each network cell will generate one box prediction. Here we visualize the cells by different colors according to the confidence of their corresponding box predictions. Note that the cells with the most confident prediction do not come from the center region here, but from some cells with the most discriminative textures.
          {\bf (b)} With multiple detection candidates from each local cell, we cluster them according to some simple distance threshold to correspond them to different instances.
          {\bf (c)} After a simple fusion of multiple candidate results, we can obtain a single robust result.}
    \label{fig:fusion}
  \end{figure}