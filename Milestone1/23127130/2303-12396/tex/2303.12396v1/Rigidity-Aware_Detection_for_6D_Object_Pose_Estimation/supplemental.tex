% CVPR 2023 Paper Template
% based on the CVPR template provided by Ming-Ming Cheng (https://github.com/MCG-NKU/CVPR_Template)
% modified and extended by Stefan Roth (stefan.roth@NOSPAMtu-darmstadt.de)

\documentclass[10pt,twocolumn,letterpaper]{article}

%%%%%%%%% PAPER TYPE  - PLEASE UPDATE FOR FINAL VERSION
% \usepackage[review]{cvpr}      % To produce the REVIEW version
\usepackage{cvpr}              % To produce the CAMERA-READY version
% \usepackage[pagenumbers]{cvpr} % To force page numbers, e.g. for an arXiv version
% 
% Include other packages here, before hyperref.
\usepackage[table]{xcolor}
\usepackage{graphicx}
\usepackage{amsmath}
\usepackage{amssymb}
\usepackage{booktabs}
\usepackage{graphics}
\usepackage{multirow}
\usepackage{pifont}
\usepackage{comment}
\usepackage{color}
\usepackage{makecell}
\usepackage{array}

\newcommand{\cmark}{\ding{51}}%
\newcommand{\xmark}{\ding{55}}%



% It is strongly recommended to use hyperref, especially for the review version.
% hyperref with option pagebackref eases the reviewers' job.
% Please disable hyperref *only* if you encounter grave issues, e.g. with the
% file validation for the camera-ready version.
%
% If you comment hyperref and then uncomment it, you should delete
% ReviewTempalte.aux before re-running LaTeX.
% (Or just hit 'q' on the first LaTeX run, let it finish, and you
%  should be clear).
\usepackage[pagebackref,breaklinks,colorlinks]{hyperref}


% Support for easy cross-referencing
\usepackage[capitalize]{cleveref}
\crefname{section}{Sec.}{Secs.}
\Crefname{section}{Section}{Sections}
\Crefname{table}{Table}{Tables}
\crefname{table}{Tab.}{Tabs.}


%%%%%%%%% PAPER ID  - PLEASE UPDATE
\def\cvprPaperID{2298} % *** Enter the CVPR Paper ID here
\def\confName{CVPR}
\def\confYear{2023}

\begin{document}


\title{Rigidity-Aware Detection for 6D Object Pose Estimation -- Appendix}

\author{%
	{Yang Hai $^1$, \quad Rui Song $^1$, \quad Jiaojiao Li $^1$, \quad Mathieu Salzmann $^{2, 3}$, \quad Yinlin Hu $^{4}$} \\
	{\small $^1$ State Key Laboratory of ISN, Xidian University, \quad $^2$ EPFL, \quad $^3$ ClearSpace, \quad $^4$ MagicLeap} \\
    % {\tt\small yanghai1218@gmail.com, \{rsong, jjli\}@xidian.edu.cn, } \\
    % {\tt\small mathieu.salzmann@epfl.ch, yhu@magicleap.com}
}

\maketitle

\section{Appendix}
% !TEX root = ./supplemental.tex
% !TEX spellcheck = en-US

\begin{table*}
    \centering
    % \rowcolors{3}{}{gray!10}
    % \scalebox{0.9}{
    \begin{tabular}{l@{\hspace{3em}}ccc@{\hspace{3em}}ccc@{\hspace{3em}}ccc}
        \toprule
        \multirow{2}{*}{Category}  & \multicolumn{3}{c@{\hspace{3em}}}{FCOSv2~\cite{fcosv2}}                &\multicolumn{3}{c@{\hspace{3em}}}{PAA~\cite{PAA}}             &\multicolumn{3}{c}{\bf Ours}         \\
        ~                           & AP        & AP$_{50}$     & AP$_{75}$     & AP            & AP$_{50}$     & AP$_{75}$ & AP            & AP$_{50}$     & AP$_{75}$ \\
        \midrule
        Toaster                     & {20.9} &{38.4}& {30.5}          & 16.1          & 34.0          & 25.7      & {\bf 31.8}    & {\bf 47.2}    & {\bf 42.3}         \\
        Bottle                      & {38.3}      & {59.4}  & {43.1}          & 37.0          & 58.5          & 41.9        & {\bf 40.5}  & {\bf 60.7}    & {\bf 45.7}     \\
        Skateboard                  & 50.1       & {69.7} & 59.4          & {50.2}  & 69.3  & {60.2}      & {\bf 52.6} & {\bf 71.5}  & {\bf 63.8}        \\
        % bicycle                     &           &               &               & 27.3          &               &           & 29.7          &               &           \\
        Suitcase                    & 34.7      & {52.5}          & 38.8          & {35.1} & 52.0   & {39.0}       & {\bf 36.5}   & {\bf 53.6}    & {\bf 40.7}        \\ 
        Cup                         & 41.3      & 62.3  & {50.6}  & {41.5}  & {62.7}  & {50.6}      & {\bf 42.8}    & {\bf 63.2}    & {\bf 52.2}        \\
        ... & & ... & & & ... & & & ... \\
        \midrule
        Cat                         & {65.7}  & {86.8} & {68.4}          & {\bf 67.2}   & {\bf 88.9} & {\bf 69.8}     & 63.0          & 84.5          & 66.2        \\
        Dog                         & {60.7} & {81.9}   & {62.4}          & {\bf 62.1}    & {\bf 83.2}    & {\bf 63.9}       & 59.3          & 80.8         & 60.7     \\  
        Cow                         & {56.4}      & {72.3}  & {60.5}          & {\bf 58.0}    & {\bf 74.0}    & {\bf 62.8}      & 56.0           & 71.0       & 60.2   \\
        Sheep                       & {51.4} & {71.2}  & {60.7}          & {\bf 51.7}   & {\bf 71.6} & {\bf 61.2}      & 50.0          & 70.3          & 58.6         \\
        Bird                        & {\bf 36.2}& {\bf 54.8}& {\bf 42.3}    & {35.6}  & {54.0}  & 41.4     & 35.1          & 53.5          & {41.6}      \\
        ... & & ... & & & ... & & & ... \\
        \midrule
        {Avg.}                  & 38.9      & \underline{57.5}  & 42.2          & {\bf 40.4}          & {\bf 58.4}          & {\bf 43.9}      & \underline{40.0}    & 57.1  & \underline{43.4}      \\
        \bottomrule
    \end{tabular}
    % }
    \caption{{\bf Evaluation of general detection on COCO.} Although the general detection dataset COCO does not fully match our assumption of rigid targets, our method outperforms the baselines significantly on categories such as toaster, bottle, etc., which are mainly rigid objects, and our method achieves similar performance to the baselines in average accuracy.
    }
    \label{tab:coco_eval}
\end{table*}

\section{Appendix}

\noindent \textbf{General scenario.} Our work is motivated by the rigidity of the targets in 6D object pose estimation. The general scenario, e.g., COCO, does not fully match our assumption. Nevertheless, we report results with the same experimental setting as FCOSv2 and PAA in Table~\ref{tab:coco_eval}. In addition to the average accuracy across the 80 COCO categories, we report the accuracy of the 5 categories on which our method outperforms the baselines the most, and the 5 categories on which our method performs the worst. Our method outperforms the baselines significantly on categories such as toaster, bottle, etc., which are mainly rigid objects. By contrast, the categories on which our method underperforms include cat, dog, etc., which are mainly non-rigid targets and break our assumption. Our method nevertheless achieves similar performance to the baselines in average accuracy.

% \noindent \textbf{Additional qualitative results.}
% In Fig.~\ref{fig:more_vis}, we visualize additional results on the LM-O, T-LESS, IC-BIN, and YCB datasets. These results demonstrate the robustness of our method.

\noindent \textbf{Additional quantitative results.}
We show the detailed object pose results using different metrics on LM-O, T-LESS, TUD-L, IC-BIN, ITODD, HB, and YCB in Table~\ref{tab:compare_lmo},~\ref{tab:compare_tless},~\ref{tab:compare_tudl},~\ref{tab:compare_icbin},~\ref{tab:compare_itodd},~\ref{tab:compare_hb}, and~\ref{tab:compare_ycbv}, respectively.
Our method combined with PFA-Pose~\cite{pfa} outperforms the state of the art in most experimental settings.

\begin{table}[h]
    \centering
    \scalebox{0.95}{
    % \rowcolors{2}{}{gray!10}
    \begin{tabular}{lcccc}
    \toprule
    Method  &  Avg.     & MSPD      & MSSD      & VSD\\
    \midrule
    \multicolumn{5}{c}{\textit{RGB (PBR)}} \\
    \midrule
    {\bf PFA+Ours}  &{\bf 0.715}&{\bf 0.876}&{\bf 0.712}&{\bf 0.559} \\
    PFA             & 0.674     & 0.819     & 0.673     & 0.531 \\
    SurfEmb         & 0.663     & 0.851     & 0.649     & 0.497 \\
    CIR             & 0.655     & 0.831     & 0.633     & 0.501 \\
    Cosypose        & 0.633      & 0.812     & 0.606     & 0.480 \\
    CDPNv2          & 0.624     & 0.815     & 0.612     & 0.445 \\
    \midrule
    \multicolumn{5}{c}{\textit{RGB-D (PBR)}} \\
    \midrule
    {\bf PFA+Ours}  &{\bf 0.797}&{\bf 0.890}&{\bf 0.712}&{\bf 0.559} \\
    PFA             & 0.751     & 0.835     & 0.673     & 0.531 \\
    SurfEmb         & 0.760     & 0.856     & 0.649     & 0.497 \\
    CIR             & 0.734     & 0.824     & 0.633     & 0.501 \\
    Cosypose+ICP    & 0.714     & 0.826     & 0.606     & 0.480 \\
    CDPNv2+ICP      & 0.630     & 0.731     & 0.612     & 0.445 \\
    \bottomrule
\end{tabular}
    }
    \caption{{\bf Additional object pose results on LM-O.}
    % We compare our method with PFA~\cite{pfa}, SurfEmb~\cite{surfemb}, CIR~\cite{coupled_iterative}, Cosypose~\cite{cosypose}, and CDPNv2~\cite{cdpn}.
    }
    \label{tab:compare_lmo}
\end{table}

\begin{table}
    \centering
     \scalebox{0.95}{
    % \rowcolors{2}{}{gray!10}
    \begin{tabular}{lcccc}
    \toprule
    Method          & Avg.     & MSPD        & MSSD      & VSD\\
    \midrule
    \multicolumn{5}{c}{\textit{RGB (PBR)}}\\
    \midrule
    {\bf PFA+Ours}  & 0.719     & 0.832     & 0.682     & 0.643 \\
    SurfEmb         &{\bf 0.735}&{\bf 0.857}&{\bf 0.686}& {\bf 0.661} \\
    Cosypose        & 0.640      & 0.761     & 0.589     & 0.571 \\
    CDPNv2          & 0.407      & 0.579     & 0.338     & 0.303 \\
    \midrule
    \multicolumn{5}{c}{\textit{RGB (Real+PBR)}}\\
    \midrule
    {\bf PFA+Ours}  &{\bf 0.778}&{\bf 0.877}&{\bf 0.749}&{\bf 0.709} \\
    SurfEmb         &  0.770    & -         & -         & - \\
    CIR             &  0.715    & 0.798     & 0.684     & 0.663\\
    Cosypose        &  0.728    & 0.821     & 0.695     & 0.669 \\
    CDPNv2          &  0.478    & 0.620     & 0.426     & 0.386 \\
    \midrule
    \multicolumn{5}{c}{\textit{RGB-D (PBR)}}\\
    \midrule
    {\bf PFA+Ours}  & 0.801     & 0.833     & 0.807     & 0.764 \\
    SurfEmb         &{\bf 0.828}&{\bf 0.859}&{\bf 0.829}& {\bf 0.797} \\
    CDPNv2+ICP      & 0.435     & 0.488     & 0.449     & 0.368 \\
    \midrule
    \multicolumn{5}{c}{\textit{RGB-D (Real+PBR)}}\\
    \midrule
    {\bf PFA+Ours}  &{\bf 0.850}&{\bf 0.878}&{\bf 0.856}&{\bf 0.816} \\
    SurfEmb         & 0.833     & -         & -         & - \\
    CIR             & 0.776     & 0.795     & 0.773     & 0.760\\
    Cosypose+ICP    & 0.701     & 0.767     & 0.749     & 0.587 \\
    CDPNv2+ICP      & 0.464     & 0.516     & 0.489     & 0.385 \\
    \bottomrule
\end{tabular}
    }
    \caption{{\bf Additional object pose results on T-LESS.}
    % We compare our method with PFA~\cite{pfa}, SurfEmb~\cite{surfemb}, CIR~\cite{coupled_iterative}, Cosypose~\cite{cosypose}, and CDPNv2~\cite{cdpn}.
    }
    \label{tab:compare_tless}
\end{table}

\begin{table}
    \centering
     \scalebox{0.95}{
    % \rowcolors{2}{}{gray!10}
    \begin{tabular}{lcccc}
    \toprule
    Method          & Avg.     & MSPD        & MSSD      & VSD\\
    \midrule
    \multicolumn{5}{c}{\textit{RGB (PBR)}}\\
    \midrule
    {\bf PFA+Ours}  &{\bf 0.733}&{\bf 0.890}&{\bf 0.721}& {\bf 0.594} \\
    SurfEmb         & 0.715     & 0.889     & 0.687     & 0.569 \\
    Cosypose        & 0.685     & 0.847     & 0.664    & 0.544 \\
    CDPNv2          & 0.588     & 0.797     & 0.577     & 0.391 \\
    \midrule
    \multicolumn{5}{c}{\textit{RGB (Real+PBR)}}\\
    \midrule
    {\bf PFA+Ours}  &{\bf 0.839}&{\bf 0.978}&{\bf 0.820}&{\bf 0.719} \\
    SurfEmb         & 0.805     & -         & -         & - \\
    Cosypose        & 0.823     & 0.973     &0.807      &0.689 \\
    CDPNv2          & 0.772     & 0.925     & 0.793     & 0.597 \\
    \midrule
    \multicolumn{5}{c}{\textit{RGB-D (PBR)}}\\
    \midrule
    {\bf PFA+Ours}  &{\bf 0.894}&{\bf 0.929}&{\bf 0.930}& {\bf 0.821} \\
    SurfEmb         & 0.854     & 0.905     & 0.891     & 0.767 \\
    CDPNv2+ICP      & 0.791     & 0.829     & 0.847     & 0.698 \\
    \midrule
    \multicolumn{5}{c}{\textit{RGB-D (Real+PBR)}}\\
    \midrule
    {\bf PFA+Ours}  & 0.960     & 0.989     & 0.986     & 0.904 \\
    CIR             &{\bf 0.968}&{\bf 0.991}&{\bf 0.991}&{\bf 0.920}\\
    SurfEmb         & 0.933     & -         & -         & - \\
    Cosypose+ICP    & 0.939     & 0.976     & 0.972     & 0.869 \\
    CDPNv2+ICP      & 0.913     & 0.947     & 0.962     & 0.832 \\
    \bottomrule
\end{tabular}
    }
    \caption{{\bf Additional object pose results on TUD-L.}
    % We compare our method with PFA~\cite{pfa}, SurfEmb~\cite{surfemb}, CIR~\cite{coupled_iterative}, Cosypose~\cite{cosypose}, and CDPNv2~\cite{cdpn}.
    }
    \label{tab:compare_tudl}
\end{table}

\begin{table}
    \centering
     \scalebox{0.95}{
    % \rowcolors{2}{}{gray!10}
    \begin{tabular}{lcccc}
    \toprule
    Method  &  Avg.     & MSPD      & MSSD      & VSD\\
    \midrule
    \multicolumn{5}{c}{\textit{RGB (PBR)}} \\
    \midrule
    {\bf PFA+Ours}  &{\bf 0.600}&{\bf 0.689}& 0.589     &{\bf 0.521}\\
    SurfEmb         & 0.588     & 0.678     &{\bf 0.573}& 0.514 \\
    Cosypose        & 0.473     & 0.675     & 0.559     & 0.515 \\
    CDPNv2          & 0.226     & 0.582     & 0.438     & 0.399 \\
    \midrule
    \multicolumn{5}{c}{\textit{RGB-D (PBR)}} \\
    \midrule
    {\bf PFA+Ours}  &{\bf 0.676}&{\bf 0.702}&{\bf 0.692}& 0.636 \\
    SurfEmb         & 0.659     & 0.680     &0.677      & 0,621 \\
    CIR             &{\bf 0.676}& 0.683     & 0.688     &{\bf 0.656} \\
    Cosypose+ICP    & 0.647     & 0.666     & 0.652     & 0.624 \\
    CDPNv2+ICP      & 0.450     & 0.459     & 0.458     & 0.433 \\
    \bottomrule
\end{tabular}
    }
    \caption{{\bf Additional object pose results on IC-BIN.}
    % We compare our method with PFA~\cite{pfa}, SurfEmb~\cite{surfemb}, CIR~\cite{coupled_iterative}, Cosypose~\cite{cosypose}, and CDPNv2~\cite{cdpn}.
    }
    \label{tab:compare_icbin}
\end{table}

\begin{table}
    \centering
     \scalebox{0.95}{
    % \rowcolors{2}{}{gray!10}
    \begin{tabular}{lcccc}
    \toprule
    Method  &  Avg.     & MSPD      & MSSD      & VSD\\
    \midrule
    \multicolumn{5}{c}{\textit{RGB (PBR)}} \\
    \midrule
    {\bf PFA+Ours}  & 0.353     & 0.484     & 0.306     & 0.269 \\
    SurfEmb         &{\bf 0.413}&{\bf 0.552}&{\bf 0.363}&{\bf 0.324} \\
    Cosypose        & 0.216     & 0.300     & 0.177     & 0.172 \\
    CDPNv2          & 0.067     & 0.161     & 0.087     & 0.059 \\
    \midrule
    \multicolumn{5}{c}{\textit{RGB-D (PBR)}} \\
    \midrule
    {\bf PFA+Ours}  & 0.460     & 0.498     & 0.495     & 0.413 \\
    SurfEmb         &{\bf 0.538}&{\bf 0.560}&{\bf 0.558}&{\bf 0.497} \\
    CIR             & 0.381     & 0.370     & 0.379     & 0.394 \\
    Cosypose+ICP    & 0.313     & 0.315     & 0.341     & 0.282 \\
    CDPNv2+ICP      & 0.186     & 0.184     & 0.206     & 0.168 \\
    \bottomrule
\end{tabular}
    }
    \caption{{\bf Additional object pose results on ITODD.}
    % We compare our method with PFA~\cite{pfa}, SurfEmb~\cite{surfemb}, CIR~\cite{coupled_iterative}, Cosypose~\cite{cosypose}, and CDPNv2~\cite{cdpn}.
    }
    \label{tab:compare_itodd}
\end{table}

\begin{table}
    \centering
    \scalebox{0.95}{
    % \rowcolors{2}{}{gray!10}
    \begin{tabular}{lcccc}
    \toprule
    Method  &  Avg.     & MSPD      & MSSD      & VSD\\
    \midrule
    \multicolumn{5}{c}{\textit{RGB (PBR)}} \\
    \midrule
    {\bf PFA+Ours}  &{\bf 0.840}& 0.879     & 0.840     &{\bf 0.804}\\
    SurfEmb         & 0.791     &{\bf 0.888}&{\bf 0.760}& 0.725 \\
    Cosypose        & 0.656     & 0.721     & 0.634     & 0.613 \\
    CDPNv2          & 0.722     & 0.845     & 0.708     & 0.614 \\
    \midrule
    \multicolumn{5}{c}{\textit{RGB-D (PBR)}} \\
    \midrule
    {\bf PFA+Ours}  &{\bf 0.869}& 0.888     &{\bf 0.879}& {\bf 0.839} \\
    SurfEmb         & 0.866     &{\bf 0.893}&0.875      & 0.829 \\
    CIR             &{\bf 0.757}& 0.757     & 0.753     & 0.760 \\
    Cosypose+ICP    & 0.712     & 0.737     & 0.717     & 0.679 \\
    CDPNv2+ICP      & 0.712     & 0.749     & 0.757     & 0.629 \\
    \bottomrule
\end{tabular}
    }
    \caption{{\bf Additional object pose results on HB.}
    % We compare our method with PFA~\cite{pfa}, SurfEmb~\cite{surfemb}, CIR~\cite{coupled_iterative}, Cosypose~\cite{cosypose}, and CDPNv2~\cite{cdpn}.
    }
    \label{tab:compare_hb}
\end{table}

\begin{table}
    \centering
    \scalebox{0.95}{
    % \rowcolors{2}{}{gray!10}
    \begin{tabular}{lcccc}
    \toprule
    Method          & Avg.     & MSPD        & MSSD      & VSD\\
    \midrule
    \multicolumn{5}{c}{\textit{RGB (PBR)}}\\
    \midrule
    {\bf PFA+Ours}  &{\bf 0.648}& 0.771     &{\bf 0.623}& {\bf 0.550} \\
    PFA             & 0.614     & 0.739     & 0.585     & 0.522 \\
    SurfEmb         & 0.647      &{\bf 0.773}& 0.620     & 0.548 \\
    Cosypose        & 0.574     & 0.653     & 0.554     & 0.516 \\
    CDPNv2          & 0.390     & 0.511     & 0.399     & 0.260 \\
    \midrule
    \multicolumn{5}{c}{\textit{RGB (Real+PBR)}}\\
    \midrule
    {\bf PFA+Ours}  & 0.806     &{\bf 0.856}& 0.809     & 0.751 \\
    PFA             & 0.748     & 0.810     & 0.742     & 0.694 \\
    CIR             &{\bf 0.824}& 0.852     & 0.835     & {\bf 0.783}\\
    SurfEmb         & 0.711     & -         & -         & - \\
    Cosypose        & 0.821     & 0.850     &{\bf 0.842}& 0.772 \\
    CDPNv2          & 0.532     & 0.631     & 0.570     & 0.396 \\
    \midrule
    \multicolumn{5}{c}{\textit{RGB-D (PBR)}}\\
    \midrule
    {\bf PFA+Ours}  &{\bf 0.826}&{\bf 0.819}&{\bf 0.867}& {\bf 0.792} \\
    PFA             & 0.804     & 0.793     & 0.842     & 0.775 \\
    SurfEmb         & 0.799     & 0.792     & 0.849     & 0.757 \\
    CDPNv2+ICP      & 0.532     & 0.483     & 0.603     & 0.511 \\
    \midrule
    \multicolumn{5}{c}{\textit{RGB-D (Real+PBR)}}\\
    \midrule
    {\bf PFA+Ours}  & 0.888     & 0.881     & 0.920     & 0.863 \\
    PFA             & 0.823     & 0.816     & 0.852     & 0.803 \\
    CIR             &{\bf 0.893}&{\bf 0.885}& {\bf 0.924}& {\bf 0.871}\\
    SurfEmb         & 0.824       & -         & -         & - \\
    Cosypose+ICP    & 0.861     & 0.849     & 0.903     & 0.831 \\
    CDPNv2+ICP      & 0.619     & 0.565     & 0.701     & 0.590 \\
    \bottomrule
\end{tabular}
    }
    \caption{{\bf Additional object pose results on YCB.}
    % We compare our method with PFA~\cite{pfa}, SurfEmb~\cite{surfemb}, CIR~\cite{coupled_iterative}, Cosypose~\cite{cosypose}, and CDPNv2~\cite{cdpn}.
    }
    \label{tab:compare_ycbv}
\end{table}

% \section{Pose Initialization Performance}
% PFA-Pose~\cite{pfa} relies on WDR-Pose~\cite{wdr}, which works on the whole image, for pose initialization.
% We do some modifications to incorporate our detection method into WDR-Pose.
% The original WDR-Pose uses the multi-level fusion strategy with the help of FPN~\cite{fpn} to handle the large depth range.
% However, the scale change becomes small when operating on the zoomed-in patch. Thus, we remove FPN and the multi-level fusion strategy, and only supervise the predictions from the foreground cells of the last stage of the backbone by the proposed 3D loss, in fact, more similar to Segdriven~\cite{segdriven}. 
% As shown in Table~\ref{tab:compare_estimator}, although the modified WDR-Pose follows a much simpler design philosophy, it outperforms other methods by a large margin with the help of our detection method.

% \noindent \textbf{Detailed detection results.}
% We show the detection results in more detailed metrics, including AP$_{50}$ and AP$_{75}$~\cite{coco, ATSS, PAA}, in Table~\ref{tab:compare_ap50_75}. Consistent with the main paper, our method outperforms other methods significantly, especially in the more strict metric AP$_{75}$.

% \begin{table*}
%     \centering
%     \rowcolors{2}{}{gray!10}
%     % \begin{tabular}{lll|lllllll|l}
%   \toprule
%   Method                      & Real    & Data        & LM-O$^{\ast}$     & T-LESS            & TUD-L             & IC-BIN$^{\ast}$   & ITODD$^{\ast}$    & HB$^{\ast}$       & YCB-V             & Avg.\\
%   \toprule
%   {\bf PFA + Ours}            &         & RGB         & {\bf 0.876}       & 0.832             & {\bf 0.890}       & {\bf 0.689}       & 0.484             & 0.879             & 0.771             & 0.774\\
%   PFA~\cite{pfa}              &         & RGB         & 0.819             & -                 &   -               &   -               &   -               & -                 & 0.739             &    -       \\
%   SurfEmb~\cite{surfemb}      &         & RGB         & 0.851             & {\bf 0.857}       & 0.889             & 0.678             & {\bf 0.552}       & {\bf 0.888}       & {\bf 0.773}       & {\bf 0.784}     \\
%   Cosypose~\cite{cosypose}    &         & RGB         & 0.812             & 0.761             & 0.847             & 0.675             & 0.300             & 0.721             & 0.653             & 0.681     \\
%   CDPNv2~\cite{cdpn}          &         & RGB         & 0.815             & 0.579             & 0.797             & 0.582             & 0.161             & 0.845             & 0.511             & 0.613     \\
%   \midrule
%   {\bf PFA+Ours}              & \cmark  & RGB         & {\bf 0.876}       & {\bf 0.877}       & {\bf 0.978}       & {\bf 0.689}       & {\bf 0.484}       & {\bf 0.879}       & {\bf 0.856}        & {\bf 0.806}\\
%   PFA~\cite{pfa}              & \cmark  & RGB         & 0.819             & -                 & -                 & -                 & -                 & -                 & 0.810                 & - \\
%   CIR~\cite{coupled_iterative}& \cmark  & RGB         & 0.831             & 0.798             & -                 &  -                &  -                &   -               & 0.852             & -     \\
%   Cosypose~\cite{cosypose}    & \cmark  & RGB         & 0.812             & 0.821             & 0.973             & 0.675             & 0.300             & 0.721             & 0.850             & 0.736\\
%   CDPNv2~\cite{cdpn}          & \cmark  & RGB         & 0.815             & 0.620             & 0.925             & 0.582             & 0.161             & 0.845             & 0.631             & 0.654 \\
%   \midrule
%   {\bf PFA+Ours}              &         & RGB-D       & {\bf 0.890}       & 0.833             & {\bf 0.929}       & {\bf 0.702}       & 0.498             & 0.888             & {\bf 0.819}        & {\bf 0.794}\\
%   PFA~\cite{pfa}              &         & RGB-D       & 0.835             & -                 & -                 & -                 & -                 & -                 & 0.793                 & - \\
%   SurfEmb~\cite{surfemb}      &         & RGB-D       & 0.856             & {\bf 0.859}       & 0.905             & 0.680             & {\bf 0.560}       & {\bf 0.893}       & 0.792                 & 0.792     \\
%   CDPNv2+ICP~\cite{cdpn}      &         & RGB-D       & 0.731             & 0.488             & 0.829             & 0.459             & 0.184             & 0.749             & 0.483                 & 0.560 \\
%   \midrule
%   {\bf PFA+Ours}              & \cmark  & RGB-D       & {\bf 0.890}       & {\bf 0.878}       & {\bf 0.989}       & {\bf 0.702}       & {\bf 0.498}       & {\bf 0.888}       & 0.881        & {\bf 0.818}\\
%   PFA~\cite{pfa}              & \cmark  & RGB-D       & 0.835             & -                 & -                 & -                 & -                 & -                 & 0.816                 & - \\
%   CIR~\cite{coupled_iterative}&\cmark   & RGB-D       & 0.824             & 0.795             & 0.991             & 0.683             & 0.370             & 0.757             & {\bf 0.885}                 &   0.758\\
%   Cosypose+ICP~\cite{cosypose}&\cmark   & RGB-D       & 0.826             & 0.767             & 0.976             & 0.666             & 0.315             & 0.737             & 0.849                 & 0.734    \\
%   CDPNv2+ICP~\cite{cdpn}      &         & RGB-D       & 0.731             & 0.516             & 0.947             & 0.459             & 0.184             & 0.749             & 0.565                 & 0.593 \\
%   \bottomrule
% \end{tabular}

\begin{tabular}{lll|lllllll|l}
  \toprule
  Method                      & Real    & Data        & LM-O$^{\ast}$     & T-LESS            & TUD-L             & IC-BIN$^{\ast}$   & ITODD$^{\ast}$    & HB$^{\ast}$       & YCB-V             & Avg.\\
  \toprule
  {\bf PFA + Ours}            &         & RGB         & {\bf 0.876}       & 0.832             & {\bf 0.890}       & {\bf 0.689}       & 0.484             & 0.879             & 0.771             & 0.774\\
  PFA                         &         & RGB         & 0.819             & -                 &   -               &   -               &   -               & -                 & 0.739             &    -       \\
  SurfEmb                     &         & RGB         & 0.851             & {\bf 0.857}       & 0.889             & 0.678             & {\bf 0.552}       & {\bf 0.888}       & {\bf 0.773}       & {\bf 0.784}     \\
  Cosypose                    &         & RGB         & 0.812             & 0.761             & 0.847             & 0.675             & 0.300             & 0.721             & 0.653             & 0.681     \\
  CDPNv2                      &          & RGB         & 0.815             & 0.579             & 0.797             & 0.582             & 0.161             & 0.845             & 0.511             & 0.613     \\
  \midrule
  {\bf PFA+Ours}              & \cmark  & RGB         & {\bf 0.876}       & {\bf 0.877}       & {\bf 0.978}       & {\bf 0.689}       & {\bf 0.484}       & {\bf 0.879}       & {\bf 0.856}        & {\bf 0.806}\\
  PFA                         & \cmark  & RGB         & 0.819             & -                 & -                 & -                 & -                 & -                 & 0.810                 & - \\
  CIR                         & \cmark  & RGB         & 0.831             & 0.798             & -                 &  -                &  -                &   -               & 0.852             & -     \\
  Cosypose                    & \cmark  & RGB         & 0.812             & 0.821             & 0.973             & 0.675             & 0.300             & 0.721             & 0.850             & 0.736\\
  CDPNv2                      & \cmark  & RGB         & 0.815             & 0.620             & 0.925             & 0.582             & 0.161             & 0.845             & 0.631             & 0.654 \\
  \midrule
  {\bf PFA+Ours}              &         & RGB-D       & {\bf 0.890}       & 0.833             & {\bf 0.929}       & {\bf 0.702}       & 0.498             & 0.888             & {\bf 0.819}        & {\bf 0.794}\\
  PFA                         &         & RGB-D       & 0.835             & -                 & -                 & -                 & -                 & -                 & 0.793                 & - \\
  SurfEmb                     &         & RGB-D       & 0.856             & {\bf 0.859}       & 0.905             & 0.680             & {\bf 0.560}       & {\bf 0.893}       & 0.792                 & 0.792     \\
  CDPNv2+ICP                  &         & RGB-D       & 0.731             & 0.488             & 0.829             & 0.459             & 0.184             & 0.749             & 0.483                 & 0.560 \\
  \midrule
  {\bf PFA+Ours}              & \cmark  & RGB-D       & {\bf 0.890}       & {\bf 0.878}       & {\bf 0.989}       & {\bf 0.702}       & {\bf 0.498}       & {\bf 0.888}       & 0.881        & {\bf 0.818}\\
  PFA                         & \cmark  & RGB-D       & 0.835             & -                 & -                 & -                 & -                 & -                 & 0.816                 & - \\
  CIR                         &\cmark   & RGB-D       & 0.824             & 0.795             & 0.991             & 0.683             & 0.370             & 0.757             & {\bf 0.885}                 &   0.758\\
  Cosypose+ICP                &\cmark   & RGB-D       & 0.826             & 0.767             & 0.976             & 0.666             & 0.315             & 0.737             & 0.849                 & 0.734    \\
  CDPNv2+ICP                  & \cmark    & RGB-D       & 0.731             & 0.516             & 0.947             & 0.459             & 0.184             & 0.749             & 0.565                 & 0.593 \\
   \bottomrule
\end{tabular}
%     \caption{{\bf Comparison against the state of the art on MSPD metric.}}
%     \label{tab:compare_mspd}
% \end{table*}

% \begin{table*}
%     \centering
%     \rowcolors{2}{}{gray!10}
%     % \begin{tabular}{lll|lllllll|l}
%   \toprule
%   Method                      & Real              & Data      & LM-O$^{\ast}$     & T-LESS             & TUD-L            & IC-BIN$^{\ast}$    & ITODD$^{\ast}$     & HB$^{\ast}$               & YCB-V             & Avg.\\
%   \midrule
%   {\bf PFA + Ours}            &                   & RGB       & {\bf 0.712}       & 0.682             & {\bf 0.721}       & {\bf 0.589}         & 0.306             & {\bf 0.840}               & {\bf 0.623}       & {\bf 0.639}      \\
%   PFA~\cite{pfa}              &                   & RGB       & 0.673             & -                 &   -               &   -                 &   -               & -                         & 0.585             &    -       \\
%   SurfEmb~\cite{surfemb}      &                   & RGB       & 0.640             & {\bf 0.686}       & 0.687             & 0.573               & {\bf 0.363}       & 0.760                     & 0.620             & 0.618      \\
%   Cosypose~\cite{cosypose}    &                   & RGB       & 0.606             & 0.589             & 0.664             & 0.559               & 0.177             & 0.634                     & 0.554             & 0.540     \\
%   CDPNv2~\cite{cdpn}          &                   & RGB       & 0.612             & 0.338             & 0.577             & 0.438               & 0.087             & 0.708                     & 0.399             & 0.451     \\
%   \midrule
%   {\bf PFA + Ours}            & \cmark            & RGB       & {\bf 0.712}       & {\bf 0.749}       & {\bf 0.820}       & {\bf 0.589}         & {\bf 0.306}       & {\bf 0.840}               & 0.809             & {\bf 0.689}  \\
%   PFA~\cite{pfa}              & \cmark            & RGB       & 0.673             & -                 &   -               &   -                 &   -               & -                         & 0.742             &    -       \\
%   CIR~\cite{coupled_iterative}& \cmark            & RGB       & 0.633             & 0.684             & -                 &  -                  & -                 & -                         & {\bf 0.835}       & -         \\
%   Cosypose~\cite{cosypose}    & \cmark            & RGB       & 0.606             & 0.695             & 0.807             & 0.559               & 0.177             & 0.634                     & 0.842             & 0.617    \\
%   CDPNv2~\cite{cdpn}          & \cmark            & RGB       & 0.612             & 0.426             & 0.793             & 0.438               & 0.087             &0.708                      & 0.570             & 0.519\\
%   \midrule
%   {\bf PFA + Ours}            & \cmark            & RGB-D     & {\bf 0.843}       & {\bf 0.807}       & {\bf 0.930}       & {\bf 0.692}         & 0.495             & {\bf 0.879}               & {\bf 0.867}       & {\bf 0.788}      \\
%   PFA~\cite{pfa}              & \cmark            & RGB-D     & 0.798             & -                 &   -               &   -                 &   -               & -                         & 0.842            &    -       \\
%   SurfEmb~\cite{surfemb}      & \cmark            & RGB-D     & 0.809             & 0.829             & 0.891             & 0.677               & {\bf 0.558}       & 0.875                     & 0.849             & 0.784      \\
%   CDPNv2+ICP~\cite{cdpn}      & \cmark            & RGB-D     & 0.689             & 0.449             & 0.847             & 0.458               & 0.206             & 0.757                     & 0.603             & 0.573    \\
%   \midrule
%   {\bf PFA + Ours}            & \cmark            & RGB-D     & {\bf 0.843}       & {\bf 0.856}       & {\bf 0.986}       & {\bf 0.692}         & {\bf 0.495}       & {\bf 0.879}               & 0.920             & {\bf 0.810}      \\
%   PFA~\cite{pfa}              & \cmark            & RGB-D     & 0.798             & -                 &   -               &   -                 &   -               & -                         & 0.852             &    -       \\
%   CIR~\cite{coupled_iterative}& \cmark            & RGB-D     & 0.778             & 0.773             & 0.991             & 0.688               & 0.379             & 0.753                     & {\bf 0.924}       & 0.755 \\ 
%   Cosypose+ICP~\cite{cosypose}&\cmark             & RGB-D     & 0.748             & 0.749             & 0.972             & 0.652               & 0.341             & 0.719                     & 0.903             & 0.726  \\
%   CDPNv2+ICP~\cite{cdpn}      & \cmark            & RGB-D     & 0.689             & 0.489             & 0.962             & 0.458               & 0.206             & 0.757                     & 0.701             & 0.609    \\
%   \bottomrule
% \end{tabular}


\begin{tabular}{lll|lllllll|l}
  \toprule
  Method                      & Real              & Data      & LM-O$^{\ast}$     & T-LESS             & TUD-L            & IC-BIN$^{\ast}$    & ITODD$^{\ast}$     & HB$^{\ast}$               & YCB-V             & Avg.\\
  \midrule
  {\bf PFA + Ours}            &                   & RGB       & {\bf 0.712}       & 0.682             & {\bf 0.721}       & {\bf 0.589}         & 0.306             & {\bf 0.840}               & {\bf 0.623}       & {\bf 0.639}      \\
  PFA                         &                   & RGB       & 0.673             & -                 &   -               &   -                 &   -               & -                         & 0.585             &    -       \\
  SurfEmb                     &                   & RGB       & 0.640             & {\bf 0.686}       & 0.687             & 0.573               & {\bf 0.363}       & 0.760                     & 0.620             & 0.618      \\
  Cosypose                    &                   & RGB       & 0.606             & 0.589             & 0.664             & 0.559               & 0.177             & 0.634                     & 0.554             & 0.540     \\
  CDPNv2                      &                   & RGB       & 0.612             & 0.338             & 0.577             & 0.438               & 0.087             & 0.708                     & 0.399             & 0.451     \\
  \midrule
  {\bf PFA + Ours}            & \cmark            & RGB       & {\bf 0.712}       & {\bf 0.749}       & {\bf 0.820}       & {\bf 0.589}         & {\bf 0.306}       & {\bf 0.840}               & 0.809             & {\bf 0.689}  \\
  PFA                         & \cmark            & RGB       & 0.673             & -                 &   -               &   -                 &   -               & -                         & 0.742             &    -       \\
  CIR                         & \cmark            & RGB       & 0.633             & 0.684             & -                 &  -                  & -                 & -                         & {\bf 0.835}       & -         \\
  Cosypose                    & \cmark            & RGB       & 0.606             & 0.695             & 0.807             & 0.559               & 0.177             & 0.634                     & 0.842             & 0.617    \\
  CDPNv2                      & \cmark            & RGB       & 0.612             & 0.426             & 0.793             & 0.438               & 0.087             &0.708                      & 0.570             & 0.519\\
  \midrule
  {\bf PFA + Ours}            & \cmark            & RGB-D     & {\bf 0.843}       & {\bf 0.807}       & {\bf 0.930}       & {\bf 0.692}         & 0.495             & {\bf 0.879}               & {\bf 0.867}       & {\bf 0.788}      \\
  PFA                         & \cmark            & RGB-D     & 0.798             & -                 &   -               &   -                 &   -               & -                         & 0.842            &    -       \\
  SurfEmb                     & \cmark            & RGB-D     & 0.809             & 0.829             & 0.891             & 0.677               & {\bf 0.558}       & 0.875                     & 0.849             & 0.784      \\
  CDPNv2+ICP                  & \cmark            & RGB-D     & 0.689             & 0.449             & 0.847             & 0.458               & 0.206             & 0.757                     & 0.603             & 0.573    \\
  \midrule
  {\bf PFA + Ours}            & \cmark            & RGB-D     & {\bf 0.843}       & {\bf 0.856}       & {\bf 0.986}       & {\bf 0.692}         & {\bf 0.495}       & {\bf 0.879}               & 0.920             & {\bf 0.810}      \\
  PFA                         & \cmark            & RGB-D     & 0.798             & -                 &   -               &   -                 &   -               & -                         & 0.852             &    -       \\
  CIR                      & \cmark            & RGB-D     & 0.778             & 0.773             & 0.991             & 0.688               & 0.379             & 0.753                     & {\bf 0.924}       & 0.755 \\ 
  Cosypose+ICP                &\cmark             & RGB-D     & 0.748             & 0.749             & 0.972             & 0.652               & 0.341             & 0.719                     & 0.903             & 0.726  \\
  CDPNv2+ICP                  & \cmark            & RGB-D     & 0.689             & 0.489             & 0.962             & 0.458               & 0.206             & 0.757                     & 0.701             & 0.609    \\
  \bottomrule
\end{tabular}
%     \caption{{\bf Comparison against the state of the art on MSSD metric.}}
%     \label{tab:compare_mssd}
% \end{table*}

% \begin{table*}
%     \centering
%     \rowcolors{2}{}{gray!10}
%     % \begin{tabular}{lll|lllllll|l}
%   \toprule
%   Method                      & Real        & Data      & LM-O$^{\ast}$     & T-LESS             & TUD-L            & IC-BIN$^{\ast}$   & ITODD$^{\ast}$      & HB$^{\ast}$         & YCB-V             & Avg.\\
%   \midrule
%   {\bf PFA + Ours}            &             & RGB       & {\bf 0.559}       & {\bf 0.643}       & {\bf 0.594}       & {\bf 0.521}       & 0.269               & {\bf 0.804}         & {\bf 0.550}       & {\bf 0.563}     \\
%   PFA~\cite{pfa}              &             & RGB       & 0.531             & -                 &   -               &   -               &   -                 & -                   & 0.522             &   -       \\
%   SurfEmb~\cite{surfemb}      &             & RGB       & 0.497             & 0.661             & 0.569             & 0.514             & {\bf 0.324}         & 0.725               & 0.548             & 0.548     \\
%   Cosypose~\cite{cosypose}    &             & RGB       & 0.480             & 0.571             & 0.544             & 0.515             & 0.172               & 0.613               & 0.516             & 0.487  \\
%   CDPNv2~\cite{cdpn}          &             & RGB       & 0.445             & 0.303             & 0.391             & 0.399             & 0.059               & 0.614               & 0.260             & 0.353     \\
%   \midrule
%   {\bf PFA + Ours}            & \cmark      & RGB       & {\bf 0.559}       & {\bf 0.709}       & {\bf 0.719}       & {\bf 0.521}       & {\bf 0.269}         & {\bf 0.804}         & 0.751             & {\bf 0.619} \\
%   PFA~\cite{pfa}              & \cmark      & RGB       & 0.531             & -                 &   -               &   -               &   -                 & -                   & 0.694             &   -       \\
%   CIR~\cite{coupled_iterative}& \cmark      & RGB       & 0.501             & 0.663             & -                 & -                 & -                   & -                   & {\bf 0.783}       & -     \\
%   Cosypose~\cite{cosypose}    & \cmark      & RGB       & 0.480             & 0.669             & 0.689             & 0.515             & 0.172               & 0.613               & 0.772             & 0.559  \\
%   % SurfEmb~\cite{surfemb}      & \cmark             & RGB       & 0.497             & -                 & -                 & 0.514                     & 0.324                     &0.725                      & -                 & -     \\
%   CDPNv2~\cite{cdpn}          & \cmark      & RGB       & 0.445             & 0.386             & 0.597             & 0.399             & 0.059               & 0.614               & 0.396             & 0.414 \\
%   \midrule
%   {\bf PFA + Ours}            &             & RGB-D     & {\bf 0.658}       & 0.764             & {\bf 0.821}       & {\bf 0.636}       & {\bf 0.413}         & {\bf 0.839}         & {\bf 0.792}       & {\bf 0.703}     \\
%   PFA~\cite{pfa}              &             & RGB-D     & 0.621             & -                 &   -               &   -               &   -                 & -                   & 0.775             &   -       \\
%   SurfEmb~\cite{surfemb}      &             & RGB-D     & 0.615             & {\bf 0.797}       & 0.767             & 0.621             & 0.497               & 0.829               & 0.757             & 0.698     \\
%   CDPNv2+ICP~\cite{cdpn}      &             & RGB-D     & 0.469             & 0.368             & 0.698             & 0.433             & 0.168               & 0.629               & 0.511             & 0.468     \\
%   \midrule
%   {\bf PFA + Ours}            & \cmark      & RGB-D     & {\bf 0.658}       & {\bf 0.816}       & 0.904             & 0.636             & {\bf 0.413}         & {\bf 0.839}         & 0.863             & {\bf 0.733}     \\
%   PFA~\cite{pfa}              & \cmark      & RGB-D     & 0.621             & -                 &   -               &   -               &   -                 & -                   & 0.803             &   -       \\
%   CIR~\cite{coupled_iterative}& \cmark      & RGB-D     & 0.601             & 0.760             & {\bf 0.920}       & {\bf 0.656}       & 0.394               & 0.760               & {\bf 0.871}       & 0.709\\
%   % SurfEmb~\cite{surfemb}      \cmark&             & R-DGB      & 0.615             & -                 & -                 & 0.621                     & 0.497                     & 0.829                     & -                 & 0.698     \\
%   Cosypose+ICP~\cite{cosypose}& \cmark      & RGB-D     & 0.567             & 0.587             & 0.869             & 0.624             & 0.282               & 0.679               & 0.831             & 0.634\\
%   CDPNv2+ICP~\cite{cdpn}      & \cmark      & RGB-D     & 0.469             & 0.385             & 0.832             & 0.433             & 0.168               & 0.629               & 0.590             & 0.501     \\
%   \bottomrule
% \end{tabular}


\begin{tabular}{lll|lllllll|l}
  \toprule
  Method                      & Real        & Data      & LM-O$^{\ast}$     & T-LESS             & TUD-L            & IC-BIN$^{\ast}$   & ITODD$^{\ast}$      & HB$^{\ast}$         & YCB-V             & Avg.\\
  \midrule
  {\bf PFA + Ours}            &             & RGB       & {\bf 0.559}       & {\bf 0.643}       & {\bf 0.594}       & {\bf 0.521}       & 0.269               & {\bf 0.804}         & {\bf 0.550}       & {\bf 0.563}     \\
  PFA                         &             & RGB       & 0.531             & -                 &   -               &   -               &   -                 & -                   & 0.522             &   -       \\
  SurfEmb                     &             & RGB       & 0.497             & 0.661             & 0.569             & 0.514             & {\bf 0.324}         & 0.725               & 0.548             & 0.548     \\
  Cosypose                    &             & RGB       & 0.480             & 0.571             & 0.544             & 0.515             & 0.172               & 0.613               & 0.516             & 0.487  \\
  CDPNv2                      &             & RGB       & 0.445             & 0.303             & 0.391             & 0.399             & 0.059               & 0.614               & 0.260             & 0.353     \\
  \midrule
  {\bf PFA + Ours}            & \cmark      & RGB       & {\bf 0.559}       & {\bf 0.709}       & {\bf 0.719}       & {\bf 0.521}       & {\bf 0.269}         & {\bf 0.804}         & 0.751             & {\bf 0.619} \\
  PFA                         & \cmark      & RGB       & 0.531             & -                 &   -               &   -               &   -                 & -                   & 0.694             &   -       \\
  CIR                         & \cmark      & RGB       & 0.501             & 0.663             & -                 & -                 & -                   & -                   & {\bf 0.783}       & -     \\
  Cosypose                    & \cmark      & RGB       & 0.480             & 0.669             & 0.689             & 0.515             & 0.172               & 0.613               & 0.772             & 0.559  \\
  % SurfEmb~\cite{surfemb}      & \cmark             & RGB       & 0.497             & -                 & -                 & 0.514                     & 0.324                     &0.725                      & -                 & -     \\
  CDPNv2                      & \cmark      & RGB       & 0.445             & 0.386             & 0.597             & 0.399             & 0.059               & 0.614               & 0.396             & 0.414 \\
  \midrule
  {\bf PFA + Ours}            &             & RGB-D     & {\bf 0.658}       & 0.764             & {\bf 0.821}       & {\bf 0.636}       & 0.413              & {\bf 0.839}         & {\bf 0.792}       & {\bf 0.703}     \\
  PFA                         &             & RGB-D     & 0.621             & -                 &   -               &   -               &   -                 & -                   & 0.775             &   -       \\
  SurfEmb                     &             & RGB-D     & 0.615             & {\bf 0.797}       & 0.767             & 0.621             & {\bf 0.497}         & 0.829               & 0.757             & 0.698     \\
  CDPNv2+ICP                  &             & RGB-D     & 0.469             & 0.368             & 0.698             & 0.433             & 0.168               & 0.629               & 0.511             & 0.468     \\
  \midrule
  {\bf PFA + Ours}            & \cmark      & RGB-D     & {\bf 0.658}       & {\bf 0.816}       & 0.904             & 0.636             & {\bf 0.413}         & {\bf 0.839}         & 0.863             & {\bf 0.733}     \\
  PFA                         & \cmark      & RGB-D     & 0.621             & -                 &   -               &   -               &   -                 & -                   & 0.803             &   -       \\
  CIR                         & \cmark      & RGB-D     & 0.601             & 0.760             & {\bf 0.920}       & {\bf 0.656}       & 0.394               & 0.760               & {\bf 0.871}       & 0.709\\
  % SurfEmb~\cite{surfemb}      \cmark&             & R-DGB      & 0.615             & -                 & -                 & 0.621                     & 0.497                     & 0.829                     & -                 & 0.698     \\
  Cosypose+ICP                & \cmark      & RGB-D     & 0.567             & 0.587             & 0.869             & 0.624             & 0.282               & 0.679               & 0.831             & 0.634\\
  CDPNv2+ICP                  & \cmark      & RGB-D     & 0.469             & 0.385             & 0.832             & 0.433             & 0.168               & 0.629               & 0.590             & 0.501     \\
  \bottomrule
\end{tabular}
%     \caption{{\bf Comparison against the state of the art on VSD metric.}}
%     \label{tab:compare_vsd}
% \end{table*}



% \begin{table}
%     \centering
%     \rowcolors{2}{}{gray!10}
%     \begin{tabular}{cc|ccccccc|c}
    \toprule
    Method                      & Real          & LM-O              & T-LESS            & TUD-L             & IC-BIN            & ITODD             & HB                & YCB-V     & Avg.\\
    \midrule
    \textbf{WDR+Ours}          &               & 0.614             & \textbf{0.588}    & \textbf{0.623}    & 0.514             & \textbf{0.247}    & 0.618             & \textbf{0.605}    & \textbf{0.544} \\
    CDPNv2~\cite{cdpn}          &               & \textbf{0.624}    & 0.407             & 0.588             & 0.473             & 0.102             & \textbf{0.722}    & 0.390             & 0.472 \\
    EPOS~\cite{epos}            &               & 0.547             & 0.467             & 0.558             & 0.363             & 0.186             & 0.580             & 0.499             & 0.457  \\
    CosyPose\dag~\cite{cosypose}&               & 0.533             & 0.520             & 0.582             & \textbf{0.515}    & 0.129             & 0.334             & 0.333             & 0.421 \\
    \midrule
    \textbf{WDR+Ours}          & \cmark    & 0.614             & \textbf{0.678}    & \textbf{0.796}    & 0.514             & \textbf{0.247}    & 0.618             & \textbf{0.722}    & \textbf{0.598} \\
    CDPNv2~\cite{cdpn}          & \cmark    & \textbf{0.624}    & 0.478             & 0.772             & 0.473             & 0.102             & \textbf{0.722}    & 0.532             & 0.529 \\
    CosyPose\dag~\cite{cosypose}& \cmark    & 0.533             & 0.616             & 0.742             & \textbf{0.515}    & 0.129             & 0.334             & 0.655             & 0.503 \\
    Pix2Pose~\cite{pix2pose}    & \cmark    & 0.363             & 0.344             & 0.420             & 0.226             & 0.134             & 0.346             & 0.446             & 0.342 \\
    % SO-Pose~\cite{so-pose}      & \checkmark    & 0.613             & -                 & -                 & -                 & -                 &   -               & 0.715             & -      \\
    \bottomrule
\end{tabular}
%     \caption{{\bf Comparison against the 6D pose estimation methods without pose refinement.}
%     Cosypose$\dag$ denotes Cosypose without pose refinement.}
%     \label{tab:compare_estimator}
% \end{table}

% \begin{figure*}
%     \centering
%     \setlength\tabcolsep{1pt}
%     \begin{tabular}{cccc}
    \includegraphics[width=0.22\linewidth, trim={0 0 0 20}, clip]{figures/qualitative results/detection_results/000002_000494.jpg} &  
    \includegraphics[width=0.22\linewidth, trim={0 0 0 20}, clip]{figures/qualitative results/detection_results/000002_000521.jpg} &
    \includegraphics[width=0.22\linewidth, trim={0 0 0 20}, clip]{figures/qualitative results/detection_results/000002_000560.jpg} &
    \includegraphics[width=0.22\linewidth, trim={0 0 0 20}, clip]{figures/qualitative results/detection_results/000002_000785.jpg} \\
    \includegraphics[width=0.22\linewidth, trim={0 0 0 20}, clip]{figures/qualitative results/refine_results/000002_000494.jpg} &  
    \includegraphics[width=0.22\linewidth, trim={0 0 0 20}, clip]{figures/qualitative results/refine_results/000002_000521.jpg} &
    \includegraphics[width=0.22\linewidth, trim={0 0 0 20}, clip]{figures/qualitative results/refine_results/000002_000560.jpg} &
    \includegraphics[width=0.22\linewidth, trim={0 0 0 20}, clip]{figures/qualitative results/refine_results/000002_000785.jpg} \\
    \includegraphics[width=0.22\linewidth, trim={0 50 0 50}, clip]{figures/qualitative results/detection_results/000016_000381.jpg} &
    \includegraphics[width=0.22\linewidth, trim={0 50 0 50}, clip]{figures/qualitative results/detection_results/000018_000246.jpg} &
    \includegraphics[width=0.22\linewidth, trim={0 50 0 50}, clip]{figures/qualitative results/detection_results/000020_000094.jpg} &
    \includegraphics[width=0.22\linewidth, trim={0 50 0 50}, clip]{figures/qualitative results/detection_results/000020_000174.jpg} \\
    \includegraphics[width=0.22\linewidth, trim={0 50 0 50}, clip]{figures/qualitative results/refine_results/000016_000381.jpg} &
    \includegraphics[width=0.22\linewidth, trim={0 50 0 50}, clip]{figures/qualitative results/refine_results/000018_000246.jpg} &
    \includegraphics[width=0.22\linewidth, trim={0 50 0 50}, clip]{figures/qualitative results/refine_results/000020_000094.jpg} &
    \includegraphics[width=0.22\linewidth, trim={0 50 0 50}, clip]{figures/qualitative results/refine_results/000020_000174.jpg} \\
    \includegraphics[width=0.22\linewidth, trim={0 50 0 0}, clip]{figures/qualitative results/detection_results/000001_000036.jpg} &
    \includegraphics[width=0.22\linewidth, trim={0 50 0 0}, clip]{figures/qualitative results/detection_results/000001_000015.jpg} &
    \includegraphics[width=0.22\linewidth, trim={0 50 0 0}, clip]{figures/qualitative results/detection_results/000003_000014.jpg} &
    \includegraphics[width=0.22\linewidth, trim={0 50 0 0}, clip]{figures/qualitative results/detection_results/000003_000057.jpg} \\
    \includegraphics[width=0.22\linewidth, trim={0 50 0 0}, clip]{figures/qualitative results/refine_results/000001_000036.jpg} &
    \includegraphics[width=0.22\linewidth, trim={0 50 0 0}, clip]{figures/qualitative results/refine_results/000001_000015.jpg} &
    \includegraphics[width=0.22\linewidth, trim={0 50 0 0}, clip]{figures/qualitative results/refine_results/000003_000014.jpg} &
    \includegraphics[width=0.22\linewidth, trim={0 50 0 0}, clip]{figures/qualitative results/refine_results/000003_000057.jpg} \\
    \includegraphics[width=0.22\linewidth, trim={0 0 0 20}, clip]{figures/qualitative results/detection_results/000053_000103.jpg} &
    \includegraphics[width=0.22\linewidth, trim={0 0 0 20}, clip]{figures/qualitative results/detection_results/000054_000032.jpg} &
    \includegraphics[width=0.22\linewidth, trim={0 0 0 20}, clip]{figures/qualitative results/detection_results/000055_000618.jpg} &
    \includegraphics[width=0.22\linewidth, trim={0 0 0 20}, clip]{figures/qualitative results/detection_results/000056_00075.jpg} \\
    \includegraphics[width=0.22\linewidth, trim={0 0 0 20}, clip]{figures/qualitative results/refine_results/000053_000103.jpg} &
    \includegraphics[width=0.22\linewidth, trim={0 0 0 20}, clip]{figures/qualitative results/refine_results/000054_000032.jpg} &
    \includegraphics[width=0.22\linewidth, trim={0 0 0 20}, clip]{figures/qualitative results/refine_results/000055_000618.jpg} &
    \includegraphics[width=0.22\linewidth, trim={0 0 0 20}, clip]{figures/qualitative results/refine_results/000056_000075.jpg} \\ 
\end{tabular}
%     \caption{{\bf Additional qualitative results on LM-O, T-LESS, IC-BIN, and YCB-V.} We present the results on different datasets from top to bottom, showing the detection and object pose results for each.}
%     \label{fig:more_vis}
% \end{figure*}

% \begin{figure}
%     \centering
%     \setlength\tabcolsep{1pt}
%     \input{figures/more_visulaization_part2}
%     \caption{{\bf More visualization results on YCB-V.}}
%     \label{fig:more_vis_part2}
% \end{figure}

% \begin{table}
%     \centering
%     \rowcolors{3}{gray!10}{}
%     % \begin{tabular}{c|ccc|ccc|ccc}
% \toprule
%     \multirow{2}{*}{Data}    & \multicolumn{2}{c|}{{\bf Ours}}    &  \multicolumn{2}{c|}{FCOSv2} & \multicolumn{2}{c}{Mask R-CNN} \\
%     ~       & AP$_{50}$     & AP$_{75}$         & AP$_{50}$     & AP$_{75}$      & AP$_{50}$     & AP$_{75}$ \\
%     \midrule
%     LM-O    &{\bf 94.2}     &{\bf 80.7}         &   90.0        & 62.8          &87.2           &  64.7\\
%     T-LESS  & 88.9          & {\bf 86.4}        &{\bf 89.1}          & 80.6          &   83.9        &  75.3  \\
%     TUD-L   & {\bf 100.0}         & {\bf 97.9}        & {\bf 100.0}       & 97.8          & {\bf 100.0}       &    96.8    \\
%     IC-BIN  & {\bf 88.2}          &  {\bf 75.9}       &   50.0        & 27.1          & 74.6          & 39.8   \\
%     ITODD   & {\bf 62.3}    & {\bf 55.4}        & 40.5          & 34.0         & 50.2          & 40.5  \\
%     HB      &{\bf 89.7}     & {\bf 81.7}        &  81.9         & 63.8          & 86.8          & 69.3       \\
%     YCB-V   & {\bf 99.4}    & {\bf 97.4}        & 98.6          & 89.1          &  92.7         & 84.2       \\
%     Avg.    & {\bf 89.0}    & {\bf 82.2}         & 78.6         & 65.0        & 82.2          & 67.2      \\
%     \bottomrule
% \end{tabular}  


% \begin{tabular}{c|ccc|ccc|ccc}
%   \toprule
%   \multirow{2}{*}{Data}    & \multicolumn{3}{c|}{AP}                        &  \multicolumn{3}{c|}{AP$_{50}$}   & \multicolumn{3}{c}{AP$_{75}$} \\
% %   ~                        & Ours & FCOSv2  & Mask R-CNN                    & Ours & FCOSv2  & Mask R-CNN       & Ours & FCOSv2  & Mask R-CNN\\
%     ~           & {\bf Ours} & ~\cite{fcosv2}  & ~\cite{maskrcnn}    & {\bf Ours} & ~\cite{fcosv2}  & ~\cite{maskrcnn} & {\bf Ours} & ~\cite{fcosv2}  & ~\cite{maskrcnn}\\
%   \midrule
%   LM-O                     & {\bf 67.5} &57.0&56.6&{\bf 94.2}     & 90.0          & 87.2          &{\bf 80.7}     & 62.8  & 64.7\\
%   T-LESS                   & {\bf 79.8} &75.0&69.3& 88.9          & {\bf 89.1}    & 83.9          & {\bf 86.4}    & 80.6  & 75.3  \\
%   TUD-L                    & {\bf 86.6} &86.0&82.6& {\bf 100.0}   & {\bf 100.0}   & {\bf 100.0}   & {\bf 97.9}    & 97.8  & 96.8    \\
%   IC-BIN                   & {\bf 63.8 }&27.2&40.1 & {\bf 88.2}    & 50.0          & 74.6          & {\bf 75.9}    & 27.1  & 39.8   \\
%   ITODD                    & {\bf 48.6} &30.4&36.5& {\bf 62.3}    & 40.5          & 50.2          & {\bf 55.4}    & 34.0  & 40.5  \\
%   HB                       &{\bf 73.5}  &60.4&63.5 &{\bf 89.7}     & 81.9          & 86.8          & {\bf 81.7}    & 63.8  & 69.3       \\
%   YCB-V                    &{\bf 85.0}  &80.0& 74.5& {\bf 99.4}    & 98.6          & 92.7          & {\bf 97.4}    & 89.1  & 84.2       \\
%   Avg.                     &{\bf 72.1}  &66.7&60.5& {\bf 89.0}    & 78.6          & 82.2          & {\bf 82.2}    & 65.0  & 67.2      \\
%   \bottomrule
% \end{tabular}


\begin{tabular}{cccc}
  \toprule
  Data        & {\bf Ours}        & \cite{fcosv2}        & ~\cite{maskrcnn}    \\
  \midrule
  \multicolumn{4}{c}{\textit{AP}}\\
  \midrule
  LM-O                     & {\bf 67.5} &57.0&56.6   \\
  T-LESS                   & {\bf 79.8} &75.0&69.3     \\
  TUD-L                    & {\bf 86.6} &86.0&82.6       \\
  IC-BIN                   & {\bf 63.8 }&27.2&40.1      \\
  ITODD                    & {\bf 48.6} &30.4&36.5     \\
  HB                       &{\bf 73.5}  &60.4&63.5          \\
  YCB-V                    &{\bf 85.0}  &80.0& 74.5         \\
  Avg.                     &{\bf 72.1}  &66.7&60.5         \\
  \midrule
  \multicolumn{4}{c}{\textit{AP$_{50}$}}\\
  \midrule
  LM-O                     &{\bf 94.2}     & 90.0          & 87.2        \\
  T-LESS                   & 88.9          & {\bf 89.1}    & 83.9          \\
  TUD-L                    & {\bf 100.0}   & {\bf 100.0}   & {\bf 100.0}     \\
  IC-BIN                   & {\bf 88.2}    & 50.0          & 74.6           \\
  ITODD                    & {\bf 62.3}    & 40.5          & 50.2          \\
  HB                       &{\bf 89.7}     & 81.9          & 86.8               \\
  YCB-V                    & {\bf 99.4}    & 98.6          & 92.7               \\
  Avg.                     & {\bf 89.0}    & 78.6          & 82.2              \\
  \midrule
  \multicolumn{4}{c}{\textit{AP$_{75}$}}\\
  \midrule
  LM-O                    &{\bf 80.7}     & 62.8  & 64.7\\
  T-LESS                  & {\bf 86.4}    & 80.6  & 75.3  \\
  TUD-L                   & {\bf 97.9}    & 97.8  & 96.8    \\
  IC-BIN                   & {\bf 75.9}    & 27.1  & 39.8   \\
  ITODD                   & {\bf 55.4}    & 34.0  & 40.5  \\
  HB                       & {\bf 81.7}    & 63.8  & 69.3       \\
  YCB-V                    & {\bf 97.4}    & 89.1  & 84.2       \\
  Avg.                    & {\bf 82.2}    & 65.0  & 67.2      \\
\bottomrule
\end{tabular}
%     \caption{{\bf Detailed detection results.}
%     We compare our method with FCOSv2~\cite{fcosv2} and Mask R-CNN~\cite{maskrcnn} in different detection metrics.
%     }
%     \label{tab:compare_ap50_75}
% \end{table}
% \begin{table*}
%     \centering
%     \rowcolors{2}{}{gray!10}
%     \begin{tabular}{c|ccccccc|c}
    \toprule
    Method      & LM-O       & T-LESS   & TUD-L     & IC-BIN    & ITODD     & HB        & YCB-V     & Avg.  \\
    \midrule
    Ours        & {\bf 94.2}& 88.9      & {\bf 100.0} & {\bf 88.2}& {\bf 62.3}& {\bf 89.7}& {\bf 99.4}&{\bf 89.0}     \\
    FCOSv2      & 90.0      & {\bf 89.1}& {\bf 100.0} & 50.0      & 40.5      & 81.9      & 98.6      &78.6  \\
    Mask R-CNN  & 87.2      & 83.9      & {\bf 100.0} & 74.6      & 50.2      & 86.8      & 92.7      &82.2    \\
    \bottomrule
\end{tabular}
%     \caption{{\bf Detection comparison on AP$_{50}$ metric.}}
%     \label{tab:compare_ap50}
% \end{table*}

% \begin{table*}
%     \centering
%     \rowcolors{2}{}{gray!10}
%     \begin{tabular}{c|ccccccc|c}
    \toprule
    Method      & LM-O          & T-LESS        & TUD-L         & IC-BIN    & ITODD     & HB        & YCB-V         & Avg.  \\
    \midrule
    Ours        & {\bf 80.7}    & {\bf 86.4}    & {\bf 97.9}    & {\bf 75.9}& {\bf 55.4}& {\bf 81.7}& {\bf 97.4}    & {\bf 82.2}     \\
    FCOSv2      & 62.8          & 80.6          & 97.8          & 27.1      & 34.0      & 63.8      & 89.1          & 65.0   \\
    Mask R-CNN  & 64.7          & 75.3          & 96.8          & 39.8      & 40.5      & 69.3      & 84.2          & 67.2  \\
    \bottomrule
\end{tabular}
%     \caption{{\bf Detection comparison on AP$_{75}$ metric.}}
%     \label{tab:compare_ap75}
% \end{table*}


% \onecolumn
% \begin{center}
%     \Large \textbf{Rigidity-Aware Detection for 6D Object Pose Estimation Appendix}
% \end{center}




% \section{PFA Improvement}
% To attend the BOP Challenge, we improve PFA~\cite{pfa} in several aspects.

% \noindent \textbf{Online Rendering}
% PFA renders the pose exemplars offline, which will cause colossal storage usage when participating in the 2022 BOP Challenge because it has a total of 125 objects, resulting in 1.25 million images to be stored.
% While rendering is still potentially time-consuming, we observe Pytorch3D~\cite{pytorch3d}, efficient at rendering in batch, can relieve this situation.
% Therefore we construct the pose exemplars of all the detected instances to batch format and render them simultaneously.
% The cost is affordable compared to the overall inference speed, and the storage usage can be reduced.



% \noindent \textbf{Iterative Refinement}
% In practice, we find multi-exemplar aggregation brings only marginal improvement but largely slows the inference speed.
% On the other hand, recurrent refinement~\cite{deepim,cosypose, repose,rnnpose} can boost performance more significantly.
% For example, refining one time using four exemplars yields a similar throughput as refining four times using the single initial pose, but the latter can achieve better performance.
% Thus, we only use the initial pose instead and recurrently refine it four times.


% \noindent \textbf{Bi-directional Flow}
% Although PFA uses the 2D-to-2D correspondences from the pose exemplar to the target patch, it is feasible to solve the pose using the 2D-to-2D correspondences from the target patch to the pose exemplar.
% To do so, we additionally predicts the target patch's foreground mask, and extracts the foreground points.
% We then interpolate the corresponding points' depth, and lift the corresponding 2D points to 3D object frame with the known object pose and camera intrinsic.
% In this way, we obtain the 2D-3D correspondences.

% \noindent \textbf{Using Depth}
% We further extend PFA to be capable of handling RGB-D input.
% For the RGB setting, PFA uses a PnP algorithm~\cite{epnp} to solve the pose from 2D-3D correspondences.
% When depth images are available, we lift the 2D points to 3D, then utilize RANSAC-Kabsch~\cite{ncf,dpodv2} to solve the pose from 3D-3D correspondences.
% Compared to the dominant RGB-D methods~\cite{DenseFusion,ffb6d}, our method works with either RGB or RGB-D, which is more flexible.

{\small
\bibliographystyle{ieee_fullname}
\bibliography{egbib}
}

\end{document}

