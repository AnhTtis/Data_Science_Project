\documentclass[9pt]{osa-supplemental-document}
\setboolean{shortarticle}{false}

\title{20 GHz fiber-integrated femtosecond pulse and supercontinuum generation with a resonant electro-optic frequency comb}
%\title{20 GHz fiber-integrated femtosecond pulse and supercontinuum generation}

\author[1,2,*]{Pooja Sekhar}
\author[1,2]{Connor Fredrick}
\author[2,4]{David R. Carlson}
\author[2,4]{Zachary Newman}
\author[1,2,3,\dag]{Scott A. Diddams}

\affil[1]{Department of Physics, University of Colorado Boulder, 440 UCB, Boulder, Colorado 80309, USA}
\affil[2]{Time and Frequency Division, National Institute of Standards and Technology, 325 Broadway, Boulder, Colorado 80305, USA}
\affil[3]{Department of Electrical, Computer and Energy Engineering, University of Colorado Boulder, Colorado 80309, USA}
\affil[4]{Octave Photonics, 325 W South Boulder Rd, Louisville, Colorado 80027, USA}

\affil[*]{Corresponding author: pooja.sekhar@colorado.edu}

\affil[$\dag$]{email: scott.diddams@colorado.edu}
\begin{abstract}
\end{abstract}
\setboolean{displaycopyright}{false} %copyright statement should not display in the  supplementary document

\begin{document}

\maketitle

\section{Resonant Electro-optic Comb Generator}

We employ a compact Fabry-P\'erot electro-optic frequency comb generator originally developed by Kourogi et. al \cite{KourogiM1993Wide-spanMeasurement}. It consists of a fiber-integrated waveguide phase modulator in a resonant Fabry-P\'erot cavity  that was fabricated by coating high-reflection films on the facets of a lithium niobate waveguide (Fig. \ref{scheme}(a)). The finesse and free spectral range (FSR) of the optical resonator are 58 and 2.5 GHz, respectively \cite{Saitoh1995AGenerator}, which allows us to generate frequency combs with repetition rates at integer multiples of the cavity FSR. This commercially available resonant electro-optic comb generator (REOCG, Optocomb model: WTEC-01 \cite{Saitoh1995AGenerator}) is kept inside a metal housing as shown in Fig. \ref{scheme}(b). The REOCG has a built-in temperature control unit and the temperature of the cavity can be easily tuned to reach the resonance condition. The main advantage of this resonant electro-optic comb generation over the cascaded EO modulators approach is its highly efficient resonance modulation. It requires $\sim$10 times lower microwave power to generate the same or even larger bandwidth spectra and it requires no external cooling for long-term use.

\begin{figure}[htb]
\centering\includegraphics[width=13cm]{K_comb_new.pdf}
\caption{(a) Schematic of the resonant electro-optic frequency comb generator (REOCG) that consists of a fiber-integrated waveguide phase modulator with high reflection films on each facet. (b) The commercially available REOCG is kept inside a metal housing.}
\label{scheme}
\end{figure}

One of the most interesting features of this resonant comb generator that distinguishes it from cascaded EOM combs is its unique output characteristics in the frequency and time domains \cite{Kobayashi1972High-repetition-rateModulator, Saitoh1998ModulationGenerator, Xiao2008TowardGenerator}. In the frequency domain, the comb envelope resembles a double-sided exponential as shown in Fig. \ref{output}(a). A symmetrical comb spectrum with maximum bandwidth is generated when both the optical and microwave signals are resonant inside the cavity. As the beam circulates forward and backward through a phase modulator inside the cavity, the comb modes above and below the seed frequency develop linear spectral phases with different slopes. This gives rise to a fixed group delay between the two time-domain pulse trains that come into resonance with the cavity as the effective cavity length is modulated (shown as the red and blue Lorentzian traces in Fig. \ref{output}(b)). As a result, the time-domain output of the REOCG consists of two interleaved pulse trains with a relative time delay. This time delay varies from zero to half the modulation period depending on the detuning between the CW laser and the cavity resonance. At resonance, the two interleaved pulse trains appear as a single pulse train at a repetition rate that is twice the modulation frequency applied. In this work, we either overlap the two interleaved pulse trains or remove one of them to achieve higher pulse energy by applying a group delay (in the case of 20 GHz) or by removing half of the optical spectrum (in the case of 10 GHz) using a pulse shaper. When both the seed laser and microwave signal are resonant inside the cavity as in the ideal condition, transmission through the REOCG can be approximated by modifying the transmission formula for a Fabry-P\'erot cavity by including the phase shift from the modulator. Using this assumption, the transmitted electric field from the REOCG in time domain can be written as \cite{Kobayashi1972High-repetition-rateModulator, Saitoh1998ModulationGenerator, Xiao2008TowardGenerator} 
\begin{equation}
E_{t}(t, \beta) \approx \sqrt{\eta} \frac{1-R}{1-R \eta \exp \left[-i \beta \sin \left(2 \pi f_{m} t\right)\right]} E_{i},
\label{eq:refname1}
\end{equation}
where $\eta$ is the single-pass power transmission efficiency of the waveguide modulator ($\eta = 0.976$ \cite{Xiao2008TowardGenerator}), R is the reflectance of the Fabry-P\'erot mirrors (R = 0.97), $\beta = \pi(V/V_{\pi})$ is the modulation index ($\beta \sim 3$ for 1 W RF power) and $f_{m}$ is the modulation frequency. In our device, the total insertion loss is measured to be $\sim$ 25 dB. This includes the fiber-to-waveguide coupling loss as well as the waveguide propagation loss. Interestingly, Loncar \textit{et al.} have demonstrated an improved pump-to-comb conversion efficiency for the resonant EO comb generator on thin-film lithium niobate using two mutually coupled resonators \cite{Hu2022High-efficiencyGenerators}.  

\begin{figure}[htb]
\centering\includegraphics[width=13.3cm]{K_comb_output_char.pdf}
\caption{(a) Optical spectral output of the REOCG. The measured spectrum is shown by the green trace. The simulated spectrum is given by the red and blue traces. The rise in noise floor due to amplification has not been considered in the simulated spectrum. The inset figure shows the zoomed-in 10 GHz comb modes. (b) Temporal output of the REOCG showing the interleaved pulse trains. Each pulse train corresponds to one side of the spectrum as indicated by the red and blue colors.}
\label{output}
\end{figure}

Fig. \ref{lock} shows the Pound-Drever-Hall (PDH) technique \cite{Drever1983LaserResonator, Black2001AnStabilization} employed for locking the comb generator cavity length so that the detuning between the CW laser and the cavity resonance is held at zero. In this scheme, the CW laser is amplified using an erbium-doped fiber amplifier (EDFA) and then sent to the resonant electro-optic comb generator (REOCG) via a fiber circulator. The light reflected from the comb generator is then collected through one of the circulator ports and measured on a photodetector (PD). The PD output is compared with a phase-shifted modulation signal (10 or 20 GHz) via a mixer. The mixer output signal is then sent to a servo (image shown in Fig. \ref{lock}). The phase of the modulation signal is adjusted to compensate for the unequal delay between the two paths. The proportional and integral gain knobs are then tuned so that the DC output from the servo locks the cavity length to the minimum of the reflected signal. 

\begin{figure}[htb]
\centering\includegraphics[width=13.3cm]{pdh_new3.pdf}
\caption{The basic layout of the Pound-Drever-Hall technique used for locking the comb generator cavity to the cavity-stabilized continuous wave (CW) laser. EDFA is erbium-doped fiber amplifier, Circ is a fiber circulator, REOCG is the resonant electro-optic comb generator, PD is a photodiode, and PS is a phase shifter.}
\label{lock}
\end{figure}

\section{Temporal Compression of a Cascaded EOM Comb}

To further show the versatility of our all-fiber temporal compression design, we tried employing the same approach to the output of a frequency comb generated using cascaded electro-optic modulators. In what follows, we describe the details of numerically modeling the comb generator as well as the fiber temporal compression design.

\begin{figure}[htb]
\centering\includegraphics[width=13.3cm]{cascaded_eom_comp_new.pdf}
\caption{(a) Optical spectral output of the cascaded EOM consisting of two phase modulators (PM) and one intensity modulator (IM). (b) Schematic of the all-fiber temporal compressor. (c) A compressed pulse duration of 58 fs is obtained on simulating the propagation of the band-limited EOM output through 27 m PM1550 and then through the temporal compression fiber design consisting of 3.2 m of ND HNLF (D = -2.6 ps/(nm$\cdot$km)) and 56 cm of PM1550. The inset shows the band-limited pulse of the EOM output assuming a flat spectral phase (blue trace) and then propagating it through 27 m of PM1550 for initial temporal compression through soliton effects (orange trace).}
\label{cascaded eom}
\end{figure}
Initially, we simulated a general case in which a CW laser at 1550 nm seeds two phase modulators (PM) and one intensity modulator (IM) arranged in series. Each PM is driven by a 2 W microwave signal at 10 GHz. The resulting simulated 10 GHz comb spanning $\sim$ 0.34 THz (or ~3 nm) is shown in Fig. \ref{cascaded eom}(a). The normal chirp from the phase modulators can be compensated on propagation through PM1550 to obtain a band-limited pulse duration of 2.2 ps (blue trace in the inset of Fig. \ref{cascaded eom}(c)). The above band-limited pulses are then amplified to 3W (pulse energy of 300 pJ at 10 GHz). In order to compress an input pulse duration greater than 1.5 ps with 300 pJ energy using our current all-fiber approach, the simulation (Fig. 1(g) in the main document) predicts more than 36 m of ND HNLF (D = -2.6 ps/(nm. km)) to produce sub-100 fs pulse. %which is not experimentally feasible in terms of optical loss through very long fibers as well as HNLF cost.
It may be possible to shorten the length of ND HNLF required by choosing the HNLF with a larger normal dispersion value at 1550 nm or using additional phase modulators. %However, we are limited by the commercially available options. the required length of ND HNLF might still be beyond the range ($>$ 10 m) that we normally use for these applications in lab. 
However, in order to get around this issue, we simulated propagating the 2.2 ps pulse through PM1550 for an initial temporal compression via soliton self-compression (Fig. \ref{cascaded eom}(b)). This occurs due to the combined effect of anomalous dispersion and self-phase modulation which leads to the formation and propagation of a higher-order soliton. The simulations show that the pulse duration reduces to 411 fs after propagation through 27 m of PM1550 (orange trace in the inset of Fig. \ref{cascaded eom}(c)). 
%This soliton self-compression can be more efficiently done in a shorter length of anomalous dispersion HNLF that has a larger anomalous dispersion value than PM1550 at 1550 nm.

The resultant pulse duration of 411 fs is then considered as input to our temporal compression fiber design. Further compression can be efficiently achieved by spectral broadening in ND HNLF and chirp compensation in PM1550. The simulation results give a compressed pulse duration of 58 fs on propagating through 3.2 m of ND HNLF and 56 cm of PM1550 as shown in Fig. \ref{cascaded eom}(c). Using this all-fiber design, we are able to predict a compression factor of $\sim$ 38 which is even greater than that achieved in nonlinear nanophotonic waveguides \cite{Oliver2021Soliton-effectChip}. 


\begin{figure}[h!]
\centering\includegraphics[width=13.3cm]{beat.pdf}
\caption{Heterodyne beat between the optically filtered supercontinuum and a 1319 nm CW laser with (a) an Agilent synthesizer, and (b) a DRO as source of the initial 20 GHz signal. The linewidth and SNR of the beat signals are noted.}
\label{het}
\end{figure}
\section{Supercontinuum coherence}

We performed optical heterodyne measurements as an initial step to check the coherence of the 20 GHz supercontinuum obtained with multi-segment AD HNLF. Initially, the supercontinuum is passed through a bandpass filter centered at 1319 nm with a bandwidth of 12 nm. The heterodyne beat between the filtered 20 GHz comb modes and a stable 1319 nm CW laser (JDSU model: M126N-1319-350) on a photodiode is amplified and observed on the radio frequency spectrum analyzer (RFSA) as shown in Fig. \ref{het}. We also studied the impact that the choice of the RF oscillator had on the phase noise of the comb modes. The signal-to-noise ratio of the heterodyne beat increased from $\sim$ 33 dB (Fig. \ref{het}(a)) with an Agilent E8257N synthesizer to 37 dB  using a dielectric resonant oscillator (DRO) as RF source (Fig. \ref{het}(b)). Both signals were measured on an RFSA with a resolution bandwidth (RBW) of 100 kHz. It is also observed that the 3 dB linewidth of the beat signal reduces by more than a factor of two on using the DRO as an RF source instead of a synthesizer. These results indicate that we can improve the coherence of our supercontinuum to a large extent by using low-noise RF oscillators and phase-locked loops. In the future, we plan to extend this heterodyne beat measurement to frequencies further away from the pump and study in detail the impact of input amplitude noise, and of the built-in cavity's filtering of the broadband microwave thermal noise, on the coherence of the generated supercontinuum. 
%The measured high SNR shows that the all-fiber technique in this work provides a robust way to generate a low-noise coherent broadband supercontinuum from off-the-shelf components.

\bibliography{osa-supplemental-document-template}



\end{document}