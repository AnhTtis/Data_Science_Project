\section{Related Work}
\textbf{MIL for WSI classification.}
MIL-based methods~\cite{ABMIL,DSMIL,DTFD,CLAM} have gained popularity in WSI classification due to their high effectiveness. These methods typically involve using a feature extractor to extract features from image patches of WSI, followed by an aggregation step to obtain a feature representation at the WSI level. A lightweight classifier is then employed to predict the WSI category~\cite{ABMIL}.
In WSI classification, an effective feature extractor that generates representative feature is crucial for accurate classification results. 
However, existing MIL-based methods for WSI classification mainly adopt a pre-trained feature extractor without fine-tuning, which results in sub-optimal performance~\cite{hu2022predicting}. 
This is due to the domain shift and task discrepancies between the pre-training task (\eg, ImageNet) and the downstream task (\eg, histopathology)~\cite{stacke2020measuring}. 
For this issue, we introduce visual prompts  for WSI classification, enabling smooth feature modulation from the upstream dataset to the downstream WSI classification.

\noindent\textbf{Prompt Learning.}
Prompt learning has recently emerged as a lightweight and efficient transfer learning paradigm in NLP and has achieved remarkable success~\cite{jia2022visual}. 
The fundamental idea behind prompt learning is to freeze large-scale NLP models, such as BERT~\cite{devlin2018bert} and GPT-3~\cite{brown2020language}, that have been pre-trained on vast datasets and use task-specific prompts to adapt them to diverse downstream tasks without updating any parameters~\cite{bahng2022exploring}.
Building on the NLP prompt learning paradigm, several studies~\cite{luddecke2022image,nie2022pro,chen2022conv} have proposed to extend prompt learning to natural images in computer vision. 
For example, L{\"u}ddecke et al.~\cite{luddecke2022image} used text and image prompts to adapt the frozen pre-trained CLIP model \cite{radford2021learning} to new image segmentation tasks. 
However, the effectiveness of prompt learning in the field of histopathology analysis is under-investigated.
