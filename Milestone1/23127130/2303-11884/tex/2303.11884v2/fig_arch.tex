\begin{figure*}[t]
\centering
\vspace{-.25em}
    \begin{subfigure}{0.24\linewidth}
        \centering
        \includegraphics[width=0.9\linewidth]{figures/arch_gridpg.png}
        \caption{\pg}
        \label{fig:arch:pg}
    \end{subfigure}
    \rulesep
    \begin{subfigure}{0.31\linewidth}
        \centering
        \includegraphics[width=0.9\linewidth]{figures/arch_difull.png}
        \caption{\disc}
        \label{fig:arch:disc}
    \end{subfigure}
    \rulesep
    \begin{subfigure}{0.27\linewidth}
        \centering
        \includegraphics[width=0.9\linewidth]{figures/arch_dipart.png}
        \caption{\cs}
        \label{fig:arch:cs}
    \end{subfigure}
    \caption{
    \textbf{Our three evaluation settings.} In \pg, the classification scores are influenced by the entire input. In \disc, on the other hand, we explicitly control which inputs can influence the classification score. For this, we pass each subimage separately through the spatial layers, and then construct individual classification heads for each of the subimages. \cs serves as a more natural setting to \disc, that still provides partial control over information. We show a $1\times 2$ grid for readability, but the experiments use $2\times 2$ grids.
    }
    \label{fig:arch}
    \vspace{-.85em}
\end{figure*}