\IEEEraisesectionheading{\section{Introduction}}
\label{sec:intro}

\begin{figure*}[t]
\centering
\begin{center}
\small
\setlength{\fboxrule}{1pt}
\setlength{\tabcolsep}{2pt}
\begin{tabular}{ccc}
\centeredtab{\begin{tikzpicture}
 \foreach \X [count=\Z]in {fig/teaser/input/000000.jpg,fig/teaser/input/000194.jpg,fig/teaser/input/000395.jpg,fig/teaser/input/000600.jpg,fig/teaser/input/000799.jpg,fig/teaser/input/000989.jpg,fig/teaser/input/001081.jpg}
 {\node[opacity=1] at (\Z/5,-\Z/2.5,0) {\fbox{\includegraphics[width=3cm]{\X}}};}
\end{tikzpicture} \\
Input: casually captured long video \\
\\ \\ \\
\includegraphics[width=5cm]{fig/teaser/teaser_plot3.png} \\
Output: jointly estimated camera poses \\ and local radiance fields
} &
\setlength{\tabcolsep}{1pt}
\renewcommand{\arraystretch}{0.8}
\begin{tabular}{cccc}
{\includegraphics[width=3.9cm]{fig/teaser/000020.jpg}} & 
{\includegraphics[width=3.9cm]{fig/teaser/000530.jpg}} & 
{\includegraphics[width=3.9cm]{fig/teaser/001190.jpg}} \\
\multicolumn{3}{c}{LocalRF (ours): high-quality novel view synthesis} \\
\rule{0pt}{2.35cm}{\includegraphics[width=3.9cm]{fig/teaser/rgb_2.jpg}} & 
{\includegraphics[width=3.9cm]{fig/teaser/rgb_53.jpg}} & 
{\includegraphics[width=3.9cm]{fig/teaser/rgb_119.jpg}} \\
\multicolumn{3}{c}{BARF~\cite{lin2021barf}: the estimated poses often fall into local minima for long sequences} \\
\rule{0pt}{2.35cm}{\includegraphics[width=3.9cm]{fig/teaser/color_002.jpg}} & 
{\includegraphics[width=3.9cm]{fig/teaser/color_053.jpg}} & 
{\includegraphics[width=3.9cm]{fig/teaser/color_119.jpg}} \\
\multicolumn{3}{c}{Mip-NeRF360~\cite{barron2022mipnerf360}: the spatial resolution is often limited throughout the video}
\end{tabular}
\end{tabular}%
\captionof{figure}{
\label{fig:teaser}
\textbf{High-quality novel view synthesis from a long casually captured video.} 
We jointly optimize camera poses and a scene representation using a progressive scheme that dynamically allocates local radiance fields (blue boxes).
Our method robustly handles casual hand-held captures, scales to processing arbitrarily long videos with limited memory usage, and maintains high resolution throughout the entire video.
}
\end{center}
\end{figure*}

\IEEEPARstart{D}{eep} neural networks (DNNs) are highly successful on many computer vision tasks.
However, their black box nature makes it hard to interpret and thus trust their decisions. 
To shed light on the models' decision-making process, several methods have been proposed that aim to attribute importance values to individual input features (see \cref{sec:related}).
However, given the lack of ground truth importance values, it has proven difficult to compare and evaluate these \emph{attribution methods} in a holistic and systematic manner.

In this work that extends \cite{rao2022towards}, we take a three-pronged approach towards addressing this issue. In particular, we focus on three important components for such evaluations: reliably measuring the methods' \emph{model-faithfulness}, ensuring a \emph{fair comparison} between methods, and providing a framework that allows for \emph{systematic} visual inspections of their attributions. 

First, we propose an evaluation scheme (\textbf{DiFull}), which allows distinguishing possible from impossible importance attributions. This effectively provides ground truth annotations for whether or not an input feature can possibly have influenced the model output. As such, it can highlight distinct failure modes of attribution methods (\cref{fig:teaser}, left). 

Second, a fair evaluation requires attribution methods to be compared on equal footing. However, we observe that different methods explain DNNs to different depths (e.g., full network or classification head only). 
Thus, some methods in fact solve a much easier problem (i.e., explain a much shallower network). To even the playing field, we propose a multi-layer evaluation scheme for attributions ({\bf ML-Att}) and 
thoroughly evaluate
commonly used methods across multiple layers and models (\cref{fig:teaser}, left). 
When compared on the same level, we find that performance differences between some methods essentially vanish.

Third, relying on individual examples for a qualitative comparison is prone to skew the comparison and cannot fully represent the evaluated attribution methods. To overcome this, we propose a qualitative evaluation scheme for which we aggregate attribution maps ({\bf AggAtt}) across many input samples. This allows us to observe trends in the performance of attribution methods across complete datasets, in addition to looking at individual examples (\cref{fig:teaser}, right).

\myparagraph{Contributions.}
\textbf{(1)} We propose a novel evaluation setting, {\bf\disc}, 
in which we control which regions \emph{cannot possibly} influence a model's output, which allows us to highlight definite failure modes of attribution methods.
\textbf{(2)} 
We argue that methods can only be compared fairly when evaluated \emph{on the same layer}. To do this, we introduce {\bf ML-Att} and evaluate all attribution methods at multiple layers. 
We show that, when compared fairly, apparent performance differences between some methods effectively vanish. 
\textbf{(3)} We propose a novel aggregation method, {\bf\aggatt}, to qualitatively evaluate attribution methods across all images in a dataset. This allows to qualitatively assess a method's performance across many samples (\cref{fig:teaser}, right), which complements the evaluation on individual samples.
\textbf{(4)} We propose a post-processing smoothing step that significantly improves localization performance on some attribution methods. 
We observe significant differences when evaluating these smoothed attributions on different architectures, which highlights how architectural design choices can influence an attribution method's applicability.

In this extended version of \cite{rao2022towards}, we additionally provide the following:
\textbf{(1)} We evaluate on a wider variety of network architectures, in particular deeper networks with higher classification accuracies, including VGG19 \cite{simonyan2014very}, ResNet152 \cite{he2016deep}, ResNeXt \cite{xie2017aggregated}, Wide ResNet \cite{zagoruyko2016wide}, and GoogLeNet \cite{szegedy2015going}. We show that the results and trends discussed in \cite{rao2022towards} generalize well to diverse CNN architectures.
\textbf{(2)} We evaluate our settings on multiple configurations of the layer-wise relevance propagation (LRP) \cite{bach2015pixel} family of attribution methods, that modify the gradient flow during backpropagation to identify regions in the image important to the model. We show that while LRP can outperform all other methods, achieving good localization requires carefully choosing propagation rules and their parameters, and is sensitive to the model formulation and architecture.
\textbf{(3)} We show that the trends in performance of attribution methods at multiple layers (\mlatt), which was visualized at a subset of layers (input, middle, and final) in \cite{rao2022towards}, generalizes across layers and architectures for each method.
Our code is available at \href{https://github.com/sukrutrao/Attribution-Evaluation}{https://github.com/sukrutrao/Attribution-Evaluation}.



