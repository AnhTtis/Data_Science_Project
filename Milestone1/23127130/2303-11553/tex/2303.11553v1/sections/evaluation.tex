\section{Evaluation}

We perform three types of analysis to better understand the quantitative and qualitative characteristics of the DyVeRG model.
In the first quantitative benchmark, we task the model with distinguishing genuine temporal dynamics from realistic imposter data created by other generative models.
The second quantitative analysis asks all of the models, including DyVeRG, to generate a graph corresponding to a slice of time from the data; the generated graphs are then compared to the ground truth.
We conclude with a short qualitative analysis and interpretation of the temporal transitions DyVeRG induces between grammar rules.

\subsection{Datasets}


\begin{table}[t]
    \centering
    \caption{Summary of the datasets used in the evaluation.}\label{tab:datasets}
    \scalebox{0.6}{
    \begin{tabular}{l|ccccc}
        \hline
         &~node count~&~edge count~&~\# timestamps~&~\# interactions~&~\# snapshots~\\
        \hline
        % \texttt{email-enron} & \numprint{184} & \numprint{2216} & \numprint{22633} & \numprint{38172} & 31\\
        DNC~Emails & \numprint{1891} & \numprint{4465} & \numprint{19389} & \numprint{32878} & 11\\
        EU~Emails & \numprint{986} & \numprint{16064} & \numprint{207880} & \numprint{327333} & 19\\
        DBLP & \numprint{95391} & \numprint{164479} & \numprint{21} & \numprint{200792} & 21\\
        Facebook & \numprint{61096} & \numprint{614797} & \numprint{736674} & \numprint{788135} & 29\\
        \hline
    \end{tabular}
    }
\end{table}



In this evaluation, we consider four dynamic datasets, listed in \autoref{tab:datasets}.
DNC~Emails and EU~Emails are email networks where user email accounts are nodes and an email from one user to another at a given time is represented by an undirected edge labeled with a UNIX time.
Both of these datasets are aggregated by month; DNC~Emails contains a number of self-edge loops, while EU~Emails contains none.
The DBLP dataset is an undirected academic coauthorship graph where nodes correspond to researchers and an edge is drawn between two researchers during a particular year \emph{iff} they coauthor a paper during that year.
Finally, the Facebook dataset is an undirected graph tracking friendships on a monthly basis, with two users sharing an edge if they were friends during that month.

\begin{figure}[t]
    \centering
    % PTMTorrrent
\newcommand{\numberOfModelHub}{5\xspace}

\newcommand{\TotalNumberOfPackages}{{15,913}\xspace}
% 12401 from Hugging Face
% 185 from ONNX
% 33 from Model Hub
% 3245 from Model Zoo
% 49 from PyTorch Hub
% SUM (by Nick): 15,913

\newcommand{\HFNumberOfPackages}{{12,401}\xspace}
\newcommand{\HFNumberOfPackagesMetadata}{{124,427}\xspace}
\newcommand{\MZNumberOfPackages}{3,245\xspace}
\newcommand{\PHNumberOfPackages}{{49}\xspace}
\newcommand{\MHNumberOfPackages}{{33}\xspace}
\newcommand{\ONNXNumberOfPackages}{{185}\xspace}

\newcommand{\TotalDataSize}{\textasciitilde{61TB}\xspace}
\newcommand{\HFDataSize}{{61TB}\xspace}
\newcommand{\MZDataSize}{{115GB}\xspace}
\newcommand{\PHDataSize}{{1.5GB}\xspace}
\newcommand{\MHDataSize}{{721MB}\xspace}
\newcommand{\ONNXDataSize}{{441MB}\xspace}
%%%



% ICSE submission - HFTorrent v1

\newcommand{\PTMDatasetNPackages}{63,182\xspace}
\newcommand{\PTMDatasetPercentage}{{99.7\%}\xspace}
\newcommand{\PTMDatasetFailedPackages}{{186}\xspace}
\newcommand{\PTMDatasetFailedPercentage}{{0.3\%}\xspace}

\newcommand{\PTMDatasetNReposWithSignedCommits}{{132}\xspace}
\newcommand{\PTMDatasetPercentOfSignedCommits}{{0.208\%}\xspace}


\newcommand{\PercentOfVerifiedOrgs}{{3.188\%}\xspace}
\newcommand{\NOrganizations}{{6,243}\xspace}
\newcommand{\NVerifedOrgs}{{199}\xspace}

\newcommand{\NOfRepositoriesWithMalware}{{1}\xspace}
\newcommand{\PercentageOfRepositoriesWithMalware}{{0.002\%}\xspace}
\newcommand{\TotalRepositoriesForMalwareScanning}{{63,366}\xspace}
    \vspace{-5ex}
    \caption{Number of nodes and edges in the datasets over time.}
    \label{fig:data}
\end{figure}

We take snapshots $0$ through $10$ for each dataset.
Because these datasets are dynamic, we summarize their orders and sizes in \autoref{fig:data},
noting they tend to grow over time.

\subsection{Baselines}

We compare the DyVeRG model against 5 baselines.
The Erd\H{o}s-Renyi model generates random graphs of a fixed size $n$ with probability $p$ of an edge between any two nodes~\cite{erdos1959random}; for evaluation we set $n$ and $p = \sfrac{2m}{n(n - 1)}$ to the ground-truth values within each timestep.
The configuration model of Chung and Lu generates a random graph approximating a given degree distribution~\cite{chunglu2006book,hagberg2008networkx}; for this baseline, we use the degree distribution from the dataset.

The Erd\H{o}s-Renyi and Configuration models learn very rudimentary features from an input graph.
The following three graph models are different in that they take a whole graph as input and use their own inductive biases to learn features.
The Stochastic Block Model (SBM) uses matrix reductions to represent graphs with structured communities~\cite{holland1982sbm,peixoto2014graphtool}.
Likewise, the more advanced graph recurrent neural network (GraphRNN)~\cite{you2018graphrnn} is able to learn a generative model from an input collection of graphs by adapting walks over nodes as sequential data.
We also provide a static implementation of DyVeRG based on CNRG~\cite{sikdar2019cnrg}, which we call VeRG, as a final point of comparison.

An important note should be made here that some of the data for GraphRNN is missing from the figures.
This is because, when training and testing the model on the two NVIDIA GeForce RTX 2080 Ti cards available to us, with 10~GB of RAM each, we regularly ran out of memory on the larger datasets.

\subsection{Inference}

\begin{figure}[t]
    \centering
    \input{plots/llEMAILDNC}
% \input{plots/llEMAILENRON}
\input{plots/llEMAILEUCORE}
\input{plots/llCOAUTHDBLP}
\input{plots/llFACEBOOKLINKS}


\pgfplotsset{
    every axis plot/.append style={line width=1.5pt},
    every axis title/.style={below left,at={(0.99,0.99)}},
    every axis title/.append style={font=\scriptsize},
    every tick label/.append style={font=\scriptsize},
    % every axis label/.append style={font=\scriptsize}
    % every axis xlabel/.append style={font=\scriptsize}
    every axis ylabel/.append style={font=\scriptsize}
}

\begin{tikzpicture}

\begin{groupplot}[
        group style={
            group name=dyverggroup,
            group size=2 by 3,
            vertical sep=1.7cm, 
            horizontal sep=0.75cm, 
            ylabels at=edge left,
            xlabels at=edge bottom
        },
        height=4cm,
        width=\linewidth/1.75,
        ylabel={\footnotesize Dyvergence},
        xmin=0,
        xmax=11,
        %ymax=11.0,
        %ymin=2,
        xtick align=inside,
        ytick align=inside,
        % ymajorticks = false,
        xmajorticks = false,
        major tick length=0.5ex,
        % legend style={at={(-0.1,-0.5)}, anchor=north, font=\sffamily\small, draw=none},
        legend style={at={(-0.125,-0.8)}, anchor=north, draw=none},
        legend columns=3,
        legend image post style={mark options={thick, scale=1}},
    ]
    

    \nextgroupplot[title={DNC Emails}, mark=.]%, ymax=6.5, ymin=1.5, xmajorticks=false,]
        \addplot [color=ploter]
                  table [x=ts, y=realism] {\llEMAILDNCer};
        \addplot [color=plotcl]
                  table [x=ts, y=realism] {\llEMAILDNCcl};
        \addplot [color=plotsbm]
                  table [x=ts, y=realism] {\llEMAILDNCsbm};
        \addplot [color=plotgraphrnn]
                  table [x=ts, y=realism] {\llEMAILDNCgraphrnn};
        \addplot [color=plotverg]
                  table [x=ts, y=realism] {\llEMAILDNCverg};
        \addplot [color=plotdyverg]
                  table [x=ts, y=realism] {\llEMAILDNCdyverg};

    \nextgroupplot[title={EU Emails}, mark=., ymax=1.25, ymin=-0.1, xmajorticks=false,]
        \addplot [color=ploter]
                  table [x=ts, y=realism] {\llEMAILEUCOREer};
        \addplot [color=plotcl]
                  table [x=ts, y=realism] {\llEMAILEUCOREcl};
        \addplot [color=plotsbm]
                  table [x=ts, y=realism] {\llEMAILEUCOREsbm};
        \addplot [color=plotgraphrnn]
                  table [x=ts, y=realism] {\llEMAILEUCOREgraphrnn};
        \addplot [color=plotverg]
                  table [x=ts, y=realism] {\llEMAILEUCOREverg};
        \addplot [color=plotdyverg]
                  table [x=ts, y=realism] {\llEMAILEUCOREdyverg};

    \nextgroupplot[title={DBLP}, mark=., ymax=0.9]
        \addplot [color=ploter]
                  table [x=ts, y=realism] {\llCOAUTHDBLPer};
        \addplot [color=plotcl]
                  table [x=ts, y=realism] {\llCOAUTHDBLPcl};
        \addplot [color=plotsbm]
                  table [x=ts, y=realism] {\llCOAUTHDBLPsbm};
        \addplot [color=plotgraphrnn]
                  table [x=ts, y=realism] {\llCOAUTHDBLPgraphrnn};
        \addplot [color=plotverg]
                  table [x=ts, y=realism] {\llCOAUTHDBLPverg};
        \addplot [color=plotdyverg]
                  table [x=ts, y=realism] {\llCOAUTHDBLPdyverg};

    \nextgroupplot[title={Facebook}, mark=., ymax=0.28, ymin=-0.01]
        \addplot [color=ploter]
                  table [x=ts, y=realism] {\llFACEBOOKLINKSer};
        \addplot [color=plotcl]
                  table [x=ts, y=realism] {\llFACEBOOKLINKScl};
        \addplot [color=plotsbm]
                  table [x=ts, y=realism] {\llFACEBOOKLINKSsbm};
        \addplot [color=plotgraphrnn]
                  table [x=ts, y=realism] {\llFACEBOOKLINKSgraphrnn};
        \addplot [color=plotverg]
                  table [x=ts, y=realism] {\llFACEBOOKLINKSverg};
        \addplot [color=plotdyverg]
                  table [x=ts, y=realism] {\llFACEBOOKLINKSdyverg};

\legend{Erdos-Reny\'i~~,Chung-Lu~~,SBM~~,GraphRNN~~,VeRG~~,Ground~Truth}

\end{groupplot}


\begin{axis}[
    xmin = 0, xmax = 11,
    ymin = 0.2, ymax = 7,
    at ={(dyverggroup c1r1.south)},
    xtick={1, 2, 3, 4, 5, 6, 7, 8, 9, 10},
    ytick={1, 3, 5},
    anchor=north, % anchor
    xlabel={},
    ylabel={\footnotesize Ranking},
    height=2.7cm,
    xtick align=inside,
    ytick align=inside,
    major tick length=0.5ex,
    width=\linewidth/1.75,
    ]
    %\addplot [color=ploter, only marks, mark=-]
                  %table [x=ts, y=realrank] {\llEMAILDNCer};
    \addplot [color=ploter, only marks, mark=-]
              table [x=ts, y=realrank] {\llEMAILDNCer};
    \addplot [color=plotcl, only marks, mark=-]
              table [x=ts, y=realrank] {\llEMAILDNCcl};
    \addplot [color=plotsbm, only marks, mark=-]
              table [x=ts, y=realrank] {\llEMAILDNCsbm};
    \addplot [color=plotgraphrnn, only marks, mark=-]
              table [x=ts, y=realrank] {\llEMAILDNCgraphrnn};
    \addplot [color=plotverg, only marks, mark=-]
              table [x=ts, y=realrank] {\llEMAILDNCverg};
    \addplot [color=plotdyverg, only marks, mark=-]
              table [x=ts, y=realrank] {\llEMAILDNCdyverg};                  
  
\end{axis}



\begin{axis}[
    xmin = 0, xmax = 11,
    ymin = 0.2, ymax = 7,
    at ={(dyverggroup c2r1.south)},
    xtick={1, 2, 3, 4, 5, 6, 7, 8, 9, 10},
    ytick={1, 3, 5},
    anchor=north, % anchor
    xlabel={},
    ylabel={},
    height=2.7cm,
    xtick align=inside,
    ytick align=inside,
    major tick length=0.5ex,
    width=\linewidth/1.75,
    ]
    \addplot [color=ploter, only marks, mark=-]
              table [x=ts, y=realrank] {\llEMAILEUCOREer};
    \addplot [color=plotcl, only marks, mark=-]
              table [x=ts, y=realrank] {\llEMAILEUCOREcl};
    \addplot [color=plotsbm, only marks, mark=-]
              table [x=ts, y=realrank] {\llEMAILEUCOREsbm};
    \addplot [color=plotgraphrnn, only marks, mark=-]
              table [x=ts, y=realrank] {\llEMAILEUCOREgraphrnn};
    \addplot [color=plotverg, only marks, mark=-]
              table [x=ts, y=realrank] {\llEMAILEUCOREverg};
    \addplot [color=plotdyverg, only marks, mark=-]
              table [x=ts, y=realrank] {\llEMAILEUCOREdyverg};  
  
\end{axis}


\begin{axis}[
    xmin = 0, xmax = 11,
    ymin = 0.2, ymax = 7,
    at ={(dyverggroup c1r2.south)},
    xtick={1, 2, 3, 4, 5, 6, 7, 8, 9, 10},
    ytick={1, 3, 5},
    anchor=north, % anchor
    xlabel={$t$},
    ylabel={\footnotesize Ranking},
    height=2.7cm,
    xtick align=inside,
    ytick align=inside,
    ylabel style={yshift=0.18cm},
    major tick length=0.5ex,
    width=\linewidth/1.75,
    ]
    \addplot [color=ploter, only marks, mark=-]
              table [x=ts, y=realrank] {\llCOAUTHDBLPer};
    \addplot [color=plotcl, only marks, mark=-]
              table [x=ts, y=realrank] {\llCOAUTHDBLPcl};
    \addplot [color=plotsbm, only marks, mark=-]
              table [x=ts, y=realrank] {\llCOAUTHDBLPsbm};
    \addplot [color=plotgraphrnn, only marks, mark=-]
              table [x=ts, y=realrank] {\llCOAUTHDBLPgraphrnn};
    \addplot [color=plotverg, only marks, mark=-]
              table [x=ts, y=realrank] {\llCOAUTHDBLPverg};
    \addplot [color=plotdyverg, only marks, mark=-]
              table [x=ts, y=realrank] {\llCOAUTHDBLPdyverg};  
  
\end{axis}

\begin{axis}[
    xmin = 0, xmax = 11,
    ymin = 0.2, ymax = 7,
    at ={(dyverggroup c2r2.south)},
    xtick={1, 2, 3, 4, 5, 6, 7, 8, 9, 10},
    ytick={1, 3, 5},
    anchor=north, % anchor
    xlabel={$t$},
    ylabel={},
    height=2.7cm,
    xtick align=inside,
    ytick align=inside,
    major tick length=0.5ex,
    width=\linewidth/1.75,
    ]
    \addplot [color=ploter, only marks, mark=-]
              table [x=ts, y=realrank] {\llFACEBOOKLINKSer};
    \addplot [color=plotcl, only marks, mark=-]
              table [x=ts, y=realrank] {\llFACEBOOKLINKScl};
    \addplot [color=plotsbm, only marks, mark=-]
              table [x=ts, y=realrank] {\llFACEBOOKLINKSsbm};
    \addplot [color=plotgraphrnn, only marks, mark=-]
              table [x=ts, y=realrank] {\llFACEBOOKLINKSgraphrnn};
    \addplot [color=plotverg, only marks, mark=-]
              table [x=ts, y=realrank] {\llFACEBOOKLINKSverg};
    \addplot [color=plotdyverg, only marks, mark=-]
              table [x=ts, y=realrank] {\llFACEBOOKLINKSdyverg};  
  
\end{axis}

\end{tikzpicture}


    \vspace{-5ex}
    \caption{Dyvergence scores and model rankings. The top subplots show, for each model, the deviations over time from the mean dyvergence score.
    The relative rankings of the models are then shown in the corresponding bottom subplots.
    Note that lower is better, so the best-performing model is listed at the bottom.}
    \label{fig:dyvergence}
\end{figure}

The goal of this task, given a temporal sequence of graphs $\langle G_t \rangle_{t = 0}^{10}$, is to distinguish the graph that genuinely comes next from an assortment of impostors.

% \textbf{JUSTUS MODIFIED VERSION STARTS HERE}
For each timestep $t \in \{0, \dots 9\}$, we extract a VRG from $G_t$ and update it using $G_{t + 1}$, yielding DyVeRG grammars $\langle \mathcal{G}_t \rangle_{t = 1}^{10}$.
These grammars are used to compute dyvergence scores (\textit{cf.} \autoref{sec:dyvergence}) for the ground truth. 
This is performed $10$ times independently for each $(G_t, G_{t + 1})$ pair, and we let $D_t$ denote the mean.% of the 10 dyvergences for this pair.

We use the average ground-truth dyvergences $\{D_0, \dots D_{t - 1}\}$ to compute an estimate $\hat{D}_t$ for the expected dyvergence of the next graph pair $(G_t, G_{t + 1})$---\textit{i.e.}, an estimate for $D_t$.
Specifically,
we let $A_t = \sfrac{(\sum_{i = 0}^{t} D_i)}{(t + 1)}$ and compute
\begin{equation}
    \hat{D}_t = A_{t - 1} + (D_{t - 1} - A_{t - 2}).
\end{equation}
% Given this estimate, we can now calculate a realism score for $G_{t + 1}$ and the graphs generated by the imposter models.

% We now explain how to compute the \emph{realism} of a proposed graph.
Separately, each impostor model $\mathcal{M}$ is trained on $G_{t + 1}$ and $10$ graphs $\langle M_{t + 1, i} \rangle_{i = 1}^{10}$ are sampled from its distribution.
Dyvergence scores are calculated for these graphs by extracting a VRG from $G_t$ and then updating it with each of the $M_{t + 1, i}$; aggregate edits are then computed as in \autoref{eq:dyvergence}.
Average dyvergences $D_{\mathcal{M},t}$ are then found for the $(G_t, M_{t + 1, i})$.
We define the \emph{dyvergence} of $\mathcal{M}_t$ by
\begin{equation}
\text{dyvergence}(G_t, \mathcal{M}_t = |\hat{D}_t - D_{\mathcal{M}, t}|)
\end{equation}
Dyvergence for the ground truth is similarly defined by $\text{dyvergence}(G_t, G_{t + 1}) = |\hat{D}_t - D_t|$.
The lower this score is, the higher our confidence would be that the scored graph comes from the same generating distribution as the data.

We illustrate our results in \autoref{fig:dyvergence}.
Here, we determine success by assigning the ground truth a lower dyvergence score than the impostor graphs.
We outperform the competing baselines on the EU~Emails and Facebook datasets.

Our model is also largely successful on the DNC email graph, ranking the ground truth as least-dyvergent the majority of the time, shown clearly by the ranking subfigures in \autoref{fig:dyvergence}.
The model performs poorly only on the DBLP graph.
We conjecture that the amount of dyvergance in DBLP from one time step to another fluctuates more drastically due to the longer timescale for data aggregation in this dataset; whereas the other three datasets were grouped into monthly snapshots, DBLP snapshots are taken annually.
This might lead to inaccuracies in $\hat{D}_t$, negatively impacting the dyvergence scores for the real graph while boosting performance on imposters that are not as temporally turbulent.

% \textbf{AND ENDS HERE}

\iffalse

%For each timestep $t \in \{0, \dots 9\}$, we extract a VRG from $G_t$ and update it using $G_{t + 1}$, yielding DyVeRG grammars $\langle \mathcal{G}_t \rangle_{t = 1}^{10}$, which we use to compute a dyvergence score (\textit{cf.} \autoref{sec:dyvergence}) for the ground truth.
%This is performed $10$ times independently for each $(G_t, G_{t + 1})$ in the given dataset.
%Each impostor model $\mathcal{M}$ is then trained on each $G_{t + 1}$ and $10$ graphs $\langle M_{t + 1, i} \rangle_{i = 1}^{10}$ are sampled from its distribution.
%Dyvergence scores are calculated for these graphs by extracting a VRG from $G_t$ and then updating it with each of the $M_{t + 1, i}$ and computing the aggregate edits as in \autoref{eq:dyvergence}.

%The results of the inference task with dyvergence scores are illustrated in \autoref{fig:dyvergence}.
%Here, we can gauge success in one of two ways: ideally, the ground truth should be assigned the lowest dyvergence score among all graphs, but we can alternatively settle for being able to consistently distinguish between graphs across time by finding stable temporal rankings.
%The DNC~Emails dataset sees particularly poor performance, as none of the graphs are readily distinguishable, primarily due to its extremely small size.
%However, we outperform baselines on the EU~Emails dataset, consistently picking out the ground truth by a sizable margin against the impostors.

%The remaining two datasets are noticeably larger than the prior two, and performance degrades predictably as time increases.
%On DBLP, our model is able to marginally discern the ground truth from the others, but after around the fifth timestep this is no longer the case.
%Despite this, the graph rankings remain relatively stable over time.
%Finally, our model is completely confused on the Facebook dataset, believing that the ground truth is actually least likely to be the genuine graph.
%However, interestingly, the rankings across time are very stable for this largest dataset, indicating that our model is still finding some salient differences between the graphs.
%Each data point presented in \autoref{fig:dyvergence} is the mean across 10 independent trials.

\fi

\begin{figure}[t]
    \centering
    \input{plots/pdEMAILDNC}
% \input{plots/pdEMAILENRON}
\input{plots/pdEMAILEUCORE}
\input{plots/pdCOAUTHDBLP}
\input{plots/pdFACEBOOKLINKS}


\pgfplotsset{
    every axis plot/.append style={line width=1.5pt},
    every axis title/.style={below left,at={(0.99,0.99)}},
    every axis title/.append style={font=\scriptsize},
    every tick label/.append style={font=\scriptsize},
    % every axis label/.append style={font=\scriptsize}
    % every axis xlabel/.append style={font=\scriptsize}
    every axis ylabel/.append style={font=\scriptsize}
}

\begin{tikzpicture}

\begin{groupplot}[
        group style={
            group size=2 by 2,
            vertical sep=0.6cm, 
            horizontal sep=0.75cm, 
            ylabels at=edge left,
            xlabels at=edge bottom
        },
        height=4cm,
        width=\linewidth/1.75,
        xtick={1, 2, 3, 4, 5, 6, 7, 8, 9, 10},
        xlabel={$t$},
        ylabel={\footnotesize Portrait Divergence},
        xmin=0,
        xmax=11,
        ymin=-0.1,
        ymax=1.25,
        xtick align=inside,
        ytick align=inside,
        major tick length=0.5ex,
        % legend style={at={(-0.1,-0.5)}, anchor=north, font=\sffamily\small, draw=none},
        legend style={at={(-0.2,-0.5)}, anchor=north, draw=none},
        legend columns=3,
        legend image post style={mark options={thick, scale=1}},
    ]
    

    \nextgroupplot[title={DNC Emails}, mark=.]
        \addplot [color=ploter]
                  table [x=ts, y=avg] {\pdEMAILDNCer};
        \addplot [color=plotcl]
                  table [x=ts, y=avg] {\pdEMAILDNCcl};
        \addplot [color=plotsbm]
                  table [x=ts, y=avg] {\pdEMAILDNCsbm};
        \addplot [color=plotgraphrnn]
                  table [x=ts, y=avg] {\pdEMAILDNCgraphrnn};
        \addplot [color=plotverg]
                  table [x=ts, y=avg] {\pdEMAILDNCverg};
        \addplot [color=plotdyverg]
                  table [x=ts, y=avg] {\pdEMAILDNCdyverg};

    \nextgroupplot[title={EU Emails}, mark=.]
        \addplot [color=ploter]
                  table [x=ts, y=avg] {\pdEMAILEUCOREer};
        \addplot [color=plotcl]
                  table [x=ts, y=avg] {\pdEMAILEUCOREcl};
        \addplot [color=plotsbm]
                  table [x=ts, y=avg] {\pdEMAILEUCOREsbm};
        \addplot [color=plotgraphrnn]
                  table [x=ts, y=avg] {\pdEMAILEUCOREgraphrnn};
        \addplot [color=plotverg]
                  table [x=ts, y=avg] {\pdEMAILEUCOREverg};
        \addplot [color=plotdyverg]
                  table [x=ts, y=avg] {\pdEMAILEUCOREdyverg};

    \nextgroupplot[title={DBLP}, mark=.]
        \addplot [color=ploter]
                  table [x=ts, y=avg] {\pdCOAUTHDBLPer};
        \addplot [color=plotcl]
                  table [x=ts, y=avg] {\pdCOAUTHDBLPcl};
        \addplot [color=plotsbm]
                  table [x=ts, y=avg] {\pdCOAUTHDBLPsbm};
        \addplot [color=plotgraphrnn]
                  table [x=ts, y=avg] {\pdCOAUTHDBLPgraphrnn};
        \addplot [color=plotverg]
                  table [x=ts, y=avg] {\pdCOAUTHDBLPverg};
        \addplot [color=plotdyverg]
                  table [x=ts, y=avg] {\pdCOAUTHDBLPdyverg};

%HIDAN!
    \nextgroupplot[title={Facebook}, mark=.]
        \addplot [color=ploter]
                  table [x=ts, y=avg] {\pdFACEBOOKLINKSer};
        \addplot [color=plotcl]
                  table [x=ts, y=avg] {\pdFACEBOOKLINKScl};
        \addplot [color=plotsbm]
                  table [x=ts, y=avg] {\pdFACEBOOKLINKSsbm};
        \addplot [color=plotgraphrnn]
                  table [x=ts, y=avg] {\pdFACEBOOKLINKSgraphrnn};
        \addplot [color=plotverg]
                  table [x=ts, y=avg] {\pdFACEBOOKLINKSverg};
        \addplot [color=plotdyverg]
                  table [x=ts, y=avg] {\pdFACEBOOKLINKSdyverg};

\legend{Erdos-Reny\'i~~,Chung-Lu~~,SBM~~,GraphRNN~~,VeRG~~,DyVeRG}

\end{groupplot}
\end{tikzpicture}


    \vspace{-5ex}
    \caption{Portrait Divergence comparing a generated graph from each model and timestep against a corresponding ground truth graph. Lower is better.}
    \label{fig:portraitdivergence}
\end{figure}

\subsection{Generation}
A natural way to interrogate a generative graph model---like a graph grammar---is to generate graphs with it.
Generative graph models are widely used in modern AI systems for contrastive and adversarial learning.
Here, we use these models in the more traditional way they might be used for a task like hypothesis-testing; we fit the models, generate a graph at a particular time, and then compare the generated graph with the ground truth.
% For each baseline model, we train on the ground-truth at time $t$ and then generate a graph corresponding to the same timestep.
For each baseline model, we train on the ground truth at time $t$ and then generate at this same time.
If the two graphs are similar according to some empirical measure of graph similarity, then we would say that the model performed well.
% For DyVeRG, we train on time $t - 1$, update with time $t$, and then generate targeting time $t$.
For DyVeRG, we train on time $t - 1$, update with time $t$, and then generate at time $t$.
% Although it is possible for the DyVeRG model to train on all timesteps prior to $t$

Comparing two (or more) graphs is a nontrivial task since the distributions from which graphs can be sampled can behave erratically and are often very high-dimensional.
The most natural way to determine similarity between two graphs is by an isomorphism test; however, in addition to being computationally intractable, this provides a far-too-narrow view of graph similarity.
We instead take two alternative views to graph similarity.
Graph portrait divergence~\cite{bagrow2019portrait} provides a holistic view of a graph based on a matrix of random-walk counts sorted by length; these results will be averaged across 10 independent trials.
Maximum mean discrepancy (MMD)~\cite{gretton2012kernel} is a kernel-based sampling test---which will thus not require any averaging---with desirable stability and computational efficiency characteristics.
For both of these, lower is better.

\begin{figure}[t]
    \centering
    \input{plots/spectrumEMAILDNC}
% \input{plots/spectrumEMAILENRON}
\input{plots/spectrumEMAILEUCORE}
\input{plots/spectrumCOAUTHDBLP}
\input{plots/spectrumFACEBOOKLINKS}


\pgfplotsset{
    every axis plot/.append style={line width=1.5pt},
    every axis title/.style={below left,at={(0.99,0.99)}},
    every axis title/.append style={font=\scriptsize},
    every tick label/.append style={font=\scriptsize},
    % every axis label/.append style={font=\scriptsize}
    % every axis xlabel/.append style={font=\scriptsize}
    every axis ylabel/.append style={font=\scriptsize}
}

\begin{tikzpicture}

\begin{groupplot}[
        group style={
            group size=2 by 2,
            vertical sep=0.6cm, 
            horizontal sep=0.75cm, 
            ylabels at=edge left,
            xlabels at=edge bottom
        },
        height=4cm,
        width=\linewidth/1.75,
        xtick={1, 2, 3, 4, 5, 6, 7, 8, 9, 10},
        xlabel={$t$},
        ylabel={\footnotesize Spectrum MMD},
        xmin=0,
        xmax=11,
        ymin=-0.2,
        ymax=2.3,
        xtick align=inside,
        ytick align=inside,
        major tick length=0.5ex,
        % legend style={at={(-0.1,-0.5)}, anchor=north, font=\sffamily\small, draw=none},
        legend style={at={(-0.2,-0.5)}, anchor=north, draw=none},
        legend columns=3,
        legend image post style={mark options={thick, scale=1}},
    ]
    

    \nextgroupplot[title={DNC Emails}, mark=.]
        \addplot [color=ploter]
                  table [x=ts, y=mmd] {\spectrumEMAILDNCer};
        \addplot [color=plotcl]
                  table [x=ts, y=mmd] {\spectrumEMAILDNCcl};
        \addplot [color=plotsbm]
                  table [x=ts, y=mmd] {\spectrumEMAILDNCsbm};
        \addplot [color=plotgraphrnn]
                  table [x=ts, y=mmd] {\spectrumEMAILDNCgraphrnn};
        \addplot [color=plotverg]
                  table [x=ts, y=mmd] {\spectrumEMAILDNCverg};
        \addplot [color=plotdyverg]
                  table [x=ts, y=mmd] {\spectrumEMAILDNCdyverg};

    \nextgroupplot[title={EU Emails}, mark=.]
        \addplot [color=ploter]
                  table [x=ts, y=mmd] {\spectrumEMAILEUCOREer};
        \addplot [color=plotcl]
                  table [x=ts, y=mmd] {\spectrumEMAILEUCOREcl};
        \addplot [color=plotsbm]
                  table [x=ts, y=mmd] {\spectrumEMAILEUCOREsbm};
        \addplot [color=plotgraphrnn]
                  table [x=ts, y=mmd] {\spectrumEMAILEUCOREgraphrnn};
        \addplot [color=plotverg]
                  table [x=ts, y=mmd] {\spectrumEMAILEUCOREverg};
        \addplot [color=plotdyverg]
                  table [x=ts, y=mmd] {\spectrumEMAILEUCOREdyverg};

    \nextgroupplot[title={DBLP}, mark=.]
        \addplot [color=ploter]
                  table [x=ts, y=mmd] {\spectrumCOAUTHDBLPer};
        \addplot [color=plotcl]
                  table [x=ts, y=mmd] {\spectrumCOAUTHDBLPcl};
        \addplot [color=plotsbm]
                  table [x=ts, y=mmd] {\spectrumCOAUTHDBLPsbm};
        \addplot [color=plotgraphrnn]
                  table [x=ts, y=mmd] {\spectrumCOAUTHDBLPgraphrnn};
        \addplot [color=plotverg]
                  table [x=ts, y=mmd] {\spectrumCOAUTHDBLPverg};
        \addplot [color=plotdyverg]
                  table [x=ts, y=mmd] {\spectrumCOAUTHDBLPdyverg};

    \nextgroupplot[title={Facebook}, mark=.]
        \addplot [color=ploter]
                  table [x=ts, y=mmd] {\spectrumFACEBOOKLINKSer};
        \addplot [color=plotcl]
                  table [x=ts, y=mmd] {\spectrumFACEBOOKLINKScl};
        \addplot [color=plotsbm]
                  table [x=ts, y=mmd] {\spectrumFACEBOOKLINKSsbm};
        \addplot [color=plotgraphrnn]
                  table [x=ts, y=mmd] {\spectrumFACEBOOKLINKSgraphrnn};
        \addplot [color=plotverg]
                  table [x=ts, y=mmd] {\spectrumFACEBOOKLINKSverg};
        \addplot [color=plotdyverg]
                  table [x=ts, y=mmd] {\spectrumFACEBOOKLINKSdyverg};

\legend{Erdos-Reny\'i~~,Chung-Lu~~,SBM~~,GraphRNN~~,VeRG~~,DyVeRG}

\end{groupplot}
\end{tikzpicture}


    \vspace{-5ex}
    \caption{The MMD of the eigenvalues (Spectrum) of a generated graph from each model and timestep compared a corresponding ground truth graph. Lower is better.}
    \label{fig:mmdspectrum}
\end{figure}

We begin with the Portrait Divergence results, shown in \autoref{fig:portraitdivergence}.
% As expected, the performance of the model degrades as time moves onward.
In general, we can see that the DyVeRG-generated graphs tend to have lower portrait divergence compared to the other models, thus outperforming them.

Next, we analyse the MMD of the eigenvalue spectra of the graphs' Laplacian matrices.
MMD values are bounded between 0 and 2, with a value of 0 indicating belief that the spectrum of the ground truth and the sample spectra of the generated graphs were certainly sampled from the same underlying distribution.
These results are shown in \autoref{fig:mmdspectrum}.
Here, we find that DyVeRG performs no worse than VeRG, its static counterpart, on three of the datasets.
However, on the DBLP dataset, DyVeRG performs worse than almost all of the other models, despite is static analogue VeRG outperforming every model.

%
%\begin{figure}
%    \centering
%    \input{plots/degreedistributionEMAILDNC}
% \input{plots/degreedistributionEMAILENRON}
\input{plots/degreedistributionEMAILEUCORE}
\input{plots/degreedistributionCOAUTHDBLP}
\input{plots/degreedistributionFACEBOOKLINKS}


\pgfplotsset{
    every axis plot/.append style={line width=1.5pt},
    every axis title/.style={below left,at={(0.99,0.99)}},
    every axis title/.append style={font=\scriptsize},
    every tick label/.append style={font=\scriptsize},
    % every axis label/.append style={font=\scriptsize}
    % every axis xlabel/.append style={font=\scriptsize}
    every axis ylabel/.append style={font=\scriptsize}
}

\begin{tikzpicture}

\begin{groupplot}[
        group style={
            group size=2 by 2,
            vertical sep=0.5cm, 
            ylabels at=edge left,
            xlabels at=edge bottom
        },
        height=4cm,
        width=\linewidth/2,
        xtick={1, 2, 3, 4, 5, 6, 7, 8, 9, 10},
        xlabel={$t$},
        ylabel={Deg.\ Distr.\ MMD},
        xmin=0,
        xmax=11,
        ymin=-0.2,
        ymax=2.3,
        xtick align=inside,
        ytick align=inside,
        major tick length=0.5ex,
        % legend style={at={(-0.1,-0.5)}, anchor=north, font=\sffamily\small, draw=none},
        legend style={at={(-0.2,-0.5)}, anchor=north, draw=none},
        legend columns=6,
        legend image post style={mark options={thick, scale=1}},
    ]
    

    \nextgroupplot[title={DNC Emails}, mark=.]
        \addplot [color=ploter]
                  table [x=ts, y=mmd] {\degreedistributionEMAILDNCer};
        \addplot [color=plotcl]
                  table [x=ts, y=mmd] {\degreedistributionEMAILDNCcl};
        \addplot [color=plotsbm]
                  table [x=ts, y=mmd] {\degreedistributionEMAILDNCsbm};
        \addplot [color=plotgraphrnn]
                  table [x=ts, y=mmd] {\degreedistributionEMAILDNCgraphrnn};
        \addplot [color=plotverg]
                  table [x=ts, y=mmd] {\degreedistributionEMAILDNCverg};
        \addplot [color=plotdyverg]
                  table [x=ts, y=mmd] {\degreedistributionEMAILDNCdyverg};

    \nextgroupplot[title={EU Emails}, mark=.]
        \addplot [color=ploter]
                  table [x=ts, y=mmd] {\degreedistributionEMAILEUCOREer};
        \addplot [color=plotcl]
                  table [x=ts, y=mmd] {\degreedistributionEMAILEUCOREcl};
        \addplot [color=plotsbm]
                  table [x=ts, y=mmd] {\degreedistributionEMAILEUCOREsbm};
        \addplot [color=plotgraphrnn]
                  table [x=ts, y=mmd] {\degreedistributionEMAILEUCOREgraphrnn};
        \addplot [color=plotverg]
                  table [x=ts, y=mmd] {\degreedistributionEMAILEUCOREverg};
        \addplot [color=plotdyverg]
                  table [x=ts, y=mmd] {\degreedistributionEMAILEUCOREdyverg};

    \nextgroupplot[title={DBLP}, mark=.]
        \addplot [color=ploter]
                  table [x=ts, y=mmd] {\degreedistributionCOAUTHDBLPer};
        \addplot [color=plotcl]
                  table [x=ts, y=mmd] {\degreedistributionCOAUTHDBLPcl};
        \addplot [color=plotsbm]
                  table [x=ts, y=mmd] {\degreedistributionCOAUTHDBLPsbm};
        \addplot [color=plotgraphrnn]
                  table [x=ts, y=mmd] {\degreedistributionCOAUTHDBLPgraphrnn};
        \addplot [color=plotverg]
                  table [x=ts, y=mmd] {\degreedistributionCOAUTHDBLPverg};
        \addplot [color=plotdyverg]
                  table [x=ts, y=mmd] {\degreedistributionCOAUTHDBLPdyverg};

    \nextgroupplot[title={Facebook}, mark=.]
        \addplot [color=ploter]
                  table [x=ts, y=mmd] {\degreedistributionFACEBOOKLINKSer};
        \addplot [color=plotcl]
                  table [x=ts, y=mmd] {\degreedistributionFACEBOOKLINKScl};
        \addplot [color=plotsbm]
                  table [x=ts, y=mmd] {\degreedistributionFACEBOOKLINKSsbm};
        \addplot [color=plotgraphrnn]
                  table [x=ts, y=mmd] {\degreedistributionFACEBOOKLINKSgraphrnn};
        \addplot [color=plotverg]
                  table [x=ts, y=mmd] {\degreedistributionFACEBOOKLINKSverg};
        \addplot [color=plotdyverg]
                  table [x=ts, y=mmd] {\degreedistributionFACEBOOKLINKSdyverg};

\legend{Erdos-Reny\'i~~,Chung-Lu~~,SBM~~,GraphRNN~~,VeRG~~,DyVeRG}

\end{groupplot}
\end{tikzpicture}


%    \caption{Lower is better.}
%    \label{fig:mmddegree}
%\end{figure}
%Likewise, the results of the MMD on the degree distribution are illustrated in %Fig.~\ref{fig:mmddegree}. Again we find that VeRG and DyVeRG tend to perform quite well. 
There are, of course, many additional metrics by which to compare these models, but the main power of DyVeRG comes from its ability to express graph dynamics in a human-interpretable way. 

\iffalse
\begin{figure}
    \centering
    \subsection{Interplay between particle clusters and near-wall coherent structures}
\label{sec:mechanism}
In this section, we show that modulating the skin-friction drag depends to a large extent on how particle clusters interact with near-wall coherent structures.

\begin{figure}
  \includegraphics[width=5in]{tikzgraphics/fig13.pdf}
  \caption{Isocontours of normalized particle volume fraction in a wall-normal plane showing the presence of clusters and the accumulation of particles near the walls. As in figure \ref{fig:isocontour_u}, the larger domain for $\Sto^+ = 30$ particles is truncated to the same size as the domain for $\Sto^+ = 6$ particles to facilitate visual comparison.}
  \label{fig:vfp}
\end{figure}
The distribution of $\Sto^+=6$ and $\Sto^+=30$ particles within the channel is strongly inhomogeneous.
Visualization of normalized particle volume fraction in a wall-normal plane in figure \ref{fig:vfp} shows that the particles concentrate in long filamentous clusters that may span the entire channel height.
$\Sto^+=30$ particles form clusters that are relatively denser and further elongated in the streamwise direction compared to clusters formed by $\Sto^+=6$ particles. Figure \ref{fig:vfp} also shows that the normalized particle volume fraction within the bulk of the channel is lower at mass loading $M=0.1$, compared to the bulk normalized volume fraction at $M=0.6$ and 1.0. This points to a tendency of particles to accumulate near the walls that is stronger at $M=0.1$  than at $M=0.6$ and $M=1$.  Note that the formation of such clusters is expected owing to the fact that the particles considered in this study have significant inertia. As previously discussed by several investigators, inertial particles in wall-bounded turbulent flows tend to form clusters due to two effects, namely, turbophoresis, i.e., the migration of inertial particles to lower turbulence regions near the walls \citep{caporaloniTransferParticlesNonisotropic1975,reeksTransportDiscreteParticles1983,nowbaharTurbophoresisAttenuationTurbulent2013,kuertenTurbulenceModificationHeat2011}, and preferential concentration, i.e., the migration of inertial particles from vortical regions to straining regions of the flow \citep{eatonPreferentialConcentrationParticles1994,marchioliStatisticsParticleDispersion2008,kasbaouiTurbulenceModulationSettling2019,fongVelocitySpatialDistribution2019}. It follows that the particle feedback force is concentrated along these structures, and that the resulting flow modulation depends largely on the cluster morphology and dynamics.

\begin{figure}\centering
  \begin{subfigure}{0.49\linewidth}
    \centering
    \includegraphics[width=3.2in]{plotting/fig14a.pdf}
    \caption{}
    \label{fig:pnd_st6}
  \end{subfigure}\hfill
  \begin{subfigure}{0.49\linewidth}
    \centering
    \includegraphics[width=3.2in]{plotting/fig14b.pdf}
    \caption{}
    \label{fig:pnd_st30}
	\end{subfigure}
  \caption{Particle number density normalized by the average particle number density as a function of the wall normal distance for (a) $\Sto^+ = 6$ and (b) $\Sto^+ = 30$ at various mass loadings. Symbols as in figure \ref{fig:uavg}.}
  \label{fig:PND}
\end{figure}

Although particle clusters can be observed throughout the channel, it is near the walls that the majority of particles accumulate. Figure \ref{fig:PND} shows the variation of the normalized plane-averaged volume fraction ${\langle \phi\rangle}/ \phi_0$ with the wall normal distance. Within the region $y^+ < 10$, the local particle volume fraction is several times larger than the mean volume fraction $\phi_0$, which shows that the majority of the particles accumulate near the walls. $\Sto^+=30$ particles lead to the largest wall accumulation reaching ${\langle \phi\rangle}/ \phi_0\simeq 4.98$ at $M=1$ compared to ${\langle \phi\rangle}/ \phi_0\simeq 2.62$ for $\Sto^+=6$ particles at the same mass loading. Similar observations were made by \citet{nilsenVoronoiAnalysisPreferential2013} and \citet{yuanThreedimensionalVoronoiAnalysis2018} who, despite considering only one-way coupling, found that particles with $\Sto^+=30$ have the greatest wall-accumulation among particles with $\Sto^+$ in the range 1-100. Interestingly, the particle wall accumulation reduces when mass loading increases. At $M=0.1$, the particle volume fraction at the wall rises to $\langle\phi\rangle/\phi_0\simeq 14.96$ and 6.7 for $\Sto^+=30$ and $\Sto^+=6$, respectively. This finding is in agreement with the observation from figure \ref{fig:vfp} that the relative bulk particle volume fraction is lowest at $M=0.1$ as relatively more particles accumulate at the walls with decreasing $M$. This effect likely results from two-way coupling, given that particle-particle collisions are weak in the present semi-dilute regime.

Here, we stress that capturing the particle ropes accurately and the subsequent flow modulation requires much larger domains than those generally used in simulations of particle-laden turbulent channel flows \citep{rousonPreferentialConcentrationSolid2001,zhaoTurbulenceModulationDrag2010a,bernardiniReynoldsNumberScaling2014,costaInterfaceresolvedSimulationsSmall2020,jieExistenceFormationMultiscale2022}. The present large domain used for simulations with $\Sto^+=30$ particles is sufficiently wide to allow a natural development of flow and particle structures in the spanwise direction. However, even with a streamwise length of $12\pi h\sim 38h$, the domain remains too short to properly characterize the average streamwise length of the particle ropes.

\begin{figure}
    \centering
  \begin{subfigure}{\linewidth}
    \centering
    \includegraphics[width=4.5in]{tikzgraphics/fig15a.pdf}
    \caption{}
    \label{fig:cluster_length_st30_2d1}
  \end{subfigure}
  \begin{subfigure}{\linewidth}
    \centering
    \includegraphics[width=4in]{tikzgraphics/fig15b.pdf}
    \caption{}
    \label{fig:cluster_length_st30_2d2}
  \end{subfigure}
  \begin{subfigure}{\linewidth}
    \centering
    \includegraphics[width=4in]{tikzgraphics/fig15c.pdf}
    \caption{}
    \label{fig:cluster_length_st6_2d1}
\end{subfigure}
  \caption{\textcolor{revision}{Isocontours of normalized particle volume fraction at $y^+ = 10$ for (a,b) $\Sto^+ = 30, M = 1.0$ and (c) $\Sto^+ = 6, M = 1.0$. The view in (b) corresponds to the area marked by the red rectangle in (a).}}
  \label{fig:cluster_length}
\end{figure}
With most of the particles concentrating near the walls, clusters found therein have the largest impact on the carrier flow. As shown in figure \ref{fig:cluster_length}, the topology of these structures varies significantly depending on whether the particles are drag-reducing ($\Sto^+=30$) or drag-increasing ($\Sto^+=6$). For better comparison of the scales, figure \ref{fig:cluster_length_st30_2d2} shows a view of the particle volume fraction field for $\Sto^+=30$ particles cropped to the same dimensions as the smaller domain used with $\Sto^+=6$ particles and shown in figure \ref{fig:cluster_length_st6_2d1}. In contrast with $\Sto^+=6$ particles, the higher inertia particles at $\Sto^+=30$ form distinctively long and stable clusters. These structures, which we call \emph{ropes}, span the entire length of the domain in the streamwise direction, i.e, over 6000 wall units. The ropes travel downstream but remain stable and coherent for dynamically significant times. Further, the ropes repeat periodically in the spanwise direction in a fashion reminiscent of low-speed streaks discussed in \S\ref{sec:unladen}. This suggests that formation of these ropes results from the interaction of particle clusters with coherent flow structures in the buffer layer. The fact that no such ropes are observed with $\Sto^+=6$ particles suggests that intermittent flow structures in the buffer layer are capable of breaking down clusters formed by low inertia particles, whereas clusters formed by particles with large inertia retain their spatial and temporal coherence. The stable particle ropes may in turn alter the near-wall coherent flow structures.

\begin{figure}
    \centering
    \begin{subfigure}{\linewidth}
      \centering
      \includegraphics[width=4.5in]{tikzgraphics/fig16a.pdf}
      \caption{$\Sto^+ =30, M = 1.0$}
      \label{fig:riblet_S30M10_large}
    \end{subfigure}
    \begin{subfigure}{\linewidth}
      \centering
      \includegraphics[width=4in]{tikzgraphics/fig16b.pdf}
      \caption{$\Sto^+ =30, M = 1.0$}
      \label{fig:riblet_S30M10}
    \end{subfigure}
    \begin{subfigure}{\linewidth}
    \centering
    \includegraphics[width=4in]{tikzgraphics/fig16c.pdf}
    \caption{$\Sto^+ = 6, M = 1.0$}
    \label{fig:riblet_S6M10}
  \end{subfigure}
  \caption{\textcolor{revision}{Overlay of the isocontours of fluid streamwise velocity, and the contour of the relative particle volume fraction $\phi/\phi_0 = 3$ at $y^+ = 10$ for (a,b) $\Sto=30$, $M=1.0$ and (c) $\Sto^+ = 6, M = 1.0$. The view in (b) corresponds to the area marked by the red rectangle in (a).}}
  \label{fig:riblet_comparison}
\end{figure}
In order to shed light on how particle ropes interact with near-wall coherent flow structures, we report in figure  \ref{fig:riblet_comparison} isocontours of streamwise velocity at $y^+ = 10$ with the iso-level $\phi=3\times\phi_0$ overlayed on top. The latter shows the regions where the particles cluster. For the flow laden with $\Sto^+ = 30$ particles at $M=1$, we observe that the long ropes align well with the low-speed streaks, showing that the dynamics of these two coherent structures are interlinked. Compared to the unladen flow (see figure \ref{fig:streak_SP_a}), the low-speed streaks are visibly further elongated in a way similar to how the particle ropes extend in the streamwise direction. The spanwise spacing of the low-speed streaks also increases and appears comparable to the spanwise spacing of the ropes. In the case of the flow laden with $\Sto^+=6$ particles at $M=1$, the clusters are also primarily found in the low-speed streaks. However, the streamwise length of these clusters is much shorter in comparison with the low-speed streaks and with the ropes formed by $\Sto^+ = 30$ particles. In addition, the streamwise length and spanwise spacing of low-speed streaks increase compared to the particle-free flow, although not to the same extent as with $\Sto^+=30$ particles.



\begin{figure}
	\centering
	\begin{subfigure}{0.45\linewidth}
		\includegraphics[width=\linewidth]{plotting/fig17a.pdf}
		\caption{\label{fig:spacing_st30_f}}
	\end{subfigure}
	\hfill
	\begin{subfigure}{0.45\linewidth}
		\includegraphics[width=\linewidth]{plotting/fig17b.pdf}
		\caption{\label{fig:spacing_st30_p}}
	\end{subfigure}
	\begin{subfigure}{0.45\linewidth}
		\includegraphics[width=\linewidth]{plotting/fig17c.pdf}
		\caption{\label{fig:spacing_st30_p}}
	\end{subfigure}
	\hfill
	\begin{subfigure}{0.45\linewidth}
		\includegraphics[width=\linewidth]{plotting/fig17d.pdf}
		\caption{\label{fig:spacing_st6_p}}
	\end{subfigure}
	\caption{Variation with spanwise spacing of the two-point autocorrelation of the streamwise fluid fluctuations and particle volume fraction fluctuations in the spanwise direction for the (a,b) drag-reducing case $\Sto^+=30$ (\textcolor{blue}{\protect\scalebox{1.25}{$\blacksquare$}}) and (c,d) drag increasing case $\Sto^+ = 6$ (\textcolor{red}{\protect\scalebox{1.75}{$\bullet$}}). Darker symbols correspond to larger mass loading which varies from 0.2 to 1.0. The solid black line represents the particle-free channel flow.
	\label{fig:spacing}}
\end{figure}
To characterize quantitatively the spanwise spacing of particle clusters and their impact on the spanwise spacing of low-speed streaks, we compute the two-point autocorrelation of the particle volume fraction fluctuations,
\begin{equation}
	    R^p_{\phi\phi}(\Delta z;y_0)=\frac{\langle{\phi' (x,y_0,z,t)\phi' (x,y_0,z+\Delta z,t)}\rangle}{\langle{\phi'^2}\rangle},
\end{equation}
and the the two-point autocorrelation of the streamwise velocity fluctuations $R^f_{uu}$. Figure \ref{fig:spacing} shows the variation $ R^p_{\phi\phi}$ and $ R^f_{\phi\phi}$ with spanwise spacing at $y^+=10$. Similar to how the low-speed streak spacing $\lambda_f^+$ is defined, we define $\lambda_p^+$, the spanwise spacing of particle clusters, as twice the distance between the origin and $\Delta z^+$ where $R^p_{\phi\phi}$ reaches a first minimum.

\begin{table}
  \caption{Spanwise spacing of the low-speed streaks and particle ropes. \label{tab:spacing}}
  \begin{ruledtabular}
    \begin{tabular}{llll}
      Stokes number ($\Sto^+$) & Mass loading ($M$) & $\lambda^+_f$ & $\lambda^+_p$ \\\hline
      (Particle-free) & 0 & 106 & $-$ \\
      6             & 0.2          & 125      & 99     \\
                    & 0.6          & 134      & 108    \\
                    & 1.0          & 116      & 90     \\
      30            & 0.2          & 126      & 108    \\
                    & 0.6          & 161      & 130    \\
                    & 1.0          & 170	  & 135    
    \end{tabular}
  \end{ruledtabular}
\end{table}

Table \ref{tab:spacing} shows the values of $\lambda^+_f$ and $\lambda^+_p$ for all cases simulated.
For the drag-reducing cases at $\Sto^+ = 30$, it is clear that as the mass loading is increased from 0.2 to 1.0 the low-speed streak spanwise spacing increases from $\lambda^+_f = 126$ to $170$. These are significant increases compared to the low-speed streak spacing of $\lambda^+_f = 106$ in the particle-free channel. The rope spacing $\lambda_p^+$ increases from $\lambda^+_p = 108$ to $135$ as mass loading is increased. 
\textcolor{revision}{The disparity between $\lambda_p^+$ and $\lambda_f^+$ is likely due to small particle clusters that detach from the main ropes due to the spanwise meandering of ropes and low-speed streaks.}
%
%\textcolor{revision}{We have also reported the difference between the spacing of low speed and particle streaks, i.e., $\lambda_f^+-\lambda_p^+$ in table \ref{tab:spacing}. With the drag-reducing $\Sto^+ = 30$  particles, $\lambda_f^+-\lambda_p^+$ increases from 18 wall units for $M=0.2$ to 35 wall units at $M=1.0$. This suggests that the particle ropes are able to push in and out of the low-speed streaks thanks to the large particle inertia.}
%In doing so, these particles may help suppress coherent structures and relaminarize the near-wall region.
In comparison, $\Sto^+ = 6$ particles lead to substantially lower modulation of the low-speed streaks. As shown in table \ref{tab:spacing}, the spanwise spacing of the low-speed streaks varies between $\lambda^+_f = 116$ and $134$ when $\Sto^+ = 6$ particles are dispersed. The corresponding spacing of particle clusters varies in the range of $\lambda^+_p = 90-108$, \textcolor{revision} {with less disparity between $\lambda^+_p$ and $\lambda^+_f$ compared to the flow laden with $\Sto^+ = 30$ particles. This suggests that $\Sto^+ = 6$ clusters are more closely aligned with the high-strain low-vorticity regions found within the low-speed streaks, likely due to their lower inertia.}
%
%\textcolor{revision}{ but the difference $\lambda_f^+-\lambda_p^+$ remains constant at 26 wall units for all mass loadings considered. This lower disparity between low-speed streaks and particle spacing suggests that  $\Sto^+ = 6$ particles are primarily located close to the low-speed streaks and that their comparatively lower inertia prevents them from escaping further way}.

Note that the two-way coupling plays a critical role in the arrangement of low-speed streaks and particle clusters. In a prior study by \citet{jieExistenceFormationMultiscale2022}, where the authors considered one-way coupled Euler-Lagrange simulations of particle-laden channel flows at $\Rey_\tau$ between 600 and 2000, the absence of feedback force from the particles leads to low-speed streaks that have identical characteristics to those of a particle-free turbulent channel flow. The data presented by the authors further suggests that the particle cluster spanwise spacing varies little with Reynolds number and is about $\lambda_p^+ \sim 114$ for $\Sto^+ = 30$ particles. However, as we have shown in this study,  $\lambda_p^+$ and $\lambda_f^+$ reach considerably higher values when two-way coupling is significant since the dynamics of clusters and near-wall coherent structures become more inter-dependent.

\begin{figure}
  \centering
  \includegraphics[width=4in]{tikzgraphics/fig18.pdf}
  \caption{\textcolor{revision}{Instantaneous velocity vectors overlayed by contour of particle volume fraction $\phi/\phi_0 = 3$, for the case $\Sto^+ = 30, M = 1.0$, show particle ropes forming in the high strain region between quasi-streamwise vortices.}}
  \label{fig:vorticity}
\end{figure}

%The mechanism underpinning the modulation of low-speed streaks relates to how inertial particles interact with quasi-streamwise vortices surrounding these streaks.
Figure \ref{fig:vorticity} shows an example of how $\Sto^+=30$ particles are distributed in the vicinity of a pair of quasi-streamwise vortices. The particles form ropes by concentrating in the straining region between the pair of vortices, consistently with the preferential concentration mechanism. Pockets of particles can be seen ejected upward towards the centerline, which results in a downward feedback force on the fluid. This process is self-sustaining because the ejected particles eventually return to the near-wall region due to turbophoresis, where they accumulate again along particle ropes. The feedback force from these clusters contributes to the the suppression of bursting and stabilization of quasi-streamise vortices. Consequently, low-speed streaks nested in-between quasi-streamwise vortices extend further than possible in particle-free flows. Because bursting events contribute largely to the Reynolds shear stress production \citep{willmarthStructureReynoldsStress1972}, the stabilizing role of  $\Sto^+=30$ particles is likely the main reason these particles reduce the fluid-phase Reynolds shear stress to the extent shown in \S \ref{sec:stress_two_phase}, and \emph{in fine}, skin-friction drag reduction.

    \caption{Lower is better.}
    \label{fig:mmdclustering}
\end{figure}

\begin{figure}
    \centering
    \input{plots/transitivityEMAILDNC}
% \input{plots/transitivityEMAILENRON}
\input{plots/transitivityEMAILEUCORE}
\input{plots/transitivityCOAUTHDBLP}
\input{plots/transitivityFACEBOOKLINKS}


\pgfplotsset{
    every axis plot/.append style={line width=1.5pt},
    every axis title/.style={below left,at={(0.99,0.99)}},
    every axis title/.append style={font=\scriptsize},
    every tick label/.append style={font=\scriptsize},
    % every axis label/.append style={font=\scriptsize}
    % every axis xlabel/.append style={font=\scriptsize}
    every axis ylabel/.append style={font=\scriptsize}
}

\begin{tikzpicture}

\begin{groupplot}[
        group style={
            group size=2 by 2,
            vertical sep=0.5cm, 
            ylabels at=edge left,
            xlabels at=edge bottom
        },
        height=4cm,
        width=\linewidth/2,
        xtick={1, 2, 3, 4, 5, 6, 7, 8, 9, 10},
        xlabel={$t$},
        ylabel={Transitivity MMD},
        xmin=0,
        xmax=11,
        ymin=-0.2,
        ymax=2.3,
        xtick align=inside,
        ytick align=inside,
        major tick length=0.5ex,
        % legend style={at={(-0.1,-0.5)}, anchor=north, font=\sffamily\small, draw=none},
        legend style={at={(-0.2,-0.5)}, anchor=north, draw=none},
        legend columns=6,
        legend image post style={mark options={thick, scale=1}},
    ]
    

    \nextgroupplot[title={DNC Emails}, mark=.]
        \addplot [color=ploter]
                  table [x=ts, y=mmd] {\transitivityEMAILDNCer};
        \addplot [color=plotcl]
                  table [x=ts, y=mmd] {\transitivityEMAILDNCcl};
        \addplot [color=plotsbm]
                  table [x=ts, y=mmd] {\transitivityEMAILDNCsbm};
        \addplot [color=plotgraphrnn]
                  table [x=ts, y=mmd] {\transitivityEMAILDNCgraphrnn};
        \addplot [color=plotverg]
                  table [x=ts, y=mmd] {\transitivityEMAILDNCverg};
        \addplot [color=plotdyverg]
                  table [x=ts, y=mmd] {\transitivityEMAILDNCdyverg};

    \nextgroupplot[title={EU Emails}, mark=.]
        \addplot [color=ploter]
                  table [x=ts, y=mmd] {\transitivityEMAILEUCOREer};
        \addplot [color=plotcl]
                  table [x=ts, y=mmd] {\transitivityEMAILEUCOREcl};
        \addplot [color=plotsbm]
                  table [x=ts, y=mmd] {\transitivityEMAILEUCOREsbm};
        \addplot [color=plotgraphrnn]
                  table [x=ts, y=mmd] {\transitivityEMAILEUCOREgraphrnn};
        \addplot [color=plotverg]
                  table [x=ts, y=mmd] {\transitivityEMAILEUCOREverg};
        \addplot [color=plotdyverg]
                  table [x=ts, y=mmd] {\transitivityEMAILEUCOREdyverg};

    \nextgroupplot[title={DBLP}, mark=.]
        \addplot [color=ploter]
                  table [x=ts, y=mmd] {\transitivityCOAUTHDBLPer};
        \addplot [color=plotcl]
                  table [x=ts, y=mmd] {\transitivityCOAUTHDBLPcl};
        \addplot [color=plotsbm]
                  table [x=ts, y=mmd] {\transitivityCOAUTHDBLPsbm};
        \addplot [color=plotgraphrnn]
                  table [x=ts, y=mmd] {\transitivityCOAUTHDBLPgraphrnn};
        \addplot [color=plotverg]
                  table [x=ts, y=mmd] {\transitivityCOAUTHDBLPverg};
        \addplot [color=plotdyverg]
                  table [x=ts, y=mmd] {\transitivityCOAUTHDBLPdyverg};

    \nextgroupplot[title={Facebook}, mark=.]
        \addplot [color=ploter]
                  table [x=ts, y=mmd] {\transitivityFACEBOOKLINKSer};
        \addplot [color=plotcl]
                  table [x=ts, y=mmd] {\transitivityFACEBOOKLINKScl};
        \addplot [color=plotsbm]
                  table [x=ts, y=mmd] {\transitivityFACEBOOKLINKSsbm};
        \addplot [color=plotgraphrnn]
                  table [x=ts, y=mmd] {\transitivityFACEBOOKLINKSgraphrnn};
        \addplot [color=plotverg]
                  table [x=ts, y=mmd] {\transitivityFACEBOOKLINKSverg};
        \addplot [color=plotdyverg]
                  table [x=ts, y=mmd] {\transitivityFACEBOOKLINKSdyverg};

\legend{Erdos-Reny\'i~~,Chung-Lu~~,SBM~~,GraphRNN~~,VeRG~~,DyVeRG}

\end{groupplot}
\end{tikzpicture}


    \caption{Lower is better.}
    \label{fig:mmdtransitivity}
\end{figure}

\begin{figure}
    \centering
    \input{plots/trianglecountEMAILDNC}
% \input{plots/trianglecountEMAILENRON}
\input{plots/trianglecountEMAILEUCORE}
\input{plots/trianglecountCOAUTHDBLP}
\input{plots/trianglecountFACEBOOKLINKS}


\pgfplotsset{
    every axis plot/.append style={line width=1.5pt},
    every axis title/.style={below left,at={(0.99,0.99)}},
    every axis title/.append style={font=\scriptsize},
    every tick label/.append style={font=\scriptsize},
    % every axis label/.append style={font=\scriptsize}
    % every axis xlabel/.append style={font=\scriptsize}
    every axis ylabel/.append style={font=\scriptsize}
}

\begin{tikzpicture}

\begin{groupplot}[
        group style={
            group size=2 by 2,
            vertical sep=0.5cm, 
            ylabels at=edge left,
            xlabels at=edge bottom
        },
        height=4cm,
        width=\linewidth/2,
        xtick={1, 2, 3, 4, 5, 6, 7, 8, 9, 10},
        xlabel={$t$},
        ylabel={Triangles MMD},
        xmin=0,
        xmax=11,
        ymin=-0.2,
        ymax=2.3,
        xtick align=inside,
        ytick align=inside,
        major tick length=0.5ex,
        % legend style={at={(-0.1,-0.5)}, anchor=north, font=\sffamily\small, draw=none},
        legend style={at={(-0.2,-0.5)}, anchor=north, draw=none},
        legend columns=6,
        legend image post style={mark options={thick, scale=1}},
    ]
    

    \nextgroupplot[title={DNC Emails}, mark=.]
        \addplot [color=ploter]
                  table [x=ts, y=mmd] {\trianglecountEMAILDNCer};
        \addplot [color=plotcl]
                  table [x=ts, y=mmd] {\trianglecountEMAILDNCcl};
        \addplot [color=plotsbm]
                  table [x=ts, y=mmd] {\trianglecountEMAILDNCsbm};
        \addplot [color=plotgraphrnn]
                  table [x=ts, y=mmd] {\trianglecountEMAILDNCgraphrnn};
        \addplot [color=plotverg]
                  table [x=ts, y=mmd] {\trianglecountEMAILDNCverg};
        \addplot [color=plotdyverg]
                  table [x=ts, y=mmd] {\trianglecountEMAILDNCdyverg};

    \nextgroupplot[title={EU Emails}, mark=.]
        \addplot [color=ploter]
                  table [x=ts, y=mmd] {\trianglecountEMAILEUCOREer};
        \addplot [color=plotcl]
                  table [x=ts, y=mmd] {\trianglecountEMAILEUCOREcl};
        \addplot [color=plotsbm]
                  table [x=ts, y=mmd] {\trianglecountEMAILEUCOREsbm};
        \addplot [color=plotgraphrnn]
                  table [x=ts, y=mmd] {\trianglecountEMAILEUCOREgraphrnn};
        \addplot [color=plotverg]
                  table [x=ts, y=mmd] {\trianglecountEMAILEUCOREverg};
        \addplot [color=plotdyverg]
                  table [x=ts, y=mmd] {\trianglecountEMAILEUCOREdyverg};

    \nextgroupplot[title={DBLP}, mark=.]
        \addplot [color=ploter]
                  table [x=ts, y=mmd] {\trianglecountCOAUTHDBLPer};
        \addplot [color=plotcl]
                  table [x=ts, y=mmd] {\trianglecountCOAUTHDBLPcl};
        \addplot [color=plotsbm]
                  table [x=ts, y=mmd] {\trianglecountCOAUTHDBLPsbm};
        \addplot [color=plotgraphrnn]
                  table [x=ts, y=mmd] {\trianglecountCOAUTHDBLPgraphrnn};
        \addplot [color=plotverg]
                  table [x=ts, y=mmd] {\trianglecountCOAUTHDBLPverg};
        \addplot [color=plotdyverg]
                  table [x=ts, y=mmd] {\trianglecountCOAUTHDBLPdyverg};

    \nextgroupplot[title={Facebook}, mark=.]
        \addplot [color=ploter]
                  table [x=ts, y=mmd] {\trianglecountFACEBOOKLINKSer};
        \addplot [color=plotcl]
                  table [x=ts, y=mmd] {\trianglecountFACEBOOKLINKScl};
        \addplot [color=plotsbm]
                  table [x=ts, y=mmd] {\trianglecountFACEBOOKLINKSsbm};
        \addplot [color=plotgraphrnn]
                  table [x=ts, y=mmd] {\trianglecountFACEBOOKLINKSgraphrnn};
        \addplot [color=plotverg]
                  table [x=ts, y=mmd] {\trianglecountFACEBOOKLINKSverg};
        \addplot [color=plotdyverg]
                  table [x=ts, y=mmd] {\trianglecountFACEBOOKLINKSdyverg};

\legend{Erdos-Reny\'i~~,Chung-Lu~~,SBM~~,GraphRNN~~,VeRG~~,DyVeRG}

\end{groupplot}
\end{tikzpicture}


    \caption{Lower is better.}
    \label{fig:mmdtriangles}
\end{figure}
\fi

\subsection{Interpretability}
To illustrate how the DyVeRG model can help a practitioner understand a complex temporal dataset, we illustrate some specific examples of frequent \emph{rule transitions}---analogous to subgraph-to-subgraph transitions---learned by the model. 

%VRGs 
%Recall that a grammar not only encodes low-level rules consisting of terminal nodes and edges, and high-level, macro-scale rules consisting of mostly nonterminal symbols, but also meso-scale rules representing various levels of refinement. So, although they are similar in nature to the subgraph-to-subgraph transitions of related work, the rule transitions in the DyVeRG model describe changes at all levels of granularity within the graph. Furthermore, DyVeRG's heirarchical filtration scheme automatically focuses the model on which changes are the most relevant to understanding the graph.

\begin{figure}[t]
    \centering
    \scalebox{0.775}{\tikzmath{%
    \colone = 0;
    \coltwo = 5;
    \yoffset = 1;
    \yshiftrulefour = -0.75;
}

\begin{tikzpicture}
    \coordinate (label1l) at (\colone - 0.25 - 3/4, -1 * \yoffset + 0.4);
    \coordinate (left1l) at (\colone, -1 * \yoffset + 0.4);
    \coordinate (right1l) at (\colone + 1.5, -1 * \yoffset + 0.4);
    \coordinate (label1r) at (\coltwo - 3/4, -1 * \yoffset + 0.4);
    \coordinate (left1r) at (\coltwo, -1 * \yoffset + 0.4);
    \coordinate (right1r) at (\coltwo + 1.5, -1 * \yoffset + 0.4);
    \coordinate (arrow1) at (\colone + 3.95, -1 * \yoffset + 0.4);

    \coordinate (label2l) at (\colone - 0.25 - 3/4, -2 * \yoffset + 0.425);
    \coordinate (left2l) at (\colone, -2 * \yoffset + 0.425);
    \coordinate (right2l) at (\colone + 1.5, -2 * \yoffset + 0.425);
    \coordinate (label2r) at (\coltwo - 3/4, -2 * \yoffset + 0.425);
    \coordinate (left2r) at (\coltwo, -2 * \yoffset + 0.425);
    \coordinate (right2r) at (\coltwo + 1.5, -2 * \yoffset + 0.425);
    \coordinate (arrow2) at (\colone + 3.95, -2 * \yoffset + 0.425);

    \coordinate (label3l) at (\colone - 0.25 - 3/4, -3 * \yoffset + 0.4);
    \coordinate (left3l) at (\colone, -3 * \yoffset + 0.4);
    \coordinate (right3l) at (\colone + 1.5, -3 * \yoffset + 0.4);
    \coordinate (label3r) at (\coltwo - 3/4, -3 * \yoffset + 0.4);
    \coordinate (left3r) at (\coltwo, -3 * \yoffset + 0.4);
    \coordinate (right3r) at (\coltwo + 1.5, -3 * \yoffset + 0.4);
    \coordinate (arrow3) at (\colone + 3.95, -3 * \yoffset + 0.4);

    \coordinate (label4l) at (\colone - 0.25 - 3/4, -4 * \yoffset);
    \coordinate (left4l) at (\colone, -4 * \yoffset);
    \coordinate (right4l) at (\colone + 1.5, -4 * \yoffset);
    \coordinate (label4r) at (\coltwo - 3/4, -4 * \yoffset);
    \coordinate (left4r) at (\coltwo, -4 * \yoffset);
    \coordinate (right4r) at (\coltwo + 1.5, -4 * \yoffset);
    \coordinate (arrow4) at (\colone + 3.95, -4 * \yoffset);

    % rule1
    \begin{scope}[scale=1]
        \node [] at (label1l) () {\(\dot{P}_1\)};
        \begin{scope}[scale=1/2, shift={(left1l)}, transparency group, opacity=0.5]
            \node [lhs] at (0, 0) (lhs) {\scriptsize \phantom{0}};
            \node [blank] at (right1l) (rhs) {};
            \draw [->, edge] (lhs) -- (rhs);
        \end{scope}
        \begin{scope}[scale=1/2, shift={(right1l)}]
            \node [lhs, opacity=0.5] at (0, 0) (lhs) {\scriptsize \phantom{0}};
        \end{scope}
        \node [] at (arrow1) () {$\xRightarrow{\times 61}$};
        \begin{scope}[scale=1/2, shift={(left1r)}]
            \node [text opacity=1] at (label1r) () {};
            \node [addlhs] at (0, 0) (lhs) {\scriptsize 2};
            \node [blank] at (right1r) (rhs) {};
            \draw [->, edge] (lhs) -- (rhs);
        \end{scope}
        \begin{scope}[scale=1/2, shift={(right1r)}]
            \node [addnode] at (0, 0) (rhs0) {};
            \draw [addedge] (rhs0) -- (0.53033, 0.53033);
            \draw [addedge] (rhs0) -- (0.53033, -0.53033);
        \end{scope}
    \end{scope}

    % rule2
    \begin{scope}[scale=1]
        \node [] at (label2l) () {\(\dot{P}_2\)};
        \begin{scope}[scale=1/2, shift={(left2l)}, transparency group, opacity=0.5]
            \node [lhs] at (0, 0) (lhs) {\scriptsize \phantom{0}};
            \node [blank] at (right2l) (rhs) {};
            \draw [->, edge] (lhs) -- (rhs);
        \end{scope}
        \begin{scope}[scale=1/2, shift={(right2l)}]
            \node [lhs, opacity=0.5] at (0, 0) (lhs) {\scriptsize \phantom{0}};
        \end{scope}
        \node [] at (arrow2) () {$\xRightarrow{\times 5}$};
        \begin{scope}[scale=1/2, shift={(left2r)}]
            \node [text opacity=1] at (label2r) () {};
            \node [addlhs] at (0, 0) (lhs) {\scriptsize 3};
            \node [blank] at (right2r) (rhs) {};
            \draw [->, edge] (lhs) -- (rhs);
        \end{scope}
        \begin{scope}[scale=1/2, shift={(right2r)}]
            \node [addnode] at (0.75, 0) (rhs0) {};
            \node [addnode] at (2.25, 0) (rhs1) {};
            \draw [addedge] (rhs0) -- (rhs1);
            \draw [addedge] (rhs0) -- (0.2196699, 0.53033);
            \draw [addedge] (rhs0) -- (0.2196699, -0.53033);
            \draw [addedge] (rhs1) -- (3.0, 0.0);
        \end{scope}
    \end{scope}

    % rule3
    \begin{scope}[scale=1]
        \node [] at (label3l) () {\(\dot{P}_3\)};
        \begin{scope}[scale=1/2, shift={(left3l)}]
            \node [lhs] at (0, 0) (lhs) {\scriptsize 10};
            \node [blank] at (right3l) (rhs) {};
            \draw [->, edge] (lhs) -- (rhs);
        \end{scope}
        \begin{scope}[scale=1/2, shift={(right3l)}]
            \node [node] at (0.75, 0) (rhs0) {};
            \node [node] at (2.25, 0) (rhs1) {};
            \draw [edge] (rhs0) -- (rhs1);
            \draw [edge] (rhs0) -- (1.28033, 0.53033);
            \draw [edge] (rhs0) -- (0.75, 0.75);
            \draw [edge] (rhs0) -- (0.2196699, 0.53033);
            \draw [edge] (rhs0) -- (0.0, 0.0);
            \draw [edge] (rhs0) -- (0.2196699, -0.53033);
            \draw [edge] (rhs0) -- (0.75, -0.75);
            \draw [edge] (rhs0) -- (1.28033, -0.53033);
            \draw [edge] (rhs1) -- (3.0, 0.0);
            \draw [edge] (rhs1) -- (2.625, 0.649519);
            \draw [edge] (rhs1) -- (2.625, -0.649519);
        \end{scope}
        \node [] at (arrow3) () {$\xRightarrow{\times 3}$};
        \begin{scope}[scale=1/2, shift={(left3r)}]
            \node [text opacity=1] at (label3r) () {};
            \node [dellhs] at (0, 0) (lhs) {\scriptsize 8};
            \node [blank] at (right3r) (rhs) {};
            \draw [->, edge] (lhs) -- (rhs);
        \end{scope}
        \begin{scope}[scale=1/2, shift={(right3r)}]
            \node [node] at (0.75, 0) (rhs0) {};
            \node [node] at (2.25, 0) (rhs1) {};
            \draw [edge] (rhs0) -- (rhs1);
            \draw [edge] (rhs0) -- (1.28033, 0.53033);
            \draw [edge] (rhs0) -- (0.75, 0.75);
            \draw [edge] (rhs0) -- (0.2196699, 0.53033);
            \draw [edge] (rhs0) -- (0.0, 0.0);
            \draw [edge] (rhs0) -- (0.2196699, -0.53033);
            \draw [edge] (rhs0) -- (0.75, -0.75);
            \draw [edge] (rhs0) -- (1.28033, -0.53033);
            \draw [deledge] (rhs1) -- (3.0, 0.0);
            \draw [edge] (rhs1) -- (2.625, 0.649519);
            \draw [deledge] (rhs1) -- (2.625, -0.649519);
        \end{scope}
    \end{scope}

    % rule4
    \begin{scope}[scale=1]
        \node [] at (label4l) () {\(\dot{P}_4\)};
        \begin{scope}[scale=1/2, shift={(left4l)}]
            \node [lhs] at (0, 0) (lhs) {\scriptsize 10};
            \node [blank] at (right4l) (rhs) {};
            \draw [->, edge] (lhs) -- (rhs);
        \end{scope}
        \begin{scope}[scale=1/2, shift={(right4l)}]
            \node [node] at (0.75, 0) (rhs0) {};
            \node [node] at (1.85, 0.649519) (rhs1) {};
            \node [node] at (1.85, -0.649519) (rhs2) {};
            \draw [edge] (rhs0) -- (rhs2);
            \draw [edge] (rhs1) -- (rhs2);

            \draw [edge] (rhs0) -- (0.375, 0.649519);
            \draw [edge] (rhs0) -- (0.0, 0.0);
            \draw [edge] (rhs0) -- (0.375, -0.649519);

            \draw [edge] (rhs1) -- (1.3196699, 1.179849);
            \draw [edge] (rhs1) -- (2.38033, 1.179849);

            \draw [edge] (rhs2) -- (1.2754666676607664, -1.131609707264904);
            \draw [edge] (rhs2) -- (1.7197638667498023, -1.388124814759156);
            \draw [edge] (rhs2) -- (2.225, -1.2990380528383292);
            \draw [edge] (rhs2) -- (2.5547694655894313, -0.9060341074942514);
            \draw [edge] (rhs2) -- (2.5547694655894313, -0.3930038925057484);
        \end{scope}
        \node [] at (arrow4) () {$\xRightarrow{\times 2}$};
        \begin{scope}[scale=1/2, shift={(left4r)}]
            \node [text opacity=1] at (label4r) () {};
            \node [addlhs] at (0, 0) (lhs) {\scriptsize 14};
            \node [blank] at (right4r) (rhs) {};
            \draw [->, edge] (lhs) -- (rhs);
        \end{scope}
        \begin{scope}[scale=1/2, shift={(right4r)}]
            \node [node] at (0.75, 0) (rhs0) {};
            \node [node] at (1.85, 0.649519) (rhs1) {};
            \node [node] at (1.85, -0.649519) (rhs2) {};
            \draw [deledge] (rhs0) -- (rhs2);
            \draw [edge] (rhs1) -- (rhs2);

            \draw [addedge] (rhs0) -- (0.75, 0.75);
            \draw [edge] (rhs0) -- (0.375, 0.649519);
            \draw [addedge] (rhs0) -- (0.100480947161671, 0.37499999999999994);
            \draw [edge] (rhs0) -- (0.0, 0.0);
            \draw [addedge] (rhs0) -- (0.10048094716167089, -0.375);
            \draw [edge] (rhs0) -- (0.375, -0.649519);

            \draw [addedge] (rhs1) -- (1.1255556302831988, 0.8436332838268907);
            \draw [edge] (rhs1) -- (1.3196699, 1.179849);
            \draw [addedge] (rhs1) -- (1.6558857161731095, 1.3739633697168012);
            \draw [addedge] (rhs1) -- (2.0441142838268904, 1.3739633697168012);
            \draw [edge] (rhs1) -- (2.38033, 1.179849);
            \draw [addedge] (rhs1) -- (2.5744443697168014, 0.8436332838268905);

            \draw [deledge] (rhs2) -- (1.2754666676607664, -1.131609707264904);
            \draw [edge] (rhs2) -- (1.7197638667498023, -1.388124814759156);
            \draw [deledge] (rhs2) -- (2.225, -1.2990380528383292);
            \draw [edge] (rhs2) -- (2.5547694655894313, -0.9060341074942514);
            \draw [deledge] (rhs2) -- (2.5547694655894313, -0.3930038925057484);
        \end{scope}
        \begin{pgfonlayer}{background} 
            \coordinate (lsw) at ([shift={(-0.2 + 0.15 + 0.87, -0.2)}]current bounding box.south west);
            \coordinate (lne) at ([shift={(0.2 - 4.9, 0.05)}]current bounding box.north east);
            \coordinate (rsw) at ([shift={(-0.2 + 0.15 + 5.9, -0.2)}]current bounding box.south west);
            \coordinate (rne) at ([shift={(0.2, 0.05)}]current bounding box.north east);

            \draw [draw=plotgrey, fill=plotgrey, opacity=0.75] (lsw) rectangle (lne);
            \draw [draw=plotgrey, fill=plotgrey, opacity=0.75] (rsw) rectangle (rne);
            \draw [very thick, draw=white] ([shift={(0.0, 3.85)}]lsw) -- ([shift={(4.65, 3.85)}]rsw);
            \draw [very thick, draw=white] ([shift={(0.0, 2.85)}]lsw) -- ([shift={(4.65, 2.85)}]rsw);
            \draw [very thick, draw=white] ([shift={(0.0, 1.75)}]lsw) -- ([shift={(4.65, 1.75)}]rsw);
        \end{pgfonlayer}
        \node [] at ([shift={(9.75, 0.0)}]label1l) () {\(\ddot{P}_1\)};
        \node [] at ([shift={(9.75, 0.0)}]label2l) () {\(\ddot{P}_2\)};
        \node [] at ([shift={(9.75, 0.0)}]label3l) () {\(\ddot{P}_3\)};
        \node [] at ([shift={(9.75, 0.0)}]label4l) () {\(\ddot{P}_4\)};
    \end{scope}
\end{tikzpicture}
}
    \caption{A sample of the top rule transitions from the EU~Emails dataset. $\times 61$ denotes that the first rule transition was repeated 61 times. These rule transitions describe various changes in graph structure over time.}
    \label{fig:interpret}
\end{figure}

We focus our analysis here on the first $10$ timesteps of the EU~Emails dataset.
For each $t \in \{0, \dots 9\}$, we extract a grammar on $G_t$ and then update it according to the procedure described in \hyperref[sec:dyverg]{Section~3},
giving us a list of DyVeRG grammars $\langle \mathcal{G}_t \rangle_{t = 1}^{10}$.
Then, given two rules $\dot{P}$ and $\ddot{P}$, each of which could be a rule from any of the grammars $\mathcal{G}_t$ for $t \in \{1, \dots 10\}$, we say that a transition of \emph{type} $\dot{P} \implies \ddot{P}$ has occurred \emph{iff} there is a grammar $\mathcal{G}_i$ such that, during the temporal updating procedure, a rule isomorphic to $\dot{P}$ was modified into a rule isomorphic to $\ddot{P}$.
We then go through our list of grammars and tally up the frequency with which every possible rule transition occurs with the idea in mind that the most frequent rule transitions might provide some salient insight into the dynamics of the dataset.
In \autoref{fig:interpret}, we have a sample of four of the most frequent rule transitions learned from EU~Emails, which we will refer to as $\dot{P}_1 \implies \ddot{P}_1$, $\dots$ $\dot{P}_4 \implies \ddot{P}_4$ respectively.

Both $\dot{P}_1 \implies \ddot{P}_1$ and $\dot{P}_2 \implies \ddot{P}_2$ illustrate that a new structure emerged at time $t + 1$ among nodes that did not already exist in $G_t$.
In the first case, we have the introduction of a new user participating in a email exchange with two other people, and this occurs $61$ times throughout the whole dataset.
In the second, we can see a pair of new users emailing each other, one of whom has sent two emails elsewhere in the network and the other of whom has sent out one additional email,
a structure that occurs $5$ times in the data.
In either case, because there was no rule at time $t$ that $\ddot{P}_1$ and $\ddot{P}_2$ are updated versions of, we can be certain they were additions that participated in a larger connected component that was introduced wholly at time $t + 1$.
This reveals a temporal property of the EU~Email network: it is much more frequent for users to send out emails following periods of inactivity when other previously-inactive users are also sending out emails to new people, than it would be for them to suddenly begin sending emails to active users.

The next transition, $\dot{P}_3 \implies \ddot{P}_3$, shows us that three times throughout the data, a heterophilous dyad%
---a communicating pair of users where one is involved in many emails and the other is not---%
will see a reduction in the number of emails in which the less popular user participates.
By contrast, the final transition $\dot{P}_4 \implies \ddot{P}_4$ exemplifies a more extreme version of the \emph{opposite} phenomenon:
twice in the data, when a heterophilous wedge consisting of two unpopular users is bridged by a high-volume email-sender, the bridging user will experience a reduction in email output while the unpopular users become more popular.

The insights obtained by this analysis are over-specific due largely to the precise nature of rule isomorphism.
However, if a more relaxed view of rule isomorphism is adopted, and the definition of rule transition is broadened, then our model could describe even more general temporal trends.
Even so, our model has shown its ability to provide significant insight into network dynamics.
