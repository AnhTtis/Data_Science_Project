\pdfoutput=1
\documentclass[twoside]{article}
\usepackage[letterpaper, left=3.5cm, right=3.5cm, top=4cm, bottom=2.5cm]{geometry}
\usepackage{graphicx}
\usepackage{amsmath,amssymb,amsthm}
\usepackage{authblk}
\usepackage[utf8]{inputenc}
\usepackage{microtype}
\usepackage{mathrsfs} 
\usepackage{enumitem}
\usepackage{calc}
\usepackage{esint}
\usepackage{fancyhdr}
\usepackage{xcolor}
\usepackage[labelfont = bf]{caption}
\usepackage{doi}
\usepackage{subfigure}
\usepackage[numbers]{natbib}
\usepackage{float}
\usepackage{stmaryrd}
\usepackage{mathtools}
\usepackage{booktabs}
\setlength{\aboverulesep}{0pt}
\setlength{\belowrulesep}{0pt}
\usepackage{colortbl}
\usepackage{tabulary}
\usepackage{diagbox}
\usepackage{caption}
%\usepackage{refcheck}

\newtheorem{theorem}{Theorem}[section]
\numberwithin{equation}{section}
\newtheorem{proposition}[theorem]{Proposition}
\newtheorem{definition}[theorem]{Definition}
\newtheorem{corollary}[theorem]{Corollary}
\newtheorem{remark}[theorem]{Remark}
\newtheorem{lemma}[theorem]{Lemma}
\newtheorem{example}[theorem]{Example}
\newtheorem{algorithm}[theorem]{Algorithm}
\newtheorem{assumption}[theorem]{Assumption}

\usepackage{titlesec}
\titleformat{\section}{\normalfont\scshape\centering}{\thesection.}{0.5em}{}
%\titleformat*{\section}{\large \bfseries}
\titleformat*{\subsection}{\itshape}
\titleformat*{\subsubsection}{\itshape}

\providecommand{\keywords}[1]
{
	{\small\hspace{-5mm}\textit{Keywords:} #1}
}

\providecommand{\MSC}[1]
{
	{\small\hspace{-5mm}\textit{AMS MSC (2020):~~} #1}
}

\usepackage{hyperref}
\hypersetup{
	colorlinks=true,
	linkcolor=blue,
	citecolor = blue,
	filecolor=magenta,  
	urlcolor=cyan,
}

\providecommand{\jumptmp}[2]{#1\llbracket{#2}#1\rrbracket}
\providecommand{\jump}[1]{\jumptmp{}{#1}}

\AtBeginDocument{%
	\def\MR#1{}
}

\newcommand{\AAA}{\boldsymbol{\mathcal{A}}}


\begin{document}
	\setlength{\abovedisplayskip}{5.5pt}
	\setlength{\belowdisplayskip}{5.5pt}
	\setlength{\abovedisplayshortskip}{5.5pt}
	\setlength{\belowdisplayshortskip}{5.5pt}
	

	\title{\vspace{-21mm}Error analysis for a Crouzeix--Raviart approximation of the variable exponent Dirichlet~problem}
	\author[1]{Anna Kh.\ Balci\thanks{Email: \texttt{akhripun@math.uni-bielefeld.de}
	
	Anna Kh.Balci research is funded by by the Deutsche Forschungsgemeinschaft (DFG, German Research Foundation) - SFB 1283/2 2021 - 317210226}}
	\author[2]{Alex Kaltenbach\thanks{Email: \texttt{alex.kaltenbach@mathematik.uni-freiburg.de\vspace*{-4mm}}}}
	\date{\today}
	\affil[1]{\small{Faculty of Mathematics, University of Bielefeld, Universitätsstra\ss e~25, 33615 Bielefeld}}
	\affil[2]{\small{Department of Applied Mathematics, University of Freiburg, Ernst--Zermelo--Stra\ss e~1, 79104 Freiburg}}
	\maketitle
	

	\pagestyle{fancy}
	\fancyhf{}
	\fancyheadoffset{0cm}
	\addtolength{\headheight}{-0.25cm}
	\renewcommand{\headrulewidth}{0pt} 
	\renewcommand{\footrulewidth}{0pt}
	\fancyhead[CO]{\textsc{CR for the $p(\cdot)$-Dirichlet problem}}
	\fancyhead[CE]{\textsc{A. Kh.\ Balci and A. Kaltenbach}}
	\fancyhead[R]{\thepage}
	\fancyfoot[R]{}
	
	\begin{abstract}
		In the present paper, we examine a Crouzeix--Raviart approximation of the $p(\cdot)$-Dirichlet problem. We derive a medius error estimate, i.e., a best-approximation result, which holds  for uniformly continuous  exponents and implies a priori error estimates, which~apply~for~Hölder continuous exponents and are optimal for Lipschitz continuous exponents. 
        Numerical experiments are carried out to review the theoretical findings.
	\end{abstract}

	\keywords{\hspace{-0.1mm}$p(\cdot)$-Dirichlet \hspace{-0.1mm}problem; \hspace{-0.1mm}Crouzeix--Raviart \hspace{-0.1mm}element; \hspace{-0.1mm}a \hspace{-0.1mm}priori \hspace{-0.1mm}error \hspace{-0.1mm}analysis;~\hspace{-0.1mm}medius~\hspace{-0.1mm}error~\hspace{-0.1mm}\mbox{analysis}.}
	\MSC{\hspace{-0.1mm}49M29; 65N15; 65N30; 35J60; 46E30.}
	
	
	\section{Introduction}\label{sec:intro}
    \thispagestyle{empty}
  	
  	\hspace{5mm}We examine the numerical approximation of a non-linear system of $p(\cdot)$-Dirichlet type, i.e.,\enlargethispage{5mm}
  	\begin{align}\label{eq:pDirichlet}
  	 \begin{aligned}
  	 -\mathrm{div}(\AAA(\cdot,\nabla u))&=f&&\quad\textup{ in }\Omega\,,\\
  	 u&= 0&&\quad\textup{ on }\Gamma_D\,,\\
  	  \AAA(\cdot,\nabla u)\cdot n &= 0&&\quad\textup{ on }\Gamma_N\,,
  	 \end{aligned}
  	\end{align}
 using the Crouzeix--Raviart element, cf.\ \cite{CR73}. More precisely, given $f\in L^{p'(\cdot)}(\Omega)$,~where~$p'(x)\hspace{-0.1em}\coloneqq\hspace{-0.1em} \smash{\frac{p(x)}{p(x)-1}}$ for all $x\hspace{-0.15em}\in\hspace{-0.15em} \Omega$, and $p\hspace{-0.15em}\in\hspace{-0.15em} C^0(\overline{\Omega})$ with $p^-\hspace{-0.15em}\coloneqq\hspace{-0.15em}\min_{x\in \overline{\Omega}}{p(x)}\hspace{-0.15em}>\hspace{-0.15em}1$, we seek $u\hspace{-0.15em} \in\hspace{-0.15em} \smash{W^{1,p(\cdot)}_D(\Omega)}$~\mbox{solving}~\eqref{eq:pDirichlet}. Here, ${\Omega \subseteq \mathbb{R}^d}$, ${d \in \mathbb{N}}$, is a bounded Lipschitz domain whose topological boundary $\partial\Omega$ is disjointly \hspace{-0.1mm}divided \hspace{-0.1mm}into \hspace{-0.1mm}a \hspace{-0.1mm}Dirichlet \hspace{-0.1mm}part \hspace{-0.1mm}$\Gamma_D$ \hspace{-0.1mm}and \hspace{-0.1mm}a \hspace{-0.1mm}Neumann \hspace{-0.1mm}part \hspace{-0.1mm}$\Gamma_N$, and $\smash{W^{1,p(\cdot)}_D(\Omega)}\coloneqq \{v\in \smash{W^{1,p(\cdot)}(\Omega)}\mid \textup{tr}\, v=0\textup{ on }\Gamma_D\}$.
 \hspace{-0.1mm}The \hspace{-0.1mm}non-linear~\hspace{-0.1mm}operator~\hspace{-0.1mm}${\AAA\colon\hspace{-0.1em}\Omega\times\mathbb{R}^d\hspace{-0.1em}\to\hspace{-0.1em} \mathbb{R}^d}$ is supposed to assume the form 
  	\begin{align}\label{example}
  	 \AAA(\cdot,\nabla u)\coloneqq (\delta+\vert \nabla u\vert)^{p(\cdot)-2}\nabla u\,.
  	\end{align}
   Since $\AAA\colon\Omega\times \mathbb{R}^d\to \mathbb{R}^d$ possesses a potential with respect~to~its~second~component, i.e., there exists a  function $\varphi\colon \Omega\times\mathbb{R}_{\ge 0}\to \mathbb{R}_{\ge 0}$, which is strictly convex with respect to its~second~component, such~that $D(\varphi(x,\vert\cdot\vert))(a)=\AAA(x,a)$ %$D(\varphi(x,\cdot)\circ \vert \cdot\vert)(a)=\AAA(x,a)$
   for all $(x,a)^\top\in \mathbb{R}^d$, each solution  $u\in  \smash{W^{1,p(\cdot)}_D(\Omega)}$ of \eqref{eq:pDirichlet} is the unique minimizer of the functional ${I\colon {W^{1,p(\cdot)}_D(\Omega)}\to \mathbb{R}_{\ge 0}}$, for every $v\in{ W^{1,p(\cdot)}_D(\Omega)}$~defined~by
        \begin{align}
            I(v)\coloneqq \int_{\Omega}{\varphi(\cdot,\vert \nabla v\vert)\,\mathrm{d}x}-\int_{\Omega}{f\,v\,\mathrm{d}x}\,,\label{eq:pDirichletMin}
        \end{align}
        and vice-versa, leading to a primal and a dual formulation of \eqref{eq:pDirichlet}, and to convex~duality~relations.\enlargethispage{7mm}
        
The $p(\cdot)$-Dirichlet problem \eqref{eq:pDirichlet} is a prototypical example of a non-linear system with variable growth conditions. It  appears in physical models for smart fluids, e.g.,~\mbox{electro-rheological~fluids}, cf.\ \cite{RR96,Ru00}, micro-polar electro-rheological fluids, cf.\ \cite{winr,eringenbook}, chemically reacting fluids,~cf.~\cite{LKM78,HMPR10}, and thermo-rheological fluids, cf.\ \cite{Z97,AR06}. In these models, the variable exponent depends on~physical quantities such as, e.g., an electric field, a concentration field or a~temperature~field.~In~addition, the $p(\cdot)$-Dirichlet problem \eqref{eq:pDirichlet} has applications in the field of image reconstruction, cf. \cite{AMS08,CLR06,LLP10}. The main difficulty in treating models of type \eqref{eq:pDirichlet} is the non-autonomous structure (i.e., the dependency on the space variable).   In general, without additional assumptions on the space variable, there could be difficulties in numerical treatment of such problems. The essential feature of such models is the possibility of so-called energy gap (or Lavrentiev phenomenon), which could lead to the disconvergence of  conforming schemes to the global minimizer of the problem.  In comparison to the Lagrange elements, non-conforming methods, in particular, the Crouzeix--Raviart method, converges, see for example \cite{BalDieSto22}. In this paper, we consider the situation without energy gap, so additional regularity assumption on the variable exponent $p\in C^{0, \alpha}(\overline{\Omega})$ is needed. The obtained convergence rate result is new for  the non-conforming~\mbox{Crouzeix--Raviart}~method. 


   \subsection{Related contributions}\vspace{-1mm}

    \hspace{5mm}The numerical approximation of \eqref{eq:pDirichlet} in the case of a constant exponent $p\in (1,\infty)$ has already been subject of numerous contributions: 
    The best approximation property in terms of the natural distance for these kind of problems 
  has first been proven  for the $p$-Dirichlet problem \label{eq:pDirichlet}  in \cite{BarLiu94}.    This had been extended in \cite{DieRuz07} to the funtionals with Orlicz growth. A-priori estimates in terms of the mesh-size were obtained in~\cite{EbmLiu05,DieRuz07}. This was done in \cite{DieRuz07} by extending the approximation properties of standard interpolation operators to Orlicz spaces.  Results for generalized Newtonian fluids could be found  \cite{ST20}.    The~$p$-Laplace problem with DG methods was studied in~\cite{BE08} and 
\cite{BO09}; for~the~Orlicz~problems~in~\cite{DKRT14}.~In~\cite{BelDieKre12},  it has been shown that the  adaptive finite element method with element-wise affine functions and D\"orfler marking converges with optimal rates. As the problem is non-linear, it has to be solved by iterative methods. A \hspace{-0.1mm}stable \hspace{-0.1mm}procedure, \hspace{-0.1mm}that \hspace{-0.1mm}is \hspace{-0.1mm}very \hspace{-0.1mm}efficient \hspace{-0.1mm}in \hspace{-0.1mm}the \hspace{-0.1mm}experiments,~\hspace{-0.1mm}was~\hspace{-0.1mm}introduced \hspace{-0.1mm}in \hspace{-0.1mm}\cite{DieForTomWan20}~\hspace{-0.1mm}for~\hspace{-0.1mm}the~\hspace{-0.1mm}case~\hspace{-0.1mm}${p\hspace{-0.1em}<\hspace{-0.1em}2}$ and in~\cite{BalDieSto22} for the case~$p>2$. In~\cite{K22CR}, an  error analysis for a Crouzeix--Raviart approximation of the $p$-Dirichlet problem was carried out, including optimal a priori~error estimates,~a~medius~error estimate, i.e., a best-approximation~result,~and~a~posteriori~error~estimates. 
  
    However, there are only a few contributions in the case of a variable exponent $p\in C^0(\overline{\Omega})$. The paper \cite{CHP10} is concerned with the (weak) convergence of a conforming, discretely inf-sup stable finite element approximation of the model for electro-rheological fluids. The first contribution addressing a priori error estimates for finite element~approximations~of~\eqref{eq:pDirichlet}~can~be~found~in~\cite{BDS13}; see also \cite{BBD15}, for a extension to the model for electro-rheological fluids.
    
    In \cite{DPLM13}, the (weak)~conver-gence of discontinuous Galerkin type methods is studied; the result~contains~no~convergence~rates.   In \cite{KPS18,KS19}, the (weak) convergence of a conforming, discretely inf-sup stable finite element~approx-imation of the model for chemically reacting %(i.e., synovial) 
    fluids is proved.  In  \cite{BalOrtSto22}, the weak convergence of an approximation using the Crouzeix--Raviart element was established.
    To~the~best~of~the~authors'~knowledge, however, no a priori error analysis for \eqref{eq:pDirichlet} for approximations using non-conforming ansatz~classes --in particular, the Crouzeix--Raviart element, which is usually the first step towards a fully-non-conforming a priori error analysis-- has been carried  out yet.\vspace{-1mm}\enlargethispage{11mm}

   \subsection{New contributions}\vspace{-1mm}

    \hspace{5mm}Deriving local efficiency estimates in terms of shifted $N$-functions and deploying the so-called node-averaging quasi-interpolation operator, cf. \cite{Osw93,Sus96}, we generalize the medius error analysis in 
    \cite{Bre15} from $p=2$ and $\delta=0$ in \eqref{example}, i.e., $\AAA=\textup{id}_{\mathbb{R}^d}\colon \mathbb{R}^d\to \mathbb{R}^d$, and of \cite{K22CR} from $p\in (1,\infty)$ and $\delta\ge 0$ in \eqref{example}
    to variable exponents $p\in C^0(\overline{\Omega})$ with $p^->1$.
    This medius error analysis implies that the performances of the 
    conforming Lagrange finite element method applied to \eqref{eq:pDirichlet} and the non-conforming Crouzeix--Raviart finite element method applied~to~\eqref{eq:pDirichlet}~are~comparable.~As~a~result, we obtain a priori error estimates for the approximation of \eqref{eq:pDirichlet}
    using 
    Crouzeix--Raviart  element, which apply for Hölder continuous variable exponents 
    $p\in C^{0,\alpha}(\overline{\Omega})$, where $\alpha\in (0,1]$ and $p^->1$, and $\delta\ge  0$, and are optimal for Lipschitz continuous variable exponents $p\in C^{0,1}(\overline{\Omega})$~and~$\delta> 0$. Since $\AAA\colon\Omega\times \mathbb{R}^d\to \mathbb{R}^d$ has a potential and, therefore, \eqref{eq:pDirichlet} admits an equivalent formulation as a convex minimization problem, cf. \eqref{eq:pDirichletMin},  we have access to a (discrete) convex duality theory, cf.~\cite{LLC18,Bar21,BK22B}, and \eqref{eq:pDirichlet} as well as the approximation of \eqref{eq:pDirichlet} using the Crouzeix--Raviart  element admit dual formulations with a dual solution as well as a discrete~dual~solution,~respectively. We derive a priori error estimates for the dual solution and the discrete~dual~solution.\vspace{-1mm}%, measured in the so-called conjugate natural distance.\vspace{-1mm}
 
  	
  \subsection{Outline}\vspace{-1mm}

  \hspace{5mm}\textit{This article is organized as follows:} \!In Section \ref{sec:preliminaries}, we introduce the employed notation,~the~\mbox{basic} properties of non-linear operator $\AAA\colon\Omega\times\mathbb{R}^d\to \mathbb{R}^d$ and its  consequences, the relevant finite element spaces, and give brief review of the continuous and the discrete $p(\cdot)$-Dirichlet~problem. 
   In~Section~\ref{sec:medius}, we establish a medius error analysis, i.e., best-approximation result, for the Crouzeix--Raviart finite element method applied to \eqref{eq:pDirichlet}.
   In Section \ref{sec:a_priori}, by~means~of~this~medius~error~analysis, we derive a priori error estimates for the  Crouzeix--Raviart finite element method applied~to~\eqref{eq:pDirichlet}, which are optimal for all Lipschitz continuous exponents $p\in C^{0,1}(\overline{\Omega})$ with $p^->1$ and all $\delta>0$. 
   In Section \ref{sec:experiments}, we review our theoretical findings via numerical experiments.
   
  	 
	\section{Preliminaries}\label{sec:preliminaries}\enlargethispage{7.5mm}
	
	\qquad Throughout the article, let ${\Omega\subseteq \mathbb{R}^d}$,~${d\in\mathbb{N}}$, always be a bounded polyhedral Lipschitz domain, whose topological boundary $\partial\Omega$ is disjointly divided into a closed Dirichlet part $\Gamma_D$, for which we always assume that $\vert \Gamma_D\vert>0$,~and~a~Neumann~part~$\Gamma_N$,~i.e., ${\partial\Omega=\Gamma_D\cup\Gamma_N}$~and~${\emptyset=\Gamma_D\cap\Gamma_N}$. 
 The integral mean of a locally integrable function $f\colon\Omega\to\mathbb{R}$ over a (Lebesgue)~measurable~set~${M\subseteq \Omega}$ is denoted by $\fint_M{ f \,\textup{d}x} \coloneqq \smash{\frac 1 {|M|}\int_M f \,\textup{d}x}$. For (Lebesgue) measurable functions $f,g\colon\Omega\to \mathbb{R}$ and a (Lebesgue) measurable set $M\subseteq \Omega$,
 we write $(f,g)_M\coloneqq \int_M f g\,\textup{d}x$, 
 whenever~the~right-hand side is well-defined. We use the notation~$\wedge$ and $\vee$, for the minimum and maximum,~respectively. 
 

 \subsection{Variable Lebesgue spaces and variable Sobolev spaces}

 \qquad Let $\Omega\subseteq \mathbb{R}^d$, $d\in \mathbb{N}$, be an open set 
and $p\colon \Omega\to [1,+\infty]$ be a (Lebesgue)~\mbox{measurable}~function, a \hspace{-0.1mm}so-called \hspace{-0.1mm}\textit{variable
\hspace{-0.1mm}exponent}. \hspace{-0.1mm}By \hspace{-0.1mm}$\mathcal{P}(\Omega)$, \hspace{-0.1mm}we \hspace{-0.1mm}denote \hspace{-0.1mm}the \hspace{-0.1mm}\textit{set \hspace{-0.1mm}of \hspace{-0.1mm}variable~exponents}.~\hspace{-0.1mm}Then,~\hspace{-0.1mm}for~\hspace{-0.1mm}${p\in \mathcal{P}(\Omega)} $, we denote by
${p^+\coloneqq \textup{ess\,sup}_{x\in
 \Omega}{p(x)}}$~and~${p^-\coloneqq \textup{ess\,inf}_{x\in
 \Omega}{p(x)}}$ its constant~\textit{limit~exponents}.~Then,~by
$\mathcal{P}^{\infty}(\Omega)\coloneqq \{p\in\mathcal{P}(\Omega)\mid
p^+<\infty\}$, we denote the \textit{set of bounded variable exponents}.~For~${p\in\mathcal{P}^\infty(\Omega)}$ and $l\in \mathbb{N}$, \hspace{-0.1mm}we \hspace{-0.1mm}denote \hspace{-0.1mm}by \hspace{-0.1mm}$L^{p(\cdot)}(\Omega;\mathbb{R}^l)$, \hspace{-0.1mm}the \textit{\hspace{-0.1mm}variable Lebesgue \hspace{-0.1mm}space},~\hspace{-0.1mm}i.e.,~\hspace{-0.1mm}the~\hspace{-0.1mm}vector~\hspace{-0.1mm}space~\hspace{-0.1mm}of~\hspace{-0.1mm}(Lebesgue) measurable functions $v\colon\Omega\to \mathbb{R}^l$~for~which~the~\textit{modular}~${\rho_{p(\cdot),\Omega}(v)\coloneqq \smash{\int_{\Omega}{\vert v\vert^{p(\cdot)}\,\mathrm{d}x}}<\infty}$.
Then, the \textit{Luxembourg norm} $\| v\|_{p(\cdot),\Omega}\coloneqq \inf\{\lambda> 0\mid \rho_{p(\cdot),\Omega}(\frac{v}{\lambda})\leq 1\}$ turns $L^{p(\cdot)}(\Omega;\mathbb{R}^l)$~into~a~\mbox{Banach}~space. 

Moreover, for $p\in\mathcal{P}^\infty(\Omega)$ and $l\in \mathbb{N}$, we define the spaces
\begin{align}\label{eq:spaces}
		\begin{aligned}
		W^{1,p(\cdot)}_D(\Omega;\mathbb{R}^l)&\coloneqq \big\{v\in L^{p(\cdot)}(\Omega;\mathbb{R}^l)&&\hspace*{-3.25mm}\mid \nabla v\in L^{p(\cdot)}(\Omega;\mathbb{R}^{l\times d})\,,\; \textup{tr}\,v=0\text{ in }L^{p^-}(\Gamma_D;\mathbb{R}^l)\big\}\,,\\
		W^{p(\cdot)}_N(\textup{div};\Omega)&\coloneqq \big\{y\in L^{p(\cdot)}(\Omega;\mathbb{R}^d)&&\hspace*{-3.25mm}\mid \textup{div}\,y\in L^{p(\cdot)}(\Omega)\,,\;\textup{tr}_n\,y=0\text{ in }W^{\smash{-\frac{1}{p^-},p^-}}(\Gamma_N)\big\}\,,
	\end{aligned}
\end{align}
$W^{1,p(\cdot)}(\Omega;\mathbb{R}^l)\coloneqq W^{1,p(\cdot)}_D(\Omega;\mathbb{R}^l)$ if $\Gamma_D=\emptyset$, and $W^{p(\cdot)}(\textup{div};\Omega)\coloneqq W^{p(\cdot)}_N(\textup{div};\Omega)$ if $\Gamma_N=\emptyset$. Here,  $\textup{tr}\colon \smash{W^{1,p(\cdot)}(\Omega;\mathbb{R}^l)}\to\smash{L^{p^-}(\partial\Omega;\mathbb{R}^l)}$ and by $
	\textup{tr}_n\colon\smash{W^{p(\cdot)}(\textup{div};\Omega)}\to W^{\smash{-\frac{1}{p^-},p^-}}(\partial\Omega)$ denote the \textit{trace operator} and the \textit{normal trace operator}, respectively. In particular, we will always omit $\textup{tr}$ and $\textup{tr}_n$. %Furthermore, 
 We write $L^{p(\cdot)}(\Omega) \coloneqq L^{p(\cdot)}(\Omega;\mathbb{R}^1)$, ${W^{1,p(\cdot)}(\Omega)\coloneqq W^{1,p(\cdot)}(\Omega;\mathbb{R}^1)}$,~and~${W^{1,p(\cdot)}_D(\Omega)\coloneqq W^{1,p(\cdot)}_D(\Omega;\mathbb{R}^1)}$.

	
 \subsection{(Generalized) $N$-functions}
	
	\qquad A \hspace{-0.1em}(real) \hspace{-0.1em}convex function
 $\psi\colon\hspace{-0.1em}\mathbb{R}_{\geq 0} \to \mathbb{R}_{\geq 0}$ is called
 \textit{$N$-function},~if~${\psi(0)=0}$,~${\psi(t)>0}$~for~all~${t>0}$,
 $\lim_{t\rightarrow0} \psi(t)/t=0$, and
 $\lim_{t\rightarrow\infty} \psi(t)/t=\infty$. If, in addition, $\psi\in C^1(\mathbb{R}_{\geq 0})\cap C^2(\mathbb{R}_{> 0})$~and~${\psi''(t)\hspace{-0.1em}>\hspace{-0.1em}0}$ for \hspace{-0.1mm}all \hspace{-0.1mm}$t>0$, \hspace{-0.1mm}we \hspace{-0.1mm}call \hspace{-0.1mm}$\psi$ \hspace{-0.1mm}a \hspace{-0.1mm}\textit{regular
  \hspace{-0.1mm}$N$-function}. \hspace{-0.1mm}For \hspace{-0.1mm}a \hspace{-0.1mm}regular \hspace{-0.1mm}$N$-function \hspace{-0.1mm}${\psi \colon \hspace{-0.1em}\mathbb{R}_{\geq 0}\to \mathbb{R}_{\geq 0}}$,~\hspace{-0.1mm}we~\hspace{-0.1mm}have~\hspace{-0.1mm}that $\psi (0)=\psi'(0)=0$,
 $\psi'\colon\hspace{-0.1em}\mathbb{R}_{\geq 0} \to \mathbb{R}_{\geq 0}$ is increasing and $\lim _{t\to \infty} \psi'(t)=\infty$.~For~a~given~\mbox{$N$-function} ${\psi \colon\mathbb{R}_{\geq 0} \to \mathbb{R}_{\geq 0}}$, we define the \textit{(Fenchel) conjugate \mbox{$N$-function}} $\psi^*\colon\mathbb{R}_{\geq 0} \to \mathbb{R}_{\geq 0}$,~for~every~$t\ge 0$,~by
 ${\psi^*(t)\coloneqq \sup_{s \geq 0} (st
  -\psi(s))}$, which satisfies $(\psi^*)' =
 (\psi')^{-1}$ in $\mathbb{R}_{\ge 0}$. An $N$-function $\psi$ satisfies the \textit{$\Delta_2$-condition}
 (in short, $\psi \hspace*{-0.1em}\in\hspace*{-0.1em} \Delta_2$), if there exists $K\hspace*{-0.1em}>\hspace*{-0.1em} 2$ such that~for~all~${t \hspace*{-0.1em}\ge\hspace*{-0.1em} 0}$,~it~holds~${\psi(2\,t) \hspace*{-0.1em}\leq\hspace*{-0.1em} K\, \psi(t)}$. Then, \hspace*{-0.15mm}we \hspace*{-0.15mm}denote \hspace*{-0.15mm}the
 \hspace*{-0.1mm}smallest \hspace*{-0.15mm}such \hspace*{-0.15mm}constant \hspace*{-0.15mm}by \hspace*{-0.15mm}$\Delta_2(\psi)\hspace*{-0.15em}>\hspace*{-0.15em}0$. \hspace*{-0.15mm}We \hspace*{-0.15mm}say \hspace*{-0.15mm}that \hspace*{-0.15mm}an \hspace*{-0.15mm}$N$-function~\hspace*{-0.15mm}${\psi\colon\hspace*{-0.1em}\mathbb{R}_{\ge 0}\hspace*{-0.15em}\to \hspace*{-0.15em}\mathbb{R}_{\ge 0}}$ satisfies the \textit{$\nabla_2$-condition} (in short, $\psi\in \nabla_2$), if its (Fenchel) conjugate $\psi^*\colon\mathbb{R}_{\ge 0}\to \mathbb{R}_{\ge 0}$ is an $N$-function satisfying the $\Delta_2$-condition. 
 If $\psi\colon\mathbb{R}_{\ge 0}\to \mathbb{R}_{\ge 0}$ satisfies the $\Delta_2$- and the $\nabla_2$-condition (in~short, $\psi\in \Delta_2\cap \nabla_2$), then, there holds 
 %the %following refined version~of~
 the~\mbox{\textit{$\varepsilon$-Young}}~\textit{inequality}: for every
 $\varepsilon> 0$, there exists a constant $c_\varepsilon>0 $, depending only on
 $\Delta_2(\psi),\Delta_2( \psi ^*)<\infty$, such that for every $ s,t\geq0 $, it holds
 \begin{align}
  \label{ineq:young}
 s\,t\leq  c_\varepsilon \,\psi^*(s)+\varepsilon \, \psi(t)\,.
 \end{align}
 A function $\psi \colon \Omega \times \mathbb{R}_{\ge 0} \to \mathbb{R}_{\ge 0}$ is called \textit{generalized $N$-function} if it is Carath\'eodory mapping and $\psi(x,\cdot)\colon\hspace{-0.15em} \mathbb{R}_{\ge 0}\hspace{-0.15em}\to \hspace{-0.15em}\mathbb{R}_{\ge 0}$ is for a.e.\ $x \hspace{-0.15em}\in\hspace{-0.15em} \Omega$ an
 $N$-function. For a generalized $N$-function~${\psi \colon \hspace{-0.15em}\Omega\hspace{-0.15em} \times\hspace{-0.15em} \mathbb{R}_{\ge 0} \hspace{-0.15em}\to \hspace{-0.15em}\mathbb{R}_{\ge 0}}$, a (Lebesgue) measurable function $f\colon\Omega\to \mathbb{R}$, and a (Lebesgue) measurable set $M\subseteq \Omega$, we write $\rho_{\psi,M}(f)\coloneqq \int_M \psi(\cdot,\vert f\vert )\,\textup{d}x $, 
 whenever~the~\mbox{right-hand}~side~is~\mbox{well-defined}.
 
 \subsection{Basic properties of the non-linear operators}\label{sec:basic}\enlargethispage{2mm}
 
 \qquad Throughout the entire article, we always assume that  $\AAA\colon\Omega\times \mathbb{R}^d\to \mathbb{R}^d$ has \textit{$(p(\cdot),\delta)$-structure}, where $p\in \mathcal{P}^{\infty}(\Omega)$ with $p^->1$ and $\delta\ge 0$, i.e., for a.e.\ $x\in \Omega$ and~every~${a\in \mathbb{R}^d}$,~it~holds
 \begin{align}\label{def:A}
 \AAA(x,a)\coloneqq (\delta+\vert a\vert )^{p(x)-2}a\,.
 \end{align}
 For given $p\in \mathcal{P}^{\infty}(\Omega)$ with $p^->1$ and $\delta\ge 0$, we introduce the special generalized $N$-function
$\varphi\coloneqq \varphi_{p,\delta}\colon\Omega\times\mathbb{R}_{\ge 0}\to \mathbb{R}_{\ge 0}$, for a.e.\ $x\in \Omega$ and all $t\ge 0$, defined by
\begin{align} 
 \label{eq:def_phi} 
 \varphi(x,t)\coloneqq \int _0^t \varphi'(x,s)\, \mathrm ds\,,\quad\text{where}\quad
 \varphi'(x,t) \coloneqq (\delta +t)^{p(x)-2} t\,.
\end{align}
For (Lebesgue) measurable functions $f,g\colon\Omega \to \mathbb{R}_{\ge 0}$, we write
$f\sim g$ (or $f\lesssim g$) if~there~exists~a constant \hspace{-0.1mm}$c\hspace{-0.1em}>\hspace{-0.1em}0$ \hspace{-0.1mm}such \hspace{-0.1mm}that \hspace{-0.1mm}$c^{-1}g\hspace{-0.1em}\leq\hspace{-0.1em} f\hspace{-0.1em}\leq\hspace{-0.1em} c\,g$ \hspace{-0.1mm}(or \hspace{-0.1mm}$ f\hspace{-0.1em}\leq\hspace{-0.1em} c\,g$) \hspace{-0.1mm}a.e.\ \hspace{-0.1mm}in \hspace{-0.1mm}$\Omega$. \hspace{-0.1mm}In \hspace{-0.1mm}particular,~i\hspace{-0.1mm}f~\hspace{-0.1mm}not~\hspace{-0.1mm}otherwise~\hspace{-0.1mm}specified, we always assume that the hidden constant in $\sim$ and $\lesssim$ depends only on $p^-,p^+>1$ and $\delta\ge 0$. % , and the chunkiness $\omega_0>0$, cf.Subsection \ref{subsec:finite_elements}.

Then, $\varphi\colon \Omega\times\mathbb{R}_{\ge 0}\to \mathbb{R}_{\ge 0}$ satisfies, uniformly in $\delta\ge  0$ and a.e.\ $x\in \Omega$, the
$\Delta_2$-condition~with $\textup{ess\,sup}_{x\in \Omega}{\Delta_2(\varphi(x,\cdot))}\lesssim 2^{\smash{\max \{2,p^+\}}}$. In addition, 
the (Fenchel) conjugate function (with respect to the second argument) $\varphi^*\colon\Omega\times\mathbb{R}_{\ge 0}\to \mathbb{R}_{\ge 0}$ satisfies, uniformly~in~both~${t \ge 0}$,~${\delta \ge 0}$,~and a.e.\ $x\in \Omega$, $\varphi^*(x,t) \sim
(\delta^{p(x)-1} + t)^{p'(x)-2} t^2$ and the $\Delta_2$-condition with
$\textup{ess\,sup}_{x\in \Omega}{\Delta_2(\varphi^*(x,\cdot))} \lesssim 2^{\smash{\max \{2,(p^-)'\}}}$.

For a generalized $N$-function $\psi\colon\Omega\times \mathbb{R}_{\ge 0}\to \mathbb{R}_{\ge 0}$, we introduce \textit{shifted generalized $N$-functions} $\psi_a\colon\Omega\times \mathbb{R}_{\ge 0}\to \mathbb{R}_{\ge 0}$, ${a\ge 0}$, for a.e.\ $x\in \Omega$ and all $a,t\ge 0$, defined by
\begin{align}
 \label{eq:phi_shifted}
 \psi_a(x,t)\coloneqq \int _0^t \psi_a'(x,s)\, \mathrm ds\,,\quad\text{where}\quad
 \psi'_a(x,t)\coloneqq \psi'(x,a+t)\frac {t}{a+t}\,.
\end{align}

\begin{remark} \label{rem:phi_a}
 For the above defined $N$-function $\varphi\colon \Omega\times \mathbb{R}_{\ge 0}\to \mathbb{R}_{\ge 0}$, cf.\ \!\eqref{eq:def_phi}, uniformly~in~${a,t\ge 0}$ and a.e.\ $x\in \Omega$, it holds
 $\varphi_a(x,t) \sim (\delta+a+t)^{p(x)-2} t^2$ and $(\varphi_a)^*(x,t)
 \sim ((\delta+a)^{p(x)-1} + t)^{\smash{p'(x)-2}} t^2$. The \hspace{-0.1mm}families
 \hspace{-0.1mm}$\{\varphi_a\}_{\smash{a \ge 0}},\{(\varphi_a)^*\}_{\smash{a \ge 0}}\colon\hspace{-0.15em}\Omega\hspace{-0.05em}\times\hspace{-0.05em} \mathbb{R}_{\ge 0}\hspace{-0.15em}\to \hspace{-0.15em}\mathbb{R}_{\ge 0}$ \hspace{-0.1mm}satisfy, \hspace{-0.1mm}uniformly \hspace{-0.1mm}in \hspace{-0.1mm}$a\hspace{-0.15em}\ge\hspace{-0.15em} 0$,~\hspace{-0.1mm}the~\hspace{-0.1mm}\mbox{$\Delta_2$-condition}~\hspace{-0.1mm}with ${\textup{ess\,sup}_{x\in \Omega}{\Delta_2(\varphi_a(x,\cdot))} \lesssim 2^{\smash{\max \{2,p^+\}}}}\!$ and
 ${\textup{ess\,sup}_{x\in \Omega}{\Delta_2((\varphi_a)^*(x,\cdot))} \lesssim 2^{\smash{\max \{2,(p^-)'\}}}}\!$,~respectively.
\end{remark}

Closely related to $\AAA\colon\Omega\times\mathbb{R}^d\to \mathbb{R}^d$ defined by \eqref{def:A}~are~the~non-linear operators $F,F^*\colon\Omega\times\mathbb{R}^d\to \mathbb{R}^d$, for a.e.\ $x\in \Omega$ and every $a\in \mathbb{R}^d$ defined by
\begin{align}
 F(x,a)\coloneqq\smash{(\delta+\vert a\vert)^{\frac{p(x)-2}{2}}a}\,,\qquad F^*(x,a)\coloneqq \smash{(\delta^{p(x)-1}+\vert a\vert)^{\frac{p'(x)-2}{2}}a}
 \,.\label{eq:def_F}
\end{align}


The relations between
$\AAA,F,F^*\colon\Omega\times\mathbb{R}^d
\to \mathbb{R}^d$ and
$\varphi_a,(\varphi^*)_a,(\varphi_a)^*\colon\Omega\times\mathbb{R}_{\ge
 0}\to \mathbb{R}_{\ge
 0}$,~${a\ge 0}$, are presented in
 the following proposition.

\begin{proposition}
 \label{lem:hammer}
 Uniformly in a.e.\ $ x,y\in \Omega$, $t\ge 0$, and 
 $a, b \in \mathbb{R}^d$, we have that
 \begin{align}\label{eq:hammera}
 \begin{aligned}
  \hspace{-8mm}(\AAA(x,a) - \AAA(x,b))
  \cdot(a-b ) &\sim \smash{\vert F(x,a) - F(x,b)\vert^2}
  \\&
  \sim \varphi_{\vert a\vert }(x,\vert a - b\vert )
  %\\
  %&\sim (\varphi_{\vert a\vert })^*(x,\vert \AAA(x,a) - \AAA(x,b)\vert )
 % \\&
  %\sim (\varphi^*)_{\vert \AAA(x,a)\vert }(\vert \AAA(x,a) - \AAA(x,b)\vert )\notag
  %\\&\sim \smash{\vert F^*(x,\AAA(x,a)) - F^*(x,\AAA(x,b))\vert^2}
  \,,\\[-3mm]
  \end{aligned}
  \end{align}
  \begin{align}
  \smash{\vert F^*(x,a) - F^*(x,b)\vert^2}
  \label{eq:hammerf}
  &\sim \smash{\smash{(\varphi^*)}_{\smash{\vert a\vert }}(x,\vert a - b\vert )}\,,\\
   \label{eq:hammerg}
 \smash{\smash{(\varphi^*)}_{\smash{\vert \AAA(x,a)\vert }}(x,t)}
   &\sim \smash{\smash{(\varphi}_{\smash{\vert a\vert }})^*(x,t)}\,,\\
     \label{eq:hammerh}
\smash{\vert F^*(x,\AAA(x,a)) - F^*(x,\AAA(y,b))\vert^2}
   &\sim \smash{\smash{(\varphi}_{\smash{\vert a\vert }})^*(x,\vert \AAA(x,a)-\AAA(y,b)\vert)}\,.
 \end{align}
\end{proposition} 

\begin{proof}
 For the equivalences \eqref{eq:hammera}--\eqref{eq:hammerg}, we refer to \cite[Remark A.9]{BDS15}. The equivalence 
 \eqref{eq:hammerh} follows from the equivalences \eqref{eq:hammerf} and \eqref{eq:hammerg}.
 \enlargethispage{4mm}
\end{proof}

In addition, we need the following shift change result.

\begin{lemma}\label{lem:shift-change}
 For every $\varepsilon>0$, there exists $c_\varepsilon \geq 1$ (depending only
 on~$\varepsilon>0$, $p^-,p^+>1$ and $\delta\ge 0$) such that for a.e.\ $x\in \Omega$, for every $a,b\in\mathbb{R}^d$ and for all $t\geq 0$, it holds
 \begin{align}
 \varphi_{\vert a\vert}(x,t)&\leq c_\varepsilon\, \varphi_{\vert b\vert }(x,t)
 +\varepsilon\, \vert F(x,a) - F(x,b)\vert^2\,,\label{lem:shift-change.1}
 \\
 (\varphi_{\vert a\vert})^*(x,t)&\leq c_\varepsilon\, (\varphi_{\vert b\vert })^*(x,t)
 +\varepsilon\, \vert F(x,a) - F(x,b)\vert^2\,.\label{lem:shift-change.3}
 \end{align}
\end{lemma}

\begin{proof}
 See \cite[Remark A.9]{BDS15}.
\end{proof}



\begin{remark}[Natural distance]
 \label{rem:natural_dist}
 Due to \eqref{eq:hammera}, uniformly in $u, v \in W^{1,p(\cdot)}(\Omega)$, it holds
 \begin{align*}
  (\AAA(\cdot,\nabla u) -
  \AAA(\cdot,\nabla v),\nabla u - \nabla v)_\Omega
  &\sim
 \|F(\cdot,\nabla u)-F(\cdot,\nabla v)\|_{2,\Omega}^2 \,\sim \rho_{\varphi_{\vert \nabla u\vert},\Omega}(\nabla u -
 \nabla v)\,.
 \end{align*}
 We refer to all three equivalent quantities as the \textup{natural distance}.  In particular, note that $\varphi_{\vert \nabla v\vert}\colon \Omega\times\mathbb{R}_{\ge 0}\to \mathbb{R}_{\ge 0}$ for every $v\in  W^{1,p(\cdot)}(\Omega)$ is a generalized $N$-function. 
\end{remark}

\begin{remark}[Conjugate natural distance]
 \label{rem:conjugate_natural_dist}
 For a.e.\ $x\in \Omega$, $\AAA(x,\cdot)\colon\mathbb{R}^d\to \mathbb{R}^d$ 
 is a continuous, strictly monotone and coercive operator, so that from the theory of monotone~operators, cf.~\cite{Zei90B}, it follows that $\AAA(x,\cdot)\colon\mathbb{R}^d\to \mathbb{R}^d$ for a.e.\ $x\in \Omega$ is bijective and its inverse $\AAA^{-1}(x,\cdot):\mathbb{R}^d\to \mathbb{R}^d$ continuous. Due to \eqref{eq:hammera}, uniformly in ${z, y \in L^{p'(\cdot)}(\Omega;\mathbb{R}^d)}$, it holds
 \begin{align*}
  (\AAA^{-1}(\cdot,z)-\AAA^{-1}(\cdot,y),z-y)_\Omega
  &\sim
 \|F^*(\cdot,z)-F^*(\cdot,y)\|_{2,\Omega}^2 \,\sim \rho_{(\varphi^*)_{\vert z\vert},\Omega}( z -
 y)\,.
 \end{align*}
 We refer to all
 three equivalent quantities as the \textup{conjugate natural distance}. In particular, note that $(\varphi^*)_{\vert y\vert}\colon \Omega\times\mathbb{R}_{\ge 0}\to \mathbb{R}_{\ge 0}$ for every $y\in L^{p'(\cdot)}(\Omega;\mathbb{R}^d)$ is a generalized $N$-function. 
\end{remark}
 	
 \subsection{Triangulations and standard finite element spaces}\label{subsec:finite_elements}
	
	%\qquad Throughout the entire paper, we denote by $\mathcal{T}_h$, $h>0$, a family~of~regular, i.e., uniformly shape regular and conforming, triangulations of $\Omega\subseteq \mathbb{R}^d$, $d\in\mathbb{N}$, cf.\ \!\cite{EG21}. 
	%Here,~${h>0}$~refers to the \textit{average mesh-size}, i.e., if we set $h_T\coloneqq \textup{diam}(T)$ for all $T\in \mathcal{T}_h$,~then,~we~have~that~${h\coloneqq \frac{1}{\textup{card}(\mathcal{T}_h)}\sum_{T\in \mathcal{T}_h}{h_T}}$.
	%For every element $T \in \mathcal{T}_h$, we denote by $\rho_T>0$, the supremum of diameters of~inscribed~balls. We assume that there exists a constant $\omega_0>0$, independent of $h>0$, such that $\max_{T\in \mathcal{T}_h}{h_T}{\rho_T^{-1}}\le \omega_0$. The smallest such constant is called the \textit{chunkiness} of $(\mathcal{T}_h)_{h>0}$. Also note that, in what follows, all constants may depend on the chunkiness, but are independent~of~${h>0}$. For~every~${T \in \mathcal{T}_h}$, let $\omega_T$ denote the \textit{patch} of~$T$, i.e., the union of all elements of~$\mathcal{T}_h$ touching~$T$.We assume that $\textup{int}(\omega_T)$ is connected for all $T\in \mathcal{T}_h$.  Under~these~assumptions, $\vert T\vert \sim\vert \omega_T\vert$ uniformly in $T\in \mathcal{T}_h$ and $h>0$, and the number of elements in $\omega_T$ and patches to which an element $T$ belongs to are uniformly bounded with respect to $T \in \mathcal{T}_h$~and~$h>0$. 

    \hspace{5mm}Throughout the article, let  $\mathcal{T}_h$, $h>0$, be a family of  uniformly shape regular and conforming, triangulations of $\Omega\subseteq \mathbb{R}^d$, $d\in\mathbb{N}$, cf.\ \cite{EG21}. Here,
	the parameter $h>0$ denotes~the~\textit{average~\mbox{mesh-size}}, i.e., $h\!\coloneqq  \!
 \smash{\frac{1}{\textup{card}(\mathcal{T}_h)}}\sum_{T\in \mathcal{T}_h}{h_T} 
 $,  where $h_T\!\coloneqq \!\textup{diam}(T)>0$ for all $T\!\in \!\mathcal{T}_h$.
  We assume~there~exists~$\omega_0\!>\!0$,  independent of $h\hspace{-0.1em}>\hspace{-0.1em}0$, such that $\max_{T\in \mathcal{T}_h}{h_T}{\smash{\rho_T^{-1}}}\hspace{-0.1em}\le\hspace{-0.1em}
    \omega_0$,~where~${\rho_T\hspace{-0.1em}\coloneqq \hspace{-0.1em}
    \sup\{r\hspace{-0.1em}>\hspace{-0.1em}0\mid \exists x\hspace{-0.1em}\in\hspace{-0.1em} T: B_r^d(x)\hspace{-0.1em}\subseteq \hspace{-0.1em}T\}}$ for all $T\in \mathcal{T}_h$. The smallest such constant is called the \textit{chunkiness} of $(\mathcal{T}_h)_{h>0}$.   For every ${T \in \mathcal{T}_h}$, 
    let $\omega_T\coloneqq \bigcup\{T'\in \mathcal{T}_h\mid T'\cap T\neq \emptyset\} $ denote the \textit{element patch} of $T$. Then, 
    we  assume that $\textup{int}(\omega_T)$ is connected for all $T\in \mathcal{T}_h$, so that $\textup{card}(\bigcup\{T'\in\mathcal{T}_h\mid T'\subseteq \omega_T\})+\textup{card}(\bigcup\{T'\in\mathcal{T}_h\mid T\subseteq \omega_{T'}\})\leq c $, where $c>0$ depends only on the chunkiness $\omega_0>0$, and $\vert T\vert \sim
    \vert \omega_T\vert$ for all  $T\in \mathcal{T}_h$. Eventually, we define the \textit{maximum mesh-size} by $h_{\max}\coloneqq \max_{T\in \mathcal{T}_h}{h_T}>0$.

 We define interior and boundary sides of $\mathcal{T}_h$ in the following way: an \textit{interior side} is the closure of the 
    non-empty relative interior of $\partial T \cap \partial T'$,~where~${T, T'\in \mathcal{T}_h}$~are~two~adjacent~elements.
    For an interior side $S\coloneqq 
    \partial T \cap \partial T'\in \mathcal{S}_h$, where $T,T'\in \mathcal{T}_h$, the \textit{side patch} is~defined~by~$\omega_S\coloneqq  T \cup
    T'$. A \textit{boundary side} is the closure of the non-empty relative interior of
    $\partial T \cap \partial \Omega$, where $T\in \mathcal{T}_h$ denotes a boundary~element~of~$\mathcal{T}_h$.  For a boundary side $S\coloneqq  \partial T \cap \partial
    \Omega$, the side patch is defined by  $\omega_S\coloneqq  T $. 
    Eventually, 
    by $\mathcal{S}_h^{i}$, we denote the set of 
 interior sides,
    and by $\mathcal{S}_h$, we denote~the~set~of~all~sides.
 %We define~the~\textit{sides}~of~$\mathcal{T}_h$~in~the~\mbox{following}~way: an interior side of $\mathcal{T}_h$ is the closure of the 
 %non-empty interior of $\partial T \cap \partial T'$, where $T, T'\in \mathcal{T}_h$~are~two~adjacent~elements.
 %For an interior side $S\coloneqq 
 %\partial T \cap \partial T'\in \mathcal{S}_h$, where $T,T'\in \mathcal{T}_h$, we employ the notation $\omega_S\coloneqq T \cup
 %T'$. A boundary side of $\mathcal{T}_h$ is the closure of the non-empty interior of
 %$\partial T \cap \partial \Omega$, where $T\in \mathcal{T}_h$ denotes a boundary~element~of~$\mathcal{T}_h$. For a boundary side $S\coloneqq \partial T \cap \partial
 %\Omega$, we employ the notation $\omega_S\coloneqq T $. By $\mathcal{S}_h^{i}$ and $\mathcal{S}_h$, we denote the sets of 
 %all interior~sides~and~the~set~of~all~sides,~respectively.~Eventually, 
 %we define the \textit{maximal mesh-size} 
 %$h_{\max}\coloneqq \max_{T\in \mathcal{T}_h}{h_T}$ and ${h_S\coloneqq \textup{diam}(S)}$ for~all~${S\in \mathcal{S}_h}$.

    For (Lebesgue) measurable~functions~${u,v\colon\mathcal{S}_h\to \mathbb{R}}$ and $\mathcal{M}_h\subseteq \mathcal{S}_h$, we write
    \begin{align*}
        (u,v)_{\mathcal{M}_h}\coloneqq \sum_{S\in \mathcal{M}_h}{(u,v)_S}\,,\quad\text{ where }(u,v)_S\coloneqq\int_S{uv\,\mathrm{d}s}\,,
    \end{align*}
    whenever all integrals are well-defined. Analogously, for  (Lebesgue) measurable vector fields $z,y\colon \mathcal{S}_h\hspace{-0.1em}\to\hspace{-0.1em} \mathbb{R}^d$ \hspace{-0.1mm}and \hspace{-0.1mm}$\mathcal{M}_h\subseteq \mathcal{S}_h$, we write ${(z,y)_{\mathcal{M}_h}\hspace{-0.15em}\coloneqq\hspace{-0.15em} \sum_{S\in \mathcal{M}_h}{(z,y)_S}}$,~where~${(z,y)_S\coloneqq\int_S{z\cdot y\,\mathrm{d}s}}$.\enlargethispage{3mm}%, whenever the right-hand side is well-defined. 
	
	For $k\in \mathbb{N}\cup\{0\}$ and $T\in \mathcal{T}_h$, let $\mathcal{P}_k(T)$ denote the set of polynomials of maximal~degree~$k$~on~$T$. Then, for $k\in \mathbb{N}\cup\{0\}$~and $l\in \mathbb{N}$, the sets of continuous and~\mbox{element-wise}~polynomial functions or vector~fields,~respectively, are defined by
	\begin{align*}
	\begin{aligned}
	\mathcal{S}^k(\mathcal{T}_h)^l&\coloneqq 	\big\{v_h\in C^0(\overline{\Omega};\mathbb{R}^l)\hspace*{-3mm}&&\mid v_h|_T\in\mathcal{P}_k(T)^l\text{ for all }T\in \mathcal{T}_h\big\}\,,\\
	\mathcal{L}^k(\mathcal{T}_h)^l&\coloneqq \big\{v_h\in L^\infty(\Omega;\mathbb{R}^l)\hspace*{-3mm}&&\mid v_h|_T\in\mathcal{P}_k(T)^l\text{ for all }T\in \mathcal{T}_h\big\}\,.
	\end{aligned}
	\end{align*}
	The \textit{element-wise constant mesh-size function} $h_\mathcal{T}\in \mathcal{L}^0(\mathcal{T}_h)$ is defined~by~${h_\mathcal{T}|_T\coloneqq h_T}$~for~all~${T\in \mathcal{T}_h}$.
	The \textit{side-wise constant mesh-size function} $h_\mathcal{S}\in \mathcal{L}^0(\mathcal{S}_h)$ is defined~by~${h_\mathcal{S}|_S\coloneqq h_S}$~for~all~${S\in \mathcal{S}_h}$,  where $h_S\coloneqq \textup{diam}(S)$ for all $S\in \mathcal{S}_h$.
	Denoting by $\mathcal{N}_h$, the set of all vertices of $\mathcal{T}_h$,~for~every~$T\in \mathcal{T}_h$ and $S\in \mathcal{S}_h$, we denote by $\smash{x_T\coloneqq \frac{1}{d+1}\sum_{z\in \mathcal{N}_h\cap T}{z}}$ and $\smash{x_S\coloneqq \frac{1}{d}\sum_{z\in \mathcal{N}_h\cap S}{z}}$, the  barycenters~of~$T$ and $S$, respectively. The \hspace{-0.2mm}\textit{(local) \hspace{-0.4mm}$L^2$\hspace{-0.2mm}-projection \hspace{-0.2mm}operator}  \hspace{-0.2mm}onto \hspace{-0.2mm}\mbox{element-wise} \hspace{-0.2mm}constant \hspace{-0.2mm}functions \hspace{-0.2mm}or~\hspace{-0.2mm}vector \hspace{-0.2mm}fields, \mbox{respectively}, is denoted by $\Pi_h\colon L^1(\Omega;\mathbb{R}^l)\to \mathcal{L}^0(\mathcal{T}_h)^l$.
	%\begin{align*}
	%\smash{\Pi_h\colon L^1(\Omega;\mathbb{R}^l)\to \mathcal{L}^0(\mathcal{T}_h)^l\,.}
	%\end{align*}
	%For every $v_h\in \smash{\mathcal{L}^1(\mathcal{T}_h)^l}$, it holds $\Pi_hv_h|_T=v_h(q_T)$ in $T$ for all $T\in \mathcal{T}_h$. 
 The \textit{element-wise~gradient} % operator}
 $\nabla_{\!h}\colon \mathcal{L}^1(\mathcal{T}_h)^l\to \mathcal{L}^0(\mathcal{T}_h)^{l\times d}$, for every $v_h\in \mathcal{L}^1(\mathcal{T}_h)^l$,~is~defined by $\nabla_{\!h}v_h|_T\coloneqq \nabla(v_h|_T)$ for all~$T\in \mathcal{T}_h$.\enlargethispage{12mm}
 
	
	\subsubsection{Crouzeix--Raviart element}\vspace{-0.5mm}
	
	\hspace{5mm}The Crouzeix--Raviart finite element space, cf.\ \cite{CR73}, is defined as the space of element-wise affine functions that are continuous in the barycenters of inner element sides, i.e.,\footnote{Here, for every inner element side $S\in\mathcal{S}_h^{i}$, $\jump{v_h}_S\coloneqq v_h|_{T_+}-v_h|_{T_-}$ on $S$, where $T_+, T_-\in \mathcal{T}_h$ satisfy $\partial T_+\cap\partial T_-=S$, and for every boundary side $S\in\mathcal{S}_h\setminus \mathcal{S}_h^{i}$, $\jump{v_h}_S\coloneqq v_h|_T$ on $S$, where $T\in \mathcal{T}_h$ satisfies $S\subseteq \partial T$.}
	\begin{align*}\mathcal{S}^{1,\textit{cr}}(\mathcal{T}_h)\coloneqq \big\{v_h\in \mathcal{L}^1(\mathcal{T}_h)\mid \jump{v_h}_S(x_S)=0\text{ for all }S\in \mathcal{S}_h^{i}\big\}\,.
	\end{align*}
	The Crouzeix--Raviart finite element space with homogeneous Dirichlet boundary condition is defined as the space of Crouzeix--Raviart functions that vanish in the barycenters of boundary sides that belong to  $\Gamma_D$, i.e.,
	\begin{align*}
			\mathcal{S}^{1,\textit{cr}}_D(\mathcal{T}_h)\coloneqq \big\{v_h\in\smash{\mathcal{S}^{1,\textit{cr}}(\mathcal{T}_h)}\mid v_h(x_S)=0\text{ for all }S\in \mathcal{S}_h\cap \Gamma_D\big\}\,.
	\end{align*}
	%In particular, we have that $	\smash{\mathcal{S}^{1,\textit{cr}}_D(\mathcal{T}_h)}=	\smash{\mathcal{S}^{1,\textit{cr}}(\mathcal{T}_h)}$ if $\Gamma_D=\emptyset$.
	
	\subsubsection{Raviart--Thomas element}\vspace{-0.5mm}
 
	\hspace{5mm}The lowest order Raviart--Thomas finite element space, cf.\ \cite{RT75}, is defined as the space of element-wise  affine vector fields that have continuous constant normal components on  inner elements sides, i.e.,\!\footnote{Here, for every inner element side $S\in\mathcal{S}_h\setminus\partial \Omega$, $\jump{y_h\cdot n}_S\coloneqq \smash{y_h|_{T_+}\cdot n_{T_+}+y_h|_{T_-}\cdot n_{T_-}}$ on $S$, where $T_+, T_-\in \mathcal{T}_h$ satisfy $\smash{\partial T_+\cap\partial T_-=S}$, and for every $T\in \mathcal{T}_h$, $\smash{n_T\colon\partial T\to \mathbb{S}^{d-1}}$ denotes the outward unit normal vector field~to~$ T$, 
	and for every boundary side $\smash{S\in\mathcal{S}_h\setminus \mathcal{S}_h^{i}}$, $\smash{\jump{y_h\cdot n}_S\coloneqq \smash{y_h|_T\cdot n}}$ on $S$, where $T\in \mathcal{T}_h$ satisfies $S\subseteq \partial T$ and $\smash{n\colon\partial\Omega\to \mathbb{S}^{d-1}}$ denotes the outward unit normal vector field to $\Omega$.}
	\begin{align*}
 \mathcal{R}T^0(\mathcal{T}_h)\coloneqq \big\{y_h\in \mathcal{L}^1(\mathcal{T}_h)^d\mid &\,\smash{y_h|_T\cdot  n_T=\textup{const}\text{ on }\partial T\text{ for all }T\in \mathcal{T}_h\,,}\\ 
 &\smash{	\jump{y_h\cdot n}_S=0\text{ on }S\text{ for all }S\in \mathcal{S}_h^{i}\big\}\,.}
	\end{align*}
    The	Raviart--Thomas finite element space with homogeneous slip boundary condition is defined as the space of Raviart--Thomas functions that have vanishing normal components~on~$\Gamma_N$, i.e.,
	\begin{align*}
		\mathcal{R}T^{0}_N(\mathcal{T}_h)\coloneqq \big\{y_h\in	\mathcal{R}T^0(\mathcal{T}_h)\mid y_h\cdot n=0\text{ on }\Gamma_N\big\}\,.
	\end{align*}
	%In particular, we have that $\smash{\mathcal{R}T^{0}_N(\mathcal{T}_h)}=\mathcal{R}T^0(\mathcal{T}_h)$ if $\Gamma_N=\emptyset$. 
% Note that $\mathcal{R}T^{0}_N(\mathcal{T}_h)\subseteq W^\infty_N(\textup{div};\Omega)$.
	
	\subsubsection{Discrete integration-by-parts formula}\vspace{-0.5mm}
 
 	\hspace{5mm}For every $v_h\in \mathcal{S}^{1,\textit{cr}}(\mathcal{T}_h)$ and ${y_h\in \mathcal{R}T^0(\mathcal{T}_h)}$, we have the \textit{discrete integration-by-parts
	formula}
	\begin{align}
	(\nabla_{\!h}v_h,\Pi_h y_h)_\Omega+(\Pi_h v_h,\,\textup{div}\,y_h)_\Omega=(v_h,y_h\cdot n)_{\partial\Omega}\,,\label{eq:pi}
	\end{align}
	which follows from the fact that, by definition,  for every $y_h\in \mathcal{R}T^0(\mathcal{T}_h)$, it holds $y_h|_T\cdot n_T=\textrm{const}$ on $\partial T$ for all $T\in \mathcal{T}_h$ and	$\jump{y_h\cdot n}_S=0$ on $S$ for all $S\in \mathcal{S}_h^{i}$, and for every~${v_h\in \mathcal{S}^{1,\textit{\textrm{cr}}}(\mathcal{T}_h)}$,~it~holds $\int_{S}{\jump{v_h}_S\,\textup{d}s}\hspace{-0.15em}=\hspace{-0.15em}\jump{v_h}_S(x_S)\hspace{-0.15em}=\hspace{-0.15em}0$ for all $S\hspace{-0.15em}\in\hspace{-0.15em} \mathcal{S}_h^{i}$.
%	As a result, 
 For~every~$v_h\hspace{-0.15em}\in\hspace{-0.15em} \smash{\mathcal{S}^{1,\textit{\textrm{cr}}}_D(\mathcal{T}_h)}$~and~${y_h\hspace{-0.15em}\in \hspace{-0.15em}\smash{\mathcal{R}T^0_N(\mathcal{T}_h)}}$,~\eqref{eq:pi}~reads
	\begin{align}
		(\nabla_{\!h}v_h,\Pi_h y_h)_\Omega=-(\Pi_h v_h,\,\textup{div}\,y_h)_\Omega\,.\label{eq:pi0}
	\end{align}
	%In \cite{Bar21,BKAFEM22}, the integration-by-parts formula \eqref{eq:pi0} formed a cornerstone in the derivation~of~\mbox{discrete} convex duality relations and, as such, plays a central 
	%role 
	%in the hereinafter analysis.~
    In~addition, cf.\ \cite[Section 2.4]{BW21}, there holds the orthogonal decomposition
    \begin{align}
        \mathcal{L}^0(\mathcal{T}_h)^d=\textup{ker}(\textup{div}|_{\smash{\mathcal{R}T^0_N(\mathcal{T}_h)}})\oplus \nabla_{\!h}(\smash{\mathcal{S}^{1,\textit{\textrm{cr}}}_D(\mathcal{T}_h)})\,,\label{eq:decomposition}
    \end{align}
    which shows that for every vector field $y_h\in \mathcal{L}^0(\mathcal{T}_h)^d$, the following implication applies:
    \begin{align}
        (y_h,\nabla_{\!h} v_h)_{\Omega}=0\quad\text{ for all }v_h\in \mathcal{S}^{1,\textit{\textrm{cr}}}_D(\mathcal{T}_h)\qquad\Rightarrow \qquad y_h\in \mathcal{R}T^0_N(\mathcal{T}_h)\cap \mathcal{L}^0(\mathcal{T}_h)^d\,.\label{eq:decomposition.1}
    \end{align}
    The implication \eqref{eq:decomposition.1} is of crucial importance in the derivation of discrete strong duality relations and a discrete recontruction formula in Proposition \ref{prop:discrete_convex_duality}.
	 
 \subsection{$p(\cdot)$-Dirichlet problem}
	
	\qquad In this section, we briefly review  the variational formulation, the primal formulation, and the dual formulation of the $p(\cdot)$-Dirichlet problem \eqref{eq:pDirichlet}.\vspace{-2mm}
	
	\subsubsection{Variational problem}\vspace{-0.5mm}
		
		\qquad Given 
		\hspace{-0.1mm}a \hspace{-0.1mm}right-hand \hspace{-0.1mm}side \hspace{-0.1mm}$f\hspace{-0.1em}\in \hspace{-0.1em} L^{p'(\cdot)}(\Omega)$, \hspace{-0.1mm}where \hspace{-0.1mm}$p\hspace{-0.1em}\in\hspace{-0.1em} \mathcal{P}^{\infty}(\Omega)$ \hspace{-0.1mm}with \hspace{-0.1mm}$p^-\hspace{-0.1em}>\hspace{-0.1em}1$, \hspace{-0.1mm}the \hspace{-0.1mm}\textit{$p(\cdot)$-Dirichlet~\hspace{-0.1mm}\mbox{problem}} seeks for a function ${u\in \smash{W^{1,p(\cdot)}_D(\Omega)}}$ such that for every $v\in \smash{W^{1,p(\cdot)}_D(\Omega)}$, it holds
		\begin{align}
		 (\AAA(\cdot,\nabla u),\nabla v)_\Omega=(f,v)_\Omega\,.\label{eq:pDirichletW1p}
		\end{align}
		The theory of monotone operators, cf.\ \cite{Zei90B,Ru04}, proves the existence of a  unique solution to \eqref{eq:pDirichletW1p}. In what follows, we~reserve~the~notation~$u\in \smash{W^{1,p(\cdot)}_D(\Omega)}$~for~this~solution.\vspace{-1mm}
		
	\subsubsection{Minimization problem and convex duality relations}
	
	\hspace{5mm}The variational problem \eqref{eq:pDirichletW1p} emerges as an optimality condition of an equivalent convex minimization problem, leading to a primal and a dual formulation, and convex duality relations.\smallskip
	
	\hspace{5mm}\textit{Primal problem.} The problem \eqref{eq:pDirichletW1p} is equivalent to the minimization of the \textit{$p(\cdot)$-Dirichlet energy}, i.e.,  $I\colon \smash{W^{1,p(\cdot)}_D(\Omega)}\to \mathbb{R}$, for every $v\in \smash{W^{1,p(\cdot)}_D(\Omega)}$ defined by
	\begin{align}\label{eq:pDirichletPrimal}
	 I(v)\coloneqq \rho_{\varphi,\Omega}( \nabla v)-(f,v)_\Omega\,.
	\end{align}
	We will always refer to the minimization of the $p(\cdot)$-Dirichlet energy \eqref{eq:pDirichletPrimal}~as~the~\textit{primal~\mbox{problem}}.
	Since the $p(\cdot)$-Dirichlet energy \eqref{eq:pDirichletPrimal} is proper, 
	strictly convex, weakly coercive, 
	and~lower~semi-continuous, the existence of a unique minimizer of \eqref{eq:pDirichletPrimal}, called the \textit{primal solution}, follows using
 the direct method in the calculus of variations, cf.\ \cite{Dac08}.
  In particular,
	since the $p(\cdot)$-Dirichlet~energy is Fr\'echet~differentiable~and for every $\smash{v,w\in W^{1,p(\cdot)}_D(\Omega)}$, it holds
	\begin{align*}
	 \langle DI(v),w\rangle_{\smash{W^{1,p(\cdot)}_D(\Omega)}}=(\AAA(\cdot,\nabla v),\nabla w)_\Omega\,,
	\end{align*}
	the optimality conditions of the primal problem and the strict convexity of the $p(\cdot)$-Dirichlet~energy imply that $\smash{u\in W^{1,p(\cdot)}_D(\Omega)}$ is the unique minimizer of the $p(\cdot)$-Dirichlet energy.\smallskip
	
	\hspace{5mm}\textit{Dual problem.} \hspace{-0.1mm}Generalizing \hspace{-0.1mm}the \hspace{-0.1mm}methods \hspace{-0.1mm}in \hspace{-0.1mm}\mbox{\cite[p.\ \hspace{-0.1mm}113 \hspace{-0.1mm}ff.]{ET99}} \hspace{-0.1mm}to \hspace{-0.1mm}the \hspace{-0.1mm}spaces \hspace{-0.1mm}\eqref{eq:spaces}, \hspace{-0.1mm}one~\hspace{-0.1mm}finds~\hspace{-0.1mm}that
		the dual problem consists in the maximization of the functional $\smash{D\colon W^{p'(\cdot)}_{N}(\textup{div};\Omega)\to \mathbb{R}\cup\{-\infty\}}$, for every $y\in \smash{W^{p'(\cdot)}_N(\textup{div};\Omega)}$ defined by
		\begin{align}
			D(y)\coloneqq -\rho_{\varphi^*,\Omega}(y) -I_{\{-f\}}(\textup{div}\,y)\,,\label{eq:pDirichletDual}
		\end{align}
		where $I_{\{-f\}}\colon L^{p'(\cdot)}(\Omega)\to  \mathbb{R}\cup\{+\infty\}$ for every $g\in L^{p'(\cdot)}(\Omega) $ is defined by 
 \begin{align}
  I_{\{-f\}}(g)\coloneqq\begin{cases}
  0&\text{ if }g=-f\text{ a.e. in }\Omega\\+\infty&\text{ else }
  \end{cases}\,.
 \end{align}
		Due to \cite[Remark 4.1 (4.21), p.\ 60]{ET99}, the dual problem admits a unique solution $\smash{z\in W^{p'(\cdot)}_N(\textup{div};\Omega)}$, i.e., a maximizer of \eqref{eq:pDirichletDual}, called the \textit{dual solution}, and a \textit{strong duality relation} applies, i.e., 
 \begin{align*}
  I(u)=D(z)\,.
 \end{align*}
  In addition, cf. \cite[Remark 4.1 (4.22)--(4.25), p.\ 60]{ET99}, there hold the \textit{convex optimality relations}
		\begin{alignat}{2}
				\textup{div}\,z &=-f&&\quad\text{ in }L^{p'(\cdot)}(\Omega)\label{eq:pDirichletOptimality1.1}\,,\\ 
 z&=\AAA(\cdot,\nabla u)&&\quad\text{ in } \smash{L^{p'(\cdot)}(\Omega;\mathbb{R}^d)}\,.\label{eq:pDirichletOptimality1.2}
		\end{alignat}
		By the Fenchel--Young identity, cf.\ \cite[Proposition 5.1, p.\ 21]{ET99}, \eqref{eq:pDirichletOptimality1.2} is equivalent to 
		\begin{align}
			 z\cdot\nabla u=\varphi^*(\cdot,\vert z\vert )+\varphi(\cdot,\vert \nabla u\vert )\quad\textup{ in }L^1(\Omega)\,.\label{eq:pDirichletOptimality2}
		\end{align}

    
 
 
 \subsection{Discrete $p_h(\cdot)$-Dirichlet problem}

 \hspace{5mm}One important aspect in the numerical approximation of the $p(\cdot)$-Dirichlet~problem~\eqref{eq:pDirichletW1p} consists in the discretization of the $x$-dependent non-linearity. Here, it is convenient to use a simple one-point quadrature rule. More precisely, if $p\in C^0(\overline{\Omega})$ with $p^->1$,~then~we~define~the~element-wise constant variable exponent $p_h\in \mathcal{L}^0(\mathcal{T}_h)$, the generalized $N$-function $\varphi_h\colon\Omega\times\mathbb{R}_{\ge 0}\to \mathbb{R}_{\ge 0}$, and the non-linear~operators ${\AAA_h,F_h,F_h^*\colon\Omega\times\mathbb{R}^d\hspace{-0.05em}\to\hspace{-0.05em} \mathbb{R}^d}$ for all $T\hspace{-0.05em}\in\hspace{-0.05em} \mathcal{T}_h$, a.e. $x\hspace{-0.05em}\in\hspace{-0.05em} T$, and~all~${a\hspace{-0.05em}\in\hspace{-0.05em} \mathbb{R}^d}$~by
 \begin{align}
  \begin{aligned}
  p_h(x)&\coloneqq p(q_T)\,,\qquad
  &&\hspace{-2mm}\varphi_h(x,\vert a\vert )\coloneqq \varphi(q_T,\vert a\vert)\,,\\
  \AAA_h(x,a)&\coloneqq \AAA(q_T,a)\,,\qquad
  &&F_h(x,a)\coloneqq F(q_T,a)\,,\qquad
  F_h^*(x,a)\coloneqq F^*(q_T,a)\,,
  \end{aligned}\label{def:A_h}
 \end{align}
 where $q_T\in T$ is some arbitrary quadrature point, e.g., the barycenter of the element $T$.

 \begin{remark}\label{rem:uniform}
    Since the hidden constants in all equivalences in Section \ref{sec:basic} depend only on $p^-,p^+>1$ and $\delta\ge 0$ and since $p^-\leq p_h^-\le p^+_h\leq p^+$ a.e. in $\Omega$ for all $h>0$,     
    the same equivalences apply to the discretizations \eqref{def:A_h} with the hidden constants depending~only~on~$p^-,p^+\in (1,\infty)$.
 \end{remark}
 
 \subsubsection{$\mathcal{S}^1_D(\mathcal{T}_h)$-approximation of the $p(\cdot)$-Dirichlet problem}
		
		\qquad Given a right-hand side $f\hspace{-0.1em}\in\hspace{-0.1em} L^{p'(\cdot)}(\Omega)$, where $p\hspace{-0.1em}\in\hspace{-0.1em} C^0(\overline{\Omega})$ with $p^-\hspace{-0.1em}>\hspace{-0.1em}1$,~the~\mbox{$\smash{\mathcal{S}^1_D(\mathcal{T}_h)}$-approximation} of the $p(\cdot)$-Dirichlet problem, where $\mathcal{S}^1_D(\mathcal{T}_h)\vcentcolon=\mathcal{S}^1(\mathcal{T}_h)\cap \smash{W^{1,p(\cdot)}_D(\Omega)}$, seeks for $u_h^{\textit{c}}\in \mathcal{S}^1_D(\mathcal{T}_h)$~such~that for every $v_h\in \mathcal{S}^1_D(\mathcal{T}_h)$,~it~holds
		\begin{align}
		 (\AAA_h(\cdot,\nabla u_h^{\textit{c}}),\nabla v_h)_\Omega=(f,v_h)_\Omega\,.\label{eq:pDirichletS1D}
		\end{align}
		The theory of monotone operators, cf.\ \cite{Zei90B}, 
        proves the existence of a unique~solution~to~\eqref{eq:pDirichletS1D}. In what follows, we reserve~the~notation~$\smash{u_h^{\textit{c}}\in \mathcal{S}^1_D(\mathcal{T}_h)}$~for~this~solution. 
 
  In \cite{BDS15}, the following best-approximation result was derived:
		
		\begin{theorem}[Best-approximation]\label{P1_best-approx}
  Let $p\in C^{0,\alpha}(\overline{\Omega})$ with $\alpha\in (0,1]$ and $p^->1$~and~let~$\delta\ge 0$. 
  Then, there exists some $s>1$, which can chosen to be close to $1$ if $h_{\max}>0$ is close~to~$0$,~such~that if 
    $u\in W^{1,p(\cdot)s}(\Omega)$, then 
		 \begin{align*}
		 \|F_h(\cdot, \nabla u_h^{\textit{c}})- F_h(\cdot, \nabla u)\|_{2,\Omega}^2\lesssim \inf_{v_h\in \mathcal{S}^1_D(\mathcal{T}_h)}{\|F_h(\cdot, \nabla v_h)-F_h(\cdot,\nabla u)\|_{2,\Omega}^2}+h_{\max}^{2\alpha}\,\big(1+\rho_{p(\cdot)s,\Omega}(\nabla u)\big)\,,
		 \end{align*}
         where the hidden constant also depends on $s>1$ and the chunkiness $\omega_0>0$. 
		\end{theorem}
		
		
		\begin{proof} In \cite[Lemma 4.7]{BDS15}, only the case $q_T\coloneqq \textrm{arg\,min}_{x\in T}{p(x)}$ for all $T\in \mathcal{T}_h$ has been considered. However, an  analysis of the proof of \cite[Lemma 4.7]{BDS15} reveals that this particular quadrature rule is not needed there.
		\end{proof}

        Resorting \hspace{-0.1mm}in \hspace{-0.1mm}Theorem \hspace{-0.1mm}\ref{P1_best-approx} \hspace{-0.1mm}to \hspace{-0.1mm}the \hspace{-0.1mm}approximation \hspace{-0.1mm}properties \hspace{-0.1mm}of \hspace{-0.1mm}the \hspace{-0.1mm}\textit{Scott--Zhang~\hspace{-0.1mm}quasi-interpolation \hspace{-0.1mm}operator}  $I_h^{\textit{sz}}\colon \smash{W^{1,p(\cdot)}_D(\Omega)}\hspace*{-0.15em}\to\hspace*{-0.15em}\mathcal{S}^1_D(\mathcal{T}_h)$, cf.\ \cite{SZ90}, one arrives at  the following a priori error estimate~for the $\mathcal{S}^1_D(\mathcal{T}_h)$-approximation \eqref{eq:pDirichletS1D} of the $p(\cdot)$-Dirichlet problem \eqref{eq:pDirichletW1p}.
		
		\begin{theorem}[A priori error estimate]\label{P1_apriori}
		 Let $p\in C^{0,\alpha}(\overline{\Omega})$ with $\alpha\in (0,1]$ and $p^->1$~and~let~$\delta\ge 0$.  Moreover, let $F(\cdot,\nabla u)\in W^{1,2}(\Omega;\mathbb{R}^d)$.  
        Then, there exists some $s>1$, which can chosen to be close to $1$ if $h_{\max}>0$ is close to $0$, such that 
		 \begin{align*}
		 \|F_h(\cdot,\nabla u_h^{\textit{c}})-F_h(\cdot,\nabla u)\|_{2,\Omega}^2\lesssim h_{\max}^2\,\|\nabla F(\cdot,\nabla u)\|_{2,\Omega}^2+h_{\max}^{2\alpha}\,\big(1+\rho_{p(\cdot)s,\Omega}(\nabla u)\big)\,,
		 \end{align*}
        where the hidden constant also depends on $s>1$ and the chunkiness $\omega_0>0$.\enlargethispage{1mm}
		\end{theorem}
		
		\begin{proof}
		 See \cite[Theorem 4.8]{BDS15}.
		\end{proof}

        \begin{remark}\label{rem:ps}
            If $F(\cdot,\nabla u)\in W^{1,2}(\Omega;\mathbb{R}^d)$, then  $F(\cdot,\nabla u)\in L^{2^*}(\Omega;\mathbb{R}^d)$, where $2^*\coloneqq \frac{2d}{d-2}$ if $2<d$ and $2^*\hspace{-0.1em}\in\hspace{-0.1em} [1,\infty)$ if $2\hspace{-0.1em}\ge\hspace{-0.1em} d$. Thus, for $s\hspace{-0.1em}>\hspace{-0.1em}1$ close to $1$, i.e., $h_{\max}\hspace{-0.1em}>\hspace{-0.1em}0$  close to $0$,~it~holds~${\nabla u\hspace{-0.1em} \in\hspace{-0.1em} L^{p(\cdot)s}(\Omega)}$. 
            Similarly, if $F^*(\cdot,z)\in W^{1,2}(\Omega;\mathbb{R}^d)$, then for $s>1$ close to $1$, i.e., $h_{\max}>0$  close to $0$, it holds $z \in L^{p'(\cdot)s}(\Omega)$. 
        \end{remark}

 The following lemma is of crucial importance for the hereinafter analysis; it bounds the error resulting from switching from $\AAA_h\colon\Omega\times\mathbb{R}^d\to \mathbb{R}^d$, $h>0$, to $\AAA\colon\Omega\times\mathbb{R}^d\to \mathbb{R}^d$ or from~switching~from
 $F_h\colon\Omega\times\mathbb{R}^d\to \mathbb{R}^d$, $h>0$, to $F\colon\Omega\times\mathbb{R}^d\to \mathbb{R}^d$ and vice versa, respectively.
 
 \begin{proposition}\label{lem:A-Ah}
 Let $p\in C^0(\overline{\Omega})$ with $p^->1$ and let $\delta\ge 0$.   Then, there exists~some~${s>1}$,~which can chosen to be close to $0$ if $h_T>0$ is close to $0$, such that
  for every $T\in \mathcal{T}_h$, $g\in L^{p'(\cdot)s}(T)$, $v\in W^{1,p(\cdot)s}(T)$, and $\lambda\in [0,1]$, it holds
 \begin{align}
  \| F_h(\cdot,\nabla v)-F(\cdot,\nabla v)\|_{2,T}^2&\lesssim  \|\omega_{p,T}(h_T)^{2}\,(1+\vert \nabla v\vert^{p(\cdot)s})\|_{1,T}
  \,,\label{eq:Fh-F}\\
  \| F_h^*(\cdot,\AAA_h(\cdot,\nabla v))-F^*_h(\cdot,\AAA(\cdot,\nabla v))\|_{2,T}^2&\lesssim \|\omega_{p,T}(h_T)^{2}\,(1+\vert \nabla v\vert^{p(\cdot)s})\|_{1,T}
  \,,\label{eq:Ah-A}\\
  \rho_{((\varphi_h)_{\vert \nabla v \vert})^*,T}(\lambda\,g) &\lesssim \rho_{(\varphi_{\vert \nabla v \vert})^*,T}(\lambda\,g)\label{eq:phih-phi}\\&\quad+\smash{\lambda^{\smash{{2\wedge (p^+)'}}}}\,\|\omega_{p,T}(h_T)\,(1\hspace{-0.05em}+\hspace{-0.05em}\vert \nabla v\vert^{p(\cdot)s}\hspace{-0.05em}+\hspace{-0.05em}\vert g\vert^{p'(\cdot)s})\|_{1,T}\,,\notag
 \end{align} 
  where the hidden constants also depend on $s>1$ and the chunkiness $\omega_0>0$. Furthermore,  $\omega_{p,T}$ denotes the modulus of continuity~of~$p|_T\in C^0(T)$.
 \end{proposition}

Proposition \ref{lem:A-Ah} is based on the following point-wise estimates.\vspace{-1mm}\enlargethispage{5.5mm}

 \begin{lemma}\label{lem:Ax-Axh}
  Let $p\in C^0(\overline{\Omega})$ with $p^->1$ and let $\delta\ge 0$. 
  Then, there exists some $s>1$, which can chosen to be close to $0$ if $\vert x-y\vert$ is close to $0$,  such that 
  for every $x,y\in \overline{\Omega}$, $t\ge 0$, $a,b\in \mathbb{R}^d$, and $\lambda\in [0,1]$, it holds
 \begin{align}
  \vert F(x,a)-F(y,a)\vert^2&\lesssim \vert p(x)-p(y)\vert^2\,(1 +\vert a\vert^{p(x)s})\,,\label{eq:Fxh-Fx}\\
   \vert F^*(x,a)-F^*(y,a)\vert^2&\lesssim \vert p(x)-p(y)\vert^2\,(1 +\vert a\vert^{p(x)s})\,,\label{eq:F*xh-F*x}\\
 \vert F^*(x,\AAA(x,a))-F^*(x,\AAA(y,a))\vert^2&\lesssim\vert p(x)-p(y)\vert^2\,(1 +\vert a\vert^{p(x)s}) \,,\label{eq:Axh-Ax}\\
  \begin{split}
  (\varphi_{\vert b\vert})^*(x,t) &\lesssim (\varphi_{\vert b\vert})^*(y,t)\\[-0.5mm]&\quad+
  \smash{\lambda^{\smash{{2\wedge (p^+)'}}}}\,\vert p(x)-p(y)\vert \,(1+\vert b\vert^{p(y)s}+t^{p'(y)s})\,,
  \end{split}\label{eq:phixh-phix}
 \end{align}
 where the hidden constants also depend on $s>1$ and the chunkiness $\omega_0>0$.
 \end{lemma}

 \begin{proof}\let\qed\relax
    \textit{ad \eqref{eq:Fxh-Fx}.}
 The Newton--Leibniz formula yields~for~all~${x,y\in \overline{\Omega}}$~and~${a\in \mathbb{R}^d}$~that
  \begin{align}\label{eq:Fh-F.1}
  \begin{aligned}
\vert F(x,a)-F(y,a)\vert^2&\lesssim \vert p(x)-p(y)\vert^2\,\ln(\delta +\vert a\vert)^2\,(
 \varphi(x,\vert a\vert )+\varphi(y,\vert a\vert ))
 \\&\lesssim 
  \vert p(x)-p(y)\vert^2\,\ln(\delta +\vert a\vert)^2\,(1+(\delta +\vert a\vert)^{p(y)-p(x)})\,\varphi(x,\vert a\vert)%\\&\quad +
  %\vert p(x)-p(y)\vert^2\ln(\delta +\vert a\vert)^2(\delta +\vert a\vert)^{p(y)-p(x)}\varphi(x,\vert a\vert)
  \\&\lesssim \vert p(x)-p(y)\vert^2\,(1 +\vert a\vert^{p(x)s})\,,
  \end{aligned}
  \end{align}
 %By the aid of \eqref{eq:Fh-F.1}, we deduce that
 %\begin{align*}
 %\vert F(x,a)-F(y,a)\vert^2&\lesssim 
 %\vert p(x)-p(y)\vert^2\ln(\delta +\vert a\vert)^2(1+(\delta +\vert a\vert)^{p(y)-p(x)})\varphi(x,\vert a\vert)%\\&\quad +
 %%\vert p(x)-p(y)\vert^2\ln(\delta +\vert a\vert)^2(\delta +\vert a\vert)^{p(y)-p(x)}\varphi(x,\vert a\vert)
 %\\&\lesssim \vert p(x)-p(y)\vert^2\,(1 +\vert a\vert^{p(x)s})\,,
 %\end{align*}
 for some $s>1$, which can chosen to be close to $1$ if $\vert x-y\vert$ is close to $0$.
 
  \textit{ad \eqref{eq:Axh-Ax}.} The Newton--Leibniz formula yields~for~all~${x,y\in  \overline{\Omega}}$~and~${a\in \mathbb{R}^d}$ that
  \begin{align}\label{eq:Ah-A.1}
  \begin{aligned}
 \vert \AAA(x,a)-\AAA(y,a)\vert&\lesssim \vert p(x)-p(y)\vert\,\vert\ln(\delta +\vert a\vert)\vert\,(
 \varphi'(x,\vert a \vert)+\varphi'(y,\vert a\vert))\\&\lesssim 
 \vert p(x)-p(y)\vert\, \vert \ln(\delta +\vert a\vert)\vert\, (1+(\delta +\vert a\vert)^{p(y)-p(x)})\, \varphi'(x,\vert a\vert)\,.
 \end{aligned}
  \end{align}
  Using \eqref{eq:Ah-A.1}, \eqref{eq:hammerf},  \eqref{eq:hammerg}, the monotonicity of $(\varphi_{\vert a\vert })^*(x,\cdot)$, that 
  $\smash{\Delta_2((\varphi_{\vert a\vert })^*(x,\cdot))\lesssim 2^{\max\{2,(p^-)'\}}}$, cf. Remark \ref{rem:phi_a},\vspace{-1.5mm}
  \begin{align}
      (\varphi_{\vert a\vert })^*(x,\lambda\, t)&\lesssim \smash{\max\{\lambda^{p'(x)},\lambda^2\}} (\varphi_{\vert a\vert })^*(x, t)\,,\label{estimate}\\
      (\varphi_{\vert a\vert})^*(x,\lambda\,\varphi'(x,\vert a\vert))&\sim \smash{\lambda^2}\, \varphi(x,\vert a\vert)\,,
  \end{align}
  cf. \cite[Lemma A.7 \& Lemma A.8]{BDS13}, in conjunction with Remark \ref{rem:uniform}, we deduce that
  \begin{align*}
 \vert F^*(x,\,&\AAA(x,a))-F^*(x,\AAA(y,a))\vert^2\lesssim (\varphi_{\vert a\vert })^*(x,\vert \AAA(x,a)-\AAA(y,a)\vert )%\\&\lesssim 
  %(\varphi_{\vert a\vert })^*(x,\vert p(x)-p(y)\vert\, \vert \ln(\delta +\vert a\vert)\vert\, (1+(\delta +\vert a\vert)^{p(y)-p(x)})\, \varphi'(x,\vert a\vert))
  \\&\lesssim 
  ((1+\vert \ln(\delta +\vert a\vert)\vert)\, (1+(\delta +\vert a\vert)^{p(y)-p(x)}))^{\max\{2,p'(x)\}}\,(\varphi_{\vert a\vert })^*(x,\vert p(x)-p(y)\vert\,\varphi'(x,\vert a\vert))
  \\&\lesssim 
  \vert p(x)-p(y)\vert^2\,((1+\vert \ln(\delta +\vert a\vert)\vert) \,(1+(\delta +\vert a\vert)^{p(y)-p(x)}))^{\max\{2,p'(x)\}}\,\varphi(x,\vert a\vert)
  \\&\lesssim \vert p(x)-p(y)\vert^2\,(1 +\vert a\vert^{p(x)s})\,,
 \end{align*}
 for some $s>1$, which can chosen to be close to $1$ if $\vert x-y\vert$ is close to $0$.\enlargethispage{2.5mm}

  \textit{ad \eqref{eq:phixh-phix}.} 
  Using twice the Newton--Leibniz formula yields for all  $x,y\in  \overline{\Omega}$ and $a\in \mathbb{R}^d$ that\enlargethispage{5mm}
  \begin{align}\label{eq:phih-phi.1}
  \begin{aligned}
  &(\varphi_{\vert b\vert})^*(x,\lambda\, t) 
  \lesssim ((\delta+\vert b\vert)^{p(x)-1}+\lambda\,t)^{p'(x)-2}(\lambda\,t)^2
  \\& = ((\delta+\vert b\vert)^{p(y)-1}+\lambda\,t)^{p'(x)-2}(\lambda\,t)^2
  \\&\quad+ \int_{p(x)\wedge p(y)}^{p(x)\vee p(y)}{(p'(x)-2)(\delta+\vert b\vert)^{r-1}\ln(\delta+\vert b\vert)((\delta+\vert b\vert)^{r-1}+\lambda\,t)^{p'(x)-3}(\lambda\,t)^2\,\textrm{d}r} 
  \\&=((\delta+\vert b\vert)^{p(y)-1}+\lambda\,t)^{p'(y)-2}(\lambda\,t)^2
 \\&\quad+ \int_{p(x)\wedge p(y)}^{p(x)\vee p(y)}{(p'(x)-2)(\delta+\vert b\vert)^{r-1}\ln(\delta+\vert b\vert)((\delta+\vert b\vert)^{r-1}+\lambda\,t)^{p'(x)-3}(\lambda\,t)^2\,\textrm{d}r}
  \\&\quad+\int_{p'(x)\wedge p'(y)}^{p'(x)\vee p'(y)}{\ln((\delta+\vert b\vert)^{p(y)-1}+\lambda\,t)((\delta+\vert b\vert)^{p(y)-1}+\lambda\,t)^{r-2}(\lambda\,t)^2\,\textrm{d}r}\,. 
  \\&\eqqcolon ((\delta+\vert b\vert)^{p(y)-1}+\lambda\,t)^{p'(y)-2}(\lambda\,t)^2+I_1+I_2\,.
  \end{aligned}\hspace{-1cm}
  \end{align}
  Next, we need to estimate the terms $I_1$ and $I_2$:

  \textit{ad $I_1$.} Using $((\delta+\vert b\vert)^{r-1}+\lambda\,t)^{p'(x)-2}(\lambda\,t)^2\lesssim \lambda^{\min\{2,p'(x)\}} ((\delta+\vert b\vert)^{r-1}+t)^{p'(x)-2}t^2$,~we~obtain
  \begin{align}\label{eq:phih-phi.2}
  \begin{aligned}
 \hspace{-1mm}I_1&\lesssim \vert \ln(\delta+\vert b\vert)\vert \int_{p(x)\wedge p(y)}^{p(x)\vee p(y)}{((\delta+\vert b\vert)^{r-1}+\lambda\,t)^{p'(x)-2}(\lambda\,t)^2\,\textrm{d}r}
  %\vert p(x)-p(y)\vert\vert \ln(\delta+\vert b\vert)\vert ((\delta+\vert b\vert)^{p(x)-1}+(\delta+\vert b\vert)^{p(y)-1}+\lambda\,t)^{p'(y)-2}(\lambda\,t)^2
 % \\&\lesssim
%  \lambda^{\smash{{2\wedge (p^-)'}}}\,\vert p(x)-p(y)\vert\vert \ln(\delta+\vert b\vert)\vert ((\delta+\vert b\vert)^{p(x)-1}+(\delta+\vert b\vert)^{p(y)-1}+t)^{p'(y)-2}t^2
  \\&\lesssim
  \lambda^{\smash{2\wedge p'(x)}}\,\vert p(x)-p(y)\vert\,\vert \ln(\delta+\vert b\vert)\vert \,((\delta+\vert b\vert)^{p(x)-1}+(\delta+\vert b\vert)^{p(y)-1}+t)^{p'(x)}
 \\&\lesssim
  \lambda^{\smash{{2\wedge (p^+)'}}}\,\vert p(x)-p(y)\vert\, (1+(\delta+\vert b\vert)^{p(x)-1}+(\delta+\vert b\vert)^{p(y)-1}+t)^{p'(x)s}
  \\&\lesssim
  \lambda^{\smash{{2\wedge (p^+)'}}}\,\vert p(x)-p(y)\vert\, (1+(\delta+\vert b\vert)^{p(x)s}+(\delta+\vert b\vert)^{p(y)(p'(x)/p'(y))s}+t^{p'(y)(p'(x)/p'(y))s})
  \\&\lesssim
  \lambda^{\smash{{2\wedge (p^+)'}}}\, \vert p(x)-p(y)\vert\,(1+\vert b\vert^{p(y)s}+t^{p'(y)s})\,,
  \end{aligned}\hspace{-2.5mm}
  \end{align}
  for some $s>1$, which can chosen to be close to $1$ if $\vert x-y\vert$ is close to $0$.
  
  \textit{ad $I_2$.} Using $((\delta+\vert b\vert)^{p(y)-1}+\lambda\,t)^{r-2}(\lambda\,t)^2\lesssim \lambda^{\min\{2,r\}} ((\delta+\vert b\vert)^{p(y)-1}+t)^{r-2}t^2$,~we~obtain
  \begin{align}
  \label{eq:phih-phi.3}
  \begin{aligned}
  I_2&\lesssim \lambda^{{2\wedge (p^+)'}}
  \vert p'(x)-p'(y)\vert\,\vert\ln((\delta+\vert b\vert)^{p(y)-1}+t)\vert\,(1+(\delta+\vert b\vert)^{p(y)-1}+t)^{p'(x)-2}t^2
  \\&\quad+\lambda^{{2\wedge (p^+)'}}\vert p'(x)-p'(y)\vert\,\vert\ln((\delta+\vert b\vert)^{p(y)-1}+t)\vert\,(1+(\delta+\vert b\vert)^{p(y)-1}+t)^{p'(y)-2}t^2\\
  %&\lesssim \lambda^{\smash{{2\wedge (p^-)'}}}\,\vert p'(x)-p'(y)\vert\vert\ln((\delta+\vert b\vert)^{p(y)-1}+t)\vert((\delta+\vert b\vert)^{p(y)-1}+t)^{p'(x)-2}t^2
  %\\&\quad+\lambda^{\smash{{2\wedge (p^-)'}}}\,\vert p'(x)-p'(y)\vert\vert\ln((\delta+\vert b\vert)^{p(y)-1}+t)\vert((\delta+\vert b\vert)^{p(y)-1}+t)^{p'(y)-2}t^2\\
  &\lesssim \lambda^{\smash{{2\wedge (p^+)'}}}\,\tfrac{\vert p(x)-p(y)\vert}{(p(x)-1)(p(y)-1)}\,\vert\ln((\delta+\vert b\vert)^{p(y)-1}+t)\vert\,(1+(\delta+\vert b\vert)^{p(y)-1}+t)^{p'(x)}
  \\&\quad+\lambda^{\smash{{2\wedge (p^+)'}}}\,\tfrac{\vert p(x)-p(y)\vert}{(p(x)-1)(p(y)-1)}\,\vert\ln((\delta+\vert b\vert)^{p(y)-1}+t)\vert\,(1+(\delta+\vert b\vert)^{p(y)-1}+t)^{p'(y)}
 \\&\lesssim \lambda^{\smash{{2\wedge (p^+)'}}}\,\vert p(x)-p(y)\vert\, \vert\ln((\delta+\vert b\vert)^{p(y)-1}+t)\vert\,(1+(\delta+\vert b\vert)^{p(y)-1}+t)^{p'(y)(p'(x)/p'(y))}
  \\&\quad+\lambda^{\smash{{2\wedge (p^+)'}}}\,\vert p(x)-p(y)\vert\,\vert\ln((\delta+\vert b\vert)^{p(y)-1}+t)\vert\,(1+(\delta+\vert b\vert)^{p(y)-1}+t)^{p'(y)}
 %\\&\lesssim \lambda^{\smash{{2\wedge (p^-)'}}}\,\vert p(x)-p(y)\vert (1+(\delta+\vert b\vert)^{p(y)-1}+t)^{p'(y)s}
 \\&\lesssim \lambda^{\smash{{2\wedge (p^+)'}}}\,\vert p(x)-p(y)\vert\, (1+\vert b\vert^{p(y)s}+t^{p'(y)s})\,,
  \end{aligned}%\hspace{-5mm}
  \end{align}
  for some $s>1$, which can chosen to be close to $1$ if $\vert x-y\vert$ is close to $0$.
  
  Eventually, combining \eqref{eq:phih-phi.2} and \eqref{eq:phih-phi.3} in \eqref{eq:phih-phi.1}, appealing to Remark \ref{rem:phi_a}, we conclude that
  \begin{align*}
  (\varphi_{\vert b\vert})^*(x, \lambda\,t)\lesssim (\varphi_{\vert b\vert})^*(y, \lambda\,t)+\lambda^{{2\wedge (p^+)'}}\,\vert p(x)-p(y)\vert (1+\vert b\vert^{p(y)s}+t^{p'(y)s})\,.
  \end{align*}

  \textit{ad \eqref{eq:F*xh-F*x}.} Using the Newton--Leibniz formula and proceeding as for \eqref{eq:F*xh-F*x} and \eqref{eq:phih-phi.3} yields for all  $x,y\in  \overline{\Omega}$ and $a\in \mathbb{R}^d$ that
  \begin{align*}
      \vert F^*(x,a)-F^*(y,a)\vert^2 &\lesssim  \vert (\delta^{p(x)-1}+\vert a\vert)^{\frac{p'(x)-2}{2}}a-(\delta^{p(x)-1}+\vert a\vert)^{\frac{p'(y)-2}{2}}a\vert^2
      \\&\quad +\int_{p(x)\wedge p(y)}^{p(x)\vee p(y)}{\vert \ln(\delta)\vert(\delta^{r-1}+\vert a\vert)^{p'(x)-2}\vert a\vert^2\,\textrm{d}r}
      \\&\lesssim \vert p(x)-p(y)\vert \,(1+\vert a\vert^{p(y)s})\,.\tag*{$\qedsymbol$}
  \end{align*}
 \end{proof}\newpage
	


 

 \subsection{$\smash{\mathcal{S}^{1,\textit{cr}}_D(\mathcal{T}_h)}$-approximation of the $p(\cdot)$-Dirichlet problem}
	
		
		 \hspace{5mm}Given a right-hand side $f\hspace{-0.05em}\in \hspace{-0.05em}L^{p'(\cdot)}(\Omega)$, $p\hspace{-0.05em}\in\hspace{-0.05em} C^0(\overline{\Omega})$ with $p^-\hspace{-0.05em}>\hspace{-0.05em}1$, and setting ${f_h\hspace{-0.05em}\coloneqq \hspace{-0.05em}\Pi_h f\hspace{-0.05em}\in \hspace{-0.05em}\mathcal{L}^0(\mathcal{T}_h)}$,
  the $\mathcal{S}^{1,\textit{cr}}_D(\mathcal{T}_h)$-approximation of the $p(\cdot)$-Dirichlet problem seeks for $u_h^{\textit{cr}}\in \mathcal{S}^{1,\textit{cr}}_D(\mathcal{T}_h)$ such that for every $v_h\in \mathcal{S}^{1,\textit{cr}}_D(\mathcal{T}_h)$, it holds
		\begin{align}
		 (\AAA_h(\cdot,\nabla_{\!h}u_h^{\textit{cr}}),\nabla_{\! h} v_h)_\Omega=(f_h,\Pi_h v_h)_\Omega\,.\label{eq:pDirichletS1crD}
		\end{align}
		The theory of monotone operators, cf.\ \cite{Zei90B,Ru04}, 
        proves the existence of a unique~solution~to~\eqref{eq:pDirichletS1crD}. In what follows, we reserve~the~notation~$u_h^{\textit{cr}}\in \smash{\mathcal{S}^{1,\textit{cr}}_D(\mathcal{T}_h)}$~for~this~solution. 
		\subsubsection{Discrete minimization problem and discrete convex duality relations}\enlargethispage{5mm}
		
		\hspace{5mm}The discrete variational problem \eqref{eq:pDirichletS1crD} emerges as an optimality condition of an equivalent convex minimization problem.
		
		\hspace{5mm}\textit{Discrete primal problem.} The problem \eqref{eq:pDirichletS1crD} is equivalent to the minimization of the \textit{discrete $p_h(\cdot)$-Dirichlet energy}, i.e.,  ${I_h^{\textit{cr}} \colon \mathcal{S}^{1,\textit{cr}}_D(\mathcal{T}_h) \to \mathbb{R}}$, for every $v_h\in \smash{\mathcal{S}^{1,\textit{cr}}_D(\mathcal{T}_h)}$ defined by
	 \begin{align}\label{eq:pDirichletPrimalCR}
		 I_h^{\textit{cr}}(v_h)\coloneqq \rho_{\varphi_h,\Omega}(\nabla_{\!h} v_h)-(f_h,\Pi_hv_h)_\Omega\,.
		\end{align}
		Hereinafter, we refer the minimization of the discrete $p_h(\cdot)$-Dirichlet energy \eqref{eq:pDirichletPrimalCR}~to~as~the~\textit{discrete primal problem}.
		Since the discrete $p_h(\cdot)$-Dirichlet energy \eqref{eq:pDirichletPrimalCR} is proper, 
		strictly convex, weakly coercive, 
		and lower semi-continuous, the existence of a unique minimizer of \eqref{eq:pDirichletPrimalCR}, called the \textit{discrete primal solution}, follows using 
  the direct method in the calculus of variations, cf.\ \cite{Dac08}. More precisely,
	since the discrete $p_h(\cdot)$-Dirichlet energy \eqref{eq:pDirichletPrimalCR} is Fr\'echet differentiable and for every $v_h,w_h\in \smash{\mathcal{S}^{1,\textit{cr}}_D(\mathcal{T}_h)}$,~it~holds
	\begin{align*}
	 \langle DI_h^{\textit{cr}}(v_h),w_h\rangle_{\smash{\mathcal{S}^{1,\textit{cr}}_D(\mathcal{T}_h)}}=(\AAA_h(\cdot,\nabla_{\!h} v_h),\nabla_{\! h} w_h)_\Omega\,,
	\end{align*}
	the \hspace{-0.2mm}optimality \hspace{-0.2mm}conditions \hspace{-0.2mm}of \hspace{-0.2mm}the \hspace{-0.2mm}discrete \hspace{-0.2mm}primal \hspace{-0.2mm}problem \hspace{-0.2mm}and \hspace{-0.2mm}convexity \hspace{-0.2mm}of \hspace{-0.2mm}the \hspace{-0.2mm}discrete~\hspace{-0.2mm}\mbox{$p_h(\cdot)$-Dirichlet} energy~\eqref{eq:pDirichletPrimalCR} imply that $\smash{u_h^{\textit{cr}}\in \smash{\mathcal{S}^{1,\textit{cr}}_D(\mathcal{T}_h)}}$ solves the discrete primal problem, i.e., is the unique minimizer of the discrete $p_h(\cdot)$-Dirichlet energy~\eqref{eq:pDirichletPrimalCR}.
	
		\hspace{5mm}\textit{Discrete dual problem.} The discrete dual problem consists in the maximization of the functional $D_h^{\textit{rt}}:\mathcal{R}T^0_N(\mathcal{T}_h)\to \mathbb{R}\cup\{-\infty\}$, for every $y_h\in \mathcal{R}T^0_N(\mathcal{T}_h)$ defined by
	\begin{align}
		D_h^{\textit{rt}}(y_h)\coloneqq -\rho_{\varphi_h^*,\Omega}(\Pi_hy_h)-I_{\{-f_h\}}(\textup{div}\,y_h)\,.\label{eq:pDirichletDualCR}
	\end{align}
 The following proposition establishes the well-posedness of the discrete dual problem, i.e., the existence of a maximizer, the so-called \textit{discrete dual solution},  and
 discrete strong duality.~In~addition, it provides a reconstruction formula for~this~maximizer from the discrete primal solution.

 \begin{proposition}\label{prop:discrete_convex_duality}
 The following statements apply:
 \begin{itemize}[noitemsep,topsep=1pt,labelwidth=\widthof{(ii)},leftmargin=!]
 \item[(i)] There holds a discrete weak duality relation, i.e.,
 \begin{align}
  \inf_{v_h\in \mathcal{S}^{1,\textit{cr}}_D(\mathcal{T}_h)}{I_h^{\textit{cr}}(v_h)}\ge \sup_{y_h\in \mathcal{R}T^0_N(\mathcal{T}_h)}{D_h^{\textit{rt}}(y_h)}\,.\label{eq:discrete_weak_duality}
 \end{align}
  \item[(ii)] The discrete flux $z_h^{\textit{rt}}\in \mathcal{L}^1(\mathcal{T}_h)$, defined via the generalized Marini formula
  \begin{align}
		z_h^{\textit{rt}}= \AAA_h(\cdot,\nabla_{\! h}u_h^{\textit{cr}})-\frac{f_h}{d}\big(\textup{id}_{\mathbb{R}^d}-\Pi_h\textup{id}_{\mathbb{R}^d}\big)\quad\text{ in } \mathcal{R}T^0_N(\mathcal{T}_h)\,,\label{eq:gen_marini}
	\end{align}
 satisfies $z_h^{\textit{rt}}\in \mathcal{R}T^0_N(\mathcal{T}_h)$ and the discrete convex optimality relations
 \begin{alignat}{2}
		\textup{div}\,z_h^{\textit{rt}}&=-f_h&&\quad\text{ in }\mathcal{L}^0(\mathcal{T}_h)\,,\label{eq:pDirichletOptimalityCR1.1}\\
 \Pi_hz_h^{\textit{rt}}&=\AAA_h(\cdot,\nabla_{\! h}u_h^{\textit{cr}})&&\quad\text{ in }\mathcal{L}^0(\mathcal{T}_h)^d\,.\label{eq:pDirichletOptimalityCR1.2}
	\end{alignat}
 \item[(iii)] The discrete flux $z_h^{\textit{rt}}\in \mathcal{R}T^0_N(\mathcal{T}_h)$ is the unique maximizer of \eqref{eq:pDirichletDualCR} and discrete~strong~duality applies, i.e., 
 \begin{align}\label{eq:discrete_strong_duality}
     I_h^{\textit{cr}}(u_h^{\textit{cr}})=D_h^{\textit{rt}}(z_h^{\textit{rt}})\,.
 \end{align}
 \end{itemize}
 \end{proposition}

By the Fenchel--Young identity, cf.\ \cite[Proposition 5.1, p. 21]{ET99}, \eqref{eq:pDirichletOptimalityCR1.2}~is~equivalent~to
	\begin{align}\label{eq:pDirichletOptimalityCR2}
	  	\Pi_hz_h^{\textit{rt}}\cdot\nabla_{\! h} u_h^{\textit{cr}}=\varphi_h^*(\cdot,\vert\Pi_hz_h^{\textit{rt}}\vert)+\varphi_h(\cdot,\vert\nabla_{\! h}u_h^{\textit{cr}}\vert)\quad\text{ in }\mathcal{L}^0(\mathcal{T}_h)\,.
	\end{align}
 
 \begin{proof}
 \textit{ad (i).} Using element-wise for each $T\in \mathcal{T}_h$ that $\varphi(q_T,\cdot)=\varphi^{**}(q_T,\cdot)$, the definition of the convex conjugate, and the discrete integration-by-parts formula \eqref{eq:pi0}, we find that
 \begin{align*}
  \inf_{v_h\in \mathcal{S}^{1,\textit{cr}}_D(\mathcal{T}_h)}{I_h^{\textit{cr}}(v_h)}&=\inf_{v_h\in \mathcal{S}^{1,\textit{cr}}_D(\mathcal{T}_h)}{\rho_{\varphi_h^{**},\Omega}(\nabla_{\! h} v_h)-(f_h,\Pi_h v_h)_\Omega}
 \\
 &=
  \inf_{v_h\in \mathcal{S}^{1,\textit{cr}}_D(\mathcal{T}_h)}{ \sup_{\overline{y}_h\in \mathcal{L}^0(\mathcal{T}_h)^d}{(\overline{y}_h,\nabla_{\! h} v_h)_\Omega-\rho_{\varphi_h^*,\Omega}(\overline{y}_h)-(f_h,\Pi_h v_h)_\Omega}}
 \\&\ge 
  \inf_{v_h\in \mathcal{S}^{1,\textit{cr}}_D(\mathcal{T}_h)}{ \sup_{y_h\in \mathcal{R}T^0_N(\mathcal{T}_h)}{-\rho_{\varphi_h^*,\Omega}(\Pi_h y_h)+(\Pi_h y_h,\nabla_{\! h} v_h)_\Omega-(f_h,\Pi_h v_h)_\Omega}}
  \\&=
  \inf_{v_h\in \mathcal{S}^{1,\textit{cr}}_D(\mathcal{T}_h)}{ \sup_{y_h\in \mathcal{R}T^0_N(\mathcal{T}_h)}{-\rho_{\varphi_h^*,\Omega}(\Pi_h y_h)-(\textup{div}\,y_h+f_h,\Pi_h v_h)_\Omega}}
  \\&\ge 
  \inf_{\overline{v}_h\in \mathcal{L}^0(\mathcal{T}_h)}{ \sup_{y_h\in \mathcal{R}T^0_N(\mathcal{T}_h)}{-\rho_{\varphi_h^*,\Omega}(\Pi_h y_h)-(\textup{div}\,y_h+f_h,\overline{v}_h)_\Omega}}
 \\&\ge 
  \sup_{y_h\in \mathcal{R}T^0_N(\mathcal{T}_h)}{\inf_{\overline{v}_h\in \mathcal{L}^0(\mathcal{T}_h)}{ -\rho_{\varphi_h^*,\Omega}(\Pi_h y_h)-(\textup{div}\,y_h+f_h,\overline{v}_h)_\Omega}}
  \\&=
  \sup_{y_h\in \mathcal{R}T^0_N(\mathcal{T}_h)}{- \rho_{\varphi_h^*,\Omega}(\Pi_h y_h)-\sup_{\overline{v}_h\in \mathcal{L}^0(\mathcal{T}_h)}{(\textup{div}\,y_h+f_h,\overline{v}_h)_\Omega}}
 \\&=
  \sup_{y_h\in \mathcal{R}T^0_N(\mathcal{T}_h)}{-\rho_{\varphi_h^*,\Omega}(\Pi_h y_h)-I_{\{-f_h\}}(\textup{div}\,y_h)}
 \\&=
  \sup_{y_h\in \mathcal{R}T^0_N(\mathcal{T}_h)}{D_h^{\textit{rt}}(y_h)}\,.
 \end{align*}

 \textit{ad (ii).} By definition, the discrete flux $z_h^{\textit{rt}}\in \mathcal{L}^1(\mathcal{T}_h)^d$, defined by \eqref{eq:gen_marini}, satisfies the discrete convex optimality condition \eqref{eq:pDirichletOptimalityCR1.2} and $\textup{div}\,(z_h^{\textit{rt}}|_T)=-f_h|_T$ in $T$ for all $T\in \mathcal{T}_h$. Due to $\vert \Gamma_D\vert >0$, the divergence operator $\textup{div}\colon 
 \mathcal{R}T^0_N(\mathcal{T}_h)\to \mathcal{L}^0(\mathcal{T}_h)$ is surjective. Hence, there exists
  $y_h\in \mathcal{R}T^0_N(\mathcal{T}_h)$ such that $\textup{div}\, y_h=-f_h$ in $\mathcal{L}^0(\mathcal{T}_h)$. Then, we have that $\textup{div}\,((z_h^{\textit{rt}}-y_h)|_T)=0$ in $T$ for all $T\in \mathcal{T}_h$, i.e., $z_h^{\textit{rt}}-y_h\in \mathcal{L}^0(\mathcal{T}_h)^d$. In addition, for every $v_h\in \mathcal{S}^{1,\textit{cr}}_D(\mathcal{T}_h)$, it holds
 \begin{align*}
  \begin{aligned}
  (\Pi_h y_h,\nabla_{\! h} v_h)_\Omega&=-(\textup{div}\, y_h,\Pi_h v_h)_\Omega\\&=(f_h,\Pi_h v_h)_\Omega\\&=(\AAA_h(\cdot,\nabla_{\! h} u_h^{\textit{cr}}),\nabla_{\! h} v_h)_\Omega
 \\& =(\Pi_h z_h^{\textit{rt}},\nabla_{\! h} v_h)_\Omega\,.
 \end{aligned}
 \end{align*}
 In other words, for every $v_h\in \mathcal{S}^{1,\textit{cr}}_D(\mathcal{T}_h)$, it holds
 \begin{align*}
  (y_h-z_h^{\textit{rt}},\nabla_{\! h} v_h)_\Omega=(\Pi_h y_h-\Pi_h z_h^{\textit{rt}},\nabla_{\! h} v_h)_\Omega=0\,,
 \end{align*}
 i.e., $y_h-z_h^{\textit{rt}}\in \nabla_{\! h}(\mathcal{S}^{1,\textit{cr}}_D(\mathcal{T}_h))^{\perp}$. By the decomposition \eqref{eq:decomposition}, we have that $\nabla_{\! h}(\mathcal{S}^{1,\textit{cr}}_D(\mathcal{T}_h))^{\perp}=\textup{ker}(\textup{div}|_{\mathcal{R}T^0_N(\mathcal{T}_h)})\subseteq \mathcal{R}T^0_N(\mathcal{T}_h)$. 
 As a result, we have that $y_h-z_h^{\textit{rt}}\in \mathcal{R}T^0_N(\mathcal{T}_h)$.~Due~to~${y_h\in \mathcal{R}T^0_N(\mathcal{T}_h)}$, we conclude that $z_h^{\textit{rt}}\in \mathcal{R}T^0_N(\mathcal{T}_h)$. In particular, now from
 $\textup{div}\,(z_h^{\textit{rt}}|_T)=-f_h|_T$ in $T$ for all $T\in \mathcal{T}_h$, it follows the discrete optimality condition
  \eqref{eq:pDirichletOptimalityCR1.1}.

 \textit{ad (iii).} Using \eqref{eq:pDirichletOptimalityCR2}, \eqref{eq:pDirichletOptimality1.1}, and the discrete integration-by-parts formula \eqref{eq:pi0},~we~find~that\enlargethispage{7mm}
 \begin{align*}
  I_h^{\textit{cr}}(u_h^{\textit{cr}})&=\rho_{\varphi_h,\Omega}(\nabla_{\! h} u_h^{\textit{cr}})-(f_h,\Pi_h u_h^{\textit{cr}})_\Omega
 \\& =-\rho_{\varphi_h^*,\Omega}(\Pi_h z_h^{\textit{rt}})+(\Pi_h z_h^{\textit{rt}},\nabla_{\! h} u_h^{\textit{cr}})_\Omega+(\textup{div}\,z_h^{\textit{rt}},\Pi_h u_h^{\textit{cr}})_\Omega
  \\&=-\rho_{\varphi_h^*,\Omega}(\Pi_h z_h^{\textit{rt}})-I_{\{-f_h\}}(\textup{div}\,z_h^{\textit{rt}})
  \\&=D_h^{\textit{rt}}(z_h^{\textit{rt}})\,,
 \end{align*}
 i.e., the discrete strong duality relation applies, which in conjunction with the discrete~weak~duality relation \eqref{eq:discrete_weak_duality} implies the maximality of $z_h^{\textit{rt}}\in \mathcal{R}T^0_N(\mathcal{T}_h)$ for \eqref{eq:pDirichletDualCR}. Since \eqref{eq:pDirichletDualCR} is strictly convex, $z_h^{\textit{rt}}\in \mathcal{R}T^0_N(\mathcal{T}_h)$ is unique.
 \end{proof}
	
 \newpage

 \subsection{Natural regularity assumption for $p\in C^{0,1}(\overline{\Omega})$}\enlargethispage{5mm}
	
	\qquad In this section, for $p\in C^{0,1}(\overline{\Omega})$, we briefly examine %collect  important consequences of 
 the natural regularity assumption 
    \begin{align}\label{natural_regularity}
        F(\cdot,\nabla u)\in W^{1,2}(\Omega;\mathbb{R}^d)\,,
    \end{align}   
    on the solution $u\in W^{1,p(\cdot)}_D(\Omega)$ of \eqref{eq:pDirichletW1p}, which is satisfied under mild assumptions on the domain $\Omega\subseteq\mathbb{R}^d$, $d\in \mathbb{N}$, the exponent $p\in C^{0,1}(\overline{\Omega})$ with $p^->1$ and the right-hand~side~$f\in \smash{L^{p'(\cdot)}(\Omega)}$, cf. \cite{ELM04}; see also
    \cite[Remark 4.5]{BDS13}.
	
	\begin{lemma}\label{lem:reg_primal_source}
            Let $p\in C^{0,1}(\overline{\Omega})$ with $p^->1$ and $\delta>0$. Then, for every  $v\in W^{1,p(\cdot)}(\Omega)$ with $F(\cdot,\nabla v)\in W^{1,2}(\Omega;\mathbb{R}^d)$, for $\mu(v)\coloneqq\vert \ln(\delta +\vert \nabla v\vert)\vert^2(\delta +\vert \nabla v\vert)^{p(\cdot)-2}\vert\nabla p\otimes \nabla v\vert^2\in L^1(\Omega)$,~it~holds
                \begin{align*}
                   (\delta+\vert \nabla v\vert )^{p(\cdot)-2}\vert \nabla^2 v\vert^2+\mu(v)\sim  \vert \nabla F(\cdot,\nabla v)\vert^2+\mu(v)\quad\text{ a.e.\ in }\Omega\,.
                \end{align*}
		\end{lemma}
		
		\begin{proof}
            Since the claimed equivalence reads $0\sim 0$ on $\{\vert\nabla v\vert =0\}$, we restrict to the case $\vert\nabla v\vert>0$. Here, by Rademacher's theorem, the product and chain rule, we find that
            \begin{align*}
                \nabla F(\cdot,\nabla v)&= (\delta+\vert \nabla v\vert)^{\frac{p-2}{2}}\big(\tfrac{\nabla p\otimes \nabla v}{2}\ln(\delta +\vert \nabla v\vert)+\tfrac{p-2}{2}\tfrac{\nabla\vert\nabla v\vert\otimes \nabla v}{\delta+\vert \nabla v\vert }+\nabla^2 v
                \big)\quad\text{ a.e.\ in }\Omega\,.
            \end{align*}
            Since $\nabla p\in L^\infty(\Omega;\mathbb{R}^d)$ as well as $ (\delta+\vert \nabla v\vert)^{p-2}\vert\nabla^2 v\vert^2\sim (\delta+\vert \nabla v\vert)^{p-4}\vert\nabla\vert\nabla v\vert\otimes \nabla v\vert^2$ a.e.\ in $\Omega$, cf.\ \cite[Lemma 2.8]{K22CR},
            we conclude the claimed equivalence on $\{\vert\nabla v\vert>0\}$.
		\end{proof}

  The following lemma %is of crucial importance for the derivation of optimal a priori estimates, since it 
  translates the natural regularity assumption \eqref{natural_regularity} to  $z\coloneqq \AAA(\cdot,\nabla u)\in \smash{W^{p'(\cdot)}_N(\textup{div};\Omega)}$ and vice versa. %This enables us later to estimate oscillation terms optimally.
		
		\begin{lemma}\label{lem:reg_equiv} Let $p\in C^{0,1}(\overline{\Omega})$ with $p^->1$ and $\delta\ge 0$. Then, for every $v\in W^{1,p(\cdot)}(\Omega)$ and $y\coloneqq \AAA(\cdot,\nabla v)\in L^{p'(\cdot)}(\Omega;\mathbb{R}^d)$, it holds  $F(\cdot,\nabla v)\in W^{1,2}(\Omega;\mathbb{R}^d)$ if and only if $F^*(\cdot,y)\in W^{1,2}(\Omega;\mathbb{R}^d)$. In particular, it holds $\vert F(\cdot,\nabla v)\vert^2 \sim \vert  F^*(\cdot,y)\vert^2$ a.e.\ in $\Omega$ and
        $\vert \nabla F(\cdot,\nabla v)\vert^2+(1+\vert \nabla v\vert^{p(\cdot)s}) \sim \vert \nabla F^*(\cdot,y)\vert^2+(1+\vert y\vert^{p'(\cdot)s})$ a.e.\ in $\Omega$ for some $s>1$ with can chosen to be close to $1$.
		\end{lemma}
		
		\begin{proof}
                The first equivalence is evident.  For the second equivalence, we denote by
          $\tau_h f \coloneqq \vert h\vert^{-1}(f (\cdot + h) - f)$ the difference quotient and exploit that, by Lemma \ref{lem:Ax-Axh}, for all $h\in \mathbb{R}^d$~small~enough
          \begin{align*}
             \vert \tau_h[F(\cdot,\nabla v)]\vert^2
             &\lesssim \vert F(\cdot\hspace{-0.15em}+\hspace{-0.15em}h,(\nabla v)(\cdot\hspace{-0.15em}+\hspace{-0.15em}h))\hspace{-0.15em}-\hspace{-0.15em}F(\cdot\hspace{-0.15em}+\hspace{-0.15em}h,\nabla v)\vert^2
             +\vert F(\cdot\hspace{-0.15em}+\hspace{-0.15em}h,\nabla v)\hspace{-0.15em}-\hspace{-0.15em}F(\cdot,\nabla v)\vert^2
              \\&\lesssim \vert F^*(\cdot\hspace{-0.15em}+\hspace{-0.15em}h,\AAA(\cdot\hspace{-0.15em}+\hspace{-0.15em}h,(\nabla v)(\cdot+h))\hspace{-0.15em}-\hspace{-0.15em}F^*(\cdot\hspace{-0.15em}+\hspace{-0.15em}h,\AAA(\cdot\hspace{-0.15em}+\hspace{-0.15em}h,\nabla v))\vert^2
           \hspace{-0.15em}+\hspace{-0.15em}\vert \tau_h p\vert (1\hspace{-0.15em}+\hspace{-0.15em}\vert \nabla v\vert^{p(\cdot)s})
             \\&\lesssim \vert \tau_h[F^*(\cdot,\AAA(\cdot,\nabla v))]\vert^2\hspace{-0.15em}+\hspace{-0.15em} \vert \tau_h p\vert (1\hspace{-0.05em}+\hspace{-0.05em}\vert\nabla v\vert^{p(\cdot)s})
            \\&\quad+\vert F^*(\cdot\hspace{-0.15em}+\hspace{-0.15em}h,\AAA(\cdot,\nabla v))\hspace{-0.15em}-\hspace{-0.15em}F^*(\cdot\hspace{-0.15em}+\hspace{-0.15em}h,\AAA(\cdot\hspace{-0.15em}+\hspace{-0.15em}h,\nabla v))\vert^2
             \\&\quad+\vert F^*(\cdot\hspace{-0.15em}+\hspace{-0.15em}h,\AAA(\cdot,\nabla v))\hspace{-0.15em}-\hspace{-0.15em}F^*(\cdot,\AAA(\cdot,\nabla v))\vert^2
            \\&\lesssim \vert \tau_h[F^*(\cdot,y)]\vert^2\hspace{-0.15em}+\hspace{-0.15em}\vert \tau_h p\vert (1+\vert y\vert^{p'(\cdot)s})\quad\textup{ a.e.\ in }\{x\in \Omega\mid \textup{dist}(x,\partial\Omega)>\vert h\vert\}\,.
          \end{align*}
         % \begin{align*}
         %    \vert \Delta_h[F(\cdot,\nabla v)](x)\vert^2
         %    &\lesssim \vert \Delta_h[F(x+h,\nabla v)](x)\vert^2 + \vert \Delta_h[F(\cdot,(\nabla v)(x)](x)\vert
         %     \\&\lesssim \vert \Delta_h[F^*(x+h,\AAA(x+h,\nabla v))](x)\vert^2 + \vert \Delta_h[F^*(x+h,\AAA(\cdot,(\nabla v)(x))](x)\vert^2
         %   \\&\quad + \vert \Delta_h[p](x)\vert (1+\vert (\nabla v)(x)\vert^{p(x)s})
         %    \\&\lesssim \vert \Delta_h[F^*(x+h,\AAA(\cdot,\nabla v))](x)\vert  + \vert \Delta_h[F^*(x+h,\AAA(\cdot,(\nabla v)(x)))](x)\vert^2
         %    \\&\quad+ \vert \Delta_h[p](x)\vert(1+\vert (\nabla v)(x)\vert^{p(x)s})
         %    \\&\lesssim\vert \Delta_h[F^*(x+h,\AAA(\cdot,\nabla v))](x)\vert^2 + \vert \Delta_h[p](x)\vert (1+\vert (\nabla v)(x)\vert^{p(x)s})
         %    \\&\lesssim \vert \Delta_h[F^*(\cdot,\AAA(\cdot,\nabla v))](x)\vert^2
         %+ \vert \Delta_h[F^*(\cdot,\AAA(x,(\nabla v)(x))](x)\vert^2
         %    \\&\quad+ \vert \Delta_h[p](x)\vert(1+\vert (\nabla v)(y)\vert^{p(y)s})
         %    \\&\lesssim \vert \Delta_h[F^*(\cdot,\AAA(\cdot,\nabla v))](x)\vert + \vert \Delta_h[p](x)\vert (1+\vert (\nabla v)(y)\vert^{p(y)s})
         % \end{align*}
        Similarly, \hspace{-0.1mm}we \hspace{-0.1mm}find \hspace{-0.1mm}that \hspace{-0.1mm}$\vert \tau_h[F^*(\cdot,y)]\vert\hspace{-0.15em}\lesssim\hspace{-0.15em}\vert \tau_h[F(\cdot,\nabla v)]\vert+(1+\vert \nabla v\vert^{p(\cdot)s})$ \hspace{-0.1mm}a.e.\ \hspace{-0.1mm}in \hspace{-0.1mm}${\{x\hspace{-0.15em}\in\hspace{-0.15em} \Omega\mid \textup{dist}(x,\partial\Omega)\hspace{-0.15em}>\hspace{-0.15em}\vert h\vert\}}$. Passing to the limit $\vert h\vert \to 0$ proves the claim.
		\end{proof}

        Lemma \ref{lem:reg_equiv}, in turn, motivates to prove the following dual counterparts of Lemma \ref{lem:reg_primal_source}.
		
		\begin{lemma}\label{lem:reg_dual_source}
        Let $p\in C^{0,1}(\overline{\Omega})$ with $p^->1$ and $\delta> 0$. Then, for every  $y\in L^{p'(\cdot)}(\Omega;\mathbb{R}^d)$~with $F^*(\cdot,y)\hspace{-0.15em}\in \hspace{-0.15em} W^{1,2}(\Omega;\mathbb{R}^d)$, for  $\mu^*(y)\hspace{-0.15em}\coloneqq\hspace{-0.15em}\vert \ln(\delta^{p(\cdot)-1} +\vert y\vert)\vert^2(\delta^{p(\cdot)-1} +\vert y\vert)^{p'(\cdot)-2}\vert \nabla p'\otimes y\vert^2\hspace{-0.15em}\in\hspace{-0.15em} L^1(\Omega)$,~it~holds
                \begin{align*}
                 (\delta^{p(\cdot)-1}+\vert y\vert )^{p'(\cdot)-2}\vert \nabla y\vert^2
                +\mu^*(y)\sim  \vert \nabla F^*(\cdot, y)\vert^2+\mu^*(y)\quad\text{ a.e.\ in }\Omega\,.
                \end{align*}
		\end{lemma}
		
		\begin{proof}
        Since the claimed equivalence reads $0\sim 0$ on $\{\vert y\vert =0\}$, we restrict to the case $\vert y\vert>0$. Here, by Rademacher's theorem, the product and chain rule, we find that
            \begin{align*}
                \nabla F^*(\cdot,y)&= (\delta^{p-1}\!+\vert y\vert)^{\frac{p'-2}{2}}\big(\tfrac{\nabla p'\otimes y}{2}\ln(\delta^{p-1}\! +\vert y\vert)+\tfrac{p'-2}{2}\tfrac{(\nabla p \ln(\delta)\delta^{p-1}+\nabla\vert y\vert)\otimes y}{\delta^{p-1}+\vert y\vert }+
                \nabla y\big)\quad\text{ a.e.\ in }\Omega\,.
            \end{align*}
        Since $\nabla p'\in L^\infty(\Omega;\mathbb{R}^d)$ as well as $ (\delta^{p-1}+\vert y\vert)^{p'-2}\vert\nabla y\vert^2\sim (\delta^{p-1}+\vert y\vert)^{p'-4}\vert  \nabla\vert y\vert\otimes y\vert^2$ a.e.\ in $\Omega$, cf.\ \cite[Lemma 2.11]{K22CR}, 
            we conclude the claimed equivalence on $\{\vert y\vert >0\}$.
		\end{proof}\newpage
	\section{Medius error analysis}\label{sec:medius}

 	\qquad In this section, we prove a best-approximation result for the $\mathcal{S}^{1,cr}_D(\mathcal{T}_h)$-approximation \eqref{eq:pDirichletS1crD} of \eqref{eq:pDirichletW1p}.%, cf.\ \mbox{Theorem} \ref{P1_best-approx}.
  
 \begin{theorem}\label{thm:best-approx} 
 Let $p\in C^0(\overline{\Omega})$ with $p^->1$ and $\delta\ge 0$ and let $f\in L^{p'(\cdot)}(\Omega)\cap \bigcap_{h\in (0,h_0]}{L^{p_h'(\cdot)}(\Omega)}$ for some $h_0\hspace{-0.15em}>\hspace{-0.15em}0$. \hspace{-0.2em}Then, there exists some $s\hspace{-0.15em}>\hspace{-0.15em}1$, which can chosen to be close~to~$1$~if~${h\hspace{-0.15em}>\hspace{-0.15em}0}$~is~close~to~$0$, such that if $u\in W^{1,p(\cdot)s}_D(\Omega)$, then for every $h\in (0,h_0]$, it holds
		\begin{align*}%\label{eq:best-approx}
    \begin{aligned}
		\|F_h(\cdot,\nabla_{\!h} u_h^{\textit{cr}})-F_h(\cdot,\nabla u)\|_{2,\Omega}^2&\lesssim \inf_{v_h\in\mathcal{S}^1_D(\mathcal{T}_h)}{\big[\|F_h(\cdot,\nabla v_h)-F_h(\cdot,\nabla u)\|_{2,\Omega}^2+\mathrm{osc}_h(f,v_h)}\\&\;\;+\|\omega_p(h_{\mathcal{T}})^2\,(1+\vert \nabla u\vert^{p(\cdot)s}+(\vert \nabla v_h\vert+\vert \nabla v_h-\nabla_{\!h} u_h^{\textit{cr}}\vert)^{p_h(\cdot)s})\|_{1,\Omega}\big]\,,
 \end{aligned}  
		\end{align*}
		where the hidden constant  also depends on $s\hspace{-0.1em}>\hspace{-0.1em}1$ and the chunkiness $\omega_0\hspace{-0.1em}>\hspace{-0.1em}0$, and~for~all~${v_h\hspace{-0.1em}\in\hspace{-0.1em} \smash{\mathcal{S}^1_D(\mathcal{T}_h)}}$ and $\mathcal{M}_h\subseteq \mathcal{T}_h $, we define $\mathrm{osc}_h(f,v_h,\mathcal{M}_h)\coloneqq \sum_{T\in \mathcal{M}_h}{\mathrm{osc}_h(f,v_h,T)}$, 
		%\begin{align*}
		% \mathrm{osc}_h(f,v_h,\mathcal{M}_h)\coloneqq \sum_{T\in \mathcal{M}_h}{\mathrm{osc}_h(f,v_h,T)}\,,
		%\end{align*}
		where we define $\mathrm{osc}_h(f,v_h,T)$ $\coloneqq \rho_{((\varphi_h)_{\vert \nabla v_h\vert})^*,T}(h_T (f-f_h))$
 for all $T\in \mathcal{T}_h$ 
 and $\mathrm{osc}_h(f,v_h)\coloneqq\mathrm{osc}_h(f,v_h,\mathcal{T}_h)$.
	\end{theorem}

     The proof of Theorem \ref{thm:best-approx}
     involves three tools.\vspace{-1mm}
		% Before we will prove Theorem \ref{thm:best-approx} below, we will first introduce some tools.\vspace{-1mm}
		 
		 \subsection{Node-averaging quasi-interpolation operator}\label{subsec:node-averaging}

        \hspace{5mm}The \hspace{-0.1mm}first \hspace{-0.1mm}tool \hspace{-0.1mm}is \hspace{-0.1mm}the \hspace{-0.1mm}\textit{node-averaging \hspace{-0.1mm}quasi-interpolation \hspace{-0.1mm}operator} $I_h^{\textit{av}}\colon \mathcal{L}^1(\mathcal{T}_h)\to \mathcal{S}^1_D(\mathcal{T}_h)$,~that, 
		  denoting~for~every~${z\in \mathcal{N}_h}$,
		% \qquad The first tool is the node-averaging operator and its uniform approximation and stability properties with respect to shifted $N$-functions, cf.\ \cite{Osw93,Sus96,EG21}.
		% The \textit{node-averaging quasi-interpolation operator} $I_h^{\textit{av}}\colon \mathcal{L}^1(\mathcal{T}_h)\to \mathcal{S}^1_D(\mathcal{T}_h)$, 
		% denoting~for~${z\hspace{-0.05em}\in\hspace{-0.05em} \mathcal{N}_h}$, 
        by ${\mathcal{T}_h(z)\coloneqq \{T\in \mathcal{T}_h\mid z\in T\}}$, the set of elements sharing~$z$, for every ${v_h\in \mathcal{L}^1(\mathcal{T}_h)}$,~is~defined~by
	\begin{align*}
		I_h^{\textit{av}}v_h\coloneqq \sum_{z\in \smash{\mathcal{N}_h}}{\langle v_h\rangle_z\varphi_z}\,,\qquad \langle v_h\rangle_z\coloneqq \begin{cases}
			\frac{1}{\textup{card}(\mathcal{T}_h(z))}\sum_{T\in \mathcal{T}_h(z)}{(v_h|_T)(z)}&\;\text{ if }z\in \Omega\cup \Gamma_N\\
			0&\;\text{ if }z\in \Gamma_D
		\end{cases}\,,
	\end{align*}
	where we denote by $(\varphi_z)_{\smash{z\in \mathcal{N}_h}}$, the nodal basis of $\mathcal{S}^1(\mathcal{T}_h)$.
	If $p\in C^0(\overline{\Omega})$ with $p^->1$ and $\delta\ge 0$, then there exists some $s>1$, which can chosen to be close to $1$ if $h_T>0$ is close to $1$,  such that 
 for every $a\ge 0$, $v_h\in \smash{\smash{\mathcal{S}^{1,\textit{cr}}_D(\mathcal{T}_h)}}$, $T\in \mathcal{T}_h$, and $m\in \{0,1\}$, cf.\ \cite[Corollary~A.2]{BK22B}, it holds%\footnote{Here, $\nabla_{\!h}^m\colon\mathcal{L}^1(\mathcal{T}_h)\to \mathcal{L}^{1-m}(\mathcal{T}_h)^{d^m}$, for every $v_h\in\mathcal{L}^1(\mathcal{T}_h)$ defined by $(\nabla_{\!h}^mv_h)|_T\coloneqq \nabla^m(v_h|_T)$ for all $T\in \mathcal{T}_h$, denotes the element-wise $m$-th gradient operator.}
	\begin{align}\label{eq:a6}
	 %\begin{aligned}
	 &\int_T{\varphi_a(q_T,h_T^m\vert\nabla^m_{\!h}(v_h-I_h^{\textit{av}}v_h)\vert)\,\mathrm{d}x}
	\lesssim \int_{\omega_T}{\varphi_a(q_T,h_T\vert\nabla_{\!h}v_h\vert)\,\mathrm{d}x}\\
  &\quad\lesssim \int_{\omega_T}{(\varphi_h)_a(\cdot,h_T\vert\nabla_{\!h}v_h\vert)\,\mathrm{d}x}+\omega_{p,\omega_T}(h_T)^2\,\|1+a^{p_h(\cdot)s}+\vert\nabla_{\!h}v_h\vert^{p_h(\cdot)s}\|_{1,\omega_T} \,,\notag
	 %\end{aligned}
	\end{align}
    where we exploit that $\varphi_a(q_T,t)\lesssim \varphi_a(q_{T'},t) +\vert p(q_T)-p(q_{T'})\vert^2\,(1+a^{p(q_{T'})s}+t^{p(q_{T'})s})$~for~all~$T'\in \mathcal{T}_h$ with $T'\subseteq \omega_T$ for 
    the second inequality, which follows analogously to the estimate \eqref{eq:phixh-phix}.
	
	\subsection{Local efficiency estimates}

    \qquad The second tool are local efficiency estimates that are  based on %standard 
    bubble function techniques.\enlargethispage{2mm}%, cf. \cite{Ver94}.
    
	% \qquad The second tool involves the following local efficiency estimates.\vspace{-1mm}\enlargethispage{12mm}
		
		\begin{lemma}\label{lem:efficiency}
		Let $p\in C^0(\overline{\Omega})$ with $p^->1$ and $\delta\ge 0$ and let $f\in L^{p'(\cdot)}(\Omega)\cap \bigcap_{h\in (0,h_0]}{L^{p_h'(\cdot)}(\Omega)}$ for some $h_0>0$. Then, there exists some $s>1$, which can chosen to be close to $1$ if $h>0$ is close to $0$, such that if $u\in W^{1,p(\cdot)s}_D(\Omega)$, then for~every~${h\in (0,h_0]}$,~${v_h\in \mathcal{S}^1_D(\mathcal{T}_h)}$, $T\in \mathcal{T}_h$,~and~$S\in \mathcal{S}_h^{i}$,~it~holds 
		 \hspace{-3mm}\begin{align}
		 \rho_{((\varphi_h)_{\vert \nabla v_h\vert})^*,T}(h_Tf_h)&\lesssim \|F_h(\cdot,\nabla v_h)-F_h(\cdot,\nabla u)\|_{2,T}^2 \notag\\&\quad+\|\omega_{p,T}(h_T)^2\,(1+\vert \nabla u\vert^{p(\cdot)s})\|_{1,T}+\mathrm{osc}_h(f,v_h,T)\,,\label{lem:efficiency.1}\\
		 h_S\|\jump{F_h(\cdot,\nabla v_h)}_S\|_{2,S}^2&\lesssim \|F_h(\cdot,\nabla v_h)-F_h(\cdot,\nabla u)\|_{2,\omega_S}^2\notag\\&\quad+\|\omega_{p,\omega_S}(h_S)^2\,(1+\vert \nabla u\vert^{p(\cdot)s}+\vert \nabla v_h\vert^{p_h(\cdot)s})\|_{1,\omega_S}+\mathrm{osc}_h(f,v_h,\omega_S)\label{lem:efficiency.2}\,,\hspace{-1mm}
		 \end{align}
        where the hidden constants in \eqref{lem:efficiency.1} and \eqref{lem:efficiency.2} also depend on $s>1$ and the chunkiness $\omega_0>0$, and for every $\mathcal{M}_h\subseteq\mathcal{T}_h$, we define $\omega_{p,\mathcal{M}_h}(t)|_T\coloneqq \omega_{p,T}(t)$~for~all~$t\ge 0$ and $T\in \mathcal{T}_h$ and $\omega_p\coloneqq \omega_{p,\mathcal{T}_h}$.
		\end{lemma}
		
		\begin{proof}
		 We generalize the procedure in the proof of \cite[Lemma 3.2]{K22CR}.
		 
		 \textit{ad \eqref{lem:efficiency.1}.} Let $T \hspace{-0.1em}\in \hspace{-0.1em} \mathcal{T}_h$ be fixed, but arbitrary. Then, there exists a bubble~function~${b_T \hspace{-0.1em}\in  \hspace{-0.1em}W^{1,\infty}_0(T)}$ such that $0 \hspace{-0.1em}\leq\hspace{-0.1em} b_T  \hspace{-0.1em}\lesssim \hspace{-0.1em} 1 $ in $T$, $\vert \nabla b_T\vert \hspace{-0.1em}\lesssim\hspace{-0.1em}\smash{h^{-1}_T}$ in $T$, and $\smash{\fint_T{b_T\,\mathrm{d}x}} = 1$, where~the~hidden~\mbox{constant}~\mbox{depends} only on the chunkiness $\omega_0>0$.
		 Using \eqref{eq:pDirichletW1p} and integration-by-parts, taking~into~account that $\AAA_h(\cdot,\nabla v_h)\in \mathcal{L}^0(\mathcal{T}_h)^d$ and ${b_T\in W^{1,\infty}_0(T)}$, for every $\lambda\in \mathbb{R}$, we find that
		 \begin{align}
		 (\AAA(\cdot,\nabla u)-\AAA_h(\cdot,\nabla v_h), \nabla(\lambda b_T))_T=(f,\lambda b_T)_T\,.\label{lem:efficiency.5}
		 \end{align}
		 For the special choice $\lambda_T\coloneqq\textup{sgn}(f_h)\partial_a((\varphi_{\vert\nabla v_h\vert})^*)(q_T,h_T\vert f_h\vert )\in\mathbb{R}$, 
		 by the Fenchel--Young~identity, cf.\ \cite[Proposition~5.1,~p.~21]{ET99}, we obtain
		 \begin{align}
		  (h_T f_h)\lambda_T=(\varphi_{\vert\nabla v_h\vert})^*(q_T,h_T\vert f_h\vert)+\varphi_{\vert\nabla v_h\vert}(q_T,\vert\lambda_T\vert)\,.\label{lem:efficiency.6}
		 \end{align}
		 Then, choosing $\lambda=h_T\lambda_T\in \mathbb{R}$, cf.\ \eqref{lem:efficiency.6}, in \eqref{lem:efficiency.5}, we observe that
		 \begin{align}\label{lem:efficiency.7}
		 \begin{aligned}
		  \rho_{((\varphi_h)_{\vert\nabla v_h\vert})^*,T}(h_Tf_h)+\rho_{(\varphi_h)_{\vert\nabla v_h\vert},T}(\lambda_T)&=(f,h_T\lambda_T b_T)_T\\&\quad+( f_h-f,h_T\lambda_T b_T)_T
		  \\&= (\AAA_h(\cdot,\nabla u)-\AAA_h(\cdot,\nabla v_h), \nabla(h_T\lambda_T b_T))_T\\&\quad+
 (\AAA(\cdot,\nabla u)-\AAA_h(\cdot,\nabla u), \nabla(h_T\lambda_T b_T))_T\\&\quad +(f_h-f,h_T\lambda_T b_T)_T
  \\&\eqqcolon I_1+I_2+I_3\,.
		  \end{aligned}
		 \end{align}
		 Applying the $\varepsilon$-Young inequality \eqref{ineq:young} with $\psi=\varphi_{\vert\nabla v_h\vert}(q_T,\cdot)$ together~with~\eqref{eq:hammera}, also using that $\vert b_T\vert+h_T\vert \nabla b_T\vert\lesssim 1$ in $T$, we find that
		 \begin{align}\label{lem:efficiency.8}
		 \begin{aligned}
		  I_1&\leq c_\varepsilon\,\|F_h(\cdot,\nabla v_h)-F_h(\cdot,\nabla u)\|_{2,T}^2+\varepsilon\,\rho_{(\varphi_h)_{\vert\nabla v_h\vert},T}(\lambda_T)\,,\\
		  I_3&\leq c_\varepsilon\,\mathrm{osc}_h(f,v_h,T)+\varepsilon\,\rho_{(\varphi_h)_{\vert\nabla v_h\vert},T}(\lambda_T)\,.
		 \end{aligned}
		 \end{align}
  Applying the $\varepsilon$-Young inequality \eqref{ineq:young} with $\psi=\varphi_{\vert\nabla u\vert}(q_T,\cdot)$ together~with~\eqref{eq:hammerh},  $\vert b_T\vert+h_T\vert \nabla b_T\vert\lesssim 1$ in $T$, the shift change \eqref{lem:shift-change.1}, and Lemma~\ref{lem:A-Ah}~\eqref{eq:Ah-A}, we obtain
		 \begin{align}\label{lem:efficiency.8.1}
   \begin{aligned}
 \hspace{-2mm}I_2&
 \leq c_\varepsilon\,\|F_h^*(\cdot,\AAA_h(\cdot,\nabla u))-F_h^*(\cdot,\AAA(\cdot,\nabla u))\|_{2,T}^2
 +\varepsilon\,\rho_{(\varphi_h)_{\vert\nabla u\vert},T}(\lambda_T)
 \\&\lesssim c_\varepsilon\,\|\omega_{p,T}(h_T)^2\, (1+\vert \nabla u\vert)^{p(\cdot)s}\|_{1,T}
 +\varepsilon\,\big[\rho_{(\varphi_h)_{\vert\nabla v_h\vert},T}(\lambda_T)+\|F_h(\cdot,\nabla v_h)-F_h(\cdot,\nabla u)\|_{2,T}^2\big]\,,\hspace{-2mm}
 \end{aligned}
		 \end{align}
  Taking into account  \eqref{lem:efficiency.8} and \eqref{lem:efficiency.8.1} in \eqref{lem:efficiency.7}, for sufficiently small $\varepsilon>0$,~we~conclude~that
		 \begin{align}\label{lem:efficiency.9}
		 \begin{aligned}
		  \rho_{((\varphi_h)_{\vert\nabla v_h\vert})^*,T}(h_T f_h)&\lesssim \|F_h(\cdot,\nabla v_h)-F_h(\cdot,\nabla u)\|_{2,T}^2\\&\quad+\|\omega_{p,T}(h_T)^2 (1+\vert \nabla u\vert^{p(\cdot)s})\|_{1,T}+\mathrm{osc}_h(f,v_h,T)\,.
		  \end{aligned}
		 \end{align}
		 
		 \textit{ad \eqref{lem:efficiency.2}.} Let $S \hspace{-0.17em}\in \hspace{-0.17em} \mathcal{S}_h^{i}$ be fixed, but arbitrary. \!Then, there exists a bubble~function~${b_S \hspace{-0.17em}\in\hspace{-0.17em} W^{1,\infty}_0(\omega_S)}$ such that $0 \hspace{-0.15em}\leq \hspace{-0.15em}b_S\hspace{-0.15em} \lesssim\hspace{-0.15em} 1 $ in $\omega_S$, $\vert \nabla b_S\vert \hspace{-0.15em}\lesssim\hspace{-0.15em}  \smash{h^{-1}_S}$ in $\omega_S$, and $\smash{\fint_S{b_S\,\mathrm{d}s}} \hspace{-0.1em}=\hspace{-0.1em} 1$, where~the~hidden~\mbox{constant}~\mbox{depends} only on the chunkiness $\omega_0>0$.
		 Using \eqref{eq:pDirichletW1p} and integration-by-parts, taking into account~that $\AAA_h(\cdot,\nabla v_h)\in \mathcal{L}^0(\mathcal{T}_h)^d$ and $b_S\in W^{1,\infty}_0(\omega_S)$ with $\smash{\fint_S{b_S\,\mathrm{d}s}}=1$, for every $\lambda\in \mathbb{R}$,~we~find~that\enlargethispage{3mm}
		 \begin{align}\label{lem:efficiency.10}
		 (\AAA(\cdot,\nabla u)-\AAA_h(\cdot,\nabla v_h), \nabla(\lambda b_S))_{\omega_S}=(f,\lambda b_S)_{\omega_S}- \vert S\vert \jump{\AAA_h(\cdot,\nabla v_h)\cdot n}_S\lambda\,.
		 \end{align}
		 Let $T\in \mathcal{T}_h$ be such that $T\subseteq \omega_S$. Then, using the notation $\nabla v_h(T)\coloneqq\nabla v_h|_T$\footnote{In what follows, we employ this notation to indicate that the value of $\nabla v_h(T)\in \mathbb{R}^d$ depends only on the value of $\nabla v_h\in \mathcal{L}^0(\mathcal{T}_h)$ on $T\in \mathcal{T}_h$.}, for the choice
  \begin{align*}
  \lambda_{S,T}\coloneqq \textup{sgn}(\jump{\AAA_h(\cdot,\nabla v_h)\cdot n}_S)\partial_a((\varphi_{\vert\nabla v_h(T)\vert})^*)(q_T,\vert \jump{\AAA_h(\cdot,\nabla v_h)\cdot n}_S\vert)\in \mathbb{R}\,,
  \end{align*}
		 by the Fenchel--Young identity, cf.\ \cite[Proposition~5.1,~p.~21]{ET99},~it~holds
		 \begin{align}\label{lem:efficiency.11}
		 \jump{\AAA_h(\cdot,\nabla v_h)\cdot n}_S\lambda_{S,T}=(\varphi_{\vert\nabla v_h(T)\vert})^*(q_T,\vert\jump{\AAA_h(\cdot,\nabla v_h)\cdot n}_S\vert)+\varphi_{\vert\nabla v_h(T)\vert}(q_T,\vert\lambda_{S,T}\vert)\,.
		 \end{align}
		 Next, let $T'\in \mathcal{T}_h\setminus\{T\}$ be such that $T'\subseteq \omega_S$. Then, due to the convexity of $(\varphi_{\vert\nabla v_h(T)\vert})^*(q_T,\cdot)$~and $\Delta_2((\varphi_{\vert\nabla v_h(T)\vert})^*(q_T,\cdot)) \lesssim 2^{\smash{\max\{2,(p^-)'\}}}$, also using the shift change \eqref{lem:shift-change.3} and \eqref{eq:hammerh},~we~have~that
		 \begin{align}\label{lem:efficiency.12.0}
  \begin{aligned}
		 (\varphi_{\vert\nabla v_h(T)\vert})^*&(q_T,\vert\jump{\AAA_h(\cdot,\nabla v_h)\cdot n}_S\vert)%\\&=(\varphi_{\vert\nabla v_h(T)\vert})^*(q_T,\vert (\AAA(q_T,\nabla v_h(T))-\AAA(q_{T'},\nabla v_h(T')))\cdot n_{T}\vert)
  \\&\gtrsim (\varphi_{\vert\nabla v_h(T)\vert})^*(q_T,\vert \jump{\AAA(q_T,\nabla v_h)}_S\cdot n_{T}\vert)\\&\quad-
  (\varphi_{\vert\nabla v_h(T)\vert})^*(q_T,\vert (\AAA(q_T,\nabla v_h(T'))-\AAA(q_{T'},\nabla v_h(T')))\cdot n_{T}\vert)
  \\&\gtrsim (\varphi_{\vert\nabla v_h(T)\vert})^*(q_T,\vert \jump{\AAA(q_T,\nabla v_h)}_S\cdot n_{T}\vert)\\&\quad-
  c_\varepsilon\,(\varphi_{\vert\nabla v_h(T')\vert})^*(q_T,\vert (\AAA(q_T,\nabla v_h(T'))-\AAA(q_{T'},\nabla v_h(T')))\vert)
  \\&\quad-
  \varepsilon\,\vert F(q_T,\nabla v_h(T'))-F(q_T,\nabla v_h(T))\vert^2
  \\&\gtrsim (\varphi_{\vert\nabla v_h(T)\vert})^*(q_T,\vert \jump{\AAA(q_T,\nabla v_h)}_S\cdot n_{T}\vert)
 \\&\quad-c_\varepsilon\,\vert F^*(q_T,\AAA(q_T,\nabla v_h(T')))-F^*(q_T,\AAA(q_{T'},\nabla v_h(T')))\vert^2 
 \\&\quad-
  \varepsilon\,\vert\jump{F(q_T,\nabla v_h)}_S\vert^2
  \eqqcolon I_1-c_\varepsilon\, I_2-\varepsilon\, I_3\,.
  \end{aligned}
		 \end{align}
 Resorting to Lemma \ref{lem:Ax-Axh} \eqref{eq:Axh-Ax}, we deduce that
 \begin{align}\label{lem:efficiency.12.1}
 \smash{ I_2\lesssim \omega_{p,\omega_S}(h_S)^2\, (1+\vert \nabla_{\! h} v_h(T')\vert^{p(q_{T'})s})\,.}
 \end{align}
 Using that $n_T=\pm\frac{\jump{\nabla v_h}_S}{\vert \jump{\nabla v_h}_S\vert}$ since $v_h\in \mathcal{S}^1_D(\mathcal{T}_h)$, cf.\ \cite[p.\ 12]{DK08}, and \eqref{eq:hammera}, we find that
 \begin{align*}
 \smash{\vert \jump{\AAA(q_T,\nabla v_h)}_S\cdot n_{T}\vert  
 \sim \tfrac{\varphi_{\vert \nabla v_h(T)\vert }(q_T,\vert \jump{\nabla v_h}_S\vert)}{\vert \jump{\nabla v_h}_S\vert}\sim\partial_a(\varphi_{\vert \nabla v_h(T)\vert })(q_T,\vert \jump{\nabla v_h}_S\vert) \,,}
 \end{align*}
 and, thus, using $(\varphi_{\vert\nabla v_h(T)\vert})^*(q_T,\cdot)\circ \partial_a(\varphi_{\vert\nabla v_h(T)\vert})(q_T,\cdot)\sim \vert F(q_T,\cdot)\vert^2$, cf.\ \cite[(2.6)]{DK08},~we~get~$I_1\sim I_3$.
 Therefore, for sufficiently small $\varepsilon>0$, resorting to Lemma \ref{lem:Ax-Axh} \eqref{eq:Fxh-Fx}, we deduce that
 \begin{align}\label{lem:efficiency.12.2}
\begin{aligned}
 I_1-\varepsilon\, I_3&\gtrsim\vert\jump{F_h(\cdot,\nabla v_h)}_S\vert^2- \vert F(q_{T'},\nabla v_h(T'))-F(q_T,\nabla v_h(T'))\vert^2  \\&\gtrsim\vert\jump{F_h(\cdot,\nabla v_h)}_S\vert^2-\omega_{p,\omega_S}(h_S)^2\, (1+\vert \nabla v_h(T')\vert^{p(q_{T'})s})
  \,.\end{aligned}
 \end{align}
 Combining \eqref{lem:efficiency.12.1} and \eqref{lem:efficiency.12.2} in \eqref{lem:efficiency.12.0}, we arrive at
 \begin{align}\label{lem:efficiency.12}
 \hspace{-2mm}\vert\jump{F_h(\cdot,\hspace{-0.1em}\nabla v_h)}_S\vert^2\hspace{-0.1em}\lesssim \hspace{-0.1em}(\varphi_{\vert\nabla v_h(T)\vert})^*(q_T,\hspace{-0.1em}\vert\jump{\AAA_h(\cdot,\hspace{-0.1em}\nabla v_h)\cdot n}_S\vert)\hspace{-0.1em}+\hspace{-0.1em}\omega_{p,\omega_S}(h_S)^2\, (1\hspace{-0.1em}+\hspace{-0.1em}\vert \nabla v_h(T')\vert^{p(q_{T'})s})\,.\hspace{-1mm}
 \end{align}
		 For $\smash{\lambda=\frac{\vert \omega_S\vert}{\vert S\vert}\lambda_{S,T}\in \mathbb{R}}$,~cf.~\eqref{lem:efficiency.11}, in \eqref{lem:efficiency.10}, also using \eqref{lem:efficiency.12}, we observe that
		 \begin{align}\label{lem:efficiency.12.2.1}
		 \begin{aligned}
		  h_S\|\jump{F_h(\cdot,\nabla v_h)}_S\|_{2,S}^2
		  &+\rho_{(\varphi_h)_{\vert\nabla v_h(T)\vert},\omega_S}(\lambda_{S,T})-\| \omega_{p,\omega_S}(h_S)^2(1+\vert \nabla v_h\vert^{p_h(\cdot)s})\|_{1,\omega_S}
		  \\&\lesssim \vert \omega_S\vert \jump{\AAA_h(\cdot,\nabla v_h)\cdot n}_S\lambda_{S,T}
 \\&= \tfrac{\vert \omega_S\vert}{\vert S\vert}(\AAA_h(\cdot,\nabla v_h)-\AAA_h(\cdot,\nabla u), \nabla(\lambda_{S,T} b_S))_{\omega_S}\\&\quad+\tfrac{\vert \omega_S\vert}{\vert S\vert}(\AAA_h(\cdot,\nabla u)-\AAA(\cdot,\nabla u), \nabla(\lambda_{S,T} b_S))_{\omega_S}
		  \\&\quad+\tfrac{\vert \omega_S\vert}{\vert S\vert}(f_h,\lambda_{S,T} b_S)_{\omega_S}+\tfrac{\vert \omega_S\vert}{\vert S\vert}(f-f_h,\lambda_{S,T} b_S)_{\omega_S}
  \\&\eqqcolon I_1+I_2+I_3+I_4\,.
		 \end{aligned}
		 \end{align}
		 Applying the $\varepsilon$-Young inequality \eqref{ineq:young} with $\psi=\varphi_{\vert\nabla v_h(T')\vert }(q_{T'},\cdot)$ together with \eqref{eq:hammera} for all $T'\in \mathcal{T}_h$ with $T'\subseteq \omega_S$, and $\vert b_S\vert+h_S\vert \nabla b_S\vert\lesssim 1$ in $\omega_S$, we obtain\enlargethispage{3mm}
		 \begin{align}\label{lem:efficiency.13}
		 \begin{aligned}
		 I_1 &\leq \smash{c_\varepsilon\,\|F_h(\cdot,\nabla v_h)-F_h(\cdot,\nabla u)\|_{2,\omega_S}^2+\varepsilon\,\rho_{(\varphi_h)_{\vert\nabla v_h\vert},\omega_S}(\lambda_{S,T})}\,,\\[0.5mm]
		  I_3&\leq 
		 \smash{ c_\varepsilon\,\rho_{((\varphi_h)_{\vert\nabla v_h\vert})^*,\omega_S}(h_{\mathcal{T}}f)+\varepsilon\,\rho_{(\varphi_h)_{\vert\nabla v_h\vert},\omega_S}(\lambda_{S,T})}\,,\\[0.5mm]
     I_4&\leq 
		  \smash{c_\varepsilon\,\textup{osc}(f,v_h,\omega_S)+\varepsilon\,\rho_{(\varphi_h)_{\vert\nabla v_h\vert},\omega_S}(\lambda_{S,T})}\,.
		  \end{aligned}
		 \end{align}
  Applying the $\varepsilon$-Young inequality \eqref{ineq:young} with $\psi=\varphi_{\vert\nabla u\vert }(q_{T'},\cdot)$ together with \eqref{eq:hammerh} for all $T'\in \mathcal{T}_h$ with $T'\subseteq \omega_S$, $\vert b_S\vert+h_S\vert \nabla b_S\vert\lesssim 1$ in $\omega_S$, the shift change \eqref{lem:shift-change.1},  and Lemma~\ref{lem:A-Ah} \eqref{eq:Ah-A},~we~obtain
 \begin{align}\label{lem:efficiency.13.2}
		 %\begin{aligned}
  I_2& 
  \leq c_\varepsilon\,\|F_h^*(\cdot,\AAA_h(\cdot, \nabla u))-F_h^*(\cdot,\AAA(\cdot,\nabla u))\|_{2,\omega_S}^2
+\varepsilon\,\rho_{(\varphi_h)_{\vert\nabla u\vert},\omega_S}(\lambda_{S,T})
  \\&\lesssim c_\varepsilon\,\|\omega_p(h_{\mathcal{T}})^2\,(1+\vert \nabla u\vert^{p(\cdot)s})\|_{1,\omega_S}
  +\varepsilon\,\big[\rho_{(\varphi_h)_{\vert\nabla v_h\vert},\omega_S}(\lambda_{S,T})+\|F_h(\cdot,\nabla v_h)-F_h(\cdot,\nabla u)\|_{2,\omega_S}^2\big]\,.\notag
		  %\end{aligned}
		 \end{align}
    The shift change \eqref{lem:shift-change.1} on $T'\in \mathcal{T}_h\setminus\{T\}$ with $T'\subseteq \omega_S$ further yields that
        \begin{align}\label{lem:efficiency.14}
            \begin{aligned}
                \rho_{(\varphi_h)_{\vert\nabla v_h\vert},\omega_S}(\lambda_T^S)&\lesssim \rho_{(\varphi_h)_{\vert\nabla v_h(T)\vert},\omega_S}(\lambda_T^S)+h_S\|\jump{F(q_{T'},\nabla v_h)}_S\|_{2,S}^2
                \\&\lesssim \rho_{(\varphi_h)_{\vert\nabla v_h(T)\vert},\omega_S}(\lambda_T^S)+ h_S\|\jump{F_h(\cdot,\nabla v_h)}_S\|_{2,S}^2\\&\quad+\|\omega_{p,\omega_S}(h_S)^2\,(1+\vert \nabla v_h\vert^{p_h(\cdot)s})\|_{1,\omega_S}\,. \end{aligned}
        \end{align}
        Combining \eqref{lem:efficiency.12.2.1}--\eqref{lem:efficiency.14}, 
		 for sufficiently small $\varepsilon>0$,  we conclude that \eqref{lem:efficiency.2} applies.\enlargethispage{5mm}
		\end{proof}

 
		
		\subsection{Patch-shift-to-element-shift inequality}

  \qquad The third tool is an estimate that enables to pass from element-patch-shifts to element-shifts. This is essential in the application of  quasi-interpolation operators that are only element-to-patch stable, e.g., the node-averaging quasi-interpolation operator $I_h^{\textit{av}}\colon\mathcal{S}^{1,\textit{cr}}_D(\mathcal{T}_h)\to \mathcal{S}^1_D(\mathcal{T}_h)$,~cf.~\eqref{eq:a6}.
		
		\begin{lemma}\label{lem:patch_to_element}
		 Let $p\in C^0(\overline{\Omega})$ with $p^->1$ and let $\delta\ge 0$.
		 Then, there exists some $s>1$, which can chosen to be close to $1$ if $h_T>0$ is close to $0$, such that for every $y_h\in L^{p_h(\cdot)}(\Omega;\mathbb{R}^d)$,~${v_h\in \mathcal{L}^1(\mathcal{T}_h)}$, and $T\in \mathcal{T}_h$, it holds
		 \begin{align*}
  \rho_{(\varphi_h)_{\vert \nabla_{\! h} v_h(T)\vert},\omega_T}(y_h)&\lesssim \rho_{(\varphi_h)_{\vert \nabla_{\! h} v_h\vert},\omega_T}(y_h)+\|\smash{h_{\mathcal{S}}^{1/2}}\jump{F_h(\cdot,\nabla_{\! h} v_h)}\|_{2,\mathcal{S}_h^{i}(T)}^2\\&\quad+\|\omega_{p,\omega_T}(h_T)^2\,(1+\vert \nabla_{\!h}v_h\vert^{p_h(\cdot)s})\|_{1,T}\,.
		 \end{align*}
		 where $\mathcal{S}_h^{i}(T)\coloneqq \mathcal{S}_h(T)\cap \mathcal{S}_h^{i}$ and $\mathcal{S}_h(T)\coloneqq \{S\in \mathcal{S}_h\mid S\cap T\neq \emptyset\}$.
		\end{lemma}
		
		\begin{proof}
        Applying for every $T'\in \mathcal{T}_h$ with $T'\subseteq \omega_T$, the shift change \eqref{lem:shift-change.1}, we arrive at
		 \begin{align}\label{lem:patch_to_element.1}
		  \smash{\rho_{(\varphi_h)_{\vert \nabla_{\! h} v_h(T)\vert},\omega_T}(y_h)\lesssim \rho_{(\varphi_h)_{\vert \nabla_{\! h} v_h\vert},\omega_T}(y_h)+\|F_h(\cdot,\nabla_{\! h} v_h(T))-F_h(\cdot,\nabla_{\! h} v_h)\|_{2,\omega_T}^2\,.}
		 \end{align}
		 Since each $T'\in \mathcal{T}_h$ with $T'\subseteq \omega_T$ can be reached by passing through a uniformly bounded~number (depending on $\omega_0 > 0$) of sides ${S\in \mathcal{S}_h^{i}(T)}$, for every $T' \in \mathcal{T}_h$ with $T' \subseteq \omega_T$,~using~\eqref{eq:Fh-F.1},~it~holds
     \begin{align}\label{lem:patch_to_element.2}
  \begin{aligned}
  &\vert F(q_{T'},\nabla_{\! h} v_h(T))-F(q_{T'},\nabla_{\! h} v_h(T'))\vert^2\\&\quad\lesssim \vert F(q_T,\nabla_{\! h} v_h(T))-F(q_{T'},\nabla_{\! h} v_h(T'))\vert^2+\vert F(q_{T'},\nabla_{\! h} v_h(T))-F(q_T,\nabla_{\! h} v_h(T))\vert^2 
 \\&\quad\lesssim \smash{\vert T\vert^{-1}\| \smash{h_{\mathcal{S}}^{1/2}}\jump{F_h(\cdot,\nabla_{\! h} v_h)}_S\|^2_{2,\mathcal{S}_h^{i}(T)}}
  +\vert T\vert^{-1}\|\omega_{p,\omega_T}(h_T)^2\,(1+\vert \nabla_{\!h}v_h\vert^{p_h(\cdot)s})\|_{1,T}\,.
  \end{aligned}
  \end{align}
  Eventually, multiplying \eqref{lem:patch_to_element.2} by $\vert T'\vert$ for all $T'\in \mathcal{T}_h$ with $T'\subseteq \omega_T$,
  due to $\vert T'\vert \sim \vert T\vert$, 
  we~arrive~at the claimed estimate.
	\end{proof}
	 
 \begin{proof}[Proof (of Theorem \ref{thm:best-approx})]
				Abbreviating $e_h\coloneqq v_h-u_h^{\textit{cr}}\in \smash{\mathcal{S}^{1,\textit{cr}}_D(\mathcal{T}_h)}$ and resorting~to~\eqref{eq:pDirichletW1p}~and~\eqref{eq:pDirichletS1crD} as well as that $f-f_h\perp \Pi_he_h$ in $L^2(\Omega)$, we arrive at
				\begin{align}
				\begin{aligned}
					( \AAA_h(\cdot,\nabla v_h)-\AAA_h(\cdot,\nabla_{\!h} u_h^{\textit{cr}}),\nabla_{\!h} e_h )_\Omega&=
					(\AAA_h(\cdot,\nabla v_h),\nabla_{\!h}( e_h- I_h^{\textit{av}} e_h) )_\Omega	\\&\quad+( f,I_h^{\textit{av}} e_h-e_h)_\Omega
					\\&\quad+( \AAA_h(\cdot,\nabla v_h)-\AAA_h(\cdot,\nabla u) ,\nabla I_h^{\textit{av}} e_h)_\Omega
					\\&\quad+(f-f_h,e_h-\Pi_he_h)_\Omega
			 \\&
    \eqqcolon I_h^1+I_h^2+I_h^3+I_h^4\,.
			 \end{aligned}\label{thm:best-approx.1}
				\end{align}
				
				\textit{ad $I_h^1$.} An element-wise integration-by-parts, 
             $\jump{\AAA_h(\cdot,\nabla v_h)\cdot n( e_h- I_h^{\textit{av}} e_h)}_S=\jump{\AAA_h(\cdot,\nabla v_h)\cdot n }_S $ $\{e_h-I_h^{\textit{av}} e_h\}_S +\{\AAA_h(\cdot,\nabla v_h)\cdot n\}_S$ 
             $\jump{e_h-I_h^{\textit{av}} e_h}_S$ on $S$, $\int_S{\jump{e_h-I_h^{\textit{av}} e_h}_S\,\textup{d}s}=0$ and $\{\AAA_h(\cdot,\nabla v_h)\cdot n\}_S=\textup{const}$ on $S$ for all ${S\in \mathcal{S}_h^{i}}$, the discrete trace inequality \cite[Lemma~12.8]{EG21}, and \eqref{eq:a6} with $\psi=\vert\cdot\vert$ and $a=0$ yield\vspace{-0.5mm}
				\begin{align}
					\begin{aligned}
					I_h^1&=\sum_{S\in \mathcal{S}_h^{i}}{\jump{\AAA_h(\cdot,\nabla v_h)\cdot n}_S\int_S{\{e_h- I_h^{\textit{av}} e_h\}_S\,\textup{d}s}}
      \\&
					\lesssim \sum_{S\in \mathcal{S}_h^{i}}{ \sum_{T\in \mathcal{T}_h;T\subseteq \omega_S}{\int_{\omega_T}{\vert \jump{\AAA_h(\cdot,\nabla v_h)\cdot n}_S\vert\vert \nabla_{\!h}e_h\vert\,\textup{d}x}}}\,.
				\end{aligned}	\label{thm:best-approx.2}
				\end{align}
 Next, let $T'\in \mathcal{T}_h\setminus\{T\}$ with $T'\subseteq \omega_S$. Then, resorting to the convexity of $(\varphi_{\vert\nabla v_h(T)\vert})^*(q_T,\cdot)$ and $\Delta_2((\varphi_{\vert\nabla v_h(T)\vert})^*(q_T,\cdot)) \lesssim 2^{\max\{2,(p^-)'\}}$, also using the shift change \eqref{lem:shift-change.3} and \eqref{eq:hammerh},~we~have~that
		 \begin{align}\label{thm:best-approx.3.0}
  \begin{aligned}
		 (\varphi_{\vert\nabla v_h(T)\vert})^*&(q_T,\vert\jump{\AAA_h(\cdot,\nabla v_h)\cdot n}_S\vert)
  \\&\lesssim (\varphi_{\vert\nabla v_h(T)\vert})^*(q_T,\vert \jump{\AAA(q_T,\nabla v_h)\cdot n_T}_S\vert)
  \\&\quad+(\varphi_{\vert\nabla v_h(T)\vert})^*(q_T,\vert (\AAA(q_T,\nabla v_h(T'))-\AAA(q_{T'},\nabla v_h(T')))\cdot n_{T}\vert)
  \\&\lesssim 
  (\varphi_{\vert\nabla v_h(T)\vert})^*(q_T,\vert \jump{\AAA(q_T,\nabla v_h)\cdot n_T}_S\vert)
  \\&\quad+\vert F(q_T,\nabla v_h(T'))-F(q_T,\nabla v_h(T))\vert^2
  \\&\quad+
  \vert F^*(q_T,\AAA(q_T,\nabla v_h(T')))-F^*(q_T,\AAA(q_{T'},\nabla v_h(T')))\vert^2 
  \\&\lesssim \vert\jump{F(q_T,\nabla v_h)}_S\vert^2+\omega_{p,\omega_S}(h_S)^2\, (1+\vert \nabla v_h(T')\vert^{p(q_{T'})s})
  \\& \lesssim\vert\jump{F_h(\cdot,\nabla v_h)}_S\vert^2+ \vert F(q_{T'},\nabla v_h(T'))-F(q_T,\nabla v_h(T'))\vert^2 \\&\quad + \omega_{p,\omega_S}(h_S)^2\, (1+\vert \nabla v_h(T')\vert^{p(q_{T'})s})
  \\&\lesssim\vert\jump{F_h(\cdot,\nabla v_h)}_S\vert^2+\omega_{p,\omega_S}^2\, (1+\vert \nabla v_h(T')\vert^{p(q_{T'})s})\,.
  \end{aligned}
		 \end{align}
				Applying for every $T'\in \mathcal{T}_h$ with $T'\subseteq \omega_T$, the $\varepsilon$-Young inequality~\eqref{ineq:young}~with~${\psi=\smash{\varphi_{\vert\nabla v_h(T)\vert}}(q_{T'},\cdot)}$, \eqref{thm:best-approx.3.0},
				and the finite overlapping of the element~patches~$\omega_T$,~${T\in \mathcal{T}_h}$, in \eqref{thm:best-approx.2}, for every $\varepsilon>0$, we conclude that
				\begin{align}\label{thm:best-approx.3}
				 \begin{aligned}
					 I_h^1&\lesssim c_{\varepsilon}\sum_{S\in \mathcal{S}_h^{i}}{\sum_{T\in \mathcal{T}_h;T\subseteq \omega_S}{\int_{\omega_T}{((\varphi_h)_{\vert \nabla v_h(T)\vert})^*(\cdot,\vert \jump{\AAA_h(\cdot,\nabla v_h)\cdot n}_S\vert)\,\textup{d}x}}}
  \\&\quad +\varepsilon\sum_{S\in \mathcal{S}_h^{i}}{\sum_{T\in \mathcal{T}_h;T\subseteq \omega_S}{\int_{\omega_T}{(\varphi_h)_{\vert\nabla v_h(T)\vert}(\cdot,\vert \nabla_{\!h} e_h\vert)\,\textup{d}x}}}
					 	\\
					 	&
					 	\lesssim  c_\varepsilon\,\big[\|\smash{h_{\mathcal{S}}^{1/2}}\jump{F_h(\cdot,\nabla v_h)}\|_{2,\mathcal{S}_h^{i}}^2 +\|\omega_p(h_{\mathcal{T}})^2\,(1+\vert\nabla_{\! h} v_h\vert^{p_h(\cdot)s})\|_{1,\Omega} \big]\\&\quad + \varepsilon\,  \sum_{T\in \mathcal{T}_h}{ \rho_{(\varphi_h)_{\vert\nabla v_h(T)\vert},\omega_T}( \nabla_{\!h} e_h)}
					 	\,.
					\end{aligned}
				\end{align}
  Appealing to Lemma \ref{lem:patch_to_element} with $y_h= \nabla_{\!h} e_h\in L^{p_h(\cdot)}(\Omega;\mathbb{R}^d)$, we have that
				\begin{align}\label{thm:best-approx.6}
 \begin{aligned}
				 \sum_{T\in \mathcal{T}_h}{\rho_{(\varphi_h)_{\vert\nabla v_h(T)\vert},\omega_T}( \nabla_{\!h} e_h)}&\lesssim \rho_{(\varphi_h)_{\vert\nabla v_h\vert},\Omega}(\nabla_{\!h} e_h)
     %\\[-2mm]&\quad
     +\|\smash{h_{\mathcal{S}}^{1/2}}\jump{F_h(\cdot,\nabla v_h)}\|_{2,\mathcal{S}_h^{i}}^2
 \\[-2mm]&\quad+\|\omega_p(h_{\mathcal{T}})^2\,(1+\vert \nabla v_h\vert^{p_h(\cdot)s})\|_{1,\Omega}\,.
					\end{aligned}
				\end{align}
				Thus, resorting in \eqref{thm:best-approx.3} to \eqref{lem:efficiency.2} and \eqref{eq:hammera}, we deduce that
				\begin{align}\label{thm:best-approx.3.1}
				  \begin{aligned}
					 I_h^1&
					 \lesssim  c_\varepsilon\,\big[\|F_h(\cdot,\nabla v_h)-F_h(\cdot,\nabla u)\|_{2,\Omega}^2\big]\\&\quad\quad+\|\omega_p(h_{\mathcal{T}})^2\,(1+\vert \nabla u\vert^{p(\cdot)s}+\vert \nabla v_h\vert^{p_h(\cdot)s})\|_{1,\Omega}+\mathrm{osc}_h(f,v_h) \big]
 \\&\quad+ \varepsilon\,  
 \|F_h(\cdot,\nabla v_h)-F_h(\cdot,\nabla_{\!h} u_h^{\textit{cr}})\|_{2,\Omega}^2
 \,.
					\end{aligned}
				\end{align}
				
				\textit{ad $I_h^2$.}
				 Applying for every $T\in \mathcal{T}_h$ the $\varepsilon$-Young inequality \eqref{ineq:young} with ${\psi=\varphi_{\vert\nabla v_h(T)\vert}(q_T,\cdot)}$, for every ${\varepsilon>0}$, we obtain
				\begin{align}
					\begin{aligned}
						I_h^2&\leq c_\varepsilon\,\rho_{((\varphi_h)_{\vert \nabla v_h\vert})^*,\Omega}(h_{\mathcal{T}}
  f)+
						\varepsilon\,\rho_{(\varphi_h)_{\vert \nabla v_h\vert},\Omega}\big(h_{\mathcal{T}}^{-1}( e_h-I_h^{\textit{av}}e_h)\big)\,.
					\end{aligned}\label{thm:best-approx.4}
				\end{align}
				Then, \hspace{-0.1mm}using \hspace{-0.1mm}for \hspace{-0.1mm}every \hspace{-0.1mm}$T\hspace{-0.1em}\in \hspace{-0.1em}\mathcal{T}_h$ \hspace{-0.1mm}the \hspace{-0.1mm}Orlicz-approximation \hspace{-0.1mm}property \hspace{-0.1mm}of \hspace{-0.1mm}$I_h^{\textit{av}}\colon\smash{\mathcal{S}^{1,\textit{cr}}_D(\mathcal{T}_h)}\hspace{-0.1em}\to \hspace{-0.1em}\mathcal{S}^1_D(\mathcal{T}_h)$,~\hspace{-0.1mm}cf.~\hspace{-0.1mm}\eqref{eq:a6}, with $\psi=\varphi_h(q_T,\cdot)$ and $a=\vert \nabla v_h(T)\vert$, 
    we find that
				\begin{align}\label{thm:best-approx.5}
                    \begin{aligned}
				 \rho_{(\varphi_h)_{\vert \nabla v_h\vert},\Omega}\big(h_{\mathcal{T}}^{-1}( e_h-I_h^{\textit{av}}e_h)\big)&\lesssim\,\sum_{T\in \mathcal{T}_h}{\rho_{(\varphi_h)_{\vert\nabla v_h(T)\vert},\omega_T}( \nabla_{\!h} e_h)}\\&\quad+\|\omega_p(h_{\mathcal{T}})^2\,(1+\vert \nabla v_h\vert^{p_h(\cdot)s}+\vert\nabla_{\! h} e_h\vert^{p_h(\cdot)s})\|_{1,\Omega}\,.
                    \end{aligned}
		  		\end{align}
				Thus, using \eqref{thm:best-approx.5} and \eqref{thm:best-approx.6} in conjunction with \eqref{lem:efficiency.1} and \eqref{lem:efficiency.2} in \eqref{thm:best-approx.4}, we arrive at
				\begin{align}\label{thm:best-approx.6.1}
				 \begin{aligned}
				 	I_h^2&
					 	\lesssim  c_\varepsilon\,\big[\|F_h(\cdot,\nabla v_h)-F_h(\cdot,\nabla u)\|_{2,\Omega}^2 \\&\quad \quad+\|\omega_p(h_{\mathcal{T}})^2\,(1+\vert \nabla u\vert^{p(\cdot)s}+\vert \nabla v_h\vert^{p_h(\cdot)s}+\vert \nabla_{\!h } e_h\vert^{p_h(\cdot)s})\|_{1,\Omega}+\mathrm{osc}_h(f,v_h)\big]
					 	\\&\quad+ \varepsilon\,  
					 	\|F_h(\cdot,\nabla v_h)-F_h(\cdot,\nabla_{\!h} u_h^{\textit{cr}})\|_{2,\Omega}^2
					 	\,.
					 \end{aligned}
				\end{align}
				
				\textit{ad $I_h^3$.}
				Applying for every $T\in \mathcal{T}_h$ the $\varepsilon$-Young inequality \eqref{ineq:young} with $\psi=\varphi_{\vert\nabla v_h(T)\vert }(q_T,\cdot)$ together with \eqref{eq:hammera}, for~every~${\varepsilon>0}$, we obtain
				\begin{align}
					\begin{aligned}
					I_h^3&\leq c_\varepsilon\,\|F_h(\cdot,\nabla v_h)-F_h(\cdot,\nabla u)\|_{2,\Omega}^2+\varepsilon\,\rho_{(\varphi_h)_{\vert\nabla v_h\vert },\Omega}(\nabla I_h^{\textit{av}}e_h )\,.
				\end{aligned}\label{thm:best-approx.7}
				\end{align}
				Then, using for every $T\in \mathcal{T}_h$ the Orlicz-stability property of $I_h^{\textit{av}}\colon\smash{\mathcal{S}^{1,\textit{cr}}_D(\mathcal{T}_h)}\to \mathcal{S}^1_D(\mathcal{T}_h)$, cf.\ \eqref{eq:a6}, with $\psi=\varphi(q_T,\cdot)$ and $a=\vert \nabla v_h(T)\vert$,
    we~find~that
				\begin{align}\label{thm:best-approx.8}
                    \begin{aligned}
				        \rho_{(\varphi_h)_{\vert\nabla v_h\vert },\Omega}( \nabla I_h^{\textit{av}}e_h)&\lesssim\sum_{T\in \mathcal{T}_h}{\rho_{(\varphi_h)_{\vert\nabla v_h(T)\vert},\omega_T}( \nabla_{\!h} e_h)}\\&\quad +\|\omega_p(h_{\mathcal{T}})^2\,(1+\vert \nabla v_h\vert^{p_h(\cdot)s}+\vert \nabla_{\! h} e_h\vert^{p_h(\cdot)s})\|_{1,\Omega}\,.
                    \end{aligned}
				\end{align}
				Thus, using \eqref{thm:best-approx.8} and \eqref{thm:best-approx.6} in conjunction with \eqref{lem:efficiency.2} in \eqref{thm:best-approx.7}, we arrive at
				\begin{align}\label{thm:best-approx.8.1}
				 \begin{aligned}
				 	I_h^3&
					 	\lesssim c_\varepsilon\,\big[\|F_h(\cdot,\nabla v_h)-F_h(\cdot,\nabla u)\|_{2,\Omega}^2 \\&\quad \quad+\|\omega_p(h_{\mathcal{T}})^2\,(1+\vert \nabla v_h\vert^{p_h(\cdot)s}+\vert \nabla_{\! h} e_h\vert^{p_h(\cdot)s})\|_{1,\Omega}+\mathrm{osc}_h(f,v_h)\big]\\&\quad+\varepsilon\,  
					 	\|F_h(\cdot,\nabla v_h)-F_h(\cdot,\nabla_{\!h} u_h^{\textit{cr}})\|_{2,\Omega}^2
					 	\,.
					 \end{aligned}
				\end{align}
				
				\textit{ad $I_h^4$.}
				Applying for every $T\in \mathcal{T}_h$ the $\varepsilon$-Young inequality \eqref{ineq:young} with $\psi=\smash{\varphi_{\vert\nabla v_h\vert}}$,~for~every~${\varepsilon>0}$, we obtain
				\begin{align}
					\begin{aligned}
					I_h^4&\leq c_\varepsilon\,\mathrm{osc}_h(f,v_h) +\varepsilon\,\rho_{\varphi_{\vert\nabla v_h\vert },\Omega}\big(h_{\mathcal{T}}^{-1} (e_h- I_h^{\textit{av}}e_h) \big)\,.
				\end{aligned}	\label{thm:best-approx.9}
				\end{align}
				Thus, using \eqref{thm:best-approx.5} and \eqref{thm:best-approx.6} in conjunction with \eqref{lem:efficiency.2} in \eqref{thm:best-approx.9}, we arrive at
				\begin{align}\label{thm:best-approx.9.1}
				 \begin{aligned}
				 	I_h^4&
					 	\lesssim c_\varepsilon\,\big[\|F_h(\cdot,\nabla v_h)-F_h(\cdot,\nabla u)\|_{2,\Omega}^2 \\&\quad \quad+\|\omega_p(h_{\mathcal{T}})^2\,(1+\vert \nabla v_h\vert^{p_h(\cdot)s}+\vert \nabla_{\! h} e_h\vert^{p_h(\cdot)s})\|_{1,\Omega}+\mathrm{osc}_h(f,v_h)\big]\\&\quad + \varepsilon\, 
					 	\|F_h(\cdot,\nabla v_h)-F_h(\cdot,\nabla_{\!h} u_h^{\textit{cr}})\|_{2,\Omega}^2
					 	\,.
					 \end{aligned}
				\end{align}
				Then, combining \eqref{eq:hammera}, \eqref{thm:best-approx.3.1}, \eqref{thm:best-approx.6.1}, \eqref{thm:best-approx.8.1}, and \eqref{thm:best-approx.9.1} 
				in \eqref{thm:best-approx.1}, for every $\varepsilon>0$, we arrive at
				\begin{align*}%\label{thm:best-approx.11}
					%\hspace{-3mm}\begin{aligned}
				  \|F_h(\cdot,\nabla v_h)-F_h(\cdot,\nabla_{\!h}u_h^{\textit{cr}})\|_{2,\Omega}^2%\big( \AAA_h(\cdot,\nabla v_h)-\AAA_h(\cdot,\nabla_{\!h} u_h^{\textit{cr}}) ,\nabla_{\!h}e_h \big)_\Omega
      &
					 	\lesssim  c_\varepsilon\,\big[\|F_h(\cdot,\nabla v_h)-F_h(\cdot,\nabla u)\|_{2,\Omega}^2+\mathrm{osc}_h(f,v_h) \\&\quad \quad+\|\omega_p(h_{\mathcal{T}})^2\,(1+\vert \nabla u\vert^{p(\cdot)s}+(\vert \nabla v_h\vert+\vert \nabla_{\! h} e_h\vert)^{p_h(\cdot)s})\|_{1,\Omega}%\\&\quad\quad+\mathrm{osc}_h(f,v_h)
                        \big]
					 	\\&\quad+ \varepsilon\,  
					 	\|F_h(\cdot,\nabla v_h)-F_h(\cdot,\nabla_{\!h} u_h^{\textit{cr}})\|_{2,\Omega}^2
					 	\,.
			 %\end{aligned}\hspace{-1mm}
				\end{align*}
	%			Resorting in \eqref{thm:best-approx.11} to \eqref{eq:hammera}, f
    Next, choosing $\varepsilon>0$ sufficiently small, for every $v_h\in \smash{\mathcal{S}^{1,\textit{cr}}_D(\mathcal{T}_h)}$, we obtain
				\begin{align}\label{thm:best-approx.12}
					\begin{aligned}
						\|F_h(\cdot,\nabla v_h)-F_h(\cdot,\nabla_{\!h}u_h^{\textit{cr}})\|_{2,\Omega}^2&
					 	\lesssim  \|F_h(\cdot,\nabla v_h)-F_h(\cdot,\nabla u)\|_{2,\Omega}^2+\mathrm{osc}_h(f,v_h) \\&\quad +\|\omega_p(h_{\mathcal{T}})^2\,(1+\vert \nabla u\vert^{p(\cdot)s}+(\vert \nabla v_h\vert+\vert \nabla_{\! h} e_h\vert)^{p_h(\cdot)s})\|_{1,\Omega}%\\&\quad
       \,.
					\end{aligned}\hspace{-2mm}
				\end{align}	
				From \eqref{thm:best-approx.12}, in turn, we deduce that\enlargethispage{2mm}
				\begin{align}\label{thm:best-approx.13}
				 \begin{aligned}
				 \|F_h(\cdot,\nabla_{\!h}u_h^{\textit{cr}})-F_h(\cdot,\nabla u)\|_{2,\Omega}^2&\lesssim \|F_h(\cdot,\nabla v_h)-F_h(\cdot,\nabla_{\!h}u_h^{\textit{cr}})\|_{2,\Omega}^2\\&\quad+
				  \|F_h(\cdot,\nabla v_h)-F_h(\cdot,\nabla u)\|_{2,\Omega}^2\\&
				 \lesssim \|F_h(\cdot,\nabla v_h)-F_h(\cdot,\nabla u)\|_{2,\Omega}^2 +\mathrm{osc}_h(f,v_h)\\&\quad +\|\omega_p(h_{\mathcal{T}})^2\,(1+\vert \nabla u\vert^{p(\cdot)s}+(\vert \nabla v_h\vert+\vert \nabla_{\! h} e_h\vert)^{p_h(\cdot)s})\|_{1,\Omega} \,.
				 \end{aligned}\hspace{-5mm}
				\end{align}
				Taking in \eqref{thm:best-approx.13} the infimum with respect to $v_h\in \smash{\mathcal{S}^1_D(\mathcal{T}_h)}$, we conclude~the~claimed~estimate.
			\end{proof}
		
 \section{A priori error analysis}\label{sec:a_priori}
	
		\qquad In this section, we establish a priori error estimates for the $S^{1,\textit{\textrm{cr}}}_D(\mathcal{T}_h)$-approximation \eqref{eq:pDirichletS1crD} of the $p(\cdot)$-Dirichlet problem \eqref{eq:pDirichletW1p}. To this end, we resort to the medius error analysis of Section~\ref{sec:medius}, which allows us to tranfer the approximation rate capabilities of the $S^1_D(\mathcal{T}_h)$-approximation \eqref{eq:pDirichletS1D} (cf.\ Theorem \ref{P1_apriori}) to the $S^{1,\textit{\textrm{cr}}}_D(\mathcal{T}_h)$-approximation \eqref{eq:pDirichletS1crD}.
 
		\begin{theorem}\label{thm:rate_u}
		  Let $p\in C^{0,\alpha}(\overline{\Omega})$ with $\alpha\in (0,1]$ and $p^->1$ and let $\delta\ge 0$.~Moreover,~let~$F(\cdot,\nabla u)\in W^{1,2}(\Omega;\mathbb{R}^d)$, $(\delta^{p(\cdot)-1}+\vert z\vert )^{p'(\cdot)-2}\vert f\vert^2\in L^1(\{p> 2\})$, and 
 $ f\in L^{(p^-)'}(\Omega)$ or $\Gamma_D=\partial \Omega$.
		 Then, assuming that $h_{\max}\sim h_{\mathcal{T}}$, there exists some $s\hspace{-0.1em}>\hspace{-0.1em}1$, which can chosen to be close to $1$ if $h_{\max}>0$ is close to $0$, such that if~$f\hspace{-0.1em}\in\hspace{-0.1em} L^{p'(\cdot)s}(\Omega)$ and $2\alpha s +d(1-s^2)\ge 0$, then
		 \begin{align*}
		 \|F_h(\cdot,\nabla_{\!h} u_h^{\textit{cr}})-F_h(\cdot,\nabla u)\|_{2,\Omega}^2&\lesssim h_{\max}^{2\alpha}\,\big(1+\|\nabla F(\cdot,\nabla u)\|_{2,\Omega}^{2}+\|(\delta^{p(\cdot)-1}+\vert z\vert)^{p'(\cdot)-2}\vert f\vert^2\|_{1,\{p>2\}}\\&\quad+
  \sigma(f,s)+\rho_{p'(\cdot)s,\Omega}(f)+\rho_{p(\cdot)s,\Omega}( \nabla u)\big)^s\,,
		 \end{align*}
        where the hidden constants also depend on $s>1$ and $\sigma(f,s)\coloneqq 1+\rho_{(p^-)',\Omega}(f)$ if $f\in L^{(p^-)'}(\Omega)$ and $\sigma(f,s)\coloneqq 1+\rho_{p'(\cdot)s,\Omega}(f)^{\smash{(p^-)'/(p^+)'}}$ if $\Gamma_D=\partial \Omega$.
		\end{theorem}

 \begin{remark}[Comments on the regularity assumptions in Theorem \ref{thm:rate_u}]\hphantom{                   }
 \begin{itemize}[noitemsep,topsep=2pt,labelwidth=\widthof{(iii)},leftmargin=!]
    \item[(i)] If $p\in C^{0,\alpha}(\overline{\Omega})$ with $\alpha\in (0,1)$, one cannot expect that $F(\cdot,\nabla u)\in W^{1,2}(\Omega;\mathbb{R}^d)$, even locally.
    However, appealing to \cite[Remark 4.5]{BDS13}, one can expect that $F(\cdot,\nabla u)\in \mathcal{N}^{\alpha,2}(\Omega)$, where $\mathcal{N}^{\alpha,2}(\Omega)$ is the Nikolski\u{\i} space with order of differentiability $\alpha\in (0,1)$, which should still be enough the justify the arguments below, but is beyond the scope of this article.
  \item[(ii)] If $p\leq 2$ in $\overline{\Omega}$ in Theorem \ref{thm:rate_u}, then the assumption  $(\delta^{p(\cdot)-1}+\vert z\vert )^{p'(\cdot)-2}\vert f\vert^2\in L^1(\{p> 2\})$ is trivially satisfied.

  \item[(iii)]  
  If $(\delta^{p(\cdot)-1}+\vert z\vert )^{p'(\cdot)-2}\vert \nabla z\vert^2\hspace{-0.1em}\in L^1(\{p\hspace{-0.1em}>\hspace{-0.1em} 2\})$ in Theorem \ref{thm:rate_u}, then, due to $f\hspace{-0.1em}=\hspace{-0.1em}-\textup{div}\,z$~in~$L^{p'(\cdot)}(\Omega)$, it holds  $(\delta^{p(\cdot)-1}+\vert z\vert )^{p'(\cdot)-2}\vert f\vert^2\in L^1(\{p> 2\})$.

  \item[(iv)] If $f\in L^2(\{p> 2\})$ and $\delta>0$, then  $(\delta^{p(\cdot)-1}+\vert z\vert )^{p'(\cdot)-2}\vert f\vert^2\leq \delta^{2-p(\cdot)}\vert f\vert^2$ a.e.\ in $\{p> 2\}$, i.e., it holds $(\delta^{p(\cdot)-1}+\vert z\vert )^{p'(\cdot)-2}\vert f\vert^2\in L^1(\{p> 2\})$.

  \item[(v)] If $p\in (1,\infty)$, cf. \cite[Lemma 2.3]{DKRI14}, then it holds $F(\nabla u)\in W^{1,2}(\Omega;\mathbb{R}^d)$ if and only if $F^*(z)\in W^{1,2}(\Omega;\mathbb{R}^d)$ with $\vert F(\nabla u)\vert \sim \vert F^*(z)\vert$ a.e.\ in $\Omega$ and $\vert \nabla F(\nabla u)\vert \sim  \vert \nabla F^*(z)\vert$~a.e.\ in~$\Omega$.~In~addition, appealing to \cite[Lemma 2.11]{K22CR}, it holds $\vert \nabla F^*(z)\vert\sim (\delta^{p-1}+\vert z\vert )^{(p'-2)/2}\vert \nabla z\vert$~a.e.~in~$\Omega$.
  Thus, Theorem \ref{thm:rate_u} extends the a priori error analysis in \cite{K22CR}.
 \end{itemize}
 
 \end{remark}

 Since the right-hand side in Theorem \ref{thm:best-approx} still involves the discrete primal solution, before proving Theorem \ref{thm:rate_u}, we first derive the following a priori estimate.%\enlargethispage{5mm}

 \begin{lemma}\label{lem:a_priori}
     Let the assumptions of Theorem  \ref{thm:rate_u} be satisfied. Then, there exists  some $s>1$, which can chosen to be close to $1$ if $h_{\max}>0$ is close to $0$, such that if $f\in L^{p'(\cdot)s}(\Omega)$, then 
     \begin{align*}
         \rho_{\varphi_h,\Omega}(\nabla_{\! h}u_h^{\textit{\textrm{cr}}})\lesssim  \sigma(f,s)\,,
     \end{align*}
      where the hidden constants also depend on $s>1$ and $\sigma(f,s)$ is defined as in Theorem  \ref{thm:rate_u}.
 \end{lemma}

\begin{proof}
    We distinguish the cases $f\in L^{(p^-)'}(\Omega)$ and $\Gamma_D=\partial \Omega$:

    \textit{Case $f\hspace{-0.15em}\in\hspace{-0.15em} L^{(p^-)'}(\Omega)$.}
    Since $I_h^{\textit{\textrm{cr}}}(u_h^{\textit{\textrm{cr}}})\hspace{-0.15em}\leq\hspace{-0.15em} I_h^{\textit{\textrm{cr}}}(0)$, applying  the $\varepsilon$-Young inequality \eqref{ineq:young}~with~${\psi\hspace{-0.15em}=\hspace{-0.15em}\vert\hspace{-0.15em} \cdot\hspace{-0.15em}\vert^{p^-}}\!\!$,  the 
    $L^{p^-}$-stability of $\Pi_h$, the discrete Poincar\'e inequality in $\mathcal{S}^{1,\textit{\textrm{cr}}}_D(\mathcal{T}_h)$, cf. \cite[Proposition I.4.13]{Tem77}, and $p^-\lesssim p_h$ a.e.\ in $\Omega$,
    we find that
    \begin{align*}
        \rho_{\varphi_h,\Omega}(\nabla_{\! h}u_h^{\textit{\textrm{cr}}})&\lesssim (f_h,\Pi_h  u_h^{\textit{\textrm{cr}}})_{\Omega}
        %\\&\lesssim c_\varepsilon\,\rho_{(p^-)',\Omega}(f_h)+\varepsilon\, \rho_{p^-,\Omega}( u_h^{\textit{\textrm{cr}}})
        \\&\lesssim c_\varepsilon\,\rho_{(p^-)',\Omega}(f_h)+\varepsilon\,
        \rho_{p^-,\Omega}(\nabla_{\! h} u_h^{\textit{\textrm{cr}}})
        \\&\lesssim c_\varepsilon\,\rho_{(p^-)',\Omega}(f)+\varepsilon\,
        \rho_{p_h(\cdot),\Omega}(\nabla_{\! h} u_h^{\textit{\textrm{cr}}})
        \,.
    \end{align*}
    For $\varepsilon>0$ sufficiently small, using $p_h\lesssim 1+\varphi_h$, we conclude that $\rho_{\varphi_h,\Omega}(\nabla_{\! h}u_h^{\textit{\textrm{cr}}})\lesssim 1+\rho_{(p^-)',\Omega}(f)$. 

    \textit{Case $\Gamma_D=\partial\Omega$.} Let $R>0$ be such that $\Omega\subseteq B_R^d(0)$. Then, if $f\in L^{p'(\cdot)s}(\Omega)$,
    appealing to \cite[Theorem 14.1.2]{DHHR11},  the vector field $G \coloneqq (-\nabla u)|_{\Omega} $, where $u\in W^{2,p(\cdot)}(B_R^d(0))\cap W^{1,p(\cdot)}_0(B_R^d(0))$ is the unique solution of  $-\Delta u =\overline{f}$ a.e.\ in $B_R^d(0)$, where $\overline{f}\coloneqq f$ a.e.\ in $\Omega$ and $\overline{f}\coloneqq 0$ a.e.\ in $B_R^d(0)\setminus\Omega$, satisfies 
     $G\in W^{1,p'(\cdot)s}(\Omega)^d$, $\textup{div}\,G=f$ in $\mathcal{L}^0(\mathcal{T}_h)$,  and 
    $\|G\|_{p'(\cdot)s,\Omega}+\|\nabla G\|_{p'(\cdot)s,\Omega}\lesssim \|f\|_{p'(\cdot)s,\Omega}$, with a constant depending~on~$s$,~$p$, and $\Omega$. Using \cite[Lemma 3.2.5]{DHHR11}, we find that
    \begin{align}
        \rho_{p'(\cdot)s,\Omega}(G)+\rho_{p'(\cdot)s,\Omega}(\nabla G)\lesssim \rho_{p'(\cdot)s,\Omega}(f)^{\smash{(p^-)'/(p^+)'}}\,.\label{lem:a_priori.1}
    \end{align}    
    Denote by $I_h^{\textit{\textrm{rt}}}\colon W^{1,1}(\Omega)^d\to \mathcal{R}T^0(\mathcal{T}_h)$ the Raviart--Thomas quasi-iterpolation operator. Then, we have that $\textup{div}\,I_h^{\textit{\textrm{rt}}} G=\Pi_h \textup{div}\, G=f_h$ a.e.\ in $\Omega$. Appealing to  \cite[Theorem 16.4]{EG21},~for~every~$T\in \mathcal{T}_h$, we have that
    \begin{align}
        \rho_{p'(q_T),T}(I_h^{\textit{\textrm{rt}}}G)\lesssim \rho_{p'(q_T),T}(G)+\rho_{p'(q_T),T}(\nabla G)\,.\label{lem:a_priori.2}
    \end{align}
    Therefore, since $I_h^{\textit{\textrm{cr}}}(u_h^{\textit{\textrm{cr}}})\leq I_h^{\textit{\textrm{cr}}}(0)$, using $\textup{div}\,I_h^{\textit{\textrm{rt}}} G=f_h$  in $\mathcal{L}^0(\mathcal{T}_h)$, the discrete integration-by-parts formula \eqref{eq:pi0}, for every $T\in \mathcal{T}_h$ the $\varepsilon$-Young inequality \eqref{ineq:young} with $\psi=\vert \cdot\vert^{p(q_T)}$, \eqref{lem:a_priori.2},~and~\eqref{lem:a_priori.1},
    we find that
    \begin{align*}
        \rho_{\varphi_h,\Omega}(\nabla_{\! h}u_h^{\textit{\textrm{cr}}})&\lesssim (f_h,\Pi_h  u_h^{\textit{\textrm{cr}}})_{\Omega}
        =(\textup{div}\,I_h^{\textit{\textrm{rt}}} G,\Pi_h  u_h^{\textit{\textrm{cr}}})_{\Omega}
        =-(\Pi_hI_h^{\textit{\textrm{rt}}} G,\nabla_{\! h}  u_h^{\textit{\textrm{cr}}})_{\Omega}
        \\&\lesssim c_\varepsilon\,\rho_{p_h'(\cdot),\Omega}(I_h^{\textit{\textrm{rt}}} G)+\varepsilon\,
        \rho_{p_h(\cdot),\Omega}(\nabla_{\! h} u_h^{\textit{\textrm{cr}}})\,,
        \\&\lesssim c_\varepsilon\,\big(\rho_{p_h'(\cdot)}(G)+\rho_{p_h'(\cdot),\Omega}(\nabla G)\big)+\varepsilon\,
        \rho_{p_h(\cdot),\Omega}(\nabla_{\! h} u_h^{\textit{\textrm{cr}}})
        \\&\lesssim c_\varepsilon\,\big(1+\rho_{p'(\cdot)s,\Omega}(G)+\rho_{p'(\cdot)s,\Omega}(\nabla G)\big)+\varepsilon\,
        \rho_{p_h(\cdot),\Omega}(\nabla_{\! h} u_h^{\textit{\textrm{cr}}})
        \\&\lesssim c_\varepsilon\,\big(1+\rho_{p'(\cdot)s,\Omega}(f)^{\smash{(p^-)'/(p^+)'}}\big)+\varepsilon\,
        \rho_{p_h(\cdot),\Omega}(\nabla_{\! h} u_h^{\textit{\textrm{cr}}})\,.
    \end{align*}
    For $\varepsilon>0$ sufficiently small, we conclude that $ \rho_{\varphi_h}(\nabla_{\! h}u_h^{\textit{\textrm{cr}}})\lesssim 1+\rho_{p'(\cdot)s}(f)^{\smash{(p^-)'/(p^+)'}}$.
    %\begin{align*}
    %    \rho_{p_h(\cdot)}(\nabla_{\! h}u_h^{\textit{\textrm{cr}}})\lesssim 1+\rho_{p'(\cdot)s}(f)^{\smash{(sp^-)'/(sp^+)'}}\,.
    %\end{align*}
\end{proof}
		
		\begin{proof}[Proof (of Theorem \ref{thm:rate_u}).]
		  The convexity of $(\varphi_{\vert \nabla_{\!h} v_h(T)\vert})^*(q_T,\cdot)$ and that $\Delta_2((\varphi_{\vert \nabla_{\!h} v_h(T)\vert})^*(q_T,\cdot))\lesssim 2^{\smash{\max\{2,(p^-)'\}}}$ for all $T\in \mathcal{T}_h$, the Orlicz-stability of $\Pi_h$ (cf.\ \cite[Corollary~A.8,~(A.12)]{kr-phi-ldg}), and the shift change \eqref{lem:shift-change.3}, for every $v_h\in\mathcal{S}^1_D(\mathcal{T}_h)$, yield 
		 \begin{align}\label{cor:rate_improve.1}
		  \begin{aligned}
		  \textrm{osc}_h(f,v_h)
	%	 & \lesssim \rho_{((\varphi_h)_{\vert \nabla v_h\vert})^*,\Omega}(h_{\mathcal{T}} f )
	%	  \\&
    \lesssim \rho_{((\varphi_h)_{\vert \nabla u\vert})^*,\Omega}(h_{\mathcal{T}}f )+\|F_h(\cdot,\nabla v_h)-F_h(\cdot,\nabla u)\|_{2,\Omega}^2\,.
		  \end{aligned}
		 \end{align}
 		 Using \eqref{cor:rate_improve.1} in Theorem \ref{thm:best-approx}, for every $v_h\in\mathcal{S}^1_D(\mathcal{T}_h)$, we find that
  \begin{align}
 \begin{aligned}
		  \|F_h(\cdot,\nabla_{\!h} u_h^{\textit{cr}})-F_h(\cdot,\nabla u)\|_{2,\Omega}^2&\lesssim \|F_h(\cdot,\nabla v_h)-F_h(\cdot,\nabla u)\|_{2,\Omega}^2\\&\quad+h_{\max}^{2\alpha}\,\rho_{p_h(\cdot)s,\Omega}(\nabla v_h)+h_{\max}^{2\alpha}\,\rho_{p_h(\cdot)s,\Omega}(\nabla_{\! h} u_h^{\textit{cr}})
  \\&\quad+h_{\max}^{2\alpha}\,\big(1+\rho_{p(\cdot)s,\Omega}(\nabla u)\big)+\rho_{((\varphi_h)_{\vert \nabla u\vert})^*,\Omega}(h_{\mathcal{T}}f ) \,.
 \end{aligned}\label{cor:rate_improve.2}
		 \end{align}
  Using  that, appealing to \cite[Lemma 4.3 \& Corollary 3.6]{BDS13},  it holds
  \begin{align}\label{cor:rate_improve.2.0}
 \begin{aligned}
 \|F_h(\cdot,\nabla I_h^{\textit{sz}}u)-F_h(\cdot,\nabla u)\|_{2,\Omega}^2&\lesssim h_{\max}^2\|\nabla F(\cdot,\nabla u)\|_{2,\Omega}^2+h_{\max}^{2\alpha}\,\big(1+\rho_{p(\cdot)s,\Omega}(\nabla u)\big)\,,\\
 \rho_{p_h(\cdot)s,\Omega}(\nabla I_h^{\textit{sz}}u)&\lesssim 1+\rho_{p(\cdot)s,\Omega}(\nabla I_h^{\textit{sz}}u)\lesssim 1+\rho_{p(\cdot)s,\Omega}(\nabla u)\,,
 \end{aligned}
  \end{align}
  choosing $v_h=I_h^{\textit{sz}}u\in \mathcal{S}^1_D(\mathcal{T}_h)$ in \eqref{cor:rate_improve.2}, we arrive at
  \begin{align}\label{cor:rate_improve.3}
  \begin{aligned}
 \|F_h(\cdot,\nabla_{\!h} u_h^{\textit{cr}})-F_h(\cdot,\nabla u)\|_{2,\Omega}^2&\lesssim h_{\max}^2\|\nabla F(\cdot,\nabla u)\|_{2,\Omega}^2+ h_{\max}^{2\alpha}\,\big(1+\rho_{p(\cdot)s,\Omega}(\nabla u)\big)\\&\quad +h_{\max}^{2\alpha}\,\rho_{p_h(\cdot)s,\Omega}(\nabla_{\! h} u_h^{\textit{cr}})+\rho_{((\varphi_h)_{\vert \nabla u\vert})^*,\Omega}(h_{\mathcal{T}}f )\,.
 \end{aligned}
  \end{align}
  By the aid of Lemma \ref{lem:A-Ah} \eqref{eq:phih-phi}, also using that $2\alpha\leq {2\wedge (p^+)'}+\alpha$, it holds
  \begin{align}\label{cor:rate_improve.3.2}
 \begin{aligned}
 \rho_{((\varphi_h)_{\vert \nabla u\vert})^*,\Omega}(h_{\mathcal{T}} f )%&\lesssim \rho_{(\varphi_{\vert \nabla u\vert})^*,\Omega}(h_{\mathcal{T}} f )+ h_{\max}^{{2\wedge (p^-)'}+\alpha}\,\big(1+\rho_{p'(\cdot)s,\Omega}(f)+\rho_{p(\cdot)s,\Omega}(\nabla u)\big)
 %\\&
 \lesssim \rho_{(\varphi_{\vert \nabla u\vert})^*,\Omega}(h_{\mathcal{T}} f )+h_{\max}^{2\alpha}\,\big(1+\rho_{p'(\cdot)s,\Omega}(f)+\rho_{p(\cdot)s,\Omega}(\nabla u)\big)\,.
 \end{aligned}
  \end{align}
  Next, for every $T\in \mathcal{T}_h$ and $x\in T$, we need to distinguish the cases $p(x)\in (1,2)$~and~$p(x)\in [2,\infty)$:

  \textit{Case $p(x)\in (1,2]$.} If $p(x)\in (1,2]$, then, there holds the elementary inequality
		 \begin{align*}
		 (\varphi_{\vert a\vert })^*(x,h\,t)\lesssim \lambda^2\,\big(\varphi^*(x,t)+\varphi(x,\vert a\vert)\big)\quad\textup{ for all }a\in \mathbb{R}^d\,,\; t\ge 0\,,\; \lambda\in [0,1]\,,
		  \end{align*}
		  which follows from the definition of shifted \mbox{$N$-functions}, cf.\ \eqref{eq:phi_shifted}, and the shift change~\eqref{lem:shift-change.3}~(i.e., with $b = 0$ and using that $\vert F(x,a)\vert^2=\varphi(x,\vert a\vert )$ for all $a\in \mathbb{R}^d$), so that
  \begin{align}\label{cor:rate_improve.4}
  \begin{aligned} 
 (\varphi_{\vert \nabla u(x)\vert })^*(x,h_T\vert f(x)\vert )&\lesssim h^2_T\,\big(\varphi^*(x,\vert f(x)\vert )+\varphi(x,\vert \nabla u(x)\vert)\big)\,.
 \end{aligned}
  \end{align}

  \textit{Case $p(x)\in (2,\infty)$.} 
  Since  $(\varphi_{\vert a\vert})^*(x,\lambda\,t)\lesssim \lambda^2\,(\delta^{p(x)-1}+\vert a\vert)^{p'(x)-2}t^2$ for all $a\in \mathbb{R}^d$~and~${t,\lambda\ge 0}$, we have that\vspace{-1mm}
  \begin{align}\label{cor:rate_improve.5}
		 (\varphi_{\smash{\vert \nabla u\vert}})^*(x,h_T \vert f(x)\vert )
 &\lesssim h_T^2\,(\delta^{p(x)-1}+\vert z(x)\vert)^{p'(x)-2} \vert f(x)\vert^2 \,.
		 \end{align}
		  Combining \eqref{cor:rate_improve.4} and \eqref{cor:rate_improve.5}, we deduce that
 \begin{align}\label{cor:rate_improve.8}
  \begin{aligned}
 \rho_{(\varphi_{\smash{\vert \nabla u\vert}})^*,\Omega}(h_{\mathcal{T}}f)&\lesssim h_{\max}^2\,\big(
 \rho_{\varphi^*,\Omega}(f)+\rho_{\varphi,\Omega}(\nabla u)+\|(\delta^{p(\cdot)-1}+\vert z\vert)^{p'(\cdot)-2} \vert f\vert^2\|_{1,\{p>2\}}\big)\,.
 \end{aligned}
 \end{align}
 Using \eqref{cor:rate_improve.8} in \eqref{cor:rate_improve.3.2}, we find that
 \begin{align}\label{cor:rate_improve.9}
 \begin{aligned}
 \rho_{((\varphi_h)_{\vert \nabla u\vert})^*,\Omega}(h_{\mathcal{T}} f )&\lesssim h_{\max}^{2\alpha}\,\big(1+\rho_{p'(\cdot)s,\Omega}(f)+\rho_{p(\cdot)s,\Omega}(\nabla u)\big)\\&\quad + h_{\max}^2\,
 \|(\delta^{p(\cdot)-1}+\vert z\vert)^{p'(\cdot)-2} \vert f\vert^2\|_{1,\{p>2\}}\,.
 \end{aligned}
 \end{align}
 In addition, due to Lemma \ref{lem:a_priori}, \cite[Lemma 12.1]{EG21}, $h_{\max}\sim h_{\mathcal{T}}$, and $p_h\lesssim 1+\varphi_h$, we have that
 \begin{align}\label{cor:rate_improve.10}
    \begin{aligned}
     \rho_{p_h(\cdot)s,\Omega}(\nabla_{\! h} u_h^{\textit{cr}})&\lesssim 
     h_{\max}^{d(1-s)}\rho_{p_h(\cdot),\Omega}(\nabla_{\! h} u_h^{\textit{cr}})^s
     \\&\lesssim h_{\max}^{d(1-s)}\big(1+ \sigma(f;s)\big)^s\,.
     \end{aligned}
 \end{align}
 Next, using \eqref{cor:rate_improve.9}  and \eqref{cor:rate_improve.10} in \eqref{cor:rate_improve.3}, we arrive at 
 \begin{align}
 \label{cor:rate_improve.11}
    \begin{aligned}
     \|F_h(\cdot,\nabla_{\!h} u_h^{\textit{cr}})-F_h(\cdot,\nabla u)\|_{2,\Omega}^2&\lesssim h_{\max}^{2\alpha +d(1-s)}\,\big(1+\|\nabla F(\cdot,\nabla u)\|_{2,\Omega}^{2}\\&\quad+\|(\delta^{p(\cdot)-1}+\vert z\vert)^{p'(\cdot)-2}\vert f\vert^2\|_{1,\{p>2\}}\\&\quad+
  \sigma(f;s)^s+\rho_{p'(\cdot)s,\Omega}(f)+\rho_{p(\cdot)s,\Omega}( \nabla u)\big)\,.
 \end{aligned}
 \end{align}
 Aided by the a priori error estimate \eqref{cor:rate_improve.11}, we can improve the a priori estimate \eqref{cor:rate_improve.10} and,~in~turn, the a priori error estimate \eqref{cor:rate_improve.11}.
 Using the a priori error estimate \eqref{cor:rate_improve.11}, \eqref{cor:rate_improve.2.0}$_1$, the shift change \eqref{lem:shift-change.1}, and \eqref{cor:rate_improve.2.0}$_2$, we find that 
  \begin{align*}
     \rho_{\varphi_h,\Omega}(\nabla_{\! h} u_h^{\textit{cr}}-\nabla I_h^{sz} u)
     &\lesssim  c_\varepsilon\,\|F_h(\cdot,\nabla_{\!h} u_h^{\textit{cr}})-F_h(\cdot,\nabla I_h^{sz} u)\|_{2,\Omega}^2+\varepsilon\,\rho_{\varphi_h,\Omega}(\nabla_{\! h} u_h^{\textit{cr}})
     \\&\lesssim c_\varepsilon\big(\|F_h(\cdot,\nabla_{\!h} u_h^{\textit{cr}})-F_h(\cdot,\nabla u)\|_{2,\Omega}^2+\|F_h(\cdot,\nabla u)-F_h(\cdot,\nabla I_h^{sz} u)\|_{2,\Omega}^2\big)\\&\quad+\varepsilon\,\rho_{\varphi_h,\Omega}(\nabla_{\! h} u_h^{\textit{cr}}-\nabla I_h^{sz} u) +\varepsilon\,\rho_{\varphi_h,\Omega}(\nabla I_h^{sz} u)
     \\&\lesssim  c_\varepsilon\,h_{\max}^{2\alpha +d(1-s)}\big(1+\|\nabla F(\cdot,\nabla u)\|_{2,\Omega}^{2}+\|(\delta^{p(\cdot)-1}+\vert z\vert)^{p'(\cdot)-2}\vert f\vert^2\|_{1,\{p>2\}}\\&\quad+
  \sigma(f;s)^s+\rho_{p'(\cdot)s,\Omega}(f)+\rho_{p(\cdot)s,\Omega}( \nabla u)\big)\\&\quad+\varepsilon\,\rho_{p_h(\cdot),\Omega}(\nabla_{\! h} u_h^{\textit{cr}}-\nabla I_h^{sz} u) +\varepsilon\,(1+\rho_{p(\cdot)s,\Omega}( \nabla u))\,.
 \end{align*}
 Next, choosing $\varepsilon>0$ sufficiently small, using again \eqref{cor:rate_improve.2.0}$_2$, we obtain 
  \begin{align*}%\label{cor:rate_improve.12}
  %\begin{aligned}
     \rho_{p_h(\cdot),\Omega}(\nabla_{\! h} u_h^{\textit{cr}})&\lesssim  \max\{1,h_{\max}^{2\alpha +d(1-s)}\}\big(1+\|\nabla F(\cdot,\nabla u)\|_{2,\Omega}^{2}+\|(\delta^{p(\cdot)-1}+\vert z\vert)^{p'(\cdot)-2}\vert f\vert^2\|_{1,\{p>2\}}\notag\\&\quad+
\sigma(f;s)^s+\rho_{p'(\cdot)s,\Omega}(f)+\rho_{p(\cdot)s,\Omega}( \nabla u)\big)\,.%\label{cor:rate_improve.12}
  %\end{aligned}
 \end{align*}
As a result, using \eqref{cor:rate_improve.10}$_1$ and $2\alpha s +d(1-s^2)\ge 0$, we find that
\begin{align}%\label{cor:rate_improve.13}
%\begin{aligned}
    \rho_{p_h(\cdot)s,\Omega}(\nabla_{\! h} u_h^{\textit{cr}})&\lesssim  \max\{1,h_{\max}^{2\alpha s +d(1-s^2)}\}\big(1+\|\nabla F(\cdot,\nabla u)\|_{2,\Omega}^{2}+\|(\delta^{p(\cdot)-1}+\vert z\vert)^{p'(\cdot)-2}\vert f\vert^2\|_{1,\{p>2\}}\notag\\&\quad+
  \sigma(f;s)+\rho_{p'(\cdot)s,\Omega}(f)+\rho_{p(\cdot)s,\Omega}( \nabla u)\big)^s\notag
  \\&\lesssim  \big(1+\|\nabla F(\cdot,\nabla u)\|_{2,\Omega}^{2}+\|(\delta^{p(\cdot)-1}+\vert z\vert)^{p'(\cdot)-2}\vert f\vert^2\|_{1,\{p>2\}}\notag\\&\quad+
  \sigma(f;s)+\rho_{p'(\cdot)s,\Omega}(f)+\rho_{p(\cdot)s,\Omega}( \nabla u)\big)^s\,. \label{cor:rate_improve.13}
 % \end{aligned}
\end{align}
Eventually,   using \eqref{cor:rate_improve.9}  and \eqref{cor:rate_improve.13} in \eqref{cor:rate_improve.3}, we arrive at claimed a priori error estimate.
		\end{proof}
\newpage

 Aided by the (discrete) convex optimality relations \eqref{eq:pDirichletOptimality2} and \eqref{eq:pDirichletOptimalityCR2}, together with the generalized Marini formula, cf.\ \eqref{eq:gen_marini}, we can derive from Corollary \ref{thm:rate_u} an a priori error estimate for the error between the  dual solution and the discrete dual solution.%, measured in the conjugate~natural~distance,~cf.~Remark~\ref{rem:conjugate_natural_dist}.
 
 \begin{lemma}\label{lem:rate_z} 
		 Let $p\in C^0(\overline{\Omega})$ with $p^->1$ and $\delta\ge 0$ and let $f\in L^{p'(\cdot)}(\Omega)\cap \bigcap_{h\in (0,h_0]}{L^{p_h'(\cdot)}(\Omega)}$ for some $h_0>0$. Then, there exists some $s>1$,  which can chosen to be close to $1$ if $h>0$~is~close~to~$0$, such that if $u\in W^{1,p(\cdot)s}_D(\Omega)$, then for every $h\in (0,h_0]$, it holds
			\begin{align*}
				\|F^*_h(\cdot,z_h^{\textit{rt}})-F^*_h(\cdot,z)\|^2_{2,\Omega}&\lesssim \|F_h(\cdot,\nabla_{\!h} u_h^{\textit{cr}})-F_h(\cdot,\nabla u)\|^2_{2,\Omega}+ h_{\max}^{2\alpha}\,\big(1+\rho_{p(\cdot)s,\Omega}(\nabla u)\big)\\&\quad+
				\rho_{((\varphi_h)_{\smash{\vert \nabla u\vert}})^*,\Omega}(h_{\mathcal{T}}f)
				\,.
			\end{align*}
		\end{lemma}
		
		\begin{proof}
		 Using the discrete convex optimality relations \eqref{eq:pDirichletOptimality2} and \eqref{eq:pDirichletOptimalityCR2}, two equivalences in \eqref{eq:hammera}, the generalized Marini formula \eqref{eq:gen_marini}, again, the discrete convex optimality relations \eqref{eq:pDirichletOptimalityCR2}, and the Orlicz-stability of $\Pi_h$ (cf.\ \cite[Corollary A.8, (A.12)]{kr-phi-ldg}), we find that
		 \begin{align}\label{cor:convex_rate_z.1} 
		  \begin{aligned}
		 	\|F^*_h(\cdot,z_h^{\textit{rt}})-F^*_h(\cdot,z)\|^2_{2,\Omega}&\lesssim \|F^*_h(\cdot,\Pi_h z_h^{\textit{rt}})-F^*_h(\cdot,z)\|^2_{2,\Omega}
		 	+\|F^*_h(\cdot,z_h^{\textit{rt}})-F^*_h(\cdot,\Pi_h z_h^{\textit{rt}})\|^2_{2,\Omega}
		 	\\&
		 	\lesssim \|F^*_h(\cdot,\AAA_h(\cdot,\nabla_{\!h} u_h^{\textit{cr}}))-F^*_h(\cdot,\AAA_h(\cdot,\nabla u))\|^2_{2,\Omega}\\&\quad+ \|F^*_h(\cdot,\AAA_h(\cdot,\nabla u)-F^*_h(\cdot,\AAA(\cdot,\nabla u))\|^2_{2,\Omega}\\&\quad+\rho_{((\varphi_h)^*)_{\smash{\vert \Pi_h z_h^{\textit{rt}}\vert}},\Omega}(z_h^{\textit{rt}}-\Pi_h z_h^{\textit{rt}}) 
		 	%\\&
%		 	\lesssim \|F_h(\cdot,\nabla_{\!h} u_h^{\textit{cr}})-F_h(\cdot,\nabla u)\|^2_{2,\Omega}
 % + h_{\max}^{2\alpha}\,\big(1+\rho_{p(\cdot)s}(\nabla u)\big)\\&\quad +\rho_{((\varphi_h)^*)_{\smash{\vert \AAA_h(\cdot,\nabla_{\! h }u_h^{\textit{cr}})\vert}},\Omega}(h_{\mathcal{T}}f_h)
		 	\\&
		 	\lesssim \|F_h(\cdot,\nabla_{\!h} u_h^{\textit{cr}})-F_h(\cdot,\nabla u)\|^2_{2,\Omega}
 + h_{\max}^{2\alpha}\,\big(1+\rho_{p(\cdot)s}(\nabla u)\big)\\&\quad+\rho_{((\varphi_h)^*)_{\smash{\vert \AAA_h(\cdot,\nabla_{\! h }u_h^{\textit{cr}})\vert}},\Omega}(h_{\mathcal{T}}f)\,.
		 	 \end{aligned}
		 \end{align}
		 Using that $(\varphi^*)_{\vert \AAA(q_T,a)\vert }(q_T,\cdot)\sim (\varphi_{\vert a\vert })^*(q_T,\cdot)$ for all $T\in \mathcal{T}_h$ and $a\in \mathbb{R}^d$ (cf.\ Lemma \ref{lem:hammer} \eqref{eq:hammerg}) and the shift change \eqref{lem:shift-change.3},~we~observe~that
		 \begin{align}\label{cor:convex_rate_z.2} 
		 \begin{aligned}
		  \rho_{((\varphi_h)^*)_{\smash{\vert \AAA_h(\cdot,\nabla_{\! h }u_h^{\textit{cr}})\vert}},\Omega}(h_{\mathcal{T}}f)
		  %\\&
            \lesssim \rho_{((\varphi_h)_{\smash{\vert \nabla u\vert}})^*,\Omega}(h_{\mathcal{T}}f)+\|F_h(\cdot,\nabla_{\!h} u_h^{\textit{cr}})-F_h(\cdot,\nabla u)\|^2_{2,\Omega}\,.
		 \end{aligned}
		 \end{align}
		 Eventually, combining \eqref{cor:convex_rate_z.1} and \eqref{cor:convex_rate_z.2}, we arrive at the claimed inequality.
		\end{proof}
		
		\begin{theorem}\label{thm:rate_z} Let $p\in C^{0,\alpha}(\overline{\Omega})$ with $\alpha\in (0,1]$ and $p^->1$ and $\delta\ge 0$.~Moreover,~let~$F(\cdot,\nabla u)\in W^{1,2}(\Omega;\mathbb{R}^d)$, $(\delta^{p(\cdot)-1}+\vert z\vert )^{p'(\cdot)-2}\vert f\vert^2\in L^1(\{p> 2\})$, and $f\in L^{(p^-)'}(\Omega)$ or $\Gamma_D=\partial \Omega$.~Then, assuming that $h_{\max}\sim h_{\mathcal{T}}$, there exists some $s>1$, which can chosen to be close to $1$ if $h_{\max}>0$ is close to $0$, such that if $f\in L^{p'(\cdot)s}(\Omega)$ and $2\alpha s+d(1-s^2)\ge 0$, then
		\begin{align*}
		 	\|F^*_h(\cdot,z_h^{\textit{rt}})-F^*_h(\cdot,z)\|^2_{2,\Omega}&\lesssim h_{\max}^2\,\big(1+\|\nabla F(\cdot,\nabla u)\|_{2,\Omega}^{2}+\|(\delta^{p(\cdot)-1}+\vert z\vert)^{p'(\cdot)-2}\vert f\vert^2\|_{1,\{p>2\}}\\&\quad+
            \sigma(f;s)+\rho_{p'(\cdot)s,\Omega}(f)+\rho_{p(\cdot)s,\Omega}( \nabla u)\big)^s\,,
		\end{align*}
        where the hidden constant also depends on $s>1$ and $\sigma(f;s)$ is defined as in Theorem \ref{thm:rate_u}.
		\end{theorem}

        \begin{proof}
            Immediate consequence of Lemma \ref{lem:rate_z} in conjunction with Theorem \ref{thm:rate_u}.
        \end{proof}

        \begin{corollary}\label{cor:rate}
        Let $p\in C^{0,1}(\overline{\Omega})$ with $p^->1$ and $\delta> 0$. Moreover, let $F(\cdot,\nabla u)\in W^{1,2}(\Omega;\mathbb{R}^d)$ and $f\hspace{-0.1em}\in\hspace{-0.1em} L^{(p^-)'}(\Omega)$ or $\Gamma_D\hspace{-0.1em}=\hspace{-0.1em}\partial \Omega$.
		Then, assuming that $h_{\max}\hspace{-0.1em}\sim\hspace{-0.1em} h_{\mathcal{T}}$, there exists some $s\hspace{-0.1em}>\hspace{-0.1em}1$,~which~can chosen to be close to $1$ if $h_{\max}$ is close to $0$, such that if $f\hspace{-0.1em}\in\hspace{-0.1em} L^{p'(\cdot)s}(\Omega)$~and~${2\alpha s+d(1-s^2)\hspace{-0.1em}\ge\hspace{-0.1em} 0}$,~then
        \begin{align*}
		 	&\|F_h(\cdot,\nabla_{\!h} u_h^{\textit{cr}})-F_h(\cdot,\nabla u)\|_{2,\Omega}^2+\|F^*_h(\cdot,z_h^{\textit{rt}})-F^*_h(\cdot,z)\|^2_{2,\Omega}\\&\quad\lesssim h_{\max}^2\,\big(1+\|\nabla F(\cdot,\nabla u)\|_{2,\Omega}^2+
           \sigma(f;s)+ \rho_{p'(\cdot)s,\Omega}(f)+\rho_{p(\cdot)s,\Omega}( \nabla u)\big)^s\,,
		\end{align*}
        where the hidden constant also depends on $s>1$ and $\sigma(f;s)$ is defined as in Theorem \ref{thm:rate_u}.\enlargethispage{5mm}
        \end{corollary}
        
        \begin{proof}
            Due to Lemma \ref{lem:reg_equiv}, from $F(\cdot,\nabla u)\in W^{1,2}(\Omega;\mathbb{R}^d)$, it follows that $F^*(\cdot,z)\in W^{1,2}(\Omega;\mathbb{R}^d)$ and $\vert \nabla F(\cdot,\nabla u)\vert^2+(1+\vert \nabla v\vert^{p(\cdot)s})\sim \vert \nabla F^*(\cdot,z)\vert^2+(1+\vert \nabla z\vert^{p'(\cdot)s})$ a.e.\ in $\Omega$ for some $s>1$ which can chosen to be close to $1$. In addition, due to Lemma \ref{lem:reg_dual_source}, it holds $\vert \nabla F^*(\cdot,z)\vert^2+(1+\vert z\vert^{p'(\cdot)s})\sim (\delta^{p(\cdot)-1}+\vert z\vert)^{p'(\cdot)-2} \vert \nabla z\vert^2$ a.e.\ in $\Omega$. As a result, the claimed a priori error estimate follows from Theorem \ref{thm:rate_u} together with Theorem \ref{thm:rate_z}.
        \end{proof}

        \section{Numerical experiments}\label{sec:experiments}
        
	\hspace{5mm}In this section, we review the theoretical findings of  Section \ref{sec:a_priori} via numerical experiments.
    All experiments were carried out using the finite element software package \mbox{\textsf{FEniCS}} (version~2019.1.0), cf. \cite{LW10}. 
	
	
	We apply the $\smash{\mathcal{S}^{1,\textit{cr}}_D(\mathcal{T}_h)}$-approximation \eqref{eq:pDirichletS1crD} of the variational 
	$p$-Dirichlet problem \eqref{eq:pDirichletW1p} with 
    $\delta\coloneqq 1\textrm{e}{-}4$ and $p\in C^{0,\alpha}(\overline{\Omega})$, where $\alpha \in (0,1]$ and $p^->1$, for every $x\in \overline{\Omega}$ defined by
    \begin{align*}
        p(x)\coloneqq p^-+\varepsilon\,\vert x\vert^{\alpha}\,,
    \end{align*}
    where $\varepsilon>0$. As quadrature points of the one-point quadrature rule used to discretize $p\in C^{0,\alpha}(\overline{\Omega})$, we employ barycenters  of elements, i.e., for every $T\in \mathcal{T}_h$, we employ $q_T\coloneqq x_T= \frac{1}{d+1}\sum_{z\in \mathcal{N}_h\cap T}{z}$.
    
    Then, we approximate the discrete primal solution ${u_h^{\textit{cr}}\in \smash{\mathcal{S}^{1,\textit{cr}}_D(\mathcal{T}_h)}}$  deploying the Newton line search algorithm  of \mbox{\textsf{PETSc}} (version     3.17.3), cf. \cite{LW10}, with an absolute tolerance~of~${\tau_{abs}= 1\textrm{e}{-}8}$~and a relative tolerance of $\tau_{rel}=1\textrm{e}{-}10$. The linear system  emerging  in each Newton~step~is~solved using a sparse direct solver from \textup{\textsf{MUMPS}} (version 5.5.0), cf. \cite{mumps}.
    
    For  our numerical experiments, we choose $\Omega= (-1,1)^2$, $\Gamma_D=\partial \Omega$, and as a manufactured solution of \eqref{eq:pDirichlet}, the function $u\in W^{1,p(\cdot)}_D(\Omega)$, for every $x\coloneqq (x_1,x_2)^\top\in \Omega$ defined by
    \begin{align*}
    	\smash{u(x)\coloneqq d(x)\,\vert x\vert^{\beta}}\,,
    \end{align*}
    i.e., $f\coloneqq -\textup{div}\,\AAA(\cdot,\nabla u)$,
    where 
      $d\in C^\infty(\overline{\Omega})$, defined by $d(x)\coloneqq (1-x_1^2)\,(1-x_2^2)$ for every $x\coloneqq(x_1,x_2)^\top \in\overline{\Omega}$,
    is a smooth cut-off function enforcing homogeneous Dirichlet~\mbox{boundary}~\mbox{conditions}. Moreover, we  choose $\beta=1.01$, which just yields that $u\in W^{1,p(\cdot)}_D(\Omega)$ satisfies 
    \begin{align*}
        \smash{F(\cdot,\nabla u),F^*(\cdot,z)\in W^{1,2}(\Omega;\mathbb{R}^2)}\quad\text{ and }\quad \smash{(\delta^{p(\cdot)-1}+\vert z\vert)^{p'(\cdot)-2}\vert \nabla z\vert^2}\in L^1(\Omega)\,.
    \end{align*}
    Due \hspace{-0.1mm}to \hspace{-0.1mm}Theorem \hspace{-0.1mm}\ref{thm:rate_u} \hspace{-0.1mm}and \hspace{-0.1mm}Theorem \hspace{-0.1mm}\ref{thm:rate_z}, \hspace{-0.1mm}we \hspace{-0.1mm}can \hspace{-0.1mm}expect \hspace{-0.1mm}the
    \hspace{-0.1mm}convergence \hspace{-0.1mm}rate \hspace{-0.1mm}$\alpha$ \hspace{-0.1mm}for \hspace{-0.1mm}the~\hspace{-0.1mm}quantities~\hspace{-0.1mm}\eqref{errors}.

    An initial triangulation $\mathcal
    T_{h_0}$, $h_0=\smash{\frac{3}{2\sqrt{2}}}$, is constructed by subdividing a rectangular~Cartesian grid into regular triangles with different orientations.  Refined     triangulations $\mathcal T_{h_k}$,~$k=1,\dots,7$, where $h_{k+1}=\frac{h_k}{2}$ for all $k=1,\dots,7$, are 
    obtained by
    %regular subdivision of the previous grid: each triangle is subdivided
    %into four equal triangles by connecting the midpoints~of~the~edges,~i.e., 
    applying the red-refinement rule, cf.\ \cite{Car04}.
    %We construct an initial triangulation $\mathcal T_{h_0}$, where $h_0=\frac{1}{\sqrt{2}}$, by subdividing a~rectangular~Cartesian grid into regular triangles with different orientations.  Finer     triangulations $\mathcal T_{h_k}$, $k=1,\dots,10$, where $h_{k+1}=\frac{h_k}{2}$ for all $k=1,\dots,10$, are obtained by regular subdivision of the previous grid: each triangle is subdivided into four equal triangles by connecting the midpoints of the edges, i.e., applying the red-refinement rule, cf. \cite{Car04}.
    
    Then, for the resulting series of triangulations $\mathcal T_k\coloneqq \mathcal T_{h_k}$, $k=1,\dots,10$, we apply the above Newton scheme to compute the discrete primal solution $u_k^{\textit{cr}}\coloneqq u_{h_k}^{\textit{cr}}\in \mathcal{S}^{1,\textit{cr}}_D(\mathcal{T}_k)$,~${k=1,\dots,10}$,~and, resorting to the generalized Marini formula \eqref{eq:gen_marini}, the discrete dual solution ${z_k^{\textit{rt}}\coloneqq z_{h_k}^{\textit{rt}}\in\mathcal{R}T^0_N(\mathcal{T}_k)}$, $k=1,\dots,10$. Subsequently, we compute the error quantities
    \begin{align}\label{errors}
    	\left.\begin{aligned}
    		e_{F,k}&\coloneqq \|F_h(\cdot,\nabla_{\!h_k}u_k^{\textit{cr}} )-F_h(\cdot,\nabla u)\|_{2,\Omega}^2\,,\\
    		e_{F^*,k}&\coloneqq \|F^*_h(\cdot,z_k^{\textit{cr}})-F^*_h(\cdot,z)\|_{2,\Omega}^2\,,
    \end{aligned}\quad\right\}\quad k=1,\dots,10\,.
    \end{align}
    For the determination of the convergence rates,  the experimental order of convergence~(EOC)
    %As estimation of the convergence rates,  the experimental order of convergence~(EOC)
    \begin{align*}
    	\texttt{EOC}_k(e_k)\coloneqq \frac{\log(e_k/e_{k-1})}{\log(h_k/h_{k-1})}\,, \quad k=1,\dots,10\,,
    \end{align*}
    where for every $k= 1,\dots,10$, we denote by $e_k$,
    either $e_{F,k}$ or 
    $e_{F^*,k}$, 
    ~\mbox{respectively},~is~recorded.\enlargethispage{7mm}
    
    For different values of $p^-\in \{1.5, 2, 2.5\}$, $\alpha\in \{0.1,0.25,0.5,1.0\}$, $\varepsilon\in \{0.5,1.0\}$, and a
    series of triangulations \hspace{-0.1mm}$\mathcal{T}_k$, \hspace{-0.1mm}$k = 1,\dots,10$,
    \hspace{-0.1mm}obtained \hspace{-0.1mm}by \hspace{-0.1mm}uniform \hspace{-0.1mm}mesh \hspace{-0.1mm}refinement~\hspace{-0.1mm}as~\hspace{-0.1mm}\mbox{described}~\hspace{-0.1mm}above,~\hspace{-0.1mm}the~\hspace{-0.1mm}EOC is
    computed and for $k = 5,\dots,10$ presented in Table~\ref{tab1}, Table~\ref{tab2}, Table~\ref{tab3}~and~Table~\ref{tab4},   
    respectively. In each case, we report a convergence ratio of about $\texttt{EOC}_k(e_k)\approx 1$, $k=5,\dots,10$, conforming the optimality of the  a priori error estimates derived in Corollary~\ref{cor:rate} and also indicating the sub-optimality of the a priori error estimates derived in Theorem \ref{thm:rate_u} and Theorem \ref{thm:rate_z} for $\alpha\in (0,1)$. We believe that this can be traced back to an imbalance of between the regularity assumptions $F(\cdot,\nabla u)\in W^{1,2}(\Omega;\mathbb{R}^d)$ and $p\in C^{0,\alpha}(\overline{\Omega})$ with $\alpha\in (0,1)$ in Theorem \ref{thm:rate_u} and Theorem \ref{thm:rate_z};~and expect that  the assumptions $F(\cdot,\nabla u)\in \mathcal{N}^{\alpha,2}(\Omega)$ and $p\in C^{0,\alpha}(\overline{\Omega})$ with $\alpha\in (0,1)$ 
    are more balanced and may lead to optimal a priori error estimates. The examination of this hypothesis, however, is beyond the scope of this article and, therefore, left open for follow-up research.
\begin{table}[H]
     \setlength\tabcolsep{3pt}
 	\centering
 	\begin{tabular}{c |c|c|c|c|c|c|c|c|c|c|c|c|} 
 	\hline 
    \multicolumn{1}{|c||}{\cellcolor{lightgray}$\varepsilon$}	
    & \multicolumn{12}{c|}{\cellcolor{lightgray}$1.0$}\\\hline 
    \multicolumn{1}{|c||}{\cellcolor{lightgray}$\alpha$}	
    & \multicolumn{3}{c||}{\cellcolor{lightgray}$0.1$} & \multicolumn{3}{c||}{\cellcolor{lightgray}$0.25$} & \multicolumn{3}{c||}{\cellcolor{lightgray}$0.5$} & \multicolumn{3}{c|}{\cellcolor{lightgray}$1.0$}\\\hline 
\multicolumn{1}{|c||}{\cellcolor{lightgray}\diagbox[height=1.1\line,width=0.11\dimexpr\linewidth]{\vspace{-0.6mm}$k$}{\\[-5mm] $p^-$}}
& \cellcolor{lightgray}1.5
& \cellcolor{lightgray}2.0 &  \multicolumn{1}{c||}{\cellcolor{lightgray}2.5}  &  \multicolumn{1}{c|}{\cellcolor{lightgray}1.5} &\cellcolor{lightgray}2.0  & \multicolumn{1}{c||}{\cellcolor{lightgray}2.5}  &  \multicolumn{1}{c|}{\cellcolor{lightgray}1.5} & \cellcolor{lightgray}2.0  &  \multicolumn{1}{c||}{\cellcolor{lightgray}2.5}   &  \multicolumn{1}{c|}{\cellcolor{lightgray}1.5} & \cellcolor{lightgray}2.0    & \cellcolor{lightgray}2.5   \\ \hline\hline
\multicolumn{1}{|c||}{\cellcolor{lightgray}$5$} & 0.978 & 0.986 & \multicolumn{1}{c||}{0.987} & \multicolumn{1}{c|}{0.958} & 0.972 & \multicolumn{1}{c||}{0.974} & \multicolumn{1}{c|}{0.971} & 0.986 & \multicolumn{1}{c||}{0.989} & \multicolumn{1}{c|}{0.969} & 0.988 & 0.992 \\ \hline
\multicolumn{1}{|c||}{\cellcolor{lightgray}$6$} & 0.979 & 0.989 & \multicolumn{1}{c||}{0.992} & \multicolumn{1}{c|}{0.975} & 0.985 & \multicolumn{1}{c||}{0.985} & \multicolumn{1}{c|}{0.971} & 0.987 & \multicolumn{1}{c||}{0.992} & \multicolumn{1}{c|}{0.970} & 0.988 & 0.993 \\ \hline
\multicolumn{1}{|c||}{\cellcolor{lightgray}$7$} & 0.981 & 0.990 & \multicolumn{1}{c||}{0.994} & \multicolumn{1}{c|}{0.975} & 0.988 & \multicolumn{1}{c||}{0.988} & \multicolumn{1}{c|}{0.972} & 0.988 & \multicolumn{1}{c||}{0.993} & \multicolumn{1}{c|}{0.972} & 0.988 & 0.994 \\ \hline
\multicolumn{1}{|c||}{\cellcolor{lightgray}$8$} & 0.981 & 0.990 & \multicolumn{1}{c||}{0.994} & \multicolumn{1}{c|}{0.976} & 0.989 & \multicolumn{1}{c||}{0.989} & \multicolumn{1}{c|}{0.973} & 0.988 & \multicolumn{1}{c||}{0.994} & \multicolumn{1}{c|}{0.973} & 0.989 & 0.994 \\ \hline
\multicolumn{1}{|c||}{\cellcolor{lightgray}$9$} & 0.981 & 0.990 & \multicolumn{1}{c||}{0.994} & \multicolumn{1}{c|}{0.976} & 0.989 & \multicolumn{1}{c||}{0.990} & \multicolumn{1}{c|}{0.974} & 0.989 & \multicolumn{1}{c||}{0.994} & \multicolumn{1}{c|}{0.974} & 0.989 & 0.994 \\ \hline
\multicolumn{1}{|c||}{\cellcolor{lightgray}$10$}& 0.982 & 0.990 & \multicolumn{1}{c||}{0.995} & \multicolumn{1}{c|}{0.977} & 0.989 & \multicolumn{1}{c||}{0.990} & \multicolumn{1}{c|}{0.975} & 0.989 & \multicolumn{1}{c||}{0.994} & \multicolumn{1}{c|}{0.975} & 0.989 & 0.994 \\ \hline\hline
\multicolumn{1}{|c||}{\cellcolor{lightgray}\small \textrm{expected}}   & 0.10  & 0.10  & \multicolumn{1}{c||}{0.10}  & \multicolumn{1}{c|}{0.25} & 0.25  & \multicolumn{1}{c||}{0.25}  & \multicolumn{1}{c|}{0.50}  & 0.50  & \multicolumn{1}{c||}{0.50}  & \multicolumn{1}{c|}{1.00}  & 1.00  & 1.00 \\ \hline
\end{tabular}\vspace{-2mm}
 	\caption{Experimental order of convergence: $\texttt{EOC}_k(e_{F,k})$,~${k=5,\dots,10}$.} 
 	\label{tab1}
 \end{table}\vspace{-5mm}

\begin{table}[H]
     \setlength\tabcolsep{3pt}
 	\centering
 	\begin{tabular}{c |c|c|c|c|c|c|c|c|c|c|c|c|} 
 	\hline 
    \multicolumn{1}{|c||}{\cellcolor{lightgray}$\varepsilon$}	
    & \multicolumn{12}{c|}{\cellcolor{lightgray}$1.0$}\\\hline 
    \multicolumn{1}{|c||}{\cellcolor{lightgray}$\alpha$}	
    & \multicolumn{3}{c||}{\cellcolor{lightgray}$0.1$} & \multicolumn{3}{c||}{\cellcolor{lightgray}$0.25$} & \multicolumn{3}{c||}{\cellcolor{lightgray}$0.5$} & \multicolumn{3}{c|}{\cellcolor{lightgray}$1.0$}\\\hline 
\multicolumn{1}{|c||}{\cellcolor{lightgray}\diagbox[height=1.1\line,width=0.11\dimexpr\linewidth]{\vspace{-0.6mm}$k$}{\\[-5mm] $p^-$}}
& \cellcolor{lightgray}1.5
& \cellcolor{lightgray}2.0 &  \multicolumn{1}{c||}{\cellcolor{lightgray}2.5}  &  \multicolumn{1}{c|}{\cellcolor{lightgray}1.5} &\cellcolor{lightgray}2.0  & \multicolumn{1}{c||}{\cellcolor{lightgray}2.5}  &  \multicolumn{1}{c|}{\cellcolor{lightgray}1.5} & \cellcolor{lightgray}2.0  &  \multicolumn{1}{c||}{\cellcolor{lightgray}2.5}   &  \multicolumn{1}{c|}{\cellcolor{lightgray}1.5} & \cellcolor{lightgray}2.0    & \cellcolor{lightgray}2.5   \\ \hline\hline
\multicolumn{1}{|c||}{\cellcolor{lightgray}$5$} & 0.978 & 0.986 & \multicolumn{1}{c||}{0.987} & \multicolumn{1}{c|}{0.975} & 0.985 & \multicolumn{1}{c||}{0.988} & \multicolumn{1}{c|}{0.971} & 0.986 & \multicolumn{1}{c||}{0.989} & \multicolumn{1}{c|}{0.969} & 0.988 & 0.992 \\ \hline
\multicolumn{1}{|c||}{\cellcolor{lightgray}$6$} & 0.979 & 0.989 & \multicolumn{1}{c||}{0.992} & \multicolumn{1}{c|}{0.975} & 0.988 & \multicolumn{1}{c||}{0.992} & \multicolumn{1}{c|}{0.971} & 0.987 & \multicolumn{1}{c||}{0.992} & \multicolumn{1}{c|}{0.970} & 0.988 & 0.993 \\ \hline
\multicolumn{1}{|c||}{\cellcolor{lightgray}$7$} & 0.981 & 0.990 & \multicolumn{1}{c||}{0.994} & \multicolumn{1}{c|}{0.976} & 0.989 & \multicolumn{1}{c||}{0.993} & \multicolumn{1}{c|}{0.972} & 0.988 & \multicolumn{1}{c||}{0.993} & \multicolumn{1}{c|}{0.972} & 0.988 & 0.994 \\ \hline
\multicolumn{1}{|c||}{\cellcolor{lightgray}$8$} & 0.981 & 0.990 & \multicolumn{1}{c||}{0.994} & \multicolumn{1}{c|}{0.976} & 0.989 & \multicolumn{1}{c||}{0.994} & \multicolumn{1}{c|}{0.973} & 0.988 & \multicolumn{1}{c||}{0.994} & \multicolumn{1}{c|}{0.973} & 0.989 & 0.994 \\ \hline
\multicolumn{1}{|c||}{\cellcolor{lightgray}$9$} & 0.981 & 0.990 & \multicolumn{1}{c||}{0.994} & \multicolumn{1}{c|}{0.977} & 0.989 & \multicolumn{1}{c||}{0.994} & \multicolumn{1}{c|}{0.974} & 0.989 & \multicolumn{1}{c||}{0.994} & \multicolumn{1}{c|}{0.974} & 0.989 & 0.994 \\ \hline
\multicolumn{1}{|c||}{\cellcolor{lightgray}$10$}& 0.982 & 0.990 & \multicolumn{1}{c||}{0.995} & \multicolumn{1}{c|}{0.977} & 0.989 & \multicolumn{1}{c||}{0.994} & \multicolumn{1}{c|}{0.975} & 0.989 & \multicolumn{1}{c||}{0.994} & \multicolumn{1}{c|}{0.975} & 0.989 & 0.994 \\ \hline\hline
\multicolumn{1}{|c||}{\cellcolor{lightgray}\small \textrm{expected}}   & 0.10  & 0.10  & \multicolumn{1}{c||}{0.10}  & \multicolumn{1}{c|}{0.25} & 0.25  & \multicolumn{1}{c||}{0.25}  & \multicolumn{1}{c|}{0.50}  & 0.50  & \multicolumn{1}{c||}{0.50}  & \multicolumn{1}{c|}{1.00}  & 1.00  & 1.00 \\ \hline
\end{tabular}\vspace{-2mm}
 	\caption{Experimental order of convergence: $\texttt{EOC}_k(e_{F^*,k})$,~${k=5,\dots,10}$.} 
 	\label{tab2}
 \end{table}\vspace{-5mm}


\begin{table}[H]
     \setlength\tabcolsep{3pt}
 	\centering
 	\begin{tabular}{c |c|c|c|c|c|c|c|c|c|c|c|c|} 
 	\hline 
    \multicolumn{1}{|c||}{\cellcolor{lightgray}$\varepsilon$}	
    & \multicolumn{12}{c|}{\cellcolor{lightgray}$0.5$}\\\hline 
    \multicolumn{1}{|c||}{\cellcolor{lightgray}$\alpha$}	
    & \multicolumn{3}{c||}{\cellcolor{lightgray}$0.1$} & \multicolumn{3}{c||}{\cellcolor{lightgray}$0.25$} & \multicolumn{3}{c||}{\cellcolor{lightgray}$0.5$} & \multicolumn{3}{c|}{\cellcolor{lightgray}$1.0$}\\\hline 
\multicolumn{1}{|c||}{\cellcolor{lightgray}\diagbox[height=1.1\line,width=0.11\dimexpr\linewidth]{\vspace{-0.6mm}$k$}{\\[-5mm] $p^-$}}
& \cellcolor{lightgray}1.5
& \cellcolor{lightgray}2.0 &  \multicolumn{1}{c||}{\cellcolor{lightgray}2.5}  &  \multicolumn{1}{c|}{\cellcolor{lightgray}1.5} &\cellcolor{lightgray}2.0  & \multicolumn{1}{c||}{\cellcolor{lightgray}2.5}  &  \multicolumn{1}{c|}{\cellcolor{lightgray}1.5} & \cellcolor{lightgray}2.0  &  \multicolumn{1}{c||}{\cellcolor{lightgray}2.5}   &  \multicolumn{1}{c|}{\cellcolor{lightgray}1.5} & \cellcolor{lightgray}2.0    & \cellcolor{lightgray}2.5   \\ \hline\hline
\multicolumn{1}{|c||}{\cellcolor{lightgray}$5$} & 0.959 & 0.979 & \multicolumn{1}{c||}{0.986} & \multicolumn{1}{c|}{0.975} & 0.985 & \multicolumn{1}{c||}{0.988} & \multicolumn{1}{c|}{0.971} & 0.978 & \multicolumn{1}{c||}{0.986} & \multicolumn{1}{c|}{0.944} & 0.979 & 0.988 \\ \hline
\multicolumn{1}{|c||}{\cellcolor{lightgray}$6$} & 0.965 & 0.981 & \multicolumn{1}{c||}{0.989} & \multicolumn{1}{c|}{0.975} & 0.988 & \multicolumn{1}{c||}{0.992} & \multicolumn{1}{c|}{0.971} & 0.979 & \multicolumn{1}{c||}{0.989} & \multicolumn{1}{c|}{0.958} & 0.980 & 0.989 \\ \hline
\multicolumn{1}{|c||}{\cellcolor{lightgray}$7$} & 0.966 & 0.983 & \multicolumn{1}{c||}{0.990} & \multicolumn{1}{c|}{0.976} & 0.989 & \multicolumn{1}{c||}{0.993} & \multicolumn{1}{c|}{0.972} & 0.981 & \multicolumn{1}{c||}{0.990} & \multicolumn{1}{c|}{0.958} & 0.981 & 0.990 \\ \hline
\multicolumn{1}{|c||}{\cellcolor{lightgray}$8$} & 0.968 & 0.983 & \multicolumn{1}{c||}{0.991} & \multicolumn{1}{c|}{0.976} & 0.989 & \multicolumn{1}{c||}{0.994} & \multicolumn{1}{c|}{0.973} & 0.982 & \multicolumn{1}{c||}{0.990} & \multicolumn{1}{c|}{0.962} & 0.982 & 0.990 \\ \hline
\multicolumn{1}{|c||}{\cellcolor{lightgray}$9$} & 0.969 & 0.984 & \multicolumn{1}{c||}{0.991} & \multicolumn{1}{c|}{0.977} & 0.989 & \multicolumn{1}{c||}{0.994} & \multicolumn{1}{c|}{0.974} & 0.982 & \multicolumn{1}{c||}{0.990} & \multicolumn{1}{c|}{0.964} & 0.983 & 0.991 \\ \hline
\multicolumn{1}{|c||}{\cellcolor{lightgray}$10$}& 0.970 & 0.984 & \multicolumn{1}{c||}{0.991} & \multicolumn{1}{c|}{0.977} & 0.989 & \multicolumn{1}{c||}{0.994} & \multicolumn{1}{c|}{0.975} & 0.983 & \multicolumn{1}{c||}{0.991} & \multicolumn{1}{c|}{0.966} & 0.983 & 0.991 \\ \hline\hline
\multicolumn{1}{|c||}{\cellcolor{lightgray}\small \textrm{expected}}   & 0.10  & 0.10  & \multicolumn{1}{c||}{0.10}  & \multicolumn{1}{c|}{0.25} & 0.25  & \multicolumn{1}{c||}{0.25}  & \multicolumn{1}{c|}{0.50}  & 0.50  & \multicolumn{1}{c||}{0.50}  & \multicolumn{1}{c|}{1.00}  & 1.00  & 1.00 \\ \hline
\end{tabular}\vspace{-2mm}
 	\caption{Experimental order of convergence: $\texttt{EOC}_k(e_{F,k})$,~${k=5,\dots,10}$.} 
 	\label{tab3}
 \end{table}\vspace{-5mm}

\begin{table}[H]
     \setlength\tabcolsep{3pt}
 	\centering
 	\begin{tabular}{c |c|c|c|c|c|c|c|c|c|c|c|c|} 
 	\hline 
    \multicolumn{1}{|c||}{\cellcolor{lightgray}$\varepsilon$}	
    & \multicolumn{12}{c|}{\cellcolor{lightgray}$0.5$}\\\hline 
    \multicolumn{1}{|c||}{\cellcolor{lightgray}$\alpha$}	
    & \multicolumn{3}{c||}{\cellcolor{lightgray}$0.1$} & \multicolumn{3}{c||}{\cellcolor{lightgray}$0.25$} & \multicolumn{3}{c||}{\cellcolor{lightgray}$0.5$} & \multicolumn{3}{c|}{\cellcolor{lightgray}$1.0$}\\\hline 
\multicolumn{1}{|c||}{\cellcolor{lightgray}\diagbox[height=1.1\line,width=0.11\dimexpr\linewidth]{\vspace{-0.6mm}$k$}{\\[-5mm] $p^-$}}
& \cellcolor{lightgray}1.5
& \cellcolor{lightgray}2.0 &  \multicolumn{1}{c||}{\cellcolor{lightgray}2.5}  &  \multicolumn{1}{c|}{\cellcolor{lightgray}1.5} &\cellcolor{lightgray}2.0  & \multicolumn{1}{c||}{\cellcolor{lightgray}2.5}  &  \multicolumn{1}{c|}{\cellcolor{lightgray}1.5} & \cellcolor{lightgray}2.0  &  \multicolumn{1}{c||}{\cellcolor{lightgray}2.5}   &  \multicolumn{1}{c|}{\cellcolor{lightgray}1.5} & \cellcolor{lightgray}2.0    & \cellcolor{lightgray}2.5   \\ \hline\hline
\multicolumn{1}{|c||}{\cellcolor{lightgray}$5$} & 0.959 & 0.979 & \multicolumn{1}{c||}{0.986} & \multicolumn{1}{c|}{0.953} & 0.978 & \multicolumn{1}{c||}{0.986} & \multicolumn{1}{c|}{0.947} & 0.978 & \multicolumn{1}{c||}{0.986} & \multicolumn{1}{c|}{0.943} & 0.979 & 0.988 \\ \hline
\multicolumn{1}{|c||}{\cellcolor{lightgray}$6$} & 0.965 & 0.981 & \multicolumn{1}{c||}{0.989} & \multicolumn{1}{c|}{0.960} & 0.980 & \multicolumn{1}{c||}{0.989} & \multicolumn{1}{c|}{0.956} & 0.979 & \multicolumn{1}{c||}{0.989} & \multicolumn{1}{c|}{0.958} & 0.980 & 0.989 \\ \hline
\multicolumn{1}{|c||}{\cellcolor{lightgray}$7$} & 0.966 & 0.983 & \multicolumn{1}{c||}{0.990} & \multicolumn{1}{c|}{0.961} & 0.981 & \multicolumn{1}{c||}{0.990} & \multicolumn{1}{c|}{0.958} & 0.981 & \multicolumn{1}{c||}{0.990} & \multicolumn{1}{c|}{0.958} & 0.981 & 0.990 \\ \hline
\multicolumn{1}{|c||}{\cellcolor{lightgray}$8$} & 0.968 & 0.983 & \multicolumn{1}{c||}{0.991} & \multicolumn{1}{c|}{0.963} & 0.982 & \multicolumn{1}{c||}{0.990} & \multicolumn{1}{c|}{0.961} & 0.982 & \multicolumn{1}{c||}{0.990} & \multicolumn{1}{c|}{0.962} & 0.982 & 0.990 \\ \hline
\multicolumn{1}{|c||}{\cellcolor{lightgray}$9$} & 0.969 & 0.984 & \multicolumn{1}{c||}{0.991} & \multicolumn{1}{c|}{0.964} & 0.983 & \multicolumn{1}{c||}{0.991} & \multicolumn{1}{c|}{0.963} & 0.982 & \multicolumn{1}{c||}{0.990} & \multicolumn{1}{c|}{0.964} & 0.983 & 0.991 \\ \hline
\multicolumn{1}{|c||}{\cellcolor{lightgray}$10$}& 0.970 & 0.984 & \multicolumn{1}{c||}{0.991} & \multicolumn{1}{c|}{0.966} & 0.983 & \multicolumn{1}{c||}{0.991} & \multicolumn{1}{c|}{0.965} & 0.983 & \multicolumn{1}{c||}{0.991} & \multicolumn{1}{c|}{0.966} & 0.983 & 0.991 \\ \hline\hline
\multicolumn{1}{|c||}{\cellcolor{lightgray}\small \textrm{expected}}   & 0.10  & 0.10  & \multicolumn{1}{c||}{0.10}  & \multicolumn{1}{c|}{0.25} & 0.25  & \multicolumn{1}{c||}{0.25}  & \multicolumn{1}{c|}{0.50}  & 0.50  & \multicolumn{1}{c||}{0.50}  & \multicolumn{1}{c|}{1.00}  & 1.00  & 1.00 \\ \hline
\end{tabular}\vspace{-2mm}
 	\caption{Experimental order of convergence: $\texttt{EOC}_k(e_{F^*,k})$,~${k=5,\dots,10}$.} 
 	\label{tab4}
 \end{table}\vspace{-5mm}

 
 {\setlength{\bibsep}{0pt plus 0.0ex}\small
		
	
    \providecommand{\bysame}{\leavevmode\hbox to3em{\hrulefill}\thinspace}
\providecommand{\noopsort}[1]{}
\providecommand{\mr}[1]{\href{http://www.ams.org/mathscinet-getitem?mr=#1}{MR~#1}}
\providecommand{\zbl}[1]{\href{http://www.zentralblatt-math.org/zmath/en/search/?q=an:#1}{Zbl~#1}}
\providecommand{\jfm}[1]{\href{http://www.emis.de/cgi-bin/JFM-item?#1}{JFM~#1}}
\providecommand{\arxiv}[1]{\href{http://www.arxiv.org/abs/#1}{arXiv~#1}}
\providecommand{\doi}[1]{\url{https://doi.org/#1}}
\providecommand{\MR}{\relax\ifhmode\unskip\space\fi MR }
% \MRhref is called by the amsart/book/proc definition of \MR.
\providecommand{\MRhref}[2]{%
  \href{http://www.ams.org/mathscinet-getitem?mr=#1}{#2}
}
\providecommand{\href}[2]{#2}
\begin{thebibliography}{10}

\bibitem{AMS08}
\bgroup\scshape{}R.~Aboulaich\egroup{}, \bgroup\scshape{}D.~Meskine\egroup{},
  and \bgroup\scshape{}A.~Souissi\egroup{}, New diffusion models in image
  processing,  \emph{Comput. Math. Appl.} \textbf{56}
  (2008), 874--882. \doi{10.1016/j.camwa.2008.01.017}.

\bibitem{mumps}
\bgroup\scshape{}P.~R. Amestoy\egroup{}, \bgroup\scshape{}I.~S. Duff\egroup{},
  \bgroup\scshape{}J.~Koster\egroup{}, and \bgroup\scshape{}J.-Y.
  L'Excellent\egroup{}, A fully asynchronous multifrontal solver using
  distributed dynamic scheduling,  \emph{SIAM J. Matrix Anal. Appl.} \textbf{23} no.~1 (2001), 15--41.

\bibitem{AR06}
\bgroup\scshape{}S.~Antontsev\egroup{} and
  \bgroup\scshape{}J.~Rodrigues\egroup{}, On stationary thermo-rheological
  viscous flows,  \emph{ANNALI DELL'UNIVERSITA' DI FERRARA} \textbf{52} (2006),
  19--36. \doi{10.1007/s11565-006-0002-9}.

\bibitem{BalDieSto22}
\bgroup\scshape{}A.~K. Balci\egroup{}, \bgroup\scshape{}L.~Diening\egroup{},
  and \bgroup\scshape{}J.~Storn\egroup{}, Relaxed kacanov scheme for the
  p-laplacian with large p, 2022. \doi{10.48550/ARXIV.2210.06402}.

\bibitem{BalOrtSto22}
\bgroup\scshape{}A.~K. Balci\egroup{}, \bgroup\scshape{}C.~Ortner\egroup{}, and
  \bgroup\scshape{}J.~Storn\egroup{}, Crouzeix-{R}aviart finite element method
  for non-autonomous variational problems with {L}avrentiev gap,  \emph{Numer.
  Math.} \textbf{151} no.~4 (2022), 779--805.
  \doi{10.1007/s00211-022-01303-1}.

\bibitem{BarLiu94}
\bgroup\scshape{}J.~W. Barrett\egroup{} and \bgroup\scshape{}W.~B.
  Liu\egroup{}, Finite element approximation of degenerate quasilinear elliptic
  and parabolic problems,  in \emph{Numerical analysis 1993 ({D}undee, 1993)},
  \emph{Pitman Res. Notes Math. Ser.} \textbf{303}, Longman Sci. Tech., Harlow,
  1994, pp.~1--16.

\bibitem{Bar21}
\bgroup\scshape{}S.~Bartels\egroup{}, Nonconforming discretizations of convex
  minimization problems and precise relations to mixed methods,  \emph{Comput.
  Math. Appl.} \textbf{93} (2021), 214--229. \mr{4253261}.
  \doi{10.1016/j.camwa.2021.04.014}.

\bibitem{BKAFEM22}
\bgroup\scshape{}S.~Bartels\egroup{} and
  \bgroup\scshape{}A.~Kaltenbach\egroup{}, Explicit and efficient error
  estimation for convex minimization problems,  \emph{Math. Comp.} (2023),
  (accepted).


\bibitem{BW21}
\bgroup\scshape{}S.~Bartels\egroup{} and \bgroup\scshape{}Z.~Wang\egroup{},
  Orthogonality relations of {C}rouzeix-{R}aviart and {R}aviart-{T}homas finite
  element spaces,  \emph{Numer. Math.} \textbf{148} no.~1 (2021), 127--139.
   \doi{10.1007/s00211-021-01199-3}.

\bibitem{BK22B}
\bgroup\scshape{}S.~Bartels\egroup{} and
  \bgroup\scshape{}A.~Kaltenbach\egroup{}, Error estimates for total-variation
  regularized minimization problems with singular dual solutions,  \emph{Numer.
  Math.} \textbf{152} no.~4 (2022), 881--906. 
  \doi{10.1007/s00211-022-01324-w}.

\bibitem{BelDieKre12}
\bgroup\scshape{}L.~Belenki\egroup{}, \bgroup\scshape{}L.~Diening\egroup{}, and
  \bgroup\scshape{}C.~Kreuzer\egroup{}, Optimality of an adaptive finite
  element method for the {$p$}-{L}aplacian equation,  \emph{IMA J. Numer.
  Anal.} \textbf{32} no.~2 (2012), 484--510.
  \doi{10.1093/imanum/drr016}.

\bibitem{BBD15}
\bgroup\scshape{}L.~C. Berselli\egroup{}, \bgroup\scshape{}D.~Breit\egroup{},
  and \bgroup\scshape{}L.~Diening\egroup{}, Convergence analysis for a finite
  element approximation of a steady model for electrorheological fluids,
  \emph{Numer. Math.} \textbf{132} no.~4 (2016), 657--689.
  \doi{10.1007/s00211-015-0735-4}.

\bibitem{BDS13}
\bgroup\scshape{}D.~Breit\egroup{}, \bgroup\scshape{}L.~Diening\egroup{}, and
  \bgroup\scshape{}S.~Schwarzacher\egroup{}, Solenoidal {L}ipschitz truncation
  for parabolic {PDE}s,  \emph{Math. Models Methods Appl. Sci.} \textbf{23}
  no.~14 (2013), 2671--2700.  \doi{10.1142/S0218202513500437}.

\bibitem{BDS15}
\bgroup\scshape{}D.~Breit\egroup{}, \bgroup\scshape{}L.~Diening\egroup{}, and
  \bgroup\scshape{}S.~Schwarzacher\egroup{}, Finite element approximation of
  the $p(\cdot)$-laplacian,  \emph{SIAM J. Numer. Anal.}
  \textbf{53} no.~1 (2015), 551--572. \doi{10.1137/130946046}.

\bibitem{Sus96}
\bgroup\scshape{}S.~C. Brenner\egroup{}, Two-level additive schwarz
  preconditioners for nonconforming finite element methods,  \emph{Math.
  Comp.} \textbf{65} no.~215 (1996), 897--921.

\bibitem{Bre15}
\bgroup\scshape{}S.~C. Brenner\egroup{}, Forty years of the
  {C}rouzeix-{R}aviart element,  \emph{Numer. Methods Partial Differ. Equ.} \textbf{31} no.~2 (2015), 367--396.
  \doi{10.1002/num.21892}.

\bibitem{BO09}
\bgroup\scshape{}A.~Buffa\egroup{} and \bgroup\scshape{}C.~Ortner\egroup{},
  Compact embeddings of broken {S}obolev spaces and applications,  \emph{IMA J.
  Numer. Anal.} \textbf{29} no.~4 (2009), 827--855.
  \doi{10.1093/imanum/drn038}.

\bibitem{BE08}
\bgroup\scshape{}E.~Burman\egroup{} and \bgroup\scshape{}A.~Ern\egroup{},
  Discontinuous {G}alerkin approximation with discrete variational principle
  for the nonlinear {L}aplacian,  \emph{C. R. Math. Acad. Sci. Paris}
  \textbf{346} no.~17-18 (2008), 1013--1016.
  \doi{10.1016/j.crma.2008.07.005}.

\bibitem{CHP10}
\bgroup\scshape{}E.~Carelli\egroup{}, \bgroup\scshape{}J.~Haehnle\egroup{}, and
  \bgroup\scshape{}A.~Prohl\egroup{}, Convergence analysis for incompressible
  generalized {N}ewtonian fluid flows with nonstandard anisotropic growth
  conditions,  \emph{SIAM J. Numer. Anal.} \textbf{48} no.~1 (2010), 164--190.
  \doi{10.1137/080718978}.

\bibitem{Car04}
\bgroup\scshape{}C.~Carstensen\egroup{}, An adaptive mesh-refining algorithm
  allowing for an {$H^1$} stable {$L^2$} projection onto {C}ourant finite
  element spaces,  \emph{Constr. Approx.} \textbf{20} no.~4 (2004), 549--564.
  \doi{10.1007/s00365-003-0550-5}.

\bibitem{CLR06}
\bgroup\scshape{}Y.~Chen\egroup{}, \bgroup\scshape{}S.~Levine\egroup{}, and
  \bgroup\scshape{}M.~Rao\egroup{}, Variable exponent, linear growth
  functionals in image restoration,  \emph{SIAM J. Appl. Math.} \textbf{66}
  no.~4 (2006), 1383--1406.  \doi{10.1137/050624522}.

\bibitem{CR73}
\bgroup\scshape{}M.~Crouzeix\egroup{} and \bgroup\scshape{}P.-A.
  Raviart\egroup{}, Conforming and nonconforming finite element methods for
  solving the stationary {S}tokes equations. {I},  \emph{Rev. Fran\c{c}aise
  Automat. Informat. Recherche Op\'{e}rationnelle S\'{e}r. Rouge} \textbf{7}
  no.~{\rm R}-3 (1973), 33--75.

\bibitem{Dac08}
\bgroup\scshape{}B.~Dacorogna\egroup{}, \emph{Direct methods in the calculus of
  variations}, second ed., \emph{Appl. Math. Sci.} \textbf{78},
  Springer, New York, 2008.

\bibitem{DPLM13}
\bgroup\scshape{}L.~M. Del~Pezzo\egroup{}, \bgroup\scshape{}A.~L.
  Lombardi\egroup{}, and \bgroup\scshape{}S.~Mart\'{\i}nez\egroup{}, Interior
  penalty discontinuous galerkin fem for the \$p(x)\$-{L}aplacian,  \emph{SIAM J. Numer. Anal.} \textbf{50} no.~5 (2012), 2497--2521.
  \doi{10.1137/110820324}.

\bibitem{die-ett}
\bgroup\scshape{}L.~Diening\egroup{} and \bgroup\scshape{}F.~Ettwein\egroup{},
  Fractional estimates for non-differentiable elliptic systems with general
  growth,  \emph{Forum Math.} \textbf{20} no.~3 (2008), 523--556.

\bibitem{DieForTomWan20}
\bgroup\scshape{}L.~Diening\egroup{}, \bgroup\scshape{}M.~Fornasier\egroup{},
  \bgroup\scshape{}R.~Tomasi\egroup{}, and \bgroup\scshape{}M.~Wank\egroup{}, A
  relaxed {K}a\v{c}anov iteration for the {$p$}-{P}oisson problem,
  \emph{Numer. Math.} \textbf{145} no.~1 (2020), 1--34.
  \doi{10.1007/s00211-020-01107-1}.

\bibitem{DHHR11}
\bgroup\scshape{}L.~Diening\egroup{}, \bgroup\scshape{}P.~Harjulehto\egroup{},
  \bgroup\scshape{}P.~H\"{a}st\"{o}\egroup{}, and
  \bgroup\scshape{}M.~R\r{u}\v{z}i\v{c}ka\egroup{}, \emph{Lebesgue and
  {S}obolev spaces with variable exponents}, \emph{Lect. Notes Math.} \textbf{2017}, Springer, Heidelberg, 2011.
  \doi{10.1007/978-3-642-18363-8}.

\bibitem{DKRI14}
\bgroup\scshape{}L.~Diening\egroup{}, \bgroup\scshape{}D.~K\"{o}ner\egroup{},
  \bgroup\scshape{}M.~R\r{u}\v{z}i\v{c}ka\egroup{}, and
  \bgroup\scshape{}I.~Toulopoulos\egroup{}, A local discontinuous {G}alerkin
  approximation for systems with {$p$}-structure,  \emph{IMA J. Numer. Anal.}
  \textbf{34} no.~4 (2014), 1447--1488
  \doi{10.1093/imanum/drt040}.

\bibitem{DK08}
\bgroup\scshape{}L.~Diening\egroup{} and \bgroup\scshape{}C.~Kreuzer\egroup{},
  Linear convergence of an adaptive finite element method for the
  {$p$}-{L}aplacian equation,  \emph{SIAM J. Numer. Anal.} \textbf{46} no.~2
  (2008), 614--638.  \doi{10.1137/070681508}.

\bibitem{DieRuz07}
\bgroup\scshape{}L.~Diening\egroup{} and
  \bgroup\scshape{}M.~R\r{u}\v{z}i\v{c}ka\egroup{}, Interpolation operators in
  {O}rlicz-{S}obolev spaces,  \emph{Numer. Math.} \textbf{107} no.~1 (2007),
  107--129.  \doi{10.1007/s00211-007-0079-9}.

\bibitem{DKRT14}
\bgroup\scshape{}L.~Diening\egroup{}, \bgroup\scshape{}D.~K\"{o}ner\egroup{},
  \bgroup\scshape{}M.~R\r{u}\v{z}i\v{c}ka, M.\egroup{}, and
  \bgroup\scshape{}I.~Toulopoulos\egroup{}, A local discontinuous {G}alerkin
  approximation for systems with {$p$}-structure,  \emph{IMA J. Numer. Anal.}
  \textbf{34} no.~4 (2014), 1447--1488.
  \doi{10.1093/imanum/drt040}.

\bibitem{EbmLiu05}
\bgroup\scshape{}C.~Ebmeyer\egroup{} and \bgroup\scshape{}W.~Liu\egroup{},
  Quasi-norm interpolation error estimates for the piecewise linear finite
  element approximation of {$p$}-{L}aplacian problems,  \emph{Numer. Math.}
  \textbf{100} no.~2 (2005), 233--258.
  \doi{10.1007/s00211-005-0594-5}.

\bibitem{winr}
\bgroup\scshape{}W.~Eckart\egroup{} and \bgroup\scshape{}M.~R{\r
  u}\v{z}i\v{c}ka\egroup{}, Modeling micropolar electrorheological fluids,
  \emph{Int. J. Appl. Mech. Eng.} \textbf{11} (2006), {813--844}.

\bibitem{ET99}
\bgroup\scshape{}I.~Ekeland\egroup{} and
  \bgroup\scshape{}R.~T\'{e}mam\egroup{}, \emph{Convex analysis and variational
  problems}, english ed., \emph{Classics Appl. Math.} \textbf{28},
  Society for Industrial and Applied Mathematics (SIAM), Philadelphia, PA,
  1999, Translated from the French.
  \doi{10.1137/1.9781611971088}.

\bibitem{eringenbook}
\bgroup\scshape{}A.~C. Eringen\egroup{}, \emph{Microcontinuum field theories.
  {I,II}.}, Springer-Verlag, New York, 1999.

\bibitem{EG21}
\bgroup\scshape{}A.~Ern\egroup{} and \bgroup\scshape{}J.~L. Guermond\egroup{},
  \emph{Finite Elements I: Approximation and Interpolation}, \emph{Texts in
  Applied Mathematics} no.~1, Springer International Publishing, 2021.
  \doi{10.1007/978-3-030-56341-7}.

  \bibitem{ELM04}
\bgroup\scshape{}L.~Esposito\egroup{}, \bgroup\scshape{}F.~Leonetti\egroup{},
  and \bgroup\scshape{}G.~Mingione\egroup{}, Sharp regularity for functionals
  with (p,q) growth,  \emph{J. Differ. Equ.} \textbf{204} no.~1 (2004), 5--55.
  \doi{https://doi.org/10.1016/j.jde.2003.11.007}.

\bibitem{HMPR10}
\bgroup\scshape{}J.~Hron\egroup{}, \bgroup\scshape{}J.~M\'alek\egroup{},
  \bgroup\scshape{}P.~Pust\v{e}jovsk\'a\egroup{}, and \bgroup\scshape{}K.~R.
  Rajagopal\egroup{}, On the modeling of the synovial fluid,  \emph{Advances in
  Tribology} \textbf{2010} (2010). \doi{10.1155/2010/104957}.

\bibitem{K22CR}
\bgroup\scshape{}A.~Kaltenbach\egroup{}, \emph{Error analysis for a
  Crouzeix--Raviart approximation of the $p$-Dirichlet problem}, submitted, 2023.

\bibitem{kr-phi-ldg}
\bgroup\scshape{}A.~Kaltenbach\egroup{} and
  \bgroup\scshape{}M.~R{\r{u}}{\v{z}}i{\v{c}}ka\egroup{}, \emph{Convergence
  analysis of a {L}ocal {D}iscontinuous {G}alerkin approximation for nonlinear
  systems with {O}rlicz-structure}, submitted, 2022. 

\bibitem{KPS18}
\bgroup\scshape{}S.~Ko\egroup{},
  \bgroup\scshape{}P.~Pust\v{e}jovsk\'{a}\egroup{}, and
  \bgroup\scshape{}E.~S\"{u}li\egroup{}, Finite element approximation of an
  incompressible chemically reacting non-{N}ewtonian fluid,  \emph{ESAIM Math.
  Model. Numer. Anal.} \textbf{52} no.~2 (2018), 509--541.
  \doi{10.1051/m2an/2017043}.

\bibitem{KS19}
\bgroup\scshape{}S.~Ko\egroup{} and \bgroup\scshape{}E.~S\"{u}li\egroup{},
  Finite element approximation of steady flows of generalized {N}ewtonian
  fluids with concentration-dependent power-law index,  \emph{Math. Comp.}
  \textbf{88} no.~317 (2019), 1061--1090.
  \doi{10.1090/mcom/3379}.

\bibitem{LKM78}
\bgroup\scshape{}W.~M. Lai\egroup{}, \bgroup\scshape{}S.~C. Kuei\egroup{}, and
  \bgroup\scshape{}V.~C. Mow\egroup{}, {Rheological Equations for Synovial
  Fluids},  \emph{J. Biomech. Eng.} \textbf{100} no.~4
  (1978), 169--186. \doi{10.1115/1.3426208}.

\bibitem{LLP10}
\bgroup\scshape{}F.~Li\egroup{}, \bgroup\scshape{}Z.~Li\egroup{}, and
  \bgroup\scshape{}L.~Pi\egroup{}, Variable exponent functionals in image
  restoration,  \emph{Appl. Math. Comput.} \textbf{216} no.~3 (2010), 870--882.
   \doi{10.1016/j.amc.2010.01.094}.

\bibitem{LLC18}
\bgroup\scshape{}D.~J. Liu\egroup{}, \bgroup\scshape{}A.~Q. Li\egroup{}, and
  \bgroup\scshape{}Z.~R. Chen\egroup{}, Nonconforming {FEM}s for the
  {$p$}-{L}aplace problem,  \emph{Adv. Appl. Math. Mech.} \textbf{10} no.~6
  (2018), 1365--1383.  \doi{10.4208/aamm}.

\bibitem{LW10}
\bgroup\scshape{}A.~Logg\egroup{} and \bgroup\scshape{}G.~N. Wells\egroup{},
  Dolfin: Automated finite element computing,  \emph{ACM Transactions on
  Mathematical Software} \textbf{37} no.~2 (2010). 
  \doi{10.1145/1731022.1731030}.

\bibitem{Osw93}
\bgroup\scshape{}P.~Oswald\egroup{}, On the robustness of the
  bpx-preconditioner with respect to jumps in the coefficients,
 \emph{Math. Comp.} \textbf{68} no.~226 (1999), 633--650.

\bibitem{RR96}
\bgroup\scshape{}K.~R. Rajagopal\egroup{} and
  \bgroup\scshape{}M.~R\r{u}\v{z}i\v{c}ka\egroup{}, On the modeling of
  electrorheological materials,  \emph{Mech. Res. Commun.}
  \textbf{13} (2001), 59--78. \doi{10.1007/s001610100034}.

\bibitem{RT75}
\bgroup\scshape{}P.-A. Raviart\egroup{} and \bgroup\scshape{}J.~M.
  Thomas\egroup{}, A mixed finite element method for 2nd order elliptic
  problems,  in \emph{Mathematical aspects of finite element methods ({P}roc.
  {C}onf., {C}onsiglio {N}az. delle {R}icerche ({C}.{N}.{R}.), {R}ome, 1975)},
  1977, pp.~292--315. Lecture Notes in Math., Vol. 606. 

\bibitem{Ru00}
\bgroup\scshape{}M.~R\r{u}\v{z}i\v{c}ka\egroup{}, \emph{Electrorheological
  fluids: modeling and mathematical theory}, \emph{Lect. Notes Math.} \textbf{1748}, Springer-Verlag, Berlin, 2000. 
  \doi{10.1007/BFb0104029}.

\bibitem{Ru04}
\bgroup\scshape{}M.~R{\r u}{\v z}i{\v c}ka\egroup{}, \emph{{N}ichtlineare
  {F}unktionalanalysis. {E}ine {E}inf\"uhrung}, Berlin: Springer. xii, 2004
  (German).

\bibitem{SZ90}
\bgroup\scshape{}L.~Scott\egroup{} and \bgroup\scshape{}S.~Zhang\egroup{},
  Finite element interpolation of nonsmooth functions satisfying boundary
  conditions,  \emph{Math. Comp.} \textbf{54}
  (1990), 483--493. \doi{10.1090/S0025-5718-1990-1011446-7}.

\bibitem{ST20}
	\bgroup\scshape{}E.~ S\"{u}li\egroup{} and \bgroup\scshape{} T.~Tscherpel\egroup{},
	Fully discrete finite element approximation of unsteady flows
	of implicitly constituted incompressible fluids,    \emph{IMA J. Numer. Anal.}\textbf{40},(202), 801-849.\doi{10.1093/imanum/dry097}.

\bibitem{Tem77}
\bgroup\scshape{}R.~Temam\egroup{}, \emph{Navier-{S}tokes equations. {T}heory
  and numerical analysis}, North-Holland Publishing Co., Amsterdam-New
  York-Oxford, 1977, Studies in Mathematics and its Applications, Vol. 2.

\bibitem{Zei90B}
\bgroup\scshape{}E.~Zeidler\egroup{}, \emph{Nonlinear functional analysis and
  its applications. {II}/{B} {N}onlinear monotone operators}, Springer-Verlag,
  New York, 1990.  \doi{10.1007/978-1-4612-0985-0}.

\bibitem{Z97}
\bgroup\scshape{}V.~V. Zhikov\egroup{}, Meyer-type estimates for solving the
  nonlinear {S}tokes system,  \emph{Differ. Uravn.} \textbf{33} no.~1 (1997),
  {107--114}.

\end{thebibliography}


	%\bibliographystyle{aomplain}
	%\bibliography{literatur}
	
}

\end{document}




   