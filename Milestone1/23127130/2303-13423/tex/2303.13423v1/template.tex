%  LaTeX support: latex@mdpi.com 
%  For support, please attach all files needed for compiling as well as the log file, and specify your operating system, LaTeX version, and LaTeX editor.

%=================================================================
%\documentclass[journal,article,submit,pdftex,moreauthors]{Definitions/mdpi} 
\documentclass[preprints,article,accept,moreauthors,pdftex]{Definitions/mdpi}
% For posting an early version of this manuscript as a preprint, you may use "preprints" as the journal and change "submit" to "accept". The document class line would be, e.g., \documentclass[preprints,article,accept,moreauthors,pdftex]{mdpi}. This is especially recommended for submission to arXiv, where line numbers should be removed before posting. For preprints.org, the editorial staff will make this change immediately prior to posting.

%--------------------
% Class Options:
%--------------------
%----------
% journal
%----------
% Choose between the following MDPI journals:
% acoustics, actuators, addictions, admsci, adolescents, aerospace, agriculture, agriengineering, agronomy, ai, algorithms, allergies, alloys, analytica, animals, antibiotics, antibodies, antioxidants, applbiosci, appliedchem, appliedmath, applmech, applmicrobiol, applnano, applsci, aquacj, architecture, arts, asc, asi, astronomy, atmosphere, atoms, audiolres, automation, axioms, bacteria, batteries, bdcc, behavsci, beverages, biochem, bioengineering, biologics, biology, biomass, biomechanics, biomed, biomedicines, biomedinformatics, biomimetics, biomolecules, biophysica, biosensors, biotech, birds, bloods, blsf, brainsci, breath, buildings, businesses, cancers, carbon, cardiogenetics, catalysts, cells, ceramics, challenges, chemengineering, chemistry, chemosensors, chemproc, children, chips, cimb, civileng, cleantechnol, climate, clinpract, clockssleep, cmd, coasts, coatings, colloids, colorants, commodities, compounds, computation, computers, condensedmatter, conservation, constrmater, cosmetics, covid, crops, cryptography, crystals, csmf, ctn, curroncol, currophthalmol, cyber, dairy, data, dentistry, dermato, dermatopathology, designs, diabetology, diagnostics, dietetics, digital, disabilities, diseases, diversity, dna, drones, dynamics, earth, ebj, ecologies, econometrics, economies, education, ejihpe, electricity, electrochem, electronicmat, electronics, encyclopedia, endocrines, energies, eng, engproc, ent, entomology, entropy, environments, environsciproc, epidemiologia, epigenomes, est, fermentation, fibers, fintech, fire, fishes, fluids, foods, forecasting, forensicsci, forests, foundations, fractalfract, fuels, futureinternet, futureparasites, futurepharmacol, futurephys, futuretransp, galaxies, games, gases, gastroent, gastrointestdisord, gels, genealogy, genes, geographies, geohazards, geomatics, geosciences, geotechnics, geriatrics, hazardousmatters, healthcare, hearts, hemato, heritage, highthroughput, histories, horticulturae, humanities, humans, hydrobiology, hydrogen, hydrology, hygiene, idr, ijerph, ijfs, ijgi, ijms, ijns, ijtm, ijtpp, immuno, informatics, information, infrastructures, inorganics, insects, instruments, inventions, iot, j, jal, jcdd, jcm, jcp, jcs, jdb, jeta, jfb, jfmk, jimaging, jintelligence, jlpea, jmmp, jmp, jmse, jne, jnt, jof, joitmc, jor, journalmedia, jox, jpm, jrfm, jsan, jtaer, jzbg, kidney, kidneydial, knowledge, land, languages, laws, life, liquids, literature, livers, logics, logistics, lubricants, lymphatics, machines, macromol, magnetism, magnetochemistry, make, marinedrugs, materials, materproc, mathematics, mca, measurements, medicina, medicines, medsci, membranes, merits, metabolites, metals, meteorology, methane, metrology, micro, microarrays, microbiolres, micromachines, microorganisms, microplastics, minerals, mining, modelling, molbank, molecules, mps, msf, mti, muscles, nanoenergyadv, nanomanufacturing, nanomaterials, ncrna, network, neuroglia, neurolint, neurosci, nitrogen, notspecified, nri, nursrep, nutraceuticals, nutrients, obesities, oceans, ohbm, onco, oncopathology, optics, oral, organics, organoids, osteology, oxygen, parasites, parasitologia, particles, pathogens, pathophysiology, pediatrrep, pharmaceuticals, pharmaceutics, pharmacoepidemiology, pharmacy, philosophies, photochem, photonics, phycology, physchem, physics, physiologia, plants, plasma, pollutants, polymers, polysaccharides, poultry, powders, preprints, proceedings, processes, prosthesis, proteomes, psf, psych, psychiatryint, psychoactives, publications, quantumrep, quaternary, qubs, radiation, reactions, recycling, regeneration, religions, remotesensing, reports, reprodmed, resources, rheumato, risks, robotics, ruminants, safety, sci, scipharm, seeds, sensors, separations, sexes, signals, sinusitis, skins, smartcities, sna, societies, socsci, software, soilsystems, solar, solids, sports, standards, stats, stresses, surfaces, surgeries, suschem, sustainability, symmetry, synbio, systems, taxonomy, technologies, telecom, test, textiles, thalassrep, thermo, tomography, tourismhosp, toxics, toxins, transplantology, transportation, traumacare, traumas, tropicalmed, universe, urbansci, uro, vaccines, vehicles, venereology, vetsci, vibration, viruses, vision, waste, water, wem, wevj, wind, women, world, youth, zoonoticdis 

%---------
% article
%---------
% The default type of manuscript is "article", but can be replaced by: 
% abstract, addendum, article, book, bookreview, briefreport, casereport, comment, commentary, communication, conferenceproceedings, correction, conferencereport, entry, expressionofconcern, extendedabstract, datadescriptor, editorial, essay, erratum, hypothesis, interestingimage, obituary, opinion, projectreport, reply, retraction, review, perspective, protocol, shortnote, studyprotocol, systematicreview, supfile, technicalnote, viewpoint, guidelines, registeredreport, tutorial
% supfile = supplementary materials

%----------
% submit
%----------
% The class option "submit" will be changed to "accept" by the Editorial Office when the paper is accepted. This will only make changes to the frontpage (e.g., the logo of the journal will get visible), the headings, and the copyright information. Also, line numbering will be removed. Journal info and pagination for accepted papers will also be assigned by the Editorial Office.

%------------------
% moreauthors
%------------------
% If there is only one author the class option oneauthor should be used. Otherwise use the class option moreauthors.

%---------
% pdftex
%---------
% The option pdftex is for use with pdfLaTeX. If eps figures are used, remove the option pdftex and use LaTeX and dvi2pdf.

%=================================================================
% MDPI internal commands
\firstpage{1} 
\makeatletter 
\setcounter{page}{\@firstpage} 
\makeatother
\pubvolume{1}
\issuenum{1}
\articlenumber{0}
\pubyear{2023}
\copyrightyear{2023}
%\externaleditor{Academic Editor: Firstname Lastname}
\datereceived{} 
%\daterevised{} % Only for the journal Acoustics
\dateaccepted{} 
\datepublished{} 
%\datecorrected{} % Corrected papers include a "Corrected: XXX" date in the original paper.
%\dateretracted{} % Corrected papers include a "Retracted: XXX" date in the original paper.
\hreflink{https://doi.org/} % If needed use \linebreak
%\doinum{}
%------------------------------------------------------------------
% The following line should be uncommented if the LaTeX file is uploaded to arXiv.org
%\pdfoutput=1

%=================================================================
% Add packages and commands here. The following packages are loaded in our class file: fontenc, inputenc, calc, indentfirst, fancyhdr, graphicx, epstopdf, lastpage, ifthen, lineno, float, amsmath, setspace, enumitem, mathpazo, booktabs, titlesec, etoolbox, tabto, xcolor, soul, multirow, microtype, tikz, totcount, changepage, attrib, upgreek, cleveref, amsthm, hyphenat, natbib, hyperref, footmisc, url, geometry, newfloat, caption

%=================================================================
%% Please use the following mathematics environments: Theorem, Lemma, Corollary, Proposition, Characterization, Property, Problem, Example, ExamplesandDefinitions, Hypothesis, Remark, Definition, Notation, Assumption
%% For proofs, please use the proof environment (the amsthm package is loaded by the MDPI class).

%=================================================================
\usepackage{url}
% Full title of the paper (Capitalized)
\Title{CE$\nu$NS Experiment Proposal at CSNS}

% MDPI internal command: Title for citation in the left column
\TitleCitation{CE$\nu$NS Experiment Proposal at CSNS}

% Author Orchid ID: enter ID or remove command
%\newcommand{\orcidauthorA}{0000-0000-0000-000X} % Add \orcidA{} behind the author's name
%\newcommand{\orcidauthorB}{0000-0000-0000-000X} % Add \orcidB{} behind the author's name

% Authors, for the paper (add full first names)
\Author{Chenguang Su$^{1,\dagger}$, Qian Liu$^{1,*}$ and Tianjiao Liang$^{2,3}$ on be half of the CE$\nu$NS@CSNS Collaboration}

%\longauthorlist{yes}

% MDPI internal command: Authors, for metadata in PDF
\AuthorNames{Chenguang Su, Qian Liu and Tianjiao Liang on behalf of the CE$\nu$NS@CSNS Collaboration}

% MDPI internal command: Authors, for citation in the left column
\AuthorCitation{Chenguang Su, Qian Liu and Tianjiao Liang on be half of the CE$\nu$NS@CSNS Collaboration.}
% If this is a Chicago style journal: Lastname, Firstname, Firstname Lastname, and Firstname Lastname.

% Affiliations / Addresses (Add [1] after \address if there is only one affiliation.)
\address{%
$^{1}$ \quad School of Physics Science, University of Chinese Academy of Sciences, Beijing, 100049, China,\\
$^{2}$ \quad Institute of High Energy Physics, Chinese Academy of Sciences, Beijing, 100049, China\\
$^{3}$ \quad Dongguan Institute of Neutron Science, Dongguan, Guangdong, 523808, China\\
}
%$^{2}$ \quad Affiliation 2; e-mail@e-mail.com}

% Contact information of the corresponding author
\corres{Correspondence: liuqian@ucas.ac.cn}

% Current address and/or shared authorship
\firstnote{Speaker} 
%\secondnote{These authors contributed equally to this work.}
% The commands \thirdnote{} till \eighthnote{} are available for further notes

%\simplesumm{} % Simple summary

%\conference{} % An extended version of a conference paper

% Abstract (Do not insert blank lines, i.e. \\) 
\abstract{The detection and cross section measurement of Coherent Elastic Neutrino-Nucleus Scattering (CE$\nu$NS) is vital for particle physics, astrophysics and nuclear physics. Therefore, a new CE$\nu$NS detection experiment is proposed in China. Undoped CsI crystals coupled with two Photon Multiplier Tubes (PMTs) each, will be cooled down to 77K and placed at China Spallation Neutron Source (CSNS) to detect the CE$\nu$NS signals produced by neutrinos from stopped pion decays happening within the Tungsten target of CSNS. Owing to the extremely high light yield of pure CsI at 77K, even though only having a neutrino flux 60\% weaker than COHERENT, the detectable signal event rate is still expected to be $0.14/day/kg$. Low radioactivity materials and devices will be used to construct the detector and strong shielding will be applied to reduce the radioactive and neutron background. Dual-PMT readout should be able to reject PMT-related background like Cherenkov light and PMT dark noise. With all the strategies above, we are hoping to reach a 5.1$\sigma$ signal detection significance by a half-year data taking with a $12kg$ CsI. In this presentation, the design of the experiment will be presented. In addition, the estimation of signal, various kinds of background and expected signal sensitivity will be discussed.}

%In 2017, the COHERNET collaboration reported the first observation of CE$\nu$NS signal.

% Keywords
\keyword{neutrino scattering physics; neutrino detectors; spallation neutron source} 

% The fields PACS, MSC, and JEL may be left empty or commented out if not applicable
%\PACS{J0101}
%\MSC{}
%\JEL{}

%%%%%%%%%%%%%%%%%%%%%%%%%%%%%%%%%%%%%%%%%%
% Only for the journal Diversity
%\LSID{\url{http://}}

%%%%%%%%%%%%%%%%%%%%%%%%%%%%%%%%%%%%%%%%%%
% Only for the journal Applied Sciences
%\featuredapplication{Authors are encouraged to provide a concise description of the specific application or a potential application of the work. This section is not mandatory.}
%%%%%%%%%%%%%%%%%%%%%%%%%%%%%%%%%%%%%%%%%%

%%%%%%%%%%%%%%%%%%%%%%%%%%%%%%%%%%%%%%%%%%
% Only for the journal Data
%\dataset{DOI number or link to the deposited data set if the data set is published separately. If the data set shall be published as a supplement to this paper, this field will be filled by the journal editors. In this case, please submit the data set as a supplement.}
%\datasetlicense{License under which the data set is made available (CC0, CC-BY, CC-BY-SA, CC-BY-NC, etc.)}

%%%%%%%%%%%%%%%%%%%%%%%%%%%%%%%%%%%%%%%%%%
% Only for the journal Toxins
%\keycontribution{The breakthroughs or highlights of the manuscript. Authors can write one or two sentences to describe the most important part of the paper.}

%%%%%%%%%%%%%%%%%%%%%%%%%%%%%%%%%%%%%%%%%%
% Only for the journal Encyclopedia
%\encyclopediadef{For entry manuscripts only: please provide a brief overview of the entry title instead of an abstract.}

%%%%%%%%%%%%%%%%%%%%%%%%%%%%%%%%%%%%%%%%%%
% Only for the journal Advances in Respiratory Medicine
%\addhighlights{yes}
%\renewcommand{\addhighlights}{%

%\noindent This is an obligatory section in “Advances in Respiratory Medicine”, whose goal is to increase the discoverability and readability of the article via search engines and other scholars. Highlights should not be a copy of the abstract, but a simple text allowing the reader to quickly and simplified find out what the article is about and what can be cited from it. Each of these parts should be devoted up to 2~bullet points.\vspace{3pt}\\
%\textbf{What are the main findings?}
% \begin{itemize}[labelsep=2.5mm,topsep=-3pt]
% \item First bullet.
% \item Second bullet.
% \end{itemize}\vspace{3pt}
%\textbf{What is the implication of the main finding?}
% \begin{itemize}[labelsep=2.5mm,topsep=-3pt]
% \item First bullet.
% \item Second bullet.
% \end{itemize}
%}

%%%%%%%%%%%%%%%%%%%%%%%%%%%%%%%%%%%%%%%%%%
\begin{document}

%%%%%%%%%%%%%%%%%%%%%%%%%%%%%%%%%%%%%%%%%%
%\setcounter{section}{-1} %% Remove this when starting to work on the template.
%\section{How to Use this Template}

%The template details the sections that can be used in a manuscript. Note that the order and names of article sections may differ from the requirements of the journal (e.g., the positioning of the Materials and Methods section). Please check the instructions on the authors' page of the journal to verify the correct order and names. For any questions, please contact the editorial office of the journal or support@mdpi.com. For LaTeX-related questions please contact latex@mdpi.com.%\endnote{This is an endnote.} % To use endnotes, please un-comment \printendnotes below (before References). Only journal Laws uses \footnote.

% The order of the section titles is different for some journals. Please refer to the "Instructions for Authors” on the journal homepage.

\section{Introduction}

In general, the non-trivial interplay between neutrino and individual nucleons and the complex structure of the nucleus make the precise understanding of the interaction between neutrino and nucleus difficult. However, when the momentum transfer between neutrino and nucleus is small enough, which means the scattering would be elastic and the de Broglie wavelength corresponding to the transferred momentum is much larger than the scale of nucleus, from the point of view of the neutrino, the nucleons inside the nucleus hold almost the same position. Therefore, when calculating the cross section, the phases of scattering amplitudes contributed by different nucleons are almost the same, leading to a coherent enhancement of the cross section. This is so called coherent elastic neutrino-nucleus scattering (CE$\nu$NS) process.

The measurement of CE$\nu$NS signal would benefit varieties of aspects of physics. For particle physics, it provides a good way to measure the Weak-Mixing Angle in low momentum transfer circumstance and another inspection of the Standard Model. For dark matter searching, a good knowledge of CE$\nu$NS signal helps the searching of WIMP dark matter candidate in which the CE$\nu$NS signals, as an important background, are hard to be distinguished from real dark matter signals. For astrophysics, it points out a new way to detect solar neutrinos and supernova neutrinos as their energy perfectly meet the coherent criterion. The accurate measurement of the cross section of CE$\nu$NS process also helps to understand the outburst of core-collapsed supernova whose energy are released mainly through neutrinos with energy of tens of $MeV$ \cite{ott2013core}. For nuclear physics, it can be used to measure the form factor of the nucleus when the exchanged momentum is large enough to introduce some non-coherent effect due to the structure of nucleus but also small enough to keep the scattering elastic. In addition, CE$\nu$EN process has no threshold limit. Hence, reactor neutrino spectrum under the 1.8MeV threshold of IBD process can be measured by CE$\nu$NS process and the experiment result could be a good examination of nuclear physics models.

Since the measurement of CE$\nu$NS process is vital for different fields of physics, many scientists has been devoting into this area since the first theoretical prediction was published by Daniel Z. Freedman in 1974 \cite{freedman1974coherent}. Owing to the coherent enhancement, its cross section is approximately proportional to the square of the neutron number in the nucleus \cite{drukier1984principles} making it much larger than any other neutrino-matter interactions. However, the signal produced by the recoiled nucleus is so weak that the CE$\nu$NS signal had not been observed until 2017 by COHERENT collaboration using $\pi^{+}$ decay-at-rest neutrinos from the Spallation Neutron Source (SNS) in Oak Ridge \cite{akimov2017observation}. There are also many other groups trying to detect CE$\nu$NS signal by reactor neutrinos with various kinds of technologies applied. For instance, cryogenic superconductor calorimeter is selected by NUCLEUS in France \cite{angloher2019exploring} while a astronomical CCD is applied by CONNIE in Mexico \cite{aguilar2019exploring}. 

Despite much effort has been put into the searching of CE$\nu$NS signal, the independent CE$\nu$NS signal detection verification still remains a blank to fill. Here we propose a CE$\nu$NS experiment at China Spallation Neutron Source (CSNS) which provides neutrinos with almost the same spectrum as SNS in Oak Ridge. The experiment design and the estimation of signal and background are discussed in section \ref{Exp_Deg} and section \ref{Sig_Bkg_Esti}. A sensitivity estimation and the experiment schedule are presented in section \ref{Sens} and section \ref{Sched}.

%The introduction should briefly place the study in a broad context and highlight why it is important. It should define the purpose of the work and its significance. The current state of the research field should be reviewed carefully and key publications cited. Please highlight controversial and diverging hypotheses when necessary. Finally, briefly mention the main aim of the work and highlight the principal conclusions. As far as possible, please keep the introduction comprehensible to scientists outside your particular field of research. Citing a journal paper \cite{solinhac2004fabrication}. Now citing a book reference \cite{solinhac2004fabrication,lintereur2012neutron} or other reference types. Please use the command \citep{lazauskas2007neutrino,Lee2016KIMS} for the following MDPI journals, which use author--date citation: Administrative Sciences, Arts, Econometrics, Economies, Genealogy, Humanities, IJFS, Journal of Intelligence, Journalism and Media, JRFM, Languages, Laws, Religions, Risks, Social Sciences, Literature.
%%%%%%%%%%%%%%%%%%%%%%%%%%%%%%%%%%%%%%%%%%

\section{Experiment Design}\label{Exp_Deg}

The common difficulties faced by neutrino detection experiments are low cross section, weak signal and high background. The CE$\nu$NS experiment shares the last two of them while the first one is relieved by the coherent enhancement of cross section. The observable energy generated by recoiled nucleus is only several $keV$, which requires threshold of the detector to be very low. Since the signals are weak, background must be strongly suppressed to make sure the signals would not be overwhelmed. Therefore, an optimized shielding structure is a must. The following part of this section describes our selection of neutrino source and our design of detector and shielding structure.

\subsection{Selection of neutrino source}\label{Neu_Soc}

%The measurement of CE$\nu$NS signal requires the neutrino energy to be lower than 100$MeV$ to meet the coherent criterion. There exit two types of neutrino sources satisfying this demand in the present time: reactors and spallation neutron sources. Reactors generate $\bar{\nu_{e}}$ via the fission and decay process of isotopes. While in spallation neutron sources, $\nu_{\mu}$, $\bar{\nu_{\mu}}$ and $\nu_{e}$ are generated by the decay of stopped $\pi^{+}$ generated by protons hitting target. The total neutrino flux generated by a $4GW$ reactor surpasses that of a spallation neutron source with $1MW$ proton beam power for magnitudes. However, pulsed proton beams in spallation neutron source produce pulsed neutrinos which are highly concentrated in a couple of microseconds after the moment when proton impinging the target. The disadvantage of weaker neutrino flux in spallation neutron source could be balanced out by this feature since it can be used to suppress the steady-state background for magnitudes. Meanwhile, the energy spectrum of reactor neutrinos is below $10MeV$. Low neutrino energy leads to low recoil energy and observable energy. Taking reactors as neutrino source requires the threshold of detector to be as low as $0.1keV$ recoil energy which is very difficult to achieve by now. While the energy spectrum of neutrinos from spallation neutron souces extends mainly between $20-50MeV$, relieving the threshold to around 1$keV$ recoil energy which is hopefully reachable by the detector design presented in section \ref{Det_Deg}. Fig. \ref{fig_Nu_Spec} shows the energy spectrum of neutrinos generated by reactors (red) and spallation neutron source (blue) \cite{Avignone2002kc, riyana2019calculation}. These reasons above indicate that in present time, using spallatino neutron source as neutrino source seems to be a more reasonable choice.

%Therefore, the China Spallation Neutron Source (CSNS) is selected as our neutrino source. It is located in Dongguan, Guangdong province in China. In CSNS, a beam of protons is accelerated to $1.6GeV$ and impinges on a Tungsten target with a repetition rate of $25Hz$. It is running with a beam power of $140kW$ for now. According to a simulation implemented by FLUKA, the neutrino production rate is about $0.17/proton/flavor$ in CSNS \cite{HuangMY2016}. Our detector is to be placed on a platform $8.2m$ right above the target. Considering the thickness of shielding structure and detector encapsulation to be $2.3m$, the neutrino flux of the detector location is calculated to be $2.42 \times 10^{10} /cm^{2}/h$. Fig. This is about 40\% of the flux in COHERENT \cite{scholz2018first}. Fig. \ref{fig_Loc} shows the scene picture of the platform (left) and its relative position in CSNS (right).

The China Spallation Neutron Source (CSNS) is selected as our neutrino source. It is located in Dongguan, Guangdong province in China. In CSNS, a beam of protons is accelerated to $1.6GeV$ and impinges on a Tungsten target with a repetition rate of $25Hz$. $\nu_{\mu}$, $\bar{\nu_{\mu}}$ and $\nu_{e}$ are generated by the decay of target-stopped $\pi^{+}$ generated by the impingement. Thus, the neutrinos are highly pulsed which is very beneficial for the suppression of the background evenly distributed in time. The energy of neutrinos from $\pi^{+}$ decay-at-rest expends mainly between $20-50MeV$, about one magnitude higher than the reactor neutrinos (Fig. \ref{fig_Nu_Spec}), making the detection of CE$\nu$NS signal much easier. 

CSNS is running with a beam power of $140kW$ now. According to a simulation implemented by FLUKA, the neutrino production rate is about $0.17/proton/flavor$ in CSNS \cite{HuangMY2016}. Our detector is to be placed on a platform $8.2m$ right above the target. Considering the thickness of shielding structure and detector encapsulation to be $2.3m$, the neutrino flux of the detector location is calculated to be $2.42 \times 10^{10} /cm^{2}/h$. This is about 40\% of the flux in COHERENT \cite{scholz2018first}. Fig. \ref{fig_Loc} shows the scene picture of the platform (left) and its relative position in CSNS (right).

\begin{figure}[H]
\centering
\includegraphics[width=10.5 cm]{Graphs/Nu_Eng_Spec.png}
\caption{Neutrino energy spectra of reactors (\textbf{red}) and spallation neutron source (\textbf{blue}). The energy neutrinos from spallation neutron  is significantly higher reactor neutrinos \cite{Avignone2002kc, riyana2019calculation}. \label{fig_Nu_Spec}}
\end{figure}   
%\unskip

\begin{figure}[H]
\centering
\includegraphics[width=12 cm]{Graphs/Loc.png}
\caption{Scene picture of the platform (\textbf{left}) and its relative postion in CSNS (\textbf{right}). The platfrom is $8.2m$ right above the Tungsten target. \label{fig_Loc}}
\end{figure}   
%\unskip

\subsection{Detector design}\label{Det_Deg}

To achieve a threshold of around $1keV$ recoil energy, a detector as shown in Fig. \ref{fig_DetSche} is designed. The detector would contain several sub-detectors held in a big Dewar. Each sub-detector is composed of a PTFE fabrication shell, two R11065 Hamamatsu photomultiplier tubes (PMTs) and one $3kg$ undoped cesium iodide (CsI) crystal. The Dewar will be filled with liquid nitrogen to immerse the sub-detectors to provide a stable cryogenic temperature of $77K$. Even though the light yield of undoped CsI is much lower than CsI(Na) and CsI(Tl) in room temperature \cite{AMSLER2002494, Woody106667}, it would increase more than 15 times when cooled down to $77K$ \cite{AMSLER2002494, Mikhailik_CsI}. When coupled with R11065 PMTs, the light yield of undoped CsI was reported to reach $33.5PE/keV_{ee}$ \cite{Ding2020uxu}, more than twice of that of CsI(Na) in room temperature measured by COHERENT collaboration \cite{scholz2018first}. High light yield enable us to lower the threshold.

The two PMTs in each sub-detector form a coincidence system to reject backgrounds generated by the PMT dark count background including electron emission on cathode or dynodes and Cherenkov light generated by charged particles passing through PMT window. This background dominate in COHERENT CsI(Na) experiment \cite{scholz2018first} in a few photon electrons (PE) region. Since the PMT dark count background is independent from one PMT to another, this background can be suppressed by 3 magnitudes by applying a two PMT readout coincidence system requiring at least one photon electron is detected in each PMT. The suppression effect is discussed in detail in section \ref{est_bkg}.

The four sub-detectors also form an anti-coincidence system. The probability of one particle producing signals in more than one sub-detectors is negligible for neutrinos but much larger for fast neutrons and $\gamma$ rays. Events in which scintillation signals are observed in more than one sub-detectors would be regarded as background. This strategy can strongly reduce the fast neutron background. Details of the effect of this strategy is in section \ref{est_bkg}.

\begin{figure}[H]
\centering
\includegraphics[width=10.5 cm]{Graphs/Detector.png}
\caption{A schematic of the detector. The detector contains four sub-detectors in a Dewar filled with liquid nitrogen. Each sub-detector is composed of one $3kg$ undoped CsI and two R11065 PMTs. \label{fig_DetSche}}
\end{figure}   

\subsection{Shielding structure}\label{Shielding}

%The main background expected on the platform are fast neutrons leaking from the Tungsten target as the CSNS was built to produce lots of neutrons. These fast neutrons produce similar nuclear recoil signals as neutrinos in undpoed CsI detectors and they are can not be suppressed by the proton beam trigger as they are beam related. Therefore, the shield of neutrons would be vital for this experiment.

In order to achieve a $5\sigma$ detection of CE$\nu$NS signal in one year or so, the event rate of background needs to be reduced to the same magnitude of CE$\nu$NS signal. A preliminary shielding structure as shown in Fig. \ref{fig_Shield_Sche} is designed to achieve this goal. Inside the Dewar (grey), four sub-detectors are surrounded by $5cm$ of OHFC (oxygen-free high-conductivity copper) to shield the radioactive background produced by the stainless steel in Dewar and inner layer shielding materials. Outside the Dewar is a $30cm$ thick layer of HDPE (high density polyethylene), followed by $60cm$ of Lead. The Lead shield aims at reducing $\gamma$ ray background while the innermost layer of HDPE slows down and stops fast neutrons produced by high energy neutrons (>50MeV) reacting with lead nuclei. The Lead shield is encased by a 5cm thick $\mu$ veto plastic scintillator to tag comic ray events. The outermost layer is $80cm$ HDPE serves as a strong moderator to fast neutrons. The total thickness of HDPE reaches 1.1m in this design because the fast neutrons escaping from the Tungsten target are expected to be the main background on the platform. Detailed discussion of the fast neutron background is demonstrated in section \ref{est_bkg}. 

This shielding structure is just a preliminary one. As the on-site background measurement on the platform in CSNS is undertaking, this design will be adjusted and optimized according to our measurement result.

%As CSNS was built to produce lots of neutrons, 
%beam related pulsed fast neutrons which produce similar nuclear recoil signals like neutrinos would be our main background. So the shielding structure contains lots of HDPE (high density polyethylene) with a total thickness of $1.1m$ to slow down and stop the fast neutrons. 

\subsection{Data taking strategy and event selection}

A data taking strategy is proposed with characteristics of neutrino source and detector taken into consideration. It is stated as follows:

\begin{enumerate}
\item	The $25Hz$ proton beam trigger signal provided by CSNS would be taken as an external trigger for the experiment data taking, suppressing the steady-state background like cosmic ray and environmental background by 4 magnitudes. Each trigger corresponds to one event referred latter.
\item	A complete waveform signal of each PMT would be recorded by a flash ADC with sampling rate of $1GHz$. Each waveform would extends $50\mu s$ with $10\mu s$ signal region and $40\mu s$ pretrace. Offline waveform analysis would be applied to extract CE$\nu$NS signal candidates.
\item	Every event is recorded with a time tag and a $\mu$ veto tag. By referring the the time tag to the beam power monitor of CSNS, the proton beam power fluctuation can be neutralized. And the $\mu$ veto tag signal from the $\mu$ veto system rejects events possibly contaminated by cosmic rays.
\end{enumerate}

\begin{figure}[H]
\centering
\includegraphics[width=10.5 cm]{Graphs/Shielding.png}
\caption{A schematic of the preliminary shielding structure design. The shielding components from the inside out are as follows. (\textbf{1}) Yellow green: $5cm$ OHFC. (\textbf{2}) Grey: Dewar. (\textbf{3}) Yellow: $30cm$ HDPE. (\textbf{4}) Green: $60cm$ Lead. (\textbf{5}) Red: $5cm$ $\mu$ veto plastic scintillator. (\textbf{6}) Blue: $80cm$ HDPE. \label{fig_Shield_Sche}}
\end{figure}  

Although the $25Hz$ proton beam trigger can reduce the steady-state background by 4 magnitudes, some event selection criterion are still needed to further reduce the background to meet the standard mentioned in section \ref{Shielding}. They are listed as follows.
\begin{enumerate}
\item	The event is not tagged by $\mu$ veto system to reject events possibly contaminated by cosmic ray.
\item	For each waveform, the PE number found in pretrace should be smaller than 3 to suppress the afterglow background introduced by other particles hitting CsI detector just a few microseconds before the trigger.
\item	For each sub-detector, at least one PE should be detected in both PMTs. The purpose of this criteria is to reduce the PMT dark count background which is independent from one PMT to another.
\item   For the whole detector system, events with more than one sub-detectors satisfying criteria 3 would be excluded since neutrons and $\gamma$ rays are much more likely to produce signals in different sub-detectors.
\end{enumerate}

%\item   The number of photon electrons (NPE) reconstructed for each event should fall within a signal region selected to maximize the signal to background ratio. According to the research in section \ref{Sig_Bkg_Esti}, this region is selected to be [4,72] NPE for now.

\begin{figure}[H]
\centering
\includegraphics[width=10.5 cm]{Graphs/Eff.png}
\caption{Efficiency of event selection criterion. \textbf{Green}: Efficiency of $\mu$ veto cut\cite{scholz2018first}. \textbf{Blue}: Efficiency of afterglow cut \cite{scholz2018first}. \textbf{Black}: Total Efficiency. \label{fig_Eff}}
\end{figure}

The selection efficiency of all criterion to CE$\nu$NS signal are also investigated. The efficiencies of the first two criterion needs to be determined on site with the whole detector and shield constructed which is not ready yet. Hence the efficiencies of similar cuts applied by COHERENT collaboration are adopted in the sensitivity estimation in section \ref{Sens} \cite{scholz2018first}. The efficiency  is $98.9\%$ for criteria 1 and $73.8\%$ for criteria 2. Considering the cosmic ray level and other kinds of steady-state background responsible for most afterglow events should be similar in CSNS and SNS while the afterglow of undoped CsI at $77K$ is much weaker than CsI(Na) at room temperature, this estimation of efficiencies of the first two criterion should be conservative totally.

The efficiency of criteria 3 is considered by assuming the probabilities of  scintillation photon detected by two PMTs are equal in an average sense which should be reasonable enough considering the detector is symmetric longitudinally. An analytical calculation based on binomial distribution has been carried out to calculate the selection efficiency of scintillation signal with different total NPE. The efficiency of criteria 4 is considered as $100\%$ since the possibility of one neutrino producing signals in different sub-detectors is negligible. 

Fig. \ref{fig_Eff} shows the estimated CE$\nu$NS signal selection efficiency with respect to different numbers of detected NPE. The green line and bule line show the efficiencies of $\mu$ veto cut and afterglow cut taken from \cite{scholz2018first}. They are both overall cuts holding the same value for all events. The black line shows the total efficiency of all criterion by multiplying efficiency of each cut together. Criteria 3 gives the rising shape of the curve and its efficiency is equal to all scintillation events including signals generated by particles depositing energy in CsI.

%%%%%%%%%%%%%%%%%%%%%%%%%%%%%%%%%%%%%%%%%%
\section{Estimation of CE$\nu$NS signal and background}\label{Sig_Bkg_Esti}

With the experiment design described above, the estimation of CE$\nu$NS signal and background in this experiment can be implemented. In this part, the detector is assumed to be composed of four $3kg$ undoped CsI sub-detectors, totaly $12kg$ of undoped CsI and the data taking time is taken as half an year. Based on the results of this part, the expected sensitivity of this experiment can be obtained.

\subsection{Estimation of CE$\nu$NS signal}\label{est_sig}

The estimation of CE$\nu$NS signal is composed of two steps. First, the CE$\nu$NS recoiled energy distribution of nuclear recoils induced by neutrinos should be considered. Second, the energy response of the detector to recoiled nuclei should be taken into account to acquire an expected spectrum of NPE detected by PMTs. 

The expected CE$\nu$NS recoiled energy distribution can be calculated numerically when the neutrino flux, detector mass and CE$\nu$NS differential cross section \cite{drukier1984principles} are considered.  Fig. \ref{fig_CEvNS_Recoil} shows the calculated result.  Total event rate distribution as well as the event rates of three different neutrinos generated in CSNS are shown. The total CE$\nu$NS events rate reaches $303/half year/12kg$, equivalently $0.14/day/kg$.

\begin{figure}[H]
\centering
\includegraphics[width=10.5 cm]{Graphs/CEvNS_Recoil.png}
\caption{Expected recoiled energy distribution of CE$\nu$NS interaction detected by a $12kg$ cryogenic undoped CsI detector $10.5m$ away from the Tungsten target with a half-year data taking. The contribution from different flavors of neutrinos and different isotopes are also shown. \label{fig_CEvNS_Recoil}}
\end{figure}

\begin{figure}[H]
\centering
\includegraphics[width=10.5 cm]{Graphs/CEvNS_NPE.png}
\caption{Expected NPE spectra of CE$\nu$NS signal. Contribution from different flavors of neutrinos are also shown. \label{fig_CEvNS_NPE}}
\end{figure}

The energy response of the detector to recoiled nuclei also includes two steps. First, the light yield with respect to energy deposited through ionization process which is often calibrated by $\gamma$ rays or $\beta$ rays. Second, the quenching factor (QF) of the detector material which is the efficiency of nuclear recoil energy transforming into ionization energy. A $33.5PE/keV_{ee}$ light yield of undoped CsI in 77K \cite{Ding2020uxu} is adopted in this estimation. And the quenching factor of undoped CsI uses the result measured by COHERENT collaboration \cite{COHERENT2021pcd}. 

Convolute the recoiled energy distribution with energy response of detector to nuclear recoils and Fig. \ref{fig_CEvNS_NPE} is obtained. It shows the expected spectrum of NPE detected by PMTs.

%%%%%%%%%%%%%%%%%%%%%%%%%%%%%%%%%%%%%%%%%%

\subsection{Estimation of background}\label{est_bkg}
The background comes from various kinds of sources. (i) Lots of beam related neutrons (BRN) would be produced upon the impinging of protons on the Tungsten target. Even though a shield composed of $7m$ of steel and $1m$ of concrete are settled between the target and the platform, some fast neutrons can still escape and reach the platform. The fast neutrons generate nuclear recoil signals by scattering with Cs and I nucleus which are indistinguishable from the real CE$\nu$NS signals and can not be reduced by proton beam trigger. (ii) the PMT dark count event can happen to fall within the signal region of the recorded waveform. Since most CE$\nu$NS signals and most PMT dark count signals both only generate a few detectable PE, the PMT dark count background could be important in low NPE region. (iii), materials and devices used to construct the detector and shield unavoidably contain some long-lived radioactive isotopes. Their decay would also introduce $\gamma$ and $\beta$ background on the detector. (iv), the environmental $\gamma$ background from the decay of long-lived radioactive isotopes in rock and building materials always exit everywhere including the platform. A simulation software framework has been developed based on Geant4 to evaluate the influence of these background on this experiment. The following parts show the detailed research upon different kinds of background. All results are acquired with a detector containing four $3kg$ undoped CsI sub-detectors assumed and a data taking time of half an year.

\begin{figure}[H]
\centering
\includegraphics[width=10.5 cm]{Graphs/NeutronSpec.png}
\caption{Neutron spectrum (\textbf{black}) measured by a ${}^{3}$He multi-sphere neutron spectrometer and the initial guess spectrum (\textbf{red}) feed into the unfolding program.\label{fig_Neutron_Spec}}
\end{figure} 

\subsubsection{Beam related neutron background}\label{BRN}

A ${}^{3}$He multi-sphere neutron spectrometer has been used to measure the neutron spectrum upon the platform and outside the facility. Fig. \ref{fig_Neutron_Spec} shows the neutron spectrum on the platform unfolded by a method similar to the description in \cite{Li2022cjw_Bonar_Spec}. The integrated neutron flux of Fig. \ref{fig_Neutron_Spec} is $4.8 \times 10^{-2} n/cm^{2}/s$, about one magnitude higher than that measured outside the facility, which is  $5.3 \times 10^{-3} n/cm^{2}/s$. Taking this spectrum as an input, with the whole shielding structure considered, a simulation has been done to access how many neutron background would be generated. After all the event selection criterion applied, the survived neutron background spectrum is shown by the orange line in Fig. \ref{fig_AllBkg}. It is worth mentioning that 
the event selection criteria 4 could reject more than 70\% of neutron background according to the simulation.

\subsubsection{PMT dark count background}\label{PMTDC}

A PMT dark count spectrum at $77K$ has been taken by setting the data taking ADC to a self-trigger mode and adjusting the threshold low enough to trigger single photon electron (SPE) signals. The dark count rate is measured to be averagely $111Hz$ and stays stable in a $24h$ data taking period. A toy Monte-Carlo analysis has been carried out to investigate the effect of criteria 3 on this background. Four couple of PMTs of four sub-detectors are considered. Fig. \ref{fig_Cr3_PMTDC} shows the background level with and without applying event selection criteria 3. The PMT dark count background could be suppressed by 3 magnitudes.

\begin{figure}[H]
\centering
\includegraphics[width=10.5 cm]{Graphs/Cr3_PMTDC.png}
\caption{PMT dark count spectra with (\textbf{blue}) and without (\textbf{orange}) applying event selection criteria 3. Four couple of PMTs of four sub-detectors are considered. The event selection criteria 3 can suppress this background by 3 magnitudes. \label{fig_Cr3_PMTDC}}
\end{figure} 

\subsubsection{Long-lived radioactive isotopes background}\label{Radioac}

The concentration of different long-lived radioactive isotopes in different materials and devices are listed in Table \ref{table_radioac}. The influence of the decay radiation of these isotopes are simulated. The decay chain of isotopes are considered and the decay equilibrium is assumed to be reached. Fig. \ref{fig_RadioactiveBkg} shows the spectrum of energy deposited in CsI from different materials and devices. It can be clearly seen that the radioactive background from CsI crystal dominates the radioactive background.

\subsubsection{Environmental $\gamma$ background}\label{Env}

The contribution of environmental $\gamma$ background is estimated using spectrum shown in Fig. \ref{fig_Env} as an input. The spectrum is measured by CDEX collaboration in CJPL (China Jinping Underground Laboratory) \cite{Ma2020rpd_CDEX_Gamma}. As stronger radioactivity from ${}^{222}Rn$ and other radioactive isotopes from rock is expected in an underground circumstance like CJPL, this estimation should be conservative. According to our simulation, owing to the strong shielding effect of Lead to $\gamma$, the number of $\gamma$ able to penetrate the shield and depositing energy in CsI would be so low that the contribution of environmental $\gamma$ is invisible in Fig. \ref{fig_AllBkg}. 

\subsubsection{Summary of background}

After all event selection criterion employed, Fig \ref{fig_AllBkg} summaries the contributions of background surviving all cuts from different sources. In the region of NPE smaller than 5, the background from PMT dark count dominates. While the BRN background prevails in higher NPE region. The radioactive background contribute a rather low-flat component and the environmental background is too weak to be visible. 

\begin{figure}[H]
\centering
\includegraphics[width=10.5 cm]{Graphs/RadioactiveBkg.png}
\caption{The spectra of energy deposited in CsI from different materials and devices. Four $3kg$ undpoed CsI sub-detectors are assumed. The background from CsI (\textbf{blue}) is the dominate component followed by background from stainless steel of the Dewar (\textbf{purple}),  PMTs (\textbf{red}) and Copper (\textbf{orange}). The contribution from liquid nitrogen (\textbf{green}) is very little. \label{fig_RadioactiveBkg}}
\end{figure} 

\begin{figure}[H]
\centering
\includegraphics[width=10.5 cm]{Graphs/Env_Spec.png}
\caption{The environmental $\gamma$ background redraw from the measurement by CDEX collaboration in CJPL \cite{Ma2020rpd_CDEX_Gamma}. \label{fig_Env}}
\end{figure} 

There are also other possible background sources like neutrino induced neutrons (NIN) and cosmic ray induced short-lived radioactive isotopes (CRSRI). They are found to be negligible by COHERENT collaboration \cite{scholz2018first}. The event rate of both of them depends on the shielding structure. The NIN event rate also relays on the neutrino flux and the CRSRI event rate is proportional to the cosmic ray rate. Since our experiment share a similar neutrino flux, shielding structure and overburden to cosmic ray with COHERENT experiment with difference within a magnitude, these background are neglected by us in this estimation.

\begin{figure}[H]
\centering
\includegraphics[width=10.5 cm]{Graphs/AllBkg_Situ.png}
\caption{The summary of contributions of background from difference sources after all cuts applied. PMT dark count background (\textbf{green}) dominate in low NPE region while the BRN background (\textbf{orange}) prevails in high NPE region. The radioactive background (\textbf{red}) contribute a low-flat component and the environmental background (\textbf{purple}) is too weak to be seen. Four $3kg$ undoped CsI sub-detectors are considered. \label{fig_AllBkg}}
\end{figure} 

\begin{table}[H]
\caption{The concentration of radioactive isotopes in different materials and devices.\label{table_radioac}}
	\begin{adjustwidth}{-\extralength}{0cm}
		\newcolumntype{C}{>{\centering\arraybackslash}X}
		\begin{tabularx}{\fulllength}{@{}|C|C|C|C|C|C|C|C|C|@{}}
			\hline
			               & \textbf{PTFE} & \textbf{Fe} & \textbf{HDPE} & \textbf{PMT} & \textbf{Lead} & \textbf{LN2}\textsuperscript{1} & \textbf{CsI}                & \textbf{OFHC}              \\ \hline
\textbf{K40}   & 0.343         & 60          & 43.3          & 37.1         & -             & -            & -                           & -                          \\
\textbf{Ra226} & 0.12          & -           & -             & -            & 3             & -            & -                           & -                          \\
\textbf{Ra228} & 0.11          & -           & -             & -            & -             & -            & -                           & -                          \\
\textbf{Th228} & 0.065         & -           & -             & -            & 1             & -            & -                           & -                          \\
\textbf{U238}  & 1.96          & -           & 19.8          & 5.2          & 0.12          & -            & -                           & 0.077                      \\
\textbf{Ac228} & -             & 70          & -             & -            & -             & -            & -                           & -                          \\
\textbf{Bi214} & -             & 25          & -             & -            & -             & -            & -                           & -                          \\
\textbf{Pb212} & -             & 70          & -             & -            & -             & -            & -                           & -                          \\ 
\textbf{Pb214}          & -             & 25          & -             & -            & -             & -            & -                           & -                          \\
\textbf{Th234}           & -             & 200         & -             & -            & -             & -            & -                           & -                          \\
\textbf{Tl208}          & -             & 70          & -             & -            & -             & -            & -                           & -                          \\
\textbf{Th232}          & -             & -           & 12.2          & 13.4         & -             & -            & -                           & 0.005                      \\
\textbf{Pb210}          & -             & -           & -             & -            & 240000        & -            & -                           & -                          \\
\textbf{Ar39}           & -             & -           & -             & -            & -             & 0.01         & -                           & -                          \\
\textbf{Co60}           & -             & 17          & -             & -            & -             & -            & -                           & -                          \\
\textbf{Cs137}          & 0.17          & 6           & -             & -            & -             & -            & 150 & -  \\
\textbf{Cs134}          & -             & -           & -             & -            & -             & -            & 50  & - \\\hline
\textbf{Unit}           & mBq/kg        & mBq/kg      & mBq/kg        & mBq/PMT      & mBq/kg        & mBq/kg       & mBq/kg                      & mBq/kg                     \\\hline
\textbf{Reference}      & Xenon1T \cite{XENON2017fdb_Bkg_PTFE_Cu_HDPE}       & ILIAS ANAIS \cite{FeBkg} & Xenon1T \cite{XENON2017fdb_Bkg_PTFE_Cu_HDPE}       & Taishan LAr \cite{Wei2020rbm} & EDELWEISS \cite{EDELWEISS2013wrh_Bkg_Lead}    & Taishan LAr \cite{Wei2020rbm}  & KIMS \cite{Kim2003ms_CsIBkg}                        & Xenon1T \cite{XENON2017fdb_Bkg_PTFE_Cu_HDPE}\\
        \hline
		\end{tabularx}
	\end{adjustwidth}
	\noindent{\footnotesize{\textsuperscript{1} The background of liquid nitrogen is estimated by assuming the nitrogen reaching a purity of 99.99\% and the rest impurities are all argon. The radioactive background level of atmosphere argon is taken from \cite{Wei2020rbm}}}
\end{table}


%\subsubsection{Neutrino induced neutron background and cosmic ray induced short-lived radioactive isotopes background}
 %Background from neutrino induced neutrons and cosmic ray induced short-lived radioactive isotopes could also contribute to the total background.



%Background from beamcontributino related neutrons, PMT dark count coincidence and long-lived radioactive isotopes are studied with caution 
%A more precise neutron background field measurement by liquid scintillators capable of $N/\gamma$ discrimination is on going. 

%%%%%%%%%%%%%%%%%%%%%%%%%%%%%%%%%%%%%%%%%%
\section{Expected Sensitivity}\label{Sens}

Using the estimation of signal and background spectra above, the expected sensitivity of this experiment can be evaluated. Fig. \ref{fig_Sens_Esti} shows the expected spectra of CE$\nu$NS events, background events and their summation with all event selection criteria applied considering a detector containing four $3kg$ undoped CsI sub-detectors and a data taking time of half an year. A signal region between [4,72] NPE is selected to maximize the signal to background ratio. The 4 NPE lower limit  is chosen to exclude most of the PMT dark count background and it corresponds to a equivalently $1.5keV_{nr}$ detection threshold to nuclear recoils. Within this signal region, the total signal event rate is $0.074/day/kg$ and the total background event rate is $0.386/day/kg$ , equivalently $160/half year/12kg$ and $833/half year/12kg$ respectively.Its 
compositions are listed in Table \ref{table_bkg_afcut}. The expected confidence level (C.L.) of this experiment is calculated by the following formula:
\begin{linenomath}
\begin{equation}
C.L. = \frac{N_{sig}}{\sqrt{N_{sig} + N_{bkg}}},
\end{equation}
\end{linenomath}
$N_{sig}$ is the expected event number of CE$\nu$NS signal and $N_{bkg}$ is that of background. By the experiment setup assumed above, the C.L. is expected to reach $5.1\sigma$ in half an year. Fig. \ref{fig_CL_m_t} shows the expected C.L. varying with different detector mass and data taking time. Note that the contribution to C.L. from the arrival time profile of events regarding the proton beam trigger is not considered in this estimation yet. Since the arrival time profile of CE$\nu$NS signals are highly correlated to that of proton beam, it significantly differs from those of the PMT dark count and the radioactive background which are evenly distributed in time. Thus, if the contribution to C.L. from arrival time distribution is taken into account additionally, the confidence level would certainly be promoted.

\begin{figure}[H]
\centering
\includegraphics[width=10.5 cm]{Graphs/SensEsti.png}
\caption{The expected spectra of CE$\nu$NS events (\textbf{dashed red}), background events (\textbf{shadowed grey}) and their summation (\textbf{solid blue}) with all event selection criteria applied. Four $3kg$ undoped CsI sub-detectors are considered. The first a few bins of background and the summation reach out of the y-axis range because the PMT dark count background is very high in this region. \label{fig_Sens_Esti}}
\end{figure} 

\begin{table}[H]
\caption{The event rates in signal region after cut of different kinds of background. \label{table_bkg_afcut}}
\newcolumntype{c}{>{\centering\arraybackslash}X}
\begin{tabularx}{\textwidth}{|c|c|}
	\hline
    \textbf{Background Type} & \textbf{Event rate in signal region after cut / half year / 12kg} \\ \hline
    \textbf{Beam related neutron}      & 666\\ \hline
    \textbf{PMT dark count}           & 160\\ \hline
    \textbf{Radioactive isotopes}    & 7\\ \hline
    \textbf{Environmental $\gamma$}          & negligible\\ \hline
    \textbf{Neutrino induced neutron}     & negligible\\ \hline
    \textbf{Cosmic ray induced radioactive isotopes}     & negligible\\ \hline
\end{tabularx}

\end{table}


%%%%%%%%%%%%%%%%%%%%%%%%%%%%%%%%%%%%%%%%%%
\section{Experiment schedule}\label{Sched}

By referring to Table \ref{table_bkg_afcut}, it can be clearly seen that the BRN is the critical background in this experiment. An excellent knowledge of the flux, spectrum and filed distribution of the beam related neutrons on the platform is necessary. Several liquid scintillator detectors capable of $N/\gamma$ discrimination have been placed on the platform to acquire a more precise measurement of neutron background.

Meanwhile, the test of a detector prototype able to contain two $3kg$ CsI crystals is also ongoing. The cryogenic system works stably and the feasibility of this detector design has been verified. 

In 2023, we plan to finish the commissioning of the detector and the construction of the shield. If everything goes well, hopefully the data taking could start in 2024. 

\begin{figure}[H]
\centering
\includegraphics[width=10.5 cm]{Graphs/CL.png}
\caption{The expected confidence level varying with different detector mass and data taking time. If a $12kg$ CsI detector is empolyed and taking data for half an year (180 days), the C.L. can reach $5.1\sigma$ (the pentagram).  \label{fig_CL_m_t}}
\end{figure} 

%%%%%%%%%%%%%%%%%%%%%%%%%%%%%%%%%%%%%%%%%%
\section{Summary}

The measurement of CE$\nu$NS signal enjoys great significance among various aspects of physics. By placing a $12kg$ cryogenic undoped CsI detector inside a strong shield on a platform $8.2m$ away from the Tungsten target in CSNS, with some event selection criterion employed to enhance the signal to background ratio, the detector threshold is expected to be lower down to $1.5keV_{nr}$ nuclear recoil energy. The detectable CE$\nu$NS events rate is expected to be $160/half year$ and the total background rate could be suppressed to be $833/half year$. Within a half-year data taking period, a $5\sigma$ detection of CE$\nu$NS signal is anticipated. If the dedicated measurement of neutron background and the test of prototype progress smoothly, the data taking is anticipated to be started in 2 years.

%%%%%%%%%%%%%%%%%%%%%%%%%%%%%%%%%%%%%%%%%%
\vspace{6pt} 

%%%%%%%%%%%%%%%%%%%%%%%%%%%%%%%%%%%%%%%%%%
%% optional
%\supplementary{The following supporting information can be downloaded at:  \linksupplementary{s1}, Figure S1: title; Table S1: title; Video S1: title.}

% Only for the journal Methods and Protocols:
% If you wish to submit a video article, please do so with any other supplementary material.
% \supplementary{The following supporting information can be downloaded at: \linksupplementary{s1}, Figure S1: title; Table S1: title; Video S1: title. A supporting video article is available at doi: link.}

%%%%%%%%%%%%%%%%%%%%%%%%%%%%%%%%%%%%%%%%%%
%\authorcontributions{Conceptualization, Chenguang Su, Qian Liu and Tianjiao Liang; Data curation, Chenguang Su; Formal analysis, Chenguang Su; Funding acquisition, Qian Liu; Investigation, Chenguang Su; Methodology, Chenguang Su and Qian Liu; Project administration, Qian Liu; Resources, Tianjiao Liang; Software, Chenguang Su and Qian Liu; Supervision, Qian Liu and Tianjiao Liang; Validation, Chenguang Su; Visualization, Chenguang Su; Writing – original draft, Chenguang Su; Writing – review & editing, Qian Liu.}

\funding{This work was supported by the National Natural Science Foundation of China (Grant No. 12221005 and 12175241) and the Fundamental Research Funds for the Central Universities.}
%Please add: ``This research received no external funding'' or ``This research was funded by NAME OF FUNDER grant number XXX.'' and  and ``The APC was funded by XXX''. Check carefully that the details given are accurate and use the standard spelling of funding agency names at \url{https://search.crossref.org/funding}, any errors may affect your future funding.

%\institutionalreview{In this section, you should add the Institutional Review Board Statement and approval number, if relevant to your study. You might choose to exclude this statement if the study did not require ethical approval. Please note that the Editorial Office might ask you for further information. Please add “The study was conducted in accordance with the Declaration of Helsinki, and approved by the Institutional Review Board (or Ethics Committee) of NAME OF INSTITUTE (protocol code XXX and date of approval).” for studies involving humans. OR “The animal study protocol was approved by the Institutional Review Board (or Ethics Committee) of NAME OF INSTITUTE (protocol code XXX and date of approval).” for studies involving animals. OR “Ethical review and approval were waived for this study due to REASON (please provide a detailed justification).” OR “Not applicable” for studies not involving humans or animals.}

%\informedconsent{Any research article describing a study involving humans should contain this statement. Please add ``Informed consent was obtained from all subjects involved in the study.'' OR ``Patient consent was waived due to REASON (please provide a detailed justification).'' OR ``Not applicable'' for studies not involving humans. You might also choose to exclude this statement if the study did not involve humans.

%Written informed consent for publication must be obtained from participating patients who can be identified (including by the patients themselves). Please state ``Written informed consent has been obtained from the patient(s) to publish this paper'' if applicable.}

%\dataavailability{In this section, please provide details regarding where data supporting reported results can be found, including links to publicly archived datasets analyzed or generated during the study. Please refer to suggested Data Availability Statements in section ``MDPI Research Data Policies'' at \url{https://www.mdpi.com/ethics}. If the study did not report any data, you might add ``Not applicable'' here.} 

%\acknowledgments{To be added.}

\conflictsofinterest{The authors declare no conflict of interest.}
%Declare conflicts of interest or state ``The authors declare no conflict of interest.'' Authors must identify and declare any personal circumstances or interest that may be perceived as inappropriately influencing the representation or interpretation of reported research results. Any role of the funders in the design of the study; in the collection, analyses or interpretation of data; in the writing of the manuscript; or in the decision to publish the results must be declared in this section. If there is no role, please state ``The funders had no role in the design of the study; in the collection, analyses, or interpretation of data; in the writing of the manuscript; or in the decision to publish the~results''.

%%%%%%%%%%%%%%%%%%%%%%%%%%%%%%%%%%%%%%%%%%
%% Optional
%\sampleavailability{Samples of the compounds ... are available from the authors.}

%% Only for journal Encyclopedia
%\entrylink{The Link to this entry published on the encyclopedia platform.}

%\abbreviations{Abbreviations}{
%The following abbreviations are used in this manuscript:\\

%\noindent 
%\begin{tabular}{@{}ll}
%MDPI & Multidisciplinary Digital Publishing Institute\\
%DOAJ & Directory of open access journals\\
%TLA & Three letter acronym\\
%LD & Linear dichroism
%\end{tabular}
%}

%%%%%%%%%%%%%%%%%%%%%%%%%%%%%%%%%%%%%%%%%%
%% Optional
%\appendixtitles{no} % Leave argument "no" if all appendix headings stay EMPTY (then no dot is printed after "Appendix A"). If the appendix sections contain a heading then change the argument to "yes".
%\appendixstart
%\appendix
%\section[\appendixname~\thesection]{}
%\subsection[\appendixname~\thesubsection]{}
%The appendix is an optional section that can contain details and data supplemental to the main text---for example, explanations of experimental details that would disrupt the flow of the main text but nonetheless remain crucial to understanding and reproducing the research shown; figures of replicates for experiments of which representative data are shown in the main text can be added here if brief, or as Supplementary Data. Mathematical proofs of results not central to the paper can be added as an appendix.

%\begin{table}[H] 
%\caption{This is a table caption.\label{tab5}}
%\newcolumntype{C}{>{\centering\arraybackslash}X}
%\begin{tabularx}{\textwidth}{CCC}
%\toprule
%\textbf{Title 1}	& \textbf{Title 2}	& \textbf{Title 3}\\
%\hline
%Entry 1		& Data			& Data\\
%Entry 2		& Data			& Data\\
%\bottomrule
%\end{tabularx}
%\end{table}

%\section[\appendixname~\thesection]{}
%All appendix sections must be cited in the main text. In the appendices, Figures, Tables, etc. should be labeled, starting with ``A''---e.g., Figure A1, Figure A2, etc.

%%%%%%%%%%%%%%%%%%%%%%%%%%%%%%%%%%%%%%%%%%
\begin{adjustwidth}{-\extralength}{0cm}
%\printendnotes[custom] % Un-comment to print a list of endnotes

\reftitle{References}

% Please provide either the correct journal abbreviation (e.g. according to the “List of Title Word Abbreviations” http://www.issn.org/services/online-services/access-to-the-ltwa/) or the full name of the journal.
% Citations and References in Supplementary files are permitted provided that they also appear in the reference list here. 

%=====================================
% References, variant A: external bibliography
%=====================================
\bibliography{./Reference/ref.bib}

%=====================================
% References, variant B: internal bibliography
%=====================================
%\begin{thebibliography}{999}
% Reference 1
%\bibitem[Author1(year)]{ref-journal}
%Author~1, T. The title of the cited article. {\em Journal Abbreviation} {\bf 2008}, {\em 10}, 142--149.
% Reference 2
%\bibitem[Author2(year)]{ref-book1}
%Author~2, L. The title of the cited contribution. In {\em The Book Title}; Editor 1, F., Editor 2, A., Eds.; Publishing House: City, Country, 2007; pp. 32--58.
% Reference 3
%\bibitem[Author3(year)]{ref-book2}
%Author 1, A.; Author 2, B. \textit{Book Title}, 3rd ed.; Publisher: Publisher Location, Country, 2008; pp. 154--196.
% Reference 4
%\bibitem[Author4(year)]{ref-unpublish}
%Author 1, A.B.; Author 2, C. Title of Unpublished Work. \textit{Abbreviated Journal Name} year, \textit{phrase indicating stage of publication (submitted; accepted; in press)}.
% Reference 5
%\bibitem[Author5(year)]{ref-communication}
%Author 1, A.B. (University, City, State, Country); Author 2, C. (Institute, City, State, Country). Personal communication, 2012.
% Reference 6
%\bibitem[Author6(year)]{ref-proceeding}
%Author 1, A.B.; Author 2, C.D.; Author 3, E.F. Title of presentation. In Proceedings of the Name of the Conference, Location of Conference, Country, Date of Conference (Day Month Year); Abstract Number (optional), Pagination (optional).
% Reference 7
%\bibitem[Author7(year)]{ref-thesis}
%Author 1, A.B. Title of Thesis. Level of Thesis, Degree-Granting University, Location of University, Date of Completion.
% Reference 8
%\bibitem[Author8(year)]{ref-url}
%Title of Site. Available online: URL (accessed on Day Month Year).
%\end{thebibliography}

% If authors have biography, please use the format below
%\section*{Short Biography of Authors}
%\bio
%{\raisebox{-0.35cm}{\includegraphics[width=3.5cm,height=5.3cm,clip,keepaspectratio]{Definitions/author1.pdf}}}
%{\textbf{Firstname Lastname} Biography of first author}
%
%\bio
%{\raisebox{-0.35cm}{\includegraphics[width=3.5cm,height=5.3cm,clip,keepaspectratio]{Definitions/author2.jpg}}}
%{\textbf{Firstname Lastname} Biography of second author}

% For the MDPI journals use author-date citation, please follow the formatting guidelines on http://www.mdpi.com/authors/references
% To cite two works by the same author: \citeauthor{ref-journal-1a} (\citeyear{ref-journal-1a}, \citeyear{ref-journal-1b}). This produces: Whittaker (1967, 1975)
% To cite two works by the same author with specific pages: \citeauthor{ref-journal-3a} (\citeyear{ref-journal-3a}, p. 328; \citeyear{ref-journal-3b}, p.475). This produces: Wong (1999, p. 328; 2000, p. 475)

%%%%%%%%%%%%%%%%%%%%%%%%%%%%%%%%%%%%%%%%%%
%% for journal Sci
%\reviewreports{\\
%Reviewer 1 comments and authors’ response\\
%Reviewer 2 comments and authors’ response\\
%Reviewer 3 comments and authors’ response
%}
%%%%%%%%%%%%%%%%%%%%%%%%%%%%%%%%%%%%%%%%%%
\end{adjustwidth}
\end{document}

