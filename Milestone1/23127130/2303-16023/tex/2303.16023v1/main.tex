\documentclass[11pt,a4paper]{amsart}
\usepackage[utf8]{inputenc}
\usepackage[english]{babel}
\usepackage[dvipsnames]{xcolor}
\usepackage{rotating}
\usepackage[all]{xy}
\usepackage{comment}
\usepackage{pdflscape}
%\usepackage{showkeys}
\usepackage{hyperref}
\usepackage{tikz}
\usepackage{tikz-cd}
\usepackage{soul,color}
\usepackage[colorinlistoftodos]{todonotes}
\usepackage[width=\textwidth,font=footnotesize,labelfont=bf, aboveskip=30pt]{caption}
\usepackage[symbol]{footmisc}
\usepackage[OT2,T1]{fontenc}
\DeclareSymbolFont{cyrletters}{OT2}{wncyr}{m}{n}
\DeclareMathSymbol{\Sha}{\mathalpha}{cyrletters}{"58}

\renewcommand{\thefootnote}{\fnsymbol{footnote}}

\newcommand\stm[1]{\todo[color=magenta]{#1}}           %%% Sofia
\newcommand\sti[1]{\todo[inline,color=magenta]{#1}}%% Sofia
\newcommand\mji[1]{\todo[inline,color=yellow]{#1}}% Minyoung
\newcommand{\Ques}[1]{\todo[inline,color=red!50!white]{{\textbf{Question:}#1}}}
\DeclareRobustCommand{\hlgreen}[1]{{\sethlcolor{green}\hl{#1}}}
\newcommand*{\fullref}[1]{\hyperref[{#1}]{\ref*{#1}. \nameref*{#1}}} 

\newcommand{\sO}{\mathcal{O}}
\newcommand{\sM}{\mathscr{M}}
\newcommand{\sP}{\mathscr{P}}
\newcommand{\cH}{\mathrm{H}}
\newcommand{\Num}{\operatorname{Num}}
\newcommand{\set}[1]{\{{#1}\}}
\newcommand{\Ker}{\operatorname{Ker}}
\newcommand{\ts}[1]{\Sha({#1})}
\usepackage[osf]{mathpazo}
\usepackage{charter}
\linespread{1.0} 
\usepackage[a4paper, top=3.2cm, bottom=3.2cm,left=2.8cm, right=2.8cm, heightrounded,bindingoffset=0mm]{geometry}

\usepackage{hyperref}
\hypersetup{bookmarksnumbered=true, linkcolor=due, citecolor=Green, urlcolor=black,}
%\usepackage{fullpage}
%opening



% After the document class declaration comes the preamble.
% The preamble begins here.

   % First we activate any packages that we may need.
   %
   % The amssymb package provides \mathbb and other
   % math symbols.  The amsmath package provides sophisticated math
   % constructions.  The amsthm package provides \theoremstyle and
   % the \proof environment.
   %
   % The amsmath and amsthm packages are automatically activated by
   % \documentclass{amsart}, so there is no need to activate them here.
\usepackage{fancyhdr}

\usepackage{amsfonts}
\usepackage{amssymb}
\usepackage{extarrows}
\usepackage{cleveref}
\usepackage{amsthm}
\usepackage{amscd}
\usepackage{amsmath}
\usepackage{mathtools}
\usepackage{mathrsfs}
\usepackage{float}
%\usepackage[all,cmtip]{xy}

\usepackage{color}	%scrivere a colori
\definecolor{due}{RGB}{0,76,147}

\usepackage{graphicx}	
\usepackage{multicol}	
\usepackage{wrapfig}
\usepackage[english]{babel} 
\usepackage[utf8]{inputenc}
\usepackage[T1]{fontenc}
\usepackage{enumitem}

\usepackage{longtable}
\usepackage{cite}

%\usepackage{eufrak}
%\usepackage{euler}
%\usepackage{eulervm}

   % Next we use \newtheorem to specify our theorem-like environments
   % (theorem, definition, etc.) and how to display and number them.
   %
   % Note: The \theoremstyle declarations affect the appearance of the
   % Theorems, Definitions, etc.

\theoremstyle{definition}
\newtheorem{defi}{Definition}[section]
\theoremstyle{plain}
\newtheorem{thm}[defi]{Theorem}
\newtheorem{SPC}[defi]{Special Case}
\newtheorem*{mainthm}{Main Theorem}
\newtheorem{introthm}{Theorem}[section]
\renewcommand*{\theintrothm}{\Alph{introthm}}
\newtheorem{introcor}[introthm]{Corollary}
\newtheorem{prop}[defi]{Proposition}
\newtheorem{cor}[defi]{Corollary}
\newtheorem{lemma}[defi]{Lemma}
\theoremstyle{remark}
\newtheorem*{question}{Question}
\newtheorem{problem}[defi]{Problem}
\newtheorem{conj}[defi]{Conjecture}
\newtheorem{rmk}[defi]{Remark}
\newtheorem{ex}[defi]{Example}
\newtheorem{nota}[defi]{Notation}
\newtheorem*{notation}{Notation}
\newtheorem{fact}[defi]{Fact}
\newtheorem{claim}[defi]{Claim}
\newtheorem{exercise}[defi]{Exercise}
\theoremstyle{definition}
\newtheorem*{pf}{Proof}
\newtheorem*{ack}{Acknowledgement}

\newtheorem{innercustomthm}{Theorem}
\newenvironment{customthm}[1]
  {\renewcommand\theinnercustomthm{#1}\innercustomthm}
  {\endinnercustomthm}

   % The preamble is also a good place to define new commands and macros.
   % This part of the preamble is strictly optional according to your taste.

%ARROWS
\newcommand{\hra}{\hookrightarrow}
\newcommand{\lra}{\leftrightarrow}
\newcommand{\la}{\leftarrow}
\newcommand{\ra}{\rightarrow}
\newcommand{\longlra}{\longleftrightarrow}
\newcommand{\longla}{\longleftarrow}
\newcommand{\longra}{\longrightarrow}
\newcommand{\Lra}{\Leftrightarrow}
\newcommand{\La}{\Leftarrow}
\newcommand{\Ra}{\Rightarrow}
\newcommand{\Longlra}{\Longleftrightarrow}
\newcommand{\Longla}{\Longleftarrow}
\newcommand{\Longra}{\Longrightarrow}
\newcommand{\dashra}{\dashrightarrow}
\newcommand{\dashla}{\dashleftarrow}
\newcommand{\hookra}{\hookrightarrow}
\newcommand{\hooka}{\hookleftarrow}
\newcommand{\fitt}{\mathscr{F}\mathit{itt}}
\newcommand{\inv}{^{-1}}
\newcommand{\hermitian}[1]{^{H}\!{#1}}
\newcommand{\transpose}[1]{^{T}\!{#1}}
\newcommand{\crit}{\mathscr{C}}


%Mengensymbole in Kurzform definiert
\newcommand{\ii}{\ensuremath{\mathsf{i}}}	


\newcommand{\ka}{{\mathcal A}}
\newcommand{\kb}{{\mathcal B}}
\newcommand{\kc}{{\mathcal C}}
\newcommand{\kd}{{\mathcal D}}
\newcommand{\ke}{{\mathcal E}}
\newcommand{\kf}{{\mathcal F}}
\newcommand{\kg}{{\mathcal G}}
\newcommand{\kh}{{\mathcal H}}
\newcommand{\ki}{{\mathcal I}}
\newcommand{\kj}{{\mathcal J}}
\newcommand{\kk}{{\mathcal K}}
\newcommand{\kl}{{\mathcal L}}
\newcommand{\km}{{\mathcal M}}
\newcommand{\kn}{{\mathcal N}}
\newcommand{\ko}{{\mathcal O}}
\newcommand{\kp}{{\mathcal P}}
\newcommand{\kq}{{\mathcal Q}}
\newcommand{\kr}{{\mathcal R}}
\newcommand{\ks}{{\mathcal S}}
\newcommand{\kt}{{\mathcal T}}
\newcommand{\ku}{{\mathcal U}}
\newcommand{\kv}{{\mathcal V}}
\newcommand{\kw}{{\mathcal W}}
\newcommand{\kx}{{\mathcal X}}
\newcommand{\ky}{{\mathcal Y}}
\newcommand{\kz}{{\mathcal Z}}
% Mathbb
\newcommand{\IA}{{\mathbb A}}
\newcommand{\IB}{{\mathbb B}}
\newcommand{\IC}{{\mathbb C}}
\newcommand{\ID}{{\mathbb D}}
\newcommand{\IE}{{\mathbb E}}
\newcommand{\IF}{{\mathbb F}}
\newcommand{\IG}{{\mathbb G}}
\newcommand{\IH}{{\mathbb H}}
\newcommand{\II}{{\mathbb I}}
%\newcommand{\IJ}{{\mathbb J}}
\newcommand{\IK}{{\mathbb K}}
\newcommand{\IL}{{\mathbb L}}
\newcommand{\IM}{{\mathbb M}}
\newcommand{\IN}{{\mathbb N}}
\newcommand{\IO}{{\mathbb O}}
\newcommand{\IP}{{\mathbb P}}
\newcommand{\Q}{{\mathbb Q}}
\newcommand{\R}{{\mathbb R}}
\newcommand{\IS}{{\mathbb S}}
\newcommand{\IT}{{\mathbb T}}
\newcommand{\IU}{{\mathbb U}}
\newcommand{\IV}{{\mathbb V}}
\newcommand{\IW}{{\mathbb W}}
\newcommand{\IX}{{\mathbb X}}
\newcommand{\IY}{{\mathbb Y}}
\newcommand{\Z}{{\mathbb Z}}

% Mathfrak
\newcommand{\ga}{{\mathfrak a}}
\newcommand{\gb}{{\mathfrak b}}
\newcommand{\gc}{{\mathfrak c}}
\newcommand{\gd}{{\mathfrak d}}
\newcommand{\gee}{{\mathfrak e}}
\newcommand{\gf}{{\mathfrak f}}
\newcommand{\gothg}{{\mathfrak g}}
\newcommand{\gh}{{\mathfrak h}}
\newcommand{\gi}{{\mathfrak i}}
\newcommand{\gj}{{\mathfrak j}}
\newcommand{\gk}{{\mathfrak k}}
\newcommand{\gl}{{\mathfrak l}}
\newcommand{\gm}{{\mathfrak m}}
\newcommand{\gn}{{\mathfrak n}}
%\newcommand{\go}{{\mathfrak o}}
\newcommand{\gp}{{\mathfrak p}}
\newcommand{\gq}{{\mathfrak q}}
\newcommand{\gr}{{\mathfrak r}}
\newcommand{\gs}{{\mathfrak s}}
\newcommand{\gt}{{\mathfrak t}}
\newcommand{\gu}{{\mathfrak u}}
\newcommand{\gv}{{\mathfrak v}}
\newcommand{\gw}{{\mathfrak w}}
\newcommand{\gx}{{\mathfrak x}}
\newcommand{\gy}{{\mathfrak y}}
\newcommand{\gz}{{\mathfrak z}}
%
\newcommand{\gA}{{\mathfrak A}}
\newcommand{\gB}{{\mathfrak B}}
\newcommand{\gC}{{\mathfrak C}}
\newcommand{\gD}{{\mathfrak D}}
\newcommand{\gE}{{\mathfrak E}}
\newcommand{\gF}{{\mathfrak F}}
\newcommand{\gG}{{\mathfrak G}}
\newcommand{\gH}{{\mathfrak H}}
\newcommand{\gI}{{\mathfrak I}}
\newcommand{\gJ}{{\mathfrak J}}
\newcommand{\gK}{{\mathfrak K}}
\newcommand{\gL}{{\mathfrak L}}
\newcommand{\gM}{{\mathfrak M}}
\newcommand{\gN}{{\mathfrak N}}
\newcommand{\gO}{{\mathfrak O}}
\newcommand{\gP}{{\mathfrak P}}
\newcommand{\gQ}{{\mathfrak Q}}
\newcommand{\gR}{{\mathfrak R}}
\newcommand{\gS}{{\mathfrak S}}
\newcommand{\gT}{{\mathfrak T}}
\newcommand{\gU}{{\mathfrak U}}
\newcommand{\gV}{{\mathfrak V}}
\newcommand{\gW}{{\mathfrak W}}
\newcommand{\gX}{{\mathfrak X}}
\newcommand{\gY}{{\mathfrak Y}}
\newcommand{\gZ}{{\mathfrak Z}}

\newcommand{\ima}{{\mathsf i}}

\newcommand{\altD}{\textfrak{D}}

%cohomology
\newcommand{\rH}{{\rm H}}
\newcommand{\rHet}[1]{{\rm H}_{\rm \acute{e}t}^{#1}}
\newcommand{\rHcr}[1]{{\rm H}_{\rm crys}^{#1}}
\newcommand{\Nwt}{{\rm Nwt}}
\newcommand{\rT}{{\rm T}}

%groups
\newcommand{\Aut}{\operatorname{Aut}}
\newcommand{\Gal}{\operatorname{Gal}}
\newcommand{\Div}{\operatorname{Div}}
\newcommand{\Pic}{\operatorname{Pic}}
\newcommand{\NS}{\operatorname{NS}}
\newcommand{\CH}{\operatorname{CH}}
\newcommand{\MW}{\operatorname{MW}}
\newcommand{\Ext}{\operatorname{Ext}}

%matrices
\newcommand{\im}{\operatorname{im}}
\newcommand{\rk}{\operatorname{rk}}
\newcommand{\rank}{\operatorname{rank}}
\newcommand{\Mat}{\operatorname{Mat}}
\newcommand{\End}{\operatorname{End}}
\newcommand{\Hom}{\operatorname{Hom}}
\newcommand{\GL}{\operatorname{GL}}
\newcommand{\SL}{\operatorname{SL}}
\newcommand{\PGL}{\operatorname{PGL}}
\newcommand{\Sp}{\operatorname{Sp}}

%CFT of quadratic forms 
\newcommand{\disc}{\operatorname{disc}}
\newcommand{\Stab}{\operatorname{Stab}}
\newcommand{\Spl}{\operatorname{Spl}}
\newcommand{\lcm}{\operatorname{lcm}}

%miscellany
\newcommand{\Spec}{\operatorname{Spec}}
\newcommand{\codim}{\operatorname{codim}}
\newcommand{\Sym}{\operatorname{Sym}}
\newcommand{\Sing}{\operatorname{Sing}}
\newcommand{\id}{\operatorname{id}}
\newcommand{\pt}{\operatorname{pt}}
\newcommand{\const}{\operatorname{const}}
\newcommand{\Centre}{\operatorname{centre}}
\renewcommand{\div}{\operatorname{div}}
\newcommand{\irr}{\operatorname{irr}}
\newcommand{\ord}{\operatorname{ord}}
\newcommand{\Km}{\operatorname{Km}}
\newcommand{\supp}{\operatorname{supp}}
\newcommand{\NL}{\mathrm{NL}}
\newcommand{\md}{{\;}\text{mod}{\;}}
\newcommand{\ol}{\overline}
\newcommand{\wt}{\widetilde}
\newcommand{\wh}{\widehat}
\newcommand{\Isom}{\operatorname{Isom}}

\newenvironment{aside}{\begin{quote}\sffamily}{\end{quote}}
\newcommand{\marg}[1]{\normalsize{{\color{red}\footnote{{\color{blue}#1}}}{\marginpar[\vskip -.3cm {\color{red}\hfill\tiny\thefootnote$\rightarrow$}]{\vskip -.3cm{ \color{red}$\leftarrow$\tiny\thefootnote}}}}}
\newcommand{\Klaus}[1]{\marg{(Klaus) #1}}
\newcommand{\Roberto}[1]{\marg{(Roberto) #1}}

  %The following mysterious maneuver gets rid of AMS junk at the top
   % and bottom of the first page.
   
  \makeatletter
%\newcommand{\xleftrightarrow}[2][]{\ext@arrow 3359\leftrightarrowfill@{#1}{#2}}
\newcommand{\xdashrightarrow}[2][]{\ext@arrow 0359\rightarrowfill@@{#1}{#2}}
\newcommand{\xdashleftarrow}[2][]{\ext@arrow 3095\leftarrowfill@@{#1}{#2}}
\newcommand{\xdashleftrightarrow}[2][]{\ext@arrow 3359\leftrightarrowfill@@{#1}{#2}}
\def\rightarrowfill@@{\arrowfill@@\relax\relbar\rightarrow}
\def\leftarrowfill@@{\arrowfill@@\leftarrow\relbar\relax}
\def\leftrightarrowfill@@{\arrowfill@@\leftarrow\relbar\rightarrow}
\def\arrowfill@@#1#2#3#4{%
  $\m@th\thickmuskip0mu\medmuskip\thickmuskip\thinmuskip\thickmuskip
   \relax#4#1
   \xleaders\hbox{$#4#2$}\hfill
   #3$%
}
\makeatother

% This ends the preamble.  We now proceed to the document itself.
\newcommand{\escorta}[3]{0\rightarrow{#1}\longrightarrow{#2}
\longrightarrow{#3}\rightarrow0}
\newcommand{\fas}[1]{\mathscr{#1}}
\newcommand{\fascio}{\fas{F}}
\newcommand{\F}{\mathcal{F}}
\newcommand{\A}{\fas{A}}
\newcommand{\M}{\fas{M}}
\newcommand{\pp}{\fas{P}}
\newcommand{\LL}{\fas{L}}
\newcommand{\E}{\fas{e}}
\newcommand{\OO}[1]{\fas{O}_{#1}}
\newcommand{\ox}{\OO{X}}
\newcommand{\coo}[3]{H^{#1}({#2},\:{#3})}
\newcommand{\cox}[1]{\coo{0}{X}{{#1}}}
\newcommand{\coxi}[2]{\coo{#2}{X}{{#1}}}
\newcommand{\cok}[1]{\coo{0}{K_X}{{#1}}}
\newcommand{\coki}[2]{\coo{#2}{K_X}{{#1}}}
\newcommand{\Coo}[1]{H^{0}({#1})}


\numberwithin{equation}{section}
\begin{document}
	\title[Syzygies of Kummer Varieties]{Cohomological rank functions and syzygies of Kummer Varieties}
	\author{Minyoung Jeon}
	  \address{Department of Mathematics\\University of Georgia, Athens GA 30605, USA}
\email{minyoung.jeon@uga.edu}
  \author{Sofia Tirabassi}
  \address{Department of Mathematics\\ Stockholm University, Albano campus Hus 1, Stockholm, Sweden}
\email{\url{tirabassi@math.su.se}}


%	  Lddress{TU M\"unchen, Zentrum Mathematik -- M11,  Boltzmannstrchen (Germany)}
%	\email{laface@ma.tum.de}
	%\dedicatory{Dedicated to my mother on the occasion of her 50th birthday}

\clearpage\maketitle 
%\thispagestyle{empty}
%\vspace{.5cm}
\begin{abstract}
We use Jiang--Pareschi cohomological rank functions and techniques developed by Caucci and Ito to study syzygies of Kummer varieties, improving existing results by the second author.
\end{abstract}

%\setcounter{tocdepth}{1}
%level -1: part, 0: chapter, 1: section, etc.
%\tableofcontents 
%\setcounter{tocdepth}{1}
%\tableofcontents
\section{Introduction}
Let $X$ be an abelian variety over an algebraically closed field of characteristic not 2. Its \textit{associated Kummer variety}, $K_X$, is the quotient of $X$ by the natural $(\Z/2\Z)$-action induced by the morphism $(-1)_X:X\rightarrow X$ defined by $x\mapsto -x$. In \cite{Tirabassi}, the second author applied the theory of $M$-regularity of Pareschi--Popa (see \cite{pareschi2012basic} for a nice survey) to study higher syzygies of embeddings of Kummer varieties. In the present article, we aim to study the same problem, but instead of $M$-regularity we want to apply the more recent theory of \emph{cohomological rank functions}. As a corollary of our main theorem we improve some of the results in \cite{Tirabassi}.\par
Given $(X,l)$ a polarized abelian variety and an object $\mathcal{F}\in D^b(X)$, Jiang--Pareschi and Caucci (see \cite{JiPa2020,Caucci}) define the {cohomological rank functions} associated to $\mathcal{F}$ and $l$
\[
h^i_{\mathcal{F},l}:\mathbb{Q}\rightarrow\mathbb{Q}^+
\]
and use them to introduce certain numerical invariants associated to the polarization $l$. For example, if $\mathcal{\F}=\mathcal{I}_p$ is the ideal sheaf of a point $p\in X$, the \emph{basepoint-freeness threshold} is defined as
\[\beta(l)\coloneqq\in\{x\in\mathbb{{Q}}^+\:|\:h^1_{\mathcal{I}_p,l}(x)=0\},\] 
and has the following properties:
\begin{itemize}
    \item $\beta(l)\leq 1$ and $\beta(l)<1$ if and only if all line bundles $L$ with $c_1(L)=l$ are globally generated;
    \item by defintion (see \ref{crf} and \cite[Proof of Cor. 1.2]{Caucci}), we have that, for every $m$ positive integer, $\beta(ml)=\frac{\beta(l)}{m}$;
    \item $\beta(l)<\frac{1}{2}$ if, and only if, for every $L$ with $c_1(L)=0$, we have that $L$ is very ample and normally generated (see \cite[Cor. E]{JiPa2020});
    \item if $\beta(l)<\frac{1}{p+2}$, then, for every $L$ with $c_1(L)=0$, we have that $L$ is very ample and satisfies property $N_p$ (see \cite[Thm. 1.1]{Caucci});
\end{itemize}
Our starting point is the well known fact that ample line bundles on a Kummer variety $K_X$ have a nice description in terms of ample line bundles on $X$. More
precisely, denoting by $\pi_X:X\rightarrow
K_X$ the quotient map, for every $A$ ample on $K_X$ there
exists $M$ ample on $X$ such that $\pi_X^*A\simeq M^{\otimes
2}$. Our main result is the following
\begin{introthm}\label{TheoremA}
Let $A$ be a very ample line bundle on the Kummer variety $K_X$. Write $\pi^*A=M^{\otimes 2}$ with $M$ ample and denote by $\lambda$ the class of $M$ in $\operatorname{NS}(X)$. If
\[\beta(\lambda)<\frac{2}{p+2},\]
then $A$ satisfies property $N_p$.\\
More generally, if
\[\beta(\lambda)<\frac{2(r+1)}{p+r+2},\]
then $A$ satisfies property $N_p^r$.\\

\end{introthm}
As a corollary we get
\begin{introcor}\label{corllaryA}
Given $A$ an ample line bundle on the Kummer variety $K_X$, then $A^{\otimes m}$ satisfies property $N_p$ for every $m> \frac{p}{2}+1$.
\end{introcor}
In \cite[Thm. A]{Tirabassi}, it was shown that $A^{\otimes{m}}$ satisfies $N_p$ whenever $m\geq p+2$. Thus, with these new techniques, we obtain an improvement of a "factor two" of the previous known results about higher syzygies of Kummer varieties. \par
This paper is organized as follows. In Section \ref{prem} we outline some background material, as syzygies, cohomological rank functions, and the classical theory of ample line bundles on Kummer varieties. Section \ref{invariants} is the technical hearth of the paper, where we compare some invariants arising from cohomological rank functions. Once this is done, we prove our main results in \ref{proofs}, by following arguments of Caucci. \par
After finishing writing our paper we were informed by Caucci tha in \cite{caucci2} he obtained the same results using the same tools but with a different argument. We hope we have managed to coordinate in such way that the two preprints will appear on the same day. We are grateful to Caucci to show us the draft of his paper.
%\sti{Sofia's comments are red}
%\mji{Minyoung's comment are yellow}
\subsection*{Notation}
We work on an algebraically closed field $\mathbb{K}$ with $\operatorname{char}(\mathbb{K})\neq 2$. We use standard notation for abelian varieties. If $X$ is an abelian variety, we denote by $\hat{X}$ its dual abelian variety. Given an element $\alpha\in\hat{X}$, we let $P_\alpha$ be the corresponding line bundle in $\operatorname{Pic}^0(X)$, that is 
$P_\alpha=\mathscr{P}_{|X\times\{\alpha\}}$, where $\mathscr{P}$ is the normalized Poincar\'e line bundle on $X\times \hat X$. Given a integer $n$, we let $n_X:X\rightarrow X$ be the "multiplication by $n$" map. When $n$ is positive, we denote by $X[n]\coloneqq \operatorname{Ker}n_X$: the group-subscheme of $n$-torsion points of $X$. For every $x\in X$ we let $t_x:X\rightarrow X$ be the map defined by $a\mapsto a+x$. Given a polarization $l$ on $X$, we let $\varphi_l:X\rightarrow \hat{X}$ be the isogeny defined by $x\mapsto t_x^*L\otimes L^{-1}$, where $L$ is an ample line bundle with $c_1(L)=l$. It is a standard fact about abelian varieties that this does not depend from the choice of $L$. We will denote the  kernel of $\varphi_l$ by $K(l)$. The identity element of $X$ is denoted simply by 0.\\ By $D^b(X)$ we denote $D^b(\operatorname{Coh}(X))$, the derived category of the category of coherent sheaves on $X$.
\subsection*{Aknowledgements}
This project has started during Virtual Workshop II for Women in Commutative Algebra and Algebraic Geometry hosted by the Fields Institute and organized by M. Harada and C. Miller. We are really grateful to the Fields Institute, C. Miller and M. Harada for this opportunity to make new connection and create collaborations. We, in addition, wish to thanks both the organizers and the participants to the workshop for creating a friendly and stimulating environment, and for the engaging mathematical discussion. S. Tirabassi is grateful to F. Caucci and G. Pareschi for pointing out that the sequence \eqref{eqnonsplit} does not split general. She would like to extend a special thank to G. Pareschi for teaching her all she knows about syzygies: the paper \cite{Tirabassi} was the first problem Pareschi assigned to her when, more than 10 years ago, she started her PhD.\par
S. Tirabassi was partially supported by the Knut and Alice Wallenberg Foundation under grant no. KAW 2019.0493. M. Jeon was partially supported by the AMS-Simons travel grant.
\section{Preliminaries}\label{prem}
\subsection{Syzygies of projective varieties}\label{sub:N_P}
Let $Z$ an algebraic variety over an algebraically closed field $\mathbb{K}$ and let $L$ an ample
invertible sheaf on $Z$. Let $S_L$ be the symmetric algebra over $H^0(Z,L)$. Then \textit{section ring associated to  $L$},
\[R_L\coloneqq \bigoplus_{m\in\mathbb{Z}}H^{0}({Z},{L^{
\otimes m}}),\]
is a finitely generated graded $S_L$-algebra and as such it admits a \textit{minimal free
resolution} 
\begin{equation}\label{minalfree}
E_\bullet = 0\rightarrow\cdots\xrightarrow{f_{p+1}}
E_p\xrightarrow{f_p}\cdots\xrightarrow{f_2} E_1\xrightarrow{f_1}
E_0\xrightarrow{f_0}R_L\rightarrow 0
\end{equation}
with  $E_i\simeq\bigoplus_j S_L(-a_{ij})$, $a_{ij}\in\mathbb  Z_{>0}$.
Following Green (\cite{green1984koszul}),  we say that $L$ satisfies
property $N_p$ for some nonnegative integer $p$, if, in the notations above,
$$E_0=S_L%\quad\text{when $p\geq 0$},
$$
and
$$E_i=\oplus S_L(-i-1)\quad 1\leq i\leq p.$$
In (\cite{pareschi2000syzygies}), Pareschi  extended this by introducing property $N_p^r$, where $r\geq 0$ is an integer which measures how much property $N_p$ fails (that is property $N_p^0$ is equivalent to property $N_p$). More precisely, we say that $L$ satisfies property $N_p^r$ if, in the notation above, $a_{0j}\leq 1+r$ for
every $j$. Inductively, we see that property $N_p^r$ holds for $L$ if $L$ satisfies  property $N_{p-1}^r$ and $a_{pj}\leq p+1+r$ for every $j$.\par
 It is well known that condition $N_p$ is equivalent to the exactness in the middle of the complex
\begin{equation}\label{eq:np1}
 \bigwedge^{p+1}H^0(L)\otimes H^0(L^{\otimes h})\rightarrow \bigwedge^{p}H^0(L)\otimes H^0(L^{\otimes
h+1})\rightarrow \bigwedge^{p-1}H^0(L)\otimes H^0(L^{\otimes h+2})
\end{equation}
for any $h\geq 1$. More  generally, condition $N_p^r$ is equivalent to exactness in the middle of
\eqref{eq:np1} for every $h\geq r+1$. When $L$ is globally generated, we can consider the following exact sequence:
\begin{equation}
 \escorta{M_L}{H^0(L)\otimes\OO{Z}}{L}.
\end{equation}
Caucci in \cite[Prop.4.1]{Caucci} has shown that, independently from the the characteristic of the base field, we have that property $N_p$ (respectively $N_p^r$) is implied by the vanishing 
\begin{equation}\label{importan vanishing}
    H^1(Z,M_{L}^{\otimes p+1}\otimes L^{\otimes h})=0
\end{equation}
for every $h\geq 1$ (respectively, for every $h\geq r+1$). Therefore, our problem is reduced to check the vanishing of certain cohomology groups. We are going to show this vanishing using $\mathbb{Q}$-twisted sheaves and cohomological rank functions, of which we give an overview in the next subsections.
\subsection{$\mathbb{Q}$-twisted sheaves}
Given a polarization $l$ on an abelian variety $X$, we define \emph{coherent objects $\mathbb{Q}$-twisted by $l$} as equivalence classes pairs $(\mathcal{F}, xl)$ where $\mathcal{F}$ is an object in $D^b(X)$ and $x$ is a rational number, with the equivalence relation generated by
$$
(\mathcal{F}\otimes L^m,xl)\sim(\mathcal{F},(m+x)l)
$$
for every integer $m$, and every line bundle $L$ with $c_1(L)=l$. The equivalence class of $(\mathcal{F},xl)$ is denoted by $\mathcal{F}\langle xl\rangle$. If $\mathcal{F}$ is a coherent sheaf, we speak of \emph{coherent $\mathbb{Q}$-twisted sheaf}, instead of object. Given a $\mathbb{Q}$-twisted sheaf $\mathcal{F}\langle xl\rangle$, we choose a representative $L$ for $l$ and define its cohomological support loci as
$$
V^i\left(X, \mathcal{F}\left\langle \frac{a}{b}l\right\rangle, L\right)\coloneqq\{  \alpha\in\hat{X}\:|\:h^i(X, b_X^*\mathcal{F}\otimes L^{ab}\otimes P_  \alpha)=0\}.
$$
If we change the representative for $l$, we obtain a new locus that is a translate of $V^i\left(X, \mathcal{F}\langle \frac{a}{b}l\rangle, L\right)$ by an element in $\hat{X}$. In particular, the dimension of these loci does not depend on the choice of the line bundle $L$ and we can introduce the following definition.
\begin{defi} We say that $\mathcal{F}\langle \frac{a}{b}l\rangle$ is GV if $\codim V^i(X,\mathcal{F}\langle \frac{a}{b}l\rangle, L)\ge i$ for every $i\ge 1$.\par
%We say that $\mathcal{F}\langle \frac{a}{b}l\rangle$ is M-regular if $\codim V^i(X,\mathcal{F}\langle \frac{a}{b}l\rangle, L)\ge i+1$ for every $i\ge 1$.\par
We say that a $\mathbb{Q}$-twisted sheaf $\mathcal{F}\langle \frac{a}{b}l\rangle$ satisfies I.T.(0) if its cohomological support loci are empty for every $i>0$. 
\end{defi}
Clearly we have that GV, implies I.T.(0). In \cite{Caucci} and \cite{Ito22} various properties of GV, and I.T.(0) sheaves are proven. Of most interest for us will be how these properties are preserved under taking tensor products:
\begin{prop}[{\cite[Prop. 3.4]{Caucci}}]\label{caucci}Let $(X,l)$ a polarized abelian variety and  $\mathcal{F}\langle sl\rangle$ and $\mathcal{G}\langle tl\rangle$ two $\mathbb{Q}$-twisted sheaves such that one of them is locally free, one is I.T.(0) and the other GV. Then $\mathcal{F}\otimes\mathcal{G}\langle(s+t)l\rangle$ is I.T.(0).
\end{prop}
As an application we get the following lemma:
\begin{lemma}\label{lem:generalCaucci}
Fix $r\geq 0$ and integer and let $\mathcal{E}$ be a vector bundle on $X$ such that $\mathcal{E}\left\langle \frac{r+1}{q}l\right\rangle$ satisfies I.T.(0) for some positive integer $q$. Then $\mathcal{E}^{\otimes q}\langle hl\rangle$ satisfies I.T.(0) for every $h\ge r+1$. 
\end{lemma}
\begin{proof}
The argument essentially appears in \cite[Proof of Prop 3.5]{Caucci}. We write
\begin{align*}
   \mathcal{E}^{\otimes q}\langle hl\rangle &=\mathcal{E}^{\otimes q}\langle (r+1+h--r1)l\rangle\\
   &=\mathcal{E}^{\otimes q}\left\langle \left(\frac{q(r+1)}{q}+h-r-1\right)l\right\rangle\\
    &=\left(\mathcal{E}\left\langle \frac{r+1}{q}l\right\rangle\right)^{\otimes q}\otimes\mathcal{O}_X\langle (h-r-1)l\rangle\\
\end{align*}
As $\mathcal{O}_X\langle xl\rangle$ is GV when $x=0$ and I.T.(0) otherwise, we have that $\mathcal{O}_X\langle (h-r-1)l\rangle$ is GV for every $h\ge r +1$. Our assumptions grant that $\left(\mathcal{E}\left\langle \frac{r+1}{q}l\right\rangle\right)^{\otimes q}$ satisfies I.T.(0). We conclude by Proposition \ref{caucci}
\end{proof}
\subsection{Cohomological rank functions on abelian varieties}\label{crf}
Let $X$ be an abelian variety, and $l\in\operatorname{NS}(X)$ a polarization. 
For an object $\mathcal{F}\in D^b(X)$  we can consider
$$
h^i_{\text{gen}}(X,\mathcal{F}),
$$
the dimension of the i-th (hyper)cohomology group $H^i(X,\mathcal{F}\otimes P_  \alpha)$ for the general $  \alpha\in\hat{X}$. Then the \emph{cohomological rank functions associated to $\mathcal{F}$ and $l$} is 
$$h^i_{\mathcal{F},l}:\mathbb{Q}\rightarrow\mathbb{Q}_{\ge 0}$$ defined via the assignment
$$\frac{a}{b}\mapsto b^{-2g}h^i_{\text{gen}}(X,b_X^*\mathcal{F}\otimes L^{ab}),$$
with any line bundle $L$ such that $c_1(L)=l$.\par
We see that the value of $h^i_{\mathcal{F},l}(x)$ depends only of $\mathcal{F}\langle xl\rangle$. Furthermore, we have that a $\mathbb{Q}$-twisted sheaf $\mathcal{F}\langle xl\rangle$ satisfies I.T.(0) if, and only if, for every $i>0$, we have that $h^i_{\mathcal{F},l}(x)=0$. In addition to this, cohomological rank functions have been used to define invariants associated to polarizations of abelian varieties. Let $p$ be a closed point in $X$, and we denote by $\mathcal{I}_p$ its associated sheaf of ideals. Given a polarization $l$ on $X$, we have that the quantity
\[\beta(l)\coloneqq\operatorname{inf}\{x\in\mathbb{Q}\:|\: h^1_{\mathcal{I}_p,l}(x)=0\}\]
does not depends from the choice of $p$. Following the work of Jiang--Pareschi and Caucci, we call it the \emph{basepoint-freenes treshold} of $l$. The reason behind the name is that $\beta(l)<1$ if and only if all representative $L$ of $l$
 are globally generated, thus this quantity, in some ways , measures how much representative of $l$ are globally generated.\par
 When $l$ is globally generated, then, for any line bundle $L$ with $c_1(L)=l$, we can consider the short exact sequence
 \begin{equation}\label{sesimportant}
     0\rightarrow M_L\longrightarrow H^0(X,L)\otimes\mathcal{O}_X\longrightarrow L\rightarrow 0
 \end{equation}
 as in \ref{sub:N_P}. We then define the invariant
 \[
 k(l)\coloneqq \inf\{x\in\mathbb{Q}^+\:|\: h^1_{M_L,l}(x)=0\}.
 \]
 When $L$ is very ample, this invariant measures the complexity of syzygies of the embedding induced by $L$, as shown by Caucci in \cite{Caucci}.\par
 By the work of Jiang--Pareschi (see \cite[Thm. D]{JiPa2020}), we have that the invariants $\beta(l)$ and $k(l)$ are related by the following equation:
 \begin{equation}\label{eq:kappabeta}
     k(l)=\frac{\beta(l)}{1-\beta(l)}.
 \end{equation}
  
 \subsection{Symmetric line bundles and Kummer varieties}
We say that a line bundle $M$ on an abelian variety $X$ is \emph{symmetric} if there exists and isomorphism $M\simeq(-1)_X^*M$. When $M$ is ample and symmetric, among all the possible choices of isomorphisms $M\simeq(-1)_X^*M$, there is a canonical one: Mumford in \cite{mumford1966equations} shows that there is just one \emph{normalized isomorphism}
(see \cite[p. 304]{mumford1966equations} for the precise definition) $\psi_M:M\rightarrow (-1)_X^*M$. Using $\psi_M$ we can lift  the $\mathbb{Z}/2\mathbb{Z}$ action on $X$ induced by
the involution $(-1)_X:X\longrightarrow X$ to $M$ in a canonical way : the composition
\[\xymatrix{H^0(X,\:  M)  \ar[rr]^{(-1)_X^*}&&H^0(X,\:(-1)_X^*  M)  \ar[rr]^{
(-1)_X^*(\psi_  M) }&&H^0(X,\:  M)}\]  
is denoted by $[-1]_  L$. We let
\[H^0(X,\:  M)^\pm=\{s\in H^0(X,\:  M)\text{ such that
}[-1]_  M s=\pm s\}.\]
Now let us consider $K_X$ the \emph{Kummer variety associated to $X$}, that is the quotient $X/(-1)_X$. Denote by $\pi:X\rightarrow K_X$ the quotient map. Given $A$ an ample line bundle on $ K_X$, by classical results, we know that $\pi^*A\simeq {M}^{\otimes 2}$ with $ M$ an ample and symmetric line bundle on $X$. When $A$ is globally generated, then, by pulling back the short exact sequence
\begin{equation}
0\rightarrow M_A\longrightarrow H^0(K_X,A)\otimes\mathcal{O}_{K_X}\longrightarrow A\rightarrow 0,
\end{equation}
we get the short exact sequence
\begin{equation}
0\rightarrow \pi^*M_A\longrightarrow H^0(X, M^{\otimes 2})^+\otimes\mathcal{O}_{X}\longrightarrow  M^{\otimes 2}\rightarrow 0.
\end{equation}
Let, as before $M_W\coloneqq \pi^*M_A$. As $\pi_*\mathcal{O}_X$ is a split rank 2 vector bundle on $K_X$, with one summand isomorphic to $\mathcal{O}_{K_X}$, the vanishing of $H^1(K_X,M_A^{\otimes p+1}\otimes A^{\otimes h})$ is implied by the vanishing 
\begin{equation*}
    H^1(X,M_W^{\otimes p+1}\otimes  M^{\otimes 2h})=0.
\end{equation*}
Thus, when $A$ is very ample, property $N_p^r$ for $A$ is implied by a vanishing of the higher cohomology of a sheaf on $X$. In order to investigate that, we use cohomological rank functions. Let $L\coloneqq M^{\otimes 2}$ and denote by $l$ its numerical class, our goal is to give condition for $M_W\left\langle\frac{r+1}{p+1}l\right\rangle$ to satisfy I.T.(0), and then use Lemma \ref{lem:generalCaucci}. To this aim, we introduce a new invariant 
\begin{defi}\label{k+}
    Let $A$ be an ample and globally generated line bundle on $K_X$. Denote by $L$ its pullback $\pi^*A$, and by $l$ the polarization associated to $L$. We define
\[
k^+(A)\coloneqq\inf\{x\in\mathbb{Q}^+\:|\: h^1_{M_W,l}(x)=0\}
\]
    \end{defi}
The key technical step of the proof of Theorem A consists in comparing this new invariant with those introduced in \ref{crf}. We conclude this paragraph with an useful remark which we will need in the sequel.
\begin{rmk}\label{rmk:useful}
Let $x=\frac{a}{b}$ with $a$ and $b$ positive integers. Observe that the $\mathbb{Q}$-twisted objects $\mathcal{I}_p\langle xl\rangle$, $M_W\langle xl\rangle$ and $M_L\langle xl\rangle$ share an important property: their first cohomological rank function $h^1_{F,l}$ vanishes in $x$ if and only if they satisfy I.T.(0). In fact, let $\mathcal{F}$ be any of the sheaves $\mathcal{I}_p$, $M_W$ or $M_L$, then $\mathcal{F}$ fits in a short exact sequence as the one below:
\begin{equation*}
    0\rightarrow \mathcal{F}\longrightarrow V\otimes\mathcal{O}_X\longrightarrow L\rightarrow 0
\end{equation*}
where $V$ is some finite dimensional $\mathbb{K}$-vector space. Then, for any $\alpha\in\hat{X}$, we get the following exact sequence
\begin{equation*}
    0\rightarrow b_X^*\mathcal{F}\otimes L^{\otimes ab}\otimes P_\alpha\longrightarrow V\otimes L^{\otimes ab}\otimes P_\alpha\longrightarrow b_X^*L\otimes L^{\otimes ab}\otimes P_\alpha\rightarrow 0.
\end{equation*}
If we take cohomology, we see directly that $h^i_{\mathcal{F},l}(\frac{a}{b})=0$ for every $i>1$. Therefore, property I.T.(0) depends only on the vanishing of the first cohomological rank function.\par
In particular we deduce that, if $h^1_{F,l}(x)=0$, then for every $\varepsilon\in\mathbb{Q}^+$ we have that $h^1_{F,l}(x+\varepsilon)=0$. In fact, $h^1_{F,l}(x)=0$ implies that $\mathcal{F}\langle xl\rangle$ is I.T.(0). By Proposition \ref{caucci}, we deduce that $\mathcal{F}\langle (x+\varepsilon)l\rangle$ is I.T.(0) for every $\varepsilon>0$. Thus $h^1_{F,l}(x+\varepsilon)=0$.
\end{rmk}

\section{Comparison between the invariants}\label{invariants}
The main technical results of this section is the following comparison of the invariants $k(l)$ and $k^+(A)$:
\begin{thm}\label{thm:great}
    Let $A$, $L$ and $l$ as Definition \ref{k+}, then
    $$ k(l)=k^+(A).
    $$
\end{thm}
We begin with an easy lemma.
\begin{lemma}\label{easy}
    In the notation of Theorem \ref{thm:great} we have that 
    \[k(l)\leq k^+(A)\]
\end{lemma}
\begin{proof}
    By looking at the following diagram
    \[
    \xymatrix{
    && &0\ar[d]&&&\\
    0\ar[r] &M_W\ar[rr]\ar[d] & &H^0(X,L)^+\otimes\mathcal{O}_X\ar[rr]\ar[d]&&L\ar[r]\ar@{=}[d]&0\\
     0\ar[r] &M_{L}\ar[rr] & &H^0(X,L)\otimes\mathcal{O}_X\ar[rr]\ar[d]&&L\ar[r]&0\\
      & & &H^0(X,L)^-\otimes\mathcal{O}_X\ar[d]&&&\\
       && &0&&&
    }
    \]
    we see that there is a short exact sequence
    \begin{equation}\label{eqnonsplit}
        0\rightarrow M_W\longrightarrow M_{L}\longrightarrow H^0(X,L)^-\otimes\mathcal{O}_X \rightarrow 0.
    \end{equation}
    For every $a$ and $b$ positive integers, and every $\alpha\in\hat{X},$ we get the short-exact sequence 
    \begin{equation*}
        0\rightarrow b_X^*M_W\otimes L^{\otimes ab}\otimes P_\alpha\longrightarrow b_X^*M_{L}\otimes L^{\otimes ab}\otimes P_\alpha\longrightarrow H^0(X,L)^-\otimes L^{\otimes ab}\otimes P_\alpha \rightarrow 0.
    \end{equation*}
    We take cohomology and get (suppressing the $X$ in the notation)
    \begin{equation*}
        \cdots H^0(L^{\otimes 2})^-\otimes H^0(L^{\otimes ab}\otimes P_\alpha)\rightarrow H^1( b_X^*M_W\otimes L^{\otimes ab}\otimes P_\alpha)\rightarrow H^1( b_X^*M_{L}\otimes L^{\otimes ab}\otimes P_\alpha) \rightarrow 0
    \end{equation*}
    In particular, if $h^1_{M_W,l}\left(\frac{a}{b}\right)=0$, we get that $h^1_{M_{L},l}\left(\frac{a}{b}\right)=0$, and the statement is proven.
\end{proof}
In the remainder of the section we will show that strict inequality cannot hold, and thus $k^+(A)=k(l)$. Observe that the sequence \eqref{eqnonsplit} is not split in general. In fact, the first two objects in it have vanishing 0-th cohomology by definition, while 
$$
H^0(X,H^0(X,L)^-\otimes\mathcal{O}_X)\simeq H^0(X,L)^-,
$$
which is $0$ if, and only if, $K(l)=X[2]$ (cf. \cite[Inverse Formula]{mumford1966equations}). In particular, we cannot deduce the vanishing of $h^1_{M_W,l}(x)$ directly from the vanishing of $h^1_{M_{L},l}(x)$.  On the other side, we can show the following
\begin{prop}\label{prop:epsilon}
    Let $x\in\mathbb{Q}^+$ such that $h^1_{M_{L},l}(x)$, then, for every $\varepsilon\in\mathbb{Q}^+$ we have that $h^1_{M_W,l}(x+\varepsilon)=0$.
\end{prop}
Observe that Proposition \ref{prop:epsilon} directly implies Theorem \ref{thm:great}. In fact, suppose, by contradiction, that $k(l)<k^+(A)$, and let $k^+(A)>x>k(l)$ a rational number. By Remark \ref{rmk:useful}, $h^1_{M_{L},l}(x)=0$, and thus, by Proposition \ref{prop:epsilon}, we have that $h^1_{M_W,l}(x+\varepsilon)=0$ for every $\varepsilon\in\mathbb{Q}^+$. In particular, for every $\varepsilon\in\mathbb{Q}^+$ we have $x<k^+(A)\leq x+\varepsilon$. By sending $\varepsilon$ to zero, we get a contradiction.
\begin{proof}[Proof of Proposition \ref{prop:epsilon}]
    Let $x=\frac{a}{b}$ a rational number, with $a,b>0$.  Since $L$ is (totally) symmetric, we have that $b_X^*L\simeq L^{\otimes b^2}$. By taking cohomology on the short exact sequence
    \[
    0\rightarrow b_X^*M_{L}\otimes L^{\otimes ab}\otimes P_\alpha \rightarrow H^0(L)\otimes L^{\otimes ab}\otimes P_\alpha\rightarrow L^{\otimes b^2}\otimes L^{\otimes ab}\otimes P_\alpha\rightarrow 0,
    \]
    we see that the vanishing $h^1_{M_{L},l}(x)=0$, is equivalent to the surjectivity of the following multiplication map of global sections
    \[
    H^0(X,L)\otimes H^0(X,L^{\otimes ab})\longrightarrow H^0(X,L^{\otimes b^2+ab}),
    \]
    where we see $H^0(X,L)$ as a subspace of $H^0(X,L^{\otimes b^2})$, via the pullback map $b_X^*$. Since cohomological rank function do not depend from the choice of representation of $x$ as a fraction, we have that, for every $m$ positive integer, the vanishing of  $h^1_{M_{L},l}\left(\frac{a}{b}\right)$ is equivalent to the surjectivity of the following:
    \begin{equation}\label{eq:onto}
        H^0(X,L)\otimes H^0(X,L^{\otimes m^2ab})\longrightarrow H^0(X,L^{\otimes (mb)^2+m^2ab}).
    \end{equation}
    where, again, we see $H^0(X,L)$ as a subspace of $H^0(X,L^{\otimes (mb)^2})$, via the pullback map $(mb)_X^*$. \par
    In the same way, by taking cohomology on the short exact sequence
    \[
    0\rightarrow b_X^*M_{W}\otimes L^{\otimes cd}\otimes P_\alpha \rightarrow H^0(L)^+\otimes L^{\otimes cd}\otimes P_\alpha\rightarrow L^{\otimes d^2}\otimes L^{\otimes cd}\otimes P_\alpha\rightarrow 0,
    \] 
     it is possible to show that the vanishing of $ h^1_{M_W,l}(\frac{c}{d})$ is equivalent to the surjectivity of
     \begin{equation}\label{eq:onto+}
        H^0(X,L)^+\otimes H^0(X,L^{\otimes cd})\longrightarrow H^0(X,L^{\otimes d^2+cd}).
    \end{equation}
    where, via $d_X^*$, we identify $H^0(L)^+$ with a subspace of $H^0(X,L^{\otimes d^2})$.\\
    
\textit{\bfseries Claim:} For every positive integer $m$ such that  $2mb-2\geq 3$, the surjectivity of \eqref{eq:onto} implies the surjetivity of \eqref{eq:onto+} when $c=ma+1$ and $d=mb$.\\

Before proving the claim, let us see how it implies the main statement. Let $x=\frac{a}{b}$ a rational number. By the reasoning above, we have that $h^1_{M_{L},l}(x)=0$ if, and only if \eqref{eq:onto} is surjective. The Claim would then imply that 
\[
0=h^1_{M_W,l}\left(\frac{ma+1}{mb}\right)=h^1_{M_W,l}\left(x+\frac{1}{mb}\right),
\]
when $m$ is sufficiently big. Fix $\varepsilon\in \mathbb{Q}^+$ and chose $m$ as in the claim and such that $\frac{1}{mb}<\varepsilon$. By Remark \ref{rmk:useful}, we have that $h^1_{M_W,l}\left(x+\varepsilon\right)=0$, as we wanted.\par
\textit{Proof of the Claim}
    Let us consider the commutative diagram in Figure \ref{fig:my_label}.
    %\begin{tiny}
    \begin{landscape}
    \begin{figure}[p]
        \centering
%\begin{turn}{90}
%\begin{minipage}{1.57\linewidth}
\begin{small}
\[
\xymatrix{
H^0(L)^+\otimes H^0(L^{\otimes (ma+1)mb)}\otimes P_\alpha)\ar[rr]^{(1)}&&H^0(L^{\otimes m^2b^2+(ma+1)mb}\otimes P_\alpha)\\
H^0(L)^+\otimes H^0(L^{\otimes mb-1})\otimes H^0(L)\otimes H^0(L^{\otimes m^2ab}\otimes P_\alpha)\ar[u]\ar[dr]_{(2)}&&H^0(L^{\otimes mb })\otimes H^0(L^{\otimes m^2b^2+m^2ab}\otimes P_\alpha)\ar[u]_{(4)} \\
&H^0(L)^+\otimes H^0(L^{\otimes mb-1})\otimes H^0(L^{\otimes m^2b^2+m^2ab}\otimes P_\alpha)\ar[ur]_{(3)}&
}
\]
\caption{Commutative Diagram}\label{fig:my_label}
\end{small}
%\end{minipage}
%\end{turn}

        
    \end{figure}
    \end{landscape}
%\end{tiny}
We have that (2) is just the map \eqref{eq:onto}, which is surjective by assumption.  Now recall that $L\simeq M^{\otimes 2}$ with $M$ ample and symmetric. Thus, by \cite{Khaled1993}, we have that (3) is surjective as soon as $2(mb-1)\geq 3$. When $mb\geq 3$ we also have that (4) is surjective by classical results on projective normality of line bundles on abelian varieties (see \cite{Koizumi}). In particular, we deduce that (1), which is exactly \eqref{eq:onto+}, is surjective.


    
     
\end{proof}
As an easy corollary we get the following
\begin{cor}\label{MLandMW}
In the notation above, $M_L\langle xl\rangle$
 satisfies I.T.(0) if, and only if $M_W\langle xl\rangle$ does.\end{cor}
 \begin{proof}
     This depends from the fact that to satisfy I.T.(0) is open (cfr. \cite[Thm 5.2 (c)]{JiPa2020}). In fact, by loc. cit. we have that $M_L\langle xl\rangle$ satisfies I.T.(0) if and only if $M_L\langle (x-\varepsilon)l\rangle$ for sufficently small $\varepsilon >0$. Thanks to our statement  this implies that $M_W\langle xl\rangle$ satisfies I.T.(0).\\
     The other direction of the statement is a direct consequence of Lemma \ref{easy}.
 \end{proof}
\section{Proofs of the Main Results}\label{proofs}
\begin{proof}[Proof of Theorem \ref{TheoremA}]
    Denote by $L=M^{\otimes 2}$, then $L\simeq\pi^*A$ and the class of $L$ in $\operatorname{NS}(X)$ is $2\lambda$. Thus we have that
    \begin{align*}
        k^+(A) &= k(2\lambda)\\
         &=\frac{\beta(2\lambda)}{1-\beta(2\lambda)}\\
         &=\frac{\frac{\beta(\lambda)}{2}}{2-\frac{\beta(\lambda)}{2}}\\
         &=\frac{\beta(\lambda)}{2-\beta(\lambda)},
    \end{align*}
    where the first equality is exactly Theorem \ref{thm:great}, and the second is \eqref{eq:kappabeta}. Suppose that $\beta(\lambda)<\frac{2}{p+2}$, then we get
    \begin{align*}
        k^+(A) &= \frac{\beta(\lambda)}{2-\beta(\lambda)}<\frac{1}{p+1}.
    \end{align*}
    We deduce from Remark \ref{rmk:useful}, that $M_W\left\langle\frac{1}{p+1}(2\lambda)\right\rangle$ satisfies I.T.(0). By Lemma \ref{lem:generalCaucci} we have that 
    $M_W^{\otimes p+1}\otimes M^{\otimes 2h}$ satisfies I.T.(0) for every $h\geq 1$. In particular
    $H^1(K_X, M_A\otimes A^{\otimes h})=0$ and we can conclude.\par
    At the same way if $\beta(\lambda)<\frac{2(r+1)}{p+r+2}$, we get that 
    \[k^+(A)<\frac{r+1}{p+1}.\]
    Thus we have that $M_W\left\langle\frac{r+1}{p+1}\right\rangle$ is I.T.(0). We apply again  Lemma \ref{lem:generalCaucci} and show that $M_W\otimes L^{\otimes h}$ is I.T.(0) whenever $h\geq r+1$, thus the statement is proven.
    \end{proof}
    \begin{proof}[Proof of Corollary \ref{corllaryA}]
As before, there is an ample line bundle $M$ on $X$ such that $\pi^*A\simeq M^{\otimes 2}$. Denote by $\lambda$ the polarization induced by $M$, we have that
    \begin{align*}
        \beta(m\lambda)=\frac{\beta(\lambda)}{m}\leq \frac{1}{m}<\frac{2}{p+2}
    \end{align*}
    whenever $2m> p+2$. Thus the statement is proven.

    \end{proof}
    \subsection{Final remarks:}
    We conclude this paper with a question.
    \begin{quote}
        What happens if $\beta(\lambda)=\frac{2}{p+2}$?
    \end{quote}
In this case we have that $k(l)=k^+(A)=\frac{1}{p+1}$ and $M_W\left\langle\frac{1}{p+1} l\right\rangle$ is GV but does not satisfies I.T.(0)  (see \cite[Thm. 5.2]{JiPa2020}) and  we do not know anything without adding extra positivity assumptions. For what it concerns abelian varieties, Ito in \cite{Ito22} proves that, if $\mathcal{I}_p\left\langle\frac{1}{p+2}\right\rangle$ is M -regular (see \cite{JiPa2020}, roughly speaking an M-regular sheaf is more positive than a GV but less positive than an I.T.(0)) then $L$ satisfies $N_p$. On might wonder if a similar result can be obtained on Kummer varieties, improving so \cite[Thm. B]{Tirabassi}, where conditions for property $N_p^r$ are given under the assumption that $M$ does not have a base divisor. In order to do that, the first obvious step is to compare M-regularity of
$M_W\left\langle x l\right\rangle$ and of $M_L\left\langle x l\right\rangle$. In \cite{Tirabassi}, the second author has shown that $M_L\langle l\rangle$ is M-regular if, and only if, $M_W\langle l\rangle$ is. This, together with Corollary \ref{MLandMW}, brings us to the following question:
\begin{question}
Is it true in general that $M_L\langle xl\rangle$
 is M-regular if, and only if $M_W\langle xl\rangle$ is?
\end{question}



    
    \bibliographystyle{alpha}
\bibliography{biblio}
   
\end{document}
