\section{Results \label{sec:results}}


\begin{table*} [t]
  \centering
  \begin{tabular}{|l||l|c|c|c|c|c|} \hline
    \multirow{2}{*}{ID} & \multirow{2}{*}{Model} & \multirow{2}{*}{Training Data} & \multicolumn{4}{c|}{WER (\%)}  \\
    \cline{4-7}
    & & & VS & SXS & TTS & Spoken Text \\ \hline
    \textit{B0} & RNN-T & Paired & 6.6 & 28.3 & 41.9 & 26.9 \\ \hline
    \textit{B1} & RNN-T & Mixed & 6.2 & 30.4 & 39.6 & 22.5 \\ \hline
    \textit{B2} & LAS & Paired & 6.7 & 25.6 & 41.4 & 27.4 \\ \hline
    \textit{B3} & LAS & Mixed & 6.7 & 24.6 & 38.4 & 24.1 \\ \hline
    \textit{B4} & LAS-JATD & Mixed & 7.0 & 26.1 & \underline{33.1} & 24.0 \\ \hline
    \textit{B5} & Deliberation & Paired & \underline{\textbf{5.7}} & 22.0 & 39.5 & 22.2 \\ \hline
    \textit{B6} & Deliberation & Mixed & 5.8 & \underline{21.9} & 34.8 & \underline{20.5} \\ \hline
    \hline
    \textit{E0} & Deliberation-JATD (Partial) & Mixed & 5.8 & 21.9 & 30.9 & 20.7 \\ \hline
    \textit{E1} & Deliberation-JATD (Full) & Mixed & \textbf{5.7} & \textbf{21.7} & \textbf{30.6} & \textbf{18.7} \\ \hline
  \end{tabular}
  \caption{Model performance comparison. Lowest baseline WER values are \underline{underlined}, lowest overall WER values are \textbf{bolded}.
}
  \label{table:results_main}
  \vspace{-0.1in}
\end{table*}

We now analyze the performance of our Deliberation-JATD model. Table \ref{table:results_main} compares the performance of our two model variants (E0, E1) with a set of baseline models (B0 to B6). All models are trained on either paired data or the mixed audio training set (described in section \ref{sec:methods_training}).

B0 is an RNN-T model and serves as a baseline trained on paired data. It is also the model we used to initialize all LAS and deliberation variants. B1 has the same architecture as B0, but trained from scratch on the mixed audio training set. The addition of TTS data to training improves RNN-T performance on the TTS, VS, and Spoken Text test sets while degrading on the SXS set.

B2 is an LAS model trained on paired data and B3 is the same model trained on the mixed audio training set. B4 is our LAS-JATD implementation. Training on mixed audio (B3) provides some modest performance improvements compared to B2. The LAS-JATD model shows the lowest WER among baseline models on the TTS set. It also improves on the Spoken Text relative to regular LAS, but degrades VS and SXS performance.

B5 and B6 are implementations of the two-pass deliberation model trained on the paired and mixed audio training sets, respectively. They show the strongest metrics on Spoken Text as well as the VS and SXS sets. They also have the lowest TTS WER among non-JATD models. Training deliberation on the mixed audio (B6), as opposed to only the paired data (B5), results in significant improvements on all but the VS test set.

We compare the aforementioned baselines against our Deliberation-JATD models: the full variant (E0) and the partial variant (E1), both trained on the mixed audio training set. Both variants show significant gains on all sets sets aside from VS, which is roughly unchanged. The full variant (E1) produces the lowest WER of all models on the SXS, TTS, and Spoken Text test sets and matches the lowest WER obtained on VS by B5. On the TTS test set, it shows a 22.5\% improvement relative to deliberation trained on paired data (B5), and a 12\% improvement relative to deliberation trained on mixed data (B6). Similar gains are seen on the Spoken Text set.

Comparing the Deliberation-JATD models, we notice that the full variant outperforms the partial variant despite the fact that the full JATD LM term (described in section \ref{sec:methods_delib_jatd}) ignores the first-pass decoder outputs, while the partial JATD LM term uses them. We speculate that partial JATD would benefit from a real/TTS bit passed to the bidirectional LSTMs that encode the first-pass decoder output. This would allow its LM component to distinguish between paired and TTS audio. We leave this as future work.

\definecolor{dark_red}{rgb}{0.8, 0, 0}
\definecolor{dark_green}{rgb}{0, 0.55, 0}
\newcommand{\ERR}[1]{{\color{dark_red}\textbf{#1}}}
\newcommand{\CORR}[1]{{\color{dark_green}\textbf{#1}}}

\begin{table} [h!]
  \centering
  \begin{tabular}{ll} \hline
    Deliberation (B5) & Full Deliberation-JATD (E1) \\ \hline \hline
    \ERR{tough trees} leasing office & \CORR{toftrees} leasing office \\ \hline
    \ERR{chow mein} jackson & \CORR{cal-maine} jackson \\
    mississippi & mississippi \\ \hline
    distance from \ERR{wanderleo} & distance from \CORR{juan dolio} \\
    to punta cana & to punta cana \\ \hline
    \ERR{nellis ford} realty & \CORR{nellysford} realty \\ \hline \hline
    \CORR{southline} & \ERR{south lyon} \\ \hline
    the mansions of & the mansions of \\
    \CORR{shadowbriar} & \ERR{shadow briar} \\
    houston texas & houston texas \\ \hline
    \CORR{delias} near me & \ERR{delia's} near me \\ \hline
  \end{tabular}
  \caption{Sample wins and losses comparing full deliberation-JATD (E1) and deliberation (B5) on the Spoken Text test set. Correct and incorrect portions highlighted in green and red, respectively.}
  \label{table:results_err}
  \vspace{-0.1in}
\end{table}


Finally, Table \ref{table:results_err} shows a sample of wins and losses when comparing deliberation (B5) to the the full variant of deliberation-JATD (E2). The deliberation-JATD wins mostly by correcting transcription errors for proper nouns such as ``toftrees'' and ``nellysford''. The losses are sometimes also related to proper nouns (e.g. ``southline'' to ``south lyon''), but mostly due to spelling errors, e.g. ``delia's'' in place of ``delias''.

