\section{Data}

MTCs in DUGs can originate from doctors' suggestions, patient education materials, or guidelines for prescription medications. We utilize three datasets in this work: the FDA dataset, the Medscape dataset, and the EHR dataset \cite{openFDA, Medscape, mtsamples}. The first two datasets are derived from patient education materials for prescription medications, while the latter is sourced from prescriptions in EHRs. We use a variety of data sources to demonstrate that our MTC formalization generalizes across DUG domains. The three datasets contain a total of $N=836$ DUGs with labeled MTCs. The labeled datasets are made publicly available to enable future research in MTC extraction\footnote{\url{https://zenodo.org/record/7712934#.ZAnurj3MJD9}}. Examples from each dataset are presented, along with their labeled MTCs, in Table \ref{mtc_examples}. 

\subfile{../tables/mtc_examples}

\subsection{FDA Dataset}
The openFDA database contains drug product labels for both prescription and over-the-counter drugs submitted to the U.S. Food and Drug Administration (FDA), with text fields such as indications for use, adverse reactions, etc \cite{openFDA}. In this work we utilize the dosage and administration text field, which contains "information about the drug product's dosage and administration recommendations, including starting dose, dose range, titration regimens, and any other clinically significant information that affects dosing recommendations" \cite{openFDA}. To obtain the FDA dataset, a random sample of 600 drug labels was taken from this database. For each of the 600 drug products sampled, the dosage and administration instructions were annotated for MTCs by two annotators. Using Krippendorff's alpha coefficient for nominal data, a common measure of inter-annotator agreement for multi-label annotations \cite{krippendorff2011computing}, this annotation resulted in an agreement of 0.74, indicating good agreement. Of the 600 dosage and administration instructions, 371 contained MTCs as defined by our CFG. We refer to these drug product dosage and administration instructions as DUGs, and we refer to these 371 labeled DUGs as the FDA dataset.

\subsection{Medscape Dataset}
The Medscape dataset is sourced from 35 real prescriptions of patients with multiple chronic diseases \cite{mtsamples}, which combined include 83 unique medications. These medications treat several chronic diseases, including but not limited to diabetes mellitus (type I and type II), bipolar affective disorder, depression, hypertension, hypotension, chronic pain, morbid obesity, osteoarthritis, and obstructive sleep apnea. For each of these medications, one or more corresponding DUGs are extracted from a DUG corpus, Medscape \cite{preum2018corpus, Medscape}. From there, the MTC annotation in the DUG was a three-step process. First, three annotators annotated each sentence in each DUG for whether that sentence contained an MTC or other medical constraints, with 99.4\% agreement among all annotators as described in \cite{preum2018corpus}. Second, using a rule base, common temporal phrases were automatically assigned to these DUGs. Finally, a single annotator normalized these automatically extracted phrases to conform to the CFG. It was feasible to assign these MTCs semi-automatically because of recurring lexical patterns of MTCs in the DUG corpus. This process resulted in 121 DUGs, each annotated with one or more normalized MTCs. % and These DUGs are manually annotated by two annotators to identify and categorize their corresponding MTCs according to the proposed CFG. \textcolor{red}{[more annotation details?] see this paper: https://aclanthology.org/L18-1190.pdf (Table 2).}

\subsection{EHR Dataset}
The EHR dataset was extracted automatically from MTSamples, a site containing a large collection of publicly-available, de-identified medical reports submitted by clinics in various medical fields, such as Gastroenterology and Pediatrics \cite{mtsamples}. These reports are submitted by many different clinicians, ensuring heterogeneity among extracted DUGs in the EHR dataset. The automatic extraction process involved searching each EHR sample for abbreviated forms of common MTCs. Healthcare professionals use medical abbreviations when writing prescriptions and medical records, some of which directly correspond to MTCs. For example, in this DUG taken from the EHR dataset "the patient has a history of lupus, currently on Plaquenil 200-mg b.i.d.", the abbreviation "b.i.d." (Latin "bis in die") means twice a day. This abbreviation maps to the frequency constraint MTC "two times a day" (MTC type 2). While there are many abbreviations in EHRs, we select 8 that map directly to MTCs. These are listed below with their matching MTCs.
\begin{enumerate}
    \item \textit{b.i.d.}: 2 times day (MTC type 2)
    \item \textit{q.d.}: 1 times day (MTC type 2)
    \item \textit{q.h.}: 1 times hour (MTC type 2)
    \item \textit{q.i.d.}: 4 times day (MTC type 2)
    \item \textit{t.i.d.}: 3 times day (MTC type 2)
    \item \textit{h.s.}: before sleep (MTC type 4)
    \item \textit{p.c.}: after eating (MTC type 4)
    \item \textit{a.c.}: before eating (MTC type 4)
\end{enumerate}
We automatically search through all the EHRs on MTSamples and extract single-sentence statements of appropriate length which include these abbreviations. Using this method, we extract 344 medical report statements and automatically assign MTC labels. %As this process is fully automatic, we calculate no annotation metrics.

\subsection{Data Characterization}
\label{sec:data_char}

There are 836 labeled DUGs across the three datasets; 371 from the FDA dataset, 121 from the Medscape dataset, and 344 from the EHR dataset. Combined these 836 DUGs contain 1,051 MTCs.% Of the 836 DUGs, 163 (19.50\%) contain multiple MTCs. While it is possible for MTCs to be negated within our proposed CFG, very few negated MTCs appear in our datasets - only 10 MTCs (1.20\%) are negated. 

The use of three datasets from different sources supports the generalizability of our novel MTC taxonomy. This taxonomy can be used to identify MTCs on both the patient and provider sides since the FDA and Medscape datasets are patient-facing while the EHR dataset is clinician-facing. Statements in the FDA and Medscape datasets typically use the 2nd person perspective when discussing the patient, e.g. "to help \textit{you} remember, use it at the same time each day." Statements about patients in the EHR dataset, however, are expressed in 3rd person, e.g. "\textit{she} was finally put on Effexor 25 mg two tablets h.s."

While each dataset contains several MTC types, the distribution of MTC types differs across datasets, as seen in Fig. \ref{fig:mtc_types_dist}.  For instance, MTC type 6 is the most common MTC in the Medscape dataset, while it does not appear in the FDA dataset. Additionally, the EHR dataset contains almost exclusively frequency MTCs (type 2); 96.51\% of MTCs in this dataset are type 2, while the rest are type 4 MTCs. Such idiosyncrasies occur as DUGs from different sources vary with underlying medical conditions and corresponding medications/drugs. This suggests that the CFG is appropriate for multiple types of DUGs. However, MTC-type distributions may vary across DUG domains. While we explore MTCs in drug product dosage and administration labels, prescription drug labels, and de-identified medical records, MTCs may occur in other DUGs such as those found on health education websites, doctor recommendations, and elsewhere.

% FIGURE
\begin{figure}[htbp]
\centerline{\includegraphics[width=\columnwidth]{../figures/mtc_dist_plot.png}}
\caption{Distribution of MTC types across the EHR, FDA, and Medscape datasets. Along the y-axis, there are the 7 different MTC types, and the height of each bar represents the percentage of the given dataset made up of that MTC type.}
\label{fig:mtc_types_dist}
\end{figure}

The ability to computationally represent MTCs is vital for downstream tasks. Hence, MTC labels provided by the annotators conform to the proposed CFG and can be represented symbolically. As an example of a downstream task that utilizes MTC extraction, consider the task of discovering whether a chronic disease patient has violated an MTC of one of their prescription medications. Based on activity patterns recognized by a human activity recognition system, a system could use MTCs extracted from the patient's prescription medication label to determine whether the patient has violated an MTC, which may lead to poor health outcomes. 