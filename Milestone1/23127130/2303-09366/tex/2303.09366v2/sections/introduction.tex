\section{Introduction}

According to the CDC, over 48\% of the US population uses at least one prescription medicine, and 24\% take three or more \cite{CDCPrescription}. However, only four out of every five new prescriptions are filled, and half of those are administered inappropriately \cite{osterberg2005adherence}. Non-adherence includes incorrectly taking medication concerning a prescription's suggested time, dosage, frequency, or duration. Non-adherence also includes the mistiming of medication intake with respect to other activities when medication efficacy is temporally dependent on those activities, e.g. eating, exercising, or sleeping \cite{osterberg2005adherence, ingersoll2008impact, ferguson2017barriers, klakegg2018assisted}. We refer to any temporal constraints associated with medications as medical temporal constraints (MTCs). Non-adherence to MTCs is linked to higher hospital admission rates, increased morbidity, higher healthcare expenses, poor health outcomes, and even death \cite{dimatteo2004variations, chisholm2012cost, CDCPrescription2, ferguson2017barriers, klakegg2018assisted}. The effect of violating MTCs can range from minor discomfort to emergency room visits \cite{pham2011national}.

MTCs are found in drug usage guidelines (DUGs), or medication guidelines. These textual guidelines appear in both formal patient education materials (e.g., drug labels or public health websites \cite{openFDA, Medscape}), as well as in clinical texts (e.g., prescriptions and after-visit summaries recorded in electronic health records (EHRs) \cite{mtsamples}). The variety of MTC sources calls for a generalizable approach to extract and normalize MTCs from such heterogeneous sources. 

Although MTCs are critical for medical safety and treatment adherence, to our knowledge, there is no existing solution to formulate and model patient-centric MTCs. This requires (i) creating a flexible and robust computational representation of MTCs, (ii) a dataset of natural language descriptions of MTCs annotated with their computational representations, and (iii) a generalizable solution for mapping descriptions of MTCs to their corresponding computational representations. Addressing these challenges can enhance intelligent systems that improve medication adherence and patient safety \cite{roca2021validation, klakegg2018assisted, preum2021review, stankovic2021challenges} or text-based solutions to recommend safe, personalized health information \cite{preum2017preclude2, preclude}.
% patient-centric system of this nature can enhance intelligent systems by targeting medication adherence and patient safety.

We formulate and model MTCs for treatment adherence and health safety, in addition to benchmarking the task of extracting MTCs from DUGs. Specifically, (i) we develop a novel taxonomy of potential MTCs and a novel context-free grammar (CFG) based model to represent MTCs from unstructured DUGs computationally. Next, (ii) the taxonomy and CFG are used to label MTCs in three datasets of free-format textual DUGs from heterogeneous sources. Finally, (iii) we define and benchmark the MTC extraction and normalization task using state-of-the-art in-context learning (ICL) strategies, achieving an average F1 score of $0.62$ across all datasets. Recent work has demonstrated the generalizability of ICL for extracting health information in the few-shot setting \cite{agrawal2022large, dunn2022structured, torii2023task}. ICL utilizes a large language model (LLM) to perform a task by conditioning on a few input-output examples. We also compare ICL to a rule-based baseline model, explore several prompting techniques for ICL, and conduct a thorough error analysis to determine the scope of ICL for this new, safety-critical medical NLP task. 
% The solution and resources developed for this task can enable various intelligent health assistants to increase treatment adherence and patient safety \cite{preum2021review, stankovic2021challenges}. 


%  \textcolor{blue}{Should mention/summarize our other experiments in a sentence to demonstrate the depth and rigor of our work, e.g., "(iv) We also explore ABCD to determine the scope of ICL for this new and safety-critical medical NLP task."} % total macro average = 0.59

% TABLE
%\subfile{../tables/ex_table}

% FIGURE
%\begin{figure}[htbp]
%\centerline{\includegraphics{../figures/ex_fig.png}}
%\caption{Example of a figure caption.}
%\label{fig}
%\end{figure}