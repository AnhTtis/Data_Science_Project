\section{Related Works}

\subsection{Medical Information Extraction Tasks and Context-Free Grammars}

While our work is the first to formalize the MTC extraction task, the broader field of medical information extraction is a vibrant area of research. Most related to the MTC extraction task, much prior work has been done in modeling and extracting temporal and medication information in medical and health texts. A common task is the extraction of temporal relations between clinical events, such as problems and treatments, in discharge summaries \cite{sun2013evaluating, alfattni2020extraction}. Another is identifying medication information such as drug names, strengths, and routes in electronic medical records \cite{xu2010medex, wei2020study}. MTC extraction is similar to both of these tasks, however it is a new task that is far more patient-centric, in that it focuses on extracting temporal constraints placed on patient activity patterns in DUGs.

Our work additionally focuses on modeling extracted MTCs using a context-free grammar. The modeling of temporal phenomena in medical text using CFGs has been leveraged by Hao et al. \cite{pan2020temporal}, who introduce a model to leverage CFG to extract temporal expressions in clinical texts. In a similar work, Viani et al. utilize a CFG to parse mental health records and extract duration of untreated psychosis \cite{viani2020temporal}. Our work is the first to use a CFG to define and extract MTCs in DUGs, and we are the first to experiment with ICL for this task.


\subsection{In-Context Learning for Medical Information Extraction}

Agrawal, Hegselmann, Lang, Kim, and Sontag have shown that LLMs are able to extract clinical information from medical text in both the few-shot and zero-shot settings \cite{agrawal2022large}. Specifically, they show that given inputs of clinical discharge summaries or medical abstracts, along with guided prompts, GPT-3 \cite{brown2020language} is able to competently perform many medical information extraction tasks such as clinical sense disambiguation, biomedical evidence extraction, and medication extraction. We experiment with similar strategies to benchmark the MTC extraction and normalization task. Related works that utilize ICL for structured scientific information extraction include \cite{dunn2022structured}, which extracts entities and entity relationships from scientific documents into json format, and \cite{torii2023task}, which formulates the task of extracting social determinants of health from clinical narratives.