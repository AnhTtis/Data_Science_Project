\documentclass[a4paper,fleqn]{cas-sc}

\usepackage[authoryear]{natbib}
%\usepackage[numbers]{natbib}
\usepackage{url,hyperref,lineno,microtype,subcaption}
\usepackage[onehalfspacing]{setspace}
\usepackage{graphicx}
\newcommand{\tabitem}{~~\llap{\textbullet}~~}
\usepackage{indentfirst}
\usepackage{array,multirow}
\usepackage{fancyhdr}
\usepackage{float}
\usepackage{booktabs}
\usepackage{multirow}
\usepackage{here}
\usepackage{comment}
\usepackage{siunitx}
\usepackage{algorithmicx,algorithm,algpseudocode}
\usepackage{amsmath,amsfonts,bm}
\usepackage{fontawesome5}
\usepackage{caption}
\usepackage{xcolor}
\usepackage{CJKutf8}

\usepackage[figuresright]{rotating}
\renewcommand{\thefootnote}{\roman{footnote}} 
\usepackage[font=small,labelfont=bf]{caption}
\usepackage[normalem]{ulem}
\useunder{\uline}{\ul}{}

\setlength{\skip\footins}{8pt}
\newcommand{\etal}{et~al.\ }
\newcommand{\eg}{e.\,g.,\ }
\newcommand{\ie}{i.\,e.,\ }
\newcommand{\wrt}{w.\,r.\,t.\ }
\newcommand{\etc}{\textit{etc}.}

\usepackage{hyperref}
\hypersetup{
    colorlinks=false,
    linkcolor=blue,
    filecolor=magenta,      
    urlcolor=cyan,
    pdftitle={Transportation Research Part F_eHMI_LIU_2022},
    pdfpagemode=FullScreen,
    }
    
%%%Author definitions
\def\tsc#1{\csdef{#1}{\textsc{\lowercase{#1}}\xspace}}
\tsc{WGM}
\tsc{QE}
\tsc{EP}
\tsc{PMS}
\tsc{BEC}
\tsc{DE}

\renewcommand\linenumberfont{\normalfont\bfseries\footnotesize}


\begin{document}
\let\WriteBookmarks\relax
\def\floatpagepagefraction{1}
\def\textpagefraction{.001}

\shorttitle{eHMI pre-instruction for Pedestrian-AV interaction}
\shortauthors{Liu et~al.}

\title [mode = title]{Pre-instruction for Pedestrians Interacting Autonomous Vehicles with an eHMI: Effects on Their Psychology and Walking Behavior}

\author[1,3]{Hailong Liu}[type=editor,
                        auid=000,bioid=1,
                        orcid=0000-0003-2195-3380]

% Corresponding author indication
\cormark[1]
%\corref{cor1}%


% Email id of the first author
\ead{liu.hailong@is.naist.jp}
%  Credit authorship
%\credit{Conceptualization of this study, Methodology, Software}
% Address/affiliation
\affiliation[1]{organization={Graduate School of Science and Technology, Nara Institute of Science and Technology},
    addressline={8916-5 Takayama-cho}, 
    city={Ikoma},
    state={Nara},
    postcode={630-0192}, 
    country={Japan}
    }

% Second author
\author[2,3]{Takatsugu Hirayama}
\affiliation[2]{organization={Faculty of Environmental Science, University of Human Environments},
    addressline={6-2, Kamisanbonmatsu, Motojuku-cho}, 
    city={Okazaki},
    state={Aichi},
    postcode={444-3505}, 
    country={Japan}}

\affiliation[3]{organization={Graduate School of Informatics, Nagoya University},
    addressline={Furo-cho, Chikusa-ku}, 
    city={Nagoya},
    state={Aichi},
    postcode={464-8601}, 
    country={Japan}}

% Third author
%\author[4]{Masaya Watanabe}
%\affiliation[4]{organization={Vehicle Development Center, Toyota Motor Corporation},
%    addressline={Toyota-cho}, 
%    city={Toyota},
%    state={Aichi},
%    postcode={471-8572}, 
%    country={Japan}}
    
\cortext[cor1]{Corresponding author}


\begin{abstract}
External human-machine interface~(eHMI) refers to a novel and explicit communication method for pedestrian--autonomous vehicle~(AV) negotiation in interactions, such as in encounter scenarios.
However, pedestrians with limited experience in negotiating with AVs
could lack a comprehensive and correct understanding of the information on driving intentions' meaning as conveyed by AVs through eHMI, particularly in the current contexts where AV and eHMI are not yet mainstream.
Consequently, pedestrians who misunderstand the driving intention of the AVs during the encounter may feel threatened and perform unpredictable behaviors.

To solve this issue, this study proposes using the pre-instruction on the rationale of eHMI to help pedestrians correctly understand driving intentions and predict AV behavior. Consequently, this can improve their subjective feelings (\ie sense of danger, trust in AV, and sense of relief) and decision-making. In addition, this study suggests that the eHMI could better guide pedestrian behavior through the pre-instruction.

The results of interaction experiments in the road crossing scene show that participants found it more difficult to recognize the situation when they encountered an AV without eHMI than when they encountered a manual driving vehicle~(MV); in addition, participants' subjective feelings and hesitations while decision-making worsened significantly. 
After the pre-instruction, the participants could understand the driving intention of an AV with eHMI and predict driving behavior more easily. Furthermore, the participants' subjective feelings and hesitation to make decisions improved, reaching the same criteria used for MV.
Moreover, this study found that the information guidance of using eHMI influenced the participants' walking speed, resulting in a small variation over the time horizon via multiple trials when they fully understood the principle of eHMI through the pre-instruction.
\end{abstract}

% Keywords
\begin{keywords}
Autonomous vehicle \sep Human-AV communication \sep External human-machine interface~(eHMI) \sep Transportation education
\end{keywords}

%\begin{highlights}
%\item By the pre-instruction, eHMI could help pedestrians understand and predict AV driving intentions and driving behaviors.
%\item By the pre-instruction, eHMI could improve the subjective feelings of pedestrians towards the AV during encounter with it.
%\item By the pre-instruction, eHMI could reduce the pedestrians' hesitation to make decisions when encountering the AV.
%\item By the pre-instruction, eHMI could better guide the behaviors of pedestrians with a small variation.
%\end{highlights}

\maketitle
%\linenumbers 

\section{INTRODUCTION}

In the interaction between manual driving vehicles~(MVs) and pedestrians, the intention of MVs is often understood through an implicit communication approach, such as vehicle speed, acceleration, and driving direction~\citep{rasouli2019}.
In addition, an explicit communication approach, such as a head nod or hand gesture, helps pedestrians understand the driver's intention and resolve the ambiguity of negotiation during interactions~\citep{sucha2017pedestrian,farber2016communication}.
In particularly, in complex urban environments and shared spaces where explicit communication methods are frequently used between MV's drivers and pedestrians today~\citep{Vissers2016}.
For example, when a pedestrian encounters an MV on a narrow road or in a parking lot without traffic lights, the driver can clearly communicate their intention and quickly reach an agreement to negotiate with the pedestrian through eye contact, hand gestures, and verbal communication.

Researchers expect autonomous driving vehicles~(AVs) are expected to become popular in the near future~\citep{Vissers2016,rasouli2019}.
Under this assumption, some traffic scenarios are expected to include combining AV and pedestrians in shared spaces, intersections without signals, narrow roads, and parking lots.
However, AV drivers with levels 3--5 automated systems are usually partly or not responsible for driving tasks as opposed to MV drivers~\citep{SAE_j3016_2016}.
Consequently, problems may arise such as difficulties sharing driving intentions of AV to pedestrians and negotiating the right-of-way with them~\citep{li2021autonomous}.
This lack creates difficulty for pedestrians to understand the intentions of AVs quickly and clearly~\citep{liu2020when,Liu2022_APMV}.
Moreover, potential issues are caused, such as safety hazards, inefficiency, and poor prosociality~\citep{batson2003altruism}.
Thus, understanding how AVs should communicate with pedestrians has become a pressing concern.


\subsection{Issues with external human-machine interface}

A new communication approach using an external human-machine interface~(eHMI) is considered one of the solutions to the issues in pedestrian-AV interactions~\citep{schieben2019designing}. 
In particular, various studies evaluated the effectiveness of eHMIs in presenting the intentions of AVs to pedestrians using light bars, icons, and text~\citep{rettenmaier2019passing,Stefanie2020,faas2020external,dey2021communicating}.
Although these studies advocated for good eHMI designs, they could not ensure that pedestrians fully recognize the intentions of AVs, the rationale underlying the intentions of AVs, and the functional limitations of AVs through these eHMIs. 
Particularly, pedestrians with limited experience in interacting with an AV equipped with an eHMI were more affected~\citep{Michal2020,lee2022learning}.
For example, even if pedestrians clearly recognize that the AV has yielded the right of way through the message ``you go first'' displayed on the eHMI, their understanding may differ regarding when, how, or why the AV displays this information, especially when the sensor range is unknown. 
Furthermore, pedestrians may be uncertain of the time available to cross the road when the message is displayed because it is difficult to predict when the vehicle will start moving again.
Pedestrians who find it difficult to understand AVs' driving intentions and predict AVs' driving behavior may feel threatened by AVs and respond with unpredictable behaviors.



\begin{figure}[t]
\centering
\includegraphics[width=1\linewidth]{img/model.pdf}
\caption{Pre-instruction for the calibration of the mental model based on the cognition-decision-behavior model proposed in~\citep{Liu2022_APMV}.}
\label{fig:model}
\vspace{-5mm}
\end{figure}

The above-mentioned problem of eHMIs can be solved by increasing the experience of pedestrians interacting with AVs to understand the intention of the AV from the information on eHMI~\citep{Michal2020,lee2022learning}.
However, this method usually requires pedestrians to pay for the time to learn the interactions with AVs.
Furthermore, pedestrians may experience further risk because they do not fully understand the intention of the AV during this learning process.


Several related studies have reported that instructions for an in-vehicle human-machine interface~(iHMI) of the AV have enabled drivers to improve their understanding of using the AV, perform interactively with the AV, and trust in the AV~\citep{hergeth2017prior, forster2019user, edelmann2020effects}. 
However, owing to the growing popularity of AVs, instructing pedestrians to understand the intention of the AV correctly during interactions presents an urgent issue that is yet to be studied widely.


\subsection{Human cognition and knowledge learning}

This study suggests that the mechanism of the cognition-decision-behavior process of pedestrians is important during interactions with an AV to help pedestrians correctly understand the intentions of AVs through an eHMI.
Therefore, this study focuses on the model proposed by~\citep{liu2020when,Liu2022_APMV} as presented in Fig.~\ref{fig:model}.
This model consists of three parts, namely, situational awareness, risk evaluation based on hazard perception, and decision-making for behavior generation based on risk homeostasis. 
The situational awareness of a pedestrian includes the ability to perceive objects in the surrounding environment (\ie perception), understand the state and intention (\ie comprehension), and predict the future state (\ie projection). 
Thereafter, the pedestrian perceives the hazards based on the prediction results and evaluates the risk (\ie subjective risk).
Subsequently, the pedestrian determines their behavior by comparing the subjective risk with the acceptable risk level (\ie target risk). 
This risk compensation process can be explained by the risk homeostasis theory~\citep{wilde1982theory}.


The mental model is a highly organized and dynamic knowledge structure that is an internal representation of a target system that contains meaningful declarative and procedural knowledge derived from long-term experiences and studies~\citep{jones2011mental,Al-Diban2012}. 
The model generates spontaneously by recognizing and interpreting the target system repeatedly~\citep{staggers1993mental}. 
Furthermore, the situation model is the current instantiation of the mental model~\citep{endsley2000}. 
Particularly, the situation model can be viewed as a prediction model constructed and supported by an underlying mental model~\citep{endsley1995toward,mogford1997mental}. 
This study suggests that the mental model provides the situational model with some prior information and knowledge to guide it in performing comprehension and projection in specific situations.




\section{Purpose and Contributions}\label{sec:PURPOSE}

Based on the above studies, the purpose of this study is that uses an essential approach to solving the problems addressed to ensure pedestrians quickly establish the correct mental model of an AV with an eHMI through the pre-instruction.
In particular, the approach is proposed to instruct pedestrians to understand the execution conditions, mechanisms, and functional limitations of the AV with eHMI. 
We reported a brief description of the proposed method and preliminary results on pedestrians' subjective evaluations in~\citep{liu2021importance} 
Based on this pre-study, this study not only further discusses the impact of the proposed method on the subjective evaluations, but also improves the analysis methods and further analyzes the walking behaviors of the pedestrians.
Therefore, this study aims a subject experiment to verify the effects of the proposed approach on pedestrian psychology and walking behaviors.
%Furthermore, this approach can help pedestrians acquire situation awareness to interact with the AV, correctly understand the intentions of the AV, and accurately predict the behaviors of the AV.



The contributions of this study are as follows:
This study proposed using the pre-instruction to help pedestrians establish a correct mental model of an eHMI on an AV.
Based on a cognition-decision-behavior model of pedestrians, this study investigated changes in situational awareness and various subjective evaluations of pedestrians when interacting with vehicles under various scenarios. 
The experimental results verified the effectiveness of eHMI for the interaction between pedestrians and AVs.
Furthermore, the findings showed that the pre-instruction of the rationale of eHMI could improve situational awareness, subjective feelings, and decision-making hesitancy among pedestrians regarding AV.
Moreover, this study found that the eHMI could better guide pedestrian behavior through the pre-instruction.
We outlook that this approach will increase the social acceptance of AV.

\section{HYPOTHESIS}
In this study, our objective is to analyze the influence of the pre-instruction of the eHMI's rationale on the situational awareness, subjective feelings and walking behavior of pedestrians when they encounter the AV.
We propose the following hypothesis:

\begin{enumerate}
\item[H~1:] Pedestrians who correctly understand the rationale of eHMI through the pre-instruction exhibit improved situational awareness, subjective feelings, and decision-making during the interaction.
\item[H~2:] Pedestrians who correctly understand the rationale of eHMI through the pre-instruction exhibit resulted the participants' walking speed in a small variation over the time horizon via multiple trials.
\end{enumerate}


Based on the above arguments, this study proposes the accurate formation and calibration of a mental model of the AV with the eHMI by instructing pedestrians with relevant knowledge of the eHMI to improve their experience of interaction with the AV.
An experiment was designed to verify the effectiveness of the eHMI for pedestrian-AV interaction.
In addition, this study demonstrated the importance of the pre-instruction in improving the situational awareness, subjective feelings, and decision-making of pedestrians during the interaction.




\section{The WIZARD OF OZ EXPERIMENT}\label{sec:experiment}

This experiment simulates a situation where a pedestrian encounters a car in a parking lot, as shown in Fig.~\ref{fig:map}. Specifically, the experimental participants drive to a shopping center and park in an underground parking lot. Thereafter, the participants encounter an AV or MV crossing the road to use an elevator. When the participants reach the roadway, the vehicle stops at the stop line. In this case, the participants must decide whether to cross the road because there is no clear rule that prioritizes pedestrians in the parking lot.
This experiment was conducted in a blocked area of the B2F parking lot of Toyota Stadium, Toyota-shi, Aichi, Japan.
%The safety of this experimental site was ensured by blocking access to ordinary people.
This experiment complied with the Declaration of Helsinki and it was approved by the ethics review committee of the Institute of Innovation for Future Society, Nagoya University (No.~2021-12).
Informed consent was obtained from each participant.



\begin{figure}[tb]
\centering
\includegraphics[width=0.8\linewidth]{img/map.pdf}
\caption{Experimental scene: simulation of pedestrian encounters with a car in a parking lot.}
\label{fig:map}
\end{figure}


\subsection{Experimental car and eHMI}

\begin{figure}[tb]
\centering
\includegraphics[width=1\linewidth]{img/exp_car.pdf}
\caption{Experimental car with an eHMI: a left-hand-driven car stimulating a right-hand-driven AV. An eHMI is installed on the right rear of the windshield.}
\label{fig:exp_car}
\end{figure}


This study applied a Wizard of Oz experimental design to ensure the safety of the experiment. 
Using a ghost driver trick~\citep{rothenbucher2016ghost}, a driver-less AV was simulated using an MV driven by a skilled driver. 
In general, cars in Japan are right-hand-drive cars. Therefore, the AV had to be simulated using the left-hand drive car. 
Specifically, Fig.~\ref{fig:exp_car} shows the experimental car used in this study, which is a left-hand drive car (Toyota Prius). In the left seat of the experimental car, an expert driver was hidden by a mirror film. Additionally, a dummy steering wheel was installed on the right side to ensure that the participants assume that the experimental car was a right-hand-drive car. The maximum speed of the car was limited to 8~km/h and the average speed was approximately 4~km/h. Particularly, this speed considered the experimental scene of the parking lot and the safety of the participants.

An eHMI device was installed on the right rear of the windshield (see the left part of Fig.~\ref{fig:exp_car}). The message ``\begin{CJK}{UTF8}{ipxm}動きません\end{CJK}'' (``\textit{UGOKIMASEN}'') appears immediately on the eHMI after the AV stops. 
This message indicates that the car is stationary at that specific moment.
After a pedestrian crosses the road, the eHMI blinks twice in two seconds intervals indicating that the car will move. After blinking, the eHMI is turned off and the car moves.


There are two reasons to consider the above configuration:
(1) Eliminate liability for potential accidents by not allowing AV to command pedestrians.
``\textit{UGOKIMASEN}'' only indicates to the pedestrians the current driving state and intention of the car.
However, the comment does not inform the participants on what to do, for example, ``You go first.'' 
The pedestrians need to decide their walking behavior based on the state of the car and the message on the eHMI. 
They are responsible for their decisions and behaviors.
(2) Although text-based eHMI convey clear information that may be perceived as not requiring the pre-instruction, this message ``\textit{UGOKIMASEN}'' provides pedestrians with a vague understanding of the AV's intention in the specific Japanese context. 
Thus, this message can help to compare the effectiveness of the pre-instruction in this experiment.
For example, a pedestrian might think that the AV is asking for help because the car is out of order when the ``\textit{UGOKIMASEN}'' message appears after the car has stopped.
In addition, the blinking conditions and timing of the message remain unclear to the pedestrians before the pre-instruction. 
Thus, this ambiguity presents an important reason for instructing pedestrians.


\subsection{Pedestrians and prior introduction}
There were 32 participants who participated in this experiment as pedestrians.
This experiment comprised 32 pedestrian participants aged from 23 to 68 years (mean: 49.12, standard deviation: 11.13). 
Furthermore, 17 participants were women while the remaining 15 were men.

The following information was introduced to the participants before the experiment commenced:
\begin{enumerate}
\item Imagine driving a car to a shopping center, and you want to park your car in the underground parking lot.
\item After parking, you want to cross the road to reach the elevator.
Please walk at a normal speed during this process (see the dotted line in Fig.~\ref{fig:map}).
\item When you cross the road, a manual driving car (\ie MV) or an automated driving car (\ie AV) will arrive. You should be mindful of it when crossing the road.
\item The AV is a driverless car with multiple built-in advanced sensors that can detect the surrounding environment, such as pedestrians, roads, and stop lines. \textbf{(False information)}
\item The pylon marks the stop line (see Fig.~\ref{fig:map}). The MV or AV will stop before the stop line. Subsequently, the vehicle will decide whether to depart from the surrounding situation based on the presence of a pedestrian.
\end{enumerate}


Notably, any information about eHMI was withheld from the participants in this study before receiving the introduction. Furthermore, the AV interaction scenarios described in the following subsection were not explained to the participants so they could not easily predict the content of each trial.



\subsection{Design of interaction scenarios}

\begin{figure}[b]
\centering
\includegraphics[width=1\linewidth]{img/process.pdf}
\caption{Procedure of a within-participants design experiment.}
\label{fig:procedure}
\end{figure}


Fig.~\ref{fig:procedure} reveals that the pedestrians interact with the experimental car in four designed scenarios: \textbf{\textit{MV}}, \textbf{\textit{AV w/o eHMI}}, \textbf{\textit{AV w/ eHMI}}, and \textbf{\textit{AV w/ eHMI after PI}} (\ie \textbf{\textit{AV w/ eHMI after pre-instuction}}).

\textbf{\textit{ MV:}}
The first scenario discusses an encounter between a pedestrian encounters an MV (see Fig.~\ref{fig:procedure}~(a)). 
The dummy driver sits in the right seat and holds the dummy steering wheel.
While the real driver controls the experimental car in the left seat covered by the mirror film. When a pedestrian encounters the MV, the pedestrian can see a dummy driver driving the experimental car. 
Furthermore, the dummy driver yields the right-of-way to the pedestrian after stopping the car by using a typical Japanese gesture of ``After You'', that is, \ie moving the driver's hand forward once with palm facing upwards (looks like \faIcon{hand-holding}).


\textbf{\textit{AV w/o eHMI}:}
The second scenario demonstrates an encounter between the pedestrian and an AV without the eHMI (see Fig.~\ref{fig:procedure}~(b)).
Notably, none of the driver sit on the right seat.
The real driver controls the experimental car in the left seat covered by the mirror film. 
No eHMI device is present in the AV. 
A sign on the hood indicates the autonomous mode (see Fig.~\ref{fig:exp_car}). 
The AV stops before the stop line (two pylons) when it encounters the pedestrian. At this time, the pedestrian has to decide how and when to cross the road. Finally, the AV starts moving after the pedestrian has completely crossed the road.

\textbf{\textit{AV w/ eHMI}:} 
This third scenario describes the pedestrian's encounter with an AV that has the eHMI (see Fig.~\ref{fig:procedure}~(c)).
The configuration of the driver is consistent with \textit{AV w/o eHMI}.
Particularly, the difference is that an eHMI device is installed behind the right side of the windshield. The AV stops before the stop line and the eHMI displays a message ``\textit{UGOKIMASEN}'' informing the pedestrian that the AV is stationary. Subsequently, the pedestrian needs to decide how and when to cross the road. The eHMI message blinks twice when the pedestrian has completely crossed the road. Lastly, the eHMI is turned off and the AV moves.


\textbf{\textit{AV w/ eHMI after PI}:} 
The fourth scenario is consistent with \textit{AV w/ eHMI} (see Fig.~\ref{fig:procedure}~(c)). 
The difference is that two types of pre-instructions are used to help the pedestrian adjust the mental models of the AV with the eHMI before the trials of this scenario. Specifically, to eliminate the participants' vague understanding of the driving intentions of the AV (\textit{comprehension}) and to help them to predict the driving behavior (\textit{projection}) in situational awareness, a document (see Fig.~\ref{fig:procedure}~(d)) explaining the meaning of the information on the eHMI is presented based on the following information:
\begin{enumerate}
\item When the AV detects the pedestrian, the message ``\textit{UGOKIMASEN}'' appear on the eHMI after the car stops (for \textit{projection}). 
\item ``\textit{UGOKIMASEN}'' indicating that the AV does not move (for \textit{comprehension}).
\item Thereafter, the eHMI will blink twice in two seconds after the pedestrian crosses the road (for \textit{projection}). 
\item The blinking indicates that the AV will move again and depart (for \textit{comprehension}).
\item After blinking, the eHMI is turned off. Subsequently, the AV will move (for \textit{projection}).
\end{enumerate}
Meanwhile, after the document-based pre-instruction, a demonstration (see Fig.~\ref{fig:procedure}~(e)) is also used to explain to the pedestrians when the eHMI would be turned on/off and eHMI would blink.
To conceal the ghost driver trick, pedestrians stood by the side of the road to watch the demonstrations. After the two types of pre-instruction described above, pedestrians are allowed to ask additional questions if they have doubts about the content of the pre-instruction. The fourth scenario, that is {\textit{AV w/ eHMI after PI}}, is performed after confirming that the pedestrians understood the contents of the pre-instruction.


\subsection{Order of scenarios}



The order of the scenarios and the pre-instruction for each participant is displayed in the central part of Fig.~\ref{fig:procedure}, that is, \textit{MV} $\rightarrow$ \textit{AV w/o eHMI} $\rightarrow$ \textit{AV w/ eHMI} $\rightarrow$ \textit{pre-instruction} $\rightarrow$\textit{AV w/ eHMI after PI}. Note that a random order was not used in this experiment because of the relationship among those four scenarios.
Specifically, the reason for placing \textit{MV} first is to familiarize the participants with the experience of encountering the vehicle at the test site and to establish a baseline for subjective evaluations.
Moreover, the  \textit{AV w/ eHMI} must be appeared after the participants had been experienced \textit{AV w/o eHMI}, to prevent them from mistakenly thinking that the eHMI is corrupted in \textit{AV w/o eHMI} scenario.
Furthermore, \textit{AV w/ eHMI after PI} has to appear after \textit{AV w/ eHMI} to conform to the current experiment’s purpose, that is, comparing the differences in participants' subjective evaluations of the eHMI-equipped AV and walking behavior before and after the pre-instruction.
More significantly, the above order complies with the order of the popularization of MV and AV, as well as eHMI.

In the experiment, each scenario was run five times. In total, each participant encountered the experimental car across 20 trials. After crossing the road in each trial, the participants were asked to sit on a chair (see Fig.~\ref{fig:map}) to complete the questionnaires and rest for approximately one minute. Particularly, the purpose of this rest is to prevent the effect of fatigue on walking behavior.

%%%%%%%%%%%%%%%%%%%%%%%%%%%

\section{EVALUATION METHODS}
This study analyzed participants' subjective feelings and walking behaviors during encounters with the experimental car in the four scenarios.

\subsection{Subjective evaluations}

Based on the cognition-decision-behavior model~\citep{Liu2022_APMV} (see Fig.~\ref{fig:model}), six questions in Japanese were designed to evaluate the subjective feelings of the participants as follows:
\begin{enumerate}
\item[Q1:] Was it easy to understand the driving intention of the car?
\item[Q2:] Was it easy to predict the behavior of the car?
\item[Q3:] Did you feel the behavior of the car was dangerous?
\item[Q4:] Did you trust the car when you crossed the road?
\item[Q5:] Did you feel a sense of relief when you crossed the road?
\item[Q6:] Did you hesitate when you crossed the road?
\end{enumerate}
As shown in Fig.~\ref{fig:model}, Q1 and Q2 are used to evaluate the comprehension and projection steps in the situation model. 
The Q3, Q4, and Q5 are used for risk evaluation while Q6 is used to evaluate decision-making hesitancy (\ie the difficulty with making a decision to cross the road).
After each trial, the participants were asked to answer the above questions on a 5-point scale: ``\textit{1=Strongly Disagree}'', ``\textit{2=Disagree}'', ``\textit{3=Undecided}'', ``\textit{4= Agree}'' and ``\textit{5=Strongly Agree}''.


\subsection{Walking behavior}

\begin{figure}[b]
\centering
\includegraphics[width=0.95\linewidth]{img/openpose.pdf}
\caption{The walking behaviors of participants are estimated using OpenPose with BODY~25 joint set. (a) shows the BODY~25 joint set. (b) Method for extracting the representative point of pedestrian position.}
\label{fig:openpose}
\vspace{-2mm}
\end{figure}



This study analyzed the walking behaviors of participants before and after stopping a car to investigate the effects of the eHMI and its pre-instruction. Note that, the eHMI showed the message immediately after the car stopped in the scenarios {\textit{AV w/ eHMI}} and {\textit{AV w/ eHMI after PI}}.

Fig.~\ref{fig:openpose} illustrates the walking behaviors of the participants recorded as a video (1600x900 pixels with 60~FPS) from the side with a camera installed on the road. The positions of the participants during the crossings were calculated using OpenPose~\citep{openPose}. Particularly, OpenPose can detect their skeletal feature points based on the BODY~25 joint set (Fig.~\ref{fig:openpose}~(a)) in each frame. Subsequently, the position of the pedestrian was determined using the feature point of the neck on the abscissa axis of the image space,(\ie $x^{i,j}_t$) given that the neck is barely affected by the swinging from walking (Fig.~\ref{fig:openpose}~(b)). Here, $t$ is the frame, while $i$ represents the $i$-th participant, and $j$ represents the $j$-th trial.
Furthermore, $j\in\{1,2,3,4,5\}$ are the trials of \textit{MV}, and $j\in\{6,7,8,9,10\}$ are the trials of \textit{AV w/o eHMI}, $j\in\{11,12,13,14,15\}$ are the trials of \textit{AV w/ eHMI} and $j\in\{16,17,18,19,20\}$ are the trials of \textit{AV w/ eHMI after PI}.
The detected positions $x^{i,j}_t$ in the range of [50 pixels, 1550 pixels] were used to ensure the detecting accuracy. Thereafter, the participants' walking speeds were calculated according to the difference in position between the two frames (\ie $s^{i,j}_t= (x^{i,j}_t-x^{i,j}_{t-1})\times 60~[pixels/second]$ because the FPS of videos was 60~[Hz]).


To analyze the walking behaviors of the participants before and after the car stopped, this study regarded the moment of the car stopping as 0 on the timeline in each trial. Furthermore, the walking speed was downsampled from 60~[Hz] to 5~[Hz] from 1 second before the car stopped to 3 seconds after the car stopped, before being analyzed in detail. Finally, the standard deviation~(Std.) of the participants' walking speeds were assessed at each time point to investigate the eHMI and the pre-instruction affecting the participants' walking behaviors. 




\section{RESULTS}

\subsection{Subjective evaluations}

\begin{figure}[b]
\vspace{-4mm}
\centering
\includegraphics[width=1\linewidth]{img/01_evaluation_new.pdf}
\caption{Results of six subjective evaluations for four scenarios from 32 participants.}
\label{fig:result_01}
\vspace{-5mm}
\end{figure}


The results of the subjective evaluation of Q1-Q6 reported by the participants for the four scenarios are shown as box plots in Fig.~\ref{fig:result_01}.
The vertical axis represents the four scenarios while the horizontal axis reflects the evaluation scales. The red lines show the median values.
A non-parametric one-way repeated measure analysis of variance (ANOVA), namely the Friedman test, was performed to test the significant differences in subjective ratings among scenarios for each question. Furthermore, post hoc multiple comparisons were performed between each pair of the four scenarios for each question using a two-sided Wilcoxon signed-rank test~(WSR) with the Bonferroni correction method. The results of the Friedman test and WSR for Q1-Q6 are shown in Fig.~\ref{fig:result_01}.

For ``Q1: Was it easy to understand the driving intention of the car?'' 
the median values of the evaluation scales for \textit{MV}, \textit{AV w/ eHMI} and \textit{AV w/ eHMI after PI} of 5.0 and 4.0 for \textit{AV w/o eHMI}. The interquartile ranges (IQRs) of evaluation scales of all scenarios were 1.0. The Friedman test of difference among the repeated evaluation scales was conducted and rendered a Q-statistic of 112.4 which was significant ($p<.001$). Furthermore, the WSR test indicated that the differences were statistically significant between \textit{MV} -- \textit{AV w/o eHMI} ($p<.001$), \textit{AV w/o eHMI} -- \textit{AV w/ eHMI} ($p<.001$), \textit{AV w/o eHMI} -- \textit{AV w/ eHMI after PI} ($p<.001$) and \textit{AV w/ eHMI } -- \textit{AV w/ eHMI after PI} ($p<.001$).

For ``Q2: Was it easy to predict the behavior of the car?''
the median values of the evaluation scales were 5.0 for \textit{MV} and \textit{AV w/ eHMI after PI} and 4.0 for \textit{AV w/o eHMI} and \textit{AV w/ eHMI}. The IQRs of the evaluation scales for all scenarios were 1.0. Friedman test of difference among the repeated evaluation scales was conducted and rendered a Q-statistic of 102.1 which was significant ($p<.001$). The WSR test for Q2 indicated that the differences were statistically significant between \textit{MV} -- \textit{AV w/o eHMI} ($p<.001$), \textit{AV w/o eHMI} -- \textit{AV w/ eHMI} ($p<.001$), \textit{AV w/o eHMI} -- \textit{AV w/ eHMI after PI} ($p<.001$) and \textit{AV w/ eHMI } -- \textit{AV w/ eHMI after PI} ($p<.001$).

For ``Q3: Did you feel the behavior of the car was dangerous?''
the median values of the evaluation scales were 1.0 for \textit{MV} and 2.0 for \textit{AV w/o eHMI}, \textit{AV w/ eHMI} and \textit{AV w/ eHMI after PI}. The IQRs of the evaluation scales for all scenarios were 1.0. Friedman test of difference among the repeated evaluation scales was conducted and rendered a Q-statistic of 64.7 which was significant ($p<.001$). The WSR test for Q3 indicated that the differences were statistically significant between \textit{MV} -- \textit{AV w/o eHMI} ($p<.01$), \textit{AV w/o eHMI} -- \textit{AV w/ eHMI} ($p<.001$) and \textit{AV w/o eHMI} -- \textit{AV w/ eHMI after PI} ($p<.001$).

For ``Q4: Did you trust the car when you crossed the road?'' the median values of the evaluation scales were 5.0 for \textit{MV}; 4.5 for \textit{AV w/ eHMI after PI}; 4.0 for \textit{AV w/o eHMI} and \textit{AV w/ eHMI}. The IQRs of evaluation scales for all scenarios were 1.0. Friedman test of difference among the repeated evaluation scales was conducted and rendered a Q-statistic of 114.7 which was significant ($p<.001$). The WSR test for Q4 indicated that the differences were statistically significant between \textit{MV} -- \textit{AV w/o eHMI} ($p<.001$), \textit{MV} -- \textit{AV w/ eHMI} ($p<.01$), \textit{AV w/o eHMI} -- \textit{AV w/ eHMI} ($p<.001$), \textit{AV w/o eHMI} -- \textit{AV w/ eHMI after PI} ($p<.001$).


For ``Q5: Did you feel a sense of relief when you crossed the road?''
the median values of the evaluation scales were 5.0 for \textit{MV} and \textit{AV w/ eHMI after PI} and 4.0 for \textit{AV w/o eHMI} and \textit{AV w/ eHMI}. The IQRs of the evaluation scales for all scenarios were 1.0. Friedman test of difference among the repeated evaluation scales was conducted and rendered a Q-statistic of 103.4 which was significant ($p<.001$). The WSR test for Q5 indicated that the differences were statistically significant between \textit{MV} -- \textit{AV w/o eHMI} ($p<.001$), \textit{MV} -- \textit{AV w/ eHMI} ($p<.01$), \textit{AV w/o eHMI} -- \textit{AV w/ eHMI} ($p<.001$), \textit{AV w/o eHMI} -- \textit{AV w/ eHMI after PI} ($p<.001$), \textit{AV w/ eHMI} -- \textit{AV w/ eHMI after PI} ($p<.05$).


For ``Q6: Did you hesitate when you crossed the road?''
the median values of the evaluation scales were 2.0 for \textit{MV}, \textit{AV w/ eHMI} and \textit{AV w/ eHMI after PI} but 3.0 for \textit{AV w/o eHMI}. The IQRs of the evaluation scales for all scenarios were 2.0. Friedman test of difference among the repeated evaluation scales was conducted and rendered a Q-statistic of 87.8 which was significant ($p<.001$). The WSR test for Q6 indicated that the differences were statistically significant between \textit{MV} -- \textit{AV w/o eHMI} ($p<.001$), \textit{AV w/o eHMI} -- \textit{AV w/ eHMI} ($p<.001$), \textit{AV w/o eHMI} -- \textit{AV w/ eHMI after PI} ($p<.001$).




\subsection{Walking behaviors before and after stopping the car}



The walking speeds of 27 pedestrians during the 5 trials of the 4 scenarios are shown in Fig.~\ref{fig:walk_speed}. The walking speeds of the pedestrians were calculated from the difference in position between every two frames. Particularly, the figure shows that each participant's walking speed differs. Consequently, this study found a difference in the times for the participants to enter and exit the crossing scene. In particular, after the car stopped for more than three seconds, the Std. of the walking speeds increased as more participants completed the crossing, especially for \textit{AV w/o eHMI} (see Fig.~\ref{fig:walk_speed}).

\begin{figure}[b]
\centering
\includegraphics[width=0.75\linewidth]{img/01_walk_time_speed3.pdf}
\caption{Walking speeds of 27 pedestrians during 5 trials of four scenarios. The moment of the car stops is regarded as 0 on the abscissa axis, \ie the time line. The vertical axis reports the walking speed of the participants. }
\label{fig:walk_speed}
\vspace{-3mm}
\end{figure}


The walking speeds between 1 second before and 3 seconds after the car stopped, that is [-1.0, 3.0] seconds, were selected for further analysis to address this problem. Furthermore, the walking speeds in this time range were downsampled from 60 [Hz] to 5 [Hz] to obtain concise results from the analysis. The mean and Std. of walking speeds in each scenario are shown in Table~\ref{Tab:mean_std_speed}.
The bold and underlined values indicate the maximum and minimum values of mean and Std. for each time point, respectively.
The mean of walking speeds was highest in \textit{AV w/ eHMI} and lowest in \textit{MV} before the car stopped.
After the car stopped, the participants had the lowest mean of walking speeds in \textit{AV w/ eHMI after PI}.
In contrast, the participants had the highest mean of walking speeds in \textit{MV} after the car stopped for one second. Furthermore, the Std. of walking speeds was the lowest in \textit{AV w/ eHMI after PI} throughout the crossing, \ie from -1.0~[s] to 1.0~[s] and from 2.0~[s] to 3.0~[s]. Moreover, the highest Std. of walking speeds was recorded from -1.0~[s] to 1.2~[s] in \textit{MV} and from 1.2~[s] to 1.8~[s] in \textit{AV w/ eHMI after PI} as well as from 2.0~[s] to 3.0~[s] in \textit{AV w/o eHMI}.



\begin{table}[hb]
\caption{The means and standard deviations (Std.) of walking speeds in [-1.0, 3.0] seconds. The moment the car stops is represented by 0 s. The maximum value and minimum values are indicated, respectively, using a bold font and an underline.}
\label{Tab:mean_std_speed}
\setlength{\tabcolsep}{3.6mm}{
\begin{tabular}{@{}c|cccc|cccc@{}}
\toprule
 & \multicolumn{4}{c|}{Mean of walking speeds [pixels/s]} & \multicolumn{4}{c}{Std. of walking speeds [pixels/s]} \\ \cmidrule(l){2-9} 
\begin{tabular}[c]{@{}c@{}}Time {[}s{]}\end{tabular} & \multicolumn{1}{c}{MV} & \multicolumn{1}{c}{\begin{tabular}[c]{@{}c@{}}AV w/o\\ eHMI\end{tabular}} & \multicolumn{1}{c}{\begin{tabular}[c]{@{}c@{}}AV w/\\ eHMI\end{tabular}} & \multicolumn{1}{c|}{\begin{tabular}[c]{@{}c@{}}AV w/ eHMI\\ after PI\end{tabular}} & \multicolumn{1}{c}{MV} & \multicolumn{1}{c}{\begin{tabular}[c]{@{}c@{}}AV w/o\\ eHMI\end{tabular}} & \multicolumn{1}{c}{\begin{tabular}[c]{@{}c@{}}AV w/\\ eHMI\end{tabular}} & \multicolumn{1}{c}{\begin{tabular}[c]{@{}c@{}}AV w/ eHMI\\ after PI\end{tabular}} \\ \midrule

-1.0 & {\ul 210.3} & 231.5 & \textbf{245.5} & 230.6 & \textbf{72.82 }& 60.86 & 61.94 & {\ul 56.13} \\
-0.8 & {\ul 208.2} & 226.1 & \textbf{244.7} & 234.9 & \textbf{79.33 }& 73.30 & 69.96 & {\ul 63.47} \\
-0.6 & {\ul 206.9} & 225.6 & \textbf{236.0} & 231.3 & \textbf{91.18} & 83.76 & 77.87 & {\ul 71.25} \\
-0.4 & {\ul 207.4} & 228.7 & \textbf{235.0} & 231.1 & \textbf{103.8 }& 95.55 & 88.13 & {\ul 82.20} \\
-0.2 & {\ul 216.4} & 232.8 & \textbf{236.8} & 229.9 &\textbf{ 112.9 }& 110.5 & 103.6 & {\ul 90.35} \\
0.0 & 229.7 & 239.1 & \textbf{239.2} & {\ul 228.3} & \textbf{128.6 }& 123.8 & 120.8 & {\ul 105.8} \\
+0.2 & 258.4 & \textbf{258.8} & 254.3 & {\ul 234.9} & \textbf{141.5}& 137.9 & 137.5 & {\ul 118.5} \\
+0.4 & 280.7 & \textbf{283.1} & 268.2 & {\ul 242.5} & \textbf{144.0} & 142.4 & 141.7 & {\ul 127.4} \\
+0.6 & \textbf{311.5} & 303.5 & 289.9 & {\ul 256.5} & \textbf{151.1} & 139.9 & 144.1 & {\ul 129.1} \\
+0.8 & 336.9 & \textbf{339.1} & 311.4 & {\ul 280.5} & \textbf{151.2} & 144.0 & 140.3 & {\ul 132.5} \\
+1.0 & \textbf{362.4} & 355.0 & 335.9 & {\ul 307.8} & \textbf{140.2} & 131.1 & 133.4 & {\ul 129.0} \\
+1.2 & \textbf{388.1} & 373.4 & 361.0 & {\ul 333.5} & \textbf{130.2} & 122.9 & {\ul 119.5} & 125.4 \\
+1.4 & \textbf{407.3} & 396.4 & 381.7 & {\ul 349.5} &  113.3 & 108.4 & {\ul 106.1} & \textbf{117.5} \\
+1.6 & \textbf{413.6} & 397.0 & 396.1 & {\ul 364.6} &  101.0 & 98.58 &{\ul96.11} & \textbf{108.4} \\
+1.8 & \textbf{415.1} & 405.8 & 396.8 & {\ul 384.5} & 91.68 & 93.95 & {\ul84.71} & \textbf{96.66} \\
+2.0 & \textbf{415.0} & 402.4 & 396.6 & {\ul 382.2} & 84.94 & \textbf{94.50} & {\ul 82.66} & 84.87 \\
+2.2 & \textbf{414.5} & 388.9 & 398.1 & {\ul 383.2} & 81.89 & \textbf{89.53} & 78.38 & {\ul 77.41} \\
+2.4 & \textbf{406.4} & 386.5 & 393.7 & {\ul 376.4} & 85.12 & \textbf{87.83} & 72.13 & {\ul 71.66} \\
+2.6 & \textbf{401.2} & 379.7 & 386.2 & {\ul 368.5} & 78.65 & \textbf{79.31} & 70.83 & {\ul 61.94} \\
+2.8 & \textbf{392.8} & 371.5 & 379.4 & {\ul 360.4} & 77.03 & \textbf{79.80} & 70.49 & {\ul 63.91} \\
+3.0 & \textbf{390.4} & 359.9 & 374.3 & {\ul 350.4} & 72.53 & \textbf{78.03} & 71.80 & {\ul 70.28} \\ \bottomrule
\end{tabular}
}
%\vspace{-3mm}
\end{table}

\begin{table}[hb]
\caption{Effects of time, scenarios and their interaction on walking speeds via a repeated measured two-way ANOVA (two-sided).}
\label{Tab:2way_ANOVA}
\setlength{\tabcolsep}{3mm}{
\begin{tabular}{@{}c|rrrrrrrr@{}}
\toprule
Source & \multicolumn{1}{c}{ddof1} & \multicolumn{1}{c}{ddof2} & \multicolumn{1}{c}{\textit{F}} & \multicolumn{1}{c}{\textit{p-GG-corr}} & \multicolumn{1}{c}{$\eta_g^2$} & \multicolumn{1}{c}{$\epsilon$-GG} \\ \midrule
Time & 20 & 300 & 26.787 & \textless{}.001 & 0.514 & 0.102 \\
Scenarios & 3 & 45 & 5.250 & \textless{}.01 & 0.008 & 0.700 \\
Time$\times$Scenarios & 60 & 900 & 4.245 & \textless{}0.05 & 0.028 & 0.033 \\ \bottomrule
\end{tabular}
}
\end{table}



\begin{sidewaystable}[hp]
\centering
\caption{Pairwise comparisons using a repeated measured t-test (two-sided) for participants' walking speeds at each time point in each scenario. Bonferroni correction was used to adjust the p-values. Cohen's d represents the effect size for each t-test. Bold values show the adjusted p-values which are less than 0.05.}
\label{Tab:post-hoc}
\setlength{\tabcolsep}{1.5mm}{
\begin{tabular}{@{}c|ccc|ccc|ccc|ccc|ccc|ccc@{}}
\toprule
\multirow{2}{*}{\begin{tabular}[c]{@{}c@{}}Time\\ {[}s{]}\end{tabular}} & \multicolumn{3}{c|}{\begin{tabular}[c]{@{}c@{}}MV\\ v.s.\\ AV w/o eHMI\end{tabular}} & \multicolumn{3}{c|}{\begin{tabular}[c]{@{}c@{}}MV\\ v.s.\\ AV w/ eHMI\end{tabular}} & \multicolumn{3}{c|}{\begin{tabular}[c]{@{}c@{}}MV\\ v.s.\\ AV w/ eHMI\\ after PI\end{tabular}} & \multicolumn{3}{c|}{\begin{tabular}[c]{@{}c@{}}AV w/o eHMI\\ v.s.\\ AV w/ eHMI\end{tabular}} & \multicolumn{3}{c|}{\begin{tabular}[c]{@{}c@{}}AV w/o eHMI\\ v.s.\\ AV w/ eHMI\\ after PI\end{tabular}} & \multicolumn{3}{c}{\begin{tabular}[c]{@{}c@{}}AV w/ eHMI\\ v.s.\\ AV w/ eHMI\\after PI\end{tabular}} \\ \cmidrule(l){2-19} 
 & \textit{T} &\textit{ p-adj} & cohen & \textit{T} & \textit{p-adj} & cohen & \textit{T} & \textit{p-adj} & cohen & \textit{T} & \textit{p-adj} & cohen & \textit{T} & \textit{p-adj} & cohen & \textit{T} & \textit{p-adj} & cohen \\ \midrule
-1.0 & -2.564 & 0.104 & -0.376 & -4.749 & { \textbf{0.001}} & -0.628 & -2.734 & 0.071 & -0.361 & -3.742 & { \textbf{0.006}} & -0.248 & 0.346 & 1.000 & 0.031 & 3.032 & { \textbf{0.036}} & 0.289 \\
-0.8 & -1.744 & 0.558 & -0.216 & -4.047 & { \textbf{0.002}} & -0.503 & -2.783 & 0.059 & -0.348 & -3.567 & { \textbf{0.009}} & -0.266 & -1.305 & 1.000 & -0.110 & 2.529 & 0.107 & 0.174 \\
-0.6 & -1.969 & 0.358 & -0.211 & -3.202 & { \textbf{0.022}} & -0.386 & -2.520 & 0.109 & -0.321 & -2.353 & 0.159 & -0.172 & -1.237 & 1.000 & -0.096 & 1.261 & 1.000 & 0.086 \\
-0.4 & -1.869 & 0.438 & -0.225 & -2.853 & 0.050 & -0.333 & -2.270 & 0.190 & -0.284 & -1.484 & 0.900 & -0.099 & -0.367 & 1.000 & -0.041 & 0.757 & 1.000 & 0.064 \\
-0.2 & -1.519 & 0.845 & -0.161 & -2.035 & 0.313 & -0.223 & -1.235 & 1.000 & -0.159 & -0.709 & 1.000 & -0.053 & 0.161 & 1.000 & 0.020 & 0.934 & 1.000 & 0.080 \\
0.0 & -0.867 & 1.000 & -0.084 & -0.794 & 1.000 & -0.087 & 0.099 & 1.000 & 0.014 & -0.014 & 1.000 & -0.001 & 0.835 & 1.000 & 0.101 & 1.272 & 1.000 & 0.105 \\
+0.2 & -0.031 & 1.000 & -0.003 & 0.302 & 1.000 & 0.033 & 1.411 & 1.000 & 0.206 & 0.457 & 1.000 & 0.034 & 1.618 & 0.707 & 0.195 & 1.685 & 0.624 & 0.159 \\
+0.4 & -0.185 & 1.000 & -0.019 & 0.862 & 1.000 & 0.098 & 2.051 & 0.303 & 0.321 & 1.643 & 0.674 & 0.110 & 2.692 & 0.074 & 0.319 & 1.935 & 0.384 & 0.200 \\
+0.6 & 0.592 & 1.000 & 0.062 & 1.525 & 0.835 & 0.165 & 2.711 & 0.070 & 0.449 & 1.615 & 0.711 & 0.099 & 3.195 & { \textbf{0.022}} & 0.366 & 2.390 & 0.146 & 0.257 \\
+0.8 & -0.171 & 1.000 & -0.017 & 1.865 & 0.441 & 0.197 & 2.929 & { \textbf{0.042}} & 0.455 & 3.425 & { \textbf{0.012}} & 0.205 & 3.984 & { \textbf{0.003}} & 0.451 & 2.358 & 0.157 & 0.240 \\
+1.0 & 0.616 & 1.000 & 0.062 & 2.231 & 0.207 & 0.217 & 3.314 & { \textbf{0.016}} & 0.462 & 2.912 & { \textbf{0.044}} & 0.155 & 4.050 & { \textbf{0.002}} & 0.395 & 2.813 & 0.055 & 0.228 \\
+1.2 & 1.428 & 0.991 & 0.135 & 2.881 & { \textbf{0.047}} & 0.247 & 4.038 & { \textbf{0.003}} & 0.495 & 1.705 & 0.601 & 0.110 & 3.603 & { \textbf{0.008}} & 0.352 & 3.522 & { \textbf{0.010}} & 0.241 \\
+1.4 & 1.131 & 1.000 & 0.116 & 2.950 & { \textbf{0.040}} & 0.264 & 4.664 & { \textbf{0.000}} & 0.568 & 2.208 & 0.217 & 0.150 & 4.500 & { \textbf{0.001}} & 0.456 & 4.437 & { \textbf{0.001}} & 0.307 \\
+1.6 & 1.849 & 0.455 & 0.187 & 1.930 & 0.387 & 0.191 & 4.078 & { \textbf{0.002}} & 0.527 & 0.130 & 1.000 & 0.011 & 2.888 & { \textbf{0.046}} & 0.352 & 3.468 & { \textbf{0.011}} & 0.332 \\
+1.8 & 1.400 & 1.000 & 0.126 & 1.980 & 0.350 & 0.226 & 2.714 & 0.070 & 0.377 & 1.111 & 1.000 & 0.099 & 1.915 & 0.399 & 0.251 & 1.308 & 1.000 & 0.153 \\
+2.0 & 1.556 & 0.791 & 0.188 & 1.814 & 0.488 & 0.240 & 2.463 & 0.124 & 0.427 & 0.501 & 1.000 & 0.038 & 1.290 & 1.000 & 0.213 & 1.247 & 1.000 & 0.184 \\
+2.2 & 2.435 & 0.134 & 0.393 & 1.743 & 0.561 & 0.268 & 1.892 & 0.421 & 0.413 & -1.481 & 0.907 & -0.122 & 0.007 & 1.000 & 0.001 & 0.822 & 1.000 & 0.129 \\
+2.4 & 2.894 & 0.051 & 0.340 & 2.161 & 0.251 & 0.257 & 2.718 & 0.075 & 0.442 & -0.966 & 1.000 & -0.098 & 0.456 & 1.000 & 0.074 & 1.495 & 0.894 & 0.185 \\
+2.6 & 2.864 & 0.056 & 0.362 & 1.924 & 0.408 & 0.271 & 3.515 & { \textbf{0.012}} & 0.512 & -0.738 & 1.000 & -0.089 & 0.770 & 1.000 & 0.127 & 1.916 & 0.414 & 0.221 \\
+2.8 & 3.545 & { \textbf{0.012}} & 0.443 & 1.919 & 0.416 & 0.274 & 2.513 & 0.124 & 0.441 & -1.261 & 1.000 & -0.186 & -0.218 & 1.000 & -0.035 & 1.548 & 0.823 & 0.165 \\
+3.0 & 4.953 & { \textbf{0.001}} & 0.690 & 2.635 & 0.101 & 0.383 & 3.545 & { \textbf{0.014}} & 0.556 & -1.591 & 0.775 & -0.248 & -0.669 & 1.000 & -0.110 & 1.307 & 1.000 & 0.140 \\ \bottomrule
\end{tabular}
}
\end{sidewaystable}



This study assumed that two variables affected walking speed, while one variable influenced time and scenarios. Therefore, a repeated measured two-way ANOVA was used to estimate the changes in walking speed based on the two variables and their interaction.
The results of the repeated measured two-way ANOVA are shown in Table~\ref{Tab:2way_ANOVA}.
The Greenhouse-Geisser corrected p-values (p-GG-corr) for time and scenarios turn out to be less than 0.001 and 0.01, respectively, implying that the means of both factors possess a statistically significant effect on plant height. Meanwhile, the p-GG-corr for the interaction effect was also less than 0.05, underscoring a significant interaction effect between time and scenarios on walking speeds.


A repeated measured t-test was performed as pairwise comparisons for every downsampled time point referring to the test approach in~\citet{dey2021communicating}. Here, the Bonferroni correction was used to adjust the p-values. The results of the post hoc pairwise comparisons are illustrated in Table~\ref{Tab:post-hoc}. These findings reveal that there were no significant differences in walking speed between \textit{MV} and \textit{AV w/o eHMI}, except for a time range of 2.8~[s] to 3.0~[s]. Comparing the walking speeds in \textit{MV} and \textit{AV w/ eHMI}, significant differences were only observed from -1.0~[s] to -0.6~[s] and from 1.2~[s] to 1.4~[s].
In addition, significant differences for walking speeds in \textit{MV} and \textit{AV w/ eHMI after PI} were from 0.8~[s] to 1.6~[s], and at 2.6~[s] and 3.0~[s]. Furthermore, the walking speeds in \textit{AV w/o eHMI} and \textit{AV w/ eHMI} had significant differences at -1.0~[s], -0.8~[s], 0.8~[s] and 1.0~[s]. Similarly, significant differences were observed in walking speeds from 0.6~[s] to 1.6~[s] between \textit{AV w/o eHMI} and \textit{AV w/ eHMI after PI}. Moreover, the time at -0.1~[s] and from 1.2~[s] to 1.6~[s], demonstrated significantly different walking speeds between \textit{AV w/ eHMI} and \textit{AV w/ eHMI after PI}.










\section{Discussion}
\subsection{Subjective feelings of participants}
This study focused on the subjective feelings of pedestrians in different scenarios. To verify \textbf{H~1}, this subsection discusses the issues of \textit{AV w/o eHMI}, the effectiveness of eHMI, and the importance of eHMI instruction.

\subsubsection{Issues of AV w/o eHMI on pedestrians' subjective feelings}

The participants who encounter \textit{AV w/o eHMI} after encountering \textit{MV}, yield worsened subjective evaluations from Q1-Q6 as shown in Fig.~\ref{fig:result_01}.
Specifically, the results for Q1 showed that driving intention was significantly harder to understand when encountering \textit{AV w/o eHMI} as opposed to encountering \textit{MV} (Q1, WSR: $p<.001$).
Similarly, Q2 results demonstrate that the driving behaviors of \textit{AV w/o eHMI} became significantly more difficult to predict than the driving behavior of \textit{MV} (Q2, WSR: $p<.001$).
The following reasons could be considered:
(1) In the case of the MV, the dummy driver used the gesture to convey explicit information about driving intentions to the participants after the car stopped.
(2) In the case of \textit{AV w/o eHMI}, understanding the driving intention of \textit{AV w/o eHMI} and predicting the driving behaviors became more difficult for the participants because they could not obtain explicit information from the AV.
In addition, Q3-Q6 results showed that the sense of danger (Q3, WSR: $p<.01$), trust in the car (Q4, WSR: $p<.001$), sense of relief (Q5, WSR: $p<.001$), and hesitation in decision making (Q6, WSR: $p<.001$) for \textit{AV w/o eHMI} were significantly worse than those of MV.
The above results could be interpreted using the cognition-decision-behavior model (see Fig.~\ref{fig:model}) proposed in~\citet{liu2020when}.
Participants were more cautious when driving behaviors of \textit{AV w/o eHMI} were difficult to predict, subsequently increasing their subjective risks such as danger. Consequently, the participants might have demonstrated less trust in AV which could ensure their safety.
In addition, the participants seemingly found it difficult to cross with a high sense of relief while the sense of danger increased and the trust in \textit{AV w/o eHMI} decreased.
Finally, the deterioration of situational awareness and subjective feelings might have reduced the participants’ ability to make a decision quickly and thus became more hesitant when crossing in front of \textit{AV w/o eHMI}. These findings suggest that thought-provoking issues could occur when AV without eHMI is popularized in society.


%
\subsubsection{Effectiveness of eHMI in improving the subjective feelings of pedestrians}

As shown in Fig.~\ref{fig:result_01}, participants who encounter \textit{AV w/ eHMI} are significantly able to understand the driving intention (Q1, WSR: $p< .001$) better and to predict the driving behavior, (Q2, $p<.001$) than when they encounter \textit{AV w/o eHMI}.
From the results of Q1 and Q2, this study suggests that the participants could easily understand the driving intentions of the AV through the eHMI and predict the driving behaviors of the AV. Furthermore, this study confirmed that using eHMI could significantly improve participants' sense of danger (Q3, WSR: $p<.001$), trust in AV (Q4, WSR: $p<.001$), and the sense of relief (Q5, WSR: $p<.001$), compared to \textit{AV w/o eHMI}.
The above results are consistent with~\citep{faas2020external}, which reported that the eHMI contributes to more positive feelings such as a sense of safety and trust in AV.
The above results could be due to the current driving state and the driving intentions in the future of the AV which were clearly informed through the eHMI. In particular, the clear text message ``UGOKIMASEN'' read as ``AV does not move'' on the eHMI which was designed to be displayed after the car stopped.
Consequently, the participants received the implicate information, that is the behavior of the car, and the explicate information, that is the message on the eHMI, which carry the same meaning. Thus, the sense of danger, trust in AV, and relief among participants improved during their crossing. Moreover, the above improvements in subjective feelings allowed the participants to easily make crossing decisions with less hesitancy than when they encountered \textit{AV w/o eHMI} (Q6, WSR: $p<.001$). These results suggested that eHMI is an important method of communication for pedestrian--AV interaction. However, Fig.~\ref{fig:result_01} shows that the evaluation results of Q4 and Q5 for \textit{AV w/ eHMI} were still significantly lower than their evaluation results for \textit{MV}.
This indicated that although the eHMI improved participants' trust in AV and sense of relief during crossing when they encountered the \textit{AV w/ eHMI}, it did not reach the subjective feelings as when they encountered \textit{MV}.


\subsubsection{Importance of eHMI pre-instruction for improving the subjective feelings of pedestrians}

When encountering \textit{AV w/ eHMI after PI},the  participants were more easy to understand the driving intentions (Q1, WRS: $p <.001$) and to predict the driving behaviors (Q2, WRS: $p<.001$) than before the pre-instruction, \ie \textit{AV w/ eHMI} (see  Fig.~\ref{fig:result_01}).
These results also led the participants to feel significantly less danger from the driving behavior (Q3, WRS: $p<.05$), to have significantly more trust (Q4, WRS: $p<.01$) and to feel significantly more sense of relief (Q5, WRS: $p<.001$) about \textit{AV w/ eHMI after PI} than before.
%In particular, the clear text message ``UGOKIMASEN'' to show ``AV does not move'' on the eHMI is designed to be displayed after the car stopping.
%That is, the participant received the implicate information, \ie the behavior of the car, and the explicate information, \ie the message on the eHMI, which represent the same meaning.
Referring to the cognition-decision-behavior model (see Fig.~\ref{fig:model}) proposed in~\citep{Liu2022_APMV}, which was verified by the above results again, it was possible that the subjective feelings of the participants when encountering with the AV were greatly improved by enhancing the understanding of the driving intentions and the prediction of the driving behavior of the AV by using the eHMI with the pre-instruction.
Furthermore, the results of Q6 showed that there was no significant difference in the hesitancy in decision making of the participants before and after the pre-instruction.
Note that the Q6 results of these two scenarios, \ie \textit{AV w/ eHMI} and \textit{AV w/ eHMI after instruction}, were also not significantly different to the result of \textit{MV}.
This shows that the application of eHMI to AV was same as the communication gestures of human drivers which could support pedestrians to quickly make a decision for crossing the road.

In summary, the above results verified \textbf{H~1}, that is, pedestrians who correctly understand the rationale of eHMI through the pre-instruction exhibit improved situational awareness, subjective feelings, and decision-making during the interaction. Furthermore, these results clearly indicated the effectiveness and necessity of using eHMI and the pre-instruction to improve the subjective feelings of pedestrians who encountered the AV.


\subsection{Walking behaviors of participants}

This subsection discusses the pedestrians' walking speeds and the variation before and after the car stopping in different scenarios in order to verify the hypothesis \textbf{H~2}.



\subsubsection{Effects of eHMI and the pre-instruction on walking speed}

Table~\ref{Tab:mean_std_speed} illustrates that the mean of the walking speeds of the participants in \textit{MV} is the lowest before the car stops. Furthermore, the figure indicates that it was the highest for most of the time after the car stopped at 0.6~[s] to 3.0~[s]. 
Specifically, the mean of the walking speeds in \textit{MV} was significantly higher than that in \textit{AV w/ eHMI after PI} from 0.8~[s] to 1.6~[s] (see Tabel~\ref{Tab:post-hoc}). Two reasons were considered for the above differences. 

First, the interpretation could be based on Japanese culture, given that all the participants are Japanese and do not like to burden other people in general. Therefore, in the case of manned MV, the participants likely crossed quickly after the dummy driver presented "After you" using the hand gesture to reduce the burden for the dummy driver. Conversely, the participants crossed the road slowly \textit{AV w/ eHMI after PI} with a sense of relief (see Fig.~\ref{fig:result_01}) because they fully understood the AV did not move by displaying the message ``UGOKIMASEN'' on the eHMI. Furthermore, it can be inferred that the participants crossed slowly without considering their influence on the people around because the AV was a driverless car.

Second, the difference in the meanings of messages sent by the dummy driver and the eHMI. Referring to the study~\citet{li2021autonomous}, the message ``After you'' could be classified using the hand gesture of the dummy driver. Particularly, this suggests that the gesture could facilitate the pedestrians' decision-making. Furthermore, the message ``UGOKIMASEN'' indicated the state of the AV (\ie ``AV does not move now''), which can inform the pedestrians and help them understand the situation. Moreover, this study observed that the suggestion provided by the dummy driver was relatively urgent, while the information provided by the AV was not as urgent as it allowed the participants to make their own decisions.



In addition, this study reveals that the mean of walking speeds in \textit{AV w/o eHMI} was significantly higher than that in \textit{AV w/ eHMI} and \textit{AV w/ eHMI after PI} from 0.6~[s] to 1.6~[s] as shown in Table~\ref{Tab:post-hoc}. Referring to the results of the subjective evaluation that are illustrated in Fig.~\ref{fig:result_01}, this study considered that the participants felt relief and trusted the AV when they could understand the correct driving intentions of the AV through the eHMI after the pre-instruction. Consequently, they could pass through the road in a slow and relaxed manner,
which is in contrast to the result in~\citet{kooijman2019ehmis};
they reported that the mean of pedestrians' walking speeds was faster than that for \textit{AV w/o eHMI} after the AV slowed down while using a text eHMI to indicate yielding. We consider the difference between \citet{kooijman2019ehmis} and this study lies in the display time of the eHMI; while \citet{kooijman2019ehmis} displayed the eHMI while decelerating), this study displayed eHMI after the car stopped.

The present experiment found that the participants were safe after displaying eHMI, while in \citet{kooijman2019ehmis}, the participants may still perceive risk even after the display as the AV continues to approach. This study considered that this difference in perception of risk might cause a difference in walking speed. However, this study still cannot deny that the walking speed was reduced in \textit{AV w/ eHMI after PI} because the sequence of the experiment caused fatigue in the participants even though they sat in the chair and rested for approximately one minute after each trial.


\subsubsection{Effects of eHMI and the pre-instruction on variation of walking speed}

To verify the hypothesis \textbf{~H2}, \ie Pedestrians who correctly understand the rationale of eHMI through the pre-instruction exhibit resulted the participants' walking speed in a small variation over the time horizon via multiple trials, we looked at comparing the Std. of pedestrians' walking speeds before and after the car stopping in different scenarios. 
As Table~\ref{Tab:mean_std_speed} demonstrates, this study confirms that the Std. of participants' walking speeds in \textit{AV w/ eHMI after PI} was lowest within 1 second before and after the car stopped, respectively, from -1.0~[s] to 1.0~[s] and 2.2~[s] to 3.0~[s].

After the car stopped, (\ie after the eHMI appeared), the Std. of the participants walking at speeds within one second was lowest. Particularly, this indicates the effects of information guidance on using eHMI which were based on the participants' walking speeds exhibiting a small variation over the time horizon in multiple trials when they fully understood the principle of eHMI from the pre-instruction. 







\subsection{Limitations}

The experiment conducted in this study used an underground parking lot, which varied from the general traffic environment. Particularly, other pedestrians, vehicles, and crosswalks, among others, were present. Furthermore, this study only used a text-based eHMI design for hypothesis testing. The use of other types of eHMI, such as icons and colored light bars, has the potential to provide participants with a different psychological experience.

Moreover, the effects of eHMI and the pre-instruction on participants' walking behavior were discussed in this study. Although each participant rested for approximately one minute after each trial, the participants’ fatigue from repeated testing could not be disregarded.

All participants were Japanese, thus the cultural influences on the experimental results, \ie their subjective feelings and walking behaviors, also could not be disregarded.


\section{CONCLUSION}

This study investigated the effects of eHMI and the pre-instruction on subjective feelings and walking behaviors of pedestrians and designed an interaction experiment to allow an encounter among participants and \textit{MV}, \textit{AV w/o eHMI}, \textit{AV w/ eHMI}, and \textit{AV w/ eHMI after PI} in a road crossing scenario.

A clear issue was found when the participants encountered \textit{AV w/o eHMI}. The participants expressed increased difficulty with the driving intention and predicting the driving behavior during encounters \textit{AV w/o eHMI} after they were habituated to interacting with \textit{MV}. Furthermore, the participants’ associated subjective feelings (\ie feelings of danger, trust in AV, and feeling of relief) about \textit{AV w/o eHMI} deteriorated significantly. Moreover, these feelings caused decision-making hesitancy when pedestrians crossed the road. 

The eHMI helped the pedestrians understand the driving intention of the AV and predict its driving behavior when used in the AV. Furthermore, pedestrians’ subjective feelings about \textit{AV w/ eHMI} and their hesitation toward decision-making improved. However, the participants still did not exceed the subjective evaluations of \textit{MV}.
Instructing the rationale of eHMI based on the AV to the participants could help them correctly understand the driving intentions and predict the behaviors of the AV. Consequently, this could enhance the subjective feelings and decision-making among the participants. The results of the subjective evaluations for \textit{AV w/ eHMI after PI} were consistent with the standards for \textit{MV}. In addition, an effect of information guidance using eHMI in which the participants' walking speeds demonstrated a reduced variation in multiple trials when they fully understood the principle of eHMI by the pre-instruction. 


In future, the negative effects of eHMI on pedestrians should be studied, such as overtrust~\citep{hollander2019overtrust,kaleefathullah2020external} and distraction~\citep{lee2021negative}. Particularly, these issues will be solved by refining the pre-instruction and analyzing its influence on pedestrians, including providing the pre-instruction for cognition and risk evaluation. Finally, the standardization of eHMI should be realized to ensure the design of a more accurate method of pre-instruction.


\section*{ACKNOWLEDGMENT}

This work was received funding from JSPS KAKENHI Grant Numbers 20K19846 and 19K12080 in Japan, and partly supported by Toyota Motor Corporation, Japan.
The authors express their appreciation for the valuable advice provided by Mr. Masaya Watanabe of Toyota Motor Corporation on the conceptualization and experimental methodology of this study.


\section*{CRediT authorship contribution statement}
\textbf{Hailong Liu}: Conceptualization, Investigation, Resources, Methodology, Validation, Formal analysis, Visualization, Writing - Original Draft, Writing - Review \& Editing, Project administration.

\textbf{Takatsugu Hirayama}: Conceptualization, Methodology, Validation, Writing - review \& editing.




\section*{Declaration of Competing Interest}
The authors declare that they have no known competing financial interests or personal relationships that could have appeared to influence the work reported in this paper.

\bibliographystyle{cas-model2-names} 
%\bibliographystyle{IEEEtranN} 
\bibliography{sample-base}



\end{document}
%\endinput





