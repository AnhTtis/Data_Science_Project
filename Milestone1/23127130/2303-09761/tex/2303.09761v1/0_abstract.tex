\begin{abstract}

Peer-to-peer (P2P) networks underlie a variety of decentralized paradigms including blockchains, distributed file storage and decentralized domain name systems. A central primitive in P2P networks is the peer selection algorithm, which decides how a node should select a fixed number of neighbors to connect with.
In this paper, we consider the design of a peer-selection algorithm for unstructured P2P networks with the goal of minimizing the broadcast  latency.
We propose $\GF$, a novel solution that dynamically decides the neighbor set by exploiting the past experiences as well as exploring new neighbors. The key technical contributions come from bringing ideas of matrix completion for estimating message delivery times for every possible message for every peer ever connected, and a streaming algorithm to efficiently perform the estimation while achieving good performance.
%\soubhik{Current style of writing can confuse people. I first thought why is GF doing such intensive computation. Only after reading intro completely I understood streaming algo was simplifying this computation. In current writing, it seems as if streaming algo is separate from the above key contribution. Maybe you want to say "key contribution is to model...and GF employs streaming algo to...."}. 
The matrix completion interpolates the delivery times to all virtual connections in order to select the best combination of neighbors. $\GF$ employs 
%\soubhik{$\GF$ employs...} 
a streaming algorithm that only uses a short recent memory to finish matrix interpolation. 
%We perform detailed simulations and demonstrate that $\GF$ attains good connections by exploring every peer only once. We evaluate the algorithm in two scenario. 
%\soubhik{two scenarios}. 
When the number of publishing source is  equal to a node's maximal number of connections, %where the global optimal solution is unique, %\soubhik{this sentence phrasing is weird. Is it incomplete?}. 
$\GF$ found the global optimal solution with $92.7\%$ probability by exploring every node only once. In more complex situations where nodes are publishing based on exponential distribution and adjusting connection in real time, we compare $\GF$ with a baseline peer selection system, Perigee\cite{mao2020perigee}, and show $\GF$ saves approximately $14.5\%$ less time under real world geolocation and propagation latency.

\end{abstract}