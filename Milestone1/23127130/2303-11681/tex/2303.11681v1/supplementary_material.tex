\documentclass[10pt,twocolumn,letterpaper]{article}

\usepackage{iccv}
\usepackage{times}
\usepackage{epsfig}
\usepackage{graphicx}
\usepackage{amsmath}
\usepackage{amssymb}

\usepackage{subcaption}
\usepackage{multirow}
\usepackage{multicol}
\usepackage{floatrow}
\usepackage{algorithm}
\usepackage{algorithmic}
\usepackage{booktabs}
\usepackage{mathrsfs, eucal}
\newcommand{\conditioner}{\tau_\theta}
\newcommand{\R}{\mathbb{R}}
\usepackage[pagebackref=true,breaklinks=true,letterpaper=true,colorlinks,bookmarks=false]{hyperref}



\input CSMacrosV2.tex


% Include other packages here, before hyperref.

% If you comment hyperref and then uncomment it, you should delete
% egpaper.aux before re-running latex.  (Or just hit 'q' on the first latex
% run, let it finish, and you should be clear).
\usepackage[pagebackref=true,breaklinks=true,letterpaper=true,colorlinks,bookmarks=false]{hyperref}


\newcommand{\diffmask}{\textup{DiffuMask}\xspace}
\newcommand{\Ours}{\textup{DiffuMask}\xspace}

% \newcommand{\cisdq}{\textsc{\textbf{C}i\textbf{SDQ}}\xspace}

% \usepackage[ruled]{algorithm2e}             
\usepackage{array}
\newcolumntype{I}{!{\vrule width 3pt}}
\newlength\savedwidth
\newcommand\whline{\noalign{\global\savedwidth\arrayrulewidth
                           \global\arrayrulewidth 2pt}%
                  \hline
                  \noalign{\global\arrayrulewidth\savedwidth}}
\newlength\savewidth
\newcommand\shline{\noalign{\global\savewidth\arrayrulewidth
                           \global\arrayrulewidth 0.5pt}%
                  \hline
                  \noalign{\global\arrayrulewidth\savewidth}}

\DeclareMathOperator*{\argmax}{arg\,max}
\DeclareMathOperator*{\argmin}{arg\,min}


\newcommand{\green}[1]{\textcolor[RGB]{96,177,87}{#1}}
% \newcommand{\blue}[1]{\textcolor[RGB]{0,0,255}{#1}}



\usepackage{tabulary,multirow,overpic,xcolor,subfloat}

\definecolor{linkcolor}{HTML}{ED1C24}

\newcommand{\bd}[1]{\textbf{#1}}
\newcommand{\app}{\raise.17ex\hbox{$\scriptstyle\sim$}}
\newcommand{\ncdot}{{\mkern 0mu\cdot\mkern 0mu}}
\def\x{\times}
\newcolumntype{x}[1]{>{\centering\arraybackslash}p{#1pt}}
\newcolumntype{y}[1]{>{\raggedright\arraybackslash}p{#1pt}}
\newcommand{\dt}[1]{\fontsize{5pt}{0.1em}\selectfont (#1)}

\definecolor{Gray}{gray}{0.5}
\newcommand{\tablestyle}[2]{\setlength{\tabcolsep}{#1}\renewcommand{\arraystretch}{#2}\centering\footnotesize}
\makeatletter\renewcommand\paragraph{\@startsection{paragraph}{4}{\z@}
  {.5em \@plus1ex \@minus.2ex}{-.5em}{\normalfont\normalsize\bfseries}}\makeatother


\newcommand{\lmatch}[1]{{\cal L}_{\rm match}(#1)}

\def\eg{\emph{e.g.}}
\def\Eg{\emph{E.g.,}}
\def\ie{\emph{i.e.}}
\def\etal{\emph{et al.}}



% \iccvfinalcopy % *** Uncomment this line for the final submission
\vspace{-4mm}
\def\iccvPaperID{4328} % *** Enter the ICCV Paper ID here
\def\httilde{\mbox{\tt\raisebox{-.5ex}{\symbol{126}}}}

% Pages are numbered in submission mode, and unnumbered in camera-ready
\ificcvfinal\pagestyle{empty}\fi

\begin{document}

%%%%%%%%% TITLE
\title{Supplementary Material for `\Ours'}

\author{First Author\\
Institution1\\
Institution1 address\\
{\tt\small firstauthor@i1.org}
% For a paper whose authors are all at the same institution,
% omit the following lines up until the closing ``}''.
% Additional authors and addresses can be added with ``\and'',
% just like the second author.
% To save space, use either the email address or home page, not both
\and
Second Author\\
Institution2\\
First line of institution2 address\\
{\tt\small secondauthor@i2.org}
}

\maketitle
% Remove page # from the first page of camera-ready.
\ificcvfinal\thispagestyle{empty}\fi


\section{More Details}
\textbf{Evaluation Metrics.} \textit{Mean intersection-over-union (mIoU)}~\cite{(voc)everingham2010pascal,cheng2022masked}, as the common metric of semantic segmentation, is used to evaluate the performance.
%
For open-vocabulary segmentation, following the prior~\cite{ding2022decoupling,cheng2021sign}, the mIoU averaged on seen classes, unseen classes, and their \textit{harmonic mean} are used.
%



\textbf{Mask Smoothness.} The mask $B_{\hat{\gamma}}$ generated by the Dense CRF often contains jagged edges and numerous small regions that do not correspond to distinct objects in the image.
%
To address these issues, we trained a segmentation model $\bm{\theta}$~(\ie{} Mask2Former), using the mask $B_{\hat{\gamma}}$ generated by the Dense CRF as input. 
%
We then used this model to predict the pseudo labels for the training set of synthetic data, resulting in a final semantic mask annotation
%
% In our experiment, we use the segmentation model $\bm{\theta}$~(\ie{} Mask2Former) to smooth the mask, \ie{} we train the model with $B_{\hat{\gamma}}$ and predict the pseudo label of the training set for synthetic data, which are the final semantic mask annotation.


\textbf{Cross Validation for Noise Learning.}
In the experiment, we performed the three-fold cross-validation for each class.
%
The five-fold cross-validation (CV) is a process in which all data is randomly split into $k$ folds, in our case $k$ $=$ $3$, and then the model is trained on the $k - 1$ folds, while one fold is left to test the quality.
%


\begin{figure*}[t]
% 	\centering
	\includegraphics[width=0.99\linewidth]{iccv2023AuthorKit/figures/fig1_voc.pdf}
	\vspace{-0.1cm}
	\caption{\textbf{Visualization of \diffmask for $20$ classes on Pascal-VOC 2012~\cite{(voc)everingham2010pascal}.} 
    }
    \vspace{-0.2cm}
\label{vis_voc}
\end{figure*}




\begin{table}[t]
    \centering
    \small 
    % \renewcommand\arraystretch{1.0}
    \setlength{\tabcolsep}{1mm}
    \tablestyle{4pt}{1.2}\scriptsize\begin{tabular}{l|l |l | ccc | c}
  & & & \multicolumn{3}{c}{Category/\%} & \\
Train Set & Number & Backbone  & bus & car     & person   & mIoU \\
  \shline
  
  % \textit{Training with Real Data} & &   &  &   &  &         \\
  \multicolumn{7}{l}{\textit{Train with Pure \textbf{Real} Data}}\\
  \multirow{2}{*}{ADE20K} & R: 20.2k & R50  & 87.9  & 82.5 &  79.4 &  83.3 \\
  & R: 20.2k & Swin-B  &  93.6 & 86.1  & 84.0 &   87.9   \\
   \hline
   \multicolumn{7}{l}{\textit{Train with Pure \textbf{Synthetic} Data}}\\
   % \textit{Training with \textbf{Pure} Synthetic Data}  &  &  &     & &        \\
  \multirow{2}{*}{\textbf{\diffmask}}
  & S: 6.0k & R50\phantom{$^{\text{\textdagger}}$}  & 43.4 & 67.3 &  60.2 &  57.0     \\
  & S: 6.0k & Swin-B\phantom{$^{\text{\textdagger}}$}  & 72.8 & 73.4 &  62.6 &  69.6     \\
  %  \hline
  %  \textit{Finetune with \textbf{Real} Data} &  &  &   &  &  & &        \\
  % \multirow{2}{*}{ADE20K, \textbf{\diffmask}}
  % & - & R50\phantom{$^{\text{\textdagger}}$} & - &  - & - & - & - &  -    \\
  % & - & Swin-B\phantom{$^{\text{\textdagger}}$} & - & - & - & - & - & - & - & -  &
  
  \end{tabular}



  % 'mIoU': 44.55776460284815, 'fwIoU': 69.87780487995555, 'IoU-wall': 74.7313789134194, 'BoundaryIoU-wall': 75.93483365825622, 'min(IoU, B-Iou)-wall': 74.7313789134194, 'IoU-building': 79.55577187194908, 'BoundaryIoU-building': 10.830138428901783, 'min(IoU, B-Iou)-building': 10.830138428901783, 'IoU-sky': 93.84338395830892, 'BoundaryIoU-sky': 0.0, 'min(IoU, B-Iou)-sky': 0.0, 'IoU-floor': 78.95609769927562, 'BoundaryIoU-floor': 0.0, 'min(IoU, B-Iou)-floor': 0.0, 'IoU-tree': 72.06293019422134, 'BoundaryIoU-tree': 0.0, 'min(IoU, B-Iou)-tree': 0.0, 'IoU-ceiling': 81.47069146778928, 'BoundaryIoU-ceiling': 0.0, 'min(IoU, B-Iou)-ceiling': 0.0, 'IoU-road, route': 81.97318080586258, 'BoundaryIoU-road, route': 0.0, 'min(IoU, B-Iou)-road, route': 0.0, 'IoU-bed': 86.24661399035641, 'BoundaryIoU-bed': 0.0, 'min(IoU, B-Iou)-bed': 0.0, 'IoU-window ': 60.57269043882506, 'BoundaryIoU-window ': 0.0, 'min(IoU, B-Iou)-window ': 0.0, 'IoU-grass': 69.73011114960318, 'BoundaryIoU-grass': 0.0, 'min(IoU, B-Iou)-grass': 0.0, 'IoU-cabinet': 59.236624279843355, 'BoundaryIoU-cabinet': 0.0, 'min(IoU, B-Iou)-cabinet': 0.0, 'IoU-sidewalk, pavement': 67.65325864805651, 'BoundaryIoU-sidewalk, pavement': 0.0, 'min(IoU, B-Iou)-sidewalk, pavement': 0.0, 'IoU-person': 79.39391624775693, 'BoundaryIoU-person': 0.0, 'min(IoU, B-Iou)-person': 0.0, 'IoU-earth, ground': 32.27316142173686, 'BoundaryIoU-earth, ground': 0.0, 'min(IoU, B-Iou)-earth, ground': 0.0, 'IoU-door': 42.91990997231772, 'BoundaryIoU-door': 0.0, 'min(IoU, B-Iou)-door': 0.0, 'IoU-table': 55.41693450839069, 'BoundaryIoU-table': 0.0, 'min(IoU, B-Iou)-table': 0.0, 'IoU-mountain, mount': 59.486994608591694, 'BoundaryIoU-mountain, mount': 0.0, 'min(IoU, B-Iou)-mountain, mount': 0.0, 'IoU-plant': 48.801757904223116, 'BoundaryIoU-plant': 0.0, 'min(IoU, B-Iou)-plant': 0.0, 'IoU-curtain': 69.0145522893462, 'BoundaryIoU-curtain': 0.0, 'min(IoU, B-Iou)-curtain': 0.0, 'IoU-chair': 55.630134555865595, 'BoundaryIoU-chair': 0.0, 'min(IoU, B-Iou)-chair': 0.0, 'IoU-car': 82.45025160601578, 'BoundaryIoU-car': 0.0, 'min(IoU, B-Iou)-car': 0.0, 'IoU-water': 53.65950140770839, 'BoundaryIoU-water': 0.0, 'min(IoU, B-Iou)-water': 0.0, 'IoU-painting, picture': 60.21585508173589, 'BoundaryIoU-painting, picture': 0.0, 'min(IoU, B-Iou)-painting, picture': 0.0, 'IoU-sofa': 59.51923987748787, 'BoundaryIoU-sofa': 0.0, 'min(IoU, B-Iou)-sofa': 0.0, 'IoU-shelf': 38.559134203211855, 'BoundaryIoU-shelf': 0.0, 'min(IoU, B-Iou)-shelf': 0.0, 'IoU-house': 42.28442426135578, 'BoundaryIoU-house': 0.0, 'min(IoU, B-Iou)-house': 0.0, 'IoU-sea': 61.029567473942556, 'BoundaryIoU-sea': 0.0, 'min(IoU, B-Iou)-sea': 0.0, 'IoU-mirror': 60.04703330771871, 'BoundaryIoU-mirror': 0.0, 'min(IoU, B-Iou)-mirror': 0.0, 'IoU-rug': 61.56210560720684, 'BoundaryIoU-rug': 0.0, 'min(IoU, B-Iou)-rug': 0.0, 'IoU-field': 29.811248557840038, 'BoundaryIoU-field': 0.0, 'min(IoU, B-Iou)-field': 0.0, 'IoU-armchair': 36.55648828331198, 'BoundaryIoU-armchair': 0.0, 'min(IoU, B-Iou)-armchair': 0.0, 'IoU-seat': 61.76239041442556, 'BoundaryIoU-seat': 0.0, 'min(IoU, B-Iou)-seat': 0.0, 'IoU-fence': 38.45953409086686, 'BoundaryIoU-fence': 0.0, 'min(IoU, B-Iou)-fence': 0.0, 'IoU-desk': 47.29645588693508, 'BoundaryIoU-desk': 0.0, 'min(IoU, B-Iou)-desk': 0.0, 'IoU-rock, stone': 42.03229661222097, 'BoundaryIoU-rock, stone': 0.0, 'min(IoU, B-Iou)-rock, stone': 0.0, 'IoU-wardrobe, closet, press': 42.90803867636571, 'BoundaryIoU-wardrobe, closet, press': 0.0, 'min(IoU, B-Iou)-wardrobe, closet, press': 0.0, 'IoU-lamp': 48.96663501836561, 'BoundaryIoU-lamp': 0.0, 'min(IoU, B-Iou)-lamp': 0.0, 'IoU-tub': 72.1768373874475, 'BoundaryIoU-tub': 0.0, 'min(IoU, B-Iou)-tub': 0.0, 'IoU-rail': 30.200017733057955, 'BoundaryIoU-rail': 0.0, 'min(IoU, B-Iou)-rail': 0.0, 'IoU-cushion': 47.56587565048953, 'BoundaryIoU-cushion': 0.0, 'min(IoU, B-Iou)-cushion': 0.0, 'IoU-base, pedestal, stand': 22.353379137858127, 'BoundaryIoU-base, pedestal, stand': 0.0, 'min(IoU, B-Iou)-base, pedestal, stand': 0.0, 'IoU-box': 18.873080016990055, 'BoundaryIoU-box': 0.0, 'min(IoU, B-Iou)-box': 0.0, 'IoU-column, pillar': 48.107784232025516, 'BoundaryIoU-column, pillar': 0.0, 'min(IoU, B-Iou)-column, pillar': 0.0, 'IoU-signboard, sign': 29.919548819817216, 'BoundaryIoU-signboard, sign': 0.0, 'min(IoU, B-Iou)-signboard, sign': 0.0, 'IoU-chest of drawers, chest, bureau, dresser': 40.263596755878424, 'BoundaryIoU-chest of drawers, chest, bureau, dresser': 0.0, 'min(IoU, B-Iou)-chest of drawers, chest, bureau, dresser': 0.0, 'IoU-counter': 41.21578010093914, 'BoundaryIoU-counter': 0.0, 'min(IoU, B-Iou)-counter': 0.0, 'IoU-sand': 30.15332290511149, 'BoundaryIoU-sand': 0.0, 'min(IoU, B-Iou)-sand': 0.0, 'IoU-sink': 63.000227475546374, 'BoundaryIoU-sink': 0.0, 'min(IoU, B-Iou)-sink': 0.0, 'IoU-skyscraper': 45.683109370879535, 'BoundaryIoU-skyscraper': 0.0, 'min(IoU, B-Iou)-skyscraper': 0.0, 'IoU-fireplace': 60.70952620471151, 'BoundaryIoU-fireplace': 0.0, 'min(IoU, B-Iou)-fireplace': 0.0, 'IoU-refrigerator, icebox': 80.34009373635588, 'BoundaryIoU-refrigerator, icebox': 0.0, 'min(IoU, B-Iou)-refrigerator, icebox': 0.0, 'IoU-grandstand, covered stand': 38.288421029654124, 'BoundaryIoU-grandstand, covered stand': 0.0, 'min(IoU, B-Iou)-grandstand, covered stand': 0.0, 'IoU-path': 27.396546132000093, 'BoundaryIoU-path': 0.0, 'min(IoU, B-Iou)-path': 0.0, 'IoU-stairs': 33.56029226802793, 'BoundaryIoU-stairs': 0.0, 'min(IoU, B-Iou)-stairs': 0.0, 'IoU-runway': 75.53548317663909, 'BoundaryIoU-runway': 0.0, 'min(IoU, B-Iou)-runway': 0.0, 'IoU-case, display case, showcase, vitrine': 48.88631968384764, 'BoundaryIoU-case, display case, showcase, vitrine': 0.0, 'min(IoU, B-Iou)-case, display case, showcase, vitrine': 0.0, 'IoU-pool table, billiard table, snooker table': 92.77210525814688, 'BoundaryIoU-pool table, billiard table, snooker table': 0.0, 'min(IoU, B-Iou)-pool table, billiard table, snooker table': 0.0, 'IoU-pillow': 52.94615222455376, 'BoundaryIoU-pillow': 0.0, 'min(IoU, B-Iou)-pillow': 0.0, 'IoU-screen door, screen': 46.99803998835841, 'BoundaryIoU-screen door, screen': 0.0, 'min(IoU, B-Iou)-screen door, screen': 0.0, 'IoU-stairway, staircase': 34.2278386175111, 'BoundaryIoU-stairway, staircase': 0.0, 'min(IoU, B-Iou)-stairway, staircase': 0.0, 'IoU-river': 15.850241071492727, 'BoundaryIoU-river': 0.0, 'min(IoU, B-Iou)-river': 0.0, 'IoU-bridge, span': 71.72093785378061, 'BoundaryIoU-bridge, span': 0.0, 'min(IoU, B-Iou)-bridge, span': 0.0, 'IoU-bookcase': 31.225591348840403, 'BoundaryIoU-bookcase': 0.0, 'min(IoU, B-Iou)-bookcase': 0.0, 'IoU-blind, screen': 21.979356691579767, 'BoundaryIoU-blind, screen': 0.0, 'min(IoU, B-Iou)-blind, screen': 0.0, 'IoU-coffee table': 54.372062489296525, 'BoundaryIoU-coffee table': 0.0, 'min(IoU, B-Iou)-coffee table': 0.0, 'IoU-toilet, can, commode, crapper, pot, potty, stool, throne': 84.98076879765571, 'BoundaryIoU-toilet, can, commode, crapper, pot, potty, stool, throne': 0.0, 'min(IoU, B-Iou)-toilet, can, commode, crapper, pot, potty, stool, throne': 0.0, 'IoU-flower': 25.61136385038268, 'BoundaryIoU-flower': 0.0, 'min(IoU, B-Iou)-flower': 0.0, 'IoU-book': 46.60875983390284, 'BoundaryIoU-book': 0.0, 'min(IoU, B-Iou)-book': 0.0, 'IoU-hill': 7.76279972941984, 'BoundaryIoU-hill': 0.0, 'min(IoU, B-Iou)-hill': 0.0, 'IoU-bench': 39.77289289013593, 'BoundaryIoU-bench': 0.0, 'min(IoU, B-Iou)-bench': 0.0, 'IoU-countertop': 48.3588728004675, 'BoundaryIoU-countertop': 0.0, 'min(IoU, B-Iou)-countertop': 0.0, 'IoU-stove': 72.56641371525645, 'BoundaryIoU-stove': 0.0, 'min(IoU, B-Iou)-stove': 0.0, 'IoU-palm, palm tree': 49.79268384619137, 'BoundaryIoU-palm, palm tree': 0.0, 'min(IoU, B-Iou)-palm, palm tree': 0.0, 'IoU-kitchen island': 33.81426672942219, 'BoundaryIoU-kitchen island': 0.0, 'min(IoU, B-Iou)-kitchen island': 0.0, 'IoU-computer': 57.737847700840604, 'BoundaryIoU-computer': 0.0, 'min(IoU, B-Iou)-computer': 0.0, 'IoU-swivel chair': 42.814667807325876, 'BoundaryIoU-swivel chair': 0.0, 'min(IoU, B-Iou)-swivel chair': 0.0, 'IoU-boat': 67.28427670210212, 'BoundaryIoU-boat': 0.0, 'min(IoU, B-Iou)-boat': 0.0, 'IoU-bar': 40.830580835836244, 'BoundaryIoU-bar': 0.0, 'min(IoU, B-Iou)-bar': 0.0, 'IoU-arcade machine': 63.67015992052046, 'BoundaryIoU-arcade machine': 0.0, 'min(IoU, B-Iou)-arcade machine': 0.0, 'IoU-hovel, hut, hutch, shack, shanty': 64.98358473090536, 'BoundaryIoU-hovel, hut, hutch, shack, shanty': 0.0, 'min(IoU, B-Iou)-hovel, hut, hutch, shack, shanty': 0.0, 'IoU-bus': 87.9129901317965, 'BoundaryIoU-bus': 0.0, 'min(IoU, B-Iou)-bus': 0.0, 'IoU-towel': 49.01452964341253, 'BoundaryIoU-towel': 0.0, 'min(IoU, B-Iou)-towel': 0.0, 'IoU-light': 42.73524384843941, 'BoundaryIoU-light': 0.0, 'min(IoU, B-Iou)-light': 0.0, 'IoU-truck': 19.91655565027051, 'BoundaryIoU-truck': 0.0, 'min(IoU, B-Iou)-truck': 0.0, 'IoU-tower': 27.575803583047758, 'BoundaryIoU-tower': 0.0, 'min(IoU, B-Iou)-tower': 0.0, 'IoU-chandelier': 60.59769904203144, 'BoundaryIoU-chandelier': 0.0, 'min(IoU, B-Iou)-chandelier': 0.0, 'IoU-awning, sunshade, sunblind': 22.496899664490826, 'BoundaryIoU-awning, sunshade, sunblind': 0.0, 'min(IoU, B-Iou)-awning, sunshade, sunblind': 0.0, 'IoU-street lamp': 13.837073371994407, 'BoundaryIoU-street lamp': 0.0, 'min(IoU, B-Iou)-street lamp': 0.0, 'IoU-booth': 40.98599766627771, 'BoundaryIoU-booth': 0.0, 'min(IoU, B-Iou)-booth': 0.0, 'IoU-tv': 63.51333730482382, 'BoundaryIoU-tv': 0.0, 'min(IoU, B-Iou)-tv': 0.0, 'IoU-plane': 52.167955024595926, 'BoundaryIoU-plane': 0.0, 'min(IoU, B-Iou)-plane': 0.0, 'IoU-dirt track': 3.003610273283978, 'BoundaryIoU-dirt track': 0.0, 'min(IoU, B-Iou)-dirt track': 0.0, 'IoU-clothes': 30.81750170029472, 'BoundaryIoU-clothes': 0.0, 'min(IoU, B-Iou)-clothes': 0.0, 'IoU-pole': 12.901523485546576, 'BoundaryIoU-pole': 0.0, 'min(IoU, B-Iou)-pole': 0.0, 'IoU-land, ground, soil': 3.012774243932783, 'BoundaryIoU-land, ground, soil': 0.0, 'min(IoU, B-Iou)-land, ground, soil': 0.0, 'IoU-bannister, banister, balustrade, balusters, handrail': 10.16441910079806, 'BoundaryIoU-bannister, banister, balustrade, balusters, handrail': 0.0, 'min(IoU, B-Iou)-bannister, banister, balustrade, balusters, handrail': 0.0, 'IoU-escalator, moving staircase, moving stairway': 22.44124639758373, 'BoundaryIoU-escalator, moving staircase, moving stairway': 0.0, 'min(IoU, B-Iou)-escalator, moving staircase, moving stairway': 0.0, 'IoU-ottoman, pouf, pouffe, puff, hassock': 37.63416815742397, 'BoundaryIoU-ottoman, pouf, pouffe, puff, hassock': 0.0, 'min(IoU, B-Iou)-ottoman, pouf, pouffe, puff, hassock': 0.0, 'IoU-bottle': 14.083582809551508, 'BoundaryIoU-bottle': 0.0, 'min(IoU, B-Iou)-bottle': 0.0, 'IoU-buffet, counter, sideboard': 45.73163938736314, 'BoundaryIoU-buffet, counter, sideboard': 0.0, 'min(IoU, B-Iou)-buffet, counter, sideboard': 0.0, 'IoU-poster, posting, placard, notice, bill, card': 24.32211347749791, 'BoundaryIoU-poster, posting, placard, notice, bill, card': 0.0, 'min(IoU, B-Iou)-poster, posting, placard, notice, bill, card': 0.0, 'IoU-stage': 14.401993196910926, 'BoundaryIoU-stage': 0.0, 'min(IoU, B-Iou)-stage': 0.0, 'IoU-van': 44.788442503181614, 'BoundaryIoU-van': 0.0, 'min(IoU, B-Iou)-van': 0.0, 'IoU-ship': 6.529060912854162, 'BoundaryIoU-ship': 0.0, 'min(IoU, B-Iou)-ship': 0.0, 'IoU-fountain': 20.75739565977526, 'BoundaryIoU-fountain': 0.0, 'min(IoU, B-Iou)-fountain': 0.0, 'IoU-conveyer belt, conveyor belt, conveyer, conveyor, transporter': 73.70190314943481, 'BoundaryIoU-conveyer belt, conveyor belt, conveyer, conveyor, transporter': 0.0, 'min(IoU, B-Iou)-conveyer belt, conveyor belt, conveyer, conveyor, transporter': 0.0, 'IoU-canopy': 9.132436976983465, 'BoundaryIoU-canopy': 0.0, 'min(IoU, B-Iou)-canopy': 0.0, 'IoU-washer, automatic washer, washing machine': 65.39980227559951, 'BoundaryIoU-washer, automatic washer, washing machine': 0.0, 'min(IoU, B-Iou)-washer, automatic washer, washing machine': 0.0, 'IoU-plaything, toy': 19.42734932975077, 'BoundaryIoU-plaything, toy': 0.0, 'min(IoU, B-Iou)-plaything, toy': 0.0, 'IoU-pool': 38.05390639227827, 'BoundaryIoU-pool': 0.0, 'min(IoU, B-Iou)-pool': 0.0, 'IoU-stool': 51.4664751997462, 'BoundaryIoU-stool': 0.0, 'min(IoU, B-Iou)-stool': 0.0, 'IoU-barrel, cask': 31.128617539252694, 'BoundaryIoU-barrel, cask': 0.0, 'min(IoU, B-Iou)-barrel, cask': 0.0, 'IoU-basket, handbasket': 24.545876783885067, 'BoundaryIoU-basket, handbasket': 0.0, 'min(IoU, B-Iou)-basket, handbasket': 0.0, 'IoU-falls': 65.8371356273287, 'BoundaryIoU-falls': 0.0, 'min(IoU, B-Iou)-falls': 0.0, 'IoU-tent': 94.12259673326643, 'BoundaryIoU-tent': 0.0, 'min(IoU, B-Iou)-tent': 0.0, 'IoU-bag': 8.487054117124124, 'BoundaryIoU-bag': 0.0, 'min(IoU, B-Iou)-bag': 0.0, 'IoU-minibike, motorbike': 68.45675828359492, 'BoundaryIoU-minibike, motorbike': 0.0, 'min(IoU, B-Iou)-minibike, motorbike': 0.0, 'IoU-cradle': 88.24225252015187, 'BoundaryIoU-cradle': 0.0, 'min(IoU, B-Iou)-cradle': 0.0, 'IoU-oven': 18.583277843075038, 'BoundaryIoU-oven': 0.0, 'min(IoU, B-Iou)-oven': 0.0, 'IoU-ball': 42.56516165788349, 'BoundaryIoU-ball': 0.0, 'min(IoU, B-Iou)-ball': 0.0, 'IoU-food, solid food': 54.11295009094271, 'BoundaryIoU-food, solid food': 0.0, 'min(IoU, B-Iou)-food, solid food': 0.0, 'IoU-step, stair': 10.377069868828634, 'BoundaryIoU-step, stair': 0.0, 'min(IoU, B-Iou)-step, stair': 0.0, 'IoU-tank, storage tank': 49.859212591261624, 'BoundaryIoU-tank, storage tank': 0.0, 'min(IoU, B-Iou)-tank, storage tank': 0.0, 'IoU-trade name': 23.83490830781979, 'BoundaryIoU-trade name': 0.0, 'min(IoU, B-Iou)-trade name': 0.0, 'IoU-microwave': 32.48164480479336, 'BoundaryIoU-microwave': 0.0, 'min(IoU, B-Iou)-microwave': 0.0, 'IoU-pot': 19.240628737204098, 'BoundaryIoU-pot': 0.0, 'min(IoU, B-Iou)-pot': 0.0, 'IoU-animal': 50.6686405132001, 'BoundaryIoU-animal': 0.0, 'min(IoU, B-Iou)-animal': 0.0, 'IoU-bicycle': 49.612652146294096, 'BoundaryIoU-bicycle': 0.0, 'min(IoU, B-Iou)-bicycle': 0.0, 'IoU-lake': 51.77399721694862, 'BoundaryIoU-lake': 0.0, 'min(IoU, B-Iou)-lake': 0.0, 'IoU-dishwasher': 63.50638334195827, 'BoundaryIoU-dishwasher': 0.0, 'min(IoU, B-Iou)-dishwasher': 0.0, 'IoU-screen': 75.37359919202183, 'BoundaryIoU-screen': 0.0, 'min(IoU, B-Iou)-screen': 0.0, 'IoU-blanket, cover': 19.238534470814116, 'BoundaryIoU-blanket, cover': 0.0, 'min(IoU, B-Iou)-blanket, cover': 0.0, 'IoU-sculpture': 42.0889209940696, 'BoundaryIoU-sculpture': 0.0, 'min(IoU, B-Iou)-sculpture': 0.0, 'IoU-hood, exhaust hood': 71.07348483307224, 'BoundaryIoU-hood, exhaust hood': 0.0, 'min(IoU, B-Iou)-hood, exhaust hood': 0.0, 'IoU-sconce': 20.76788026759619, 'BoundaryIoU-sconce': 0.0, 'min(IoU, B-Iou)-sconce': 0.0, 'IoU-vase': 7.685233551348926, 'BoundaryIoU-vase': 0.0, 'min(IoU, B-Iou)-vase': 0.0, 'IoU-traffic light': 21.090095104994493, 'BoundaryIoU-traffic light': 0.0, 'min(IoU, B-Iou)-traffic light': 0.0, 'IoU-tray': 7.256603756623982, 'BoundaryIoU-tray': 0.0, 'min(IoU, B-Iou)-tray': 0.0, 'IoU-trash can': 33.549808192055444, 'BoundaryIoU-trash can': 0.0, 'min(IoU, B-Iou)-trash can': 0.0, 'IoU-fan': 46.84614959367571, 'BoundaryIoU-fan': 0.0, 'min(IoU, B-Iou)-fan': 0.0, 'IoU-pier': 10.298450220525426, 'BoundaryIoU-pier': 0.0, 'min(IoU, B-Iou)-pier': 0.0, 'IoU-crt screen': 17.516809210772884, 'BoundaryIoU-crt screen': 0.0, 'min(IoU, B-Iou)-crt screen': 0.0, 'IoU-plate': 39.445888222982866, 'BoundaryIoU-plate': 0.0, 'min(IoU, B-Iou)-plate': 0.0, 'IoU-monitor': 52.21405409076246, 'BoundaryIoU-monitor': 0.0, 'min(IoU, B-Iou)-monitor': 0.0, 'IoU-bulletin board': 24.572773994278336, 'BoundaryIoU-bulletin board': 0.0, 'min(IoU, B-Iou)-bulletin board': 0.0, 'IoU-shower': 0.0, 'BoundaryIoU-shower': 0.0, 'min(IoU, B-Iou)-shower': 0.0, 'IoU-radiator': 49.960802436025446, 'BoundaryIoU-radiator': 0.0, 'min(IoU, B-Iou)-radiator': 0.0, 'IoU-glass, drinking glass': 2.589101607134258, 'BoundaryIoU-glass, drinking glass': 0.0, 'min(IoU, B-Iou)-glass, drinking glass': 0.0, 'IoU-clock': 2.7253903408136995, 'BoundaryIoU-clock': 0.0, 'min(IoU, B-Iou)-clock': 0.0, 'IoU-flag': 24.407059518211938, 'BoundaryIoU-flag': 0.0, 'min(IoU, B-Iou)-flag': 0.0,
    \vspace{-0.1cm}
    \caption{\textbf{The mIoU (\%) of Semantic Segmentation on the ADE20K \texttt{val}.}}
    % \vspace{+2mm}
    % \vspace{+2mm
    \label{ADE20k_semantic}
    % \vspace{-2mm}
\end{table}

\begin{table}[t]
    \centering
    \small 
    % \renewcommand\arraystretch{1.0}
    \setlength{\tabcolsep}{1mm}
    \tablestyle{4pt}{1.2}\scriptsize\begin{tabular}{l|cccc|c}
   Annotation & Bird & Dog & Person & Sofa & \textit{mIoU} \\
  \shline
   Real Image, Manual Label & 93.7   & 96.8 & 92.5& 65.6& 87.2\\
   Synthetic Image, Pseudo Label & 95.2   & 86.2 & 89.9 & 59.5& 82.7 \\
   Synthetic Image, \diffmask & 92.9  & 86.0 &   76.5 & 49.8 & 76.3

  
  \end{tabular}


    \vspace{-0.2cm}
    \caption{\textbf{Impact of Mask Precision and Domain Gap on VOC 2012 \texttt{val}.} Mask2former~\cite{cheng2022masked} with Swin-B is used as the baseline. `Pseudo' denotes pseudo mask annotation from Mask2former~\cite{cheng2022masked} pre-trained on VOC 2012.}
    % \vspace{+2mm}
    \label{gap}
    % \vspace{-2mm}
\end{table}


\begin{table}[t]
    \centering
    \small 
    % \renewcommand\arraystretch{1.0}
    \setlength{\tabcolsep}{1mm}
    \tablestyle{4pt}{1.2}\scriptsize\begin{tabular}{l|cc|c}
  Attention Map $\mathcal{A}$ & Bird & Dog  &\textit{mIoU} \\
  \shline
   $8\times8$ & 40.5   & 46.0 &  43.3 \\ $16\times16$ & 58.8   & 69.9 &  64.4\\
   $32\times32$ & 86.2   & 82.3 &  84.3\\
   $64\times64$ & 45.2   & 41.1 &  43.2\\
   $16\times16$,$32\times32$,$32\times32$ & 89.9   & 84.2 &  87.1\\
   Average & 92.9   & 86.0 &  89.5

  \end{tabular}


    \vspace{-0.2cm}
    \caption{\textbf{Impact of different attention maps from different layers.} Mask2former~\cite{cheng2022masked} with Swin-B is used as the baseline.}
    % \vspace{+2mm}
    \label{fature}
    % \vspace{-2mm}
\end{table}

\begin{table}[t]
    \centering
    \small 
    % \renewcommand\arraystretch{1.0}
    \setlength{\tabcolsep}{1mm}
    \tablestyle{4pt}{1.2}\scriptsize\begin{tabular}{l|ccccc|c}
   Backbone & Bird & Dog & Sheep & Horse & Person &\textit{mIoU} \\
  \shline
   RseNet 50 & 86.7   & 65.1 &  64.7 & 64.6 & 71.0 & 70.3\\
   RseNet 101 & 86.7 & 66.8  & 65.3 & 63.4& 70.2& 70.5\\
   Swin-B & 92.9  & 86.0 & 92.2 & 89.0 & 76.5 & 87.3 \\
   Swin-L & 92.8  & 86.4 & 92.3 & 88.3 & 77.3& 87.4

  \end{tabular}


    \vspace{-0.2cm}
    \caption{\textbf{Impact of Backbone on VOC 2012 \texttt{val}.} Mask2former~\cite{cheng2022masked} is used as the baseline. }
    % \vspace{+2mm}
    % \vspace{+2mm
    \label{backbone}
    % \vspace{-2mm}
\end{table}

\begin{figure}[!t]
% 	\centering
	\includegraphics[width=0.99\linewidth]{iccv2023AuthorKit/figures/fig2_attention_graident.pdf}
	\vspace{-0.2cm}
	\caption{\textbf{Gradient from Text Tokens for Stable Diffusion.} Prompt language: `\texttt{a horse on the grass}'.}
	\vspace{-0.1cm}
\label{Gradient}
\end{figure}


\begin{figure}[!t]
% 	\centering
	\includegraphics[width=0.99\linewidth]{iccv2023AuthorKit/figures/backbone.pdf}
	\vspace{-0.2cm}
	\caption{\textbf{Impact of Backbone.} Stronger backbone is robust for classification, False Negative, and mask precision.}
	\vspace{-0.1cm}
\label{backbone_vis}
\end{figure}


\section{More Ablation Study}

\textbf{What causes the performance gap between synthetic and real data.}
Domain gap and mask precision are the main reasons for the performance gap between synthetic and real data.
%
Tab.~\ref{gap} is set To further explore the problem.
%
Li \textit{et al.}~\cite{li2023guiding} shows that the pseudo mask of the synthetic image from Mask2former~\cite{cheng2022masked} pre-trained on VOC 2012 is quite accurate, and can as the ground truth.
%
Thus, we also use the pseudo label from the pre-trained Mask2former to train the model, where we argue that the pseudo label is accurate.
%
As shown in Tab.~\ref{gap}, mask precision cause $6.4\%$ mIoU gap, and the domain gap of images causes  $4.5\%$ mIoU gap.
%
Notably, for the \texttt{bird} class, the use of synthetic data with a pseudo label resulted in better results than the corresponding real images. 
%
This observation suggests that there may be no domain gap for the \texttt{bird} class in the VOC 2012 dataset.

\begin{figure*}[!t]
% 	\centering
	\includegraphics[width=0.99\linewidth]{iccv2023AuthorKit/figures/fig2_cityscapes.pdf}
	% \vspace{-0.2cm}
	\caption{\textbf{Visualization for the model trained with \textit{only} \diffmask on Cityscapes.} \diffmask presents a competitive performance on challenging driving scenario.}
	% \vspace{-0.2cm}
\label{VIS122}
\end{figure*}
\textbf{Backbone}
Tab.~\ref{backbone} presents the ablation study for the backbone.
%
For some classes, \eg{} \texttt{sheep}, the stronger backbone can bring obvious gains, \ie{} Swin-B achieves $27.5\%$ mIoU improvement than that of ResNet 50.
%
And the mIoU of all classes with Swin-B achieves $19.2\%$ mIoU improvements.
%
It is an interesting and novel insight that a stronger backbone can reduce the domain gap between synthetic and real data. 
%
To give a further analysis for that, we present some results comparison of visualizations, as shown in Fig.~\ref{backbone_vis}.
%
Swin-B brings an obvious improvement in classification, False Negatives, and mask precision.
%
Compared with the gain between different backbones, different versions (size) of the same backbone seems can not obtain an effective gain, \eg{} ResNet101 only obtain $0.5\%$ mIoU improvements than that of ResNet50.


\textbf{Attention Maps of different resolutions.}
Table \ref{fature} shows the results of an ablation study conducted on cross attention maps with varying resolutions from different layers. The performance of both high resolution ($64\times64$) and low resolution ($8\times8$) maps was found to be unsatisfactory. This can be attributed to the lack of detail in low-resolution maps and the presence of noise in high-resolution maps. On the other hand, integrating (by averaging) all attention maps produced the best performance.
%




\section{Experiment on ADE20K}
%
ADE20K, as one more challenging dataset, is also used to evaluate the \Ours.
%
Tab.~\ref{ADE20k_semantic} presents the results of three categories~(\texttt{bus}, \texttt{car}, \texttt{person}) on ADE20K.
%
With fewer synthetic images~($6$k), we achieve a competitive performance than that of a mass of real images~($20.2$k).
%
Compared with the other two categories, Class \texttt{car} achieves the best performance, with $73.4\%$ mIoU. 

\section{Visual explanation with gradients.} The gradient is another way to provide an excellent visual explanation of the generative model, Fig.~\ref{Gradient} presents the corresponding gradient visualization from different text tokens.
%
Given a text prompt $\mathcal{P}$, \ie{}, `\texttt{a horse on the grass}' and a random Gaussian image noise $z$, the text-guided generative model is in principle capable of modeling conditional distributions of the form $\mathcal{I}:= p(z \vert \conditioner(\mathcal{P}))$, where $\conditioner(\mathcal{P}) \in \R^{N\times d }$ and $\conditioner$ refers to the text encoder~\cite{radford2021learning}.
%
For the $k$-th word $t_k$~(\eg{}, `\textit{horse}' from Fig.~\ref{Gradient}) from $\mathcal{P}$, we can compute the corresponding gradient as following: $\alpha^{}_{k} = \vphantom{\sum_{i}\sum_{j}} \frac{\partial \mathcal{I}}{\partial t_k}$
% \begin{equation} 
% \label{eq:alpha1}
%     \alpha{}_{i} = \vphantom{\sum_{i}\sum_{j}} \frac{\partial A_{ij}}{\partial t_i}
% \end{equation}
%
% , where $A_{ij}$ refers to the pixel of generative image $\mathcal{I}$. 
%
where $\alpha{}_{k}$ is the gradient weight from the $k$-th word $t_k$.
%
The corresponding gradient weight can be computed by adding a small variate (that is, Numbers close to zero) to the $t_k$.
%
For convenience, we add a small variate $\bigtriangleup \beta \in \R^{d} $~($\bigtriangleup \beta = \mu \vec{\textbf{1}}_{d}$, where $\vec{\textbf{1}}_{d}$ and $\mu$ refer to the unit matrix and weight) to the text feature map $\conditioner(t_k)$ and obtain the corresponding gradient visualization, as shown in Fig.~\ref{Gradient}.
%
The gradient visualization is highly class-discriminative (\ie{} the `\texttt{horse}' explanation exclusively highlights the `\textit{horse}' regions).



\section{Limitation.} \diffmask mainly includes two limitations: 1) The inference speed of the text-to-image diffusion model is relatively slow.
%
With 8 Tesla V100 GPUs, generating $10k$ images usually need to spend around 8 hours.
%
Therefore, scaling up the synthetic dataset to a million level is difficult for some institutions.
%
And it is the main reason why we do not provide more experiments for other datasets with rich categories, \eg{} ADE20K or COCO.
%
Similarly, we can not scale up the synthetic data to the million level due to the limitation of time and computational cost.
%
But we argue the cost can be reduced by adopting advanced faster Sampling for Diffusion Models~\cite{song2020score,karras2022elucidating}.
%
2) There are still existing obvious result gaps for some classes, \eg{} \texttt{person} on VOC 2012.
%
The main reason is the obvious domain gap for these classes.
%
The synthetic image usually presents a simple foreground and background, while the real image is more examples with multi-views, multi-scales, blur, and occlusion.
%
Even so, our \diffmask, as the first work to synthesize image and mask annotation using an image-text pre-trained diffusion model, provide a promising performance and many new insights.
%
We verify the feasibility of training with text-driven synthetic data and applications in the real world, where worth mentioning the diffusion model is trained with only language-image pairs.


{\small
\bibliographystyle{ieee_fullname}
\bibliography{egbib}
}
\end{document}