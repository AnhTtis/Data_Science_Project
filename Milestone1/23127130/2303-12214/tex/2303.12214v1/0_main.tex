% This is samplepaper.tex, a sample chapter demonstrating the
% LLNCS macro package for Springer Computer Science proceedings;
% Version 2.21 of 2022/01/12
%
\documentclass[runningheads]{llncs}
%
\usepackage[T1]{fontenc}
% T1 fonts will be used to generate the final print and online PDFs,
% so please use T1 fonts in your manuscript whenever possible.
% Other font encondings may result in incorrect characters.
%
\usepackage{graphicx}
% Used for displaying a sample figure. If possible, figure files should
% be included in EPS format.
%
% If you use the hyperref package, please uncomment the following two lines
% to display URLs in blue roman font according to Springer's eBook style:
\usepackage{color}
% \renewcommand\UrlFont{\color{blue}\rmfamily}

\usepackage{times}
\usepackage{epsfig}
\usepackage{amsmath}
\usepackage{amssymb}
\usepackage{bm}
\usepackage{booktabs}
\usepackage{tablefootnote}
\usepackage{color}


\newcommand{\ceil}[1]{\lceil {#1} \rceil}
\usepackage{tikz}
\usepackage{pifont}
            
\newcommand{\KM}[1]{{\color{red} \textbf{KM}: #1}}
\newcommand{\JZ}[1]{{\color{blue} \textbf{JZ}: #1}}
% \newcommand{\JZ}[1]{#1}
\newcommand{\PP}[1]{{\color{cyan} \textbf{PP}: #1}}
\newcommand{\MV}[1]{{\color{magenta} \textbf{MV}: #1}}

\newcommand{\resultsection}[1]{\noindent\textbf{{#1}}}
% \newcommand{\resultsection}[1]{\paragraphdata{#1}}
\newcommand{\datasection}[1]{\textbf{#1}}
% \newcommand{\datasection}[1]{\paragraphdata{#1}}

\newcommand{\vspaceadj}{\vspace{0.8\baselineskip}}

%
\begin{document}
%
\title{Prompt-MIL: Boosting Multi-Instance Learning Schemes via Task-specific Prompt Tuning}
% Prompt-MIL: Boosting Multi-Instance Learning Schemes for Whole Slide Image Classification
%
\titlerunning{Prompt-MIL: Boosting MIL Schemes via Prompt Tuning}
% If the paper title is too long for the running head, you can set
% an abbreviated paper title here
%
\author{Jingwei Zhang\inst{1} \and
Saarthak Kapse\inst{1} \and Ke Ma \inst{2} \and Prateek Prasanna\inst{1} \and
Joel Saltz\inst{1} \and Maria Vakalopoulou\inst{3} 
\and Dimitris Samaras\inst{1}}
%
\authorrunning{F. Author et al.}
% % First names are abbreviated in the running head.
% % If there are more than two authors, 'et al.' is used.
% %
\institute{
    Stony Brook University, USA \and Snap Inc., USA
        \and
        CentraleSupélec, University of Paris-Saclay, France\\
    \email{\email{\{jingwezhang, kemma, samaras\}@cs.stonybrook.edu}} \\
    \email{\{saarthak.kapse, prateek.prasanna\}@stonybrook.edu} \\
    \email{Joel.Saltz@stonybrookmedicine.edu   maria.vakalopoulou@centralesupelec.fr}
}

% \author{Anonymous}
% \authorrunning{Anonymous et al.}
% \institute{Anonymous Organization\\
% { \email{**@******.***} } }

%%%% llncs was modified by adding paragraphdata

%
\maketitle              % typeset the header of the contribution
%
\begin{abstract}
Whole slide image (WSI) classification is a critical task in computational pathology, requiring the processing of gigapixel-sized images, which is challenging for current deep-learning methods. 
Current state of the art methods are based on multi-instance learning schemes (MIL), which usually rely on pretrained features to represent the instances. 
Due to the lack of task-specific annotated data, these features are either obtained from well-established backbones on natural images, or, more recently from self-supervised models pretrained on histopathology. 
However, both approaches yield task-agnostic features, resulting in performance loss compared to the appropriate task-related supervision, if available.
In this paper, we show that when task-specific annotations are limited, we can inject such supervision into downstream task training, to reduce the gap between fully task-tuned and task agnostic features. 
We propose Prompt-MIL, an MIL framework that integrates prompts into WSI classification. 
Prompt-MIL adopts a prompt tuning mechanism, where only a small fraction of parameters calibrates the pretrained features to encode task-specific information, rather than the conventional full fine-tuning approaches.
Extensive experiments on three WSI datasets, TCGA-BRCA, TCGA-CRC, and BRIGHT, demonstrate the superiority of Prompt-MIL over conventional MIL methods, achieving a relative improvement of 1.49\%-4.03\% in accuracy and 0.25\%-8.97\% in AUROC while using fewer than 0.3\% additional parameters. 
Compared to conventional full fine-tuning approaches, we fine-tune less than 1.3\% of the parameters, yet achieve a relative improvement of 1.29\%-13.61\% in accuracy and 3.22\%-27.18\% in AUROC and reduce GPU memory consumption by 38\%-45\% while training 21\%-27\% faster.


\keywords{Whole slide image classification  \and Multiple instance
learning \and Prompt tuning.}
\end{abstract}
%
%
%

\section{Introduction}
\label{sec:intro}
\begin{figure}[t]
\begin{center}
    \includegraphics[width=1\linewidth]{figures/teaser.pdf}
\end{center}
\vspace{-0.1in}
\caption{\textbf{{\em Foggy} vs {\em Clear} NeRF.} Our \ournerf gets rid of reconstruction errors manifested as foggy ``floaters" in the density volume without additional input or significant computational overhead. 
%
Below are density profiles along a given ray before and after our geometry correction procedure, where we discard density peaks corresponding to floaters.
}
\label{fig:teaser}
\vspace{-0.2in}
\end{figure}



%The emergence of 
Neural Radiance Fields (NeRFs)~\cite{mildenhall2020nerf}  %and its variants 
have made revolutionary contributions in %photo-realistic 
novel view synthesis~\cite{barron2021mip,barron2022mip}, 
autonomous driving~\cite{rematas2022urban,tancik2022block}, digital human~\cite{hong2022headnerf,zhao2022humannerf}, and 3D content generation~\cite{eg3d,poole2022dreamfusion,lin2022magic3d}.
%by leveraging a multi-layer perceptron (MLP) to implicitly model the mapping from input 5D coordinates (i.e., 3D coordinates $\mathbf{x} = (x,y,z)$ and 2D viewing directions $\mathbf{d}=(\theta,\phi)$) to volume density $\sigma$ and view-dependent emitted radiance color $\mathbf{c} = (r,g,b)$. 
%
%They then use traditional volume rendering mechanisms on the obtained continuous 5D function (i.e., MLP) to generate novel views. 
To date, unfortunately, most NeRF-based methods encounter challenges when tackling large-scale cluttered scenes (e.g., Fig.~\ref{fig:teaser}):
\begin{enumerate}[leftmargin=0.16in, topsep=2pt,itemsep=-1ex,partopsep=1ex,parsep=1ex]
\item Input observations used for NeRF are often too sparse  compared to forward-facing or synthetic looking-inward scenes;
%\item Recovering fine-grained objects within a large volume is challenging for NeRF; %in capturing details accurately.
\item View-dependent visual effects give rise to ambiguity, resulting in a ``foggy" density field as shown in Fig.~\ref{fig:teaser}. 
%
Such artifacts are particularly pronounced in indoor scenes strewn with view-dependent appearances, such as specular highlights, glossy surface reflections from man-made objects. 
\end{enumerate}

Despite attempts to enhance NeRF's rendering quality given suboptimal input, such as using 3D conical frustums~\cite{barron2021mip,barron2022mip}, physically-grounded augmentations~\cite{chen2022aug}, and misalignment correction~\cite{jiang2022alignerf},  these challenges have yet to be fully resolved.
%
Depth supervision~\cite{deng2022depth, wei2021nerfingmvs} or proxy geometry~\cite{xu2021scalable,wu2022scalable} images can help alleviate the challenges in handling large-scale with sparse input, at the expense of %but they come at the cost of requiring 
expensive pre-processing or additional input.
%
Another line of work~\cite{wang2021neus, oechsle2021unisurf, wang2022neuris} achieves better reconstruction of surface geometry by using signed distances instead of volume density as scene representation. However, they sacrifice the ability to synthesize photo-realistic novel views.

%We observe that NeRF has been suffering from foggy ``floater" artifacts in large-scale cluttered scenes.
%
%Such artifacts are particularly pronounced in indoor scenes strewn with view-dependent appearances from man-made objects. 
%
To address the above issues, we propose an extension to NeRF, dubbed as {\bf \ournerf}, which enforces effective {\em appearance} and {\em geometry} constraints conducive to accurate colors and 3D densities estimation. We believe \ournerf can contribute beyond novel view synthesis, such as NeRF object detection~\cite{hu2022nerf}, NeRF object segmentation~\cite{zhi2021place, liu2022unsupervised, fan2022nerf,ren2022neural}, and NeRF registration~\cite{goli2022nerf2nerf}, where the rooms for improvement are substantial if more accurate color and density estimation are available.

Correspondingly, there are two steps in \ournerf. First, for appearance correction, the view-independent and view-dependent color components are predicted from the underlying 3D scene, which is combined to produce the final color estimation (Fig.~\ref{fig:toaster}).
%
The view-independent component (diffuse color and shading) captures the overall scene color, while the view-dependent component (highlights or reflections) captures color variations due to changes in viewing angle.
%
\ournerf then discards these view-dependent appearances in the training views to prevent them from interfering with the density estimation.
%
Second, a simple and effective geometry correction procedure will be performed to further eliminate the foggy ``floaters" or density errors. This geometry correction procedure is based on an assumption in line with traditional ray tracing in computer graphics.
\begin{comment}
% xh: basically copying method
On the other hand, ClearNeRF performs a geometric correction procedure performed on each traced ray during inference to refine the density estimation and better tackle the floater artifacts. 
%
The geometry correction procedure assumes that there should only be one salient peak along each traced ray during NeRF inference. 
Only the salient peak closest to the ray origin (the camera center) corresponds to  true geometry while the others will be manifested as foggy floaters hovering in the density volume. 
%
This assumption is in line with traditional ray tracing in computer graphics where in the absence of noise, only one intersection per ray should be returned to indicate the closest ray-object intersection.
%
\end{comment}
%%%%%%%%%%%
%As shown in Fig.~\ref{fig:teaser}, when reconstructing an indoor scene with sparse input and highly view-dependent objects, NeRF produces severe floating artifacts due to its attempt to explain view-dependent appearances.
%
Experiments verify that our proposed \ournerf can effectively get rid of floater artifacts without additional input.% or significant computational overhead. 


In summary, our contributions include the following:
\begin{itemize}[leftmargin=0.16in, topsep=2pt,itemsep=-1ex,partopsep=1ex,parsep=1ex]
    \item We propose a concise method for decomposing view-independent and view-dependent appearance during NeRF training and eliminate the interference of view-dependent appearance.
    \item We propose a geometric correction procedure performed on each traced ray during inference to refine the density estimation and better tackle the floater artifacts.
    \item Extensive experiments and ablations verify the effectiveness of our core designs and results in improvements over the vanilla NeRF and other state-of-the-art alternatives.
    %without additional computational resources or other inputs.
\end{itemize}




%%%%%%%%%%%%%%%%%%%%%%%%%%%%%%%%%%%%%%%%%%%%%%%
%%%%%%%        2. Proposed Approach         %%%%%%%
%%%%%%%%%%%%%%%%%%%%%%%%%%%%%%%%%%%%%%%%%%%%%%%

\section{Proposed Approach}
\label{sec:methods}

A depiction of our approach is shown in Figure \ref{fig:model}. The model consists of a MOS prediction model (shown left) and a speech enhancement model (shown right). Our MOS prediction model is tailored to provide estimates for subjective-MOS (as rated by humans), and going forward, we will use MOS to refer to subjective-MOS unless explicitly stated otherwise, for ease of understanding. We next will provide notation and then describe each of these sub-modules.

\subsection{Notation}

We define a clean speech signal as $s_t$ and background noise as $n_t$ at time $t$. The mixture of clean speech and noise is denoted as $m_t=s_t+n_t$. We aim to extract the speech from the mixture by removing the unwanted noise. The short-time Fourier transform (STFT) converts the time-domain mixture into a T-F representation, $M_{t,f}$, that is defined at time $t$ and frequency $f$. The complex-valued STFT matrix, $\bm{M}$, can be written as $\bm{M}=|\bm{M}|e^{i\bm{\theta}^M}$ with magnitude $|\bm{M}|\in \bm{\Re}^{T\times F}_+$ and phase $\bm{\theta}^M \in \bm{\Re}^{T\times F}$, where $T$ is the length of speech in time and $F$ is the total number of frequency channels.

Enhancing the magnitude response of noisy speech results in an estimate of the clean speech magnitude response, $|\hat{\bm{S}}|$, using an enhancement function $\mathcal{F}_\delta$ such that $|\hat{\bm{S}}| =\mathcal{F}_\delta(|\bm{M}|)$. The enhancement function is modeled with a deep neural network which is trained to maximize the conditional log-likelihood of the training dataset, 
\begin{align*}
    &\max \frac{1}{N} \sum^N \log P\Big( |{\bm{S}}| \, \Big| \, |\bm{M}|\Big) \\
    \Rightarrow &\max_\delta \frac{1}{N} \sum^N \log P\Big( \mathcal{F}_\delta(|\bm{M}|) \, \Big| \, |\bm{M}|\Big)
\end{align*}
% $$\max \frac{1}{N} \sum^N \log P\Big( |{\bm{S}}| \, \Big| \, |\bm{M}|\Big) \Rightarrow \max_\delta \frac{1}{N} \sum^N \log P\Big( |\hat{\bm{S}}| \, \Big| \, |\bm{M}|\Big) $$
where $\delta$ denotes the set of tunable parameters and $N$ is the number of training examples. The estimated magnitude response $|\hat{\bm{S}}|$ is then combined with the noisy phase, $\bm{\theta}^M$, where the inverse STFT produces an enhanced speech signal in the time domain, $\hat{s}_t$. 

\subsection{Speech quality assessment model}
\label{subsec:mos_model}

A MOS prediction model proposed by \cite{dong2020pyramid} is adapted to estimate the MOS from noisy speech. This model has been developed with real-world captured data and it has been shown to outperform comparison approaches~\cite{fu2018quality, avila2019non, mittag2019non}, according to multiple metrics. The MOS prediction model consists of an attention-based encoder-decoder structure that uses stacked pyramid bi-directional long-short term memory (pBLSTM)~\cite{chan2016listen} networks in the encoder. We denote this model as Pyramid-MOS (PMOS). A pBLSTM architecture gives the advantages of processing sequences at multiple time resolutions, which effectively captures short- and long-term dependencies. Speech has spectral and temporal dependencies over short and long durations, and a multi-resolution framework is effective in learning these complex relations. 


A single T-F frame of the noisy-speech mixture, $|\bm{M}_t|$, is the input to the PMOS encoder. In a pyramid structure, the lower layer outputs from $\Upsilon$ consecutive time frames are concatenated and used as inputs to the next pBLSTM layer, along with the recurrent hidden states from the previous time step. The output of a pBLSTM node is an embedding vector, $h^l_t$, that is as defined below:
\begin{align}
    h^l_t &= pBLSTM\Big( h^l_{t-1}, \big[ h^{l-1}_{\Upsilon\times t -\Upsilon+1}, h^{l-1}_{\Upsilon\times t}\big] \Big)
\end{align}
where $\Upsilon$ is the reduction factor (e.g., number of concatenated frames) between successive pBLSTM layers and $l$ is the layer number. A pBLSTM reduces the time resolution from the input speech to the final latent representation $\bm{H}$. Figure~\ref{fig:pBLSTM} shows the internal structure of pBLSTM module.
This compressed vector accumulates the useful features for measuring speech perceptual quality that resides in a range of time-frames and ignores the least important features.
The encoder outputs a concatenated version of the hidden states of the last pBLSTM layer as vector $\bm{H}=\{\bm{h}_1, \dotsb, \bm{h}_\tau, \dotsb, \bm{h}_\wp\}$, where $\wp$ is the total number of final embedding vectors with time index $\tau$.

The output of the PMOS encoder becomes the input to the PMOS decoder unit. This decoder is implemented as an attention layer followed by a fully-connected (FC) layer and it outputs an estimated MOS of the input speech utterance. Attention models learn key attributes of a latent sequence, since adjacent time frames can provide important information, which is particularly crucial for our task.  
The attention mechanism~\cite{luong2015effective} uses the pyramid encoder output at the $i$-th and $k$-th time steps to compute the attention weights, $\alpha^{PMOS}_{i,k}$. Attention weights are used to compute a context vector, $c^{PMOS}_i$, using the following equations:
\begin{align}
    \alpha^{PMOS}_{i,k} &= \frac{\exp{(\bm{h}_i^\top \bm{Q} \bm{h}_k)}}{\sum^{\wp}_{\phi=1} \exp{(\bm{h}_i^\top \bm{Q} \bm{h}_\phi)}}\\
    % \alpha^{PMOS}_{i,k} &= Attention(\bm{h}_i, \bm{h}_k)\\
    c^{PMOS}_i &= \sum^\wp_{k=1} \alpha^{PMOS}_{i,k} \cdot \bm{h}_k
\end{align}
$\bm{Q}^{\wp\times\wp}$ is the trainable PMOS attention weight matrix. We learn $\bm{Q}$ using a feed-forward neural network that attempts to capture the alignment between the embeddings $\bm{h}_i$ and $\bm{h}_k$. 

The context vector is provided to a fully-connected layer to estimate the MOS. Note that the pyramid structure of the encoder results in a shorter sequence of latent representations than the original input sequence, and it leads to fewer encoding states for attention calculation at the decoding stage. Therefore, strictly  $\wp<T$, and in our case $\wp = \lceil T/\Upsilon^L \rceil$, where $L$ is the number of pBLSTM layers.
We train the PMOS model separately with the parameters defined in~\cite{nayem2019incorporating}. After training, this model is held frozen during inference.

\begin{figure}[t!]
    \centering
    \includegraphics[width = 0.95\linewidth]{figs/pBLSTM.png}
    \caption{Illustration of pBLSTM structure with reduction factor $\Upsilon=2$ and number of layer $L=2$.}
    % \vspace{-2em}
    \label{fig:pBLSTM}
    % \vspace{-0.4cm}
\end{figure}

\subsection{Proposed speech enhancement model}
\label{subsec:se_model}
Our proposed speech-enhancement (SE) model follows an encoder-decoder structure, and it is shown in Figure \ref{fig:model} (right). The SE encoder takes a single T-F frame of a noisy-speech mixture, $|\bm{M}_t|$, as input and multiple BLSTM layers, are stacked together to create a hidden representation of the frame, $\bm{g}_t$. In our SE encoder, we utilize BLSTM layers instead of pBLSTM layers since we aim to estimate an embedding frame for each time frame and pBLSTM layers reduce the number of output frames. 
An attention mechanism is applied using the mixture encoding from the SE model, $\bm{G}=\{\bm{g}_1, \bm{g}_2, \dotsb, \bm{g}_T\}$, and the PMOS encoding, $\bm{H}$, from the MOS prediction model. This allows the SE model to exploit the MOS estimator's encoding and utilize the important perceptual feature embedding that correlates with human assessment. Considering that the pBLSTM structure of the PMOS encoder condenses the final encoding vector $\bm{H}$ along time, PMOS yields a smaller time resolution than the encoding from the SE encoder, so we compute a score for each embedding vector $\bm{h}_{\tau}$ using an alignment  weight matrix, $\bm{W}^{T\times\wp}$. Then the attention weights for the SE model, $\alpha_{t,\tau}$, are obtained using a softmax operation over the scores of all $\bm{h}_\tau$. Now, the PMOS encoding is summarized in a context vector $\bm{c}_t$ for each mixture frame $\bm{g}_t$. Prior to computing $\bm{c}_t$, $\bm{h}_\tau$ passes through a linear layer $\ell$, so that we learn a different representation for the SE task. The computations are below:
\begin{align}
    \alpha_{t,\tau} &= \frac{\exp{(\bm{g}_t^\top \bm{W} \bm{h}_\tau})}{\sum^{\wp}_{\phi=1} \exp{(\bm{g}_t^\top \bm{W} \bm{h}_\phi)}} \\
    \bm{c}_t &= \sum_{\tau=1}^\wp \alpha_{t,\tau} \cdot \ell (\bm{h}_\tau)
\end{align}
\noindent
Then, the context vector and SE-model embedding vector are concatenated (e.g., $[\bm{c}_t, \bm{g}_t]$) and passed to the decoder module. The SE-decoder module follows the network structure from \cite{schulze2020joint}. It consists of a linear layer with a $tanh(\cdot)$ activation function, two BLSTM layers, and a linear layer with ReLU activation. It outputs the estimated enhanced speech $|\hat{\bm{S}}|$. This estimated speech magnitude with noisy phase produce the estimated clean speech, i.e. $\hat{\bm{S}} = |\hat{\bm{S}}|e^{i\bm{\theta}^M}$. Since we are estimating two targets MOS and enhanced speech simultaneously, the unified model will learn different representations for these tasks. Thus both PMOS and SE models will learn their corresponding targets with perceptual feature sharing. We freeze the PMOS model while training this SE model.


\subsection{Joint-learning of PMOS and SE model}
\label{subsec:joint_model}
We also develop an approach that allows the PMOS and SE models to be jointly trained. Our joint-learning objective function uses a weighted average of a {time-domain} signal-approximation loss $\mathcal{L}_{sa}$ (from the SE model), the MSE of the magnitude spectrum $\mathcal{L}_{mse}$ (from the SE model) and the MSE of the MOS estimation $\mathcal{L}_{mos}$ (from the PMOS model). We compute the signal-approximation loss from the time-domain signal difference between the reference speech $s$ and enhanced speech $\hat{s}$. The overall loss function of our network is defined as below, using hyper-parameters $\lambda_1$ and $\lambda_2$ that control the impact of individual loss terms:
\begin{align}
    \mathcal{L} &= \lambda_1\left[\lambda_2\mathcal{L}_{mse} + (1-\lambda_2)\mathcal{L}_{sa}\right] + (1-\lambda_1)\mathcal{L}_{mos}
    \label{eq:loss}
\end{align}
\noindent
The model training order is as such. First, we train the PMOS model using $\mathcal{L}_{mos}$ (e.g. $\lambda_1 = 0$). Then we train the SE model using $\lambda_1 = 1$, while running the PMOS model in inference mode (e.g. it is held fixed). This is done to ensure that the trained PMOS model effectively encodes the key features in the embedding vector that are important to perceptual speech quality. Finally, we train both the models jointly (e.g. $0 < \lambda_1 < 1$) using $\mathcal{L}$ to further reduce any correctional differences between the true and estimated MOS in the PMOS model, and to increase the perceptual quality of the enhanced speech.
\begin{figure}[t!]
    \centering
    \includegraphics[width = 0.8\linewidth]{figs/quant_fig2.png}
    \caption{Quantization of a clean magnitude spectrum.}
    % \vspace{-2em}
    \label{fig:quant}
    % \vspace{-0.4cm}
\end{figure}
\subsection{Quantized Spectral Model}
\label{subsec:QSM}
% An external language model can integrate additional information regarding speech correlation which is helpful for improving enhancement performance. Typical LM is applied at the phoneme or word level and the performance of the LM depends on the text and its vocabulary. Additionally, parallel corpus of speech and text is a requirement for training which rules out a huge number of corpus from usage. 
%A language model (LM) serves as prior knowledge on acoustic input that constrains the alternative word (or phonetic) hypothesis during speech recognition by learning which sequences of words (or phonetics) are most likely to be spoken. LM predicts which words will follow on from the current words and with what probability. $\mathbb{N}$-gram LM is a widely used approach which estimates the probability of a given sequence of words $w_{1\cdots\Omega}$ within the assumption that the probability of word $w_\delta$ depends only on previous $(\mathbb{N}-1)$ words, and the probability can be expressed as: 
%\begin{align}
%    P(w_{1\cdots\Omega})=\prod_{w_\delta} P(w_\delta|w_{\delta-1}, w_{\delta-2},\cdots,w_{\delta-\mathbb{N}+1})
%\end{align}
%Compared to conventional ASR approaches, deep ASR systems model learn in-house LM \cite{yu2016automatic}; and they can be coupled with SE task~\cite{weninger2015speech, wang2020complex}. LM helps a SE model by predicting probability of next utterance, which is otherwise will be any utterance in the whole speech spectrum. However, deep LM typically require more data to achieve comparable results. Additionally, parallel corpus of speech and text is a requirement for training which rules out a huge number of raw audio collections from usage.
%Therefore, we adapt an alternative view of a LM from \cite{nayem2021towards}, where quantized t-f values are considered as word. 
From written and spoken language, we can determine the sequences of words that are most likely to occur. This knowledge is captured by a language model (LM) of an automatic speech recognition system which we can expressed as,
\begin{align}
    \hat{words}=\argmax_{words\in Language} \overbrace{P(input|words)}^{acoustic\;model} \overbrace{P(words)}^{language\;model}
\end{align}
%Here, the most likely word sequence, $\hat{words}$, is estimated by an acoustic model that calculates the probability of the input audio given the word sequence $words$, and by a language model that gives the likelihood of the word sequence. Hence, the LM predicts the probability of a sequence of words. 
The LM is useful in eliminating rare and grammatically incorrect word sequences, and it enhances the performance of ASR systems. In the case of speech enhancement, models learn spectral information within frames over time, but they often neglect the temporal correlations. Our approach, as proposed in \cite{nayem2021towards}, suggests incorporating a ``LM" to fuse temporal correlations and overcome this limitation. Therefore, we construct a bi-gram Quantized Spectral Model (QSM), which functions in a similar way to a language model (LM), in order to produce more realistic spectra. The QSM estimates the probability of spectral magnitudes along time for each frequency channel conditioned on its previous T-F spectral magnitude. %Range of T-F unit values is constrained in a signal approximating SE system and is far smaller than typical spoken language vocabulary size. As a result, the training time and computational resource requirement are quite small fo spectral LM.
On a reference speech corpora, we apply a normalization scaling function, $\mathcal{N}_{[o,r]}(\cdot)$, that normalizes the magnitude spectrogram and re-scales the range to $[0,r]$. Then a quantization function, $\mathcal{Q}_\chi(\cdot)$, converts the range constrained magnitude spectrogram into $\mathcal{D}$ number of bins that are $\chi$ steps apart. This produces quantized speech, i.e. $|S|^q = \mathcal{Q}_\chi\big(\mathcal{N}_{[0,r]}(|S|)\big)$. Fig.~\ref{fig:quant} shows an example of the original clean and quantized clean magnitude spectra, where $\chi=2$ for display purposes. Our proposed QSM has $\mathcal{D}$ spectral levels. We construct the QSM using quantized speech magnitudes from the clean speech corpora. The QSM is less likely to suffer from the out of vocabulary problem when the model parameters, $\chi$ and $r$, are adequately defined.

%\begin{figure}[tbh!]
%    \centering
%    \includegraphics[width = \linewidth]{IEEEtran/figs/fQSM.png}
%    \caption{Proposed Quantized Spectral Models (QSMs) for per-frequency-channel.}
%    % \vspace{-2em}
%    \label{fig:fQSM}
%    % \vspace{-0.4cm}
%\end{figure}

We compute per-frequency-channel QSMs along the time axis where each entry, $d$, refers to a quantization attenuation level. We then compute the transition probability between two time consecutive T-F units, $fQSM_f = P(d_{t+1,f}|d_{t,f})$. The probabilities are calculated by counting the level transitions, and then normalizing by the appropriate scalar. These probabilities are stored in the per-frequency-channel QSM resulting in a $F\times \mathcal{D}\times \mathcal{D}$ probability matrix. %Figure~\ref{fig:fQSM} shows proposed QSMs along per-frequency-channel. 
We re-evaluate the transition probabilities using Good-Turing smoothing~\cite{jurafskyMartin2009} to overcome the zero-probability problem in N-grams. Shallow fusion~\cite{gulcehre2015using} is a simple method to incorporate an external LM into an encoder-decoder model, and it produces better results compared to others. Hence, we use shallow fushion to combine our QSM and SE model based on log-linear interpolations at inference time. This is shown in the below equations:

\begin{align}
    P^{QSM}_f(|\hat{\bm{S}}_{:,f}|) &= \prod^T_{i=1} P(d_{i,f}|d_{i-1,f}) \\
    |\hat{\bm{S}}_{:,f}|^* = \argmax_{|\hat{\bm{S}}_{:,f}|} &\log P\big(|\hat{\bm{S}}_{:,f}| \big| |\bm{M}|\big) + \mu \log P^{QSM}_f\big(|\hat{\bm{S}}_{:,f}| \big)
    \label{eq:S_hat}
\end{align}
\noindent
Here $P^{QSM}_f$ denotes the transitional probability of QSM at frequency $f$, $P\big(|\hat{\bm{S}}_{:,f}| \big| |\bm{M}|\big)$ represents the estimated magnitude output of the LSTM layers of the SE decoder, and $\mu$ is a hyper-parameter that is tuned to maximize the performance on a development set. Note that we train our QSM in advance on a clean speech corpus and use it in inference mode during enhancement. The tunable parameter $\mu$ of (\ref{eq:S_hat}) is set to zero when we do not have a trained QSM. 





\section{Experiments and Discussion}
\subsection{Datasets}
We assessed Prompt-MIL using three histopathological WSI datasets: TCGA-BRCA \cite{tcga_brca}, TCGA-CRC~\cite{cancer2012comprehensive}, and BRIGHT~\cite{bright}.
These datasets were utilized for both the self-supervised feature extractor pretraining and the end-to-end fine-tuning (with or without prompts), including the MIL component. 
Note that the testing data were not used in the SSL pretraining. \datasection{TCGA-BRCA} contains 1034 diagnostic digital slides of two breast cancer subtypes: invasive ductal carcinoma (IDC) and invasive lobular carcinoma (ILC). 
We used the same training, validation, and test split as that in the first fold cross validation in~\cite{chen2022scaling_hipt}. 
The cropped patches (790K training, 90K test) were extracted at 5$\times$  magnification. 
\datasection{TCGA-CRC} contains 430 diagnostic digital slides of colorectal cancer for a binary classification task: chromosomal instability (CIN) or genome stable (GS). 
Following the common 4-fold data split~\cite{bilal2021development,liu2018comparative}, we used the first three folds for training (236 GS, 89 CIN), and the fourth for testing (77 GS, 28 CIN). 
We further split 20\% (65 slides) training data as a validation set. The cropped patches (1.07M training, 370K test) were extracted at 10$\times$  magnification. 
\datasection{BRIGHT} contains 503 diagnostic slides of breast tissues. 
We used the official training (423 WSIs) and test (80 WSIs) splits. 
The task involves classifying non-cancerous (196 training, 25 test) vs. pre-cancerous (66 training, 23 test) vs. cancerous (161 training, 32 test). 
We further used 20\% (85 slides) training slides for validation. 
The cropped patches (1.24M training, 195K test) were extracted at 10$\times$ magnification. 

\begin{table}[t]
\caption{Comparison of accuracy and AUROC on three datasets. Reported metrics (in $\%$age) are the average across 3 runs. "Num. of Parameters" represents the number of optimized parameters}
\label{table:result:accuracy}
\begin{center}
\setlength{\tabcolsep}{0.9mm}{

\begin{tabular}{l |c c c c c c |c}
\toprule
\multicolumn{1}{c|}{Dataset} 
    & \multicolumn{2}{c}{TCGA-BRCA} 
        & \multicolumn{2}{c}{TCGA-CRC}
            & \multicolumn{2}{c}{BRIGHT}
    & \multicolumn{1}{|c}{Num. of}
\\
% \midrule
\multicolumn{1}{c|}{Metric} 
    & \multicolumn{1}{c}{Accuracy} & \multicolumn{1}{c}{AUROC} 
        & \multicolumn{1}{c}{Accuracy} & \multicolumn{1}{c}{AUROC}
            & \multicolumn{1}{c}{Accuracy} & \multicolumn{1}{c}{AUROC}
    & \multicolumn{1}{|c}{Parameters} 
\\
\midrule
Conventional MIL 
    & $92.10$          & $96.65$       
        & $73.02$         & $69.24$
            & $62.08$         & $80.96$
    & 70k
\\
Full fine-tuning 
    & $88.14$ & $93.78$
        & $74.53$ & $56.63$
            & $56.13$ & $75.87$
    & 5.6M
\\ 
Prompt-MIL (ours) 
    & $\bm{93.47}$ & $\bm{96.89}$
        & $\bm{75.47}$  & $\bm{75.45}$ 
            & $\bm{64.58}$  & $\bm{81.31}$ 
            
    & 70k+192
\\
\bottomrule
\end{tabular}
}
\end{center}
\end{table}


\subsection{Implementation Details}
We cropped non-overlapping 224 $\times$ 224 sized patches in all our experiments and used ViT-Tiny (ViT-T/16)~\cite{vit} for feature extraction.
For SSL pretraining, we leveraged the DINO framework~\cite{dino} with the default hyperparameters, but adjusted the batch size to 256 and employed the global average pooling for token aggregation. 
We pretrained separate ViT models on the TCGA-CRC datasets for 50 epochs, on the BRIGHT dataset for 50 epochs, and on the BRCA dataset for 30 epochs. 
For TCGA-BRCA, we used the AdamW~\cite{loshchilov2017adamW} optimizer with a learning rate of $1e-4$, $1e-2$ weight decay, and trained for 40 epochs.
For TCGA-CRC, we also used the AdamW optimizer with a learning rate of $5e-4$ and trained for 40 epochs.
For Bright, we used the Adam~\cite{adam} optimizer with a learning rate of $1e-4$, $5e-2$ weight decay and trained for 40 epochs.
We applied a cosine annealing learning rate decay policy in all our experiments.
For the MIL baselines, we employed the same hyperparameters as above.
For all full fine-tuning experiments, we used the learning rate in the corresponding prompt experiment as the base learning rate. For parameters in the feature model $F(\cdot)$, which are SSL pretrained, we use 1/10 of the base learning rate. For parameters in the Classifier $G(\cdot)$, which are randomly initialized, we use the base learning rate. We train the full tuning model for 10 more epochs than our prompt training to allow full convergence. This training strategy is optimized using the validation datasets.
All model implementations were in PyTorch~\cite{paszke2019pytorch} on a NVIDIA Tesla V100 or a Nvidia Quadro RTX 8000.

\subsection{Results}
% We tested the effectiveness of our proposed method on the three downstream datasets comprising prostate, colon, and breast cancer. 
% \KM{may add some details about the methods you compared to here if space allows.}
We chose overall accuracy and Area Under Receiver Operating Characteristic curve (AUROC) as the evaluation metrics. 


% \noindent\textbf{Evaluation of prompt tuning performance:} \label{sec:result_prompt}
\resultsection{Evaluation of prompt tuning performance:} \label{sec:result_prompt}
We compared the proposed Prompt-MIL with two baselines: 1) a conventional MIL model with a frozen feature extractor~\cite{li2021dual_dsmil}, 2) fine-tuning all parameters in the feature model (full fine-tuning).
Table~\ref{table:result:accuracy} highlights that our Prompt-MIL consistently outperformed both.
%the conventional MIL method and the full fine-tuning method. 
%added another 192 para, which is less than 0.3\% of the total parameters of the conventional MIL competitor.
% With such a negligible parameter overhead, our Prompt-MIL 
Compared to the conventional MIL method, Prompt-MIL added negligible parameters (192, less than 0.3\% of the total parameters), 
achieving a relative improvement of 1.49\% in accuracy and 0.25\% in AUROC on TCGA-BRCA, 3.36\% in accuracy and 8.97\% in AUROC on TCGA-CRC, and 4.03\% in accuracy and 0.43\% in AUROC on BRIGHT.
The observed improvement can be attributed to a more optimal alignment between the feature representation learned during the SSL pretraining and the downstream task, i.e., the prompt explicitly calibrated the features toward the downstream task.  


The computationally intensive full fine-tuning method under-performed conventional MIL and Prompt-MIL. 
Compared to the full fine-tuning method, our method achieved a relative improvement of 1.29\% to 13.61\% in accuracy and 3.22\% to 27.18\% in AUROC on the three datasets.
Due to the relatively small amount of slide-level labels (few hundred to a few thousands) fully fine tuning 5M parameters in the feature model might suffer from overfitting. 
In contrast, our method contained less than 1.3\% of parameters compared to full fine-tuning, leading to robust training.


\begin{table}[t]
\caption{Comparison of GPU memory consumption and training speed per slide benchmarked on the BRIGHT dataset between the full fine-tuning and our prompt tuning on four slides with different sizes. Our prompt method requires far less memory and is significantly faster.}  
\begin{center}
\setlength{\tabcolsep}{1.6mm}{
\begin{tabular}{clcccc}
\toprule
\multicolumn{2}{c}{WSI size}  
    & $44k\times21k$ & $26k\times21k$ 
        & $22k\times17k$ & $11k\times16k$ \\
\multicolumn{2}{c}{\#Tissue patches} 
    & 9212               & 4765               
        & 2307               & 1108               \\ 
\midrule
GPU        & Full fine-tuning 
    & 21.81G             & 18.22G             
        & 16.37G             & 12.71G             \\
Mem.     & Prompt (ours)    
    & $\bm{12.04}$G             & $\bm{10.66}$G             
        & $\bm{10.00}$G              & $\bm{7.90}$G              \\ 
\cmidrule{2-6}
        & Reduction percentage 
    & 44.79\%  & 41.50\% & 38.92\% & 37.84\% \\
\midrule
Time      & Full fine-tuning 
    & 17.73s             & 8.92s              
        & 4.37s              & 2.15s              \\
per slide  & Prompt (ours)    
    & $\bm{13.92}$s             & $\bm{7.09}$s              
        & $\bm{3.35}$s              & $\bm{1.56}$s              \\
\cmidrule{2-6}
& Reduction percentage 
    & 21.49\%  & 20.51\% & 23.32\% & 27.27\% \\
\bottomrule
\end{tabular}
}
\end{center}
\label{table:result:gpu}
\end{table}

\begin{table}[b]
\caption{Comparison of accuracy and AUROC on three datasets for a pathological foundation model.
}
\label{table:result:universal_models}
\begin{center}
% \setlength{\tabcolsep}{1.6mm}{

\begin{tabular}{l c c c c }
\toprule
\multicolumn{1}{c}{Dataset} & \multicolumn{2}{c}{TCGA-BRCA} & \multicolumn{2}{c}{BRIGHT} \\
% \midrule
\multicolumn{1}{c}{Metric} & \multicolumn{1}{c}{Accuracy} & \multicolumn{1}{c}{AUROC} & \multicolumn{1}{c}{Accuracy} & \multicolumn{1}{c}{AUROC}  \\
\midrule
ViT-small~\cite{wang2022transformer}
    & $91.75$          & $97.03$       
        & $54.17$  & $76.76$ \\
ViT-small w/ Prompt-MIL
    & $\bm{92.78} $ & $\bm{97.53}$
        & $\bm{57.50}$  & $\bm{78.29}$ \\
\bottomrule
\end{tabular}
% }
\end{center}
\end{table}


\resultsection{Evaluation of time and GPU memory efficiency:} Prompt-MIL is an efficient method requiring less GPU memory to train and running much faster than full fine-tuning methods. 
We evaluated the training speed and memory consumption of our method and compared to the full fine-tuning baseline on four different sized WSIs in the BRIGHT dataset.
% As sizes and areas of tissue regions varies among slides in a dataset, instead of evaluating the memory consumption and training speed on three datasets, 
% We evaluated it on four different sized WSIs in the BRIGHT dataset.
As shown in Table~\ref{table:result:gpu}, our method consumed around 38\% to 45\% less GPU memory compared to full fine-tuning and was 21\% to 27\% faster. 
As we scaled up the WSI size (i.e. WSIs with more number of patches), the memory cost difference between Prompt-MIL and full fine-tuning further widened. 
%Therefore our method is particularly even more crucial for processing large WSIs or for processing at higher magnification (eg. 20X, 40X) which contains much more number of patches per WSI. 


\resultsection{Evaluation on the pathological foundation models:} 
% The effectiveness of our prompt-MIL in augmenting the performance of foundational pathology models has been also evaluated. 
We demonstrated our Prompt-MIL also had a better performance when used with the pathological foundation model.
Foundational models refer to those trained on large-scale pathology datasets (e.g. the entire TCGA Pan-cancer dataset~\cite{weinstein2013cancer}). 
We utilized the publicly available~\cite{wang2022transformer,transpath} ViT-Small network pretrained using MoCo v3~\cite{mocov3} on all the slides from TCGA~\cite{weinstein2013cancer} and PAIP~\cite{paip}. 
In Table~\ref{table:result:universal_models}, we showed that our method robustly boosted the performance on both TCGA (the same domain as the foundation model trained on) and BRIGHT (a different domain). 
The improvement is more prominent in BRIGHT, which further confirmed that Prompt-MIL aligns the feature extractor to be more task-specific.

\begin{table}[h]
\caption{Performance with a different number of prompt tokens. For two different WSI classification tasks, one token was enough to boost the performance of the conventional MIL schemes.}
\label{table:result:abalation}
\begin{center}
\setlength{\tabcolsep}{1.6mm}{

\begin{tabular}{ccccc}
\toprule
\multicolumn{1}{c}{Dataset} 
    & \multicolumn{2}{c}{TCGA-BRCA}        
        & \multicolumn{2}{c}{BRIGHT}           \\
\#prompt tokens $k$             
    & \multicolumn{1}{l}{Accuracy} & AUROC 
        & \multicolumn{1}{l}{Accuracy} & AUROC \\ 
\midrule
$k=1$                           
    & $\bm{93.47}$  & $\bm{96.89}$ 
        & $\bm{64.58}$  & $\bm{81.31}$ \\
$k=2$                       
    & 93.13   & $\bm{96.93}$ 
        & 60.41 & 79.74 \\
$k=3$                           
    &  $\bm{93.47}$  &  $96.86$     
        &  59.17   &  76.75     \\ 
\bottomrule
\end{tabular}
}
\end{center}
\end{table}

\resultsection{Ablation study:}
An ablation was performed to study the effect of the number of trainable prompt tokens on downstream tasks. 
%We used the same pretrained feature models as that in section~\ref{sec:result_prompt}.
Table~\ref{table:result:abalation} shows the accuracy and AUROC of our Prompt-MIL model with 1, 2 and 3 trainable prompt tokens ($k=1, 2, 3$) on the TCGA-BRCA and the BRIGHT datasets.
On the TCGA-BRCA dataset, our Prompt-MIL model with 1 to 3 prompt tokens reported similar performance.
On the BRIGHT dataset, the performance of our model dropped with the increased number of prompt tokens. 
% This drop is consistant with
% Such a performance drop is consistent with the large performance drop of our method and full fine-tuning method in Table~\ref{table:result:accuracy}.
Empirically, this ablation study shows that for classification tasks, one prompt token is sufficient to boost the performance of conventional MIL methods.
% It is probably because the number of training samples in the BRIGHT dataset is only around half of that in the TCGA-BRCA dataset, and thus the increased trainable tokens may cause over-fitting.
% Overall, this ablation study showed that for classification tasks, one prompt token is sufficient to boost the performance of conventional MIL methods.




\section{Conclusion}
In this work, we introduced a new framework, Prompt-MIL, which combines the use of Multiple Instance Learning (MIL) with prompts to improve the performance of WSI classification.  Prompt-MIL adopts a prompt tuning mechanism rather than a conventional full fine-tuning of the entire feature representation. In such a scheme, only a small fraction of parameters calibrates the pretrained representations to encode task-specific information, so the entire training can be performed in an end-to-end manner. 
We applied our proposed method to three publicly available datasets.
Extensive experiments demonstrated the superiority of Prompt-MIL over the conventional MIL as well as the conventional fully fine-tuning methods.
Moreover, by fine-tuning much fewer parameters compared to fully fine-tuning, our method is GPU memory efficient and fast.
Our proposed approach also showed promising potentials in transferring foundation models. 
We will further explore the task-specific features that are captured by our prompt toward explainability of these models.

% \subsubsection{Acknowledgements} Please place your acknowledgments at
% the end of the paper, preceded by an unnumbered run-in heading (i.e.
% 3rd-level heading).


%
% ---- Bibliography ----
%
% BibTeX users should specify bibliography style 'splncs04'.
% References will then be sorted and formatted in the correct style.
%
\bibliographystyle{splncs04}
\bibliography{bibliography}
%
\end{document}
