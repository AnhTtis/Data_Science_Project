\section{Optimization of Deployments on GPU}
\label{sec:alg}
% The overall architecture of the proposed high performance accelerator is shown in Fig xxx. It is accelerated on NVIDIA GPU and hosted by an ARM CPU processor. 
% Backbone is implemented to regress linear combination coefficients.
% 8155 Conv is adopted to assemble pixel-adaptive filters.
% Einstein is invented to apply adaptive filters to the upsampled image.

\subsection{Dictionary Compression}
\label{sec:pruning}

% The SR framework we chose from \cite{SR-2020NIPS-LAPAR} already shows the lightweight model which performs the state-of-art 
% results in practice with a limited size of parameters($<$1M) and retrained number of MultiAdds operations. 
The state-of-the-art SR model LAPAR \cite{SR-2020NIPS-LAPAR} shows outstanding performance with a limited size of parameters ($<1$M). 
% However, to fulfill a real-time video super-resolution requirement, we pick up the model sliming as the very first step of our speed enhancement framework. 
% To fulfill the real-time video super-resolution requirement, running-time profiling and model sliming are adopted in our framework to compress the model. 
However, even using the state-of-the-art NVIDIA TensorRT \cite{GPU-NVIDIA-TensorRT}, the inference performance cannot meet the demanding real-time requirement. 
The reason is that the high-resolution feature maps play the key role instead of the parameters in the SR models, as mentioned above in \Cref{sec:sr-algorithm}. 
The running time profiling is shown in \Cref{fig:time-breakdown}. 
The time distribution demonstrates that dictionary learning consumes the most time and is obviously the bottleneck. 
Such analysis result comes from the fact that the existing tools and methods, \textit{e.g.}, TensorRT, as effective in acceleration for normal DNN kernels such as convolution, ReLU. 
For certain layers in the computation graph with less customized efficient implementation from TensorRT, they will cost inevitably a large amount of time. 

% \begin{figure}[tb!]
%     \centering
%     \includegraphics[width=.66\linewidth]{profiling} 
%     \caption{Time profiling (in milliseconds) on different parts of LAPAR SR inference process. Model is deployed on NVIDIA GTX 2080ti. Time cost breaks into 3 
%     separate components: (1) common kernels such as Convolution and ReLU; (2) data reformatting, concatenation or other operations; (3) dictionary query and filtering step}
%     \label{fig:time-breakdown}
% \end{figure}
% We observe that the state-of-the-art SR model LAPAR has all kinds of layers. 
% We use the roofline model to implement a layer by layer profiling in terms of both computation and data transfer. 
% Each layer in the computation graphs has different shapes of tensors, which can cause dissimilar computation and communication behaviors.
% As shown in figure \Cref{fig:time-breakdown}, it is easy to find out the dictionary learning occupies the major part of the inference time. 

% We selected the most $\vec{\alpha}$ representative layers where $\vec{\alpha} \in (0, 1)$ and ignore the redundant ones with a selective strategy. 
% We reduce $\alpha$ to accelerate the dictionary query step repeatedly until a precision threshold was met. 

Compressing the dictionary can ease both the computation and communication workloads, as has been analyzed in \Cref{sec:sr-algorithm}. 
Consequently, a structural dictionary selection strategy is proposed in this paper to compress the dictionary. 
On the one hand, an ideal dictionary $\vec{D}$ can provide sufficient information for the restoration of image details. 
On the other hand, the dictionary $\vec{D}$ is expected to be less bulky to perform an inference within the required time limit, without degradation to the quality of results. 
A threshold value $\alpha \in (0, 1)$ is set to specify the sparsity of the dictionary $\vec{D} \in \mathbb{R}^{L\times k^2}$, 
\textit{i.e.}, after compression, only the most representative $\alpha \cdot L$ items from $L$ will be reserved.  
To avoid the greedy compression which would fall into the local optimum, the dictionary items are compressed iteratively until the sparsity threshold $\alpha$ is reached. 
In step $t$, the compression ratio is set to be $\alpha_t$, with $\alpha_t < \alpha_{t-1}$. The most representative $\alpha_t L$ items are reserved while others are discarded. 
In the next step, we set $\alpha_{t+1}=\alpha_t -\Delta_{\alpha}$, to further prune more items. 
Meanwhile, parameters $\vec{W}$ of the \textit{LaparNet} is fine-tuned accordingly to minimize the reconstruction error \cite{SPEED-he2019filter-He,SPEED-ICCV2017-He,SPEED-huang2018data-Huang}. 
The problem can be formulated as:
% \todo{as indicated by \cite{SPEED-ICCV2017-He}}:
\begin{equation}
    \label{eq:prune}
    \begin{array}{rl}
        \vec{\beta}, \vec{W} ={}& \mathop{\arg\min}_{\vec{\beta}, \vec{W}} \frac{1}{N} \left\|\vec{Y} - \sum_{i=0}^{L} \beta_{i}\vec{\Phi} \vec{D}\right\|_{2}^{2}, \\[2\jot]
        \mathrm{s.t.} \ & \vec{\Phi} = \textit{LaparNet}(\vec{X}, \vec{W}), \\[2\jot]
        & \| \vec{\beta}\|_{0} \leq \alpha L,
    \end{array}
\end{equation}
where $N$ is the size of input batch of images, $\vec{\Phi}$ is the coefficient vector extracted from \textit{LaparNet} with parameters $\vec{W}$ and $\vec{Y}$ is the output matrix of this layer after dictionary query. 
$\vec{\beta}$ is the selecting vector on filters of $\vec{D}$ where $\beta_{i} = 0 $ means the $i$-th item in the dictionary will be ignored during the compression process. 
% within the selected Dictionary $\vec{\beta}\vec{D}$'s i-th item on channel-dimension will be ignored for slimming purpose.
% We can make modifications to the problem as the $\vec{Y}$ output is not the final result of the SR flow. 

\begin{figure}[tb!]
    \centering
    \includegraphics[width=0.9\linewidth]{dictionary-compression_2/dictionary-compression} 
    \caption{Visual illustration of dictionary compression, the upper flow represents original dictionary query and filtering, namely stage 3 + stage 4 in \Cref{fig:LAPAR-architecture}, 
    The flow below demonstrates the compression process of the dictionary query. }
    \label{fig:prune}
\end{figure}

Further, we improve \Cref{eq:prune} by considering the loss of the final results of the SR flow. 
In other words, the filtering stage after the dictionary query stage is considered, to guarantee the quality of results after compression. 
% Within the minimization objective in \Cref{eq:prune}, the dictionary filtering step is not involved for dictionary item selection. 
% We can simply insert this step to the loss function by computing the difference between ground truth high-resolution image $\vec{H}_{gt}$ and re-write the objective function: 
The reconstruction error $\mathscr{L}$ of the dictionary compression is defined as the difference between the ground truth high-resolution image $\vec{H}_{gt}$ and the images generated by the compressed model, as shown in \Cref{eq:prune_2}. 
\begin{equation}
    \label{eq:prune_2}
    \begin{array}{rl}
        % \mathscr{L} ={}& \frac{1}{N} \left\|\vec{H}_{gt} - \vec{F}_{W,\beta}\vec{B}^{\top}\right\|_{2}^{2}, \\ [2\jot]
        \vec{\beta}, \vec{W} ={}& \mathop{\arg\min}_{\vec{\beta}, \vec{W}} \frac{1}{N} \left\|\vec{H}_{gt} - \vec{F}_{W,\beta}\vec{B}^{\top}\right\|_{2}^{2}, \\ [2\jot]
        \mathrm{s.t.} \ \ & \vec{F}_{W,\beta} = \sum_{i=0}^{L} \beta_{i}\vec{\Phi} \vec{D}, \\ [2\jot]
        & \vec{\Phi} = \textit{LaparNet}(\vec{X}, \vec{W}), \\ [2\jot]
        & \| \vec{\beta}\|_{0} \leq \alpha L.
    \end{array}
\end{equation}
With two objectives $\vec{W}$ and $\vec{\beta}$, to solve this optimization problem efficiently, an alternating method including two steps is adopted. 
The first step is to search the suitable selecting vector $\beta$ corresponding to the required $\alpha_t$. 
The second step is to tune the parameters $\vec{W}$ corresponding to the reserved dictionary items with the minimization objective in \Cref{eq:prune_2}. 
The first step is actually an NP-hard problem. 
\cite{SPEED-ICCV2017-He} suggested to relax the problem to $\ell_{1}$ regulation. % $\ell_{0}$
Therefore the objective $\vec{\beta}$ can be solved by utilizing the LASSO regression with parameter $\lambda$, as shown in \Cref{eq:lasso}. 
\begin{equation}
    \label{eq:lasso}
    \begin{array}{rl}
        \vec\beta=&\mathop{\arg\min}_{\vec\beta} \frac{1}{N}{}\left\|\vec{H}_{gt} - \vec{F}_{W,\beta}\vec{B}^{\top}\right\|_{2}^{2} + \lambda\|\vec{\beta}\|_{1}, \\[2\jot]
        & \mathrm{s.t.} \|\vec{\beta}\|_{0} \leq \alpha L. 
    \end{array}
\end{equation}
The complete selection strategy is illustrated in \Cref{alg:Pruning Strategy}. 

\begin{algorithm}
    \small
    \caption{Dictionary Selection Strategy}
    \label{alg:Pruning Strategy}
    \begin{algorithmic}[1]
    \State {\textbf{Input:} $\vec{D} \in \mathbb{R}^{L\times k^2}$, small $\lambda_{0}$, target $\alpha$, tolerance $\epsilon$}; 
    \State {\textbf{Input:} pre-trained $\vec{W}_{0}$ , coefficient matrix $\vec{\Phi}$};
    % \State {\textbf{Input:} loss function $\mathscr{L}$ (reconstruction error in \Cref{eq:prune_2})};
    \State $t\leftarrow 0$, $\alpha_{0} \leftarrow 1.0$, $\beta_{0}\leftarrow\vec{1} \in \mathbb{R}^{L}$, $\gamma_{0}\leftarrow\vec{1} \in \mathbb{R}^{L}$;
    \State {$\mathscr{L} \leftarrow$ reconstruction error \Comment{\Cref{eq:prune_2}}}
    \Repeat 
    \State {$\alpha_{t+1} \leftarrow\alpha_{t} - \Delta\alpha$; }
    \State {$\lambda_{t+1} \leftarrow \lambda_{t}$;}
        \While {$|\beta_{t+1}|_{0} \textgreater \alpha_{t+1} \cdot L$}
        \State {Fix $\vec{W}_{t}$, update $\beta_{t+1} \leftarrow \mathop{\arg\min}_{\vec{\beta}} \mathscr{L}(\vec{W}_{t}, \vec{\beta}\vec{D})$ \Statex{$ + \lambda_{t+1}\left|\vec{\beta}\right|$; \Comment{\Cref{eq:lasso}}} }
        \State {$\lambda_{t+1} \leftarrow 2 \cdot \lambda_{t+1}$}
        \EndWhile
        \State {$\lambda_{left} \leftarrow 0.5\lambda_{t+1}$, $\lambda_{right} \leftarrow \lambda_{t+1}$; }
        \While{$\left|\alpha_{t+1} \cdot L - \left|\beta_{t+1}\right|_{0}\right| \textgreater \epsilon \cdot L$}
        \State {$\lambda_{t+1} = 1/2(\lambda_{left} + \lambda_{right})$; }
        \State {Fix $\vec{W}_{t}$, update $\beta_{t+1} \leftarrow \mathop{\arg\min}_{\vec{\beta}} \mathscr{L}(\vec{W}_{t}, \vec{\beta}\vec{D})$ \Statex{$ + \lambda_{t+1}\left|\vec{\beta}\right|$;}}
        \If{$\left|\beta_{t+1}\right|_{0} \textless \alpha_{t+1} \cdot L$}
        \State {$\lambda_{left} \leftarrow \lambda_{t+1}$;}
        \ElsIf {$\left|\beta_{t+1}\right|_{0} \textgreater \alpha_{t+1} \cdot L$}
        \State {$\lambda_{right} \leftarrow \lambda_{t+1}$;}
        \EndIf
        \EndWhile
    \State {Fix $\beta_{t+1}$, update $\vec{W}_{t+1} \leftarrow \mathop{\arg\min}_{\vec{W}} \mathscr{L}(\vec{W}, \beta_{t+1}\vec{D}) $; } \Statex{\Comment{\Cref{eq:weight}}}
    % \State {$Precision_{t+1} \leftarrow \vec{\Phi} (\vec{W}, Batch_{test}, \beta_{t+1}\vec{D})$}
    \State {$t = t + 1$; }
    \Until {$\alpha_{t} \leq \alpha$ }
    \end{algorithmic}
\end{algorithm}

For channel selecting, we accumulated a batch of input feature map from the \textit{LaparNet} prior to the dictionary query step as well as the ground-truth high-resolution images. 
The LASSO regression feature will be randomly sampled from the input feature map within width each plane. 
The $\lambda$ in \Cref{eq:lasso} is carefully set to adjust the pruning ratio for filters. 
We begin with a tiny $\lambda$ and increase the value exponentially until the reduced filter fits the required number. 
% Then a binary search with the range of each step will be put into use on the pruning numbers in case the step size for $\lambda$ is too large.
In lines 12--20 in \Cref{alg:Pruning Strategy}, a binary search is applied on $\lambda_{t+1}$ within the range of last step size to adjust the compression ratio close to $\alpha_{t+1}$.
% In line.6 of Algo.\ref{alg:Pruning Strategy}, to tune the parameters of the whole model to fit the new dictionary at each step is time-consuming. As shown in \Cref{eq:weight}, The 
% $\vec{D^\prime}$ is the Dictionary which is the compressed dictionary with layers neglected in the previous LASSO step. 

The second step is to update the parameters in line 21 of \Cref{alg:Pruning Strategy}. The problem is formulated into:
\begin{equation}
    \label{eq:weight}
    \vec{W} = \mathop{\arg\min}_{\vec{W}} \frac{1}{N} \left\|\vec{H}_{gt} - \vec{F}_{W,D^\prime}\vec{B}^{\top}\right\|_{2}^{2}. 
\end{equation}
To fine-tune the parameters of the whole model to fit the new dictionary through training at each step is time-consuming. As shown in \Cref{eq:weight}, The 
$\vec{D}^\prime$ is the Dictionary which is the compressed dictionary with layers neglected in the previous LASSO step. 
We can use linear regression in the iterative steps to reconstruct the parameters of the dictionary query layer which generates coefficient vector $\vec{\Phi}$ for 
fast tuning requirement. 
We note parameters of this layer as $\vec{W}_{D^\prime}$ and parameters at $i$-th channel will be adjusted by the regression coefficient $\vec{\gamma}$. The iterative tuning step 
of \Cref{eq:weight} will be re-written into \Cref{eq:weight_2} where $\vec{W}_{D^\prime}^{new}$ is the updated parameters. Note that $\vec{\gamma}$ is actually a weight coefficient on $\vec{W}_{D^\prime}$ along channel dimension for update.
The weighted coefficient matrix $\mathbf{\Phi}^{\prime}$ is the new query to the selected dictionary $\vec{D}^\prime$.
The whole modified dictionary query and filtering flow are illustrated in \Cref{fig:prune}.
\begin{equation}
    \label{eq:weight_2}
    \begin{array}{rl}
         \vec{\gamma} = \mathop{\arg\min}_{\vec\gamma} {}& \frac{1}{N} \left\|\vec{H}_{gt} - \sum_{i=0}^{L} \gamma_{i}\vec{F}_{W,{D}^\prime}\vec{B}^{\top}\right\|_{2}^{2}, \\[2\jot]
          &\vec{W}_{D^\prime}^{new} =  \vec{\gamma}\vec{W}_{D^\prime}. 
    \end{array}
\end{equation}

As shown in \Cref{fig:prune_ratio_performance}, the pruning process sustains the SR performance with barely no accuracy degradation. 
Essentially the well trained backbone network is capable of extracting sparse information for dictionary so the zero-out layers of the dictionary will not incur information loss. 
We can shrink to dictionary to size of 10\% without noticeable accuracy loss. 
Compared with other widely-used SR models, \textit{e.g.}, \cite{SR-eccv2016-frcnn, SR-2016cvpr-vsdr} , our performance is the optimal. 

\begin{figure}[tb!]
    \centering
    \hspace{-.2in}
    \subfloat{
        \includegraphics[height=0.28\linewidth]{psnr-ssim/psnr}  
    }
    \hspace{-.1in}
    \subfloat{
        \includegraphics[height=0.28\linewidth]{psnr-ssim/ssim} 
    }
    \hspace{-.2in}
    \caption{Single image super-resolution (SISR) performance of our model with different dictionary compression ratios, in comparison with other SR methods. 
    LAPAR-A (Per.\%) represents our model with dictionary size shrunk to Per.\%. PSNR means peak signal-to-noise ratio. SSIM means structural similarity index measure. 
    PSNR and SSIM are two common metrics to measure the quality of images. The higher the better. }
    \label{fig:prune_ratio_performance}
\end{figure}

\subsection{Constraint-Based Optimization of Deployments On GPU}
\label{sec:cuda}
After shrinking the volume of the dictionary, we successfully slim the volume of the model. 
Another bottleneck for accelerating the computations is the filtering operation following the dictionary query, which can be regarded as a Hadamard product of matrices followed by a reducesum operation along channel dimension. 
Such kind of computations are common in SR tasks but ignored by the existing deployment tools. % normally arise at the very end of some SR flows. 
In this section, we take advantage of the parallelizing mechanism of GPU to improve the computation throughput from a low-level perspective.
An example of the proposed computation engine is shown in \Cref{fig:engine}. 

\begin{figure}[tb!]
    \centering
    \includegraphics[width=\linewidth]{engine} 
    \caption{An example of the proposed computation engine for image filtering operation.}
    \label{fig:engine}
\end{figure}

The data (either images or filters) are stored in the NCHW format in linear memory addresses continuously, as shown in \Cref{fig:engine}. 
The data with the same color are assigned to the same block to conduct the computations, \textit{e.g.}, the data in purple are assigned to block 0, and the data in orange are assigned to block 1. 
The data with the same index but from various channels are assigned to the same thread. 
For example, the data at location $(1, 1)$ from these channels are all assigned to thread 0 in block 0, while the data at $(1, 2)$ are assigned to thread 1 in block 0. 
Two values with the same location index from $\vec{F}$ and $\vec{B}$ are firstly multiplied. 
Then, the products of the data with the same location index but from various channels are summed up as the final result. 
In other words, each thread in \Cref{fig:engine} computes the product of two vectors, where each vector contains the data from all of the channels. 
Note that in our implementation, each thread directly adds each intermediate product to the final result. 
With this implementation, all of the threads can execute the same code segment and achieve parallel reduction perfectly without falling into the paradox of thread divergence \cite{CUDA-C-Programming}. 
Moreover, frequent cache interaction with main memory is avoided with our design. Blocks within an SM share same shared memory/L1 cache as shown in \Cref{fig:gpu-architecture}.
Thus cache miss rate is reduced since data assigned to consecutive blocks in each channel is carefully stored consequently on memory as illustrated in \Cref{fig:engine}.

The resources in GPU are limited, which concurrently restricts the computation patterns with respect to the threads, blocks, and \textit{etc}. 
Different GPUs have distinct compute capabilities. For clarity, the edge embedded GPU NVIDIA Jetson Xavier NX which uses the Volta microarchitecture is taken as an example to illustrate this. 
From the perspective of hardware architectures, there are 6 streaming multiprocessors (SMs) in it. 
Each SM occupies a 96 KB shared memory/L1 cache. Each SM is further partitioned into 4 processing blocks. 
Every processing block has 16 FP32 cores, 8 FP64 cores, 16 INT32 cores, 2 Tensor cores, and a 64 KB shared register file.
Besides, each processing block has a warp scheduler, to schedule the threads assigned to this processing block. 
From the perspective of the programming model, the computation kernel is executed as a grid of thread blocks. 
Each thread block (different from the processing block mentioned above) is assigned to a single streaming multiprocessor. 
Once the block is scheduled to an SM, threads in this block are further partitioned into warps. 
Every warp consists of 32 consecutive threads and all threads in a warp are executed in SIMT fashion. 
While the warps within a thread block may be scheduled in any order, the number of active warps is limited by SM resources. 
Four processing blocks in every SM of NX means there are at most four active warps in executing at the same moment. 
Besides, the number of warps in a thread block is constrained by the programming model to fit the sizes of warp schedulers, instruction registers, and \textit{etc}. 
Sharing data in the shared register files among the parallel threads in the same processing block, or sharing data among the processing blocks in the same SM 
may cause a race condition: multiple threads accessing the same data in the memory simultaneously. 
Once a warp idles for the race conditions, the SM is free to schedule other available warps. 
The number of warps for a thread block can be determined as follows:
\begin{equation}
    \textrm{Warps Per Block} = \left \lceil \frac{\textrm{Threads Per Block}}{\textrm{Warp Size}} \right \rceil,
\end{equation}
where $\textrm{Warp Size}=32$ for mainstream NVIDIA GPUs. 
% From the hardware perspective, the high-dimensional thread block is a 1-dimensional collection of warps. 
% For example, a 2-dimensional block with 36 threads in the $x$ dimension and 2 threads in the $y$ dimension will be groups as 3 warps, while the last warp has only 12 threads. 
% Although the left 20 threads in the third warp are unused, the resources are still consumed. 

Suitable choices of blocks usually strike a good balance between the parallelism and resource race conditions and therefore stimulating the computations. 
In summary, the sizes of the blocks are constrained by the sizes of input data and the available on-board resources. 
Meanwhile, to accelerate the computations as much as possible, once the resources are available, the tasks will be assigned to occupy the resources. 
In other words, the parallelism is maximized so as to reach the upper-bound value of the resource utilization. 
Assume that the size of the three-dimensional input data is $D=H \times W \times C$, corresponding to the three dimensions of the thread blocks. 
Denote the number of SMs in GPU as $S$, the number of processing blocks in each SM as $P$, the size of register file in each processing block as $R$, the maximum number of threads in each warp as $WS$. 
The GPU compute capability constrains the number of warps in each block as smaller than $T_{sm}$. 
Denote the three dimensions of the thread block as $(nx, ny, nz)$. 
Therefore, we have the following equations and constraints: 
\begin{equation}
    \begin{aligned}
        & T_r = (H \times W \times C) / (S \times P \times R), \\
        & T \leq \min(T_r, T_{sm}), \\
        & nx \times ny \times nz \leq WS \times P \times T, \\
    \end{aligned}
    \label{eq:constraints}
\end{equation}
where $T_r$ is computed by distributing data evenly to different SMs.  
The computational resources are implicitly organized and scheduled by the processing blocks. 
Therefore constraints on the computational resources are reflected in $T_{sm}$, which is a constant value determined by the compute capability. 
Constrained by $T_r$ and $T_{sm}$, $T$ represents the upper bound of the number of warps assigned to each processing block.
Besides, the sizes are also constrained by the size of input data as follows:
\begin{equation}
    \begin{aligned}
        & 1 \leq nx \leq H, \\ 
        & 1 \leq ny \leq W, \\ 
        & 1 \leq nz \leq C.
    \end{aligned}
\end{equation}
These constraints are usually ignored by designers, which wastes lots of optimization workloads. 
For examples, for $T \in [T_r, T_{sm}]$, these $T$ values are legal while on-board resources are not fully utilized and the system parallelism can be further improved. 
With the above constraints, the feasible domain of the block sizes is shrunken significantly. 
The visualization view of the constraints are shown in \Cref{fig:space}. 
To the best of our knowledge, this is the first to consider these constraints for deployments of DNN models on GPUs, 
compared with previous arts, \textit{e.g.}, \cite{SPEED-OSDI2018-TVM}. 

\begin{figure}[tb!]
    \centering
    \pgfplotsset{
    width =.86\linewidth,
    height=.66\linewidth
}
\begin{tikzpicture}[scale=.88]
    \pgfplotsset{every tick label/.append style={font=\scriptsize}}
    \begin{axis}[grid=major, axis lines=center,view={60}{30}, 
        xmin=0, xmax=60, ymin=0, ymax=60, zmin=0, zmax=80,
        xtick={4,50}, ytick={4,50}, ztick={4,50}, 
        xticklabels={1,H}, yticklabels={1,W}, zticklabels={1,C}, 
        xlabel={$nx$}, ylabel={$ny$}, zlabel={$nz$}, 
        xlabel near ticks
    ]
    \addplot3[mark=none,CUlpurple,thick] coordinates {(0,0,50) (50,0,50) (50,50,50) (0,50,50) (0,0,50)};
    \addplot3[mark=none,CUlpurple,thick] coordinates {(0,0,0) (50,0,0) (50,50,0) (0,50,0) (0,0,0)};
    \addplot3[mark=none,CUlpurple,thick] coordinates {(0,0,0) (0,0,50)};
    \addplot3[mark=none,CUlpurple,thick] coordinates {(50,0,0) (50,0,50)};
    \addplot3[mark=none,CUlpurple,thick] coordinates {(50,50,0) (50,50,50)};
    \addplot3[mark=none,CUlpurple,thick] coordinates {(0,50,0) (0,50,50)};
    \addplot3[mark=*,CUpurple,point meta=explicit symbolic,nodes near coords]
    coordinates {(50,50,50)[$(H, W, C)$]};
    \addplot3+[mesh,scatter,samples=10,domain=4:50,mark=*,shader=faceted] 
        {1024/x/y};
    \end{axis}
\end{tikzpicture}

    \caption{The visualized solution space. The solution points below the dotted points are legal configurations. }
    \label{fig:space}
\end{figure}

\begin{table*}[tb!]
    \centering
    \caption{Inference Time (ms) and Acceleration ratios}
    \label{tab:runtime}
    %\resizebox{1\linewidth}{!}
    {
        \begin{tabular}{c|c|ccccc|ccc}
            \toprule
            \multirow{2}{*}{Input size} & \multirow{2}{*}{Scale} & \multicolumn{5}{c|}{NVIDIA GeForce RTX 2080 Ti} & \multicolumn{3}{c}{NVIDIA  Jetson  Xavier  NX} \\
            \cline{3-10}
            & & PyTorch & TensorRT & Ours & Acc. (PyTorch) & Acc. (TensorRT) & TensorRT & Ours & Acc. (TensorRT) \\
            \midrule
            \multirow{3}{*}{$64\times64$}    &$\times$2  &   6.94    &   1.30    &  1.02 & $\times$680.39\% & $\times$127.45\%   &   12.37 &     9.04   &  $\times$136.84\% \\ 
                                             &$\times$3  &   8.26    &   1.94    &  1.40 & $\times$590.00\% & $\times$138.57\%   &   22.62 &    14.28   &  $\times$158.40\% \\ 
                                             &$\times$4  &   9.86    &   2.79    &  1.88 & $\times$524.46\% & $\times$148.40\%   &   35.83 &    20.54   &  $\times$174.44\% \\ \midrule
            \multirow{3}{*}{$128\times128$}  &$\times$2  &   8.74    &   3.59    &  2.66 & $\times$328.57\% & $\times$134.96\%   &   52.12 &    37.25   &  $\times$139.92\% \\ 
                                             &$\times$3  &  13.04    &   6.19    &  4.16 & $\times$313.46\% & $\times$148.80\%   &   90.33 &    54.26   &  $\times$166.48\% \\ 
                                             &$\times$4  &  18.07    &   9.71    &  6.13 & $\times$294.78\% & $\times$158.40\%   &  144.34 &    81.29   &  $\times$177.56\% \\ \midrule
            \multirow{3}{*}{$180\times320$}  &$\times$2  &  17.12    &  12.40    &  9.25 & $\times$185.08\% & $\times$134.05\%   &  177.57 &   124.12   &  $\times$143.06\% \\ 
                                             &$\times$3  &  30.83    &  21.66    & 14.63 & $\times$210.73\% & $\times$148.05\%   &  325.07 &   200.02   &  $\times$162.52\% \\ 
                                             &$\times$4  &  44.69    &  34.69    & 22.12 & $\times$202.03\% & $\times$156.82\%   &  534.99 &   318.60   &  $\times$167.92\% \\ \midrule
            \multirow{3}{*}{$360\times640$}  &$\times$2  &  67.36    &  50.26    & 37.47 & $\times$179.77\% & $\times$134.13\%   &  748.72 &   530.23   &  $\times$141.21\% \\ 
                                             &$\times$3  & 105.32    &  88.45    & 59.20 & $\times$177.90\% & $\times$149.41\%   & 1466.91 &   973.25   &  $\times$150.72\% \\ 
                                             &$\times$4  & 406.93    & 141.08    & 91.09 & $\times$540.02\% & $\times$154.88\%   &       - &        -   &                -  \\ \midrule
            Average                          & -         &  61.43    &  31.17    & \textbf{20.91} & \textbf{$\times$352.27\%} & \textbf{$\times$144.49\%}   & 328.26  &   \textbf{214.81}   & \textbf{$\times$156.28\%} \\ \bottomrule
        \end{tabular}
        \begin{tablenotes}
            \small
            \item Inference time on NVIDIA Jetson Xavier NX with input size $360\times640$ and scale 4 is not available due to the memory limit of the edge device. 
          \end{tablenotes}
    }
\end{table*}


With the target of minimizing the inference latency, we tend to build a regression model with respect to candidate values of $nx$, $ny$, and $nz$. 
The key challenge is that the clear form is the objective function is unknown because of the invisible execution process of GPU and CUDA programming model. 
Bayesian optimization is adopted in this paper as the searching algorithm to search the optimal configuration of the blocks 
with Gaussian process (GP) model utilized as the surrogate model \cite{ML-2018NeurIPS-GPyTorch}. 
Firstly, several configurations are randomly sampled from the design space to initialize the GP model. 
And then the Bayesian optimization is used to iteratively select new configurations which have higher predictive performance reported by the GP model. 
The GP model is further optimized with newly sampled configurations and their on-chip inference latencies. 
Finally, the best configuration selected by the Bayesian algorithm in this exploration process is our optimal design. 


