\documentclass[12pt]{article}
\usepackage[utf8]{inputenc}
\usepackage{amssymb,amsmath,amsthm,mathtools}
\usepackage{dsfont}  %
\usepackage{graphicx}
\usepackage{enumerate}
\usepackage{natbib}
\usepackage{url}  %
\usepackage{xcolor}
\usepackage[shortcuts]{extdash}
\usepackage{caption,setspace}
\captionsetup{font={stretch=1}}

\def\spacingset#1{\renewcommand{\baselinestretch}%
{#1}\small\normalsize} \spacingset{1}

\addtolength{\oddsidemargin}{-.5in}%
\addtolength{\evensidemargin}{-1in}%
\addtolength{\textwidth}{1in}%
\addtolength{\textheight}{1.7in}%
\addtolength{\topmargin}{-1in}%



\usepackage{xr}  %

\makeatletter
\newcommand*{\addFileDependency}[1]{%
\typeout{(#1)}%
\@addtofilelist{#1}
\IfFileExists{#1}{}{\typeout{No file #1.}}
}\makeatother

\newcommand*{\myexternaldocument}[1]{%
\externaldocument{#1}%
\addFileDependency{#1.tex}%
\addFileDependency{#1.aux}%
}

\myexternaldocument{mainb}


\newcommand{\Pb}{\mathbb{P}}
\newcommand{\Ev}{\mathbb{E}}
\newcommand{\Var}{\mathbb{V}}

\newcommand{\bB}{\boldsymbol{B}}
\newcommand{\bF}{\boldsymbol{F}}
\newcommand{\bP}{\boldsymbol{P}}
\newcommand{\by}{\boldsymbol{y}}
\newcommand{\bY}{\boldsymbol{Y}}
\newcommand{\bu}{\boldsymbol{u}}
\newcommand{\bU}{\boldsymbol{U}}
\newcommand{\bs}{\boldsymbol{s}}
\newcommand{\bI}{\boldsymbol{I}}
\newcommand{\bJ}{\boldsymbol{J}}

\newcommand{\btheta}{\boldsymbol{\theta}}

\newcommand{\diff}{\mathop{}\!\mathrm{d}}
\newcommand{\indFun}[1]{\mathds{1}_{\{#1\}}}


\newcommand{\blind}{1}

\title{\bf Supplementary Material for \\ `Nonparametric Two-Sample Test for Networks Using Joint Graphon Estimation'}
\author{Benjamin Sischka and G\"{o}ran Kauermann
}



\begin{document}

\if1\blind
{
  \maketitle
} \fi


\spacingset{1.9} %



\section{Detecting Differences on the Microscopic Scale}
\subsection{Methodological Construction}
\label{sec:diffMicr}
In addition to the testing aspect, the network alignment based on the joint smooth graphon model also allows for determining structural differences on the edge level. In that regard, we are especially interested in differences that occur when inferring the structure separately. To address this, we first fit two separate graphon models to the two networks individually. To be precise, for this operation, we employ the node positions obtained from estimating the \textit{joint} graphon model and then perform the M-step as described in Section~\ref{sec:spline} of the paper but reduced to the use of only $\by^{(1)}$ or $\by^{(2)}$. This yields the separate estimates $\hat{w}^{(1)}(\cdot,\cdot)$ and $\hat{w}^{(2)}(\cdot,\cdot)$, respectively. Based on that, for $g_1,g_2 = 1, 2$ with $g_1 \neq g_2$, we calculate
\begin{align*}
    \hat{w}_{(g_1)(g_2)}^{\text{diff}}(u,v) = \frac{\hat{w}^{(g_1)}(u,v) - \hat{w}^{(g_2)}(u,v)}{\sqrt{\left\{ \hat{w}^{(1)}(u,v) \, [1 - \hat{w}^{(1)}(u,v)] + \hat{w}^{(2)}(u,v) \, [1 - \hat{w}^{(2)}(u,v)] \right\} / 2}}
\end{align*}
for $(u,v)^\top \in [0,1]^2$. With regard to data-generating process~(\ref{eq:dataGen}), this can be interpreted as the difference between expectations in relation to the averaged standard deviation. Hence, $\vert \hat{w}_{(g_1) (g_2)}^{\text{diff}}(\cdot, \cdot) \vert$ provides an appropriate measure to quantify the local differences between the found graphon structures. Moreover, it can be considered as a smoothed version of the impact on test statistic~(\ref{eq:testStat}). In turn, the contribution of the present or absent edge between node pair $(i, j)$ of network $g_1$ to the difference in the inferred structure can be quantified by evaluating
\begin{gather*}
    \begin{cases}
        \hat{w}_{(g_1) (g_2)}^{\text{diff},+}(\hat{u}_i^{(g_1)},\hat{u}_j^{(g_1)}) \, , & \text{if } y_{ij}^{(g_1)} = 1 \\
        \hat{w}_{(g_2) (g_1)}^{\text{diff},+}(\hat{u}_i^{(g_1)},\hat{u}_j^{(g_1)}) \, , & \text{otherwise}
    \end{cases} 
\end{gather*}
with $\hat{w}_{(g_1) (g_2)}^{\text{diff},+}(u,v) = \max \{ 0, \hat{w}_{(g_1)(g_2)}^{\text{diff}}(u,v) \}$. In particular, this means that the contribution is zero if the considered present or absent edge is contrary to the direction of the detected difference. Note that also here, the estimated node positions $\hat{u}_i^{(g_1)}$ are the ones stemming from the network alignment, meaning the positions resulting from estimating the joint graphon model. We stress that this approach is different from determining the deviation of present or absent edges from their ``transferred'' expectation, i.e.\ from simply calculating $\vert y_{ij}^{(g_1)} - \hat{w}^{(g_2)}(\hat{u}_i^{(g_1)}, \allowbreak \hat{u}_j^{(g_1)}) \vert$. Instead, here we aim to highlight those connections that, in one way or another, \textit{collectively} have enough impact to actually affect the inferred structure. To provide an illustrative intuition, this approach for detecting deviating behavior on the microscopic scale is additionally illustrated in Figure~\ref{fig:micDiff}. 
\begin{figure}%
	\centering
    \vspace*{-0.4cm}
    
    \makebox[\textwidth][c]{\includegraphics[width=1.01\textwidth]{graphics/microDiffs/illustrFig8b.png}}
    \vspace*{-0.0cm}
    
    \hspace*{-1.15cm}
    \begin{minipage}{1.13\textwidth}
      \centering
      \caption{Detecting differences in networks at the microscopic level by employing the joint graphon-based alignment technique. The final network representations at the bottom row illustrate extraordinary \textit{absent} edges in network~1 and extraordinary \textit{present} edges in network~2 on the left and the converse on the right. The steps to get there are as follows: \mbox{(i) Align} networks based on joint smooth graphon estimation. (ii) Estimate individual graphons for disjoint networks separately. \mbox{(iii) Ca}l\-cu\-late relative differences between graphon estimates [$\rightarrow$ $\hat{w}_{(g_1)(g_2)}^{\text{diff}}(\cdot, \cdot)$]. (iv) Evaluating the function of relative differences at the edge positions provides information about contributions to the inferred structural deviation. This assessment is restricted to present or absent edges that are opposed to the formation of equivalent structures.
      }
	   \label{fig:micDiff}
    \end{minipage}
\end{figure}
This representation allows to graphically demonstrate the single steps and thus to make the procedure much clearer.


\subsection{Application to Human Brain Coactivation Example}
\label{sec:brainLocDiff}
The analysis of the functional coactivation in the human brain has yielded a rather narrow test decision with regard to the two clinical groups, see Section~\ref{sec:humBrainCo} of the paper. Hence, we now additionally analyze the differences at the microscopic level. To quantify these differences, we make use of the approach described above. The corresponding results for this method are illustrated in Figure~\ref{fig:realWorld2b}, where the ASD and the TD group are represented on the left and right, respectively. In all these illustrations, the node coloring refers to the positions inferred through the joint graphon estimation procedure. For comparison reasons, the first row depicts again the functional connectivity networks just as obtained after preprocessing (cf.\ Figure~\ref{fig:realWorld2} of the paper). 
\begin{figure}%
	\centering
	\begin{minipage}[t]{1\textwidth}
		\centering
		\begin{minipage}[c]{0.49\textwidth}
			\centering
			\includegraphics[width=.99\textwidth]{graphics/brain/both/1_26_networkbb_EM_net1.png}
		\end{minipage}
		\hfill
		\begin{minipage}[c]{0.49\textwidth}
			\centering
			\includegraphics[width=.99\textwidth]{graphics/brain/both/1_26_networkbb_EM_net2.png}
		\end{minipage}
		\vfill
		\begin{minipage}[c]{0.49\textwidth}
			\centering
			\includegraphics[width=.99\textwidth]{graphics/brain/both/1_26_networkDiff3_EM_net1.png}
		\end{minipage}
		\hfill
		\begin{minipage}[c]{0.49\textwidth}
			\centering
			\includegraphics[width=.99\textwidth]{graphics/brain/both/1_26_networkDiff3_EM_net2.png}
		\end{minipage}
		\vfill
		
		\begin{minipage}[c]{0.49\textwidth}
			\centering
			\includegraphics[width=.99\textwidth]{graphics/brain/both/1_26_networkDiffRev3_EM_net1.png}
		\end{minipage}
		\hfill
		\begin{minipage}[c]{0.49\textwidth}
			\centering
			\includegraphics[width=.99\textwidth]{graphics/brain/both/1_26_networkDiffRev3_EM_net2.png}
		\end{minipage}
	\end{minipage}
	\caption{Localization of differences in the functional coactivation within the human brain. The results for the two clinical groups ASD and TD are represented in the left and the right column, respectively. Top: resulting connectivity between the $116$ considered brain regions after preprocessing; degree of nodes is illustrated by log-scaled node size. The lower four plots show observed present edges (middle) and absent edges (bottom) that form extraordinary structural patterns with respect to the structure found in the respective other network. The node sizes visualize the nodes' impact (log scale) as aggregation over the impact of attached edges.}
	\label{fig:realWorld2b}
\end{figure}
The second row shows the present edges in both networks which form collectives that are exceptional with respect to the structure uncovered from the respective other network. Here, the intensity of the depicted connections represents the magnitude of distinctiveness on an inverse log scale. The most extraordinary absent edges---also with respect to the structure of the respective other network and with magnitude represented by the inversely log-scaled intensity---are visualized at the bottom row. Based on these illustrations, we can conclude that, for example, the interconnection between the dark bluish nodes is much denser in the ASD group than in the TD group. Similar results are revealed for the interconnection between nodes from the green to the yellow color spectrum. In contrast, the connections between the dark orange and the cyan node bundle seem to be much more for the TD group. With regard to the test procedure carried out in the paper, these microscopic differences can be seen as a rough division of the calculated overall discrepancy represented in the form of the test statistic $t$.

Taking these analytical results together with the finding from the paper, we can (i) infer that the functional connectivity significantly differs between the ASD and the TD group and (ii) provide information on what these differences are composed of.



\section{Comparison of Brain Coactivation Between Typical-Development Subgroups}
\label{sec:showTDres}
To further demonstrate that the found significant differences in the brain coactivation between the ASD and the TD group are meaningful (see Section~\ref{sec:humBrainCo} of the paper), we here repeat the analysis for two randomly selected subgroups of the TD group. The results are illustrated in Figure~\ref{fig:realWorld2c}. 
\begin{figure}%
	\centering
	\begin{minipage}[t]{1\textwidth}
		\centering
		\begin{minipage}[c]{0.49\textwidth}
			\centering
			\includegraphics[width=.99\textwidth]{graphics/brain/tdc/1_83_network_EM_net1.png}
		\end{minipage}
		\hfill
		\begin{minipage}[c]{0.49\textwidth}
			\centering
			\includegraphics[width=.99\textwidth]{graphics/brain/tdc/1_83_network_EM_net2.png}
		\end{minipage}
		\vfill
		\vspace*{0.5cm}
		
		\begin{minipage}[c]{0.49\textwidth}
			\centering
			\includegraphics[width=.99\textwidth]{graphics/brain/tdc/1_83_relDiff_of_nets.png}
		\end{minipage}
		\hfill
		\begin{minipage}[c]{0.49\textwidth}
			\centering
			\includegraphics[width=.99\textwidth]{graphics/brain/tdc/1_83_testStat_distr.png}
		\end{minipage}
	\end{minipage}
	\caption{Comparison of the human brain functional coactivation between two subgroups of the TD group. Top: networks of subgroups with coloring referring to the inferred node positions. Bottom left: differences in rectangles according to the construction of test statistic~(\ref{eq:testStat}) in the paper. Bottom right: result of the test statistic, including the simulated and the theoretical distribution plus their corresponding critical values used for comparison.}
	\label{fig:realWorld2c}
\end{figure}
Considering the formation of the two networks in the top row, inclusive of the nodes' positional embedding found by the algorithm, this seems very similar for the two subgroups. This is also reflected by the differences in rectangles (bottom left) and the final realization of the test statistic (bottom right). Comparing the latter with the corresponding critical value shows that no significant difference on the global scale can be found.




\end{document}