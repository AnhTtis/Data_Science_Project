\noindent
Cross-silo and cross-device FL make up two major settings in federated learning. Cross-silo typically has fewer clients participating, each with strong computation and communication abilities. On the other hand, cross-device FL has a much larger quantity of clients participating in training, however, each coming with communication or computational limitations. Cross-device has another problem compared to cross-silo FL being in stragglers or user dropout. Within a training round, users can have problems that prevent successful completion of training. As a result, the client will not participate in aggregation. For \name, this means one less client used to recover images.

Non-IID client data is more common in application, resulting from different user location, data skew, or other factors. As discussed before, client data will not have identical dataset distribution, but they can be learned individually over several training iterations. While our experiments worked under the assumption that the same clients participated in each training round, this may not always be the case. Instead, different groups of clients will usually participate in each round. Even if clients do not participate sequentially, their distributions can be observed each time they do participate. In the case that the server has absolutely no knowledge of the user participation, they would then be forced to observe a combined dataset distribution across all users. Although fewer images would be leaked in total as shown in the experiments, the number will still easily be in the thousands.

Much work in FL study the non-IID setting, but from the best of our knowledge this work is the first to discuss the utilization of non-IID distributions specifically for data leakage attacks. We hope that future work can continue to explore the application of reconstruction attacks on non-IID clients.
