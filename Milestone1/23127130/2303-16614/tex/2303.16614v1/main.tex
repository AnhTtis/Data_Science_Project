\documentclass[a4paper,12pt]{extarticle}
\usepackage{geometry}
\usepackage[english]{babel}
\usepackage{amsmath,amssymb,amsfonts,amsthm}
\usepackage[mathscr]{eucal}
\usepackage[all]{xy}
\usepackage{bigints}
\usepackage{hyperref}
\usepackage{setspace}
\usepackage{upgreek}
\usepackage{bm}
\usepackage[center]{caption}
\usepackage{hyperref}
\usepackage{cite}
\usepackage{graphicx}
\usepackage{times}
\usepackage{color}
\renewcommand{\baselinestretch}{1.5}
\tolerance=2000

\textwidth 170mm \textheight 240mm \thispagestyle{empty} \topmargin
-5mm \oddsidemargin -0.5cm \evensidemargin -5mm \language=0
\usepackage{indentfirst}

\allowdisplaybreaks
\title{Coulomb problem for classical spinning particle}
\author{D.S. Kaparulin\thanks{E-mail: \texttt{dsc@phys.tsu.ru}}\; and
N.A. Sinelnikov\thanks{E-mail: \texttt{nikitasineln@gmail.com}}\\[0.5em]
{\normalsize Physics Faculty, Tomsk State University, Tomsk 634050, Russia}
}
\date{}

\begin{document}
\maketitle
\begin{abstract}
We consider a motion of a weakly relativistic charged particle with an arbitrary spin in central potential $e/r$ in terms of classical mechanics. We show that the spin-orbital interaction causes the precession of the plane of orbit around the vector of total angular momentum. The angular velocity of precession depends on the distance of the particle from the center. The effective potential for in-plane motion is central, with the corrections to Coulomb terms coming from spin-orbital interaction. The possible orbits of a quantum particle are determined by the Bohr-Sommerfeld quantization rule. We give examples of orbits corresponding to small quantum numbers, which were obtained by numerical integration of equations of motion. The energies of stationary states are determined by spin-orbital interaction.
\end{abstract}
\section{Introduction}

The equations of quantum mechanics provide the most precise and systematic setting for a description of the dynamics of elementary particles with spin \cite{LL}. The problem also admits quasi-classical consideration developed within the framework of the co-orbit method \cite{Kirillov}, \cite{Kostant}, \cite{Souriau}. The classical limit of a quantum dynamical system with spin is a point particle with proper angular momentum or a spinning particle. By spinning particle, we mean a point particle with a proper angular momentum \cite{Fryd}. For a modern review of the problem, we cite article \cite{Deriglazov}. One of the interesting applications of spinning particle theory includes the motion of an elementary particle with spin or twisted wave packet in the external or gravitational field. In the case of electromagnetic field, the problem has been pioneered by \cite{Frenkel}, the motion in curved space-time has been first considered by Mathisson and Papapetrou \cite{Mathisson}, \cite{Papapetrou}. The current applications of spinning particle theory include accelerator physics \cite{ap-spin1}, \cite{ap-spin2}, celestial mechanics \cite{Adler}, laser physics \cite{Karlovets}, statistical physics \cite{BKN-2022}, and astrophysics \cite{TFC-1976}, \cite{Semerak}, \cite{OST-2013}, \cite{Han-2017}, \cite{APP-2019}. Despite the significant studies, the number of exact solutions for spinning particle trajectories is relatively small. Among the recent advances in the field, we mention articles \cite{DP-2014} and \cite{KR-2022} considering the motion in the uniform magnetic and electric field.

One of the interesting problems of quantum physics is the motion of a charged particle in the Coulomb field. The quasi-classical trajectories of a point particle in the Coulomb field, are first considered by Bohr in \cite{Bohr}. The quantization of energy and angular momentum has been proposed. The problem has been reconsidered with an account of relativistic effects and elliptical motion by Sommerfeld \cite{Sommerfeld}. Both authors considered a spinless particle, and their predictions well agreed with the experiments. The quantum solution for the Coulomb problem for a non-relativistic spinless particle has been found by Schrödinger in article \cite{Schroedinger}. The predicted energy levels were the same as in the Bohr model. The Coulomb problem for a relativistic field of spin $1/2$ has been solved by Dirac \cite{Dirac}. Due to the accidental coincidence, the energy levels of the model have been given by the relativistic Sommerfeld formula, even though the systems are different (a spinning and a spinless particle). There is no contradiction between the results because the predictions of quasi-classical theory should be exact only in the limit of large quantum numbers. The relativistic wave equation for an arbitrary spin quantum particle in the Coulomb field has been solved in \cite{FN}. The implicit formula for the energy levels has been given. The explicit solution has been presented in the weakly relativistic limit. The energy levels of the model depend on three quantum numbers $n$, $j$, $l$. This result means that, in the case of arbitrary spin, both the orbital and total angular momenta of the particle are relevant. 

In the current work, we consider the Coulomb problem for a particle of arbitrary spin in the quasi-classical setting. The problem has not been considered up to date, even though some preliminary discussions on the subject can be found in the recent article \cite{Der-2016}. We address three following questions: i) to derive the equations of motion for a weakly relativistic particle; ii) to integrate the equations of motions and find classical orbits; iii) to apply the Bohr-Sommerfeld quantization rule to determine the quasi-classical orbits for the particle. 
We solve the problem in the weakly relativistic approximation proposed in \cite{Der-2016}, where 
only the non-relativistic terms and the first relativistic correction are included in the action functional. The effective potential energy of the model takes into account the Coulomb term and the spin-orbital interaction, being proportional to the orbital angular momentum. Solving the equations of motion, we show that the spin-orbital interaction has two different effects. First, it causes the precession of the orbital plane of the particle around the vector of total angular momentum. Second, the spin-orbit coupling changes the shape of the trajectory, making the particle orbit non-Keplerian. The frequency of the apse line precession is found to be a rational multiple of the orbital plane precession in the non-relativistic limit for the quasi-classical states satisfying Bohr-Sommerfeld quantization rules. This implies a special form of quasi-classical particle orbits resembling the Lissajous figures known in oscillation theory. It is important to note that the rational ratio of the frequencies is true only in the non-relativistic limit, so the particle trajectories are not closed in the exact sense with the small difference due to relativistic effects. The final task of our work is the comparison of quasi-classical predictions with the exact formula derived for a relativistic particle in the \cite{FN}. We do not expect identical results from the formulas because the difference is observed for $s=1/2$ particle (quasi-classical Sommerfeld formula for spinless particle describes the energy levels of Dirac quantum particle), so we compare them in the limit of large quantum numbers where they agree. 

The article is organized as the following. In the next section, we describe the classical dynamics of the model. We start with the presentation of the model of a weakly relativistic spinning particle traveling in the external electromagnetic field proposed in \cite{Der-2016}. The equations of motion of the particle are derived in the first-order form and integrated in the analytical form. In Section 3, we apply the Bohr-Sommerfeld quantization rules to determine the stationary orbits of the particle in the Coulomb field. We conclude that the quasi-classical states are determined by four quantum numbers: the radial quantum number $n_r$, the total angular momentum quantum number $j$, the orbital quantum number $l$, and the magnetic quantum number $m$. The solution for the particle energy levels agrees with the formula of Fushich and Nikitin \cite{FN} in the limit of large quantum numbers. At the end of the section, we present the results of the numerical simulations of quasi-classical orbits for selected quantum numbers. The conclusion summarizes the results. 

\section{Classical dynamics}
We consider the model of a point weakly relativistic massive charged spinning particle with the mass $m$ and spin $s$ proposed in \cite{Der-2016}. The particle state is determined by the particle position $\mathbf{x}$, particle linear momentum $\mathbf{p}$, and spin vector $\mathbf{s}$. The normalizing condition for spin vector reads $(\mathbf{s},\mathbf{s})=\hbar^2s(s+1)$. The round bracket determines the scalar product with respect to the Euclidean metric. The Poisson bracket on the phase space has the following form:
\begin{equation}
\begin{gathered}
    \{x^i,p^j\}=\delta^{ij}\,,\qquad \{s^i,s^j\}=\epsilon^{ijk}s^k\,;\\
    \{x^i,x^j\}=\{p^i,p^j\}=\{x^i,s^j\}=\{p^i,s^j\}=0\,.
    \end{gathered}
\end{equation}
Here, $x^i$, $p^i$, $s^i$, $i=1,2,3$ denote the components of the spatial vectors $\boldsymbol{x}$, $\boldsymbol{p}$, $\boldsymbol{s}$, and $\epsilon$ stands for the $3d$ Levi-Civita symbol. We use the convention $\epsilon^{123}=1$ throughout the paper. All the spatial indices $i,j,k,\ldots$ are raised and lowered by the Euclidean metric. The following Hamiltonian describes the motion of the charged spinning particle in the field of unit electric charge located at the origin, 
\begin{equation}\label{Ham0}
    H=\frac{1}{2m}\mathbf{p}^2-\frac{1}{8m^3c^2}\mathbf{p}^4-\frac{e^2}{r}+\frac{(g-1)e^2}{2m^2c^2}(\boldsymbol{\mathbf{s}},\boldsymbol{\mathbf{L}})\,.
\end{equation}
Here, $m$ is the particle mass, $s$ is the particle spin, the orbital angular momentum is $\mathbf{L}=[\mathbf{x},\mathbf{p}]$. The constant $e$ is an elementary electric charge, and $c$ is the speed of light. The quantity $r$ denotes the distance from the origin.  

In the current article, we use a special unit system where the distances are measured in the units of Bohr radius, the momentum in the units of momentum on the Bohr orbit, energies in the particle rest energy, and spin in the Planck units. The new dynamical variables $\widetilde{\mathbf{x}}$, $\widetilde{\mathbf{p}}$, $\widetilde{\mathbf{s}}$ are introduced by the rule
\begin{equation}
    \mathbf{x}=r_B\widetilde{\mathbf{x}}=\frac{\hbar^2}{me^2}\widetilde{\mathbf{x}}\,,\qquad p=mv_B\widetilde{\mathbf{p}}=\frac{me^2}{\hbar}\widetilde{\mathbf{p}}\,,\qquad \mathbf{s}=\hbar\widetilde{\mathbf{s}}\,.
\end{equation}
($r_B$ and $v_b$ denote the Bohr radius and the speed on the first Bohr orbit.) In the subsequent computations, we exclusively use the variables  $\widetilde{\mathbf{x}}$, $\widetilde{\mathbf{p}}$, $\widetilde{\mathbf{s}}$, with the tilde being omitted to avoid cumbersome notation. We rewrite the Hamiltonian (\ref{Ham0}) in the following form: 
\begin{equation}\label{Ham}
    H=\frac{\alpha^2}{2}\boldsymbol{\mathbf{p}}^2-\frac{\alpha^4}{8}\boldsymbol{\mathbf{p}}^4-\frac{\alpha^2}{r}+\frac{(g-1)}{2}\frac{\alpha^4}{r^3}(\boldsymbol{\mathbf{s}},\boldsymbol{\mathbf{L}})\,.
\end{equation}
Here, $\alpha=e^2/\hbar c$ is the fine structure constant. The Hamiltonian (\ref{Ham}) can be considered as the power series in the small parameter $\alpha$. The leading contribution in $\alpha$ corresponds to the non-relativistic Hamiltonian, and the quartic contribution accounts for the first relativistic correction. The exact expression for the Hamiltonian (\ref{Ham0}) is given by the infinite series in the inverse powers of the speed of light $1/c$. This corresponds to the infinite power series in $\alpha$ in (\ref{Ham}). The next-order correction will be proportional to $\alpha^6$. In the current article, we restrict ourselves with the precision $o(\alpha^4)$, with all the higher-order corrections being systematically ignored. This assumption is not a serious limitation because the quasi-classical theory is expected to be correct for energy levels with big quantum numbers. The relativistic corrections for these energy levels are expected to be small.

The Hamiltonian equations of motion for the dynamical variables $\mathbf{x}$, $\mathbf{p}$, $\mathbf{s}$ read
\begin{equation}\label{EoM}\left\{\begin{array}{l}\displaystyle
    \frac{d\mathbf{x}}{dt}=\bigg(1-\frac{\alpha^2}{2}\mathbf{p}^2\bigg)\mathbf{p}+\frac{(g-1)}{2}\frac{\alpha^2}{r^3}[\mathbf{s},\mathbf{x}]\,;\\[7mm]
    \displaystyle
    \frac{d\mathbf{p}}{dt}=-\bigg(\frac{1}{r^3}-\frac{3(g-1)}{2}\frac{\alpha^2}{r^5}(\mathbf{s},\mathbf{L})\bigg)\mathbf{x} +\frac{(g-1)}{2}\frac{\alpha^2}{r^3}[\mathbf{s},\mathbf{p}];\\[7mm]
    \displaystyle
    \frac{d\mathbf{s}}{dt}=\frac{(g-1)}{2}\frac{\alpha^2}{r^3}[\mathbf{L},\mathbf{s}]\,.
\end{array}\right.\end{equation}
Integration of these equations gives the particle trajectory with the initial position $\mathbf{x}_0$, initial momentum $\mathbf{p}_0$, and initial value of spin $\mathbf{s}_0$. The system has four obvious integrals of motion, the vector of total angular momentum $\boldsymbol{J}$, and one scalar quantity $\mathbf{L}^2$. As the spin vector is normalized, the latter also implies that the scalar product $(\mathbf{s},\mathbf{L})$ is also a constant on each classical trajectory. The physical meaning of the observation is that the vector of spin $\mathbf{s}$ and the vector of angular momentum $\mathbf{L}$ have precession around the vector of total angular momentum $\mathbf{J}$. This is confirmed by the fact that the equations of motion for $\mathbf{s}$, $\mathbf{L}$ can be represented in the form:
\begin{equation}
\label{LS-der}
    \frac{d\mathbf{L}}{dt}=[\mathbf{\Omega}_j,\mathbf{L}]\,,\qquad    \frac{d\mathbf{S}}{dt}=[\mathbf{\Omega}_j,\mathbf{s}]
    \,, \qquad \mathbf{\Omega}_{j}=\frac{(g-1)}{2}\frac{\alpha^2}{r^3}\mathbf{J}\,.
\end{equation}
The angle $\gamma$ between vectors $\mathbf{J}$, $\mathbf{L}$ is determined by the rule
\begin{equation}
\label{angle_gamma}
    \gamma=\arccos\bigg(\frac{\mathbf{J}^2+\mathbf{L}^2-\mathbf{s}^2}{2\sqrt{\mathbf{J}^2\mathbf{L}^2}}\bigg)\,,
\end{equation}
and it is a constant of motion. 

To find an exact solution of equations of motion, it is convenient to introduce the non-inertial system (\ref{EoM}) rotating around the spin vector. Following the book \cite{LLm}, we denote the time derivative in the rotating coordinate system by $d'/dt$. The derivative of an arbitrary vector, being a function on the phase space of variables $\mathbf{x}$, $\mathbf{p}$, $\mathbf{s}$ (and measured in non-inertial frame) reads
\begin{equation}
\frac{d}{dt}=\frac{d'}{dt}+
\bigg[\mathbf{\Omega}_s, \cdot\bigg]\,,\qquad \mathbf{\Omega}_s=\frac{(g-1)}{2}\frac{\alpha^2}{r^3}\mathbf{s}\,.
\end{equation} 
In the rotating coordinate system, the equations (\ref{EoM}) take the following form:
\begin{equation}\label{EoM-rot}
\left\{\begin{array}{c}\displaystyle
\frac{d'\mathbf{x}}{dt}=\bigg(1-\frac{\alpha^2}{2}\mathbf{p}^2\bigg)\mathbf{p}\,,\qquad \frac{d'\mathbf{p}}{dt}=-\bigg(\frac{1}{r^3}-\frac{3(g-1)}{2}\frac{\alpha^2}{r^5}(\mathbf{s},\mathbf{L})\bigg)\mathbf{x}\,;\\[7 mm]
    \displaystyle
    \frac{d'\mathbf{s}}{dt}=\frac{(g-1)}{2}\frac{\alpha^2}{r^3}[\mathbf{L},\mathbf{s}]\,.
\end{array}\right.\end{equation}
The first and second equations describe the dynamics of translational degrees of freedom. They correspond to the motion in the central field with the effective potential
\begin{equation}\label{U-eff}
    U_{\text{eff}}=-\frac{\alpha^2}{r}+\frac{(g-1)}{2}\frac{\alpha^4}{r^3}(\mathbf{s},\mathbf{L})\,.
\end{equation}
In the last expression, the scalar product $(\mathbf{s},\mathbf{L})$ is considered as the parameter, being independent of the radial variable. The radial potential (\ref{U-eff}) is given by the sum of the Coulomb term and spin-orbital contribution, being a small correction. The translational motion in the rotating coordinate system is planar, as the orbital angular momentum is conserved in the non-inertial frame. The last equation in the system (\ref{EoM-rot}) describes the precession of spin, and it has the original form (\ref{EoM}). Equations (\ref{EoM-rot}) are important because they allow us to decouple the dynamics of translational and spin degrees of freedom. After this, both degrees of freedom can be considered independently.   

Let us consider translational motion. Without loss of generality, we assume that the vector of orbital angular momentum $L$ is directed along the third coordinate axis of the non-inertial frame, so the motion occurs in the orthogonal to $\mathbf{L}$ plane. We describe the particle position in the plane by polar coordinates $r,\varphi$. The system has two obvious quantities preserved by the planar motion: the norm of orbital angular momentum $L$, and the energy of translational motion. The first relation determines the angular velocity of motion around the origin in the  second-Kepler-law style, 
\begin{equation}\label{L-df}
    \frac{d'\varphi}{dt}=\frac{L}{r^2}\,.
\end{equation}
The law of conservation of energy determines the velocity of radial motion. Taking into account that
\begin{equation}
    \mathbf{p}^2=\bigg(\frac{d'r}{dt}\bigg)^2+\frac{\mathbf{L}^2}{r^2}\,,
\end{equation}
with the precision $o(\alpha^4)$, we get
\begin{equation}\label{E-dr}
    \frac12\bigg(\frac{d'r}{dt}\bigg)^2=\frac{1}{\alpha^2}\Bigg[\bigg(E+\frac{\alpha^2}{r}\bigg)+\frac{1}{2}\bigg(E+\frac{\alpha^2}{r}\bigg)^2-\frac{\alpha^2L^2}{2r^2}-\frac{(g-1)}{2}\frac{\alpha^4}{r^3}(\mathbf{s},\mathbf{L})\Bigg]\,.
\end{equation}
Combining (\ref{L-df}), (\ref{E-dr}), we obtain two ODEs with respect to $\varphi(r)$ and $t(r)$. Their solution reads
\begin{equation}\label{EoMpolar}
t-t_0=\pm \bigintsss_{r_0}^r\frac{\alpha dr}{\displaystyle
\sqrt{2\bigg(E+\frac{\alpha^2}{r}\bigg)+\bigg(E+\frac{\alpha^2}{r}\bigg)^2-\frac{\alpha^2L^2}{r^2}-\frac{(g-1)\alpha^4}{r^3}(\mathbf{s},\mathbf{L})}}\,;
\end{equation}
\begin{equation}\label{EoMpolar-1}
\varphi-\varphi_0=\pm \bigintsss_{r_0}^r\frac{\alpha L r^{-2}dr}{\displaystyle
\sqrt{2\bigg(E+\frac{\alpha^2}{r}\bigg)+\bigg(E+\frac{\alpha^2}{r}\bigg)^2-\frac{\alpha^2L^2}{r^2}-\frac{(g-1)\alpha^4}{r^3}(\mathbf{s},\mathbf{L})}}\,;
\end{equation}
Both integrals admit an exact solution in terms of the Weierstrass elliptic functions. The explicit expressions for the integrals are long and not too informative, we do not present them here. Here, we restrict ourselves to the periods of integrals, that have the sense of the orbital period of the particle, and the angle of rotation between the two turning points of one type of radial motion. Both quantities are easily computed using the techniques of the book \cite{LLm},
\begin{equation}\label{per}
    T=\frac{\pi \alpha^3}{\sqrt{-2E^3}}\bigg(1-\frac{1}{4}E\bigg)\,,\qquad \Phi=2\pi+\frac{\pi\alpha^2}{L^2}\bigg(1-3(g-1)\frac{(\mathbf{s},\mathbf{L})}{L^2}\bigg)\,.
\end{equation}
The quantity $\Delta \Phi_a=\Phi-2\pi$ determines the rotation of the apse line per one radial oscillation (between two pericenters or two apocenters). 

We finalize the section by returning to the inertial frame, where the particle orbit is involved in two precessions: the rotation of the orbital plane (\ref{LS-der}) and rotation of the apse line (\ref{per}). To compare the (average) angular velocities of precession, we compute the rotation angle for orbital motion per one radial oscillation,
\begin{equation}
   \Delta\Phi_{j}=\int^{T}_{0}\Omega_{j}(r)dt=\frac{\pi(g-1)\alpha^2J}{L^3}\,. 
\end{equation}
The ratio of these two angles reads
\begin{equation}
\label{prec_ratio}
    \frac{\Delta \Phi_{a}}{\Delta \Phi_{j}}=\frac{L}{J}\bigg(\frac{1}{g-1}-\frac{3(\mathbf{s},\mathbf{L})}{L^2}\bigg)\,.
\end{equation}
Even though this ratio can be an arbitrary real number it is a rational quantity for quasi-classical orbits for a particle without anomalous magnetic moment. Indeed, the quasi-classical quantization rules imply that $L$, and $J$ are integer (integer and half-integer) numbers (see Section 3), while the scalar product $(\mathbf{s},\mathbf{L})$ is rational because of the formula (\ref{angle_gamma}). Finally, for a particle without anomalous angular momentum, $g=2$ and $g-1=1$. To our knowledge, the fact the precession velocities of the apse line and orbital plane are rational multiple has been observed here for the first time. It is important to note that the quantity (\ref{prec_ratio}) is rational only in the non-relativistic limit. The relativistic effects make this ratio irrational even for quasi-classical orbits.


\section{Quasi-classical orbits}

The Coulomb problem for a relativistic quantum particle is known to be an exactly solvable model, with the solution provided in the book \cite{FN}. In this section, we are going to reproduce the solution for the energies of stationary states by applying the Bohr-Sommerfeld quantization rule for the weakly relativistic particle and compare the result with the previously known exact solution.

We start with the consideration of the rotational motion. The symmetry of the model (\ref{EoM}) has three following integrals of motion in involution: $\mathbf{J}^2$, $\mathbf{L}^2$, and $J_z$. The quantization of orbital angular momentum follows from the so-called "spatial quantization" rules proposed in \cite{Sommerfeld}. We use the quantization condition for $\mathbf{L}^2$ because $\mathbf{L}$ is not conserved in the model. The quantization of total angular momentum follows from the fact that the spin projection on a fixed axis takes (half-)integer numbers, while the classical vector $\mathbf{J}$ is directed along some vector. In its turn, the quantization of spin in quasiclassical theory is predicted by the co-orbit method \cite{Kirillov}, \cite{Kostant}, \cite{Souriau}. Both mentioned conditions imply that $J_z$ takes the (half-)integer numbers, while $\mathbf{J}^2$, $\mathbf{L}^2$ are given by the squares of (half-)integer and integer numbers,
\begin{equation}
\label{stand_quant}
    \phantom{\frac12}\mathbf{J}^2=j^2\,,\qquad \mathbf{L}^2=\ell^2\,,\qquad J_z= m\,.\phantom{\frac12}
\end{equation}
(We recall that $\hbar=1$ in our unit system.) The quantum numbers are the total angular momentum quantum number $j$, the orbital quantum number $\ell$, and the magnetic quantum number $m$. The quantum numbers are restricted by the condition that the vectors $\mathbf{J}$, $\mathbf{L}$, $\mathbf{s}$ form a triangle with normalized $s$. Quantum numbers $j$, $l$, $m$ run over the set
\begin{equation}\label{jl-quant}
    \phantom{\frac12}\ell=1,2,\ldots,\qquad m=-j,\ldots,j\,,\qquad j=\ell-m_{s\ell},\ldots,\ell+s \,.\phantom{\frac12}
\end{equation}
where $m_{s\ell}=\text{min}\{s,\ell\}$. The quantization rules for the vectors $\mathbf{J}^2$, $\mathbf{L}^2$, and $J_z$ are universal for any particle model with the spherical symmetry of the potential and are not connected with the specifics of the model. 

The non-trivial condition comes from the discretization of radial motion. The quantization rule for radial variable reads
\begin{equation}\label{nr}
    \int p_r dr=2\pi n_r\,,\qquad n_r=0,1,\ldots\,.
\end{equation}
Expressing the radial momentum from the Hamiltonian (\ref{Ham}), we arrive at the following condition:
\begin{equation}\label{nr1}
   \bigointsss \alpha^{-1}\sqrt{2\bigg(E+\frac{\alpha^2}{r}\bigg)+\bigg(E+\frac{\alpha^2}{r}\bigg)^2-\frac{\alpha^2L^2}{r^2}-\frac{(g-1)\alpha^4}{r^3}(\mathbf{s},\mathbf{L})}\; dr =2\pi n_r\,,
\end{equation}
where $E$ is the energy of the particle, being negative for finite motion. Calculating the integral (\ref{nr1}) with the precision $o(\alpha^4)$ and expressing energy, we obtain 
\begin{equation}\label{energy_njl}
\displaystyle
       E_{njl}=-\frac{\alpha^2}{2n^2}-\frac{\alpha^4}{2n^4}\left(\frac{n}{\ell}-\frac{3}{4}-(g-1)\frac{n(j^2-\ell^2-s(s+1))}{2\ell^3}\right)\,, 
\end{equation}
with $n=n_r + \ell$ being the principal quantum number. The first term is the Bohr energy, and the second is the first relativistic correction to energy. When $s=0$, so $j=\ell$, this formula coincides with the Sommerfeld one (and the quantum levels of Dirac particle). For a particle with the general value of spin, energy depends on three quantum numbers, except $m$. This means that the account of spin decreases the degeneracy of energy levels in the Coulomb problem. This fact has also been observed in \cite{FN}. 

Comparing the values of energy levels with the energies of the quantum state derived from the solution of relativistic wave equation \cite{FN}, we note that the coincidence of the results (even in the order $o(\alpha^4)$) will be a too strong requirement. Indeed, the classical relativistic Sommerfeld formula \cite{Sommerfeld} reproduces the energy levels of Dirac particle \cite{Dirac} with spin $s=1/2$. The Sommerfeld formula agrees with the energies of the quantum scalar particle only in the limit of large quantum numbers $l,n\to\infty$. Thus, the requirement of large quantum numbers seems to be necessary for comparing quantum and quasi-classical results. One more condition 
includes the spin value. We account for the spin-orbit term by perturbation theory, even though this contribution is proportional to spin, being an unlimited number. For extremely high spins, the perturbation theory no longer applies. To avoid contradiction, we need to assume that spin $s$ is small in comparison with other quantum numbers, i.e $n,\ell\gg s\gg1$. Finally, the work \cite{FN} considers the special value of the magnetic moment of the particle, while in our case the $g$-factor is an arbitrary number. 

Now, we can compare the formula (\ref{energy_njl}) with the predictions of quantum theory. The work \cite{FN} gives the following expression for the energy levels with the precision $o(\alpha^4)$:
\begin{equation}
    E_{njl}=-\frac{\alpha^2}{2n^2}-\frac{\alpha^4}{2n^4}\bigg(\frac{2n}{2\ell+1}-\frac{3}{4}-\frac{1}{4s^2}\frac{n}{2\ell+1}\bigg(\frac{a^{sj}_{j-\ell+s}}{\ell+1}-\frac{a^{sj}_{j-\ell+s+1}}{\ell}\bigg)\bigg)\, , 
\end{equation}
where
\begin{equation}
    a^{sj}_{\mu}=\frac{\mu(2j-\mu+1)(2s-\mu+1)(2j+2s-\mu+2)}{(2j+2s-2\mu+1)(2j+2s-2\mu+3)}\,, 
\end{equation}
with $\mu$ being an arbitrary half-integer number. The leading term, being proportional to $\alpha^2$ is the same in both formulas. As for $\alpha^4$ correction, the first and the second contributions become equal for large quantum numbers. The comparison for the third spin-dependent term requires an accurate computation of the coefficients $a^{sj}_{\mu}$ in the large quantum number limit (in the described above sense). The identification of two formulas follows from the fact that 
\begin{equation}
\frac{1}{2\ell+1}\bigg(\frac{a^{sj}_{j-\ell+s}}{\ell+1}-\frac{a^{sj}_{j-\ell+s+1}}{\ell}\bigg)\approx\frac{(j^2-\ell^2-s(s+1))}{2\ell^3}\approx \frac{j-\ell}{l^2}\,.
\end{equation}
Both formulas give the same result if $g=1+1/4s^2$. In particular, for an electron with spin $s=1/2$, we get $g=2$. The obtained result shows that the quantum and quasi-classical formulas agree with each other. The quantum formula gives a more fundamental exact result, while the quasi-classical formula seems to be more flexible as it involves the anomalous magnetic moment as the free parameter. 

We finalize the section by presenting the results of numerical simulations of quasi-classical orbits for some small quantum numbers. Equations of motions (\ref{EoM}) were numerically integrated for selected quantum numbers to demonstrate features of quasi-classical trajectories of an electron with spin $1/2$. We have chosen the increased value of $g$-factor $g=2000$ to increase the precession speed to get more illustrative images. 
The green point is the position of the Coulomb attraction center. The red arrow represents the vector of the total angular momentum. Figure (\ref{fig1}) represents the circular orbits with zero radial quantum number 
$n_r$. As the orbit has no pericenter, the only effect is the precession of the orbital plane. The angular width of the spherical segment is $2\gamma$, with gamma being the angle (\ref{angle_gamma}). For a non-zero radial quantum number, there is a radial motion, so there are two factors affecting the orbit: the precession of the orbit plane and the precession of the pericenter of the orbit. Formula (\ref{prec_ratio}) tells us that the ratio of the frequencies of precession is a rational number for each quasi-classical state whenever $g$-factor is a rational quantity. Figure (\ref{fig5}) shows us that the classical path forms the Lissajous-like figures due to this effect. At a very long integration time, the effect disappears due to the approximate character of the ratio of angular velocities of precession.

\begin{figure}[ht]
  \centering
  \includegraphics[width=1\linewidth]{circ.png}
  \captionof{figure}{Orbits of an electron for zero radial quantum number. From left to right:\\ $\ell=1, \; j=1/2; \quad \ell=2, \; j=3/2; \quad \ell=3, \; j=5/2.$}
  \label{fig1}
\end{figure}

\begin{figure}[ht]
  \centering
  \includegraphics[width=0.45\linewidth]{4.png}
  \captionof{figure}{An elliptical orbit for an electron with quantum numbers:\\$n_r = 1, \; \ell = 2, \; j=5/2.$}
  \label{fig5}
\end{figure}

\section{Conclusion}

In the current article, we have considered the Coulomb problem for a weakly relativistic particle of arbitrary spin from the quasi-classical point of view. Using the results of the work \cite{Der-2016}, we derived the equations of motion of the particle in the field of a point charge and integrated them. It has been shown that the presence of spin affects particle motion in two ways. It causes the precession of the orbital plane of the particle at first, and it generates the precession of the apse line, making the in-plane motion non-Keplerian. The ratio of precession angular velocities has been found to be rational for each quasi-classical state. Using the Bohr-Sommerfeld quantization rule, we found the energies of stationary states in the quasi-classical formalism with the precision $o(\alpha^4)$. Our formula agrees with the quantum mechanical result of Fushich and Nikitin \cite{FN} in the limit of large quantum numbers for a particle without anomalous magnetic moment. Once a particle has the anomalous magnetic moment $g-2=\frac{\alpha}{\pi}+o(\alpha)$, it is very attractive to interpret the additional contribution to the energy as the Lamb shift. Both quantities are proportional to $\alpha^5/n^3\ell^2$. Finally, we made the visualization of quasi-classical orbits for small quantum numbers. The orbits with zero radial quantum numbers lie on spherical segments whose width depends on the total angular momentum and orbital quantum number. This shape of circular obits better resembles the spherically symmetric quantum states that the planar Bohr orbits. The non-planar motion of an electron can be considered as the special form of Zitterbewegung.  The results of the article demonstrate once again that a spinning particle concept is a useful tool for the study of various phenomena involving quantum particles, including the motion in the external field.
\\\\
\textbf{Acknowledgments.} The authors thank the hospitality of organizers of the "AYSS-2022" conference of young scientists and specialists at JINR (Dubna, Russia) where preliminary results of the study have been presented by N.A. Sinelnikov. The authors thank A.A. Sharapov and N. Makhaldiani for valuable discussions of this work. We express our special gratitude to Yu.V. Brezhnev who participated in the initial stages of the project. The work was supported by the RSF project 19-71-10051-P.

\begin{thebibliography}{}


\bibitem{LL} L.D. Landau and E.M. Lifshitz, Quantum Mechanics. Non-relativistic Theory. Second edition, revised and enlarged. Course of Theoretical Physics (Pergamon Press, Oxford, 1977)

\bibitem{Kirillov} A.A. Kirillov, Elements of the theory of group representations (Springer-Verlag, Berlin, 1976).

\bibitem{Kostant} B. Kostant, Quantization and unitary representations. Lectures in modern analysis and applications III (1970) 87-208.

\bibitem{Souriau} J.M. Souriau, Structure of dynamical systems: a symplectic view of physics. Vol. 149. (Springer Science \& Business Media, 2012).

\bibitem{Fryd} A. Frydryszak. Lagrangian models of the particles with spin: the first seventy years, in: From Field Theory To Quantum Groups, p.151-172 (World Scientific Publishing, 1996)

\bibitem{Deriglazov} A.A. Deriglazov, W.G. Ramírez, Recent progress on the description of relativistic spin: vector model of spinning particle and rotating body with gravimagnetic moment in General Relativity, Adv. in Math. Phys. 2017, 7397159 (2017).

\bibitem{Frenkel} J. Frenkel. Die Elektrodynamik des rotierenden Elektrons, Z. Phys. 37, 243-262 (1926).

\bibitem{Mathisson} M. Mathisson, Neue mechanik materieller systeme, Acta Phys. Pol. 6, 163-200 (1937).

\bibitem{Papapetrou} A. Papapetrou, Spinning test-particles in general relativity, Proc. R. Soc. A 209, 248-258
(1951).

\bibitem{ap-spin1} G. H. Hoffstaetter, H. S. Dumas, and J. A. Ellison. Adiabatic invariance of spin-orbit motion in accelerators, Phys. Rev. ST Accel. Beams 9, 014001 (2006).

\bibitem{ap-spin2} J.P. Miller, E. de Rafael, and B. L. Roberts. Muon (g-2): experiment and theory, Reports on Progress in Physics 70, 795
(2007).

\bibitem {Adler} R.J. Adler, The three-fold theoretical basis of the Gravity Probe B gyro precession calculation, Class. Quant. Grav. 32, 224002 (2015).

\bibitem{Karlovets} D.V. Karlovets. Electron with orbital angular momentum in a strong laser wave, Phys. Rev. A 86, 062102 (2012).

\bibitem{BKN-2022} M.A. Bubenchikov, D.S. Kaparulin and O.D. Nosyrev, Chiral effects in classical spinning gas, J. Phys. A: Math. Theor. 55, 395006 (2022).

\bibitem{TFC-1976} K.P. Tod, F. de Felice, and M. Calvani, Spinning test particles in the field of a black hole Nuovo Cimento B 34, 365 (1976).

\bibitem{Semerak} O. Semerak, Spinning test particles in a Kerr field. 1 Mon. Not. R. Astron. Soc. 308, 863–875 (1999).

\bibitem{OST-2013}Y.N. Obukhov, A.J. Silenko and O.V. Teryaev, Spin in an arbitrary gravitational field Phys. Rev. D 88, 084014 (2013).


\bibitem{Han-2017}  W.-B. Han and S.-C. Yang, Exotic orbits due to spin–spin coupling around Kerr black holes Int. J. Mod. Phys. D 27, 1750179 (2017).

\bibitem{APP-2019} I. Antoniou, D. Papadopoulos, and L. Perivolaropoulos, Spinning particle orbits around a black hole in an expanding background,
Class. Quantum Grav. 36, 085002 (2019).


\bibitem{DP-2014} A.A. Deriglazov, A.M. Pupasov-Maksimov, Frenkel electron on an arbitrary electromagnetic background and magnetic Zitterbewegung, Nucl. Phys. B. 885, 1-24 (2014). 

\bibitem{KR-2022} D.S. Kaparulin, I.A. Retuntsev, On the world sheet of anyon in the external electromagnetic field,
Nucl. Phys. B. 980, 115836 (2022).


\bibitem{Bohr} N. Bohr, On the Constitution of Atoms and Molecules, Phil. Mag. Ser. 6, 1–24 (1913).

\bibitem{Sommerfeld} A. Sommerfeld, Atombau und Spektrallinien (Braunschweig : F. Vieweg \& Sohn, Braunschweig, 1921).

\bibitem{Schroedinger} E. Schrödinger, Quantisierung als Eigenwertproblem, Annalen Phys. 386 (1926) 18, 109-139.

\bibitem{Dirac}  P. A. M. Dirac, The quantum theory of the electron, Proc. Roy. Soc. Lond. A 117 (1928) 610-624.

\bibitem{FN} W.I. Fushchich, A.G. Nikitin, Symmetries of Equations of Quantum Mechanics (Allerton Press Inc., New York, 1994, 480 pp.)

\bibitem{Der-2016} A.A. Deriglazov, A.M. Pupasov-Maximov, Relativistic corrections to the algebra of position variables and spin-orbital interaction, Phys. Lett. B.
761, 207-212 (2016).

\bibitem{LLm} L.D. Landau, E.M. Lifshitz, Mechanics, vol. 1 (Butterworth-Heinemann, 1976).

\end{thebibliography}{}
\end{document}
