%-----------------------------------------------------------------------
%  arXiv version
%-----------------------------------------------------------------------
\documentclass[twoside,leqno,11pt]{article}
\usepackage{graphicx}                   % graphics (includegraphics)
\usepackage{amsmath}                    % AMS math
\usepackage{amssymb}                    % special AMS math symbols
\usepackage{amsfonts}                   % more special AMS math symbols
\usepackage{mathrsfs}                   % mathscr script fonts
\usepackage{url}                        % \url{...} formatting URLs
\usepackage{color}                      % colored text and backgrounds
\usepackage{cite}                       % sort bibliographical entries
\usepackage{hyperref}                   % add hyper-references
\usepackage{enumitem}                   % added controls for enumerations
\hypersetup{colorlinks=true,linkcolor=[rgb]{0.75,0,0},citecolor=[rgb]{0,0,0.75}}
\usepackage{amsthm}                     % AMS theorem/proof
\usepackage{thmtools}                    
\usepackage{thm-restate}
\usepackage{fullpage}
\usepackage[parfill]{parskip}

\declaretheorem[name=Theorem]{theorem}
\declaretheorem[name=Lemma,numberwithin=section]{lemma}
\declaretheorem[name=Corollary,sibling=lemma]{corollary}
%-----------------------------------------------------------------------
% General special symbols
%-----------------------------------------------------------------------
% With arguments
\newcommand{\floor}[1]{\left\lfloor #1\right\rfloor}
\newcommand{\ceil}[1]{\left\lceil #1\right\rceil}
\newcommand{\ang}[1]{\langle #1\rangle}
\newcommand{\half}[1]{\frac{#1}{2}}
\newcommand{\inv}[1]{\frac{1}{#1}}
\renewcommand{\ang}[1]{\langle #1\rangle}

% Letters, symbols, and words
\newcommand{\RE}{\mathbb{R}}            % real space
\newcommand{\ZZ}{\mathbb{Z}}            % integers
\newcommand{\eps}{\varepsilon}          % my preferred epsilon
\newcommand{\ST}{\,:\,}                 % { x \ST y }
\newcommand{\RR}{\mathscr{R}}
\renewcommand{\AA}{\mathscr{A}}
\newcommand{\CC}{\mathscr{C}}
\newcommand{\MM}{\mathscr{M}}
\newcommand{\PP}{\mathscr{P}}
\newcommand{\HH}{\mathscr{H}}
\newcommand{\WW}{\mathscr{W}}
\newcommand{\Transpose}{\intercal}
\newcommand{\bd}{\partial}
\newcommand{\etal}{\textit{et al.}}
\newcommand{\SP}{\kern+1pt}

% Math functions
\DeclareMathOperator{\diam}{diam}
\DeclareMathOperator{\sz}{size}
\DeclareMathOperator{\vol}{vol}
\DeclareMathOperator{\area}{area}
\DeclareMathOperator{\conv}{conv}
\DeclareMathOperator{\radius}{radius}
\DeclareMathOperator{\width}{wid}
\DeclareMathOperator{\ray}{ray}
\DeclareMathOperator{\dist}{dist}
\DeclareMathOperator{\base}{base}
\DeclareMathOperator{\shadow}{shadow}
\DeclareMathOperator{\U}{U}
\DeclareMathOperator{\SoftOh}{\widetilde{O}}
\DeclareMathOperator{\polylog}{polylog}
\DeclareMathOperator{\EX}{\mathbb{E}}
\DeclareMathOperator{\interior}{int}
\newcommand{\icone}[2]{\textup{icone}(#1,#2)}
\newcommand{\ocone}[2]{\textup{ocone}(#1,#2)}
\newcommand{\dcap}[2]{\textup{dcap}_{#1}(#2)}
\newcommand{\pcap}[2]{\textup{cap}_{#1}(#2)}

\newtheorem*{theorem*}{Theorem}

%-----------------------------------------------------------------------
% Use this for the arxiv version
%-----------------------------------------------------------------------
\newcommand{\arxivonly}[1]{#1}
\newcommand{\confonly}[1]{}
%-----------------------------------------------------------------------
% use this for the conference version
%-----------------------------------------------------------------------
%\newcommand{\arxivonly}[1]{}
%\newcommand{\confonly}[1]{#1}
%-----------------------------------------------------------------------

\begin{document}

\title{Optimal Volume-Sensitive Bounds for Polytope Approximation}

%=======================================================================
% Title
%=======================================================================

\author{
	Sunil Arya\thanks{Research supported by the Research Grants Council of Hong Kong, China under projects number 16213219 and 16214721.}\\
		Department of Computer Science and Engineering \\
		The Hong Kong University of Science and Technology, Hong Kong\\
		arya@cse.ust.hk \\
		\and
	David M. Mount\\
		Department of Computer Science and 
		Institute for Advanced Computer Studies \\
		University of Maryland, College Park, Maryland \\
		mount@umd.edu \\
}

\date{}

\maketitle

%-----------------------------------------------------------------------
\begin{abstract}
Approximating convex bodies is a fundamental question in geometry and has a wide variety of applications. Consider a convex body $K$ of diameter $\Delta$ in $\RE^d$ for fixed $d$. The objective is to minimize the number of vertices (alternatively, the number of facets) of an approximating polytope for a given Hausdorff error $\eps$. It is known from classical results of Dudley (1974) and Bronshteyn and Ivanov (1976) that $\Theta((\Delta/\eps)^{(d-1)/2})$ vertices (alternatively, facets) are both necessary and sufficient. While this bound is tight in the worst case, that of Euclidean balls, it is far from optimal for skinny convex bodies.

A natural way to characterize a convex object's skinniness is in terms of its relationship to the Euclidean ball. Given a convex body $K$, define its \emph{volume diameter} $\Delta_d$ to be the diameter of a Euclidean ball of the same volume as $K$, and define its \emph{surface diameter} $\Delta_{d-1}$ analogously for surface area. It follows from generalizations of the isoperimetric inequality that $\Delta \geq \Delta_{d-1} \geq \Delta_d$. 

Arya, da Fonseca, and Mount (SoCG 2012) demonstrated that the diameter-based bound could be made surface-area sensitive, improving the above bound to $O((\Delta_{d-1}/\eps)^{(d-1)/2})$. In this paper, we strengthen this by proving the existence of an approximation with $O((\Delta_d/\eps)^{(d-1)/2})$ facets.

This improvement is a result of the combination of a number of new ideas. As in prior work, we exploit properties of the original body and its polar dual. In order to obtain a volume-sensitive bound, we explore the following more general problem. Given two convex bodies, one nested within the other, find a low-complexity convex polytope that is sandwiched between them. We show that this problem can be reduced to a covering problem involving a natural intermediate body based on the harmonic mean. Our proof relies on a geometric analysis of a relative notion of fatness involving these bodies.
%
\end{abstract}

%=======================================================================
\section{Introduction} \label{sec:intro}
%=======================================================================

Approximating convex bodies by polytopes is a fundamental problem, which has been extensively studied in the literature (see, e.g., Bronstein~\cite{Bro08}). We are given a convex body $K$ in Euclidean $d$-dimensional space and an error parameter $\eps > 0$. The problem is to determine the minimum combinatorial complexity of a polytope that is $\eps$-close to $K$ according to some measure of similarity. In this paper, we define similarity in terms of the Hausdorff distance~\cite{Bro08}, and we define combinatorial complexity in terms of the number of facets. Throughout, we assume that the dimension $d$ is a constant.

Approximation bounds presented in the literature are of two common types. In both cases, it is shown that there exists $\eps_0 > 0$ such that the bounds hold for all $\eps \leq \eps_0$. The first of these are \emph{nonuniform bounds}, where the value of $\eps_0$ may depend on properties of $K$, for example, bounds on its maximum curvature~\cite{Bor00, Cla06, Gru93a, McV75, Sch87, Tot48}. This is in contrast to \emph{uniform bounds}, where the value of $\eps_0$ is independent of $K$ (but may depend on $d$).

Examples of uniform bounds include the classical work of Dudley~\cite{Dud74} and Bronshteyn and Ivanov~\cite{BrI76}. Dudley showed that, for $\eps \leq 1$, any convex body $K$ can be $\eps$-approximated by a polytope $P$ with $O((\Delta/\eps)^{(d-1)/2})$ facets, where $\Delta$ is $K$'s diameter. Bronshteyn and Ivanov showed the same bound holds for the number of vertices.  Constants hidden in the $O$-notation depend only on $d$. These results have numerous applications in computational geometry, for example the construction of coresets~\cite{AHV05,ArC14,AFM17b}. 

The approximation bounds of both Dudley and Bronshteyn-Ivanov are tight in the worst case up to constant factors (specifically when $K$ is a Euclidean ball)~\cite{Bro08}. However, these bounds may be significantly suboptimal if $K$ is ``skinny''. A natural way to characterize a convex object's skinniness is in terms of its relationship to the Euclidean ball. Given a convex body $K$, define its \emph{volume diameter} $\Delta_d$ to be the diameter of a Euclidean ball of the same volume as $K$, and define its \emph{surface diameter} $\Delta_{d-1}$ analogously for surface area. These quantities are closely related (up to constant factors) to the classical concepts of \emph{quermassintegrals} and of \emph{intrinsic volumes} of the convex body \cite{McM75,McM91}. It follows from generalizations of the isoperimetric inequality that $\Delta \geq \Delta_{d-1} \geq \Delta_d$~\cite{McM91}.

Arya, da Fonseca, and Mount~\cite{AFM12b} proved that the diameter-based bound could be made surface-area sensitive, improving the above bound to $O((\Delta_{d-1}/\eps)^{(d-1)/2})$. In this paper, we strengthen this to the following volume-sensitive bound.  

%-----------------------------------------------------------------------
\begin{theorem} \label{thm:main}
%
Consider real $d$-space, $\RE^d$. There exists a constant $c_d$ (depending on $d$) such that for any convex body $K \subseteq \RE^d$ and any $\eps > 0$, if the width of $K$ in any direction is at least $\eps$, then there exists an $\eps$-approximating polytope $P$ whose number of facets is at most
\[
	\left(\frac{c_d \Delta_d}{\eps} \right)^{\kern-2pt\frac{d-1}{2}}.
\]
\end{theorem}
%-----------------------------------------------------------------------

This bound is the strongest to date. For example, in $\RE^3$, the area-sensitive bound yields better bounds for pencil-like objects that are thin along two dimensions, while the volume-sensitive bound yields better bounds for pancake-like objects as well, which are thin in just one dimension.

The minimum-width assumption seems to be a technical necessity, since it is not difficult to construct counterexamples where this condition does not hold. But this is not a fundamental impediment. If the body's width is less than $\eps$ in some direction, then by projecting the body onto a hyperplane orthogonal to this direction, it is possible to reduce the problem to a convex approximation problem in one lower dimension. This can be repeated until the body's width is sufficiently large in all remaining dimensions, and the stated bound can be applied in this lower dimensional subspace, albeit with volume measured appropriate to this dimension.

While our uniform bound trivially holds in the nonuniform setting, we present a separate (and much shorter) proof that the same bounds hold in the nonuniform setting, assuming that $K$'s boundary is $C^2$ continuous.
%
\confonly{This is presented in the full version~\cite{SoCG23arxiv}.}%
%
\arxivonly{This is presented in Section~\ref{s:nonuniform}.} 

\begin{restatable}{theorem}{RLnonunifbound}
\label{thm:nonunif-bound}
Consider real $d$-space, $\RE^d$. There exists a constant $c_d$ (depending on $d$) such that for any convex body $K \subseteq \RE^d$ of $C^2$ boundary, as $\eps$ approaches zero, there exists an $\eps$-approximating polytope $P$ whose number of facets is at most
\[
	\left(\frac{c_d \Delta_d}{\eps} \right)^{\kern-2pt\frac{d-1}{2}}.
\]
\end{restatable}

%=======================================================================
\section{Overview of Techniques} \label{s:techniques}
%=======================================================================

Broadly speaking, the problem of approximating a convex body by a polytope involves ``sandwiching'' a polytope between two nested convex bodies, call them $K_0$ and $K_1$. For example, $K_0$ may be the original body to be approximated and $K_1$ is an expansion based on the allowed error bound. Most of the prior work in this area has focused on the specific manner in which $K_1$ is defined relative to $K_0$, which is typically confined to Euclidean space (for Hausdorff distance) or affine space (for the Banach-Mazur distance).

Recent approaches to convex approximation have been based on covering the body to be approximated with convex objects that respect the local shape of the body being approximated~\cite{AAFM22,AFM23}. Macbeath regions have been a key tool in this regard. Given a convex body $K$ and a point $x$ in $K$'s interior, the Macbeath region at $x$, $M_K(x)$, is the largest centrally symmetric body nested within $K$ and centered at $x$ (see Figure~\ref{f:macbeath-cover}(a)). A Macbeath region that has been shrunken by some constant factor $\lambda$ is denoted by $M_K^{\lambda}(x)$. Shrunken Macbeath regions have nice packing and covering properties, and they behave much like metric balls. 

%-----------------------------------------------------------------------
\begin{figure}[htbp]
\centering
\includegraphics[scale=0.8]{fig/macbeath-cover}
\caption{\label{f:macbeath-cover} (a) Macbeath regions and (b) covering $K_0$ by Macbeath regions.}
\end{figure}
%-----------------------------------------------------------------------

A natural way to construct a sandwiching polytope between two nested bodies $K_0$ and $K_1$ is to construct a collection of shrunken Macbeath regions that cover $K_0$ but lie entirely within $K_1$ (see Figure~\ref{f:macbeath-cover}(b)). If done properly, a sandwiching polytope can be constructed by sampling a constant number of points from each of these Macbeath regions, and taking the convex hull of their union. Thus, the number of Macbeath regions provides an upper bound on the number of vertices in the sandwiched polytope.

The ``sandwiching'' perspective described above yields additional new challenges. Consider the two bodies $K_0$ and $K_1$ shown in Figure~\ref{f:rel-fat}, where $K_0$ is a diamond shape nested within the square $K_1$. Consider $1/2$-scaled Macbeath region centered at a point $x$ that lies at the top vertex of $K_0$. Observe that almost all of its volume lies outside of $K_0$. This is problematic because our analysis is based on the number of Macbeath regions needed to cover the boundary of a body, in this case $\bd K_0$. We want a significant amount of the volume of each Macbeath region to lie within $K_0$. In cases like that shown in Figure~\ref{f:rel-fat}, only a tiny fraction of the volume can be charged in this manner against $K_0$. 

%-----------------------------------------------------------------------
\begin{figure}[htbp]
\centering
\includegraphics[scale=0.8]{fig/rel-fat}
\caption{\label{f:rel-fat} Relative fatness.}
\end{figure}
%-----------------------------------------------------------------------

Intuitively, while the body $K_0$ is ``fat'' in a standard sense%
%
\footnote{That is, the largest ball enclosed in $K_0$ and the smallest ball containing $K_0$ differ in size by a constant.},
%
it is not fat ``relative'' to the enclosing body $K_1$. To deal with this inconvenience, we will replace $K_1$ with an intermediate body between $K_0$ and $K_1$ that satisfies this property. In Section~\ref{s:am-hm} we formally define this notion of relative fatness, and we present an intermediate body, called the \emph{harmonic-mean body}, that satisfies this notion of fatness. We will see that this body can be used as a proxy for the sake of approximation.

%=======================================================================
\section{Preliminaries} \label{s:prelim}
%=======================================================================

In this section, we introduce terminology and notation, which will be used throughout the paper. This section can be skipped on first reading (moving directly to Section~\ref{s:hm-fat}).

Let us first recall some standard notation. Given vectors $u, v \in \RE^d$, let $\ang{u,v}$ denote their dot product, and let $\|v\| = \sqrt{\ang{v,v}}$ denote $v$'s Euclidean length. Throughout, we will use the terms \emph{point} and \emph{vector} interchangeably. Given points $p,q \in \RE^d$, let $\|p q\| = \|p - q\|$ denote the Euclidean distance between them. Let $\vol(\cdot)$ and $\area(\cdot)$ denote the $d$-dimensional and $(d-1)$-dimensional Lebesgue measures, respectively.

%-=-=-=-=-=-=-=-=-=-=-=-=-=-=-=-=-=-=-=-=-=-=-=-=-=-=-=-=-=-=-=-=-=-=-=-
\subsection{Polarity and Centrality Properties} \label{s:centrality}
%-=-=-=-=-=-=-=-=-=-=-=-=-=-=-=-=-=-=-=-=-=-=-=-=-=-=-=-=-=-=-=-=-=-=-=-

Given a bounded convex body $K \subseteq \RE^d$ that contains the origin $O$ in its interior, define its \emph{polar}, denoted $K^*$, to be the convex set
\[
	K^*
		~ = ~ \{ u \ST \ang{u,v} \le 1, \hbox{~for all $v \in K$} \}.
\]
The polar enjoys many useful properties (see, e.g., Eggleston~\cite{Egg58}). For example, it is well known that $K^*$ is bounded and $(K^*)^* = K$. Further, if $K_1$ and $K_2$ are two convex bodies both containing the origin such that $K_1 \subseteq K_2$, then $K_2^* \subseteq K_1^*$. 

Given a nonzero vector $v \in \RE^d$, we define its ``polar'' $v^*$ to be the hyperplane that is orthogonal to $v$ and at distance $1/\|v\|$ from the origin, on the same side of the origin as $v$. The polar of a hyperplane is defined as the inverse of this mapping. We may equivalently define $K^*$ as the intersection of the closed halfspaces that contain the origin, bounded by the hyperplanes $v^*$, for all $v \in K$. 

Given a convex body $K \subseteq \RE^d$ and $x \in \interior(K)$, there are many ways to characterize the property that $x$ is ``central'' within $K$~\cite{Gru63, Tot15}. For our purposes, we will make it precise using the concept of Mahler volume. Define $K$'s \emph{Mahler volume}, denoted $\mu(K)$, to be the product $\vol(K) \cdot \vol(K^*)$. The Mahler volume is well studied (see, e.g.~\cite{San49,MeP90,Sch93}). It is invariant under linear transformations, and it depends on the location of the origin within $K$. We say that $K$ is \emph{well-centered} with respect to a point $x \in \interior(K)$ if the Mahler volume $\mu(K-x)$ is at most $O(1)$. When $x$ is not specified, it is understood to be the origin. We have the following lemma~\cite{AFM23,MiP00}.

%-----------------------------------------------------------------------
\begin{lemma}
\label{lem:centroid}
Any convex body $K$ is well-centered with respect to its centroid.
\end{lemma}
%-----------------------------------------------------------------------

Lower bounds on the Mahler volume have also been extensively studied and it is known that the following bound holds irrespective of the location of the origin~\cite{BoM87,Kup08,Naz12}. 

%-----------------------------------------------------------------------
\begin{lemma}
\label{lem:mahler-bounds}
Given a convex body $K \subseteq \RE^d$ whose interior contains the origin, $\mu(K) = \Omega(1)$.
\end{lemma}
%-----------------------------------------------------------------------


%-=-=-=-=-=-=-=-=-=-=-=-=-=-=-=-=-=-=-=-=-=-=-=-=-=-=-=-=-=-=-=-=-=-=-=-
\subsection{Caps, Rays, and Relative Measures} \label{s:cap-prop}
%-=-=-=-=-=-=-=-=-=-=-=-=-=-=-=-=-=-=-=-=-=-=-=-=-=-=-=-=-=-=-=-=-=-=-=-

Consider a compact convex body $K$ in $d$-dimensional space $\RE^d$ with the origin $O$ in its interior. A \emph{cap} $C$ of $K$ is defined to be the nonempty intersection of $K$ with a halfspace. Letting $h_1$ denote a hyperplane that does not pass through the origin, let $\pcap{K}{h_1}$ denote the cap resulting by intersecting $K$ with the halfspace bounded by $h_1$ that does not contain the origin (see Figure~\ref{f:widray}(a)). Define the \emph{base} of $C$, denoted $\base(C)$, to be $h_1 \cap K$. Letting $h_0$ denote a supporting hyperplane for $K$ and $C$ parallel to $h_1$, define an \emph{apex} of $C$ to be any point of $h_0 \cap K$.

%-----------------------------------------------------------------------
\begin{figure}[htbp]
\centering
\includegraphics[scale=0.8]{fig/widray}
\caption{\label{f:widray} Convex body $K$ and polar $K^*$ with definitions used for width and ray.}
\end{figure}
%-----------------------------------------------------------------------

We define the \emph{absolute width} of cap $C$ to be $\dist(h_1,h_0)$. When a cap does not contain the origin, it will be convenient to define distances in relative terms. Define the \emph{relative width} of such a cap $C$, denoted $\width_K(C)$, to be the ratio $\dist(h_1,h_0) / \dist(O,h_0)$ and, to simplify notation, define $\width_K(h_1) = \width_K(\pcap{K}{h_1})$. Observe that as a hyperplane is translated from a supporting hyperplane to the origin, the relative width of its cap ranges from $0$ to a limiting value of $1$.

We also characterize the closeness of a point to the boundary in both absolute and relative terms. Given a point $p_1 \in K$, let $p_0$ denote the point of intersection of the ray $O p_1$ with the boundary of $K$. Define the \emph{absolute ray distance} of $p_1$ to be $\|p_1 p_0\|$, and define the \emph{relative ray distance} of $p_1$, denoted $\ray_K(p_1)$, to be the ratio $\|p_1 p_0\| / \|O p_0\|$. Relative widths and relative ray distances are both affine invariants, and unless otherwise specified, references to widths and ray distances will be understood to be in the relative sense.

We can also define volumes in a manner that is affine invariant. Recall that $\vol(\cdot)$ denotes the standard Lebesgue volume measure. For any region $\Lambda \subseteq K$, define the \emph{relative volume} of $\Lambda$ with respect to $K$, denoted $\vol_K(\Lambda)$, to be $\vol(\Lambda)/\vol(K)$.

With the aid of the polar transformation we can extend the concepts of width and ray distance to objects lying outside of $K$. Consider a hyperplane $h_2$ parallel to $h_1$ that lies beyond the supporting hyperplane $h_0$ (see Figure~\ref{f:widray}(a)). It follows that $h_2^* \in K^*$, and we define $\width_K(h_2) = \ray_{K^*}(h_2^*)$ (see Figure~\ref{f:widray}(b)). Similarly, for a point $p_2 \notin K$ that lies along the ray $O p_1$, it follows that the hyperplane $p_2^*$ intersects $K^*$, and we define $\ray_K(p_2) = \width_{K^*}(p_2^*)$. By properties of the polar transformation, it is easy to see that $\width_K(h_2) = \dist(h_0,h_2) / \dist(O,h_2)$. Similarly, $\ray_K(p_2) = \|p_0 p_2\| / \|O p_2\|$. Henceforth, we will omit references to $K$ when it is clear from context.

Some of our results apply only when we are sufficiently close to the boundary of $K$. Given $\alpha \le \frac{1}{2}$, we say that a cap $C$ is \emph{$\alpha$-shallow} if $\width(C) \le \alpha$, and we say that a point $p$ is \emph{$\alpha$-shallow} if $\ray(p) \le \alpha$. We will simply say \emph{shallow} to mean $\alpha$-shallow, where $\alpha \le \frac{1}{2}$ is a sufficiently small constant.

\arxivonly{
We state some useful technical results on ray distances and cap widths. The missing proofs can be found in~\cite[Section~2.3]{AFM23arxiv}.

%-----------------------------------------------------------------------
\begin{lemma} \label{lem:raydist-width}
Let $C$ be a cap of $K$ that does not contain the origin and let $p$ be a point in $C$. Then $\ray(p) \leq \width(C)$.
\end{lemma}
%-----------------------------------------------------------------------

There are two natural ways to associate a cap with any point $p \in K$. The first is the \emph{minimum volume cap}, which is any cap whose base passes through $p$ of minimum volume among all such caps. For the second, assume that $p \neq O$, and let $p_0$ denote the point of intersection of the ray $O p$ with the boundary of $K$. Let $h_0$ be any supporting hyperplane of $K$ at $p_0$. Take the cap $C$ induced by a hyperplane parallel to $h_0$ passing through $p$. As shown in the following lemma this is the cap of minimum width containing $p$.

%-----------------------------------------------------------------------
\begin{lemma}
\label{lem:min-width-cap}
For any $p \in K \setminus \{O\}$, consider the cap $C$ defined above. Then $\width(C) = \ray(p)$ and further, $C$ has the minimum width over all caps that contain $p$.
 \end{lemma}
%-----------------------------------------------------------------------

The next lemma shows that cap widths behave nicely under containment.

%-----------------------------------------------------------------------
\begin{lemma} \label{lem:cap-containment-width}
Let $C_1$ and $C_2$ be two caps not containing the origin such that $C_1 \subseteq C_2$. Then $\width(C_1) \le \width(C_2)$.
\end{lemma}
%-----------------------------------------------------------------------

%-----------------------------------------------------------------------
\begin{proof}
Consider the point of intersection $p$ of the base of $C_1$ with the ray joining $O$ to $C_1$'s apex. By Lemma~\ref{lem:min-width-cap} and the remarks preceding it, $\ray(p) = \width(C_1)$. Since $C_1 \subseteq C_2$, it follows that $p \in C_2$, and so by Lemma~\ref{lem:raydist-width}, $\ray(p) \le \width(C_2)$. The lemma follows.
\end{proof}
%-----------------------------------------------------------------------

Given any cap $C$ and a real $\lambda > 0$, we define its $\lambda$-expansion, denoted $C^{\lambda}$, to be the cap of $K$ cut by a hyperplane parallel to the base of $C$ such that the absolute width of $C^{\lambda}$ is $\lambda$ times the absolute width of $C$. (Notice that the expansion of a cap may contain the origin, and indeed, if the expansion is large enough, it may be the same as $K$.) An easy consequence of convexity is that, for $\lambda \ge 1$, $C^{\lambda}$ is a subset of the region obtained by scaling $C$ by a factor of $\lambda$ about its apex. This implies the following lemma.

%-----------------------------------------------------------------------
\begin{lemma} \label{lem:cap-exp}
Given any cap $C$ and a real $\lambda \ge 1$, $\vol(C^{\lambda}) \leq \lambda^d \vol(C)$.
\end{lemma}
%-----------------------------------------------------------------------
}%\arxivonly


%-=-=-=-=-=-=-=-=-=-=-=-=-=-=-=-=-=-=-=-=-=-=-=-=-=-=-=-=-=-=-=-=-=-=-=-
\subsection{Macbeath Regions and MNets} \label{s:mac-prop}
%-=-=-=-=-=-=-=-=-=-=-=-=-=-=-=-=-=-=-=-=-=-=-=-=-=-=-=-=-=-=-=-=-=-=-=-

Given a convex body $K$ and a point $x \in K$, and a scaling factor $\lambda > 0$, the \emph{Macbeath region} $M_K^\lambda(x)$ is defined as
\[
    M_K^\lambda(x) ~= ~ x + \lambda ((K - x) \cap (x - K)).
\]
It is easy to see that $M_K^1(x)$ is the intersection of $K$ with the reflection of $K$ around $x$, and so $M_K^1(x)$ is centrally symmetric about $x$. Indeed, it is the largest centrally symmetric body centered at $x$ and contained in $K$. Furthermore, $M_K^\lambda(x)$ is a copy of $M_K^1(x)$ scaled by the factor $\lambda$ about the center $x$ (see Figure~\ref{f:macbeath-cover}(a)). We will omit the subscript $K$ when the convex body is clear from the context. As a convenience, we define $M(x) = M^1(x)$. 

\arxivonly{
We summarize important properties of Macbeath regions and MNets. The missing proofs can be found in~\cite[Section~2.5]{AFM23arxiv} unless indicated otherwise. The first lemma implies that a (shrunken) Macbeath region can act as a proxy for any other (shrunken) Macbeath region overlapping it~\cite{BCP93,ELR70,AFM17a}. 

%-----------------------------------------------------------------------
\begin{lemma} \label{lem:mac-mac}
Let $K$ be a convex body and let $\lambda \le \frac{1}{5}$ be any real. If $x, y \in K$ such that $M^{\lambda}(x) \cap M^{\lambda}(y) \neq \emptyset$, then $M^{\lambda}(y) \subseteq M^{4\lambda}(x)$.
\end{lemma}
%-----------------------------------------------------------------------

The following lemmas are useful in situations when we know that a Macbeath region overlaps a cap of $K$, and allow us to conclude that a constant factor expansion of the cap will fully contain the Macbeath region. The first applies to shrunken Macbeath regions and the second to Macbeath regions with any scaling factor. The proof of the first appears in~\cite{AFM17c} (Lemma~2.5), and the second is an immediate consequence of the definition of Macbeath regions.

%-----------------------------------------------------------------------
\begin{lemma} \label{lem:mac-cap}
Let $K$ be a convex body. Let $C$ be a cap of $K$ and $x$ be a point in $K$ such that $C \cap M^{1/5}(x) \neq \emptyset$. Then $M^{1/5}(x) \subseteq C^2$.
\end{lemma}
%-----------------------------------------------------------------------

%-----------------------------------------------------------------------
\begin{lemma} \label{lem:mac-cap-var}
Let $K$ be a convex body and $\lambda > 0$. If $x$ is a point in a cap $C$ of $K$, then $M^\lambda(x) \cap K \subseteq C^{1+\lambda}$.
\end{lemma}
%-----------------------------------------------------------------------

The next three lemmas relate the volume of caps and associated Macbeath regions. 

%-----------------------------------------------------------------------
\begin{lemma}[B{\'a}r{\'a}ny~\cite{Bar07}] \label{lem:min-vol-cap1}
Given a convex body $K \subseteq \RE^d$, let $C$ be a $\frac{1}{3}$-shallow cap of $K$, and let $p$ be the centroid of $\base(C)$. Then $C \subseteq M^{2d}(p)$.
\end{lemma}
%-----------------------------------------------------------------------

%-----------------------------------------------------------------------
\begin{lemma}
\label{lem:wide-cap}
Let $0 < \beta < 1$ be any constant. Let $K \subseteq \RE^d$ be a well-centered convex body, $p \in K$, and $C$ be the minimum volume cap associated with $p$. If $C$ contains the origin or $\width(C) \ge \beta$, then $\vol_K(M(p)) = \Omega(1)$.
\end{lemma}
%-----------------------------------------------------------------------

%-----------------------------------------------------------------------
\begin{lemma} \label{lem:min-vol-cap2}
Given a convex body $K \subseteq \RE^d$, let $C$ be a $\frac{1}{3}$-shallow cap of $K$, and let $p$ be the centroid of $\base(C)$. Then $\vol(M(p)) = \Theta(\vol(C))$.
\end{lemma}
%-----------------------------------------------------------------------

In the next lemma, we show that the width of the minimum volume cap for $p$ is within a constant factor of the ray distance of $p$.

%-----------------------------------------------------------------------
\begin{lemma} \label{lem:min-vol-cap3} 
Let $K$ be a convex body, $p \in K$, and $C$ be the minimum volume cap associated with $p$. If $C$ is $\frac{1}{3}$-shallow, then $\width(C) \le (2d+1) \cdot \ray(p)$.
\end{lemma}
%-----------------------------------------------------------------------

%-----------------------------------------------------------------------
\begin{proof}
We may assume that $\ray(p) \leq 1/(3(2d+1))$, since otherwise the lemma holds trivially. By a well-known property of minimum volume caps, $p$ is the centroid of the base of $C$~\cite{ELR70}. By Lemma~\ref{lem:min-vol-cap1}, we have $C \subseteq M^{2d}(p)$. By definition, $C \subseteq K$, and so $C \subseteq M^{2d}(p) \cap K$. Applying Lemma~\ref{lem:mac-cap-var} to point $p$ and the minimum width cap $W$ for $p$, we have $M^{2d}(p) \cap K \subseteq W^{2d+1}$. Thus $C \subseteq W^{2d+1}$. By Lemma~\ref{lem:min-width-cap}, $\width(W) = \ray(p)$, and so $\width(W^{2d+1}) = (2d+1) \ray(p)\leq 1/3$. Since $C$ and $W^{2d+1}$ are both $(1/3)$-shallow, and $C \subseteq W^{2d+1}$, it follows from Lemma~\ref{lem:cap-containment-width} that $\width(C) \le \width(W^{2d+1})$. Thus $\width(C) \le (2d+1) \ray(p)$, as desired.
\end{proof}
%-----------------------------------------------------------------------

The next lemma states lower and upper bounds on the relative volume of a Macbeath region based on the width of the associated cap or the ray distance of its center.

%-----------------------------------------------------------------------
\begin{lemma} \label{lem:vol-mac-bounds}
Let $\eps > 0$ be sufficiently small and let $K \subseteq \RE^d$ be a well-centered convex body. Then:
\begin{enumerate}
\item[$(i)$] Let $M$ be a Macbeath region centered at the centroid of the base of a cap $C \subseteq K$ of width $\eps$. Then $\vol_K(M) = O(\eps)$ and  $\vol_K(M) = \Omega(\eps^d)$.
\item[$(ii)$] Let $M$ be a Macbeath region centered at a point $x \in K$ whose ray distance is $\eps$. Then $\vol_K(M) = O(\eps)$ and  $\vol_K(M) = \Omega(\eps^d)$.
\end{enumerate}
\end{lemma}
%-----------------------------------------------------------------------

%-----------------------------------------------------------------------
\begin{proof}
By Lemma~\ref{lem:mac-cap-var}, $M \subseteq C^2$. Also, $C^2 \subseteq S_K$, where $S_K = K \setminus (1-2\eps)K$. Thus
\[
\vol_K(M) \leq \vol_K(C^2) \le \vol_K(S_K) = 1 - (1-2\eps)^d = O(\eps).
\]
Similarly, in part (ii), by considering the cap defined by the supporting hyperplane of $K(1-\eps)$ at $x$, we can show that $\vol_K(M) = O(\eps)$.

Next we show the lower bound on $\vol_K(M)$ in part (i). Let $y$ denote the point $\psi(C) \in K^*$ and let $M'$ denote the Macbeath region $M^{1/5}_{K^*}(y)$. Note that $\ray_{K^*}(y) = \eps$. As the cap $C$ and Macbeath region $M'$ satisfy the conditions of Lemma~\ref{lem:mahler-mac}, we have $\vol_K(C) \cdot \vol_{K^*}(M') = \Omega(\eps^{d+1})$. By the upper bound in (ii), we have $\vol_{K^*}(M') = O(\eps)$ and thus $\vol_K(C) = \Omega(\eps^d)$. Also, by Lemma~\ref{lem:min-vol-cap2}, we have $\vol_K(M) = \Omega(\vol(C)$. Thus $\vol_K(M) = \Omega(\eps^d)$, as desired. Similarly, we can establish the lower bound in part (ii).
\end{proof}
%-----------------------------------------------------------------------
} %arxivonly

The following lemma states that points in a shrunken Macbeath region all have similar ray distances. \confonly{The proof appears in ~\cite[Section~2.5]{AFM23arxiv}.}

%-----------------------------------------------------------------------
\begin{lemma} \label{lem:core-ray}
Let $K$ be a convex body. If $x$ is a $\frac{1}{2}$-shallow point in $K$ and $y \in M^{1/5}(x)$, then $\ray(x)/2 \leq \ray(y) \leq 2 \ray(x)$.
\end{lemma}
%-----------------------------------------------------------------------

The next lemma shows that translated copies of a Macbeath region act as proxies for Macbeath regions in the vicinity.
\confonly{The proof appears in the full version~\cite[Section~3.3]{SoCG23arxiv}.}

%-----------------------------------------------------------------------
\begin{lemma} \label{lem:mac-trans}
Let $\lambda \le 1/2$ and $\gamma \le 1/10$. Let $x$ be a point in a convex body $K$. Let $R = M(x)-x$. Let $y$ be a point in $x + \lambda R$. Then $y + \gamma R \subseteq M^{2\gamma}(y)$.
\end{lemma}
%-----------------------------------------------------------------------

\arxivonly{
%-----------------------------------------------------------------------
\begin{proof}
Treating $x$ as the origin, we have $R \subseteq K$ and $y \in \lambda R$. It follows that $y + (1-\lambda)R \subseteq K$. Recall that the Macbeath region $M(y)$ is the maximal centrally symmetric convex body centered at $y$ and contained within $K$. Thus $y + (1-\lambda) R \subseteq M(y)$. This implies that $y + \gamma R \subseteq M^{\frac{\gamma}{1-\lambda}}(y) \subseteq M^{2\gamma}(y)$.
\end{proof}
%-----------------------------------------------------------------------
} %arxivonly

We employ Macbeath region-based coverings in our polytope approximation scheme. In particular, we employ the concept of MNets, as defined in~\cite{AFM23}. Let $K \subseteq \RE^d$ be a convex body, let $\Lambda$ be an arbitrary subset of $\interior(K)$, and let $c \geq 5$ be any constant. Given $X \subseteq K$, define $\MM_K^{\lambda}(X) = \{ M_K^{\lambda}(x) : x \in X\}$. Define a \emph{$(K, \Lambda,c)$-MNet} to be any maximal set of points $X \subseteq \Lambda$ such that the shrunken Macbeath regions $\MM_K^{1/4c}(X)$ are pairwise disjoint. We refer to $c$ as the expansion factor of the MNet. The following lemma, proved in~\cite{AFM23}, summarizes the key properties of MNets.

%-----------------------------------------------------------------------
\begin{lemma}[\cite{AFM23}] \label{lem:delone}
Given a convex body $K \subseteq \RE^d$, $\Lambda \subset \interior(K)$, and $c \ge 5$, a $(K,\Lambda,c)$-MNet $X$ satisfies the following properties:
\begin{itemize}
    \item (Packing) The elements of $\MM_K^{1/4c}(X)$ are pairwise disjoint.
    \item (Covering) The union of $\MM_K^{1/c}(X)$ covers $\Lambda$.
    \item (Buffering) The union of $\MM_K(X)$ is contained within $K$.
\end{itemize}
\end{lemma}
%-----------------------------------------------------------------------

For the purposes of this paper, $c$ will be any sufficiently large constant, specifically $c \geq 5$. To simplify notation, we use $(K,\Lambda)$-MNet to refer to such an MNet.

As mentioned before, we reduce our polytope approximation problem to that of finding a polytope which is sandwiched between two convex bodies. In turn we tackle this problem using MNets as indicated in the next lemma. 
\confonly{The proof appears in the full version~\cite[Section~3.3]{SoCG23arxiv}.}

%-----------------------------------------------------------------------
\begin{lemma} \label{lem:MNet-approx}
Let $K_0 \subset K_1$ be two convex bodies. Let $X$ be a $(K_1,\partial K_0)$-MNet. Then there exists a polytope $P$ with $O(|X|)$ vertices such that $K_0 \subseteq P \subseteq K_1$.
\end{lemma}
%-----------------------------------------------------------------------

\arxivonly{
%-----------------------------------------------------------------------
\begin{proof}
Define a \emph{half-ellipsoid} to be the intersection of an ellipsoid with an halfspace whose bounding hyperplane passes through its center. Let $c$ be the expansion factor of the MNet $X$. For each Macbeath region $M \in \MM^{1/c}(X)$, choose a net~\cite{Mus22} so that, for a suitable constant $c'$, any half-ellipsoid contained within $M$ of volume at least $c' \vol(M)$ contains at least one point of the net. It follows from standard results~\cite{BEHM89,Har11} that half-ellipsoids have constant VC-dimension, and so the size of the resulting net is $O(1)$. The polytope $P$ is defined to be the convex hull of the points of the nets associated with all the Macbeath regions of $\MM^{1/c}(X)$. 

We claim that $K_0 \subset P \subset K_1$. The second containment follows from the fact that the Macbeath regions of $\MM^{1/c}(X)$ are contained within $K_1$. To prove that $K_0 \subset P$, we will show that our construction chooses a point in every cap of $K_1$ induced by a supporting hyperplane of $K_0$. Towards this end, let $h$ be a supporting hyperplane at point $p \in \bd K_0$, let $H$ be the halfspace bounded by $h$ and not containing $K_0$, and let $C$ be the cap $K_1 \cap H$. Let $M_p = M^{1/4c}(p)$. By the packing property of MNets, there is a point $x \in X$ such that $M_p \cap M_x \neq \emptyset$, where $M_x = M^{1/4c}(x)$. Letting $M'_x = M^{1/c}(x)$ and applying Lemma~\ref{lem:mac-mac}, we have $M_p \subseteq M'_x$ and $M_x \subseteq M'_p$. Thus, $\vol(M_p) = \Omega(\vol(M'_p)) = \Omega(\vol(M_x))$. By John's Theorem~\cite{Joh48}, $M_p$ contains an ellipsoid $E$ centered at $p$, such that $\vol(E) = \Omega(\vol(M_p))$. Putting it together, it follows that the half-ellipsoid $E' = E \cap H$ has volume $\Omega(\vol(M_x))$. Since $\vol(M'_x) = O(\vol(M_x))$, it follows that a point of the net constructed for $M'_x$ is contained in $E'$ (for sufficiently small constant $c'$). Noting that $E' \subseteq C$ completes the proof.
\end{proof}
%-----------------------------------------------------------------------
}%arxivonly

\confonly{
The following lemma bounds the sizes of MNets in important special cases involving points at roughly the same ray distance. These bounds will be useful in obtaining our volume-sensitive bounds. The proof appears in the full version~\cite[Section~4]{SoCG23arxiv}.

%-----------------------------------------------------------------------
\begin{lemma} \label{lem:fixed-ray}
Let $0 < \eps \leq 1/2$ be sufficiently small and let $K \subseteq \RE^d$ be a well-centered convex body. Let $\Lambda$ be the points of $K$ at ray distances between $\eps$ and $2\eps$, and let $X$ be a $(K,\Lambda)$-MNet. Then:
\begin{enumerate}
    \item[$(i)$] $|X| = O(1/\eps^{(d-1)/2})$.
    \item[$(ii)$] For any positive real $f \le 1$, let $X_f \subseteq K$ be such that the total relative volume of the Macbeath regions of $\MM^{1/4c}(X_f)$ is $O(f\eps)$. Then $|X_f| = O(\sqrt{f}/\eps^{(d-1)/2})$.
\end{enumerate}
\end{lemma} 
%-----------------------------------------------------------------------
}%confonly

%-----------------------------------------------------------------------
\subsection{Concepts from Projective Geometry} \label{s:projective}
%-----------------------------------------------------------------------

In this section we present some relevant standard concepts from projective geometry. For further details see any standard reference (e.g., \cite{Ric11}). Given four collinear points, $a,b,c,d$ (not necessarily in this order), the \emph{cross ratio} $(a,b;c,d)$ is defined to be $(\|ac\| / \|ad\|) / (\|bc\| / \|bd\|)$, where these are understood to be signed distances determined by the orientations of the segments along the line. We follow the convention of using symbols $a,b,c,d,\ldots$ for points, and the distinction from other uses (such as $d$ for the dimension) should be clear from the context.

It is well known that cross ratios are preserved under projective transformations. If the cross ratio $(a,b;c,d)$ is $-1$, we say that this quadruple of points forms a \emph{harmonic bundle} (see Figure~\ref{f:harmonic-bundle}). This is an important special case which occurs frequently in constructions. In this case, the points lie on the line in the order of $a, d, b, c$ and the ratio in which $a$ divides $c$ and $d$ externally (i.e., $\|ac\| / \|ad\|$) is the same as the ratio in which $b$ divides $c$ and $d$ internally (i.e., $\|bc\| / \|bd\|$). The sign is negative since $bc$ and $bd$ have opposite directions. If the point $a$ is at infinity, the cross ratio degenerates to $\|bd\|/\|bc\|$, implying that $b$ is midway between $c$ and $d$.

%-----------------------------------------------------------------------
\begin{figure}[htbp]
\centering
\includegraphics[scale=0.8]{fig/harmonic-bundle}
\caption{Harmonic bundle (from the quadrilateral construction~\cite{Ric11}).} \label{f:harmonic-bundle}
\end{figure}
%-----------------------------------------------------------------------

\arxivonly{
Given a convex body $K$, it induces a cross ratio and a distance metric, called the \emph{Hilbert distance}, for any two points in its interior. This notion was introduced by David Hilbert as a generalization of Cayley's formula for distance in the Cayley-Klein model of hyperbolic geometry~\cite{Hil95}. Given two points $x,y$ in the interior of $K$, suppose that the line passing through $x$ and $y$ intersects $\bd K$ at points $a$ and $b$ such that $a,x,y,b$ appear in this order on the line. Then the Hilbert distance between $x$ and $y$, induced by $K$, is defined as $d_K(x,y) = \inv{2} \ln (x,y;b,a)$. As a convenience, we define the cross ratio between $x$ and $y$, induced by $K$, to be $r_K(x,y) = -(a,y;b,x)$. It is easy to verify that $(x,y;b,a) \geq 1$. By standard results, $(a,y;b,x) = 1 - (x,y;b,a)$, which implies that $r_K(x,y) =  (x,y;b,a) - 1$. Thus $r_K(x,y) \geq 0$. Note also that the Hilbert distance can be equivalently expressed as $d_K(x,y) = \ln (1+r_K(x,y)) / 2$.

Let $B_H(x,r)$ denote a Hilbert ball of radius $r$ centered at $x$. An important property of Macbeath regions is that they can act as proxies to Hilbert balls~\cite{AbM18,VeW16}.

%-----------------------------------------------------------------------
\begin{lemma} \label{lem:nesting}
Let $K$ be a convex body and let $x$ be a point in $K$. For any $0 < \lambda < 1$,
\[
    B_H\left(x, \frac{1}{2} \ln(1+\lambda)\right) 
        ~\subseteq~ M^{\lambda}(x) 
        ~\subseteq~ B_H\left(x, \frac{1}{2} \ln\frac{1+\lambda}{1-\lambda}\right).
\]
\end{lemma}
%-----------------------------------------------------------------------

Recalling that the Hilbert distance $d_K(x,y) = \ln (1+r_K(x,y)) / 2$, we have the following corollary from the first containment in the statement of the lemma. 

%-----------------------------------------------------------------------
\begin{corollary} \label{cor:nesting}
Let $K$ be a convex body and let $x$ be a point in $K$. For any $0 < \lambda < 1$, any point $y \in K$ that is not contained in the interior of $M^{\lambda}(x)$ satisfies $r_K(x,y) \ge \lambda$.
\end{corollary}
%-----------------------------------------------------------------------
}%\arxivonly

%-----------------------------------------------------------------------
\subsection{Intermediate Bodies}\label{s:am-hm}
%-----------------------------------------------------------------------

In this section we explore the concept of relative fatness, which was introduced in Section~\ref{s:techniques}. Given two convex bodies $K_0$ and $K_1$ such that $K_0 \subset K_1$ and $0 < \gamma < 1$, we say that $K_0$ is \emph{relatively $\gamma$-fat} with respect to $K_1$ if, for any point $p \in \bd K_0$, and any scaling factor $0 < \lambda \leq 1$, at least a constant fraction $\gamma$ of the volume of the Macbeath region $M = M_{K_1}^{\lambda}(p)$ lies within $K_0$, that is, $\vol(M \cap K_0)/\vol(M) \geq \gamma$. 
%
We say that $K_0$ is \emph{relatively fat} with respect to $K_1$ if it is relatively $\gamma$-fat for some constant $\gamma$. Relative fatness will play an important role in our analyses. Since an arbitrary nested pair $K_0 \subset K_1$ may not necessarily satisfy this property, it will be useful to define an intermediate body sandwiched between $K_0$ and $K_1$ that does.

There are a few natural ways to define such an intermediate body. Given two convex bodies $K_0$ and $K_1$, where $K_0 \subseteq K_1$, the \emph{arithmetic-mean body}, $K_A(K_0, K_1)$, is defined to be the convex body $\frac{1}{2}(K_0 \oplus K_1)$, where ``$\oplus$'' denotes Minkowski sum. Equivalently, for any unit vector $u$ consider the two supporting halfspaces of $K_0$ and $K_1$ orthogonal to $u$, and take the halfspace that is midway between the two. The arithmetic-mean body is obtained by intersecting such halfspaces for all unit vectors $u$. 

%-----------------------------------------------------------------------
\begin{figure}[htbp]
\centering
\includegraphics[scale=0.8]{fig/arihar}
\caption{\label{f:arihar} (a) Arithmetic and (b) harmonic-mean bodies.}
\end{figure}
%-----------------------------------------------------------------------

Another natural choice arises from a polar viewpoint. Assume that $K_0 \subset K_1$ and the origin $O \in \interior(K_0)$. The \emph{harmonic-mean body}, $K_H(K_0, K_1)$, was introduced by Firey~\cite{Fir61} and is defined as follows. For any ray $r$ from the origin $O$, let $b_r$ and $d_r$ denote the points of intersection of $r$ with $\bd K_0$ and $\bd K_1$, respectively (see Figure~\ref{f:arihar}(b)). Let $c_r$ be the point on the ray such that $1/\|O c_r\| = (1/\|O b_r\| + 1/\|O d_r\|)/2$. Equivalently, the cross ratio $(O, c_r; d_r, b_r)$ equals $-1$, that is, this quadruple forms a harmonic bundle. Clearly, $c_r$ lies between $b_r$ and $d_r$, and hence the union of these points over all rays $r$ defines the boundary of a body that is sandwiched between $K_0$ and $K_1$. This body is the harmonic-mean body. By considering the supporting hyperplanes orthogonal to the ray $r$, it is easy to see that the arithmetic-mean body of $K_0$ and $K_1$ is mapped to the harmonic-mean body of $K_0^*$ and $K_1^*$ under polarity, that is, $(K_A(K_0, K_1))^* = K_H(K_0^*, K_1^*)$. Therefore, $K_H(K_0, K_1)$ is convex. When $K_0$ and $K_1$ are clear from context, we will just write $K_A$ and $K_H$, omitting references to their arguments. 

In order to understand why these intermediate bodies are useful to us, recall the diamond and square bodies $K_0$ and $K_1$ from Figure~\ref{f:rel-fat} (see Figure~\ref{f:rel-fat-har}(a)). Recall the issue that a large fraction of the volume of the Macbeath region $M^{1/2}_{K_1}(x)$ lies outside of $K_0$. If we replace $K_1$ with $K_H = K_H(K_0, K_1)$ and compute the Macbeath region with respect to $K_H$ instead (see Figure~\ref{f:rel-fat-har}(b) and (c)), we see that a constant fraction of the volume of the Macbeath region lies within $K_0$ and so relative fatness is satisfied.

%-----------------------------------------------------------------------
\begin{figure}[htbp]
\centering
\includegraphics[scale=0.8]{fig/rel-fat-har}
\caption{\label{f:rel-fat-har} Relative fatness of $K_H$.}
\end{figure}
%-----------------------------------------------------------------------

In Section~\ref{s:hm-fat}, we will present an important result by showing that the inner body $K_0$ is relatively fat with respect to the harmonic-mean body $K_H(K_0, K_1)$. The proof makes heavy use of concepts from projective geometry, such as the harmonic bundle. This fact will be critical to establishing the volume-sensitive bounds given in this paper.

\arxivonly{
%-=-=-=-=-=-=-=-=-=-=-=-=-=-=-=-=-=-=-=-=-=-=-=-=-=-=-=-=-=-=-=-=-=-=-=-
\section{Bounding MNet Sizes}
\label{a:mnet-sizes}
%-=-=-=-=-=-=-=-=-=-=-=-=-=-=-=-=-=-=-=-=-=-=-=-=-=-=-=-=-=-=-=-=-=-=-=-

In this section, we bound the sizes of MNets in important special cases involving points at roughly the same ray distance. These bounds will be useful in obtaining our volume-sensitive bounds. We begin by recalling definitions and technical tools from~\cite{AFM23arxiv}. We say that two caps $C_1$ and $C_2$ are \emph{$\lambda$-similar} for $\lambda \ge 1$, if $C_1 \subseteq C_2^{\lambda}$ and $C_2 \subseteq C_1^{\lambda}$. If two caps are $\lambda$-similar for constant $\lambda$, we say that the caps are \emph{similar}. Note that this is an affine-invariant notion of closeness between caps. 

Arya {\etal}~\cite[Section~2.6 and Section~3.2]{AFM23arxiv} showed certain important relationships between caps in $K$ and associated Macbeath regions in $K^*$. In order to state their result, consider the following mapping. Consider a  point $z \in K^*$. Let $\hat{z} \not\in K^*$ be the point on the ray $Oz$ such that $\ray(\hat{z}) = \eps$. The dual hyperplane $\hat{z}^*$ intersects $K$, and so induces a cap, which we call $z$'s \emph{$\eps$-representative cap}. They showed that points lying within the same shrunken Macbeath regions have similar representative caps, which implies Lemma~\ref{lem:mahler-mac}(i). Further, by extending and generalizing results in \cite{AAFM22,AFM12b,NNR20}, they established a Mahler-type reciprocal relationship between the volume of caps in $K$ and the associated Macbeath regions in $K^*$. This is stated in Lemma~\ref{lem:mahler-mac}(ii).

%-----------------------------------------------------------------------
\begin{lemma}[\cite{AFM23arxiv}] \label{lem:mahler-mac}
Let $0 < \eps \leq \frac{1}{16}$ and let $K \subseteq \RE^d$ be a well-centered convex body. Let $C$ be a cap of $K$ such that $\eps/2 \leq \width(C) \leq 2\eps$. Suppose that the ray shot from the origin orthogonal to the base of $C$ intersects a Macbeath region $M = M^{1/5}(y)$ of $K^*$, where $\ray(y) = \eps$ (see Figure~\ref{f:sandwich}(b)). Then:
\begin{enumerate}
\item[$(i)$] The cap $C$ and the $\eps$-representative cap of any point $z \in M$ are 16-similar.
\item[$(ii)$] $\vol_K(C) \cdot \vol_{K^*}(M) = \Omega(\eps^{d+1})$.
\end{enumerate}
\end{lemma}
%-----------------------------------------------------------------------

%-----------------------------------------------------------------------
\begin{figure}[htbp]
\centering
\includegraphics[scale=0.8,page=3]{fig/sandwich}
\caption{\label{f:sandwich} Statement of Lemma~\ref{lem:mahler-mac}(i).}
\end{figure}
%-----------------------------------------------------------------------

Next we present a general tool which will be useful in bounding the sizes of the MNets of interest to us.  Let $K \subseteq \RE^d$ be a well-centered convex body. For any shallow cap $C$ of $K$, define a point $\psi(C)$ in $K^*$ as follows. In the polar space, consider the ray shot from $O$ orthogonal to the base of $C$. We let $\psi(C) \in K^*$ be the point on this ray with ray distance $\width(C)$.

Let $\CC$ be a set of shallow caps of $K$, let $\Lambda \subseteq K$ denote the set of centroids of the bases of the caps of $\CC$, and let $\Lambda' = \{\psi(C) : C \in \CC\}$. Let $X$ be a $(K,\Lambda)$-MNet, and let $Y$ be a $(K^*,\Lambda')$-MNet. For each $x \in \Lambda$, let $C_x$ denote a cap of $\CC$ such that $x$ is the centroid of its base. (Clearly, such a cap exists. If there is more than one, then we choose one arbitrarily.) Also, for each $x \in X$, define $M_x = M_K^{1/4c}(x)$, where $c$ is the expansion factor of the MNets. Similarly, for $y \in Y$, define $M_y = M_{K^*}^{1/4c}(y)$. The following lemma shows that it is possible to construct a bipartite graph $(X,Y)$ with certain properties. 

%-----------------------------------------------------------------------
\begin{lemma} \label{lem:bipartite} 
Given a well-centered convex body $K \subseteq \RE^d$, and the entities $\CC, \Lambda, \Lambda', X, Y$ as defined above, there is a bipartite graph $(X,Y)$ such that there is exactly one edge incident to each vertex of $X$ and the degree of each vertex of $Y$ is $O(1)$. Furthermore, for any $x \in X$ and $y \in Y$, if there is an edge $(x,y)$, then $\vol_K(M_x) \cdot \vol_{K^*}(M_y) = \Omega(\delta^{d+1})$, where $\delta = \width(C_x)$. 
\end{lemma}
%-----------------------------------------------------------------------

%-----------------------------------------------------------------------
\begin{proof}
First we show how to construct the bipartite graph $(X,Y)$. Let $x$ be any point of $X$ and let $y' = \psi(C_x)$. By the covering property of MNets, there exists $y \in Y$ such that $M^{1/c}(y)$ contains $y'$. We add an edge in the bipartite graph between $x$ and $y$. It follows from our construction that there is exactly one edge incident to each vertex of $X$.

Next we show that if there is an edge $(x,y)$, then $\vol_K(M_x) \cdot \vol_{K^*}(M_y) = \Omega(\delta^{d+1})$. By definition, $\ray(y') = \width(C_x) = \delta$. Letting $\eps = \ray(y)$ and applying Lemma~\ref{lem:core-ray}, we have $\eps/2 \leq \ray(y') \leq 2 \eps$. Thus $\eps/2 \leq \width(C_x) \leq 2 \eps$. Observe that the cap $C_x$ and the Macbeath region $M^{1/5}(y)$ satisfy the conditions of Lemma~\ref{lem:mahler-mac}. Recalling that $c$ is a constant $\geq 5$,  and $M_y$ and $M^{1/c}(y)$ differ by a constant scaling factor, by part (ii) of this lemma, we have $\vol_K(C_x) \cdot \vol_{K^*}(M_y) = \Omega(\eps^{d+1})$. Also, by Lemma~\ref{lem:min-vol-cap2}, $\vol(M_x) = \Omega(\vol(C_x))$. Thus  $\vol_K(M_x) \cdot \vol_{K^*}(M_y) = \Omega(\eps^{d+1})$.

It remains to prove that the degree of each vertex of $Y$ is $O(1)$. Let $y$ be any vertex of $Y$ and let $\eps = \ray(y)$. For any edge $(x,y)$, we showed above that the cap $C_x$ and the Macbeath region $M^{1/5}(y)$ satisfy the conditions of Lemma~\ref{lem:mahler-mac}. By part (i) of this lemma, it follows that the cap $C_x$ and the $\eps$-representative cap of $y$ are $16$-similar. 

Letting $C_y$ denote the $\eps$-representative cap of $y$, we have $C_x \subseteq C_y^{16}$ and $C_y \subseteq C_x^{16}$. Applying Lemma~\ref{lem:cap-exp}, we have $\vol(C_x) = \Omega(\vol(C_x^{16})) = \Omega(\vol(C_y))$, and by Lemma~\ref{lem:min-vol-cap2}, we have $\vol(M_x) = \Omega(\vol(C_x))$. Thus, $\vol(M_x) = \Omega(\vol(C_y))$. Recall that half of the Macbeath region $M_x$ lies inside $C_x$, and hence inside $C_y^{16}$. By Lemma~\ref{lem:cap-exp}, $\vol(C_y^{16}) = O(\vol(C_y))$. Since the Macbeath regions of $\MM^{1/4c}(X)$ are disjoint, a straightforward packing argument implies that $y$ has $O(1)$ neighbors.
\end{proof}
%-----------------------------------------------------------------------

Expressing the total number of edges in the graph as the sum of the degrees of the vertices of $Y$, we see this quantity is $O(|Y|)$. The following corollary is immediate.

%-----------------------------------------------------------------------
\begin{corollary}
\label{cor:bipartite} 
Given a convex body $K \subseteq \RE^d$, and the entities $\CC, \Lambda, \Lambda', X, Y$ as defined above, then $|X| = O(|Y|)$. 
\end{corollary}
%-----------------------------------------------------------------------

We are now ready to bound the sizes of MNets in the important special case involving caps of roughly the same width, which map in the polar to points at roughly the same ray distance.  Lemmas~\ref{lem:fixed-width} and \ref{lem:fixed-ray} bound the sizes of MNets in these cases. We also bound the cardinality of important subsets that arise in our applications. 

%-----------------------------------------------------------------------
\begin{lemma} \label{lem:fixed-width}
Let $0 < \eps \leq 1/2$ and let $K \subseteq \RE^d$ be a well-centered convex body. Let $\CC$ be the set of caps of $K$ of width between $\eps$ and $2\eps$, let $\Lambda \subseteq K$ denote the set of centroids of the bases of the caps of $\CC$, and let $X$ be a $(K,\Lambda)$-MNet. Then:
\begin{enumerate}
\item[$(i)$] $|X| = O(1/\eps^{(d-1)/2})$.
\item[$(ii)$] For any positive real $f \le 1$, let $X_f \subseteq X$ be such that the total relative volume of the Macbeath regions of $\MM^{1/4c}(X_f)$ is $O(f \eps)$. Then $|X_f|$ is $O(\sqrt{f} / \eps^{(d-1)/2})$.
%
\end{enumerate}
\end{lemma} 
%-----------------------------------------------------------------------

Note that if $f$ is $o(\eps^{d-1})$, then $\sqrt{f}/\eps^{(d-1)/2}$ is $o(1)$ and so $X_f = \emptyset$.

%-----------------------------------------------------------------------
\begin{proof}
If $\eps > \eps_0$, where $\eps_0$ is any constant, we can show that $|X| = O(1)$ as follows. Associate a minimum volume cap $C_x$ with each point $x \in X$. Recall that $x$ is the centroid of the base of $C_x$. Let $M_x = M^{1/4c}(x)$. By Lemma~\ref{lem:wide-cap}, if the width of $C_x$ exceeds a constant,  then $\vol_K(M_x) = \Omega(1)$. Thus, the number of points $x \in X$ such that $\width(C_x) > \eps_0$ is at most $O(1)$ and the lemma follows trivially. 

In the remainder of the proof, we will assume that $\eps \leq \eps_0$, where $\eps_0$ is a sufficiently small constant. Let $\Lambda'$ be the points in $K^*$ at ray distances between $\eps$ and $2\eps$, and let $Y$ be a $(K^*,\Lambda')$-MNet. Note that $\Lambda' = \{\psi(C) : C \in \CC\}$, where $\psi$ is as defined above. Thus the entities $\CC, \Lambda, \Lambda', X, Y$ satisfy the preconditions of Lemma~\ref{lem:bipartite}.

Arguing as in Lemma~\ref{lem:vol-mac-bounds}, we can show that all the Macbeath regions of $\MM^{1/4c}(X)$ lie in the shell $S_K = K \setminus (1-4\eps)K$, all the Macbeath regions of $\MM^{1/4c}(Y)$ lie in the shell $S_{K*} = K^* \setminus (1-4\eps)K^*$, $\vol_K(S_K) = O(\eps)$ and $\vol_{K^*}(S_{K^*}) = O(\eps)$.

Define the \emph{fractional volume} of a Macbeath region $M \in \MM^{1/4c}(X)$, denoted $\vol_f(M)$, to be $\vol(M) / \vol(S_K)$. Similarly, for $M \in \MM^{1/4c}(Y)$, define $\vol_f(M) = \vol(M) / \vol(S_{K^*})$. Consider the bipartite graph with vertex sets $X$ and $Y$ described in Lemma~\ref{lem:bipartite}. Recall that there is exactly one edge incident to each vertex of $X$ and the degree of each vertex of $Y$ is $O(1)$.  Further, if there is an edge $(x,y)$, then $\vol_K(M_x) \cdot \vol_{K^*}(M_y) = \Omega(\eps^{d+1})$. Thus
\[
\vol_f(M_x) \cdot \vol_f(M_y) = 
\Omega\left(\frac{\vol_K(M_x)}{\vol_K(S_K)} \cdot \frac{\vol_{K^*}(M_y)}{\vol_{K^*}(S_{K^*})} \right) =
\Omega\left(\frac{\eps^{d+1}}{\eps \cdot \eps} \right) =
\Omega\left(\eps^{d-1}\right).
\]
It follows that the quantity $\vol_f(M_x) + \vol_f(M_y)$ is $\Omega(\eps^{(d-1)/2})$ for any edge $(x,y)$. Summing this quantity over all the edges in the graph, we obtain a lower bound of $\Omega(|X| \, \eps^{(d-1)/2})$. To upper bound this quantity, note that by disjointness, $\sum_{x \in X} \vol_f(M_x) = O(1)$, $\sum_{y \in Y} \vol_f(M_y) = O(1)$, and the degree of each vertex is $O(1)$. Thus, the sum of this quantity over all the edges is $O(1)$. The lower and upper bounds together imply that $|X| = O(1/\eps^{(d-1)/2})$. 

The proof of (ii) is similar to (i). (In fact, (i) is a special case of (ii) for $f=1$.) By Lemma~\ref{lem:vol-mac-bounds}(i), the relative volume of any Macbeath region of $\MM^{1/4c}(X)$ is $\Omega(\eps^d)$. It follows that if $f = o(\eps^{d-1})$ then $X_f = \emptyset$ and so (ii) holds. We may therefore assume that $f = \Omega(\eps^{d-1})$. Letting $S'_K \subseteq S_K$ denote the union of the Macbeath regions of $\MM^{1/4c}(X_f)$, we are given that $\vol_K(S'_K) = O(f \eps)$. We modify the definition of fractional volume of a Macbeath region $M \in \MM^{1/4c}(X_f)$, denoted $\vol_f(M)$, to be $\vol(M) / \vol(S'_K)$. Note that we keep the same definition of fractional volume for the Macbeath regions of $\MM^{1/4c}(Y)$, that is, for $M \in \MM^{1/4c}(Y)$, $\vol_f(M) = \vol(M) / \vol(S_{K^*})$. Arguing as in (i), but using the bound $\vol_K(S'_K) = O(f \eps)$ in place of  $\vol_K(S_K) = O(\eps)$, it follows that for any edge $(x,y)$ such that $x \in X_f$ and $y \in Y$, we have
\[
\vol_f(M_x) \cdot \vol_f(M_y) = \Omega\left(\frac{\eps^{d-1}}{f} \right).
\]
Thus $\vol_f(M_x) + \vol_f(M_y) = \Omega(\eps^{(d-1)/2}/\sqrt{f})$ for any such edge $(x,y)$. As in (i), summing this quantity over all the edges incident to the vertices of $X_f$, we obtain a lower bound of $\Omega(|X_f| \, \eps^{(d-1)/2}/\sqrt{f})$, and an upper bound of $O(1)$. The lower and upper bounds together imply that $|X_f| = O(\sqrt{f}/\eps^{(d-1)/2})$, as desired.
\end{proof}
%-----------------------------------------------------------------------

The following lemma is analogous to Lemma~\ref{lem:fixed-width}, but for points at similar ray distances. We will use this lemma in Section~\ref{s:hausdorff} together with the relative fatness properties of the harmonic-mean body to establish our volume-sensitive bound. 

%-----------------------------------------------------------------------
\begin{lemma} \label{lem:fixed-ray}
Let $0 < \eps \leq 1/2$ be sufficiently small and let $K \subseteq \RE^d$ be a well-centered convex body. Let $\Lambda$ be the points of $K$ at ray distances between $\eps$ and $2\eps$, and let $X$ be a $(K,\Lambda)$-MNet. Then:
\begin{enumerate}
    \item[$(i)$] $|X| = O(1/\eps^{(d-1)/2})$.
    \item[$(ii)$] For any positive real $f \le 1$, let $X_f \subseteq K$ be such that the total relative volume of the Macbeath regions of $\MM^{1/4c}(X_f)$ is $O(f\eps)$. Then $|X_f| = O(\sqrt{f}/\eps^{(d-1)/2})$.
\end{enumerate}
\end{lemma} 
%-----------------------------------------------------------------------

Note that if $f$ is $o(\eps^{d-1})$, then $\sqrt{f}/\eps^{(d-1)/2}$ is $o(1)$ and so $X_f = \emptyset$.

%-----------------------------------------------------------------------
\begin{proof}
We associate a minimum volume cap $C_x$ with each point $x \in X$. Recall that $x$ is the centroid of the base of $C_x$. Let $M_x = M^{1/4c}(x)$. By Lemma~\ref{lem:wide-cap}, if the width of $C_x$ exceeds a constant, say $1/3$, then $\vol_K(M_x) = \Omega(1)$. Thus, the number of points $x \in X$ such that $\width(C_x) > 1/3$ is at most $O(1)$. Next we bound the remaining points of $X$.

Since $\eps \leq \ray(x) \leq 2\eps$, it follows from Lemmas~\ref{lem:raydist-width} and \ref{lem:min-vol-cap3} that $\eps \leq \width(C_x) \leq 2(2d+1)\eps$. Let $\eps_i = 2^i \eps$. We partition the remaining points of $X$ into $O(\log d)$ groups, where the points in group $i$ have associated minimum volume caps whose widths lie between $\eps_i$ and $2 \eps_i$. By Lemma~\ref{lem:fixed-width}(i), the number of points in group $i$ is $O(1/\eps_i^{(d-1)/2})$. Summing over all groups $i$, it follows that $|X| = O(1/\eps^{(d-1)/2})$, which proves (i).

The proof of (ii) is similar. By Lemma~\ref{lem:vol-mac-bounds}(ii), the relative volume of any Macbeath region of $M^{1/4c}(X)$ is $\Omega(\eps^d)$. It follows that if $f = o(\eps^{d-1})$ then $X_f = \emptyset$ and so (ii) holds. We may therefore assume that $f = \Omega(\eps^{d-1})$. Arguing as in (i), we can show that the number of points $x \in X_f$ such that $\width(C_x) > 1/3$ is $O(1)$. We partition the remaining points into $O(\log d)$ groups as before. Applying Lemma~\ref{lem:fixed-width}(ii) to each group, and summing the result proves (ii).
\end{proof}
%-----------------------------------------------------------------------
}%\arxivonly


%-=-=-=-=-=-=-=-=-=-=-=-=-=-=-=-=-=-=-=-=-=-=-=-=-=-=-=-=-=-=-=-=-=-=-=-
\section{Relative Fatness and the Harmonic-Mean Body} \label{s:hm-fat}
%-=-=-=-=-=-=-=-=-=-=-=-=-=-=-=-=-=-=-=-=-=-=-=-=-=-=-=-=-=-=-=-=-=-=-=-

In this section, we establish properties of the harmonic-mean body that are critical to the main results of this paper. In particular, given two bodies $K_0 \subset K_1$, we show that $K_0$ is relatively fat with respect to $K_H$. In fact, we present a stronger result in Lemma~\ref{lem:HM-fat-main}, which implies relative fatness as an immediate consequence. We will employ this stronger result in Section~\ref{s:hausdorff} to obtain our volume-sensitive bounds for polytope approximation.

The proof of Lemma~\ref{lem:HM-fat-main} is based on the following technical lemma. For constant $\lambda$, it implies that for any point $b \in K_0$ that is not too close to the boundary of $K_0$, the Macbeath regions centered at $b$ with respect to $K_0$ and $K_H$, respectively, are roughly similar up to a constant scaling factor. This is formally stated in the corollary following the lemma. 

%-----------------------------------------------------------------------
\begin{lemma} 
\label{lem:HM-fat-aux}
Let $0 < \lambda < 1$ be a parameter. Let $K_0 \subset K_1$ be two convex bodies, where the origin $O$ lies in the interior of $K_0$. Let $K_H$ denote the harmonic-mean body of $K_0$ and $K_1$. Consider any ray emanating from the origin $O$. Let $c$ and $d$ denote the points of intersection of this ray with $\bd K_0$ and $\bd K_1$, respectively (see figure). Let $b \in K_0$ be a point on this ray such that the cross ratio $(O,c;d,b) \leq -\lambda$. Consider any line passing through $b$. Let $c'$ and $c''$ denote the points of intersection of this line with $\bd K_H$. Then
\[
    \min(\|b c' \cap K_0\| , \|b c'' \cap K_0\|) 
        ~ \geq ~ s(\lambda) \cdot \min(\|b c'\|, \|b c''\|), \qquad\text{where $s(\lambda) = \lambda / 6$}.
\]
\end{lemma}
%-----------------------------------------------------------------------

%-----------------------------------------------------------------------
\begin{figure}[htbp]
\centering
\includegraphics[scale=0.8]{fig/harmonic-mean}
\caption{Lemma~\ref{lem:HM-fat-aux} and its proof.} \label{f:harmonic-mean-aux}
\end{figure}
%-----------------------------------------------------------------------


%-----------------------------------------------------------------------
\begin{proof}
\confonly{We sketch the key ideas and present the full proof in~\cite{SoCG23arxiv}.}
Consider the two dimensional flat that contains the origin and the line $\ell$ that passes through the points $c'$, $b$, and $c''$. Henceforth, let $K_0, K_1, K_H$ refer to the two dimensional convex bodies obtained by intersecting the respective bodies with this flat. Let $b'$ and $d'$ denote the points of intersection of the ray $O c'$ with $\bd K_0$ and $\bd K_1$, respectively, and define $b''$ and $d''$ analogously for $O c''$. All these points lie on the flat, and it follows from the definition of the harmonic-mean body that $(O,c';d',b') = (O,c'';d'',b'') = -1$ (see Figure~\ref{f:harmonic-mean-aux}(a)). 

By rotating space, we may assume that $\ell$ is horizontal and above the origin. Through an infinitesimal perturbation, we may assume that there is a supporting line for $K_1$ at $d$ that is not parallel to $\ell$. Without loss of generality, we may assume that it intersects $\ell$ to the left of $b$. Since $c'$ and $c''$ are symmetrical in the statement of the lemma, we may assume that $c'$ lies to left of $b$ and $c''$ lies to its right. Let $f$ denote the intersection point of the line $d d'$ with $\ell$ (see Figure~\ref{f:harmonic-mean-aux}(a)). Clearly, the left-to-right order of points along $\ell$ is $\ang{f, c', b, c''}$. Observe that the points $c$, $d$, $d'$, and $d''$ all lie strictly above $\ell$, and the points $b'$ and $b''$ lie strictly below. 

Let $e'$ denote the point of intersection of the segment $c b'$ with segment $b c'$, and define $e''$ analogously for segment $c b''$. Since $c$, $b'$ and $b''$ all lie on $\bd K_0$, by convexity, $e'$ and $e''$ are contained in $K_0$. Thus, to prove the lemma, it suffices to show that 
\begin{equation} \label{eq:hm-fat-1}
    \min(\|b e'\|, \|b e''\|) 
        ~ \geq ~ s(\lambda) \cdot \min(\|b c'\|, \|b c''\|).
\end{equation}

We begin by proving bounds on two cross ratios:
\begin{enumerate}
    \item [$(i)$] $-(f,e'; c',b) \geq \lambda/2$, and
    \item[$(ii)$] $-(f,e''; c'',b) \geq \lambda/2$.
\end{enumerate}
Because projective transformations preserve cross ratios, it will be convenient to prove these bounds after first applying a projective transformation. In particular, this transformation maps $O$ and $f$ to infinity so that lines through $O$ map to vertical lines and lines through $f$ map to horizontal lines (see Figure~\ref{f:harmonic-mean-aux}(b)). 
After this transformation, $O c'$, $O c$, and $O c''$ are vertical and directed upwards and $d' d$ and $c' b$ are horizontal and directed to the right. Clearly, $\|c' d'\| = \|b d\|$. Since $d''$ lies above $\ell$ and below the line $d' d$ we have $\|c'' d''\| \leq \|b d\|$. By definition of $b$, we have $(O,c; d,b) = -1/(\|c d\| / \|c b\|) \leq -\lambda$. Since $\|c b\| + \|c d\| = \|b d\|$, we have $\|c b\| \geq \|b d\| \lambda/(1+\lambda)$.

Given that $f$ is at infinity, the above cross ratios reduce to simple ratios. Thus, it suffices to show:
\begin{enumerate}
    \item [$(i)$] $\|e' b\|/\|e' c'\| \geq \lambda/2$, and
    \item[$(ii)$] $\|e'' b\|/\|e'' c''\| \geq \lambda/2$.
\end{enumerate}

To show~(i), observe that since $(O, c'; d', b') = -1$ and since $O$ is at infinity and $c'$ lies between $b'$ and $d'$, this is equivalent to $1/(\|c' d'\| / \|c' b'\|) = 1$, that is, $\|c' b'\| = \|c' d'\|$. By similar triangles $\triangle e'b c$ and $\triangle e'c'b'$, the fact that $\|c'b'\| = \|c'd'\| = \|b d\|$, and our bounds on $\lambda$, we have
\begin{equation} \label{eq:hm-fat-2}
    \frac{\|e' c'\|}{\|e' b\|}
        ~ =    ~ \frac{\|c' b'\|}{\|c b\|}
        ~ \leq ~ \frac{\|b d\|}{\|b d\|\lambda/(1+\lambda)}
        ~ =    ~ \frac{1+\lambda}{\lambda}
        ~ \leq ~ \frac{2}{\lambda},
\end{equation}
which implies~(i).

The analysis for~(ii) is essentially the same as above. Since $(O, c''; d'', b'') = -1$ we have $\|c'' b''\| = \|c'' d''\|$. By similar triangles $\triangle e''b c$ and $\triangle e''c''b''$ and the fact that $\|c''b''\| = \|c''d''\| \leq \|b d\|$, the inequalities of Eq.~\eqref{eq:hm-fat-2} (with double primes for single primes) show that
\[
    \frac{\|e'' c''\|}{\|e'' b\|}
        ~ \leq ~ \frac{2}{\lambda},
\]
which implies~(ii).

These inequalities hold only in transformed configuration, but the cross ratios of~(i) and~(ii) hold unconditionally. 
%
\confonly{Returning to the original configuration and using (i), we can show that  $\|be'\| / \|bc'\| \geq \lambda/3$ and from (ii), we can show that either $\|be''\| / \|bf\| \geq \lambda/6$ or $\|be''\| / \|e''c''\|  \ge \lambda/5$. We omit the details of this calculation, which can be found in the full version~\cite{SoCG23arxiv}. In both cases, we are able to establish Eq.~\eqref{eq:hm-fat-1}, as desired. 
} %confonly
%
\arxivonly{Let's return to the original configuration. Since $\|f c'\|/\|f b\| < 1$, observation~(i) implies that $\|e' c'\|/\|e' b\| \leq 2/\lambda$. Thus, we have
\begin{equation} \label{eq:hm-fat-3}
    \frac{\|e' b\|}{\|c' b\|}
        ~ =    ~ \frac{\|e' b\|}{\|e' c'\| + \|e' b\|}
        ~ \geq ~ \frac{\|e' b\|}{\|e' b\|(2/\lambda) + \|e' b\|}
        ~ =    ~ \frac{\lambda}{\lambda + 2}
        ~ \geq ~ \frac{\lambda}{3}.
\end{equation}

Next, we claim that~(ii) implies that either
\begin{equation} \label{eq:hm-fat-5}
    \frac{\|b e''\|}{\|b f\|}
        ~ \geq ~ \frac{\lambda}{6} \qquad\text{or}\qquad
    \frac{\|b e''\|}{\|e'' c''\|}
        ~ \geq ~ \frac{\lambda}{5}.
\end{equation}
To see why, suppose to the contrary that both inequalities fail to hold. Then we would then have
\begin{align*}
    -(f, e''; c'', b)
        & ~ = ~ \frac{\|f c''\| / \|f b\|}{\|e'' c''\| / \|e'' b\|} 
          ~ = ~ \frac{\|f b\| + \|b e''\| + \|e'' c''\|}{\|f b\|} \cdot \frac{\|e'' b\|}{\|e'' c''\|} \\[2pt]
        & ~ = ~ \frac{\|e'' b\|}{\|e'' c''\|} + \left( \frac{\|b e''\|}{\|b f\|} \cdot \frac{\|b e''\|}{\|e'' c''\|} \right)
                    + \frac{\|b e''\|}{\|b f\|} \\[2pt]
        & ~ < ~ \frac{\lambda}{5} + \left( \frac{\lambda}{6} \cdot \frac{\lambda}{5} \right) + \frac{\lambda}{6}
          ~ = ~ \frac{2}{5} \lambda 
          ~ < ~ \frac{\lambda}{2},
\end{align*}
 which contradicts~(ii).

To complete the proof, we consider two cases depending on which inequality of Eq.~\eqref{eq:hm-fat-5} holds. First, if $\|b e''\|/\|b f\| \geq \lambda/6$, then clearly $\|b e''\|/\|b c'\| \geq \lambda/6$, and so using Eq.~\eqref{eq:hm-fat-3} we have
\[
    \min(\|b e'\|, \|b e''\|) 
        ~ \geq ~ \min \left( \frac{\lambda}{3} \|b c'\|, ~ \frac{\lambda}{6} \|b c'\| \right)
        ~ =    ~ \frac{\lambda}{6} \|b c'\|
        ~ \geq ~ \frac{\lambda}{6} \min(\|b c'\|, \|b c''\|),
\]
as desired. 

Otherwise, $\|b e''\|/\|e'' c''\| \geq \lambda/5$, which implies that 
\[
    \frac{\|b e''\|}{\|b c''\|}
        ~ =    ~ \frac{\|b e''\|}{\|b e''\| + \|e'' c''\|} 
        ~ \geq ~ \frac{\|b e''\|}{\|b e''\| + \|b e''\|(5/\lambda)} 
        ~ =    ~ \frac{\lambda}{\lambda+5} 
        ~ \geq ~ \frac{\lambda}{6}.
\]
Therefore,
\[
    \min(\|b e'\|, \|b e''\|) 
        ~ \geq ~ \min \left( \frac{\lambda}{3} \|b c'\|, ~ \frac{\lambda}{6} \|b c''\| \right)
        ~ \geq ~ \frac{\lambda}{6} \min(\|b c'\|, \|b c''\|),
\]
again, as desired.
}%arxivonly
\end{proof}
%-----------------------------------------------------------------------

The following corollary is immediate from the definition of Macbeath regions.

%-----------------------------------------------------------------------
\begin{corollary} \label{cor:HM-fat-aux}
Assume all entities to be as defined in the statement of Lemma~\ref{lem:HM-fat-aux}. Then $M_{K_H}^{s(\lambda)}(b) \subseteq M_{K_0}(b)$, where $s(\lambda) = \lambda / 6$.
\end{corollary}
%-----------------------------------------------------------------------

We have the following lemma which in conjunction with Corollary~\ref{cor:HM-fat-aux} will be useful in proving Lemma~\ref{lem:HM-fat-main}.
\confonly{The proof is presented in the full version~\cite{SoCG23arxiv}.}

%-----------------------------------------------------------------------
\begin{lemma} \label{lem:cr-lb}
Let $\lambda, K_0, K_1, K_H$, the origin $O$, and points $c$ and $d$ be as in Lemma~\ref{lem:HM-fat-aux}. Let $h$ denote the point of intersection of the ray $Oc$ with the boundary of $K_H$. Then:
\begin{enumerate}
    \item[$(i)$] $\|Oc\| \geq \|hc\|$. 
    \item[$(ii)$] Let $b$ be a point on segment $Oc$, which is not contained in the interior of $M_{K_H}^{\lambda}(c)$. Then $(O,c;d,b) \leq -\lambda / 2$.
\end{enumerate}
\end{lemma}
%-----------------------------------------------------------------------

\arxivonly{
%-----------------------------------------------------------------------
\begin{proof}
Let $h$ and $h'$ denote the points of intersection of the line $Oc$ with the boundary of $K_H$ such that $h', b, c,$ and $h$ appear in this order on the line. Since $(O,h;d,c)$ forms a harmonic bundle, we have $\|Od\|/\|Oc\| =\|hd\|/|hc\|$. Since $\|Od\| \geq \|hd\|$, it follows that $\|Oc\| \geq \|hc\|$, which proves (i).

To prove (ii), observe that since $b$ is not contained in the interior of $M_{K_H}^{\lambda}(c)$, by Corollary~\ref{cor:nesting}, $r_{K_H}(b,c) = -(h',c;h,b) \geq \lambda$.  Thus $(h',c;h,b) \leq -\lambda$. It is straightforward from the definition that this cross ratio decreases on replacing $h'$ by $O$. Thus $(O,c;h,b) \leq -\lambda$. Next we will show that $(O,c;d,b) = (O,c;h,b) / 2$, which will prove the lemma.

To see this, we apply a projective transformation that maps $O$ to a point at infinity. (Recall that collinearity of points and cross ratios are preserved under a projective transformation.) We have $(O,c;d,b) = -\|cb\| / \|cd\|$, $(O,c;h,b) = -\|cb\| / \|ch\|$, and $(O,h;d,c) = -\|ch\| / \|hd\|$. Also, by definition of harmonic body, $(O,h;d,c)$ is a harmonic bundle and so $\|ch\| = \|hd\|$. Thus
\[
(O,c;d,b) = -\frac{\|cb\|}{\|cd\|} = -\frac{\|cb\|}{\|ch\| + \|hd\|} = -\frac{\|cb\|}{2\|ch\|} = \frac{1}{2} \, (O,c;h,b),
\]
as desired.
\end{proof}
%-----------------------------------------------------------------------
} %arxivonly

We now have all the key ingredients to present the main result of this section. The relative fatness of $K_0$ with respect to $K_H$ is an immediate consequence of parts (i) and (ii) of this lemma. In order to state part (iii), we need a definition. Given a convex body $K$ with the origin $O$ in its interior and a region $R \subseteq K$, define the \emph{shadow} of $R$ with respect to $K$, denoted \emph{$\shadow_K(R)$}, to be the set of points $x \in K$ such that the segment $Ox$ intersects $R$. 

%-----------------------------------------------------------------------
\begin{lemma} \label{lem:HM-fat-main}
Let $0 < \beta \le 1$ be a real parameter. Let $K_0 \subset K_1$ be two convex bodies, let the origin $O$ lie in the interior of $K_0$, and let $K_H$ denote the harmonic-mean body of $K_0$ and $K_1$. Let $c$ be any point on the boundary of $K_0$ and let $M = M_{K_H}^{\beta}(c)$. Then there exists a convex body $M'$ such that 
\begin{enumerate}
\item[$(i)$] $\vol(M') = \Omega(\vol(M))$,
\item[$(ii)$] $M' \subseteq M \cap K_0$, and
\item[$(iii)$] $\shadow_{K_0}(M') \subseteq M$.
\end{enumerate}
\end{lemma}
%-----------------------------------------------------------------------

%-----------------------------------------------------------------------
\begin{proof} 
%
\confonly{We sketch the proof of (i) and (ii) here, and present the full proof in~\cite{SoCG23arxiv}.}
For the sake of convenience, assume that the ray $Oc$ is directed vertically upwards. Let $h$ be the point of intersection of the ray $Oc$ with $\bd K_H$. Let $R = M_{K_H}(c)-c$ be the recentering of $M_{K_H}(c)$ about the origin. By definition, $M = M_{K_H}^{\beta}(c) = c + \beta R$. Let $b$ be the point of intersection of the segment $Oc$ with the boundary of $M_{K_H}^{\lambda}(c) = c + \lambda R$, where $\lambda = \beta/\kappa$ for a suitable large constant $\kappa \geq 2$ (independent of dimension). Recalling from Lemma~\ref{lem:cr-lb}(a) that $\|ch\| \leq \|Oc\|$, it follows that $b$ is vertically below $c$ at a distance of $\lambda \|ch\|$. Recalling $s(\lambda)$ from Corollary~\ref{cor:HM-fat-aux}, let $M' = b + \gamma R$ for
\[
    \gamma 
        ~ = ~ \frac{s(\lambda/2)}{10} 
        ~ = ~ \frac{s(\beta/2\kappa)}{10}  
        ~ = ~ \frac{\beta}{120\kappa}
\]
(see Figure~\ref{f:hm-fat-main}(a)). Since $M'$ and $M$ are translated copies of $R$ scaled by a factor of $\gamma$ and $\beta$, respectively, we have $\vol(M') = (\gamma/\beta)^d \vol(M) = (1/120\kappa)^d \vol(M)$. This proves (i).

%-----------------------------------------------------------------------
\begin{figure}[htbp]
\centering
\includegraphics[scale=0.8]{fig/hm-fat-main}
\caption{\label{f:hm-fat-main} Proof of Lemma~\ref{lem:HM-fat-main}. (Objects are not drawn to scale.)}
\end{figure}
%-----------------------------------------------------------------------

To prove (ii), we will show that $M' \subseteq M$ and $M' \subseteq K_0$. Since $b \in c + \lambda R$ and $M' = b + \gamma R$, it follows that $M' \subseteq c + (\lambda + \gamma) R$. For large $\kappa$, we have $\lambda + \gamma \le \beta$, and thus $M' \subseteq c + \beta R = M$. 

Next we show that $M' \subseteq K_0$. Let $d$ denote the point of intersection of the ray $Oc$ with $\bd K_1$. Applying Lemma~\ref{lem:cr-lb}(b), it follows that the cross ratio $(O,c;d,b) \leq -\lambda / 2$. Applying Corollary~\ref{cor:HM-fat-aux} with $\lambda/2$ in place of $\lambda$ and recalling that $s(\lambda/2) = 10\gamma$, we have $M_{K_H}^{10 \gamma}(b) \subseteq M_{K_0}(b)$. Also, by Lemma~\ref{lem:mac-trans}, we have $M' = b + \gamma R \subseteq M_{K_H}^{2\gamma}(b)$. Thus $M' \subseteq M_{K_0}^{1/5}(b)$. By definition of Macbeath regions, $M_{K_0}(b) \subseteq K_0$, and so $M' \subseteq K_0$, as desired. 

\arxivonly{
To prove (iii), let $S = \shadow_{K_0}(M')$, and let $M''$ be the convex body obtained by scaling $M'$ by the factor 
\[
    f 
        ~ = ~ 1 + 4 \lambda \frac{\|ch\|}{\|Oc\|}
\]
about $O$ (see Figure~\ref{f:hm-fat-main}(b)). Letting $b''$ denote the center of $M''$, we have $M'' = b'' + f \gamma R$. We claim that
\begin{enumerate}
%
\item[(a)] $S$ is contained in the convex hull of $M' \cup M''$, and

\item[(b)] the convex hull of $M' \cup M''$ is contained in $M$.
%
\end{enumerate} 
Together, this would imply that $S$ is contained in $M$, and complete the proof.

To prove (a), let $c'$ be any point in $S \cap \bd K_0$ and let $b'$ be any point in the intersection of segment $Oc'$ with $M'$. Since $b' \in M'$ and $M' \subseteq M_{K_0}^{1/5}(b)$, we have $b' \in M_{K_0}^{1/5}(b)$. By Lemma~\ref{lem:core-ray}, we have $\ray_{K_0}(b') \le 2 \ray_{K_0}(b)$, that is, $\|b'c'\| / \|Oc'\| \le 2 \|bc\| / \|Oc\|$. Recalling that $\|bc\| = \lambda \|ch\|$, it follows that
\[
    \frac{\|Oc'\|}{\|Ob'\|} 
        ~ =    ~ \frac{\|Oc'\|}{\|Oc'\| - \|b'c'\|} 
        ~ =    ~ \frac{1}{1 - \frac{\|b'c'\|}{\|Oc'\|}} 
        ~ \leq ~ \frac{1}{1 - 2 \lambda \frac{\|ch\|}{\|Oc\|}} 
        ~ \leq ~ 1 + 4 \lambda \frac{\|ch\|}{\|Oc\|}.
\]
Recall that we defined the quantity on the right hand side to be the scaling factor $f$ and $M''$ to be the $f$-factor expansion of $M'$ about $O$. Since $\|Oc'\| \le f \|Ob'\|$, it follows that for any ray passing through $M'$, the points of $S$ on this ray lie between the lowest point on the ray contained in $M'$ and the highest point on the ray contained in $M''$. It follows that $S$ is contained in the convex hull of $M' \cup M''$.

It remains to prove (b).  By convexity of $M$, it suffices to show that both $M'$ and $M''$ are contained in $M$. We have already shown in part (ii) that $M' \subseteq M$. To complete the proof, it suffices to show that $M'' \subseteq M$. Note that the point corresponding to $c$, obtained by scaling by a factor of $f$ about the origin, is at distance $4 \lambda \|ch\|$ vertically above $c$. Clearly $b''$ lies on the segment $bc''$, and so $b'' \in c + 4 \lambda R$. Recalling that $M''= b'' + f \gamma R$, it follows that $M'' \subseteq c + (4 \lambda + f \gamma) R$. Since $\lambda = \beta/\kappa$, and $\gamma = \beta/120\kappa$, and $f = 1 + 4\lambda \|ch\|/\|Oc\| \le 1 + 4 \lambda$, we have $4\lambda + f\gamma \le \beta$, for large $\kappa$. Thus $M'' \subseteq c + \beta R = M$, which completes the proof.
}%arxivonly
\end{proof}
%-----------------------------------------------------------------------

The following corollary is immediate from parts (i) and (ii) of the above lemma.

\begin{corollary}
Let $K_0 \subset K_1$ be two convex bodies, let the origin $O$ lie in the interior of $K_0$, and let $K_H$ denote the harmonic-mean body of $K_0$ and $K_1$. Then $K_0$ is relatively fat with respect to $K_H$.    
\end{corollary}

\section{Uniform Volume-Sensitive Bounds}
\label{s:hausdorff}

In this section, we present the proof of Theorem~\ref{thm:main}. Let $\eps > 0$ and let $K_0$ denote the convex body $K$ described in this theorem. Let $K_1 = K_0 \oplus \eps$ denote the Minkowski sum of $K_0$ with a ball of radius $\eps$. Also recall that $\Delta_d(K_0)$ denotes the \emph{volume diameter} of $K_0$. Let $C(K_0,\eps)$ be a shorthand for $(\Delta_d(K_0)/\eps)^{(d-1)/2}$, the desired number of facets. 

We will show that there exists a polytope with $O(C(K_0,\eps))$ facets sandwiched between $K_0$ and $K_1$. As mentioned above, we will transform the problem by mapping to the polar. Through an appropriate translation, we may assume that the origin $O$ coincides with the centroid of $K_0$. Note that the arithmetic-mean body $K_A$ of $K_0$ and $K_1$ is given by $K_0 \oplus \frac{\eps}{2}$, and recall from Section~\ref{s:am-hm} that $K_H = K_A^*$ is the harmonic-mean body of $K_1^*$ and $K_0^*$. 

Our construction is based on Lemma~\ref{lem:MNet-size-hausdorff}, which shows that there is a $(K_H,\bd K_1^*)$-MNet $X$ of size $O(C(K_0,\eps))$. Applying Lemma~\ref{lem:MNet-approx}, it follows that there exists a polytope $P$ sandwiched between $K_1^*$ and $K_H$ with $O(|X|)$ vertices. By polarity, this implies that $P^*$ is a polytope sandwiched between $K_A$ and $K_1$ having $O(|X|)$ facets. Since $K_0 \subseteq K_A$, this polytope is also sandwiched between $K_0$ and $K_1$, which proves Theorem~\ref{thm:main}.

All that remains is showing that $|X| = O(C(K_0,\eps))$. For this purpose, we will utilize the tools for bounding the sizes of MNets in conjunction with the relative fatness of the harmonic-mean body (established in Section~\ref{s:hm-fat}).

%-----------------------------------------------------------------------
\begin{lemma} \label{lem:MNet-size-hausdorff}
Let $\eps > 0$ and let $K_0, K_1, K_A, K_H$ be convex bodies as defined above. Let $X$ be a $(K_H,\bd K_1^*)$-MNet. Then $|X| = O(C(K_0,\eps))$.
\end{lemma}
%-----------------------------------------------------------------------

%-----------------------------------------------------------------------
\begin{proof}
We begin by showing that $\vol(K_H) = \Omega(1/\vol(K_0))$, and its Mahler volume $\mu(K_H)$ is at most $O(1)$ (implying that $K_H$ is well-centered). To see this, recall that the width of $K_0$ in any direction is at least $\eps$ and $K_A = K_0 \oplus \frac{\eps}{2}$. It is well-known that the ratio of the distances of the centroid from any pair of supporting hyperplanes is at most $d$~\cite{Gru63,Min1897,Rad1916}. It follows that a ball of radius $\eps/(d+1)$ centered at the origin lies within $K_0$. Thus, a constant-factor expansion of $K_0$ contains $K_A$, implying that $\vol(K_A) = O(\vol(K_0))$. Also, because $K_H = K_A^*$, by Lemma~\ref{lem:mahler-bounds}, $\vol(K_A) \cdot \vol(K_H) = \Omega(1)$. Thus, $\vol(K_H) = \Omega(1/\vol(K_0))$. To upper bound $\mu(K_H)$, note that by polarity, $K_H \subseteq K_0^*$, and thus
\[
    \mu(K_H) 
        ~ = ~ \vol(K_A) \cdot \vol(K_H) 
        ~ = ~ O(\vol(K_0) \cdot \vol(K_0^*)) 
        ~ = ~ O(\mu(K_0))
        ~ = ~ O(1),
\] 
where in the last step, we have used Lemma~\ref{lem:centroid} and our assumption that the origin coincides with the centroid of $K_0$.

To simplify notation, for the remainder of the proof we assume that ray distances, Macbeath regions, and volumes are defined relative to $K_H$, that is, $\ray \equiv \ray_{K_H}$, $M \equiv M_{K_H}$, and $\vol \equiv \vol_{K_H}$.

For any point $p \in \bd K_1^*$, let $p'$ denote the point of intersection of the ray $O p$ with $\bd K_H$. We first establish a bound on the relative ray distance $\ray(p)$. Observe that since $p$ and $p'$ lie on $\bd K_1^*$ and $\bd K_H$, respectively, their polar hyperplanes, $p^*$ and ${p'}^*$, are supporting hyperplanes for $K_1$ and $K_H^* = K_A$, respectively. Letting $r$ denote the distance between ${p'}^*$ and the origin, it follows from the definition of $K_A$ that the distance between $p^*$ and the origin is $r + \frac{\eps}{2}$. The distance of $p'$ and $p$ from the origin are the reciprocals of these. Therefore, we have
\[
    \ray(p)
        ~ = ~ \frac{\|p p'\|}{\|O p'\|}
        ~ = ~ \frac{\|O p'\| - \|O p\|}{\|O p'\|}
        ~ = ~ \frac{\frac{1}{r} - \frac{1}{r + (\eps/2)}}{\frac{1}{r}}
        ~ = ~ 1 - \frac{r}{r + (\eps/2)}
        ~ = ~ \frac{\eps/2}{r + (\eps/2)}.
\]
Since $\frac{1}{\|O p'\|} = r = \Omega(\eps)$, we have $\ray(p) = \Theta(\eps/r) = \Theta(\eps \|O p'\|)$. (It is noteworthy and somewhat surprising that this relative ray distance is not a dimensionless quantity, since it depends linearly on $\|O p'\|$.)

To analyze $|X|$, we partition it into groups based on $\|O x'\|$ for each $x \in X$. Define $R_0 = (\vol(K_H))^{1/d}$. By our earlier remarks, $\vol(K_H) = \Omega(1/\vol(K_0))$, and so $R_0 = \Omega(1/\Delta_d(K_0))$. For any integer $i$ (possibly negative), define $R_i = 2^i R_0$ and $\eps_i = \eps R_i$. We can express $X$ as the disjoint union of sets $X_i$, where $X_i$ consists of points $x$ such that $R_i \leq \|Ox'\| < 2 R_i$. Recall that for any $x \in X_i$, we have $\ray(x) = \Theta(\eps \|O x'\|) = \Theta(\eps R_i) = \Theta(\eps_i)$. 

We will bound the contributions of the $|X_i|$ to $|X|$ based on the sign of $i$. Let us first consider the nonnegative values of $i$. We remark that $|X_i| = 0$ for large $i$ (specifically, for $i = \omega(\log(1/\eps R_0))$) because a ball of radius $\Omega(\eps)$ centered at the origin is contained within $K_0$, and so by polarity $K_0^*$, and hence $K_1^*$, is contained within a ball of radius $O(1/\eps)$. Recalling that $K_H$ is well-centered and applying Lemma~\ref{lem:fixed-ray}(i), we have (up to constant factors)
\begin{align*}
    \sum_{i \geq 0} |X_i|
        & ~ \leq ~ \sum_{i \geq 0} \left( \frac{1}{\eps_i}\right)^{\kern-2pt\frac{d-1}{2}}
          ~ =    ~ \sum_{i \geq 0} \left( \frac{1}{\eps 2^i R_0}\right)^{\kern-2pt\frac{d-1}{2}}
          ~ \leq ~ \sum_{i \geq 0} \left( \frac{\Delta_d(K_0)}{\eps 2^i}\right)^{\kern-2pt\frac{d-1}{2}} \\
          & ~ =    ~ \left( \frac{\Delta_d(K_0)}{\eps} \right)^{\kern-2pt\frac{d-1}{2}}  \sum_{i \geq 0} \left(\frac{1}{2}\right)^{\kern-2pt\frac{i(d-1)}{2}}        
            ~ \leq   ~ \left( \frac{\Delta_d(K_0)}{\eps} \right)^{\kern-2pt\frac{d-1}{2}} 
          ~ =    ~ O(C(K_0,\eps)).
\end{align*}

In order to bound the contributions to $|X|$ for negative values of $i$, we need a more sophisticated strategy. Our approach is to first bound the total relative volume of the Macbeath regions of $\MM^{1/4c}(X_i)$, which we assert to be $O(\eps_i 2^{id})$. Assuming this assertion for now, we complete the proof as follows. By applying Lemma~\ref{lem:fixed-ray}(ii)  with $f = O(2^{id})$ and recalling that $\eps_i = \eps R_i = 2^i \eps R_0$, we have (up to constant factors)
\begin{align*}
    \sum_{i < 0} |X_i| 
        & ~ \leq ~ \sum_{i < 0} \frac{\sqrt{f}}{\eps_i^{(d-1)/2}} 
          ~ =    ~ \sum_{i < 0} \frac{2^{i d/2}}{(2^i \eps R_0)^{(d-1)/2}}
          ~ =    ~ \sum_{i < 0} \frac{2^{i(d-(d-1))/2}}{(\eps R_0)^{(d-1)/2}}
          ~ =    ~ \sum_{i < 0} \frac{2^{i/2}}{(\eps R_0)^{(d-1)/2}} \\
        & ~ =    ~ \sum_{i < 0} 2^{i/2} C(K_0,\eps)
          ~ =    ~ C(K_0,\eps) \sum_{i > 0} \left( \frac{1}{2} \right)^{\kern-1pt\frac{i}{2}}
          ~ =    ~ O(C(K_0, \eps)).
\end{align*}

It remains only to prove the assertion on the total relative volume of $\MM^{1/4c}(X_i)$. Let $x \in X_i$ and let $M_x = M^{1/4c}(x)$. By Lemma~\ref{lem:HM-fat-main} (with $x$, $K_1^*$, and $K_H$ playing the roles of $c$, $K_0$, and $K_H$, respectively), there is an associated convex body $M'_x$ such that 
\begin{center}
    (i) $\vol(M'_x) = \Omega(\vol(M_x))$, ~~
    (ii) $M'_x \subseteq M_x \cap K_1^*$, ~~and~~ 
    (iii) $\shadow_{K_1^*}(M'_x) \subseteq M_x$.
\end{center}
We will use $S_x$ as a shorthand for $\shadow_{K_1^*}(M'_x)$.  Since $\vol(M_x) = O(\vol(M'_x)) = O(\vol(S_x))$, it suffices to show that the total relative volume of the shadows $\{S_x : x \in X_i\}$ is $O(\eps_i 2^{id})$.

For $x \in X_i$, we define cone $\Psi_x$ to be the intersection of $K_H$ with the infinite cone consisting of rays emanating from the origin that contain a point of $S_x$ (see Figure~\ref{f:mnet-size-hausdorff}). Since the Macbeath regions of $\MM^{1/4c}(X_i)$ are disjoint, it follows from (iii) that the associated shadows intersect $\bd K_1^*$ in patches that are also disjoint. Thus the set of cones $\Psi = \{\Psi_x : x \in X_i\}$ are disjoint.

%-----------------------------------------------------------------------
\begin{figure}[htbp]
\centering
\includegraphics[scale=0.8]{fig/mnet-size-hausdorff}
\caption{\label{f:mnet-size-hausdorff}Proof of Lemma~\ref{lem:MNet-size-hausdorff}.}
\end{figure}
%-----------------------------------------------------------------------


Consider a ray emanating from the origin that is contained in any cone $\Psi_x$. Let $q$ and $q'$ be the points of intersection of this ray with $\bd K_1^*$ and $\bd K_H$, respectively. Let $q''$ be any point on this ray that lies inside shadow $S_x$. Since $q'' \in M_x$, by Lemma~\ref{lem:core-ray}, we have $\ray(q'') = \Theta(\ray(x)) = \Theta(\eps_i)$. By the same reasoning, $\ray(q) = \Theta(\eps_i) = \Theta(\eps R_i)$. Also, recalling our earlier bounds on the relative ray distance of points on $\bd K_1^*$, we have $\ray(q) = \Theta(\eps \|Oq'\|)$. Equating the two expressions for $\ray(q)$, we obtain $\|Oq'\| = \Theta(R_i)$.

Since the cones of $\Psi$ are disjoint and any ray emanating from the origin and contained in a cone of $\Psi$ has length $\Theta(R_i)$, it follows that the total volume of these cones is $O(R_i^d)$. Further, since only a fraction $\eps_i$ of any such ray is contained in the associated shadow, it follows that the total volume of all the shadows $\{S_x : x \in X_i\}$ is $O(\eps_i R_i^d)$. Recalling that $\vol(K_H) = R_0^d$ and $R_i = 2^i R_0$, it follows that the total relative volume of these shadows is $O(\eps_i R_i^d / R_0^d) = O(\eps_i 2^{id})$. This establishes the assertion on the total relative volume of $\MM^{1/4c}(X_i)$ and completes the proof.
\end{proof}
%-----------------------------------------------------------------------


\arxivonly{
%=======================================================================
\section{Nonuniform Volume-Sensitive Bounds} \label{s:nonuniform}
%=======================================================================
A nonuniform bound very similar to ours can be derived from a result due to Gruber \cite{Gru93a}, who showed that if $K$ is a strictly convex body and $\bd K$ is twice differentiable ($C^2$ continuous), then as $\eps$ approaches zero, the number of bounding halfspaces needed to achieve an $\eps$-approximation of $K$ is
\begin{equation} \label{eq:gruber}
    O\left( \left( \frac{1}{\eps} \right)^{\kern-2pt(d-1)/2} \int_{\bd K} \kappa(x)^{1/2} d\sigma(x) \right), 
\end{equation}
where $\kappa$ and $\sigma$ denote the Gaussian curvature of $K$ and ordinary surface area measure, respectively. (B{\" o}r{\" o}czky showed that the requirement that $K$ be ``strictly'' convex can be eliminated \cite{Bor00}.) 

Assume that the origin coincides with the centroid of $K$. Let $S^{d-1}$ denote the unit Euclidean sphere in $\RE^d$. For $u \in S^{d-1}$, let $h(u) = \max \, \{ \ang{x,u} : x \in K\}$ denote the support function of $K$ and let $\rho(u) = \max \, \{\lambda > 0: \lambda u \in K^*\}$ denote the radial function of $K^*$. 

Letting $n$ denote the exterior normal unit vector of $K$ and applying the Cauchy-Schwarz inequality, we obtain
\[
    \left(\int_{\bd K} \kappa(x)^{1/2} \, d\sigma(x) \right)^2 
    ~ \leq ~ \left( \int_{\bd K} \frac{\kappa(x)}{h(n(x))} \, d\sigma(x) \right) \cdot
             \left(\int_{\bd K} h(n(x)) \, d\sigma(x)\right).
\]
The second integral on the right hand side is easily seen to be $d \cdot \vol(K)$. To bound the first integral on the right hand side, we express it as an integral over the unit sphere $S^{d-1}$.
\[
    \int_{\bd K} \frac{\kappa(x)}{h(n(x))} \, d\sigma(x) 
        ~ = ~ \int_{S^{d-1}} \frac{1}{h(u)} \, d\sigma(u).
\]
Letting $\varsigma_{d-1} = \area(S^{d-1})$ and applying Jensen's inequality, we have
\[
    \frac{1}{\varsigma_{d-1}} \int_{S^{d-1}} \frac{1}{h(u)} \, d\sigma(u)
        ~ \leq ~ \left(\frac{1}{\varsigma_{d-1}} \int_{S^{d-1}} \frac{1}{h(u)^d} \, d\sigma(u)\right)^{1/d}
        ~ = ~ \left(\frac{1}{\varsigma_{d-1}} \int_{S^{d-1}} \rho(u)^d \, d\sigma(u)\right)^{1/d},
\]
where we have used the polar relationship $\rho(u) = 1/h(u)$ in the last step. It is easy to see that this integral is $d \cdot \vol(K^*)$. Neglecting constant factors depending on $d$, we have thus shown that the first integral on the right hand side of Eq.~\eqref{eq:gruber} is $O(\vol(K^*)^{1/d})$. Thus
\[
    \left(\int_{\bd K} \kappa(x)^{1/2} \, d\sigma(x) \right)^2 
        ~ = ~ O(\vol(K^*)^{1/d} \cdot \vol(K))
        ~ = ~ O(\vol(K)^{1-1/d}),
\]
where we have used the Blaschke–Santaló inequality, which implies that the product $\vol(K) \cdot \vol(K^*) = O(1)$, when the origin coincides with $K$'s centroid (see Lemma~\ref{lem:centroid}). Substituting in Eq.~\eqref{eq:gruber} and recalling that the volume diameter of $K$, $\Delta_d(K) = \Theta((\vol(K))^{1/d})$, we obtain the following theorem.

{\RLnonunifbound*}

Note that the bound in this theorem matches the uniform bound of Theorem~\ref{thm:main}. However, this approach does not produce a uniform bound, since Eq.~\eqref{eq:gruber} only holds in the limit as $\eps$ approaches zero. (See~\cite{AFM12b} for a counterexample showing that this equation could be violated otherwise.)

}%\arxivonly

%=======================================================================
\section*{Acknowledgements}
%=======================================================================

The authors would like to acknowledge the insights and feedback from Rahul Arya and Guilherme da Fonseca.

\pdfbookmark[1]{References}{s:ref}
\bibliographystyle{plainurl}
\bibliography{convex}

\end{document}