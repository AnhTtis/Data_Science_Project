%%%%%%%%%%%%%%%%%%%%%%%%%%%%%%%%%%%%%%%%%%%%%%%%%%%
\documentclass[12pt]{article}

%\usepackage{showkeys}
\usepackage{epsfig}
\usepackage{amssymb,amsmath}
\usepackage{multirow}
\usepackage{rotating}
\usepackage{graphicx}
\usepackage{lscape}
\usepackage{hyperref}
%\usepackage[dvips]{graphicx,psfrag}
%\usepackage{color}
\usepackage{xcolor}
%\usepackage{array}
\usepackage{url}
%\usepackage{feynmp} % to draw Feynman diagram
\usepackage{feynmp-auto} % to draw Feynman diagram, and automataically recognize the figure file of the form *.1

\usepackage{comment}

\usepackage{cite} % to cite multiple reference in the same brackets in short way.

%\usepackage{biblatex} % to use BibLatex

\setlength{\oddsidemargin}{-3mm}
\setlength{\evensidemargin}{0mm} 
\setlength{\textwidth}{17.0cm}
\setlength{\topmargin}{0cm} \setlength{\headheight}{0cm}
\setlength{\headsep}{0cm} \setlength{\textheight}{22.5cm}
%\setlength{\extrarowheight}{2pt}

%\newcommand{\tabtopsp}[1]{\vbox{\vbox to#1{}\vbox to1zw{}}}
\newcommand{\bea}{\begin{eqnarray}}
\newcommand{\eea}{\end{eqnarray}}
\def\tr{\mathrm{tr}}
 \makeatletter
\def\alt{\mathrel{\mathpalette\gl@align<}}
\def\agt{\mathrel{\mathpalette\gl@align>}}
\def\gl@align#1#2{\lower.6ex\vbox{\baselineskip\z@skip\lineskip\z@
\ialign{$\m@th#1\hfil##\hfil$\crcr#2\crcr\sim\crcr}}} \makeatother

\begin{document}





\begin{flushright}
%preprint number \\
\end{flushright}

\vspace*{1.0cm}

\begin{center}
\baselineskip 20pt 
{\Large\bf 
%$b \rightarrow s \gamma$ transition in
%a model with vectorlike fermions \\
%and secluded scalars \\
%: (present and prospect)
%
%B\'elanger$-$Delaunay$-$Westhoff model
%
%$b \rightarrow s$ transitions and CKM unitarity %violation 
%from vectorlike fermions
%with $U(1)_X$ charges \\
%prospect at Belle II \\

A model with vectorlike fermions and $U(1)_X$ symmetry: CKM unitarity, $b \rightarrow s$ transitions,\\
and prospect at Belle II
}
%
\vspace{1cm}

%
{\large 
Sang Quang Dinh$^{a}$
and 
Hieu Minh Tran$^{b,}$\footnote{ E-mail: hieu.tranminh@hust.edu.vn} 
} 
\vspace{.5cm}

{\baselineskip 20pt \it
$^a$VNU University of Science, Vietnam National University - Hanoi,  \\
334 Nguyen Trai Road, Hanoi, Vietnam \\
\vspace{2mm} 
$^b$Hanoi University of Science and Technology, 1 Dai Co Viet Road, Hanoi, Vietnam 
}


\vspace{.5cm}

\vspace{1.5cm} {\bf Abstract}
\end{center}

To address the muon $g-2$ anomaly and the violation of the lepton flavor universality in the semileptonic decays of $B$ mesons, 
B\'elanger, Delaunay, and Westhoff introduced a new sector consisting of vectorlike fermions and two scalar charged under an extra $U(1)_X$ gauge symmetry.
%
The exotic Yukawa interactions in this model lead to the quark mixing responsible for the additional contributions to the flavor changing neutral currents in $B$-meson decays.
%
In this paper, we derive the analytic expression of the new physics contributions to the Wilson coefficient $C_7$ in the effective Hamiltonian.
%
By calculating the branching ratio of the inclusive radiative $B$ decay, the impact of current experimental data of the $b \rightarrow s \gamma$ transition on the model and the future prospect at the Belle II experiment are investigated.
%
Taking into account the recent data on
the CKM unitarity violation, the updated constraints on the flavor observables relevant to the $b \rightarrow s$ transitions, and the perturbation limits of the couplings,
the viable parameter regions of the model are identified.

%============================
%
%The constraints on the semileptonic decays of $B$ mesons are also taken into account.
%
%
%LEP and LHC constraints in the searches for $Z'$ boson




\thispagestyle{empty}

%\bigskip
\newpage

\addtocounter{page}{-1}


%%%%%%%%%%%%%%%%%%%%%%%%%%
%\baselineskip 36pt
% Main body
%%%%%%%%%%%%%%%%%%%%%%%%%%
\baselineskip 18pt
%%%%%%%%%%%%%%%%%%%%%%%%%%



%\begin{fmffile}{feyndiagrams}
%to generate unique file "feyndiagrams.mp" to run with "mpost" command (avoiding multiple .mp files)


%%%%%%%%%%%%%%%%%%%%%%%%%%
\section{Introduction}  
%%%%%%%%%%%%%%%%%%%%%%%%%%

Although the standard model (SM) is in agreement with most of the experimental results, there are numerous evidences showing that it is not enough to explain them, ranging from cosmological observations to measurements at colliders.
%
Therefore, this model is considered as an effective theory, and the new physics is likely somewhere around the corner.
%
%
Recently, the updated determination of the Cabibbo-Kobayashi-Maskaw (CKM) matrix elements showed that there is a 2.2$\sigma$ deviation from unitary in the first row of this mixing matrix
\cite{Workman:2022ynf}:
\begin{eqnarray}
|V_{ud}|^2 + |V_{us}|^2 + |V_{ub}|^2 &=& 
	0.9985 \pm 0.0007.
\label{CKM_unitarity}
\end{eqnarray}
%
This CKM anomaly implies a non-negligible effect of new physics   \cite{Belfatto:2019swo}.
%
Beside the violation of the CKM unitarity, 
other constraints such as those on the 
muon $g-2$, the violations of the lepton flavor universality ($R_K$ and $R_{K^*}$) also indicate the existence of new physics
\cite{Arnan:2019uhr, Crivellin:2015mga}.


%
Among the sensitive probes to the new physics beyond the SM, the 
$b \rightarrow s \gamma$ transition is of particular importance.
%
Being a flavor changing neutral current (FCNC) process, this transition is forbidden at the tree level and only arises due to quantum corrections at loop levels in the SM.
Interestingly, its rate is of order $G_F^2 \alpha$ that is larger than the rates of most of other FCNC processes 
\cite{Buras:1993xp, Buchalla:1995vs}.
At the hadronic level, this transition is identified as the $B \rightarrow X_s \gamma$ decay, of which the branching ratio for $E_\gamma > 1.6$ GeV
 predicted by the SM is given by 
\cite{Chetyrkin:1996vx, Misiak:2006zs, Misiak:2015xwa, Czakon:2015exa, Misiak:2020vlo}
\begin{eqnarray}
BR(B \rightarrow X_s \gamma)_\text{SM}	&=& 
	(3.40 \pm 0.17) \times 10^{-4}.
\end{eqnarray}
%
%
The measurements of this inclusive decay process have been performed by several experiments including
CLEO \cite{CLEO:2001gsa}, 
BaBar \cite{BaBar:2007yhb, BaBar:2012eja, BaBar:2012fqh}, and
Belle \cite{Belle:2009nth, Belle:2014nmp}.
The world average value of the branching ratio was evaluated by the Heavy Flavor Averaging Group \cite{HFLAV:2022pwe} as
\begin{eqnarray}
BR(B \rightarrow X_s \gamma)_\text{exp}	&=& 
	(3.49 \pm 0.19) \times 10^{-4},
\end{eqnarray}
for the same cut on the radiated photon energy.
%
%
The good agreement between the theoretical prediction and the measurement implies that the contributions of new physics to the $b \rightarrow s \gamma$ decay should not be too large.
%
In the near future, when the relative uncertainty is reduced to the level of a few per cent with the luminosity of 50 ab$^{-1}$ at the Belle II experiment
\cite{Belle-II:2022cgf, DiCanto:2022icc}, it is expected that the upcoming data will strongly constrain the new physics contribution to this process if the center value of the decay rate remains unchanged.




It has been shown that many new physics models are strictly constrained by the $b \rightarrow s \gamma$ transition,
see for example Refs. 
\cite{Chang:2000gz, Akeroyd:2001gf,
Idarraga:2005ia, Lunghi:2007ak, Branco:2011iw, Hermann:2012fc, Jung:2012vu, Crivellin:2013wna, Das:2015kea, Misiak:2017bgg, Haller:2018nnx, Arco:2020ucn, Atkinson:2021eox, Arco:2022xum, Enomoto:2022rrl,
Akeroyd:2020nfj,
Bertolini:1990if, Barbieri:1993av, Borzumati:1994te, Degrassi:2000qf, Carena:2000uj, Demir:2001yz, Baek:2002wm, Hurth:2003vb, Ellis:2006ix, Gomez:2006uv, Ellis:2007fu, Heinemeyer:2008fb, Olive:2008vv, Okada:2010xe, Zhang:2014nya, GAMBIT:2017snp, Yang:2018fvw,
Haisch:2007vb, Freitas:2008vh, Moch:2015oka, Blanke:2012tv, Datta:2016flx,
Cheung:2017efc,
Aliev:1996cyj,
NguyenTuan:2020xls,
Gabrielli:2016cut,
Aoki:2000ze, Aoki:2001xr, Morozumi:2018cnc, Vatsyayan:2020jan}.
%
Among those, models with vectorlike quarks turn out to be interesting since they can naturally address the CKM unitarity violation by the mixing between the SM and the vectorlike quarks  \cite{Kawamura:2019rth, Cheung:2020vqm, Crivellin:2020ebi, Cherchiglia:2021vhe, Belfatto:2021jhf, Branco:2021vhs, Balaji:2021lpr, CarcamoHernandez:2021yev, Accomando:2022ouo, Guedes:2022cfy, Branco:2022fmj}.
%
%
%
%
%- Model with vector quarks: \cite{Chang:2000gz, Akeroyd:2001gf}
%
%* THDM: \cite{Hewett:1992is,Idarraga:2005ia, Lunghi:2007ak, Branco:2011iw, Hermann:2012fc, Jung:2012vu, Crivellin:2013wna, Das:2015kea, Misiak:2017bgg, Haller:2018nnx, Arco:2020ucn, Atkinson:2021eox, Arco:2022xum, Enomoto:2022rrl}
%
%* 3HDM: \cite{Akeroyd:2020nfj}
%
%* SUSY: \cite{Bertolini:1990if, Barbieri:1993av, Borzumati:1994te, Degrassi:2000qf, Carena:2000uj, Demir:2001yz, Baek:2002wm, Hurth:2003vb, Ellis:2006ix, Gomez:2006uv, Ellis:2007fu, Heinemeyer:2008fb, Olive:2008vv, Okada:2010xe,
%Zhang:2014nya, GAMBIT:2017snp, Yang:2018fvw}
%
%* Extra dimensions: 
%\cite{Haisch:2007vb, Freitas:2008vh, Moch:2015oka, Blanke:2012tv, Datta:2016flx}
%
%* Leptoquarks: \cite{Cheung:2017efc}
%
%* Four generation: \cite{Aliev:1996cyj}
%
%* 3-3-1-1 model: \cite{NguyenTuan:2020xls}
%
%* Vectorlike fermion: \cite{Aoki:2000ze, Aoki:2001xr, Morozumi:2018cnc, Vatsyayan:2020jan}
%
%
%
%
In this paper, we are interested in the model with additional vectorlike fermions and a secluded scalar sector charged under an extra Abelian gauge symmetry $U(1)_X$ proposed 
by B\'elanger, Delaunay, and Westhoff
in Refs.
\cite{Belanger:2015nma, Belanger:2016ywb}.
%
%
Due to the exotic Yukawa coupling between the muon, the vectorlike lepton and the scalar, of which the field value vanishes at the minimum of the scalar potential, the model can explain the measured muon anomalous magnetic moment.
%
The other type of Yukawa couplings between the SM quarks, the vectorlike ones and another scalar that develops nonzero vacuum expectation value leads to the mixing in the quark sector among the SM and the vectorlike quarks.
This is the source for the the additional contributions to the FCNC processes such as the semileptonic decays of $B$ mesons.
%
%
The new physics contribution Wilson coefficients $C^{(')}_{9,10}$ were calculated analytically at the leading order in Ref. \cite{Dinh:2020inx}
in the general case with a non-vanishing gauge kinetic mixing term, allowing the evaluation of multiple relevant flavor observables such as the decay rates of the $b \rightarrow s \ell^+ \ell^-$ processes, $R_{K}$, and $R_{K^*}$.
%
%
In this follow-up paper, we investigate the ability of this model to reconcile both the constraints on the CKM unitarity violation and the $b \rightarrow s \gamma$ decay, 
while keeping other predicted observables consistent with their experimental values.
%
The future prospect of this model at the Belle II experiment will be discussed as well.





The structure of the paper is as follows.
%
In Section 2, the setup of the model is briefly reviewed.
%
In Section 3, the analytic expression of the new physics contributions to the Wilson coefficient $C_7$ in the effective Hamiltonian is derived.
%
In Section 4, the numerical analyses are carried out.
The dependence of the branching ratio of $b\rightarrow s \gamma$ decay on the input parameters is presented. 
Taking into account various constraints, we then identify the viable parameter regions of the model.
Here, the impact of the expected result at the Belle II experiment is also considered.
%
The last section is devoted to the conclusion.




%=========================================



%\begin{eqnarray}
%\text{BR}(b\rightarrow s \gamma)_\text{Belle2} &=& (3.490 \pm 0.698) \times 10^{-4}.
%\end{eqnarray}



%Brief review of recent progress on $b \rightarrow s \gamma$








%%%%%%%%%%%%%%%%%%%%%%%%%%%%%%
\section{The model}
%%%%%%%%%%%%%%%%%%%%%%%%%%%%%%



In the considered model, beside the SM particles, the new particles introduced  are the vectorlike lepton and quark doublets of the gauge group $SU(2)_L$,
%
\begin{eqnarray}
L_{L,R} = 
	\begin{pmatrix}
	N_{L,R} \\ E_{L,R}
	\end{pmatrix}	,	\qquad
Q_{L,R} = 
	\begin{pmatrix}
	U_{L,R} \\ D_{L,R}
	\end{pmatrix}	,
\end{eqnarray}
and two complex scalars, $\chi$ and $\phi$, that are singlets under the SM gauge groups.
%
The symmetry of this model is 
$SU(3)_C \otimes SU(2)_L \otimes U(1)_Y \otimes U(1)_X$
that extends the SM symmetry by adding an extra Abelian gauge group.
%
The SM particles are invariant under $U(1)_X$ transformation, while the new particles transform nontrivially with the $U(1)_X$ charges given in Table \ref{NP} together with other properties.




%%%%%%%%%%%%%%%%%%%%%

\begin{table}[h]
\caption{Properties of new particles introduced in the model \cite{Belanger:2015nma}.}
\label{NP}
\begin{center}
\begin{tabular}{|c|c|c|c|c|c|}
\hline
Particles	&	Spin	&	$SU(3)_C$	&	$SU(2)_L$	&	$U(1)_Y$	&	$U(1)_X$	\\
\hline
$L_L, L_R$	&	1/2	&	\textbf{1}	&	\textbf{2}	&	-1/2	&	1	\\
$Q_L, Q_R$	&	1/2	&	\textbf{3}	&	\textbf{2}	&	1/6	&	-2	\\
\hline
$\chi$		&	0	&	\textbf{1}	&	\textbf{1}	&	0	&	-1	\\
$\phi$		&	0	&	\textbf{1}	&	\textbf{1}	&	0	&	2	\\
\hline
\end{tabular}
\end{center}
\end{table}
%%%%%%%%%%%%%%%%%%%%%





The Lagrangian consists of the SM part and the part involving new physics:
\begin{eqnarray}
\mathcal{L} &=& \mathcal{L}_\text{SM} + \mathcal{L}_\text{NP},
\end{eqnarray}
where
\begin{eqnarray}
\mathcal{L}_\text{NP} & \supset &
	- \; \lambda_{\phi H} |\phi|^2 |H|^2
	- \lambda_{\chi H} |\chi|^2 |H|^2
	- \left[
	y \overline{\ell_L} L_R \chi +
	w \overline{q_L} Q_R \phi + h.c.
	\right] 
	- V_0(\phi,\chi) \nonumber	\\
&&
	- \; ( M_L \overline{L_L} L_R
	+ M_Q \overline{Q_L} Q_R	+ h.c. ) \, .
\label{Lnp}
\end{eqnarray}
%
In this equation, the SM left-handed lepton and quark doublets are denoted as
\begin{eqnarray}
\ell_L^i = 
	\begin{pmatrix}
	\nu^e_L \\ e_L
	\end{pmatrix}_i ,	\qquad
q_L^i =
	\begin{pmatrix}
	u_L	\\ d_L
	\end{pmatrix}_i ,	\qquad
(i = 1,2,3) .
\end{eqnarray}
%
The mass terms of the vectorlike leptons and quarks are allowed in the Lagrangian (\ref{Lnp}) with the corresponding mass matrices:
\begin{eqnarray}
M_L = 
	\begin{pmatrix}
	m_N	&	0	\\
	0	&	m_E	
	\end{pmatrix} , 
	\quad
%
M_Q	=
	\begin{pmatrix}
	m_U	&	0	\\
	0	&	m_D	
	\end{pmatrix}	.
\end{eqnarray}
%
In our analysis, for simplicity, we assume the mass degeneration in the two vectorlike doublets, i.e.
$m_N = m_E = m_L$,
$m_U = m_D = m_Q$.
%
The scalar potential $V_0(\chi,\phi)$ is given explicitly by
\begin{eqnarray}
V_0(\chi,\phi) &=&
	\lambda_\phi |\phi|^4 + m^2_\phi |\phi|^2 +
	\lambda_\chi |\chi|^4 + m^2_\chi |\chi|^2 + 
	\lambda_{\phi\chi} |\phi|^2 |\chi|^2 +
	\left( r \phi \chi^2 + h.c. \right) .
\label{V0}
\end{eqnarray}



We assume that the gauge group $U(1)_X$ is spontaneously broken by the vacuum expectation value (VEV) of the scalar $\phi$,
\begin{eqnarray}
\langle \phi \rangle &=& 		
	\sqrt{\frac{-m'^2_\phi}	
		{2\lambda_\phi}}	,
\end{eqnarray}
while the other scalar $\chi$ does not develop a nonzero VEV.
%
In the above equation, we denote
\begin{eqnarray}
m'^2_\phi &=& m^2_\phi + \lambda_{\phi H} \langle H \rangle^2	,
\end{eqnarray}
where $\langle H \rangle = 174$ GeV is the VEV of the SM Higgs field.
%
Due to the nonzero VEV, $\langle \phi \rangle$, the $U(1)_X$ gauge boson $Z'$ acquires a mass
\begin{eqnarray}
m_{Z'} &=& 2\sqrt{2} g_X \langle \phi \rangle	.
\label{mZp}
\end{eqnarray}
with $g_X$ being the $U(1)_X$ gauge coupling.


Decomposing the complex scalar field $\phi$ into the real and imaginary components,
\begin{eqnarray}
\phi &=& 
	\langle \phi \rangle + 
	\frac{1}{\sqrt{2}}
	\left( \varphi_r + i \varphi_i \right)	,
\label{phi}
\end{eqnarray} 
their masses are respectively found to be
\begin{eqnarray}
m_{\varphi_r} &=& 
	2 \sqrt{\lambda_\phi} \langle \phi \rangle,	\\
m_{\varphi_i} &=& 	0	\,	.
\end{eqnarray}
%
In the unitary gauge, the massless Nambu-Goldstone boson $\varphi_i$ is absorbed by the $Z'$ gauge boson.
%
For the scalar field $\chi$, by the similar decomposition
\begin{eqnarray}
\chi &=& 
	\frac{1}{\sqrt{2}}
	\left( \chi_r + i \chi_i \right)	\, ,
\end{eqnarray} 
the $2\times 2$ mass matrix for these real component fields is derived as
\begin{eqnarray}
\frac{1}{2}
	\begin{pmatrix}
	\chi_r	&	\chi_i
	\end{pmatrix}
M^2_\chi
	\begin{pmatrix}
	\chi_r	\\	\chi_i
	\end{pmatrix}
&=&
\frac{1}{2}
	\begin{pmatrix}
	\chi_r	&	\chi_i
	\end{pmatrix}
	\begin{pmatrix}
	m'^2_\chi + (r+r^*) \langle \phi \rangle	&	i(r-r^*) \langle \phi \rangle	\\
	i(r-r^*) \langle \phi \rangle	&	m'^2_\chi - (r+r^*) \langle \phi \rangle
	\end{pmatrix}
	\begin{pmatrix}
	\chi_r	\\	\chi_i
	\end{pmatrix}	,
\end{eqnarray}
where
\begin{eqnarray}
m'^2_\chi	&=&
	m^2_\chi + 
	\lambda_{\chi H} \langle H \rangle^2 +
	\lambda_{\phi\chi} \langle \phi \rangle^2	.
\end{eqnarray}
In the case that the coupling $r$ is real, the matrix $M_\chi^2$ is diagonal, and the masses of the particles $\chi_r$ and $\chi_i$ are respectively
\begin{eqnarray}
m_{\chi_r}^2	&=&	m'^2_\chi + 2r \langle \phi \rangle	,	\\
m_{\chi_i}^2	&=&	m'^2_\chi - 2r \langle \phi \rangle	.
\end{eqnarray}


%Mixing between vector-like and SM fermions

In the lepton sector, there is no mass mixing between the SM leptons and the vectorlike ones because 
$\langle \chi \rangle =0$.
%
However, in the quark sector, the VEV of $\phi$ generates mass mixing terms among the SM quarks and the vectorlike ones via the new exotic Yukawa interactions with the couplings 
$w = ( w_1, w_2, w_3 )$ in Eq. (\ref{Lnp}).
To diagonalize the quark mass matrices, $M^u$ and $M^d$, four $4\times 4$ unitary matrices are necessary to transform the quark gauge eigenstates, 
$(u^1, u^2, u^3, U)$ and 
$(d^1, d^2, d^3, D)$,
into the mass eigenstates, 
$(u, c, t, \mathcal{U})$ and 
$(d, s, b, \mathcal{D})$:
\begin{eqnarray}
\begin{pmatrix}
u_{L,R} \\	c_{L,R} \\	t_{L,R} \\	\mathcal{U}_{L,R}
\end{pmatrix}	=
	\left( V^u_{L,R} \right)_{4 \times 4}
	\begin{pmatrix}
	u_{L,R}^1 \\	u_{L,R}^2 \\
	u_{L,R}^3 \\	U_{L,R}	
	\end{pmatrix}	,	\qquad
%
\begin{pmatrix}
d_{L,R} \\	s_{L,R} \\	b_{L,R} \\	\mathcal{D}_{L,R}
\end{pmatrix}	=
	\left( V^d_{L,R} \right)_{4 \times 4}
	\begin{pmatrix}
	d_{L,R}^1 \\	d_{L,R}^2 \\
	d_{L,R}^3 \\	D_{L,R}	
	\end{pmatrix}	.
\label{qmix}
\end{eqnarray}
As a consequence, the diagonal mass matrices of the up-type and down-type quarks are then given by
\begin{eqnarray}
M^u_\text{diag}	&=&	V^u_L M^u (V^u_R)^\dagger	,	\\
M^d_\text{diag}	&=&	V^d_L M^d (V^d_R)^\dagger	.
\end{eqnarray}









%%%%%%%%%%%%%%%%%%%%%%%%%%%%%%%
\section{New physics contributions to the Wilson coefficient $C_7$}
%%%%%%%%%%%%%%%%%%%%%%%%%%%%%%%





The effective Hamiltonian describing the $ b \rightarrow s \gamma$ transitions is given by
\cite{Buchalla:1995vs, Greub:1996tg}
%
\begin{eqnarray}
\mathcal{H}_{\text{eff}}&=&
	-\frac{4 G_F}{\sqrt{2}}V_{tb}V^*_{ts}\sum_{i=1}^{8} (C_i \mathcal{O}_i + C_i' \mathcal{O}_i') + h.c.
\end{eqnarray}
%
Here, the operators most directly relevant to the $b\rightarrow s \gamma$ process are:
\begin{eqnarray}
\mathcal{O}_7 &=&
	\frac{e}{16\pi^{2}} m_b \left( \overline{s_L} \sigma^{\mu\nu} b_R \right) F_{\mu\nu} , \\
\mathcal{O}'_7 &=&
	\frac{e}{16\pi^{2}} m_s \left( \overline{s_R} \sigma^{\mu\nu} b_L \right) F_{\mu\nu} .
\end{eqnarray}
Due to the suppression factor $\frac{m_s}{m_b} \ll 1$ emerging from the mass-insertion on the external $s$ quark, the contribution of the operator $\mathcal{O}'_7$ is negligible in comparison to that of $\mathcal{O}_7$.


%%%%%%%%%%%%%%%%%%%%%%%%%%%%%%%%%%
\begin{figure}[h!]
\begin{center}
\includegraphics[scale=1]{1bsg.pdf}
\caption{Leading new physics contributions to the $b\rightarrow s \gamma$ transition.}
\label{Feynmandiagram}
\end{center}
\end{figure}
%%%%%%%%%%%%%%%%%%%%%%%%%%%%%%%%%%


In the considered model, the Feynman diagrams in the unitary gauge corresponding to the leading new physics contributions to the $b\rightarrow s \gamma$ transition are depicted in Figure \ref{Feynmandiagram}.
%
We see that beside the contributions of the new particles such as the vectorlike quarks $\mathcal{U}$, $\mathcal{D}$, the $U(1)_X$ gauge boson $Z'$, and the scalar $\varphi_r$, there are also contributions due to the new coupling among the SM particles 
as shown in Figures \ref{Feynmandiagram}b and \ref{Feynmandiagram}d.
%
These new couplings are induced from the mixings between the SM and the vectorlike quarks \cite{Hieu:2020hti}.


Utilizing the package FeynCalc 
\cite{Mertig:1990an, Shtabovenko:2016sxi, Shtabovenko:2020gxv} for the analytic manipulation of Dirac matrices
 with some further algebraic calculations, the leading new physics contributions to the  Wilson coefficient $ C_7$ have been derived. 
The results are then cross-checked numerically with
 the program Package-X version 2.1.1 
\cite{Patel:2015tea, Patel:2016fam}.
%
The analytic expression of $C_7^\text{NP}$ is found to be
%
\begin{eqnarray}
C_7^\text{NP}	
%&=&
%	\frac{ 1}{4\sqrt{2}G_F(V_{tb}V^*_{ts}) m_b}
%	\int_0^1 dx \int_0^{1-x} dy 
%	\left[
%		F_{W\mathcal{U}\mathcal{U}}
%		+ F_{Zqq}
%		+ F_{Z' qq}
%		+ F_{hqq}
%		+ F_{\varphi_r qq}
%		+ F_{WW\mathcal{U}}
%	\right]
%	\nonumber	\\
&=&
	\frac{ 1}{4\sqrt{2}G_F(V_{tb}V^*_{ts}) m_b}
	\int_0^1 dx \int_0^{1-x} dy 
	\left[
		F_{\mathcal{V}ff} 
\Big \vert_{\mathcal{V}=W, \, 		f=\mathcal{U}} +
		\sum_{\mathcal{V}=Z, Z'} \sum_{f=d,s,b,\mathcal{D}} F_{\mathcal{V}ff} 
	\right.
	\nonumber	\\
&&	
	\hspace{7cm}
	\left.
		+ \, F_{WW\mathcal{U}}
		+ \sum_{\mathcal{S}=h, \varphi_r} \sum_{f=d,s,b,\mathcal{D}} F_{\mathcal{S}ff} 
	\right]
\end{eqnarray}
%
where the loop functions are given by
\begin{eqnarray}
F_{\mathcal{V}ff}	&=&	
	\frac{ Q_f}{m_\mathcal{V}^2}
	\left\lbrace A_1^{\mathcal{V}ff} 
	\Bigl[ 
		(1-3x)
		\ln \bigl[ \beta_{f\mathcal{V}} (x+y)+1-x-y \bigr]
		- x   
	\Bigr] 
	- \frac{A_2^{\mathcal{V}ff}}{\beta_{f\mathcal{V}} (x+y)+1-x-y}   \right\rbrace	
	\nonumber \\		\\
%
F_{\mathcal{S}ff}	&=&
	\frac{ Q_f}{m_\mathcal{S}^2} \times
	\frac{B^{\mathcal{S}ff}}{\beta_{f\mathcal{S}} (x+y)+1-x-y}	\\
%
F_{WW\mathcal{U}}	&=&	
	\frac{ 1}{m_W^2}  
	\left\lbrace 
		C_1^{WW\mathcal{U}}  
		\ln \bigl[ (1-\beta_{\mathcal{U}W} ) (x+y)+\beta \bigr]  
		- C_2^{WW\mathcal{U}} 
		+ \frac{C_3^{WW\mathcal{U}} }{(1-\beta_{\mathcal{U}W} ) (x+y)+\beta_{\mathcal{U}W} }   \right\rbrace	
\end{eqnarray}


In the above equations, we have defined 
\begin{eqnarray}
A_1^{\mathcal{V}ff}	&=&	
	g_A^{bf\mathcal{V}} g_A^{sf\mathcal{V}} (2m_f +m_b+m_s) 
	+ g_V^{bf\mathcal{V}} g_V^{sf\mathcal{V}} (-2m_f +m_b+m_s) \nonumber \\
&&	+ \; g_V^{bf\mathcal{V}} g_A^{sf\mathcal{V}} (2m_f -m_b+m_s) 
	+ g_A^{bf\mathcal{V}} g_V^{sf\mathcal{V}} (-2m_f -m_b+m_s)	, \\
%
%
A_2^{\mathcal{V}ff}	&=&
	g_A^{bf\mathcal{V}} g_A^{sf\mathcal{V}} 
	\Bigl\{ -4m_f (x+y-1) + (m_b + m_s)\bigl[ \beta_{f\mathcal{V}} (x+y)x + 2(x+y-1)(x-1) \bigr]  
	\Bigr\}  \notag	\\
&&	+\; g_A^{bf\mathcal{V}} g_V^{sf\mathcal{V}} 
	\Bigl\{ 
	4m_f (x+y-1) + (-m_b + m_s) \bigl[ \beta_{f\mathcal{V}} (x+y)x + 2(x+y-1)(x-1) \bigr]  
	\Bigr\}  \notag	\\
&&	+\; g_V^{bf\mathcal{V}} g_A^{sf\mathcal{V}} 
	\Bigl\{ 
	-4m_f (x+y-1) + (-m_b + m_s)\bigl[ \beta_{f\mathcal{V}} (x+y)x + 2(x+y-1)(x-1) \bigr]   
	\Bigr\}  \notag	\\
&&	+\; g_V^{bf\mathcal{V}} g_V^{sf\mathcal{V}} 
	\Bigl\{ 
	4m_f (x+y-1) + (m_b + m_s) \bigl[ \beta_{f\mathcal{V}} (x+y)x + 2(x+y-1)(x-1) \bigr]  
	\Bigr\} ,	\nonumber	\\	
%
\end{eqnarray}
\begin{eqnarray}
%
B^{\mathcal{S}ff}	&=&
%	g^{bf\mathcal{S}} g^{sf\mathcal{S}} 
%	\Bigl[ m_f (x+y) - (m_b + m_s)x(x+y-1)  \vphantom{\frac{1}{2}} 
%	\Bigr] ,		\\
	g_P^{bf\mathcal{S}} g_P^{sf\mathcal{S}} 
	\Bigl[ m_f (x+y) + (m_b + m_s)x(x+y-1)  
	\Bigr]  \notag\\
&&	+\; g_P^{bf\mathcal{S}} g_S^{sf\mathcal{S}} 
	\Bigl[ m_f (x+y) - (m_s - m_b)x(x+y-1)  
	\Bigr]  \notag\\
&&	+\; g_S^{bf\mathcal{S}} g_P^{sf\mathcal{S}} 
	\Bigl[ m_f (x+y) + (m_s - m_b)x(x+y-1)  \vphantom{\frac{1}{2}} 
	\Bigr]  \notag\\
&&	+\; g_S^{bf\mathcal{S}} g_S^{sf\mathcal{S}} 
	\Bigl[ m_f (x+y) - (m_b + m_s)x(x+y-1)  \vphantom{\frac{1}{2}} 
	\Bigr] ,		
%
\end{eqnarray}
\begin{eqnarray}
%
C_1^{WW\mathcal{U}}	&=&
	g_A^{b\,\mathcal{U}W} g_A^{s\,\mathcal{U}W} 
	\Bigl\{ (m_b + m_s) \bigl[ 4x^2+2x(2y-3)+1 \bigr]  
	\Bigr\}  \notag\\
&&	+\; g_A^{b\,\mathcal{U}W} g_V^{s\,\mathcal{U}W}
	\Bigl\{ (-m_b + m_s) \bigl[ 4x^2+2x(2y-3)+1 \bigr]   
	\Bigr\}  \notag\\
&&	+\; g_V^{b\,\mathcal{U}W} g_A^{s\,\mathcal{U}W} 
	\Bigl\{ (-m_b + m_s) \bigl[4x^2+2x(2y-3)+1 \bigr]  
	\Bigr\}  \notag\\
&&	+\; g_V^{b\,\mathcal{U}W} g_V^{s\,\mathcal{U}W} 
	\Bigl\{ (m_b + m_s) \bigl[4x^2+2x(2y-3)+1 \bigr]  
	\Bigr\} ,	\\
%
%
C_2^{WW\mathcal{U}}	&=&
	g_A^{b\,\mathcal{U}W} g_A^{s\,\mathcal{U}W} 
	\Bigl\{ -m_\mathcal{U} - (m_b + m_s) \bigl[ x^2+x(y-2)+1 \bigr] 
	\Bigr\}  \notag\\
&&	+\; g_A^{b\,\mathcal{U}W} g_V^{s\,\mathcal{U}W} 
	\Bigl\{ m_\mathcal{U} + (m_b - m_s) \bigl[x^2+x(y-2)+1 \bigr]  
	\Bigr\}  \notag\\
&&	+\; g_V^{b\,\mathcal{U}W} g_A^{s\,\mathcal{U}W} 
	\Bigl\{ -m_\mathcal{U} + (m_b - m_s) \bigl[x^2+x(y-2)+1 \bigr]   
	\Bigr\}  \notag\\
&&	+\; g_V^{b\,\mathcal{U}W} g_V^{s\,\mathcal{U}W} 
	\Bigl\{ m_\mathcal{U} - (m_b + m_s) \bigl[x^2+x(y-2)+1 \bigr]  
	\Bigr\} ,	\\
%
%
%\end{eqnarray}
%\begin{eqnarray}
C_3^{WW\mathcal{U}}	&=&
	g_A^{b\,\mathcal{U}W} g_A^{s\,\mathcal{U}W} 
	\Bigl[ 3m_\mathcal{U} (x+y) + (m_b + m_s)(2x^2+2xy+y)  \Bigr] \notag\\
&&		
	+\; g_A^{b\,\mathcal{U}W} g_V^{s\,\mathcal{U}W} 
	\Bigl[ -3m_\mathcal{U} (x+y) + (-m_b + m_s)(2x^2+2xy+y)  \Bigr]  \notag\\
&&		
	+ \; g_V^{b\,\mathcal{U}W} g_A^{s\,\mathcal{U}W} 
	\Bigl[ 3m_\mathcal{U} (x+y) +\; (-m_b + m_s)(2x^2+2xy+y)  \Bigr]  \notag\\
&&		
	+\; g_V^{b\,\mathcal{U}W} g_V^{s\,\mathcal{U}W} 
	\Bigl[ -3m_\mathcal{U} (x+y) + (m_b + m_s)(2x^2+2xy+y)  \Bigr] ,
\end{eqnarray}
and $\beta_{ab} =\dfrac{m_a^2}{m_b^2}$.
%
In these above equations, $g_V$ and $g_A$ are the vector and axial-vector couplings between fermions and a gauge boson,
while $g_S$ and $g_P$ are the scalar and pseudo-scalar couplings between fermions and a scalar field.



%
%===================
%SM contribution is at NNLO,






%%%%%%%%%%%%%%%%%%%%%%%%%%%%%%%
\section{Numerical analysis}
%%%%%%%%%%%%%%%%%%%%%%%%%%%%%%%


In the numerical analysis, we start with a quark basis where the $3 \times 3$ blocks of the up-type and down-type quark mass matrices are diagonal with the values being the center experimental values of the SM quark masses.
%
In this basis, the full $4 \times 4$ quark mass matrices are represented as
\begin{eqnarray}
M^u	=
	\begin{pmatrix}
	m_u^0	&	0	&	0	&	w_1 \langle \phi \rangle	\\
	0	&	m_c^0	&	0	&	w_2 \langle \phi \rangle	\\
	0	&	0	&	m_t^0	&	w_3 \langle \phi \rangle	\\
	0	&	0	&	0	&	m_U
	\end{pmatrix} ,
	\quad
%
M^d	=
	\begin{pmatrix}
	m_d^0	&	0	&	0	&	w_1 \langle \phi \rangle	\\
	0	&	m_s^0	&	0	&	w_2 \langle \phi \rangle	\\
	0	&	0	&	m_b^0	&	w_3 \langle \phi \rangle	\\
	0	&	0	&	0	&	m_D
	\end{pmatrix} ,
\end{eqnarray}
%
Here, for simplicity, we consider the case of degenerate up-type and down-type vectorlike fermion masses, namely
$m_U = m_D = m_Q$ and
$m_N = m_E = m_L$.
%
%
After the diagonalization of these mass matrices, the masses of the SM quarks are deflected from their center values due to the mixing.
They are required to stay within the $2\sigma$ ranges of the experimental measurements.



In this model, the SM CKM matrix is a $3 \times 3$ block of a $4 \times 4$ matrix determined by 
$\left( V_L^u V_L^{d\dagger} \right)_{4\times 4}$.
Due to the mixing between the SM and the vectorlike quarks, the unitarity violation of the SM CKM matrix is obvious.
Here, we consider the experimental constraint on the sum of the squared modules of the first-row elements of the SM CKM matrix \cite{Workman:2022ynf}:
\begin{eqnarray}
0.9971
\leq |V_{ud}|^2 + |V_{us}|^2 + |V_{ub}|^2 \leq
0.9999 \quad (2 \sigma).
\label{ckm}
\end{eqnarray}


In our numerical analysis, 
we consider the current 2$\sigma$ range of the branching ratio of radiative $B$-meson decay \cite{HFLAV:2022pwe}:
\begin{eqnarray}
3.11 \times 10^{-4} < \text{BR}(B \rightarrow X_s \gamma) < 3.87 \times 10^{-4}.
\label{bsg_now}
\end{eqnarray}
%
Assuming the center value of this branching ratio remains unchanged, we can expect that the relative precision can reach 2\% with the luminosity 50 ab$^{-1}$ at the Belle II experiment
\cite{Belle-II:2022cgf, DiCanto:2022icc}.
%
The impact of the future results is investigated by taking into account the following expected 2$\sigma$ range:
\begin{eqnarray}
3.35 \times 10^{-4} < \text{BR}(B \rightarrow X_s \gamma)_{\text{Belle2}} < 3.63 \times 10^{-4}.
\label{bsg_belle2}
\end{eqnarray}




The constraints from semileptonic $B$-meson decays are also considered. They are the branching ratios of the processes 
$B^+ \rightarrow K^+ \mu^+ \mu^-$ 
\cite{BR, Aaij:2012vr, Aaij:2014pli},
$B^0 \rightarrow K^*(892)^{0} \mu^+ \mu^-$ 
\cite{BR, Aaij:2016flj}, and the observables
$R_K$ \cite{HFLAV:2022pwe, Aaij:2019wad, Aaij:2014ora, Aaij:2021vac}
and 
$R_{K^*}$ \cite{HFLAV:2022pwe, Aaij:2017vbb, Abdesselam:2019wac}
measuring the lepton flavor violation. 
The current 2$\sigma$ allowed ranges corresponding to these quantities for the muon invariant mass in the region 
$q^2 = [1.1, 6.0]$ GeV$^2$
are given as follows
%
\begin{eqnarray}
1.050 \times 10^{-7} < \text{BR}(B^+ \rightarrow K^+ \mu^+ \mu^-) < 1.322 \times 10^{-7} , 
\label{BRK}		\\
%
1.382 \times 10^{-7} < \text{BR}(B^0 \rightarrow K^*(892)^{0} \mu^+ \mu^-) < 1.970 \times 10^{-7} ,
\label{BRKs}		\\
%
0.762 < R_K = \frac{\text{BR}(B^+ \rightarrow K^+ \mu^+ \mu^-)}{\text{BR}(B^+ \rightarrow K^+ e^+ e^-)} < 0.930 , 
\label{RK}	\\
%
0.54 < R_{K ^*} = \frac{\text{BR}(B^0 \rightarrow K^*(892)^{0} \mu^+ \mu^-)}{\text{BR}(B^0 \rightarrow K^*(892)^{0} e^+ e^-)} < 0.96 .
\label{RKs}
\end{eqnarray}



The slepton searches at the ATLAS and CMS experiments at 13 GeV impose the constraints on the charged vectorlike lepton masses 
\cite{Aad:2014vma}
that satisfy either 
$m_L \gtrsim \mathcal{O}(1)$ TeV, or 
\begin{eqnarray}
m_L - m_{\chi_r} \lesssim 60 \text{ GeV}.
\label{vectorlikelepton}
\end{eqnarray}
%
%
On the other hand, in order to explain the deviation between the experimental value and the SM prediction of the muon anomalous magnetic moment
\cite{Tanabashi:2018oca, Abi:2021gix, Aoyama:2020ynm}
\begin{eqnarray}
\Delta a_\mu \equiv 
	a_\mu^\text{exp} - a_\mu^\text{SM}
	= (25.1 \pm 5.9) \times 10^{-10} \, ,
\label{g-2}
\end{eqnarray}
the vectorlike leptons must be light enough.
%


The set of free inputs of the model includes
\begin{eqnarray}
y_{1,2,3},
g_X,
w_{1,2,3},
m_{Z'},
\lambda_\phi,
m_Q,
m_{\chi_r},
\tau = \frac{m_L^2}{m_{\chi_r}^2},
\delta=\frac{m_{\chi_i}^2}{m_{\chi_r}^2} - 1.
\end{eqnarray}
%
Noticing that the new physics contribution to the muon $g-2$ depends on the parameters $y_2$, 
$m_{\chi_r}$,
$\tau = \frac{m_L^2}{m_{\chi_r}^2}$, and 
$\delta= \frac{m_L^2}{m_{\chi_r}^2}$,
to reconcile the two constraints in
Eqs. (\ref{vectorlikelepton}) and (\ref{g-2}),
we choose these parameters to be
%
$y_2 = 3$,
$m_{\chi_r} = 120$ GeV,
$\tau = 1.78$, and
$\delta = 1$, 
while $y_{1,3}$ are set to zero for simplicity
\cite{Dinh:2020inx}.
%
%
In order to suppress the new physics contributions to the FCNCs related to the first two generations, we consider the case of vanishing $w_1$.
%
Since the effects of $\lambda_\phi$ on the considered observables are negligible, we take $\lambda_\phi = 3$ in our numerical analysis as an example without lost of generality.
%
%
The remaining set of inputs are
$g_X$,
$w_2$, 
$w_3$,
$m_{Z'}$, and
$m_Q$.
%
For the $U(1)_X$ gauge couplings and the exotic Yukawa couplings, we further impose the perturbation limits as
\begin{eqnarray}
g_X,\, w_{2,3} \leqslant \sqrt{4\pi}.
\label{perturbation}
\end{eqnarray}


%==============================
%
%Constraint from LHC in searches for leptons, scalars, vectorlike quarks (~1-1.5 TeV: 
%2212.05263,
%2210.15413,
%2201.07045,
%1808.01771, 
%1806.10555, 
%1509.04261), vectorlike leptons
%
%=====================
%
%Explain about the choice of basis so that the 3x3 block of the mass matrix and 3x3 block of the CKM matrix is close to those the SM 
%$=>$ this is used for convenience in numerical analysis.
%
%==================================


%%%%%%%%%%%%%%%%%%%%%%%%%%%%%%%%%%
\begin{figure}[h!]
\begin{center}
\includegraphics[scale=0.65]{2gxBRbsg.pdf}
\caption{The branching ratio of the $b\rightarrow s \gamma$ transition as a function of $g_X$ when
$m_{Z'}=700$ GeV,
$m_Q = 1700$ GeV, 
%$\lambda_\phi = 3$,
%$w_1 = 0$,
$w_2 = 1.55$, 
$w_3 = 0.145$.
The green and yellow regions indicate the current bounds and expected ones at the Belle II experiment in the near future.
The vertical black dashed line indicates the perturbation limit (\ref{perturbation}).}
\label{BRgx}
\end{center}
\end{figure}
%%%%%%%%%%%%%%%%%%%%%%%%%%%%%%%%%%


The branching ratio of the $b\rightarrow s \gamma$ process is calculated according to the method in Refs. 
\cite{Misiak:2006ab, Czakon:2015exa, Misiak:2020vlo, HFLAV:2022pwe}.
%
%
In Figure \ref{BRgx}, this branching ratio is plotted as a function of $g_X$ in the benchmark case with 
$m_{Z'}=700$ GeV,
$m_Q = 1700$ GeV, 
%$\lambda_\phi = 3$,
%$w_1 = 0$,
$w_2 = 1.55$, 
$w_3 = 0.145$.
%
For this parameter setting, 
the most dominant contribution to this process come from the loop involving the $U(1)_X$ gauge boson $Z'$ in Figures \ref{Feynmandiagram}c.
%
Note that the coupling of $Z'$ and the down-type quark current proportional to the product of $g_X$ and the mixing matrices that, in turn, is roughly proportional to $g_X \langle \phi \rangle$.
Therefore, according to Eq. (\ref{mZp}), once $m_{Z'}$ is fixed, this coupling does not depend on $g_X$.
This explains the behavior of the branching ratio in Figure \ref{BRgx}.
%
This benchmark satisfies the current bounds (\ref{bsg_now}) depicted by the green region for the whole plotted range of $g_X$ upto the perturbation limit (the vertical black dashed line).
%
In the near future, if the center value of the branching ratio remains unchanged,
the 2$\sigma$ allowed region (\ref{bsg_belle2}) expected at the Belle II experiment is depicted in this figure as the yellow band.
The expected lower bound will be able to exclude a certain range of $g_X$ between 0.05 and 0.60 in this case.




%%%%%%%%%%%%%%%%%%%%%%%%%%%%%%%%%%
\begin{figure}[h!]
\begin{center}
\includegraphics[scale=0.65]{3m4BRbsg.pdf}
\caption{The branching ratio of the $b\rightarrow s \gamma$ transition as a function of $m_Q$ when
$m_{Z'}=700$ GeV,
$g_X = 2$ GeV, 
%$\lambda_\phi = 3$,
%$w_1 = 0$,
$w_2 = 1.55$, 
$w_3 = 0.145$.
The color codes are the same as those in Figure \ref{BRgx}.}
\label{BRm4}
\end{center}
\end{figure}
%%%%%%%%%%%%%%%%%%%%%%%%%%%%%%%%%%




In Figure \ref{BRm4}, $\text{BR}(b \rightarrow s \gamma)$ is shown as a function of the vectorlike quark mass $m_Q$ for the case of $g_X = 2$, while other parameters are the same as those in Figure \ref{BRgx}.
%
We observe that the branching ratio can be significantly enhanced by larger values of the vectorlike quark mass $m_Q$.
As $m_Q$ increases, the vector-like quarks gradually decouple from the SM sector at low energies.
Therefore, the branching ratio in this model approaches the SM limit for large values of $m_Q$.
%
The current $2\sigma$ bounds (\ref{bsg_now}) of this branching ratio (the green region) set the lower limit on $m_Q$ that is approximately 810 GeV.
%
After getting the results from the Belle II experiment, the expected 2$\sigma$ bounds in Eq. (\ref{bsg_belle2}) (the yellow region) will raise the lower limit of $m_Q$ up to about 1600 GeV.




%%%%%%%%%%%%%%%%%%%%%%%%%%%%%%%%%%
\begin{figure}[h!]
\begin{center}
\includegraphics[scale=0.65]{4mzpBRbsg.pdf}
\caption{The branching ratio of the $b\rightarrow s \gamma$ transition as a function of $m_{Z'}$ when
$m_{Q}=1700$ GeV,
$g_X = 2$, 
%$\lambda_\phi = 3$,
%$w_1 = 0$,
$w_2 = 1.55$, 
$w_3 = 0.145$.
The color codes are the same as those in Figure \ref{BRgx}.}
\label{BRmzp}
\end{center}
\end{figure}
%%%%%%%%%%%%%%%%%%%%%%%%%%%%%%%%%%


The dependence of $\text{BR}(b\rightarrow s\gamma)$ on $m_{Z'}$ is depicted in Figure \ref{BRmzp} for the case with $g_X = 2$ and other parameters are the same as those in Figure \ref{BRgx}.
%
We observe that the transition rate is slightly reduced when $m_{Z'}$ is increased.
Since the main contribution comes from the diagram in Figure \ref{Feynmandiagram}c, this dependence is not strong for a fixed value of $g_X$.
% because there is a compensation between the dependences of the squared coupling and Z' propagator on $m_{Z'}$ in the loop diagram of Figure 1c.
%
The branching ratio satisfies the current 2$\sigma$ bounds (\ref{bsg_now}) (the green region) for the whole considered range of $m_{Z'}$.
%
However, it is expected that the constraint (\ref{bsg_belle2}) from the future Belle II result (the yellow region) will set a severe upper limit on the $Z'$ boson mass to be about 870 GeV for this benchmark case.




%%%%%%%%%%%%%%%%%%%%%%%%%%%%%%%%%%
\begin{figure}[h!]
\begin{center}
\includegraphics[scale=0.65]{5W2BRbsg.pdf}
\caption{The branching ratio of the $b\rightarrow s \gamma$ transition as a function of $w_2$ when
$m_{Q}=1700$ GeV,
$m_{Z'} = 700$ GeV,
$g_X = 2$, 
%$\lambda_\phi = 3$,
%$w_1 = 0$,
%$w_2 = 1.55$, 
$w_3 = 0.145$.
The color codes are the same as those in Figure \ref{BRgx}.}
\label{BRw2}
\end{center}
\end{figure}
%%%%%%%%%%%%%%%%%%%%%%%%%%%%%%%%%%


%%%%%%%%%%%%%%%%%%%%%%%%%%%%%%%%%%
\begin{figure}[h!]
\begin{center}
\includegraphics[scale=0.65]{6W3BRbsg.pdf}
\caption{The branching ratio of the $b\rightarrow s \gamma$ transition as a function of $w_3$ when
$m_{Q}=1700$ GeV,
$m_{Z'} = 700$ GeV,
$g_X = 2$, 
%$\lambda_\phi = 3$,
%$w_1 = 0$,
$w_2 = 1.55$. 
%$w_3 = 0.145$.
The color codes are the same as those in Figure \ref{BRgx}.}
\label{BRw3}
\end{center}
\end{figure}
%%%%%%%%%%%%%%%%%%%%%%%%%%%%%%%%%%





Figure \ref{BRw2} shows the dependence of BR$(b \rightarrow s \gamma)$ on the parameter $w_2$ in the case with 
$m_{Q}=1700$ GeV,
$m_{Z'} = 700$ GeV,
$g_X = 2$, 
%$\lambda_\phi = 3$,
%$w_1 = 0$,
%$w_2 = 1.55$, 
$w_3 = 0.145$.
It is observed that the branching ratio is slightly reduced when $w_2$ increases.
For this benchmark, the branching ratio stays in the green region allowed by the current constraint (\ref{bsg_now}) on the $b \rightarrow s \gamma$ transition for the whole range of $w_2$ upto the perturbation limit.
In the near future, the Belle II result (\ref{bsg_belle2}) is expected to set the upper limit of about 1.8 on the parameter $w_2$ 
that is well below the perturbation limit (\ref{perturbation}).
%
%
%
The branching ratio of the process $b\rightarrow s \gamma$ is plotted as a function of the parameter $w_3$ in Figure \ref{BRw3}.
Similar to Figure \ref{BRw2}, here we observe that the branching ratio is also inversely proportional to $w_3$.
The current constraint (\ref{bsg_now}) requires that $w_3$ must be smaller than 0.9.
Due to the strong dependence of BR$(b\rightarrow s \gamma)$ on $w_3$, this upper bound is expected to be reduced to about 0.17 by the foreseen constraint (\ref{bsg_belle2}) after the Belle II experiment accumulates enough data.












%==============================
%Combined rare $B$ decay constraints
%==============================



%==============
%Plot of RK, RK*, BR(B to K mumu), BR(B to K* mumu)
%==============


%%%%%%%%%%%%%%%%%%%%%%%%%%%%%%%%%%
\begin{figure}[h!]
\begin{center}
\includegraphics[scale=0.65]{4bsllMzpMQ.pdf}
\caption{Semileptonic decay constraints
(Eqs. (\ref{BRK})-(\ref{BRKs})) at the level of 2$\sigma$ on the 
$(m_{Z'}, m_Q)$ plane
 when
%$m_{Z'}=700$ GeV,
$g_X = 2$, 
%$\lambda_\phi = 3$,
%$w_1 = 0$,
$w_2 = 1.55$, 
$w_3 = 0.145$.
}
\label{bsllmzpmq}
\end{center}
\end{figure}
%%%%%%%%%%%%%%%%%%%%%%%%%%%%%%%%%%



In Figure \ref{bsllmzpmq}, the constraints 
on the semileptontic decays of $B$ mesons are plotted on the ($m_{Z'}, m_Q$) plane for the case with
$g_X = 2$, 
%$\lambda_\phi = 3$, 
$w_2 = 1.55$, and
%$w_1 = 0$,
%$w_2 = 1.55$, and
$w_3 = 0.145$.
The 2$\sigma$ allowed regions corresponding to the constraints 
%(Eqs. (\ref{BRK})-(\ref{BRKs})) 
on 
BR$(B^+ \rightarrow K^+ \mu^+ \mu^-)$
(\ref{BRK}),
$R_K$ (\ref{RK}), 
BR($B^0 \rightarrow K^{0*} \mu^+ \mu^-$)
(\ref{BRKs}), and
$R_{K^*}$ (\ref{RKs})
are shown as the red back-hatched, the blue hatched, the yellow and the green areas.
Each constraint has two allowed regions, the narrow one corresponding to small values of both $m_{Z'}$ and $m_Q$, 
and
the wider region including points with larger values of these two parameters.
%
From this figure, we see that only the latter has an overlapping thin strip satisfying all the above four constraints.
%
Therefore, the constraints on the semileptonic decays of $B$ mesons imply a strict correlation of $m_Q$ and $m_{Z'}$.



%==============================
%Combined rare $B$ decay constraints
%==============================




%%%%%%%%%%%%%%%%%%%%%%%%%%%%%%%%%%
\begin{figure}[h!]
\begin{center}
\includegraphics[scale=0.65]{MzpMQ.pdf}
\caption{Constraints on the $(m_{Z'},m_Q)$ plane
 when
%$m_{Q}=1700$ GeV,
$g_X = 2$, 
%$\lambda_\phi = 3$,
%$w_1 = 0$,
$w_2 = 1.55$, 
$w_3 = 0.145$.
The light and the dark green colors indicate the current 2$\sigma$ allowed region (\ref{bsg_now})
and the expected one after the Belle II experiment (\ref{bsg_belle2}).
The constraints on the SM quark masses are shown by the red hatched region.
The blue back-hatched region satisfies the constraint  (\ref{ckm}) on the CKM unitarity violation of the first row.}
\label{mzpmq}
\end{center}
\end{figure}
%%%%%%%%%%%%%%%%%%%%%%%%%%%%%%%%%%


The above overlapping thin region is extracted and plotted in Figure \ref{mzpmq} as a yellow strip.
In addition, we consider other constraints on the $b\rightarrow s \gamma$ transition (\ref{bsg_now}),
the SM quark masses, and
the violation of the CKM unitarity in the first row (\ref{ckm}).
%
%The constraints on the plane ($m_{Z'}, m_Q$) are plotted in Figure \ref{mzpmq} for the case with
%$g_X = 2$, 
%$\lambda_\phi = 3$,
%$w_1 = 0$,
%$w_2 = 1.55$, and
%$w_3 = 0.145$.
%
In this figure, the CKM unitarity violation (the blue back-hatched region) imposes the allowed range on the mass ratio $\frac{m_Q}{m_{Z'}}$ to be [1.1, 6.4].
For a given value of $m_Q$, the region with too large $m_{Z'}$ generates large mixing between the vectorlike and the SM quarks causing the CKM to be too far away from unitarity. Hence, it is excluded.
On the other hand, the region with too small $m_{Z'}$ does not generate enough violation of the CKM unitarity according to the experimental result in Eq. (\ref{ckm}). Thus, this region is also not favored.
%
%
The lower bound on $m_Q$ set by the current constraint on the $b\rightarrow s \gamma$ decay (the light green region) is 500 GeV.
This bound is expected to rise up to nearly 1300 GeV after the Belle II experiment imposes a more restrictive constraint (\ref{bsg_belle2}) (the dark green region).
%
%
Taking into account the constraints (\ref{BRK})-(\ref{RKs}) from the data of the semileptonic decays $b\rightarrow s \ell^+ \ell^-$ (the yellow strip),
we obtain even more severe allowed range for $m_{Z'}$ and $m_Q$.
%
The current allowed ranges for $m_{Z'}$ and $m_Q$ are
[264, 1500] GeV and
[1610, 1750] GeV, respectively.
The allowed ranges for these two parameters after the Belle II experiment are expected to be reduced to
[264, 810] GeV and
[1670, 1750] GeV.










%%%%%%%%%%%%%%%%%%%%%%%%%%%%%%%%%%
\begin{figure}[h!]
\begin{center}
\includegraphics[scale=0.65]{MQgx.pdf}
\caption{Constraints on the $(m_Q,g_X)$ plane
 when
$m_{Z'}=700$ GeV,
%$g_X = 2$, 
%$\lambda_\phi = 3$,
%$w_1 = 0$,
$w_2 = 1.55$, 
$w_3 = 0.145$.
The color codes are the same as those in Figure \ref{mzpmq}.
The horizontal black dashed line indicates the perturbation limit (\ref{perturbation}).
}
\label{mqgx}
\end{center}
\end{figure}
%%%%%%%%%%%%%%%%%%%%%%%%%%%%%%%%%%


In Figure \ref{mqgx}, we show the 
2$\sigma$ constraints on the plane of ($m_Q, g_X$) with fixed values of other parameters:
$m_{Z'} = 700$ GeV,
%$\lambda_\phi = 3$,
%$w_1 = 0$,
$w_2 = 1.55$, and
$w_3 = 0.145$.
%
Using the same color codes as those in Figure \ref{mzpmq},
the red hatched region satisfies the constraints on the SM quark masses, while the blue back-hatched region indicates the constraint (\ref{ckm}) from the measurements of the first-row elements of the CKM.
We observe that these two constraints are compatible, with the one on the CKM unitarity violation being more severe.
It excludes a significant part of the parameter space with small values of $m_Q$ and $g_X$ that corresponds to the region with too large mixing between the SM and the vectorlike quarks.
%
The region with too small mixing, corresponding to too large $g_X$ and $m_Q$ (the upper right corner of the plot), is also excluded since it could not generate enough unitarity violation according to Eq. (\ref{ckm}).
%
The constraints on the $b\rightarrow s \gamma$ decay derived from the current experimental data (\ref{bsg_now}) and the expected Belle II data
(\ref{bsg_belle2}) are depicted in this plot by the light and dark green regions, respectively.
While the current bounds exclude the white region on the left with both $m_Q \lesssim 700$ GeV and 
$g_X \gtrsim \mathcal{O}(1)$ simultaneously,
the expected result from Belle II experiment will be able to set the lower limit on $m_Q$ to be about 1650 GeV.
%
%
The yellow strip in this figure indicates the combined constraint from the
$b \rightarrow s \ell^+ \ell^-$ transitions ((\ref{BRK})-(\ref{RKs})).
The overlap between this thin strip and the constraint on the CKM unitarity violation severely restrict the allowed range for $m_Q$, which is about 1700 GeV for this benchmark point.
%
Since the dark green area only marginally overlaps the yellow strip, the Belle II experiment will be able to test the allowed region in the near future.




%%%%%%%%%%%%%%%%%%%%%%%%%%%%%%%%%%
\begin{figure}[h!]
\begin{center}
\includegraphics[scale=0.65]{MQW2Positive.pdf}
\caption{Constraints on the $(m_Q,w_2)$ plane
 when
$m_{Z'}=700$ GeV,
$g_X = 2$, 
%$\lambda_\phi = 3$,
%$w_1 = 0$,
%$w_2 = 1.55$, 
$w_3 = 0.145$.
The color codes are the same as those in Figure \ref{mzpmq}.
The horizontal black dashed line indicates the perturbation limit (\ref{perturbation}).
}
\label{mqw2}
\end{center}
\end{figure}
%%%%%%%%%%%%%%%%%%%%%%%%%%%%%%%%%%


The constraints on the ($m_Q, w_2$) plane
are shown in Figure \ref{mqw2}.
Here, the benchmarks for other parameters are chosen to be 
$m_{Z'}=700$ GeV,
$g_X = 2$, 
%$\lambda_\phi = 3$,
%$w_1 = 0$,
%$w_2 = 1.55$, 
$w_3 = 0.145$.
%
The region with small $w_2$ corresponding to very little mixing between the SM and vectorlike quarks 
does not satisfy the constraint (\ref{ckm}) on the CKM unitarity violation.
Therefore, this constraint represented by blue back-hatched regions requires $w_2$ to have sizable values.
%
Due to the interplay between $w_2$ and $m_Q$ in terms of their effects on the 
BR$(b \rightarrow s \gamma)$ and the CKM unitarity violation, a large value of $w_2$ is not enough if the vectorlike quarks are relatively light.
It is observed that the combination of the two constraints (\ref{ckm}) and (\ref{bsg_now}) implies that $m_Q$ must be larger than about 300 GeV.
It also ensures that the constraint on the SM quark masses (the red hatched region) is fulfilled.
%
The lower limit on $m_Q$ rises to about 600 GeV when we further take into account the constraints ((\ref{BRK})-(\ref{RKs}))
on the semileptonic rare decays of $B$ mesons (the yellow region).
This set of constraints also imposes a lower limit of about 0.2 on the parameter $w_2$.
%
In the near future, the Belle II experiment is expected to push the lower bound of $m_Q$ up to about 1450 GeV, and the lower bound of $w_2$ to 
about 1.1 for this benchmark when superimposing the dark green area and the yellow one.
%
The upper bound on $m_Q$, in this case, is about 2580 GeV that is determined by the perturbation limit (\ref{perturbation}) on the Yukawa coupling $w_2$ (the horizontal black dashed line).





%%%%%%%%%%%%%%%%%%%%%%%%%%%%%%%%%%
\begin{figure}[h!]
\begin{center}
\includegraphics[scale=0.65]{MQW3Positive.pdf}
\caption{Constraints on the $(m_Q,w_3)$ plane
 when
$m_{Z'}=700$ GeV,
$g_X = 2$, 
%$\lambda_\phi = 3$,
%$w_1 = 0$,
$w_2 = 1.55$. 
%$w_3 = 0.145$.
The color codes are the same as those in Figure \ref{mzpmq}.
}
\label{mqw3positive}
\end{center}
\end{figure}
%%%%%%%%%%%%%%%%%%%%%%%%%%%%%%%%%%


In Figure \ref{mqw3positive}, we show the allowed regions with respect to the considered constraints on the plane ($m_Q, w_3$) for the case with
$m_{Z'}=700$ GeV,
$g_X = 2$, and
%$\lambda_\phi = 3$,
%$w_1 = 0$, and
$w_2 = 1.55$.
%
In this case, the lower bound for $m_Q$ is determined by the constraint on the CKM unitarity violation (\ref{ckm}) (the blue back-hatched region) to be about 770 GeV for $w_3 \lesssim 0.5$.
It is due to the fact that smaller $m_Q$ will lead to too much mixing between the SM and the vectorlike quark.
For $w_3 \gtrsim 0.5$, the constraint on the SM quark masses becomes severe and rule out most of the parameter space because of the same reason as the above case with small $m_Q$.
%
The current constraint (\ref{bsg_now}) on the $b \rightarrow s \gamma$ decay (the light green region) excludes further some portion of the parameter space with relatively large $w_3$.
%
The constraints (\ref{BRK})-(\ref{RKs})
on the semileptonic decay  $b \rightarrow s \ell^+ \ell^-$ play an important role in excluding the parameter space such that only a thin yellow strip survives on this plot.
Although this yellow strip satisfies the current constraint on $BR(b\rightarrow s \gamma)$, the expected result at the Belle II experiment (the dark green region) will be able to rule out a certain part of this strip with $m_Q \lesssim 1360$ GeV.
%
Since $w_3$ increase with $m_Q$, the Belle II experiment will also impose a lower bound of $\mathcal{O}(0.1)$ on the parameter $w_3$.









%%%%%%%%%%%%%%%%%%%%%%%%%%%%%%%%%%
\begin{figure}[h!]
\begin{center}
\includegraphics[scale=0.65]{Mzpgx.pdf}
\caption{Constraints on the $(m_{Z'},g_X)$ plane
 when
$m_{Q}=1700$ GeV,
%$g_X = 2$, 
%$\lambda_\phi = 3$,
%$w_1 = 0$,
$w_2 = 1.55$, 
$w_3 = 0.145$.
The color codes are the same as those in Figure \ref{mzpmq}.
The horizontal black dashed line indicates the perturbation limit (\ref{perturbation}).
}
\label{mzpgx}
\end{center}
\end{figure}
%%%%%%%%%%%%%%%%%%%%%%%%%%%%%%%%%%




The allowed regions on the plane ($m_{Z'}, g_X$) are plotted in Figure \ref{mzpgx}
in the case with $m_{Q}=1700$ GeV,
%$g_X = 2$, 
%$\lambda_\phi = 3$,
%$w_1 = 0$,
$w_2 = 1.55$, and
$w_3 = 0.145$.
With this choice of benchmark, the considered parameter space in the figure satisfies the current constraint
(\ref{bsg_now}) on the $b\rightarrow s \gamma$ decay.
However, the upcoming result (\ref{bsg_belle2}) at the Belle II experiment (the dark green region)
will be expected to set a severe upper bound on $m_{Z'}$ to be about 760 GeV.
%
On the one hand, the constraint (\ref{ckm}) on the CKM unitarity violation (the blue back-hatched region) imposes a lower bound of about 130 GeV on the ratio 
$\frac{m_{Z'}}{g_X}$.
The region with the ratio smaller than this limit is excluded because it generates too small mixing between the SM and the vectorlike quarks.
%
On the other hand, the combination of constraints (\ref{BRK})-(\ref{RKs})
on the semileptonic decays $b \rightarrow s \ell^+ \ell^-$ (the yellow region) sets an upper bound on $\frac{m_{Z'}}{g_X}$ to be about 540 GeV.













%%%%%%%%%%%%%%%%%%%%%%%%%%%%%%%%%%
\begin{figure}[h!]
\begin{center}
\includegraphics[scale=0.65]{MzpW2Positive.pdf}
\caption{Constraints on the $(m_{Z'},w_2)$ plane
 when
$m_{Q}=1700$ GeV,
$g_X = 2$, 
%$\lambda_\phi = 3$,
%$w_1 = 0$,
%$w_2 = 1.55$, 
$w_3 = 0.145$.
The color codes are the same as those in Figure \ref{mzpmq}.
}
\label{mzpw2}
\end{center}
\end{figure}
%%%%%%%%%%%%%%%%%%%%%%%%%%%%%%%%%%



In Figure \ref{mzpw2},
the constraints on the plane ($m_{Z'},w_2$) are shown in the case with
$m_{Q}=1700$ GeV,
$g_X = 2$, 
%$\lambda_\phi = 3$,
%$w_1 = 0$, and
%$w_2 = 1.55$, 
$w_3 = 0.145$.
The points in the region with small $w_2$ and $m_{Z'}$ correspond to too small mixing between the SM and the vectorlike quarks.
Therefore, they can not explain the CKM unitarity violation in the first row.
This results in the excluded region below the blue back-hatched area.
%
Although the current $b\rightarrow s\gamma$ constraint (\ref{bsg_now}) is not severe for this benchmark, the Belle II experiment is expected to make a significant contribution in excluding a large portion of the parameter space.
As we can see in this figure, the dark green region corresponding to Eq. (\ref{bsg_belle2}) becomes much smaller than the red hatched region allowed by the quark mass constraint.
%
The presently allowed regions on this plane are set by the constraints
(\ref{BRK})-(\ref{RKs}) on the $b\rightarrow s \ell^+ \ell^-$ transitions (the yellow strip) and the constraint (\ref{ckm}) from the CKM unitarity violation (the blue back-hatched region).
%
With the given choice of other parameters as above, the Yukawa coupling $w_2$ is restricted to stay within a narrow range [1.47, 1.74],
while the allowed range for $m_{Z'}$ is [265, 1560] GeV.
%
After the Belle II experiment finishes, this allowed region will be reduced by a factor of more than one half.




%%%%%%%%%%%%%%%%%%%%%%%%%%%%%%%%%%
\begin{figure}[h!]
\begin{center}
\includegraphics[scale=0.65]{MzpW3Positive.pdf}
\caption{Constraints on the $(m_{Z'},w_3)$ plane
 when
$m_{Q}=1700$ GeV,
$g_X = 2$, 
%$\lambda_\phi = 3$,
%$w_1 = 0$,
$w_2 = 1.55$. 
%$w_3 = 0.145$.
The color codes are the same as those in Figure \ref{mzpmq}.
}
\label{mzpw3}
\end{center}
\end{figure}
%%%%%%%%%%%%%%%%%%%%%%%%%%%%%%%%%%

In Figure \ref{mzpw3}, the considered constraints are depicted on the plane ($m_{Z'}, w_3$).
The free inputs are chosen to be
$m_{Q}=1700$ GeV,
$g_X = 2$, and 
%$\lambda_\phi = 3$,
%$w_1 = 0$, and
$w_2 = 1.55$.
For this benchmark, the CKM unitarity violation constraint (the blue back-hatched region) requires the $Z'$-boson mass to be within the range
[275, 1530] GeV.
The constraints on the semileptonic decays $b\rightarrow s \ell^+ \ell^-$ (the yellow strip) restrict the values of $w_3$ to be about 0.15 
in this case.
The expected result at the Belle II experiment will provide us with a stringent constraint on the $b\rightarrow s \gamma$ decay (the dark green region).
According to that the allowed range of $m_{Z'}$ will be reduced to 
[275, 950] GeV for this benchmark point.









%%%%%%%%%%%%%%%%%%%%%%%%%%%%%%
\section{Conclusion}
%%%%%%%%%%%%%%%%%%%%%%%%%%%%%%




In the considered model with a new sector consisting of vectorlike fermions and two scalars charged under an extra $U(1)_X$ symmetry, the exotic Yukawa interactions between this sector and the SM fermions are essential to address various experimental anomalies such as the muon $g-2$, the lepton flavor universal violation in the rare decays of $B$ mesons.
%
Within this context, we have analytically calculated 
the new physics contributions to the Wilson coefficient $C_7$ in the effective Hamiltonian.
Based on that, the dependence of the $b\rightarrow s \gamma$ decay rate on the free parameters is obtained.
%
We have shown that this model is possible to explain the measured CKM unitarity violation and the current data relevant to the $b \rightarrow s \gamma$ transition at the same time, while predicting other flavor observables relating to the $b \rightarrow s \ell^+ \ell^-$ processes and the SM quark masses compatible with their updated measurements.
%
By investigating the space of input parameters, the allowed region satisfying various constraints has been identified.
%
Together with the combined constraint on the semileptonic decays of $B$ mesons, the constraint on the violation of the CKM unitarity plays an important role in pinpointing the viable parameter space.
%
%
The impact of the expected outcome at the Belle II experiment on the $b\rightarrow s \gamma$ decay has been analyzed in detail.
The result showed that this foreseen constraint will be able to exclude a significant portion of the currently allowed parameter regions in the near future.















%\end{fmffile}


%%%%%%%%%%%%%%%%%%%%%%%%%%%




%%%%%%%%%%%%%%%%%%%%%%%%%%%%%%
\section*{Acknowledgment}
%%%%%%%%%%%%%%%%%%%%%%%%%%%%%%

Sang Quang Dinh was funded by Vingroup JSC and supported by the Master, PhD Scholarship Programme of Vingroup Innovation Foundation (VINIF), Institute of Big Data, code VINIF.2021.TS.037.



%%%%%%%%%%%%%%%%%%%%%%%%%%%%%%
%\section*{Appendix}
%%%%%%%%%%%%%%%%%%%%%%%%%%%%%%





%%%%%%%%%%%%%%%%%%%%%%%%%%%%%%%%%
\begin{thebibliography}{99}
%%%%%%%%%%%%%%%%%%%%%%%%%%%%%%%%%

%\cite{Workman:2022ynf}
\bibitem{Workman:2022ynf}
R.~L.~Workman [Particle Data Group],
%``Review of Particle Physics,''
PTEP \textbf{2022}, 083C01 (2022).
%doi:10.1093/ptep/ptac097
%120 citations counted in INSPIRE as of 08 Oct 2022



%\cite{Belfatto:2019swo}
\bibitem{Belfatto:2019swo}
B.~Belfatto, R.~Beradze and Z.~Berezhiani,
%``The CKM unitarity problem: A trace of new physics at the TeV scale?,''
Eur. Phys. J. C \textbf{80}, no.2, 149 (2020)
%doi:10.1140/epjc/s10052-020-7691-6
[arXiv:1906.02714 [hep-ph]];
%89 citations counted in INSPIRE as of 08 Oct 2022
%\cite{Coutinho:2019aiy}
%\bibitem{Coutinho:2019aiy}
A.~M.~Coutinho, A.~Crivellin and C.~A.~Manzari,
%``Global Fit to Modified Neutrino Couplings and the Cabibbo-Angle Anomaly,''
Phys. Rev. Lett. \textbf{125}, no.7, 071802 (2020)
%doi:10.1103/PhysRevLett.125.071802
[arXiv:1912.08823 [hep-ph]];
%87 citations counted in INSPIRE as of 29 Mar 2023
%\cite{Crivellin:2020lzu}
%\bibitem{Crivellin:2020lzu}
A.~Crivellin and M.~Hoferichter,
%``\ensuremath{\beta} Decays as Sensitive Probes of Lepton Flavor Universality,''
Phys. Rev. Lett. \textbf{125}, no.11, 111801 (2020)
%doi:10.1103/PhysRevLett.125.111801
[arXiv:2002.07184 [hep-ph]];
%74 citations counted in INSPIRE as of 29 Mar 2023
%\cite{Kirk:2020wdk}
%\bibitem{Kirk:2020wdk}
M.~Kirk,
%``Cabibbo anomaly versus electroweak precision tests: An exploration of extensions of the Standard Model,''
Phys. Rev. D \textbf{103}, no.3, 035004 (2021)
%doi:10.1103/PhysRevD.103.035004
[arXiv:2008.03261 [hep-ph]];
%46 citations counted in INSPIRE as of 29 Mar 2023
%\cite{Crivellin:2021rbf}
%\bibitem{Crivellin:2021rbf}
A.~Crivellin, C.~A.~Manzari and M.~Montull,
%``Correlating nonresonant di-electron searches at the LHC to the Cabibbo-angle anomaly and lepton flavor universality violation,''
Phys. Rev. D \textbf{104}, no.11, 115016 (2021)
%doi:10.1103/PhysRevD.104.115016
[arXiv:2103.12003 [hep-ph]];
%25 citations counted in INSPIRE as of 29 Mar 2023
%\cite{Crivellin:2022rhw}
%\bibitem{Crivellin:2022rhw}
A.~Crivellin, M.~Kirk, T.~Kitahara and F.~Mescia,
%``Global Fit of Modified Quark Couplings to EW Gauge Bosons and Vector-Like Quarks in Light of the Cabibbo Angle Anomaly,''
[arXiv:2212.06862 [hep-ph]].
%3 citations counted in INSPIRE as of 29 Mar 2023



%\cite{Arnan:2019uhr}
\bibitem{Arnan:2019uhr}
P.~Arnan, A.~Crivellin, M.~Fedele and F.~Mescia,
%``Generic Loop Effects of New Scalars and Fermions in $b\to s\ell^+\ell^-$, $(g-2)_\mu$ and a Vector-like $4^{\rm th}$ Generation,''
JHEP \textbf{06}, 118 (2019)
%doi:10.1007/JHEP06(2019)118
[arXiv:1904.05890 [hep-ph]];
%79 citations counted in INSPIRE as of 29 Mar 2023
%\cite{Crivellin:2019mvj}
%\bibitem{Crivellin:2019mvj}
A.~Crivellin and M.~Hoferichter,
%``Combined Explanations of $(g-2)_\mu$, $(g-2)_e$ and Implications for a Large Muon EDM,''
PoS \textbf{ALPS2019}, 009 (2020)
%doi:10.22323/1.360.0009
[arXiv:1905.03789 [hep-ph]];
%21 citations counted in INSPIRE as of 29 Mar 2023
%\cite{Crivellin:2021rbq}
%\bibitem{Crivellin:2021rbq}
A.~Crivellin and M.~Hoferichter,
%``Consequences of chirally enhanced explanations of (g \ensuremath{-} 2)$_{μ}$ for h \textrightarrow{} \ensuremath{\mu}\ensuremath{\mu} and Z \textrightarrow{} \ensuremath{\mu}\ensuremath{\mu},''
JHEP \textbf{07}, 135 (2021)
[erratum: JHEP \textbf{10}, 030 (2022)]
%doi:10.1007/JHEP07(2021)135
[arXiv:2104.03202 [hep-ph]].
%72 citations counted in INSPIRE as of 29 Mar 2023




%\cite{Crivellin:2015mga}
\bibitem{Crivellin:2015mga}
A.~Crivellin, G.~D'Ambrosio and J.~Heeck,
%``Explaining $h\to\mu^\pm\tau^\mp$, $B\to K^* \mu^+\mu^-$ and $B\to K \mu^+\mu^-/B\to K e^+e^-$ in a two-Higgs-doublet model with gauged $L_\mu-L_\tau$,''
Phys. Rev. Lett. \textbf{114}, 151801 (2015)
%doi:10.1103/PhysRevLett.114.151801
[arXiv:1501.00993 [hep-ph]];
%432 citations counted in INSPIRE as of 29 Mar 2023
%\cite{Crivellin:2015lwa}
%\bibitem{Crivellin:2015lwa}
A.~Crivellin, G.~D'Ambrosio and J.~Heeck,
%``Addressing the LHC flavor anomalies with horizontal gauge symmetries,''
Phys. Rev. D \textbf{91}, no.7, 075006 (2015)
%doi:10.1103/PhysRevD.91.075006
[arXiv:1503.03477 [hep-ph]];
%330 citations counted in INSPIRE as of 29 Mar 2023
%\cite{Crivellin:2020oup}
%\bibitem{Crivellin:2020oup}
A.~Crivellin, C.~A.~Manzari, M.~Alguero and J.~Matias,
%``Combined Explanation of the Z\textrightarrow{}bb\textasciimacron{} Forward-Backward Asymmetry, the Cabibbo Angle Anomaly, and \ensuremath{\tau}\textrightarrow{}\ensuremath{\mu}\ensuremath{\nu}\ensuremath{\nu} and b\textrightarrow{}s\ensuremath{\ell}+\ensuremath{\ell}- Data,''
Phys. Rev. Lett. \textbf{127}, no.1, 011801 (2021)
%doi:10.1103/PhysRevLett.127.011801
[arXiv:2010.14504 [hep-ph]].
%49 citations counted in INSPIRE as of 29 Mar 2023






%\cite{Buras:1993xp}
\bibitem{Buras:1993xp}
A.~J.~Buras, M.~Misiak, M.~Munz and S.~Pokorski,
%``Theoretical uncertainties and phenomenological aspects of B ---\ensuremath{>} X(s) gamma decay,''
Nucl. Phys. B \textbf{424}, 374-398 (1994)
%doi:10.1016/0550-3213(94)90299-2
[arXiv:hep-ph/9311345 [hep-ph]].
%503 citations counted in INSPIRE as of 30 Sep 2022

%\cite{Buchalla:1995vs}
\bibitem{Buchalla:1995vs}
G.~Buchalla, A.~J.~Buras and M.~E.~Lautenbacher,
%``Weak decays beyond leading logarithms,''
Rev. Mod. Phys. \textbf{68}, 1125-1144 (1996)
%doi:10.1103/RevModPhys.68.1125
[arXiv:hep-ph/9512380 [hep-ph]].
%2794 citations counted in INSPIRE as of 27 Sep 2022



%\cite{Chetyrkin:1996vx}
\bibitem{Chetyrkin:1996vx}
K.~G.~Chetyrkin, M.~Misiak and M.~Munz,
%``Weak radiative B meson decay beyond leading logarithms,''
Phys. Lett. B \textbf{400}, 206-219 (1997)
[erratum: Phys. Lett. B \textbf{425}, 414 (1998)]
%doi:10.1016/S0370-2693(97)00324-9
[arXiv:hep-ph/9612313 [hep-ph]].
%742 citations counted in INSPIRE as of 30 Sep 2022

%\cite{Misiak:2006zs}
\bibitem{Misiak:2006zs}
M.~Misiak, H.~M.~Asatrian, K.~Bieri, M.~Czakon, A.~Czarnecki, T.~Ewerth, A.~Ferroglia, P.~Gambino, M.~Gorbahn and C.~Greub, \textit{et al.}
%``Estimate of $\mathcal{B} (\bar B \to X_s \gamma)$ at $O(\alpha_s^2)$,''
Phys. Rev. Lett. \textbf{98}, 022002 (2007)
%doi:10.1103/PhysRevLett.98.022002
[arXiv:hep-ph/0609232 [hep-ph]].
%921 citations counted in INSPIRE as of 30 Sep 2022

%\cite{Misiak:2015xwa}
\bibitem{Misiak:2015xwa}
M.~Misiak, H.~M.~Asatrian, R.~Boughezal, M.~Czakon, T.~Ewerth, A.~Ferroglia, P.~Fiedler, P.~Gambino, C.~Greub and U.~Haisch, \textit{et al.}
%``Updated NNLO QCD predictions for the weak radiative B-meson decays,''
Phys. Rev. Lett. \textbf{114}, no.22, 221801 (2015)
%doi:10.1103/PhysRevLett.114.221801
[arXiv:1503.01789 [hep-ph]].
%382 citations counted in INSPIRE as of 29 Sep 2022

%\cite{Czakon:2015exa}
\bibitem{Czakon:2015exa}
M.~Czakon, P.~Fiedler, T.~Huber, M.~Misiak, T.~Schutzmeier and M.~Steinhauser,
%``The $(Q_{7}, Q_{1,2})$ contribution to $ \overline{B}\to {X}_s\gamma $ at $ \mathcal{O}\left({\alpha}_{\mathrm{s}}^2\right) $,''
JHEP \textbf{04}, 168 (2015)
%doi:10.1007/JHEP04(2015)168
[arXiv:1503.01791 [hep-ph]].
%103 citations counted in INSPIRE as of 29 Sep 2022

%\cite{Misiak:2020vlo}
\bibitem{Misiak:2020vlo}
M.~Misiak, A.~Rehman and M.~Steinhauser,
%``Towards $ \overline{B}\to {X}_s\gamma $ at the NNLO in QCD without interpolation in m$_{c}$,''
JHEP \textbf{06}, 175 (2020)
%doi:10.1007/JHEP06(2020)175
[arXiv:2002.01548 [hep-ph]].
%62 citations counted in INSPIRE as of 29 Sep 2022


%\cite{CLEO:2001gsa}
\bibitem{CLEO:2001gsa}
S.~Chen \textit{et al.} [CLEO],
%``Branching fraction and photon energy spectrum for $b \to s \gamma$,''
Phys. Rev. Lett. \textbf{87}, 251807 (2001)
%doi:10.1103/PhysRevLett.87.251807
[arXiv:hep-ex/0108032 [hep-ex]].
%727 citations counted in INSPIRE as of 14 Sep 2022




%\cite{BaBar:2007yhb}
\bibitem{BaBar:2007yhb}
B.~Aubert \textit{et al.} [BaBar],
%``Measurement of the $B \to X_s \gamma$ branching fraction and photon energy spectrum using the recoil method,''
Phys. Rev. D \textbf{77}, 051103 (2008)
%doi:10.1103/PhysRevD.77.051103
[arXiv:0711.4889 [hep-ex]].
%155 citations counted in INSPIRE as of 14 Sep 2022

%\cite{BaBar:2012eja}
\bibitem{BaBar:2012eja}
J.~P.~Lees \textit{et al.} [BaBar],
%``Exclusive Measurements of $b \to s\gamma$ Transition Rate and Photon Energy Spectrum,''
Phys. Rev. D \textbf{86}, 052012 (2012)
%doi:10.1103/PhysRevD.86.052012
[arXiv:1207.2520 [hep-ex]].
%120 citations counted in INSPIRE as of 14 Sep 2022

%\cite{BaBar:2012fqh}
\bibitem{BaBar:2012fqh}
J.~P.~Lees \textit{et al.} [BaBar],
%``Precision Measurement of the $B \to X_s \gamma$ Photon Energy Spectrum, Branching Fraction, and Direct CP Asymmetry $A_{CP}(B \to X_{s+d}\gamma)$,''
Phys. Rev. Lett. \textbf{109}, 191801 (2012)
%doi:10.1103/PhysRevLett.109.191801
[arXiv:1207.2690 [hep-ex]].
%112 citations counted in INSPIRE as of 14 Sep 2022

%\cite{Belle:2009nth}
\bibitem{Belle:2009nth}
A.~Limosani \textit{et al.} [Belle],
%``Measurement of Inclusive Radiative B-meson Decays with a Photon Energy Threshold of 1.7-GeV,''
Phys. Rev. Lett. \textbf{103}, 241801 (2009)
%doi:10.1103/PhysRevLett.103.241801
[arXiv:0907.1384 [hep-ex]].
%159 citations counted in INSPIRE as of 14 Sep 2022

%\cite{Belle:2014nmp}
\bibitem{Belle:2014nmp}
T.~Saito \textit{et al.} [Belle],
%``Measurement of the $\bar{B} \rightarrow X_s \gamma$ Branching Fraction with a Sum of Exclusive Decays,''
Phys. Rev. D \textbf{91}, no.5, 052004 (2015)
%doi:10.1103/PhysRevD.91.052004
[arXiv:1411.7198 [hep-ex]].
%82 citations counted in INSPIRE as of 14 Sep 2022



%\cite{HFLAV:2022pwe}
\bibitem{HFLAV:2022pwe}
Y.~Amhis \textit{et al.} [HFLAV],
%``Averages of $b$-hadron, $c$-hadron, and $\tau$-lepton properties as of 2021,''
[arXiv:2206.07501 [hep-ex]].
%27 citations counted in INSPIRE as of 14 Sep 2022



%\cite{Belle-II:2022cgf}
\bibitem{Belle-II:2022cgf}
L.~Aggarwal \textit{et al.} [Belle-II],
%``Snowmass White Paper: Belle II physics reach and plans for the next decade and beyond,''
[arXiv:2207.06307 [hep-ex]].
%21 citations counted in INSPIRE as of 20 Jan 2023


%\cite{DiCanto:2022icc}
\bibitem{DiCanto:2022icc}
A.~Di Canto and S.~Meinel,
%``Weak Decays of $b$ and $c$ Quarks,''
[arXiv:2208.05403 [hep-ex]].
%3 citations counted in INSPIRE as of 20 Jan 2023




%\cite{Chang:2000gz}
\bibitem{Chang:2000gz}
C.~H.~V.~Chang, D.~Chang and W.~Y.~Keung,
%``Vector quark model and B ---\ensuremath{>} X-s gamma decay,''
Phys. Rev. D \textbf{61} (2000), 053007.
%doi:10.1103/PhysRevD.61.053007
%25 citations counted in INSPIRE as of 07 Oct 2022

%\cite{Akeroyd:2001gf}
\bibitem{Akeroyd:2001gf}
A.~G.~Akeroyd and S.~Recksiegel,
%``Direct CP asymmetry of B ---\ensuremath{>} X(d,s) gamma in a model with vector quarks,''
Phys. Lett. B \textbf{525}, 81-88 (2002)
%doi:10.1016/S0370-2693(01)01419-8
[arXiv:hep-ph/0109091 [hep-ph]].
%14 citations counted in INSPIRE as of 07 Oct 2022


%\cite{Idarraga:2005ia}
\bibitem{Idarraga:2005ia}
J.~P.~Idarraga, R.~Martinez, J.~A.~Rodriguez and N.~Poveda,
%``$B \to$ X($s \gamma^{)}$ and $B^{+} \to \ell^{+} \nu$ in the 2 HDM type III,''
[arXiv:hep-ph/0509072 [hep-ph]].
%7 citations counted in INSPIRE as of 07 Oct 2022

%\cite{Lunghi:2007ak}
\bibitem{Lunghi:2007ak}
E.~Lunghi and A.~Soni,
%``Footprints of the Beyond in flavor physics: Possible role of the Top Two Higgs Doublet Model,''
JHEP \textbf{09}, 053 (2007)
%doi:10.1088/1126-6708/2007/09/053
[arXiv:0707.0212 [hep-ph]].
%63 citations counted in INSPIRE as of 08 Oct 2022

%\cite{Branco:2011iw}
\bibitem{Branco:2011iw}
G.~C.~Branco, P.~M.~Ferreira, L.~Lavoura, M.~N.~Rebelo, M.~Sher and J.~P.~Silva,
%``Theory and phenomenology of two-Higgs-doublet models,''
Phys. Rept. \textbf{516}, 1-102 (2012)
doi:10.1016/j.physrep.2012.02.002
[arXiv:1106.0034 [hep-ph]].
%2417 citations counted in INSPIRE as of 07 Oct 2022

%\cite{Hermann:2012fc}
\bibitem{Hermann:2012fc}
T.~Hermann, M.~Misiak and M.~Steinhauser,
%``$\bar{B}\to X_s \gamma$ in the Two Higgs Doublet Model up to Next-to-Next-to-Leading Order in QCD,''
JHEP \textbf{11}, 036 (2012)
%doi:10.1007/JHEP11(2012)036
[arXiv:1208.2788 [hep-ph]].
%205 citations counted in INSPIRE as of 08 Oct 2022

%\cite{Jung:2012vu}
\bibitem{Jung:2012vu}
M.~Jung, X.~Q.~Li and A.~Pich,
%``Exclusive radiative B-meson decays within the aligned two-Higgs-doublet model,''
JHEP \textbf{10}, 063 (2012)
%doi:10.1007/JHEP10(2012)063
[arXiv:1208.1251 [hep-ph]].
%56 citations counted in INSPIRE as of 08 Oct 2022

%\cite{Crivellin:2013wna}
\bibitem{Crivellin:2013wna}
A.~Crivellin, A.~Kokulu and C.~Greub,
%``Flavor-phenomenology of two-Higgs-doublet models with generic Yukawa structure,''
Phys. Rev. D \textbf{87}, no.9, 094031 (2013)
%doi:10.1103/PhysRevD.87.094031
[arXiv:1303.5877 [hep-ph]].
%282 citations counted in INSPIRE as of 08 Oct 2022

%\cite{Das:2015kea}
\bibitem{Das:2015kea}
S.~P.~Das, J.~Hern\'andez-S\'anchez, S.~Moretti, A.~Rosado and R.~Xoxocotzi,
%``Flavor violating signatures of lighter and heavier Higgs bosons within the Two Higgs Doublet Model Type-III at the LHeC,''
Phys. Rev. D \textbf{94}, no.5, 055003 (2016)
%doi:10.1103/PhysRevD.94.055003
[arXiv:1503.01464 [hep-ph]].
%29 citations counted in INSPIRE as of 08 Oct 2022

%\cite{Misiak:2017bgg}
\bibitem{Misiak:2017bgg}
M.~Misiak and M.~Steinhauser,
%``Weak radiative decays of the B meson and bounds on $M_{H^\pm }$ in the Two-Higgs-Doublet Model,''
Eur. Phys. J. C \textbf{77}, no.3, 201 (2017)
%doi:10.1140/epjc/s10052-017-4776-y
[arXiv:1702.04571 [hep-ph]].
%247 citations counted in INSPIRE as of 08 Oct 2022

%\cite{Haller:2018nnx}
\bibitem{Haller:2018nnx}
J.~Haller, A.~Hoecker, R.~Kogler, K.~M\"onig, T.~Peiffer and J.~Stelzer,
%``Update of the global electroweak fit and constraints on two-Higgs-doublet models,''
Eur. Phys. J. C \textbf{78}, no.8, 675 (2018)
%doi:10.1140/epjc/s10052-018-6131-3
[arXiv:1803.01853 [hep-ph]].
%310 citations counted in INSPIRE as of 08 Oct 2022

%\cite{Arco:2020ucn}
\bibitem{Arco:2020ucn}
F.~Arco, S.~Heinemeyer and M.~J.~Herrero,
%``Exploring sizable triple Higgs couplings in the 2HDM,''
Eur. Phys. J. C \textbf{80}, no.9, 884 (2020)
%doi:10.1140/epjc/s10052-020-8406-8
[arXiv:2005.10576 [hep-ph]].
%22 citations counted in INSPIRE as of 08 Oct 2022

%\cite{Atkinson:2021eox}
\bibitem{Atkinson:2021eox}
O.~Atkinson, M.~Black, A.~Lenz, A.~Rusov and J.~Wynne,
%``Cornering the Two Higgs Doublet Model Type II,''
JHEP \textbf{04}, 172 (2022)
%doi:10.1007/JHEP04(2022)172
[arXiv:2107.05650 [hep-ph]].
%22 citations counted in INSPIRE as of 08 Oct 2022

%\cite{Arco:2022xum}
\bibitem{Arco:2022xum}
F.~Arco, S.~Heinemeyer and M.~J.~Herrero,
%``Triple Higgs couplings in the 2HDM: the complete picture,''
Eur. Phys. J. C \textbf{82}, no.6, 536 (2022)
%doi:10.1140/epjc/s10052-022-10485-9
[arXiv:2203.12684 [hep-ph]].
%3 citations counted in INSPIRE as of 08 Oct 2022

%\cite{Enomoto:2022rrl}
\bibitem{Enomoto:2022rrl}
K.~Enomoto, S.~Kanemura and Y.~Mura,
%``New benchmark scenarios of electroweak baryogenesis in aligned two Higgs double models,''
JHEP \textbf{09}, 121 (2022)
%doi:10.1007/JHEP09(2022)121
[arXiv:2207.00060 [hep-ph]].
%0 citations counted in INSPIRE as of 08 Oct 2022





%\cite{Akeroyd:2020nfj}
\bibitem{Akeroyd:2020nfj}
A.~G.~Akeroyd, S.~Moretti, T.~Shindou and M.~Song,
%``CP asymmetries of ${\overline B}\to X_s/X_d\gamma$ in models with three Higgs doublets,''
Phys. Rev. D \textbf{103}, no.1, 015035 (2021)
%doi:10.1103/PhysRevD.103.015035
[arXiv:2009.05779 [hep-ph]].
%10 citations counted in INSPIRE as of 08 Oct 2022


%\cite{Bertolini:1990if}
\bibitem{Bertolini:1990if}
S.~Bertolini, F.~Borzumati, A.~Masiero and G.~Ridolfi,
%``Effects of supergravity induced electroweak breaking on rare $B$ decays and mixings,''
Nucl. Phys. B \textbf{353}, 591-649 (1991).
%doi:10.1016/0550-3213(91)90320-W
%892 citations counted in INSPIRE as of 07 Oct 2022

%\cite{Barbieri:1993av}
\bibitem{Barbieri:1993av}
R.~Barbieri and G.~F.~Giudice,
%``b ---\ensuremath{>} s gamma decay and supersymmetry,''
Phys. Lett. B \textbf{309}, 86-90 (1993)
%doi:10.1016/0370-2693(93)91508-K
[arXiv:hep-ph/9303270 [hep-ph]].
%385 citations counted in INSPIRE as of 07 Oct 2022

%\cite{Borzumati:1994te}
\bibitem{Borzumati:1994te}
F.~Borzumati, M.~Olechowski and S.~Pokorski,
%``Constraints on the minimal SUSY SO(10) model from cosmology and the b ---\ensuremath{>} s gamma decay,''
Phys. Lett. B \textbf{349}, 311-318 (1995)
%doi:10.1016/0370-2693(95)00270-U
[arXiv:hep-ph/9412379 [hep-ph]].
%72 citations counted in INSPIRE as of 07 Oct 2022

%\cite{Degrassi:2000qf}
\bibitem{Degrassi:2000qf}
G.~Degrassi, P.~Gambino and G.~F.~Giudice,
%``B ---\ensuremath{>} X(s gamma) in supersymmetry: Large contributions beyond the leading order,''
JHEP \textbf{12}, 009 (2000)
%doi:10.1088/1126-6708/2000/12/009
[arXiv:hep-ph/0009337 [hep-ph]].
%438 citations counted in INSPIRE as of 07 Oct 2022

%\cite{Carena:2000uj}
\bibitem{Carena:2000uj}
M.~Carena, D.~Garcia, U.~Nierste and C.~E.~M.~Wagner,
%``$b \to s \gamma$ and supersymmetry with large $\tan\beta$,''
Phys. Lett. B \textbf{499}, 141-146 (2001)
%doi:10.1016/S0370-2693(01)00009-0
[arXiv:hep-ph/0010003 [hep-ph]].
%407 citations counted in INSPIRE as of 07 Oct 2022

%\cite{Demir:2001yz}
\bibitem{Demir:2001yz}
D.~A.~Demir and K.~A.~Olive,
%``B ---\ensuremath{>} X(s) gamma in supersymmetry with explicit CP violation,''
Phys. Rev. D \textbf{65} (2002), 034007
%doi:10.1103/PhysRevD.65.034007
[arXiv:hep-ph/0107329 [hep-ph]].
%99 citations counted in INSPIRE as of 07 Oct 2022

%\cite{Baek:2002wm}
\bibitem{Baek:2002wm}
S.~Baek, P.~Ko and W.~Y.~Song,
%``SUSY breaking mediation mechanisms and (g-2) ($\mu$), $B \to X_{s} \gamma$, $B \to X_{s} \ell^{+} \ell^{-}$ and $B_s \to \mu^{+} \mu^{-}$,''
JHEP \textbf{03}, 054 (2003)
%doi:10.1088/1126-6708/2003/03/054
[arXiv:hep-ph/0208112 [hep-ph]].
%64 citations counted in INSPIRE as of 07 Oct 2022

%\cite{Hurth:2003vb}
\bibitem{Hurth:2003vb}
T.~Hurth,
%``Present status of inclusive rare B decays,''
Rev. Mod. Phys. \textbf{75}, 1159-1199 (2003)
%doi:10.1103/RevModPhys.75.1159
[arXiv:hep-ph/0212304 [hep-ph]].
%239 citations counted in INSPIRE as of 07 Oct 2022

%\cite{Ellis:2006ix}
\bibitem{Ellis:2006ix}
J.~R.~Ellis, S.~Heinemeyer, K.~A.~Olive and G.~Weiglein,
%``Phenomenological indications of the scale of supersymmetry,''
JHEP \textbf{05}, 005 (2006)
%doi:10.1088/1126-6708/2006/05/005
[arXiv:hep-ph/0602220 [hep-ph]], and Refs. there in.
%73 citations counted in INSPIRE as of 07 Oct 2022

%\cite{Gomez:2006uv}
\bibitem{Gomez:2006uv}
M.~E.~Gomez, T.~Ibrahim, P.~Nath and S.~Skadhauge,
%``An Improved analysis of b --\ensuremath{>} s gamma in supersymmetry,''
Phys. Rev. D \textbf{74}, 015015 (2006)
%doi:10.1103/PhysRevD.74.015015
[arXiv:hep-ph/0601163 [hep-ph]].
%75 citations counted in INSPIRE as of 07 Oct 2022

%\cite{Ellis:2007fu}
\bibitem{Ellis:2007fu}
J.~R.~Ellis, S.~Heinemeyer, K.~A.~Olive, A.~M.~Weber and G.~Weiglein,
%``The Supersymmetric Parameter Space in Light of $B^-$ physics Observables and Electroweak Precision Data,''
JHEP \textbf{08}, 083 (2007)
%doi:10.1088/1126-6708/2007/08/083
[arXiv:0706.0652 [hep-ph]].
%155 citations counted in INSPIRE as of 08 Oct 2022

%\cite{Heinemeyer:2008fb}
\bibitem{Heinemeyer:2008fb}
S.~Heinemeyer, X.~Miao, S.~Su and G.~Weiglein,
%``$B^-$ Physics Observables and Electroweak Precision Data in the CMSSM, mGMSB and mAMSB,''
JHEP \textbf{08}, 087 (2008)
%doi:10.1088/1126-6708/2008/08/087
[arXiv:0805.2359 [hep-ph]].
%52 citations counted in INSPIRE as of 08 Oct 2022

%\cite{Olive:2008vv}
\bibitem{Olive:2008vv}
K.~A.~Olive and L.~Velasco-Sevilla,
%``Constraints on Supersymmetric Flavour Models from b ---\ensuremath{>} s gamma,''
JHEP \textbf{05}, 052 (2008)
%doi:10.1088/1126-6708/2008/05/052
[arXiv:0801.0428 [hep-ph]].
%19 citations counted in INSPIRE as of 08 Oct 2022

%\cite{Okada:2010xe}
\bibitem{Okada:2010xe}
N.~Okada and H.~M.~Tran,
%``Discrimination of Supersymmetric Grand Unified Models in Gaugino Mediation,''
Phys. Rev. D \textbf{83}, 053001 (2011)
%doi:10.1103/PhysRevD.83.053001
[arXiv:1011.1668 [hep-ph]].
%6 citations counted in INSPIRE as of 07 Oct 2022

%\cite{Zhang:2014nya}
\bibitem{Zhang:2014nya}
H.~B.~Zhang, G.~H.~Luo, T.~F.~Feng, S.~M.~Zhao, T.~J.~Gao and K.~S.~Sun,
%``$\bar{B}\rightarrow X_s\gamma$ in the $\mu\nu$SSM,''
Mod. Phys. Lett. A \textbf{29}, no.38, 1450196 (2014)
%doi:10.1142/S021773231450196X
[arXiv:1409.6837 [hep-ph]].
%5 citations counted in INSPIRE as of 08 Oct 2022

%\cite{GAMBIT:2017snp}
\bibitem{GAMBIT:2017snp}
P.~Athron \textit{et al.} [GAMBIT],
%``Global fits of GUT-scale SUSY models with GAMBIT,''
Eur. Phys. J. C \textbf{77}, no.12, 824 (2017)
%doi:10.1140/epjc/s10052-017-5167-0
[arXiv:1705.07935 [hep-ph]].
%125 citations counted in INSPIRE as of 08 Oct 2022

%\cite{Yang:2018fvw}
\bibitem{Yang:2018fvw}
J.~L.~Yang, T.~F.~Feng, S.~M.~Zhao, R.~F.~Zhu, X.~Y.~Yang and H.~B.~Zhang,
%``Two loop electroweak corrections to $\bar B\rightarrow X_s\gamma$ and $B_s^0\rightarrow \mu^+\mu^-$ in the B-LSSM,''
Eur. Phys. J. C \textbf{78}, no.9, 714 (2018)
%doi:10.1140/epjc/s10052-018-6174-5
[arXiv:1803.09904 [hep-ph]].
%19 citations counted in INSPIRE as of 08 Oct 2022









%\cite{Haisch:2007vb}
\bibitem{Haisch:2007vb}
U.~Haisch and A.~Weiler,
%``Bound on minimal universal extra dimensions from anti-B ---\ensuremath{>} X(s)gamma,''
Phys. Rev. D \textbf{76}, 034014 (2007)
%doi:10.1103/PhysRevD.76.034014
[arXiv:hep-ph/0703064 [hep-ph]].
%116 citations counted in INSPIRE as of 08 Oct 2022

%\cite{Freitas:2008vh}
\bibitem{Freitas:2008vh}
A.~Freitas and U.~Haisch,
%``Anti-B ---\ensuremath{>} X(s) gamma in two universal extra dimensions,''
Phys. Rev. D \textbf{77}, 093008 (2008)
%doi:10.1103/PhysRevD.77.093008
[arXiv:0801.4346 [hep-ph]].
%39 citations counted in INSPIRE as of 08 Oct 2022

%\cite{Moch:2015oka}
\bibitem{Moch:2015oka}
P.~Moch and J.~Rohrwild,
%``$\bar B\to X_s \gamma$ with a warped bulk Higgs,''
Nucl. Phys. B \textbf{902}, 142-161 (2016)
%doi:10.1016/j.nuclphysb.2015.11.014
[arXiv:1509.04643 [hep-ph]].
%8 citations counted in INSPIRE as of 08 Oct 2022


%\cite{Blanke:2012tv}
\bibitem{Blanke:2012tv}
M.~Blanke, B.~Shakya, P.~Tanedo and Y.~Tsai,
%``The Birds and the Bs in RS: The $b to s \gamma$ penguin in a warped extra dimension,''
JHEP \textbf{08}, 038 (2012)
%doi:10.1007/JHEP08(2012)038
[arXiv:1203.6650 [hep-ph]].
%46 citations counted in INSPIRE as of 01 Oct 2022

%\cite{Datta:2016flx}
\bibitem{Datta:2016flx}
A.~Datta \textit{et al.} [Indian Association for the Cultivation of Science],
%``Effects of non-minimal Universal Extra Dimension on $B\rightarrow X_s\gamma$,''
Phys. Rev. D \textbf{95}, no.1, 015033 (2017)
%doi:10.1103/PhysRevD.95.015033
[arXiv:1610.09924 [hep-ph]].
%12 citations counted in INSPIRE as of 08 Oct 2022

%\cite{Cheung:2017efc}
\bibitem{Cheung:2017efc}
K.~Cheung, T.~Nomura and H.~Okada,
%``A Three-loop Neutrino Model with Leptoquark Triplet Scalars,''
Phys. Lett. B \textbf{768}, 359-364 (2017)
%doi:10.1016/j.physletb.2017.03.021
[arXiv:1701.01080 [hep-ph]].
%14 citations counted in INSPIRE as of 08 Oct 2022

%\cite{Aliev:1996cyj}
\bibitem{Aliev:1996cyj}
T.~M.~Aliev, D.~A.~Demir and N.~K.~Pak,
%``Constraining four generation SM with $b \to s \gamma$ and $b \to s$ g decays,''
Phys. Lett. B \textbf{389}, 83-88 (1996)
%doi:10.1016/S0370-2693(96)01244-0
[arXiv:hep-ph/9809354 [hep-ph]].
%14 citations counted in INSPIRE as of 01 Oct 2022


%\cite{NguyenTuan:2020xls}
\bibitem{NguyenTuan:2020xls}
D.~Nguyen Tuan, T.~Inami and H.~Do Thi,
%``Physical constraints derived from FCNC in the 3-3-1-1 model,''
Eur. Phys. J. C \textbf{81}, no.9, 813 (2021)
%doi:10.1140/epjc/s10052-021-09583-x
[arXiv:2009.09698 [hep-ph]].
%10 citations counted in INSPIRE as of 01 Oct 2022

%\cite{Gabrielli:2016cut}
\bibitem{Gabrielli:2016cut}
E.~Gabrielli, B.~Mele, M.~Raidal and E.~Venturini,
%``FCNC decays of standard model fermions into a dark photon,''
Phys. Rev. D \textbf{94}, no.11, 115013 (2016)
%doi:10.1103/PhysRevD.94.115013
[arXiv:1607.05928 [hep-ph]].
%30 citations counted in INSPIRE as of 08 Oct 2022




%\cite{Aoki:2000ze}
\bibitem{Aoki:2000ze}
M.~Aoki, E.~Asakawa, M.~Nagashima, N.~Oshimo and A.~Sugamoto,
%``Contributions of vector like quarks to radiative B meson decay,''
Phys. Lett. B \textbf{487}, 321-326 (2000)
%doi:10.1016/S0370-2693(00)00818-2
[arXiv:hep-ph/0005133 [hep-ph]].
%14 citations counted in INSPIRE as of 08 Oct 2022

%\cite{Aoki:2001xr}
\bibitem{Aoki:2001xr}
M.~Aoki, G.~C.~Cho, M.~Nagashima and N.~Oshimo,
%``Large effects on B(s) anti-B(s) mixing by vectorlike quarks,''
Phys. Rev. D \textbf{64}, 117305 (2001)
%doi:10.1103/PhysRevD.64.117305
[arXiv:hep-ph/0102165 [hep-ph]].
%9 citations counted in INSPIRE as of 08 Oct 2022

%\cite{Morozumi:2018cnc}
\bibitem{Morozumi:2018cnc}
T.~Morozumi, Y.~Shimizu, S.~Takahashi and H.~Umeeda,
%``Effective theory analysis for vector-like quark model,''
PTEP \textbf{2018}, no.4, 043B10 (2018)
%doi:10.1093/ptep/pty042
[arXiv:1801.05268 [hep-ph]].
%2 citations counted in INSPIRE as of 08 Oct 2022

%\cite{Vatsyayan:2020jan}
\bibitem{Vatsyayan:2020jan}
D.~Vatsyayan and A.~Kundu,
%``Constraints on the quark mixing matrix with vector-like quarks,''
Nucl. Phys. B \textbf{960}, 115208 (2020)
%doi:10.1016/j.nuclphysb.2020.115208
[arXiv:2007.02327 [hep-ph]].
%5 citations counted in INSPIRE as of 08 Oct 2022



%\cite{Kawamura:2019rth}
\bibitem{Kawamura:2019rth}
J.~Kawamura, S.~Raby and A.~Trautner,
%``Complete vectorlike fourth family and new U(1)' for muon anomalies,''
Phys. Rev. D \textbf{100}, no.5, 055030 (2019)
%doi:10.1103/PhysRevD.100.055030
[arXiv:1906.11297 [hep-ph]].
%53 citations counted in INSPIRE as of 12 Oct 2022

%\cite{Cheung:2020vqm}
\bibitem{Cheung:2020vqm}
K.~Cheung, W.~Y.~Keung, C.~T.~Lu and P.~Y.~Tseng,
%``Vector-like Quark Interpretation for the CKM Unitarity Violation, Excess in Higgs Signal Strength, and Bottom Quark Forward-Backward Asymmetry,''
JHEP \textbf{05}, 117 (2020)
%doi:10.1007/JHEP05(2020)117
[arXiv:2001.02853 [hep-ph]].
%28 citations counted in INSPIRE as of 12 Oct 2022

%\cite{Crivellin:2020ebi}
\bibitem{Crivellin:2020ebi}
A.~Crivellin, F.~Kirk, C.~A.~Manzari and M.~Montull,
%``Global Electroweak Fit and Vector-Like Leptons in Light of the Cabibbo Angle Anomaly,''
JHEP \textbf{12}, 166 (2020)
%doi:10.1007/JHEP12(2020)166
[arXiv:2008.01113 [hep-ph]].
%69 citations counted in INSPIRE as of 12 Oct 2022

%\cite{Cherchiglia:2021vhe}
\bibitem{Cherchiglia:2021vhe}
A.~L.~Cherchiglia, G.~De Conto and C.~C.~Nishi,
%``Flavor constraints for a vector-like quark of Nelson-Barr type,''
JHEP \textbf{11}, 093 (2021)
%doi:10.1007/JHEP11(2021)093
[arXiv:2103.04798 [hep-ph]].
%4 citations counted in INSPIRE as of 12 Oct 2022

%\cite{Belfatto:2021jhf}
\bibitem{Belfatto:2021jhf}
B.~Belfatto and Z.~Berezhiani,
%``Are the CKM anomalies induced by vector-like quarks? Limits from flavor changing and Standard Model precision tests,''
JHEP \textbf{10}, 079 (2021)
%doi:10.1007/JHEP10(2021)079
[arXiv:2103.05549 [hep-ph]].
%27 citations counted in INSPIRE as of 12 Oct 2022

%\cite{Branco:2021vhs}
\bibitem{Branco:2021vhs}
G.~C.~Branco, J.~T.~Penedo, P.~M.~F.~Pereira, M.~N.~Rebelo and J.~I.~Silva-Marcos,
%``Addressing the CKM unitarity problem with a vector-like up quark,''
JHEP \textbf{07}, 099 (2021)
%doi:10.1007/JHEP07(2021)099
[arXiv:2103.13409 [hep-ph]].
%22 citations counted in INSPIRE as of 12 Oct 2022

%\cite{Balaji:2021lpr}
\bibitem{Balaji:2021lpr}
S.~Balaji,
%``Asymmetry in flavour changing electromagnetic transitions of vector-like quarks,''
JHEP \textbf{05}, 015 (2022)
%doi:10.1007/JHEP05(2022)015
[arXiv:2110.05473 [hep-ph]].
%4 citations counted in INSPIRE as of 12 Oct 2022

%\cite{CarcamoHernandez:2021yev}
\bibitem{CarcamoHernandez:2021yev}
A.~E.~C\'arcamo Hern\'andez, S.~F.~King and H.~Lee,
%``Z mediated flavor changing neutral currents with a fourth vectorlike family,''
Phys. Rev. D \textbf{105}, no.1, 015021 (2022)
%doi:10.1103/PhysRevD.105.015021
[arXiv:2110.07630 [hep-ph]].
%3 citations counted in INSPIRE as of 12 Oct 2022

%\cite{Accomando:2022ouo}
\bibitem{Accomando:2022ouo}
E.~Accomando, J.~Brannigan, J.~Gunn, Y.~Huyan and S.~Mulligan,
%``Constrained Vector-like quark model with perturbative unitarity,''
[arXiv:2202.05936 [hep-ph]].
%0 citations counted in INSPIRE as of 12 Oct 2022

%\cite{Guedes:2022cfy}
\bibitem{Guedes:2022cfy}
G.~Guedes and P.~Olgoso,
%``A bridge to new physics: proposing new \textemdash{} and reviving old \textemdash{} explanations of a$_{μ}$,''
JHEP \textbf{09}, 181 (2022)
%doi:10.1007/JHEP09(2022)181
[arXiv:2205.04480 [hep-ph]].
%4 citations counted in INSPIRE as of 12 Oct 2022

%\cite{Branco:2022fmj}
\bibitem{Branco:2022fmj}
G.~C.~Branco, J.~F.~Bastos and J.~I.~Silva-Marcos,
%``Do the Small Numbers in the Quark Mixing arise from New Physics?,''
[arXiv:2207.14235 [hep-ph]].
%0 citations counted in INSPIRE as of 12 Oct 2022






%\cite{Belanger:2015nma}
\bibitem{Belanger:2015nma}
G.~B\'elanger, C.~Delaunay and S.~Westhoff,
%``A Dark Matter Relic From Muon Anomalies,''
Phys. Rev. D \textbf{92}, 055021 (2015)
%doi:10.1103/PhysRevD.92.055021
[arXiv:1507.06660 [hep-ph]].
%134 citations counted in INSPIRE as of 27 Sep 2022

%\cite{Belanger:2016ywb}
\bibitem{Belanger:2016ywb}
G.~B\'elanger and C.~Delaunay,
%``A Dark Sector for $g_\mu-2$, $R_K$ and a Diphoton Resonance,''
Phys. Rev. D \textbf{94}, no.7, 075019 (2016)
%doi:10.1103/PhysRevD.94.075019
[arXiv:1603.03333 [hep-ph]].
%20 citations counted in INSPIRE as of 27 Sep 2022

%\cite{Dinh:2020inx}
\bibitem{Dinh:2020inx}
S.~Q.~Dinh and H.~M.~Tran,
%``Muon g-2 and semileptonic B decays in the B\'elanger-Delaunay-Westhoff model with gauge kinetic mixing,''
Phys. Rev. D \textbf{104}, no.11, 115009 (2021)
%doi:10.1103/PhysRevD.104.115009
[arXiv:2011.07182 [hep-ph]].
%3 citations counted in INSPIRE as of 27 Sep 2022


%\cite{Greub:1996tg}
\bibitem{Greub:1996tg}
C.~Greub, T.~Hurth and D.~Wyler,
%``Virtual O (alpha-s) corrections to the inclusive decay b ---\ensuremath{>} s gamma,''
Phys. Rev. D \textbf{54}, 3350-3364 (1996)
%doi:10.1103/PhysRevD.54.3350
[arXiv:hep-ph/9603404 [hep-ph]].
%336 citations counted in INSPIRE as of 27 Mar 2023



%\cite{Hieu:2020hti}
\bibitem{Hieu:2020hti}
T.~M.~Hieu, Q.~S.~Sang and T.~Q.~Trang,
%``On a standard model extension with vector-like fermions and Abelian symmetry,''
Commun. in Phys. \textbf{30}, no.3, 231-244 (2020).
%doi:10.15625/0868-3166/30/3/15071
%1 citations counted in INSPIRE as of 08 Jan 2023




%\cite{Mertig:1990an}
\bibitem{Mertig:1990an}
R.~Mertig, M.~Bohm and A.~Denner,
%``FEYN CALC: Computer algebraic calculation of Feynman amplitudes,''
Comput. Phys. Commun. \textbf{64}, 345-359 (1991).
%doi:10.1016/0010-4655(91)90130-D
%1198 citations counted in INSPIRE as of 08 Jan 2023

%\cite{Shtabovenko:2016sxi}
\bibitem{Shtabovenko:2016sxi}
V.~Shtabovenko, R.~Mertig and F.~Orellana,
%``New Developments in FeynCalc 9.0,''
Comput. Phys. Commun. \textbf{207}, 432-444 (2016)
%doi:10.1016/j.cpc.2016.06.008
[arXiv:1601.01167 [hep-ph]].
%649 citations counted in INSPIRE as of 08 Jan 2023

%\cite{Shtabovenko:2020gxv}
\bibitem{Shtabovenko:2020gxv}
V.~Shtabovenko, R.~Mertig and F.~Orellana,
%``FeynCalc 9.3: New features and improvements,''
Comput. Phys. Commun. \textbf{256}, 107478 (2020)
%doi:10.1016/j.cpc.2020.107478
[arXiv:2001.04407 [hep-ph]].
%250 citations counted in INSPIRE as of 08 Jan 2023



%\cite{Patel:2015tea}
\bibitem{Patel:2015tea}
H.~H.~Patel,
%``Package-X: A Mathematica package for the analytic calculation of one-loop integrals,''
Comput. Phys. Commun. \textbf{197}, 276-290 (2015)
%doi:10.1016/j.cpc.2015.08.017
[arXiv:1503.01469 [hep-ph]].
%393 citations counted in INSPIRE as of 08 Jan 2023

%\cite{Patel:2016fam}
\bibitem{Patel:2016fam}
H.~H.~Patel,
%``Package-X 2.0: A Mathematica package for the analytic calculation of one-loop integrals,''
Comput. Phys. Commun. \textbf{218}, 66-70 (2017)
%doi:10.1016/j.cpc.2017.04.015
[arXiv:1612.00009 [hep-ph]].
%142 citations counted in INSPIRE as of 08 Jan 2023


\bibitem{BR}
Heavy Flavor Averaging Group,
\url{https://hflav-eos.web.cern.ch/hflav-eos/rare/April2019/RADLL/OUTPUT/HTML/radll_table1.html}

\bibitem{Aaij:2012vr}
R.~Aaij \textit{et al.} [LHCb],
%``Differential branching fraction and angular analysis of the $B^{+} \rightarrow K^{+}\mu^{+}\mu^{-}$ decay,''
JHEP \textbf{02}, 105 (2013)
%doi:10.1007/JHEP02(2013)105
[arXiv:1209.4284 [hep-ex]].
%112 citations counted in INSPIRE as of 07 Jul 2020

\bibitem{Aaij:2014pli}
R.~Aaij \textit{et al.} [LHCb],
%``Differential branching fractions and isospin asymmetries of $B \to K^{(*)} \mu^+ \mu^-$ decays,''
JHEP \textbf{06}, 133 (2014)
%doi:10.1007/JHEP06(2014)133
[arXiv:1403.8044 [hep-ex]].
%341 citations counted in INSPIRE as of 09 Jul 2020


\bibitem{Aaij:2016flj}
R.~Aaij \textit{et al.} [LHCb],
%``Measurements of the S-wave fraction in $B^{0}\rightarrow K^{+}\pi^{-}\mu^{+}\mu^{-}$ decays and the $B^{0}\rightarrow K^{\ast}(892)^{0}\mu^{+}\mu^{-}$ differential branching fraction,''
JHEP \textbf{11}, 047 (2016)
%doi:10.1007/JHEP11(2016)047
[arXiv:1606.04731 [hep-ex]].
%129 citations counted in INSPIRE as of 07 Jul 2020


%\bibitem{RK_RKs}
%%\cite{HFLAV:2022pwe}
%%\bibitem{HFLAV:2022pwe}
%Y.~S.~Amhis \textit{et al.} [HFLAV],
%%``Averages of $b$-hadron, $c$-hadron, and $\tau$-lepton properties as of 2021,''
%[arXiv:2206.07501 [hep-ex]].
%103 citations counted in INSPIRE as of 23 Mar 2023
%Heavy Flavor Averaging Group,
%\url{https://hflav-eos.web.cern.ch/hflav-eos/rare/April2019/RADLL/OUTPUT/HTML/radll_table5.html}

\bibitem{Aaij:2019wad}
R.~Aaij \textit{et al.} [LHCb],
%``Search for lepton-universality violation in $B^+\to K^+\ell^+\ell^-$ decays,''
Phys. Rev. Lett. \textbf{122}, no.19, 191801 (2019)
%doi:10.1103/PhysRevLett.122.191801
[arXiv:1903.09252 [hep-ex]].
%187 citations counted in INSPIRE as of 07 Jul 2020

\bibitem{Aaij:2014ora}
R.~Aaij \textit{et al.} [LHCb],
%``Test of lepton universality using $B^{+}\rightarrow K^{+}\ell^{+}\ell^{-}$ decays,''
Phys. Rev. Lett. \textbf{113}, 151601 (2014)
%doi:10.1103/PhysRevLett.113.151601
[arXiv:1406.6482 [hep-ex]].
%996 citations counted in INSPIRE as of 09 Jul 2020

%\cite{Aaij:2021vac}
\bibitem{Aaij:2021vac}
R.~Aaij \textit{et al.} [LHCb],
%``Test of lepton universality in beauty-quark decays,''
[arXiv:2103.11769 [hep-ex]].
%39 citations counted in INSPIRE as of 26 Apr 2021


\bibitem{Aaij:2017vbb}
R.~Aaij \textit{et al.} [LHCb],
%``Test of lepton universality with $B^{0} \rightarrow K^{*0}\ell^{+}\ell^{-}$ decays,''
JHEP \textbf{08}, 055 (2017)
%doi:10.1007/JHEP08(2017)055
[arXiv:1705.05802 [hep-ex]].
%639 citations counted in INSPIRE as of 09 Jul 2020
%

\bibitem{Abdesselam:2019wac}
A.~Abdesselam \textit{et al.} [Belle],
%``Test of lepton flavor universality in ${B\to K^\ast\ell^+\ell^-}$ decays at Belle,''
[arXiv:1904.02440 [hep-ex]].
%95 citations counted in INSPIRE as of 09 Jul 2020



%\cite{Aad:2014vma}
\bibitem{Aad:2014vma}
G.~Aad \textit{et al.} [ATLAS],
%``Search for direct production of charginos, neutralinos and sleptons in final states with two leptons and missing transverse momentum in $pp$ collisions at $\sqrt{s} =$ 8 TeV with the ATLAS detector,''
JHEP \textbf{05}, 071 (2014)
%doi:10.1007/JHEP05(2014)071
[arXiv:1403.5294 [hep-ex]];
%434 citations counted in INSPIRE as of 16 Jun 2021
%\cite{Aad:2019vnb}
%\bibitem{Aad:2019vnb}
G.~Aad \textit{et al.} [ATLAS],
%``Search for electroweak production of charginos and sleptons decaying into final states with two leptons and missing transverse momentum in $\sqrt{s}=13$ TeV $pp$ collisions using the ATLAS detector,''
Eur. Phys. J. C \textbf{80}, no.2, 123 (2020)
%doi:10.1140/epjc/s10052-019-7594-6
[arXiv:1908.08215 [hep-ex]];
%116 citations counted in INSPIRE as of 16 Jun 2021
%\cite{Khachatryan:2014qwa}
%\bibitem{Khachatryan:2014qwa}
V.~Khachatryan \textit{et al.} [CMS],
%``Searches for electroweak production of charginos, neutralinos, and sleptons decaying to leptons and W, Z, and Higgs bosons in pp collisions at 8 TeV,''
Eur. Phys. J. C \textbf{74}, no.9, 3036 (2014)
%doi:10.1140/epjc/s10052-014-3036-7
[arXiv:1405.7570 [hep-ex]];
%344 citations counted in INSPIRE as of 19 Jun 2021
%\cite{Sirunyan:2020eab}
%\bibitem{Sirunyan:2020eab}
A.~M.~Sirunyan \textit{et al.} [CMS],
%``Search for supersymmetry in final states with two oppositely charged same-flavor leptons and missing transverse momentum in proton-proton collisions at $\sqrt{s} =$ 13 TeV,''
JHEP \textbf{04}, 123 (2021)
%doi:10.1007/JHEP04(2021)123
[arXiv:2012.08600 [hep-ex]].
%17 citations counted in INSPIRE as of 19 Jun 2021


\bibitem{Tanabashi:2018oca}
M.~Tanabashi \textit{et al.} [Particle Data Group],
%``Review of Particle Physics,''
Phys. Rev. D \textbf{98}, no.3, 030001 (2018);
%doi:10.1103/PhysRevD.98.030001
%5077 citations counted in INSPIRE as of 13 Jul 2020
%\bibitem{Mohr:2012tt}
P.~J.~Mohr, B.~N.~Taylor and D.~B.~Newell,
%``CODATA Recommended Values of the Fundamental Physical Constants: 2010,''
Rev. Mod. Phys. \textbf{84}, 1527-1605 (2012)
%doi:10.1103/RevModPhys.84.1527
[arXiv:1203.5425 [physics.atom-ph]];
%568 citations counted in INSPIRE as of 13 Jul 
%\bibitem{Bennett:2002jb}
G.~W.~Bennett \textit{et al.} [Muon g-2],
%``Measurement of the positive muon anomalous magnetic moment to 0.7 ppm,''
Phys. Rev. Lett. \textbf{89}, 101804 (2002)
%doi:10.1103/PhysRevLett.89.101804
[arXiv:hep-ex/0208001 [hep-ex]];
%566 citations counted in INSPIRE as of 13 Jul 2020
%\bibitem{Bennett:2002jb}
G.~W.~Bennett \textit{et al.} [Muon g-2],
%``Measurement of the positive muon anomalous magnetic moment to 0.7 ppm,''
Phys. Rev. Lett. \textbf{89}, 101804 (2002)
%doi:10.1103/PhysRevLett.89.101804
[arXiv:hep-ex/0208001 [hep-ex]];
%566 citations counted in INSPIRE as of 13 Jul 2020
%\bibitem{Bennett:2004pv}
G.~W.~Bennett \textit{et al.} [Muon g-2],
%``Measurement of the negative muon anomalous magnetic moment to 0.7 ppm,''
Phys. Rev. Lett. \textbf{92}, 161802 (2004)
%doi:10.1103/PhysRevLett.92.161802
[arXiv:hep-ex/0401008 [hep-ex]];
%789 citations counted in INSPIRE as of 13 Jul 2020
%\bibitem{Bennett:2006fi}
G.~W.~Bennett \textit{et al.} [Muon g-2],
%``Final Report of the Muon E821 Anomalous Magnetic Moment Measurement at BNL,''
Phys. Rev. D \textbf{73}, 072003 (2006)
%doi:10.1103/PhysRevD.73.072003
[arXiv:hep-ex/0602035 [hep-ex]].
%2169 citations counted in INSPIRE as of 13 Jul 2020


%\cite{Abi:2021gix}
\bibitem{Abi:2021gix}
B.~Abi \textit{et al.} [Muon g-2],
%``Measurement of the Positive Muon Anomalous Magnetic Moment to 0.46 ppm,''
Phys. Rev. Lett. \textbf{126}, no.14, 141801 (2021)
%doi:10.1103/PhysRevLett.126.141801
[arXiv:2104.03281 [hep-ex]].
%51 citations counted in INSPIRE as of 14 Apr 2021



%\cite{Aoyama:2020ynm}
\bibitem{Aoyama:2020ynm}
T.~Aoyama \textit{et al.}
%``The anomalous magnetic moment of the muon in the Standard Model,''
Phys. Rept. \textbf{887}, 1-166 (2020)
%doi:10.1016/j.physrep.2020.07.006
[arXiv:2006.04822 [hep-ph]];
%944 citations counted in INSPIRE as of 29 Mar 2023
%\bibitem{Aoyama:2012wk}
T.~Aoyama, M.~Hayakawa, T.~Kinoshita and M.~Nio,
%``Complete Tenth-Order QED Contribution to the Muon g-2,''
Phys. Rev. Lett. \textbf{109}, 111808 (2012)
%doi:10.1103/PhysRevLett.109.111808
[arXiv:1205.5370 [hep-ph]];
%326 citations counted in INSPIRE as of 29 Jul 2020
%\bibitem{Aoyama:2017uqe}
T.~Aoyama, T.~Kinoshita and M.~Nio,
%``Revised and Improved Value of the QED Tenth-Order Electron Anomalous Magnetic Moment,''
Phys. Rev. D \textbf{97}, no.3, 036001 (2018)
%doi:10.1103/PhysRevD.97.036001
[arXiv:1712.06060 [hep-ph]];
%96 citations counted in INSPIRE as of 29 Jul 2020
%\cite{Aoyama:2019ryr}
%\bibitem{Aoyama:2019ryr}
T.~Aoyama, T.~Kinoshita and M.~Nio,
%``Theory of the Anomalous Magnetic Moment of the Electron,''
Atoms \textbf{7}, no.1, 28 (2019);
%doi:10.3390/atoms7010028
%287 citations counted in INSPIRE as of 29 Mar 2023
%
%\cite{Czarnecki:2002nt}
%\bibitem{Czarnecki:2002nt}
A.~Czarnecki, W.~J.~Marciano and A.~Vainshtein,
%``Refinements in electroweak contributions to the muon anomalous magnetic moment,''
Phys. Rev. D \textbf{67}, 073006 (2003)
[erratum: Phys. Rev. D \textbf{73}, 119901 (2006)]
%doi:10.1103/PhysRevD.67.073006
[arXiv:hep-ph/0212229 [hep-ph]];
%572 citations counted in INSPIRE as of 29 Mar 2023
%
%\bibitem{Gnendiger:2013pva}
C.~Gnendiger, D.~Stöckinger and H.~Stöckinger-Kim,
%``The electroweak contributions to $(g-2)_\mu$ after the Higgs boson mass measurement,''
Phys. Rev. D \textbf{88}, 053005 (2013)
%doi:10.1103/PhysRevD.88.053005
[arXiv:1306.5546 [hep-ph]];
%200 citations counted in INSPIRE as of 29 Jul 2020
%\bibitem{Blum:2013xva}
T.~Blum, A.~Denig, I.~Logashenko, E.~de Rafael, B.~L.~Roberts, T.~Teubner and G.~Venanzoni,
%``The Muon (g-2) Theory Value: Present and Future,''
[arXiv:1311.2198 [hep-ph]];
%243 citations counted in INSPIRE as of 29 Jul 2020
%\bibitem{Davier:2010nc}
M.~Davier, A.~Hoecker, B.~Malaescu and Z.~Zhang,
%``Reevaluation of the Hadronic Contributions to the Muon g-2 and to alpha(MZ),''
Eur. Phys. J. C \textbf{71}, 1515 (2011)
%doi:10.1140/epjc/s10052-012-1874-8
[arXiv:1010.4180 [hep-ph]];
%831 citations counted in INSPIRE as of 29 Jul 2020
%\bibitem{Davier:2017zfy}
M.~Davier, A.~Hoecker, B.~Malaescu and Z.~Zhang,
%``Reevaluation of the hadronic vacuum polarisation contributions to the Standard Model predictions of the muon $g-2$ and ${\alpha (m_Z^2)}$ using newest hadronic cross-section data,''
Eur. Phys. J. C \textbf{77}, no.12, 827 (2017)
%doi:10.1140/epjc/s10052-017-5161-6
[arXiv:1706.09436 [hep-ph]];
%192 citations counted in INSPIRE as of 29 Jul 2020
%\bibitem{Davier:2019can}
M.~Davier, A.~Hoecker, B.~Malaescu and Z.~Zhang,
%``A new evaluation of the hadronic vacuum polarisation contributions to the muon anomalous magnetic moment and to $\mathbf{\boldsymbol\alpha(m_Z^2)}$,''
Eur. Phys. J. C \textbf{80}, no.3, 241 (2020)
%doi:10.1140/epjc/s10052-020-7792-2
[arXiv:1908.00921 [hep-ph]];
%
%\cite{Keshavarzi:2018mgv}
%\bibitem{Keshavarzi:2018mgv}
A.~Keshavarzi, D.~Nomura and T.~Teubner,
%``Muon $g-2$ and $\alpha(M_Z^2)$: a new data-based analysis,''
Phys. Rev. D \textbf{97}, no.11, 114025 (2018)
%doi:10.1103/PhysRevD.97.114025
[arXiv:1802.02995 [hep-ph]];
%633 citations counted in INSPIRE as of 29 Mar 2023
%\cite{Keshavarzi:2019abf}
%\bibitem{Keshavarzi:2019abf}
A.~Keshavarzi, D.~Nomura and T.~Teubner,
%``$g-2$ of charged leptons, $\alpha (M^2_Z)$ , and the hyperfine splitting of muonium,''
Phys. Rev. D \textbf{101}, no.1, 014029 (2020)
%doi:10.1103/PhysRevD.101.014029
[arXiv:1911.00367 [hep-ph]];
%398 citations counted in INSPIRE as of 29 Mar 2023
%\cite{Colangelo:2018mtw}
%\bibitem{Colangelo:2018mtw}
G.~Colangelo, M.~Hoferichter and P.~Stoffer,
%``Two-pion contribution to hadronic vacuum polarization,''
JHEP \textbf{02}, 006 (2019)
%doi:10.1007/JHEP02(2019)006
[arXiv:1810.00007 [hep-ph]];
%353 citations counted in INSPIRE as of 29 Mar 2023
%\cite{Colangelo:2017fiz}
%\bibitem{Colangelo:2017fiz}
G.~Colangelo, M.~Hoferichter, M.~Procura and P.~Stoffer,
%``Dispersion relation for hadronic light-by-light scattering: two-pion contributions,''
JHEP \textbf{04}, 161 (2017)
%doi:10.1007/JHEP04(2017)161
[arXiv:1702.07347 [hep-ph]];
%341 citations counted in INSPIRE as of 29 Mar 2023
%\cite{Colangelo:2019uex}
%\bibitem{Colangelo:2019uex}
G.~Colangelo, F.~Hagelstein, M.~Hoferichter, L.~Laub and P.~Stoffer,
%``Longitudinal short-distance constraints for the hadronic light-by-light contribution to $(g-2)_\mu$ with large-$N_c$ Regge models,''
JHEP \textbf{03}, 101 (2020)
%doi:10.1007/JHEP03(2020)101
[arXiv:1910.13432 [hep-ph]];
%259 citations counted in INSPIRE as of 29 Mar 2023
%\cite{Colangelo:2014qya}
%\bibitem{Colangelo:2014qya}
G.~Colangelo, M.~Hoferichter, A.~Nyffeler, M.~Passera and P.~Stoffer,
%``Remarks on higher-order hadronic corrections to the muon g\ensuremath{-}2,''
Phys. Lett. B \textbf{735}, 90-91 (2014)
%doi:10.1016/j.physletb.2014.06.012
[arXiv:1403.7512 [hep-ph]];
%350 citations counted in INSPIRE as of 29 Mar 2023
%\cite{Hoferichter:2019mqg}
%\bibitem{Hoferichter:2019mqg}
M.~Hoferichter, B.~L.~Hoid and B.~Kubis,
%``Three-pion contribution to hadronic vacuum polarization,''
JHEP \textbf{08}, 137 (2019)
%doi:10.1007/JHEP08(2019)137
[arXiv:1907.01556 [hep-ph]];
%299 citations counted in INSPIRE as of 29 Mar 2023
%\cite{Hoferichter:2018kwz}
%\bibitem{Hoferichter:2018kwz}
M.~Hoferichter, B.~L.~Hoid, B.~Kubis, S.~Leupold and S.~P.~Schneider,
%``Dispersion relation for hadronic light-by-light scattering: pion pole,''
JHEP \textbf{10}, 141 (2018)
%doi:10.1007/JHEP10(2018)141
[arXiv:1808.04823 [hep-ph]];
%321 citations counted in INSPIRE as of 29 Mar 2023
%\cite{Kurz:2014wya}
%\bibitem{Kurz:2014wya}
A.~Kurz, T.~Liu, P.~Marquard and M.~Steinhauser,
%``Hadronic contribution to the muon anomalous magnetic moment to next-to-next-to-leading order,''
Phys. Lett. B \textbf{734}, 144-147 (2014)
%doi:10.1016/j.physletb.2014.05.043
[arXiv:1403.6400 [hep-ph]];
%407 citations counted in INSPIRE as of 29 Mar 2023
%\cite{Melnikov:2003xd}
%\bibitem{Melnikov:2003xd}
K.~Melnikov and A.~Vainshtein,
%``Hadronic light-by-light scattering contribution to the muon anomalous magnetic moment revisited,''
Phys. Rev. D \textbf{70}, 113006 (2004)
%doi:10.1103/PhysRevD.70.113006
[arXiv:hep-ph/0312226 [hep-ph]];
%585 citations counted in INSPIRE as of 29 Mar 2023
%\cite{Masjuan:2017tvw}
%\bibitem{Masjuan:2017tvw}
P.~Masjuan and P.~Sanchez-Puertas,
%``Pseudoscalar-pole contribution to the $(g_{\mu}-2)$: a rational approach,''
Phys. Rev. D \textbf{95}, no.5, 054026 (2017)
%doi:10.1103/PhysRevD.95.054026
[arXiv:1701.05829 [hep-ph]];
%284 citations counted in INSPIRE as of 29 Mar 2023
%\cite{Gerardin:2019vio}
%\bibitem{Gerardin:2019vio}
A.~G\'erardin, H.~B.~Meyer and A.~Nyffeler,
%``Lattice calculation of the pion transition form factor with $N_f=2+1$ Wilson quarks,''
Phys. Rev. D \textbf{100}, no.3, 034520 (2019)
%doi:10.1103/PhysRevD.100.034520
[arXiv:1903.09471 [hep-lat]];
%276 citations counted in INSPIRE as of 29 Mar 2023
%\cite{Bijnens:2019ghy}
%\bibitem{Bijnens:2019ghy}
J.~Bijnens, N.~Hermansson-Truedsson and A.~Rodr\'\i{}guez-S\'anchez,
%``Short-distance constraints for the HLbL contribution to the muon anomalous magnetic moment,''
Phys. Lett. B \textbf{798}, 134994 (2019)
%doi:10.1016/j.physletb.2019.134994
[arXiv:1908.03331 [hep-ph]];
%251 citations counted in INSPIRE as of 29 Mar 2023
%\cite{Blum:2019ugy}
%\bibitem{Blum:2019ugy}
T.~Blum, N.~Christ, M.~Hayakawa, T.~Izubuchi, L.~Jin, C.~Jung and C.~Lehner,
%``Hadronic Light-by-Light Scattering Contribution to the Muon Anomalous Magnetic Moment from Lattice QCD,''
Phys. Rev. Lett. \textbf{124}, no.13, 132002 (2020)
%doi:10.1103/PhysRevLett.124.132002
[arXiv:1911.08123 [hep-lat]];
%286 citations counted in INSPIRE as of 29 Mar 2023
%
%
%
%\cite{Borsanyi:2020mff}
%\bibitem{Borsanyi:2020mff}
S.~Borsanyi, Z.~Fodor, J.~N.~Guenther, C.~Hoelbling, S.~D.~Katz, L.~Lellouch, T.~Lippert, K.~Miura, L.~Parato and K.~K.~Szabo, \textit{et al.}
%``Leading hadronic contribution to the muon magnetic moment from lattice QCD,''
Nature \textbf{593}, no.7857, 51-55 (2021)
%doi:10.1038/s41586-021-03418-1
[arXiv:2002.12347 [hep-lat]];
%571 citations counted in INSPIRE as of 29 Mar 2023
%
%
For early analyses of the muon $g-2$, see for example,
%\bibitem{Terazawa:1968jh}
H.~Terazawa,
%``All the hadronic contributions to the anomalous magnetic moment of the muon and the lamb shift in the hydrogen atom,''
Prog. Theor. Phys. \textbf{39}, 1326-1332 (1968);
%doi:10.1143/PTP.39.1326
%12 citations counted in INSPIRE as of 24 Nov 2020
%\bibitem{Terazawa:1969ih}
H.~Terazawa,
%``Spectral function of the photon propagator-mass spectrum and timelike form-factors of particles,''
Phys. Rev. \textbf{177}, 2159-2166 (1969);
%doi:10.1103/PhysRev.177.2159
%6 citations counted in INSPIRE as of 24 Nov 2020
%\bibitem{Terazawa:1968mx}
H.~Terazawa,
%``Note on another expansion parameter in quantum electrodynamics,''
Prog. Theor. Phys. \textbf{40}, 830-833 (1968).
%doi:10.1143/PTP.40.830
%3 citations counted in INSPIRE as of 24 Nov 2020



%\cite{Misiak:2006ab}
\bibitem{Misiak:2006ab}
M.~Misiak and M.~Steinhauser,
%``NNLO QCD corrections to the anti-B ---\ensuremath{>} X(s) gamma matrix elements using interpolation in m(c),''
Nucl. Phys. B \textbf{764}, 62-82 (2007)
%doi:10.1016/j.nuclphysb.2006.11.027
[arXiv:hep-ph/0609241 [hep-ph]].
%339 citations counted in INSPIRE as of 11 Jan 2023



\end{thebibliography}


\end{document}
