\section{Conclusion}

In this paper, we identified program rewriting as a lightweight technique for
optimizing ad-hoc, single-machine EDA workloads. Performing rewrites is valid
only under conditions, which need to be checked at runtime, a setting that
imposes strict latency boundaries. We implemented \system{}, a system which
rewrites \code{pandas} code automatically and transparently, while
simultaneously addressing the requirements and constraints of
condition-checking. \system{} applies rewrite rules automatically, and it
verifies whether applying a rule is correct by either injecting checks in the
code or by slicing the execution and performing checks in between.

We experimentally showed that \system{} was able to achieve significant speedups
(up to 57$\times$ for individual cells and 3.5$\times$ for whole notebooks),
both compared to \code{pandas} and \code{modin}, in \emph{real-world}, randomly
sampled notebooks. At the same time, \system{} incurs minimal runtime and memory
overheads, while making no use of the disk, whether \system{} succeeds to
optimize code or not. Last but not least, this paper showed a new direction for
optimization, that of crossing library boundaries, the key aspect of which is to
employ techniques that understand both the library and the client code.