\section{Details on the Experimental Setup}

\paragraph{\textbf{Machine}}

\begin{itemize}
    \item \textbf{CPU}: AMD Ryzen 9 5900X 12-Core, clocked at 3.7 GHz
    \item \textbf{RAM}: 32 GB DRAM
    \item \textbf{SSD}: Samsung 980 PRO
    \item \textbf{OS}: Ubuntu 22.04.1 LTS
    \item \textbf{Quiescing Settings}:
        \begin{itemize}
            \item Boot in terminal mode
            \item Disable turbo boost
            \item Disable frequency scaling
        \end{itemize}
\end{itemize}

We picked this as a conventional consumer machine, except for the SSD which is unconventionally fast.

\paragraph{\textbf{Python/Pandas Setup}}
\begin{itemize}
    \item Python: 3.10.6
    \item IPython: 8.5.0
    \item Numpy: 1.23.4
    \item Pandas: 1.5.1
\end{itemize}

\paragraph{\textbf{Benchmark}}

We disabled some code in these notebooks that we consider out of scope. In particular:
\begin{itemize}
    \item Networking or Shell code, i.e., cells like \code{pip install} or like \code{nltk.download()}.
        \begin{itemize}
            \item This means that to reproduce our experiments, you may need to run the original notebook once e.g., to download the required \code{nltk} dataset.
        \end{itemize}
    \item Plotting and machine-learning code
    \item IPython magic functions because our tool can only handle valid Python code. For these notebooks, we only had to remove \code{\%matplotlib inline}.
\end{itemize}
