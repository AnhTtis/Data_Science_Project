% % Center column names
% \newcolumntype{P}[1]{>{\centering\arraybackslash}p{#1}}

% {\footnotesize
% \begin{table*}
% \begin{tabular}{|P{0.33\textwidth}|P{0.33\textwidth}|P{0.33\textwidth}|}
% {\LARGE LHS} &  {\LARGE RHS} & {\LARGE Preconditions} \\
% \hline \\
% \begin{lstlisting}[language=Python,basicstyle=\ttfamily, breaklines=true]
% @{Name: a}, @{Name: b} =
%   @{expr: ser}.str.split(
%     @{Constant(str): sep}, 
%       expand=@{Constant(bool): expand})
% \end{lstlisting}
% &
% \vspace{-15pt}
% \begin{lstlisting}[language=Python,basicstyle=\ttfamily, breaklines=true]
% a, b = [], []
% for it in @{ser}.tolist():
%     spl = it.split(@{sep})
%     a.append(spl[0])
%     y = spl[1] if len(spl) > 1 \
%           else None
%     b.append(y)
% @{a} = pandas.Series(a, @{ser}.index)
% @{b} = pandas.Series(b, @{ser}.index)
% \end{lstlisting}
% &
% \begin{lstlisting}[language=Python,basicstyle=\ttfamily, breaklines=true, escapeinside=||]
% |$\mathfrak S$:| @{expand} == True
% |$\mathfrak R$:| type(@{ser}) == pandas.Series
% \end{lstlisting}

% \\
% \hline
% \\

% \vspace{-12pt}
% \begin{lstlisting}[language=Python,basicstyle=\ttfamily, breaklines=true, escapeinside=||]
% @{expr: ser}.apply(
%   lambda @{Name: par1}:
%     @{Constant(str): needle} 
%       in @{Name: par2})
% \end{lstlisting}
% &
% \vspace{-5pt}
% \begin{lstlisting}[language=Python,basicstyle=\ttfamily, breaklines=true, escapeinside=||]
% res = @{ser}.tolist()
% res = [(@{needle} in s) for s in res]
% pandas.Series(res, @{ser}.index)
% \end{lstlisting}
% &
% \begin{lstlisting}[language=Python,basicstyle=\ttfamily, breaklines=true, escapeinside=||]
% |$\mathfrak R$:| type(@{ser}) == pandas.Series
% \end{lstlisting}

% \\
% \hline
% \\

% \vspace{-10pt}
% \begin{lstlisting}[language=Python,basicstyle=\ttfamily, breaklines=true, escapeinside=||]
% @{Name: df1}[@{Constant(str): c1}] = 
%   @{Name: df2}[@{Constant(str): c2}]
%     .fillna(@{expr: arg})
% \end{lstlisting}
% &
% \vspace{-5pt}
% \begin{lstlisting}[language=Python,basicstyle=\ttfamily, breaklines=true, escapeinside=||]
% @{df1}[@{c1}].fillna(@{arg}, inplace=True)
% \end{lstlisting}
% &
% \vspace{-10pt}
% \begin{lstlisting}[language=Python,basicstyle=\ttfamily, breaklines=true, escapeinside=||]
% |$\mathfrak S$:| @{df1} == @{df2}
% |$\mathfrak S$:| @{c1} == @{c2}
% |$\mathfrak R$:| type(@{df1}) == pandas.DataFrame
% \end{lstlisting}

% \\
% \hline
% \\

% \vspace{-15pt}
% \begin{lstlisting}[language=Python,basicstyle=\ttfamily, breaklines=true, escapeinside=||]
% pd.Series(
%   @{expr: e1}.tolist() + 
%   @{expr: e2}.tolist())
% \end{lstlisting}
% &
% \vspace{-10pt}
% \begin{lstlisting}[language=Python,basicstyle=\ttfamily, breaklines=true, escapeinside=||]
% pd.concat([@{e1}, @{e2}], ignore_index=True)
% \end{lstlisting}
% &
% \vspace{-10pt}
% \begin{lstlisting}[language=Python,basicstyle=\ttfamily, breaklines=true, escapeinside=||]
% |$\mathfrak R$:| pd == pandas
% |$\mathfrak R$:| type(@{e1}) == pandas.Series
% |$\mathfrak R$:| type(@{e2}) == pandas.Series 
% \end{lstlisting}

% \end{tabular}
% \vspace{10pt}
% \caption{Rewrite Rule Examples \stef{There are line cuts in the 3rd column}}
% \label{tbl:rewr_rules}
% \end{table*}
% }

% % MINTED DIDN'T WORK.
% %
% % \begin{figure*}
% % \begin{tabular}{ |p{0.3\textwidth} | p{0.3\textwidth} | p{0.3\textwidth} | }
% %   \begin{minted}[escapeinside=||]{python}
% %     @{expr: called_on}
% %     .sort_values()
% %     .head(n=@{Constant(int): first_n})
% %   \end{minted}
% %   &
% %   \begin{minted}[escapeinside=||]{python}
% %     @{expr: called_on}
% %     .sort_values()
% %     .head(n=@{Constant(int): first_n})
% %   \end{minted}
% %   &
% %   \begin{minted}[escapeinside=||]{python}
% %     @{expr: called_on}
% %     .sort_values()
% %     .head(n=@{Constant(int): first_n})
% %   \end{minted}
% % \end{tabular}
% % \end{figure*}

\begin{table*}[t]
  \centering
  \includegraphics[height=0.85\textheight]{figures/sources/tbl_rewr_rules.pdf}
  \caption{Examples of Rewrite Rules. If any of the LHS's is matched, it can be
  replaced with the corresponding RHS, provided that the preconditions hold. The
  symbol $\mathfrak S$ denotes syntactic preconditions while $\mathfrak R$
  denotes runtime ones. The name of the rule appears as a comment in the LHS
  column.}
  \label{tbl:rewr_rules}
\end{table*}