\begin{figure}
  \begin{subfigure}{\columnwidth}
\begin{minted}[bgcolor=light-gray]{python}
def foo(x):
  return True if x['Fare'] > 10 else False
df.apply(foo, axis=1)
\end{minted}
    \caption{A function applied to the whole dataframe.}
    \label{fig:remove-axis-1-orig}
  \end{subfigure}
  \hfill
  \begin{subfigure}{\columnwidth}
\begin{minted}[bgcolor=light-gray]{python}
def foo(x):
  return True if x > 10 else False
df['Fare'].apply(foo)
\end{minted}
    \caption{The function touches only one column so we can increase locality by iterating only that column.}
    \label{fig:remove-axis-1-rewr}
  \end{subfigure}
  \caption{Apply a function to a single column instead of a whole dataframe and increase locality}
  \label{fig:remove-axis-1-ex}
\end{figure}