\begin{abstract}

In recent years, dataframe libraries, such as \texttt{pandas} have exploded in
popularity. Due to their flexibility, they are increasingly used in
\emph{ad-hoc} exploratory data analysis (EDA) workloads. These workloads are
diverse, including custom functions which can span libraries or be written in
pure Python. The majority of systems available to accelerate EDA workloads focus
on bulk-parallel workloads, which contain vastly different computational
patterns, typically within a single library. As a result, they can introduce
excessive overheads for ad-hoc EDA workloads due to their expensive optimization
techniques. Instead, we identify source-to-source, external program rewriting as
a lightweight technique which can optimize across representations, and offer
substantial speedups while also avoiding slowdowns. We implemented \system{},
which rewrites notebook cells to be more efficient for ad-hoc EDA workloads. We
develop techniques for efficient rewrites in \system{}, including checking the
preconditions under which rewrites are correct, dynamically, at fine-grained
program points. We show that \system{} can rewrite individual cells to be
57$\times$ faster compared to \code{pandas} and 1909$\times$ faster compared to
optimized systems such as \code{modin}. Furthermore, \system{} can accelerate
whole notebooks by up to 3.6$\times$ compared to \code{pandas} and 27.1$\times$
compared to \code{modin}.



%
%In recent years, dataframe libraries, such as \texttt{pandas} have exploded in
%popularity. Due to their flexibility, they are increasingly used in
%\emph{ad-hoc} exploratory data science (EDA) workloads. These ad-hoc EDA are
%diverse, possibly including custom processing functions which can span libraries
%or be fully contained within pure Python. Previous systems fail to cope with
%this diversity graciously, and can introduce excessive overheads. In this paper,
%we identify program rewriting as a technique which can offer substantial
%speedups while also naturally avoiding slowdowns. We implemented our techniques
%in \system{}, which \emph{rewrites} notebook cells to be more efficient.
%\system{} targets workloads which are developed in interactive notebook
%environments, and thus it must operate at runtime, restricting it to a tight
%latency budget. Furthermore, \system{} must maintain code semantics, which is
%difficult in a language such as Python, with a liberal, dynamic type system,
%loose scoping rules and imprecise semantics. In order to achieve both goals, we
%introduce a lightweight \code{pandas} code rewrite system that dynamically
%checks preconditions under which rewrites are correct and just-in-time rewrites
%original pandas code with faster variants. We show that \system{} can rewrite
%individual cells to be 57$\times$ faster compared to \code{pandas} and
%1909$\times$ faster compared to optimized systems such as \code{modin}.
%Furthermore, \system{} can accelerate whole notebooks by up to 3.6$\times$
%compared to \code{pandas} and \emph{27.1$\times$} compared to \code{modin}.

% --- Previous ---
% In recent years, dataframe libraries, such as \texttt{pandas} have exploded in
% popularity. Due to their flexibility, they are increasingly used in
% \emph{ad-hoc} exploratory data science (EDA) workloads. In these ad-hoc EDA
% workloads, data scientists and social scientists write diverse code, including
% custom processing functions which can span libraries or be fully contained
% within pure Python. To accelerate these workloads, we propose \system{}, which
% \emph{rewrites} notebook cells to be more efficient. \system{} targets workloads
% which are developed in interactive notebook environments, and thus it must
% operate at runtime, restricting it to a tight latency budget. Furthermore,
% \system{} must maintain code semantics, which is difficult in a language such as
% Python, with a liberal, dynamic type system, loose scoping rules and imprecise
% semantics. At the same time, \system{} should handle the diversity of these
% workloads graciously and not introduce excessive overheads.

% In order to achieve these goals, we introduce a lightweight \code{pandas} code
% rewrite system that dynamically checks preconditions under which rewrites are
% correct and just-in-time rewrites original pandas code with faster variants,
% while also avoiding to rewrite cells that cause slowdowns. We show that
% \system{} can rewrite individual cells to be 57$\times$ faster compared
% to \code{pandas} and 1909$\times$ faster compared to optimized systems
% such as \code{modin}. Furthermore, \system{} can accelerate whole notebooks by
% up to 3.6$\times$ compared to \code{pandas} and \emph{27.1$\times$} compared to
% \code{modin}.

\end{abstract}
