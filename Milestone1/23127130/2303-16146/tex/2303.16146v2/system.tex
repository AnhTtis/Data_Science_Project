\section{\system{} Overview}
\label{sec:System}

\begin{figure}
\begin{tikzpicture}[
  node distance=2cm,
  font=\small,
  CodeBlock/.style = {
  align=center,
  rectangle,
  draw=border-blue,
  thick,
  },
  ComponentBlock/.style={
  rectangle,
  draw=blue,
  thick,
  fill=blue!20,
  text width=5em,
  align=center,
  rounded corners,
  minimum height=2em
  }
]



\node (codeSource) [CodeBlock] {%
\centering

\begin{minipage}{7.2cm}
{\fontfamily{qbk}\selectfont
Jupyter Cell Source
}
\centering
\begin{minted}[bgcolor=light-gray]{python}
print(...)
foo().sort_values().head(n=5)
\end{minted}
\end{minipage}%
};




\draw node[CodeBlock, below=0.7cm of codeSource] (PattMatcher) {%
\centering
\begin{minipage}{7.2cm}
\centering
{\fontfamily{qbk}\selectfont
Pattern Matcher
}

\begin{minted}[bgcolor=light-gray, escapeinside=||]{python}
print(...)
|\colorbox{highlight-green}{foo().sort\_values().head(n=5)}|
\end{minted}
\end{minipage}%
};



\draw node[CodeBlock, below=0.5cm of PattMatcher] (OnlRewr) {%
\centering
\begin{minipage}{7.2cm}
\centering
{\fontfamily{qbk}\selectfont
Rewrite the Code
}

\begin{minted}[bgcolor=light-gray, escapeinside=||]{python}
print(...)

def sort_head(tmp):
  if type(tmp) == pd.Series:
    return tmp.nsmallest()
  else:
    tmp.sort_values().head(n=5)

sort_head(foo())

\end{minted}
\end{minipage}%
};


\draw node[CodeBlock, below=0.7cm of OnlRewr] (RunCell) {%
\centering
\begin{minipage}{7.2cm}
\centering
{\fontfamily{qbk}\selectfont
Execute the Rewritten Code
}

\begin{minted}[bgcolor=light-gray, escapeinside=||]{python}
ipython.run_cell(new_source)
\end{minted}
\end{minipage}%
};


\draw [-stealth] (codeSource) -- (PattMatcher);
\draw [-stealth] (PattMatcher) -- (OnlRewr);
\draw [-stealth] (OnlRewr) -- (RunCell);


\end{tikzpicture}
\caption{\system{} overview. \system{} identifies patterns in the source code,
which it rewrites using its rewriter. The optimized version is used only if
certain dynamic checks pass, to ensure correctness.}
\label{fig:system_overview}
\end{figure}

%We now present \system{}, a system that accelerates interactive data science workloads by transparently rewriting Python that interfaces with Pandas code and which addresses all the previously mentioned challenges. In order to do so, \system{} has two high-level components. First, \system{}' \textit{syntactic pattern-matcher} identifies matches the input code against the LHS parts of the rewrite rules. The second component is a \emph{rewriter}, which validates the preconditions of the rewrite rules and on passing them, rewrites the code to the RHS version and executes it. We show a high-level overview in Figure \ref{fig:system_overview}.

We now present the high-level architecture of \system{}, a rewrite engine that
automatically applies rewrite rules to improve the performance of ad-hoc EDA
workloads.

We designed \system{} with two high-level components. First, \system{}'
\textit{syntactic pattern matcher} matches the input code against the syntactic
patterns the rewrite rules. The second component is a \emph{rewriter}, which
checks the runtime preconditions of the rewrite rules and on success, rewrites
the code to an equivalent, but faster version and executes it. We show a
high-level overview in Figure \ref{fig:system_overview}.

We have several desiderata for \system: it should facilitate applying complex
rewrites automatically with minimal overhead. Further, it should guarantee that
the rewritten code is semantically equivalent to the original code i.e.,
that it is sound, which presents the main technical challenge.

To guarantee soundness, we first formalize the rewrites
(Section~\ref{sec:pandas_rewr_rules}). Second, we accurately describe the
conditions under which a rewrite can be applied, some of which can be quite
involved. For example, some require \system{} to check the form of whole
functions. And finally, most of these conditions can only be checked at runtime,
and checking them at the correct program points requires delicate program
transformations (Section~\ref{sub-sec:rewriter}).

\hfill{}

% In the subsequent sections, we describe how we designed the pattern matcher (Section~\ref{sub-sec:patt-match}) and the 
% rewriter (Section~\ref{sub-sec:rewriter}) to overcome these challenges. First, we introduce the structure of the rewrite rules briefly in Section~\ref{sec:pandas_rewr_rules}.

% In this Figure, the pattern-matcher matched an LHS (highlighted in green); a substring search for the string \code{'G'} in \code{df['Name']}. We can do the same substring search but much faster. Note that the pattern-matcher, in accordance with our descriptions of the LHS's, identifies purely syntactic patterns. So, it does not (and cannot) perform semantic-related or dynamic checks.

% Having matched an LHS, \system{} uses. For example, to rewrite the substring search, the name \code{df} should be of type \code{DataFrame}, otherwise the rewriting is (possibly) invalid. Most of the time, the rewriter can bake these dynamic checks as part of the rewritten code (as in Figure \ref{fig:system_overview}). This is not always possible though but we will return to this later.



%\begin{figure}
  \begin{subfigure}[t]{\columnwidth}
\begin{minted}[bgcolor=light-gray]{python}
df['original_price'] = 
  df['original_price'].str.split('$')
df['original_price'] = df['original_price'].str[1]
\end{minted}
  \caption{Original}
  \end{subfigure}
  \hfill
  \begin{subfigure}[t]{\columnwidth}
\begin{minted}[bgcolor=light-gray]{python}
res = []
col = df['original_price']
for s in col.tolist():
    spl = s.split('$')
    res.append(spl[1])
df['original_price'] = pd.Series(res, ser.index)
\end{minted}
  \caption{Rewritten}
  \end{subfigure}
  \caption{An example of a multi-statement pattern (omitting runtime preconditions)}
  \label{fig:multi_stmt_patt}
\end{figure}

% It looks better to me if we split it vertically but I couldn't figure out how
% to align it properly. The web says to use [t], which I did, but still it looks
% weird.

% \begin{figure}
%   \begin{subfigure}[t]{0.45\columnwidth}
%     \begin{minipage}{\linewidth}%
% \begin{minted}[bgcolor=light-gray]{python}
% df['A'].sort_values()
%        .head(n=5)
% \end{minted}
% \end{minipage}
%     \caption{Select the 5 smallest elements by sorting first}
%     \label{fig:sort_values}
%   \end{subfigure}
%   \hfill %%
%   \begin{subfigure}[t]{0.45\columnwidth}
%     \begin{minipage}{\linewidth}%
% \begin{minted}[bgcolor=light-gray]{python}
% df['A'].nsmallest(n=5)
% \end{minted}
% \end{minipage}
%     \caption{Select the 5 smallest elements directly}
%     \label{fig:nsmallest}
%   \end{subfigure}
%   \caption{Rewriting Example \stef{This figure looks bad, not sure how to fix it now}
%   \ddkang{manual v/hspacing}
%   }
%   \label{fig:example_rewrite}
% \end{figure}




\begin{figure}
  \begin{subfigure}{\columnwidth}
\begin{minted}[bgcolor=light-gray]{python}
df['A'].sort_values().head(n=5)
\end{minted}
    \caption{Select the 5 smallest elements by sorting first. Extracted from a
    Kaggle notebook \cite{real_nb_sort_values}.}
    \label{fig:sort_values}
  \end{subfigure}
  \hfill
  \begin{subfigure}{\columnwidth}
\begin{minted}[bgcolor=light-gray]{python}
df['A'].nsmallest(n=5)
\end{minted}
    \caption{Select the 5 smallest elements directly. This avoids sorting.}
    \label{fig:nsmallest}
  \end{subfigure}
  \caption{Selecting the 5 smallest elements. By comprehending
  the \code{pandas} API, \system{} is able to recognize that the second version
  is equivalent to, and faster than, the first.}
  \label{fig:example_rewrite}
\end{figure}
\subsection{Pandas Rewrite Rules}
\label{sec:pandas_rewr_rules}

% \begin{figure}
%     \begin{minted}[bgcolor=light-gray,escapeinside=||]{python}

% @{expr: called_on}
%    .sort_values()
%    .head(n=@{Constant(int): first_n}) |$\mapsto$|
% @{called_on}.nsmallest(n=@{first_n})
%    |$\mathcal{C}$|(type(@{called_on}) == DataFrame)
%     \end{minted}

%     \caption{A rewrite rule example}
%     \label{fig:sort-values-rule}
% \end{figure}


\begin{figure}
    \begin{subfigure}{\columnwidth}
  \begin{minted}[bgcolor=light-gray,escapeinside=||]{python}
@{expr: called_on}.sort_values().head(n=@{Const(int): first_n})
|{\LARGE\color{brown}$\mapsto$}|
@{called_on}.nsmallest(n=@{first_n})
  \end{minted}
      \caption{LHS {\color{brown}$\mapsto$} RHS}
      \label{fig:sort-values-rule-lhs-rhs}
    \end{subfigure}
    \hfill
    \begin{subfigure}{\columnwidth}
  \begin{minted}[bgcolor=light-gray,escapeinside=||]{python}
type(@{called_on}) == pandas.Series
  \end{minted}
      \caption{Preconditions}
      \label{fig:sort-values-rule-preconds}
    \end{subfigure}
    \caption{\revis{An example of a rewrite rule, named \textbf{SortHead}}. If we match the LHS in the source code,
    we can replace it with the RHS only if the preconditions hold (at runtime).}
    \label{fig:sort-values-rule}
  \end{figure}

% % Center column names
% \newcolumntype{P}[1]{>{\centering\arraybackslash}p{#1}}

% {\footnotesize
% \begin{table*}
% \begin{tabular}{|P{0.33\textwidth}|P{0.33\textwidth}|P{0.33\textwidth}|}
% {\LARGE LHS} &  {\LARGE RHS} & {\LARGE Preconditions} \\
% \hline \\
% \begin{lstlisting}[language=Python,basicstyle=\ttfamily, breaklines=true]
% @{Name: a}, @{Name: b} =
%   @{expr: ser}.str.split(
%     @{Constant(str): sep}, 
%       expand=@{Constant(bool): expand})
% \end{lstlisting}
% &
% \vspace{-15pt}
% \begin{lstlisting}[language=Python,basicstyle=\ttfamily, breaklines=true]
% a, b = [], []
% for it in @{ser}.tolist():
%     spl = it.split(@{sep})
%     a.append(spl[0])
%     y = spl[1] if len(spl) > 1 \
%           else None
%     b.append(y)
% @{a} = pandas.Series(a, @{ser}.index)
% @{b} = pandas.Series(b, @{ser}.index)
% \end{lstlisting}
% &
% \begin{lstlisting}[language=Python,basicstyle=\ttfamily, breaklines=true, escapeinside=||]
% |$\mathfrak S$:| @{expand} == True
% |$\mathfrak R$:| type(@{ser}) == pandas.Series
% \end{lstlisting}

% \\
% \hline
% \\

% \vspace{-12pt}
% \begin{lstlisting}[language=Python,basicstyle=\ttfamily, breaklines=true, escapeinside=||]
% @{expr: ser}.apply(
%   lambda @{Name: par1}:
%     @{Constant(str): needle} 
%       in @{Name: par2})
% \end{lstlisting}
% &
% \vspace{-5pt}
% \begin{lstlisting}[language=Python,basicstyle=\ttfamily, breaklines=true, escapeinside=||]
% res = @{ser}.tolist()
% res = [(@{needle} in s) for s in res]
% pandas.Series(res, @{ser}.index)
% \end{lstlisting}
% &
% \begin{lstlisting}[language=Python,basicstyle=\ttfamily, breaklines=true, escapeinside=||]
% |$\mathfrak R$:| type(@{ser}) == pandas.Series
% \end{lstlisting}

% \\
% \hline
% \\

% \vspace{-10pt}
% \begin{lstlisting}[language=Python,basicstyle=\ttfamily, breaklines=true, escapeinside=||]
% @{Name: df1}[@{Constant(str): c1}] = 
%   @{Name: df2}[@{Constant(str): c2}]
%     .fillna(@{expr: arg})
% \end{lstlisting}
% &
% \vspace{-5pt}
% \begin{lstlisting}[language=Python,basicstyle=\ttfamily, breaklines=true, escapeinside=||]
% @{df1}[@{c1}].fillna(@{arg}, inplace=True)
% \end{lstlisting}
% &
% \vspace{-10pt}
% \begin{lstlisting}[language=Python,basicstyle=\ttfamily, breaklines=true, escapeinside=||]
% |$\mathfrak S$:| @{df1} == @{df2}
% |$\mathfrak S$:| @{c1} == @{c2}
% |$\mathfrak R$:| type(@{df1}) == pandas.DataFrame
% \end{lstlisting}

% \\
% \hline
% \\

% \vspace{-15pt}
% \begin{lstlisting}[language=Python,basicstyle=\ttfamily, breaklines=true, escapeinside=||]
% pd.Series(
%   @{expr: e1}.tolist() + 
%   @{expr: e2}.tolist())
% \end{lstlisting}
% &
% \vspace{-10pt}
% \begin{lstlisting}[language=Python,basicstyle=\ttfamily, breaklines=true, escapeinside=||]
% pd.concat([@{e1}, @{e2}], ignore_index=True)
% \end{lstlisting}
% &
% \vspace{-10pt}
% \begin{lstlisting}[language=Python,basicstyle=\ttfamily, breaklines=true, escapeinside=||]
% |$\mathfrak R$:| pd == pandas
% |$\mathfrak R$:| type(@{e1}) == pandas.Series
% |$\mathfrak R$:| type(@{e2}) == pandas.Series 
% \end{lstlisting}

% \end{tabular}
% \vspace{10pt}
% \caption{Rewrite Rule Examples \stef{There are line cuts in the 3rd column}}
% \label{tbl:rewr_rules}
% \end{table*}
% }

% % MINTED DIDN'T WORK.
% %
% % \begin{figure*}
% % \begin{tabular}{ |p{0.3\textwidth} | p{0.3\textwidth} | p{0.3\textwidth} | }
% %   \begin{minted}[escapeinside=||]{python}
% %     @{expr: called_on}
% %     .sort_values()
% %     .head(n=@{Constant(int): first_n})
% %   \end{minted}
% %   &
% %   \begin{minted}[escapeinside=||]{python}
% %     @{expr: called_on}
% %     .sort_values()
% %     .head(n=@{Constant(int): first_n})
% %   \end{minted}
% %   &
% %   \begin{minted}[escapeinside=||]{python}
% %     @{expr: called_on}
% %     .sort_values()
% %     .head(n=@{Constant(int): first_n})
% %   \end{minted}
% % \end{tabular}
% % \end{figure*}

\begin{table*}[t]
  \centering
  \includegraphics[height=0.85\textheight]{figures/sources/tbl_rewr_rules.pdf}
  \caption{Examples of Rewrite Rules. If any of the LHS's is matched, it can be
  replaced with the corresponding RHS, provided that the preconditions hold. The
  symbol $\mathfrak S$ denotes syntactic preconditions while $\mathfrak R$
  denotes runtime ones. The name of the rule appears as a comment in the LHS
  column.}
  \label{tbl:rewr_rules}
\end{table*}

%\begin{figure}

\begin{mathpar}
\inferrule
{(\code{@\{name\}}, \Gamma_{1}) \Downarrow Obj, \Gamma_{2} \\ (\code{type(Obj)}, \Gamma_{2}) \Downarrow \code{DataFrame}, \Gamma_{2} \\ (LHS, \Gamma_{1}) \Downarrow VLHS, \Gamma_{4} \\ (RHS, \Gamma_{1}) \Downarrow VRHS, \Gamma_{5} }
{VLHS = VRHS \wedge \Gamma_{4} = \Gamma_{5}}
\end{mathpar}

    \caption{Formal Semantics the Rewrite Rule in Figure \ref{fig:sort-values-rule}}
    \label{fig:sort-values-formal}
\end{figure}

The abstract form of the rewrite rules \system{} supports can be modeled as transforming a Left Hand Side (LHS) set of statements to Right Hand Side (RHS) set of statements subject to certain preconditions on the LHS. 
We introduce some notation to show the structure of our parameterized rewrite rules. The parameterized portions of the rewrite rules are general and can match multiple valid code segments subject to certain conditions (e.g. types). For example consider the original code in Figure~\ref{fig:example_rewrite}(a) rewritten to Figure~\ref{fig:example_rewrite}(b) using the rewrite rule shown in Figure~\ref{fig:sort-values-rule}. %For example consider the rewrite rule shown in Figure~\ref{fig:sort-values-rule}. \system{} uses it to perform the rewrite shown in Figure~\ref{fig:example_rewrite}. We begin with the LHS. 
A \code{@\{...\}} entry denotes a parameterized part of the rule. These parts
can be matched to multiple valid options by \system{}. Inside the curly brackets,
we describe these valid options using a derivation rule of the Python grammar
\cite{python_ast}. For example, \code{@\{expr\}} denotes that any expression can
appear in its place. For \code{Constant}s, we optionally specify the type of the
constant inside parentheses.\footnote{We can determine the type of constants
from the AST \cite{python_ast_constants}.} So, \code{@\{Constant(int)\}} denotes
that any integer constant can appear in its place. We need to refer to the parts of the LHS that are parameterized in the preconditions and the RHS. So, we bind these parts to names. For example, the code string \code{df.sort\_values().head()} matches the LHS of Figure~\ref{fig:sort-values-rule} and \code{called\_on} is bound to \code{df}. Everything that is not in \code{@\{\}} should appear as is.
%It is important to note that the LHS section is concerned purely with syntax. This implies that the bindings, and their uses in the other sections, are purely \textit{textual}; no evaluation happens (a name binding in the LHS is like a macro definition in C). 
With these in mind, we can read the LHS of Figure~\ref{fig:sort-values-rule} as matching any Python expression on which \code{sort\_values()} is applied, followed by \code{head()} with any constant integer as the argument of the formal parameter \code{n}.

There are two kinds of preconditions, syntactic and runtime ones. Syntactic preconditions describe conditions related to the matched \textit{text}. Usually, they require that two matched entries of the LHS are \emph{syntactically} equal. The runtime preconditions describe conditions which have to hold at \emph{runtime} for the original (LHS) and the rewritten code (RHS) to be semantically equivalent and they are expressed in Python syntax and semantics. For example, in Figure \ref{fig:sort-values-rule}, the result of the \code{called\_on} expression that was matched in the LHS should be a \code{pandas.DataFrame}. The runtime preconditions implicitly impose an order of evaluation. In this example, \code{called\_on} must be evaluated first, then the preconditions are checked on the resulting object, and then this object is used in place of \code{called\_on} in the RHS. Note that unconditionally evaluating \code{called\_on} is correct even if the conditions do not hold because it would be evaluated anyway in the original.

%Section~\ref{sec:Evaluation} presents in-depth case studies on real world Pandas codes where non-trivial rewrites enable significant performance improvements. For brevity, we only show the parameterized form of the rewrite rule when the fixed and parameterized parts of the rewrite is not apparent from the context.

% The preconditions should be evaluated when the LHS would be evaluated. If the preconditions pass, then evaluating the RHS should be functionally equivalent with evaluating the RHS. %Finally, in the RHS, we just list the code with which we replace the LHS. %We give a more formal description in Figure \ref{fig:sort-values-formal} in the form of operational semantics \stef{Give citation; not worth describing here}. %\stef{Not sure how to write this last sentence}. %We consider such formal descriptions to be out of scope for this paper, so for the rest of it, we will use our simplified notation of preconditions as in Figure \ref{fig:sort-values-rule}.


% The $\Gamma$'s are contexts, which for simplicity can be thought as snapshots of the Python namespace. Above horizontal line there is a set of premises, which, if true, they entail the statements below the line. The premise $(E, \Gamma_{1}) \Downarrow V, \Gamma_{2}$ states that if the expression $E$ is evaluated under context $\Gamma_{1}$, then its value is $V$ and it updates the context to $\Gamma_{2}$. In short, the rule says that if we evaluate \code{called\_on} first and then check the precondition in Figure \ref{fig:sort-values-rule}, then evaluating the LHS should be functionally equivalent with evaluating the RHS.

%The interpretation of the RHS section is the same as that of the LHS section.

%\charith{Too low level explain with high-level constructs}. \stef{Not sure what this means :/}


Table~\ref{tbl:rewr_rules} shows three more rewrite rules we use in \system{}. The first two correspond to the examples in Figure~\ref{fig:split} and Figure~\ref{fig:concat-with-lists}, respectively. Rules can have both runtime and syntactic conditions. For example, in the second rule, we have the syntactic precondition \code{@\{par1\} == @\{par2\}} requires that the two names be equal. To differentiate between the two kinds of preconditions, we prefix the syntactic preconditions with $\mathfrak S$ and the runtime ones with $\mathfrak R$.

% Easter egg: This style of typesetting was used in Turing's seminal paper.

%These rewrite rules span simple .... to complex ... \stef{Can't have diversity in how complex the rules are. Either we have such simple rules or way more complicated like vectorization. . As an alternative, I put rules that are tricky and explained all of them concisely.} \charith{Explain 2 rules concisely from the table}. 

% The first rule is the one in Figure~\ref{fig:split} that we discussed earlier. In Section~\ref{sub-sec:case-studies} we present an in-depth case study. %Finally, we have yet another rule that crosses the library boundaries as we move from Pandas to Python (by converting to a list) and then back to Pandas. The interesting thing here is in the preconditions: the name \code{pd} must be bound to the Pandas module. It is possible that there is another module that the user imported as \code{pd} which has a function \code{Series} that accepts lists.

%We note that Series are not just collections of elements. They also have an index with which the elements are accessed. Usually, this is a \code{RangeIndex} \cite{pandas_range_idnex}, i.e. a contiguous range of integers as in Python lists. But it need not be, and thus in rules where the rewritten version is implemented using pure Python and lists, when converting back to a \code{Series}, we need to preserve the original index.

 

%We will now present \system{}, a system that accelerates interactive data science workloads by transparently rewriting Python that interfaces with Pandas code and which addresses all the previously mentioned challenges. In order to do so, \system{} has two high-level components. First, \system{}' \textit{syntactic pattern-matcher} identifies matches the input code against the LHS parts of the rewrite rules. The second component is a \emph{rewriter}, which validates the preconditions of the rewrite rules and on passing them, rewrites the code to the RHS version and executes it. We show a high-level overview in Figure \ref{fig:system_overview}.

%\stef{Elaboration on the Figure?}

%We now describe each component in depth.


\section{\system{} Rewrite System}

\system{} consists of two main parts: a syntactic pattern matcher and a rewriter
that rewrites the code matched against patterns. We now describe how the two
parts were designed in detail.

\subsection{\system{} Pattern Matcher}
\label{sub-sec:patt-match}

% % Thanks to Hydride team. Got part of the latex from there.

% It's not an algorithm but I liked the \begin{algorithm} format.
\floatname{algorithm}{Listing}

%%\renewcommand{\algorithmiccomment}[1]{\hfill\eqparbox{COMMENT}{#1}}
\renewcommand{\algorithmiccomment}[1]{\bgroup\hfill\footnotesize//~#1\egroup}
\algrenewcommand\algorithmicindent{1.0em}%

\begin{algorithm}[t!]
\caption{A Sketch of \system{}' Pattern Matcher}
\label{lst:patt-match}
%%\akash{Algorithm modified. Font size is smaller now and comments are in the same line as the pseudocode}
\normalsize{
\begin{algorithmic}[1]
\Function{LambdaSubstrSearch}{stmt}

\State \textit{// Return True if stmt is a ast.Lambda}
\State \textit{// that performs substring search.}

\EndFunction
\Function{PatternMatch}{stmt}
\ForAll{node \textbf{in} stmt} 
    \If{node \textbf{is} ast.Call}
        \If{node.func.attr = "apply"}
            \State arg0 $\gets$ node.func.args[0]
            \If{\Call{LambdaSubstrSearch}{arg0}}
                \State \Return SubstringSearchApply
            \EndIf
            \If{isinstance(arg0, ast.Name)}
              \If{node.func.args[1] \textbf{is} axis=1}
                \State \Return ApplyAxis1
              \EndIf
            \EndIf
        \EndIf
    \EndIf
\EndFor
\State \Return \textbf{None}
\EndFunction

\end{algorithmic}
}
\end{algorithm}
\setlength{\textfloatsep}{15pt}


The pattern matcher is responsible for matching a sequence of statements
with the \textbf{LHS} part of any rewrite rule. Whether any \textbf{LHS}
(represented as an AST) matches any part of the source AST, is essentially a
sub-tree search problem. The pattern matcher performs a greedy search and it
returns the first LHS it matches.

To minimize matching overhead, we designed the pattern matcher to be
hierarchical, by factoring patterns based on their commonalities. The common
parts are matched first before hierarchically matching more specific components
of a rule. This eliminates repeatedly matching components that are common to
multiple rules.

% Consider the pattern-matching code that matches two patterns: the third pattern
% of Table~\ref{tbl:rewr_rules} and the one that enables the rewrite of
% Figure~\ref{fig:apply_only_math}. The LHS of the former is shown in the table.
% The LHS of the latter is \code{@\{expr: e\}.apply(@\{Name: fun\}, axis=1)} (see
% Section~\ref{sub-sec:rewriter}). Notice that these LHS's share parts; they both
% require a function call, that is an attribute of some expression and the name of
% the function is \code{apply}. We want to check the common parts of the pattern
% at a single place to exploit commonalities across patterns.
% Listing~\ref{lst:patt-match} shows a sketch of the pattern-matching code that
% matches these two patterns. %The input is a statement represented as an AST. The
% function returns an object representing a pattern if it matches one, or None
% otherwise. It recursively loops through all the AST nodes of type \code{stmt}
% and checks for the two patterns by first checking for an attribute function
% called \code{apply} and then matching against either one of the patterns
% specifically.

Lastly, the pattern matcher needs to be able to match patterns that span
multiple statements. Having a function that matches single-statement patterns by
performing a greedy search, there is another function that matches multiple
statements. The latter function operates on a higher level, viewing
multi-statement patterns as sets of smaller ones. So, for a 2-statement pattern,
if it matches the first part, it then checks the next statement for the second
part.

\subsection{\system{} Rewriter}
\label{sub-sec:rewriter}

%
\subsection{Pandas Code Rewriting}

\stef{Section overview?}



\charith{Let's use the following paragraph structure:
\begin{itemize}
    \item Introducing rewrites
    \item Non-obvious rewrites
    \item Why an automatic rewrite system is neededs of such a system over exisiting solutions (it does not have the 3 drawbacks, plus it gets good speedups)
    \item What are the non-obvious benefits we get from a rewrite system (going beyond abstraction boundary)
\end{itemize}}

%% purpose of the para: introducing rewrites and why other systems may not address it.
Rewriting, for optimization purposes, is the process of replacing some part of code with a functionally equivalent but faster version. As an example, consider the task of selecting the 5 smallest elements of a \code{DataFrame} column. Figure \ref{fig:sort_values} shows user-written code, extracted from a Kaggle notebook, that performs this task. The same code can be rewritten as in Figure \ref{fig:nsmallest}. Figure \ref{fig:sort_values} first sorts the values and then selects the first 5 elements whereas the code in Figure \ref{fig:nsmallest} uses the Pandas function \code{nsmallest}, which selects the \code{n} smallest elements directly. As we can expect, Figure \ref{fig:nsmallest} is faster. Similarly, there are many opportunities in real-world notebooks to perform rewrites into more optimized versions. %We describe such rewrite opportunities in detail in Section~\ref{}.

Rewriting appears simple, but it can be challenging when performed manually. There are many non-obvious rewrites that the user may not be able to discover easily. For example, it might seem that the only way to make \code{pandas} code faster through rewriting is by replacing it with other \code{pandas} code, or using a similar library such as \code{numpy}. This has been reinforced over years of data scientists being trained to remain within \code{pandas}/\code{numpy} as much as possible because these use native, vectorized implementations and are thus much faster pure Python.

\begin{figure}
  \begin{subfigure}{\columnwidth}
\begin{minted}[bgcolor=light-gray]{python}
df[['a', 'b']] = df['C'].str.split('(', expand=True)
\end{minted}
    \caption{Splitting a \code{pandas.Series} using
    \code{pandas.Series.str.split()}. Extracted from a Kaggle notebook \cite{real_nb_split}.}
    \label{fig:split_pandas}
  \end{subfigure}
  \hfill
  \begin{subfigure}{\columnwidth}
\begin{minted}[bgcolor=light-gray]{python}
a = []
b = []
ls = df['C'].tolist()
for it in ls:
    spl = it.split('(', 1)
    a.append(spl[0])
    b.append(spl[1] if len(spl) > 1 else None)
df['a'] = pd.Series(a, df['C'].index)
df['b'] = pd.Series(b, df['C'].index)
\end{minted}
    \caption{Splitting a \code{pandas.Series} in pure Python}
    \label{fig:split_python}
  \end{subfigure}
  \caption{Splitting in \code{pandas} and Python. Surprisingly, the pure Python
  implementation is up to 7$\times$ faster.}
  \label{fig:split}
\end{figure}

It might, then, be surprising that moving out of \code{pandas} and into pure Python can lead to (significant) speedups. One example is shown in Figure \ref{fig:split}. The task here is to split a \code{Series} of strings by the delimiter \code{'('}. The code in Figure~\ref{fig:split_pandas} (extracted from a Kaggle notebook) does it by using a \code{pandas}-provided function. One would expect that this is the best way to perform this operation. Nevertheless, the version in Figure \ref{fig:split_python} is 3.5$\times$ faster. %What this code does is seemingly completely unorthodox. 
It moves from \code{pandas} to pure Python (by converting \code{df['C']} to a Python list) and performs the operation with a sequential Python loop \footnote{It then converts the results to \code{Series} for functional equivalence with the original. Note that it has to have the same index; more on that later.} (in our case studies in Section~\ref{sub-sec:case-studies}, we explain why this version is faster). We detail some of the rewrite rules we use in Section~\ref{sec:pandas_rewr_rules} including these complex rules.

It is unreasonable to expect general \code{pandas} users to comprehend Python, \code{pandas}, and \code{numpy} to such an extensive level so as to be able to discover such equivalent versions and evaluate their relative performance. Second, even if the user succeeds in these tasks, the rewritten version can be significantly harder to write and read, as is evident from Figure \ref{fig:split}. This can further lead to correctness concerns about the rewrite. Third, manual rewriting breaks the library abstraction. In the original code of Figure \ref{fig:split}, the user has to think only of \emph{what} \code{split()} does. But, to come up with the rewritten version, this abstraction is broken as the user needs to think of \emph{how} to implement it. %Finally, rewriting becomes more complicated once correctness concerns arise. 

%\begin{figure}
  \begin{subfigure}{\columnwidth}
\begin{minted}[bgcolor=light-gray]{python}
def foo(row):
  if row['A'] == row['B'] and row['A'] < row['C']:
    return 'X'
  elif row['A'].startswith('Y'):
    return 'Y'
  elif row['B'] in ls:
    return 'Z'
  else:
    return 'NA'

df.apply(foo, axis=1)
\end{minted}
    \caption{Original \code{pandas} \code{apply()}. It processes each row
    sequentially, using the interpreter.}
    \label{fig:apply_vectorized_orig}
  \end{subfigure}
  \hfill
  \begin{subfigure}{\columnwidth}
\begin{minted}[bgcolor=light-gray]{python}
conditions = [
  (df['A'] == df['B']) & (df['A'] < df['C']),
  df['A'].str.startswith('Y'),
  df['B'].isin(ls)
]
choices = [
  'X', 'Y', 'Z'
]
np.select(conditions, choices, default='NA')
\end{minted}
    \caption{Vectorized execution using \code{numpy.select()}}
    \label{fig:apply_vectorized_rewr}
  \end{subfigure}
  \caption{VectorizedConditionals: Vectorized \code{apply()} with conditions, which can be hundreds of
  times faster \cite{pygotham_apply_vectorized}. However, performing this
  rewrite automatically is challenging.}
  \label{fig:apply_vectorized}
\end{figure}

This motivated us to build \system{}, a system that performs such rewrites \textit{automatically}, by guaranteeing \textit{correctness} and with minimal overhead. Section \ref{sec:System} describes the internal workings of \system{} in detail. 

Rewriting avoids the previously mentioned drawbacks of library-based optimization systems. First, it inherently does not suffer from a lack of API support since the rewritten code uses only \code{pandas} and Python features that are already supported. \stef{The previous seems to miss the important and exciting fact that the rewriter is fundamentally different. The lack of API support does not even make sense in this context. I'd rewrite it as: "because it is fundamentally different: it is not a replacment for \code{pandas} and it can always leave the code untouched if it cannot handle it.}  Second, the rewriting overheads scale proportionally only to the code, not the data. %In fact, the code of a single cell. 
%IPython notebooks (see Section \ref{sec:Implementation)}) are populated incrementally and thus the rewriter cannot know all the code. 
\system{} considers only a single cell at a time \footnote{it could also consider the history, although we have not found that necessary} and needs to save only the AST representation of the code and a few light and ephemeral internal data structures during rewrite code generation. %Further, the rewritten code only have a few dynamic checks. 
These overheads are thus negligible both in terms of memory and execution time \charith{Do we have numbers to show this overhead is low?}. \stef{Yes. I have to add plots}
%On the other hand, pattern-matching and rewriting reduce to fast tree search and a few dynamic checks.
%These overheads are thus negligible both in terms of memory and execution time.
Finally, when the rewriter succeeds, the rewritten code is almost always faster than the original, to the extent that there are such patterns. As we will show in Section \ref{sec:Evaluation}, the patterns we have used always result in speedups except for degenerate cases. \stef{Do we need data that the patterns always result in a speedup?}

Additionally, there are fundamental advantages \system{} has over library based optimization approaches. The rewrite system is transparent. When the user observes a speedup, they can always see the code that the rewriter used. In other words, the user does not need to understand the system to understand the cause of the speedup. At the same time, the user's code remains intact. Further, rewriting has the benefit of being able to optimize across library boundaries. For example, \system{} can automatically perform the rewrite that we described in Figure~\ref{fig:split}. \stef{This example does not drive the point home because a library could do it. That's why I had another simple example which cross library boundaries} To perform this rewrite, a tool needs to view all the code and understand semantic equivalences and differences across library boundaries (in this case \code{pandas} and the host language, Python). This is not possible with optimization appproaches that purely aim at accelerating Panadas functions.  %However, a rewriter is naturally such a tool as it views all the user code. 
However, because the rewriter views, and understands, both the library semantics and also the host language semantics (in this case, Python), it can optimize across library boundaries.


%There are many advantages of using automatic Pandas code rewriting to optimize Pandas workloads. One such advantage is that an automatic rewriting system preserves the readability and interpretability of the code. When the user observes a speedup, they can always see the code that the rewriter used. In other words, the user does not need to understand the system to understand the speedup cause. At the same time, the user's code remains intact in case they want to change it later. Second, a rewriting system has low implementation complexity. By complexity, we do not mean lines of code, as this is an irrelevant metric. Rather, the really burdensome complexity is the mental one, which is increased mostly as the number of interactions between different parts of code increases. A rewriting system's complexity is low because it needs only clearly separated components. Furthermore, the complexity does not increase with every new pattern because the patterns are independent.

%Rewriting also avoids the previously mentioned drawbacks. First, a rewriting system for Pandas is not a replacement for Pandas but a clear addition. Therefore, it inherently does not suffer from a lack of API support. Second, the rewriting overheads scale proportionally only to the code, not the data. In fact, the code of a single cell. IPython notebooks (see Section \ref{sec:Implementation)} are populated incrementally and thus the rewriter cannot know all the code. Thus, it has to consider a single cell at a time \footnote{it could also consider the history, although we have not found that necessary}. These overheads are thus negligible both in terms of memory and execution time. On the one hand, a rewriting system needs to save only the AST representation of the code and a few light internal data structures. On the other hand, pattern-matching and rewriting reduce to fast tree search and a few dynamic checks. Finally, when the rewriter succeeds, the rewritten code is almost always faster than the original, to the extent that there are such patterns. As we will show in Section \ref{sec:Evaluation}, the patterns we have used always result in speedups except for degenerate cases. \stef{Do we need data that the patterns always result in a speedup?}

%\begin{figure}
  \begin{subfigure}{\columnwidth}
\begin{minted}[bgcolor=light-gray]{python}
pd.Series(df['A'].tolist() + df['B'].tolist())
\end{minted}
    \caption{Original: Concatenate \code{Series} by first turning them into
    lists. Extracted from a Kaggle notebook \cite{real_nb_concat}.}
    \label{fig:concat_orig}
  \end{subfigure}
  \hfill
  \begin{subfigure}{\columnwidth}
\begin{minted}[bgcolor=light-gray]{python}
pd.concat([df['A'], df['B']], ignore_index=True)
\end{minted}
    \caption{Rewritten: Use a \code{pandas}-provided function for concatenation}
    \label{fig:concat_rewr}
  \end{subfigure}
  \caption{Rewrite example that crosses library boundaries, and thus cannot be
  performed by previous techniques. The rewritten version can be up to 11$\times$
  faster.}
  \label{fig:concat-with-lists}
\end{figure}


%As a last note, we note that rewriting has the non-obvious benefit of optimizing across library boundaries. Consider the case of concatenating two Pandas Series. Figure \stef{add it} shows user-written code extracted from a Kaggle notebook. The user (e.g., because they don't remember what Pandas API call performs the equivalent operation) first converts the Series into lists, concatenates them, then converts the result to a Series. Pandas provides a function to concatenate two Series directly, as shown in the rewritten version in Figure \stef{add it}. To perform this rewrite, a tool needs to view all the code because when converting the Series to lists, we \textit{cross} the library boundaries. At this point, Pandas (or any replacement for Pandas) has no control. A rewriter is naturally such a tool as it views all the user code. In summary, because the rewriter views, and understands, both the library semantics but also the host language semantics (in this case, Python), it can optimize across library boundaries.



%\subsection{Pandas Rewrite Rules}
\label{sec:pandas_rewr_rules}

% \begin{figure}
%     \begin{minted}[bgcolor=light-gray,escapeinside=||]{python}

% @{expr: called_on}
%    .sort_values()
%    .head(n=@{Constant(int): first_n}) |$\mapsto$|
% @{called_on}.nsmallest(n=@{first_n})
%    |$\mathcal{C}$|(type(@{called_on}) == DataFrame)
%     \end{minted}

%     \caption{A rewrite rule example}
%     \label{fig:sort-values-rule}
% \end{figure}


\begin{figure}
    \begin{subfigure}{\columnwidth}
  \begin{minted}[bgcolor=light-gray,escapeinside=||]{python}
@{expr: called_on}.sort_values().head(n=@{Const(int): first_n})
|{\LARGE\color{brown}$\mapsto$}|
@{called_on}.nsmallest(n=@{first_n})
  \end{minted}
      \caption{LHS {\color{brown}$\mapsto$} RHS}
      \label{fig:sort-values-rule-lhs-rhs}
    \end{subfigure}
    \hfill
    \begin{subfigure}{\columnwidth}
  \begin{minted}[bgcolor=light-gray,escapeinside=||]{python}
type(@{called_on}) == pandas.Series
  \end{minted}
      \caption{Preconditions}
      \label{fig:sort-values-rule-preconds}
    \end{subfigure}
    \caption{\revis{An example of a rewrite rule, named \textbf{SortHead}}. If we match the LHS in the source code,
    we can replace it with the RHS only if the preconditions hold (at runtime).}
    \label{fig:sort-values-rule}
  \end{figure}

% % Center column names
% \newcolumntype{P}[1]{>{\centering\arraybackslash}p{#1}}

% {\footnotesize
% \begin{table*}
% \begin{tabular}{|P{0.33\textwidth}|P{0.33\textwidth}|P{0.33\textwidth}|}
% {\LARGE LHS} &  {\LARGE RHS} & {\LARGE Preconditions} \\
% \hline \\
% \begin{lstlisting}[language=Python,basicstyle=\ttfamily, breaklines=true]
% @{Name: a}, @{Name: b} =
%   @{expr: ser}.str.split(
%     @{Constant(str): sep}, 
%       expand=@{Constant(bool): expand})
% \end{lstlisting}
% &
% \vspace{-15pt}
% \begin{lstlisting}[language=Python,basicstyle=\ttfamily, breaklines=true]
% a, b = [], []
% for it in @{ser}.tolist():
%     spl = it.split(@{sep})
%     a.append(spl[0])
%     y = spl[1] if len(spl) > 1 \
%           else None
%     b.append(y)
% @{a} = pandas.Series(a, @{ser}.index)
% @{b} = pandas.Series(b, @{ser}.index)
% \end{lstlisting}
% &
% \begin{lstlisting}[language=Python,basicstyle=\ttfamily, breaklines=true, escapeinside=||]
% |$\mathfrak S$:| @{expand} == True
% |$\mathfrak R$:| type(@{ser}) == pandas.Series
% \end{lstlisting}

% \\
% \hline
% \\

% \vspace{-12pt}
% \begin{lstlisting}[language=Python,basicstyle=\ttfamily, breaklines=true, escapeinside=||]
% @{expr: ser}.apply(
%   lambda @{Name: par1}:
%     @{Constant(str): needle} 
%       in @{Name: par2})
% \end{lstlisting}
% &
% \vspace{-5pt}
% \begin{lstlisting}[language=Python,basicstyle=\ttfamily, breaklines=true, escapeinside=||]
% res = @{ser}.tolist()
% res = [(@{needle} in s) for s in res]
% pandas.Series(res, @{ser}.index)
% \end{lstlisting}
% &
% \begin{lstlisting}[language=Python,basicstyle=\ttfamily, breaklines=true, escapeinside=||]
% |$\mathfrak R$:| type(@{ser}) == pandas.Series
% \end{lstlisting}

% \\
% \hline
% \\

% \vspace{-10pt}
% \begin{lstlisting}[language=Python,basicstyle=\ttfamily, breaklines=true, escapeinside=||]
% @{Name: df1}[@{Constant(str): c1}] = 
%   @{Name: df2}[@{Constant(str): c2}]
%     .fillna(@{expr: arg})
% \end{lstlisting}
% &
% \vspace{-5pt}
% \begin{lstlisting}[language=Python,basicstyle=\ttfamily, breaklines=true, escapeinside=||]
% @{df1}[@{c1}].fillna(@{arg}, inplace=True)
% \end{lstlisting}
% &
% \vspace{-10pt}
% \begin{lstlisting}[language=Python,basicstyle=\ttfamily, breaklines=true, escapeinside=||]
% |$\mathfrak S$:| @{df1} == @{df2}
% |$\mathfrak S$:| @{c1} == @{c2}
% |$\mathfrak R$:| type(@{df1}) == pandas.DataFrame
% \end{lstlisting}

% \\
% \hline
% \\

% \vspace{-15pt}
% \begin{lstlisting}[language=Python,basicstyle=\ttfamily, breaklines=true, escapeinside=||]
% pd.Series(
%   @{expr: e1}.tolist() + 
%   @{expr: e2}.tolist())
% \end{lstlisting}
% &
% \vspace{-10pt}
% \begin{lstlisting}[language=Python,basicstyle=\ttfamily, breaklines=true, escapeinside=||]
% pd.concat([@{e1}, @{e2}], ignore_index=True)
% \end{lstlisting}
% &
% \vspace{-10pt}
% \begin{lstlisting}[language=Python,basicstyle=\ttfamily, breaklines=true, escapeinside=||]
% |$\mathfrak R$:| pd == pandas
% |$\mathfrak R$:| type(@{e1}) == pandas.Series
% |$\mathfrak R$:| type(@{e2}) == pandas.Series 
% \end{lstlisting}

% \end{tabular}
% \vspace{10pt}
% \caption{Rewrite Rule Examples \stef{There are line cuts in the 3rd column}}
% \label{tbl:rewr_rules}
% \end{table*}
% }

% % MINTED DIDN'T WORK.
% %
% % \begin{figure*}
% % \begin{tabular}{ |p{0.3\textwidth} | p{0.3\textwidth} | p{0.3\textwidth} | }
% %   \begin{minted}[escapeinside=||]{python}
% %     @{expr: called_on}
% %     .sort_values()
% %     .head(n=@{Constant(int): first_n})
% %   \end{minted}
% %   &
% %   \begin{minted}[escapeinside=||]{python}
% %     @{expr: called_on}
% %     .sort_values()
% %     .head(n=@{Constant(int): first_n})
% %   \end{minted}
% %   &
% %   \begin{minted}[escapeinside=||]{python}
% %     @{expr: called_on}
% %     .sort_values()
% %     .head(n=@{Constant(int): first_n})
% %   \end{minted}
% % \end{tabular}
% % \end{figure*}

\begin{table*}[t]
  \centering
  \includegraphics[height=0.85\textheight]{figures/sources/tbl_rewr_rules.pdf}
  \caption{Examples of Rewrite Rules. If any of the LHS's is matched, it can be
  replaced with the corresponding RHS, provided that the preconditions hold. The
  symbol $\mathfrak S$ denotes syntactic preconditions while $\mathfrak R$
  denotes runtime ones. The name of the rule appears as a comment in the LHS
  column.}
  \label{tbl:rewr_rules}
\end{table*}

%\begin{figure}

\begin{mathpar}
\inferrule
{(\code{@\{name\}}, \Gamma_{1}) \Downarrow Obj, \Gamma_{2} \\ (\code{type(Obj)}, \Gamma_{2}) \Downarrow \code{DataFrame}, \Gamma_{2} \\ (LHS, \Gamma_{1}) \Downarrow VLHS, \Gamma_{4} \\ (RHS, \Gamma_{1}) \Downarrow VRHS, \Gamma_{5} }
{VLHS = VRHS \wedge \Gamma_{4} = \Gamma_{5}}
\end{mathpar}

    \caption{Formal Semantics the Rewrite Rule in Figure \ref{fig:sort-values-rule}}
    \label{fig:sort-values-formal}
\end{figure}

The abstract form of the rewrite rules \system{} supports can be modeled as transforming a Left Hand Side (LHS) set of statements to Right Hand Side (RHS) set of statements subject to certain preconditions on the LHS. 
We introduce some notation to show the structure of our parameterized rewrite rules. The parameterized portions of the rewrite rules are general and can match multiple valid code segments subject to certain conditions (e.g. types). For example consider the original code in Figure~\ref{fig:example_rewrite}(a) rewritten to Figure~\ref{fig:example_rewrite}(b) using the rewrite rule shown in Figure~\ref{fig:sort-values-rule}. %For example consider the rewrite rule shown in Figure~\ref{fig:sort-values-rule}. \system{} uses it to perform the rewrite shown in Figure~\ref{fig:example_rewrite}. We begin with the LHS. 
A \code{@\{...\}} entry denotes a parameterized part of the rule. These parts
can be matched to multiple valid options by \system{}. Inside the curly brackets,
we describe these valid options using a derivation rule of the Python grammar
\cite{python_ast}. For example, \code{@\{expr\}} denotes that any expression can
appear in its place. For \code{Constant}s, we optionally specify the type of the
constant inside parentheses.\footnote{We can determine the type of constants
from the AST \cite{python_ast_constants}.} So, \code{@\{Constant(int)\}} denotes
that any integer constant can appear in its place. We need to refer to the parts of the LHS that are parameterized in the preconditions and the RHS. So, we bind these parts to names. For example, the code string \code{df.sort\_values().head()} matches the LHS of Figure~\ref{fig:sort-values-rule} and \code{called\_on} is bound to \code{df}. Everything that is not in \code{@\{\}} should appear as is.
%It is important to note that the LHS section is concerned purely with syntax. This implies that the bindings, and their uses in the other sections, are purely \textit{textual}; no evaluation happens (a name binding in the LHS is like a macro definition in C). 
With these in mind, we can read the LHS of Figure~\ref{fig:sort-values-rule} as matching any Python expression on which \code{sort\_values()} is applied, followed by \code{head()} with any constant integer as the argument of the formal parameter \code{n}.

There are two kinds of preconditions, syntactic and runtime ones. Syntactic preconditions describe conditions related to the matched \textit{text}. Usually, they require that two matched entries of the LHS are \emph{syntactically} equal. The runtime preconditions describe conditions which have to hold at \emph{runtime} for the original (LHS) and the rewritten code (RHS) to be semantically equivalent and they are expressed in Python syntax and semantics. For example, in Figure \ref{fig:sort-values-rule}, the result of the \code{called\_on} expression that was matched in the LHS should be a \code{pandas.DataFrame}. The runtime preconditions implicitly impose an order of evaluation. In this example, \code{called\_on} must be evaluated first, then the preconditions are checked on the resulting object, and then this object is used in place of \code{called\_on} in the RHS. Note that unconditionally evaluating \code{called\_on} is correct even if the conditions do not hold because it would be evaluated anyway in the original.

%Section~\ref{sec:Evaluation} presents in-depth case studies on real world Pandas codes where non-trivial rewrites enable significant performance improvements. For brevity, we only show the parameterized form of the rewrite rule when the fixed and parameterized parts of the rewrite is not apparent from the context.

% The preconditions should be evaluated when the LHS would be evaluated. If the preconditions pass, then evaluating the RHS should be functionally equivalent with evaluating the RHS. %Finally, in the RHS, we just list the code with which we replace the LHS. %We give a more formal description in Figure \ref{fig:sort-values-formal} in the form of operational semantics \stef{Give citation; not worth describing here}. %\stef{Not sure how to write this last sentence}. %We consider such formal descriptions to be out of scope for this paper, so for the rest of it, we will use our simplified notation of preconditions as in Figure \ref{fig:sort-values-rule}.


% The $\Gamma$'s are contexts, which for simplicity can be thought as snapshots of the Python namespace. Above horizontal line there is a set of premises, which, if true, they entail the statements below the line. The premise $(E, \Gamma_{1}) \Downarrow V, \Gamma_{2}$ states that if the expression $E$ is evaluated under context $\Gamma_{1}$, then its value is $V$ and it updates the context to $\Gamma_{2}$. In short, the rule says that if we evaluate \code{called\_on} first and then check the precondition in Figure \ref{fig:sort-values-rule}, then evaluating the LHS should be functionally equivalent with evaluating the RHS.

%The interpretation of the RHS section is the same as that of the LHS section.

%\charith{Too low level explain with high-level constructs}. \stef{Not sure what this means :/}


Table~\ref{tbl:rewr_rules} shows three more rewrite rules we use in \system{}. The first two correspond to the examples in Figure~\ref{fig:split} and Figure~\ref{fig:concat-with-lists}, respectively. Rules can have both runtime and syntactic conditions. For example, in the second rule, we have the syntactic precondition \code{@\{par1\} == @\{par2\}} requires that the two names be equal. To differentiate between the two kinds of preconditions, we prefix the syntactic preconditions with $\mathfrak S$ and the runtime ones with $\mathfrak R$.

% Easter egg: This style of typesetting was used in Turing's seminal paper.

%These rewrite rules span simple .... to complex ... \stef{Can't have diversity in how complex the rules are. Either we have such simple rules or way more complicated like vectorization. . As an alternative, I put rules that are tricky and explained all of them concisely.} \charith{Explain 2 rules concisely from the table}. 

% The first rule is the one in Figure~\ref{fig:split} that we discussed earlier. In Section~\ref{sub-sec:case-studies} we present an in-depth case study. %Finally, we have yet another rule that crosses the library boundaries as we move from Pandas to Python (by converting to a list) and then back to Pandas. The interesting thing here is in the preconditions: the name \code{pd} must be bound to the Pandas module. It is possible that there is another module that the user imported as \code{pd} which has a function \code{Series} that accepts lists.

%We note that Series are not just collections of elements. They also have an index with which the elements are accessed. Usually, this is a \code{RangeIndex} \cite{pandas_range_idnex}, i.e. a contiguous range of integers as in Python lists. But it need not be, and thus in rules where the rewritten version is implemented using pure Python and lists, when converting back to a \code{Series}, we need to preserve the original index.

 

%We will now present \system{}, a system that accelerates interactive data science workloads by transparently rewriting Python that interfaces with Pandas code and which addresses all the previously mentioned challenges. In order to do so, \system{} has two high-level components. First, \system{}' \textit{syntactic pattern-matcher} identifies matches the input code against the LHS parts of the rewrite rules. The second component is a \emph{rewriter}, which validates the preconditions of the rewrite rules and on passing them, rewrites the code to the RHS version and executes it. We show a high-level overview in Figure \ref{fig:system_overview}.

%\stef{Elaboration on the Figure?}

%We now describe each component in depth.


%\subsection{Applying Rewrites}

When a piece of code is successfully matched with a rewrite rule's
\textbf{LHS}, if there are no runtime preconditions (i.e.,
\textbf{RuntimePrecond} just returns \code{True}), then the rewriter can invoke
\textbf{TransformLHS} on the \textbf{LHS}, and replace the \textbf{LHS} with the
result.

For example, consider the rule in Figure~\ref{fig:sort-values-rule}.
\code{@\{called\_on\}} needs to be evaluated first, let us name the resulting
object \code{res}, then execution needs to \textit{stop}, check the precondition
on \code{res}, and then continue (i.e., evaluating either
\code{res.sort\_values().head()} or \code{res.nsmallest()}, depending on the
precondition result).

% \begin{figure}
\begin{minted}[bgcolor=light-gray,escapeinside=||]{python}
tmp = @{called_on}
if type(tmp) == pandas.Series:
  tmp.nsmallest()
else:
  tmp.sort_values().head()
\end{minted}
  \caption{A naive precondition check.}
  \label{fig:sort-values-precond-naive}
\end{figure}

This is more difficult than it looks because we do not have arbitrary
control over the evaluation of the code, since we are operating at the source
level. So, we need to \textit{effectively} do the same thing but using
source-level transformations. This is difficult because we are not allowed to
evaluate \code{@\{called\_on\}} twice. The obvious solution is to just evaluate
it once and reuse it.

\begin{figure}
  \begin{minted}[bgcolor=light-gray,escapeinside=||]{python}
test(mod_x(), foo().read_x().sort_values().head())
  \end{minted}
    \caption{A nested expression with global state access.}
    \label{fig:sort-values-nested}
  \end{figure}
% \begin{figure}
\begin{minted}[bgcolor=light-gray,escapeinside=||]{python}
tmp = foo().read_x()
if type(tmp) == pandas.Series:
  test(mod_x(), tmp.nsmallest())
else:
  test(mod_x(), tmp.sort_values().head())
\end{minted}
  \caption{An incorrect precondition check when having a nested expression.}
  \label{fig:sort-values-precond-naive2}
\end{figure}
% 

\begin{figure}
  \begin{minted}[bgcolor=light-gray,escapeinside=||]{ocaml}
test(mod_x(), 
  let tmp = foo().read_x() in
    if type(tmp) == pd.|Series| 
      then tmp.nsmallest()
      else tmp.sort_values().head())
  \end{minted}
    \caption{A local binding in the style of OCaml.}
    \label{fig:ocaml-local-binding}
\end{figure}
\begin{figure}
  \begin{minted}[bgcolor=light-gray,escapeinside=||]{python}
def sort_head(tmp):
  return tmp.nsmallest() if type(tmp) == pd.Series 
    else tmp.sort_values().head()

test(mod_x(), sort_head(foo.read_x()))
  \end{minted}
    \caption{A correct dynamic check with a local binding.}
    \label{fig:sort-values-precond-correct}
  \end{figure}

However, this requires careful orchestration. Consider for example the code in
Figure~\ref{fig:sort-values-nested}. The evaluation-and-reuse should happen
\textit{exactly} where the original sub-expression (involving
\code{sort\_values()}) appears. Otherwise (e.g., if we save it in a variable by
adding a statement above), it is possible that \code{read\_x()} will read a
stale value.

To solve that in general, we need a local binding of the evaluation of
\code{@\{called\_on\}} \footnote{In the style of a \code{let} expression in
OCaml.}. However, Python does not have local bindings, so the workaround is to
create a function and call it at the place of the original expression, as in
Figure~\ref{fig:sort-values-precond-correct}. The local binding here is the
binding to the function's parameter.

\paragraph{\textbf{Dynamic RHS}}

Observe that in the solution we just mentioned, to create the function
\code{sort\_head()}, we need to know the RHS a-priori. This is true when
\textbf{TransformLHS} does not depend on dynamic information and thus we can
``run'' it offline. This is the case for the rule in
Figure~\ref{fig:sort-values-rule}. However, this is not the case for the
\code{RemoveAxis1} rule, because it depends on the body of \code{@\{func\}},
which is not known offline. To implement such rules, all of which involve
\code{apply()} and which face the same problem, we just overwrite
\code{apply()}. In the overwritten body, we invoke \textbf{RuntimePrecond},
which depends on the body of \code{@\{func\}}, which is however available
because it is passed as an argument. If the preconditions pass, we rewrite the
body on the fly and invoke it appropriately. For example, in the case of
\code{RemoveAxis1}, we rewrite the body as described earlier and we call
\code{self[theOneSeries].apply(new\_body)}. Note that \code{self} is the
evaluated \code{@\{expr\}}. We never see this \code{@\{expr\}}, but we know this
fact because \code{self} is bound to the object on which (the overwritten) \code{apply()}
gets called.

%\begin{figure}
  \begin{subfigure}[b]{0.45\columnwidth}
    \begin{minipage}{\linewidth}%
\begin{minted}[bgcolor=light-gray]{python}
def foo(row):
  if row['Fare'] > 10:
    return True
  return False

df.apply(foo, axis=1)
\end{minted}
\end{minipage}
    \caption{Row-Mjaor}
    \label{fig:row-major}
  \end{subfigure}
  \hfill %%
  \begin{subfigure}[b]{0.45\columnwidth}
    \begin{minipage}{\linewidth}%
\begin{minted}[bgcolor=light-gray]{python}
def foo(fare):
  if fare > 10:
    return True
  return False

df['Fare'].apply(foo)
\end{minted}
\end{minipage}
    \caption{Columnar}
    \label{fig:columnar}
  \end{subfigure}
  \caption{Row-Major vs Columnar Access}
\end{figure}


% In retrospect, there is a much better solution (in terms of code complexity of \system{} and in terms of the correctness it provides and in terms of being more transparent to the user) which is an improvement over the problematic solution we wanted to avoid. Instead of inserting the code that performs the checks and the rewriting as part of the rewritten code, we put them in functions and in the rewritten code we just insert a call to them (after all, \system{} can act as a library). This would achieve the same effect as sliced execution but all the complexity of the latter goes away as it now happens organically. We aim to do this in a future version.
