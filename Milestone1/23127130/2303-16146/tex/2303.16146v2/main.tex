%%
%% This is file `sample-sigconf.tex',
%% generated with the docstrip utility.
%%
%% The original source files were:
%%
%% samples.dtx  (with options: `sigconf')
%% 
%% IMPORTANT NOTICE:
%% 
%% For the copyright see the source file.
%% 
%% Any modified versions of this file must be renamed
%% with new filenames distinct from sample-sigconf.tex.
%% 
%% For distribution of the original source see the terms
%% for copying and modification in the file samples.dtx.
%% 
%% This generated file may be distributed as long as the
%% original source files, as listed above, are part of the
%% same distribution. (The sources need not necessarily be
%% in the same archive or directory.)
%%
%% Commands for TeXCount
%TC:macro \cite [option:text,text]
%TC:macro \citep [option:text,text]
%TC:macro \citet [option:text,text]
%TC:envir table 0 1
%TC:envir table* 0 1
%TC:envir tabular [ignore] word
%TC:envir displaymath 0 word
%TC:envir math 0 word
%TC:envir comment 0 0
%%
%%
%% The first command in your LaTeX source must be the \documentclass command.
% \documentclass[sigconf]{acmart}
%% NOTE that a single column version is required for 
%% submission and peer review. This can be done by changing
%% the \doucmentclass[...]{acmart} in this template to 
% \documentclass[manuscript,screen,review,anonymous]{acmart}
\documentclass[acmsmall]{acmart}
%% 
%% To ensure 100% compatibility, please check the white list of
%% approved LaTeX packages to be used with the Master Article Template at
%% https://www.acm.org/publications/taps/whitelist-of-latex-packages 
%% before creating your document. The white list page provides 
%% information on how to submit additional LaTeX packages for 
%% review and adoption.
%% Fonts used in the template cannot be substituted; margin 
%% adjustments are not allowed.

%%
%% \BibTeX command to typeset BibTeX logo in the docs
\providecommand\BibTeX{{%
  Bib\TeX}}

%% Rights management information.  This information is sent to you
%% when you complete the rights form.  These commands have SAMPLE
%% values in them; it is your responsibility as an author to replace
%% the commands and values with those provided to you when you
%% complete the rights form.
\setcopyright{rightsretained}
\acmJournal{PACMMOD} \acmYear{2024} \acmVolume{2} \acmNumber{1 (SIGMOD)}
\acmArticle{58} \acmMonth{2} \acmPrice{}\acmDOI{10.1145/3639313}

%% These commands are for a PROCEEDINGS abstract or paper.
% \acmConference[Conference acronym 'XX]{Make sure to enter the correct
%   conference title from your rights confirmation emai}{June 03--05,
%   2018}{Woodstock, NY}
%
%  Uncomment \acmBooktitle if th title of the proceedings is different
%  from ``Proceedings of ...''!
%
%\acmBooktitle{Woodstock '18: ACM Symposium on Neural Gaze Detection,
%  June 03--05, 2018, Woodstock, NY} 
% \acmPrice{15.00}
% \acmISBN{978-1-4503-XXXX-X/18/06}


%%
%% Submission ID.
%% Use this when submitting an article to a sponsored event. You'll
%% receive a unique submission ID from the organizers
%% of the event, and this ID should be used as the parameter to this command.
%%\acmSubmissionID{123-A56-BU3}

%%
%% For managing citations, it is recommended to use bibliography
%% files in BibTeX format.
%%
%% You can then either use BibTeX with the ACM-Reference-Format style,
%% or BibLaTeX with the acmnumeric or acmauthoryear sytles, that include
%% support for advanced citation of software artefact from the
%% biblatex-software package, also separately available on CTAN.
%%
%% Look at the sample-*-biblatex.tex files for templates showcasing
%% the biblatex styles.
%%

%%
%% The majority of ACM publications use numbered citations and
%% references.  The command \citestyle{authoryear} switches to the
%% "author year" style.
%%
%% If you are preparing content for an event
%% sponsored by ACM SIGGRAPH, you must use the "author year" style of
%% citations and references.
%% Uncommenting
%% the next command will enable that style.
%%\citestyle{acmauthoryear}

\usepackage{algorithm}
\usepackage{algpseudocode}
\usepackage{color, soul}
\usepackage{xcolor}
\usepackage[frozencache,cachedir=.]{minted}
\usepackage{array}
\usepackage{listings}
\usepackage{tikz}
\usepackage{caption}
\usepackage{subcaption}
\usetikzlibrary{positioning}
\usetikzlibrary{calc,bending}
\usepackage{mathpartir}
\usepackage{float}
\usepackage{stfloats}
\usepackage{enumitem}
\usepackage{float}
\usepackage{marginalia}

\colorlet{light-gray}{gray!10}

\definecolor{highlight-green}{rgb}{0.6, 0.81, 0.66}
% \definecolor{border-blue}{rgb}{0.37, 0.62, 0.69}
\definecolor{border-blue}{rgb}{0.38, 0.25, 0.34}

\definecolor{codegray}{gray}{0.9}
% If you insert \raggedright, this will
% remove the paragraph indentation for some reason.
% \newcommand{\code}[1]{\colorbox{codegray}{\texttt{#1}}}
\newcommand{\code}[1]{\texttt{#1}}


%% System name
\def\system{\textsc{Dias}}

% Unfortunately, it doesn't highlight it. But it'll do...
\soulregister{\system}{0}


% Authors macros
% \definecolor{ddkang-blue}{rgb}{0.25, 0.87, 0.81}
\definecolor{ddkang-blue}{rgb}{0.69, 0.87, 0.94}
\DeclareRobustCommand{\ddkang}[1]{{\sethlcolor{ddkang-blue}\hl{#1}}}

% \definecolor{stef-color}{rgb}{0.984, 0.808, 0.694}
\definecolor{stef-color}{rgb}{1, 0.898, 0.706}
% \DeclareRobustCommand{\stef}[1]{{\sethlcolor{stef-color}\hl{#1}}}
\DeclareRobustCommand{\stef}[1]{}

\definecolor{charith-color}{rgb}{1, 0.5, 0.5}
\DeclareRobustCommand{\charith}[1]{{\color{magenta} #1}}

\definecolor{revis-color}{rgb}{0.1, 0.1, 1}
% \DeclareRobustCommand{\revis}[1]{{\color{revis-color} {#1}}}
\DeclareRobustCommand{\revis}[1]{{#1}}

\setminted{fontsize=\small} 

\definecolor{purple}{rgb}{1, 0, 1}

\newcommand{\ie}{\emph{i.e.,}\xspace}
\newcommand{\eg}{\emph{e.g.,}\xspace}
\newcommand{\abr}{\emph{abbr.}\xspace}
\newcommand{\ea}{\emph{et al.}\xspace}
\newcommand{\gensync}{\emph{GenSync}\xspace}
\newcommand{\colosseum}{\emph{Colosseum}\xspace}
\newcommand{\srep}{\emph{SREP}\xspace} % Set Reconciliation Enhances
\newcommand{\srepsim}{\emph{SREPSim}\xspace}
% Propagation
\newcommand{\esrep}{\emph{E-SREP}\xspace}
\newcommand{\epsrep}{\emph{EP-SREP}\xspace}
\newcommand{\mesrep}{\emph{ME-SREP}\xspace}
\newcommand{\mempoolsync}{\emph{MempoolSync}}

\newcommand{\fref}[1]{Fig.~\ref{#1}}
\newcommand{\tref}[1]{Table~\ref{#1}}
\newcommand{\aref}[1]{Algorithm~\ref{#1}}
\newcommand{\procref}[1]{Procedure~\ref{#1}}
\newcommand{\sref}[1]{Section~\ref{#1}}
\newcommand{\lineref}[1]{line~\ref{#1}}
\newcommand{\appref}[1]{Appendix~\ref{#1}}

% Change \eqref
\LetLtxMacro{\originaleqref}{\eqref}
\renewcommand{\eqref}{Eq.~\originaleqref}

% Theorems and corollaries
\newcounter{theoremcount}
\setcounter{theoremcount}{0}
\DeclareRobustCommand{\theorem}[1]{%
  \refstepcounter{theoremcount}%
  \noindent\textit{\textbf{Theorem \thetheoremcount\label{theorem:#1}: }}%
}
\DeclareRobustCommand{\theoremref}[1]{Theorem~\ref{theorem:#1}}

\DeclareRobustCommand{\proof}{\emph{Proof:}\xspace}
\DeclareRobustCommand{\qqed}{\hfill$\blacksquare$}

\newcounter{corollcount}
\setcounter{corollcount}{0}
\DeclareRobustCommand{\coroll}[1]{%
  \refstepcounter{corollcount}%
  \noindent\textit{\textbf{Corollary \thecorollcount\label{coroll:#1}: }}%
}
\DeclareRobustCommand{\corollref}[1]{Corollary~\ref{coroll:#1}}

\newcounter{lemmacount}
\setcounter{lemmacount}{0}
\DeclareRobustCommand{\lemma}[1]{%
  \refstepcounter{lemmacount}%
  \noindent\textit{\textbf{Lemma \thelemmacount\label{lemma:#1}: }}%
}
\DeclareRobustCommand{\lemmaref}[1]{Lemma~\ref{lemma:#1}}

\newcounter{definitioncount}
\setcounter{definitioncount}{0}
\DeclareRobustCommand{\definition}[1]{%
  \refstepcounter{definitioncount}%
  \noindent\textit{\textbf{Definition \thedefinitioncount\label{definition:#1}: }}%
}
\DeclareRobustCommand{\defref}[1]{Definition~\ref{definition:#1}}

%notes of different authors
\newif\ifnotes
\notestrue
\notesfalse

\newif\ifdiff
\difftrue
\difffalse

\newcommand{\anote}[1]{\ifnotes $\ll$\textsf{\textcolor{purple}{Ari: {#1}}}$\gg$ \fi}
\newcommand{\nnote}[1]{\ifnotes $\ll$\textsf{\textcolor{orange}{Novak: {#1}}}$\gg$ \fi}
\newcommand{\diff}[1]{\ifdiff\textcolor{orange}{#1}\else#1\fi}

%%% Local Variables:
%%% mode: latex
%%% TeX-master: "main"
%%% End:


\makeatletter
\gdef\@copyrightpermission{
  \begin{minipage}{0.2\columnwidth}
   \href{https://creativecommons.org/licenses/by/4.0/}{%
     \includegraphics[width=0.90\textwidth]{figures/4ACM-CC-by-88x31.eps}}
  \end{minipage}\hfill
  \begin{minipage}{0.8\columnwidth}
   \href{https://creativecommons.org/licenses/by/4.0/}{This work is licensed under a Creative Commons Attribution International 4.0 License.}
  \end{minipage}
  \vspace{5pt}
}
\makeatother

%%
%% end of the preamble, start of the body of the document source.
\begin{document}

%%
%% The "title" command has an optional parameter,
%% allowing the author to define a "short title" to be used in page headers.

\title[Dias: Dynamic Rewriting of Pandas Code]{%
\system{}: Dynamic Rewriting of Pandas Code}

%%
%% The "author" command and its associated commands are used to define
%% the authors and their affiliations.
%% Of note is the shared affiliation of the first two authors, and the
%% "authornote" and "authornotemark" commands
%% used to denote shared contribution to the research.
\author{Stefanos Baziotis}
\affiliation{%
  \institution{University of Illinois (UIUC)}
  \city{Champaign-Urbana}
  \country{U.S.A}
}
\email{sb54@illinois.edu}

\author{Daniel Kang}
\affiliation{%
  \institution{University of Illinois (UIUC)}
  \city{Champaign-Urbana}
  \country{U.S.A}
}
\email{ddkang@illinois.edu}

\author{Charith Mendis}
\affiliation{%
  \institution{University of Illinois (UIUC)}
  \city{Champaign-Urbana}
  \country{U.S.A}
}
\email{charithm@illinois.edu}
%%
%% By default, the full list of authors will be used in the page
%% headers. Often, this list is too long, and will overlap
%% other information printed in the page headers. This command allows
%% the author to define a more concise list
%% of authors' names for this purpose.

%%
%% The abstract is a short summary of the work to be presented in the
%% article.


Over the past few years, there has been a significant amount of research focused on studying the ReLU activation function, with the aim of achieving neural network convergence through over-parametrization. However, recent developments in the field of Large Language Models (LLMs) have sparked interest in the use of exponential activation functions, specifically in the attention mechanism.

Mathematically, we define the neural function $F: \R^{d \times m} \times  \mathbb{R}^d \rightarrow \mathbb{R}$ using an exponential activation function. Given a set of data points with labels $\{(x_1, y_1), (x_2, y_2), \dots, (x_n, y_n)\} \subset \mathbb{R}^d \times \mathbb{R}$ where $n$ denotes the number of the data. Here $F(W(t),x)$ can be expressed as $F(W(t),x) := \sum_{r=1}^m a_r \exp(\langle w_r, x \rangle)$, where $m$ represents the number of neurons, and $w_r(t)$ are weights at time $t$. It's standard in literature that $a_r$ are the fixed weights and it's never changed during the training. We initialize the weights $W(0) \in \mathbb{R}^{d \times m}$ with random Gaussian distributions, such that $w_r(0) \sim \mathcal{N}(0, I_d)$ and initialize $a_r$ from random sign distribution for each $r \in [m]$.

Using the gradient descent algorithm, we can find a weight $W(T)$ such that $\| F(W(T), X) - y \|_2 \leq \epsilon$ holds with probability $1-\delta$, where $\epsilon \in (0,0.1)$ and $m = \Omega(n^{2+o(1)}\log(n/\delta))$. To optimize the over-parametrization bound $m$, we employ several tight analysis techniques from previous studies [Song and Yang arXiv 2019, Munteanu, Omlor, Song and Woodruff ICML 2022]. 

 

%%
%% The code below is generated by the tool at http://dl.acm.org/ccs.cfm.
%% Please copy and paste the code instead of the example below.
%%
\begin{CCSXML}
<ccs2012>
<concept>
<concept_id>10002951.10002952</concept_id>
<concept_desc>Information systems~Data management systems</concept_desc>
<concept_significance>500</concept_significance>
</concept>
</ccs2012>
\end{CCSXML}

\ccsdesc[500]{Information systems~Data management systems}

%%
%% Keywords. The author(s) should pick words that accurately describe
%% the work being presented. Separate the keywords with commas.
\keywords{pandas, rewriting, dynamic, cross-representation}

\received{July 2023}
\received[revised]{October 2023}
\received[accepted]{November 2023}

%%
%% This command processes the author and affiliation and title
%% information and builds the first part of the formatted document.
\maketitle

\section{Introduction}
\label{sec:introduction}
% \begin{itemize}
%     % Diffusion of FL
%     \item {\st{Diffusion of FL}}
%     % Security threats to FL
%     \item {\st{Security threats to FL with particular focus on model poisoning}}
%     % Limitations of existing countermeasures
%     \item {\st{Current countermeasures (e.g., KRUM) and their limitations}}
%     % Proposed method and its advantages
%     \item {\st{Intuitive description of the proposed method and its difference (i.e., advantages) w.r.t. state of the art}}
%     % Main contributions
%     \item {\st{Summary of the main contributions of this work}}
%     % Paper's structure and organization
%     \item {\st{Paper's structure and organization}}
% \end{itemize}

% Diffusion of FL
Recently, {\em federated learning} (FL) has emerged as the leading paradigm for training distributed, large-scale, and privacy-preserving machine learning (ML) systems~\cite{mcmahan2017googleai,mcmahan2017aistats}. 
The core idea of FL is to allow multiple edge clients to collaboratively train a shared, global model without disclosing their local private training data.
%Specifically, an FL system consists of a central server and many edge clients; 
A typical FL round involves the following steps: {\em(i)} the server randomly picks some clients and sends them the current, global model; {\em(ii)} each selected client locally trains its model with its own private data; then, it sends the resulting local model to the server;\footnote{Whenever we refer to global/local model, we mean global/local model {\em parameters}.} {\em(iii)} the server updates the global model by computing an \emph{aggregation function}, usually the average (FedAvg), on the local models received from clients.
% \begin{enumerate}
%     \item[{\em(i)}] the server sends the current, global model to the clients and appoints some of them for training;
%     \item[{\em(ii)}] each selected client locally trains its copy of the global model with its own private data; then, it sends the resulting local model back to the server;\footnote{Whenever we refer to global/local model, we mean global/local model {\em parameters}.}
%     \item[{\em(iii)}] the server updates the global model by computing an \emph{aggregation function} on the local models received from clients (by default, the average, also referred to as FedAvg~\cite{mcmahan2017aistats}).
% \end{enumerate}
This process goes on until the global model converges. %(e.g., after a certain number of rounds or other similar stopping criteria).
%\\
% The advantages of FL over the traditional, centralized learning paradigm are undoubtedly clear in terms of flexibility/scalability (clients can join/disconnect from the FL network dynamically), network communications (only model weights\footnote{We will use \textit{parameters} and \textit{weights} interchangeably.} are exchanged between clients and server), and privacy (each client's private training data is kept local at the client's end and not uploaded to the server).
\\
% Security threats to FL
%However, the growing adoption of FL also raises security concerns~\cite{costa2022covert}, particularly about its confidentiality, integrity, and availability.
Although its advantages over standard ML, FL also raises security concerns~\cite{costa2022covert}. %, particularly about its confidentiality, integrity, and availability~\cite{costa2022covert}.
% OLD, LONG VERSION
% Indeed, some work deals with privacy leakage that may expose the local data of some clients~\cite{melis2019sp}. 
% A large body of work, instead, investigates attacks that usually aim to detriment the predictive accuracy of the learned global model. For instance, \emph{data poisoning} attacks achieve this goal by letting an adversary pollute the training set of some corrupt FL clients with maliciously crafted examples~\cite{jagielski2018sp}.
% Similarly, in \emph{model poisoning} the attacker attempts to tweak the global model weights~\cite{bhagoji2019pmlr} by directly perturbing the local model's weights of some infected FL clients before these are sent to the central server for aggregation, usually via so-called Byzantine attacks. 
% It turns out that Byzantine model poisoning attacks severely impact standard FedAvg; therefore, more robust aggregation functions must be designed to make FL systems secure.
Here, we focus on \emph{untargeted model poisoning} attacks~\cite{bhagoji2019pmlr}, where an adversary attempts to tweak the global model weights %\footnote{We will use the terms \textit{parameters} and \textit{weights} interchangeably.} 
by directly perturbing the local model's parameters of some infected clients before these are sent to the central server for aggregation.
In doing so, the adversary aims to jeopardize the global model \textit{indiscriminately} at inference time.
Such model poisoning attacks severely impact standard FedAvg; therefore, more robust aggregation functions must be designed to secure FL systems.
\\
% In this paper, we focus on designing a novel robust aggregation scheme at the server's end to contrast the effect of Byzantine model poisoning attacks.
%
% Current countermeasures and their limitations
%Several countermeasures have been proposed in the literature to combat model poisoning attacks on FL systems.
% Some methods use simple statistics more robust than plain average to smooth the impact of malicious updates (e.g., Trimmed Mean and FedMedian~\cite{yin2018icml}). 
% Other defenses implement outlier detection techniques to discard malicious updates from the aggregation performed at the server's end. Those are either based on heuristics (e.g., Krum/Multi-Krum~\cite{blanchard2017nips} and Bulyan~\cite{mhamdi2018pmlr}) or data-driven approaches (e.g., K-means clustering~\cite{shen2016acm} or DnC via spectral analysis~\cite{shejwalkar2021ndss}). 
% Finally, some strategies rely on a centralized ``source of trust'' to spot potential malicious updates (e.g., FLTrust~\cite{cao2020fltrust}).
% Several countermeasures have been proposed in the literature to combat model poisoning attacks on FL systems, i.e., to discard possible malicious local updates from the aggregation performed at the server's end. 
% These techniques range from simple statistics more robust than plain average (e.g., Trimmed Mean and FedMedian~\cite{yin2018icml}) to outlier detection heuristics (e.g., Krum/Multi-Krum~\cite{blanchard2017nips} and Bulyan~\cite{mhamdi2018pmlr}) or data-driven approaches (e.g., spectral analysis via K-means clustering~\cite{shen2016acm} or spectral analysis), or methods based on ``source of trust'' (e.g., FLTrust~\cite{cao2020fltrust}).
% OLD, LONG VERSION
%Several countermeasures have been proposed in the literature to combat Byzantine model poisoning attacks on FL systems.
% Descriptive statistics
% For example, Trimmed Mean and FedMedian aggregate local model updates using more robust statistics than standard average~\cite{yin2018icml}.
%
% % Heuristics for outlier detection
% Many existing Byzantine-resilient strategies implement some outlier detection heuristics to discard the model updates sent by potentially malicious clients from the input of the aggregation function.
% One of the most popular heuristics is Krum~\cite{blanchard2017nips}.
% This strategy tries to mitigate the impact of Byzantine attacks by selecting as a global model the local model with the smallest sum of Euclidean distances to {\em all} the other local models.
% Although powerful, Krum requires the server to know (or, at least, estimate) the number of malicious FL clients upfront, which is generally impossible in a realistic attack scenario. %
% Moreover, Krum may become ineffective for complex, high-dimensional model parameter spaces due to the curse of dimensionality.
% Bulyan~\cite{mhamdi2018pmlr} tries to overcome this issue by combining Krum with a variant of Trimmed Mean.
% % Data-driven outlier detection
% Other strategies use data-driven outlier detection techniques -- e.g., via K-means clustering~\cite{shen2016acm} -- to spot potential malicious local model updates. 
% %For instance, Shen et al. propose to cluster local model updates with K-means and thus identify outliers.
%
% % Other techniques
% As far as the server is concerned, any local model received can be from a potential malicious client. 
% FLTrust~\cite{cao2020fltrust} assumes the server acts as a client, i.e., trains a local model on an additional {\em trustworthy} dataset at the server's end and compares it against all the local models from other clients. 
% This way, the server can rely on some ``source of trust'' when discarding potentially malicious clients.
%\\
% Limitations of existing Byzantine-resilient strategies
Unfortunately, existing defense mechanisms either rely on simple heuristics (e.g., Trimmed Mean and FedMedian by~\cite{yin2018icml}) or need strong and unrealistic assumptions to work effectively (e.g., foreknowledge or estimation of the number of malicious clients in the FL system, as for Krum/Multi-Krum~\cite{blanchard2017nips} and Bulyan~\cite{mhamdi2018pmlr}, which, however, cannot exceed a fixed threshold).
Furthermore, outlier detection methods using K-means clustering~\cite{shen2016acm} or spectral analysis like DnC~\cite{shejwalkar2021ndss} do not directly consider the temporal evolution of local model updates received.
Finally, strategies like FLTrust~\cite{cao2020fltrust} require the server to collect its own dataset and act as a proper client, thereby altering the standard FL protocol.
\\
% OLD, LONG VERSION
% Overall, existing Byzantine-resilient strategies are either simple heuristics (e.g., FedMedian) or, if they are more complex, they rely on strong and unrealistic assumptions to work effectively (e.g., knowing the number of malicious clients in the FL system in advance, as for Krum and alike).
% Furthermore, data-driven outlier detection methods do not consider the temporary evolution of local model updates received (e.g., K-means clustering). 
% Finally, strategies like FLTrust requires the server to collect its own dataset and act as a proper client, thereby altering the standard FL protocol.
%
% Description of the proposed method
This work introduces a novel pre-aggregation \textit{filter} robust to untargeted model poisoning attacks. Notably, this filter $(i)$ operates without requiring prior knowledge or constraints on the number of malicious clients and $(ii)$ inherently integrates temporal dependencies. 
The FL server can employ this filter as a preprocessing step before applying \textit{any} aggregation function, be it standard like FedAvg or robust like Krum or Bulyan.
Specifically, we formulate the problem of identifying corrupted updates as a multidimensional (i.e., matrix-valued) time series anomaly detection task. 
The key idea is that legitimate local updates, resulting from well-calibrated iterative procedures like stochastic gradient descent (SGD) with an appropriate learning rate, show \textit{higher predictability} compared to malicious updates. This hypothesis stems from the fact that the sequence of gradients (thus, model parameters) observed during legitimate training exhibit regular patterns, as validated in Section~\ref{subsec:intuition}. %until convergence. 
%This regularity may be more pronounced for smooth convex loss functions, but it can still be captured within an appropriate time window, even for more complex and convoluted loss surfaces. 
%We provide evidence of this claim in Appendix~B, where we show that the average mutual information (i.e., ``predictability''), calculated over pairs of legitimate model updates sent at different FL rounds, is significantly higher than the corresponding computation for a malicious client.
\\
Inspired by the matrix autoregressive (MAR) framework for multidimensional time series forecasting~\cite{chen2021je}, we propose the FLANDERS ({\em \textbf{F}ederated \textbf{L}earning meets \textbf{AN}omaly \textbf{DE}tection for a \textbf{R}obust and \textbf{S}ecure}) filter.
The main advantages of FLANDERS over existing strategies like FLDetector~\cite{zhao2020multivariate} are its resilience to large-scale attacks, where $50\%$ or more FL participants are hostile, and the capability of working under realistic non-iid scenarios.
We attribute such a capability to two key factors: $(i)$ FLANDERS works without knowing a priori the ratio of corrupted clients, and $(ii)$ it embodies temporal dependencies between intra- and inter-client updates, quickly recognizing local model drifts caused by evil players. Below, we summarize our main contributions:

\begin{itemize}
\item[{\em(i)}]
We provide empirical evidence that the sequence of models sent by legitimate clients is more predictable than those of malicious participants performing untargeted model poisoning attacks.
\\
\item[{\em(ii)}] 
We introduce FLANDERS, the first pre-aggregation filter for FL robust to untargeted model poisoning based on multidimensional time series anomaly detection.
\\
\item[{\em(iii)}] 
We integrate FLANDERS into Flower,\footnote{\scriptsize{\url{https://flower.dev/}}} a popular FL simulation framework for reproducibility.
\\
\item[{\em(iv)}] 
We show that FLANDERS improves the robustness of the existing aggregation methods under multiple settings: different datasets, client's data distribution (non-iid), models, and attack scenarios.
\\
\item[{\em(v)}] 
We publicly release all the implementation code of FLANDERS along with our experiments.\footnote{\scriptsize{\url{https://anonymous.4open.science/r/flanders_exp-7EEB}}}
\end{itemize}

% Paper's structure and organization
The remainder of the paper is structured as follows. %some related work and the current state-of-the-art solutions to security issues that FL entails. 
Section~\ref{sec:background} covers background and preliminaries. 
In Section~\ref{sec:related}, we discuss related work.
Section~\ref{sec:problem} and Section~\ref{sec:method} describe the problem formulation and the method proposed. % to tackle it. 
Section~\ref{sec:experiments} gathers experimental results. %, and Section~\ref{sec:limitations} discusses some limitations of this work.
Finally, we conclude in Section~\ref{sec:conclusion}.
 %discusses the limitations of this work and draws future research directions.
%reports conclusions and draws perspectives for future research directions.

%%%%%%% OLD %%%%%%%
%to overcome the resilience of Byzantine failures in distributed Stochastic Gradient Descent computations. 
% The strength of Krum is its time complexity, which is linear in the gradient dimension. 
% However, the robustness of the approach is guaranteed for gradient-based learning applications only when the majority of the clients are not compromised. 
% Besides, the aggregation mechanism of Krum, as well as that of similar methods, is robust from a coarse-grained perspective and does not provide solutions to errors and perturbations that may occur at inference time.
%A related approach to~\cite{blanchard2017nips} is the work of Su et al.~\cite{su2016dc}. Here, the authors propose an iterated approximate agreement to tackle a multi-layer scenario attacked by Byzantine agents. 
%However, the method works efficiently on the sole discrete context and it is inapplicable to continuous state environments.
%\gabri{Maybe, we should just talk about the main limitations of existing countermeasures without digging into their details (or, we can just mention Krum as this is the most popular one). I will move the description of all these methods to the Related Work section.}
\section{Background on Network Calculus}
\label{sec: background}


\begin{figure*}[tbh]
\centering
\begin{subfigure}[b]{0.3\textwidth}
    \centering
    \includegraphics[width=\linewidth]{images/in-out.png}
    \caption{Arrival and departure data and their relation with delay $d(t)$ and backlog $b(t)$. For a FIFO system, the delay is the horizontal distance between $R(t)$ and $R^*(t)$ but some other multiplexing techniques may shift the data to a later priority, causing a longer delay.}
    \label{fig: data in-out}
\end{subfigure}
\hfill
\begin{subfigure}[b]{0.35\textwidth}
    \centering
    \includegraphics[width=\linewidth]{images/arrival-service.png}
    \caption{Characteristics of an arrival curve and a service curve. From any point of observation, the arriving data never exceeds its arrival curve; the departure data is also never less than the service curve with respect to the data arrival.}
    \label{fig: arrival-service curves}
\end{subfigure}
\hfill
\begin{subfigure}[b]{0.33\textwidth}
    \centering
    \includegraphics[width=\linewidth]{images/bound.png}
    \caption{Delay and backlog bounds of a system. Backlog is the maximum vertical distance between $\alpha(t)$ and $\beta(t)$; FIFO delay is their maximum horizontal distance; but for arbitrary multiplexing, the delay guarantee is when the system clears its buffer, thus it's the intersection of $\alpha(t)$ and $\beta(t)$.}
    \label{fig: system bounds}
\end{subfigure}
\caption{Network calculus framework. We let $R(t)$ and $R^*(t)$ be the arrival and departure data flow of a system; $\alpha(t)$ be the piecewise linear concave arrival curve and $\beta(t)$ be the piecewise linear convex service curve of a system.}
% \hossein{Better to show piece-wise linear concave arrival curve and piece-wise linear convex service curve instead of token-bucket and rate-latency.}}
\end{figure*}

We recall some of the network calculus essentials for a better understanding of the framework used in Saihu. In the following context, we use the following notation: $\mbb{R}^+$ is the set of non-negative real numbers; $[x]_+$ denotes $\max(0, x)$

The data flow is by convention modeled as a left-continuous wide-sense increasing function $R(t): \mbb{R}^+ \mapsto \mbb{R}^+$ with respect to time $t$~\cite{ncbook2001leboudec}. 

A system $\mcal{S}$ receives arrival data described as a cumulative function $R(t)$ and delivers departure data as another cumulative function $R^*(t)$. Figure~\ref{fig: data in-out} illustrates such a system $\mcal{S}$. The benefit of representing a system like this is that we can observe system backlog and delay with such a model. 

\begin{definition}[Backlog and Delay~\cite{ncbook2001leboudec}]
    The backlog of a system at time~$t$ is
    \begin{equation}
        b(t) = R(t) - R^*(t)
    \end{equation}
    
    The virtual delay of a FIFO system at time $t$ is
    \begin{equation}
        d_{FIFO}(t) = \inf \lbp \tau \geq 0 : R(t) \leq R^*(t+\tau) \rbp
    \end{equation}
\end{definition}



The backlog of a system can be viewed as the vertical distance between $R$ and $R^*$. The FIFO (\textit{First-in First-out}) delay is the horizontal distance between $R$ and $R^*$. One may obtain other delay values if the multiplexing technique is not FIFO.

% \begin{figure}
%     \centering
%     \includegraphics[width=0.9\linewidth]{images/in-out.png}
%     \caption{In/out data flow; delay and backlog}
%     \label{fig: data in-out}
% \end{figure}

Since we are interested in the system guarantee instead of a single instance of data flow, we would like to have general bounds to the arrival and departure data flows. Therefore, we define \textit{arrival curve} and \textit{service curve} as the bounds of arrival and departure data flows.

\begin{definition}[Arrival Curve~\cite{ncbook2001leboudec}]
    Given a wide-sense increasing function $\alpha: \mbb{R}^+ \mapsto \mbb{R}^+$, we say that a flow $R(t)$ is $\alpha$-constrained if and only if for all $s \leq t$:
    \begin{equation}
        R(t) - R(s) \leq \alpha(t-s)
    \end{equation}
    We say $R(t)$ has $\alpha$ as an arrival curve.
\end{definition}

\begin{definition}[Service Curve~\cite{ncbook2001leboudec}]
    Given a wide-sense increasing function $\beta: \mbb{R}^+ \mapsto \mbb{R}^+$ and $\beta(0) = 0$. A system $\mcal{S}$ having $R(t)$ and $R^*(t)$ as its arrival and departure flows. We say $\mcal{S}$ offers a service curve $\beta$ if and only if
    \begin{equation}
        R^*(t) \geq (R \otimes \beta)(t) =: \inf_{s \leq t} \lbp R(s) + \beta(t-s) \rbp
    \end{equation}
    where $\otimes$ denotes the min-plus convolution
\end{definition}

Figure~\ref{fig: arrival-service curves} illustrates the arrival and service curves. Any segment of arrival flow $R(t)$ is constrained by arrival curve $\alpha$ and the output curve $R^*(t)$ is always no less than the curve $R\otimes\beta$. As a result, an arrival curve upper bounds the incoming traffic, and a service curve lower bounds the outgoing traffic.

% \begin{figure}
%     \centering
%     \includegraphics[width=\linewidth]{images/arrival-service.png}
%     \caption{Arrival/Service curve}
%     \label{fig: arrival-service curves}
% \end{figure}

We consider 2 special types of curves throughout this paper, \textit{token-bucket} (or sometimes called \textit{leaky-bucket}) curve and \textit{rate-Latency} curve.

\begin{definition}[Token-bucket and Rate-latency~\cite{ncbook2001leboudec}]
    A token-bucket curve $\gamma_{r,b}$ with arrival rate $r$ and burst $b$ is defined as
    \begin{equation}
        \gamma_{r,b}(t) = b + rt
    \end{equation}

    A rate-latency curve $\beta_{R,T}$ with service rate $R$ and latency $T$ is defined as
    \begin{equation}
        \beta_{R,T}(t) = R \lb t - T \rb_+
    \end{equation}
\end{definition}

A token-bucket curve is determined by a burst $b$ and an arrival rate~$r$. Burst represents the maximum possible data volume that can arrive simultaneously, and arrival rate represents the maximum long-term data rate~\cite{bouillard2022tradeoff}.
A rate-latency curve is determined by a latency~$T$ and a service rate~$R$. Latency represents the time a server needs before starting to process the incoming data, and service rate represents the minimum rate to process data after the initial latency.

With the help of arrival and service curves, we can derive delay and backlog bounds for a system $\mcal{S}$ illustrated in Figure~\ref{fig: system bounds}. Suppose a system $\mcal{S}$ has arrival curve $\alpha$ and service curve~$\beta$, its worst-case backlog $b^*$ is the maximum vertical distance between~$\alpha$ and~$\beta$. Similarly, depending on the multiplexing technique applied to the system, its worst-case delay bound $d^*$ is the maximum horizontal distance between $\alpha$ and $\beta$ if $\mcal{S}$ is a FIFO system. If we don't have any information about its multiplexing technique, referred to as arbitrary multiplexing, the best we can say is that when $\alpha$ and $\beta$ intersect each other, where all data has been delivered out of the system. Consequently, the worst-case delay bound for arbitrary multiplexing is the time required for $\mcal{S}$ to clear its buffer.

% \begin{figure}
%     \centering
%     \includegraphics[width=\linewidth]{images/bound.png}
%     \caption{System delay/backlog bounds}
%     \label{fig: system bounds}
% \end{figure}

While a service curve captures the slowest possible output speed of a system, a link's transmission capacity limits the speed as well. Hence, we model this phenomenon using a \textit{greedy shaper} with a sub-additive function $\sigma: \mbb{R}^+ \mapsto \mbb{R}^+$ concatenated with a server. We consider a concatenation as shown in Figure \ref{fig: system}. By convention we assume $\sigma(0) = 0$ and $\beta(t) \leq \sigma(t), \forall t \in \mbb{R}^+$, meaning that the buffer is cleared at the beginning and the service never exceed its physical limitation. With the above definition, such greedy shaper conserves the service provided by the system due to theorem \ref{thm: shaping}.

\begin{figure}[thb]
    \centering
    \includegraphics[width=0.7\linewidth]{images/system.png}
    \caption{Shaping of departure data. A flow that has an arrival curve $\alpha$ feeds into a server with an arrival data flow $R(t)$. The server having service curve $\beta$ takes $R(t)$ and gives a departure data flow $R^*(t)$ to a shaper with shaping function $\sigma$. The shaper takes $R^*(t)$ and shape the data flow as another departure $D(t)$.}
    \label{fig: system}
\end{figure}


\begin{theorem}[Shaping conserves service \cite{ncbook2001leboudec}]
\label{thm: shaping}
Following the system shown in Figure \ref{fig: system}, we have
\begin{equation}
     D = R^* \otimes \sigma \geq \lp R \otimes \beta \rp \otimes \sigma = R \otimes \lp \beta \otimes \sigma \rp = R \otimes \beta
\end{equation}
\end{theorem}

In the following context, we model the shaping function $\sigma$ as a token-bucket curve $\gamma_{C,L}$ with transmission capacity $C$ and the packet size $L$ to capture the link capacity and packetization~\cite{bouillard2022tradeoff}.

\section{\system Framework Design}
\label{sec:system}
We now explain how \system helps author widgets that support transparent, reusable, and customizable user actions. 

% \danc{here do we want to emphasize a) we revise conventional widget design to a statefule design or b) megneton can convert your existing widgets to improve it with our proposed characteristics? It sounds more like b) to me but not sure if that's the intention. } \saj{it's actually a}
%In this section, we discuss the key components that provide the foundation for \system widgets, which instruments the design goals outlined in Section~\ref{sec:design_goal}. 
%We then explain the system architecture of \system.

\subsection{Widget Frameworks: Design and Limitations}
Widgets are interactive elements, \eg sliders, text boxes, buttons, that have representations both in the kernel, \ie where code is executed, and the front-end, \ie the notebook web interface. However, recent frameworks for authoring widgets~\cite{idomjp} also enable integration of interactive dashboards in the front-end~\cite{wu2020b2,bauerle2022symphony, zhang2023meganno}.  

 \begin{figure}[!htb] 
 \centering
  \includegraphics[width=0.8\linewidth]{figures/stateful-widget-redesign-basic.png}
  \caption{Design of basic, \ie traditional widgets.}
  \label{fig:base_widget} 
  \Description{The basic widget design.}
\end{figure}

 
 As shown in Figure~\ref{fig:base_widget}, Widgets (\eg \emph{ipywidgets}~\cite{IPyWidgets}) maintain their state both at the back-end kernel (called \emph{Widget Base}) and the front-end (called \emph{Widget Model}.) The Widget Base and Widget Model remain in-sync via the communication API called \emph{Comm}. 
 %\danc{The previous statement is bit hard to follow. Break down into defining what are widget base and model, and then explain how they work together?} 
However, only the most recent state is maintained, making the widgets essentially \emph{memoryless}. The \emph{Widget Manager} coordinates the display of the widget in the front-end \emph{Widget View}. The Widget View is a container for rendering interactive components using front-end libraries and web frameworks. The Widget View only registers low-level event listeners corresponding to user interactions on the components. %For example, a \emph{drag} interaction that updates position of slider is registered as an \emph{onChange} listener. 
 For example, a \emph{selection} interaction on the graph node in Figure~\ref{fig:teaser}B that updates the bar charts is registered as an \emph{onClick} listener.
 Therefore, these widgets are \emph{agnostic} of the user's high-level interaction types and additional context, such as where the interaction happened and which components were updated. The \emph{memoryless} and \emph{interaction agnostic} nature of widgets prevent tracking of the user's interaction history and the corresponding widget states.
Moreover, such a design primarily serves to parameterize data operations in the kernel using front-end events --- a widget state variable (\eg current node identifier) impacted by a low-level event (\eg \emph{onClick}) serves as an input parameter to a data operation (\eg distribution computation). Any change in the widget variable triggers a recomputation of the data operation. In the notebook, the users can programmatically access and update the parameters of the data operations. However, the data operations in the kernel, designed by widget developers, are neither accessible nor customizable from the front-end. The lack of affordances to override data operations limit 
the end-user's capability to customize the widgets designed by the developers. We describe enhancement of existing widgets with such features next.

\subsection{Towards Persistent, Interaction-Aware, and Customizable Widgets}

%We augment existing widgets to introduce new features such as interaction history, reusable sates, and on-demand customization of data operations. 
We create a persistent and interaction-aware widget called \emph{stateful widget} by extending the Widget Base with state and interaction history management capabilities (see Figure~\ref{fig:stateful_widget}.) Within a stateful widget, the state manager maintains each state updates corresponding to user interactions within a list called \emph{Data States}. The state manager registers the following in the \emph{action history}: (a) context of each event (\eg the front-end interaction type and the component and element where interaction occurred) and (b) the corresponding state identifier in Data States. Since the default Widget View only registers low-level events, we create a Widget View Wrapper that records each event's context as an action via an Action Wrapper. The action wrapper dispatches an action consisting of the event context mentioned earlier via the Comm API. Users can view the interaction history in a separate notebook cell which shows the details of an interaction and the corresponding data state as shown in Figure~\ref{fig:history}, thereby ensuring transparency. The history view is synchronized with the corresponding widget. Therefore, users can leverage the history to load previous states in the Widget View using the \emph{Restore} button. Moreover, users can also access the widget state as a \code{JSON} object using a declarative command as shown in Figure~\ref{fig:teaser}E, thereby ensuring reusability. Such a design also enables users to employ visualizations as a medium for capturing 
and exporting ``actions interactively
performed in the component''~\cite{batch2017interactive} --- 
the outcomes of these interactions are often utilized in 
subsequent steps of a data science workflow~\cite{rahman2022ie}.

 \begin{figure}[!htb] 
  \centering
  \includegraphics[width=\linewidth]{figures/stateful-widget-redesign-mag.png}
  \caption{Design of \system widgets. The dashed (``- -'') elements, \ie the stateful widget and widget view wrappers, are introduced by \system.}
  \label{fig:stateful_widget} 
  \Description{The stateful widget design.}
\end{figure}

% Therefore, interaction-aware state management \todo{via stateful widget} ensures transparency and reusability of user actions. \todo{elaborate}

\begin{figure*}[!htb] 
  \centering
  \includegraphics[width=\linewidth]{figures/history.png}
  \caption{The history view of a widget (\code{widget.history.show()}). Clicking the \emph{Restore} button loads previous state visualizations. }
  \label{fig:history} 
  \Description{The history view of a widget. Clicking the Restore button loads the previous states and their visualizations.}
\end{figure*}

%\hkc{why is it called 'shared'? also which opreations are customizable and which are not?} all data operations are customizable if they are defined as shared
Since data operations in the kernel correspond to user interactions in the front-end component, we introduce the concept of \emph{shared actions}.
Shared actions are data operations that end-users can override from the notebook. The operation definitions are essentially shared between the kernel and front-end. In the \system framework, developers can define a data operation to be shared. 
 For example, a shared data operation may return a distribution sorted by descending order of frequency. However, the user may prefer viewing the distribution in the alphabetic order of labels. As shown in Figure~\ref{fig:teaser}C and~\ref{fig:teaser}D, a user-defined function (UDF) written in the notebook --- which reflects the updated sort order --- is mapped to these the actions during widget instantiation time. 
In the kernel, the state manager parses the UDFs using custom serializers and overrides the data operation corresponding to the shared action. 
Such a design expands the ``events parameterizing code'' paradigm of widgets to ``operations parameterizing code'' and offers more flexible customization capabilities --- users can keep updating the shared actions to explore different objectives by modifying the function defined in the notebook.
Note that developers may implement data operations such as schema generation and distributions computation using standalone libraries or from scratch. In the case studies described in Section~\ref{sec:study}, we used an in-house graph query library, which was published as a Python package. 
% , among others. The operations are part of an in-house graph query library, which is published as a Python package for internal usage.
%We provide examples of these features in the supplementary material. 


%\todo{ADD CODE BLOCK}
\stitle{Components in \system Widget View.} 
%\danc{this paragraph renamed to technical/implementation details? Or is component a special term in megneton? } 
We use the React web framework~\cite{react} to develop the front-end components and the IDOM-Jupyter package~\cite{idomjp} for component rendering in the Widget View. 
The components are TypeScript~\cite{typescript} modules that enable the rendering of a wide range web-based visualization libraries. For example, we used a custom graph visualization library to render the schema graph~\cite{franz2016cytoscape}, Vega-lite~\cite{satyanarayan2016vega} to render the bar charts, and a JavaScript library to render tables. 
%We provide a detailed list of all the components in the supplementary material.
As TypeScript supports static typing, developers can define application-specific data types and use those across the modules. 
Therefore, using TypeScript ensures a tighter integration between the Widget Base in the kernel and the Widget Model in the front-end. 
Moreover, when customizing data operations defined as shared actions, the pre-defined types provide hints to the user about the expected return type of the customized function. Each of the components rendered in the Widget View is fully interactive. These interactions, derived from existing visualization research~\cite{yi2007toward, amar2005low} are reactively synchronized across \system components, enabling multiple-coordinated visualizations (\eg Figure~\ref{fig:teaser}B.) 
% 
\subsection{Pandas Code Rewriting}

\stef{Section overview?}



\charith{Let's use the following paragraph structure:
\begin{itemize}
    \item Introducing rewrites
    \item Non-obvious rewrites
    \item Why an automatic rewrite system is neededs of such a system over exisiting solutions (it does not have the 3 drawbacks, plus it gets good speedups)
    \item What are the non-obvious benefits we get from a rewrite system (going beyond abstraction boundary)
\end{itemize}}

%% purpose of the para: introducing rewrites and why other systems may not address it.
Rewriting, for optimization purposes, is the process of replacing some part of code with a functionally equivalent but faster version. As an example, consider the task of selecting the 5 smallest elements of a \code{DataFrame} column. Figure \ref{fig:sort_values} shows user-written code, extracted from a Kaggle notebook, that performs this task. The same code can be rewritten as in Figure \ref{fig:nsmallest}. Figure \ref{fig:sort_values} first sorts the values and then selects the first 5 elements whereas the code in Figure \ref{fig:nsmallest} uses the Pandas function \code{nsmallest}, which selects the \code{n} smallest elements directly. As we can expect, Figure \ref{fig:nsmallest} is faster. Similarly, there are many opportunities in real-world notebooks to perform rewrites into more optimized versions. %We describe such rewrite opportunities in detail in Section~\ref{}.

Rewriting appears simple, but it can be challenging when performed manually. There are many non-obvious rewrites that the user may not be able to discover easily. For example, it might seem that the only way to make \code{pandas} code faster through rewriting is by replacing it with other \code{pandas} code, or using a similar library such as \code{numpy}. This has been reinforced over years of data scientists being trained to remain within \code{pandas}/\code{numpy} as much as possible because these use native, vectorized implementations and are thus much faster pure Python.

\begin{figure}
  \begin{subfigure}{\columnwidth}
\begin{minted}[bgcolor=light-gray]{python}
df[['a', 'b']] = df['C'].str.split('(', expand=True)
\end{minted}
    \caption{Splitting a \code{pandas.Series} using
    \code{pandas.Series.str.split()}. Extracted from a Kaggle notebook \cite{real_nb_split}.}
    \label{fig:split_pandas}
  \end{subfigure}
  \hfill
  \begin{subfigure}{\columnwidth}
\begin{minted}[bgcolor=light-gray]{python}
a = []
b = []
ls = df['C'].tolist()
for it in ls:
    spl = it.split('(', 1)
    a.append(spl[0])
    b.append(spl[1] if len(spl) > 1 else None)
df['a'] = pd.Series(a, df['C'].index)
df['b'] = pd.Series(b, df['C'].index)
\end{minted}
    \caption{Splitting a \code{pandas.Series} in pure Python}
    \label{fig:split_python}
  \end{subfigure}
  \caption{Splitting in \code{pandas} and Python. Surprisingly, the pure Python
  implementation is up to 7$\times$ faster.}
  \label{fig:split}
\end{figure}

It might, then, be surprising that moving out of \code{pandas} and into pure Python can lead to (significant) speedups. One example is shown in Figure \ref{fig:split}. The task here is to split a \code{Series} of strings by the delimiter \code{'('}. The code in Figure~\ref{fig:split_pandas} (extracted from a Kaggle notebook) does it by using a \code{pandas}-provided function. One would expect that this is the best way to perform this operation. Nevertheless, the version in Figure \ref{fig:split_python} is 3.5$\times$ faster. %What this code does is seemingly completely unorthodox. 
It moves from \code{pandas} to pure Python (by converting \code{df['C']} to a Python list) and performs the operation with a sequential Python loop \footnote{It then converts the results to \code{Series} for functional equivalence with the original. Note that it has to have the same index; more on that later.} (in our case studies in Section~\ref{sub-sec:case-studies}, we explain why this version is faster). We detail some of the rewrite rules we use in Section~\ref{sec:pandas_rewr_rules} including these complex rules.

It is unreasonable to expect general \code{pandas} users to comprehend Python, \code{pandas}, and \code{numpy} to such an extensive level so as to be able to discover such equivalent versions and evaluate their relative performance. Second, even if the user succeeds in these tasks, the rewritten version can be significantly harder to write and read, as is evident from Figure \ref{fig:split}. This can further lead to correctness concerns about the rewrite. Third, manual rewriting breaks the library abstraction. In the original code of Figure \ref{fig:split}, the user has to think only of \emph{what} \code{split()} does. But, to come up with the rewritten version, this abstraction is broken as the user needs to think of \emph{how} to implement it. %Finally, rewriting becomes more complicated once correctness concerns arise. 

%\begin{figure}
  \begin{subfigure}{\columnwidth}
\begin{minted}[bgcolor=light-gray]{python}
def foo(row):
  if row['A'] == row['B'] and row['A'] < row['C']:
    return 'X'
  elif row['A'].startswith('Y'):
    return 'Y'
  elif row['B'] in ls:
    return 'Z'
  else:
    return 'NA'

df.apply(foo, axis=1)
\end{minted}
    \caption{Original \code{pandas} \code{apply()}. It processes each row
    sequentially, using the interpreter.}
    \label{fig:apply_vectorized_orig}
  \end{subfigure}
  \hfill
  \begin{subfigure}{\columnwidth}
\begin{minted}[bgcolor=light-gray]{python}
conditions = [
  (df['A'] == df['B']) & (df['A'] < df['C']),
  df['A'].str.startswith('Y'),
  df['B'].isin(ls)
]
choices = [
  'X', 'Y', 'Z'
]
np.select(conditions, choices, default='NA')
\end{minted}
    \caption{Vectorized execution using \code{numpy.select()}}
    \label{fig:apply_vectorized_rewr}
  \end{subfigure}
  \caption{VectorizedConditionals: Vectorized \code{apply()} with conditions, which can be hundreds of
  times faster \cite{pygotham_apply_vectorized}. However, performing this
  rewrite automatically is challenging.}
  \label{fig:apply_vectorized}
\end{figure}

This motivated us to build \system{}, a system that performs such rewrites \textit{automatically}, by guaranteeing \textit{correctness} and with minimal overhead. Section \ref{sec:System} describes the internal workings of \system{} in detail. 

Rewriting avoids the previously mentioned drawbacks of library-based optimization systems. First, it inherently does not suffer from a lack of API support since the rewritten code uses only \code{pandas} and Python features that are already supported. \stef{The previous seems to miss the important and exciting fact that the rewriter is fundamentally different. The lack of API support does not even make sense in this context. I'd rewrite it as: "because it is fundamentally different: it is not a replacment for \code{pandas} and it can always leave the code untouched if it cannot handle it.}  Second, the rewriting overheads scale proportionally only to the code, not the data. %In fact, the code of a single cell. 
%IPython notebooks (see Section \ref{sec:Implementation)}) are populated incrementally and thus the rewriter cannot know all the code. 
\system{} considers only a single cell at a time \footnote{it could also consider the history, although we have not found that necessary} and needs to save only the AST representation of the code and a few light and ephemeral internal data structures during rewrite code generation. %Further, the rewritten code only have a few dynamic checks. 
These overheads are thus negligible both in terms of memory and execution time \charith{Do we have numbers to show this overhead is low?}. \stef{Yes. I have to add plots}
%On the other hand, pattern-matching and rewriting reduce to fast tree search and a few dynamic checks.
%These overheads are thus negligible both in terms of memory and execution time.
Finally, when the rewriter succeeds, the rewritten code is almost always faster than the original, to the extent that there are such patterns. As we will show in Section \ref{sec:Evaluation}, the patterns we have used always result in speedups except for degenerate cases. \stef{Do we need data that the patterns always result in a speedup?}

Additionally, there are fundamental advantages \system{} has over library based optimization approaches. The rewrite system is transparent. When the user observes a speedup, they can always see the code that the rewriter used. In other words, the user does not need to understand the system to understand the cause of the speedup. At the same time, the user's code remains intact. Further, rewriting has the benefit of being able to optimize across library boundaries. For example, \system{} can automatically perform the rewrite that we described in Figure~\ref{fig:split}. \stef{This example does not drive the point home because a library could do it. That's why I had another simple example which cross library boundaries} To perform this rewrite, a tool needs to view all the code and understand semantic equivalences and differences across library boundaries (in this case \code{pandas} and the host language, Python). This is not possible with optimization appproaches that purely aim at accelerating Panadas functions.  %However, a rewriter is naturally such a tool as it views all the user code. 
However, because the rewriter views, and understands, both the library semantics and also the host language semantics (in this case, Python), it can optimize across library boundaries.


%There are many advantages of using automatic Pandas code rewriting to optimize Pandas workloads. One such advantage is that an automatic rewriting system preserves the readability and interpretability of the code. When the user observes a speedup, they can always see the code that the rewriter used. In other words, the user does not need to understand the system to understand the speedup cause. At the same time, the user's code remains intact in case they want to change it later. Second, a rewriting system has low implementation complexity. By complexity, we do not mean lines of code, as this is an irrelevant metric. Rather, the really burdensome complexity is the mental one, which is increased mostly as the number of interactions between different parts of code increases. A rewriting system's complexity is low because it needs only clearly separated components. Furthermore, the complexity does not increase with every new pattern because the patterns are independent.

%Rewriting also avoids the previously mentioned drawbacks. First, a rewriting system for Pandas is not a replacement for Pandas but a clear addition. Therefore, it inherently does not suffer from a lack of API support. Second, the rewriting overheads scale proportionally only to the code, not the data. In fact, the code of a single cell. IPython notebooks (see Section \ref{sec:Implementation)} are populated incrementally and thus the rewriter cannot know all the code. Thus, it has to consider a single cell at a time \footnote{it could also consider the history, although we have not found that necessary}. These overheads are thus negligible both in terms of memory and execution time. On the one hand, a rewriting system needs to save only the AST representation of the code and a few light internal data structures. On the other hand, pattern-matching and rewriting reduce to fast tree search and a few dynamic checks. Finally, when the rewriter succeeds, the rewritten code is almost always faster than the original, to the extent that there are such patterns. As we will show in Section \ref{sec:Evaluation}, the patterns we have used always result in speedups except for degenerate cases. \stef{Do we need data that the patterns always result in a speedup?}

%\begin{figure}
  \begin{subfigure}{\columnwidth}
\begin{minted}[bgcolor=light-gray]{python}
pd.Series(df['A'].tolist() + df['B'].tolist())
\end{minted}
    \caption{Original: Concatenate \code{Series} by first turning them into
    lists. Extracted from a Kaggle notebook \cite{real_nb_concat}.}
    \label{fig:concat_orig}
  \end{subfigure}
  \hfill
  \begin{subfigure}{\columnwidth}
\begin{minted}[bgcolor=light-gray]{python}
pd.concat([df['A'], df['B']], ignore_index=True)
\end{minted}
    \caption{Rewritten: Use a \code{pandas}-provided function for concatenation}
    \label{fig:concat_rewr}
  \end{subfigure}
  \caption{Rewrite example that crosses library boundaries, and thus cannot be
  performed by previous techniques. The rewritten version can be up to 11$\times$
  faster.}
  \label{fig:concat-with-lists}
\end{figure}


%As a last note, we note that rewriting has the non-obvious benefit of optimizing across library boundaries. Consider the case of concatenating two Pandas Series. Figure \stef{add it} shows user-written code extracted from a Kaggle notebook. The user (e.g., because they don't remember what Pandas API call performs the equivalent operation) first converts the Series into lists, concatenates them, then converts the result to a Series. Pandas provides a function to concatenate two Series directly, as shown in the rewritten version in Figure \stef{add it}. To perform this rewrite, a tool needs to view all the code because when converting the Series to lists, we \textit{cross} the library boundaries. At this point, Pandas (or any replacement for Pandas) has no control. A rewriter is naturally such a tool as it views all the user code. In summary, because the rewriter views, and understands, both the library semantics but also the host language semantics (in this case, Python), it can optimize across library boundaries.



% \subsection{Pandas Rewrite Rules}
\label{sec:pandas_rewr_rules}

% \begin{figure}
%     \begin{minted}[bgcolor=light-gray,escapeinside=||]{python}

% @{expr: called_on}
%    .sort_values()
%    .head(n=@{Constant(int): first_n}) |$\mapsto$|
% @{called_on}.nsmallest(n=@{first_n})
%    |$\mathcal{C}$|(type(@{called_on}) == DataFrame)
%     \end{minted}

%     \caption{A rewrite rule example}
%     \label{fig:sort-values-rule}
% \end{figure}


\begin{figure}
    \begin{subfigure}{\columnwidth}
  \begin{minted}[bgcolor=light-gray,escapeinside=||]{python}
@{expr: called_on}.sort_values().head(n=@{Const(int): first_n})
|{\LARGE\color{brown}$\mapsto$}|
@{called_on}.nsmallest(n=@{first_n})
  \end{minted}
      \caption{LHS {\color{brown}$\mapsto$} RHS}
      \label{fig:sort-values-rule-lhs-rhs}
    \end{subfigure}
    \hfill
    \begin{subfigure}{\columnwidth}
  \begin{minted}[bgcolor=light-gray,escapeinside=||]{python}
type(@{called_on}) == pandas.Series
  \end{minted}
      \caption{Preconditions}
      \label{fig:sort-values-rule-preconds}
    \end{subfigure}
    \caption{\revis{An example of a rewrite rule, named \textbf{SortHead}}. If we match the LHS in the source code,
    we can replace it with the RHS only if the preconditions hold (at runtime).}
    \label{fig:sort-values-rule}
  \end{figure}

% % Center column names
% \newcolumntype{P}[1]{>{\centering\arraybackslash}p{#1}}

% {\footnotesize
% \begin{table*}
% \begin{tabular}{|P{0.33\textwidth}|P{0.33\textwidth}|P{0.33\textwidth}|}
% {\LARGE LHS} &  {\LARGE RHS} & {\LARGE Preconditions} \\
% \hline \\
% \begin{lstlisting}[language=Python,basicstyle=\ttfamily, breaklines=true]
% @{Name: a}, @{Name: b} =
%   @{expr: ser}.str.split(
%     @{Constant(str): sep}, 
%       expand=@{Constant(bool): expand})
% \end{lstlisting}
% &
% \vspace{-15pt}
% \begin{lstlisting}[language=Python,basicstyle=\ttfamily, breaklines=true]
% a, b = [], []
% for it in @{ser}.tolist():
%     spl = it.split(@{sep})
%     a.append(spl[0])
%     y = spl[1] if len(spl) > 1 \
%           else None
%     b.append(y)
% @{a} = pandas.Series(a, @{ser}.index)
% @{b} = pandas.Series(b, @{ser}.index)
% \end{lstlisting}
% &
% \begin{lstlisting}[language=Python,basicstyle=\ttfamily, breaklines=true, escapeinside=||]
% |$\mathfrak S$:| @{expand} == True
% |$\mathfrak R$:| type(@{ser}) == pandas.Series
% \end{lstlisting}

% \\
% \hline
% \\

% \vspace{-12pt}
% \begin{lstlisting}[language=Python,basicstyle=\ttfamily, breaklines=true, escapeinside=||]
% @{expr: ser}.apply(
%   lambda @{Name: par1}:
%     @{Constant(str): needle} 
%       in @{Name: par2})
% \end{lstlisting}
% &
% \vspace{-5pt}
% \begin{lstlisting}[language=Python,basicstyle=\ttfamily, breaklines=true, escapeinside=||]
% res = @{ser}.tolist()
% res = [(@{needle} in s) for s in res]
% pandas.Series(res, @{ser}.index)
% \end{lstlisting}
% &
% \begin{lstlisting}[language=Python,basicstyle=\ttfamily, breaklines=true, escapeinside=||]
% |$\mathfrak R$:| type(@{ser}) == pandas.Series
% \end{lstlisting}

% \\
% \hline
% \\

% \vspace{-10pt}
% \begin{lstlisting}[language=Python,basicstyle=\ttfamily, breaklines=true, escapeinside=||]
% @{Name: df1}[@{Constant(str): c1}] = 
%   @{Name: df2}[@{Constant(str): c2}]
%     .fillna(@{expr: arg})
% \end{lstlisting}
% &
% \vspace{-5pt}
% \begin{lstlisting}[language=Python,basicstyle=\ttfamily, breaklines=true, escapeinside=||]
% @{df1}[@{c1}].fillna(@{arg}, inplace=True)
% \end{lstlisting}
% &
% \vspace{-10pt}
% \begin{lstlisting}[language=Python,basicstyle=\ttfamily, breaklines=true, escapeinside=||]
% |$\mathfrak S$:| @{df1} == @{df2}
% |$\mathfrak S$:| @{c1} == @{c2}
% |$\mathfrak R$:| type(@{df1}) == pandas.DataFrame
% \end{lstlisting}

% \\
% \hline
% \\

% \vspace{-15pt}
% \begin{lstlisting}[language=Python,basicstyle=\ttfamily, breaklines=true, escapeinside=||]
% pd.Series(
%   @{expr: e1}.tolist() + 
%   @{expr: e2}.tolist())
% \end{lstlisting}
% &
% \vspace{-10pt}
% \begin{lstlisting}[language=Python,basicstyle=\ttfamily, breaklines=true, escapeinside=||]
% pd.concat([@{e1}, @{e2}], ignore_index=True)
% \end{lstlisting}
% &
% \vspace{-10pt}
% \begin{lstlisting}[language=Python,basicstyle=\ttfamily, breaklines=true, escapeinside=||]
% |$\mathfrak R$:| pd == pandas
% |$\mathfrak R$:| type(@{e1}) == pandas.Series
% |$\mathfrak R$:| type(@{e2}) == pandas.Series 
% \end{lstlisting}

% \end{tabular}
% \vspace{10pt}
% \caption{Rewrite Rule Examples \stef{There are line cuts in the 3rd column}}
% \label{tbl:rewr_rules}
% \end{table*}
% }

% % MINTED DIDN'T WORK.
% %
% % \begin{figure*}
% % \begin{tabular}{ |p{0.3\textwidth} | p{0.3\textwidth} | p{0.3\textwidth} | }
% %   \begin{minted}[escapeinside=||]{python}
% %     @{expr: called_on}
% %     .sort_values()
% %     .head(n=@{Constant(int): first_n})
% %   \end{minted}
% %   &
% %   \begin{minted}[escapeinside=||]{python}
% %     @{expr: called_on}
% %     .sort_values()
% %     .head(n=@{Constant(int): first_n})
% %   \end{minted}
% %   &
% %   \begin{minted}[escapeinside=||]{python}
% %     @{expr: called_on}
% %     .sort_values()
% %     .head(n=@{Constant(int): first_n})
% %   \end{minted}
% % \end{tabular}
% % \end{figure*}

\begin{table*}[t]
  \centering
  \includegraphics[height=0.85\textheight]{figures/sources/tbl_rewr_rules.pdf}
  \caption{Examples of Rewrite Rules. If any of the LHS's is matched, it can be
  replaced with the corresponding RHS, provided that the preconditions hold. The
  symbol $\mathfrak S$ denotes syntactic preconditions while $\mathfrak R$
  denotes runtime ones. The name of the rule appears as a comment in the LHS
  column.}
  \label{tbl:rewr_rules}
\end{table*}

%\begin{figure}

\begin{mathpar}
\inferrule
{(\code{@\{name\}}, \Gamma_{1}) \Downarrow Obj, \Gamma_{2} \\ (\code{type(Obj)}, \Gamma_{2}) \Downarrow \code{DataFrame}, \Gamma_{2} \\ (LHS, \Gamma_{1}) \Downarrow VLHS, \Gamma_{4} \\ (RHS, \Gamma_{1}) \Downarrow VRHS, \Gamma_{5} }
{VLHS = VRHS \wedge \Gamma_{4} = \Gamma_{5}}
\end{mathpar}

    \caption{Formal Semantics the Rewrite Rule in Figure \ref{fig:sort-values-rule}}
    \label{fig:sort-values-formal}
\end{figure}

The abstract form of the rewrite rules \system{} supports can be modeled as transforming a Left Hand Side (LHS) set of statements to Right Hand Side (RHS) set of statements subject to certain preconditions on the LHS. 
We introduce some notation to show the structure of our parameterized rewrite rules. The parameterized portions of the rewrite rules are general and can match multiple valid code segments subject to certain conditions (e.g. types). For example consider the original code in Figure~\ref{fig:example_rewrite}(a) rewritten to Figure~\ref{fig:example_rewrite}(b) using the rewrite rule shown in Figure~\ref{fig:sort-values-rule}. %For example consider the rewrite rule shown in Figure~\ref{fig:sort-values-rule}. \system{} uses it to perform the rewrite shown in Figure~\ref{fig:example_rewrite}. We begin with the LHS. 
A \code{@\{...\}} entry denotes a parameterized part of the rule. These parts
can be matched to multiple valid options by \system{}. Inside the curly brackets,
we describe these valid options using a derivation rule of the Python grammar
\cite{python_ast}. For example, \code{@\{expr\}} denotes that any expression can
appear in its place. For \code{Constant}s, we optionally specify the type of the
constant inside parentheses.\footnote{We can determine the type of constants
from the AST \cite{python_ast_constants}.} So, \code{@\{Constant(int)\}} denotes
that any integer constant can appear in its place. We need to refer to the parts of the LHS that are parameterized in the preconditions and the RHS. So, we bind these parts to names. For example, the code string \code{df.sort\_values().head()} matches the LHS of Figure~\ref{fig:sort-values-rule} and \code{called\_on} is bound to \code{df}. Everything that is not in \code{@\{\}} should appear as is.
%It is important to note that the LHS section is concerned purely with syntax. This implies that the bindings, and their uses in the other sections, are purely \textit{textual}; no evaluation happens (a name binding in the LHS is like a macro definition in C). 
With these in mind, we can read the LHS of Figure~\ref{fig:sort-values-rule} as matching any Python expression on which \code{sort\_values()} is applied, followed by \code{head()} with any constant integer as the argument of the formal parameter \code{n}.

There are two kinds of preconditions, syntactic and runtime ones. Syntactic preconditions describe conditions related to the matched \textit{text}. Usually, they require that two matched entries of the LHS are \emph{syntactically} equal. The runtime preconditions describe conditions which have to hold at \emph{runtime} for the original (LHS) and the rewritten code (RHS) to be semantically equivalent and they are expressed in Python syntax and semantics. For example, in Figure \ref{fig:sort-values-rule}, the result of the \code{called\_on} expression that was matched in the LHS should be a \code{pandas.DataFrame}. The runtime preconditions implicitly impose an order of evaluation. In this example, \code{called\_on} must be evaluated first, then the preconditions are checked on the resulting object, and then this object is used in place of \code{called\_on} in the RHS. Note that unconditionally evaluating \code{called\_on} is correct even if the conditions do not hold because it would be evaluated anyway in the original.

%Section~\ref{sec:Evaluation} presents in-depth case studies on real world Pandas codes where non-trivial rewrites enable significant performance improvements. For brevity, we only show the parameterized form of the rewrite rule when the fixed and parameterized parts of the rewrite is not apparent from the context.

% The preconditions should be evaluated when the LHS would be evaluated. If the preconditions pass, then evaluating the RHS should be functionally equivalent with evaluating the RHS. %Finally, in the RHS, we just list the code with which we replace the LHS. %We give a more formal description in Figure \ref{fig:sort-values-formal} in the form of operational semantics \stef{Give citation; not worth describing here}. %\stef{Not sure how to write this last sentence}. %We consider such formal descriptions to be out of scope for this paper, so for the rest of it, we will use our simplified notation of preconditions as in Figure \ref{fig:sort-values-rule}.


% The $\Gamma$'s are contexts, which for simplicity can be thought as snapshots of the Python namespace. Above horizontal line there is a set of premises, which, if true, they entail the statements below the line. The premise $(E, \Gamma_{1}) \Downarrow V, \Gamma_{2}$ states that if the expression $E$ is evaluated under context $\Gamma_{1}$, then its value is $V$ and it updates the context to $\Gamma_{2}$. In short, the rule says that if we evaluate \code{called\_on} first and then check the precondition in Figure \ref{fig:sort-values-rule}, then evaluating the LHS should be functionally equivalent with evaluating the RHS.

%The interpretation of the RHS section is the same as that of the LHS section.

%\charith{Too low level explain with high-level constructs}. \stef{Not sure what this means :/}


Table~\ref{tbl:rewr_rules} shows three more rewrite rules we use in \system{}. The first two correspond to the examples in Figure~\ref{fig:split} and Figure~\ref{fig:concat-with-lists}, respectively. Rules can have both runtime and syntactic conditions. For example, in the second rule, we have the syntactic precondition \code{@\{par1\} == @\{par2\}} requires that the two names be equal. To differentiate between the two kinds of preconditions, we prefix the syntactic preconditions with $\mathfrak S$ and the runtime ones with $\mathfrak R$.

% Easter egg: This style of typesetting was used in Turing's seminal paper.

%These rewrite rules span simple .... to complex ... \stef{Can't have diversity in how complex the rules are. Either we have such simple rules or way more complicated like vectorization. . As an alternative, I put rules that are tricky and explained all of them concisely.} \charith{Explain 2 rules concisely from the table}. 

% The first rule is the one in Figure~\ref{fig:split} that we discussed earlier. In Section~\ref{sub-sec:case-studies} we present an in-depth case study. %Finally, we have yet another rule that crosses the library boundaries as we move from Pandas to Python (by converting to a list) and then back to Pandas. The interesting thing here is in the preconditions: the name \code{pd} must be bound to the Pandas module. It is possible that there is another module that the user imported as \code{pd} which has a function \code{Series} that accepts lists.

%We note that Series are not just collections of elements. They also have an index with which the elements are accessed. Usually, this is a \code{RangeIndex} \cite{pandas_range_idnex}, i.e. a contiguous range of integers as in Python lists. But it need not be, and thus in rules where the rewritten version is implemented using pure Python and lists, when converting back to a \code{Series}, we need to preserve the original index.

 

%We will now present \system{}, a system that accelerates interactive data science workloads by transparently rewriting Python that interfaces with Pandas code and which addresses all the previously mentioned challenges. In order to do so, \system{} has two high-level components. First, \system{}' \textit{syntactic pattern-matcher} identifies matches the input code against the LHS parts of the rewrite rules. The second component is a \emph{rewriter}, which validates the preconditions of the rewrite rules and on passing them, rewrites the code to the RHS version and executes it. We show a high-level overview in Figure \ref{fig:system_overview}.

%\stef{Elaboration on the Figure?}

%We now describe each component in depth.

\section{Implementation}
\label{sec:impl}

At \company, we have deployed \sysname in our internal clusters to serve daily DL workloads.
The internal clusters consist of heterogeneous GPUs, including NVIDIA T4 GPU and NVIDIA A10 GPU.
Integrated with Kubernetes~\cite{k8s}, \sysname manages thousands of GPUs in each cluster and more than 20,000 GPUs in all.

\parabf{Service manager.}
For online workloads, we use the existing service manager at \company which deploys containers, discovers service, and autoscales horizontal pods.

\parabf{Global manager.}
We modify the Kubernetes scheduler to schedule offline workloads.
The workload profiler takes several dedicated GPUs, whose number is negligible to the total number of GPUs.
When a new offline workload comes, the workload profiler performs a few dry runs of the workload and utilizes the NVIDIA Data Center GPU Manager (DCGM) tools~\cite{dcgm} and NVIDIA Management Library (NVML)~\cite{nvml} libraries to collect GPU metrics.
We collect about 2,000 data for each GPU type to train the speed predictor.
The MLPs of the speed predictor have four layers with hidden size $64\times 64$.
The MLPs are trained with momentum SGD optimizer~\cite{ruder2016overview} in PyTorch v1.8.0~\cite{paszke2019pytorch} until they converge.
\sysname invokes the scheduler periodically to schedule all offline workloads.
When moving workloads, we record checkpoints of offline workloads and restart the workloads after transmitting the models and checkpoints.
As the datasets are usually colossal, we store the datasets in a remote file system and fetch data during the execution.
We implement the scheduler as a third-party plugin to the Kubernetes scheduler.


\parabf{Local executor.}
Each local executor executes online workloads according to the service manager and offline workloads according to the global manager.
DL workloads are executed in Docker containers with our customized components.
We add Best-Effort GPU DevicePlugin in Kubernetes and relevant control paths with Kubelet and \sysprobe for offline workloads.
To control SM percentage, we leverage the environment variable $CUDA\_MPS\_ACTIVE\_THREAD\_PERCENTAGE$ provided by MPS.
The GPU monitor collects resource metrics through DCGM~\cite{dcgm} and NVML~\cite{nvml} for NVIDIA GPU.
The \sysprobe updates the state machine with the collected resource metrics and empirically-set thresholds.
When the state is unhealthy, the \sysprobe will ask the NodeManager in Kubernetes to evict offline workloads.
\bytecuda intercepts nearly 800 CUDA driver APIs for GPU memory allocation and kernel launch.
The GPU memory quota of offline workloads is fixed to $40\%$ as Figure~\ref{fig:motiv_gpu_resource} reports that most online workloads use less than $60\%$ GPU memory.
We adopt the cpuset of Cgroup for CPU isolation.
For memory, \sysname will evict offline workloads if memory usage is higher than a threshold or the kernel swap daemon is busy for a long time.
The parameters to calculate GPU load in Equation~\ref{equ:gpu_load}$\&$\ref{equ:clock_factor} are empirically selected through trial-and-error.

 \section{Benchmarks and Evaluation}
\label{sec:eval}

We evaluate \krakenSpace to answer the following set of questions:
\begin{itemize}
\item How much improvement does partial evaluation and our implemented compiler optimizations give \kraken? %(\S \ref{sec:eval2})
\item How much faster is our purely functional f-expr language, \krakenSpace, compared to other implementations of fexprs? %(\S \ref{sec:eval1} - \ref{sec:eval2})
\item How does \kraken's performance, with its fexprs, compare to macros? %(\S \ref{sec:eval1}, \S \ref{sec:eval3})
\item How do the different partial evaluation mechanisms/optimizations in \krakenSpace contribute towards reduction in overall runtime?
%\item What does \krakenSpace do internally when we create a data structure and evaluate it for some function? (\S \ref{sec:casestudy})
\end{itemize}

\textbf{Experimental Setup}: 
We ran these experiments in a reproducible Nix environment on a NixOS install \cite{10.1145/1411203.1411255} (Kernel 6.0.0) on a laptop with 8 cores / 16 threads and 64 GB of RAM.
Our code contains the scripts and Nix Flakes needed to reproduce the exact set of dependencies to run our tests.
%The code can be found at \url{https://github.com/limvot/kraken}.

The Kraken benchmarks were run using both the Wasmtime and WAVM WebAssembly engines for most benchmarks.
The Wasmtime WebAssembly engine is one of the most popular, developed by the Bytecode Alliance itself, and uses the CraneLift code generation backend.
The WAVM WebAssembly engine is interesting for its use of LLVM, and it often produces the fastest code on benchmarks but has a higher startup time.
We eliminated the Cfold Wasmtime benchmark due to problems running out of stack space (a known property of the Cfold benchmark).

\textbf{Benchmarks}: 
To showcase the capability of Kraken, we created benchmarks that are commonly implemented in functional languages and have been used as benchmarks in other papers \cite{reinking2021perceus, 10.1145/3547646}.
The benchmarks are
\begin{itemize}
\item Fib - Calculating the nth Fibonacci number
\item RB-Tree - Inserting n items into a red-black tree, then traversing the tree to sum its values
\item Deriv - Computing a symbolic derivative of a large expression
\item Cfold - Constant-folding a large expression
\item NQueens - Placing n number of queens on the board such that no two queens are diagonal, vertical, or horizontal from each other
\end{itemize}
All benchmarks besides Fibonacci use the fexpr version of match for pattern matching in \kraken, which is equivalent to the macro version in NewLisp. We also RB-Tree using NewLisp's~\cite{mueller2018newlisp} version of fexpr match. We modified the sizes of the problems presented to the benchmark to account for the longer running times of some of the less-optimized implementations.
The code for Kraken and NewLisp is very similar, and we should note that it is very unidiomatic NewLisp.
Our goal was not to compare Kraken and NewLisp as implementation languages for Red-Black Trees, but to stress test a single reasonably complex fexpr/macro, namely pattern matching.
% \textbf{Comparison with other languages}: We evaluated \krakenSpace against a language that contains f-exprs, as well as against itself with various optimizations disabled. The only other language we could find which contains a real f-expr mechanism is NewLisp~\cite{mueller2018newlisp} and so we ported \kraken's benchmark implementation to NewLisp.

%The six state-of-the-art languages are Java 17.0.1, Swift 5.4.2, Koka 2.3.2, C++, Haskell 8.10.7, and OCaml 4.12.
%The language choices were taken directly from Perceus reference-counting paper \cite{reinking2021perceus}.
%The Fibonacci benchmark additionally tests Python 3.9.11 and Chez Scheme 9.5.4.
%Koka, Ocaml and Haskell are good comparison points as statically-typed, compiled, functional programming languages, while Chez Scheme is a good comparison point as a mature and industrial strength dynamically-typed Scheme implementation known for its performance. 
%\subsection{Basic Level Comparison}
\subsection{The Effect of Partial Evaluation on Eval Calls}

\begin{table}[h]
\caption{Number of eval calls with no partial evaluation for Fexprs}
	\begin{tabular}{||c | c c c c c ||} 
		\hline
		&Evals & Eval w1 Calls & Eval w0 Calls & Comp Dyn & Comp Dyn\\ 
        & & & & w1 Calls & w0 Calls\\ [0.5ex] 
		\hline\hline
		Cfold 5 & 10897376 & 2784275 & 879066  & 1 & 0 \\ 
		\hline
		  Deriv 2  & 11708558 & 2990090 & 946500 & 1 & 0 \\ 
        \hline
		  NQueens 7 & 13530241 & 3429161 & 1108393 & 1 & 0 \\ 
    \hline
		  Fib 30 & 119107888 & 30450112 & 10770217 & 1 & 0 \\ 
    \hline
		  RB-Tree 10 & 5032297 & 1291489 & 398104 & 1 & 0 \\ 
		\hline
	\end{tabular}
    \label{npe:calls}
 \end{table}

As mentioned before, using fexprs without partial evaluation will prelude optimization and cause a massive amount of repeated work. Table \ref{npe:calls} and Table \ref{pe:calls} show the number of calls to the \krakenSpace runtime's eval function, the number of times the runtime's eval function executed a call to an applicative with wrap\_level=1, the number of times the runtime's eval function executed a call to an operative with wrap\_level=0, the number of compiled dynamic calls to applicatives with wrap\_level=1, and the number of compiled dynamic calls to operatives with wrap\_level=0.
These are shown for \krakenSpace test cases with partial evaluation turned off and turned on. 
\begin{table}[h]
\caption{Number of eval calls in Partially Evaluated Fexprs}
	\begin{tabular}{||c | c c c c c ||} 
		\hline
		&Evals & Eval w1 Calls & Eval w0 Calls & Comp Dyn & Comp Dyn\\ 
        & & & & w1 Calls & w0 Calls\\ [0.5ex] 
		\hline\hline
		Cfold 5 & 0 & 0 & 0  & 0 & 0 \\ 
		\hline
		  Deriv 2  & 0 & 0 & 0 & 2 & 0 \\ 
        \hline
		  NQueens 7 & 0 & 0 & 0 & 0 & 0 \\ 
    \hline
		  Fib 30 & 0 & 0 & 0 & 0 & 0 \\ 
    \hline
		  RB-Tree 10 & 0 & 0 & 0 & 10 & 0 \\ 
		\hline
	\end{tabular}
    \label{pe:calls}
 \end{table}

\begin{table}[h]
\caption{Number of calls to the runtime's eval function for RB-Tree. The table shows the non-partial evaluation numbers -> partial evaluation numbers.}
	\begin{tabular}{||c | c c c c c ||} 
		\hline
		&Evals & Eval w1 Calls & Eval w0 Calls & Comp Dyn & Comp Dyn\\ 
        & & & & w1 Calls & w0 Calls\\ [0.5ex] 
		\hline\hline
		  RB-Tree 7 & 2952848 -> 0 & 757932 -> 0 & 233513 -> 0 & 1 -> 7 & 0 -> 0\\ 
        \hline
		  RB-Tree 8 & 3532131 -> 0 & 906548 -> 0 & 279379 -> 0 & 1 -> 8 & 0 -> 0\\ 
        \hline
		  RB-Tree 9 & 4278001 -> 0 & 1097965 -> 0 & 3383831 -> 0 & 1 -> 9 & 0 -> 0\\ 
		\hline
	\end{tabular}
    \label{pe:rb}
    \vspace{-4mm}
 \end{table}

Without partial evaluation, no compilation can be done because it is impossible to tell if arguments to calls will be evaluated. In all benchmarks, partial evaluation removed all calls to the runtime's eval function, resulting in a completely compiled program. Looking at RB-Tree, there are over a million calls to combiners with wrap level 1 (normal functions), and 398,000 calls to combiners with wrap level 0 (operatives replacing macros). This massive blowup in the number of calls is due to the repeated and exponential re-execution of macro-like-combiners in the definition of other macro-like-combiners, as discussed in the Introduction.

The non-partially-evaluated benchmarks show 1 compiled dynamic call to an applicative (its the first call into eval) and 0 compiled dynamic calls to operatives, because there is no compilation at all. For the partially evaluated benchmarks, there are a few compiled dynamic calls to applicatives due to higher-order function use in the benchmarks, and there are no compiled dynamic calls to operatives, as all operative use has been eliminated.
We also varied the inputs for RB-Tree shown in Table \ref{pe:rb} to give a sense for how the number scale with respect to input size.

The incredible slowdown implied by these tables comes to full fruition in our RB-Tree test in Fig.~\ref{fig:kraken_nqueens_rbtree}.
We kept this run shorter because Kraken's non-partial-evaluating interpreter takes an incredibly long time even for 100 insertions (40 minutes).
The compounding layers of repeated macro-like operative calls in the non-partially-evaluated Kraken version cause a ~70,000x slowdown relative to the partial evaluated, optimized, and compiled version.
For the remaining benchmarks, we remove the naive interpreted \krakenSpace version, as in each case its performance is so bad as to blow out the graph and make it impossible to do any comparison.
In our optimized Kraken, our partial evaluation algorithm is able to fully collapse these levels of inefficiency, evaluate and inline the results, and give the backend more specialized code to optimize, emitting a compiled version that handily beats not only the NewLisp-fexpr implementation but even the NewLisp-macro implementation, as can be seen in Fig.~\ref{fig:kraken_vs_world_fib}.
We kept the benchmark sizes small in this test because the stack limits of NewLisp prevent sizes larger then ~880, while the Tail Call Elimination performed by the \krakenSpace compiler allows us to run much larger benchmarks, including the run of 4,800,000 inserts to the RB-Tree.
This result shows the dramatic effect of partial evaluation and compiler optimizations on runtime for \kraken. Our technique takes the performance of a fully fexpr based language from being completely infeasible to being faster than a macro-based dynamic scripting language currently in use.
% \begin{center}
% \begin{table}[ht]
% \caption{Number of call to the runtime's eval function for Fib. The table shows the non-partial evaluation numbers -> partial evaluation numbers}
% 	\begin{tabular}{||c | c c c c c ||} 
% 		\hline
% 		&Evals & Eval w1 Calls & Eval w0 Calls & Comp Dyn w1 Calls & Comp Dyn w0 Calls\\ [0.5ex] 
% 		\hline\hline
% 		Fib 10 & 8468 -> 0 & 2167 -> 0  & 777 -> 0 & 1 -> 0 & 0 -> 0 \\ 
% 		\hline
% 		  Fib 15  & 87916 -> 0 & 22478 -> 0 & 7961 -> 0 & 1 -> 0 & 0 -> 0 \\ 
%         \hline
% 		  Fib 20 & 969010 -> 0 & 247731 -> 0 & 87633 -> 0 & 1 -> 0 & 0 -> 0 \\ 
%     \hline
% 		  Fib 25 & 10740492 -> 0 & 2745825 -> 0  & 971209 -> 0 & 1 -> 0 & 0 -> 0 \\ 
% 		\hline
% 	\end{tabular}
%     \label{pe:fib}
%  \end{table}
% \end{center}

\begin{figure}[h]
\caption{Constant Fold and Deriv}
\includegraphics[width=0.45\textwidth]{cfold_table.csv_}
\includegraphics[width=0.45\textwidth]{deriv_table.csv_}
\label{fig:kraken_const_deriv}
\vspace{-6mm}
\end{figure}
\subsection{Comparison between Kraken Versions}
Beyond the massive speedup from partial-evaluation, Fig. \ref{fig:kraken_const_deriv} and \ref{fig:kraken_nqueens_rbtree} show the effect of the various compiler optimizations we described by disabling them one by one.
 Our main four optimizations have a strong positive effect on runtime, with the exception of lazy environment instantiation. Lazy environment instantiation helps massively on fib, and some on Deriv, but generally hurts the rest slightly.


\begin{figure}[h]
\caption{N-Queens}
\includegraphics[width=0.45\textwidth]{nqueens_table.csv_}
\includegraphics[width=0.45\textwidth]{slow_rbtree_table.csv_}
\label{fig:kraken_nqueens_rbtree}
\vspace{-4mm}
\end{figure}


\subsection{Comparison against Others}


To give a general idea of our current performance, we also show a Fibonacci benchmark that mostly exercises pure function-call speed and inlining as seen in Fig. ~\ref{fig:kraken_vs_world_fib}.
We include Python and Chez Scheme to give a general idea for where an exemplar slow and an exemplar fast dynamic language would fall.
With the benefit of our partial evaluation, compilation, and leaning upon mature WebAssembly implementations, we beat both, but this should be taken with a grain of salt, as this is a very limited micro-benchmark only meant to give a general sense of the order of magnitude of our performance.



\label{sec:eval1}
\begin{figure}[h]
\caption{Kraken vs. Others. Ordered by fastest to slowest}
\includegraphics[width=0.45\textwidth]{fib_table.csv_}
\includegraphics[width=0.45\textwidth]{rbtree_table.csv_}
\label{fig:kraken_vs_world_fib}
\end{figure}

%\label{sec:eval_nqueens}
%\begin{figure}[h]
%\caption{N-Queens}
%\includegraphics[width=0.45\textwidth]{nqueens_table.csv_}
%\includegraphics[width=0.45\textwidth]{slow_nqueens_table.csv_}
%\label{fig:kraken_nqueens}
%\end{figure}

%\label{sec:eval_nqueens}
%\begin{figure}[h]
%\caption{Kraken, N-Queens, absolute value and log-scale}
%\includegraphics[width=0.45\textwidth]{nqueens_table.csv_}
%\includegraphics[width=0.45\textwidth]{nqueens_table.csv_log}
%\label{fig:kraken_nqueens}
%\end{figure}
%\label{sec:eval_nqueensp}
%\begin{figure}[h]
%\caption{Kraken, N-Queens, absolute value and log-scale}
%\includegraphics[width=0.45\textwidth]{slow_nqueens_table.csv_}
%\includegraphics[width=0.45\textwidth]{slow_nqueens_table.csv_log}
%\label{fig:kraken_nqueensp}
%\end{figure}

%\label{sec:eval_cfold}
%\begin{figure}[h]
%\caption{C-Fold}
%\includegraphics[width=0.45\textwidth]{cfold_table.csv_}
%\includegraphics[width=0.45\textwidth]{slow_cfold_table.csv_}
%\label{fig:kraken_cfold}
%\end{figure}
%\label{sec:eval_cfold}
%\begin{figure}[h]
%\caption{Kraken, C-Fold, absolute value and log-scale}
%\includegraphics[width=0.45\textwidth]{cfold_table.csv_}
%\includegraphics[width=0.45\textwidth]{cfold_table.csv_log}
%\label{fig:kraken_cfold}
%\end{figure}
%\label{sec:eval_cfoldp}
%\begin{figure}[h]
%\caption{Kraken, C-Fold, absolute value and log-scale}
%\includegraphics[width=0.45\textwidth]{slow_cfold_table.csv_}
%\includegraphics[width=0.45\textwidth]{slow_cfold_table.csv_log}
%\label{fig:kraken_cfoldp}
%\end{figure}

%\label{sec:eval_deriv}
%\begin{figure}[h]
%\caption{Deriv}
%\includegraphics[width=0.45\textwidth]{deriv_table.csv_}
%\includegraphics[width=0.45\textwidth]{slow_deriv_table.csv_}
%\label{fig:kraken_deriv}
%\end{figure}
%\label{sec:eval_deriv}
%\begin{figure}[h]
%\caption{Kraken, Deriv, absolute value and log-scale}
%\includegraphics[width=0.45\textwidth]{deriv_table.csv_}
%\includegraphics[width=0.45\textwidth]{deriv_table.csv_log}
%\label{fig:kraken_deriv}
%\end{figure}
%\label{sec:eval_derivp}
%\begin{figure}[h]
%\caption{Kraken, Deriv, absolute value and log-scale}
%\includegraphics[width=0.45\textwidth]{slow_deriv_table.csv_}
%\includegraphics[width=0.45\textwidth]{slow_deriv_table.csv_log}
%\label{fig:kraken_derivp}
%\end{figure}

%\subsection{Comparison against state-of-the-art languages}
%\label{sec:eval3}

%\begin{figure}[h]
%\caption{Kraken vs. S.o.t.A.}
%\includegraphics[width=0.45\textwidth]{cfold_table.csv_}
%\includegraphics[width=0.45\textwidth]{rbtree_table.csv_}
%\label{fig:kraken_vs_world1}
%\end{figure}

%\begin{figure}[h]
%\caption{Kraken vs. S.o.t.A.}
%\includegraphics[width=0.45\textwidth]{deriv_table.csv_}
%\includegraphics[width=0.45\textwidth]{nqueens_table.csv_}
%\label{fig:kraken_vs_world2}
%\end{figure}

% \begin{figure}[h]
% \caption{Kraken vs. S.o.t.A. (Log)}
% \includegraphics[width=0.45\textwidth]{cfold_table.csv_log}
% \includegraphics[width=0.45\textwidth]{rbtree_table.csv_log}
% \label{fig:kraken_vs_world_log_1}
% \end{figure}
% \begin{figure}[h]
% \caption{Kraken vs. S.o.t.A. (Log)}
% \includegraphics[width=0.45\textwidth]{deriv_table.csv_log}
% \includegraphics[width=0.45\textwidth]{nqueens_table.csv_log}
% \label{fig:kraken_vs_world_log_2}
% \end{figure}

%As we noted before with the Fib(30) microbenchmark in Section \ref{sec:eval1}, we remain significantly slower than state-of-the-art compiled languages.
%This is particularly true for memory-intensive benchmarks due to our naive reference-counting and malloc/free implementations.
%However, our results are of a similar order of magnitude to the difference between the state-of-the-art compiled languages and dynamic scripting languages, like Python's results in the Fib(30) microbenchmark.
%We assert that is not a fundamental limitation because the classic f-expr slowness is being eliminated, as shown by Fig. \ref{fig:kraken_vs_newlisp1} and Fig. \ref{fig:kraken_vs_newlisp2}.
%In future work, we plan to expand our compile-time analysis and optimization to implement a modified, dynamic-language version of Perceus reference counting.
%With this change, we belive \krakenSpace can be competitive with these state-of-the-art languages.

%\subsection{Case Study: Red-Black Tree}
%\label{sec:casestudy}

%\begin{figure}[h]
%\caption{Kraken vs. S.o.t.A. - RB-Tree Focus}
%\includegraphics[width=0.4\textwidth]{rbtree_table.csv_}
%\includegraphics[width=0.4\textwidth]{rbtree_table.csv_log}
%\label{fig:kraken_vs_world_rbtree}
%\end{figure}


%To evaluate our partial evaluation algorithm and compiler, we extracted the benchmarks used by the Koka language project from their code repository and added Kraken versions, as well as implementing a naive Fibonacci microbenchmark ourselves to evaluate pure function call speed.\\
%With partial evaluation and the compiler optimizations listed above, we get fairly strong performance on purely numerical computations, such as the naive Fibonacci microbenchmark.
%Unfortunately, the overhead of our unsophisticated reference counting, dynamic type checking, and bounds checking causes poor performance on benchmarks involving data structures relative to mainstream programming language implementations.
%This is not a fundamental limitation, and will be addressed in future work, as recounted in the next section.
%It should be noted, however, that while the performance relative to established language implementations is very poor for the memory-intensive benchmarks (600-900x slower), we still realize a massive speedup compared to an unoptimized and non-partial-evaluated f-expr implementation (100,000x faster)!

\section{Related work}
% There is extensive recent work on speeding up analytical queries due to the need for consistent execution times in the face of the explosive growth in the volume of available data.
% In this section, we divide existing work into two categories: maintaining data freshness in MVs (\Cref{sec:server_side}) and optimizations for minimizing ad-hoc query latency (\Cref{sec:client_side}).

% \subsection{Maintaining Data Freshness in MVs}
% \label{sec:server_side}
% There exists a variety of data warehousing applications aimed at supporting low-latency analytical queries on fresh data.
% In particular, these applications require efficiency in the propagation of newly ingested data into downstream MVs.
 
\mypara{Efficient MV Refresh}
Incremental view maintenance (IVM) aims to update MVs to reflect newly ingested data, taking advantage of already computed results to perform the update in a manner more efficient than computing from scratch (full refresh)
~\cite{ahmad2012dbtoaster,mcsherry2013differential,armbrust2013generalized,zeng2016iolap, palpanas2002incremental, griffin1995incremental, agiwal2021napa, braun2015analytics}. 
There is an abundance of work in IVM, including incremental updates on duplicate values~\cite{griffin1995incremental}, non-distributive aggregate functions~\cite{palpanas2002incremental}, higher-order views~\cite{ahmad2012dbtoaster}, and sliding windows~\cite{braun2015analytics}. 
More recent works also investigate the scalability aspect of IVM, proposing scale-independent updates~\cite{armbrust2013generalized} and sampled views~\cite{zeng2016iolap}. Since \system is applicable to arbitrary SQL statements, \system is orthogonal to and is fully compatible with existing IVM techniques.

\mypara{MV Refresh Scheduling}
There exist works on scheduling the refresh of a MV set focusing on resolving cyclic dependencies~\cite{folkert2005optimizing}, minimizing weighted average staleness~\cite{golab2009scheduling}, and prioritizing MVs with the highest speedups on predicted future queries~\cite{ahmed2020automated}.
\system's scheduling to speed up the end-to-end refresh of the MV set is not addressed in existing works.

\mypara{DAG Workflow Scheduling}
The execution of workloads consisting of individual jobs with acyclic dependencies is a well-studied topic~\cite{apacheoozie,sparkdag,marchal2018parallel,bathie2020revisiting,baruah2022ilp}; many of these techniques can be applied to MV refresh runs studied in this paper.
Existing workflow scheduling systems such as Apache Oozie~\cite{apacheoozie}, Apache Airflow~\cite{airflow}, and Spark DAG scheduler~\cite{sparkdag} automate the execution of user-defined workflows following a topological order.
There are recent works aimed at finding more optimal execution orders in terms of peak memory usage~\cite{marchal2018parallel, bathie2020revisiting} and execution time on parallel platforms~\cite{baruah2022ilp}.
While \system is designed for use with MV refresh runs/workloads, our technique on joint scheduling and optimization can be reasonably applied to general workloads as a possible future direction.

% \paragraph{Incremental MV indexing}
% Update-optimized indices such as the log-structured merge-trees (LSM)~\cite{o1996log} are used for indexing MVs due to frequent updates induced by data ingestion~\cite{gupta2016mesa,agiwal2021napa}.
% \system is orthogonal to indexing: \system is capable of efficiently performing MV refresh runs regardless of whether the individual MVs are indexed or not.

% \subsection{Ad-hoc Query Latency Reduction}
% \label{sec:client_side}

% The minimization of ad-hoc analytical query response times is a well-studied topic due to latency being negatively correlated with the productivity of a data analyst during a data analysis session~\cite{liu2014effects}.
% Sessions are commonly conducted within visualization systems that contain a variety of optimization techniques to ensure that query response times fall within a certain latency tolerance.

% \mypara{Data prefetching}
% Data is often loaded into memory on a by-need basis in visualization systems to minimize interference with user-issued query computations~\cite{mani2017effective,xin2021enhancing,galakatos2017revisiting, yan2020auto, battle2016dynamic, crotty2016case, jalaparti2018netco}. 
% Query-time data retrieval can be significantly expedited by anticipating the data usage of the user in future queries and pre-loading the data into memory during the downtime between user queries (`think time'). SMART~\cite{mani2017effective} prefetches data for modified versions of current user-issued queries with different filters and dimensions. A-WARE~\cite{crotty2016case} maintains a sample store constantly refined through ingesting data based on speculations of future plots.
% ForeCache~\cite{battle2016dynamic} uses an SVM to predict the user's current analysis phase and accordingly prefetches data tiles partitioned based on different numerical values. NetCo predicts future queries via log analysis, and solves an ILP formulation to prefetch data to maximize the number of SLO-meeting queries~\cite{jalaparti2018netco}.
% In the case of MV refresh workloads, `think time' is nonexistent as individual MVs are refreshed back-to-back, rendering data prefetching techniques non-applicable.

\mypara{Intermediate Data Caching}
Some existing data visualization systems cache user-defined variables to support the typical incremental construction of data visualizations~\cite{zgraggen2016progressive, eichmann2020idebench} during data analysis sessions~\cite{jupyter, rstudio, colab}. 
Recent work proposes a management scheme for these cached variables under a memory constraint that greedily keeps variables with the highest estimated time savings based on predicted future user behavior via neural networks~\cite{xin2021enhancing}.
While useful for data visualization, a greedy approach to memory management fails to achieve satisfactory results compared to \system.

\mypara{Intermediate Result Reuse}

There exist works on storing intermediate results from computations to speedup future computations~\cite{yang2018intermediate, dursun2017revisiting, nagel2013recycling, michiardi2019memory, galakatos2017revisiting}.
Studied topics include the identification of reuse opportunities by finding overlaps in computation graphs of successive jobs~\cite{yang2018intermediate, michiardi2019memory},
selective storage under a space constraint with heuristics such as reuse probability~\cite{dursun2017revisiting}, expected savings~\cite{yang2018intermediate}, and recompute-storage cost difference~\cite{nagel2013recycling},
and rewriting incoming jobs to take advantage of stored intermediates~\cite{galakatos2017revisiting}.
These works share similarity with \system in their selection of items to store under a memory constraint, however, \system's problem setting requires it to uniquely consider the joint (re)ordering of job executions along with the selection of items.

% work that considers both job execution (re)order as well as intermediate result caching with a bounded amount of memory. but notably lack the joint aspect of \system and cannot be used to achieve immediate speedup on an incoming MV refresh run if no intermediates are stored beforehand. 

\mypara{Incremental Query Processing} Incremental processing (IQP) is useful for cases where not all data required for a query is immediately available. Similar to online aggregation~\cite{hellerstein1997online}, initial results of a query are computed on a subset of required data and progressively refined as the rest of the required data arrives in a predictable pattern~\cite{tang2019intermittent,wangtempura}. Tang et al. propose a dynamic programming formulation to pick intermediate states to store in memory given a limited memory budget~\cite{tang2019intermittent}. Tempura rewrites the query plan for more efficient execution based on predicted data arrival patterns~\cite{wangtempura}. While similarities exist between the problem setting of IQP and \system, such as management of bounded memory, \system notably includes additional joint optimization for the order of MV updates.

% \paragraph{Sampling}
% Sampling has seen wide use in visualization systems for reducing the computation time of ad-hoc queries by computing an approximate result over a subset of data as exact results are not always required by the user~\cite{crotty2016case, mani2017effective, zgraggen2014panoramicdata, kraska2021northstar, galakatos2017revisiting, kandula2016quickr}. 
% Commonly studied topics in sampling for ad-hoc queries include complex query sampling~\cite{kandula2016quickr}, rare event aggregation~\cite{kraska2021northstar, galakatos2017revisiting}, and maintaining consistency between related sampled visualizations~\cite{zgraggen2014panoramicdata}.
% Sampling server-side at the MV level compromises the assumptions of downstream applications and is thus not considered in \system.

% \paragraph{Progressive visualization}
% The latency tolerance for time-consuming queries can be circumvented by presenting a partially-computed visualization to the user within the tolerance, which is then incrementally refined until it is fully accurate~\cite{rahman2017ve, zgraggen2016progressive, crotty2015vizdom, kraska2021northstar, kamat2017infiniviz}.
% Example plots which benefit from progressive visualization include bar charts~\cite{kamat2017infiniviz} and heatmaps~\cite{rahman2017ve}.
% Similar to sampling, study on this topic is orthogonal to \system as pushing out partially-updated MVs compromises downstream assumptions.
%\section{Limitations and Future Work}

We summarize the limitations we have identified for our method and propose
future research directions.

\textbf{Parallel implementation:} 
With a focus on accuracy and algorithms, our implementation for this work is
serial. Some of the most time-consuming routines in our method can easily
benefit from a parallel implementation, while the same is not obvious for the
SAP solver and the Schur complement computation. Leveraging the power of
parallelization on modern hardware for these computations is an interesting area
for future investigation.

\textbf{Rotational invariance:} 
As with all other linear constitutive models, our linearized model with lagged
rotational component is not rotationally invariant. Thus it is not suitable for
simulation of extreme deformations using large time steps. For those scenarios,
we fall back to traditional nonlinear models with Hessian positive definite
corrections proposed in \cite{bib:teran2005robust}.

\textbf{Self-contact:} 
We do not consider self-contact at the moment due to the lack of support by our
geometry engine. Self-contact can be incorporated into our method by updating the
geometry engine to augment the set of contacts reported.

\textbf{Tunneling at high speeds:} Though our method has a lower computational
cost, it could benefit from continuous collision detection strategies
\cite{bib:li2020ipc} to provide constraints before contact is established. This
would allow to mitigate issues such as objects tunneling past each other at high
speeds. Efficient solution to mitigate this issue is a topic of active research
for the authors.

\textbf{Redundant constraints:} Our geometry engine often introduces a large
number of constraints to resolve contact. Similarly, welding a large number of
deformable mesh vertices to a rigid body (as done in Section
\ref{sec:bubble_gripper}) introduces many constraints. Even though our SAP
solver \cite{bib:castro2022unconstrained} provides existence and uniqueness
guarantees, a large number of constraints hurts performance as can be observed
in the \emph{Soft-bubble} example. We are currently investigating strategies to
significantly reduce the number of constraints without sacrificing accuracy.

\section{Conclusion}\label{sec:conclusion}
In this work, we focus on addressing the fundamental challenge of OOD detection tasks, which is how to fully understand the semantic discrepancy between the ID/OOD samples. We reveal that the key to success in the realistic SCOOD task is to allocate as many ID samples in the unlabeled set correctly as possible. To this end, we propose a novel uncertainty-aware optimal transport scheme that introduces class-specific energy scores as guidance for effective label assignment. Experimental results show that our method achieves better performance than previous state-of-the-art methods on SCOOD benchmarks.

\textbf{Limitations.} In addition to temperature scaling, other techniques such as feature clipping applied in ReAct~\cite{sun2021react} also enhance the performance of energy score, so how to obtain an OOD score that best fits the SCOOD task can be further explored. Moreover, a setting highly related to SCOOD has been proposed in \cite{katz2022training} and formulated as a constrained optimization problem. We will also theoretically analyze these practical OOD settings in our feature work.

% \section*{Acknowledgments}
\textbf{Acknowledgments.} 
This work is supported by National Key R\&D Program of China under Grant 2020AAA0105701, National Natural Science Foundation of China (NSFC) under Grants 61872327, Major Special Science and Technology Project of Anhui, National Natural Science Foundation of China (62033012) and Ant Group through Ant Research Intern Program.


\begin{acks}
We would like to thank Marc Canby, Stratos Vamvourellis, Edward Gan and the anonymous reviewers for insightful comments and suggestions. This work was supported by the AWS Cloud Credit for Research and the Open Philanthropy project.
\end{acks}

\appendix
%\section{Details on the Experimental Setup}

\paragraph{\textbf{Machine}}

\begin{itemize}
    \item \textbf{CPU}: AMD Ryzen 9 5900X 12-Core, clocked at 3.7 GHz
    \item \textbf{RAM}: 32 GB DRAM
    \item \textbf{SSD}: Samsung 980 PRO
    \item \textbf{OS}: Ubuntu 22.04.1 LTS
    \item \textbf{Quiescing Settings}:
        \begin{itemize}
            \item Boot in terminal mode
            \item Disable turbo boost
            \item Disable frequency scaling
        \end{itemize}
\end{itemize}

We picked this as a conventional consumer machine, except for the SSD which is unconventionally fast.

\paragraph{\textbf{Python/Pandas Setup}}
\begin{itemize}
    \item Python: 3.10.6
    \item IPython: 8.5.0
    \item Numpy: 1.23.4
    \item Pandas: 1.5.1
\end{itemize}

\paragraph{\textbf{Benchmark}}

We disabled some code in these notebooks that we consider out of scope. In particular:
\begin{itemize}
    \item Networking or Shell code, i.e., cells like \code{pip install} or like \code{nltk.download()}.
        \begin{itemize}
            \item This means that to reproduce our experiments, you may need to run the original notebook once e.g., to download the required \code{nltk} dataset.
        \end{itemize}
    \item Plotting and machine-learning code
    \item IPython magic functions because our tool can only handle valid Python code. For these notebooks, we only had to remove \code{\%matplotlib inline}.
\end{itemize}


\section{Extended Results}

In Section~\ref{sec:Evaluation} we focused only on ten out of the twenty random
notebooks we picked (see Section~\ref{sub-sec:exp-setup}). Here, we include
results for all twenty notebooks.

\paragraph{\textbf{Per-Cell Speedups}} Figure~\ref{fig:ext:cell_level} shows the
cell-level speedups, corresponding to Figure~\ref{fig:cell_level}. The plots
look almost identical, and this is because \system{} does not decelerate
notebooks it does not rewrite. Thus, since this plot includes only slowdowns or
speedups that are outside the 10\% range, there is hardly any discernible
difference. Similar observations are derived from
Figure~\ref{fig:ext:cells-slowdowns}, where the slowdowns are still under
interactive latency, i.e., 300ms.

These results further validate our hypothesis in
Section~\ref{sub-sec:vs-pandas}. That is, the slowdowns we observe are the
result of rewriting, independent of who performs it (in this case, \system{}).

When we include all twenty notebooks, the geometric mean speedup is 1.1$\times$
and the maximum slowdown is 28\%.

\paragraph{\textbf{Per-Notebook Speedups}} In Figure~\ref{fig:ext:nb_level} we
show the notebook-level speedups. This figure corresponds to
Figure~\ref{fig:nb_level}. As we mentioned, \system{} does not rewrite code in
the the ten new notebooks, so we do not see any additional speedup. However, it
remains that the slowdowns, when \system{} does not succeed, are minimal. The
geometric mean speedup is now 1.13$\times$ while the maximum slowdown is 3.5\%.

\paragraph{\textbf{Comparison with Modin}} In
Figure~\ref{fig:ext:modin_nb_level}, which corresponds to
Figure~\ref{fig:modin_nb_level}, our conclusions are again unaltered.
\code{modin} slows down all the ten new notebooks and it rarely scales with the
number of cores. The geometric mean and maximum speedup remain the same (see
Figure~\ref{fig:modin_nb_level}).


In Figure~\ref{fig:ext:modin_nb_mem}, which corresponds to
Figure~\ref{fig:modin_nb_mem}, we show the memory consumption of \system{},
\code{pandas}, and \code{modin}, when we consider all twenty notebooks. The
results are not significantly different for \code{pandas} and \system{}. However,
\code{modin}'s memory consumption becomes even more aggressive. We see that
for one notebook, \code{modin} consumes almost 250GB when \code{pandas} and
\system{} consume less than 5GB.

\hfill{}\\
\hfill{}\\
\hfill{}\\
\hfill{}\\
\hfill{}\\
\hfill{}\\
\hfill{}\\
\begin{figure}[b]
  \centering
  \includegraphics[width=\columnwidth]{figures/extended/sources/cells_only_slowdowns.pdf}
  \caption{Corresponding to Figure~\ref{fig:cells-slowdowns}. The conclusions
  are the same. The slowdowns are within interactive latencies (i.e., less than
  300ms).}
  \label{fig:ext:cells-slowdowns}
\end{figure}
\begin{figure}[t]
  \centering
  \includegraphics[width=\columnwidth]{figures/extended/sources/mem_usage.pdf}
  \caption{Corresponding to Figure~\ref{fig:modin_nb_mem}. When we include all
  20 notebooks, we see even more aggressive memory+disk usage from
  \code{modin}. \system{} and \code{pandas} remain on the same scales.}
  \label{fig:ext:modin_nb_mem}
\end{figure}
\begin{figure}[!h]
    \centering
    \includegraphics[width=0.8\columnwidth]{figures/extended/sources/nb_level.pdf}
    \caption{Relative speedups on whole notebooks. We see the same speedups as
    in Figure~\ref{fig:nb_level} with no extra substantial slowdowns when
    considering all 20 notebooks.}
    \label{fig:ext:nb_level}
\end{figure}


\begin{figure*}[!b]
    \centering
    \includegraphics[width=\textwidth]{figures/extended/sources/cell_level.pdf}
    \caption{Cell-level relative speedups (excluding cells that originally ran
    for less than 50ms and also all the cells that got a speedup or slowdown
    within the 10\% range) for all 20 notebooks. Still, \system{} provides
    significant speedups with no substantial slowdowns (see
    Figure~\ref{fig:ext:cells-slowdowns}).}
    \label{fig:ext:cell_level}
\end{figure*}

\begin{figure*}[t]
  \centering
  \includegraphics[width=\textwidth]{figures/sources/modin_nb.pdf}
  \caption{Comparing \system{} with \code{modin} \cite{modin}. \system{} is
  faster for 9 out of 10 notebooks (Up to 26.4$\times$ faster with 4.1$\times$
  geometric mean). \code{modin} is, in many cases significantly, slower than the
  original for these 9 notebooks. For the one notebook where \system{} is
  slower, it is no more slower than 3\% compared to the original.}
  \label{fig:modin_nb_level}
\end{figure*}


\bibliographystyle{ACM-Reference-Format}
\bibliography{sample, rewrite, jit}

\end{document}
\endinput
%%
%% End of file `sample-sigconf.tex'.
