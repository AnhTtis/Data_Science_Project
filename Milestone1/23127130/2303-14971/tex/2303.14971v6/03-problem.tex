% \RSTODO{riscrivi \S3 e \S4 di
% conseguenza, spiegando prima l'intuiizione e poi l'esempio.}
  In this section, we present a theoretical analysis of the impact of different CNF-izations on %\ignoreinlong{the AllSAT task}\ignoreinshort{
the enumeration of short partial truth assignments. %}. 
In particular, we focus on \TseitinCNF~\cite{tseitinComplexityDerivationPropositional1983} and \PlaistedCNF~\cite{plaistedStructurepreservingClauseForm1986}.
% \ignoreinshort{ %
% }
We point out how CNF-izing AllSAT problems using these transformations can introduce unexpected drawbacks for enumeration. % when they are used to preprocess non-CNF formulas. 
In fact, we show that the resulting encodings can force the enumerator
to produce partial assignments that are larger in size and in
number than necessary.
%
In our analysis we refer to AllSAT, but it applies to All\smt{} as well by restricting to theory-satisfiable truth assignments.\@ Moreover, the analysis applies to both disjoint and non-disjoint enumeration.

\subsection{The impact of Tseitin CNF transformation}%
\label{sec:problem:label}

We show that %preprocessing the input formula
using the \TseitinCNF{}
transformation~\cite{tseitinComplexityDerivationPropositional1983} can
be problematic for enumeration.
In particular, we point out a fundamental weakness of \TseitinCNF{}:
%
%\ADDED{
\begin{fact}%
  \label{fact:tseitin}
  If a partial assignment \muA{} suffices to satisfy $\vi$, this does not imply that \muA{} suffices to satisfy $\exists
  \allB.\vicnfts$.
   \end{fact}
   \noindent
   In fact, we recall that \TseitinCNF{} works by applying recursively the rewriting step (\sref{sec:bg:cnf}):
    \begin{eqnarray}
    \label{eq:rewritingTseitin}
    \vi\Longrightarrow\vi[\vi_i|B_i] \wedge (B_i\iff\vi_i)
    \end{eqnarray}
    \noindent and then by recursively CNF-izing the two conjuncts.
%
    A    {\em partial} assignment \muA{} may suffice to satisfy a non-CNF formula 
    $\vi(\allA)$  because it does not need to assign a truth value to
    the atoms in {\em all} sub-formulas of $\vi$. (E.g., \muA{} can satisfy
    $\vi\defas\vi_1\vee\vi_2$ without assigning values to the atoms in
    $\vi_2$ if it satisfies $\vi_1$.)
    %
    Consider \eqref{eq:rewritingTseitin} s.t.\ $\vi_i$ is some
    sub-formula of \vi{} whose atoms are not assigned by \muA{}.
    Although \muA{} suffices to satisfy $\vi$,
    \muA{} does not
    suffice to satisfy $\exists B_i.( \vi[\vi_i|B_i]\wedge
    (B_i\iff\vi_i))$. In fact, to satisfy the second conjunct it
    is necessary to assign some truth value not only to $B_i$ but also
    to some of the unassigned atoms in $\vi_i$, so
    that to make $\vi_i$ evaluate to the same truth value assigned to
    $B_i$.
   
 %   \noindent
% %    In general, \cref{fact:tseitin} can be explained as follows.
%     In fact, a
%     {\em partial} assignment \muA{} may suffice to satisfy a non-CNF formula 
%     $\vi(\allA)$  because it does not need to assign a truth value to the
%     atoms in \emph{all}
%     subformulas of $\vi$. (E.g., \muA{} can satisfy
%     $\vi\defas\vi_1\vee\vi_2$ without assigning values to the atoms in
%     $\vi_2$ if it satisfies $\vi_1$.)
%     %
%     Consider some subformula  $\vi_i$ of \vi{} whose atoms are not assigned by \muA{}.
%     %
%     We recall that \TseitinCNF{} works by applying recursively the rewriting step (\sref{sec:bg:cnf}):
%     %\GMSIDENOTE{In \sref{sec:bg:cnf} abbiamo usato \DeMorganCNF($B_i\iff\vi_i$)}
%     \begin{eqnarray}
%     \label{eq:rewritingTseitin}
%     \vi\Longrightarrow\vi[\vi_i|B_i] \wedge (B_i\iff\vi_i)
%     \end{eqnarray}
%     %\vi\Longrightarrow\vi[\vi_i|B_i] \wedge \DeMorganCNF(B_i\iff\vi_i)$$
%     %
%     %$$\TseitinCNF{(\vi)}=\TseitinCNF{(\vi[\vi|B_i])}\wedge\DeMorganCNF{(B_i\iff\vi_i)}$$.
%     \noindent and then by recursively CNF-izing the two conjuncts.
%     Unfortunately, although \muA{} suffices to satisfy $\vi$,
%     \muA{} does not
%     suffice to satisfy $\exists B_i.( \vi[\vi_i|B_i]\wedge
%     (B_i\iff\vi_i))$, because to satisfy the second conjunct it
%     is necessary to assign some truth value not only to $B_i$ but also
%     to some of the atoms in $\vi_i$, so
%     that to make $\vi_i$ evaluate to the same truth value given to $B_i$.


    As a consequence of \cref{fact:tseitin}, given $\muA$ satisfying \vi{}, in order to
  produce an assignment \muAprime{} satisfying $\exists 
  \allB.\vicnfts$  the enumerator is most often forced to %(unnecessarily)
  assign
  other 
  atoms in \allA{}, so that $\muAprime\supset\muA$.
%
  Given the fact that the amount of total assignments covered by a partial
  assignment decreases exponentially with its length (see \sref{sec:background:propositional-logic}), the above weakness 
  causes a blow-up in
  the number of partial assignments 
  %which we need generating
  needed to cover all
  models.
  This may drastically affect the effectiveness and efficiency of the enumeration.

  This is illustrated in the following example, where instead of
  one single short partial assignment the enumerator
  is forced to enumerate 9 longer ones. 


%We first illustrate this issue with an example.
%\newpage
\begin{example}%
    \label{ex1}
    Consider the propositional formula  over %the set of atoms
    $\allA\defas\set{A_1, A_2, A_3, A_4, A_5, A_6, A_7}$:
    \begin{equation}
        \label{eq:ex1:vi}
        % \vi \defas 
        %     \overbrace{(\underbrace{(A_1 \wedge A_2)}_{B_1} \vee  A_3)}^{B_2} \iff
        %     \overbrace{ (\underbrace{(A_4 \wedge A_5 )}_{B_3} \vee  A_6) }^{B_4}
        % \overbrace{(A_1 \wedge A_2)}^{B_1} \vee  
        % \overbrace{(A_3 \iff \underbrace{((A_4\vee A_5) \wedge 
        % \overbrace{(A_6 \vee A_7)}^{B_2})}_{B_3})}^{B_4}
        \vi \defas
        \overbrace{(A_1\wedge A_2)}^{B_1}\vee
        \overbrace{(
            \overbrace{(
                \overbrace{(A_3\vee A_4)}^{B_2}\wedge
                \overbrace{(A_5\vee A_6)}^{B_3}
                )}^{B_4}\iff
            A_7
            )}^{B_5}.
    \end{equation}
      \noindent
$\vi$ is not in CNF, and thus it must be CNF-ized before starting the enumeration process.
    If \TseitinCNF{} is used, then the following CNF formula is obtained:
    
    \begin{subequations}%
        \label{eq:ex1:vicnf}
        \begin{alignat}{2}
            % \vicnf \defas
            % &(\neg B_1\vee\pos A_1)\wedge(\neg B_1\vee\pos A_2)\wedge(\pos B_1\vee\neg A_1\vee\neg A_2)&\wedge\label{eq:ex1:vicnf:line1}\\
            % &(\pos B_2\vee\neg B_1)\wedge(\pos B_2\vee\neg A_3)\wedge(\neg B_2\vee\pos B_1\vee\pos A_3)&\wedge\label{eq:ex1:vicnf:line2}\\
            % &(\neg B_3\vee\pos A_4)\wedge(\neg B_3\vee\pos A_5)\wedge(\pos B_3\vee\neg A_4\vee\neg A_5)&\wedge\label{eq:ex1:vicnf:line3}\\
            % &(\pos B_2\vee\neg B_3)\wedge(\pos B_2\vee\neg A_6)\wedge(\neg B_2\vee\pos B_3\vee\pos A_6)&\wedge\label{eq:ex1:vicnf:line4}\\
            % &(\neg B_4 \vee\pos B_2)\wedge(\pos B_4\vee\neg B_2)&\label{eq:ex1:vicnf:line5}
            %-------------------------------
            % \psi\defas
            % &(\neg B_1\vee\pos A_1)\wedge(\neg B_1\vee\pos A_2)\wedge(\pos B_1\vee\neg A_1\vee\neg A_2)&\wedge\label{eq:ex1:vicnf:line1}\\
            % &(\pos B_2\vee\neg A_5)\wedge(\pos B_2\vee\neg A_6)\wedge(\neg B_2\vee\pos A_5\vee\pos A_6)&\wedge\label{eq:ex1:vicnf:line2}\\
            % &(\neg B_3\vee\pos A_4)\wedge(\neg B_3\vee\pos B_2)\wedge(\pos B_3\vee\neg A_4\vee\neg B_2)&\wedge\label{eq:ex1:vicnf:line3}\\
            % &(\neg B_4\vee\neg A_3\vee B_3)\wedge(\neg B_4\vee A_3\vee\neg B_3)\wedge
            % ( B_4\vee A_3\vee B_3)\wedge( B_4\vee\neg A_3\vee\neg B_3)
            % &\wedge\label{eq:ex1:vicnf:line4}\\
            % &(\pos B_1\vee\pos B_4)&
            % --------------------------------------
             & \vicnfts\defas\nonumber                                                                                                         \\
             & \enspace(\neg B_1\vee\pos A_1)\wedge(\neg B_1\vee\pos A_2)\wedge(\pos B_1\vee\neg A_1\vee\neg A_2) & \wedge
             & \quad\eqcomment{(B_1\iff (A_1\wedge A_2))}\label{eq:ex1:vicnf:line1}                                                            \\
             & \enspace(\pos B_2\vee\neg A_3)\wedge(\pos B_2\vee\neg A_4)\wedge(\neg B_2\vee\pos A_3\vee\pos A_4) & \wedge
             & \quad\eqcomment{(B_2\iff (A_3\vee A_4))}\label{eq:ex1:vicnf:line2}                                                              \\
             & \enspace(\pos B_3\vee\neg A_5)\wedge(\pos B_3\vee\neg A_6)\wedge(\neg B_3\vee\pos A_5\vee\pos A_6) & \wedge
             & \quad\eqcomment{(B_3\iff (A_5\vee A_6))}\label{eq:ex1:vicnf:line3}                                                              \\
             & \enspace(\neg B_4\vee\pos B_2)\wedge(\neg B_4\vee\pos B_3)\wedge(\pos B_4\vee\neg B_2\vee\neg B_3) & \wedge
             & \quad\eqcomment{(B_4\iff (B_2\wedge B_3))}\label{eq:ex1:vicnf:line4}                                                            \\
             & \enspace(\neg B_5\vee\pos B_4\vee\neg A_7)\wedge(\neg B_5\vee\neg B_4\vee\pos A_7)                 & \wedge
             & \quad\eqcomment{(B_5\iff (B_4\iff A_7))}\label{eq:ex1:vicnf:line5}                                                              \\
             & \enspace(\pos B_5\vee\pos B_4\vee\pos A_7)\wedge(\pos B_5\vee\neg B_4\vee\neg A_7)                 & \wedge\nonumber            %\tag{...}
            \\
             & \enspace(\pos B_1\vee\pos B_5)                                                                     & \label{eq:ex1:vicnf:line6}
        \end{alignat}
    \end{subequations}
    where the fresh atoms $\allB\defas\set{B_1, B_2, B_3, B_4, B_5}$ label
    sub-formulas as in~\eqref{eq:ex1:vi}.

    Consider the (minimal) partial truth assignment:
    \begin{equation}
        \label{eq:ex1:muA}
        \muA\defas\set{\neg A_3,\neg A_4,\neg A_7}.
    \end{equation}
    \muA{} suffices to satisfy $\vi$ \eqref{eq:ex1:vi}, {even though it does not assign a truth value to the sub-formulas $(A_1\wedge A_2)$ and $(A_5\vee A_6)$.} % since the atoms $A_1, A_2, A_5, A_6$ are not assigned.
%
    %Unfortunately, 
    {Yet, }$\muA$ does not suffice to
satisfy $\exists\allB.\vicnfts$.
    In fact,  {there is no total truth
      assignment $\etaB$ on \allB{} such that
      $\muA\cup\etaB\pmodels\vicnfts$},
    because~\eqref{eq:ex1:vicnf:line1} and~\eqref{eq:ex1:vicnf:line3}
    cannot be satisfied by assigning only variables in $\allB$;
    rather, it is necessary to further assign  at least one atom in
$\set{A_1,A_2}$ to satisfy~\eqref{eq:ex1:vicnf:line1} and at least one 
in $\set{A_5,A_6}$ to satisfy~\eqref{eq:ex1:vicnf:line3}.
    %
    
Suppose an enumerator assigns  first the literals in \muA~\eqref{eq:ex1:muA}, which force to assign also $\muB\defas\set{\neg B_2, \neg B_4, B_5}$ due to~\eqref{eq:ex1:vicnf:line2},~\eqref{eq:ex1:vicnf:line4}, and~\eqref{eq:ex1:vicnf:line5} respectively. Since $\muA\cup\muB$ satisfies all clauses
except those in~\eqref{eq:ex1:vicnf:line1}
and~\eqref{eq:ex1:vicnf:line3}, 
then the enumerator needs extending $\muA\cup\muB$ by enumerating the
partial assignments on the
unassigned atoms \set{A_1,A_2,A_5,A_6,B_1,B_3} which satisfy~\eqref{eq:ex1:vicnf:line1} and~\eqref{eq:ex1:vicnf:line3}.
Regardless of the search strategy adopted, this requires generating no
less than
$9$ satisfying partial assignments on
$\set{A_1,A_2,A_5,A_6}$.~%
%$\set{A_1,A_2,A_5,A_6,B_1,B_3}$.~%
\footnote{
    The set of models for~\eqref{eq:ex1:vicnf:line1}
    is
\set{
\set{\pos B_1,\pos A_1,\pos A_2},
\set{\neg B_1,\pos A_1,\neg A_2},
\set{\neg B_1,\neg A_1,\pos A_2},
\set{\neg B_1,\neg A_1,\neg A_2}
}, which can be covered only either by 
\set{
\set{\pos B_1,\pos A_1,\pos A_2},
\set{\neg B_1,\neg A_1},
\set{\neg B_1,\pos A_1,\neg A_2}
} or by
\set{
\set{\pos B_1,\pos A_1,\pos A_2},
\set{\neg B_1,\neg A_2},
\set{\neg B_1,\neg A_1,\pos A_2}
} 
in the case of disjoint enumeration,
and by
\set{
\set{\pos B_1,\pos A_1,\pos A_2},
\set{\neg B_1,\neg A_1},
\set{\neg B_1,\neg A_2}
} in the case of non-disjoint enumeration.
In all cases,   we need no less than 3 distinct partial assignments on
\set{A_1,A_2}. Similar considerations hold for
\eqref{eq:ex1:vicnf:line3}. Thus, we need no
less than
$3\times3=9$ partial assignments on $\set{A_1,A_2}\cup\set{A_5,A_6}$.
%$3\times3=9$ partial assignments on $\set{B_1,A_1,A_2}\cup\set{B_3,A_5,A_6}$.
%representing the $4\times4=16$ total ones.
% \GMNOTE{
% Bisogna coprire tutti i modelli totali projected su \set{A_1, A_2}
% }
}
%
    For instance, %\ignoreinshort{
    in the case of disjoint AllSAT, %}
    instead of the single partial assignment
    $\muA$~\eqref{eq:ex1:muA}, the solver may return the following
    list of 9 partial assignments satisfying $\exists\allB.\vicnfts$
    which extend \muA{}:
    \begin{equation}%
        \label{eq:ex1:muA:all}
        \begin{array}{llllllll}
            \multicolumn{2}{c}{\overbrace{\rule{1.5cm}{0pt}}^{B_1}} &           &           & \multicolumn{2}{c}{\overbrace{\rule{1.5cm}{0pt}}^{B_3}} &           &                        \\
            \{\neg A_1,                                             &           & \neg A_3, & \neg A_4,                                               & \neg A_5, & \neg A_6, & \neg A_7\}
            \quad\eqcomment{\set{\neg B_1,\neg B_2,\neg B_3,\neg B_4,\pos B_5}}                                                                                                                                       \\
            \{\neg A_1,                                             &           & \neg A_3, & \neg A_4,                                               & \pos A_5, &           & \neg A_7\}
            \quad\eqcomment{\set{\neg B_1,\neg B_2,\pos B_3,\neg B_4,\pos B_5}}                                                                                                                                       \\
            \{\neg A_1,                                             &           & \neg A_3, & \neg A_4,                                               & \neg A_5, & \pos A_6, & \neg A_7\}
            \quad\eqcomment{\set{\neg B_1,\neg B_2,\pos B_3,\neg B_4,\pos B_5}}                                                                                                                                       \\
            \{\pos A_1,                                             & \neg A_2, & \neg A_3, & \neg A_4,                                               & \neg A_5, & \neg A_6, & \neg A_7\}
            \quad\eqcomment{\set{\neg B_1,\neg B_2,\neg B_3,\neg B_4,\pos B_5}}                                                                                                                                       \\
            \{\pos A_1,                                             & \neg A_2, & \neg A_3, & \neg A_4,                                               & \pos A_5, &           & \neg A_7\}
            \quad\eqcomment{\set{\neg B_1,\neg B_2,\pos B_3,\neg B_4,\pos B_5}}                                                                                                                                       \\
            \{\pos A_1,                                             & \neg A_2, & \neg A_3, & \neg A_4,                                               & \neg A_5, & \pos A_6, & \neg A_7\}
            \quad\eqcomment{\set{\neg B_1,\neg B_2,\pos B_3,\neg B_4,\pos B_5}}                                                                                                                                       \\
            \{\pos A_1,                                             & \pos A_2, & \neg A_3, & \neg A_4,                                               & \neg A_5, & \neg A_6, & \neg A_7\}
            \quad\eqcomment{\set{\pos B_1,\neg B_2,\neg B_3,\neg B_4,\pos B_5}}                                                                                                                                       \\
            \{\pos A_1,                                             & \pos A_2, & \neg A_3, & \neg A_4,                                               & \pos A_5, &           & \neg A_7\}
            \quad\eqcomment{\set{\pos B_1,\neg B_2,\pos B_3,\neg B_4,\pos B_5}}                                                                                                                                       \\
            \{\pos A_1,                                             & \pos A_2, & \neg A_3, & \neg A_4,                                               & \neg A_5, & \pos A_6, & \neg A_7\}
            \quad\eqcomment{\set{\pos B_1,\neg B_2,\pos B_3,\neg B_4,\pos B_5}}                                                                                                                                       \\
        \end{array}
    \end{equation}
In the case of non-disjoint AllSAT, the solver may enumerate a similar
set of partial assignments, with \set{\neg A_2} instead of \set{A_1,\neg A_2} and \set{A_6} instead of \set{\neg A_5,A_6}.\exdone{}
\end{example}

%   \RSTODO{RISCRITTO FIN QUi}
  
\subsection{The impact of Plaisted and Greenbaum CNF transformation}%
\label{sec:problem:polarity}

We point out that also the \PlaistedCNF{}
transformation~\cite{plaistedStructurepreservingClauseForm1986}
  suffers for the same weakness as \TseitinCNF{} ---that
  is, \cref{fact:tseitin} holds also for \PlaistedCNF{}---
although its effects are mitigated. 

%   \noindent
   In fact, we recall that \PlaistedCNF{} works by applying recursively the rewriting step (\sref{sec:bg:cnf}):
    \begin{eqnarray}
    \label{eq:rewritingPlaisted}
      \vi\Longrightarrow\vi[\vi_i|B_i] \wedge \left \{
      \begin{array}{lll}
      (B_i\imp\vi_i)  & \mbox{if $\vi_i$ occurs only positively in $\vi$}\\
      (B_i\limp\vi_i) & \mbox{if $\vi_i$ occurs only negatively in $\vi$}\\
       (B_i\iff\vi_i) & \mbox{if $\vi_i$ occurs both positively and negatively in $\vi$}\\       
      \end{array}\right \},
    \end{eqnarray}
    \noindent and then by recursively CNF-izing the two conjuncts.
    %
    As with \TseitinCNF{}, consider~\eqref{eq:rewritingPlaisted} s.t.\ $\vi_i$ is some
    sub-formula of \vi{} whose atoms are not assigned by \muA{}.

    If $\vi_i$ occurs both positively and negatively in $\vi$, then~\eqref{eq:rewritingPlaisted} reduces to~\eqref{eq:rewritingTseitin} and \PlaistedCNF{} behaves like \TseitinCNF{}, so that     \muA{} does not
    suffice to satisfy $\exists B_i.( \vi[\vi_i|B_i]\wedge
    (B_i\iff\vi_i))$. 

    If instead $\vi_i$ occurs only positively \resp{negatively}  in
    $\vi$, then it is possible to extend \muA{} by assigning $B_i$ to
    $\bot$ \resp{$\top$} to satisfy $(B_i\imp\vi_i)$ \resp{$(B_i\limp\vi_i)$} without assigning any atom in $\vi_i$.
    Thus $\muA$ satisfies
    $\exists B_i.( \vi[\vi_i|B_i]\wedge (B_i\imp\vi_i))$
    \resp{$\exists B_i.( \vi[\vi_i|B_i]\wedge (B_i\limp\vi_i))$}.

    As with \TseitinCNF{}, %as a consequence of \cref{fact:tseitin},
    given $\muA$ satisfying \vi{}, in order to
  produce an assignment \muAprime{} satisfying $\exists 
  \allB.\vicnfpg$  the enumerator is most often forced to %(unnecessarily)
  assign
  other 
  atoms in \allA{}, so that $\muAprime\supset\muA$. We notice,
  however, that the effect of this
  problem is mitigated by the presence of single-polarity
  sub-formulas among those left unassigned by \muA{}.
  As an extreme case, if all sub-formulas in $\vi$ occur with single polarity, then no further assignment to atoms in \allA{} is needed. 
%

\begin{example}%
    \label{ex2}
    Consider the formula $\vi$~\eqref{eq:ex1:vi} as in
    \cref{ex1}. Suppose that $\vi$ is converted into CNF using
    \PlaistedCNF{}. Then, we have:
    \begin{subequations}%
        \label{eq:ex2:vicnf}
        \begin{alignat}{2}
             & \vicnfpg\defas\nonumber                                                                                                         \\
             & \enspace(\neg B_1\vee\pos A_1)\wedge(\neg B_1\vee\pos A_2)                                         & \wedge
             & \quad\eqcomment{(B_1\imp (A_1\wedge A_2))}\label{eq:ex2:vicnf:line1}                                                            \\
             & \enspace(\pos B_2\vee\neg A_3)\wedge(\pos B_2\vee\neg A_4)\wedge(\neg B_2\vee\pos A_3\vee\pos A_4) & \wedge
             & \quad\eqcomment{(B_2\iff (A_3\vee A_4))}\label{eq:ex2:vicnf:line2}                                                              \\
             & \enspace(\pos B_3\vee\neg A_5)\wedge(\pos B_3\vee\neg A_6)\wedge(\neg B_3\vee\pos A_5\vee\pos A_6) & \wedge
             & \quad\eqcomment{(B_3\iff (A_5\vee A_6))}\label{eq:ex2:vicnf:line3}                                                              \\
             & \enspace(\neg B_4\vee\pos B_2)\wedge(\neg B_4\vee\pos B_3)\wedge(\pos B_4\vee\neg B_2\vee\neg B_3) & \wedge
             & \quad\eqcomment{(B_4\iff (B_2\wedge B_3))}\label{eq:ex2:vicnf:line4}                                                            \\
             & \enspace(\neg B_5\vee\pos B_4\vee\neg A_7)\wedge(\neg B_5\vee\neg B_4\vee\pos A_7)                 & \wedge
             & \quad\eqcomment{(B_5\imp (B_4\iff A_7))}\label{eq:ex2:vicnf:line5}                                                              \\
             & \enspace(\pos B_1\vee\pos B_5).                                                                     & \label{eq:ex2:vicnf:line6}
        \end{alignat}
    \end{subequations}
    We remark that~\eqref{eq:ex2:vicnf:line1}
    and~\eqref{eq:ex2:vicnf:line5} are shorter
    than~\eqref{eq:ex1:vicnf:line1} and~\eqref{eq:ex1:vicnf:line5}
    respectively, since the corresponding sub-formulas $(A_1\wedge
    A_2)$ and $((...)\iff A_7)$ occur only with
    positive polarity in \vi{}~\eqref{eq:ex1:vi}, so that only the one-way implication
    $(B_i\imp\vi_i)$ is needed.

    As
    in \cref{ex1}, consider the 
    partial assignment $\muA\defas\set{\neg A_3,\neg A_4,\neg A_7}$
    \eqref{eq:ex1:muA} which suffices to satisfy $\vi{}$~\eqref{eq:ex1:vi}.
     As before, $\muA$ does not suffice to
satisfy $\exists\allB.\vicnfpg$.
    In fact,  {there is no total truth
      assignment $\etaB$ on \allB{} such that
      $\muA\cup\etaB\pmodels\vicnfpg$},
    because~\eqref{eq:ex2:vicnf:line3}
    cannot be satisfied by assigning only variables in $\allB$;
    rather, it is necessary to further assign  at least one atom in
%    $\set{A_1,A_2}$ to satisfy ~
%    \eqref{eq:ex2:vicnf:line1}.
%    and at least one in
    $\set{A_5,A_6}$ to satisfy~\eqref{eq:ex2:vicnf:line3}.
Notice that, unlike with \cref{ex1}, in order to satisfy~\eqref{eq:ex2:vicnf:line1} %and  \eqref{eq:ex2:vicnf:line5}
it suffices to set $B_1=\bot$ with no need to assign any atom in $\set{A_1,A_2}$.
    

    Suppose an enumerator assigns first the literals in \muA, which
    force it to assign also $\muB\defas\set{\neg B_2, \neg B_4}$ due to~\eqref{eq:ex2:vicnf:line2} and~\eqref{eq:ex2:vicnf:line4} respectively.
%
Since $\muA\cup\muB$ satisfies all clauses
except those in \eqref{eq:ex2:vicnf:line1},
\eqref{eq:ex2:vicnf:line3},
%\eqref{eq:ex2:vicnf:line5},
and
\eqref{eq:ex2:vicnf:line6},
the enumerator needs extending $\muA\cup\muB$ by enumerating
partial assignments on the
unassigned atoms \set{A_1,A_2,A_5,A_6,B_1,B_3,B_5} which satisfy them.
%
Regardless of the search strategy adopted, {to
  satisfy~\eqref{eq:ex2:vicnf:line3} it is necessary to generate no
less than
$3$ partial assignments} on
$\set{A_5,A_6}$ (see \cref{ex1});
to satisfy 
\eqref{eq:ex2:vicnf:line1},
%\eqref{eq:ex2:vicnf:line5},
and
\eqref{eq:ex2:vicnf:line6}, instead, the enumerator needs only
assigning $B_1=\bot$, which forces it to assign $B_5=\top$ due to
\eqref{eq:ex2:vicnf:line6}.

% , and
% \eqref{eq:ex2:vicnf:line5} respectively. 
     For instance, %\ignoreinshort{
    in the case of disjoint AllSAT, %}
    instead of the single partial assignment
    $\muA$, the solver may return the following
    list of 3 partial assignments satisfying $\exists\allB.\vicnfpg$
    which extend \muA{}:
        \begin{equation}%
        \label{eq:ex1:muA:all}
        \begin{array}{llllllll}
            \multicolumn{2}{c}{} &           &           & \multicolumn{2}{c}{\overbrace{\rule{1.5cm}{0pt}}^{B_3}} &           &                        \\
            \{                                             &           & \neg A_3, & \neg A_4,                                               & \neg A_5, & \neg A_6, & \neg A_7\}
            \quad\eqcomment{\set{\neg B_1,\neg B_2,\neg B_3,\neg B_4,\pos B_5}}                                                                                                                                       \\
            \{                                             &           & \neg A_3, & \neg A_4,                                               & \pos A_5, &           & \neg A_7\}
            \quad\eqcomment{\set{\neg B_1,\neg B_2,\pos B_3,\neg B_4,\pos B_5}}                                                                                                                                       \\
            \{                                             &           & \neg A_3, & \neg A_4,                                               & \neg A_5, & \pos A_6, & \neg A_7\}
            \quad\eqcomment{\set{\neg B_1,\neg B_2,\pos B_3,\neg
                                                                                                                                                                                B_4,\pos B_5}}                                                                                                                                       \\
        \end{array}
    \end{equation}
In the case of non-disjoint AllSAT, the solver may enumerate a similar
set of partial assignments, with 
\set{A_6}
instead of  \set{\neg A_5,A_6} in the third assignment.\exdone{}


    % Indeed, sub-formulas occurring with double polarity are labeled using double implications as for \TseitinCNF{}, raising the same problems as the latter. For instance, the sub-formula $(A_5\vee A_6)$ occurs with double polarity, since it is under the scope of an ``$\iff$''. Hence, the clauses in~\eqref{eq:ex2:vicnf:line3} must be satisfied by assigning a truth value also to $A_5$ or $A_6$, and so the partial truth assignment $\muA$ in~\eqref{eq:ex1:muA} does not suffice to satisfy $\exists\allB.\vicnfpg$. %because $\residual{\vicnfpg}{\muA\cup\etaB}=(\neg A_5)\wedge(\neg A_6)$.
    % \exdone{}
\end{example}

% The example above shows that \PlaistedCNF{} has an advantage over
% \TseitinCNF{} when enumerating partial assignments, but it overcomes
% its effectiveness issues only in part, {\em because a minimal assignment
%         $\muA$ satisfying $\vi$ may not suffice to satisfy
%         $\exists\allB.\vicnfpg$}, as with \TseitinCNF.

Notice that, to maximize the benefits of \PlaistedCNF{}, the
sub-formulas occurring with positive  \resp{negative} polarity only must have their
label assigned to false \resp{true}. {In practice, this can be achieved in part by instructing the solver to split on negative values in decision branches~\footnote{To exploit this heuristic also for sub-formulas occurring only negatively, the latter can be labeled with a negative label $\neg B_i$ as $(\neg B_i\limp\vi_i)$.}.
        Even though the solver is not guaranteed to always assign to false these labels, we empirically verified that this heuristic provides a good approximation of this behavior.}

    

