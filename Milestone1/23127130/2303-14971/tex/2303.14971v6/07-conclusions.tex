We have presented a theoretical and empirical analysis of the impact of different CNF-ization approaches on SAT %\ignoreinshort{\GMCHANGE{
and SMT %}}
enumeration, %\ignoreinshort{\GMCHANGE{, 
both disjoint and non-disjoint. %}}. 
We have shown how the most popular transformations conceived for SAT %\ignoreinshort{\GMCHANGEp{
and SMT %}} 
solving, namely the Tseitin and the Plaisted and Greenbaum CNF-izations, prevent the solver from producing short partial assignments, thus seriously affecting the effectiveness of the enumeration. To overcome this limitation, we have proposed to preprocess the formula by converting it into NNF before applying the Plaisted and Greenbaum transformation. We have shown, both theoretically and empirically, that the latter approach can fully overcome the problem and can drastically reduce both the number of partial assignments and the execution time.

% we plan to further investigate the
% impact of CNF conversion also on disjoint SMT enumeration. We expect that in
% this domain the impact can be even more relevant, since in SMT
% multiple instances of the same theory atoms are typically rarer than for atoms in
% the Boolean case. Also, disjoint SMT enumeration has a fundamental role in Weighted Model Integration~\cite{morettin-wmi-ijcar17,morettin-wmi-aij19,spallittaSMTbasedWeightedModel2022}, an important framework for probabilistic inference in hybrid domains. Hence, we believe that our contribution can have a great impact on this application, where non-CNF formulas occur frequently.
% Finally, we think that work should be done to understand the impact on enumeration with repetitions, i.e.\ where models may not be disjoint, for instance in Predicate Abstraction~\cite{lahiriSMTTechniquesFast2006}. %Moreover, there is an alternative definition of partial assignment satisfiability that is based on the notion of \emph{entailment}~\cite{sebastianiAreYouSatisfied2020,mohleFourFlavorsEntailment2020}. Understanding the impact of the CNF conversion on solvers that use this definition is an interesting direction.

This work opens an interesting research avenue: investigate the role
of CNF-ization in neighbor fields as d-DNNF compilation and model
counting, possibly adapting d-DNNF compilers and model counters
so that to exploit different forms of CNF-izations.
