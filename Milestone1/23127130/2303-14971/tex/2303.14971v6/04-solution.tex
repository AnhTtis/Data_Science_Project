In this section, we propose a solution to address the
shortcomings of \TseitinCNF{} and
\PlaistedCNF{} %transformations in
%\ignoreinlong{model}\ignoreinshort{
for SAT and SMT %} 
enumeration described in~\sref{sec:problem}.
% We show that a
% simple preprocessing can avoid this situation.
The idea is simple: \emph{we transform first
the input formula \vi{} into NNF, % an NNF DAG.\@
{and then we apply \PlaistedCNF{} to \NNF{\vi}.}}
In fact, NNF guarantees that every non-atomic sub-formula of \NNF{\vi} occurs only
positively, because every sub-formula $\vi_i$ occurring with double
polarity in \vi{} is converted into two syntactically-different
sub-formulas $\poslab{\vi_i}\defas\NNF{\vi_i}$ and
$\neglab{\vi_i}\defas\NNF{\neg\vi_i}$, each occurring only
positively. Thus, when computing \vinnfcnfpg{}, $\poslab{\vi_i}$ and $\neglab{\vi_i}$ are 
labelled with two distinct atoms  $\poslab{B_i}$ and
$\neglab{B_i}$ respectively, adding the one-way
implications $(\poslab{B_i}\imp\poslab{\vi_i})$ and
$(\neglab{B_i}\imp\neglab{\vi_i})$.
(If $\vi_i$ occurs only positively \resp{negatively} in \vi, then only
\poslab{\vi_i} \resp{\neglab{\vi_i}} occurs in \NNF{\vi}, so that only
\poslab{B_i} \resp{\neglab{B_i}} is introduced and only
$(\poslab{B_i}\imp\poslab{\vi_i})$
\resp{$(\neglab{B_i}\imp\neglab{\vi_i})$} is added.)

We remark that
with this preprocessing we maintain the
correctness and completeness of the enumeration process, because $\vi$ is
equivalent to $\NNF{\vi}$, which is equivalent to $\exists\allB.\vinnfcnfpg$.
  We remark also that we produce a linear-size CNF encoding, since %the $\NNF{\vi}$ DAG 
{$\NNF{\vi}$} has linear size w.r.t.\ $\vi$ and
\vinnfcnfpg{} has linear size w.r.t.\ \NNF{\vi} 
(see~\sref{sec:background:propositional-logic}).


We prove the following fact:
{\em %\ignoreinlong{every partial model $\mualpha$ for $\vi$ is also a model for}\ignoreinshort{
if a partial truth assignment $\muA$ suffices to satisfy $\vi$, then
it suffices to satisfy also 
$\exists\allB.\vinnfcnfpg$}.
% , that is, if $\mualpha\pmodels\vi$, then there exists $\etaB$ s.t.\
% $\mualpha\cup\etaB\pmodels\vinnfcnfpg$.
(The vice versa holds trivially.)\@
% \ignoreinlong{A complete formal proof of this fact is presented in
%     an extended version of this paper \cite{masina_cnf_2023}.
%     Intuitively, it is easy to see that the suitable $\etaB$ is defined  so that,
%     for each sub-formula $\vi_i$ of $\vi$,
%     if $\vi_i$  is made true, false or
%     is unassigned by  $\mualpha$,
%     then   \tuple{\etaB(B_i^+),\etaB(B_i^-)} is
%     %defined as
%     \tuple{\top,\bot}, \tuple{\bot,\top}, or \tuple{\bot,\bot}
%     respectively. }
%\end{rschange}
%
% \ignoreinshort{
%This is proved in the following theorem. %, which is proved in~\cref{sec:proofexistsetaB}.
\begin{theorem}%
    \label{th:existsetaB}
    Consider a formula $\vi$, and a \emph{partial} truth assignment $\muA$ such that $\muA\pmodels\vi$. 
    Then $\muA\pmodels\exists\allB.\vinnfcnfpg$.
        % Indeed, there exists a total truth assignment $\etaB$ such that $\mualpha\cup\etaB$ satisfies $\vinnfcnfpg$.
    % For every partial truth assignment $\mualpha$ s.t.\ $\mualpha\pmodels\vi(\allalpha)$, there exists a total truth assignment \etaB s.t.\ $\mualpha\cup\etaB\pmodels\vinnfcnfpg$.
\end{theorem}
% \TODO{Riscrivere come $\mualpha\pmodels\vi(\allalpha)\implies\mualpha\pmodels\vinnfcnfpg(\allalpha)$}

\begin{proof}%
    We show that, for every $\muA$ such that $\muA\pmodels\vi$,
    there exists a total truth assignment $\etaB$ such that $\muA\cup\etaB\pmodels\vinnfcnfpg$.
    We first show how such a $\etaB$ can be built, then we prove that it satisfies \vinnfcnfpg{} if conjoined with $\muA$.

    In the following, the symbol ``\any'' denotes any formula which is not in $\set{\top,\bot}$, so that ``$\residual{\vi_i}{\muA}=\any$'' means that $\vi_i$ is not assigned a truth value by $\muA$.
    
    %In order to simplify the proof, we assume that all sub-formulas occur with double polarity in $\vi$.

    For each sub-formula $\vi_i$ of $\vi$, whose positive and negative occurrences in $\vinnfcnfpg$ are associated with the variables $\poslab{B_i}$ and $\neglab{B_i}$ respectively, do:
    \begin{enumerate}[(a)]
        \item\label{item:existsetaB:true} if $\residual{\vi_i}{\muA} = \top$, and hence $\residual{\NNF{\vi_i}}{\muA}=\top$ and $\residual{\NNF{\neg\vi_i}}{\muA}=\bot$ by \cref{th:munnf}, then 
        %set $\etaB(\poslab{B_i})=\top$ and $\etaB(\neglab{B_i})=\bot$;
        add \set{\poslab{B_i},\neg\neglab{B_i}} to \etaB;
        \item\label{item:existsetaB:false} if $\residual{\vi_i}{\muA} = \bot$, and hence $\residual{\NNF{\vi_i}}{\muA}=\bot$ and $\residual{\NNF{\neg\vi_i}}{\muA}=\top$ by \cref{th:munnf}, then %set $\etaB(\poslab{B_i})=\bot$ and $\etaB(\neglab{B_i})=\top$;
         add \set{\neg\poslab{B_i},\neglab{B_i}} to \etaB;
        \item\label{item:existsetaB:star} otherwise if $\residual{\vi_i}{\muA}=\any$, then %set $\etaB(\poslab{B_i})=\etaB(\neglab{B_i})=\bot$.
         add \set{\neg\poslab{B_i},\neg\neglab{B_i}} to \etaB;
    \end{enumerate}
    (If $\vi_i$ occurs only positively or negatively, then only assign $\poslab{B_i}$ or $\neglab{B_i}$ respectively.)
    
    We prove that $\muA\cup\etaB\pmodels\vinnfcnfpg$ by induction on the structure of the formula $\vinnfcnfpg$.
    Consider a sub-formula $\vi_i$ of $\vi$:
    \begin{enumerate}[(i)]
        \item if $\vi_i$ occurs positively in $\vi$, then
        %\GMSIDENOTE{Scrivere con $\imp$ invece di $\vee$ come in \eqref{eq:rewritingTseitin} e \eqref{eq:rewritingPlaisted}?}
        \begin{equation}
                \vinnfcnfpg\defas\PlaistedCNF(
                \overbrace{\NNF{\vi}[\NNF{\vi_i}|\poslab{B_i}]}^{\subtermeqlabel{eq:proofmuB:pos:vi}} \wedge
                \overbrace{\left(\poslab{B_i}\imp\NNF{\vi_i}\right)}^{\subtermeqlabel{eq:proofmuB:pos:lab}})
        \end{equation}
        \item if $\vi_i$ occurs negatively in $\vi$, then
        \begin{equation}
            \vinnfcnfpg\defas\PlaistedCNF(
                \overbrace{\NNF{\vi}[\NNF{\neg\vi_i}|\neglab{B_i}]}^{\subtermeqlabel{eq:proofmuB:neg:vi}} \wedge
                \overbrace{\left(\neglab{B_i}\imp\NNF{\neg\vi_i}\right)}^{\subtermeqlabel{eq:proofmuB:neg:lab}})
        \end{equation}
    \end{enumerate}
    % \begin{subequations}
    % \begin{align}
    %     \vinnfcnfpg\defas\PlaistedCNF\Big(
    %         \label{eq:proof:a}&\NNF{\vi}[\NNF{\vi_i},\NNF{\neg\vi_i}/\poslab{B_i},\neglab{B_i}]
    %      \wedge\\
    %      \label{eq:proof:b}& \left(\neg \poslab{B_i}\vee\NNF{\vi_i}\right) \wedge \\
    %      \label{eq:proof:c}& \left(\neg \neglab{B_i}\vee\NNF{\neg\vi_i}\right)\Big).
    % \end{align}
    % \end{subequations}
    For each pair of cases we have:
    \begin{enumerate}[(a)]
        \item\begin{enumerate}[(i)]
            \item $\muA\cup\etaB\pmodels\eqref{eq:proofmuB:pos:vi}$ by substituting $\top$ for $\top$; $\muA\cup\etaB\pmodels\eqref{eq:proofmuB:pos:lab}$ by $\muA\pmodels\NNF{\vi_i}$;
            \item $\muA\cup\etaB\pmodels\eqref{eq:proofmuB:neg:vi}$ by substituting $\bot$ for $\bot$; $\muA\cup\etaB\pmodels\eqref{eq:proofmuB:neg:lab}$ by $\etaB\pmodels\neg\neglab{B_i}$;
        \end{enumerate}
        \item \begin{enumerate}[(i)]
                  \item $\muA\cup\etaB\pmodels\eqref{eq:proofmuB:pos:vi}$ by substituting $\bot$ for $\bot$; $\muA\cup\etaB\pmodels\eqref{eq:proofmuB:pos:lab}$ by $\etaB\pmodels\neg\poslab{B_i}$;
                  \item $\muA\cup\etaB\pmodels\eqref{eq:proofmuB:neg:vi}$ by substituting $\top$ for $\top$; $\muA\cup\etaB\pmodels\eqref{eq:proofmuB:neg:lab}$ by $\muA\pmodels\NNF{\neg\vi_i}$;
              \end{enumerate}
        \item \begin{enumerate}[(i)]
                  \item $\muA\cup\etaB\pmodels\eqref{eq:proofmuB:pos:vi}$ by substituting $\bot$ for $*$; $\muA\cup\etaB\pmodels\eqref{eq:proofmuB:pos:lab}$ by $\etaB\pmodels\neg\poslab{B_i}$;
                  \item $\muA\cup\etaB\pmodels\eqref{eq:proofmuB:neg:vi}$ by substituting $\bot$ for $*$; $\muA\cup\etaB\pmodels\eqref{eq:proofmuB:neg:lab}$ by $\etaB\pmodels\neg\neglab{B_i}$;
              \end{enumerate}
              %----------------------------------------
              % \item $\muA\cup\etaB\pmodels%\eqref{eq:proof:a}
              % (1)$ because we substitute $\top$ for $\top$ and $\bot$ for $\bot$; $\muA\cup\etaB\pmodels%\eqref{eq:proof:b}
              % (2)$ because $\muA\pmodels\NNF{\vi_i}$; $\muA\cup\etaB\pmodels%\eqref{eq:proof:c}
              % TODO$ because $\etaB\pmodels\neg\neglab{B_i}$;
              % \item $\muA\cup\etaB\pmodels%\eqref{eq:proof:a}
              % TODO$ because we substitute $\bot$ for $\bot$ and $\top$ for $\top$; $\muA\cup\etaB\pmodels%\eqref{eq:proof:b}
              % TODO$ because $\etaB\pmodels\neg\poslab{B_i}$; $\muA\cup\etaB\pmodels%\eqref{eq:proof:c}
              % TODO$ because $\muA\pmodels\NNF{\neg\vi_i}$;
              % \item $\muA\cup\etaB\pmodels%\eqref{eq:proof:a}
              % TODO$ because $\muA$ satisfies $\vi$ even without assigning a truth value to $\vi_i$, so that we can safely substitute both $\NNF{\vi_i}$ and $\NNF{\neg\vi_i}$ with $\bot$; $\muA\cup\etaB\pmodels%\eqref{eq:proof:b}
              % TODO$ because $\etaB\pmodels\neg\poslab{B_i}$;
              % $\muA\cup\etaB\pmodels%\eqref{eq:proof:c}
              % TODO$ because $\etaB\pmodels\neg\neglab{B_i}$.
    \end{enumerate}
    Therefore, $\muA\cup\etaB\pmodels\vinnfcnfpg$.
\end{proof}

% }
  
As a consequence of \cref{th:existsetaB}, given $\muA$ satisfying \vi{},  the enumerator is no more forced to assign any
  more atom in \allA to
  satisfy $\exists 
  \allB.\vinnfcnfpg$. This prevents the enumerator from producing multiple
  assignments extending \muA{}, avoiding thus the blow-up in the number
  of assignments for \TseitinCNF{(\vi)} and
  \PlaistedCNF{(\vi)} described in \sref{sec:problem}.
%
  We stress the fact that~{\Cref{th:existsetaB}} is agnostic of the %\ignoreinlong{disjoint-}
AllSAT %\ignoreinshort{\ 
(or AllSMT) %} 
algorithm adopted, %\ignoreinshort{
and that it holds for both disjoint and non-disjoint enumeration. %}.

\begin{remark}%
    \label{rem:ex3:preconv}
    Unlike with AllSAT 
or AllSMT, the pre-conversion into NNF is typically never used in plain SAT %\ignoreinshort{
    or SMT
    %}
    \emph{solving}, because it causes the unnecessary duplication of labels $\poslab{B_i}$ and $\neglab{B_i}$, with extra overhead and no benefit for the solver.
\end{remark}


  We notice that the proof of \cref{th:existsetaB} is {\em constructive},
  that is,
  it not only says that an assignment $\etaB$
  s.t.\ $\muA\cup\etaB\pmodels\vinnfcnfpg$ exists, but also it shows
  how to construct it. %, bottom-up:
  In particular, points~\ref{item:existsetaB:true},~\ref{item:existsetaB:false}, and~\ref{item:existsetaB:star} implicitly suggest a strategy for assigning the right
  values to the \allB{} atoms given \muA{}, which is based on the
  iterative applications of the following steps, interleaved with the
  assignment of values which are forced by residual constraints:
%  assign all truth values which are forced by constraints;
%  for each unassigned  \poslab{B_i} \resp{\neglab{B_i}}, 
%  if some constraint forces assigning it some truth value, do it;
    \begin{enumerate}[(a)]
        \item\label{item:rule:true} if \poslab{B_i} occurs negatively in already-satisfied
        clauses only, then add \set{\poslab{B_i},\neg\neglab{B_i}} to \etaB;
        \item\label{item:rule:false} if \neglab{B_i} occurs negatively in already-satisfied
        clauses only, then  add \set{\neglab{B_i},\neg\poslab{B_i}} to \etaB; 
        \item\label{item:rule:star} if \poslab{B_i} \resp{\neglab{B_i}} occurs negatively in a not-yet-satisfied clause whose
        other unassigned literals are all \allA{}-literals, then add \set{\neg\poslab{B_i}} \resp{\set{\neg\neglab{B_i}}} to \etaB.
    \end{enumerate}
%   (a) 
%   \\(b) 
%   \\(c) \\%
%   \\
  %\GMSIDENOTE{Usare notazione uniforme qui e nella proof: ``set $\etaB(\poslab{B_i})=\top$ and $\etaB(\neglab{B_i})=\bot$'' oppure
  %``add \set{\poslab{B_i},\neg\neglab{B_i}} to \etaB'' }
  \noindent This strategy 
  mimics  the application of points~\ref{item:existsetaB:true},~\ref{item:existsetaB:false}, and~\ref{item:existsetaB:star} in the
  proof to the sub-formulas $\poslab{\vi_i}$ and $\neglab{\vi_i}$ of \NNF{\vi}
  in a bottom-up fashion, from the leaves to the root.~%
  \footnote{In fact, we notice that \poslab{B_i} \resp{\neglab{B_i}} occurs
    negatively only in the clauses encoding the
    $(\poslab{B_i}\imp\poslab{\vi_i})$
    \resp{$(\neglab{B_i}\imp\neglab{\vi_i})$ } constraints, because by
    construction it always occurs positively elsewhere, since it
    substitutes some sub-formula \poslab{\vi_i} \resp{\neglab{\vi_i}}
    which occurs only positively in \NNF{\vi}. Also, by construction,
    we can have at most one negative \allB-literal per clause.}

We also notice that, due to the constraints
$(\poslab{B_i}\imp\poslab{\vi_i})$ and
$(\neglab{B_i}\imp\neglab{\vi_i})$ and to the fact that \poslab{\vi_i}
and \neglab{\vi_i} are mutually inconsistent by construction, no
assignment satisfying $\vinnfcnfpg$ may assign both \poslab{B_i} and
\neglab{B_i} to $\top$. 
Thus, 
to further improve the efficiency of the enumeration procedure without affecting its
outcome, we also add to \vinnfcnfpg{} the mutual-exclusion clauses $(\neg
    \poslab{B_i}\vee\neg\neglab{B_i})$ when both $\poslab{B_i}$ and
$\neglab{B_i}$ are introduced, which prevent the solver from assigning
both $\poslab{B_i}$ and $\neglab{B_i}$ to $\top$, and thus from
wasting time in exploring inconsistent search sub-trees for residual
formulas like $(\ldots\wedge\poslab{\vi_i}\wedge\neglab{\vi_i}\wedge\ldots)$. 


 We illustrate the benefit of our proposed technique with the following example.

%\newpage
\begin{example}%
  \label{ex3}
    Consider the formula $\vi$ of~\cref{ex1}. By converting it into NNF, we obtain:
    \begin{align*}
        &\NNF{\vi}\defas\\
        &\overbrace{(A_1\wedge A_2)}^{\poslab{B_1}}\vee
        \overbrace{(
        \overbrace{(
        \overbrace{(
        \overbrace{(\neg A_3\wedge\neg A_4)}^{\neglab{B_2}}\vee
        \overbrace{(\neg A_5\wedge\neg A_6)}^{\neglab{B_3}}
        )}^{\neglab{B_4}}\vee A_7
        )}^{\poslab{B_5}} \wedge
        \overbrace{(
        \overbrace{(
        \overbrace{(     A_3\vee       A_4)}^{\poslab{B_2}}\wedge
        \overbrace{(     A_5\vee       A_6)}^{\poslab{B_3}}
        )}^{\poslab{B_4}}\vee\neg A_7
        )}^{\poslab{B_6}}
        )}^{\poslab{B_7}}
    \end{align*}  
    % We remark that, by definition, each sub-formula of $\vinnf$ occurs only with positive polarity. For instance, consider the sub-formula $(A_3\vee A_4)$ that occurs in $\vi$ with double polarity. In $\vinnf$, the positive occurrence remains the same, while the negative occurrence is converted into $(\neg A_3\wedge\neg A_4)$. This implies that the two occurrences correspond to two different sub-formula, each occurring only positively, and thus they will be labeled with two different atoms.
\noindent
Applying \PlaistedCNF{} and adding the mutual-exclusion clauses we obtain the CNF formula:
%Suppose, then, that the formula is converted into CNF using \PlaistedCNF{}. Then, the following CNF formula is obtained:
    \begin{subequations}%
        \label{eq:ex3:vicnf}
        \begin{alignat}{2}
             & \vinnfcnfpg\defas                                                                                                                                                                                                \\
             & \qquad(\neg \poslab{B_1}\vee\pos A_1)\wedge(\neg \poslab{B_1}\vee\pos A_2)                                     & \wedge
             & \quad\eqcomment{(\poslab{B_1}\imp (\pos A_1\wedge\pos A_2))}\label{eq:ex3:vicnf:line1}                                                                                                                                    \\
             & \qquad(\neg \neglab{B_2}\vee\neg A_3)\wedge(\neg \neglab{B_2}\vee\neg A_4)                   & \wedge & \quad\eqcomment{(\neglab{B_2}\imp (\neg A_3\wedge\neg A_4))}\label{eq:ex3:vicnf:line2}                   \\
             & \qquad(\neg \neglab{B_3}\vee\neg A_5)\wedge(\neg \neglab{B_3}\vee\neg A_6)                   & \wedge & \quad\eqcomment{(\neglab{B_3}\imp (\neg A_5\wedge\neg A_6))}\label{eq:ex3:vicnf:line3}                   \\
             & \qquad(\neg \neglab{B_4}\vee\pos \neglab{B_2}\vee\pos \neglab{B_3})                          & \wedge & \quad\eqcomment{(\neglab{B_4}\imp (\pos \neglab{B_2}\vee\pos \neglab{B_3}))}\label{eq:ex3:vicnf:line4}   \\
             & \qquad(\neg \poslab{B_5}\vee\pos \neglab{B_4}\vee\pos A_7)                                            & \wedge & \quad\eqcomment{(\poslab{B_5}\imp (\pos \neglab{B_4}\vee\pos A_7))}\label{eq:ex3:vicnf:line5}                     \\
             & \qquad(\neg \poslab{B_2}\vee\pos A_3\vee\pos A_4)                                            & \wedge & \quad\eqcomment{(\poslab{B_2}\imp (\pos A_3\vee\pos A_4))}\label{eq:ex3:vicnf:line6}                     \\
             & \qquad(\neg \poslab{B_3}\vee\pos A_5\vee\pos A_6)                                            & \wedge & \quad\eqcomment{(\poslab{B_3}\imp (\pos A_5\vee\pos A_6))}\label{eq:ex3:vicnf:line7}                     \\
             & \qquad(\neg \poslab{B_4}\vee\pos \poslab{B_2})\wedge(\neg \poslab{B_4}\vee\pos \poslab{B_3}) & \wedge & \quad\eqcomment{(\poslab{B_4}\imp (\pos \poslab{B_2}\wedge\pos \poslab{B_3}))}\label{eq:ex3:vicnf:line8} \\
             & \qquad(\neg \poslab{B_6}\vee\pos \poslab{B_4}\vee\neg A_7)                                            & \wedge & \quad\eqcomment{(\poslab{B_6}\imp (\pos \poslab{B_4}\vee\neg A_7))}\label{eq:ex3:vicnf:line9}                     \\
             & \qquad(\neg \poslab{B_7}\vee\pos \poslab{B_5})\wedge(\neg \poslab{B_7}\vee\pos \poslab{B_6})                                 & \wedge & \quad\eqcomment{(\poslab{B_7}\imp (\pos \poslab{B_5}\wedge\pos \poslab{B_6}))}\label{eq:ex3:vicnf:line10}                         \\
             & \qquad(\pos \poslab{B_1}\vee\pos \poslab{B_7})                                                               & \wedge & \label{eq:ex3:vicnf:line11}                                                                              \\
             & \qquad(\neg \poslab{B_2}\vee \neg \neglab{B_2})
             & \wedge & \quad\eqcomment{\textit{mutual exclusion for \poslab{B_2}, \neglab{B_2}}}                                                                                                         \\
             & \qquad(\neg \poslab{B_3}\vee \neg \neglab{B_3})                                              & \wedge &  \quad\eqcomment{\textit{mutual exclusion for \poslab{B_3}, \neglab{B_3}}}                                                                                                         \\
             & \qquad(\neg \poslab{B_4}\vee \neg \neglab{B_4})                                              &&\quad\eqcomment{\textit{mutual exclusion for \poslab{B_4}, \neglab{B_4}}} 
        \end{alignat}
      \end{subequations}
\noindent
Notice that the labels
\neglab{B_1}, \neglab{B_5}, \neglab{B_6}, \neglab{B_7} and their
respective one-way constraints are not
introduced, because there is no negative occurrence of the respective
sub-formulas in \vi{}.
      
    As in \cref{ex1,ex2}, consider the 
    partial assignment $\muA\defas\set{\neg A_3,\neg A_4,\neg A_7}$~\eqref{eq:ex1:muA} which suffices to satisfy $\vi{}$.
First, it is easy to see that \muA{} suffices to satisfy also \NNF{\vi}, in
compliance with \cref{th:munnf}.
      Also, \muA{} suffices to satisfy
      $\exists\allB.\vinnfcnfpg$, because
      $\muA\cup\etaB\pmodels\vinnfcnfpg$ where
      $\etaB\defas\set{\neg \poslab{B_1}, \neg
        \poslab{B_2},\neglab{B_2},\neg \poslab{B_3},\neg
        \neglab{B_3},\neg
        \poslab{B_4},\neglab{B_4},\poslab{B_5},\poslab{B_6},\poslab{B_7}}$.

The above assignment can be produced by adopting the strategy
described above. First, we assign  the literals in \muA, which force 
to add  also
$\set{\neg \poslab{B_2}, \neg \poslab{B_4}}$ due to 
\eqref{eq:ex3:vicnf:line6} and 
\eqref{eq:ex3:vicnf:line8} respectively. 
Then we add \set{\poslab{B_6}} by step~\ref{item:rule:true} on~\eqref{eq:ex3:vicnf:line9},
\set{\neglab{B_2}} by step~\ref{item:rule:false} on~\eqref{eq:ex3:vicnf:line2}, and
\set{\neg\poslab{B_1},\neg\neglab{B_3},\neg\poslab{B_3}} by step~\ref{item:rule:star}
on \eqref{eq:ex3:vicnf:line1}, \eqref{eq:ex3:vicnf:line3},
\eqref{eq:ex3:vicnf:line7} respectively.
These force to add \set{\poslab{B_7}} and \set{\poslab{B_5}} due to 
\eqref{eq:ex3:vicnf:line11} and \eqref{eq:ex3:vicnf:line10}
respectively.
Finally, we add \set{\neglab{B_4}} by step~\ref{item:rule:false} on
\eqref{eq:ex3:vicnf:line4}.
Overall, this corresponds to enumerate only the partial assignment $\muA{}$
satisfying $\exists\allB.\vinnfcnfpg$, with both disjoint and
non-disjoint enumeration:
\begin{equation*}
  \label{eq:ex3:final}
  \set{\neg A_3, \neg A_4, \neg A_5}  \quad\eqcomment{\set{\neg \poslab{B_1}, \neg
        \poslab{B_2},\neglab{B_2},\neg \poslab{B_3},\neg
        \neglab{B_3},\neg
        \poslab{B_4},\neglab{B_4},\poslab{B_5},\poslab{B_6},\poslab{B_7}}}.
        \hfill\exdone{}
\end{equation*}

\end{example}

%\ignoreinshort{
% }

Notice that, the shorter is $\muA$ w.r.t.\ a total assignment, the higher is
the chance that \TseitinCNF{} and \PlaistedCNF{} force the
production of a high
number of extra assignments, the more beneficial is the usage of
\NNFPlaisted{} which avoids it.
This said, if the enumerator is able to produce short partial
assignments \muA{}
satisfying the formula, we expect a high benefit from using
\NNFPlaisted{} 
instead of 
\TseitinCNF{} and \PlaistedCNF{}; vice versa, if the enumerator
produces only total or nearly-total assignments, we expect no or
very-limited benefit respectively. 

Implementation-wise, the strategy to assign the values of \etaB{}
described above is difficult to implement inside the current
enumerators.
Therefore, there is no formal guarantee that a generic
enumeration procedure always finds exactly the $\etaB$ which prevents the
generation of longer assignments.
For example, the enumeration procedure of
\sref{sec:background:allsat} finds a total truth assignment
$\etaA\cup\etaB$ that satisfies the formula, and then finds
$\muA\subseteq\etaA$ that is minimal w.r.t.\ that specific $\etaB$ such
that $\muA\cup\etaB\pmodels\vinnfcnfpg$, so that the $\etaB$ found is not guaranteed to be the one that allows for the most effective minimization of $\muA$. %}
Ad-hoc enumeration heuristics could be %\ignoreinlong{investigated}\ignoreinshort{
adopted.
%
Nevertheless, in \sref{sec:experiments} we show that a
very simple heuristic ---i.e., force the assignments of false values first
to decision atoms--- suffices to guarantee dramatic improvements w.r.t.\
previous approaches using two state-of-the-art AllSAT/AllSMT enumerators.
%\RSTODO{RISCRITTO FIN QUI} %%%%%%%%%%%%%%%%%%%%%%%%%%%%%%5



