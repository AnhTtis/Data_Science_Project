% \begin{gmchange}
\section{Proofs}
\subsection{Proof for~\cref{th:nnfdaglinear}  in~\cref{sec:background:propositional-logic}}\label{sec:proofnnfdaglinear}%
%
\begin{figure}[th]
    \centering
    \begin{subfigure}[t]{0.48\textwidth}
        \begin{tikzpicture}[remember picture,->,auto,node distance=1.8cm,semithick]
  \begin{footnotesize}
    % First layer
    \node[circle, draw] (A) {$\wedge$};
    \node[circle, draw] (B) [right of=A] {$\vee$};

    % Second layer
    \node[circle, draw] (D) [below=.4cm of A] {\phantom{$\vee$}};
    \node[circle, draw] (C) [left of=D] {\phantom{$\vee$}};
    \node[circle, draw] (E) [right of=D] {\phantom{$\vee$}};
    \node[circle, draw] (F) [right of=E] {\phantom{$\vee$}};


    % % Edges
    \path (A) edge node {} (C)
    (A) edge node {} (E)
    (B) edge node {} (D)
    (B) edge node {} (F);

    % Text labels
    \node[above left=.4cm and -.8cm of A] at (A) {$\NNF{\vi_1\wedge\vi_2}$};
    \node[above right=.4cm and -.8cm of B] at (B) {$\NNF{\neg(\vi_1\wedge\vi_2)}$};
    \node[below=.4cm of C]                at (C) {$\NNF{\vi_1}$};
    \node[below=.4cm of D]                at (D) {$\NNF{\neg\vi_1}$};
    \node[below=.4cm of E]                at (E) {$\NNF{\vi_2}$};
    \node[below=.4cm of F]                at (F) {$\NNF{\neg\vi_2}$};

    \node[draw,densely dotted,red,fit=(A) (B)] {};
    \node[draw,densely dotted,red,fit=(C) (D)] {};
    \node[draw,densely dotted,red,fit=(E) (F)] {};
  \end{footnotesize}
\end{tikzpicture}
        \caption{Case $\vi\defas\vi_1\wedge\vi_2$.}%
        \label{fig:2root:and}
    \end{subfigure}
    \begin{subfigure}[t]{0.48\textwidth}
        \begin{tikzpicture}[remember picture,->,auto,node distance=1.8cm,semithick]
  \begin{footnotesize}
    % First layer
    \node[circle, draw] (A) {$\wedge$};
    \node[circle, draw] (B) [right of=A] {$\wedge$};

    % Second layer
    \node[circle, draw] (D) [below=.4cm of A] {$\vee$};
    \node[circle, draw] (C) [left of=D] {$\vee$};
    \node[circle, draw] (E) [right of=D] {$\vee$};
    \node[circle, draw] (F) [right of=E] {$\vee$};

    % % Third layer
    \node[circle, draw] (G) [below=.6cm of C] {\phantom{$\vee$}};
    \node[circle, draw] (H) [right of=G] {\phantom{$\vee$}};
    \node[circle, draw] (I) [right of=H] {\phantom{$\vee$}};
    \node[circle, draw] (J) [right of=I] {\phantom{$\vee$}};

    % % Edges
    \path (A) edge node {} (C)
    (A) edge node {} (D)
    (B) edge node {} (E)
    (B) edge node {} (F)
    (C) edge node {} (G)
    (C) edge node {} (J)
    (D) edge node {} (H)
    (D) edge node {} (I)
    (E) edge node {} (G)
    (E) edge node {} (I)
    (F) edge node {} (H)
    (F) edge node {} (J);

    % Text labels
    \node[above left=.4cm and -.8cm of A] at (A) {$\NNF{\vi_1\iff\vi_2}$};
    \node[above right=.4cm and -.8cm of B] at (B) {$\NNF{\neg(\vi_1\iff\vi_2)}$};
    \node[below=.4cm of G]                at (G) {$\NNF{\vi_1}$};
    \node[below=.4cm of H]                at (H) {$\NNF{\neg\vi_1}$};
    \node[below=.4cm of I]                at (I) {$\NNF{\vi_2}$};
    \node[below=.4cm of J]                at (J) {$\NNF{\neg\vi_2}$};

    \node[draw,densely dotted,red,fit=(A) (B)] {};
    \node[draw,densely dotted,red,fit=(G) (H)] {};
    \node[draw,densely dotted,red,fit=(I) (J)] {};
  \end{footnotesize}
\end{tikzpicture}
        \caption{Case $\vi\defas\vi_1\iff\vi_2$.}%
        \label{fig:2root:iff}
    \end{subfigure}
    %    \vspace{-.3cm}
    \caption{2-root DAGs for the pair $\langle{\NNF{\vi}},\NNF{\neg\vi}\rangle$ for $\vi\defas\vi_1\wedge\vi_2$ and $\vi\defas\vi_1\iff\vi_2$ (those for $\vi_1\vee\vi_2$ and $\vi_1\imp\vi_2$ are similar to that of $\vi_1\wedge\vi_2$).}%
    \label{fig:2root}
\end{figure}
%We present the proof for~\cref{th:nnfdaglinear} in~\cref{sec:background:propositional-logic}.
\begin{proof}
    %The NNF DAG that represents 
    \NNF{\vi} is a sub-graph of the 2-root DAG for the
        pair $\langle{\NNF{\vi}},\NNF{\neg\vi}\rangle$. %\GMCHANGE{
    %}. 
    We prove that the latter grows linearly in size w.r.t.\ $\vi$ by reasoning
    inductively on the structure of \vi. (The size of a DAG ``$\dots$'' is denoted
    with ``$|\dots|$''.)
    %i.e.\ ``|\dots|'' is \#nodes(\dots)+\#arcs(\dots).
    %    We prove it by induction on the structure of \vi.
    \begin{description}
        \item[if $\vi$ is an atom:] $\NNF{\vi}=\vi$ and $\NNF{\neg\vi}=\neg\vi$, so that:
              % \begin{equation}\label{eq:recursizebase}
              $|\langle\NNF{\vi},\NNF{\neg\vi}\rangle|=2$.
              % \end{equation}
        \item[if $\vi\defas\neg\vi_1$:] we assume by induction that
              %we have computed
              $|\tuple{\NNF{\vi_1},\NNF{\neg\vi_1}}|$ is linear in $|\vi_1|$.
              Then
              $\tuple{\NNF{\vi},\NNF{\neg\vi}}=\tuple{\NNF{\neg\vi_1},\NNF{\vi_1}}$
              (i.e., we just invert the order), so that
              \begin{equation}\label{eq:recursizenot}
                  |\tuple{\NNF{\vi},\NNF{\neg\vi}}|=|\tuple{\NNF{\neg\vi_1},\NNF{\vi_1}}|=|\tuple{\NNF{\vi_1},\NNF{\neg\vi_1}}|
              \end{equation}
              % $|\langle\NNF{\neg\vi_1},\NNF{\vi_1}\rangle|=|\langle\NNF{\vi_1},\NNF{\neg\vi_1}\rangle|$.
              %        \item[if $\vi\defas(\vi_1\bowtie\vi_2)$ s.t.\ $\bowtie\ \in\set{\wedge,\vee,\imp,\iff}$:]
        \item[if $\vi\defas(\vi_1\bowtie\vi_2)$ s.t.\ $\bowtie\ \in\set{\wedge,\iff}$:] we
              assume by induction that
              %we have computed
              $|\tuple{\NNF{\vi_1},\NNF{\neg\vi_1}}|$ and
                  $|\tuple{\NNF{\vi_2},\NNF{\neg\vi_2}}|$ are linear in $|\vi_1|$ and $|\vi_2|$, respectively.
              (See \Cref{fig:2root}):
              % Specifically:
                  \begin{description}
                      \item[if $\bowtie$ is $\wedge$:] the DAGs for $\NNF{\vi}, \NNF{\neg\vi}$ add 2
                            ``$\wedge/\vee$'' nodes and $2+2$ arcs:\\ $
                                \begin{aligned}
                                    \NNF{\pos\phantom{(}\vi_1\wedge\vi_2\phantom{)}} & \defas\NNF{\pos\vi_1}\wedge\NNF{\pos\vi_2}\text{ and} \\
                                    \NNF{\neg(\vi_1\wedge\vi_2)}                     & \defas\NNF{\neg\vi_1}\vee\NNF{\neg\vi_2}\text{, thus}
                                \end{aligned}
                            $
                            {\setlength{\mathindent}{-.4cm}%
                                    \begin{equation}\label{eq:recursizeand}
                                        |\tuple{\NNF{\vi},\NNF{\neg\vi}}| =
                                        6+|\tuple{\NNF{\vi_1},\NNF{\neg\vi_1}}|+
                                        |\tuple{\NNF{\vi_2},\NNF{\neg\vi_2}}|
                                    \end{equation}%
                                }
                      \item[if $\bowtie$ is $\iff$:] the DAGs for $\NNF{\vi}, \NNF{\neg\vi}$ %share the sub-DAGs for
                            %$\NNF{\vi_1}$, $\NNF{\neg\vi_1}$, $\NNF{\vi_2}$, $\NNF{\neg\vi_2}$, adding
                            add 3+3 ``$\wedge$''/``$\vee$'' nodes and $6+6$ arcs:\\ $
                                \begin{aligned}
                                    \NNF{\pos\phantom{(}\vi_1\iff\vi_2\phantom{)}} & \defas(\NNF{\neg\vi_1}\vee\NNF{\vi_2})\wedge(\NNF{\pos\vi_1}\vee \NNF{\neg\vi_2})\text{ and}  \\
                                    \NNF{\neg(\vi_1\iff\vi_2)}                     & \defas(\NNF{\pos\vi_1}\vee\NNF{\vi_2})\wedge(\NNF{\neg\vi_1}\vee\NNF{\neg\vi_2})\text{, thus}
                                \end{aligned}
                            $
                            {\setlength{\mathindent}{-.4cm}%
                                    \begin{equation}\label{eq:recursizeiff}
                                        |\tuple{\NNF{\vi},\NNF{\neg\vi}}| =
                                        18+|\tuple{\NNF{\vi_1},\NNF{\neg\vi_1}}|+
                                        |\tuple{\NNF{\vi_2},\NNF{\neg\vi_2}}|
                                    \end{equation}%
                                }
                  \end{description}
                  % Then:
                  % \begin{equation}\label{eq:recursize}
                  %     |\langle \NNF{\vi},\NNF{\neg\vi}\rangle| \leq
                  %     18+|\langle \NNF{\vi_1},\NNF{\neg\vi_1}\rangle|+|\langle
                  %     \NNF{\vi_2},\NNF{\neg\vi_2}\rangle|:
                  % \end{equation}
        \item[if $\vi\defas(\vi_1\bowtie\vi_2)$ s.t.\ $\bowtie\ \in\set{\vee,\imp}$:]
              %          \item[if $\bowtie\ \in \set{\vee, \imp}$:]
              these cases can be reduced to the previous cases, since $\NNF{\vi_1\vee\vi_2}=
                  \NNF{\neg(\neg\vi_1\wedge\neg\vi_2)}$ and $\NNF{\vi_1\imp\vi_2}=
                  \NNF{\neg(\vi_1\wedge\neg\vi_2)}$.
    \end{description}
    Therefore, from~\eqref{eq:recursizenot},~\eqref{eq:recursizeand} and~\eqref{eq:recursizeiff} we have
        that $|\langle \NNF{\vi},\NNF{\neg\vi}\rangle|$ is $O(|\vi|)$.
    %18\cdot|\vi|$.}
    %
    % RS; gia' detto nel main part del paper
    % Intuitively, we only need at most 2 nodes for each sub-formula $\vi_i$ of $\vi$,
    % representing $\NNF{\vi_i}$ and $\NNF{\neg\vi_i}$ for positive and negative
    % occurrences of $\vi_i$ respectively. These nodes are shared among up
    % to exponentially-many branches generated by expanding the nested iffs.
\end{proof}
%\end{gmchange}

%%%%%%%%%%%%%%%%%%%%%%%%%%%%%%%%%%%%%%%%%%%%%%%%%%%%%%%%%%%%%
%%%
%%%%%%%%%%%%%%%%%%%%%%%%%%%%%%%%%%%%%%%%%%%%%%%%%%%%%%%%%%%%%
\newpage
%\begin{gmchange}
\newcommand{\muof}[1]{\residual{#1}{\mu}}
\subsection{Proof for~\cref{th:munnf} in~\cref{sec:background:propositional-logic}}\label{sec:proofmunnf}%

\begin{figure}[th]
  % \footnotesize
  \small
  % \centering
  %  \begin{tabular}{cc}
  %  \begin{minipage}[t]{0.49\textwidth}
  \centering
  $
    \begin{array}{||c||l|l|l|l|l|l|l|l|l||}
      \hline
      \muof{\vi_1}                   & \top & \top & \top & \any & \any & \any & \bot & \bot & \bot \\
      \muof{\vi_2}                   & \top & \any & \bot & \top & \any & \bot & \top & \any & \bot \\
      \hline
      \neg(\muof{\vi_1})             & \bot & \bot & \bot & \any & \any & \any & \top & \top & \top \\
      \muof{\vi_1}\wedge\muof{\vi_2} & \top & \any & \bot & \any & \any & \bot & \bot & \bot & \bot \\
      \muof{\vi_1}\vee\muof{\vi_2}   & \top & \top & \top & \top & \any & \any & \top & \any & \bot \\
      \muof{\vi_1}\imp\muof{\vi_2}   & \top & \any & \bot & \top & \any & \any & \top & \top & \top \\
      \muof{\vi_1}\iff\muof{\vi_2}   & \top & \any & \bot & \any & \any & \any & \bot & \any & \top \\
      \hline
    \end{array}
  $
  \caption{\label{fig:threeval}
    Three-value-semantics of $\muof{\vi}$ in terms of \set{\top ,\bot ,\any }
    (``true'', ``false'', ``unknown''). \newline
    % \ignoreinlong{The definition of
    %   $\muof{\vi_1\bowtie\vi_2}$ s.t. $\bowtie\ \in\set{\vee,\imp,\limp,\iff}$
    %   follows straighforwardly.}
  }
  %  \end{minipage}
  \ignore{
    &
    \begin{minipage}[t]{0.4\textwidth}
      %\centering
      $
        \begin{array}{|cl|cl|}
          \hline
          \neg\top                     & \Rightarrow \bot & \neg \bot                     & \Rightarrow \top    \\
          \top\wedge\vi, \vi\wedge\top & \Rightarrow \vi  & \bot\wedge\vi,  \vi\wedge\bot & \Rightarrow \bot    \\
          \top\vee\vi, \vi\vee\top     & \Rightarrow \top & \bot\vee\vi, \vi\vee\bot      & \Rightarrow \vi     \\
          \top\imp\vi                  & \Rightarrow \vi  & \bot\imp\vi                   & \Rightarrow \top    \\
          \vi\imp\top                  & \Rightarrow \top & \vi\imp\bot                   & \Rightarrow \neg\vi \\
          \top\iff\vi, \vi\iff\top     & \Rightarrow \vi  & \bot\iff\vi, \vi\iff\bot      & \Rightarrow \neg\vi \\
                                       &                  &                               &                     \\
          \hline
        \end{array}
      $\\
      \caption{\label{fig:boolprop}
        Propagation of truth values
        through the Boolean connectives.}
    \end{minipage}%
  }
  %\end{tabular}
\end{figure}

In the following the symbol ``\any'' denotes any formula which is not
in $\set{\top,\bot}$. Following~\citeA{sebastianiAreYouSatisfied2020}, we adopt a 3-value
semantics for residuals $\residual{\vi}{\mu}\in\set{\top,\bot,\any}$,
so that ``$\residual{\vi}{\mu}=\any$'' means
``$\residual{\vi}{\mu}\not\in\set{\top,\bot}$'' and
``$\residual{\vi_1}{\mu}=\residual{\vi_2}{\mu}$'' means that the two
residuals $\residual{\vi_1}{\mu}$ and $\residual{\vi_2}{\mu}$ are either both $\top$, or both $\bot$, or neither is in
\set{\top,\bot}. (Notice that, in the latter case,
$\residual{\vi_1}{\mu}=\residual{\vi_2}{\mu}$ even if
$\residual{\vi_1}{\mu}$ and $\residual{\vi_2}{\mu}$ are different
formulas.)
We extend the definition to tuples in an obvious way:
$\tuple{\residual{\vi_1}{\mu},\dots,\residual{\vi_n}{\mu}}=
    \tuple{\residual{\psi_1}{\mu},\dots,\residual{\psi_n}{\mu}}$ iff
$\residual{\vi_i}{\mu}=\residual{\psi_i}{\mu}$ for each $i\in[1\dots n]$.

The 3-value semantics of the Boolean operators is reported for convenience in
Figure~\ref{fig:threeval}. As a straightforward consequence of the above
semantics, we have that:
\begin{eqnarray*}
    \residual{(\neg\vi)}{\mu}&=&\neg(\residual{\vi}{\mu})\ \  \mbox{(hereafter
        simply ``$\residual{\neg\vi}{\mu}$'')}\\
    \residual{(\vi_1\bowtie\vi_2)}{\mu}&=&
    \residual{\vi_1}{\mu}\bowtie\residual{\vi_2}{\mu}\ \  \mbox{for $\bowtie\ \in \set{\wedge,\vee,\imp,\iff}$}.
\end{eqnarray*}
Also, the usual transformations apply:
$\muof{\neg(\vi_1\wedge\vi_2)}=\muof{\neg\vi_1}\vee\muof{\neg\vi_2}$,
$\muof{\neg(\vi_1\vee\vi_2)}=\muof{\neg\vi_1}\wedge\muof{\neg\vi_2}$,
$\muof{(\vi_1\iff\vi_2)}=
    (\neg\muof{\vi_1}\vee\muof{\vi_2})\wedge(\muof{\vi_1}\vee\neg\muof{\vi_2})$
and
$\muof{\neg(\vi_1\iff\vi_2)}=
    (\muof{\vi_1}\vee\muof{\vi_2})\wedge(\neg\muof{\vi_1}\vee\neg\muof{\vi_2})$.
%
For convenience, sometimes we denote as $\bar{v}$ the complement of
$v\in\set{\top,\bot,\any}$, i.e., $\bar{v}\defas\neg v$, so that
$\bar{\top}=\bot, \bar{\bot}=\top, \bar{\any}=\any$.

We prove the following lemma, from which \cref{th:munnf}
in~\cref{sec:background:propositional-logic} follows directly.

% \ignore{\begin{property}%
%     \label{th:munnf2}
%     Consider a Boolean formula $\vi(\allA)$, and let $\NNF{\vi}$ be its NNF DAG. 
%     Consider a partial assignment $\muA$ on $\allA$. Then
%     $\residual{\vi}{\muA}=v$ iff $\residual{\NNF{\vi}}{\muA}=v$ for $v\in\set{\top,\bot,\any}$.
% \end{property}}

\begin{lemma}
    Consider a %\ignoreinlong{Boolean }
    formula $\vi$, %and let $\NNF{\vi}$ be its NNF DAG.\@ 
    and a partial assignment $\mu$. % on $\allA$.
    % Let $\langle\NNF{\vi},\NNF{\neg\vi}\rangle$ be the 2-root DAG as
    % in~\cref{sec:proofnnfdaglinear}.
    Then:
    \begin{equation}
        \label{eq:nnfeq}
        \pair{\residual{\vi}{\mu}}{\residual{\neg\vi}{\mu}}=\pair{\residual{\NNF{\vi}}{\mu}}{\residual{\NNF{\neg\vi}}{\mu}}.
    \end{equation}

\end{lemma}

\begin{proof}%
    % RS: spostato sopra
    % We first notice that $\residual{\vi}{\mu}=v$ iff
    % $\residual{\neg\vi}{\mu}=\bar{v}$, where $\bar{v}$ is the
    % complement of $v$, i.e.\ $\bar{\top}=\bot, \bar{\bot}=\top,
    % \bar{\any}=\any$.
    As in~\cref{sec:proofnnfdaglinear}, we prove this fact by reasoning on the
    2-root DAG for the pair $\langle\NNF{\vi},\NNF{\neg\vi}\rangle$. Specifically,
    we prove~\eqref{eq:nnfeq} by induction on the structure of $\vi$.

    %% RS: spostato a sopra
    % (With a little abuse of notation, we say that $\residual{\vi}{\mu}=\residual{\psi}{\mu}$ even in the case when $\residual{\vi}{\mu}$ and $\residual{\psi}{\mu}$ are different formulas s.t.\ $\residual{\vi}{\mu},\residual{\psi}{\mu}\notin\set{\top,\bot}$, meaning that they both fall in the case $\any$.)
    \begin{description}
        \item[if $\vi$ is an atom:] then $\NNF{\vi}=\vi$ and $\NNF{\neg\vi}=\neg\vi$, so that
              $\residual{\vi}{\mu}=\residual{\NNF{\vi}}{\mu}$ and
              $\residual{\neg\vi}{\mu}=\residual{\NNF{\neg\vi}}{\mu}$.
              %------------------- NOT -------------------
        \item[if $\vi\defas\neg\vi_1$:] we assume by induction that
              $\pair{\residual{\vi_1}{\mu}}{\residual{\neg\vi_1}{\mu}}=\pair{\residual{\NNF{\vi_1}}{\mu}}{\residual{\NNF{\neg\vi_1}}{\mu}}$.
                  {Let $v$ be s.t.
                      $\pair{\residual{\vi}{\mu}}{\residual{\neg\vi}{\mu}}=\pair{v}{\bar{v}}$. Then:}
              \[
                  \begin{aligned}
                      \pair{\residual{\neg\vi_1}{\mu}}{\residual{\pos\vi_1}{\mu}}=\pair{v}{\bar{v}}             & \Iff    \\
                      \pair{\residual{\pos\vi_1}{\mu}}{\residual{\neg\vi_1}{\mu}}=\pair{\bar{v}}{v}             & \Iffind \\
                      \pair{\residual{\NNF{\pos\vi_1}}{\mu}}{\residual{\NNF{\neg\vi_1}}{\mu}}=\pair{\bar{v}}{v} & \Iff    \\
                      \pair{\residual{\NNF{\neg\vi_1}}{\mu}}{\residual{\NNF{\pos\vi_1}}{\mu}}=\pair{v}{\bar{v}} &
                  \end{aligned}
              \]

              % then $\langle\residual{\neg\vi_1}{\mu},\residual{\vi_1}{\mu}\rangle=\langle v,\bar{v}\rangle$ iff $\langle\residual{\vi_1}{\mu},\residual{\neg\vi_1}{\mu}\rangle=\langle\bar{v},v\rangle$. By induction hypothesis, this holds iff $\langle\residual{\NNF{\vi_1}}{\mu},\residual{\NNF{\neg\vi_1}}{\mu}\rangle=\langle\bar{v},v\rangle$, and so iff $\langle\residual{\NNF{\vi}}{\mu},\residual{\NNF{\neg\vi}}{\mu}\rangle=\langle v,\bar{v}\rangle$;%, hence $\residual{\NNF{\vi}}{\mu}=v$ and $\residual{\NNF{\neg\vi}}{\mu}=\bar{v}$. 
              %        \item[if $\vi\defas(\vi_1\bowtie\vi_2)$:] s.t.\ $\bowtie~\in\set{\wedge,\vee,\imp,\iff}$. We assume by induction that
        \item[if $\vi\defas(\vi_1\bowtie\vi_2)$:] s.t.\ $\bowtie~\in\set{\wedge,\iff}$. We
              assume by induction that
              \[
                  \begin{aligned}
                      \pair{\residual{\vi_1}{\mu}}{\residual{\neg\vi_1}{\mu}}=\pair{\residual{\NNF{\vi_1}}{\mu}}{\residual{\NNF{\neg\vi_1}}{\mu}}, \\
                      \pair{\residual{\vi_2}{\mu}}{\residual{\neg\vi_2}{\mu}}=\pair{\residual{\NNF{\vi_2}}{\mu}}{\residual{\NNF{\neg\vi_2}}{\mu}}.
                  \end{aligned}
              \]
              Then,
              \begin{description}
                  %--------------- AND ------------------------
                  \item[if $\bowtie$ is $\wedge$:]
                        \[
                            \begin{aligned}[t]
                                \pair{\residual{(\vi_1\wedge\vi_2)}{\mu}}{\residual{\neg(\vi_1\wedge\vi_2)}{\mu}}                                                       & =      \\
                                %\pair{\residual{\vi_1}{\mu}\wedge\residual{\vi_2}{\mu}}{\neg(\residual{\vi_1}{\mu}\wedge\residual{\vi_2}{\mu})}=
                                \pair{\residual{\vi_1}{\mu}\wedge\residual{\vi_2}{\mu}}{\residual{\neg\vi_1}{\mu}\vee\residual{\neg\vi_2}{\mu}}                         & \eqind \\
                                \pair{\residual{\NNF{\vi_1}}{\mu}\wedge\residual{\NNF{\vi_2}}{\mu}}{\residual{\NNF{\neg\vi_1}}{\mu}\vee\residual{\NNF{\neg\vi_2}}{\mu}} & =      \\
                                \pair{\residual{(\NNF{\vi_1}\wedge\NNF{\vi_2})}{\mu}}{\residual{(\NNF{\neg\vi_1}\vee\NNF{\neg\vi_2})}{\mu}}                             & =      \\
                                \pair{\residual{\NNF{\vi_1\wedge\vi_2}}{\mu}}{\residual{\NNF{\neg(\vi_1\wedge\vi_2)}}{\mu}}                                             &
                            \end{aligned}
                        \]
                        %     %--------------- OR ------------------------
                        % \item[if $\bowtie$ is $\vee$:]
                        %     \[
                        %         \begin{aligned}[t]
                        %             \langle\residual{(\vi_1\vee\vi_2)}{\mu},\residual{\neg(\vi_1\vee\vi_2)}{\mu}\rangle=
                        %             %\langle\residual{\vi_1}{\mu}\vee\residual{\vi_2}{\mu},\neg(\residual{\vi_1}{\mu}\vee\residual{\vi_2}{\mu})\rangle=
                        %             \langle\residual{\vi_1}{\mu}\vee\residual{\vi_2}{\mu},\residual{\neg\vi_1}{\mu}\wedge\residual{\neg\vi_2}{\mu}\rangle                         & \eqind \\
                        %             \langle\residual{\NNF{\vi_1}}{\mu}\vee\residual{\NNF{\vi_2}}{\mu},\residual{\NNF{\neg\vi_1}}{\mu}\wedge\residual{\NNF{\neg\vi_2}}{\mu}\rangle & =      \\
                        %             \langle\residual{(\NNF{\vi_1}\vee\NNF{\vi_2})}{\mu},\residual{(\NNF{\neg\vi_1}\wedge\NNF{\neg\vi_2})}{\mu}\rangle                             & =      \\
                        %             \langle\residual{\NNF{\vi_1\vee\vi_2}}{\mu},\residual{\NNF{\neg(\vi_1\vee\vi_2)}}{\mu}\rangle                                                 &
                        %         \end{aligned}
                        %     \]
                        %     %--------------- IMPLIES ------------------------
                        % \item[if $\bowtie$ is $\imp$:]
                        %     \[
                        %         \begin{aligned}[t]
                        %             \langle\residual{(\vi_1\imp\vi_2)}{\mu},\residual{\neg(\vi_1\imp\vi_2)}{\mu}\rangle=
                        %             %\langle\residual{\neg\vi_1}{\mu}\vee\residual{\vi_2}{\mu},\neg(\residual{\neg\vi_1}{\mu}\vee\residual{\vi_2}{\mu})\rangle=
                        %             \langle\residual{\neg\vi_1}{\mu}\vee\residual{\vi_2}{\mu},\residual{\vi_1}{\mu}\wedge\residual{\neg\vi_2}{\mu}\rangle                         & \eqind \\
                        %             \langle\residual{\NNF{\neg\vi_1}}{\mu}\vee\residual{\NNF{\vi_2}}{\mu},\residual{\NNF{\vi_1}}{\mu}\wedge\residual{\NNF{\neg\vi_2}}{\mu}\rangle & =      \\
                        %             \langle\residual{(\NNF{\neg\vi_1}\vee\NNF{\vi_2})}{\mu},\residual{(\NNF{\vi_1}\wedge\NNF{\neg\vi_2})}{\mu}\rangle                             & =      \\
                        %             \langle\residual{\NNF{\vi_1\imp\vi_2}}{\mu},\residual{\NNF{\neg(\vi_1\imp\vi_2)}}{\mu}\rangle                                                 &
                        %         \end{aligned}
                        %     \]

                        %--------------- OR and IMPLIES ------------------------
                        % \item[if $\bowtie\ \in \set{\vee, \imp}$:] \GMCHANGE{these cases can be reduced to the previous cases, since $\NNF{\vi_1\vee\vi_2}=\NNF{\neg(\neg\vi_1\wedge\neg\vi_2)}$ and $\NNF{\vi_1\imp\vi_2}= \NNF{\neg(\vi_1\wedge\neg\vi_2)}$.}
                        %--------------- IFF ------------------------
                  \item[if $\bowtie$ is $\iff$:]
                        \[
                            \begin{aligned}
                                \langle\residual{(\vi_1\iff\vi_2)}{\mu},\residual{\neg(\vi_1\iff\vi_2)}{\mu}\rangle                                                                                                                                                                         & =      \\
                                \langle\residual{((\neg\vi_1\vee\vi_2)\wedge(\vi_1\vee\neg\vi_2))}{\mu},\residual{((\vi_1\vee\vi_2)\wedge(\ADDED{\neg}\vi_1\vee\ADDED{\neg}\vi_2))}{\mu}\rangle                                                                                             & =      \\
                                % push the residual inside the conjunctions and disjunctions
                                \langle(\residual{\neg\vi_1}{\mu}\vee\residual{\vi_2}{\mu})\wedge(\residual{\vi_1}{\mu}\vee\residual{\neg\vi_2}{\mu}),(\residual{\vi_1}{\mu}\vee\residual{\vi_2}{\mu})\wedge(\residual{\ADDED{\neg}\vi_1}{\mu}\vee\residual{\ADDED{\neg}\vi_2}{\mu})\rangle & \eqind \\
                                % apply induction hypothesis by changing each psi into NNF(psi)
                                \langle(\residual{\NNF{\neg\vi_1}}{\mu}\vee\residual{\NNF{\vi_2}}{\mu})\wedge(\residual{\NNF{\vi_1}}{\mu}\vee\residual{\NNF{\neg\vi_2}}{\mu}),                                                                                                                       \\
                                (\residual{\NNF{\vi_1}}{\mu}\vee\residual{\NNF{\vi_2}}{\mu})\wedge(\residual{\NNF{\ADDED{\neg}\vi_1}}{\mu}\vee\residual{\NNF{\ADDED{\neg}\vi_2}}{\mu})\rangle                                                                                               & =      \\
                                % extract only the residual (and not NNF) from the conjunctions and disjunctions
                                \langle\residual{((\NNF{\neg\vi_1}\vee\NNF{\vi_2})\wedge(\NNF{\vi_1}\vee\NNF{\neg\vi_2}))}{\mu},                                                                                                                                                            &        \\\residual{((\NNF{\vi_1}\vee\NNF{\vi_2})\wedge(\NNF{\ADDED{\neg}\vi_1}\vee\NNF{\ADDED{\neg}\vi_2}))}{\mu}\rangle&=\\
                                % say that NNF(...) is NNF(\vi_1 \iff \vi_2)
                                \langle\residual{\NNF{\vi_1\iff\vi_2}}{\mu},\residual{\NNF{\neg(\vi_1\iff\vi_2)}}{\mu}\rangle                                                                                                                                                               & .
                            \end{aligned}
                        \]

                        % %--------------- OR ------------------------
                        % \item[if $\bowtie$ is $\vee$:] then $\residual{\NNF{\vi}}{\mu}=\residual{\NNF{\vi_1}}{\mu}\vee\residual{\NNF{\vi_2}}{\mu}$ and 
                        % $\residual{\NNF{\neg\vi}}{\mu}=\residual{\NNF{\neg\vi_1}}{\mu}\wedge\residual{\NNF{\neg\vi_2}}{\mu}$, so that:
                        % \begin{alphaenumerate}
                        %     \item $\langle\residual{\vi}{\mu},\residual{\neg\vi}{\mu}\rangle=\langle\top,\bot\rangle$ iff $\residual{\vi_1}{\mu}=\top,\residual{\neg\vi_1}{\mu}=\bot$ or $\residual{\vi_2}{\mu}=\top,\residual{\neg\vi_2}{\mu}=\bot$. By induction hypothesis, this holds iff $\residual{\NNF{\vi_1}}{\mu}=\top,\residual{\NNF{\neg\vi_1}}{\mu}=\bot$ or $\residual{\NNF{\vi_2}}{\mu}=\top,\residual{\NNF{\neg\vi_2}}{\mu}=\bot$, hence iff $\langle\residual{\NNF{\vi}}{\mu},\residual{\NNF{\neg\vi}}{\mu}\rangle=\langle\top,\bot\rangle$;

                        %     \item $\langle\residual{\vi}{\mu},\residual{\neg\vi}{\mu}\rangle=\langle\bot,\top\rangle$ iff $\residual{\vi_1}{\mu}=\residual{\vi_2}{\mu}=\bot,\residual{\neg\vi_1}{\mu}=\residual{\neg\vi_2}{\mu}=\top$. By induction hypothesis, this holds iff $\residual{\NNF{\vi_1}}{\mu}=\residual{\NNF{\vi_2}}{\mu}=\bot,\residual{\NNF{\neg\vi_1}}{\mu}=\residual{\NNF{\neg\vi_2}}{\mu}=\top$, hence iff $\langle\residual{\NNF{\vi}}{\mu},\residual{\NNF{\neg\vi}}{\mu}\rangle=\langle\bot,\top \rangle$;
                        % \end{alphaenumerate}
                        % %------------------- IMPL ----------------------
                        % \item[if $\bowtie$ is $\imp$:] then $\residual{\NNF{\vi}}{\mu}=\residual{\NNF{\neg\vi_1}}{\mu}\vee\residual{\NNF{\vi_2}}{\mu}$ and 
                        % $\residual{\NNF{\neg\vi}}{\mu}=\residual{\NNF{\vi_1}}{\mu}\wedge\residual{\NNF{\neg\vi_2}}{\mu}$, so that:
                        % \begin{alphaenumerate}
                        %     \item $\langle\residual{\vi}{\mu},\residual{\neg\vi}{\mu}\rangle=\langle\top,\bot\rangle$ iff $\residual{\vi_1}{\mu}=\bot,\residual{\neg\vi_1}{\mu}=\top$ or $\residual{\vi_2}{\mu}=\top,\residual{\neg\vi_2}{\mu}=\bot$. By induction hypothesis, this holds iff $\residual{\NNF{\vi_1}}{\mu}=\bot,\residual{\NNF{\neg\vi_1}}{\mu}=\top$ or $\residual{\NNF{\vi_2}}{\mu}=\top,\residual{\NNF{\neg\vi_2}}{\mu}=\bot$, hence iff $\langle\residual{\NNF{\vi}}{\mu},\residual{\NNF{\neg\vi}}{\mu}\rangle=\langle\top,\bot\rangle$;

                        %     \item $\langle\residual{\vi}{\mu},\residual{\neg\vi}{\mu}\rangle=\langle\bot,\top\rangle$ iff $\residual{\neg\vi_1}{\mu}=\residual{\vi_2}{\mu}=\bot,\residual{\vi_1}{\mu}=\residual{\neg\vi_2}{\mu}=\top$. By induction hypothesis, this holds iff $\residual{\NNF{\neg\vi_1}}{\mu}=\residual{\NNF{\vi_2}}{\mu}=\bot,\residual{\NNF{\vi_1}}{\mu}=\residual{\NNF{\neg\vi_2}}{\mu}=\top$, hence iff $\langle\residual{\NNF{\vi}}{\mu},\residual{\NNF{\neg\vi}}{\mu}\rangle=\langle\bot,\top \rangle$;
                        % \end{alphaenumerate}

                        % %------------------- IFF ----------------------
                        % \item[if $\bowtie$ is $\iff$:] then $\residual{\NNF{\vi}}{\mu}=(\residual{\NNF{\vi_1}}{\mu}\vee\residual{\NNF{\neg\vi_2}}{\mu})\wedge(\residual{\NNF{\neg\vi_1}}{\mu}\vee\residual{\NNF{\vi_2}}{\mu})$ and $\residual{\NNF{\neg\vi}}{\mu}=(\residual{\NNF{\vi_1}}{\mu}\vee\residual{\NNF{\vi_2}}{\mu})\wedge(\residual{\NNF{\neg\vi_1}}{\mu}\vee\residual{\NNF{\neg\vi_2}}{\mu})$, so that:    
                        % \begin{alphaenumerate}
                        %     \item $\langle\residual{\vi}{\mu},\residual{\neg\vi}{\mu}\rangle=\langle\top,\bot\rangle$ iff $\residual{\vi_1}{\mu}=\residual{\vi_2}{\mu}=v,\residual{\neg\vi_1}{\mu}=\residual{\neg\vi_2}{\mu}=\bar{v}$. By induction hypothesis, this holds iff $\residual{\NNF{\vi_1}}{\mu}=\residual{\NNF{\vi_2}}{\mu}=v,\residual{\NNF{\neg\vi_1}}{\mu}=\residual{\NNF{\neg\vi_2}}{\mu}=\bar{v}$, hence iff $\langle\residual{\NNF{\vi}}{\mu},\residual{\NNF{\neg\vi}}{\mu}\rangle=\langle\top,\bot\rangle$;

                        %     \item $\langle\residual{\vi}{\mu},\residual{\neg\vi}{\mu}\rangle=\langle\bot,\top\rangle$ iff $\residual{\vi_1}{\mu}=\residual{\neg\vi_2}{\mu}=v,\residual{\neg\vi_1}{\mu}=\residual{\vi_2}{\mu}=\bar{v}$. By induction hypothesis, this holds iff $\residual{\NNF{\vi_1}}{\mu}=\residual{\NNF{\neg\vi_2}}{\mu}=v,\residual{\NNF{\neg\vi_1}}{\mu}=\residual{\NNF{\vi_2}}{\mu}=\bar{v}$, hence iff $\langle\residual{\NNF{\vi}}{\mu},\residual{\NNF{\neg\vi}}{\mu}\rangle=\langle\bot,\top\rangle$;
                        % \end{alphaenumerate}

                        % \item[if $\bowtie$ is $\vee$:] then $\residual{\NNF{\vi}}{\mu}=\residual{\NNF{\vi_1}}{\mu}\vee\residual{\NNF{\vi_2}}{\mu}$, so that:
                        % \begin{alphaenumerate}
                        %     \item if $\residual{\vi}{\mu}=\top$ then $\residual{\vi_1}{\mu}=\top$ or $\residual{\vi_2}{\mu}=\top$. By induction hypothesis, $\residual{\NNF{\vi_1}}{\mu}=\top$ or $\residual{\NNF{\vi_2}}{\mu}=\top$, hence $\residual{\NNF{\vi}}{\mu}=\top$;

                        %     \item if $\residual{\vi}{\mu}=\bot$ then $\residual{\vi_1}{\mu}=\residual{\vi_2}{\mu}=\bot$. By induction hypothesis, $\residual{\NNF{\vi_1}}{\mu}=\residual{\NNF{\vi_2}}{\mu}=\bot$, hence $\residual{\NNF{\vi}}{\mu}=\bot$;

                        %     \item if $\residual{\vi}{\mu}=\star$ then
                        %     $\langle\residual{\vi_1}{\mu},\residual{\vi_2}{\mu}\rangle\in\set{\langle\bot,\star\rangle,\langle\star,\bot\rangle,\langle\star,\star\rangle}$. By induction hypothesis, $\langle\residual{\NNF{\vi_1}}{\mu},\residual{\NNF{\vi_2}}{\mu}\rangle\in\set{\langle\bot,\star\rangle,\langle\star,\bot\rangle,\langle\star,\star\rangle}$, hence $\residual{\NNF{\vi}}{\mu}=\star$;
                        % \end{alphaenumerate}

                        % \item[if $\bowtie$ is $\imp$:] then $\residual{\NNF{\vi}}{\mu}=\residual{\NNF{\neg\vi_1}}{\mu}\vee\residual{\NNF{\vi_2}}{\mu}$, so that:
                        % \begin{alphaenumerate}
                        %     \item if $\residual{\vi}{\mu}=\top$ then $\residual{\vi_1}{\mu}=\bot$ or $\residual{\vi_2}{\mu}=\top$. By induction hypothesis, $\residual{\NNF{\vi_1}}{\mu}=\bot$ or $\residual{\NNF{\vi_2}}{\mu}=\top$, hence $\residual{\NNF{\vi}}{\mu}=\top$;

                        %     \item if $\residual{\vi}{\mu}=\bot$ then $\residual{\vi_1}{\mu}=\top$ and $\residual{\vi_2}{\mu}=\bot$. By induction hypothesis, $\residual{\NNF{\vi_1}}{\mu}=\top$ and $\residual{\NNF{\vi_2}}{\mu}=\bot$, hence $\residual{\NNF{\vi}}{\mu}=\bot$;

                        %     \item if $\residual{\vi}{\mu}=\star$ then
                        %     $\langle\residual{\neg\vi_1}{\mu},\residual{\vi_2}{\mu}\rangle\in\set{\langle\top,\star\rangle,\langle\star,\bot\rangle,\langle\star,\star\rangle}$. By induction hypothesis, $\langle\residual{\NNF{\neg\vi_1}}{\mu},\residual{\NNF{\vi_2}}{\mu}\rangle\in\set{\langle\top,\star\rangle,\langle\star,\bot\rangle,\langle\star,\star\rangle}$, hence $\residual{\NNF{\vi}}{\mu}=\star$;
                        % \end{alphaenumerate}
                        % \item[if $\bowtie$ is $\iff$:] then $\residual{\NNF{\vi}}{\mu}=(\residual{\NNF{\neg\vi_1}}{\mu}\vee\residual{\NNF{\vi_2}}{\mu})\wedge(\residual{\NNF{\vi_1}}{\mu}\vee\residual{\NNF{\neg\vi_2}}{\mu})$, so that:
                        % \begin{alphaenumerate}
                        %     % \item if $\residual{\vi}{\mu}=\top$ then either $\residual{\vi_1}{\mu}=\residual{\vi_2}{\mu}=\top$, $\residual{\neg\vi_1}{\mu}=\residual{\neg\vi_2}{\mu}=\bot$ or $\residual{\vi_1}{\mu}=\residual{\vi_2}{\mu}=\bot$,$\residual{\neg\vi_1}{\mu}=\residual{\neg\vi_2}{\mu}=\top$. By induction hypothesis, either $\residual{\NNF{\vi_1}}{\mu}=\residual{\NNF{\vi_2}}{\mu}=\top$,$\residual{\NNF{\neg\vi_1}}{\mu}=\residual{\NNF{\neg\vi_2}}{\mu}=\bot$ or $\residual{\NNF{\vi_1}}{\mu}=\residual{\NNF{\vi_2}}{\mu}=\bot$,$\residual{\NNF{\neg\vi_1}}{\mu}=\residual{\NNF{\neg\vi_2}}{\mu}=\top$, hence $\residual{\NNF{\vi}}{\mu}=\top$;
                        %     \item if $\residual{\vi}{\mu}=\top$ then $\langle\residual{\vi_1}{\mu},\residual{\vi_2}{\mu},\residual{\neg\vi_1}{\mu},\residual{\neg\vi_2}{\mu}\rangle\in\set{\langle\top,\top,\bot,\bot\rangle,\langle\bot,\bot,\top,\top\rangle}$. By induction hypothesis, $\langle\residual{\NNF{\vi_1}}{\mu},\residual{\NNF{\vi_2}}{\mu},\residual{\NNF{\neg\vi_1}}{\mu},\residual{\NNF{\neg\vi_2}}{\mu}\rangle\in\set{\langle\top,\top,\bot,\bot\rangle,\langle\bot,\bot,\top,\top\rangle}$, hence $\residual{\NNF{\vi}}{\mu}=\top$;

                        %     \item if $\residual{\vi}{\mu}=\bot$ then $\langle\residual{\vi_1}{\mu},\residual{\vi_2}{\mu},\residual{\neg\vi_1}{\mu},\residual{\neg\vi_2}{\mu}\rangle\in\set{\langle\top,\bot,\bot,\top\rangle,\langle\bot,\top,\top,\bot\rangle}$. By induction hypothesis, $\langle\residual{\NNF{\vi_1}}{\mu},\residual{\NNF{\vi_2}}{\mu},\residual{\NNF{\neg\vi_1}}{\mu},\residual{\NNF{\neg\vi_2}}{\mu}\rangle\in\set{\langle\top,\bot,\bot,\top\rangle,\langle\bot,\top,\top,\bot\rangle}$, hence $\residual{\NNF{\vi}}{\mu}=\bot$;

                        %     \item if $\residual{\vi}{\mu}=\star$ then $\langle\residual{\vi_1}{\mu},\residual{\vi_2}{\mu},\residual{\neg\vi_1}{\mu},\residual{\neg\vi_2}{\mu}\rangle\in\{\langle\top,\star,\bot,\star\rangle$, $\langle\bot,\star,\top,\star\rangle$, $\langle\star,\top,\star,\bot\rangle$, $\langle\star,\bot,\star,\top\rangle$, $\langle\star,\star,\star,\star\rangle\}$. By induction hypothesis, \\$\langle\residual{\NNF{\vi_1}}{\mu},\residual{\NNF{\vi_2}}{\mu},\residual{\NNF{\neg\vi_1}}{\mu},\residual{\NNF{\neg\vi_2}}{\mu}\rangle\in\{\langle\top,\star,\bot,\star\rangle$, $\langle\bot,\star,\top,\star\rangle$, $\langle\star,\top,\star,\bot\rangle$, $\langle\star,\bot,\star,\top\rangle$, $\langle\star,\star,\star,\star\rangle\}$, hence $\residual{\NNF{\vi}}{\mu}=\star$;
                        % \end{alphaenumerate}
              \end{description}
              %--------------- OR and IMPLIES ------------------------
        \item[if $\vi\defas(\vi_1\bowtie\vi_2)$:] s.t.\ $\bowtie~\in\set{\vee,\imp}$. These
              cases can be reduced to the previous cases, since
              $\NNF{\vi_1\vee\vi_2}=\NNF{\neg(\neg\vi_1\wedge\neg\vi_2)}$ and
              $\NNF{\vi_1\imp\vi_2}= \NNF{\neg(\vi_1\wedge\neg\vi_2)}$.
    \end{description}
\end{proof}
{Thus,
$\pair{\residual{\vi}{\mu}}{\residual{\neg\vi}{\mu}}=\pair{\residual{\NNF{\vi}}{\mu}}{\residual{\NNF{\neg\vi}}{\mu}}$
so that $\residual{\vi}{\mu}=v$ iff
$\residual{\NNF{\vi}}{\mu}=v$ for every
$v\in\set{\top,\bot}$, so that \cref{th:munnf}
in~\cref{sec:background:propositional-logic} holds.}
%\end{gmchange}

%%%%%%%%%%%%%%%%%%%%%%%%%%%%%%%%%%%%%%%%%%%%%%%%%%%%%%%%%%%%%
%%%
%%%%%%%%%%%%%%%%%%%%%%%%%%%%%%%%%%%%%%%%%%%%%%%%%%%%%%%%%%%%%

% \newpage

% % \begin{gmchange}
% \subsection{Proof for~\cref{th:existsetaB} in~\cref{sec:solution}}%
% \label{sec:proofexistsetaB}
% %We present the proof for~\cref{th:existsetaB} in~\cref{sec:solution}.

% \end{gmchange}
\newpage
\section{An analysis of candidate solvers}%
\label{appendix:tooltable}

In Table~\ref{tab:allsat-solvers} we report an analysis of the features of
every candidate AllSAT and AllSMT solver, as discussed in
\sref{sec:survey-allsat-solvers}.

\begin{table}[th]
    \newcommand{\cmark}{\textcolor{green4}{\ding{51}}}%
    \newcommand{\xmark}{\textcolor{red}{\ding{55}}}%
    \newcommand{\nomark}{}
    \newcommand*\rot{\rotatebox{90}}
    \newcolumntype{C}{>{\centering\arraybackslash}m{0.4cm}}
    \centering
    \begin{tabularx}{.8\textwidth}{c|l|CCCCC|CC|X}
        % \hline
                                               &                              &
        \multicolumn{5}{c|}{Required features} & \multicolumn{2}{c|}{Options} &                                                                                        \\
        \hline
                                               & Solver                       &
        \rot{Available}                        &
        \rot{CNF input\ }                      &
        \rot{Projected}                        &
        \rot{Partial}                          &
        \rot{Minimal}                          &
        \rot{Disjoint}                         &
        \rot{Non-disjoint}                     &
        {Notes}                                                                                                                                                        \\
        \hline
        \parbox[t]{4mm}{\multirow{15}{*}{\rotatebox[origin=c]{90}{AllSAT}}}
                                               & \textsc{RELSAT}              & \cmark & \cmark & \xmark & \xmark & \xmark & \cmark & \xmark &                         \\
                                               & \textsc{Grumberg}            & \xmark & \cmark & \cmark & \cmark & \cmark & \cmark & \cmark & Ex-post minimization    \\
                                               & \textsc{SOLALL}              & ?      & \cmark & \cmark & \cmark & \xmark & \cmark & \xmark & Result stored in (O)BDD \\
                                               & \textsc{Jin}                 & \xmark & \xmark & \xmark & \cmark & \cmark & \xmark & \cmark & AIG input               \\
                                               & \textsc{clasp}               & \cmark & \cmark & \cmark & \xmark & \xmark & \cmark & \xmark & ASP solver              \\
                                               & \textsc{PicoSAT}             & \cmark & \cmark & \xmark & \xmark & \xmark & \cmark & \xmark &                         \\
                                               & \textsc{Yu}                  & ?      & \cmark & \xmark & \cmark & \cmark & \cmark & \cmark &                         \\
                                               & \textsc{BC}                  & \cmark & \cmark & \xmark & \cmark & \cmark & \cmark & \cmark &                         \\
                                               & \textsc{NBC}                 & \cmark & \cmark & \xmark & \xmark & \xmark & \cmark & \xmark &                         \\
                                               & \textsc{BDD}                 & \cmark & \cmark & \xmark & \cmark & \xmark & \xmark & \cmark & Result stored in (O)BDD \\
                                               & \textsc{depbdd}              & \xmark & \cmark & \cmark & \cmark & \xmark & \cmark & \xmark & Result stored in (O)BDD \\
                                               & \textsc{Dualiza}             & \cmark & \cmark & \cmark & \cmark & \xmark & \cmark & \cmark & Dual-reasoning-based    \\
                                               & \textsc{BASolver}            & \xmark & \cmark & \xmark & \cmark & \cmark & \xmark & \cmark &                         \\
                                               & \textsc{AllSATCC}            & \xmark & \cmark & \xmark & \cmark & \cmark & \cmark & \xmark &                         \\
                                               & \textsc{HALL}                & \cmark & \xmark & \xmark & \cmark & \cmark & \cmark & \cmark & AIG input               \\
                                               & \tabularallsat{}             & \cmark & \cmark & \cmark & \cmark & \xmark & \cmark & \xmark &                         \\
                                               & \decdnnf{}                   & \cmark & \xmark & \xmark & \cmark & \xmark & \cmark & \xmark & d-DNNF input            \\
        \hline
        \parbox[t]{4mm}{\multirow{5}{*}{\rotatebox[origin=c]{90}{AllSMT}}}
                                               &                              &        &        &        &        &        &        &                                  \\
                                               & \mathsat{}                   & \cmark & \cmark & \cmark & \cmark & \cmark & \cmark & \cmark &                         \\
                                               & \textsc{aZ3}                 & \cmark & \cmark & \cmark & \xmark & \xmark & \cmark & \xmark &                         \\
                                               & \tabularallsmt{}             & \cmark & \cmark & \cmark & \cmark & \xmark & \cmark & \xmark &                         \\
                                               &                              &        &        &        &        &        &        &
        \\
        % \textsc{TALE} & \xmark & \xmark & \cmark & \xmark & \cmark & \cmark & AIG input \\
        % \textsc{MARS} & \xmark & \xmark & \cmark & \cmark & \cmark & \cmark & AIG input \\
        % \textsc{DUTY} & \xmark & \xmark & \cmark & \xmark & \cmark & \cmark & AIG input \\
    \end{tabularx}
    \caption{List of AllSAT and AllSMT solvers (rows) and supported
        features (columns).
        % Required features
        % (\ref{item:tool:avail}-\ref{item:tool:minimal}): the tool is
        % publicly available, input CNF formula, enumerate projected
        % assignments, enumerate partial assignments. Optional features:
        % partial assignments are minimal, enumerate disjoint assignments,
        % enumerate non-disjoint assignment.
        % \newline
        % In the ``Available'' column, ``\xmark'' and ``?'' indicate respectively that the
        % authors confirmed to us the non-availability of the tool and
        % that they did not reply to our request.
        %
        (In the ``Available'' column, ``\xmark'' indicates that the
        authors confirmed to us the unavailability of their tool, whereas
        ``?'' indicates that they did not reply to our enquiry.)
    }
    \label{tab:allsat-solvers}
\end{table}




\newpage
    \section{Experimental results on plain SAT and SMT solving}%
    \label{appendix:sat}
    In this section, we report the results of the experiments on plain SAT and SMT solving with \mathsat{} using the different CNF-izations.
    The plots in~\cref{fig:sat:ecdf} show that \NNFPlaisted{} does not bring any advantage for plain SAT solving, supporting the analysis in~\sref{sec:experiments:sat}.
    \begin{figure}[h!]
        %
        \centering
        \captionsetup[subfigure]{justification=centering}
        \begin{subfigure}[t]{\textwidth}
            \centering
            \includegraphics[width=\textwidth]{plots/msat-sat/bool/legend.pdf}
        \end{subfigure}
        \begin{subfigure}[b]{0.33\textwidth}
            \centering
            \includegraphics[width=\textwidth]{plots/msat-sat/bool/time_ecdf_syn-bool.pdf}
            \caption{Boolean synthetic.}%
            \label{fig:sat:syn:bool:ecdf}
        \end{subfigure}%
        \begin{subfigure}[b]{0.33\textwidth}
            \centering
            \includegraphics[width=\textwidth]{plots/msat-sat/bool/time_ecdf_iscas85.pdf}
            \caption{ISCAS'85.}%
            \label{fig:sat:circ:ecdf}
        \end{subfigure}%
        \begin{subfigure}[b]{0.33\textwidth}
            \centering
            \includegraphics[width=\textwidth]{plots/msat-sat/bool/time_ecdf_aig.pdf}
            \caption{AIG.}%
            \label{fig:sat:aig:ecdf}
        \end{subfigure}
        \begin{subfigure}[b]{0.33\textwidth}
            \centering
            \includegraphics[width=\textwidth]{plots/msat-sat/lra/time_ecdf_syn-lra.pdf}
            \caption{\smtlarat{} synthetic.}%
            \label{fig:sat:syn:lra:ecdf}
        \end{subfigure}%
        \begin{subfigure}[b]{0.33\textwidth}
            \centering
            \includegraphics[width=\textwidth]{plots/msat-sat/lra/time_ecdf_wmi.pdf}
            \caption{WMI.}%
            \label{fig:sat:wmi:ecdf}
        \end{subfigure}
        \caption{Time taken for plain SAT %\ignoreinshort{\GMCHANGEp{
            and SMT %}}
            solving using the different CNF transformations. The $y$-axis reports the number of instances for which the solver finished within the cumulative time on the $x$-axis.
        }%
        \label{fig:sat:ecdf}
    \end{figure}

% ---- d-DNNF enum ----
\newpage
% \begin{gmchange}
        \section{Experimental results on d-DNNF-based enumeration}%
        \label{appendix:d4:enum}
        Recently, \citeA{lagniezLeveragingDecisionDNNFCompilation2024} presented a tool, named \decdnnf{}, to enumerate all models of a d-DNNF formula~\cite{darwicheKnowledgeCompilationMap2002}.

        Following the \dfdecdnnf{} approach by~\citeA{lagniezLeveragingDecisionDNNFCompilation2024}, we first compile the CNF-ization of the
        input formula into a d-DNNF using the \textsc{d4}
        compiler~\cite{lagniezImprovedDecisionDNNFCompiler2017}, and then enumerate its models using \decdnnf{}. To avoid enumerating on CNF labels, we project the d-DNNF onto the original atoms \allA{} of the input formula, and then use \decdnnf{} to enumerate the models of the projected d-DNNF.

        The results of the experiments on the Boolean benchmarks are shown in \cref{fig:plt:d4:all:bool:scatter}. We see that \TseitinCNF{} is uniformly the best-performing CNF-ization, supporting the analysis in~\sref{sec:experiments:d4}
        % \PlaistedCNF{} and
        % \NNFPlaisted{} only introduce time-overhead, without any benefit in terms of the number of models enumerated. This can be explained by the fact that d-DNNF compilation does not aim at representing short partial truth assignments, but rather at effectively decomposing the formula into atom-disjoint components, so to allow for caching and sharing of sub-formulas.
        % In fact, compilation to d-DNNF is typically performed following a DPLL-like procedure, where the formula is decomposed into a DAG of decision nodes, each representing an assignment to one or more atoms. In the case of projected d-DNNF, decisions are made on the relevant atoms first, and whenever no relevant atom is left, a SAT solver is used to complete the assignment. This search strategy prevents finding short partial assignments~\cite{lagniezRecursiveAlgorithmProjected2019}, and thus the
        % benefit of our encoding for enumeration is lost.
% \end{gmchange}
% --------------- d4 enum ----------------
\begin{figure}[h!]
    \centering
    \begin{subfigure}[t]{\textwidth}
        \centering
        \includegraphics[height=2em]{plots/d4/enum/all/legend.pdf}
    \end{subfigure}
    \begin{subfigure}[t]{\textwidth}
        \begin{subfigure}[t]{0.29\textwidth}
            \centering
            \includegraphics[width=.85\textwidth]{plots/d4/enum/all/models_compare_LAB_vs_LABELNEG_POL.pdf}%
            \label{fig:plt:d4:all:bool:norep:models:lab_vs_pol}
        \end{subfigure}\hfill
        \begin{subfigure}[t]{0.29\textwidth}
            \centering
            \includegraphics[width=.85\textwidth]{plots/d4/enum/all/models_compare_LABELNEG_POL_vs_NNF_MUTEX_POL.pdf}%
            \label{fig:plt:d4:all:bool:norep:models:pol_vs_nnfpol}
        \end{subfigure}\hfill
        \begin{subfigure}[t]{0.29\textwidth}
            \centering
            \includegraphics[width=.85\textwidth]{plots/d4/enum/all/models_compare_LAB_vs_NNF_MUTEX_POL.pdf}%
            \label{fig:plt:d4:all:bool:norep:models:lab_vs_nnfpol}
        \end{subfigure}\hfill
        \begin{subfigure}[t]{0.29\textwidth}
            \centering
            \includegraphics[width=.85\textwidth]{plots/d4/enum/all/time_compare_LAB_vs_LABELNEG_POL.pdf}%
            \label{fig:plt:d4:all:bool:norep:time:lab_vs_pol}
        \end{subfigure}\hfill
        \begin{subfigure}[t]{0.29\textwidth}
            \centering
            \includegraphics[width=.85\textwidth]{plots/d4/enum/all/time_compare_LABELNEG_POL_vs_NNF_MUTEX_POL.pdf}%
            \label{fig:plt:d4:all:bool:norep:time:pol_vs_nnfpol}
        \end{subfigure}\hfill
        \begin{subfigure}[t]{0.29\textwidth}
            \centering
            \includegraphics[width=.85\textwidth]{plots/d4/enum/all/time_compare_LAB_vs_NNF_MUTEX_POL.pdf}%
            \label{fig:plt:d4:all:bool:norep:time:lab_vs_nnfpol}
            % \end{subfigure}
        \end{subfigure}
        \caption{Results for disjoint enumeration.}%
        \label{fig:plt:d4:all:bool:norep:scatter}
    \end{subfigure}
    \begin{subfigure}[t]{\textwidth}
        \centering
        {\small
            \newcommand{\best}[1]{\textbf{#1}}
% \begin{figure}[th]
%     \centering
\begin{tabularx}{.44\textwidth}{l|c|ccc}
    % \toprule
    \multirow{3}{*}{Bench.} & \multirow{3}{*}{Instances} & \multicolumn{3}{c}{T.O.\ for disjoint AllSMT}                                            \\
                            &                            & \TseitinCNF{}                                  & \PlaistedCNF{} & \NNFPlaisted{}
    \\[0.2em]
    \hline
    Syn-LRA                 & 300                        & 147                                            & 92             & \best{73}               \\
    WMI                     & 40                         & 24                                             & 24             & \best{23}               \\
    % \bottomrule
\end{tabularx}
%     \caption{Number of timeouts for the plots in \cref{fig:plt:all:lra:scatter}.}%
%     \label{tab:timeouts:lra}
% \end{figure}
        }
        \caption{Number of timeouts.}%
        \label{tab:timeouts:d4:bool}
    \end{subfigure}
    \caption{Results on the Boolean benchmarks using \dfdecdnnf{}.
            Scatter plots in~\ref{fig:plt:d4:all:bool:norep:scatter} compare CNF-izations by \TAna{} size (first row) and execution time (second row).
            Points on dashed lines represent timeouts, summarized in~\ref{tab:timeouts:d4:bool}.
            All axes use a logarithmic scale.}%
    \label{fig:plt:d4:all:bool:scatter}
\end{figure}

% ---- d-DNNF counting ----
\newpage
    % \begin{gmchange}
        \section{Experimental results on d-DNNF-based model counting}%
        \label{appendix:d4:counting}
        We tested the different CNF-izations also for model counting using \df{}. Since \PlaistedCNF{} and \NNFPlaisted{} do not preserve the model count, we perform projected model counting on the original atoms of the input formula. Even though this would not be necessary for \TseitinCNF{}, we apply the same procedure to ensure a fair comparison.

        The results of the experiments on the Boolean benchmarks are shown in \cref{fig:plt:d4:counting:all:bool:scatter}. We can observe that also for model counting, \TseitinCNF{} is uniformly the best-performing CNF-ization, supporting the analysis in~\sref{sec:experiments:d4}.
        % \PlaistedCNF{} and \NNFPlaisted{} only introduce time-overhead.
        % Similar considerations as for enumeration apply here, since model counting strategies are typically based on effective decompositions of the formula into atom-disjoint components, so to allow for caching and sharing of sub-formulas, rather than on finding short partial truth assignments. Hence, our encoding does not provide any benefit for model counting using state-of-the-art counters.
    % \end{gmchange}
% --------------- d4 enum ----------------
\begin{figure}[h!]
    \centering
    \begin{subfigure}[t]{\textwidth}
        \centering
        \includegraphics[height=2em]{plots/d4/counting/all/legend.pdf}
    \end{subfigure}
    \begin{subfigure}[t]{\textwidth}
        \begin{subfigure}[t]{0.29\textwidth}
            \centering
            \includegraphics[width=.85\textwidth]{plots/d4/counting/all/time_compare_LAB_vs_LABELNEG_POL.pdf}%
            \label{fig:plt:d4:counting:all:bool:norep:time:lab_vs_pol}
        \end{subfigure}\hfill
        \begin{subfigure}[t]{0.29\textwidth}
            \centering
            \includegraphics[width=.85\textwidth]{plots/d4/counting/all/time_compare_LABELNEG_POL_vs_NNF_MUTEX_POL.pdf}%
            \label{fig:plt:d4:counting:all:bool:norep:time:pol_vs_nnfpol}
        \end{subfigure}\hfill
        \begin{subfigure}[t]{0.29\textwidth}
            \centering
            \includegraphics[width=.85\textwidth]{plots/d4/counting/all/time_compare_LAB_vs_NNF_MUTEX_POL.pdf}%
            \label{fig:plt:d4:counting:all:bool:norep:time:lab_vs_nnfpol}
            % \end{subfigure}
        \end{subfigure}
        \caption{Results for model counting.}%
        \label{fig:plt:d4:counting:all:bool:norep:scatter}
    \end{subfigure}
    \begin{subfigure}[t]{\textwidth}
        \centering
        {\small
            \newcommand{\best}[1]{\textbf{#1}}
% \begin{figure}[th]
%     \centering
\begin{tabularx}{.44\textwidth}{l|c|ccc}
    % \toprule
    \multirow{3}{*}{Bench.} & \multirow{3}{*}{Instances} & \multicolumn{3}{c}{T.O.\ for disjoint AllSMT}                                            \\
                            &                            & \TseitinCNF{}                                  & \PlaistedCNF{} & \NNFPlaisted{}
    \\[0.2em]
    \hline
    Syn-LRA                 & 300                        & 147                                            & 92             & \best{73}               \\
    WMI                     & 40                         & 24                                             & 24             & \best{23}               \\
    % \bottomrule
\end{tabularx}
%     \caption{Number of timeouts for the plots in \cref{fig:plt:all:lra:scatter}.}%
%     \label{tab:timeouts:lra}
% \end{figure}
        }
        \caption{Number of timeouts.}%
        \label{tab:timeouts:d4:bool:counting}
    \end{subfigure}
    \caption{Results on the Boolean benchmarks using \df{} for model counting.
            Scatter plots in~\ref{fig:plt:d4:counting:all:bool:norep:scatter} compare CNF-izations by execution time.
            Points on dashed lines represent timeouts, summarized in~\ref{tab:timeouts:d4:bool:counting}.
            All axes use a logarithmic scale.}%
    \label{fig:plt:d4:counting:all:bool:scatter}
\end{figure}

\newpage

\section{Details on experimental results in the paper}%
\label{appendix:experiments}
    In this section, we report the scatter plots on individual
    benchmarks for the experiments presented
    in~\cref{sec:experiments:results}, \cref{fig:plt:all:bool:scatter,,fig:plt:tabula:all:bool:scatter,,fig:plt:all:lra:scatter}.

    For AllSAT,
    \cref{fig:plt:syn:bool:scatter,,fig:plt:circ:scatter,,fig:plt:aig:scatter} show
    the results for \mathsat{} on the Boolean synthetic, ISCAS'85 and AIG
    benchmarks, respectively.
    \cref{fig:plt:tabula:syn:bool:scatter,,fig:plt:tabula:circ:scatter,,fig:plt:tabula:aig:scatter}
    show the results for \tabularallsat{} on the same benchmarks.

    For AllSMT, \cref{fig:plt:syn:lra:scatter,,fig:plt:wmi:scatter} show the
    results for \mathsat{} on the \smtlarat{} synthetic and WMI benchmarks, respectively.
    \cref{fig:plt:tabula:syn:lra:scatter,,fig:plt:tabula:wmi:scatter} show the
    results for \tabularallsmt{} on the same benchmarks.
% --------------- AllSAT ----------------
% --------------- AllSAT MathSAT ----------------
\begin{figure}
    \centering
    % \begin{subfigure}[t]{\textwidth}
    \begin{subfigure}[t]{\textwidth}
        \begin{subfigure}[t]{0.29\textwidth}
            \includegraphics[width=.85\textwidth]{plots/msat/bool/no-rep/syn-bool/models_compare_LAB_vs_LABELNEG_POL.pdf}%
            \label{fig:plt:syn:bool:norep:models:lab_vs_pol}
        \end{subfigure}\hfill
        \begin{subfigure}[t]{0.29\textwidth}
            \includegraphics[width=.85\textwidth]{plots/msat/bool/no-rep/syn-bool/models_compare_LABELNEG_POL_vs_NNF_MUTEX_POL.pdf}%
            \label{fig:plt:syn:bool:norep:models:pol_vs_nnfpol}
        \end{subfigure}\hfill
        \begin{subfigure}[t]{0.29\textwidth}
            \includegraphics[width=.85\textwidth]{plots/msat/bool/no-rep/syn-bool/models_compare_LAB_vs_NNF_MUTEX_POL.pdf}%
            \label{fig:plt:syn:bool:norep:models:lab_vs_nnfpol}
        \end{subfigure}\hfill
        \begin{subfigure}[t]{0.29\textwidth}
            \includegraphics[width=.85\textwidth]{plots/msat/bool/no-rep/syn-bool/time_compare_LAB_vs_LABELNEG_POL.pdf}%
            \label{fig:plt:syn:bool:norep:time:lab_vs_pol}
        \end{subfigure}\hfill
        \begin{subfigure}[t]{0.29\textwidth}
            \includegraphics[width=.85\textwidth]{plots/msat/bool/no-rep/syn-bool/time_compare_LABELNEG_POL_vs_NNF_MUTEX_POL.pdf}%
            \label{fig:plt:syn:bool:norep:time:pol_vs_nnfpol}
        \end{subfigure}\hfill
        \begin{subfigure}[t]{0.29\textwidth}
            \includegraphics[width=.85\textwidth]{plots/msat/bool/no-rep/syn-bool/time_compare_LAB_vs_NNF_MUTEX_POL.pdf}%
            \label{fig:plt:syn:bool:norep:time:lab_vs_nnfpol}
            % \end{subfigure}
        \end{subfigure}
        \caption{Results for disjoint enumeration. %\TseitinCNF{}, \PlaistedCNF{} and $\NNFPlaisted{}$ reported 163, 94 and 20 timeouts, respectively (points on the dashed lines).
        }%
        \label{fig:plt:syn:bool:norep:scatter}
    \end{subfigure}
    %%%%%%%%%%%% REP %%%%%%%%%%%%%
    \begin{subfigure}[t]{\textwidth}
        \begin{subfigure}[t]{0.29\textwidth}
            \includegraphics[width=.85\textwidth]{plots/msat/bool/rep/syn-bool/models_compare_LAB_vs_LABELNEG_POL.pdf}%
            \label{fig:plt:syn:bool:rep:models:lab_vs_pol}
        \end{subfigure}\hfill
        \begin{subfigure}[t]{0.29\textwidth}
            \includegraphics[width=.85\textwidth]{plots/msat/bool/rep/syn-bool/models_compare_LABELNEG_POL_vs_NNF_MUTEX_POL.pdf}%
            \label{fig:plt:syn:bool:rep:models:pol_vs_nnfpol}
        \end{subfigure}\hfill
        \begin{subfigure}[t]{0.29\textwidth}
            \includegraphics[width=.85\textwidth]{plots/msat/bool/rep/syn-bool/models_compare_LAB_vs_NNF_MUTEX_POL.pdf}%
            \label{fig:plt:syn:bool:rep:models:lab_vs_nnfpol}
        \end{subfigure}\hfill
        \begin{subfigure}[t]{0.29\textwidth}
            \includegraphics[width=.85\textwidth]{plots/msat/bool/rep/syn-bool/time_compare_LAB_vs_LABELNEG_POL.pdf}%
            \label{fig:plt:syn:bool:rep:time:lab_vs_pol}
        \end{subfigure}\hfill
        \begin{subfigure}[t]{0.29\textwidth}
            \includegraphics[width=.85\textwidth]{plots/msat/bool/rep/syn-bool/time_compare_LABELNEG_POL_vs_NNF_MUTEX_POL.pdf}%
            \label{fig:plt:syn:bool:rep:time:pol_vs_nnfpol}
        \end{subfigure}\hfill
        \begin{subfigure}[t]{0.29\textwidth}
            \includegraphics[width=.85\textwidth]{plots/msat/bool/rep/syn-bool/time_compare_LAB_vs_NNF_MUTEX_POL.pdf}%
            \label{fig:plt:syn:bool:rep:time:lab_vs_nnfpol}
            % \end{subfigure}
        \end{subfigure}
        \caption{Results for non-disjoint enumeration. %\TseitinCNF{}, \PlaistedCNF{} and $\NNFPlaisted{}$ reported 122, 69 and 0 timeouts, respectively (points on the dashed lines).
        }%
        \label{fig:plt:syn:bool:rep:scatter}
    \end{subfigure}
    \caption{Results on the Boolean synthetic benchmarks using \mathsat{}.
            Scatter plots in~\ref{fig:plt:syn:bool:norep:scatter} and~\ref{fig:plt:syn:bool:rep:scatter} compare CNF-izations by \TAna{} size (first row) and execution time (second row).
            Points on dashed lines represent timeouts, summarized in~\ref{tab:timeouts:bool}.
            All axes use a logarithmic scale.}%
    \label{fig:plt:syn:bool:scatter}
\end{figure}

\begin{figure}
    \centering
    % \begin{subfigure}[t]{\textwidth}
    \begin{subfigure}[t]{\textwidth}
        \begin{subfigure}[t]{0.29\textwidth}
            \includegraphics[width=.85\textwidth]{plots/msat/bool/no-rep/iscas85/models_compare_LAB_vs_LABELNEG_POL.pdf}%
            \label{fig:plt:circ:norep:models:lab_vs_pol}
        \end{subfigure}\hfill
        \begin{subfigure}[t]{0.29\textwidth}
            \includegraphics[width=.85\textwidth]{plots/msat/bool/no-rep/iscas85/models_compare_LABELNEG_POL_vs_NNF_MUTEX_POL.pdf}%
            \label{fig:plt:circ:norep:models:pol_vs_nnfpol}
        \end{subfigure}\hfill
        \begin{subfigure}[t]{0.29\textwidth}
            \includegraphics[width=.85\textwidth]{plots/msat/bool/no-rep/iscas85/models_compare_LAB_vs_NNF_MUTEX_POL.pdf}%
            \label{fig:plt:circ:norep:models:lab_vs_nnfpol}
        \end{subfigure}\hfill
        \begin{subfigure}[t]{0.29\textwidth}
            \includegraphics[width=.85\textwidth]{plots/msat/bool/no-rep/iscas85/time_compare_LAB_vs_LABELNEG_POL.pdf}%
            \label{fig:plt:circ:norep:time:lab_vs_pol}
        \end{subfigure}\hfill
        \begin{subfigure}[t]{0.29\textwidth}
            \includegraphics[width=.85\textwidth]{plots/msat/bool/no-rep/iscas85/time_compare_LABELNEG_POL_vs_NNF_MUTEX_POL.pdf}%
            \label{fig:plt:circ:norep:time:pol_vs_nnfpol}
        \end{subfigure}\hfill
        \begin{subfigure}[t]{0.29\textwidth}
            \includegraphics[width=.85\textwidth]{plots/msat/bool/no-rep/iscas85/time_compare_LAB_vs_NNF_MUTEX_POL.pdf}%
            \label{fig:plt:circ:norep:time:lab_vs_nnfpol}
            % \end{subfigure}
        \end{subfigure}
        \caption{Results for disjoint enumeration. %\TseitinCNF{}, \PlaistedCNF{} and $\NNFPlaisted{}$ reported 49, 44 and 27 timeouts, respectively (points on the dashed lines).
        }%
        \label{fig:plt:circ:norep:scatter}
    \end{subfigure}
    %%%%%%%%%%%% REP %%%%%%%%%%%%%
    \begin{subfigure}[t]{\textwidth}
        \begin{subfigure}[t]{0.29\textwidth}
            \includegraphics[width=.85\textwidth]{plots/msat/bool/rep/iscas85/models_compare_LAB_vs_LABELNEG_POL.pdf}%
            \label{fig:plt:circ:rep:models:lab_vs_pol}
        \end{subfigure}\hfill
        \begin{subfigure}[t]{0.29\textwidth}
            \includegraphics[width=.85\textwidth]{plots/msat/bool/rep/iscas85/models_compare_LABELNEG_POL_vs_NNF_MUTEX_POL.pdf}%
            \label{fig:plt:circ:rep:models:pol_vs_nnfpol}
        \end{subfigure}\hfill
        \begin{subfigure}[t]{0.29\textwidth}
            \includegraphics[width=.85\textwidth]{plots/msat/bool/rep/iscas85/models_compare_LAB_vs_NNF_MUTEX_POL.pdf}%
            \label{fig:plt:circ:rep:models:lab_vs_nnfpol}
        \end{subfigure}\hfill
        \begin{subfigure}[t]{0.29\textwidth}
            \includegraphics[width=.85\textwidth]{plots/msat/bool/rep/iscas85/time_compare_LAB_vs_LABELNEG_POL.pdf}%
            \label{fig:plt:circ:rep:time:lab_vs_pol}
        \end{subfigure}\hfill
        \begin{subfigure}[t]{0.29\textwidth}
            \includegraphics[width=.85\textwidth]{plots/msat/bool/rep/iscas85/time_compare_LABELNEG_POL_vs_NNF_MUTEX_POL.pdf}%
            \label{fig:plt:circ:rep:time:pol_vs_nnfpol}
        \end{subfigure}\hfill
        \begin{subfigure}[t]{0.29\textwidth}
            \includegraphics[width=.85\textwidth]{plots/msat/bool/rep/iscas85/time_compare_LAB_vs_NNF_MUTEX_POL.pdf}%
            \label{fig:plt:circ:rep:time:lab_vs_nnfpol}
            % \end{subfigure}
        \end{subfigure}
        \caption{Results for non-disjoint enumeration. %\TseitinCNF{}, \PlaistedCNF{} and $\NNFPlaisted{}$ reported 41, 38 and 3 timeouts, respectively (points on the dashed lines).
        }%
        \label{fig:plt:circ:rep:scatter}
    \end{subfigure}
    \caption{Results on the ISCAS'85 benchmarks using \mathsat{}.
            Scatter plots in~\ref{fig:plt:circ:norep:scatter} and~\ref{fig:plt:circ:rep:scatter} compare CNF-izations by \TAna{} size (first row) and execution time (second row).
            Points on dashed lines represent timeouts, summarized in~\ref{tab:timeouts:bool}.
            All axes use a logarithmic scale.}%
    \label{fig:plt:circ:scatter}
\end{figure}
\begin{figure}
    \centering
    % \begin{subfigure}[t]{\textwidth}
    \begin{subfigure}[t]{\textwidth}
        \begin{subfigure}[t]{0.29\textwidth}
            \includegraphics[width=.85\textwidth]{plots/msat/bool/no-rep/aig/models_compare_LAB_vs_LABELNEG_POL.pdf}%
            \label{fig:plt:aig:norep:models:lab_vs_pol}
        \end{subfigure}\hfill
        \begin{subfigure}[t]{0.29\textwidth}
            \includegraphics[width=.85\textwidth]{plots/msat/bool/no-rep/aig/models_compare_LABELNEG_POL_vs_NNF_MUTEX_POL.pdf}%
            \label{fig:plt:aig:norep:models:pol_vs_nnfpol}
        \end{subfigure}\hfill
        \begin{subfigure}[t]{0.29\textwidth}
            \includegraphics[width=.85\textwidth]{plots/msat/bool/no-rep/aig/models_compare_LAB_vs_NNF_MUTEX_POL.pdf}%
            \label{fig:plt:aig:norep:models:lab_vs_nnfpol}
        \end{subfigure}\hfill
        \begin{subfigure}[t]{0.29\textwidth}
            \includegraphics[width=.85\textwidth]{plots/msat/bool/no-rep/aig/time_compare_LAB_vs_LABELNEG_POL.pdf}%
            \label{fig:plt:aig:norep:time:lab_vs_pol}
        \end{subfigure}\hfill
        \begin{subfigure}[t]{0.29\textwidth}
            \includegraphics[width=.85\textwidth]{plots/msat/bool/no-rep/aig/time_compare_LABELNEG_POL_vs_NNF_MUTEX_POL.pdf}%
            \label{fig:plt:aig:norep:time:pol_vs_nnfpol}
        \end{subfigure}\hfill
        \begin{subfigure}[t]{0.29\textwidth}
            \includegraphics[width=.85\textwidth]{plots/msat/bool/no-rep/aig/time_compare_LAB_vs_NNF_MUTEX_POL.pdf}%
            \label{fig:plt:aig:norep:time:lab_vs_nnfpol}
            % \end{subfigure}
        \end{subfigure}
        \caption{Results for disjoint enumeration.
            %\TseitinCNF{},\ \PlaistedCNF{} and $\NNFPlaisted{}$ reported 79, 73 and 72 timeouts, respectively.
        }%
        \label{fig:plt:aig:norep:scatter}
    \end{subfigure}
    %%%%%%%%%%%% REP %%%%%%%%%%%%%
    \begin{subfigure}[t]{\textwidth}
        \begin{subfigure}[t]{0.29\textwidth}
            \includegraphics[width=.85\textwidth]{plots/msat/bool/rep/aig/models_compare_LAB_vs_LABELNEG_POL.pdf}%
            \label{fig:plt:aig:rep:models:lab_vs_pol}
        \end{subfigure}\hfill
        \begin{subfigure}[t]{0.29\textwidth}
            \includegraphics[width=.85\textwidth]{plots/msat/bool/rep/aig/models_compare_LABELNEG_POL_vs_NNF_MUTEX_POL.pdf}%
            \label{fig:plt:aig:rep:models:pol_vs_nnfpol}
        \end{subfigure}\hfill
        \begin{subfigure}[t]{0.29\textwidth}
            \includegraphics[width=.85\textwidth]{plots/msat/bool/rep/aig/models_compare_LAB_vs_NNF_MUTEX_POL.pdf}%
            \label{fig:plt:aig:rep:models:lab_vs_nnfpol}
        \end{subfigure}\hfill
        \begin{subfigure}[t]{0.29\textwidth}
            \includegraphics[width=.85\textwidth]{plots/msat/bool/rep/aig/time_compare_LAB_vs_LABELNEG_POL.pdf}%
            \label{fig:plt:aig:rep:time:lab_vs_pol}
        \end{subfigure}\hfill
        \begin{subfigure}[t]{0.29\textwidth}
            \includegraphics[width=.85\textwidth]{plots/msat/bool/rep/aig/time_compare_LABELNEG_POL_vs_NNF_MUTEX_POL.pdf}%
            \label{fig:plt:aig:rep:time:pol_vs_nnfpol}
        \end{subfigure}\hfill
        \begin{subfigure}[t]{0.29\textwidth}
            \includegraphics[width=.85\textwidth]{plots/msat/bool/rep/aig/time_compare_LAB_vs_NNF_MUTEX_POL.pdf}%
            \label{fig:plt:aig:rep:time:lab_vs_nnfpol}
            % \end{subfigure}
        \end{subfigure}
        \caption{Results for non-disjoint enumeration.
            %\TseitinCNF{},\ \PlaistedCNF{} and $\NNFPlaisted{}$ reported 78, 60 and 51 timeouts, respectively.
        }%
        \label{fig:plt:aig:rep:scatter}
    \end{subfigure}
    \caption{Results on the AIG benchmarks using \mathsat{}.
            Scatter plots in~\ref{fig:plt:aig:norep:scatter} and~\ref{fig:plt:aig:rep:scatter} compare CNF-izations by \TAna{} size (first row) and execution time (second row).
            Points on dashed lines represent timeouts, summarized in~\ref{tab:timeouts:bool}.
            All axes use a logarithmic scale.}%
    \label{fig:plt:aig:scatter}
\end{figure}
% ---- AllSAT Tabula ----

\begin{figure}
    \centering
    % \begin{subfigure}[t]{\textwidth}
    \begin{subfigure}[t]{\textwidth}
        \begin{subfigure}[t]{0.29\textwidth}
            \includegraphics[width=.85\textwidth]{plots/tabularallsat/syn-bool/models_compare_LAB_vs_LABELNEG_POL.pdf}%
            \label{fig:plt:tabula:syn:bool:norep:models:lab_vs_pol}
        \end{subfigure}\hfill
        \begin{subfigure}[t]{0.29\textwidth}
            \includegraphics[width=.85\textwidth]{plots/tabularallsat/syn-bool/models_compare_LABELNEG_POL_vs_NNF_MUTEX_POL.pdf}%
            \label{fig:plt:tabula:syn:bool:norep:models:pol_vs_nnfpol}
        \end{subfigure}\hfill
        \begin{subfigure}[t]{0.29\textwidth}
            \includegraphics[width=.85\textwidth]{plots/tabularallsat/syn-bool/models_compare_LAB_vs_NNF_MUTEX_POL.pdf}%
            \label{fig:plt:tabula:syn:bool:norep:models:lab_vs_nnfpol}
        \end{subfigure}\hfill
        \begin{subfigure}[t]{0.29\textwidth}
            \includegraphics[width=.85\textwidth]{plots/tabularallsat/syn-bool/time_compare_LAB_vs_LABELNEG_POL.pdf}%
            \label{fig:plt:tabula:syn:bool:norep:time:lab_vs_pol}
        \end{subfigure}\hfill
        \begin{subfigure}[t]{0.29\textwidth}
            \includegraphics[width=.85\textwidth]{plots/tabularallsat/syn-bool/time_compare_LABELNEG_POL_vs_NNF_MUTEX_POL.pdf}%
            \label{fig:plt:tabula:syn:bool:norep:time:pol_vs_nnfpol}
        \end{subfigure}\hfill
        \begin{subfigure}[t]{0.29\textwidth}
            \includegraphics[width=.85\textwidth]{plots/tabularallsat/syn-bool/time_compare_LAB_vs_NNF_MUTEX_POL.pdf}%
            \label{fig:plt:tabula:syn:bool:norep:time:lab_vs_nnfpol}
            % \end{subfigure}
        \end{subfigure}
        \caption{Results for disjoint enumeration. %\TseitinCNF{}, \PlaistedCNF{} and $\NNFPlaisted{}$ reported 163, 94 and 20 timeouts, respectively (points on the dashed lines).
        }%
        \label{fig:plt:tabula:syn:bool:norep:scatter}
    \end{subfigure}
    \caption{Results on the Boolean synthetic benchmarks using \tabularallsat{}.
            Scatter plots in~\ref{fig:plt:tabula:syn:bool:norep:scatter} compare CNF-izations by \TAna{} size (first row) and execution time (second row).
            Points on dashed lines represent timeouts, summarized in~\ref{tab:timeouts:tabula:bool}.
            All axes use a logarithmic scale.}%
    \label{fig:plt:tabula:syn:bool:scatter}
\end{figure}

\begin{figure}
    \centering
    % \begin{subfigure}[t]{\textwidth}
    \begin{subfigure}[t]{\textwidth}
        \begin{subfigure}[t]{0.29\textwidth}
            \includegraphics[width=.85\textwidth]{plots/tabularallsat/iscas85/models_compare_LAB_vs_LABELNEG_POL.pdf}%
            \label{fig:plt:tabula:circ:norep:models:lab_vs_pol}
        \end{subfigure}\hfill
        \begin{subfigure}[t]{0.29\textwidth}
            \includegraphics[width=.85\textwidth]{plots/tabularallsat/iscas85/models_compare_LABELNEG_POL_vs_NNF_MUTEX_POL.pdf}%
            \label{fig:plt:tabula:circ:norep:models:pol_vs_nnfpol}
        \end{subfigure}\hfill
        \begin{subfigure}[t]{0.29\textwidth}
            \includegraphics[width=.85\textwidth]{plots/tabularallsat/iscas85/models_compare_LAB_vs_NNF_MUTEX_POL.pdf}%
            \label{fig:plt:tabula:circ:norep:models:lab_vs_nnfpol}
        \end{subfigure}\hfill
        \begin{subfigure}[t]{0.29\textwidth}
            \includegraphics[width=.85\textwidth]{plots/tabularallsat/iscas85/time_compare_LAB_vs_LABELNEG_POL.pdf}%
            \label{fig:plt:tabula:circ:norep:time:lab_vs_pol}
        \end{subfigure}\hfill
        \begin{subfigure}[t]{0.29\textwidth}
            \includegraphics[width=.85\textwidth]{plots/tabularallsat/iscas85/time_compare_LABELNEG_POL_vs_NNF_MUTEX_POL.pdf}%
            \label{fig:plt:tabula:circ:norep:time:pol_vs_nnfpol}
        \end{subfigure}\hfill
        \begin{subfigure}[t]{0.29\textwidth}
            \includegraphics[width=.85\textwidth]{plots/tabularallsat/iscas85/time_compare_LAB_vs_NNF_MUTEX_POL.pdf}%
            \label{fig:plt:tabula:circ:norep:time:lab_vs_nnfpol}
            % \end{subfigure}
        \end{subfigure}
        \caption{Results for disjoint enumeration. %\TseitinCNF{}, \PlaistedCNF{} and $\NNFPlaisted{}$ reported 49, 44 and 27 timeouts, respectively (points on the dashed lines).
        }%
        \label{fig:plt:tabula:circ:norep:scatter}
    \end{subfigure}
    \caption{Results on the ISCAS'85 benchmarks using \tabularallsat{}.
            Scatter plots in~\ref{fig:plt:tabula:circ:norep:scatter} compare CNF-izations by \TAna{} size (first row) and execution time (second row).
            Points on dashed lines represent timeouts, summarized in~\ref{tab:timeouts:tabula:bool}.
            All axes use a logarithmic scale.}%
    \label{fig:plt:tabula:circ:scatter}
\end{figure}
\begin{figure}
    \centering
    % \begin{subfigure}[t]{\textwidth}
    \begin{subfigure}[t]{\textwidth}
        \begin{subfigure}[t]{0.29\textwidth}
            \includegraphics[width=.85\textwidth]{plots/tabularallsat/aig/models_compare_LAB_vs_LABELNEG_POL.pdf}%
            \label{fig:plt:tabula:aig:norep:models:lab_vs_pol}
        \end{subfigure}\hfill
        \begin{subfigure}[t]{0.29\textwidth}
            \includegraphics[width=.85\textwidth]{plots/tabularallsat/aig/models_compare_LABELNEG_POL_vs_NNF_MUTEX_POL.pdf}%
            \label{fig:plt:tabula:aig:norep:models:pol_vs_nnfpol}
        \end{subfigure}\hfill
        \begin{subfigure}[t]{0.29\textwidth}
            \includegraphics[width=.85\textwidth]{plots/tabularallsat/aig/models_compare_LAB_vs_NNF_MUTEX_POL.pdf}%
            \label{fig:plt:tabula:aig:norep:models:lab_vs_nnfpol}
        \end{subfigure}\hfill
        \begin{subfigure}[t]{0.29\textwidth}
            \includegraphics[width=.85\textwidth]{plots/tabularallsat/aig/time_compare_LAB_vs_LABELNEG_POL.pdf}%
            \label{fig:plt:tabula:aig:norep:time:lab_vs_pol}
        \end{subfigure}\hfill
        \begin{subfigure}[t]{0.29\textwidth}
            \includegraphics[width=.85\textwidth]{plots/tabularallsat/aig/time_compare_LABELNEG_POL_vs_NNF_MUTEX_POL.pdf}%
            \label{fig:plt:tabula:aig:norep:time:pol_vs_nnfpol}
        \end{subfigure}\hfill
        \begin{subfigure}[t]{0.29\textwidth}
            \includegraphics[width=.85\textwidth]{plots/tabularallsat/aig/time_compare_LAB_vs_NNF_MUTEX_POL.pdf}%
            \label{fig:plt:tabula:aig:norep:time:lab_vs_nnfpol}
            % \end{subfigure}
        \end{subfigure}
        \caption{Results for disjoint enumeration.
            %\TseitinCNF{},\ \PlaistedCNF{} and $\NNFPlaisted{}$ reported 79, 73 and 72 timeouts, respectively.
        }%
        \label{fig:plt:tabula:aig:norep:scatter}
    \end{subfigure}
    \caption{Results on the AIG benchmarks using \tabularallsat{}.
            Scatter plots in~\ref{fig:plt:tabula:aig:norep:scatter} compare CNF-izations by \TAna{} size (first row) and execution time (second row).
            Points on dashed lines represent timeouts, summarized in~\ref{tab:timeouts:tabula:bool}.
            All axes use a logarithmic scale.}%
    \label{fig:plt:tabula:aig:scatter}
\end{figure}

% ---- ALLSMT Mathsat ----
\begin{figure}
    \centering
    % \begin{subfigure}[t]{\textwidth}
    \begin{subfigure}[t]{\textwidth}
        \begin{subfigure}[t]{0.29\textwidth}
            \includegraphics[width=.85\textwidth]{plots/msat/lra/no-rep/syn-lra/models_compare_LAB_vs_LABELNEG_POL.pdf}%
            \label{fig:plt:syn:lra:norep:models:lab_vs_pol}
        \end{subfigure}\hfill
        \begin{subfigure}[t]{0.29\textwidth}
            \includegraphics[width=.85\textwidth]{plots/msat/lra/no-rep/syn-lra/models_compare_LABELNEG_POL_vs_NNF_MUTEX_POL.pdf}%
            \label{fig:plt:syn:lra:norep:models:pol_vs_nnfpol}
        \end{subfigure}\hfill
        \begin{subfigure}[t]{0.29\textwidth}
            \includegraphics[width=.85\textwidth]{plots/msat/lra/no-rep/syn-lra/models_compare_LAB_vs_NNF_MUTEX_POL.pdf}%
            \label{fig:plt:syn:lra:norep:models:lab_vs_nnfpol}
        \end{subfigure}\hfill
        \begin{subfigure}[t]{0.29\textwidth}
            \includegraphics[width=.85\textwidth]{plots/msat/lra/no-rep/syn-lra/time_compare_LAB_vs_LABELNEG_POL.pdf}%
            \label{fig:plt:syn:lra:norep:time:lab_vs_pol}
        \end{subfigure}\hfill
        \begin{subfigure}[t]{0.29\textwidth}
            \includegraphics[width=.85\textwidth]{plots/msat/lra/no-rep/syn-lra/time_compare_LABELNEG_POL_vs_NNF_MUTEX_POL.pdf}%
            \label{fig:plt:syn:lra:norep:time:pol_vs_nnfpol}
        \end{subfigure}\hfill
        \begin{subfigure}[t]{0.29\textwidth}
            \includegraphics[width=.85\textwidth]{plots/msat/lra/no-rep/syn-lra/time_compare_LAB_vs_NNF_MUTEX_POL.pdf}%
            \label{fig:plt:syn:lra:norep:time:lab_vs_nnfpol}
            % \end{subfigure}
        \end{subfigure}
        \caption{Results for disjoint enumeration. %\TseitinCNF{}, \PlaistedCNF{} and $\NNFPlaisted{}$ reported 171, 98 and 58 timeouts, respectively (points on the dashed lines).
        }%
        \label{fig:plt:syn:lra:norep:scatter}
    \end{subfigure}
    %%%%%%%%%%%% REP %%%%%%%%%%%%%
    \begin{subfigure}[t]{\textwidth}
        \begin{subfigure}[t]{0.29\textwidth}
            \includegraphics[width=.85\textwidth]{plots/msat/lra/rep/syn-lra/models_compare_LAB_vs_LABELNEG_POL.pdf}%
            \label{fig:plt:syn:lra:rep:models:lab_vs_pol}
        \end{subfigure}\hfill
        \begin{subfigure}[t]{0.29\textwidth}
            \includegraphics[width=.85\textwidth]{plots/msat/lra/rep/syn-lra/models_compare_LABELNEG_POL_vs_NNF_MUTEX_POL.pdf}%
            \label{fig:plt:syn:lra:rep:models:pol_vs_nnfpol}
        \end{subfigure}\hfill
        \begin{subfigure}[t]{0.29\textwidth}
            \includegraphics[width=.85\textwidth]{plots/msat/lra/rep/syn-lra/models_compare_LAB_vs_NNF_MUTEX_POL.pdf}%
            \label{fig:plt:syn:lra:rep:models:lab_vs_nnfpol}
        \end{subfigure}\hfill
        \begin{subfigure}[t]{0.29\textwidth}
            \includegraphics[width=.85\textwidth]{plots/msat/lra/rep/syn-lra/time_compare_LAB_vs_LABELNEG_POL.pdf}%
            \label{fig:plt:syn:lra:rep:time:lab_vs_pol}
        \end{subfigure}\hfill
        \begin{subfigure}[t]{0.29\textwidth}
            \includegraphics[width=.85\textwidth]{plots/msat/lra/rep/syn-lra/time_compare_LABELNEG_POL_vs_NNF_MUTEX_POL.pdf}%
            \label{fig:plt:syn:lra:rep:time:pol_vs_nnfpol}
        \end{subfigure}\hfill
        \begin{subfigure}[t]{0.29\textwidth}
            \includegraphics[width=.85\textwidth]{plots/msat/lra/rep/syn-lra/time_compare_LAB_vs_NNF_MUTEX_POL.pdf}%
            \label{fig:plt:syn:lra:rep:time:lab_vs_nnfpol}
            % \end{subfigure}
        \end{subfigure}
        \caption{Results for non-disjoint enumeration. %\TseitinCNF{}, \PlaistedCNF{} and $\NNFPlaisted{}$ reported 168, 95 and 26 timeouts, respectively (points on the dashed lines).
        }%
        \label{fig:plt:syn:lra:rep:scatter}
    \end{subfigure}
    \caption{Results on the \smtlarat{} synthetic benchmarks using \mathsat{}.
            Scatter plots in~\ref{fig:plt:syn:lra:norep:scatter} and \ref{fig:plt:syn:lra:rep:scatter} compare CNF-izations by \TAna{} size (first row) and execution time (second row).
            Points on dashed lines represent timeouts, summarized in~\ref{tab:timeouts:lra}.
            All axes use a logarithmic scale.}%
    \label{fig:plt:syn:lra:scatter}
\end{figure}
%
\begin{figure}
    \centering
    % \begin{subfigure}[t]{\textwidth}
    \begin{subfigure}[t]{\textwidth}
        \begin{subfigure}[t]{0.29\textwidth}
            \includegraphics[width=.85\textwidth]{plots/msat/lra/no-rep/wmi/models_compare_LAB_vs_LABELNEG_POL.pdf}%
            \label{fig:plt:wmi:norep:models:lab_vs_pol}
        \end{subfigure}\hfill
        \begin{subfigure}[t]{0.29\textwidth}
            \includegraphics[width=.85\textwidth]{plots/msat/lra/no-rep/wmi/models_compare_LABELNEG_POL_vs_NNF_MUTEX_POL.pdf}%
            \label{fig:plt:wmi:norep:models:pol_vs_nnfpol}
        \end{subfigure}\hfill
        \begin{subfigure}[t]{0.29\textwidth}
            \includegraphics[width=.85\textwidth]{plots/msat/lra/no-rep/wmi/models_compare_LAB_vs_NNF_MUTEX_POL.pdf}%
            \label{fig:plt:wmi:norep:models:lab_vs_nnfpol}
        \end{subfigure}\hfill
        \begin{subfigure}[t]{0.29\textwidth}
            \includegraphics[width=.85\textwidth]{plots/msat/lra/no-rep/wmi/time_compare_LAB_vs_LABELNEG_POL.pdf}%
            \label{fig:plt:wmi:norep:time:lab_vs_pol}
        \end{subfigure}\hfill
        \begin{subfigure}[t]{0.29\textwidth}
            \includegraphics[width=.85\textwidth]{plots/msat/lra/no-rep/wmi/time_compare_LABELNEG_POL_vs_NNF_MUTEX_POL.pdf}%
            \label{fig:plt:wmi:norep:time:pol_vs_nnfpol}
        \end{subfigure}\hfill
        \begin{subfigure}[t]{0.29\textwidth}
            \includegraphics[width=.85\textwidth]{plots/msat/lra/no-rep/wmi/time_compare_LAB_vs_NNF_MUTEX_POL.pdf}%
            \label{fig:plt:wmi:norep:time:lab_vs_nnfpol}
            % \end{subfigure}
        \end{subfigure}
        \caption{Results for disjoint enumeration. %\TseitinCNF{}, \PlaistedCNF{} and $\NNFPlaisted{}$ reported 24, 23 and 10 timeouts, respectively (points on the dashed lines).
        }%
        \label{fig:plt:wmi:norep:scatter}
    \end{subfigure}
    %%%%%%%%%%%% REP %%%%%%%%%%%%%
    \begin{subfigure}[t]{\textwidth}
        \begin{subfigure}[t]{0.29\textwidth}
            \includegraphics[width=.85\textwidth]{plots/msat/lra/rep/wmi/models_compare_LAB_vs_LABELNEG_POL.pdf}%
            \label{fig:plt:wmi:rep:models:lab_vs_pol}
        \end{subfigure}\hfill
        \begin{subfigure}[t]{0.29\textwidth}
            \includegraphics[width=.85\textwidth]{plots/msat/lra/rep/wmi/models_compare_LABELNEG_POL_vs_NNF_MUTEX_POL.pdf}%
            \label{fig:plt:wmi:rep:models:pol_vs_nnfpol}
        \end{subfigure}\hfill
        \begin{subfigure}[t]{0.29\textwidth}
            \includegraphics[width=.85\textwidth]{plots/msat/lra/rep/wmi/models_compare_LAB_vs_NNF_MUTEX_POL.pdf}%
            \label{fig:plt:wmi:rep:models:lab_vs_nnfpol}
        \end{subfigure}\hfill
        \begin{subfigure}[t]{0.29\textwidth}
            \includegraphics[width=.85\textwidth]{plots/msat/lra/rep/wmi/time_compare_LAB_vs_LABELNEG_POL.pdf}%
            \label{fig:plt:wmi:rep:time:lab_vs_pol}
        \end{subfigure}\hfill
        \begin{subfigure}[t]{0.29\textwidth}
            \includegraphics[width=.85\textwidth]{plots/msat/lra/rep/wmi/time_compare_LABELNEG_POL_vs_NNF_MUTEX_POL.pdf}%
            \label{fig:plt:wmi:rep:time:pol_vs_nnfpol}
        \end{subfigure}\hfill
        \begin{subfigure}[t]{0.29\textwidth}
            \includegraphics[width=.85\textwidth]{plots/msat/lra/rep/wmi/time_compare_LAB_vs_NNF_MUTEX_POL.pdf}%
            \label{fig:plt:wmi:rep:time:lab_vs_nnfpol}
            % \end{subfigure}
        \end{subfigure}
        \caption{Results for non-disjoint enumeration. %\TseitinCNF{}, \PlaistedCNF{} and $\NNFPlaisted{}$ reported 24, 24 and 10 timeouts, respectively (points on the dashed lines).
        }%
        \label{fig:plt:wmi:rep:scatter}
    \end{subfigure}
    \caption{Results on the WMI benchmarks using \mathsat{}.
            Scatter plots in~\ref{fig:plt:wmi:norep:scatter} and \ref{fig:plt:wmi:rep:scatter} compare CNF-izations by \TAna{} size (first row) and execution time (second row).
            Points on dashed lines represent timeouts, summarized in~\ref{tab:timeouts:lra}.
            All axes use a logarithmic scale.}%
    \label{fig:plt:wmi:scatter}
\end{figure}


% ---- AllSMT Tabula ----

\begin{figure}
    \centering
    % \begin{subfigure}[t]{\textwidth}
    \begin{subfigure}[t]{\textwidth}
        \begin{subfigure}[t]{0.29\textwidth}
            \includegraphics[width=.85\textwidth]{plots/tabularallsmt/syn-lra/models_compare_LAB_vs_LABELNEG_POL.pdf}%
            \label{fig:plt:tabula:syn:lra:norep:models:lab_vs_pol}
        \end{subfigure}\hfill
        \begin{subfigure}[t]{0.29\textwidth}
            \includegraphics[width=.85\textwidth]{plots/tabularallsmt/syn-lra/models_compare_LABELNEG_POL_vs_NNF_MUTEX_POL.pdf}%
            \label{fig:plt:tabula:syn:lra:norep:models:pol_vs_nnfpol}
        \end{subfigure}\hfill
        \begin{subfigure}[t]{0.29\textwidth}
            \includegraphics[width=.85\textwidth]{plots/tabularallsmt/syn-lra/models_compare_LAB_vs_NNF_MUTEX_POL.pdf}%
            \label{fig:plt:tabula:syn:lra:norep:models:lab_vs_nnfpol}
        \end{subfigure}\hfill
        \begin{subfigure}[t]{0.29\textwidth}
            \includegraphics[width=.85\textwidth]{plots/tabularallsmt/syn-lra/time_compare_LAB_vs_LABELNEG_POL.pdf}%
            \label{fig:plt:tabula:syn:lra:norep:time:lab_vs_pol}
        \end{subfigure}\hfill
        \begin{subfigure}[t]{0.29\textwidth}
            \includegraphics[width=.85\textwidth]{plots/tabularallsmt/syn-lra/time_compare_LABELNEG_POL_vs_NNF_MUTEX_POL.pdf}%
            \label{fig:plt:tabula:syn:lra:norep:time:pol_vs_nnfpol}
        \end{subfigure}\hfill
        \begin{subfigure}[t]{0.29\textwidth}
            \includegraphics[width=.85\textwidth]{plots/tabularallsmt/syn-lra/time_compare_LAB_vs_NNF_MUTEX_POL.pdf}%
            \label{fig:plt:tabula:syn:lra:norep:time:lab_vs_nnfpol}
            % \end{subfigure}
        \end{subfigure}
        \caption{Results for disjoint enumeration. %\TseitinCNF{}, \PlaistedCNF{} and $\NNFPlaisted{}$ reported 163, 94 and 20 timeouts, respectively (points on the dashed lines).
        }%
        \label{fig:plt:tabula:syn:lra:norep:scatter}
    \end{subfigure}
    \caption{Results on the \smtlarat{} synthetic benchmarks using \tabularallsmt{}.
            Scatter plots in~\ref{fig:plt:tabula:syn:lra:norep:scatter} compare CNF-izations by \TAna{} size (first row) and execution time (second row).
            Points on dashed lines represent timeouts, summarized in~\ref{tab:timeouts:tabula:lra}.
            All axes use a logarithmic scale.}%
    \label{fig:plt:tabula:syn:lra:scatter}
\end{figure}

\begin{figure}
    \centering
    % \begin{subfigure}[t]{\textwidth}
    \begin{subfigure}[t]{\textwidth}
        \begin{subfigure}[t]{0.29\textwidth}
            \includegraphics[width=.85\textwidth]{plots/tabularallsmt/wmi/models_compare_LAB_vs_LABELNEG_POL.pdf}%
            \label{fig:plt:tabula:wmi:norep:models:lab_vs_pol}
        \end{subfigure}\hfill
        \begin{subfigure}[t]{0.29\textwidth}
            \includegraphics[width=.85\textwidth]{plots/tabularallsmt/wmi/models_compare_LABELNEG_POL_vs_NNF_MUTEX_POL.pdf}%
            \label{fig:plt:tabula:wmi:norep:models:pol_vs_nnfpol}
        \end{subfigure}\hfill
        \begin{subfigure}[t]{0.29\textwidth}
            \includegraphics[width=.85\textwidth]{plots/tabularallsmt/wmi/models_compare_LAB_vs_NNF_MUTEX_POL.pdf}%
            \label{fig:plt:tabula:wmi:norep:models:lab_vs_nnfpol}
        \end{subfigure}\hfill
        \begin{subfigure}[t]{0.29\textwidth}
            \includegraphics[width=.85\textwidth]{plots/tabularallsmt/wmi/time_compare_LAB_vs_LABELNEG_POL.pdf}%
            \label{fig:plt:tabula:wmi:norep:time:lab_vs_pol}
        \end{subfigure}\hfill
        \begin{subfigure}[t]{0.29\textwidth}
            \includegraphics[width=.85\textwidth]{plots/tabularallsmt/wmi/time_compare_LABELNEG_POL_vs_NNF_MUTEX_POL.pdf}%
            \label{fig:plt:tabula:wmi:norep:time:pol_vs_nnfpol}
        \end{subfigure}\hfill
        \begin{subfigure}[t]{0.29\textwidth}
            \includegraphics[width=.85\textwidth]{plots/tabularallsmt/wmi/time_compare_LAB_vs_NNF_MUTEX_POL.pdf}%
            \label{fig:plt:tabula:wmi:norep:time:lab_vs_nnfpol}
            % \end{subfigure}
        \end{subfigure}
        \caption{Results for disjoint enumeration. %\TseitinCNF{}, \PlaistedCNF{} and $\NNFPlaisted{}$ reported 49, 44 and 27 timeouts, respectively (points on the dashed lines).
        }%
        \label{fig:plt:tabula:wmi:norep:scatter}
    \end{subfigure}
    \caption{Results on the WMI benchmarks using \tabularallsmt{}.
            Scatter plots in~\ref{fig:plt:tabula:wmi:norep:scatter} compare CNF-izations by \TAna{} size (first row) and execution time (second row).
            Points on dashed lines represent timeouts, summarized in~\ref{tab:timeouts:tabula:lra}.
            All axes use a logarithmic scale.}%
    \label{fig:plt:tabula:wmi:scatter}
\end{figure}