In this section, we propose a possible solution to address the
shortcomings of \TseitinCNF{} and \PlaistedCNF{} CNF-izations in
%\ignoreinlong{model}\ignoreinshort{
SAT and SMT %} 
enumeration, described in~\sref{sec:problem}. We show that a
simple preprocessing can avoid this situation. We transform first
the input formula into an NNF DAG.\@
In fact, NNF guarantees that each sub-formula occurs only
positively, as every sub-formula $\vi_i$ occurring with double
polarity is converted into two syntactically-different
sub-formulas $\poslab{\vi_i}\defas\NNF{\vi_i}$ and
$\neglab{\vi_i}\defas\NNF{\neg\vi_i}$ ---each occurring only
positively--- which are then labelled ---with single
implications--- with two distinct atoms $\poslab{B_i}$ and
$\neglab{B_i}$ respectively.
%    \begin{rschange}
To improve the efficiency of the enumeration procedure without affecting its
outcome, we also add the clauses $(\neg
    \poslab{B_i}\vee\neg\neglab{B_i})$ when both $\poslab{B_i}$ and
$\neglab{B_i}$ are introduced, which prevent the solver from assigning both $\poslab{B_i}$ and $\neglab{B_i}$ to true, and thus from exploring inconsistent search branches.
%    \end{rschange}



%    \begin{gmchange}
We remark that even with this preprocessing we produce a linear-size CNF encoding, since the $\NNF{\vi}$ DAG has linear size w.r.t.\ $\vi$ (see~\sref{sec:background:propositional-logic}), and \PlaistedCNF{} introduces one label definition for each DAG node, each consisting of 1 or 2 clauses.
%    \end{gmchange}
We illustrate the benefit of this additional preprocessing with the following example.

\begin{example}%
    \label{ex3}
    Consider the formula $\vi$ of~\cref{ex1}. By converting it into NNF, we obtain:
    \begin{equation}
        \vinnf\defas
        \overbrace{(A_1\wedge A_2)}^{B_1}\vee
        \underbrace{(
        \overbrace{(
        \underbrace{(
        \overbrace{(\neg A_3\wedge\neg A_4)}^{\neglab{B_2}}\vee
        \overbrace{(\neg A_5\wedge\neg A_6)}^{\neglab{B_3}}
        )}_{\neglab{B_4}}\vee A_7
        )}^{B_5} \wedge
        \overbrace{(
        \underbrace{(
        \overbrace{(     A_3\vee       A_4)}^{\poslab{B_2}}\wedge
        \overbrace{(     A_5\vee       A_6)}^{\poslab{B_3}}
        )}_{\poslab{B_4}}\vee\neg A_7
        )}^{B_6}
        )}_{B_{7}}
    \end{equation}
    % We remark that, by definition, each sub-formula of $\vinnf$ occurs only with positive polarity. For instance, consider the sub-formula $(A_3\vee A_4)$ that occurs in $\vi$ with double polarity. In $\vinnf$, the positive occurrence remains the same, while the negative occurrence is converted into $(\neg A_3\wedge\neg A_4)$. This implies that the two occurrences correspond to two different sub-formula, each occurring only positively, and thus they will be labelled with two different atoms.
    Suppose, then, that the formula is converted into CNF using \PlaistedCNF{}. Then, the following CNF formula is obtained:
    \begin{subequations}%
        \label{eq:ex3:vicnf}
        \begin{alignat}{2}
             & \vinnfcnfpg\defas                                                                                                                                                                                                \\
             & \qquad(\neg B_1\vee\pos A_1)\wedge(\neg B_1\vee\pos A_2)                                     & \wedge
             & \quad\eqcomment{(B_1\imp (\pos A_1\wedge\pos A_2))}\label{eq:ex3:vicnf:line1}                                                                                                                                    \\
             & \qquad(\neg \neglab{B_2}\vee\neg A_3)\wedge(\neg \neglab{B_2}\vee\neg A_4)                   & \wedge & \quad\eqcomment{(\neglab{B_2}\imp (\neg A_3\wedge\neg A_4))}\label{eq:ex3:vicnf:line2}                   \\
             & \qquad(\neg \neglab{B_3}\vee\neg A_5)\wedge(\neg \neglab{B_3}\vee\neg A_6)                   & \wedge & \quad\eqcomment{(\neglab{B_3}\imp (\neg A_5\wedge\neg A_6))}\label{eq:ex3:vicnf:line3}                   \\
             & \qquad(\neg \neglab{B_4}\vee\pos \neglab{B_2}\vee\pos \neglab{B_3})                          & \wedge & \quad\eqcomment{(\neglab{B_4}\imp (\pos \neglab{B_2}\vee\pos \neglab{B_3}))}\label{eq:ex3:vicnf:line4}   \\
             & \qquad(\neg B_5\vee\pos \neglab{B_4}\vee\pos A_7)                                            & \wedge & \quad\eqcomment{(B_5\imp (\pos \neglab{B_4}\vee\pos A_7))}\label{eq:ex3:vicnf:line5}                     \\
             & \qquad(\neg \poslab{B_2}\vee\pos A_3\vee\pos A_4)                                            & \wedge & \quad\eqcomment{(\poslab{B_2}\imp (\pos A_3\vee\pos A_4))}\label{eq:ex3:vicnf:line6}                     \\
             & \qquad(\neg \poslab{B_3}\vee\pos A_5\vee\pos A_6)                                            & \wedge & \quad\eqcomment{(\poslab{B_3}\imp (\pos A_5\vee\pos A_6))}\label{eq:ex3:vicnf:line7}                     \\
             & \qquad(\neg \poslab{B_4}\vee\pos \poslab{B_2})\wedge(\neg \poslab{B_4}\vee\pos \poslab{B_3}) & \wedge & \quad\eqcomment{(\poslab{B_4}\imp (\pos \poslab{B_2}\wedge\pos \poslab{B_3}))}\label{eq:ex3:vicnf:line8} \\
             & \qquad(\neg B_6\vee\pos \poslab{B_4}\vee\neg A_7)                                            & \wedge & \quad\eqcomment{(B_6\imp (\pos \poslab{B_4}\vee\neg A_7))}\label{eq:ex3:vicnf:line9}                     \\
             & \qquad(\neg B_{7}\vee\pos B_5)\wedge(\neg B_{7}\vee\pos B_6)                                 & \wedge & \quad\eqcomment{(B_{7}\imp (\pos B_5\wedge\pos B_6))}\label{eq:ex3:vicnf:line10}                         \\
             & \qquad(\pos B_1\vee\pos B_{7})                                                               & \wedge & \label{eq:ex3:vicnf:line11}                                                                              \\
             & \qquad(\neg \poslab{B_2}\vee \neg \neglab{B_2})                                              & \wedge &                                                                                                          \\
             & \qquad(\neg \poslab{B_3}\vee \neg \neglab{B_3})                                              & \wedge &                                                                                                          \\
             & \qquad(\neg \poslab{B_4}\vee \neg \neglab{B_4})                                              &
        \end{alignat}
    \end{subequations}
    Suppose, e.g., that the solver picks non-deterministic choices, deciding the atoms in the order
    $\set{B_1,A_1,A_2,\neglab{B_3},A_5,A_6,\neglab{B_2},A_3,A_4,\neglab{B_4},B_5,A_7,\poslab{B_2},\poslab{B_3},\poslab{B_4},B_6,B_{7}}$,
    branching with a negative value first.
    Then, the first total truth assignment found is:
    \begin{equation}
        \label{ex3:eta}
        \eta\defas\set{
        \underbrace{\neg B_1, \neglab{B_2},\neg \neglab{B_3},\neglab{B_4},B_5,\neg \poslab{B_2},\neg \poslab{B_3},\neg \poslab{B_4},B_6,B_{7}}_{\etaB},
        \underbrace{\neg A_1,\neg A_2,\neg A_3,\neg A_4,\neg A_5,\neg A_6,\neg A_7}_{\etaA}}.
    \end{equation}

    In this case, the minimization procedure returns $\muA\defas\set{\neg A_3,\neg A_4, \neg A_7}$ as in~\eqref{eq:ex1:muAprime}, achieving full minimization. With this additional preprocessing, in fact, the solver is no longer forced to assign a truth value to $A_5$ or $A_6$.
    This is possible because, even though $(A_5\vee A_6)$ occurs with double polarity in $\vi$, in $\NNF{\vi}$ its positive and negative occurrences are converted into $(A_5\vee A_6)$ and $(\neg A_5\wedge\neg A_6)$ respectively. Since they appear as two syntactically-different sub-formulas, \PlaistedCNF{} labels them ---with single implications--- using two different atoms $\poslab{B_3}$ and $\neglab{B_3}$ respectively. This allows the solver to find a %\ignoreinlong{model}\ignoreinshort{
    truth assignment %} 
    $\eta$ that assigns both $\neglab{B_3}$ and $\poslab{B_3}$ to false. Hence, the clauses in \eqref{eq:ex3:vicnf:line3} and~\eqref{eq:ex3:vicnf:line7} are satisfied even without assigning $A_5$ and $A_6$, and thus these atoms can be dropped by the minimization procedure.
    \exdone{}
\end{example}


\ignore{%%%%%%%%%%%%%%%% from rebuttal
    if a subformula \Phi_i of \Phi is
    made True, False or unassigned by \muA, then the {B_i+,B_i-} are
    assigned {T,F},{F,T},{F,F} respectively.
    The inductive proof that the extended assignment satisfies
    CNF_PG(NNF(\Phi)) follows straightforwardly.
    This example shows that the solver can enumerate shorter partial
    assignments if the formula is converted into NNF before applying
    \PlaistedCNF{}.}

The key idea behind this additional preprocessing is that each sub-formula of $\NNF{\vi}$ occurs only positively, so that \PlaistedCNF{} labels them with single implications, and the solver is no longer forced to assign them a truth value.
Consider a sub-formula $\vi_i$ that occurs with double polarity in
$\vi$.
In $\NNF{\vi}$ the two sub-formulas
$\poslab{\vi_i}\defas\NNF{\vi_i}$ and
$\neglab{\vi_i}\defas\NNF{\neg\vi_i}$ occur only positively.
Then,
instead of adding $(B_i\iff\vi_i)$, we add
$(\poslab{B_i}\imp\poslab{\vi_i})\wedge(\neglab{B_i}\imp\neglab{\vi_i})$,
and the solver can find a truth assignment $\eta$ that assigns both
$\neglab{B_i}$ and $\poslab{B_i}$ to false. (In~\cref{ex3}, instead of
$(B_3\iff(A_5\vee A_6))$ we add $(\poslab{B_3}\imp(A_5\vee A_6))$ and
$(\neglab{B_3}\imp(\neg A_5\wedge\neg A_6))$.) Thus, the clauses
deriving from $\vi_i$ can be satisfied even without assigning a truth
value to $\vi_i$, whose atoms can be dropped by the minimization
procedure ---provided that they are not forced to be assigned by some
other sub-formula of $\vi$. (In the example, by setting
$\etaB(\poslab{B_3})=\etaB(\neglab{B_3})=\bot$, the clauses
in~\eqref{eq:ex3:vicnf:line3} and~\eqref{eq:ex3:vicnf:line7} are
satisfied even without assigning $A_5$ and $A_6$.)

%\begin{rschange}
We have the following general fact:
{\em %\ignoreinlong{every partial model $\mualpha$ for $\vi$ is also a model for}\ignoreinshort{
every partial truth assignment $\mualpha$ satisfying $\vi$, satisfies also %}
$\exists\allB.\vinnfcnfpg$}, that is, if $\mualpha\pmodels\vi$, then there exists $\etaB$ s.t.\
$\mualpha\cup\etaB\pmodels\vinnfcnfpg$.
(The vice versa holds trivially.)\@
% \ignoreinlong{A complete formal proof of this fact is presented in
%     an extended version of this paper \cite{masina_cnf_2023}.
%     Intuitively, it is easy to see that the suitable $\etaB$ is defined  so that,
%     for each sub-formula $\vi_i$ of $\vi$,
%     if $\vi_i$  is made true, false or
%     is unassigned by  $\mualpha$,
%     then   \tuple{\etaB(B_i^+),\etaB(B_i^-)} is
%     %defined as
%     \tuple{\top,\bot}, \tuple{\bot,\top}, or \tuple{\bot,\bot}
%     respectively. }
%\end{rschange}
%
% \ignoreinshort{
This is illustrated by the following theorem, which is proved in~\cref{sec:proofexistsetaB}.
\begin{theorem}%
    \label{th:existsetaB}
    For every partial truth assignment $\mualpha$ s.t.\ $\mualpha\pmodels\vi(\allalpha)$, there exists a total truth assignment \etaB s.t.\ $\mualpha\cup\etaB\pmodels\vinnfcnfpg$.
\end{theorem}
% }

%\ignoreinshort{
Notice that the above fact is agnostic w.r.t.\ the %\ignoreinlong{disjoint-}
AllSAT %\ignoreinshort{\ 
or AllSMT %} 
algorithm adopted %\ignoreinshort{
and holds for both disjoint and non-disjoint enumeration. %}.
% }

% \ignoreinshort{
%\ignore{%%%% tolto perche' secondo me misleading
We %\ignoreinlong{stress the fact}\ignoreinshort{
notice %} 
that this does not guarantee that the
enumeration procedure always finds this $\etaB$, but only that
such $\etaB$ exists.\@
%\ignoreinshort{
E.g., the enumeration procedure of \sref{sec:background:allsat} finds a total truth assignment $\etaA\cup\etaB$ that satisfies the formula, and then finds $\muA\subseteq\etaA$ that is minimal w.r.t.\ that specific $\etaB$ so that $\muA\cup\etaB\pmodels\vinnfcnfpg$, thus the $\etaB$ found is not guaranteed to be the one that allows for the most effective minimization of $\muA$. %}
Ad-hoc enumeration heuristics should be %\ignoreinlong{investigated}\ignoreinshort{
adopted. %}. %\\
% }


\begin{remark}%
    \label{rem:ex3:preconv}
    We %\ignoreinlong{notice}\ignoreinshort{
    stress the fact %} 
    that the pre-conversion into NNF is typically never used in plain SAT %\ignoreinshort{
    or SMT
    %}
    \emph{solving}, because it causes the unnecessary duplication of labels $\poslab{B_i}$ and $\neglab{B_i}$, with extra overhead and no benefit for the solver.
\end{remark}