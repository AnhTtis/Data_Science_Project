In this section we analyse the impact of different CNF-izations on %\ignoreinlong{the AllSAT task}\ignoreinshort{
the enumeration of short partial truth assignments. %}. 
In particular, we focus on \TseitinCNF~\cite{tseitin68} and \PlaistedCNF~\cite{plaisted1986structure}.
% \ignoreinshort{
In the analysis, we refer to AllSAT, but it applies to All\smt{} as well, by restricting to theory-satisfiable truth assignments.\@ Furthermore, the analysis applies to both disjoint and non-disjoint enumeration. %
% }
We point out how CNF-izing AllSAT problems using these transformations can introduce unexpected drawbacks in the enumeration process. % when they are used to preprocess non-CNF formulas. 
In fact, we show that the resulting encodings can prevent the solver from effectively minimizing the truth assignments, and thus from enumerating a small set of short partial truth assignments.

\subsection{The impact of Tseitin CNF transformation}%
\label{sec:problem:label}

We show that preprocessing the input formula using the \TseitinCNF{} transformation~\cite{tseitin68} can be problematic for enumeration.
We first illustrate this issue with an example.
\begin{example}%
    \label{ex1}
    Consider the propositional formula
    \begin{equation}
        \label{eq:ex1:vi}
        % \vi \defas 
        %     \overbrace{(\underbrace{(A_1 \wedge A_2)}_{B_1} \vee  A_3)}^{B_2} \iff
        %     \overbrace{ (\underbrace{(A_4 \wedge A_5 )}_{B_3} \vee  A_6) }^{B_4}
        % \overbrace{(A_1 \wedge A_2)}^{B_1} \vee  
        % \overbrace{(A_3 \iff \underbrace{((A_4\vee A_5) \wedge 
        % \overbrace{(A_6 \vee A_7)}^{B_2})}_{B_3})}^{B_4}
        \vi \defas
        \overbrace{(A_1\wedge A_2)}^{B_1}\vee
        \overbrace{(
            \underbrace{(
                \overbrace{(A_3\vee A_4)}^{B_2}\wedge
                \overbrace{(A_5\vee A_6)}^{B_3}
                )}_{B_4}\iff
            A_7
            )}^{B_5}
    \end{equation}
    over the set of atoms $\allA\defas\set{A_1, A_2, A_3, A_4, A_5, A_6, A_7}$.
    We first notice that the minimal partial truth assignment:
    \begin{equation}
        \label{eq:ex1:muA}
        \muA\defas\set{\neg A_3,\neg A_4,\neg A_7}
    \end{equation}
    suffices to satisfy $\vi$, even though it does not assign a truth value to the sub-formulas $(A_1\wedge A_2)$ and $(A_5\vee A_6)$ since the atoms $A_1, A_2, A_5, A_6$ are not assigned.


    Nevertheless, $\vi$ is not in CNF, and thus it must be CNF-ized by the solver before starting the enumeration process.
    If \TseitinCNF{} is used, then the following CNF formula is obtained:
    \begin{subequations}%
        \label{eq:ex1:vicnf}
        \begin{alignat}{2}
            % \vicnf \defas
            % &(\neg B_1\vee\pos A_1)\wedge(\neg B_1\vee\pos A_2)\wedge(\pos B_1\vee\neg A_1\vee\neg A_2)&\wedge\label{eq:ex1:vicnf:line1}\\
            % &(\pos B_2\vee\neg B_1)\wedge(\pos B_2\vee\neg A_3)\wedge(\neg B_2\vee\pos B_1\vee\pos A_3)&\wedge\label{eq:ex1:vicnf:line2}\\
            % &(\neg B_3\vee\pos A_4)\wedge(\neg B_3\vee\pos A_5)\wedge(\pos B_3\vee\neg A_4\vee\neg A_5)&\wedge\label{eq:ex1:vicnf:line3}\\
            % &(\pos B_2\vee\neg B_3)\wedge(\pos B_2\vee\neg A_6)\wedge(\neg B_2\vee\pos B_3\vee\pos A_6)&\wedge\label{eq:ex1:vicnf:line4}\\
            % &(\neg B_4 \vee\pos B_2)\wedge(\pos B_4\vee\neg B_2)&\label{eq:ex1:vicnf:line5}
            %-------------------------------
            % \psi\defas
            % &(\neg B_1\vee\pos A_1)\wedge(\neg B_1\vee\pos A_2)\wedge(\pos B_1\vee\neg A_1\vee\neg A_2)&\wedge\label{eq:ex1:vicnf:line1}\\
            % &(\pos B_2\vee\neg A_5)\wedge(\pos B_2\vee\neg A_6)\wedge(\neg B_2\vee\pos A_5\vee\pos A_6)&\wedge\label{eq:ex1:vicnf:line2}\\
            % &(\neg B_3\vee\pos A_4)\wedge(\neg B_3\vee\pos B_2)\wedge(\pos B_3\vee\neg A_4\vee\neg B_2)&\wedge\label{eq:ex1:vicnf:line3}\\
            % &(\neg B_4\vee\neg A_3\vee B_3)\wedge(\neg B_4\vee A_3\vee\neg B_3)\wedge
            % ( B_4\vee A_3\vee B_3)\wedge( B_4\vee\neg A_3\vee\neg B_3)
            % &\wedge\label{eq:ex1:vicnf:line4}\\
            % &(\pos B_1\vee\pos B_4)&
            % --------------------------------------
             & \vicnfts\defas\nonumber                                                                                                         \\
             & \enspace(\neg B_1\vee\pos A_1)\wedge(\neg B_1\vee\pos A_2)\wedge(\pos B_1\vee\neg A_1\vee\neg A_2) & \wedge
             & \quad\eqcomment{(B_1\iff (A_1\wedge A_2))}\label{eq:ex1:vicnf:line1}                                                            \\
             & \enspace(\pos B_2\vee\neg A_3)\wedge(\pos B_2\vee\neg A_4)\wedge(\neg B_2\vee\pos A_3\vee\pos A_4) & \wedge
             & \quad\eqcomment{(B_2\iff (A_3\vee A_4))}\label{eq:ex1:vicnf:line2}                                                              \\
             & \enspace(\pos B_3\vee\neg A_5)\wedge(\pos B_3\vee\neg A_6)\wedge(\neg B_3\vee\pos A_5\vee\pos A_6) & \wedge
             & \quad\eqcomment{(B_3\iff (A_5\vee A_6))}\label{eq:ex1:vicnf:line3}                                                              \\
             & \enspace(\neg B_4\vee\pos B_2)\wedge(\neg B_4\vee\pos B_3)\wedge(\pos B_4\vee\neg B_2\vee\neg B_3) & \wedge
             & \quad\eqcomment{(B_4\iff (B_2\wedge B_3))}\label{eq:ex1:vicnf:line4}                                                            \\
             & \enspace(\neg B_5\vee\pos B_4\vee\neg A_7)\wedge(\neg B_5\vee\neg B_4\vee\pos A_7)                 & \wedge
             & \quad\eqcomment{(B_5\iff (B_4\iff A_7))}\label{eq:ex1:vicnf:line5}                                                              \\
             & \enspace(\pos B_5\vee\pos B_4\vee\pos A_7)\wedge(\pos B_5\vee\neg B_4\vee\neg A_7)                 & \wedge\nonumber            %\tag{...}
            \\
             & \enspace(\pos B_1\vee\pos B_5)                                                                     & \label{eq:ex1:vicnf:line6}
        \end{alignat}
    \end{subequations}
    The fresh atoms $\allB\defas\set{B_1, B_2, B_3, B_4, B_5}$ label sub-formulas as in~\eqref{eq:ex1:vi}. The solver proceeds to compute $\TA{\exists\allB.\vicnfts}$ by enumerating the assignments satisfying $\vicnfts$ projected over \allA{}.
    % As described in~\cref{sec:background}, the solver first enumerates a total truth assignment $\mu\defas\muA\cup\etaB$ such that $\mu\pmodels\vicnf$. Then, the minimization step finds a minimal $\muAprime\subseteq\muA$ s.t.\ $\muAprime\cup\etaB\pmodels\vicnf$.
    % Suppose the solver finds the total model
    % \begin{equation}
    %     \mu\defas\set{\underbrace{\neg B_1, B_2, B_3, B_4}_{\etaB}, \underbrace{\neg A_1,\neg A_2, A_3, A_4, A_5, \neg A_6}_{\muA}}
    % \end{equation}
    % and then the minimization step outputs
    % $\muAprime\defas\set{\neg A_1, A_3, A_4, A_5}$ s.t.\ $\muAprime\cup\etaB\pmodels\vicnf$. Notice that $\muAprime$ is minimal, even though it assigns a truth value also to $A_1$. In fact, whereas first two clauses in~\eqref{eq:ex1:vicnf:line1} are satisfied by $\neg B_1$, the last one must be satisfied by assigning either $\neg A_1$ or $\neg A_2$.
    Suppose, e.g., that the solver picks non-deterministic choices, deciding the atoms in the order
    % they appear in $\vicnf$ 
    $\set{B_1, A_1, A_2, B_2, A_3, A_4, B_3, A_5, A_6, B_4, B_5, A_7}$ and branching with negative value first. Then, the first (sorted) total truth assignment found is:
    \begin{equation}
        \label{eq:ex1:eta}
        \eta\defas\set{
            \underbrace{\neg B_1,\neg B_2,\neg B_3,\neg B_4, B_5}_{\etaB},
            \underbrace{\neg A_1,\neg A_2,\neg A_3,\neg A_4,\neg A_5,\neg A_6, \neg A_7}_{\etaA}
        }
    \end{equation}
    which contains $\muA$~\eqref{eq:ex1:muA}.
    The minimization procedure looks for a \emph{minimal} subset
    $\muAprime$ of $\etaA{}$ s.t.\
    $\muAprime\cup\etaB{}\pmodels\vicnfts$.
    One possible output of this procedure is the minimal assignment:
    \begin{equation}%
        \label{eq:ex1:muAprime}
        \muAprime\defas\set{\neg A_1,\neg A_3,\neg A_4,\neg A_5,\neg A_6,\neg A_7}.
    \end{equation}
    %Notice that $\muAprime$ is minimal for $\vicnf$, meaning that any subset $\muAsecond\subset\muAprime$ is s.t.\ $\muAsecond\cup\etaB\not\pmodels\vicnf$. %We remark that this is not a coincidence, in fact, there does not exist any $\etaBprime$ s.t.\ $\muAprime\cup\etaBprime\pmodels\vicnf$.
    %$\residual{\vicnf}{\muAprime\cup\etaB}\neq\top$. 
    We notice that the partial truth assignment
    $\muA$~\eqref{eq:ex1:muA} satisfies $\vi$ and it is s.t.\
    $\muA\subset\muAprime$, but it {\em does not satisfy $\exists\allB.\vicnfts$}.
    In fact, three clauses of $\vicnfts$ in~\eqref{eq:ex1:vicnf:line1} and~\eqref{eq:ex1:vicnf:line3} are not satisfied by $\muA\cup\etaB$, since
    $\residual{\vicnfts}{\muA\cup\etaB}=
        (\neg A_1\vee\neg A_2)\wedge(\neg A_5)\wedge(\neg A_6)$.
    We remark that this is not a coincidence, since {there is no $\etaBprime$ such that $\muA\cup\etaBprime\pmodels\vicnfts$}, because~\eqref{eq:ex1:vicnf:line1} and~\eqref{eq:ex1:vicnf:line3} cannot be satisfied without assigning any atom in $\set{A_1,A_2}$ and $\set{A_5,A_6}$ respectively.

    Finding $\muAprime$~\eqref{eq:ex1:muAprime} instead of $\muA$~\eqref{eq:ex1:muA} clearly causes an %\ignoreinlong{efficiency}\ignoreinshort{
    effectiveness
    %} 
    problem, since finding longer partial truth assignments implies that the total number of enumerated truth assignments could be up to exponentially larger.
    For instance, %\ignoreinshort{
    in the case of disjoint AllSAT, %}
    instead of the single partial assignment $\muA$~\eqref{eq:ex1:muA}, the solver may return the following list of 9 partial assignments satisfying $\exists\allB.\vicnfts$:
    \begin{equation}%
        \label{eq:ex1:muA:all}
        \begin{array}{llllllll}
            \multicolumn{2}{c}{\overbrace{\rule{1.5cm}{0pt}}^{B_1}} &           &           & \multicolumn{2}{c}{\overbrace{\rule{1.5cm}{0pt}}^{B_3}} &           &                        \\
            \{\neg A_1,                                             &           & \neg A_3, & \neg A_4,                                               & \neg A_5, & \neg A_6, & \neg A_7\}
            \quad\eqcomment{\set{\neg B_1,\neg B_3}}                                                                                                                                       \\
            \{\pos A_1,                                             & \neg A_2, & \neg A_3, & \neg A_4,                                               & \neg A_5, & \neg A_6, & \neg A_7\}
            \quad\eqcomment{\set{\neg B_1,\neg B_3}}                                                                                                                                       \\
            \{\neg A_1,                                             &           & \neg A_3, & \neg A_4,                                               & \pos A_5, &           & \neg A_7\}
            \quad\eqcomment{\set{\neg B_1,\pos B_3}}                                                                                                                                       \\
            \{\neg A_1,                                             &           & \neg A_3, & \neg A_4,                                               & \neg A_5, & \pos A_6, & \neg A_7\}
            \quad\eqcomment{\set{\neg B_1,\pos B_3}}                                                                                                                                       \\
            \{\pos A_1,                                             & \neg A_2, & \neg A_3, & \neg A_4,                                               & \pos A_5, &           & \neg A_7\}
            \quad\eqcomment{\set{\neg B_1,\pos B_3}}                                                                                                                                       \\
            \{\pos A_1,                                             & \neg A_2, & \neg A_3, & \neg A_4,                                               & \neg A_5, & \pos A_6, & \neg A_7\}
            \quad\eqcomment{\set{\neg B_1,\pos B_3}}                                                                                                                                       \\
            \{\pos A_1,                                             & \pos A_2, & \neg A_3, & \neg A_4,                                               & \neg A_5, & \neg A_6, & \neg A_7\}
            \quad\eqcomment{\set{\pos B_1,\neg B_3}}                                                                                                                                       \\
            \{\pos A_1,                                             & \pos A_2, & \neg A_3, & \neg A_4,                                               & \pos A_5, &           & \neg A_7\}
            \quad\eqcomment{\set{\pos B_1,\pos B_3}}                                                                                                                                       \\
            \{\pos A_1,                                             & \pos A_2, & \neg A_3, & \neg A_4,                                               & \neg A_5, & \pos A_6, & \neg A_7\}
            \quad\eqcomment{\set{\pos B_1,\pos B_3}}                                                                                                                                       \\
        \end{array}
    \end{equation}
    where $\muAprime$~\eqref{eq:ex1:muAprime} is the first in the list.
    % \ignoreinshort{
    In the case of non-disjoint AllSAT, instead, a possible output is the following:
    \begin{equation}%
        \label{eq:ex1:muA:all:rep}
        \begin{array}{llllllll}
            \multicolumn{2}{c}{\overbrace{\rule{1.5cm}{0pt}}^{B_1}} &           &           & \multicolumn{2}{c}{\overbrace{\rule{1.5cm}{0pt}}^{B_3}} &           &                        \\
            \{\neg A_1,                                             &           & \neg A_3, & \neg A_4,                                               & \neg A_5, & \neg A_6, & \neg A_7\}
            \quad\eqcomment{\set{\neg B_1,\neg B_3}}                                                                                                                                       \\
            \{                                                      & \neg A_2, & \neg A_3, & \neg A_4,                                               & \neg A_5, & \neg A_6, & \neg A_7\}
            \quad\eqcomment{\set{\neg B_1,\neg B_3}}                                                                                                                                       \\
            \{\neg A_1,                                             &           & \neg A_3, & \neg A_4,                                               & \pos A_5, &           & \neg A_7\}
            \quad\eqcomment{\set{\neg B_1,\pos B_3}}                                                                                                                                       \\
            \{\neg A_1,                                             &           & \neg A_3, & \neg A_4,                                               &           & \pos A_6, & \neg A_7\}
            \quad\eqcomment{\set{\neg B_1,\pos B_3}}                                                                                                                                       \\
            \{                                                      & \neg A_2, & \neg A_3, & \neg A_4,                                               & \pos A_5, &           & \neg A_7\}
            \quad\eqcomment{\set{\neg B_1,\pos B_3}}                                                                                                                                       \\
            \{                                                      & \neg A_2, & \neg A_3, & \neg A_4,                                               &           & \pos A_6, & \neg A_7\}
            \quad\eqcomment{\set{\neg B_1,\pos B_3}}                                                                                                                                       \\
            \{\pos A_1,                                             & \pos A_2, & \neg A_3, & \neg A_4,                                               & \neg A_5, & \neg A_6, & \neg A_7\}
            \quad\eqcomment{\set{\pos B_1,\neg B_3}}                                                                                                                                       \\
            \{\pos A_1,                                             & \pos A_2, & \neg A_3, & \neg A_4,                                               & \pos A_5, &           & \neg A_7\}
            \quad\eqcomment{\set{\pos B_1,\pos B_3}}                                                                                                                                       \\
            \{\pos A_1,                                             & \pos A_2, & \neg A_3, & \neg A_4,                                               &           & \pos A_6, & \neg A_7\}
            \quad\eqcomment{\set{\pos B_1,\pos B_3}}                                                                                                                                       \\
        \end{array}
    \end{equation}
    where $\muAprime$~\eqref{eq:ex1:muAprime} is the first in the list. Notice that in this case the assignments may be shorter and not pairwise disjoint.
    % }
    \exdone{}
    % Consider the partial truth assignment $\mu$ and its projection over $\allA$ as follows:
    % \begin{equation}
    %     \label{eq:ex1:mu}
    %     \mu\defas\set{A_3, A_4, A_5, \neg B_1, B_2, B_3, B_4}, \qquad
    %     \muA\defas\set{A_3, A_4, A_5}
    % \end{equation}
    % Whereas $\muA$ evaluates $\vi$ to true, $\mu$ does not evaluate to true $\vicnf$.
    % In fact, $\residual{\vicnf}{\mu}=(\neg A_1\vee\neg A_2)$.
    % Thus, in order to evaluate $\vicnf$ to true, $\mu$ must also assign either $\neg A_1$ or $\neg A_2$.
    %     % Suppose, for instance, that the solver finds $\mu$ and its projection over $\allA$ as follows
    %     % \begin{equation}
    %     %     \mu'\defas\set{\neg A_1, A_3, A_4, A_5, \neg B_1, B_2, B_3, B_4}, \qquad
    %     % \muAprime\defas\set{\neg A_1, A_3, A_4, A_5}
    %     % \end{equation}
    %     % and then adds the blocking clauses $(A_1\vee A_3, )$
    %     % Suppose such an assignment exists. Then, $\mu'$ must assign $B_2$, $B_3$ and $B_4$ to true because of the clauses in~\eqref{eq:ex1:vicnf:line2,,eq:ex1:vicnf:line3,,eq:ex1:vicnf:line4},
    %     % which represent the label definitions $(B_2\iff(B_1\vee A_3))$, $(B_3\iff(A_4\wedge B_2))$ and $(B_4\iff(A_3\iff B_3))$ respectively. 
    %     % Such an assignment does not yet evaluate to true the clauses in~\eqref{eq:ex1:vicnf:line1}, which represents the label definitions $(B_1\iff(A_1\wedge A_2))$.
    %     % In order to do so, $\mu'$ must also assign a truth value to $B_1$. However, if it assigns $B_1$ to true, then both $A_1$ and $A_2$ must be assigned to true. 
    %     % Similarly, if it assigns $B_1$ to false, then either $A_1$ or $A_2$ must be assigned to false.
    %     % Thus, $\mu'$ must assign a truth value also to some other atoms in \allA{} to evaluate $\vicnf$ to true, which contradicts the hypothesis.\exdone{}
\end{example}

%\newpage
The example above shows an intrinsic problem of \TseitinCNF{} when
used for enumeration: {\em if a minimal partial assignment \muA{} suffices
to satisfy
$\vi$, this does not imply that \muA{} suffices to satisfy $\exists
    \allB.\vicnfts$, i.e., that some $\etaB$ exists
such that $\muA\cup\etaB$ satisfies $\vicnfts$}. %\ignoreinlong{~\cite{sebastiani_are_2020}}.
% RS: mi e' venuto in mente che questa cosa, formulata cosi', l'avevo
% gia' della in [18].

% Consider a generic non-CNF formula $\vi(\allA)$ and a minimal
% partial truth assignment $\muA$ that satisfies $\vi$.
% %\begin{rschange}
% {\em Then, there is no guarantee that $\muA$ suffices to
%   satisfy $\exists \allB.\vicnfts$}, i.e., that there exists some $\etaB$
% such that $\muA\cup\etaB$ suffices to satisfy $\vicnfts$.

In fact, consider a generic non-CNF formula $\vi(\allA)$ and a minimal
partial truth assignment $\muA$ that satisfies $\vi$, and
let $\vi_i$ be some sub-formula of \vi{} which is not assigned a truth
value by $\muA$
% suppose $\muA$ does not assign
% a truth value to some sub-formula $\vi_i$
---for
instance, because $\vi_i$ occurs into some positive sub-formula $\vi_i\vee\vi_j$ and $\muA$ satisfies $\vi_j$. (In~\cref{ex1}, $\muA\defas\set{\neg A_3, \neg A_4, \neg A_7}$, $\vi_i\defas(A_1\wedge A_2)$ %and $\vi_j\defas(((A_3\vee A_4)\wedge(A_5\vee A_6))\iff A_7)$
or $\vi_i\defas(A_5\vee A_6)$ respectively.)
Then
\TseitinCNF{}
conjoins to the main formula the definition $(B_i\iff \vi_i)$, so that
every satisfying partial truth assignment $\muAprime$ is forced to
assign a truth value to $\vi_i$ and thus to some of its atoms,
which may not occur in $\muA$, so that $\muAprime\supset\muA$.
%\end{rschange}
%
(In the example, the clauses
in~\eqref{eq:ex1:vicnf:line1} and~\eqref{eq:ex1:vicnf:line3} force
$\muAprime$ to assign a truth value also to $(A_1\wedge A_2)$ and
$(A_5\vee A_6)$ respectively.)

%
Thus, by using \TseitinCNF{}, instead of enumerating one
minimal partial truth assignment $\muA$ for \vi, the solver may {be forced to} enumerate
many partial truth assignments $\muAprime$ that are minimal for
${\exists \allB.}\TseitinCNF(\vi)$ but %\ignoreinshort{
they are %}
not %\ignoreinshort{
minimal %}
for
$\vi$, so that their number can be up to exponentially larger in the
number of unassigned atoms in \muA.
In fact, each such truth assignment \muAprime{} conjoins to \muA{} one of the (up to $2^{|\allA|-|\muA|}$) partial
assignments which are needed to evaluate  to either $\top$ or $\bot$ all unassigned $\vi_i$'s.
(E.g., in~\eqref{eq:ex1:muA:all} %\ignoreinshort{\ 
and~\eqref{eq:ex1:muA:all:rep}, %}, 
the solver enumerates nine
$\muAprime$s by conjoining $\muA$~\eqref{eq:ex1:muA} with an
exhaustive enumeration of partial assignments to $A_1,A_2,A_5,A_6$
that  evaluate $(A_1\wedge A_2)$ and $(A_5\vee A_6)$ to either $\top$ or $\bot$.)
This may drastically affect the effectiveness of the enumeration.


\ignore{%%
    The example above shows an intrinsic problem of \TseitinCNF{} when used for enumeration.\@ Consider a generic non-CNF formula $\vi(\allA)$ and a minimal
    partial truth assignment $\muA$ that satisfies $\vi$ without
    assigning a truth value to some sub-formula $\vi_i$ ---for
    instance, because $\vi_i$ occurs into some positive sub-formula $\vi_i\vee\vi_j$ and $\muA$ satisfies $\vi_j$. (In~\cref{ex1}, $\muA\defas\set{\neg A_3, \neg A_4, \neg A_7}$, $\vi_i\defas(A_1\wedge A_2)$ %and $\vi_j\defas(((A_3\vee A_4)\wedge(A_5\vee A_6))\iff A_7)$
    or $\vi_i\defas(A_5\vee A_6)$ respectively.)
    \em Then, there is no guarantee that $\muA$ suffices to
    satisfy $\exists \allB.\vicnfts$, i.e., that there exists some $\etaB$
    such that $\muA\cup\etaB$ satisfies $\vicnfts$. In fact,
    \TseitinCNF{}
    conjoins to the main formula the definition $(B_i\iff \vi_i)$, so that every satisfying partial truth assignment $\muAprime$ is forced to assign a truth value to $\vi_i$ and to some of its atoms as well. (In the example, the clauses
    in~\eqref{eq:ex1:vicnf:line1} and~\eqref{eq:ex1:vicnf:line3} force
    $\muAprime$ to assign a truth value also to $(A_1\wedge A_2)$ and $(A_5\vee A_6)$ respectively).
    As a consequence, by using \TseitinCNF{}, the solver may enumerate
    partial truth assignments that, although minimal for $\exists \allB.\TseitinCNF(\vi)$, are not minimal for $\vi$, so that their number can be up to exponentially bigger in the number of unassigned atoms, thus affecting the effectiveness of the enumeration.
}
\subsection{The impact of Plaisted and Greenbaum CNF transformation}%
\label{sec:problem:polarity}

We point out how \PlaistedCNF{}~\cite{plaisted1986structure} can be used to solve these issues, but only in part. We first illustrate it with an example.
\begin{example}%
    \label{ex2}
    Consider the formula $\vi$~\eqref{eq:ex1:vi} and the minimal satisfying assignment $\muA$~\eqref{eq:ex1:muA} as in \cref{ex1}. Suppose that $\vi$ is converted into CNF using \PlaistedCNF{}. Then, the following CNF formula is obtained:
    \begin{subequations}%
        \label{eq:ex2:vicnf}
        \begin{alignat}{2}
             & \vicnfpg\defas\nonumber                                                                                                         \\
             & \enspace(\neg B_1\vee\pos A_1)\wedge(\neg B_1\vee\pos A_2)                                         & \wedge
             & \quad\eqcomment{(B_1\imp (A_1\wedge A_2))}\label{eq:ex2:vicnf:line1}                                                            \\
             & \enspace(\pos B_2\vee\neg A_3)\wedge(\pos B_2\vee\neg A_4)\wedge(\neg B_2\vee\pos A_3\vee\pos A_4) & \wedge
             & \quad\eqcomment{(B_2\iff (A_3\vee A_4))}\label{eq:ex2:vicnf:line2}                                                              \\
             & \enspace(\pos B_3\vee\neg A_5)\wedge(\pos B_3\vee\neg A_6)\wedge(\neg B_3\vee\pos A_5\vee\pos A_6) & \wedge
             & \quad\eqcomment{(B_3\iff (A_5\vee A_6))}\label{eq:ex2:vicnf:line3}                                                              \\
             & \enspace(\neg B_4\vee\pos B_2)\wedge(\neg B_4\vee\pos B_3)\wedge(\pos B_4\vee\neg B_2\vee\neg B_3) & \wedge
             & \quad\eqcomment{(B_4\iff (B_2\wedge B_3))}\label{eq:ex2:vicnf:line4}                                                            \\
             & \enspace(\neg B_5\vee\pos B_4\vee\neg A_7)\wedge(\neg B_5\vee\neg B_4\vee\pos A_7)                 & \wedge
             & \quad\eqcomment{(B_5\imp (B_4\iff A_7))}\label{eq:ex2:vicnf:line5}                                                              \\
             & \enspace(\pos B_1\vee\pos B_5)                                                                     & \label{eq:ex2:vicnf:line6}
        \end{alignat}
    \end{subequations}
    We highlight that~\eqref{eq:ex2:vicnf:line1} and~\eqref{eq:ex2:vicnf:line5} are shorter than~\eqref{eq:ex1:vicnf:line1} and~\eqref{eq:ex1:vicnf:line5} respectively, since the corresponding sub-formulas occur only with positive polarity. Suppose, as in~\cref{ex1}, that the solver finds the total truth assignment~$\eta\defas\etaB\cup\etaA$ in~\eqref{eq:ex1:eta}. In this case, one possible output of the minimization procedure is the minimal partial truth assignment:
    \begin{equation}
        \muAsecond\defas\set{\neg A_3,\neg A_4,\neg A_5,\neg A_6,\neg A_7}.
    \end{equation}
    %s.t.\ $\muA$ is minimal and $\muA\cup\etaB\pmodels\vi$.
    %$\residual{\vi}{\muA\cup\etaB}=\top$.
    This assignment is a strict sub-assignment of $\muAprime$
    in~\eqref{eq:ex1:muAprime}, since the atom $A_1$ is not
    assigned. This is possible because the sub-formula $(A_1\wedge
        A_2)$ is labelled by $B_1$ using a single implication, and the
    clauses representing $(B_1\imp(A_1\wedge A_2))$ are satisfied by
    $\etaB(B_1)=\bot$ even without assigning $A_1$ and
    $A_2$. Nevertheless, the assignment $\muA$~in~\eqref{eq:ex1:muA}
    %\GMCHANGE{, which is s.t.\ $\muA\subset\muAsecond$,}
    that satisfies $\vi$ {\em still does not satisfy
            $\exists\allB.\vicnfpg$}.

    Indeed, sub-formulas occurring with double polarity are labelled using double implications as for \TseitinCNF{}, raising the same problems as the latter. For instance, the sub-formula $(A_5\vee A_6)$ occurs with double polarity, since it is under the scope of an ``$\iff$''. Hence, the clauses in~\eqref{eq:ex2:vicnf:line3} must be satisfied by assigning a truth value also to $A_5$ or $A_6$, and so the partial truth assignment $\muA$ in~\eqref{eq:ex1:muA} does not suffice to satisfy $\exists\allB.\vicnfpg$. %because $\residual{\vicnfpg}{\muA\cup\etaB}=(\neg A_5)\wedge(\neg A_6)$.
    \exdone{}
\end{example}

The example above shows that \PlaistedCNF{} has an advantage over
\TseitinCNF{} when enumerating partial assignments, but it overcomes
its effectiveness issues only in part, {\em because a minimal assignment
        $\muA$ satisfying $\vi$ may not suffice to satisfy
        $\exists\allB.\vicnfpg$}, as with \TseitinCNF.

Consider, as in~\sref{sec:problem:label}, a generic non-CNF formula $\vi(\allA)$ and a partial truth assignment $\muA$ that satisfies $\vi$ without assigning a truth value to some sub-formula $\vi_i$. Suppose that $\vi_i$ occurs only positively in $\vi$ ---for the negative case the reasoning is dual. (In~\cref{ex2}, $\muA\defas\set{\neg A_3, \neg A_4, \neg A_7}$, $\vi_i\defas(A_1\wedge A_2)$.)\@
% and $\vi_j\defas((A_3\vee A_4)\wedge(A_5\vee A_6))\iff A_7$
Since \PlaistedCNF{} introduces only the clauses representing $(B_i\imp\vi_i)$ ---and not those representing $(B_i\limp\vi_i)$--- the solver is no longer forced to assign a truth value to $\vi_i$, because it suffices to assign $\etaB(B_i)=\bot$. (In the example, $(A_1\wedge A_2)$ is labelled with $B_1$ in~\eqref{eq:ex2:vicnf:line1}.) In this case, $\vi_i$ plays the role of a ``don't care'' term, and this property allows for the enumeration of shorter partial assignments.

Nevertheless, a sub-formula can be ``don't care'' only if it occurs with single polarity.
In fact, if $\vi_i$ occurs with double polarity ---as it is the case, e.g., of sub-formulas under the scope of an ``$\iff$''--- then $\vi_i$ is labelled with a double implication $(B_i\iff\vi_i)$, yielding the same drawbacks as with \TseitinCNF{}. (In the example, $(A_5\vee A_6)$ occurs with double polarity, and $\muAprime$ is forced to assign a truth value also to $A_5$ or $A_6$ to satisfy the clauses in~\eqref{eq:ex2:vicnf:line3}.)

Notice that, to maximize the benefits of \PlaistedCNF{}, the sub-formulas that should be treated as ``don't care'' must have their label assigned to false. {In practice, this can be achieved in part by instructing the solver to split on negative values in decision branches\footnote{To exploit this heuristic also for sub-formulas occurring only negatively, the latter can be labelled with a negative label $\neg B_i$ as $(\neg B_i\limp\vi_i)$.}.
        Even though the solver is not guaranteed to always assign to false the labels of ``don't care'' sub-formulas, we empirically verified that this heuristic provides a good approximation of this behaviour.}


