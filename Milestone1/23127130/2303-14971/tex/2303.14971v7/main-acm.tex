%% This is a JAIR Example File Compiled by Nicholas Mattei (nsmattei@tulane.edu) 
%% and Odd Erik Gundersen (odderik@ntnu.no)
%% and Mykel Kochenderfer (mykel@stanford.edu)
%% March 29, 2025
%%
%% This file is based off the ACM Latex Template https://www.acm.org/publications/proceedings-template
%% Revision 2.12 (12/28/2024)
%% 
%% Please see https://www.jair.org/index.php/jair for more information and submission instructions.
%%

%% The first command in your LaTeX source must be the \documentclass
%% command.
%%
%% For submission and review of your manuscript please change the
%% command to \documentclass[manuscript, screen, review]{jair}.
%%
\PassOptionsToPackage{fleqn}{amsmath}
% \PassOptionsToPackage{showframe}{geometry}
\documentclass[manuscript]{jair}
\makeatletter
\newcounter{subterm}[equation] % Define a new counter, reset with every equation
\renewcommand{\thesubterm}{\theequation\alph{subterm}} % New subterm counter is \theequation with alphabetic subterm numbering suffix

\newcommand{\subtermeqlabel}[1]{%
    \refstepcounter{subterm}% Step the counter
    \textup{\tagform@{\thesubterm}}% Display the tag
    \protected@write\@auxout{}{%
        \string\newlabel{#1}{{\thesubterm}{\thepage}{\@currentlabelname}{\@currentHref}{}}%
    }%
}
\makeatother
\usepackage{amsthm}
\usepackage{amsfonts}
\usepackage{environ}
\usepackage{xspace}
\usepackage{xcolor}
\usepackage{graphicx}
\usepackage{ifthen}
\usepackage{algorithm}
\usepackage[noend]{algorithmic}
\usepackage[capitalise, noabbrev]{cleveref}
\usepackage{subcaption}
\usepackage{placeins}
\usepackage{pifont}
\usepackage{url}
\usepackage{tabularx}
\usepackage{tikz}
\usepackage{multirow}
\usepackage[draft]{todonotes}
\usepackage[shortlabels, inline]{enumitem}

\usetikzlibrary{decorations.pathmorphing}
\usetikzlibrary{positioning}
\usetikzlibrary{fit}
\input{rs_macros_general.tex}

\setcopyright{cc}
\copyrightyear{2025}
\acmYear{2025}
\acmDOI{10.1613/jair.1.xxxxx}

%%
\JAIRAE{Kuldeep Meel}
\JAIRTrack{Articles}
\acmVolume{4}
\acmArticle{111}
\acmMonth{8}
\acmYear{2025}

%%
%% For managing citations, it is recommended to use bibliography
%% files in BibTeX format.
%%
%% You can then either use BibTeX with the ACM-Reference-Format style,
%% or BibLaTeX with the acmnumeric or acmauthoryear sytles, that include
%% support for advanced citation of software artefact from the
%% biblatex-software package, also separately available on CTAN.
%%
%% Look at the sample-*-biblatex.tex files for templates showcasing
%% the biblatex styles.
%%

%%
%% The majority of ACM publications use numbered citations and
%% references.  The command \citestyle{authoryear} switches to the
%% "author year" style.
%%
%% If you are preparing content for an event
%% sponsored by ACM SIGGRAPH, you must use the "author year" style of
%% citations and references.
%% Uncommenting
%% the next command will enable that style.
%%\citestyle{acmauthoryear}


%%
%% end of the preamble, start of the body of the document source.
\begin{document}

%%
%% The "title" command has an optional parameter,
%% allowing the author to define a "short title" to be used in page headers.
%\title{JAIR Example Template}
\title{On CNF Conversion for SAT and SMT Enumeration}

%%
%% The "author" command and its associated commands are used to define
%% the authors and their affiliations.
%% Of note is the shared affiliation of the first two authors, and the
%% "authornote" and "authornotemark" commands
%% used to denote shared contribution to the research and/or corresponding author.
\author{Gabriele Masina}
\authornote{Corresponding Author.}
\email{gabriele.masina@unitn.it}
\orcid{0000-0001-8842-4913}
% \orcid{https://orcid.org/0000-0002-2037-3694} % This works too if you want to see the full URL
\affiliation{%
  \institution{University of Trento, DISI}
  \city{Trento}
  % \state{British Columbia}
  \country{Italy}
}

\author{Giuseppe Spallitta}
\orcid{0000-0002-4321-4995}
\email{gs81@rice.edu}
\affiliation{%
  \institution{Rice University}
  \city{Houston}
  \state{Texas}
  \country{USA}}

\author{Roberto Sebastiani}
\orcid{0000-0002-0989-6101}
\email{roberto.sebastiani@unitn.it}
\affiliation{%
  \institution{University of Trento, DISI}
  \city{Trento}
  % \state{British Columbia}
  \country{Italy}
}

%\author{Ben Trovato}

%%
%% By default, the full list of authors will be used in the page
%% headers. Often, this list is too long, and will overlap
%% other information printed in the page headers. This command allows
%% the author to define a more concise list
%% of authors' names for this purpose.
\renewcommand{\shortauthors}{Masina et al.}

%%
%% The abstract is a short summary of the work to be presented in the
%% article.
\begin{abstract}
  Modern SAT and SMT solvers are designed to handle problems expressed in Conjunctive Normal Form (CNF) so that non-CNF problems must be CNF-ized upfront, typically by using variants of either Tseitin or Plaisted and Greenbaum transformations.
  %
  When passing from plain solving to enumeration, however, the capability of
  producing partial satisfying assignments that are as small as possible becomes
  crucial, which raises the question of whether such CNF encodings are also
  effective for enumeration.

  In this paper, we investigate both theoretically and empirically the
  effectiveness of CNF conversions for SAT and SMT enumeration. On the negative
  side, we show that: (i)~Tseitin transformation prevents the solver from
  producing short partial assignments, thus seriously affecting the effectiveness
  of enumeration; (ii)~Plaisted and Greenbaum transformation overcomes this
  problem only in part. On the positive side, we prove theoretically and we show
  empirically that combining Plaisted and Greenbaum transformation with NNF
  preprocessing upfront \mbox{---which} is typically not used in solving--- can
  fully overcome the problem and can drastically reduce both the number of
  partial assignments and the execution time.
\end{abstract}

%\begin{abstract}
%      A clear and well-documented \LaTeX\ document is presented as an
%  article formatted for publication by ACM in a conference proceedings
%  or journal publication. Based on the ``acmart'' document class, this
%  article presents and explains many of the common variations, as well
%  as many of the formatting elements an author may use in the
%  preparation of the documentation of their work.
%\end{abstract}

%% JAIR Note: 
%% Do not include ACM CCS Concepts or Keywords

%% To be updated by authors.
\received{9 February 2024}
\received[revised]{19 November 2024}
\received[accepted]{30 April 2025}

%%
%% This command processes the author and affiliation and title
%% information and builds the first part of the formatted document.
\maketitle

\section{Introduction}%
\section{introduction}

% 1. importance of TKGs and reasoning on TKGs. 
% 2. low resource languages, main main idea.
% 3. relations and limitations of current works.
% 4. summarize our solutions and contributions.

Temporal Knowledge Graphs (TKGs)~\cite{YAGO,ICEWS18,WIKI,acekg} characterize temporally evolving events, where each event, represented as ({\em subject}, {\em relation}, {\em object}), is associated with temporal information ({\em time}), e.g., ({\em Macron}, {\em reelected}, {\em French president}, {\em 2022}). TKGs has facilitated various knowledge-intensive Web applications with timeliness, such as question answering~\cite{KBQA}, product recommendation~\cite{RippleNet,TKG4Rec,TKG4Rec2,RETE}, and social event forecasting~\cite{KG4Social,DyDiff-VAE,andgan,belief,misinfo,polarization}. 

As new events are continually emerging, modern TKGs are still far from being complete. Conventionally, the TKG construction process relies primarily on information extraction from unstructured corpus~\cite{WIKI,YAGO, EventKG}, which necessitates extensive manual annotations to keep up with changing events. For instance, the recent transition from Trump to Biden as the President of the United States has not been reflected in many TKGs, highlighting the need for timely updates. This spurs research on temporal knowledge graph reasoning to automate evolving events prediction over time~\cite{TA-DistMult,Know-Evolve,Renet,RE-GCN}. Unfortunately, the problem of TKG incompleteness is particularly pronounced in low-resource languages, where it is unable to collect enough corpus and annotations to support robust TKG construction. This results in suboptimal reasoning performance and distinctly unsatisfying accuracy in predicting recent and future events.

% whose performance can degrade significantly in low-resource language TKGs that suffer from severe incompleteness over time. 
% \jingfeng{why don't people  study cross-lingual TKG previously, (i.e. use language alignment to improve TKG). Is it really helpful intuitively to use high resource language to help TKGC? For instance, is it enough to use static langauge-alignment to help KGC, ignoring the temporal information? Are those langauge-alignment changing across time?}



\begin{figure}
    \centering
    \includegraphics[width = 1.0\linewidth]{fig/task.pdf}
    \caption{An illustrative example of cross-lingual reasoning on TKGs. 1) We aim to transfer knowledge from English TKG to Japanese TKG, where the English version provides more complete information; 2) Cross-lingual alignments only cover a small ratio of entities, e.g., Apple Inc; 3) Cross-lingual alignments can be noisy and misleading, e.g., A city called Ventura is linked to new macOS Ventura at $t_2$, introducing noise for reasoning in Japanese.}
    \label{fig:illustration}
    %\vspace{-6mm}
\end{figure}

Inspired by the incompleteness issue facing low-resource languages in constructing TKGs, we introduce a novel task named Cross-Lingual Temporal Knowledge Graph Reasoning (as shown in Figure~\ref{fig:illustration}). This task aims to alleviate the reliance on supervision for TKGs in low-resource languages (referred to as the target language) by transferring temporal knowledge from high-resource languages (referred to as the source language)~\footnote{In this paper, for the sake of brevity, we interchangeably use the terms high-resource/low-resource and source/target.}. In contrast, all the existing efforts are either limited to reasoning in monolingual TKGs (usually high-resource languages, e.g., English)~\cite{TA-DistMult,Know-Evolve,Renet,RE-GCN}, or multilingual static KGs~\cite{KEnS,AlignKGC,SS-AGA}. To the best of our knowledge, cross-lingual TKG reasoning that transfers temporal knowledge between TKGs has not been investigated. 

%Motivated by this, we study a new task named {\em cross-lingual temporal knowledge graph reasoning} as shown in Figure~\ref{fig:illustration}, to alleviate the heavy dependence on supervision for any resource-poor language TKGs by distilling the temporal knowledge from resource-rich ones. Differently, all the existing efforts are either limited to reasoning in monolingual (usually high-resource languages, e.g., English) temporal KGs~\cite{TA-DistMult,Know-Evolve,Renet,RE-GCN}, or multilingual static KG~\cite{KEnS,AlignKGC,SS-AGA}, but neglecting the reasoning in a both temporal and cross-lingual manner that highly requires capturing time-evolving patterns and language discrepancy. To the best of our knowledge, this problem, regarding how to transfer cross-lingual knowledge between TKGs, has still not been formally investigated. 

% Unlike conventional TKG reasoning, 
The fulfillment of this task poses tremendous challenges in two aspects: 1) \textbf{Scarcity of cross-lingual alignment}: as the informative bridge of two separate TKGs, cross-lingual alignment is imperative for cross-lingual knowledge transfer~\cite{AlignKGC,KEnS,SS-AGA}. However, obtaining alignments between languages is a time-consuming and resource-intensive process that heavily relies on human annotations. The transfer of knowledge through a limited number of alignments is often insufficient to fully enhance the TKG in the target language. 2) \textbf{Temporal knowledge discrepancy}: the information associated with two aligned entities is not necessarily identical, especially with regards to temporal patterns. Utilizing a rough approach to equate the aligned entities at all times can result in the transfer of misleading knowledge and negatively impact performance. This becomes more pronounced when the alignments are noisy and unreliable. For example, at the time step $t_2$, a new event about operating system ``{\it Ventura}'' from Apple company occurs in the source English TKG, and meanwhile there is a noisy aligned entity ``{\it Ventura city}'' in the target Japanese TKG. Directly pulling those two entities at this point, can inevitably introduce  noise and fail to predict a set of related events in the target TKG. Therefore, it is crucial to dynamically regulate the alignment strength of each local graph structure over time in order to maximize the effectiveness of cross-lingual knowledge distillation.

% Pulling those entities together cannot augment information in target languages. Small alignment strength is beneficial in the unreliable alignment cases, otherwise the misleading knowledge transferring can even hurt the performance.

% Moreover, in a case that the alignments are not fully reliable, directly pulling the two aligned entities together 


% optimally dynamic alignment strength
% {\em Optimal alignment strength to maximize the benefits of knowledge distillation is difficult to obtain, especially in the temporal manner.} 
% In practical, although the aligned entities can share similar information, they may still differ in other perspectives, including but not limited to frequency, interactions, and temporal patterns. How to adjust the alignment strength (i.e., the distance constrains of the aligned entities in the uni-space) accordingly for different entities at different time is unclear. \zheng{Ruijie TODO: add Ventura case}Moreover, in a case that the alignments are not fully reliable, directly pulling the two aligned entities together can even hurt the performance.



% scarcity of hinders the efficient
% knowledge transfer across languages. 
% {\em Transferring knowledge through a small set of alignments is hard to augment information for all entities.} 

% Aligning the same entities across languages rely heavily on manual labeling or rule-based inference~\cite{EA1,EA2,EA3,selfKG}, which is too time-consuming and impractical to obtain the alignments covering most of the entities in target language. 

% In this paper, we study how to boost the TKG reasoning performance in low-resource languages by explicitly increasing the completeness of those TKGs in history. Instead of improving the underlying information extraction techniques in low-data regime, we propose a new task called {\em Cross-lingual Temporal Knowledge Graph Reasoning}, motivated by the facts that there exists common or complementary knowledge shared by the TKGs in different languages under similar topics. The new task aims to facilitate TKG reasoning in low-resource languages (target languages) by distilling knowledge from a corresponding TKG in high-resource language (source language)  through a small set of entity alignments as bridges~\footnote{In this paper, we interchangeably use the terminology high-resource/low-resource and source/target for briety.}. Figure~\ref{fig:illustration} provides an illustrative example of the proposed task.


% Unfortunately, recent breakthroughs in temporal knowledge graph reasoning model~\cite{TA-DistMult,Know-Evolve,Renet,RE-GCN} highly rely on the completeness of the TKGs, especially for the most recent events. 

% However, the completeness of TKGs varies a lot across different languages, even under similar topics. Conventionally, the TKG construction process relies primarily on information extraction techniques built on the unstructured corpus~\cite{WIKI,YAGO, EventKG}. Therefore, the amount of corpus and human annotations in different languages significantly influence the quality of the corresponding TKGs . 
% Therefore, automatically completing/updating TKGs has been attracting enormous interests in recently years, which aims to predict recent/future events on TKGs based on historical events~\cite{TA-DistMult,Know-Evolve,Renet,RE-GCN}, namely temporal knowledge graph reasoning~\footnote{Broadly speaking, TKG reasoning includes interpolation to predict historical events and extrapolation to predict future events. In this paper, we refer to extrapolation task as TKG reasoning, since it is more vital for time-sensitive downstream tasks.}.


% For languages with large-scale and carefully labeled corpus (we refer to as high-resource languages, e.g., English), the constructed TKGs are more comprehensive than TKGs in other languages that lack the high-quality corpus (we refer to as low-resource languages, e.g., Spanish, Slovene, Danish, etc). Such completeness discrepancy leads to distinctly uneven TKG reasoning performances in different languages, which in turn affects the quality of service of the downstream applications. 


% Compared with the traditional TKG reasoning task, the new task imposes non-trivial challenges. An intuitive solution is to construct a unified graph including two TKGs in both source and target languages, and the knowledge distillation can be fulfilled by pulling the aligned entities from two languages close to each other in the uni-space~\cite{AlignKGC,KEnS}. However, there are still two challenges to be addressed. 

% \zheng{Ruijie TODO, Place this part to related works.}
% Existing works in related areas fail to address the aforementioned challenges. Monolingual reasoning methods on static/temporal knowledge graphs~\cite{TransE,TranR,ComplEX,RotatE,TA-DistMult,Know-Evolve,Renet,RE-GCN} is incapable of the desired knowledge transferring due to the insufficient alignment modeling. Although they can be extended on the cross-lingual scenario by viewing the alignments as a new relation on the merged TKGs, the limited amount of alignments prevent them from augmenting information for most of the entities. Entity alignment methods on KGs~\cite{EA1,EA2,EA3,EA4,EA5,selfKG} can automatically enlarge the alignments by  predicting the correspondence between the two TGs. But most of them, if not all, require the relatively even completeness of two TGs to capture the structural similarities, which can not be satisfied in our case, as target TKGs are far from complete. Some recent works start to study the multilingual TK reasoning on static graphs~\cite{AlignKGC,KEnS,SS-AGA}, which similarly aim to extract knowledge from several source KGs to boost the reasoning performance in the target KG, while they still require a sufficient amount of cross-lingual alignments and totally ignore the temporal perspective in our task.

% to facilitate temporal knowledge graph reasoning in low-resource languages. 
% increase the TKG connection and target TKG capacity
% In light of the mutual benefits, we iteratively generate pseudo alignment pairs and pseudo temporal events to address the co-existing scarcity issue in both cross-lingual alignment and target TKGs. 


In this paper, we propose a novel Mutually-paced Knowledge Distillation (\model) framework, where a teacher network learns more enriched temporal knowledge and reasoning skills from the source TKG to facilitate the learning of a student network in the low-data target one. The knowledge transfer is enabled via an alignment module, which estimates entity correspondence across languages based on temporal patterns. Firstly, to alleviate the limited language alignments (\textbf{Challenge \#1}), such a knowledge distillation process is mutually paced over time. This means, on one hand, we encourage the mutually interactive learning between the teacher and student. Concretely, the alignment module between the teacher and the student learns to generate pseudo alignment between TKGs to maximally expand the upper bound of knowledge transfer. And subsequently, it empowers the student to encode more informative knowledge in target TKG, which can in turn boost the alignment module to explore more reasonable alignments as the bridge across TKGs. One the other hand, inspired by self-paced learning~\cite{spl-1,spl-2}, we make the generations as a progressively easy-to-hard process over time. We start from generating reliable pseudo data with high confidence. As time goes by, we then gradually increase the generation amount by relieving the restriction over time. Secondly, to inhibit the temporal knowledge mismatch (\textbf{Challenge \#2}), the attention module can estimate the graph alignment strength distribution over time. This is achieved by a temporal cross-lingual attention in terms of the local graph structure and temporal-evolving patterns of aligned entities. As such, it can dynamically control the negative effect and suppress noise  propagation from the source TKG. Moreover, we provide a theoretical convergence guarantee for the training objective on both initial ground-truth data and pseudo data. To evaluate \model, we conduct extensive experiments of 12 cross-lingual TKG transfer tasks in multilingual EventKG dataset~\cite{EventKG}. Our empirical results show that the \model method outperforms state-of-the-art baselines in both with and without alignment noise settings, where only $20\%$ of temporal events in the target KG and $10\%$ of cross-lingual alignments are preserved.

% To validate the effectiveness of \model, we conduct extensive experiments of 12 cross-lingual TKG transfer tasks in multilingual EventKG benchmark dataset~\cite{EventKG} . Our experimental results empirically demonstrate the superiority of the \model method over state-of-the-art baselines, ranging from static KG embedding~\cite{TransE,TransR,DistMult,RotatE}, temporal KG reasoning~\cite{TA-DistMult,Renet,RE-GCN} to multilingual KG completion~\cite{KEnS,AlignKGC,SS-AGA}, in both with and without alignment noise settings. We further conduct comprehensive ablation and hyperparameter studies to validate the effectiveness of each design choices. Moreover, we provide theoretical analysis of convergence guarantee for the training objective on both initial groundtruth data and pseudo generative data.



To sum up, our contributions are three-fold:

\begin{itemize}[leftmargin = 15pt]
    \item \textbf{Problem formulation}: We propose the cross-lingual temporal knowledge graph reasoning task, to boost the temporal reasoning performance in target TKG by transferring knowledge from source TKG;
    \item \textbf{Novel framework}: We propose a novel \model framework, which enables the mutually-paced learning between the teacher and student networks, to promote both pseudo alignments and knowledge transfer reliability. Besides, \model involves a dynamic alignment estimation across TKGs that inhibits the influence of temporal knowledge discrepancy.
    \item \textbf{Extensive evaluations}: Empirically, extensive experiments on 12 cross-lingual TKG transfer tasks in multilingual EventKG benchmark dataset demonstrate the effectiveness of \model.
\end{itemize}
% pseudo data generation technique to progressively enhance the training data. The generated pseudo alignments can help the training of the representation modules by the knowledge distillation, and in turn adding pseudo events in the target TKG can improves alignment module by providing high-quality representations. 




% interactively
% TKGs in a source language and a target language are represented by a teacher representation module and a student one into a uni-space, respectively. 
% The knowledge distillation is enabled by a cross-lingual alignment module which pulls the aligned entities close to each other and push other entities far away. 
% To address the challenge caused by the scarcity of cross-lingual alignment, 



\section{Background}%
\label{sec:background}
\section{Background}
\label{sec:background}

\subsubsection{Plagiarism in Programming Contests}

Source code plagiarism is a well-researched area of studies~\cite{marins2014survey}, however, the developed solutions are usually focused on finding plagiarism in homework assignments~\cite{novak2016academia, plugiarismreview2019},  
there are only a few works devoted to plagiarism in competitive programming~\cite{contest_bangladesh2021, contest_warsaw2019}. Even then, they are mainly focused on integrating a particular plagiarism detection tool into the online judging system, and not on comparing different existing tools. 

Unlike some other cases, source code in programming contests has its own specifics that directly relate to finding plagiarism in it. In addition to the usual plagiarism hiding techniques~\cite{comparison2009}, the solutions will have a lot of similar code that is natural for contests. Firstly, there is \textit{template} code, \textit{i.e.}, some common implementations of popular algorithms that the contestant copies into every solution. Secondly, there is code that will be almost the same in every solution to the given problem but not actually plagiarized, \textit{e.g.}, reading the input or printing the answer.
Currently, active research is underway on how to find similar code and reduce its influence on the similarity~\cite{common2016, common2020}.

The template code is particularly difficult to take into account, because it can be completely different for different contestants in both size and implementation. Also, functions from the template code can be actively used or ignored completely by the contestant, depending on a particular task, making it very difficult to automatically preprocess all submissions by simply removing all template code from them. It is clear that template code can easily become a weak spot for many algorithms, and this must be taken into account when using plagiarism detection tools on the contest code and when building a benchmark for their comparison.

\subsubsection{Source Code Plagiarism Detection Tools}

Many different tools were developed aimed at detecting source code plagiarism~\cite{plugiarismreview2019}. \textit{Text-based} algorithms treat a program as a simple text, without taking into account its programming language. The advantages of such approaches are language-independence and high performance~\cite{comparison2009}, however, this comes at the expense of lower accuracy. A popular text-based tool is Sherlock~\cite{sherlock}. This tool converts the file into a sequence of string tokens, hashes it, and extracts a subsample of hashes. To determine the similarity of two programs, Sherlock calculates the similarity of the sequences of hashes.

\textit{Token-based} algorithms run a specific lexer on the program and compare token streams. Such approaches are still fast but consider a deeper representation of the program. One of the earliest plagiarism detection tools, SIM~\cite{sim1999}, applies an algorithm for finding the maximum sequence alignment to the resulting token sequence of given programs. The similarity of two programs is then defined as their alignment score. Another popular token-based tool, JPlag~\cite{jplag2003}, defines the similarity as the percent of tokens from the first sequence that can be covered by tokens from the second one. A different token-based tool, MOSS~\cite{moss2003}, is based on comparing the fingerprints of programs. A fingerprint is constructed in three steps: (1) all the \textit{n}-grams for the token stream are built, (2) these \textit{n}-grams are hashed, and (3) to avoid comparing big sets of hashes, MOSS uses a \textit{winnowing} algorithm to select a certain subset of hashes for each program. The idea of winnowing has got popular, and several tools appeared that are based on it. One such tool is Dolos~\cite{dolos2022}. Unlike MOSS, Dolos is open-source, supports more programming languages, and provides powerful visualizations of the results.

Finally, \textit{graph-based} algorithms build a graph (usually, a program dependence graph) of the program, which shows the dependencies of the data within the program. To avoid the naive solving of an NP-hard problem, they use certain heuristics. For example, BPlag~\cite{bplag2021} uses the idea of a Greedy-String-Tiling algorithm to find similar parts in the graphs of two programs.

Overall, it can be seen that there exist a lot of approaches for finding plagiarism in the source code, however, given the specifics of programming contests, it is not clear how well they perform in such a setting. To evaluate the existing tools, to help researchers further improve them for competitive programming, as well as to provide a benchmark for future solutions, in this work, we aim to collect the first dedicated dataset of programming contest plagiarism.

\begin{figure*}[htbp]
\centering
    \includegraphics[width=\textwidth]{figures/pipeline.pdf}
    \centering
    \vspace{-0.5cm}
    \caption{The pipeline of the proposed approach for collecting the dataset.}
    \label{fig:pipeline}
    \vspace{-0.5cm}
\end{figure*}

\section{The impact of CNF transformations on %\ignoreinlong{AllSAT} \ignoreinshort{
  Enumeration} %}%
\label{sec:problem}
% \RSTODO{riscrivi \S3 e \S4 di
% conseguenza, spiegando prima l'intuiizione e poi l'esempio.}
In this section, we present a theoretical analysis of the impact of different
CNF-izations on the enumeration of short partial truth assignments. In
particular, we focus on
\TseitinCNF~\cite{tseitinComplexityDerivationPropositional1983} and
\PlaistedCNF~\cite{plaistedStructurepreservingClauseForm1986}. %\ignoreinlong{the AllSAT task}\ignoreinshort{%}. 
% \ignoreinshort{ %
% }
We point out how CNF-izing AllSAT problems using these transformations can
introduce unexpected drawbacks for enumeration. In fact, we show that the
resulting encodings can force the enumerator to produce partial assignments
that are larger in size and in number than necessary. % when they are used to preprocess non-CNF formulas. 
%
In our analysis we refer to AllSAT, but it applies to All\smt{} as well by
restricting to theory-satisfiable truth assignments.\@ Moreover, the analysis
applies to both disjoint and non-disjoint enumeration.

\subsection{The impact of Tseitin CNF transformation}%
\label{sec:problem:label}

We show that using the \TseitinCNF{}
transformation~\cite{tseitinComplexityDerivationPropositional1983} can be
problematic for enumeration. %preprocessing the input formula
%
In particular, we point out a fundamental weakness of \TseitinCNF{}:
%
%\ADDED{
%\GMSIDENOTE{RSTODO: Change ``suffices to satisfy''?}
\begin{fact}%
    \label{fact:tseitin}
    If a partial assignment \muA{} satisfies $\vi$, this does not imply that \muA{} satisfies $\exists
        \allB.\vicnfts$.
\end{fact}
\noindent
In fact, we recall that \TseitinCNF{} works by applying recursively the rewriting step (\sref{sec:bg:cnf}):
\begin{eqnarray}
    \label{eq:rewritingTseitin}
    \vi\Longrightarrow\vi[\vi_i|B_i] \wedge (B_i\iff\vi_i)
\end{eqnarray}
\noindent and then by recursively CNF-izing the two conjuncts.
%
A {\em partial} assignment \muA{} may satisfy a non-CNF formula $\vi(\allA)$
because it does not need to assign a truth value to the atoms in {\em all}
sub-formulas of $\vi$. (E.g., \muA{} can satisfy $\vi\defas\vi_1\vee\vi_2$
without assigning values to the atoms in $\vi_2$ if it satisfies $\vi_1$.)
%
Consider \eqref{eq:rewritingTseitin} s.t.\ $\vi_i$ is some sub-formula of \vi{}
whose atoms are not assigned by \muA{}. Although \muA{} satisfies $\vi$, \muA{}
does not satisfy $\exists B_i.( \vi[\vi_i|B_i]\wedge (B_i\iff\vi_i))$. In fact,
to satisfy the second conjunct it is necessary to assign some truth value not
only to $B_i$ but also to some of the unassigned atoms in $\vi_i$, so that to
make $\vi_i$ evaluate to the same truth value assigned to $B_i$.

%   \noindent
% %    In general, \cref{fact:tseitin} can be explained as follows.
%     In fact, a
%     {\em partial} assignment \muA{} may suffice to satisfy a non-CNF formula 
%     $\vi(\allA)$  because it does not need to assign a truth value to the
%     atoms in \emph{all}
%     subformulas of $\vi$. (E.g., \muA{} can satisfy
%     $\vi\defas\vi_1\vee\vi_2$ without assigning values to the atoms in
%     $\vi_2$ if it satisfies $\vi_1$.)
%     %
%     Consider some subformula  $\vi_i$ of \vi{} whose atoms are not assigned by \muA{}.
%     %
%     We recall that \TseitinCNF{} works by applying recursively the rewriting step (\sref{sec:bg:cnf}):
%     %\GMSIDENOTE{In \sref{sec:bg:cnf} abbiamo usato \DeMorganCNF($B_i\iff\vi_i$)}
%     \begin{eqnarray}
%     \label{eq:rewritingTseitin}
%     \vi\Longrightarrow\vi[\vi_i|B_i] \wedge (B_i\iff\vi_i)
%     \end{eqnarray}
%     %\vi\Longrightarrow\vi[\vi_i|B_i] \wedge \DeMorganCNF(B_i\iff\vi_i)$$
%     %
%     %$$\TseitinCNF{(\vi)}=\TseitinCNF{(\vi[\vi|B_i])}\wedge\DeMorganCNF{(B_i\iff\vi_i)}$$.
%     \noindent and then by recursively CNF-izing the two conjuncts.
%     Unfortunately, although \muA{} suffices to satisfy $\vi$,
%     \muA{} does not
%     suffice to satisfy $\exists B_i.( \vi[\vi_i|B_i]\wedge
%     (B_i\iff\vi_i))$, because to satisfy the second conjunct it
%     is necessary to assign some truth value not only to $B_i$ but also
%     to some of the atoms in $\vi_i$, so
%     that to make $\vi_i$ evaluate to the same truth value given to $B_i$.

As a consequence of \cref{fact:tseitin}, given $\muA$ satisfying \vi{}, in
order to produce an assignment \muAprime{} satisfying $\exists \allB.\vicnfts$
the enumerator is most often forced to assign other atoms in \allA{}, so that
$\muAprime\supset\muA$. %(unnecessarily)
%
Given the fact that the amount of total assignments covered by a partial
assignment decreases exponentially with its length (see
\sref{sec:background:propositional-logic}), the above weakness causes a blow-up
in the number of partial assignments
%which we need generating
needed to cover all models. This may drastically affect the effectiveness and
efficiency of the enumeration.

This is illustrated in the following example, where instead of one single short
partial assignment the enumerator is forced to enumerate 9 longer ones.

%We first illustrate this issue with an example.
%\newpage
\begin{example}%
    \label{ex1}
    Consider the propositional formula  over %the set of atoms
    $\allA\defas\set{A_1, A_2, A_3, A_4, A_5, A_6, A_7}$:
    \begin{equation}
        \label{eq:ex1:vi}
        % \vi \defas 
        %     \overbrace{(\underbrace{(A_1 \wedge A_2)}_{B_1} \vee  A_3)}^{B_2} \iff
        %     \overbrace{ (\underbrace{(A_4 \wedge A_5 )}_{B_3} \vee  A_6) }^{B_4}
        % \overbrace{(A_1 \wedge A_2)}^{B_1} \vee  
        % \overbrace{(A_3 \iff \underbrace{((A_4\vee A_5) \wedge 
        % \overbrace{(A_6 \vee A_7)}^{B_2})}_{B_3})}^{B_4}
        \vi \defas
        \overbrace{(A_1\wedge A_2)}^{B_1}\vee
        \overbrace{(
            \overbrace{(
                \overbrace{(A_3\vee A_4)}^{B_2}\wedge
                \overbrace{(A_5\vee A_6)}^{B_3}
                )}^{B_4}\iff
            A_7
            )}^{B_5}.
    \end{equation}
    \noindent
    $\vi$ is not in CNF, and thus it must be CNF-ized before starting the enumeration process.
    If \TseitinCNF{} is used, then the following CNF formula is obtained:

    \begin{subequations}%
        \label{eq:ex1:vicnf}
        \begin{alignat}{2}
            % \vicnf \defas
            % &(\neg B_1\vee\pos A_1)\wedge(\neg B_1\vee\pos A_2)\wedge(\pos B_1\vee\neg A_1\vee\neg A_2)&\wedge\label{eq:ex1:vicnf:line1}\\
            % &(\pos B_2\vee\neg B_1)\wedge(\pos B_2\vee\neg A_3)\wedge(\neg B_2\vee\pos B_1\vee\pos A_3)&\wedge\label{eq:ex1:vicnf:line2}\\
            % &(\neg B_3\vee\pos A_4)\wedge(\neg B_3\vee\pos A_5)\wedge(\pos B_3\vee\neg A_4\vee\neg A_5)&\wedge\label{eq:ex1:vicnf:line3}\\
            % &(\pos B_2\vee\neg B_3)\wedge(\pos B_2\vee\neg A_6)\wedge(\neg B_2\vee\pos B_3\vee\pos A_6)&\wedge\label{eq:ex1:vicnf:line4}\\
            % &(\neg B_4 \vee\pos B_2)\wedge(\pos B_4\vee\neg B_2)&\label{eq:ex1:vicnf:line5}
            %-------------------------------
            % \psi\defas
            % &(\neg B_1\vee\pos A_1)\wedge(\neg B_1\vee\pos A_2)\wedge(\pos B_1\vee\neg A_1\vee\neg A_2)&\wedge\label{eq:ex1:vicnf:line1}\\
            % &(\pos B_2\vee\neg A_5)\wedge(\pos B_2\vee\neg A_6)\wedge(\neg B_2\vee\pos A_5\vee\pos A_6)&\wedge\label{eq:ex1:vicnf:line2}\\
            % &(\neg B_3\vee\pos A_4)\wedge(\neg B_3\vee\pos B_2)\wedge(\pos B_3\vee\neg A_4\vee\neg B_2)&\wedge\label{eq:ex1:vicnf:line3}\\
            % &(\neg B_4\vee\neg A_3\vee B_3)\wedge(\neg B_4\vee A_3\vee\neg B_3)\wedge
            % ( B_4\vee A_3\vee B_3)\wedge( B_4\vee\neg A_3\vee\neg B_3)
            % &\wedge\label{eq:ex1:vicnf:line4}\\
            % &(\pos B_1\vee\pos B_4)&
            % --------------------------------------
             & \vicnfts\defas\nonumber                                                                                                         \\
             & \enspace(\neg B_1\vee\pos A_1)\wedge(\neg B_1\vee\pos A_2)\wedge(\pos B_1\vee\neg A_1\vee\neg A_2) & \wedge
             & \quad\eqcomment{(B_1\iff (A_1\wedge A_2))}\label{eq:ex1:vicnf:line1}                                                            \\
             & \enspace(\pos B_2\vee\neg A_3)\wedge(\pos B_2\vee\neg A_4)\wedge(\neg B_2\vee\pos A_3\vee\pos A_4) & \wedge
             & \quad\eqcomment{(B_2\iff (A_3\vee A_4))}\label{eq:ex1:vicnf:line2}                                                              \\
             & \enspace(\pos B_3\vee\neg A_5)\wedge(\pos B_3\vee\neg A_6)\wedge(\neg B_3\vee\pos A_5\vee\pos A_6) & \wedge
             & \quad\eqcomment{(B_3\iff (A_5\vee A_6))}\label{eq:ex1:vicnf:line3}                                                              \\
             & \enspace(\neg B_4\vee\pos B_2)\wedge(\neg B_4\vee\pos B_3)\wedge(\pos B_4\vee\neg B_2\vee\neg B_3) & \wedge
             & \quad\eqcomment{(B_4\iff (B_2\wedge B_3))}\label{eq:ex1:vicnf:line4}                                                            \\
             & \enspace(\neg B_5\vee\pos B_4\vee\neg A_7)\wedge(\neg B_5\vee\neg B_4\vee\pos A_7)                 & \wedge
             & \quad\eqcomment{(B_5\iff (B_4\iff A_7))}\label{eq:ex1:vicnf:line5}                                                              \\
             & \enspace(\pos B_5\vee\pos B_4\vee\pos A_7)\wedge(\pos B_5\vee\neg B_4\vee\neg A_7)                 & \wedge\nonumber            %\tag{...}
            \\
             & \enspace(\pos B_1\vee\pos B_5)                                                                     & \label{eq:ex1:vicnf:line6}
        \end{alignat}
    \end{subequations}
    where the fresh atoms $\allB\defas\set{B_1, B_2, B_3, B_4, B_5}$ label
    sub-formulas as in~\eqref{eq:ex1:vi}.

    Consider the (minimal) partial truth assignment:
    \begin{equation}
        \label{eq:ex1:muA}
        \muA\defas\set{\neg A_3,\neg A_4,\neg A_7}.
    \end{equation}
    \muA{} satisfies $\vi$ \eqref{eq:ex1:vi}, {even though it does not assign a truth value to the sub-formulas $(A_1\wedge A_2)$ and $(A_5\vee A_6)$.} % since the atoms $A_1, A_2, A_5, A_6$ are not assigned.
        %
        %Unfortunately, 
        {Yet, }$\muA$ does not
    satisfy $\exists\allB.\vicnfts$.
    In fact,  {there is no total truth
            assignment $\etaB$ on \allB{} such that
            $\muA\cup\etaB\pmodels\vicnfts$},
    because~\eqref{eq:ex1:vicnf:line1} and~\eqref{eq:ex1:vicnf:line3}
    cannot be satisfied by assigning only variables in $\allB$;
    rather, it is necessary to further assign  at least one atom in
    $\set{A_1,A_2}$ to satisfy~\eqref{eq:ex1:vicnf:line1} and at least one
    in $\set{A_5,A_6}$ to satisfy~\eqref{eq:ex1:vicnf:line3}.
    %

    Suppose an enumerator assigns first the literals in \muA~\eqref{eq:ex1:muA},
    which force to assign also $\muB\defas\set{\neg B_2, \neg B_4, B_5}$ due
    to~\eqref{eq:ex1:vicnf:line2},~\eqref{eq:ex1:vicnf:line4},
    and~\eqref{eq:ex1:vicnf:line5} respectively. Since $\muA\cup\muB$ satisfies all
    clauses except those in~\eqref{eq:ex1:vicnf:line1}
    and~\eqref{eq:ex1:vicnf:line3}, then the enumerator needs extending
    $\muA\cup\muB$ by enumerating the partial assignments on the unassigned atoms
    \set{A_1,A_2,A_5,A_6,B_1,B_3} which satisfy~\eqref{eq:ex1:vicnf:line1}
    and~\eqref{eq:ex1:vicnf:line3}. Regardless of the search strategy adopted, this
    requires generating no less than $9$ satisfying partial assignments on
    $\set{A_1,A_2,A_5,A_6}$.~%
    %$\set{A_1,A_2,A_5,A_6,B_1,B_3}$.~%
    \footnote{
        The set of models for \eqref{eq:ex1:vicnf:line1}
        is
        \set{
            \set{\pos B_1,\pos A_1,\pos A_2},
            \set{\neg B_1,\pos A_1,\neg A_2},
            \set{\neg B_1,\neg A_1,\pos A_2},
            \set{\neg B_1,\neg A_1,\neg A_2}
        }, which can be covered only either by
        \set{
            \set{\pos B_1,\pos A_1,\pos A_2},
            \set{\neg B_1,\neg A_1},
            \set{\neg B_1,\pos A_1,\neg A_2}
        } or by
        \set{
            \set{\pos B_1,\pos A_1,\pos A_2},
            \set{\neg B_1,\neg A_2},
            \set{\neg B_1,\neg A_1,\pos A_2}
        }
        in the case of disjoint enumeration,
        and by
        \set{
            \set{\pos B_1,\pos A_1,\pos A_2},
            \set{\neg B_1,\neg A_1},
            \set{\neg B_1,\neg A_2}
        } in the case of non-disjoint enumeration.
        In all cases,   we need no less than 3 distinct partial assignments on
        \set{A_1,A_2}. Similar considerations hold for
        \eqref{eq:ex1:vicnf:line3}. Thus, we need no
        less than
        $3\times3=9$ partial assignments on $\set{A_1,A_2}\cup\set{A_5,A_6}$.
        %$3\times3=9$ partial assignments on $\set{B_1,A_1,A_2}\cup\set{B_3,A_5,A_6}$.
        %representing the $4\times4=16$ total ones.
        % \GMNOTE{
        % Bisogna coprire tutti i modelli totali projected su \set{A_1, A_2}
        % }
    }
    %
    For instance, in the case of disjoint AllSAT, instead of the single partial
    assignment $\muA$~\eqref{eq:ex1:muA}, the solver may return the following list
    of 9 partial assignments satisfying $\exists\allB.\vicnfts$ which extend
    \muA{}:%\ignoreinshort{%}
    \begin{equation}%
        \label{eq:ex1:muA:all}
        \begin{array}{llllllll}
            \multicolumn{2}{c}{\overbrace{\rule{1.5cm}{0pt}}^{B_1}} &           &           & \multicolumn{2}{c}{\overbrace{\rule{1.5cm}{0pt}}^{B_3}} &           &                        \\
            \{\neg A_1,                                             &           & \neg A_3, & \neg A_4,                                               & \neg A_5, & \neg A_6, & \neg A_7\}
            \quad\eqcomment{\set{\neg B_1,\neg B_2,\neg B_3,\neg B_4,\pos B_5}}                                                                                                            \\
            \{\neg A_1,                                             &           & \neg A_3, & \neg A_4,                                               & \pos A_5, &           & \neg A_7\}
            \quad\eqcomment{\set{\neg B_1,\neg B_2,\pos B_3,\neg B_4,\pos B_5}}                                                                                                            \\
            \{\neg A_1,                                             &           & \neg A_3, & \neg A_4,                                               & \neg A_5, & \pos A_6, & \neg A_7\}
            \quad\eqcomment{\set{\neg B_1,\neg B_2,\pos B_3,\neg B_4,\pos B_5}}                                                                                                            \\
            \{\pos A_1,                                             & \neg A_2, & \neg A_3, & \neg A_4,                                               & \neg A_5, & \neg A_6, & \neg A_7\}
            \quad\eqcomment{\set{\neg B_1,\neg B_2,\neg B_3,\neg B_4,\pos B_5}}                                                                                                            \\
            \{\pos A_1,                                             & \neg A_2, & \neg A_3, & \neg A_4,                                               & \pos A_5, &           & \neg A_7\}
            \quad\eqcomment{\set{\neg B_1,\neg B_2,\pos B_3,\neg B_4,\pos B_5}}                                                                                                            \\
            \{\pos A_1,                                             & \neg A_2, & \neg A_3, & \neg A_4,                                               & \neg A_5, & \pos A_6, & \neg A_7\}
            \quad\eqcomment{\set{\neg B_1,\neg B_2,\pos B_3,\neg B_4,\pos B_5}}                                                                                                            \\
            \{\pos A_1,                                             & \pos A_2, & \neg A_3, & \neg A_4,                                               & \neg A_5, & \neg A_6, & \neg A_7\}
            \quad\eqcomment{\set{\pos B_1,\neg B_2,\neg B_3,\neg B_4,\pos B_5}}                                                                                                            \\
            \{\pos A_1,                                             & \pos A_2, & \neg A_3, & \neg A_4,                                               & \pos A_5, &           & \neg A_7\}
            \quad\eqcomment{\set{\pos B_1,\neg B_2,\pos B_3,\neg B_4,\pos B_5}}                                                                                                            \\
            \{\pos A_1,                                             & \pos A_2, & \neg A_3, & \neg A_4,                                               & \neg A_5, & \pos A_6, & \neg A_7\}
            \quad\eqcomment{\set{\pos B_1,\neg B_2,\pos B_3,\neg B_4,\pos B_5}}                                                                                                            \\
        \end{array}
    \end{equation}
    In the case of non-disjoint AllSAT, the solver may enumerate a similar
    set of partial assignments, with \set{\neg A_2} instead of \set{A_1,\neg A_2} and \set{A_6}
    instead of  \set{\neg A_5,A_6}.\exdone{}
    % \begin{IGNORE}

    %     % In fact, three clauses of $\vicnfts$ in~\eqref{eq:ex1:vicnf:line1} and~\eqref{eq:ex1:vicnf:line3} are not satisfied by $\muA\cup\etaB$, since
    %     % $\residual{\vicnfts}{\muA\cup\etaB}=
    %     %     (\neg A_1\vee\neg A_2)\wedge(\neg A_5)\wedge(\neg A_6)$.
    %     % We remark that this is not a coincidence, since {there is no $\etaBprime$ such that $\muA\cup\etaBprime\pmodels\vicnfts$}, because~\eqref{eq:ex1:vicnf:line1} and~\eqref{eq:ex1:vicnf:line3} cannot be satisfied without assigning any atom in $\set{A_1,A_2}$ and $\set{A_5,A_6}$ respectively.

    % %

    % For instance, suppose we adopt the procedure producing disjoint
    % minimal models described in \sref{sec:background:allsat}, and assume
    % it assigns  first the literals in  \muA \eqref{eq:ex1:muA}.

    % % For instance, we consider the procedure in \sref{}  

    %     The solver proceeds to compute $\TA{\exists\allB.\vicnfts}$ by enumerating the assignments satisfying $\vicnfts$ projected over \allA{}.
    %     % As described in~\cref{sec:background}, the solver first enumerates a total truth assignment $\mu\defas\muA\cup\etaB$ such that $\mu\pmodels\vicnf$. Then, the minimization step finds a minimal $\muAprime\subseteq\muA$ s.t.\ $\muAprime\cup\etaB\pmodels\vicnf$.
    %     % Suppose the solver finds the total model
    %     % \begin{equation}
    %     %     \mu\defas\set{\underbrace{\neg B_1, B_2, B_3, B_4}_{\etaB}, \underbrace{\neg A_1,\neg A_2, A_3, A_4, A_5, \neg A_6}_{\muA}}
    %     % \end{equation}
    %     % and then the minimization step outputs
    %     % $\muAprime\defas\set{\neg A_1, A_3, A_4, A_5}$ s.t.\ $\muAprime\cup\etaB\pmodels\vicnf$. Notice that $\muAprime$ is minimal, even though it assigns a truth value also to $A_1$. Whereas the first two clauses in~\eqref{eq:ex1:vicnf:line1} are satisfied by $\neg B_1$, the last one must be satisfied by assigning either $\neg A_1$ or $\neg A_2$.
    %     Suppose, e.g., that the solver picks non-deterministic choices, deciding the atoms in the order
    %     % they appear in $\vicnf$ 
    %     $\set{B_1, A_1, A_2, B_2, A_3, A_4, B_3, A_5, A_6, B_4, B_5, A_7}$ and branching with negative value first. Then, the first (sorted) total truth assignment found is:
    %     \begin{equation}
    %         \label{eq:ex1:eta}
    %         \eta\defas\set{
    %             \underbrace{\neg B_1,\neg B_2,\neg B_3,\neg B_4, B_5}_{\etaB},
    %             \underbrace{\neg A_1,\neg A_2,\neg A_3,\neg A_4,\neg A_5,\neg A_6, \neg A_7}_{\etaA}
    %         }
    %     \end{equation}
    %     which contains $\muA$~\eqref{eq:ex1:muA}.
    %     The minimization procedure looks for a \emph{minimal} subset
    %     $\muAprime$ of $\etaA{}$ s.t.\
    %     $\muAprime\cup\etaB{}\pmodels\vicnfts$.
    %     One possible output of this procedure is the minimal assignment:
    %     \begin{equation}%
    %         \label{eq:ex1:muAprime}
    %         \muAprime\defas\set{\neg A_1,\neg A_3,\neg A_4,\neg A_5,\neg A_6,\neg A_7}.
    %     \end{equation}
    %     %Notice that $\muAprime$ is minimal for $\vicnf$, meaning that any subset $\muAsecond\subset\muAprime$ is s.t.\ $\muAsecond\cup\etaB\not\pmodels\vicnf$. %We remark that this is not a coincidence there does not exist any $\etaBprime$ s.t.\ $\muAprime\cup\etaBprime\pmodels\vicnf$.
    %     %$\residual{\vicnf}{\muAprime\cup\etaB}\neq\top$. 
    % \ignore{    We notice that the partial truth assignment
    %     $\muA$~\eqref{eq:ex1:muA} satisfies $\vi$ and it is s.t.\
    %     $\muA\subset\muAprime$, but it {\em does not satisfy $\exists\allB.\vicnfts$}.
    %     In fact, three clauses of $\vicnfts$ in~\eqref{eq:ex1:vicnf:line1} and~\eqref{eq:ex1:vicnf:line3} are not satisfied by $\muA\cup\etaB$, since
    %     $\residual{\vicnfts}{\muA\cup\etaB}=
    %         (\neg A_1\vee\neg A_2)\wedge(\neg A_5)\wedge(\neg A_6)$.
    %     We remark that this is not a coincidence, since {there is no $\etaBprime$ such that $\muA\cup\etaBprime\pmodels\vicnfts$}, because~\eqref{eq:ex1:vicnf:line1} and~\eqref{eq:ex1:vicnf:line3} cannot be satisfied without assigning any atom in $\set{A_1,A_2}$ and $\set{A_5,A_6}$ respectively.
    % }
    %     Finding $\muAprime$~\eqref{eq:ex1:muAprime} instead of $\muA$~\eqref{eq:ex1:muA} clearly causes an %\ignoreinlong{efficiency}\ignoreinshort{
    %     effectiveness
    %     %} 
    %     problem, since finding longer partial truth assignments implies that the total number of enumerated truth assignments could be up to exponentially larger.
    %     For instance, %\ignoreinshort{
    %     in the case of disjoint AllSAT, %}
    %     instead of the single partial assignment $\muA$~\eqref{eq:ex1:muA}, the solver may return the following list of 9 partial assignments satisfying $\exists\allB.\vicnfts$:
    %     \begin{equation}%
    %         \label{eq:ex1:muA:all}
    %         \begin{array}{llllllll}
    %             \multicolumn{2}{c}{\overbrace{\rule{1.5cm}{0pt}}^{B_1}} &           &           & \multicolumn{2}{c}{\overbrace{\rule{1.5cm}{0pt}}^{B_3}} &           &                        \\
    %             \{\neg A_1,                                             &           & \neg A_3, & \neg A_4,                                               & \neg A_5, & \neg A_6, & \neg A_7\}
    %             \quad\eqcomment{\set{\neg B_1,\neg B_3}}                                                                                                                                       \\
    %             \{\neg A_1,                                             &           & \neg A_3, & \neg A_4,                                               & \pos A_5, &           & \neg A_7\}
    %             \quad\eqcomment{\set{\neg B_1,\pos B_3}}                                                                                                                                       \\
    %             \{\neg A_1,                                             &           & \neg A_3, & \neg A_4,                                               & \neg A_5, & \pos A_6, & \neg A_7\}
    %             \quad\eqcomment{\set{\neg B_1,\pos B_3}}                                                                                                                                       \\
    %             \{\pos A_1,                                             & \neg A_2, & \neg A_3, & \neg A_4,                                               & \neg A_5, & \neg A_6, & \neg A_7\}
    %             \quad\eqcomment{\set{\neg B_1,\neg B_3}}                                                                                                                                       \\
    %             \{\pos A_1,                                             & \neg A_2, & \neg A_3, & \neg A_4,                                               & \pos A_5, &           & \neg A_7\}
    %             \quad\eqcomment{\set{\neg B_1,\pos B_3}}                                                                                                                                       \\
    %             \{\pos A_1,                                             & \neg A_2, & \neg A_3, & \neg A_4,                                               & \neg A_5, & \pos A_6, & \neg A_7\}
    %             \quad\eqcomment{\set{\neg B_1,\pos B_3}}                                                                                                                                       \\
    %             \{\pos A_1,                                             & \pos A_2, & \neg A_3, & \neg A_4,                                               & \neg A_5, & \neg A_6, & \neg A_7\}
    %             \quad\eqcomment{\set{\pos B_1,\neg B_3}}                                                                                                                                       \\
    %             \{\pos A_1,                                             & \pos A_2, & \neg A_3, & \neg A_4,                                               & \pos A_5, &           & \neg A_7\}
    %             \quad\eqcomment{\set{\pos B_1,\pos B_3}}                                                                                                                                       \\
    %             \{\pos A_1,                                             & \pos A_2, & \neg A_3, & \neg A_4,                                               & \neg A_5, & \pos A_6, & \neg A_7\}
    %             \quad\eqcomment{\set{\pos B_1,\pos B_3}}                                                                                                                                       \\
    %         \end{array}
    %     \end{equation}
    %     where $\muAprime$~\eqref{eq:ex1:muAprime} is the first in the list.
    %     % \ignoreinshort{
    %     In the case of non-disjoint AllSAT, instead, a possible output is the following:
    %     \begin{equation}%
    %         \label{eq:ex1:muA:all:rep}
    %         \begin{array}{llllllll}
    %             \multicolumn{2}{c}{\overbrace{\rule{1.5cm}{0pt}}^{B_1}} &           &           & \multicolumn{2}{c}{\overbrace{\rule{1.5cm}{0pt}}^{B_3}} &           &                        \\
    %             \{\neg A_1,                                             &           & \neg A_3, & \neg A_4,                                               & \neg A_5, & \neg A_6, & \neg A_7\}
    %             \quad\eqcomment{\set{\neg B_1,\neg B_3}}                                                                                                                                       \\
    %             \{                                                      & \neg A_2, & \neg A_3, & \neg A_4,                                               & \neg A_5, & \neg A_6, & \neg A_7\}
    %             \quad\eqcomment{\set{\neg B_1,\neg B_3}}                                                                                                                                       \\
    %             \{\neg A_1,                                             &           & \neg A_3, & \neg A_4,                                               & \pos A_5, &           & \neg A_7\}
    %             \quad\eqcomment{\set{\neg B_1,\pos B_3}}                                                                                                                                       \\
    %             \{\neg A_1,                                             &           & \neg A_3, & \neg A_4,                                               &           & \pos A_6, & \neg A_7\}
    %             \quad\eqcomment{\set{\neg B_1,\pos B_3}}                                                                                                                                       \\
    %             \{                                                      & \neg A_2, & \neg A_3, & \neg A_4,                                               & \pos A_5, &           & \neg A_7\}
    %             \quad\eqcomment{\set{\neg B_1,\pos B_3}}                                                                                                                                       \\
    %             \{                                                      & \neg A_2, & \neg A_3, & \neg A_4,                                               &           & \pos A_6, & \neg A_7\}
    %             \quad\eqcomment{\set{\neg B_1,\pos B_3}}                                                                                                                                       \\
    %             \{\pos A_1,                                             & \pos A_2, & \neg A_3, & \neg A_4,                                               & \neg A_5, & \neg A_6, & \neg A_7\}
    %             \quad\eqcomment{\set{\pos B_1,\neg B_3}}                                                                                                                                       \\
    %             \{\pos A_1,                                             & \pos A_2, & \neg A_3, & \neg A_4,                                               & \pos A_5, &           & \neg A_7\}
    %             \quad\eqcomment{\set{\pos B_1,\pos B_3}}                                                                                                                                       \\
    %             \{\pos A_1,                                             & \pos A_2, & \neg A_3, & \neg A_4,                                               &           & \pos A_6, & \neg A_7\}
    %             \quad\eqcomment{\set{\pos B_1,\pos B_3}}                                                                                                                                       \\
    %         \end{array}
    %     \end{equation}
    %     where $\muAprime$~\eqref{eq:ex1:muAprime} is the first in the list. Notice that in this case the assignments may be shorter and not pairwise disjoint.
    %     % }
    %     \exdone{}
    %     % Consider the partial truth assignment $\mu$ and its projection over $\allA$ as follows:
    %     % \begin{equation}
    %     %     \label{eq:ex1:mu}
    %     %     \mu\defas\set{A_3, A_4, A_5, \neg B_1, B_2, B_3, B_4}, \qquad
    %     %     \muA\defas\set{A_3, A_4, A_5}
    %     % \end{equation}
    %     % Whereas $\muA$ evaluates $\vi$ to true, $\mu$ does not evaluate to true $\vicnf$.
    %     % In fact, $\residual{\vicnf}{\mu}=(\neg A_1\vee\neg A_2)$.
    %     % Thus, in order to evaluate $\vicnf$ to true, $\mu$ must also assign either $\neg A_1$ or $\neg A_2$.
    %     %     % Suppose, for instance, that the solver finds $\mu$ and its projection over $\allA$ as follows
    %     %     % \begin{equation}
    %     %     %     \mu'\defas\set{\neg A_1, A_3, A_4, A_5, \neg B_1, B_2, B_3, B_4}, \qquad
    %     %     % \muAprime\defas\set{\neg A_1, A_3, A_4, A_5}
    %     %     % \end{equation}
    %     %     % and then adds the blocking clauses $(A_1\vee A_3, )$
    %     %     % Suppose such an assignment exists. Then, $\mu'$ must assign $B_2$, $B_3$ and $B_4$ to true because of the clauses in~\eqref{eq:ex1:vicnf:line2,,eq:ex1:vicnf:line3,,eq:ex1:vicnf:line4},
    %     %     % which represent the label definitions $(B_2\iff(B_1\vee A_3))$, $(B_3\iff(A_4\wedge B_2))$ and $(B_4\iff(A_3\iff B_3))$ respectively. 
    %     %     % Such an assignment does not yet evaluate to true the clauses in~\eqref{eq:ex1:vicnf:line1}, which represents the label definitions $(B_1\iff(A_1\wedge A_2))$.
    %     %     % In order to do so, $\mu'$ must also assign a truth value to $B_1$. However, if it assigns $B_1$ to true, then both $A_1$ and $A_2$ must be assigned to true. 
    %     %     % Similarly, if it assigns $B_1$ to false, then either $A_1$ or $A_2$ must be assigned to false.
    %     %     % Thus, $\mu'$ must assign a truth value also to some other atoms in \allA{} to evaluate $\vicnf$ to true, which contradicts the hypothesis.\exdone{}
    % \end{IGNORE}
\end{example}

%\ADDED{The example above shows an intrinsic problem of \TseitinCNF{} when used for enumeration.}
% \begin{IGNORE}    
%     In general, \cref{fact:tseitin} can be explained as follows. A
%     partial assignment \muA{} may suffice to satisfy a non-CNF formula 
%     $\vi(\allA)$  because it does not need to assign a truth value to the
%     atoms in \emph{all}
%     subformulas of $\vi$.
%     (In~\cref{ex1}, $\muA\defas\set{\neg A_3, \neg A_4, \neg A_7}$
%     satisfies \vi{} \eqref{eq:ex1:vi} without assigning a truth value to
%     the atoms in
%     the subformulas $\vi_1\defas(A_1\wedge A_2)$ and 
%     $\vi_3\defas(A_5\vee A_6)$.)
%     Let $\vi_i$ be one such subformula.
%     %
%     We recall that \TseitinCNF{} works by applying recursively the rewriting step (\sref{sec:bg:cnf}):
%     %\GMSIDENOTE{In \sref{sec:bg:cnf} abbiamo usato \DeMorganCNF($B_i\iff\vi_i$)}
%     \begin{eqnarray}
%     \label{eq:rewritingTseitin}
%     \vi\Longrightarrow\vi[\vi_i|B_i] \wedge (B_i\iff\vi_i)
%     \end{eqnarray}
%     %\vi\Longrightarrow\vi[\vi_i|B_i] \wedge \DeMorganCNF(B_i\iff\vi_i)$$
%     %
%     %$$\TseitinCNF{(\vi)}=\TseitinCNF{(\vi[\vi|B_i])}\wedge\DeMorganCNF{(B_i\iff\vi_i)}$$.
%     \noindent and then by recursively CNF-izing the two conjuncts.
%     Unfortunately, although \muA{} suffices to satisfy $\vi$,
%     \muA{} does not
%     suffice to satisfy $\exists B_i.( \vi[\vi_i|B_i]\wedge
%     (B_i\iff\vi_i))$, because to satisfy the second conjunct it
%     must assign a value also to some unassigned \allA{}-atoms in $\vi_i$,
%     no matter the value of $B_i$.
%     (In~\cref{ex1}, \eqref{eq:ex1:vicnf:line1} forces assigning also some
%     atom in \set{A_1,A_2} and \eqref{eq:ex1:vicnf:line3}  forces assigning also some
%     atom in \set{A_5,A_6}.)
% \end{IGNORE}
%   \RSTODO{RISCRITTO FIN QUi}

% \begin{IGNORE}
% %In fact, consider 
% \ADDED{Consider} a generic non-CNF formula $\vi(\allA)$ and a %minimal
% partial truth assignment $\muA$ that satisfies $\vi$, and
% let $\vi_i$ be some sub-formula of \vi{} which is not assigned a truth
% value by $\muA$
% % suppose $\muA$ does not assign
% % a truth value to some sub-formula $\vi_i$
% ---for
% instance, because $\vi_i$ occurs into some positive sub-formula $\vi_i\vee\vi_j$ and $\muA$ satisfies $\vi_j$. (In~\cref{ex1}, $\muA\defas\set{\neg A_3, \neg A_4, \neg A_7}$, $\vi_i\defas(A_1\wedge A_2)$ %and $\vi_j\defas(((A_3\vee A_4)\wedge(A_5\vee A_6))\iff A_7)$
% or $\vi_i\defas(A_5\vee A_6)$ respectively.)
% Then
% \TseitinCNF{}
% conjoins to the main formula the definition $(B_i\iff \vi_i)$, so that
% every
% % satisfying 
% partial truth assignment $\muAprime$
% \ADDED{satisfying $\exists \allB.\vicnfts$}
% is forced to
% assign a truth value to $\vi_i$ and thus to some of its atoms,
% which may not occur in $\muA$, so that $\muAprime\supset\muA$.
% %\end{rschange}
% %
% (In the example, the clauses
% in~\eqref{eq:ex1:vicnf:line1} and~\eqref{eq:ex1:vicnf:line3} force
% $\muAprime$ to assign a truth value also to $(A_1\wedge A_2)$ and
% $(A_5\vee A_6)$ respectively.)

% %
% Thus, by using \TseitinCNF{}, instead of enumerating one
% minimal partial truth assignment $\muA$ for \vi, the solver may {be forced to} enumerate
% many partial truth assignments $\muAprime$ that are minimal for
% ${\exists \allB.}\TseitinCNF(\vi)$ but %\ignoreinshort{
% they are %}
% not %\ignoreinshort{
% minimal %}
% for
% $\vi$, so that their number can be up to exponentially larger in the
% number of unassigned atoms in \muA.
% In fact, each such truth assignment \muAprime{} conjoins to \muA{} one of the (up to $2^{|\allA|-|\muA|}$) partial
% assignments which are needed to evaluate  to either $\top$ or $\bot$ all unassigned $\vi_i$'s.
% (E.g., in~\eqref{eq:ex1:muA:all} %\ignoreinshort{\ 
% and~\eqref{eq:ex1:muA:all:rep}, %}, 
% the solver enumerates nine
% $\muAprime$s by conjoining $\muA$~\eqref{eq:ex1:muA} with an
% exhaustive enumeration of partial assignments to $A_1,A_2,A_5,A_6$
% that  evaluate $(A_1\wedge A_2)$ and $(A_5\vee A_6)$ to either $\top$ or $\bot$.)
% This may drastically affect the effectiveness of the enumeration.

% \ignore{%%
%     The example above shows an intrinsic problem of \TseitinCNF{} when used for enumeration.\@ Consider a generic non-CNF formula $\vi(\allA)$ and a minimal
%     partial truth assignment $\muA$ that satisfies $\vi$ without
%     assigning a truth value to some sub-formula $\vi_i$ ---for
%     instance, because $\vi_i$ occurs into some positive sub-formula $\vi_i\vee\vi_j$ and $\muA$ satisfies $\vi_j$. (In~\cref{ex1}, $\muA\defas\set{\neg A_3, \neg A_4, \neg A_7}$, $\vi_i\defas(A_1\wedge A_2)$ %and $\vi_j\defas(((A_3\vee A_4)\wedge(A_5\vee A_6))\iff A_7)$
%     or $\vi_i\defas(A_5\vee A_6)$ respectively.)
%     \em Then, there is no guarantee that $\muA$ suffices to
%     satisfy $\exists \allB.\vicnfts$, i.e., that there exists some $\etaB$
%     such that $\muA\cup\etaB$ satisfies $\vicnfts$. In fact,
%     \TseitinCNF{}
%     conjoins to the main formula the definition $(B_i\iff \vi_i)$, so that every satisfying partial truth assignment $\muAprime$ is forced to assign a truth value to $\vi_i$ and to some of its atoms as well. (In the example, the clauses
%     in~\eqref{eq:ex1:vicnf:line1} and~\eqref{eq:ex1:vicnf:line3} force
%     $\muAprime$ to assign a truth value also to $(A_1\wedge A_2)$ and $(A_5\vee A_6)$ respectively).
%     As a consequence, by using \TseitinCNF{}, the solver may enumerate
%     partial truth assignments that, although minimal for $\exists \allB.\TseitinCNF(\vi)$, are not minimal for $\vi$ so that their number can be up to exponentially bigger in the number of unassigned atoms, thus affecting the effectiveness of the enumeration.
% }

% \end{IGNORE}

\subsection{The impact of Plaisted and Greenbaum CNF transformation}%
\label{sec:problem:polarity}

We point out that also the \PlaistedCNF{}
transformation~\cite{plaistedStructurepreservingClauseForm1986} suffers for the
same weakness as \TseitinCNF{} ---that is, \cref{fact:tseitin} holds also for
\PlaistedCNF{}--- although its effects are mitigated.

%   \noindent
In fact, we recall that \PlaistedCNF{} works by applying recursively the
rewriting step (\sref{sec:bg:cnf}):
\begin{eqnarray}
    \label{eq:rewritingPlaisted}
    \vi\Longrightarrow\vi[\vi_i|B_i] \wedge \left \{
    \begin{array}{lll}
        (B_i\imp\vi_i)  & \mbox{if $\vi_i$ occurs only positively in $\vi$}                \\
        (B_i\limp\vi_i) & \mbox{if $\vi_i$ occurs only negatively in $\vi$}                \\
        (B_i\iff\vi_i)  & \mbox{if $\vi_i$ occurs both positively and negatively in $\vi$} \\
    \end{array}\right \},
\end{eqnarray}
\noindent and then by recursively CNF-izing the two conjuncts.
%
As with \TseitinCNF{}, consider~\eqref{eq:rewritingPlaisted} s.t.\ $\vi_i$ is
some sub-formula of \vi{} whose atoms are not assigned by \muA{}.

If $\vi_i$ occurs both positively and negatively in $\vi$,
then~\eqref{eq:rewritingPlaisted} reduces to~\eqref{eq:rewritingTseitin} and
\PlaistedCNF{} behaves like \TseitinCNF{}, so that \muA{} does not satisfy
$\exists B_i.( \vi[\vi_i|B_i]\wedge (B_i\iff\vi_i))$.

If instead $\vi_i$ occurs only positively \resp{negatively} in $\vi$, then it
is possible to extend \muA{} by assigning $B_i$ to $\bot$ \resp{$\top$} to
satisfy $(B_i\imp\vi_i)$ \resp{$(B_i\limp\vi_i)$} without assigning any atom in
$\vi_i$. Thus $\muA$ satisfies $\exists B_i.( \vi[\vi_i|B_i]\wedge
    (B_i\imp\vi_i))$ \resp{$\exists B_i.( \vi[\vi_i|B_i]\wedge (B_i\limp\vi_i))$}.

As with \TseitinCNF{}, given $\muA$ satisfying \vi{}, in order to produce an
assignment \muAprime{} satisfying $\exists \allB.\vicnfpg$ the enumerator is
most often forced to assign other atoms in \allA{}, so that
$\muAprime\supset\muA$. We notice, however, that the effect of this problem is
mitigated by the presence of single-polarity sub-formulas among those left
unassigned by \muA{}. As an extreme case, if all sub-formulas in $\vi$ occur
with single polarity, then no further assignment to atoms in \allA{} is needed. %as a consequence of \cref{fact:tseitin},%(unnecessarily)
%

\begin{example}%
    \label{ex2}
    Consider the formula $\vi$~\eqref{eq:ex1:vi} as in
    \cref{ex1}. Suppose that $\vi$ is converted into CNF using
    \PlaistedCNF{}. Then, we have:
    \begin{subequations}%
        \label{eq:ex2:vicnf}
        \begin{alignat}{2}
             & \vicnfpg\defas\nonumber                                                                                                         \\
             & \enspace(\neg B_1\vee\pos A_1)\wedge(\neg B_1\vee\pos A_2)                                         & \wedge
             & \quad\eqcomment{(B_1\imp (A_1\wedge A_2))}\label{eq:ex2:vicnf:line1}                                                            \\
             & \enspace(\pos B_2\vee\neg A_3)\wedge(\pos B_2\vee\neg A_4)\wedge(\neg B_2\vee\pos A_3\vee\pos A_4) & \wedge
             & \quad\eqcomment{(B_2\iff (A_3\vee A_4))}\label{eq:ex2:vicnf:line2}                                                              \\
             & \enspace(\pos B_3\vee\neg A_5)\wedge(\pos B_3\vee\neg A_6)\wedge(\neg B_3\vee\pos A_5\vee\pos A_6) & \wedge
             & \quad\eqcomment{(B_3\iff (A_5\vee A_6))}\label{eq:ex2:vicnf:line3}                                                              \\
             & \enspace(\neg B_4\vee\pos B_2)\wedge(\neg B_4\vee\pos B_3)\wedge(\pos B_4\vee\neg B_2\vee\neg B_3) & \wedge
             & \quad\eqcomment{(B_4\iff (B_2\wedge B_3))}\label{eq:ex2:vicnf:line4}                                                            \\
             & \enspace(\neg B_5\vee\pos B_4\vee\neg A_7)\wedge(\neg B_5\vee\neg B_4\vee\pos A_7)                 & \wedge
             & \quad\eqcomment{(B_5\imp (B_4\iff A_7))}\label{eq:ex2:vicnf:line5}                                                              \\
             & \enspace(\pos B_1\vee\pos B_5).                                                                    & \label{eq:ex2:vicnf:line6}
        \end{alignat}
    \end{subequations}
    We remark that~\eqref{eq:ex2:vicnf:line1}
    and~\eqref{eq:ex2:vicnf:line5} are shorter
    than~\eqref{eq:ex1:vicnf:line1} and~\eqref{eq:ex1:vicnf:line5}
    respectively, since the corresponding sub-formulas $(A_1\wedge
        A_2)$ and $((\dots)\iff A_7)$ occur only with
    positive polarity in \vi{}~\eqref{eq:ex1:vi}, so that only the one-way implication
    $(B_i\imp\vi_i)$ is needed.

    As in \cref{ex1}, consider the partial assignment $\muA\defas\set{\neg A_3,\neg
            A_4,\neg A_7}$~\eqref{eq:ex1:muA} which satisfies $\vi{}$~\eqref{eq:ex1:vi}. As
    before, $\muA$ does not satisfy $\exists\allB.\vicnfpg$. In fact, {there is no
            total truth assignment $\etaB$ on \allB{} such that
            $\muA\cup\etaB\pmodels\vicnfpg$}, because~\eqref{eq:ex2:vicnf:line3} cannot be
    satisfied by assigning only variables in $\allB$; rather, it is necessary to
    further assign at least one atom in
    %    $\set{A_1,A_2}$ to satisfy ~
    %    \eqref{eq:ex2:vicnf:line1}.
    %    and at least one in
    $\set{A_5,A_6}$ to satisfy~\eqref{eq:ex2:vicnf:line3}.
    Notice that, unlike with \cref{ex1}, in order to satisfy~\eqref{eq:ex2:vicnf:line1} %and  \eqref{eq:ex2:vicnf:line5}
    it is sufficient to set $B_1=\bot$ with no need to assign any atom in $\set{A_1,A_2}$.

    Suppose an enumerator assigns first the literals in \muA, which force it to
    assign also $\muB\defas\set{\neg B_2, \neg B_4}$ due
    to~\eqref{eq:ex2:vicnf:line2} and~\eqref{eq:ex2:vicnf:line4} respectively.
    %
    Since $\muA\cup\muB$ satisfies all clauses except those in
    \eqref{eq:ex2:vicnf:line1}, \eqref{eq:ex2:vicnf:line3},
    %\eqref{eq:ex2:vicnf:line5},
    and \eqref{eq:ex2:vicnf:line6}, the enumerator needs extending $\muA\cup\muB$
    by enumerating partial assignments on the unassigned atoms
    \set{A_1,A_2,A_5,A_6,B_1,B_3,B_5} which satisfy them.
    %
    Regardless of the search strategy adopted, {to
            satisfy~\eqref{eq:ex2:vicnf:line3} it is necessary to generate no less than $3$
            partial assignments} on $\set{A_5,A_6}$ (see \cref{ex1}); to satisfy
    \eqref{eq:ex2:vicnf:line1},
    %\eqref{eq:ex2:vicnf:line5},
    and \eqref{eq:ex2:vicnf:line6}, instead, the enumerator needs only assigning
    $B_1=\bot$, which forces it to assign $B_5=\top$ due to
    \eqref{eq:ex2:vicnf:line6}.

    % , and
    % \eqref{eq:ex2:vicnf:line5} respectively. 
    For instance, in the case of disjoint AllSAT, instead of the single partial
    assignment $\muA$, the solver may return the following list of 3 partial
    assignments satisfying $\exists\allB.\vicnfpg$ which extend \muA{}:%\ignoreinshort{%}
    \begin{equation}%
        \label{eq:ex2:muA:all}
        \begin{array}{llllllll}
            \multicolumn{2}{c}{} &  &           & \multicolumn{2}{c}{\overbrace{\rule{1.5cm}{0pt}}^{B_3}} &           &                        \\
            \{                   &  & \neg A_3, & \neg A_4,                                               & \neg A_5, & \neg A_6, & \neg A_7\}
            \quad\eqcomment{\set{\neg B_1,\neg B_2,\neg B_3,\neg B_4,\pos B_5}}                                                                \\
            \{                   &  & \neg A_3, & \neg A_4,                                               & \pos A_5, &           & \neg A_7\}
            \quad\eqcomment{\set{\neg B_1,\neg B_2,\pos B_3,\neg B_4,\pos B_5}}                                                                \\
            \{                   &  & \neg A_3, & \neg A_4,                                               & \neg A_5, & \pos A_6, & \neg A_7\}
            \quad\eqcomment{\set{\neg B_1,\neg B_2,\pos B_3,\neg
            B_4,\pos B_5}}                                                                                                                     \\
        \end{array}
    \end{equation}
    In the case of non-disjoint AllSAT, the solver may enumerate a similar
    set of partial assignments, with
    \set{A_6}
    instead of  \set{\neg A_5,A_6} in the third assignment.\exdone{}

    % Indeed, sub-formulas occurring with double polarity are labeled using double implications as for \TseitinCNF{}, raising the same problems as the latter. For instance, the sub-formula $(A_5\vee A_6)$ occurs with double polarity, since it is under the scope of an ``$\iff$''. Hence, the clauses in~\eqref{eq:ex2:vicnf:line3} must be satisfied by assigning a truth value also to $A_5$ or $A_6$, and so the partial truth assignment $\muA$ in~\eqref{eq:ex1:muA} does not suffice to satisfy $\exists\allB.\vicnfpg$. %because $\residual{\vicnfpg}{\muA\cup\etaB}=(\neg A_5)\wedge(\neg A_6)$.
    % \exdone{}
\end{example}

% The example above shows that \PlaistedCNF{} has an advantage over
% \TseitinCNF{} when enumerating partial assignments, but it overcomes
% its effectiveness issues only in part, {\em because a minimal assignment
%         $\muA$ satisfying $\vi$ may not suffice to satisfy
%         $\exists\allB.\vicnfpg$}, as with \TseitinCNF.

% \begin{IGNORE}
%     Consider, as in~\sref{sec:problem:label}, a generic non-CNF formula $\vi(\allA)$ and a partial truth assignment $\muA$ that satisfies $\vi$ without assigning a truth value to some sub-formula $\vi_i$. Suppose that $\vi_i$ occurs only positively in $\vi$ ---for the negative case the reasoning is dual. (In~\cref{ex2}, $\muA\defas\set{\neg A_3, \neg A_4, \neg A_7}$, $\vi_i\defas(A_1\wedge A_2)$.)\@
%     % and $\vi_j\defas((A_3\vee A_4)\wedge(A_5\vee A_6))\iff A_7$
%     Since \PlaistedCNF{} introduces only the clauses representing $(B_i\imp\vi_i)$
%     ---and not those representing $(B_i\limp\vi_i)$--- the solver is no longer
%     forced to assign a truth value to $\vi_i$, because it suffices to assign
%     $\etaB(B_i)=\bot$. (In the example, $(A_1\wedge A_2)$ is labelled with $B_1$
%     in~\eqref{eq:ex2:vicnf:line1}.) In this case, $\vi_i$ plays the role of a
%     ``don't care'' term, and this property allows for the enumeration of shorter
%     partial assignments.

%     Nevertheless, a sub-formula can be ``don't care'' only if it occurs with single
%     polarity. In fact, if $\vi_i$ occurs with double polarity ---as it is the case,
%     e.g., of sub-formulas under the scope of an ``$\iff$''--- then $\vi_i$ is
%     labeled with a double implication $(B_i\iff\vi_i)$, yielding the same drawbacks
%     as with \TseitinCNF{}. (In the example, $(A_5\vee A_6)$ occurs with double
%     polarity, and $\muAprime$ is forced to assign a truth value also to $A_5$ or
%     $A_6$ to satisfy the clauses in~\eqref{eq:ex2:vicnf:line3}.)
% \end{IGNORE}

Notice that, to maximize the benefits of \PlaistedCNF{}, the sub-formulas
occurring with positive \resp{negative} polarity only must have their label
assigned to false \resp{true}. {In practice, this can be achieved in part by
instructing the solver to split on negative values in decision
branches.~\footnote{To exploit this heuristic also for sub-formulas occurring
    only negatively, the latter can be labeled with a negative label $\neg B_i$ as
    $(\neg B_i\limp\vi_i)$.} Even though the solver is not guaranteed to always
assign to false these labels, we empirically verified that this heuristic
provides a good approximation of this behavior.}

% \end{rschange}



\section{Enhancing enumeration via NNF preprocessing}%
\label{sec:solution}
In this section, we propose a possible solution to address the shortcomings of \TseitinCNF{} and \PlaistedCNF{} CNF-izations in model enumeration, described in~\sref{sec:problem}. We show that a simple preprocessing can avoid this situation. We transform the input formula into NNF, using a DAG representation to avoid the exponential blow-up due to ``$\iff$''s, and then use \PlaistedCNF{} on the resulting NNF formula.\@ In fact, NNF guarantees that each sub-formula occurs only positively, as every sub-formula $\vi_i$ occurring with double polarity is converted into two syntactically different sub-formulas $\poslab{\vi_i}\defas\NNF{\vi_i}$ and $\neglab{\vi_i}\defas\NNF{\neg\vi_i}$ ---each occurring only positively--- which are then labelled ---with single implications--- with two distinct atoms $\poslab{B_i}$ and $\neglab{B_i}$ respectively. We first illustrate the benefit of this additional preprocessing with the following example.

\begin{example}%
    \label{ex3}
    Consider the formula $\vi$ of~\cref{ex1}. By converting it into NNF, we obtain:
    \begin{equation}
        \vinnf\defas
        \overbrace{(A_1\wedge A_2)}^{B_1}\vee
        \underbrace{(
            \overbrace{(
                \underbrace{(
                    \overbrace{(\neg A_3\wedge\neg A_4)}^{\neglab{B_2}}\vee
                    \overbrace{(\neg A_5\wedge\neg A_6)}^{\neglab{B_3}}
                )}_{\neglab{B_4}}\vee A_7
            )}^{B_5} \wedge
            \overbrace{(
                \underbrace{(
                    \overbrace{(     A_3\vee       A_4)}^{\poslab{B_2}}\wedge
                    \overbrace{(     A_5\vee       A_6)}^{\poslab{B_3}}
                )}_{\poslab{B_4}}\vee\neg A_7
            )}^{B_6}
        )}_{B_{7}}
    \end{equation}
    % We remark that, by definition, each sub-formula of $\vinnf$ occurs only with positive polarity. For instance, consider the sub-formula $(A_3\vee A_4)$ that occurs in $\vi$ with double polarity. In $\vinnf$, the positive occurrence remains the same, while the negative occurrence is converted into $(\neg A_3\wedge\neg A_4)$. This implies that the two occurrences correspond to two different sub-formula, each occurring only positively, and thus they will be labelled with two different atoms.
    Suppose, then, that the formula is converted into CNF using \PlaistedCNF{}. Then, the following CNF formula is obtained:
    \begin{subequations}%
        \label{eq:ex3:vicnf}
    \begin{alignat}{2}
        \vinnfcnfpg\defas
        &(\neg B_1\vee\pos A_1)\wedge(\neg B_1\vee\pos A_2)&\wedge
        &\quad\eqcomment{(B_1\imp (\pos A_1\wedge\pos A_2))}\label{eq:ex3:vicnf:line1}\\
        &(\neg \neglab{B_2}\vee\neg A_3)\wedge(\neg \neglab{B_2}\vee\neg A_4)&\wedge&\quad\eqcomment{(\neglab{B_2}\imp (\neg A_3\wedge\neg A_4))}\label{eq:ex3:vicnf:line2}\\
        &(\neg \neglab{B_3}\vee\neg A_5)\wedge(\neg \neglab{B_3}\vee\neg A_6)&\wedge&\quad\eqcomment{(\neglab{B_3}\imp (\neg A_5\wedge\neg A_6))}\label{eq:ex3:vicnf:line3}\\
        &(\neg \neglab{B_4}\vee\pos \neglab{B_2}\vee\pos \neglab{B_3})&\wedge&\quad\eqcomment{(\neglab{B_4}\imp (\pos \neglab{B_2}\vee\pos \neglab{B_3}))}\label{eq:ex3:vicnf:line4}\\
        &(\neg B_5\vee\pos \neglab{B_4}\vee\pos A_7)&\wedge&\quad\eqcomment{(B_5\imp (\pos \neglab{B_4}\vee\pos A_7))}\label{eq:ex3:vicnf:line5}\\
        &(\neg \poslab{B_2}\vee\pos A_3\vee\pos A_4)&\wedge&\quad\eqcomment{(\poslab{B_2}\imp (\pos A_3\vee\pos A_4))}\label{eq:ex3:vicnf:line6}\\
        &(\neg \poslab{B_3}\vee\pos A_5\vee\pos A_6)&\wedge&\quad\eqcomment{(\poslab{B_3}\imp (\pos A_5\vee\pos A_6))}\label{eq:ex3:vicnf:line7}\\
        &(\neg \poslab{B_4}\vee\pos \poslab{B_2})\wedge(\neg \poslab{B_4}\vee\pos \poslab{B_3})&\wedge&\quad\eqcomment{(\poslab{B_4}\imp (\pos \poslab{B_2}\wedge\pos \poslab{B_3}))}\label{eq:ex3:vicnf:line8}\\
        &(\neg B_6\vee\pos \poslab{B_4}\vee\neg A_7)&\wedge&\quad\eqcomment{(B_6\imp (\pos \poslab{B_4}\vee\neg A_7))}\label{eq:ex3:vicnf:line9}\\
        &(\neg B_{7}\vee\pos B_5)\wedge(\neg B_{7}\vee\pos B_6)&\wedge&\quad\eqcomment{(B_{7}\imp (\pos B_5\wedge\pos B_6))}\label{eq:ex3:vicnf:line10}\\
        &(\pos B_1\vee\pos B_{7})&&\label{eq:ex3:vicnf:line11}
    \end{alignat}
    \end{subequations}
    Suppose, e.g., that the solver picks non-deterministic choices, deciding the atoms in the order 
    $\set{B_1,A_1,A_2,\neglab{B_3},A_5,A_6,\neglab{B_2},A_3,A_4,\neglab{B_4},B_5,A_7,\poslab{B_2},\poslab{B_3},\poslab{B_4},B_6,B_{7}}$,
    branching with a negative value first.
    Then, the first total truth assignment found is:
    \begin{equation}
        \label{ex3:eta}
        \eta\defas\set{
            \underbrace{\neg B_1, \neglab{B_2},\neg \neglab{B_3},\neglab{B_4},B_5,\neg \poslab{B_2},\neg \poslab{B_3},\neg \poslab{B_4},B_6,B_{7}}_{\etaB},
            \underbrace{\neg A_1,\neg A_2,\neg A_3,\neg A_4,\neg A_5,\neg A_6,\neg A_7}_{\etaA}}.
    \end{equation}

    In this case, the minimization procedure returns $\muA\defas\set{\neg A_3,\neg A_4, \neg A_7}$ as in~\eqref{eq:ex1:muAprime}, achieving full minimization. With this additional preprocessing, in fact, the solver is no longer forced to assign a truth value to $A_5$ or $A_6$.
    This is possible because, even though $(A_5\vee A_6)$ occurs with double polarity in $\vi$, in $\NNF{\vi}$ its positive and negative occurrences are converted into $(A_5\vee A_6)$ and $(\neg A_5\wedge\neg A_6)$ respectively. Since they appear as two syntactically-different sub-formulas, \PlaistedCNF{} labels them ---with single implications--- using two different atoms $\poslab{B_3}$ and $\neglab{B_3}$ respectively. This allows the solver to find a model $\eta$ that assigns both $\neglab{B_3}$ and $\poslab{B_3}$ to false. Hence, the clauses in \eqref{eq:ex3:vicnf:line3} and~\eqref{eq:ex3:vicnf:line7} are satisfied even without assigning $A_5$ and $A_6$, and thus these atoms can be dropped by the minimization procedure.
    \exdone{}
\end{example}

This example shows that the solver can enumerate shorter partial assignments if the formula is converted into NNF before applying \PlaistedCNF{}. The key idea behind this additional preprocessing is that each sub-formula of $\NNF{\vi}$ occurs only positively, so that \PlaistedCNF{} labels them with single implications, and the solver is no longer forced to assign them a truth value. 
Consider a sub-formula $\vi_i$ that occurs with double polarity in
$\vi$.
In $\NNF{\vi}$ the two subformulas
$\poslab{\vi_i}\defas\NNF{\vi_i}$ and
$\neglab{\vi_i}\defas\NNF{\neg\vi_i}$ occur only positively.
Then, 
instead of adding $(B_i\iff\vi_i)$, we add $(\poslab{B_i}\imp\poslab{\vi_i})\wedge(\neglab{B_i}\imp\neglab{\vi_i})$, and the solver can find a truth assignment $\eta$ that assigns both $\neglab{B_i}$ and $\poslab{B_i}$ to false. (In~\cref{ex3}, instead of $(B_3\iff(A_5\vee A_6))$ we add $(\poslab{B_3}\imp(A_5\vee A_6))$ and $(\neglab{B_3}\imp(\neg A_5\wedge\neg A_6))$.) Thus, the clauses deriving from $\vi_i$ can be satisfied even without assigning a truth value to $\vi_i$, whose atoms can be dropped by the minimization procedure ---provided that they are not forced to be assigned by some other sub-formula of $\vi$. (In the example, by setting $\etaB(\poslab{B_3})=\etaB(\poslab{B_3})=\bot$, the clauses in~\eqref{eq:ex3:vicnf:line3} and~\eqref{eq:ex3:vicnf:line7} are satisfied even without assigning $A_5$ and $A_6$.)
To improve the efficiency of the procedure, we also add the clause $(\neg \poslab{B_i}\vee\neg\neglab{B_i})$ in order to prevent the solver from assigning both $\poslab{B_i}$ and $\neglab{B_i}$ to true, and thus from exploring inconsistent search branches.

\begin{remark}%
    \label{rem:ex3:preconv}
    We notice that the pre-conversion into NNF is typically never used in plain SAT \emph{solving}, because it causes the unnecessary duplication of labels $\poslab{B_i}$ and $\neglab{B_i}$, with extra overhead and no benefit for the solver.
\end{remark}


\section{Experimental evaluation}%
\label{sec:experiments}
\section{Experiments}

We compare our proposed architecture against baselines on both single object datasets (\ourdatacad, \ourdatareal from \citet{jiang2022opd}) and the new multiple object dataset we created (\ourdatamulti).
We also conduct an analysis of part consistency on single objects and the challenges of handling multiple objects.
In the main paper, we present experiments for RGB input images.
See the supplement for results with depth only (D) and RGBD, and additional analysis.

\subsection{Implementation details} 
Our architecture is based on Mask2Former~\cite{cheng2021masked} as implemented in Detectron2~\cite{wu2019detectron2}.
We use the R-50 backbone Mask2Former model pretrained on COCO~\cite{lin2014microsoft} instance segmentation to initialize our weights, and train with AdamW~\cite{loshchilov2017decoupled}.
The learning rate and other hyperparameters match those used by Mask2Former.
Our experiments are carried out on a machine with 64GB RAM and an RTX 2080Ti GPU.
We train each model end-to-end for $60000$ steps and pick the best checkpoint based on val set performance (+\mtype{}\axis{}\orig).
Models evaluated on \ourtask are first pretrained on \ourdatareal and then finetuned on the \ourdatamulti train split.
The \opdrcnn baselines are first pretrained on \ourdatacad, then \ourdatareal, and finally \ourdatamulti.

We note that the predicted object pose rotation matrix is not guaranteed to be a valid rotation matrix. \citet{jiang2022opd} did not address this issue.
We convert the predicted rotation matrix into a unit quaternion and back using PyTorch3D~\cite{ravi2020accelerating} to ensure a valid rotation.
The results for \opdnet are approximately the same as without such post-processing.
For \ourdatamulti, we use a confidence threshold of $0.8$ to determine whether a predicted part is valid.


\subsection{Experimental setup}

For single objects, we evaluate our method on two datasets introduced in OPD \cite{jiang2022opd}, \ourdatacad and \ourdatareal.
For multiple objects, we evaluate on \ourdatamulti.

\mypara{Metrics.}
We use the evaluation metrics for part detection and motion prediction from \citet{jiang2022opd}. 
The metrics extend the traditional mAP metric for detection to the motion prediction task, including two main metrics: mAP@IoU=0.5 for the predicted part label and 2D bounding box (\partdet).
For each metric, the detection is further constrained by whether: motion type is matched (+\mtype{}), motion type and motion axis are matched (+\mtype{}\axis{}), and whether motion type, axis and origin are all matched (+\mtype{}\axis{}\orig), within predefined error thresholds.
Note that \citet{jiang2022opd}'s metrics were only defined for inputs with openable parts.
Since we have frames with no openable parts, we measure the percentage of those we correctly predicted as having no openable parts.

\mypara{Methods.}
We compare variants of our \opdformerbaseline with the MaskRCNN-based \opdrcnn~\cite{jiang2022opd}.
We compare the following variants: predicting directly in camera coordinates (\textsc{-C}), vs predicting a single global pose (\textsc{-O}) vs predicting per-part object poses (\textsc{-P}). 

\begin{table}
\resizebox{\linewidth}{!}
{
\begin{tabular}{@{} ll rrrr @{}}
\toprule
& & \multicolumn{4}{c}{Part-averaged mAP $\% \uparrow$} \\
\cmidrule(l{0pt}r{2pt}){3-6}
Dataset & Model & \partdet & +\mtype & +\mtype{}\axis & +\mtype{}\axis{}\orig \\
\midrule
\multirow{7}{*}{\ourdatacad}
& \opdrcnnc~\cite{jiang2022opd} & 74.3\std{0.27} & 72.3\std{0.29} & 40.2\std{0.09} & 36.5\std{0.17} \\
& \opdrcnno~\cite{jiang2022opd} & 74.2\std{0.34} & 72.4\std{0.32} & 52.4\std{0.27} & 47.0\std{0.36} \\
& \opdrcnnop & 73.2\std{0.64} & 71.2\std{0.69} & 51.6\std{0.47} & 44.8\std{0.32} \\
& \opdformerc & 77.3\std{0.40} & 74.9\std{0.42} & 48.9\std{0.23} & 43.9\std{0.09} \\
& \opdformero & 77.8\std{0.54} & 75.7\std{0.47} & 57.5\std{0.15} & 52.4\std{0.35} \\
& \opdformerp & \best{79.0}\std{0.23} & \best{76.7}\std{0.23} & \best{58.6}\std{0.94} & \best{53.4}\std{0.28} \\
\midrule
\multirow{7}{*}{\ourdatareal}
& \opdrcnnc~\cite{jiang2022opd} & 57.6\std{0.10} & 55.5\std{0.24} & 15.6\std{0.28} & 14.7\std{0.29}\\
& \opdrcnno~\cite{jiang2022opd} & 57.0\std{0.49} & 54.7\std{0.57} & 27.9\std{0.49} & 25.7\std{0.41} \\
& \opdrcnnop & 57.6\std{0.62} & 54.7\std{0.59} & 26.9\std{0.03} & 25.1\std{0.19} \\
& \opdformerc & 57.9\std{1.31} & 56.0\std{1.09} & 29.7\std{0.51} & 28.3\std{0.49} \\
& \opdformero & \best{61.8}\std{0.58} & \best{59.4}\std{0.55} & 31.2\std{0.58} & 28.9\std{0.57} \\
& \opdformerp  & 58.8\std{0.66} & 56.2\std{0.58} & \best{35.4}\std{0.20} & \best{33.7}\std{0.18} \\
\midrule
\multirow{7}{*}{\ourdatamulti}
& \opdrcnnc~\cite{jiang2022opd} & 27.3\std{0.10} & 25.7\std{0.10} & 8.8\std{0.25} & 7.8\std{0.20} \\
& \opdrcnno~\cite{jiang2022opd} & 20.2\std{0.42} & 18.3\std{0.62} & 3.9\std{0.07} & 0.5\std{0.12} \\
& \opdrcnnop & 20.9\std{0.44} & 19\std{0.35} & 7.2\std{0.25} & 5.7\std{0.22} \\
& \opdformerc & 30.3\std{1.02} & 28.9\std{0.99} & 13.1\std{0.55} & 12.1\std{0.49} \\
& \opdformero & 30.1\std{0.15} & 28.5\std{0.18} & 5.2\std{0.10} & 1.6\std{0.03} \\
& \opdformerp & \best{32.9}\std{0.69} & \best{31.6}\std{0.72} & \best{19.4}\std{0.38} & \best{16.0}\std{0.03} \\
\bottomrule
\end{tabular}
}
\caption{Comparison of \opdrcnn and \opdformerbaseline on validation set RGB input images for the three datasets (\ourdatacad and \ourdatareal for single-object, and \ourdatamulti for multiple-object real scenes).
Our \opdformerbaseline variants outperform baselines especially on the multi-object inputs from \ourdatamulti.}
\label{tab:results-OPDAll-val-mini}
\end{table}


\begin{figure*}
\centering
\setkeys{Gin}{width=\linewidth}
\begin{tabularx}{\linewidth}{Y Y Y Y Y Y}
\toprule
\small{GT} &
\imgclip{0}{figure/qua_single/gt/174-8460__0.png} & 
\imgclip{0}{figure/qua_single/gt/174-8460__1.png} &
\imgclip{0}{figure/qua_single/gt/174-8460__2.png} &
\imgclip{0}{figure/qua_single/gt/241-5820__0.png} &
\imgclip{0}{figure/qua_single/gt/241-5820__1.png}\\

\small{\opdrcnnop} &
\imgclip{0}{figure/qua_single/opdrcnnp/174-8460__1.png} & 
\imgclip{0}{figure/qua_single/opdrcnnp/174-8460__0.png} &
Miss &
\imgclip{0}{figure/qua_single/opdrcnnp/241-5820__0.png} &
\imgclip{0}{figure/qua_single/opdrcnnp/241-5820__1.png}\\
\small{Axis (origin) error} & 7.537 & 2.098 & - & 6.078 (0.067) & 8.752 (0.149)\\

\small{\opdformerp} &
\imgclip{0}{figure/qua_single/opdformerp/174-8460__1.png} & 
\imgclip{0}{figure/qua_single/opdformerp/174-8460__0.png} &
\imgclip{0}{figure/qua_single/opdformerp/174-8460__2.png} &
\imgclip{0}{figure/qua_single/opdformerp/241-5820__0.png} &
\imgclip{0}{figure/qua_single/opdformerp/241-5820__1.png}\\
\small{Axis (origin) error} & 2.131 & 2.956 & 2.153 & 3.247 (0.019) & 1.975 (0.06)\\

\midrule

\small{GT} &
\imgclip{0}{figure/qua_multi/gt/106-5040__0.png} & 
\imgclip{0}{figure/qua_multi/gt/106-5040__1.png} &
\imgclip{0}{figure/qua_multi/gt/252-4800__0.png} &
\imgclip{0}{figure/qua_multi/gt/252-4800__1.png} &
\imgclip{0}{figure/qua_multi/gt/252-4800__2.png}\\

\small{\opdrcnnop} &
\imgclip{0}{figure/qua_multi/opdrcnnp/106-5040__0.png} & 
Miss &
\imgclip{0}{figure/qua_multi/opdrcnnp/252-4800__1.png} &
\imgclip{0}{figure/qua_multi/opdrcnnp/252-4800__2.png} &
\imgclip{0}{figure/qua_multi/opdrcnnp/252-4800__0.png}\\
\small{Axis (origin) error} & 4.458 (0.013) & - & 6.329 (0.057) & 5.757 (0.056) & 8.305 (0.029)\\

\small{\opdformerp} &
\imgclip{0}{figure/qua_multi/opdformerp/106-5040__0.png} & 
\imgclip{0}{figure/qua_multi/opdformerp/106-5040__1.png} &
\imgclip{0}{figure/qua_multi/opdformerp/252-4800__0.png} &
\imgclip{0}{figure/qua_multi/opdformerp/252-4800__2.png} &
\imgclip{0}{figure/qua_multi/opdformerp/252-4800__1.png}\\
\small{Axis (origin) error} & 2.021 (0.087) & 4.099(0.234) & 1.038 (0.058) & 1.619 (0.063) & 1.569 (0.096)\\

\bottomrule
\end{tabularx}
\caption{
Example predictions on the \ourdatamulti val split. 
The first row in each group is the ground truth (GT) with the motion axis in green.  
The following rows are predictions from \opdrcnnop and \opdformerp with axis error and origin error indicated if the motion type is rotation.
The GT axis is in \textcolor{blue}{blue} and the predicted axis is in \textcolor{green}{green} if it is within $5^\circ$ of the GT, \textcolor{orange}{orange} if between $5^\circ$ and $10^\circ$, and \textcolor{red}{red} if the angle difference is greater than $10^\circ$.
The axis origin is visualized with the same color scheme using error thresholds of 0.1 and 0.25.
Overall, \opdformerp provides significantly more accurate openable part predictions, in particular for the scenarios in the bottom that contain multiple objects or multiple parts.
}
\label{fig:vis-compare-real-multi}
\end{figure*}


\subsection{Results}

\Cref{tab:results-OPDAll-val-mini} evaluates the different methods on RGB input images from the val set of the three datasets.  We report the mean and standard error across three runs with different seeds. 
See the supplement for depth and RGB-D input results, motion averaged metrics, and for performance on the test set.
\Cref{fig:vis-compare-real-multi} shows example predictions on \ourdatamulti, and the supplement provides qualitative results on \ourdatacad and \ourdatareal.


Our \opdformerbaseline variants outperform the \opdrcnn baselines on all metrics.
One reason is the stronger part detection (\partdet) provided by the Mask2Former backbone.
We note that the \opdformerbaseline variants with the R50 backbone actually have fewer parameters than \opdrcnn methods, indicating that the performance gains are not due to increased parameters.
For example, \opdrcnnop has 46.1M parameters whereas \opdformerp has 42.0M parameters.
This observation is similar for other \opdformerbaseline variants and corresponding \opdrcnn baselines (see supplement).   


\mypara{Are camera coordinates useful?}
As observed in \citet{jiang2022opd}, predicting motion parameters in object coordinates and predicting the object pose (\opdrcnno) outperform prediction in camera coordinates (\opdrcnnc).
This is true for the single object case (\ourdatacad and \ourdatareal), but not for \ourdatamulti where the assumption of one global object coordinate does not hold.

\mypara{Is having per-part object pose prediction important?}
When we take motion parameters into account, we see the advantage of per-part object pose predictions with the transformer-based architecture (\opdformerp).
On the main metric (+\mtype{}\axis{}\orig), our per-part \opdformerp consistently outperforms the global \opdformero, which in turn outperforms \opdrcnno by \citet{jiang2022opd}.
Interestingly, \opdrcnnop does not help over \opdrcnno for the single object scenario.

\begin{table}
\resizebox{\linewidth}{!}{
\begin{tabular}{@{} l r rr rr @{}}
\toprule
 & No AO $\% \uparrow$ & \multicolumn{2}{c}{Single AO $\% \uparrow$} & \multicolumn{2}{c}{Multiple AO $\% \uparrow$} \\
\cmidrule(l{0pt}r{2pt}){2-2} \cmidrule(l{2pt}r{0pt}){3-4} \cmidrule(l{2pt}r{0pt}){5-6}
Model & \textbf{Accuracy} & \partdet & +\mtype{}\axis{}\orig & \partdet & +\mtype{}\axis{}\orig  \\
\midrule
\opdrcnnc~\cite{jiang2022opd}  & \best{58.6} & 43.6 & 11.8 & 37.6 & 8.6\\
\opdrcnno~\cite{jiang2022opd}  & 57.5 & 34.8 & 0.5 & 30.5 & 0.4\\
\opdrcnnop & 50.8 & 40.0 & 10.0 & 34.0 & 8.9 \\
\opdformerc & 27.3 & 60.1 & 21.9 & 36.1 & 14.6 \\
\opdformero  & 16.7 & 59.4 & 2.5 & 35.8 & 1.5\\
\opdformerp & 35.0 & \best{61.4} & \best{28.7} & \best{40.2} & \best{15.2} \\
\bottomrule
\end{tabular}
}
\caption{We compare the performance of the models for images with no/one/multiple articulated objects (AO) on the \ourdatamulti validation set.  For `No AO', we compute the percent of frames for which the method correctly predicted there was no openable parts.
}
\label{tab:results-OPDMulti-split}
\end{table}


\begin{table}[t]
\resizebox{\linewidth}{!}{
\begin{tabular}{@{} ll rr rr @{}}
\toprule
& & \multicolumn{2}{c}{Pose Rotation} & \multicolumn{2}{c}{Pose Translation} \\
\cmidrule(l{0pt}r{2pt}){3-4} \cmidrule(l{2pt}r{0pt}){5-6}
Dataset & Model & MedErr $\downarrow$ & Acc:5 $\uparrow$ & MedErr $\downarrow$ & Acc:0.1 $\uparrow$ \\
\midrule
\multirow{2}{*}{\ourdatacad} 
& \opdrcnnop & 4.28 & 0.58 & 0.16 & 0.28 \\
& \opdformerp &  \best{2.47} & \best{0.78} & \best{0.11} & \best{0.46}  \\
\midrule
\multirow{2}{*}{\ourdatareal} 
& \opdrcnnop & 8.33 & 0.23 & 0.19 & 0.16 \\
& \opdformerp & \best{4.96} & \best{0.51} & \best{0.14} & \best{0.29} \\
\midrule
\multirow{2}{*}{\ourdatamulti} 
& \opdrcnnop & 19.86 & 0.05 & 0.27 & 0.06 \\
& \opdformerp & \best{8.09} & \best{0.27} & \best{0.21} & \best{0.12} \\
\bottomrule
\end{tabular}
}
\caption{Object pose error on the val set for all three datasets. Rotation error is in degrees and translation error is normalized by the diagonal length of the object. For accuracy, we use thresholds of $5^\circ$ for rotation and 0.1 (of object diagonal) for translation. Averages are computed part-wise. Accuracy counts matched pairs of GT and prediction in the same way as the mAP@50 metric, with higher confidence masks picking GT first and IoU = 50 threshold.}
\label{tab:results-pose-error-mini}
\end{table}





\mypara{How challenging is \ourdatamulti?}
\ourdatamulti is much more challenging than the single object \ourdatareal data.
As expected, the best performing model (\opdformerp) for \ourdatamulti makes use of the per-part object pose prediction.  
There is a significant difference in performance between \opdformerp and \opdformero  for \ourdatamulti, but less for single objects.
This is because in the single object scenario having one global pose is sufficient.


\mypara{Analysis by number of openable objects.}
To better understand performance on \ourdatamulti we evaluate on all images in \ourdatamulti grouping into images with zero, one, or multiple openable (articulated) objects (AO).
\Cref{tab:results-OPDMulti-split} shows that \opdrcnn-based methods are better at avoiding false predictions on images without any openable parts.
For images with one or more openable objects, \opdformerp makes the most accurate predictions (highest +\mtype{}\axis{}\orig).
We also see that multiple AO is more challenging with both the part detection (\partdet) and motion parameter predictions (+\mtype{}\axis{}\orig) being much lower than the single AO case.



\mypara{What part types are more challenging?}
We find that \lid is the most challenging to detect on \ourdatacad.
We suspect this is due to limited data and variability of the \lid shape.
See the supplement for a detailed analysis.

\begin{figure*}[htbp]
\centering
\setkeys{Gin}{width=\linewidth}
\begin{tabularx}{0.9\linewidth}{Y Y Y Y Y Y}
\toprule

\small{GT} &
\imgclip{0}{figure/qua_fail/gt/18-1200__0.png} & 
\imgclip{0}{figure/qua_fail/gt/106-480__0.png} &
\imgclip{0}{figure/qua_fail/gt/106-480__1.png} &
\imgclip{0}{figure/qua_fail/gt/106-480__2.png} &
\imgclip{0}{figure/qua_fail/gt/174-480__1.png}\\

\small{\opdrcnnop} &
Miss & 
Miss &
\imgclip{0}{figure/qua_fail/opdrcnnp/106-480__0.png} &
\imgclip{0}{figure/qua_fail/opdrcnnp/106-480__1.png} &
Miss \\
\small{\opdformerp} &
\imgclip{0}{figure/qua_fail/opdformerp/18-1200__1.png} & 
Miss &
\imgclip{0}{figure/qua_fail/opdformerp/106-480__0.png} &
\imgclip{0}{figure/qua_fail/opdformerp/106-480__2.png} &
\imgclip{0}{figure/qua_fail/opdformerp/174-480__2.png}\\
\bottomrule
\end{tabularx}
\vspace{-5pt}
\caption{
Example failure cases.
Failures are due to limited camera field-of-view leading to cropping (1st column), unclear part edges (2nd, 3rd columns), confusion between door and drawer parts (4th column), and confusion between walls and doors.
}
\label{fig:fail}
\end{figure*}

\mypara{How good are the predicted object poses?}
To check whether \opdformerbaseline provides improved object pose predictions, we evaluate the object pose directly by measuring the rotation error (angle between two rotation matrices) and translation error (Euclidean distance normalized by the object diagonal length).
Following prior work~\cite{li2021leveraging}, we report the median error and accuracy at different thresholds.
For rotation accuracy, we use thresholds of $5^\circ$ degree, and for translation accuracy, we use a threshold of $0.1$.
\Cref{tab:results-pose-error-mini} shows the results of above object pose evaluation metric on the val sets of \ourdatacad and \ourdatamulti. We can see that \opdformerbaseline variants all  outperform \opdrcnno, indicating that the transformer structure can give better pose predictions.  For \ourdatamulti, the rotation and translation error are much higher than for \ourdatacad, illustrating the challenge of our \ourdatamulti scenario. 
Furthermore, our \opdformerp has better object pose prediction than \opdformero, indicating the importance of having per-part object pose prediction.  
In the single setting (\ourdatacad), the part-weight-average global pose gives the best object pose prediction. 



\begin{table}
\centering
\resizebox{0.8\linewidth}{!}
{
\begin{tabular}{@{} ll rrrr @{}}
\toprule
& & \multicolumn{4}{c}{Part-averaged mAP $\% \uparrow$} \\
\cmidrule(l{0pt}r{2pt}){3-6}
Dataset & backbone & \partdet & +\mtype & +\mtype{}\axis & +\mtype{}\axis{}\orig \\
\midrule
\multirow{2}{*}{\ourdatacad}
& R50 & 79.0 & 76.7 & 58.6 & 53.4 \\
& Swin-L & \best{79.6} & \best{77.0} & \best{64.1} & \best{57.9} \\
\midrule
\multirow{2}{*}{\ourdatareal}
& R50 & 58.8 & 56.2 & 35.4 & 33.7\\
& Swin-L & \best{69.2} & \best{66.6} & \best{44.0} & \best{40.7} \\
\midrule
\multirow{2}{*}{\ourdatamulti}
& R50 & 32.9 & 31.6 & 19.4 & 16.0 \\
& Swin-L & \best{42.2} & \best{40.6} & \best{26.4} & \best{23.4} \\
\bottomrule
\end{tabular}
}
\vspace{-5pt}
\caption{Comparison of backbones with \opdformerp architecture on the val set for all three datasets.} 
\label{tab:results-backbone}
\end{table}
\mypara{Effect of backbone.}
Most of our experiments use the R50 backbone as it is smaller and requires fewer resources to train.
We check performance with Swin-L, a more powerful backbone compared to R50.
\Cref{tab:results-backbone} shows that with the Swin-L backbone, \opdformerp outperforms the R50 backbone in all cases.
Even when the part detection performance is roughly the same for OPDSynth dataset, the motion prediction is considerably higher (by 4.5\%).
Note that \opdformerp with Swin-L backbone (with 200 queries for the transformer decoder) has 205.6M parameters, which is around $5\times$ larger than the R50 backbone.


\mypara{Failure case analysis.}
\Cref{fig:fail} shows some failure cases.
Many errors occur due to the limited field-of-view and significant cropping of openable parts (see first column).
In the second and third column unclear edges lead to part detection failures.
In the fourth column motion type prediction fails due to a rotating door with drawer-like features.
The last column is an incorrect prediction of a wall as a door.


% \ignoreinshort{
\section{Related work}
\label{sec:related-work}
% \begin{gmchange}
% \GMTODO{Other AllSAT applications (from~\citeR{morgadoGoodLearningImplicit2005,todaImplementingEfficientAll2016,zhangAcceleratingAllSATComputation2020}): in data mining (e.g., frequent itemsets), network verification, image and preimage computation in unbounded Model Checking. AllSMT: predicate abstraction, in probabilistic reasoning (e.g., \#SMT, WMI).}
\paragraph{Applications of AllSAT and AllSMT.}
SAT and \smt{} enumeration has an important role in a variety of applications, ranging from artificial intelligence to formal verification.
AllSAT and AllSMT, mainly in their disjoint version, play a foundational role in several frameworks for \emph{probabilistic reasoning}, such as model counting in \smt{}~(\emph{\#\smt})~\cite{chistikovApproximateCountingSMT2017}
and \emph{Weighted Model Integration (WMI)}~\cite{belleProbabilisticInferenceHybrid2015,morettin-wmi-ijcar17,morettin-wmi-aij19,spallitta_smt-based_2022,spallittaEnhancingSMTbasedWeighted2024a}.
Specifically, \#\smtlarat{}~\cite{maVolumeComputationBoolean2009,zhouEstimatingVolumeSolution2015,geComputingEstimatingVolume2018} consists in summing up the volumes of the convex polytopes defined by each of the \larat{}-satisfiable truth assignments propositionally satisfying a \smtlarat{} formula, and has been employed for value estimation of probabilistic programs~\cite{chistikovApproximateCountingSMT2017} and for quantitative program analysis~\cite{liuProgramAnalysisQualitative2011}.\@ WMI can be seen as a generalization of \#\smtlarat{} that additionally considers a weight function $w$ that has to be integrated over each of such polytopes, and has been used to perform inference in hybrid probabilistic models such as Bayesian and Markov networks~\cite{belleProbabilisticInferenceHybrid2015} and Density Estimation Trees~\cite{spallitta_smt-based_2022}.
%Hence, in these cases, it is essential to enumerate disjoint partial truth assignments that are as small and as few as possible.
%
%
AllSAT has applications also in \emph{data mining}, where the problem of frequent itemsets can be encoded into a propositional formula whose satisfying assignments are the frequent itemsets~\cite{boudaneSATbasedApproachMining2016,dlalaComparativeStudySATBased2016}.
%Specifically, in \#SMT(\larat) consists in summing up the volumes of the convex polytopes defined by each of the assignments, whereas in WMI some function $w$ has to be integrated over each of such polytopes. %Hence, in these cases, it is essential to enumerate disjoint partial \ignoreinlong{models}\ignoreinshort{\GMCHANGE{truth assignments}} that are as small and as few as possible.
% In the context of knowledge compilation (cite), AllSAT can be used to compile a formula into deterministic Decomposable Normal Form (d-DNNF) (cite), that has found applications, e.g., in planning (cite). 
It has also been used in the context of \emph{software testing} to generate a suite of test inputs that should match a given output~\cite{khurshidCaseEfficientSolution2004}, and in \emph{circuit design}, to convert a formula from CNF to DNF~\cite{minatoFindingAllSimple1998,miltersenConvertingCNFDNF2005,bernasconiCompactDSOPPartial2013},
and for Static Timing Analysis to determine the inputs that can trigger a timing violation in a circuit~\cite{friedAllSATCombinationalCircuits2023}.
AllSAT and AllSMT have been applied also in \emph{network verification} for checking network reachability and for analysing the correctness and consistency of network connectivity policies~\cite{lopes2013network,jayaraman2014automated,lopesCheckingBeliefsDynamic2015}.
Moreover, they have been used for computing the image and preimage of a given set of states in \emph{unbounded model checking}~\cite{mcmillanApplyingSATMethods2002b,grumbergMemoryEfficientAllSolutions2004,liNovelSATAllsolutions2004}, and they are also at the core of algorithms for computing \emph{predicate abstraction}, a concept widely used in formal verification for automatically computing finite-state abstractions for systems with potentially infinite state space~\cite{lahiriSymbolicApproachPredicate2003,clarkePredicateAbstractionANSIC2004,hutchison_smt_2006}.

\paragraph{AllSAT.}
Most of the works on AllSAT have focused on the enumeration of satisfying assignments for CNF formulas (e.g.,~\citeR{morgadoGoodLearningImplicit2005,huangUsingDPLLEfficient2005,yu_all-sat_2014,todaBDDConstructionAll2015,todaImplementingEfficientAll2016,liangAllSATCCBoostingAllSAT2022,spallittaDisjointPartialEnumeration2024}), with several efforts in developing efficient and effective techniques for minimizing partial assignments (e.g.,~\citeR{raviMinimalAssignmentsBounded2004,morgadoGoodLearningImplicit2005,todaImplementingEfficientAll2016}).
The problem of minimizing truth assignments for Tseitin-encoded problems was addressed by~\citeA{hutchison_minimizing_2013}. They propose to
%first simplify the formula by considering its original structure and the current model; then they 
make iterative calls to a SAT solver imposing increasingly tighter cardinality constraints to obtain a minimal assignment. Whereas this approach can be used to find a single short truth assignment, it can be very expensive, and thus it is unsuitable for enumeration.

Other works have concentrated on the enumeration of satisfying assignments for combinatorial circuits, exploiting the structural information of the circuits to minimize the partial assignments over the input variables (e.g.,~\citeR{jinEfficientConflictAnalysis2005,jin2005prime,tibebu_augmenting_2018,friedAllSATCombinationalCircuits2023}).

A problem closely-related to AllSAT is that of finding all the prime implicants of a formula (e.g.,~\citeR{previtiPrimeCompilationNonClausal,jabbourEnumeratingPrimeImplicants2014a,luoEfficientTwophaseMethod2021}). AllSAT is a much simpler problem, and can be viewed as the problem of finding a not-necessarily-prime implicant cover for the formula.

\paragraph{Projected AllSAT.}
Projected enumeration, i.e., enumeration of satisfying assignments over a set of relevant atoms, has been studied mainly for CNF formulas, e.g., by~\citeA{grumbergMemoryEfficientAllSolutions2004,liNovelSATAllsolutions2004,hutchison_smt_2006,todaExploitingFunctionalDependencies2017}. The ambiguity of the notion of ``satisfiability by partial assignment'' for non-CNF and existentially-quantified formulas has been raised by~\citeA{sebastiani_are_2020,mohle_four_2020}, highlighting the difference between ``evaluation to true'', which is simpler to check and typically used by SAT solvers, and ``logical entailment'', %\GMCHANGEp{, 
which allows for producing shorter assignments. %}. 
The approach based on dual reasoning by \citeA{mohleDualizingProjectedModel2018,mohleEnumeratingShortProjected2021}, although able to detect logical entailment and thus to produce shorter partial assignments, is very inefficient even for small formulas.


\paragraph{AllSMT.}
The literature on AllSMT is very limited, and AllSMT algorithms are highly based on AllSAT techniques and tools. E.g., \mathsatfive{}~\cite{mathsat5_tacas13} implements an AllSMT functionality based on the procedure by~\citeA{hutchison_smt_2006}. A similar procedure has been described and implemented by~\citeA{phanAllSolutionSatisfiabilityModulo2015f}.

\paragraph{The role of CNF encodings.}
Although most AllSAT solvers assume the formulas to be in CNF, little or no work has been done to investigate the impact of the different CNF encodings on their effectiveness and efficiency. The role of the CNF-ization has been widely studied for SAT solving (e.g.,~\citeR{boy_de_la_tour_optimality_1992,jacksonClauseFormConversions2005,bjork_successful_2009,kuiter_tseitin_2022}) and in a recent work also for propositional model counting in the context of feature model analysis~\cite{kuiter_tseitin_2022}.

\subparagraph{Dual-rail encoding.}%
\label{sec:dualrail}
The idea of using two different variables to represent the positive and negative occurrences of a sub-formula shares some similarities with the so-called \emph{dual-rail} encoding~\cite{bryantCOSMOSCompiledSimulator1987,palopoliAlgorithmsSelectiveEnumeration1999}, which has been shown to be successful for prime implicant enumeration~\cite{%palopoliAlgorithmsSelectiveEnumeration1999,
    previtiPrimeCompilationNonClausal,luoEfficientTwophaseMethod2021}, and recently also for SAT enumeration for combinatorial circuits~\cite{friedAllSATCombinationalCircuits2023}.

Given a CNF formula, the dual-rail encoding maps each atom $A_i$ into a pair of dual-rail atoms $\pair{\poslab{A_i}}{\neglab{A_i}}$, so that $A_i$ and $\neg{}A_i$ are substituted with $\poslab{A_i}$ and $\neglab{A_i}$, respectively, and the clause $(\neg\poslab{A_i}\vee\neg\neglab{A_i})$ is conjoined with the formula. The resulting CNF formula is equisatisfiable to the original one, and every total assignment $\eta$ satisfying the dual-rail encoding corresponds to a partial assignment $\mu$ satisfying the original formula. Such $\mu$ can be obtained by assigning $\mu(A_i)$ to $\top$ or $\bot$ if the pair $\pair{\eta(\poslab{A_i})}{\eta(\neglab{A_i})}$ is $\pair{\top}{\bot}$ or $\pair{\bot}{\top}$, respectively, while leaving it unassigned if $\pair{\eta(\poslab{A_i})}{\eta(\neglab{A_i})}=\pair{\bot}{\bot}$. Short partial assignments can be obtained by maximizing the number of pairs assigned to $\pair{\bot}{\bot}$, which can be done either exactly by solving a MaxSAT problem or approximately by assigning negative value first in decision branches~\cite{friedAllSATCombinationalCircuits2023}.

% \begin{gmchangep}

Using the dual-rail encoding for enumeration, however, requires ad-hoc enumeration algorithms, since the solver must take into account the three-valued semantics of truth assignments over dual-rail atoms when enumerating the assignments~\cite{friedAllSATCombinationalCircuits2023}.
Our contribution, instead, focuses on CNF-ization approaches, which can be used in combination with any enumeration algorithm matching the properties described in~\sref{sec:background:allsat}.
%
Also, comparing the $\NNFna{}+\PlaistedCNF{}$ encoding with the dual-rail encoding, we observe that the former duplicates only the label atoms in $\allB{}$, so that the number of introduced atoms is smaller than in the dual-rail encoding.
Moreover, because of this, the clauses in the form $(\neg\poslab{B_i}\vee\neg\neglab{B_i})$ are not needed for correctness but only for efficiency, and can in principle be omitted. In the dual-rail encoding, instead, their presence is essential to prevent the illegal assignment $\eta(\poslab{B_i})=\eta(\neglab{B_i})=\top$.

% \end{gmchangep}


% Also, differently from the approach by~\citeA{friedAllSATCombinationalCircuits2023}, our approach does not require implementing ad-hoc enumeration algorithms, and any enumeration strategy matching conditions~\ref{item:mumodelsvi}-\ref{item:muAminimal} in \sref{sec:background:allsat} can benefit from it.
%
% A technique that has gained some success in SAT enumeration is the so called dual-rail encoding~\cite{bryantCOSMOSCompiledSimulator1987%,palopoliAlgorithmsSelectiveEnumeration1999
% }. E.g.,~\citeA{friedAllSATCombinationalCircuits2023} exploit it to enumerate short partial assignments for circuits, and also~\citeA{%palopoliAlgorithmsSelectiveEnumeration1999,
%     previtiPrimeCompilationNonClausal,luoEfficientTwophaseMethod2021} use it in the context of prime implicants enumeration. %We highlight the main similarities and differences of this encoding with the encoding proposed in this paper in~\sref{sec:dualrail}.


% Other works on AllSAT:
% \begin{itemize}
%     \item AllSAT-CNF
%     \begin{itemize}
%         \item \cite{morgadoGoodLearningImplicit2005} survey, minimization techniques for AllSAT-CNF.
%         \item \cite{todaImplementingEfficientAll2016} survey (BC, NBC)
%         \item \cite{raviMinimalAssignmentsBounded2004} minimization for CNF
%         \item \cite{yu_all-sat_2014}. Non-disjoint all-sat (CNF), add blocking clauses involving only decision variables.
%         \item \cite{liangAllSATCCBoostingAllSAT2022} Exploit component analysis in enumeration (CNF). No projected. 
%     \end{itemize}
%     \item AllSAT-Circuits: the general idea is to first convert the formula into CNF and once a satisfying assignment is found, exploit the circuit structural information to find a minimal partial assignment.
%     \begin{itemize}
%         \item \cite{jinEfficientConflictAnalysis2005}. Enumerate models of AIG.
%         \item \cite{jin2005prime}. Enumeration of minimal model for circuits.
%         \item \cite{tibebu_augmenting_2018}. Enumeration of minimal models for AIG.
%         \item Paper Nadel. Enumeration of minimal models for circuits.
%     \end{itemize}
% \end{itemize}
% \begin{itemize}
%     \item \cite{huangUsingDPLLEfficient2005,todaBDDConstructionAll2015,todaImplementingEfficientAll2016} compile a propositional formula into a BDD and then enumerate all the satisfying assignments by traversing the BDD.\@ OBDDs can be used also to cache intermediate results.
%     \item \cite{todaExploitingFunctionalDependencies2017} Extend result above to projected enumeration. However, the satisfiability for the existentially quantified formula is not based on entailment (?).
% \end{itemize}

% Enumeration of all the prime implicants:
% \begin{itemize}
%     \item \cite{previtiPrimeCompilationNonClausal} find all the prime implicants and a prime implicates cover for a non-CNF formula. Exploit dual-rail encoding. (tool Primer, not available) 
%     \item \cite{luoEfficientTwophaseMethod2021} first compile the formula into an \textbf{equivalent} CNF, by finding a cover of prime or small implicates, and then find all the prime implicants of the CNF, also in this case exploiting dual-rail encoding. (tool CoAPI, available)
% \end{itemize}

% Dual rail encoding

% The impact of using different CNF encodings on the performance for SAT and SMT solving has been widely studied in the literature~\cite{boy_de_la_tour_optimality_1992,jackson_optimality_2004,bjork_successful_2009,kuiter_tseitin_2022}.


% \end{gmchange}

% }

\section{Conclusions and future work}
%
\label{sec:conclusions}
We have presented a theoretical and empirical analysis of the impact of different CNF-ization approaches on SAT %\ignoreinshort{\GMCHANGE{
and SMT %}}
enumeration, %\ignoreinshort{\GMCHANGE{, 
both disjoint and non-disjoint. %}}. 
We have shown how the most popular transformations conceived for SAT %\ignoreinshort{\GMCHANGEp{
and SMT %}} 
solving, namely the Tseitin and the Plaisted and Greenbaum CNF-izations, prevent the solver from producing short partial assignments, thus seriously affecting the effectiveness of the enumeration. To overcome this limitation, we have proposed to preprocess the formula by converting it into NNF before applying the Plaisted and Greenbaum transformation. We have shown, both theoretically and empirically, that the latter approach can fully overcome the problem and can drastically reduce both the number of partial assignments and the execution time.

% we plan to further investigate the
% impact of CNF conversion also on disjoint SMT enumeration. We expect that in
% this domain the impact can be even more relevant, since in SMT
% multiple instances of the same theory atoms are typically rarer than for atoms in
% the Boolean case. Also, disjoint SMT enumeration has a fundamental role in Weighted Model Integration~\cite{morettin-wmi-ijcar17,morettin-wmi-aij19,spallittaSMTbasedWeightedModel2022}, an important framework for probabilistic inference in hybrid domains. Hence, we believe that our contribution can have a great impact on this application, where non-CNF formulas occur frequently.
% Finally, we think that work should be done to understand the impact on enumeration with repetitions, i.e.\ where models may not be disjoint, for instance in Predicate Abstraction~\cite{lahiriSMTTechniquesFast2006}. %Moreover, there is an alternative definition of partial assignment satisfiability that is based on the notion of \emph{entailment}~\cite{sebastianiAreYouSatisfied2020,mohleFourFlavorsEntailment2020}. Understanding the impact of the CNF conversion on solvers that use this definition is an interesting direction.

This work opens an interesting research avenue: investigate the role
of CNF-ization in neighbor fields as d-DNNF compilation and model
counting, possibly adapting d-DNNF compilers and model counters
so that to exploit different forms of CNF-izations.


\FloatBarrier

%%
%% The acknowledgments section is defined using the "acks" environment
%% (and NOT an unnumbered section). This ensures the proper
%% identification of the section in the article metadata, and the
%% consistent spelling of the heading.
\begin{acks}
GS and RS were partially supported by the project ``AI@TN'' funded by the Autonomous Province of Trento. 
%
RS was partially supported by the MUR PNRR project FAIR - Future AI
Research (PE00000013) funded by the NextGenerationEU.
%
RS was partially funded under the NRRP, Mission
4 Component 2 Investment 1.4, by the European Union – NextGenerationEU (proj. nr. CN 00000013).
%
RS was supported in part by the TANGO project funded by the EU Horizon Europe research and innovation program
under GA No 101120763, funded by the European Union.
%
Views and opinions expressed are however those of
the author(s) only and do not necessarily reflect those of the European Union, the European Health and Digital
Executive Agency (HaDEA) or The European Research Council. Neither the European Union nor the granting
authority can be held responsible for them.
\end{acks}

%%
%% The next two lines define the bibliography style to be used, and
%% the bibliography file.
\bibliographystyle{ACM-Reference-Format}
\bibliography{jair-allsat-cnfs}

%%
%% If your work has an appendix, this is the place to put it.
\newpage
\appendix

\section{Appendix for Proofs}

\paragraph{Proof of Theorem \ref{thm:main}.}

\begin{proof}
\label{proof:main}
Our proof has two steps. In Step 1, we will show that SimCLR is equivalent to minimizing the cross entropy loss defined in Eqn.~(\ref{eqn:cross-entropy}). 
In Step 2, we will show  that minimizing the cross-entropy loss 
is equivalent to spectral clustering on $\bfpi$. 
Combining the two steps together, we have proved our theorem. 

\textbf{Step 1: } SimCLR is equivalent to minimizing the cross entropy loss.

The cross-entropy loss takes expectation over 
$\bfW_\bfX\sim \mathbb{P}(\cdot ; \bfpi)$, 
which means $\bfW_\bfX$ has exactly one non-zero entry in each row $i$. By Lemma~\ref{lem:multinomial}, we know every row $i$ of $\bfW_\bfX$ is independent of other rows. Moreover, 
$\bfW_{\bfX,i}\sim \mathcal{M}(1, \bfpi_i/\sum_j \bfpi_{i,j})=\mathcal{M}(1, \bfpi_i)$, because $\bfpi_i$ itself is a probability distribution.
Similarly, we know $\bfW_\bfZ$ also has the row-independent property by sampling over $\mathbb{P}(\cdot;\bfK_\bfZ)$.
Therefore, by Lemma~\ref{lem:cross_split}, we know Eqn.~(\ref{eqn:cross-entropy}) is equivalent to:
\[
 -\sum_{i=1}^n \mathbb{E}_{\bfW_{\bfX,i}}[\log \mathbb{P}(\bfW_{\bfZ,i}=\bfW_{\bfX,i};\bfK_\bfZ)],
\]

This expression takes expectation over $\bfW_{\bfX,i}$ for the given row $i$. Notice that 
$\bfW_{\bfX,i}$ has exactly one non-zero entry, which equals $1$ (same for $\bfW_{\bfZ,i}$). 
As a result
we expand the above expression to be:
\begin{equation}
 -\sum_{i=1}^n \sum_{j\neq i} \Pr(\bfW_{\bfX,i,j}=1)\log \Pr(\bfW_{\bfZ,i,j}=1).
\label{eqn:detailed-expansion}    
\end{equation}


By Lemma~\ref{lem:multinomial}, $\Pr(\bfW_{\bfZ,i,j}=1)=\bfK_{\bfZ,i,j}/\|\bfK_{\bfZ,i}\|_1$ for $j\neq i$. Recall that $\bfK_\bfZ=(k(\bfZ_i-\bfZ_j))_{(i,j)\in[n]^2}$, which means 
$\bfK_{\bfZ,i,j}/\|\bfK_{\bfZ,i}\|_1=\frac{\exp(-\|\bfZ_i-\bfZ_j\|^2/{2\tau})}{\sum_{k\neq i}
\exp(-\|\bfZ_i-\bfZ_k\|^2/{2\tau})
}$ for $j\neq i$, when $k$ is the Gaussian kernel with variance $\tau$. 

Notice that $\bfZ_i=f(\bfX_i)$, so we know
\begin{equation}
-\log \Pr(\bfW_{\bfZ,i,j}=1)=
-\log \frac{\exp(-\|f(\bfX_i)-f(\bfX_j)\|^2/{2\tau})}{\sum_{k\neq i}
\exp(-\|f(\bfX_i)-f(\bfX_k)\|^2/{2\tau}),
}
\label{eqn:infonce-equivalence}    
\end{equation}


The right hand side is exactly the InfoNCE loss defined in Eqn.~(\ref{eqn:infonce}).
Inserting Eqn.~(\ref{eqn:infonce-equivalence}) into Eqn.~(\ref{eqn:detailed-expansion}), we get the SimCLR algorithm, which first samples augmentation pairs $(i,j)$ with $\Pr(\bfW_{\bfX,i,j}=1)$ for each row $i$, and then optimize the InfoNCE loss. 

\textbf{Step 2: } minimizing the cross entropy loss 
is equivalent to spectral clustering on $\bfpi$.


By Lemma~\ref{lem:convert_to_spectral}, we may further convert the loss to 
\begin{equation}
\label{eqn:main-theorem-repul-attr}
\min_{\bfZ}
-\sum_{(i,j)\in [n]^2} \mathbf{P}_{i,j}
\log k (\bfZ_i-\bfZ_j)+\log \mathbf{R}(\bfZ).
\end{equation}
Since $k$ is the Gaussian kernel, this reduces to \[
\min_\bfZ \mathrm{tr}(\bfZ^\top \mathbf{L}(\bfpi) \bfZ)
+\log \mathbf{R}(\bfZ),
\]

where we use the fact that $\mathbb{E}_{\bfW_\bfX\sim \mathbb{P}(\cdot; \bfpi)}[\mathbf{L}(\bfW_\bfX)]
=\mathbf{L}(\bfpi)
$, because the Laplacian operator is linear and $
\mathbb{E}_{\bfW_\bfX\sim \mathbb{P}(\cdot; \bfpi)}(\bfW_\bfX)=\bfpi
$.
\end{proof}

\paragraph{Proof of Theorem \ref{thm:clip}.}
\begin{proof}
Since $\bfW_\bfX\sim \mathbb{P}(\cdot;\bfpi_{\mathbf{A}, \mathbf{B}})$, we know 
$\bfW_\bfX$ has exactly one non-zero entry in each row, denoting the pair that got sampled. 
A notable difference compared to the previous proof is we now have $n_\mathcal{A}+n_\mathcal{B}$ objects in our graph. CLIP deals with this by taking a mini-batch of size $2N$, 
such that $n_\mathcal{A}=n_\mathcal{B}=N$, and adding the $2N$ InfoNCE losses together. We label the objects in $\mathcal{A}$ as $[n_\mathcal{A}]$, and the objects in $\mathcal{B}$ as $\{n_\mathcal{A}+1, \cdots, n_\mathcal{A}+n_\mathcal{B}\}$. 

Notice that $\bfpi_{\mathbf{A}, \mathbf{B}}$ is a bipartite graph, so the edges of objects in $\mathcal{A}$ will only connect to object in $\mathcal{B}$ and vice versa. We can define the similarity matrix in $\cZ$ as $\bfK_\bfZ$, 
where $\bfK_\bfZ(i, j+n_\mathcal{A})=\bfK_\bfZ(j+n_\mathcal{A},i)= k(\bfZ_i-\bfZ_j)$ for $i\in [n_\mathcal{A}], j\in [n_\mathcal{B}]$, and otherwise we set $\bfK_\bfZ(i,j)=0$. 
The rest is same as the previous proof. 
\end{proof}

\paragraph{Proof of Theorem \ref{thm:exponential}.}

\begin{proof}
\label{proof:exponential}
Since the objective function consists of a linear term combined with an entropy regularization, which is a strongly concave function, the maximization problem is a convex optimization problem. Owing to the implicit constraints provided by the entropy function, the problem is equivalent to having only the equality constraint. We then introduce the Lagrangian multiplier $\lambda$ and obtain the following relaxed problem:

$$
\widetilde{E}(\boldsymbol{\alpha})=\psi_{1}-\sum_{i=1}^n \alpha_{i} \psi_{i}+\tau \sum_{i=1}^n \alpha_{i}\log \alpha_{i}+\lambda\left(\boldsymbol{\alpha}^{\top} \mathbf{1}_n-1\right).
$$

As the relaxed problem is unconstrained, taking the derivative with respect to $\alpha_{i}$ yields

$$
\frac{\partial \widetilde{E}(\boldsymbol{\alpha})}{\partial \alpha_{i}}=-\psi_{i}+\tau\left(\log \alpha_{i}+\alpha_{i} \frac{1}{\alpha_{i}}\right)+\lambda=0.
$$

Solving the above equation implies that $\alpha_{i}$ takes the form
$
\alpha_{i}=\exp \left(\frac{1}{\tau} \psi_{i}\right) \exp \left(\frac{-\lambda}{\tau}-1\right).
$ Since $\alpha_{i}$ lies on the probability simplex, the optimal $\alpha_{i}$ is explicitly given by
$
\alpha^{*}_{i}=\frac{\exp \left(\frac{1}{\tau} \psi_{i}\right)}{\sum_{i^{\prime}=1}^n \exp \left(\frac{1}{\tau} \psi_{i^{\prime}}\right)} .
$ Substituting the optimal point into the objective function, we obtain
$$
\begin{aligned}
E\left(\boldsymbol{\alpha}^*\right)  &=\psi_1-\sum_{i=1}^n \frac{\exp \left(\frac{1}{\tau} \psi_{i}\right)}{\sum_{i^{\prime}=1}^n \exp \left(\frac{1}{\tau} \psi_{i^{\prime}}\right)} \psi_{i}+\tau \sum_{i=1}^n \frac{\exp \left(\frac{1}{\tau} \psi_{i}\right)}{\sum_{i^{\prime}=1}^n \exp \left(\frac{1}{\tau} \psi_{i^{\prime}}\right)}\log \frac{\exp \left(\frac{1}{\tau} \psi_{i}\right)}{\sum_{i^{\prime}=1}^n \exp \left(\frac{1}{\tau} \psi_{i^{\prime}}\right)} \\
& =\psi_1 - \tau \log \left(\sum_{i=1}^n \exp \left(\frac{1}{\tau} \psi_{i}\right)\right).
\end{aligned}
$$
Thus, the Lagrangian dual function is given by
\begin{equation*}
-E\left(\boldsymbol{\alpha}^*\right)= -\tau \log \frac{\exp \left(\frac{1}{\tau} \psi_{1}\right)}{\sum_{i=1}^n \exp \left(\frac{1}{\tau} \psi_{i}\right)}.\qedhere
\end{equation*}
\end{proof}



\section{More on Experiments} \label{section: experiment_details}

\paragraph{CIFAR-10 and CIFAR-100} CIFAR-10 ~\citep{krizhevsky2009learning} and CIFAR-100 ~\citep{krizhevsky2009learning} are well-known classic image classification datasets. Both CIFAR-10 and CIFAR-100 contain a total of 60k $32 \times 32$ labeled images of different classes, with 50k for training and 10k for testing. CIFAR-10 is similar to CIFAR-100, except there are 10 different classes in CIFAR-10 and 100 classes in CIFAR-100.

\paragraph{TinyImageNet} TinyImageNet ~\citep{le2015tiny} is a subset of ImageNet ~\citep{deng2009imagenet}. There are 200 different object classes in TinyImageNet, with 500 training images, 50 validation images, and 50 test images for each class. All the images in TinyImageNet are colored and labeled with a size of $64 \times 64$.

\textbf{Pseudo-code.} Algorithm \ref{alg:Training Procedure} presents the pseudo-code for our empirical training procedure.

\begin{algorithm}[!htbp]
\caption{Training Procedure}
\label{alg:Training Procedure}
\begin{algorithmic}[1]
\REQUIRE trainable encoder network $f$, batch size $N$, augmentation strategy \textit{aug}, loss function $L$ with hyperparameters \textit{args}
\FOR {sampled minibatch ${x_i}_{i=1}^N$}
\FORALL{$i \in { 1, ..., N }$}
\STATE draw two augmentations $t_i = \textit{aug}\left(x_i\right) $, $t_i' = \textit{aug}\left(x_i\right) $
\STATE $z_i = f\left(t_i\right)$, $z_i' = f\left(t_i'\right)$
\ENDFOR
\STATE compute loss $\mathcal{L} = L(N, z, z', \textit{args})$
\STATE update encoder network $f$ to minimize $\mathcal{L}$
\ENDFOR
\STATE \textbf{Return} encoder network $f$
\end{algorithmic}
\end{algorithm}

We also provide the pseudo-code for our core loss function used in the training procedure in Algorithm \ref{alg:Core loss}. The pseudo-code is almost identical to SimCLR's loss function, with the exception of an extra parameter $\gamma$.

\begin{algorithm}[!htbp]
\caption{Core loss function $\mathcal{C}$}
\label{alg:Core loss}
\begin{algorithmic}[1]
\REQUIRE batch size $N$, two encoded minibatches $z_1, z_2$, $\gamma$, temperature $\tau$
\STATE $z = \textit{concat}\left(z_1, z_2\right)$
\FOR {$i \in {1, ..., 2N }, j \in {1, ..., 2N}$ }
\STATE $s_{i,j} = \Vert z_i - z_j \Vert_2^{\gamma}$
\ENDFOR
\STATE \textbf{define} $l(i, j)$ \textbf{as} $l(i, j) = - \log \frac{exp\left(s_{i,j}/\tau \right)}{\sum_{k=1}^{2N} \mathbf{1}{[k \ne i]} exp\left(s{i, j} / \tau \right)} $
\STATE \textbf{Return} $\frac{1}{2N} \sum_{k=1}^N\left[l(i, i+N) + l(i+N, i)\right]$
\end{algorithmic}
\end{algorithm}

Utilizing the core loss function $\mathcal{C}$, we can define all kernel loss functions used in our experiments in Table \ref{table: loss definition}. For all $z_i \in z$ with even dimensions $n$, we define $z_{L_i} = z_i\left[0:n/2\right]$ and $z_{R_i} = z_i\left[n/2:n\right]$.

\begin{table}[ht]
\centering
\begin{tabular}{{@{}l|l@{}}}
Kernel  &  Loss function \\ \midrule
Laplacian & $\mathcal{C}\left(N, z, z', \gamma=1, \tau\right)$\\ \midrule
Sum       & $\lambda * \mathcal{C}\left(N, z, z', \gamma=1, \tau_1\right) + (1-\lambda) * \mathcal{C}\left(N, z, z', \gamma=2, \tau_2\right)$  \\ \midrule
Concatenation Sum&$\lambda * \mathcal{C}\left(N, z_L, z'_L, \gamma=1, \tau_1\right) + (1-\lambda) * \mathcal{C}\left(N, z_R, z'_R, \gamma=2, \tau_2\right)$\\ \midrule
$\gamma = 0.5$ & $\mathcal{C}\left(N, z, z', \gamma=0.5, \tau\right)$          \\ 

\end{tabular}

\caption{Definition of kernel loss functions in our experiments}
\label {table: loss definition}
\end{table}

\textbf{Baselines.} We reproduce the SimCLR algorithm using PyTorch Lightning~\citep{PytorchLightning}.

\textbf{Encoder details.}
The encoder $f$ consists of a backbone network and a projection network. We employ ResNet50~\citep{ResNet} as the backbone and a 2-layer MLP (connected by a batch normalization~\citep{ioffe2015batch} layer and a ReLU \cite{nair2010rectified} layer) with hidden dimensions 2048 and output dimensions 128 (or 256 in the concatenation kernel case).

\textbf{Encoder hyperparameter tuning.}
For each encoder training case, we randomly sample 500 hyperparameter groups (sample details are shown in Table \ref{table: Hyperparameter sample}) and train these samples simultaneously using Ray Tune ~\citep{RayTune}, with the ASHA scheduler~\citep{li2018massively}. Ultimately, the hyperparameter group that maximizes the online validation accuracy (integrated in PyTorch Lightning) within 5000 validation steps is chosen for the given encoder training case.

\begin{table}[ht]
\centering

\begin{tabular}{@{}l|l|l@{}}
\midrule
Hyperparameter  & Sample Range & Sample Strategy \\ \midrule
start learning rate & $\left[10^{-2}, 10\right]$ & log uniform \\ \midrule
$\lambda$       & $\left[0, 1\right]$ & uniform \\ \midrule
$\tau$, $\tau_1$, $\tau_2$ & $\left[0, 1\right]$ & log uniform \\ \midrule
\end{tabular}

\caption{Hyperparameters sample strategy}
\label {table: Hyperparameter sample}
\end{table}

\textbf{Encoder training.} 
We train each encoder using the LARS optimizer~\citep{LARSOptimizer}, LambdaLR Scheduler in PyTorch, momentum 0.9, weight decay $10^{-6}$, batch size 256, and the aforementioned hyperparameters for 400 epochs on a single A-100 GPU.

\textbf{Image transformation.} The image transformation strategy, including augmentation, is identical to the default transformation strategy provided by PyTorch Lightning.

\textbf{Linear evaluation.}
The linear head is trained using the SGD optimizer with a cosine learning rate scheduler, batch size 64, and weight decay $10^{-6}$ for 100 epochs. The learning rate starts at $0.3$ and ends at $0$.

\textbf{Moco Experiments.} We also tested our method based on MoCo~\citep{he2019moco}. The results are summarized in Table \ref{tab:results-moco}. Here we choose ResNet18~\citep{ResNet} as the backbone and set a temperature of $0.1$ as default. For our simple sum kernel, we set $\lambda=0.8$. The results show that our method outperforms the original MoCo method.

\begin{table}[thb]
\centering
\caption{MoCo Experiment Results on CIFAR-10 and CIFAR-100.}
\label{tab:results-moco}
\resizebox{\textwidth}{!}{%
\begin{tabular}{@{}c|ccc|ccc@{}}
\toprule
\multirow{3}{*}{Method} & \multicolumn{3}{c|}{CIFAR-10} & \multicolumn{3}{c}{CIFAR-100} \\ \cmidrule(lr){2-4} \cmidrule(lr){5-7} 
                        & 200 epochs & 400 epochs    & 1000 epochs   & 200 epochs & 400 epochs & 1000 epochs         \\ \midrule
MoCo (repro.)         & $76.41 \pm 0.12$    & $80.01 \pm 0.15$          & $84.45 \pm 0.08$    & $\mathbf{47.02 \pm 0.11}$ & $52.50 \pm 0.07$ & $57.62 \pm 0.15$            \\
\midrule
Laplacian Kernel        & ${78.09 \pm 0.10}$    & $\mathbf{83.85 \pm 0.09}$          & $\mathbf{88.34 \pm 0.16}$    & $46.12 \pm 0.22$   & $53.44 \pm 0.17$ & $59.10 \pm 0.14$        \\
Simple Sum Kernel & $\mathbf{78.12 \pm 0.15}$   & $83.23 \pm 0.18$ & $87.50 \pm 0.20$ & $46.65 \pm 0.06$ & $\mathbf{53.62 \pm 0.19}$ & $\mathbf{59.83 \pm 0.12}$\\
\bottomrule
\end{tabular}
}
\end{table}



\section{More Experiments on Synthetic Data}


Consider a scenario with $n$ clusters, each containing $k$ vertices. Let the probability of vertices $u$ and $v$ from the same cluster belonging to $\bfpi$ be $p$. Conversely, for vertices $u$ and $v$ from different clusters, let the probability of belonging to $\pi$ be $q$. We generate the graph $\bfpi$ randomly, based on $p$ and $q$. We experiment with values of $k=100$ and $n=6$ for ease of visualization, embedding all points in a two-dimensional space. Each vertex's initial position originates from a normal distribution. In each iteration, we sample a subgraph of $\bfpi$ uniformly, ensuring each vertex has an out-degree of $1$. We then optimize the corresponding vectors using InfoNCE loss with an SGD optimizer and iterate until convergence. Our experimental setup consists of an SGD learning rate of $1$, an InfoNCE loss temperature of $0.5$, and a batch size of $50$. We evaluate two scenarios with different $p$ and $q$ values: $p=1$, $q=0$, and $p=0.75$, $q=0.2$. The results of these experiments are visualized in Figure \ref{fig:vis-spectral-cluster}. The obtained embeddings exhibit the hallmark pattern of spectral clustering of graph $\bfpi$.

\begin{figure}[!tb]
\centering
\subfigure{
\includegraphics[width=1\textwidth]{Figures/cluster_pi.png}
\label{fig:vis-cluster}
}
\subfigure{
\includegraphics[width=1\textwidth]{Figures/noised_cluster_pi.png}
\label{fig:vis-noised-cluster}
}
\caption{Visualizations of the optimization process using InfoNCE Loss on the vectors corresponding to $\bfpi$. Points of identical color belong to the same cluster within $\bfpi$. To showcase the internal structure of $\bfpi$, we randomly select 10 vertices from each cluster to display the edge distribution of $\bfpi$.}
\label{fig:vis-spectral-cluster}
\end{figure}



\end{document}
\endinput
%%
%% End of file `sample-acmlarge.tex'.
