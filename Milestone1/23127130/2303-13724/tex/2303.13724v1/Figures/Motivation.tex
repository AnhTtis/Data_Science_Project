 \begin{figure*}[t]
  \centering
    \includegraphics[width=1\linewidth]{Pics/Motivation.pdf}
    \caption{Illustration of the difference between our method and prior methods. In the FSSeg task, the model predicts only novel classes provided by support samples in the form of binary masks for query images. However, this approach requires prior knowledge of the support samples, which can be a challenging and time-consuming task. Additionally, FSSeg only evaluates novel classes and ignores the segmentation of base classes in the test samples. 
    On the other hand, GFSSeg models predict both base and novel classes without requiring prior knowledge of the support samples. However, this approach faces its own challenges due to the imbalance of training samples between base and novel classes, leading to a bias towards base classes with abundant samples. As shown in the figures, even the previous method CAPL~\cite{capl} is able to locate the objects, but it incorrectly labels the objects as a truck instead of a car.
    To address these challenges, we propose using a class contrastive loss and class relationship loss to encourage a large distance between the features from different classes. With our method, we are able to identify the correct class label, improving upon prior approaches.}
    
    \label{Figure: Motivation}
\end{figure*}