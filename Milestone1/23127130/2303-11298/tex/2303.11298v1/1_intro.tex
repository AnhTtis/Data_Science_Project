\section{Introduction}
\label{sec:intro}

Humans tend to overestimate their abilities, a cognitive bias known as Dunning-Kruger
effect~\cite{Kruger1999UnskilledAndUnaware}. Unfortunately, so do deep neural
networks. Despite impressive performance on a wide range of tasks, deep learning models tend to be overconfident---that is, they 
predict with high-confidence
even when they are wrong~\cite{GuoICML17OnCalibrationOfModernNNs}. This effect is even more severe under domain shifts, where models tend to underperform in general~\cite{ovadia2019can, RechtICLR19DuImageNetClassifiersGeneralizeToImageNet, HendrycksICLR19BenchmarkingNNRobustnessCommonCorruptionsPerturbations}.

\begin{figure}
\centering
\includegraphics[width=\linewidth]{Figures/splash_v2_compressed.pdf}

\caption{\textbf{Top: mIoU and ECE \vs domain shift.} Errors are normalized with respect to the lowest error on the training distribution (Cityscapes). We compare recent segmentation models, both transformer-based (SETR \cite{ZhengCVPR21RethinkingSiSSeq2SeqPerspTransformers}, SegFormer \cite{XieNIPS21SegFormerSemSegmTransformers} and Segmenter \cite{StrudelICCV21SegmenterTransformerForSemSegm}) and convolution-based (ConvNext \cite{LiuCVPR22AConvNet4The2020s}) with ResNet baselines (UPerNet\cite{XiaoECCV18UnifiedPerceptualParsingSceneUnderstanding} and DLV3+\cite{ChenX17RethinkingAtrousConvolutionSemSegm}). All recent models (both transformers and CNNs) are remarkably more robust than ResNet baselines (whose lines in mIoU overlap), however, ECE increases sharply for all methods. \textbf{Bottom:} Sample images for each dataset.
}
\vspace{-5pt}
\label{figure:splash}
\end{figure}

%Comments from when it was in Introduction
% Greg --  I would detail which methods are transformer / Conv / ResNet (with citations) so it's stand alone for non experts too
% Greg -- couldn't we make the samples a bit bigger? putting the dataset name below for instance
While these vulnerabilities affect deep models in general, they are often studied for classification models and are comparably less explored for semantic segmentation, a fundamental task in computer vision that is key to many
% semantic image image segmentation but when
% it comes to specific tasks, they have been mostly studied for simple case studies such as classification.
%
% In this work, we put segmentation image segmentation models under the lens, studying them both in terms of \textit{robustness} and \textit{reliability} under domain shifts. 
% Semantic image segmentation is a fundamental task in computer vision,
% leveraged in 
critical applications such as autonomous driving and
AI-assisted medical imaging. In those applications, domain shifts are more the rule than the exception (\eg, changes in weather for
a self-driving car or differences across patients for a medical imaging system).
Therefore, brittle performance and overconfidence under domain
shifts are two important and challenging problems 
% \rict{that undermine the \textit{reliability}
% ~\cite{TranX22PlexReliability} 
% of machine learning systems and we need} to address \rict{them} 
to address
for a
safe deployment of 
% such
artificial intelligence
systems in the real world.

With that in mind, we argue that a \textit{reliable} model should 
% \textit{1)} 
\textit{i)}
be robust to domain shifts and 
% \textit{2)} 
\textit{ii)}
provide good uncertainty estimates. The core goal of this study is providing an answer to the following, crucial question:
\textbf{are state-of-the-art semantic segmentation models improving in terms of robustness \textit{and} uncertainty estimation?}

To shed light on this, we evaluate a large body of segmentation models, assessing their in-domain (\id) \vs out-of-domain (\ood) prediction quality (\textbf{robustness}) together with their calibration, misclassification detection and \ood detection (\textbf{uncertainty estimation}).

We argue that a study of this kind is crucial to understand whether research on semantic segmentation is moving in the right direction.
%
Following the rise of transformer architectures in computer vision~\cite{DosovitskiyICLR21AnImageIsWorth16x16WordsTransformersAtScale, touvron2021training, carion2020end,LiuICCV21SwinTransformerHierarchicalViTShiftedWindows},
several studies have compared recent self-attention and CNN-based \textit{classification} models in terms of robustness \cite{BhojanapalliICCV21UnderstandingRobustnessTransformersImageClass, naseer2021intriguing, bai2021transformers, paul2022vision, mao2022towards, LiuCVPR22AConvNet4The2020s} and predictive uncertainty \cite{minderer2021revisiting, pinto2022impartial}. Yet, when it comes to \textit{semantic segmentation}, prior studies \cite{XieNIPS21SegFormerSemSegmTransformers, zhou2022understanding} only focused on robustness, using synthetic corruptions as domain shifts (\eg, blur, noise) \cite{KamannCVPR20BenchmarkingRobustnessSemSegmModels}. In contrast, we 
% study 
consider
natural, realistic domain shifts and study segmentation models both in terms of robustness and uncertainty, leveraging datasets captured in different conditions---see \cref{figure:splash} (bottom). 

% Although improvements in classification and segmentation often go hand in hand, we argue a thorough study focusing on segmentation models is warranted since different trends may arise. 
Task-specific studies are important, since task-specific architectures and learning algorithms may carry different behaviors and some observations made for classification might not hold true when switching to segmentation.
For instance, contrary to Minderer~\etal \cite{minderer2021revisiting}, we observe that improvements in calibration are far behind those in robustness, see \cref{figure:splash} (top). Furthermore, previous analyses only
consider simple 
% temperature scaling 
calibration approaches~\cite{GuoICML17OnCalibrationOfModernNNs}
% to improve 
while assessing
model reliability;
% meanwhile, 
in contrast,
we make a step forward and
explore content-dependent calibration
strategies~\cite{gong2021confidence, ding2021local}, which show promise to improve reliability out of domain.

% This allows us individuating important research directions to improve
% reliability of segmentation models when deployed out of domain.
% By clarifying 
Our analysis allows us individuating
in which directions 
% are we 
we are
improving and in which 
% do we lag behind.
we are lagging behind.
% our study can
% To the best of our knowledge, this
This is the first work to systematically study robustness
and uncertainty under domain shift for a large suite of segmentation models and
we believe it
% this will
can
help 
% both 
practitioners and researchers 
% building on the state of the art of 
working on
semantic segmentation. 
% Our main observations are:
We summarize our main observations in the following.


\myparintro{i) Remarkable improvements in robustness, but poor in calibration} Under domain shifts, recent segmentation models perform significantly better (in terms of mIoU)---with larger improvements for stronger shifts. Yet, \ood calibration error increases dramatically for all models.

\noindent
\textit{\textbf{ii) Content-dependent calibration~\cite{ding2021local} can improve \ood calibration}}, especially under strong domain shifts, where models are poorly calibrated.
% These are even more effective when they have access to \ood images during calibration. 
%\todo{Mention proof?}

\myparintro{iii) Misclassification detection shows different 
% behavior 
model ranking
in and out of domain} 
% \rict{Perhaps surprisingly, } 
When tested in domain, recent models 
% actually 
underperform the ResNet baseline. As the domain shift increases, 
% more 
recent models take the lead.

\myparintro{iv) \ood detection is inversely correlated 
% to 
with
performance} 
% When it comes to \ood detection, 
% In this task, 
Indeed,
a small ResNet-18 backbone performs best.

\myparintro{v) Content-dependent calibration~\cite{ding2021local} can improve \ood detection and misclassification out of domain} We observe a significant increase in misclassification detection under strong domain shifts after improving calibration. We also observe improvements for \ood detection, albeit milder.


% \todo{Make fig/table caption more self-consistent by adding title/bold sentences}

% \todo{use ood/id instead of \ood/in-domain throughout text}

\begin{table}[t!]
\begin{center}
{\scriptsize
\setlength{\tabcolsep}{2.5pt}
\begin{tabular}{@{}lccccc@{}}
\toprule
 & 
\begin{tabular}{@{}c@{}}\textbf{Sem.} \\ \textbf{segm.}\end{tabular} &
\begin{tabular}{@{}c@{}}\textbf{Robust} \\ \textbf{performance}\end{tabular} &
\begin{tabular}{@{}c@{}}\textbf{Uncertainty} \\ \textbf{estimation}\end{tabular} &
\begin{tabular}{@{}c@{}}\textbf{Natural} \\ \textbf{shifts}\end{tabular} &
\begin{tabular}{@{}c@{}}\textbf{OOD calib} \\ \textbf{methods}\end{tabular}

% \textbf{Segm.} & 
% \textbf{Robust.} & 
% \textbf{Reliab.} &  
% \textbf{Nat. shifts} &
% \textbf{Calib. meth.}

\\

\midrule
Kamann~\etal 
% (2021)~
\cite{KamannCVPR20BenchmarkingRobustnessSemSegmModels} & \checkmark & \checkmark & & & \\

\midrule
Bhojanapalli~\etal 
% (2021)~
\cite{BhojanapalliICCV21UnderstandingRobustnessTransformersImageClass} &  & \checkmark & & \checkmark & \\

\midrule
Xie~\etal 
% (2021)~
\cite{XieNIPS21SegFormerSemSegmTransformers} &  \checkmark & \checkmark & &  & \\

\midrule
Naseer~\etal 
% (2021)~
\cite{naseer2021intriguing} &  & \checkmark & & & \\

\midrule
Bai~\etal 
% (2021)~
\cite{bai2021transformers} &  & \checkmark &  & \checkmark & \\

\midrule
Minderer~\etal 
% (2021)~
\cite{minderer2021revisiting} &  & \checkmark & \checkmark & \checkmark & \\

\midrule
Paul and Cheng 
% (2022)~
\cite{paul2022vision} &  & \checkmark & & \checkmark & \\

\midrule
Mao~\etal 
% (2022)~
\cite{mao2022towards} &  & \checkmark & & \checkmark & \\

\midrule
Liu~\etal 
% (2022)~
\cite{LiuCVPR22AConvNet4The2020s} &  & \checkmark & & \checkmark & \\

\midrule
Zhou~\etal 
% (2022)~
\cite{zhou2022understanding} &  \checkmark & \checkmark & & & \\

\midrule
Pinto~\etal 
% (2022)~
\cite{pinto2022impartial} &  & \checkmark & \checkmark & \checkmark & \\

\midrule
\midrule

\textit{Ours} &  \checkmark & \checkmark & \checkmark & \checkmark  & \checkmark \\

\bottomrule
\end{tabular}
}
\end{center}
% \vspace{-10pt}
\caption{\textbf{Studies of recent architectures}. While several prior works studied robustness and uncertainty of transformer- and CNN- based \textit{classifiers}, studies on \textit{segmentation} limited to robustness. This is the first study assessing robustness \textit{and} uncertainty of modern segmentation models. Moreover, we consider natural domain shifts and are the only analysis to include content-dependent methods~\cite{gong2021confidence, ding2021local} to improve calibration
in \ood settings. 
}
\label{tab:comparison} 
\end{table}

