\section{Discussion}
\mypara{Privacy concerns. }As our method relies on collecting unlabeled repeated traversal of the same routes, there are privacy concerns that have to be addressed before public deployment. This could be achieved by making data collection an opt-in option for drivers. Also, the collected data should be properly annoymized, or reduced to random road segments to remove any potential personal identifiable information. 

\mypara{Limitations.} Our method currently focuses on adapting \emph{dynamic}, \ie mobile, object detectors to target domains using multiple traversals. However, \methodshort could be extended to \emph{static} objects easily via selecting the appropriate thresholds for \ppfilterlong and \ppsupervisionlong. We leave this exploration for future work. Also, we assume the source and target domain share the same object frequency for \postfilterlong. However, this assumption could be alleviated via querying local authorities for the object frequency or by estimating the object frequencies from similar regions (we assume access to the locations of target domain). 


\section{Conclusion}
End-user domain adaptation is one of the key challenges towards safe and reliable self-driving vehicles. 
In this paper we claim that unlike most domain adaptation settings in machine learning, the self-driving car setting naturally gives rise to a weak supervision signal that is exceptionally well-suited to adapt 3D object detector to a new environment. As drivers share roads, unlabeled LiDAR data automatically comes in the form of multiple traversals of the same routes. We show that with such data we can iteratively refine a detector to new domains. This is effective because we prevent it from reinforcing mistakes with three ``safe guards'': 1. \postfilterlong, 2.\ppfilterlong, 3. \ppsupervisionlong. 
Although the experiments in this paper already indicate that \methodlong may currently be the most effective approach for adaptation in the self-driving context, we believe that the true potential of this method may be even greater than our paper seems to suggest. As cars with driver assist features become common place, collecting unlabeled data will become easier and cheaper. This could give rise to unlabeled data sets that are several orders of magnitudes larger than the original source data set, possibly yielding consistently more accurate detectors than are obtainable with purely hand-labeled training sets. 


