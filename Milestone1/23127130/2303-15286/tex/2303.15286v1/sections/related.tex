\section{Related Works}
We seek to adapt a 3D object detector from a source to a target domain with the help of unlabeled target data of \textit{repeated traversals}. 


\mypara{Unsupervised Domain Adaptation in 3D.} Improving generalizability of visual recognition systems (trained on a source domain) without annotated data from the testing environment (target domain) falls under the purview of unsupervised domain adaptation (UDA). The key to successful adaptation is leveraging the right information about the target domain. After all, without any knowledge of the target domain, adapting any learning systems would be extremely challenging, if not impossible. The most common source of information used for adaptation in the literature is the unlabeled data from the target domain; ST3D~\citep{yang2021st3d} improves conventional self-training adaption using stronger data augmentations and maintaining a memory bank of high quality predictions for self-training throughout adaptation;  inspired by success in 2D UDA approaches that leverage feature alignment techniques \citep{long2015learning, tzeng2017adversarial, hoffman2018cycada, peng2019moment, saito2019strong}, MLC-Net \citep{luo2021MLC-Net} proposes to encourage domain alignment by imposing consistency between a source detector and its exponential moving average~\citep{tarvainen2017mean} at point, instance, and neural-statistics-level on the target unlabeled data. 

Other than unlabeled data, other work has also sought to use other information from the target domain to improve adaptation. 
One notable work along these lines is \textit{statistical normalization}~\citep{wang2020train} where the authors identify car size difference as the biggest source of domain gap and propose to scale the source data with the target data car size for adaptation. Knowing that the difference in weather conditions between source and target domain would cause changes in point cloud distributions, SPG \citep{xu2021spg} seeks to fill in points at foreground regions to address the domain gap. 
In addition to unlabeled data, other work~\citep{saltori2020sf, you2022exploiting} has also explored temporal consistency --- or more precisely, tracking of rigid 3D objects --- to improve adaptation. Our work explores another rich yet easily attainable source of information --- repeated traversals. In principle, one could combine our approach with prior UDA methods that use only the unlabeled data \citep{yang2021st3d, luo2021MLC-Net} for additional marginal gains at the cost of increased algorithmic complexity. We choose to keep our contribution simple and clear and focus only on self-training with repeated traversals, which in itself is very effective and straightforward to replicate. 


\mypara{Repeated Traversals.} Repeated traversals contain rich information that have already been used in a variety of scenarios. 
Early works utilize multiple traversals of the same route for localization~\citep{barnesephemerality,levinson2010robust}.
Repeated traversals of the same location allows discovering of non-stationary points in a point cloud captured by modern self-driving sensor since non-stationary points are less likely to persist across different traversals of the same location. To formalize this intuition, \citep{barnesephemerality} develop an entropy-based measure, termed ephemerality score (see background \ref{section:background}) to determine dynamic points in a scene and subsequently, uses the signal to learn a representation for 2D visual odometry in a self-supervised manner. Building upon ephemerality, \citep{you2022learning} utilize multiple common sense rules to discover a set of mobile objects for self-training a mobile object detector without any human supervision. Similar to \cite{you2022learning} we leverage information from repeated traversals and use self-training; however in contrast to our work they focus on single class object discovery  
and our work is the first to show how multiple traversals can be used for domain adaptation. In addition to detecting foreground points/objects, repeated traversals have also been utilized by Hindsight~\citep{you2022hindsight} to decorate 3D point clouds with learned features for better 3D object detection. In principle, we could combine our approach with Hindsight to bring forth better generalizability but we did not explore such combination for simplicity and leave the exploration for future work. 

