\section{Conclusions}
\label{Conclusions} 
The \jpsi \pt-differential yields and nuclear modification factors \RAA 
measured from $\pt = 0.15$~GeV/c up to $15$~GeV/c in the 0–10\% and 30–50\% centrality ranges at midrapidity and from $\pt = 0.3$~GeV/c up to $\pt = 20$~GeV/c in the 0--20\%, 20--40\% and 40--90\% centrality ranges at forward rapidity, and the centrality dependent \jpsi \meanpt and \rAA  measured at midrapidity, are reported and discussed in comparison with model calculations.


The centrality dependent \pt-integrated \RAA in peripheral collisions shows similar values at both midrapidity and forward rapidity, while a hint of an increasing trend of the \RAA is observed at midrapidity towards central collisions. When looking at the \pt-differential \RAA, relatively large values are observed at low \pt, which are compatible with unity for $\pt<3$~\GeVc in the 0--10\% centrality interval at $|y|<0.9$, while a strong nuclear suppression is seen at higher \pt in central and semicentral collisions. In addition, \RAA is higher at midrapidity 
than at forward rapidity  for $\pt < 3$~GeV/c in the most central collisions. A weaker \pt dependence of the \RAA values is observed for more peripheral collisions. Such a behaviour, for the central and semicentral collisions, can be explained by a large contribution from (re)generation to the \jpsi yields. This is supported by the statistical hadronisation model and by two microscopic transport model calculations when compared to the data. However, the large model uncertainties, which are mainly due to the assumptions on the collision initial conditions, prevent from drawing a clear conclusion on the phenomenology of the \jpsi production in heavy-ion collisions at LHC energies.
The \jpsi nuclear modification factor at high \pt is well described by the transport models and also by a \jpsi energy loss model, while it is largely underestimated in the hydro-inspired freeze-out approach implemented together with the SHMc model. For most central events, the \RAA converge to similar values at high \pt at mid and forward rapidity suggesting a weaker dependence on rapidity for the suppression effects. The centrality dependent \jpsi \meanpt and \rAA measurements are compared with similar results at lower energies from RHIC~\cite{PHENIX:2006gsi,Adare:2008sh,Adare:2011vq} and SPS~\cite{Abreu:2000xe}, with the centrality trends showing an opposite behaviour between the LHC and the lower energy results. This behaviour is compatible with a strong contribution from the regeneration component which tends to soften the \pt distributions. The two microscopic transport models describe the data within the uncertainties.

The ratio of inclusive \pt-integrated \jpsi to \dzero yields measured by ALICE at midrapidity in \PbPb collisions is presented for the first time in the 0--10\% and 30--50\% centrality ranges. The data shows a larger value of this ratio in 0--10\% compared to 30--50\% collisions, which is in good agreement with the expectation from the SHMc model that fixes the charm fugacity parameter based on the \dzero yields measured by ALICE. 

The large improvements in experimental accuracy expected for the LHC Run 3 and 4 for charmonium measurements and more in general for heavy-quark production, the improved measurements of total charm quark cross section and CNM effects are critical to constrain the phenomenological model calculations, will allow to settle the longstanding questions regarding the mechanisms behind charmonium production in heavy-ion collisions.
