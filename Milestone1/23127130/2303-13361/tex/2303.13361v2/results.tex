\section{Results}\label{sec:results} 

\subsection{Inclusive \textbf{J/$\psi$} yields} \label{subsec:spectrum}

The fully corrected inclusive \jpsi \pt-differential yields, \dndydpt, were obtained according to Eq.\ref{eq:d2ndydpt} in centrality intervals in $\PbPb$ collisions at \fivenn at midrapidity (\yrange{0.9}) and at forward rapidity ($2.5 < \y < 4$). Figure~\ref{fig:JpsiPtSpectra_mid} shows the \jpsi yields obtained at midrapidity for the $0$--$10$\% and $30$--$50$\% centrality intervals, while Fig.~\ref{fig:JpsiPtSpectra_forward} shows the yields measured at forward rapidity in the $0$--$20$\%, $20$--$40$\% and $40$--$90$\% centrality intervals. For all of the results, the statistical and systematic uncertainties are indicated by the vertical error bars and the open boxes around the data points, respectively. These results are compared with calculations performed using  the statistical hadronisation model (SHMc) by Andronic et al.~\cite{Andronic:2019wva}, and two microscopic transport models by Rapp et al.~\cite{Zhao:2007hh} and Zhuang et al.~\cite{Zhou:2014kka}. The physics assumptions in these model calculations are discussed in more detail in the next section. The lower panels of the Figs.~\ref{fig:JpsiPtSpectra_mid} and ~\ref{fig:JpsiPtSpectra_forward} depict the ratio between the experimental data and the different model calculations, with the width of the bands representing the model uncertainties. These uncertainties are due to uncertainties on input parameters, mainly the total charm-quark production cross section and CNM effects. The filled boxes around unity represent the uncertainties of the measured results, shown as the quadratic sum of statistical and systematic uncertainties. Both transport models describe the \pt-differential yields in central and semicentral collisions, while they overestimate them at forward rapidity in peripheral collisions. The SHMc calculations coupled with a hydrodynamics inspired freeze-out parameterisation are in good agreement with the data in the low-$\pt$ region, but underestimate the measurements at higher \pt. 


\begin{figure}[!htb]
\begin{center}
  \includegraphics[width = 7.5 cm]{figures/midy/Spectrum_Vs_pt_0_10_015_model_ee.pdf}
  \includegraphics[width = 7.5 cm]{figures/midy/Spectrum_Vs_pt_30_50_015_model_ee.pdf}
\caption{\jpsi \pt-differential production yields in $\PbPb$ collisions at \fivenn at midrapidity in the $0$--$10$\% (left panel) and $30$--$50$\% (right panel) centrality intervals. The statistical and systematic uncertainties are indicated, respectively, by the vertical error bars and the open boxes. The horizontal bars indicate the \pt intervals. Data are compared to model calculations from Refs.~\cite{Andronic:2019wva,Zhao:2007hh,Zhou:2014kka}. The ratios between data and models are shown in the lower panels. The filled boxes around unity depict the quadratic sum of statistical and systematic uncertainties from the measurement, while the bands indicate model uncertainties.}
\label{fig:JpsiPtSpectra_mid}
\end{center}
\end{figure}

\begin{figure}[!htb]
\begin{center}
  \includegraphics[width = 7.5 cm]{figures/fwdy/Spectrum_Vs_pt_0_20_015_model_mm.pdf}
  \includegraphics[width = 7.5 cm]{figures/fwdy/Spectrum_Vs_pt_20_40_015_model_mm.pdf}
  \includegraphics[width = 7.5 cm]{figures/fwdy/Spectrum_Vs_pt_40_90_015_model_mm.pdf}
\caption{\jpsi \pt-differential production yields in $\PbPb$ collisions at \fivenn at forward rapidity in the $0$--$20$\%, $20$--$40$\%, and $40$--$90$\% centrality intervals. The statistical and systematic uncertainties are indicated, respectively, by the vertical error bars and the open boxes. The horizontal bars indicate the \pt intervals. Data are compared to model calculations from Refs.~\cite{Andronic:2019wva,Zhao:2007hh,Zhou:2014kka}. The ratio between data and models is shown in the lower panels. The filled boxes around unity depict the quadratic sum of statistical and systematic uncertainties from the measurement, while the bands indicate model uncertainties.} 
\label{fig:JpsiPtSpectra_forward}
\end{center}
\end{figure}

\FloatBarrier

\subsection{The \textbf{J/$\psi$} nuclear modification factor \RAA } \label{subsec:RAA}


The nuclear modification factor \RAA was obtained using the measured yields, according to Eq.~\ref{eq:Raa}. Figure~\ref{fig:Raa_vs_cent} shows the \pt-integrated \jpsi \RAA as a function of \Npart in \PbPb collisions at \fivenn, obtained in the current analysis at midrapidity in comparison to the results at forward rapidity, previously reported by the ALICE Collaboration in Ref.~\cite{ALICE:2016flj}. Global uncertainties represented the centrality-correlated uncertainties, shown as filled boxes around unity, and are largely uncorrelated between the forward and midrapidity analyses. Both results exclude low-\pt \jpsi, with a selection of $\pt>0.15$~\GeVc and $\pt>0.3$~\GeVc at midrapidity and forward rapidity, respectively, in order to reject \jpsi produced via photoproduction processes~\cite{ALICE:2012yye,ALICE:2019tqa,ALICE:2021gpt}, which contribute significantly to the \jpsi yield in particular in peripheral collisions~\cite{ALICE:2015mzu,ALICE:2022zso}. The \RAA is compatible with unity in the most peripheral collisions, while a suppression of the \jpsi production in \PbPb collisions with respect to binary scaled \pp collisions is observed in semicentral and central collisions, in particular at forward rapidity. At midrapidity, \RAA exhibits a slightly increasing trend from approximately $\langle\Npart\rangle$~=~100 towards the most central collisions, with slightly larger \RAA values at midrapidity than at forward rapidity, which confirms previous observations reported by ALICE~\cite{ALICE:2019nrq}. The results at midrapidity are larger than those measured at forward rapidity, with a significance of the difference of $2.2\sigma$ when considering the data points in the centrality range 0--10\%. 
The larger \RAA in central collisions at midrapidity is expected in phenomenological models due to the larger ${\rm d}\sigma_{\ccbar}/{\rm d}y$ at midrapidity which leads to larger fraction of \jpsi produced via (re)generation~\cite{Andronic:2019wva,Zhou:2014kka,Wu:2020zbx}.


 
\begin{figure}[!htb]
\begin{center}
  \includegraphics[width = 7.5 cm]{figures/midy/Raa_Vs_cent_015_data.pdf}
\caption{Inclusive \jpsi nuclear modification factor at midrapidity and forward rapidity~\cite{ALICE:2016flj}, integrated over \pt, as a function of the number of participants in \PbPb collisions at \fivenn. The statistical and systematic uncertainties are indicated, respectively, by the vertical error bars and the open boxes around the data points. The filled boxes around unity show the global uncertainties.}
\label{fig:Raa_vs_cent}
\end{center}
\end{figure}

The \pt-differential \RAA results are shown in Fig.~\ref{fig:RAA_vs_pt} for midrapidity (left panel) and forward rapidity (right panel) in various centrality intervals. The main feature of these measurements is that the \RAA values are relatively large at low \pt ($\pt<5$~\GeVc), in contrast with the strong suppression in the $\pt > 5$~\GeVc range, for central and semicentral collisions. A weaker \pt dependence of the \RAA values is observed from central to peripheral collisions, up to a constant \RAA, within uncertainties, in the 40–90\% centrality interval at forward rapidity. Such a behaviour in the data can be qualitatively understood by the dominance of hot nuclear matter effects for central and semicentral collisions, acting on top of the CNM ones, which were discussed in Refs.~\cite{ALICE:2015sru,ALICE:2021lmn}. In the low-\pt region and in particular in central and semicentral collisions, where the density of charm quarks is larger, the coalescence of charm quark pairs has an important contribution counterbalancing the impact of quarkonium suppression in the QGP. At higher \pt, the \jpsi production is dominated by effects such as dissociation and energy loss, which are expected to be stronger in the most central collisions.
A comparison of the \jpsi \pt-differential \RAA in the most central Pb–Pb collisions at \fivenn between midrapidity and forward rapidity is shown in Fig.~\ref{fig:RAA_vs_pt_mid_vs_forward_y}. Neglecting the slight difference in centrality intervals, the \RAA is higher at midrapidity with respect to forward rapidity at low \pt ($\pt < 3$~\GeVc) with a 2.7$\sigma$ significance, highlighting the strong dependence of \RAA on the local charm quark density, further supporting the picture of quarkonium production via coalescing charm quarks. The two measurements converge to similar values at higher \pt suggesting a weaker dependence on rapidity for the suppression effects.


\begin{figure}[!htb]
\begin{center}
  \includegraphics[width = 7.5 cm]{figures/midy/Raa_Vs_pt_0_10_30_50_015_data_ee.pdf}
  \includegraphics[width = 7.5 cm]{figures/fwdy/Raa_Vs_pt_all_015_model_mm.pdf}
\caption{Inclusive \jpsi \pt-differential \RAA in Pb--Pb collisions at \fivenn in various centrality intervals. The left panel shows the comparison of \RAA measured in central (0--10\%) and semicentral (30--50\%) collisions at midrapidity. For the data point in the \pt bin 10 $<$ \pt $<$ 15 \GeVc an open symbol is used to highlight the usage of \pp reference from the extrapolation approach. The right panel shows the measured \RAA in three centrality classes, 0--20\%,  20--40\%, and 40--90\%, at forward rapidity. The statistical and systematic uncertainties are indicated, respectively, by the vertical error bars and the open boxes around the data points. The filled boxes around unity show the global uncertainties.}
\label{fig:RAA_vs_pt}
\end{center}
\end{figure}


\begin{figure}[!htb]
\begin{center}
  \includegraphics[width = 7.5 cm]{figures/midy/Raa_Vs_pt_midy_fwdy.pdf}
\caption{Inclusive \jpsi \RAA as a function of \pt in Pb--Pb collisions at \fivenn at midrapidity and forward rapidity, in the 0--10\% centrality class and 0--20\% centrality class, respectively.For the data point in the \pt bin 10 $<$ \pt $<$ 15 \GeVc an open symbol is used to highlight the usage of \pp reference from the extrapolation approach. The statistical and systematic uncertainties are indicated, respectively, by the vertical error bars and the open boxes around the data points. The filled boxes around unity show the global uncertainties.}
\label{fig:RAA_vs_pt_mid_vs_forward_y}
\end{center}
\end{figure}

The measurements presented so far are discussed in the rest of this section in comparison to calculations which employ different approaches in modelling the collision fireball and the hot medium effects having an impact on \jpsi production in \PbPb collisions.


The statistical hadronisation model by Andronic et al.~\cite{Andronic:2019wva} assumes that all charm quarks are produced during the initial hard partonic interactions and then thermalize in the QGP. The relative yields of the charmed hadrons are then determined solely by the equilibrium thermodynamical parameters at the chemical freeze-out, which are fixed from fits to the yields of light-flavoured hadrons. A recent extension of the model~\cite{Andronic:2019wva,Braun-Munzinger:2000csl} implements a hydro-inspired freeze-out hypersurface which allows the calculation of \pt-differential yields in addition to the integrated ones.

The microscopic transport model of Zhuang et al.~\cite{Zhou:2014kka,Chen_2019} implements a real time evolution of the \jpsi, $\psi(2S)$ and $\chi_{\rm c}$ production using a Boltzmann-type rate equation that includes both dissociation and coalescence terms. The dissociation term includes contributions from the temperature dependent colour screening effect and scatterings with thermal partons, i.e. gluon dissociation. The (re)generation of charmonia is implemented by exploiting the detailed balance of the gluon dissociation process. The space--time evolution of the fireball is described using the equations for (2+1)D ideal hydrodynamics. This model includes also the production of non-prompt \jpsi, with the precursor beauty quarks being propagated through the QGP using the Langevin equation. 


Similarly to the previously described model, the transport model proposed by Rapp et al. ~\cite{Grandchamp:2001pf,Grandchamp:2003uw,Zhao:2010nk,Zhao:2011cv,Wu:2020zbx}, is also based on a kinetic rate equation to compute the time evolution of charmonium (\jpsi, $\psi(2S)$, and $\chi_c$) yields. The dissociation term of the rate equation employs an inelastic parton scattering cross section of charmonia in the QGP, computed using next-to-leading order perturbative QCD Feynman diagrams, and also includes the effect of in-medium reduced binding energy. The (re)generation rate depends on the charmonium dissociation temperature, which is extracted from lattice QCD calculations, and equilibrium limits computed based on the thermal model. The space--time evolution of heavy-ion collisions is simulated by a cylindrically expanding fireball model with the regenerated charmonium \pt-spectra being calculated in a thermal blast-wave approximation at a temperature and flow velocity reflecting the average production time of each charmonium state~\cite{PhysRevC.48.2462}.

All of the models described above consider the initial state of the nuclear collisions by making assumptions on the total charm quark density produced during the hard partonic collisions and modified by the CNM effects. The two transport models obtain the total charm density based on the measured total charm cross section in pp collisions~\cite{PhysRevD.105.L011103} multiplied by the number of binary nucleon--nucleon collisions. The CNM effects are introduced via different approaches. The microscopic transport model of Rapp et al.~\cite{Grandchamp:2001pf,Grandchamp:2003uw} estimates CNM effects using fits of the measured \pA data, while the transport model of Zhuang et al.~\cite{Zhao:2010nk,Zhao:2011cv} uses the nuclear parton distribution functions and their uncertainties from EPS09~\cite{Eskola:2009uj}.
The SHMc extracts the total charm cross section from the ALICE measurements of D meson production in \PbPb collisions~\cite{alicedmesons}. The large uncertainty on the estimation of CNM effects is inherited by these model calculations.


 At large \pt, the fragmentation of high-energy partons may become the main mechanism for \jpsi production. In that case, energy loss of partons due to multiple scattering in the QGP leads to \jpsi suppression at high \pt. In the model by Arleo et al.~\cite{Arleo:2017ntr}
, the quenching of large-\pt particles ($\pt > 10$~\GeVc) is assumed to be mostly due to radiative parton energy loss. In this approach, the \pt dependence of \RAA is fully predicted from the model proposed by Baier et al.~\cite{Baier:1996kr,Baier:1996sk}, which employed a medium-induced gluon spectrum. The \RAA value is computed from the mean energy loss, which is extracted from a fit to the charged hadron \RAA, measured in various collision systems, the average fragmentation function, and the colour coupling factor of the parton. At forward rapidity, the mean energy loss is further corrected for the charged-hadron multiplicity difference between midrapidity and forward rapidity. The model uncertainties arise from the uncertainties on these inputs. This model does not include the production of non-prompt \jpsi, but the \RAA variation, when accounting for this contribution, is expected to lie within the theoretical uncertainties.  

The \pt-integrated nuclear modification factor measured in \PbPb collisions at \fivenn at midrapidity is shown in Fig.~\ref{fig:Raa_vs_cent_models} in comparison with results from the SHMc and the two transport-model calculations. The calculations are shown as coloured bands, illustrating the uncertainties on the initial effects, mainly CNM effects, described above. Within the model uncertainties, all three predictions agree with the data. One can note though that the data lie on the upper edge of the transport-model calculations, while they are in good agreement with the central values from the SHMc calculations for semicentral and central collisions.

Figures~\ref{fig:RAA_vs_pt_model} and ~\ref{fig:RAA_vs_pt_model_forward} show the \pt-differential \RAA measurements for various centrality intervals at midrapidity and forward rapidity, respectively, in comparison with the available model calculations. With the exception of the energy-loss calculations, available only for $\pt>10$~\GeVc, all of the models cover the full \pt range in which these measurements were performed. The SHMc model calculations are in good agreement with the data at low \pt at both midrapidity and forward rapidity. However, the \RAA is underestimated for $\pt>5$~\GeVc in all centrality intervals in both rapidity ranges. 
This might be attributed to physical sources missing in this approach, such as the contributions from surviving primordial \jpsi or non-prompt \jpsi from beauty-hadron decays, but also to an underestimated amount of radial flow acquired by the charm quarks during the system evolution. 
 A similar conclusion can also be drawn from the comparison of the \pt-differential yields in \PbPb collisions shown in Fig.~\ref{fig:JpsiPtSpectra_mid} where the measured spectrum is harder than the one from the SHMc calculations. The two transport models are in better quantitative agreement with data than the SHMc model. Both of the transport models provide a good description of the \RAA at both low and high \pt. However, the model calculations in the low-\pt region, where \jpsi production is dominated by coalescence in these models, do not describe the detailed shape of the \pt dependence of \RAA, in particular in semicentral collisions, which points to a still not perfectly understood dynamics of charm-quark coalescence.

The energy-loss calculations by Arleo et al~\cite{Arleo:2017ntr}, performed in all studied centrality ranges for $\pt>10$~\GeVc, are in good agreement with the measurements, which, based on the model assumptions, suggests that the dominant mechanism in this kinematic regime is indeed energy loss, similar to that of the other hadrons measured at LHC energies.

\begin{figure}[!htb]
\begin{center}
  \includegraphics[width = 7.5 cm]{figures/midy/Raa_Vs_cent_015_model.pdf}
\caption{Inclusive \jpsi \RAA at midrapidity, integrated over \pt, as a function of \Npart in Pb--Pb collisions at \fivenn and compared to model calculations from Refs.~\cite{Andronic:2019wva,Zhao:2007hh,Zhou:2014kka}. 
}
\label{fig:Raa_vs_cent_models}
\end{center}
\end{figure}


\begin{figure}[!htb]
\begin{center}
  \includegraphics[width = 7.5 cm]{figures/midy/Raa_Vs_pt_0_10_015_model_ee.pdf}
  \includegraphics[width = 7.5 cm]{figures/midy/Raa_Vs_pt_30_50_015_model_ee.pdf}
\caption{Transverse-momentum dependence of the \jpsi \RAA in $\PbPb$ collisions at \fivenn
at midrapidity in the $0$--$10$\% (left panel) and $30$--$50$\% (right panel) centrality intervals. For the data point in the \pt bin 10 $<$ \pt $<$ 15 \GeVc an open symbol is used to highlight the usage of \pp reference from the extrapolation approach. The data are compared with model calculations from Refs.~\cite{Andronic:2019wva,Zhao:2007hh,Zhou:2014kka,Arleo:2017ntr}.}
\label{fig:RAA_vs_pt_model}
\end{center}
\end{figure}


\begin{figure}[!htb]
\begin{center}
  \includegraphics[width = 7.5 cm]{figures/fwdy/Raa_Vs_pt_00_20_015_model_mm.pdf}
  \includegraphics[width = 7.5 cm]{figures/fwdy/Raa_Vs_pt_20_40_015_model_mm.pdf}
   \includegraphics[width = 7.5 cm]{figures/fwdy/Raa_Vs_pt_40_90_015_model_mm.pdf}
\caption{Transverse-momentum dependence of the \jpsi \RAA at forward rapidity in the $0$--$20$\%, $20$--$40$\% and $40$--$90$\%  centrality intervals. The data are compared with model calculations from Refs.~\cite{Andronic:2019wva,Zhao:2007hh,Zhou:2014kka,Arleo:2017ntr}}
\label{fig:RAA_vs_pt_model_forward}
\end{center}
\end{figure}



\FloatBarrier

\subsection{The inclusive \textbf{J/$\psi$} \meanpt and \meanptsq } \label{subsec:meanpt}

Observables that allow for a more differential study of the \jpsi \pt spectrum with respect to the centrality of the collisions are the \jpsi \meanpt and \meanptsq. The latter is typically quantified using the \rAA, defined in Eq.\ref{eq:little_raa}, which is related to the broadening or narrowing of the \jpsi \pt spectrum relative to that in \pp collisions. The \meanpt and the \rAA measured at midrapidity are shown in Fig.~\ref{fig:mean_pt_rAA_with_data}  in the left and right panels, respectively, as a function of $\langle\Npart\rangle$. Similar measurements were done at the forward rapidity as well~\cite{ALICE:2019lga}. These results are compared with similar measurements in heavy-ion collisions at RHIC~\cite{PHENIX:2006gsi,PhysRevLett.101.122301,Adare:2011vq} and SPS~\cite{Abreu:2000xe} energies. The \meanpt results are also compared with the value reported by the ALICE Collaboration in \pp collisions at midrapidity at $\s=5.02$~TeV~\cite{Acharya:2019lkw}, showing a good agreement with the value measured in the most peripheral \PbPb collisions. The data show that for a given \Npart, the \jpsi \meanpt grows with increasing collision energy. However, while the data indicate no centrality dependence of the \meanpt at the SPS and RHIC energies, a monotonically decreasing trend from the most peripheral to the most central collisions is observed in the ALICE measurements, which reflects the gradual increase of the low \pt (re)generation component. The \rAA results support the observations for the \meanpt, showing a decrease from unity towards central collisions for the ALICE measurements. The RHIC results are compatible with unity over the whole covered centrality range, while the SPS data indicate a strong increase from peripheral to central collisions, suggesting that CNM effects such as the Cronin effect ~\cite{PhysRevLett.88.232303} have an impact on the \jpsi \pt shape.

The \meanpt and \meanptsq results are also compared with the aforementioned transport model calculations, which show a good agreement with the trends observed in data as demonstrated in Fig.~\ref{fig:mean_pt_rAA_with_model}. Although overall in good quantitative agreement, the calculations by Rapp et al. overestimate the \jpsi \meanpt for the most central collisions, while the \meanptsq results are slightly underestimated by both models in the semicentral and peripheral range ($50<\Npart<150$).   


\begin{figure}[!htb]
\begin{center}
  \includegraphics[width = 7.5 cm]{figures/meanpt/MeanPt_vs_Cent_data.pdf}
  \includegraphics[width = 7.5 cm]{figures/meanpt/raa_vs_Cent_data.pdf}
\caption{Left panel: Inclusive \jpsi \meanpt as a function of the mean number of participants in Pb--Pb collisions at \fivenn at midrapidity. Right panel: Inclusive \jpsi \rAA as a function of centrality at \fivenn and compared with measurements at lower energies from RHIC~\cite{PHENIX:2006gsi,Adare:2008sh,Adare:2011vq} and the SPS~\cite{Abreu:2000xe}. The statistical and systematic uncertainties are indicated, respectively, by the vertical error bars and the open boxes around the data points. The filled box around unity on the right panel shows the global uncertainty.}
\label{fig:mean_pt_rAA_with_data}
\end{center}
\end{figure}


\begin{figure}[!htb]
\begin{center}
  \includegraphics[width = 7.5 cm]{figures/meanpt/MeanPt_vs_Cent_model.pdf}
  \includegraphics[width = 7.5 cm]{figures/meanpt/raa_vs_Cent_model.pdf}
\caption{Inclusive \jpsi \meanpt as a function of the mean number of participants in Pb--Pb collisions at \fivenn at midrapidity (left panel), and \jpsi \rAA as a function of centrality (right panel).
The results are compared with transport model calculations~\cite{Zhao:2007hh,Zhou:2014kka}}
\label{fig:mean_pt_rAA_with_model}
\end{center}
\end{figure}

\FloatBarrier

\subsection{The \textbf{J/$\psi$} to D$^{0}$ yield ratio} 
\label{subsec:jpsi_D0_ratio}

A long awaited measurement which helps understanding the details of the \jpsi production in heavy-ion collisions is the ratio between the \jpsi and the \dzero yields, both measured in the same collision system. Such a measurement provides a tight constraint to models because some of the model parameters and most model uncertainties related to the \ccbar cross section
cancel in the ratio. This ratio is sensitive to the hadronisation mechanisms of the different charm hadrons. While a model independent measurement of the \ccbar production density in heavy-ion collisions is not available, it is still useful to compare the \jpsi yield with the recently published ALICE measurements of the \dzero yield down to zero \pt~\cite{alicedmesons}.


Figure~\ref{fig:D0tojpsi} shows the measured \pt-integrated \jpsi to \dzero yield ratio in central (0--10\%) and semicentral (30--50\%) collisions. The largest source of systematic uncertainty for both measurements comes from tracking efficiency and it is considered correlated between the \dzero and \jpsi measurements, and consequently cancels in the ratio. Without this uncertainty, the numerical values of the \jpsi yields are $0.12 \pm 0.005$~(stat.)~$\pm 0.012$~(syst.)~and $0.016 \pm 0.0008$~(stat.)~$\pm 0.0004$~(syst.) for (0--10\%) and (30--50\%) centrality intervals, respectively. 
The statistical (systematical) uncertainty of the ratio is the quadratic sum of the statistical (systematical) uncertainties of the two measurements. 
The results suggest a higher value for this ratio in central compared to semicentral collisions. This is supported by the SHMc calculations~\cite{Andronic:2019wva}, which suggests both the \jpsi and \dzero are produced via the coalescence of charm quarks at the phase boundary, the ratio being determined by the charm fugacity. The SHMc model gives a good description of the data. The model uncertainty from the SHMc model is due to uncertainties on the charm fugacity parameter, which is fitted to the ALICE D$^{0}$ data ~\cite{alicedmesons}.
 
\begin{figure}[!htb]
\begin{center}
  \includegraphics[width = 8.5 cm]{figures/midy/Jpsi_to_D0_ratio_data.pdf}
\caption{Inclusive \jpsi to $D^{0}$ yield~\cite{alicedmesons} ratio at \fivenn at midrapidity for the 0--10\% and 30--50\% centrality intervals. Vertical lines and open boxes represent the statistical and systematical uncertainties, respectively. The measurements are compared with SHMc model predictions~\cite{Andronic:2019wva}. }
\label{fig:D0tojpsi}
\end{center}
\end{figure} 

\FloatBarrier
