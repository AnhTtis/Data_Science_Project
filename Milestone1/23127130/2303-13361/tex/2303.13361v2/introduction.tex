\section{Introduction} \label{sec:Introduction}
Quantum chromodynamics (QCD) is the theory describing the strong interaction. Lattice QCD, i.e. the discrete formulation of QCD, predicts the existence of a state of deconfined matter at high energy density that is characterised by quark and gluon degrees of freedom~\cite{Bazavov:2009zn,Borsanyi:2013bia}. This quark--gluon plasma (QGP) is created during the early hot and dense stage of heavy-ion collisions at ultra-relativistic energies. 

The heavy quarks, charm (c) and beauty (b), are unique probes for this phase of matter~\cite{Prino:2016cni,Rothkopf:2019ipj}. Due to their large masses, they are produced as quark--antiquark pairs in hard partonic scattering processes in the early stage of the collision, and they thus experience the full evolution of the system. While the majority of the produced heavy quarks and antiquarks hadronise independently into open heavy-flavour hadrons, bound quarkonium states can also be formed~\cite{Brambilla:2010cs}. However, it has been predicted that the formation of bound states should be suppressed due to the mechanism of colour screening where the large density of colour charges in the QGP hinders the production of bound quarkonia~\cite{Matsui:1986dk,Digal:2001ue}. The degree of suppression of the various quarkonium states depends on their binding energy along with the medium properties, such as its temperature. Consequently, the measurements of quarkonium production rates in heavy-ion collisions have been considered as a potential thermometer of the medium.

The production of \jpsi, the charmonium ground state with quantum numbers $J^{PC} = 1^{--}$, has been studied extensively in heavy-ion collisions over the last several decades. Suppression of the \jpsi yield was observed in nucleus--nucleus collisions with respect to the expectation from proton--proton collisions at the SPS (up to centre-of-mass energy per nucleon pair \snn $\mathrm{=}$ 17~GeV)~\cite{NA38:1994yau,NA50:2004sgj,NA60:2007ewx}, at RHIC (\snn up to 200~GeV)~\cite{PHENIX:2006gsi,PHENIX:2011img,STAR:2009irl,STAR:2013eve} and at the LHC (\snn $\mathrm{=}$ 2.76~TeV~\cite{CMS:2012bms,ALICE:2012jsl,ALICE:2013osk,ALICE:2015jrl} and 5.02~TeV~\cite{ALICE:2016flj,ALICE:2019lga, ALICE:2019nrq,ALICE:2022wpn, ATLAS:2018hqe,CMS:2017uuv}). However, contrary to the prediction from the colour-screening scenario, the measured suppression does not increase with increasing collision energy from RHIC to LHC despite of an increased energy density of the produced QGP. At LHC energies, \jpsi production is found to be less suppressed than at the lower RHIC energies, in particular at low transverse momentum, \pt, of the \jpsi~\cite{ALICE:2015jrl,starjpsi39-200,PhysRevLett.101.122301,PhysRevC.93.034903}. In addition, a significant azimuthal anisotropy in the \jpsi production was observed via the elliptic and triangular flow measurements reported in Refs.~\cite{ALICE:2017quq,ALICE:2020pvw}. These observations are explained by an additional \jpsi production mechanism, referred to as (re)generation in the following, in which copiously produced uncorrelated charm quarks and antiquarks bind into \jpsi mesons~\cite{Braun-Munzinger:2000csl,Thews:2000rj}. This process can only take place in a deconfined medium, and its contribution to the measured \jpsi yield increases with the density of c$\overline{\rm c}$ pairs and, therefore, with increasing collision energy and decreasing \pt~\cite{Andronic:2019wva,Zhou:2014kka,Du:2015wha}. With increasing \pt, (re)generation becomes less relevant for \jpsi production and, instead, charmonium dissociation and the fragmentation of high-energy partons into charmonia become dominant. In the latter case, the suppression of high-\pt \jpsi yields should reflect the energy loss of partons~\cite{Arleo:2017ntr}, which is mostly of radiative nature in this kinematic regime.

The formation process of charmonia in heavy-ion collisions is complex and various phenomenological approaches are considered. In the statistical hadronization scenario, the relative abundances of charmomium states with respect to other charmed hadrons  are determined by thermal weights~\cite{Braun-Munzinger:2000csl,Andronic:2019wva} at the system chemical freeze-out. In microscopic transport and comover interaction models, charmonia are continuously produced and broken up during their propagation through the QGP~\cite{Thews:2000rj,Zhou:2014kka,Du:2015wha,Ferreiro:2012rq}. 
Furthermore, it is important to consider cold nuclear matter (CNM) effects. In particular, the modification of the parton distribution functions in nuclei with respect to nucleons~\cite{Armesto:2006ph} has to be taken into account for the interpretation of the results. 
These CNM effects were investigated in ALICE especially with proton--nucleus collisions~\cite{ALICE:2013snh,ALICE:2015sru, ALICE:2015kgk,ALICE:2018mml,ALICE:2018szk, ALICE:2020tsj, ALICE:2021lmn}.

For a better assessment of the production mechanisms, systematic measurements of the centrality, \pt, and rapidity dependence of \jpsi production are pivotal. In this article, the ALICE results on inclusive \jpsi production at midrapidity (\yrange0.9) for $0.15 < \pt < 15$~GeV/c and forward rapidity (2.5 $< y <$ 4) for $0.3 < \pt < 20$~GeV/c, from the full Run~2 data sample at the LHC, are reported. Inclusive \jpsi measurements contain a prompt \jpsi contribution from direct \jpsi and decay from heavier charmonium states, and a non-prompt \jpsi contribution from the decay of beauty hadrons. The precision of the measurements, using the entire Run 2 data sample, improved significantly compared to previous ones~\cite{ALICE:2019nrq,ALICE:2019lga} and the measurements could be extended up to 15~GeV/c and 20~GeV/c at mid and forward rapidity, respectively. The \pt-differential \jpsi yields in \PbPb collisions at \fivenn are measured in various centrality classes. At midrapidity, the average transverse momentum $\langle\pt\rangle$ as well as squared transverse momentum $\langle\pt^{2}\rangle$ is determined, which provide a quantitative estimation of the evolution of the \pt spectra as a function of centrality. The nuclear modification factor \RAA defined, as the ratio of the yield in \PbPb to the corresponding yield in pp collisions scaled by the number of binary nucleon--nucleon collisions is calculated. The results as a function of \pt and collision centrality are compared with model calculations employing the statistical hadronisation~\cite{Andronic:2019wva}, microscopic parton transport~\cite{Zhou:2014kka,Du:2015wha}, comover~\cite{Ferreiro:2012rq}, and energy loss~\cite{Arleo:2017ntr} approaches.
