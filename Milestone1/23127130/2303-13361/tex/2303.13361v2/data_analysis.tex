\section{Analysis details}
\label{sec:analysis} 

The primary observable  is the $\pt$-differential \jpsi yield per unit of rapidity $\mathrm{d}^2N/(\mathrm{d}y \; \mathrm{d}\pt)$. For a given interval of centrality, rapidity ($\mathrm{\Delta}y$), and transverse momentum ($\mathrm{\Delta}\pt$), this is obtained as
\begin{equation}
\frac{\mathrm{d}^{2} N}{\mathrm{~d} y \mathrm{~d} \pt} = \frac{N_{\jpsi}}{N_{\rm{ev}} \times \mathrm{BR}_{\jpsi \rightarrow \mathrm{l}^+\mathrm{l}^-} \times (A \times \epsilon) \times \Delta y \times \Delta \pt},
\label{eq:d2ndydpt}
\end{equation}
where $N_{\jpsi}$ is the number of reconstructed \jpsi mesons, $N_{\mathrm{ev}}$ is the number of events corresponding to the analysed centrality interval, and ($A\times\epsilon$) is the acceptance times efficiency factor. The branching ratio (BR) corresponds to either the dielectron or the dimuon \jpsi decay channel. 
Since the analysis at forward rapidity was based on a sample of dimuon-triggered events, the equivalent $N_{\mathrm{ev}}$ was obtained as the product of the number of dimuon-triggered events times the inverse of the probability of having a dimuon trigger in a MB triggered event, $F_{\mathrm{norm}}$~\cite{ALICE:2015kgk,ALICE:2020jff}. The number of equivalent $N_{\mathrm{ev}}$ was first obtained for the 0--90\% centrality class and was then scaled to the centrality classes considered in the analysis. 

The nuclear modification factor, \RAA, is obtained as
\begin{equation}
\RAA=\frac{\mathrm{d}^{2} N / (\mathrm{d} y \mathrm{~d} \pt)}{\langle T_{\mathrm{AA}} \rangle  \, \mathrm{d}^{2} \sigma_{\mathrm{pp}} / (\mathrm{d} y \mathrm{~d} \pt)},
\label{eq:Raa}
\end{equation}
where $\langle$\TAA$\rangle$ is the average nuclear overlap function
as described in Ref.~\cite{ALICE-PUBLIC-2018-011} and given in Table~\ref{tab:TAA} for the centrality intervals used for the analyses at midrapidity and forward rapidity. The \RAA is evaluated as a function of the average number of participant nucleons, \Npart, corresponding to a given centrality class (as shown in Table~\ref{tab:TAA}), and as a function of the \jpsi \pt. The differential \jpsi production cross section in \pp collisions, $\mathrm{d}^{2} \sigma_{\mathrm{pp}} / (\mathrm{d} y \mathrm{~d} \pt)$, was measured at both midrapidity and forward rapidity as reported in Refs.~\cite{ALICE:2019pid} and~\cite{ALICE:2021qlw}, respectively. At midrapidity, the data sample size does not allow to obtain the cross section for $\pt > 10$~GeV/c and an extrapolation is applied for the last \pt interval (10 $<$ \pt $<$ 15 \GeVc), the result of the extrapolation is 6.82 $\pm$ 0.91 nb. The details of this approach are described in Ref.~\cite{Bossu:2011qe,ALICE:2022zig}, and the corresponding \RAA is shown as the open points in Figs.~\ref{fig:RAA_vs_pt}, ~\ref{fig:RAA_vs_pt_mid_vs_forward_y} and ~\ref{fig:RAA_vs_pt_model}.


The \pt-differential \jpsi yields at midrapidity were studied further by extracting the \meanpt and \meanptsq in fine centrality intervals, as described later in this section. For quantitative comparisons of the \pt distributions, the ratio \rAA of the \jpsi \meanptsq measured in \PbPb collisions to the one obtained in \pp collisions at the same energy is calculated as
\begin{equation}
\rAA=\frac{\meanptsq_{\rm PbPb}}{\meanptsq_{{\rm pp}}}.
\label{eq:little_raa}
\end{equation}

\begin{table}[!htb] 
\begin{center}
\caption{Average nuclear overlap function $\langle$\TAA$\rangle$ and the average number of participants $\langle$\Npart$\rangle$ in \PbPb collisions at $\sqrt{s_{\rm{NN}}}= 5.02 $~TeV for the centrality classes used in the analyses at midrapidity (upper) and forward rapidity (lower).}
\begin{tabular}[t]{c|c|c}
Centrality & $\langle$$T_{\rm{AA}}$$\rangle$ (1/mb) & $\langle$\Npart$\rangle$ \\
\hline
0--5\% &  26.08 $\pm$ 0.18   & 383.40  $\pm$ 0.57 \\
5--10\% &  20.44 $\pm$ 0.17  & 331.20 $\pm$ 1.03 \\
10--20\% &  14.4 $\pm$ 0.13  & 262.00 $\pm$ 1.15 \\
20--30\% &  8.77 $\pm$ 0.10  & 187.90 $\pm$ 1.34  \\
30--40\% &  5.09 $\pm$ 0.08  & 130.80 $\pm$ 1.33\\
40--50\% &  2.75 $\pm$ 0.05  & 87.14 $\pm$  0.93\\
50--70\% &  0.98 $\pm$ 0.02  & 42.65 $\pm$ 0.69 \\
70--90\% &  0.16 $\pm$ 0.004 & 11.34 $\pm$ 0.16 \\
\hline
0--20\% &  18.83 $\pm$ 0.14 & 309.7 $\pm$ 0.89 \\ 
20--40\% & 6.93  $\pm$ 0.09 & 159.4 $\pm$ 1.32\\
40--90\% &  1.00 $\pm$ 0.02 & 39.03 $\pm$ 0.53\\
\hline
\end{tabular}
\label{tab:TAA}
\end{center}
\end{table}

\subsection{ \textbf{J/$\psi$} raw yield extraction}
\label{signal extraction}
The \jpsi mesons are reconstructed employing the \ee decay channel at midrapidity and the \mumu decay channel at forward rapidity. The analysis techniques are discussed in detail in Refs.~\cite{ALICE:2019nrq, ALICE:2015jrl,ALICE:2019lga}. Here, only a brief overview is given and differences with respect to previous analyses are highlighted. 

Electron candidates for the analysis at midrapidity are tracks reconstructed in both the ITS and TPC in the pseudorapidity range \etarange0.9 and with a \pt $>$ 1~\GeVc to suppress combinatorial background.  All tracks are required to have at least one hit in the SPD layers and at least 70 out of a maximum of 159 clusters reconstructed in the TPC. These and other quality criteria that were applied in addition (see Ref.~\cite{ALICE:2019lga}) ensure good tracking resolution and particle identification. They reduce the background electrons from the conversion of photons in the detector material or from long lived weakly-decaying hadrons as well as tracks from pileup collisions, occuring within the readout time of the TPC. 
Electrons and positrons are identified using selections on the specific energy loss, \dEdx, in the TPC gaseous volume. The measured \dEdx is required to be within 3 standard deviations ($\sigma$) relative to the expected electron specific energy loss corresponding to the track momentum, and more than 3.5$\sigma$ different from either the $\pi$ or proton specific energy loss hypotheses. Electrons from photon conversions surviving the track quality criteria are rejected using a technique where candidate electrons are paired with other electrons selected with less strict criteria to enhance the probability of finding the conversion partner, as described in detail in Ref.~\cite{ALICE:2019nrq}.

Muon candidates were selected such that the track pseudorapidity is within the geometrical acceptance of the muon spectrometer, $-4 < \eta < -2.5$, and that the reconstructed track  matches a track segment reconstructed in the trigger system. The transverse position of the tracks at the end of the absorber is required to be $17.6 < R_{\rm abs} < 89.5$~cm in order to reject tracks crossing the thickest part of the absorber. In addition, a selection was applied on the product of the track momentum and the transverse distance to the primary vertex in order to reduce the contamination produced by particles that do not originate from the interaction point.

The number of reconstructed \jpsi mesons, i.e. the raw \jpsi yield, was obtained by constructing the invariant-mass distribution of all the possible opposite-sign dileptons with rapidity selections of $2.5 < y < 4$ for dimuons and $|y| < 0.9$ for dielectrons.
At midrapidity, the signal extraction was performed in two steps. First, the combinatorial background was estimated using an event-mixing technique~\cite{ALICE:2019nrq} and subtracted from the invariant-mass distribution. In the second step, the remaining distribution was fitted with a two-component function, one corresponding to the \jpsi signal and the other to the residual background, which mainly arises from correlated semileptonic decays of heavy-flavour hadrons. The \jpsi signal line shape was obtained from Monte Carlo (MC) simulations of \jpsi mesons decaying in the dielectron channel embedded in simulated \PbPb collisions, as described below, while for the residual background a second-order polynomial function was employed. The raw \jpsi yield was obtained by first counting the dielectron pairs in the mass range $2.92 < \mee < 3.16$~\GeVmass\ in the combinatorial-background subtracted invariant-mass distribution, and then subtracting the residual background based on a two-component fit. Finally, the raw \jpsi-meson yield is corrected for the fraction of \jpsi reconstructed outside of the counting mass interval, as described in more detail in Section~\ref{subs_effAcc}. This procedure is illustrated in the left panels of Fig.~\ref{fig:signal_extraction} for the collision centrality interval 0--5\% and $\pt>0.15$~\GeVc. The upper left panel shows the invariant-mass distributions for the opposite-sign dielectrons constructed from the same event (black) and mixed events (red). The fitting procedure of the combinatorial background subtracted invariant-mass distribution discussed above is illustrated in the lower left panel.



\begin{figure}[!htb]
\begin{center}
  \includegraphics[width = 7.5 cm]{figures/midy/InvMass_midy_0_5.pdf}
  \includegraphics[width = 7.5 cm]{figures/fwdy/InvMass_fordy_10_20.pdf}
\caption{ Upper panels: invariant-mass distribution of opposite-sign lepton pairs from the same event (black points) and mixed events (red histograms) at midrapidity (left) and forward rapidity (right) in \PbPb collisions at \fivenn. Lower panels: invariant-mass distribution after the background subtraction with the event-mixing technique. The fit curves, shown in red, represent the sum of the signal and background shapes and the blue curves correspond to the residual background.}
\label{fig:signal_extraction}
\end{center}
\end{figure}

At forward rapidity, two different methods were used to extract the number of \jpsi counts. In the first method, the invariant-mass distributions were fitted with a sum of a signal and a background function, while in the second method the event-mixing technique 
was employed, as described in Ref.~\cite{ALICE:2015jrl}.
The fit functions corresponding to the signal are either a double-sided Crystal Ball function (CB2) or a pseudo-Gaussian with a mass-dependent width~\cite{ALICE-PUBLIC-2015-006}. In both cases, the \jpsi pole mass and width were free parameters of the fit, while the non-Gaussian tail parameters were fixed. Two sets of tail parameters were obtained, one based on MC simulations
and one extracted from a large data sample of \pp collisions at \s = 13 TeV~\cite{ALICE:2017leg}. The MC simulations were embedded into real MB events in order to properly account for the effect of the detector occupancy. The \psitwos resonance was also included in the fit to the invariant-mass spectrum, using the same signal function as for the \jpsi with mass and width bound to those of the \jpsi~\cite{ALICE:2017leg,ALICE:2022jeh}. 
The background functions employed in the first method were either a variable-width Gaussian~\cite{ALICE-PUBLIC-2015-006} or a ratio of a second order to a third order polynomial. The residual background in the second method was parameterised with a sum of two exponential functions.  
Finally, two invariant-mass ranges were considered for the fit procedure: $2.2 < m_{\mu^{+}\mu^{-}} < 4.5$ and $2.4 < m_{\mu^{+}\mu^{-}} < 4.7$ \GeVmass. 
An example of the signal extraction fit is shown in the right panel of Fig.~\ref{fig:signal_extraction} before (upper plot) and after (lower plot) the subtraction of the combinatorial background estimated with the event-mixing technique.
For each \pt and centrality interval, several fits were performed with the two different approaches, different combinations of signal and background functions, signal tail parameters, and fitting ranges. The number of \jpsi was obtained as the average of the results from the various fitting methods~\cite{ALICE:2021qlw}. 
These various fitting methods are used to determine the systematic uncertainties on the yield extraction as described in Section~\ref{sec:systematics}. 

About 9.0$\times$10$^5$ and 8.2$\times$$10^4$ raw \jpsi counts are measured at forward rapidity and midrapidity, respectively, integrated over all available centrality and \pt intervals. 
\FloatBarrier

\subsection{ \textbf{J/$\psi$} \meanpt and \meanptsq extraction}\label{jpsiMeanpT}
At midrapidity, a quantitative study of the \jpsi \pt spectrum in fine centrality intervals is conducted by extracting the \jpsi mean \pt, $\meanpt_{\jpsi}$, and mean \pt squared, $\meanptsq_{\jpsi}$. For a given centrality interval, these quantities are obtained based on a fit to the mass dependent \meanpt and \meanptsq distributions after efficiency correction, using a function defined as:

\begin{equation}
X (\mee) = f(\mee) \times X_{\jpsi} + (1-f(\mee)) \times X_{\mathrm{bkg}}(\mee)
\label{eq:jpsimeanpt}
\end{equation}

where $X$ stands for either the \meanpt or \meanptsq, and $f$ is the invariant-mass dependent fraction of \jpsi signal determined in the signal-extraction procedure explained above. The invariant-mass dependent background component $X_{\mathrm{bkg(m_{e^{+}e^{-})}}}$ is determined from the event-mixing procedure plus a second order polynomial function for the residual background. Examples of these fits for the \meanpt observable are shown in Fig.~\ref{fig:meanpt}.


\begin{figure}[!htb]
\begin{center}
  \includegraphics[width = 7.5 cm]{figures/meanpt/InvMass_midy_MeanpT_0_5.pdf}
  \includegraphics[width = 7.5 cm]{figures/meanpt/InvMass_midy_MeanpT_70_90.pdf}
\caption{\jpsi \meanpt extraction  in \PbPb collisions at \fivenn at midrapidity for the 0--5\% (left panel) and the 70--90\% (right panel) centrality interval. The data points correspond to opposite-sign \ee pairs from the same event, the blue line to the \ee pairs from mixed events, and the red line is the combined fit that includes the mixed events and residual background which is described by the polynomial function.} 
\label{fig:meanpt}
\end{center}
\end{figure}

\FloatBarrier

\subsection{Acceptance and efficiency correction}\label{subs_effAcc}

The acceptance times reconstruction efficiency factor (\AccEff), which enters into Eq.~\ref{eq:d2ndydpt}, was computed employing both MC and data-driven methods. At midrapidity, this factor includes the kinematic acceptance, track-reconstruction and particle-identification efficiencies, and the fraction of \jpsi with an invariant mass in the signal counting range. With the exception of the particle identification, obtained with a data-driven method, the corrections were obtained using a MC simulation of \jpsi embedded in simulated \PbPb collisions. The \PbPb collisions were generated using the HIJING 1.0 model~\cite{Wang:1991hta}, while the \jpsi were generated using a cocktail of prompt \jpsi with a kinematic distribution tuned to existing measurements and non-prompt \jpsi from beauty hadrons forced to decay into channels containing \jpsi, using PYTHIA 6.4~\cite{pythia}. The \ee decay of the embedded \jpsi was handled using PHOTOS ~\cite{photos}. Both the prompt \jpsi and the beauty hadrons forced to decay into non-prompt \jpsi were assumed to be unpolarised, in agreement with existing measurements, which indicate small or no polarisation~\cite{acharya_measurement_2018,ALICE:2020iev}. All generated particles were transported through the ALICE detector setup using GEANT3~\cite{geant3}, taking into account the time dependence of detector conditions during the 2015 and 2018 data-taking periods. For the determination of the particle-identification efficiency, a clean sample of electrons from photon conversions, passing similar quality selection criteria as primary electrons in the TPC, is used to compute differential maps in pseudorapidity, azimuthal angle $\varphi$ and momentum $p$ for the single-electron selection efficiency. These were then propagated to the \jpsi dielectron pairs using the phase-space distribution of the \jpsi decay simulated in the above mentioned MC simulations.
The total ($A\times \varepsilon$) for the \pt-integrated \jpsi yields is about 6.5\% in the 0--10\% centrality interval, and it slightly increases towards more peripheral collisions. As a function of \pt, the ($A \times \varepsilon$) has a non-monotonic behaviour, with a minimum value of 5.6\% around \pt=~2~\GeVc, and a maximum of about 9\% towards zero and high \pt.

At forward rapidity, the acceptance and reconstruction efficiency values were determined using simulated \jpsi mesons forced to decay via the dimuon channel, embedded into real events. The \jpsi \pt- and \y-differential distributions used in the simulation were adjusted to measurements via an iterative procedure, and separately for all centrality intervals employed in this analysis. 
The \jpsi were assumed to be unpolarised, in agreement with the small polarisation, compatible with zero, measured in \PbPb collisions for $2<\pt<10$~\GeVc~\cite{ALICE:2020iev}. As in the analysis at midrapidity, the simulations take into account the time dependence of the detector conditions, such as the status of the tracking chambers and the residual detector element misalignment. The trigger-chamber efficiency was determined from data and used as input in the simulations. The (\AccEff) reaches a minimum of 11\% at \pt $\approx$~2~GeV/c and increases up to 13\% at low \pt and up to 46\% at high \pt in the 0--20\% centrality interval. It increases towards peripheral collisions by a few percent~\cite{Abelev:2014ffa}. 


\FloatBarrier

\subsection{Systematic uncertainties}
\label{sec:systematics}
The considered sources of systematic uncertainty for the analysis at midrapidity include central-barrel tracking, electron identification, signal extraction, and the kinematics of the \jpsi injected in the MC simulations. For the analysis at forward rapidity, the main systematic uncertainties originate from the signal extraction, the muon tracking and trigger efficiencies, and the kinematics of the \jpsi used in the embedded MC simulations. In addition, for the $\RAA$, uncertainties on the pp reference and the nuclear overlap function are included in both analyses ~\cite{ALICE-PUBLIC-2018-011}. The uncertainty on the \jpsi decay branching ratio and the evaluated systematic uncertainties are summarised in Table~\ref{table:systematics} and Table~\ref{tab:muon_syst} for the analysis at midrapidity and forward rapidity, respectively. 

At midrapidity, the tracking uncertainty is the largest source of systematic uncertainty and it is dominated by the ITS--TPC track matching. This was determined based on the difference in the matching efficiency observed for single tracks between data and MC simulations. The systematic uncertainty due to the electron identification takes into account the residual miscalibration of the TPC particle identification (PID) response and also the statistical uncertainty of the clean electron sample used to compute the identification efficiency. For its estimation, the PID selection criteria (both electron inclusion and hadron rejection) are varied, each time obtaining a new set of raw yields and corresponding PID efficiencies. The assigned systematic uncertainty is taken as the standard deviation of the distribution of the corrected results obtained in this procedure. The systematic uncertainty of the signal extraction includes one component from the \jpsi signal shape obtained from MC simulations and one component related to the description of the dielectron background. The former was determined by varying the mass range in which the signal is counted and recomputing each time the corresponding signal fraction correction, while the latter was determined by repeating the fit to the invariant-mass distribution in different mass ranges. The standard deviation of the corrected yield distribution was then taken as the systematic uncertainty.
Since the \jpsi efficiency depends on \pt, the average efficiency computed over wide \pt intervals depends in turn on the underlying \jpsi \pt distribution used in the MC simulations. The corresponding uncertainty was minimised by iteratively tuning the injected \pt spectrum to match the corrected spectrum measured in this analysis. A systematic uncertainty which takes into account possible variations of the \pt spectrum, statistically compatible with the finally measured \pt spectrum, was assigned and it is typically below 1\%. The total systematic uncertainty on the \jpsi corrected yield varies in the range 6--10\% in different \pt and centrality intervals. The systematic uncertainties, which are dominated by the tracking uncertainties, are correlated over centrality and \pt to a very large extent. 
The systematic uncertainties for \meanpt and \meanptsq are evaluated via similar procedures as for the \pt integrated yields. The uncertainties from signal extraction and track selection criteria range, respectively, from 0.2 to 1.2\% and from 0.5 to 1.3\%. The electron identification and ITS--TPC matching systematic uncertainties are calculated by propagating the \pt-differential systematic uncertainties to the \meanpt based on the measured \pt-differential spectrum.

In the analysis at forward rapidity, the systematic uncertainty corresponding to the signal extraction was determined using several variations of the fit to the invariant-mass spectra, including the fit method, the signal and background functions, and the fitted mass range. 
This uncertainty varies in the range 1.5--10.7\% depending on the \pt interval and centrality class. 
The systematic uncertainty on (\AccEff) depends on the uncertainty on the \pt and \rapidity distributions of the simulated \jpsi, and on the tracking, trigger, and matching efficiency. The first two were evaluated by varying the \pt and \rapidity spectrum for each centrality interval, taking into account the correlations between the \pt and \rapidity distributions. The systematic uncertainty was estimated as the largest difference between the nominal (\AccEff) and the one estimated from the variations. It ranges between 0.2\% and 4.1\%. The systematic uncertainty on the muon tracking efficiency was estimated based on the difference between the single-muon tracking efficiency obtained in data and MC with a method that uses the redundancy of the tracking information in each station. The corresponding uncertainty for dimuons was evaluated to be 3\% and constant over \pt. An additional systematic uncertainty
is ascribed to the loss of tracking efficiency due to occupancy effects in the most central events and was estimated to range between 0.5 and 1\%, increasing towards more central events. 
The systematic uncertainty on the trigger efficiency has two components, one due to the intrinsic efficiency of the trigger chambers and another one due to the trigger response. The first component was estimated from the uncertainties on the single-muon trigger efficiency measured from data and used in the simulations.  
The second component was evaluated by comparing the \pt dependence of the trigger response function of the single muon between data and MC. 
The two sources were added in quadrature and the obtained uncertainty ranges between 1.5 and 2\%. Finally, a 1\% systematic uncertainty is assigned, related to the choice of the $\chi^2$ selection used to define the matching between the tracks reconstructed in the tracking system and the track segments reconstructed in the trigger system.
The uncertainty on $F_{\mathrm{norm}}$ was estimated by using two methods. First, the opposite-sign dimuon trigger condition was applied when analysing recorded MB events and, second, the counting rate of the dimuon and MB triggers were compared. The estimated uncertainty on $F_{\mathrm{norm}}$ was obtained by comparing the two methods and it amounts to 0.7\%. 

The systematic uncertainties on $\langle$\TAA$\rangle$ were obtained as described in Ref.~\cite{ALICE-PUBLIC-2018-011} and the values are listed in Table~\ref{tab:TAA} for both analyses, at midrapidity and forward rapidity. 
The systematic uncertainty on the definition of the centrality interval was estimated using variations of $\pm 1\%$ of the V0 signal amplitude corresponding to 90\% of the hadronic Pb--Pb cross section and redefining correspondingly the centrality intervals. The systematic uncertainty of the centrality limit depends on the width of the centrality classes and it ranges from 1 to 6\% and from 0 to 2.8\%, as shown in Tables ~\ref{table:systematics} and ~\ref{tab:muon_syst} for the analyses at midrapidity and forward rapidity, respectively. 

The systematic uncertainties on the \jpsi reference cross section in pp collisions at \five as obtained in Refs.~\cite{Acharya:2019lkw, ALICE:2021qlw} are provided in Table~\ref{table:systematics} and Table~\ref{tab:muon_syst}. 
The correlations of the systematic uncertainties over centrality and \pt depend on the mid and forward rapidity analysis and they are indicated in Table~\ref{table:systematics} and ~\ref{tab:muon_syst}.


    \begin{table}[!htb]
    	\begin{center}
    	\caption{Systematic uncertainties on the $\pt$-integrated ($0.15<p_\textrm{T}<15~ \textrm{GeV/}c$) measurement at midrapidity for different centrality intervals. The individual contributions and the total uncertainties are given in percentage. It is considered that all the uncertainties are correlated over centrality and \pt to a very large extent.}
    	\label{table:systematics}
    	\begin{tabular}{c | cccccccc} 
    		\hline
    		Centrality (\%)&0--5 &5--10 &10--20 &20--30 &30--40& 40--50 &50--70 &70--90\\
    		\hline
    		Signal extraction & 2.5 & 2.7 & 1.9 & 6.6 & 2.5 & 1.0 & 1.8 & 2.5\\
            MC input & 0.5 & 0.5 & 0.5 & 0.5  & 0.5  & 0.5 & 0.5 & 0.5 \\
    		Tracking & 10.1 & 10.1 & 8.5 & 8.5 & 8.0 & 8.0 & 7.9 & 7.9\\
    		PID & 1.6 & 1.3 & 1.4 & 1.4 & 1.1 & 1.2 & 1.1 & 1.3\\
    		Centrality limit & 1.0 & 1.0 & 1.0 & 1.0  & 1.0  & 1.0 & 5.7 & 6.0 \\
    		\hline
    		Total & 10.6 & 10.5 & 8.8 & 10.8 & 9.0 & 8.2 & 10.0 & 10.3\\ 
    		\hline
    		\TAA (only on \RAA) & 0.7 & 0.8 & 0.9 & 1.2  & 1.6  & 1.7 & 2.0 & 2.3 \\
    		pp reference (only on \RAA) & \multicolumn{8}{c}{5.8} \\
    		
    		Branching ratio (only on yield) &  \multicolumn{8}{c}{0.5} \\
            \hline
    	\end{tabular}
    	\end{center}
    \end{table}

    
    \begin{table}[htbp] 
    \caption{
    Systematic uncertainties on the $\pt$-differential measurement at forward rapidity for various centrality intervals. The individual contributions are given in percentage. When a range is given, it  corresponds to the minimum and maximum values obtained in the \pt interval. Values marked with an asterisk correspond to the uncertainties correlated over \pt. These uncertainties are considered as global ones for \RAA and added in quadrature to the uncorrelated uncertainties for the yields.}
\begin{center}
\vspace{1ex}
\begin{tabular}[t]{c|ccc}
\hline
Centrality (\%) & 0--20\% & 20--40\% & 40--90\% \\  
\hline
$F_{\rm{norm}}$ & \multicolumn{3}{c}{0.7*} \\ 
\hline 
Signal extraction & 1.5--5.8 & 1.9--4.5 & 1.6--10.7\\
MC input & 1.8--4.1 & 0.2--2.1 & 0.9--1.8\\
Tracking efficiency & 3.0 + 1.0* & 3.0 + 0.5* & 3.0\\
Trigger efficiency & 1.5--2.0 + 1.0* & 1.5--2.0 + 0.5* & 1.5--2.0\\
Matching efficiency & 1.0 & 1.0 & 1.0 \\ 
Centrality limit & -- & 0.8* & 2.8*\\
\hline
$T_{\rm{AA}}$ (only on \RAA) & 0.8* & 1.3* & 2.0*\\
pp reference (only on \RAA) & \multicolumn{3}{c}{3.5--5.6 + 1.9*}\\
Branching ratio (only on yield)&  \multicolumn{3}{c}{0.5*} \\
\hline 

\end{tabular}
\end{center}
\label{tab:muon_syst}
\end{table}

\FloatBarrier
