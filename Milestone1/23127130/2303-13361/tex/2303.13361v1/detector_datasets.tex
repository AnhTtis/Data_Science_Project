\section{Apparatus and data sample}
\label{detectors}  


A complete description of the ALICE apparatus and its performance is given in Refs.~\cite{Aamodt:2008zz, Abelev:2014ffa}. The central barrel detectors, for the dielectron analysis, and the muon spectrometer, for the dimuon analysis, covering midrapidity and forward rapidity, respectively, were used in the analyses reported in this paper. 

At midrapidity, the main detectors employed in the analysis are the Time Projection Chamber (TPC)~\cite{TPC} and the Inner Tracking System (ITS)~\cite{ITS}, both immersed in a uniform magnetic field of 0.5~T provided by a solenoid magnet. 
The TPC is used for tracking and particle identification. It covers the pseudorapidity range \etarange0.9 for tracks with full radial length and has full coverage in azimuth. It provides excellent momentum resolution and electron--hadron separation in a wide range of track transverse momentum.
The ITS is a cylindrical six-layer silicon detector, with the innermost layer located at 3.9 cm from the beam pipe, providing additional space points for tracking that enhance the spatial resolution in the reconstruction of primary and secondary vertices. 

 At forward rapidity, the muon spectometer~\cite{CERN-LHCC-99-022, CERN-LHCC-2000-046} detects muons in the range $-4 < \eta < -2.5$. It consists of a 3 Tm dipole associated with a tracking and a trigger system. A front absorber with a thickness of 10 interaction lengths is placed before the tracking system in order to filter out hadrons produced in the interaction. The tracking system consists of five tracking stations, each one made of two planes of cathode pad chambers. An iron wall with 7.2 interaction length thickness is located between the tracking and trigger stations in order to stop secondary hadrons escaping the front absorber and low momentum muons produced predominantly from $\pi$ and $K$ decays. The trigger system consists of two stations, each one made of two planes of resistive plate chambers. Finally, a conical absorber around the beam pipe protects the spectrometer against secondary particles produced by the interaction of primary particles with large pseudorapidity at the beam pipe. In the analysis at forward rapidity, the determination of the primary vertex of the collision is provided by the Silicon Pixel Detector (SPD) that constitutes the two innermost layers of the ITS.

Both analyses, at midrapidity and forward rapidity, use the V0~\cite{ALICE:2013axi} and Zero Degree Calorimeter (ZDC)~\cite{ALICE:2012aa} detectors. The V0 detector consists of two scintillator detector arrays and covers the full azimuth in the pseudorapidity regions $-3.7 < \eta < -1.7$ and 2.8 $< \eta < $ 5.1, respectively. It is used for triggering, beam--gas background rejection, and characterisation of the event centrality. The ZDC detectors are located at a distance of 112.5 m on both sides of the interaction region along the beam direction, and they detect spectator nucleons emitted at zero degree with respect to the LHC beam axis. They are used to reject electromagnetic \PbPb interactions. 

The trigger for minimum-bias (MB) events was provided by the coincidence of signals in the two scintillator arrays of the V0 detector. The dimuon analysis relies on a dimuon trigger which requires, in addition to the MB trigger, the detection of two opposite-sign tracks in the muon trigger system. The muon trigger selects muon candidates with a \pt larger than a threshold of $\sim 1$~\GeVc. The trigger efficiency reaches 50\% at this threshold value and a plateau value of 98\% at $\pt \sim 2.5$~\GeVc~\cite{Bossu:2012jt}. 


The results presented in this article are based on the data sample collected by ALICE from Pb--Pb collisions at \fivenn in 2015 and 2018 during Run 2 at the LHC. During the 2018 data taking, the \PbPb dataset for the central barrel was enhanced with central (0--10\%) and semicentral (30--50\%) events. In total, the integrated luminosity corresponding to the analysed data sample was about~105 $\rm\mu \rm b^{-1}$ and~51 $\rm\mu \rm b^{-1}$ for the central and semicentral events, respectively. For the other centrality intervals, the integrated luminosity of the data sample was~22 $\rm\mu \rm b^{-1}$. For the analysis at forward rapidity, the dimuon triggered sample corresponds to an integrated luminosity of~756~$\rm\mu b^{-1}$. 
At midrapidity, triggered events containing collisions that overlap within a time window smaller than the readout time of the TPC were removed to preserve a uniform particle identification performance of the TPC, which is sensitive to the total charge produced by the ionising tracks in the sensitive volume. Only events with the primary vertex, reconstructed within $\pm$10~cm from the nominal interaction point in the beam direction, were considered for further analysis at midrapidity. In the forward analysis, there was no selection on the primary vertex. 






 

 
 
 
 


