\section{Introduction}

Heart rate is one of the most important vital signs. Photoplethysmography (PPG)~\cite{Allen2007} is an optical technique to detect light changes caused by blood flow to noninvasively estimate heart rate. Traditional PPG employs a sensor that transmits and measures reflected light of specific frequencies on vascularized skin of the fingertip, ears, or forehead. These sensors are affixed to the body, which can be inconvenient and can present risks to neonates, the elderly, and patients with damaged skin~\cite{Zhan2020-hm, informatics9030057, 6226654, DeHaan2013}. These concerns, plus increased interest in image-based noncontact measurement of vital signs, have yielded increasing interest in remote photoplethysmography (rPPG), which can estimate blood flow changes at a distance, including such exotic applications as measurement from UAV-mounted cameras~\cite{Huang_ICCRE_2020}.

The current major databases conducive to rPPG are mainly focused on facial skin pixels. MAHNOB-HCI~\cite{5975141} and PURE~\cite{Stricker2014} contain face videos with limited or controlled head movements. MMSE-HR~\cite{Zhang_2016_CVPR} collected face videos after inducing emotional responses. UBFC-rPPG~\cite{Bobbia2019} contains occasional movements and subjects under stress from mathematical games. DDPM~\cite{Speth_IJCB_2021, Vance2022} was the first long-form dataset containing unconstrained facial movements. Synthetic face rPPG datasets containing avatars also only contain faces~\cite{mcduff2022scamps,Kadambi2022}. As rPPG becomes more common for ubiquitous health monitoring, it is important to understand the limitations presented by partial or total occlusion of the face, as well as the corresponding opportunities when skin regions {\em not} on the face are visible. 

\begin{figure*}
    \centering
    \includegraphics[width=\linewidth]{figures/teaser/cvpm-teaser-v4.pdf}
    \caption{This paper explored the (a) Multi-Site Physiological Monitoring (MSPM) dataset with {\bf contact} measurements from ten PPG sensors, including a fingertip oximeter, as shown on the left, with (b) {\bf contactless} (camera-based) measurements of multiple skin areas carried out by three state-of-the-art rPPG methods (CHROM~\cite{DeHaan2013}, POS~\cite{Wang2017}, and RPNet~\cite{Speth_CVIU_2021}) as shown on the right. In this paper we also compared de-noised and combined PPG signals with the corresponding rPPG estimates.}
    \label{fig:teaser}
\end{figure*}

This paper explores multi-site rPPG, where the blood volume pulse is simultaneously estimated from the face, arms, hands, and legs. From a physiological perspective, collecting rPPG signals from different skin regions could provide a richer map of blood transport through the skin's capillary networks, rather than a scalar measurement of periodic blood volume changes at a single location~\cite{Allen2007, Mukkamala2015, Jeong2016, Bentham2018-fd, Chan2019-mi, Block2020, Iuchi_2022_CVPR, Iuchi_BOE_2022}. Figure~\ref{fig:teaser} displays the experimental process presented in this paper. Our main contributions are summarized as follows:

\begin{enumerate}[label=(\alph*)]
    \item The newly-created Multi-Site Physiological Monitoring (MSPM) dataset collected from eighty-seven subjects. This dataset is, to our knowledge, the first dataset that allows simultaneous rPPG estimation from multiple body sites including arms, legs, and hands in addition to the face (see Sec. \ref{sec:dataset}); 
    \item A comprehensive evaluation of baseline rPPG methods applied to video from multiple skin regions of the body. We used state-of-the-art rPPG methods including CHROM~\cite{DeHaan2013}, POS\cite{Wang2017}, and RPNet~\cite{Speth_CVIU_2021} to retrieve signals from each body part, as well as sub-regions of each body part, to explore the overall performance and motivate future work based on these results (see Sec. \ref{sec:approach}, \ref{sec:local} and \ref{sec:global}).
    \item A pulse transit time (PTT) analysis from both contact PPG and rPPG measurements at different sites on the body. We showed that pulse transit time can be estimated from a camera with lower frame rate than previous studies~\cite{Jeong2016, Iuchi_BOE_2022} if the measurement sites on the body are further apart (see Sec. \ref{sec:ptt}).
\end{enumerate}

