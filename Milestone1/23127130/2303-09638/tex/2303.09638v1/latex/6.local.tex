\section{Local rPPG Experiments}
\label{sec:local}

\subsection{Local Evaluation}
In addition to analyzing performance for the entire face, arm, and legs, we performed a ``local" analysis on subregions of pixels in the native video resolution. This gives a more detailed evaluation on rPPG quality over the visible skin on the body. We selected POS~\cite{Wang2017} as the rPPG method, since it was the best performing for each of the global regions.

We used the ROIs described in section~\ref{sec:ROI} to define bounds for a non-overlapping sliding window of 20x20 pixel subregions. The POS algorithm was applied on 10-second non-overlapping time windows of the spatially averaged subregions for the whole video. The predictions were then linearly interpolated up to the native image resolution (i.e. 20 times as large along the x- and y-axes). Next, we computed error metrics at each pixel location for the 10-second windows, which results in 9 error frames for the hand raising portion, and around 25 error frames for the sitting portion per subject.

For error metrics we used MAE between the predicted and ground truth pulse rate and signal-to-noise ratio (SNR). The SNR was calculated similarly to \cite{DeHaan2013,Nowara_BOE_2021}, where the sum of spectral power around the signal bands in the first and second harmonics ($\pm 6$ bpm) was divided by the total power outside the signal bands.

To remove spurious local errors outside of the skin pixel regions, we applied the average SCHP mask (see frame (b) in Fig.~\ref{fig:frames}) for the time window to the error frames. Since subjects sat in slightly different positions throughout the interview, we aligned each body part across subjects for an accurate physiological error map. To do this we used the pose keypoints from Mediapipe (see frame (d) in Fig.~\ref{fig:frames}). We first calculated the average pose across all subjects for the hand raising and relaxed portions as the target poses. Then for each error frame for every subject, we applied a homographic transformation from the time window's pose to the hand raising or relaxed portion's target pose.

\begin{figure*}
    \centering
    \includegraphics[width=0.85\linewidth]{figures/heatmaps/heatmaps_pagewidth_dehaan_harm.pdf}
    \caption{A heatmap of the local POS signal quality across the visible skin. POS signals are predicted over 20x20 pixel subregions, masked with SCHP~\cite{li2020self}, and the human pose is aligned with a homographic transformation.}
    \label{fig:heatmap}
\end{figure*}

\subsection{Local Results}
Figure~\ref{fig:heatmap} shows a heatmap of local pulse estimation performance for the POS algorithm. The hand raising and sitting portions are visualized separately, since the body positions were much different. In general, we found the face to hold the strongest rPPG signal. In the hand raising heatmap, we can see that the palm gave relatively low errors compared with the arms and legs. This supports the hypothesis that signal quality is higher on glabrous than non-glabrous skin~\cite{Cao2023}, and shows promise for substituting the face region with the hand if necessary.

It is useful to analyze each body part separately to assess the signal quality. Firstly, the face appears to give low MAE and high SNR in all regions except the eyes and mouth. This is in line with past research that mainly utilizes the forehead and cheek regions. The arms give relatively low signal quality, but there is a slight improvement visible above the forearm over other regions in the SNR plots. The legs give perhaps the weakest rPPG signal as evidenced by the global results in Table \ref{tab:within_dataset}. Within the local analysis, we can see that the thighs give higher SNR than the shins. This could be due to the thigh's closer proximity to the illuminators, whereas the shins are visibly darker in the video.

In both the hand raising and relaxed portions, the hand gives the second best SNR. In nearly all cases the subject's right hand was resting with the palm facing downwards, indicating that the back of the fingers and hand on non-glabrous skin is still feasible for rPPG. During the hand raising, we see that the signal quality is high on the glabrous skin of the palm. With more sophisticated methods for aligning each subject's palms, we believe the error maps would give even higher signal quality.