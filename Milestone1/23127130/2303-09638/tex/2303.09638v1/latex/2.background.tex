\section{Related Work}

\subsection{Camera-based Measurements}
rPPG methods estimate pulse rate based on skin pixel color changes from blood volume fluctuations. Several rPPG algorithms have been proposed based on different mechanisms, including Blind source separation (BSS)~\cite{Wedekind_2017} and independent component analysis (ICA)~\cite{Poh2010, Poh2011}, which estimates pulse rate by applying different criteria to separate temporal RGB traces into uncorrelated or independent signal sources~\cite{Wang2017}. CHROM~\cite{DeHaan2013} assumes a standardized skin tone under white light and linearly combines the color signals for heart rate estimation. Spatial subspace rotation (2SR)~\cite{Wang2016} utilizes both spatial subspace and temporal rotation angle to calculate heart rate. POS~\cite{Wang2017} applies a plane orthogonal to the skin tone in the temporally normalized RGB space for pulse extraction.

More recently, deep learning models have been developed as the size of rPPG databases has increased. The first deep learning method~\cite{Hsu2014} used a support vector regression model on both ICA and chrominance features. Convolutional neural networks (CNN) have successfully been applied to frame differences with several adaptations~\cite{Chen2018, Liu_MTTS_2020, Liu_2023_WACV}. End-to-end waveform estimation from cropped videos passed through 3D-CNNs was first presented with PhysNet~\cite{Yu2019} and later improved with internal dilated convolutions to improve temporal context~\cite{Speth_CVIU_2021}. Several unsupervised approaches have been introduced to reduce the need for simultaneous PPG ground truth. Most approaches leverage contrastive training strategies, where similar pairs of samples are pulled closer and different samples are repelled~\cite{Gideon_2021_ICCV, Wang_SSL_2022, Yuzhe_SimPer_2022, Sun_2022_ECCV}. The first non-contrastive approach leverages strong periodic priors to encourage the model to predict sparse signals in the frequency domain~\cite{speth2023sinc}.



\subsection{Multiple Contact Measurements}
Measuring contact PPG from multiple locations simultaneously has garnered research interest primarily for blood pressure estimation from features related to pulse transit time (PTT)~\cite{Allen2007,Mukkamala2015,RibasRipoll2019,Block2020,Lubin2021,Natarajan2022}. Unfortunately, such systems are cumbersome, since multiple sensors must be properly attached and synchronized. It is therefore desirable to gather the same physiological information from a single sensor without contact, thus providing additional motivation for this work. Another advantage is that the number of potential transit time differences is drastically increased when using a camera sensor over contact PPG.

\begin{figure*}
        \centering
        \includegraphics[width=0.85\linewidth]{figures/concatenated_figures/concatenated_4.pdf}
        \caption{The subjects remained seated while collecting data (a). Using a trained SCHP~\cite{li2020self} model on the LIP~\cite{https://doi.org/10.48550/arxiv.1804.01984} dataset, the human body was segmented into 20 classes (b). rPPG Signals were extracted separately from 5 classes including 'Face', 'Left-arm', 'Right-arm', 'Left-leg', 'Right-leg'. We used Mediapipe Hand detection~\cite{Zhang2020MPHands} to detect 4 points (shown in blue) of palm and use them to form a bounding box (shown in green) (c). Mediapipe Pose detection method~\cite{Bazarevsky2020} was applied and thirty-three keypoints of human body were generated (d).}
        \label{fig:frames}
\end{figure*}

\subsection{Non-Facial rPPG}
Most of the existing research utilizes the diffuse reflection from the face region to estimate the rPPG signal, with the exception of a few works analyzing the hand~\cite{Kamshilin2011, Teplov2014, Saiko2021, Cao2023}, arms~\cite{Zaytsev2018}, and thigh~\cite{Harford2022}. A gap in current rPPG research is the capture and estimation at multiple parts of the body simultaneously. Understanding rPPG signal quality across the entire body could allow for fewer constraints on the user and create new biomarkers as blood volume dynamics are measured in the peripheral vasculature.
