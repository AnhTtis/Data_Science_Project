\section{Global rPPG Experiments}
\label{sec:global}
\subsection{Global Evaluation}
We refer to rPPG with all pixels in a region of interest as ``global" rPPG. We evaluated global rPPG performance over the face, arms, and legs with the color transformation approaches (CHROM~\cite{DeHaan2013} and POS~\cite{Wang2017}), and a 3D-CNN approach (RPNet~\cite{Speth_CVIU_2021}). We compared the rPPG quality by evaluating the pulse rate performance. We applied the same method for pulse rate estimation on both the ground truth signals and the rPPG signals for fair evaluation~\cite{Mironenko2020}. 
We used popular error metrics to compare the pulse rate estimates, including mean absolute error (MAE) and Pearson's correlation coefficient ($r$).

\subsection{Global Results}

The results of global rPPG experiments with different body parts are shown in Table~\ref{tab:within_dataset}. We show results for the entire sequence as well as for separate components in which the left arm is either relaxed or raised with left palm frontally presented. We observed the best MAE and $r$ for the face region. CHROM and RPNet give similar performance on the face region, with RPNet giving slightly better performance. RPNet performed poorly on the non-face regions, however, likely indicating that the model has learned spatial priors to look for facial features. Additionally, the deep learning model may be overfit to the skin thickness, melanin concentration, and microcirculation present in the glabrous skin of the face. The improved performance for RPNet on the palm (also with glabrous skin) helps justify this explanation.

The POS algorithm gives the best performance for all body regions in terms of both MAE and $r$. This is especially impressive given that it is a simple linear method. It also shows that color changes from blood volume are similar over different skin thicknesses and underlying microvasculature. For POS and CHROM, the order of performance from best to worst is face, arms, then legs. The palm also gives good performance for all approaches, due to physical similarities to the skin on the face.

Figure \ref{fig:result_waveforms} shows predicted waveforms from a 10-second window for the same subject at different locations. The nearest contact PPG waveforms are bandpass filtered and plotted against the predictions to show that many of the predictions contain the underlying dominant pulse even in the presence of higher frequency noise. The difference in waveform morphology across body location shows how informative full-body rPPG can be. For this particular segment, the legs contain a strong second harmonic, which may arise from either the closure of the aortic valve during forward wave propagation, or wave reflections occurring at structural discontinuities along the femoral arteries~\cite{London1999}. Future studies on this dataset will explore different waveform morphologies at a finer scale and their relation to arterial stiffness and blood pressure.

Figure~\ref{fig:result_HRs} shows pulse rates calculated from POS predictions over the entire session for 3 subjects. Interestingly, the errors generally occur as spikes of short duration rather than sustained periods of large offset. Even for the left arm, which is the worst-performing region for subject 0, the errors appear to be due to short periods where the second harmonic contains more power than the first harmonic. It is possible that simple heuristics during postprocessing could remove these transient errors. In general, the global POS signals give meaningful predictions for most applications, even in non-face regions.

\begin{figure}
    \centering
    \includegraphics[width=\linewidth]{figures/results/wave.pdf}
    \caption{Pulse waveform predictions from POS overlayed with the nearest contact-PPG waveforms for a 10 second window.}
    \label{fig:result_waveforms}
\end{figure}

\begin{figure}
    \centering
    \includegraphics[width=\linewidth]{figures/results/HR_3.pdf}
    \caption{Pulse rate predictions from POS for 3 subjects over the course of a full session compared to the ground truth.}
    \label{fig:result_HRs}
\end{figure}
