\section{Multi-Site Physiological Monitoring Dataset}
\label{sec:dataset}

We collected a large dataset of subjects seated in front of a video camera with several attached contact PPG sensors. Subjects were instructed to avoid wearing clothes that obstructed the arms and legs. A sample video frame is shown in Fig.~\ref{fig:teaser}, where the subject's face, arms, and legs are visible simultaneously. To the authors' knowledge, this is the first dataset that allows for simultaneous ground truthed rPPG at more than two sites on the body.

The collection began with the subject holding their left hand upright facing the camera for 90 seconds. This allows for evaluation of rPPG on the palm, a region with glabrous skin that is valuable for remote pulse estimation~\cite{Cao2023}. For the remainder of the session, subjects underwent a blood pressure measurement, guided breathing exercise, and relaxation period. On average, the sessions evaluated in this paper last for approximately 5.7 minutes.

\subsection{Apparatus}
Video of the full subject was recorded with a DFK 33UX290 RGB camera from The Imaging Source (TIS). Raw video was recorded at 90 FPS with a resolution of 1920 $\times$ 1080 pixels. To record blood oxygenation (SpO2) and pulse rate, we collected 60 samples per second from an FDA-certified Contec CMS50EA pulse oximeter attached to an index finger. To support research in pulse transit time, we attached nine MAX30101 contact PPG sensors that recorded red and near-infrared signals at 400 samples per second. Each leg and arm had two attached sensors via elastic straps, and the last sensor was attached to the back of the neck with medical tape. All sensors and the camera recorded raw data to a single collection machine with the associated timestamps for easy synchronization.

\subsection{Data Preprocessing}
To properly evaluate rPPG approaches, we created a global pulse rate estimate from the multiple contact sensors. Contact PPG signals recorded from the nine sites with the MAX30101 sensors contain varying levels of noise throughout the session depending on body movements. However, the pulse signal is likely present in at least one signal at any given point in time, since movements may be isolated to local regions.

Using this assumption, we combined the multiple pulse signals using a sliding window approach and bandpass filtering. We relied on the FDA-certified CMS50EA oximeter's pulse rate estimate to define the bounds of a narrow bandpass filter for the MAX30101 signals. Given the estimated pulse rate from the fingertip oximeter $Y$, we specified a padding around this value, $\Delta Y = 30$ bpm, and filtered the MAX30101 signals with a 2nd order Butterworth filter with lower and upper cutoffs of $Y - \Delta Y$ and $Y + \Delta Y$, respectively.

\begin{figure}
    \centering
    \includegraphics[width=\linewidth]{figures/ground_truth_preprocessing.pdf}
    \caption{Processing of the contact PPG waveforms to produce a robust pulse rate. (a) The z-normalized waveforms from all sensors. (b) Bandpass filtered signals around the fingertip oximeter's pulse rate $\pm$ 30 bpm. (c) The signals were added together and the envelope is calculated. (d) The combined signal was divided by the envelope.}
    \label{fig:ground_truth}
\end{figure}

Specifically, for a sliding 10-second window with a stride of a single sample at 400 Hz, the waveforms underwent z-normalization, followed by filtering around the fingertip oximeter's pulse rate, then they were summed together into the combined waveform for that window. Finally, we divided the complete combined waveform by its envelope, which was calculated from the Hilbert transform. The process for signal combination is shown in Fig.~\ref{fig:ground_truth}. In the figure, between seconds 2 and 5, there appears to be noise from motion in multiple sensors, but the combined signal in subplot (d) remains clean.
