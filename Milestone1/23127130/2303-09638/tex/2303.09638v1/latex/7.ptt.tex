\section{Pulse Transit Time Experiments}
\label{sec:ptt}

\begin{figure*}
        \centering
        \begin{minipage}[t]{0.37\linewidth}
        \begin{subfigure}{\linewidth}
        \includegraphics[width=\linewidth]{figures/PTT_subfig.pdf}
        \caption{}
        \label{subfig:ptts}
        \end{subfigure}
        \end{minipage}
        \hfil
        \begin{minipage}[t]{0.25\linewidth}
        \begin{subfigure}{\linewidth}
        \includegraphics[width=\linewidth]{figures/PTT_POS_subfig.pdf}
        \vspace{7.27em}
        \caption{}
        \label{subfig:rptts}
        \end{subfigure}
        \end{minipage}
        \hfil
        \begin{minipage}[t]{0.37\linewidth}
        \begin{subfigure}{\linewidth}
        \includegraphics[width=\linewidth]{figures/PTT_boxplots_horizontal_30.pdf}
        \caption{}
        \label{subfig:ptt_boxplots}
        \end{subfigure}
        \end{minipage}
    \caption{Pulse transit time from both contact PPG and rPPG calculated by a sliding cross-correlation between pairs of waveforms. (a) Time lags between the MAX30101 contact PPG sensors. (b) Time lags between POS waveform predictions on the five different body regions. (c) Grouped boxplots comparing the pulse transit time from PPG and rPPG. For the PPG measurements on the arms and legs we only used the forearm and knee sensors, respectively. The PTT is calculated as $X-Y$ given a label of measurement sites ``$X$ to $Y$".}
    \label{fig:ptts}
\end{figure*}

\subsection{Pulse Transit Time for Contact PPG Sensors}
Contact PPG measurements at different sites on the body contain small phase offsets due to different proximities to the heart. These offsets are sometimes referred to as differential pulse transit times (dPTT)~\cite{Block2020} and are negatively correlated with blood pressure. Throughout the paper we treat PTT and dPTT interchangeably. When fusing the PPG measurements at multiple sites into a single ground truth, we summed the waveforms without applying a phase shift. To justify this simplification, we calculated the differential pulse transit time between all pairs of sensors. Fig. \ref{subfig:ptts} shows a heatmap of the phase offsets between all pairs of PPG signals in milliseconds.

The phase differences were calculated with a sliding cross-correlation. We used a window size of 5 seconds (2,000 points) with a stride of 10 milliseconds (4 points), and a maximum lag 300 milliseconds, which is much higher than a typical transit time~\cite{Block2020}.
The sliding cross-correlation approach for PTT analysis is simple, but could be improved in future work by using foot-finding methods that measure time differences at the diastolic foot~\cite{Mukkamala2015}. 
The index with the maximum correlation was selected as the phase shift for each window. All pairs of transit times formed a skew-symmetric matrix, where $A^T = -A$.

In general, the phase offset between different sites is quite small relative to the pulse rate frequency. For example, the largest average phase offset of 51 milliseconds occurs between the left tricep and the right ankle. Given even a high pulse rate of 180 bpm, the phase angle for such a shift is only 27.66 degrees. Since most phase offsets are lower, there is very little risk of interference when summing the waveforms. The pulse transit times provide interesting physiological measurements of blood flow throughout the body. As mentioned previously, the largest difference occurs between the triceps and ankle. The triceps receive blood faster via the brachial artery and a closer proximity to the aorta than the lower legs. We had originally theorized that the neck sensor would sense the pulse wave before the other sensors, but our analysis refutes this.


\subsection{Pulse Transit Time from rPPG}
Beyond the benefit of contactless monitoring, one of the most powerful properties of a camera is the ability to sense at multiple spatial locations. To leverage this, we performed remote pulse transit time (rPTT) with a sliding cross-correlation from the POS rPPG waveforms at multiple sites on the body. The POS waveforms are noisier than the PPG waveforms in the legs and arms, but the infrequent pulse rate errors (see Fig. \ref{fig:result_HRs}) indicate that the signal quality is high enough for rPTT.

Figure \ref{subfig:rptts} shows a heatmap of the rPTT between all pairs of the face, legs, and arms. The maximum observed rPTT is 46.75 ms between the face and left leg, which is within a reasonable range compared to the largest contact-PTT measurements between triceps and legs (51.23 ms and 48.82 ms). An interesting observation is the bilateral asymmetry visible in the rPTT left and right sides of the body. The pulse waves from the right side of the body were consistently observed sooner than the left side of the body.

To compare the rPTT measurements with the PPG-derived PTT values we selected the nearest contact sensors to the rPTT measurement sites, and discarded the rest (\eg ankles and triceps). Figure \ref{subfig:ptt_boxplots} shows boxplots of the PTT values for all 5-second time windows in the relaxation portion of the dataset. The groups were sorted by the median PTT at the pair of measurement sites for easier viewing. The means of the rPTT measurements via POS are not perfectly aligned with the PTTs --- but upon further inspection, we can see that rPTT and PTT correlate well across the pairs of body sites. A potential reason for the bias in the rPTT values is that the signal quality is not uniform over the measured body sites. For example, the POS signal on the arms is likely shifted in phase towards the pulse wave at the hands. Similarly, the POS wave is likely shifted more towards the phase of the thighs than the lower legs. Furthermore, the sampling rate of the video is 90 fps, so large time lags such as 55 ms will only occur as 5 frames. Future work will perform more robust transit time calculations via the systolic foot and finer-grained spatial measurements of rPTT to better align with the contact-PPG sensors.
