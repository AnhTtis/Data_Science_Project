We measured the impact of the number of categories and choice of color palettes on people's perceptions of multiclass scatterplots.
We find that people are less accurate at assessing category means as the number of categories increases. However, certain palettes may be more robust to variations in category number. 
Our results provide new perspectives on prior findings and offer both actionable design guidance and opportunities for future research. 
%Our results can validate or be in contrast to some prior findings, be derived into design guidelines, and be improved in some future directions.


\subsection{Reflections on Prior Findings}

%Besides the above findings, in addition, our results can indicate, validate, or oppose the following hypotheses or findings proposed by prior studies.

%\fix{we should put all of them together without sub-subsection. We want to say the result validated prior findings without giving too much attention and limiting our findings.}
% \paragraph{Hardness level}

% Gleicher et al.~\cite{gleicher2013perception} proposed the definition of hardness level $\Delta$, and found that $\Delta$ can measure task hardness and the overall performance in 2-class scatterplots.
% The results of our study indicate that $\Delta$ would be a solid cue for accuracy, the accuracy of judgment will descend with the hardness level arises (see \autoref{sec-analysis-delta}), which can extend their findings to multiclass scatterplots with more than 2 classes.


%\paragraph{Number of points}

%Our analysis result cannot indicate a significant correlation between the accuracy rate and the number of points (see \autoref{tab:ancova-result}).
%In other words, increasing the number of points would not significantly reduce the accuracy of judgment even with a large number of categories, which is consistent with the finding of Gleicher et al.~\cite{gleicher2013perception} in 2-class scatterplots.

%\paragraph{Number of categories}
Our study indicates that increasing numbers of categories would lead to descending performance in relative mean estimation (see Section \ref{sec-analysis-cat}).
However, this finding is inconsistent with the proposed guidelines of Gleicher et al.~\cite{gleicher2013perception}, which found little impact of adding additional distractor classes. We anticipate that this contradiction is a result of the number of classes evaluated. For smaller numbers of classes, performance tended to be more robust across palettes. However, as the number of classes increased, performance started to degrade. This finding likely stems from a correlation between performance and color discriminability indicated in our exploratory analysis. As the number of colors increases, it becomes more difficult to ensure colors are spaced apart, especially if maximizing for metrics like pair preference~\cite{schloss2011aesthetic} or otherwise minimizing the number of large hue variations to preserve harmonies \cite{stone2006choosing}. Certain palettes may differently balance this aesthetic and performance trade-off. However, our results indicate that this trade-off is sensitive to the number of categories present in the data. 

\add{Contrary to past heuristics, we found that performance remained relatively high even for more than seven categories. We hypothesize that people may simply be better at this task than they expect: while the task may feel significantly more difficult as the number of categories increases, our visual system may be more robust than expected in working with complex categorical data. Feedback from pilot participants indicated that the tasks felt challenging as the number of categories exceeded four, but these participants, like those in the formal study, still performed well at such seemingly difficult tasks. 
%Performance still decreased, but people performed significantly higher than chance as the category count increased. 
While the tested palettes reflect best practices, our findings challenge existing heuristics around the scalability and utility of color in visualization. A more thorough and rigorous empirical examination of the robustness of categorical perception in visualization generally would benefit a wide range of applications.}
%can be ignored and suggested that "multi-class scatterplots should not necessarily be avoided in favor of simpler ones".

\add{Our results demonstrate a high overall accuracy (more than 85\%) compared to Gleicher et al. ~\cite{gleicher2013perception} ($\sim$ 75\%) even though we tested a larger number of categories. 
Gleicher et al. chose to generate uniformly sparsely distributed classes. 
%, where points from each class were uniformly and sparsely located on the entire scatterplots.
However, such distributions might not be widely applicable; in real-world use cases, 
%researchers are more interested in and find more
people more commonly build insight from densely distributed classes~\cite{van2008visualizing}.
As shown in \autoref{fig:num-stimuli}, we generated more randomly and densely distributed classes to privilege ecological validity.
Consequently, our stimuli are more likely to perform similarly to what people usually see in their daily life, but the clustering structure may have affected task accuracy by, for example, making it slightly easier to group points within categories.
%, thus making our results more practical and meaningful to real-world usage scenarios. 
While we tuned our stimulus difficulty in piloting and our results were consistent across different performance thresholds, providing evidence of their generalizability, these differences raise important future questions as to the impact of different data distributions on categorical palette design. }

% Despite carefully tuning difficulty levels from our pilot studies, as the category number grew, the accuracy rate did not drop as significantly as we expected. Instead, the results show consistently high accuracy over all conditions. different tasks regarding different category numbers or different color encodings. Hence, we learned that people are having overall good performance over hard tasks than common expectations. While people's perceptions of how hard a task is may not always correlate to their performance. However, even though we have demonstrated our findings that the performances are different while using different color palettes and with different number of categories, as we learn that people are good at such tasks that raise the future questions of conducting even more difficult tasks to robust empirical assessment of these kinds of heuristics.}
% Our different class generation decisions could impact this accuracy.


% Our early to e more like results show that there is a strong correlation between the accuracy rate and the number of categories (see \autoref{tab:ancova-result}). Not only does the number of categories impact the accuracy, but the result \autoref{fig:all}) also indicates that as adding more categories does influence the accuracy rate in a certain trend.

%\paragraph{Subitizing}
Part of the difference in results between our findings and Gleicher et al. may stem from differences in perceptual mechanisms present when processing different numbers of categories. 
Our study reveals a significant "dip" in accuracy when the number of categories increased to five or six (see Section \ref{sec-analysis-cat}).
These bumps correlate with a key number of objects for subitizing~\cite{kaufman1949discrimination}: below roughly six objects, we can instantly and precisely detect the quantity of objects present, whereas we have to actively count larger numbers. While subitizing tends to focus on individual objects rather than collections of objects, the dip in accuracy directly correlates with this subitizing threshold and echoes similar findings in past work in categorical visualization~\cite{haroz2012capacity}. Our study is not designed to probe subitizing or other specific perceptual mechanisms that may explain these results. However, this correlation offers opportunities for further understanding the relationship between categorical perception, subitizing, cluster detection, and other related perceptual phenomena in visualization. 
%where Kaufman et al.~\cite{kaufman1949discrimination} found that when the number of objects is larger than 4, the ability of human cognition would be significantly weakened.
%Our result shows that the obvious accuracy descent occurs in 6 categories, which may be consistent with the finding of Haroz and Whitney~\cite{haroz2012capacity}.
%However, because of the limited dataset, we would not assert it as a finding.
%Instead, we hypothesize that the unusual performance of 6 categories might be related to $subitizing$ in categorical visualizations.


We also found a significant overall difference between color palettes. These differences echo the findings of Liu \& Heer \cite{liu2018somewhere} for continuous colormaps: even if a palette satisfies the basic constraints of good palette design (e.g., discriminable colors), it may not perform optimally. Like Liu \& Heer, we also find that characterizing the source of these performance differences is challenging: palette effectiveness arises from a complex combination of factors. Future work should seek to further deconstruct these factors to derive more robust design guidelines. 
%, while these differences are undoubtedly due to perceptual properties 


\subsection{Design Guidelines for Multiclass Scatterplots}

%Our results reveal that the 
The data and design 
%choices can still significantly impact the accuracy of human perception in multiclass scatterplots.
of multiclass scatterplots significantly influence our abilities to reason across classes. 
Compared to Gleicher et al.'s guidelines~\cite{gleicher2013perception}, our results emphasize the 
%overlooked 
influences of category number and 
%the choices of 
color palette, which are the two essential elements in visualizing categorical data. 
%on the scatterplot.
Additionally, in contrast of some existing guidelines for color palettes~\cite{ware1988color, rogowitz2001blair}, our results indicate maximizing luminance variation may hinder analysis. While designers can use our results to directly choose the optimal palette from our tested set of palettes given the number of categories in their data, our results also provide preliminary guidance for palette selection more broadly: 
%Based on our findings, we recommend the following preliminary guidance for selecting palettes in multiclass scatterplots. 
%present our design guidelines for multiclass scatterplots as follows.

%\begin{itemize}
\vspace{3pt}\noindent \textbf{Simplifying category structure may improve performance.}
%Utilize multiclass scatterplots when required, but be aware of the performance descending.}
Our study suggests that people can reason across multiple classes encoded using color.
However, as shown in Section \ref{sec-analysis-cat}, designers should be aware that performance tends to degrade as the number of categories increases: people are slower and less accurate, especially when working with six or more categories.
%Additionally, it's worth to note when the number of categories is larger than 5, there might occur an unusual performance reduction of judgment accuracy.
We recommend designers 
%should 
consider how the number of categories influences performance on key tasks and consider collapsing relevant categories hierarchically if necessary. 

\add{As a caveat, people were relatively good at completing this task, even with larger numbers of categories than conventional heuristics recommend. Our results indicate that people can reliably distinguish colors in large palettes even though informal pilot participants indicated that the task felt quite difficult for higher numbers of categories. This contrast between perceived and objective performance suggests that even well-established design heuristics can benefit from experimental validation and refinement.  }
%notice the performance descending with categories increasing, and avoid using too many categories for complex tasks.

\vspace{3pt}\noindent \textbf{When designing new palettes,  consider fewer lightness differences, larger perceptual distances, and lighter colors.}
As shown in Section \ref{sec-analysis-color}, our results reveal that color palettes significantly impact the accuracy of human judgment.
%Specifically, for pre-defined designer-crafted palettes, we suggest \emph{SFSO Parties} because of the high performance compared to others.
%Based on our results, for creating new color palettes, we would inform them that \emph{L*} path in CIELCh color space and Perceptual Distance metric~\cite{sharma2005ciede2000} can best capture the performance of palettes, 
Our exploratory analysis confirms the benefits of maximizing the pairwise difference between colors and provides further evidence of minimizing lightness variation. However, we also find that palettes using lighter colors tend to also enhance accuracy. We anticipate that this bias may be in part due to the use of a white background enhancing contrast within categories while minimizing undesirable ``loud'' colors that have too high of a luminance contrast with the background. 
%and suggest choosing lighter colors with minimizing luminance difference and maximizing perceptual distance to promote the ability for judgment tasks.
However, the tested palettes are all handcrafted to select harmonious and aesthetically pleasing colors. Future work should investigate 
%whether 
these results 
%replicate 
on other background colors. \add{We also found little evidence of the benefits of hue or color name variation when considered in isolation. This points to the need for the systematic interrogation of designer practices to improve existing heuristics for palette design \cite{smart2019color}.}

% \item \textbf{Consider multiple choices of color palettes if you're designing scatterplots with different numbers of categories.}

\vspace{3pt}\noindent \textbf{Choose your palettes to fit your data.}
%Since we found significant interaction effects between the choice of color palettes and the number of categories in \autoref{sec-analysis-color}, which reveals that w
When the number of categories changes, the performance rank of different color palettes may also change. Different palettes are differently robust to changes in category number. We recommend designers select color palettes based on the parameters of their specific data.
For example, a designer might use \emph{SFSO Parties} or \emph{ColorBrewer Set3} for multiclass scatterplot with less than seven categories and \emph{D3 Cat10} for larger numbers of categories (see \autoref{fig:palettes-acc}).

%\end{itemize}

% \subsection{A Use Case in Real Scenario}

% \begin{figure}[htbp] 
% \centering
% \includegraphics[width=0.9\textwidth]{Figure/refill.pdf} 
% \caption{Left is the multiclass scatterplot from The Economist \cite{theEconomistarticle} article, with seven categories. We regenerated the plot and refilled the colors using the top 2 palettes from our ranking.} 
% \label{fig:refill}
% \end{figure}

% Multiclass scatterplots are commonly used on media platforms, e.g., New York Times \cite{theNYT}, and The Economist \cite{theEconomist} to demonstrate categorical data and salient patterns. We collected examples of multiclass scatterplots with four or more categories from the above sources.
% % websites or blogs of commercial visualization tools, and refilled them with the color palettes from our guidelines. 
% For example, \autoref{fig:refill} shows a replicated multiclass scatterplot from an article \cite{theEconomistarticle} of The Economist. There are seven categories in these three scatterplots, the first one originated from the article and two replicated charts using \emph{SFSO Parties} and \emph{ColorBrewer Set3} color palettes from our ranking based on overall accuracy of color palettes.
% %
% Based on our study findings, we can demonstrate that the ranked color palettes can be used with an increasing number of categories in the multiclass scatterplot. We did not conduct a case study to prove our regeneration charts have better performance. In this example, we simply demonstrate that the scatterplots having a large number of categories do have a different categorical perception using different color palettes. 

% \fix{compare and contrast color palettes with Tableau in terms of performance from results and how we can provide new guideline.}

% \fix{Getting some examples from NYT with more than 5 categories scatterplot and talk}

% \subsection{\add{Real World Application}}
% \add{Multiclass scatterplots are widely used in real-world scenarios, from media platforms like New York Times \cite{theNYT} to plots rendered using commercial visualization tools like Tableau \cite{tableau}.
% %As encoding the categorical information as third dimension by colors, the selection of 
% Our results show that the color palettes used in a visualization can affect the perception, performance and scalability of the rendered scatterplot. Based on our findings, performance is influenced directly by the number of categories and choice of color palette. 
% %with the increasing number of categories, not only does the overall performance drop significantly, but the performance also varies more between color palettes. 
% Despite the color design in a small number of categories seeming trivial in real-world usage, we have demonstrated the significant correlation between color palettes and performance, especially showing the robustness in higher category numbers. We suggest that our findings can be applied to real-world scenarios and improve the multiclass scatterplots design.}

\subsection{Limitations and Future Work}
We studied the impact of the number of categories and color palettes on multiclass scatterplots. 
%which are essential and commonly used visual channels in multiclass scatterplots.
However, scatterplots offer a wide variety of design choices for representing categorical data that may
%for other visual encodings that can 
provide different trade-offs in perception~\cite{sarikaya2018scatterplots}. 
%, and these encodings' impacts on multiclass scatterplots are still not clear.
Future work should explore the robustness of different channels to varying numbers of categories. Further, scatterplots often encode larger numbers of variables, such as multiple categorical dimensions or combining categorical and continuous dimensions~\cite{smart2019measuring}. Future work should investigate the interplay between different design factors in higher dimensional multiclass scatterplots. 
\add{Both our study and Gleicher et.al.'s work~\cite{gleicher2013perception} focused on 
%the impact of the y-axis (i.e., computing $\Delta$ based on y-value)
comparing y values. However, scatterplots are two-dimensional visualizations. Future work should consider the impact of palettes on crossdimensional tasks.}
%As a consequewe focused on nce, we tend to explore more visual encodings, such as varying shape and size of points, to understand the robustness of human perception to mean judgment tasks and assess their interaction effects with category numbers and color palettes in our future work.

We 
%considered and 
evaluated 10 pre-defined qualitative color palettes on qualitative data.
%in the experiments.
\add{We employed a random color sampling strategy from selected palettes for 
%scatterplots 
data with less than ten categories to simplify the stimulus generation to avoid potential bias from sources outside of color selection. Future work should extend our results to consider sequential strategies in comparing preconstructed palettes.}
%It might avoid the original palettes' sequential impact, however, it also limits the generalizability of our results.}
Additionally, categorical data can also be encoded using other types of palettes, such as sequential and diverging encodings~\cite{liu2018somewhere}, whose robustness to varying numbers of samples is not well understood. 
\add{
%Beside, application-specific design guidance for color-blind users has been 
Considering additional properties of color selection, such as accessible palettes for people with color vision deficiencies~\cite{jefferson2006accommodating, pugliesi2011cartographic}, is also important future work.
%however, how those color-blind palettes would perform on perception and cognition tasks such as mean judgment still require a deeper look.
}

%, a possible perspective for future work can be taking these types of palettes into consideration.
%Further, w
We sampled from predefined palettes at a fixed mark size. Varying mark size can influence mark discriminability~\cite{szafir2018modeling}. As mark size was held constant for all palettes and all palettes had large distances between all color pairs, we do not anticipate that this choice biased our results. However, future  work should explore a larger range of mark sizes and mark types. It should also seek to more systematically evaluate the robustness of our exploratory results. Such variation is challenging due to a large number of potential perceptual factors; however, our results may provide preliminary support for identifying the most promising factors. 
%Moreover, since the palettes we utilized are design-crafted, the performance of perceptual color measures, such as perceptual distance~\cite{sharma2005ciede2000}, designed to describe perceptual features of color palettes cannot be comprehensively assessed in our study.
%Future work can analyze color measures to explore whether these measures fit with their designers' original objectives according to their consistency with human perception.

%Further, we utilize the idea of $subitizing$ to explain some disturbance of judgment accuracy in our results.
%However, it's still a hypothesis, and our current results cannot strongly convince it as a finding.
%But since this phenomenon has been confirmed in other visualization type~\cite{haroz2012capacity}, we tend to explore more general cases for more populations in information visualization.
%For future work, we can focus more to find similar disturbances in larger experiments with more participants, a larger number of cases, kinds of plots, and charts, and find a model to faithfully explain that or convince $subitizing$ as a general assertion in visualization.

%Additionally, to avoid participants from developing strategies that select correct answers by ignoring the bottom part of the chart, and failing to evaluate all the points,
%we tried to generate our data points by using Gaussian distribution that makes points from different categories cross one another.
%However, it is inevitable that as the number of categories grows, the impact of such strategies will be enlarged.
%It would be crucial to find a better way to generate stimulus or find an alternative task to avoid employing such strategies in future work.

\add{We elected to use Mechanical Turk to reflect the range of viewing conditions and participants common to web-based visualizations and to recruit larger numbers of participants. However, variations in viewing conditions can influence color perception. While past studies of color perception in visualization validate the predictive ability of crowdsourced studies for color perception studies in HCI \cite{szafir2014adapting,reinecke2016enabling}, the variability introduced by the range of viewing conditions on MTurk limits the generalizability of our results and our ability to make precise claims about fine-grained mechanistic perceptual phenomena. However, given the large differences between colors in our palettes, we anticipate the affect of viewing variation to be relatively minimal \cite{moroney2003unconstrained} and followed best practices in our experimental design to minimize the impact of viewing variation. Future work seeking to quantify more precise causal mechanisms underlying our findings may wish to replicate our study under more constrained conditions. 
%Furthermore, another limitation is that we couldn't control participants' screen performance (limited by the MT study style) such as brightness and hue variance which might impact the presentation of colors.
%For example, Camgoz et al.~\cite{camgoz2002effects} introduced color brightness would impact users' preferences for background colors and Benedetto et al.~\cite{benedetto2014effects} proposed that higher levels of screen brightness would increase users' visual fatigue.
%According to their findings, we would assume that a low-brightness screen user might perform worse for low-luminance color palettes but with more consistent performance over the whole study.
}

Additionally, data-centric statistical factors that may be related to the performance of multiclass scatterplots are not considered in our study.
For example, we did not explore the impact of correlation or strength of clusters.
Extending our experiments to consider a wider range of data properties as well as statistical tasks 
%enable such consideration of statistical factors would further improve the understanding of human perception 
could help us further understand categorical data visualization for complex datasets and usage scenarios and offer broader guidance for categorical visualization generally. 
%, and could more likely be extended to general visualization types.


