
We discuss significant results and statistical analysis based on the independent factors considered in this paper (see Appendix) using both traditional inferential measures and 95\% bootstrapped confidence intervals ($\pm$ 95\% CI) for fair statistical communication~\cite{dragicevic2016fair}. Table \ref{tab:ancova-result} summarizes our ANCOVA results.
%Table~\ref{tab:} shows an overall illustration of the results of all of the independent visual factors and the corresponding interaction effect.
Additional results, charts, and details of the analysis can be found on Appendix. 
%in the supplemental materials.

%We failed to find any significant effects of the number of points (its $p$-value is $0.09$ which only reflects a marginal effect, see \autoref{tab:ancova-result} and \autoref{fig:all} (e) \& (f)) and found the predicted effect of hardness level (when $\Delta$ increase, the performance decrease, see \autoref{fig:all} (c) \& (d), with $F(1, 89)=31.2187, p<.0001$). These findings align well with previous study~\cite{gleicher2013perception}, so we included these factors as random covariates in our ANCOVAs.

\begin{figure*}[htbp] 
\centering
\includegraphics[width=0.8\textwidth]{Figure/main-acc-time.pdf}
% \vspace{-1em}
\caption{Our primary results with respect to the numbers of categories, hardness level, and numbers of points. Graphs on the left show changes in accuracy, whereas those on the right show response times Both accuracy and time do not systematically vary with the number of points. However, as the number of categories grows or the hardness level increases, the overall accuracy rate drops, and the time spent escalates. In order to show the trend clearly, we used a scale from 50--100\% (chance at our smallest number of categories to perfect performance) on the y-axis for accuracy. Error bars represent 95\% confidence intervals.} 
\label{fig:all}
% \vspace{-1em}
\end{figure*}

\subsection{Number of Categories}
\label{sec-analysis-cat}

Our results support \textbf{H1}:
we found that performance decreased as the number of categories increased. 

Our analysis reveals a significant effect of category number on judgment performance ($F(8, 82)=7.6511, p<.0001$): people were both less accurate and slower with higher numbers of categories.
\autoref{fig:all} (a) shows that accuracy rate decreases based on the number of categories from 96.4\% to 86.6\%, 
with an overall descending trend as the number of categories increases.
\autoref{fig:all} (b) presents the average spent time broken down by category number,
suggesting that participants were slower for scatterplots with more categories.

We also found anomalies in the accuracy rate
%occurs with strange $bumps$ when the category number is 5 or 6 in 
for between five and six categories (\autoref{fig:all} (a)).
While we initially assumed this anomaly to be noise, the pattern was repeated across almost all palettes. 
This category number correlates with past findings of \textit{subitizing}---the ability to instantly recognize how many objects are present without counting---in categorical data from Haroz \& Whitney~\cite{haroz2012capacity}. 
%which may cause a significant descent in the performance with more than 5 classes of objects in color-coded binned images by Haroz and Whitney~\cite{haroz2012capacity} are quite similar to our results.
%Hence we hypothesize this might cause by the $subitizing$ phenomenon.
%We would not confirm it as a finding because of the limited dataset.
While we do not confirm this hypothesis in this study, our results do raise questions about the role of subitizing or a related mechanism in categorical reasoning with visualizations.  

\begin{table}[htbp] 
\centering
\caption{ANCOVA results. Significant effects are indicated by \textbf{bold} text and the corresponding rows are highlighted in green.}
\includegraphics[width=\linewidth]{Figure/ancova-table.pdf} 
\label{tab:ancova-result}
\end{table}

% \subsection{Hardness level $\Delta$}
% \label{sec-analysis-delta}

% \autoref{fig:all} (c) illustrates the average accuracy rate of different difficulty levels, where easy refers to \begin{math}  2.5<\Delta\leq3.0 \end{math}, intermediate means \begin{math}  2.0<\Delta\leq2.5 \end{math}, and hard is \begin{math}  1.5<\Delta\leq2.0 \end{math}.
% The result clearly reveals that with the $\Delta$ value decreases, its relevant average accuracy rate descends respectively.
% This finding can also be convinced by ~\autoref{fig:all} (d), which shows that participants spent more time on questions with higher hardness levels.
% Furthermore, the $F$-value of the impact of hardness level in our ANOVA analysis is $F(1)=31.2187, p<.0001$ as shown in \autoref{tab:ancova-result}, as a result, it suggests a strong effect.

% Both results reveal that the hardness level is closely related to the accuracy of perception and $\Delta$ is useful to measure the hardness of the mean judgment task in multiclass scatterplots, hence confirming Gleicher et al.~\cite{gleicher2013perception}'s corresponding finding about $\Delta$.



% \subsection{Number of Points}
% \label{sec-analysis-points}

% \autoref{fig:all} (e) presents the relations between the number of points and the average accuracy rate.
% The results show that with the number of points increasing, its corresponding accuracy rate varies from about 95\% to 85\%.
% But there's no obvious trend correlation between number points and accuracy rate.
% According to \autoref{fig:all} (f), the response time of participants varies little with varying numbers of points.
% Similarly, its $p$-value is $0.086$ in \autoref{tab:ancova-result}, which cannot reveal an obvious impact.
% So we suggest that the number of points would not significantly impact human perception accuracy, thus validating the finding of prior study~\cite{gleicher2013perception}.

\begin{figure*}[htbp] 
\vspace{-1em}
\centering
\includegraphics[width=0.8\textwidth]{Figure/color-acc.pdf} 
\vspace{-1em}
\caption[]{The accuracy rates based on the number of categories separated per color palette, sorted by average accuracy over all categories (dash lines) sorted from most to least accurate. Color palettes are shown along with corresponding charts. See Section \ref{sec-analysis-color} for detailed analysis \add{and \autoref{tab:parameters} in the Appendix for the count of scatterplots per palette.}
} 
\label{fig:palettes-acc}
\end{figure*}

\subsection{Color Palettes}
\label{sec-analysis-color}

% \begin{table}[htbp]
%     \centering
%     \caption{Three categories of color palettes of ANOVA analysis.}
%     \label{tab:color-classes}
%     \begin{tabular}{ |p{3cm}||p{2cm}|p{2cm}|p{2cm}|p{2cm}|  }
%      \hline
%      Color Palette &Class A&Class B&Class C&Least Sq Mean\\
%      \hline
%      SFSO Parties    & \checkmark  & \checkmark & & 0.95\\
%      ColorBrewer Set3& \checkmark  & \checkmark & & 0.94\\
%      Stata S2        & \checkmark  &            & & 0.94 \\
%      Tableau Tab10   & \checkmark  & \checkmark & & 0.94 \\
%      D3 Cat10        & \checkmark  & \checkmark & & 0.92\\
%      Carto Bold      & \checkmark  & \checkmark & & 0.92\\
%      Carto Pastel    & \checkmark  & \checkmark&\checkmark&0.90\\
%      ColorBrewer Paired &   &  \checkmark&\checkmark&0.88\\
%      PaulTol Muted      &   & &\checkmark&0.84\\
%      Stata S1           &   & &\checkmark&0.83\\
%      \hline
%     \end{tabular}
% \end{table}

Our results also support \textbf{H2}: color palettes significantly affect accuracy ($F(9, 81)=8.4689, p<.0001$, see \autoref{tab:ancova-result}). 
%where we found color palettes seriously impact performance.
%The significance analysis reveals a severe impact on the choices of color palettes .
We 
%also 
found a significant interaction effect between color palettes and the number of categories for both time and accuracy. 
%with $p=.0003$.
In other words, as the number of categories increases, the accuracy ranks between color palettes might be different. Different palettes are more or less robust to increasing the number of categories. This finding indicates that there is no best palette
%It indicates that there is no magic choice of color palettes in
for multiclass scatterplots. Instead, our results provide guidance for designers to select effective palettes based on the number of categories in their data. 
%and to achieve the best performance, we should choose specific color palettes to fit with the target data.
%This result also suggests there is no magic choice of color palettes that can fit well for all kinds of multiclass scatterplots.
%Color palettes within each class are closely related.
%Class A \& B share a large proportion (6 of 7) of the same palettes, which suggests those two classes might be similar.
%And Class A represented the highest performance in the accuracy of human judgment, Class B performed only slightly lower than A, and Class C performed the worst.

%We further created palette-specific results of accuracy, 
\autoref{fig:palettes-acc} shows the
%line charts of 
accuracy rate and category number per color palette. 
%\add{We skip plotting the error bars while separating the results by color palettes and category numbers, there are not sufficient numbers for showing the meaningful error bar at each point.}
These charts reveal that:

\begin{enumerate}
    \item \emph{SFSO Parties} and \emph{ColorBrewer Set3} achieved the highest average accuracy rate in all data, whereas \emph{PaulTol Muted} and \emph{Stata S1} exhibited the worst overall performance (an 11.3\% accuracy difference on average between \emph{SFSO Parties} and \emph{Stata S1}),
%where we assume the low accuracy rate in 8 categories of \emph{SFSO Parties} is an outlier because of the high accuracy of all the rest categories,

\item lower performing palettes tend to be less robust to increasing the number of categories, and 

\item most palettes show an overall descending trend as the number of categories increases, though some palettes remained relatively robust (e.g., \emph{Stata S2}, \emph{D3 Cat10}).

\end{enumerate}
%, which is consistent with hypothesis \textbf{H1},

%4) \emph{D3 Cat10} shows a unique increasing trend when the category number is larger than 6 hence it finally ranks top for larger categories, which partially support  \textbf{H3}, and

%4) there are significant performance differences across palettes, e.g., the accuracy difference between \emph{SFSO Parties} and \emph{Stata S1} is more than 12 \%. 

%We further studied the relation between the property of color palettes and judgment accuracy to find whether the selected metrics can measure this difference.

\begin{table}[htbp] 
\centering
% \vspace{-1em}
\caption{Three performance classes of color palettes from Tukey's HSD. Performance is rated better to worst from Class A to C respectively. }
% \vspace{-1em}
\includegraphics[width=\linewidth]{Figure/palette-class.pdf} 
% \vspace{-1em}
\label{tab:color-classes}
\end{table}

\subsection{Exploratory Analysis}
\label{sec-analysis-explor}

To better analyze the impact of specific color palettes, we performed a Tukey's HSD with Bonferroni correction to identify significant performance differences between palettes. 
%for the 10 employed color palettes, resulting in 3
The test revealed three \emph{classes} of color palettes with comparable performance, shown in \autoref{tab:color-classes}.
\autoref{fig:class-acc} illustrates the combined accuracy rate of the three classes, in which Class A refers to the best performance, Class B is slightly lower, and Class C is the worst overall. 
All three classes of palettes showed a steady downward trend that is consistent with \textbf{H1}. We use the clusters created by these performance classes to scaffold an exploratory analysis of potential metrics associated with the observed performance differences. 

\begin{figure*}[htbp] 
\centering
% \vspace{-1em}
\includegraphics[width=0.95\textwidth]{Figure/acc-class.pdf} 
% \vspace{-1em}
\caption{The average accuracy rate with different numbers of categories per performance class of color palettes. Charts represent Class A to C from left to right.} 
% \vspace{-1em}
\label{fig:class-acc}
\end{figure*}



% Both the result of Class A and B shows an overall descending tread, and the accuracy of categories 4 and 5 are almost the same while the differences are less than 0.1\% in Class A and less than 1\% in Class B.
% The difference ratio is relatively low so we suggest it might be an outlier and it would not reduce the strength of our hypothesis \textbf{H1}.
% There also shows a slight increase from category 9 to 10 in Class A and B, since the difference is relatively low, we suggest that's because there are too many categories to distinguish and would not lessen the strength of our finding.
% The result of Class C is with huge jitters for all number of categories but also reveals a slight descending overall trend.

%\subsection{Color Metrics}

%Furthermore, we evaluated the mentioned 8 color metrics (see \autoref{sec:metrics}) to assess their consistency with our accuracy results.
%\subsubsection{Color Discrimination.}
We analyzed these classes using eight color metrics associated with palette design to explore the relationship between performance and common design parameters: perceptual distance~\cite{sharma2005ciede2000}, name difference~\cite{heer2012color}, and name uniqueness~\cite{heer2012color} as employed by Colorgorical~\cite{gramazio2016colorgorical} and the magnitude and variances of different dimensions in CIELCh~\cite{zeileis2009escaping} ($L^*$, $C^*$, and $h^*$). 
%, i.e., $L*$ length (lightness), $C*$ length (colorfulness), $L*$ variance (light difference), $C*$ variance (color difference), and $h*$ variance (hue difference).
The 
%mathematical equations for computing 
computations for those metrics can be found in our supplemental materials.
Since we randomly sampled colors in a palette for plots with less than 10 categories (see Section \ref{sec-stimuli-generation}), for each target color palette, we compute those metrics based on the actual colors used in each individual stimulus sampled from the target palette to explore the distribution of these features with respect to performance.

We conducted an ANOVA using these nine measures to assess the impact of each parameter on accuracy (\autoref{tab:metric-significance}).
We found significant effects ($p<0.01$) of $L^*$ variance, $L^*$ magnitude, and all-pairs perceptual distance~\cite{sharma2005ciede2000} and marginal effects ($p<0.10$) of $h*$ variance and $C*$ variance.

\begin{table}[htbp] 
\centering
\caption{Results of significance analysis from color metrics to judgment accuracy.
The right-most column shows plus or minus of the $\beta$-ratio in the OLS linear regression where plus means incremental trend, and minus means decremental trend.
Significant impacts ($p<0.01$) are in \textbf{bold} style and green color. }
\includegraphics[width=\linewidth]{Figure/metrics.pdf} 
\label{tab:metric-significance}
\end{table}

To assess the direction of the effects, we performed an OLS linear regression of each metric and average accuracy. Since the value of $\beta$-ratio (in $Y=\beta X + \epsilon$) of regression differs from the data range of source metrics, we show its 
%plus or minus sign 
directionality in the right-most column in \autoref{tab:metric-significance}, where a plus sign refers to an increasing trend and minus sign means decreasing trend.
%The value of $L*$ length and perceptual distance would have an incremental trend with accuracy, which means 
We found that larger $L^*$ magnitude (lighter colors) and larger perceptual distance both lead to better judgment accuracy, whereas palettes with less lightness variance had better performance. 
%We also found a marginal effect of both chroma and hue variance. Contrary to common design conventions, performance was higher with lower hue variance and chroma variance. 
%would have better performance.
%Additionally, the marginal effects and negative signs of $h*$ variance and $C*$ variance show that larger hue and chroma difference might hinder performance.
%\fix{Not sure should we discuss hue and chroma variance since it's marginal.}

Our findings suggest that palettes leveraging lighter colors, larger perceptual distances, and lower luminance differences led to better performance \add{whereas factors like hue variance or name uniqueness that are conventionally associated with palette design may be less important when considered in isolation}.
The use of lightness in palettes has a tenuous history: some advocate for minimizing lightness variation to avoid biasing attention \cite{munzner2014visualization,kindlmann2002face}, whereas other  guidelines suggest that designers leverage higher contrasts introduced by lightness variations to take advantage of our sensitivity to lightness variations~\cite{rogowitz2001blair}.
%enlarge the luminance difference, such as higher contrast in luminance of colormap would lead to better representations of absolute values in data~\cite{rogowitz2001blair}.
%Nonetheless, our finding may lead to another topic, isoluminant colormap, that creates color palettes by minimizing luminance differences.
%Previous studies found isoluminant colormap is helpful to aid data value comprehension~\cite{kindlmann2002face}.
%and viewers can immediately make choices for palettes containing seven isoluminant colors~\cite{healey1996choosing}. 
Our results provide further support for privileging more isoluminant palettes to avoid directing too much attention to given classes
%with higher luminance which enables advantages in sub-set search task~\
\cite{braithwaite2010visual}.
%Accordingly, our results indicate that the guideline of maxmizing luminance difference might hinder analysis.
%As a consequence, we would recommend designers consider lighter colors, larger perceptual distances, and lower luminance differences in designing color palettes.
However, future work should more systematically explore this hypothesis. 



%This result indicates that 




% \begin{figure}[htbp] 
% \centering
% \vspace{-2em}
% \includegraphics[width=0.8\textwidth]{Figure/box-hcl.pdf} 
% \vspace{-1em}
% \caption{Boxplots with the distribution of $L*$ variance, $h*$ variance, and $L*$ length with 3 classes of color palettes.} 
% \vspace{-1em}
% \label{fig:class-bos}
% \end{figure}


%In particular, as shown in \autoref{fig:class-bos}, we can see the clear difference for the $L*$ variance, $h*$ variance, and $L*$ length metrics between palette classes.
%Most palettes in Class A \& B have a lower value of both $L*$ variance and $h*$ variance than Class C, which suggests that color palettes with slightly lower lightness and hue variance would offer better performance in judgment accuracy.
%But as for $L*$ length, Class A \& B would be slighter larger than C, which may reveal that lighter color palettes might achieve better performance.
%The above results suggest that lightness and hue of colors may be the major impact factor on judgment accuracy, users might perform better to distinguish classes and capture mean values with a palette that contains lighter colors with fewer variances in both lightness and hue.
%The detailed results of the rest metrics can be found in the supplemental materials.


% \subsection{Interaction Effect}
% \label{sec-analysis-effect}

% \begin{table}[htbp]
%     \centering
%     \caption{ $F$-value and $p$-value in the results of our significance analysis. Significant impacts are in \textbf{bold} style.}
%     \label{tab:ancova-result}
% \begin{tabular}{ |c||c|c|c|c| } 
% \hline
% Source & $DF_1$ & $DF_2$ & \emph{F}-value & \emph{p}-value \\
% \hline
% category number & 8 & - & 7.6511 & \textbf{< .0001} \\ 
% hardness level & 1 & - & 31.2187 & \textbf{< .0001} \\ 
% point number & 10 & - & 1.6502 & 0.086 \\
% color palette & 9 & - & 8.4689 & \textbf{< .0001} \\ 
% category number * hardness level & 8 & 1 & 1.7448 & 0.0832 \\ 
% hardness level * color palette & 1 & 9 & 1.1141 & 0.3486 \\ 
% color palette * category number & 9 & 8 & 1.6921 & \textbf{.0003} \\ 
% \hline
% \end{tabular}
% \end{table}

% \autoref{tab:ancova-result} illustrates the results of $F$-value and $p$-value from our analysis, which in an effort to validate the existence of interaction effects between category number, color palettes, and hardness level.
% As our prior results show that the number of points does not have a significant impact on the perception, we conduct the ANCOVA with the number of points as random effects. The results show that there are no obvious effect of hardness level to color palettes ($F(1,9)=1.1141, p=0.35$) and category number ($F(8,1)=1.7448, p=0.08$).
% However, we found significant interaction effects between the color palettes and the number of categories for both metrics ($F(9,8)=1.6921, p=.0003$).
% That is to say, as the number of categories increases, the accuracy ranks between color palettes might be different, which indicates that to achieve the best performance of human judgment accuracy, we may need to choose multiple color palettes for different numbers of categories.
% This result suggests there is no magic choice of color palettes that can fit well for all kinds of multiclass scatterplots.
% The above findings partially validate our hypothesis \textbf{H3}.

