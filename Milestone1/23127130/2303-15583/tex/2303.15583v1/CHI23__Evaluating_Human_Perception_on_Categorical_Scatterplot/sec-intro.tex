%
%\fix{\paragraph{talk about the background and introduce the problem you are solving}}
% \fix{\paragraph{why that problem is important and what previously(if any) has been studied to solve that}}
% \fix{\paragraph{How you solved that problem and high-level finding and result}}
%
Scatterplots 
%is one of the most common techniques to visualize bi-variate data that enables 
enable people to conduct a wide variety of 
%demanding 
statistical tasks~\cite{quadri2021survey}, such as finding outliers \cite{saket2018task, li2010size}, estimating mean values~\cite{gleicher2013perception, hong2021weighted, kim2018assessing}, and 
%interpreting data~\cite{quadri2021survey}.
assessing correlation~\cite{yang2018correlation,rensink2010perception}. 
Multiclass scatterplots leverage people's abilities to attend to different subsets of information in order to compare patterns across different categories of data. 
When the number of categories 
%of variables arise (i.e, multi-class data), 
increases, 
%selecting a proper visualization becomes a severe challenge
people's abilities to analyze patterns across categories may degrade~\cite{blasius1998visualization}.
%\fix{might need to review the 2nd sentence}
%Combing additional visual encoding such as color palettes together with scatterplots is a typical way of presenting multi-class data~\cite{sarikaya2018scatterplots}, resulting in \emph{multiclass scatterplot}.
However, certain scatterplot designs may be more robust to larger numbers of categories than others. 
Determining robustness is challenging as the perception of multiclass scatterplots 
%is more complex since it 
requires 
%building abstraction and aggregation 
first identifying points from relevant categories and then estimating values from those points across various visual encodings~\cite{gleicher2013perception}.

%Existing studies have focused on the perceptual impact of design choices among
Existing studies provide guidance for supporting a range of tasks in
%visual channels in 
general scatterplots~\cite{quadri2021survey}, such as cluster estimation~\cite{sedlmair2012taxonomy,quadri2020modeling,quadri2022automatic}, 
%mark shape~\cite{burlinson2017open}, 
or for tuning across channels such as 
color differences~\cite{szafir2018modeling} and point size~\cite{hong2021weighted,li2010size}.
However, little attention has been paid to 
%exploring guidelines of 
design choices for rendering complex multiclass scatterplots and how such design choices may change as the number of categories increases.
%For this objective, 
Color palettes~\cite{munzner2014visualization, stone2014engineering} and shapes~\cite{burlinson2017open} are commonly used to delineate categories in scatterplots, but available design guidance for effectively supporting categorical tasks is largely heuristic rather than empirical, which raises questions as to the robustness and precision of this guidance across a range of scenarios.

Gleicher et al.~\cite{gleicher2013perception} and Burlinson et al.~\cite{burlinson2017open}
%initially conducted several experiments for assessing visual aggregation via exploring the accuracy of human 
offer preliminary experimental insight into the robustness of different visual channels on 
%the perception of means
mean estimation in multiclass scatterplots. However, these studies focus on scatterplots with two to three classes, where as we 
%are measuring the effect of 2-10 numbers of 
measure the effect of 2--10
categories across different
color palettes. 
%in perceptions of multiclass scatterplots in this study.
%Based on the empirical results, they suggested scatterplots are capable to reveal the inter-class differences and viewers can extract that information.
%Haroz and Whitney found the number of categories heavily impacts the accuracy of human perception in categorical visualizations.
The number of categories heavily impacts people's abilities to reason across categories~\cite{haroz2012capacity}, especially for color, which remains the default channel for encoding categorical data in many popular commercial applications~\cite{shmueli2011data,mackinlay2007show}. 
%However, Gleicher et al.'s study mainly focused on simple 2-classes multiclass scatterplots (with exception of some 3-class scatterplots) without considering the impact of category numbers.
Existing studies~\cite{healey1996choosing, gramazio2016colorgorical, wang2018optimizing} 
%suggested there are severe impacts on selecting 
highlight both the importance and complexity of selecting proper color palettes for categorical visualizations. Despite the popularity of using color palettes in categorical visualizations, we lack insight into how robust these palettes are to the number of presented categories and as to whether that robustness varies across different parameters of a palette. Such information is critical for effectively communicating data, especially as the size and complexity of data continue to grow. 
%However, Gleicher et al.'s study only chose a same set of two or three colors in all of their experiments, thus the impact of color palettes on categorical perception was dramatically ignored.

%To overcome the above drawbacks and provide a more comprehensive design guideline, we thus conduct
We conducted a crowdsourced study 
%to explore human perception on more complex multiclass scatterplots via using the same \textit{mean-judgment} task.
to measure how robust different palette designs are to increasing numbers of categories. 
%To better assess the impact of visual encodings, we considered 5 independent visual factors in scatterplots: 
Participants estimated mean values in a series of multiclass scatterplots with varying numbers of categories (2--10), dataset sizes (10--20 points per category), and color palettes drawn from popular visualization tools.
%\emph{numbers of categories}, \emph{hardness levels}, \emph{point distribution}, \emph{numbers of points}, and \emph{color palettes} (see \autoref{sec-factors} for details).
%Prior to the formal study, we first conducted three pilot studies to test and decide the proper parameters of \emph{numbers of categories}, \emph{hardness levels}, and \emph{point distribution} for scatterplots in stimuli generation.
%Based on the pilot studies' results, we chose the ranges of the above 3 factors for generating multiclass scatterplots.
%Besides, we considered varying \emph{numbers of points} from 10 to 20 per category and chose 10 pre-defined design-crafted \emph{color palettes} in the stimulus.
%Using the above stimulus, we conducted an extensive crowdsourced experiment with 81 participants to evaluate human perception in mean judgment estimates on multiclass scatterplots.
%The research goals of our experiment are
%Our results enable us to characterize palette robustness with increasing numbers of categories. We deconstructed our results with respect to common parameters of color encodings to find potential cues for robust palette design. 
%to human perception of mean judgment with different visual encodings and to evaluate at which condition the visual encodings would cause perceptual biases.
%Our results demonstrate the influence of different visual factors on category perception in multiclass scatterplots.
We found that both the number of categories and choices of color palettes significantly impact people's abilities to estimate category means. 
%, 2) there's a solid cue between hardness level and perceptual accuracy, and 3) the perceptual impact of the number of points is minor.
%In addition, we convince 
We deconstructed our results with respect to common parameters of color encodings to find potential cues for robust palette design and find preliminary evidence that subitizing may impact categorical estimates~\cite{phillips1977components, haroz2012capacity,nothelfer2017redundant,nothelfer2019measures}.

\noindent \textbf{Contribution:} The primary contribution of our paper is evaluating 
%human perception for mean judgments on complex
mean estimation in multiclass scatterplots with varying color palettes.
% visual encodings
Our results 
%replicate previous findings about hardness level and points number~\cite{gleicher2013perception} and $subitizing$~\cite{haroz2012capacity}, and suggest novel disclosures about the impact of 
characterize the effect of the numbers of categories and color palette in perceptions of multiclass scatterplots.
%, and their interaction effects.
%The findings allow a deeper understanding of human perception in complex multiclass scatterplots with various visual encodings, especially for plots with many categories.
Our findings challenge current guidelines on multiclass scatterplot design~\cite{gleicher2013perception}, and we present an exploratory analysis of key factors for effective color palette design.  
%therefore we formalize our results as a new set of general guidelines for designing multiclass scatterplots.

