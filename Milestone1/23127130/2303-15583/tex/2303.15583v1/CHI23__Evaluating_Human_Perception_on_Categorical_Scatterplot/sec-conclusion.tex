

We measure how different color palettes impact people's ability to distinguish classes and assess mean values on multiclass scatterplots.
Our results suggest that both the number of categories and the discriminability of color palettes heavily impact people's abilities to use multiclass scatterplots.
We found that increasing the number of categories decreases how well people can distinguish different classes. Furthermore, we found preliminary evidence that even using designer-crafter palettes, a more discriminable color palette (such as \emph{SFSO Parties} who achieves 95\% average accuracy) can perform nearly 12\% better than a less discriminable one (such as \emph{Stata S1} with only 83\% average accuracy).
Based on the experimental results, we critically reflect on past findings and derive a set of design guidelines for palette selection in multiclass scatterplots.
We believe that our findings have the potential to support a variety of other visualization types and low-level tasks that combine continuous and categorical data. We hope our work will inform future studies to construct more general guidelines for the understanding of categorical perception in information visualization.


