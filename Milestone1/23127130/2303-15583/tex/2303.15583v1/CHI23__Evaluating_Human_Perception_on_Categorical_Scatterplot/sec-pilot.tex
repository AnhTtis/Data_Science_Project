


Our study aims to understand the robustness of color palettes on the perception of multiclass scatterplots.
%Since there is no prior work focusing on how those visual factors impact the perception of complex multiclass scatterplots, we first conducted 3 
To tune the parameters of our study, we first conducted three pilot studies to identify 
%the ability of human perception 
people's abilities to recognize data about different visual factors in multiclass scatterplots and to decide the proper parameters for scatterplots in stimuli generation.

\subsection{Factors}
\label{sec-factors}
We first describe the independent visual factors 
%that 
we considered for generating multiclass scatterplots in both pilot and formal studies.

\textbf{Number of categories.}
The total category count in a scatterplot, varies from 2-10 in our experiments.

\textbf{Level of difficulty.}
We described the distance of means between the categories that have the highest mean and the second highest mean to be $\Delta$. We considered the task to be easier as the $\Delta$ is larger, and more difficult as the $\Delta$ is smaller. 

\textbf{Point distribution.}
The pre-generated x-y data of points. Points from each category were randomly sampled from the Gaussian distribution.

\textbf{Number of points.}
The number of points in one category. Each category in the same scatterplot shared the same number of points, varying from 10-20. 

\textbf{Color palettes.}
%There are 
10 color palettes in total were used in our experiments, with 10 colors in each palette. A certain number of colors were randomly picked to display in each scatterplot depending on the number of categories.

\subsection{Procedure}
We followed the same procedure in the three pilot studies.
Participants were required to carefully read the task description first and then 
%tested 
completed a tutorial check to ensure their understanding.
Afterward, for each study, all participants viewed scatterplots from the corresponding dataset to make it a fair comparison.
They were required to pick the class with the highest average y-value.


\begin{table*}[htbp] 
\centering
\caption{\add{The number of samples collected for each experimental condition after exclusions. Columns are category numbers and rows are color palettes.  }} 
\includegraphics[width=0.9\textwidth]{Figure/result-count-table.pdf} 
\label{tab:parameters}
\end{table*}


\subsection{Pilot Study 1: Hardness Level ($\Delta$) of Stimulus}

\textbf{Participants.} We recruited 106 participants for this study. Participants are all college students, other demographic information was not recorded. They all participated voluntarily and no compensation was provided. 

\textbf{Generation factors:}
Number of categories: \{2\};
Level of difficulty: \begin{math} \{ \Delta \in \mathbb{R} \, | \, 0.5<\Delta<5 \} \end{math};
Point distribution: Poisson distribution with data points (x,y) denoted as \begin{math} \{ x, y \in \mathbb{R} \, | \, 0<x,y<10 \} \end{math};
Number of points: \{15\};
Color palettes: \emph{D3 Cat10}.

\textbf{Results.} The overall accuracy of this study is 76.88\%.
The results suggested that the accuracy rate will increase with the $\Delta$ rises.
To avoid showing tasks that are too easy or too difficult for participants, 
%that both cases would descend the differences in perception precision, 
we selected $\Delta$ from 1.5 to 3.0 in the final study. In the formal study, we mark the $\Delta$ in range 1.5 - 2.0 as hard level, 2.0 - 2.5 as intermediate, and 2.5 - 3.0 as easy (c.f., \autoref{fig:delta-stimuli}).
Details of the result and figures are available in our supplemental material.

\subsection{Pilot Study 2: Number of Categories} 

\textbf{Participants.} We conducted the second study with 25 participants from the 
%local college, 
UNC campus. Other demographic information was not recorded. They all participated voluntarily and no compensation was provided. 

\textbf{Generation factors:}
Number of categories: [2, 9];
Level of difficulty: \begin{math} \{ \Delta \in \mathbb{R} \, | \, 1.5<\Delta<3.0 \} \end{math}
Point distribution: Poisson distribution with data points (x,y) denoted as \begin{math} \{ x, y \in \mathbb{R} \, | \, 0<x,y<10 \} \end{math};
Number of points: \{5, 10, 15\};
Color palettes: \emph{D3 Cat10}.

\textbf{Results.} The overall accuracy of this study was 98.30\%.
The result revealed that participants can identify mean judgment across a lot of categories and colors.
Likewise, we decided to use 2 to 10 categories in the final study, 
%resulting in three difficulty levels: 8 - 10 refers to the hard level, 5 - 7 means intermediate, and 2 - 4 is easy, 
see \autoref{fig:num-stimuli} for examples.
The extremely high accuracy rate encouraged us to think about whether the results are impacted by our choice of distribution. We conducted a third study to check if the Poisson distribution is too na\"ive for this task.
Details of the result and figures are available in our supplemental material.

\subsection{Pilot Study 3: Point Distribution}

\textbf{Participants.} 81 participants joined the third study in total. All the participants were recruited from Amazon Mechanical Turk (MTurk), aged between 24 to 65, with an average of 37 with a standard deviation of 10.7. There are 51 males and 30 females, and 69 of them are wearing corrected glasses.

\textbf{Generation factors:}
Number of categories: [2, 10];
Level of difficulty: \begin{math} \{ \Delta \in \mathbb{R} \, | \, 1.5<\Delta<3.0 \} \end{math}
Point distribution: Gaussian distribution with data points (x,y) denoted as \begin{math} \{ x, y \in \mathbb{R} \, | \, 0<x,y<10 \} \end{math};
Number of points: [10, 20];
Color palettes: All 10 color palettes, see \autoref{fig:palettes-acc}.

\textbf{Results.} The overall accuracy of this study was 80.10\%.
The result suggested that there might be a cue between category number and human judgment accuracy.
Compared to the Poisson distribution in Pilot Study 2, the accuracy rate 
%is not too high to make it hard to analyze.
did not suggest a risk of ceiling effects.
As a result, we decided to use Gaussian distribution to generate scatterplots in our final study.
Details of the result and figures are available in our supplemental material.

\subsection{\add{Metadata}}

\add{\autoref{tab:parameters} illustrates 
%the numbers of generated scatterplots in our study separated and 
the distribution of collected data samples} counted by color palettes and category numbers. Conditions were assigned based on stratified random sampling as described in Section \ref{sec-methodology}.

% \begin{table}[htbp] 
% \centering
% \caption{The experiment parameters. We refined the factors and domain range from 3 pilot studies. Category number and color palettes are our independent variables and hardness level and point number are the control variables. The experiments were built from the combination of these four factors.} 
% \includegraphics[width=0.9\textwidth]{Figure/factor-table.pdf} 
% \label{tab:parameters}
% \end{table}

% \begin{table}[htbp]
%     \centering
%     \caption{ }
%     \label{tab:parameters}
%     \begin{tabular}{ |p{3cm}||p{2.5cm}|p{4.3cm}|p{3.5cm}|  }
%      \hline
%      Factor & Domain & Group & Sampling \\
%      \hline
%      Number of categories & \begin{math} \mathbb{N}: [2, 10] \end{math} & \begin{math} \{2, 3, 4\} \in Small \end{math} \newline \begin{math} \{5, 6, 7\} \in Medium \end{math} \newline \begin{math} \{8, 9, 10\} \in Large \end{math} & Random. Uniformly distributed between groups.\\
%      \hline
%      Color palettes   & 10 palettes \newline (see \autoref{fig:palettes-acc}) & - & Randomly assigned between participants. \\
%      \hline
% b     Hardness level $\Delta$  & \begin{math} \{ \Delta \in \mathbb{R} \, | \, 1.5<\Delta\leq3.0 \} \end{math} & \begin{math} \{2.5<\Delta\leq3.0\} \in Easy \end{math} \newline \begin{math} \{2.0<\Delta\leq2.5\} \in Intermediate \end{math} \newline \begin{math} \{1.5<\Delta\leq2.0\} \in Hard \end{math} & Random. Uniformly distributed between groups.\\
%      \hline
%      Number of points & \begin{math}  \mathbb{N}: [10, 20] \end{math} & - & Random. Uniform distributed \\ 
%      \hline
%     \end{tabular}
% \end{table}