\documentclass[sigconf]{acmart}

% \usepackage{graphicx}
% \usepackage{float}
% \usepackage{url}
% \usepackage{colortbl}
% \usepackage{booktabs}
% \usepackage{caption}
% \usepackage{subcaption}
% \usepackage{hyperref}
% \usepackage{appendix}
% \usepackage{soul}

\AtBeginDocument{%
  \providecommand\BibTeX{{%
    \normalfont B\kern-0.5em{\scshape i\kern-0.25em b}\kern-0.8em\TeX}}}

\copyrightyear{2023}
\acmYear{2023}
\setcopyright{acmlicensed}\acmConference[CHI '23]{Proceedings of the 2023 CHI Conference on Human Factors in Computing Systems}{April 23--28, 2023}{Hamburg, Germany}
\acmBooktitle{Proceedings of the 2023 CHI Conference on Human Factors in Computing Systems (CHI '23), April 23--28, 2023, Hamburg, Germany}
\acmPrice{15.00}
\acmDOI{10.1145/3544548.3581416}
\acmISBN{978-1-4503-9421-5/23/04}

\newcommand{\fix}[1]{\textcolor{red}{\textbf{\textit{#1}}}}
\newcommand{\gh}[1]{\textcolor{blue}{\textbf[[GH: #1]]}}
\newcommand{\ct}[1]{\textcolor{purple}{\textbf[[CT: #1]]}}
\newcommand{\ar}[1]{\textcolor{torquise}{\textbf[[AR: #1]]}}
\newcommand{\ds}[1]{\textcolor{apricot}{\textbf[[DS: #1]]}}
\newcommand{\add}[1]{\textcolor{black}{#1}}
\newcommand{\remove}[1]{\textcolor{red}{\sout{#1}}}

\begin{document}

\title{Measuring Categorical Perception in Color-Coded Scatterplots}

\author{Chin Tseng}
\affiliation{%
   \institution{University of North Carolina at Chapel Hill}
   \city{Chapel Hill}
   \state{NC}
   \country{USA}}
\email{chint@cs.unc.edu}

\author{Ghulam Jilani Quadri}
\affiliation{%
   \institution{University of North Carolina at Chapel Hill}
   \city{Chapel Hill}
   \state{NC}
   \country{USA}}
\email{jiquad@cs.unc.edu}

\author{Zeyu Wang}
\affiliation{%
   \institution{University of North Carolina at Chapel Hill}
   \city{Chapel Hill}
   \state{NC}
   \country{USA}}
\email{zeyuwang@cs.unc.edu}

\author{Danielle Albers Szafir}
\affiliation{%
   \institution{University of North Carolina at Chapel Hill}
   \city{Chapel Hill}
   \state{NC}
   \country{USA}}
\email{danielle.szafir@cs.unc.edu}

\begin{abstract}
Scatterplots commonly use color to encode categorical data. However, as datasets increase in size and complexity, the efficacy of these channels may vary. Designers lack insight into how robust different design choices are to variations in category numbers. This paper presents a crowdsourced experiment measuring how the number of categories and choice of color encodings 
used in multiclass scatterplots 
influences the viewers’ abilities to analyze data across classes. 
Participants estimated relative means in a series of scatterplots with 2 to 10 categories encoded using ten color palettes 
drawn from popular design tools. 
Our results show that the number of categories and color 
discriminability within a color palette 
notably impact people's perception of categorical 
data in scatterplots 
and that the judgments become harder as the number of categories grows. We examine existing palette design heuristics in light of our results to help designers make robust color choices informed by the parameters of their data.
\end{abstract}

\begin{CCSXML}
<ccs2012>
   <concept>
       <concept_id>10003120.10003145.10011769</concept_id>
       <concept_desc>Human-centered computing~Empirical studies in visualization</concept_desc>
       <concept_significance>500</concept_significance>
       </concept>
 </ccs2012>
\end{CCSXML}

\ccsdesc[500]{Human-centered computing~Empirical studies in visualization}

\keywords{scatterplot, category, colors}

\maketitle

\section{Introduction}
\label{sec-intro}
%
%\fix{\paragraph{talk about the background and introduce the problem you are solving}}
% \fix{\paragraph{why that problem is important and what previously(if any) has been studied to solve that}}
% \fix{\paragraph{How you solved that problem and high-level finding and result}}
%
Scatterplots 
%is one of the most common techniques to visualize bi-variate data that enables 
enable people to conduct a wide variety of 
%demanding 
statistical tasks~\cite{quadri2021survey}, such as finding outliers \cite{saket2018task, li2010size}, estimating mean values~\cite{gleicher2013perception, hong2021weighted, kim2018assessing}, and 
%interpreting data~\cite{quadri2021survey}.
assessing correlation~\cite{yang2018correlation,rensink2010perception}. 
Multiclass scatterplots leverage people's abilities to attend to different subsets of information in order to compare patterns across different categories of data. 
When the number of categories 
%of variables arise (i.e, multi-class data), 
increases, 
%selecting a proper visualization becomes a severe challenge
people's abilities to analyze patterns across categories may degrade~\cite{blasius1998visualization}.
%\fix{might need to review the 2nd sentence}
%Combing additional visual encoding such as color palettes together with scatterplots is a typical way of presenting multi-class data~\cite{sarikaya2018scatterplots}, resulting in \emph{multiclass scatterplot}.
However, certain scatterplot designs may be more robust to larger numbers of categories than others. 
Determining robustness is challenging as the perception of multiclass scatterplots 
%is more complex since it 
requires 
%building abstraction and aggregation 
first identifying points from relevant categories and then estimating values from those points across various visual encodings~\cite{gleicher2013perception}.

%Existing studies have focused on the perceptual impact of design choices among
Existing studies provide guidance for supporting a range of tasks in
%visual channels in 
general scatterplots~\cite{quadri2021survey}, such as cluster estimation~\cite{sedlmair2012taxonomy,quadri2020modeling,quadri2022automatic}, 
%mark shape~\cite{burlinson2017open}, 
or for tuning across channels such as 
color differences~\cite{szafir2018modeling} and point size~\cite{hong2021weighted,li2010size}.
However, little attention has been paid to 
%exploring guidelines of 
design choices for rendering complex multiclass scatterplots and how such design choices may change as the number of categories increases.
%For this objective, 
Color palettes~\cite{munzner2014visualization, stone2014engineering} and shapes~\cite{burlinson2017open} are commonly used to delineate categories in scatterplots, but available design guidance for effectively supporting categorical tasks is largely heuristic rather than empirical, which raises questions as to the robustness and precision of this guidance across a range of scenarios.

Gleicher et al.~\cite{gleicher2013perception} and Burlinson et al.~\cite{burlinson2017open}
%initially conducted several experiments for assessing visual aggregation via exploring the accuracy of human 
offer preliminary experimental insight into the robustness of different visual channels on 
%the perception of means
mean estimation in multiclass scatterplots. However, these studies focus on scatterplots with two to three classes, where as we 
%are measuring the effect of 2-10 numbers of 
measure the effect of 2--10
categories across different
color palettes. 
%in perceptions of multiclass scatterplots in this study.
%Based on the empirical results, they suggested scatterplots are capable to reveal the inter-class differences and viewers can extract that information.
%Haroz and Whitney found the number of categories heavily impacts the accuracy of human perception in categorical visualizations.
The number of categories heavily impacts people's abilities to reason across categories~\cite{haroz2012capacity}, especially for color, which remains the default channel for encoding categorical data in many popular commercial applications~\cite{shmueli2011data,mackinlay2007show}. 
%However, Gleicher et al.'s study mainly focused on simple 2-classes multiclass scatterplots (with exception of some 3-class scatterplots) without considering the impact of category numbers.
Existing studies~\cite{healey1996choosing, gramazio2016colorgorical, wang2018optimizing} 
%suggested there are severe impacts on selecting 
highlight both the importance and complexity of selecting proper color palettes for categorical visualizations. Despite the popularity of using color palettes in categorical visualizations, we lack insight into how robust these palettes are to the number of presented categories and as to whether that robustness varies across different parameters of a palette. Such information is critical for effectively communicating data, especially as the size and complexity of data continue to grow. 
%However, Gleicher et al.'s study only chose a same set of two or three colors in all of their experiments, thus the impact of color palettes on categorical perception was dramatically ignored.

%To overcome the above drawbacks and provide a more comprehensive design guideline, we thus conduct
We conducted a crowdsourced study 
%to explore human perception on more complex multiclass scatterplots via using the same \textit{mean-judgment} task.
to measure how robust different palette designs are to increasing numbers of categories. 
%To better assess the impact of visual encodings, we considered 5 independent visual factors in scatterplots: 
Participants estimated mean values in a series of multiclass scatterplots with varying numbers of categories (2--10), dataset sizes (10--20 points per category), and color palettes drawn from popular visualization tools.
%\emph{numbers of categories}, \emph{hardness levels}, \emph{point distribution}, \emph{numbers of points}, and \emph{color palettes} (see \autoref{sec-factors} for details).
%Prior to the formal study, we first conducted three pilot studies to test and decide the proper parameters of \emph{numbers of categories}, \emph{hardness levels}, and \emph{point distribution} for scatterplots in stimuli generation.
%Based on the pilot studies' results, we chose the ranges of the above 3 factors for generating multiclass scatterplots.
%Besides, we considered varying \emph{numbers of points} from 10 to 20 per category and chose 10 pre-defined design-crafted \emph{color palettes} in the stimulus.
%Using the above stimulus, we conducted an extensive crowdsourced experiment with 81 participants to evaluate human perception in mean judgment estimates on multiclass scatterplots.
%The research goals of our experiment are
%Our results enable us to characterize palette robustness with increasing numbers of categories. We deconstructed our results with respect to common parameters of color encodings to find potential cues for robust palette design. 
%to human perception of mean judgment with different visual encodings and to evaluate at which condition the visual encodings would cause perceptual biases.
%Our results demonstrate the influence of different visual factors on category perception in multiclass scatterplots.
We found that both the number of categories and choices of color palettes significantly impact people's abilities to estimate category means. 
%, 2) there's a solid cue between hardness level and perceptual accuracy, and 3) the perceptual impact of the number of points is minor.
%In addition, we convince 
We deconstructed our results with respect to common parameters of color encodings to find potential cues for robust palette design and find preliminary evidence that subitizing may impact categorical estimates~\cite{phillips1977components, haroz2012capacity,nothelfer2017redundant,nothelfer2019measures}.

\noindent \textbf{Contribution:} The primary contribution of our paper is evaluating 
%human perception for mean judgments on complex
mean estimation in multiclass scatterplots with varying color palettes.
% visual encodings
Our results 
%replicate previous findings about hardness level and points number~\cite{gleicher2013perception} and $subitizing$~\cite{haroz2012capacity}, and suggest novel disclosures about the impact of 
characterize the effect of the numbers of categories and color palette in perceptions of multiclass scatterplots.
%, and their interaction effects.
%The findings allow a deeper understanding of human perception in complex multiclass scatterplots with various visual encodings, especially for plots with many categories.
Our findings challenge current guidelines on multiclass scatterplot design~\cite{gleicher2013perception}, and we present an exploratory analysis of key factors for effective color palette design.  
%therefore we formalize our results as a new set of general guidelines for designing multiclass scatterplots.



\section{Related Work}
\label{sec-related}

Visual encodings in multiclass scatterplots significantly affect people's ability to interpret categorical data correctly.
However, we still do not understand the perceptual impact of encoding choices across varying numbers of categories.
We briefly review the topics of graphical perception in scatterplots, color palette design, and tasks in scatterplots to ground our work.

\subsection{Graphical Perception in Scatterplots}

Understanding categorical perception is a fundamental task in both cognitive science~\cite{harnad2003categorical} and visualization~\cite{munzner2014visualization}. Past work has introduced a range of techniques for eliciting patterns in categorical data, such as Flexible Linked Axes~\cite{kosara2006parallel}, Parallel Sets~\cite{lex2010comparative}, and Matchmaker~\cite{claessen2011flexible}.
However, these techniques leverage specialized approaches with high learning costs, 
%and required knowledge base make it hard to employ these techniques 
making them difficult for lay audiences to work with.
Scatterplots, alternatively, are more familiar for many audiences and commonly encode categorical data~\cite{sarikaya2018scatterplots}. Consequently, 
understanding how to best design scatterplots for categorical datasets is essential for effective data communication.
%the effectiveness of visual design for scatterplots is crucial including evaluating the graphical perception.
%since scatterplot is still the most commonly used categorical visualization technique~\cite{sarikaya2018scatterplots}.
%We mainly review perceptual studies in scatterplots and a complete review of such studies in different visualizations is beyond the scope of this paper, we refer to Quadri and Rosen~\cite{quadri2021survey}.

Graphical perception studies investigate how effectively people can estimate different properties from visualized data (see Quadri \& Rosen~\cite{quadri2021survey} for a survey). 
%Graphical perception in scatterplot has been studied for decades since Cleveland and McGill~\cite{cleveland1984graphical, cleveland1986experiment} analyzed several low-level judgment tasks in bi-variate graphs.
Scatterplots are commonly used in graphical perception experiments as they are sufficiently complex to reflect real-world challenges and simultaneously sufficiently simple to control~\cite{rensink2014prospects, rensink2010perception, harrison2014ranking, kay2016beyond}.   
%Heer and Bostock~\cite{heer2010crowdsourcing} further confirmed that Weber’s Law can capture perceptual precision in graphical perception problems.
Existing studies have analyzed how scatterplots can support a variety of perceptual tasks across a range of channels. 
For example, Kim \& Heer use scatterplots as a means to assess how different visual channels support 
%a range of 
various tasks~\cite{kim2018assessing}.
Hong et al.~\cite{hong2021weighted} found that varying point size and lightness can lead to perceptual bias in mean judgments in scatterplots. Scatterplot studies commonly investigate how design influences people's abilities to estimate aggregate statistics, such as correlation~\cite{harrison2014ranking,rensink2010perception,kay2016beyond}, clustering~\cite{quadri2022automatic,sedlmair2012taxonomy,quadri2020modeling}, and means~\cite{hong2021weighted,gleicher2013perception,wei2019evaluating,whitlock2020graphical}. 
Other studies model the influence of different channels on scatterplot design, such as opacity~\cite{micallef2017towards}, color~\cite{szafir2018modeling}, and shape~\cite{burlinson2017open}. 

Most graphical perception studies focus on statistical relationships within a single category of scatterplots. However, studies of multiclass scatterplots often characterize people's abilities to separate classes by measuring just-noticeable differences in categorical encodings~\cite{smart2019measuring,burlinson2017open}. Alternatively, Gleicher et al.~\cite{gleicher2013perception} studied how different categorical encodings influenced people's abilities to compare the means of different classes 
%relative mean values of different classes were perceived in multiclass scatterplots. They conducted a crowd-sourced user study to compare the accuracy of mean value judgments 
with varying numbers of points and
%$hardness$ levels (denoted by $\Delta$ and computed by the pixel distance between classes' centers), 
differences in means, colors, and shapes.
%of scatterplots. 
%Their results revealed that hardness level and color are the most significant factors that impact human judgment, while the numbers of points are less influential.
%Based on the empirical results, they suggested the following guidelines: 1) scatterplots are capable to reveal the inter-class differences and viewers can extract that information, 2) conflicting cues do not hinder performance in the assessment of aggregates, and 3) since additional classes have little impact on performance, multi-class scatterplots should not be exempted in visualization design.
They found that scatterplots can effectively reveal interclass differences and that the design of a scatterplot influenced people's abilities to compare classes, with color being the strongest categorical cue. However, in contrast to other work on categorical visualization~\cite{haroz2012capacity}, they found that increasing the number of classes from two to three did not decrease performance.  
%Prior work in categorical visualization indicates that 
%However, they neither studied complex multiclass scatterplots with more than 3 classes nor took various color palettes into consideration.
%Since color palettes~\cite{zhou2015survey} and the number of classes~\cite{haroz2012capacity} both are important factors for human perception in visualizations, in this paper, we conduct a crowd-sourced experiment to evaluate various factors in multiclass scatterplots.
%Our studies focus on complex multiclass scatterplots from
We build on these observations to explore how robust people's estimates are in scatterplots with between 2 and 10 classes with varying hardness levels, color palettes, and numbers of points, (see Section \ref{sec-methodology}) to more deeply understand factors involved in effective multiclass scatterplot design. 
%aiming to provide a guideline with the human accuracy rate based on the number of categories, color palettes, and other variables such as hardness level (see \autoref{sec-discussion}).


\subsection{Color Palette Design}
Gleicher et al.'s findings about the effectiveness of color in multiclass scatterplots echo existing design guidance and results from other studies of categorical data encodings~\cite{gleicher2013perception,haroz2012capacity,trumbo1981theory,munzner2014visualization}.
%Color is one of the most common encoding channels in visualization. 
Choosing a proper categorical color palette\footnote{We define a color \emph{palette} as a set of colors specifically designed for categorical data.} for visualizing categorical data is a crucial task~\cite{trumbo1981theory, zhou2015survey}. Designers employ a combination of color models and heuristics to generate palettes (see Zhou \& Hansen~\cite{zhou2015survey}, Kovesi~\cite{kovesi2015good}, Bujack et al.~\cite{bujack2017good}, and Nardini et al.~\cite{nardini2019making} for surveys).
A range of studies has explicitly examined color perception for continuous data, such as characterizing limitations of rainbow colormaps \cite{ware1988color,reda2020rainbows,borland2007rainbow,quinan2019examining}, comparing the task-based effectiveness of continuous colormap designs~\cite{padilla2016evaluating,reda2018graphical,liu2018somewhere}, modeling color discrimination~\cite{ware2018measuring}, examining color semantics~\cite{anderson2021affective}, quantifying the impact of size and shape on encoding perception~\cite{smart2019measuring, szafir2018modeling} and examining perceptual biases~\cite{schloss2018mapping}.
However, significantly fewer studies have characterized color use for categorical data encoding.  

%Various aspects related to color perception have been studied empirically.
%Cleveland and McGill~\cite{cleveland1984graphical} found that compared to size and position, human perception of visual encoding of color channels would be at a lower precision in the most basic bi-variate graphs.
%For multiclass scatterplots, Gleicher et al.~\cite{gleicher2013perception} suggest that in the mean judgment task, color schema shows a stronger cue compared to other visual channels such as shape or orientation.
%Several existing studies focused on the co-effect between color and other visual encodings, such as size~\cite{szafir2018modeling}, shape~\cite{smart2019measuring} and uncertainty-relevant factors~\cite{maceachren2012visual}.
%For more details about color perception, we refer to recent surveys~\cite{zhou2015survey, kovesi2015good, nardini2019making}. 

Several principles and metrics of effective color palette design have been proposed~\cite{brewer1994guidelines,harrower2003colorbrewer,stone2006choosing,gramazio2016colorgorical}. 
%For example, Trumbo~\cite{trumbo1981theory} suggested that i) the order of colors should be comparable if presenting an ordered statistical variable and ii) the difference among colors should be obvious if presenting the differences of a variable.
Past work recommends that color palettes optimize the mapping between data semantics and color semantics~\cite{lin2013selecting,schloss2020semantic,setlur2016linguistic}; select colors that emphasize color harmonies~\cite{stone2006choosing,zeileis2009escaping}, affect~\cite{bartram2017affective}, or pair preference~\cite{schloss2011aesthetic}; and maximize perceptual and categorical separability between colors~\cite{healey1996choosing} (see Silva et al. \cite{silva2011using} for a survey).
%More recently, Zeileis et al.~\cite{zeileis2009escaping} recommended that colors should be attractive and harmonic with each other.
Designers can use predefined metrics to describe aesthetic 
%are some optimizing metrics proposed to measure aesthetic preference such as Pair Preference
(e.g., pair preference~\cite{schloss2011aesthetic}), perceptual (e.g., %or to measure color discrimination, such as Perceptual Distance (
CIEDE 2000~\cite{sharma2005ciede2000}), 
and categorical (e.g., color name difference or uniqueness~\cite{heer2012color})
%, and Name Uniqueness~\cite{heer2012color}.
attributes of color to implement these guidelines and constrain effective palette design. 
While these metrics underlie many palette design guidelines, implementing these guidelines effectively takes significant expertise. 
%We utilize the above-mentioned 3 color discrimination measures in our analysis.
%Additionally, the length or variance on paths of color spaces such as CIELCh~\cite{ihaka2003colour} can be used to model color distributions and differences in a palette.
%CIELCh is an equal space to CIELAB, but with perceptual path representations where $L*$ counts the \emph{lightness} of a color, $C*$ approximates its \emph{chroma}, and $h*$ measures the \emph{hue}.
%Compared to CIELAB ($\alpha*$ counts red-to-green ratio and $\beta*$ counts blue-to-yellow ratio) or RGB spaces, the perception-based CIELCh can better represent the property of colors~\cite{zeileis2009escaping}.
%As a consequence, we utilize the variance and length of the CIELCh paths to evaluate color palettes in our paper.

%Besides, Haroz and Whitney~\cite{haroz2012capacity} analyzed categorical binned images with different colors and found that the $subitizing$ phenomenon~\cite{kaufman1949discrimination,mandler1982subitizing}, can significantly impact the accuracy and responding time in categorical visualization.
%Their findings suggest that 5 may be the capacity limit of human attention in category numbers which could heavily impact the effectiveness of visualization design.

Several methods for creating effective color palettes have been introduced.
For example, Healey~\cite{healey1996choosing} considers linear separability, color difference, and color categorization to design discriminable color palettes.
Harrower and Brewer~\cite{harrower2003colorbrewer} introduced ColorBrewer for providing designer-crafted distinguishable color palettes for cartography.
Gramazio et al.~\cite{gramazio2016colorgorical} developed Colorgorical, which can generate categorical palettes by optimizing several perceptual and aesthetic metrics. 
%Lin et al. ~\cite{lin2013selecting} introduced categorical colors assignment with semantically-resonant colors by mapping values to representative images, and Schloss et al. ~\cite{schloss2020semantic} models semantic space distance on semantic discriminability. Bartram et al. ~\cite{bartram2017affective} and Anderson et al. ~\cite{9318559} proposed proper affective color properties selection can enhance visual communications. 
Recent efforts have also explored how palettes might be extracted from images~\cite{zheng2022image} or colors from a given palette optimally assigned to a visualization~\cite{lee2012perceptually, lin2013selecting,wang2018optimizing}. Tools such as Colorgorical~\cite{gramazio2016colorgorical} and ColorBrewer~\cite{harrower2003colorbrewer} enable people to generate or choose from a range of palette designs (see Zhou \& Hansen~\cite{zhou2015survey} for a survey).  
%Smart et al.~\cite{smart2019color} created effective color encodings based on a corpus of 222 expert-designed seed color ramps. 
%Wang et al.\cite{} combined the impact of spatial relation, density, cluster overlap, and background to assign colors.
In this study, we compare preconstructed palettes from a range of sources, 
%Since our studies do not contain comparisons of color assignment technologies, we choose pre-defined colors from existing designer-crafted palettes 
including ColorBrewer~\cite{harrower2003colorbrewer}, 
%Paul Tol~\cite{tol2012colour}, and Swiss Federal Statistical Office (SFSO)~\cite{sfso} but also from commercial visualization tools~\cite{4376133} like 
Tableau~\cite{tableau}, D3~\cite{6064996}, Stata Graphics~\cite{statagraphics19}, and Carto~\cite{carto} (see \autoref{fig:palettes} for the details of our selected color palettes). Following the model for comparing the effectiveness of continuous color ramps in Liu \& Heer \cite{liu2018somewhere}, we leverage these palettes to understand how effectively common best-practice color palettes encode data over a range of data parameters.  


%\subsection{Tasks in Scatterplot}
% Moved to 2 sentences in Sec. 4.1.

%Amar et al.~\cite{amar2005low} generalized a task-based taxonomy such as cluster, sort, and value retrieval that might impact analytics in information visualization.
%Since then, a number of perceptual studies have been proposed to conduct various low-level tasks to assess the effectiveness of scatterplots, such as assessing trend estimation in multivariate scatterplots~\cite{nguyen2016correlation}, modeling correlation perception with Weber's Law~\cite{harrison2014ranking}, evaluating outlier perception~\cite{sarikaya2018design}, and modeling cluster perception topologically~\cite{quadri2020modeling}.

%Among them, the relative mean judgment task, which is required to estimate the mean position of classes and compare its value in scatterplots, is commonly employed in perceptual experiments to evaluate the differences across multiple classes of points~\cite{sarikaya2018scatterplots}.
%For example, Gleicher et al.~\cite{gleicher2013perception} performed this task to evaluate perception accuracy for basic multiclass scatterplots.
%Karmer et al.~\cite{kramer2017visual} found by comparing the mean and variances of variables over time, people can capture trend information within data.
%Hong et al.~\cite{hong2021weighted} introduced perceptual biases in judging mean positions in scatterplots with varying colors and sizes of points.
%For multiclass scatterplots, the mean judgment task requires  making accurate judgments of mean localization and mean comparison, thus it combines the ability of both value retrieval and sort tasks, and enables to assess of the visual aggregation from human perception~\cite{gleicher2013perception, sarikaya2018scatterplots}.
%As a consequence, we perform the mean judgment task to assess human perception in multiclass scatterplots.


\section{Methodology}
\label{sec-methodology}

We analyzed how the number of categories, number of points, and color palettes used to distinguish various categories impact people's abilities to reason with multiclass scatterplots.
We performed a crowdsourced study measuring how well people were able to compare category means over varying category numbers and color palette designs. This study allowed us to characterize the effect of category number in multiclass scatterplots as well as how robust different color palette designs are across varying numbers of categories. 
%through the relative mean judgments task with the fine-tuned parameters from pilot studies.
%The overall goal of the formal study is to assess the potential cues of each visual encoding to mean judgments and at which level each visual factor would lead to bias in human perception.
We hypothesized that:

\begin{itemize}
\item[\textbf{H1:}] \textbf{Performance will decrease as the number of categories increases.}
%The number of categories would significantly impact the precision of mean judgment.}

As visual information becomes more complex, perception and cognition degrades~\cite{liberman1957discrimination, regier2009language}. Haroz \& Whitney~\cite{haroz2012capacity} found that these findings generalized to categorical visualizations: increasing the number of categories degrades visual search performance. However, Gleicher et al.'s findings contradicted this observation, instead finding no performance difference between two or three category visualizations~\cite{gleicher2013perception}.  
%Since the rising number of categories can make a scatterplot more complex, we expected that these findings would be consistent in multiclass scatterplots.
We expect that for larger numbers of categories, this robustness will likely falter, even with designer-crafted palettes. \add{Existing heuristics recommend that visualizations should not use more than seven colors for reliable data interpretation \cite{munzner2014visualization}. This guidance suggests that we should see drastic performance reductions for seven or more categories.}

\item[\textbf{H2:}] \textbf{The choices of the color palette will affect people's abilities to effectively compare means.}

Perceptual studies demonstrate that color is a strong cue in both visualization~\cite{szafir2018modeling} and categorical perception~\cite{goldstone1995effects}. Past work has shown that, even in unitless data, the choice of color palettes can affect visualization interpretation~\cite{gramazio2016colorgorical,healey1996choosing}. We likewise anticipate that color palette design may differently support varying numbers of categories: some palettes may more robustly distinguish a range of classes than others, especially as the complexity of the palette increases with larger numbers of colors. 
%\add{While we anticipate that palettes maximizing the perceptual distances between colors will perform best, current design recommendations \cite{munzner2014visualization,stone2006choosing} suggest palettes should maximize categorical features of the colors (e.g., hue or name uniqueness \cite{heer2012color}) while minimizing lightness variation, which can bias attention. We anticipate that palettes following these practices will support higher performance. }
%Hence we agreed that the choice of color palettes would impact the judgment accuracy of multiclass scatterplots and expected users would achieve better accuracy rates with more discriminable palettes.
%The discriminability can be measured by some of our above color metrics (see \autoref{sec:metrics}).

%\item[\textbf{H3:}] \textbf{There would be an interaction effect\footnote{A variable that has an interaction effect will have a different effect on dependent variables, depending on the level of a third variable.} among the number of categories, hardness level, and the discriminability of the color palette.}

%Previous study~\cite{smart2019measuring} found that there are interaction effects across various visual encoding factors in scatterplots.
%Hence we expected there would exist interaction effects of our independent variables in multiclass scatterplots.

\end{itemize}
%Gleicher et al.~\cite{gleicher2013perception} shows that there are solid cues for both $\Delta$ value and color palette to impact the accuracy of human perception.
%Haroz and Whitney~\cite{haroz2012capacity} found that there exist capacity limits when drawing different numbers of categories in visualization design.

The anonymized data, results, and infrastructure 
for our study can be found on \href{https://osf.io/wz8eb/?view_only=03db060f94ee42f29f453ed3013e3405}{OSF.}\footnote{https://osf.io/wz8eb/?view\_only=03db060f94ee42f29f453ed3013e3405}

\subsection{Task}


%Among them, the relative mean judgment task, which is required to estimate the mean position of classes and compare its value in scatterplots, is commonly employed in perceptual experiments to evaluate the differences across multiple classes of points~\cite{sarikaya2018scatterplots}.
%For example, Gleicher et al.~\cite{gleicher2013perception} performed this task to evaluate perception accuracy for basic multiclass scatterplots.
%Karmer et al.~\cite{kramer2017visual} found by comparing the mean and variances of variables over time, people can capture trend information within data.
%Hong et al.~\cite{hong2021weighted} introduced perceptual biases in judging mean positions in scatterplots with varying colors and sizes of points.
%For multiclass scatterplots, the mean judgment task requires  making accurate judgments of mean localization and mean comparison, thus it combines the ability of both value retrieval and sort tasks, and enables to assess of the visual aggregation from human perception~\cite{gleicher2013perception, sarikaya2018scatterplots}.
%As a consequence, we perform the mean judgment task to assess human perception in multiclass scatterplots.

Scatterplots have been studied across a range of tasks (see Sarikaya \& Gleicher~\cite{sarikaya2018scatterplots} for a survey). We employed a relative mean judgment task as applied in previous studies~\cite{gleicher2013perception,hong2021weighted,kramer2017visual}. As in Gleicher et al.~\cite{gleicher2013perception}, we asked participants to estimate the category with the highest average y-value. We used this task as it required participants to first find data points of different categories and then estimate statistical values 
%estimated 
over all points in that category. This task is sensitive to both overinclusion (i.e., including points that are not in a given class) and underinclusion (i.e., failing to include points in a given category), meaning that confusion between points of different categories should be reflected in participants' responses. It also represents a basic statistical quantity that most lay participants 
%would be able to complete. 
are able to compute.
%predict accurate judgments of mean localization and comparison among classes, thus it combines the ability of both value retrieval and sort tasks~\cite{amar2005low}, and enables us to assess the efficiency of visual aggregation in perception~\cite{gleicher2013perception}.
%For specific, in each of our questions, participants are required to select a class of points with the highest average y-value of all the others among the whole multiclass scatterplot.
%\autoref{fig:amt-engagement} illustrates 3 examples of our questions based on the mean judgment task.

\begin{figure*}[htbp] 
\vspace{-1em}
\centering
\includegraphics[width=0.7\textwidth]{Figure/color_palettes.pdf} 
\vspace{-1em}
\caption{The 10 color palettes used in our experiment.} 
\label{fig:palettes}
\end{figure*}


\begin{figure*}[htbp] 
\centering
\includegraphics[width=0.8\textwidth]{Figure/engagement.pdf} 
\caption{Three engagement checks with D3 color palettes. Participants were required to pass two out of these three tasks to be considered as an approved response. All engagement checks were placed in random order with other formal trials.} 
\label{fig:amt-engagement}
\end{figure*}

%\subsection{Metrics}
%\label{sec:metrics}


%\subsection{Hypotheses}

\subsection{Stimuli Generation}
\label{sec-stimuli-generation}

\begin{table*}[htbp] 
\centering
\caption{The experiment parameters. We refined the factors and domain range in three pilot studies. Category number and color palettes are our independent variables, and hardness level and point number are the control variables. The experiments were built from the combination of these four factors.
} 
\includegraphics[width=0.9\textwidth]{Figure/factor-table-clean.pdf} 
\label{tab:parameters}
\end{table*}

Participants estimated means for a series of scatterplots. We generated each scatterplot as a 400x400 pixel graph using D3~\cite{6064996}. Each scatterplot was rendered to white background and two orthogonal black axes with 13 unlabeled ticks. For every point, we 
%chose a fixed size that is 3 pixels in radius among all the tasks.
rendered a filled circle mark with a three pixel radius.
\add{We selected three pixel points based on internal piloting to ensure that points were distinguishable between classes while also minimizing the need to address overdraw and reflecting design parameters commonly seen in real-world visualizations. }
%We used Gaussian distribution to generate our set of points for the experiments. 

%We then introduce several metrics employed in our studies to measure multiclass scatterplots.

%\subsubsection{Hardness Level ($\Delta$).}

As shown in \autoref{fig:palettes}, we selected 10 qualitative color palettes: ColorBrewer/Paired~\cite{harrower2003colorbrewer}, ColorBrewer/Set3~\cite{harrower2003colorbrewer}, D3/Category10~\cite{6064996}, Tableau/Tab10~\cite{tableau}, Paul Tol/Muted~\cite{tol2012colour}, SFSO/Parties~\cite{sfso}, \newline  Stata/S1~\cite{statagraphics19}, Stata/S2~\cite{statagraphics19}, Carto/Bold~\cite{carto} and Carto/Patel~\cite{carto}. These color palettes were chosen from popular visualization tools that provide 
%over 
at least 10 categorical colors in a single palette. If there were more than 10 colors in a certain palette, we used the first 
%ten as our colors. 
10 as the palette's colors.
In each scatterplot, colors were randomly selected from the target palette and 
%rendered
mapped to corresponding categories. While some tools prescribe a fixed order to the selection of colors from a palette, this is not a universal design practice. 
%Therefore, we minimized potential effects from variations in this practice by randomization. 
%\add{We used a random subset to avoid the original palettes' sequential impact (i.e., a palette might be optimized for its unique subjective sequence). 
\add{Randomization helps avoids potential bias from differences beyond color selection as not all palettes may have been intentionally ordered, but future work should investigate differences in the ordered application of palettes.}
%However, some tools encoded colors in a fixed order for multiclass scatterplots, but we managed colors as random factors in our analysis.


We tuned our dataset parameters in a series of three extensive pilot studies, measuring performance for varying numbers of categories, points, and hardness levels (see Appendix for details). As in Gleicher et al.~\cite{gleicher2013perception}, we controlled task hardness using the distance between classes. The hardness level is denoted by $\Delta$ and is calculated by the 
%numerical 
distance between y-means of classes in multiclass scatterplots. 
To generate 
%the x-y 
positional data with the given mean and covariance, we used a function from Numpy~\cite{oliphant2006guide} that randomly samples from a multivariate normal distribution. We denoted our data points as \begin{math} \{ x,y \in \mathbb{R} \, | \, 0<x,y<10 \} \end{math}. First, we randomly sample the mean \begin{math} \mu{_1} \end{math} in the range [5, 9] for the category that possesses the highest mean, then set the mean \begin{math} \mu{_2} = \mu{_1} - \Delta\end{math} as the second highest mean based on y-values. To prevent subsequent means from drifting too far apart and artificially simplifying the task, we constrained the mean \begin{math} \mu_i \end{math} of the rest of the categories to \begin{math} \Delta < \mu{_1} - \mu_i < 1.5\Delta \end{math}. Finally, we determined the covariance for each category that has y-mean \begin{math} \mu_i \end{math} with \begin{math} cov(\lambda_i,  \lambda_i) \end{math} where \begin{math}\lambda_i=random(1, min(\mu_i, 10-\mu_i))\end{math}. We used this variance to tune the datasets such that selecting the category with the highest point did not reliably produce the correct answer, with variance tuned in piloting.  

Each scatterplot contained between 10 and 20 points per category. 
To prevent overlapping points, we applied jittering methods which add random noise to any data points that would otherwise overlap each other.
We generated 450 datasets in total.
%\add{Please see \autoref{tab:parameters} in Appendix for detailed numbers of scatterplots per color palettes and categories.}

\begin{figure*}[htbp] 
\centering
\includegraphics[width=0.8\textwidth]{Figure/hardness-stimuli.pdf} 
\caption{Instances with varying hardness level ($\Delta$) values employed in our study. The difficulty level of instances varies from easy to hard from left to right, with four categories.} 
\label{fig:delta-stimuli}
\end{figure*}


%and dividing them into 10 sets. Each set contains 45 tasks, including 3 engagement checks and 42 formal trials. 

\begin{figure*}[htbp] 
\centering
\includegraphics[width=0.8\textwidth]{Figure/catnum-stimuli.pdf}
\caption{Instances with varying numbers of categories employed in our study. Their numbers of categories are 3, 6, and 9 respectively from left to right, with the same hardness level (intermediate).} 
\label{fig:num-stimuli}
\end{figure*}

\subsection{Procedure}
Our experiment consisted of three phases: (1) informed consent and color-blindness screening, (2) task description and tutorial, and (3) formal study.
At the beginning of the study, participants were provided with informed consent in accordance with our IRB protocol. They were then asked to complete an Ishihara test for color-blindness screening~\cite{hardy1945tests}. 
%, in which we selected 6 tests that participants were required to pass all to continue the study. 
After completing the screening successfully, participants were led to a description page for the mean judgment task. They were required to successfully complete an easy tutorial question to minimize possible ambiguities in their understanding of the task.

During our formal study, each participant completed our target task (\textit{Identify the class with the highest average y-value}) for 45 stimuli presented sequentially using a single color palette (42 formal trials and three engagement checks). We used stratified random sampling to balance number of categories and difficulty levels that each participant saw. 
To ensure participants saw a range of category numbers, we grouped category numbers into three classes: small, medium, and large, which corresponded to 2-4, 5-7, and 8-10 categories, as shown in \autoref{fig:num-stimuli}. 
%We used these grouping to perform a stratified sampling of category numbers for each participant, with each participant seeing 14 tasks within each category group. 
We also grouped stimuli into three difficulty levels: easy, intermediate, and hard, as shown in \autoref{fig:delta-stimuli}. Each person saw 14 stimuli from each category and difficulty group, with combinations of category and difficulty assigned at random. 
%Our datasets included all combinations of category numbers and difficulty levels. The point number 
%The number of points was uniformly distributed over a range from 10 to 20 per category. Each participant saw 45 tasks from one set, which was randomly assigned from pre-generated 10 sets when they started the study. There was only one color palette being used for each participant over the entire experiment, which was also randomly assigned at the beginning of the study. All the tasks were presented in random order.

We randomly placed three engagement checks within 42 formal trials to assess if participants were inattentive during the test.
These engagement checks presented three classes with large differences in their means (c.f., Figure~\ref{fig:amt-engagement}). 
%Figure~\ref{fig:amt-engagement} illustrated examples of these questions. 
%i.e., the engagement tasks employed in our study.
We randomly ordered the sequence of the formal questions and the engagement checks to avoid learning or fatigue effects.
%After completing all the formal tests, each participant was required to report their demographics and received \$3.00 compensation.



\subsection{Participants}
We recruited 95 participants from the US and Canada with at least a 95\% approval rating on Amazon Mechanical Turk (MTurk). We excluded four participants who failed more than one engagement check. We analyzed data from the remaining 91 participants (46 male, 45 female; 24--65 years of age).
%The ages of our participants were between 24 and 65, on average at 45.3 with a standard deviation of 10.7. 
All participants reported normal or corrected to normal vision. 
%or wearing corrected glasses.
Our experiment took about 15 minutes on average, and each of the participants was compensated \$3.00 for their time.

\subsection{Analysis}

%The overall goal of our analysis is to validate our hypotheses and some findings from previous studies and capture the differences among different visual encodings that impact human judgment.

%During our main study, we calculated the accuracy rate for each participant and the corresponding time spent on each question, and combined these results with different combinations of our employed independent visual factors.
We measured performance as both accuracy and time spent on task.
%For each of our independent variables, we conduct the analysis of variance (ANOVA) independently to test for variations in the impact of judgment accuracy of different visual encodings.
We analyzed the resulting data using a 10 (color palette) x 9 (number of categories) mixed-factors ANCOVA, with the number of points, interparticipant variation, trial order, and hardness levels as random covariates. 
%Besides, we analyzed the interaction effects across different combinations of independent variables through the analysis of covariance (ANCOVA). First, We treated delta values as random effects, which that showed the number of points has a small effect on performance. As a result, we conducted ANCOVA with trial order, delta values, and number of points as random effects, and number of category and color palettes as covariates.
During our post-hoc analysis, we employed the Tukey's honestly significant difference test (Tukey's HSD) with $\alpha$ = 0.05 and Bonferroni correction. 

% \fix{Detailed factors for ANOVA and ANCOVA analysis, such as how to combine factors, use which random effect, finally computed a ?-level (2, 3, 4, ...) ANCOVA}

\section{Results}
\label{sec-results}

We discuss significant results and statistical analysis based on the independent factors considered in this paper (see Appendix) using both traditional inferential measures and 95\% bootstrapped confidence intervals ($\pm$ 95\% CI) for fair statistical communication~\cite{dragicevic2016fair}. Table \ref{tab:ancova-result} summarizes our ANCOVA results.
%Table~\ref{tab:} shows an overall illustration of the results of all of the independent visual factors and the corresponding interaction effect.
Additional results, charts, and details of the analysis can be found on Appendix. 
%in the supplemental materials.

%We failed to find any significant effects of the number of points (its $p$-value is $0.09$ which only reflects a marginal effect, see \autoref{tab:ancova-result} and \autoref{fig:all} (e) \& (f)) and found the predicted effect of hardness level (when $\Delta$ increase, the performance decrease, see \autoref{fig:all} (c) \& (d), with $F(1, 89)=31.2187, p<.0001$). These findings align well with previous study~\cite{gleicher2013perception}, so we included these factors as random covariates in our ANCOVAs.

\begin{figure*}[htbp] 
\centering
\includegraphics[width=0.8\textwidth]{Figure/main-acc-time.pdf}
% \vspace{-1em}
\caption{Our primary results with respect to the numbers of categories, hardness level, and numbers of points. Graphs on the left show changes in accuracy, whereas those on the right show response times Both accuracy and time do not systematically vary with the number of points. However, as the number of categories grows or the hardness level increases, the overall accuracy rate drops, and the time spent escalates. In order to show the trend clearly, we used a scale from 50--100\% (chance at our smallest number of categories to perfect performance) on the y-axis for accuracy. Error bars represent 95\% confidence intervals.} 
\label{fig:all}
% \vspace{-1em}
\end{figure*}

\subsection{Number of Categories}
\label{sec-analysis-cat}

Our results support \textbf{H1}:
we found that performance decreased as the number of categories increased. 

Our analysis reveals a significant effect of category number on judgment performance ($F(8, 82)=7.6511, p<.0001$): people were both less accurate and slower with higher numbers of categories.
\autoref{fig:all} (a) shows that accuracy rate decreases based on the number of categories from 96.4\% to 86.6\%, 
with an overall descending trend as the number of categories increases.
\autoref{fig:all} (b) presents the average spent time broken down by category number,
suggesting that participants were slower for scatterplots with more categories.

We also found anomalies in the accuracy rate
%occurs with strange $bumps$ when the category number is 5 or 6 in 
for between five and six categories (\autoref{fig:all} (a)).
While we initially assumed this anomaly to be noise, the pattern was repeated across almost all palettes. 
This category number correlates with past findings of \textit{subitizing}---the ability to instantly recognize how many objects are present without counting---in categorical data from Haroz \& Whitney~\cite{haroz2012capacity}. 
%which may cause a significant descent in the performance with more than 5 classes of objects in color-coded binned images by Haroz and Whitney~\cite{haroz2012capacity} are quite similar to our results.
%Hence we hypothesize this might cause by the $subitizing$ phenomenon.
%We would not confirm it as a finding because of the limited dataset.
While we do not confirm this hypothesis in this study, our results do raise questions about the role of subitizing or a related mechanism in categorical reasoning with visualizations.  

\begin{table}[htbp] 
\centering
\caption{ANCOVA results. Significant effects are indicated by \textbf{bold} text and the corresponding rows are highlighted in green.}
\includegraphics[width=\linewidth]{Figure/ancova-table.pdf} 
\label{tab:ancova-result}
\end{table}

% \subsection{Hardness level $\Delta$}
% \label{sec-analysis-delta}

% \autoref{fig:all} (c) illustrates the average accuracy rate of different difficulty levels, where easy refers to \begin{math}  2.5<\Delta\leq3.0 \end{math}, intermediate means \begin{math}  2.0<\Delta\leq2.5 \end{math}, and hard is \begin{math}  1.5<\Delta\leq2.0 \end{math}.
% The result clearly reveals that with the $\Delta$ value decreases, its relevant average accuracy rate descends respectively.
% This finding can also be convinced by ~\autoref{fig:all} (d), which shows that participants spent more time on questions with higher hardness levels.
% Furthermore, the $F$-value of the impact of hardness level in our ANOVA analysis is $F(1)=31.2187, p<.0001$ as shown in \autoref{tab:ancova-result}, as a result, it suggests a strong effect.

% Both results reveal that the hardness level is closely related to the accuracy of perception and $\Delta$ is useful to measure the hardness of the mean judgment task in multiclass scatterplots, hence confirming Gleicher et al.~\cite{gleicher2013perception}'s corresponding finding about $\Delta$.



% \subsection{Number of Points}
% \label{sec-analysis-points}

% \autoref{fig:all} (e) presents the relations between the number of points and the average accuracy rate.
% The results show that with the number of points increasing, its corresponding accuracy rate varies from about 95\% to 85\%.
% But there's no obvious trend correlation between number points and accuracy rate.
% According to \autoref{fig:all} (f), the response time of participants varies little with varying numbers of points.
% Similarly, its $p$-value is $0.086$ in \autoref{tab:ancova-result}, which cannot reveal an obvious impact.
% So we suggest that the number of points would not significantly impact human perception accuracy, thus validating the finding of prior study~\cite{gleicher2013perception}.

\begin{figure*}[htbp] 
\vspace{-1em}
\centering
\includegraphics[width=0.8\textwidth]{Figure/color-acc.pdf} 
\vspace{-1em}
\caption[]{The accuracy rates based on the number of categories separated per color palette, sorted by average accuracy over all categories (dash lines) sorted from most to least accurate. Color palettes are shown along with corresponding charts. See Section \ref{sec-analysis-color} for detailed analysis \add{and \autoref{tab:parameters} in the Appendix for the count of scatterplots per palette.}
} 
\label{fig:palettes-acc}
\end{figure*}

\subsection{Color Palettes}
\label{sec-analysis-color}

% \begin{table}[htbp]
%     \centering
%     \caption{Three categories of color palettes of ANOVA analysis.}
%     \label{tab:color-classes}
%     \begin{tabular}{ |p{3cm}||p{2cm}|p{2cm}|p{2cm}|p{2cm}|  }
%      \hline
%      Color Palette &Class A&Class B&Class C&Least Sq Mean\\
%      \hline
%      SFSO Parties    & \checkmark  & \checkmark & & 0.95\\
%      ColorBrewer Set3& \checkmark  & \checkmark & & 0.94\\
%      Stata S2        & \checkmark  &            & & 0.94 \\
%      Tableau Tab10   & \checkmark  & \checkmark & & 0.94 \\
%      D3 Cat10        & \checkmark  & \checkmark & & 0.92\\
%      Carto Bold      & \checkmark  & \checkmark & & 0.92\\
%      Carto Pastel    & \checkmark  & \checkmark&\checkmark&0.90\\
%      ColorBrewer Paired &   &  \checkmark&\checkmark&0.88\\
%      PaulTol Muted      &   & &\checkmark&0.84\\
%      Stata S1           &   & &\checkmark&0.83\\
%      \hline
%     \end{tabular}
% \end{table}

Our results also support \textbf{H2}: color palettes significantly affect accuracy ($F(9, 81)=8.4689, p<.0001$, see \autoref{tab:ancova-result}). 
%where we found color palettes seriously impact performance.
%The significance analysis reveals a severe impact on the choices of color palettes .
We 
%also 
found a significant interaction effect between color palettes and the number of categories for both time and accuracy. 
%with $p=.0003$.
In other words, as the number of categories increases, the accuracy ranks between color palettes might be different. Different palettes are more or less robust to increasing the number of categories. This finding indicates that there is no best palette
%It indicates that there is no magic choice of color palettes in
for multiclass scatterplots. Instead, our results provide guidance for designers to select effective palettes based on the number of categories in their data. 
%and to achieve the best performance, we should choose specific color palettes to fit with the target data.
%This result also suggests there is no magic choice of color palettes that can fit well for all kinds of multiclass scatterplots.
%Color palettes within each class are closely related.
%Class A \& B share a large proportion (6 of 7) of the same palettes, which suggests those two classes might be similar.
%And Class A represented the highest performance in the accuracy of human judgment, Class B performed only slightly lower than A, and Class C performed the worst.

%We further created palette-specific results of accuracy, 
\autoref{fig:palettes-acc} shows the
%line charts of 
accuracy rate and category number per color palette. 
%\add{We skip plotting the error bars while separating the results by color palettes and category numbers, there are not sufficient numbers for showing the meaningful error bar at each point.}
These charts reveal that:

\begin{enumerate}
    \item \emph{SFSO Parties} and \emph{ColorBrewer Set3} achieved the highest average accuracy rate in all data, whereas \emph{PaulTol Muted} and \emph{Stata S1} exhibited the worst overall performance (an 11.3\% accuracy difference on average between \emph{SFSO Parties} and \emph{Stata S1}),
%where we assume the low accuracy rate in 8 categories of \emph{SFSO Parties} is an outlier because of the high accuracy of all the rest categories,

\item lower performing palettes tend to be less robust to increasing the number of categories, and 

\item most palettes show an overall descending trend as the number of categories increases, though some palettes remained relatively robust (e.g., \emph{Stata S2}, \emph{D3 Cat10}).

\end{enumerate}
%, which is consistent with hypothesis \textbf{H1},

%4) \emph{D3 Cat10} shows a unique increasing trend when the category number is larger than 6 hence it finally ranks top for larger categories, which partially support  \textbf{H3}, and

%4) there are significant performance differences across palettes, e.g., the accuracy difference between \emph{SFSO Parties} and \emph{Stata S1} is more than 12 \%. 

%We further studied the relation between the property of color palettes and judgment accuracy to find whether the selected metrics can measure this difference.

\begin{table}[htbp] 
\centering
% \vspace{-1em}
\caption{Three performance classes of color palettes from Tukey's HSD. Performance is rated better to worst from Class A to C respectively. }
% \vspace{-1em}
\includegraphics[width=\linewidth]{Figure/palette-class.pdf} 
% \vspace{-1em}
\label{tab:color-classes}
\end{table}

\subsection{Exploratory Analysis}
\label{sec-analysis-explor}

To better analyze the impact of specific color palettes, we performed a Tukey's HSD with Bonferroni correction to identify significant performance differences between palettes. 
%for the 10 employed color palettes, resulting in 3
The test revealed three \emph{classes} of color palettes with comparable performance, shown in \autoref{tab:color-classes}.
\autoref{fig:class-acc} illustrates the combined accuracy rate of the three classes, in which Class A refers to the best performance, Class B is slightly lower, and Class C is the worst overall. 
All three classes of palettes showed a steady downward trend that is consistent with \textbf{H1}. We use the clusters created by these performance classes to scaffold an exploratory analysis of potential metrics associated with the observed performance differences. 

\begin{figure*}[htbp] 
\centering
% \vspace{-1em}
\includegraphics[width=0.95\textwidth]{Figure/acc-class.pdf} 
% \vspace{-1em}
\caption{The average accuracy rate with different numbers of categories per performance class of color palettes. Charts represent Class A to C from left to right.} 
% \vspace{-1em}
\label{fig:class-acc}
\end{figure*}



% Both the result of Class A and B shows an overall descending tread, and the accuracy of categories 4 and 5 are almost the same while the differences are less than 0.1\% in Class A and less than 1\% in Class B.
% The difference ratio is relatively low so we suggest it might be an outlier and it would not reduce the strength of our hypothesis \textbf{H1}.
% There also shows a slight increase from category 9 to 10 in Class A and B, since the difference is relatively low, we suggest that's because there are too many categories to distinguish and would not lessen the strength of our finding.
% The result of Class C is with huge jitters for all number of categories but also reveals a slight descending overall trend.

%\subsection{Color Metrics}

%Furthermore, we evaluated the mentioned 8 color metrics (see \autoref{sec:metrics}) to assess their consistency with our accuracy results.
%\subsubsection{Color Discrimination.}
We analyzed these classes using eight color metrics associated with palette design to explore the relationship between performance and common design parameters: perceptual distance~\cite{sharma2005ciede2000}, name difference~\cite{heer2012color}, and name uniqueness~\cite{heer2012color} as employed by Colorgorical~\cite{gramazio2016colorgorical} and the magnitude and variances of different dimensions in CIELCh~\cite{zeileis2009escaping} ($L^*$, $C^*$, and $h^*$). 
%, i.e., $L*$ length (lightness), $C*$ length (colorfulness), $L*$ variance (light difference), $C*$ variance (color difference), and $h*$ variance (hue difference).
The 
%mathematical equations for computing 
computations for those metrics can be found in our supplemental materials.
Since we randomly sampled colors in a palette for plots with less than 10 categories (see Section \ref{sec-stimuli-generation}), for each target color palette, we compute those metrics based on the actual colors used in each individual stimulus sampled from the target palette to explore the distribution of these features with respect to performance.

We conducted an ANOVA using these nine measures to assess the impact of each parameter on accuracy (\autoref{tab:metric-significance}).
We found significant effects ($p<0.01$) of $L^*$ variance, $L^*$ magnitude, and all-pairs perceptual distance~\cite{sharma2005ciede2000} and marginal effects ($p<0.10$) of $h*$ variance and $C*$ variance.

\begin{table}[htbp] 
\centering
\caption{Results of significance analysis from color metrics to judgment accuracy.
The right-most column shows plus or minus of the $\beta$-ratio in the OLS linear regression where plus means incremental trend, and minus means decremental trend.
Significant impacts ($p<0.01$) are in \textbf{bold} style and green color. }
\includegraphics[width=\linewidth]{Figure/metrics.pdf} 
\label{tab:metric-significance}
\end{table}

To assess the direction of the effects, we performed an OLS linear regression of each metric and average accuracy. Since the value of $\beta$-ratio (in $Y=\beta X + \epsilon$) of regression differs from the data range of source metrics, we show its 
%plus or minus sign 
directionality in the right-most column in \autoref{tab:metric-significance}, where a plus sign refers to an increasing trend and minus sign means decreasing trend.
%The value of $L*$ length and perceptual distance would have an incremental trend with accuracy, which means 
We found that larger $L^*$ magnitude (lighter colors) and larger perceptual distance both lead to better judgment accuracy, whereas palettes with less lightness variance had better performance. 
%We also found a marginal effect of both chroma and hue variance. Contrary to common design conventions, performance was higher with lower hue variance and chroma variance. 
%would have better performance.
%Additionally, the marginal effects and negative signs of $h*$ variance and $C*$ variance show that larger hue and chroma difference might hinder performance.
%\fix{Not sure should we discuss hue and chroma variance since it's marginal.}

Our findings suggest that palettes leveraging lighter colors, larger perceptual distances, and lower luminance differences led to better performance \add{whereas factors like hue variance or name uniqueness that are conventionally associated with palette design may be less important when considered in isolation}.
The use of lightness in palettes has a tenuous history: some advocate for minimizing lightness variation to avoid biasing attention \cite{munzner2014visualization,kindlmann2002face}, whereas other  guidelines suggest that designers leverage higher contrasts introduced by lightness variations to take advantage of our sensitivity to lightness variations~\cite{rogowitz2001blair}.
%enlarge the luminance difference, such as higher contrast in luminance of colormap would lead to better representations of absolute values in data~\cite{rogowitz2001blair}.
%Nonetheless, our finding may lead to another topic, isoluminant colormap, that creates color palettes by minimizing luminance differences.
%Previous studies found isoluminant colormap is helpful to aid data value comprehension~\cite{kindlmann2002face}.
%and viewers can immediately make choices for palettes containing seven isoluminant colors~\cite{healey1996choosing}. 
Our results provide further support for privileging more isoluminant palettes to avoid directing too much attention to given classes
%with higher luminance which enables advantages in sub-set search task~\
\cite{braithwaite2010visual}.
%Accordingly, our results indicate that the guideline of maxmizing luminance difference might hinder analysis.
%As a consequence, we would recommend designers consider lighter colors, larger perceptual distances, and lower luminance differences in designing color palettes.
However, future work should more systematically explore this hypothesis. 



%This result indicates that 




% \begin{figure}[htbp] 
% \centering
% \vspace{-2em}
% \includegraphics[width=0.8\textwidth]{Figure/box-hcl.pdf} 
% \vspace{-1em}
% \caption{Boxplots with the distribution of $L*$ variance, $h*$ variance, and $L*$ length with 3 classes of color palettes.} 
% \vspace{-1em}
% \label{fig:class-bos}
% \end{figure}


%In particular, as shown in \autoref{fig:class-bos}, we can see the clear difference for the $L*$ variance, $h*$ variance, and $L*$ length metrics between palette classes.
%Most palettes in Class A \& B have a lower value of both $L*$ variance and $h*$ variance than Class C, which suggests that color palettes with slightly lower lightness and hue variance would offer better performance in judgment accuracy.
%But as for $L*$ length, Class A \& B would be slighter larger than C, which may reveal that lighter color palettes might achieve better performance.
%The above results suggest that lightness and hue of colors may be the major impact factor on judgment accuracy, users might perform better to distinguish classes and capture mean values with a palette that contains lighter colors with fewer variances in both lightness and hue.
%The detailed results of the rest metrics can be found in the supplemental materials.


% \subsection{Interaction Effect}
% \label{sec-analysis-effect}

% \begin{table}[htbp]
%     \centering
%     \caption{ $F$-value and $p$-value in the results of our significance analysis. Significant impacts are in \textbf{bold} style.}
%     \label{tab:ancova-result}
% \begin{tabular}{ |c||c|c|c|c| } 
% \hline
% Source & $DF_1$ & $DF_2$ & \emph{F}-value & \emph{p}-value \\
% \hline
% category number & 8 & - & 7.6511 & \textbf{< .0001} \\ 
% hardness level & 1 & - & 31.2187 & \textbf{< .0001} \\ 
% point number & 10 & - & 1.6502 & 0.086 \\
% color palette & 9 & - & 8.4689 & \textbf{< .0001} \\ 
% category number * hardness level & 8 & 1 & 1.7448 & 0.0832 \\ 
% hardness level * color palette & 1 & 9 & 1.1141 & 0.3486 \\ 
% color palette * category number & 9 & 8 & 1.6921 & \textbf{.0003} \\ 
% \hline
% \end{tabular}
% \end{table}

% \autoref{tab:ancova-result} illustrates the results of $F$-value and $p$-value from our analysis, which in an effort to validate the existence of interaction effects between category number, color palettes, and hardness level.
% As our prior results show that the number of points does not have a significant impact on the perception, we conduct the ANCOVA with the number of points as random effects. The results show that there are no obvious effect of hardness level to color palettes ($F(1,9)=1.1141, p=0.35$) and category number ($F(8,1)=1.7448, p=0.08$).
% However, we found significant interaction effects between the color palettes and the number of categories for both metrics ($F(9,8)=1.6921, p=.0003$).
% That is to say, as the number of categories increases, the accuracy ranks between color palettes might be different, which indicates that to achieve the best performance of human judgment accuracy, we may need to choose multiple color palettes for different numbers of categories.
% This result suggests there is no magic choice of color palettes that can fit well for all kinds of multiclass scatterplots.
% The above findings partially validate our hypothesis \textbf{H3}.



\section{Discussion}
\label{sec-discussion}
We measured the impact of the number of categories and choice of color palettes on people's perceptions of multiclass scatterplots.
We find that people are less accurate at assessing category means as the number of categories increases. However, certain palettes may be more robust to variations in category number. 
Our results provide new perspectives on prior findings and offer both actionable design guidance and opportunities for future research. 
%Our results can validate or be in contrast to some prior findings, be derived into design guidelines, and be improved in some future directions.


\subsection{Reflections on Prior Findings}

%Besides the above findings, in addition, our results can indicate, validate, or oppose the following hypotheses or findings proposed by prior studies.

%\fix{we should put all of them together without sub-subsection. We want to say the result validated prior findings without giving too much attention and limiting our findings.}
% \paragraph{Hardness level}

% Gleicher et al.~\cite{gleicher2013perception} proposed the definition of hardness level $\Delta$, and found that $\Delta$ can measure task hardness and the overall performance in 2-class scatterplots.
% The results of our study indicate that $\Delta$ would be a solid cue for accuracy, the accuracy of judgment will descend with the hardness level arises (see \autoref{sec-analysis-delta}), which can extend their findings to multiclass scatterplots with more than 2 classes.


%\paragraph{Number of points}

%Our analysis result cannot indicate a significant correlation between the accuracy rate and the number of points (see \autoref{tab:ancova-result}).
%In other words, increasing the number of points would not significantly reduce the accuracy of judgment even with a large number of categories, which is consistent with the finding of Gleicher et al.~\cite{gleicher2013perception} in 2-class scatterplots.

%\paragraph{Number of categories}
Our study indicates that increasing numbers of categories would lead to descending performance in relative mean estimation (see Section \ref{sec-analysis-cat}).
However, this finding is inconsistent with the proposed guidelines of Gleicher et al.~\cite{gleicher2013perception}, which found little impact of adding additional distractor classes. We anticipate that this contradiction is a result of the number of classes evaluated. For smaller numbers of classes, performance tended to be more robust across palettes. However, as the number of classes increased, performance started to degrade. This finding likely stems from a correlation between performance and color discriminability indicated in our exploratory analysis. As the number of colors increases, it becomes more difficult to ensure colors are spaced apart, especially if maximizing for metrics like pair preference~\cite{schloss2011aesthetic} or otherwise minimizing the number of large hue variations to preserve harmonies \cite{stone2006choosing}. Certain palettes may differently balance this aesthetic and performance trade-off. However, our results indicate that this trade-off is sensitive to the number of categories present in the data. 

\add{Contrary to past heuristics, we found that performance remained relatively high even for more than seven categories. We hypothesize that people may simply be better at this task than they expect: while the task may feel significantly more difficult as the number of categories increases, our visual system may be more robust than expected in working with complex categorical data. Feedback from pilot participants indicated that the tasks felt challenging as the number of categories exceeded four, but these participants, like those in the formal study, still performed well at such seemingly difficult tasks. 
%Performance still decreased, but people performed significantly higher than chance as the category count increased. 
While the tested palettes reflect best practices, our findings challenge existing heuristics around the scalability and utility of color in visualization. A more thorough and rigorous empirical examination of the robustness of categorical perception in visualization generally would benefit a wide range of applications.}
%can be ignored and suggested that "multi-class scatterplots should not necessarily be avoided in favor of simpler ones".

\add{Our results demonstrate a high overall accuracy (more than 85\%) compared to Gleicher et al. ~\cite{gleicher2013perception} ($\sim$ 75\%) even though we tested a larger number of categories. 
Gleicher et al. chose to generate uniformly sparsely distributed classes. 
%, where points from each class were uniformly and sparsely located on the entire scatterplots.
However, such distributions might not be widely applicable; in real-world use cases, 
%researchers are more interested in and find more
people more commonly build insight from densely distributed classes~\cite{van2008visualizing}.
As shown in \autoref{fig:num-stimuli}, we generated more randomly and densely distributed classes to privilege ecological validity.
Consequently, our stimuli are more likely to perform similarly to what people usually see in their daily life, but the clustering structure may have affected task accuracy by, for example, making it slightly easier to group points within categories.
%, thus making our results more practical and meaningful to real-world usage scenarios. 
While we tuned our stimulus difficulty in piloting and our results were consistent across different performance thresholds, providing evidence of their generalizability, these differences raise important future questions as to the impact of different data distributions on categorical palette design. }

% Despite carefully tuning difficulty levels from our pilot studies, as the category number grew, the accuracy rate did not drop as significantly as we expected. Instead, the results show consistently high accuracy over all conditions. different tasks regarding different category numbers or different color encodings. Hence, we learned that people are having overall good performance over hard tasks than common expectations. While people's perceptions of how hard a task is may not always correlate to their performance. However, even though we have demonstrated our findings that the performances are different while using different color palettes and with different number of categories, as we learn that people are good at such tasks that raise the future questions of conducting even more difficult tasks to robust empirical assessment of these kinds of heuristics.}
% Our different class generation decisions could impact this accuracy.


% Our early to e more like results show that there is a strong correlation between the accuracy rate and the number of categories (see \autoref{tab:ancova-result}). Not only does the number of categories impact the accuracy, but the result \autoref{fig:all}) also indicates that as adding more categories does influence the accuracy rate in a certain trend.

%\paragraph{Subitizing}
Part of the difference in results between our findings and Gleicher et al. may stem from differences in perceptual mechanisms present when processing different numbers of categories. 
Our study reveals a significant "dip" in accuracy when the number of categories increased to five or six (see Section \ref{sec-analysis-cat}).
These bumps correlate with a key number of objects for subitizing~\cite{kaufman1949discrimination}: below roughly six objects, we can instantly and precisely detect the quantity of objects present, whereas we have to actively count larger numbers. While subitizing tends to focus on individual objects rather than collections of objects, the dip in accuracy directly correlates with this subitizing threshold and echoes similar findings in past work in categorical visualization~\cite{haroz2012capacity}. Our study is not designed to probe subitizing or other specific perceptual mechanisms that may explain these results. However, this correlation offers opportunities for further understanding the relationship between categorical perception, subitizing, cluster detection, and other related perceptual phenomena in visualization. 
%where Kaufman et al.~\cite{kaufman1949discrimination} found that when the number of objects is larger than 4, the ability of human cognition would be significantly weakened.
%Our result shows that the obvious accuracy descent occurs in 6 categories, which may be consistent with the finding of Haroz and Whitney~\cite{haroz2012capacity}.
%However, because of the limited dataset, we would not assert it as a finding.
%Instead, we hypothesize that the unusual performance of 6 categories might be related to $subitizing$ in categorical visualizations.


We also found a significant overall difference between color palettes. These differences echo the findings of Liu \& Heer \cite{liu2018somewhere} for continuous colormaps: even if a palette satisfies the basic constraints of good palette design (e.g., discriminable colors), it may not perform optimally. Like Liu \& Heer, we also find that characterizing the source of these performance differences is challenging: palette effectiveness arises from a complex combination of factors. Future work should seek to further deconstruct these factors to derive more robust design guidelines. 
%, while these differences are undoubtedly due to perceptual properties 


\subsection{Design Guidelines for Multiclass Scatterplots}

%Our results reveal that the 
The data and design 
%choices can still significantly impact the accuracy of human perception in multiclass scatterplots.
of multiclass scatterplots significantly influence our abilities to reason across classes. 
Compared to Gleicher et al.'s guidelines~\cite{gleicher2013perception}, our results emphasize the 
%overlooked 
influences of category number and 
%the choices of 
color palette, which are the two essential elements in visualizing categorical data. 
%on the scatterplot.
Additionally, in contrast of some existing guidelines for color palettes~\cite{ware1988color, rogowitz2001blair}, our results indicate maximizing luminance variation may hinder analysis. While designers can use our results to directly choose the optimal palette from our tested set of palettes given the number of categories in their data, our results also provide preliminary guidance for palette selection more broadly: 
%Based on our findings, we recommend the following preliminary guidance for selecting palettes in multiclass scatterplots. 
%present our design guidelines for multiclass scatterplots as follows.

%\begin{itemize}
\vspace{3pt}\noindent \textbf{Simplifying category structure may improve performance.}
%Utilize multiclass scatterplots when required, but be aware of the performance descending.}
Our study suggests that people can reason across multiple classes encoded using color.
However, as shown in Section \ref{sec-analysis-cat}, designers should be aware that performance tends to degrade as the number of categories increases: people are slower and less accurate, especially when working with six or more categories.
%Additionally, it's worth to note when the number of categories is larger than 5, there might occur an unusual performance reduction of judgment accuracy.
We recommend designers 
%should 
consider how the number of categories influences performance on key tasks and consider collapsing relevant categories hierarchically if necessary. 

\add{As a caveat, people were relatively good at completing this task, even with larger numbers of categories than conventional heuristics recommend. Our results indicate that people can reliably distinguish colors in large palettes even though informal pilot participants indicated that the task felt quite difficult for higher numbers of categories. This contrast between perceived and objective performance suggests that even well-established design heuristics can benefit from experimental validation and refinement.  }
%notice the performance descending with categories increasing, and avoid using too many categories for complex tasks.

\vspace{3pt}\noindent \textbf{When designing new palettes,  consider fewer lightness differences, larger perceptual distances, and lighter colors.}
As shown in Section \ref{sec-analysis-color}, our results reveal that color palettes significantly impact the accuracy of human judgment.
%Specifically, for pre-defined designer-crafted palettes, we suggest \emph{SFSO Parties} because of the high performance compared to others.
%Based on our results, for creating new color palettes, we would inform them that \emph{L*} path in CIELCh color space and Perceptual Distance metric~\cite{sharma2005ciede2000} can best capture the performance of palettes, 
Our exploratory analysis confirms the benefits of maximizing the pairwise difference between colors and provides further evidence of minimizing lightness variation. However, we also find that palettes using lighter colors tend to also enhance accuracy. We anticipate that this bias may be in part due to the use of a white background enhancing contrast within categories while minimizing undesirable ``loud'' colors that have too high of a luminance contrast with the background. 
%and suggest choosing lighter colors with minimizing luminance difference and maximizing perceptual distance to promote the ability for judgment tasks.
However, the tested palettes are all handcrafted to select harmonious and aesthetically pleasing colors. Future work should investigate 
%whether 
these results 
%replicate 
on other background colors. \add{We also found little evidence of the benefits of hue or color name variation when considered in isolation. This points to the need for the systematic interrogation of designer practices to improve existing heuristics for palette design \cite{smart2019color}.}

% \item \textbf{Consider multiple choices of color palettes if you're designing scatterplots with different numbers of categories.}

\vspace{3pt}\noindent \textbf{Choose your palettes to fit your data.}
%Since we found significant interaction effects between the choice of color palettes and the number of categories in \autoref{sec-analysis-color}, which reveals that w
When the number of categories changes, the performance rank of different color palettes may also change. Different palettes are differently robust to changes in category number. We recommend designers select color palettes based on the parameters of their specific data.
For example, a designer might use \emph{SFSO Parties} or \emph{ColorBrewer Set3} for multiclass scatterplot with less than seven categories and \emph{D3 Cat10} for larger numbers of categories (see \autoref{fig:palettes-acc}).

%\end{itemize}

% \subsection{A Use Case in Real Scenario}

% \begin{figure}[htbp] 
% \centering
% \includegraphics[width=0.9\textwidth]{Figure/refill.pdf} 
% \caption{Left is the multiclass scatterplot from The Economist \cite{theEconomistarticle} article, with seven categories. We regenerated the plot and refilled the colors using the top 2 palettes from our ranking.} 
% \label{fig:refill}
% \end{figure}

% Multiclass scatterplots are commonly used on media platforms, e.g., New York Times \cite{theNYT}, and The Economist \cite{theEconomist} to demonstrate categorical data and salient patterns. We collected examples of multiclass scatterplots with four or more categories from the above sources.
% % websites or blogs of commercial visualization tools, and refilled them with the color palettes from our guidelines. 
% For example, \autoref{fig:refill} shows a replicated multiclass scatterplot from an article \cite{theEconomistarticle} of The Economist. There are seven categories in these three scatterplots, the first one originated from the article and two replicated charts using \emph{SFSO Parties} and \emph{ColorBrewer Set3} color palettes from our ranking based on overall accuracy of color palettes.
% %
% Based on our study findings, we can demonstrate that the ranked color palettes can be used with an increasing number of categories in the multiclass scatterplot. We did not conduct a case study to prove our regeneration charts have better performance. In this example, we simply demonstrate that the scatterplots having a large number of categories do have a different categorical perception using different color palettes. 

% \fix{compare and contrast color palettes with Tableau in terms of performance from results and how we can provide new guideline.}

% \fix{Getting some examples from NYT with more than 5 categories scatterplot and talk}

% \subsection{\add{Real World Application}}
% \add{Multiclass scatterplots are widely used in real-world scenarios, from media platforms like New York Times \cite{theNYT} to plots rendered using commercial visualization tools like Tableau \cite{tableau}.
% %As encoding the categorical information as third dimension by colors, the selection of 
% Our results show that the color palettes used in a visualization can affect the perception, performance and scalability of the rendered scatterplot. Based on our findings, performance is influenced directly by the number of categories and choice of color palette. 
% %with the increasing number of categories, not only does the overall performance drop significantly, but the performance also varies more between color palettes. 
% Despite the color design in a small number of categories seeming trivial in real-world usage, we have demonstrated the significant correlation between color palettes and performance, especially showing the robustness in higher category numbers. We suggest that our findings can be applied to real-world scenarios and improve the multiclass scatterplots design.}

\subsection{Limitations and Future Work}
We studied the impact of the number of categories and color palettes on multiclass scatterplots. 
%which are essential and commonly used visual channels in multiclass scatterplots.
However, scatterplots offer a wide variety of design choices for representing categorical data that may
%for other visual encodings that can 
provide different trade-offs in perception~\cite{sarikaya2018scatterplots}. 
%, and these encodings' impacts on multiclass scatterplots are still not clear.
Future work should explore the robustness of different channels to varying numbers of categories. Further, scatterplots often encode larger numbers of variables, such as multiple categorical dimensions or combining categorical and continuous dimensions~\cite{smart2019measuring}. Future work should investigate the interplay between different design factors in higher dimensional multiclass scatterplots. 
\add{Both our study and Gleicher et.al.'s work~\cite{gleicher2013perception} focused on 
%the impact of the y-axis (i.e., computing $\Delta$ based on y-value)
comparing y values. However, scatterplots are two-dimensional visualizations. Future work should consider the impact of palettes on crossdimensional tasks.}
%As a consequewe focused on nce, we tend to explore more visual encodings, such as varying shape and size of points, to understand the robustness of human perception to mean judgment tasks and assess their interaction effects with category numbers and color palettes in our future work.

We 
%considered and 
evaluated 10 pre-defined qualitative color palettes on qualitative data.
%in the experiments.
\add{We employed a random color sampling strategy from selected palettes for 
%scatterplots 
data with less than ten categories to simplify the stimulus generation to avoid potential bias from sources outside of color selection. Future work should extend our results to consider sequential strategies in comparing preconstructed palettes.}
%It might avoid the original palettes' sequential impact, however, it also limits the generalizability of our results.}
Additionally, categorical data can also be encoded using other types of palettes, such as sequential and diverging encodings~\cite{liu2018somewhere}, whose robustness to varying numbers of samples is not well understood. 
\add{
%Beside, application-specific design guidance for color-blind users has been 
Considering additional properties of color selection, such as accessible palettes for people with color vision deficiencies~\cite{jefferson2006accommodating, pugliesi2011cartographic}, is also important future work.
%however, how those color-blind palettes would perform on perception and cognition tasks such as mean judgment still require a deeper look.
}

%, a possible perspective for future work can be taking these types of palettes into consideration.
%Further, w
We sampled from predefined palettes at a fixed mark size. Varying mark size can influence mark discriminability~\cite{szafir2018modeling}. As mark size was held constant for all palettes and all palettes had large distances between all color pairs, we do not anticipate that this choice biased our results. However, future  work should explore a larger range of mark sizes and mark types. It should also seek to more systematically evaluate the robustness of our exploratory results. Such variation is challenging due to a large number of potential perceptual factors; however, our results may provide preliminary support for identifying the most promising factors. 
%Moreover, since the palettes we utilized are design-crafted, the performance of perceptual color measures, such as perceptual distance~\cite{sharma2005ciede2000}, designed to describe perceptual features of color palettes cannot be comprehensively assessed in our study.
%Future work can analyze color measures to explore whether these measures fit with their designers' original objectives according to their consistency with human perception.

%Further, we utilize the idea of $subitizing$ to explain some disturbance of judgment accuracy in our results.
%However, it's still a hypothesis, and our current results cannot strongly convince it as a finding.
%But since this phenomenon has been confirmed in other visualization type~\cite{haroz2012capacity}, we tend to explore more general cases for more populations in information visualization.
%For future work, we can focus more to find similar disturbances in larger experiments with more participants, a larger number of cases, kinds of plots, and charts, and find a model to faithfully explain that or convince $subitizing$ as a general assertion in visualization.

%Additionally, to avoid participants from developing strategies that select correct answers by ignoring the bottom part of the chart, and failing to evaluate all the points,
%we tried to generate our data points by using Gaussian distribution that makes points from different categories cross one another.
%However, it is inevitable that as the number of categories grows, the impact of such strategies will be enlarged.
%It would be crucial to find a better way to generate stimulus or find an alternative task to avoid employing such strategies in future work.

\add{We elected to use Mechanical Turk to reflect the range of viewing conditions and participants common to web-based visualizations and to recruit larger numbers of participants. However, variations in viewing conditions can influence color perception. While past studies of color perception in visualization validate the predictive ability of crowdsourced studies for color perception studies in HCI \cite{szafir2014adapting,reinecke2016enabling}, the variability introduced by the range of viewing conditions on MTurk limits the generalizability of our results and our ability to make precise claims about fine-grained mechanistic perceptual phenomena. However, given the large differences between colors in our palettes, we anticipate the affect of viewing variation to be relatively minimal \cite{moroney2003unconstrained} and followed best practices in our experimental design to minimize the impact of viewing variation. Future work seeking to quantify more precise causal mechanisms underlying our findings may wish to replicate our study under more constrained conditions. 
%Furthermore, another limitation is that we couldn't control participants' screen performance (limited by the MT study style) such as brightness and hue variance which might impact the presentation of colors.
%For example, Camgoz et al.~\cite{camgoz2002effects} introduced color brightness would impact users' preferences for background colors and Benedetto et al.~\cite{benedetto2014effects} proposed that higher levels of screen brightness would increase users' visual fatigue.
%According to their findings, we would assume that a low-brightness screen user might perform worse for low-luminance color palettes but with more consistent performance over the whole study.
}

Additionally, data-centric statistical factors that may be related to the performance of multiclass scatterplots are not considered in our study.
For example, we did not explore the impact of correlation or strength of clusters.
Extending our experiments to consider a wider range of data properties as well as statistical tasks 
%enable such consideration of statistical factors would further improve the understanding of human perception 
could help us further understand categorical data visualization for complex datasets and usage scenarios and offer broader guidance for categorical visualization generally. 
%, and could more likely be extended to general visualization types.




\section{Conclusion}
\label{sec-conclusion}
\section{Conclusion}
% We develop MPOBERT, a parameter-efficient pre-trained language model that allows for fewer parameters and efficient training.
% MPOBERT develops a novel mechanism with which we can easily scale BERT to deep models without increasing parameters or computational costs.
% During training, we propose initialization methods for both the central tensors and the auxiliary tensors based on our theoretical analysis to alleviate the training instability issue.
% We validate the effectiveness via supervised, few-shot and multitask experiments. 
% With fewer and less training costs, MPOBERT outperforms several competing models.
% --v2
We develop MPOBERT, a parameter-efficient pre-trained language model that allows for the efficient scaling of deep models without the need for additional parameters or computational resources. 
We achieve this by introducing an MPO-based Transformer layer and sharing the central tensors across layers. During training, we propose initialization methods for the central and auxiliary tensors, which are based on theoretical analysis to address training stability issues. 
The effectiveness of MPOBERT is demonstrated through various experiments, such as supervised, multitasking, and few-shot where it consistently outperforms other competing models.


\begin{acks}
This work was supported by NSF IIS \#2046725 and by NSF CNS \#2127309
to the Computing Research Association for the CIFellows Project.
\end{acks}

\bibliographystyle{ACM-Reference-Format}
\bibliography{main}

\appendix
\section{Appendix: Pilot Study}
\label{sec:appendix}



Our study aims to understand the robustness of color palettes on the perception of multiclass scatterplots.
%Since there is no prior work focusing on how those visual factors impact the perception of complex multiclass scatterplots, we first conducted 3 
To tune the parameters of our study, we first conducted three pilot studies to identify 
%the ability of human perception 
people's abilities to recognize data about different visual factors in multiclass scatterplots and to decide the proper parameters for scatterplots in stimuli generation.

\subsection{Factors}
\label{sec-factors}
We first describe the independent visual factors 
%that 
we considered for generating multiclass scatterplots in both pilot and formal studies.

\textbf{Number of categories.}
The total category count in a scatterplot, varies from 2-10 in our experiments.

\textbf{Level of difficulty.}
We described the distance of means between the categories that have the highest mean and the second highest mean to be $\Delta$. We considered the task to be easier as the $\Delta$ is larger, and more difficult as the $\Delta$ is smaller. 

\textbf{Point distribution.}
The pre-generated x-y data of points. Points from each category were randomly sampled from the Gaussian distribution.

\textbf{Number of points.}
The number of points in one category. Each category in the same scatterplot shared the same number of points, varying from 10-20. 

\textbf{Color palettes.}
%There are 
10 color palettes in total were used in our experiments, with 10 colors in each palette. A certain number of colors were randomly picked to display in each scatterplot depending on the number of categories.

\subsection{Procedure}
We followed the same procedure in the three pilot studies.
Participants were required to carefully read the task description first and then 
%tested 
completed a tutorial check to ensure their understanding.
Afterward, for each study, all participants viewed scatterplots from the corresponding dataset to make it a fair comparison.
They were required to pick the class with the highest average y-value.


\begin{table*}[htbp] 
\centering
\caption{\add{The number of samples collected for each experimental condition after exclusions. Columns are category numbers and rows are color palettes.  }} 
\includegraphics[width=0.9\textwidth]{Figure/result-count-table.pdf} 
\label{tab:parameters}
\end{table*}


\subsection{Pilot Study 1: Hardness Level ($\Delta$) of Stimulus}

\textbf{Participants.} We recruited 106 participants for this study. Participants are all college students, other demographic information was not recorded. They all participated voluntarily and no compensation was provided. 

\textbf{Generation factors:}
Number of categories: \{2\};
Level of difficulty: \begin{math} \{ \Delta \in \mathbb{R} \, | \, 0.5<\Delta<5 \} \end{math};
Point distribution: Poisson distribution with data points (x,y) denoted as \begin{math} \{ x, y \in \mathbb{R} \, | \, 0<x,y<10 \} \end{math};
Number of points: \{15\};
Color palettes: \emph{D3 Cat10}.

\textbf{Results.} The overall accuracy of this study is 76.88\%.
The results suggested that the accuracy rate will increase with the $\Delta$ rises.
To avoid showing tasks that are too easy or too difficult for participants, 
%that both cases would descend the differences in perception precision, 
we selected $\Delta$ from 1.5 to 3.0 in the final study. In the formal study, we mark the $\Delta$ in range 1.5 - 2.0 as hard level, 2.0 - 2.5 as intermediate, and 2.5 - 3.0 as easy (c.f., \autoref{fig:delta-stimuli}).
Details of the result and figures are available in our supplemental material.

\subsection{Pilot Study 2: Number of Categories} 

\textbf{Participants.} We conducted the second study with 25 participants from the 
%local college, 
UNC campus. Other demographic information was not recorded. They all participated voluntarily and no compensation was provided. 

\textbf{Generation factors:}
Number of categories: [2, 9];
Level of difficulty: \begin{math} \{ \Delta \in \mathbb{R} \, | \, 1.5<\Delta<3.0 \} \end{math}
Point distribution: Poisson distribution with data points (x,y) denoted as \begin{math} \{ x, y \in \mathbb{R} \, | \, 0<x,y<10 \} \end{math};
Number of points: \{5, 10, 15\};
Color palettes: \emph{D3 Cat10}.

\textbf{Results.} The overall accuracy of this study was 98.30\%.
The result revealed that participants can identify mean judgment across a lot of categories and colors.
Likewise, we decided to use 2 to 10 categories in the final study, 
%resulting in three difficulty levels: 8 - 10 refers to the hard level, 5 - 7 means intermediate, and 2 - 4 is easy, 
see \autoref{fig:num-stimuli} for examples.
The extremely high accuracy rate encouraged us to think about whether the results are impacted by our choice of distribution. We conducted a third study to check if the Poisson distribution is too na\"ive for this task.
Details of the result and figures are available in our supplemental material.

\subsection{Pilot Study 3: Point Distribution}

\textbf{Participants.} 81 participants joined the third study in total. All the participants were recruited from Amazon Mechanical Turk (MTurk), aged between 24 to 65, with an average of 37 with a standard deviation of 10.7. There are 51 males and 30 females, and 69 of them are wearing corrected glasses.

\textbf{Generation factors:}
Number of categories: [2, 10];
Level of difficulty: \begin{math} \{ \Delta \in \mathbb{R} \, | \, 1.5<\Delta<3.0 \} \end{math}
Point distribution: Gaussian distribution with data points (x,y) denoted as \begin{math} \{ x, y \in \mathbb{R} \, | \, 0<x,y<10 \} \end{math};
Number of points: [10, 20];
Color palettes: All 10 color palettes, see \autoref{fig:palettes-acc}.

\textbf{Results.} The overall accuracy of this study was 80.10\%.
The result suggested that there might be a cue between category number and human judgment accuracy.
Compared to the Poisson distribution in Pilot Study 2, the accuracy rate 
%is not too high to make it hard to analyze.
did not suggest a risk of ceiling effects.
As a result, we decided to use Gaussian distribution to generate scatterplots in our final study.
Details of the result and figures are available in our supplemental material.

\subsection{\add{Metadata}}

\add{\autoref{tab:parameters} illustrates 
%the numbers of generated scatterplots in our study separated and 
the distribution of collected data samples} counted by color palettes and category numbers. Conditions were assigned based on stratified random sampling as described in Section \ref{sec-methodology}.

% \begin{table}[htbp] 
% \centering
% \caption{The experiment parameters. We refined the factors and domain range from 3 pilot studies. Category number and color palettes are our independent variables and hardness level and point number are the control variables. The experiments were built from the combination of these four factors.} 
% \includegraphics[width=0.9\textwidth]{Figure/factor-table.pdf} 
% \label{tab:parameters}
% \end{table}

% \begin{table}[htbp]
%     \centering
%     \caption{ }
%     \label{tab:parameters}
%     \begin{tabular}{ |p{3cm}||p{2.5cm}|p{4.3cm}|p{3.5cm}|  }
%      \hline
%      Factor & Domain & Group & Sampling \\
%      \hline
%      Number of categories & \begin{math} \mathbb{N}: [2, 10] \end{math} & \begin{math} \{2, 3, 4\} \in Small \end{math} \newline \begin{math} \{5, 6, 7\} \in Medium \end{math} \newline \begin{math} \{8, 9, 10\} \in Large \end{math} & Random. Uniformly distributed between groups.\\
%      \hline
%      Color palettes   & 10 palettes \newline (see \autoref{fig:palettes-acc}) & - & Randomly assigned between participants. \\
%      \hline
% b     Hardness level $\Delta$  & \begin{math} \{ \Delta \in \mathbb{R} \, | \, 1.5<\Delta\leq3.0 \} \end{math} & \begin{math} \{2.5<\Delta\leq3.0\} \in Easy \end{math} \newline \begin{math} \{2.0<\Delta\leq2.5\} \in Intermediate \end{math} \newline \begin{math} \{1.5<\Delta\leq2.0\} \in Hard \end{math} & Random. Uniformly distributed between groups.\\
%      \hline
%      Number of points & \begin{math}  \mathbb{N}: [10, 20] \end{math} & - & Random. Uniform distributed \\ 
%      \hline
%     \end{tabular}
% \end{table}

\end{document}
\endinput
%%