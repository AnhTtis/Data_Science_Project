%%   Copyright (c) 2001, 2009, 2010 The American Physical Society.
\documentclass[aps,prl,reprint,notitlepage,twocolumn,showpacs,showkeys,groupedaddress,superscriptaddress,amsmath,amssymb]{revtex4-2}
\usepackage{graphicx,subfigure}
\usepackage{epic}
\usepackage{bm}
\usepackage{bbold}
\usepackage{natbib}
\usepackage{hyperref}
\usepackage{color}
%\usepackage{soul}

%\usepackage{soul}
%\usepackage{todonotes}
%\newcommand{\JE}[2][]{\textcolor{red}{#1}\todo[inline, backgroundcolor=blue!10]{Jocelyn: #2}}
%\newcommand{\JEhide}[1]{}
%\newcommand{\JE}[2][]{}

%\newcommand{\PR}[2][]{\hl{#1}\todo[inline, backgroundcolor=green!10]{Pierre: #2}}

%\usepackage[backend=biber,...]{biblatex} 
%\usepackage[backend=biber]{biblatex}
\setcounter{equation}{0}
\setcounter{figure}{0}
\setcounter{table}{0}
\setcounter{page}{1}
\makeatletter
%\renewcommand{\theequation}{\arabic{equation}}
\renewcommand{\thefigure}{\arabic{figure}}
%\renewcommand{\bibnumfmt}[1]{[S#1]}
%\renewcommand{\citenumfont}[1]{S#1}
\newcommand{\supmat}{Appendix}

\begin{document}

\title{Initiation of motility on a compliant substrate}

%\title{Protrusion-driven cell surfing}

\author{J. \'Etienne and P. Recho}
\affiliation{Univ.\ Grenoble Alpes, CNRS, LIPhy, F-38000 Grenoble, France}
\date{\today}


\begin{abstract}
The conditions under which biological cells switch from a static to a motile state are fundamental to the understanding of many healthy and pathological processes. We show that even in the presence of a fully symmetric protrusive activity at the cell edges, such a spontaneous transition can result solely from the mechanical interaction of the cell traction forces with an elastic substrate. The loss of symmetry of the traction forces leading to the cell propulsion is rooted in the fact that the surface loading follows the substrate deformation. We analytically characterize the bifurcation between the static and motile states and, considering the measurements performed on two cell types, we show that such an instability can realistically occur on soft \emph{in vivo} substrates. 
\end{abstract}

\maketitle




Most living cells have the ability to move to perform collective tasks necessary to vital biological functions such as wound healing or the immune response. To do so, they rely on their cytoskeleton, an active biopolymer meshwork that consumes chemical energy to locally contract and turnover \cite{boal_2012}. Simple continuum models have been used to physically understand how a non-uniform contractility driven by molecular motors or turnover due to the cytoskeleton polymerization and depolymerization can drive cell crawling by producing asymmetric traction forces on the cell substrate  \cite{Julicher+Joanny.2007.1,carlsson2011mechanisms, Recho-Truskinovsky.2015.1,Drozdowski+Schwarz.2021.1, chelly2022cell}. 

Starting from a static state, cells on stiff substrates can spontaneously initiate their motion by breaking the symmetry of their locomotion apparatus \cite{verkhovsky1999self,yam2007actin,barnhart2015balance}, following for instance a chemomechanical instability forming a pattern \cite{mori2008wave,copos2020hybrid,lavi2020motility}, dynamically modulating the adhesion \cite{Sens.2020.1} or the polymerization \cite{Ron+Gov.2020.1}, or through a material instability whereby the active stress created by some molecular motors self-amplifies the cell cytoskeleton flow \cite{recho2013contraction, tjhung2012spontaneous,callan2013active, giomi2014spontaneous}. It has also been shown that a cell formed of an isotropic gel, with constant surface polymerization and bulk depolymerization, can break its initially symmetric configuration through a morphological instability when the protrusive forces overcome the membrane resistance \cite{blanch2013spontaneous}. However, such an instability requires a two-dimensional shape and has no analogue for e.g.\ cells constrained to move along a narrow track, either at the surface of a substrate \cite{maiuri2012first,Mohammed+Gabriele.2019.1} or along a fiber in a 3D matrix \cite{doyle2013dimensions}. %While such an instability requires a two-dimensional shape,  cells constrained to move along a narrow track, either at the surface of a substrate \cite{maiuri2012first,Mohammed+Gabriele.2019.1} or along a fiber in a 3D matrix \cite{doyle2013dimensions}.

In this letter, we show that substrate deformation can lead to symmetry breaking through a mechanical feedback that initiates  the motion of a cell with symmetric boundary polymerization and bulk depolymerization. %\JE[Substrate-mediated instabilities...]{}
%The instability is rooted in the non-local coupling of the sliding friction forces that the substrate elasticity entails \JE[and does not require any non-linearity in the components of the system.]{...}
In many existing models of mechanotaxis, the observed influence of the substrate stiffness or deformation on cell motility \cite{lo2000cell,duchez2019durotaxis} is included through various mechanosensing pathways controlling the cell adhesion \cite{lelidis2013interaction,lober2014modeling,feng2019cell,sens2013rigidity,chen2022unified}, the molecular motors' active stress \cite{banerjee2011substrate,shenoy2016chemo} or the cytoskeleton polymerization rate \cite{zhang2020minimal,oliveri2021theory}. In sharp contrast, only the fundamental principle of force balance at the cell--substrate interface is required for the present instability. % which relies on the non-local coupling of the sliding friction forces via the substrate.

We consider an active gel segment \cite{kruse2005generic,kruse2006contractility} moving along a one-dimensional line, modeling the cytoskeleton of a cell crawling on a thin fibronectin-coated track.
For simplicity, we consider the limit in which the active gel flow is fixed independently of the external forces \cite{blanch2013spontaneous}.
As shown previously \cite{Etienne+Asnacios.2015.1} and in the Supplemental Text, such a regime exists for a symmetric polymerization velocity of the gel at the boundaries $v_p$, leading to a uniform bulk depolymerization rate $\gamma$ that fixes the length $L=2 v_p/\gamma$ of the segment. The meshwork flow $v$ associated with these dynamics is given by  $v(x,t)=\dot C(t)-\gamma{(x-C(t))}$ where $C(t)$ is the segment center, $\dot{C}=\partial_tC(t)$ its velocity and $x\in [C(t)-L/2,C(t)+L/2]$ the spatial coordinate within the gel.

Such a symmetricaly treadmilling system can be put into motion by a spatial asymmetry in its frictional interaction with the substrate \cite{Sens.2020.1,Ron+Gov.2020.1,schreiber2021adhesion} since the traction forces are then biased in one direction. We consider here a different situation where the friction coefficient $\xi$ with the substrate is a constant but the traction forces depend on the relative velocity of the gel with respect to the substrate: 
\begin{equation}\label{e:traction}
  f = \xi\Psi_*(v- v_s). 
\end{equation}
In \eqref{e:traction},  $v_s$ is the tangential velocity of the substrate surface along the track and $\Psi_*$ is an odd function, which characterizes the non-linear behavior of the sliding friction between the gel and the substrate. This is relevant to cell migration,
since the effective friction law is known to be biphasic \cite{gardel2008traction} due to the stochastic collective dynamics of force-sensitive adhesive bonds \cite{sabass2010modeling,sens2013rigidity}. In what follows, we take the simple approximation $\Psi_*(v) = v/(1+(v/v_*)^2)$, which leads to  a switch of behavior when the relative velocity reaches the threshold velocity $v_*$. 

We aim at showing  that the non-local feedback of the friction forces through the velocity $v_s$ of a compliant substrate can lead to a symmetry breaking and spontaneous motility.

To first explain the physics of this motion, we start by considering the situation of a deformation imposed externally to a linear elastic incompressible substrate, which occupies the half space $z<0$. We assume that the deformation is created by the presence of magnetic beads at the substrate surface, along the track. The beads are actuated by an electromagnet of exactly the cell size $L$ that is quasistaticly moved along the track at a velocity $V_e$. As they are polarized in the $z$ direction, the torque acting of the beads vanishes while they experience a traction force oriented only in the $x$-direction that takes the form $f_m=f_m^0F(y(x,t))$, where the characteristic force $f_m^0$ scales with the current in the electromagnet and the beads magnetization and $y(x,t)=2(x-V_e t)/L$ is the traveling wave coordinate. By asymptotic matching of the magnetic field \cite{Landau+.1984.8} close and far from the magnet, we can approximate $F(y)=-y \text{ if } |y|<1 \text{ and } -y/|y|^{5} \text{ if } |y|>1 $. Next, restricting our analysis to small deformations, the substrate displacement at the surface $z=0$ is purely tangential and takes the form of the traveling wave  $u(x,t)=u^0U(y(x,t))$ where, in plane strain, $U$ reads \cite{Johnson1987}:
\begin{equation}\label{e:substrat_vel_0}
U(y)=-\int_{-\infty}^{\infty}\log |y-y'|F(y') \mathrm{d}y', 
\end{equation}
and $u^0=3f_m^0L/(2\pi E_s)$. The Young modulus of the substrate is denoted $E_s$.

In \eqref{e:substrat_vel_0}, we have neglected the contribution of the cell to the traction force acting on the substrate, assuming that $|f_m^0|\gg\xi v_p$. Using again the small deformation assumption, the substrate velocity is also a traveling wave given by $v_s=\partial_tu=-V_e\epsilon$ where the strain of the substrate surface is $\epsilon=\epsilon^0E(y(x,t))$ with $E(y)=\partial_yU$ and $\epsilon^0=3f_m^0/(\pi E_s)$.  The functions $F$, $U$ and $E$ representing the variations of the applied force, substrate displacement and strain are shown in Fig.~\ref{f:cell_driven_magnetic}.


 Finally, as the total force that is exerted on the treadmilling segment has to vanish, we obtain from \eqref{e:traction}  that 
\begin{align}
&0= -\int_{C(t)-L/2}^{C(t)+L/2} f(x) \mathrm{d}x,\text{ which implies that, }\nonumber\\
&0= \int_{-1}^{1} \Psi_*\left(\dot{C}-v_py+V_e\epsilon^0E\left(y+\delta(t)\right)\right) \mathrm{d}y\label{e:surfer},
\end{align}
where $\delta(t)=y(C(t),t)$. Equation \eqref{e:surfer} is a differential equation that sets the $C(t)$ dynamics. Taking $C(0)=0$ (the driving electromagnet is centered with the cell from the start) and since $E$ decays to zero at infinity, the only permanent regime of motion is for $\delta(t)=0$, i.e.\ a segment velocity equal to the actuation velocity $\dot{C}=V_e$: we call this mode of motion at the speed of the elastic wave `surfing'.
Thus, \eqref{e:surfer} becomes the  algebraic equation 
\begin{equation}
0= \int_{-1}^{1} \Psi_*(-v_py+V_e(1+\epsilon^0E(y))) \mathrm{d}y,
\end{equation}
that sets the strain magnitude $\epsilon^0$ of the traveling wave such that it can be surfed by the treadmilling gel.

\begin {figure}[!t]
\includegraphics[scale=0.4]{cell_driven_magnetic_sketch.pdf}
\caption{Top, an electromagnet moved at velocity $V_e$ induces a mechanical strain (gray-coded) in a substrate. A cell is in frictional interaction with the substrate, it is assumed to treadmill with velocity $-v_py$ and polymerize at its edges with speed $v_p$.
Bottom, shape of the force applied by the magnet, displacement and strain of the elastic substrate surface.}
\label{f:cell_driven_magnetic}
\end {figure}

If $\Psi_*$ is a linear function, this surfing condition becomes independent of $V_e$ and $v_p$ and simply reads $\bar{\epsilon}=\frac{1}{2}\int_{-1}^{1}\epsilon(y)\mathrm{d}y=-1$. Such an average elastic strain of the substrate is not admissible since it implies self-interpenetration of matter. Likewise, for non-linear $\Psi_*$, no solution exists if the treadmilling is slow, $v_p \ll V_e(1-\epsilon^0)$. We thus focus on the case $v_p\gg V_e$. Since $\int_{-1}^{1} \Psi_{*}(-v_py)\mathrm{d}y=0$, a first order expansion  leads to the surfing condition
\begin{equation}
0= \int_{-1}^{1} \Psi_{*}'(-v_py)\left(1+\epsilon^0E(y)\right) \mathrm{d}y.
\end{equation}
 The above condition, where~${}'$~denotes the derivative, shows the importance of the treadmilling dynamics and the non-linear friction for the surfing  to be physically feasible. Indeed, the prefactor $\Psi_{*}'(-v_py)$ has to change sign in order to make the cancellation of the integral possible for admissible $\bar{\epsilon}>-1$. The necessary and sufficient conditions for surfing motion are thus the presence of treadmilling and a biphasic friction $\Psi_*$. 

We now turn to the coupled problem asking whether the cell can surf on an elastic wave that it itself creates instead of relying on an external traveling wave driven by a moving magnetic field. We therefore express the surface displacement in response to the cell traction forces themselves \cite{Johnson1987}, which we had previously neglected compared to the action of the magnetic field:
\begin{equation}\label{e:substrat_vel}
u(x,t)=\frac{-3}{2\pi E_s}\int_{C(t)-L/2}^{C(t)+L/2}\!\!\log\left| \frac{x-x'}{L}\right|f(x'\!,t) \mathrm{d}x'\!. 
\end{equation}
Changing the time and space variables to follow the cell, $s=t$ and $y=y(x,t)$, we have $\partial_t=\partial_s-(2\dot{C}/L)\partial_y$. Then, the velocity of the substrate for small deformations $v_s=\partial_tu$ can be expressed as a functional of the traction force distribution from \eqref{e:substrat_vel}:
\begin{align}\label{e:v_s}
v_s(y,s)&=-\frac{3L}{4\pi E_s}\int_{-1}^{1}\log|y-y'|\partial_sf(y'\!,s) \mathrm{d}y'\nonumber\\
&+\frac{3 \dot{C}(s)}{2\pi E_s}\int_{-1}^{1} \frac{f(y'\!,s)}{y-y'} \mathrm{d}y'\!.
\end{align}
In these coordinates, the cell cytoskeleton's velocity is
\begin{equation}\label{e:v}
v(y,s)=-v_py+\dot{C}(s).
\end{equation}
Non-dimensionalizing the distance by $L$, the time by $1/\gamma$ and the stress by $2\xi v_p$, and defining the non-dimensional substrate displacement and strain ($\epsilon=\partial_xu$):
\begin{align*}
 u[f](y,s)&=-\frac{\theta}{2}\int_{-1}^{1}\log|y-y'|f(y'\!,s) \mathrm{d}y'\!,\\
\epsilon[f](y,s)&=-\theta\int_{-1}^{1}\frac{f(y'\!,s)}{y-y'} \mathrm{d}y'\!,
\end{align*} 
the three relations $\eqref{e:traction}$, \eqref{e:v_s} and \eqref{e:v} are combined in a single non-linear integral equation that determines the distribution of the cell traction forces, 
\begin{equation}\label{e:integ_pb}
2\beta f=\Psi\left(2\beta(\dot{C}-y/2 -u[\partial_sf]+\dot{C}\epsilon[f])\right),
\end{equation}
with $\Psi(v) = v/(1+v^2)$.
The non-dimensional parameters of the model, respectively 
$$\beta=\frac{v_p}{v_*}\text{ and }\theta =\frac{3\xi v_p}{\pi E_s},$$
represent the level of importance of the non-linearity of the sliding friction law ($\beta\ll 1$ being the linear viscous friction limit) and the stress imposed by the protrusion compared to the elastic modulus of the substrate. Equation \eqref{e:integ_pb}  also depends on the unknown crawling velocity $\dot{C}$, which is set by the global force balance condition
\begin{equation}\label{e:integ_cond}
\int_{-1}^{1}f(y,s)\mathrm{d}y=0.
\end{equation}

 
When the substrate is infinitely rigid ($\theta=0$), our model \eqref{e:integ_pb} directly provides the expression of the traction forces  $f=\Psi(2\beta(\dot{C}-y/2))/(2\beta)$ and \eqref{e:integ_cond} necessarily implies that $\dot{C}=0$. Therefore, as can be expected for a system with symmetric polymerization at both ends, in the absence of intra-cellular feedbacks \cite{Ron+Gov.2020.1,Sens.2020.1}, the steady state traction force profile $f_0(y)=\Psi(-\beta y)/(2\beta)$ is symmetric with respect to the cell center and the cell always remains static. While such a distribution  remains a solution of the integral problem \eqref{e:integ_pb}-\eqref{e:integ_cond},  its stability for a finite substrate elasticity is now addressed. 

To do so, we begin by investigating the situation when $\beta\ll 1$ (the sliding friction law is linear) and \eqref{e:integ_pb} reduces to the linear integral equation 
\begin{equation}\label{e:integ_pb_lin}
 f=\dot{C}-y/2 -u[\partial_sf]+\dot{C}\epsilon[f].
\end{equation}
Inserting in \eqref{e:integ_pb_lin} the perturbation of the static solution with a growth rate $\lambda$,
$$f(y,s)=f_0(y)+\eta\text{e}^{\lambda s} \delta f(y) \text{ and }\dot{C}=\eta\text{e}^{\lambda s} \delta\dot{C},$$
we find at first order in the small parameter $\eta$:
$\delta f(y)=\delta\dot{C}(1+\epsilon[f_0(y)])-\lambda u[\delta f(y)]$. Imposing the global force balance condition \eqref{e:integ_cond}, $\delta\dot{C}$ is expressed as a functional of $\delta f$  to obtain,
\begin{align}\label{e:perturb_lin} 
&\left( 2+\int_{-1}^1\epsilon[f_0]\mathrm{d}y\right) \frac{\delta f}{\lambda}=(1+\epsilon[f_0])\int_{-1}^1u[\delta f]\mathrm{d}y.\nonumber \\
& -\left( 2+\int_{-1}^1\epsilon[f_0]\mathrm{d}y\right) u[\delta f].
 \end{align}
 When $\theta$ is small, multiplying  \eqref{e:perturb_lin} by $\delta f$ and integrating, leads to
$$\int_{-1}^{1}(\delta f(y))^2 \mathrm{d}y=\frac{\lambda\theta}{2}\int_{-1}^{1}\int_{-1}^{1} \log|y-y'|\delta f(y)\delta f(y')\mathrm{d}y\mathrm{d}y'.$$
As the linear operator $u[f]$ is symmetric positive definite \cite{estrada1989integral}, we conclude from the above equation that $\lambda$ is negative and the static steady state is stable to small perturbations for $\theta\ll 1$. This limiting behavior also shows that, in general, the $u$ term in \eqref{e:integ_pb_lin} describing the elastic feedback of the substrate when the loading domain is fixed does not destabilize the central symmetry of the traction forces in the static configuration and thus does not contribute to making the system motile. However, the growth rate of the instability can vanish ($\lambda=0$)  when $\theta$ reaches a critical threshold $\theta_c$ in \eqref{e:perturb_lin} that satisfies the condition $2+\int_{-1}^1\epsilon[f_0]\mathrm{d}y=0$, leading to $\theta_c=2$. Close to this threshold, the behavior of $\lambda(\theta)$ can be expanded in power series and we find at first order that $\lambda\sim 9 (\theta -\theta_c)/(2(\pi^2-6))$, showing that the growth rate of a perturbation of the static solution becomes positive beyond $\theta=\theta_c$.	Thus, the contribution of the substrate deformation due to the fact that the loading domain moves if the cell moves  (so-called `follower load' effect \cite{bigoni2011experimental}) makes the static state unstable when the traction forces compared to the substrate elasticity increase beyond the critical threshold $\theta_c$.

The post-bifurcation regime can be  characterized  analytically by searching traveling wave solutions of \eqref{e:integ_pb_lin} for which $\partial_s f=0$ and $V=\dot{C}$ is a constant.  Indeed, the Cauchy integral equation:
\begin{equation}\label{e:integral_eq_simpl_nd_TW}
\theta V\int_{-1}^{1}\frac{f(y')}{y-y'} \mathrm{d}y'+f(y)=-\frac{y}{2}+V,
\end{equation}
has an integrable solution which reads \cite{Karpenko1966approximate}:
\begin{equation}\label{e:analytical_f}
f_{\text{eq}}(y)=-\frac{(y/2+V)w_{a}(y)}{\sqrt{1+(\theta \pi V)^2}},
\end{equation}
where $w_{a}(y)=((1-y)/(1+y))^a$ and $a=\text{atan}(\theta \pi V)/\pi$. Thus, imposing the global force balance condition \eqref{e:integ_cond}, we find that the cell velocity satisfies the implicit relation:
\begin{equation}\label{e:velocity_eq}
V=a/2=\text{atan}(\theta \pi V)/(2\pi).
\end{equation}
If $\theta\leq \theta_c= 2$, equation \eqref{e:velocity_eq} has the single root $V=0$ (stable static solution) while if $\theta>\theta_c$, it has three roots: $V=0$ corresponding to the now unstable static configuration and the two others corresponding to two symmetric motile configurations, see Fig.~\ref{f:pitchfork}(a).
\begin {figure}[t]
\includegraphics[scale=0.5]{linear_friction.pdf}
\caption{ (a) Static to motile transition for a linear sliding friction law (Eq.~\eqref{e:integral_eq_simpl_nd_TW}). Characteristic traction forces profiles \eqref{e:analytical_f} are shown in inserts. The static branch is dashed passed the critical point $\theta=\theta_c$ since it becomes unstable. (b) Profiles of the substrate displacement and strain corresponding to the traction forces shown in (a)}
\label{f:pitchfork}
\end {figure} 
\begin {figure}[t]
\includegraphics [scale=0.4]{dynatoss.pdf}
\caption{Convergence of the instantaneous cell velocity found from \eqref{e:integ_pb_lin} to its steady state value predicted by \eqref{e:velocity_eq} for an initial traction force profile of the form of \eqref{e:analytical_f} but with an initial velocity randomly chosen in the interval [-0.25,0.25] (corresponding to the maximal velocity in \eqref{e:velocity_eq}). Large transient oscillations can arise before reaching the steady state velocity.}
\label{f:stability}
\end {figure} 
Using $v_p\simeq 0.25 \,\mu\text{m s}^{-1}$ and $\xi\simeq 1\text{ kPa s}$ measured for fish keratocytes \cite{rubinstein2009actin,barnhart2011adhesion}, we can estimate a value of $\theta\simeq 2.5$ for a substrate made of extra-cellular matrix with a Young modulus of $E_s\simeq100 \text{ Pa}$  \cite{levental2009matrix, dolega2021mechanical}, close to $\theta_c$. This suggests that the present motility initiation instability could realistically be at play following a uniform actin polymerization activation at the membrane for \emph{in vivo} soft substrates. Upon the bifurcation, the symmetry of the traction forces is lost (inserts of Fig.~\ref{f:pitchfork}(a)) with a maximum at the trailing edge  and a negative part at the leading edge in qualitative agreement with experiments~\cite{hennig2020stick}. This in turn leads to an asymmetric substrate displacement and strain (see Fig.~\ref{f:pitchfork}(b)) that promotes the  cell motion, which `surfs' on the deformation it creates.  By simulating numerically \eqref{e:integ_pb_lin}, we  show on Fig.~\ref{f:stability} the robust convergence of the cell to the aforementioned steady state configurations regardless of initial conditions, in line with the small perturbation stability results. 

Although the above results with a linear friction law limit ($\beta\ll 1$) allow to characterize the physics of substrate deformation-induced motility initiation, the instability arises in a parameter range for which the deformation is not physically admissible. Indeed, as in the case of externally imposed deformations that has been presented in the introduction, the surface strain at the bifurcation threshold 
$\epsilon_c(y)=2 (y \log(\sqrt{\left| (1+y)/(1-y)\right| })-1)$ has values below $-1$ over a finite interval. %As a result, the average strain over the cell contact area $\bar{\epsilon_c}=(\int_{-1}^1\epsilon_c(y)\mathrm{d}y)/2$ is $\bar{\epsilon_c}=-1$, indicating the presence of non admissible substrate compression at that lengthscale. This also casts some doubt about the model applicability since, formally the hypothesis of small deformations was used to characterize the substrate elasticity in \eqref{e:substrat_vel}. To address this issue, we get back to the initial model where $\beta$ is finite and the sliding friction is biphasic. 
We now show that here too, a biphasic sliding friction (finite $\beta$) restores the possibility of an admissible substrate deformation.
In this case, the perturbation of the static state leads to
\begin{equation}\label{e:perturb}
\delta f(y)=\Psi'(-\beta y)(\delta\dot{C}(1+\epsilon[f_0(y)])-\lambda u[\delta f(y)])
\end{equation}
and the growth rate of the perturbation switches from a negative to a positive value at the critical value: 
$$\theta_c(\beta)=\frac{4 \left(\beta ^2+1\right)}{\left(\beta^{-2}+1\right) \text{atan}(\beta ) \left(\left(\beta ^2+1\right) \text{atan}(\beta )+2 \beta \right)-1}.$$
The strain at the bifurcation threshold reads
$$\epsilon_c=-\theta_c(\beta)\frac{\beta  y \log \left| \frac{1-y}{1+y}\right| +2 \text{atan}(\beta )}{2 \left(\beta +\beta ^3 y^2\right)},$$
which approaches zero when $\beta$ increases except in narrow boundary layers at $y=\pm 1$ close to the cell fronts. This leads to an average strain under the gel
%$$\bar{\epsilon}_c=\frac{2 \left(\beta ^2+1\right) \text{atan}(\beta )^2}{\beta ^2-\left(\beta ^2+1\right) \text{atan}(\beta ) \left(\left(\beta ^2+1\right) \text{atan}(\beta )+2 \beta \right)}$$
%with 
$\bar{\epsilon}_c=\frac{1}{2}\int_{-1}^1\epsilon_c(y)\mathrm{d}y = -\theta_c(\beta) \text{atan}^2(\beta)/(2\beta^2)$,
that tends to zero as $\beta\rightarrow \infty$. Thus for a large enough $\beta$, the small deformation assumption is verified as we show in Fig.~\ref{f:pitchfork_nl}(b). For fish keratocytes, the value of $v_*\simeq 0.125 \,\mu\text{m s}^{-1}$ in the biphasic relation was estimated in \cite{barnhart2015balance} based on experiments on stiff substrates, leading to $\beta\simeq 2$ and $\bar{\epsilon}_c\simeq -0.25$. %, four times smaller than in the linear case. 
The biphasic relation was also initially characterized for another cell type (PtK1). In this case, the polymerization velocity can be estimated from \cite{ponti2004two} to be $v_p\simeq 0.03\,\mu\text{m s}^{-1}$ and the data of \cite{gardel2008traction} indicates an effective friction coefficient of $\xi\simeq 10\text{ kPa s }\mu\text{m}^{-1}$ and a threshold velocity $v_*\simeq 0.01 \,\mu\text{m s}^{-1}$. This leads to non-dimensional parameters $\theta\simeq 3 $ and $\beta\simeq 3$ %, of the same order as those of fish keratocytes. For $\beta\simeq 3$, we obtain 
and to $\bar{\epsilon}_c\simeq -0.15$. 

\begin {figure}[t]
\includegraphics[scale=0.45]{bif_biphasic.pdf}
\caption{ (a) Pitchfork bifurcation of the segment steady state velocity from a static to a motile state for the non-linear system \eqref{e:integ_pb}-\eqref{e:integ_cond}. Typical traction forces distributions for special choices of $\theta$ (same as for Fig.~\ref{f:pitchfork}) are shown in inset. (b) Substrate displacement and strain associated to the traction forces profiles shown in (a). Parameter $\beta=3$. }
\label{f:pitchfork_nl}
\end {figure} 

We compute by numerical continuation the traction force profile and the cell velocity beyond the bifurcation threshold for this more realistic sliding friction law, see Fig.~\ref{f:pitchfork_nl}. The traction force profiles are similar to the ones of the linear case except that their divergence at the trailing edge is eliminated because of the thresholding effect of $\Psi$. The cell-generated substrate displacement is also similar to the linear case, albeit smaller, leading to admissible strain profiles. The strain may still become large very close to the cell fronts since our simple model does not account for the small decay length of the traction forces at the cell edges.

We have shown that symmetric protrusions driving the cell cytoskeleton flow can lead to cell spontaneous initiation of motility in the presence of a soft substrate. This instability is rooted in a follower load effect where the cell surfs on the substrate deformation it creates. The feedback between the substrate and the cell mechanics is fully mediated by the frictional traction forces and requires no specific mecanosensing, which suggests this phenomenon could be universal across cell types. To be more quantitative, this model could be coupled to internal cell signaling, controlling the local protrusion rates at the cell boundary as a function of  Rho GTPases concentration that undergo a drift-diffusion dynamic in the cytoplasm, further enhancing the cell polarity when the symmetry of the internal flow is broken \cite{Ambrosi-Zanzottera.2016.1,Ron+Gov.2020.1}.  
%Another interesting generalization to reach a more quantitative model of a specific experiment would be to represent the substrate rheology and the track geometry in a refined way \cite{chelly2022cell}. 

\acknowledgements{The authors thank L. Truskinovsky for his stimulating comments, as well as Alexander Erlich for his critical and helpful reading of the manuscript.}


\appendix
\newpage
\noindent
\textbf{Supplemental text}
\bigskip

{
This supplemental text contains a detailed mechanical modeling of \textcolor{black}{a treadmilling active gel model of the cell cytoskeleton leading to the linear retrograde flow  used in the main text.} 

%how it can be the limit of an active gel with a uniform \textcolor{black}{contractile} active stress in the bulk (commonly used to model the action of molecular motors) and polymerization at the boundaries \cite{kruse2006contractility}, and also a more detailed derivation for an active gel with fixed depolymerization in the bulk and polymerization at the boundaries \cite{putelat2018mechanical}.

\section{Mechanical balance}

We consider an active gel slab \cite{kruse2005generic,kruse2006contractility}
 moving along a one-dimensional line, modeling the biopolymer meshwork of a cell crawling on a fibronectin-coated track. 
Neglecting inertial effects, the momentum balance in the gel reads 
\begin{equation}\label{e:momentum_bal}
h\partial_x\sigma=f,
\end{equation}
where $h$ is the constant gel height, $\sigma(x,t)$ is the axial stress in the gel and $f(x,t)$ is the traction force  applied by the cell on the substrate. 

The spatial coordinate of material points in the gel at time $t$ is denoted $x\in[l_-(t),l_+(t)]$, where $l_-(t)$ and $l_+(t)$ are the moving fronts of the gel. We denote by $C(t)=(l_-(t)+l_+(t))/2$ the segment center and by $L(t)=l_+(t)-l_-(t)$ its length.
In the absence of any external loading at the segment ends, the stress vanishes at the cell boundaries,
\begin{equation}\label{e:momentum_bal_bc}
\sigma(l_{\pm}(t),t)=0.
\end{equation} 

%As in the main text, the sliding friction is written in its generic form
%\begin{equation}\label{e:friction}
%f = \xi \Psi_*(v-v_s),
%\end{equation}
%where $v(x,t)$ is the gel velocity field in the lab coordinates and $v_s(x,t)$ the velocity of the substrate surface.
%$\Psi_*(w)$ is an odd function for symmetry reasons, moreover it has the same sign as $w$.
%It is chosen so as to represent the so-called `biphasic' dependence of the traction forces on the sliding velocity $w=v-v_s$ \cite{gardel2008traction,sabass2010modeling,sens2013rigidity}, with a switch of behavior when the relative velocity reaches the threshold $v_*$. When $v_*$ is large compared to $v-v_s$, $\eqref{e:friction}$ reduces to a viscous sliding friction.%  $f=\xi (v-v_s)$ where the effective friction coefficient is $\xi=f_*/v_*$.

%Finally, the substrate is modeled as an elastic incompressible half-space under plane strain and we restrict our analysis to small deformations.  Then, as \eqref{e:momentum_bal}-\eqref{e:momentum_bal_bc} imply that the total traction force vanishes,
%\begin{equation}\label{e:total_force}
%\int_{l_-(t)}^{l_+(t)}f(x,t)\mathrm{d}x=0,
%\end{equation}
%the substrate displacement is \cite{Johnson1987}:
%\begin{equation}\label{e:substrat_vel}
%u(x,t)=\frac{3}{2\pi E_s}\int_{l_-(t)}^{l_+(t)}\log\left( \frac{L(t)}{|x-x'|}\right)f(x'\!,t) \mathrm{d}x'\!. 
%\end{equation}
%Using again the small deformation assumption, the substrate velocity is simply given by $v_s=\partial_tu$. 

\section{Treadmilling driven retrograde-flow}

%\subsection{Mass balance law in the cell}

In this section, %in order to clarify the physical ingredients at play in the instability that we shall present, we voluntarily only consider  the turnover of the gel and 
we neglect the active stress due to the presence of molecular motors and consider only the activity due to \textcolor{black}{the meshwork polymerization at the boundaries and its depolymerization in the bulk.}
Mass balance within the gel reads \cite{kruse2005generic}
\begin{equation}\label{e:conservation_mass}
\dot{\rho}+\rho \partial_x v=\gamma \rho, 
\end{equation}
where $\rho(x,t)$ is the mass density of the gel and $\gamma$ is a bulk depolymerization rate. %Here $\rho_0$ is the so-called homeostatic density at which polymerization and depolymerization balance. 
The superimposed dot $\dot{}=\partial_t+v\partial_x$ denotes the total time derivative. This bulk mass conservation equation is associated with the boundary conditions:
\begin{equation}\label{e:conservation_mass_bc}
\rho(l_{\pm}(t),t)(v(l_{\pm}(t),t)-\dot{l}_{\pm})=\pm j,
\end{equation}
where $j<0$ is a fixed and \emph{symmetric} polymerization flux at the two cell ends. %
%\begin {figure}
%\includegraphics [scale=0.45]{scheme.png}
%\caption{Scheme of the active gel segment on a track subjected to a symmetric polymerization flux balanced by bulk depolymerization.}
%\label{f:profiles}
%\end {figure} 

%\subsection{Cell constitutive behavior}
%The following constitutive behaviors characterize the gel mechanics and its contact with the substrate:
We assume the following constitutive behavior to characterize the gel mechanics:
\begin{equation}\label{e:const_beh}
\sigma=-E\Phi\left(\frac{\rho}{\rho_*}\right)%\text{ and } f=f_*\Psi\left(\frac{v-v_s}{v_*}\right).
\end{equation} 
In \eqref{e:const_beh}, $E$ is the bulk modulus of the gel and $\rho_*$ is a reference density of the gel. % and 
$\Phi(x)$ is an increasing function that vanishes when $\rho=\rho_*$, characterizing the short timescale non-linear elastic properties of the biopolymer meshwork penalizing both infinite compression and dilution. See \cite{putelat2018mechanical} for further discussion on the effective mechanical properties that arise in these conditions.

%\subsection{Simplified front dynamics} 

Combining the polymerization fluxes \eqref{e:conservation_mass_bc}  with the absence of applied stress at the boundaries  \eqref{e:momentum_bal_bc}, \textcolor{black}{we obtain  the widely used kinematic conditions for the moving fronts \cite{larripa2006transport,kruse2006contractility,recho2013asymmetry, blanch2013spontaneous,Ambrosi-Zanzottera.2016.1}:
\begin{equation}\label{e:conservation_mass_bc_bis}
\dot{l}_{\pm}=\pm v_p+v(l_{\pm}(t),t),
\end{equation}
where $v_p=-j/\rho_*>0$ is a \emph{symmetric} polymerization velocity at both ends \cite{carlsson2011mechanisms,Recho-Truskinovsky.2015.1,Drozdowski+Schwarz.2021.1}.}
%leads to the widely used kinetic  conditions for the moving fronts \cite{larripa2006transport,kruse2006contractility,recho2013asymmetry, blanch2013spontaneous,ambrosi2016mechanics}:
%\begin{equation}\label{e:conservation_mass_bc_bis}
%\dot{l}_{\pm}=\pm v_p+v(l_{\pm}(t),t),
%\end{equation}

Next, we assume that both the magnitude and rate of fluctuations of the gel density are small: 
\begin{equation}\label{e:tech_assumption}
|\rho-\rho_*|/\rho_*\ll 1\text{ and }  \dot{\rho}/\rho\ll \gamma
\end{equation}  
 In this case, \textcolor{black}{similarly to \cite{blanch2013spontaneous},} \eqref{e:conservation_mass} leads to $\partial_xv=-\gamma$
and the cytoskeleton flow in the cell is linear:
 $v(x,t)=-\gamma (x-l_-(t))+\bar{v}(t)$, with $\bar{v}(t)$ a time-dependent function to be determined. This expression of velocity, combined with \eqref{e:conservation_mass_bc_bis}  leads to the following fronts dynamics: $\dot{L}=-\gamma L+2v_p$ and $\dot{C}=\bar{v}-\gamma L/2$, showing that $L$ converges to a steady state fixed value $L=L_{\text{eq}} = 2v_p/\gamma$ and that $\bar{v}$ can be expressed as a function of $\dot{C}$:  
 $$v(x,t)=-\gamma (x-C(t))+\dot{C}(t).$$

%\subsection{Non-dimensional problem}

%Similar to above, we non-dimensionalize time by $1/\gamma$, distance by $L_{\text{eq}}$, and additionally density by $\rho_*$ and stress by $2\xi v_p$.
%When the interfacial traction force distribution $f$ is known, all the problem fields can be reconstructed by integration. Within the cell, $\sigma$ is found from \protect\eqref{e:momentum_bal}  and $\rho$ from \protect\eqref{e:const_beh} leading in non-dimensional form to,
%$$\sigma(y,s)=\nu\int_{-1}^yf(y'\!,s)\mathrm{d}y'\text{ and }\rho(y,s)=\Phi^{-1}\left(\vartheta\int_{-1}^yf(y'\!,s)\mathrm{d}y'\right)$$
%with the non-dimensional parameters $\nu=\gamma h/v_p$ and $\vartheta=2\xi\gamma h /E$. The smallness of $\vartheta$ is necessary to enforce the validity of the small density variations approximation \eqref{e:tech_assumption}.


%\section{Contractility-driven treadmilling}
%
%\textcolor{black}{We can also consider the contractility-driven case where instead of \eqref{e:const_beh}, we use the constitutive model of an infinitely compressible viscous active fluid as in \cite{kruse2006contractility,Julicher+Joanny.2007.1}}:
%\begin{equation}\label{e:active_gel}
%\sigma = \eta\partial_x v + \chi
%\end{equation}
%where $\eta$ is a bulk viscosity and $\chi>0$ an active prestress.
%
%
%
%We use the change of variable $y = 2(x-C(t))/L(t)$ and non-dimensionalize time by $\eta/\chi$ and velocity by $2v_p$, giving a characteristic length $\Lambda = 2v_p\eta/\chi$. Using \eqref{e:momentum_bal_bc} to integrate \eqref{e:momentum_bal} and combining with \eqref{e:active_gel}, we obtain
%\begin{equation}
%\partial_y v + \frac{L(t)}{\Lambda} = \left(\frac{L(t)}{\lambda}\right)^2 \int_{-1}^y \frac{v-v_s}{1+\beta^2(v-v_s)}\mathrm{d}y'
%\end{equation}
%where $\beta=v_p/v_*$ and $\lambda = 2\sqrt{ h \eta/\xi }$ is a hydrodynamic length. 
%
%If $L(t) \simeq \Lambda \ll \lambda$, the right hand side becomes negligible and the \emph{internal problem} of unknowns $(v(y,t),\,L(t))$ becomes independent of the \emph{external problem} in $(v_s(y,t), C(t))$. Using the kinematic conditions \eqref{e:conservation_mass_bc_bis} we find that in the steady state, $L = \Lambda$ and $v = - y + \dot{C}$.




%\bibliographystyle{apsrev4-2}
%\bibliography{substrate_turnover_polarity,jocelyn}


%apsrev4-2.bst 2019-01-14 (MD) hand-edited version of apsrev4-1.bst
%Control: key (0)
%Control: author (72) initials jnrlst
%Control: editor formatted (1) identically to author
%Control: production of article title (-1) disabled
%Control: page (0) single
%Control: year (1) truncated
%Control: production of eprint (0) enabled
\begin{thebibliography}{54}%
\makeatletter
\providecommand \@ifxundefined [1]{%
 \@ifx{#1\undefined}
}%
\providecommand \@ifnum [1]{%
 \ifnum #1\expandafter \@firstoftwo
 \else \expandafter \@secondoftwo
 \fi
}%
\providecommand \@ifx [1]{%
 \ifx #1\expandafter \@firstoftwo
 \else \expandafter \@secondoftwo
 \fi
}%
\providecommand \natexlab [1]{#1}%
\providecommand \enquote  [1]{``#1''}%
\providecommand \bibnamefont  [1]{#1}%
\providecommand \bibfnamefont [1]{#1}%
\providecommand \citenamefont [1]{#1}%
\providecommand \href@noop [0]{\@secondoftwo}%
\providecommand \href [0]{\begingroup \@sanitize@url \@href}%
\providecommand \@href[1]{\@@startlink{#1}\@@href}%
\providecommand \@@href[1]{\endgroup#1\@@endlink}%
\providecommand \@sanitize@url [0]{\catcode `\\12\catcode `\$12\catcode
  `\&12\catcode `\#12\catcode `\^12\catcode `\_12\catcode `\%12\relax}%
\providecommand \@@startlink[1]{}%
\providecommand \@@endlink[0]{}%
\providecommand \url  [0]{\begingroup\@sanitize@url \@url }%
\providecommand \@url [1]{\endgroup\@href {#1}{\urlprefix }}%
\providecommand \urlprefix  [0]{URL }%
\providecommand \Eprint [0]{\href }%
\providecommand \doibase [0]{https://doi.org/}%
\providecommand \selectlanguage [0]{\@gobble}%
\providecommand \bibinfo  [0]{\@secondoftwo}%
\providecommand \bibfield  [0]{\@secondoftwo}%
\providecommand \translation [1]{[#1]}%
\providecommand \BibitemOpen [0]{}%
\providecommand \bibitemStop [0]{}%
\providecommand \bibitemNoStop [0]{.\EOS\space}%
\providecommand \EOS [0]{\spacefactor3000\relax}%
\providecommand \BibitemShut  [1]{\csname bibitem#1\endcsname}%
\let\auto@bib@innerbib\@empty
%</preamble>
\bibitem [{\citenamefont {Boal}(2012)}]{boal_2012}%
  \BibitemOpen
  \bibfield  {author} {\bibinfo {author} {\bibfnamefont {D.}~\bibnamefont
  {Boal}},\ }\href {https://doi.org/10.1017/CBO9781139022217} {\emph {\bibinfo
  {title} {Mechanics of the Cell}}},\ \bibinfo {edition} {2nd}\ ed.\ (\bibinfo
  {publisher} {Cambridge University Press},\ \bibinfo {year}
  {2012})\BibitemShut {NoStop}%
\bibitem [{\citenamefont {Julicher}\ \emph {et~al.}(2007)\citenamefont
  {Julicher}, \citenamefont {Kruse}, \citenamefont {Prost},\ and\ \citenamefont
  {Joanny}}]{Julicher+Joanny.2007.1}%
  \BibitemOpen
  \bibfield  {author} {\bibinfo {author} {\bibfnamefont {F.}~\bibnamefont
  {Julicher}}, \bibinfo {author} {\bibfnamefont {K.}~\bibnamefont {Kruse}},
  \bibinfo {author} {\bibfnamefont {J.}~\bibnamefont {Prost}},\ and\ \bibinfo
  {author} {\bibfnamefont {J.}~\bibnamefont {Joanny}},\ }\href
  {https://doi.org/10.1016/j.physrep.2007.02.018} {\bibfield  {journal}
  {\bibinfo  {journal} {Physics Reports}\ }\textbf {\bibinfo {volume} {449}},\
  \bibinfo {pages} {3} (\bibinfo {year} {2007})}\BibitemShut {NoStop}%
\bibitem [{\citenamefont {Carlsson}(2011)}]{carlsson2011mechanisms}%
  \BibitemOpen
  \bibfield  {author} {\bibinfo {author} {\bibfnamefont {A.}~\bibnamefont
  {Carlsson}},\ }\href@noop {} {\bibfield  {journal} {\bibinfo  {journal} {New
  journal of physics}\ }\textbf {\bibinfo {volume} {13}},\ \bibinfo {pages}
  {073009} (\bibinfo {year} {2011})}\BibitemShut {NoStop}%
\bibitem [{\citenamefont {Recho}\ and\ \citenamefont
  {Truskinovsky}(2016)}]{Recho-Truskinovsky.2015.1}%
  \BibitemOpen
  \bibfield  {author} {\bibinfo {author} {\bibfnamefont {P.}~\bibnamefont
  {Recho}}\ and\ \bibinfo {author} {\bibfnamefont {L.}~\bibnamefont
  {Truskinovsky}},\ }\href@noop {} {\bibfield  {journal} {\bibinfo  {journal}
  {Math. Mech. Solids}\ }\textbf {\bibinfo {volume} {21}},\ \bibinfo {pages}
  {263} (\bibinfo {year} {2016})}\BibitemShut {NoStop}%
\bibitem [{\citenamefont {Drozdowski}\ \emph {et~al.}(2021)\citenamefont
  {Drozdowski}, \citenamefont {Ziebert},\ and\ \citenamefont
  {Schwarz}}]{Drozdowski+Schwarz.2021.1}%
  \BibitemOpen
  \bibfield  {author} {\bibinfo {author} {\bibfnamefont {O.~M.}\ \bibnamefont
  {Drozdowski}}, \bibinfo {author} {\bibfnamefont {F.}~\bibnamefont
  {Ziebert}},\ and\ \bibinfo {author} {\bibfnamefont {U.~S.}\ \bibnamefont
  {Schwarz}},\ }\bibfield  {journal} {\bibinfo  {journal} {Phys. Rev. E}\
  }\textbf {\bibinfo {volume} {104}},\ \href
  {https://doi.org/10.1103/PhysRevE.104.024406} {10.1103/PhysRevE.104.024406}
  (\bibinfo {year} {2021})\BibitemShut {NoStop}%
\bibitem [{\citenamefont {Chelly}\ \emph {et~al.}(2022)\citenamefont {Chelly},
  \citenamefont {Jahangiri}, \citenamefont {Mireux}, \citenamefont
  {{\'E}tienne}, \citenamefont {Dysthe}, \citenamefont {Verdier},\ and\
  \citenamefont {Recho}}]{chelly2022cell}%
  \BibitemOpen
  \bibfield  {author} {\bibinfo {author} {\bibfnamefont {H.}~\bibnamefont
  {Chelly}}, \bibinfo {author} {\bibfnamefont {A.}~\bibnamefont {Jahangiri}},
  \bibinfo {author} {\bibfnamefont {M.}~\bibnamefont {Mireux}}, \bibinfo
  {author} {\bibfnamefont {J.}~\bibnamefont {{\'E}tienne}}, \bibinfo {author}
  {\bibfnamefont {D.}~\bibnamefont {Dysthe}}, \bibinfo {author} {\bibfnamefont
  {C.}~\bibnamefont {Verdier}},\ and\ \bibinfo {author} {\bibfnamefont
  {P.}~\bibnamefont {Recho}},\ }\href@noop {} {\bibfield  {journal} {\bibinfo
  {journal} {International Journal of Non-Linear Mechanics}\ }\textbf {\bibinfo
  {volume} {139}},\ \bibinfo {pages} {103897} (\bibinfo {year}
  {2022})}\BibitemShut {NoStop}%
\bibitem [{\citenamefont {Verkhovsky}\ \emph {et~al.}(1999)\citenamefont
  {Verkhovsky}, \citenamefont {Svitkina},\ and\ \citenamefont
  {Borisy}}]{verkhovsky1999self}%
  \BibitemOpen
  \bibfield  {author} {\bibinfo {author} {\bibfnamefont {A.~B.}\ \bibnamefont
  {Verkhovsky}}, \bibinfo {author} {\bibfnamefont {T.~M.}\ \bibnamefont
  {Svitkina}},\ and\ \bibinfo {author} {\bibfnamefont {G.~G.}\ \bibnamefont
  {Borisy}},\ }\href@noop {} {\bibfield  {journal} {\bibinfo  {journal}
  {Current Biology}\ }\textbf {\bibinfo {volume} {9}},\ \bibinfo {pages} {11}
  (\bibinfo {year} {1999})}\BibitemShut {NoStop}%
\bibitem [{\citenamefont {Yam}\ \emph {et~al.}(2007)\citenamefont {Yam},
  \citenamefont {Wilson}, \citenamefont {Ji}, \citenamefont {Hebert},
  \citenamefont {Barnhart}, \citenamefont {Dye}, \citenamefont {Wiseman},
  \citenamefont {Danuser},\ and\ \citenamefont {Theriot}}]{yam2007actin}%
  \BibitemOpen
  \bibfield  {author} {\bibinfo {author} {\bibfnamefont {P.~T.}\ \bibnamefont
  {Yam}}, \bibinfo {author} {\bibfnamefont {C.~A.}\ \bibnamefont {Wilson}},
  \bibinfo {author} {\bibfnamefont {L.}~\bibnamefont {Ji}}, \bibinfo {author}
  {\bibfnamefont {B.}~\bibnamefont {Hebert}}, \bibinfo {author} {\bibfnamefont
  {E.~L.}\ \bibnamefont {Barnhart}}, \bibinfo {author} {\bibfnamefont {N.~A.}\
  \bibnamefont {Dye}}, \bibinfo {author} {\bibfnamefont {P.~W.}\ \bibnamefont
  {Wiseman}}, \bibinfo {author} {\bibfnamefont {G.}~\bibnamefont {Danuser}},\
  and\ \bibinfo {author} {\bibfnamefont {J.~A.}\ \bibnamefont {Theriot}},\
  }\href@noop {} {\bibfield  {journal} {\bibinfo  {journal} {The Journal of
  cell biology}\ }\textbf {\bibinfo {volume} {178}},\ \bibinfo {pages} {1207}
  (\bibinfo {year} {2007})}\BibitemShut {NoStop}%
\bibitem [{\citenamefont {Barnhart}\ \emph {et~al.}(2015)\citenamefont
  {Barnhart}, \citenamefont {Lee}, \citenamefont {Allen}, \citenamefont
  {Theriot},\ and\ \citenamefont {Mogilner}}]{barnhart2015balance}%
  \BibitemOpen
  \bibfield  {author} {\bibinfo {author} {\bibfnamefont {E.}~\bibnamefont
  {Barnhart}}, \bibinfo {author} {\bibfnamefont {K.-C.}\ \bibnamefont {Lee}},
  \bibinfo {author} {\bibfnamefont {G.~M.}\ \bibnamefont {Allen}}, \bibinfo
  {author} {\bibfnamefont {J.~A.}\ \bibnamefont {Theriot}},\ and\ \bibinfo
  {author} {\bibfnamefont {A.}~\bibnamefont {Mogilner}},\ }\href@noop {}
  {\bibfield  {journal} {\bibinfo  {journal} {Proceedings of the National
  Academy of Sciences}\ }\textbf {\bibinfo {volume} {112}},\ \bibinfo {pages}
  {5045} (\bibinfo {year} {2015})}\BibitemShut {NoStop}%
\bibitem [{\citenamefont {Mori}\ \emph {et~al.}(2008)\citenamefont {Mori},
  \citenamefont {Jilkine},\ and\ \citenamefont
  {Edelstein-Keshet}}]{mori2008wave}%
  \BibitemOpen
  \bibfield  {author} {\bibinfo {author} {\bibfnamefont {Y.}~\bibnamefont
  {Mori}}, \bibinfo {author} {\bibfnamefont {A.}~\bibnamefont {Jilkine}},\ and\
  \bibinfo {author} {\bibfnamefont {L.}~\bibnamefont {Edelstein-Keshet}},\
  }\href@noop {} {\bibfield  {journal} {\bibinfo  {journal} {Biophysical
  journal}\ }\textbf {\bibinfo {volume} {94}},\ \bibinfo {pages} {3684}
  (\bibinfo {year} {2008})}\BibitemShut {NoStop}%
\bibitem [{\citenamefont {Copos}\ and\ \citenamefont
  {Mogilner}(2020)}]{copos2020hybrid}%
  \BibitemOpen
  \bibfield  {author} {\bibinfo {author} {\bibfnamefont {C.}~\bibnamefont
  {Copos}}\ and\ \bibinfo {author} {\bibfnamefont {A.}~\bibnamefont
  {Mogilner}},\ }\href@noop {} {\bibfield  {journal} {\bibinfo  {journal}
  {Molecular biology of the cell}\ }\textbf {\bibinfo {volume} {31}},\ \bibinfo
  {pages} {1637} (\bibinfo {year} {2020})}\BibitemShut {NoStop}%
\bibitem [{\citenamefont {Lavi}\ \emph {et~al.}(2020)\citenamefont {Lavi},
  \citenamefont {Meunier}, \citenamefont {Voituriez},\ and\ \citenamefont
  {Casademunt}}]{lavi2020motility}%
  \BibitemOpen
  \bibfield  {author} {\bibinfo {author} {\bibfnamefont {I.}~\bibnamefont
  {Lavi}}, \bibinfo {author} {\bibfnamefont {N.}~\bibnamefont {Meunier}},
  \bibinfo {author} {\bibfnamefont {R.}~\bibnamefont {Voituriez}},\ and\
  \bibinfo {author} {\bibfnamefont {J.}~\bibnamefont {Casademunt}},\
  }\href@noop {} {\bibfield  {journal} {\bibinfo  {journal} {Physical Review
  E}\ }\textbf {\bibinfo {volume} {101}},\ \bibinfo {pages} {022404} (\bibinfo
  {year} {2020})}\BibitemShut {NoStop}%
\bibitem [{\citenamefont {Sens}(2020)}]{Sens.2020.1}%
  \BibitemOpen
  \bibfield  {author} {\bibinfo {author} {\bibfnamefont {P.}~\bibnamefont
  {Sens}},\ }\href {https://doi.org/10.1073/pnas.2011785117} {\bibfield
  {journal} {\bibinfo  {journal} {Proc Natl Acad Sci USA}\ }\textbf {\bibinfo
  {volume} {117}},\ \bibinfo {pages} {24670} (\bibinfo {year}
  {2020})}\BibitemShut {NoStop}%
\bibitem [{\citenamefont {Ron}\ \emph {et~al.}(2020)\citenamefont {Ron},
  \citenamefont {Monzo}, \citenamefont {Gauthier}, \citenamefont {Voituriez},\
  and\ \citenamefont {Gov}}]{Ron+Gov.2020.1}%
  \BibitemOpen
  \bibfield  {author} {\bibinfo {author} {\bibfnamefont {J.~E.}\ \bibnamefont
  {Ron}}, \bibinfo {author} {\bibfnamefont {P.}~\bibnamefont {Monzo}}, \bibinfo
  {author} {\bibfnamefont {N.~C.}\ \bibnamefont {Gauthier}}, \bibinfo {author}
  {\bibfnamefont {R.}~\bibnamefont {Voituriez}},\ and\ \bibinfo {author}
  {\bibfnamefont {N.~S.}\ \bibnamefont {Gov}},\ }\href
  {https://doi.org/10.1103/PhysRevResearch.2.033237} {\bibfield  {journal}
  {\bibinfo  {journal} {Phys. Rev. Research}\ }\textbf {\bibinfo {volume}
  {2}},\ \bibinfo {pages} {2435} (\bibinfo {year} {2020})}\BibitemShut
  {NoStop}%
\bibitem [{\citenamefont {Recho}\ \emph {et~al.}(2013)\citenamefont {Recho},
  \citenamefont {Putelat},\ and\ \citenamefont
  {Truskinovsky}}]{recho2013contraction}%
  \BibitemOpen
  \bibfield  {author} {\bibinfo {author} {\bibfnamefont {P.}~\bibnamefont
  {Recho}}, \bibinfo {author} {\bibfnamefont {T.}~\bibnamefont {Putelat}},\
  and\ \bibinfo {author} {\bibfnamefont {L.}~\bibnamefont {Truskinovsky}},\
  }\href@noop {} {\bibfield  {journal} {\bibinfo  {journal} {Physical review
  letters}\ }\textbf {\bibinfo {volume} {111}},\ \bibinfo {pages} {108102}
  (\bibinfo {year} {2013})}\BibitemShut {NoStop}%
\bibitem [{\citenamefont {Tjhung}\ \emph {et~al.}(2012)\citenamefont {Tjhung},
  \citenamefont {Marenduzzo},\ and\ \citenamefont
  {Cates}}]{tjhung2012spontaneous}%
  \BibitemOpen
  \bibfield  {author} {\bibinfo {author} {\bibfnamefont {E.}~\bibnamefont
  {Tjhung}}, \bibinfo {author} {\bibfnamefont {D.}~\bibnamefont {Marenduzzo}},\
  and\ \bibinfo {author} {\bibfnamefont {M.~E.}\ \bibnamefont {Cates}},\
  }\href@noop {} {\bibfield  {journal} {\bibinfo  {journal} {Proceedings of the
  National Academy of Sciences}\ }\textbf {\bibinfo {volume} {109}},\ \bibinfo
  {pages} {12381} (\bibinfo {year} {2012})}\BibitemShut {NoStop}%
\bibitem [{\citenamefont {Callan-Jones}\ and\ \citenamefont
  {Voituriez}(2013)}]{callan2013active}%
  \BibitemOpen
  \bibfield  {author} {\bibinfo {author} {\bibfnamefont {A.}~\bibnamefont
  {Callan-Jones}}\ and\ \bibinfo {author} {\bibfnamefont {R.}~\bibnamefont
  {Voituriez}},\ }\href@noop {} {\bibfield  {journal} {\bibinfo  {journal} {New
  Journal of Physics}\ }\textbf {\bibinfo {volume} {15}},\ \bibinfo {pages}
  {025022} (\bibinfo {year} {2013})}\BibitemShut {NoStop}%
\bibitem [{\citenamefont {Giomi}\ and\ \citenamefont
  {DeSimone}(2014)}]{giomi2014spontaneous}%
  \BibitemOpen
  \bibfield  {author} {\bibinfo {author} {\bibfnamefont {L.}~\bibnamefont
  {Giomi}}\ and\ \bibinfo {author} {\bibfnamefont {A.}~\bibnamefont
  {DeSimone}},\ }\href@noop {} {\bibfield  {journal} {\bibinfo  {journal}
  {Physical review letters}\ }\textbf {\bibinfo {volume} {112}},\ \bibinfo
  {pages} {147802} (\bibinfo {year} {2014})}\BibitemShut {NoStop}%
\bibitem [{\citenamefont {Blanch-Mercader}\ and\ \citenamefont
  {Casademunt}(2013)}]{blanch2013spontaneous}%
  \BibitemOpen
  \bibfield  {author} {\bibinfo {author} {\bibfnamefont {C.}~\bibnamefont
  {Blanch-Mercader}}\ and\ \bibinfo {author} {\bibfnamefont {J.}~\bibnamefont
  {Casademunt}},\ }\href@noop {} {\bibfield  {journal} {\bibinfo  {journal}
  {Physical review letters}\ }\textbf {\bibinfo {volume} {110}},\ \bibinfo
  {pages} {078102} (\bibinfo {year} {2013})}\BibitemShut {NoStop}%
\bibitem [{\citenamefont {Maiuri}\ \emph {et~al.}(2012)\citenamefont {Maiuri},
  \citenamefont {Terriac}, \citenamefont {Paul-Gilloteaux}, \citenamefont
  {Vignaud}, \citenamefont {McNally}, \citenamefont {Onuffer}, \citenamefont
  {Thorn}, \citenamefont {Nguyen}, \citenamefont {Georgoulia}, \citenamefont
  {Soong} \emph {et~al.}}]{maiuri2012first}%
  \BibitemOpen
  \bibfield  {author} {\bibinfo {author} {\bibfnamefont {P.}~\bibnamefont
  {Maiuri}}, \bibinfo {author} {\bibfnamefont {E.}~\bibnamefont {Terriac}},
  \bibinfo {author} {\bibfnamefont {P.}~\bibnamefont {Paul-Gilloteaux}},
  \bibinfo {author} {\bibfnamefont {T.}~\bibnamefont {Vignaud}}, \bibinfo
  {author} {\bibfnamefont {K.}~\bibnamefont {McNally}}, \bibinfo {author}
  {\bibfnamefont {J.}~\bibnamefont {Onuffer}}, \bibinfo {author} {\bibfnamefont
  {K.}~\bibnamefont {Thorn}}, \bibinfo {author} {\bibfnamefont {P.~A.}\
  \bibnamefont {Nguyen}}, \bibinfo {author} {\bibfnamefont {N.}~\bibnamefont
  {Georgoulia}}, \bibinfo {author} {\bibfnamefont {D.}~\bibnamefont {Soong}},
  \emph {et~al.},\ }\href@noop {} {\bibfield  {journal} {\bibinfo  {journal}
  {Current Biology}\ }\textbf {\bibinfo {volume} {22}},\ \bibinfo {pages}
  {R673} (\bibinfo {year} {2012})}\BibitemShut {NoStop}%
\bibitem [{\citenamefont {Mohammed}\ \emph {et~al.}(2019)\citenamefont
  {Mohammed}, \citenamefont {Charras}, \citenamefont {Vercruysse},
  \citenamefont {Versaevel}, \citenamefont {Lantoine}, \citenamefont {Alaimo},
  \citenamefont {Bruy{\`e}re}, \citenamefont {Luciano}, \citenamefont {Glinel},
  \citenamefont {Delhaye} \emph {et~al.}}]{Mohammed+Gabriele.2019.1}%
  \BibitemOpen
  \bibfield  {author} {\bibinfo {author} {\bibfnamefont {D.}~\bibnamefont
  {Mohammed}}, \bibinfo {author} {\bibfnamefont {G.}~\bibnamefont {Charras}},
  \bibinfo {author} {\bibfnamefont {E.}~\bibnamefont {Vercruysse}}, \bibinfo
  {author} {\bibfnamefont {M.}~\bibnamefont {Versaevel}}, \bibinfo {author}
  {\bibfnamefont {J.}~\bibnamefont {Lantoine}}, \bibinfo {author}
  {\bibfnamefont {L.}~\bibnamefont {Alaimo}}, \bibinfo {author} {\bibfnamefont
  {C.}~\bibnamefont {Bruy{\`e}re}}, \bibinfo {author} {\bibfnamefont
  {M.}~\bibnamefont {Luciano}}, \bibinfo {author} {\bibfnamefont
  {K.}~\bibnamefont {Glinel}}, \bibinfo {author} {\bibfnamefont
  {G.}~\bibnamefont {Delhaye}}, \emph {et~al.},\ }\href@noop {} {\bibfield
  {journal} {\bibinfo  {journal} {Nature Physics}\ }\textbf {\bibinfo {volume}
  {15}},\ \bibinfo {pages} {858} (\bibinfo {year} {2019})}\BibitemShut
  {NoStop}%
\bibitem [{\citenamefont {Doyle}\ \emph {et~al.}(2013)\citenamefont {Doyle},
  \citenamefont {Petrie}, \citenamefont {Kutys},\ and\ \citenamefont
  {Yamada}}]{doyle2013dimensions}%
  \BibitemOpen
  \bibfield  {author} {\bibinfo {author} {\bibfnamefont {A.~D.}\ \bibnamefont
  {Doyle}}, \bibinfo {author} {\bibfnamefont {R.~J.}\ \bibnamefont {Petrie}},
  \bibinfo {author} {\bibfnamefont {M.~L.}\ \bibnamefont {Kutys}},\ and\
  \bibinfo {author} {\bibfnamefont {K.~M.}\ \bibnamefont {Yamada}},\
  }\href@noop {} {\bibfield  {journal} {\bibinfo  {journal} {Current opinion in
  cell biology}\ }\textbf {\bibinfo {volume} {25}},\ \bibinfo {pages} {642}
  (\bibinfo {year} {2013})}\BibitemShut {NoStop}%
\bibitem [{\citenamefont {Lo}\ \emph {et~al.}(2000)\citenamefont {Lo},
  \citenamefont {Wang}, \citenamefont {Dembo},\ and\ \citenamefont
  {Wang}}]{lo2000cell}%
  \BibitemOpen
  \bibfield  {author} {\bibinfo {author} {\bibfnamefont {C.-M.}\ \bibnamefont
  {Lo}}, \bibinfo {author} {\bibfnamefont {H.-B.}\ \bibnamefont {Wang}},
  \bibinfo {author} {\bibfnamefont {M.}~\bibnamefont {Dembo}},\ and\ \bibinfo
  {author} {\bibfnamefont {Y.-l.}\ \bibnamefont {Wang}},\ }\href@noop {}
  {\bibfield  {journal} {\bibinfo  {journal} {Biophysical journal}\ }\textbf
  {\bibinfo {volume} {79}},\ \bibinfo {pages} {144} (\bibinfo {year}
  {2000})}\BibitemShut {NoStop}%
\bibitem [{\citenamefont {DuChez}\ \emph {et~al.}(2019)\citenamefont {DuChez},
  \citenamefont {Doyle}, \citenamefont {Dimitriadis},\ and\ \citenamefont
  {Yamada}}]{duchez2019durotaxis}%
  \BibitemOpen
  \bibfield  {author} {\bibinfo {author} {\bibfnamefont {B.~J.}\ \bibnamefont
  {DuChez}}, \bibinfo {author} {\bibfnamefont {A.~D.}\ \bibnamefont {Doyle}},
  \bibinfo {author} {\bibfnamefont {E.~K.}\ \bibnamefont {Dimitriadis}},\ and\
  \bibinfo {author} {\bibfnamefont {K.~M.}\ \bibnamefont {Yamada}},\
  }\href@noop {} {\bibfield  {journal} {\bibinfo  {journal} {Biophysical
  journal}\ }\textbf {\bibinfo {volume} {116}},\ \bibinfo {pages} {670}
  (\bibinfo {year} {2019})}\BibitemShut {NoStop}%
\bibitem [{\citenamefont {Lelidis}\ and\ \citenamefont
  {Joanny}(2013)}]{lelidis2013interaction}%
  \BibitemOpen
  \bibfield  {author} {\bibinfo {author} {\bibfnamefont {I.}~\bibnamefont
  {Lelidis}}\ and\ \bibinfo {author} {\bibfnamefont {J.-F.}\ \bibnamefont
  {Joanny}},\ }\href@noop {} {\bibfield  {journal} {\bibinfo  {journal} {Soft
  Matter}\ }\textbf {\bibinfo {volume} {9}},\ \bibinfo {pages} {11120}
  (\bibinfo {year} {2013})}\BibitemShut {NoStop}%
\bibitem [{\citenamefont {L{\"o}ber}\ \emph {et~al.}(2014)\citenamefont
  {L{\"o}ber}, \citenamefont {Ziebert},\ and\ \citenamefont
  {Aranson}}]{lober2014modeling}%
  \BibitemOpen
  \bibfield  {author} {\bibinfo {author} {\bibfnamefont {J.}~\bibnamefont
  {L{\"o}ber}}, \bibinfo {author} {\bibfnamefont {F.}~\bibnamefont {Ziebert}},\
  and\ \bibinfo {author} {\bibfnamefont {I.~S.}\ \bibnamefont {Aranson}},\
  }\href@noop {} {\bibfield  {journal} {\bibinfo  {journal} {Soft matter}\
  }\textbf {\bibinfo {volume} {10}},\ \bibinfo {pages} {1365} (\bibinfo {year}
  {2014})}\BibitemShut {NoStop}%
\bibitem [{\citenamefont {Feng}\ \emph {et~al.}(2019)\citenamefont {Feng},
  \citenamefont {Levine}, \citenamefont {Mao},\ and\ \citenamefont
  {Sander}}]{feng2019cell}%
  \BibitemOpen
  \bibfield  {author} {\bibinfo {author} {\bibfnamefont {J.}~\bibnamefont
  {Feng}}, \bibinfo {author} {\bibfnamefont {H.}~\bibnamefont {Levine}},
  \bibinfo {author} {\bibfnamefont {X.}~\bibnamefont {Mao}},\ and\ \bibinfo
  {author} {\bibfnamefont {L.~M.}\ \bibnamefont {Sander}},\ }\href@noop {}
  {\bibfield  {journal} {\bibinfo  {journal} {Soft matter}\ }\textbf {\bibinfo
  {volume} {15}},\ \bibinfo {pages} {4856} (\bibinfo {year}
  {2019})}\BibitemShut {NoStop}%
\bibitem [{\citenamefont {Sens}(2013)}]{sens2013rigidity}%
  \BibitemOpen
  \bibfield  {author} {\bibinfo {author} {\bibfnamefont {P.}~\bibnamefont
  {Sens}},\ }\href@noop {} {\bibfield  {journal} {\bibinfo  {journal} {EPL
  (Europhysics Letters)}\ }\textbf {\bibinfo {volume} {104}},\ \bibinfo {pages}
  {38003} (\bibinfo {year} {2013})}\BibitemShut {NoStop}%
\bibitem [{\citenamefont {Chen}\ \emph {et~al.}(2022)\citenamefont {Chen},
  \citenamefont {Feng},\ and\ \citenamefont {Li}}]{chen2022unified}%
  \BibitemOpen
  \bibfield  {author} {\bibinfo {author} {\bibfnamefont {P.-C.}\ \bibnamefont
  {Chen}}, \bibinfo {author} {\bibfnamefont {X.-Q.}\ \bibnamefont {Feng}},\
  and\ \bibinfo {author} {\bibfnamefont {B.}~\bibnamefont {Li}},\ }\href@noop
  {} {\bibfield  {journal} {\bibinfo  {journal} {Biophysical Journal}\ }\textbf
  {\bibinfo {volume} {121}},\ \bibinfo {pages} {3474} (\bibinfo {year}
  {2022})}\BibitemShut {NoStop}%
\bibitem [{\citenamefont {Banerjee}\ and\ \citenamefont
  {Marchetti}(2011)}]{banerjee2011substrate}%
  \BibitemOpen
  \bibfield  {author} {\bibinfo {author} {\bibfnamefont {S.}~\bibnamefont
  {Banerjee}}\ and\ \bibinfo {author} {\bibfnamefont {M.~C.}\ \bibnamefont
  {Marchetti}},\ }\href@noop {} {\bibfield  {journal} {\bibinfo  {journal} {EPL
  (Europhysics Letters)}\ }\textbf {\bibinfo {volume} {96}},\ \bibinfo {pages}
  {28003} (\bibinfo {year} {2011})}\BibitemShut {NoStop}%
\bibitem [{\citenamefont {Shenoy}\ \emph {et~al.}(2016)\citenamefont {Shenoy},
  \citenamefont {Wang},\ and\ \citenamefont {Wang}}]{shenoy2016chemo}%
  \BibitemOpen
  \bibfield  {author} {\bibinfo {author} {\bibfnamefont {V.~B.}\ \bibnamefont
  {Shenoy}}, \bibinfo {author} {\bibfnamefont {H.}~\bibnamefont {Wang}},\ and\
  \bibinfo {author} {\bibfnamefont {X.}~\bibnamefont {Wang}},\ }\href@noop {}
  {\bibfield  {journal} {\bibinfo  {journal} {Interface focus}\ }\textbf
  {\bibinfo {volume} {6}},\ \bibinfo {pages} {20150067} (\bibinfo {year}
  {2016})}\BibitemShut {NoStop}%
\bibitem [{\citenamefont {Zhang}\ \emph {et~al.}(2020)\citenamefont {Zhang},
  \citenamefont {Rosakis}, \citenamefont {Hou},\ and\ \citenamefont
  {Ravichandran}}]{zhang2020minimal}%
  \BibitemOpen
  \bibfield  {author} {\bibinfo {author} {\bibfnamefont {Z.}~\bibnamefont
  {Zhang}}, \bibinfo {author} {\bibfnamefont {P.}~\bibnamefont {Rosakis}},
  \bibinfo {author} {\bibfnamefont {T.~Y.}\ \bibnamefont {Hou}},\ and\ \bibinfo
  {author} {\bibfnamefont {G.}~\bibnamefont {Ravichandran}},\ }\href@noop {}
  {\bibfield  {journal} {\bibinfo  {journal} {Journal of the Royal Society
  Interface}\ }\textbf {\bibinfo {volume} {17}},\ \bibinfo {pages} {20200175}
  (\bibinfo {year} {2020})}\BibitemShut {NoStop}%
\bibitem [{\citenamefont {Oliveri}\ \emph {et~al.}(2021)\citenamefont
  {Oliveri}, \citenamefont {Franze},\ and\ \citenamefont
  {Goriely}}]{oliveri2021theory}%
  \BibitemOpen
  \bibfield  {author} {\bibinfo {author} {\bibfnamefont {H.}~\bibnamefont
  {Oliveri}}, \bibinfo {author} {\bibfnamefont {K.}~\bibnamefont {Franze}},\
  and\ \bibinfo {author} {\bibfnamefont {A.}~\bibnamefont {Goriely}},\
  }\href@noop {} {\bibfield  {journal} {\bibinfo  {journal} {Physical Review
  Letters}\ }\textbf {\bibinfo {volume} {126}},\ \bibinfo {pages} {118101}
  (\bibinfo {year} {2021})}\BibitemShut {NoStop}%
\bibitem [{\citenamefont {Kruse}\ \emph {et~al.}(2005)\citenamefont {Kruse},
  \citenamefont {Joanny}, \citenamefont {J{\"u}licher}, \citenamefont {Prost},\
  and\ \citenamefont {Sekimoto}}]{kruse2005generic}%
  \BibitemOpen
  \bibfield  {author} {\bibinfo {author} {\bibfnamefont {K.}~\bibnamefont
  {Kruse}}, \bibinfo {author} {\bibfnamefont {J.-F.}\ \bibnamefont {Joanny}},
  \bibinfo {author} {\bibfnamefont {F.}~\bibnamefont {J{\"u}licher}}, \bibinfo
  {author} {\bibfnamefont {J.}~\bibnamefont {Prost}},\ and\ \bibinfo {author}
  {\bibfnamefont {K.}~\bibnamefont {Sekimoto}},\ }\href@noop {} {\bibfield
  {journal} {\bibinfo  {journal} {The European Physical Journal E}\ }\textbf
  {\bibinfo {volume} {16}},\ \bibinfo {pages} {5} (\bibinfo {year}
  {2005})}\BibitemShut {NoStop}%
\bibitem [{\citenamefont {Kruse}\ \emph {et~al.}(2006)\citenamefont {Kruse},
  \citenamefont {Joanny}, \citenamefont {J{\"u}licher},\ and\ \citenamefont
  {Prost}}]{kruse2006contractility}%
  \BibitemOpen
  \bibfield  {author} {\bibinfo {author} {\bibfnamefont {K.}~\bibnamefont
  {Kruse}}, \bibinfo {author} {\bibfnamefont {J.}~\bibnamefont {Joanny}},
  \bibinfo {author} {\bibfnamefont {F.}~\bibnamefont {J{\"u}licher}},\ and\
  \bibinfo {author} {\bibfnamefont {J.}~\bibnamefont {Prost}},\ }\href@noop {}
  {\bibfield  {journal} {\bibinfo  {journal} {Physical biology}\ }\textbf
  {\bibinfo {volume} {3}},\ \bibinfo {pages} {130} (\bibinfo {year}
  {2006})}\BibitemShut {NoStop}%
\bibitem [{\citenamefont {\'Etienne}\ \emph {et~al.}(2015)\citenamefont
  {\'Etienne}, \citenamefont {Fouchard}, \citenamefont {Mitrossilis},
  \citenamefont {Bufi}, \citenamefont {Durand-Smet},\ and\ \citenamefont
  {Asnacios}}]{Etienne+Asnacios.2015.1}%
  \BibitemOpen
  \bibfield  {author} {\bibinfo {author} {\bibfnamefont {J.}~\bibnamefont
  {\'Etienne}}, \bibinfo {author} {\bibfnamefont {J.}~\bibnamefont {Fouchard}},
  \bibinfo {author} {\bibfnamefont {D.}~\bibnamefont {Mitrossilis}}, \bibinfo
  {author} {\bibfnamefont {N.}~\bibnamefont {Bufi}}, \bibinfo {author}
  {\bibfnamefont {P.}~\bibnamefont {Durand-Smet}},\ and\ \bibinfo {author}
  {\bibfnamefont {A.}~\bibnamefont {Asnacios}},\ }\href
  {https://doi.org/10.1073/pnas.1417113112} {\bibfield  {journal} {\bibinfo
  {journal} {Proc Natl Acad Sci USA}\ }\textbf {\bibinfo {volume} {112}},\
  \bibinfo {pages} {2740} (\bibinfo {year} {2015})}\BibitemShut {NoStop}%
\bibitem [{\citenamefont {Schreiber}\ \emph {et~al.}(2021)\citenamefont
  {Schreiber}, \citenamefont {Amiri}, \citenamefont {Heyn}, \citenamefont
  {R{\"a}dler},\ and\ \citenamefont {Falcke}}]{schreiber2021adhesion}%
  \BibitemOpen
  \bibfield  {author} {\bibinfo {author} {\bibfnamefont {C.}~\bibnamefont
  {Schreiber}}, \bibinfo {author} {\bibfnamefont {B.}~\bibnamefont {Amiri}},
  \bibinfo {author} {\bibfnamefont {J.~C.}\ \bibnamefont {Heyn}}, \bibinfo
  {author} {\bibfnamefont {J.~O.}\ \bibnamefont {R{\"a}dler}},\ and\ \bibinfo
  {author} {\bibfnamefont {M.}~\bibnamefont {Falcke}},\ }\href@noop {}
  {\bibfield  {journal} {\bibinfo  {journal} {Proceedings of the National
  Academy of Sciences}\ }\textbf {\bibinfo {volume} {118}},\ \bibinfo {pages}
  {e2009959118} (\bibinfo {year} {2021})}\BibitemShut {NoStop}%
\bibitem [{\citenamefont {Gardel}\ \emph {et~al.}(2008)\citenamefont {Gardel},
  \citenamefont {Sabass}, \citenamefont {Ji}, \citenamefont {Danuser},
  \citenamefont {Schwarz},\ and\ \citenamefont
  {Waterman}}]{gardel2008traction}%
  \BibitemOpen
  \bibfield  {author} {\bibinfo {author} {\bibfnamefont {M.~L.}\ \bibnamefont
  {Gardel}}, \bibinfo {author} {\bibfnamefont {B.}~\bibnamefont {Sabass}},
  \bibinfo {author} {\bibfnamefont {L.}~\bibnamefont {Ji}}, \bibinfo {author}
  {\bibfnamefont {G.}~\bibnamefont {Danuser}}, \bibinfo {author} {\bibfnamefont
  {U.~S.}\ \bibnamefont {Schwarz}},\ and\ \bibinfo {author} {\bibfnamefont
  {C.~M.}\ \bibnamefont {Waterman}},\ }\href@noop {} {\bibfield  {journal}
  {\bibinfo  {journal} {The Journal of cell biology}\ }\textbf {\bibinfo
  {volume} {183}},\ \bibinfo {pages} {999} (\bibinfo {year}
  {2008})}\BibitemShut {NoStop}%
\bibitem [{\citenamefont {Sabass}\ and\ \citenamefont
  {Schwarz}(2010)}]{sabass2010modeling}%
  \BibitemOpen
  \bibfield  {author} {\bibinfo {author} {\bibfnamefont {B.}~\bibnamefont
  {Sabass}}\ and\ \bibinfo {author} {\bibfnamefont {U.~S.}\ \bibnamefont
  {Schwarz}},\ }\href@noop {} {\bibfield  {journal} {\bibinfo  {journal}
  {Journal of Physics: Condensed Matter}\ }\textbf {\bibinfo {volume} {22}},\
  \bibinfo {pages} {194112} (\bibinfo {year} {2010})}\BibitemShut {NoStop}%
\bibitem [{\citenamefont {Landau}\ and\ \citenamefont
  {Lifshitz}(1984)}]{Landau+.1984.8}%
  \BibitemOpen
  \bibfield  {author} {\bibinfo {author} {\bibfnamefont {L.~D.}\ \bibnamefont
  {Landau}}\ and\ \bibinfo {author} {\bibfnamefont {E.}~\bibnamefont
  {Lifshitz}},\ }\href@noop {} {\emph {\bibinfo {title} {Electrodynamics of
  Continuous Media}}},\ \bibinfo {series} {Course of Theoretical Physics},
  Vol.~\bibinfo {volume} {8}\ (\bibinfo  {publisher} {Pergamon Press},\
  \bibinfo {address} {Oxford},\ \bibinfo {year} {1984})\ \bibinfo {note} {2nd
  ed}\BibitemShut {NoStop}%
\bibitem [{\citenamefont {Johnson}(1987)}]{Johnson1987}%
  \BibitemOpen
  \bibfield  {author} {\bibinfo {author} {\bibfnamefont {K.~L.}\ \bibnamefont
  {Johnson}},\ }\href@noop {} {\emph {\bibinfo {title} {Contact mechanics}}}\
  (\bibinfo  {publisher} {Cambridge university press},\ \bibinfo {year}
  {1987})\BibitemShut {NoStop}%
\bibitem [{\citenamefont {Estrada}\ and\ \citenamefont
  {Kanwal}(1989)}]{estrada1989integral}%
  \BibitemOpen
  \bibfield  {author} {\bibinfo {author} {\bibfnamefont {R.}~\bibnamefont
  {Estrada}}\ and\ \bibinfo {author} {\bibfnamefont {R.}~\bibnamefont
  {Kanwal}},\ }\href@noop {} {\bibfield  {journal} {\bibinfo  {journal} {IMA
  journal of applied mathematics}\ }\textbf {\bibinfo {volume} {43}},\ \bibinfo
  {pages} {133} (\bibinfo {year} {1989})}\BibitemShut {NoStop}%
\bibitem [{\citenamefont {Bigoni}\ and\ \citenamefont
  {Noselli}(2011)}]{bigoni2011experimental}%
  \BibitemOpen
  \bibfield  {author} {\bibinfo {author} {\bibfnamefont {D.}~\bibnamefont
  {Bigoni}}\ and\ \bibinfo {author} {\bibfnamefont {G.}~\bibnamefont
  {Noselli}},\ }\href@noop {} {\bibfield  {journal} {\bibinfo  {journal}
  {Journal of the Mechanics and Physics of Solids}\ }\textbf {\bibinfo {volume}
  {59}},\ \bibinfo {pages} {2208} (\bibinfo {year} {2011})}\BibitemShut
  {NoStop}%
\bibitem [{\citenamefont {Karpenko}(1966)}]{Karpenko1966approximate}%
  \BibitemOpen
  \bibfield  {author} {\bibinfo {author} {\bibfnamefont {L.}~\bibnamefont
  {Karpenko}},\ }\href
  {https://doi.org/https://doi.org/10.1016/0021-8928(67)90103-7} {\bibfield
  {journal} {\bibinfo  {journal} {Journal of Applied Mathematics and
  Mechanics}\ }\textbf {\bibinfo {volume} {30}},\ \bibinfo {pages} {668}
  (\bibinfo {year} {1966})}\BibitemShut {NoStop}%
\bibitem [{\citenamefont {Rubinstein}\ \emph {et~al.}(2009)\citenamefont
  {Rubinstein}, \citenamefont {Fournier}, \citenamefont {Jacobson},
  \citenamefont {Verkhovsky},\ and\ \citenamefont
  {Mogilner}}]{rubinstein2009actin}%
  \BibitemOpen
  \bibfield  {author} {\bibinfo {author} {\bibfnamefont {B.}~\bibnamefont
  {Rubinstein}}, \bibinfo {author} {\bibfnamefont {M.~F.}\ \bibnamefont
  {Fournier}}, \bibinfo {author} {\bibfnamefont {K.}~\bibnamefont {Jacobson}},
  \bibinfo {author} {\bibfnamefont {A.~B.}\ \bibnamefont {Verkhovsky}},\ and\
  \bibinfo {author} {\bibfnamefont {A.}~\bibnamefont {Mogilner}},\ }\href@noop
  {} {\bibfield  {journal} {\bibinfo  {journal} {Biophysical journal}\ }\textbf
  {\bibinfo {volume} {97}},\ \bibinfo {pages} {1853} (\bibinfo {year}
  {2009})}\BibitemShut {NoStop}%
\bibitem [{\citenamefont {Barnhart}\ \emph {et~al.}(2011)\citenamefont
  {Barnhart}, \citenamefont {Lee}, \citenamefont {Keren}, \citenamefont
  {Mogilner},\ and\ \citenamefont {Theriot}}]{barnhart2011adhesion}%
  \BibitemOpen
  \bibfield  {author} {\bibinfo {author} {\bibfnamefont {E.~L.}\ \bibnamefont
  {Barnhart}}, \bibinfo {author} {\bibfnamefont {K.-C.}\ \bibnamefont {Lee}},
  \bibinfo {author} {\bibfnamefont {K.}~\bibnamefont {Keren}}, \bibinfo
  {author} {\bibfnamefont {A.}~\bibnamefont {Mogilner}},\ and\ \bibinfo
  {author} {\bibfnamefont {J.~A.}\ \bibnamefont {Theriot}},\ }\href@noop {}
  {\bibfield  {journal} {\bibinfo  {journal} {PLoS biology}\ }\textbf {\bibinfo
  {volume} {9}},\ \bibinfo {pages} {e1001059} (\bibinfo {year}
  {2011})}\BibitemShut {NoStop}%
\bibitem [{\citenamefont {Levental}\ \emph {et~al.}(2009)\citenamefont
  {Levental}, \citenamefont {Yu}, \citenamefont {Kass}, \citenamefont {Lakins},
  \citenamefont {Egeblad}, \citenamefont {Erler}, \citenamefont {Fong},
  \citenamefont {Csiszar}, \citenamefont {Giaccia}, \citenamefont {Weninger}
  \emph {et~al.}}]{levental2009matrix}%
  \BibitemOpen
  \bibfield  {author} {\bibinfo {author} {\bibfnamefont {K.~R.}\ \bibnamefont
  {Levental}}, \bibinfo {author} {\bibfnamefont {H.}~\bibnamefont {Yu}},
  \bibinfo {author} {\bibfnamefont {L.}~\bibnamefont {Kass}}, \bibinfo {author}
  {\bibfnamefont {J.~N.}\ \bibnamefont {Lakins}}, \bibinfo {author}
  {\bibfnamefont {M.}~\bibnamefont {Egeblad}}, \bibinfo {author} {\bibfnamefont
  {J.~T.}\ \bibnamefont {Erler}}, \bibinfo {author} {\bibfnamefont {S.~F.}\
  \bibnamefont {Fong}}, \bibinfo {author} {\bibfnamefont {K.}~\bibnamefont
  {Csiszar}}, \bibinfo {author} {\bibfnamefont {A.}~\bibnamefont {Giaccia}},
  \bibinfo {author} {\bibfnamefont {W.}~\bibnamefont {Weninger}}, \emph
  {et~al.},\ }\href@noop {} {\bibfield  {journal} {\bibinfo  {journal} {Cell}\
  }\textbf {\bibinfo {volume} {139}},\ \bibinfo {pages} {891} (\bibinfo {year}
  {2009})}\BibitemShut {NoStop}%
\bibitem [{\citenamefont {Dolega}\ \emph {et~al.}(2021)\citenamefont {Dolega},
  \citenamefont {Zurlo}, \citenamefont {Le~Goff}, \citenamefont {Greda},
  \citenamefont {Verdier}, \citenamefont {Joanny}, \citenamefont {Cappello},\
  and\ \citenamefont {Recho}}]{dolega2021mechanical}%
  \BibitemOpen
  \bibfield  {author} {\bibinfo {author} {\bibfnamefont {M.}~\bibnamefont
  {Dolega}}, \bibinfo {author} {\bibfnamefont {G.}~\bibnamefont {Zurlo}},
  \bibinfo {author} {\bibfnamefont {M.}~\bibnamefont {Le~Goff}}, \bibinfo
  {author} {\bibfnamefont {M.}~\bibnamefont {Greda}}, \bibinfo {author}
  {\bibfnamefont {C.}~\bibnamefont {Verdier}}, \bibinfo {author} {\bibfnamefont
  {J.-F.}\ \bibnamefont {Joanny}}, \bibinfo {author} {\bibfnamefont
  {G.}~\bibnamefont {Cappello}},\ and\ \bibinfo {author} {\bibfnamefont
  {P.}~\bibnamefont {Recho}},\ }\href@noop {} {\bibfield  {journal} {\bibinfo
  {journal} {Journal of the Mechanics and Physics of Solids}\ }\textbf
  {\bibinfo {volume} {147}},\ \bibinfo {pages} {104205} (\bibinfo {year}
  {2021})}\BibitemShut {NoStop}%
\bibitem [{\citenamefont {Hennig}\ \emph {et~al.}(2020)\citenamefont {Hennig},
  \citenamefont {Wang}, \citenamefont {Moreau}, \citenamefont {Valon},
  \citenamefont {DeBeco}, \citenamefont {Coppey}, \citenamefont {Miroshnikova},
  \citenamefont {Albiges-Rizo}, \citenamefont {Favard}, \citenamefont
  {Voituriez} \emph {et~al.}}]{hennig2020stick}%
  \BibitemOpen
  \bibfield  {author} {\bibinfo {author} {\bibfnamefont {K.}~\bibnamefont
  {Hennig}}, \bibinfo {author} {\bibfnamefont {I.}~\bibnamefont {Wang}},
  \bibinfo {author} {\bibfnamefont {P.}~\bibnamefont {Moreau}}, \bibinfo
  {author} {\bibfnamefont {L.}~\bibnamefont {Valon}}, \bibinfo {author}
  {\bibfnamefont {S.}~\bibnamefont {DeBeco}}, \bibinfo {author} {\bibfnamefont
  {M.}~\bibnamefont {Coppey}}, \bibinfo {author} {\bibfnamefont
  {Y.}~\bibnamefont {Miroshnikova}}, \bibinfo {author} {\bibfnamefont
  {C.}~\bibnamefont {Albiges-Rizo}}, \bibinfo {author} {\bibfnamefont
  {C.}~\bibnamefont {Favard}}, \bibinfo {author} {\bibfnamefont
  {R.}~\bibnamefont {Voituriez}}, \emph {et~al.},\ }\href@noop {} {\bibfield
  {journal} {\bibinfo  {journal} {Science advances}\ }\textbf {\bibinfo
  {volume} {6}},\ \bibinfo {pages} {eaau5670} (\bibinfo {year}
  {2020})}\BibitemShut {NoStop}%
\bibitem [{\citenamefont {Ponti}\ \emph {et~al.}(2004)\citenamefont {Ponti},
  \citenamefont {Machacek}, \citenamefont {Gupton}, \citenamefont
  {Waterman-Storer},\ and\ \citenamefont {Danuser}}]{ponti2004two}%
  \BibitemOpen
  \bibfield  {author} {\bibinfo {author} {\bibfnamefont {A.}~\bibnamefont
  {Ponti}}, \bibinfo {author} {\bibfnamefont {M.}~\bibnamefont {Machacek}},
  \bibinfo {author} {\bibfnamefont {S.~L.}\ \bibnamefont {Gupton}}, \bibinfo
  {author} {\bibfnamefont {C.~M.}\ \bibnamefont {Waterman-Storer}},\ and\
  \bibinfo {author} {\bibfnamefont {G.}~\bibnamefont {Danuser}},\ }\href@noop
  {} {\bibfield  {journal} {\bibinfo  {journal} {Science}\ }\textbf {\bibinfo
  {volume} {305}},\ \bibinfo {pages} {1782} (\bibinfo {year}
  {2004})}\BibitemShut {NoStop}%
\bibitem [{\citenamefont {Ambrosi}\ and\ \citenamefont
  {Zanzottera}(2016)}]{Ambrosi-Zanzottera.2016.1}%
  \BibitemOpen
  \bibfield  {author} {\bibinfo {author} {\bibfnamefont {D.}~\bibnamefont
  {Ambrosi}}\ and\ \bibinfo {author} {\bibfnamefont {A.}~\bibnamefont
  {Zanzottera}},\ }\href {https://doi.org/10.1016/j.physd.2016.05.003}
  {\bibfield  {journal} {\bibinfo  {journal} {Physica D}\ ,\ \bibinfo {pages}
  {58}} (\bibinfo {year} {2016})}\BibitemShut {NoStop}%
\bibitem [{\citenamefont {Putelat}\ \emph {et~al.}(2018)\citenamefont
  {Putelat}, \citenamefont {Recho},\ and\ \citenamefont
  {Truskinovsky}}]{putelat2018mechanical}%
  \BibitemOpen
  \bibfield  {author} {\bibinfo {author} {\bibfnamefont {T.}~\bibnamefont
  {Putelat}}, \bibinfo {author} {\bibfnamefont {P.}~\bibnamefont {Recho}},\
  and\ \bibinfo {author} {\bibfnamefont {L.}~\bibnamefont {Truskinovsky}},\
  }\href@noop {} {\bibfield  {journal} {\bibinfo  {journal} {Physical Review
  E}\ }\textbf {\bibinfo {volume} {97}},\ \bibinfo {pages} {012410} (\bibinfo
  {year} {2018})}\BibitemShut {NoStop}%
\bibitem [{\citenamefont {Larripa}\ and\ \citenamefont
  {Mogilner}(2006)}]{larripa2006transport}%
  \BibitemOpen
  \bibfield  {author} {\bibinfo {author} {\bibfnamefont {K.}~\bibnamefont
  {Larripa}}\ and\ \bibinfo {author} {\bibfnamefont {A.}~\bibnamefont
  {Mogilner}},\ }\href@noop {} {\bibfield  {journal} {\bibinfo  {journal}
  {Physica A: Statistical Mechanics and its Applications}\ }\textbf {\bibinfo
  {volume} {372}},\ \bibinfo {pages} {113} (\bibinfo {year}
  {2006})}\BibitemShut {NoStop}%
\bibitem [{\citenamefont {Recho}\ and\ \citenamefont
  {Truskinovsky}(2013)}]{recho2013asymmetry}%
  \BibitemOpen
  \bibfield  {author} {\bibinfo {author} {\bibfnamefont {P.}~\bibnamefont
  {Recho}}\ and\ \bibinfo {author} {\bibfnamefont {L.}~\bibnamefont
  {Truskinovsky}},\ }\href@noop {} {\bibfield  {journal} {\bibinfo  {journal}
  {Physical Review E}\ }\textbf {\bibinfo {volume} {87}},\ \bibinfo {pages}
  {022720} (\bibinfo {year} {2013})}\BibitemShut {NoStop}%
\end{thebibliography}%

\end{document}
