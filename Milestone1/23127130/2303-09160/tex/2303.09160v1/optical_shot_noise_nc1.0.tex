\documentclass[aps,prl,twocolumn,superscriptaddress]{revtex4-1}
\usepackage{bm}
\usepackage{graphicx}
\usepackage{color}
\usepackage{braket}
\usepackage{amsmath,amssymb,amsfonts,amsthm,mathtools}
\usepackage{enumerate}
\usepackage{enumitem}
\usepackage[colorlinks=true,linkcolor=blue,anchorcolor=red,citecolor=blue,urlcolor=blue]{hyperref}
\usepackage{titlesec}
%\usepackage{tikz-cd}
\usepackage{float}
\usepackage{multirow}
\usepackage[title]{appendix}
\usepackage{pdfpages}
\usepackage{changes}
%\usepackage{stackengine}
\usepackage{makecell}
\usepackage{diagbox}
\newcommand{\vect}[1]{\boldsymbol{#1}}

%\renewcommand\thesubsection{\alph{subsection}}
%\def \bM {\mathsf{M}}
%\def \bL {\mathsf{L}}
%\def \bR {\mathsf{R}}
%\def \bg {\mathsf{g}}
%\def \bG {\mathsf{G}}
%\numberwithin{equation}{section}

\makeatletter
\AtBeginDocument{\let\LS@rot\@undefined}
\makeatother

\makeatletter
\renewcommand\NAT@biblabelnum[1]{#1.}
\makeatother

%\setcitestyle{super}

\begin{document}

\title{Hear the quantum noise of photocurrent}
\author{Longjun Xiang}
\affiliation{College of Physics and Optoelectronic Engineering, Shenzhen University, Shenzhen 518060, China}
%\affiliation{College of Physics and Optoelectronic Engineering, Shenzhen University, Guangdong, P. R. China}
\author{Hao Jin}
\affiliation{College of Physics and Optoelectronic Engineering, Shenzhen University, Shenzhen 518060, China}
%\affiliation{College of Physics and Optoelectronic Engineering, Shenzhen University, Guangdong, P. R. China}
\author{Jian Wang}
\email[]{jianwang@hku.hk}
\affiliation{College of Physics and Optoelectronic Engineering, Shenzhen University, Shenzhen 518060, China}
%\affiliation{College of Physics and Optoelectronic Engineering, Shenzhen University, Guangdong, P. R. China}
\affiliation{Department of Physics, University of Hong Kong, Pokfulam Road, Hong Kong, P. R. China}
\affiliation{Department of Physics, The University of Science and Technology of China, Heifei, P. R. China}

\begin{abstract}
Using light to detect the response of quantum materials is one of the most versatile tools
in modern condensed matter physics. For instance, the injection current under light illumination
can be exploited to probe the topological properties
of materials without inversion ($\mathcal{P}$) symmetry.
Rather than detecting the averaged DC photocurrent response,
we propose that the quantum fluctuation of photocurrent, in the form of DC shot noise (SN),
can also be heard even when the DC photocurrent is forbidden by symmetry.
Particularly, we develop the quantum theory for DC SN for gapped systems
and identify four types of resonant DC SNs at the second order of electric field,
dubbed shift SN, injection SN, jerk SN and snap SN in terms of their dependence on illumination time. 
%The same as the shift or injection photocurrent susceptibility tensor,
%we find that all the resonant SN susceptibility tensors
%can be decomposed into symmetric and anti-symmetric components,
%which are coupled with the linearly and circularly polarized light, respectively.
Importantly, the expressions for all SN susceptibility tensors
are also characterized by the gauge-invariant band topological quantities,
such as the local quantum metric, the local Berry curvature, as well as the shift vector,
and amenable to first-principles calculations.
Furthermore, we discuss the symmetry constraints for all SNs
and find that they are hardly forbidden by symmtry like the photocurrent.
To illustrate our theory, we investigate the SNs in time-reversal-invariant two-dimensional
materials GeS and MoS$_2$ with different symmetries.
We confirm that the SNs can be heard even when the photocurrent responses are forbidden by $\mathcal{P}$-symmetry
and also reveal the relevance of band geometric quantities on SNs.
Our work show that the quantum fluctuation of photocurrent can be informative and
offer a complementary tool to characterize the symmetry and topology of quantum materials. 
\end{abstract}

\maketitle

\noindent \textbf{INTRODUCTION}\\
It has been well known that the inversion ($\mathcal{P}$) symmetry broken materials under light illumination
can feature the bulk photovoltaic effect (BPVE) \cite{Chynoweth1956,Chen1969,Class1974,Sturman},
which refers to the nonlinear DC photocurrent generation in a single-phase material.
Superior to the conventional photovoltaic devices
based on heterogeneous junction structures, the BPVE can overcome the thermodynamic Shockley–Queisser limit \cite{Spanier2016}
and generate an above-band-gap photovoltage, showing a range of potential applications in the solar energy conversion,
mainly in nonmagnetic ferroelectric perovskite oxides \cite{ferro0,ferro1,ferro2,ferro3}.
Recently, theoretical investigations
show that the materials with broken time-reversal ($\mathcal{T}$) symmetry \cite{Yan2019, Wanghua2020, Yang2020, Duan2022},
such as the two-dimensional anti-ferromagnetic insulator CrI$_3$, can also exhibit a BPVE and
inspire the search for magnetic photovoltaic materials.
Importantly, the first-principles calculations have confirmed that
the intrinsic (free of scattering) shift current mechanism,
driven by the nontrivial band topological quantity Berry phase \cite{Xiao2010},
plays a dominant role in explaining the BPVE \cite{Sipe2, Sipe1, Nagaosa1, J-Moore1}.
In addition, the injection current mechanism for BPVE also attracted much attention in recent years
and can be used to probe the topology of quantum materials
\cite{Circu0, X-Qian1, Ahn, Hosur2011, science2022}.
Furthermore, by investigating the physical divergence of the third (fourth) order polarization susceptibility,
Fregoso \textit{et al.} have proposed the jerk (snap) photocurrent mechanism for BPVE 
\cite{Fregoso, Ventura2021, Fregoso1, Fregoso2021}.

Although both the shift and injection  photocurrent responses have been extensively investigated
experimentally and theoretically, little attention has been paid on the quantum fluctuation of
those nonlinear photocurrent, which usually behaves as the noise of photocurrent response
and is ubiquitous as a consequence of non-equilibrium process.
Particularly, the SN of electric current,
as the quantum fluctuation due to the quantization of charge in steady state transport,
can provide more information of the system than electric current itself \cite{Buttiker}.
In addition, SN can also probe the quantum statistics of the quasi-particles and measure its effective charge.
SN has been studied extensively in mesoscopic systems exhibiting a plethora of interesting phenomena.
For instance, negative current correlation has been observed due to Pauli exclusion principle \cite{negative}
while the interaction in superconductor-normal hybrid system gives rise to positive current correlation
\cite{positive1,positive2}.

In this work, we investigate the quantum fluctuation of photocurrent, in the form of SN.
We unveil that the resonant DC \cite{note2} SNs $S^{(2)}$ at the second order of electric field
are composed by four terms according to their time dependence at short time limit:
\begin{align}
S^{(2)} = \left( \sigma + \kappa t^2 \right) \delta_C + \left( \eta t^1  + \zeta t^3 \right) \delta_L, \label{shotnoise}
\end{align}
where $\delta_L \equiv |\vect{E}|^2$ stands for the linearly polarized light (LPL)
and $\delta_C \equiv |\vect{E}\times\vect{E}^*|$ the circularly polarized light (CPL) \cite{Sturman}, respectively,
$t$ is the illumination time, and
$\sigma$, $\eta$, $\kappa$, and $\zeta$ are
the susceptibility tensors of the resonant shift SN, injection SN, jerk SN, and snap SN.
We show that the shift and injection SNs are mainly arising from the correlation between
second-order photocurrent operator and interband velocity
while the jerk and snap SNs are caused by the autocorrelation of
first-order photocurrent operator.
Unlike the second-order photocurrent, we find that the second-order resonant DC SNs are immune to
$\mathcal{P}$-symmetry. However, in time-reversal ($\mathcal{T}$)
or $\mathcal{P}\mathcal{T}$ invariant systems we find that the resonant shift/jerk SNs and
injection/snap SNs vanish under the illumination of linearly polarized light (LPL)
and circularly polarized light (CPL), respectively.
On the other hand, the jerk/snap SNs, as the quantum fluctuation of linear photocurrent operator,
show the similar behavior with the jerk/snap currents at the third/fourth order of electric field.
The same as photocurrent responses, we find that various SNs are characterized
by the band topological quantities, such as the local quantum metric, the local Berry curvature and the shift vector.
Our work shows that the quantum fluctuation of photocurrent can be informative and
offers more observable physical properties under light illumination, thereby providing a new tool 
for characterizing quantum materials and playing a complementary role of shift and injection current.

\begin{table*}[htb!]
\begin{center}
\caption{\label{tab0} {
The symmetry transformation for group velocity difference $\Delta^a_{nm}$,
interband velocity matrix element $v^a_{nm}$, interband Berry connection $r^a_{nm}$,
local quantum metric $g^{bc}_{nm}$, local Berry curvature $\Omega^{bc}_{nm}$, shift vector $R^{a,b}_{nm}$,
and other composed quantities appeared in SN susceptibilities under $\mathcal{P}$, $\mathcal{T}$,
and $\mathcal{PT}$ operations. Here $\pm$ means the global sign changes under symmetry operation.
For simplicity, we have suppressed their superscripts and the transformation of their independent variable $\vect{k}$.
}}
\begin{tabular}{ c|c|c|c|c|c|c|c|c }
\hline
\hline
\diagbox{Symmetries}{Quantities}
                                 & \ \ $\Delta_{nm}(\vect{k})$   \ \  & \ \ $v_{nm}(\vect{k})$  \ \       & \ \ $r_{nm}(\vect{k})$ \ \
                                 & \ \ $g_{nm}(\vect{k})$        \ \  & \ \ $\Omega_{nm}(\vect{k})$ \ \   & \ \ $R_{nm}(\vect{k})$ \ \
                                 & \ \ $\Lambda_{nm}(\vect{k})$  \ \  & \ \ $\Pi_{nm}(\vect{k})$ \ \ \\ 
\hline
\ \ $\mathcal{P}$  \ \           & \ \ $-$ \ \   & \ \ $-$  \ \   & \ \ $-$ \ \
                                 & \ \ $+$ \ \   & \ \ $+$  \ \   & \ \ $-$ \ \
                                 & \ \ $-$ \ \   & \ \ $+$  \ \   \\ 
\hline
\ \ $\mathcal{T}$ \ \            & \ \ $-$ \ \   & \ \ $-$  \ \   & \ \ $+$ \ \
                                 & \ \ $+$ \ \   & \ \ $-$  \ \   & \ \ $+$ \ \
                                 & \ \ $-$ \ \   & \ \ $+$  \ \   \\ 
\hline
\ \ $\mathcal{P}\mathcal{T}$ \ \
                                 & \ \ $+$ \ \   & \ \ $+$ \ \    & \ \ $-$ \ \
                                 & \ \ $+$ \ \   & \ \ $-$ \ \    & \ \ $-$ \ \
                                 & \ \ $+$ \ \   & \ \ $+$ \ \    \\
\hline
\hline
\end{tabular}
\end{center}
\end{table*}


\bigskip
\noindent \textbf{RESULTS}\\
\textbf{The quantum theory for SN}
Within independent particle approximation \cite{Sipe1,Sipe2},
the current operator at the $i$th order in second quantization can be expressed as
\begin{align}
\hat{J}^{a,(i)}
=
\sum_{nm} \int_k \hat{\rho}^{(i)}_{nm} v_{mn}
\equiv
\sum_{nm} \int_k J_{nm}^{a,(i)} a^\dagger_m a_n,
\label{currentoperator}
\end{align}
where $\int_k \equiv \int d\vect{k}/(2\pi)^d$ with $d$ the spatial dimensionality,
$\hat{\rho}^{(i)}_{nm}$ is the density matrix operator in second quantization
at the $i$th order of electric field,
which will be solved iteratively 
within the nonlinear response theory \cite{sup},
$v_{nm}$ is the velocity matrix element,
and $a_m^\dagger$ ($a_n$) is the creation (anihilation) operator for Bloch states.
Particularly, in Eq.(\ref{currentoperator}), it is easily shown that
the intraband and interband component of $v_{mn}$,
weighted by the second-order density matrix operator $\hat{\rho}^{(2)}_{nm}$,
will give the nonlinear injection current operator and shift current operator \cite{photodrag},
respectively. Naturally, the quantum and statistical average of injection and shift current operators
gives the well known injection and shift currents, respectively.
Once the current operator is defined, the noise spectrum, defined as
$\delta(0)S^{ab,(i+j)} \equiv \langle \hat{J}^{a,(i)}\hat{J}^{b,(j)}\rangle
-
\langle \hat{J}^{a,(i)} \rangle \langle \hat{J}^{b,(j)}\rangle$,
is given by \cite{Buttiker, M-Wei}
\begin{align}
S^{ab,(i+j)} = \sum_{nm} \int_k \bar{f}_{nm} J^{a,(i)}_{nm} J^{b,(j)}_{mn}
\label{generalnoisespectrum}
\end{align}
where ${\bar f}_{nm} \equiv f_n (1-f_m) + f_m (1-f_n)$ with $f_n$ the equilibrium Fermi distribution function.
Note that Eq.(\ref{generalnoisespectrum}) includes both the thermal noise for $n=m$ and
the SN for $n \neq m$.
Interestingly, we find that the thermal noise features the Fermi-surface property
due to $f_n(1-f_n)=-k_BT\partial_\epsilon f_n$ and vanishes for gapped systems at low temperatures;
on the contrary, the SN displays a Fermi-sea property for gapped systems and
becomes dominant at low temperatures. Because we are interested in
the quantum fluctuation of photocurrent in semiconductors or insulators
and therefore we will focus on the SN for gapped systems throughout this work.
In addition, we assume that there is no momentum relaxation as was done in previous studies
\cite{Sipe1, Yan2019, Juan, Ahn, Rappe1, Yanase2, Moore2, Fregoso1, Park} and hence
we can classify the SN in terms of their time dependence, $t^n$,
where $t$ is the illumination time in short time limit
and $n$ indicates the power of the frequency sum divergence.
We will discuss what happens when relaxations are present in the discussion section.
Note that this classification is consistent with the nonlinear photocurrent responses.
For example, $n=1$ corresponds to the injection current and injection SN.

Similar to the DC photocurrent responses,
the lowest order DC SNs appear at the second order of the electric field.
Particularly, from Eq.(\ref{generalnoisespectrum}), we find that the SNs at second order of electric field
are contributed by $J^{a,(0)}_{nm} J^{b,(2)}_{mn}$ and $J^{a,(1)}_{nm} J^{b,(1)}_{mn}$,
where $J^{a,(0)}_{nm}=v_{nm}$ and $J^{a,(l)}$ with $l=1,2$ are given in the Supplemental Material \cite{sup}.
Note that the quantum and statistical average of $\hat{J}^{a,(2)}$ consists of the injection current and the shift current
and hence we naturally obtain the injection SN and shift SN from $J^{a,(0)}_{nm}J^{a,(2)}_{mn}$.
In addition, although the DC photocurrent response due to $J^{a,(1)}_{nm}$ is forbidden by symmetry \cite{Nagaosa2018},
we find that its autocorrelation, namely $J^{a,(1)}_{nm}J^{a,(1)}_{nm}$, displays rich physics,
in which the resonant injection SN, shift SN, jerk SN, and snap SN can be captured.

To be specific, in the following we consider the monochromatic optical field
$E^b(t)=E^b_\beta e^{-i\omega_\beta t}=E^b(\omega_\beta)e^{-i\omega_\beta t}$
with $\omega_\beta=\pm \omega$, where the Einstein summation convention has been assumed for the repeated Greek alphabet.
First of all, by defining $S^{aa,(2)}_{\text{sht}} \equiv \sigma^{abc} (-\omega_\Sigma;\omega_\beta,\omega_\gamma)E_\beta^b E_\gamma^c$,
where $\omega_\Sigma \equiv \omega_\beta+\omega_\gamma=0$,
we find that the shift SN susceptibility tensor in $\mathcal{T}$-invariant systems
is given as \cite{sup,Nagaosanoise}
\begin{equation}
\begin{aligned}
\sigma_{\text{C}}^{abc} 
&= \dfrac{\pi}{4} \int_k \sum_{nm} \bar{f}_{nm} \Delta^{a}_{nm}
\Lambda^{abc}_{nm}
\delta(\omega-\omega_{mn}),
\\
\Lambda^{abc}_{nm}
&\equiv
g^{bc}_{nm} \partial_a \ln\left|r^c_{nm}/r^b_{nm}\right|
+
\Omega^{cb}_{nm} \left( R^{a,c}_{nm} + R^{a,b}_{nm} \right),
\end{aligned}
\label{shotsigmaI}
\end{equation}
where the subscript $\text{C}$ dictates that the $\sigma^{abc}_{\text{C}}$ can only be coupled with CPL.
Note that $\sigma_{\text{C}}^{abc}$ is a rank-4 tensor,
where the first index $a$ is responsible for the direction of autocorrelated current.
The same convention will be applied to other SNs discussed below.
In Eq.(\ref{shotsigmaI}), $\Delta^a_{nm}=v^a_n - v^a_m$ with $v^a_n$ the group velocity of band $n$,
$g^{cb}_{nm} \equiv r^c_{nm}r^b_{mn} + r^b_{nm}r^c_{mn}$
with $r^c_{nm}$ the interband Berry connection is the local quantum metric \cite{Duan2022},
$\Omega^{cb}_{nm} \equiv i(r^c_{nm}r^b_{mn}-r^b_{nm}r^c_{mn})$ is the local Berry curvature \cite{Wanghua2020},
and $R^{a,c}_{nm}=-\partial_a\phi^c_{nm}+\mathcal{A}_n^a-\mathcal{A}^a_m$ is the shift vector,
where $\phi^c_{nm}=r^c_{nm}/|r^c_{nm}|$ and $\mathcal{A}^a_n$ is the intraband Berry connection.
Due to $g^{bc}_{nm}=g^{cb}_{nm}$, $\partial_a\ln|r^c_{nm}/r^b_{nm}|=-\partial_a\ln|r^b_{nm}/r^c_{nm}|$,
and $\Omega^{bc}_{nm}=-\Omega^{cb}_{nm}$,
it is easily found that $\sigma^{abc}_{\text{C}}$
is anti-symmetric about $b$ and $c$.
Note that Eq.(\ref{shotsigmaI}) only consider the optical transitions involved two-band process,
the expression including three-band process is discussed in Supplemental Material \cite{sup}.
Usually, the two-band process is two orders of magnitude larger than that of three-band \cite{Yan2019}.
Interestingly, we find that the shift SN susceptibility tensor displays a dual relation with 
the shift current susceptibility tensor in $\mathcal{T}$-invariant systems, 
because the latter one is symmetric about the last two indices and can only be coupled with LPL.
In addition, we find that the quantity $\Lambda^{abc}_{nm}$ also appears in the expression
of magnetic shift current susceptibility tensor
but in $\mathcal{T}$-broken systems \cite{Wanghua2020}.

Similarly, we find that the injection SN $\partial_t S_{\text{inj}}^{aa,(2)}$,
obtained from $J^{a,(0)}_{nm}J^{a,(2)}_{mn}$, can be expressed as 
$\partial_t S^{aa,(2)}_{\text{inj}} = \eta^{abc}(0;\omega_\beta,\omega_\gamma)E^b_\beta E_c^\gamma$
and in $\mathcal{T}$-invariant systems we obtain \cite{sup}
\begin{align}
\eta^{abc}_{\text{L}} &=
\dfrac{\pi}{2} \sum_{nm} \int_k
\bar{f}_{nm} g^{bc}_{nm} (\Delta^a_{nm})^2 \delta (\omega-\omega_{mn}),
\label{shotetaR}
\end{align}
where the subscript $\text{L}$ dictates that the $\eta^{abc}_{\text{L}}$ can only be coupled with LPL.
Different from the shift SN, we find that $\eta^{abc}_{\text{L}}$ is symmetric about $b$ and $c$,
forming a dual relation with the injection current susceptibility tensor in time-reversal invariant systems.
In addition, we note that the injection SN is closely related to the local quantum metric
whereas the injection current is closely related to the local Berry curvature.
Note that the expressions for shift and injection SN resemble their magnetic photocurrent counterparts\cite{Wanghua2020}.
Specifically, by replacing $f_{nm}$ with $\bar{f}_{nm} \Delta_{nm}^a$ in the shift and injection current expression, 
one obtains shift SN and injection SN, respectively.
Here the factor of $\bar{f}_{nm}$ is due to the quantum statistics
and $\Delta_{nm}^a$ is due to $J^{a,(0)}$.

Interestingly, we find the autocorrelation $J^{a,(1)}_{nm}J^{a,(1)}_{mn}$ also contributes to
the shift and injection SNs (discussed in Supplemental Material \cite{sup}).
However, in terms of the time dependence or the power of the frequency sum divergence,
we find that $J^{a,(1)}_{nm}J^{a,(1)}_{mn}$ can further give the jerk and snap resonant DC SNs.
For the former, by writing
$\partial_t^2 S^{aa,(2)}_{\text{jerk}} = \kappa^{abc}(0;\omega_\beta,\omega_\gamma)E^b_\beta E^c_\gamma$,
for $\mathcal{T}$-invariant systems we obtain\cite{sup}:
\begin{equation}
\begin{aligned}
\kappa^{abc}_{\text{C}} &= 2\pi\int_k \bar{f}_{nm} |v^a_{nm}|^2 
\Pi^{a,cb}_{nm}
\delta(\omega-\omega_{mn}),
\\
\Pi^{a,cb}_{nm} &\equiv \partial_c |v_{nm}^a| \Delta^b_{nm} - \partial_b |v_{nm}^a| \Delta^c_{nm},
\label{shotkappaI}
\end{aligned}
\end{equation}
where $v^a_{nm}=i\omega_{nm}r^a_{nm}$ is the interband velocity matrix element
and $|v^a_{nm}|^2$ through $|r^a_{nm}|^2$ plays the role of dipole transition rate \cite{Yan2019}.
The same as the shift SN, we find that $\kappa^{abc}_{\text{C}}$ is
an anti-symmetric tensor about $b$ and $c$ and can only be coupled with CPL.
For the latter, by writing 
$\partial_t^3 S^{a,(2)}_{\text{snap}} = \zeta^{abc}(0;\omega_\beta,\omega_\gamma)E^b_\beta E^c_\gamma$,
for $\mathcal{T}$-invariant systems \cite{sup}
\begin{align}
\zeta_{\text{L}}^{abc} &= 4\pi \sum_{m n} \int_k \bar{f}_{nm} |v_{nm}^a|^2 \Delta^b_{nm} \Delta^c_{nm} \delta(\omega-\omega_{mn}),
\label{shotzetaR}
\end{align}
Note that $\zeta_{\text{L}}^{abc}$ shows the same behavior with injection SN,
which is symmetric about $b$ and $c$ and can only be coupled with LPL.
As expected, we note that all the second-order resonant SN
susceptibility tensors are gauge-invariant under $U(1)$ gauge transformation.
In summary, Eqs. (\ref{shotsigmaI}-\ref{shotzetaR}) constitute the quantum theory of
second-order resonant SN in time-reversal invariant systems
and can be used to investigate the SN in realistic materials by combining with first-principles calculations.
Before doing that, we need to discuss the symmetry constraints for these rank-4 tensors
in $\mathcal{T}$-invariant systems.

\smallskip
\noindent\textbf{The symmetry for SN susceptibility tensor}
It is well known that the symmetry plays a pivotal role in the discussion of
the shift (injection) photocurrent \cite{Yan2020}, especially when interplaying with the polarization of light.
For example, under $\mathcal{P}$-symmetry, we have $\mathcal{P}J^a=-J^a$ and $\mathcal{P}E^b=-E^b$
and hence the susceptibility for shift (injection) current is a $\mathcal{P}$-odd tensor,
which dictates that both the shift and injection current response
vanishes in $\mathcal{P}$ invariant systems regardless of the polarization of light \cite{photodrag}.
In addition, under the illumination of LPL, the injection current is forbidden by
$\mathcal{T}$-symmetry because the susceptibility tensor for injection current is a $\mathcal{T}$-odd tensor \cite{Wanghua2020}.

In the previous section, the quantum theory for second-order resonant SN 
in $\mathcal{T}$-invariant systems has been established,
but the influence of symmetry on these SN susceptibilities under the illumination of LPL or CPL 
has not been discussed yet.
In this section, we explore the symmetry constraint on these susceptibilities.
To that purpose, we first tabulate the symmetry transformation for group velocity difference $\Delta^a_{nm}$,
interband velocity matrix element $v^a_{nm}$, interband Berry connection $r^a_{nm}$,
local quantum metric $g^{bc}_{nm}$, local Berry curvature $\Omega^{bc}_{nm}$, and shift vector $R^{a,b}_{nm}$
and other related quantities appeared in Eqs.(\ref{shotsigmaI}-\ref{shotzetaR}) under $\mathcal{P}$,
$\mathcal{T}$, and $\mathcal{PT}$ operations, respectively, as shown in TABLE (\ref{tab0}).
In terms of the transformation relations under $\mathcal{P}$ symmetry,
it is easily verified that all the second-order resonant susceptibilities in Eqs.(\ref{shotsigmaI}-\ref{shotzetaR})
belong to $\mathcal{P}$-even tensor %, as shown in TABLE (\ref{tab1}),
and hence all the second-order SNs are immune to $\mathcal{P}$-symmetry,
also regardless of the polarization of light, in sharp contrast with the
$\mathcal{P}$-odd shift or injection current susceptibility tensor.

On the other hand, the constraint from $\mathcal{P}$-symmetry for photocurrent
and its SN can be intuitively understand as follows.
As a general feature of photocurrent, the $\mathcal{P}$-symmetry has to be broken
either by crystal structure or by other external means.
This is because, if the system has $\mathcal{P}$-symmetry, the left going and right going photocurrent cancel to each other.
For the noise spectrum, however, this cancellation restriction is lifted since correlation of current
is nonzero even when current is zero. A notable example is the equilibrium noise or Nyquist-Johnson noise.
This property is also formulated as
fluctuation-dissipation theorem which can be used to calculate the non-equilibrium current, e.g., Kubo formula.
As such, the shot noise of photocurrent can exist in the systems with $\mathcal{P}$-symmetry.

The same as shift (injection) current, the constraints on SN susceptibility tensors from $\mathcal{T}$-symmetry
is tricky because we need to take the polarization of light into account at the same time.
For example, under the illumination of LPL (CPL), we find that the shift current susceptibility
tensor is a $\mathcal{T}$-even ($\mathcal{T}$-odd) tensor while the injection current susceptibility
tensor is a $\mathcal{T}$-odd ($\mathcal{T}$-even) tensor, therefore, in $\mathcal{T}$-invariant systems,
one can only detect the shift current or injection current in $\mathcal{P}$-broken materials
by illuminating LPL and CPL, respectively. In previous section, we have established the quantum theory
for DC SNs at the second-order for $\mathcal{T}$-invariant systems, in terms of TABLE (\ref{tab0}),
one can easily check all DC SN susceptibility tensors belongs to $\mathcal{T}$-even tensors, as it must be.
The $\mathcal{T}$-odd components for all DC SNs can be found in Supplemental Material\cite{sup}.
In addition, we find that these DC SNs in $\mathcal{T}$-invariant systems can also be coupled with LPL
and CPL and we comment that the shift and injection SN feature a different dependence on the polarization
of light with their current counterpart, forming a dual relation.
Finally, we note that all DC SNs show the same dependence on $\mathcal{T}$ and $\mathcal{P}\mathcal{T}$ symmetries
due to their $\mathcal{P}$-even property, different from the shift and injection photocurrent.

Except for the $\mathcal{P}$, $\mathcal{T}$, and $\mathcal{P}\mathcal{T}$ symmetries,
to consider the constraint from other point group symmetry, such as rotation and mirror symmetries,
one should resort to the Neumman's principle \cite{Xiong2022}, which determines the non-vanishing tensor elements
under point group operation. Particularly, for the rank-4 SN susceptibility tensor $\lambda^{abcd}$
in $\mathcal{T}$-invariant systems,
where $\lambda$ represents $\sigma, \eta, \kappa$, and $\zeta$,
the constraint imposed by point group symmetry operation $R$ can be expressed as:
\begin{align}
\lambda^{abcd} = R_{aa'} R_{bb'} R_{cc'} R_{dd'} \lambda^{a'b'c'd'},
\label{NeumannP}
\end{align}
where $R_{\alpha\alpha'}$ is the matrix element of the spatial point group operation $R$.
For example, if the system respects the mirror symmetry $\mathcal{M}_x$: $\mathcal{M}_x x \rightarrow -x$,
one can immediately check that that $\lambda^{yyxz}$ is forbbiden by this symmetry in terms of Eq.(\ref{NeumannP}).
Fortunately, one can also use the Bilbao Crystallographic Server \cite{Bilbao}
to achieve this purpose once for all by defining a suitable Jahn's notation.
For example, in $\mathcal{T}$-invariant systems 
the Jahn's notation for the shift (injection) SN $\sigma^{abc}_{\text{C}}$ ($\eta^{abc}_{\text{L}}$),
which is an anti-symmetric (symmetric) rank-4 $\mathcal{T}$-even tensor,
can be expressed as $VV\{V^2\}$ ($VV[V^2]$),
where $[\cdots](\{\cdots\})$, indicates the symmetric (asymmetric) feature under index interchange.
For convenience, we list all the Jahn's notation for all second-order resonant DC SN susceptibility tensors
in TABLE (\ref{tab1}). Note that jerk (snap) SN shows the same behavior with the shift (injection) SN.
In addition, we also list the Jahn's notations for shift and injection current
susceptibility tensors in $\mathcal{T}$-invariant systems, as a comparison.

\begin{center}
\begin{table}
\caption{\label{tab1} 
{The Jahn's symbols for all the SN susceptibility tensors
and the second-order DC photocurrent susceptiblity tensors in $\mathcal{T}$-invariant systems,
which are represented by $\sigma_\text{2L}$ and $\eta_\text{2C}$,
respectively. For simplicity, we have suppressed their subscripts.
}
}
\begin{tabular}{ c|c|c|c|c }
\hline
\hline
                        & \ \ $\sigma_\text{C}/\kappa_\text{C}$ \ \ & \ \ $\eta_\text{L}/\zeta_\text{L}$ \ \
                        & \ \ $\sigma_\text{2L}$ \ \                & \ \ $\eta_\text{2C}$ \\ 
\hline
Jahn's Sym. \ \         & \ \ $VV\{V^2\}$ \ \                       & \ \ $VV[V^2]$ \ \
                        & \ \ $V[V^2]$ \ \                          & \ \ $V\{V^2\}$ \\ 
\hline
\hline
\end{tabular}
\end{table}
\end{center}

\smallskip
\noindent\textbf{The GeS monolayer}
As the first example, we explore the SNs in 
single-layer monochalcogenide GeS,
which has been extensively studied \cite{J-Moore1, FregosoGeS2019, WanghuaSci2019}
and dispalyed a large BPVE and spontaneous polarization
in its ferroelectric phase with point group (PG) $mm2$, as shown in FIG.(\ref{fig:fig1}a).
The $mm2$ PG doesn't include the $\mathcal{P}$-symmetry and hence both the shift and
injection photocurrents are allowed. However, except for the ferroelectric phase,
GeS may stay in a paraelectric phase with PG $mmm$, as shown in FIG.(\ref{fig:fig1}a),
which respects $\mathcal{P}$-symmetry and hence can not generate a DC photocurrent under light illumination.
In FIG.(\ref{fig:fig1}b), we display the band structures for monolayer GeS with
$mm2$ and $mmm$ symmetries and we find that the band gap goes through a large modification
from the ferroelectric phase with PG $mm2$ to paraelectric phase with PG $mmm$.
However, due to the paraelectric phase can not generate a DC photocurrent response
and the band structure evolution can not be detected just by measuring the current response.

Interestingly, for shift SN, we find that the susceptibility tensor $\sigma^{abc}_{\text{C}}$
is forbidden by mirror symmetry $\mathcal{M}_x$ or $\mathcal{M}_y$ in both $mm2$ and $mmm$
if $b \neq c$ or by the antisymmetric property if $b=c$ and therefore one can not capture the SN
due to autocorrelation. Note that $\sigma^{abc}_{\text{C}}$ is a rank-4 tensor and the first index
is responsible for the autocorrelation of photocurrent operator,
However, in terms of the Jahn's notation listed in TABLE (\ref{tab1}),
one can check the full table for the SN susceptibility tensor is hardly forbidden by symmetry.
However, the injection SN and snap SN, such as $\eta^{yyy}_{\text{L}}$ and $\kappa^{yyy}_{\text{L}}$,
are allowed by both $mmm$ and $mm2$ phases of GeS due to its symmetric property about last two indices,
as shown in FIG.(\ref{fig:fig1}c-d),
in which the injection SNs features a quite different behavior
while the snap SNs shows a decreaing behavior for PGs $mm2$ and $mmm$.
Pparticularly, for injection SN there is a sharp peak near the band edge for
$mmm$ PG with $\mathcal{P}$-symmetry while for $mm2$ without $\mathcal{P}$-symmetry
the peak is broadened and shifted towards higher energy. 
The different response behavior for injection SN of GeS may offer a tool to probe
the band structure evolution from ferroelectric phase to paraelectric phase.
Finally, we comment that there are other symmetry-allowed elements for both 
the injection SN and snap SN of GeS in terms of Eq.(\ref{NeumannP}) or
by checking the Bilbao database with Jahn's notation listed in TABLE (\ref{tab1}),
such as $\eta^{xxx}_{\text{L}}$ and $\eta_{\text{L}}^{xzz}$, see Supplemental Material\cite{sup}.

\begin{figure}[htb!]
\centering
\includegraphics[width=0.45\textwidth]{fig1}\\
\caption{(a) The side views of GeS monolayer with point group $mm2$ and $mmm$,
where $mm2$ ($mmm$) breaks (respects) $\mathcal{P}$-symmetry.
(b) The band structures for GeS with different symmetries.
(c)-(d) The injection and snap SN susceptibility tensors for GeS with different symmetries, respectively.
}
\label{fig:fig1}
\end{figure}


\smallskip
\noindent\textbf{The MoS$_2$ bilayer}
As the second example, we explore the SNs in bilayer MoS$_2$
within $2H$ and $3R$ phases, as shown in FIG.(\ref{fig:fig2}a). 
Particularly, the 2H-MoS$_2$ bilayer respects $\mathcal{P}$ symmetry and hence
both the shift and injection currents are forbidden in this system.
However, bilayer MoS$_2$ could possess a $\mathcal{P}$-broken phase by
modifying the stack configuration of constituent monolayers.
Very recently, it is shown that bilayer MoS$_2$ with 3R stacking pattern 
could exhibit an out-of-plane electric polarization, which is also known as the sliding ferroelectricity
\cite{YangDY2022,SuiFG2023,LiJu2021,Stern2021}.
In FIG.(\ref{fig:fig2}b), we show the band structures for
bilayer MoS$_2$ within $3R$ and $2H$ phases. Different from monolayer GeS discussed before,
we find that the band structures are almost unchanged with these two different stacking patterns.

Furthermore, in FIG.(\ref{fig:fig2}c-d)
we plot the shift and injection SN for bilayer MoS$_2$ with $3R$ and $2H$ phases, respectively.
Different from monolayer GeS, we find both shift and injection SNs exists for both $\mathcal{P}$-broken $3R$
and $\mathcal{P}$-invariant $2H$ bilayer MoS$_2$,
even the mirror symmetry exists in $3R$ and $2H$ but in hexagonal lattice.
%in which the constraint from $\mathcal{P}$ symmetry is removed.
In addition, we find that the injection SN in bilayer MoS$_2$
almost features the same behavior in $2H$ and $3R$ phases while
the shift SN becomes dominant in $2H$ phases.
Of course, there are other symmetry-allowed components
for both shift and injection SNs in bilayer MoS$_2$ with or without $\mathcal{P}$-symmetry,
as discussed in Supplemental Materials \cite{sup}.
In addition, we have also discussed the jerk and snap SNs for both GeS and MoS$_2$
in Supplemental Materials \cite{sup}.

\bigskip
\noindent\textbf{DISCUSSION}\\
Having discussed symmetry constraint, we see that for $\cal T$ or $\cal PT$-even systems,
the resonant shift, injection, and jerk SNs are nonzero only for CPL, LPL, and CPL respectively and
Eqs.(\ref{shotsigmaI}), (\ref{shotetaR}), and (\ref{shotkappaI}) can be used for their calculation.
The short time scaling of SNs and the polarization of light are well correlated for $\cal T$ or
$\cal PT$-even systems. Although this is illustrated in two-band process,
it is applied for three-band and fourth band processes which also exist for resonant SN
(for details see Supplemental Materials \cite{sup}).
When the systems do not have $\cal T$ or $\cal PT$ symmetry,
the resonant shift, injection, and jerk SNs have nonzero signals regardless of polarization of light.
For resonant snap SN, however, only LPL gives nonzero SN response for systems with any symmetry.

\begin{figure}[ht!]
\centering
\includegraphics[width=0.45\textwidth]{fig2}\\
\caption{(a) The side views of MoS$_2$ bilayer with PGs $3m$ ($3R$) and $-3m$ ($2H$),
where $3m$ ($-3m$) breaks (respects) $\mathcal{P}$ symmetry.
(b) The band structures for $3R$ and $2H$ MoS$_2$ bilayer.
(c)-(d) The shift and injection SN susceptibility tensors for MoS$_2$ bilayer with PGs $3m$ and $-3m$,
respectively.
}
\label{fig:fig2}
\end{figure}

%So far, we have presented results for resonant contribution, there are non-resonant contributions
%as well which give complementary information in terms of polarization of light. For instance,
%for $\cal T$ or $\cal PT$-even systems, the resonant shift SN survives only for CPL while
%for non-resonant shift SN it is nonzero only for LPL. For shift SN,
%there are non-resonant contributions from two-band and three-band processes while 
%for injection SN, the non-resonant contributions come from three-band process.
%There is no non-resonant contribution for jerk and snap SNs.

In the above discussion, we have assumed that the relaxation time is infinitely long.
In the presence of relaxation, the shift SN is an intrinsic property and independent
of relaxation time while the relaxation time of injection, jerk, snap SN scale the same way
as that of illumination time, which is similar to the relaxation time dependence of shift,
injection\cite{Ahn, Yang2020, Duan2022}, jerk, and snap photocurrent \cite{Fregoso}.
Typically, $\tau$ is of the order of $100 fs$ \cite{G-Li}.
For the time scale much shorter than $\tau$ belongs to the short time limit where
the time scaling of various SN can be observed.

There are many experimental works on photocurrent to characterize quantum materials
including measurement of absorption spectrum \cite{exp1}, spin degrees of freedom\cite{exp2,exp3},
quantum metric \cite{exp4,exp5}, and quantum kinetics \cite{exp6}. Although the shot noise of photocurrent
has not been measured, the noise spectrum of electric current in mesoscopic systems has been extensively
studied experimentally over twenty years ago \cite{exp7,exp8,exp9,exp10}.
Therefore, experimental verification of photocurrent noise should be within the reach of current technology. 

In conclusion, we formulate the quantum theory to calculate the quantum fluctuation of photocurrent
in gapped systems and we reveal four types resonant DC SNs at the second-order of electric field,
termed as shift SN, injection SN, jerk SN, and snap SN in terms of their dependence
on illumination time. The explicit expressions for all resonant DC SN susceptibility tensors,
which are amenable to first-principles calculations,
are derived when the polarization of light is assigned.
Interestingly, we find that these resonant DC SNs also encode the information of band topological quantities,
such as the local quantum metric, the local Berry curvature, as well as the shift vector.
Furthermore, by symmetry analysis, we find that all SNs can exist in $\mathcal{P}$-invariant systems,
in sharp contrast with their photocurrent counterparts. Guided by symmetry, we apply our theory
in monolayer GeS and bilayer MoS$_2$ in $\mathcal{P}$-invariant and $\mathcal{P}$-broken phases.
Our work show that the quantum fluctuation of photocurrent can be informative and
offer a complementary tool for the characterization of quantum materials. 

\bigskip
\noindent \textbf{\large METHODS}\\
First-principles calculations are performed by using the Vienna $ab$ $initio$ simulation package (VASP)
\cite{kresse1996efficiency,kresse1996efficient}. The generalized gradient approximation (GGA) in the form
of Perdew-Burke-Ernzerhof (PBE) is used to describe the exchange correlation \cite{QE-2009,perdew1996generalized}.
We choose 500 eV for the cutoff energy and a $k$-grid of $18\times18\times1$ for the first Brillouin zone integration.
To avoid the spurious interaction, we employ at least 20 {\AA} vacuum space along the perpendicular direction.
All atoms in the supercell are fully relaxed based on the conjugate gradient algorithm,
and the convergence criteria is 0.01 eV/{\AA} for the force and $10^{-8}$ eV for the energy, respectively.
A damped van der Waals (vdW) correction based on the Grimme's scheme is also incorporated to better
describe the nonbonding interaction \cite{vdw1,vdw2}. The maximally localized Wannier functions
are then employed to construct the tight-binding model via wannier90 code,
in which Mo-$d$, Ge-4$p$, and S-$3d$ orbitals are taken into account \cite{w90-2020,wanniersc}.
The tight-binding Hamiltonian is utilized to calculate the shot noise according to equations listed in this work.
To deal with the rapid variation of the Berry curvature, the Brillouin zone integration
is carried out using a dense $k$-mesh with 600$\times$600$\times$1, which gives well convergent results.
The 3D-like coefficients of shot noise are obtained by assuming an active single-layer with a thickness of $L_a$:
\begin{equation}
SN_{3D}=\frac{L_{slab}}{L_{a}}SN_{2D}.
\end{equation}
where $SN_{2D}$ is the calculated shot noise, and $L_{slab}$ is the thickness of the supercell \cite{wanniersc}.

\bigskip
\noindent \textbf{\large DATA AVAILABILITY}\\
The data generated and analyzed during this study are available from the corresponding author upon request.

\bigskip
\noindent \textbf{\large CODE AVAILABILITY}\\
All code used to generate the plotted data is available from the corresponding author upon request.

\bigskip
%\bibliographystyle{apsrev4-1}
\def\bibsection{\ }
\noindent \textbf{REFERENCES}
\begin{thebibliography}{00}
\bibitem{Chynoweth1956}
A. G. Chynoweth, Phys. Rev. 102, 705 (1956).
\bibitem{Chen1969}
F. S. Chen, J. Appl. Phys. 40, 3389 (1969).
\bibitem{Class1974}
A. M. Glass, D. von der Linde, and T. J. Negran,
Appl. Phys. Lett. 25, 233 (1974).
\bibitem{Sturman} B. I. Sturman and V. M. Fridkin,
The Photovoltaic and Photorefractive Effects in Non-Centrosymmetric
Materials (Gordon and Breach Science Publishers, Philadelphia, 1992).
\bibitem{Spanier2016}
J. E. Spanier \textit{et al.},
%V. M. Fridkin, A. M. Rappe, A. R. Akbashev,
%A. Polemi, Y. Qi, Z. Gu, S. M. Young, C. J. Hawley, D. Imbrenda,
%G. Xiao, A. L. Bennett-Jackson, and C. L. Johnson,
Nature Photonics 10, 611–616 (2016).
%\bibitem{Yang2010}
%S. Y. Yang \textit{et al.},
%Above-bandgap voltages from ferroelectric photovoltaic devices.
%Nat. Nanotechnol. 5, 143–147 (2010).
\bibitem{ferro0}
T. Choi, S. Lee, Y. J. Choi, V. Kiryukhin, S.-W. Cheong,
Science 324, 63-66 (2009).
\bibitem{ferro1}
S. Y. Yang \textit{et al.},
%J. Seidel, S. J. Byrnes, P. Shafer, C.-H. Yang, M. D. Rossell, P. Yu,
%Y.-H. Chu, J. F. Scott, J. W. Ager III, L. W. Martin and R. Ramesh,
Nat. Nanotechnol. 5, 143–147 (2010).
\bibitem{ferro2}
I. Grinberg \textit{et al.},
%D. V. West, M. Torres, G. Y. Gou, D. M. Stein, L. Y. Wu, G. Chen,
%E. M. Gallo, A. R. Akbashev, P. K. Davies, J. E. Spanier and A. M. Rappe,
Nature 503, 509–512 (2013).
\bibitem{ferro3}
D. Daranciang, \textit{et al},
Phys. Rev. Lett. 108, 087601 (2012).
\bibitem{Yan2019}
Y. Zhang, T. Holder, H. Ishizuka, F. de Juan, N. Nagaosa, C. Felser, and B. H. Yan,
Nat. Communi. 10, 3783 (2019).
\bibitem{Wanghua2020}
H. Wang and X.F. Qian, npj Comput. Mater. 6, 199 (2020).
\bibitem{Yang2020}
R.X. Fei, W.S. Song, and L. Yang,
Phys. Rev. B 102, 035440 (2020).
\bibitem{Duan2022}
H. Chen, M. Ye, N. Zou, B.-L. Gu, Y. Xu, and W. H. Duan,
Phys. Rev. B 105, 075123 (2022).
\bibitem{Xiao2010}
D. Xiao, M.-C. Chang, and Q. Niu,
Rev. Mod. Phys. 82, 1959 (2010).
%\bibitem{Wu2016}
%L. Wu \textit{et al},
%Nat. Phys. 13, 350–355 (2016).
\bibitem{Sipe2}
C. Aversa and J.E. Sipe,
Phys. Rev. B 52, 14636 (1995).
\bibitem{Sipe1}
J. E. Sipe and A. I. Shkrebtii,
Phys. Rev. B 61, 5337 (2000).
\bibitem{Nagaosa1}
T. Morimoto and N. Nagaosa,
Sci. Adv. 2, 1501524 (2016).
\bibitem{J-Moore1}
T. Rangel, B. M. Fregoso, B. S. Mendoza, T. Morimoto, J. E. Moore, and J. B. Neaton,
Phys. Rev. Lett. 119, 067402 (2017).
\bibitem{Circu0}
F. de Juan, A. G. Grushin, T. Morimoto and J. E. Moore,
Nat. Communi. 8, 15995 (2017).
\bibitem{X-Qian1}
H. Wang and X.F. Qian,
Sci. Adv. 5, 9743 (2019).
\bibitem{Ahn}
J. Ahn, G.-Y. Guo, and N. Nagaosa,
Phys. Rev. X 10, 041041 (2020).
\bibitem{Hosur2011}
P. Hosur,
Phys. Rev. B 83, 035309 (2011).
\bibitem{science2022}
M. G. Vergniory \textit{et al.},
Science, 376, 6595, (2022).
\bibitem{Fregoso1}
B. M. Fregoso, R. A. Muniz, and J. E. Sipe,
Phys. Rev. Lett. 121, 176604 (2018).
\bibitem{Ventura2021}
G. B. Ventura \textit{et al.},
Phys. Rev. Lett. 126, 259701 (2021).
\bibitem{Fregoso}
B. M. Fregoso,
Phys. Rev. B 100, 064301 (2019).
\bibitem{Fregoso2021}
B. M. Fregoso, R. A. Muniz, and J. E. Sipe,
Phys. Rev. Lett. 126, 259702 (2021).
\bibitem{Buttiker}
Ya. M. Blanter and M. Buttiker,
Phys. Rep. 1, 336 (2000).
\bibitem{negative}
M. Henny \textit{et al.},
Science 284, 296 (1999).
%\bibitem{negative}
%W. D. Oliver \textit{et al.},
%Science 284, 299 (1999).
\bibitem{positive1}
M. P. Anantram and S. Datta,
Phys. Rev. B 53, 16 390 (1996).
\bibitem{positive2}
P. Samuelsson and M. Buttiker,
Phys. Rev. Lett. 89, 046601 (2002).
\bibitem{note2}
By DC SN, we mean zero frequency component of SN.
%\bibitem{note1} For linear polarization, $\vect{E}$ is real so that $|\vect{E} \times \vect{E}^*|=0$
%while for circular polarization $\vect{E}(t) = \vect{E}(\omega) e^{i\omega t} + \vect{E}(-\omega) e^{-i\omega t}$
%with $\vect{E}(\omega) = |\vect{E}(\omega)| (i\hat{e}_x + \hat{e}_y)$,
%$|\vect{E}|^2 \equiv \vect{E}(\omega) \cdot \vect{E}(\omega) + \vect{E}(-\omega) \cdot \vect{E}(-\omega)=0$.
\bibitem{sup}
Supplemental Material.
\bibitem{photodrag}
L.-K. Shi, D. Zhang, K. Chang, and J. C. W. Song,
Phys. Rev. Lett. 126, 197402 (2021).
In this paper, the authors propose that a nonzero shift current in centrosymmetric crystals
can be activated by a photon-drag effect, in which the shift current is propotional
to $E^3$ and hence bypass the constraint from $\mathcal{P}$ symmetry.
\bibitem{M-Wei}
M. M. Wei, L. Y. Wang, B. Wang, L. J. Xiang, F. M. Xu, B. G. Wang, and J. Wang,
Phys. Rev. Lett. 130, 036202 (2023).
\bibitem{Juan}
F. de Juan, Y. Zhang, T. Morimoto, Y. Sun, J. E. Moore, and A. G. Grushin,
Phys. Rev. Research 2, 012017(R) (2020).
\bibitem{Rappe1}
L. Gao, Z. Addison, E. J. Mele, and A. M. Rappe,
Phys. Rev. Res. 3, L042032 (2021).
\bibitem{Yanase2}
H. Watanabe and Y. Yanase,
Phys. Rev. X 11, 011001 (2021).
\bibitem{Moore2}
D. E. Parker, T. Morimoto, J. Orenstein, and J. E. Moore,
Phys. Rev. B 99, 045121 (2019).
\bibitem{Park}
J. M. Lihm and C. H. Park,
Phys. Rev. B 105, 045201 (2022).
\bibitem{Nagaosa2018}
Y. Tokura and N. Nagaosa,
Nat. Communi. 9, 3740 (2018).
\bibitem{Nagaosanoise}
T. Morimoto, M. Nakamura, M. Kawasaki, and N. Nagaosa,
Phys. Rev. Lett. 121, 267401 (2018).
We note that this paper is also discucssed the shot noise of shift current,
but these authors uses the local current operator instead of the shift current operator used in this work.
In addition, their method can only be applied to the two-band model and our theory
is anmenable to first-principles calculation.
\bibitem{Yan2020}
T. Holder, D. Kaplan , and B. H. Yan,
Phys. Rev. Research 2, 033100 (2020).
\bibitem{Xiong2022}
Z. S. Wu and W. Xiong,
J. Chem. Phys. 157, 134702 (2022).
\bibitem{Bilbao}
S. V. Gallego, J. Etxebarria, L. Elcoro, E. S. Tasci, and J. M. Perez-Mato,
Acta Crystallogr. Sect. A 75, 438 (2019).
\bibitem{G-Li}
G. Li, K. Kushnir, M. Wang, Y. Dong, S. Chertopalov, A. M.
Rao, V. N. Mochalin, R. Podila, K. Koski, and L. V. Titova, in
2018 43rd International Conference on Infrared, Millimeter, and
Terahertz Waves (IRMMW-THz) (2018).
\bibitem{WangHuan1}
H. Wang, X.Y. Tang, H.W. Xu, J. Li and X.F. Qian,
npj Quantum Materials 7, 61 (2022).
%\bibitem{W-Duan1} H.W. Chen, M. Ye, N.L. Zou, B.L. Gu, Y. Xu, and W.H. Duan, Phys. Rev. B, 105, 075123 (2022).
\bibitem{inj-exp}
N. Laman, A. I. Shkrebtii, J. E. Sipe, and H. M. van Driel,
Appl. Phys. Lett. 75, 2581 (1999).
\bibitem{FregosoGeS2019}
S. R. Panday, S. Barraza-Lopez, T. Rangel, and B. M. Fregoso,
Phys. Rev. B 100, 195305 (2019).
\bibitem{WanghuaSci2019}
H. Wang and X.F. Qian,
Sci. Adv. 5, eaav9743 (2019).
\bibitem{YangDY2022}
D.Y. Yang \textit{et al.},
Nature Photonics 16, 469–474 (2022).
\bibitem{SuiFG2023}
F.G. Sui \textit{et al.}
Nat. Communi. 14, 36 (2023).
\bibitem{LiJu2021}
M. Wu and J. Li,
Proc. Natl. Acad. Sci. USA 118, e2115703118 (2021).
\bibitem{Stern2021}
M. V. Stern \textit{et al.},
%Y. Waschitz, W. Cao, I. Nevo, K. Watanabe, T.
%Taniguchi, E. Sela, M. Urbakh, O. Hod, and M. Ben Shalom,
%Interfacial ferroelectricity by van der Waals sliding, 
Science 372, 1462 (2021).
\bibitem{exp1}
K. F. Mak, C. Lee, J. Hone, J. Shan, and T. F. Heinz,
Phys. Rev. Lett. 105, 136805 (2010). 
\bibitem{exp2}
J. McIver, D. Hsieh, H. Steinberg, P. Jarillo-Herrero, and N. Gedik,
Nat. Nanotech. 7, 96 (2012).
\bibitem{exp3}
A. Brenneis \textit{et al.},
Nat. Nanotech. 10, 135–139 (2015).
\bibitem{exp4}
Q. Ma, \textit{et al.},
Nat. Phys. 13, 842–847 (2017).
\bibitem{exp5}
G. B. Osterhoudt \textit{et al.},
Nat. Mater. 18, 471–475 (2019).
\bibitem{exp6}
N. M. Gabor, Z. Zhong, K. Bosnick,  and P. L.  McEuen,
Phys. Rev. Lett. 108, 087404 (2012).
\bibitem{exp7}
M. Henny, S. Oberholzer, C. Strunk, T. Heinzel, K. Ensslin, M. Holland, and C. Sch\"oenberger,
Science 284, 296 (1999). 
\bibitem{exp8}
H. Birk, M. J. M. de Jong, and C. Sch\"onenberger,
Phys. Rev. Lett. 75, 1610 (1995).
\bibitem{exp9}
H. Birk, K. Oostveen, and C. Sch\"oenberger,
Rev. Sci. Instrum. 67, 2977 (1996). 
\bibitem{exp10}
L. DiCarlo, Y. Zhang, D. T. McClure, C. M. Marcus, L. N. Pfeiffer, and K.W. West,
Rev. Sci. Instrum. 77, 073906 (2006).
\bibitem{kresse1996efficiency}
G. Kresse and J. Furthmüller, Comput. Mater. Sci. 6, 15 (1996).
\bibitem{kresse1996efficient}
G. Kresse and J. Furthmüller, Comput. Mater. Sci. 6, 15 (1996).
\bibitem{QE-2009}
P. Giannozzi \textit{et al.}
J. Phys. Condens. Matter 21, 395502 (2009).
\bibitem{perdew1996generalized}
J. P. Perdew, K. Burke, and M. Ernzerhof,
Phys. Rev. Lett. 77, 3865 (1996).
\bibitem{vdw1}
S. Grimme, J. Antony, S. Ehrlich, and H. Krieg,
The Journal of Chemical Physics 132, 154104 (2010).
\bibitem{vdw2}
S. Grimme, S. Ehrlich, and L. Goerigk,
Journal of Computational Chemistry 32, 1456 (2011).
\bibitem{w90-2020}
G. Pizzi \textit{et al.},
J. Phys. Condens. Matter 32, 165902 (2020).
\bibitem{wanniersc}
J.I. Azpiroz, S. S. Tsirkin, and I. Souza,
Phys. Rev. B 97, 245143 (2018).
\end{thebibliography}

\bigskip
\noindent \textbf{\large ACKNOWLEDGEMENTS}\\
This work was financially supported by the Natural Science Foundation of China (Grant No.12034014 and No. 12004442).

\bigskip
\noindent \textbf{\large AUTHOR CONTRIBUTIONS}\\
J.W. conceived the project.
L.J.X. and J.W. developed the theory and performed the symmetry analysis.
L.J.X. and H.J. performed the first-principles calculations.
L.J.X., H.J., and J.W. wrote the paper. J.W. supervised the project.
All authors analyzed the data and contributed to the discussions of the results.

\bigskip
\noindent \textbf{\large COMPETING INTERESTS}\\
The authors declare no competing interests.

\widetext
\clearpage

\end{document}
