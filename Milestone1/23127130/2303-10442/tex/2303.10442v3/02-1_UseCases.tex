\subsection{Emerging Applications and Use cases} \label{sec:usecases}

%New Wi-Fi capabilities could lead to a vast number of new applications and services. 
The key use cases in 2030 and beyond for indoor connectivity %\st{in unlicensed bands} 
are foreseen to include the following \cite{UHRProposedPAR,HexaD13}. 

\noindent\emph{Immersive communications}: Moving from augmented/virtual reality (AR/VR) glasses to holographic telepresence. 

\noindent\emph{Digital twins for manufacturing}: Establishing a virtual connection between a digital representation of a complex system or environment and its real-world counterpart.  

\noindent\emph{e-Health for all}: Providing remote medical surgery in areas where doctors and infrastructure are lacking. 

\noindent\emph{Cooperative mobile robots}: Requiring deterministic communication for handling critical motion control information. 

\noindent Table~\ref{tab:UseCases} quantifies the performance requirements for the above use cases. \blue{To approach these latency and reliability requirements,} Wi-Fi is considering a paradigm shift towards introducing more performance determinism. This is not an easy task, since \blue{unlike 3GPP technologies like 5G operating in licensed bands, Wi-Fi operates in unlicensed bands subject to channel access contention and uncontrolled interference. To cope with uncoordinated usage in the unlicensed spectrum, rather than prioritize determinism,} Wi-Fi's medium access control (MAC) was originally designed upon carrier sense multiple access with collision avoidance (CSMA/CA). Evolving from this legacy, Wi-Fi 8 \blue{intends to} pursue determinism through coordination and a \blue{more efficient use of the available spectrum.} 
