\blue{\subsection{Novel Standardization Efforts}}
\label{sec:standardization}

\subsubsection*{IEEE 802.11 Real Time Applications (RTA) Topic Interest Group (TIG)}
Back in 2019, the RTA-TIG provided a set of recommendations and guidelines to support low latency and reliability in future Wi-Fi networks \cite{RTA}. Those recommendations have been considered in the Wi-Fi 7 development (e.g., MLO), but they are also influencing the effort towards reliability in Wi-Fi 8, e.g., \blue{via time-sensitive networking (TSN)} integration.

\subsubsection*{IEEE 802.11 AI/ML Topic Interest Group (TIG)} 
Established to explore the application of artificial intelligence (AI) and machine learning (ML) \blue{directly to Wi-Fi protocols. Its aim is to discuss relevant use cases, together with their technical feasibility based on existing mechanisms and expected implementation efforts. These include channel state information (CSI) feedback compression using neural networks, enhanced roaming assisted by AI/ML, deep reinforcement learning-based channel access, and enhanced multi-AP coordination schemes driven by AI/ML \cite{AIMLTIG}.}

\subsubsection*{\blue{IEEE 802.11 Integrated mmWave Study Group (IMMW SG)}} 
%
\blue{To ensure the long-term evolution of Wi-Fi, next-generation high-end devices could also potentially operate in all three sub-7 GHz bands as well as in the mmWave realm. Indeed, there is a growing interest in better capitalizing on the up to 14 GHz of licensed-exempt spectrum available nearly worldwide in the 60 GHz bands or 5.5 GHz in the 45 GHz band in China, respectively. 
The 60 GHz band is currently used by several incumbent technologies, such as satellite, radio astronomy, and IEEE 802.11ad/ay (WiGig). However, the market adoption of WiGig has been confined to niche applications, and regulatory bodies may consider repurposing the 60 GHz band for other bandwidth-hungry technologies such as 5G and 6G. Against this background, and after initial discussions about extending the UHR scope, it was decided to create a dedicated IMMW SG to pose the basis for the development of a new 802.11 amendment, leveraging PHY/MAC functionalities of the existing Wi-Fi 7 and future Wi-Fi 8 radio interfaces for the sub-7 GHz bands, including channelization and multi-link framework to dynamically operate additional mmWave links.}

\subsubsection*{IEEE 802.11bn UHR} 
%
\blue{Fig.~\ref{fig:timeline} summarizes the ongoing IEEE standardization effort for 802.11bn (bottom) alongside the nearly completed 802.11be amendment (top) and its consolidated main features.} %\cite{lopez2019ieee,garcia2021ieee,khorov2020current,CheCheDas22}}.
\blue{The UHR Study Group (SG)} was established in July 2022 to discuss and produce a new Project Authorization Request (PAR) defining the set of objectives, frequency bands, and technologies to be considered beyond 802.11be. The \blue{resulting UHR Task Group (TG) was formed in November 2023}, with a traditional single release standardization cycle that will last until 2028. This activity will define the protocol functionalities of future Wi-Fi\,8 products, mainly focusing on these aspects to be improved with respect to 802.11be \cite{UHRProposedPAR}:
\blue{
\begin{itemize}
\item Increasing throughput by 25\%, as measured at the MAC data service AP. 
\item Reducing by 25\% the 95th percentile latency and by 25\% MAC Protocol Data Unit (MPDU) loss, even in scenarios with mobility and overlapping basic service sets (OBSSs).
\item Improving power saving mechanisms for the AP and enhancing direct peer-to-peer data exchanges.
\end{itemize}
Three main critical aspects impacting reliability in the unlicensed spectrum are being investigated: seamless connectivity, determinism, and controlled worst-case delay. Fig.~\ref{fig:WiFi8Features} depicts examples for each, with their chief opportunities and challenges discussed in the next three sections, respectively.
}