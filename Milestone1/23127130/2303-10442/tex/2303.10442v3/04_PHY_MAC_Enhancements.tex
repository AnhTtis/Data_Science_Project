\section{\blue{Determinism via PHY and MAC Enhancements}}

\blue{Considering traffic characteristics is crucial in designing low-latency mechanisms. While it would be desirable to handle traffic with predictable arrival patterns by leveraging existing solutions \cite{bankov2020tuning}, the challenge intensifies when accommodating unexpected, event-driven, time-sensitive traffic.}

\subsubsection*{Opportunities} \blue{Upon the arrival of an unexpected high-priority packet at a device, two primary sources of latency may be encountered: the remaining time of another ongoing transmission and the following channel contention procedure for its own transmission. IEEE 802.11bn can tackle both aspects by: (i) extending enhanced distributed channel access (EDCA) with additional priority classes and associated channel access parameters, e.g., for backoff; (ii) extending OFDMA implementation by introducing resource unit (RU) reservations and enabling preemption; and (iii) taking advantage of channel access opportunities in secondary channels when the primary channel is occupied by other transmissions.}

\subsubsection*{Proposed mechanisms} 
\blue{Addressing the latter two opportunities, 802.11bn is contemplating the introduction of two MAC enhancements.
\emph{Resource Reservation and Channel Preemption} could potentially reserve a small RU for low-latency traffic in all transmissions. Coupled with pre-padding, this would enable a node to promptly serve incoming low-latency packets by allocating them to the reserved RU. However, to avoid RU wastage across all transmissions, this RU could also be used for the transmission of actual data, provided that preemption is supported (see Fig.~\ref{fig:WiFi8Features}, second example). Notably, this approach does not impose receiver design changes, but it does require designing a multidimensional PPDU frame. However, if the device aiming to transmit a time-sensitive frame is not the transmission opportunity (TXOP) holder, effective preemption mechanisms must exploit short inter-frame spaces between transmitted PPDUs to seize the channel.} 
%
\blue{In addition, \emph{Secondary Channel Access (SCA)} may extend the preamble puncturing functionalities in 802.11be by removing dependency on the primary channel and better leveraging transmission opportunities on an idle secondary channel only. The introduction of SCA is anticipated to yield performance gains in scenarios with medium-to-high load without entailing excessive complexity.} 

%%%%%%%%%%% DELETED STUFF %%%%%%%%%%%

%\bcom{References left as text, not added to the bib. To discuss if there is room for any. We should balance them between sections in any case.}

%\bcom{I added a reference to the bib (to a paper of Khorov that fits well here). I think we can remove all others from this section. Now we have 16 in total. I would remove one of the initial references to Wi-Fi 7 (there are many)}

%Wi-Fi\,8 will consider the possibility to include PHY/MAC enhancements such as hybrid automatic repeat request (HARQ) and increasing the number of supported spatial streams from 8 to 16. The use of HARQ could allow devices to combine corrupted data units with their corresponding retransmissions to increase the probability of correct decoding, reducing latency in challenging channel conditions. 
%The availability of additional spatial streams could enable more users to be served simultaneously, reducing their channel access time, and provide extra degrees of freedom to mechanisms such as coordinated beamforming (discussed later in this section). 
%Additionally, other features building atop TXOP sharing functionalities will potentially allow APs to share a portion of their obtained TXOPs with associated stations for transmitting uplink frames to the AP or for direct peer-to-peer communication with another station. 
%The new functionalities, combined with the aforementioned R-TWT mechanisms and TXOP sharing principles, already promise improvements in terms of reliability. However, further enhancements may be needed to support the arrival of unexpected or event-based time-sensitive traffic during large ongoing transmissions. An interesting proposal to address this issue is the one of frame `preemption', detailed below. \bcom{All previous text is too general perhaps. I would remove most of it. Perhaps mentioning just R-TWT and TXOP sharing as the starting point for this section. A link also with previous SCA section is needed.}

% LORE: REMOVED FROM BORIS Traffic with predictable arrival patterns seamlessly integrates with solutions like R-TWT, in which Access Points (APs) and stations establish dedicated service periods to facilitate the efficient exchange of low-latency traffic. While R-TWT functions at the management level, EDCA can also be extended to accommodate predictable low-latency traffic with minimal complexity by initiating contention in advance \cite{bankov2020tuning}. This approach aligns the packet arrival time with the end of the preamble transmission, enabling immediate packet delivery to the medium. Challenges associated with this approach include addressing delayed packet arrivals, which can be effectively managed through a "pre-padding" technique to maintain channel activity. Additionally, determining the PPDU duration is essential, striking a balance between addressing arrival delays and preventing unnecessary deferrals by other APs and stations.
% (802.11-23/1155r0)
%Traffic with predictable arrival patterns is easier to be handled leveraging existing solutions \cite{bankov2020tuning}, however, accommodating unexpected or event-driven time-sensitive traffic is a further challenge. 
%LORE: REMOVED FROM BORIS \st{especially when it arrives during large ongoing transmissions, whether from the same device, another within the same BSS, or from a device in a different OBSS. To address this, proposed solutions include:
%
%LORE: REPHRASED AND MERGED FOR REDUCING TEXT\subsubsection*{Improved traffic prioritization} There is a general agreement that fine grained priority levels are required to handle the simultaneous presence of multiple high priority traffic flows, either by defining and adding more ACs, i.e., priority levels, or by improving the management of existing ones beyond the current FIFO approach.
%[lorenzo] \blue{\subsubsection*{Improved traffic prioritization} There is a general agreement that fine grained priority levels are required to handle the simultaneous presence of multiple high priority traffic flows, either by defining and adding more access categories (ACs), i.e., priority levels, or by improving the management of existing ones beyond the current FIFO approach. With the latter focusing on creating further isolation between prioritized and best-effort stations by isolating prioritized stations at the contention period’s start using a "Defer Signal" control frame for the non-prioritized traffic. This ensures only prioritized stations contend, allowing for appropriate contention parameter adjustments. However, selecting these parameters and preventing unfairness for non-prioritized stations poses challenges.}
%
%LORE: REPHRASED AND MERGED FOR REDUCING TEXT \subsubsection*{Multiple Contention Stages} Contention parameters used in high-priority ACs, such as AIFSN and CW values, were initially designed to support traffic differentiation relative to other Access Categories (ACs). However, when there are multiple high-priority contenders, these parameters can lead to high collision rates and subsequent retransmissions, ultimately compromising the efficiency of low-latency traffic delivery. To alleviate these high collision rates and retransmissions caused by multiple low-latency high-priority contenders, further isolation between prioritized and best-effort stations should be achieved. This would ensure that only prioritized stations contend to access the channel, allowing for appropriate contention parameter adjustments. However, selecting these parameters and preventing unfairness for non-prioritized stations pose challenges.
% (802.11-23/1065r0)
%LORE: REPHRASED AND MERGED FOR REDUCING TEXT \subsubsection*{RU Reservation for Non-Predictable LL Traffic} Reserving a 'small' Resource Unit (RU) for low-latency traffic in all transmissions, coupled with pre-padding, enables a node to promptly serve incoming low-latency packets by allocating them to the reserved RU. However, to prevent RU wastage in all transmissions, this RU can also be used for the transmission of actual data given ‘preemption’ is supported. In this case, the TXOP holder could seamlessly replace an ongoing best-effort PPDU with a time-sensitive one on the fly, interrupting the current PPDU transmission, which can be later retransmitted. As an illustration, consider the scenario depicted in Fig.~\ref{fig:WiFi8Features}, the third example. Here, upon receiving a time-sensitive packet destined for Device 1 (URLLC), the Access Point (AP) interrupts the ongoing transmission to Device 2, specifically PPDU 1 in the upper 80 MHz resource unit, in order to transmit the new time-sensitive frame. Notably, this approach requires no receiver design changes but necessitates designing a multidimensional PPDU frame.
%
%[lorenzo] \blue{\subsubsection*{Resource Reservation and Channel Preemption} Reserving a 'small' Resource Unit (RU) for low-latency traffic in all transmissions, coupled with pre-padding, enables a node to promptly serve incoming low-latency packets by allocating them to the reserved RU. However, to prevent RU wastage in all transmissions, this RU can also be used for the transmission of actual data given ‘preemption’ is supported. As illustrated in the third example of Fig.~\ref{fig:WiFi8Features}, upon receiving a time-sensitive packet destined for Device 1 (URLLC), the Access Point (AP) interrupts the ongoing transmission to Device 2, specifically PPDU 1 in the upper 80 MHz resource unit, in order to transmit the new time-sensitive frame. Notably, this approach does not require receiver design changes but it necessitates designing a multidimensional PPDU frame. However, when the device aiming to transmit a time-sensitive frame is not the TXOP holder of the on-going transmission is more complex. Effective preemption mechanisms must exploit temporal inter-frame spaces between transmitted PPDUs. Alternatively, a short guard period can be allocated at the end of scheduled TXOPs, allowing only devices with time-sensitive traffic to transmit.} 
%
%LORE: REPHRASED AND MERGED FOR REDUCING TEXT \subsubsection*{Channel Preemption by a Non-TXOP Holder} Preemption when the device aiming to transmit a time-sensitive frame is not the TXOP holder of the on-going transmission is more complex. Effective preemption mechanisms must exploit temporal inter-frame spaces between transmitted PPDUs. Alternatively, a short guard period can be allocated at the end of scheduled TXOPs, allowing only devices with time-sensitive traffic to transmit. However, in order to facilitate preemption by a non-TXOP holder, it is essential to steer clear of large A-MPDU transmissions, despite the potential trade-off in performance. It is important to note that these Wi-Fi 8 proposals are in the early stages and may face challenges when dealing with legacy devices and OBSS transmissions. In such cases, the SCA approach described in the previous section can help to mitigate them.

%LORE: REPHRASED AND MERGED FOR REDUCING TEXT \subsubsection*{Secondary Channel Access (SCA)} The integration of wide channels, particularly the 160 and 320 MHz variants, within Wi-Fi 8 holds substantial potential for enhancing Wi-Fi capacity. While dynamic channel access mechanisms, such as preamble puncturing, adeptly manage the available spectrum in secondary channels, when the primary channel experiences congestion due to OBSS activity, current transmission is unavoidably delayed, irrespective of the status of the secondary channels. In response to this challenge, Wi-Fi 8 is exploring to leverage available transmission opportunities on the idle secondary channels.
%%LORE: REPHRASED AND MERGED FOR REDUCING TEXT The introduction of SCA is anticipated to yield performance gains without imposing significant complexity, although the extent of these gains in terms of both throughput and latency depends on the specific scenario and OBSS configuration. An initial possibility is to limit its use to low-latency applications, as sporadic and short data transmissions are well-suited to harness the advantages offered by secondary channel access opportunities.

%[lorenzo] \blue{\subsubsection*{Secondary Channel Access (SCA)} While dynamic channel access mechanisms such as preamble puncturing adeptly manage the available spectrum in secondary channels, they still require the availability of a primary channel. Thus, when the primary channel experiences congestion due to OBSS activity, secondary channels cannot be added, irrespectively of their status. In response to this challenge, Wi-Fi 8 is exploring to leverage available transmission opportunities on the idle secondary channels. The introduction of SCA is anticipated to yield performance gains without imposing significant complexity, although the extent of these gains in terms of both throughput and latency depends on the specific scenario and OBSS configuration.} 

%the concept of supporting Secondary Channel Access (SCA).
%
%The conceived operation of Secondary Channel Access (SCA) unfolds as follows: Initially, an Access Point (AP) and its associated stations reach a consensus to utilize an auxiliary 'primary' channel, situated within one of the 20 MHz secondary channels. This auxiliary channel serves as an alternative when the primary channel becomes occupied due to transmissions from overlapping Basic Service Sets (OBSS). In such scenarios, the AP and stations detecting OBSS activity on the primary channel seamlessly transition to monitor the auxiliary primary channel. Once it is determined to be idle, they engage in standard channel contention procedures, followed by an RTS/CTS or BSRP/BRP exchange to secure the channel for data transmission. 
%
%An important constraint to adhere to is that the secondary Transmission Opportunity (TXOP) duration must conclude before the initiation of the transmission that occupies the primary channel. This allows all devices within the Basic Service Set (BSS) to return to normal operations once the primary channel becomes available again. While this approach may appear intuitive, it does challenge the conventional requirement mandating the inclusion of the primary channel in all transmissions and receptions.
%


%Preliminary studies indicate potential gains ranging from 50\% to 250\%. Importantly, these studies also demonstrate that non-SCA BSSs do not experience any adverse performance effects. While the incorporation of SCA into Wi-Fi 8 may not represent a revolutionary leap, it does hold the promise of increasing Wi-Fi performance levels at minimal cost by creating additional channel access opportunities. 
%Nevertheless, further research is necessary to fully understand how SCA interacts with channel allocation strategies and dynamic channel access protocols, as its presence could encourage more aggressive overlapping between OBSSs. Finally, we advise to initially limit SCA to applications involving low-latency and high-priority traffic. Such traffic typically comprises sporadic and limited data volumes, making it well-suited to harness the advantages offered by secondary channel access opportunities.

%\blue{\subsubsection*{Improved traffic prioritization} There is a general agreement that fine grained priority levels are required to handle the simultaneous presence of multiple high priority traffic flows, either by defining and adding more access categories (ACs), i.e., priority levels, or by improving the management of existing ones beyond the current FIFO approach. With the latter focusing on creating further isolation between prioritized and best-effort stations by isolating prioritized stations at the contention period’s start using a "Defer Signal" control frame for the non-prioritized traffic. This ensures only prioritized stations contend, allowing for appropriate contention parameter adjustments. However, selecting these parameters and preventing unfairness for non-prioritized stations poses challenges.}

%REPHRASED \noindent\blue{\emph{Resource Reservation and Channel Preemption:} Reserving a 'small' Resource Unit (RU) for low-latency traffic in all transmissions, coupled with pre-padding, enables a node to promptly serve incoming low-latency packets by allocating them to the reserved RU. However, to prevent RU wastage in all transmissions, this RU can also be used for the transmission of actual data given ‘preemption’ is supported. As illustrated in the second example of Fig.~\ref{fig:WiFi8Features}, upon receiving a time-sensitive packet destined for Device 1 (URLLC), the Access Point (AP) interrupts the ongoing transmission to Device 2, specifically PPDU 1 in the upper 80 MHz resource unit, in order to transmit the new time-sensitive frame. Notably, this approach does not require receiver design changes but it necessitates designing a multidimensional PPDU frame. However, when the device aiming to transmit a time-sensitive frame is not the TXOP holder of the on-going transmission is more complex. Effective preemption mechanisms must exploit temporal inter-frame spaces between transmitted PPDUs. Alternatively, a short guard period can be allocated at the end of scheduled TXOPs, allowing only devices with time-sensitive traffic to transmit.} 

%REPHRASED \noindent\blue{\emph{Secondary Channel Access (SCA):} While dynamic channel access mechanisms such as preamble puncturing adeptly manage the available spectrum in secondary channels, they still require the availability of a primary channel. Thus, when the primary channel experiences congestion due to OBSS activity, secondary channels cannot be added, irrespectively of their status. In response to this challenge, Wi-Fi 8 is exploring to leverage available transmission opportunities on the idle secondary channels. The introduction of SCA is anticipated to yield performance gains without imposing significant complexity, although the extent of these gains in terms of both throughput and latency depends on the specific scenario and OBSS configuration.} 