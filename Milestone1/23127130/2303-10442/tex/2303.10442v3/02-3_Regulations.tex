\green{\subsection{Spectrum Allocation and Usage}} 
%\subsection{mmWave Spectrum} 
\label{sec:regulations}

\st{To maintain Wi-Fi's competitiveness and support new demanding applications, it will be crucial to secure new spectrum. 
A significant step has been taken in this regard with the decision by many countries to make the band around 6\,GHz available for secondary licensed-exempt use by Wi-Fi.
The new 6\,GHz band will offer significant advantages to Wi-Fi\,6E and Wi-Fi\,7 devices, since older generations of Wi-Fi will not be able to use it. However, these benefits will diminish as devices populate the band. 
Moreover, the 6\,GHz band can only support (in best-case scenarios) up to three 320 MHz channels, and may be unable to meet the long-term needs of Wi-Fi applications. 
In addition, the adoption of the 6\,GHz spectrum in China is still uncertain, challenging the evolution of Wi-Fi towards supporting new use cases. 
Proposals for a more efficient and integrated use of the mmWave spectrum have therefore emerged alongside initial Wi-Fi \,8 standardization discussions.} 
%\bcom{If we need to save space, I would reduce this paragraph a bit. It just lists 'intentions'.}

\blue{The \emph{IEEE 802.11 Integrated mmWave Study Group (IMMW SG)}: To ensure the long-term evolution of Wi-Fi, next-generation high-end devices could also potentially operate in all three sub-7 GHz bands as well as in the mmWave realm. Indeed, there is a growing interest in better capitalizing on the up to 14 GHz of licensed-exempt spectrum available nearly worldwide in the 60 GHz bands or 5.5 GHz in the 45 GHz band in China, respectively. 
The 60 GHz band is currently used by several incumbent technologies, such as satellite, radio astronomy, and IEEE 802.11ad/ay (WiGig). However, the market adoption of WiGig has been confined to niche applications, and regulatory bodies may consider repurposing the 60 GHz band for other bandwidth-hungry technologies such as 5G and 6G. Against this background, and after initial discussions about extending the UHR scope, it was decided to create a dedicated Integrated mmWave (IMMW) SG to consider the development of a new 802.11 amendment. This will specify carrier frequency operation between 42.5 and 71 GHz, leveraging PHY/MAC functionalities of the existing Wi-Fi 7 and future Wi-Fi 8 radio interfaces for the Sub 7 GHz bands, including channelization and multi-link framework to dynamically operate additional mmWave links.}


%OLD R1
%The \emph{IEEE 802.18 mmWave Ad Hoc Group} is exploring options in the 45\,GHz and 60\,GHz bands, which respectively offer 5.5\,GHz and 14\,GHz of spectrum. The 60\,GHz band is currently used by several incumbent technologies, such as satellite, radio astronomy, and IEEE 802.11ad/ay (WiGig). 
%However, the market adoption of WiGig has been confined to niche applications, and regulatory bodies may consider repurposing the 60\,GHz band for other bandwidth-hungry technologies such as 5G and 6G. 
%Against this background, and after initial discussions in the UHR SG, it was decided to create a dedicated \emph{Integrated mmWave (IMW) SG} to develop a new 802.11 amendment specifying carrier frequency operation between 42.5 and 71\,GHz and leveraging PHY/MAC functionalities in the existing Sub 7 GHz bands and in the future IEEE 802.11bn amendment (Wi-Fi 8).