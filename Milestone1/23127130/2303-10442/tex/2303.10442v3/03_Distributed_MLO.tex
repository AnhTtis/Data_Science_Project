\section{\blue{Seamless Connectivity via Distributed MLO}}

\begin{figure*}
    \centering
    \includegraphics[width=0.85\textwidth]{Figures/FeaturesWiFi8.eps}
    \caption{Illustrative examples of the key features being investigated for Wi-Fi\,8.} 
   \label{fig:WiFi8Features}
\end{figure*}

\blue{The multi-link architecture introduced in IEEE 802.11be offers a high degree of flexibility, presenting a clear split between upper (multi-link level) and lower (link level) MAC functionalities, with a multi-link device (MLD) that can be viewed as an entity controlling two or more legacy APs (or STAs), each operating on a single link and co-located on the same hardware \cite{garcia2021ieee,khorov2020current,CheCheDas22}. 
%\cite{garcia2021ieee,khorov2020current,deng2020ieee,yang2020survey,CheCheDas22}. 
This multi-link framework already allows a multi-link station to switch links with minimal signaling overhead and delay, implicitly enabling seamless transitions between APs under the control of the same MLD entity, and thus allowing for a make-before-break path switch.}

\subsubsection*{Opportunities}
\blue{To improve mobility support, one of the major sources of unreliability in Wi-Fi, 802.11bn has the possibility to extend the just described multi-link architecture towards a distributed framework, where APs under the control of the same MLD entity do not necessary have to be co-located on the same physical hardware. 
This approach creates a distributed virtual cell where a device’s mobility is seamlessly handled by enabling multiple links to be concurrently activated from different distributed APs, thus ensuring that a nomadic device is always connected to at least one link, effectively embedding native roaming support into 802.11bn and significantly improving the connection’s reliability.}

\subsubsection*{Technical challenges} 
\blue{Several critical aspects would need to be addressed to implement distributed multi-link operations in 802.11bn. Primarily, the distributed MLO approach requires coordination and communication among the different distributed APs under the same controlling multi-link instance. In addition, different links would need unique addressing, considering that current 802.11be specification guarantees different identifiers only for links activated within the same physical device.}

\subsubsection*{Potential implementation}
\blue{The coordination among the different distributed APs may be implemented following different approaches. One option is to define a mobility domain where APs, either co-located or not, could be affiliated with an extended MLD entity. Another possible option is to consider a novel overarching logical entity that would provide seamless roaming between links located in two or more 802.11be AP MLDs. In addition, the 802.11bn distributed MLD architecture should define novel, reliable, and sufficiently general interfaces between the coordination entity (e.g., MLD upper MAC) and the coordinated APs (e.g., MLD lower MAC) to allow for the usage of both wired and wireless communications and to ensure interoperability among implementations provided by different vendors.}