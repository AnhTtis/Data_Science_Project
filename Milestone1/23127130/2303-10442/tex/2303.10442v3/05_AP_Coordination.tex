\section{\blue{Controlled Worst-Case Delay via\\ Multi-AP Coordination}}

Random access procedures \blue{make it difficult} to provide performance guarantees in Wi-Fi, which is why R-TWT was introduced in 802.11be to reduce contention within a basic service set (BSS) \blue{by means of scheduling coordinated service periods}. However, inter-BSS interactions are still governed by contention principles, even if the APs belong to the same administrative domain, making worst-case delays unpredictable. Wi-Fi\,8 is expected to address this issue by introducing multi-AP coordination (MAPC) to achieve greater reliability and prevent channel access contentions, especially in dense and heavily loaded environments.
%
\blue{
\subsubsection*{Channel state information acquisition}
The implementation of multi-AP coordination mechanisms relies on OBSS CSI, i.e., on estimating the channel for non-associated neighboring devices. 
A certain BSS AP can initiate the OBSS channel sounding procedure through a trigger frame that indicates the IDs of the STAs and AP in the OBSS. The OBSS AP follows by transmitting control frames for sounding (e.g., NDPA and NDP). The OBSS STAs then respond by feeding back the measured channel information to both the BSS AP and the OBSS AP. This procedure may be performed several times to acquire channel state information from multiple OBSS \cite{mentorLG_0854r0}.
Availing of such information can be essential to manage frequency resources, adjust the transmit power, or devise specific beamforming methods so as to avoid OBSS interference. As described in the sequel, each of the different AP coordination schemes may require a different amount of channel state information (e.g., overall signal strength vs. per-antenna small-scale fading estimation) with a very different periodicity. As the overheads incurred may offset the performance gains, efficient CSI acquisition will be key for some of these schemes to be part of Wi-Fi 8.
}
%
\blue{
\subsubsection*{Protocol upgrades}
New frames} will be necessary for discovering and managing multi-AP groups, sharing channel and buffer state data between APs, and triggering coordinated multi-AP transmissions to minimize inter-BSS collisions and achieve a more efficient and dynamic spectrum usage. 
AP coordination schemes in Wi-Fi\,8 are envisioned to leverage both over-the-air and wired signaling. These schemes will range from basic to advanced, depending on the amount of data that must be exchanged between access points and their implementation complexity. While it is still to be decided what aspects of the coordination mechanism will be specified by the standard and what will be left for implementation, \blue{the main schemes may possibly include some of those described in the remainder of this section.}
%
%%%%%%%%%%%%%%%%%%%%%%%%%%%%%%%%%%%%%%
%%%%%%%%%%%%%%%%%%%%%%%%%%%%%%%%%%%%%%
%
%
\subsection{\blue{Coordinated TDMA/OFDMA}}
%
\blue{These are} two basic approaches leveraging the time and frequency domain, respectively. In C-TDMA, a TXOP is divided in slots and sequentially allocated to different APs. In C-OFDMA, different portions of the band are allocated to different APs. 

\blue{For instance, with C-OFDMA, an AP that obtains a TXOP is able to share its frequency resources with a set of neighboring APs. The minimum resource unit to be employed is currently under discussion, with smaller units (20\,MHz) offering more flexibility and scheduling gain than larger ones (80\,MHz), but also potentially requiring PHY format changes.}

\blue{On the one hand, C-OFDMA can achieve latency reduction by reducing channel contention. On the other hand, the sharing AP faces computational burden and overhead, as it must first request neighboring APs to report their channel and buffer status, and then schedule and allocate resources accordingly.}



%%%%%%%%%%%%%%%%%%%%%%%%%%%%%%%%%%%%%%
%%%%%%%%%%%%%%%%%%%%%%%%%%%%%%%%%%%%%%


\subsection{\blue{Coordinated Spatial Reuse}}
\blue{In coordinated spatial reuse (C-SR), APs cooperatively control their transmit power, allowing concurrent transmissions and thus increasing the total area throughput.} 
%
\blue{
\subsubsection*{Opportunities}
This approach, incorporating cooperation, represents an upgrade over status-quo 802.11ax spatial reuse, whereby one AP transmits at maximum power and all others must reduce their power accordingly, sometimes to an extent that does not yield a sufficiently high signal-to-interference-plus-noise ratio (SINR). Instead, coordinating the transmit power among APs allows for the guarantee of an adequate SINR at all receiving STAs and to create extra spatial reuse opportunities. Additionally and unlike C-TDMA/OFDMA, C-SR allows parallel transmissions on the same time/frequency resources, and thus potentially achieves a higher throughput and reduced queuing delay.
}
%
\blue{
\subsubsection*{Technical challenges}
C-SR requires measuring the receive signal strength information (RSSI) for interfering links in order to compute the appropriate transmit power. However, since the RSSI is relatively static, such information could be acquired via beacon measurements, incurring only limited overhead. Accounting for beamforming in the computation of the RSSI (and thus of the transmit power) may yield better performance but also increase complexity and overhead.}
%
\blue{
\subsubsection*{Potential implementation}
In a measurement phase, a sharing AP can request intra-BSS STAs to measure and report their RSSI from other APs. 
Once the sharing AP gains access to a TXOP, it collects information from other APs, including which STAs those APs intend to transmit to and their target SINRs. Based on this knowledge, the sharing AP can then calculate the appropriate transmit power for each of the other APs. This information is then communicated via a trigger frame, along with the sharing AP's transmit power, allowing the other APs to set their optimal modulation and coding schemes.
}


%%%%%%%%%%%%%%%%%%%%%%%%%%%%%%%%%%%%%%
%%%%%%%%%%%%%%%%%%%%%%%%%%%%%%%%%%%%%%


\subsection{\blue{Joint Transmission}}
Joint transmission \blue{(JT)} is an advanced approach, also known as \emph{distributed MIMO}, leveraging the spatial domain and involving non-co-located APs that jointly transmit/receive data to/from multiple STAs. 
%
\blue{
\subsubsection*{Opportunities}
Remarkably, JT turns neighboring APs from potential interferers to servers. This approach has the potential to simultaneously achieve high throughput and low latency, since interference can be suppressed without sacrificing the number of spatial streams. 
}
%
\blue{
\subsubsection*{Technical challenges}
The success of JT may depend on designing a new distributed CSMA/CA protocol and ensuring tight synchronization in time, frequency, and phase among the cooperating APs. 
Moreover, this feature requires all APs involved to share the data to be transmitted. 
In order to limit the ensuing overhead and prevent an undesired increase in the queueing delay, joint transmission is likely to require an out-of-band backhaul link to connect the APs, e.g., a 10\,Gbps Ethernet cable \cite{mentorSony_1821r1}.
}
%
\blue{
\subsubsection*{Potential implementation}
A certain AP (AP$_1$) exchanges a coordination request/response with another AP (AP$_2$) to decide whether coordination should be started and which packets would be sent jointly. AP$_1$ then transmits a coordination set to AP$_2$ to start data sharing, e.g., via a wired backhaul. Once data sharing is completed, AP$_1$ sends a coordination trigger to AP$_2$ to start a coordinated transmission, at the end of which both APs receive a block acknowledgment from the receiving STAs. 
Possible solutions to limit the overhead introduced by data sharing could be: (i) completing the data sharing in advance, rather than prior to transmission, whenever possible; (ii) performing wireless packet transmissions to other STAs during wired data sharing to improve efficiency.
}

%%%%%%%%%%%%%%%%%%%%%%%%%%%%%%%%%%%%%%
%%%%%%%%%%%%%%%%%%%%%%%%%%%%%%%%%%%%%%


\subsection{\blue{Coordinated Beamforming}}
\blue{Coordinated beamforming (CBF), also leveraging the spatial domain, is an approach where collaborative APs suppress incoming OBSS interference (see Fig.~\ref{fig:WiFi8Features}, third example).}
%
\blue{
\subsubsection*{Opportunities}
With CBF, a next-generation multi-antenna AP uses its spatial degrees of freedom not only to multiplex its own STAs but also to place radiation nulls to and from neighboring non-associated STAs. This approach makes the AP and its neighboring STAs mutually \emph{invisible}, avoiding channel access contention, allowing transmissions at full power, and potentially improving worst-case latency as a byproduct. 
}
%
\blue{
\subsubsection*{Technical challenges}
Unlike JT, CBF does not require joint data processing as each STA transmits/receives data to/from a single AP, therefore not incurring the data sharing overhead and removing the off-band backhauling needs. However, impact of overhead should be carefully considered when defining the CSI acquisition framework. With the size of the antenna arrays expected to grow, 802.11bn should compare the benefit of a more accurate explicit procedure, naturally entailing higher overhead, with an implicit one that trades accuracy for overhead reduction.   
In addition, since the spatial degrees of freedom are limited by the size of the antenna arrays, an appropriate tradeoff between spatial streams carrying data, beamforming gains, and nulling accuracy should be found, along with opportunistic user scheduling during each newly created spatial reuse opportunity.
}
\blue{
\subsubsection*{Potential implementation}
CBF is likely to entail the design of the following key phases:
\begin{itemize}
    \item A control frame exchange between two or more collaborative APs to establish and maintain a coordination set. 
    %
    \item A CSI acquisition phase, for APs to communicate with OBSS STAs and configure space-domain interference suppression. The latter modifies the conventional filter employed for spatial multiplexing---e.g., a zero forcing (ZF) or minimum mean square error (MMSE) precoder---by imposing nulls on a specific channel direction (aiming for complete nulling towards a certain STA) or subspace (aiming for partial nulling towards multiple STAs).
    %
    \item A framework for dynamic nullsteering-based spatial reuse, wherein a donor AP grants a transmission opportunity to an OBSS AP by communicating its served STAs and correspondent interference suppression conditions, i.e. the obligation for the OBSS AP to place nulls towards the STAs served by the donor AP \cite{mentorUnisoc_0855r1}. 
\end{itemize}
}

\blue{While exhibiting a lower implementation complexity than JT, CBF could circumvent channel contention and, under the right circumstances, could deliver a substantial reduction in worst-case latency. Although the potential of CBF has been recently demonstrated for a single link operation \cite{mentorBroadcom_0855r1,garcia2021ieee}, in the following, we conduct a preliminary evaluation of the performance tradeoffs that arise when CBF is paired with MLO, as envisioned in 802.11bn.}

%%%%%%%%%%%%%%%%%%%%%%%%%%%%%%%%%%%%%%%%%%%%%%%%%%%%%

%\bcom{Perhaps is good to mention that we sacrifice spatial streams}

%Through an A-CTS, a donor AP$_1$ indicates that nullsteering-based spatial reuse is allowed in the following downlink transmission, specifying the target STA$_1$ of such transmission. AP$_1$ then begins transmitting a PPDU to its STA$_1$. A donee AP$_2$, possessing valid channel state information to place a null towards STA$_1$, gains a channel access opportunity and transmits a PPDU to its STA$_2$. A block acknowledgment (BA) is then transmitted from STA$_1$ to AP$_1$, followed by a BA request (BAR) and a BA between AP$_2$ and STA$_2$ \cite{mentorUnisoc_0855r1}.
%

%%OLD - REORGANIZED BY LORENZO
%\green{
%\subsubsection*{Technical challenges}
%CBF is likely to entail the design of the following key phases:
%\begin{itemize}
%    \item A control frame exchange between two or more collaborative APs to establish and maintain a coordination set. 
%    %
%    \item A CSI acquisition phase, for APs to communicate with OBSS STAs and configure space-domain interference suppression. The latter modifies the conventional filter employed for spatial multiplexing---e.g., a zero forcing (ZF) or minimum mean square error (MMSE) precoder---by imposing nulls on a specific channel direction (aiming for complete nulling towards a certain STA) or subspace (aiming for partial nulling towards multiple STAs).
%    %
%    \item A framework for dynamic and opportunistic spatial reuse, wherein a donor AP grants a transmission opportunity to an OBSS AP, as long as interference suppression conditions are met. 
%\end{itemize}
%}
%% OLD - REORGANIZED BY LORENZO 
%\green{
%\subsubsection*{Potential implementation}
%Through an A-CTS, a donor AP$_1$ indicates that nullsteering-based spatial reuse is allowed in the following downlink transmission, specifying the target STA$_1$ of such transmission. AP$_1$ then begins transmitting a PPDU to its STA$_1$. A donee AP$_2$, possessing valid channel state information to place a null towards STA$_1$, gains a channel access opportunity and transmits a PPDU to its STA$_2$. A block acknowledgment (BA) is then transmitted from STA$_1$ to AP$_1$, followed by a BA request (BAR) and a BA between AP$_2$ and STA$_2$ \cite{mentorUnisoc_0855r1}.
%}
%

