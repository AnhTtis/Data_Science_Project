\section{IEEE 802.11be: State-of-the-Art Wi-Fi\,7} \label{sec:WiFi7}

As suggested by its name (IEEE 802.11be `Extremely High Throughput' or `EHT'), Wi-Fi\,7 will augment data rates to at least 30\,Gbps per AP, about four times as fast as Wi-Fi\,6. 
In the sequel, we summarize the main features introduced in the soon-to-appear Wi-Fi\,7 commercial products \cite{garcia2021ieee,CheCheDas22}. 

\subsection{Multi-link Operation (MLO)} 

Many experts point to multi-link operation (MLO) as the main novelty Wi-Fi\,7 brings to the table, allowing Wi-Fi devices to concurrently operate on multiple channels through a single connection\cite{CarGerGal22}. MLO comes in different implementations according to the number and mode of operation of the active radios: Enhanced Multi-link Single-radio (EMLSR), Simultaneous Transmit and Receive Enhanced Multi-link Multi-radio (STR EMLMR), Non-simultaneous Transmit and Receive EMLMR (NSTR EMLMR).

Recent studies showed that in scenarios devoid of contention, STR EMLMR---the most flexible MLO implementation---supports significantly higher traffic loads (and thus throughput) than single-link for a given delay requirement. However, under high load and contention, STR EMLMR devices frequently access multiple links, often blocking contending basic service sets (BSSs) and occasionally causing even larger delays than those experienced with legacy single-link \cite{CarGerGal22}. 
Future Wi-Fi standard amendments are envisioned to prevent these worst-case events.

\subsection{320\,MHz Channels and 4K-QAM Modulation} 
These two enhancements are respectively achieved by duplicating the 160\,MHz tone plan of Wi-Fi\,6 and by adding two new modulation and coding scheme (MCS) indices. While these two upgrades jointly increase the maximum nominal rates by a factor of 2.4, wide contiguous channels of 320\,MHz are only likely to be found in the newly opened 6\,GHz band. 
Moreover, the new modulation orders require very high signal-to-noise ratios that may only be achieved in line-of-sight, close-proximity links (devoid of rich scattering and thus unsuitable for using multiple spatial streams) via beamforming, with high-quality hardware and eventually mesh-based installations.

\subsection{Multiple Resource Unit (MRU) Allocation}
To increase spectral efficiency, Wi-Fi 7 allows allocating a Multiple RU (MRU) per station (STA) consisting of a selected combination of RUs. A prime example of a scenario where such a degree of flexibility may pay off is in a BSS with a small number of users. For instance with Wi-Fi 6, an AP operating on an 80 MHz channel where the secondary 20 MHz channel is occupied was only able to assign the primary 20 MHz channel to a certain STA. Wi-Fi 7 enables the same AP to also assign the available secondary 40 MHz channel to the same STA, providing a total of 60 MHz. Such extra bandwidth can be used either to transmit faster and reduce latency, or to improve reliability by enabling more robust transmissions using lower MCSs.

\subsection{Restricted Target Wake Time (R-TWT)} 
Wi-Fi\,6 target wake time (TWT) specification \cite{nurchis2019target} aims at reducing power consumption by defining specific service periods (SP) in which a device should be awake. R-TWT builds atop this feature to define non-overlapping SPs, representing an attempt to improve support of delay-sensitive and real-time applications (RTA) through scheduled transmissions. In fact, R-TWT forces Wi-Fi\,7 STAs to end ongoing communications before the start of an advertised R-TWT SP, and it also configures a quiet interval for the entire duration of the R-TWT SP to ensure that legacy STAs remain silent.



%OLD
%As suggested by its name (IEEE 802.11be `Extremely High Throughput' or `EHT'), Wi-Fi\,7 will augment data rates to at least 30\,Gbps per AP, about four times as fast as Wi-Fi\,6. 
%%With the standardization process entering the final phase and the correspondent certification activity moving already the first steps, 
%In the sequel, we summarize the main features introduced in the soon-to-appear Wi-Fi\,7 commercial products \cite{garcia2021ieee,CheCheDas22}. 

%\bcom{We are citing 11be Draft 1.5. Is 4.0 the last one?}
% Gio: Removed \cite{80211beDraft}.fr

%%%%%%%%%%%%%%%%%%%%%%%%%%%%%%%%%%%%%%%%%%%%%%%%

%\subsection{Multi-link Operation (MLO)} 

%Many experts point to multi-link operation (MLO) as the main novelty Wi-Fi\,7 brings to the table, allowing Wi-Fi devices to concurrently operate on multiple channels through a single connection \cite{CarGerKniICC2022,CarGerGal22,lopez2022multi}. MLO comes in different implementations, with the main ones summarized as follows. 

%\subsubsection*{Enhanced Multi-link Single-radio (EMLSR)} 
%With a multi-link device (MLD) listening to two or more links simultaneously, %(e.g., by splitting its multiple radios),  performing clear channel assessment, and receiving a limited type of control frames. EMLSR supports opportunistic spectrum access at a reduced cost, as it requires a single fully functional 802.11be radio plus several other low-capability radios able only to decode 802.11 control frame preambles. Upon reception of an initial control frame on one of the links, the MLD can switch to the latter and operate using all antennas.

%\subsubsection*{Enhanced Multi-link Multi-radio (EMLMR)}
%Where all radios are 802.11be-compliant and allow operating on multiple links concurrently. EMLMR is further classified into: (i) Simultaneous Transmit and Receive (STR), where simultaneous uplink and downlink is allowed over a pair of links; (ii) Non-simultaneous Transmit and Receive (NSTR), where the above is not allowed so as to prevent self-interference.

%Lastly, non-`enhanced' versions of the above modes have also been defined, where MLSR (as opposed to EMLSR) lacks the capability of performing clear channel assessment and transmission/reception on multiple links, and MLMR (as opposed to EMLMR) lacks the capability to dynamically reconfigure spatial multiplexing over multiple links. 
%\bcom{So, what is MLSR doing different from single link? Is it that it enables faster link switching at 'management' level? How that link change is agreed between AP and STA pair?}
% Gio: In my understanding, in MLO MLSR this is handled under a single association. In SLO, this may require new association. We're using similar wording as Intel's, to be on the safe side.
%\bcom{Do we need to explain the non-enhanced ones?}
% Gio: If space allows, yes. It feels weird talking about "enhanced" without mentioning the baseline.

%Recent studies showed that in scenarios devoid of contention, STR EMLMR---the most flexible MLO implementation---supports significantly higher traffic loads (and thus throughput) than single-link for a given delay requirement. However, under high load and contention, STR EMLMR devices frequently access multiple links, often blocking contending basic service sets (BSSs) and occasionally causing even larger delays than those experienced with legacy single-link \cite{CarGerKniICC2022,CarGerGal22}. 

%\subsection{320\,MHz Channels and 4K-QAM Modulation} 
%These two enhancements are respectively achieved by duplicating the 160\,MHz tone plan of Wi-Fi\,6 and by adding two new modulation and coding scheme (MCS) indices. While these two upgrades jointly increase the maximum nominal rates by a factor of 2.4, wide contiguous channels of 320\,MHz are only likely to be found in the newly opened 6\,GHz band. Moreover, the new modulation orders require very high signal-to-noise ratios that may only be achieved in line-of-sight, close-proximity links (devoid of rich scattering and thus unsuitable for using multiple spatial streams) via beamforming, with high-quality hardware and eventually mesh-based installations.

%\subsection{Allocation of Multiple Resource Units (RU)} 
%Wi-Fi\,7 allows allocating multiple RU per station (STA) for an increased spectral efficiency. A prime example of a scenario where such degree of flexibility may pay off is with a small number of users. For instance with Wi-Fi\,6, an AP operating on an 80\,MHz channel where the secondary 20\,MHz channel is occupied was only able to assign the primary 20\,MHz channel to a certain STA. Wi-Fi\,7 enables the same AP to also assign the available secondary 40\,MHz channel, with a threefold gain in spectrum utilization. 

%\subsection{Restricted Target Wake Time (R-TWT)} 
%R-TWT comes to solve an inherent issue of the default Wi-Fi\,6 target wake time (TWT) specification \cite{nurchis2019target}: the possibility of encountering ongoing transmissions when a TWT service period (SP) starts, which adds uncertainty on when the scheduled transmission will be initiated. To avoid this situation, R-TWT forces Wi-Fi\,7 STAs to end ongoing transmissions before the start of an R-TWT SP advertised by the AP. Moreover, to ensure that non-Wi-Fi\,7 STAs also remain silent, an overlapping quiet interval is also scheduled for each R-TWT SP. R-TWT offers improved latency guarantees, representing a first attempt to support delay-sensitive applications. 
