\begin{figure*}
    \centering
    \includegraphics[width=0.80\textwidth]{Figures/FeaturesWiFi8.eps}
    \caption{Illustrative examples of the key features being investigated for Wi-Fi\,8.} 
   \label{fig:WiFi8Features}
\end{figure*}

\section{IEEE 802.11bn: Key Features for Wi-Fi\,8} \label{sec:WiFi8}

Wi-Fi\,8 will be the first generation aiming to improve the protocol's reliability, with a focus on service availability and delay guarantees. Three critical aspects impacting reliability in the unlicensed spectrum are being investigated: seamless connectivity, determinism, and controlled worst-case delay.
Fig.~\ref{fig:WiFi8Features} depicts examples for each, with their chief opportunities and challenges discussed in the sequel.

\subsection{\hl{Seamless Connectivity via Distributed MLO}}

%
\footnote{\hl{Add more technical details to this subsection.}}
%
Mobility support has never been a primary focus in previous Wi-Fi standards, although devices roaming between APs is a major cause of Wi-Fi link unreliability. 
The new multi-link architecture offers a high degree of flexibility, presenting a clear split between upper (multi-link level) and lower (link level) MAC functionalities, with an MLD that can be seen as an entity that controls two or more legacy APs (or STAs) each operating on a single link. 
One way to leverage this flexibility to improve mobility support in Wi-Fi\,8 is through a new \emph{distributed} MLO framework, where APs under the control of the same MLD entity can be non-co-located rather than implemented in the same physical hardware. While this approach requires coordination and communication among the different distributed APs under the same controlling multi-link instance, it creates a distributed \emph{virtual cell} where a device's mobility is seamlessly handled by ensuring multiple links are concurrently activating from different distributed APs (Fig.~2, leftmost). This approach ensures a nomadic device is connected to at least one link, de facto embedding native roaming support into MLO and significantly improving the connection's reliability.


\subsection{\hl{Determinism via PHY and MAC Enhancements}}

%
\footnote{\hl{Add more technical details to this subsection.}}
%
Wi-Fi\,8 will consider the possibility to include PHY/MAC enhancements such as hybrid automatic repeat request (HARQ) and increasing the number of supported spatial streams from 8 to 16. The use of HARQ could allow devices to combine corrupted data units with their corresponding retransmissions to increase the probability of correct decoding, reducing latency in challenging channel conditions. 
The availability of additional spatial streams could enable more users to be served simultaneously, reducing their channel access time, and provide extra degrees of freedom to mechanisms such as coordinated beamforming (discussed later in this section). 
Additionally, other features building atop TXOP sharing functionalities will potentially allow APs to share a portion of their obtained TXOPs with associated stations for transmitting uplink frames to the AP or for direct peer-to-peer communication with another station. 
The new functionalities, combined with the aforementioned R-TWT mechanisms and TXOP sharing principles, 
already promise improvements in terms of reliability. However, further enhancements may be needed to support the arrival of unexpected or event-based time-sensitive traffic during large ongoing transmissions. An interesting proposal to address this issue is the one of frame `preemption', detailed below.

\subsubsection*{Preemption by the TXOP holder} 
To implement preemption within an ongoing PPDU transmission, a new multidimensional aggregated physical layer protocol data unit (A-PPDU) design is being studied. By aggregating multiple PPDUs in both time and frequency, the TXOP holder can replace on the fly a best-effort PPDU with a time-sensitive one, interrupting the ongoing PPDU only over the necessary resource unit. For instance (see Fig.~\ref{fig:WiFi8Features}, third example), after receiving a time-sensitive packet directed to Device\,1 (URLLC), the AP interrupts the ongoing transmission to Device 2\,of PPDU\,1 (upper 80\,MHz resource unit) to transmit the new time-sensitive frame. An interesting and positive aspect of this approach is that it does not require changing the receiver design, as interrupted frames will simply be retransmitted later.

\subsubsection*{Preemption by a non-TXOP holder} 
To implement preemption when the device transmitting a time-sensitive frame is not the TXOP holder, a more challenging situation arises. In such a case, an effective preemption mechanism should aim to capture the ongoing TXOP by exploiting the temporal inter-frame spaces between transmitted PPDUs, so that the time-sensitive frame can be squeezed in at the expense of the current data exchange. Another possibility is to allocate a short guard period at the end of all scheduled TXOPs, where only devices with time-sensitive traffic will be allowed to transmit.

\subsection{Controlled Worst-Case Delay via AP Coordination}

Random access makes it difficult to provide performance guarantees, which is why Wi-Fi\,7 introduced R-TWT to reduce contention within a BSS. However, inter-BSS interactions are still governed by contention principles, even if the APs belong to the same administrative domain, making worst-case delays unpredictable. Wi-Fi\,8 is expected to address this issue by introducing multi-AP coordination to achieve greater reliability and prevent channel access contentions, especially in dense and heavily loaded environments.

To this end, new protocols and frames will be necessary for discovering and managing multi-AP groups, sharing channel and buffer state data between APs, and triggering coordinated multi-AP transmissions to minimize inter-BSS collisions and achieve a more efficient use of the spectrum through dynamic inter-AP resource management. AP coordination schemes in Wi-Fi\,8 are envisioned to leverage both over-the-air and wired signaling. These schemes will range from basic to advanced, depending on the amount of data that must be exchanged between access points and the level of implementation complexity. While it is still to be decided what aspects of the coordination mechanism will be specified by the standard and what will be left for implementation, the main schemes are expected to include the ones described in the following.

\subsubsection*{Coordinated TDMA/OFDMA} 
Two basic approaches, respectively where a TXOP is divided in slots and sequentially allocated to different APs (TDMA) and where different portions of the band are allocated to different APs (OFDMA). 

\subsubsection*{Joint transmission} 
An advanced approach, also known as \emph{distributed MIMO}, involving non-co-located APs that jointly transmit or receive data from multiple STAs. Remarkably, this approach turns neighboring APs from potential interferers to servers. However, its implementation requires designing a new distributed CSMA/CA as well as ensuring tight synchronization in time, frequency, and phase among the transmitters. Joint data processing across multiple APs also requires an out-of-band backhaul link, whether wired or wireless.

\subsubsection*{Coordinated beamforming} 
An intermediate approach where collaborative APs suppress incoming OBSS interference in the spatial domain (see Fig.~\ref{fig:WiFi8Features}, fourth example). 
With coordinated beamforming (CBF), a next-generation multi-antenna AP can use its multiple spatial degrees of freedom to multiplex its STAs and place radiation nulls to and from neighboring non-associated STAs  \cite{GerGarLop17}. This approach makes the AP and its neighboring STAs mutually \emph{invisible}, avoiding channel access contention, allowing concurrent collision-free transmissions, and improving worst-case latency as a byproduct. 
Although the specific implementation of CBF is still under discussion, it is likely to involve the following key elements: 
\begin{itemize}
    \item A control frame exchange between two or more collaborative APs to establish and maintain a coordination set. During this exchange, APs may communicate with OBSS STAs to acquire their channel state information (CSI) and configure necessary interference suppression through nulls towards a specific channel direction or subspace.
    \item A framework for dynamic and opportunistic spatial reuse, wherein a donor AP grants a transmission opportunity to an OBSS device on which the AP places a radiation null.
    \item CSI acquisition and data transmission phases, the former being necessary to design a filter for spatial multiplexing and bidirectional interference suppression, the latter to take advantage of the new spatial reuse opportunity.
\end{itemize}

While incurring a limited implementation complexity, CBF can circumvent channel contention and deliver a substantial worst-case latency reduction, as demonstrated next.