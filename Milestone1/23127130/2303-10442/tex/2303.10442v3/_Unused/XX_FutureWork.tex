\section{Research directions}

%\blue{[[Optional] \textbf{Table (Perhaps to be removed to comply with the limit. This could be discussed in each reference subsection:} Opportunities and challenges for each of the key directions]} \\

As presented in Section \ref{sec:WiFi8}, the path towards Wi-Fi 8 have already identified four distinct macro areas of work. 
%Although the standardization effort will most likely focus on defining the new required features to support them, 
In addition, we believe more fundamental research should focus on studying and understanding the benefits that forthcoming Wi-Fi upgrades may bring in terms of coexistence and performance, thinking about possible existing trade-offs and novel mechanisms to boost the required KPIs.   

Among the most interesting ones, we highlight:
\begin{itemize}
    \item \textit{Throughput vs latency trade-off}: given the well-know principle that nothing comes from nothing, theoretical works able to provide an indication of the maximum achievable throughput that can be reached when to guarantee a certain level of latency and reliability in the network are required. This will help to better design the new features, understand their applicability and benefits in the different deployment scenarios together with their impact over legacy Wi-Fi devices. 
    %
    \item \textit{Smart resource allocation mechanisms}: the degrees of freedom when deciding how to transmit data has increased significantly over the years. Flexibility in the assignment of RUs, carrier frequencies, control information, links and associated mode of operation, spatial domain radiation beams, number of nulls and its spatial directions, are demanding for the implementation of more sophisticated resource allocation strategies that take into account the dynamic needs of a variety of different applications \bb{and are able to satisfy extremely tight UHR requirements}.
    %
    \item \textit{AI/ML methods and associated architecture}: with the complexity of Wi-Fi system constantly growing, together with the need to support more stringent requirements, there is the need for a increased use of AI/ML techniques and solutions. Being currently already in use in the form of proprietary implementations, typically working among devices from the same vendor, we are heading now towards the need of a more structured and harmonized application of these techniques, the definition of a more clear architecture, with standardized interfaces, and the possibility to enable the access to a wider range of data. In this context, how AI/ML techniques will then exploit the full potential of Wi-Fi 8 is of paramount importance.  
\end{itemize}


%The use of AI/ML techniques embedded to provide better operation and improved environment understanding. AI/ML can be also used to identify traffic, acting accordingly once identified.



%    \item \textit{Capacity vs Latency trade-off}: Theoretical works able to provide determine the minimum capacity that has to be sacrified to provide latency guarantees are required to guide the design of specific mechanisms.
%    \item \textit{Practical mechanisms leveraging MLO}: to maximize spectrum access opportunities over multiple channels and bands.  




%%%%%%%%%%%%%%%%%%%%%%%%%%%%%%%%%%%%%%%%%%%%%%%%%%%%%%%%%%%%%

\begin{comment}
We conclude listing the obvious ones: performance evaluation, simulation development, tests, etc. to better understand how these features perfom both individually and in combination with others.


\bcom{[Thinking] What are the open challenges to make MAPC, mmW and reliable PHY/MAC mech? Well, more than research, is to conceive practical ways to make them work in a efficient way, and show how to get performance improvements from them. In MAPC, well, we need to reduce the amount of overheads (data exchange) to make it work. mmW? What to say since 11ad and 11ay are there given the idea is to make even simpler? All is known there, the big challenge is how to use that 'fast' link when available.... Reliable mechanism may imply the redefinition of the TXOP 'structure'. For instance, leaving 'holes' where stations/other APs can place a planned transmission (instead of not transmitting as in rTWT, start transmitting but leave room for the planned tx by 'sharing' the TXOP). In this case, agreement could be still in place, and so the goal is how to plan the 'holes'. }

\vspace*{2cm}

\noindent \bcom{Some general thoughts, not (exclusively) following the previous 3 directions (which makes sense to follow). Obvious: to go beyond what will developed in the TG}
\begin{enumerate}
    \item MAPC group schedulers (MAPC)
    %
    \item mmWave: ?
    %
    \item LL Frame insertion in on-going transmission:
    %
    \item Interplay between different features
    %
    \item Understanding the performance gains of the different features, specially in terms of latency and reliability, and their 'capacity' regions
    %
    \item \textbf{Integration of AI/ML}
    %
    \item New technologies as NOMA, in-band full-duplex, etc.
    %
    \item Development of analysis and simulation tools (NS3 is clearly not going to fill that hole).
    %
    \item Coexistence in unlicensed bands
\end{enumerate}

Due to the operation in unlicensed bands to guarantee a high level of reliability (i.e., 99.000) we need to over-provisioning resources. To do that we need a) technologies able to support over-provisioning (i.e., MLO) and b) a function that given the current situation estimates as much as accurately as possible how many resources should be allocated (i.e., number of links). With MAPC the amount of over-provisioning could be reduced, but it is still required.
\end{comment}