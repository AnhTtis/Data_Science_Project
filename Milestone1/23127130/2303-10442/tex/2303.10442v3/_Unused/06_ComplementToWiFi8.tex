\section{Complementary Research Direction: Integrated mmWave Operations}

To ensure the long-term evolution of Wi-Fi, next-generation devices could be operating in all three sub-7\,GHz bands as well as in the mmWave realm. 
Indeed, there is a growing interest in capitalizing on the 5.5 and 14\,GHz of spectrum available in the 45 and 60\,GHz bands, respectively. 
While mmWave bands offer plenty of additional spectrum to offload traffic---and as a by-product reduce latency and increase reliability---they also present significant challenges. 
Rapid signal attenuation, sensitivity to blockages, %(see Fig.~\ref{fig:WiFi8Features}, second example), 
antenna beam management requirements, and high power consumption have limited the commercial adoption of products like WiGig. 
% OLD In the following, we discuss three key aspects currently debated for the 802.11bn standardization.
%Lorenzo NEW: 
In the following, we discuss three key aspects for debate in the IMW SG.

\subsubsection*{Integrated vs. independent PHY design}
%OLD: One of the main points of discussion regarding Wi-Fi\,8 is whether it will adopt a physical layer (PHY) similar to the one used in Wi-Fi\,7 for operation in mmWave bands, or leverage the one currently used in IEEE 802.11ad/ay. 
% Gio NEW. Same content but more concise to save words
One important discussion point is whether to adopt a physical layer (PHY) like Wi-Fi 7 or enhance the one in IEEE 802.11ad/ay. An integrated PHY design would eliminate the need for wide channels (2.16 GHz to 8.64 GHz) used in WiGig, reducing costs and power usage. However, this introduces challenges in handling carrier offset and phase noise at mmWave frequencies. Synchronization errors increase with carrier frequency, impairing OFDM-based systems with small subcarrier spacing. To address this, bandwidth and subcarrier spacing numerology need revisiting, with flexible subcarrier spacing scaling and appropriate FFT size adjustments to avoid processing complexity growth for wider bandwidths, similar to 5G.
%Lorenzo NEW 
%One of the main points of discussion will be whether to adopt a physical layer (PHY) similar to the one used in Wi-Fi\,7, or to enhance the one currently used in IEEE 802.11ad/ay. 
%Adopting an integrated PHY design would have a positive impact on hardware, as it would eliminate the need for the extremely wide channels (from 2.16\,GHz up to 8.64\,GHz) used by WiGig products, thus avoiding expensive and power-hungry amplifiers. However, this would also introduce new challenges. 
%When operating at mmWave frequencies, properly handling carrier frequency offset and phase noise caused by fluctuations in local oscillators becomes crucial. This is because synchronization errors increase linearly with carrier frequency, leading to significant impairments in OFDM-based systems with small subcarrier spacing. To address this issue, bandwidth and subcarrier spacing numerology need to be revisited, with the latter being flexible to upscale with channel width, similar to what is applied in 5G. Additionally, the FFT size should scale accordingly to avoid exponential increases in processing complexity for wider bandwidths.

\subsubsection*{Practical throughput gains}
% Gio revised to save words
The losses from rapid signal attenuation or blockage may offset the benefits of wider bands. Operating at mmWave can lead to a significant reduction in SINR and MCS compared to sub-7 GHz operations, resulting in a logarithmic loss in achievable rates depending on the transmitter-receiver distance. At mmWave, antenna arrays are typically used for beamforming to counteract path loss, while sub-7\,GHz bands leverage multiple antennas for parallel transmission streams, linearly increasing throughput. To determine the cost-effectiveness of mmWave operations, it is crucial to quantify realistic data rates at 60\,GHz compared to existing operations in the 6\,GHz band.
%The losses due to a more rapid signal attenuation (or worse, blockage) may offset the benefits introduced by the adoption of wider bands. Depending on the distance between transmitter and receiver, operating at mmWave may incur a drastic reduction of the SINR and of the MCS employed with respect to sub-7\,GHz operations, triggering a logarithmic loss in achievable rates. 
%Additionally at mmWave, antenna arrays must typically be used to beamform and boost signals to counteract the severe path loss, whereas in sub-7\,GHz bands multiple antennas are leveraged to generate multiple parallel transmission streams, linearly increasing the throughput. 
%To assess the cost-effectiveness of mmWave operations it is crucial to quantify realistic achievable data rates at 60\,GHz compared to existing operations in the 6\,GHz band. 

\subsubsection*{Building atop the Wi-Fi\,7 multi-link framework} 
In Section~\ref{sec:WiFi7}, we discussed MLO as the key feature of Wi-Fi\,7, enabling multiple links to operate jointly through a single association in 2.4, 5, and 6\,GHz bands. The current discussion revolves around utilizing the multi-link framework as a mechanism to integrate sub-7\,GHz and mmWave channels, offering several advantages. These include dynamically activating mmWave links when propagation conditions are favorable, and leveraging sub-7\,GHz bands to exchange control information and handle normal operations. % (see Fig. \ref{fig:WiFi8Features}, second example).
