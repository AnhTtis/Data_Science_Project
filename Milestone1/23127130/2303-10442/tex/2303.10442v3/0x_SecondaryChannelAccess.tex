\section{\blue{Secondary Channel Access (SCA)}}

\bcom{References left as text, no added to the bib. To discuss if there is room for any.}

\textcolor{blue}{The integration of wide channels, particularly the 160 and 320 MHz variants, within Wi-Fi 8 holds substantial potential for enhancing Wi-Fi capacity. While dynamic channel access mechanisms, such as preamble puncturing, adeptly manage the available spectrum in secondary channels, when the primary channel experiences congestion, all transmission processes are unavoidably delayed, irrespective of the status of the secondary channels. In response to this challenge, Wi-Fi 8 is exploring the concept of supporting Secondary Channel Access (SCA).
%
The conceived operation of Secondary Channel Access (SCA) unfolds as follows: Initially, an Access Point (AP) and its associated stations reach a consensus to utilize an auxiliary 'primary' channel, situated within one of the 20 MHz secondary channels. This auxiliary channel serves as an alternative when the primary channel becomes occupied due to transmissions from overlapping Basic Service Sets (OBSS). In such scenarios, the AP and stations detecting OBSS activity on the primary channel seamlessly transition to monitor the auxiliary primary channel. Once it is determined to be idle, they engage in standard channel contention procedures, followed by an RTS/CTS or BSRP/BRP exchange to secure the channel for data transmission. An important constraint to adhere to is that the secondary Transmission Opportunity (TXOP) duration must conclude before the initiation of the transmission that occupies the primary channel. This allows all devices within the Basic Service Set (BSS) to return to normal operations once the primary channel becomes available again. While this approach may appear intuitive, it does challenge the conventional requirement mandating the inclusion of the primary channel in all transmissions and receptions.
%
The introduction of SCA is anticipated to yield performance gains without imposing significant complexity. The extent of these gains in terms of both throughput and latency depends on the specific scenario and OBSS configuration. Preliminary studies indicate potential gains ranging from 50\% to 250\%. Importantly, these studies also demonstrate that non-SCA BSSs do not experience any adverse performance effects. While the incorporation of SCA into Wi-Fi 8 may not represent a revolutionary leap, it does hold the promise of increasing Wi-Fi performance levels at minimal cost by creating additional channel access opportunities. Nevertheless, further research is necessary to fully understand how SCA interacts with channel allocation strategies and dynamic channel access protocols, as its presence could encourage more aggressive overlapping between OBSSs. Finally, we advise to initially limit SCA to applications involving low-latency and high-priority traffic. Such traffic typically comprises sporadic and limited data volumes, making it well-suited to harness the advantages offered by secondary channel access opportunities.}

\begin{comment}
Secondary Channel Access
============================================ 

- Non-Primary Channel Utilization Follow-up
-- "a busy 20MHz primary channel prevents a STA from accessing an idle 300 MHz of remaining bandwidth"
-- Motivation: to use the available bandwidth that otherwise is wasted always the primary channel is busy.

- Thoughts on Secondary Channel Access
-- How to leverage the use of secondary channels when the primary is busy
-- Primary channel is busy from OBSS tx.
-- Contention on secondary + TXOP shorter than the end of the NAV timer on the primary.
-- Is it equivalent to have a secondary 'primary' (this is what they propose).
-- They propose HiP EDCA

- Performance evaluation of non-primary channel access
-- The authors show some gains of NPC in different scenarios.
-- They key is: "DUT UHR AP and STA pre-negotiates an auxiliary non-primary channel to temporarily switch to when they detect medium busy on a primary channel. "

- UHR Secondary Channel Access
-- Motivation regarding the waste of spectrum when using 160 or 320 MHz channels.
-- Protocol: switch, control + TXOP < NAV

- Non-Primary Channel Access
-- Auxiliary Primary Channel
-- The same as the others


\end{comment}