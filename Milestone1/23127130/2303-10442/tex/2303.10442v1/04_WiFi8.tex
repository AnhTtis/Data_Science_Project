\begin{figure*}
    \centering
    \includegraphics[width=0.99\textwidth]{Figures/FeaturesWiFi8.eps}
    \caption{Illustrative examples of the key features being investigated for Wi-Fi\,8.} 
   \label{fig:WiFi8Features}
\end{figure*}

\section{IEEE 802.11bn: Key Features for Wi-Fi\,8} \label{sec:WiFi8}

Wi-Fi\,8 will be the first generation aiming to improve the protocol's reliability, with a focus on service availability and delay guarantees. Four critical aspects impacting reliability in the unlicensed spectrum are being investigated: seamless connectivity, abundant spectrum, determinism, and controlled worst-case delay. Fig.~\ref{fig:WiFi8Features} depicts examples for each, with their chief opportunities and challenges discussed in the sequel.

\subsection{Seamless Connectivity via Distributed MLO}

Although mobility support has not been a primary focus in previous Wi-Fi standards, devices roaming between APs is a major cause of Wi-Fi link unreliability. 
The new multi-link architecture offers a high degree of flexibility, which can improve mobility in Wi-Fi\,8. One way to leverage this flexibility is through a new \emph{distributed} MLO framework, where logical APs under the control of the same MLD entity can be non-co-located rather than implemented in the same physical device. While this approach requires coordination and communication among the different distributed APs under the same multi-link instance, it creates a distributed \emph{virtual cell} where a device's mobility is handled by activating multiple links from different distributed APs (Fig.~\ref{fig:WiFi8Features}, leftmost). This approach ensures a nomadic device is seamlessly connected to at least one link, embedding native roaming support into MLO and significantly improving the connection's reliability.

\subsection{Abundant Spectrum via Integrated mmWave Operations}

To ensure the long-term evolution of Wi-Fi, next-generation devices could be operating in all three sub-7\,GHz bands as well as in the mmWave realm. 
Indeed, there is a growing interest in capitalizing on the 5.5 and 14\,GHz of spectrum available in the 45 and 60\,GHz bands, respectively. 
While mmWave bands offer plenty of additional spectrum to offload traffic---and as a by-product reduce latency and increase reliability---they also present significant challenges. 
Rapid signal attenuation, sensitivity to blockages (see Fig.~\ref{fig:WiFi8Features}, second example), antenna beam management requirements, and high power consumption have limited the commercial adoption of products like WiGig. 
% OLD In the following, we discuss three key aspects currently debated for the 802.11bn standardization.
%Lorenzo NEW: 
{In the following, we discuss three key aspects for debate in the IMW SG.}

\subsubsection*{Integrated vs. independent PHY design}
%OLD: One of the main points of discussion regarding Wi-Fi\,8 is whether it will adopt a physical layer (PHY) similar to the one used in Wi-Fi\,7 for operation in mmWave bands, or leverage the one currently used in IEEE 802.11ad/ay. 
%Lorenzo NEW 
{One of the main points of discussion will be whether to adopt a physical layer (PHY) similar to the one used in Wi-Fi\,7, or to enhance the one currently used in IEEE 802.11ad/ay.} 
Adopting an integrated PHY design would have a positive impact on hardware, as it would eliminate the need for the extremely wide channels (from 2.16\,GHz up to 8.64\,GHz) used by WiGig products, thus avoiding expensive and power-hungry amplifiers. However, this would also introduce new challenges. 
When operating at mmWave frequencies, properly handling carrier frequency offset and phase noise caused by fluctuations in local oscillators becomes crucial. This is because synchronization errors increase linearly with carrier frequency, leading to significant impairments in OFDM-based systems with small subcarrier spacing. To address this issue, bandwidth and subcarrier spacing numerology need to be revisited, with the latter being flexible to upscale with channel width, similar to what is applied in 5G. Additionally, the FFT size should scale accordingly to avoid exponential increases in processing complexity for wider bandwidths.

\subsubsection*{Practical throughput gains}
The losses due to a more rapid signal attenuation (or worse, blockage) may offset the benefits introduced by the adoption of wider bands. Depending on the distance between transmitter and receiver, operating at mmWave may incur a drastic reduction of the SINR and of the MCS employed with respect to sub-7\,GHz operations, triggering a logarithmic loss in achievable rates. 
Additionally at mmWave, antenna arrays must typically be used to beamform and boost signals to counteract the severe path loss, whereas in sub-7\,GHz bands multiple antennas are leveraged to generate multiple parallel transmission streams, linearly increasing the throughput. 
To assess the cost-effectiveness of mmWave operations it is crucial to quantify realistic achievable data rates at 60\,GHz compared to existing operations in the 6\,GHz band. 

\subsubsection*{Building atop the Wi-Fi\,7 multi-link framework} 
In Section \ref{sec:WiFi7}, we discussed MLO as the key feature of Wi-Fi\,7, enabling multiple links to operate jointly through a single association in 2.4, 5, and 6\,GHz bands. The current discussion revolves around utilizing the multi-link framework as a mechanism to integrate sub-7\,GHz and mmWave channels, offering several advantages. These include dynamically activating mmWave links when propagation conditions are favorable, and leveraging sub-7\,GHz bands to exchange control information and handle normal operations (see Fig. \ref{fig:WiFi8Features}, second example).

\subsection{Determinism via PHY and MAC Enhancements}

Wi-Fi\,8 will incorporate PHY/MAC enhancements such as hybrid automatic repeat request (HARQ) and increasing the number of supported spatial streams from 8 to 16. The use of HARQ will allow devices to combine corrupted data units with their corresponding retransmissions to increase the probability of correct decoding, reducing latency in challenging channel conditions. 
The availability of additional spatial streams will enable more users to be served simultaneously, reducing their channel access time, and provide extra degrees of freedom to mechanisms such as coordinated beamforming (discussed later in this section). 
Additionally, other features building atop TXOP sharing functionalities will allow APs to share a portion of their obtained TXOPs with associated stations for transmitting uplink frames to the AP or for direct peer-to-peer communication with another station. 
The new features, combined with the aforementioned R-TWT mechanisms and TXOP sharing principles, 
already promise improvements in terms of reliability. However, further enhancements may be needed to support the arrival of unexpected or event-based time-sensitive traffic during large ongoing transmissions. An interesting proposal to address this issue is the one of frame `preemption', detailed below.

\subsubsection*{Preemption by the TXOP holder} 
To implement preemption within an ongoing PPDU transmission, a new multidimensional aggregated physical layer protocol data unit (A-PPDU) design is being studied. By aggregating multiple PPDUs in both time and frequency, the TXOP holder can replace on the fly a best-effort PPDU with a time-sensitive one, interrupting the ongoing PPDU only over the necessary resource unit. For instance (see Fig.~\ref{fig:WiFi8Features}, third example), after receiving a time-sensitive packet directed to Device\,1 (URLLC), the AP interrupts the ongoing transmission to Device 2\,of PPDU\,1 (upper 80\,MHz resource unit) to transmit the new time-sensitive frame. An interesting and positive aspect of this approach is that it does not require changing the receiver design, as interrupted frames will simply be retransmitted later.

\subsubsection*{Preemption by a non-TXOP holder} 
To implement preemption when the device transmitting a time-sensitive frame is not the TXOP holder, a more challenging situation arises. In such a case, an effective preemption mechanism should aim to capture the ongoing TXOP by exploiting the temporal inter-frame spaces between transmitted PPDUs, so that the time-sensitive frame can be squeezed in at the expense of the current data exchange. Another possibility is to allocate a short guard period at the end of all scheduled TXOPs, where only devices with time-sensitive traffic will be allowed to transmit.

\subsection{Controlled Worst-Case Delay via AP Coordination}

Random access makes it difficult to provide performance guarantees, which is why Wi-Fi\,7 introduced R-TWT to reduce contention within a BSS. However, inter-BSS interactions are still governed by contention principles, even if the APs belong to the same administrative domain, making worst-case delays unpredicatable. Wi-Fi\,8 is expected to address this issue by introducing multi-AP coordination to achieve greater reliability and prevent channel access contentions, especially in dense and heavily loaded environments.

To this end, new protocols and frames will be necessary for discovering and managing multi-AP groups, sharing channel and buffer state data between APs, and triggering coordinated multi-AP transmissions to minimize inter-BSS collisions and achieve a more efficient use of the spectrum through dynamic inter-AP resource management. AP coordination schemes in Wi-Fi\,8 will range from basic to advanced, depending on the amount of data that must be exchanged between access points and the level of implementation complexity, and may include the ones described in the following.

\subsubsection*{Coordinated TDMA/OFDMA} 
Two basic approaches, respectively where a TXOP is divided in slots and sequentially allocated to different APs (TDMA) and where different portions of the band are allocated to different APs (OFDMA). 

\subsubsection*{Joint transmission} 
An advanced approach, also known as \emph{distributed MIMO}, involving non-co-located APs that jointly transmit or receive data from multiple STAs. Remarkably, this approach turns neighboring APs from potential interferers to servers. However, its implementation requires designing a new distributed CSMA/CA as well as ensuring tight synchronization in time, frequency, and phase among the transmitters. Joint data processing across multiple APs also requires an out-of-band backhaul link, whether wired or wireless.

\subsubsection*{Coordinated beamforming} 
An intermediate approach where collaborative APs suppress incoming OBSS interference in the spatial domain (see Fig.~\ref{fig:WiFi8Features}, fourth example). 
With coordinated beamforming (CBF), a next-generation multi-antenna AP can use its multiple spatial degrees of freedom to multiplex its STAs and place radiation nulls to and from neighboring non-associated STAs  \cite{GerGarLop17}. This approach makes the AP and its neighboring STAs mutually \emph{invisible}, avoiding channel access contention, allowing concurrent collision-free transmissions, and improving worst-case latency as a byproduct. 
Although the specific implementation of CBF is still under discussion, it is likely to involve the following key elements: 
\begin{itemize}
    \item A control frame exchange between two or more collaborative APs to establish and maintain a coordination set. During this exchange, APs may communicate with OBSS STAs to acquire their channel state information (CSI) and configure necessary interference suppression through nulls towards a specific channel direction or subspace.
    \item A framework for dynamic and opportunistic spatial reuse, wherein a donor AP grants a transmission opportunity to an OBSS device on which the AP places a radiation null.
    \item CSI acquisition and data transmission phases, the former being necessary to design a filter for spatial multiplexing and bidirectional interference suppression, the latter to take advantage of the new spatial reuse opportunity.
\end{itemize}

While incurring a limited implementation complexity, CBF can circumvent channel contention and deliver a substantial worst-case latency reduction, as demonstrated next.