\section{Introduction} \label{sec:Intro}

You do not need to be tech-savvy to know Wi-Fi. With twice as many devices as people, Wi-Fi technologies carry two thirds of the world’s mobile traffic and underpin our digital economy. This generation will not easily forget what it could have meant to undergo Covid lockdown without Wi-Fi from social, economic, and safety standpoints. And even now that traveling to places is possible once again, many of us reach for the Wi-Fi password first thing upon arrival, as this is often the means to ordering a meal and sending news back home.

Wi-Fi has come a long way since its introduction in the late nineties. The easiest way to appreciate the technology’s improvement is by reading peak data rates specifications on commercial Wi-Fi access point (AP) boxes. These rates have grown roughly four orders of magnitudes in two and a half decades, from the mere 1\,Mbps of the original 802.11 standard to the near 30\,Gbps of the latest 802.11be products (alias Wi-Fi\,7) scheduled to hit the shelves as early as 2024 \cite{lopez2019ieee,garcia2021ieee,khorov2020current,deng2020ieee,yang2020survey,CheCheDas22}. This giant leap allowed Wi-Fi to move beyond email and web browsing and progressively conquer crowded co-working spaces, airports, and even the hearts of many parents who can now video-call their children without worrying about phone bills. But how many of us have complained at least once about Wi-Fi not functioning when we most need it? Unreliability would be the Achilles heel for any technology meant to be affordable, pervasive, and operating in license-exempt bands subject to uncontrolled interference. Wi-Fi is no exception.

And while it only takes patience to cope with a buffering video or to repeat our last sentence in a voice call, a lack of Wi-Fi reliability will not be tolerated by its new users: machines. 
In future manufacturing environments, Gbps communications between robots, sensors, and industrial machinery will demand reliability---with at least three (but sometimes many more) ‘nines’---in terms of both data delivery and maximum latency. Rest assured that these requirements will not get any looser for use cases involving humans. Many of us may not even want to think about undergoing robotic-assisted surgery with an unreliable Wi-Fi connection. But even just for holographic communications, a key building block of the upcoming Metaverse, excessive delays experienced by just 0.01\% of the packets could trigger nausea and user distress. As it takes up ever more challenging endeavors to fuel industrial automation, digital twinning, and tele-presence, next-generation Wi-Fi is bound to step out of its comfort zone and set reliability as its first priority~\cite{UHRProposedPAR,ResCor22}.

In this paper, we embark on a journey towards 802.11bn Ultra High Reliability (UHR), the amendment that will form the basis of Wi-Fi\,8. After providing an overview of the nearly completed Wi-Fi\,7 standard, we present emerging applications that are driving a further Wi-Fi evolution. We then review the current activities in terms of standardization, certification, and spectrum allocation, and provide fresh updates from the newly formed UHR Study Group. As the research community shifts gears to target new use cases and requirements, we introduce the envisioned new features that Wi-Fi\,8 will bring about, along with their associated research challenges. Among these features, we highlight the multi-AP coordination framework as a game changer for Wi-Fi, boosting spectrum utilization efficiency and closing in on performance determinism. We also present novel results demonstrating how such disruptive enhancements could build upon 802.11be multi-link operation to maximize their impact, making Wi-Fi\,8---and its ultra-reliability ambitions---a reality.
