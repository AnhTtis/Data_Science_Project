\subsection{Emerging Applications and Use cases} \label{sec:usecases}

New Wi-Fi capabilities could lead to a vast number of new applications and services. The key use cases in 2030 and beyond for indoor connectivity in unlicensed bands are foreseen to include the following \cite{UHRProposedPAR,HexaD13}. 

\noindent\emph{Immersive communications}: Moving from augmented/virtual reality (AR/VR) glasses to holographic telepresence. 

\noindent\emph{Digital twins for manufacturing}: Establishing a virtual connection between a digital representation of a complex system or environment and its real-world counterpart.  

\noindent\emph{e-Health for all}: Providing remote medical surgery in areas where doctors and infrastructure are lacking. 

\noindent\emph{Cooperative mobile robots}: Requiring deterministic communication for handling critical motion control information. 

\noindent Table~\ref{tab:UseCases} quantifies the performance requirements for the above use cases. To meet these ultra-reliable low-latency communication (URLLC) requirements, Wi-Fi is considering a paradigm shift towards introducing more performance determinism. This is not an easy task, 
since medium access control (MAC) was originally designed upon carrier sense multiple access with collision avoidance (CSMA/CA) to cope with uncoordinated usage in the unlicensed spectrum, rather than prioritize determinism. Evolving from this legacy, Wi-Fi 8 will pursue determinism through coordination and a more efficient use of current and additional spectrum.
%OLD given that the Wi-Fi channel access mechanism was not originally designed for this purpose. Wi-Fi 8 will pursue determinism through coordination and a more efficient use of current and additional spectrum.

