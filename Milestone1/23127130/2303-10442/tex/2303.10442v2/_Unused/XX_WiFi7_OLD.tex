\section{Wi-Fi 7 functionalities: the new state of art \\\textbf{[Gio/Boris $\sim$ 1000 words]}} \label{sec:WiFi7}
%\section{Wi-Fi 7 features supporting EHT targets\\\textbf{[Gio/Boris $\sim$ 1000 words]}} \label{sec:WiFi7}

With the IEEE 802.11be standardization process entering the final phase and the correspondent certification activity moving already the first steps, it's now the perfect time to summarize the main functionalities of the soon-to-appear Wi-Fi 7 products. 

%%%%%%%%%%%%%%%%%%%%%%%%%%%%%%%%%%%%%%%%%%%%%%%%
\subsection{Key features \textbf{[Gio]}} \label{subsec:WiFi7_KeyFeatures}

\begin{comment}
\begin{itemize}
    \item 320 MHz
    \item Multiple RUs per STA 
    \item 4K QAM
    \item Multi-Link
    \begin{itemize}
    \item Explain the novelty and why MLO will pose the basis for future generations of Wi-Fi. 
    \item Different implementations of MLO in a nutshell. Not too many details, refer to \cite{CheCheDas22} and also to our MLO magazine paper.
\end{itemize}
\end{itemize}
\end{comment}
%%%%%%%%%%%%%%%%%%%%%%%%%%%%%%%%%%%%%%%%%%%%%%%%

As suggested by its name (IEEE 802.11be `Extremely High Throughput' or `EHT'), Wi-Fi 7 will pursue augmenting data rates---at least 30~Gbps per AP, about four times as fast as Wi-Fi 6. 
Different features will contribute to achieve this goal.  
%---by upgrading the maximum channel bandwidth and highest modulation order. The former will be increased to 320~MHz from the 160~MHz of 802.11ax, whereas the latter will go up to 4096-QAM from the 1024-QAM of 802.11ax. 
\subsubsection*{320 MHz and 4K-QAM} These two upgrades will respectively be achieved by using a duplicated 160 MHz tone plan based on Wi-Fi 6 and by adding two new MCS indices, MCS 12 and MCS 13. It should be noted that while these two upgrades increase the maximum nominal rates by a factor of 2.4, wide contiguous channels of 320 MHz are only likely to be found in the newly opened 6 GHz band. Moreover, the new modulation orders require very high signal-to-noise ratios that may only be achieved in line-of-sight, close-proximity links (devoid of rich scattering and thus unsuitable for using multiple spatial streams) with good-quality hardware.
But Wi-Fi 7 will not just be about peak rates and will incorporate important features towards a more efficient use of the available radio resources. 
\subsubsection*{Allocation of multiple resource units} While in Wi-Fi 6 each STA could only be assigned a single resource unit (RU), i.e., group of OFDMA tones, Wi-Fi 7 will allow allocating multiple RUs per STA for an increased flexibility and spectral efficiency. A prime example of a scenario where such degree of flexibility may pay off is with a small number of users. For instance, with Wi-Fi 6, an AP operating on an 80 MHz channel where the secondary 20 MHz channel is occupied will only be able to assign the primary 20 MHz channel to a certain STA. Wi-Fi 7 will enable the same AP to also assign the available secondary 40 MHz channel, with a threefold gain in spectrum utilization. 

The importance of the above features notwithstanding, many experts point to multi-link operation (MLO) as the true paradigm shift Wi-Fi 7 will bring to the table \cite{CheCheDas22,CarGalJon22}. 

\subsubsection*{Multi-link operation} MLO will allow Wi-Fi devices to concurrently operate on multiple channels through a single connection. MLO will come in different implementations, with the main ones summarized as follows.
\begin{itemize}
    \item Enhanced Multi-link Single-radio (EMLSR), with a multi-link device (MLD) listening to two or more links simultaneously (e.g., by splitting their multiple antennas), performing clear channel assessment, and receiving a limited type of control frames. EMLSR supports opportunistic spectrum access at a reduced cost, as it requires a single fully functional 802.11be radio plus several other low-capability radios able only to decode 802.11 control frame preambles. Upon reception of an initial control frame on one of the links, the MLD can switch to the latter and operate using all antennas.
    \item Enhanced Multi-link Multi-radio (EMLMR), where all radios are 802.11be-compliant and allow operating on multiple links concurrently. EMLMR is further classified into: (i) Simultaneous Transmit and Receive (STR), where simultaneous uplink and downlink is allowed over a pair of links; (ii) Non-simultaneous Transmit and Receive (NSTR), where the above is not allowed so as to prevent self-interference.
\end{itemize}
Lastly, non-`enhanced' versions of the above modes have also been defined, where MLSR (as opposed to EMLSR) lacks the capability of performing clear channel assessment and transmission/reception on multiple links, and MLMR (as opposed to EMLMR) lacks the capability to dynamically reconfigure spatial multiplexing over multiple links. 
Recent studies showed that in scenarios devoid of contention, STR EMLMR---the most flexible MLO implementation---supports significantly higher traffic loads (and therefore throughput) than single-link for a given delay requirements. However, in the presence of high load and contention, STR EMLMR devices frequently access multiple links, thereby blocking contending BSSs and occasionally causing even larger delays than those experienced with a legacy SL operation \cite{CarGalJon22}.


%%%%%%%%%%%%%%%%%%%%%%%%%%%%%%%%%%%%%%%%%%%%%%%%
\subsection{Additional potential features \textbf{[Boris]}}

In addition to the features introduced in the previous section, IEEE 802.11be will also include some other relevant features, such as Restricted Target Wake Time (R-TWT) and TXOP sharing, both targeting an initial support for low-latency and deterministic communication, while improving the utilisation of the channel resources.

R-TWT comes to solve an inherent issue of the default TWT specification \cite{nurchis2019target} that prevents it from offering truly deterministic and low-latency guarantees: the possibility of encountering on-going transmissions when a TWT SP starts, which adds an uncertain delay on when the scheduled transmission will be initiated. To avoid this situation, R-TWT forces EHT stations to ensure any on-going transmission will end before the start time of a R-TWT SP advertised by the AP. Moreover, to guarantee non-EHT stations will also remain silent during R-TWT periods, an overlapping quiet interval is also scheduled for each R-TWT SP, hence all other contending stations will defer accordingly.

The Triggered TXOP sharing feature extends EHT APs scheduling capabilities. It allows APs to share a portion of their obtained TXOPs with associated stations, either for transmitting uplink frames to the AP, or to directly communicate with another station in peer to peer mode. Then, APs can use its favourable channel access parameters and privileged view of the network to allocate temporal resources to stations when required, hence improving data exchange efficiency and reducing channel contention. 

%IEEE 802.11be also enhances other features already included in IEEE 802.11ax, such as Spatial Reuse by including PSR [to check].

%\subsection{A paradigm shift: Multi-Link Operations}
%\begin{itemize}
%    \item Explain the novlety and why MLO will pose the basis for future generations of Wi-Fi. 
%    \item Different implementations of MLO in a nutshell. Not too many details, refer to \cite{CheCheDas22} and also to our MLO magazine paper.
%\end{itemize}