\newpage
	\vspace{3mm}
	{\noindent\LARGE\textbf{Reviewer 3}}
	\vspace{3mm}
	
	\noindent\makebox[\linewidth]{\rule{\textwidth}{1pt}} 
	
	\vspace{3mm}
	
	%%%%%%%%%%%%%%%%%%%% Comment %%%%%%%%%%%%%%%%%%%%

\vspace{6mm}
\begin{enumerate}[label=\arabic*., listparindent=1em]
\item \underline{\textbf{Reviewer:}} \textit{I liked the style of how the paper is written. It is suitable for the audience with zero background in Wi-Fi and it is very easy to read. However, from a scientific/technical point of view, I see several significant flaws, which cancel the value of the paper. So they have to be corrected before the paper can be published.} 

\noindent\underline{\textbf{Authors:}} 
Thanks for appreciating our efforts in the writing of this article. In the following, we detail how we have addressed your comments one by one. 

	%%%%%%%%%%%%%%%%%%%% Comment %%%%%%%%%%%%%%%%%%%%

\vspace{4mm}
\item \underline{\textbf{Reviewer:}} \textit{The main drawback is that the paper consists of several independent parts, some of which are not even related to Wi-Fi 8.} 

\noindent\underline{\textbf{Authors:}} 
We agree that our original submission dedicated Section~II to introduce the state-of-the-art Wi-Fi 7. This had the objective of making the paper self-contained and accessible to the broad readership, including those who are less familiar with Wi-Fi current state. As suggested, we have now reduced the content related to Wi-Fi 7 to just over a column, i.e., around 7\% of the entire paper. To do so, we have removed detailed descriptions of Wi-Fi 7 features such as MLO, 320~MHz channels, 4K-QAM modulation, allocation of multiple RU, and restricted TWT. Moreover, we have referred the reader to the suggested articles for further details on these features. The revised section reads as follows.

\begin{tcolorbox}[breakable]

As suggested by its name (IEEE 802.11be `Extremely High Throughput' or `EHT'), Wi-Fi\,7 will augment data rates to at least 30\,Gbps per AP, about four times as fast as Wi-Fi\,6. 
In the sequel, we summarize the main features introduced in the soon-to-appear Wi-Fi\,7 commercial products \cite{garcia2021ieee,CheCheDas22}. 

\emph{A. Multi-link Operation (MLO)} 

Many experts point to multi-link operation (MLO) as the main novelty Wi-Fi\,7 brings to the table, allowing Wi-Fi devices to concurrently operate on multiple channels through a single connection\cite{CarGerKniICC2022,CarGerGal22,lopez2022multi}. MLO comes in different implementations {according to the number and mode of operation of the active radios: Enhanced Multi-link Single-radio (EMLSR), Simultaneous Transmit and Receive Enhanced Multi-link Multi-radio (STR EMLMR), Non-simultaneous Transmit and Receive EMLMR (NSTR EMLMR).}

Recent studies showed that in scenarios devoid of contention, STR EMLMR---the most flexible MLO implementation---supports significantly higher traffic loads (and thus throughput) than single-link for a given delay requirement. However, under high load and contention, STR EMLMR devices frequently access multiple links, often blocking contending basic service sets (BSSs) and occasionally causing even larger delays than those experienced with legacy single-link \cite{CarGerKniICC2022,CarGerGal22}. 
{Future Wi-Fi standard amendments are envisioned to prevent these worst-case events.}

\emph{B. 320\,MHz Channels and 4K-QAM Modulation}\\
These two enhancements are respectively achieved by duplicating the 160\,MHz tone plan of Wi-Fi\,6 and by adding two new modulation and coding scheme (MCS) indices. While these two upgrades jointly increase the maximum nominal rates by a factor of 2.4, wide contiguous channels of 320\,MHz are only likely to be found in the newly opened 6\,GHz band. 
Moreover, the new modulation orders require very high signal-to-noise ratios that may only be achieved in line-of-sight, close-proximity links (devoid of rich scattering and thus unsuitable for using multiple spatial streams) via beamforming, with high-quality hardware and eventually mesh-based installations.

\emph{{C. Multiple Resource Units (MRU) allocation}}\\
{To increase spectral efficiency, Wi-Fi 7 allows allocating a Multiple RU (MRU) per station (STA) consisting of a selected combination of RUs.} A prime example of a scenario where such a degree of flexibility may pay off is in a BSS with a small number of users. For instance with Wi-Fi 6, an AP operating on an 80 MHz channel where the secondary 20 MHz channel is occupied was only able to assign the primary 20 MHz channel to a certain STA. Wi-Fi 7 enables the same AP to also assign the available secondary 40 MHz channel to the same STA, {providing a total of 60 MHz. Such extra bandwidth can be used either to transmit faster and reduce latency, or to improve reliability by enabling more robust transmissions using lower MCSs.}

\emph{D. Restricted Target Wake Time (R-TWT)}\\ 
{Wi-Fi\,6 target wake time (TWT) specification \cite{nurchis2019target} aims at reducing power consumption by defining specific service periods (SP) in which a device should be awake. R-TWT builds atop this feature to define non-overlapping SPs, representing an attempt to improve support of delay-sensitive and real-time applications (RTA) through scheduled transmissions. In fact, R-TWT forces Wi-Fi\,7 STAs to end ongoing communications before the start of an advertised R-TWT SP, and it also configures a quiet interval for the entire duration of the R-TWT SP to ensure that legacy STAs remain silent.} 

\end{tcolorbox}

\noindent In light of the above changes, the revised manuscript now almost entirely focuses on Wi-Fi 8. 

	%%%%%%%%%%%%%%%%%%%% Comment %%%%%%%%%%%%%%%%%%%%

\vspace{4mm}
\item \underline{\textbf{Reviewer:}} \textit{What are the contribution and the novelty of the paper? What can I get from the paper?
The authors shall clearly define what new ideas they introduce with respect to: E. Reshef and C. Cordeiro, ``Future directions for Wi-Fi 8 and beyond," IEEE Communications Magazine, pp. 1–7, 2022.} 

\noindent\underline{\textbf{Authors:}} 
We agree on the importance of highlighting our key contribution with respect to \cite{ResCor22}, a very relevant article that we have cited. Our contribution lies in the following: 
\begin{enumerate}
    \item An up-to-date summary of the drivers and features envisioned for Wi-Fi 8 after the issue of the 802.11bn PAR. This includes emerging use cases, along with reliability, latency and data rate requirements (Section~III, Figure~1, and Table~I). Ours is also the first article revealing the naming (IEEE 802.11bn) associated to the new Wi-Fi 8 standard amendment and presenting a clear timeline for its development (Fig.~1).
    \item A technical description of some of these key new features for 802.11bn. These include detailed insights gathered from industrial contributions currently under discussion in the UHR Study Group, as well as new ideas and approaches never discussed previously in the literature, such as distributed multi-link operations (Section~IV).
    \item Novel system-level studies demonstrating how Wi-Fi 8 could achieve ultra high reliability through a joint inter-working (Section V) of multi-link operation (a Wi-Fi 7 feature) and spatial-domain AP coordination (a foreseen Wi-Fi 8 feature). We believe our early results will act as a catalyst for further research in this area.
    \item A fresh overview of complementary extensions to Wi-Fi 8 products, represented by the clear description of opportunities and challenges associated to the definition of revised mmWave operations that led to the recent formation of the     Integrated mmWave (IMW) SG. The IMW SG will be in charge to develop a new 802.11 amendment specifying carrier frequency operation between 42.5 and 71 GHz, building atop PHY/MAC functionalities in the existing Sub 7 GHz bands and in the future IEEE 802.11bn amendment (Wi-Fi 8).   
\end{enumerate}
 
\noindent In light of the above, we hope you will agree that our article complements very well, and further extends, the original contributions in \cite{ResCor22}.

	%%%%%%%%%%%%%%%%%%%% Comment %%%%%%%%%%%%%%%%%%%%

\vspace{4mm}
\item \underline{\textbf{Reviewer:}} \textit{A significant part of the paper is devoted to the introduction of Wi-Fi 7, which has already been many times introduced in the literature. However, the way how the authors present these features does not form a clear picture. I do not understand why the authors need to introduce EMLSR, EMLMR, MLMR, and MLSR if they do not have space to provide a clear difference between these methods.} 

\noindent\underline{\textbf{Authors:}} 
We completely agree with your comment and have made a tangible effort to address it. Please refer to our reply to your comment \#2.

	%%%%%%%%%%%%%%%%%%%% Comment %%%%%%%%%%%%%%%%%%%%

\vspace{4mm}
\item \underline{\textbf{Reviewer:}} \textit{Some statements about Wi-Fi 7 shall be corrected. For example,  the authors, as many authors in the literature, claim that Wi-Fi 7 allows allocating Multiple Resource  Units (RU), in reality, the RU is single, but the size of some new RUs corresponds to the size of one big RU and one small. The motivation for this feature is not presented in the paper.} 

\noindent\underline{\textbf{Authors:}} 
Thanks for this important comment. We now realize that we were not accurately using the definition of Multiple Resource Units (MRU). We have now clarified in the text that an MRU consists of a selected combination of individual RUs. As suggested, we have also clarified what motivates MRU support in Wi-Fi 7. The revised text can be found in Section II-C (whose title has also been updated) and reads as follows.
\begin{tcolorbox}[breakable]
\emph{II-C. Multiple Resource Unit (MRU) Allocation}\\
{To increase spectral efficiency, Wi-Fi 7 allows allocating a Multiple RU (MRU) per station (STA) consisting of a selected combination of RUs.} A prime example of a scenario where such a degree of flexibility may pay off is in a BSS with a small number of users. For instance with Wi-Fi 6, an AP operating on an 80 MHz channel where the secondary 20 MHz channel is occupied was only able to assign the primary 20 MHz channel to a certain STA. Wi-Fi 7 enables the same AP to also assign the available secondary 40 MHz channel to the same STA, {providing a total of 60 MHz. Such extra bandwidth can be used either to transmit faster and reduce latency, or to improve reliability by enabling more robust transmissions using lower MCSs.}
\end{tcolorbox}

%From the D3.0 amendment: "An MRU consists of selected combinations of multiple RUs of 26-tone RU, 52-tone RU, 106-tone RU, 242-tone RU, 484-tone RU, 996-tone RU, and 2x996-tone RU. The tone indices of the various RUs for different EHT PPDU bandwidths are defined in 36.3.2.1 (Subcarriers and resource allocation in EHT PPDUs)."

	%%%%%%%%%%%%%%%%%%%% Comment %%%%%%%%%%%%%%%%%%%%

\vspace{4mm}
\item \underline{\textbf{Reviewer:}} \textit{TWT was introduced in 802.11ah. The authors say that R-TWT solves an inherent issue of the default TWT, while in the reality, TWT is designed to reduce power consumption, while R-TWT is for RTA.} 

\noindent\underline{\textbf{Authors:}} 
Thanks for this comment as it motivates us to better explain this important aspect and avoid confusion. As suggested, in the revised manuscript we clarified that TWT was initially introduced to reduce power consumption while R-TWT leverages TWT to define non-overlapping SPs, thus aiming at introducing non-overlapping scheduled transmission slots for improved support to delay sensitive and real time applications. Accordingly, we have updated the relevant text in Section~IV-D as follows:
\begin{tcolorbox}[breakable]
\textit{D. Restricted Target Wake Time (R-TWT)} \\
{Wi-Fi\,6 target wake time (TWT) specification \cite{nurchis2019target} aims at reducing power consumption by defining specific service periods (SP) in which a device should be awake. R-TWT builds atop this feature to define non-overlapping SPs, representing an attempt to improve support of delay-sensitive and real-time applications (RTA) through scheduled transmissions. In fact, R-TWT forces Wi-Fi\,7 STAs to end ongoing communications before the start of an advertised R-TWT SP, and it also configures a quiet interval for the entire duration of the R-TWT SP to ensure that legacy STAs remain silent.} 
\end{tcolorbox}

	%%%%%%%%%%%%%%%%%%%% Comment %%%%%%%%%%%%%%%%%%%%

\vspace{4mm}
\item \underline{\textbf{Reviewer:}} \textit{Fig 1 is outdated. For example, Phase 2 is shown as 2022 activities, which may be postponed. So, in 2023 we shall have a more clear picture.} 

\noindent\underline{\textbf{Authors:}} 
As suggested, we have now revised Fig.~1 including the latest updates, and removing details that can be unclear for a reader not familiar with the Wi-Fi standardization and commercialization process. As an example, we have now removed any mention to ``postponed'' features and have instead placed each feature under the respective amendment. The revised figure is provided in this response letter as Fig.~\ref{fig:timeline}.

	%%%%%%%%%%%%%%%%%%%% Comment %%%%%%%%%%%%%%%%%%%%

\vspace{4mm}
\item \underline{\textbf{Reviewer:}} \textit{Section III B discusses again Wi-Fi 7, and also mmWave and Machine Learning. How the latter topics are related to Wi-Fi 8?} 

\noindent\underline{\textbf{Authors:}} 
We agree that, accordingly to the recently issued PAR, 802.11bn (Wi-Fi 8) is no longer planning to use mmWave. Indeed, it was recently decided to create a dedicated Integrated mmWave (IMW) SG to develop a new 802.11 amendment specifying carrier frequency operation between 42.5 and 71\,GHz leveraging the 802.11be/bn amendments. 
As suggested, we have thus made the following changes in the manuscript:
\begin{itemize}
\item We have removed the discussion on mmWave from Section~IV (now purely focusing on IEEE 802.11bn) and we have moved it to a newly created Section~VI (focusing on future/complementary research directions).
\item We have removed mmWave operations from Fig.~1, with the new figure reported in this response letter.
\item We have revised the second paragraph of Section III-C as follows:
\begin{tcolorbox}[breakable]
The \emph{IEEE 802.18 mmWave Ad Hoc Group} is exploring options in the 45\,GHz and 60\,GHz bands, which respectively offer 5.5\,GHz and 14\,GHz of spectrum. The 60\,GHz band is currently used by several incumbent technologies, such as satellite, radio astronomy, and IEEE 802.11ad/ay (WiGig). 
However, the market adoption of WiGig has been confined to niche applications, and regulatory bodies may consider repurposing the 60\,GHz band for other bandwidth-hungry technologies such as 5G and 6G. 
{Against this background, and after initial discussions in the UHR SG, it was decided to create a dedicated \emph{Integrated mmWave (IMW) SG} to develop a new 802.11 amendment specifying carrier frequency operation between 42.5 and 71\,GHz and leveraging PHY/MAC functionalities in the existing Sub 7 GHz bands and in the future IEEE 802.11bn amendment (Wi-Fi 8).}
\end{tcolorbox}
\item We have removed the second example (\textit{Integrated mmWave Operations}) from Fig. 2.
\item We have thoroughly revised the manuscript to ensure that integrated mmWave operations are introduced as complementary to Wi-Fi 8, rather than as part of it.

\end{itemize}

Regarding AI/ML, we agree that it is not yet clear how/whether it will relate to Wi-Fi 8 as relevant AI/ML-related discussions are being held in the UHR SG and in the 802.11 AI/ML Topic Interest Group (TIG). We have clarified this aspect through a dedicated a paragraph in ``\textit{Section III. New Use Cases, Standardization, and Spectrum Allocation}'', reported as follows:  
\begin{tcolorbox}[breakable]
\subsubsection*{IEEE 802.11 AI/ML Topic Interest Group (TIG)} 
Established alongside EHT TG and UHR SG to explore the use of artificial intelligence (AI) and machine learning (ML) in Wi-Fi. {This TIG aims to evaluate the feasibility of specific AI/ML-based features that could enhance Wi-Fi\,8-and-beyond based networks while coping with their increasing complexity \cite{szott2022wifi}.} 
One potential use of AI/ML is in determining optimal configurations for OBSSs, including RU assignments, carrier frequencies, modes of operation, and radiation beams and nulls. While AI/ML-driven protocols could prevent undesirable phenomena such as worst-case delay anomalies \cite{CarGerGal22}, currently they are primarily proprietary and limited to devices from the same vendor, making standardization and access to a wider range of data statistics crucial.
\end{tcolorbox}

Regarding the standardization of 802.11be (Wi-Fi 7), only a short paragraph has been kept. We find this helpful to bring the reader up to speed before introducing the standardization of 802.11bn. However, we would consider removing it, should you and the Editor believe this would not be detrimental.

	%%%%%%%%%%%%%%%%%%%% Comment %%%%%%%%%%%%%%%%%%%%

\vspace{4mm}
\item \underline{\textbf{Reviewer:}} \textit{The idea of Seamless Connectivity via Distributed MLO seems interesting, however, it is not clear from the text. What are the main differences between existing MLO and roaming features?} 

\noindent\underline{\textbf{Authors:}} 
As suggested, we have added further details on how we envision Seamless Connectivity via Distributed MLO to take place. These details have been incorporated in Section~IV-A, where the relevant part reads as follows:
\begin{tcolorbox}[breakable]
\emph{IV-A. Seamless Connectivity via Distributed MLO}\\
Mobility support has never been a primary focus in previous Wi-Fi standards, although devices roaming between APs is a major cause of Wi-Fi link unreliability. 
The new multi-link architecture offers a high degree of flexibility, {presenting a clear split between upper (multi-link level) and lower (link level) MAC functionalities, with an MLD that can be seen as an entity that controls two or more legacy APs (or STAs) each operating on a single link.} 
One way to leverage this flexibility {to improve mobility support in Wi-Fi\,8} is through a new \emph{distributed} MLO framework, {where APs under the control of the same MLD entity can be non-co-located rather than implemented in the same physical hardware}. While this approach requires coordination and communication among the different distributed APs under the same {controlling} multi-link instance, it creates a distributed \emph{virtual cell} where a device's mobility is {seamlessly} handled by {ensuring multiple links are concurrently activating from different distributed APs (Fig.~2, leftmost)}. This approach ensures a nomadic device is connected to at least one link, {de facto} embedding native roaming support into MLO and significantly improving the connection's reliability.
\end{tcolorbox}


%%%%%%%%%%%%%%%%%%%% Comment %%%%%%%%%%%%%%%%%%%%

\vspace{4mm}
\item \underline{\textbf{Reviewer:}} \textit{By mistake the authors claim that mmWave/WiGig could be a part of Wi-Fi 8. Wi-Fi 8 will operate at frequencies below 7 GHz. The corresponding sections provide the wrong view.} 

\noindent\underline{\textbf{Authors:}} 
We agree with the reviewer that this aspect should be clarified and corrected. Accordingly, we have implemented extensive changes in the structure of the paper and in the contents. Please see our reply to your comment \#8 for further details.   

%%%%%%%%%%%%%%%%%%%% Comment %%%%%%%%%%%%%%%%%%%%

\vspace{4mm}
\item \underline{\textbf{Reviewer:}} \textit{The paper ends with some numerical results. However, the scenario is not clear. Also, it is not clear how the authors obtained the numerical results. The conclusions from these results are not clear. Why do the authors only focus on this method?} 

\noindent\underline{\textbf{Authors:}} 
We agree that our original submission did not provide sufficient details (due to lack of space) on the simulation setup considered. As suggested, we have now clarified that the purpose of our case study is to illustrate how MLO (key Wi-Fi 7 feature) and MAPC (potential key Wi-Fi 8 feature) can complement each other to reduce latency and increase reliability. While our simulations are use case-agnostic, we have now updated Section~V, hinting at the fact that the scenario considered could be representing a holographic video stream (2\,Gbps, 120\,frames-per-second) requiring latencies lower than 20\,ms (from Table~I). Our results then show that Wi-Fi\,8 using CBF atop MLO could potentially enable such use case whereas MLO alone cannot. 

We have also elaborated on the scenario considered and added reference \cite{carrascosa2022performance}, where more details about the simulator and simulation parameters are provided. The revised section reads as follows.

\begin{tcolorbox}[breakable]
To assess the need in Wi-Fi 8 to extend MLO with MAPC---and CBF in particular---we consider a single WLAN scenario that consists of two overlapping BSSs similar to the rightmost scenario shown in Fig. 2. Each BSS includes a single AP, equipped with four antennas, and a single associated STA, equipped with two antennas. The two BSSs support MLO-EMLMR, operate on the same two 160 MHz links in the 6 GHz band, and implement CBF. Each AP transmits two spatial streams to their respective STAs, using their two remaining spatial degrees of freedom to create radiation nulls towards the other BSS when CBF is enabled. 
{A 2 Gbps traffic stream is active on each AP, corresponding to a 120 frames/second holographic video stream with ON/OFF activity periods of 4.15 ms each. The same simulator used in \cite{carrascosa2022performance} is employed and overheads for CSI acquisition are not considered. All latency values refer exclusively to the AP-STA delay and other potential Wi-Fi 7 and Wi-Fi 8 features are not implemented to isolate and highlight the gains provided by CBF. The full set of simulation parameters is reported in Table II.}

As illustrated in Fig. 3, employing MLO with CBF (bottom) creates additional reuse opportunities and reduces the delay when compared to standalone MLO (top). However, the achievable performance of CBF is related to the accuracy in the null placement. Fig. 4 presents the median, 99\%-tile, and 99.9999\%-tile delay values obtained by combining MLO with CBF as the interference suppression accuracy increases from 10 to 30 dB. For comparison, the corresponding performance with standalone MLO is also displayed. The results show that when nodes contend for the medium with standalone MLO, the 99.9999\% delay exceeds 100 ms. Such performance worsens when combining MLO with CBF with a null accuracy of just 10 dB, since the benefits of a higher spatial reuse are outweighed by the resulting increased interference and degraded MCS (down to 16-QAM 3/4). However, the trend is reversed when the null accuracy increases to 20 dB and above, as the reduction in MCS incurred is more than compensated for by a lack of contention. An accuracy of 30 dB or more allows for the highest MCS (4096-QAM 5/6) and nearly an order of magnitude reduction in the 99.9999\%-tile delay.

The presented results show that even with Wi-Fi 7 features such as 4096-QAM and MLO, meeting low delay requirements with ultra-high reliability can be challenging in dense scenarios with all links showing high contention levels. {CBF can address this issue and help Wi-Fi 8 cope with use cases requiring reliably high throughput and low latency, such as future immersive holographic communications.} 

\end{tcolorbox}

    %%%%%%%%%%%%%%%%%%%%%%%%%%%%%%%%%%%%%%%%%%%%%%%%%%

\end{enumerate}

\newpage