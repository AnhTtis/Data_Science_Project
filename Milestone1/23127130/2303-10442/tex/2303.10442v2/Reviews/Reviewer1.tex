\newpage
	\vspace{3mm}
	{\noindent\LARGE\textbf{Reviewer 1}}
	\vspace{3mm}
	
	\noindent\makebox[\linewidth]{\rule{\textwidth}{1pt}} 
	
	\vspace{3mm}
	
	%%%%%%%%%%%%%%%%%%%% Comment %%%%%%%%%%%%%%%%%%%%

\vspace{6mm}
\begin{enumerate}[label=\arabic*., listparindent=1em]
\item \underline{\textbf{Reviewer:}} \textit{The paper is quite well-written, and readable to average reader in this field.} 

\noindent\underline{\textbf{Authors:}} We thank the reviewer for appreciating the overall clarity and readability of our
work. 


	%%%%%%%%%%%%%%%%%%%% Comment %%%%%%%%%%%%%%%%%%%%

\vspace{4mm}
\item \underline{\textbf{Reviewer:}} \textit{
Ultra-high reliability and low latency have been discussed in 5G, and there have been a large amount of standardization work in 3GPP. At least so far, URLLC is not that successful in commercial use. It would be interesting to consider this feature in WiFi. It would be better to have a comparison between URLLC technologies in 5G and those to be considered in WiFi. What are the potential benefits to consider URLLC in WiFi?
} 

\noindent\underline{\textbf{Authors:}} 
%
We agree that URLLC are yet to reach their expected commercial popularity. However, URLLC are still poised to be a key driver of wireless technology development for the next decade, as can be seen in the list of key performance indicators (KPIs) and key value indicators (KVIs) defined for 6G \cite{HexaD13}. The key benefit of considering URLLC in Wi-Fi is the possibility of enabling its associated use cases in indoor scenarios and in license-exempt bands. The latter represents a challenge as well as an opportunity, given the uncontrolled interference typically incurred in the unlicensed spectrum. In the revised manuscript, this point is clarified through the following sentence in Section III-A 
\begin{tcolorbox}[breakable]
New Wi-Fi capabilities could lead to a vast number of new applications and services. The key use cases in 2030 and beyond for indoor connectivity in unlicensed bands are foreseen to include the following \cite{UHRProposedPAR,HexaD13}:

\noindent\emph{Immersive communications}: [\ldots]

\noindent\emph{Digital twins for manufacturing}: [\ldots  ]

\noindent\emph{e-Health for all}: [\ldots ]

\noindent\emph{Cooperative mobile robots}: [\ldots]
\end{tcolorbox}

\noindent Since the tight word budget prevents us from adding more extensive details, we have referred the reader to the most relevant documents \cite{UHRProposedPAR,HexaD13} on URLLC in licensed and unlicensed bands.

	%%%%%%%%%%%%%%%%%%%% Comment %%%%%%%%%%%%%%%%%%%%

\vspace{4mm}
\item \underline{\textbf{Reviewer:}} \textit{
Line 5 of page 2, ``splitting its multiple radios". Is this a typo? There is only a single radio for this case.
} 

\noindent\underline{\textbf{Authors:}} Thanks for spotting this typo. Accordingly, the sentence has been removed from the updated manuscript. 



	%%%%%%%%%%%%%%%%%%%% Comment %%%%%%%%%%%%%%%%%%%%

\vspace{4mm}
\item \underline{\textbf{Reviewer:}} \textit{
In Fig. 1, it is mentioned some features in 802.11be are likely to postponed to 802.11bn, e.g., TXOP sharing, MIMO enhancements. Any reference for this? This can be misleading for people in this field.
} 

\noindent\underline{\textbf{Authors:}} 
We agree that these details can be unclear for a reader that is not familiar with the Wi-Fi standardization and commercialization process. As suggested, we have now removed any mention to ``postponed'' features and have instead placed each feature under the respective amendment. The revised figure is provided as follows.
\begin{figure*}[h]
    \centering
    \includegraphics[width=0.90\textwidth]{Figures/timeline.eps}
    %\includegraphics{Figures/timeline.eps}
    \caption{Current standardization, certification, and commercialization timelines for IEEE 802.11be (top) and IEEE 802.11bn (bottom).}
    \label{fig:timeline}
\end{figure*}

%\textcolor{red}{Boris: To check. Maybe a short explanation is required.}

	%%%%%%%%%%%%%%%%%%%% Comment %%%%%%%%%%%%%%%%%%%%

\vspace{4mm}
\item \underline{\textbf{Reviewer:}} \textit{
In Fig. 4, is the latency for E2E application? Has the time for CSI acquisition, packet processing and detection been considered? For the simplest case with two APs and two STAs, it seems 1ms latency with high reliability is quite hard to achieve. Then how to fulfill the requirements for those use cases in Table 1?
} 

\noindent\underline{\textbf{Authors:}} 
We agree that our original submission did not provide sufficient details (due to lack of space) on the simulation setup considered. Indeed, being this the first case study combining coordinated beamforming and multi-link operation, we have confined ourselves to a relatively baseline scenario where latency refers to the last Wi-Fi hop and where CSI acquisition, packet processing, and detection delays are not included. We believe packet processing and detection delays can be neglected. As for the impact of (impefect) CSI acquisition, this may on the one hand play an important role and so requires a more detailed study not suitable for a non-technical Magazine paper. On the other hand, static environments as the one considered in our case study may experience long channel coherence times, and thus allow for acceptable CSI acquisition overheads.

We also agree that it is challenging to achieve a 1\,ms latency with high reliability.
While our simulations are use case-agnostic, we have now updated Section~V, hinting at the fact that the scenario considered could be representing a holographic video stream (2\,Gbps, 120\,frames-per-second) requiring latencies lower than 20\,ms (from Table~I). Our results then show that Wi-Fi\,8 using CBF atop MLO could potentially enable such use case whereas MLO alone cannot. 

As suggested, these important insights have now been incorporated in the below revised ``\textit{Section~V. Case Study: Ultra High Reliability in Wi-Fi 8}'', together with the addition of reference \cite{carrascosa2022performance}, where more details about the simulator and simulation parameters are provided. 

\begin{tcolorbox}[breakable]
To assess the need in Wi-Fi 8 to extend MLO with MAPC---and CBF in particular---we consider a single WLAN scenario that consists of two overlapping BSSs similar to the rightmost scenario shown in Fig. 2. Each BSS includes a single AP, equipped with four antennas, and a single associated STA, equipped with two antennas. The two BSSs support MLO-EMLMR, operate on the same two 160 MHz links in the 6 GHz band, and implement CBF. Each AP transmits two spatial streams to their respective STAs, using their two remaining spatial degrees of freedom to create radiation nulls towards the other BSS when CBF is enabled. 
{A 2 Gbps traffic stream is active on each AP, corresponding to a 120 frames/second holographic video stream with ON/OFF activity periods of 4.15 ms each. The same simulator used in \cite{carrascosa2022performance} is employed and overheads for CSI acquisition are not considered. All latency values refer exclusively to the AP-STA delay and other potential Wi-Fi 7 and Wi-Fi 8 features are not implemented to isolate and highlight the gains provided by CBF. The full set of simulation parameters is reported in Table II.}

As illustrated in Fig. 3, employing MLO with CBF (bottom) creates additional reuse opportunities and reduces the delay when compared to standalone MLO (top). However, the achievable performance of CBF is related to the accuracy in the null placement. Fig. 4 presents the median, 99\%-tile, and 99.9999\%-tile delay values obtained by combining MLO with CBF as the interference suppression accuracy increases from 10 to 30 dB. For comparison, the corresponding performance with standalone MLO is also displayed. The results show that when nodes contend for the medium with standalone MLO, the 99.9999\% delay exceeds 100 ms. Such performance worsens when combining MLO with CBF with a null accuracy of just 10 dB, since the benefits of a higher spatial reuse are outweighed by the resulting increased interference and degraded MCS (down to 16-QAM 3/4). However, the trend is reversed when the null accuracy increases to 20 dB and above, as the reduction in MCS incurred is more than compensated for by a lack of contention. An accuracy of 30 dB or more allows for the highest MCS (4096-QAM 5/6) and nearly an order of magnitude reduction in the 99.9999\%-tile delay.

The presented results show that even with Wi-Fi 7 features such as 4096-QAM and MLO, meeting low delay requirements with ultra-high reliability can be challenging in dense scenarios with all links showing high contention levels. {CBF can address this issue and help Wi-Fi 8 cope with use cases requiring reliably high throughput and low latency, such as future immersive holographic communications.} 

\end{tcolorbox}


%%%%%%%%%%%%%%%%%%%%%%%%%%%%%%%%%%%%%%%%%%%%%%%%%%

\end{enumerate}

\newpage