\newpage
	\vspace{3mm}
	{\noindent\LARGE\textbf{Reviewer 2}}
	\vspace{3mm}
	
	\noindent\makebox[\linewidth]{\rule{\textwidth}{1pt}} 
	
	\vspace{3mm}
	
	%%%%%%%%%%%%%%%%%%%% Comment %%%%%%%%%%%%%%%%%%%%

\vspace{6mm}
\begin{enumerate}[label=\arabic*., listparindent=1em]
\item \underline{\textbf{Reviewer:}} \textit{This paper presents a survey of the features of the next generation Wi-Fi: Wi-Fi 8 or 802.11bn. It starts with a description of the current state of Wi-Fi 7 development and then moves on to the beyond Wi-Fi 7 scenarios and KPIs and describes approaches that will be considered in Wi-Fi 8 to satisfy the new requirements.
The paper is quite timely, it has a very good introduction, and raises important questions about the development of IEEE 802.11be and 802.11bn.
However, I see a number of drawbacks as well.} 

\noindent\underline{\textbf{Authors:}} 
Thanks for acknowledging the contribution of our article. In the following, we detail how we have addressed your comments one by one. 

	%%%%%%%%%%%%%%%%%%%% Comment %%%%%%%%%%%%%%%%%%%%

\vspace{4mm}
\item \underline{\textbf{Reviewer:}} \textit{The main question is the novelty and impact of the contents of this paper. Half of the paper describes Wi-Fi 7, but I do not think it brings something new given a number of papers already published (including this magazine) which are completely dedicated to Wi-Fi 7, e.g., [1-5]. Another half of the paper describes Wi-Fi 8, but it also has limited contribution compared to [6], and also contains some information which is not very relevant after the issue of PAR [7].}\\

\noindent\textit{[1] Deng, Cailian, Xuming Fang, Xiao Han, Xianbin Wang, Li Yan, Rong He, Yan Long, and Yuchen Guo. ``IEEE 802.11 be Wi-Fi 7: New challenges and opportunities." IEEE Communications Surveys \& Tutorials 22, no. 4 (2020): 2136-2166.}

\noindent\textit{[2] Garcia-Rodriguez, Adrian, David Lopez-Perez, Lorenzo Galati-Giordano, and Giovanni Geraci. ``IEEE 802.11 be: Wi-Fi 7 strikes back." IEEE Communications Magazine 59, no. 4 (2021): 102-108.}

\noindent\textit{[3] Khorov, Evgeny, Ilya Levitsky, and Ian F. Akyildiz. ``Current status and directions of IEEE 802.11 be, the future Wi-Fi 7." IEEE access 8 (2020): 88664-88688.}

\noindent\textit{[4] Chen, Cheng, Xiaogang Chen, Dibakar Das, Dmitry Akhmetov, and Carlos Cordeiro. ``Overview and performance evaluation of Wi-Fi 7." IEEE Communications Standards Magazine 6, no. 2 (2022): 12-18.}

\noindent\textit{[5] Chauhan, Shivam, Arpit Sharma, Shivam Pandey, Kowtharapu Nageswara Rao, and Pradeep Kumar. ``IEEE 802.11 be: A Review on Wi-Fi 7 Use Cases." In 2021 9th International Conference on Reliability, Infocom Technologies and Optimization (Trends and Future Directions)(ICRITO), pp. 1-7. IEEE, 2021.}

\noindent\textit{[6] Reshef, Ehud, and Carlos Cordeiro. ``Future Directions for Wi-Fi 8 and Beyond." IEEE Communications Magazine 60, no. 10 (2022): 50-55.}

\noindent\textit{[7] https://mentor.ieee.org/802.11/dcn/23/11-23-0480-00-0uhr-uhr-proposed-par.pdf} 

\noindent\underline{\textbf{Authors: }} 
We agree that our original submission dedicated Section~II to introduce the state-of-the-art Wi-Fi 7. This had the objective of making the paper self-contained and accessible to the broad readership, including those who are less familiar with Wi-Fi current state. As suggested, we have now reduced the content related to Wi-Fi 7 to just over a column, i.e., around 7\% of the entire paper. To do so, we have removed detailed descriptions of Wi-Fi 7 features such as MLO, 320~MHz channels, 4K-QAM modulation, allocation of multiple RU, and restricted TWT. Moreover, we have referred the reader to the suggested articles for further details on these features. The revised section reads as follows.

\begin{tcolorbox}[breakable]

As suggested by its name (IEEE 802.11be `Extremely High Throughput' or `EHT'), Wi-Fi\,7 will augment data rates to at least 30\,Gbps per AP, about four times as fast as Wi-Fi\,6. 
In the sequel, we summarize the main features introduced in the soon-to-appear Wi-Fi\,7 commercial products \cite{garcia2021ieee,CheCheDas22}. 

\emph{A. Multi-link Operation (MLO)} 

Many experts point to multi-link operation (MLO) as the main novelty Wi-Fi\,7 brings to the table, allowing Wi-Fi devices to concurrently operate on multiple channels through a single connection\cite{CarGerKniICC2022,CarGerGal22,lopez2022multi}. MLO comes in different implementations {according to the number and mode of operation of the active radios: Enhanced Multi-link Single-radio (EMLSR), Simultaneous Transmit and Receive Enhanced Multi-link Multi-radio (STR EMLMR), Non-simultaneous Transmit and Receive EMLMR (NSTR EMLMR).}

Recent studies showed that in scenarios devoid of contention, STR EMLMR---the most flexible MLO implementation---supports significantly higher traffic loads (and thus throughput) than single-link for a given delay requirement. However, under high load and contention, STR EMLMR devices frequently access multiple links, often blocking contending basic service sets (BSSs) and occasionally causing even larger delays than those experienced with legacy single-link \cite{CarGerKniICC2022,CarGerGal22}. 
{Future Wi-Fi standard amendments are envisioned to prevent these worst-case events.}

\emph{B. 320\,MHz Channels and 4K-QAM Modulation}\\
These two enhancements are respectively achieved by duplicating the 160\,MHz tone plan of Wi-Fi\,6 and by adding two new modulation and coding scheme (MCS) indices. While these two upgrades jointly increase the maximum nominal rates by a factor of 2.4, wide contiguous channels of 320\,MHz are only likely to be found in the newly opened 6\,GHz band. 
Moreover, the new modulation orders require very high signal-to-noise ratios that may only be achieved in line-of-sight, close-proximity links (devoid of rich scattering and thus unsuitable for using multiple spatial streams) via beamforming, with high-quality hardware and eventually mesh-based installations.

\emph{{C. Multiple Resource Units (MRU) allocation}}\\
{To increase spectral efficiency, Wi-Fi 7 allows allocating a Multiple RU (MRU) per station (STA) consisting of a selected combination of RUs.} A prime example of a scenario where such a degree of flexibility may pay off is in a BSS with a small number of users. For instance with Wi-Fi 6, an AP operating on an 80 MHz channel where the secondary 20 MHz channel is occupied was only able to assign the primary 20 MHz channel to a certain STA. Wi-Fi 7 enables the same AP to also assign the available secondary 40 MHz channel to the same STA, {providing a total of 60 MHz. Such extra bandwidth can be used either to transmit faster and reduce latency, or to improve reliability by enabling more robust transmissions using lower MCSs.}

\emph{D. Restricted Target Wake Time (R-TWT)}\\ 
{Wi-Fi\,6 target wake time (TWT) specification \cite{nurchis2019target} aims at reducing power consumption by defining specific service periods (SP) in which a device should be awake. R-TWT builds atop this feature to define non-overlapping SPs, representing an attempt to improve support of delay-sensitive and real-time applications (RTA) through scheduled transmissions. In fact, R-TWT forces Wi-Fi\,7 STAs to end ongoing communications before the start of an advertised R-TWT SP, and it also configures a quiet interval for the entire duration of the R-TWT SP to ensure that legacy STAs remain silent.} 

\end{tcolorbox}

\noindent In light of the above changes, the revised manuscript now almost entirely focuses on Wi-Fi 8. 

Regarding our key contribution with respect to \cite{ResCor22}---a very relevant article that we have cited---this lies in: 
\begin{enumerate}
    \item An up-to-date summary of the drivers and features envisioned for Wi-Fi 8 after the issue of the 802.11bn PAR. This includes emerging use cases, along with reliability, latency and data rate requirements (Section~III, Figure~1, and Table~I). Ours is also the first article revealing the naming (IEEE 802.11bn) associated to the new Wi-Fi 8 standard amendment and presenting a clear timeline for its development (Fig.~1).
    \item A technical description of some of these key new features for 802.11bn. These include detailed insights gathered from industrial contributions currently under discussion in the UHR Study Group, as well as new ideas and approaches never discussed previously in the literature, such as distributed multi-link operations (Section~IV).
    \item Novel system-level studies demonstrating how Wi-Fi 8 could achieve ultra high reliability through a joint inter-working (Section V) of multi-link operation (a Wi-Fi 7 feature) and spatial-domain AP coordination (a foreseen Wi-Fi 8 feature). We believe our early results will act as a catalyst for further research in this area.
    \item A fresh overview of complementary extensions to Wi-Fi 8 products, represented by the clear description of opportunities and challenges associated to the definition of revised mmWave operations that led to the recent formation of the     Integrated mmWave (IMW) SG. The IMW SG will be in charge to develop a new 802.11 amendment specifying carrier frequency operation between 42.5 and 71 GHz, building atop PHY/MAC functionalities in the existing Sub 7 GHz bands and in the future IEEE 802.11bn amendment (Wi-Fi 8).   
\end{enumerate}
 
\noindent In light of the above, we hope you will agree that our article complements very well, and further extends, the original contributions in \cite{ResCor22}.

We finally agree on the importance of providing up-to-date information with respect to the issued PAR, and we refer you to our reply to your next comment.
    
	%%%%%%%%%%%%%%%%%%%% Comment %%%%%%%%%%%%%%%%%%%%

\vspace{4mm}
\item \underline{\textbf{Reviewer:}} \textit{Much attention in the paper is paid to mmWave communications, however, according to the recently issued PAR, 802.11bn will not use mmWave.} 

\noindent\underline{\textbf{Authors:}} 
%
We agree that, accordingly to the recently issued PAR, 802.11bn is no longer planning to use mmWave. Indeed, it was recently decided to create a dedicated IMW SG to develop a new 802.11 amendment specifying carrier frequency operation between 42.5 and 71\,GHz leveraging the 802.11be/bn amendments. 
As suggested, we have thus made the following changes in the manuscript:
\begin{itemize}
\item We have removed the discussion on mmWave from Section~IV (now purely focusing on IEEE 802.11bn) and we have moved it to a newly created Section~VI (focusing on future/complementary research directions).
\item We have removed mmWave operations from Fig.~1, with the new figure included in this response letter.

\item We have revised the second paragraph of Section III-C as follows:
\begin{tcolorbox}[breakable]
The \emph{IEEE 802.18 mmWave Ad Hoc Group} is exploring options in the 45\,GHz and 60\,GHz bands, which respectively offer 5.5\,GHz and 14\,GHz of spectrum. The 60\,GHz band is currently used by several incumbent technologies, such as satellite, radio astronomy, and IEEE 802.11ad/ay (WiGig). 
However, the market adoption of WiGig has been confined to niche applications, and regulatory bodies may consider repurposing the 60\,GHz band for other bandwidth-hungry technologies such as 5G and 6G. 
{Against this background, and after initial discussions in the UHR SG, it was decided to create a dedicated \emph{Integrated mmWave (IMW) SG} to develop a new 802.11 amendment specifying carrier frequency operation between 42.5 and 71\,GHz and leveraging PHY/MAC functionalities in the existing Sub 7 GHz bands and in the future IEEE 802.11bn amendment (Wi-Fi 8).}
\end{tcolorbox}
\item We have removed the second example (\textit{Integrated mmWave Operations}) from Fig. 2.
\item We have thoroughly revised the manuscript to ensure that integrated mmWave operations are introduced as complementary to Wi-Fi 8, rather than as part of it. 
\end{itemize}

%In fact, at the moment of submitting the manuscript, integrated sub-7 GHz and mmWave operations were under discussion in the UHR SG. However, in April 2023, it was decided to create a dedicated Integrated mmWave (IMW) SG to develop a new 802.11 amendment specifying carrier frequency operation between 42.5 and 71 GHz in Wi-Fi 8 leveraging the 802.11be/bn amendments.  
%\red{To reflect this decision, and since we still consider important to highlight this opportunity for high-end Wi-Fi 8 product available in 2028-onwards, we made a complete review of how we presented mmWave operations in our manuscript and made a number of significant improvements, described in the following.}
%\textcolor{red}{Fig. 1: remove mmWave (new figure to be produced and added here)}


%%%%%%%%%%%%%%%%%%%% Comment %%%%%%%%%%%%%%%%%%%%

\vspace{4mm}
\item \underline{\textbf{Reviewer:}} \textit{The paper mentions ultra high reliability (UHR) and ultra reliable low latency communications (URLLC), but misses the results of the real-time applications topic interest group (RTA TIG) of 802.11 WG which consider similar problems (providing low latency with high reliability) and has inspired some activities for 802.11be.} 

\noindent\underline{\textbf{Authors:}} 
Thanks for this useful remark. As suggested, we have added a new paragraph in ``\textit{Section~III-B. Standardization: From IEEE 802.11be to IEEE 802.11bn}'', describing the activities and impact of the RTA TIG group in both Wi-Fi 7 and 8.
\begin{tcolorbox}[breakable]
[...]
\subsubsection*{IEEE 802.11be EHT} 
Set to define the main technical features of upcoming Wi-Fi\,7 products. 
Started in May 2019, the amendment has currently reached a mature stage with the release of multiple drafts and the definition of a set of features \cite{lopez2019ieee,garcia2021ieee,khorov2020current,deng2020ieee,yang2020survey,CheCheDas22}. The 802.11be Task Group (TG) is expected to produce the final amendment in May 2024. 
Its primary objective is to increase capacity and link throughput and also improve worst-case latency and jitter with at least one mode of operation. While the latter is a novel endeavor compared to previous Wi-Fi generations, target latency and jitter were not quantified, making this only an initial step towards reliability.

\subsubsection*{The IEEE 802.11bn UHR} 
Whose Study Group (SG) was established in July 2022 to support URLLC. The UHR SG will produce a new PAR defining the set of objectives, frequency bands, and technologies to be considered beyond 802.11be. The current plan is to form the UHR TG by November 2023, with a traditional single release standardization cycle that will last until 2028. This activity will define the protocol functionalities of future Wi-Fi\,8 products, mainly focusing on these aspects to be improved with respect to 802.11be \cite{UHRProposedPAR}:

\begin{itemize}
    \item Data rates, even at lower signal-to-interference-plus-noise ratio (SINR) levels. 
    \item Tail latency and jitter, even in scenarios with mobility and overlapping BSSs (OBSSs).
    \item Reuse of the wireless medium.
    \item Power saving and peer-to-peer operation.
\end{itemize}
Discussions are ongoing on the specific performance targets.
{\subsubsection*{IEEE 802.11 Real Time Applications (RTA) Topic Interest Group (TIG)}
Back in 2019, the RTA-TIG provided a set of recommendations and guidelines to support low latency and reliability in future Wi-Fi networks \cite{RTA}. Those recommendations have been considered in the Wi-Fi 7 development (e.g., MLO), but they are also influencing the effort towards reliability in Wi-Fi 8 (e.g., via TSN integration).}
\subsubsection*{IEEE 802.11 AI/ML Topic Interest Group (TIG)} 
Established alongside EHT TG and UHR SG to explore the use of artificial intelligence (AI) and machine learning (ML) in Wi-Fi. This TIG aims to evaluate the feasibility of specific AI/ML-based features that could enhance Wi-Fi\,8-and-beyond based networks while coping with their increasing complexity \cite{szott2022wifi}. One potential use of AI/ML is in determining optimal configurations for OBSSs, including RU assignments, carrier frequencies, modes of operation, and radiation beams and nulls. While AI/ML-driven protocols could prevent undesirable phenomena such as worst-case delay anomalies \cite{CarGerGal22}, currently they are primarily proprietary and limited to devices from the same vendor, making standardization and access to a wider range of data statistics crucial.

[...]
\end{tcolorbox}

\noindent We hope that our added explanations, together with the newly added reference \cite{RTA}, address your concerns.

	%%%%%%%%%%%%%%%%%%%% Comment %%%%%%%%%%%%%%%%%%%%

\vspace{4mm}
\item \underline{\textbf{Reviewer:}} \textit{The paper mentions that some completely new mechanisms will be added to Wi-Fi 8, like HARQ, but no details and sources are given. We know that many new functions are proposed at the beginning stage of the development of an amendment, but it does not mean that all the proposals will later be included. For example, HARQ has been proposed for Wi-Fi 7, and some version of it has even been proposed in TGax, but it turned out to be too complex to implement. I think the paper needs some suggestions which features are more or less likely to be implemented.} 

\noindent\underline{\textbf{Authors:}} 
We agree that HARQ had originally been proposed for Wi-Fi 7 but was later sidelined due to complexity concerns. As suggested, we have now toned down our discussion on HARQ to better remark that, although this feature has been proposed at the early stage of development for 802.11bn, it is yet to be confirmed whether it will eventually be included in the final amendment. The revised text in ``\textit{Section~IV-C. Determinism via PHY and MAC Enhancements}'' reads as follows:
\begin{tcolorbox}[breakable]

Wi-Fi\,8 will {consider the possibility to include} PHY/MAC enhancements such as hybrid automatic repeat request (HARQ) and increasing the number of supported spatial streams from 8 to 16. The use of HARQ {could} allow devices to combine corrupted data units with their corresponding retransmissions to increase the probability of correct decoding, reducing latency in challenging channel conditions. 
The availability of additional spatial streams {could} enable more users to be served simultaneously, reducing their channel access time, and provide extra degrees of freedom to mechanisms such as coordinated beamforming (discussed later in this section). 
Additionally, other features building atop TXOP sharing functionalities {will potentially} allow APs to share a portion of their obtained TXOPs with associated stations for transmitting uplink frames to the AP or for direct peer-to-peer communication with another station. 
The new {functionalities}, combined with the aforementioned R-TWT mechanisms and TXOP sharing principles, 
already promise improvements in terms of reliability. However, further enhancements may be needed to support the arrival of unexpected or event-based time-sensitive traffic during large ongoing transmissions. An interesting proposal to address this issue is the one of frame `preemption', detailed below.

[...]

\end{tcolorbox}

	%%%%%%%%%%%%%%%%%%%% Comment %%%%%%%%%%%%%%%%%%%%

\vspace{4mm}
\item \underline{\textbf{Reviewer:}} \textit{The mentioned coordinated TDMA/FDMA/beamforming mechanisms require exchange of information between the APs. Please elaborate how is it supposed to be done? Will it be a wireless service link or a wired one, will it be described by the standard.} 

\noindent\underline{\textbf{Authors:}}
As suggested, we have now added more details on how we foresee the coordination between APs to take place, by also clarifying how some aspects might be introduced in the standard or left for proprietary implementation. The corresponding text in ``\textit{Section IV-D. Controlled Worst-Case Delay via AP Coordination}'' now reads as follows:
\begin{tcolorbox}[breakable]

[\ldots] 

To this end, new protocols and frames will be necessary for discovering and managing multi-AP groups, sharing channel and buffer state data between APs, and triggering coordinated multi-AP transmissions to minimize inter-BSS collisions and achieve a more efficient use of the spectrum through dynamic inter-AP resource management. AP coordination schemes in Wi-Fi\,8 {are envisioned to leverage both over-the-air and wired signaling. These schemes} will range from basic to advanced, depending on the amount of data that must be exchanged between access points and the level of implementation complexity. {While it is still to be decided what aspects of the coordination mechanism will be specified by the standard and what will be left for implementation, the main schemes are expected to} include the ones described in the following.
\end{tcolorbox}


	%%%%%%%%%%%%%%%%%%%% Comment %%%%%%%%%%%%%%%%%%%%

\vspace{4mm}
\item \underline{\textbf{Reviewer:}} \textit{The case study shows that with the latest Wi-Fi 7 features meeting low delay requirements with UHR is challenging. However, are all the possibilities of Wi-Fi 6 and 7 used to provide low latency and high reliability? What about OBSS PD or r-TWT.} 

\noindent\underline{\textbf{Authors:}}
We agree that other features exist that could further reduce the delay in Wi-Fi 7, including the mentioned R-TWT and OBSS PD. Indeed, we have discussed some of these features in Section~II. However, we have now clarified that the purpose of our case study is to illustrate how MLO (key Wi-Fi 7 feature) and MAPC (potential key Wi-Fi 8 feature) can complement each other to reduce latency and increase reliability. 

As suggested, these important insights have now been incorporated in ``\textit{Section~V. Case Study: Ultra High Reliability in Wi-Fi 8}'', where the relevant parts read as follows:

\begin{tcolorbox}[breakable]
To assess the need in Wi-Fi 8 to extend MLO with MAPC---and CBF in particular---we consider a single WLAN scenario that consists of two overlapping BSSs similar to the rightmost scenario shown in Fig. 2. Each BSS includes a single AP, equipped with four antennas, and a single associated STA, equipped with two antennas. The two BSSs support MLO-EMLMR, operate on the same two 160 MHz links in the 6 GHz band, and implement CBF. Each AP transmits two spatial streams to their respective STAs, using their two remaining spatial degrees of freedom to create radiation nulls towards the other BSS when CBF is enabled. 
{A 2 Gbps traffic stream is active on each AP, corresponding to a 120 frames/second holographic video stream with ON/OFF activity periods of 4.15 ms each. The same simulator used in \cite{carrascosa2022performance} is employed and overheads for CSI acquisition are not considered. All latency values refer exclusively to the AP-STA delay and other potential Wi-Fi 7 and Wi-Fi 8 features are not implemented to isolate and highlight the gains provided by CBF. The full set of simulation parameters is reported in Table II.}

As illustrated in Fig. 3, employing MLO with CBF (bottom) creates additional reuse opportunities and reduces the delay when compared to standalone MLO (top). However, the achievable performance of CBF is related to the accuracy in the null placement. Fig. 4 presents the median, 99\%-tile, and 99.9999\%-tile delay values obtained by combining MLO with CBF as the interference suppression accuracy increases from 10 to 30 dB. For comparison, the corresponding performance with standalone MLO is also displayed. The results show that when nodes contend for the medium with standalone MLO, the 99.9999\% delay exceeds 100 ms. Such performance worsens when combining MLO with CBF with a null accuracy of just 10 dB, since the benefits of a higher spatial reuse are outweighed by the resulting increased interference and degraded MCS (down to 16-QAM 3/4). However, the trend is reversed when the null accuracy increases to 20 dB and above, as the reduction in MCS incurred is more than compensated for by a lack of contention. An accuracy of 30 dB or more allows for the highest MCS (4096-QAM 5/6) and nearly an order of magnitude reduction in the 99.9999\%-tile delay.

The presented results show that even with Wi-Fi 7 features such as 4096-QAM and MLO, meeting low delay requirements with ultra-high reliability can be challenging in dense scenarios with all links showing high contention levels. {CBF can address this issue and help Wi-Fi 8 cope with use cases requiring reliably high throughput and low latency, such as future immersive holographic communications.} 

\end{tcolorbox}


	%%%%%%%%%%%%%%%%%%%% Comment %%%%%%%%%%%%%%%%%%%%

\vspace{4mm}
\item \underline{\textbf{Reviewer:}} \textit{Generally, more information is required on the simulation for the case study.} 

\noindent\underline{\textbf{Authors:}}
We agree that our original submission did not provide sufficient details (due to lack of space) on the simulation setup considered. As suggested, these important details have now been incorporated in the revised \textit{Section~V Case Study: Ultra High Reliability in Wi-Fi 8}, together with the addition of a reference (\cite{carrascosa2022performance}) where more details about the simulator and simulation parameters are provided. Please refer to our reply to your previous comment.

	%%%%%%%%%%%%%%%%%%%% Comment %%%%%%%%%%%%%%%%%%%%

\vspace{4mm}
\item \underline{\textbf{Reviewer:}} \textit{I think that the comments 1 and 2 are critical and the paper needs rewriting. Perhaps it would be better to change its structure.} 

\noindent\underline{\textbf{Authors:}} 
We agree that your first comments were extremely useful. Indeed, they have triggered a thorough revision of our paper, including its structure, as outlined above. We hope you will now concur that our manuscript is ready for publication.

    %%%%%%%%%%%%%%%%%%%%%%%%%%%%%%%%%%%%%%%%%%%%%%%%%%

\end{enumerate}

\newpage