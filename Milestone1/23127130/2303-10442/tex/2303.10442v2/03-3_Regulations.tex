\subsection{mmWave Spectrum} 
\label{sec:regulations}

To maintain Wi-Fi's competitiveness and support new demanding applications, it will be crucial to secure new spectrum. 
A significant step has been taken in this regard with the decision by many countries to make the band around 6\,GHz available for secondary licensed-exempt use by Wi-Fi. 
The new 6\,GHz band will offer significant advantages to Wi-Fi\,6E and Wi-Fi\,7 devices, since older generations of Wi-Fi will not be able to use it. However, these benefits will diminish as devices populate the band. 
Moreover, the 6\,GHz band can only support (in best-case scenarios) up to three 320 MHz channels, and may be unable to meet the long-term needs of Wi-Fi applications. 
In addition, the adoption of the 6\,GHz spectrum in China is still uncertain, challenging the evolution of Wi-Fi towards supporting new use cases. 
Proposals for a more efficient and integrated use of the mmWave spectrum have therefore emerged alongside initial Wi-Fi \,8 standardization discussions.
%Proposal for additional spectrum have therefore emerged alongside initial Wi-Fi \,8 standardization discussions. 

%Countries worldwide have defined bandwidth options and rules for this spectrum. For example, the US, Canada, Brazil, South Korea, and Saudi Arabia have decided to release the whole 1200 MHz of spectrum, while the EU, UK, South Africa, UAE, Japan, and Australia have only optioned the first 500 MHz. However, it is still uncertain whether China will reserve the 6 GHz spectrum for licensed use of cellular technologies or make it available for Wi-Fi.
%However, the newly released 6 GHz band is inadequate to allow for a sufficient number of wider channels, such as 480 or 640 MHz, especially in Europe where only 500 MHz of spectrum have been released for unlicensed use.

The \emph{IEEE 802.18 mmWave Ad Hoc Group} is exploring options in the 45\,GHz and 60\,GHz bands, which respectively offer 5.5\,GHz and 14\,GHz of spectrum. The 60\,GHz band is currently used by several incumbent technologies, such as satellite, radio astronomy, and IEEE 802.11ad/ay (WiGig). 
However, the market adoption of WiGig has been confined to niche applications, and regulatory bodies may consider repurposing the 60\,GHz band for other bandwidth-hungry technologies such as 5G and 6G. 
%OLD Against this background, the UHR SG will have to evaluate the suitability and benefits of integrating mmWave operations into Wi-Fi\,8, either as part of the 802.11bn amendment or via a separate one.
%Lorenzo New: 
Against this background, and after initial discussions in the UHR SG, it was decided to create a dedicated \emph{Integrated mmWave (IMW) SG} to develop a new 802.11 amendment specifying carrier frequency operation between 42.5 and 71\,GHz and leveraging PHY/MAC functionalities in the existing Sub 7 GHz bands and in the future IEEE 802.11bn amendment (Wi-Fi 8).