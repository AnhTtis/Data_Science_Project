\subsection{Standardization: From IEEE 802.11be to IEEE 802.11bn} \label{sec:standardization}

Fig.~\ref{fig:timeline} summarizes the ongoing IEEE standardization effort for 802.11be (top) and 802.11bn (bottom). 

\subsubsection*{IEEE 802.11be EHT} 
Set to define the main technical features of upcoming Wi-Fi\,7 products. 
Started in May 2019, the amendment has currently reached a mature stage with the release of multiple drafts and the definition of a set of features \cite{lopez2019ieee,garcia2021ieee,khorov2020current,deng2020ieee,yang2020survey,CheCheDas22}. The 802.11be Task Group (TG) is expected to produce the final amendment in May 2024. 
Its primary objective is to increase capacity and link throughput and also improve worst-case latency and jitter with at least one mode of operation. While the latter is a novel endeavor compared to previous Wi-Fi generations, target latency and jitter were not quantified, making this only an initial step towards reliability.

\subsubsection*{The IEEE 802.11bn UHR} 
Whose Study Group (SG) was established in July 2022 to increase support for URLLC. The UHR SG will produce a new PAR defining the set of objectives, frequency bands, and technologies to be considered beyond 802.11be. The current plan is to form the UHR TG by November 2023, with a traditional single release standardization cycle that will last until 2028. This activity will define the protocol functionalities of future Wi-Fi\,8 products, mainly focusing on these aspects to be improved with respect to 802.11be \cite{UHRProposedPAR}:

\begin{itemize}
    \item Data rates, even at lower signal-to-interference-plus-noise ratio (SINR) levels. 
    %\bcom{Not clear what means here a 'mode of operation'. Any hint to show what is behind this statement? More antennas? Better channel coding?}
    % Gio: The PAR is being vague on purpose.
    % Gio: In our understanding, this is not something trivial like 4K-QAM increasing peak rate, but rather something like more spatial streams which increase throughput across all SINR values.
    %
    \item Tail latency and jitter, even in scenarios with mobility and overlapping BSSs (OBSSs).
    %
    \item Reuse of the wireless medium. %\bcom{Whare the 802.11be unefficiencies to improve?}
    % Gio: I think, primarily, CSMA/CA. But the PAR does not specify more.
    %
    \item Power saving and peer-to-peer operation.
    %
\end{itemize}
Discussions are ongoing on the specific performance targets.
\subsubsection*{IEEE 802.11 Real Time Applications (RTA) Topic Interest Group (TIG)}
Back in 2019, the RTA-TIG provided a set of recommendations and guidelines to support low latency and reliability in future Wi-Fi networks \cite{RTA}. Those recommendations have been considered in the Wi-Fi 7 development (e.g., MLO), but they are also influencing the effort towards reliability in Wi-Fi 8 (e.g., via TSN integration).

\subsubsection*{IEEE 802.11 AI/ML Topic Interest Group (TIG)} 
Established alongside EHT TG and UHR SG to explore the use of artificial intelligence (AI) and machine learning (ML) in Wi-Fi. This TIG aims to evaluate the feasibility of specific AI/ML-based features that could enhance Wi-Fi\,8-and-beyond based networks while coping with their increasing complexity \cite{szott2022wifi}. 
One potential use of AI/ML is in determining optimal configurations for OBSSs, including RU assignments, carrier frequencies, modes of operation, and radiation beams and nulls. While AI/ML-driven protocols could prevent undesirable phenomena such as worst-case delay anomalies \cite{CarGerGal22}, currently they are primarily proprietary and limited to devices from the same vendor, making standardization and access to a wider range of data statistics crucial.