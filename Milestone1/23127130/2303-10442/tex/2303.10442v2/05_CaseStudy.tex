\section{Case Study: Ultra High Reliability in Wi-Fi\,8}

\begin{figure}
    \centering
    \includegraphics[width=0.40\textwidth]{Figures/MLO-CBF.eps}
    \caption{Illustration of standalone MLO and combined MLO with CBF.}
    %\caption{Illustration of standalone MLO (top) and combined MLO with CBF (bottom). When CBF is enabled, imperfect nulling may result in the use of a lower MCS and larger transmission delays. However, this may be more than compensated for by the increased spatial reuse.}
    \label{fig:MLO+CBF}
\end{figure}


To assess the need in Wi-Fi 8 to extend MLO with MAPC---and CBF in particular---we consider a single WLAN scenario that consists of two overlapping BSSs similar to the rightmost scenario shown in Fig. 2. Each BSS includes a single AP, equipped with four antennas, and a single associated STA, equipped with two antennas. The two BSSs support MLO-EMLMR, operate on the same two 160 MHz links in the 6 GHz band, and implement CBF. Each AP transmits two spatial streams to their respective STAs, using their two remaining spatial degrees of freedom to create radiation nulls towards the other BSS when CBF is enabled.
A 2 Gbps traffic stream is active on each AP, corresponding to a 120 frames/second holographic video stream with ON/OFF activity periods of 4.15 ms each. The same simulator used in \cite{carrascosa2022performance} is employed and overheads for CSI acquisition are not considered. All latency values refer exclusively to the AP-STA delay and other potential Wi-Fi 7 and Wi-Fi 8 features are not implemented to isolate and highlight the gains provided by CBF. The full set of simulation parameters is reported in Table II.

As illustrated in Fig. 3, employing MLO with CBF (bottom) creates additional reuse opportunities and reduces the delay when compared to standalone MLO (top). However, the achievable performance of CBF is related to the accuracy in the null placement. Fig. 4 presents the median, 99\%-tile, and 99.9999\%-tile delay values obtained by combining MLO with CBF as the interference suppression accuracy increases from 10 to 30 dB. For comparison, the corresponding performance with standalone MLO is also displayed. The results show that when nodes contend for the medium with standalone MLO, the 99.9999\% delay exceeds 100 ms. Such performance worsens when combining MLO with CBF with a null accuracy of just 10 dB, since the benefits of a higher spatial reuse are outweighed by the resulting increased interference and degraded MCS (down to 16-QAM 3/4). However, the trend is reversed when the null accuracy increases to 20 dB and above, as the reduction in MCS incurred is more than compensated for by a lack of contention. An accuracy of 30 dB or more allows for the highest MCS (4096-QAM 5/6) and nearly an order of magnitude reduction in the 99.9999\%-tile delay.

The presented results show that even with Wi-Fi 7 features such as 4096-QAM and MLO, meeting low delay requirements with ultra-high reliability can be challenging in dense scenarios with all links showing high contention levels. CBF can address this issue and help Wi-Fi 8 cope with use cases requiring reliably high throughput and low latency, such as future immersive holographic communications.

%OLD
%To assess the improvements in reliability provided by CBF, we consider the rightmost scenario in Fig.~\ref{fig:WiFi8Features}. Two APs, each with four antennas and a single associated STA equipped with two antennas, implement MLO on two 160\,MHz shared links in the 6\,GHz band. Each AP transmits two spatial streams to their respective STAs. The two APs also employ CBF and use their two remaining spatial degrees of freedom to create radiation nulls towards the other BSS, creating spatial reuse opportunities and avoiding contending for the medium. 

%As illustrated in Fig.~\ref{fig:MLO+CBF}, employing MLO with CBF (bottom) creates additional reuse opportunities and reduces the delay when compared to standalone MLO (top). However, placing imperfect nulls also entails a potentially longer transmission time due to an increased OBSS interference and an MCS reduction. We therefore evaluate the effect of an increasing accuracy in the null placement in Fig.~\ref{fig:ThroughputAndDelay}, for which the full set of simulation parameters is reported in Table~\ref{table:parameters}.

%Fig.~\ref{fig:ThroughputAndDelay} presents the median, 99\%-tile, and 99.9999\%-tile delay values obtained by combining MLO with CBF as the interference suppression accuracy increases from 10 to 30\,dB. For comparison, the corresponding performance with standalone MLO is also displayed. The results show that when nodes contend for the medium with standalone MLO, the 99.9999\% delay exceeds 100\,ms. 
%Such performance worsens when combining MLO with CBF with a null accuracy of just 10\,dB, since the benefits of a higher spatial reuse are outweighed by the resulting increased interference and degraded MCS (down to 16-QAM 3/4). However, the trend is reversed when the null accuracy increases to 20\,dB and above, as the reduction in MCS incurred is more than compensated for by a lack of contention. An accuracy of 30\,dB or more allows for the highest MCS (4096-QAM 5/6) and nearly an order of magnitude reduction in the 99.9999\%-tile delay.

%Fig.~\ref{fig:ThroughputAndDelay} shows that even with the latest Wi-Fi\,7 features such as 4096-QAM and MLO, meeting low delay requirements with ultra-high reliability can be challenging in highly dense scenarios where all links are occupied. The primary source of randomness in Wi-Fi networks is contention, and CBF can address this issue and help Wi-Fi\,8 support simultaneous transmissions and achieve near-deterministic operation.

%%%%%%%%%%%%%%%%%%%%%%%%%%%%%%%%%%%%%%%%

%%%%%%%%%%%%%%%%%%%%%%%%%%%%%%%%%%%%%%%%%%%%%%%%%%%%%%%%
\begin{table}
\centering
\caption{System-level simulation parameters for the case study.}
\label{table:parameters}
\def\arraystretch{1.2}
\colorbox{BackgroundGray}{
\begin{tabulary}{\columnwidth}{ |p{2.6cm} | p{5.0cm} | }
\hline
\rowcolor{BackgroundLightBlue}
  \textbf{Deployment \& traffic}	&  \\ \hline
  AP locations			& (5\,m, 10\,m) and (10\,m, 10\,m) \\ \hline
  STA locations			& (5\,m, 12.5\,m) and (10\,m, 12.5\,m) \\ \hline
  %Traffic load			& 2\,Gbps per AP \\ \hline
  Traffic arrivals			& On/Off, both periods exponentially distributed with mean 4.15\,ms (2\,Gbps per AP) \\ \hline
  %Buffer size			& 10,240 packets \\ \hline
\rowcolor{BackgroundLightBlue}
  \textbf{MAC}	&  \\ \hline
  Buffer size, TXOP			& 10,240 packets, 5.484\,ms max.  \\ \hline
  Frame aggregation			& 1024 frames max., 10\% packet error rate \\ \hline
  MLO mode			& STR EMLMR \\ \hline
\rowcolor{BackgroundLightBlue}
  \textbf{PHY}	&  \\ \hline
  Frequency band			& 6\,GHz band with two 160\,MHz channels \\ \hline
  Channel model			& IEEE 802.11ax (indoor residential) \\ \hline
  Transmit power, noise			& 20\,dBm, -174\,dBm/Hz spectral density  \\ \hline
  %Highest MCS			& 4096-QAM 5/6 \\ \hline
  MCS selection			& SINR-based, highest: 4096-QAM 5/6 \\ \hline
  Number of antennas			& 4 per AP and 2 per STA \\ \hline
  CBF nulling			& Imperfect, ranging from 10\,dB to 30\,dB \\ \hline
  %Thermal noise			& -174~dBm/Hz spectral density \\ \hline
  %STA noise figure			& XXXXXXXXXXXX \\ \hline
  %Packet error rate			& 10\% \\ \hline
\end{tabulary}
}
\vspace{-0.5cm}
\end{table}

%%%%%%%%%%%%%%%%%%%%%%%%%%%%%%%%%%%%%%%%%%%

\begin{figure}
    \centering
    \includegraphics[width=0.48\textwidth]{Figures/delayMLOCBF.eps}
    \caption{Delay incurred by standalone MLO and when combining MLO with CBF under a variable nulling accuracy.}
    \label{fig:ThroughputAndDelay}
\end{figure}

%%%%%%%%%%%%%%%%%%%%%%%%%%%%%%%%%%%%%%%%