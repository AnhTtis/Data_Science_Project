\section{Model and Problem Statement}
\label{sec:model}

\begin{figure}[t] % TODO: change to t
    \centering
    \begin{minipage}{.49\textwidth}
        \centering
        \includegraphics[height=.65\linewidth,trim={{0} {12} {0} {24}},clip]{modelgraph}
        \caption{Tiled nodes of the underlying graph $G$ and their incident edges to empty nodes.}
        \label{fig:modelgraph}
    \end{minipage}%
    \hfill
    \begin{minipage}{.49\textwidth}
        \centering
        \includegraphics[height=.65\linewidth,trim={{0} {12} {0} {24}},clip]{modeldirections}
        \caption{A passive tile (rhombic dodecahedron) and the twelve compass directions.}
        \label{fig:modeldirections}
    \end{minipage}
\end{figure}
In the \emph{3D hybrid model}, we consider a single active agent with limited sensing and computational power that operates on a finite set of passive \emph{tiles} positioned at nodes of some underlying graph $G$ that has a fixed embedding in $\mathbb{R}^3$.
We define the graph and embedding in \cref{subsec:graph}, the agent model in \cref{subsec:agent}, and the coating problem in \cref{subsec:problem}.


\subsection{Underlying Graph}
\label{subsec:graph}

Let $G = (V,E)$ be the adjacency graph of equally sized closely packed spheres at each point of the infinite face-centered cubic lattice (see \cref{fig:modelgraph}).
This graph can be embedded in $\mathbb{R}^3$ such that nodes have alternating cubic coordinates, i.e., each node $v$ has a coordinate $\vec{v} = (x,y,z)$ with $x,y,z \in \mathbb{Z}$ and $x+y+z$ is even.
Each node has twelve neighbors whose relative positions are described by the directions $\UNE$, $\UW$, $\USE$, $\N$, $\NW$, $\SW$, $\S$, $\SE$, $\NE$, $\DNW$, $\DSW$ and~$\DE$, which correspond to the following vectors in~the~embedding~of~$G$:
\begin{equation*} % TODO: Check guidelines for the use of array
    \label{eq:embedding}
    \begin{array}{llllllllll}
        \vec{\UNE} & =  (1,1,0),   &  \vec{\UW}  & =  (0,1,1),   & \vec{\USE} & =  (1,0,1),   & \vec{\N}   & =  (0,1,-1),\\
        \vec{\NW}  & =  (-1,1,0),  &  \vec{\SW}  & =  (-1,0,1),  & \vec{\S}   & =  (0,-1,1),  & \vec{\SE}  & =  (1,-1,0),  \\
        \vec{\NE}  & =  (1,0,-1),  &  \vec{\DNW} & =  (-1,0,-1), & \vec{\DSW} & =  (-1,-1,0), & \vec{\DE}  & =  (0,-1,-1).
    \end{array}
\end{equation*}

Cells in the dual graph of $G$ w.r.t.\ the above embedding have the shape of rhombic dodecahedra, i.e., polyhedra with 12 congruent rhombic faces (see \cref{fig:modeldirections}).
This is also the shape of every cell in the Voronoi tessellation of $G$, i.e., that shape completely tessellates 3D space.
Consistent to the embedding, we denote by $v + \X$ the node $w$ that is neighboring $v$ in some direction $\X$, i.e., $\vec{w} = \vec{v} + \vec{\X}$.
%and by $-\X$ the opposite compass direction of $\X$, e.g., $\UNE = -\DSW$.
Consider a finite set of tiles of $k$ distinguishable types (until \cref{sec:construction} we only consider $k=1$).
Tiles have the shape of rhombic dodecahedra and are passive, in the sense that they cannot perform computation or movement on their own.
A node $v \in V$ is either \emph{tiled}, if there is a tile positioned at $v$, or \emph{empty}, otherwise.
Except for the material depot (which we introduce in \cref{subsec:problem}), nodes can hold at most one tile at a time.
We denote by $\occ{}$ the set of tiled nodes, and by $\emp{}$ the (infinite) set of empty nodes.




\subsection{Agent Model}
\label{subsec:agent}

The agent $r$ is the only active entity in this model.
It can place and remove tiles of any type at nodes of $G$ and loses and gains a unit of material in the process.
We assume that it initially carries no material and that it can carry at most one unit of material at any time.
The agent has the computational capabilities of a deterministic finite automaton performing \emph{Look-Compute-Move} cycles.
In the \emph{look}-phase, it observes tiles at its current position $p$ and the twelve neighbors of $p$, and if there are tiles, it observes their types as well.
%Further, it observes whether a neighboring node belongs to the impassable object (which we introduce in \cref{subsec:problem}).
The agent is equipped with a compass that allows it to distinguish the relative positioning of its neighbors.
Its initial rotation and chirality can be arbitrary, but we assume that it remains consistent throughout the execution.
%Based on the information on nodes within its limited vision range, 
In the \emph{compute}-phase the agent determines its next state transition according to the finite automaton.
In the \emph{move}-phase, the agent executes an \emph{action} that corresponds to that state transition.
It either (i) moves to an empty or tiled node adjacent to $p$, (ii) places a tile (of any type) at $p$, if $p \notin \occ{}$ and $r$ carries material (we call that \emph{tiling} node $p$), (iii) removes a tile from $p$, if $p\in \occ{}$ and $r$ carries no material, (iv) changes the tile type at $p$, or (v) terminates.
During (ii) and (iii), the agent loses and gains one unit of material, respectively.
%Note that action (iv) is only relevant if we assume $k > 1$ types of passive tiles.
%Note that the agent can move through both empty and tiled nodes. %, but it may never violate the connectivity constraint.
While the agent is technically a finite automaton, we describe algorithms from a higher level of abstraction textually and through pseudocode using a constant number~of variables of constant size domain.
%It is easy to see that a constant number~of variables of constant size domain can be incorporated into the agent's constantly many states.



\subsection{Problem Statement}
\label{subsec:problem}

%Consider a connected subset $\obj{}\subset V$ of nodes, called \emph{object}, and denote by $\emp{}$ the (infinite) set of empty nodes.
%Any node is either an object node, empty or tiled such that $\obj{}$, $\emp{}$ and the set of occupied nodes $\occ{}$ are pairwise disjoint.
%%In contrast to occupied and empty nodes, 
%We assume object nodes to be impassable and static, i.e., $\obj{}$ does not change throughout execution.
%Further, we assume that holes in the object are sufficiently large.
%To be precise, we assume that $d_\obj{}(v,w) \leq 2$ for any $v,w \in \obj$ with $d(v,w) \leq 2$, where $d(v,w)$ denotes the hop-distance (length of the shortest path) between $v$ and $w$ in $G$, and $d_\obj{}(v,w)$ the hop-distance in the subgraph $G(\obj{})$ of $G$ induced by nodes of $\obj{}$.

Denote by $G(W)$ the subgraph of $G$ induced by some set of nodes $W \subseteq V$, by $d(v,w)$ the distance (length of the shortest path) between nodes $v,w \in V$ w.r.t. $G$, and by $d_W(v,w)$ the distance w.r.t. $G(W)$.
Consider a connected subset $\obj{}\subset V$ of impassable and static nodes, called \emph{object}.
Any node is either an object node, empty or tiled such that $\obj{}$ and the sets of empty nodes $\emp{}$ and tiled nodes $\occ{}$ are pairwise disjoint.
A \emph{configuration} is the tuple $C = (\occ{},\obj, p)$, and we call $C$ \emph{valid}, if $G(\occ{} \cup \obj{} \cup \{p\})$ is connected.
%In contrast to tiled and empty nodes, 
%Object nodes are impassable by the agent and static, i.e., $\obj{}$ does not change throughout execution.
%If $v \notin W$ or $w \notin W$, define $d_W{}(v,w) = \infty$.
We assume that holes in the object have width larger than one, i.e., $d_\obj{}(v,w) \leq 2$ for any $v,w \in \obj$ with $d(v,w) \leq 2$.

%This constraint prevents the agent, tiles and the object from drifting apart.
Let $C^0 = (\occ{}^0,\obj{}, p^0)$ be a valid initial configuration with $\occ{}^0=\{p^0\}$.
Superscripts~generally refer to step numbers and may be omitted if they are clear from the context.
Define the \emph{coating layer} as the maximum subset $L\subset V \setminus \obj{}$ such that for each node $v \in L$ there is an object node $w \in \obj$ with $d(v,w) = 1$ and $d_L(v,p^0) < \infty$ (w.r.t. $G(L)$).
The latter condition excludes unreachable nodes that are separated by the object, e.g., the inner surface of a hollow sphere.
We assume a \emph{material depot} at the agent's initial position $p^0$, that is $p^0$ is a node with the special property of holding at least $|L|$ units of material.
An algorithm solves the \emph{coating problem}, if its execution results in a sequence of valid configurations $C^0,\dots,C^{t^*}$ such that $\occ{}^{t^*} = L$, $C^{t}$ results from $C^{t-1}$ for $1\leq t \leq {t^*}$ by applying some action (i)--(iv) to $p^{t-1}$, and the agent terminates (v) in step ${t^*}$.