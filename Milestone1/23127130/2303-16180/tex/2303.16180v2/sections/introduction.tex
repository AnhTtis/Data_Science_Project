\section{Introduction}
\label{sec:introduction}
Recent advances in the field of molecular engineering gave rise to a series of computing DNA robots that are capable of performing simple tasks on the nano scale, including the transportation of cargo, communication, movement on the surface of membranes, and pathfinding~\cite{dnaCargo, dnaCargoCollective, dnaWalkMembrane, dnaPathFinding}. % communication with other robots
These results foreshadow future technologies in which a collective of computing particles cooperatively act as programmable matter - a homogenous material that changes its shape and physical properties in a programmable fashion.
Robots may be deployed in the human body as part of a medical treatment: 
they may repair tissues by covering wounds with proteins or apply layers of lipids to isolate pathogens. 
The common thread uniting these applications is the \emph{coating problem}, in which a thin layer of some specific substance is applied to the surface of a given object.

In the past decades, a variety of models for programmable matter has been proposed, primarily distinguished between passive and active systems.
In passive systems, particles %do not perform any computation on their own; 
move and bond to each other solely by external stimuli, e.g., current or light, or by their structural properties, e.g., specific glues on the sides of the particle.
Prominent examples are the DNA tile assembly models aTAM, kTAM and 2HAM (see survey in~\cite{tileAssemblySurvey}).
In contrast, particles in active systems solve tasks by performing computation and movement on their own.
Noteworthy examples include the Amoebot model, modular self-reconfigurable robots and swarm robotics~\cite{amoebotAnnouncement, MSR, MSR2, swarmRobotics}.
While computing DNA robots are difficult to manufacture, passive tiles can be folded from DNA strands efficiently in large quantities~\cite{dnaTileSurvey}.
A trade-off between feasibility and utility is offered by the hybrid model for programmable matter~\cite{hybridShapeFormation, hybridShapeRecognition, hybrid3D}, in which a single active agent that acts as a deterministic finite automaton operates on a large set of passive particles (called tiles) serving as building blocks.
A key advantage of the hybrid approach lies in the reusability of the agent upon completing a task.
While agents in purely active systems are expended as they become part of the formed structure, hybrid agents can be deployed again.
This property is beneficial in scenarios requiring the coating of numerous objects, such as isolating malicious cells within the human body.
Coating multiple objects concurrently, with each being individually coated by a single agent, allows for efficient~pipelining.

In the 3D hybrid model, we consider tiles of the shape of rhombic dodecahedra, i.e., polyhedra with 12 congruent rhombic faces, positioned at nodes of the adjacency graph of face-centered cubic (FCC) stacked spheres (see \cref{fig:modelgraph,fig:modeldirections}).
In contrast to rectangular graphs (e.g., \cite{cadbots}), this allows the agent to fully revolve around tiles without losing connectivity, which prevents the agent and tiles from drifting apart, e.g., in liquid or low gravity environments.
In this paper, we investigate the coating problem in the 3D hybrid model, in which the goal is to completely cover the surface of some impassable object with tiles, where tiles can be gathered from a material depot somewhere on the object's~surface.
%We assume that the agent is initially positioned at a material depot somewhere on the object's surface, and that it can spend material to assemble tiles of $k$ distinguishable types.
%Material may be gathered from the depot or by disassembling a tile, and the agent can carry material for at most one tile assembly at any time.


\subsection{Our Results}
\label{subsec:contribution}

We present a generalized algorithm that solves the coating problem assuming that the agent operates on a graph $\tri{}$ of size $n$ and degree $\Delta \leq 6$ that is a triangulation of a closed 3D surface.
We assume a fixed embedding of $\tri{}$ in $\mathbb{R}^3$ in which edges have constantly many possible orientations, and that the boundary of each node in $\tri{}$ is a chordless cycle.
Our algorithm requires only a single type of passive tiles and solves the coating problem in $\O(n^2)$ steps, which is worst-case optimal compared to an algorithm for an agent with global knowledge and no restriction on its memory or the number of tile types. % TODO: grammar: single type of passive tile? tiles?
In the 3D hybrid model, the surface graph arises as the subgraph induced by nodes adjacent to a given object (some subset of nodes) where we assume holes in the object to be sufficiently large.
While that subgraph is not necessarily a triangulation with the properties described above, we show that our algorithm can be emulated on a restricted class of objects with a single type of tiles.
To realize the algorithm outside of that class, we construct a virtual surface graph on which our algorithm can be emulated in $\O(\Delta^2n^2)$ steps using $2^{2\Delta}$ types of passive tiles.
Notably, $\Delta$ is a constant the 3D hybrid model.
%This technique may be of independent interest.

The paper is structured as follows:
We formally define the 3D hybrid model and the coating problem in \cref{sec:model}.
In \cref{sec:algorithm}, we describe and analyze the generalized algorithm, and in \cref{sec:construction}, we show how the algorithm can be realized in the 3D hybrid model.



\subsection{Related Work}
\label{subsec:relatedwork}

In recent years much work on the 2D version of the hybrid model has been carried out, yet the only publication that considers the 3D variant is a workshop paper from EuroCG 2020 in which an arbitrary configuration of $n$ tiles is rearranged into a line in $\O(n^3)$ steps \cite{hybrid3D}. 
2D shape formation was studied in~\cite{hybridShapeFormation}; the authors provide algorithms that build an equilateral triangle in $\O(nD)$ steps where $D$ is the diameter of the initial configuration.
%Additionally, they provide algorithms for intermediate structure shape formation, e.g., forming a line or a block (a dense, hole-free structure).
The problem of recognizing parallelograms of a specific height to length ratio was studied in~\cite{hybridShapeRecognition}.
The most recent publication \cite{hybridLine} solves the problem of maintaining a line of tiles in presence of multiple agents and dynamic failures of the tiles.
Closely related to the hybrid model is the well established Amoebot model, in which computing particles move on the infinite triangular lattice via a series of expansions and contractions.
In this model, a variety of problems was researched in the last years, including convex hull formation~\cite{amoebotConvexity}, shape formation~\cite{amoebotShape1, amoebotShape2}, and leader election~\cite{amoebotLeader}.
A~recent extension considers additional circuits on top of the Amoebot structure which results in a significant speedup for fundamental problems~\cite{amoebotCircuits}.
In~\cite{amoebotCoating2, amoebotCoating}, the authors solve the coating problem in the 2D Amoebot model;
in their variant, the objective is to apply multiple layers of coating to the object. % such that each computing particle is eventually contained in some layer of coating.
In \cite{amoebot3Dcoating}, the authors provide a greedy algorithm that solves the coating problem in the 3D Amoebot model.
In their approach, the object's surface is flooded by amoebots that remain connected in a tree structure. 
The process terminates for each amoebot when there are no more surrounding empty positions to move to.
It is noteworthy that, given our agent's requirement to retrieve a tile after each placement, their approach cannot be applied or transferred to the hybrid model.

In the field of modular reconfigurable robots, coating is often part of the shape formation problem.
In the 3D Catom model, a module of robots first assembles into a scaffolding~\cite{claytronicsScaffold} that is then coated by another module of robots~\cite{claytronicsCoating}.
The robots have spherical shape and reside in the FCC lattice; 
in contrast to the hybrid model, they assume more powerful computation, sensing and communication capabilities.
The problem of leader election and local identifier assignment by generic agents in the FCC lattice is considered in~\cite{3dLeaderElection}.
%; the authors provide an algorithm that requires only constant memory space but is restricted to specific initial shapes.
Coating is approached differently in the field of swarm robotics where robots form a non-uniform spatial distribution around objects that are too heavy to be lifted alone~\cite{swarmRobotics}.

