\section{Model and Problem Statement}
\label{sec:model}

In the \emph{3D hybrid model}, we consider a single active agent with limited sensing and computational power that operates on a finite set of passive \emph{tiles} positioned at nodes of some underlying graph embedded in $\mathbb{R}^3$.
We define this graph in \cref{subsec:graph}, the agent model in \cref{subsec:agent}, and the coating problem in \cref{subsec:problem}.


\subsection{Underlying Graph}
\label{subsec:graph}

Let $G = (V,E)$ be the adjacency graph of equally sized closely packed spheres at each point of the infinite face-centered cubic lattice (see \cref{fig:modelgraph}).
This graph can be embedded in $\mathbb{R}^3$ such that nodes have alternating cubic coordinates, i.e., each node $v$ has a coordinate $\vec{v} = (x,y,z)$ with $x,y,z \in \mathbb{Z}$ and $x+y+z$ is even.
Each node has precisely twelve neighbors whose relative positions are described by the compass directions $\UNE$, $\UW$, $\USE$, $\N$, $\NW$, $\SW$, $\S$, $\SE$, $\NE$, $\DNW$, $\DSW$ and~$\DE$. 
%(see \cref{fig:modeldirections}). 
In~the embedding of $G$, compass directions correspond to vectors as follows:
\begin{equation*} % TODO: Check guidelines for the use of array
    \label{eq:embedding}
    \begin{array}{lclclclclclclcl}
        \vect{\UNE} & = & (1,1,0)   & \;\; & \vect{\UW}  & = & (0,1,1)   & \;\; & \vect{\USE} & = & (1,0,1)   & \;\; & \vect{\N}   & = & (0,1,-1)\\
        \vect{\NW}  & = & (-1,1,0)  & \;\; & \vect{\SW}  & = & (-1,0,1)  & \;\; & \vect{\S}   & = & (0,-1,1)  & \;\; & \vect{\SE}  & = & (1,-1,0)\\
        \vect{\NE}  & = & (1,0,-1)  & \;\; & \vect{\DNW} & = & (-1,0,-1) & \;\; & \vect{\DSW} & = & (-1,-1,0) & \;\; & \vect{\DE}  & = & (0,-1,-1)
    \end{array}
\end{equation*}
Cells in the dual graph of $G$ w.r.t.\ the above embedding have the shape of rhombic dodecahedra, i.e., polyhedra with 12 congruent rhombic faces (see \cref{fig:modeldirections}).
This is also the shape of every cell in the Voronoi tessellation of $G$, i.e., that shape completely tessellates 3D space.
Consistent to the embedding, we denote by $v + \X$ the node $w$ that is neighboring $v$ in some direction $\X$, i.e., $\vec{w} = \vec{v} + \vect{\X}$.
%and by $-\X$ the opposite compass direction of $\X$, e.g., $\UNE = -\DSW$.
Consider a finite set of tiles of $k$ distinguishable types. %(until \cref{sec:construction} we consider $k=1$),
Tiles have the shape of rhombic dodecahedra and are passive, in the sense that they cannot perform computation or movement on their own.
Each node $v \in V$ is either \emph{occupied}, if there is a tile positioned at $v$, or \emph{empty}, otherwise.
Except for a dedicated material depot node (which we will introduce in \cref{subsec:problem}), each tile occupies at most one node and each node is occupied by at most one tile.

\begin{figure}[t] % TODO: change to t
    \centering
    \begin{minipage}{.47\textwidth}
        \centering
        \includegraphics[height=.71\linewidth]{modelgraph}
        \captionof{figure}{Occupied nodes of $G$ and their incident edges to empty nodes.}
        \label{fig:modelgraph}
    \end{minipage}%
    \hfill
    \begin{minipage}{.47\textwidth}
        \centering
        \includegraphics[height=.71\linewidth]{modeldirections}
        \captionof{figure}{A passive tile (rhombic dodecahedron) and the compass directions.}
        \label{fig:modeldirections}
    \end{minipage}
\end{figure}


\subsection{Agent Model}
\label{subsec:agent}

The agent $r$ is the only active entity in this model.
It can place and remove tiles of any type at nodes of $G$ and loses and gains a unit of material in the process.
We assume that it initially carries no material and that it can carry at most one unit of material at any time.
The agent has the computational capabilities of a deterministic finite automaton performing \emph{Look-Compute-Move} cycles.
In the \emph{look}-phase, it observes tiles at its current position $p$ and the twelve neighbors of $p$, and if there are tiles, it observes their types as well.
Further, it observes whether a neighboring node belongs to the impassable object (which we introduce in \cref{subsec:problem}).
The agent is equipped with a compass that allows it to distinguish the relative positioning of its neighbors.
Its initial rotation and chirality can be arbitrary, but we assume that it remains consistent throughout the execution.
%Based on the information on nodes within its limited vision range, 
In the \emph{compute}-phase the agent determines its next state transition according to the finite automaton.
In the \emph{move} phase, the agent performs an \emph{action} that corresponds to the last state transition.
It either (i) moves to a node adjacent to $p$, (ii) places a tile (of any type) at $p$, if $p \notin \occ{}$ and $r$ carries material, (iii) removes a tile from $p$, if $p\in \occ{}$ and $r$ carries no material, (iv) changes the tile type at $p$, or (v) terminates.
During actions (ii) and (iii), the agent loses and gains one unit of material, respectively.
%Note that action (iv) is only relevant if we assume $k > 1$ types of passive tiles.
Note that the agent can move through both empty and occupied nodes. %, but it may never violate the connectivity constraint.
While the agent is technically a finite automaton, we describe algorithms from a higher level of abstraction textually and through pseudocode.
It is easy to see that a constant number of variables of constant size domain can be incorporated into the agent's constantly many states.



\subsection{Problem Statement}
\label{subsec:problem}

Consider a connected subset $\obj{}\subset V$ of nodes, called \emph{object}, and denote $\emp{}$ the (infinite) set of empty nodes.
Any node is either an object node, empty or occupied such that $\obj{}$, $\emp{}$ and the set of occupied nodes $\occ{}$ are pairwise disjoint.
%In contrast to occupied and empty nodes, 
We assume object nodes to be impassable and static, i.e., $\obj{}$ does not change throughout execution.
Further, we assume that $\obj$ contains no tunnels of width one.
To be precise, denote $G(W)$ the subgraph of $G$ induced by some set of nodes $W \subseteq V$, and let $d(v,w)$ be the hop-distance between nodes $v,w \in V$ (w.r.t. $G$), and $d_W(v,w)$ the hop-distance w.r.t. $G(W)$.
If $v \notin W$ or $w \notin W$, define $d_W{}(v,w) = \infty$.
We assume that $d_\obj{}(v,w) \leq 2$ for any $v,w \in \obj$ with $d(v,w) \leq 2$.

A \emph{configuration} $C = (\occ{},\obj, p)$ is the tuple containing $\occ{}$, $\obj{}$, and the agent's position $p\in V$.
A configuration is \emph{valid}, if $G(\occ{} \cup \obj{} \cup \{p\})$ is connected.
This connectivity constraint prevents the agent and tiles to drift apart from the object.

Consider an initially valid configuration $C^0 = (\occ{}^0,\obj{}, p^0)$ with $\occ{}^0=\{p^0\}$.
Note that superscripts generally refer to step numbers and may be omitted if the context is clear.
Define the \emph{coating layer} as the maximum subset $L\subset V \setminus \obj{}$ such that for each $v \in L$ there is a node $w \in \obj$ with $d(v,w) = 1$ and $d_L(v,p^0) < \infty$ (w.r.t. $G(L)$).
The latter condition excludes unreachable nodes that are separated by the object, e.g., the inner surface of a hollow sphere.
We assume a \emph{material depot} at $p^0$, that is $p^0$ is a node with the special property of holding at least $|L| - 1$ additional units of material.
An algorithm for a finite-state agent solves the \emph{coating problem}, if its execution results in a finite sequence of valid configurations $C^0,\dots,C^{t^*}$ such that $\occ{}^{t^*} = L$, $C^{t}$ results from $C^{t-1}$ for $1\leq t \leq {t^*}$ by applying some action (i)--(iv) to $p^{t-1}$, and the agent terminates (v) after step ${t^*}$.