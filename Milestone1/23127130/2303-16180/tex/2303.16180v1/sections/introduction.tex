\section{Introduction}
\label{sec:introduction}
Recent advances in the field of molecular engineering gave rise to a series of computing DNA robots that are capable of performing simple tasks on the nano scale, including the transportation of cargo, communication, movement on the surface of membranes, and pathfinding~\cite{dnaCargo, dnaCargoCollective, dnaWalkMembrane, dnaPathFinding}. % communication with other robots
These results foreshadow future technologies in which a collective of computing particles cooperatively act as programmable matter - a homogenous material that changes its shape and physical properties in a programmable fashion.
Robots may be deployed in the human body as part of a medical treatment: %or to assist its physiological functions
they may repair tissues by covering wounds with proteins or apply layers of lipids isolate pathogens. %to protect membranes or 
The common thread uniting these applications is the \emph{coating problem}, in which a thin layer of some specific substance is applied to the surface of a given object.

In the past decades, a variety of models for programmable matter has been proposed, primarily distinguished based on whether the particles are active or passive.
In passive systems, particles %do not perform any computation on their own; 
move and bond to each other solely by external stimuli, e.g., current or light, or by their structural properties, e.g., specific glues on the sides of the particle.
Prominent examples are the DNA tile assembly models aTAM, kTAM and 2HAM (see survey in~\cite{tileAssemblySurvey}).
In contrast to this, particles in active systems solve tasks by performing computation and movement on their own.
Noteworthy examples include the Amoebot model, modular self-reconfigurable robots and swarm robotics~\cite{amoebotAnnouncement, MSR, MSR2, swarmRobotics}.
While computing DNA robots are difficult to manufacture, simple passive tiles that are folded from DNA strands can already be synthesized efficiently in large quantities~\cite{dnaTileSurvey}.
A trade-off between feasibility and utility is offered by the hybrid model for programmable matter~\cite{hybridShapeFormation, hybridShapeRecognition, hybrid3D}, in which a single active agent with the computational capacity of a deterministic finite automaton operates on a large set of passive particles (called tiles) that serve as building blocks.
We consider tiles of the shape of rhombic dodecahedra, i.e., polyhedra with 12 congruent rhombic faces, positioned at nodes of the adjacency graph of face-centered cubic (FCC) stacked spheres.
In contrast to models with a rectangular graph (e.g., \cite{cadbots}), this allows the agent to fully revolve around tiles without losing connectivity, which prevents the agent and tiles to drift apart, e.g., in liquid or low gravity environments.
In this paper, we investigate the coating problem in the 3D hybrid model, in which the goal is to completely cover the surface of some impassable object with tiles, where tiles can be gathered from a material depot somewhere on the object's surface.
%We assume that the agent is initially positioned at a material depot somewhere on the object's surface, and that it can spend material to assemble tiles of $k$ distinguishable types.
%Material may be gathered from the depot or by disassembling a tile, and the agent can carry material for at most one tile assembly at any time.


\subsection{Our Results}
\label{subsec:contribution}

We present a generalized algorithm that solves the coating problem assuming that the agent operates on a graph that is a triangulation with constant degree of a closed 3D surface in which the boundary of each node is a chordless cycle, and edges have constantly many possible orientations.
We solve the coating problem on that graph of size $n$ in $\O(n^2)$ steps with a single type of passive tile, which is worst-case optimal compared to an algorithm for an agent with global knowledge and no restriction on its memory or the number of tile types.

In the 3D hybrid model, we only consider surfaces of objects that contain no tunnels or holes of width one.
We show that there is a restricted class of such surfaces that are directly coated by our algorithm.
To realize the algorithm on any other surface of size $n$ and degree $\Delta$, we construct a virtual surface graph on which our algorithm can be emulated in $\O(\Delta^2n^2)$ steps using $2^{2\Delta}$ types of passive tiles.
This technique may be of independent interest.

A formal introduction to the 3D hybrid model and the coating problem is given in \cref{sec:model}.
In \cref{sec:algorithm}, we describe and analyze the generalized coating algorithm, and in \cref{sec:construction}, we realize the algorithm in the 3D hybrid model.



\subsection{Related Work}
\label{subsec:relatedwork}

Shape formation in the 2D hybrid model was studied in~\cite{hybridShapeFormation}; the authors provide algorithms that rearrange a configuration of $n$ tiles and diameter $D$ into an equilateral triangle in $\O(nD)$ steps.
%first into an intermediate block (a dense, hole-free structure) and then 
The problem of recognizing parallelograms of a specific height to length ratio was studied in~\cite{hybridShapeRecognition}.
Interestingly, an agent without a pebble (recognizable marker) can only detect linear ratios, while a single pebble allows the agent to detect polynomials of constant degree.
The first research on the 3D hybrid model considers the problem of rearranging a configuration into a line in $\O(n^3)$ steps and was published at the EuroCG workshop in 2020~\cite{hybrid3D}.
Closely related to the hybrid model is the well established Amoebot model, in which computing particles move on the infinite triangular lattice by performing a series of expansions and contractions.
In this model, a variety of problems was researched in the last years, including convex hull formation~\cite{amoebotConvexity}, shape formation~\cite{amoebotShape1, amoebotShape2}, and leader election~\cite{amoebotLeader}.
A~recent extension considers circuits on top of the Amoebot structure that allow the instant transmission of signals, which results in a significant speedup for various fundamental problems~\cite{amoebotCircuits}. % such as leader election and shape recognition
In~\cite{amoebotCoating2, amoebotCoating}, the authors solve the coating problem in the 2D Amoebot model;
in their variant, the objective is to apply multiple layers of coating to the object. % such that each computing particle is eventually contained in some layer of coating.
In the field of modular reconfigurable robots, coating is often part of the shape formation problem.
In the 3D Catom model, a collective of modular robots first assembles into a scaffolding~\cite{claytronicsScaffold} that is then coated by another module of robots~\cite{claytronicsCoating}.
The robots have spherical shape and reside in the FCC lattice; 
in contrast to the hybrid model, they solely move via rotation and have more powerful computation, sensing and communication capabilities.
A different approach to the coating problem is considered in the field of swarm robotics, where the objective is a non-uniform spatial distribution of robots around objects that are too heavy to be lifted alone~\cite{swarmRobotics}.
The problem of leader election by generic agents in the FCC lattice is considered in~\cite{3dLeaderElection}; the authors provide an algorithm that requires only constant memory space but is restricted to specific initial shapes.

%The problem of leader election and local identifier assignment by generic agents in the FCC lattice is considered in~\cite{3dLeaderElection}; the authors provide two algorithms, one of which requires only constant memory space and allows arbitrary initial rotations of the agents but works only for specific initial shapes.
