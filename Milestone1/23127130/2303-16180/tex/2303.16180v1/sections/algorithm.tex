\section{The Coating Algorithm}
\label{sec:algorithm}



In this section, we give a generalized coating algorithm for a surface graph $\tri$ with node set $L$ that fulfills specific properties which we define below.
Note that the notation in this section is w.r.t. $\tri$ instead of $G$.
Let $W \subseteq L$ be an arbitrary subset of nodes.
Define the $i$-\emph{neighborhood} $N_i(W)$ as the set of nodes $v \in L$ with $d(v,w) \leq i$ for some $w \in W$, and the \emph{boundary} as $B(W) \coloneqq N_1(W) \setminus W$.
We write $N_i(w)$ and $B(w)$ for $W = \{w\}$, and denote subsets of only empty or only occupied nodes using a subscript, e.g., $\be(w) = B(w) \cap \emp{}$.
We assume the following properties:
(1) $\tri$ is a triangulation of a closed 3D surface (e.g., see \cref{fig:idea1}) with constantly many edge orientations, and (2) $B(v)$ is a chordless cycle for any $v \in L$, i.e., $\tri(B(v))$ contains only a single simple cycle.

\begin{figure}[t]
    \centering
    \begin{subfigure}[c]{0.49\linewidth}
        \includegraphics[width=\linewidth,trim={{0} {135} {0} {150}},clip]{idea1} % [trim={left bottom right top},clip]
        \subcaption{}
        \label{fig:idea1}
    \end{subfigure}\hfill
    \begin{subfigure}[c]{0.49\linewidth}
        \includegraphics[width=\linewidth,trim={{0} {135} {0} {150}},clip]{idea2}
        \subcaption{}
        \label{fig:idea2}
    \end{subfigure}%
    \caption{(a) Triangulation $\tri{}$ of a closed 3D surface. (b) Abstract demonstration that the link property is not sufficient for a cut node w.r.t. $\tri{}(\emp{})$ (material depot depicted as a hexagon; opaque surface area is occupied; links are red circles). }
    \label{fig:idea}
\end{figure}

%\begin{figure}[!t]
%    \centering
%    \begin{subfigure}[c]{0.33\linewidth}
%        \includegraphics[width=\linewidth,trim={{0} {135} {0} {150}},clip]{idea1} % [trim={left bottom right top},clip]
%        \subcaption{}
%        \label{fig:idea1}
%    \end{subfigure}%
%    \begin{subfigure}[c]{0.33\linewidth}
%        \includegraphics[width=\linewidth,trim={{0} {135} {0} {150}},clip]{idea2}
%        \subcaption{}
%        \label{fig:idea2}
%    \end{subfigure}%
%    \begin{subfigure}[c]{0.33\linewidth}
%        \includegraphics[width=\linewidth,trim={{0} {135} {0} {150}},clip]{idea3}
%        \subcaption{}
%        \label{fig:idea3}
%    \end{subfigure}%
%    \caption{(a) Triangulation $\tri{}$ of a closed 3D surface. (b) Visual proof that the link property is not sufficient for a cut node w.r.t. $\tri{}(\emp{})$. (c) A naive approach results in a cyclic graph $\tri(\emp)$ containing only links. The material depot is depicted as a hexagon; the opaque surface area is occupied. Links are red circles. }
%    \label{fig:idea}
%\end{figure}

From a high level perspective, the %algorithm works as follows: the agent
agent operates in graph $\tri{}$ and traverses the boundary of occupied nodes starting at the material depot by following the left-hand-rule (LHR) of labyrinth traversal until it finds a node to place its carried tile at.
It then moves back to the material depot following the right-hand-rule (RHR), gathers material and repeats the process.
Following the LHR and RHR, the agent cannot leave the connected component $\tri(\emp)$ of empty nodes that contains its current position.
Hence, it is crucial that each tile placement preserves connectivity of $\tri(\emp)$, as otherwise the agent might not find its way back to the material depot, which would result in an infinite execution sequence, or it might terminate without ever visiting some empty node.

\begin{definition}
    \label{def:1link}
    A node $v \in \emp{}$ is a \emph{link}, if $\be(v)$ is disconnected.
    A~node $v \in \emp{}$ \emph{generates} (a link) at node $w$, if placing a tile at $v$ increases $||\be(w)||$, where $||\be(w)||$ is the number of connected components of $\tri(\be(w))$.
    Similarly, $v$ \emph{consumes} (a link) at node $w$, if placing a tile at $v$ decreases $||\be(w)||$.
\end{definition}

%If placing a tile at some $v \in \emp{}$ disconnects the set of empty nodes globally, it also disconnects the set locally.
The link property is necessary for some $v \in \emp{}$ to be a cut node w.r.t. $\tri(\emp)$.
However, it is not sufficient, as there can be a simple cycle in $\tri(\emp)$ that visits $v$ as well as each connected component of $\be(v)$ (see \cref{fig:idea2}).
In fact, if the genus of the surface that is captured by $\tri$ is non-zero, and the agent never places a tile at a link, then $\tri(\emp)$ converges to a cyclic graph containing only links.
In our algorithm, we use links to build a path whose head can be found by the agent without ambiguity after retrieving material.
At the head of the path, we explore a constant size neighborhood to detect whether extending the path would introduce a cycle, in which case a tile is preemptively placed at some link.

\subsection{Preliminaries}
\label{subsec:preliminaries}

In the following, we introduce the remaining notation that is used in the detailed algorithm description as well as in the analysis.

%\subsubsection{LHR and RHR Traversal.}
%While labyrinth traversal rules are generally not applicable in 3D space, $\tri$ generalizes a graph embedded in 2D space.
The agent at node $p$ always stores a direction to some occupied node $\anchor(p) \in B(p)$, called \emph{anchor} of $p$.
Using the analogy of wall-following in a labyrinth, $\anchor(p)$ is the node at which the agent's `hand' is currently placed.
Initially, the agent moves to an arbitrary node of $B(p^0)$ and sets its anchor to $p^0$.
The agent follows the LHR by moving to the first empty node $v$ in a clockwise order of $B(p)$ starting at $a(p)$ and sets its anchor to the last occupied node between $a(p)$ and $v$ in that order afterwards (see \cref{fig:anchor}).
Following the RHR is analogous except for a counter-clockwise order.
Note that the robot's anchor does not necessarily change after each move.
Since we assume $B(p)$ to be a chordless cycle, the clockwise and counter-clockwise order on nodes of $B(p)$ are uniquely defined.
Subsequently, $\lhr{}$ and $\rhr{}$ denote the direction to the next node according to the LHR and RHR, respectively.

\begin{figure}[t]
    \centering
    \begin{subfigure}[c]{0.33\linewidth}
        \includegraphics[width=\linewidth]{anchor}
        \subcaption{}
        \label{fig:anchor}
    \end{subfigure}%
    \begin{subfigure}[c]{0.33\linewidth}
        \includegraphics[width=\linewidth]{seg}
        \subcaption{}
        \label{fig:segment}
    \end{subfigure}%
    \begin{subfigure}[c]{0.33\linewidth}
        \includegraphics[width=\linewidth]{range2}
        \subcaption{}
        \label{fig:range}
    \end{subfigure}%
    \caption{(a) An agent at $p$ together with its anchor $a(p)$ and directions $\lhr{}$ and $\rhr{}$. (b) $\seg(\bp_i), \pred(\bp_i)$ and $\suc(\bp_i)$ for some node $\bp_i \in \bp$. (c) Example of the $i$-range $R_i(p,q)$ for $i = 2$ (orange) together with $N_2(p)$ w.r.t. $\tri(\emp{} \setminus \{q\})$ (green). Occupied nodes are depicted as hexagons, empty nodes as circles, links as red.}
    \label{fig:definitions}
\end{figure}

%\subsubsection{Additional Terminology.}
Let $\start \in B(p^0)$ be a dedicated starting node chosen arbitrarily by the agent in the initialization.
Define $\bp = (\bp_0,\dots,\bp_m)$ as the path along nodes of $B(\occ{})$ according to a left-hand rule traversal that starts and ends at $\start = \bp_0 = \bp_m$. %, with $\bp_i \neq \start$ for any $0 < i < m$.
Note that $B(\occ{})$ is not necessarily a simple cycle.
In fact, any link $v$ can be contained more than once in $\bp$ since following the left-hand rule, an agent may enter $v$ multiple times with the anchor set to different occupied nodes on each visit (e.g., $\bp_i$ in \cref{fig:segment}).
%Thereby, an agent may set its anchor to multiple nodes of $\bo(v)$ during a full traversal of $\bp$ (see \cref{fig:range}).
To avoid ambiguity, $\anchor(\bp_i)$ refers to the agent's anchor \emph{after} moving from $\bp_{i-1}$ to $\bp_i$, and $\anchor(v)$ refers to $\anchor(\bp_i)$ for some $\bp_i = v$ with minimum $i$, i.e., the first visit of $v$.
We treat $\bp$ as an ordered multiset of nodes, e.g., we write $w \in \bp$, if $\bp$ contains node $w$.

Denote $\be(v,w)$ the connected component of $\tri(\be(v))$ that contains $w$.
For some node $\bp_i \in \bp$, we call $\seg(\bp_i) \coloneqq \be(\anchor(\bp_i),\bp_i)$ the \emph{segment} of $\bp_i$.
Simply put, $\seg(\bp_i)$ contains all nodes that touch the anchor $a(\bp_i)$ from the same `side'.
We call a node $\bp_j$ a \emph{successor} of some $\bp_i$, if $\bp_j \in \seg(\bp_i)$ and $j > i$.
Analogously, $\bp_j$ a \emph{predecessor} of $\bp_i$, if $j < i$.
As an example, node $\sigma_i$ in \cref{fig:segment} has two predecessors and one successor.
Denote $\suc(\bp_i)$ the node without a successor in $\seg(\bp_i)$, and $\pred(\bp_i)$ the node without a predecessor in $\seg(\bp_i)$.
Observe that $\suc(\bp_i)$ and $\pred(\bp_i)$ are the nodes at which the agent's anchor changes moving LHR.
If the agent enters $\pred(\bp_i)$ moving LHR, then its anchor is not yet set to $a(\bp_i)$, and if it moves LHR at $\suc(\bp_i)$, then its anchor changes afterwards.
%Observe that if $\pred(\bp_i) = \bp_{i-1}$ or equivalently $\suc(\bp_{i-1}) = \bp_{i-1}$, then $\anchor(\bp_i) \neq \anchor(\bp_{i-1})$.

Define $\lp = (\lp_0,\dots,\lp_l)$ as a maximal simple sub-path of $\bp$ that starts at $\start$, i.e., $\lp_i = \bp_i$ for any $0 \leq i \leq l$.
The definition of $\seg(\cdot),\suc(\cdot)$ and $\pred(\cdot)$ directly carry over from $\bp$; similarly, we treat $\lp$ as an ordered set of nodes.

Consider an agent positioned at some empty node $p$, such that the node $q = p + \rhr{}$ is a link.
The $i$-\emph{range} $R_i(p,q)$ is a specific neighborhood of empty nodes that is defined as if node $q$ were occupied (see \cref{fig:range} as an example).
Consider a node $v$ that can be reached from $p$ in at most $i$ steps without moving through $q$ or any occupied node.
If the segment $\seg(v)$ does not contain $q$, we simply add it to $R_i(p,q)$.
Otherwise, the segment is split at $q$ and only the part that contains $v$ is added to $R_i(p,q)$.
Formally, the $i$-range is defined as follows:

\begin{definition}
    \label{def:range}
    The $i$-\emph{range} of $p$ w.r.t. $q$ is the set of nodes
    \begin{flalign*}
        R_i(p,q) \coloneqq \bigcup_{v \in N_i(p)}\bigcup_{w \in \bo(v)} \be(w,v)\text{,$\,$where$\,N_i(\cdot)\,$and$\,\be(\cdot)\,$are$\,$w.r.t.}\,\tri(\emp \setminus \{q\}).
    \end{flalign*}
\end{definition}


\subsection{The Algorithm}
\label{subsec:algorithm}

The coating algorithm (see pseudocode in \cref{alg:algorithm}) consists of an initialization \init{} (lines 1--3) and two phases \coat{} (lines 4--10), dedicated to the tile placement, and \fetch{} (lines 11--16), dedicated to the gathering of material and termination.
While phase \init{} is only executed once, the agent switches between phases $\coat{}$ and $\fetch{}$ after each tile placement.
In the pseudocode, $p$ refers to the agent's position (which may change from line to line), $\links$ and $\gen{}$ denote the set of all links and generators (see \cref{def:1link}), and $\lhr{}$ and $\rhr{}$ denote the next direction of movement according to the LHR and RHR.


\begin{algorithm}[!b]
    %\scriptsize
    % % COMMENT TEMPLATE:
    %\nl \If(\Comment*[f]{comment}){if statement}{
    %    \nl commented line \Comment*[r]{comment}
    %    \nl regular line\;
    %}
    \caption{Coating Algorithm}
    \label{alg:algorithm}
    \SetAlgoVlined
    \DontPrintSemicolon
    \BlankLine
    Phase \init{}:\\
    \nl gather material from $p^0$; move to an arbitrary $\start \in B(p^0)$\;
    \nl store the direction of $p^0$ w.r.t. $\start$; $\anchor(\start)\leftarrow p^0$; $\noCheck \leftarrow false$\;
    \nl move $\lhr{}$; enter phase \coat{} \Comment*[r]{$p \leftarrow \start{} + \lhr{}$}
    \BlankLine
    Phase \coat{}:\\
    \nl \uIf{$\noCheck{}$ \normalfont{or} $R_3(p, p + \rhr{}) \cap \left(\links \cup \{\start{}\}\right) = \emptyset$}{
        \nl place a tile at $p$; $\noCheck{} \leftarrow false$; enter phase \fetch{}\;
    }
    \nl \Else{
        \nl move \rhr{} \Comment*[r]{$p \leftarrow p + \rhr{}$}
        \nl \lIf(\Comment*[f]{$p \leftarrow p + \rhr{}$}){$p \in \gen$}{move \rhr{}}
        \nl \lIf{$p \in \gen$ \normalfont{and} $p + \rhr{} \notin \links \cup \{\start{}\}$}{$\noCheck{} \leftarrow true$}
        \nl place a tile at $p$; enter phase \fetch{}\;
    }
    \BlankLine
    Phase \fetch{}:\\
    \nl \lWhile(\Comment*[f]{$p \leftarrow p + \rhr{}$}){$p \neq \start$}{
        move \rhr 
    }
    \nl move to $p^0$; gather material from $p^0$; move to $\start$\;
    \nl \lIf{$\be(\start{}) = \emptyset$}{
        place a tile at $\start$; terminate
    }
    \nl \While{$p \in \links \cup \{\start{}\}$ \normalfont{or} $v \in \links$ \normalfont{for a successor $v$ of $p$ in} $\seg(p)$}{
        \nl move \lhr \Comment*[r]{$p \leftarrow p + \lhr{}$}
    }
    \nl enter phase \coat{}\;
\end{algorithm}


In phase \init{} (lines 1--3), the agent gathers material and moves to an arbitrary node $\start\in B(p^0)$.
It stores the direction of $p^0$ w.r.t. $\start$ such that it can recognize $\start$ by the adjacent material depot at a later visit, and it initializes $\anchor{}(\start{}) \leftarrow p^0$ and $\noCheck \leftarrow false$.
Note that $\start$ is the first node of the paths $\bp$ and $\lp$, $\anchor(p)$ is the agent's anchor, and $\noCheck$ is a flag that is used in phase \coat{} which is described below.
Afterwards, the agent moves \lhr{} and enters phase \coat{}.

Phase \coat{} is always entered such that $p \notin \links \cup \{s\}$ and $p + \rhr{}$ is the last node $v\in\lp$ with $v \in \links \cup \{s\}$.
In each execution of phase \coat{} a tile is placed either directly at $p$ (lines 4--5), or at some link RHR of $p$ (lines 7--10).
In any case, the agent enters phase \fetch{} afterwards.
The position of the next tile depends on the following criteria:
If the flag $\noCheck{}$ is set to $true$, then the tile is placed at $p$ and $\noCheck{}$ is set to $false$ afterwards.
The tile is also placed at $p$, if $R_3(p,p+\rhr{})$ (see \cref{def:range}) does not contain any node of $\links \cup \{s\}$.
Otherwise, the agent places the next tile at some link RHR of $p$.
Let the agent's position w.r.t. $\bp$ be $p = \bp_i$, i.e., $\bp_{i-1},\bp_{i-2}$ and $\bp_{i-3}$ are the next three nodes RHR of $p$ (see \cref{fig:alg}).
The goal is to place the tile at a $\bp_j$ ($j \in \{i-1,i-2\}$) such that no link is generated in the connected component of $\be(\bp_j)$ that contains $\bp_{j+1}$.
If $\bp_{i-1} \notin \gen$, then the tile is placed at $\bp_{i-1}$.
Otherwise, it is placed at $\bp_{i-2}$.
In the latter case, if $\bp_{i-2} \in \gen$ and $\bp_{i-3}\notin \links$, then $\noCheck$ is set to $true$ (line 9).
The flag $\noCheck{}$ ensures that the next tile is placed at $\bp_{i-3}$ such that the agent again enters phase \coat{} LHR of the last node $v \in \lp$ with $v \in \links \cup \{s\}$ afterwards.
%In the analysis, we show that this is the only case in which $P2$ is violated for an entire execution of phase \fetch{}, however, after placing the next tile obliviously with $\noCheck$ set to true, the property holds again.

\begin{figure}[!t]
    \centering
    \begin{subfigure}[c]{0.32\linewidth}
        \includegraphics[width=\linewidth]{alg1}
        \subcaption{}
        \label{fig:alg1}
    \end{subfigure}
    \begin{subfigure}[c]{0.32\linewidth}
        \includegraphics[width=\linewidth]{alg2}
        \subcaption{}
        \label{fig:alg2}
    \end{subfigure}
    \begin{subfigure}[c]{0.31\linewidth}
        \includegraphics[width=\linewidth]{alg3}
        \subcaption{}
        \label{fig:alg3}
    \end{subfigure}%
    \caption{Examples in which the next tile is placed at some link (red) in phase \coat{}: at $\sigma_{i-1}$ in (a), and at $\sigma_{i-2}$ in (b) and (c). Only in (c), $\noCheck$ is set to $true$.}
    \label{fig:alg}
\end{figure}


In phase \fetch{}, the agent moves $\rhr{}$ until it is positioned at $\start$ (line 11), which it detects by the adjacent material depot and the direction stored in phase \init{}.
It moves to $p^0$, gathers material and returns to $\start$ (line 12).
If $\start$ has no empty neighbors, it places a tile at $\start$ and terminates (line 13).
Otherwise, the agent moves $\lhr{}$ as long as $p \in \links \cup \start$ or whenever a successor of $p$ in $\seg(p)$ is a link, and switches to phase \coat{} afterwards (lines 14--16).
Note that $p$ may have multiple successors and the agent implicitly explores $\seg(p)$ in line 14.

\subsection{Analysis}
\label{subsec:analysis}

Consider an initial configuration $C^0 = (\occ^0,\obj{},p^0)$ with $p^0 \in \occ^0 \subseteq L$, and a material depot of size at least $|L| - |\occ^0| + 1$ at $p^0$.
In the problem statement we assume that $p^0$ is the only initially occupied node, but now we allow $\occ^0$ to contain multiple occupied nodes besides $p^0$.
This will later become useful in \cref{sec:construction}.
We analyze \cref{alg:algorithm} given that $C^0$ satisfies the following definition:

\begin{definition}
    \label{def:coatability}
    A configuration $C^0 = (\occ^0,\obj{},\occ{},p^0)$ is coatable w.r.t. $\tri$, if $|B(v)| \leq 6$ for any $v \in \emp^0$, $\beo(p^0) \neq \emptyset$, $\links^0 = \emptyset$, and $\emp^0$ is connected.
\end{definition}

The five properties which we aim to maintain are the following:

\begin{itemize}
    \item[] \emph{\textbf{P1}}: Links may only occur on the simple path $\lp$, i.e., $\links \subseteq \lp$.
    \item[] \emph{\textbf{P2}}: All links are connected by a sequence of tails to the starting node, i.e., $\pred(v) \in \links \cup \{\start{}\}$ for any $v \in \links$.
    \item[] \emph{\textbf{P3}}: The subpath of $\lp$ from $\start{}$ to the last link on $\lp$ induces no cycle in $\tri$, i.e., for any $i < j \leq k$ with $\lp_k \in \links$: if $d(\lp_i,\lp_j) = 1$, then $j = i+1$.
    \item[] \emph{\textbf{P4}}: The boundary of any link contains precisely two connected components of empty nodes, i.e., $||\be(v)|| = 2$ for any $v \in \links$.
    \item[] \emph{\textbf{P5}}: There exists a node of $\lp$ at which the agent enters \coat{}, i.e., either $\links = \emptyset$ or there is an $i$ such that $\lp_i \notin \links \cup \{\start{}\}$ and $\suc(\lp_i) = \lp_i$.
\end{itemize}

Observe that all properties hold initially by $\links^0 = \emptyset$.
The structure of our proof is as follows:
We prove termination given that $P5$ is maintained, and that $\emp$ never disconnects given that $P1$--$P4$ are maintained.
Since the robot always finds a node to place the next tile at by $P5$, there must eventually be a step in which $\be(\start{}) = \emptyset$.
Since $\emp{}$ remains connected until that step, $L = \occ{}$ holds after the last tile is placed at $\start{}$.
Finally, we show that $P1$--$P5$ are maintained as invariants throughout execution.
We start with an auxiliary lemma and corollary.

\begin{lemma}
    \label{lem:generatorSharedOccupied}
    For any $v,w \in L$: If $\bo(v) \cap \bo(w) \neq \emptyset$, then $v$ cannot generate $w$.
\end{lemma}

\begin{proof}
    The lemma follows trivially if $v \notin B(w)$.
    Consider arbitrary $v,w \in L$ with $v \in B(w)$.
    Since $\tri$ is a triangulation, the edge $\{v,w\}$ is contained in precisely two triangular faces, each of which contains another node $u_1$ and $u_2$, respectively.
    Since $B(v)$ and $B(w)$ are chordless cycles, these are the only nodes adjacent to both $v$ and $w$, which implies that $B(v) \cap B(w) = \{u_1,u_2\}$ and $\{u_1,v,u_2\}$ is connected in $B(w)$.
    If any of the $u_i$ is occupied, then $||\bo(w) \cup \{v\}||\leq ||\bo(w)||$.
    Thereby, $||\be(w) \setminus \{v\}|| = ||\bo(w) \cup \{v\}|| \leq ||\bo(w)|| = ||\be(w)||$.
    \qed
\end{proof}

We now deduce the neighborhood of $v$ for the case where $v$ is both a link and a generator, i.e., $v \in \links \cap \gen$.
By \cref{def:coatability}, it holds that $|B(v)| \leq 6$, and since $v$ is a link, it holds that $|\bo(v)| \geq 2$.
If each connected component of $\be(v)$ has size at most two, then all nodes in $\be(v)$ share an occupied neighbor with $v$, which together with the previous lemma implies the following corollary:

\begin{corollary}
    \label{cor:generatorSize}
    For any $v \in \links \cap \gen$: $\bo(v)$ contains two connected components of size one, and $\be(v)$ contains two connected components of size one and three.
\end{corollary}

%We continue with the termination of the algorithm.

\begin{lemma}
    \label{lem:termination}
    If $P5$ holds in step $t$ in which the agent gathers material at $p^0$, then there is a step $t^+ > t$ in which the agent enters $p^0$ again or terminates.
\end{lemma}

\begin{proof}
    Assume by contradiction that the agent does not terminate or enter $p^0$ again in any step $t' > t$.
    There are two cases: the agent places a tile in phase \coat{} and moves to a connected component of $\emp{}$ that does not contain $\start$, or the agent never enters phase \coat{}, i.e., it moves indefinitely $\lhr{}$ in phase \fetch{}.
    If the agent traverses a simple path from $\start{}$ to some node $v$ by moving $\lhr{}$, places a tile at $v$ and moves $\rhr{}$ afterwards, it must enter the connected component of $\emp{}$ that contains $\start{}$.
    Hence, in the first case, the agent places a tile at $v$ after visiting $v$ at least twice, i.e., it has fully traversed $\lp$, which contradicts the existence of node $\lp_i$ specified by $P5$.
    In the second case, the agent never reaches $\lp_i$, as it would otherwise enter phase \coat{} and place a tile.
    This implies a cycle $(\lp_j,...,\lp_k)$ with $\lp_{k+1} = \lp_{j}$ and $j < i$, which contradicts that $\lp$ is a simple path.
    Hence, there is a step $t^+$ in which $p^0$ is entered again or $\be(\start{}) = \emptyset$ and the agent terminates.
    \qed 
\end{proof}

Subsequently, our notation refers to some step $t$ in which the agent gathers material at $p^0$.
With a slight abuse of notation we will use a superscript $^+$ ($^-$) to denote the next (previous) step $t^+$ ($t^-$) in which the agent gathers material at $p^0$ or terminates, e.g., $\emp{}^+$ denotes the set of empty nodes in step $t^+$.

\begin{lemma}
    \label{lem:enterCoat}
    If $P2$ holds in step $t$ and phase \coat{} is entered at some $\lp_i$ between step $t$ and $t^+$, then $\lp_{i-1}$ is the last node $v$ on $\lp$ with $v \in \links \cup \{\start{}\}$.
\end{lemma}

\begin{proof}
    As long as $p \in \links \cup \{\start{}\}$, the agent moves \lhr{} in phase \fetch{} such that after entering phase \coat{}, it holds that $\lp_{i-1} \in \links \cup \{\start{}\}$.
    It also moves \lhr{} if a successor of $p$ in $\seg(p)$ is a link.
    Hence, for any successor $v$ of $\lp_{i-1}$ in $\seg(\lp_{i-1})$ holds $v \notin \links$.
    Assume by contradiction that $\lp_{i-1}$ is not the last node on $\lp$ that is contained in $\links \cup \{s\}$.
    Let $\lp_k$ be the first link after $\lp_{i-1}$, i.e., $\lp_k \in \links$ and $k > i-1$.
    Since no successor of $\lp_{i-1}$ in $\seg(\lp_{i-1})$ is a link, $\lp_k$ cannot be contained in $\seg(\lp_{i-1})$.
    Let $S = (v_0, ..., v_m)$ be the sequence of nodes with $v_0 = \lp_k$, $v_m = \start$ and $v_j = \pred(v_{j-1})$ for any $0 < j \leq m$.
    The robot's anchor changes only if there is no further successor in its current segment, i.e., \emph{after} moving $\lhr{}$ at some node $v$ of $\lp$ with $\suc(v) = v$.
    This implies $\suc(\pred(v)) = \pred(v)$ for all $v \in \lp$.
    It follows that $S$ must contain some node $v_j$ with $0 < j < m$ for which $v_j = \suc(\lp_i)$.
    Since $\suc(\lp_i) \notin \links$, there must exists some $v_{j'}$ with $j' < j$ for which $\pred(v_{j'}) \notin \links \cup \{\start{}\}$ which contradicts $P2$ and concludes the lemma.
    \qed
\end{proof}

Next, we show that $\emp{}$ remains connected throughout execution.

\begin{lemma}
    \label{lem:cycles}
    If $\emp{}$ is connected and $P1$--$P4$ hold in step $t$, and a tile is placed in phase \coat{}, then $\emp^+$ is connected.
\end{lemma}

\begin{proof}
    Placing a tile at some node $v \notin \links$ cannot disconnect $\emp{}$ by \cref{def:1link}.
    This covers the cases in which the agent places a tile directly after entering phase \coat{}, especially the case where $\noCheck = true$. 
    Thus, we must only consider cases in which a tile is placed at some $v \in \links$.
    By \cref{lem:enterCoat}, the agent enters phase \coat{} at $\lp_i$ such that $\lp_{i-1}$ is the last node $w \in \lp$ with $w \in \links \cup \{\start{}\}$.
    Together with $P1$ follows that whenever it detects some node $w \in R_3(\lp_i,\lp_{i-1})$ with $w \in \links \cup \{\start{}\}$, then $w$ was visited in the previous execution of phase \fetch{}.
    Let $\lp_j$ be the last node of $\lp$ that is contained in $R_3(\lp_i,\lp_{i-1})$ with $j < i-1$, and $P$ be the shortest path from $\lp_i$ to $\lp_j$ in $\tri(R_3(\lp_i,\lp_{i-1}))$.
    Then $C = P \circ (\lp_{j+1}, \lp_{j+2},...,\lp_{i-2}, \lp_{i-1})$ is a simple cycle in $\tri$, where $\circ$ is the concatenation of paths.
    Placing a tile at $\lp_{i-1}$ or $\lp_{i-2}$ cannot disconnect $C$ since it is a cycle, and it cannot disconnect $\emp{}$ since $C$ contains nodes of distinct connected components of $\be(\lp_{i-1})$ (and $\be(\lp_{i-2})$, if $\lp_{i-2} \in \links$), and $||\be(\lp_{i-1})|| = 2$ (and $||\be(\lp_{i-2})|| = 2$, if $\lp_{i-2} \in \links$) by $P4$.
    \qed
\end{proof}

We show that $P1$--$P5$ are maintained as invariants in separate lemmas based on the value of $\noCheck{}$ and whether a tile is placed at some link or generator.

\begin{lemma}
    \label{lem:invariantSimple}
    If $P1$--$P5$ hold in step $t$, and a tile is placed at some $\lp_i \notin \links \cup \{\start{}\}$ with $\noCheck = false$, then $P1$--$P5$ hold in step $t^{+}$.
\end{lemma}  

\begin{proof}
    We first show that $P4$ holds in step $t^+$.
    The tile is not placed at a link, which implies that $\lp_i$ is the node at which the agent enters phase \coat{} and $\lp_{i-1}$ is the last node on $\lp$ that is contained in $\links \cup \{s\}$ by \cref{lem:enterCoat}.
    Since $\noCheck = false$ by assumption, $R_3(\lp_i,\lp_{i-1})$ cannot contain any link.
    Placing a tile at $\lp_i$ can only generate links in $\be(\lp_i)$, and the only possible node $w \in \be(\lp_i)$ with $||\be(w)|| > 1$ is $\lp_{i-1}$.
    Since $\lp_{i}$ shares an occupied neighbor with $\lp_{i-1}$, $\lp_{i}$ cannot increase $||\be(\lp_{i-1})||$ by \cref{lem:generatorSharedOccupied} such that $P4$ holds in step $t^+$.
    
    To show the remaining properties, we distinguish two cases based on the number of empty neighbors of $\lp_i$.
    First, consider the case $|\be(\lp_i)| > 1$.
    Let $\tilde{N}_3(\lp_i)$ be the $3$-neighborhood of $\lp_i$ w.r.t. $\tri(\emp{} \setminus \{\lp_{i-1}\})$.
    By \cref{def:range}, $R_3(\lp_i,\lp_{i-1})$ contains $\pred(v)$ for all $v \in \lp \cap \tilde{N}_3(\lp_i)$.
    $R_3(\lp_i,\lp_{i-1})$ does not contain $\start$ or any link, which together with $P2$ implies that $j > i - 1$ for any $\lp_j \in \tilde{N}_3(\lp_i) \cap \lp$.
    Thereby, the subpath of $\lp$ from $\start$ to $\lp_{i-1}$ does not change from step $t$ to $t^+$, and due to the fact that a tile is placed at $\lp_i$, each node in $\be(\lp_i)$ is contained in $\lp^+$.
    Together with $\tilde{N}_1(v) \subseteq \tilde{N}_3(\lp_i)$ for all $v \in \be(\lp_i)$, it follows that $P3$ holds in step $t^+$.
    Since each link that is generated in step $t$ is contained in $\be(\lp_i)$, $P1$ holds in step $t^+$ as well.
    It remains to show $P2$ and $P5$.
    $\lp_i$ cannot consume $\lp_{i-1}$, i.e., $\lp_{i-1} \in \links^+ \cup \{\start{}\}$, as otherwise $\lp_{i-1}$ is contained in a connected component of $\be(\lp_i)$ of size one, which would contradict $\lp_i \notin \links$ or $|\be(\lp_i)| > 1$.
    $\lp_i$ cannot generate $\lp_{i+1}$ by \cref{lem:generatorSharedOccupied}, i.e., $\lp_{i+1} \notin \links^+$.
    $\lp_{i-1}$ is the only node in $\be(\lp_i)$ without a predecessor and $\lp_{i+1}$ is the only node in $\be(\lp_i)$ without a successor.
    Then for all $v \in \be(\lp_i)$ with $v \neq \lp_{i-1}$ holds that $\pred^+(v) = \lp_{i-1} \in \links^+ \cup \{\start{}\}$  and $\suc^+(v) = \lp_{i+1} \notin \links^+$, i.e., $P2$ and $P5$ hold in step $t^+$.
    
    Second, consider the case $|\be(\lp_i)| = 1$.
    By symmetry, $\lp_i$ is a node of a connected component of $\be(\lp_{i-1})$ of size one.
    By $P4$, $\lp_i$ consumes $\lp_{i-1}$ (if $\lp_{i-1} \in \links$) and it cannot generate any link since it has no other empty neighbor.
    Then $P1$--$P4$ hold trivially in step $t^+$.
    If $\lp_i$ is a successor of $\lp_{i-1}$ in $\seg(\lp_{i-1})$ in step $t$, then $\lp_{i-1}$ has no successor in step $t^+$, i.e., $\suc(\lp_{i-1})^+ = \lp_{i-1}$ and $P5$ holds.
    \qed
\end{proof}

%Note that searching $R_2(\cdot)$ and accordingly considering $\tilde{N}_2(\cdot)$ in the proof would suffice for the previous lemma, however in the proof of the next lemma, we use the subset-relation $R_2(v,w) \subseteq R_3(\anchor(v),w)$ for any $v \in \be(\anchor(v),v)$ and reduce parts of the proof. 
The next lemma considers the case after which $\noCheck{}$ is set to $true$.
This is the only case in which $P2$ is violated in step $t^+$.
Hence, we show that the properties are maintained from step $t$ to step $t^{++}$, i.e., on the next visit of $p^0$ after $t^+$.

%\begin{figure}
%    \centering
%    \begin{tikzpicture}
%        \node (image) at (0,0) {\includegraphics[width=0.4\textwidth]{noCheck}};
%        \node (i) at (-1.3,2.55) {$\lp_i$};
%        \node (i1) at (-.4,2.9) {$\lp_{i-1}$};
%        \node (i2) at (1.9,-2.5) {$\lp_{i-2}$};
%        \node (i3) at (-1.05,-2.15) {$\lp_{i-3}$};
%        \node (u) at (2.4,-2.1) {$u$};
%        \node (w) at (0.1,-2.95) {$w$};
%        \node (x1) at (-2.2,-0.6) {$x_1$};
%        \node (x2) at (-2.2,0.98) {$x_2$};
%        \node (x2) at (-0.85,0.18) {$\anchor(\lp_i)$};
%    \end{tikzpicture}
%    \captionof{figure}{Local configuration in which a tile is placed at some $v = \lp_{i-2} \in \links\cap \gen$.}
%    \label{fig:noCheck}
%\end{figure}

\begin{lemma}
    \label{lem:invariantNoCheck}
    If $P1$--$P5$ hold in step $t$, and a tile is placed at some $\lp_i \in \links \cap \gen$ with $\lp_{i-1} \notin \links \cup \{\start{}\}$, then $P1$--$P5$ hold in step $t^{++}$.
\end{lemma}


\begin{proof}
    First, we deduce the neighborhood of $\anchor(\lp_i), \lp_i$ and $\lp_{i+1}$.
    For ease of reference, refer to \cref{fig:noCheck}.
    Since the agent places a tile at some $\lp_i \in \links \cap \gen$, $\noCheck{}$ must be $false$ in that execution of phase \coat{} and the agent moves \rhr{} twice before it places a tile and sets $\noCheck$ to $true$.
    By \cref{lem:enterCoat}, phase \coat{} is entered at $\lp_{i+2} \notin \links$ with $\lp_{i+1}, \lp_{i} \in \links \cap \gen$, and $\lp_{i+1}$ is the last $v \in \lp$ with $v \in \links \cup \{\start{}\}$.
    By \cref{cor:generatorSize}, both $\be(\lp_{i+1})$ and $\be(\lp_{i})$ contain two connected components of size one and three.
    $\lp_{i+2}$ must be contained in a connected component of $\be(\lp_{i+1})$ of size three, as otherwise $\lp_{i+2} \in \links$ which contradicts that the agent enters phase \coat{} at $\lp_{i+2}$, or $\be(\lp_{i+2}) = \{\lp_{i+1}\}$, which implies $R_3(\lp_{i+2},\lp_{i+1}) = \emptyset$ and contradicts that a tile is placed at a link.
    Thereby, $\lp_{i+2}, \lp_{i+1}, \lp_{i}$ and $\lp_{i-1}$ are all contained in the same segment $\seg(\lp_i)$.
    The lemma's assumption $\lp_{i-1} \notin \links \cup \{s\}$ implies that there must exist another node $\lp_{i-2}$ in that segment since otherwise $\lp_{i-1} = \pred(\lp_i) \notin \links \cup \{s\}$ which would contradict $P2$.
    Next, we show that $\lp_i$ and $\lp_{i+1}$ were generated by $\anchor(\lp_i)$ in some step $t' < t$.
    Initially, $\links^0 = \emptyset$, which implies that the links at $\lp_i$ and $\lp_{i+1}$ were generated by one of the two occupied nodes in $B(\lp_i) \cap B(\lp_{i+1})$.
    Let $u$ be the node that generated $\lp_i$ and $\lp_{i+1}$ in step $t'$.
    By \cref{def:coatability}, it holds that $|B(u)|\leq 6$.
    By the above deduction of the neighborhood of $\lp_i$ and $\lp_{i+1}$, $\be(u)$ contains a connected component of size four.
    Hence, $B(u) \setminus \be(u)$ is connected and contains at most two nodes.
    Since the agent never disassembles any tile, this implies that $u$ was not a link in step $t'$.
    Node $u$ generates both $\lp_i$ and $\lp_{i+1}$, and $\lp_i$ is visited before $\lp_{i+1}$ in step $t$, which implies $\anchor(\lp_i) = u$ and thus $|B(\anchor(\lp_i))| \leq 6$.
    Hence, $\seg(\lp_i)$ contains five nodes $\sigma_j$ with $j \in \{i-2,...,i+2\}$ and $\anchor(\lp_i)$ has one occupied neighbor, which precisely results in the local configuration depicted by \cref{fig:noCheck} apart from rotation.


    \begin{figure}[!t]
        \centering
        \begin{subfigure}[c]{0.33\linewidth}
            \includegraphics[width=\linewidth]{noCheck}
            \subcaption{step $t$}
            \label{fig:noCheck1}
        \end{subfigure}%
        \begin{subfigure}[c]{0.33\linewidth}
            \includegraphics[width=\linewidth]{noCheck2}
            \subcaption{step $t^+$}
            \label{fig:noCheck2}
        \end{subfigure}%
        \begin{subfigure}[c]{0.33\linewidth}
            \includegraphics[width=\linewidth]{noCheck3}
            \subcaption{step $t^{++}$}
            \label{fig:noCheck3}
        \end{subfigure}%
        \caption{Local configuration in the proof of \cref{lem:invariantNoCheck} together with the traversed path in phase \fetch{} (empty nodes depicted as circular, links depicted as red).}
        \label{fig:noCheck}
    \end{figure}
    
    Second, we consider the situation after a tile is placed at $\lp_i$ between step $t$ and $t^+$ and show that $P2$ holds in step $t^{++}$.
    Let $v$ be the node that is generated by $\lp_{i}$, and $w$ be the other node in the connected component of $\be(\lp_i)$ distinct from $\lp_{i-1}, \lp_{i+1}$ and $v$.
    As can be seen in \cref{fig:noCheck1}, $\lp_i$ consumes $\lp_{i+1}$, it generates $v$ and it cannot generate $w$ or $\lp_{i-1}$ since they share an occupied neighbor with $\lp_i$.
    Hence, in step $t^+$ the neighborhoods are precisely depicted by \cref{fig:noCheck2}.
    Since $\pred^+(v) = \lp_{i-1}$ and $\lp_{i-1} \notin \links^+$, $P2$ is violated and the agent enters phase \coat{} with $\noCheck{}$ set to $true$ at $\lp_{i-1}$ between step $t^+$ and $t^{++}$ without visiting $v$.
    In this case it places a tile at $\lp_{i-1}$ without searching for links in $R_3(\lp_{i-1},\lp_{i-2})$.
    Afterwards, it holds that $\pred^{++}(v) = \lp_{i-2}$ such that $P2$ holds again in step $t^{++}$.
    For completeness, \cref{fig:noCheck3} shows the neighborhoods w.r.t. step $t^{++}$.

    Third, we show that $P1$ and $P3$--$P5$ are maintained.
    Recall that we already concluded that $\lp_{i}$ and $\lp_{i+1}$ were generated by $\anchor(\lp_i)$ in some prior step $t' < t$.
    Note that precisely two nodes in $B(\lp_{i-1})$ are occupied in step $t^+$ (see \cref{fig:noCheck2}).
    Since $B(\anchor(\lp_i))$ contains only one occupied node in step $t$, by contraposition it follows that in step $t'$ the tile at $\anchor(\lp_i)$ is placed with $\noCheck{} = false$.
    No node from $\be(\anchor(\lp_i))$ was ever occupied prior to step $t$, and $\anchor(\lp_i)$ cannot generate $\lp_{i+2}$ in step $t'$ by \cref{lem:generatorSharedOccupied}.
    It follows that no tile was placed between step $t'$ and $t$, i.e., $t' = t^-$.
    This implies that if $\anchor(\lp_i)$ were empty in step $t$, then $R_3(\anchor(\lp_i), \lp_{i-2})$ contains neither $\start$ nor any link.
    The only links that are generated and not consumed between step $t^-$ and $t^{++}$ are contained in $\be(\lp_{i-1})$ (note that $v \in \be(\lp_{i-1})$ as well).
    Hence, $P1$ and $P3$--$P4$ hold in step $t^{++}$ analogous to the proof of \cref{lem:invariantSimple}, and $P5$ holds since $w \notin \links^{++}$ and $\suc^{++}(w) = w$.
    \qed
\end{proof}

\begin{lemma}
    \label{lem:invariantGen}
    If $P1$--$P5$ hold in step $t$, and a tile is placed at some $\lp_i \in \links \cap \gen$ with $\lp_{i-1} \in \links \cup \{\start{}\}$, then $P1$--$P5$ hold in step $t^{+}$.
\end{lemma}

\begin{lemma}
    \label{lem:invariantLinkNoGen}
    If $P1$--$P5$ hold in step $t$, and a tile is placed at some $\lp_i \in \links$ with $\lp_{i} \notin \gen$, then $P1$--$P5$ hold in step $t^{+}$.
\end{lemma}

Refer to \cref{sec:appendix} for the proof of \cref{lem:invariantGen,lem:invariantLinkNoGen}.
In the proof of \cref{lem:invariantGen}, we distinguish $\lp_{i-1} = \start$ from $\lp_{i-1} \in \links$.
In the former case, only a single link exists in step $t^+$ and that link is adjacent to $\start$ from which $P1$--$P5$ follows.
In the latter case, we again show that the tile at $\anchor(\lp_i)$ was placed with $\noCheck = false$ in step $t^-$ and the proof reduces to the proof of \cref{lem:invariantNoCheck}.
In the proof of \cref{lem:invariantLinkNoGen}, we show that the sub-path of $\lp$ from $\start$ to $\lp_{i-1}$ does not change from step $t$ to $t^+$, and that no link $\lp_j$ with $j > i-1$ exists in step $t^+$.

\begin{theorem}
    \label{thm:algorithm}
    Following \cref{alg:algorithm}, a finite-state agent solves the coating problem on $\tri$, given a configuration $C^0 = (\occ^0,\obj{},p^0)$ that is coatable w.r.t. $\tri$.
\end{theorem}

All properties hold in an intial configuration $C^0$, and they are maintained as invariants as proven in \cref{lem:invariantSimple,lem:invariantNoCheck,lem:invariantGen,lem:invariantLinkNoGen}.
By \cref{lem:enterCoat}, the agent eventually terminates, and since $\emp{}$ is initially connected by \cref{def:coatability} and cannot disconnect by \cref{lem:cycles}, $L = \occ{}$ holds afterwards, which concludes \cref{thm:algorithm}.

\subsection{Runtime Analysis}
\label{subsec:optimality}

%We now analyze the runtime of \cref{alg:algorithm} and show worst-case optimality.
%To detect whether $p \in \links$, the agent does not need to move, since it senses the occupation of adjacent nodes in its look-phase.
Since the agent does not sense any node outside of $N_1(p)$ in its look-phase, exploring $R_i(p,q)$ requires additional steps.
From $R_i(p,q) \subset N_{i+2}(p)$ follows that the required number of steps is upper bounded by $2 \cdot |N_{i+2}(p)| = \O(|N(p)|^i)$.
Since $i$ is a constant and $\tri$ has constant degree, which follows from the assumption that $\tri$ has constantly many edge orientations, each execution of phase \coat{} takes $\O(1)$ steps.
Each execution of phase \fetch{} takes $\O(|\lp|)$ steps, since the agent traverses a sub-path of $\lp$ in that phase twice.
Since $\lp$ is simple, it follows that $|\lp| = \O(n)$, where $n = |L|$.
The agent can place at most $n$ tiles before $L = \occ{}$ and thereby performs at most $n$ executions of \coat{} and \fetch{}, which gives a total runtime of $n \cdot \left(\O(1) + \O(n) \right) = \O(n^2)$ steps.

An agent $\tilde{r}$ with unlimited memory and global vision that runs an optimal algorithm is not restricted to move along the boundary of occupied nodes.
It can compute the shortest path from $p^0$ to each $v \in L$, directly move from $p^0$ to $v$ (potentially through occupied nodes) and place a tile at $v$.
Except for the last tile, it must always return to the material depot.
Hence, the last tile is placed at a node $w$ with maximum distance to $p^0$.
In the worst case, the given surface graph is the triangulation of some object resembling a straight line such that $d_L(p^0,w) = \Theta(n)$. %, with the material depot at one of its endpoints 
At each node on the shortest path $P$ from $p^0$ to $w$ a tile is placed.
Hence, $\tilde{r}$ takes at least
$\left(\sum_{u \in P} 2\cdot d_L(p^0,u)\right) - d_L(p^0,w) = \left(\sum_{i=1}^{\Theta(n)} 2 \cdot i\right) - \Theta(n) = \Theta(n^2)$ steps which implies worst-case optimality.


