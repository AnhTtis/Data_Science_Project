\section{Coating in the 3D Hybrid Model}
\label{sec:construction}

In this section, we apply the coating algorithm to the surface of objects in the 3D hybrid model.
We first give a surface graph $\tri$ that is a triangulation with degree $4 \leq \Delta \leq 8$ in which the boundary of each node is a chordless cycle.
By \cref{thm:algorithm}, our algorithm solves the coating problem only if the initial configuration is coatable w.r.t. $\tri$, which especially includes that empty nodes have degree at most six.
We show that there is a restricted class $\mathcal{S}$ of \emph{smooth objects}, for which $C^0$ is coatable w.r.t. $\tri$.
To solve the problem on surface graphs of any constant degree $\Delta > 6$, we construct a virtual graph $\tri^*$ of size at most $2 \Delta n$.
We show that \cref{alg:algorithm} can be emulated on $\tri^*$ by an agent moving on $\tri$ that utilizes $2^{2\Delta}$ types of passive tiles.
Each tile type corresponds to a bit-sequence of length $2\Delta$ that we use to encode whether nodes of $\tri^*$ are occupied or empty.

\begin{figure}[t]
    \centering
    \begin{minipage}{.48\textwidth}
        \centering
        \includegraphics[height=.8\linewidth]{coatingLayerGraph}
        \captionof{figure}{Two empty nodes (circled) that are adjacent in $G(L)$ but not in $\surf$.}
        \label{fig:coatingLayerGraph}
    \end{minipage}%
    \hfill
    \begin{minipage}{.48\textwidth}
        \centering
        \includegraphics[height=.8\linewidth]{boundary}
        \captionof{figure}{All faces of the embedding of $\surf$ that contain the centering node.}
        \label{fig:boundary}
    \end{minipage}
\end{figure}

\begin{figure}[!b]
    \centering
    \begin{tikzpicture}
        \node (image) at (0,0) {\includegraphics[width=\textwidth]{boundaries}};
        \node (a) at (-5.150,1.0) {(a)};
        \node (b) at (-3.08,1.0) {(b)};
        \node (c) at (-1.02,1.0) {(c)};
        \node (d) at (1.03,1.0) {(d)};
        \node (e) at (3.102,1.0) {(e)};
        \node (f) at (5.174,1.0) {(f)};
        \node (g) at (-5.150, -1.0) {(g)};
        \node (h) at (-3.08, -1.0) {(h)};
        \node (i) at (-1.02, -1.0) {(i)};
        \node (j) at (1.045, -1.0) {(j)};
        \node (k) at (3.102, -1.0) {(k)};
        \node (l) at (5.174, -1.0) {(l)};
        \node (m) at (-5.150,- 2.9) {(m)};
    \end{tikzpicture}
    \captionof{figure}{All possible arrangements of faces within $\surf$ except for rotation and reflection. The red dashed edges indicate vertex adjacency between its endpoints.}
    \label{fig:triangulation}
\end{figure}

\subsection{Triangulation of the Surface Graph}
\label{subsec:smoothObjects}

Recall the definition of graph $G = (V,E)$ in the 3D hybrid model and its embedding in $\mathbb{R}^3$ (see \cref{sec:model}).
If two nodes $v,w \in V$ are adjacent w.r.t. $G$, then tiles at $v$ and $w$ share a common face.
Note that in this case $|\vec{v}-\vec{w}|_2 = \sqrt{2}$, where $|\cdot|_2$ is the Euclidean distance.
We call $v$ and $w$ \emph{vertex adjacent}, if $|\vec{v}-\vec{w}|_2 = 2$, e.g., $v = w + \UNE + \SE$.
In contrast to adjacent nodes, tiles at vertex adjacent nodes only share a common vertex.
Define $\surf = (L,E')$ as the subgraph of $G(L)$ that contains only those edges $\{v,w\}$ for which $v$ and $w$ share adjacent object neighbors, i.e., $E' = \{\{v,w\}\mid d_\obj{}(N_1(v), N_1(w)) \leq 1\})$.
In contrast to $G(L)$, an agent that traverses $\surf$ must remain locally connected to the object (see \cref{fig:coatingLayerGraph}).
%The difference between $G(L)$ and $\surf$ is illustrated in \cref{fig:coatingLayerGraph}.
%The two circled empty nodes are adjacent in $G(L)$, but not in $\surf$;
%an agent that moves in $\surf$ must thereby remain locally connected to the object.

We can view $\surf$ as embedded on the surface of our 3D object.
Consider that embedding.
%It contains no cross edges, and especially, $\surf$ is planar if the object contains no holes.
It contains both triangular and tetragonal faces (see \cref{fig:boundary}), and tetragonal faces occur in one of three orientations: 
(1) $v, v+ \NE, v + \NE + \USE, v + \USE$, (2) $v, v + \NW, v + \NW + \UNE, v + \UNE$, and (3)~$v, v + \N, v + \N + \UW, v+ \UW$.
While nodes of triangular faces are pairwise adjacent, nodes on the diagonals of tetragonal faces are only vertex adjacent. 
Using (1) as an example, $v$ is vertex adjacent to $v + \NE + \USE$, and $v + \NE$ is vertex adjacent to $v + \USE$.
Apart from rotation and reflection, \cref{fig:triangulation} shows all possible arrangements of faces within $\surf$ w.r.t.\ a fixed common node $v \in L$ (centering node in the figure).
Let $\mathcal{S}$ be the class of \emph{smooth objects} that contain all objects for which $\surf$ contains only the cases (a)--(f) from \cref{fig:triangulation}.
%In order to apply the LHR and RHR traversal strategy, we require the boundary $B(v)$ of any node $v \in L$ to be connected, however, this is not the case in $\surf$, e.g., see \cref{fig:boundary}.
%In fact, if $B(v)$ is disconnected w.r.t. $\surf$, then $v$ must be a node of some tetragonal face, since they contain a node that is only vertex adjacent to $v$.
Let $\tri$ be the triangulation of $\surf$, i.e., $\tri$ equals $\surf$ except for an additional (diagonal) edge for each tetragonal face of $\surf$.
We want the agent to be able to deduce the triangulation.
Hence, the same diagonal is chosen for each tetragonal face of the same orientation (1)--(3).
%We denote $\DA, \DB$ and $\DC$ the three additional compass directions corresponding to the three added diagonal edges.
Each face of $\tri$ is triangular with pairwise adjacent nodes such that $B(v)$ is a chordless cycle w.r.t. $\tri$ for any $v \in L$.
Since $d_L(v,w) = 1$ w.r.t. $\tri$ implies $d_L(v,w) \leq 2$ w.r.t. $\surf$, the agent can emulate moving on $\tri$ with a multiplicative time and memory overhead of at most two.
Since tetragonal faces of different orientations (1)--(3) cannot contain a common edge, (d), (k) and (m) from \cref{fig:triangulation} are the only cases in which a node is contained in tetragonal faces of different orientations.
Hence, $\tri$ has degree at most six for any $\obj \in \mathcal{S}$ such that from \cref{thm:algorithm} follows:






\begin{theorem}
    \label{thm:coatableSingleType}
    A finite-state agent solves the coating problem on any object $\obj \in \mathcal{S}$ in $\O(n^2)$ steps with a single type of passive tiles, where $n = |L|$.
\end{theorem}


\subsection{Emulation of Coatable Surface Graphs}
\label{subsec:emulation}

Consider an arbitrary triangulation $\tri = (L,E)$ of constant degree $\Delta$ and an initially valid configuration $C_0$.
We construct a virtual graph $\tri^* = (L^*,E^*)$ with virtual initial configuration $C^{0*}$ such that $\tri^*$ is coatable w.r.t. $C^{0*}$.
During that construction, we define a partial surjective function $\mathcal{R}: L^* \rightarrow L$ that maps virtual nodes to real nodes.
Finally, we show that an agent $r$ that operates on $\tri$ w.r.t. $C_0$ utilizing $2^{2\Delta}$ tile types can emulate an agent $r^*$ that executes \cref{alg:algorithm} on $\tri^*$ w.r.t. $C^{0*}$ such that throughout the emulation $\mathcal{R}(p^*) = p$.


\subsubsection{Virtual Graph Construction.}


\begin{figure}[t]
    \centering
    \includegraphics[width=.55\linewidth]{emulation}
    \captionof{figure}{A triangular face $f = \{u_1,u_2,u_3\}$ of $\tri$ overlain by its corresponding virtual edges and nodes of $\tri^*$.}
    \label{fig:emulation}
\end{figure}

$\tri^*$ is the result of subdividing each face of $\tri$ into nine triangular faces (see \cref{fig:emulation}).
The node set $L^*$ contains a virtual node $v^*_u$ for each node $u \in L$, two virtual nodes $v^*_{u,w}$ and $v^*_{w,u}$ for each edge $\{u,w\} \in E$, and a virtual node $v^*_f$ for each triangular face $f$ of $\tri$.
The edge set $E^*$ contains three virtual edges $\{v^*_u,v^*_{u,w}\}$, $\{v^*_{u,w}, v^*_{w,u}\}$, $\{v^*_{w,u}, v^*_w\}$ for each edge $\{u,w\} \in E$, and six virtual edges $\{v^*_f,v^*_{u_i,u_j}\}$ for any triangular face $f = \{u_1,u_2,u_3\}$ of $\tri$.
We define $\mathcal{R}(v^*_{u,w}) = u$ for any virtual node $v^*_{u,w} \in L^*$. % that corresponds to an edge $\{u,w\}$ of $\tri$.
Consider an arbitrary but fixed order on the vectors $\vect{\X}_1,...,\vect{\X}_m$ that correspond to edges in the embedding of $\tri$.
Let $\pi$ represent that order, i.e., $\pi(\vect{\X}_i) = i$.
For some face $f = \{u_1,u_2,u_3\}$ of $\tri$, we define $\mathcal{R}(v^*_f) = u_i$, where $u_i$ is the node that minimizes $\pi(\vec{u_i} - \vec{u_j})$ for any $u_i,u_j \in f$ with $i \neq j$.
We define the virtual initial configuration $C^{0*}$ such that all $v^*_u$ are occupied, i.e., $\occ^{0*} = \cup_{u \in L} v^*_u$, $p^{0*} = v^*_{p^0}$ and assume a material depot of size at least $|L^*|-|L|$ at $v^*_{p^0}$. 

\begin{lemma}
    \label{lem:virtualGraph}
    $C^{0*}$ is coatable w.r.t. $\tri^*$.
\end{lemma}

\begin{proof}
    Each face of $\tri$ is triangular, and two virtual nodes are added for each edge of $\tri$.
    Hence, $|B(v^*_f)|= 6$ w.r.t. $\tri^*$ for any face $f$ of $\tri$.
    Any $v^*_{u,w}$ is adjacent to $v^*_{f_1}$ and $v^*_{f_2}$, where $f_1,f_2$ are the two faces of $\tri$ that both contain $u$ and $w$, to two nodes $v^*_{u,w_1}, v^*_{u,w_2}$, where $w_1 \in f_1$ and $w_2 \in f_2$, and to $v^*_{w,u}$ and $v^*_u$.
    Hence, $|B(v^*_{u,w})| = 6$ w.r.t. $\tri^*$ for any edge $\{u,w\}$ of $\tri$.
    Any other virtual node is initially occupied, which implies $|B(v^*)| \leq 6$ for any $v^* \in \emp^*$.
    
    $B(v^*)$ is chordless since all nodes in $B(v^*)$ correspond to the same or some adjacent face of $\tri$ for any $v^* \in L^*$. 
    By construction, each initially occupied node is isolated, i.e., $d(v^*,w^*) \geq 3$ for any $v^*,w^* \in \occ{}^{0*}$.
    Since $\tri^*$ is connected, it follows that $\emp^{0*}$ is connected and $\be(v^*)$ is connected for any $v^* \in \emp^{0*}$, i.e., $\links^{0*} = \emptyset$.
    Hence, each property of \cref{def:coatability} is satisfied by $C^{0*}$ w.r.t. $\tri$.
    \qed
\end{proof}

\begin{lemma}
    \label{lem:emulation}
    A finite-state agent can emulate \cref{alg:algorithm} on $\tri^*$ in $\O(\Delta^2n^2)$ steps while moving and placing tiles of at most $2^{2\Delta}$ types on $\tri$.
\end{lemma}

The proof of \cref{lem:emulation} is deferred to \cref{sec:appendix}.
The idea is to define for each node $u \in L$ a unique order on the virtual nodes $v^*$ for which $\mathcal{R}(v^*) = u$, and a bit-sequence $x(u) = (x_1,...,x_{2\Delta})$ in which $x_i$ encodes whether the $i$-th node in that order is occupied.
Utilizing $k = 2^{2\Delta}$ tile types effectively allows the agent to store $\log(k) = 2\Delta$ bits at each occupied node such that the emulation is straight forward.
Our final theorem follows from applying the virtual graph construction to the triangulation $\tri$ of $\surf$ together with $\Delta \leq 8$ and the previous lemma.

\begin{theorem}
    \label{thm:coatableManyType}
    A finite-state agent utilizing constantly many tile types can solve the coating problem on arbitrary objects in worst-case optimal $\O(n^2)$ steps.
\end{theorem}