\section{Deferred Proofs}
\label{sec:appendix}

\setcounter{lemma}{6}
\begin{lemma}
    \label{lem:invariantAppendix}
    If $P1$--$P5$ hold and a node is tiled in in step $t$, then either $P1$--$P5$ hold in step $t^+$ or the agent sets $\noCheck = true$ and $P1$--$P5$ hold in step $t^{++}$.
\end{lemma}

\begin{proof}
    \setcounter{lemma}{0}
    The lemma is proven by a case distinction on the type of node $v$ that is tiled in step $t$.
    For the sake of clarity, each case is proven individually in the following claims. 
    \begin{claim} % CLAIM 1
        If $P1$--$P5$ hold in step $t$, and a tile is placed at some $\lp_i \notin \links \cup \{\start{}\}$ with $\noCheck = false$, then $P1$--$P5$ hold in step $t^{+}$. 
    \end{claim}
    \begin{claimproof}
        We first show that $P4$ holds in step $t^+$.
        The tile is not placed at a link, which by \cref{lem:enterCoat} implies that $\lp_i$ is the node at which the agent enters phase \coat{} and $\lp_{i-1}$ is the last node on $\lp$ that is contained in $\links \cup \{\start{}\}$.
        Since $\noCheck = false$ by assumption, $R_3(\lp_i,\lp_{i-1})$ must have been searched by the agent and cannot contain any link.
        Placing a tile at $\lp_i$ can only generate links in $\be(\lp_i)$, and the only possible node $w \in \be(\lp_i)$ that is already a link, i.e., $||\be(w)|| > 1$, is $\lp_{i-1}$.
        Since $\lp_{i}$ shares a tiled neighbor with $\lp_{i-1}$, $\lp_{i}$ cannot increase $||\be(\lp_{i-1})||$ by \cref{lem:generatorSharedTiled} such that $P4$ holds in step $t^+$.

        To show the remaining properties, we distinguish two cases based on the number of empty neighbors of $\lp_i$.
        First, consider the case $|\be(\lp_i)| > 1$.
        Let $\tilde{N}_3(\lp_i)$ be the $3$-neighborhood of $\lp_i$ w.r.t. $\tri(\emp{} \setminus \{\lp_{i-1}\})$.
        By \cref{def:range}, $R_3(\lp_i,\lp_{i-1})$ contains $\pred(v)$ for all $v \in \lp \cap \tilde{N}_3(\lp_i)$.
        $R_3(\lp_i,\lp_{i-1})$ does not contain $\start$ or any link, which together with $P2$ implies that $j > i - 1$ for any $\lp_j \in \tilde{N}_3(\lp_i) \cap \lp$.
        Thereby, the subpath of $\lp$ from $\start$ to $\lp_{i-1}$ does not change from step $t$ to $t^+$, and due to the fact that a tile is placed at $\lp_i$, each node in $\be(\lp_i)$ is contained in $\lp^+$.
        Together with $\tilde{N}_1(v) \subseteq \tilde{N}_3(\lp_i)$ for all $v \in \be(\lp_i)$, it follows that $P3$ holds in step $t^+$.
        Since each link that is generated in step $t$ is contained in $\be(\lp_i)$, $P1$ holds in step $t^+$ as well.
        It remains to show $P2$ and $P5$.
        $\lp_i$ cannot consume $\lp_{i-1}$, i.e., $\lp_{i-1} \in \links^+ \cup \{\start{}\}$, as otherwise $\lp_{i-1}$ is contained in a connected component of $\be(\lp_i)$ of size one, which would contradict $\lp_i \notin \links$ or $|\be(\lp_i)| > 1$.
        $\lp_i$ cannot generate $\lp_{i+1}$ by \cref{lem:generatorSharedTiled}, i.e., $\lp_{i+1} \notin \links^+$.
        $\lp_{i-1}$ is the only node in $\be(\lp_i)$ without a predecessor and $\lp_{i+1}$ is the only node in $\be(\lp_i)$ without a successor.
        Then for all $v \in \be(\lp_i)$ with $v \neq \lp_{i-1}$ holds that $\pred^+(v) = \lp_{i-1} \in \links^+ \cup \{\start{}\}$  and $\suc^+(v) = \lp_{i+1} \notin \links^+$, i.e., $P2$ and $P5$ hold in step $t^+$.

        Second, consider the case $|\be(\lp_i)| = 1$.
        By symmetry, $\lp_i$ is a node of a connected component of $\be(\lp_{i-1})$ of size one.
        By $P4$, $\lp_i$ consumes $\lp_{i-1}$ (if $\lp_{i-1} \in \links$) and it cannot generate any link since it has no other empty neighbor.
        Then $P1$--$P4$ hold trivially in step $t^+$.
        If $\lp_i$ is a successor of $\lp_{i-1}$ in $\seg(\lp_{i-1})$ in step $t$, then $\lp_{i-1}$ has no successor in step $t^+$, i.e., $\suc(\lp_{i-1})^+ = \lp_{i-1}$ and $P5$ holds.
    \end{claimproof}
    \begin{claim}  % CLAIM 2
        If $P1$--$P5$ hold in step $t$, and a tile is placed at some $\lp_i \in \links$ with $\lp_{i} \notin \gen$, then $P1$--$P5$ hold in step $t^{+}$.
    \end{claim}
    \begin{claimproof}
        There are two cases: (1) the agent enters \coat{} at $\lp_{i+2}$ and moves $\rhr{}$ twice, i.e., $\lp_{i+1} \in \links \cap \gen$, or (2) it enters \coat{} at $\lp_{i+1}$.
        In case (1), $\lp_i$ consumes $\lp_{i+1}$ by \cref{cor:generatorSize}, as $\lp_{i}$ must be contained in a connected component of $\be(\lp_{i+1})$ of size one.
        In both cases $\lp_i \notin \lp^+$ since $\lp_i$ is tiled in step $t^+$.
        Together with \cref{lem:enterCoat}, it follows that only the last link of $\lp$ (and second last link in case (1)) is consumed and no link is generated, which implies that the sub-path of $\lp$ from $\start$ to $\lp_{i-1}$ is identical in step $t$ to $t^+$ $P1$--$P4$ hold.

        If the connected component $K$ of $\tri(\be(\lp_{i}))$ that contains $\lp_{i-1}$ has size one, then $\lp_i$ consumes $\lp_{i-1}$ by $P4$ such that $\suc^+(\lp_{i-1}) = \lp_{i-1}$ and $P5$ holds.
        Otherwise, for any $\lp_j \in K$ with $j \neq i-1$ it holds that $j > i$ by $P3$, and thereby $\lp_j \notin \links$ by $P1$.
        Together with the lemma's assumption $\lp_i \notin \gen$ follows that $\lp_j \notin \links^+$, especially $\suc(\lp_i^+) \notin \links$.
        Hence, $P5$ holds in step $t^+$.
    \end{claimproof}
    \begin{claim} % CLAIM 3
        \label{claim:firstFig}
        If $P1$--$P5$ hold in step $t$, and a tile is placed at some $\lp_i \in \links \cap \gen$ with $\lp_{i-1} \notin \links \cup \{\start{}\}$, then $P1$--$P5$ hold in step $t^{++}$.
    \end{claim}
    \begin{claimproof}
        First, we deduce the neighborhood of $\anchor(\lp_i), \lp_i$ and $\lp_{i+1}$.
        For ease of reference, refer to \cref{fig:noCheck}.
        Since the agent places a tile at some $\lp_i \in \links \cap \gen$, $\noCheck{}$ must be $false$ in that execution of phase \coat{} and the agent moves \rhr{} twice before it places a tile and sets $\noCheck$ to $true$.
        By \cref{lem:enterCoat}, phase \coat{} is entered at $\lp_{i+2} \notin \links$ with $\lp_{i+1}, \lp_{i} \in \links \cap \gen$, and $\lp_{i+1}$ is the last $v \in \lp$ with $v \in \links \cup \{\start{}\}$.
        By \cref{cor:generatorSize}, both $\be(\lp_{i+1})$ and $\be(\lp_{i})$ contain two connected components of size one and three.
        $\lp_{i+2}$ must be contained in a connected component of $\be(\lp_{i+1})$ of size three, as otherwise $\lp_{i+2} \in \links$ which contradicts that the agent enters phase \coat{} at $\lp_{i+2}$, or $\be(\lp_{i+2}) = \{\lp_{i+1}\}$, which implies $R_3(\lp_{i+2},\lp_{i+1}) = \emptyset$ and contradicts that a tile is placed at a link.
        Thereby, $\lp_{i+2}, \lp_{i+1}, \lp_{i}$ and $\lp_{i-1}$ are all contained in the same segment $\seg(\lp_i)$.
        The lemma's assumption $\lp_{i-1} \notin \links \cup \{\start{}\}$ implies that there must exist another node $\lp_{i-2}$ in that segment since otherwise $\lp_{i-1} = \pred(\lp_i) \notin \links \cup \{\start{}\}$ which would contradict $P2$.
        Next, we show that $\lp_i$ and $\lp_{i+1}$ were generated by $\anchor(\lp_i)$ in some step $t' < t$.
        Initially, $\links^0 = \emptyset$, which implies that the links at $\lp_i$ and $\lp_{i+1}$ were generated by one of the two tiled nodes in $B(\lp_i) \cap B(\lp_{i+1})$.
        Let $u$ be the node that generated $\lp_i$ and $\lp_{i+1}$ in step $t'$.
        By \cref{def:coatability}, it holds that $|B(u)|\leq 6$.
        By the above deduction of the neighborhood of $\lp_i$ and $\lp_{i+1}$, $\be(u)$ contains a connected component of size four.
        Hence, $B(u) \setminus \be(u)$ is connected and contains at most two nodes.
        Since the agent never disassembles any tile, this implies that $u$ was not a link in step $t'$.
        Node $u$ generates both $\lp_i$ and $\lp_{i+1}$, and $\lp_i$ is visited before $\lp_{i+1}$ in step $t$, which implies $\anchor(\lp_i) = u$ and thus $|B(\anchor(\lp_i))| \leq 6$.
        Hence, $\seg(\lp_i)$ contains five nodes $\sigma_j$ with $j \in \{i-2,...,i+2\}$ and $\anchor(\lp_i)$ has one tiled neighbor, which precisely results in the local configuration depicted by \cref{fig:noCheck} apart from rotation.


        \begin{figure}[!t]
            \centering
            \begin{subfigure}[c]{0.33\linewidth}
                \includegraphics[width=\linewidth]{noCheck}
                \subcaption{step $t$}
                \label{fig:noCheck1}
            \end{subfigure}%
            \begin{subfigure}[c]{0.33\linewidth}
                \includegraphics[width=\linewidth]{noCheck2}
                \subcaption{step $t^+$}
                \label{fig:noCheck2}
            \end{subfigure}%
            \begin{subfigure}[c]{0.33\linewidth}
                \includegraphics[width=\linewidth]{noCheck3}
                \subcaption{step $t^{++}$}
                \label{fig:noCheck3}
            \end{subfigure}%
            \caption{Local configuration in the proof of Claim 3 together with the traversed path in phase \fetch{} (tiled nodes are depicted as hexagons, empty nodes as circles, links are red circles).}
            \label{fig:noCheck}
        \end{figure}

        Second, we consider the situation after a tile is placed at $\lp_i$ between step $t$ and $t^+$ and show that $P2$ holds in step $t^{++}$.
        Let $v$ be the node that is generated by $\lp_{i}$, and $w$ be the other node in the connected component of $\be(\lp_i)$ distinct from $\lp_{i-1}, \lp_{i+1}$ and $v$.
        As can be seen in \cref{fig:noCheck1}, $\lp_i$ consumes $\lp_{i+1}$, it generates $v$ and it cannot generate $w$ or $\lp_{i-1}$ since they share a tiled neighbor with $\lp_i$.
        Hence, in step $t^+$ the neighborhoods are precisely depicted by \cref{fig:noCheck2}.
        Since $\pred^+(v) = \lp_{i-1}$ and $\lp_{i-1} \notin \links^+$, $P2$ is violated and the agent enters phase \coat{} with $\noCheck{}$ set to $true$ at $\lp_{i-1}$ between step $t^+$ and $t^{++}$ without visiting $v$.
        In this case it places a tile at $\lp_{i-1}$ without searching for links in $R_3(\lp_{i-1},\lp_{i-2})$.
        Afterwards, it holds that $\pred^{++}(v) = \lp_{i-2}$ such that $P2$ holds again in step $t^{++}$.
        For completeness, \cref{fig:noCheck3} shows the neighborhoods w.r.t. step $t^{++}$.

        Third, we show that $P1$ and $P3$--$P5$ are maintained.
        Recall that we already concluded that $\lp_{i}$ and $\lp_{i+1}$ were generated by $\anchor(\lp_i)$ in some prior step $t' < t$.
        Note that precisely two nodes in $B(\lp_{i-1})$ are tiled in step $t^+$ (see \cref{fig:noCheck2}).
        Since $B(\anchor(\lp_i))$ contains only one tiled node in step $t$, by contraposition it follows that in step $t'$ the tile at $\anchor(\lp_i)$ is placed with $\noCheck{} = false$.
        No node from $\be(\anchor(\lp_i))$ was ever tiled prior to step $t$, and $\anchor(\lp_i)$ cannot generate $\lp_{i+2}$ in step $t'$ by \cref{lem:generatorSharedTiled}.
        It follows that no tile was placed between step $t'$ and $t$, i.e., $t' = t^-$.
        This implies that if $\anchor(\lp_i)$ were empty in step $t$, then $R_3(\anchor(\lp_i), \lp_{i-2})$ contains neither $\start$ nor any link.
        The only links that are generated and not consumed between step $t^-$ and $t^{++}$ are contained in $\be(\lp_{i-1})$ (note that $v \in \be(\lp_{i-1})$ as well).
        Hence, $P1$ and $P3$--$P4$ hold in step $t^{++}$ analogous to Claim 1, and $P5$ holds since $w \notin \links^{++}$ and $\suc^{++}(w) = w$.
    \end{claimproof}
    
    \begin{figure}[!b]
        \centering
        \begin{minipage}[t][][b]{.63\textwidth}
            \begin{subfigure}[c]{0.5\linewidth}
                \includegraphics[width=\linewidth]{invariantGen}
                \subcaption{step $t$}
                \label{fig:invariantGen1}
            \end{subfigure}%
            \begin{subfigure}[c]{0.5\linewidth}
                \includegraphics[width=\linewidth]{invariantGen2}
                \subcaption{step $t^+$}
                \label{fig:invariantGen2}
            \end{subfigure}
            \caption{Local configuration in the proof of Claim 4. Note that while $\lp_{i-1}$ is depicted as a link (circular, red), it may be the node $\start{}$ instead.}
            \label{fig:coatingLayerGraphasd}
        \end{minipage}%
        \hfill
        \begin{minipage}[t][][b]{.315\textwidth}
            \centering
            \captionsetup[subfigure]{labelformat=empty}
            \begin{subfigure}[c]{\linewidth}
                \includegraphics[width=\linewidth]{invariantGen3}
                \subcaption{}
                \label{fig:invariantGen3}
            \end{subfigure}
            \caption{Local configuration after a tile is placed at $u_1$ with $\noCheck = true$.}
            \label{fig:boundaryasd}
        \end{minipage}
    \end{figure}
    \begin{claim}  % CLAIM 4
        If $P1$--$P5$ hold in step $t$, and a tile is placed at some $\lp_i \in \links \cap \gen$ with $\lp_{i-1} \in \links \cup \{\start{}\}$, then $P1$--$P5$ hold in step $t^{+}$.
    \end{claim}
    \begin{claimproof}
        Since a tile is placed at a generator, by \cref{lem:enterCoat} the agent enters phase \coat{} at $\lp_{i+2}$ such that $\lp_{i+1}, \lp_i \in \links \cap \gen$.
        It follows that the neighborhood of $\lp_i$ and $\lp_{i+1}$ is the same as depicted in \cref{fig:noCheck1}.
        However, since $\lp_{i-1} \in \links \cup \{\start{}\}$, the neighborhood of $\anchor(\lp_i)$ differs.
        If $\anchor(\lp_i)$ has only one tiled neighbor, then the proof reduces to the proof of Claim 3 except that $\noCheck$ is not set to $true$ and $P2$ holds directly in step $t^+$.
        Hence, we must only consider the case $|\bo(\anchor(\lp_i))| = 2$ which is precisely depicted in \cref{fig:invariantGen1} apart from rotation.
        %\cref{fig:noCheck} precisely depicts the neighborhoods of $\anchor(\lp_i), \lp_{i}$ and $\lp_{i+1}$ in that case.

        First, assume that $\lp_{i-1} = \start{}$.
        By \cref{lem:enterCoat}, $\lp_{i+2}$ is the first node $v \in \lp$ with $v\notin \links \cup \{\start{}\}$.
        Hence, $\lp_i$ and $\lp_{i+1}$ are the only existing links.
        Let $v$ be the node generated by $\lp_i$ and $w$ the node in $\be(\lp_i)$ that is not $\lp_{i+1}, \lp_i-1$ or $v$.
        As can be seen in \cref{fig:invariantGen1}, $\lp_i$ consumes $\lp_{i+1}$, it generates $v$, and by \cref{lem:generatorSharedTiled} cannot generate $w$.
        Then in step $t^+$ (see \ref{fig:invariantGen2}), $\lp^+$ is given by $\lp^+ = (\start{},v,w,...)$ with $\links^+ = \{v\}$ and $\suc(w) = w \notin \links^+$ such that $P1$--$P5$ hold.

        Second, assume that $\lp_{i-1} \neq \start{}$, i.e., by the lemma's assumption $\lp_{i-1} \in \links$.
        Note that $\be(\lp_{i-1})$ and $\bo(\lp_{i-1})$ both contain a connected component of size at least two.
        Since $\lp_{i-1}$ is empty, it holds that $|B(\lp_{i-1})| \leq 6$, and since $\lp_{i-1}$ is a link, $\be(\lp_{i-1})$ and $\bo(\lp_{i-1})$ must each contain another component of size one.
        Especially, $\bo(\lp_{i-1})$ contains no connected component of size larger than two.

        In the proof of Claim 3, we showed that apart from rotation there is only one local configuration in which a tile is placed with $\noCheck = true$.
        For ease of reference, the configuration after the tile is placed is depicted in \cref{fig:invariantGen3} with new labeling on the nodes.
        Let $u_1$ be the node at which a tile is placed with $\noCheck = true$ in that case, and $u_2$ the node from which the agent moves to $u_1$ before placing the tile.
        As can be seen in \cref{fig:invariantGen3}, $\bo(u_2)$ contains a connected component of size at least three.
        By contraposition, $\anchor(\lp_i)$ must have been placed with $\noCheck = false$ since we showed that $\bo(\lp_{i-1})$ cannot contain a connected component of size larger than two.
        In that case, the proof again reduces to the proof of Claim 3 as described above.
    \end{claimproof}
    Our claims cover all cases in which the agent places a tile according to \cref{alg:algorithm} without terminating afterwards.
    Hence, the case distinction is complete and the lemma follows.
\end{proof}

