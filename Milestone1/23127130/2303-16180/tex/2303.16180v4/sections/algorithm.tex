\section{The Coating Algorithm}
\label{sec:algorithm}

In this section, we give a generalized coating algorithm for a surface graph $\tri$ whose node set is the coating layer $L$.
The agent operates only in $\tri$ and all notation in this section is exclusively w.r.t. $\tri$.
We assume that (1)~$\tri$ is a triangulation of a closed 3D surface (e.g., \cref{fig:idea1}) with an embedding in $\mathbb{R}^3$ in which edges have constantly many possible orientations.
Define the $i$-\emph{neighborhood} $N_i(W)$ of $W$ as the set of nodes $v \in L$ with $d(v,w) \leq i$ for some node $w \in W$, and the \emph{boundary} as $B(W) \coloneqq N_1(W) \setminus W$.
We write $N_i(w)$ and $B(w)$ for $W = \{w\}$, and use subscripts to denote subsets of only empty or only tiled nodes, e.g., $\be(w) = B(w) \cap \emp{}$.
Further, we assume that (2) $\tri(B(v))$ contains precisely one simple cycle for any $v \in L$ (we say that $B(v)$ is \emph{chordless}).
Notably, this assumption does not weaken our results, since there are surface graphs in which properties (1) and (2) naturally arise, and otherwise we can emulate the properties using additional tile types (see \cref{sec:construction}).

\begin{figure}[t]
    \centering
    \begin{subfigure}[c]{0.49\linewidth}
        \includegraphics[width=\linewidth]{ideaNew1} % [trim={left bottom right top},clip]
        \subcaption{}
        \label{fig:idea1}
    \end{subfigure}\hfill
    \begin{subfigure}[c]{0.49\linewidth}
        \includegraphics[width=\linewidth]{ideaNew2} % [trim={left bottom right top},clip]
        \subcaption{}
        \label{fig:idea2}
    \end{subfigure}%
    \caption{(a) Example of a triangulation $\tri{}$. (b) The simple path $\tau$ that starts at the material depot (hexagon) along the boundary of tiled nodes (opaque surface area). Links are depicted as circles. The blue area ($i$) is the \emph{range} explored by the agent. In the yellow area ($ii$), $\tau$ is `exposed' to nodes that are not tiled and not links. The agent explores the blue area to find an overlap with the yellow area, in which case a link can be tiled while preserving connectivity of $\tri(\emp)$.}
    \label{fig:idea}
\end{figure}

%In the following section, we give a high-level overview of our algorithm together with the terminology that is essential for the pseudocode which we provide in \cref{subsec:algorithm}.
%The proof of correctness and the runtime analysis follow in \cref{subsec:analysis,subsec:optimality}.

\subsection{High-Level Description and Preliminaries}
\label{subsec:preliminaries}

The main challenge is the exploration of $\tri{}$ to find an empty node that can be tiled while maintaining a path back to the material depot, as it is the only source of additional tiles.
The exploration of all graphs of diameter $D$ and degree $\Delta$ requires $\Omega(D \log(\Delta))$ memory bits \cite{graphExploration}.
With constant memory already the exploration of plane labyrinths (grid graphs in $\mathbb{Z}^2$) requires two pebbles (placable markers) \cite{exploration1pebble,exploration2pebbles}.
Our agent has only constant memory and while it is not provided with pebbles, we can use tiles to aid the exploration of $\tri{}$.
Since at some point all nodes in $\tri{}$ must be tiled, and our algorithm uses only a single type of tiles, i.e., tiles are indistinguishable, we cannot directly use tiles as markers.
Instead, our strategy involves strategically placing tiles to create a 'narrow tunnel' of empty nodes leading to the tile depot.
We then employ the left- and right-hand-rule (LHR, RHR) commonly used in labyrinth traversal to navigate through the resulting tile structure.
In this process, the agent consistently maintains contact with the set of tiled nodes, either on its left or right.
Essentially, we `mark' empty nodes by disconnecting tiles in their boundary such that they become part of the tunnel.
This concept is formalized through the introduction of \emph{links}:

\begin{definition}
    \label{def:1link}
    A node $v \in \emp{}$ is a \emph{link}, if $\be(v)$ is disconnected.
    A~node $v \in \emp{}$ \emph{generates} (a link at) node $w$, if tiling node $v$ turns $w$ into a link.
    Similarly, $v$ \emph{consumes} (a link at) node $w$, if by tiling $v$, $w$ is no longer a link.
\end{definition}

From a high level perspective, the agent traverses the boundary of tiled nodes by following the LHR until it finds a node to place its carried tile at.
It then moves back to the material depot following the RHR, gathers material and repeats the process.
Following the LHR and RHR, the agent cannot leave the connected component of $\tri(\emp)$ that contains its position.
Hence, it is crucial that each tile placement preserves connectivity of $\tri(\emp)$, as otherwise the agent might never find its way back to the material depot, or it might terminate without tiling some node in $L$.
A naive approach to maintain connectivity would be to never place a tile at a link.
However, that strategy only works if the surface that is captured by $\tri{}$ is simply connected, i.e., it does not contain any hole.
In fact, the link property is necessary for some node $v \in \emp{}$ to be a cut node w.r.t. $\tri(\emp)$ but it is not sufficient.
If the surface contains holes, then the naive approach would converge $\tri(\emp{})$ to a cyclic graph containing only links, which again may be impossible to explore.
Hence, links must be tiled eventually.

In our algorithm, we build a simple path $\tau$ of empty nodes that contains all links and is completely contained in the boundary $B(\occ{})$ of tiled nodes (see \cref{fig:idea2}).
We cannot ensure connectivity of links on $\tau$, i.e., there can be multiple sections at which $\tau$ is `exposed' to empty nodes that are not links.
However, we can ensure that links on $\tau$ are `sufficiently close' to each other which we will formalize below by the notion of \emph{segments}.
In each traversal of $\tau$, the agent visits all links precisely once by following the LHR until there are no more links in the segment at the agent's position.
Afterwards, it explores a constant size neighborhood called \emph{range}, where we define ranges such that they contain all `close' links (all segments in some local neighborhood).
The presence of a link in that range indicates that we have found one of the above mentioned `exposures' of $\tau$ that was already visited before, i.e., there exists a link in some cycle of $\tri{}(\emp{})$ that can safely be tiled.
Note that the agent does not necessarily traverse $\tau$ fully.
As an example, the subpath of $\tau$ that follows the topmost link $v$ in \cref{fig:idea2} is not traversed, since the next segment following node $v$ contains no link.
%In this example, area $(ii)$ is the only exposure of $\tau$ that was visited before area $(i)$ is explored.

\paragraph*{Additional Terminology.}
\begin{figure}[!t]
    \centering
    \begin{subfigure}[c]{0.33\linewidth}
        \includegraphics[width=\linewidth]{anchor}
        \subcaption{}
        \label{fig:anchor}
    \end{subfigure}%
    \begin{subfigure}[c]{0.33\linewidth}
        \includegraphics[width=\linewidth]{seg}
        \subcaption{}
        \label{fig:segment}
    \end{subfigure}%
    \begin{subfigure}[c]{0.33\linewidth}
        \includegraphics[width=\linewidth]{range2}
        \subcaption{}
        \label{fig:range}
    \end{subfigure}%
    \caption{Tiled nodes are depicted as hexagons, empty nodes as disks, links are depicted in red (node $\sigma_i$ in (b), $q$ in (c) and the uppermost central node in (c)). (a) Example of the updates made to the agent's anchor $a(p)$ while following the LHR and RHR. (b) $\seg(\bp_i), \pred(\bp_i)$ and $\suc(\bp_i)$ for some node $\bp_i \in \bp$ in an example configuration. (c) Example of the $i$-range $R_i(p,q)$ for $i = 2$ (orange, opaque) together with $N_2(p)$ w.r.t. $\tri(\emp{} \setminus \{q\})$ (green, transparent).}
\end{figure}

The agent (at node $p$) maintains a direction pointer to some tiled node $\anchor(p) \in B(p)$, called \emph{anchor} of $p$.
Using the analogy of wall-following in a labyrinth, $\anchor(p)$ is the node at which the agent's `hand' is currently placed.
Initially, the agent leaves the tile depot $p^0$ to an arbitrary neighbor and sets its anchor to $p^0$.
Afterwards, it can follow the LHR (RHR) by moving to the first empty node $v$ in a clockwise (counter-clockwise) order of $B(p)$ starting at $a(p)$ and sets its anchor to the last tiled node between $a(p)$ and $v$ in that order (see \cref{fig:anchor}).
Note that the agent's anchor does not necessarily change after each move.
%Since we assume $B(p)$ to be chordless, the order on nodes of $B(p)$ is uniquely defined.
Subsequently, $\lhr{}$ and $\rhr{}$ denote the direction to the next node according to the LHR and RHR, respectively.

Let $\start \in B(p^0)$ be a dedicated starting node chosen arbitrarily in the initialization.
Define $\bp = (\bp_0,\dots,\bp_m)$ as the path along nodes of $B(\occ{})$ according to a LHR traversal that starts and ends at $\start = \bp_0 = \bp_m$. %, with $\bp_i \neq \start$ for any $0 < i < m$.
Note that $\bp$ is not necessarily a simple cycle.
In fact, in a full traversal of $\bp$, the agent can visit a link multiple times with its anchor set to different tiled nodes on each visit (e.g., $\bp_i$ in \cref{fig:segment}).
To avoid ambiguity, $\anchor(\bp_i)$ refers to the anchor \emph{after} moving from $\bp_{i-1}$ to $\bp_i$, and $\anchor(v)$ refers to the anchor at the first occurence of $v$ on $\bp$.

The \emph{segment} $\seg(\bp_i)$ of a node $\bp_i$ (see \cref{fig:segment}) contains all nodes that can be reached from $\bp_i$ by following the LHR or RHR while keeping the anchor fixed at $\anchor(\bp_i)$.
Simply put, $\seg(\bp_i)$ is the set of nodes that touch the anchor $a(\bp_i)$ from the same `side' as $\bp_i$.
Any node $\bp_j \in \seg(\bp_i)$ is a \emph{successor} of $\bp_i$, if $j > i$, or a \emph{predecessor} of $\bp_i$, if $j < i$.
As an example, node $\sigma_i$ in \cref{fig:segment} has two predecessors and one successor.
Denote by $\suc(\bp_i)$ the node without a successor in $\seg(\bp_i)$, and $\pred(\bp_i)$ the node without a predecessor in $\seg(\bp_i)$.

%Observe that $\suc(\bp_i)$ and $\pred(\bp_i)$ are by definition the nodes at which the agent's anchor changes moving $\lhr{}$.
%If the agent enters $\pred(\bp_i)$ by moving $\lhr{}$, then its anchor is still set to $a(\bp_{i-1})$, and if it moves $\lhr{}$ at $\suc(\bp_i)$, then its anchor must change prior to the move.


We can now formally introduce the above mentioned path $\lp$.
Define $\lp = (\lp_0,\dots,\lp_l)$ as a maximal simple sub-path of $\bp$ that starts at $\start$, i.e., $\lp_i = \bp_i$ for any $0 \leq i \leq l$.
The definition of $\seg(\cdot),\suc(\cdot)$ and $\pred(\cdot)$ directly carry over from $\bp$.
For simplicity, we write $v \in \lp$ or $v \in \bp$ if $\lp$ or $\bp$ contains node $v$.
%Note that $\lp$ is the path on which the agent moves back and forth after each tile placement.

Finally, we introduce the aforementioned \emph{range} that is explored by the agent before each tile placement.
Consider the agent to be positioned at some empty node $p$, such that the node $q = p + \rhr{}$ is a link.
The $i$-\emph{range} $R_i(p,q)$ (see \cref{fig:range}) is a specific neighborhood of empty nodes that is defined as if node $q$ were tiled.
Consider a node $v$ that can be reached from $p$ in at most $i$ steps without moving through $q$ or any tiled node.
If the segment $\seg(v)$ does not contain $q$, we add it to $R_i(p,q)$.
Otherwise, that segment is separated at $q$ and only the part that contains $v$ is added.
The formal definition is as follows: % (recall that $\bo(v) \coloneqq B(v) \cap \occ$):

\begin{definition}
    \label{def:range}
    The $i$-\emph{range} of $p$ w.r.t. $q$ is the set of nodes $$R_i(p,q) \coloneqq \bigcup_{v \in N_i(p)}\bigcup_{\;\;w \in \bo(v)} \seg(v)$$
    where $N_i(p)$ and $\seg(v)$ are w.r.t. $\tri(\emp \setminus \!\{q\})$ and anchor $w$.
\end{definition}


\subsection{Algorithm Details and Pseudocode}
\label{subsec:algorithm}

The coating algorithm (see pseudocode in \cref{alg:algorithm}) consists of an initialization \init{} (lines 1--3) and two phases \coat{} (lines 4--10), dedicated to the tile placement, and \fetch{} (lines 11--16), dedicated to the gathering of material, the traversal of $\lp$, and the termination.
The agent switches between phases $\coat{}$ and $\fetch{}$ after each tile placement.
In the pseudocode, %$p$ refers to the agent's position (which may change from line to line), 
$\links$ and $\gen{}$ denote the set of all links and generators (see \cref{def:1link}). %, and $\lhr{}$ and $\rhr{}$ denote the next direction of movement according to the LHR and RHR, respectively.

\begin{algorithm}[!b]
    %\scriptsize
    % % COMMENT TEMPLATE:
    %\nl \If(\Comment*[f]{comment}){if statement}{
    %    \nl commented line \Comment*[r]{comment}
    %    \nl regular line\;
    %}
    \caption{Coating Algorithm}
    \label{alg:algorithm}
    \SetAlgoVlined
    \DontPrintSemicolon
    Phase \init{}:\\
    \nl gather material from $p^0$; move to an arbitrary $\start \in B(p^0)$\;
    \nl store the direction of $p^0$ w.r.t. $\start$; $\anchor(\start)\leftarrow p^0$; $\noCheck \leftarrow false$\;
    \nl move $\lhr{}$; enter phase \coat{} \Comment*[r]{$p \leftarrow \start{} + \lhr{}$}
    \BlankLine
    Phase \coat{}:\\
    \nl \uIf{$\noCheck{}$ \normalfont{or} $R_3(p, p + \rhr{}) \cap \left(\links \cup \{\start{}\}\right) = \emptyset$}{
        \nl place a tile at $p$; $\noCheck{} \leftarrow false$; enter phase \fetch{}\;
    }
    \nl \Else{
        \nl move \rhr{} \Comment*[r]{$p \leftarrow p + \rhr{}$}
        \nl \lIf(\Comment*[f]{$p \leftarrow p + \rhr{}$}){$p \in \gen$}{move \rhr{}}
        \nl \lIf{$p \in \gen$ \normalfont{and} $p + \rhr{} \notin \links \cup \{\start{}\}$}{$\noCheck{} \leftarrow true$}
        \nl place a tile at $p$; enter phase \fetch{}\;
    }
    \BlankLine
    Phase \fetch{}:\\
    \nl \lWhile(\Comment*[f]{$p \leftarrow p + \rhr{}$}){$p \neq \start$}{
        move \rhr
    }
    \nl move to $p^0$; gather material from $p^0$; move to $\start$\;
    \nl \lIf{$\be(\start{}) = \emptyset$}{
        place a tile at $\start$; terminate
    }
    \nl \While{$p \in \links \cup \{\start{}\}$ \normalfont{or} $v \in \links$ \normalfont{for a successor $v$ of $p$ in} $\seg(p)$}{
        \nl move \lhr \Comment*[r]{$p \leftarrow p + \lhr{}$}
    }
    \nl enter phase \coat{}\;
\end{algorithm}

In phase \init{} (lines 1--3), the agent gathers material and moves to an arbitrary node $\start\in B(p^0)$.
It stores the direction of $p^0$ w.r.t. $\start$ such that it can recognize $\start$ by the adjacent material depot at a later visit, and it initializes $\anchor{}(\start{}) \leftarrow p^0$ and $\noCheck \leftarrow false$.
Note that $\start$ is the first node of the paths $\bp$ and $\lp$, $\anchor(p)$ is the agent's anchor, and $\noCheck$ is a flag that indicates that the search for links in the $3$-range is skipped in the next execution of phase \coat{} (see line 4).
The flag is necessary to maintain a crucial invariant which we elaborate in the proof of \cref{lem:invariant}.
Afterwards, the agent moves \lhr{} and enters phase \coat{}.

Phase \coat{} is always entered such that $p \notin \links \cup \{\start{}\}$ and $p + \rhr{}$ is the last node $v\in\lp$ (i.e., with maximum index) for which $v \in \links \cup \{\start{}\}$.
In each execution of phase \coat{} the tile that is carried by the agent is either placed directly at $p$ (lines 4--5), or at some link that can be reached by following the RHR for at most two steps (lines 7--10).
In any case, the agent enters phase \fetch{} afterwards.
The position of the next tile depends on the following criteria:
If the flag $\noCheck{}$ is set to $true$, then the tile is placed at $p$ and $\noCheck{}$ is set to $false$ afterwards.
The tile is also placed at $p$, if $R_3(p,p+\rhr{})$ (see \cref{def:range}) does not contain any node of $\links \cup \{\start{}\}$.
Otherwise, the agent must have found some link or the starting node, which implies that $R_3(p,p+\rhr{})$ contains a node $v \in \lp$ with smaller index than $p \in \lp$, i.e., the agent has detected a cycle in $\tri{}(\emp)$ in which it can safely place a tile at some link.
It is crucial that no link is generated on the `wrong side' of the newly placed tile as this link may later lead to a false detection.
Let the agent's position w.r.t. $\bp$ be $p = \bp_i$, i.e., $\bp_{i-1},\bp_{i-2}$ and $\bp_{i-3}$ are the next three nodes visited by following the RHR (see \cref{fig:alg}).
%$\bp_{i-1}$ or $\bp_{i-2}$ such that no link is generated in the connected component of $\be(\bp_{i-1})$ that contains $\bp_{i}$ or the connected component of $\be(\bp_{i-2})$ that contains $\bp_{i-1}$.
If $\bp_{i-1} \notin \gen$, then node $\bp_{i-1}$ is tiled, otherwise node $\bp_{i-2}$ is tiled.
In the latter case, $\noCheck$ is set to $true$ whenever $\bp_{i-2} \in \gen$ and $\bp_{i-3}\notin \links$ (line 9).
The flag $\noCheck{}$ ensures that node $\bp_{i-3}$ is tiled next such that the agent again enters phase \coat{} $\lhr{}$ of the last node $v \in \lp$ with $v \in \links \cup \{\start{}\}$.
%In the analysis, we show that this is the only case in which $P2$ is violated for an entire execution of phase \fetch{}, however, after placing the next tile obliviously with $\noCheck$ set to true, the property holds again.

\begin{figure}[!t]
    \centering
    \begin{subfigure}[c]{0.32\linewidth}
        \includegraphics[width=\linewidth]{alg1s}
        \subcaption{}
        \label{fig:alg1}
    \end{subfigure}
    \begin{subfigure}[c]{0.32\linewidth}
        \includegraphics[width=\linewidth]{alg2s}
        \subcaption{}
        \label{fig:alg2}
    \end{subfigure}
    \begin{subfigure}[c]{0.31\linewidth}
        \includegraphics[width=\linewidth]{alg3s}
        \subcaption{}
        \label{fig:alg3}
    \end{subfigure}%
    \caption{Examples in which the next tile is placed at some link (red disk) in phase \coat{}: $\sigma_{i-1}$ is tiled in (a); $\sigma_{i-2}$ is tiled in (b) and (c). Only in (c), $\noCheck$ is set to $true$ since $\sigma_{i-3} \notin \links \cup \{\start\}$. As a result, $\sigma_{i-3}$ is tiled on the next visit.}
    \label{fig:alg}
\end{figure}


In phase \fetch{}, the agent moves $\rhr{}$ until it is positioned at $\start$ (line 11), which it detects by the adjacent material depot and the direction stored in phase \init{}.
It moves to $p^0$, gathers material and returns to $\start$ (line 12).
If $\start$ has no empty neighbors, it places a tile at $\start$ and terminates (line 13).
Otherwise, the agent moves $\lhr{}$ as long as $p \in \links \cup \start$ or whenever a successor of $p$ in $\seg(p)$ is a link, and switches to phase \coat{} afterwards (lines 14--16).
In that step, the agent implicitly explores $\seg(p)$, since $p$ can have~multiple~successors.

\subsection{Analysis}
\label{subsec:analysis}

Consider an initial configuration $C^0 = (\occ^0,\obj{},p^0)$ with $p^0 \in \occ^0 \subseteq L$, and a material depot of size at least $|L| - |\occ^0| + 1$ at $p^0$.
In the problem statement we assume that $p^0$ is the only initially tiled node, but now we allow $\occ^0$ to contain multiple tiled nodes besides $p^0$.
This will later become useful in \cref{sec:construction} where we construct a virtual graph in which some nodes are tiled initially.
We analyze \cref{alg:algorithm} given that $C^0$ satisfies the following definition:

\begin{definition}
    \label{def:coatability}
    A configuration $C^0 = (\occ^0,\obj{},p^0)$ is coatable w.r.t. $\tri$, if $|B(v)| \leq 6$ for any $v \in \emp^0$, $\beo(p^0) \neq \emptyset$, $\links^0 = \emptyset$, and $\emp^0$ is connected.
\end{definition}

We aim to maintain five properties as invariants:
\emph{\textbf{P1}}: Links may only occur on the simple path $\lp$, i.e., $\links \subseteq \lp$.
\emph{\textbf{P2}}: All links are connected by a sequence of overlapping segments to the starting node, i.e., $\pred(v) \in \links \cup \{\start{}\}$ for any $v \in \links$.
\emph{\textbf{P3}}: The subpath of $\lp$ from $\start{}$ to the last link on $\lp$ induces no cycle in $\tri$, i.e., for any $i < j \leq k$ with $\lp_k \in \links$: if $d(\lp_i,\lp_j) = 1$, then $j = i+1$.
\emph{\textbf{P4}}: The boundary of any link contains precisely two connected components of empty nodes, i.e., $||\be(v)|| = 2$ for any $v \in \links$ where $||\cdot||$ denotes the number of connected components.
\emph{\textbf{P5}}: There exists a node of $\lp$ at which the agent enters phase \coat{}, i.e., either $\links = \emptyset$ or there is an $i$ such that $\lp_i \notin \links \cup \{\start{}\}$ and $\suc(\lp_i) = \lp_i$.

%\begin{itemize}
%    \item[] \emph{\textbf{P1}}: Links may only occur on the simple path $\lp$, i.e., $\links \subseteq \lp$.
%    \item[] \emph{\textbf{P2}}: All links are connected by a sequence of tails to the starting node, i.e., $\pred(v) \in \links \cup \{\start{}\}$ for any $v \in \links$.
%    \item[] \emph{\textbf{P3}}: The subpath of $\lp$ from $\start{}$ to the last link on $\lp$ induces no cycle in $\tri$, i.e., for any $i < j \leq k$ with $\lp_k \in \links$: if $d(\lp_i,\lp_j) = 1$, then $j = i+1$.
%    \item[] \emph{\textbf{P4}}: The boundary of any link contains precisely two connected components of empty nodes, i.e., $||\be(v)|| = 2$ for any $v \in \links$ where $||\cdot||$ denotes the number of connected components.
%    \item[] \emph{\textbf{P5}}: There exists a node of $\lp$ at which the agent enters phase \coat{}, i.e., either $\links = \emptyset$ or there is an $i$ such that $\lp_i \notin \links \cup \{\start{}\}$ and $\suc(\lp_i) = \lp_i$.
%\end{itemize}

Observe that all properties hold initially by $\links^0 = \emptyset$.
The structure of our proof is as follows:
We prove termination given that $P5$ is maintained, and that $\emp$ never disconnects given that $P1$--$P4$ are maintained.
Since the agent always finds a node to place the next tile at by $P5$, there must eventually be a step in which $\be(\start{}) = \emptyset$.
Since $\emp{}$ remains connected until that step, $L = \occ{}$ holds after the last tile is placed at $\start{}$.
Finally, we show that $P1$--$P5$ are maintained as invariants.
We start with the termination of the algorithm:

\begin{lemma}
    \label{lem:termination}
    If $P5$ holds in step $t$ in which the agent gathers material at $p^0$, then there is a step $t^+ > t$ in which the agent enters $p^0$ again or terminates.
\end{lemma}

\begin{proof}
    Assume by contradiction that the agent does not terminate or enter $p^0$ again in any step $t' > t$.
    There are two cases: the agent places a tile in phase \coat{} and moves to a connected component of $\emp{}$ that does not contain $\start$, or the agent never enters phase \coat{}, i.e., it moves indefinitely $\lhr{}$ in phase \fetch{}.
    If the agent traverses a simple path from $\start{}$ to some node $v$ by moving $\lhr{}$, places a tile at $v$ and moves $\rhr{}$ afterwards, it must enter the connected component of $\emp{}$ that contains $\start{}$.
    Hence, in the first case, the agent places a tile at $v$ after visiting $v$ at least twice, i.e., it has fully traversed $\lp$, which contradicts the existence of node $\lp_i$ specified by $P5$.
    In the second case, the agent never reaches $\lp_i$, as it would otherwise enter phase \coat{} and place a tile.
    This implies a cycle $(\lp_j,...,\lp_k)$ with $\lp_{k+1} = \lp_{j}$ and $j < i$, which contradicts that $\lp$ is a simple path.
    Hence, there is a step $t^+$ in which $p^0$ is entered again or $\be(\start{}) = \emptyset$ and the agent terminates.
    \end{proof}

Subsequently, our notation refers to some step $t$ in which the agent gathers material at $p^0$.
With a slight abuse of notation we will use a superscript $^+$ ($^-$) to denote the next (previous) step $t^+$ ($t^-$) in which the agent gathers material at $p^0$ or terminates, e.g., $\emp{}^+$ denotes the set of empty nodes in step $t^+$.

\begin{lemma}
    \label{lem:enterCoat}
    If $P2$ holds in step $t$ and phase \coat{} is entered at some $\lp_i$ between step $t$ and $t^+$, then $\lp_{i-1}$ is the last node $v$ on $\lp$ with $v \in \links \cup \{\start{}\}$.
\end{lemma}

\begin{proof}
    As long as $p \in \links \cup \{\start{}\}$, the agent moves \lhr{} in phase \fetch{} which implies $\lp_{i-1} \in \links \cup \{\start{}\}$.
    It also moves \lhr{} if a successor of $p$ in $\seg(p)$ is a link.
    Hence, no successor $v$ of $\lp_{i-1}$ in $\seg(\lp_{i-1})$ is a link.
    Assume by contradiction that $\lp_{i-1}$ is not the last node on $\lp$ that is contained in $\links \cup \{\start{}\}$.
    Let $\lp_k$ be the first link after $\lp_{i-1}$, i.e., $\lp_k \in \links$ and $k > i-1$.
    Since no successor of $\lp_{i-1}$ in $\seg(\lp_{i-1})$ is a link, $\lp_k$ cannot be contained in $\seg(\lp_{i-1})$.
    Let $S = (v_0, ..., v_m)$ be the sequence of nodes with $v_0 = \lp_k$, $v_m = \start$ and $v_j = \pred(v_{j-1})$ for any $0 < j \leq m$.
    The agent's anchor changes only if there is no further successor in its current segment, i.e., \emph{after} moving $\lhr{}$ at some node $v$ of $\lp$ with $\suc(v) = v$.
    This implies $\suc(\pred(v)) = \pred(v)$ for all $v \in \lp$.
    It follows that $S$ must contain some node $v_j$ with $0 < j < m$ for which $v_j = \suc(\lp_i)$.
    Since $\suc(\lp_i) \notin \links$, there must exists some $v_{j'}$ with $j' < j$ for which $\pred(v_{j'}) \notin \links \cup \{\start{}\}$ which contradicts $P2$ and concludes the lemma.
    \end{proof}

%Next, we show that $\emp{}$ remains connected throughout execution.

\begin{lemma}
    \label{lem:cycles}
    If $\emp{}$ is connected and $P1$--$P4$ hold in step $t$, then $\emp^+$ is connected.
\end{lemma}

\begin{proof}
    Tiling a node $v \notin \links$ cannot disconnect $\emp{}$ by \cref{def:1link}.
    This covers the tile placement at the start of phase \coat{}, especially the case $\noCheck = true$, and the last placement before termination in phase \fetch{}.
    Thus, we must only consider cases in which a tile is placed at some link $v \in \links$.
    By \cref{lem:enterCoat}, the agent enters phase \coat{} at $\lp_i$ such that $\lp_{i-1}$ is the last node $w \in \lp$ with $w \in \links \cup \{\start{}\}$.
    Together with $P1$ follows that whenever it detects some node $w \in R_3(\lp_i,\lp_{i-1})$ with $w \in \links \cup \{\start{}\}$, then $w$ was visited in the previous execution of phase \fetch{}.
    Let $\lp_j$ be the last node of $\lp$ that is contained in $R_3(\lp_i,\lp_{i-1})$ with $j < i-1$, and $P$ be the shortest path from $\lp_i$ to $\lp_j$ in $\tri(R_3(\lp_i,\lp_{i-1}))$.
    Then $C = P \circ (\lp_{j+1}, \lp_{j+2},...,\lp_{i-2}, \lp_{i-1})$ is a simple cycle in $\tri$, where $\circ$ is the concatenation of paths.
    Placing a tile at $\lp_{i-1}$ or $\lp_{i-2}$ cannot disconnect $C$ since it is a cycle, and it cannot disconnect $\emp{}$ since $C$ contains nodes of all connected components of $\be(\lp_{i-1})$ (and $\be(\lp_{i-2})$, if $\lp_{i-2} \in \links$) by $P4$.
    \end{proof}


\begin{lemma}
    \label{lem:generatorSharedTiled} %For all $v,w \in L$: 
    If $\bo(v) \cap \bo(w) \neq \emptyset$, then tiling $v$ cannot increase $||\be(w)||$.
\end{lemma}

\begin{proof}
    The lemma follows trivially if $v \notin B(w)$.
    Consider arbitrary $v,w \in L$ with $v \in B(w)$.
    Since $\tri$ is a triangulation, the edge $\{v,w\}$ is contained in precisely two triangular faces, each of which contains another node $u_1$ and $u_2$, respectively.
    Since $B(v)$ and $B(w)$ are chordless, these are the only nodes adjacent to both $v$ and $w$, which implies that $B(v) \cap B(w) = \{u_1,u_2\}$ and that $\{u_1,v,u_2\}$ is connected in $B(w)$.
    If any $u_i$ is tiled, then $||\bo(w) \cup \{v\}||\leq ||\bo(w)||$.
    Thereby, $||\be(w) \setminus \{v\}|| = ||\bo(w) \cup \{v\}|| \leq ||\bo(w)|| = ||\be(w)||$.
    \end{proof}

We can now deduce the precise neighborhood of $v$ for the case where $v$ is both a link and a generator, i.e., $v \in \links \cap \gen$.
By \cref{def:coatability}, $B(v)$ contains at most six nodes, at least two of which must be tiled, as otherwise $v$ cannot be a link.
Hence, at most four empty nodes in at least two connected components of $\be(v)$ remain.
If each connected component has size at most two, then all nodes in $\be(v)$ share a tiled neighbor with $v$, which contradicts that $v$ is a generator by the previous lemma.
Hence, as a corollary we obtain the following:

\begin{corollary}
    \label{cor:generatorSize}
    For any $v \in \links \cap \gen$: $\bo(v)$ contains two connected components of size one, and $\be(v)$ contains two connected components of size one and three.
\end{corollary}


\begin{lemma}
    \label{lem:invariant}
    If $P1$--$P5$ hold and a node is tiled in step $t$, then either $P1$--$P5$ hold in step $t^+$ or the agent sets $\noCheck = true$ and $P1$--$P5$ hold in step $t^{++}$.
\end{lemma}

The proof of \cref{lem:invariant} is deferred to \cref{sec:appendix}.
Essentially, we distinguish the type of node $v$ that is tiled in step $t$, i.e., whether $v$ is not a link, a link but no generator, or a link and a generator, and finally whether $v + \rhr{} \in \links \cup \{\start\}$ (see line 9).
In all but the last case, we can show that $P1$--$P5$ immediately hold in step $t^+$, and in the last case, we know by the algorithm that $\noCheck{}$ is set to true after tiling $v$ in step $t^+$.
Here we can show that only property $P2$ is violated in step $t^+$, and only within the neighborhood of the previously placed tile.
Using \cref{cor:generatorSize}, we can precisely determine at which node the next tile is placed (with $\noCheck{} = true$), and that this tile placement restores $P2$ in step $t^{++}$.
In the other cases the properties mostly follow from \cref{lem:generatorSharedTiled}.

All properties hold in an initial configuration, and they are maintained as invariants by \cref{lem:invariant}.
By \cref{lem:enterCoat}, the agent eventually terminates in some step $t^*$, and by \cref{lem:cycles} $\emp{}$~never disconnects.
Hence, $\occ{}^{t^*} = L$ holds after termination which concludes the following:

\begin{theorem}
    \label{thm:algorithm}
    Following \cref{alg:algorithm}, a finite-state agent solves the coating problem on $\tri$, given a configuration $C^0 = (\occ^0,\obj{},p^0)$ that is coatable w.r.t. $\tri$.
\end{theorem}

\subsection{Runtime Analysis}
\label{subsec:optimality}

%We now analyze the runtime of \cref{alg:algorithm} and show worst-case optimality.
%To detect whether $p \in \links$, the agent does not need to move, since it senses the occupation of adjacent nodes in its look-phase.
Since the agent does not sense any node outside of $N_1(p)$ in its look-phase, exploring $R_i(p,q)$ requires additional steps.
From $R_i(p,q) \subset N_{i+2}(p)$ it follows that the number of steps is upper bounded by $2 \cdot |N_{i+2}(p)| = \O(|N(p)|^i)$. %which follows from the assumption that $\tri$ has constantly many edge orientations
Since $i$ is a constant and $\tri$ has constant degree, each execution of phase \coat{} takes $\O(1)$ steps.
Each execution of phase \fetch{} takes $\O(|\lp|)$ steps as the agent traverses a sub-path of $\lp$ twice.
Since $\lp$ is simple, it follows that $|\lp| = \O(n)$, where $n = |L|$.
The agent can place at most $n$ tiles until $L = \occ{}$, thereby performs at most $n$ executions of \coat{} and \fetch{}, which results in $\O(n^2)$ steps in total.

An agent $\tilde{r}$ with unlimited memory and global vision can reach any node via a shortest path instead of sticking to the boundary of tiled nodes.
Except for the last placed tile it must always return to the material depot which implies that the last tile is placed at a node $w$ with maximum distance to $p^0$.
In the worst case, the surface graph is the triangulation of an object resembling a straight line such that $d_L(p^0,w) = \Theta(n)$. %, with the material depot at one of its endpoints 
Each node on the shortest path $P$ from $p^0$ to $w$ must be tiled.
Hence, $\tilde{r}$ takes at least
$2\left( \sum_{u \in P} d_L(p^0,u)\right) - d_L(p^0,w) = \left(\sum_{i=1}^{\Theta(n)} 2 \cdot i\right) - \Theta(n) = \Theta(n^2)$ steps which implies worst-case optimality of \cref{alg:algorithm}.

