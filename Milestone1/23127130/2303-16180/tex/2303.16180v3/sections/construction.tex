\section{Coating in the 3D Hybrid Model}
\label{sec:construction}

In this section, we apply our coating algorithm to the 3D hybrid model.
We first define a triangulation on nodes of $L$ with degree $\Delta \leq 8$, and afterwards construct a virtual graph on which we emulate our algorithm using $2^{2\Delta}$ different types of tiles.
%We first give a surface graph $\tri$ that is a triangulation with degree $4 \leq \Delta \leq 8$ in which the boundary of each node is chordless.
%We show that there is a restricted class of objects for which $C^0$ is coatable w.r.t. $\tri$, i.e., empty nodes have degree at most six.
%To solve the problem on surface graphs of degree $\Delta > 6$, we construct a virtual graph $\tri^*$ of size at most $2 \Delta n$ on which we emulate the algorithm using $2^{2\Delta}$ different types of tiles.
%Each tile type corresponds to a bit-sequence of length $2\Delta$ that we use to encode whether nodes of $\tri^*$ are tiled or empty.

\begin{figure}[t]
    \centering
    \begin{minipage}{.43\textwidth}
        \centering
        \includegraphics[width=\linewidth]{snapshot}
        \caption{Snapshot of $\surf$: the circled nodes are adjacent in $G(L)$ but not in $\surf$.}
        \label{fig:coatingLayerGraph}
    \end{minipage}%
    \hfill
    \begin{minipage}{.55\textwidth}
        \centering
        \includegraphics[width=\linewidth]{emulation}
        \caption{A triangular face $f = \{u_1,u_2,u_3\}$ of $\tri$ and its corresponding virtual edges and nodes in $\tri^*$.}
        \label{fig:emulation}
    \end{minipage}
\end{figure}


\begin{figure}[b]
    \centering
    \hfill
    \foreach \x in {a,...,f}{%
        \begin{subfigure}[c]{0.12\linewidth}
            \includegraphics[width=\linewidth]{boundaries_\x}
            \subcaption{}
            \label{subfig:triangulation_\x}
        \end{subfigure}%
        \hfill
    }%
    \null\hfill
    \\
    \foreach \x in {g,...,m}{%
        \begin{subfigure}[c]{0.12\linewidth}
            \includegraphics[width=\linewidth]{boundaries_\x}
            \subcaption{}
            \label{subfig:triangulation_\x}
        \end{subfigure}%
        \hfill
    }%
    \caption{All possible arrangements of faces in $\surf$ apart from rotation. Dashed edges indicate that the distance between its endpoints is precisely two w.r.t. $G$.}
    \label{fig:triangulation}
\end{figure}

%\subsection{Triangulation of the Surface Graph}
%\label{subsec:smoothObjects}

Recall the definition of graph $G = (V,E)$ and its embedding in $\mathbb{R}^3$ from \cref{sec:model}.
%Any two adjacent nodes $v,w \in V$ have euclidean distance $|\vec{v}-\vec{w}|_2 = \sqrt{2}$, and tiles at $v$ and $w$ share a common face.
%If two nodes $v,w \in V$ are adjacent w.r.t. $G$, then tiles at $v$ and $w$ share a common face.
%Note that in this case $|\vec{v}-\vec{w}|_2 = \sqrt{2}$, where $|\cdot|_2$ is the Euclidean distance.
%We call $v$ and $w$ \emph{vertex adjacent}, if $|\vec{v}-\vec{w}|_2 = 2$, e.g., $v = w + \UNE + \SE$.
%In contrast to adjacent nodes, tiles at vertex adjacent nodes only share a common vertex.
Define $\surf = (L,E')$ as the subgraph of $G(L)$ that contains only those edges $\{v,w\}$ for which $v$ and $w$ share adjacent object neighbors (see \cref{fig:coatingLayerGraph}), i.e., $E' = \{\{v,w\}\mid d_\obj{}(N_1(v), N_1(w)) \leq 1\}$.
%The difference between $G(L)$ and $\surf$ is illustrated in \cref{fig:coatingLayerGraph}.
%The two circled empty nodes are adjacent in $G(L)$, but not in $\surf$;
%an agent that moves in $\surf$ must thereby remain locally connected to the object.
We can view $\surf$ as embedded on the surface of our 3D object.
%It contains no cross edges, and especially, $\surf$ is planar if the object contains no holes.
That embedding contains triangular and tetragonal faces (see \cref{fig:coatingLayerGraph}) where tetragonal faces can occur in one of three orientations: 
(1) $v, v+ \NE, v + \NE + \USE, v + \USE$, (2) $v, v + \NW, v + \NW + \UNE, v + \UNE$, and (3)~$v, v + \N, v + \N + \UW, v+ \UW$.
%Note that any two nodes $v,w$ on the diagonal of a tetragonal face have euclidean distance $|\vec{v}-\vec{w}|_2 = 2$ while adjacent nodes have distance $|\vec{v}-\vec{w}|_2 = \sqrt{2}$.
%While nodes of triangular faces are pairwise adjacent, nodes on the diagonals of tetragonal faces are only vertex adjacent. 
%Using (1) as an example, $v$ is vertex adjacent to $v + \NE + \USE$, and $v + \NE$ is vertex adjacent to $v + \USE$.
Apart from rotation, \cref{fig:triangulation} shows all possible arrangements of faces within $\surf$. % w.r.t.\ a fixed common node $v \in L$ (centering node in the figure).
We define the class of \emph{smooth objects} $\mathcal{S}$ as all objects for which $\surf$ contains only the cases (a)--(f) from \cref{fig:triangulation}.
Let $\tri$ be the triangulation of $\surf$ in which the same diagonal edge is added for each tetragonal face of the same orientation (1)--(3) (since we want the agent to be able to deduce the triangulation).
%Each face of $\tri$ is triangular with pairwise adjacent nodes such that $B(v)$ is chordless w.r.t. $\tri$ for any $v \in L$.
Since $d_L(v,w) = 1$ w.r.t. $\tri$ implies $d_L(v,w) \leq 2$ w.r.t. $\surf$, the agent can emulate moving on $\tri$ with a multiplicative time and memory overhead of at most two.
%Since tetragonal faces of different orientations (1)--(3) cannot contain a common edge, (d), (k) and (m) from \cref{fig:triangulation} are the only cases in which a node is contained in tetragonal faces of different orientations.
%Hence, $\tri$ has degree at most six for any $\obj \in \mathcal{S}$ such that our next theorem follows directly from \cref{thm:algorithm}:
It is easy to see that $B(v)$ is chordless and $v$ has degree at most six for all $v \in L$ within the class $\mathcal{S}$.
Together with \cref{thm:algorithm} follows:

\begin{theorem}
    \label{thm:coatableSingleType}
    A finite-state agent with a single type of passive tiles solves the coating problem on any object $\obj \in \mathcal{S}$ with coating layer $L$ in $\O(n^2)$ steps, where $n = |L|$.
\end{theorem}


\subsection{Emulation of Coatable Surface Graphs}
\label{subsec:emulation}

Consider an arbitrary triangulation $\tri = (L,E)$ of constant degree $\Delta$ and an initially valid configuration $C_0$.
We construct a virtual graph $\tri^* = (L^*,E^*)$ with virtual initial configuration $C^{0*}$ such that $\tri^*$ is coatable w.r.t. $C^{0*}$.
During that construction, we define a partial surjective function $\mathcal{R}: L^* \rightarrow L$ that maps virtual nodes to real nodes.
We show that an agent $r$ operating on $\tri$ w.r.t. $C_0$ with $2^{2\Delta}$ tile types can emulate an agent $r^*$ that executes \cref{alg:algorithm} on $\tri^*$ w.r.t. $C^{0*}$ such that throughout the emulation $\mathcal{R}(p^*) = p$.


\subsubsection{Virtual Graph Construction}

The virtual graph $\tri^*$ is the result of subdividing each face of $\tri$ into nine triangular faces (see \cref{fig:emulation}).
The node set $L^*$ contains a virtual node $v^*_u$ for each node $u \in L$, two virtual nodes $v^*_{u,w}$ and $v^*_{w,u}$ for each edge $\{u,w\} \in E$, and a virtual node $v^*_f$ for each triangular face $f$ of $\tri$.
For each edge $\{u,w\} \in E$ the edge set $E^*$ contains three virtual edges $\{v^*_u,v^*_{u,w}\}$, $\{v^*_{u,w}, v^*_{w,u}\}$ and $\{v^*_{w,u}, v^*_w\}$.
For each triangular face $f = \{u_1,u_2,u_3\}$ of $\tri$ the edge set $E^*$ contains six virtual edges $\{v^*_f,v^*_{u_i,u_j}\}$ and three virtual edges $\{v^*_{u_i,u_j}, v^*_{u_i,u_k}\}$, where $u_i, u_j, u_k \in f$ are pairwise distinct.
We define $\mathcal{R}(v^*_{u,w}) = u$ for any virtual node $v^*_{u,w} \in L^*$. % that corresponds to an edge $\{u,w\}$ of $\tri$.
Consider an arbitrary but fixed order on the vectors $\vec{\X}_1,...,\vec{\X}_m$ that correspond to edges in the embedding of $\tri$.
Let $\pi$ represent that order, i.e., $\pi(\vec{\X}_i) = i$.
For some face $f = \{u_1,u_2,u_3\}$ of $\tri$, we define $\mathcal{R}(v^*_f) = u_i$, where $u_i$ is the node that minimizes $\pi(\vec{u_i} - \vec{u_j})$ for any $u_i,u_j \in f$ with $i \neq j$.
We define the virtual initial configuration $C^{0*}$ such that all $v^*_u$ are tiled, i.e., $\occ^{0*} = \cup_{u \in L} v^*_u$, $p^{0*} = v^*_{p^0}$ and assume a material depot of size at least $|L^*|-|L|$ at $v^*_{p^0}$. 

\begin{lemma}
    \label{lem:virtualGraph}
    $C^{0*}$ is coatable w.r.t. $\tri^*$.
\end{lemma}

\begin{proof}
    Each face of $\tri$ is triangular, and two virtual nodes are added for each edge of $\tri$.
    Hence, $|B(v^*_f)|= 6$ w.r.t. $\tri^*$ for any face $f$ of $\tri$.
    Any $v^*_{u,w}$ is adjacent to $v^*_{f_1}$ and $v^*_{f_2}$, where $f_1,f_2$ are the two faces of $\tri$ that both contain $u$ and $w$, to two nodes $v^*_{u,w_1}, v^*_{u,w_2}$, where $w_1 \in f_1$ and $w_2 \in f_2$, and to $v^*_{w,u}$ and $v^*_u$.
    Hence, $|B(v^*_{u,w})| = 6$ w.r.t. $\tri^*$ for any edge $\{u,w\}$ of $\tri$.
    Any other virtual node is initially tiled, which implies $|B(v^*)| \leq 6$ for any $v^* \in \emp^*$.
    %$B(v^*)$ is chordless since all nodes in $B(v^*)$ correspond to the same or some adjacent face of $\tri$ for any $v^* \in L^*$. 
    By construction, each initially tiled node is isolated, i.e., $d(v^*,w^*) \geq 3$ for any $v^*,w^* \in \occ{}^{0*}$.
    Since $\tri^*$ is connected, it follows that $\emp^{0*}$ is connected and $\be(v^*)$ is connected for any $v^* \in \emp^{0*}$, i.e., $\links^{0*} = \emptyset$.
    Hence, each property of \cref{def:coatability} is satisfied.
\end{proof}

\begin{lemma}
    \label{lem:emulation}
    A finite-state agent can emulate \cref{alg:algorithm} on $\tri^*$ in $\O(\Delta^2n^2)$ steps while moving and placing tiles of at most $2^{2\Delta}$ types on $\tri$.
\end{lemma}

\begin{proof}
    Let $F^* \subset L^*$ be the set of virtual nodes $v^*_f$ that correspond to some face $f$ of $\tri$ in the construction of $\tri^*$.
    Since $\tri^*(L^* \setminus F^*)$ is a subdivision of $\tri$, it can be embedded in the same 3D surface as $\tri$ using vectors that are collinear to vectors in the embedding of $\tri$.
    It follows that we can use the same fixed order $\pi$ from the construction of $\tri^*$.

    In the following, we define for each node $u \in L$ a bit-sequence $x(u) = (x_1,...,x_{2\Delta})$ that encodes the occupation of all nodes $v^* \in L^*$ with $\mathcal{R}(v^*) = u$, where a $0$ encodes an empty, and a $1$ encodes an occupied virtual node.
    By the construction of $\tri^*$, there are at most $2 \Delta$ nodes $v^*$ with $\mathcal{R}(v^*) = u$ such that $2 \Delta$ bits suffice.
    The order of bits in $x(u)$ is uniquely given by $\pi$ where the first $\Delta$ bits encode virtual nodes that correspond to edges of $\tri$, and the following bits encode virtual nodes that correspond to faces of $\tri$.
    There is no bit for the virtual node $v^*_u \in L^*$ since it is initially occupied and remains occupied until termination by following \cref{alg:algorithm}.
    In fact, $\mathcal{R}$ is undefined for $v^*_u \in L^*$.

    Consider an agent $r$ on $\tri$ that utilizes $k = 2^{2\Delta}$ types of passive tiles.
    Each tile type uniquely describes a bit-sequence of length $\log(k) = 2\Delta$ such that $r$ emulates an agent $r^*$ on $\tri^*$ with initial configuration $C^{0*}$ as follows:
    If $r^*$ moves from $v^*$ to $w^*$, then $r$ moves from $\mathcal{R}(v^*)$ to $\mathcal{R}(w^*)$ (if $\mathcal{R}(v^*) \neq \mathcal{R}(w^*)$).
    If $r^*$ places a tile at $v^*$ and $\mathcal{R}(v^*)$ is empty, then $r$ places a tile at $\mathcal{R}(v^*)$ that corresponds to the bit-sequence $x$ in which only $v^*$ is encoded as occupied, otherwise $r$ incorporates the occupation of $v^*$ by changing the tile type.
    If $r^*$ gathers material and $r$ carries no material, then $r$ also gathers material.

    By \cref{thm:algorithm} and \cref{lem:virtualGraph}, $r^*$ solves the coating problem on $\tri^*$.
    Since $\mathcal{R}$ is surjective and any node $\mathcal{R}(v^*) \in L$ is occupied, if $v^*\in L^*$ is occupied, the emulation solves the coating problem on $\tri$ in $\O(|L^*|^2) = \O(\Delta n)$ steps.
\end{proof}

Our final theorem follows from the virtual graph construction on top of our triangulation $\tri$ of $\surf$ (with $\Delta \leq 8$) and the previous lemma:

\begin{theorem}
    \label{thm:coatableManyType}
    A finite-state agent utilizing constantly many tile types can solve the coating problem on arbitrary objects in worst-case optimal $\O(n^2)$ steps.
\end{theorem}