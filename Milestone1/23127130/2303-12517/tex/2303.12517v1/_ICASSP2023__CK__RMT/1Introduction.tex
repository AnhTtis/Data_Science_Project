Virtual sensing (VS) algorithms utilise physical remote monitoring sensors to estimate the acoustic pressure at a virtual position, and thus it is often employed in active noise control (ANC) applications with design constraints in error microphone placements where control is desired, such as at the human ears. \citep{Moreau2008, Edamoto2016, Deng2018, Sun2022}. One of the prominent VS methods is the remote microphone technique (RMT) \citep{Elliott2015}, which directly estimates the acoustic pressure at these virtual positions through pre-calibrated observation filters (OF), as shown in \fref{fig:block_diagram_RMT_noDelay}. This technique as described in \sref{sec:formulation_rmt}, however, assumes a stationary acoustic field to achieve a robust estimation at the virtual locations with the pre-calibrated fixed-coefficient OFs \citep{Elliott2020, Zhang2021}. This limits its application to where the acoustic field remains relatively stationary throughout the active control period, such as in road noise ANC in automobile cabins, where head-tracking techniques were applied with the RMT to continuously update the location of the virtual error microphones due to head movement \citep{Jung2018, Elliott2018a}. In cases where noise sources are time-varying and could arise from unknown directions, such as in the active control of noise through an open-aperture \citep{Shi2020, Lam2020} or mobile phones \citep{Cheer2018}, estimation performance will be degraded. While it was shown previously that the RMT estimation performance can be improved by reconstructing the correlation matrices (CMs) between microphones based on the superposition of CMs associated with its respective incoherent noise source \citep{Lai2022}, the reconstruction requires knowledge of the relative source strengths between these incoherent noise sources. Whereas source-localization techniques, such as the deconvolution approach for the mapping of acoustic sources (DAMAS), inverse acoustic method, or CM fitting (CMF) \citep{Yardibi2010a, Nelson2000}, could be utilised to estimate the source strengths, none of these methods was suitable for a VS application due to its modelling assumptions as detailed in \sref{sec:formulation_sourcetrack}. Through an open-aperture ANC implementation \citep{Lai2022}, the significance of the source ratio parameter is first described in \sref{sec:effect_mismatch}, followed by the proposed algorithm in \sref{sec:formulation_sourcetrack} to obtain the source ratio parameter through a source localization method, and finally its verification by simulation in \sref{sec:SourceTrack_result}.