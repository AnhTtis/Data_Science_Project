\fref{fig:block_diagram_RMT_noDelay} shows the block diagram arrangement of a virtual sensing system using the remote microphone technique \citep{Moreau2008, Jung2018}. The observation filter $\boldsymbol{O}$ can be expressed in either the frequency domain or in the causally-constrained time domain to minimise the expected mean squared error between the estimated disturbance signal at the virtual microphone, $\hat{\boldsymbol{d}}_{e}$, and the actual disturbance signal, $\boldsymbol{d}_{e}$. Thus, the optimal observation filter in the frequency and causally-constrained time domain are expressed as \citep{Elliott2015, Jung2019}
{ \small
\begin{align} 
  \boldsymbol{O}_{opt}(j\omega) &= \boldsymbol{S}_{d_m d_e} \left(\boldsymbol{S}_{d_m d_m} + \beta\boldsymbol{I}\right)^{-1} \label{eqn:Oopt_freq_beta} \\
  &= \boldsymbol{P}_{e} \boldsymbol{S}_{vv} \boldsymbol{P}^{H}_{m} \left(\boldsymbol{P}_{m} \boldsymbol{S}_{vv} \boldsymbol{P}^{H}_{m} + \beta\boldsymbol{I} \right)^{-1} \nonumber
\end{align}
}%
and
{ \small
\begin{align} 
  \mathbf{O}_{opt} = \mathbf{R}_{d_m d_e}^{T}\left[0\right]\left(\mathbf{R}_{d_m d_m}\left[0\right]+\beta \mathbf{I}\right)^{-1}
 \label{eqn:Oopt_regularised_fir}
\end{align}
}%
respectively, where $\boldsymbol{P}_{e}$ and $\boldsymbol{P}_{m}$ are matrices of responses from the array of modelled primary sources $\boldsymbol{v}$ to the vectors of error and monitoring microphones $\boldsymbol{e}$ and $\boldsymbol{m}$, $E\left[ \cdot \right]$ is the expectation operator and $\beta$ is a regularisation parameter, $\boldsymbol{S}_{d_m d_e}$, $\boldsymbol{S}_{d_m d_m}$ and $\boldsymbol{S}_{vv}$ are the spectral density matrices and $\mathbf{R}_{d_m d_e}$, $\mathbf{R}_{d_m d_m}$ are the correlation matrices, each defined with a general notation of $\boldsymbol{S}_{xy}(\omega) = E\left[ \mathbf{y}(\omega) \mathbf{x}(\omega)^{H} \right]$ and $\mathbf{R}_{xy}[n_0] = E\left[ \mathbf{x}[n]\mathbf{y}[n-n_0]^{T}\right]$ respectively. For brevity, the $``\left[0\right]"$ notation from \eref{eqn:Oopt_regularised_fir} will be omitted throughout the paper, and while regularization can improve the robustness of the RMT when subject to uncertainty in the acoustic field \citep{Jung2018}, it is omitted to limit the scope (i.e. $\beta=0$). To evaluate the accuracy of the RMT, the overall estimation error of the RMT in the frequency domain is used, that is \citep{Jung2019}
{ \small
\begin{align}
  L_\epsilon = 10 \log_{10} \left| \frac{ \text{tr}\left\lbrace \boldsymbol{S}_{\epsilon \epsilon}\right\rbrace }{ \text{tr}\left\lbrace \boldsymbol{S}_{d_e d_e}\right\rbrace } \right| \label{eqn:EstimationError}
 \end{align}
 }%
 where $\epsilon = \boldsymbol{d}_{e} - \hat{\boldsymbol{d}}_{e} $ and $\text{tr}\{\cdot\}$ is the trace operator. 
 \begin{figure}[t] \centering 
	\includegraphics[width=\linewidth]{block_diagram_rmt.png}
	\caption{Block diagram of the virtual sensing control algorithm using remote microphone technique \citep{Jung2018}.}\label{fig:block_diagram_RMT_noDelay}
    \vspace{-0.5\baselineskip}
\end{figure}
