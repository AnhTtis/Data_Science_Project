In RMT implementation, multiple physical monitoring microphones are typically used, and thus the true cross-correlation matrix (CCM) between monitoring microphones $\mathbf{R}_{d_m d_m}$ can be obtained. By assuming incoherence between noise sources, and each CCM associated with that noise source was measured, the estimated CCM for $\hat{\mathbf{R}}_{d_m d_m}$ can be expressed as
{ \small
\begin{align}
    \hat{\mathbf{R}}_{d_m d_m} = \sum_{n_v = 1}^{N_v} z_{n_v} \mathbf{R}_{d_m d_m}^{(n_v)}, \label{eqn:Rmm_cross_decompose}
\end{align}
}%
with its respective source ratio parameter $z_{n_v} = r_{n_v}^2$ derived from \eref{eqn:Oopt_regularised_fir_decompose}. As $\mathbf{R}_{d_m d_m}$ changes with time due to changes in  $z_{n_v}$, the cost function can be formulated by minimizing its $l_2$ norm of the difference between the estimated CCM and its true CCM measured at that time. Thus, the minimization problem can be formulated as
{ \small
\begin{alignat}{2} 
    \min \; J(n)& = &&\left\| \mathbf{R}_{d_m d_m}(n) - \hat{\mathbf{R}}_{d_m d_m}(n)\right\|_2^2 \label{eqn:Cost_function_sourceTrack}\\
    s.t.& &&\quad \mathbf{a} \leq \mathbf{z} \leq \mathbf{b} \nonumber
\end{alignat}
}%
where $\mathbf{z} = [z_1 \; z_2 \; \cdots \; z_{N_v}]^T$ is the squared source ratio vector; $\mathbf{a}=[a_1 \; a_2 \cdots \; a_{N_v}]^T$ and $\mathbf{b}=[b_1 \; b_2 \cdots \; b_{N_v}]^T$ are the element-wise positive lower and upper bound vector constraints, i.e. $a_n, \; b_n \geq 0$ for all $n$. The objective function can thus be formulated with the use of the quadratic penalty function, defined as
{ \small
\begin{align}
    Q &= \left\|\mathbf{R}_{d_m d_m}-\mathbf{\hat{R}}_{d_m d_m}\right\|^2_2 \nonumber \\
    &+ \sum_{n_v=1}^{N_v}\sigma_{n_v} \left[h_{n_v}(z_{n_v}-a_{n_v})^{2} + g_{n_v}(z_{n_v}-b_{n_v})^{2}\right] \label{eqn:Quadratic_penalty_function}
\end{align}
}%
where $\mathbf{h} = [h_1 \; h_2 \; \cdots \; h_{N_v}]^T$ and $\mathbf{g} = [g_1 \; g_2 \; \cdots \; g_{N_v}]^T$ are penalty vectors that serves as a Heaviside function if the constraints were violated, given by 
{ \small
\begin{align}
    h_{n_v} = 
    \begin{cases}
        1, &  z_{n_v} < a_{n_v} \\
        0, & \text{otherwise}
    \end{cases}\; , \;
    g_{n_v}= 
    \begin{cases}
        1, &  z_{n_v} > b_{n_v} \\
        0, & \text{otherwise}
    \end{cases},
\end{align}
}%
and $\sigma_{n_v}$ is the penalty weight. Thus, the derivatives can be shown to be
{ \small
\begin{align}
    \frac{\partial Q}{\partial z_{n_v}} = &2 tr\left\lbrace -\mathbf{R}_{d_m d_m}\mathbf{R}_{d_m d_m}^{(n_v), T} 
    + \mathbf{R}_{d_m d_m}^{(n_v)}\mathbf{\hat{R}}_{d_m d_m}^{T} \right\rbrace\\
    + &2\sigma_{n_v}\left[(z_{n_v}-a_{n_v})h_{n_v} + (z_{n_v}-b_{n_v})g_{n_v}\right] \nonumber.
\end{align}
}%
Assuming that optimal $\mathbf{z}$ remained unconstrained, i.e. $\mathbf{a} < \mathbf{z}_{opt}< \mathbf{b} \text{, and thus } \mathbf{h} = \mathbf{g} = 0$, the optimal squared source ratio $\mathbf{z}_{opt}$ can be obtained by setting the derivatives to zero, that is
{ \small
\begin{align}
    \mathbf{z}_{opt} = \mathbf{A}^{-1}\mathbf{c},
    \label{eqn:zopt}
\end{align}
}%
where
{ \footnotesize
\begin{align}
\mathbf{A} = 
\begin{bmatrix}
\text{tr}\left\lbrace\mathbf{R}_{d_m d_m}^{(1)}\mathbf{R}_{d_m d_m}^{(1), T}\right\rbrace & \cdots & \text{tr}\left\lbrace\mathbf{R}_{d_m d_m}^{(1)}\mathbf{R}_{d_m d_m}^{(N_v), T}\right\rbrace \\
\text{tr}\left\lbrace\mathbf{R}_{d_m d_m}^{(2)}\mathbf{R}_{d_m d_m}^{(1), T}\right\rbrace &
\cdots & \text{tr}\left\lbrace\mathbf{R}_{d_m d_m}^{(2)}\mathbf{R}_{d_m d_m}^{(N_v), T}\right\rbrace \\
\cdots  & \ddots &\vdots \\ 
\text{tr}\left\lbrace\mathbf{R}_{d_m d_m}^{(N_v)}\mathbf{R}_{d_m d_m}^{(1), T}\right\rbrace &
\cdots & \text{tr}\left\lbrace\mathbf{R}_{d_m d_m}^{(N_v)}\mathbf{R}_{d_m d_m}^{(N_v), T}\right\rbrace
\end{bmatrix}
\end{align}
}%
and 
{ \footnotesize
\begin{align}
\mathbf{c} = 
\begin{Bmatrix}
\text{tr}\left\lbrace \mathbf{R}_{d_m d_m} \mathbf{R}_{d_m d_m}^{(1), T} \right\rbrace \\
\text{tr}\left\lbrace  \mathbf{R}_{d_m d_m} \mathbf{R}_{d_m d_m}^{(2), T} \right\rbrace \\
\vdots \\
\text{tr}\left\lbrace \mathbf{R}_{d_m d_m} \mathbf{R}_{d_m d_m}^{(N_v), T} \right\rbrace
\end{Bmatrix}.
\end{align}
}%
Since the optimal approach in \eref{eqn:zopt} is unconstrained and may lead to matrix ill-conditioning with large $N_v$ \citep{Yardibi2010}, an iterative gradient descent approach is adopted, whereby
{ \small
\begin{align}
    z_{n_v}^{(n+1)} = &z_{n_v}^{(n)} - \alpha_{n_v}\Biggl\lbrace \text{tr}\left\lbrace \mathbf{R}_{d_m d_m}^{(n_v)}\mathbf{\hat{R}}_{d_m d_m}^{T}-\mathbf{R}_{d_m d_m}\mathbf{R}_{d_m d_m}^{(n_v), T} \right\rbrace \nonumber \\
    + &\sigma_{n_v}\left[(z_{n_v}^{(n)}-a_{n_v})h_{n_v} + (z_{n_v}^{(n)}-b_{n_v})g_{n_v}\right] \Biggr\rbrace. \label{eqn:SourceRatio_gradientDescent}
\end{align}
}%
As $\mathbf{z}$ is being iterated closer to the optimal value, and therefore getting an accurate reconstruction of $\mathbf{R}_{d_m d_m}$, an accurate estimation of $\mathbf{R}_{d_m d_e}$ will be achieved indirectly as both correlation terms share the same source ratio term, which allows the reconstruction shown from \eref{eqn:Oopt_regularised_fir_decompose} to be implemented in real-time. The normalised estimation error of the source tracking technique across the $n$-th iteration can therefore be expressed given the general form of
{ \small
\begin{align}
    L_{xy}(n) = 10\log_{10}\left( \frac{\left\|\mathbf{R}_{xy}(n)-\hat{\mathbf{R}}_{xy}(n)\right\|^2_2}{\left\|\mathbf{R}_{xy}(n)\right\|^2_2}\right) \label{eqn:estimationError_sourceTrack}
\end{align}
}%
where the $xy$ subscript from \eref{eqn:estimationError_sourceTrack} can either be $d_m d_m$ or $d_m d_e$. 
\newline\newline\noindent
There is a distinction in this approach as compared to other traditional source-localization methods. Unlike the conventional CMF approach or DAMAS that makes use of a steering vector \citep{Yardibi2010a}, this method does not assume a free-field propagation from the noise source to the microphones which allows us to obtain the source ratio in a diffuse field. While it is certainly comparable to the acoustic inverse method \citep{Nelson2000}, this formulation strictly assumes that the noise source clusters are incoherent as explained in \sref{sec:effect_mismatch}. Additionally, the time domain CM is used instead of the cross-spectral density (CSD) in the frequency domain in this formulation which allows the causally constrained time-domain observation filter from \eref{eqn:Oopt_regularised_fir_decompose} to be reconstructed. This method, therefore, is well suited for certain virtual sensing applications, such as separating road noise and wind noise in automobile ANC as the CM caused by the road noise and wind noise can be measured separately.