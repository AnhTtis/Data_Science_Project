\begin{figure}[!t] \centering
    \begin{subfigure}{0.49\linewidth}             \centering
        \includegraphics[width=\linewidth]{Group1_rIteration.pdf}    \vspace{-1.5\baselineskip}
        \caption{}\label{fig:riterate_Group1} 
    \end{subfigure}
        \begin{subfigure}{0.49\linewidth} \centering
        \includegraphics[width=\linewidth]{Group1_jIteration.pdf}    \vspace{-1.5\baselineskip}
        \caption{} \label{fig:Jiterate_Group1}
    \end{subfigure}
    \caption{The plot of (\subref{fig:riterate_Group1}): Source ratio and (\subref{fig:Jiterate_Group1}): Source-tracking estimation error from \eref{eqn:estimationError_sourceTrack} over time using previous measurements from \citep{Lai2022}. $\alpha$ = 5, $\mathbf{a}=0$ and $\mathbf{b} = 5$ were used in this simulation, and the exact source-ratio used for each loudspeakers are $\mathbf{z}=1$. The true correlation matrix $\mathbf{R}_{d_m d_m}$ with the length of 400 is obtained for every time-frame of 200 samples (overlap of 50\%).} \label{fig:r_j_iterateSourceTrack} \vspace{-1\baselineskip}
\end{figure}
To verify the proposed algorithm in \sref{sec:formulation_sourcetrack}, simulations were made on the VS system in the case where all four primary loudspeakers from \fref{fig:Birdeye_placement} were present with a given source ratio parameter of 1, but the source ratio parameters were iterated along sample frame using \eref{eqn:SourceRatio_gradientDescent}. \fref{fig:riterate_Group1}--\subref{fig:Jiterate_Group1} show the iteration plot of the source ratio parameter and the normalised source-tracking estimation error over a period of 10 seconds using \eref{eqn:SourceRatio_gradientDescent} and \eref{eqn:estimationError_sourceTrack}. The iteration plot of the source ratio parameter showed convergence for all noise sources to around 1, with $r_1$ and $r_3$ converging quicker than $r_2$ and $r_4$. As the true $\mathbf{R}_{d_m d_m}$ with a length of 400 and overlap of 50\% has been obtained throughout the time frame, it is expected for the source ratio parameter to have random fluctuations from its stochastic nature, thus validating \eref{eqn:SourceRatio_gradientDescent} from obtaining its source ratio parameter. This convergence is also supported in \fref{fig:Jiterate_Group1} where $L_{d_m d_m}$ converges to around -13dB with expected random fluctuations. Interestingly, the estimation error of $\mathbf{R}_{d_m d_e}$ has decreased with time and converges at about -10dB even if it's indirectly estimated, which verifies the estimation concept described in \sref{sec:formulation_sourcetrack} where a good correlation in $L_{d_m d_m}$ and $L_{d_m d_e}$ is shown in \fref{fig:Jiterate_Group1}. Once $\mathbf{z}$ has been found through iteration, the observation filter due to source-tracking algorithm $\hat{\boldsymbol{O}}_{st}$ will be iteratively updated and was used to simulate the estimated error signals. It has been shown from the estimation error plot from \fref{fig:EstimationError_SourceTrack} that the estimation spectra between the nominal observation filter $\boldsymbol{O}_{opt}$ and source tracking observation filter $\hat{\boldsymbol{O}}_{st}$, which ultimately verifies the source-tracking algorithm.
\begin{figure}[!t] \centering 
    \tikzsetnextfilename{Group1_estimationSourceTrack.tikz}
	\includegraphics[width=\linewidth, height=0.171\textheight]{Group1_estimationSourceTrack.tikz}
	\caption{The estimation error spectra when all 4 loudspeakers from \fref{fig:Birdeye_placement} were present, using the optimal observation filter $\boldsymbol{O}_{opt}$, estimated observation filter $\hat{\boldsymbol{O}}$ with $\mathbf{z}=1$ and estimated observation filter with the use of source tracking iteration $\hat{\boldsymbol{O}}_{st}$.} \label{fig:EstimationError_SourceTrack}
\end{figure}
\begin{comment}
previous measurements obtained from \citep{Lai2022} will be used to estimate the source ratio parameter $\mathbf{z}$ through the source-tracking algorithm from \sref{sec:formulation_sourcetrack}. \fref{fig:r_j_iterateSourceTrack} shows the source ratio parameter and the normalised source-tracking estimation error plot over time using \eref{eqn:SourceRatio_gradientDescent}. Quick convergence of $\mathbf{z}$ can be achieved as shown from \fref{fig:riterate_Group1} where the source ratio vector has converged to close to 1 in 1 second. While the estimation error of $\mathbf{R}_{d_m d_m}$ has decrease along iteration to about -13dB as expected, shown in \fref{fig:Jiterate_Group1}, the estimation error of $\mathbf{R}_{d_m d_e}$ has decrease with time and converges at about -10dB, which verifies the estimation concept described in \sref{sec:formulation_sourcetrack}. Once $\mathbf{z}$ has been found through iteration, the observation filter due to source-tracking algorithm $\hat{\boldsymbol{O}}_{st}$ will be iteratively updated and was used to simulate the estimated error signals. With the use of the source-tracking algorithm, a good estimation has been achieved as shown in \fref{fig:EstimationError_SourceTrack} as the estimation spectra is similar to the optimal observation case $\boldsymbol{O}_{opt}$.
\end{comment}
