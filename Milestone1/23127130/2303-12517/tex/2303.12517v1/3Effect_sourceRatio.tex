On the assumption of incoherence between modelled noise sources, it can be shown that the CM from \eref{eqn:Oopt_freq_beta} and \eref{eqn:Oopt_regularised_fir} can be further decomposed into a sum of CMs, with each associated to the respective noise sources \citep{Lai2022}. This allows the OF to be reconstructed in real-time based on the current primary acoustic field, given by
{ \footnotesize
\begin{align}
    \boldsymbol{O}_{opt}(j\omega) = \left( \sum_{n_v = 1}^{N_v} r_{n_v}^{2}\boldsymbol{S}_{d_m d_e}^{(n_v)} \right) \left[\left(\sum_{n_v = 1}^{N_v} r_{n_v}^{2}\boldsymbol{S}_{d_m d_m}^{(n_v)} \right) + \beta\boldsymbol{I}\right]^{-1} \label{eqn:Oopt_freq_decompose}
\end{align}
}%
and
{ \footnotesize
\begin{align}
    \boldsymbol{O}_{opt} = \left(\sum_{n_v = 1}^{N_v} r_{n_v}^{2}\mathbf{R}_{d_m d_e}^{(n_v)}\right)^{T}\left[\left(\sum_{n_v = 1}^{N_v} r_{n_v}^{2}\mathbf{R}_{d_m d_m}^{(n_v)}\right)+\beta \mathbf{I}\right]^{-1}, \label{eqn:Oopt_regularised_fir_decompose}
\end{align}
}%
where $N_v$ is the total number of the modelled primary sources in the system and $r_{n_v}$ denotes the source strength ratio at the $n_v$-th modelled primary source relative to its calibrated source strength.
\begin{figure}[!t] \centering
\includegraphics[width=0.9\linewidth]{Figures/Birdeye_placement_drawing.png}
\caption{Top view of the virtual sensing system for an open aperture, showing the arrangement of primary loudspeakers (Genelec 8302A) $v_1\;$--$\;v_4$, physical monitoring microphones $m_1\;$--$\;m_3$ and virtual error microphones $e_1\;$--$\;e_5$ (Pro Signal NPA415-OMNI). Signals were acquired and computed on a low-latency system (NI PXIe-8880) at a sampling frequency $F_s = 10 \text{kHz}$. \citep{Lai2022}.}\label{fig:Birdeye_placement}
    \vspace{-0.5\baselineskip}
\end{figure}
\newline \newline
As $r_{n_v}$ varies with time in a real-time implementation, it is important to understand the significance of this parameter. Additional real-time experiments from \citep{Lai2022} were conducted with its arrangement shown in \fref{fig:AmpCompare_Oopt}. Each loudspeaker reproduced white Gaussian noise during the calibration stage to obtain the individual CMs $\mathbf{R}_{d_m d_e}^{(n_v)}$ and $\mathbf{R}_{d_m d_m}^{(n_v)}$, which will be used to reconstruct the OF from \eref{eqn:Oopt_regularised_fir_decompose}. \fref{fig:AmpCompare_Group21_22_compare}--\ref{fig:AmpCompare_Group23_24_compare} showed the estimation error spectra when both $v_1$ and $v_2$ reproduced known amplitude ratios. While the nominal OF $\boldsymbol{O}_{opt}$ was directly obtained from both loudspeakers with the new amplitude ratio, the correctly estimated and mismatched OF $\hat{\boldsymbol{O}}$ and $\hat{\boldsymbol{O}}_{mis}$ uses the CM obtained from the calibration stage where the original amplitude was used for each individual loudspeaker, followed by the superposition formulation from \eref{eqn:Oopt_regularised_fir_decompose} using the correct and mismatched amplitude ratio input. The correctly estimated OF for both \fref{fig:AmpCompare_Group21_22_compare} and \ref{fig:AmpCompare_Group23_24_compare} showed a similar estimation spectra with the nominal OF which effectively validates \eref{eqn:Oopt_regularised_fir_decompose}. However, the estimation error can degrade when the wrong amplitude ratio is used. While \fref{fig:AmpCompare_Group21_22_compare} showed a decrease in estimation error at frequencies 400--600Hz, higher frequency region from 800--1000Hz did not show much change. \fref{fig:AmpCompare_Group23_24_compare} showed a larger difference in estimation error in a wider frequency range, suggesting a larger degradation in estimation error when its mismatch becomes larger. Thus, it can be concluded that the source ratio $r_{n_v}$ would play an important role to achieve robust estimation.
\begin{figure}[!t] \centering
\begin{subfigure}{0.49\linewidth}    \centering
\includegraphics[width=\linewidth]{EstimationError_Oopt_AmplitudeMatching_Group21_22.pdf}
\caption{$r_1 = 1.2, r_2 = 0.8$\\$r_{1, mis} = 0.8, r_{2, mis} = 1.2$}\label{fig:AmpCompare_Group21_22_compare} 
\end{subfigure}
\begin{subfigure}{0.49\linewidth} \centering
\includegraphics[width=\linewidth]{EstimationError_Oopt_AmplitudeMatching_Group23_24.pdf}
\caption{$r_1 = 1.5, r_2 = 0.5$\\ $r_{1, mis} = 0.5, r_{2, mis} = 1.5$} \label{fig:AmpCompare_Group23_24_compare}
\end{subfigure}
\caption{The estimation spectra when loudspeaker 1 and 2 as arranged from \fref{fig:Birdeye_placement} \citep{Lai2022} were present, using the optimal OF $\boldsymbol{O}_{opt}$, correctly predicted OF  $\hat{\boldsymbol{O}}_{opt}$  and the mismatched OF $\hat{\boldsymbol{O}}_{mis}$ calculated from \eref{eqn:Oopt_regularised_fir_decompose}.} \label{fig:AmpCompare_Oopt}
\end{figure}
\begin{comment}
\begin{figure}[t]
\begin{subfigure}{\linewidth}    \centering
\tikzsetnextfilename{EstimationError_Oopt_AmplitudeMatching_Group21_22}
\includegraphics[width=4cm]{EstimationError_Oopt_AmplitudeMatching_Group21_22.tikz}
    \caption{ \label{fig:AmpCompare_Group21_22_compare} }
\end{subfigure}

\begin{subfigure}{\linewidth} \centering
\tikzsetnextfilename{EstimationError_Oopt_AmplitudeMatching_Group23_24}
\includegraphics[width=4cm]{EstimationError_Oopt_AmplitudeMatching_Group23_24.tikz}
\caption{\label{fig:AmpCompare_Group23_24_compare} }
\end{subfigure}
\end{figure}
\end{comment}