\documentclass[a4paper, twoside, 12pt]{article}

% Add the code in the 'template.tex' file. Check this file to see which packages are already loaded.
\documentclass{article}

\usepackage{arxiv}

\usepackage[utf8]{inputenc} % allow utf-8 input
\usepackage[T1]{fontenc}    % use 8-bit T1 fonts
\usepackage{hyperref}       % hyperlinks
\usepackage{url}            % simple URL typesetting
\usepackage{booktabs}       % professional-quality tables
\usepackage{amsfonts}       % blackboard math symbols
\usepackage{nicefrac}       % compact symbols for 1/2, etc.
\usepackage{microtype}      % microtypography
\usepackage{lipsum}		% Can be removed after putting your text content
\usepackage{graphicx}
\usepackage{float} %force graph positionning
\usepackage{subcaption}
\usepackage[square,numbers]{natbib} %package for biblio with numbers
\usepackage{doi}
%Image-related packages
\usepackage{graphicx}
\usepackage{subcaption}
\usepackage[export]{adjustbox}
\usepackage{wrapfig}
\usepackage{siunitx}
\usepackage{xcolor, color, soul}
\usepackage{multirow}
\sisetup{output-exponent-marker=\ensuremath{\mathrm{e}}}

\title{Evaluating self-attention interpretability through human-grounded experimental protocol}

%\date{September 9, 1985}	% Here you can change the date presented in the paper title
%\date{} 					% Or removing it

\author{ \href{https://myedb.edite-de-paris.fr/Fiche/41003/}{\includegraphics[scale=0.06]{orcid.pdf}\hspace{1mm}Milan Bhan} \\
	Ekimetrics\\
	Sorbonne University\\
	Paris \\
	\texttt{milan.bhan@ekimetrics.com} \\
	%% examples of more authors
	\And
	\href{https://www.linkedin.com/in/nina-achache/}{\includegraphics[scale=0.06]{orcid.pdf}\hspace{1mm}Nina Achache} \\
	Ekimetrics\\
	Paris \\
	\texttt{nina.achache@ekimetrics.com} \\
     \And
    	\href{https://www.linkedin.com/in/legrand-victor/}{\includegraphics[scale=0.06]{orcid.pdf}\hspace{1mm}Victor Legrand} \\
    	Ekimetrics\\
    	Paris \\
    	\texttt{victor.legrand@ekimetrics.com} \\
     \And
    	\href{https://www.linkedin.com/in/annabelleblangero/?originalSubdomain=fr}{\includegraphics[scale=0.06]{orcid.pdf}\hspace{1mm}Annabelle Blangero} \\
        Aix-Marseille University \\
    	Ekimetrics\\
    	Paris \\
    	\texttt{annabelle.blangero@ekimetrics.com} \\
    \And
    	\href{https://www.linkedin.com/in/nicolas-chesneau-0416b94/}{\includegraphics[scale=0.06]{orcid.pdf}\hspace{1mm}Nicolas Chesneau} \\
    	Ekimetrics\\
    	Paris \\
    	\texttt{nicolas.chesneau@ekimetrics.com} \\
	%% \AND
	%% Coauthor \\
	%% Affiliation \\
	%% Address \\
	%% \texttt{email} \\
	%% \And
	%% Coauthor \\
	%% Affiliation \\
	%% Address \\
	%% \texttt{email} \\
	%% \And
	%% Coauthor \\
	%% Affiliation \\
	%% Address \\
	%% \texttt{email} \\
}

% Uncomment to remove the date
%\date{}

% Uncomment to override  the `A preprint' in the header
% \renewcommand{\headeright}{Technical Report}
% \renewcommand{\undertitle}{Technical Report}
% \renewcommand{\shorttitle}{\textit{arXiv} Template}
\renewcommand{\shorttitle}{Bhan et al.}

%%% Add PDF metadata to help others organize their library
%%% Once the PDF is generated, you can check the metadata with
%%% $ pdfinfo template.pdf
\hypersetup{
pdftitle={A template for the arxiv style},
pdfsubject={q-bio.NC, q-bio.QM},
pdfauthor={David S.~Hippocampus, Elias D.~Striatum},
pdfkeywords={Interpretability, Explainability, Attention, Semantic congruency, Natural language processing},
}

\begin{document}
\maketitle

\begin{abstract}
Attention mechanisms have played a crucial role in the development of complex architectures such as Transformers in natural language processing. However, Transformers remain hard to interpret and are considered as black-boxes. This paper aims to assess how attention coefficients from Transformers can help in providing interpretability.  A new attention-based interpretability method called CLaSsification-Attention (CLS-A) is proposed. CLS-A computes an interpretability score for each word based on the attention coefficient distribution related to the part specific to the classification task within the Transformer architecture. A human-grounded experiment is conducted to evaluate and compare CLS-A to other interpretability methods. The experimental protocol relies on the capacity of an interpretability method to provide explanation in line with human reasoning. Experiment design includes measuring reaction times and correct response rates by human subjects. CLS-A performs comparably to usual interpretability methods regarding average participant reaction time and accuracy. The lower computational cost of CLS-A compared to other interpretability methods and its availability by design within the classifier make it particularly interesting. Data analysis also highlights the link between the probability score of a classifier prediction and adequate explanations. Finally, our work confirms the relevancy of the use of CLS-A and shows to which extent self-attention contains rich information to explain Transformer classifiers.
\end{abstract}


% keywords can be removed
\keywords{Interpretability \and Explainability \and Attention \and Semantic congruency \and Natural language processing}


\section{Introduction}
The field of machine learning (ML) has witnessed great advance in recent years. ML algorithms have achieved high levels of performance in a wide variety of tasks thanks to their rapid development and improving complexity. The complexity of these new models has led to an increasing difficulty in understanding and interpreting their predictions. The field of eXplainable Artificial Intelligence (XAI) has emerged to overcome this lack of transparency by developing methods of "interpretability" or "explicability". Most commonly used interpretability methods are time consuming and usually based on strong hypothesis such as features’ independence and linear approximation~\cite{poursabzi2021manipulating}. The explanation provided can differ significantly from one method to another.
Natural language processing (NLP) has been particularly concerned by recent breakthroughs in ML with the development and democratization of Transformer-type models \cite{Attention}. Attention mechanism \cite{Neural} is a crucial component in Transformer architecture, enabling models to focus on specific parts of the input text. However, the interpretability of attention remains questionable~\cite{notnotXP, notXP}. The plurality of interpretability methods and the differences in the resulting explanations raise the question of their assessment and comparison.

This paper aims to assess the interpretability of attention mechanisms in Transformers models. A new method which we call CLaSsification-Attention (CLS-A) is built based on attention weights from Transformers. Additionally, this work aims to illustrate how human-based experimental protocols can be good alternatives for interpretability comparison.
We first introduce key notions of Transformers architecture and interpretability. We then introduce our new attention-based method CLS-A and the experimental protocol used to assess its interpretability. Finally, we analyze the experimental results which validate the CLS-A's interpretability.
Our work adds to the literature questioning the interpretability of attention coefficients in recent Transformer models. CLS-A seems particularly appropriate to explain attention-based classifiers in NLP since we validate its interpretability and because attention is available at no cost within the model architecture.

\section{Background \& related work}
\label{sec:headings}
This section introduces some key notions about Transformers architectures and interpretability as well as an overview of existing approaches to compare XAI methods.  
\subsection{Background}
\paragraph{Transformers, attention and BERT} In NLP, Transformer-like models have achieved high levels of performance in a variety of tasks, such as text summarization, question answering or named entity recognition. These models are particularly complex, with a number of parameters that can exceed the billion \cite{DistilBERT}. The Transformer architecture is based on multi-head self-attention mechanisms, aiming at making learning more efficient \cite{Neural} by encoding the relations between words. The model attends to different parts of the input in parallel, using multiple self-attention heads. A self-attention head takes as input a triplet ($Q$, $K$, $V$) and outputs a representation as formalized in the following formula :
\begin{equation}
\label{eqn:attention}
\texttt{Attention}(Q,K,V) = \texttt{softmax}(\frac{QK^{T}}{\sqrt{d_{k}}})V
\end{equation}
Where:
\begin{itemize}
    \item $Q$ (query) is a matrix that represents the input in which the attention mechanism focuses on. 
    \item $K$ (key) is a matrix that represents the different elements in the input that the attention mechanism can attend to.
    \item $V$ (value) is a matrix that represents the output of the attention mechanism.
    \item $d_{k}$ is the dimension of matrices $Q$ and $V$ and allows to stabilize the model during the training phase
\end{itemize}
Hence, each head has its own set of parameters, allowing the model to learn different types of attention patterns. The attentions resulting from each of the heads are then concatenated and projected on a dense layer. 

Bidirectional Encoder Representations from Transformers (BERT) models \cite{devlin_bert_2019} is a stack of $n$ encoders from the Transformer architecture. Each BERT layer contains $h$ attention heads with its own set of weights, which have been learned during training. These weights determine how the model will attend to different parts of the input when making a prediction. In this way, words are related to each other even in the case of long-term dependency. BERT has been widely adopted and has achieved state-of-the-art performance on a variety of benchmarks such as GLUE\cite{wang-etal-2018-glue} for natural language understanding (NLU). 

One of the key features of BERT is its bidirectional nature. Unlike previous models that were only trained to look \textit{before} or \textit{after} a word, BERT is trained to look \textit{before} and \textit{after} a word at the same time. This allows BERT to understand the full context of a word and improve its performance on NLU tasks. Moreover, BERT has several advantages such as its scalability, the fact that it can be parallelized\cite{deep_mind_transfo} and that it can compute longer sequences. The \textit{CLS} token (short for "classification") in BERT is a special token added to the beginning of the input text that the model uses as a representation of the entire input sentence or document. The final hidden state of the \textit{CLS} token, which is a fixed-size vector, is typically used as the input to a classifier or other downstream task. This allows BERT to take into account information from the entire sentence or document when making a prediction.

\paragraph{Local feature importance.} There are several ways to interpret black-box systems such as BERT models \cite{guidotti_counterfactual_2022}. One of the main approaches consists in computing local feature importance \cite{molnar_interpretable_2020}. When the model to explain is a classifier, the purpose is to compute contribution coefficients to the probability score of the predicted class. It can be done by considering ML models as black-boxes and explaining their predictions \textit{ex post}, without referring to their inherent parameters. This kind of approach is called \textit{post-hoc model-agnostic}\cite{MLint_Survey}. Another way to compute local feature importance is to use the information contained within the model architectures\cite{Axio, Importance}, which can significantly reduce the computation time. However, it makes the method rely strongly on the model it tries to explain and makes its use less flexible. This kind of approaches is referred to as \textit{post-hoc model-specific}. The large number of local feature importance methods can make it difficult to choose the most appropriate one \cite{SanityChecks}.

Due to their flexibility and the plurality of data types on which they can be applied, Linear Interpretable Model-agnostic Explainer (LIME) \cite{trust} and SHapley Additive exPlanations (SHAP) \cite{unified} are the most frequently used interpretability methods in the industry \cite{XAI}. LIME offers to explain a prediction locally using models that are interpretable by design such as sparse regressions. The algorithm artificially generates data points in a neighborhood around the instance to explain and fits an interpretable model on these new examples. The SHAP method is inspired by the Shapley values\cite{shapley1953value} from economics and game theory. It aims to distribute fairly the rewards from a set of games to all the players. By associating the features of a model to the players to whom the gains are distributed, the prediction associated with an instance is decomposed by feature, which allows to compute each feature contribution.

\subsection{Related work}
The interpretability of the attention coefficients is still an open question \cite{notnotXP, notXP, Vashishth2019AttentionIA, serrano-smith-2019-attention}. Several methods based on self-attention coefficients have been proposed to explain the predictions made by Transformer-type models, such as attention flow and attention roll-out~\cite{flow_att}. These methods are based on complex aggregators to synthesize the information contained in the attention layers. Visualization tools allowing to dive in detail into the self-attention coefficients have been developed as well~\cite{vig-2019-multiscale}. Visualizing attention is the basis of saliency map approaches specific to Computer Vision for Vision Transformers~\cite{bastings-filippova-2020-elephant, DBLP:journals/corr/abs-2012-09838}. If these approaches allow to assign a local importance to input data, the quality of the explanation produced is not rigorously assessed. This raises the question of the evaluation of interpretability.

One way to assess the interpretability of a given method is to compare it quantitatively \cite{notXP} to common interpretable approaches such as SHAP. A given method would thus be considered interpretable if it strongly correlates with one target local feature importance method. This comparison criterion cannot be applied to approaches based on attention coefficients though, since attention is a \textit{positive} probability while LIME and SHAP contributions can be both \textit{positive} and \textit{negative}. Moreover, this is restrictive because the latter methods would represent ground truths to replicate. 

Human-grounded evaluation is needed in order to reach a rigorous science of interpretable machine learning and thus experimental approaches can be alternatives~\cite{been_kim_rigourous}. Therefore, interpretability methods can also be compared by evaluating to what extend they improve human performance during a specific annotation task \cite{ poursabzi2021manipulating}. In the context of NLP, it consists in asking humans to guess which class a text belongs to, in which words have been highlighted according to their local feature importance\cite{Quantifying, empirical_xai}. The response time and the average accuracy are then measured and the results are compared between the different methods. The method with the best response time/accuracy compromise is then considered as the most appropriate. This kind of experimental protocol has the advantage of quantifying the quality of an explanation, as long as the explanation is intended as a decision (in this case annotation) support tool. However, in the absence of robust statistical analysis, we find the work conducted by the authors to be insufficient to establish the validity of their conclusion.

\section{Methodology}
\label{sec:method}
We present a novel method, CLS-A, based on Transformer model attention coefficients. Then we define an experimental protocol inspired by the one previously introduced\cite{Quantifying} that we apply on three different annotation tasks of binary classification. Thereafter we provide implementation details of our experiment. Finally we introduce the evaluation protocol with data description, linear and non linear modeling.
\subsection{CLS-A}
We introduce our local feature importance method based on BERT self-attention that we call CLS-A. Since all the information needed for the classification task of a BERT goes through the \textit{CLS} token, we use the rich information contained in its related attention coefficients. We call context the distribution of attention linking a word in the input text to the rest of the sequence. Thus, CLS-A is constructed to represent the average context of the \textit{CLS} token.

Motivated by the fact that intermediate representations of deep neural networks become more abstract with network depth \cite{DBLP:journals/corr/YosinskiCNFL15}, we focus on the last layer of the BERT architecture. Since this last layer contains $h$ different self-attention matrices, the information has to be aggregated in order to build a one-dimensional local feature importance explanation. We aggregate the information contained by the $h$ attention heads by applying the average operator. This results in an average context of the classification token in the last layer of BERT. A weight is assigned to each word of the input text, representing its importance in the context that induced the prediction of the classifier.

Since the CLS token plays a central role in the computation of CLS-A, it is recommended that the BERT forward pass passes through the CLS token in order to perform its prediction. In the case where the BERT forward pass does not pass exclusively through the CLS token, an alternative would be to compute the average of all the coefficients of the attention heads.

\subsection{Experimental protocol}
The experimental protocol consists in asking participants to annotate one hundred texts in a binary classification task. Each text has some of its words colored with a more or less intense shade of blue, based on an underlying interpretability method or a random generator. The higher the coefficient of the method, the more strongly the word will be highlighted in blue. Accuracy and response time are measured to evaluate each method's ability to assist the participant in the annotation task. The higher the accuracy and the shorter the response time, the more relevant the method as it facilitates the human semantic processing of the text.

\paragraph{Setup and instructions.} 

All participants take part in the experiment in the same room and can be up to three at the same time. They are isolated in order to limit any other exogenous influence (visual, sound) and are placed in front of a computer as depicted in Figure~\ref{fig:schemaXP}. An explanation of the protocol is displayed on the screen to put the participants in the right conditions. In order to perform the annotation task, participants are asked to press either one of two buttons corresponding to the two possible answers as shown in Figure~\ref{fig:schemaXP}. The buttons correspond to keys on the keyboard of the computer used. Two colored stickers are stuck on the keys to help locate them. When a text is annotated (response given/key pressed), the next one is displayed on the screen.  

Three different classification tasks are tested. The first one (Experiment 1) is to evaluate the sentiment associated with a movie review. The participant must choose between the "positive" and "negative" sentiment. The second and third (Experiment 2 and 3) are film genre evaluations. In Experiment 2, the task is to distinguish between horror and comedy films, in Experiment 3 between action and drama. The information given in the explanation of the protocol differs depending on the classification task. Participants annotating film genres are asked to strive for a response time/accuracy trade-off. They are told that displayed colors can potentially be useful in the annotation task. Participants annotating movie review sentiments have no information about the colors displayed and no incentive to respond quickly.

\begin{figure}[t] 
    \centering
    \includegraphics[width=0.7\textwidth]{Images/schemaXP.png}
    \caption{Scheme of the experimental protocol. Each participant labels 100 different texts after reading the instructions. The participant has two possible answers, depending on the  experiment he/she is participating in. The text is colored according to the interpretability method used to explain the classifier's prediction. The selected texts are all classified properly by the classifier.}
    \label{fig:schemaXP}
\end{figure}

\paragraph{Experiment characteristics.} All the 100 participants work in the same data science consulting firm. The first experiment involved 50 participants while the other two had 25 each. The participants were predominantly male: about 76\% compared to 24\% for women. They were between 22 and 40  years old and none of the subjects were cognitively impaired to our knowledge.

Every participant is asked to annotate 100 different texts in a binary classification task. The asked classification task remains the same during the whole experiment. A participant cannot see the same text twice during the experiment. Each text has its words colored in different shades of blue. This coloring is proportional to the coefficients chosen at random among a random baseline, SHAP, LIME and CLS-A  built on the DistilBERT classifier attention. The random baseline assigns randomly a coefficient to each word. The feature importance obtained from LIME and SHAP to color the text are truncated to be exclusively positive. Thus, the negative SHAP and LIME values are set to 0 in order to highlight only the words contributing positivly to the predicted class. This truncation of SHAP and LIME make them more comparable to the attention coefficients. Participants are thus subjected to exogenous attentional orientation effects in order to compare the methods one-to-one. An example of the text displayed during the experiment is plotted in Figure~\ref{fig:schemaXP}.
 
The classes of the various classification tasks are all equally represented among the displayed texts. Only the instances that the models predicted correctly were selected. We wanted to look at the effect of the review length and the prediction probabilities in Experiments 2 and 3. The reviews corresponding to the sentiment analysis task contain between 32 and 50 words. The text sequence lengths of the other classification tasks vary between 19 and 145 words. The probability scores of belonging to the target class are highly polarized for the sentiment analysis and the horror/comedy classification while probability score is more uniformly distributed for the action/drama classification task.
We assume that an interpretability method provides good explanations to the extent that it helps an annotator to go faster and be more efficient. An explanation will then be the object of a semantic congruence between the label to be predicted and the words highlighted. Therefore, the response time is precisely measured for each text and the correctness of the answers is assessed.

\subsection{Implementation details}
The three classifiers analyzed during the three annotation tasks were based on a DistilBERT\cite{sanh_distilbert_2020}, a stack of 6 encoders being the result of a distillation of a classical BERT. Each pre-trained DistilBERT was retrieved from Hugging Face\footnote{www.huggingface.co/}. A dense layer was added to perform the classification and fine-tune each model. The forward pass was defined as getting the embedding of the CLS token to perform the classification task. The library used to compile and fine-tune the models were Keras on the TensorFlow framework. Each model was trained with an initial learning rate of $10^{-5}$ and a reducing learning strategy when reaching a plateau. The number of epochs was for each model of 5 and the batch size was 32. The models were trained with a categorical crossentropy loss and the Adam optimizer. The first model was fine-tuned for sentiment analysis on the IMDB database\cite{maas_learning_2011}. The second and third one were fine-tuned to perform movie genre classification on a Kaggle dataset\footnote{www.kaggle.com/competitions/movie-genre-classification/overview}.

For each text, SHAP was computed with the \verb|shap| library~\cite{unified}. The Shapley values were computed in a permutation way. Finally, LIME was computed with the \verb|lime|~\cite{trust} library. The whole experiment was performed on the \verb|psychopy|~\cite{peirce2019psychopy2} framework on Python.

\subsection{Data analyses}
Here we define the methods used to analyze the data produced by the experimental protocol presented above. Each experiment produces $n \times 100$ answers, with $n$ the number of participants, and 100 the number of plotted text samples during each experiment. The indicators of interest are the labeling time, which we call "reaction time", and whether or not the participant is wrong, which we call "accuracy". These variables of interest are then analyzed through their relationship with other characteristics such as features about the text (sequence length, probabiltiy score, trial number, relative position of impacting word) and the interpretability method used to color it.

\paragraph{Data description.} The descriptive analysis is first performed by calculating the average reaction time and the average accuracy. The one-tail $t$-test is then used to compare the distributions of reaction times between interpretability methods and the random baseline in order to have statistically significant comparisons. The one-tailed $t$-test is a statistical hypothesis test used to determine whether the mean of a first sample is lower the mean of a second one. This test is applied here to the average difference between each interpretability method and the random baseline, per participant, per experiment.
\paragraph{Linear modeling.} The impact of interpretability methods on reaction time is estimated with a linear regression by incorporating the effect of other explanatory variables. 
% A linear model assumes that the explanatory variables are independent and can be formalized as follows:

% \begin{equation}
% y = \beta_0 + \sum_{i}\beta_i x_i 
% \end{equation}

% Where:
% \begin{itemize}
%     \item $y$ is the reaction time 
%     \item $\beta_0$ is the intercept
%     \item $\beta_i$ is the coefficients for the variables $x_i,$
% \end{itemize}

The random baseline is used for reference. Thus, the coefficients of the linear regression associated with the method used to color the text are expressed with respect to this baseline. 

For each experiment, a linear model is built per participant to explain the reaction time to the labeling task. The explanatory variables of the models for an experiment are the same for all participants. The mean value of the regression parameters and their distribution are then analyzed using the one-tail $t$-test presented above.
\paragraph{Non-linear modeling.} Decision tree boosting algorithms allow to model complex and non-linear phenomena. We apply this type of algorithm to model the participant accuracy. This binary classification problem is addressed via Explainable Boosting Machine (EBM) \cite{nori2019interpretml}. EBM obtain performance levels equivalent to other boosting approaches based on decision trees, while decomposing its prediction into contributions of the explanatory variables. EBM is a generalized additive model (GAM) of the form:
\begin{equation}
g(\mathbb{E}[y]) = \beta_0 + \sum_{i}f_{i}(x_i) + \sum_{i,j, i\ne j}f_{i,j}(x_{i},x_{j})
\end{equation}

Where:
\begin{itemize}
    \item $y$ is the variable indicating whether a participant has successfully completed its labeling 
    \item $g$ is the link function
    \item $\beta_0$ is the intercept
    \item $f_i$ is the feature response function of the variable $x_i,$
    \item $f_{i,j}$ is the pairwise interaction function of the two variables $x_{i}$ and $x_{j}$
\end{itemize}
 A response curve reflects the impact of a given explanatory variable by plotting the evolution of its contribution to the target variable. One model is fitted per method and per experiment in order to compare the response curves of the methods within a given experiment. Each model has to be trained with the same explanatory variables. Since the participants generally perform their annotation tasks accurately, the data are largely imbalanced. Sub-sampling is then performed to run the EBM on a balanced dataset with as many right and wrong answers. Since this sub-sampling induces sampling bias, this operation is run again 50 times. Average response curves and standard deviations are then calculated to produce the results in the analysis. 

\section{Results}
\label{sec:results}
This section reports the analysis of the data produced by the three experiments with the data analysis methods introduced in Section~\ref{sec:method}. LIME and SHAP are compared to CLS-A and a random baseline. We show that CLS-A improves both speed and accuracy of annotation in a statistically significant way compared to the random baseline. CLS-A, SHAP and LIME result in statistically similar response times and accuracy. Furthermore, we highlight the relationship between the quality of an explanation and the certainty of the classifier's prediction.

\subsection{Data description}
\begin{table}[t]
\centering
\begin{tabular}{@{}cccccc@{}}
\toprule
Metrics                            & Experiment & CLS-A          & LIME          & SHAP          & Random \\ \midrule
\multirow{3}{*}{Reaction Time (s)} & Exp 1      & \textbf{10.52} & 10.83         & 10.74         & 11.21  \\
                                   & Exp 2      & 9.13           & 8.72          & 8.74          & 8.79   \\
                                   & Exp 3      & \textbf{10.85} & 11.25         & 11.66         & 12.28  \\ \midrule
\multirow{3}{*}{Accuracy (\%)}     & Exp 1      & \textbf{96.1}  & \textbf{96.4} & 95.4          & 95.2   \\
                                   & Exp 2      & \textbf{80.9}  & 79.1          & \textbf{80.4} & 79.9   \\
                                   & Exp 3      & \textbf{85.9}  & \textbf{84.9} & 80.0          & 83.0   \\ \bottomrule
\end{tabular}
\caption{Average reaction time and accuracy per experiment per method.}
      \label{tab:table1}
\end{table}

% \begin{table}[t]
%       \centering
%       \begin{tabular}{llllll}
%         \toprule
%         \cmidrule(){1-6}
%         Metrics     & Experiment   & CLS-A & LIME & SHAP & Random \\
%         \midrule
%         \multirow{Reaction Time (s)}{3}
%         & Exp 1 & $\textbf{10.52}$ & $10.83$ & $10.74$ & $11.21$\\
%         & Exp 2 & $9.13$ &$\textbf{8.72}$ & $8.74$ & $8.79$ \\
%         & Exp 3 & $\textbf{10.85}$ & $11.25$ & $11.66$ & $12.28$\\
%         \midrule
%         \multirow{Accuracy (\%)} & Exp 1 & $\textbf{96.1}$ & $\textbf{96.4}$ & $95.4$ & $95.2$\\
%         & Exp 2 & $\textbf{80.9}$ &$79.1$ & $\textbf{80.4}$ & $79.9$ \\
%         & Exp 3 & $\textbf{85.9}$ & $\textbf{84.9}$ & $80.0$ & $83.0$\\
%         \bottomrule
%       \end{tabular}
%       \caption{Average reaction time and accuracy per experiment per method.}
%       \label{tab:table1}
%     \end{table}
    
This section provides an exploratory analysis of participant's responses to the three experiments as presented in Section~\ref{sec:method}. The first experiment consisted of responses from 100 participants while the other two had 50 each. Table~\ref{tab:table1} relates the average reaction time and the average accuracy per experiment and per method used to color the text. This shows that the average reaction time related to CLS-A is lower for experiments 1 and 3. Accuracy is also on average higher for participants who were exposed to CLS-A. The random baseline induces less accurate and slower responses overall.

We also compare the distributions of the average response time of the CLS-A, LIME and SHAP methods in comparison to the random baseline. We perform this distribution comparison using the one-tailed $t$-test on the average difference between the coloring method and the baseline, per participant as presented in Section~\ref{sec:method} . Figure~\ref{fig:distrib time reaction minus random} plots the distribution of the average reaction time deviation from the random baseline with the results of the $t$-tests, by method and by experiment. 

% \begin{figure}[H]
% \begin{subfigure}{0.3333\textwidth}
% \includegraphics[width=0.9\linewidth, height=5cm]{Images/xp1/xp1_distrib_time.png} 
% \caption{Experiment 1}
% \label{fig:xp1_distrib_time}
% \end{subfigure}
% \begin{subfigure}{0.3333\textwidth}
% \includegraphics[width=0.9\linewidth, height=5cm]{Images/xp2/xp2_distrib_time.png}
% \caption{experiment 2}
% \label{fig:xp2_distrib_time}
% \end{subfigure}
% \begin{subfigure}{0.3333\textwidth}
% \includegraphics[width=0.9\linewidth, height=5cm]{Images/xp3/xp3_distrib_time.png}
% \caption{experiment 3}
% \label{fig:xp3_distrib_time}
% \end{subfigure}

\begin{figure}[t]
\centerline{\includegraphics[width=1\linewidth, height=7cm]{Images/all_exp_methods_distrib.png}}
\caption{Distribution of mean reaction time deviation from random baseline by participant, by experiment. The results of the one-tailed $t$-test are represented with stars above the violin plots. With $p$ as the $p$-value of the $t$-test, *$p<5$\%, **$p<1$\%, and ***$p<0.5$\%}
\label{fig:distrib time reaction minus random}
\end{figure}

The mean difference in reaction time between CLS-A and the random baseline is statistically significant in the first experiment with a risk level of 0.1\%. The second experiment shows no significant difference in response time between the different methods and the random baseline. Finally, the third experiment highlights a mean distribution difference significantly lower than 0 between CLS-A, SHAP and the random baseline at risk levels of 5\% and 1\% respectively. Therefore, participants went faster on average in the text annotation task in Experiments 1 and 3 when they were exposed to CLS-A compared to the baseline. This difference from baseline is exclusive to CLS-A for Experiment 1, and shared with LIME for Experiment 3.
    
% \subsection{Linear modeling}
% \begin{table}[t]
%       \centering
%       \begin{tabular}{lllllll}
%         \toprule
%         \cmidrule(){1-5}
%         Metrics  & Experiment & \hfil CLS-A & \hfil LIME & \hfil SHAP \\
%         \midrule
%         \multirow{Reaction Time (s)}
%          %OLD MEAN
%          % & Exp 1 & \textbf{\hfil\num{-6.10e-3}} & \hfil\num{-3.43e-3} & \hfil\num{-4.79e-3}\\
%          % & Exp 2 & \hfil\num{-2.70e-3} & \textbf{\hfil\num{-1.00e-2}} & \hfil\num{-9.39e-3} \\
%          % & Exp 3 & \textbf{\hfil\num{-1.07e-2}} & \hfil\num{-1.04e-2} & \hfil\num{-3.209e-3}\\
%         \bottomrule
%       \end{tabular}
%       \caption{Linear modeling median coefficients by method}
%       \label{tab:table_coef_methods}
%     \end{table}
 
This section presents the analysis of the difference in reaction time between the interpretability methods and the random baseline by linear modeling . We also assess the impact of other important features affecting the reaction time, such as review length and probability score of the target label.

For each experiment, we run a linear regression per participant as presented in Section~\ref{sec:method} to model the reaction time to the labeling task. The explanatory variables differ only slightly from one experiment to another. The explanatory variables are information about the text and the interpretability method used to color it and are presented in Table~\ref{tab:variables} in Appendix~\ref{sec:appendix}. 

\paragraph{Method effect.} The performance metrics of the linear regressions on reaction times are presented in Table~\ref{tab:table_methods_R2} Appendix~\ref{sec:appendix} and the distributions of the average reaction time deviation from the random baseline per experiment are shown in Table~\ref{fig:distrib time reaction minus random}. The average and median coefficients of all methods, over all experiments are negative. Since all coefficients are calculated with respect to the random baseline, this suggests a lower average response time when participants are exposed to interpretability methods. CLS-A has the greatest impact compared to the random baseline for the first and third experiments, while SHAP induces faster responses for the second experiment. In order to evaluate the statistical significance of the results obtained, we analyze the distribution of the parameters of each linear model for each experiment. We then assess the mean sign of these distributions by performing a one-tailed $t$-test in the same way as in the previous section. The Figure~\ref{fig:all_reg_methods} shows the distributions of coefficients of linear regressions on reaction times associated with the interpretability method used to color the text.



% \begin{figure}[H]
% \begin{subfigure}{0.3333\textwidth}
% \includegraphics[width=0.9\linewidth, height=6cm]{Images/xp1/reg_coef_exp1_methods.png} 
% \caption{experiment 1}
% \label{fig:reg_coef_exp1_methods}
% \end{subfigure}
% \begin{subfigure}{0.3333\textwidth}
% \includegraphics[width=0.9\linewidth, height=6cm]{Images/xp2/reg_coef_exp2_methods.png}
% \caption{experiment 2}
% \label{fig:reg_coef_exp2_methods}
% \end{subfigure}
% \begin{subfigure}{0.3333\textwidth}
% \includegraphics[width=0.9\linewidth, height=6cm]{Images/xp3/reg_coef_exp3_methods.png}
% \caption{experiment 3}
% \label{fig:reg_coef_exp3_methods}
% \end{subfigure}
% \caption{Distribution of linear modeling coefficients of each interpretability method with respect to the baseline. The results of the one-tailed $t$-test are represented with stars above the box plots. With $p$ as the $p$-value of the one-tailed $t$-test, *$p<5$\%, **$p<1$\%, and ***$p<0.1$\%}
% \label{fig:distrib_coeffients_linear_modeling_methods}
% \end{figure}

The average linear regression coefficients on reaction times of CLS-A, SHAP and LIME are significantly negative for Experiment 1 and 3. The level of associated risk is however lower for CLS-A, with 0.1\% against 1\% and 5\% respectively for LIME and SHAP. The results for the CLS-A method are broadly consistent with the previous exploratory analysis. Participants took therefore less time on average to complete their annotation task on Experiment 1 and 3 when important words in the text were colored via the CLS-A method compared to the random baseline. However, the results differ for SHAP and LIME, and Experiment 2 does not show a statistically significant difference between these 3 methods compared to the random baseline.
\begin{figure}[t]
\centerline{\includegraphics[width=1\linewidth, height=8cm]{Images/all_exp_methods_reg.png}}
\caption{Distribution of linear modeling on reaction times coefficients of each interpretability method variable with respect to the baseline. The results of the one-tailed $t$-test are represented with stars above the box plots. With $p$ as the $p$-value of the one-tailed $t$-test, *$p<5$\%, **$p<1$\%, and ***$p<0.5$\%}
\label{fig:all_reg_methods}
\end{figure}
\paragraph{Probability score and review length effects.} We similarly assess the distributions of two more features in the linear regression model on reaction times, namely the probability of belonging to the target class, and the length of the word sequence (review). The Figure~\ref{fig:all_exp_methods_reg_proba_rev} represents the distribution of these coefficients, by experiment. The significance of the means of the distributions is assessed with one-tailed paired $t$-test in the same way than CLS-A, LIME and SHAP.

% \begin{figure}[H]
% \begin{subfigure}{0.3333\textwidth}
% \includegraphics[width=0.9\linewidth, height=6cm]{Images/xp1/reg_coef_exp1_proba_length.png} 
% \caption{experiment 1}
% \label{fig:reg_coef_exp1_proba_length}
% \end{subfigure}
% \begin{subfigure}{0.3333\textwidth}
% \includegraphics[width=0.9\linewidth, height=6cm]{Images/xp2/reg_coef_exp2_proba_rev.png}
% \caption{experiment 2}
% \label{fig:reg_coef_exp2_proba_rev}
% \end{subfigure}
% \begin{subfigure}{0.3333\textwidth}
% \includegraphics[width=0.9\linewidth, height=6cm]{Images/xp3/reg_coef_exp3_proba_rev.png}
% \caption{experiment 3}
% \label{fig:reg_coef_exp3_proba_rev}
% \end{subfigure}
% \caption{Distribution of probability score and review length coefficients in linear modeling. The results of the one-tailed $t$-test are represented with stars above the box plots. Noting $p$ as the $p$-value of the statistical test, the notations are as follows. *$p<5$\%, **$p<1$\%, and ***$p<0.1$\%}
% \label{fig:distrib_coeffients_linear_modeling_probaandrevlength}
% \end{figure}

The sign of each of the two coefficients is consistent across all three experiments. The effect of the probability score variable on response time is negative on average. This impact is statistically significant for experiment 1 and 2. The effect of the sequence length variable is positive and statistically significant on all experiments. Thus, all things being equal, the higher the probability score of belonging to the target class, the lower the reaction time. This highlights the relationship between the quality of an explanation and the certainty of a prediction from a time reaction perspective. In the same way, all things being equal, the annotation time increases with the length of the textual sequence processed.

Therefore, the linear modeling allowed to highlight that CLS-A fosters quicker responses on average compared to the random baseline. This result is found in experiment 1 and 3. If LIME and SHAP generate faster responses on average in these two experiments, CLS-A does so as well at a lower level of statistical risk. Furthermore, the probability score of the target class impacts negatively the reaction time.

\begin{figure}[t]
\centerline{\includegraphics[width=1\linewidth, height=8cm]{Images/all_exp_methods_reg_proba_rev.png}}
\caption{Distribution of probability score and review length coefficients in linear modeling. The results of the one-tailed $t$-test are represented with stars above the box plots. Noting $p$ as the $p$-value of the statistical test, the notations are as follows. *$p<5$\%, **$p<1$\%, and ***$p<0.5$\%}
\label{fig:all_exp_methods_reg_proba_rev}
\end{figure}

\subsection{Non-linear modeling}
This section compares CLS-A and the random baseline through the prism of the participant's accuracy, which is modelled using non-linear Explainable Boosting Machines (EBM) introduced in Section~\ref{sec:method}. 
The explanatory variables used to explain participant's response to the experiment are presented in Appendix~\ref{sec:appendix}. We fit one model per method and per experiment in order to compare the response curves of the methods within a given experiment. Each model is trained with the same explanatory variables. The same analysis integrating LIME and SHAP is in Appendix~\ref{sec:appendix}.   

We focus on this section on the impact of the probability score and the reaction time on the annotation task. Figure~\ref{fig:response_curve_xp1}, \ref{fig:response_curve_xp2} and \ref{fig:response_curve_xp3} represent the EBM response curves of the  probability score and the reaction time, by method, by experiment. These curves represent the contributions to the probability scores that the participant performs the annotation task well. The response curves of the variables sequence length, relative position of important words and trial number are shown in Appendix~\ref{sec:appendix}.The interval around the mean curve integrates the standard deviation measured on 50 sampling iterations for a given model. For each of the analyzed variables, we focus on the comparison between CLS-A and the random baseline. The analyses of LIME and SHAP are in  Appendix~\ref{sec:appendix}. 
\begin{figure}[t!]
\centerline{\includegraphics[width=1\linewidth, height=5cm]{Images/EBM_xp1.png}}
\caption{Explainable boosting machine response curves of probability score and time reaction in the first experiment. The contribution of the probability score variable becomes positive at a lower probability threshold for CLS-A compared to the random generator. The contributions of the reaction time variables are positive for the fast reactions for the CLS-A method unlike the random generator.}
\label{fig:response_curve_xp1}
\centerline{\includegraphics[width=1\linewidth, height=5cm]{Images/EBM_xp2.png}}
\caption{Explainable boosting machine response curves of probability score and time reaction in the second experiment. The contribution of the probability score increases at a faster rate than the random generator. CLS-A favors more fast reactions.}
\label{fig:response_curve_xp2}
\centerline{\includegraphics[width=1\linewidth, height=5cm]{Images/EBM_xp3.png}}
\caption{Explainable boosting machine response curves of probability score and time reaction in the third experiment. The contribution of the probability score variable increases at a faster rate than the random generator and decreases for very high prediction probability score unlike the random generator. The contributions are slightly higher for fast response times for CLS-A.}
\label{fig:response_curve_xp3}
\end{figure}

% \begin{figure}[t]
% \centerline{\includegraphics[width=1\linewidth, height=6cm]{Images/EBM_xp1.png}}
% \caption{EBM response curves of review length, probability score and time reaction, first experiment}
% \label{fig:response_curve_xp1}
% \end{figure}


% \begin{figure}[H]
% \begin{subfigure}{0.3333\textwidth}
% \includegraphics[width=0.9\linewidth, height=5cm]{Images/xp1/ebm_length_xp1.png} 
% \caption{Review length}
% \label{fig:ebm_xp1_length}
% \end{subfigure}
% \begin{subfigure}{0.3333\textwidth}
% \includegraphics[width=0.9\linewidth, height=5cm]{Images/xp1/ebm_proba_xp1.png}
% \caption{Probability score}
% \label{fig:ebm_xp1_proba}
% \end{subfigure}
% \begin{subfigure}{0.3333\textwidth}
% \includegraphics[width=0.9\linewidth, height=5cm]{Images/xp1/ebm_time_reaction_xp1.png}
% \caption{Time reaction}
% \label{fig:ebm_xp1_time_reaction}
% \end{subfigure}
% \caption{EBM response curves of review length, probability score and time reaction, first experiment}
% \label{fig:response_curve_xp1}
% \end{figure}

\paragraph{Sentiment analysis.} The first experiment highlights a higher target class probability score contribution for CLS-A compared to the random baseline in Figure~\ref{fig:response_curve_xp1}. The response curves tend to merge for the polarized probability scores and CLS-A falls a bit below the baseline for the probability score distribution tail. Accuracy contributions related to reaction time under CLS-A influence are higher for fast reactions, and lower for medium time reactions. Accuracy contribution tend to the same for very long response times. We can conclude that the interest of CLS-A compared to the random generator lies in the relatively low probabilities in the first experiment. Note however that the probability score distribution is very high in the first experiment, and covers very few non-polarized predictions. Moreover, the contribution of CLS-A is significant for fast predictions, and tends to vanish gradually.

% \begin{figure}[t]
% \centerline{\includegraphics[width=1\linewidth, height=6cm]{Images/EBM_xp2.png}}
% \caption{EBM response curves of review length, probability score and time reaction, second experiment}
% \label{fig:response_curve_xp2}
% \end{figure}

% \begin{figure}[H]
% \begin{subfigure}{0.3333\textwidth}
% \includegraphics[width=0.9\linewidth, height=5cm]{Images/xp2/ebm_length_xp2.png} 
% \caption{Review length}
% \label{fig:ebm_xp2_length}
% \end{subfigure}
% \begin{subfigure}{0.3333\textwidth}
% \includegraphics[width=0.9\linewidth, height=5cm]{Images/xp2/ebm_proba_xp2.png}
% \caption{Probability score}
% \label{fig:ebm_xp2_proba}
% \end{subfigure}
% \begin{subfigure}{0.3333\textwidth}
% \includegraphics[width=0.9\linewidth, height=5cm]{Images/xp2/ebm_time_reaction_xp2.png}
% \caption{Time reaction}
% \label{fig:ebm_xp2_time_reaction}
% \end{subfigure}
% \caption{EBM response curves of review length, probability score and time reaction, second experiment}
% \label{fig:response_curve_xp2}
% \end{figure}

\paragraph{Movie genre classification, action vs drama.} The second experiment has a more dispersed distribution of target class probability scores than the first experiment. The CLS-A response curve associated with the target class probability score variable is higher for polarized predictions as shown in in Figure~\ref{fig:response_curve_xp2}. The contribution of CLS-A compared to the baseline thus seems to be related to the certainty of the classifier prediction. The area in which the CLS-A response curve is higher corresponds to the majority of the probability score distribution of the target variable. Finally, the accuracy contributions of the reaction time variables are higher for CLS-A for short and very long responses. Finally and similarly to the first experiment, CLS-A has a strong impact to form rapid responses when labeling a high target class probability score text.



% \begin{figure}[t]
% \centerline{\includegraphics[width=1\linewidth, height=6cm]{Images/EBM_xp3.png}}
% \caption{EBM response curves of review length, probability score and time reaction, third experiment}
% \label{fig:response_curve_xp3}
% \end{figure}


% \begin{figure}[H]
% \begin{subfigure}{0.3333\textwidth}
% \includegraphics[width=0.9\linewidth, height=5cm]{Images/xp3/ebm_length_xp3.png} 
% \caption{Review length}
% \label{fig:ebm_xp3_length}
% \end{subfigure}
% \begin{subfigure}{0.3333\textwidth}
% \includegraphics[width=0.9\linewidth, height=5cm]{Images/xp3/ebm_proba_xp3.png}
% \caption{Probability score}
% \label{fig:ebm_xp3_proba}
% \end{subfigure}
% \begin{subfigure}{0.3333\textwidth}
% \includegraphics[width=0.9\linewidth, height=5cm]{Images/xp3/ebm_time_reaction_xp3.png}
% \caption{Time reaction}
% \label{fig:ebm_xp3_time_reaction}
% \end{subfigure}
% \caption{EBM response curves of review length, probability score and time reaction, third experiment}
% \label{fig:response_curve_xp3}
% \end{figure}

\paragraph{Movie genre classification, horror vs comedy.} The distribution of the target class probability score variable is less dispersed in the last experiment than in the second one. Figure~\ref{fig:response_curve_xp3} depicts that the response curve is higher for CLS-A for high probability scores and falls below it at the distribution tail in a slightly more marked way than the first experiment. Finally, the effect of CLS-A on the response time variable with respect to the baseline in experiment 3 is relatively similar to Experiment 2. Short and very long answers are more accurate with CLS-A compared to the random baseline. 

Altogether, the analysis of the response curves highlights the non-linear relationships between the explanatory variables and the target variable. The interest of CLS-A seems strongest for high probability scores of classification, but tends to decrease for texts whose probability score is at the tail of the distribution. In the same way as in the linear modeling analysis, the interest of CLS-A seems to be less important or even non-existent for texts whose probability scores are relatively low. This highlights a strong link between the quality of an explanation and the certainty of a prediction, insofar as participants respond less well overall when exposed to such texts colored by the CLS-A method. The fact that the interest of CLS-A fades for high probabilities for experiment 1 and 3 may be related to the intrinsic ease of classifying text with such a high probability score. We also found that the impact of the CLS-A method is concentrated around fast and slow responses, and tends to merge with the baseline for medium fast responses, confirming the facilitating effect of the method on semantic processing. 

\section{Conclusion and future work}
We applied an experimental protocol to compare a new method called CLS-A based on Transformer self-attention to SHAP, LIME and a random baseline. We found that CLS-A helps in the same proportions as SHAP and LIME to annotate text on three different tasks and is significantly better than a random baseline. This work adds to the literature aiming to evaluate the interpretability of attention coefficients in recent deep learning models. Moreover, analyses have shown that the interest of an CLS-A relies highly on the probability score of the prediction. The higher the probability score, the more relevant the explanation. As far as we know, this is the first time that the relationship between the quality of an explanation and the certainty of its associated prediction has come to light. 

There is currently no consensus on a quantitative method to validate interpretability techniques. We believe human-based experiments measuring the similarity of processing between the model and human reasoning is the best option available. The present study confirms its relevance, especially when combined with statistical analysis.

This work has been done by analyzing binary classifiers. Analyzing multi-class classifiers could likewise lead to more refined conclusions. Finally, it would be insightful to include in such experiments other feature importance methods usually used in NLP based, for example, on gradient computation in order to to benchmark them in relation to common approaches. To our knowledge, there is no experimental protocol to compare other type of explainability method such as counterfactual explanations. We plan in a future work to build such a protocol, which will also allow to assess the quality of a given counterfactual explanation.

\section{Acknowledgments}
We thank Gabriel Olympie and Jean-Baptiste Gette for their advice and support. We also thank all the participants in the experiment who made this work possible.

\section{Ethic statement}
Each participant signed a consent form containing an overview of the project and the intended use of the data they would generate. The data was anonymized and processed only by our team. The data produced is stored in a file in respect with the GDPR regulations in force. Participation in the study was fully voluntary. It was possible to stop performing the labeling tasks at any time.  

%\bibliography{references}
\bibliography{template}
\bibliographystyle{unsrt}
\newpage
\appendix
\section{Appendix}
\label{sec:appendix}

\subsection{Regression results} \label{A}
\begin{table}[H]
\centering
\begin{tabular}{@{}ccc@{}}
\toprule
Metrics                            & Experiment & \textit{$R^2$} \\ \midrule
\multirow{3}{*}{Reaction Time (s)} & Exp 1      & 4.23e-1        \\
                                   & Exp 2      & 6.01e-1        \\
                                   & Exp 3      & 6.59e-1        \\ \bottomrule
\end{tabular}
\caption{R-square of linear regression explaining participant reaction time}
      \label{tab:table_methods_R2}
\end{table}

% \begin{table}[H]
%       \centering
%       \begin{tabular}{lllllll}
%         \toprule
%         \cmidrule(){1-3}
%         Metrics  & Experiment & $R^2$ \\
%         \midrule
%         \multirow{Reaction Time (s)}
%         & Exp 1 & \hfil\num{4.23e-1}\\
%         & Exp 2  & \hfil\num{6.01e-1}\\
%         & Exp 3 & \hfil\num{6.59e-1}\\
%         \bottomrule
%       \end{tabular}
%       \caption{R-square of linear regression explaining participant reaction time}
%       \label{tab:table_methods_R2}
% \end{table}


\begin{table}[H]
      \centering
      \begin{tabular}{lllllll}
        \toprule
        \cmidrule(){1-4}
        Metrics & Experiment 1 & Experiment 2 & Experiment 3\\
        \midrule
         Intercept & \hfil\num{3.68e-1} & \hfil\num{1.50e-1} & \hfil\num{9.83e-2}\\
         Method CLS-A  & \hfil\num{-6.10e-3} & \hfil\num{-2.70e-3} & \hfil\num{-1.07e-2}\\
         Method LIME & \hfil\num{-3.43e-3} & \hfil\num{-1.00e-2} & \hfil\num{-1.04e-2}\\
         Method SHAP & \hfil\num{-4.79e-3} & \hfil\num{-9.39e-3} & \hfil\num{-3.21e-3}\\
         Probability  & \hfil\num{-2.24e-1} & \hfil\num{4.89e-3} & \hfil\num{3.19e-2}\\
         Accurate answer & \hfil\num{-5.25e-2} & \hfil\num{-1.88e-2} & \hfil\num{-1.66e-2}\\
         Review length & \hfil\num{4.38e-2} & \hfil\num{1.15e-1} & \hfil\num{8.25e-2}\\
         First word position & \hfil\num{5.21e-3} & \hfil\num{1.19e-3} & \hfil\num{-8.50e-3}\\
         Second word position & \hfil x & \hfil\num{-1.02e-2} & \hfil\num{7.75e-3}\\
         Third word position & \hfil x & \hfil\num{9.71e-3} & \hfil\num{3.33e-3} \\
         Trial number & \hfil\num{-5.32e-4} & \hfil\num{-6.93e-4} & \hfil\num{-9.11e-4}\\
        \bottomrule
      \end{tabular}
      \caption{Average coefficients of linear regression modeling  participant reaction time. All the coefficients related to the interpretability methods are negative. It indicates a positive effect on reaction time. The value of the CLS-A parameter is lower than the others methods for experiment 1 and 3.}
      \label{tab:table_coef_regression}
\end{table}



% \begin{table}[H]
%       \centering
%       \begin{tabular}{lllllll}
%         \toprule
%         \cmidrule(){1-4}
%         Variables & Experiment 1 & Experiment 2 & Experiment 3\\
%         \midrule
%          Method CLS-A  & \hfil\num{0.012584744748163037} & \hfil\num{0.32056912633302037} & \hfil\num{0.02951797740112013}\\
%          Method LIME & \hfil\num{0.3905096547276356} & \hfil\num{0.060460476141128566} & \hfil\num{0.005989711602859692}\\
%          Method SHAP & \hfil\num{0.07026288339062253} & \hfil\num{0.089475537340785} & \hfil\num{0.21321199157751242}\\
%          Probability  & \hfil\num{5.535298778303383e-17} & \hfil\num{0.8159848880548115} & \hfil\num{0.4538214695311823}\\
%          Review length & \hfil\num{2.1664735735619272e-09} & \hfil\num{.0589989452579111e-05} & \hfil\num{0.00012345483664856515}\\
%         \bottomrule
%       \end{tabular}
%       \caption{p-value reg to complete}
%       \label{tab:table_pval_regression}
% \end{table}

% \begin{table}[H]
%       \centering
%       \begin{tabular}{lllllll}
%         \toprule
%         \cmidrule(){1-4}
%         Variables & Experiment 1 & Experiment 2 & Experiment 3\\
%         \midrule
%          Method CLS-A  & \hfil\num{0.004724278302540167} & \hfil\num{0.542361730584281} & \hfil\num{0.02273998976376002}\\
%          Method LIME & \hfil\num{0.12208270588301569} & \hfil\num{0.0771591227442335} & \hfil\num{0.006012616313143069}\\
%          Method SHAP & \hfil\num{0.08057729125111142} & \hfil\num{0.1278848614159821} & \hfil\num{0.21732488853204662}\\
%         \bottomrule
%       \end{tabular}
%       \caption{p-value distrib to complete}
%       \label{tab:table_pval_distrib}
% \end{table}
\subsection{Modeling summary} \label{B}
\begin{table}[H]
\centering
\begin{tabular}{|c|l|c|c|}
\hline
\textbf{\begin{tabular}[c]{@{}c@{}}Task \\ type\end{tabular}} & \textbf{Target variable} & \textbf{Model}               & \textbf{\begin{tabular}[c]{@{}c@{}}Explanatory \\ variables\end{tabular}}                                                                                                                                                    \\ \hline
Regression                                                    & Reaction time            & Linear model                 & \begin{tabular}[c]{@{}c@{}}expected answer, classifier probability score, \\ review length, trial number, interpretability method, \\ relative positions of the first, \\ second and third most impacting words\end{tabular} \\ \hline
Classification                                                & Accurate                 & Explainable boosting machine & \begin{tabular}[c]{@{}c@{}}reaction time, classifier probability score, \\ review length, trial number, interpretability method, \\ relative position of the first most impacting word\end{tabular}                          \\ \hline
\end{tabular}
\caption{Linear regression and explainable boosting machine explanatory variables. The variables of the relative positions of the second and third most important words were used only for reaction time modeling in the first .}
      \label{tab:variables}
\end{table}

% \begin{table}[H]
% \centering
% \begin{tabular}{|c|l|c|c|}
% \hline
% \textbf{\begin{tabular}[c]{@{}c@{}}Task \\ type\end{tabular}} & \textbf{Target variable} & \textbf{Model}               & \textbf{\begin{tabular}[c]{@{}c@{}}Explanatory \\ variables\end{tabular}}                                                                                                                                                 \\ \hline
% Regression                                                    & Reaction time            & Linear model                 & \begin{tabular}[c]{@{}c@{}}\textbf{expected answer}, classifier probability score, \\ review length, trial number, interpretability method, \\ relative positions of the first, second and third most impacting words\end{tabular} \\ \hline
% Classification                                                & Accurate                 & Explainable boosting machine & \begin{tabular}[c]{@{}c@{}}\textbf{reaction time}, classifier probability score, \\ review length, trial number, interpretability method, \\ relative position of the first most impacting word\end{tabular}                       \\ \hline
% \end{tabular}
% \caption{Linear regression and explainable boosting machine explanatory variables. The variables are the same, except for the bolded variables. The variables of the relative positions of the second and third most important words were used only for reaction time modeling in the first .}
%       \label{tab:ebm_scores}
% \end{table}


% \begin{table}[H]
% \centering
% \begin{tabular}{|c|c|c|}
% \hline
% \textbf{\begin{tabular}[c]{@{}c@{}}Task \\ type\end{tabular}} & \textbf{Model}               & \textbf{\begin{tabular}[c]{@{}c@{}}Explanatory \\ variables\end{tabular}}                                                                                                                                                 \\ \hline
% Regression                                                    & Linear model                 & \begin{tabular}[c]{@{}c@{}}\textbf{expected} answer, classifier probability score, \\ review length, trial number, interpretability method, \\ relative positions of the first, \textbf{second} and \textbf{third} most impacting words\end{tabular} \\ \hline
% Classification                                                & Explainable boosting machine & \begin{tabular}[c]{@{}c@{}}\textbf{reaction time}, classifier probability score, \\ review length, trial number, interpretability method, \\ relative position of the first most impacting word\end{tabular}                       \\ \hline
% \end{tabular}
% \caption{Linear regression and explainable boosting machine explanatory variables. The variables are the same, except for the bolded variables. The variables of the relative positions of the second and third most important words were used only for reaction time modeling in the first .}
%       \label{tab:ebm_scores}
% \end{table}

\subsection{Explainable boosting machine performance metrics} \label{D}
\begin{table}[H]
\centering
\begin{tabular}{@{}cccccc@{}}
\toprule
\textbf{Experiment}    & \textbf{Method} & \textbf{Accuracy} & \textbf{Precision Score} & \textbf{F1 Score} & \textbf{Recall Score} \\ \midrule
\multirow{4}{*}{Exp 1} & CLS-A           & 0.952             & 0.945                    & 0.953             & 0.961                 \\
                       & LIME            & 0.992             & 0.991                    & 0.992             & 0.993                 \\
                       & SHAP            & 0.961             & 0.955                    & 0.961             & 0.976                 \\
                       & Random          & 0.982             & 0.988                    & 0.982             & 0.976                 \\ \midrule
\multirow{4}{*}{Exp 2} & CLS-A           & 0.889             & 0.894                    & 0.888             & 0.884                 \\
                       & LIME            & 0.897             & 0.888                    & 0.898             & 0.909                 \\
                       & SHAP            & 0.913             & 0.917                    & 0.913             & 0.909                 \\
                       & Random          & 0.878             & 0.867                    & 0.880             & 0.894                 \\ \midrule
\multirow{4}{*}{Exp 3} & CLS-A           & 0.957             & 0.950                    & 0.957             & 0.965                 \\
                       & LIME            & 0.946             & 0.943                    & 0.946             & 0.949                 \\
                       & SHAP            & 0.920             & 0.916                    & 0.920             & 0.925                 \\
                       & Random          & 0.920             & 0.921                    & 0.920             & 0.919                 \\ \cmidrule(l){2-6} 
\end{tabular}
\caption{Average explainable boosting machine performance per experiment, per method.}
      \label{tab:ebm_scores}
\end{table}

% \begin{table}[H]
%     \centering
%       \begin{tabular}{llllll}
%         \toprule
%         \cmidrule(){1-6}
%         Experiments & Method & Accuracy & Precision Score & F1 Score &  Recall Score\\
%         \midrule
%         \multirow{Experiment 1}
%          & CLS-A  & \hfil\num{0.952} & \hfil\num{0.945} & \hfil\num{0.953} & \hfil\num{0.961}\\
%          & LIME & \hfil\num{0.992} & \hfil\num{0.991} & \hfil\num{0.992} & \hfil\num{0.993}\\
%          & SHAP & \hfil\num{0.961} & \hfil\num{0.955} & \hfil\num{0.961} & \hfil\num{0.976}\\
%          & Random & \hfil\num{0.982} & \hfil\num{0.988} & \hfil\num{0.982} & \hfil\num{0.976}\\
%          \cmidrule(){1-6}
%         \multirow{Experiment 2}
%          & CLS-A  & \hfil\num{0.889} & \hfil\num{0.894} & \hfil\num{0.888} & \hfil\num{0.884}\\
%          & LIME & \hfil\num{0.897} & \hfil\num{0.888} & \hfil\num{0.898} & \hfil\num{0.909}\\
%          & SHAP & \hfil\num{0.913} & \hfil\num{0.917} & \hfil\num{0.913} & \hfil\num{0.909}\\
%          & Random & \hfil\num{0.878} & \hfil\num{0.867} & \hfil\num{0.880} & \hfil\num{0.894}\\
%          \cmidrule(){1-6}
%          \multirow{Experiment 3}
%          & CLS-A  & \hfil\num{0.957} & \hfil\num{0.950} & \hfil\num{0.957} & \hfil\num{0.965}\\
%          & LIME & \hfil\num{0.946} & \hfil\num{0.943} & \hfil\num{0.946} & \hfil\num{0.949}\\
%          & SHAP & \hfil\num{0.920} & \hfil\num{0.916} & \hfil\num{0.920} & \hfil\num{0.925}\\
%          & Random & \hfil\num{0.920} & \hfil\num{0.921} & \hfil\num{0.920} & \hfil\num{0.919}\\
%         \bottomrule
%       \end{tabular}
%       \caption{Average explainable boosting machine performance per experiment, per method.}
%       \label{tab:ebm_scores}
% \end{table}
\newpage
\subsection{EBM reponse curves} \label{B}

\begin{figure}[h]
\centerline{\includegraphics[width=1\linewidth, height=5cm]{Images/EBM_proba_allmethods.png}}
\caption{Explainable boosting machine response curves of probability score for all methods. The contribution of the probability score variable becomes higher for CLS-A, SHAP and LIME above a certain probability threshold compared to the random baseline for Experiment 2 and 3.}
\label{fig:EBM_proba_allmethods}
\end{figure}

\begin{figure}[h]
\centerline{\includegraphics[width=1\linewidth, height=5cm]{Images/EBM_time_allmethods.png}}
\caption{Explainable boosting machine response curves of time reaction for all methods. }
\label{fig:EBM_time_allmethods}
\end{figure}

\begin{figure}[h]
\centerline{\includegraphics[width=1\linewidth, height=5cm]{Images/EBM_first_word.png}}
\caption{Explainable boosting machine response curves of first word relative position.}
\label{fig:EBM_first_word}
\end{figure}
\begin{figure}[h]
\centerline{\includegraphics[width=1\linewidth, height=5cm]{Images/EBM_second_word.png}}
\caption{Explainable boosting machine response curves of second word relative position}
\label{fig:EBM_second_word}
\end{figure}
\begin{figure}[h]
\centerline{\includegraphics[width=1\linewidth, height=5cm]{Images/EBM_third_word.png}}
\caption{Explainable boosting machine response curves of third word relative position}
\label{fig:EBM_third_word}
\end{figure}
\begin{figure}[h]
\centerline{\includegraphics[width=1\linewidth, height=5cm]{Images/EBM_rev_length_xp23.png}}
\caption{Explainable boosting machine response curves of review length. The contribution of the review length variable is higher for CLS-A for very short and very long reviews.}
\label{fig:EBM_rev_length_xp23}
\end{figure}




% \bibliography{references}  %%% Uncomment this line and comment out the ``thebibliography'' section below to use the external .bib file (using bibtex) .

\end{document}



% Preamble: load the packages that you need and define your own commands here
\newcommand{\todo}[1]{\textcolor{red}{[todo: #1]}}
\newcommand{\vic}[1]{\textcolor{magenta}{[vic: #1]}}
\newcommand{\rob}[1]{\textcolor{blue}{#1}}
\newcommand{\maika}[1]{\textcolor{green}{#1}}

\begin{document}

%%%%%%%%%%%%%%%%%%%%%%%%%%%%%%%%%%%%%%%%%%%%%%%%%%%%%%%%%%%%%%%%%
%           I N F O R M A T I O N   T O   C H A N G E
%%%%%%%%%%%%%%%%%%%%%%%%%%%%%%%%%%%%%%%%%%%%%%%%%%%%%%%%%%%%%%%%%

% Author name displayed in the running head
\newcommand{\runningauthor}{R. Courant \textit{et al.}} 

% Title displayed in the running head
\newcommand{\runningheadtitle}{Transformers}

% Chapter number
\newcommand{\chapternumber}{6}

% E-mail address of the corresponding author
\newcommand{\emailaddress}{vicky.kalogeiton@lix.polytechnique.fr}

% Title of the chapter
\title{Transformers and visual Transformers} 

% Authors' names and affiliation numbers
\author[1,2]{Robin Courant}
\author[1]{Maika Edberg}
\author[1]{Nicolas Dufour}
\author[*,1]{Vicky Kalogeiton}  % Use the symbol '*' for the corresponding author

% Affiliations
\affil[1]{LIX, CNRS, Ecole Polytechnique, IP Paris}
\affil[2]{Univ.\ Rennes, CNRS, IRISA, INRIA}

%%%%%%%%%%%%%%%%%%%%%%%%%%%%%%%%%%%%%%%%%%%%%%%%%%%%%%%%%%%%%%%%%
%\affil[$\dagger$]{Equal Contribution}
\affil[*]{Corresponding author: e-mail address: \href{mailto:\emailaddress}{\emailaddress}}

\maketitle

% Restore the geometry and change the page style for the other pages
\afterpage{\aftergroup\restoregeometry}
\pagestyle{otherpages}

\begin{abstract}
Transformers were initially introduced for natural language processing (NLP) tasks, but fast they were adopted by most deep learning fields, including computer vision. They measure the relationships between pairs of input tokens (words in the case of text strings, parts of images for visual Transformers), termed attention. The cost is exponential with the number of tokens. For image classification, the most common Transformer Architecture uses only the Transformer Encoder in order to transform the various input tokens. However, there are also numerous other applications in which the decoder part of the traditional Transformer Architecture is also used. Here, we first introduce the Attention mechanism (Section~\ref{sec:attention}), and then the Basic Transformer Block including the Vision Transformer (Section~\ref{sec:transformer}). Next, we discuss some improvements of visual Transformers to account for small datasets or less computation (Section~\ref{sec:extensions}). Finally, we introduce Visual Transformers applied to tasks other than image classification, such as detection, segmentation, generation and training without labels (Section~\ref{sec:tasks}) and other domains, such as  video or multimodality using text or audio data (Section~\ref{sec:domains}). 

\end{abstract}

\begin{keywords}
attention, Transformers, visual Transformers, multimodal attention
\end{keywords}

\section{Attention} 
\label{sec:attention}
Attention is a technique in Computer Science that imitates the way in which the brain can focus on the relevant parts of the input. In this section, we introduce attention: its history (Section~\ref{sub:history_attention}), its definition (Section~\ref{sub:definition_attention}), its types and variations  (Sections~\ref{sub:attention_types} and \ref{sub:variation_attention}) and its properties (Section~\ref{sub:property_attention}). \\

To understand what attention is and why it is so useful, consider the following film review:
\begin{center}
	\textit{While others claim the story is boring, I found it fascinating.}
\end{center}
Is this film review positive or negative?   
The first part of the sentence is unrelated to the critic's opinion, while the second part suggests a positive sentiment with the word ‘fascinating’. 
To a human, the answer is obvious; however, this type of analysis is not necessarily obvious to a computer. 

Typically, sequential data require \emph{context} to be understood. 
In natural language, a word has a meaning because of its position in the sentence, with respect to the other words: its \emph{context}. 
In our example, while “boring” alone suggests that the review is negative, its contextual relationship with the other words allows the reader to reach the appropriate conclusion. 
In computer vision, in a task like object detection, the nature of a pixel alone cannot be identified: we need to account for its neighbourhood, its \emph{context}. 
So, how can we formalize the concept of \emph{context} in sequential data?

%%%%%%%%%%%%%%%%%%%%%%%%%%%%%%%%%%%%%%%
\subsection{The History of Attention}
\label{sub:history_attention}
%%%%%%%%%%%%%%%%%%%%%%%%%%%%%%%%%%%%%%%

This notion of \emph{context} is the motivation behind the introduction of the attention mechanism in 2015 \cite{Bahdanau2015NeuralMT}. 
Before this, language translation was mostly relying on encoder-decoder architectures:  recurrent neural networks (RNNs) ~\cite{cho_RNN_translation_2014} and in particular long short-term memory (LSTMs) networks were used to model the relationship among words~\cite{ Sutskever_LSTM_Translation_2014}. 
Specifically, each word of an input sentence is processed by the encoder sequentially. At each step, the past and present information are summarized and encoded into a fixed-length vector. 
In the end, the encoder has processed every word, and outputs a final fixed-length vector, which summarizes all input information. This final vector is then decoded, and finally translates the input information into the target language.

However, the main issue of such structure is that all the information is compressed into one fixed-length vector. 
Given that the sizes of sentences vary and as the sentences get longer, a fixed-length vector is a real bottleneck: it gets increasingly difficult not to lose any information in the encoding process due to the vanishing gradient problem~\cite{Bahdanau2015NeuralMT}. 

As a solution to this issue, Bahdanue et al.~\cite{Bahdanau2015NeuralMT} proposed the attention module in 2015. 
The attention module allows the model to consider the parts of the sentence that are relevant to predicting the next word. Moreover, this facilitates the understanding of relationships among words that are further apart.

%%%%%%%%%%%%%%%%%%%%%%%%%%%%%%%%%%%%%%%
\subsection{Definition of Attention}
\label{sub:definition_attention}
%%%%%%%%%%%%%%%%%%%%%%%%%%%%%%%%%%%%%%%
\begin{figure}
    \centering
    \includegraphics[scale=0.8]{figures/attention_v2.pdf}
    \caption{\textbf{Attention block.} Next to each element, we denote its dimensionality. Figure inspired from~\cite{vaswani2017attention}.  }
    \label{fig:attention}
\end{figure}

Given two lists of tokens, $\bm{X}\in\mathbb{R}^{N \times d_x}$ and $\bm{Y}\in\mathbb{R}^{N \times d_y}$, attention encodes information from $\bm{Y}$ into $\bm{X}$, where $N$ is the length of inputs $\bm{X}$ and $\bm{Y}$, and $d_x$ and $d_y$ are their respective dimensions. 
For this, we first define three linear mappings: 
query mapping $\bm{W}^Q\in\mathbb{R}^{d_x \times d_{q}}$,  key mapping $\bm{W}^K\in\mathbb{R}^{d_y \times d_{k}}$ and value mapping $\bm{W}^V\in\mathbb{R}^{d_y \times d_{v}}$, where $d_q$, $d_k$, and $d_v$ is the embedding dimension in which the query, key, and value are going to be computed, respectively. 

Then, we define the query $\bm{Q}$, key $\bm{K}$ and value $\bm{V}$~\cite{vaswani2017attention} as:
\begin{align*}
    \bm{Q} &= \bm{X} \bm{W}^Q \\
    \bm{K} &= \bm{Y} \bm{W}^K \\
    \bm{V} &= \bm{Y} \bm{W}^V 
\end{align*}

Next, the \emph{attention matrix} is defined as:
\begin{equation}
    \label{eq:attention_matrix}
    A(\bm{Q}, \bm{K}) = \text{Softmax}
    \left(
    \frac{\bm{Q}\bm{K}^\top}{\sqrt{d_k}}
    \right) \quad . 
\end{equation}

This is illustrated in the left part of Figure~\ref{fig:attention}. The nominator $\bm{Q K}^T \in \mathbb{R}^{N \times N}$ represents how each part of the input in $\bm{X}$ attends to each part of the input in $\bm{Y}$\footnote{Note that in the literature, there are two main attention functions: additive attention~\cite{Bahdanau2015NeuralMT} and dot-product attention (Equation~\ref{eq:attention_matrix}). In practice, the dot-product is more efficient since it is implemented using highly optimized matrix multiplication, compared to the feed forward network of the additive attention; hence, the dot-product is the dominant one.}. This dot-product is then put through the Softmax function to normalize its values and get positive values that add to 1. However, for large values of $d_k$, this may result in the Softmax to have incredibly small gradients, so it is scaled down by $\sqrt{d_k}$. 

The resulting $N \times N$ matrix encodes the relationship between $\bm{X}$ with respect to $\bm{Y}$: it measures how important a token in $\bm{X}$ is with respect to another one in $\bm{Y}$.

Finally, the \emph{attention output} is defined as:
\begin{equation}
    \label{eq:attention_output}
    \text{Attention}(\bm{Q},\bm{K}, \bm{V})= A(\bm{Q}, \bm{K}) \bm{V} \quad .
\end{equation}

Figure~\ref{fig:attention} displays this.  
The attention output encodes the information of each token by taking into account the contextual information. 
Therefore, through the learnable parameters -- queries, keys, and values--, the attention layers learn a token embedding that takes into account their relationship. 

\paragraph{Contextual relationships.}
How does Equation~\ref{eq:attention_output} encode contextual relationships? To answer this question, let us reconsider analysing the sentiment of film reviews. To encode contextual relationships into the word embedding, we first want a matrix representation of the relationship between all words. To do so, given a sentence of length $N$, we take each word vector and feed it to two different linear layers, calling one output `query' and the other output `key'. We pack the queries into the matrix $\bm{Q}$ and the keys into the matrix $\bm{K}$, by taking their product ($\bm{Q} \bm{K}^T$). The result is a $N \times N$ matrix that explains how important the $i$-th word (row-wise) is to understand the $j$-th word (column-wise). This matrix is then scaled and normalized by the division and Softmax. Next, we feed the word vectors into another linear layer, calling its output `value'.   
We multiply these two matrices together. The result of their product are attention vectors that encode the meaning of each word, by including their contextual meaning as well. Given that each of these queries, keys, and values are learnable parameters, as the attention layer is trained, the model learns how relationships among words are encoded in the data. 

%%%%%%%%%%%%%%%%%%%%%%%%%%%%%%%%%%%%%%%
\subsection{Types of Attention}
\label{sub:attention_types}
%%%%%%%%%%%%%%%%%%%%%%%%%%%%%%%%%%%%%%%

There exist two dominant types of attention mechanisms: \emph{self-attention} and  \emph{cross-attention} \cite{vaswani2017attention}. 
In \emph{self-attention}, the queries, keys and values come from the same input, i.e., $\bm{X} = \bm{Y}$; 
in \emph{cross-attention}, the queries come from a different input than the key and value vectors, i.e., $\bm{X} \neq \bm{Y}$. 
These are described below in Sections~\ref{subsub:self_attention} and \ref{subsub:cross_attention}, respectively.  

%%%%%%%%%%%%%%%%%%%%%%%%%%%%%%%%%%%%%%%
\subsubsection{Self-Attention}
\label{subsub:self_attention}
%%%%%%%%%%%%%%%%%%%%%%%%%%%%%%%%%%%%%%%

In self-attention, the tokens of $\bm{X}$ attend to themselves ($\bm{X}=\bm{Y}$). Therefore, it is modelled as follows: 
\begin{equation}
    \text{SA}(\bm{X}) = \text{Attention}(
                           \bm{X}\bm{W}^Q , 
                           \bm{X}\bm{W}^K, 
                           \bm{X}\bm{W}^V ) \quad .
\end{equation}

Self-attention formalizes the concept of context. It learns the patterns underlying how parts of the input correspond to each other. By gathering information from the same set, given a sequence of tokens, a token can attend to its neighbouring tokens to compute its output.

%%%%%%%%%%%%%%%%%%%%%%%%%%%%%%%%%%%%%%%
\subsubsection{Cross-Attention}
\label{subsub:cross_attention}
%%%%%%%%%%%%%%%%%%%%%%%%%%%%%%%%%%%%%%%

Most real-world data are multimodal -- for instance, videos contain frames, audios and subtitles, images come with captions, etc. Therefore, models that can deal with such types of multimodal information have become essential. 

Cross-attention is an attention mechanism designed to handle multimodal inputs.
Unlike self-attention, it extracts queries from one input source and key-value pairs from another one ($\bm{X} \neq \bm{Y}$). It answers the question: `which parts of input $\bm{X}$ and input $\bm{Y}$ correspond to each other?'
Cross-attention (CA) is defined as:
\begin{equation}
    \text{CA}(\bm{X}, \bm{Y}) = \text{Attention}(
        \bm{X}\bm{W}^Q , 
        \bm{Y}\bm{W}^K, 
        \bm{Y}\bm{W}^V) \quad .
\end{equation}

%%%%%%%%%%%%%%%%%%%%%%%%%%%%%%%%%%%%%%%
\subsection{Variation of Attention}
\label{sub:variation_attention}
%%%%%%%%%%%%%%%%%%%%%%%%%%%%%%%%%%%%%%%

Attention is typically employed in two ways: 
(1) Multi-head Self-Attention (MSA, Section~\ref{subsub:multihead}) and (2) Masked Multi-head Attention (MMA, Section~\ref{subsub:maskedmultihead}). 

\paragraph{Attention Head.}
We call Attention Head the mechanism presented in Section~\ref{sub:definition_attention}, i.e., query-key-value projection, followed by scaled dot product attention (Equations~\ref{eq:attention_matrix} and \ref{eq:attention_output}). 



When employing an attention-based model, relying only on a single attention head can inhibit learning. 
Therefore, the Multi-head Attention block is introduced \cite{vaswani2017attention}. 

\paragraph{Multi-head Self-Attention (MSA).} 
\label{subsub:multihead}
MSA is shown in Figure~\ref{fig:multi_head_attention} and is defined as: 

\begin{equation}
\label{eq:multihead_self_attention}
\begin{array}{c}
    \text{MSA}(\bm{X}) = \text{Concat} \big( \text{head}_{1}(\bm{X}), \dots, \text{head}_{h}(\bm{X}) \big) \bm{W}^O \quad, \\
     \quad 
    \text{head}_{i}(\bm{X}) = \text{SA}(\bm{X}) \text{ , }\forall i \in \{1,h\} \quad ,
\end{array}
\end{equation}

\noindent where Concat is the concatenation of $h$ attention heads and $\bm{W}^O\in\mathbb{R}^{hd_v \times d}$ is projection matrix. 
This means that the initial embedding dimension $d_x$ is decomposed into $h \times d_v$, and the computation per head is carried out independently. The independent attention heads are usually concatenated and multiplied by a linear layer to match the desired output dimension. The output dimension is often the same as the input embedding dimension $d$. This allows an easier stacking of multiple blocks. 

\paragraph{Multi-head Cross-Attention (MCA).}
Similar to MSA, MCA is defined as:

\begin{equation}
\label{eq:multihead_cross_attention}
\begin{array}{c}
\text{MCA}(\bm{X}, \bm{Y}) = \text{Concat}(\text{head}_1(\bm{X}, \bm{Y}), \dots, \text{head}_h(\bm{X}, \bm{Y}))\bm{W}^O \quad, \\ 
     \quad 
    \text{head}_{i}(\bm{X}, \bm{Y}) = \text{CA}(\bm{X}, \bm{Y}) \text{ , }\forall i \in \{1,h\} \quad .
\end{array}
\end{equation}

\begin{figure}
    \centering
    \includegraphics[scale=0.8]{figures/msa_v2.pdf}
    \caption{\textbf{Multi-head Self-Attention block (MSA).} 
    First, the input $\bm{X}$ is projected to queries, keys and values and then passed through $h$ attention blocks. The $h$ resulting attention outputs are then concatenated together and finally projected to a $d$-dimensional output vector.
    Next to each element, we denote its dimensionality. Figure inspired from \cite{vaswani2017attention}. 
    }
    \label{fig:multi_head_attention}
\end{figure}



\paragraph{Masked Multi-head Self-Attention (MMSA).}
\label{subsub:maskedmultihead}
The MMSA layer~\cite{vaswani2017attention} is another variation of attention. It has the same structure as the Multi-head Self-Attention block (Section~\ref{subsub:multihead}), but all the later vectors in the target output are masked. When dealing with sequential data, this can help make training parallel. 

%%%%%%%%%%%%%%%%%%%%%%%%%%%%%%%%%%%%%%%
\subsection{Properties of Attention}
\label{sub:property_attention}
%%%%%%%%%%%%%%%%%%%%%%%%%%%%%%%%%%%%%%%

While attention encodes contextual relationships, it is \emph{permutation equivalent}, as the mechanism does not account for the order of the input data. As shown in Equation~\ref{eq:attention_output}, the attention computations are all matrix multiplication and normalizations. Therefore, a permuted input results in a permuted output. In practice, however, this may not be an accurate representation of the information. For instance, consider the sentences `the monkey ate the banana' and `the banana ate the monkey'. They have distinct meanings because of the order of the words. 
If the order of the input is important, various mechanisms, such as the Positional Encoding, discussed in Section~\ref{subsub:positional_encoding}, are used to capture this subtlety. 

\section{Visual Transformers} 
\label{sec:transformer}
The Transformer architecture was introduced in~\cite{vaswani2017attention} and is the first architecture that relies purely on attention to draw connections between the inputs and outputs. 
Since its debut, it revolutionized Deep Learning, making breakthroughs in numerous fields, including Natural Language Processing, Computer Vision, Chemistry, Biology, thus making its way to becoming the \textit{default} architecture for learning representations. 
Recently, the standard Transformer~\cite{vaswani2017attention} has been adapted for vision tasks~\cite{dosovitskiy2020vit}. And again, visual Transformer has become one of the central architectures in computer vision.

In this section, we first introduce the basic architecture of Transformers (Section~\ref{sub:basic_transformer}) and then present its advantages (Section~\ref{sub:advantage_transformers}). Finally, we describe the Vision Transformer (Section~\ref{sub:vit}).

%%%%%%%%%%%%%%%%%%%%%%%%%%%%%%%%%%%%%%%
\subsection{Basic Transformers}
\label{sub:basic_transformer}
%%%%%%%%%%%%%%%%%%%%%%%%%%%%%%%%%%%%%%%

\begin{figure}[hbtp]
    \centering
    \includegraphics[width=0.8\textwidth]{figures/transformer_v2.pdf}
    \caption{\textbf{The Transformer architecture.} It consists of an encoder (left) and a decoder (right) block, each one consisting from a series of attention blocks (Multi-head and Masked Multi-head attention) and MLP layers. Next to each element, we denote its dimentionality. Figure inspired from~\cite{vaswani2017attention}. 
    }
    \label{fig:transformer_arch}
\end{figure}

As shown in Figure~\ref{fig:transformer_arch}, the Transformer architecture~\cite{vaswani2017attention} is an encoder-decoder model. First, it embeds input tokens $\bm{X} = (\bm{x_1}, \dots, \bm{x_N})$ into a latent space, resulting in latent vectors $\bm{Z} = (\bm{z_1}, \dots, \bm{z_N})$, which are fed to the decoder to output $\bm{Y} = (\bm{y_1}, \dots, \bm{y_M})$. 
The encoder is a stack of $L$ layers, with each one consisting of two sub-blocks: Multi-head Self-Attention (MSA) layers and a Multi-Layer Perceptron (MLP). 
The decoder is also a stack of $L$ layers, with each one consisting of three sub-blocks: Masked Multi-head Self-Attention (MMSA),  Multi-head Cross-Attention (MCA), and a MLP.

\paragraph{Overview.}
Below, we describe the various parts of the Transformer architecture, following Figure~\ref{fig:transformer_arch}. 
First, the input tokens are converted into the embedding tokens (Section~\ref{subsub:embedding}). 
Then, the Positional encoding adds a positional token to each embedding token to denote the order of tokens (Section~\ref{subsub:positional_encoding}). 
Then, the Transformer encoder follows (Section~\ref{subsub:transformer_encoder}). This consists of a stack of $L$ multi-head attention, nomalization and MLP layers and encodes the input to a set of semantically meaningful features. 
After, the decoder follows (Section~\ref{subsub:transformer_decoder}). This consists of a stack of $L$ masked multi-head attention, multi-head attention, and MLP layers followed by normalizations and decodes the input features with respect to the output embedding tokens. 
Finally, the output is projected to linear and Softmax layers. 

%%%%%%%%%%%%%%%%%%%%%%%%%%%%%%%%%%%%%%%
\subsubsection{Embedding}
\label{subsub:embedding}
%%%%%%%%%%%%%%%%%%%%%%%%%%%%%%%%%%%%%%%

The first step of Transformers consists in converting input tokens\footnote{Note, the initial Transformer architecture was proposed for natural language processing (NLP), and therefore the inputs were words.} into embedding tokens, i.e., vectors with meaningful features. 
To do so, following standard  practice~\cite{word_embedding_Press_2016}, each input is projected into an embedding space to obtain embedding tokens $\bm{Z^{\text{e}}}$. 
The embedding space is structured in a way that the distance between a pair of vectors is relative to the semantic similarity of their associated words. For the initial NLP case, this means that we get a vector of each word, such that the vectors that are closer together have similar meanings.

%%%%%%%%%%%%%%%%%%%%%%%%%%%%%%%%%%%%%%%
\subsubsection{Positional Encoding}
\label{subsub:positional_encoding}
%%%%%%%%%%%%%%%%%%%%%%%%%%%%%%%%%%%%%%%

As discussed in Section~\ref{sub:property_attention}, the attention mechanism is positional agnostic, which means that it does not store the information on the position of each input. 
However, in most cases, the order of input tokens is relevant and should be taken into account, such as the order of words in a sentence matter as they may change its meaning. 
Therefore, \cite{vaswani2017attention} introduced the \emph{\textbf{P}ositional \textbf{E}ncoding} $\textbf{PE}\in\mathbb{R}^{N \times d_x}$, which adds a positional token to each embedding token $\bm{Z^{\text{e}}} \in\mathbb{R}^{N \times d_x}$. 

\paragraph{Sinusoidal Positional Encoding.}
The Sinusoidal Positional Encoding~\cite{vaswani2017attention} is the main positional encoding method, which encodes the position of each token with sinusoidal waves of multiples frequency. For an embedding token $\bm{Z^{\text{e}}} \in\mathbb{R}^{N\times d_x}$, its Positional Encoding $\textbf{PE}\in\mathbb{R}^{N\times d_x}$ is defined as: 

\begin{equation}
\label{eq:positional_encoding}
\begin{array}{l}
    \textbf{PE}(i,2j)   = \text{sin} \left(\frac{i}{10000^{2j/d}} \right) \\
    \textbf{PE}(i,2j+1) = \text{cos} \left(\frac{i}{10000^{2j/d}} \right), \forall i,j \in [|1,n|]\times[|1,d|] \quad .
\end{array}
\end{equation}

\paragraph{Learnable Positional Encoding.}
An orthogonal approach is to let the model learn the positional encoding. In this case, $\textbf{PE}\in\mathbb{R}^{N\times d_x}$ becomes a learnable parameter. This, however, increases the memory requirements, without necessarily bringing improvements over the sinusoidal encoding.

\paragraph{Positional Embedding.}
After its computation, the Positional Encoding \textbf{PE} is either added to the embedding tokens, or they are concatenated as follows: 

\begin{equation}
\label{eq:positional_embedding}
\begin{array}{l}
    \bm{Z}^{\text{pe}} = \bm{Z^{e}}+\bm{\textbf{PE}} \quad, \text{ or} \\
    \bm{Z}^{\text{pe}} = \text{Concat}(\bm{Z^{e}}, \bm{\textbf{PE}}) \quad, 
\end{array}
\end{equation}


\noindent where Concat denotes vector concatenation. 
Note that the concatenation has the advantage of not altering the information contained in $\bm{Z}^e$, since the positional information is only added to the unused dimension. Nevertheless, it augments the input dimension, leading to higher memory requirements.
Instead, the addition does preserve the same input dimension, while altering the content of the embedding tokens. When the input dimension is high, this content altering is trivial, as most of the content is preserved. 
Therefore, in practice, for high-dimension summing positional encodings is preferred, whereas for low dimensions, concatenating them prevails. 

%%%%%%%%%%%%%%%%%%%%%%%%%%%%%%%%%%%%%%%
\subsubsection{Encoder Block}
\label{subsub:transformer_encoder}
%%%%%%%%%%%%%%%%%%%%%%%%%%%%%%%%%%%%%%%

The encoder block takes as input the embedding and positional tokens and outputs features of the input, to be decoded by the decoder block. It consists of a stack of $L$ Multi-head Self-Attention (MSA) layers and a Multi-Layer Perceptron (MLP). Specifically, the embedding and positional tokens, $\bm{Z}_{x}^{\text{pe}} \in \mathbb{R}^{N \times d}$, go through a multi-head self-attention block. Then, a residual connection with layer normalization is deployed. In the Transformer, this operation is performed after each sub-layer. Next, we feed its output it to a MLP and a normalization layer. This operation is performed $L$ times and each time the output of each encoder block (of size $N \times d$) is the input of the subsequent block. The $L-$th time, the output of the normalization is the input of the cross attention block in the decoder (Section~\ref{subsub:transformer_decoder}). 

%%%%%%%%%%%%%%%%%%%%%%%%%%%%%%%%%%%%%%%
\subsubsection{Decoder Block}
\label{subsub:transformer_decoder}
%%%%%%%%%%%%%%%%%%%%%%%%%%%%%%%%%%%%%%%

The decoder has two inputs: first, an input that constitutes the queries $\bm{Q} \in \mathbb{R}^{N \times d}$ of the encoder, and, second, the output of the encoder that constitutes the key-value $\bm{K}, \bm{V} \in \mathbb{R}^{N \times d}$ pair. Similar to Sections~\ref{subsub:embedding}-\ref{subsub:positional_encoding}, the first step constitutes encoding the output token to output embedding token and output positional token. These tokens are fed into the main part of the decoder, which consists of a stack of $L$ Masked Multi-head Self-Attention (MMSA) layers, Multi-Head Cross-Attention (MCA) layers, and Multi-Layer Perceptron (MLP) followed by normalizations. Specifically, the embedding and positional tokens, $\bm{Z}_{y}^{\text{pe}} \in \mathbb{R}^{N \times d}$, go through a MMSA block. Then, a residual connection with layer normalization follow. Next, a MCA layer (followed by a normalization) maps the queries to the encoded keys-values before forwarding the output to a MLP. Finally, we project the output of the $L$ decoder blocks (of dimension $N \times d_y$) through a linear layer and get output probability through a soft-max layer.

%%%%%%%%%%%%%%%%%%%%%%%%%%%%%%%%%%%%%%%
\subsection{Advantages of Transformers}
\label{sub:advantage_transformers}
%%%%%%%%%%%%%%%%%%%%%%%%%%%%%%%%%%%%%%%

Since their introduction, the Transformers have had a significant impact on deep learning approaches. 

In natural language processing (NLP), before Transformers, most  architectures used to rely on recurrent modules, such as RNNs~\cite{cho_RNN_translation_2014} and in particular LSTMs~\cite{Sutskever_LSTM_Translation_2014}. 
However, recurrent models process the input sequentially, meaning that, to compute the current state, they require the output of the previous state. This makes them tremendously inefficient, as they are impossible to parallelize.
On the contrary, in Transformers, each input is processed independent of the others, and the multi-head attention can perform multiple attention computations at once. This makes Transformers highly efficient, as they are highly parallelizable.  

This results in not only exceptional scalability, both in the complexity of the model and size of datasets, but also relatively fast training. Notably, the recent Switch Transformers~\cite{fedus_switch_2021} was pre-trained on 34 billion tokens from the C4 dataset~\cite{RAffel_C4_dataset_2020}, scaling the model to over 1 trillion parameters.

This scalability~\cite{fedus_switch_2021} is the principal reason for the power of the Transformer. While it was originally introduced for translation, it refrains from introducing many inductive biases, i.e., the set of assumptions that the user makes about the structure of the model input. In doing so, the Transformer relies on data to learn how they are structured. Compared to its counterparts with more biases, the Transformer requires much more data to produce comparable results~\cite{dosovitskiy2020vit}. However, if a sufficient amount of data is available, the lack of inductive bias becomes a strength. By learning the structure of the data from the data, the Transformer is able to learn better without human assumptions hindering~\cite{khan_vision_survey_2021}.

In most tasks involving Transformers, the model is first pre-trained on a large dataset, and then fine-tuned for the task at hand on a smaller dataset. The pre-training phase is essential for Transformers to learn the global structure of the specific input modality. For fine-tuning, typically fewer data suffice as the model is already rich. 
For instance, in natural language processing, BERT~\cite{devlin2018bert}, a state-of-the-art language model, is pre-trained on a Wikipedia-based dataset \cite{wikidump}, with over 6 million articles and Book Corpus \cite{zhu_corpus_2015} with over 10,000 books. Then, this model can be fine-tuned on much more specific tasks.
In Computer Vision, the Vision Transformer (ViT) is pre-trained on the JFT-300M dataset, containing over one billion labels for 300 million images \cite{dosovitskiy2020vit}.
Hence, with a sufficient amount of data, Transformers achieve results that were never possible before in various areas of Machine Learning.

%%%%%%%%%%%%%%%%%%%%%%%%%%%%%%%%%%%%%%%
\subsection{Vision Transformer}
\label{sub:vit}
%%%%%%%%%%%%%%%%%%%%%%%%%%%%%%%%%%%%%%%

Transformers offer an alternative to CNNs that have long held a stranglehold on Computer Vision.  
Before 2020, most attempts to use Transformers for vision tasks were still highly reliant on CNNs, either by using self-attention jointly with convolutions~\cite{wang_vision_2017,carion_vision_2020} 
or by keeping the general structure of CNNs while using self-attention~\cite{Ramachandran_vision_2019,wang_vision_2020}.

The reason for this is rooted in the two main weaknesses of the Transformers. First, the complexity of the attention operation is high. As attention is a quadratic operation, the number of parameters skyrockets quickly when dealing with visual data, i.e., images --and even more so with videos--. For instance, in the case of ImageNet~\cite{deng2009imagenet}, inputting a single image with $256 \times 256 = 65,536$ pixels in an attention layer would be too heavy computationally. Second, Transformers suffer from lack of inductive biases. Since CNNs were specifically created for vision tasks, their architecture includes spatial inductive biases, like translation equivariance and locality. Therefore, the Transformers have to be pre-trained on a significantly large dataset to achieve similar performances.

\begin{figure}%[hbtp]
    \centering
    \includegraphics[width=1.0\textwidth]{figures/vit_v2.pdf}
    \caption{\textbf{The Vision Transformer architecture (ViT).} First, the input image is split into patches (bottom), which are linearly projected (embedding), and then concatenated with positional embedding tokens. The resulting tokens are fed into a Transformer, and finally the resulting classification token is passed through an MLP to compute output probabilities.
    Figure inspired from~\cite{dosovitskiy2020vit}.}
    \label{fig:vit_arch}
\end{figure}

The Vision Transformer (ViT)~\cite{dosovitskiy2020vit} is the first systematic approach that uses directly Transformers for vision tasks by addressing both aforementioned issues. It rids the concept of convolutions altogether, using purely a Transformer-based architecture. In doing so, it achieves the state of the art on image recognition on various datasets, including ImageNet~\cite{deng2009imagenet} and CIFAR-100~\cite{krizhevsky2009cifar}. 

Figure~\ref{fig:vit_arch} illustrates the ViT architecture. 
The input image is first split into $16 \times 16$ patches, flattened and mapped to the expected dimension through a learnable linear projection. Since the image size is reduced to $16 \times 16$, the complexity of the attention mechanism is no longer a bottleneck. Then, ViT encodes the positional information and attaches a learnable embedding to the front of the sequence, similarly to BERT’s classification token~\cite{devlin2018bert}. The output of this token represents the entirety of the input -- it encodes the information from each part of the input. Then, this sequence is fed into an encoder block, with the same structure as in the standard Transformers~\cite{vaswani2017attention}. The output of the classification token is then fed into an MLP that outputs class probabilities.

Due to lack of inductive biases, when ViT is trained only on mid-sized datasets such as ImageNet, it scores some percentage points lower than the state of the art. Therefore, the proposed model is first pre-trained on the JFT-300M dataset~\cite{sun_jft300m_2017} and then fine-tuned on smaller datasets, thereby increasing its accuracy by $13\%$. 

For a complete overview of visual Transformers and follow-up works, we invite the readers to study~\cite{khan_vision_survey_2021,selva_video_survey_2022}. 



\section{Improvements over the Vision Transformer} 
\label{sec:extensions}
%%%%%%%%%%%%%%%%%%%%%%%%%%%%%%%%%%%%%%%%%%%%%%%%%%%
In this section, we present Transformer-based methods that improve over the original Vision Transformer (Section~\ref{sub:vit}) in two main ways. First, we introduce approaches that are trained on smaller datasets, unlike ViT~\cite{dosovitskiy2020vit} that requires pre-training on 300 million labelled images (Section~\ref{sub:data_efficiency}). 
Second, we present extensions over ViT that are more computational efficient than ViT, given that training a ViT is directly correlated to the image resolution and the number of patches (Section~\ref{sub:compute_efficiency}). 


%%%%%%%%%%%%%%%%%%%%%%%%%%%%%%%%%%%%%%%%%%%%%%%%%%%
\subsection{Data Efficiency}
\label{sub:data_efficiency}
%%%%%%%%%%%%%%%%%%%%%%%%%%%%%%%%%%%%%%%%%%%%%%%%%%%

As discussed in Section~\ref{sub:vit}, the Vision Transformer (ViT)~\cite{dosovitskiy2020vit} is pretrained on a massive proprietary dataset (JFT-300M) which contains 300 million labelled images. This need arises with Transformers because we remove the inductive biases from the architecture compared to convolutional based networks. Indeed, convolutions contain some translation equivariance. ViT does not benefit from this property and thus has to learn such biases, requiring more data. JFT-300M is an enormous dataset and to make ViT work in practice, better data-efficiency is needed. Indeed, collecting that amount of data is costly and can be infeasible for most tasks.


\begin{figure}[hbtp]
	\centering
		\includegraphics[width=0.8\textwidth]{figures/deit_v2.pdf}
	   \caption{\textbf{The DeiT architecture.} The architecture features an extra token, the distillation token. This token is used similarly to the class token. Figure inspired from~\cite{deit}.}
	\label{fig:deit}
\end{figure}

\paragraph{Data-efficient image Transformers (DeiT)~\cite{deit}.}
The first work to achieve an improved data efficiency is DeiT~\cite{deit} . The main idea of DeiT is to distil the inductive biases from a CNN into a Transformer (Figure~\ref{fig:deit}). DeiT adds another token that works similarly to the class token. When training, ground truth labels are used to train the network according to the class token output with a cross entropy loss. However, for the distillation network, the output labels are compared to the labels provided from a teacher network with a cross entropy loss. The final loss for a $N$-categorical classification task is defined as follows:
\begin{equation}
\begin{array}{c}
    \mathcal{L}_{global}^{hardDistill} = \frac{1}{2}(\mathcal{L}_{CE}(\Psi(\bm{Z}_{class}), \bm{y}) + \mathcal{L}_{CE}(\Psi(\bm{Z}_{distill}), \bm{y}_T) ) \quad , \\
    \mathcal{L}_{CE}(\bm{\hat{y}}, \bm{y}) = - \frac{1}{N}\sum_{i = 1}^{N} \left[ y_i \log \hat{y}_i + (1 - y_i) \log(1 - \hat{y}_i) \right] 
\end{array}
\end{equation}
with $\Psi$ the Softmax function, $\bm{Z}_{class}$ the class token output,  $\bm{Z}_{distill}$ the class token output, $\bm{y}$ the ground truth label and $\bm{y}_T$ the teacher label prediction. 

The teacher network is a Convolutional Neural Network (CNN). The main idea is that the distillation head will provide the inductive biases needed to improve the data efficiency of the architecture. By doing this, DeiT achieves remarkable performance on the ImageNet dataset, by training `only' on ImageNet-1K~\cite{deng2009imagenet}, which contains 1.3 million images.

\paragraph{Convit~\cite{convit}.}
The main disadvantage of DeiT~\cite{deit} is that is requires a pretrained CNN, which is not ideal, and it would be more convenient to not have this requirement. The CNN has a hard inductive bias constraint that can be a major limitation. Indeed, if enough data is available, learning the biases from the data can result in better representations. 

Convit~\cite{convit} overpasses this issue by including the inductive bias of CNNs into a Transformer in a soft way. Specifically, if the inductive bias is limiting the  training, the Transformer can discard it. The main idea is to include the inductive bias into the ViT initialization. Therefore, before beginning training, the ViT is equivalent to a CNN. Then, the network can progressively learn the needed biases and diverge from the CNN initialization. 

\begin{figure}[hbtp]
	\centering
		\includegraphics[width=\textwidth]{figures/cct_v2.pdf}
	   \caption{\textbf{Compact Convolutional Transformers.} This architecture  features a convolutional based patch extraction to leverage a smaller Transformer network, leading to higher data-efficiency. Figure inspired from~\cite{cct}.}
	\label{fig:cct}
\end{figure}

\paragraph{Compact Convolutional Transformer~\cite{cct}.}
DeiT~\cite{deit} and ConVit~\cite{convit} successfully achieve data efficiency at the ImageNet scale. However, ImageNet is a big dataset with 1.3 million images, whereas most datasets are significantly smaller. 

To reach higher data efficiency, the Compact Convolutional Transformer~\cite{cct} uses a CNN operation to extract the patches, and then uses these patches in a Transformer network (Figure~\ref{fig:cct}). 
The Compact Convolutional Transformer comes with some modifications that lead to  major improvements. First, by having a more complex encoding of patches, the system relies on the convolutional inductive biases at the lower scales and then uses a Transformer network to remove the locality constraint of the CNN. Second, the authors show that discarding the `class' token results in higher efficiency. Specifically, instead of the class token, the Compact Convolutional Transformer pools together all the patches token and classifies on top of this pooled token. These two modifications enable using smaller Transformers, while improving both the data efficiency and the computational efficiency. Therefore, these improvements allow the Compact Convolutional Transformer to be successfully trained on smaller datasets, such as CIFAR or MNIST.
 

%%%%%%%%%%%%%%%%%%%%%%%%%%%%%%%%%%%%%%%%%%%%%%%%%%%
\subsection{Computational Efficiency}
\label{sub:compute_efficiency}
%%%%%%%%%%%%%%%%%%%%%%%%%%%%%%%%%%%%%%%%%%%%%%%%%%%

The Vision Transformer architecture (Section~\ref{sub:vit}) suffers from a $\mathcal{O}(n^2)$ complexity with respect to the number of tokens. When considering small resolution images or big patch size, this is not a limitation; for instance, for an image of $224 \times 224$ resolution with $16 \times 16$ patches, this amounts to $196$ tokens. However, when needing to process larger images (for instance 3D images in medical imaging) or when considering smaller patches, using and training such models becomes \emph{prohibitive}. 
For instance, in tasks such as segmentation or image generation, it is needed to have more granular representations than $16 \times 16$ patches; hence, it is crucial to solve this issue to enable more applications of Vision Transformer.

\begin{figure}[hbtp]
	\centering
		\includegraphics[width=0.9\textwidth]{figures/swin.pdf}
	   \caption{\textbf{Shifting operation in the Swin Transformer~\cite{swin}.} Between each attention operation, the attention window is shifted so that each patch can communicate with different patches than before. This allows the network to gain more global knowledge with the network's depth. Figure inspired from~\cite{swin}.
	   }
	\label{fig:swin}
\end{figure}

\paragraph{Swin Transformer~\cite{swin}.}
One idea to make Transformers more computation efficient is the Swin Transformer~\cite{swin}. Instead of attending every patch in the image, the Swin Transformer proposes to add a locality constraint. Specifically, the patches can  only attend other patches that are limited to a vicinity window $K$. This restores the local inductive bias of CNNs. To allow communication across patches throughout the network, the Swin Transformer shifts the attention windows from one operation to another (Figure~\ref{fig:swin}). Therefore, the Swin Transformer is quadratic with regard to the size of the window $K$ but linear with respect to the number of tokens $n$ with complexity $\mathcal{O}(nK^2)$. In practice, however, $K$ is small, and this solves the quadratic complexity problem of attention.

\begin{figure}[hbtp]
	\centering
		\includegraphics[width=\textwidth]{figures/perceiver_v2.pdf}
	   \caption{\textbf{The Perceiver architecture~\cite{jaegle2021perceiver,perceiverio}.} A set of latent tokens retrieve information from the image through Cross-Attention. Self-Attention is performed between the tokens to refine the learned representation. These operations are linear with respect to the number of image tokens. Figure inspired from~\cite{jaegle2021perceiver,perceiverio}.}
	\label{fig:perceiver}
\end{figure}


\paragraph{Perceiver~\cite{jaegle2021perceiver,perceiverio}.}
Another idea for a more computation efficient visual Transformers is to make a more drastic change to the architecture. If instead of using self-attention, the model uses cross-attention, the problem of the quadratic complexity with regard to the number of tokens can be solved. Indeed, computing the cross attention between two sets of length $m$ and $n$, respectively, has complexity $\mathcal{O}(mn)$. This idea is introduced in the Perceiver~\cite{jaegle2021perceiver,perceiverio}. The key idea is to have a smaller set of latent variables that will be used as queries and that will retrieve information in the image token set (Figure~\ref{fig:perceiver}). Since this solves the quadratic complexity issue, it also removes the need of using patches; hence, in the case of Transformers, each pixel is mapped to a single token.

\section{Vision Transformers for tasks other than classification} 
\label{sec:tasks}
Sections~\ref{sec:attention}-\ref{sec:extensions} introduce visual Transformers for one main application: classification. 
Nevertheless, Transformers can be used for numerous other tasks than classification. 

In this section, we present some fundamental vision tasks where Transformers have had a major impact: 
object detection in images (Section~\ref{sub:detection}), image segmentation (Section~\ref{sub:segmentation}), 
training visual Transformers without labels (Section~\ref{sub:no_labels}), and image generation using Generative Adversarial Networks (GANs) (Section~\ref{sub:generative_models}). 


%%%%%%%%%%%%%%%%%%%%%%%%%%%%%%%%%%%%%%%
\subsection{Object detection with Transformers}
\label{sub:detection}
%%%%%%%%%%%%%%%%%%%%%%%%%%%%%%%%%%%%%%%

\begin{figure}[hbtp]
	\centering
		\includegraphics[width=1.0\textwidth]{figures/detr.png}
	   \caption{\textbf{The DETR architecture.} It refines a CNN visual representation to extract object localization and classes. Figure inspired from~\cite{carion2020end}.
	   }
	\label{fig:detr}
\end{figure}

Detection is one of the early tasks that have seen improvements thanks to Transformers. 
Detection is a combined recognition and localization problem; this means that a successful detection system should both recognize whether an object is present in an image and localize it spatially in the image. \cite{carion2020end} is the first approach that uses Transformers for detection. 

\paragraph{DEtection TRansformer (DETR)~\cite{carion2020end}.}
DETR first extracts visual representations with a convolutional network (Figure~\ref{fig:detr})\footnote{Note that in DETR, the Transformer is not directly used to extract the visual representation. Instead, it focuses on refining the visual representation to extract the object information.
}. 
Then, the encodings are processed by a Transformer network. Finally, the processed tokens are provided to a Transformer decoder. The decoder uses cross-attention between a set of learned tokens and the image tokens encoded by the encoder and outputs a set of tokens. Each output token is then passed through a feed-forward network that predicts if an object is present in an image or not; if the object is indeed present, the network also predicts the class and spatial location of the object, i.e., coordinates within the image. 


%%%%%%%%%%%%%%%%%%%%%%%%%%%%%%%%%%%%%%%
\subsection{Image segmentation with Transformers}
\label{sub:segmentation}
%%%%%%%%%%%%%%%%%%%%%%%%%%%%%%%%%%%%%%%

\begin{figure}[hbtp]
	\centering
		\includegraphics[width=0.8\textwidth]{figures/segmenter.png} 
	   \caption{\textbf{The Segmenter architecture}. It is a purely ViT based approach to perform semantic segmentation. Figure inspired from~\cite{strudel2021segmenter}.}
	\label{fig:segmenter}
\end{figure}

The goal of image segmentation is to assign to each pixel of an image the label of the object it belongs to. 
The Segmenter~\cite{strudel2021segmenter} is a purely ViT approach addressing image segmentation. The idea is to first use ViT to encode the image. Then, the Segmenter learns a token per semantic label. The encoded patch tokens and the semantic tokens are then fed to a 2nd Transformer. Finally, by computing the scalar product between the semantic tokens and the image tokens, the network assigns a label to each patch. Figure~\ref{fig:segmenter} displays this. 

%%%%%%%%%%%%%%%%%%%%%%%%%%%%%%%%%%%%%%%
\subsection{Training Transformers without labels}
\label{sub:no_labels}
%%%%%%%%%%%%%%%%%%%%%%%%%%%%%%%%%%%%%%%

Visual Transformers have initially been trained for classification tasks. However, this tasks requires having access to massive amounts of labelled data, which can be hard to obtain (as discussed in Section~\ref{sub:data_efficiency}). Sections~\ref{sub:data_efficiency}-\ref{sub:compute_efficiency} present ways to train ViT more efficiently. However, it would also be interesting to be able to train this type of networks with ``cheaper'' data. Therefore, the goal of this part is to introduce unsupervised learning with Transformers, i.e., training Transformers without any labels. 

\begin{figure}[hbtp]
	\centering
		\includegraphics[width=0.6\textwidth]{figures/dino_v2.pdf}
	   \caption{\textbf{The DINO training procedure.} It consists in matching the outputs between two networks ($\bm{p}_1$ and $\bm{p}_2$) having two different augmentations ($\bm{X}_1$ and $\bm{X}_2$) of the same image as input ($\bm{X}$). The parameters of the teacher model are updated with an exponential moving average (ema) of the student parameters. 
	   Figure inspired from~\cite{dino}.}
	\label{fig:dino_arch}
\end{figure}

\begin{figure}[hbtp]
	\centering
		\includegraphics[width=\textwidth]{figures/DINO_examples.pdf}
	   \caption{\textbf{DINO samples.} Visualization of the attention matrix of ViT heads trained with DINO. The ViT discovers the semantic structure of an image in an unsupervised way.}
	\label{fig:dino_sample}
\end{figure}

\paragraph{Self-DIstillation with NO labels (DINO)~\cite{dino}.}
DINO is one of the first works that trains a ViT with self-supervised learning (Figure~\ref{fig:dino_arch}). The main idea is to have two ViT models following the teacher-student paradigm: the first model is updated through gradient descent and the second is an exponential moving average of the first one. 
Then, the whole two-stream DINO network is trained using two augmentations of the same image, that are each passed to one of the two networks. The goal of the training is to match the output between the two networks, i.e., no matter the augmentation in the input data, both networks should produce the same result. 
The main finding of DINO is that the ViT is capable of learning a semantic understanding of the image, as the attention matrices display some semantic information. Figure~\ref{fig:dino_sample} visualizes the attention matrix of the various ViT heads trained with DINO. 

\begin{figure}[hbtp]
	\centering
		\includegraphics[width=\textwidth]{figures/mae.pdf}
	   \caption{\textbf{The MAE training procedure}. After masking some tokens of an image, the remaining tokens are fed to an encoder. Then a decoder tries to reconstruct the original image from this representation. 
	   Figure inspired from~\cite{he2021masked}.  }
	\label{fig:mae_arch}
\end{figure}

\paragraph{Masked Autoencoders (MAE)~\cite{he2021masked}.}
Another way to train a ViT without supervision is by using an autoencoder architecture. Masked Autoencoders (MAE)~\cite{he2021masked} perform a random masking of the input token and give the task to reconstruct the original image to a decoder. The encoder learns a representation that performs well in a given downstream task. This is illustrated in Figure~\ref{fig:mae_arch}. 
One of the key observations of the MAE work~\cite{he2021masked} is that the decoder does not need to be very good for the encoder to achieve good performance: by using only a small decoder, MAE successfully trains a ViT in an auto-encoder fashion.

%%%%%%%%%%%%%%%%%%%%%%%%%%%%%%%%%%%%%%%
\subsection{Image generation with Transformers and Attention}
\label{sub:generative_models}
%%%%%%%%%%%%%%%%%%%%%%%%%%%%%%%%%%%%%%%

Attention and Vision Transformers have also helped in developing fresh ideas and creating new architectures for generative models, and in particular for Generative Adversarial Networks (GANs). 

\begin{figure}[hbtp]
	\centering
		\includegraphics[width=0.4\textwidth]{figures/gansformers.png}
	   \caption{\textbf{GANsformers architecture}. A set of latents contribute to bring information to a CNN feature map. Figure inspired from~\cite{hudson2021generative}.}
	\label{fig:gansformers}
\end{figure}

\paragraph{GANsformers~\cite{hudson2021generative}.} 
GANsformers are the most representative work of GANs with Transformers, as they are a hybrid architecture using both attention and CNNs. 
The GANsformers architecture is illustrated in Figure~\ref{fig:gansformers}. 
The model first splits the latent vector of a GAN into multiple tokens. Then, a cross attention mechanism is used to improve the generated feature maps and, at the same time, the GANsformers architecture retrieves information from the generated features map to enrich the tokens. This mechanism allows the GAN to have better and richer semantic knowledge, which is showed to be useful for generating multimodal images.

\paragraph{StyleSwin~\cite{zhang2021styleswin}.}
Another approach for generative modeling is to purely use a ViT architecture like StyleSwin~\cite{zhang2021styleswin}. StyleSwin is a GAN that leverages a similar type of attention as the Swin Transformer~\cite{swin}. This allows to generate high definition images without having to deal with the quadratic cost problem.




\section{Vision Transformers for other domains} 
\label{sec:domains}
In this section, we present applications of visual Transformers to other domains. 
First, we describe multimodal Transformers operating with vision and language (Section~\ref{sub:multimodal_transformers}), then we describe video-level attention and video Transformers (Sections~\ref{sub:video_attention}-\ref{sub:video_transformers}), and finally we present multimodal video Transformers operating with vision, language and audio (Section~\ref{sub:multi_modal}). 

%%%%%%%%%%%%%%%%%%%%%%%%%%%%%%%%%%%%%%%
\subsection{Multimodal Transformers: vision and language}
\label{sub:multimodal_transformers}
%%%%%%%%%%%%%%%%%%%%%%%%%%%%%%%%%%%%%%%

As Transformers have found tremendous success in both natural language processing and computer vision, their use in vision-language tasks is also of interest. In this section, we describe some representative multimodal methods for vision and language: ViLBERT (Section~\ref{subsub:vilbert}), DALL-E (Section~\ref{subsub:dalle}) and CLIP (Section~\ref{subsub:clip}). 

%%%%%%%%%%%%%%%%%%%%%%%%%%%%%%%%%%%%%%%
\subsubsection{ViLBERT}
\label{subsub:vilbert}
%%%%%%%%%%%%%%%%%%%%%%%%%%%%%%%%%%%%%%%

Vision-and-language BERT (VilBERT)~\cite{lu_vilbert_2019} is an example of architecture that fuses two modalities. 
It consists of two parallel streams, each one working with one modality. The vision stream extracts bounding boxes from images via an object detection network, by encoding their position. The language stream embeds word vectors and extracts feature vectors using the basic Transformers encoder block~\cite{vaswani2017attention} (Figure~\ref{fig:transformer_arch} left). These two resulting feature vectors are then fused together by a Cross-Attention layer (Section~\ref{subsub:cross_attention}). This follows the standard architecture of the Transformers encoder block, where the keys and values of one modality are passed onto the MCA block of the other modality. The output of the Cross-Attention Layer is passed into another Transformers encoder block and these two layers are stacked multiple times. 

The language stream is initialized with BERT trained on Book Corpus~\cite{zhu_corpus_2015} and Wikipedia~\cite{wikidump}, while the visual stream is initialized with Faster R-CNN~\cite{ren_fastercnn(vilbert)_2015}. On top of the pre-training of each stream, the whole architecture is pre-trained on the Conceptual Captions dataset~\cite{sharma2018conceptual} on two pretext tasks. 

ViLBERT has been proven powerful for a variety of multimodal tasks. In the original paper, ViLBERT was fined-tuned to a variety of tasks, including visual question answering, visual common-sense reasoning, referring expressions, and caption-based image retrieval.


%%%%%%%%%%%%%%%%%%%%%%%%%%%%%%%%%%%%%%%
\subsubsection{CLIP}
\label{subsub:clip}
%%%%%%%%%%%%%%%%%%%%%%%%%%%%%%%%%%%%%%%

Connecting Text and Images (CLIP)~\cite{clip} is designed to address two major issues of deep learning models: costly datasets and inflexibility. While most deep learning models are trained on labelled datasets, CLIP is trained on 400 million text-image pairs that are scraped from the internet. This reduces the labour of having to manually label millions of images that are required to train powerful deep learning models. When models are trained on one specific dataset, they also tend to be difficult to extend to other applications. For instance, the accuracy of a model trained on ImageNet is generally limited to its own dataset and cannot be applied to real-world problems. To optimize training, CLIP models learn to perform a wide variety of tasks during pretraining, and this task allows for zero-shot transfer to many existing datasets. While there are still several potential improvements, this approach is competitive to supervised models that are trained on specific datasets.

\paragraph{CLIP architecture and training.}
CLIP is used to measure the similarity between the text input and the image generated from a latent vector. At the core of the approach is the idea of learning perception from supervision contained in natural language. Methods which work on natural language can learn passively from the supervision contained in the vast amount of text on the internet. 

Given a batch of $N$ (image, text) pairs, CLIP is trained to predict which of the $N \times N$ possible (image, text) pairings across a batch actually occurred. To do this, CLIP learns a multimodal embedding space by jointly training an image encoder and a text encoder to  maximize the cosine similarity of the image and text embeddings of the $N$ real pairs in the batch, while minimizing the cosine similarity of the embeddings of the $N^{2} {\--} N$ incorrect pairings. A symmetric cross entropy loss over these similarity scores is optimized.

Two different architectures were considered for the image encoder. For the first, ResNet-50~\cite{he_resnet50_2017} is used as the base architecture for the image encoder due to its widespread adoption and proven performance. Several modifications were made to the original version of ResNet.
For the second architecture, ViT is used with some minor modifications: first, adding an extra layer normalization to the combined patch and position embeddings before the Transformer, and second, using a slightly different initialization scheme.

The text encoder is a standard Transformer~\cite{vaswani2017attention} (Section~\ref{sub:basic_transformer}) with the architecture modifications described in~\cite{clip}. As a base size, CLIP uses a 63M- parameter 12- layer 512-wide model with 8 attention heads. The Transformer operates on a lower-cased byte pair encoding (BPE) representation of the text with a 49,152 vocab size~\cite{sennrich2015neural}. The max sequence length is capped at 76. The text sequence is bracketed with [SOS] and [EOS] tokens\footnote{[SOS]: start-of-sequence; [EOS]: end-of-sequence} and the activations of the highest layer of the Transformer at the [EOS] token are treated as the feature representation of the text which is layer normalized and then linearly projected into the multimodal embedding space.

%%%%%%%%%%%%%%%%%%%%%%%%%%%%%%%%%%%%%%%
\subsubsection{DALL-E and DALL-E 2}
\label{subsub:dalle}
%%%%%%%%%%%%%%%%%%%%%%%%%%%%%%%%%%%%%%%

DALL-E~\cite{ramesh_dalle_2021} is another example of the application of Transformers in vision. It generates images from a natural language prompt -- some examples include `an armchair in the shape of an avocado' and `a penguin made of watermelon'. It uses a decoder-only model, which is similar to GPT-3~\cite{brown_gpt3_2020}. DALL-E uses 12 billion parameters and is pre-trained on Conceptual Captions~\cite{sharma2018conceptual} with over 3.3 million text-image pairs. 
DALL-E 2~\cite{ramesh2022dalle2} is the upgraded version of DALL-E, based on diffusion models and CLIP (Section~\ref{subsub:clip}), and allows better performances with more realistic and accurate generated images. 
In addition to producing more realistic results with a better resolution than DALL-E, DALL-E 2 is also able to edit the outputs. Indeed, with DALL-E 2, one can add or remove realistically an element in the output, and can also generate different variations of the same output.
These two models clearly demonstrate the powerful nature and scalability of Transformers that are capable of efficiently processing a web-scale amount of data.

%%%%%%%%%%%%%%%%%%%%%%%%%%%%%%%%%%%%%%%
\subsubsection{Flamingo}
\label{subsub:flamingo}
%%%%%%%%%%%%%%%%%%%%%%%%%%%%%%%%%%%%%%%

Flamingo~\cite{alayrac2022flamingo} is a visual language model (VLM) tackling a wide range of multimodal tasks based on few-shot learning. This is an adaptation of large language models (LLMs) handling an extra visual modality with 80B parameters.

Flamingo consists of three main components: a vision encoder, a Perceiver resampler and a language model.
%
First, to encode images or videos, a vision convolutional encoder~\cite{brock2021nfnet} is pretrained in a contrastive way, using image and text pairs~\footnote{The text is encoded using a pretrained BERT model~\cite{devlin2018bert}.}.
%
Then, inspired by the Perceiver architecture~\cite{jaegle2021perceiver} (detailed in Section~\ref{subsub:cross_attention}), the Perceiver resampler takes a variable number of encoded visual features and outputs a fixed length latent code.
%
Finally, this visual latent code conditions the language model by querying language tokens through cross attention blocks. Those cross-attention blocks are interleaved with pretrained and frozen language model blocks.

The whole model is trained using three different kinds of datasets without annotations (text with image content from webpages~\cite{alayrac2022flamingo}, text and image pairs~\cite{alayrac2022flamingo, jia2021align} and text and video pairs~\cite{alayrac2022flamingo}).
Once the model is trained, it is fine-tuned using few-shot learning techniques to tackle specific tasks.


%%%%%%%%%%%%%%%%%%%%%%%%%%%%%%%%%%%%%%%
\subsection{Video Attention} 
\label{sub:video_attention}
%%%%%%%%%%%%%%%%%%%%%%%%%%%%%%%%%%%%%%%

Video understanding is a long-standing problem, and despite incredible Computer Vision advances, obtaining the best video representation is still an active research area. Videos require employing effective spatio-temporal processing of RGB and time streams to capture long-range interactions~\cite{epstein2021learning,marin2019laeo}, while focusing on important video parts~\cite{nagrani2021attention} with minimum computational resources~\cite{ryoo2021tokenlearner}. 

Typically, video understanding benefits from 2D Computer Vision, by adapting 2D image processing methods to 3D spatio-temporal methods~\cite{hara2018cnn3d}.  
And through the Video Vision Transformer (ViViT)~\cite{arnab2021vivit}, history repeats itself. 
Indeed, with the rise of Transformers~\cite{vaswani2017attention} and the recent advances in image classification~\cite{dosovitskiy2020vit}, video Transformers appear as logical successors of CNNs. 

However, in addition to the computationally expensive video processing, Transformers  also  require a lot of computational resources. Thus, developing efficient spatio-temporal attention mechanisms is essential~\cite{arnab2021vivit,bertasius2021timesformer,jaegle2021perceiver}.

In this section, we first describe the general principle of video Transformers (Section~\ref{subsub:general_principle}), and then, we detail three different attention mechanisms used for video representation (Sections~\ref{subsub:full_space_time_attention}, \ref{subsub:divided_space_time_attention} and \ref{subsub:attention_bottleneck}).

%%%%%%%%%%%%%%%%%%%%%%%%%%%%%%%%%%%%%%%
\subsubsection{General principle}
\label{subsub:general_principle}
%%%%%%%%%%%%%%%%%%%%%%%%%%%%%%%%%%%%%%%

Generally, inputs of video Transformers are RGB video clips $\bm{\mathsfit{X}} \in \mathbb{R}^{F\times H \times W \times 3}$, with $F$ frames of size $H \times W$.

To begin with, video Transformers split the input video clip $\bm{\mathsfit{X}}$ into $ST$ tokens $\bm{x}_i \in \mathbb{R}^K$, where $S$ and $T$ are respectively the number of tokens along the spatial and temporal dimension and $K$ is the size of a token.

To do so, the simplest method extracts non-overlapping 2D patches of size $P\times P$ from each frame~\cite{dosovitskiy2020vit}, as used in TimeSformer~\cite{bertasius2021timesformer}. This results in $S = HW/P^2$, $T=F$ and $K = P^2$.

However, there exist more elegant and efficient token extraction methods for videos. For instance, in ViViT~\cite{arnab2021vivit}, the authors propose to extract 3D volumes from videos (involving $T \neq F$) to capture spatio-temporal information within tokens. In TokenLearner~\cite{ryoo2021tokenlearner}, they propose a learnable token extractor to select the most important parts of the video.

Once raw tokens  $\bm{x}_i$ are extracted, Transformer architectures aim to map them into $d$-dimensional embedding vectors $\bm{Z} \in \mathbb{R}^{ST \times d}$ using a linear embedding $\bm{E} \in \mathbb{R}^{d \times K}$: 
\begin{equation}
    \bm{Z} = [\bm{z}_{cls}, E\bm{x}_1, E\bm{x}_2, \dots, E\bm{x}_{ST}] + \textbf{PE} \quad ,
\end{equation}
where $\bm{z}_{cls} \in \mathbb{R}^{d}$ is a classification token that encodes information from all tokens of a single sample~\cite{devlin2018bert}, and $\textbf{PE} \in \mathbb{R}^{ST \times d}$ is a positional embedding that encodes the spatio-temporal position of tokens, since the subsequent attention blocks are permutation invariant~\cite{vaswani2017attention}.

In the end, embedding vectors $\bm{Z}$ pass through a sequence of $L$ Transformer layers. A Transformer layer $\ell$, is composed of a series of Multi-head Self-Attention (MSA)~\cite{vaswani2017attention}, Layer Normalisation (LN)~\cite{ba2016layernorm} and MLP blocks:
\begin{equation}
\begin{array}{c}
\bm{Y}^\ell = \text{MSA}(\text{LN}(\bm{Z}^\ell)) + \bm{Z}^\ell \quad ,\\
\bm{Z}^{\ell + 1} = \text{MLP}(\text{LN}(\bm{Y}^\ell)) + \bm{Y}^\ell \quad .
\end{array}
\end{equation}

In this way, as shown in Figure~\ref{fig:multi_head_attention}, we denote four different components in a video Transformer layer: the Query-Key-Value (QKV) projection, the MSA block, the MSA projection and the MLP.
For a layer with $h$ heads, the complexity of each component is~\cite{vaswani2017attention}:
\begin{itemize}
    \item \textit{QKV projection}: $\mathcal{O}(h.(2STdd_k + STdd_v)$ 
    \item \textit{MSA}: $\mathcal{O}(hS^2T^2.(d_k + d_v))$ 
    \item \textit{MSA projection}: $\mathcal{O}(SThd_vd)$ 
    \item \textit{MLP}: $\mathcal{O}(STd^2)$ 
\end{itemize}

We note that the MSA complexity is the most impacting component, with a quadratic complexity with respect to the number of tokens. 
Hence, for comprehension and clarity purposes, in the rest of the Section, we consider the global complexity of a video Transformer with $L$ layers to equal to $\mathcal{O}(LS^2T^2)$. 

\begin{figure}[hbtp]
	\centering
	    \includegraphics[width=0.6\textwidth]{figures/full_spacetime_attention_v2.pdf}
	    \caption{\textbf{Full space-time attention mechanism.} 
	    Embedding tokens at layer $\ell - 1$, $\bm{Z}^{(\ell -1)}$ are all fed simultaneously through a unique spatio-temporal attention block. Finally, the spatio-temporal embedding is passed through a MLP and normalized to output embedding tokens of the next layer, $\bm{Z}^{\ell}$.
	    Figure inspired from~\cite{bertasius2021timesformer}. }
	\label{fig:full_space_time_attention}
\end{figure}

%%%%%%%%%%%%%%%%%%%%%%%%%%%%%%%%%%%%%%%
\subsubsection{Full space-time attention}
\label{subsub:full_space_time_attention}
%%%%%%%%%%%%%%%%%%%%%%%%%%%%%%%%%%%%%%%

As described in~\cite{bertasius2021timesformer,arnab2021vivit}, \textit{full space-time attention} mechanism is the most basic and direct spatio-temporal attention mechanism. As shown in Figure~\ref{fig:full_space_time_attention}, it consists in computing self-attention across all pairs of extracted tokens.

This method results in a heavy complexity of $\mathcal{O}(LS^2T^2)$~\cite{bertasius2021timesformer,arnab2021vivit}. This quadratic complexity can fast be memory consuming, which it is especially true when considering videos.
Therefore, using full space-time attention mechanism is  impractical~\cite{bertasius2021timesformer}.  

\begin{figure}[hbtp]
	\centering
	    \includegraphics[width=0.6\textwidth]{figures/divided_spacetime_attention_v2.pdf}
	    \caption{\textbf{Divided space-time attention mechanism.} 
	    Embedding tokens at layer $\ell - 1$, $\bm{Z}^{(\ell -1)}$ are first processed along the temporal dimension through a first MSA block, and the resulting tokens are processed along the spatial dimension. Finally, the spatio-temporal embedding is passed through a MLP and normalized to output embedding tokens of the next layer, $\bm{Z}^{\ell}$.
	    Figure inspired from~\cite{bertasius2021timesformer}.}
	\label{fig:divided_space_time_attention}
\end{figure}

%%%%%%%%%%%%%%%%%%%%%%%%%%%%%%%%%%%%%%%
\subsubsection{Divided space-time attention}
\label{subsub:divided_space_time_attention}
%%%%%%%%%%%%%%%%%%%%%%%%%%%%%%%%%%%%%%%

A smarter and more efficient way to compute spatio-temporal attention is the \textit{divided space-time attention} mechanism, first described in~\cite{bertasius2021timesformer}.

As shown in Figure~\ref{fig:divided_space_time_attention}, it relies on computing spatial and temporal attention separately in each Transformer layer. Indeed, we first compute the spatial attention, i.e., self-attention within each temporal index, and then, the temporal attention, i.e., self-attention across all temporal indices.

The complexity of this attention mechanism is $\mathcal{O}(LST.(S+T))$~\cite{bertasius2021timesformer}.
By separating the calculation of the self-attention over the different dimensions, one tames the quadratic complexity of the MSA module. This mechanism highly reduces the complexity of a model with respect to the full space-time complexity. Therefore, it is reasonable to use it to process videos~\cite{bertasius2021timesformer}.




\begin{figure}[hbtp]
	\centering
	    \includegraphics[width=0.6\textwidth]{figures/bottleneck_attention_v2.pdf}
	    \caption{\textbf{Attention bottleneck mechanism.} 
	    Raw input patches and embedding tokens at layer $\ell - 1$, $\bm{Z}^{(\ell -1)}$ are fed to a cross attention block (CA), then normalized and projected. Finally, the resulting embedding is passed through a transformer to output embedding tokens of the next layer, $\bm{Z}^{\ell}$.
	    Figure inspired from~\cite{jaegle2021perceiver}.}
	\label{fig:attention_bottleneck}
\end{figure}

%%%%%%%%%%%%%%%%%%%%%%%%%%%%%%%%%%%%%%%
\subsubsection{Cross-attention bottlenecks}
\label{subsub:attention_bottleneck}
%%%%%%%%%%%%%%%%%%%%%%%%%%%%%%%%%%%%%%%

An even more refined way to reduce the computational cost of attention calculation, consists of using cross-attention as a bottleneck. For instance, as shown in Figure~\ref{fig:attention_bottleneck} and mentioned in Section~\ref{sub:compute_efficiency}, the Perceiver~\cite{jaegle2021perceiver} projects the extracted tokens $\bm{x}_i$ into a very low-dimensional embedding through a cross-attention block placed before the Transformer layers. 

Here, the cross attention block placed before the $L$ Transformer layers reduces the input dimension from $ST$ to $N$, where $N \ll ST$~\footnote{In practice, $N \le 512$ for a Perceiver~\cite{jaegle2021perceiver}, against $ST = 16 \times 16 \times (32 / 2) = 4,096$ for a ViViT-L~\cite{arnab2021vivit}}, thus resulting in a complexity of $\mathcal{O}(STN)$. Hence, the total complexity of this attention block is $\mathcal{O}(STN + LN^2)$.
It reduces again the complexity of a model with respect to the \textit{divided space-time attention} mechanism. We note that it enables to design deep architectures, as in the Perceiver~\cite{jaegle2021perceiver}, and then, it enables the extraction of higher level features.



\begin{figure}[hbtp]
	\centering
	    \includegraphics[width=\textwidth]{figures/factorised_encoder.pdf}
	    \caption{\textbf{Factorized encoder mechanism.} 
	    First, a spatial Transformer processes input tokens along the spatial dimension, Then, a temporal Transformer processes the resulting spatial embedding along the temporal dimension. 
	    Figure inspired from~\cite{jaegle2021perceiver}.}
	\label{fig:factorised_encoder}
\end{figure}

%%%%%%%%%%%%%%%%%%%%%%%%%%%%%%%%%%%%%%%
\subsubsection{Factorised encoder}
\label{subsub:factorised_encoder}
%%%%%%%%%%%%%%%%%%%%%%%%%%%%%%%%%%%%%%%

Lastly, the \textit{factorised encoder}~\cite{arnab2021vivit} architecture is the most efficient with respect to the complexity/performance trade-off.

As in \textit{divided space-time attention}, the \textit{factorised encoder} aims to compute spatial and temporal attention separately. Nevertheless, as shown in Figure~\ref{fig:factorised_encoder}, instead of mixing spatio-temporal tokens in each Transformer layer, here there exist two separate encoders. First, a representation of each temporal index is obtained thanks to a spatial encoder with $L_s$ layers, then, these tokens are passed through a temporal encoder with $L_t$ layers (i.e. $L = L_s + L_t$).

Hence, the complexity of a such architecture, has two main components: the spatial encoder complexity of $\mathcal{O}(L_sS^2)$, and the temporal encoder complexity of $\mathcal{O}(L_tT^2)$. It results in a global complexity of $\mathcal{O}(L_sS^2 + L_tT^2)$.
Thus, it leads to very lightweight models. However, as it first extracts per-frame features, and then aggregates them to a final representation, it corresponds to a late fusion mechanism, which can sometimes be a drawback as it does not mix spatial and temporal information simultaneously~\cite{nagrani2021bottlenecksmultimodalfusion}.

%%%%%%%%%%%%%%%%%%%%%%%%%%%%%%%%%%%%%%%
\subsection{Video Transformers} 
\label{sub:video_transformers}
%%%%%%%%%%%%%%%%%%%%%%%%%%%%%%%%%%%%%%%

In this section, we present two modern Transformer-based architectures for video classification.
We start by introducing the TimeSformer architecture in Section~\ref{subsub:timesformer}, and then the ViViT architecture in Section~\ref{subsub:vivit}.

%%%%%%%%%%%%%%%%%%%%%%%%%%%%%%%%%%%%%%%
\subsubsection{TimeSformer}
\label{subsub:timesformer}
%%%%%%%%%%%%%%%%%%%%%%%%%%%%%%%%%%%%%%%

\begin{figure}[hbtp]
	\centering
	    \includegraphics[width=0.4\textwidth]{figures/timesformer_v2.pdf}
	    \caption{\textbf{TimeSformer architecture.} 
	    The TimeSformer first projects input to embedding tokens, which are summed to positional embedding tokens. The resulting tokens are then passed through $L$ divided space-time attention blocks and then linearly projected to obtain output probabilities.}
	\label{fig:timesformer}
\end{figure}

TimeSformer~\cite{bertasius2021timesformer} is one of the first architectures with space-time attention that impacted the video classification field. It follows the same structure and principle described in Section~\ref{subsub:general_principle}.

First, it takes as input an RGB video clip sampled at a rate of $1/32$ and decomposed into 2D $16 \times 16$ patches. 

As shown in Figure~\ref{fig:timesformer}, the TimeSformer architecture is based on the ViT architecture (Section~\ref{sub:vit}), with $12$ $12$-headed MSA layers. However, the added value compared to the ViT is that TimeSfomer uses the \textit{divided space-time attention} mechanism (Section~\ref{subsub:divided_space_time_attention}). Such attention mechanism enables to capture high-level spatio-temporal features, while taming the complexity of the model. 
%
Moreover, the authors introduce three variants of the architecture: (i) TimeSformer, the standard version of the model, that operates on $8$ frames of $224 \times 224$; (ii) TimeSformer-L, a configuration with high spatial resolution, that operates on $16$ frames of $448 \times 448$; and (iii) TimeSformer-HR, a long temporal range setup, that operates on $96$ frames of $224 \times 224$.

Finally, the terminal classification token embedding is passed through an MLP to output a probability for all video classes. During inference, the final prediction is obtained by averaging the output probabilities from three different spatial crops of the input video clip (top-left, centre and bottom-right). 

TimeSformer achieves similar state-of-the-art performances as the 3D CNNs~\cite{feichtenhofer2019slowfast,carreira2017i3d} on various video classification datasets, such as Kinetics-400 and Kinetics-600~\cite{kay2017kinetics}. Note, the TimeSformer is much faster to train (416 training hours against 3840 hours~\cite{bertasius2021timesformer} for a SlowFast architecture~\cite{feichtenhofer2019slowfast}), and also, more efficient ($0.59$ TFLOPs against $1.97$ TFLOPs~\cite{bertasius2021timesformer} for a SlowFast architecture~\cite{feichtenhofer2019slowfast}). 

%%%%%%%%%%%%%%%%%%%%%%%%%%%%%%%%%%%%%%%
\subsubsection{ViViT}
\label{subsub:vivit}
%%%%%%%%%%%%%%%%%%%%%%%%%%%%%%%%%%%%%%%

\begin{figure}[hbtp]
	\centering
	    \includegraphics[width=0.4\textwidth]{figures/vivit_v2.pdf}
	    \caption{\textbf{ViViT architecture.}
	    The ViViT first projects input to embedding tokens, which are summed to positional embedding tokens. The resulting tokens are first passed through $L_s$ spatial attention blocks and then through $L_t$ temporal attention blocks. The resulting output is linearly projected to obtain output probabilities.
	    }
	\label{fig:vivit}
\end{figure}

ViViT~\cite{arnab2021vivit} is the main extension of the ViT~\cite{dosovitskiy2020vit} architecture (Section~\ref{sub:vit})  for video classification.

First, the authors use a $16 \time 16 \time 2$ tubelet embedding instead of a 2D patch embedding, as mentioned in Section~\ref{subsub:general_principle}. This alternate embedding method aims to capture the spatio-temporal information from the tokenization step, unlike standard architectures that fuse spatio-temporal information from the first attention block.

As shown in Figure~\ref{fig:vivit}, the ViViT architecture is based on \textit{factorised encoder} architecture (Section~\ref{subsub:factorised_encoder}), and consists of one spatial and one temporal encoder operating on input clips with $32$ frames of $224 \times 224$. 
The spatial encoder uses one of the three ViT variants as backbone~\footnote{ViT-B: $12$ $12$-headed MSA layers; ViT-L: $24$ $16$-headed MSA layers; and ViT-H: $32$ $16$-headed MSA layers.}. For the temporal encoder, the number of layers does not impact much the performance, so that, according to the performance/complexity trade-off, the number MSA layers is fixed at $4$. 
The authors show that such architecture reaches high performances, while reducing drastically the complexity.

Finally, as in TimeSformer (Section~\ref{subsub:timesformer}), ViViT outputs probabilities for all video classes through the last classification token embedding, and averages the obtained probabilities across three crops of each input clip (top-left, centre and bottom-right).

ViViT outperforms both 3D CNNs~\cite{feichtenhofer2019slowfast,carreira2017i3d} and TimeSformer~\cite{bertasius2021timesformer} on the Kinetics-400 and Kinetics-600 datasets~\cite{kay2017kinetics}. 
Note, the complexity of this architecture is highly reduced in comparison to other state-of-the-art models. For instance, the number of FLOPs for a ViViT-L/$16 \times 16 \times 2$ is $3.89 \times 10^{12}$, against $7.14 \times 10^{12}$ for a TimeSformer-L~\cite{bertasius2021timesformer}, and $7.14 \times 10^{12}$ for a SlowFast~\cite{feichtenhofer2019slowfast} architecture.

%%%%%%%%%%%%%%%%%%%%%%%%%%%%%%%%%%%%%%%
\subsection{Multimodal video Transformers} 
\label{sub:multi_modal}
%%%%%%%%%%%%%%%%%%%%%%%%%%%%%%%%%%%%%%%

Nowadays, one of the main gaps between artificial and human intelligence is the ability for us to process multimodal signals and to enrich the analysis by mixing the different modalities. Moreover, until recently, deep learning models have been focusing mostly on very specific visual tasks, typically based on a single modality, such as image classification~\cite{deng2009imagenet,krizhevsky2009cifar,dosovitskiy2020vit,wu2021cvt,touvron2021cait}, audio classification~\cite{gemmeke2017audioset,gong2021ast,nagrani2021bottlenecksmultimodalfusion,jaegle2021perceiver} and machine translation~\cite{bojar2014wmt,devlin2018bert,liu2020transformerBT,edunov2018backtranslation,lin2020pre}. These two factors combined have pushed researchers to take up multimodal challenges.

The \textit{default} solution for multimodal tasks consists in first creating an individual model (or network) per modality, and then in fusing the resulting single-modal features together~\cite{owens2018audio,ramanishka2016multimodal}.  
Yet, this approach fails to model interactions or correlations among different modalities.
However, the recent rise of attention~\cite{vaswani2017attention,dosovitskiy2020vit,arnab2021vivit} is promising for multimodal applications, since attention performs very well at combining multiple inputs~\cite{chen2022mmvit,jaegle2021perceiver,nagrani2021bottlenecksmultimodalfusion,narasimhan2021clipit}.
\\

Here, we present two main ways of dealing with several modalities: 

\paragraph{1. Concatenating tokens from different modalities into one vector~\cite{chen2022mmvit,jaegle2021perceiver}.}
The multimodal Video Transformer (MM-ViT)~\cite{chen2022mmvit} combines raw RGB frames, motion features and audio spectrogram for video action recognition. To do so, the authors fuse tokens from all different modalities into a single input embedding and pass it through Transformer layers. 
However, a drawback of this method is that it fails to distinguish well one modality to another. 
To overcome this issue, the authors of the Perceiver~\cite{jaegle2021perceiver} propose to learn a modality embedding in addition to the positional embedding (see Sections~\ref{sub:compute_efficiency} and \ref{subsub:general_principle}). This allows associating each token with its modality. 
Nevertheless, given that (i) the complexity of a Transformer layer is quadratic with respect to the number of tokens  (Section~\ref{subsub:general_principle}), and (ii) with this method, the number of tokens is multiplied by the number of modalities: it may lead to skyrocketing computational cost~\cite{chen2022mmvit}.

\paragraph{2. Exploiting cross attention~\cite{nagrani2021bottlenecksmultimodalfusion,narasimhan2021clipit,zadeh2019factorized}.}
Several modern approaches exploit cross attention to mix multiple modalities, such as~\cite{nagrani2021bottlenecksmultimodalfusion} for audio and video, \cite{narasimhan2021clipit} for text and video, and \cite{zadeh2019factorized} for audio, text and video. 
The commonality among all these methods is that they exploit the intrinsic properties of cross attention by querying one modality with a key-value pair from the other one~\cite{nagrani2021bottlenecksmultimodalfusion,narasimhan2021clipit}. 
This idea can be easily generalized to more than two modalities by computing cross-attention across each combination of modalities~\cite{zadeh2019factorized}.



\section{Conclusion} 
\label{sec:conclusion}
\vspace{-3pt}
\section{Conclusion}
\label{sec:conclusion}
\vspace{-2pt}

In this work, we propose to reduce the size of Conformer-based models through parameter weight reuse at four levels: (i) repeating conformer block layer transformations, (ii) sharing specific conformer modules, (iii) sharing or not sharing sub-components per conformer module, and (iv) sharing low-rank decomposed sub-weights. 
By sharing model weight across layers, we find that we can increase the number of virtual transformations of our input data without further increasing the size of our model and thus we can retain our model in-memory for always-on ambient ASR leveraging low-power and low-resource neural accelerators such as edge TPU hardware.
Through our evaluations, we find that sharing model weights and applying a low-rank Conformer architecture (LRS3) offers the greatest performance for our $5$M parameter models, achieving a WER of $2.84$ and $2.98$ for LibriSpeech dev-clean and test-clean respectively. 



% References section
\bibliographystyle{spbasicsort}
\bibliography{longstrings,main}

\end{document}