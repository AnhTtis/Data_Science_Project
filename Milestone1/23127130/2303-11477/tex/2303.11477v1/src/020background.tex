Deep learning based generative models for histopathology images have seen tremendous progress in recent years due to advances in digital pathology, compute power, and neural network architectures. Several GAN-based generative models have been proposed to generate histology patches~\cite{levine2020synthesis, xue2021selective, zhou2022u}. However, GANs suffer from problems of frequent mode collapse and overfitting their discriminator~\cite{xiao2021tackling}. It is also challenging to capture long-tailed distributions and synthesize rare samples from imbalanced datasets using GANs. More recently, denoising diffusion models have been shown to generate highly compelling images by incrementally adding information to noise~\cite{ho2020denoising}. Success of diffusion models in generating realistic images led to various conditional~\cite{ kawar2022denoising, saharia2022palette, saharia2022image} and unconditional~\cite{dhariwal2021diffusion, ho2022cascaded, nichol2021improved} diffusion models that generate realistic samples with high fidelity. Following this, a morphology-focused diffusion model has been presented for generating tissue patches based on genotype~\cite{moghadam2023morphology}. Semantic image synthesis is a task involving generating diverse realistic images from semantic layouts. GAN-based semantic image synthesis works~\cite{tan2021diverse, tan2021efficient, park2019semantic} generally struggled at generating high quality and enforcing semantic correspondence at the same time. To this end, a semantic diffusion model has been proposed that uses conditional denoising diffusion probabilistic model and achieves both better fidelity and diversity~\cite{wang2022semantic}. We use this progress in the field of conditional diffusion models and semantic image synthesis to formulate our NASDM framework.





% Generating Histopathology images has been getting popular in recent years because of advances in digital pathology
% imaging and computational infrastructures as well as the introduction of powerful deep generative models that are able
% to tackle specific difficulties in this domain.