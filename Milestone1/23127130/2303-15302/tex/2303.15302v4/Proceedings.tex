\documentclass[a4paper,11pt]{article}
\pdfoutput=1

\usepackage{jinstpub}

\usepackage{lineno}
%\linenumbers

\usepackage{siunitx}

\graphicspath{{Figs/}}

%%Title
\title{Techniques for mass production of large-sized GEM foil by the Korean CMS group for CMS phase-2 upgrade}

%%Author
\author{I. Yoon, \note{Corresponding author.}}
\affiliation{Seoul National University,\\1 Gwanak-ro, Gwanak-gu, Seoul 08826, Republic of Korea}

%%E-mail
\emailAdd{inseok.yoon@cern.ch}

%%Abtract
\abstract{
    %Production techniques for mass production of large-sized GEM foils at Korea for CMS phase-2 upgrade is presented. 
    %The foil production facility is designed a focus on mass production such as adopting double-mask technique.
    %A polyimide wet etching technology using mono ethanolamine is reported. 
    %The use of mono ethanolamine, which has a lower inhalation toxicity compared to ethylene diamine, can create a safer working environment. 
    %The denaturation of the etchant over time and the process of retuning are presented. 
    %R\&D results on soldering SMD resistors with hot air for faster production is discussed.
  This study presents techniques for the mass production of large-sized GEM foils for the CMS phase-2 upgrade by the Korean CMS group. 
  The foil production facility is designed with a focus on mass production, including the adoption of the double-mask technique. 
  A polyimide wet etching technology that uses mono ethanolamine is reported, providing a safer working environment due to its lower inhalation toxicity compared to ethylene diamine. 
  The study also covers the denaturation of the etchant over time and the process of retuning. 
  Finally, R\&D results on soldering surface mount resistors with hot air for faster production are discussed.
}

%%Collaboration
\collaboration[c]{on behalf of CMS Muon Group}

%%Keywords
\keywords{GEM, CMS, Double-mask technique, Mass production}

%%Proceedings
\proceeding{The 7$^{th}$ International Conference on Micro Pattern Gaseous Detectors 2022\\
  December 11-16, 2022\\
  Weizmann Institute of Science, Rehovot, Israel}

\begin{document}
\maketitle
\flushbottom

\section{GEM foil production by the Korean CMS group}
\label{sec:overview}
%For the luminosity upgrade of the large hadron collider (HL-LHC), several upgrades of CMS detector is ongoing.
%The installation of three detector stations based on gas electron multiplier (GEM) technology is one of them.
%These stations, order of their distance from the interaction point, are called ME0, GE1/1, and GE2/1.
%The installation of the GE1/1 stations is complete and commissioning is ongoing.
%Production and assembly of GE2/1 detectors has begun and will be followed by production for ME0.
Several upgrades of the CMS detector are ongoing for the luminosity upgrade of the Large Hadron Collider (HL-LHC).
One of these upgrades is the installation of three detector stations based on the Gas Electron Multiplier (GEM) technology, known as the ME0, GE1/1, and GE2/1 stations, in order of their distance from the interaction point.
The production and the installation of the GE1/1 system was completed in 2020.
Production and assembly of the GE2/1 detectors has begun and will be followed by the production of the ME0 detectors.

%The Korean CMS group (KCMS) and MECARO Co., Ltd., a Korean company producing components and materials for semiconductor production, had formed a consortium that designates the KCMS as the second supplier of large-sized GEM foils for the CMS GEM upgrades.
%As GEM technology had garnered attention from experimental high energy and nuclear physics community, it became harder for CERN Micro Pattern Technologies workshop (MPT) alone to satisfy the growing demend.
%The KCMS became the second supplier to contribute toward the CMS upgrade to relax the situation which affects the upgrades as well.
%It is in charge of producing GEM foils equivalent to half of GE2/1 and all of ME0 stations.
%The quality validation results of GEM foils produced by KCMS is shown by ref.~\cite{ICHEP2018}.
%So far 285 GE2/1 foils have been produced by KCMS and passes QC criteria established by the CMS GEM upgrade projects \cite{QC_Criteria}.

%Due to the growing demand for GEM technology from the experimental high energy and nuclear physics communities, the Korean CMS group (KCMS) and MECARO Co., Ltd., a Korean company that produces components and materials for semiconductor production, formed a consortium.
%CERN Micro Pattern Technologies workshop (MPT) alone can no longer satisfy the increasing demand.
%KCMS was designated as the second supplier of large-sized GEM foils for the CMS GEM upgrades, and it will contribute to the upgrade by producing GEM foils equivalent to half of the GE2/1 and all of the ME0 stations.
%The shortage of GEM foils is a hurdle for the CMS upgrade and KCMS's contribution will help alleviate this issue.
%So far, KCMS has produced 285 GE2/1 foils that have passed the quality control criteria established by the CMS GEM upgrade projects.% \cite{QC_Criteria}.

%Due to the increasing demand for GEM technology from the experimental high energy and nuclear physics communities, it may become difficult for the CERN Micro Pattern Technologies workshop (MPT) to keep up with the growing needs alone. 
%Therefore, to help address this demand, the Korean CMS group (KCMS) has been designated as the second supplier of large-sized GEM foils for the CMS GEM upgrades. 
The Korean CMS group (KCMS), an association of Korean research institutes participating in the CMS experiment, has been designated as the second supplier of large-sized GEM foils together with the CERN Micro Pattern Technologies workshop (MPT) for the CMS GEM upgrades.
KCMS is contributing significantly to the upgrade by producing GEM foils equivalent to half of the GE2/1 and all of the ME0 stations.
So far, KCMS has produced 292 GE2/1 foils that have passed the quality control (QC) criteria established by the CMS GEM upgrade projects.

KCMS formed a consortium with Mecaro Co., Ltd. for the production of the foil, but after the consortium ended, the equipment for the production is being moved to the Institute of Basic Science (IBS - South Korea) for the production of ME0 foils.

\section{Production processes}
%GEM foil production facilities of KCMS are designed for fast production as KCMS is in charge of mass production.
%The very first step of GEM foil production is laminating dry film resist (DFR) to flexible copper clad liminate (FCCL).
%Both side of FCCL are liminated with DFR.
%Particularly different parts of GEM foil production technology of KCMS from CERN MPT will be discussed in below.
KCMS's GEM foil production facilities have been optimized for faster mass production.
In the next section, we will discuss the specific aspects of KCMS's GEM foil production technology that distinguish it from that of CERN MPT.

\subsection{Bipolar photolithograpy}
%KCMS adopts the double-mask technique while CERN MPT uses the single-mask technique to produce large-sized GEM foils \cite{Pinto}.
%By choosing it, production processes become simpler and faster production is possible comparing to using the single-mask technique.
%However, mask align becomes very crucial.
%Left of Figure~\ref{fig:facilities} shows large-size bipolar UV exposure machine which not only transcribes pattern but also aligns top and bottom-side of masks.
%The machine can align masks with misalignment less than \SI{5}{\micro\meter} while the maximum allowed misaligment is \SI{7}{\micro\meter} (\SI{10}{\percent} of hole diameter) for the optimum operation.
%Emulsion glass masks are necessary for the alignment.
%The machine can be used to form pattern of maximum \SI{125x58}{\centi\meter\squared} area.
KCMS uses the double-mask technique for producing large-sized GEM foils, whereas CERN MPT is focusing the single-mask GEM foil production only \cite{Pinto}.
The definitions of the double and single-mask technique are described in \cite{Pinto}.
The double-mask technique simplifies the production process and enables faster production than the single-mask technique.
However, mask alignment becomes crucial.
The left side of figure~\ref{fig:facilities} shows a large-size bipolar UV exposure machine that not only transcribes patterns but also aligns the top and bottom sides of the masks.
%The exposure machine has five built-in microscopes which are used for the alignment in four corners and center.
The machine can align masks with a misalignment of less than \SI{5}{\micro\meter}, while the maximum allowed misalignment for optimal operation is \SI{7}{\micro\meter} (\SI{10}{\percent} of the hole diameter).
Emulsion glass masks are required for the alignment process.
The machine can form patterns of up to \SI{125}{\centi\meter}$\,\times\,$\SI{58}{\centi\meter} in size.

\begin{figure}[htbp]
	\centering
	\includegraphics[width=.25\textwidth]{UV_Exposure.png}
	\qquad
	\includegraphics[width=.25\textwidth]{Developer.png}
        %\vspace*{-2mm}
	\caption{\label{fig:facilities} (Left) A large-sized bipolar UV exposure. (Right) A pattern developer and Cu etcher.}
\end{figure}

\subsection{Development and copper etching}
%Right of Figure~\ref{fig:facilities} shows pattern developer and copper (Cu) etcher.  
%For faster production, the equipments are integrated and have a conveyor belt built in.
%When flexible Cu clad laminate (FCCL) that has completed the photo process is put in, it reacts with the reactant sprayed while moving at a predefined speed on the conveyor belt.
%The diameter of the developed hole can be adjusted by the moving speed of the belt, the concentration and spray pressure of the reactant.
On the right side of figure~\ref{fig:facilities}, the pattern developer and copper (Cu) etcher machines are integrated, with a built-in conveyor belt to enable faster production.
When the flexible Cu clad laminate (FCCL) that has completed the photo process is inserted, it reacts with the reactant sprayed while moving on the conveyor belt.
The diameter of the developed hole can be adjusted by the moving speed of the belt, the concentration, and spray pressure of the reactant.

\subsection{Dry film resist stripping}
%To strip dry film resist (DFR), the processed FCCL is manually dipped into bath filled with sodium hydroxide (NaOH). 
%During the strip process, the progess should be checked continuously unless NaOH may damage Cu layer.
%NaOH is chosen for faster reaction.
To strip the dry film resist, the processed FCCL is manually dipped into a bath filled with sodium hydroxide (NaOH).
%During the stripping process, the progress should be checked continuously to avoid damage to the Cu layer by NaOH.
NaOH is chosen for its fast reaction rate.

\subsection{Polyimide etching}
%Polyimide (PI) etching is done by manually dipping processed FCCL to bach filled with etchant. 
%PI etchant is mixture of potassium hydroxide (KOH) and mono ethanolamine (MEA).
%Similarly to ethylenediamine (EDA) \cite{Pinto}, MEA shows anisotropic etching properties.
%By adjusting the ratio of MEA and KOH, which is isotropic etchant, geometry of developed hole is tunable.
%The geometry is not too sensitive to duration of etching. 
%MEA is chosen for safer working condition.
%Acute lethal concentration (LC) for mice exposed to MEA vapor is greater than \SI{2430}{\milli\gram\per\cubic\meter} \cite{PubChem_MEA} while median LC of EDA vapor is \SI{300}{\milli\gram\per\cubic\meter} \cite{PubChem_EDA}.
%Etchant should be tuned before use because etchant exposed to air is denatured slowly by reaction with carbon dioxide.
%Etchant can be calibrated through a few sample etching.  
%Figure~\ref{fig:cross_section} shows cross sectional view of produced GEM foil.
%PI hole with proper  geometry is developed. 
%It is confirmed that mask alignment is in acceptable range. 
Polyimide (PI) etching is done by manually dipping the processed FCCL into a bath filled with an etchant mixture of potassium hydroxide (KOH) and monoethanolamine (MEA).
Like ethylenediamine (EDA) \cite{Pinto}, MEA exhibits anisotropic etching properties .
By adjusting the ratio of MEA and KOH, which is an isotropic etchant, the geometry of the developed hole can be tuned.
The geometry is not too sensitive to the duration of etching.
MEA is chosen for a safer working condition because the acute lethal concentration (LC) for mice exposed to MEA vapor is greater than \SI{2430}{\milli\gram\per\cubic\meter} \cite{PubChem_MEA}, while the median LC of EDA vapor is \SI{300}{\milli\gram\per\cubic\meter} \cite{PubChem_EDA}.
The etchant should be calibrated before use because the etchant exposed to air is slowly denatured by reacting with carbon dioxide.
The etchant can be calibrated through a few sample etchings.

Figure~\ref{fig:cross_section} shows the cross-sectional view of the produced GEM foil with a PI hole of proper geometry, indicating acceptable mask alignment.
In this case, the residual misalignment is estimated to be at the level of \SI{3}{\micro\meter}, which satisfies the prescribed value of \SI{7}{\micro\meter}.

\begin{figure}[htbp]
	\centering
	\includegraphics[width=.3\textwidth]{Cross_Section.png}
	\caption{\label{fig:cross_section}
		A cross-sectional view of the produced GEM foil, where the blue lines were added by the author to enhance the visibility of the PI edge after taking the SEM image.}
\end{figure}

\section{Low-cost soldering technique for mass production}
%GEM foils usually equips surface mount resistors for high voltage stability.
%Soldering the small resistors to FCCL is demanding because the resistors are easily pushed out by soldering tips.
%In addition, excessive solder may reduce gap between foils and cause sparks.
GEM foils typically require surface mount resistors for high voltage protection.
However, soldering these small resistors onto FCCL can be challenging as they are easily pushed out by soldering tips, and excessive solder may reduce the gap between foils and cause sparks.

%KCMS invented low cost soldering technique using cream solder dispenser, and hot air rework station.   
%The air pressure-operated solder dispenser deploys uniform and accurate amount of cream solder.
%Jig helps align the resistors and prevents GEM foils from shaking. 
%With the technique, the solder process, which used to be a bottle neck for GEM foil production, can be done more quickly and accurately with less working fatigue. 
To address these challenges, KCMS has developed a low-cost soldering technique using a cream solder dispenser and hot air rework station.
The air pressure-operated solder dispenser deploys a uniform and accurate amount of cream solder, and a jig helps align the resistors and prevent GEM foils from sagging during the soldering process.
With this technique, the soldering process can be completed more quickly and accurately with less work fatigue.

%\begin{figure}[htbp]
%	\centering
%	\includegraphics[width=.75\textwidth]{Solder.png}
%	\caption{\label{fig:soldering}
%		%Photographs explaining the soldering process. In the picture, the silver object covering GEM foils is jig.
%		These photographs illustrate the processes used for soldering resistors to GEM foils, with a silver jig seen covering the foils in the images.
%	}
%\end{figure}

\section{Validation of the production techniques}
%The validation of the production techniques was conducted by assembling GEM detectors with the foils produced by the new vendor and measuring characteristics such as gain, gain uniformity, etc.
The validation of the production techniques was conducted by assembling GEM detectors with the foils produced by the new vendor and measuring characteristics such as gain, gain uniformity, rate capability, aging property and other performance metrics \cite{ICHEP18_Validation}.
%In addition, the techniques were verified again through the results of the mass production as described in section.~\ref{sec:overview}.
In addition, the techniques were verified again through the results of the
mass production for the CMS upgrades, described in section~\ref{sec:overview}.

\acknowledgments{
	%We would like to thank Dr. Rui De Oliveira (CERN) for providing invaluable assistance.
	%We are grateful to Taeseong Jeong (Mecaro) and Inseung Jeong (Mecaro) for their support.
	%This study is supported by National Research Foundation of Korea.
	We would like to express our gratitude to Dr. Rui De Oliveira (CERN) for his invaluable assistance.
	We also extend our thanks to Taeseong Jeong and Inseung Jeong (Mecaro) for their support.
	This study was supported by the National Research Foundation of Korea
}

\begin{thebibliography}{99}
	%\bibitem{ICHEP2018}
	%I. Yoon, \emph{Results of quality control of large size GEM detectors based on the Korean GEM foil for the future upgrades of the CMS muon system}, \emph{PoS ICHEP2018} {\bf 133} (2019).

	%\bibitem{QC_Criteria}
	%D. Abbaneo et al, \emph{Quality control and beam test of GEM detectors for future upgrades of the CMS muon high rate region at the LHC}, \emph{JINST} {\bf 10} (2015) C03039.

	\bibitem{Pinto}
	S Duarte Pinto et al, \emph{Progress on large area GEMs}, \emph{JINST} {\bf 4} (2009) P12009.

	\bibitem{PubChem_MEA}
	NCBI, \emph{PubChem Compound Summary for CID 700, Ethanolamine}.%, \url{https://pubchem.ncbi.nlm.nih.gov/compound/Ethanolamine}, Accessed Feb. 22, 2023.

	\bibitem{PubChem_EDA}
	NCBI, \emph{PubChem Compound Summary for CID 3301, Ethylenediamine}.%, \url{https://pubchem.ncbi.nlm.nih.gov/compound/Ethylenediamine}, Accessed Feb. 22, 2023.

	\bibitem{ICHEP18_Validation}
	CMS collaboration, I. Yoon, \emph{Results of quality control of large size GEM detectors based on the Korean GEM foil for the future upgrades of the CMS muon system} in \emph{Proceedings of the ICHEP2018}, 2019
	%\bibitem{Jeremie_Double_Segmented}
	%J.A. Merlin, \emph{Study of discharges and their effects in GEM detectors}, \emph{MPGD19}, (2019), \url{https://indico.cern.ch/event/757322/contributions/3396501/
	%} 

\end{thebibliography}

\end{document}
