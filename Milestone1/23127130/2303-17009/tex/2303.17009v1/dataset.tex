%\documentclass{article}

%\begin{document}

\subsection{Datasets}
\label{Sec:datasets}
To conduct our experiments several datasets of histological images were collected. Whole slide images (WSIs) were acquired with a Zeiss AxioScan scanner (Carl 
Zeiss, Jena, Germany) with a $20 \times$ objective at a resolution of 0.221 \textmu m/{pixel} from mouse liver tissue samples stained with H\&E and MT according to established protocols.
The WSI were then subsampled with a factor of 1:2, which resulted in a 0.442 \textmu m/{pixel} resolution.
The \textit{training dataset} consists of around $26000$ $256 \times 256$ tiles, extracted from WSIs, for each of the two types of staining. 
This dataset is used to train I2I methods in both directions.

We collected a disjoint \textit{validation dataset} of around 1300 image tiles for each type of staining. The tiles were extracted from the WSIs used for the training dataset.
Therefore, even though there is no overlap between tiles of the training and validation sets, both datasets are coming from the same distribution. 
%The validation dataset is used for performance evaluation of image translation methods when distribution of input images corresponds to the distribution of the training dataset.
We also collected a \textit{test dataset} of around 1300 image tiles for each type of staining extracted from WSIs originating from different histological studies so that the image tiles' distribution is different from the training dataset. 
%We will use this dataset for performance evaluation testing the robustness of I2I translation methods when processing images with shifted distribution relatively to the training data.

  

%We additionally collected \textit{a fibrotic tissue dataset} for testing performance of a classifier, which was trained to assess this condition in MT stained tissue images, when evaluating it on artificially created $H\&E \rightarrow MT$ images.
%This dataset contains ... tiles extracted from ... WSIs of tissue stained with H\&E and Masson's trichrome procedures. Close tissue cuts (3-4\textmu m) gave us closely matching images stained with two different procedures, which is needed for this experiment with assesment of classificatin performance. Similarly to the first  validation dataset, the WSIs come from histopathological study different to the training data.  


%    A dataset with images of both stains for training Deep Learning models was created for training models.
%    Each domain has approximately 26k patches of tissues that come from WSI. Each patch has a resolution of $256 \times 256$ pixels.
%    An additional dataset with approximately 1.3k patches was created for testing.     
%   $ l_1 (u, v) = \inf_{\pi \in \Gamma (u, v)} \int_{\mathbb{R} \times
%        \mathbb{R}} |x-y| \mathrm{d} \pi (x, y)$
%
%    where :math:$\Gamma (u, v)$ is the set of (probability) distributions on
%    :math:$\mathbb{R} \times \mathbb{R}$ whose marginals are :math:$u$ and
%    :math:$v$ on the first and second factors respectively.
%
%    If :math:$U$ and :math:$V$ are the respective CDFs of :math:$u$ and
%    :math:$v$, this distance also equals to:
%
%    .. math::
%
%        $l_1(u, v) = \int_{-\infty}^{+\infty} |U-V|$
%\end{document}