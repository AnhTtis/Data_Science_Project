\section{Robustness of computer-aided grading of tissue to artificial images}
\label{Sec:NASH_classification}

A growing number of tasks in quantitative tissue analysis, such as disease grading, are being performed with the aid of machine learning systems. We  investigated whether artificial images can be utilized by such systems. This would allow to apply such systems when the stain used for training is not readily available. We use a deep-learning-based system \cite{heinemann2019deep} that was trained on MT stained tissues to replace pathologist grading of non-alcoholic fatty liver disease, which includes a quantification of liver inflammation. The condition, if exists, is typically spread homogeneously over tissue. The system was fed with artificially created $\mbox{H\&E} \rightarrow \mbox{MT}$ images.
It analyzes $300 \times 300$ tiles and assigns inflammation scores in the $[0,2]$ range, which are then averaged for the whole tissue sample.
Our study is based on 77 rodent liver tissue sections (used also for the sampling of ~1300 tiles for the test dataset, see \sectionref{Sec:datasets}).
In \tableref{Tab:classification_exp} we report the Mean Square Error (MSE) and Mean Absolute Error (MAE) between the inflammation scores generated by the system fed with real and artificial MT stained tissue images. We did not perform an analysis of Pix2Pix and UTOM, because the generator implementations did not allow to process $300 \times 300$ image sizes. 
When the system was fed with (real) $\mbox{H\&E}$ tiles (instead of required MT tiles), we obtained a $\mbox{MSE} =0.031$, $\mbox{MAE} =0.143$. Therefore, the methods performing worse do not provide a benefit, compared to the frequently available H\&E stain.
In alignment with the results from \sectionref{Sec:main_results}, CycleGAN, derived from it StainGAN, and UNIT showed strong performance with MSE and MAE close to 0 (the theoretical optimum), while the traditional methods performed poorly. The success of CycleGAN, StainGAN, and UNIT can be attributed to high SSIM measures and simultaneously low values of FID, see \tableref{Tab:comp_results}.
%\vspace{-0.5cm} 
\begin{table}[tb]
    \centering
    \floatconts    
    {Tab:classification_exp}%
    {\caption{MSE, first row, and MAE, second row, of inflammation score [0, 2] when the system was fed with real versus generated MT images. MSE, MAE, and the standard error have a factor $10^{-2}$.} \vspace{-1.7em}}
	
%	\def\arraystretch{1}
%    \def\aboverulesep{0.ex}
%    \def\belowrulesep{0.ex}      
    \begingroup
    \setlength{\tabcolsep}{3.5pt}
    {\begin{tabular}{cccccccccc}
    \hline   
     {\scriptsize CycleGAN} & {\scriptsize UNIT} & {\scriptsize StainGAN} & {\scriptsize DRIT} & {\scriptsize MUNIT} & {\scriptsize CUT} & {\scriptsize StainNet} & {\scriptsize ColorStat} & {\scriptsize Macenko} & {\scriptsize Vahadane} \\
    \hline
     {\scriptsize $1.1 \pm 0.2$} & {\scriptsize $1.2 \pm 0.2$} & {\scriptsize $1.3 \pm 0.2$} & {\scriptsize $1.7 \pm 0.2$} & {\scriptsize $2.4 \pm 0.7$} & {\scriptsize $3.3 \pm 0.6$} & {\scriptsize $3.7 \pm 0.3$} & {\scriptsize $5.3 \pm 1.5$} & {\scriptsize $9.7 \pm 0.7$} & {\scriptsize $10.3 \pm 0.7$} \\
     
     {\scriptsize $8.6 \pm 0.7$} & {\scriptsize $8.9 \pm 0.8$} & {\scriptsize $8.9 \pm 0.8$} & {\scriptsize $11.3 \pm 0.8$} & {\scriptsize $11.2 \pm 1.2$} & {\scriptsize $13.3 \pm 1.4$} & {\scriptsize $17.2 \pm 1.0$} & {\scriptsize $14.5 \pm 2.1$} & {\scriptsize $28.6 \pm 1.4$} & {\scriptsize $29.6 \pm 1.4$} \\
        
    \end{tabular} \vspace{-1em}}
    \endgroup
\end{table}


 

 
