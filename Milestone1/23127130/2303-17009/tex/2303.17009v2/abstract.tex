%\documentclass[12pt]{article}
%\begin{document}
%\noindent \textbf{Abstract} \\
\begin{abstract}
Image-to-image translation (I2I) methods allow the generation of artificial images that share the content of the original image but have a different style. With the advances in Generative Adversarial Networks (GANs)-based methods, I2I methods enabled the generation of artificial images that are indistinguishable from natural images. Recently, I2I methods were also employed in histopathology for generating artificial images of in silico stained tissues from a different type of staining. We refer to this process as stain transfer. The number of I2I variants 
%(relying on  cardinal or fine modifications)
 is constantly increasing, which makes a well justified choice of the most suitable I2I methods for stain transfer challenging. In our work, we compare twelve stain transfer approaches, three of which are based on traditional and nine on GAN-based image processing methods. The analysis relies on complementary quantitative measures for the quality of image translation, the assessment of the suitability for deep learning-based tissue grading, and the visual evaluation by pathologists. Our study highlights the strengths and weaknesses of the stain transfer approaches, thereby allowing a rational choice of the underlying I2I algorithms. Code, data, and trained models for stain transfer between H\&E and Masson's Trichrome staining will be made available online.
%at \url{https://github.com/Boehringer-Ingelheim}.         

\end{abstract}


%\end{document}