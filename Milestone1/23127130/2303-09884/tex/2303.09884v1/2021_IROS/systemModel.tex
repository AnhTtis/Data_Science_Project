\section{System Model} \label{sec:system_model}


\subsection{Target Dynamics}\label{ssec:target_dynamics}
This work considers one target (e.g., a rogue drone) which can appear/disappear at random times anywhere inside the surveillance area and thus at any time-step the target can exist in one of two states i.e., present or absent. When the target is present, it moves in 3D space according to a stochastic dynamical model. More specifically, the state of the target at time $t$ is modeled as a Bernoulli random finite set (RFS) \cite{VoBook2015,Mahler2014book} $X_t \in \{\emptyset,\{x_t\}\}$, with Bernoulli RFS \cite{Ristic2013} density $F_{t}(X_t)$ with parameters $(p_{t}(\epsilon_t|Z_{1:t}), f_{t}(x_t|Z_{1:t}))$, where $e_{t}=p_{t}(\epsilon_t=1|Z_{1:t})$ is the probability of target existence given measurements $Z_{1:t}$ up to time $t$ and $s_{t} = f_{t}(x_t|Z_{1:t})$ is the spatial density of the target with state $x_t$. The dynamics of the target are modeled as a Bernoulli Markov Process \cite{Mahler2007book} with transitional RFS density $\psi_{t|t-1}(X_t|X_{t-1})$ given by:

\vspace{+3mm}
\begin{tabular}{ l|c|c| }
\multicolumn{1}{r}{}
 &  \multicolumn{1}{c}{$X_t =\emptyset$}
 & \multicolumn{1}{c}{$X_t =\{x_t\}$} \\
\cline{2-3}
$X_{t-1} =\emptyset$ & $1-p_b$ & $p_b b_t(x_t)$ \\
\cline{2-3}
$X_{t-1} =\{x_{t-1}\}$& $1-p_s$ & $p_s \pi_{t|t-1}(x_t|x_{t-1})$  \\
\cline{2-3}
\end{tabular}
\vspace{+3mm}   

\noindent where $p_b$ denotes the probability of target birth, $b_t(x)$ is the birth density (uniform inside the surveillance area), $p_s$ is the probability that the target survives to the next time-step, and $\pi_{t|t-1}(x_t|x_{t-1})$ is the target transitional density which in this work it is assumed to be governed by the following discrete-time dynamical model:
\begin{equation} \label{eq:target_dynamics}
    x_t = \Phi x_{t-1} + \Gamma \nu_{t}
\end{equation}
where $x_t = [\text{x},\text{y},\text{z},\dot{\text{x}},\dot{\text{y}},\dot{\text{z}}]_t^\top \in \mathcal{X}$ denotes the target state at time $t$ which consists of the position and velocity components in 3D Cartesian coordinates and $\nu_{t} \sim \mathcal{N}(0,\Sigma_\text{v})$ denotes the perturbing acceleration noise which is drawn from a zero mean multivariate normal distribution with covariance matrix $\Sigma_\text{v}$. The matrices $\Phi$ and $\Gamma$ are defined as:
\begin{equation}
\Phi = 
\begin{bmatrix}
    \text{I}_3 & \Delta T \cdot \text{I}_3\\
    \text{0}_3 & \text{I}_3
   \end{bmatrix},
\Gamma = 
\begin{bmatrix}
    0.5\Delta T^2 \cdot \text{I}_3\\
     \Delta T \cdot \text{I}_3
   \end{bmatrix}
\end{equation}

\noindent where $\Delta T$ is the sampling period and $\text{I}_3$, $\text{0}_3$ are the $3 \times 3$ identity and zero matrix, respectively.



\subsection{Pursuer Agent Kinematics} \label{ssec:AgentDynamics}
A team of $N$ autonomous pursuer agents (e.g., UAV agents) operate inside the surveillance area,  each of which is subject to the following kinematic model:

\begin{equation} \label{eq:controlVectors}
u^j_{t}\!\! = \! u^j_{t-1} + \begin{bmatrix}
						\Delta_R(l_1) \sin(l_2 \Delta_\phi) \cos(l_3 \Delta_\theta)\\
						\Delta_R(l_1) \sin(l_2 \Delta_\phi) \sin(l_3 \Delta_\theta)\\
						\Delta_R(l_1) \cos(l_2 \Delta_\phi)
					\end{bmatrix}\!\!\!,\!\!  
					\begin{array}{l} 
					    l_1 = 1,...,|\Delta_R|\\
					    l_2 = 0,...,N_\phi\\ 
						l_3 = 1,...,N_\theta
				    \end{array} 
\end{equation}
where $j \in [1,..,N]$, $u^j_{t} = [u^j_x,u^j_y,u^j_z]^\top_{t} \in \mathbb{R}^3$ denotes the state (i.e., position) of pursuer agent $j$ at time $t$, $\Delta_R$ is a vector of possible radial step sizes, $\Delta_\phi=\pi/N_\phi$, $\Delta_\theta=2\pi/N_\theta$, and the parameters $(|\Delta_R|,N_\phi,N_\theta)$ determine the number of possible mobility control actions. The set of all admissible control actions of agent $j$ at time $t$ is denoted as $\mathbb{U}^j_{t}=\{u^{j,1}_{t},u^{j,2}_{t},...,u^{j,|\mathbb{U}^j_{t}|}_{t}  \}$ according to Eq. (\ref{eq:controlVectors}). Although in this work a simplified kinematic model for the agents is utilized in order to demonstrate the proposed approach, depending on the application scenario more realistic kinematic/dynamic models can also be incorporated.



\subsection{Agent Sensing and Jamming Model} \label{ssec:agent_sensing}
The pursuer agents are equipped with an onboard directional active 3D range-finding radio which they use in order to detect the target, acquire target measurements, and transmit power to the target in order to radio-jam its communications circuitry. The range-finding characteristics are as follows:

\subsubsection{Sensing Profile} The radio's sensing profile $S$ in 3D is modeled as a circular right angle cone with Cartesian coordinates $(x,y,z)$ given by: $[ x=u\cos(v), y=u\sin(v), z=\frac{h_a}{r} u ]$ where $ u \in [0, r], ~ v \in [0, 2\pi]$, $h_a$ characterizes the effective sensing range, $r=\tan(\frac{\theta_a}{2})h_a$ is the base radius of the cone, and $\theta_a$ is the opening angle of the cone. Thus, a target with position coordinates $Hx_t = [\text{x},\text{y},\text{z}]_t^\top$ (where $H$ is a matrix which extracts the position coordinates from state $x_t$) resides inside agent's $j$ sensing range when $Hx_t \in S^j$. It should be mentioned that the direction of the agent's sensor is given at each time-step by the direction of the vector $\vec{d}_t = x_t - u^j_t$, where $x_t$ and $u^j_t$ are the positions of the target and  agent $j$, respectively. 


\subsubsection{Measurement Model} Each agent $j$ uses its radio received signal to acquire 3D target measurements $z^j_t=[\rho,\theta,\phi]^\top_t \in \mathcal{Z}$ (i.e., radial distance $\rho$, azimuth angle $\theta$, and inclination angle $\phi$) according to the following measurement model: 

\begin{equation}
    z^j_t = h(x_t,u^j_t) + w^j_t
\end{equation}
The function $h(x_t,u^j_t)$ is given by:
\begin{equation}
    \left[ \eta_t, ~\tan^{-1}\left(\frac{\Delta_{t,y}}{\Delta_{t,x}}\right),~ \tan^{-1}\left(\frac{\sqrt{\Delta_{t,x}^2+\Delta_{t,y}^2}}{\Delta_{t,z}}\right) \right]^\top
\end{equation} 
where $\eta^j_t=\norm{H x_t-u^j_t}_2$, $\Delta_{t,y} = [\text{y}-u^j_y]_t$, $\Delta_{t,x} = [\text{x}-u^j_x]_t$, $\Delta_{t,z} = [\text{z}-u^j_z]_t$, and $w^j_t \sim \mathcal{N}(0,\Sigma_w)$ is zero mean Gaussian measurement noise with covariance matrix $\Sigma_w$. 
Due to sensing imperfections, at time $t$, agent $j$ also receives false-alarm measurements or clutter $\{c^{j1}_t,\ldots,c^{jn}_t\}$ (in addition to the target measurement) with a Poisson rate of $\lambda_c$ (i.e., $\mathbb{E}(n) = \lambda_c$), which are uniformly distributed (i.e., with clutter density denoted as $p_c(c_t)$) inside the measurement space. To summarize, at each time-step, agent $j$ receives a set $Y^j_t$ of measurements which is given by: 
\begin{equation}\label{eq:Upsilon}
    Y^j_t=\bigcup \Big\{ y_1\subset\{ \emptyset,z^j_t\}, y_2 \subseteq \{c^{j1}_t,\ldots,c^{jn}_t\} \Big\}
\end{equation}
%
\noindent Thus, agent $j$ can receive at time-step $t$, zero or one target measurements $z^j_t$ and a set of false-alarms measurements. 




\subsubsection{Target Detection and Jamming}
The agents use their onboard radio to detect and jam the target. A target detection occurs, with certain probability, when the target is illuminated (i.e., when the agent transmits power). Hence, it is important to note here that the same transmit signals are used to detect the target and at the same time jam it. The detection probability i.e., $p^j_D = p^j_D(x_t,u^j_t,l^j_t)$, that agent $j$ with state $u^j_t$ and transmit power-level $l^j_t$ detects a target with state $x_t$ is given by: 
\begin{equation}\label{eq:sensing_model}
 p^j_D(x_t,u^j_t,l^j_t) = 
  \begin{cases} 
   \frac{l^j_t}{l^j_{\max}} p^{\max}_D & \!\!\!\!\!\!\text{if } \eta^j_t < R_0 \wedge x_t \in S^j \\
   \frac{l^j_t}{l^j_{\max}} p^{\max}_D  \left(\frac{R_0}{\eta_t}\right)^{n_e} &\!\!\!\!\!\! \text{if } \eta_t > R_0 \wedge x_t \in S^j \notag
  \end{cases}
\end{equation}

\noindent where $l^j_{\max}$ is the maximum transmit power-level, $\eta^j_t=\norm{H x_t-u^j_t}_2$ denotes the Euclidean distance between the agent and the target in 3D space, $p^{\max}_D$ denotes the maximum attainable detection probability of the sensor which can be obtained when a target resides within $R_0$ distance from the agent's position and the target also resides inside the agent's sensing profile $S^j$, and $n_e$ is the path-loss exponent. Note that, $p^j_D = 0$, when $x_t \notin S^j$.
Finally, the transmit power-level $l^j_t$ takes its values from the discrete set of admissible transmit power-levels $\mathbb{L}^j$, i.e., $l^j_t \in \mathbb{L}^j = \{l^{j1}_t,...,l^{jn}_t \}$ with $l^j_{\max} = \max ~\mathbb{L}^j$. Additionally, the received power at the target with location $Hx_t$ from an agent with state $u^j_t$ which transmits at power-level $l^j_t$ is given by the following path-loss model:
\begin{equation}\label{eq:pathLoss}
R^j_t(Hx_t,u^j_t,l^j_t) = 
\begin{cases} 
    l^j_t & \text{if } \eta^j_t < R_0 \wedge x_t \in S^j \\
    l^j_t  \left(\frac{R_0}{\eta_t}\right)^{n_e} & \text{if } \eta_t > R_0 \wedge x_t \in S^j \notag\\
\end{cases} 
\end{equation}
Note that, $R^j_t(Hx_t,u^j_t,l^j_t) = 0$, when $x_t \notin S^j$.
