\section{Related Work} \label{sec:related_work}


A recent survey on detection, tracking, and interdiction techniques for malicious drones or unmanned aerial vehicles (UAVs) can be found in \cite{Guvenc2018}, where the authors discuss various drone threats and review several counter-measures that have been investigated in the literature. In \cite{Shi2018} the authors present a detailed survey on the state-of-the-art anti-drone systems and discuss techniques and technologies used for drone surveillance. The work in \cite{Ganti2016} discusses the advantages and disadvantages of various drone detection methods including audio-visual, thermal, and RF and proposes a low-cost stationary system for drone detection and tracking which can be easily incorporated into third-party anti-drone platforms. The authors in \cite{Srigrarom2020} focus on the problem of detection and tracking of small and fast moving drones with the ultimate goal of developing a system that can be used to prevent such small drones from accessing restricted areas and facilities. %Similarly, the work in \cite{Dressel2019} uses a consumer drone outfitted with antennas and commodity radios to localize another drone using its telemetry radio signature. 

Moreover, in \cite{Karas2020,Papaioannou2019_3,Papaioannou2019_2,Papaioannou2019_1,Papaioannou4}, the problems of formation control, target tracking, and target interception with multiple agents are investigated but without considering jamming capabilities for the pursuers. The work in \cite{Perkins2015} develops a UAV based solution for localizing a GPS jammer and the  work in \cite{Multerer2017} proposes a low-cost ground jamming system to counteract the operation of small drones. In \cite{Brust2017} a team of defense agents forms a cluster around a malicious target in order to prevent it from entering restricted airspaces. The authors assume however the availability of a tracking system for detecting and tracking the target. Finally, in \cite{Papaioannou2019,Valianti} the problem of jamming a rogue drone with a team of mobile agents is investigated, however without  considering target appearance/disappearance events. 

Compared to the existing techniques, the proposed framework develops a novel cooperative technique for the problem of target tracking and target radio-jamming in 3D by a team of mobile agents, while considering the induced jamming interference amongst the team. The problem is investigated in the presence of target detection uncertainty, clutter, and target appearance/disappearance events which to the best of our knowledge has not been investigated before. 

