



\section{Evaluation}\label{sec:Evaluation}

\subsection{Simulation Setup} \label{ssec:sim_setup}
The simulation setup used to evaluate the performance of the proposed approach is as follows: The agents and the target maneuver inside a bounded 3D surveillance area of size $100\text{m} \times 100\text{m} \times 100\text{m}$. The dynamics of the target are according to Eq. \eqref{eq:target_dynamics} with $\Sigma_\text{v}=0.5  \text{I}_3 ~\text{m}/\text{s}^2$ and $\Delta T = 1$s. The dynamics of the pursuer agent are according to Eq. \eqref{eq:controlVectors} with $\Delta_R = [1, 3, 5]$m, $N_\phi = 8$, and $N_\theta= 8$, unless otherwise specified. The pursuer agent detection and jamming model is according to Eq. (\ref{eq:sensing_model}) with $p_D^{\text{max}} = 0.95$, and transmit power levels $\mathbb{L} = [\text{off},-50,-7,0.5]$dB, i.e., $l^j_\text{max} = 0.5$dB $~\forall j$. The path-loss exponent $n_e=2$ and $R_0 = 6$m. The agent measurement model has $\Sigma_\text{w}= \text{diag}([0.8\text{m}, \pi/50 \text{rad}, \pi/50 \text{rad}])$ and clutter rate $\lambda_c = 3$. The conic sensing profile $S$ of each agent has $h_a=40$m and $\theta_a=80$deg and the interference value for each agent $\delta^j$ is set at $-40$dB. The Bernoulli filter parameters are $p_b = 0.02$, $p_s = 0.98$ and the birth density $b_t(x)$ is uniform inside the surveillance space. Finally, the tracking-and-jamming objective function used in the experiments is given by Eq. (\ref{eq:joint_control1}) as $\Xi_2^N$ with $N=3$ and the jamming constraints are given by Eq. (\ref{eq:joint_control2}). To handle the non-linear measurement model, the Bernoulli filter was implemented as a particle filter \cite{Ristic2013}. Finally, in the implementation of the GRASP-based optimization $n_s = 10^4$ samples and $n_I = 100$ iterations were used.





\subsection{Results}
The first experiment, shown in Fig. \ref{fig:res1}, aims to demonstrate the overall behavior of the proposed approach. More specifically, in this simulated scenario which takes place over 30 time-steps, a team of $N=3$ pursuer agents cooperate in order to track-and-jam the target. In particular, Fig. \ref{fig:res1}(a) shows the trajectories of the agents (i.e., blue, purple and orange) while tracking and jamming the target (i.e., ground truth is shown with black line, estimated is shown with cyan line). The agents are initially located at the $(x,y,z)$ coordinates $[30,30,30]^\top, [20,10,10]^\top, [10,15,15]^\top$ and the target is spawned inside their sensing range with initial state $[20, 20, 20, 1.5, 2, 1.7]^\top$. As shown in the figure, the agents maneuver around the target in such a way so that (a) the tracking-and-jamming performance is maximized while (b) the jamming interference amongst the team is kept below the critical value of -40dB. More specifically, 
Fig. \ref{fig:res1}(b) shows the transmit power level of each agent during this experiment and  Fig. \ref{fig:res1}(c) shows the jamming agents (i.e., due to jamming interference). For instance, agent 1 is being jammed by agent 3 during time-steps 1 to 3 and then again during time-steps 16 and 17. Additionally, during time-steps 16 to 18 agent 1 is also being jammed by agent 2 as shown in the first row of Fig. \ref{fig:res1}(c). Then, Fig. \ref{fig:res1}(d) shows the target received power (i.e., red line) and the agent jamming interference in dB indicated by the colored circles. Figure \ref{fig:res1}(e) shows the estimated target existence probability and Fig. \ref{fig:res1}(f) shows the tracking error i.e., the optimal sub-pattern assignment (OSPA) metric \cite{Schuhmacher2008c} of order 2 with cut-off value of 10m i.e., OSPA(p=2,c=10). In this experiment the target is occluded between time-steps 14 to 17 and thus from the agents' perspective the target seems to disappear at time-step 14 and then to re-appear at time-step 18 as shown in Fig. \ref{fig:res1}(e). The agents try to estimate the target's appearance and disappearance events captured by the existence probability (i.e., red line) in Fig. \ref{fig:res1}(e). 

From the results it is also observed that during the first few time-steps the agents cause interference to each other. For this reason, the agents transmit at lower power levels to accommodate this interference as shown in Figs. \ref{fig:res1}(b)-(c). Moreover, at time-step 16, agent 1 is being jammed by agent 2 and agent 3 as shown in Fig. \ref{fig:res1}(c). Subsequently, agents 2 and 3 switch off their antennas as shown in Fig. \ref{fig:res1}(b) in order to respect the interference limit of agent 1. Figure \ref{fig:res1}(d) verifies that at time-step 16 agent 1 does not receive any power from agents 2 and 3. Figure \ref{fig:res1}(g) shows the configuration of the agents' sensing profile at this particular time step and as it can be seen, agent 1 resides inside the jamming range of agents 2 and 3. Then, the spike in the tracking error at time-step 15 is due to a cardinality estimation error, i.e., the agents believe that the target is present at time-step 15, however in reality the target at time-step 15 is absent. This error, however, is corrected in the next time-step. Finally, Fig. \ref{fig:res1}(h) shows the configuration of the agents at time-step 30 along with their antenna orientations. As illustrated, the agents take positions which avoid interferences with each other, while at the same time the positions taken by the agents maximize the target received power. 

\begin{figure}
	\centering
	\includegraphics[width=\columnwidth]{res2.pdf}
	\caption{Impact of the number of mobility controls and the size of the opening angle $\theta_a$ of sensing profile $\mathcal{S}$ on the performance of the proposed approach.}	
	\label{fig:res2}
	\vspace{-3mm}
\end{figure}




The second experiment aims to investigate in more detail the impact of the various parameters on the behavior of the proposed approach. In particular, it investigates the impact of the opening angle $\theta_a$ of the sensing profile $S$ and the number of total mobility control actions on the performance of the system. 
Intuitively, it is expected that as $\theta_a$ increases, the sensing range increases, and thus the number of times the agents interfere with each other increases as well. However, as the system's degrees of freedom, i.e., the number of admissible mobility control actions, increase, additional options are provided for the agents to try to avoid the interference. 


\begin{table} \label{table:1}
\caption{Parameter Configurations}
\renewcommand{\arraystretch}{0.8}
\begin{center}
	\begin{tabular}{| c | c | c |}
		\hline
		Config. \# & \# Mob. Controls & $\theta_a$ (deg)\\
		\hline \hline
		1 & 11 & 60\\  \hline
		2 & 11 & 80\\  \hline
		3 & 11 & 100\\ \hline
		4 & 11 & 120\\ \hline
		5 & 79 & 60\\  \hline
	    6 & 79 & 80\\  \hline
        7 & 79 & 100\\  \hline
        8 & 79 & 120\\
	    \hline
	\end{tabular}
\end{center}
\vspace{-7mm}
\end{table}



To verify this assumption the parameter configurations shown in Table I were used. This table shows that for configurations 1-4 each agent has a total of 11 mobility control actions, whereas configurations 5-8 allow for a total of 79  mobility actions per agent. Moreover, the opening angle $\theta_a$ takes increasing values from 60deg to 120deg. The experimental setup is as follows (with all the other parameters taking values according to  Section \ref{ssec:sim_setup}): First the target is randomly spawned within the surveillance area. Then, the initial locations of 3 purser agents are randomly sampled, from within a sphere with radius 20m centered at the target location. The antennas of the agents point to the center of the sphere. For each configuration shown in Table I the system runs for 30 time-steps (i.e., one trial). This process is performed for each configuration 20 times and the results are averaged and shown in Fig. \ref{fig:res2}. More specifically, Fig. \ref{fig:res2}(a) shows the average jamming incidents per trial per agent for each configuration, Fig. \ref{fig:res2}(b) shows the average received power for the target and agents, Fig. \ref{fig:res2}(c) shows the average transmit power-level in Watts at each time-step for the 8 configurations, and finally Fig. \ref{fig:res2}(d) shows the average tracking error. It can be observed from these results that with 11 mobility control actions as the opening angle $\theta_a$ becomes larger the number of times that the agents are jamming each other increases and so does the agent received power due to interference as shown in Figs. \ref{fig:res2}(a)-(b). This is due to the limited number of admissible control actions which prohibit the agents to acquire positions which are interference free. In addition, since the target detection is linked to the transmit power-level (as shown in Eq. (\ref{eq:sensing_model})), the agents by turning off their antennas put themselves in a disadvantage regarding their tracking performance which can result in complete tracking failure. To overcome this issue, the agents try to transmit at low power-levels (i.e., Fig. \ref{fig:res2}(c)). This however, not only causes reduced tracking performance (i.e., Fig. \ref{fig:res2}(d)) but also reduced target jamming performance (i.e., Fig. \ref{fig:res2}(b)). On the other hand, once the system's degrees of freedom are increased, large opening angles can be handled more efficiently with improved jamming and tracking performance and reduced interference as illustrated in Fig. \ref{fig:res2} for configurations 5-8. 


\begin{figure}
	\centering
	\includegraphics[width=\columnwidth]{res3.pdf}
	\caption{Impact of the interference constraints i.e., Eq. (\ref{eq:joint_control2}), on the tracking and jamming performance.}
	\label{fig:res3}
	\vspace{-3mm}
\end{figure}







The last experiment aims to investigate how the interference constraints in Eq. (\ref{eq:joint_control2}) affect the tracking and jamming performance of the system. Again, the same simulation setup discussed in the previous paragraph is followed, where a target is randomly spawned inside the surveillance area and then within a sphere around the target location 3 agents are spawned. The system runs for trials with duration 30 time-steps and evaluated with the interference constraints enabled and disabled. Twenty (20) trials are conducted for each case with the simulation parameters and values set according to Section \ref{ssec:sim_setup}. The average results are shown in Fig. \ref{fig:res3}. As expected, by disabling the interference constraints the number of jamming incidents per agent increases as shown in Fig. \ref{fig:res3}(a). As a result, the average received power per agent also increases as shown by Fig. \ref{fig:res3}(b) significantly above the critical value, and thus the system is driven into failure. On the other hand, it is shown in Fig. \ref{fig:res3}(c) that the target received power is higher with the interference constraints turned off. The results show that when the interference constraints are disabled the agents can get much closer to the target and to each other. As a result, the target receives much higher power compared to the power received with the interference constraints enabled. Similarly, with the interference constraints turned off, the agents optimize the joint target detection probability of Eq. \eqref{eq:joint_control1} without any constraints and this results in better tracking performance as shown in Fig. \ref{fig:res3}(d). However, from the aforementioned results it is clear that the interference constraints are vital for keeping the system operational at all times while achieving a satisfactory tracking and jamming performance as shown in Fig. \ref{fig:res3}.






















