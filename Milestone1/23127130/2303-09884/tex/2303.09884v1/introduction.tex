\section{Introduction} \label{sec:intro}


Drones which have nowadays become the new emerging trend, are currently being utilized in many applications ranging from aerial photography and critical infrastructure inspection to emergency-response missions and aerial monitoring tasks. Unfortunately, drones have also become a threat and a risk to public safety. In particular, numerous times drones have threatened public safety by targeting airports and restricted airspaces \cite{Schneider2019}, attacking critical infrastructures \cite{AC1} and endangering people's lives \cite{Humphreys2015}.
For this reason there is a necessity for counter-drone systems that can detect, track, and interdict rogue or malicious drones \cite{Wesson2013}. Although, counter-drone approaches and systems have already been proposed in the literature \cite{Guvenc2018,Loeb2017,Shi2018}, these are still in their infancy and considerably more work is needed in order for this technology to reach the required level of maturity. 

In this work, a distributed multi-agent control framework is proposed, in which a team of pursuer agents (i.e., pursuer drones) cooperate in order to continuously track and radio-jam a target (i.e., rogue drone) in 3D space (as depicted in Fig. \ref{fig:problem}), disrupting its communication and sensing circuitry and thus ultimately forcing it to execute its fail-safe protocols \cite{Revill2016} i.e., auto-landing. The assumption is that the pursuer agents are equipped with a 3D range-finding active sensor \cite{Park2001} with limited sensing range, which they use to obtain noisy target measurements (i.e., radial distance, azimuth angle, and inclination angle) in the presence of false-alarm measurements (or clutter). In this work it is assumed that the target detection process is uncertain and that the target can appear and disappear within the surveillance area at random times (i.e., the target can spawn from anywhere inside the surveillance area and additionally during tracking it can move behind obstacles or other occlusions which results in its disappearance from the field of view of the pursuer agents). Finally, it is assumed that the pursuer agents have the ability to transmit power to the target via their on-board active sensor, at discrete power-levels, with the ultimate purpose of radio-jamming its circuitry. 
\noindent The main contributions of this work are two-fold:

\begin{figure}
	\centering
	\includegraphics[width=\columnwidth]{figA.pdf}
	\caption{A team of pursuer agents choose their mobility controls and radio transmission levels which result in accurate tracking and uninterrupted radio jamming of the roque target.}	
	\label{fig:problem}
	\vspace{-4 mm}
\end{figure}



\begin{itemize}
    \item A novel distributed control framework is proposed for the problem of target tracking and target radio-jamming in 3D by a team of cooperative mobile agents. The proposed approach allows the team of pursuer agents to optimally (a) choose their mobility control actions that result in accurate target tracking and (b) choose their transmit power levels to cause uninterrupted target radio jamming. The agents cooperate for improving the target tracking-and-jamming performance of the team while minimizing the jamming interference amongst them.
    \item In the scenario considered, the target can exist in one of two states i.e., present or absent. Thus in order to be jammed, its existence probability along with its spatial density must be estimated from a sequence of noisy measurements in the presence of clutter, by a team of mobile agents equipped with conic directional antennas with limited sensing range.
    \end{itemize}


\noindent The rest of the paper is organized as follows. Section \ref{sec:related_work} discusses the related work. Section \ref{sec:system_overview} formulates the problem and illustrates the proposed system architecture. Section \ref{sec:system_model} develops the system model and Section \ref{sec:proposed_approach} discuses the details of the proposed approach. Finally, Section \ref{sec:Evaluation} conducts an extensive performance evaluation and Section \ref{sec:Conclusion} concludes the paper and discusses future directions.


















