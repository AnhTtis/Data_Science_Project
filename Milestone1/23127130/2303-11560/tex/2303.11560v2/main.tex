\documentclass{article}
\usepackage[utf8]{inputenc}
\usepackage{geometry}
\usepackage{booktabs}
 \geometry{
 a4paper,
 total={170mm,257mm},
 left=20mm,
 top=20mm,
 }
 \usepackage{graphicx}
 \usepackage{titling}

 \title{Machine Learning Based Nitrate Concentration Soft-Sensor}
\author{Harry Dobbs}
\date{December 2022}
 
 \usepackage{fancyhdr}
\fancypagestyle{plain}{%  the preset of fancyhdr 
    \fancyhf{} % clear all header and footer fields
    %\fancyfoot[R]{\includegraphics[width=2cm]{KULEUVEN_GENT_RGB_LOGO.png}}
    \fancyfoot[L]{\thedate}
    \fancyhead[L]{RiverWatch}
    \fancyhead[R]{\theauthor}
}
\makeatletter
\def\@maketitle{%
  \newpage
  \null
  \vskip 1em%
  \begin{center}%
  \let \footnote \thanks
    {\LARGE \@title \par}%
    \vskip 1em%
    %{\large \@date}%
  \end{center}%
  \par
  \vskip 1em}
\makeatother

\usepackage{lipsum}  
\usepackage{cmbright}

\begin{document}

\maketitle

\noindent\begin{tabular}{@{}ll}
    Author & \theauthor\\
\end{tabular}

\section*{Summary}


\section*{Introduction}
% Talk about what has causes the increase in nitrate....
Excessive amounts of nitrogen in rivers is a major concern worldwide. The increased amount of nitrogen in rivers is a concern for aquatic life, drinking water and recreational activities. In New Zealand, the increase in nitrates is primarily caused by intensification of agriculture \cite{doi:10.1080/00288233.2012.747185}. To enable mitigation of the effects caused by nitrate leeching, monitoring of nitrate levels is required. Traditionally this has been achieved by river grab sampling which has numerous drawbacks like delays in data acquisition, high costs and difficulty to detect events that take place during the time interval between two samples \cite{miller2016quantifying}. New Zealand has recognized that monthly discrete (spot) water quality measurements cannot capture or characterize rapid changes in water quality. High frequency nitrate monitoring is becoming popular important for several reasons, such as improving uncertainty in the examination of trends in nitrate concentrations, improving accuracy of estimated nitrate loads, calibration of surface water and ground water nitrate transport models, improving ability to predict downstream water quality and identifying periods when nitrate concentrations require additional management \cite{niwa_2019}. 
\newline \newline
The common commercially available sensor technologies can be categorized as ion-selective electrodes (ISE), wet-chemical analysers and optical sensors. ISE is relatively inexpensive and easy to use, however they are often inaccurate and prone to drifting. Wet chemical and optical sensors are considered to be higher accuracy but are expensive and require significant maintenance \cite{castrillo2020estimation}. These issues hinder the possibility of nitrate sensors being able to perform high-frequency in-situ nitrate monitoring.
Recently soft-sensor machine learning based approaches have been explored. Soft-sensor based approaches use available surrogate data such as water temperature, conductivity, turbidity, dissolved oxygen and pH to determine target variables such as nitrate or phosphorus concentrations. Soft-sensor based approaches can be considered low cost and low maintenance. However, their accuracy significantly relies on the ability to have an appropriate model - which generally requires large quantities of ground truth data to train and validate the model.
\newline \newline
The work presented in this paper aims to aid in the development of a robust, high frequency, low cost and high accuracy nitrate concentration sampling methodology.
The main objectives of this work are as follows:

\begin{itemize}
  \item Report on the current state-of-art nitrate concentration soft-sensors.
  \item Determine which machine learning model would be most suitable for real-time nitrate monitoring (considering both non-linear and linear models).
  \item Further understand which predictor variables are most beneficial to the nitrate concentration model performance.
  \item Quantify how much training data is required to train a model that can work on real-time data with acceptable accuracy for long-term stable deployement.
\end{itemize}

\section*{Related Work}
The most popular models used for predicting nitrate concentrations from surrogate variables include random forests (RF), support vector machines (SVM), Linear Regression (LR), neural networks (NN) and Gaussian mixture models (GAMMS).
\newline \newline
\textbf{Random Forest (RF):}
Random forests regression is a powerful and widely used supervised machine learning method - capable of modelling non-linear patterns. RF builds a multitude of regression trees (forest) by making many random subsets of the data and using random variable sets to build many predictions of the dependent variable. Those many predictions are averaged to produce the final estimated dependent variable. A number of high frequency nitrate soft-sensor implementations have used random forest regression \cite{green2021predicting,  harrison2021prediction, tran2022predicting, castrillo2020estimation}. 
\newline
\newline
Green et al \cite{green2021predicting} implements a RF and SVM model using high frequency data (15 minute intervals) from four years (October 2012 - January 2017) worth of data collected from the Hubbard Brooke Experimental Forest River based in the White Mountains of New Hampshire, USA. The catchment is a steep (mean slope = 17.1\%) glaciated hillside of spodic soils. The sensor data collected includes EC, pH, fDOM, turbidity, DO, NO3. Additionally discrete routine weekly samples and multiple storm and seasonal snow melt sampling campaigns were taken. To incorporate seasonality as a predictor variable, day of the year was encoded. The models were trained using the manual samples (n=382) which then allows the models to predict nitrate concentrations at 15-minute intervals. The results from this study showed great promise for nitrate concentration prediction when an appropriate sensor or proxy is not available (Nash-Sutcliffe efficiencies; \(R^2\) = 0.68 and 0.64 for the SVM and RF models, respectively).

The results achieved by Green et al show the SVM model slightly out preforms the RF model (RF preformed slightly better for Aluminium and Potassium concentration). However the RF model was found to be more robust when predictor variables were removed. This paper mentions how additional independent variables such as meteorological conditions (temperature, precipitation, vapour pressure deficit, evapo-transpiration) could aid in better predictions. Future work recommended by this paper was to build a number of models to represent a range of catchments, then test the ability for the models to generalize across time and place (could not be achieved in this paper due to lack of available data).
\newline
\newline
Harrison et al \cite{harrison2021prediction}, proposed a solution to estimate stream nitrogen and phosphorus concentrations from sensor data from 11 locations within a forested, mountainous drainage area in upstate New York by using a RF model. The catchment is predominantly forested, though parts of the lake's shoreline are relatively well-developed. The high frequency sensor data used in this study includes 
hydro-static pressure (HP), soil moisture (SM), water temperature, pH, specific conductance, turbidity, fluorescence dissolved organic matter (fDOM). The model was trained using these sensors as well as their 1-h, 5-h, 24-h, and 120-h lags and delta values.
The manual water nutrient samples were taken monthly, and the sensor values were captured every 1-15 minutes. During storm events, water nutrients samples were taken automatically at a rate of 1-2hrs. The manual water nutrient samples were paired with the high frequency data hourly median values.
Leave one out cross validation was used to train and test this method (LOOCV). LOOCV involves omitting one case (row) at a time from the dataset while training (fitting) the model, then
predicting the value for the omitted case. This is done n times, where n is the number of values of the dependent variable. This method was modified slightly, omitting data from one sampling event at a time (i.e., 1 case from baseflow dates, multiple cases from storm event dates when multiple observations were made on the same day). The results from this study were (MAE=0.075mgN/l, RMSE=0.12mgN/l, R2=0.89) for detecting nitrates where the nitrate ranges were from 0.01-1.01 mg/L .

This study found fDOM and soil moisture to be important predictors for calculating nutrient concentrations. The authors noted a disadvantage of random forests is their inability to make predictions outside of the data used for training. However this problem exists in all machine learning models (inability to accurately extrapolate outside the training values).
\newline
\newline
Jingshui et al \cite{tran2022predicting} also developed a RF model to determine nitrates using high frequency predictor data (15 minute intervals) from three locations on the Danube River, Germany. The relevant predictor variables used include dissolved oxygen, temperature, conductivity, pH, and discharge rate (Q). Monthly, daily and hourly features were also implemented into the model. The RF results were compared against a multi-variable linear regression model. The RF demonstrated better results when using 3 or more predictor variables. This study implements a moving mean filter to remove outliers  by using a sliding windows of 6h (24 data points). If a value is more of less than three local std deviations from the local mean the value will be defined as an outlier. %However this implementation likely introduces look-ahead bias, as the training and testing set are created using random K-fold sampling - which means testing dataset is intertwined, and at higher frequency sampling (15 minute) the changes between measurement are relatively small.
\newline 
\newline
Similarly Castrillo et al \cite{castrillo2020estimation} implemented a RF model to determine nitrate concentrations from two different monitoring sites from contributors to the River Thames England (Enborne and The Cut). The River Enborne is relatively rural, receiving sewage effluent from small sewage treatment works, whereas The Cut is highly urbanised, receiving large quantities of sewage from major towns.
This study implements a preprocessing stage to help remove data skew, prior to creating training, testing and validation data-sets. This method similar to Jingshui et al\cite{tran2022predicting}, creates the data-sets using a K-fold sampling scheme (K=10) , which results in 90\% of the data being used as training data.
The model to achieve impress results (nRMSE of 15\% and RMSE 0.025 mg/l on the Enborne data-set), with EC and pH being the most powerful variables. The work in this study shows that RF is an appropriate model choice and are able to effectively represent complex environmental phenomena. 
\newline 
\newline
Recently Paepae et al \cite{paepae2022virtual} expanded on the work done by Castrillo by testing different machine learning models, data imputation methods for missing values (and data scaling methods) and proposing the best practises for soft-sensor implementations. The authors evaluated decision tree (DT), random forest (RF), extremegradient boosting (XGB), light gradient boosting machine (LGBM), multi-layer perceptron (MLP), k-nearest neighbor (kNN), support vector machine (SVM), extremely randomized trees (ET), gradient boosting regressor (GB), stochastic gradient descent (SGD), linear regression (LR), ridge regression (Ridge), and bagging regressor (BR). It was observed that ET preformed the best (R2=0.9682), followed by RF (R2=0.9544) and LR preformed the worst. Little to no-benefit was found by scaling the values. The effects of value imputation and pre-transformations were found to be marginal.
\newline
\newline
\textbf{Neural Networks:} Neural networks are another popular methods for determining Nitrate concentrations from surrogate sensors and predictors. Neural networks work by learning a function embedding which is encoded by connecting artificial neurons through sequential layers.
\newline
\newline
An neural network was proposed by Diamantopoulou et al \cite{diamantopoulou2005use} for the prediction of
the monthly values of three water quality parameters of the Strymon river at a station located in
Sidirokastro Bridge near the Greek - Bulgarian borders by using the monthly values of the
other existing water quality parameters as input variables. The monthly data of thirteen
parameters and the discharge, at the Sidirokastro station, for the time period 1980-1990 were
selected for this analysis. The results demonstrate the ability of the appropriate ANN models
for the prediction of water quality parameters. This provides a very useful tool for filling the
missing values that is a very serious problem in most of the Greek monitoring stations. 
\newline
\newline
A nonlinear autoregressive exogenous (NARX) network was proposed in \cite{di2022nonlinear} to predict nitrate concentrations using, water discharge (Q), electrical conductivity (EC), dissolved oxygen (DO) and temperature (temp). A NARX network is a kind of neural network which uses values previously predicted by the network to aid in future predictions, this architecture is typically used in time-series problems. The Kreutz Creek data-set used daily sampled values collected from May 2020 to January 2021. This model was trained for 6 months and was able to achieve overall accurate results for the following 60 days (\(R^2 = 0.77, MAE = 0.95 mg/l, RMSE = 1.16 mg/l, RAE = 48.21\% \)), with the best results within the first 30 days. 
This paper concluded that to achieve accurate results at least 2 years of training data is required, in particular if only water discharge is used.
\newline
\newline
\textbf{Generalised Additive Mixed Model (GAMM):} 
Kermorvant et al \cite{kermorvant2021understanding}, implemented a GAMM to explain the nonlinear relationships at each site between nitrate concentration and conductivity, turbidity, dissolved oxygen, water temperature, and elevation. The dataset is from the NEON database and comprises of three locations; the Arikaree River in Colorado, Caribou-Poker Creeks Research Watershed in Alaska, and Lewis
Run in Virginia. Measurements were taken every 15 minutes of NO3, EC, DO, Water Level and temperature over two years. There is some missing data, which is largely due to the weather events e.g. The river at Caribou freezes from approximately October to May each year and NEON removes most sensors from the site to prevent damage or loss. This study, emphasises the need to have a model for each river location i.e. At Caribou, nitrate, dissolved oxygen, specific conductance and turbidity
increased while temperature decreased during the afternoon and the night. At Arikaree, the
diel patterns exhibited comparatively more variation than at Caribou, and both nitrate and
dissolved oxygen fluctuated in the opposite direction (i.e. decreasing at night and increasing
during the day). When a flow event occurred at Arikaree site (i.e. when the
elevation level suddenly rose), nitrate concentration, oxygen concentration and turbidity
increased, while specific conductance and temperature decreased. Conversely, at Lewis Run,
a sudden increase in water level coincided a decrease in nitrate concentration. At Caribou,
the rise in water level was accompanied by an increase in turbidity, but the relationships
between water level and the other water-quality covariates were more complex.
Furthermore, there were differences in which water-quality co-variate was most beneficial for predicting nitrate concentrations - with the most important for Arikaree being water temperature, the most important for Lewis Run being specific conductance and time and at Caribou the most important was time and water surface elevation. The result from this work suggest that a single model cannot be used across multiple sites due to the unique relationships between nitrates and water-quality variables. Future research may reveal that models fit to data from sites with more similar land use, climate conditions
and flow regimes are more transferable.
\newline 
\newline
\textbf{Limitations:} It is important to note that some these implementations use data sampling schemes which cause the training and testing data-sets to be intertwined (e.g. the train and test data set points are randomly chosen - meaning that a test point could be chronologially neighbouring two training points). This introduces look-ahead bias and only determines how well the model preforms at interpolating values as opposed to how the model might preforms on unseen data. Some implementations use K-fold (with cross-validation), which is reasonable - although the amount of training data (4x the amount of testing data) used in these examples is significant and there appears to be large variation across each fold - likely due to seasonal variation. 
Other implementations have shown the ability to predict high frequency nitrates values by using low frequency manual sampling (weekly). This can be useful to greatly augment the temporal resolution of nutrient data-sets (i.e. if you have discrete manual nitrate samples and you want to estimate nitrate levels between samples using high-frequency sensor data).

\section{Linear vs Non-linear Models}
Jingshui et al \cite{tran2022predicting} demonstrated the benefits of using a non-linear model to predict NO3. When detecting nitrates with a multivariable linear regression model an R values of 0.67, 0.89 were achieved as opposed to R values of 0.99 and 0.99, when using a RF model.

Non-linear machine learning models present some advantages
in comparison to linear regression. There is no need to find obvious
relationships between the target variable and the surrogates, and
they allow capturing phenomena that have a non-linear behaviour,
as usually occurs with environmental processes \cite{castrillo2020estimation}.

Overall it is unlikely that the a linear regression based model will be able to achieve desirable accuracy for modelling nitrates, this is due to the complex non-linear relationship between the predictors variables. There is no available literature to the best of my knowledge suggesting that a purely linear model will be suitable for our application. 

\section{Surrogate Sensors}
Green et al \cite{green2021predicting}, found that EC was the most important independent variable for predicting nitrate, likely due to the direct dependence of EC on ionic strength of a solution. Model predictions also were sensitive to the inclusion of FDOM which is likely due to FDOM being an effective proxy for dissolved organic carbon. Nitrate predictions were also sensitive to pH, perhaps because most stream nitrate at Hubbard Brook is produced by nitrification, which is an acidifying process. 
\newline
\newline
Harrison et al \cite{harrison2021prediction}, tested dropping different surrogate variables from the random forest model. It was found that soil moisture had the biggest affect followed by hydro-static pressure, then EC. Time of day and pH were found to be insignificant and were not predictors in the best models.
EC was the most important independent variable, likely due to the direct dependence of EC on ionic strength of a solution. Model predictions also were sensitive to the inclusion of FDOM which is likely due to FDOM being an effective proxy for dissolved organic carbon. Nitrate predictions were also sensitive to pH, prehaps because most stream nitrate at Hubbard Brook is produced by nitrification, which is an acidifying process. Turbidity was found to be of little importance, which is likely due to it being a proxy for suspended material rather than dissolved solutes.
\newline
\newline
As mentioned earlier Kermorvant et al \cite{kermorvant2021understanding} found the most significant predictor variable varied greatly between sites, with EC, temperature, time and water level each being the best predictor for each site. 
\newline
\newline
Due to the complicated non-linear interaction between the predictor variables, the choice of variables cannot be considered independent. Generally EC has found to be the most important predictor variable, however this is not always the case and it can be observed that it is beneficial to capture a range of water quality parameters. Additionally the work done by Harrison et al, finds that there is benefit to incorporating soil moisture and hydro-static pressure sensors, they also find that lagged and delta values are important. It can also be observed that there is benefits to incorporating time of day and time of year as predictor variables, however this depends how much consistent seasonal and daily variation the site has. Further research would help to understand how well models can generalize to represent a wide range of catchments.

\section{Experiments}
\subsection*{Evaluation Metrics}
To evaluate the performance of the various nitrate models, different metrics are used.
\newline 
\newline
\textbf{R-Squared (A.K.A. NSE):} Is a measure (between -1 and 1) of how well the model can explain the variance of the data (goodness of fit). Where \(\bar{y}\) is the mean of the ground truth data, \(y\) is the ground truth value and \(\hat{y}\) is the model prediction. A R-Squared of 0 can be considered taking the mean value (i.e. anything close to zero or negative is considered poor).


%Consider what R2
% means: proportion of variability explained, compared to a baseline model that always guesses the average value of the pooled response variable.

%If you’re higher than R2=0
%, which you probably will be with in-sample data when you use an intercept, then you’re beating the baseline %performance.



\[R^2=1-\frac{\sum_{i=1}^n (y_i - \hat{y})^2}{\sum_{i=1}^n ({y_i} - \bar{y})^2}\]
\newline
\newline
\textbf{Mean Absolute Error:} Is a measure of the absolute difference between the predicted value and the ground truth value.
\[MAE=\frac{1}{N}\sum_{i=1}^{N}|y_i-\hat{y}_i|\]
\newline
\newline
\textbf{Mean Absolute Percentage Error:} Is a measure of the absolute difference between the predicted value and the ground truth value over the ground truth value.
\[MAPE=\frac{1}{N}\sum_{i=1}^{N}\frac{|y_i-\hat{y}_i|}{\hat{y}_i}\]
\newline
\newline
\textbf{Mean Squared  Error:}  Is a measure of the difference between the predicted value and the ground truth value but it penalizes larger errors more heavily.
\[MSE=\frac{1}{N}\sum_{i=1}^{N}(y_i-\hat{y}_i)^2\]
\newline
\newline
\textbf{Root Mean Squared  Error:} Is a measure of the difference between the predicted value and the ground truth value but it penalizes larger errors more heavily but it takes the root of this value to put it in a more interpretable form.
\[RMSE=\sqrt{\frac{1}{N}\sum_{i=1}^{N}(y_i-\hat{y}_i)^2}\]
\newline
\newline
\textbf{Normalized Root Mean Squared  Error:} Is a measure of the difference between the predicted value and the ground truth value but it penalizes larger errors more heavily but it takes the root of this value to put it in a more interpret-able form (as a percentage). In our experiments we normalize by the mean.
\[nRMSE=\frac{1}{\hat{y}}\sqrt{\frac{1}{N}\sum_{i=1}^{N}(y_i-\hat{y}_i)^2}\]
\newpage
\subsection*{Datasets}
To be able to undertake experimental testing we collate data-sets from previous research. The data-sets include the Arikaree River (from NEON) and the River Enborne dataset.
\newline \newline
The Aikaree River is located in the central plains of North Eastern Colorado. It has a large watershed size of 2631.8 \(km^2\) (65,332 acres). The area is sensitive to invasive plant species. The Arikaree is also impacted by agricultural runoff and cattle grazing. In the winter the river is a free-flowing stream and dries into standing pools in summer. The area can be subject to tornados, flooding, blizzards, and severe winter storms. The Arikaree River flows through a sand-alluvial basin and its bed and banks are composed mostly of sand. The grassy shores of Arikaree River are dominated by grasses. Blue grama grass (Bouteloua gracilis), needle-and-thread grass (Hesperostipa comata), and big bluestem (Andropogon gerardii) grow strong here among others. Data is sampled at 15-minute intervals and includes; NO3, EC, DO, pH, chlorophyll, turbidity, fDOM and temperature. This data-set contains data from 2018-09-13 till 2019-12-31 (474 days) with 14721 valid samples.

\begin{table}[h]
\begin{center}
\begin{tabular}{@{}lllllll@{}}
\toprule
\multicolumn{7}{c}{Aikaree (n=14721)}                                    \\ \midrule
     & NO3 (mg/l) & Turb (FNU) & Temp (C) & EC (uS/cm) & pH    & DO (\%)\\
mean & 0.26      & 38.31     & 11.02    & 547.05     & 7.75  & 53.16   \\
std  & 0.15      & 228.08  & 9.3       & 46.93      & 0.32 & 23.41    \\
min  & 0.006     & 0.01      & -0.644     & 374.00     & 7.18  & 31.65   \\
max  & 1.097     & 3797.0     & 31.52    & 709.75     & 8.41  & 112.92  \\ \bottomrule
\end{tabular}
\end{center}
\end{table}
The River Enborne is a relatively rural river, near Brimpton. It has some small population centres. The relief of the Enborne catchment is characterised by gently sloping valleys with a maximum altitude of 296 m above sea level.
The dataset contains hourly measurements of total reactive phosphorus, nitrate, conductivity, temperature, dissolved oxygen, pH and total chlorophyll. Throughout the monitoring program, weekly “grab-samples” were collected at the Brimpton site. In detail analysis of this data is conducted thoroughly in \cite{halliday2014water}. This data set has 12723 valid samples from 2009-12-08 till 2011-12-19 (859 days) .

\begin{table}[h]
\begin{center}
\begin{tabular}{@{}llllllll@{}}
\toprule
\multicolumn{8}{c}{Enborne (n=12723)}                                    \\ \midrule
     & NO3 ($mg\/l$) & Turb (NTU) & Temp (C) & EC (uS/cm) & pH    & DO (\%) & Flow ($m^3\/s$) \\
mean & 4.05       & 8.05       & 10.8     & 486.52     & 7.99  & 90.27   & 0.97 \\
std  & 0.74       & 10.27      & 4.4      & 79.63      & 0.224 & 15.58   & 1.32 \\
min  & 1.69       & 2.00       & 0.30     & 215.00     & 7.24  & 53.00   & 0.10\\
max  & 6.24       & 215.8      & 19.80    & 769.00     & 8.94  & 170.00  & 12.60\\ \bottomrule
\end{tabular}
\end{center}
\end{table}

\subsection{How much training data is required?}
Due to each river having a unique complex non-linear  relationship between the water quality parameters and the nitrate concentration; it is important to understand how much data is required to get a model that suitably fits each river. Additionally it is important to understand what is the best way to temporally collect the data (i.e. can the data be collected at a high frequency in one month or does it need collected through the year). 
\newline \newline
The Aikaree River data-set has many missing rows which is due to the extreme weather conditions and sensor reliability. Therefore, for the tests conducted with this data set we remove the cases in which we cannot get at least 50\% of the training samples for that time period.  
\newline \newline
Although there is multiple options of which model to use for this experiment, a logical choice is to use a RF model (with 200 trees). As it is straight forward to implement, able to accurately represent non-linear relationships and has support from previous literature about it's viability. Therefore it is a sensible model to use as a baseline for some initial testing. 
\newline \newline
For the following experiments we set a amount of days for the training data and an amount of days for the test days. We then shift the training and testing days by 1 (move a day forward in time) and repeat this process until we have reached the end of the data-set. This allows us to perform the experiment across the full data-set, thereby improving the reliability of our results. For each split of the data-set we evaluate our model using the prior mentioned metrics. For the Enborne dataset we use NO3, EC, pH, do and turbidity as predictor variables. 

\begin{table}[h]
\begin{center}
\begin{tabular}{@{}lllll@{}}
\toprule
\multicolumn{5}{c}{Enborne Temporal Analysis (30 days of training)}                                    \\ \midrule
Days After:    & 0-30 & 30-60  & 60-180  & 180-360 \\ \hline
$R^2$   &  0.715  & $-0.272$     & 0.028    & 0.039       \\
MAE (mg/l)  & 0.176      & 0.308      & 0.423    & 0.485      \\
MAPE (\%) & 4.6       & 7.3       & 10.4     & 12.2       \\
MSE  & 0.068       & 0.157     & 0.315    &  0.372 \\
RMSE (mg/l) & 0.243       & 0.384     & 0.556    & 0.607     \\ 
nRMSE (\%)  & 0.065   & 0.087     & 0.131   & 0.154     \\\bottomrule
\end{tabular}
\end{center}
\end{table}

As we can see from the table above, the The $R^2$ results are acceptable for the following 0-30 days. 
However for the remaining time periods, the results are unacceptable. The $R^2$ scores are close to zero and negative, this means that the model predictions are worse than a horizontal mean line (average of the training values - $\bar{y}$). Showing that the model is unable to explain the variability in the data. This is likely due to the seasonal variability as-well as extreme weather events that occur. However we can see that our MAE, MAPE, MSE, RMSE and nRMSE all seem to be within acceptable ranges. This is interesting and shows the effect of having data that has little std deviation.

% \begin{table}[h]
% \begin{center}
% \begin{tabular}{@{}lllll@{}}
% \toprule
% \multicolumn{5}{c}{Enborne Temporal Analysis (30 days of training)}                                    \\ \midrule
% Days After:    & 0-30 & 30-60  & 60-180  & 180-360 \\ \hline
% $R^2$   &  0.715  & $-0.272$     & 0.028    & 0.039       \\
% MAE (mg/l)  & 0.176      & 0.308      & 0.423    & 0.485      \\
% MAPE (\%) & 0.046       & 0.073       & 0.104     & 0.122       \\
% MSE  & 0.068       & 0.157     & 0.315    &  0.372 \\
% RMSE (mg/l) & 0.243       & 0.384     & 0.556    & 0.607     \\ 
% nRMSE (\%)  & 6.5   & 8.7     & 13.1   & 15.4     \\\bottomrule
% \end{tabular}
% \end{center}
% \end{table}


\section*{Results}


\section*{Conclusion}
Current research proposes solutions which appear promising however, only a few studies determine how well the solution would work by training with data that precedes the testing data. The research that does propose using training data that precedes the testing data is only able to achieve acceptable results for a short period (less than 60 days). Although if the training data was collected over a long period of time (2 years +) it might be possible to achieve better long term results. Other research that has proposed to use K-Fold cross-validation, leads to the training data set being K-1 times bigger than the testing set (training set is considerably bigger than the testing set). This is impractical to implement as we require the training data to be smaller to ensure our solution is cost-effective to implement.  
\newline 
\newline
A few solutions provide an interesting idea of using manual grab samples (usually weekly) in parallel with a machine learning model, to essentially augment the data from weekly resolution into hourly resolution. This seems to be reasonably effective, however it seems better suited as a post-processing stage as a opposed to a solution that can provide real-time information. 
\newline
\newline
Regarding machine learning model choice, Random Forest regression is a good starting point (XGBoost potential next step). However, the main bottleneck for any model is getting appropriate data. Getting suitable data for Rivers is challenging due to their dynamic nature. Furthermore, models fail to extrapolate when receiving values outside of their training data - meaning that it would be difficult for a model to work accurately in extreme cases (unless appropriate training data was provided).


\bibliographystyle{splncs04}
\bibliography{ref}


\end{document}

