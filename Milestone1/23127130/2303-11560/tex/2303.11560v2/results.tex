\section{Results} \label{results_section}
Performance evaluation of skeletonization algorithms is incredibly challenging and remains an open problem. Hence, there is no widely accepted metric used for evaluation. We compare our algorithms skeleton for quantitative evaluation using a form of precision and recall which matches points along the ground truth skeleton against points along the estimated skeleton.

We evaluate our method against the state-of-the-art AdTree algorithm \cite{du2019adtree}. As our metrics do not evaluate topological errors directly, additional qualitative analysis is conducted by visually inspecting the algorithm outputs against the ground truth.

Due to augmentations, some of the finer branches may become excessively noisy or disappear. To ensure the metrics measure what is possible to reconstruct, we prune the ground truth skeleton and point cloud based on a branch radius and length threshold - respective to tree size. 

\subsection{Metrics}

We evaluate our skeletons using point cloud metrics by sampling our skeletons at a 1mm resolution.
We use the following metrics to assess our approach: f-score, precision, recall and AUC. 
For the following metrics, we consider \(p \in \mathcal{S} \) points along the ground truth skeleton and \(p^* \in \mathcal{S}^* \) points obtained by sampling the output skeleton. \( p_r \) is the radius at each point. 
We use a threshold variable \({t}\), which sets the distance points must be within based on a factor of the ground truth radius. We test this over the range of \(0.0 - 1.0\). The f-score is the harmonic mean of the precision and recall. 

 \subsubsection{Skeletonization Precision}
To calculate the precision, we first get the nearest points from the output skeleton $p_i \in \mathcal{S}$ to the ground truth skeleton $p_j^* \in \mathcal{S^*}$, using a distance metric of the euclidean distance relative to the ground truth radius $r_j^*$.
The operator $\llbracket . \rrbracket$ is the Iverson bracket, which evaluates to 1 when the condition is true; otherwise, 0.

\begin{equation}
\begin{split}
d_{ij} = ||p_i - p_j^*||
\end{split}
\end{equation}
\begin{equation}\label{precision}
\begin{split}
P(t) = \frac{100}{|S|} \sum_{i \in \mathcal{S}} \llbracket  d_{ij} < t \ r_j^* \land \mathop{\forall}_{k \in \mathcal{S}} d_{ij} \leq d_{kj} \rrbracket 
\end{split}
\end{equation}
\subsubsection{Skeletonization Recall}
To calculate the recall, we first get the nearest points from the ground truth skeleton $p_j^* \in \mathcal{S^*}$ to the output skeleton $p_i \in \mathcal{S}$. We then calculate which points fall inside the thresholded ground truth radius. This gives us a measurement of the completeness of the output skeleton.
 
%At most one point $p^*$ (included in the set of recall true positives ($Q$))  is allowed to match the same $p$ where the distance is less than %the threshold. 

\begin{equation}\label{recall_truth_positives}
\begin{split}
R(t) = \frac{100}{|S^*|} \sum_{j \in \mathcal{S^*}} \llbracket  d_{ij} < t \ r_j^* \land \mathop{\forall}_{k \in \mathcal{S^*}} d_{ij} \leq d_{ik} \rrbracket 
\end{split}
\end{equation}

\subsection{Quantitative Results}
We evaluate our method on our test set of sixty synthetic ground truth skeletons (made up of six species). Our results are summarized in Table \ref{table:results} and illustrated in Figure \ref{fig:result-graphs}. We compute the AUC for F1, precision, and recall using the radius threshold \({t}\) ranging from 0 to 1.
\begin{table}[htp]
\centering
\caption{Skeletonization Results.}
\begin{tabular*}{\textwidth}{
     P{0.30\textwidth} |                 
     P{0.30\textwidth} |                 
     P{0.30\textwidth} }
Metric & Smart-Tree & AdTree \\
\hline
Precision AUC & 0.53 & 0.21\\
Recall AUC & 0.40 & 0.38\\
F1 AUC & 0.45 & 0.26 
\end{tabular*}
\label{table:results}
\end{table}

Smart-Tree achieves a high precision score, with most points being close to the ground truth skeleton (Fig. \ref{fig:result-graphs}a). 
Compared to AdTree, Smart-Tree has lower recall at the most permissive thresholds, and this is due to Smart-Tree's inability to approximate missing regions of the point cloud, making it prone to gaps. AdTree, on the other hand, benefits from approximating missing regions. However, this also leads to AdTree having more topological errors and lower precision. Smart-Tree consistently achieves a higher F1 score and AUC for precision, recall, and F1, respectively. 
% We plan to curate an annotated set of real point clouds in the future, once we have acquired data from a range of species. 
\begin{figure}
\centering
\begin{tabular}{ccc}
  \includegraphics[width=40mm]{images/results-graphs/precision_threshold.png} &   \includegraphics[width=40mm]{images/results-graphs/recall_threshold.png}  &
 \includegraphics[width=40mm]{images/results-graphs/f1_threshold.png} \\
\end{tabular}
\caption{Left to right: Precision Results (a), Recall Results (b), F1 Results (c).}
\label{fig:result-graphs}
\vspace{-2mm}%Put here to reduce too much white space after your table 
\end{figure}



% \begin{table}[htp ]
% \centering

% \caption{Skeletonization Results Across Thresholds.}

% \medskip
% \begin{tabular*}{\textwidth}{
%     P{0.12\textwidth}                   % Column 1
%     P{0.14\textwidth}                   % Column 1
%     P{0.14\textwidth}                   % Column 1
%     P{0.14\textwidth}                   % Column 1
%     P{0.14\textwidth}                   % Column 1
%     P{0.14\textwidth}                   % Column 1
%     P{0.14\textwidth}                   % Column 1
% }

% \toprule
% Threshold \({t}\)
% & \multicolumn{2}{c}{Precision Mean (\%)} 
% & \multicolumn{2}{c}{Recall Mean (\%)} 
% & \multicolumn{2}{c}{F1  Mean (\%)} \\
% \cmidrule(lr){2-3} \cmidrule(l){4-5} \cmidrule(l){6-7}
%  &  
% {AdTree} &   
% {Smart-Tree} & 
% {AdTree} &   
% {Smart-Tree} & 
% {AdTree} &
% {Smart-Tree}\\
% \midrule
% 0.1&0.54&9.75&1.98&6.73&0.89&7.80 \\
% \midrule
% 0.2&2.23&28.86&7.98&19.12&3.36&22.52 \\
% \midrule
% 0.3&5.05&45.77&17.06&29.27&7.53&34.95 \\
% \midrule
% 0.4&9.09&57.32&28.03&35.79&13.30&43.12 \\
% \midrule
% 0.5&14.54&64.64&39.58&39.78&20.70&48.15 \\
% \midrule
% 0.6&21.89&69.42&50.87&42.48&29.91&51.50 \\
% \midrule
% 0.7&32.27&72.89&60.72&44.64&40.40&54.10 \\
% \midrule
% 0.8&42.22&75.64&68.33&46.45&51.08&56.23 \\
% \bottomrule
% \label{table:results}
% \end{tabular*}
% \vspace{-15mm}%Put here to reduce too much white space after your table 
% \end{table}

\subsection{Qualitative Results}
In Figure \ref{fig:qualitiative}, we show qualitative results for each species in our synthetic dataset. We can see that Smart-Tree produces an accurate skeleton representing the tree topology well. Adtree produces additional branches that would require post-processing to remove. 

AdTree often fails to capture the correct topology of the tree. This is due to Adtree using Delaunay triangulation to initialise the neighbourhood graph. This can lead to branches that have no association being connected. 

Smart-Tree, however, does not generate a fully connected skeleton - but rather one with multiple sub-graphs. The biggest sub-graph can still capture the majority of the major branching structure, although to provide a full topology of the tree with the finer branches, additional work is required to connect the sub-graphs by inferring the branching structure in occluded regions. 

%\subsection{Real Data}
To demonstrate our method's ability to work on real-world data. We test our method on a tree from the Christchurch Botanic Gardens, New Zealand. As this tree has foliage, we train our network to segment away the foliage points and then run the skeletonization algorithm on the remaining points. In Figure \ref{fig:botanic-real}c, we can see that Smart-Tree can accurately reconstruct the skeleton. 
\begin{figure}[H]
\centering
\begin{tabular}{ccc}
\includegraphics[width=0.29\textwidth, trim={10cm 0cm 12cm 0cm},clip]{images/botanic/botanic-pcd.png}&
\includegraphics[width=0.29\textwidth, trim={10cm 0cm 12cm 0cm},clip]{images/botanic/botanic-branch-mesh.png}& 
\includegraphics[width=0.29\textwidth, trim={10cm 0cm 12cm 0cm},clip]{images/botanic/botanic-skeleton.png}\\ 
\end{tabular}
\caption{Left to Right: Input point cloud (a), Branch meshes (b), Skeleton Sub-graphs (c).}
\label{fig:botanic-real}
\vspace{-10mm}%Put here to reduce too much white space after your table 
\end{figure}


% To demonstrate our method's ability to work on real-life data. We test our method on a Kowhai tree point cloud that was captured using a Phantom 4 RTK Drone (Figure \ref{fig:kowhai-real}).

% As the Kowhai tree contains foliage points, we train our network to segment the leaves from the wood as well as estimate the direction and radius. In Figure \ref{fig:kowhai-real}c, we can see that Smart-Tree can accurately reconstruct the skeleton. 

% \begin{figure}[H]
% \centering
% \begin{tabular}{ccc}
%   \includegraphics[width=32mm]{images/kowhai-pcd.png} &   \includegraphics[width=32mm]{images/kowhai-surface.png} & \includegraphics[width=32mm]{images/kowhai-skel.png} \\
% \end{tabular}
% \caption{Left to Right: Input point cloud (a), Allocated branch surface points (b), Skeleton output (c).}
% \label{fig:kowhai-real}
% %\vspace{-10mm}%Put here to reduce too much white space after your table 
% \end{figure}

% Furthermore, we qualitatively compare our method with Adtree, using a real-life apple tree dataset supplied by Straub et al. \cite{straub2022approach}.


\begin{figure}[H]
\centering
\begin{tabular}{cccc}
  \includegraphics[width=0.24\textwidth]{images/results/cherry-4-pcd.png} &  
  \includegraphics[width=0.24\textwidth]{images/results/cherry-4-gt.png}  &
  \includegraphics[width=0.24\textwidth]{images/results/cherry-4-smart.png}  &
  \includegraphics[width=0.24\textwidth]{images/results/cherry-4-adtree.png}  \\
  \includegraphics[width=0.24\textwidth]{images/results/euc-74-pcd.png} &  
  \includegraphics[width=0.24\textwidth]{images/results/euc-74-gt.png}  &
  \includegraphics[width=0.24\textwidth]{images/results/euc-74-smart.png}  &
  \includegraphics[width=0.24\textwidth]{images/results/euc-74-adtree.png}  \\
  \includegraphics[width=0.24\textwidth]{images/results/walnut-7-point-cloud.png} &  
  \includegraphics[width=0.24\textwidth]{images/results/walnut-7-gt-skeleton.png}  &
  \includegraphics[width=0.24\textwidth]{images/results/walnut-7-smart-tree.png}  &
  \includegraphics[width=0.24\textwidth]{images/results/walnut-7-adtree.png}  \\
  \includegraphics[width=0.24\textwidth]{images/results/apple-53-pcd.png} &  
  \includegraphics[width=0.24\textwidth]{images/results/apple-53-gt.png}  &
  \includegraphics[width=0.24\textwidth]{images/results/apple-52-smart.png}  &
  \includegraphics[width=0.24\textwidth]{images/results/apple-53-adtree.png}  \\
  \includegraphics[width=0.24\textwidth]{images/results/pine-11-pcd.png} &  
  \includegraphics[width=0.24\textwidth]{images/results/pine-11-gt.png}  &
  \includegraphics[width=0.24\textwidth]{images/results/pine-11-smart-tree.png}  &
  \includegraphics[width=0.24\textwidth]{images/results/pine-11-adtree.png}  \\
  \includegraphics[width=0.24\textwidth]{images/results/ginkgo-77-pcd.png} &  
  \includegraphics[width=0.24\textwidth]{images/results/ginko-77-gt.png}  &
  \includegraphics[width=0.24\textwidth]{images/results/ginkgo-77-smart.png}  &
  \includegraphics[width=0.24\textwidth]{images/results/ginkgo-77-adtree.png}  \\
\end{tabular}
\caption{Several results of skeletonization of synthetic point clouds. From left to right: synthetic point cloud, ground truth skeleton, Smart-Tree skeleton, and Adtree skeleton. From top to bottom (species): cherry, eucalyptus, walnut, apple, pine, ginkgo}
\label{fig:qualitiative}
\vspace{-1.2cm}
\end{figure}