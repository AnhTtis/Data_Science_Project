\section{Conclusion and Future Work} \label{conclusion_section}
We proposed Smart-Tree, a supervised method for generating skeletons from tree point clouds. A major area for improvement in the literature on tree point-cloud skeletonization is quantitative evaluation, which we contribute towards with a synthetic tree point cloud dataset with ground-truth and error metrics.

We demonstrated that using a sparse convolutional neural network can help improve the robustness of tree point cloud skeletonization. One novelty of our work is that a neighbourhood graph can be created based on the radius at each region, improving the accuracy of our skeleton. 

We used a precision and recall-based metric to compare Smart-Tree with the state-of-the-art AdTree. Smart-Tree is generally much more precise than AdTree, but it currently does not handle point clouds containing gaps (due to occlusions and reconstruction errors). AdTree has problems with over-completeness on this dataset, with many duplicate branches. 

In the future, we would like to work towards robustness to gaps in the point cloud by filling gaps in the medial-axis estimation phase. We plan to train our method on a wider range of synthetic and real trees; to do this, we will expand our dataset to include more variety, trees with foliage, and human annotation on real trees. We are also working towards error metrics which better capture topology errors.