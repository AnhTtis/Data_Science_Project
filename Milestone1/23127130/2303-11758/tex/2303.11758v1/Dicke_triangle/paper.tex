\documentclass[pra,twocolumn,showpacs,superscriptaddress,amssymb]{revtex4}


\usepackage[colorlinks=true,linkcolor=magenta,citecolor=magenta,urlcolor=magenta,linktocpage=true]{hyperref}
\usepackage[utf8]{inputenc}
\usepackage[english]{babel}
\usepackage[T1]{fontenc}
\usepackage{amsmath}
\usepackage{cleveref}
\usepackage{svg}
\usepackage{tabularx}
\usepackage{array, makecell}
\usepackage{multirow}
\renewcommand\theadfont{\bfseries}


\usepackage{listings}
\usepackage{xcolor}

\lstdefinestyle{mystyle}{
    backgroundcolor=\color{backcolour},   
    commentstyle=\color{codegreen},
    keywordstyle=\color{magenta},
    numberstyle=\tiny\color{codegray},
    stringstyle=\color{codepurple},
    basicstyle=\ttfamily\footnotesize,
    breakatwhitespace=false,         
    breaklines=true,                 
    captionpos=b,                    
    keepspaces=true,                 
    numbers=none,                    
    numbersep=5pt,                  
    showspaces=false,                
    showstringspaces=false,
    showtabs=false,                  
    tabsize=2
}
\lstset{style=mystyle}

\usepackage{psfrag,graphicx}
\usepackage{dcolumn}
\usepackage{bm}
\usepackage{amsfonts,amssymb,amsmath}        
\usepackage{slashed}
\usepackage[utf8]{inputenc}
\usepackage{graphicx}
\usepackage[caption=false]{subfig}
\usepackage{cancel}
\usepackage{xcolor}
\usepackage{amsmath}
\setcitestyle{numbers,square}

\newcommand\numberthis{\addtocounter{equation}{1}\tag{\theequation}}


\newenvironment{red}{\color{red}}
    
\newenvironment{blue}{\color{blue}}

\newcommand\myeqT{\mathrel{\overset{\makebox[0pt]{\mbox{\normalfont\tiny\sffamily $t>0$}}}{=}}}

\newcommand\myeqTT{\mathrel{\overset{\makebox[0pt]{\mbox{\normalfont\tiny\sffamily $t<0$}}}{=}}}

%%%%%%%%%%%%%%%%%%%%%%%%
\makeatletter

\renewenvironment{widetext@grid}{%
  \par\ignorespaces
  \setbox\widetext@top\vbox{%
   \vskip15\p@
   \hb@xt@\hsize{%
    \leaders\hrule\hfil
    \vrule\@height6\p@
   }%
   \vskip6\p@
  }%
  \setbox\widetext@bot\hb@xt@\hsize{%
    \vrule\@depth6\p@
    \leaders\hrule\hfil
  }%
  \onecolumngrid

  \let\set@footnotewidth\set@footnotewidth@ii
}{%
  \par

  \twocolumngrid\global\@ignoretrue
  \@endpetrue
}%

\makeatother
%%%%%%%%%%%%%%%%%%%%%%%%%%%%%%




\begin{document}
\title{Novel Critical Scalings and Finite Critical Fluctuations Across the Frustrated Superradiant Phase Transition}

\author{Cheng Zhang}
\affiliation{Key Laboratory for Microstructural
Material Physics of Hebei Province, School of Science, Yanshan
University, Qinhuangdao 066004, China}

\author{Pengfei Liang}
\email{pfliang@gscaep.ac.cn}
\affiliation{Graduate School of China Academy of Engineering Physics, Haidian District, Beijing, 100193, China}

\author{Neill Lambert}
\email{nwlambert@gmail.com}
\affiliation{Theoretical Quantum Physics Laboratory, Cluster for Pioneering Research, RIKEN, Wakoshi, Saitama 351-0198, Japan}

\author{Mauro Cirio}
\email{cirio.mauro@gmail.com}
\affiliation{Graduate School of China Academy of Engineering Physics, Haidian District, Beijing, 100193, China}



\date{\today}

\begin{abstract}
We introduce a generalized frustrated Dicke trimer model where three Dicke models are coupled in sequence via direct photon hopping and investigate the stability of a recently found frustrated superradiant phase and the associated two critical scalings [Phys.~Rev.~Lett.~128,~163601] in the presence of two symmetry-breaking perturbations. The first type of perturbation breaks time-reversal symmetry by synthesizing an artificial gauge field in the cavity and is manifested in the phase of the photon hopping amplitude. We find that this type of perturbation demolishes the two critical scalings but allows the emergence of a new scaling behavior with unconventional exponent $1.5$ in the frustrated superradiant phase and {\it finite} critical fluctuations when approaching the critical point from the normal phase. Remarkably, the two critical scalings appear again at a tricritical point in {\it both} the normal phase and superradiant phase. The second type of perturbation comes from tuning the anisotropy of the Dicke model, such that in the isotropic case the $Z_2$ parity symmetry of the Dicke model turns into a $U(1)$ continuous symmetry of the Tavis-Cummings model. We find the emergence of a zero-energy mode in the superradiant phase which can be attributed to a phase redundancy of the ground state solutions in the isotropic model. 

\end{abstract}

\maketitle


%%%%%%%%%%%%%%%%%%%%%%%%%%%%%%%%%%%%%%%%%%%%%%%%%%%%%%%%%%%%%%%%%%%%%%%%%%%%%%%%%%%%%%%%%%%%%%%%%%%%%%%%
\section{Introduction}\label{sec:intro}

The theory of critical phenomena lies at the heart of our understanding of quantum phase transitions (QPTs) \cite{RevModPhys.69.315,qptbook}. Continuous QPTs occur at zero temperature and exhibit a number of unique characteristics, including the presence of degenerate ground states with spontaneously broken symmetries and the closing of the spectral gap. 
A QPT is normally associated with some diverging length and time scales. As a consequence, both the statistics and the dynamics near the transition are characterized by universal scaling laws which are independent from the microscopic details of the model. In turn, this allows to classify different QPTs according to critical exponents which can be used to describe the scaling of the divergent properties \cite{scalingbook}. 

Due to considerable progress in the experimental control and manipulation of  quantum degrees of freedom, quantum systems made of bosonic modes, spins, and atomic ensembles have emerged as promising platforms for exploring QPTs and the associated critical phenomena. A paradigmatic example of such systems is the Dicke model~\cite{https://doi.org/10.1002/qute.201800043}, where a single bosonic mode is homogeneously coupled to a large ensemble of two-level atoms via  a dipole interaction. In this system, the atoms can coherently and constructively interact with light, leading to enhanced levels of radiation in the ground state~\cite{PhysRev.93.99} or the steady state~\cite{PhysRevA.7.831,PhysRevA.8.1440,PhysRevA.8.2517,PhysRevA.98.063815} of dissipative-driven systems. This prototypical Dicke model undergoes a superradiant phase transition (SPT) characterized by mean-field critical exponents when the light-matter coupling is comparable to the frequencies of the bosonic mode and atoms. Experimental realizations of the Dicke model and SPT have been achieved in cavity QED systems \cite{Baumann2010,PhysRevLett.107.140402,doi:10.1073/pnas.1417132112,PhysRevLett.121.163601,doi:10.1126/science.abd4385}, trapped ions \cite{PhysRevLett.121.040503, doi:10.1126/science.abi5226} and ultracold atoms in a cavity \cite{PhysRevLett.91.203001,Zhiqiang2018DickemodelSV}.

In this context, a recent paper by Zhao and Hwang~\cite{PhysRevLett.128.163601} proposes the realization of a "frustrated superradiance" phase by placing an odd number of Dicke models in a ring geometry and allowing them to interact by photon hopping. In the superradiant phase, the ground state energy function of each Dicke model is a double-well potential and the resulting macroscopic classical cavity field can, intuitively, be interpreted as an Ising spin with a variable amplitude. From this point of view, the photon hopping effectively acts as a magnetic exchange-coupling which introduces frustration in the antiferromagnetic case. Ultimately, this leads to the existence of a frustrated superradiant phase (FSP) characterized by the breaking of translational symmetry \cite{PhysRevLett.128.163601}. Interestingly, the phase transition to the normal phase (NP) exhibits a novel scaling feature in which critical exponents of mean-field and unconventional type coexist.

In this work we analyze the FSP, its stability, and its critical scaling in the presence of two types of symmetry breaking perturbations.
The first type of perturbation breaks the time-reversal symmetry and is introduced via a phase factor in the photon hopping, which may be produced by imposing an artificial gauge field in the cavity ring. This is not purely out of theoretical interest since artificial gauge fields have been successfully created in cold atomic gases \cite{Lin2009,RevModPhys.83.1523,PhysRevA.89.013610} and photonic systems \cite{PhysRevLett.108.206809,PhysRevLett.116.220502,Roushan2017} in recent years. We note that a similar consideration was made in the context of the Rabi triangle model in Ref.~\cite{PhysRevLett.127.063602}, where some new phases are found in the presence of artificial gauge fields. The second type of perturbation comes from tuning the anisotropy of the Dicke model, such that rotating and counter-rotating terms in the coupling compete with each other. Previous works~\cite{PhysRevLett.119.220601,PhysRevLett.120.183603} have revealed that such a competition may produce significant consequences such as a tricritical point and the appearance of new critical points. 
We remark that in the isotropic case where the Tavis-Cummings model is reached, the $Z_2$ parity symmetry of the Dicke model breaks into a $U(1)$ continuous symmetry related to the conservation of excitation quanta. We find that the FSP survives both perturbations while the critical scalings across the frustrated SPT crucially depends on the underlying symmetry. Remarkably in the isotropic case a zero energy mode arises in both the FSP and non-frustrated superradiant phase (nFSP) originating from an easy axis in the phase of the macroscopic cavity field. 

For consistency, throughout this article, we keep most of our notations compatible with the ones in  Ref.~\cite{PhysRevLett.128.163601}.


%%%%%%%%%%%%%%%%%%%%%%%%%%%%%%%%%%%%%%%%%%%%%%%%%%%%%%%%%%%%%%%%%%%%%%%%%%%%%%%%%%%%%%%%%%%%%%%%%%%%%%%%
\section{The Generalized Dicke Trimer}\label{sec:model}

We start by introducing the Hamiltonian for the generalized Dicke trimer model as
\begin{eqnarray}
H_N = \sum_{n=1}^N H_n^{\mathrm{Dicke}} + J(e^{i\varphi}a_n^\dagger a_{n+1}+e^{-i\varphi}a_{n+1}^\dagger a_n),
\end{eqnarray}
where $N=3$, and the Hamiltonian for the Dicke model as
\begin{eqnarray}
H_n^{\mathrm{Dicke}} = \omega_0 a_n^\dagger a_n + \Omega J_n^z + \frac{2\sqrt2\lambda}{\sqrt{N_a}}\Big[\eta_+q_nJ_n^x + \eta_-p_nJ_n^y \Big]. \nonumber\\
\end{eqnarray}
Here, $\eta_\pm = (1\pm\eta)/2$ and $\lambda$ tune the anisotropy and strength of the Dicke coupling, respectively. We take periodic boundary condition $a_{N+1} = a_1$,  where $a_n$ is the annihilation operator of the  bosonic resonant mode in the $n$th cavity. This mode has frequency $\omega_0$ and its position and momentum quadratures are defined by $q_n=(a_n+a_n^\dagger)/\sqrt2, p_n=i(a_n-a_n^\dagger)/\sqrt2$. The atomic ensemble in each cavity is made of $N_a$ two-level atoms with frequency $\Omega$, and it is described by the collective spin operators $J_n^{x,y,z}=\sum_{m=1}^{N_a} s_m^{x,y,z}$ with $s_m^{x,y,z}$ representing a single spin half. Without loss of generality, we assume $J>0$ and the phase $\varphi\in[0,\pi]$  to interpolate between the antiferromagnetic ($\varphi=0$) and ferromagnetic ($\varphi=\pi$) limits which were already thoroughly investigated in Ref.~\cite{PhysRevLett.128.163601}. 
The anisotropy parameter $\eta\in[-1,1]$ is introduced to further interpolate between the isotropic ($\eta=0$) and anisotropic ($\eta\neq0$) cases. In the former case $H_n^{\mathrm{Dicke}}$ owns a $U(1)$ symmetry defined by $G_\zeta=\prod_n \exp{[i\zeta(a_n^\dagger a_n + J_n^z + N_a/2)]}$ with $\zeta\in\mathbb{R}$,  while in the latter case only the $Z_2$ parity symmetry, described by $G_\pi$, is left. We will see later this causes important effects in the excitation spectra. Apart from this symmetry, $H_N$ possesses the $Z_N$ translational symmetry $T$ which acts as $Ta_n(J_n^{x,y,z})T=a_{n+1}(J_{n+1}^{x,y,z})$ and satisfies $T^N=1$ due to periodic boundary condition. At $\varphi=0,\pi$, $H_N$ also has time-reversal symmetry $K$ with $K$ representing the complex conjugate operation. 

To solve the model, it is customary to separate the classical, linear and quadratic parts of $H_N$ by a rotation of the collective spins and a displacement of the cavity field specified with the unitary operator
\begin{eqnarray}
\label{eq:U}
U = \prod_n e^{-i\phi_nJ_n^z}e^{-i\theta_nJ_n^y}e^{\alpha_na_n^\dagger-\alpha_n^*a_n},
\end{eqnarray}
where $\theta_n,\;\phi_n$ are spherical coordinates of the collective spins and $\alpha_n$ is the classical displacement of the $n$th cavity field. The resulting rotated and displaced Hamiltonian is (see Appendix.~\ref{sec:clq} for detailed derivation and the expression for $H_\text{l}$)
\begin{eqnarray}
\label{eq:decom}
U^\dagger H_N U = E_{\text{GS}} + H_\text{l} + H_\text{q},
\end{eqnarray}
where
\begin{eqnarray}\label{eq:EGS}
\frac{E_{\text{GS}}}{N_a\Omega} &=& \sum_n\Big[ \lvert\bar\alpha_n\rvert^2+ \frac{1}{2}\cos\theta_n+g\sin\theta_n(\eta_+\cos\phi_n\Re\bar\alpha_n \nonumber\\
&&+g\eta_-\sin\phi_n\Im\bar\alpha_n) +\bar J(e^{i\varphi}\bar\alpha_n^*\bar\alpha_{n+1}+\text{H. c.})\Big], 
\end{eqnarray}
and
\begin{widetext}
\begin{eqnarray}\label{eq:Hq}
\frac{H_\text{q}}{\omega_0} &=& \sum_n \bigg[ a_n^\dagger a_n -\bar\Omega\Big(\cos\theta_n + 2g\eta_+\sin\theta_n\cos\phi_n\Re\bar\alpha_n+2g\eta_-\sin\theta_n\sin\phi_n\Im\bar\alpha_n\Big)b_n^\dagger b_n+\bar J(e^{i\varphi}a_n^\dagger a_{n+1}+e^{-i\varphi}a_{n+1}^\dagger a_n) \nonumber \\
&&+\frac12g\sqrt{\bar\Omega}\eta_+(a_n+a_n^\dagger)\Big(\cos\theta_n\cos\phi_n(b_n+b_n^\dagger)+i\sin\phi_n(b_n-b_n^\dagger)\Big) -i\frac12g\sqrt{\bar\Omega}\eta_-(a_n-a_n^\dagger)\Big(\cos\theta_n\sin\phi_n(b_n+b_n^\dagger) \nonumber\\
&&-i\cos\phi_n(b_n-b_n^\dagger)\Big)\bigg].
\end{eqnarray}
\end{widetext}
Here, $s=\pm1$ and, following Ref.~\cite{PhysRevLett.128.163601}, we defined the rescaled parameters $\bar\alpha_n = \sqrt{\omega_0/N_a\Omega}\alpha_n$, $g=2\lambda/\sqrt{\omega_0\Omega}$, and $\bar\Omega=\Omega/\omega_0,\bar J=J/\omega_0$. Furthermore, in the thermodynamic limit, we replaced  the collective spin operators by the bosonic operators $b_n$ using the Holstein-Pirmakoff (HP) transformation $J_n^z = N_a/2-b_n^\dagger b_n,\; J_n^+ = \sqrt{N_a-b_n^\dagger b_n}b_n$. The expressions in  Eq.~(\ref{eq:EGS}) and Eq.~(\ref{eq:Hq}) constitute our starting point to study the properties of the phases (such as the critical scaling and the location of the phase transitions) of the generalized Dicke trimer model.



%%%%%%%%%%%%%%%%%%%%%%%%%%%%%%%%%%%%%%%%%%%%%%%%%%%%%%%%%%%%%%%%%%%%%%%%%%%%%%%%%%%%%%%%%%%%%%%%%%%%%%%%
\section{Results}\label{sec:results}

\subsection{Phase Diagram}

\begin{figure}%[!htbp]
\includegraphics[clip, width = 0.98\columnwidth]{pd3d.pdf}
\caption{3d plot of the phase diagram. Colored surfaces and curves correspond to continuous phase transitions featuring different critical scalings; while gray surface corresponds to first-order phase transitions. Red curve is characterized by two critical scalings in the FSP and blue curve is made of tricritical points with two critical scalings on both sides of the transition. Here we use $\bar J=0.3$ to satisfy the constraints Eq.~(\ref{eq:Jphi}) such that for all values of $\varphi$ the NP is stable in numerical simulations.
}\label{fig:pd3d}
\end{figure}
We start by calculating the phase diagram of the model. To achieve this, we first simplify the expression of the  ground-state energy $E_{\text{GS}}$ by minimizing it with respect to the variables $\theta_n,\phi_n$ (details are given in Appendix.~\ref{sec:app_sphc}) to write
\begin{eqnarray}\label{eq:EGSsimplifed}
\bar{E}_{\text{GS}}&=&\sum_n\bigg[\lvert\bar\alpha_n\rvert^2-\frac12 \sqrt{1+4g^2A_n^2} \nonumber\\
&&\quad\quad+\bar J\left(e^{i\varphi}\bar\alpha_n^*\bar\alpha_{n+1}+\text{H. c.}\right) \bigg],
\end{eqnarray}
where $A_n=\sqrt{\eta_+^2\Re^2\bar\alpha_n+\eta_-^2\Im^2\bar\alpha_n}$, and where we defined the rescaled ground state energy as $\bar{E}_{\text{GS}}=E_{\text{GS}}/N_a\Omega$. The ground state solution $\bar\alpha_n^\text{gs}$ is determined by minimizing $\bar E_{\text{GS}}$ with respect to $\bar\alpha_n$. 
Since the NP solution $\bar\alpha_n=0$ is always an extreme point of $\bar{E}_{\text{GS}}$, the phase diagram can be obtained by examining its stability. This is achieved by evaluating the eigenvalues of the $6\times6$ Hessian matrix $\partial^2\bar E_{\text{GS}}/\partial\bar\alpha_n\partial\bar\alpha_m$ at the origin $\bar{\alpha}_n=0$.  Interestingly, it is possible to find analytical expressions for the six eigenvalues $\xi_{1,2} = 2 + 4\bar J\cos\varphi - g^2(1\pm\eta)^2/2$, $\xi_{3,4} = 2-2\bar J\cos\varphi-g^2(1+\eta^2)/2\pm\sqrt{12\bar J^2\sin^2\varphi+g^4\eta^2}$ and $\xi_{5(6)}=\xi_{3(4)}$. 
Since a stable NP corresponds to having all eigenvalues positive, the phase boundary separating the NP  and the superradiant phase can be written as
$g_\text{c}=\min(g_\text{c}^\text{nf},g_\text{c}^\text{f})$ where $g_\text{c}^\text{nf}=\sqrt{1+2\bar J\cos\varphi}/\max(\eta_+,\,\eta_-)$ and 
\begin{eqnarray}\label{eq:gcf}
g_\text{c}^\text{f} = 
\begin{cases}
\sqrt{\frac{3(1-\bar J^2)}{1-\bar J\cos\varphi}-4\bar J\cos\varphi-2},~~\text{for}~\eta=\pm1,  \\
\min\bigg(\frac{\sqrt{M+\sqrt{N}}}{1-\eta^2},\,\frac{\sqrt{M-\sqrt{N}}}{1-\eta^2}\bigg),~~\text{for}~\eta\neq\pm1, 
\end{cases}
\end{eqnarray}
in terms of the variables $M=(1+\eta^2)(1-\bar J\cos\varphi)$ and $N=4\eta^2(1-\bar J\cos\varphi)^2+3\bar J^2\sin^2\varphi(1-\eta^2)^2$. Note that, since $g_\text{c}^\text{f},\,g_\text{c}^\text{nf}$ are real, the following constraints on $\bar J,\,\varphi$ should be imposed 
\begin{eqnarray}\label{eq:Jphi}
1+2\bar J\cos\varphi > 0,~~~~~ 1-2\bar J\cos(\varphi-\pi/3) > 0. 
\end{eqnarray} 
The corresponding phase diagram is shown in Fig.~\ref{fig:pd3d}, where surfaces and curves with different colors are characterized by distinct critical scalings as we will elaborate below. The phase boundary of the FSP (yellow surface) and the nFSP (green surface) intersect at a tricritical line $\varphi_{tr}$ (blue curve), which vertically extends into a first-order phase boundary.
Note that we did not find any analytical expression for the first-order phase boundary, which we verified numerically. In the NP and nFSP, the values for $\bar\alpha_n^\text{gs}$ are identical in all cavities and they are twofold degenerate in the nFSP (see Appendix.~\ref{sec:analy_nFSP}). 
Furthermore, in Appendix.~\ref{sec:app_prop_FSP}, we show the following generic properties of the sixfold degenerate FSP solutions (which break translational symmetry): (i) $\Im\bar\alpha_n=0,\;\bar\alpha_{n+1}=\bar\alpha_{n-1}^*$ for $\eta>0$; (ii) $\Re\bar\alpha_n=0,\;\bar\alpha_{n+1}=-\bar\alpha_{n-1}^*$ for $\eta<0$; (iii) At $\eta=0$, there exist special solutions satisfying either (i) or (ii), and from which the remaining solutions can be obtained by multiplying a phase factor $e^{i\zeta}$. Property (iii) describes a phase redundancy of the ground state solutions at $\eta=0$ which can be understood by inserting $\eta=0$ and $\bar\alpha_n = \lvert\bar\alpha_n\rvert e^{i\zeta_n}$ into Eq.~(\ref{eq:EGS}) to obtain
\begin{eqnarray}
\bar{E}_{\text{GS}}&=&\sum_n\bigg[\lvert\bar\alpha_n\rvert^2 - \frac12
\sqrt{1+g^2\lvert\bar\alpha_n\rvert^2} \nonumber\\
&&+ 2\bar J\lvert\bar\alpha_n\bar\alpha_{n+1}\rvert
\cos(\varphi+\zeta_{n+1}-\zeta_n)
\bigg]. 
\end{eqnarray}
From this expression, we can appreciate how $\bar E_{\text{GS}}$ only depends on the phase difference $\zeta_{n+1}-\zeta_n$ of neighbouring cavity fields, implying that once a minimum $\bar\alpha_n^\text{gs}$ of $\bar E_{\text{GS}}$ is found, all other minima can be constructed as $\bar\alpha_n^\text{gs}e^{i\zeta}$. In the following, we will analyze the  consequences of this phase redundancy on the excitation spectra.
We finish this section noting that the phase diagram is symmetric with respect to $\eta=0$ as a consequence of the symmetric coupling in the Dicke Hamiltonian.



\subsection{Excitation Spectra and Critical Scalings}

\begin{figure*}%[!htbp]
\includegraphics[clip,width = 13cm]{exspec_all.pdf}%
\caption{Excitation spectra for $\eta=1$ (solid lines) and $\eta=0$ (dashed lines) calculated at (a) $\varphi=0$, (b) $\pi/4$, (c) $\varphi_{tr}$ and (d) $\pi$ respectively. The two zero momentum modes (blue) are plotted using Eq.~(\ref{eq:dispersion0}) while the four finite momentum modes (red) and all modes in the FSP (black) are from numerics. In panel (c) the spectra in the FSP are plotted along the lines $\varphi_+$ (red and blue lines) and $\varphi_-$ (black lines). We used $\bar J=0.3$, $\bar\Omega=1$ in numerical simulations. 
}
\label{fig:excspec}
\end{figure*}


In this section, we analyze the quantum properties of the model on top of the classical energy landscape described by $E_{\text{GS}}$. To do this, we substitute the values of $\theta_n,\;\phi_n$ which minimize $E_{\text{GS}}$ into Eq.~(\ref{eq:Hq}),  see Appendix.~\ref{sec:app_sphc}. In the superradiant phase, this leads to
\begin{widetext}
\begin{eqnarray}\label{eq:HqsimplifiedSP}
\frac{H_\text{q}}{\omega_0}&=&\sum_n\bigg[ \frac12(q_n^2+p_n^2) +  \frac{\bar\Omega}{2}\sqrt{1+4g^2A_n^2}(Q_n^2+P_n^2) +\bar J\cos\varphi(q_nq_{n+1}+p_np_{n+1})-\bar J\sin\varphi(q_np_{n+1}-p_nq_{n+1}) \nonumber\\
&&+\frac{g\sqrt{\bar\Omega}}{A_n}\Big(\frac{\eta_+^2\Re\bar\alpha_n}{\sqrt{1+4g^2A_n^2}}q_nQ_n + \eta_+\eta_-\Im\bar\alpha_n q_nP_n+\frac{\eta_-^2\Im\bar\alpha_n}{\sqrt{1+4g^2A_n^2}}p_nQ_n - \eta_+\eta_-\Re\bar\alpha_n p_nP_n\Big)
\bigg],  
\end{eqnarray}
\end{widetext}
where $Q_n=(b_n+b_n^\dagger)/\sqrt2, P_n = -i(b_n-b_n^\dagger)/\sqrt2$ are the  position and momentum quadratures of collective spins and we have expressed $H_\text{q}$ in terms of the quadrature variables $q_n,p_n,Q_n,P_n$. In the NP, $H_\text{q}$ is obtained by applying the replacements  $\eta_+\Re\bar\alpha_n/A_n \to -1,\,\eta_-\Im\bar\alpha_n/A_n \to 0$ and then setting $A_n=0$ in the previous expression. In both cases, the quadratic Hamiltonian is a bilinear form so it can be written as 
$H_\text{q}/\omega_0 = \mathbf{r}^T \mathcal{H}_\text{q} \mathbf{r}/2$, 
where $\mathbf{r} = (q_1,p_1,Q_1,P_1,q_2,p_2,Q_2,P_2,q_3,p_3,Q_3,P_3)^T$, in terms of the following $12\times12$ real symmetric matrix 
\begin{eqnarray}
\mathcal H_\text{q} = 
\begin{pmatrix}
\mathcal H_1 & \mathcal H_J & \mathcal H_J^T \\
\mathcal H_J^T & \mathcal H_2 & \mathcal H_J \\
\mathcal H_J & \mathcal H_J^T & \mathcal H_3 \\
\end{pmatrix},
\end{eqnarray}
where the explicit expressions for the submatrices $\mathcal{H}_n,\;\mathcal{H}_J$ are given in Appendix.~\ref{sec:app_HnHJ}. Since $\mathcal{H}_\text{q}$ is positive definite, by Williamson’s theorem~\cite{Serafini}, we can find a symplectic matrix $S$ which simultaneously satisfies both $S^T\Omega_0S=\Omega_0$ and $S^T\mathcal{H}_\text{q}S = \Sigma$. Here, we defined the symplectic form $\Omega_0=\bigoplus_{n=1}^6 i\sigma_y$ (in terms of  the Pauli matrix   $\sigma_y$) and the matrix $\Sigma=\mathrm{diag}(\epsilon_1,\epsilon_1,\epsilon_2,\epsilon_2,\epsilon_3,\epsilon_3,\epsilon_4,\epsilon_4,\epsilon_5,\epsilon_5,\epsilon_6,\epsilon_6)$ whose parameters $\epsilon_i>0$ are the absolute value of the $12$ eigenvalues of the matrix $i\Omega_0\mathcal H_\text{q}$. Importantly, the commutation relations are invariant under this simplectic transformation. 

The translational invariance of the NP and nFSP solutions, i.~e., $\bar\alpha_n=\bar\alpha,\,A_n=A=\sqrt{\eta_+^2\Re^2\bar\alpha+\eta_-^2\Im^2\bar\alpha}$, guarantees the translational invariance of $H_\text{q}$. This allows to make some progress by defining the 
Fourier transformation $q_n = \sum_k e^{-ikn}q_k/\sqrt N$, $p_n = \sum_k e^{ikn}p_k/\sqrt N$
in terms of the quasimomentum $k=2\pi m/N$ with $m=0,\pm1$ which results into   
\begin{widetext}
\begin{eqnarray}
\frac{H_\text{q}}{\omega_0} &=& \sum_k \Bigg[
\frac12\Big(q_kq_{-k}+p_kp_{-k}\Big) + 
\frac{\bar\Omega}{2}\sqrt{1+4g^2A^2}\Big(Q_kQ_{-k}+P_kP_{-k}\Big) + \bar J\cos\varphi\Big(e^{ik}q_kq_{-k}+e^{-ik}p_kp_{-k}\Big) \nonumber\\
&&- 2i\bar J\sin\varphi\sin k\;q_kp_k +\frac{g\sqrt{\bar\Omega}}{A}\Big(\frac{\eta_+^2\Re\bar\alpha}{\sqrt{1+4g^2A^2}}Q_kq_{-k} + \eta_+\eta_-\Im\bar\alpha P_kq_{-k}+\frac{\eta_-^2\Im\bar\alpha}{\sqrt{1+4g^2A^2}}Q_kp_{-k} - \eta_+\eta_-\Re\bar\alpha P_kp_{-k}\Big)
\Bigg]. \nonumber\\
\end{eqnarray}
\end{widetext}
The $k=0$ subspace is now decoupled from the finite momentum sector and diagonalization of the corresponding Hamiltonian $H_\text{q}^{k=0}$ leads to two analytical solutions for the NP and nFSP
\begin{eqnarray}\label{eq:dispersion0}
\epsilon_{1,2}^{k=0} = 
\sqrt{F\pm\sqrt{F^2+G}}, 
\end{eqnarray}
where $F=(D_1^2+D_2^2-2R_1R_2-2I_1I_2)/2$, $G=D_1D_2(R_1^2+R_2^2+I_1^2+I_2^2)-D_1^2D_2^2-(R_1R_2+I_1I_2)^2$ 
with 
$D_1 = 1+2\bar J\cos\varphi$, 
$D_2 = \bar\Omega\sqrt{1+4g^2A^2}$, 
$R_1 = g\sqrt{\bar\Omega}\eta_+^2\Re\bar\alpha/A\sqrt{1+4g^2A^2}$, 
$R_2 = g\sqrt{\bar\Omega}\eta_+\eta_-\Re\bar\alpha/A$, 
$I_1 = g\sqrt{\bar\Omega}\eta_-^2\Im\bar\alpha/A\sqrt{1+4g^2A^2}$, $I_2=g\sqrt{\bar\Omega}\eta_+\eta_-\Im\bar\alpha/A$. 
We do not find any closed-form solutions in the remaining momentum subspace. 

\begin{figure}%[t!]
\includegraphics[clip,width=8.5cm]{fitting_mode.pdf}%
\caption{(a) Fitting of the soft modes for $\varphi=\pi/4,\;\eta=1$ (blue) and $\varphi=0,\;\eta=0$ (red). Extracted exponents are given in Table.~\ref{tab:exponents} with the  asterisk superscript. (b) Weights $w_i,\,v_i$ of the soft mode for $\varphi=\pi/4,\;\eta=1$. Note that curves for $q_2,\;q_3$ overlap. On both sides of the transition, the relation $\sum_i w_i=0$ holds. 
}\label{fig:fitting_mode}
\end{figure}

\begin{table}%[!htbp]
\resizebox{8cm}{!}{
\begin{tabular}{|c||c|c|c|c|}
\hline 
$\varphi$ & $0$ & $(0,\,\varphi_{tr}$) & $\varphi_{tr}$ & $(\varphi_{tr},\,\pi]$ \\
\hline\hline
\# & 2+2 & 1+1 & 2+2 & 1+1  \\
\hline\hline
\multirow{2}{*}{$\eta=0$} & $1_n,\,1_n$ & $1_n$ & $1_n,\,1_0$ & $1_0$ \\
\cline{2-5}
& $0.99^*,\,0$ & $0$ & $1_n,\,0_0$ & $0_0$ \\
\hline
\multirow{2}{*}{$\eta\neq0$} & $0.5_n,\,0.5_n$ & $1_n$ & $1_n,\,0.5_0$ & $0.5_0$ \\
\cline{2-5}
& $1,\,0.5$ & $1.52^*$ & $1_n,\,0.5_0$ & $0.5_0$ \\
\hline
\end{tabular}}
\caption{Values of the critical exponent $\gamma$. The second row shows the number of soft modes in the NP (before plus symbol) and superradiant phase (after plus symbol). Exponents with the subscripts $0$ or $n$ correspond to soft modes originating from the zero and finite momentum subspaces respectively whereas exponents with the asterisk superscript are extracted by numerical fitting. The exponent $0$ is due to the zero energy mode at $\eta=0$ and the pair $(1, 0.5)$ at lower left corner is taken from Ref.\cite{PhysRevLett.128.163601}. 
}\label{tab:exponents}
\end{table}

In Fig.~\ref{fig:excspec} we show the full spectra obtained by numerical diagonalization (black and red lines)
and the two branches (blue lines) using Eq.~(\ref{eq:dispersion0}) for the anisotropic case $\eta=1$ and the isotropic case $\eta=0$ at $\varphi=0,\;\pi/4,\;\varphi_{tr},\;\pi$. The excitation spectra and the critical scaling of the soft modes (defined as $\epsilon\sim\lvert\delta g\rvert^\gamma$, in terms of the distance $\delta g=g-g_\text{c}$ from the critical point) show a number of remarkable properties. In fact, the branches from the finite momentum subspace at $\varphi=0,\;\pi$ have a twofold degeneracy in the NP and nFSP  as a consequence of time-reversal symmetry. This degeneracy is lifted in the FSP (at $\varphi=0$), where the ground state solutions break this translational symmetry, resulting in two distinct soft modes (following Ref.~\cite{PhysRevLett.128.163601}, we use the terms ``mean-field" and ``frustrated" to label the modes scaling as $\lvert\delta g\rvert^{1/2}$ and $\lvert\delta g\rvert^1$, respectively). This is an essential prerequisite for the appearance of the two critical scalings first observed in Ref.\cite{PhysRevLett.128.163601}. As time-reversal symmetry breaks for $\varphi\neq 0,\,\pi$, these notable two critical scalings disappear and only one critical mode survives on both sides of the transition. Of particular interest is the one in the FSP which, by numerical fitting, scales as $\epsilon\sim\lvert\delta g\rvert^{1.52}$, see Fig.~\ref{fig:fitting_mode}(a). The weights $w_i$ ($v_i$) of the quadrature operators $q_i$ ($Q_i$) for this mode are plotted in Fig.~\ref{fig:fitting_mode}(b). Close to the critical point, we find the relations $v_i=0$ ($\forall\,i$), $w_2=w_3=-w_1/2$ hold in the FSP provided the solution $\Im\bar\alpha_1=0$, $\bar\alpha_2=\bar\alpha_3^*$ is used, indicating that the critical mode for $0<\varphi<\varphi_\text{tr}$ coincides with the mean-field mode found at $\varphi=0$. By contrast, the frustrated mode satisfies $w_1=v_1=0$, $w_2=-w_3$ and $v_2=-v_3$, instead~\cite{PhysRevLett.128.163601}. 

In order to gain more intuition about the emergence of this mode and its critical scaling, we consider a semi-classical model in which all matter degrees of freedom are evaluated at their energy minimum and all quantum effects are encoded by light. A similar analysis, involving both light and matter, has been developed to investigate the role of quantum chaos in the Dicke model, c.f.~Sec.~II in~Ref.~\cite{PhysRevA.44.1022} and Refs.~\cite{PhysRevA.50.2040,PhysRevE.54.1449,PhysRevLett.80.5524,PhysRevA.64.043801,PhysRevE.67.066203} for the discussion of the semilclassical limit of the Dicke model. Here, we show that a classical treatment of the matter degrees of freedom is sufficient for a self-consistent estimation of the critical exponents. To build this model, we will operate a quantization procedure over the light degrees of freedom in the potential in Eq.~(\ref{eq:EGSsimplifed}) which can be interpreted as a classical Hamiltonian $\bar E_\text{GS}(\{\mathfrak q_i,\mathfrak p_i\})$,  where we performed the replacements $\Re\bar\alpha_i\to\mathfrak q_i,\;\Im\bar\alpha_i\to\mathfrak p_i$ to simplify the notation.
In this context, we can also define the Poisson bracket acting over two generic phase-space functions $\mathfrak Q(\{\mathfrak q_i,\mathfrak p_i\})$, $\mathfrak P(\{\mathfrak q_i,\mathfrak p_i\})$ as  
\begin{eqnarray}
\{\mathfrak Q,\mathfrak P
\}=\sum_l\bigg(
\frac{\partial\mathfrak Q}{\partial\mathfrak q_l}
\frac{\partial\mathfrak P}{\partial\mathfrak p_l} 
- 
\frac{\partial\mathfrak Q}{\partial\mathfrak p_l}
\frac{\partial\mathfrak P}{\partial\mathfrak q_l} 
\bigg), 
\end{eqnarray}
in terms of the conjugate pair $\mathfrak p_i, \mathfrak q_i$ which satisfy 
$\{\mathfrak q_i,\mathfrak p_j\}=\delta_{ij}$, $\{\mathfrak q_i,\mathfrak q_j\}=\{\mathfrak p_i,\mathfrak p_j\}=0$. Near the local energy minimum $\{\mathfrak q_i^m, \mathfrak p_i^m\}$, $\bar E_\text{GS}$ can be expanded as 
\begin{eqnarray}
\bar E_\text{GS}\approx\bar E_\text{GS}^\text{m} + \sum_{i,j=1}^3 \mathfrak D_{ij} \delta\mathfrak q_i\delta\mathfrak p_j,   
\end{eqnarray}
where the deviations $\delta\mathfrak q_i=\mathfrak q_i-\mathfrak q_i^\text{m}$, $\delta\mathfrak p_i=\mathfrak p_i-\mathfrak p_i^\text{m}$ are conjugate with respect to the Poisson bracket, where $\mathfrak D_{ij} = \partial^2\bar E_\text{GS}/\partial\delta\mathfrak q_i\partial\delta\mathfrak p_j$, and where $\bar E_\text{GS}^\text{m}=\bar E_\text{GS}(\{\mathfrak q_i^\text{m},\mathfrak p_i^\text{m}\})$. Since the matrix $\mathfrak D$ is real symmetric, invoking Williamson’s theorem again, one can find a symplectic matrix $S$ which brings it into a diagonal form while preserving the Poisson brackets. We can then write
\begin{eqnarray}\label{eq:Enormal}
\bar E_\text{GS}\approx \bar E_\text{GS}^\text{m} + \frac12\sum_{i=1}^3 (k_{\mathfrak q_i'}\delta \mathfrak q_i'^2 + k_{\mathfrak p_i'}\delta\mathfrak p_i'^2), 
\end{eqnarray}
where $\pm\sqrt{k_{\mathfrak q_i'}k_{\mathfrak p_i'}}$ are eigenvalues of the matrix $i\Omega_0\mathfrak D$ with $\Omega_0=\bigoplus_{n=1}^3 i\sigma_y$. 
We remark that, since we are interested in analyzing the presence of diverging behaviour at the critical point, we can always operate the substitutions $\delta\mathfrak q_i'\to\sqrt c\delta\mathfrak q_i',\;\delta\mathfrak p_i'\to\delta\mathfrak p_i'/\sqrt c,\;k_{\mathfrak q_i'}\to k_{\mathfrak q_i'}/c,\;k_{\mathfrak p_i'}\to ck_{\mathfrak p_i'}$ in Eq.~(\ref{eq:Enormal}), in terms of a $\delta g$-independent positive constant  $c$. To fix this freedom we assume, without loss of generality, that $\lVert\delta\mathfrak q_i'\rVert_2=1$ (in terms of the Euclidean 2-norm $\lVert\cdot\rVert_2$). 
The quantum ground state and low energy excited modes can be obtained by applying the quantization condition $\oint\delta\mathfrak p_i'd\delta\mathfrak q_i'=n_i\omega_0/\Omega N_a$ for each conjugate pair $\delta\mathfrak q_i', \delta\mathfrak p_i'$, where $n_i\in\mathbb{N}$ and the effective Planck's constant $\omega_0/\Omega N_a$ originates from the rescaling of $\bar{\alpha}_n$. Here, the presence of $N_a$ in the denominator is crucial to recover the mean-field nature of the Dicke model in the thermodynamic limit. 
Integrating the left side, one finds the relation $\epsilon_i\sim\sqrt{k_{\mathfrak q_i'}k_{\mathfrak p_i'}}/N_a$. 

Now we employ this formalism to compute $\gamma$ in the NP and FSP for the special case $\eta=1$. In the NP, the positive eigenvalues of $i\Omega_0\mathfrak D$ can be found analytically as $d_1=2g_\text{c}^\text{nf}\sqrt{((g_\text{c}^\text{nf})^2-g^2)}$ and $d_{2,3}=2\sqrt6\bar J\sin\varphi\sqrt{1-D\big(g^2-(g_\text{c}^f)^2\big)\pm\sqrt{1-2D\big(g^2-(g_\text{c}^\text{f})^2\big)}}$, where $D=(1-\bar J\cos\varphi)/6\bar J^2\sin^2\varphi$. For $\varphi>\varphi_\text{tr}$ ($0<\varphi<\varphi_\text{tr}$), only $d_1$ ($d_3$) vanishes at $g_\text{c}$ resulting in $\gamma=1/2$ ($1$). For $\varphi=0$, $d_{2,3}=2\sqrt{(1-\bar J)(1-\bar J-g^2)}$ leading to two soft modes, both characterized by $\gamma=1/2$. For $\varphi=\varphi_\text{tr}$, both $d_1$ and $d_3$ vanish at $g_\text{c}$ originating the two critical scalings at the tricritical point. In the FSP, despite the absence of analytical solutions, we found approximations valid near the critical point up to order $\mathcal O(\delta g^{3/2})$, see Appendix.~\ref{sec:app_fspsol}. Since only one critical mode exists for $0<\varphi<\varphi_\text{tr}$, it is possible  to  compute its correspondent determinant (in alternative toits diagonalization) which scales as $\det i\Omega_0\mathfrak D\sim\delta g^3$, implying that $\gamma=3/2$, in agreement with previous numerical fitting.  

Remarkably at the tricritical point $\varphi_\text{tr}$, the two critical scalings reappear on both sides of the transition. In fact, in the symmetry-broken phase we can approach the critical point from two different directions $\varphi_{\pm}=\varphi_\text{tr}\pm(g-g_{\text{c}})$ which are located in the nFSP and FSP respectively. In both cases, the two critical scalings appear and share the same exponents, as shown in Fig.~\ref{fig:excspec}(c). At $\eta=0$, the excitation spectra shows its most prominent feature in the emergence of a zero energy mode in the superradiant phase, which can be attributed to the phase redundancy of $\bar\alpha_n^\text{gs}$ mentioned above. In fact, this phase redundancy indicates that the system can be excited without any extra energy-cost. This implies the appearance of a zero-energy mode in the excitation spectra, leading to the  exponent $\gamma=0$.

The values of the critical exponents $\gamma$  can be determined by various methods and they are summarized in Table.~\ref{tab:exponents}. In the NP and nFSP, the exponents for the soft modes in the zero momentum subspace (subscript $0$ in Table.~\ref{tab:exponents}) can be obtained by expanding $\epsilon_{1,2}^{k=0}$ around $g_\text{c}$, while those in the finite momentum subspace  (subscript $n$ in Table.~\ref{tab:exponents}) can be obtained by evaluating the determinant $\det i\Omega_0\mathcal{H}_\text{q}^{k\neq0} = \prod_{i=1}^4(\epsilon_{i}^{k\neq0})^2$ with $\epsilon_i^{k\neq0}$ in ascending order and expanding both sides around $g_\text{c}$. More details can be found in Appendix.~\ref{sec:app_exponents}. The remaining exponents for the soft modes in the FSP are marked by the asterisk superscript and are obtained by numerical fitting, as shown in Fig.~\ref{fig:fitting_mode}(a).  


\begin{figure}%[!htbp]
\includegraphics[clip,width=8.8cm]{fluctuations.pdf}%
\caption{Scaling of the cavity photon numbers for different values of $\varphi$: (a) $0$, (b) $\varphi_{tr}$, (c) $\pi$ and (d) $\pi/4$ computed for $\bar J=0.1$. Note that in panel (b) empty (filled) symbols are obtained along the line $\varphi_+$ ($\varphi_-$). We mark different scaling behavior by different line styles, specifically, dashed and dotted lines correspond to $\lvert\delta g\rvert^{-1/2}$ and $\lvert\delta g\rvert^{-1}$ respectively. We used the parameters $\bar J=0.1,\;\bar \Omega=1$. 
}\label{fig:fluctuations}
\end{figure}

\begin{figure}%[t!]
\includegraphics[clip,width=8.5cm]{kqkp.pdf}%
\caption{(a) Coefficients $k_{\mathfrak q_i'}$ (solid lines),  $k_{\mathfrak p_i'}$ (dashed lines) as a function of $\delta g$ with the constraint $\lVert\delta\mathfrak q_i'\rVert_2=1$. For $\varphi=0$, $k_{\mathfrak q_i'}$ vanishes while $k_{\mathfrak p_i'}$ remains finite at  the critical point, consistent with a divergent fluctuation of the ground state wavefunction in the $\delta\mathfrak q_i'$ direction. In contrast, for $\varphi=\pi/4$, both coefficients vanish for $\delta g\rightarrow 0$ at the same rate, as demonstrated by the identical slopes of the two red lines. This implies that the width of the Gaussian ground state wavefunction remains finite even at the critical point. See more detailed discussion in the main text. (b) Variances of the quadrature $q_n$ as functions of $\delta g$ in the NP for two values of $\varphi$ and $\eta=1$. Approximate results from the semiclassical method are marked by cross and plus symbols. In both panels we used $\bar J=0.1,\;\bar \Omega=1$. 
}\label{fig:kqkp}
\end{figure}

\subsection{Local Cavity Photon Numbers}

In this section, we analyze 
the local cavity photon numbers $\langle a_n^\dagger a_n\rangle$ on top of the classical background
which also bears non-trivial information about the phase transition. These fluctuations  can be obtained from the covariance matrix $C=SS^T/2 = \langle\mathbf{r}^T\mathbf{r}\rangle_\text{gs}$ where the expectation is taken over the Gaussian ground state. In Fig.~\ref{fig:fluctuations} we show the scaling behavior of the cavity photon number $\langle a_n^\dagger a_n\rangle\sim\lvert\delta g\rvert^{-\beta}$ for $\eta=1$. As mentioned above, for $\varphi=0$, the FSP ground state solutions satisfy the property $\Im\bar\alpha_n=0,\;\bar\alpha_{n-1}=\bar\alpha_{n+1}^*$, leading to the distinct scaling behaviors $\langle a_n^\dagger a_n\rangle\sim\lvert\delta g\rvert^{-1/2}$ and $\langle a_{n\pm1}^\dagger a_{n\pm1}\rangle\sim\lvert\delta g\rvert^{-1}$  \cite{PhysRevLett.128.163601}; whereas in the NP $\langle a_n^\dagger a_n\rangle\sim\lvert\delta g\rvert^{-1/2}$ for all sites. For $\varphi=\varphi_{tr},\;\pi$, the photon number diverges as $\langle a_n^\dagger a_n\rangle\sim\lvert\delta g\rvert^{-1/2}$ on both sides of the transition. However, for $\varphi=\pi/4$, we observe that, in the NP, the photon numbers saturate at the critical point, i. e., $\langle a_n^\dagger a_n\rangle\sim\lvert\delta g\rvert^{0}$, seemly conflicting with the general recognition of divergent fluctuations of the order parameter for continuous phase transitions, see Fig.~\ref{fig:fluctuations}(d). However, this finite fluctuation at the critical point is related to the non-divergent quadrature weights in the NP observed in  Fig.~\ref{fig:fitting_mode}(b). We now give an intuitive interpretation of this phenomenon. In the paradigmatic Landau's picture, a continuous phase transition is triggered when one, or several, of the curvatures $k_{\mathfrak q_i'}$, $k_{\mathfrak p_i'}$ in the Landau potential described in Eq.~(\ref{eq:Enormal})
(forth-order terms are neglected) vanish at $g_c$. As a consequence, it is possible to distinguish two different scenarios depending on the leading expansion of the curvatures, $k_{\mathfrak q_i'(\mathfrak p_i')}\sim\lvert\delta g\rvert^{d_{\mathfrak q_i'(\mathfrak p_i')}}$. When $d_{\mathfrak q_i'}\neq d_{\mathfrak p_i'}$ (for example $d_{\mathfrak q_i'} > d_{\mathfrak p_i'}$) the ground state wavefunction of the quantized Hamiltonian $H_i/k_{\mathfrak p_i'} \sim ({k_{\mathfrak q_i'}}/{k_{\mathfrak p_i'}})\delta\mathfrak q_i'^2+\delta\mathfrak p_i'^2$ is 
\begin{equation}
\psi(\delta\mathfrak q_i') = \left(\frac{\sqrt{k_{\mathfrak q_i'}/k_{\mathfrak p_i'}}}{\pi}\right)^{1/4} e^{-\sqrt{k_{\mathfrak q_i'}/k_{\mathfrak p_i'}}\delta\mathfrak q_i'^2/2},
\end{equation}
which becomes infinitely wide in the $\delta\mathfrak q_i'$ direction, leading to divergent variances in the original quadratures $\mathfrak q_i,\;\mathfrak p_i$ (note that the constraint $\lVert\delta\mathfrak q_i'\rVert_2=1$ is necessary) and thus divergent photon numbers. On the contrary, when $d_{\mathfrak q_i'}= d_{\mathfrak p_i'}$,  the quantized Hamiltonian can be written as $H_i/k_{\mathfrak p_i'} \sim c'\delta\mathfrak q_i'^2 + \delta\mathfrak p_i'^2$ (in terms of $c'$ which is constant in $\delta g$) and its ground state can be represented by the wavefunction 
\begin{equation}
\label{eq:semiclass_wf}
\psi(\delta\mathfrak q_i') = \Big(\frac{\sqrt{c'}}{\pi}\Big)^{1/4} e^{-\sqrt{c'} \delta\mathfrak q_i'^2}.
\end{equation}
As a consequence, neither of the variances of the original quadratures or the average photon numbers diverge at the critical point. In Fig.~\ref{fig:kqkp}(a), we show that a numerical evaluation of the variables $k_{\mathfrak q_i'},\;k_{\mathfrak p_i'}$ for $\varphi=0,\;\pi/4$ at $\eta=1$ is consistent with the above intuitive interpretation. In Fig.~\ref{fig:kqkp}(b), we show the variance $\langle q_n^2\rangle$ of the ground state is indeed finite for $\varphi=\pi/4$, in sharp contrast to the divergent behavior for $\varphi=0$. A comparison between the quantity $\langle q_n^2\rangle$ calculated via the  wavefunction in Eq.~(\ref{eq:semiclass_wf}) and via numerical diagonalization, gives further evidence of the consistency of our semiclassical model. In the FSP, although the photon numbers differ in different cavities, they show the same scaling behavior $\langle a_n^\dagger a_n\rangle\sim\lvert\delta g\rvert^{-1/2}$ visible only on small scales. Lastly for $\eta=0$, we always find finite fluctuations (not shown) in the NP due to the existence of the phase mode.




\section{Conclusions}\label{sec:conclusion}

Motivated by the recent results in \cite{PhysRevLett.128.163601}, we studied a generalized Dicke trimer model where two types of perturbations(breaking time-reversal and $Z_2$ symmetries) are introduced to deepen the understanding of frustrated superradiant phases and corresponding phase transitions. The critical scaling behavior of the generalized model we present here is characterized by a range of interesting properties. For example, we found an unconventional critical exponent $1.5$ and finite critical fluctuations appearing when time-reversal symmetry is broken, a tricritical line hosting two critical scalings on both sides of the transition, and a zero energy mode in the superradiant phase in the $U(1)$ isotropic case. By extending the findings present in the original proposal \cite{PhysRevLett.128.163601}, these results  confirm that frustrated SPT are a promising platform for exploring novel critical phenomena. Future works may further analyze the behavior of the critical exponents for an arbitrary odd number of cavities, the effects of cavity dissipation and driving \cite{PhysRevLett.122.193605, PhysRevA.98.023804,Kirton_2018,PhysRevLett.123.260401}, entanglement properties across the frustrated SPT \cite{PhysRevLett.92.073602}, the effects of staggered Zeeman magnetic field \cite{PhysRevLett.122.193201}, two-photon light-matter coupling \cite{Garbe2020} and other generalized Dicke models \cite{PhysRevA.104.043708,PhysRevA.104.063705}.



\section{Acknowledgements}

C.Z. acknowledges support from the Startup Fund of Yanshan University. M.C. acknowledges support from NSFC (Grants No.~12050410264 and No.~11935012) and NSAF (Grant No.~U1930403). N.L.~acknowledges partial support from JST PRESTO through Grant No.~JPMJPR18GC. 

\newpage

\appendix

\section{Derivation of the Classical, Linear and Quadratic Hamiltonians}\label{sec:clq}
In this section, we analyze details about the derivation of the decomposition in Eq.~(\ref{eq:decom}) which is obtained by a change of frame defined by the operator $U$ in Eq.~(\ref{eq:U}). This unitary operator $U$ acts on the collective spin operators as
\begin{widetext}
\begin{eqnarray}
\begin{split}
&U^\dagger J_n^x U = \cos\theta_n\sin\phi_n J_n^x - \sin\phi_n J_n^y +\sin\theta_n\sin\phi_n J_n^z, \\
&U^\dagger J_n^y U = \cos\theta_n\cos\phi_n J_n^x + \cos\phi_n J_n^y +\sin\theta_n\cos\phi_n J_n^z, \\
&U^\dagger J_n^z U = -\sin\theta_n J_n^x + \cos\theta_n J_n^z, 
\end{split}
\end{eqnarray}
\end{widetext}
and  on the cavity field as $U^\dagger a_n U = a_n+\alpha_n$. By using of these relations and the Holstein-Pirmakoff transformation, we obtain the ground state energy in Eq.~(\ref{eq:EGS}), the quadratic Hamiltonian in Eq.~(\ref{eq:Hq}), and the linear Hamiltonian
\begin{widetext}
\begin{eqnarray}
\frac{H_\text{l}}{\sqrt{N_a\Omega\omega_0}} &=& 
\sum_n\Bigg[(\bar\alpha_n^*a_n+\bar\alpha_na_n^\dagger)-\sqrt{\bar\Omega}\sin\theta_n(b_n+b_n^\dagger) +\sqrt{\bar\Omega}g\eta_+\Re\bar\alpha_n\Big(\cos\theta_n\cos\phi_n(b_n+b_n^\dagger)+i\sin\phi_n(b_n-b_n^\dagger)\Big) \nonumber\\
&&+\sqrt{\bar\Omega}g\eta_-\Im\bar\alpha_n\Big(\cos\theta_n\sin\phi_n(b_n+b_n^\dagger)-i\cos\phi_n(b_n-b_n^\dagger)\Big)
+\frac12g\eta_+\sin\theta_n\cos\phi_n(a_n+a_n^\dagger) \nonumber\\
&&+\frac12g\eta_-\sin\theta_n\sin\phi_n(a_n-a_n^\dagger)+\bar J\Big(e^{i\varphi}(\bar\alpha_n^*a_{n+1}+\bar\alpha_{n+1}a_n^\dagger)+\text{H. c.}\Big)
\Bigg]. 
\end{eqnarray}
\end{widetext}
This Hamiltonian vanishes when the variables $\theta_n$ and $\phi_n$ are evaluated, as a function of $\bar{\alpha}_n$ at the points which minimize the ground state energy $E_{\text{GS}}$, see Eq.~(\ref{eq:phi}) and Eq.~(\ref{eq:theta}).



\section{Elimination of Spherical Coordinates}\label{sec:app_sphc}
In this section we present the explicit expression for the values of the spherical coordinates $\theta_n$ and $\phi_n$ which minimize the ground state energy $E_{\text{GS}}$. We start by rewriting Eq.~(\ref{eq:EGS}) in the main text as
\begin{eqnarray}\label{eq:EGSapp}
\frac{E_{\text{GS}}}{N_a\Omega} &=& \sum_n\Big[ \lvert\bar\alpha_n\rvert^2+ \frac{1}{2}\cos\theta_n+g\sin\theta_n(\eta_+\cos\phi_n\Re\alpha_n \nonumber\\
&&+g\eta_-\sin\phi_n\Im\alpha_n) +J(e^{i\varphi}\alpha_n^*\alpha_{n+1}+\text{H. c.})\Big].  \nonumber\\
\end{eqnarray}
To eliminate $\phi_n$, we minimize $\bar E_{\text{GS}} = E_{\text{GS}}/N_a\Omega$ over $\phi_n$
\begin{eqnarray}
0=\frac{\partial\bar{E}_{\text{GS}}}{\partial\phi_{n}}=g\eta_-\Im\bar\alpha_n\sin\theta_n\cos\phi_n-g\eta_+\Re\bar{\alpha}_{n}\sin\theta_{n}\sin\phi_{n}, \nonumber\\
\end{eqnarray}
leading to 
\begin{eqnarray}
\eta_+\Re\bar\alpha_n\sin\phi_n = \eta_-\Im\bar\alpha_n\cos\phi_n.
\end{eqnarray}
We define $A_n = \sqrt{\eta_+^2\Re^2\bar\alpha_n+\eta_-^2\Im^2\bar\alpha_n}$ and express $\phi_n$ in terms of $\bar\alpha_n$ as
\begin{eqnarray}\label{eq:phi}
\sin\phi_n = \frac{s\eta_-\Im\bar\alpha_n}{A_n},~~~ 
\cos\phi_n = \frac{s\eta_+\Re\bar\alpha_n}{A_n},
\end{eqnarray}
which allows us to eliminate $\phi_n$ in $\bar E_{\text{GS}}$. Similarly, we can minimize over $\theta_n$ to obtain
\begin{eqnarray}
0=\frac{\partial\bar{E}_{\text{GS}}}{\partial\theta_{n}}&=&-\frac{1}{2}\sin\theta_{n}+g\eta_+\Re\bar{\alpha}_{n}\cos\theta_{n}\cos\phi_{n} \nonumber\\
&& +g\eta_-\Im\bar\alpha_n\cos\theta_n\sin\phi_n. 
\end{eqnarray} 
Since $\theta_n\in[0,\pi]$, $\sin\theta_n\geq 0$, we have 
\begin{eqnarray}\label{eq:theta}
\cos\theta_{n}=\frac{s}{\sqrt{1+4g^{2}A_n^2}},~~ 
\sin\theta_{n}=\frac{2gA_n}{\sqrt{1+4g^{2}A_n^2}}.
\end{eqnarray}
Due to the fact that in the ground state of the uncoupled Dicke model with $\lambda=0$ all atoms populate the spin down state, we need to impose $s=-1$. Substituting Eq.~(\ref{eq:phi}) and Eq.~(\ref{eq:theta}) 
back into Eq.~(\ref{eq:EGSapp}) and Eq.~(\ref{eq:Hq}), we finally arrive at Eq.~(\ref{eq:EGSsimplifed}) and Eq.~(\ref{eq:HqsimplifiedSP}) in the main text, respectively. 


\section{Analytical Results of the nFSP Solutions}\label{sec:analy_nFSP}

In the NP and nFSP, all $\bar\alpha_n$ are identical so we can write $\bar\alpha_n=\bar\alpha$ to simplify Eq.~(\ref{eq:EGSsimplifed}) as
\begin{eqnarray}
\bar E_{\text{GS}}/N = (1+2\bar J\cos\varphi)\lvert\bar\alpha\rvert^2 - \frac 12\sqrt{1+4g^2A^2}, 
\end{eqnarray}
where $A=\sqrt{\eta_+^2\Re^2\bar\alpha+\eta_-^2\Im^2\bar\alpha}$. 
Minimizing $\bar E_{\text{GS}}$ over $\bar\alpha$ leads to the solutions 
\begin{equation}
    \bar{\alpha} =\pm\frac{1}{2g\eta_+}\sqrt{\frac{g^4\eta_+^4}{(1+2\bar{J}\cos\varphi)^2}-1}
\end{equation}
for $\eta>0$, 
\begin{eqnarray}
\bar\alpha = \pm i \frac{1}{2g\eta_-}\sqrt{\frac{g^4\eta_-^4}{(1+2\bar J\cos\varphi)^2}-1}, 
\end{eqnarray}
for $\eta<0$, and
\begin{eqnarray}
\bar\alpha = \frac1g\sqrt{\frac{g^4}{16(1+2\bar J\cos\varphi)^2}-1}\;e^{i\zeta}, 
\end{eqnarray}
for $\eta=0$ and with arbitrary real $\zeta$.


\section{Generic Properties of the FSP Solutions}\label{sec:app_prop_FSP}

In this section we present a semi-analytical proof of  the properties satisfied by the FSP solutions discussed in the main text. We start by simplifying the expression of $\bar E_{\text{GS}}$ at $\eta=1$. 
Substituting $\eta=1$ into Eq.~(\ref{eq:EGS}) in the main text we obtain
\begin{eqnarray}\label{eq:EGSeta1}
\bar{E}_{\text{GS}}&=&\sum_n\bigg[ \lvert\bar{\alpha}_{n}\rvert^{2}
-
\frac 12\sqrt{1+4g^{2}\Re^2\bar{\alpha}_{n}} \nonumber\\
&&~~~~~~~~~~~~~+\bar J\left(e^{i\varphi}\bar{\alpha}_{n}^{*}\bar{\alpha}_{n+1}
+\text{H. c.}\right) \bigg]. 
\end{eqnarray}
We can further eliminate $\Im\bar\alpha_n$ by calculating
\begin{eqnarray}\label{alpha}
0=\frac{\partial\bar{E}_{\text{GS}}}{\partial\Im\bar{\alpha}_{n}} &=& 2\Im\bar{\alpha}_{n}+2\bar{J}\cos\varphi\left(\Im\bar{\alpha}_{n-1}+\Im\bar{\alpha}_{n+1}\right) \nonumber\\
&&+2\bar{J}\sin\varphi\left(\Re\bar{\alpha}_{n+1}-\Re\bar{\alpha}_{n-1}\right), 
\end{eqnarray}
and summing over $n$ 
\begin{equation}
\qquad\sum_{n}\Im\bar{\alpha}_{n}+\bar{J}\cos\varphi\sum_{n}\left(\Im\bar{\alpha}_{n+1}+\Im\bar{\alpha}_{n-1}\right)=0, 
\end{equation}
which leads to $\sum_n\Im\bar\alpha_n = 0$. Since in the present model $N=3$, we find the relation 
\begin{equation}
\Im\bar{\alpha}_{n}=\frac{-\bar{J}\sin\varphi}{1-\bar{J}\cos\varphi}\left(\Re\bar{\alpha}_{n+1}-\Re\bar{\alpha}_{n-1}\right).
\end{equation}
Substituting this expression into Eq.~(\ref{eq:EGSeta1}) leads to a simplified ground state energy function 
\begin{equation}\label{eq:EGSeta1s}
\bar E_{\text{GS}} = \sum_n \Bigg[ 
\xi_0\Re^2\alpha_n 
- \frac12\sqrt{1+4g^2\Re^2\bar\alpha_n} 
+ \xi_1\Re\bar\alpha_n \Re\bar\alpha_{n+1}
\Bigg],
\end{equation}
where we defined $\xi_0 = 1-2\bar J^2\sin^2\varphi/(1-\bar J \cos\varphi)$ and 
$\xi_1 = 2\bar J\cos\varphi + 2\bar J^2\sin^2\varphi/(1-\bar J \cos\varphi)$. Differentiating over $\Re\bar\alpha_n$ and rearranging the terms we obtain
\begin{eqnarray}\label{eq:RealphaEq}
\frac{\xi_1}{2}\sum_n\Re\bar\alpha_n = f(\Re\bar\alpha_n),
\end{eqnarray}
in terms of the function \cite{PhysRevLett.128.163601}
\begin{equation}\label{fx}
f(x)=\frac{g^2x}{\sqrt{1+4g^2x^2}}-\left(\xi_{0}-\frac{1}{2}\xi_{1}\right)x. 
\end{equation}
We can now follow the same arguments presented in Ref.\cite{PhysRevLett.128.163601} to prove the property (i) presented in the main text for the FSP solutions at $\eta=1$. We also numerically checked that the same holds for a generic $\eta>0$. We performed similar arguments in the cases $\eta<0$ and $\eta=0$ to prove the properties (ii) and (iii), respectively.

\section{The Matrices $\mathcal{H}_n$ and $\mathcal{H}_J$ in the NP and Superradiant Phase}\label{sec:app_HnHJ}
In this section we provide the explicit expressions for the matrices $\mathcal{H}_n$ and $\mathcal{H}_J$ in the NP and Superradiant Phase.
In the superradiant phase we have
\begin{widetext}
\begin{eqnarray}
\mathcal H_n = 
\begin{pmatrix}
1 & 0 & \frac{2g\sqrt{\bar\Omega}\eta_+^2\Re\bar\alpha_n}{A_n\sqrt{1+g^2A_n^2}} & \frac{2g\sqrt{\bar\Omega}\eta_+\eta_-\Im\bar\alpha_n}{A_n} \\
0 & 1 & \frac{2g\sqrt{\bar\Omega}\eta_-^2\Im\bar\alpha_n}{A_n\sqrt{1+g^2A_n^2}} & -\frac{2g\sqrt{\bar\Omega}\eta_+\eta_-\Re\bar\alpha_n}{A_n} \\
\frac{2g\sqrt{\bar\Omega}\eta_+^2\Re\bar\alpha_n}{A_n\sqrt{1+g^2A_n^2}} & \frac{2g\sqrt{\bar\Omega}\eta_-^2\Im\bar\alpha_n}{A_n\sqrt{1+g^2A_n^2}} & \bar\Omega\sqrt{1+g^2A_n^2} & 0 \\
\frac{2g\sqrt{\bar\Omega}\eta_+\eta_-\Im\bar\alpha_n}{A_n} & -\frac{2g\sqrt{\bar\Omega}\eta_+\eta_-\Re\bar\alpha_n}{A_n} & 0 & \bar\Omega\sqrt{1+g^2A_n^2}
\end{pmatrix},\;
\mathcal H_J = 
\begin{pmatrix}
\bar J\cos\varphi & -\bar J\sin\varphi & 0 & 0 \\
\bar J\sin\varphi & \bar J\cos\varphi & 0 & 0 \\
0 & 0 & 0 & 0 \\
0 & 0 & 0 & 0 
\end{pmatrix}. 
\end{eqnarray}
\end{widetext}
In the NP, $\mathcal{H}_n$ is obtained by applying the replacements $\eta_+\Re\bar\alpha_n/A_n \to -1,\,\eta_-\Im\bar\alpha_n/A_n \to 0$ and by setting $A_n=0$ to explicitly obtain
\begin{widetext}
\begin{eqnarray}
\mathcal{H}_n = 
\begin{pmatrix}
1 & 0 & -g\sqrt{\bar\Omega}\eta_+ & 0 \\
0 & 1 & 0 & g\sqrt{\bar\Omega}\eta_- \\
-g\sqrt{\bar\Omega}\eta_+ & 0 & \bar\Omega & 0 \\
0 & g\sqrt{\bar\Omega}\eta_- & 0 & \bar\Omega
\end{pmatrix},\;
\mathcal{H}_J = 
\begin{pmatrix}
\bar J\cos\varphi & -\bar J\sin\varphi & 0 & 0 \\
\bar J\sin\varphi & \bar J\cos\varphi & 0 & 0 \\
0 & 0 & 0 & 0 \\
0 & 0 & 0 & 0 
\end{pmatrix}.
\end{eqnarray}
\end{widetext}


\section{Approximate FSP Ground State Solutions Near the Critical Point For $\eta=1$}\label{sec:app_fspsol}

In this section we derive approximate ground state solutions of the FSP near the critical point for $\eta=1$. We substitute the ansatz $\bar\alpha_2=\bar\alpha_3$ into Eq.~(\ref{eq:RealphaEq}) and rearrange the equation to arrive at 
\begin{eqnarray}\label{eq:1}
\begin{split}
\xi_0\Re\bar\alpha_1 + \frac12\xi_1\Re\bar\alpha_2 - \frac{g^2\Re\bar\alpha_1}{\sqrt{1+4g^2\Re\bar\alpha_1^2}} = 0, \\
\frac12 \xi_1\Re\bar\alpha_1 + (\xi_0+\frac12\xi_1)\Re\bar\alpha_2 - \frac{g^2\Re\bar\alpha_2}{\sqrt{1+4g^2\Re\bar\alpha_2^2}} = 0.
\end{split}
\end{eqnarray}
Near the critical point, we have the expansions $\Re\bar\alpha_1 \approx r_0 \delta g^{\beta_1} + r_1 \delta g^{\beta_1+1} + r_2 \delta g^{\beta_1+2}$, $\Re\bar\alpha_2 \approx s_0 \delta g^{\beta_2} + s_1 \delta g^{\beta_2+1} + s_2 \delta g^{\beta_2+2}$, where $\beta_{1,2}>0$, $r_0,s_0\neq0$ and $\delta g>0$ in the FSP. Now we insert these expansions into Eq.~(\ref{eq:1}) then get the equations order by order. For the lowest order, we have 
\begin{eqnarray}
\begin{split}
\xi_0r_0\delta g^{\beta_1} + \xi_1s_0\delta g^{\beta_2} - g_c^2r_0\delta g^{\beta_1} = 0, \\
\frac12\xi_1r_0\delta g^{\beta_1} + (\xi_0+\frac12\xi_1)s_0\delta g^{\beta_2} - g_c^2s_0\delta g^{\beta_2} = 0. 
\end{split}
\end{eqnarray}
It is easy to see that $\beta_1=\beta_2$ otherwise either $r_0$ or $s_0$ vanishes. From now on we assume $\beta_1=\beta_2=\beta$. Making use of this relation and $g_c^2=\xi_0-\xi_1/2$ at $\eta=1$, one can show that $s_0=-r_0/2$. 
The equations for the next order are 
\begin{eqnarray}
\begin{split}
&(\xi_0r_1 + \xi_1s_1 - g_c^2r_1 - 2g_cr_0 )\delta g^{\beta+1} + 2g_c^4r_0^3\delta g^{3\beta} = 0, \\
&\Big[\frac12\xi_1r_1+(\xi_0+\frac12\xi_1)s_1-g_c^2s_1-2g_cs_0\Big]\delta g^{\beta+1} \\
&~~~~~~~~~~~~~~~~~~~~~~~~~~~~~~~~~~~~~~~~~~~~+ 2g_c^4s_0^3 \delta g^{3\beta} = 0. \\
\end{split}
\end{eqnarray}
One can readily see that $\beta\neq1/2$ leads to contradiction thus only $\beta=1/2$ is possible and solving the above equation gives
\begin{eqnarray}
r_0 = \pm \frac{2}{\sqrt3g_c^{3/2}},~~~~s_0 = \mp \frac{1}{\sqrt3g_c^{3/2}}.   
\end{eqnarray}
Proceeding along similar lines, we find 
\begin{eqnarray}
r_1 = \pm \frac{1}{6\sqrt3g_c^{5/2}},~s_1 = \frac{4}{3\sqrt3\xi_1g_c^{1/2}}\mp\frac{1}{12\sqrt3g_c^{5/2}}. 
\end{eqnarray}
Summarizing, in the FSP near the critical point we have the approximate solutions
\begin{eqnarray}
\begin{split}
&\Re\bar\alpha_1\approx\pm\frac{2\delta g^{1/2}}{\sqrt3g_c^{3/2}}\pm\frac{\delta g^{3/2}}{6\sqrt3g_c^{5/2}}, \\
&\Re\bar\alpha_2\approx\mp\frac{\delta g^{1/2}}{\sqrt3g_c^{3/2}}+\bigg(\frac{4}{3\sqrt3\xi_1g_c^{1/2}}\mp\frac{1}{12\sqrt3g_c^{5/2}}\bigg)\delta g^{3/2}.
\end{split}
\end{eqnarray}




\section{Evaluation of Critical Exponents}\label{sec:app_exponents}

Here we present details about the  computation of the critical exponents. 
In the NP, the determinant $\det i\Omega_0\mathcal{H}_\text{q}^{k\neq0}$ can be found analytically, and takes the form 
\begin{eqnarray}
\det i\Omega_0\mathcal{H}_\text{q}^{k\neq0} &=& \bar\Omega^4(16-8g^2+g^4-16\bar J^2-8g^2\eta^2-2g^4\eta^2 \nonumber\\
&&+g^4\eta^4+8\bar J\cos\varphi(g^2+g^2\eta^2-4) \nonumber\\
&&+32\bar J^2\cos2\varphi)^2/256.
\end{eqnarray}
An expansion around $g_\text{c}$ 
at $\eta=0,\varphi=0$ gives $\det i\Omega_0\mathcal{H}_\text{q}^{k\neq0} \sim (\delta g)^4$ whereas in all other cases we have $\det i\Omega_0\mathcal{H}_\text{q}^{k\neq0} \sim (\delta g)^2$. In the former case, we can further use the degeneracy of the soft modes to find $\gamma=0.5$ for  $\eta\neq0,\varphi=0$ and $\gamma=1$ in other cases. 

In the nFSP, the expression for the determinant is more complicated and, for clarity, we focus on some specific cases. For $\eta=1$, we find
\begin{widetext}
\begin{eqnarray}
\det i\Omega_0\mathcal{H}_\text{q}^{k\neq0} &=& 
g^8\bar\Omega^4\Big(4-4g^4-6\bar J^2+4g^4\bar J^2-3\bar J^4+\bar J\cos\varphi(11+8g^4-21\bar J^2)^2+\bar J^2\cos2\varphi(3-8g^4+2\bar J^2) \nonumber\\
&&+5\bar J^3\cos3\varphi+5\bar J^4\cos4\varphi \Big) \Big/ (1+2\bar J\cos\varphi)^4\big(3(1+2\bar J\cos\varphi)^2-4g^4\big)^2, 
\end{eqnarray}
\end{widetext}
while, for $\eta=0$, we find 
\begin{widetext}
\begin{eqnarray}
\det i\Omega_0\mathcal{H}_\text{q}^{k\neq0} &=& 
9\bar J^2\bar\Omega^4\Big((16-g^4+144\bar J^2)\cos\varphi+\bar J\cos2\varphi(48+g^4+64\bar J^2) \nonumber\\
&&+16\bar J(3+3\bar J^2 + 3\bar J\cos3\varphi+\bar J^2\cos4\varphi)\Big)^2\Big/256(1+2\bar J\cos\varphi)^4. 
\end{eqnarray}

\end{widetext}
These expressions lead to the same exponent $\gamma=1$ only at $\varphi=\varphi_{tr}$. For other values of $\eta$, we have checked numerically that $\gamma$ is always $1$. 



\newpage

\nocite{apsrev41Control}
\bibliographystyle{apsrev4-2}

\bibliography{bib}




\end{document}