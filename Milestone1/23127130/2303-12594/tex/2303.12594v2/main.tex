\documentclass[letterpaper]{article}

\usepackage{natbib,alifeconf}  %% The order is important


\usepackage[utf8]{inputenc} % allow utf-8 input
\usepackage[T1]{fontenc}    % use 8-bit T1 fonts
\usepackage{hyperref}       % hyperlinks
\usepackage{url}            % simple URL typesetting
\usepackage{booktabs}       % professional-quality tables
\usepackage{amsfonts}       % blackboard math symbols
\usepackage{nicefrac}       % compact symbols for 1/2, etc.
\usepackage{microtype}      % microtypography
\usepackage{lipsum}		% Can be removed after putting your text content
\usepackage{graphicx}
% \usepackage[square,sort,comma,numbers]{natbib}
% \usepackage{booktabs} % For formal tables
\usepackage{subcaption}
\usepackage{graphicx}
\usepackage{xcolor} % http://www.ctan.org/tex-archive/macros/latex/contrib/xcolor
\usepackage{tikz}
% packages for Flow diagrams
\usetikzlibrary{shapes.geometric, arrows, positioning, automata,positioning,fit,backgrounds}
\tikzstyle{process} = [rectangle, minimum width=2cm, minimum height=1cm, text centered, draw=black, fill=white!30]
\tikzstyle{sum} = \tikzstyle{sum} = [draw, circle, minimum size=.5cm]
\tikzstyle{arrow} = [thick,->,>=stealth]
\def\checkmark{\tikz\fill[scale=0.4](0,.35) -- (.25,0) -- (1,.7) -- (.25,.15) -- cycle;}
\usepackage{amsmath}
\DeclareMathOperator{\taninv}{tan^{-1}}
\usepackage{bm}
\usepackage{units}
\usepackage{float}
\usepackage{dblfloatfix}
\usepackage{algorithm}
\usepackage{algpseudocode}
\makeatother
\usepackage[T1]{fontenc}

\usetikzlibrary{fit,calc}
%define a marking command
\newcommand*{\tikzmk}[1]{\tikz[remember picture,overlay,] \node (#1) {};\ignorespaces}
%define a boxing command, argument = colour of box
\newcommand{\boxit}[1]{\tikz[remember picture,overlay]{\node[yshift=0pt,fill=#1,opacity=.25,fit={(A)($(B)+(.9\linewidth,.5\baselineskip)$)}] {};}\ignorespaces}
%define some colours according to algorithm parts (or any other method you like)
\colorlet{yellow}{yellow!100}
\colorlet{blue}{cyan!60}



% *****************
%  Requirements:
% *****************
%
% - All pages sized consistently at 8.5 x 11 inches (US letter size).
% - PDF length <= 8 pages for full papers, <=2 pages for extended
%    abstracts (not including citations).
% - Abstract length <= 250 words.
% - No visible crop marks.
% - Images at no greater than 300 dpi, scaled at 100%.
% - Embedded open type fonts only.
% - All layers flattened.
% - No attachments.
% - All desired links active in the files.

% Note that the PDF file must not exceed 5 MB if it is to be indexed
% by Google Scholar. Additional information about Google Scholar
% can be found here:
% http://www.google.com/intl/en/scholar/inclusion.html.


% If your system does not generate letter format documents by default,
% you can use the following workflow:
% latex example
% bibtex example
% latex example ; latex example
% dvips -o example.ps -t letterSize example.dvi
% ps2pdf example.ps example.pdf


% For pdflatex users:
% The alifeconf style file loads the "graphicx" package, and
% this may lead some users of pdflatex to experience problems.
% These can be fixed by editing the alifeconf.sty file to specify:
% \usepackage[pdftex]{graphicx}
%   instead of
% \usepackage{graphicx}.
% The PDF output generated by pdflatex should match the required
% specifications and obviously the dvips and ps2pdf steps become
% unnecessary.


% Note:  Some laser printers have a serious problem printing TeX
% output. The use of ps type I fonts should avoid this problem.

\title{A Comparative Study of Brain Reproduction Methods for Morphologically Evolving Robots}
% \title{Comparing Brain Evolution Methods for Morphologically Evolving Robots}
% \title{A Comparative Study of Brain Evolution Mechanisms for Morphologically Evolving Robots}

\author{Jie Luo$^{1}$, Carlo Longhi$^{2}$ \and Agoston E.
Eiben$^{1}$ \\
\mbox{}\\
$^1$Vrije Universiteit Amsterdam, Amsterdam, The Netherlands \\
$^2$University of Bologna, Bologna, Italy \\
% $^*$Each submission will undergo a double-blind review process. To this end, submissions should NOT contain any element that could reveal \\ the identity of the authors (author names, affiliations, funding details and acknowledgments), and should use the third person to refer to \\ previous work by the authors. These notes should be removed/commented out, but please remember that the page limit will remain, strictly, \\8 pages (not including citations) in case of acceptance, with mandatory name(s), affiliation(s), and email address of a corresponding author. \\
j2.luo@vu.nl} % email of corresponding author

% For several authors from the same institution use the same number to
% refer to one address.

% If the names do not fit well on one line use
%         Author 1, Author 2 ... \\ {\Large\bf Author n} ...\\ ...

% If the title and author information do not fit in the area
% allocated, place \setlength\titlebox{<new height>} after the
% \documentclass line where <new height> is 2.25in

\begin{document}
\maketitle

\begin{abstract}
In the most extensive robot evolution systems, both the bodies and the brains of the robots undergo evolution and the brains of `infant' robots are also optimized by a learning process immediately after `birth'. This paper is concerned with the brain evolution mechanism in such a system. In particular, we compare four options obtained by combining asexual or sexual brain reproduction with Darwinian or Lamarckian evolution mechanisms. We conduct experiments in simulation with a system of evolvable modular robots on two different tasks. The results show that sexual reproduction of the robots' brains is preferable in the Darwinian framework, but the effect is the opposite in the Lamarckian system (both using the same infant learning method). Our experiments suggest that the overall best option is asexual reproduction combined with the Lamarckian framework, as it obtains better robots in terms of fitness than the other three. Considering the evolved morphologies, the different brain reproduction methods do not lead to differences. This result indicates that the morphology of the robot is mainly determined by the task and the environment, not by the brain reproduction methods. 

% Abstract length should not exceed 250 words
%Evolutionary Robotics (ER) draw inspiration from natural evolution to solve the problem of robot design. A key principle in the evolutionary process is reproduction, when the genotype of one or more parents is inherited by their offspring. Commonly ER researchers have been using either sexual or asexual reproduction to generate new solutions to improve the fitness of the robot. However, to our best knowledge, no research has been done yet to investigate the impact of different reproduction methods, especially testing on different evolution frameworks. 

%In this work, we study the effects of sexual and asexual reproduction on the controllers with morphological evolvable learning robots in two evolutionary frameworks (Lamarckian evolution and Darwinian evolution). In our system, both morphologies and controllers are jointly evolved to solve two tasks (Point navigation and Panoramic rotation). We adopt the Triangle of Life framework, in which the controllers go through a phase of learning before reproduction.

%Using extensive simulations we show that sexual reproduction of the robots' brains is preferable over asexual reproduction in the Darwinian framework as it obtains better robots in terms of fitness, however, the effect is the opposite in the Lamarckian framework. Moreover, we investigate the morphologies produced by both reproduction methods. The result shows that the evolved bodies in the final generation are similar in terms of tasks. 

% Finally, we study the effects of the reproduction mechanism on the robots' learning capabilities. By measuring the difference between the inherited and the learned brain we find that robots that evolved using sexual reproduction have better-inherited brains and are also better learners.
\end{abstract}

\section{Introduction}
In Evolutionary Robotics (ER), the simultaneous development of robot morphologies and control systems is a difficult task, and we have only seen relatively simple results so far, as noted by \cite{Cheney2016}. Some of the difficulty is due to the increased dimensionality of the search, but a more pernicious aspect may be the increased ruggedness of the search space: a small variation in the morphology can easily offset the performance of the controller-body combination found earlier. The design choice for reproduction varies per study. \cite{Cheney2016} and \cite{Nygaard2018} employ asexual reproduction for the generation of both morphologies and controllers. \cite{Medvet2021} generate each offspring by either mutation or geometric crossover according to a probability. \cite{Lehman2011}, \cite{miras2020environmental}, \cite{Stensby2021}, \cite{auerbach2012} and \cite{DeCarlo2020} utilize sexual reproduction.

Several researchers have combined evolutionary methods with learning techniques to drive the joint evolution of controllers and morphologies deeper. Different reproduction methods for their robots have been used too. \cite{Cheney2018} produce new offspring by mutation of the parent's morphology or controller but not both. \cite{Wang2019}, \cite{Nygaard2017}, \cite{Kriegman2018} and \cite{Goff2021} only use mutation to generate the robotic offspring and \cite{Gupta2021} generate new morphologies by asexual reproduction but use randomly initialized controllers. Sexual reproduction is used, instead, to produce the morphology and controller of the offspring by \cite{Luo2022}, \cite{Jelisavcic2019}, and \cite{miras2018effects}.

To the best of our knowledge, no studies have addressed the arity of brain reproduction, comparing unary/asexual with binary/sexual reproduction in a morphologically evolving robotic system, let alone in combination with a Darwinian vs.~Lamarckian framework for combining evolution with learning. Our study aims to fill this gap by answering the following research questions:

\textbf{Research Question 1:} How do asexual and sexual brain reproduction compare within Lamarckian and Darwinian evolution frameworks in terms of task performance?

% The answer to this question would indicate the differences that arise between two Evolutionary Robotics systems that utilize sexual and asexual reproduction and which of the two reproduction methods is preferable.

\textbf{Research Question 2:} Will the different brain reproduction methods lead to different robot morphologies? 
% \textbf{Research Question 2:} How do the mechanisms of genetic variation and inheritance differ between asexual and sexual reproduction, and how do these differences shape the genetic diversity and evolution of populations?

% \cite{Luo2022} and \cite{Miras2020} investigate the effect that learning has on the morphologies produced in an Evolutionary Robotics system and find that the introduction of learning drives evolution toward diverse morphologies showing, in a way, how the brain can shape the body. The reproduction mechanism of the brain is another factor that could potentially drive the evolutionary search in different directions. Additionally, we look into the robots' ability to learn and how it changes for the effect of the reproduction mechanism.

\textbf{Research Question 3:} Will the different brain reproduction methods lead to different robot behaviours?
% To answer this question we will compare the behaviour of the best robots produced by the evolutionary process while using sexual or asexual reproduction and investigate their differences.


To answer these questions, we design an evolutionary robot system in which robot morphologies (bodies) and controllers (brain) are jointly evolved and, before reproduction, the controllers go through a phase of learning. In such systems, the best robots are chosen to produce new robotic offspring and the new genes are formed by sexual or asexual reproduction of their parent's genes. The body of every offspring is produced by recombination and mutation of the genotypes of its parents' bodies. For the inheritance of the brain, we design two different mechanisms. The first one, dubbed sexual reproduction, generates the new brain by recombination and mutation starting from the parent's genotype. The second mechanism, asexual reproduction, produces the new brain by the sole mutation of the genotype of its best parent. The new robots are then tested on two separate tasks, panoramic rotation and point navigation, and a fitness value is assigned to them based on their performance. The best individuals are selected to form the new population.

\section{Methods}
\subsection{Robot Morphology(Body)}
\subsubsection{Body Phenotype}
The phenotype of the body is a subset of RoboGen's 3D-printable components \cite{Auerbach2014}: a morphology consists of one core component, one or more brick components, and one or more active hinges. The phenotype follows a tree structure, with the core module being the root node from which further components branch out. Child modules can be rotated 90 degrees when connected to their parent, making 3D morphologies possible. The resulting bodies are suitable for both simulation and physical robots through 3D printing.

\subsubsection{Body Genotype}
The phenotype of bodies is encoded in a Compositional Pattern Producing Network (CPPN) which was introduced by Stanley \cite{Stanley2007}. The structure of the CPPN has four inputs and five outputs. The first three inputs are the x, y, and z coordinates of a component, and the fourth input is the distance from that component to the core component in the tree structure. The first three outputs are the probabilities of the modules being a brick, a joint, or empty space, and the last two outputs are the probabilities of the module being rotated 0 or 90 degrees. For both module type and rotation the output with the highest probability is always chosen; randomness is not involved.

The body's genotype to phenotype decoder operates as follows:\\
The core component is generated at the origin. We move outwards from the core component until there are no open sockets(breadth-first exploration), querying the CPPN network to determine the type and rotation of each module. Additionally, we stop when ten modules have been created. The coordinates of each module are integers; a module attached to the front of the core module will have coordinates (0,1,0). If a module would be placed on a location already occupied by a previous module, the module is simply not placed and the branch ends there. In the evolutionary loop for generating the body of offspring, we use the same mutation and crossover operators as in MultiNEAT (\url{https://github.com/MultiNEAT/}).

\subsection{Robot Controller(Brain)}
\subsubsection{Brain Phenotype}
We use Central Pattern Generators (CPGs)-based controllers to drive the modular robots. Each joint of the robot has an associated CPG that is defined by three neurons: an $x_i$-neuron, a $y_i$-neuron and an $out_i$-neuron. The recursive connection of the tree neurons is shown in Figure \ref{fig:cpg}. 
The change of the $x_i$ and $y_i$ neurons' states with respect to time is obtained by multiplying the activation value of the opposite neuron with the corresponding weight  $\dot{x}_i = w_i y_i$, $\dot{y}_i = -w_i x_i$. To reduce the search space we set $w_{x_iy_i}$ to be equal to $-w_{y_ix_i}$ and call their absolute value $w_i$. The resulting activations of neurons $x_i$ and $y_i$ are periodic and bounded. The initial states of all $x$ and $y$ neurons are set to $\frac{\sqrt{2}}{2}$ because this leads to a sine wave with amplitude 1, which matches the limited rotating angle of the joints.

% \begin{equation}
%     \begin{split}
%         \dot{x}_i &= w_i y_i \\
%         \dot{y}_i &= -w_i x_i
%     \end{split}
% \end{equation}

To enable more complex output patterns, connections between CPGs of neighbouring joints are implemented. An example of the CPG network of a "+" shape robot is shown in Figure \ref{fig:cpg_network}. Two joints are said to be neighbours if their distance in the morphology tree is less than or equal to two. 
% A robot and its network of CPGs are shown in Figure \ref{fig:cpg_network}.
Consider the $i_{th}$ joint, and $\mathcal{N}_i$ the set of indices of the joints neighbouring it, $w_{ij}$ the weight of the connection between $x_i$ and $x_j$. Again, $w_{ij}$ is set to be $-w_{ji}$. The extended system of differential equations becomes:

\begin{equation}
    \begin{split}
        \dot{x}_i &= w_i y_i + \sum_{j \in \mathcal{N}_i} w_{ji} x_j \\
        \dot{y}_i &= -w_i x_i
    \end{split}
\end{equation}

Because of this addition, $x$ neurons are no longer bounded between $[-1,1]$. For this reason, we use the hyperbolic tangent function (\emph{tanh}) as the activation function of $out_i$-neurons.

\begin{equation}
    out_{(i,t)}(x_{(i,t)}) = \frac{2}{1+e^{-2x_{(i,t)}}} - 1
\end{equation}

\begin{figure}
    \centering
    \includegraphics[width=0.9\linewidth]{figures/images/CPG_single.png}
    \caption{The structure of the CPG associated to the $i_{th}$ joint. $w_{x_iy_i}$, $w_{y_ix_i}$ and $w_{x_io_i}$ are the weights of the connections between the neurons and out is the activation value of $out_i$ neuron that controls the servo in a joint}
    \label{fig:cpg}
\end{figure}
\begin{figure}
   % \begin{minipage}{0.49\textwidth}
   %   \centering
   %   \includegraphics[width=0.7\linewidth]{figures/images/spider.png}
   % \end{minipage}\hfill
   \begin{minipage}{0.49\textwidth}
     \centering
     \includegraphics[width=.9\linewidth]{figures/images/cpg_network.png}
    \end{minipage}
   \centering
    \caption{\label{fig:cpg_network}Brain phenotype (CPG network) of a "+" shape robot. In our design, the topology of the brain is determined by the topology of the body.}
\end{figure}
\subsubsection{Brain Genotype}
For our brain genotype, we use a direct encoding of the CPG weights as an array of values. We have seen how every modular robot can be represented as a 3D grid in which the core module occupies the central position and each module's position is given by a triple of coordinates. When building the controller from our genotype, we use the coordinates of the joints in the grid to locate the corresponding CPG weight. To reduce the size of our genotype, instead of the 3D grid, we use a simplified 3D in which the third dimension is removed. For this reason, some joints might end up with the same coordinates and will be dealt with accordingly. 

Since our robots have a maximum of 10 modules, every robot configuration can be represented in a grid of $21 \times 21$. Each joint in a robot can occupy any position of the grid except the center. For this reason, the possible positions of a joint in our morphologies are exactly $(21 \cdot 21) - 1=440$. We can represent all the internal weights of every possible CPG in our morphologies as a $440$-long array. When building the phenotype from this array, we can simply retrieve the corresponding weight starting from a joint's coordinates in the body grid.

To represent the external connections between CPGs, we need to consider all the possible neighbours a joint can have. In the 2-dimensional grid, the number of cells in a distance-2 neighbourhood for each position is represented by the Delannoy number $D(2,2) = 13$, including the central element. Each one of the neighbours can be identified using the relative position from the joint taken into consideration. Since our robots can assume a 3D position, we need to consider an additional connection for modules with the same 2D coordinates.

To conclude, for each of the $440$ possible joints in the body grid, we need to store 1 internal weight for its CPG, 12 weights for external connections, and 1 weight for connections with CPGs at the same coordinate for a total of 14 weights. The genotype used to represent the robots' brains is an array of size $440 \times 14$. An example of the brain genotype of a "+" shape robot is shown in Figure \ref{fig:brain_geno}.

It is important to notice that not all the elements of the genotype matrix are going to be used by each robot. This means that their brain's genotype can carry additional information that could be exploited by their children with different morphologies.

The recombination operator for the brain genotype is implemented as a uniform crossover where each gene is chosen from either parent with equal probability. The new genotype is generated by essentially flipping a coin for each element of the parents' genotype to decide whether or not it will be included in the offspring's genotype. In the uniform crossover operator, each gene is treated separately.
The mutation operator applies a Gaussian mutation to each element of the genotype by adding a value, with a probability of 0.8, sampled from a Gaussian distribution with 0 mean and 0.5 standard deviation.

\begin{figure}
    \centering
    \includegraphics[width=0.47\textwidth]{figures/images/brain_geno.png}
    \caption{Brain genotype to phenotype mapping of a "+" shape robot. The left image (brain phenotype) shows the schema of the "+" shape robot with the coordinates of its joints in the 2D body grid. The right image (brain genotype)is the distance 2 neighbour of the joint at (1,0). The coordinates reported in the neighbourhood are relative to this joint. The CPG weight of the joint is highlighted in purple and its 2-distance neighbours are in blue.}
    \label{fig:brain_geno}
\end{figure}

\subsection{Asexual \& Sexual Reproduction}
In this research, the bodies of the robots are evolved only with sexual reproduction while the brains of the robots are evolved with asexual or sexual reproduction.

Body - sexual reproduction: Parents are selected from the current generation using binary tournaments with replacement. We perform two tournaments in which two random potential parents each are selected. In each tournament the potential parents are compared, the one with the highest fitness wins the tournament and becomes a parent. The body of every new offspring is created through recombination and mutation of the genotypes of its parents.

Brain - asexual \& sexual reproduction: For the generation of the brain, we use two different strategies. The first strategy is called asexual because the brain genotype of the offspring is generated from only one parent. The brain genotype of the best-performing parent is mutated before being inherited by its offspring. For sexual reproduction, instead, the child's brain is created through the recombination and mutation of its parents' brain genotypes.

The Algorithm \ref{alg:EL} displays the pseudocode of the complete integrated process of evolution and learning. With the highlighted blue code, it is the sexual reproduction method, without it is the asexual reproduction. With the highlighted yellow code, it is the Lamarckian learning mechanism, without it is the Darwinian learning mechanism. Note that for the sake of generality, we distinguish two types of quality testing depending on the context, evolution or learning. Within the evolutionary cycle (line 2 and line 14) a test is called an evaluation and it delivers a fitness value. Inside the learning cycle (line 11) a test is called an assessment and it delivers a performance value. 

We apply Reversible Differential Evolution (RevDE) \cite{Tomczak2020} as a learning method for `newborn' robots. In particular, it will be used to optimize the weights of the CPGs of our modular robots for the tasks during the Infancy stage. 

The code for replicating this work and carrying out the experiments is available online: \url{https://bit.ly/XXXX}. 

\begin{algorithm}[h!]
  \caption{Evolution+Learning}
  \label{alg:EL}
  \begin{algorithmic}[1]
    \State INITIALIZE robot population (genotypes + phenotypes with body and brain)  
    \State EVALUATE each robot  (evaluation delivers a fitness value)
    %\State EVALUATE each robot's CPG weights $\overrightarrow{\mathcal{W}}_1, \overrightarrow{\mathcal{W}}_2, ..., \overrightarrow{\mathcal{W}}_n$ to obtain the fitness $f_1, f_2, ...,f_n$.
    \While{not STOP-EVOLUTION}
    %\While{generation g<=30 \& g>1}
        \State SELECT parents; (based on fitness)
        \State RECOMBINE+MUTATE parents' bodies; (this delivers a new body genotype)
        
        \tikzmk{A}
        \State RECOMBINE parents' brains;
        
        \tikzmk{B} \boxit{blue}
        \State MUTATE parents' brains; (this delivers a new brain genotype)
        \State CREATE offspring robot body; (this delivers a new body phenotype)
        \State CREATE offspring robot brain; (this delivers a new brain phenotype)
        
        \State INITIALIZE brain(s) for the learning process; (in the new body)
        \While{not STOP-LEARNING}
        %\While{learning l<=100 \& l>0}
            \State ASSESS offspring; (assessment delivers a performance value)
            \State GENERATE new brain for offspring;
        \EndWhile 
        \State EVALUATE offspring with learned brain; (evaluation delivers a fitness value) 
        
        \tikzmk{A}
        \State UPDATE brain genotype 
        %(using genome of best fitness) 

        \tikzmk{B} \boxit{yellow}
        \State SELECT survivors / UPDATE population
          
    \EndWhile
    %\State \textbf{return} data $(\overrightarrow{\mathcal{W}}_{1:(g*n)},f_{1:(g*n)})$.
 \end{algorithmic}
\end{algorithm}

% \subsection{Learning algorithm}
% In a recent study on modular robots \cite{Diggelen2021}, it was demonstrated that Reversible Differential Evolution (RevDE) \cite{Tomczak2020, weglarz2021population}, an altered version of Differential Evolution, performs and generalizes well across various morphologies. This method works as follows \cite{Tomczak2020}:
% \begin{enumerate}
%     \item Initialize a population with \textit{$\mu$} samples ($n$-dimensional vectors), $\mathcal{P}_{\mu}$. 
%     %Here we add Gaussian noise to self-mutate the weights of CPGs ten times to obtain the initial population.
%     \item Evaluate all \textit{$\mu$} samples.
%     \item Apply the reversible differential mutation operator and the uniform crossover operator.\\
%     \textit{The reversible differential mutation operator}: Three new candidates are generated by randomly picking a triplet from the population, $(\mathbf{w}_i,\mathbf{w}_j,\mathbf{w}_k)\in \mathcal{P}_{\mu}$, then all three individuals are perturbed by adding a scaled difference in the following manner:
%         \begin{equation}\label{eq:de3}
%             \begin{split}
%             \mathbf{w}_1 &= \mathbf{w}_i + F \cdot (\mathbf{w}_j-\mathbf{w}_k) \\
%             \mathbf{v}_2 &= \mathbf{w}_j + F \cdot (\mathbf{w}_k-\mathbf{v}_1) \\
%             \mathbf{v}_3 &= \mathbf{w}_k + F\cdot (\mathbf{v}_1-\mathbf{v}_2) 
%             \end{split}
%         \end{equation}
%         where $F\in R_+$ is the scaling factor. New candidates $y_1$ and $y_2$ are used to calculate perturbations using points outside the population. This approach does not follow the typical construction of an EA where only evaluated candidates are mutated.\\
%         \textit{The uniform crossover operator}: Following the original DE method \cite{Storn1997}, we first sample a binary mask $\mathbf{m} \in \{0, 1\}^D$ according to the Bernoulli distribution with probability \textit{$CR$} shared across $D$ dimensions, and calculate the final candidate according to the following formula:
%         \begin{equation}\label{eq:de2}
%               \mathbf{u} = \mathbf{m} \odot \mathbf{w}_n+(1-m) \odot \mathbf{w}_n .
%         \end{equation}
%         Following general recommendations in literature \cite{Pedersen2010} to obtain stable exploration behaviour, the crossover probability CR is fixed to a value of $0.9$ and according to the analysis provided in \cite{Tomczak2020}, the scaling factor $F$ is fixed to a value of 0.5. 
%     \item Perform a selection over the population based on the fitness value and select \textit{$\mu$} samples.
%     \item Repeat from step (2) until the maximum number of iterations is reached.
% \end{enumerate}

% As explained above, we apply RevDE here as a learning method for `newborn' robots. In particular, it will be used to optimize the weights of the CPGs of our modular robots for the tasks during the Infancy stage. The initial population of $X = 10$ weight vectors for RevDE is created by using the inherited brain of the given robot. Specifically, the values of the inherited weight vector are altered by adding Gaussian noise to create mutant vectors and the initial population consists of nine such mutants and the vector with the inherited weights.
% % add Gaussian noise to self-mutate the weights of CPGs ten times to obtain the initial population.

\subsection{Tasks and Fitness functions}
\paragraph{Point Navigation}
Point navigation is a closed-loop controller task which needs feedback from the environment passing to the controller to steer the robot. A robot is spawned at the centre of a flat arena (10 × 10 m2) to reach a sequence of target points $P_1,...,P_N$. In each evaluation, the robot has to reach as many targets in order as possible. Success in this task requires the ability to move fast to reach one target and then quickly change direction to another target in a short duration. A target point is considered to be reached if the robot gets within 0.01 meters from it. 
The fitness function for this task is designed to maximize the number of targets reached and minimize the path followed by the robot to reach the targets.

The data collected from the simulator is the following:
\begin{itemize}
    \item The coordinates of the core component of the robot at the start of the simulation are approximate to $O (0,0)$;
    \item The coordinates of the robot, sampled during the simulation at 5Hz, allowing us to plot and approximate the length of the followed path;
    \item The coordinates of the robot at the end of the simulation $P_T(x_T,y_T)$;
    \item The coordinates of the target points $P_1(x_1,y_1)$... $P_n(x_n,y_n)$.
\end{itemize}

Being $k$ the number of target points reached by the robot at the end of the evaluation, and $L$ the path followed, the fitness function is the following.
\begin{multline}
    F=\sum_{i=1}^{k}dist(P_i,P_{i-1})+(dist(P_{k+1},P_k) \\
    - dist(P,P_{k+1})) - \omega \cdot L
\end{multline}
The first term of the function is a sum of the distances between the target points the robot has reached. The second term is necessary when the robot has not reached all the targets and is the distance travelled toward the next one. The last term is used to penalize longer paths and $\omega$ is a constant scalar that is set to 0.1 in the experiments. 

\paragraph{Panoramic Rotation}
 Panoramic Rotation task is an open-loop controller task which does not need any feedback from the environment to feed the controller. Same as the Point navigation, the initial coordinate of the robot is [0,0]. Success in this task requires the ability to rotate 360 degrees around the robot's vertical axis as many times as possible in the evaluation time. 

To solve this task, we collect from the simulator the orientation of the robot sampled at 5 Hz during the evaluation. Since the robot's orientations are represented as quaternions in our simulation, we first have to convert them to the equivalent triplet of vectors $v_i, v_j \text{ and } v_k$. The conversion formulas are:
\begin{multline}
    \cr v_i = [1 - 2(q_2^2 + q_3^2), 2(q_1 \cdot q_2 - q_3 \cdot q_0), 2(q_1 \cdot q_3 + q_2 \cdot q_0)]
    \cr v_j = [2(q_1 \cdot q_2 + q_3 \cdot q_0, 1 - 2(q_1 ^ 2 + q_3 ^ 2,
    2(q_2 \cdot q_3 - q_1 \cdot q_0)]
    \cr v_k = [2(q_1 \cdot q_3 - q_2 \cdot q_0), 2(q_2 \cdot q_3 + q_1 \cdot q_0), 1 - 2(q_1^2 + q_2^2)]
\end{multline}
where $q = (q_0, q_1, q_2, q_3)$ is the quaternion representing the orientation.
$v_i, v_j \text{ and } v_k$ are the vectors obtained by applying the rotation described by the quaternion to the base vectors $(i,j,k)$.

The robots, due to their modularity, can have a wide range of shapes and orientations. For each one of them, we need to identify the orientation vector whose rotation is going to be measured. At the start of the simulation, the orientation vector with the lowest x-axis component is identified. The rotation of this vector is computed to measure the total rotation of the robot. 

The angle between two vectors can be computed using their determinant and dot product. The dot product of two vectors $a, b$ is proportional to the angle between them and their determinant is proportional to the sine. We can compute the angle as:
\begin{equation}
    \theta_i = atan2(det, dot)
\end{equation}
where $det = a_x \cdot b_y - b_x \cdot a_y$ and $dot = a_x \cdot a_y + b_x \cdot b_y$.

Given the chosen vector at two consecutive timestamps $T_{i-1}, T_i$, the rotation between them can be computed using the method just described.

The final fitness function is the total rotation (in radians) of the robot computed as the sum of the rotation of the orientation vector at each consecutive timestamp.

\begin{equation}
    F = \sum_{n=1}^{30} \theta_i
\end{equation}

\vspace{-0.4em}
We assign a positive sign to the counter-clockwise rotations and a negative one to the clockwise rotations in our tests.

\section{Experimental setup}
The stochastic nature of evolutionary algorithms requires multiple runs under the same conditions and a sound statistical analysis (\cite{bartz2007experimental}). We perform 10 runs for each evaluation task, reproduction mechanism and evolutionary framework, namely Rotation Asexual Darwinian, Rotation Asexual Lamarckian, Rotation Sexual Darwinian, Rotation Sexual Lamarckian, Point Navigation Asexual Darwinian, Point Navigation Asexual Lamarckian, Point Navigation Sexual Darwinian, Point Navigation Sexual Lamarckian. In total, 80 experiments.

Each experiment consists of 30 generations with a population size of 50 individuals and 25 offspring. A total of $50+(25\cdot(30-1))=775$ morphologies and controllers are generated, and then the learning algorithm RevDE is applied on each controller. For RevDE we use a population of 10 controllers for 10 generations, for a total of $(10+30\cdot(10-1))=280$ performance assessments.

The fitness measures used to guide the evolutionary process are the same as the performance measure used in the learning loop. For this reason, we use the same test process for both.
The tests for the task of point navigation use 40 seconds of evaluation time with two target points at the coordinates of $(1, -1)$ and $(0, -2)$. The evaluation time for for panoramic rotation is 30 seconds.

All the experiments are run with Mujoco simulator-based wrapper called Revolve2 on a 64 cores Linux computer, where they each take approximately 15 hours to finish, totalling 1,200 hours of computing time.

% \begin{table}[h]
% \caption{Main experiment parameters}
% % \centering
% \begin{tabular}{{p{0.18\linewidth} | p{0.1\linewidth}| p{0.6\linewidth}}}
% \toprule
% Parameters       & Value & Description                                    \\ \midrule
% Population size  & ~50    & Number of individuals per generation     \\
% Offspring size  & ~25    & Number of offspring produced per generation     \\
% % Mutation         & ~0.8   & Probability of mutation for individuals        \\ 
% % Crossover         & ~0.8   & Probability of crossover for individual        \\ 
% Generations      & ~30   & Termination condition for each run             \\ 
% Learning trials  & ~280    & Number of the evaluations performed by RevDE on each robot \\ 
% % $\mu$ 			 & ~10     & RevDE - Population size 	\\
% % N 			 & ~30     & RevDE - New candidates per iteration	\\
% % $\lambda$ 			 & ~10     & RevDE - Top-sample size 	\\
% % $F$ 			 & ~0.5    & RevDE - Scaling factor 	\\
% % $CR$  & ~0.9    & RevDE - Crossover probability \\ 
% % Iterations  & ~20    & RevDE - Number of iterations \\ 
% %Evaluation time  & ~30    & Duration of evaluation in seconds for rotation task \\ 
% %Evaluation time  & ~40    & Duration of evaluation in seconds for point navigation task\\ 
% Tournament size  & ~2     & Number of individuals used in the parent selection - (k-tournament)		 \\ 
% % $\lambda \slash \mu$ & ~0.5    & The ratio used in the survivor selection - ($\mu + \lambda)$  \\
% Repetitions      &  ~10    & Number of repetitions per experiment \\ 
% \bottomrule 
% \end{tabular}
% \label{tab:parameters}
% \end{table}
\section{Results}
To compare the effects of asexual and sexual reproduction, we consider two generic performance indicators: efficiency and efficacy, meanwhile we also look into robots' behaviour and morphologies.
\subsubsection{Efficacy}
We measure efficacy by the mean and maximum fitness value within the simulation time at the end of the evolutionary process (30 generations), then we take the average over 10 independent repetitions. 

Figure \ref{fig:fitness_mean_avg} shows both reproduction methods can generate robots that can solve the two tasks successfully. It also shows that asexual and sexual reproduction has the opposite effect for different evolution frameworks on two tasks. With the point navigation task, sexual reproduction achieved a higher fitness value in the Darwinian framework, while in the Lamarckian framework, asexual reproduction has a higher mean fitness value across generations.

With the panoramic rotation task, there is no significant difference between the two reproduction methods in the Darwinian framework, however, in the Lamarckian framework, the asexual method has a significantly higher mean fitness value across generations than the sexual method.

\begin{figure*}[ht!] 
  \centering
     \begin{subfigure}[b]{0.49\textwidth}
         \centering
         \includegraphics[width=0.7\textwidth]{figures/images/fitness_avg_max_lineplot_dar_point_nav.png}
         \caption{}
     \end{subfigure}
     \hfill
     \begin{subfigure}[b]{0.49\textwidth}
         \centering
         \includegraphics[width=0.7\textwidth]{figures/images/fitness_avg_max_lineplot_lam_point_nav.png}
         \caption{}
     \end{subfigure}
     \hfill
     \begin{subfigure}[b]{0.49\textwidth}
         \centering
         \includegraphics[width=0.7\textwidth]{figures/images/fitness_avg_max_lineplot_dar_rotation.png}
         \caption{}
     \end{subfigure}
     \hfill
     \begin{subfigure}[b]{0.49\textwidth}
         \centering
         \includegraphics[width=0.7\textwidth]{figures/images/fitness_avg_max_lineplot_lam_rotation.png}
         \caption{}
     \end{subfigure}
     \hfill
  \caption{Mean fitness over 30 generations (averaged over 10 runs) for asexual reproduction in purple and sexual reproduction in blue. Subfigures (a)(b) exhibit mean average fitness for the point navigation task, and Subfigures (c)(d) are for the rotation task. The bands indicate the 95\% confidence intervals ($\pm1.96\times SE$, Standard Error).}
  \label{fig:fitness_mean_avg} 
\end{figure*}

Second, another way to measure the efficacy of the solution is by giving the same computational budget (number of generation) and measuring which method finds the best solution (maximum fitness) faster. In Figure \ref{fig:efficacy_boxplot}, for the Darwinian framework, sexual reproduction has a better mean fitness on point navigation task. Although the difference in mean fitness is not significant on the rotation task, the best sexually reproducing robot for the rotation task with the highest fitness value, 80.21, is almost twice as better as the best robot by asexual reproduction, whose fitness is only 44.52. For the Lamarckian framework, mean and max fitness values of asexual reproduction are significantly better than sexual reproduction's on both tasks.
\begin{figure*}
   \begin{minipage}{0.48\textwidth}
     \centering
     \includegraphics[width=0.8\linewidth]{figures/images/boxplot_final_gen_dar_point_nav.png}
   \end{minipage}\hfill
   \begin{minipage}{0.48\textwidth}
     \centering
     \includegraphics[width=.8\linewidth]{figures/images/boxplot_final_gen_lam_point_nav.png}
    \end{minipage}
       \begin{minipage}{0.48\textwidth}
     \centering
     \includegraphics[width=0.8\linewidth]{figures/images/boxplot_final_gen_dar_rotation.png}
   \end{minipage}\hfill
   \begin{minipage}{0.48\textwidth}
     \centering
     \includegraphics[width=.8\linewidth]{figures/images/boxplot_final_gen_lam_rotation.png}
    \end{minipage}
   \centering
    \caption{\label{fig:efficacy_boxplot}Efficacy boxplots. The fitness values of two reproduction methods at generation 30 for two tasks and two evolution frameworks. Red dots show mean values.}
\end{figure*}

\begin{figure*}
    \centering
    \begin{subfigure}[b]{0.485\textwidth}
        \centering
        \includegraphics[width=0.9\textwidth]{figures/images/trajectory_darwinian.png}
        % \caption{}
    \end{subfigure}
    \hfill
    \begin{subfigure}[b]{0.485\textwidth}  
        \centering 
        \includegraphics[width=0.9\textwidth]{figures/images/trajectory_lamarckian.png}
        % \caption{}
    \end{subfigure}
    \caption{Trajectories of the best 10 robots from both reproduction methods in the point navigation task with Darwinian and Lamarckian evolution frameworks. The purple square is the starting point. Two yellow circles are the target points which robots aim to go through. The blue lines are the trajectory paths of robots produced by the asexual reproduction method ending at the blue squares. The green lines are from the sexual reproduction method ending at the green squares.}
    \label{fig:trajectories}
\end{figure*}

\begin{figure*}
   \begin{minipage}{0.49\textwidth}
     \centering
     \includegraphics[width=0.80\linewidth]{figures/images/asexual_rotation_traj.png}
   \end{minipage}\hfill
   \begin{minipage}{0.49\textwidth}
     \centering
     \includegraphics[width=0.80\linewidth]{figures/images/sexual_rotation_traj.png}
    \end{minipage}
   \centering
    \caption{\label{fig:rotation_traj}Vertical coordinate of the trajectories of the 20 best robots from the last generation in the panoramic rotation task for both reproduction methods with Darwinian framework. The robots from sexual reproduction with Darwinian framework exhibit a jumping behaviour that helps them rotate faster.}
\end{figure*}

\begin{figure*}[ht!] 
  \centering
     \begin{subfigure}[b]{0.49\textwidth}
         \centering
         \includegraphics[width=0.94\textwidth]{figures/images/best_sexual_pn.png}
         \caption{}
     \end{subfigure}
     \hfill
     \begin{subfigure}[b]{0.49\textwidth}
         \centering
         \includegraphics[width=0.94\textwidth]{figures/images/best_asexual_pn.png}
         \caption{}
     \end{subfigure}
     \hfill
     \begin{subfigure}[b]{0.49\textwidth}
         \centering
         \includegraphics[width=0.94\textwidth]{figures/images/best_sexual_rot.png}
         \caption{}
     \end{subfigure}
     \hfill
     \begin{subfigure}[b]{0.49\textwidth}
         \centering
         \includegraphics[width=0.94\textwidth]{figures/images/best_asexual_rot.png}
         \caption{}
     \end{subfigure}
     \hfill
  \caption{Subfigures (a)(b): The 5 best robots produced for the point navigation task by sexual and asexual reproduction and their fitnesses. sexually reproducing robots have higher fitness values. Subfigures (c)(d): The 5 best robots produced for the panoramic rotation task by sexual and asexual reproduction and their fitnesses. sexually reproducing robots have higher fitness values and are more diverse.}
  \label{fig:best5} \vspace{-1em}
\end{figure*}

\subsubsection{Efficiency}
Efficiency indicates how much effort is needed to reach a given quality threshold (the fitness level). In this paper, we use the average number of evaluations to solution to measure it. In figure \ref{fig:fitness_mean_avg}, the most efficient method for the point navigation task is asexual reproduction with the Lamarckian framework. It already surpassed in its 15, 21, 21 generations the fitness levels that asexual reproduction with Darwinian, sexual reproduction with Darwinian and sexual reproduction with Lamarckian methods achieved at the end of the evolutionary period respectively. So does the rotation task, asexual reproduction with Lamarckian framework surpassed in its 16, 17, 25,  generations the fitness levels that sexual with Lamarckian method, asexual with Darwinian, and sexual with Darwinian achieved at the end of the evolutionary period respectively.

\subsubsection{Robot Behavior}
To get a better understanding of the robots' behaviour, we visualize the trajectories of the 10 best-performing robots from both reproduction methods for the point navigation task in the last generation across all runs. Figure\ref{fig:trajectories} shows that with the Lamarckian framework, all the robots reached the two target points much earlier than the ones with the Darwinian framework. In the Darwinian framework, both reproduction methods reached the first target point successfully. However, the majority of the robots that use sexual reproduction with the Darwinian framework reach the second target point while only some of the robots produced using asexual reproduction can do the same. In the Lamarckian framework, both reproduction methods reached two target points successfully.  

In order to get an insight into why the best-performing robots are produced using sexual reproduction with the Darwinian framework for the rotation task, we plot the vertical coordinate of the best robots from the last generations. Figure \ref{fig:rotation_traj} shows the best-performing robots produced using sexual reproduction with Darwinian framework keep a higher position on the vertical coordinate on average. This is mostly due to a very effective behaviour which is composed of little jumps. On the other hand, the best robots of asexual reproduction with Darwinian framework adopt a simpler behaviour by which they rotate while staying on the ground.

\subsubsection{Robot Morphologies}
Figure \ref{fig:best5} present the 5 best robots for each method. The morphologies evolved for both tasks are reaching maximum size, having close to 10 modules on average, and are mostly made of hinges with no bricks (except one).
The best robots for point navigation have 3 or 4 limbs made of hinges while those for rotation task only have 2 or 3.

\section{Conclusions and Future Work}
% In Evolutionary Robotics, the choice between the two reproduction methods will depend on the characteristics of the task and environment. If the goal of the optimization is to optimize a particular trait or set of traits, asexual reproduction may be the better choice. On the contrary, if we aim to explore a wider range of solutions and adapt to changing conditions, sexual reproduction might be more advantageous. In practice, combining the two mechanisms may be the better choice to benefit from the advantages of both.
We compared asexual and sexual brain reproduction methods in a morphologically evolving robot system. Since our system also lets `infant' robots optimize their inherited brains by learning, we also had two options regarding the combination of learning and evolution: Darwinian and Lamarckian. Given the two tasks we  considered --point navigation and panoramic rotation-- all together we conducted eight experiments.The results show that to achieve the highest task performance, the use of sexual brain reproduction is advisable in the Darwinian system for both tasks. (Although for one task, the the two reproduction systems performed similarly.) However, the effect is the opposite in the Lamarckian framework, since asexual reproduction leads to robots with higher fitness. This answers our first research question. Our experiments also show that to maximize task performance, the use asexual reproduction and the Lamarckian framework is the best choice. This is a novel result that can impact the design of evolutionary robot systems of the future.

With regards to the evolved morphologies, both reproduction methods drive the evolution process towards maximum-size robots composed of many active hinges. The morphologies of the best-performing robots for the same task are similar: they are made of only hinges attached to the core module without bricks. For the rotation task, bodies mainly converged to an "L" shape and to and "X" shape for the point navigation task. We conclude that under the given experimental conditions the morphology of the robot is mainly determined by the tasks, not the brain reproduction methods.

Regarding our third research question, we highlight a difference in the behaviours of the robots that evolved using the two reproduction methods for both tasks. In point navigation, the trajectories of the best robots produced by both reproduction of the Lamarckian framework reached the target points much earlier than the ones from the Darwinian framework where the majority of the best robots evolved using sexual reproduction reach the second target point while only a few of those asexually evolved do so. In the task of rotation, one interesting behaviour occurs from the Darwinian sexual production method. It evolves a jumping behaviour that let the robots rotate faster. 

Future work will be directed to test the superiority of asexual brain reproduction and a Lamarckian combination of evolution and learning on more tasks and environments.  
%validate the study the different reproduction methods in changing environments and test reproduction methods using multiple parents. Other future works will also allow for the possibility of encoding the reproduction strategy in the genotype of the robots and finding the best one using evolution. 
\clearpage
\bibliographystyle{apalike}
\bibliography{references} 
\end{document}
