

In this work, we propose a practical scenario, namely partial KB inference, in biomedical EL and give a detailed definition and evaluation procedures for it.
We review and categorize current state-of-the-art entity linking models into three paradigms.
Through experiments, we show NER-NED and simultaneous generation paradigms have vulnerable performance toward partial KB inference which is mainly caused by mention detection precision drop.
The NED-NER paradigm is more robust due to well-modeled mention-concept reliance.
We also propose two methods to redeem the performance drop in partial KB inference and discover out-KB annotations may enhance the in-KB performance.
Post-pruning and thresholding can both improve the performance of NER-NED and simultaneous generation paradigms. Although post-pruning is easy-to-use, it needs to store the large KB $\mathcal{E}_1$ (with their embeddings or trie) which has large memory consumption.
Thresholding does not rely on large KB $\mathcal{E}_1$ which also has better performance on the NER-NED paradigm.
Our findings illustrate the importance of partial KB inference in EL which shed light on the future research direction.
% Our work 