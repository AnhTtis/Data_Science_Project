We only investigate representative methods of three widely-used EL paradigms.
However, there are more EL methods and paradigms we may not cover, and we leave them as future works.
Furthermore, more auxiliary information in the biomedical domain can be introduced to address the NIL issue we identify in this work.
For example, a hierarchical structure exists for concepts in KBs in the biomedical domain.
Therefore, NIL may be solved by linking them to hypernym concepts in the partial KBs \cite{nilinker}.
We consider the hierarchical mapping between NILs and in-KB concepts as a potential solution for performance degradation in partial KB inference.

Users can obtain different entity-linking results based on their own KBs which have the potential risk of missing important clinical information from the texts.

\section*{Ethics Statement}
Datasets used for building partial KB inference do not contain any patient privacy information.

\section*{Acknowledgement}

We would like to express our appreciation and gratitude to Professor Sheng Yu from Center for Statistical Science, Tsinghua University and Professor Muhao Chen from University of Southern California who have provided computational resources for this research. Vive l'amiti\'e parmi les auteurs.


