
In this section, we introduce our experimental setup,
which includes implementation details of EL methods we investigated (\Cref{sec:methods}) and datasets we create for investigating partial KB inference (\Cref{sec:datasets}).

\begin{table*}[th]
\small 
\centering
\setlength{\tabcolsep}{1.5mm}{
\begin{tabular}{p{1mm}ll|ccc|ccc|ccc}
\toprule
& \multicolumn{2}{c}{Target KB.}& \multicolumn{3}{c}{EntQA} & \multicolumn{3}{c}{GENRE}& \multicolumn{3}{c}{KeBioLM+CODER} \\
& Train KB & Eval KB &Precision & Recall & F1&Precision & Recall & F1&Precision & Recall & F1 \\ 
\hline
\cellcolor{blue!10}& UMLS & UMLS  & 45.99&23.68&31.27&42.44&43.69&43.05&33.58&34.94&34.25\\
\cellcolor{blue!10}&&\SNOMEDint  & 46.04&27.01&34.05&34.40&49.40&40.56&28.19&48.28&35.59\\
\cellcolor{blue!10}&&\SNOMEDext  & 
36.75 & 23.12 & 28.38 & 19.82 & 39.28 & 26.35 & 14.18 & 37.54 & 20.59 \\
\cellcolor{blue!10}&&\TAint  & 41.52&31.56&35.86&17.26&49.53&25.60&9.78&50.28&16.37\\
\cellcolor{blue!10}&&\TAext  & 
43.43 & 23.24 & 30.28 & 34.97 & 42.45 & 38.35 & 26.52 & 34.59 & 30.02 \\
\cellcolor{blue!10}& &\TBint  & 30.01&25.56&27.61&7.69&36.06&12.68&4.76&41.51&8.54\\
\cellcolor{blue!10}\multirow{-7}{*}{\rotatebox[origin=c]{90}{\textit{MedMentions}}} &&\TBext  & 
 46.02 & 24.34 & 31.84 & 40.45 & 44.76 & 42.50 & 31.95 & 37.74 & 34.61 \\
\hline
\multicolumn{3}{c}{Avg. Drop}\vline & 5.36 & -2.13 & -0.7 & 16.68 & 0.11 & 12.04 & 14.35 & -6.71 & 9.96 \\
\hline
\cellcolor{blue!10}&MeSH&MeSH & 83.59 & 66.48 & 74.06 &70.92&68.71&69.80&72.21&74.84&73.5 \\
\cellcolor{blue!10}&&\MEDICint  & 81.92&70.45&75.75&31.53&68.19&43.12&29.24&68.38&40.96\\
\cellcolor{blue!10}\multirow{-3}{*}{\rotatebox[origin=c]{90}{\textit{BC5.}}}&&\MEDICext & 
87.10 & 66.92 & 75.69 & 37.55 & 65.33 & 47.69 & 42.57 & 80.67 & 55.73 \\
\hline
\multicolumn{3}{c}{Avg. Drop}\vline & -0.92 & -2.21 & -1.66 & 36.38 & 1.95 & 24.40 & 36.31 & 0.32 & 25.16 \\
\bottomrule
\end{tabular}}
\caption{Results for entitly linking in parital KB inference. The first section shows results on MedMentions with UMLS as training KB. The last section shows results on BC5CDR with MeSH as training KB. Eval KB represents different partial KBs for inference. 
% The first row of each section are evaluated using training KBs. 
% And others are results that we evaluate models on subsets of training KB in zero-shot setting. 
The average drops are averaged among metrics between full evaluation (first row in each section) and partial KB evaluation (other rows).}
\label{tab:main_end-to-end_result}
\small 
\vspace{-1em}
\end{table*}



\subsection{Direct Partial KB Inference}\label{sec:methods}
\section{Experiments}\label{sec:experiments}

In this section, DISTRO is evaluated for a variety of robust ID and OOD tests and is compared to previous approaches.
As baseline, we consider the pre-trained models\footnote{\href{https://github.com/AlexMeinke/Provable-OOD-Detection}{https://github.com/AlexMeinke/Provable-OOD-Detection}} from \citet{prood}.
The normal trained (\textbf{Plain}) and outlier exposure (\textbf{OE})~\cite{oe} models share the same ResNet18~\cite{resnet} architecture and hyperparameters as \textbf{ProoD}~\cite{prood}.
\textbf{GOOD}~\cite{good} uses a 'XL' convolutional neural network.
Additionally, we evaluate the pretrained DenseNet101~\cite{densenet} models for \textbf{ATOM}~\cite{atom} and \textbf{ACET}~\cite{acet}; and the standard OOD detection methods: \textbf{VOS}\footnote{\href{https://github.com/deeplearning-wisc/vos}{https://github.com/deeplearning-wisc/vos}}~\cite{vos} and \textbf{LogitNorm}\footnote{\href{https://github.com/hongxin001/logitnorm_ood}{https://github.com/hongxin001/logitnorm\_ood}}~\cite{logitnorm} with the pretrained WideResNet40~\cite{wideresnet} models provided in the respective works.
We consider \textbf{DDS}~\cite{dds} with a pre-trained diffusion model\footnote{\href{https://github.com/openai/improved-diffusion}{https://github.com/openai/improved-diffusion}} from \citet{nichol2021improved} in front of the OE classifier.
With \textbf{DISTRO}, we incorporate the same pre-trained diffusion model of DDS before the main classifier of ProoD, and maintain its discriminator.
The diffusion models have been used with the settings described in \citet{dds}.
In the context of $\ell_\infty$, we set $\sigma = \sqrt{d} \cdot \epsilon$.

We evaluate all methods on the standard datasets \texttt{CIFAR10/100}~\cite{cifar} as ID.
For the OOD detection evaluation we consider the following set of datasets: 
\texttt{CIFAR100/10}, \texttt{SVHN}~\cite{svhn}, LSUN~\cite{lsun} cropped (\texttt{LSUN\_CR}) and resized (\texttt{LSUN\_RS}),  TinyImageNet~\cite{tiny} cropped (\texttt{TinyImageNet\_CR}), \texttt{Textures}~\citep{textures} and synthetic (\texttt{Gaussian} and \texttt{Uniform}) noise distributions.
We use a random but fixed subset of 1000 images for all datasets considered as a test for OOD.
For ID, we consider the entire dataset.
We run all our experiments on a single NVIDIA A100. 

\subsection{In-Distribution Results}\label{sec:id-results}

Here, we compare clean, adversarial, and certified accuracy for ID samples.
Adversarial accuracy is evaluated with AutoAttack~\citep{apgd} for $\ell_\infty$-norm attacks of budget $\epsilon \in \{\nicefrac{2}{255}, \nicefrac{8}{255}\}$.
We ran the standard version of AutoAttack without additional hyper-parameters. 
Certified accuracy is evaluated for $\ell_2$-norm robustness of deviation $\sigma \in \{0.12, 0.25\}$.
To this end, random smoothing is performed on 10'000 Gaussian distributed samples around the input with a failure probability of $0.001$.
All $R>0$ are considered for the certified accuracy.
In the context of DISTRO and DDS we run 100 evaluation of the entire test set of \texttt{CIFAR10} to estimate the clean accuracy and report the average.
Further, we ran AutoAttack in both \textit{rand} and \textit{standard} modes, and considered the lowest results for DISTRO and DDS.


\begin{table}[htb]
\vspace{-0.5em}
    \centering
    \caption{\textbf{ID Accuracy}: Results of clean, adversarial and certified accuracy (\%) on the \texttt{CIFAR10} test set.
    The grayed-out models have an accuracy drop greater than $3\%$ relative to the model with the highest accuracy.}
    \label{tab:in-distribution}
    \begin{adjustbox}{width=0.5\textwidth,center}
        \begin{tabular}{llccccc}
            \toprule
            \multirow{2}{*}{Method} &\multirow{2}{*}{Clean} &\multicolumn{2}{c}{Adversarial ($\ell_\infty$)} &\multicolumn{2}{c}{Certified ($\ell_2$)} \\
            & &$\epsilon = \nicefrac{2}{255}$ &$\epsilon = \nicefrac{8}{255}$ &$\sigma=0.12$ &$\sigma = 0.25$ \\
            \midrule
            Plain$^*$       &95.01  &2.16   &0.00   &28.14  &14.17 \\
            OE$^*$          &95.53 &1.97   &0.00   &31.48  &10.88 \\
            VOS$^\dag$      &94.62  &2.24   &0.00   &13.13   &10.02       \\
            LogitNorm$^\ddag$  &94.48  &2.65   &0.00   &12.53  &10.25 \\
            \gray{ATOM$^*$}    &\gray{92.33}  &\gray{0.00}   &\gray{0.00}   &\gray{0.00}   &\gray{0.00}  \\
            \gray{ACET$^*$}    &\gray{91.49}  &\gray{69.01}  &\gray{6.04}   &\gray{57.13}  &\gray{12.48} \\
            \gray{GOOD$^*_{80}$} &\gray{90.13}  &\gray{11.65}  &\gray{0.23}   &\gray{17.33}  &\gray{10.31} \\
            ProoD$^*$ $\Delta=3$  &95.46  &2.69   &0.00   &33.92  &13.50 \\
            DDS                   &\textbf{95.55} &72.97 &24.09 &82.26 &64.58 \\
            DISTRO (our)          &95.47  &\textbf{73.34} &\textbf{27.14}  &\textbf{82.77}   &\textbf{65.63} \\
            \bottomrule
        \end{tabular}
    \end{adjustbox}
    \scriptsize{$*$ Pre-trained models from \citet{prood}, $\dagger$ Pre-trained from \citet{vos}, \\ $\ddag$ Pre-trained from \citet{logitnorm}.
    }
\vspace{-2em}
\end{table}

In \autoref{tab:in-distribution}, we show the results.
As expected, Plain and OE are not robust to adversarial attacks.
This applies to ProoD as well, since OE is its primary classifier.
Similarly, standard OOD detection methods, as LogitNorm and VOS, show poor robustness for ID data.
GOOD demonstrates better results than ProoD for adversarial attacks and worse in terms of certified accuracy.
Suprisingly, ACET reveals strong adversarial and certified accuracy despite of its reduced clean accuracy.
Meanwhile, ATOM results in zero for all tests since any slight perturbation of the input triggers the last neuron used for OOD detection.

\subsubsection*{Discussion}

It is clear that diffusion models can enhance adversarial and certified robustness while maintaining high clean accuracy.
As diffusion introduces variance into gradient estimators, standard attacks become much less effective.
Nevertheless, robustness accuracy of diffusion models varies over different runs for the same input, so it should be defined differently from deterministic accuracy, e.g. as expectation.
Luckily, one-shot diffusion introduces such a tiny variance that throughout a few of runs, our results were similar.

\subsection{Datasets}\label{sec:datasets}

We conduct experiments on two widely-used biomedical EL datasets and select several partial KBs used as inference.
Selection biases of partial KBs may be introduced into our setting because different partial KBs may result in different target distributions of mention-concept annotations, as this may lead to different difficulties in EL due to different KB sizes, the semantics of entities, and entity occurrence frequencies in the training set.
% This might prejudice our empirical findings.
To eliminate this effect as much as possible, we not only evaluate on partial KBs mentioned above but also their complement KBs to the training KBs.
We add $\complement$ to indicate the complements. 
The detailed statistics of datasets are listed in \Cref{tab:stats} of \Cref{app:data_stat}.

\textbf{BC5CDR}~\cite{li2016biocreative} is a dataset that annotates 1,500 PubMed abstracts with 4,409 chemicals, 5818 disease entities, and 3,116 chemical-disease interactions. 
All annotated mentions are linked to concepts in the target knowledge base MeSH.
We use MeSH as the training KB and we consider a smaller KB \textit{MEDIC} \cite{davis2012medic} as the partial KB for inference.
MEDIC is a manually curated KB composed of 9,700 selected disease concepts mainly from MeSH.
% To meet the assumption that MEDIC forms a subset of MeSH, we ditch the concepts in MEDIC that do not exist in MeSH.

\textbf{MedMentions} \cite{medmentions} is a large-scale biomedical entity linking datasets curated from annotated PubMed abstracts.
We use the \textit{st21pv} subset which comprises 4,392 PubMed abstracts, and over 350,000 annotated mentions linked to concepts of 21 selected semantic types in UMLS \cite{bodenreider2004unified}.
We use UMLS as the training KB and we select three representative partial KBs which are concepts from semantic types \textit{T038 (Biologic Function)} and \textit{T058 (Health Care Activity)} in UMLS and \textit{SNOMED}.
% To meet the assumption that the partial KBs do not contain concepts out of training KB, we ditch the concepts in partial KBs that do not exist in UMLS. 


% \subsection{Metrics}\label{sec:metrics}

