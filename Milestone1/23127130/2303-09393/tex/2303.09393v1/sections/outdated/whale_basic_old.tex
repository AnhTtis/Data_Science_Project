NFT whales have compared to the other whales in markets such as stock. In this section, we characterize NFT whales by scrutinizing trading patterns of NFT collections and how they manage assets in terms of holding period and trading price. 

% In this section, we investigate the overall features of NFT whales that are distinct from whales of other markets(e.g., stock market). We then scrutinize their holdings in terms of collections and price. Finally, we analyze the changes in whale group's composition.

% \subsection{Understanding NFT Whale}
\subsection{Whales' NFT Portfolio Management}
%콜렉션마다, 콜렉션 내의 토큰마다 완전 다른 가격 분포
% We take a close look at NFT whales in terms of the number of NFTs they have. First, we examine characteristics of 1) whale group and 2) whale individual in separate.
We describe how NFT whales manage their NFT portfolio in terms of token types, the number of holdings, collection preferences, and price distribution.

% \noindent{\bf Whale Group.}
\noindent{\bf The number of collections in portfolio.}
We investigate how many NFTs and how much NFT holding value whales have in the market.
As mentioned in Section 3, there are two types of tokens, ERC-1155 and ERC-721. Due to different interest of whale group on each token type, the quantity and holding value vary over time. 
In case of ERC-1155 tokens, until June 2021, the number of tokens held by whales is close to 0\% of the total tokens, which shows that whales were not interested in ERC-1155 tokens. However, from July, the quantity and holding value of their ERC-1155 tokens began to increase and reached 9.4\% and 24.3\% in August. This is likely to be due to the launch of \textit{Punks Comic} collection, and this interest continues to the early 2022 with the emergence of other famous ERC-1155 collections. 
% In the case of ERC-1155 tokens, it is possible to exist multiple tokens with the same pair of (\textit{token\_contract}, \textit{token\_id}).
Meanwhile, whales steadily had interest on ERC-721 tokens. The total quantity and average price of ERC-721 tokens are 4.3 times and 8.4 times larger than those of ERC-1155 tokens, respectively (as of February 2022). Therefore, whales group's holding value of ERC-721 tokens takes majority portion of entire holding value. To sum these two types of tokens all together, while whale group only possesses less than 5\% of the total quantity of NFTs, the total holding value accounts for around 20\% of the NFT market cap. This indicates that they have high-value NFTs than the other groups. The percentage of the number and value of tokens held by whale group is summarized in Table \ref{f:whale_distribution} of Appendix. 


\begin{figure}[h]
    \centering
    \includegraphics[width=0.8\linewidth]{figures/whale_dist.pdf}
    \caption{Token Distribution of Whales in February, 2022}
    \label{f:whale_distribution}
\end{figure}

% \noindent{\bf Whale Individual.}
Then we scrutinize each individual in the whale group by looking at the distribution of whales according to tokens they have. In Figure \ref{f:whale_distribution}, more than half of whales have only ERC-721 tokens and 93\% have more ERC-721 tokens than ERC-1155 tokens. Token quantity of whales ranges from 1 to 6,516, and the holding value ranges from \$1.2M to \$31.3M. Even within the whale group, the number and holding value of each whale's NFTs are very different. Interestingly, a whale with the highest holding value (the darkest point in the graph) has only 20 NFTs, which confirms that the number of NFTs they have is not significantly related to their NFT holding value. \\

\noindent{\bf Findings and Insight.}
Among two types of NFTs, almost all of whales prefer ERC-721 tokens. In NFT market, having a large quantity of NFTs does not lead to large holding value. This implies that whales mainly focus on possessing distinctive and valuable NFTs.    


\begin{figure}[ht!]
  \centering
  \includegraphics[width=0.8\linewidth]{figures/holding_distribution.pdf}
  \caption{Price distribution of holding tokens among different groups. For ERC-1155 tokens, there does not exist any token in range R6.}
  \label{f:price_dist}
\end{figure}


% \subsection{Whales' Holdings}
\subsection{Whales' Portfolio Holdings}
Participants in NFT market pursue financial gains. Thus, it is necessary to examine holdings in portfolio. In this section, we examine whale's NFT holdings in portfolio, preferences, and dominance in NFT collections.

% In this section, we enquire into the NFTs that are in hands of whales. We examine whether whales indeed monopolize the most of high-price tokens. 

\noindent \textbf{Top collection holdings.} We begin by searching for top 10 collections that whales mostly hold. Majority of the collections in top 10 are popular collections with large trading volume. Whales' most holding collection is \textit{Art Blocks} where 9.6K tokens are held by whales. Interestingly, whales hold 1.6K \textit{CryptoPunks} tokens which accounts for 16\% of total tokens in this collection. Whales' holding token numbers for remaining collections can be found on Table \ref{t:whale_collections} of Appendix.

\begin{figure*}[t!]
  \centering
  \includegraphics[width=0.9\linewidth]{figures/sankey.pdf}
  \caption{Top : Changes in the group of traders in \textit{whale} at least once from January 2021 (t = 1) to August 2021 (t = 8), Bottom : Changes in the group of traders in \textit{whale} at least once until February 2022 (t = 14). (Dark-blue, sky-blue, olive, and grey in the graph each represents whales, dolphins, minnows, and traders who never traded before, and they are noted as from A to D. Transitions from one color to another depict a quantity of traders with group changes.) } 
  \label{f:sankey}
\end{figure*}


\noindent\textbf{Price distribution.} To look into price of each group's holdings, we divide the token price into a number of ranges, from \$0 to \$23.3M. Due to this large range, we divide each range with log scale. Figure \ref{f:price_dist} is the price distribution of tokens that each group holds on February 28, 2022. Figure \ref{f:price_dist}(a) and (b) each shows distribution of ERC-721 and ERC-1155 tokens on all 91 collections. On the other hand, \ref{f:price_dist}(c) shows token price within a collection, \textit{Cryptopunks}. Note that the price distribution of popular collections (e.g., \textit{BAYC, Art Blocks}) also resembles the distribution of \textit{Cryptopunks}. Almost all of the tokens with price over \$10M are in hands of whales. In addition, whales are the most holders tokens in range \$1M to \$10M across all collections and within a collection. All three of the graphs in Figure \ref{f:price_dist} reveal whales are the only holders of the high-price tokens.\\



\noindent\textbf{Findings and Insight.} 
Generally speaking, price of a good in a market is an indicator of the `value'. The result of this section suggests that whales gain dominance over almost all of highly valuable tokens.




\subsection{Changes in Whales' Composition}

NFT market changed dramatically during 14 months with the influx of new traders and the advent of various types of new collections. In this rapidly changing market, how do traders become the whale group?
%In this highly volatile NFT market, has there been any change in whale composition?
To answer this question, we study how traders move between groups during 14 months, focused on whales. 
% We understand characteristics of whales according to the group they belonged to before, in terms of how they become whales.
%nft market은 새로운 nft도 많이 등장하고, 굉장히 급변하고 있는 시기인데, 그럼에도 불구하고 whale 대부분 남아있음
%whale 사이즈가 커지면서 돌핀이 남은 자리를 차지하는 건 당연.
%minnow에서 오는 비율 굉장히 적음.
%whale에서 non-whale되는 trader들의 절반 이상의 토큰이 send를 통해 new traders한테 감. -> 원래 whale과 specific relationship
%from minnows --> 대부분 recv 거의 없고, buy 혹은 mint --> 다른 유저와의 관계 x

Figure \ref{f:sankey} shows the changes in composition of whale group until February 2022. To better visualize the changes in the first half of the period, the top graph in Figure \ref{f:sankey} only covers traders who were in whale group at least once until August 2021. This is because the number of traders in February 2022 is around 70 times higher than that of January 2021.
The range of \textit{t} is from 1 to 14, indicating each month from January 2021 to February 2022.


In the beginning, since the size of whale group is small, the composition of the group is highly volatile as the market size increases. However, from June 2021 (t = 6), more than 80\% of whales remain in the next whale group, except for August 2021 (t = 8). From July to August, there is the biggest change in the composition of whale group. This is due to the fact that August 2021 is when the number of new traders appeared the most compared to the previous months, with the surge of public interest in NFT [?]. 
To understand how traders become whale group and how whales keep their status, we investigate the characteristics according to whale's origin as follows.

\noindent{\textbf{Whales from \textit{whale}.}}
They maintain their status through various ways: buy, receive, and mint. In the first half, more than half of whales use buy as their major way to increase holding value. However, in early 2022, whales who mainly receive tokens grow in number. Meanwhile, some actively mint NFTs. For example, a whale minted tremendous number of tokens in several collections and sold them to dolphins and minnows. There also exist traders who have not been active for more than a few months. Their extremely expensive tokens let them stay in whale group. \\
% These whales can be considered as \textit{whale} in specific collection, as well as in NFT ecosystem.\\  
% In this regards, information is important in \textit{mint} because it is often done at a specific time or may require certain conditions. As time goes by, most of the traders who participate in mint are belong to whale, which can be interpreted as sharing information among themselves. Therefore, it might be difficult to obtain information for mint if traders are not in whale. We will discuss about whale's mint in section 5.3.
\noindent{\textbf{Whales from \textit{dolphin}.}}
As the size of the whale group increases, it is natural for dolphins to become whales. In August, when the market size grow most rapidly, dolphins take up 39\% of new whale group. 
They raise their holding value in various ways like whales from \textit{whale}. However, in early 2022, new whales from dolphin reduce to 15\%.

% A special case is one from dolphin in August who minted about 1,000 \textit{Punks Comics} tokens, which is Pixel Vault account, the company behind the collection.

\noindent{\textbf{Whales from \textit{minnow}.}}
We rarely observe whales who formerly belonged to \textit{minnow}, usually none. 
They overall increase their NFT holding value through buy or mint. They barely receive tokens from other traders, which indicates that they do not have traders in a particular relationship. This is completely different to whales from traders who have never traded before described below.   


\begin{figure*}[ht]
  \centering
  \includegraphics[width=0.9\linewidth]{figures/long_tx_cnt.pdf}
  \caption{The average number of transactions involving \textit{Whale/Dolphin/Minnow} classes over time }
  \label{f:long_tx}
\end{figure*}

\noindent{\textbf{Whales from traders who have never traded before.}}
There are a lot of traders who became whales as soon as they participate in the NFT market. When the size of the market increases rapidly, whales from this group sometimes outnumber whales from dolphin group. They usually become whales through buy or receive, but many of them become whales only through receive.
Usually, the tokens they receive are NFTs of popular collections with high trading volume (e.g. \textit{CrytoPunks} and \textit{Art Blocks}). Interestingly, most of tokens they received come from traders who were whales at $t-1$ but are not at $t$.
This implies that they have a close tie with the former whales.\\ 
%9월 민트 많이해서 올라온 앧르 있음
\noindent{\bf Findings and Insight.}
Although the NFT ecosystem is rapidly growing, whales keep their place firmly. In addition, newly emerging whales are often associated with former whales. This clearly shows that it is hard for minnows to be whales, and thus to be a whale, one should strive to act out of the group.

% Overall the high ranks of traders who buy or receive a lot rarely overlap. When we observe traders who receive NFTs a lot in each group, almost all of them have their own a few main senders, which is completely contrast to buy. This implies that some traders have a close relationship through transfer, especially whales from D. We will discuss about relationship between whales and other traders in section 5.2. 
% We find out that it is very difficult for a trader to have a high holding value, other than buying  high value tokens or receiving tokens from previous whales in certain relationships. 
% , by investigating how they became whales depending on where they came from.

