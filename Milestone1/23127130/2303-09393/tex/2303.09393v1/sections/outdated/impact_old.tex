To examine how influential whales are in NFT ecosystem, we analyze whales' trading behavior in terms of their impact on the market sentiment and traders.

\subsection{Whales' Impact on NFT Market}
In this section, we analyze whales' transaction patterns in comparison with other groups. We study their effect on NFT ecosystem along with the evolution of NFT market.
%전반적으로 whale, dolphin, minnow 순으로 buy, recv, mint 많이  / sell, send 적게
%시장의 크기는 12월 점점 커지는데, 유저수에 비해 거래량은 많지 않음.
% Since whale's behavior greatly affects the flow of the market, it is essential to study how they trade across the evolution of NFT ecosystem. 
Figure \ref{f:long_tx} shows the monthly number of each transaction type of each group on average. We consider \textit{token\_count} in transaction data as one transaction. As mentioned in section 3.2., transaction types include \textit{buy, sell, receive, send, mint}, and \textit{burn}. 

Overall the average number of transactions by whales is overwhelmingly large. 
Note that y axis of each graph in Figure \ref{f:long_tx} has different scale. 
% Their activity seems to be decrease a lot in the second half, however, in the case of whale, transactions between traders have decreased and transfers through contract addresses have increased (Figure ? in the Appendix). 
Particularly noteworthy is the fact that trading activities of dolphins and minnows tend to be similar, while whales show much more fluctuations. To understand whales' behavior, we study how they respond to real-world events related to NFT. 

\noindent{\bf Sale (Buy/Sell).}
Whale group's buying pattern is similar to other groups, but their selling pattern is different. 
%돌고래, 민노우는 비슷한 패턴
%고래는 buy는 비슷, sell은 달라
This indicates that they buy tokens according to the market trend, but they only sell them with their own standards. 
Their behavior pattern over time is explained with several events (E1$\sim$E3) that whales had a huge impact on the market.
First, we observe a rapid but short-lived peak in February 2021 in all groups. With the growing popularity of \textit{Hashmasks}\cite{hashmasks}, traders buy a huge number of tokens of the collection, which accounts for more than half of their purchases.

\textbf{(E1)} However, in April, NFT market conditions cooled abruptly. Traders in all groups barely participated in transactions, because of the sharp decline (nearly 70\%) in the average price of NFTs \cite{NFT_burst}. In fact, before the market crash, whales' activities drop to the lowest position first, in March. This implies that decrease in whale's activities have had a big impact on the NFT market sentiment.
% price and flow of the NFT ecosystem.
\textbf{(E2)} In May, there exists an uplift in number of sells only in the whales' transaction graph. They sell tokens they minted on this month. The interesting fact is that 95\% of these tokens are from two new collections, \textit{Meebits} and \textit{Bored Ape Yacht Club (BAYC)}. 
\textit{Meebits} received the spotlight even before the launch, since it is made by the creators behind \textit{CryptoPunks}. 
On the other hand, \textit{BAYC} suddenly gained popularity due to a large quantity of whales' mint \cite{bayc_boom}. This promoted the sales of BAYC tokens, especially in dolphin and minnow group. In fact, \textit{BAYC} accounts for 40\% and 54\% of dolphins' and minnows' sales, respectively. 
% The interesting fact is that one of whales, who minted \textit{BAYC} the most (around 1K) and even advertised his minting on Twitter, sold almost all of the tokens to dolphins and minnows by June.
\textbf{(E3)} Overall sales volume ramps up steadily by July 2021, and increases dramatically in August. This is related to a surge in price of \textit{CryptoPunks} by whales. One day in August, a whale bought over 100 \textit{CryptoPunks} NFTs worth more than \$6M in total. Not long after, some of other whales bought tokens of this collection for \$3.7M and \$5.4M. Since then, whales' purchase of the collection continued. Whales' activities led to public interest in NFT market and new traders drastically swelled \cite{CryptoPunk_resurge}.
% an increased trading volume due to growing interest in \textit{Art Blocks} according to Google Trend.

In December, as famous fashion brands (e.g., such as Adidas and Nike) formed NFT partnerships or acquired NFT startups\cite{fashion_nft}, NFT gained greater attention from the public, which made a small increase in three groups. 

To sum up, it turns out three events (E1 $\sim$ E3) out of five were caused by whales, which verifies a great impact of whales in the NFT market.

\noindent{\bf Transfer (Receive/Send).}
%receive를 거의 다, send는 거의 앙ㄴ 함
In the beginning, whales barely do transfer, but from August, the number of transfer becomes greater than or equal to the number of sale. However, for the entire period of our data, among transfer types, whale rarely send NFTs but only receive them. 
%From September, most of the received tokens are related to game, which is consistent with the surge in popularity of decentralized NFT games at that time. 
This is in stark contrast to the fact that dolphins and minnows participate in sale much more than transfer. This represents dolphins increase their holding value through buying tokens, while whales increase by receive them during the second half. 
Interestingly, when we observe whales who receive NFTs a lot, they have a few main senders, and they get hundreds of tokens in a specific collection from each sender.
% When we observe traders who receive NFTs a lot, almost all of them have their own a few main senders, which is completely contrast to buy. This implies that some traders have a close relationship through transfer. 

\noindent{\bf Mint and Burn.}
% Unlike \textit{Meebits}, \textit{Bored Ape Yacht Club} did not gain much attention at the beginning of their launch. However, as soon as several whales minted a large number of \textit{BAYC} tokens, the rest of the tokens were all minted within a few hours\cite{bayc_boom}. This event also verifies the influence of whales on the market. 
We observe all three groups actively participate in mint, especially whale group. By September, mint is the most common transaction type of whales, meaning that they increase their holding value not only through buy, but also through mint.
There are three conspicuous points in their mint. 
In January 2021, they mint about four times more than any other transaction types, and most of them are \textit{Hashmasks}. It surged again in May, due to \textit{Meebits} and \textit{BAYC} as mentioned above. 
In later months including August, some collections(e.g.,\textit{Punks Comic} and \textit{MetaHero Universe: Planet Tokens}) embed governance rights on tokens(a.k.a. DAO tokens), that rose the demand for mint. For example, traders received \textit{Punks Comic} DAO tokens by burning their previous tokens of the collection.
This suggests that whales are interested in exercising governance rights on a collection. 
Noteworthy is whales only mint a huge amount of specific collections, while other groups mint various collections. Besides, we find and explain several unique characteristics of whales' mint in Section 5.3.

NFT market capitalization is increasing due to rise in the number of collections and traders. However, traders' practical activities rather decreased compared to early months. In fact, the number of active minnows is only 20\% of minnow group as of February 2022 (in Table \ref{t:active_users} of Appendix). This implies that NFT market is mainly operated by whale and dolphin group. In addition, we cannot observe any noticeable increase in whales' transactions after August, which means that no more influential collections emerged (e.g., \textit{CryptoPunks}). Despite the large number of collections' launch, a few major collections account for the majority of NFT market capitalization \cite{?}. Hence, holding NFTs of major collections can let whales keep their place.\\
%8월 이후에는 눈에 띄게 큰 변화가 없다. 그 뜻은 cryptopunks, bayc처럼 큰 콜렉션이 등장하지 않았다는 뜻. > 실제로 봤더니 엄청나게 많은 콜렉션이 등장했음에도 불구하고, 전체 시장 가치의 ?퍼센트를 얘네가 차지하더라. 결국 원래 고래였던 애들이 그 콜렉션을 계속 들고 있으면 계속 고래. 

\noindent{\bf Findings and Insight.}
NFT market is predominantly being driven by whales. They take various behaviors in the right place, resulting in their trading pattern being very distinct from other groups. Moreover, they have a huge impact on the market and often alter market sentiment.

% Specifically, the main focus of whales is {\it mint}, {\it } and whales commonly cherry pick high profile NFT collections and participate in the minting process of those collections.  

\begin{figure}[h]
     \centering
     \begin{subfigure}[b]{0.4\textwidth}
         \centering
         \includegraphics[width=\textwidth]{figures/sale_w.pdf}
         \caption{Sale graph}
         \label{f:sale_graph}
     \end{subfigure}
    %  \hfill
     \begin{subfigure}[b]{0.4\textwidth}
         \centering         \includegraphics[width=\textwidth]{figures/trans_w.pdf}
         \caption{Transfer graph}
         \label{f:trans_graph}
     \end{subfigure}
        \caption{Trading network including whales from January 2022 to February 2022. Red, purple and green each represents whale, dolphin, and minnow. The size of the node is proportional to its PageRank.}
        \label{f:transaction_graph}
\end{figure}


% size가 작은 그룹일수록 holding leader가 많음, transfer 비율이 높음 
\subsection{Whales' Importance in Trading Network}
In this section, we explore the influence of whales on other traders by looking at the \textit{trading network}. \textit{Trading network} is a network formed by transactions between traders. Each node represents a trader, and edge represents transaction (sale or transfer) between them. The direction of the edge represents the transaction flow from sender to receiver.
% Dark-blue, sky-blue and olive each represent whale, dolphin, and minnow and the node size is proportional to its pagerank.
Since the structure of trading network gradually becomes apparent over time, Figure \ref{f:transaction_graph} only shows transactions from January 2022 to February 2022. Each graph shows trading network formed either by sale or transfer. To understand relationships between whale and the others, we only use transactions including whales for analysis. Whales' importance in the trading network is represented by PageRank \cite{brin1998anatomy}, generally used in web network graphs. 

The major finding is that whales show different relationships with traders in sale and transfer.
% 
That is, (1) The transfer network shows many noticeably large whales with high PageRank, unlike the sale network. This indicates that whales play a more important role in the transfer netwoTo examine how influential whales are in NFT ecosystem, we analyze whales' trading behavior in terms of their impact on the market sentiment and traders.

\subsection{Whales' Impact on NFT Market}
In this section, we analyze whales' transaction patterns in comparison with other groups. We study their effect on NFT ecosystem along with the evolution of NFT market.
%전반적으로 whale, dolphin, minnow 순으로 buy, recv, mint 많이  / sell, send 적게
%시장의 크기는 12월 점점 커지는데, 유저수에 비해 거래량은 많지 않음.
% Since whale's behavior greatly affects the flow of the market, it is essential to study how they trade across the evolution of NFT ecosystem. 
Figure \ref{f:long_tx} shows the monthly number of each transaction type of each group on average. We consider \textit{token\_count} in transaction data as one transaction. As mentioned in section 3.2., transaction types include \textit{buy, sell, receive, send, mint}, and \textit{burn}. 

Overall the average number of transactions by whales is overwhelmingly large. 
Note that y axis of each graph in Figure \ref{f:long_tx} has different scale. 
% Their activity seems to be decrease a lot in the second half, however, in the case of whale, transactions between traders have decreased and transfers through contract addresses have increased (Figure ? in the Appendix). 
Particularly noteworthy is the fact that trading activities of dolphins and minnows tend to be similar, while whales show much more fluctuations. To understand whales' behavior, we study how they respond to real-world events related to NFT. 

\noindent{\bf Sale (Buy/Sell).}
Whale group's buying pattern is similar to other groups, but their selling pattern is different. 
%돌고래, 민노우는 비슷한 패턴
%고래는 buy는 비슷, sell은 달라
This indicates that they buy tokens according to the market trend, but they only sell them with their own standards. 
Their behavior pattern over time is explained with several events (E1$\sim$E3) that whales had a huge impact on the market.
First, we observe a rapid but short-lived peak in February 2021 in all groups. With the growing popularity of \textit{Hashmasks}\cite{hashmasks}, traders buy a huge number of tokens of the collection, which accounts for more than half of their purchases.

\textbf{(E1)} However, in April, NFT market conditions cooled abruptly. Traders in all groups barely participated in transactions, because of the sharp decline (nearly 70\%) in the average price of NFTs \cite{NFT_burst}. In fact, before the market crash, whales' activities drop to the lowest position first, in March. This implies that decrease in whale's activities have had a big impact on the NFT market sentiment.
% price and flow of the NFT ecosystem.
\textbf{(E2)} In May, there exists an uplift in number of sells only in the whales' transaction graph. They sell tokens they minted on this month. The interesting fact is that 95\% of these tokens are from two new collections, \textit{Meebits} and \textit{Bored Ape Yacht Club (BAYC)}. 
\textit{Meebits} received the spotlight even before the launch, since it is made by the creators behind \textit{CryptoPunks}. 
On the other hand, \textit{BAYC} suddenly gained popularity due to a large quantity of whales' mint \cite{bayc_boom}. This promoted the sales of BAYC tokens, especially in dolphin and minnow group. In fact, \textit{BAYC} accounts for 40\% and 54\% of dolphins' and minnows' sales, respectively. 
% The interesting fact is that one of whales, who minted \textit{BAYC} the most (around 1K) and even advertised his minting on Twitter, sold almost all of the tokens to dolphins and minnows by June.
\textbf{(E3)} Overall sales volume ramps up steadily by July 2021, and increases dramatically in August. This is related to a surge in price of \textit{CryptoPunks} by whales. One day in August, a whale bought over 100 \textit{CryptoPunks} NFTs worth more than \$6M in total. Not long after, some of other whales bought tokens of this collection for \$3.7M and \$5.4M. Since then, whales' purchase of the collection continued. Whales' activities led to public interest in NFT market and new traders drastically swelled \cite{CryptoPunk_resurge}.
% an increased trading volume due to growing interest in \textit{Art Blocks} according to Google Trend.

In December, as famous fashion brands (e.g., such as Adidas and Nike) formed NFT partnerships or acquired NFT startups\cite{fashion_nft}, NFT gained greater attention from the public, which made a small increase in three groups. 

To sum up, it turns out three events (E1 $\sim$ E3) out of five were caused by whales, which verifies a great impact of whales in the NFT market.

\noindent{\bf Transfer (Receive/Send).}
%receive를 거의 다, send는 거의 앙ㄴ 함
In the beginning, whales barely do transfer, but from August, the number of transfer becomes greater than or equal to the number of sale. However, for the entire period of our data, among transfer types, whale rarely send NFTs but only receive them. 
%From September, most of the received tokens are related to game, which is consistent with the surge in popularity of decentralized NFT games at that time. 
This is in stark contrast to the fact that dolphins and minnows participate in sale much more than transfer. This represents dolphins increase their holding value through buying tokens, while whales increase by receive them during the second half. 
Interestingly, when we observe whales who receive NFTs a lot, they have a few main senders, and they get hundreds of tokens in a specific collection from each sender.
% When we observe traders who receive NFTs a lot, almost all of them have their own a few main senders, which is completely contrast to buy. This implies that some traders have a close relationship through transfer. 

\noindent{\bf Mint and Burn.}
% Unlike \textit{Meebits}, \textit{Bored Ape Yacht Club} did not gain much attention at the beginning of their launch. However, as soon as several whales minted a large number of \textit{BAYC} tokens, the rest of the tokens were all minted within a few hours\cite{bayc_boom}. This event also verifies the influence of whales on the market. 
We observe all three groups actively participate in mint, especially whale group. By September, mint is the most common transaction type of whales, meaning that they increase their holding value not only through buy, but also through mint.
There are three conspicuous points in their mint. 
In January 2021, they mint about four times more than any other transaction types, and most of them are \textit{Hashmasks}. It surged again in May, due to \textit{Meebits} and \textit{BAYC} as mentioned above. 
In later months including August, some collections(e.g.,\textit{Punks Comic} and \textit{MetaHero Universe: Planet Tokens}) embed governance rights on tokens(a.k.a. DAO tokens), that rose the demand for mint. For example, traders received \textit{Punks Comic} DAO tokens by burning their previous tokens of the collection.
This suggests that whales are interested in exercising governance rights on a collection. 
Noteworthy is whales only mint a huge amount of specific collections, while other groups mint various collections. Besides, we find and explain several unique characteristics of whales' mint in Section 5.3.

NFT market capitalization is increasing due to rise in the number of collections and traders. However, traders' practical activities rather decreased compared to early months. In fact, the number of active minnows is only 20\% of minnow group as of February 2022 (in Table \ref{t:active_users} of Appendix). This implies that NFT market is mainly operated by whale and dolphin group. In addition, we cannot observe any noticeable increase in whales' transactions after August, which means that no more influential collections emerged (e.g., \textit{CryptoPunks}). Despite the large number of collections' launch, a few major collections account for the majority of NFT market capitalization \cite{?}. Hence, holding NFTs of major collections can let whales keep their place.\\
%8월 이후에는 눈에 띄게 큰 변화가 없다. 그 뜻은 cryptopunks, bayc처럼 큰 콜렉션이 등장하지 않았다는 뜻. > 실제로 봤더니 엄청나게 많은 콜렉션이 등장했음에도 불구하고, 전체 시장 가치의 ?퍼센트를 얘네가 차지하더라. 결국 원래 고래였던 애들이 그 콜렉션을 계속 들고 있으면 계속 고래. 

\noindent{\bf Findings and Insight.}
NFT market is predominantly being driven by whales. They take various behaviors in the right place, resulting in their trading pattern being very distinct from other groups. Moreover, they have a huge impact on the market and often alter market sentiment.

% Specifically, the main focus of whales is {\it mint}, {\it } and whales commonly cherry pick high profile NFT collections and participate in the minting process of those collections.  

\begin{figure}[h]
     \centering
     \begin{subfigure}[b]{0.4\textwidth}
         \centering
         \includegraphics[width=\textwidth]{figures/sale_w.pdf}
         \caption{Sale graph}
         \label{f:sale_graph}
     \end{subfigure}
    %  \hfill
     \begin{subfigure}[b]{0.4\textwidth}
         \centering         \includegraphics[width=\textwidth]{figures/trans_w.pdf}
         \caption{Transfer graph}
         \label{f:trans_graph}
     \end{subfigure}
        \caption{Trading network including whales from January 2022 to February 2022. Red, purple and green each represents whale, dolphin, and minnow. The size of the node is proportional to its PageRank.}
        \label{f:transaction_graph}
\end{figure}


% size가 작은 그룹일수록 holding leader가 많음, transfer 비율이 높음 
\subsection{Whales' Importance in Trading Network}
In this section, we explore the influence of whales on other traders by looking at the \textit{trading network}. \textit{Trading network} is a network formed by transactions between traders. Each node represents a trader, and edge represents transaction (sale or transfer) between them. The direction of the edge represents the transaction flow from sender to receiver.
% Dark-blue, sky-blue and olive each represent whale, dolphin, and minnow and the node size is proportional to its pagerank.
Since the structure of trading network gradually becomes apparent over time, Figure \ref{f:transaction_graph} only shows transactions from January 2022 to February 2022. Each graph shows trading network formed either by sale or transfer. To understand relationships between whale and the others, we only use transactions including whales for analysis. Whales' importance in the trading network is represented by PageRank \cite{brin1998anatomy}, generally used in web network graphs. 

The major finding is that whales show different relationships with traders in sale and transfer.
% 
That is, (1) The transfer network shows many noticeably large whales with high PageRank, unlike the sale network. This indicates that whales play a more important role in the transfer network than the sale network. 
% Also, whales who buy tokens a lot, also sell a lot, however, they do not send them again.
% (1) In Figure \ref{f:sale_graph}, we observe whales of various sizes exist, and large dolphins and minnows are also seen. However, in Figure \ref{f:trans_graph}, a few extremely large whales and large size of non-whale traders barely exist. This means that the effect of a small number of whales on other trader is greater in transfer graph than in sale graph.
(2) Almost all whales transfer only with dolphins and minnows surrounding them, but they sale with whales (or the others) regardless of cluster.
% This is due to the observation of a lot of distinct clusters of dolphins and minnows surrounding a whale in Figure \ref{f:trans_graph}, which indicates that transactions occur mainly within each cluster. This is different with whale's \textit{sale}, such clusters appear unclear. 
(3) In the transfer graph, there are a lot of small independent components (WCCs) that are not connected to the biggest component. This phenomenon is hard to see in sale graph.

Since transfer is a transaction without actual payment, we can assume that traders are likely to be in a specific relationship. In other words, whales appear to form a \textit{whale-centered clusters} through transfer.
% (The number of weakly connected components in the transfer network is around three times that in the sale network). 
\\ 
%  This suggests that communities become more solid over time. In section 5.1., in the second half, we find the majority of whales' transactions are \textit{transfer}, which indicates that they mainly trade within the community rather than with external traders.\\
% Previous works \cite{casale2021networks}\cite{nadini2021mapping} show that traders are clustered around NFT collections. However, we discover it is only true when traders are in transfer relationship, not in sale relationship.
% 0.0029 max
% Diameter: 11,  Avg Path length: 3.0963156367856106
% Diameter: 24, Avg Path length: 8.026665147315253


\noindent{\bf Findings and Insight.}
Whales form close relationships with small number of traders through \textit{transfer-major communities}, focusing on specific collections. Some of them participate in wash trading, raising NFT prices.


%t = 10 모든 멤버가 holding leader로 이루어진 그룹이 있는데, 8명이 CyberKongz를 125번 transfer
%t = 13 4명이 800번, 6명이 470번 meebits transfer --> meebits 가격 증가 
rk than the sale network. 
% Also, whales who buy tokens a lot, also sell a lot, however, they do not send them again.
% (1) In Figure \ref{f:sale_graph}, we observe whales of various sizes exist, and large dolphins and minnows are also seen. However, in Figure \ref{f:trans_graph}, a few extremely large whales and large size of non-whale traders barely exist. This means that the effect of a small number of whales on other trader is greater in transfer graph than in sale graph.
(2) Almost all whales transfer only with dolphins and minnows surrounding them, but they sale with whales (or the others) regardless of cluster.
% This is due to the observation of a lot of distinct clusters of dolphins and minnows surrounding a whale in Figure \ref{f:trans_graph}, which indicates that transactions occur mainly within each cluster. This is different with whale's \textit{sale}, such clusters appear unclear. 
(3) In the transfer graph, there are a lot of small independent components (WCCs) that are not connected to the biggest component. This phenomenon is hard to see in sale graph.

Since transfer is a transaction without actual payment, we can assume that traders are likely to be in a specific relationship. In other words, whales appear to form a \textit{whale-centered clusters} through transfer.
% (The number of weakly connected components in the transfer network is around three times that in the sale network). 
\\ 
%  This suggests that communities become more solid over time. In section 5.1., in the second half, we find the majority of whales' transactions are \textit{transfer}, which indicates that they mainly trade within the community rather than with external traders.\\
% Previous works \cite{casale2021networks}\cite{nadini2021mapping} show that traders are clustered around NFT collections. However, we discover it is only true when traders are in transfer relationship, not in sale relationship.
% 0.0029 max
% Diameter: 11,  Avg Path length: 3.0963156367856106
% Diameter: 24, Avg Path length: 8.026665147315253


\noindent{\bf Findings and Insight.}
Whales form close relationships with small number of traders through \textit{transfer-major communities}, focusing on specific collections. Some of them participate in wash trading, raising NFT prices.


%t = 10 모든 멤버가 holding leader로 이루어진 그룹이 있는데, 8명이 CyberKongz를 125번 transfer
%t = 13 4명이 800번, 6명이 470번 meebits transfer --> meebits 가격 증가 
