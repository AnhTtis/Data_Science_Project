% In this section, we endeavor to answer our final research question regarding profit of whales. 
% We start off by investing how much profit whales compared to other groups in NFT ecosystem.

% For ERC-1155 token, whenever each trader makes transaction, we update the trader’s average unit price and quantity, then calculate profit of the token. For minted tokens before January 1, 2021, we refer the last sale/mint price before 2021. If it is not possible, we exclude the profit of the first sale of the token. 

In this section, we evaluate the investment performance of each group and discuss how whales achieve high returns compared to other market participants. 


% \subsection{Overall Profit of Each Group}
\subsection{Investment Performance}
Figure \ref{f:profit_basic} illustrates the investment performance of each trading group. We divide the profit range with a log scale due to the wide range of profit among traders. Note that we do not include traders who have never participated in profit activity (i.e., zero profit). The trader who gained the maximum profit is a whale (\$18.9M) and the trader with the largest loss is a dolphin (-\$2.7M). The most noticeable observation is that 35.9\% of traders who belong to whales make profits larger than \$1M while there hardly exists any from dolphins and minnows. Moreover, while 79.8\% of whales produced profits greater than \$100K, the percentages are much lower in other groups. This suggests that a large fraction of whales produces a considerably higher profit compared to any other group.\\
\noindent\textbf{Findings and Insight.}
We identify that whales have achieved significant financial gain. A large fraction of whales achieved profit over \$1M which shows great contrast with other groups.

% \subsection{Source of Whales' Profit}
\subsection{Sources of Financial Gains from Whales}
Finally, we scrutinize the source of profit from whales. We divide the profit source into two parts: buy profit and mint profit. 

% \begin{itemize}
%     \item \textbf{profit from buy} - profit from ordinary sales (except the very first sale of a token)
%     \item \textbf{profit from mint} - profit of the first sale (occurs only once per token)
%\qquad(i.e., profit of first sell, occurs only once per token)
% \end{itemize}

% Although Figure \ref{f:profit_basic} illustrates profits of all types of tokens, for this section, we only analyze ERC-721 tokens due to difference in mint; there exists only one mint for an ERC-721 token but number of mints for an ERC-1155 token. Still, the result can be generalized since number of ERC-721 tokens account for 99.41\% of whole tokens in our data. 
%mint 수익 vs buy 수익
%mint 수익 고래수 vs buy 수익 고래수
%전체 profit에서 고래가 얼만ㄴ큼 참여?

% \begin{figure}[t!]
%   \centering
%   \includegraphics[width=0.9\linewidth]{figures/profit_type.pdf}
% %   \caption{Average Profit of each profit type}
%   \caption{Trends of financial gains in each profit type}
%   \label{f:profit_type}
% \end{figure}

\noindent\textbf{Buy profit.} This type of profit occurs from ordinary sales, except the very first sale of a token. For profit generated from buy, we focus on the collections that whales take advantage of. Whales gained profit from 74 collections out of 91 in total. Among them, we sort the top collections in which the whales were most profitable. If a collection does not generate any profit for several months (e.g., launching in the middle of the entire period), we only mark zero profit once for graph's visibility. 

Figure \ref{f:col_profit} shows the average profit of whales from the top 7 collections (top 10\% of 74 collections) for each month. Overall, the majority of these collections are well-known for their large trading volume. Among them, profits from \textit{CryptoPunks, Art Blocks} and \textit{BAYC} are consistently large, which indicates that whales continuously gain profit from very popular collections. For the peak of the rest of the collections, the profit ranks the top due to the selling of several expensive tokens by small number of whales. For example, a peak of \textit{ASM AIFA Genesis} on December 2021 was generated by a single whale that sold 56 tokens.
% In the early months until April 2021, the buy profit was only from \textit{CryptoPunks} and \textit{Art blocks}. These collections, which were launched in the early phase of the NFT market, garnered interest from whales from the beginning of our data period. The highest peak is from \textit{CryptoPunks} on August 2021 when the transaction of the traders were most active. 31 whales participated in 299 sales with an average profit of over \$1.4M. The second largest peak is from a very popular collection, \textit{BAYC} with an average profit over \$0.5M. Another peak with similar height is from \textit{CyberKongz} on September 2021, made by 3 whales that sold 11 tokens in total. For most of the collections, the peak is either on August or December 2021, which matches the peaks from Figure \ref{f:long_tx}, where the traders performed active transactions.

\noindent\textbf{Mint profit.}
Whales consistently gained profit from mint profit, which is the profit from the first sale just after mint. Surprisingly, although mint profit occurs only once per token, mint profits are comparable to buy profits and even are larger on some months. This is noteworthy as, 1) the number of mint logs accounts for 28.2\% of total transfer logs in our data (1.1M out of 3.9M) and 2) most collections have a limited number of tokens available for mint (e.g., only 10K tokens for \textit{BAYC}).  

% On the first month, none of the whales sold tokens that they minted on any of the collections. By February 2021, two whales each sold 29 and 2 tokens of \textit{Hashmasks} and obtained \$313.3K and \$19.4K, respectively. 
Until July 2021, whales sell the tokens that they have minted from a limited number of collections. For example, The mint profit of whales consistently occur from \textit{Hashmasks} in the first half of the year. However, starting from August where the number of traders in the NFT market drastically increased, whales start to gain mint profit from a wide range of collections. This is likely to be due to the launch of many collections after August. Nevertheless, most of the mint profit are obtained from popular collections mentioned above, e.g., \textit{Art Blocks, Meebits, BAYC} until the last month (February 2022).

Meanwhile, some collections utilize \textit{airdrop} as a way to advertise themselves during the launch period; they let traders mint the tokens without paying any mint fees \cite{airdrop}. To amplify the effects of advertising, some collections even choose to send their tokens to popular wallet addresses in a unilateral way. Since whale accounts are easily available in NFT platforms (e.g.,NFTGo \cite{nftgo}), it is likely for whales to receive lots of tokens through airdrop. Therefore, we investigate the profits produced by whales through airdrop. 

Figure \ref{f:airdrop} shows the top 5 collections in which the total profit of whales is the largest. The number of airdrop tokens sold by whales take up 9 to 20\% of airdrop tokens in these collections. This is a large portion considering the small number of whales and implies that whales actively participate in selling airdrop tokens. Interestingly, the five collections in Figure \ref{f:airdrop} are very popular collections with large trading volumes. Overall, among the 41 collections from which the whales obtained profit through airdrop, we find that the tokens sold by whales are generally more expensive than tokens by other groups on 36 collections. This indicates two possibilities: 1) whales already have the public's trust in the NFT market and the traders are willing to pay high prices for the tokens minted by whales or 2) whales receive relatively high-value tokens of a collection via airdrop. Still, in either case, it is an undeniable fact that whales have power in the minting process that can lead to larger profit while no others can. \\
% 몇 명의 고래들이 참여했는데, 이건 매우 작은 portion인데(10%를 에어드랍 민트). 이건 고래 수에 비해 매우 많이 에어드랍에 참여한 것. 게다가 수익도 좋음
\noindent\textbf{Findings and Insight.} Whales are mainly lucrative through selling NFTs of popular collections. More importantly, they actively participate in minting process, sell NFTs at higher price than any other group. This allows whales to become successful investors. 
\begin{figure}[t!]
  \centering
  \includegraphics[width=0.8\linewidth]{figures/profit_group.pdf}
  \caption{Investment performance per trading group}
  \label{f:profit_basic}
  \vspace{-10pt}
\end{figure}
\begin{figure*}[t!]
  \centering
  \includegraphics[width=0.9\linewidth]{figures/profit_col.pdf}
  \caption{Average Profit of whales per collection}
  \label{f:col_profit}
  \vspace{-10pt}
\end{figure*}
\label{s:profit-mint}
\begin{figure}[h!]
  \centering
  \includegraphics[width=0.9\linewidth]{figures/airdrop_profit.pdf}
  \caption{Average Profit of airdrop tokens (\textit{MAYC} and \textit{CLONE X} are abbreviations for \textit{Mutant Ape Yacht Club} and \textit{RTFKT CLONE X + Murakami}, respectively.)}
  \label{f:airdrop}
  \vspace{-10pt}
\end{figure}