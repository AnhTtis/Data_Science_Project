In this section, we characterize NFT whales through a deep analysis of their behavior patterns such as unique trading methods for NFT items and portfolio management.
%NFT whales have compared to the other whales in markets such as stock. In this section, we characterize NFT whales by scrutinizing trading patterns of NFT collections, asset management patterns in terms of holding period and trading price, relationships with other traders, and asset transfer.

% In this section, we investigate the overall features of NFT whales that are distinct from whales of other markets(e.g., stock market). We then scrutinize their holdings in terms of collections and price. Finally, we analyze the changes in whale group's composition.

% \subsection{Understanding NFT Whale}
% \subsection{Whales' NFT Portfolio Management}
% %콜렉션마다, 콜렉션 내의 토큰마다 완전 다른 가격 분포
% % We take a close look at NFT whales in terms of the number of NFTs they have. First, we examine characteristics of 1) whale group and 2) whale individual in separate.
% We describe how NFT whales manage their NFT portfolio in terms of token value and the number of holdings.

% % \noindent{\bf Whale Group.}
% % \noindent{\bf The token types in portfolio.}
% We investigate how many NFTs and how much NFT holding value whales have in the market.
% % As mentioned in Section 3, there are two types of tokens, ERC-1155 and ERC-721. Due to different interest of whale group on each token type, the quantity and holding value vary over time. 
% % In case of ERC-1155 tokens, until June 2021, the number of tokens held by whales is close to 0\% of the total tokens, which shows that whales were not interested in ERC-1155 tokens. However, from July, the quantity and holding value of their ERC-1155 tokens began to increase and reached 9.4\% and 24.3\% in August. This is likely to be due to the launch of \textit{Punks Comic} collection, and this interest continues to the early 2022 with the emergence of other famous ERC-1155 collections. 
% % In the case of ERC-1155 tokens, it is possible to exist multiple tokens with the same pair of (\textit{token\_contract}, \textit{token\_id}).
% % Meanwhile, whales steadily had interest on ERC-721 tokens. The total quantity and average price of ERC-721 tokens are 4.3 times and 8.4 times larger than those of ERC-1155 tokens, respectively (as of February 2022). Therefore, whales group's holding value of ERC-721 tokens takes majority portion of entire holding value. To sum these two types of tokens all together, while whale group only possesses less than 5\% of the total quantity of NFTs, the total holding value accounts for around 20\% of the NFT market cap. This indicates that they have high-value NFTs than the other groups. The percentage of the number and value of tokens held by whale group is summarized in Table \ref{t:whale_basic} in Appendix. 


% \begin{figure}[t]
%     \centering
%     \includegraphics[width=0.8\linewidth]{figures/whale_dist.pdf}
%     \caption{Token Distribution of Whales in February, 2022}
%     \label{f:whale_distribution}
% \end{figure}

% % \noindent{\bf Whale Individual.}
% Then we scrutinize each individual in the whale group by looking at the distribution of whales according to tokens they have. In Figure \ref{f:whale_distribution}, more than half of whales have only ERC-721 tokens and 93\% have more ERC-721 tokens than ERC-1155 tokens. Token quantity of whales ranges from 1 to 6,516, and the holding value ranges from \$1.2M to \$31.3M. Even within the whale group, the number and holding value of each whale's NFTs are very different. Interestingly, a whale with the highest holding value (the darkest point in the graph) has only 20 NFTs, which confirms that the number of NFTs they have is not significantly related to their NFT holding value. \\
% \noindent{\bf Findings and Insight.}
% Among two types of NFTs, almost all of whales prefer ERC-721 tokens. In NFT market, having a large quantity of NFTs does not lead to large holding value. This implies that whales mainly focus on possessing distinctive and valuable NFTs.    


% \begin{figure}[ht!]
%   \centering
%   \includegraphics[width=0.8\linewidth]{figures/holding_distribution.pdf}
%   \caption{Price distribution of tokens hold by each group. (Note that for ERC-1155 tokens, there does not exist any token in range R6. Dark-blue,
% sky-blue, olive each represents holdings of whales, dolphins, minnows, respectively.}
%   \label{f:price_dist}
% \end{figure}

% \begin{figure}[h]
%      \centering
%      \begin{subfigure}[b]{0.35\textwidth}
%          \centering
%          \includegraphics[width=\textwidth]{figures/sale_w.pdf}
%          \caption{Sale graph}
%          \label{f:sale_graph}
%      \end{subfigure}
%       \hfill
%      \begin{subfigure}[b]{0.35\textwidth}
%          \centering         \includegraphics[width=\textwidth]{figures/trans_w.pdf}
%          \caption{Transfer graph}
%          \label{f:trans_graph}
%      \end{subfigure}
%         \caption{Trading network including whales from January 2022 to February 2022. Red, purple and green each represents whale, dolphin, and minnow. Each node represents a trader, and edge represents transaction (sale or transfer) between them. The direction of the edge represents the transaction flow from sender to receiver. The size of the node is proportional to its PageRank.}
%         \label{f:transaction_graph}
% \end{figure}

% \subsection{Whales' Holdings}


\begin{figure*}[t]
\vspace{0.1cm}
  \centering
  \includegraphics[width=0.8\linewidth]{figures/sankey.pdf}
  \caption{Changes in the group of traders in \textit{whale} from January 2021 (t = 1) to February 2022 (t = 14). The top and bottom only depict traders that have belonged to the \textit{whale} group at least once, until August and February 2022, respectively.
 Transitions from one color to another depict a quantity of traders with group changes. }
  \label{f:sankey}
\end{figure*}



\subsection{Changes in Whales' Composition}

The NFT market has changed dramatically over the past 14 months with the influx of new traders and the advent of various types of new collections. 
In this rapidly changing market, \textit{how do traders become whales}?
To answer this question, we study how traders move across between groups during 14 months, especially focusing on whales. Figure \ref{f:sankey} shows the changes in composition of the whale group until February 2022. 
% For simplicity, we denote each of 14 months as $\textit{t} \in \{1, 2, \cdots 14 \}$.
% Since the number of traders in February 2022 is around 70 times higher than in January 2021, for better visualization, 
% In the figure, the top and bottom only depict traders that have belonged to the \textit{whale} group at least once, until August and February 2022, respectively.

In the beginning, since the size of whale group is small, the composition of the group is highly volatile as the market size increases. However, from June 2021, more than 80\% of whales remain in the next whale group, except in August 2021. August is when the most significant change in the whale group occurs. The surge of public interest at that time~\cite{nft_august} caused the largest number of new traders compared to the previous months. 
To understand how traders become whales and how they maintain their status, we investigate the characteristics according to the whales' origin as follows. The percentage of each group where the whales belonged to is summarized in Table \ref{t:across_users} of the Appendix.


\noindent{\textbf{Whales from \textit{whales}.}}
Whales maintain their status through various ways: buying, receiving, and minting. In the first half, more than half of whales buy NFTs in order to increase the holding value. However, in early 2022, whales who mainly receive tokens grow in number. Meanwhile, some whales actively mint NFTs. For example, one whale minted a tremendous number of tokens in several collections and sold them to dolphins and minnows. There also exist whales who were inactive for more than a few months. Their extremely expensive tokens let them stay in the whale group. \\
% These whales can be considered as \textit{whale} in specific collection, as well as in NFT ecosystem.\\  
% In this regards, information is important in \textit{mint} because it is often done at a specific time or may require certain conditions. As time goes by, most of the traders who participate in mint are belong to whale, which can be interpreted as sharing information among themselves. Therefore, it might be difficult to obtain information for mint if traders are not in whale. We will discuss about whale's mint in section 5.3.
\noindent{\textbf{Whales from \textit{dolphins}.}}
As the size of the whale group increases, it is natural for dolphins to become whales. 
In August, when the market size grows most rapidly, dolphins take up 39\% of the new whale group. 
Dolphins raise their holding value in various ways like whales from \textit{whales}. 
However, in early 2022, the percentage of new whales from dolphins see a decrease to 15\%.

% A special case is one from dolphin in August who minted about 1,000 \textit{Punks Comics} tokens, which is Pixel Vault account, the company behind the collection.

\noindent{\textbf{Whales from \textit{minnows}.}}
We rarely observe whales who formerly belonged to \textit{minnows}.
They increase their NFT holding value through buying or minting. 
They barely receive tokens from other traders, which indicates that they do not form particular relationships with other traders. 
This is completely different to first-time trading whales, which is described in detail below.   


\noindent{\textbf{First-time trading whales.}}
There are a lot of traders who become whales as soon as they begin participating in the NFT market. When the size of the market increases rapidly, whales from this group sometimes outnumber the whales from the dolphin group. They usually become whales through buying or receiving, but many of them become whales only through the act of receiving tokens.
Usually, the tokens they receive are NFTs of popular collections with high trading volume (e.g. \textit{CrytoPunks}). Interestingly, most of the NFTs they received come from the former whales, implying a close relationship between them.\\
% they hold close ties with the former whales.\\ 
\noindent{\bf Findings and Insight.}
Although the NFT ecosystem is rapidly growing, whales maintain their place firmly. In addition, newly emerging whales are often associated with former whales. This clearly shows that it is hard for minnows to become whales.
% , and thus to be a whale, one should strive to act out of the group.

% Overall the high ranks of traders who buy or receive a lot rarely overlap. When we observe traders who receive NFTs a lot in each group, almost all of them have their own a few main senders, which is completely contrast to buy. This implies that some traders have a close relationship through transfer, especially whales from D. We will discuss about relationship between whales and other traders in section 5.2. 
% We find out that it is very difficult for a trader to have a high holding value, other than buying  high value tokens or receiving tokens from previous whales in certain relationships. 
% , by investigating how they became whales depending on where they came from.


\begin{figure}[ht]
  \centering
  \includegraphics[width=0.8\linewidth]{figures/price_dist.pdf}
  \caption{Price distribution of tokens held by each group. Dark-blue,
sky-blue, olive each represents holdings of whales, dolphins, and minnows.}
  \label{f:price_dist}
\end{figure}

\begin{figure*}[ht!]
  \centering
  \includegraphics[width=0.9\linewidth]{figures/long_tx_cnt.pdf}
  \caption{Monthly statistics of each transaction type of each group on average. (stacked plot)}
  \label{f:long_tx}
\end{figure*}






\subsection{Whales' Portfolio Holdings}
Common participants in the NFT market pursue financial gains. 
%Thus, it is necessary to examine holdings in portfolio. 
In this regard, examining their portfolio from various angles is highly valuable for measuring significant financial factors on the NFT ecosystem.
In this section, we examine the whales' NFT portfolio, preferences, and dominance in NFT collections.

% In this section, we enquire into the NFTs that are in hands of whales. We examine whether whales indeed monopolize the most of high-price tokens. 

\noindent \textbf{Top collection holdings.} We begin by searching for top 10 collections that whales mostly hold. Interestingly, we discover that the majority of the collections in the top 10 are popular collections with large trading volume.
The whales' most held collection is \textit{Art Blocks} (9.6K tokens held). Interestingly, whales hold 1.6K \textit{CryptoPunks} tokens which accounts for 16\% of total tokens in this collection. The number of tokens held by Whales in other collections can be found on Table \ref{t:whale_collections} of the Appendix.



\noindent\textbf{Price distribution.} To look into the price of each group's holdings, we divide the token price into six price points, ranging from \$0 to the maximum price in our data, which is \$23.3M.
%Due to this large range, we divide each range with log scale. 
Figure \ref{f:price_dist} describes the price distribution of tokens that each group held on February 28, 2022. Figure \ref{f:price_dist}(a) shows the distribution of tokens over total 91 collections. In addition, to look at distribution within a collection, \ref{f:price_dist}(b) shows the holding range of the most popular NFT project, \textit{CryptoPunks}. Note that the price distribution of popular collections (e.g., \textit{BAYC, Art Blocks}) also resembles the distribution of \textit{CryptoPunks}. Interestingly, almost all of the tokens with prices ranging over \$10M are in hands of whales. Whales also hold most of the tokens in the price range of \$1M to \$10M across all collections. On the whole, Figure \ref{f:price_dist} indicates that whales are the only holders of the high-price tokens, both for overall collections and the top collections.\\
\noindent\textbf{Findings and Insight.} 
Generally speaking, price of goods in a market is an indicator of its `value'. The result obtained in this section suggests that whales gain dominance over almost all highly valuable tokens.



% % size가 작은 그룹일수록 holding leader가 많음, transfer 비율이 높음 
% \subsection{Whales' Trading Network}
% Whales' importance in the trading network is represented by PageRank~\cite{brin1998anatomy}, generally used in web network graphs.
% We explore the influence of whales on other traders base on \textit{trading network}, which is a network formed by transactions between traders. 
% Figure \ref{f:transaction_graph} shows transaction networks of the last two months of our data when the market size was the largest (from January 2022 to February 2022).
% %Since the structure of trading network gradually becomes apparent over time, Figure \ref{f:transaction_graph} only shows transactions from January 2022 to February 2022. 
% Each graph shows trading network formed either by sale or transfer. To understand relationships between whale and the others, we only use transactions including whales for analysis. 

% \begin{figure}[t!]
%   \centering
%   \includegraphics[width=0.9\linewidth]{figures/long_community.pdf}
%   \caption{Communities including whales (stacked plot)}
%   \label{f:long_community}
% \end{figure}



% The major finding is that whales show different relationships with traders in sale and transfer: (i) The transfer network shows many noticeably large whales with a bunch of connections, unlike the sale network. This indicates that whales play a more important role in the transfer network than the sale network. 
% % Also, whales who buy tokens a lot, also sell a lot, however, they do not send them again.
% % (1) In Figure \ref{f:sale_graph}, we observe whales of various sizes exist, and large dolphins and minnows are also seen. However, in Figure \ref{f:trans_graph}, a few extremely large whales and large size of non-whale traders barely exist. This means that the effect of a small number of whales on other trader is greater in transfer graph than in sale graph.
% (ii) Almost all whales transfer only with dolphins and minnows surrounding them, but they sale with whales (or the others) regardless of cluster.
% % This is due to the observation of a lot of distinct clusters of dolphins and minnows surrounding a whale in Figure \ref{f:trans_graph}, which indicates that transactions occur mainly within each cluster. This is different with whale's \textit{sale}, such clusters appear unclear. 
% (iii) In the transfer graph, there are a lot of small independent components (i.e., Weakly Connected Components) that are not connected to the biggest component. This phenomenon is hard to see in sale graph.

% Since transfer is a transaction without actual payment, we can assume that traders are likely to be in a specific relationship. In other words, whales appear to form a \textit{whale-centered communities} through transfer.



% To close look at how whale-centered communities changes over time, we construct transaction graph $G_t$ by combining both sale and transfer of each month $t$. For each $G_t$, we apply Louvain method~\cite{blondel2008fast}, an optimization method for community detection using modularity. 


% Figure \ref{f:long_community} shows the communities including whales over time. The red dot indicates the number of \textit{transfer-major communities} where number of transfer is more than 10 times compare to that of sale within the community. In early months, there are more than half of the communities in which its main transaction type is sale. However, from May 2021, the ratio of transfer-major communities increase dramatically. Regardless of time, most of them mainly trade one particular collection, a maximum of five. In case of communities that trade multiple collections, the collections are usually interrelated (e.g., made by the same creator). This indicates that communities formed around collections, especially through transfer.\\ 
% \noindent{\bf Findings and Insight.}
% Whales form close relationships with small number of traders through \textit{transfer-major communities}. These communities grow in number over time focusing on specific collections. 



%t = 10 모든 멤버가 holding leader로 이루어진 그룹이 있는데, 8명이 CyberKongz를 125번 transfer
%t = 13 4명이 800번, 6명이 470번 meebits transfer --> meebits 가격 증가 





