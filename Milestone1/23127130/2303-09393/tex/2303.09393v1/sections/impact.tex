To examine how influential whales are in the NFT ecosystem, we analyze the whales' trading behavior in terms of their impact on market sentiment and traders.
%We consider \textit{token\_count} in transaction data as one transaction. 
% As mentioned in Section \ref{s:background}, transaction types include \textit{buy, sell, receive, send, mint}, and \textit{burn}. 

In Figure \ref{f:long_tx}, we observe that the average number of transactions by whales is overwhelmingly large. 
% Note that the $y$-axis of each graph in Figure \ref{f:long_tx} has a different scale. 
% Their activity seems to be decrease a lot in the second half, however, in the case of whale, transactions between traders have decreased and transfers through contract addresses have increased (Figure ? in the Appendix). 
% Trading activities of dolphins and minnows tend to be similar, while whales show much larger fluctuations. 
We observe larger fluctuations in the trading activities of whales than dolphins and minnows.
To understand the whales' behavior, we study how they respond to real-world events related to NFT. 


% \noindent{\bf Sale (Buy/Sell).}
\noindent{\bf Impact on market volatility during liquidation.}
Although the number of whales makes up a small portion of all traders, whales impact market volatility when liquidating NFT collections. Liquidation by whales has drastically increased market volatility. We discuss important events (E1$\sim$E3) in which the whales had a huge impact on the market.

% Contrast to dolphins and minnow, whales' liquidation 
% % Whale group's buying pattern is similar to other groups, but their selling pattern is different. 
% This indicates that they buy tokens according to the market trend, but they only sell them with their own standards.
% Their behavior pattern over time is explained with several events (E1$\sim$E3) that whales had a huge impact on the market.
First, we observe a rapid but short-lived peak in February 2021 in all groups. With the growing popularity of one collection, \textit{Hashmasks}~\cite{hashmasks}, traders buy a huge number of tokens from the collection, which accounts for more than half of their purchases. 

\textbf{(E1)} However, in April, NFT market conditions cooled abruptly. Traders in all groups barely participated in transactions due to a sharp decline (nearly 70\%) in the average price of NFTs~\cite{NFT_burst}. In fact, before the market crash, we observe whale activities drop to its lowest position in March. This implies that the decrease in whale activities has had a big impact on the NFT market sentiment.
% price and flow of the NFT ecosystem.

\textbf{(E2)} In May, we observe an uplift in the number of sells only in the transaction graph for whales. In particular, the whales sell tokens that they have minted on this month. Interestingly, 95\% of such tokens are from two new collections, \textit{Meebits} and \textit{Bored Ape Yacht Club (BAYC)}. 
\textit{Meebits} received the spotlight even before launch, since it is made by the creators behind \textit{CryptoPunks}. 
On the other hand, \textit{BAYC} suddenly gained popularity due to a large quantity of whales' mintings~\cite{bayc_boom}. This promoted the sales of \textit{BAYC} tokens, especially in the dolphins and minnows group. In fact, \textit{BAYC} accounts for 40\% and 54\% of dolphins' and minnows' sales, respectively. 
% The interesting fact is that one of whales, who minted \textit{BAYC} the most (around 1K) and even advertised his minting on Twitter, sold almost all of the tokens to dolphins and minnows by June.

\textbf{(E3)} We observe a steady increase in overall sales volume in July 2021, which increases dramatically by August. This is related to a surge in price of \textit{CryptoPunks} caused by whales. One day in August, a whale bought over 100 \textit{CryptoPunks} NFTs worth more than \$6M in total. Since then, the whales' purchase of the collection continued. Such whale activity sparked public interest in the NFT market, resulting in the influx of new traders~\cite{CryptoPunk_resurge}.

Except for specific periods in which whales actively liquidate their assets, whales tend to transfer (specifically, the act of receiving) instead of selling assets on the market. They prefer to hold assets relatively longer compared to dolphins and minnows. This is discussed in more detail in Section~\ref{s:strategy-hold}.

%% comment out (sh)
% In December, as famous fashion brands (e.g., such as Adidas and Nike) formed NFT partnerships or acquired NFT startups\cite{fashion_nft}, NFT gained greater attention from the public, which made a small increase in three groups. 

% To sum up, it turns out three events (E1 $\sim$ E3) out of five were caused by whales, which verifies a great impact of whales in the NFT market.
%%

% \noindent{\bf Transfer (Receive/Send).}
% \noindent{\bf Transfer (Receive/Send).}

% In the beginning, whales barely do transfer, but from August, the number of transfer becomes greater than or equal to the number of sale. However, for the entire period of our data, among transfer types, whale rarely send NFTs but only receive them. 
% %From September, most of the received tokens are related to game, which is consistent with the surge in popularity of decentralized NFT games at that time. 
% This is in stark contrast to the fact that dolphins and minnows participate in sale much more than transfer. This represents dolphins increase their holding value through buying tokens, while whales increase by receive them during the second half. 
% Interestingly, when we observe whales who receive NFTs a lot, they have a few main senders, and they get hundreds of tokens in a specific collection from each sender.
% % When we observe traders who receive NFTs a lot, almost all of them have their own a few main senders, which is completely contrast to buy. This implies that some traders have a close relationship through transfer. 

%%%%%%%%%%%%%%%%%% HERE!

% \noindent{\bf Mint and Burn.}

% Unlike \textit{Meebits}, \textit{Bored Ape Yacht Club} did not gain much attention at the beginning of their launch. However, as soon as several whales minted a large number of \textit{BAYC} tokens, the rest of the tokens were all minted within a few hours\cite{bayc_boom}. This event also verifies the influence of whales on the market. 
% We observe all three groups actively participate in mint, especially whale group. By September, mint is the most common transaction type of whales, meaning that they increase their holding value not only through buy, but also through mint.
% There are three conspicuous points in their mint. 
\noindent{\bf Leading investment trends. }
Whales have actively participated in \textit{mint} as early stage investors. Also, they prefer to invest specific collections that they are interested in, while other groups participate in a variety of collections. Here, we uncover several remarkable events as follows.

In January 2021, whales mint about four times more than any other transaction types, and most of them are \textit{Hashmasks}. Their mintings surged again in May, due to \textit{Meebits} and \textit{BAYC} as mentioned in E2. 
In the later months including August, some collections (e.g.,\textit{Punks Comic}) embed governance rights on tokens (a.k.a. DAO tokens), that increased the demand for mint. 
% For example, traders received \textit{Punks Comic} DAO tokens by burning their previous tokens of the collection.
This suggests that whales are interested in exercising governance rights on a collection. 
% Noteworthy is whales only mint a huge amount of specific collections, whereas other groups mint various collections. Besides, we find and explain several unique characteristics related to whales' mint in Section 6.2.

NFT market capitalization is increasing due to rise in the number of collections and traders. However, we see a drop in practical trader activities compared to the early months. In fact, only 20\% of minnows participates in trading in the last month (see Table \ref{t:active_users} in the Appendix).  
% In fact, the number of active minnows only makes up 20\% of the total minnows group as of February 2022 (see Table \ref{t:active_users} in the Appendix). 
This implies that the NFT market is mainly operated by whales and dolphins. In addition, we cannot observe any noticeable increase in whale transactions after August, which suggests the lack of emergence of influential collections (e.g., \textit{CryptoPunks}). Despite the large number of collection launches, only a few major collections account for the majority of the NFT market capitalization~\cite{nftgo_marketoverview}. Therefore, holding major collection NFTs allows the whales to maintain their position as whales.\\
%8월 이후에는 눈에 띄게 큰 변화가 없다. 그 뜻은 cryptopunks, bayc처럼 큰 콜렉션이 등장하지 않았다는 뜻. > 실제로 봤더니 엄청나게 많은 콜렉션이 등장했음에도 불구하고, 전체 시장 가치의 ?퍼센트를 얘네가 차지하더라. 결국 원래 고래였던 애들이 그 콜렉션을 계속 들고 있으면 계속 고래. 
\noindent{\bf Findings and Insight.}
The NFT market is predominantly being driven by whales. Whales have distinct trading patterns compared to that of other groups; whales have a huge impact on the market and often alter market sentiment.
% Specifically, the main focus of whales is {\it mint}, {\it } and whales commonly cherry pick high profile NFT collections and participate in the minting process of those collections.  
