\noindent\textbf{Non-fungible token.} An \textit{NFT} is unique, irreplaceable, and its value is decided individually depending on demand. An NFT is a digital asset that represents objects like art, video, in-game entities, etc. NFTs are stored in blockchain, with Ethereum being the representative choice.
% \textit{ERC-721} is a non-fungible token standard that explains how to create NFT in Ethereum blockchain, and most NFTs follow this standard.
% \textit{ERC-1155} extends the representation of fungible and non-fungible tokens, which has an interface to contain multiple tokens. In the case of ERC-1155, a number of tokens with the same identifier can be minted over several times. 
NFTs are categorized in a \textit{collection}, which refers to a set of NFTs. 
Commonly, NFTs in the same collection share common features. For example, one of most popular collection named \textit{Bored Ape Yacht Club (BAYC)} is a collection consisting of 10,000 ape tokens. 
% \textit{CryptoPunks} is one of the earliest NFT collection which accounts for the highest portion of NFT market capitalization. 
The price of NFTs varies widely even within the same collection, although the range varies depending on the collection.


%Due to its anonymous nature of cryptocurrency wallet, we define an unique wallet address as a \textit{trader}. Note that an individual can create multiple wallet addresses. 
 % trading 이랑 같은 말로 쓴다
%\textit{Transaction} is an act of a wallet address passing a token to another wallet address. In this paper, transaction can be divided into four types: \textit{sale, transfer, mint} and \textit{burn}.   
%Usually, a trader passes an NFT to another and get paid accordingly with crpytocurrency, which is called a sale. Meanwhile, some traders only pass tokens without any crypto-payment. This type of transaction is called transfer. Throughout this paper, we also use the word \textit{trading} which includes both sale and transfer.
%Mint is an act of publishing a token. Burn is exactly the opposite, which is destroying a token.



\noindent\textbf{NFT life-cycle.} Due to the anonymous nature of cryptocurrency wallet, we define a unique wallet address as a \textit{trader}. 
% Note that an individual can create multiple wallet addresses and each wallet can have multiple NFTs as well. 
Here, we detail the procedures for how an NFT transaction operates, which consists of three steps: (i) mint, (ii) transfer/sale, and (iii) burn. 
\textit{Mint} refers to the process of offering newly created tokens to the public. Generally, NFT creators post the detailed schedule and price on various channels, such as discord, which is then conducted by a smart contract on a blockchain.
% Most commonly, NFT creators (or operators) post the detailed schedule and price on various channels, such as discord, telegram, and their web page, which is then conducted by a smart contract on a public blockchain.
\textit{Transfer} is an act of simple value transfer which only passes ownership of NFT tokens to another wallet. \textit{Sale} is the process of transferring ownership of NFTs for a price, and is commonly held in NFT markets~\cite{opensea, looksrare}. 
% Note that, throughout this paper, we also use the term \textit{trading} which includes both \textit{sale} and \textit{transfer}. 
\textit{Burn} refers to the method of removing ownership of NFT tokens on purpose. Once an NFT is burned, no one is able to gain its control.

\noindent\textbf{Classifying NFT traders.} There has been no official term for traders (individiuals or entities) that hold differing amounts of NFTs.
Thus, we consider each wallet as an individual \textit{trader} and classify the traders into three groups according to their holding values of NFTs, for every time range in our dataset:

\begin{itemize}
    \item \textbf{Whale} - top 0.1\% traders
    \item \textbf{Dolphin} - top 10\% traders excluding whales
    \item \textbf{Minnow} - all other traders excluding whales and dolphins (89.9\%)
\end{itemize}

To clearly identify the trader groups, we define \textit{holding value} to be used as a concrete criteria for our classification. 
The estimation of the current market price of NFTs may be imprecise due to the high volatility of the market value of NFTs compared to traditional economics. 
Instead, we believe that the latest trading price of each NFT could represent its value well. 
We define \textit{holding value of a trader} as the sum of the last traded price for each token held by the trader. 


% To clearly identify the trader groups, we define \textit{holding value} to be used as a concrete criteria for our classification.
% Traditionally, the current market price of financial assets is representative of its worth.
% In the NFT ecosystem, the estimation of the current market price of NFT assets may be imprecise as the market value of NFT assets is usually highly volatile compared to traditional economics.
% Instead, we believe that the latest trading price of each NFT could represent its value well, so we define the overall holding value of a trader as the sum of the last traded price for each token.
% Also, we construct such trader groups for every single time period range existing in our dataset. 
To focus on the top holding value traders, we refer to whales and dolphins collectively as \textit{holding value leaders}.
The number of traders in each group is shown in Table~\ref{t:data_collection} and is further described in Table~\ref{t:num_of_trader} of the Appendix.
% \lstset{basicstyle=\ttfamily\footnotesize,breaklines=true}
% \begin{lstlisting}[language=Python, caption = An example of an NFT transfer log (we only include the first eight characters for each hash value to preserve anonymity.), label={l:transfer_log}, captionpos=b][ht]

% 'address': '0xbc4ca0ed',
% 'blockHash': '0x0812c2e9'
% 'blockNumber': '13718867'
% 'topics': [
%     '0xddf252ad',
%     '0xb9f0758e',
%     '0xa8cc6763',
%     4949
% ]
% 'transactionHash': '0x13614268'
% 'data': '0x'
% \end{lstlisting}


%We divide the traders into three groups based on their NFT holding value. Since the NFT market cap has been highly volatile, we define them with ranking between traders, not the fixed number of NFT holding value. We classify type of traders every month and their NFT holding values are calculated as the sum of USD of all tokens held by each trader on the last day of each month. There are three groups each for 14 months in total.

% We classify type of traders each month, therefore 3 groups each for 14 months in total. 
%For example, a whale of month 5 are top 0.1\% of traders who traded until May 31, 2021. The number of each trader group in each month is specifically described in Table \ref{t:num_of_trader} of Appendix. 

%\cmt{1. We need to provide more clear classification criteria for each holder. (How to measure value of NFTs and 'when'. Does the type of holder change over time?. Transactions have continuously processed with frequency and these transactions affect the type of holders?).}

