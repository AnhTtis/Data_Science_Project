In this section, we discuss several investment strategies that whales utilize to achieve high financial gains. We discover three strategies: (i) minting or buying expensive NFTs and holding them for a long time until liquidation, (ii) intensively investing during the minting period, and (iii) taking advantage of self-trading.

% In this section, we examine whales' unique investment strategies to achieve financial gain.  
% We first track the top 1\% valuable NFTs of each collection and how whales handle these tokens. Then, we scrutinize their strategies right after mint. Moreover, we have detected extreme methods whales take advantage of, that is, wash trading.  

% \subsection{Strategy 1: Hold till the end}

\subsection{Long-term Investment}
\label{s:strategy-hold}
In a market, the value of a token is decided by its price; valuable tokens are limited but takes up a major portion of the market cap. Therefore, we define and track \textit{most-valuable NFTs} as the top 1\% most expensive NFTs from each collection. By accumulating such tokens, we obtain 14,192 most-valuable NFTs. 

To investigate the strategies whales use on the most-valuable NFTs, we begin by calculating how long each group holds the tokens before selling them. Figure \ref{f:holding_time} shows the holding time of each group on most-valuable NFTs. Here, holding time is the duration between the last buy/receive time of each token and the last day of our data (February 28, 2022). As clearly shown from the graph, whales are likely to hold tokens longer than any other group. It turns out that some whales even hold tokens for the entire period of our data. On the other hand, for minnows, the peak in the left range of the graph is due to their relatively short holding period on collections that have low average token price. While the holding time distributions of minnows and whales are very distinct from each other, the distribution of dolphins resemble that of whales. Hence, the graph indicates that holding value leaders tend to hold the valuable tokens longer. 

These observations bring about a question related to the liquidity of high-value tokens; do the other traders have any chance at acquiring such tokens? To answer this, we track the sale history of these tokens. Figure \ref{f:mvt_sale} illustrates the number of sales by whales on most-valuable NFTs. The graph clearly shows that the number of buys across all months heavily outweigh the number of sells. This indicates that high-value NFTs have low-liquidity. 

These two strategies are likely to be long-term tactics by whales to wait for valuable tokens to reach their desired price. Indeed, the results suggest that whales consistently collect most-valuable NFTs but barely sell them for a long duration. \\
\noindent\textbf{Findings and Insight.}
Whales are patient; they invest by holding valuable NFTs for a long period and do not sell them to other traders. NFT traders must be aware of low-liquidity of high-value tokens caused by whales.
% 비트코인 시장에서는 많이 들고 있으면 장땡이지만, token의 rarity나 예술적 가치가 가격을 결정하는 nft시장에서 valuable한 토큰들이 대부분 특정 트레이더들에 의해 독점되고 있다는 것은 noteworthy하며, 트레이더들은 이 사실에 be aware!
\begin{figure}[t!]
  \centering
  \includegraphics[width=0.8\linewidth]{figures/holding_time.pdf}
  \caption{The holding time of most-valuable tokens per group}
  \label{f:holding_time}
\end{figure}

\begin{figure}[t!]
  \centering
  \includegraphics[width=0.9\linewidth]{figures/mvt_sale.pdf}
  \caption{Buy/sell plot by whales on most-valuable tokens (stacked plot)}
  \label{f:mvt_sale}
  \vspace{-10pt}
\end{figure}
% \subsection{Strategy 2: Minter's Plans}
\subsection{Deliberate Investment During the Minting Phase}
We find out that whales take advantage of minting via unique methods. They use a number of strategies to raise the price of tokens they minted. We describe each strategy throughout this section.\\  
\noindent\textbf{Strategy 2-A.} The first strategy is related to the number of transfers between the first minting period and the first sale. Around 10\% of tokens minted by each group are transferred at least once before the first sale. To compare each group's transfer tendency, we look into those tokens in detail. Figure \ref{f:mint_trasfer} shows the number of transfers between the mint and the first sale, where the number is shown in percentage. As shown in the figure, a large portion of NFTs minted by whales are transferred many times (up to 25 times). 72.0\% of tokens minted by whales are transferred at least two times whereas the percentages are much smaller in dolphins and minnows. Furthermore, 13.8\% of tokens minted by whales go through transfer at least 5 times. 

\noindent\textbf{Strategy 2-B.} Another noticeable strategy is relevant to time duration before the first sale. This can be seen from Figure \ref{f:mint_days} where the time duration is divided into four ranges: \textit{within a day}, \textit{a day to a week}, \textit{a week to a month} and \textit{more than a month}. Surprisingly, whales tend to wait for long periods of time before the first sale; 82.6\% of tokens take more than a week and 58.5\% take more than a month. In contrast, dolphins and minnows wait less and the majority of tokens are sold within a week for both groups.
% dolphin, minnows
% 1 transfer: [3.994732221246708, 5.912665957433725, 12.47018031852732]
% more than 2 : [10.294117647058822, 4.096127711124913, 4.148919730729483]

\noindent\textbf{Strategy 2-C.} This strategy is used by some of the minters. Usual sales after minting do not involve the address (i.e., minter) that minted the token. However, we find that some minters later receive back the token that they minted. To give an instance that occurred in September 2021, the minter first minted the token and sold it to another wallet address for 6.25 ETH. Then, the token was transferred back to the minter. Finally, the minter sold the token to another trader for 6 ETH. Indeed, we find that 13.7K tokens are involved in this type of transaction pattern. Whales and dolphins are involved in 10.4K tokens which accounts for 77.4\% of such tokens. This is a significant number considering the fact that usually the total number of tokens for a collection is around 10K. 

All three strategies are effectively utilized to maximize the profit of whales. We find out that the profit of the first sale is proportional to all of these strategies. Details of these profits can be found in Table \ref{t:mint_appendix1} and \ref{t:mint_appendix2}. Further details for mint profits from whales are discussed in Section \ref{s:profit-mint}\\
\noindent\textbf{Findings and Insight.} Whales use their own special strategies to maximize the profit of tokens they minted. 

\begin{figure}[t!]
  \centering
  \includegraphics[width=0.6\linewidth]{figures/mint_transfer_.pdf}
  \caption{Number of transfers between \textit{mint} and first \textit{sell}}
  \label{f:mint_trasfer}
\end{figure}
\begin{figure}[t!]
  \centering
  \includegraphics[width=0.6\linewidth]{figures/mint_time_.pdf}
  \caption{Time duration between \textit{mint} and first \textit{sell}}
  \label{f:mint_days}
  \vspace{-15pt}
\end{figure}
% 2번쨰는...
% 그러다가 mint한 사람이 트랜스퍼 거쳐서 자기가 다시 파는걸 몇번 발견 그게 다른 그룹에서도 있는지 그런 행동은 whale이 몇퍼센트 차지하는지 -> figure 하나 설명

% \begin{figure}[t!]
%   \centering
%   \includegraphics[width=0.8\linewidth]{figures/minter_overlap.pdf}
%   \caption{Instance of self trading in \textit{CyberKongz} on September, 2021}
%   \label{f:mint_overlap}
% \end{figure}
\subsection{Wash Trading}
% The number of transfer-major communities is similar to the transaction pattern in Figure \ref{f:long_tx} except the months of early 2022. Indeed, the increase of communities at that time is closely related to wash trading. \\
% \noindent{\bf Wash trading}
%초반에는 opensea 마켓에 콜렉션을 입증시키기 위해 --> 콜렉션 set을 주고 받음
%2022년 초반에는 reward를 받기 위해 --> 토큰 1개를? 
\begin{figure}[t!]
  \centering
  \includegraphics[width=0.9\linewidth]{figures/wash.pdf}
  \caption{Number of wash trading instances(stacked plot)}
  \label{f:long_wash}
  \vspace{-15pt}
\end{figure}
Wash trading is a collusion by the buyer and the seller to artificially inflate the trading volume of an asset. They repeatedly trade their assets between them, which results in cycles in the sale graph, where nodes are traders and edges are sales between them. Therefore, wash tradings are captured by finding strongly connected components (SCCs) in each token sale graph~\cite{victor2021detecting}. 
Indeed, multiple unconditional token transfers can be used in trading malpractices to avoid monitoring~\cite{das2021understanding, victor2021detecting}. 
Thus, we construct each NFT transaction graph, where nodes are traders and edges are transactions between them, and find SCCs in each graph including sales at least once. 

We detect 3,558 instances of wash trading in 1,676 NFTs across 82 out of 91 collections including 3,676 traders in our data. Note that the collected NFT transaction data is different for each period and for each collection, the number of wash trading instances can be different from the work done by Victor et al.~\cite{victor2021detecting}. Surprisingly, the holding value leaders perform 69\% of wash trading. Furthermore, 42 whales, which makes up 10\% of all whales, are involved. We find that popular collections with high trading volume (e.g., \textit{CryptoPunks, BAYC}) also belong to manipulations done by the holding value leaders.

To understand when and why wash trading occurs over time, we locate SCCs in each token transaction graph from the monthly transaction records.  
Figure \ref{f:long_wash} shows the number of wash trading over time.
First, we observe a noticeable peak in October. Wash trading for \textit{The n project} started in September (when it was launched) and recorded 506 cases in October. We confirm that the average price of the collection tokens traded sharply rises whenever washing trading is involved as shown in Figure \ref{fig:wash_price} of the Appendix. 

This is a common phenomenon in newly launched collections. In order for a newly-launched collection to be verified by \textit{Opensea} (the largest NFT marketplace), the trading volume must be over 100 ETH~\cite{opensea_verify}. For this reason, many traders are tempted to perform wash trading. Similarly, we can see a increase in the number of wash trading in August 2021, when many new collections emerged. 

The number of wash trading shows another small peak in January 2022, which is related to the policy of LooksRare~\cite{meebits_wash}, a newly-launched NFT marketplace. It rewards valuable tokens to traders according to each trader’s trading volume, which lures traders into wash trading. \\
\noindent\textbf{Findings and Insight.}
Whales and dolphins actively participate in wash trading to receive verification of newly-launched collections in NFT matketplaces or earn rewards by abusing NFT market policy. NFT prices can be driven up and down along with their wash trading. Popular collections (e.g., \textit{CryptoPunks, BAYC}) are also unavoidable.