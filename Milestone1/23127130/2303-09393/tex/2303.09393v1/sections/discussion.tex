
% whale has the power to drive market
Throughout this paper, we find out that whales are highly influential traders in the NFT ecosystem. Whales hold a mere 5 percent of all NFT tokens on the Ethereum blockchain, but their worth accounts for nearly 20 percent of the entire NFT market value. In addition, although the NFT market features relatively wide variations in prices, nearly all of the high value items worth more than one million dollars belong to the whales. This means that whales exclusively own almost all of the dominant NFT items. Considering that a small number of NFT projects with a high market cap (e.g., \textit{CryptoPunks}, \textit{BAYC}, \textit{Doodles}) is bringing success to the NFT ecosystem, to say that the NFT market is being driven by whales is not an overstatement.

% get rich? be the whale, kid.
In summary, the average profit from whales is far greater than that of the other groups. In addition, the most highly profitable traders (i.e., make profits of more than \$1M) make up over 35\% of all whale traders, while there such traders only make up under one percent of dolphins and minnows. A large portion of traders from minnows did not make profits, with some even faced with considerable losses. This indicates that a small number of whales (note that we only consider the top 0.1 percent of traders as whales) takes most of the rewards for success in the NFT market. In other words, the NFT ecosystem is a harsh environment to find success for the majority of the traders involved (non-whales), and stakes are high for the minority traders. Therefore, it is imperative that various studies about the NFT ecosystem are conducted and measures to protect investors are established.


