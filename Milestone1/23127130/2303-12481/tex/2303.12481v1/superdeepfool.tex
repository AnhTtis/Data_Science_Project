%%%%%%%% ICML 2021 EXAMPLE LATEX SUBMISSION FILE %%%%%%%%%%%%%%%%%

\documentclass{article}

% Recommended, but optional, packages for figures and better typesetting:
\usepackage{microtype}
\usepackage{graphicx, xcolor}
\usepackage{bbm}%
\usepackage{relsize}
\usepackage{float}
% \floatstyle{plaintop}%
% \restylefloat{table}
\usepackage{pgffor}

\usepackage[space]{grffile} %
% \usepackage{subfigure}

\usepackage{datatool}

\usepackage{csvsimple}
\usepackage{xspace}
\usepackage{etoolbox} %
\usepackage{mathtools} %


\usepackage{xhfill}

\usepackage[breakable, theorems, skins]{tcolorbox}

\usepackage[normalem]{ulem} %
\usepackage{booktabs} % for professional tables
\usepackage{amsmath}
\usepackage{amssymb}
\usepackage{mathrsfs}
\usepackage{subcaption}
\usepackage{multirow}
\usepackage{makecell}
\usepackage{xcolor}
\usepackage{amssymb}% http://ctan.org/pkg/amssymb
\usepackage{pifont}% http://ctan.org/pkg/pifont
\newcommand{\cmark}{\ding{51}}%
\newcommand{\xmark}{\ding{55}}%
\usepackage{nicematrix}
\usepackage{xcolor}
% \usepackage{tabularray}
% \usepackage{colortbl}
% hyperref makes hyperlinks in the resulting PDF.
% If your build breaks (sometimes temporarily if a hyperlink spans a page)
% please comment out the following usepackage line and replace
% \usepackage{icml2021} with \usepackage[nohyperref]{icml2021} above.
\usepackage{hyperref}

% Attempt to make hyperref and algorithmic work together better:
\newcommand{\theHalgorithm}{\arabic{algorithm}}
\usepackage{mathrsfs}
\usepackage{color}
\usepackage{subcaption}
\usepackage{footmisc}
\usepackage[]{algorithm2e}
\usepackage{makecell}
\usepackage{multirow}
\usepackage{fancyhdr}
\usepackage{appendix}
\usepackage{textcomp}
% \usepackage[dvipsnames]{xcolor}
% \usepackage{wrapfig}

% Use the following line for the initial blind version submitted for review:
% \usepackage{icml2023}

% If accepted, instead use the following line for the camera-ready submission:
\usepackage[accepted]{icml2023}

% The \icmltitle you define below is probably too long as a header.
% Therefore, a short form for the running title is supplied here:
\icmltitlerunning{Revisiting DeepFool: Generalization and Improvement}


%%%%%%%%%%%%%% Notation %%%%%%
\def\x{\boldsymbol{x}}
\def\y{\boldsymbol{y}}
\def\r{\boldsymbol{r^*}_{adv}(\x)}
\def\w{\boldsymbol{w}}
%%%%%%%%%%%%%%%%%%%%%%%%%%%%%%
\newtheorem{lemma}{Lemma}
\newtheorem{theorem}{Theorem}
\newtheorem{proposition}{Proposition}
\newtheorem{proof}{Proof}
%%%%%%%%%%%%%%%%%%%%%%%%%%%%%%%%%%%%%%%%%%%

\DeclareMathOperator*{\argmin}{arg\,min}
\DeclareMathOperator*{\argmax}{arg\,max}
\DeclareMathOperator*{\sign}{sign}
\newcolumntype{Y}{>{\centering\arraybackslash}X}
\newcolumntype{s}{>{\hsize=.4\hsize}X}
\newcommand{\olive}[2]{{\color{red}{ #1 }}{\small\color{olive}{ #2 }}}
\newcommand{\mahed}[2]{{\color{blue}{#1}}{\small\color{blue}{#2}}}

\newcommand{\cw}{\texttt{C$\&$W}}
\newcommand{\DF}{\texttt{DF}}
\newcommand{\DDN}{\texttt{DDN}}
\newcommand{\project}{\textsf{project}}
\newcommand{\SDF}{\texttt{SDF}}
\newcommand{\grad}{\nabla}
\newcommand{\C}{\mathcal{C}}
\newcommand{\bbR}{\mathbb{R}}
\newcommand{\bbN}{\mathbb{N}}
\newcommand{\E}{\mathbb{E}}
\newcommand{\std}[1]{\tiny{$\pm$ #1}}
\begin{document}

\twocolumn[

\icmltitle{Revisiting DeepFool: generalization and improvement}
\icmlsetsymbol{equal}{*}

\begin{icmlauthorlist}
\icmlauthor{Alireza Abdollahpourrostam}{a}
\icmlauthor{Mahed Abroshan}{b}
\icmlauthor{Seyed-Mohsen Moosavi-Dezfooli}{c}
\end{icmlauthorlist}

\icmlaffiliation{c}{Imperial College London, UK}
\icmlaffiliation{b}{Optum Labs, London, UK}
\icmlaffiliation{a}{Tehran Polytechnic, Iran}

\icmlcorrespondingauthor{Alireza Abdollahpourrostam}{alirezaabdollahpour1380@gmail.com}
% \icmlcorrespondingauthor{Eee Pppp}{ep@eden.co.uk}

% You may provide any keywords that you
% find helpful for describing your paper; these are used to populate
% the "keywords" metadata in the PDF but will not be shown in the document
\icmlkeywords{Robustness, Adversarial examples, Adversarial Attacks}
\vskip 0.3in]

% this must go after the closing bracket ] following \twocolumn[ ...

% This command actually creates the footnote in the first column
% listing the affiliations and the copyright notice.
% The command takes one argument, which is text to display at the start of the footnote.
% The \icmlEqualContribution command is standard text for equal contribution.
% Remove it (just {}) if you do not need this facility.

\printAffiliationsAndNotice{}  % leave blank if no need to mention equal contribution
% \printAffiliationsAndNotice{\icmlEqualContribution} % otherwise use the standard text





Over the past few years, there has been a significant amount of research focused on studying the ReLU activation function, with the aim of achieving neural network convergence through over-parametrization. However, recent developments in the field of Large Language Models (LLMs) have sparked interest in the use of exponential activation functions, specifically in the attention mechanism.

Mathematically, we define the neural function $F: \R^{d \times m} \times  \mathbb{R}^d \rightarrow \mathbb{R}$ using an exponential activation function. Given a set of data points with labels $\{(x_1, y_1), (x_2, y_2), \dots, (x_n, y_n)\} \subset \mathbb{R}^d \times \mathbb{R}$ where $n$ denotes the number of the data. Here $F(W(t),x)$ can be expressed as $F(W(t),x) := \sum_{r=1}^m a_r \exp(\langle w_r, x \rangle)$, where $m$ represents the number of neurons, and $w_r(t)$ are weights at time $t$. It's standard in literature that $a_r$ are the fixed weights and it's never changed during the training. We initialize the weights $W(0) \in \mathbb{R}^{d \times m}$ with random Gaussian distributions, such that $w_r(0) \sim \mathcal{N}(0, I_d)$ and initialize $a_r$ from random sign distribution for each $r \in [m]$.

Using the gradient descent algorithm, we can find a weight $W(T)$ such that $\| F(W(T), X) - y \|_2 \leq \epsilon$ holds with probability $1-\delta$, where $\epsilon \in (0,0.1)$ and $m = \Omega(n^{2+o(1)}\log(n/\delta))$. To optimize the over-parametrization bound $m$, we employ several tight analysis techniques from previous studies [Song and Yang arXiv 2019, Munteanu, Omlor, Song and Woodruff ICML 2022]. 

 

\section{Introduction}

The increasing complexity of source code poses a key challenge to the reliability of large-scale software systems. Software bugs in these systems can lead to safety issues~\cite{bug_safety} for users around the world as well as cause non-negligible financial losses~\cite{bug_loss}. As such, developers have to spend a large amount of time and effort on bug fixing. Consequently, \aprfull (\apr), designed to automatically generate patches to fix software bugs, has attracted wide attention from both academia and industry~\cite{long2016prophet, legoues2012genprog, long2015spr, lou2020can, tufano2018empstudy}. 


To achieve \apr, one popular approach is known as Generate-and-Validate (G\&V)~\cite{qi2015gv, ghanbari2019prapr, lou2020can, le2016hdrepair, legoues2012genprog, wen2018capgen, hua2018sketchfix, martinez2016astor, koyuncu2020fixminder, liu2019tbar, liu2019avatar}, which is typically based on the following pipeline: First, fault localization techniques~\cite{wong2016fl, abreu2007ochiai, zhang2013injecting, papadakis2015metallaxis, li2019deepfl, li2017transforming} are applied to determine the suspicious locations in programs where bugs are likely to exist. Then, the buggy locations are used by the \apr tools to generate a list of patches that replace buggy lines with correct lines. Afterward, each patch is validated against the original test suite to identify any \emph{plausible patches} (i.e., passing all tests in the test suite). Finally, to determine the \emph{correct patches}, developers examine the list of plausible patches to see if any of them can correctly fix the bug. 

Traditional \apr tools can mainly be categorized into heuristic-based~\cite{legoues2012genprog, le2016hdrepair, wen2018capgen}, constraint-based~\cite{mechtaev2016angelix, le2017s3, demacro2014nopol, long2015spr} and \template~\cite{ghanbari2019prapr, hua2018sketchfix, martinez2016astor, liu2019tbar, liu2019avatar}. Among these traditional tools, \template \apr tools~\cite{ghanbari2019prapr, liu2019tbar, benton2020effectiveness} have been able to achieve state-of-the-art results. \Template \apr tools typically leverage pre-defined templates (e.g., adding a nullness check) for bug fixing. However, since these fix templates are typically handcrafted, the number and types of bugs they are able to fix can be limited. 



To address the limitations of traditional \apr, researchers have proposed various \learning \apr tools~\cite{li2020dlfix, chen2018sequencer, jiang2021cure, lutellier2020coconut, zhu2021recoder, ye2022rewardrepair} based on the \nmtfull (\nmt) architecture~\cite{sutskever2014mt} where the input is the buggy code snippets and the goal is to translate the buggy code snippets into a fixed version. To accomplish this, \learning \apr tools require supervised training datasets with pairs of both buggy and fixed code snippets in order to learn how to perform this translation step. These training data are usually obtained by mining historical bug fixes using heuristics/keywords~\cite{dallmeier2007benchmark}, which can be imprecise for identifying bug-fixing commits; even the actual bug-fixing commits can include irrelevant code changes, leading to further pollution in the dataset~\cite{xia2022alpharepair}.
% 
Moreover, it can be hard for such \apr tools to generalize and fix bug types unseen during training. 



To better leverage recent advances in \plmfull{s} (\plm{s}), researchers~\cite{xia2022alpharepair, xia2023repairstudy, kolak2022patch, prenner2021codexws} have directly applied \plm{s} to generate patches without bug-fixing datasets. These \llm-based \apr tools work by either directly generating a complete code function~\cite{prenner2021codexws, xia2023repairstudy} or predict/infill the correct code snippet given its surrounding context~\cite{xia2022alpharepair, xia2023repairstudy}. By directly using \llm{s} that are pre-trained on billions of open-source code snippets, \llm-based \apr tools can achieve state-of-the-art performance on many repair datasets~\cite{xia2022alpharepair}. 


% 
%
%

Traditional \apr tools have long used the insight of the \emph{plastic surgery hypothesis}~\cite{barr2014plastic} where it states that the code ingredients to fix a bug already exist within the same project. Traditional \apr tools have manually designed pattern-~\cite{ghanbari2019prapr, saha2017elixir} or heuristic-based~\cite{jiang2018simfix, legoues2012genprog} approaches to finding and using such relevant code ingredients to generate fixes for bugs. However, the plastic surgery hypothesis has been largely ignored in \llm-based \apr. In fact, \llm provides a unique opportunity to fully automate the plastic surgery hypothesis idea via fine-tuning (learning project-specific information via model updates from the buggy project) and prompting (directly providing relevant code ingredients to the model), and make it directly applicable to different languages (since the \llm{s} are typically multi-lingual).%
Moreover, despite the intensive manual efforts involved, traditional \apr tools still cannot fully leverage project-specific information due to large search space for leveraging/composing existing code ingredients. In contrast, the project-specific information can effectively leveraged by \llm{s} due to their power in code understanding/vectorization, e.g., even partial/imprecise information may still guide \llm{s} in correct patch generation!
 To this end, we ask the question: \emph{How useful is the plastic surgery hypothesis in the era of \plm{s}}?








\mypara{Our Work.} To answer the question, we present \ourtech{\xspace} -- a \llm-based approach that automatically utilizes the plastic surgery hypothesis by systematically combining multiple fine-tuning and prompting strategies for \apr. \ourtech fine-tunes \plm{s} using two novel domain-specific training strategies: \textbf{\epfinetune} -- we fine-tune using the original buggy project by aggressively masking out a high percentage of tokens, which allows \plm to learn project-specific code tokens and programming styles; and \textbf{\rofinetune} -- which only masks out a single continuous code sequence per training sample, allowing the model to get used to the final \csapr task of predicting a single continuous code sequence. Furthermore, we directly leverage the ability for \plm{s} to understand natural language instructions and introduce a novel prompting strategy, \textbf{\idprompting}, which uses information retrieval and static analysis to obtain a list of relevant identifiers for the buggy lines. While such relevant identifiers are critical for fixing some difficult bugs, they may not be seen by the \llm during inference due to limited context window size. Through the use of prompting, we directly tell the model to use these extracted identifiers (relevant code ingredients) to generate the correct code. Finally, to perform repair, we combine all four model variants (including the base model, both fine-tuned models and the base model with prompting) for the final repair.





While our insight of leveraging the plastic surgery hypothesis for \llm-based \apr is generalizable across different types of \plm{s}, to implement \ourtech, we choose a recent \plm{\xspace}, \ctfive~\cite{wang2021codet5}, which is pre-trained on millions of open-source code snippets. \ctfive is an encoder-decoder model trained using \mspfull (\msp) objective where a percentage of tokens are masked out and each continuous masked token sequence is referred to as a masked span. Also, although we only extract relevant identifiers from the current buggy project (since this paper focuses on the plastic surgery hypothesis), our work can be easily extended to obtain other code information (such as relevant statements or functions) from other sources, such as  the massive pre-training corpora~\cite{husain2020codesearchnet} or historical bug-fixing datasets~\cite{jiang2019infer}, which can provide more coding knowledge for \llm{s}. Besides, although we mainly focus on using traditional string comparison algorithms for information retrieval in this paper, these techniques can be easily replaced by other frequency-based retrieval~\cite{robertson2009probabilistic} and neural search (or embedding-based search)~\cite{reimers2019sentence}.
  In summary, this paper makes the following contributions:


%


\begin{itemize}[noitemsep, leftmargin=*, topsep=0pt]
    \item \textbf{Dimension.} This paper is the first to revisit the important plastic surgery hypothesis in the era of \llm{s}. It opens up a new dimension for \llm-based \apr to incorporate previously neglected information from the buggy project itself to boost \apr performance. Furthermore, it demonstrates the promising future of retrieval-based prompting for modern \llm-based \apr.
    \item \textbf{Implementation.} We implement \ourtech based on the recent \ctfive model. We augment the model using two novel fine-tuning strategies: \epfinetune and \rofinetune, along with a novel prompting strategy based on information retrieval and static analysis: \idprompting. We combine the patches generated by all four models together and perform patch ranking to speed up \apr.% 
    \item \textbf{Evaluation Study.} We conduct an extensive evaluation against state-of-the-art \apr tools. On the widely studied \dfj 1.2 and 2.0 datasets~\cite{just2014dfj}, \ourtech is able to achieve the new state-of-the-art results of 89 and 44 correct bug fixes (15 and 8 more than best baseline) respectively.  Furthermore, we perform a broad ablation study to justify our design. \ourtech demonstrates for the first time that the plastic surgery hypothesis can substantially boost \llm-based \apr and advance state-of-the-art \apr, while being fully automated and general. Moreover, even partial/imprecise code ingredients may still effectively guide \llm{s} for \apr!
\end{itemize}


\vspace{-0.08in}
\paragraph{Why does $\ell_2$ white-box adversarial robustness matter?}
\vspace{-0.11in}
The reasons for using $\ell_2$ norm perturbations are manifold.
We acknowledge that $\ell_2$ threat model may not seem particularly realistic in practical scenarios (at least for images); however, it can be perceived as a basic threat model amenable to both theoretical and empirical analyses, potentially leading insights in tackling adversarial robustness in more complex settings. The fact that, despite considerable advancements in AI/ML, we are yet to solve adversarial vulnerability, motivates part of our community to return to the basics and work towards finding fundamental solutions to this issue~\cite{Jailbreakbench-Maksym,Carlini-Prompt,Wong-Prompt}.
In particular, thanks to their intuitive geometric interpretation, $\ell_2$ perturbations provide valuable insights into the geometry of classifiers. They can serve as an effective tool in the "interpretation/explanation" toolbox to shed light on what/how these models learn.
Moreover, it has been demonstrated that~\cite{Guille_Optimism,engstrom2019adversarial}, $\ell_2$ robustness has several applications beyond security~(for more details on the necessity of robustness to $\ell_{p}$ norms, please refer to Appendix~\ref{L-P}).\looseness=-1
%A note on the spectral properties of $\ell_2$ perturbations: It is not entirely accurate to categorize $\ell_2$ perturbations as “high-frequency”. On the contrary, these perturbations tend to align more with "low-frequency" bands (see, e.g., [11]).
\vspace{-0.12in}
\section{DeepFool (DF) and Minimal Adversarial Perturbations}
\vspace{-0.10in}
\label{DeepFool}
In this section, we first discuss the geometric interpretation of the minimum-norm adversarial perturbations, i.e., solutions to the optimization problem in~\eqref{global-formula}.
We then examine DF to demonstrate why it may fail to find the optimal minimum-norm perturbation. Then in the next section, we introduce our proposed method that exploits DF to find smaller perturbations.

Let $f$ : $\mathbb{R}^{d} \rightarrow \mathbb{R}^{C}$ denote a $C$-class classifier, where $f_k$ represents the classifier's output associated to the $k$th class. Specifically, for a given datapoint $\x$ $\in$ $\mathbb{R}^{d}$, the estimated label is obtained by $\hat{k}(\x)= \text{argmax}_{k} f_{k}(\x)$, where $f_{k}(\x)$ is the $k^{\text{th}}$ component of $f(\x)$ that corresponds to the $k^{\text{th}}$ class.
Note that the classifier $f$ can be seen as a mapping that partitions the input space $\mathbb{R}^{d}$ into classification regions, each of which has a constant estimated label (i.e., $\hat{k}(.)$ is constant for each such region). The decision boundary $\mathscr{B}$ is defined as the set of points in $\mathbb{R}^d$ such that $f_i(\x)=f_j(\x)=\max_{k}f_k(\x)$ for some distinct $i$ and $j$.
Additive $\ell_2$-norm adversarial perturbations are inherently related to the geometry of the decision boundary. More formally, Let $\x \in \mathbb{R}^{d}$, and $\boldsymbol{r}^{*}(\x)$ be the minimal adversarial perturbation defined as the minimizer of~\eqref{global-formula}. Then: 
\insight{\textbf{\textit{Properties of \textbf{minimal adversarial perturbation $\rightarrow$ $\boldsymbol{r}^{*}(\x)$}}}}
{

\circled{1} It is orthogonal to the decision boundary of the classifier $\mathscr{B}$.
	
\circled{2} Its norm, i.e., $\|\boldsymbol{r}^{*}(\x)\|_2$ measures the Euclidean distance between $\x$ and $\mathscr{B}$, that is $\x+\boldsymbol{r}^{*}$ lies on $\mathscr{B}$.}
%Then  $\boldsymbol{r}^{*}(\x)$,
% 1) is orthogonal to the decision boundary of the classifier $\mathscr{B}$, and 2) its norm $\|\boldsymbol{r}^{*}(\x)\|_2$ measures the Euclidean distance between $\x$ and $\mathscr{B}$, that is $\x+\boldsymbol{r}^{*}$ lies on $\mathscr{B}$. 
 We aim to investigate whether the perturbations generated by DF satisfy the aforementioned two conditions. Let $\boldsymbol{r}_\text{DF}$ denote the perturbation found by DF for a datapoint $\x$.
We expect $\boldsymbol{x}+\boldsymbol{r}_\text{DF}$ to lie on the decision boundary. Hence, if $\boldsymbol{r}$ is the minimal perturbation, for all $0<\gamma <1$, we expect the perturbation $\gamma \boldsymbol{r}$  to remain in the same decision region as of $\x$ and thus fail to fool the model.
%on about the geometrical behavior of deep neural networks.
\begin{wrapfigure}[16]{r}{0.40\textwidth}
	\vspace{-0.30cm}
	\centering
	\includegraphics[width=0.3\textwidth]{photos/orthogonality_illustration.pdf}
	\caption{Illustration of the optimal adversarial example $\x+\boldsymbol{r}^*$ for a binary classifier $f$; the example lies on the decision boundary (set of points where $f(\x)=0$) and the perturbation vector $\boldsymbol{r}^*$ is orthogonal to this boundary.}
	\label{fig:orthogonality_illust}
\end{wrapfigure}
%\paragraph{\textbf{Geometric conditions of optimal perturbations.}} 

Fig.~\ref{fig:orthogonality_illust} illustrates the two conditions discussed in Section~\ref{DeepFool}. In the figure, \(n_1\) and \(n_2\) represent two orthogonal vectors to the decision boundary. The optimal perturbation vector \(\boldsymbol{r}^*\) aligns parallel to \(n_2\). On the other hand, a non-optimal perturbation \(\boldsymbol{r}_{\text{DF}}\) forms an angle \(\alpha\) with \(n_1\).

In Fig.~\ref{fig:DF_analysis}~(left), we consider the fooling rate of $\gamma\,\boldsymbol{r}_\text{DF}$ for $0.2<\gamma<1$. For a minimum-norm perturbation, we expect an immediate sharp decline for $\gamma$ close to one. However, in Fig.~\ref{fig:DF_analysis}~(top-left) we cannot observe such a decline (a sharp decline happens close to $\gamma=0.9$, not 1).
This is a confirmation that DF typically finds an overly perturbed point. One potential reason for this is the fact that DF stops when a misclassified point is found, and this point might be an overly perturbed one within the adversarial region, and not necessarily on the decision boundary.

%\begin{wrapfigure}[11]{r}{0.30\textwidth}
%	\vspace{-0.780cm}
%    \centering
%    \includegraphics[width=0.25\textwidth]{photos/orthogonality_illustration.pdf}
%    \caption{\orig{Illustration of the optimal adversarial example $\x+\boldsymbol{r}^*$ for a binary classifier $f$; the example lies on the decision boundary (set of points where $f(\x)=0$) and the perturbation vector $\boldsymbol{r}^*$ is orthogonal to this boundary.}}
%    \label{fig:orthogonality_illust}
%\end{wrapfigure}

Now, let us consider the other characteristic of the minimal adversarial perturbation. That is, the perturbation should be orthogonal to the decision boundary. We measure the angle between the found perturbation $\boldsymbol{r}_\text{DF}$ and the normal vector orthogonal to the decision boundary~($\nabla f(\x+\boldsymbol{r}_\text{DF})$). To do so, we first scale $\boldsymbol{r}_\text{DF}$ such that $\x+\gamma\boldsymbol{r}_\text{DF}$ lies on the decision boundary. It can be simply done via performing a line search along $\boldsymbol{r}_\text{DF}$. We then compute the cosine of the angle between $\boldsymbol{r}_\text{DF}$ and the normal to the decision boundary at $\x+\gamma\boldsymbol{r}_\text{DF}$~(this angle is denoted by $\cos(\alpha)$). 
A necessary condition for $\gamma\boldsymbol{r}_\text{DF}$ to be an optimal perturbation is that it must be parallel to the normal vector of the decision boundary.
In Fig.~\ref{fig:DF_analysis}~(right) , we show the distribution of cosine of this angle. Ideally, we wanted this distribution to be accumulated around one. However, it clearly shows that this is not the case, which is a confirmation that $\boldsymbol{r}_\text{DF}$ is not necessarily the minimal perturbation.
% \begin{figure}[t]
% % \begin{wrapfigure}[7]{r}{0.5\textwidth}
%     \centering
%     \includegraphics[width=0.3\columnwidth]{photos/Overly_perturbed_DF_S.pdf}
%     \caption{We generated $1000$ images with one hundred $\gamma$ between zero and one, and the fooling rate of the DF is reported. This experiment is done on the CIFAR-$10$ dataset and ResNet-$18$~\cite{he2016deep} model. The accuracy of this Network is $94\%$.}
%     \label{fig:overly}
% \end{figure}
% % \end{wrapfigure}

% \begin{figure}[t]
%     \centering
%     \includegraphics[width=0.3\columnwidth]{photos/orthogonality-2_S.pdf}
%     \caption{Histogram of the cosine angle distribution between the gradient in the last step of DF and the perturbation vector obtained by DF.
%     This experiment has been performed on $1000$ images from the CIFAR-$10$~\cite{krizhevsky2009learning} dataset with the ResNet-$18$~\cite{he2016deep} model.}
%     \label{fig:orthogonality}
% \end{figure}

% We generated $1000$ images with one hundred $\gamma$ between zero and one, and the fooling rate of the DF is reported. This experiment is done on the CIFAR-$10$ dataset and ResNet-$18$~\cite{he2016deep} model. The accuracy of this Network is $94\%$

% Histogram of the cosine angle distribution between the gradient in the last step of DF and the perturbation vector obtained by DF.
        % This experiment has been performed on $1000$ images from the CIFAR-$10$~\cite{krizhevsky2009learning} dataset with the ResNet-$18$~\cite{he2016deep} model.

        
% \begin{figure}[t]
% \centering
%     \begin{subfigure}[b]{0.30\columnwidth}
%         % \caption*{$\leq0.01\%$}
%         \includegraphics[width=\linewidth]{photos/Overly_perturbed_DF_S.pdf}
%         \vspace{-2mm}
%         \caption{Overly perturbed analysis of DF.}
%     \end{subfigure}\!
%     \begin{subfigure}[b]{0.31\columnwidth}
%         % \caption*{$(0.01\%,0.1\%)$}
%         \includegraphics[width=\linewidth]{photos/orthogonality-2_S.pdf}
%         \vspace{-2.1mm}
%         \caption{Orthogonality analysis of DF.}
%     \end{subfigure}\!
%     \caption{In \textbf{(a)} we generated 1000 images with one hundred $\gamma$ between zero and one, and the fooling rate of the DF is reported. This experiment is done on the CIFAR10 dataset and ResNet18 model.
%     In \textbf{(b)} histogram of the cosine angle distribution between the gradient in the last step of DF and the perturbation vector obtained by DF.
%         This experiment has been performed on 1000 images from the CIFAR-10 dataset with the ResNet-18 model. }
%     \label{fig:DF_analysis}
% \end{figure}

\begin{figure}[h]
\centering
\begin{tabular}{c c}
% \raisebox{-0.5\height}
%{\includegraphics[width=0.33\linewidth]{photos/Overly_perturbed_DF_S_iclr.pdf}}&
% \raisebox{-0.5\height}
%{\includegraphics[width=0.33\textwidth]{photos/orthogonality-2_S_iclr.pdf}} \\
{\includegraphics[width=0.35\textwidth]{photos/Overly_perturbed_SDF_S_iclr.pdf}} &
{\includegraphics[width=.35\linewidth]{photos/orthogonality_SDF_S_iclr.pdf}}
% \raisebox{-0.5\height}

\end{tabular}

\caption{\textbf{(\textit{Left})} we generated 1000 images with one hundred $\gamma$ between zero and one, and the fooling rate of the DeepFool and SuperDeepFool is reported. This experiment is done on the CIFAR10 dataset and ResNet18 model. \textbf{(\textit{Right})} histogram of the cosine angle between the normal to the decision boundary and the perturbation vector obtained by DeepFool and SuperDeepFool has been showed.\looseness=-1}
\label{fig:DF_analysis}
\end{figure}



\section{Efficient Algorithms to Find Minimal Perturbations}
\label{How-to-Improve-DF}

In this section, we propose a new class of methods that modifies DF to address the aforementioned challenges in the previous section. The goal is to maintain the desired characteristics of DF, i.e., computational efficiency and the fact that it is parameter-free while finding smaller adversarial perturbations. We achieve this by introducing an additional projection step which its goal is to steer the direction of perturbation towards the optimal solution of \eqref{global-formula}.

Let us first briefly recall how DF finds an adversarial perturbations for a classifier $f$. Given the current point $\x_i$, DF updates it according to the following equation:
\begin{align}
    \label{eq:DeepFool-step}
    {\x}_{i+1} = \x_{i} - \frac{f(\x_{i})}{\| \nabla f(\x_i)\|^{2}_{2}}\nabla f(\x_{i}).
\end{align}
Here the gradient is taken w.r.t. the input. The intuition is that, in each iteration, DF finds the minimum perturbation for a linear classifier that approximates the model around $\x_i$.
The below proposition shows that under certain conditions, repeating this update step eventually converges to a point on the decision boundary.

\begin{proposition}
    Let the binary classifier $f:\mathbb{R}^{d} \rightarrow \mathbb{R}$ be continuously differentiable and its gradient $\nabla f$ be $L^{'}$-Lipschitz. For a given input sample $\x_0$, suppose $B(\x_{0},\epsilon)$ is a ball centered around $\x_0$ with radius $\epsilon$, such that there exists $\x\in B(\x_{0},\epsilon)$ that $f(\x)=0$. If $\|\nabla f\|_{2}\geq \zeta$ for all $\x\in B$ and $\epsilon < \frac{\zeta^2}{{L^{'}}^{2}}$, then DF iterations converge to a point on the decision boundary.
\end{proposition}
\textit{Proof:} \textit{We defer the proof to the Appendix}.

Note that while the proposition guarantees the perturbed sample to lie on the decision boundary, it does not state anything about the orthogonality of the perturbation to the decision boundary. Moreover, in practice, DF typically terminates after less than four iterations when an adversarial example is found. As discussed in the previous section, such a solution may not necessarily lie on the decision boundary. 

To find perturbations that are more aligned with the normal to the decision boundary, we introduce an additional projection step that steers the perturbation direction towards the optimal solution of \eqref{global-formula}. Formally, the optimal perturbation, $\boldsymbol{r}^*$, and the normal to the decision boundary at $\x_0+\boldsymbol{r}^*$, $\nabla f(\x_0+\boldsymbol{r}^*)$, should be parallel. Equivalently, $\boldsymbol{r}^*$ should be a solution of the following maximization problem:
\begin{equation}
    \max_{\boldsymbol{r}} \frac{{\boldsymbol{r}}^\top\nabla f(\x_0+\boldsymbol{r})}{\|\nabla f(\x_0+\boldsymbol{r})\| \|\boldsymbol{r}\|},
    \label{eq:proj_max}
\end{equation}
which is the cosine of the angle between $\boldsymbol{r}$ and $\nabla f(\x_0+\boldsymbol{r})$. A necessary condition for $\boldsymbol{r}^*$ to be a solution of \eqref{eq:proj_max} is
that the projection of $\boldsymbol{r}^*$ on the subspace orthogonal to $\nabla f(\x_0+\boldsymbol{r}^*)$ should be zero.
Then, $\boldsymbol{r}^*$ can be seen as a fixed point of the following iterative map:
\begin{equation}
    \boldsymbol{r}_{i+1} = T(\boldsymbol{r}_i)=\frac{{\boldsymbol{r}_i}^\top\nabla f(\x_0+\boldsymbol{r}_i)}{\|\nabla f(\x_0+\boldsymbol{r}_i)\|^2} \nabla f(\x_0+\boldsymbol{r}_i).
    \label{eq:iterative_proj}
\end{equation}

The following proposition shows that this iterative process can find a solution of \eqref{eq:proj_max}. 

\begin{proposition}
For a differentiable $f$ and a given $\boldsymbol{r}_0$, $\boldsymbol{r}_i$ in the iterations \eqref{eq:iterative_proj} either converge to a solution of \eqref{eq:proj_max} or a trivial solution (i.e., $\boldsymbol{r}_i\rightarrow 0$).
\end{proposition}
\textit{Proof:} \textit{We defer the proof to the Appendix.}

% \AlgoDontDisplayBlockMarkers
% \RestyleAlgo{ruled}
% \SetAlgoNoLine
% \LinesNumbered
\begin{algorithm}[t]
    \SetKwFor{RepTimes}{repeat}{times}{end}
	\SetKwFunction{Union}{Union}\SetKwFunction{DeepFool}{\texttt{DeepFool}}\SetKwFunction{l}{\texttt{projection step}}
 
 	\KwIn{image $\x_0$, classifier $f$, $m$, and $n$.}
 	\KwOut{perturbation $\boldsymbol{r}$}

	Initialize: $\x\leftarrow\x_0$
	
    \While{$\sign(f(\x))= \sign(f(\x_{0}))$}
    	 {  
    	 \smallskip
         \RepTimes{$m$}{
         
         
    	    \smallskip               	    
  	    
    	    \smallskip
    	    $\x\gets \x-
         \frac{\left|f(\x)\right|}{\|\nabla f(\x)\|_2^2}\nabla f(\x)$
    	  
    	  \smallskip
    	  }
	  \smallskip
	  \RepTimes{$n$}{
	       $\x \gets \x_0 + \frac{(\x-\x_0)^\top\nabla f(\x)}{\|\nabla f(\x)\|^2} \nabla f(\x)$	  
	  }
    	  }	
 	\KwRet $\boldsymbol{r}=\x-\x_{0}$
\caption{SDF~($m$,$n$) for binary classifiers}
\label{alg:SuperDeepFool}
\end{algorithm}
\subsection{A Family of Adversarial Attacks}
Finding minimum-norm adversarial perturbations can be seen as a multi-objective optimization problem, where we want $f(\x+\boldsymbol{r})=0$ and the perturbation $\boldsymbol{r}$ to be orthogonal to the decision boundary. So far we have seen that DF finds a solution satisfying the former objective and the iterative map \eqref{eq:iterative_proj} can be used to find a solution for the latter. A natural approach to satisfy both objectives is to alternate between these two iterative steps, namely \eqref{eq:DeepFool-step} and \eqref{eq:iterative_proj}. We propose a family of adversarial attack algorithms, coined \textit{SuperDeepFool}, by varying how frequently we alternate between these two steps.
We denote this family of algorithms with \textit{SDF}$(m,n)$, where $m$ is the number of DF steps \eqref{eq:DeepFool-step} followed by $n$ repetition of the projection step \eqref{eq:iterative_proj}. This process is summarized in Algorithm~\ref{alg:SuperDeepFool}.

One interesting case is SDF~$(\infty,1)$ which, in each iteration, continues DF steps till a point on the decision boundary is found and then applies the projection step. This particular case has a resemblance with the strategy used in~\cite{rahmati2020geoda} to find black-box adversarial perturbations. This algorithm can be interpreted as iteratively approximating the decision boundary with a hyperplane and then analytically calculating the minimal adversarial perturbation for a linear classifier for which this hyperplane is the decision boundary.
It is justified by the observation that the decision boundary of state-of-the-art deep networks has a small mean curvature around data samples~\cite{fawzi2017robustness,fawzi2018empirical}.
A geometric illustration of this procedure is shown in Figure~\ref{fig:illus-SDF}.



\subsection{SDF Attack}
We empirically compare the performance of SDF$(m,n)$ for different values of $m$ and $n$ in Section~\ref{sec:exp-sdf}. Interestingly, we observe that we get better attack performance when we apply several DF steps followed by a single projection. Since the standard DF typically finds an adversarial example in less than four iterations for state-of-the-art image classifiers, 
one possibility is to continue DF steps till an adversarial example is found and then apply a single projection step. We simply call this particular version of our algorithm SDF, which we will extensively evaluate in Section~\ref{sec:experiments}.

SDF can be understood as a generic algorithm that can also work for the multi-class case by simply substituting the first inner loop of Algorithm~\ref{alg:SuperDeepFool} with the standard multi-class DF algorithm. The label of the obtained adversarial example determines the boundary on which the projection step will be performed. A summary of multi-class SDF is presented in Algorithm~\ref{alg:SuperDeepFool-multi}. Compared to the standard DF, this algorithm has an additional projection step. We will see later that such a simple modification leads to significantly smaller perturbations.


\begin{figure}
\center
\includegraphics[width=0.45\textwidth]{photos/SuperDF_illus_S.pdf}
\caption{\label{fig:illus-SDF}Illustration of two iterations of the SDF($\infty$,1) algorithm. Here $\x_0$ is the original data point and $\x_*$ is the minimum-norm adversarial example, that is the closest point on the decision boundary to $\x_0$. $\tilde{\x}_i$ and $\x_i$ indicate the DF and the orthogonal projection steps respectively. The algorithm will eventually converges to $\x_*$.}
\end{figure}



% \AlgoDontDisplayBlockMarkers
% \RestyleAlgo{ruled}
% \SetAlgoNoLine
% \LinesNumbered
\begin{algorithm}[tb]
	\SetKwFunction{Union}{Union}\SetKwFunction{DeepFool}{\texttt{DeepFool}}\SetKwFunction{l}{\texttt{projection step}}
 	\KwIn{image $\x_0$, classifier $f$.}
 	\KwOut{perturbation $\boldsymbol{r}$}

	Initialize: $\x\leftarrow\x_0$
	
    \While{$\hat{k}(\x)= \hat{k}(\x_{0})$}
    	 {
    	    \smallskip
    		$\widetilde{\x}\leftarrow \DeepFool(\x)$    		
      
    		\smallskip
       
    		$\w\leftarrow\nabla f_{\hat{k}(\widetilde{\x})}(\widetilde{\x}) - \nabla f_{\hat{k}(\x_0)}(\widetilde{\x})$
    		
    		\smallskip

            $\x \gets \x_0 + \frac{(\widetilde{\x}-\x_0)^\top\w}{\|\w\|^2} \w$
    	  }	
 	\KwRet $\boldsymbol{r}=\x-\x_{0}$
\caption{SDF for multi-class classifiers}
\label{alg:SuperDeepFool-multi}
\end{algorithm}


We present in section~\ref{ssec:faces} an application of PnP-HVAE on face images, using a pretrained state-of-the-art hierarchical VAE. 
Next, we study the application of our framework to natural images. To that end, we introduce  in section~\ref{ssec:patchVDVAE}  a patch hierachical VAE architecture, that is able to model natural images of different resolutions. In section~\ref{ssec:app_nat}, we provide deblurring, super-resolution and inpainting experiments to demonstrate the relevance of the proposed method.

Additional results are presented in Appendix~\ref{app:add}. All experiments can be reproduced using the code available at \url{https://github.com/jprost76/PnP-HVAE}.



\subsection{Face Image restoration (FFHQ)}\label{ssec:faces}
We first demonstrate the effectiveness of PnP-HVAE on highly structured data, by performing face image restoration.
Latent variable generative models can accurately model structured images such as face images \cite{karras2019style,vahdat2020nvae,child2021very,kingma2018glow}, and then be used to produce high quality restoration of such data. 
In our experiments, we use the VDVAE model of~\cite{child2021very}, pre-trained on the FFHQ dataset~\cite{karras2019style}, as our hierarchical VAE prior.
VDVAE has $L=66$ latent variable groups in its hierarchy and generates images at resolution $256\times256$.

We compare PnP-HVAE with the intermediate layer optimization algorithm (ILO)~\cite{daras2021intermediate} that is based on a different class of generative models than HVAE. ILO is a GAN inversion method which optimizes the image latent code along with the intermediate layer representation of a StyleGAN to generate an image consistent with a degraded observation.
We use the official implementation of ILO, along with a StyleGAN2 model~\cite{karras2020analyzing, stylegan2pytorch}, that was trained for 550k iterations on images of resolution $256\times256$ from FFHQ.  
As VDVAE and StyleGAN models are not trained on the same train-test split of FFHQ, we chose to evaluate the methods on a subset of 100 images from the CelebA dataset~\cite{liu2018large}. 
For super-resolution, the degradation model corresponds to the application of a gaussian low-pass filter followed by a $\times 4$ sub-sampling, and the addition of a gaussian white noise with $\sigma=3$.
For the deblurring, we considered motion blur and  gaussian kernels, both with a noise level $\sigma=8$. %

We provide quantitative comparisons in table~\ref{table:comp_ILO}, along with a visual comparison of the results in figure~\ref{fig:face_restoration}.
PnP-HVAE has the best  PSNR and SSIM results for all the considered restoration tasks, while ILO provides better results  for the perceptual distance.
By jointly optimizing the image and its latent variable, PnP-HVAE provides  results that are both realistic and consistent with the degraded observation.
On the other hand,  ILO  only optimizes on an extended latent space. This method generates  sharp and realistic images with better LPIPS scores,   
but the results lack  of consistency with respect to the observation, which explains the overall lower PSNR performance. 






\subsection{PatchVDVAE: a HVAE for natural images}\label{ssec:patchVDVAE}
Available generative models in the literature operate on images of  fixed resolutions and
are either restrained to datasets of limited diversity, or even to registered face images~\cite{kingma2018glow,child2021very, vahdat2020nvae, karras2019style}, or requiring additional class information~\cite{brock2018large, dhariwal2021diffusion, song2020score, luhman2022optimizing}.
Fitting an unconditional model on natural images appears to be a more difficult task, as their resolution can change, and their content is highly diverse.
The complexity of the problem can be reduced by learning a prior model on patches of reduced dimension. 
For image restoration problems, the patch model can be reused on images of higher dimensions~\cite{zoran2011learning,prost2021learning,altekruger2022patchnr}. When the model is a full CNN, the prior on the set of the  patches can  be computed efficiently by applying the network on the full image~\cite{prost2021learning}.

We thus introduce  patchVDVAE, a fully convolutional hierarchical VAE.
Contrary to existing HVAE models whose resolution is constrained by the constant tensor at the input of the top-down block, patchVDVAE can generate images of different resolutions by controlling the dimension of the input latent. 
This amounts to defining a prior on patches whose dimension corresponds to the receptive field of the VAE. A similar model is used for image denoising in~\cite{prakash2021interpretable}.

 
For PatchVDVAE architecture, we use the same bottom-up and top-down blocks as VDVAE~\cite{child2021very}, and replace the constant trainable input in the first top-down block by a latent variable, to make the model fully convolutional (details on the  architecture are given in Appendix~\ref{app:details}). 
The training dataset is composed of $128\times 128$ patches extracted from a combination of DIV2K~\cite{agustsson2017ntire} and Flickr2K~\cite{Lim_2017_CVPR_workshops} datasets.
We perform data augmentation by extracting  patches at $3$ resolutions: HR-images and $\times 2$ and $\times 4$ downscaled images. 
The model is trained for $7.10^5$ iterations with a batch size of $64$. Following the recommendation of~\cite{hazami2022efficient}, we use Adamax optimizer with an exponential moving average and gradient smoothing of the variance.
We set the decoder model to be a gaussian with diagonal covariance, as in~\cite{luhman2022optimizing}.
PatchVDVAE is fully convolutional and can generate images of dimension that are multiples of $64$ as illustrated by
figure~\ref{fig:vdvae}.

\newlength{\patchwidth}
\setlength{\patchwidth}{0.135\columnwidth}
\begin{figure}[!ht]
    \centering
    \begin{subfigure}[t]{.34\columnwidth}\hspace{0.1cm}
        \setlength{\tabcolsep}{0.02pt}
\renewcommand{\arraystretch}{0}
        \begin{tabular}{*{2}{p{1.03\patchwidth}}}
            \includegraphics[width=\patchwidth]{figures_arxiv/patchVDVAE/samples/generated/64x64/setup-5-image-0018.png} &
            \includegraphics[width=\patchwidth]{figures_arxiv/patchVDVAE/samples/generated/64x64/setup-5-image-0016.png} \\
            \includegraphics[width=\patchwidth]{figures_arxiv/patchVDVAE/samples/generated/64x64/setup-5-image-0008.png} &
            \includegraphics[width=\patchwidth]{figures_arxiv/patchVDVAE/samples/generated/64x64/setup-5-image-0019.png}   
        \end{tabular}
    \end{subfigure}\hspace{-0.15cm}
    \begin{subfigure}[t]{.64\columnwidth}
\begin{tabular}{cc}\vspace{-0.1cm}
\includegraphics[width=2\patchwidth]{figures_arxiv/patchVDVAE/samples/generated/256x256/setup-2-image-0009.png}&
        \includegraphics[width=2\patchwidth]{figures_arxiv/patchVDVAE/samples/generated/256x256/setup-2-image-0002.png}\end{tabular}

    \end{subfigure}
    \caption{\label{fig:vdvae} Left: $64\times64$ patches samples from our patchVDVAE model trained on patches from natural images.
    Right: PatchVDVAE is fully convolutional and it can generate images of higher resolution (here: $128\times128$).\vspace{-0.2cm}}
\end{figure}

\subsection{Natural images restoration}\label{ssec:app_nat}
We  evaluate PnP-HVAE on natural image restoration.
For each task, we report the average value of the PSNR, the SSIM, and the LPIPS metrics on $20$ images from the test set of the BSD dataset~\cite{MartinFTM01}.\\


\noindent
{\bf Image deblurring.}
In the experiments, we consider $2$ gaussian kernels and $2$ motion blur kernels from~\cite{levin2009understanding}, with $3$ different noise levels 
$\sigma \in \{2.55, 7.65, 12.75\}$.
As a baseline we consider  EPLL~\cite{zoran2011learning}, which learns a prior on image patches with a gaussian mixture model.
We also compare PnP-HVAE  with PnP-MMO and GS-PnP, $2$ competing convergent Plug-and-Play methods based on CNN denoisers.
PnP-MMO~\cite{pesquet2021learning} restricts the denoiser to be contraction in order to guarantee the convergence of the PnP forward-backard algorithm. GS-PnP~\cite{hurault2022gradient} considers a gradient step denoiser and reaches state-of-the-art performances of non converging methods~\cite{zhang2021plug}.
We set the temperature $\tau$  in our method as $0.95$, $0.8$ and $0.6$ for noise levels $2.55$, $7.65$ and $12.75$ respectively, and we let it run for a maximum of $50$ iterations. 
For the three compared methods we use the official implementations and pre-trained models provided by the respective authors. 
Details on the choice of hyperparameters for the concurrent methods are provided in the Appendix~\ref{app:details}
Figure~\ref{fig:deblurring_bsd} illustrates that our method provides correct deblurring results. 

According to table~\ref{tab:deb}, the performance of PnP-HVAE is between those of EPLL and GS-PnP and it outperforms PnP-MMO for large noise levels.\\

\begin{table}
\begin{center}\footnotesize
    \begin{tabular}{>{\centering}m{.3cm}*{5}{c}}
    $\sigma$ &Method & PSNR$\uparrow$ & SSIM$\uparrow$ & LPIPS$\downarrow$  \\ 
    \hline
    \multirow{4}{*}{\vcell{$2.55$}}
    & PnP-HVAE & $27.75$ & $0.79$ & $0.31$\\
    & GS-PNP \cite{hurault2022gradient} & $\mathbf{29.59}$ & $\mathbf{0.84}$ & $\mathbf{0.22}$\\
    & EPLL \cite{zoran2011learning} & $26.49$ & $0.71$ & $0.36$\\ 
    & PnP-MMO \cite{pesquet2021learning} & $\underbar{29.50}$ & $\underbar{0.83}$ & $\underbar{0.20}$ \\ \hline
    \multirow{4}{*}{\vcell{$7.65$}}
    & PnP-HVAE & $\underbar{26.36}$ & $\underbar{0.72}$ & $\underbar{0.40}$\\
    & GS-PNP \cite{hurault2022gradient} & $\mathbf{27.33}$ & $\mathbf{0.77}$ & $\mathbf{0.31}$\\
    & EPLL \cite{zoran2011learning} & $24.04$ & $0.66$ & $0.45$ \\ 
    & PnP-MMO \cite{pesquet2021learning} & $25.34$ & $0.69$ & $0.34$\\
    \hline
    \multirow{4}{*}{\vcell{$12.75$}}
    & PnP-HVAE & $\underbar{25.12}$ & $\mathbf{0.73}$ & $\underbar{0.47}$\\
    & GS-PNP \cite{hurault2022gradient} & $\mathbf{26.32}$ & $\mathbf{0.73}$ & $\mathbf{0.37}$\\
    & EPLL \cite{zoran2011learning} & $23.28$ & $0.61$ & $0.51$ \\ 
    & PnP-MMO \cite{pesquet2021learning} & $22.42$ & $0.53$& $0.54$ \\
    \hline
    &\vspace*{-.3cm}\\
            \multicolumn{2}{c}{Blur and motion kernels}& \multicolumn{3}{c}{
        \includegraphics*[scale=1]{figures_arxiv/kernels/4.png}\;\includegraphics*[scale=1]{figures_arxiv/kernels/7.png}\;\includegraphics*[scale=1]{figures_arxiv/kernels/9.png}\;\includegraphics*[scale=1]{figures_arxiv/kernels/11.png}} 
    \end{tabular}
        \caption{\label{tab:deb}Comparison  of PnP-HVAE  and other restoration methods on deblurring. Results are averaged on $4$ kernels.\vspace{-0.2cm}}% on image deblurring.}
    \end{center}
\end{table}

\begin{figure}
    
    \begin{subfigure}[h]{\linewidth}
        \centering
        \includegraphics*[width=\columnwidth]{figures_arxiv/deb_s255_k7.pdf}\vspace{-0.1cm}
        \caption{Gaussian blur, $\sigma=2.55$}
    \end{subfigure}
    \begin{subfigure}[h]{\linewidth}
        \centering
        \includegraphics*[width=\columnwidth]{figures_arxiv/deb_s765_k11.pdf}\vspace{-0.1cm}
        \caption{Motion blur, $\sigma=7.65$}
    \end{subfigure}\vspace*{-0.1cm}
    \caption{\label{fig:deblurring_bsd} Natural image deblurring\vspace{-0.1cm}}
\end{figure}

\noindent {\bf Effect of the temperature.}
PnP-HVAE gives control on the temperature of the prior over the latent space.
In figure~\ref{fig:temp_effect}, we illustrate that reducing the temperature increases the strength of the regularization prior. In this example the tuning $\tau=0.7$ produces the best performance.\\
\begin{figure}[!ht]
   
    \includegraphics[width=\columnwidth]{figures_arxiv/demo_temp.pdf}\vspace{-0.15cm}
    \caption{ \label{fig:temp_effect} Effect of the temperature in PnP-VAE on a deblurring problem, with $\sigma=7.65$.\vspace{-0.15cm}}
\end{figure}


\noindent
{\bf Image inpainting.}
Next we consider the task of noisy image inpainting. 
We compose a test-set of 10 images from the validation set of BSD~\cite{MartinFTM01} and we create masks
  by occluding diverse objects of small size in the images. 
A gaussian white noise with $\sigma=3$ is added to the images.
As a comparaison, we still consider GS-PnP and EPLL.
For PnP-HVAE, the temperature is set to $\tau=0.6$, and the algorithm is run for a maximum of $200$ iterations, unless the residual $||\x_{k+1}-\x_k||$ is on a plateau.
We provide on Table~\ref{tab:inpainting_bsd} the distortion metrics with the ground truth, as well as a visual
\begin{table}



\begin{center}
    \begin{tabular}{cccc}
        & PSNR$\uparrow$ & SSIM$\uparrow$ &LPIPS$\downarrow$ \\\hline
        PnP-HVAE  & $\mathbf{29.54}$ & $\mathbf{0.93}$ & $\mathbf{0.06}$\\
        GS-PNP & $28.52$ & $\mathbf{0.93}$ & $0.09$\\
        EPLL & $\underline{29.16}$ & $\mathbf{0.93}$ & $\mathbf{0.06}$\\
    \end{tabular}
    \caption{\label{tab:inpainting_bsd}Quantitative evaluation for inpainting on BSD.}
    \end{center}
\end{table}
comparison on figure~\ref{fig:inpainting_bsd}. 
With its hierarchical structure,  PnP-HVAE outperforms the compared methods. \vspace{0.05cm}



\begin{figure}[!h]
    \includegraphics[width=\columnwidth]{figures_arxiv/demo_inp_bsd2.pdf}\vspace{-0.1cm}
    \caption{\label{fig:inpainting_bsd}Natural image inpainting\vspace{-0.3cm}}
\end{figure}











%\section{}
%\label{sec:resDir}


\section{Conclusion}
\label{sec:conclusion}
% <>
Since its advent in 1931, Koopman operator theory \cite{koopman:1931} has only recently been actively utilized for solving practical problems, thanks to the introduction of the DMD algorithm in 2008 \cite{schmid:2008}. Since then, a multitude of DMD algorithm variations have risen to prominence and found utility across various fields. A notable feature of our survey paper was reviewing and categorizing the results of over 100 research papers based on both application and algorithm type in smart mobility and vehicle engineering  (see Table~\ref{tab1} and Section~\ref{sec:vehicApp}).  Additionally, this survey paper identified potential research gaps in smart mobility and vehicular engineering applications (Remarks~\ref{remGap1}--\ref{remGap6}). Finally, this review paper discussed theoretical aspects of Koopman operator theory that have been largely neglected by the smart mobility and vehicle engineering community and yet have large potential for contributing to solving open problems in these areas (see Section~\ref{subsec:theorIssue}).

\noindent{\textbf{Future Research Directions.}}	Given the emergence of cyber-threats against connected and autonomous vehicles as well as robotic systems (see, e.g.,~\cite{nekouei2021randomized,mohammadi2022generation}), a future research direction might include utilizing Koopman operator-based algorithms for designing cyber-resilient vehicular and smart mobility applications (see, e.g.,~\cite{taheri2022data} for a related line of research). Another potential research direction is using Koopman operator-based algorithms for predicting the motion of vulnerable road users (VRUs), e.g., pedestrians and cyclists (see, e.g.,~\cite{pool2019context,scholler2020constant}). Finally, rehabilitation robotics and robotic exoskeletons can be the benefactors of the predictive capabilities of Koopman operator-based algorithms for detecting tripping events and/or system  identification in various modes of locomotion (see, e.g.,~\cite{kumar2019extremum,aprigliano2019pre}).



%Fig. 1 depicts the accumulation of such algorithms since 2014, which are particular to vehicle engineering and smart mobility, i.e., the focus of this review. Table 1 summarizes the varieties of relevant algorithms developed in those studies. Furthermore, we have highlighted theoretical issues, whose expansion will have potential applications to the wide research area of smart mobility and vehicle engineering.  

%Although fairly comprehensive, we have found several gaps in this research area. In particular, we could not find any studies related to elevators, robots/vehicles employing crawling, slithering, hopping or peristaltic locomotion, arctic or special-terrain vehicles such as those employing screws or tracks, hovercraft and other amphibious vehicles or subsystems which tolerate flexible environments, classification or guidance systems related to vehicles for drilling or agriculture, or for current-ripple, power-split, battery health monitoring, nuclear propulsion, exoskeletons/prosthetics, personal mobility, motorsports, specialized rovers or similar open problems in emerging areas.  These examples are, of course, not exhaustive.  
%
%The purely data-driven nature of Koopman operators holds the promise of capturing unknown and complex dynamics for reduced-order model generation and system identification, through which the rich machinery of linear control techniques can be utilized. The emergent nature of the smart mobility and vehicular-related applications, where  the Koopman operator  in each particular application needs to be approximated, implies that the development of various Koopman operator approximation  algorithms is expected to grow along with the vehicular problems they aim to solve.  Given the ongoing development of this research area and the many existing open problems in the fields of smart mobility and vehicle engineering, a survey of techniques and open challenges of applying Koopman operator theory to this vibrant area is warranted.  To the best of our knowledge, this survey paper is the \emph{first of its kind} reviewing the applications of Koopman operator theory within a focused research area, namely, smart mobility and vehicle engineering applications. A \emph{notable feature} of our survey paper is reviewing and categorizing the results of over 100 research papers based on both application and algorithm type  (see Tables~\ref{tab1}--~\ref{tab4} and Section~\ref{sec:vehicApp}) that are concerned with the applications of Koopman operator theory to the field of smart mobility and vehicular engineering. Such a \emph{comprehensive and  detailed categorization} will be beneficial to the research practitioners working in the field.  Furthermore, this review paper discusses theoretical aspects of Koopman operator theory that have been largely neglected by the smart mobility and vehicle engineering community and yet have large potential for contributing to solving open problems in these areas. Additionally, our survey paper seeks to \emph{identify gaps} in the smart mobility and vehicle engineering research where new and existing Koopman operator-based methods have the potential to further develop and address unsolved problems  potentially benefiting from the perspectives of nonlinear system identification, control, global linearization, and the predictive powers that Koopman operator theory has to offer (see, e.g., Remarks~\ref{remGap1}--\ref{remGap6}). 




% \clearpage
% \newpage
{\small
\bibliographystyle{icml2023}
\bibliography{egbib}
}
\clearpage
\newpage
\section{Appendix for Proofs}

\paragraph{Proof of Theorem \ref{thm:main}.}

\begin{proof}
\label{proof:main}
Our proof has two steps. In Step 1, we will show that SimCLR is equivalent to minimizing the cross entropy loss defined in Eqn.~(\ref{eqn:cross-entropy}). 
In Step 2, we will show  that minimizing the cross-entropy loss 
is equivalent to spectral clustering on $\bfpi$. 
Combining the two steps together, we have proved our theorem. 

\textbf{Step 1: } SimCLR is equivalent to minimizing the cross entropy loss.

The cross-entropy loss takes expectation over 
$\bfW_\bfX\sim \mathbb{P}(\cdot ; \bfpi)$, 
which means $\bfW_\bfX$ has exactly one non-zero entry in each row $i$. By Lemma~\ref{lem:multinomial}, we know every row $i$ of $\bfW_\bfX$ is independent of other rows. Moreover, 
$\bfW_{\bfX,i}\sim \mathcal{M}(1, \bfpi_i/\sum_j \bfpi_{i,j})=\mathcal{M}(1, \bfpi_i)$, because $\bfpi_i$ itself is a probability distribution.
Similarly, we know $\bfW_\bfZ$ also has the row-independent property by sampling over $\mathbb{P}(\cdot;\bfK_\bfZ)$.
Therefore, by Lemma~\ref{lem:cross_split}, we know Eqn.~(\ref{eqn:cross-entropy}) is equivalent to:
\[
 -\sum_{i=1}^n \mathbb{E}_{\bfW_{\bfX,i}}[\log \mathbb{P}(\bfW_{\bfZ,i}=\bfW_{\bfX,i};\bfK_\bfZ)],
\]

This expression takes expectation over $\bfW_{\bfX,i}$ for the given row $i$. Notice that 
$\bfW_{\bfX,i}$ has exactly one non-zero entry, which equals $1$ (same for $\bfW_{\bfZ,i}$). 
As a result
we expand the above expression to be:
\begin{equation}
 -\sum_{i=1}^n \sum_{j\neq i} \Pr(\bfW_{\bfX,i,j}=1)\log \Pr(\bfW_{\bfZ,i,j}=1).
\label{eqn:detailed-expansion}    
\end{equation}


By Lemma~\ref{lem:multinomial}, $\Pr(\bfW_{\bfZ,i,j}=1)=\bfK_{\bfZ,i,j}/\|\bfK_{\bfZ,i}\|_1$ for $j\neq i$. Recall that $\bfK_\bfZ=(k(\bfZ_i-\bfZ_j))_{(i,j)\in[n]^2}$, which means 
$\bfK_{\bfZ,i,j}/\|\bfK_{\bfZ,i}\|_1=\frac{\exp(-\|\bfZ_i-\bfZ_j\|^2/{2\tau})}{\sum_{k\neq i}
\exp(-\|\bfZ_i-\bfZ_k\|^2/{2\tau})
}$ for $j\neq i$, when $k$ is the Gaussian kernel with variance $\tau$. 

Notice that $\bfZ_i=f(\bfX_i)$, so we know
\begin{equation}
-\log \Pr(\bfW_{\bfZ,i,j}=1)=
-\log \frac{\exp(-\|f(\bfX_i)-f(\bfX_j)\|^2/{2\tau})}{\sum_{k\neq i}
\exp(-\|f(\bfX_i)-f(\bfX_k)\|^2/{2\tau}),
}
\label{eqn:infonce-equivalence}    
\end{equation}


The right hand side is exactly the InfoNCE loss defined in Eqn.~(\ref{eqn:infonce}).
Inserting Eqn.~(\ref{eqn:infonce-equivalence}) into Eqn.~(\ref{eqn:detailed-expansion}), we get the SimCLR algorithm, which first samples augmentation pairs $(i,j)$ with $\Pr(\bfW_{\bfX,i,j}=1)$ for each row $i$, and then optimize the InfoNCE loss. 

\textbf{Step 2: } minimizing the cross entropy loss 
is equivalent to spectral clustering on $\bfpi$.


By Lemma~\ref{lem:convert_to_spectral}, we may further convert the loss to 
\begin{equation}
\label{eqn:main-theorem-repul-attr}
\min_{\bfZ}
-\sum_{(i,j)\in [n]^2} \mathbf{P}_{i,j}
\log k (\bfZ_i-\bfZ_j)+\log \mathbf{R}(\bfZ).
\end{equation}
Since $k$ is the Gaussian kernel, this reduces to \[
\min_\bfZ \mathrm{tr}(\bfZ^\top \mathbf{L}(\bfpi) \bfZ)
+\log \mathbf{R}(\bfZ),
\]

where we use the fact that $\mathbb{E}_{\bfW_\bfX\sim \mathbb{P}(\cdot; \bfpi)}[\mathbf{L}(\bfW_\bfX)]
=\mathbf{L}(\bfpi)
$, because the Laplacian operator is linear and $
\mathbb{E}_{\bfW_\bfX\sim \mathbb{P}(\cdot; \bfpi)}(\bfW_\bfX)=\bfpi
$.
\end{proof}

\paragraph{Proof of Theorem \ref{thm:clip}.}
\begin{proof}
Since $\bfW_\bfX\sim \mathbb{P}(\cdot;\bfpi_{\mathbf{A}, \mathbf{B}})$, we know 
$\bfW_\bfX$ has exactly one non-zero entry in each row, denoting the pair that got sampled. 
A notable difference compared to the previous proof is we now have $n_\mathcal{A}+n_\mathcal{B}$ objects in our graph. CLIP deals with this by taking a mini-batch of size $2N$, 
such that $n_\mathcal{A}=n_\mathcal{B}=N$, and adding the $2N$ InfoNCE losses together. We label the objects in $\mathcal{A}$ as $[n_\mathcal{A}]$, and the objects in $\mathcal{B}$ as $\{n_\mathcal{A}+1, \cdots, n_\mathcal{A}+n_\mathcal{B}\}$. 

Notice that $\bfpi_{\mathbf{A}, \mathbf{B}}$ is a bipartite graph, so the edges of objects in $\mathcal{A}$ will only connect to object in $\mathcal{B}$ and vice versa. We can define the similarity matrix in $\cZ$ as $\bfK_\bfZ$, 
where $\bfK_\bfZ(i, j+n_\mathcal{A})=\bfK_\bfZ(j+n_\mathcal{A},i)= k(\bfZ_i-\bfZ_j)$ for $i\in [n_\mathcal{A}], j\in [n_\mathcal{B}]$, and otherwise we set $\bfK_\bfZ(i,j)=0$. 
The rest is same as the previous proof. 
\end{proof}

\paragraph{Proof of Theorem \ref{thm:exponential}.}

\begin{proof}
\label{proof:exponential}
Since the objective function consists of a linear term combined with an entropy regularization, which is a strongly concave function, the maximization problem is a convex optimization problem. Owing to the implicit constraints provided by the entropy function, the problem is equivalent to having only the equality constraint. We then introduce the Lagrangian multiplier $\lambda$ and obtain the following relaxed problem:

$$
\widetilde{E}(\boldsymbol{\alpha})=\psi_{1}-\sum_{i=1}^n \alpha_{i} \psi_{i}+\tau \sum_{i=1}^n \alpha_{i}\log \alpha_{i}+\lambda\left(\boldsymbol{\alpha}^{\top} \mathbf{1}_n-1\right).
$$

As the relaxed problem is unconstrained, taking the derivative with respect to $\alpha_{i}$ yields

$$
\frac{\partial \widetilde{E}(\boldsymbol{\alpha})}{\partial \alpha_{i}}=-\psi_{i}+\tau\left(\log \alpha_{i}+\alpha_{i} \frac{1}{\alpha_{i}}\right)+\lambda=0.
$$

Solving the above equation implies that $\alpha_{i}$ takes the form
$
\alpha_{i}=\exp \left(\frac{1}{\tau} \psi_{i}\right) \exp \left(\frac{-\lambda}{\tau}-1\right).
$ Since $\alpha_{i}$ lies on the probability simplex, the optimal $\alpha_{i}$ is explicitly given by
$
\alpha^{*}_{i}=\frac{\exp \left(\frac{1}{\tau} \psi_{i}\right)}{\sum_{i^{\prime}=1}^n \exp \left(\frac{1}{\tau} \psi_{i^{\prime}}\right)} .
$ Substituting the optimal point into the objective function, we obtain
$$
\begin{aligned}
E\left(\boldsymbol{\alpha}^*\right)  &=\psi_1-\sum_{i=1}^n \frac{\exp \left(\frac{1}{\tau} \psi_{i}\right)}{\sum_{i^{\prime}=1}^n \exp \left(\frac{1}{\tau} \psi_{i^{\prime}}\right)} \psi_{i}+\tau \sum_{i=1}^n \frac{\exp \left(\frac{1}{\tau} \psi_{i}\right)}{\sum_{i^{\prime}=1}^n \exp \left(\frac{1}{\tau} \psi_{i^{\prime}}\right)}\log \frac{\exp \left(\frac{1}{\tau} \psi_{i}\right)}{\sum_{i^{\prime}=1}^n \exp \left(\frac{1}{\tau} \psi_{i^{\prime}}\right)} \\
& =\psi_1 - \tau \log \left(\sum_{i=1}^n \exp \left(\frac{1}{\tau} \psi_{i}\right)\right).
\end{aligned}
$$
Thus, the Lagrangian dual function is given by
\begin{equation*}
-E\left(\boldsymbol{\alpha}^*\right)= -\tau \log \frac{\exp \left(\frac{1}{\tau} \psi_{1}\right)}{\sum_{i=1}^n \exp \left(\frac{1}{\tau} \psi_{i}\right)}.\qedhere
\end{equation*}
\end{proof}



\section{More on Experiments} \label{section: experiment_details}

\paragraph{CIFAR-10 and CIFAR-100} CIFAR-10 ~\citep{krizhevsky2009learning} and CIFAR-100 ~\citep{krizhevsky2009learning} are well-known classic image classification datasets. Both CIFAR-10 and CIFAR-100 contain a total of 60k $32 \times 32$ labeled images of different classes, with 50k for training and 10k for testing. CIFAR-10 is similar to CIFAR-100, except there are 10 different classes in CIFAR-10 and 100 classes in CIFAR-100.

\paragraph{TinyImageNet} TinyImageNet ~\citep{le2015tiny} is a subset of ImageNet ~\citep{deng2009imagenet}. There are 200 different object classes in TinyImageNet, with 500 training images, 50 validation images, and 50 test images for each class. All the images in TinyImageNet are colored and labeled with a size of $64 \times 64$.

\textbf{Pseudo-code.} Algorithm \ref{alg:Training Procedure} presents the pseudo-code for our empirical training procedure.

\begin{algorithm}[!htbp]
\caption{Training Procedure}
\label{alg:Training Procedure}
\begin{algorithmic}[1]
\REQUIRE trainable encoder network $f$, batch size $N$, augmentation strategy \textit{aug}, loss function $L$ with hyperparameters \textit{args}
\FOR {sampled minibatch ${x_i}_{i=1}^N$}
\FORALL{$i \in { 1, ..., N }$}
\STATE draw two augmentations $t_i = \textit{aug}\left(x_i\right) $, $t_i' = \textit{aug}\left(x_i\right) $
\STATE $z_i = f\left(t_i\right)$, $z_i' = f\left(t_i'\right)$
\ENDFOR
\STATE compute loss $\mathcal{L} = L(N, z, z', \textit{args})$
\STATE update encoder network $f$ to minimize $\mathcal{L}$
\ENDFOR
\STATE \textbf{Return} encoder network $f$
\end{algorithmic}
\end{algorithm}

We also provide the pseudo-code for our core loss function used in the training procedure in Algorithm \ref{alg:Core loss}. The pseudo-code is almost identical to SimCLR's loss function, with the exception of an extra parameter $\gamma$.

\begin{algorithm}[!htbp]
\caption{Core loss function $\mathcal{C}$}
\label{alg:Core loss}
\begin{algorithmic}[1]
\REQUIRE batch size $N$, two encoded minibatches $z_1, z_2$, $\gamma$, temperature $\tau$
\STATE $z = \textit{concat}\left(z_1, z_2\right)$
\FOR {$i \in {1, ..., 2N }, j \in {1, ..., 2N}$ }
\STATE $s_{i,j} = \Vert z_i - z_j \Vert_2^{\gamma}$
\ENDFOR
\STATE \textbf{define} $l(i, j)$ \textbf{as} $l(i, j) = - \log \frac{exp\left(s_{i,j}/\tau \right)}{\sum_{k=1}^{2N} \mathbf{1}{[k \ne i]} exp\left(s{i, j} / \tau \right)} $
\STATE \textbf{Return} $\frac{1}{2N} \sum_{k=1}^N\left[l(i, i+N) + l(i+N, i)\right]$
\end{algorithmic}
\end{algorithm}

Utilizing the core loss function $\mathcal{C}$, we can define all kernel loss functions used in our experiments in Table \ref{table: loss definition}. For all $z_i \in z$ with even dimensions $n$, we define $z_{L_i} = z_i\left[0:n/2\right]$ and $z_{R_i} = z_i\left[n/2:n\right]$.

\begin{table}[ht]
\centering
\begin{tabular}{{@{}l|l@{}}}
Kernel  &  Loss function \\ \midrule
Laplacian & $\mathcal{C}\left(N, z, z', \gamma=1, \tau\right)$\\ \midrule
Sum       & $\lambda * \mathcal{C}\left(N, z, z', \gamma=1, \tau_1\right) + (1-\lambda) * \mathcal{C}\left(N, z, z', \gamma=2, \tau_2\right)$  \\ \midrule
Concatenation Sum&$\lambda * \mathcal{C}\left(N, z_L, z'_L, \gamma=1, \tau_1\right) + (1-\lambda) * \mathcal{C}\left(N, z_R, z'_R, \gamma=2, \tau_2\right)$\\ \midrule
$\gamma = 0.5$ & $\mathcal{C}\left(N, z, z', \gamma=0.5, \tau\right)$          \\ 

\end{tabular}

\caption{Definition of kernel loss functions in our experiments}
\label {table: loss definition}
\end{table}

\textbf{Baselines.} We reproduce the SimCLR algorithm using PyTorch Lightning~\citep{PytorchLightning}.

\textbf{Encoder details.}
The encoder $f$ consists of a backbone network and a projection network. We employ ResNet50~\citep{ResNet} as the backbone and a 2-layer MLP (connected by a batch normalization~\citep{ioffe2015batch} layer and a ReLU \cite{nair2010rectified} layer) with hidden dimensions 2048 and output dimensions 128 (or 256 in the concatenation kernel case).

\textbf{Encoder hyperparameter tuning.}
For each encoder training case, we randomly sample 500 hyperparameter groups (sample details are shown in Table \ref{table: Hyperparameter sample}) and train these samples simultaneously using Ray Tune ~\citep{RayTune}, with the ASHA scheduler~\citep{li2018massively}. Ultimately, the hyperparameter group that maximizes the online validation accuracy (integrated in PyTorch Lightning) within 5000 validation steps is chosen for the given encoder training case.

\begin{table}[ht]
\centering

\begin{tabular}{@{}l|l|l@{}}
\midrule
Hyperparameter  & Sample Range & Sample Strategy \\ \midrule
start learning rate & $\left[10^{-2}, 10\right]$ & log uniform \\ \midrule
$\lambda$       & $\left[0, 1\right]$ & uniform \\ \midrule
$\tau$, $\tau_1$, $\tau_2$ & $\left[0, 1\right]$ & log uniform \\ \midrule
\end{tabular}

\caption{Hyperparameters sample strategy}
\label {table: Hyperparameter sample}
\end{table}

\textbf{Encoder training.} 
We train each encoder using the LARS optimizer~\citep{LARSOptimizer}, LambdaLR Scheduler in PyTorch, momentum 0.9, weight decay $10^{-6}$, batch size 256, and the aforementioned hyperparameters for 400 epochs on a single A-100 GPU.

\textbf{Image transformation.} The image transformation strategy, including augmentation, is identical to the default transformation strategy provided by PyTorch Lightning.

\textbf{Linear evaluation.}
The linear head is trained using the SGD optimizer with a cosine learning rate scheduler, batch size 64, and weight decay $10^{-6}$ for 100 epochs. The learning rate starts at $0.3$ and ends at $0$.

\textbf{Moco Experiments.} We also tested our method based on MoCo~\citep{he2019moco}. The results are summarized in Table \ref{tab:results-moco}. Here we choose ResNet18~\citep{ResNet} as the backbone and set a temperature of $0.1$ as default. For our simple sum kernel, we set $\lambda=0.8$. The results show that our method outperforms the original MoCo method.

\begin{table}[thb]
\centering
\caption{MoCo Experiment Results on CIFAR-10 and CIFAR-100.}
\label{tab:results-moco}
\resizebox{\textwidth}{!}{%
\begin{tabular}{@{}c|ccc|ccc@{}}
\toprule
\multirow{3}{*}{Method} & \multicolumn{3}{c|}{CIFAR-10} & \multicolumn{3}{c}{CIFAR-100} \\ \cmidrule(lr){2-4} \cmidrule(lr){5-7} 
                        & 200 epochs & 400 epochs    & 1000 epochs   & 200 epochs & 400 epochs & 1000 epochs         \\ \midrule
MoCo (repro.)         & $76.41 \pm 0.12$    & $80.01 \pm 0.15$          & $84.45 \pm 0.08$    & $\mathbf{47.02 \pm 0.11}$ & $52.50 \pm 0.07$ & $57.62 \pm 0.15$            \\
\midrule
Laplacian Kernel        & ${78.09 \pm 0.10}$    & $\mathbf{83.85 \pm 0.09}$          & $\mathbf{88.34 \pm 0.16}$    & $46.12 \pm 0.22$   & $53.44 \pm 0.17$ & $59.10 \pm 0.14$        \\
Simple Sum Kernel & $\mathbf{78.12 \pm 0.15}$   & $83.23 \pm 0.18$ & $87.50 \pm 0.20$ & $46.65 \pm 0.06$ & $\mathbf{53.62 \pm 0.19}$ & $\mathbf{59.83 \pm 0.12}$\\
\bottomrule
\end{tabular}
}
\end{table}



\section{More Experiments on Synthetic Data}


Consider a scenario with $n$ clusters, each containing $k$ vertices. Let the probability of vertices $u$ and $v$ from the same cluster belonging to $\bfpi$ be $p$. Conversely, for vertices $u$ and $v$ from different clusters, let the probability of belonging to $\pi$ be $q$. We generate the graph $\bfpi$ randomly, based on $p$ and $q$. We experiment with values of $k=100$ and $n=6$ for ease of visualization, embedding all points in a two-dimensional space. Each vertex's initial position originates from a normal distribution. In each iteration, we sample a subgraph of $\bfpi$ uniformly, ensuring each vertex has an out-degree of $1$. We then optimize the corresponding vectors using InfoNCE loss with an SGD optimizer and iterate until convergence. Our experimental setup consists of an SGD learning rate of $1$, an InfoNCE loss temperature of $0.5$, and a batch size of $50$. We evaluate two scenarios with different $p$ and $q$ values: $p=1$, $q=0$, and $p=0.75$, $q=0.2$. The results of these experiments are visualized in Figure \ref{fig:vis-spectral-cluster}. The obtained embeddings exhibit the hallmark pattern of spectral clustering of graph $\bfpi$.

\begin{figure}[!tb]
\centering
\subfigure{
\includegraphics[width=1\textwidth]{Figures/cluster_pi.png}
\label{fig:vis-cluster}
}
\subfigure{
\includegraphics[width=1\textwidth]{Figures/noised_cluster_pi.png}
\label{fig:vis-noised-cluster}
}
\caption{Visualizations of the optimization process using InfoNCE Loss on the vectors corresponding to $\bfpi$. Points of identical color belong to the same cluster within $\bfpi$. To showcase the internal structure of $\bfpi$, we randomly select 10 vertices from each cluster to display the edge distribution of $\bfpi$.}
\label{fig:vis-spectral-cluster}
\end{figure}


\onecolumn
\end{document}
