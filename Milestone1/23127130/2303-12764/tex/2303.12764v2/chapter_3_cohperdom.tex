%}!TEX root = draft.tex
\setenumerate[0]{leftmargin=15mm,itemsep=\the\smallskipamount}
\section{Cohomological computations}\label{s:CoPerDom}
%\todo[inline]{\textbf{TODO}: Recall the beginning of 1.3 in \cite{OS}}
\subsection{Setup} \label{s:Setup}
Let $K=\BQ_p$. In order to apply the results of \cite{OSch, OSt} and ideas of \cite{MR}, we consider $\sF^{\wa}:=\sF^{\wa}(\bG, \{\mu \}, 1)$ with 
 $\bG$ a split connected reductive group over $K$ and $\{\mu\} \subset X_*(\bG)$ such that $\bB:=\bP(\mu)$ is a Borel subgroup of $\bG_E$. \\

This entails a lot of simplifications. Notice first that $1 \in [1]$ is $1$-decent as $\nu:=\nu_1$ is trivial. Furthermore, $\bJ:=\bJ_1=\bG$ by \cite[Remark 9.5.9]{DOR} and $E=K$ since the action of $\Gamma_{K}$ is trivial on $\{\mu\}$. Hence, $\sF$ and $\sF^{\wa}$ are defined over $K$. \\

We set $n:=\dim \sF$ and choose a uniformizer $\pi$ of $K$. Further, we fix an $\inva$ on $\bG$ (cf. section \ref{s:rootdatum}).
We choose a split maximal torus $\bT$ of $\bG$ of rank $d$ such that $\mu \in X_*(\bT)_\BQ$. 
Since all Borel subgroups over $K$ of $\bG$ are $\bG(K)$-conjugated (cf. \cite[Theorem 20.9]{Bor}), we can assume that $(\bT,\bB)$ is a Borel pair (cf. section \ref{s:rootdatum}). 
 This gives rise to a set of simple roots (cf. section \ref{s:rootdatum})
    $$\Delta:=\{\alpha_1, \ldots , \alpha_d\} \subset X^*(\bT)_\BQ.$$
% to which we associate the dual basis 
% $$\{\varpi_\alpha \mid \alpha \in \Delta \} \subset X_*(\bT)_\BQ$$
% i.e $\langle \beta, \varpi_\alpha \rangle = \delta_{\alpha,\beta}$ for all $\alpha, \beta \in \Delta$. 
% Then $\bP(\varpi_\alpha)$ are the maximal $\BQ_p$-parabolic subgroups of $\bG$ containing $\bB$.
% For $I \subset \Delta$ they define by 
% $$\bP_I:= \bigcap_{\alpha \not\in I} \bP(\varpi_\alpha)$$ 
% a standard-parabolic subgroup of $\bG$ defined over $\BQ_p$. \\yw
After conjugating $\mu$ with an element of $W$, if necessary, we can assume that $\mu$ lies in the positive Weyl chamber with respect to $\bB$, i.e. 
\begin{equation}\label{positiveChamber}
\langle \alpha, \mu \rangle >0
\end{equation}
for all $\alpha \in \Delta$ (here we used that $\bP(\mu)=\bB$ to get $>$).  
Notice that since $\bG$ is split over $K$ we have $\Gamma_\mu=\Gamma_K$, so $\ov\mu=\mu$. By Theorem \ref{necessary}, we assume that $[1] \in A(\bG, \{\mu\})$, i.e. 
\begin{equation}\label{existence}
\mu=\mu-\nu=\sum_{\alpha \in \Delta} n_\alpha \alpha^{\vee}
\end{equation}
with $n_\alpha \in \BQ_{\geq 0}$. 

\begin{lemma}\label{n_a} For $\mu= \sum_{\alpha \in \Delta} n_\alpha \alpha^{\vee}$, we have $n_\alpha \in \BQ_{> 0}.$ 
\end{lemma}

\begin{proof} 
    Let $\beta \in \Delta$ and consider 
    $$0 < \langle  \beta, \mu \rangle = \sum_{\alpha \in \Delta} n_\alpha \langle \beta , \alpha^{\vee} \rangle = 2n_\beta+   \sum_{\alpha \in \Delta\backslash \{\beta\} } n_\alpha \langle \beta, \alpha^{\vee} \rangle  \leq 2n_\beta$$
    where we used that $ \langle \beta ,  \alpha^{\vee}  \rangle \leq 0$ for all simple roots $\alpha \neq \beta$ (cf. Lemma \ref{rootrel}). 
    % implies that $\sum_{\substack{\alpha \in \Delta \\ \alpha \neq \beta }} n_\alpha \langle \alpha^{\vee}, \beta \rangle=0$. Thus  
   % $n_\alpha=0$ if $\langle \alpha^{\vee}, \beta \rangle \neq 0$ i.e for all adjacent nodes $\alpha$ of $\beta$ in the Dynkin diagram $D$ associated to $\Phi$ with respect to $\Delta$. Repeating the argument we see
    %that $n_\alpha=0$ for all $\alpha$ in the irreducible component of $D$ containing $\beta$. %Therefore we can assume that $n_\alpha \in \BQ_{>0}$ for all $\alpha \in \Delta$.
\end{proof} 
For a weight $\lambda \in X^{*}(\bT)$, let $\CL_\lambda$ be the sheaf on $\sF$ with 
\begin{equation}\label{linebundle}
    \CL_\lambda(U)=\Big\{f \in \CO_{\bG}(\pi^{-1}(U)) \mid f(gb)=-\lambda(b)f(g) \text{ for all } g \in \bG(\ov{K}), b \in \bB(\ov{K} )\Big\}
\end{equation}
for $U \subset \sF$ open (cf. \cite[Part I, 5.8]{J}; note the sign in the definition). 
Here,  $\pi:\bG\rightarrow \sF$ is the natural projection. It is a locally free sheaf of rank 1 (cf. \cite[Part II, 4.1]{J}). For example, $\CL_{2\rho}=\omega_{\sF}$.  We fix a dominant $\lambda \in X^*(\bT)^+$ and set $\CE_\lambda :=\CL_\lambda \otimes \omega_{\sF}.$ 
\subsection{Geometrical properties of the complement of $\sF^{\wa}$}\label{s:GeoProp}  Following \cite[Section 3, p. 536]{O1}, each $\tau \in X_*(\bG)_\BQ$ defines a closed subvariety of $\sF$ by setting $Y_\tau:=\{x \in \sF \mid \mu^{\sL}(x, \tau)<0 \}.$ Here $\mu^{\sL}(\, , \,)$ is the slope function (cf. \cite[Definition 2.2]{M}). Then, for $I \subsetneq \Delta$, we set
\begin{equation*}
    Y_I:= \bigcap_{\alpha \not\in I} Y_{\varpi_\alpha}
\end{equation*}
which is again a closed subvariety of $\sF$.

\begin{lemma}\cite[Lemma 3.1]{O1} Let $I \subsetneq \Delta$. The variety $Y_I$ is defined over $K$. The natural action of $\bG(K)$  on $\sF$ restricts to an action of $\bP_I(K)$ on $Y_I$.
\end{lemma}
Let $Y:=\sF^{\ad} \backslash \sF^{\wa}$. For $I \subsetneq \Delta$ and any subset $W \subset \bG/\bP_I(K)$, we set
$$Z_I^W:=\bigcup_{g \in W} gY_I^{\ad}$$ 
which, in view of the previous lemma, is indeed well-defined. 
\begin{lemma}\cite[Lemma 3.2]{O1} The subset $Z_I^W$ is a closed pseudo-adic subspace of $\sF^{\ad}$ for every compact open subset $W \subset \bG/\bP_I(K)$.
\end{lemma}
Then, by \cite[Corollary 2.4]{O1}, we have the following identification
$$Y=\bigcup_{\substack{I \subset \Delta \\ \lvert \Delta \backslash I \rvert =1}} Z_I^{\bG/\bP_I(K)}.$$ 
For an alternative description of the $Y_I$, which will be important hereinafter, we set 
 \begin{equation}\label{omegaI}
    \Omega_I:=\{w \in W \mid (w\mu, \varpi_\alpha)> 0 \text{ for all } \alpha \not\in I  \}
 \end{equation} 
 for $I \subsetneq \Delta$ (cf. \cite[p. 530]{O1}). Reformulating Lemma \ref{equivalentcond}, we get the following statement. 
 \begin{lemma}\label{fweightandpairing} 
Let $I \subsetneq \Delta $. Then, $w \in \Omega_I$ if and only if $\langle \check{\varpi}_\alpha, w\mu\rangle_\der >0$ for all $\alpha \not\in I$. 
 \end{lemma} 

 \begin{definition}\label{defC_Iw}Let $I \subset \Delta$ and $w \in W^I$. 
    The \textit{generalized Schubert cell} in $\sF$ associated to $w$ is 
         $$C_I(w):=\bP_Iw\bB/\bB=\bigcup_{v \in W_I}C(vw).$$ 
    If $I=\emptyset$, we omit the subscript and call it Schubert cell. 
 \end{definition}


First, it turns out that the $Y_I$ are a union of Schubert cells. 

\begin{proposition}\cite[Proposition 4.1]{O1} For $I \subsetneq \Delta$, we have 
    $$Y_I= \bigcup_{w \in \Omega_I} C(w).$$
\end{proposition}

But for our purposes, we need a description in terms of generalized Schubert cells.  

\begin{proposition}\label{genSchCe} For $I \subsetneq \Delta$, we have 
    $$Y_I= \bigcup_{w \in W^I \cap \, \Omega_I} C_I(w).$$
\end{proposition}

\begin{proof}
    We know by \cite[Proposition 11.1.6]{DOR} that $Y_I= \bigcup_{w \in \Omega_I} \bP_Iw\bB/\bB$. 
    For $w' \in \Omega_I$ exist unique $w \in W^I$ and $v \in W_I$ such that $w'=vw$ and $l(w')=l(v)+l(w)$. Hence, we have 
    $$  \bP_Iw'\bB/\bB=\bP_Ivw\bB/\bB=\bP_Iw\bB/\bB.$$ 
    Since $Y_I$ is closed, this implies that $$Y_I= \bigcup_{w \in W^I \cap \, \Omega_I} C_I(w).\qedhere $$ 
\end{proof}
    % Let $w \in W$ and $v \in W_I$ with reduced expression $v=s_1\ldots s_r$  where $s_i \in \{s_{\alpha}\}_{\alpha \in I}$. 
    % By \cite[Chapter VI, 1.10]{Bo}, we have $s_i\check{\varpi}_\beta=\check{\varpi}_\beta$ for $\beta \in \Delta \backslash I$ and all $i$. Hence, we see that 
    % \begin{equation*}
    % \langle \check{\varpi}_\beta, vw\mu  \rangle_\der =\langle s_1\check{\varpi}_\beta, s_2\ldots s_rw\mu \rangle_\der=\langle \check{\varpi}_\beta, s_2\ldots s_rw\mu \rangle_\der = \ldots = \langle \check{\varpi}_\beta, w\mu \rangle_\der
    % \end{equation*}
    % for $\beta \in \Delta \backslash I$. Further, for any $w' \in W$, we have $w'\mu=\sum_{\alpha \in \Delta} m_\alpha \alpha^{\vee}$ with $m_\alpha \in \BQ$. %by (\ref{existence}). 
    % Then, it follows by Lemma \ref{fweightandpairing} that 
    % \begin{equation}\label{condition} w \in \Omega_I \text{ if and only if } vw \in \Omega_I. \end{equation} 
    % Hence, $W_Iw \subset \Omega_I$ for every $w \in \Omega_I$ and we see that 
    % $$\bigcup_{w \in W^I \cap \, \Omega_I} C_I(w)=\bigcup_{w \in W^I \cap \, \Omega_I}\bigcup_{v \in W_I} C(vw) \subset \bigcup_{w \in \Omega_I} C(w) =Y_I.$$
    % On the other hand, if $w' \in \Omega_I$, there exist unique $w \in W^I$ and $v \in W_I$ such that $w'=vw$ and $l(w')=l(v)+l(w)$. 
    % This implies that $C(w') \subset C_I(w)$.  Let $v=s_1\ldots s_r$ be a reduced expression where $s_i \in \{s_{\alpha}\}_{\alpha \in I}$. 
    % By \cite[Chapter VI, 1.10]{Bo}, we have $s_i\check{\varpi}_\beta=\check{\varpi}_\beta$ for $\beta \in \Delta \backslash I$ and all $i$. Hence, we see that 
    %  \begin{equation*}
    % \langle \check{\varpi}_\beta, vw\mu  \rangle_\der =\langle s_1\check{\varpi}_\beta, s_2\ldots s_rw\mu \rangle_\der=\langle \check{\varpi}_\beta, s_2\ldots s_rw\mu \rangle_\der = \ldots = \langle \check{\varpi}_\beta, w\mu \rangle_\der
    %  \end{equation*}
    % for $\beta \in \Delta \backslash I$.
    % and by (\ref{condition}), we have $w \in W^I \cap \Omega_I$. Thus,
    % $$Y_I= \bigcup_{w \in \Omega_I} C(w) \subset \bigcup_{w \in W^I \cap \, \Omega_I} C_I(w).$$ 
    % That the union is disjoint follows from the unique decomposition of $w' \in \Omega_I$ mentioned above and that Schubert cells are disjoint for distinct Weyl group elements.  
In addition, we make the following observation for the complement of the $Y_I$ in $\sF$. 
% \subsection{Algebraic local cohomology}
\begin{lemma}\label{complement} 
    Let $I \subsetneq \Delta$ and $w_0 \in W$ the longest element. Then, 
    $$ \sF \backslash Y_I= \bigcup_{v \in W \backslash \Omega_I} vw_0C(w_0).$$
\end{lemma}
\begin{proof} 
Let $v \in W$. We first notice that $vw_0C(w_0)=vw_0 \bB w_0v^{-1}v\bB/ \bB$ is the \glqq coordinate neighborhood \grqq \,of $v\bB/\bB$ in $\sF$, which Kempf describes in \cite[Section 3]{K} (cf. \cite[Corollary 3.5]{K}).
 Then, by \cite[Proposition 6.3 a)]{K},
$$ C(v) \subset vw_0C(w_0).$$ 
Hence, 
$$ \sF \backslash Y_I= \bigcup_{v \in W \backslash \Omega_I} C(v) \subset \bigcup_{v \in W \backslash \Omega_I} vw_0C(w_0).$$
For the other inclusion, let $w \in \Omega_I$ and $v \not\in \Omega_I$. Then, we consider 
$$X:=v^{-1}\overline{C(w)} \cap w_0C(w_0).$$  
It is a closed $\bT$-invariant subset of $w_0C(w_0)$. By \cite[Theorem 3.1]{K}, this is in bijection to a closed $\bT$-invariant subset 
$$H \subset \bU_\bB^-.$$
Here $\bT$ acts by conjugation. We suppose that $X$ is non-empty. Thus, $H$ is non-empty. Furthermore, by \cite[Exercise 8.4.6 (5)]{Sp},
$$ \bU_\bB^-(\ov K)=\big\{g \in \bG(\ov K)  \mid \lim_{t \rightarrow 0} (w_0\mu)(t)g(w_0\mu)(t)^{-1}=1  \big\}.$$
As $H$ is closed and $\bT$-invariant, this description implies that $1 \in H$ (cf. \cite[Lemma 9]{Kn}). Therefore, $\bB/\bB \in X$ and 
$v\bB/\bB \in \overline{C(w)}$, respectively. This implies that $v \leq w$ and therefore $v \in \Omega_I$ since $Y_I$ is closed. That is a contradiction. Hence, 
$$ C(w) \cap vw_0C(w_0) = \emptyset$$ 
which implies  $$Y_I \cap \bigcup_{v \in W \backslash \Omega_I} vw_0C(w_0)= \emptyset.\qedhere$$ 
\end{proof}

\subsection{Algebraic local cohomology}\label{s:AlgLocCo}
In this subsection, we consider the local cohomology groups of $\sF$ with support in a (generalized) Schubert cell and in the closed varieties $Y_I$, respectively, with coefficients in $\CE_\lambda$. \\ 

Jantzen states in \cite[Introduction and Part II, Section 1]{J} that split reductive groups and constructions like Borel und Parabolic subgroups can be carried out over $\BZ$, and therefore, by base change, over any integral domain (cf. \cite[Exp. XXV, Corollary 1.3]{SGA3III}).  That means that there is a split connected reductive algebraic group $\sG$ over $\BZ$ with split maximal torus $\sT$ and Borel $\sB$ such that $\sG_K = \bG, \sT_K = \bT$ and $\sB_K = \bB$. Let $\sF_{\BZ}:=\sG/\sB$ and $$C(w)_{\BZ}:=\sB w\sB/\sB \subset \sF_{\BZ}$$ for $w \in W$. They are flat $\BZ$-schemes by \cite[Part I, Section 5.7 (2)]{J}. Moreover, $$\sF=(\sF_{\BZ})_K \text{ and } C(w)=(C(w)_{\BZ})_K.$$ 
The first identity follows from the fact that the base change commutes with the quotient (cf. \cite[Part I, Section 5.5 (4)]{J}). The latter one can be seen after identifying both sides with affine spaces (cf. \cite[Part II, Section 13.3 (1)]{J}).
In \cite[Section 3]{KL} it is mentioned that the flag varieties and Schubert cells admit \glqq flat lifts to 
$\mathbb{Z}$-schemes\grqq. Furthermore, as described in \cite[Section 13, p. 389]{K} (cf. \cite[Part I, Section 5.8]{J}), we have an invertible sheaf $\CE_{\lambda,\BZ}$ on $\sF_\BZ$ which is defined similarly to $\CE_\lambda$. The same arguments apply if we assume that $\bG$ and all introduced objects are defined over $\BQ$ and $\BC$, respectively. We denote the ground field, if it is not $K$, as a subscript in the 
following proof. \\  %(as it agrees with the base change by the previous considerations).\\

Then, we have the following two identifications of local cohomology groups on $\sF$. They are already known over $\BC$ (cf. \cite[(3.3)]{MR} and \cite[Theorem 1 \& Theorem 3]{MR}). 

\begin{lemma}\label{loccocell}For $w \in W$, one has 
    $$H^i_{C(w)}(\sF,\CE_\lambda)\cong\begin{cases} M(w\cdot\lambda) &i=n-l(w),\\ 0 &\text{else} \end{cases}$$
in $\CO_{\alg}$.
\end{lemma}
\begin{proof}
    As $C(w)$ is affine, it follows that $H^i_{C(w)}(\sF,\CE_\lambda)=0$ for $i \neq n-l(w)$ (cf. \cite[Theorem 10.9]{K}). Since $\CE_\lambda$ has a natural $\fkg$-module structure (cf. \cite[Section 1.2]{O2}), we see by functoriality that $H^{n-l(w)}_{C(w)}(\sF,\CE_\lambda)$ is a $\fkg$-module. 
    Furthermore, by \cite[Lemma 12.8.]{K}, we have that $H^{n-l(w)}_{C(w)}(\sF,\CE_\lambda)$ is $\fkt$-semisimple and  $$\ch\big(H^{n-l(w)}_{C(w)}(\sF, \CE_\lambda)\big)=\ch\big(M(w \cdot \lambda)\big).$$ This implies that $H^{n-l(w)}_{C(w)}(\sF,\CE_\lambda)$ lies in the category $\CO_{\alg}$ (cf. \cite[Example 1.1]{AL}). 
    In particular for $w=e$, we see by the last remark in \cite[Section 12]{K} and the proof of \cite[Proposition 1.4.2]{O2} that 
    \begin{align*}
        H^{n}_{C(e)}(\sF, \CE_\lambda)&\cong H^{n}_{C(e)}(\sF, \CO_{\sF}) \otimes_{K} (K)_{2 \rho + \lambda} \
    \cong M(-2\rho)^\vee \otimes_{K} (K)_{2 \rho + \lambda} \\
    &\cong M(-2\rho)\otimes_{K} (K)_{2 \rho + \lambda} \cong M(\lambda)%U(\fkn^-) \otimes_{\BQ_p} (\BQ_p)_{-2\rho} \otimes_{\BQ_p} (\BQ_p)_\lambda \\
    \end{align*}
    holds in the category $\CO_{\alg}$. Here, we used \cite[Proposition 7]{B} for the second isomorphism and the fact that $-2\rho$ is antidominant for the third.
    Thus, by Lemma \ref{Verma}, it remains  to prove that 
    there is a non-trivial injective morphism 
    \begin{equation}\label{search}
        H^{n-l(w)}_{C(w)}(\sF,\CE_\lambda) \longrightarrow  H^{n}_{C(e)}(\sF, \CE_\lambda).
    \end{equation}
    For this, let $k \in \{K, \BQ, \BC\}$. Further, we let $X_1:=\overline{C(w)_\BZ}$ and $X_2:=X_1 \backslash C(w)_\BZ$. By Lemma \ref{locrel} and Proposition \ref{excision}, we have  
    $$H^q_{C(w)_\BZ}(\sF_\BZ, \CE_{\lambda,\BZ}  \otimes k ) \cong H^q_{X_1/X_2}(\sF_\BZ, \CE_{\lambda,\BZ}  \otimes k) \text{ and } H^q_{C(w)_k}(\sF_k, \CE_{\lambda,k} ) \cong H^q_{X_{1,k}/X_{2,k}}(\sF_k, \CE_{\lambda,k} ).$$  
    Then, by \cite[Lemma 13.8]{K}, we obtain an isomorphism 
    $$H^q_{C(w)_\BZ}(\sF_\BZ, \CE_{\lambda,\BZ} \otimes k) \cong H^q_{C(w)_k}(\sF_k, \CE_{\lambda,k})$$
    of $k$-vector spaces. As $\CO_{\sF_\BZ}$ is flat over $\BZ$ and $\CE_{\lambda,\BZ}$ is locally free, it follows that $\CE_{\lambda,\BZ}$ is flat over $\BZ$ since it is a local property. 
     Following \cite[Section 4]{KL}, this yields a spectral sequence 
    $$ E_2^{p,q}=\Tor^{\BZ}_{-p}\big(H^q_{C(w)_\BZ}(\sF_\BZ, \CE_{\lambda,\BZ} ), k\big) \Rightarrow H^{p+q}_{C(w)_k}(\sF_k,\CE_{\lambda,k}).$$
    Since $k$ is flat over $\BZ$, we have an isomorphism 
    $$H^{n-l(w)}_{C(w)_\BZ}(\sF_\BZ, \CE_{\lambda,\BZ} )\otimes k \cong H^{{n-l(w)}}_{C(w)_k}(\sF_k,\CE_{\lambda,k})$$
    of $k$-vector spaces. This implies 
    \begin{align}
        H^{{n-l(w)}}_{C(w)_\BC}(\sF_\BC,\CE_{\lambda,\BC}) &\cong  H^{n-l(w)}_{C(w)_\BQ}(\sF_\BQ,\CE_{\lambda,\BQ}) \otimes_\BQ \BC \label{C}
    \intertext{ and }
    H^{{n-l(w)}}_{C(w)}(\sF,\CE_{\lambda}) &\cong  H^{n-l(w)}_{C(w)_\BQ}(\sF_\BQ,\CE_{\lambda,\BQ}) \otimes_\BQ K. \label{K}
    \end{align}
    If we choose $X_1=\sF_\BZ$ and $X_2=\emptyset$ instead, then using the same arguments as before, we get that 
    \begin{align}
        H^q(\sF_\BC,\CE_{\lambda,\BC})&\cong  H^q(\sF_\BQ,\CE_{\lambda,\BQ}) \otimes \BC \label{HC}
        % H^q(\sF,\CE_{\lambda})&\cong  H^q(\sF_\BQ,\CE_{\lambda,\BQ}) \otimes K\label{HK}
    \end{align}
    for all integers $q$. Next, let $Z_j \subset \sF_\BQ$ be the union of the closure of Schubert cells of codimension greater than or equal to $j$. This defines a filtration on $\sF_\BQ$ by closed subsets 
    \begin{equation}\label{globalfilt}
        \sF_\BQ=Z_0 \supset Z_1 \supset \ldots \supset Z_n=C(e)_\BQ.
    \end{equation}
    Furthermore, 
    $$Z_j\backslash Z_{j+1} = \bigsqcup_{\substack{w \in W \\ l(w)=n-j}} C(w)_\BQ.$$ 
    Then, we get, by Lemma \ref{locrel} and Proposition \ref{excision} (cf. \cite[p. 385]{K}), that 
    $$ H^{i}_{Z_j\backslash  Z_{j+1}}(\sF_\BQ,\CE_{\lambda,\BQ})\cong\bigoplus_{\substack{w \in W \\ l(w)=n-j}} H^i_{C(w)}(\sF_\BQ,\CE_{\lambda,\BQ})$$ 
    for all integers $i$. 
    Thus, by Lemma \ref{cousinspectral} and Lemma \ref{loccocell}, we can compute $H^*(\sF_\BQ, \CE_{\lambda,\BQ})$ by the complex 
    \begin{equation}\label{BGGreso}\bigoplus_{\substack{w \in W \\ l(w)=n}} H^0_{C(w)_\BQ}(\sF_\BQ, \CE_{\lambda,\BQ})\rightarrow \ldots \rightarrow  \bigoplus_{\substack{w \in W \\ l(w)=1}} H^{n-1}_{C(w)_\BQ}(\sF_\BQ,\CE_{\lambda,\BQ})\rightarrow H^n_{C(e)_\BQ}(\sF_\BQ, \CE_{\lambda,\BQ}).
    \end{equation} 
    On the other hand, we have by Serre duality (cf. \cite[Part II, 4.2 (9)]{J}, note that the setting in loc. cit. induces different signs) that 
    $$H^i(\sF_\BQ, \CE_{\lambda,\BQ})=H^i(\sF_\BQ, \CL_{\lambda,\BQ} \otimes \omega_{\sF_\BQ}\\) \cong (H^{n-i}(\sF_\BQ, (\CL_{\lambda,\BQ})^\vee))'=(H^{n-i}(\sF_\BQ, \CL_{-\lambda,\BQ}))'.$$
    For the latter one, the Borel-Weil-Bott theorem (cf. \cite[Part II, Corollary 5.5]{J}, note again the setting in loc.cit) gives  $H^{i}(\sF_\BQ, \CL_{-\lambda,\BQ})=0$ for $i \neq 0$. Hence, the complex (\ref{BGGreso}) is a resolution of $H^n(\sF_\BQ, \CE_{\lambda,\BQ})$. By (\ref{C}), (\ref{HC}) and (faithfully) flatness of field extensions, we get an acyclic complex 
    \begin{align}\label{BGGresoC} 0 \rightarrow \bigoplus_{\substack{w \in W \\ l(w)=n}} H^0_{C(w)_\BC}(\sF_\BC, \CE_{\lambda,\BC})\rightarrow \ldots &\rightarrow  \bigoplus_{\substack{w \in W \nonumber \\ l(w)=1}} H^{n-1}_{C(w)_\BC}(\sF_\BC,\CE_{\lambda,\BC}) \\ &\rightarrow H^n_{C(e)_\BC}(\sF_\BC, \CE_{\lambda,\BC})\rightarrow H^n(\sF_\BC, \CE_{\lambda,\BC}) \rightarrow 0.
    \end{align} 
    Again by the Borel-Weil-Bott theorem, we know that $H^n(\sF_\BC, \CE_{\lambda,\BC})= L(\lambda)_\BC$. 
    Here $L(\lambda)_\BC$ is the unique simple quotient of the Verma module $M(\lambda)_\BC$ in the usual BGG Category \CO over the complex numbers (cf. \cite[Section 1.3]{H2}). Then, by \cite[(3.3)]{MR}, we have 
    $$ H^{n-l(w)}_{C(w)_\BC}(\sF_\BC,\CE_{\lambda,\BC}) \cong M(w \cdot \lambda)_\BC$$ 
    for all $w \in W$. Therefore, the complex (\ref{BGGresoC}) is a BGG resolution of $L(\lambda)_\BC$ (cf. \cite[Section 6.1]{H2}).
    Thus, by \cite[Theorem, Section 6.8]{H2}, the natural morphism 
    $$H^{n-l(w)}_{C(w)_\BC}(\sF_\BC,\CE_{\lambda,\BC}) \longrightarrow H^{n-l(w')}_{C(w')_\BC}(\sF_\BC,\CE_{\lambda,\BC})$$ 
    is non-trivial for $w' \leq w$ with $l(w)=l(w)+1$. Moreover, it is injective by \cite[Theorem, Section 4.2]{H2}. This implies that 
    the morphism 
    $$H^{n-l(w)}_{C(w)_\BQ}(\sF_\BQ,\CE_{\lambda,\BQ}) \longrightarrow H^{n-l(w')}_{C(w')_\BQ}(\sF_\BQ,\CE_{\lambda,\BQ})$$ 
    in the complex (\ref{BGGreso}) was already injective by the faithfully flatness of field extensions. Again by the faithfully flatness and by (\ref{K}), we get an injective morphism 
    $$H^{n-l(w)}_{C(w)}(\sF,\CE_{\lambda}) \longhookrightarrow H^{n-l(w')}_{C(w')'}(\sF,\CE_{\lambda})$$ 
    for all $w, w' \in W$ with $w' \leq w$ and $l(w)=l(w')+1$. Let $w \in W$ with reduced expression $w=s_1\ldots s_t$ and let $w_i:=s_1\ldots s_i$, i.e. $w=w_t$. Then, we get the desired morphism (\ref{search}) from the sequence of injections 
    $$ H^{n-l(w)}_{C(w)}(\sF,\CE_{\lambda}) \longhookrightarrow H^{n-l(w_{t-1})}_{C(w_{t-1})}(\sF,\CE_{\lambda}) \longhookrightarrow \ldots \longhookrightarrow H^{n}_{C(e)}(\sF,\CE_{\lambda}).$$ 
    \end{proof}
% \begin{remark}\label{morphismlocal}
% Let $Z_i$ be the union of the closure of Schubert cells of codimension greater than or equal to $i$. This defines a filtration on $\sF$ by closed subsets 
% \begin{equation}\label{globalfilt}
%     \sF=Z_0 \supset Z_1 \supset \ldots \supset Z_n=C(e).
% \end{equation}
% Furthermore, 
% $$Z_k\backslash Z_{k+1} = \bigsqcup_{\substack{w \in W \\ l(w)=n-k}} C(w).$$ 
% Then, we get by Lemma \ref{locrel} and Proposition \ref{excision} (cf. \cite[p. 385]{K}) that 
% $$ H^{i}_{Z_k\backslash  Z_{k+1}}(\sF,\CE_\lambda)\cong\bigoplus_{\substack{w \in W \\ l(w)=n-k}} H^i_{C(w)}(\sF,\CE_\lambda)$$ 
% for all integers $i$. 
% Thus, by Lemma \ref{cousinspectral} and Lemma \ref{loccocell}, we can compute $H^*(\sF, \CE_\lambda)$ by the complex 
% \begin{equation}\label{BGG}\bigoplus_{\substack{w \in W \\ l(w)=n}} H^0_{C(w)}(\sF, \CE_\lambda)\rightarrow \ldots \rightarrow  \bigoplus_{\substack{w \in W \\ l(w)=1}} H^{n-1}_{C(w)}(\sF, \CE_\lambda)\rightarrow H^n_{C(w)}(\sF, \CE_\lambda).
% \end{equation} 
% On the other hand, we have by Serre duality (cf. \cite[Part II, 4.2 (9)]{J}) that 
% $$H^i(\sF, \CE_\lambda)=H^i(\sF, \CL_\lambda \otimes \omega_{\sF}\\) \cong (H^{n-i}(\sF, (\CL_\lambda)^\vee))'=(H^{n-i}(\sF, \CL_{-\lambda}))'.$$
% For the latter one, the Borel-Weil-Bott theorem (cf. \cite[Part II, Corollary 5.5]{J}, but note the setting) gives 
% $$  H^{i}(\sF, \CL_{-\lambda})\cong \begin{cases} L(\lambda)' &i=0 \\
%                                                         0  &\text{else.}\end{cases}$$ 
% Therefore, the complex (\ref{BGG}) is a BGG-resolution of $L(\lambda)$ (cf. \cite[Section 6.1]{H2}). Then, \cite[Theorem, Section 6.8]{H2} implies that the natural induced morphism 
% $$ H^{n-l(w)}_{C(w)}(\sF, \CE_\lambda) \rightarrow H^{n-l(w')}_{C(w')}(\sF, \CE_\lambda)$$ 
% is non-trivial and injective for $ w'\leq w$ with $l(w)=l(w')+1$. \\ 
% \end{remark}
Similar to Lemma \ref{loccocell}, we have the following identification for generalized Schubert cells. 
\begin{lemma}\label{C_I(w)} For $I \subset \Delta$ and $w \in W^I$, one has
    $$H^i_{C_I(w)}(\sF,\CE_\lambda)\cong\begin{cases} M_I(w\cdot\lambda) &i=n-l(w),\\ 0 &\text{else} \end{cases}$$
in $\CO_{\alg}^{\fkp_I}$.
\end{lemma}

\begin{proof}
Over $\mathbb{C}$ this is \cite[Theorem 1/Theorem 3]{MR}. The arguments of loc. cit. are applicable as well. For completeness, we will recall them. \\

Let $Z_j$ be the union of the closure of Schubert cells of codimension greater than or equal to $j$ and $U_w=\sF\backslash \big(\overline{C_I(w)}\backslash C_I(w)\big)$. Hence, $C_I(w)$ is closed in the open subset $U_w$. Let $r_I:=\dim(\bP_I/\bB)$ and $t:=n-l(w)-r_I$. Then, we consider the filtration on $U_w$ by closed subsets 
\begin{equation}\label{filt}
    U_w \supset C_I(w) \supset C_I(w)  \cap Z_{t+1} \supset \ldots \supset C_I(w)  \cap Z_{r_I+t}=C(w).
\end{equation}
As 
$$ (C_I(w)  \cap Z_{t+j})\backslash (C_I(w)  \cap Z_{t+j+1})= C_I(w)  \cap  (Z_{t+j}\backslash Z_{t+j+1})=\bigsqcup_{\substack{v \in W_I \\ l(v)=r_I-j}} C(vw),$$
we get, as in Lemma \ref{loccocell}, that
$$H^i_{(C_I(w) \cap Z_{t+j})/(C_I(w) \cap Z_{t+j+1})}(U_w,\CE_\lambda)\cong\bigoplus_{\substack{v \in W_I \\ l(v)=r_I-j}} H^i_{C(vw)}(U_w,\CE_\lambda)$$
for all integers $i$. Notice that by Proposition \ref{excision}, we have 
$$H^i_{C(vw)}(U_w,\CE_\lambda)\cong H^i_{C(vw)}(\sF,\CE_\lambda).$$ 
Then, applying Lemma \ref{cousinspectral} to (\ref{filt}) and taking Lemma \ref{loccocell}
into account, we see that the cochain complex 
\begin{equation}\label{complexparabolic}
     \bigoplus_{\substack{v \in W_I \\ l(v)=r_I}} M(v\cdot \lambda) \rightarrow \ldots \rightarrow \bigoplus_{\substack{v \in W_I \\ l(v)=1}} M(v\cdot \lambda) \rightarrow  M(\lambda), 
\end{equation} 
starting in degree $t$, computes $H^*_{C_I(w)}(U_w,\CE_\lambda)$, and therefore, by Proposition \ref{excision}, also 
$H^*_{C_I(w)}(\sF,\CE_\lambda)$. Then, by the work of Lepowsky (cf. \cite[p. 506, Proof of Theorem 4.3]{Le1}), we get 
 $$H^{n-l(w)}_{C_I(w)}(\sF,\CE_\lambda)\cong M_I(w \cdot \lambda).$$  
On the other hand, the complex (\ref{complexparabolic}) is obtained from the BGG-resolution of $V_I(\lambda)$ by Verma modules for $L_{P_I}$ by tensoring with $U(\fkg)$ over $U(\fkp_I)$. This functor preserves exactness.  Therefore, $H^i_{C_I(w)}(\sF,\CE_\lambda)=0$ for $i \neq n-l(w)$. 

\end{proof}
Another application of Lemma \ref{cousinspectral} is the computation of the local cohomology groups $H_{Y_I}^*(\sF,\CE)$.
     \begin{lemma}\label{cohY_I} Let $I \subsetneq \Delta$, $d_I:= \dim(Y_I)$ and $r_I:= \dim(\bP_I/\bB)$. Then, the
        cochain complex

        $$C_I^\bullet:\bigoplus_{\substack{w \in W^I \cap \,\Omega_I \\ l(w)=d_I-r_I}} H^{n-l(w)}_{C_I(w)}(\sF,\CE_\lambda) \rightarrow \bigoplus_{\substack{w \in W^I \cap \,\Omega_I \\ l(w)=d_I-r_I-1}} H^{n-l(w)}_{C_I(w)}(\sF,\CE_\lambda) \rightarrow \ldots \rightarrow  H^{n}_{C_I(e)}(\sF,\CE_\lambda), $$
        with the natural morphisms starting in degree $n+r_I-d_I$, computes the cohomology groups $H^j_{Y_I}(\sF,\CE_\lambda)$. More specifically, $H^j(C_I^\bullet)=H^j_{Y_I}(\sF,\CE_\lambda)$.
    \end{lemma}

    \begin{proof}
    %Set $d=\dim(Y_I)$ and $r=\dim(\bP_I/\bB)$. Notice that $r$ coincides with the length of the longest element in $W_I$. 
    Let $\tilde{Z}_j$ be the closure of the union of those $P_I$-orbits $C_I(w)$ whose codimension is greater or equal to $j$. This defines a filtration 
    $$\sF=\tilde{Z}_0 \supset \tilde{Z}_1 \supset \ldots \supset \tilde{Z}_{n-r}$$by closed subsets. 
    Then, we consider the filtration of closed subsets on $Y_I$ induced by setting $Z_j:=Y_I\cap \tilde{Z}_{n-d_I+j}$ 
    \begin{equation}\label{fil1} 
        Y_I=Z_0\supset  Z_1 \supset \ldots \supset Z_{d_I}
    \end{equation}
    where, by Lemma \ref{genSchCe}, one has $${Z}_j\backslash {Z}_{j+1}=\bigsqcup_{\substack{w \in W^I \cap \,\Omega_I \\ l(w)=d_I-r_I-j}} C_I(w).$$
    Therefore, as in the proof of Lemma \ref{loccocell} , we obtain that 
    $$H^i_{Z_j/Z_{j+1}}(\sF,\CE_\lambda)\cong\bigoplus_{\substack{w \in W^I \cap \,\Omega_I \\ l(w)=d_I-r_I-j}} H^i_{C_I(w)}(\sF,\CE_\lambda).$$ 
    for all integers $i$. Applying Lemma \ref{cousinspectral} to the filtration (\ref{fil1}) and $\CE_\lambda$, and taking Lemma \ref{C_I(w)} into account, we see that the induced spectral sequence 
    $$E_1^{pq}=H^{p+q}_{Z_p/Z_{p+1}}(X, \CE_\lambda)\Rightarrow H^{p+q}_{Y_I}(X,\CE_\lambda)$$ degenerates at the $E_2$-page. Thus, the result follows. 
\end{proof}

\begin{remark}\label{morphismBGG}
    As pointed out in \cite[Theorem 2]{MR}, the morphisms $$H^{n-l(w)}_{C_I(w)}(\sF, \CE_\lambda) \rightarrow H^{n-l(w')}_{C_I(w')}(\sF, \CE_\lambda)$$ for $w' \leq w$ with $l(w)=l(w')+1$ that appear in the differentials are those
    from the Lepowsky BGG resolution (cf. \cite[Theorem 4.3]{Le1}). 
\end{remark}

\begin{corollary}\label{loccohcatOp}
        For each $i \in \BN_0$, the $U(\fkg)$-module $H_{Y_I}^i(\sF,\CE_\lambda)$  lies in $\CO_{\alg}^{\fkp_I}$.
\end{corollary}
    
\begin{proof}
        As the category $\CO_{\alg}^{\fkp_I}$ is closed under taking submodules and quotients, this follows immediately from Lemma \ref{C_I(w)} and \ref{cohY_I}.
\end{proof}


\subsection{Analytic local cohomology}\label{s:AnaLocCo} In the following, we would like to relate the algebraic local cohomology groups of $\sF$ with support in the $Y_I$ and coefficients in $\CE_\lambda$ to some analytic local cohomology groups of $\sF^{\rig}.$ \\

For this, we recall first from \cite[Section 1.3]{O2} what we mean by analytic local cohomology. Let $X$ be a rigid analytic variety over $K$ and $U \subset X$ an admissible open subset with $Z:=X\backslash U$, the set theoretical complement. Further, be $\CE$ be a coherent sheaf on $X$. 
Then, similar to the last section, we define 
$$ \Gamma_Z(X,\CE):=\ker\big(\Gamma(X,\CE)\rightarrow \Gamma(U,\CE)\big) $$ 
and $H^*_Y(X,\CG)$ to be the right derived functors. In case $X$ is a separated rigid analytic variety of countable type, the local cohomology groups carry a natural structure of a locally convex $K$-vector space which is in general not Hausdorff (cf. \cite[Section 1.3]{O2}, \cite[Section 1.6]{vP}). \\

For our purposes, we fix an embedding $\sF \hookrightarrow{} \BP_{K}^N$ defined by the vanishing ideal $\sI \subset \CO_K[T_0,\ldots, T_N]$. We introduce, adapted from \cite[Section 2, p. 1398]{O6}, the notion of special neighborhoods of 
a closed subvariety of $\sF$. They play a crucial role in the computation of the cohomology of a period domain. 
\begin{definition}\label{epsnbh}
Let $\epsilon \in \lb \overline{K}^\times \rb$. Let $Y \subset \sF$ be a closed subvariety 
and $f_1, \ldots, f_r \in \CO_K[T_0,\ldots, T_N]$ homogeneous polynomials such that they generate the vanishing ideal of the Zariski closure of $Y$ in $\sF_{\CO_K}$. Additionally, each $f_i$ has at least one coefficient in $\CO_K^\times$. 
\begin{enumerate}[label=\roman*)]
    \item  We call a tuple $(z_0,\ldots, z_{N}) \in \BA_K^{N+1}(C)$ \textsl{unimodular} if $z_i \in \CO_{C}$ 
           for all $i$ and there exists an $i$ such that $z_i \in \CO_{C}^\times$.
    \item  We define the \textsl{open $\epsilon$-neighborhood} of $Y$ in $\sF^{\rig}$ by 
\begin{align*}
    Y(\epsilon):=\Big\{z \in \sF^{\rig} \, \Big\vert \, &\text{for any unimodular representative $\tilde{z}$ of $z$, we have } \\ &\lb f_j(\tilde{z})\rb \leq \epsilon \text{ for all $j$}\Big\}.
\end{align*}
\item We define the \textsl{closed $\epsilon$-neighborhood} of $Y$ in $\sF^{\rig}$ by 
\begin{align*}
    Y^{-}(\epsilon):=\Big\{z \in \sF^{\rig} \, \Big\vert \, &\text{for any unimodular representative $\tilde{z}$ of $z$, we have } \\ &\lb f_j(\tilde{z})\rb < \epsilon \text{ for all $j$}\Big\}.
\end{align*}
\end{enumerate}
\end{definition}
% In the case that $K$ is a local field with ring of integers $\CO_K$ and $\bG$ a split connected reductive group over $K$, this implies 
% that there is a split reductive group $\bG_0$ over $\CO_K$ such that $(\bG_0)_K \cong \bG$. Furthermore, $(\bG_0)_k$ and $\bG$ have the same root datum for $k$ the residue field of $\CO_K$. 
% We call $\bG_0$ a \textit{split reductive group model} of $\bG$ over $\bO_K$. \\
Let $I \subsetneq \Delta$. Let $\Phi_{\fku_{\bP_I}^-}=\{{\alpha_1}, \ldots, {\alpha_r}\}$ be the set of roots appearing in $\fku_{\bP_I}^-$ (under the adjoint action of $\bT$) and $y_{\alpha_1}, \ldots, y_{\alpha_r}$ be a basis of the $K$-vector space $\fku_{\bP_I}^-$. Then, for $\epsilon \in \lb \overline{K^*} \rb$, the norm ${\lb \enspace\, \rb}_\epsilon$ on $U(\fku_{\bP_I}^-)$ is given by 
\begin{equation}\label{normUp}
   \Bigg \lb \sum_{(i_1,\ldots,i_r) \in \BN_0^r} a_{i_1,\ldots,i_r}y_{\alpha_1}^{i_1}\cdots  y_{\alpha_r}^{i_r} \Bigg\rb_\epsilon= \sup_{(i_1,\ldots,i_r) \in \BN_0^r}\Big\lb i_1! \cdots i_r! \cdot a_{i_1,\ldots,i_r}\Big\rb \epsilon^{i_1 + \ldots +i_r}.
   \end{equation}
Completing $U(\fku_{\bP_I}^-)$ with respect to ${\lb \enspace\, \rb}_\epsilon$ yields the $K$-Banach space
 \begin{align}\label{completeUEA}
   U(\fku_{\bP}^-)_\epsilon:=\Bigg\{&\sum_{(i_1,\ldots,i_r) \in \BN_0^r} a_{i_1,\ldots,i_r}y_{\alpha_1}^{i_1}\cdots  y_{\alpha_r}^{i_r}\, \bigg\vert \, a_{i_1,\ldots,i_r} \in K, \nonumber \\
   &\lb i_1! \cdots i_r! \cdot a_{i_1,\ldots,i_r}\rb \epsilon^{i_1 + \ldots +i_r} \rightarrow 0 \text{ for } i_1+\ldots + i_r \rightarrow 0
   \Bigg\}.
\end{align}
Let $m\in \BN$ and $\epsilon_m:=\lb \pi \rb^m$. We will write $U(\fku_{\bP_I}^-)_m$ for $U(\fku_{\bP_I}^-)_{\frac{1}{\epsilon_m}}$. Let $i \in \BN_0$.  We know from Corollary \ref{loccohcatOp} that $H^i_{Y_I}(\sF,\CE_\lambda) \in \mathcal{O}_{\alg}^{\fkp_I}$. Thus, we have an $\CO^{\fkp_I}_{\alg}$-pair $(H^i_{Y_I}(\sF,\CE_\lambda),W)$ with a short exact sequence 
\begin{equation}\label{seqCoh}
      0 \rightarrow \fkd \rightarrow U(\fku_{\bP_I}^-) \otimes_{K} W \longrightarrow H^i_{Y_I}(\sF,\CE_\lambda) \rightarrow 0.
\end{equation}
Then, the above defines a norm on $U(\fku_{\bP_I}^-) \otimes_{K} W$ and induces, by (\ref{seqCoh}), the quotient norm on $H^i_{Y_I}(\sF,\CE_\lambda)$. 
Since the quotient map is open, it is strict (cf. \cite[Section 1.1.9, Proposition 3 ii)]{BGR}) and we obtain, by \cite[Section 1.1.9, Corollary 6]{BGR}, a short exact sequence of $K$-Banach spaces
\begin{equation}\label{seqComplete}
    0 \longrightarrow \fkd_{m} \longrightarrow U(\fku_{\bP_I}^-)_m \otimes_K W \longrightarrow \hat{H}^i_{Y_I,m} \longrightarrow 0
\end{equation}
with $\fkd_{m}$ the completion of $\fkd$ in $U(\fku_{\bP_I}^-)_m$ and $\hat{H}^i_{Y_I,m}$ denotes the completion of $H^i_{Y_I}(\sF,\CE_\lambda)$ with respect to the quotient norm. \\

Moreover, we define $D_w:=ww_0C(w_0)$ and $H_w:=\sF \backslash D_w$ for $w \in W$. By Lemma \ref{complement}, we know that 
\begin{equation}\label{affinecovering}
    \sF \backslash Y_I=\bigcup_{w \in W \backslash \Omega_I} D_w 
\end{equation}
which is an affine open covering of the complement because $C(w_0)$ is affine open. 
Thus, we can compute $H^*(\sF \backslash Y_I,\CE_\lambda)$ by the \v{C}ech-complex 
$$ \bigoplus_{w \in W \backslash \Omega_I} \Gamma(D_{w}, \CE_\lambda) \rightarrow \bigoplus_{\substack{w, w' \in W \backslash \Omega_I \\ w \neq w'}} \Gamma(D_{w} \cap D_{w'} , \CE_\lambda)  \rightarrow \ldots \rightarrow \Gamma(\bigcap_{w \in W \backslash \Omega_I} D_{w} , \CE_\lambda).$$ Furthermore, we can easily deduce from (\ref{affinecovering}) that for $\epsilon \in \lb \overline{K^\times} \rb$, we have 
$$ Y_I^-(\epsilon)= \bigcap_{w \in W \backslash \Omega_I} H_w^-(\epsilon).$$ 
Therefore, we consider the subset 
$$ D_{w,\epsilon}:=\sF^{\rig} \backslash H_w^-(\epsilon)$$
for $w \in W$. 

\begin{lemma} Let $w \in W$. The subset $D_{w,\epsilon}$ is affinoid. 
\end{lemma}

\begin{proof}
    Since $H_w$ is of codimension 1, there is $f \in \CO_K[T_0,\ldots, T_N]$ homogenous of degree $t$ with at least one coefficient in $\CO_K^{\times}$ and generating the vanishing ideal of the Zariski closure of $H_w$ in $\sF_{\CO_K}$. Let $N_0:= \binom{n+t}{t}-1$. We embed $ \BP^N_K$ into $\BP^{N_0}_K$ via the $t$-th Veronese embedding. Then, by substituting monomials, $f$ yields a homogeneous linear polynomial $g \in \CO_K[T_0,\ldots, T_{N_0}]$ defining a hyperplane $H_0 \subset \BP^{N_0}_K$, such that $$H_w=\sF \cap H_0.$$ It is known that $(\BP^{N_0}_K)^{\rig}\backslash H_0^-(\epsilon)$ is affinoid (cf. \cite[Section 1, Proof of Proposition 4]{SS}) and we notice that 
    $$ \sF^{\rig}\backslash \Big(\sF^{\rig} \cap H_0^-(\epsilon)\Big) = \sF^{\rig}\cap \Big((\BP^{N_0}_K)^{\rig}\backslash H_0^-(\epsilon)\Big).$$ Thus, $\sF^{\rig}\backslash \Big(\sF^{\rig} \cap H_0^-(\epsilon)\Big)$
    is also affinoid since it is a zero set in $(\BP^{N_0}_K)^{\rig}\backslash H_0^-(\epsilon)$. However, we have $\sF^{\rig} \cap H_0^-(\epsilon)=H_w^-(\epsilon)$. 
\end{proof}
% Therefore, we define the affinoid subset 
% $$ D_{w,\epsilon}:=\sF^{\rig} \backslash H_w(\epsilon)$$
% for $w \in W$ 
This results in the affinoid covering 
$$\sF^{\rig} \backslash Y_I^-(\epsilon)= \bigcup_{w \in W \backslash \Omega_I} D_{w,\epsilon}$$
which has two consequences. On the one hand we get an admissible covering 
$$\sF^{\rig} \backslash Y_I(\epsilon_m)=\bigcup_{{\substack{\epsilon \rightarrow \epsilon_m \\ \epsilon_m < \epsilon \in \lb \overline{K^\times}\rb}}} \sF^{\rig} \backslash Y_I^-(\epsilon) $$
by quasi-compact admissible open subsets (cf. \cite[Section 1.3, p. 601]{O2}).
On the other hand we can compute $H^*(\sF^{\rig} \backslash Y_I^-(\epsilon),\CE_\lambda)$ by the \v{C}ech-complex 
$$ \bigoplus_{w \in W \backslash \Omega_I} \Gamma(D_{w,\epsilon}, \CE_\lambda) \rightarrow \bigoplus_{\substack{w, w' \in W \backslash \Omega_I \\ w \neq w'}} \Gamma(D_{w,\epsilon} \cap D_{w',\epsilon} , \CE_\lambda)  \rightarrow \ldots \rightarrow \Gamma(\bigcap_{w \in W \backslash \Omega_I} D_{w,\epsilon} , \CE_\lambda)
$$ which terms are $K$-Banach spaces. So far, we have to make the following assumption and conjecture (cf. \cite[Proposition 2.5.2]{Li} and \cite[Lemma 3.4.10]{Li} for the Drinfeld case). 
\begin{hac}\label{hypo1}
   Let $I \subsetneq \Delta$ and $i \in \BN_0$. Both, the cohomology groups $H^i(\sF^{\rig} \backslash Y_I^-(\epsilon),\CE_\lambda)$, and $H^i_{Y_I^-(\epsilon)}(\sF^{\rig},\CE_\lambda)$ are
   $K$-Banach spaces in which the algebraic cohomology group $H^i(\sF \backslash Y_I,\CE_\lambda)$ and $H^i_{Y_I}(\sF,\CE_\lambda)$, respectively, is a dense subspace. 
   Moreover, we have an isomorphism of topological $K$-vector spaces 
   $$ \varprojlim_{m \in \BN} H^i_{Y_I^-(\epsilon_m)}(\sF^{\rig},\CE_\lambda) \cong \varprojlim_{m \in \BN} \hat{H}^i_{Y_I,m}.$$ 
\end{hac}
%\begin{proof}
 %   \todo[inline]{\textbf{TODO/Question}: I'd like to adapt the proof of Lemma 1.3.1 in \cite{O2} via covering with schubert cells,
  %   but still don't understand 
   % the crucial part.}   
%\end{proof}



\begin{corollary} \label{limitloc} Let $I \subsetneq \Delta$. For $i \in \BN_0$, we have the following isomorphisms of topological $K$-vector spaces:
    \begin{align*}
        H^i(\sF^{\rig} \backslash Y_I(\epsilon_m),\CE_\lambda)&\cong \varprojlim_{\substack{\epsilon \rightarrow \epsilon_m \\ \epsilon_m < \epsilon \in \lb\overline{K^\times}\rb}} H^i(\sF^{\rig} \backslash Y_I^-(\epsilon),\CE_\lambda) \\
       \intertext{and} 
        H^i_{Y_I(\epsilon_m)}(\sF^{\rig},\CE_\lambda)&\cong \varprojlim_{\substack{\epsilon \rightarrow \epsilon_m \\ \epsilon_m < \epsilon \in \lb \overline{K^\times}\rb}} H^i_{Y_I^{-}(\epsilon)}(\sF^{\rig},\CE_\lambda).
    \end{align*}
\end{corollary}

\begin{proof}
    The proof is same as that of \cite[Lemma 1.3.2]{O2}. 
 %   \todo[inline]{\textbf{TODO}: Add Proposition 1.3.3 from \cite{O2}}
\end{proof}

% Let $\bG_0$ be a split reductive group model of $\bG$ over $\BZ_p$ and 
% $\pi \in \BZ_p$ a uniformizer. For any positive number $n \in \BN$, we consider the reduction map 
% \begin{equation*}
% p_n:\bG_0(\BZ_p) \rightarrow \bG_0(\BZ_p/(\pi^n)). 
% \end{equation*}
% We set $G_0=\bG_0(\BZ_p)$. For $I \subset \Delta$ and a standard parabolic $\bP_{0,I}$ subgroup of $\bG_0$, we define $P_I^n:=p_n^{-1}(\bP_{0,I}(\BZ_p/(\pi^n)) \subset G_0$. 

Let $\bG_0$ be a split reductive group model of $\bG$ over $\CO_K$ with Borel pair $(\bT_0, \bB_0)$ and a standard parabolic subgroup $\bP_{I,0}$ containing $\bB_0$ such that the base change to $K$ yields the pair $(\bT,\bB)$ and $\bP_I$, respectively. 
For any positive number $m \in \BN$ let 
\begin{equation*}
p_m:\bG_0(\CO_K) \rightarrow \bG_0(\CO_K/(\pi^m))
\end{equation*}
be the natural reduction map. Then, we set $G_0:=\bG_0(\CO_K)$ and define $$P_I^m:=p_m^{-1}(\bP_{I,0}(\CO_K/(\pi^m)) \subset G_0.$$ Notice that $G_0$ is compact. Moreover, let $\sF_{\mathcal{O}_K}:=\bG_0/\bB_0$.
 
\begin{lemma}Let $I \subsetneq \Delta$ and $m \in \BN$. Then, the subset $Y_I(\epsilon_m)$ is $P_I^m$- invariant. 
\end{lemma}
\begin{proof}
    We identify $\sF^{\rig}$ with the closed points of $\sF$, i.e. for $x \in \sF^{\rig}$ exists a finite extension $L:=k(x)$ of $K$ such that $x \in \sF(L)$. Denote by $\lb \text{\,\,\,} \rb_L$ be the unique absolute value on $L$ which extends the valuation on $K$. 
    % \begin{equation}\label{ramification}
    %     \pi=u \pi_L^e
    % \end{equation}
    % for $u \in \mathcal{O}_L^\times$. 
    Since $\sF_{\mathcal{O}_K}$ is proper over $\mathcal{O}_K$, we have $\sF_{\mathcal{O}_K}(O_L)=\sF_{\mathcal{O}_K}(L)=\sF(L)$. Let 
    $$ q_{m,L}:\sF_{\mathcal{O}_K}(\mathcal{O}_L) \rightarrow \sF_{\mathcal{O}_K}(\mathcal{O}_L/\pi^m\mathcal{O}_L)$$ 
    be the natural projection. The (free) action of $\bG_0$ on $\sF_{\mathcal{O}_K}$ induces the commutative diagram 
    $$\begin{CD}
        \bG_0(\mathcal{O}_L)\times \sF_{\mathcal{O}_K}(\mathcal{O}_L) @> \mathrm{mult.} >>  \sF_{\mathcal{O}_K}(\mathcal{O}_L)\\
        @V (p_{m,L},\, q_{m,L}) V V @VV q_{m,L} V\\
        \bG_0(\mathcal{O}_L/\pi^m\mathcal{O}_L)\times \sF_{\mathcal{O}_K}(\mathcal{O}_L/\pi^m\mathcal{O}_L) @>\mathrm{mult.} >>  \sF_{\mathcal{O}_K}(\mathcal{O}_L/\pi^m\mathcal{O}_L).
    \end{CD}$$
    Moreover, it is clear that $Y_{I,0}:=\bigcup_{w \in \Omega_I} \bB_0w\bB_0/\bB_0$ is the Zariski closure of $Y_I$ in $\sF_{\mathcal{O}_K}$ defined by homogenous polynomial $f_1, \ldots, f_r \in \CO_K[T_0,\ldots, T_N]$ as in Definition \ref{epsnbh}. Then, $Y_{I,0}$ is $\bP_{I,0}$ invariant (cf. Proposition \ref{genSchCe}). Therefore, the above diagram implies that \begin{equation} P^m_I \cdot q_{m,L}^{-1}\big(Y_{I,0}(\mathcal{O}_L/\pi^m\mathcal{O}_L)\big)=q_{m,L}^{-1}\big(Y_{I,0}(\mathcal{O}_L/\pi^m\mathcal{O}_L)\big). \end{equation}
    Next we will show that
    $$ q_{m,L}^{-1}\big(Y_{I,0}(\mathcal{O}_L/\pi^m\mathcal{O}_L)\big)=\{y \in \sF_{\mathcal{O}_K}(\mathcal{O}_L) \mid \lvert f_i(y)\lvert_L\leq \epsilon_m \text{ for all }i\}=Y_I(\epsilon_m)(L).$$
    Let $y \in \sF_{\mathcal{O}_K}(\mathcal{O}_L)$. If $y \in q_{m,L}^{-1}\big(Y_{I,0}(\mathcal{O}_L/\pi^m\mathcal{O}_L)\big)$, it follows that $f_i(q_{m,L}(y))=0$ for all $i$. But $$f_i(q_{m,L}(y))=[f_i(y)]\in \mathcal{O}_L/\pi^m\mathcal{O}_L.$$ Hence, $f_i(y) \in \pi^m\mathcal{O}_L$ and $\lb f_i(y) \rb_L\leq \epsilon_m$ for all $i$. If, on the other hand, $\lb f_i(y)\rb_L\leq \epsilon_m$ for all  $i$, we deduce that $f_i(y) \in \pi^m\mathcal{O}_L$ for all $i$. Thus, $f_i(q_{m,L}(y))=[f_i(y)]=0$ for all $i$ and $y \in q_{m,L}^{-1}\big(Y_{I,0}(\mathcal{O}_L/\pi^m\mathcal{O}_L)\big)$. \\

    \noindent From that we conclude that $Y_I(\epsilon_m)(L)$ is $P_I^m$-invariant for all finite extensions $L$ of $K$. This implies that $Y_I(\epsilon_m)$ is $P_I^m$-invariant. 
\end{proof}
The previous lemma yields a natural $P^m_I$-module structure on $H^i_{Y_I(\epsilon_m)}(\sF^{\rig},\CE_\lambda)$ which we fix. 


\begin{lemma} \label{Bi-Functor} For $I \subsetneq \Delta$ and $i \in \BN_0$, we have  
\begin{equation*} 
    \Big(\varprojlim_{m \in \BN} \Ind_{P_I^m}^{G_0}\big(H^i_{Y_I(\epsilon_m)}(\sF^{\rig},\CE_\lambda)\big)\Big)'= \CF_{P_I}^G\big(H^i_{Y_I}(\sF,\CE_\lambda)\big).
    \end{equation*}
\end{lemma}

\begin{proof}
    We know from (\ref{seqComplete}), that 
    \begin{equation*}
       \varprojlim_{m \in \BN} \big(U(\fku_{\bP_I}^-)_m \otimes_{K} W/\fkd_{m}\big) \cong  \varprojlim_{m \in \BN} \hat{H}^i_{Y_I,m}.
    \end{equation*}
    Furthermore, by Assumption \ref{hypo1} and in view of Corollary \ref{limitloc}, we have 
    $$ \varprojlim_{m \in \BN} H^i_{Y_I(\epsilon_m)}(\sF^{\rig},\CE_\lambda) \cong \varprojlim_{m \in \BN} \big(U(\fku_{\bP_I}^-)_m \otimes_{K} W/\fkd_{m}\big) $$ 
    compatible with the action of $\varprojlim_{m \in \BN} P^m_I=P_{I,0}$ (cf. \cite[Proposition 1.3.10 + Proof]{O2}).  
    Then, we get (cf. \cite[p. 633]{O2})
    \begin{equation*}
        \varprojlim_{m \in \BN} \Ind_{P_I^m}^{G_0}\big(H^i_{Y_I(\epsilon_m)}(\sF^{\rig},\CE_\lambda)\big)\cong \varprojlim_{m \in \BN} \Ind_{P_I^m}^{G_0}\big(U(\fku_{\bP_I}^-)_m \otimes_{K} W/\fkd_{m}\big).
    \end{equation*}
    Passing to the dual, which is exact on $K$-Fréchet spaces (cf. \cite[Section I, Corollary 1.4]{BS}), the required statement follows from \cite[Corollary 3.12]{OSt}. .

    % We know from (\ref{seqComplete}), that 
    % \begin{equation*}
    % \Big(\varprojlim_{m \in \BN} \Ind_{P_I^m}^{G_0}\big(U(\fku_{\bP_I}^-)_m \otimes_{K} W/\fkd_{m}\big)\Big)'  \cong   \Big(\varprojlim_{m \in \BN} \Ind_{P_I^m}^{G_0} \hat{H}^i_{Y_I,m}\Big)'.
    %  \end{equation*}
    % holds. Furthermore, by Assumption \ref{hypo1}, we have 
    % $$\Big(\varprojlim_{m \in \BN} \Ind_{P_I^m}^{G_0} H^i_{Y_I^-(\epsilon_m)}(\sF^{\rig},\CE_\lambda)\Big)' \cong   \Big(\varprojlim_{m \in \BN} \Ind_{P_I^m}^{G_0} \hat{H}^i_{Y_I,m}\Big)'.$$ 
    % Hence, we obtain 
    %     \begin{equation*}
    %         \Big(\varprojlim_{m \in \BN} \Ind_{P_I^m}^{G_0} H^i_{Y_I^-(\epsilon_m)}(\sF^{\rig},\CE_\lambda)\Big)'\cong \Big(\varprojlim_{m \in \BN} \Ind_{P_I^m}^{G_0}\big(U(\fku_{\bP_I}^-)_m \otimes_{K} W/\fkd_{m}\big)\Big)'.
    %     \end{equation*}
    % In view of Corollary \ref{limitloc}, we can replace $H^i_{Y_I^-(\epsilon_m)}(\sF^{\rig},\CE_\lambda)$ with $H^i_{Y_I(\epsilon_m)}(\sF^{\rig},\CE_\lambda)$ in the last isomorphism. Then, the required statement follows from Proposition \ref{alternative}. 
%     We know from Corollary \ref{loccohcatOp} that $H^i_{Y_I}(\sF,\CE_\lambda) \in \mathcal{O}_{\alg}^{\fkp_I}$.
%     Thus, we have an $\CO^{\fkp_I}_{\alg}$-pair $(H^i_{Y_I}(\sF,\CE_\lambda),W)$
%     with a short exact sequence 
%     \begin{equation*}
%        0 \rightarrow \fkd \rightarrow U(\fku_{\bP_I}^-) \otimes_{K} W \longrightarrow H^i_{Y_I}(\sF,\CE_\lambda) \rightarrow 0.
%     \end{equation*}
%     % In Section \ref{{s:FGP}}, w

%     % As in \cite[section 3.8]{OSt}, for $\epsilon \in \lb \overline{\BQ_p^\times} \rb$ we equip $U(\fku_I^-)$ with the norm ${\lb \enspace\, \rb}_\epsilon$ defined by 
%     % \begin{equation*}
%     % \Big \lb \sum_{(i_1,\ldots,i_r) \in \BN_0^r} a_{i_1,\ldots,i_r}y_{\alpha_1}^{i_1}\cdots  y_{\alpha_r}^{i_r} \Big\rb_\epsilon= \sup_{(i_1,\ldots,i_r) \in \BN_0^r}\Big\lb i_1! \cdots i_r! \cdot a_{i_1,\ldots,i_r}\Big\rb \epsilon^{i_1 + \ldots +i_r}.
%     % \end{equation*}

%     % Completing $U(\fku_I^-)$ with respect to ${\lb \enspace\, \rb}_\epsilon$ yields the $\BQ_p$-Banach space
%     % \begin{align*}
%     % U(\fku_I^-)_\epsilon:=\Big\{&\sum_{(i_1,\ldots,i_r) \in \BN_0^r} a_{i_1,\ldots,i_r}y_{\alpha_1}^{i_1}\cdots  y_{\alpha_r}^{i_r}\mid a_{i_1,\ldots,i_r} \in \BQ_p, \\
%     % &\lb i_1! \cdots i_r! \cdot a_{i_1,\ldots,i_r}\rb \epsilon^{i_1 + \ldots +i_r} \rightarrow 0 \text{ for } i_1+\ldots + i_r \rightarrow 0
%     % \Big\}.
%     % \end{align*}
%     % We will write $U(\fku_I^-)_n$ for $U(\fku_I^-)_{\frac{1}{\epsilon_n}}$. 
%     As in in the proof of \cite[Proposition 1.3.10]{O2}, by Assumption \ref{hypo1} 
%     and \cite[Corollary 6 in 1.1.9]{BGR}, we obtain a short exact sequence of $K$-Banach spaces (cf. (\ref{completeUEA}) for the notation) 
%    % \todo[inline]{\textbf{TODO/Question}: Similar to the proof of Lemma 1.3.1 I don't understand how Corollary 6 can applied in this situation. 
%     %I just took it from the proof of proposition 1.3.10.}  
%     \begin{equation*}
%         0 \longrightarrow \fkd_{n} \longrightarrow U(\fku_{\bP_I}^-)_n \otimes_K W \longrightarrow H^i_{Y_I^-(\epsilon_n)}(\sF,\CE)\longrightarrow 0.
%     \end{equation*}
%     for every $n \in \BN$. Hence, one has 
%     \begin{equation*}
%        \varinjlim_{n \in \BN} \Ind_{P_I^n}^{G_0} H^i_{Y_I^-(\epsilon_n)}(\sF^{\rig},\CE_\lambda)'\cong \Big(\varprojlim_{n \in \BN} \Ind_{P_I^n}^{G_0}\big(U(\fku_{\bP_I}^-)_n \otimes_{K} W/\fkd_{n}\big)\Big)'.
%     \end{equation*}x
%     In view of Lemma \ref{limitloc}, we can replace $H^i_{Y_I^-(\epsilon_n)}(\sF^{\rig},\CE_\lambda)$ with $H^i_{Y_I(\epsilon_n)}(\sF^{\rig},\CE_\lambda)$ in the last isomorphism. Then, the required statement follows from Proposition \ref{alternative}. 
\end{proof}


\subsection{Results}\label{s:results}
 We start by recalling Orlik's fundamental complex on $Y_{\acute{e}t}$, the étale site on $Y$. This is taken from \cite[Section 6.2.1/6.2.2]{CDHN} which is based on \cite[Section 3]{O1}. \\

 For the constant étale sheaf $\BZ \in \Sh(Y_{\acute{e}t})$ and a closed pseudo-adic subspace $Z$ of $Y$ with inclusion $i:Z \rightarrow Y$, define $\BZ_Z:=i_*i^*(\BZ)$.

 \begin{definition}\cite[Definition 6.7]{CDHN} Let $I \subsetneq \Delta$. Define $\BZ_I \in \Sh(Y_{{\acute{e}t}})$ as the \textit{subsheaf of locally constant sections}
of $\prod_{g \in \bG/\bP_I(K)} \BZ_{gY_I^{\ad}}$, i.e. 
$$ \BZ_I= \varinjlim_{c \in \sC_I} \BZ_c$$
the limit being taken over the (pseudo-filtered) category $\sC_I$ of compact open disjoint coverings of $\bG/\bP_I(K)$ ordered by refinement where $\BZ_c$ denotes the image of the natural embedding $\bigoplus_{j \in A} \BZ_{Z_I^{T_j}} \hookrightarrow \prod_{g \in \bG/\bP_I(K)} \BZ_{gY_I^{\ad}}$ for $c=\{T_j\}_{j \in A} \in \sC_I$. 
 \end{definition}

Let $ I \subset I' \subsetneq \Delta$ and $\pi_{I,I'}:\bG/\bP_I(K)\rightarrow \bG/\bP_{I'}(K)$ be the natural surjection. Then, for all $g \in \bG/\bP_I(K)$ and $h \in \bG/\bP_{I'}(K)$, we have a natural morphism 
$\BZ_{gY_I^{\ad}} \rightarrow \BZ_{hY_{I'}^{\ad}}$ which is trivial if $\pi_{I,I'}(g) \neq h$ and otherwise, it coincides with the map induced by the closed embedding $gY_I \hookrightarrow hY_{I'}$.
Then, by definition, we get a natural morphism $$p_{I,I'}:\BZ_{I'}\rightarrow \BZ_I.$$ Fix an ordering on $\Delta$. Assuming that$\lb I' \rb- \lb I \rb =1 $ and $I'=\{\alpha_1 < \ldots < \alpha_r\}$ we set 
$$d_{I,I'}:= \left\{\begin{array}{lll} (-1)^i p_{I,I'} &\text{ if } &I'=I \cup \{\alpha_i\}, \\
                            0 &\text{ else.}  \end{array}\right.$$
This defines by standard procedure the following complex
\begin{equation}\label{fundamentalcomplex}
    0 \longrightarrow \BZ \longrightarrow \bigoplus\limits_{\substack{I \subset \Delta \\ \lb \Delta \backslash I \rb = 1 }} \BZ_I \longrightarrow \bigoplus\limits_{\substack{I \subset \Delta \\ \lb \Delta \backslash I \rb = 2 }} \BZ_I \longrightarrow \ldots \longrightarrow \bigoplus\limits_{\substack{I \subset \Delta \\ \lb \Delta \backslash I \rb = \lb \Delta \rb -1 }} \BZ_I \longrightarrow \BZ_\emptyset \longrightarrow 0
\end{equation}
on $Y_{\acute{e}t}$ 
%  Consider the topological exact sequence of locally convex $E$-vector spaces with continuous $G$-action 
% \begin{align*}
% 0 \longrightarrow H^0_Y(\sF^\mathrm{rig}, \CE) \longrightarrow &H^0(\sF^\mathrm{rig}, \CE) \longrightarrow H^0(\sF^{\mathrm{wa}},\CE) \longrightarrow \\ &H^1_Y(\sF^\mathrm{rig}, \CE) \longrightarrow H^1(\sF^\mathrm{rig}, \CE) \longrightarrow H^1(\sF^{\mathrm{wa}},\CE) \longrightarrow  \ldots
% \end{align*}
% We see that by computing $H^*_Y(\sF^\mathrm{rig}, \CE)$ we can tackle $H^*(\sF^{\mathrm{wa}},\CE)$. 
%\todo[inline]{\textbf{TODO}: Define everything for the fundamental complex}
% On $Y$ we have the following (fundamental) complex of sheaves (in the notation of \cite[chapter 6]{CDHN})
% \begin{equation}\label{fundamentalcomplex}
%     0 \longrightarrow \BZ \longrightarrow \bigoplus\limits_{\substack{I \subset \Delta \\ \lb \Delta \backslash I \rb = 1 }} \BZ_I \longrightarrow \bigoplus\limits_{\substack{I \subset \Delta \\ \lb \Delta \backslash I \rb = 2 }} \BZ_I \longrightarrow \ldots \longrightarrow \bigoplus\limits_{\substack{I \subset \Delta \\ \lb \Delta \backslash I \rb = \lb \Delta \rb -1 }} \BZ_I \longrightarrow \BZ_\emptyset \longrightarrow 0
% \end{equation}
which is acyclic by  \cite[Theorem 6.9]{CDHN}. It is referred to as the \textit{fundamental complex}. \\

Denote by $\iota:Y \hookrightarrow \sF^{\mathrm{ad}}$ the closed embedding. Then, by \cite[Exp. I, Proposition 2.3]{SGA2}, we have 
\begin{equation*}
    \Ext^*(\iota_*(\BZ_{Y}),\CE_\lambda) \cong H^*_{Y}(\sF^{\mathrm{ad}},\CE_\lambda). 
\end{equation*}
By applying $\Ext^*(\iota_*(-),\CE_\lambda)$ to the complex (\ref{fundamentalcomplex}), we get the spectral sequence 
\begin{equation}\label{spectral}
\hat{E}_1^{-p,q}=\Ext^q(\bigoplus\limits_{\substack{I \subsetneq \Delta \\ \lb \Delta \backslash I \rb = p+1 }} \iota_*(\BZ_I), \CE_\lambda) \Rightarrow \Ext^{-p+q}(\iota_*(\BZ_{Y}),\CE_\lambda)=H^{-p+q}_{Y}(\sF^{\mathrm{ad}},\CE_\lambda). 
\end{equation}
For the $\hat{E}_1$-terms, we have the following identification. 

\begin{proposition}\label{extgroups} For all $I \subsetneq \Delta$, there exists an isomorphism 
    \begin{equation*}
        \Ext^*(\iota_*(\BZ_I), \CE_\lambda) \cong \varprojlim_{m \in \BN} \Ind_{P_I^m}^{G_0} \big( H^*_{Y_I(\epsilon_m)}(\sF^{\rig},\CE_\lambda)\big).
    \end{equation*}
\end{proposition}
% For the proof, we have to make another assumption/conjecture (cf. \cite[Lemma 2.2.3]{O2}).

% \begin{hac}\label{hypo2}
%     The projective system $$\Big(\bigoplus_{g \in G_0/P_I^m}H^i_{Z^{gP^m_I}}(\sF^{\ad},\CE_\lambda)\Big)_{m \in \BN}$$ consists of $K$-Fréchet spaces and 
%     satisfies the topological Mittag-Leffler property for all $i\geq 0$.  
% \end{hac}
\begin{proof} This is essentially the proof of \cite[Proposition 2.2.1]{O2}, where the Drinfeld case is treated. 
    The family $$\{gP^m_I \mid g \in G_0,\, m \in \BN \}$$ 
    of compact open subsets in $G_0/P_I$ yields $$\BZ_I = \varinjlim_{m\in \BN} \bigoplus_{g \in G_0/P_I^m}\BZ_{Z^{gP^m_I}}.$$ 
    Then, by choosing an injective resolution $\CI^\bullet$ of $\CE_\lambda$, we have 
    \begin{align*}
        \Ext^i( \iota_*(\BZ_I), \CE_\lambda)&=H^i\big(\Hom(\iota_*(\BZ_I),\CI^\bullet)\big)= H^i\big(\Hom(\iota_*(\varinjlim_{m\in \BN} \bigoplus_{g \in G_0/P_I^m}\BZ_{Z_I^{gP^m_I}}),\CI^\bullet)\big) \\
                                &=H^i\big(\varprojlim_{m\in \BN} \bigoplus_{g \in G_0/P_I^m} \Hom(\iota_*(\BZ_{Z_I^{gP^m_I}}), \CI^\bullet)\big)\\&=H^i\big(\varprojlim_{m\in \BN}\bigoplus_{g \in G_0/P_I^m} H^0_{Z_I^{gP^m_I}}(\sF^{\ad}, \CI^\bullet)\big).
    \end{align*}
    We set $Z_{I,m}:=P^m_I\cdot Y_I^{\rig} \subset \sF^{\rig}$ for $m \in \BN$. Then, we have chains of open admissible subsets 
    $$\ldots \sF^{\rig} \backslash Z_{I,m} \subset \sF^{\rig} \backslash Z_{I,{m+1}} \subset \ldots $$
    and  
    $$ \ldots \sF^{\rig} \backslash  Y_I(\epsilon_m) \subset \sF^{\rig} \backslash Y_I(\epsilon_{m+1}) \subset  \ldots $$ 
    which each cover $\sF^{\rig}\backslash Y_I^{\rig}$. Then, we know from the proof of \cite[Section 2, Proposition 4]{SS} that 
    $$ \varprojlim_{m\in \BN} H^0_{Z_{I,m}}(\sF^{\rig}, \CI^p)= H^0_{Y_I^{\rig}}(\sF^{\rig}, \CI^p)= \varprojlim_{m\in \BN} H^0_{Y_I(\epsilon_m)}(\sF^{\rig}, \CI^p).$$  The same holds for translates of  $Z_{I,m}$ and $Y_I(\epsilon_m)$. Hence, we get (cf. \cite[p. 1415]{O6}) 
    $$  \varprojlim_{m\in \BN} \bigoplus_{g \in G_0/P_I^m} H^0_{gZ_{I,m}}(\sF^{\rig}, \CI^p) \cong  \varprojlim_{m\in \BN} \bigoplus_{g \in G_0/P_I^m} H^0_{gY_I(\epsilon_m)}(\sF^{\rig}, \CI^p) $$ 
    for all injective sheafs of the resolution $\CI^\bullet$. Therefore, using that $\sF^{\rig}$ and $\sF^{\ad}$ have equivalent topoi (cf. \cite[Proposition 2.1.4]{Hu}), we get by functoriality an isomorphism of complexes 
    $$ \varprojlim_{m\in \BN}\bigoplus_{g \in G_0/P_I^m} H^0_{Z_I^{gP^m_I}}(\sF^{\ad}, \CI^\bullet) \cong \varprojlim_{m\in \BN}\bigoplus_{g \in G_0/P_I^m} H^0_{gY_I(\epsilon_m)}(\sF^{\rig}, \CI^\bullet).$$ 
    This implies 
    $$ \Ext^i( \iota_*(\BZ_I), \CE_\lambda)= H^i\big(\varprojlim_{m\in \BN}\bigoplus_{g \in G_0/P_I^m} H^0_{gY_I(\epsilon_m)}(\sF^{\rig}, \CI^\bullet)\big).$$

Before we can continue, we need a technical lemma where $\varprojlim_{m \in \BN}^{(r)}$ denote the $r$-th right derived functor of ${\varprojlim_{m \in \BN}}$.
\begin{lemma} Let $\CI$ be an injective sheaf on $\sF^\ad$. Then,
    $$ {\varprojlim_{m \in \BN}}^{(r)}\big(\bigoplus_{g \in G_0/P_I^m} H^0_{gY_I(\epsilon_m)}(\sF^{\rig}, \CI)\big)=0 \text{ for all } r \geq 1.$$
   % \todo[inline]{\textbf{TODO}: The plan is to adapt the proof of Proposition 2.2.1 in \cite{OS}, but there are steps which I don't understand.}
\end{lemma}
\begin{proof}
    It is sufficient to reproduce the proof of  \cite[Lemma 2.2.2]{O2}.
\end{proof}

Then, with the two standard hypercohomology spectral sequences 
% \begin{align*}
% E_1^{pq}&={\varprojlim_{m \in \BN}}^{(q)}(\bigoplus_{g \in G_0/P_I^m} H^0_{Z^{gP^m_I}}(\sF^{\rig}, \CI^p))\Rightarrow \bH^{p+q}{\varprojlim_{m \in \BN}} (\bigoplus_{g \in G_0/P_I^m} H^0_{Z^{gP^m_I}}(\sF^{\rig}, \CI^\bullet)), \\
% E_2^{pq}&={\varprojlim_{m \in \BN}}^{(p)}H^q(\bigoplus_{g \in G_0/P_I^m} H^0_{Z^{gP^m_I}}(\sF^{\rig}, \CI^\bullet))\Rightarrow \bH^{p+q}{\varprojlim_{m \in \BN}} (\bigoplus_{g \in G_0/P_I^m} H^0_{Z^{gP^m_I}}(\sF^{\rig}, \CI^\bullet)), 
% \end{align*}
\begin{align*}
    E_1^{pq}&={\varprojlim_{m \in \BN}}^{(q)}\big(\bigoplus_{g \in G_0/P_I^m} H^0_{gY_I(\epsilon_m)}(\sF^{\rig}, \CI^p)\big)\Rightarrow \bH^{p+q}{\varprojlim_{m \in \BN}} \big(\bigoplus_{g \in G_0/P_I^m} H^0_{gY_I(\epsilon_m)}(\sF^{\rig}, \CI^\bullet)\big), \\
    E_2^{pq}&={\varprojlim_{m \in \BN}}^{(p)}H^q\big(\bigoplus_{g \in G_0/P_I^m} H^0_{gY_I(\epsilon_m)}(\sF^{\rig}, \CI^\bullet)\big)\Rightarrow \bH^{p+q}{\varprojlim_{m \in \BN}} \big(\bigoplus_{g \in G_0/P_I^m} H^0_{gY_I(\epsilon_m)}(\sF^{\rig}, \CI^\bullet)\big), 
    \end{align*}
and by knowing that $\varprojlim_{m \in \BN}^{(p)}=0$ for $p\geq 2$ (cf. \cite{Je}), we get the following short exact sequence (cf. \cite[Section 2, Proof of Proposition 4]{SS})
\begin{align*} 0 \rightarrow {\varprojlim_{m \in \BN}}^{(1)}\bigoplus_{g \in G_0/P_I^m} H^{i-1}_{gY_I(\epsilon_m)}(\sF^{\rig}, \CE_\lambda) &\rightarrow \Ext^i( i_*(\BZ_I), \CE_\lambda) \\ &\rightarrow {\varprojlim_{m \in \BN}}\bigoplus_{g \in G_0/P_I^m} H^{i}_{gY_I(\epsilon_m)}(\sF^{\rig}, \CE_\lambda) \rightarrow 0
\end{align*}
for all $i \in \BN$. Moreover, Assumption \ref{hypo1} and Corollary \ref{limitloc}, respectively, imply that the projective system of $K$-Fréchet spaces $\big(\bigoplus_{g \in G_0/P_I^m} H^{i}_{gY_I(\epsilon_m)}(\sF^{\rig}, \CE_\lambda)\big)_{m \in \BN}$ satisfies the topological Mittag-Leffler property for all $i\geq 0$ (cf. \cite[p. 626]{O2}). Therefore, by \cite[Remark 13.2.4]{EGAIII}, the ${\varprojlim_{m \in \BN}}^{(1)}$-term vanishes, i.e. 

% Moreover, Hypothesis \ref{hypo2} implies, by \cite[Remark 13.2.4]{EGAIII}, that the ${\varprojlim_{m \in \BN}}^{(1)}$-term vanishes, i.e. 
$$ \Ext^i( \iota_*(\BZ_I), \CE_\lambda) \cong {\varprojlim_{m \in \BN}}\bigoplus_{g \in G_0/P_I^m} H^{i}_{gY_I(\epsilon_m)}(\sF^{\rig}, \CE_\lambda).$$
% $$ \Ext^i( \iota_*(\BZ_I), \CE_\lambda) \cong {\varprojlim_{m \in \BN}}\bigoplus_{g \in G_0/P_I^m} H^{i}_{Z^{gP^m_I}}(\sF^{\ad}, \CE_\lambda).$$
% Notice that one has 
% $$\bigcap_{m\in \BN} Z^{P^m_I}=Y_I^\ad=\bigcap_{m\in \BN} Y_I(\epsilon_m)^\ad.$$
% Thus,  from \cite[Proposition 1.3.3]{O2}, we can deduce that 
% \begin{align} 
%     {\varprojlim_{m \in \BN}}  H^{i}_{Z^{P^m_I}}(\sF^{ad}, \CE_\lambda) &\cong {\varprojlim_{m \in \BN}}  H^{i}_{Y_I(\epsilon_m)}(\sF^{\rig}, \CE_\lambda) \nonumber. \label{ind}
% \end{align}
% Hence, we obtain 
% \begin{align} 
%     {\varprojlim_{m \in \BN}}\bigoplus_{g \in G_0/P_I^m} H^{i}_{Z^{gP^m_I}}(\sF^{ad}, \CE_\lambda) 
%     \cong {\varprojlim_{m \in \BN}}\bigoplus_{g \in G_0/P_I^m} H^{i}_{gY_I(\epsilon_m)}(\sF^{\rig}, \CE_\lambda).
% \end{align}
% Here, we used that $\sF^{\rig}$ and $\sF^{\ad}$ have equivalent topoi (cf. \cite[Proposition 2.1.4]{Hu}). 
The statement of the proposition is then just rewriting the latter term. 
\end{proof}

\begin{proposition}
    We have a spectral sequence 
    \begin{equation*}\label{spectral2}
        {E}_1^{-p,q}=\bigoplus\limits_{\substack{I \subset \Delta \\ \lb \Delta \backslash I \rb = p }}\varprojlim_{m \in \BN} \Ind_{P_I^m}^{G_0} \big(H^q_{Y_I(\epsilon_m)}(\sF^{\rig},\CE_\lambda)\big)\Rightarrow H^{-p+q}(\sF^{\wa},\CE_\lambda)
    \end{equation*}
    where we use the abbreviation $Y_\Delta$ for $\sF$.
\end{proposition}
\begin{proof}
We follow the arguments used in the proof of \cite[Proposition 4.2]{O3}. First, we consider the second quadrant double complex $(\tilde{E}_1^{\bullet, \bullet},d^{\bullet, \bullet}, d'^{\bullet, \bullet})$ defined by 
$$ \tilde{E}_1^{p, q}= \begin{cases}  H^q(\sF^{\rig},\CE_\lambda) &\text{if } p=0 \\
    0 &\text{else, }   \end{cases}$$ 
with all differentials being trivial. Hence, it defines a spectral sequence converging to $H^*(\sF^{\rig},\CE_\lambda)$. Further, let $I \subset \Delta$ such that $\lb\Delta \backslash I\rb=1$ and $m \in \BN$. The inclusion $gY_I(\epsilon_m) \subset \sF^{\rig}$ induces a morphism (cf. \cite[Lemma 1.3]{vP}) 
$$ H^*_{gY_I(\epsilon_m)}(\sF^{\rig}, \CE_\lambda) \rightarrow H^*(\sF^{\rig}, \CE_\lambda)$$
for $g \in G/P^m_I$. Then, by the universal property of the direct sum we get a morphism 
$$ \bigoplus_{g \in G_0/P_I^m} H^{*}_{gY_I(\epsilon_m)^{\rig}}(\sF^{\rig}, \CE_\lambda) \rightarrow H^*(\sF^{\rig}, \CE_\lambda).$$
Thus, the functoriality of  ${\varprojlim_{m \in \BN}}$  yields 
$$  D_I^*:{\varprojlim_{m \in \BN}} \bigoplus_{g \in G_0/P_I^m} H^{*}_{gY_I(\epsilon_m)^{\rig}}(\sF^{\rig}, \CE_\lambda)\rightarrow  {\varprojlim_{m \in \BN}}  H^*(\sF^{\rig}, \CE_\lambda)=H^*(\sF^{\rig}, \CE_\lambda).$$ 
Then, we consider the morphism of double complexes (cf. (\ref{spectral}))
$$ f_1^{\bullet,\bullet}: \hat{E}_1^{{\bullet,\bullet}} \rightarrow \tilde{E}_1^{\bullet, \bullet}$$ 
given by 
$$ f_1^{p,q}=\begin{cases} \oplus_I D_I^q & \text{if }p=0 \\
   0 & \text{else.} \end{cases}$$ 
It induces the morphism of total complexes 
$$ \Tot(f_1^{\bullet,\bullet}): \Tot(E_1^{{\bullet,\bullet}}) \rightarrow \Tot(\tilde{E}_1^{\bullet, \bullet}) $$
where we denote the mapping cone of $\Tot(f_1^{\bullet,\bullet})$ by $\mathrm{Cone}(\Tot(f_1^{\bullet,\bullet})) ^\bullet$. By the definitions, the triangle for this mapping cone induces a long exact sequence which identifies with 
$$ \ldots \rightarrow H^q_{Y}(\sF^{\rig},\CE_\lambda) \rightarrow 
H^q(\sF^{\rig},\CE_\lambda) \rightarrow H^q(\sF^{\wa},\CE_\lambda)\rightarrow \ldots. $$ 
Hence, the cohomology of $\mathrm{Cone}(\Tot(f_1^{\bullet,\bullet})) ^\bullet$ coincides with $H^*(\sF^{\wa},\CE_\lambda)$. Furthermore, the total complex of the double complex ${E}_1^{\bullet,\bullet}$ in the statement is exactly $\mathrm{Cone}(\Tot(f_1^{\bullet,\bullet})) ^\bullet$ which finishes the proof. 
\end{proof}
% With the arguments used in the proof of \cite[Satz 4.2]{O3} we then get 
% \begin{equation}\label{spectral2}
%     E_1^{-p,q}=\bigoplus\limits_{\substack{I \subset \Delta \\ \lb \Delta \backslash I \rb = p }}\varprojlim_{n \in \BN} \Ind_{P_I^n}^{G_0} H^*_{Y_I(\epsilon_n)}(\sF^{\rig},\CE_\lambda) \Rightarrow H^{-p+q}(\sF^{\wa},\CE_\lambda)
% \end{equation}
% where we use the abbreviation $Y_\Delta=\sF$. \textbf{TODO formulate as proposition+proof} \\

% From now on we assume that $\bG=\mathbf{GL_n}$ for $n \in  \BN$ with simple roots $$\alpha_i=e_i-e_{i+1} \in \BQ^n\cong X^*(\bT)_\BQ.$$
% Then we identify $\mu$ under the isomorphism $X_*(\bT)_\BQ\cong \BQ^n$ with $(\mu_1,\ldots, \mu_n) \in \BQ^{n}$. Here by assumption we have $\mu_i>\mu_{i+1}$ for all $i$ and $\mu=\sum n_\alpha \alpha^{\vee}$ with $n_\alpha >0$ for all $\alpha$. \\ 
Before stating and proving the main theorem we need the following lemma. 


\begin{lemma}\label{help} Let $I \subsetneq \Delta$ and $w \in W^I \cap \Omega_I$. Then, $w \in \Omega_\emptyset$. 
\end{lemma}
\begin{proof}
    By Lemma \ref{n_a}, we know that $$\mu=\sum_{\alpha \in \Delta} n_\alpha \alpha^{\vee}$$ for $n_\alpha \in \BQ_{>0}$. 
    As $W$ acts by permutation on $\Phi$, we have  that $$w\mu=\sum_{\alpha \in \Delta} m_\alpha \alpha^{\vee}$$ with $m_\alpha \in \BQ$ for $w \in W^I \cap \Omega_I$.
    Then, by Lemma \ref{fweightandpairing} it is enough to show that $ \langle \check{\varpi}_{\alpha}, w\mu  \rangle_\der = m_\alpha>0$ for all $\alpha \in I$. By (\ref{linearcombi}), we have 
    $$ \alpha = \sum_{\beta \in \Delta} \langle \alpha, \beta^\vee{}\rangle_\der \check{\varpi}_\beta$$
    for $\alpha \in \Delta$. Moreover, from Lemma \ref{Kostant} we know that $w^{-1}\alpha \in \Phi^+ $ for all $\alpha \in I$. Since we assumed $\mu$ to lie in the positive Weyl chamber (cf. (\ref{positiveChamber})), we have 
    $$\langle \alpha, w\mu \rangle= \langle w^{-1}\alpha, \mu \rangle> 0$$ for $\alpha \in I$. By the very definition of $\langle \text{ , } \rangle_\der$ it follows that 
    $$ \langle \alpha, w\mu \rangle_\der > 0$$
    for all $\alpha \in I$. 
    Furthermore, for $\alpha, \beta \in \Delta$ and $\alpha \neq \beta$, we know by Lemma \ref{rootrel} that $\langle \alpha, \beta^\vee \rangle_\der \leq 0$. Recall that $w \in W^I \cap \Omega_I$ implies that $\langle \check{\varpi}_\beta  , w\mu \rangle_\der>0$ for $\beta \in \Delta \backslash I$ by Lemma \ref{fweightandpairing}. 
    Thus, for $\alpha \in I$ and $w \in W^I \cap \Omega_I$, we have that  
    \begin{equation}\label{b}
        b_\alpha:=\sum_{\beta \in I} \langle \alpha,  \beta^{\vee} \rangle_\der \langle \check{\varpi}_\beta, w\mu  \rangle_\der = \langle \alpha , w\mu \rangle_\der - \sum_{\beta \in \Delta \backslash I} \langle \alpha, \beta^{\vee} \rangle_\der \langle \check{\varpi}_\beta  , w\mu \rangle_\der > 0.
     \end{equation}
    We fix an ordering on $I=\{\alpha_1 > \alpha_2 > \ldots > \alpha_r \}$ and define 
    \begin{align*}
    &C \in \BQ^{\lb I \rb \times \lb I \rb } \text{ with } C_{ij}:=\langle \alpha_i, \alpha_{j}^\vee  \rangle_\der,  \\ 
    &x:=(m_{\alpha_i})_{i \in \{1, \ldots, r\}} \in \BQ^{\lb I \rb}, \\
     &b:=(b_{\alpha_i})_{i \in \{1, \ldots, r\}} \in \BQ^{\lb I \rb}.
    \end{align*}
    Then, 
    \begin{equation}\label{LGS}
    Cx=b. 
    \end{equation}
    After reordering the simple roots, if necessary, we can assume that $C$ has blocks $C_1,\ldots, C_t$ on the main diagonal and has zeroes everywhere else.
    Then, the $C_i$'s are the (transposed) Cartan matrices (cf. (\ref{cartanmatrix})) of the irreducible components of the Dynkin diagram of $\Phi_I$. Thus, $C^{-1}$ has blocks $C_i^{-1}$ on the main diagonal
    and has zeroes everywhere else. The entries of the $C_i^{-1}$ are, by Lemma \ref{inversecartan}, known to be positive rational. Then, (\ref{b}) and (\ref{LGS}) imply immediately that $m_\alpha>0$ for all $\alpha \in I$.
\end{proof}


    % We have to show  that $(w\mu, \varpi_\alpha ) >0$ for all $\alpha \in I$.
    % For that we write $\Delta \backslash I=\{\alpha_{i_1}, \ldots, \alpha_{i_k}\}$ 
    % with $i_l<i_{l+1}$ for all $l$. In the following we will frequently use that since $w \in W^I$ and $\mu$ lies in the positive Weyl chamber 
    % one has $\langle w\mu, \alpha \rangle \geq 0$ for all $\alpha \in I$.
    
    % \begin{equation}\label{chamber}
    %     \mu_{w^{-1}(j)}-\mu_{w^{-1}(j+1)}\geq0
    % \end{equation}
    % for all $a_j \in I$. Now assume there is an $\alpha_{j} \in I$ with $(w\mu, \varpi_{\alpha_j})\leq 0$. 
    % There are three possible cases for $j$ and we consider for each a system of inequalities \\

    % \underline{$j<i_1:$}
    % \begin{align*}
    % \text{I) }&0\geq (w\mu, \varpi_{\alpha_j})=\sum_{i=1}^{j} \mu_{w^{-1}(i)} \\
    % \text{II) }&0<(w\mu, \varpi_{\alpha_{i_1}})-(w\mu, \varpi_{\alpha_j})=\sum_{i=j+1}^{i_1} \mu_{w^{-1}(i)}.
    % \end{align*}

    % \underline{$i_l<j<i_{l+1}:$}
    % \begin{align*}
    %     \text{I) }&0>(w\mu, \varpi_{\alpha_j})-(w\mu, \varpi_{\alpha_{i_{l}}})=\sum_{i=i_l+1}^{j} \mu_{w^{-1}(i)}, \\
    %     \text{II) }&0<(w\mu, \varpi_{\alpha_{i_{l+1}}})-(w\mu, \varpi_{\alpha_j})=\sum_{i=j+1}^{i_{l+1}} \mu_{w^{-1}(i)}.
    % \end{align*}

    % \underline{$i_k<j:$}
    % \begin{align*}
    %     \text{I) }&0>(w\mu, \varpi_{\alpha_j})-(w\mu, \varpi_{\alpha_{i_{k}}})=\sum_{i=i_k+1}^{j} \mu_{w^{-1}(i)}, \\
    %     \text{II) }&0\leq\sum_{i=1}^n \mu_i - (w\mu, \varpi_{\alpha_j})= \sum_{i=j+1}^{n} \mu_{w^{-1}(i)}.
    % \end{align*}
    % In each case (\ref{chamber}) implies that $\mu_{w^{-1}(j)}-\mu_{w^{-1}(j+1)}< 0$ which is a contradiction to (\ref{chamber}).



\begin{theorem}\label{theorem1}
    Let $i_0:=\dim\sF-\lb\Delta\rb$. The homology of the (chain) complex 
    $$C_\bullet:  \bigoplus_{\substack{w \in \Omega_\emptyset \\ l(w)=\dim Y_\emptyset}}V^G_B(w) \leftarrow \ldots \leftarrow \bigoplus_{\substack{w \in \Omega_\emptyset \\ \ l(w)=1}} V_B^G(w) \leftarrow  V^G_B(\lambda) $$
    starting in degree $i_0$ coincides with $H^*(\sF^{\wa},\CE_\lambda)'$, i.e. $H_i(C_\bullet)=H^i(\sF^{\wa},\CE_\lambda)'$.
\end{theorem}

\begin{proof}

We consider the double complex $D_{\bullet, \bullet}$, similar to the one from \cite[p. 662]{OSch}, defined as  a second quadrant double chain complex, given by 
\begin{equation}\label{double}
D_{p,q}= \bigoplus_{{\substack{I \subset \Delta \\ \lb \Delta \backslash I \rb = -p}}} \bigoplus_{{\substack{w \in W^I \cap \Omega_I \\ l(w)=n-q}}} I_{P_I}^G(w) \, \Big(=\bigoplus_{{\substack{I \subset \Delta \\ \lb \Delta \backslash I \rb = -p}}} \bigoplus_{{\substack{w \in W^I \cap \Omega_I \\ l(w)=n-q}}}\CF^G_{P_I}\big(H^q_{C_I(w)}(\sF,\CE_\lambda)\big)\Big)
\end{equation}
(cf. (\ref{twisted}) for the objects). The vertical differentials are the ones coming from Lemma \ref{cohY_I}. The horizontal ones come from the transition maps $$H^q_{C_I(w)}(\sF,\CE_\lambda) \rightarrow H^q_{C_{I'}(w)}(\sF,\CE_\lambda)$$ for $I \subset I'$ and $w \in W^{I'}$ induced by the fact that $C_I(w) \subset C_{I'}(w)$ is a closed subset. They are the same as in Example \ref{genVM}. The commutativity is shown as in the proof of \cite[Theorem 4.2]{OSch}. \\
We are especially interested in the two spectral sequences converging towards the homology of the total complex $\Tot(D_{\bullet, \bullet})$ associated to $D_{\bullet, \bullet}$.  Namely,  
\begin{align*} {}^IE_{p,q}^0=D_{p,q} &\Rightarrow H_{p+q}(\Tot(D_{\bullet, \bullet})), \\
               {}^{II}E_{p,q}^0=D_{q,p} &\Rightarrow H_{p+q}(\Tot(D_{\bullet, \bullet})). 
\end{align*}
Then, by Lemma \ref{cohY_I} and Lemma \ref{Bi-Functor} in combination with the functoriality and the exactness of the functor $\CF_P^G$ (cf. \ref{exact}), 
we see that ${}^IE^1_{\bullet,\bullet}=(E_1^{\bullet,\bullet})'$ (cf. Proposition \ref{spectral2}). We know from Proposition \ref{extgroups} that the entries of $E_1^{\bullet,\bullet}$ are $K$-Fréchet spaces. Furthermore, the duality functor is exact on the category of $K$-Fréchet spaces (cf. \cite[Section I, Corollary 1.4]{BS}). Hence, $H_{p}(\Tot(D_{\bullet, \bullet}))\cong H^{p}(\sF^{\wa},\CE_\lambda)'.$ Due to Lemma \ref{help}, we have $${}^{II}E^0_{p,\bullet}=\bigoplus_{\substack{w \in \Omega_\emptyset \\ l(w)=n-p}} E^{0,w}_{p,\bullet}$$
with chain complexes
\begin{equation} \label{reso}
    E^{0,w}_{p,\bullet}:  I_{P_{I(w)}}^G(w) \rightarrow \bigoplus_{\substack{ I \subset I(w) \\ \lb I(w)\backslash I \rb =1 }} I_{P_{I}}^G(w)\rightarrow \ldots \rightarrow \bigoplus_{\substack{ I \subset I(w) \\ \lb I \rb =1 }} I_{P_{I}}^G(w) \rightarrow I_B^G(w)
\end{equation}
ending in degree $-\lb \Delta \rb$. From Corollary \ref{relativeresolution}, we know that these complexes are exact except at the very right position where the cokernel is $V_B^G(w)$. Thus, we get $${}^{II}E^1_{p,q}= \begin{cases} \bigoplus_{\substack{ w \in \Omega_\emptyset \\ l(w)=n-p }} V_{B}^G(w) &\text{ if } q=-\lb \Delta \rb, \\
0 &\text{ else.}   
\end{cases}$$
Therefore, ${}^{II}E^2={}^{II}E^\infty$ and we are done. 
\end{proof}
\begin{corollary}
    Let $i_0:=\dim\sF-\lb\Delta\rb$. Then, 
    $H^{i_0}(\sF^{\wa}, \CE_\lambda) \neq 0.$ 
\end{corollary}

\begin{proof}
   We know from \cite[Corollary 4.3]{OSch} that 
   $$v^G_B(\lambda)=\Ker\Big(V^G_B(\lambda) \rightarrow \bigoplus_{\substack{w \in W \\ \ l(w)=1}}V^G_B(w)\Big).$$ 
   But then it follows from the previous theorem that 
     $$v^G_B(\lambda)=\Ker\Big(V^G_B(\lambda) \rightarrow \bigoplus_{\substack{w \in W \\ \ l(w)=1}}V^G_B(w)\Big) \subset \Ker\Big(V^G_B(\lambda) \rightarrow \bigoplus_{\substack{w \in \Omega_\emptyset \\ \ l(w)=1}}V^G_B(w)\Big)=H^{i_0}(\sF^{\wa}, \CE_\lambda)'.$$ 
   Therefore, $H^{i_0}(\sF^{\wa}, \CE_\lambda)$ cannot be trivial. 
\end{proof}

\begin{lemma} \label{surjection} Let $w, w' \in \Omega_\emptyset$ with $w' \leq w$ and $l(w)=l(w')+1$. Then, the morphism 
    $$p_{w',w}:V^G_B(w')\rightarrow V^G_B(w)$$
 appearing in the differentials of $C_\bullet$ is surjective. 
 \end{lemma}
 \begin{proof} As seen in the proof of Theorem \ref{theorem1}, the morphism $p_{w',w}:V^G_B(w')\rightarrow V^G_B(w)$ is induced by a morphism $\varphi:I_{B}^G(w') \rightarrow I_{B}^G(w)$. This one in turn comes from a non-trivial morphism $$i_{w,w'}:M(w \cdot \lambda )=H^{n-l(w)}_{C(w)}(\sF,\CE_\lambda) \rightarrow H^{n-l(w')}_{C(w')}(\sF,\CE_\lambda)=M(w' \cdot \lambda )$$ (cf. Remark \ref{morphismBGG}). Thus, $i_{w,w'}$ is injective (cf. \cite[p. 46]{B}) and therefore $\varphi=\CF^G_B(i_{w,w'})$ is surjective (cf. Proposition \ref{exact}). Then, we have the following commutative diagram 
 $$\begin{CD}
 I_{B}^G(w')@>\CF_B^G(i_{w,w'})>>  I_{B}^G(w) \\ 
 @VV\pi V @VV \pi V \\
 V^G_B(w')@>p_{w',w}>>V^G_B(w)\\ 
 \end{CD}$$
 where $\pi$ denote the natural projection onto the quotient. Since all morphism except $p_{w',w}$ in the commutative diagram are surjective, it follows that $p_{w',w}$ is also surjective. 
 \end{proof}

% Before we continue we have to introduce more notation. For $w \in W$ the support 
% $\supp(w)$ of $w$ is the set of simple reflections contained in a (hence in any) reduced expression of $w$. If $w' \in W$ is another element of the Weyl group, then we denote by $m(w,v) \in \BZ_{\geq 0}$
% the multiplicity of $L(v \cdot 0)$ in $M(w \cdot 0)$. It is well known that $m(w,v)>0$ if and only $ w \leq v$ with respect to the Bruhat order $\leq$ on $W$. 
% Moreover the multiplicities can be computed using Kazhdan-Lusztig polynomials, which in general is only possible with the help of a computer. \\

% \begin{theorem}\label{multiplicities}
%     Fix $w,v \in W$ and let $I_0:=I(w)$ respectively $I:=I(v)$ be as above. For a subset $J \subset \Delta$ with $J \subset I$, 
%     let $v_{P_J}^{P_I}$ be the generalized smooth Steinberg representation of $L_{P_I}$. Then the multiplicity of the irreducible $G$-representation 
%     $\CF^G_{P_I}(L(v\cdot \lambda),v_{P_J}^{P_I})$ in $V^G_B(w)$ is 
%     $$ \sum_{\substack{w' \in W  \\ \supp(w')=J \cap I_0}} (-1)^{\ell(w')+\lb J \cap I_0 \rb} m(w'w,v)$$ 
%     and we obtain in this way all the Jordan-Hölder factors of $V^G_B(w)$. 
% \end{theorem} 

% \begin{proof} 
%     We only have to slightly modify the proof of \cite[Theorem 4.6]{OSch}.  %check language 
%     From the resolution for $V^G_B(w)$ (\ref{reso}) we get for the multiplicity of the simple object 
%     $\CF^G_{P_I}(L(v\cdot \lambda),v_{P_J}^{P_I})$ in $V^G_B(w)$:
%         $$[V_B^G(w): \CF^G_{P_I}(L(v\cdot \lambda),v_{P_J}^{P_I})]=\sum_{K \subset I_0}(-1)^{\lb K \rb}[I^G_{P_K}(w):\CF^G_{P_I}(L(v\cdot \lambda),v_{P_J}^{P_I})].$$
%     By the arguments mentioned in loc. cit it follows that $[I^G_{P_K}(w):\CF^G_{P_I}(L(v\cdot \lambda),v_{P_J}^{P_I})] \neq 0$ if only if  $K \subset J \cap I_0$ and that then 
%     $$ [I^G_{P_K}(w):\CF^G_{P_I}(L(v\cdot \lambda),v_{P_J}^{P_I})] = [M_K(w \cdot \lambda): L(v\cdot \lambda)]$$
%     holds. From the character formula 
%         $$\ch(M_K(w\cdot \lambda))= \sum_{w' \in W_K} (-1)^{\ell(w')} \ch(M(w'w\cdot \lambda)),$$ 
%     which can be deduced from the exact complex (3.11) in \cite{MR}, we obtain 
%     \begin{align*} [V_B^G(w): \CF^G_{P_I}(L(v\cdot \lambda),v_{P_J}^{P_I})] &= \sum_{K \subset J \cap I_0}(-1)^{\lb K \rb} \sum_{w' \in W_K} (-1)^{\ell(w')} [M(w'w\cdot \lambda):L(v \cdot \lambda)] \\
%                                                                             &= \sum_{w' \in W} (-1)^{\ell(w')}[M(w'w\cdot \lambda):L(v \cdot \lambda)]  \sum_{\supp(w') \subset K \subset J\cap I_0} (-1)^{\lb K \rb}.
%     \end{align*}
%     Finally we have 
%     $$ \sum_{\supp(w') \subset K \subset J\cap I_0} (-1)^{\lb K \rb}=(1-1)^{\lb (J \cap I_0) \backslash \supp(w')\rb}$$
%     which is non-zero if and only if $\supp(w')=J \cap I_0$. Then the formula follows. \\

%     The natural morphism $p_{1,w}:V^G_B(\lambda)\rightarrow V^G_B(w)$ is surjectiv for all $w \in W$. 
%     Therefore \cite[Theorem 4.6]{OSch} implies that we obtain in this way all Jordan-Hölder factors of $V^G_B(w)$. 
% \end{proof}
In the following examples, we will compute the composition factors of the homology groups of the complex $C_\bullet$ of Theorem \ref{theorem1} for $\bG=\SL_4$ and some $\mu \in X_*(\bT)$. The strategy is first to 
compute all composition factors with multiplicities of the objects in $C_\bullet$  with the help of Theorem \ref{multiplicities}. This is done with a small program in SAGE (cf. Appendix \ref{AppendixCode}). Then, we can deduce the composition factors of the homology groups by knowing by the previous lemma that the morphism  $p_{w',w}:V^G_B(w')\rightarrow V^G_B(w)$ in the complex $C_\bullet$ is surjective for $w',w \in W$ with  $w' \leq w$ and how composition factors behave under short exact sequences. 
\begin{definition}
    Let $D$ be a composition factor of $V^G_B(\lambda)$ and $n_w:=\big[V^G_B(w):D\big]$ the multiplicity of $D$ in $V^G_B(w)$ for $w \in W$. Then, we define the \textsl{distribution type} of $D$ in the complex $C_\bullet$ by 
    $$\big(n_e,\{n_w\}_{w \in \Omega_\emptyset, \, l(w)=1},\ldots,\{n_w\}_{w \in \Omega_\emptyset, \, l(w)=\dim Y_\emptyset}\big) \in \BN_0^{\lb \Omega_\emptyset\rb}.$$
\end{definition}
\begin{remark}
    The distribution type depends on an ordering on $\Omega_\emptyset$. We will implicitly give such an ordering in each example and hope that causes no confusion with the notation. 
\end{remark}
\begin{example}\label{exampleCohomology}
        Let $\bG=\GL_4$, $\Delta=\{\alpha_1, \alpha_2, \alpha_3 \}$, $S=\{s_1,s_2,s_3\} \subset W$ with $s_i$ corresponding to $\alpha_i$, and $s_1$ commutes with $s_3$. We set $\bP_i=\bP_{\{\alpha_i\}}$ and $\bP_{i,j}=\bP_{\{\alpha_i,\alpha_j\}}$.
        Furthermore let $\mu=(x_1,x_2,x_3,x_4) \in X_*(\bT)\cong \BZ^4$ with $x_1>x_2>x_3>x_4$.
        \begin{enumerate}[label=\alph*),leftmargin=*]
            \item \label{exampleCohomologyA} $\mu=(x_1,x_2,x_3,x_4)$ with $\sum x_i=0$ and $x_3>0$. 
            Then $$\Omega_\emptyset=\{e, s_1, s_2, s_1s_2,s_2s_1, s_1s_2s_1\}$$
            and $$C_\bullet: V_B^G(\lambda) \xrightarrow{f} \bigoplus_{\substack{w \in \Omega_\emptyset \\ \ l(w)=1}} V_B^G(w) \xrightarrow{g}  \bigoplus_{\substack{w \in \Omega_\emptyset \\ \ l(w)=2}} V_B^G(w)  \xrightarrow{h}  V^G_B(s_1s_2s_1).$$
            The appearing distribution types of $C_\bullet$ are %(cf. Appendix \ref{s:AppA2}) 
            \begin{align*}
                &\big(\{2\},\{2,1\},\{1,1\},\{1\}\big),\, \big(\{2\},\{1,2\},\{1,1\},\{1\}\big), \\
                &\big(\{1\},\{1,1\},\{1,1\},\{1\}\big),\, \big(\{1\},\{1,1\},\{1,0\},\{0\}\big), \\
                &\big(\{1\},\{1,1\},\{0,1\},\{0\}\big),\, \big(\{1\},\{1,0\},\{0,0\},\{0\}\big),\\
                & \big(\{1\},\{0,1\},\{0,0\},\{0\}\big),\,  \big(\{1\},\{0,0\},\{0,0\},\{0\}\big).\\ 
            \end{align*}
            As an example for the computations, we consider the distribution type $$\big(\{2\},\{2,1\},\{1,1\},\{1\}\big)$$ and denote a corresponding factor by $D$. 
            Then, $$[\Ker(f):D] \leq [\Ker\big(V^G_B(\lambda)\rightarrow V^G_B(s_1)\big):D]=0.$$ This implies that 
            $[\Im(f):D]=2$. Moreover, $[\Im(h):D]=1$ since $$V^G_B(s_1s_2) \rightarrow V^G_B(s_1s_2s_1)$$ is surjective. Thus, 
            $[\Ker(h):D]=1$. As  the composition $$\bigoplus_{\substack{w \in \Omega_\emptyset \\ \ l(w)=1}} V_B^G(w) \stackrel{g}{\rightarrow} \bigoplus_{\substack{w \in \Omega_\emptyset \\ \ l(w)=2}} V_B^G(w) \stackrel{\pi_1}{\rightarrow} V^G_B(s_1s_2)$$
            is surjective, we have the chain of inequalities 
            $$ 1\leq[\Im(g):D]\leq [\Ker(h):D]=1.$$ 
            Therefore, $[\Im(g):D]=1$ and $[\Ker(g):D]=2$. Finally, we see that 
            $$ [H_i(C_\bullet):D]=0$$ for all $i$. The same arguments applied to all distribution types show that 
            $H_i(C_\bullet)=0$ for $i \neq \dim(\sF)-\lb \Delta \rb=3$ and that $H_3(C_\bullet)=\Ker(f)$ has composition factors precisely
            \begin{equation*}v^G_B(\lambda),\, \CF_{P_{1,2}}^G\Big(L(s_3\cdot \lambda),v^{P_{1,2}}_{B}\Big), \CF_{P_{1,3}}^G\Big(L(s_2s_3\cdot \lambda),v^{P_{1,3}}_{P_{3}}\Big),\,\CF_{P_{2,3}}^G\Big(L(s_1s_2s_3\cdot \lambda),1\Big)
            \end{equation*}
            each with multiplicity one.
            \item \label{exampleCohomologyB}$\mu=(x_1,x_2,x_3,x_4)$ with $\sum x_i=0$ and $x_3=0$.  Then, $$\Omega_\emptyset=\{e, s_1, s_2, s_2s_1 \}$$
            and $$C_\bullet: V_B^G(\lambda) \xrightarrow{f}  \bigoplus_{\substack{w \in \Omega_\emptyset \\ \ l(w)=1}} V_B^G(w) \xrightarrow{g}  V^G_B(s_2s_1). $$ 
            The appearing distribution types in $C_\bullet$ are %(cf. Appendix \ref{s:AppA2}) 
            \begin{align*}
                &\big(\{2\},\{2,1\},\{1\}\big),\, \big(\{2\},\{1,2\},\{1\}\big), \\
                &\big(\{1\},\{1,1\},\{1\}\big),\, \big(\{1\},\{1,1\},\{0\}\big), \\
                & \big(\{1\},\{1,0\},\{0\}\big),\, \big(\{1\},\{0,1\},\{0\}\big),\\
                &  \big(\{1\},\{0,0\},\{0\}\big).
            \end{align*}
            With the same arguments as above, we get that 
            $H_i(C_\bullet)=0$ for $i \neq 2,3$. Furthermore, $H_3(C_\bullet)$ has composition factors precisely
            \begin{align*}  &v^G_B(\lambda),\, \CF_{P_{1,2}}^G\Big(L(s_3\cdot \lambda),v^{P_{1,2}}_{B}\Big), \\ &\CF_{P_{1,3}}^G\Big(L(s_2s_3\cdot \lambda),v^{P_{1,3}}_{P_{3}}\Big),\,\CF_{P_{2,3}}^G\Big(L(s_1s_2s_3\cdot \lambda),1\Big)
            \end{align*}
            each with multiplicity one. Moreover, $H_2(C_\bullet)$ has composition factors precisely
            \begin{align*}  &\CF_{P_{2,3}}^G\Big(L(s_1s_2\cdot \lambda),v^{P_{2,3}}_{B}\Big), \CF_{P_{2,3}}^G\Big(L(s_1s_2s_3\cdot \lambda),v^{P_{2,3}}_{B}\Big), \\ 
                            &\CF_{P_{2}}^G\Big(L(s_3s_1s_2\cdot \lambda),v^{P_{2}}_{B}\Big),\,\CF_{P_{2}}^G\Big(L(s_1s_2s_3s_2\cdot \lambda),v^{P_{2}}_{B}\Big), \\
                            &\CF_{P_{1,3}}^G\Big(L(s_2s_3s_1s_2\cdot \lambda),1\Big),\,\CF_{P_{1,3}}^G\Big(L(s_2s_3s_1s_2\cdot \lambda),v^{P_{1,3}}_{P_{3}}\Big), \\
                            &\CF_{P_{3}}^G\Big(L(s_1s_2s_3s_1s_2\cdot \lambda),1\Big)
            \end{align*}
            each with multiplicity one as well. 
        \item \label{exampleCohomologyC}$\mu=(x_1,x_2,x_3,x_4)$ with $\sum x_i=0$, $x_2>0,x_3<0, x_1+x_4>0, x_2+x_3<0$.  Then,$$\Omega_\emptyset=\{e, s_1, s_2, s_3, s_1s_3, s_2s_3  \}$$
        and $$C_\bullet: V_B^G(\lambda) \xrightarrow{f} \bigoplus_{\substack{w \in \Omega_\emptyset \\ \ l(w)=1}} V_B^G(w) \xrightarrow{g}  V^G_B(s_1s_3) \oplus V^G_B(s_2s_3). $$ 
        The appearing distribution types in $C_\bullet$  are %(cf. Appendix \ref{s:AppA2}) are 
        \begin{align*}
            &\big(\{2\},\{2,1,2\},\{2,1\}\big),\, \big(\{2\},\{1,2,1\},\{1,1\}\big),\, \big(\{1\},\{1,1,1\},\{1,1\}\big), \\
            &\big(\{1\},\{1,1,1\},\{1,0\}\big),\, \big(\{1\},\{1,0,1\},\{1,0\}\big),\, \big(\{1\},\{0,1,1\},\{0,1\}\big), \\
            &\big(\{1\},\{1,1,0\},\{0,0\}\big),\, \big(\{1\},\{0,1,1\},\{0,0\}\big),\, \big(\{1\},\{1,0,0\},\{0,0\}\big), \\
            &\big(\{1\},\{0,1,0\},\{0,0\}\big),\, \big(\{1\},\{0,0,1\},\{0,0\}\big),\, \big(\{1\},\{0,0,0\},\{0,0\}\big).
        \end{align*}
        First, we notice that $g$ is surjective since $V^G_B(s_1)$ and $V^G_B(s_2)$ map onto a single but distinct direct summand. Then, we can apply the same arguments as before. We compute that  
        $H_i(C_\bullet)=0$ for $i \neq 2,3$. Furthermore, $H_3(C_\bullet)=v^G_B(\lambda)$. 
     Moreover, $H_2(C_\bullet)$ has composition factors precisely
        \begin{align*}  &\CF_{P_{2,3}}^G\Big(L(s_1s_2\cdot \lambda),v^{P_{2,3}}_{B}\Big),\, \CF_{P_{1,3}}^G\Big(L(s_2s_1\cdot \lambda),v^{P_{1,3}}_{B}\Big),\\
                        &\CF_{P_{1,2}}^G\Big(L(s_3s_2\cdot \lambda),v^{P_{1,2}}_{B}\Big),\, \CF_{P_{3}}^G\Big(L(s_1s_2s_1\cdot \lambda),v^{P_{3}}_{B}\Big),\\
                        &\CF_{P_{1,2}}^G\Big(L(s_3s_2s_1\cdot \lambda),v^{P_{1,2}}_{B}\Big), \, \CF_{P_{1,2}}^G\Big(L(s_3s_2s_1\cdot \lambda),v^{P_{1,2}}_{P_2}\Big), \\
                        &\CF_{P_{2}}^G\Big(L(s_3s_1s_2\cdot \lambda),v^{P_{2}}_{B}\Big), \,\CF_{P_{1,3}}^G\Big(L(s_2s_3s_1s_2\cdot \lambda),1\Big),\\
                        &\CF_{P_{1,3}}^G\Big(L(s_2s_3s_1s_2\cdot \lambda),v^{P_{1,3}}_{P_{1}}\Big), \, \CF_{P_{2}}^G\Big(L(s_3s_1s_2s_1\cdot \lambda),1\Big),\\
                        &\CF_{P_{2}}^G\Big(L(s_3s_1s_2s_1\cdot \lambda),v^{P_{2}}_{B}\Big),\, \CF_{P_{1}}^G\Big(L(s_2s_3s_1s_2s_1\cdot \lambda),1\Big)
        \end{align*}
        each with multiplicity one. 
        \end{enumerate}
\end{example}
% \subsection{Outlook to the parabolic case} \label{s:generalizations}

% Last but not least, we would like to illustrate why the arguments of the last section are not so easily transferable to the case where $\{\mu\}$ is arbitrary. Even if we make the Assumption \ref{hypo1} adjusted to this case. \\

% One crucial point in the previous computations is Lemma \ref{C_I(w)}, namely, the fact that $H^*_{C_I(w)}(\sF, \CE_\lambda)$ has only non-trivial cohomology in degree $n-l(w)$. This made it possible to compute $H^*_{Y_I}(\sF, \CE_\lambda)$ with a suitable chain complex (cf. Lemma \ref{cohY_I}). We will see in this section that in the general parabolic case, the local cohomology groups with support in a generalized Schubert cell with coefficients in a line bundle $\CE_\lambda$, analogous to the Borel case, do not have this vanishing property. For this, we first define generalized Schubert cells for the general situation and show a result similar to Proposition \ref{genSchCe}. \\

% Therefore, we consider a local Shtuka-datum $(\bG,\{\mu\}, [1])$  with arbitrary $\{\mu \}$; $\bG$ is still assumed to be split. As before, we fix an $\inva$ on $\bG$ and choose a split maximal torus $\bT$ of $\bG$ of rank $d$ such that $\mu \in X_*(\bT)_\BQ$. Let $(\bT, \bB)$ be a Borel pair of rank $d$ which gives rise to a set of simple roots (cf. section \ref{s:rootdatum})
%     $$\Delta:=\{\alpha_1, \ldots , \alpha_d\} \subset X^*(\bT)_\BQ.$$
% Again, we can assume that $\mu$ lies in the positive Weyl chamber with respect to $\bB$, so $\bP:=\bP(\mu) \supset \bB$, i.e. it is a standard parabolic subgroup with respect to $\bB$. Let $\sF:=\bG/\bP$ which is defined over $K$. Further, we let $W_\mu$ be the stabilizer of $\mu$ under the action of $W$. Then, it can be easily shown that $W_\mu=W_{J_\mu}$ for $$J_\mu:=\{\alpha \in \Delta \mid \langle \alpha, \mu \rangle =0\} \subset \Delta$$  (cf. \cite[Section 10.3, Lemma B and Proof]{H3}). We denote by ${}^{J_\mu}W$ the left \textit{Kostant representatives}, i.e. the set of minimal length left coset representatives in $W/W_{J_\mu}$. Then, adjusted to this case, we let 
% \begin{equation}\label{Omega_Ipara}
%     \Omega_I:=\{w \in {}^{J_\mu}W \mid (w\mu, \varpi_\alpha)> 0 \text{ for all } \alpha \not\in I  \}
% \end{equation}
% for $I \subsetneq \Delta$ (cf. \cite[p. 530]{O1}). Again, we have the following useful lemma induced by Lemma \ref{equivalentcond}. 
% \begin{lemma}\label{fweightandpairingparabolic} 
%     Let $I \subsetneq \Delta $. Then, $w \in \Omega_I$ if and only if $\langle \check{\varpi}_\alpha, w\mu\rangle_\der >0$ for all $\alpha \not\in I$. 
%  \end{lemma} 
%  The definition of $Y_I \subset \sF$ is still the same (cf. (\ref{Y_I})) and furthermore, similar to the Borel case, $Y_I$ is also a union of Schubert cells in $ \sF$.  
%     \begin{proposition}\cite[Proposition 4.1]{O1} For $I \subsetneq \Delta$, we obtain 
%         $$Y_I= \bigcup_{w \in \Omega_I} \bB w\bP/\bP.$$
%     \end{proposition}
%     For $I \subset \Delta$ and $w \in W$, we have (cf. \cite[Section 2.1]{Pe})
%     $$\bP_Iw\bP/\bP=\bigcup_{(v,u) \in W_I \times W_{J_\mu}} \bB vwu\bP/\bP.$$ 
%     From \cite[Proposition 2.7]{BKPST}, we know that each double coset $W_I\backslash W/W_{J_\mu}$ contains a unique element of minimal length which can be found in ${}^{J_\mu} W^I:={}^{J_\mu} W \cap W^I $. That motivates the following definition.
%     \begin{definition} Let $I \subset \Delta$ and $w \in {}^{J_\mu} W^I$.
%         \begin{enumerate}[label=\roman*)]
%             \item The \textit{generalized Schubert cell} in $\sF$ associated to $I$ and $w$ is $$C_I^\mu(w):=\bP_Iw\bP/\bP.$$
%             If $I=\emptyset$, we omit the subscript.
%             \item For $J \subset \Delta$, let $S_J=\{s_\alpha \in S\mid \alpha \in J \}$ (cf. (\ref{Waction})). For $w \in W$ we define $H_w \subset \Delta$ to be the subset such that $$S_{H_w}=S_I \cap wS_{J_\mu}w^{-1}.$$ 
%             Then, we let ${}^{H_w}W_I$ be the set of left Kostant representatives of $W_I/W_{H_w}$.
%         \end{enumerate}
%     \end{definition}   
%     \begin{lemma} \label{doublecoset}\cite[Corollary 2.8]{BKPST} Let $I \subset \Delta$ and $w \in {}^{J_\mu}W^I$. Then, $vw \in {}^{J_\mu}W$ for $v \in W_I$ if and only if $v \in {}^{H_w}W_I$. Consequently, every element of $W_IwW_{J_\mu}$ can be written uniquely as $vwu$, where $v \in {}^{H_w}W_I$, $u \in W_{J_\mu}$, and $l(vwu)=l(v)+l(w)+l(u)$. 
%     \end{lemma}
%     \begin{lemma}\label{generalizedparaboliccell} Let $I \subset \Delta$ and $w \in {}^{J_\mu} W^I$. Then, 
%         $$ C_I^\mu(w)=  \bigsqcup_{v \in {}^{H_w}W_I} C^\mu(vw).$$
%     \end{lemma}
%     \begin{proof}
%     We have seen that 
%     $$C_I^\mu(w)= \bigcup_{(v,u) \in W_I \times W_{J_\mu}}\bB vwu\bP/\bP.$$
%     Therefore, one inclusion is obvious. For the other inclusion, notice that $u \in W_{J_\mu}$ implies $u\bP=\bP$. Furthermore, if $v' \in W_I$, there exist unique $v \in {}^{H_w} W_I$ and $ v'' \in W_{H_w}$ such that $v'=vv''$ and $l(v')=l(v)+l(v'')$. As $v'' \in W_{H_w}$, we can write $v''=wu'w^{-1}$ with $u' \in W_{J_\mu}$. Thus, 
%     $$\bB v'w\bP/\bP=\bB vwu'w^{-1}w\bP/\bP=\bB vw\bP/\bP$$
%     and 
%     $$ C_I^\mu(w) = \bigcup_{(v,u) \in W_I \times W_{J_\mu}}\bB vwu\bP/\bP\subset \bigcup_{v \in {}^{H_w}W_I} C^\mu(vw).$$ 
%     Then, the disjointedness follows by \cite[Proposition 6.2]{K} as the $C^\mu(vw)$ are Schubert cells of $\sF$. 
%     \end{proof}
%     \begin{proposition}\label{paraboliccovering}For $I \subsetneq \Delta$, we have 
%         $$Y_I= \bigsqcup_{w \in {}^{J_\mu} W^I\cap \Omega_I } C_I^\mu(w).$$
%     \end{proposition}
%     \begin{proof}
%         We know from \cite[Proposition 11.1.6]{DOR} that $Y_I= \bigcup_{w \in \Omega_I} \bP_Iw\bP/\bP$. For $w' \in \Omega_I$ exist, by Lemma \ref{doublecoset}, unique  $w \in {}^{J_\mu} W^I$, $v \in {}^{H_w} W_I$, and $u \in W_{J_\mu}$ such that $w'=vwu$ with $l(w')=l(v)+l(w)+l(u)$, and $vw \in {}^{J_\mu} W$. Since $w' \in {}^{J_\mu} W$ (cf. (\ref{Omega_Ipara})), we have that
%         $u=e$. Then, we see that 
%         $$ \bP_Iw'\bP/\bP= \bP_Ivw\bP/\bP=\bP_Iw\bP/\bP.$$
%         Since $Y_I$ is closed, we get 
%         $$ Y_I=  \bigcup_{w \in {}^{J_\mu} W^I\cap \Omega_I } C_I^\mu(w).$$ 
%     %     Let  $w \in  {}^{J_\mu} W^I$ and $v \in W_I$. Further let $v=s_1\ldots s_r$  where $s_i \in \{s_{\alpha}\}_{\alpha \in I}$ be a reduced expression. Then, Lemma \ref{doublecoset} tells us that $vw \in {}^{J_\mu}W$ if and only if $v \in  {}^{H_w}W_I$.
%     %     Moreover, for $\beta \in \Delta \backslash I$ and $\check{\varpi}_\beta \in X^*(\bT)$, we have
%     %     \begin{equation*}
%     %     \langle \check{\varpi}_\beta, vw\mu  \rangle_\der =\langle s_1\check{\varpi}_\beta, s_2\ldots s_rw\mu \rangle_\der=\langle \check{\varpi}_\beta, s_2\ldots s_rw\mu \rangle_\der = \ldots = \langle \check{\varpi}_\beta, w\mu \rangle_\der.
%     %     \end{equation*}
%     %     by \cite[Chapter 6, 1.10]{Bo}. Further, for any $w' \in W$, we have $w'\mu=\sum_{\alpha \in \Delta} m_\alpha \alpha^{\vee}$ with $m_\alpha \in \BQ$. Then, it follows by Lemma \ref{fweightandpairingparabolic} that 
%     %     \begin{equation}\label{conditionparabolic} w \in \Omega_I \text{ and } v \in {}^{H_w}W_I \text{ if and only if } vw \in \Omega_I . \end{equation}  
%     %    % Hence $W_Iw \subset \Omega_I$ for every $w \in \Omega_I$ and 
%     %     We see that 
%     %     $$\bigcup_{w \in {}^{J_\mu} W^I\cap \Omega_I } C^\mu_I(w)=\bigcup_{w \in {}^{J_\mu} W^I\cap \Omega_I }\bigcup_{v \in {}^{H_w}W_I} C^\mu(vw) \subset \bigcup_{w \in \Omega_I} C^\mu(w) =Y_I.$$
%     %     On the other hand, if $w' \in \Omega_I \subset W$, there exists a unique $w \in {}^{J_\mu} W^I$ representing the image of $w'$ in $ W_I\backslash W/W_{J_\mu}$. Then, by Lemma \ref{doublecoset}, there exist unique $v \in {}^{H_w} W_I$ and $u \in W_{J_\mu}$ such that $w'=vwu$ with $l(w')=l(v)+l(w)+l(u)$, and $vw \in {}^{J_\mu} W$. Since $w' \in {}^{J_\mu} W$ (cf. (\ref{Omega_Ipara})), we have that $u=e$. 
%     %     Hence,  $C^\mu(w') \subset C_I^{\mu}(w)$ and we have $w \in {}^{J_\mu} W^I \cap \Omega_I$ by (\ref{conditionparabolic}). Thus, 
%     %     $$Y_I= \bigcup_{w \in \Omega_I} C^\mu(w) \subset \bigcup_{w \in {}^{J_\mu} W^I \cap \Omega_I} C_I^\mu(w).$$ 
%     The union is disjoint for the same reason as in the proof of Proposition \ref{genSchCe}. 
%     \end{proof}
%     Let $\lambda \in X^*(\bT)^+$ be a dominant weight and $\CE_\lambda=\CL_\lambda \otimes  \omega_{\sF}$, analogously defined to  (\ref{linebundle}). Then, one could ask if we can compute $H^*_{Y_I}(\sF, \CE_\lambda)$ by a similar complex as in Lemma \ref{cohY_I}. The following example at least gives the answer that these cohomology groups are not so easy to deduce as in the Borel case. 

%     \begin{example}
%         We are in the situation of Example \ref{GL_n} for $n=3$. Further, we let $\mu=(2,-1,-1) \in X_*(\bT)$. Then, $\sF=\bG/\bP(\mu)\cong \BP^2_K$ and $J_\mu=\{\alpha_2\}$. Let $I=\{\alpha_1\}$. Hence, $W^I=\{e, s_2, s_2s_1 \}$, ${}^{J_\mu}W=\{e, s_1,s_2s_1\}$, and thus ${}^{J_\mu}W^I=\{e, s_2s_1\}$. We choose $w=e \in {}^{J_\mu}W^I$. This implies $S_I \cap wS_Jw^{-1}= \{e\}$ and therefore $H_w=\emptyset$. From that, it follows that  
%         $$ C_I^\mu(e)=C^\mu(e) \sqcup C^\mu(s_1).$$ 
%        Let $\lambda=(0, \ldots,0) \in X^*(\bT)$. Then, $\CE_\lambda=\omega_{\sF}$ and $H^*_{C^\mu_I(e)}(\sF,\omega_\sF)$ is the cohomology of the cochain complex 
%        $$ H^{1}_{ C^\mu(s_1)}(\sF,\omega_\sF) \rightarrow H^{2}_{ C^\mu(e)}(\sF,\omega_\sF)$$
%        by Lemma \ref{cousinspectral}. Here we used that the Schubert cell $C^\mu(w)$ is affine for $w \in {}^{J_\mu} W$. Hence, $H^*_{ C^\mu(w)}(\sF,\omega_\sF)$ is only non-trivial in degree $n-l(w)$. As 
%        $$ \sF= C^\mu(e) \sqcup C^\mu(s_1) \sqcup C^\mu(s_2s_1),$$ 
%        we see, by the same arguments as before, that the cochain complex 
%        $$H^{0}_{ C^\mu(s_2s_1)}(\sF,\omega_\sF) \rightarrow  H^{1}_{ C^\mu(s_1)}(\sF,\omega_\sF) \rightarrow H^{2}_{ C^\mu(e)}(\sF,\omega_\sF)$$ 
%        computes $H^*(\sF,\omega_\sF)$ and the morphism are the same as for $H^*_{C^\mu_I(e)}(\sF,\omega_\sF)$. Hence, we have  
%        $$H^2_{C^\mu_I(e)}(\sF,\omega_\sF)= H^2(\sF,\omega_\sF) \neq 0.$$ 
%        Moreover, as shown in \cite[Proposition 3.2.1]{O2}, $H^1_{C^\mu_I(e)}(\sF,\omega_\sF) \neq 0.$  
    
%     \end{example}
%     By the previous example, we see that in contrast to the Borel case, the cohomology groups  $H^*_{ C^\mu_I(w),}(\sF,\CE_\lambda)$ can be non-trivial in more than one degree. 
%     As in the proof of Lemma \ref{cohY_I}, the covering of $Y_I$ from Lemma \ref{paraboliccovering} induces a filtration on $Y_I$ by closed subspaces with disjoint union of generalized Schubert cells as differences.  Therefore, the associated $E_1$-page of the spectral sequence of Lemma \ref{cousinspectral} can have more than one non-trivial line. Hence, the proof of Lemma \ref{cohY_I} breaks down at this point. 

    
%Computations for other types than $A_n$ let us conjecture that Lemma \ref{help} is independent of the type and therefore Theorem \ref{theorem1} is expected to hold for general split connected reductive groups $\bG$.




% Then following \cite{MR} we define for $w \in W^I$ $$C_I(w):=\bP_Iw\bB/\bB=\bigcup_{v \in W_I}C(vw)$$
% and similar to before we have
% \begin{lemma}\label{C_I(w)} For $w \in W^I$ 
%     $$H^i_{C_I(w)}(\sF,\CE)\cong\begin{cases} M_I(w\cdot\lambda) &i=n-l(w),\\ 0 &\text{else.} \end{cases}$$
% \end{lemma}
% \begin{proof}
% \end{proof}

% For later use we follow  \cite{OSch} and set, with respect to lemma \ref{C_I(w)}, $$I^G_{P_I}(w):=\CF_{P_I}^G(H^i_{C_I(w)}(\sF,\CE)).$$

% Lets come back to \ref{spectral} applied to our setting, where we start with some observations. 

% \begin{lemma} For $I \subset \Delta$ 
%     $$ Y_I=\bigcup_{w \in W^I \cap \Omega_I} C_I(w). $$
% \end{lemma}



% Additionally for $\alpha \in I$ we denote by $ \check{\varpi}_\alpha \in X^*(\bT)$ the fundamental weight i.e. 
% $$ \langle \alpha^{\vee}, \check{\varpi}_\beta \rangle =\delta_{\alpha,\beta}.$$
% Then 
%     $$(\alpha^{\vee}, \varpi_\beta) = (\alpha^{\vee}, \varpi_\beta)=(\frac{2}{(\alpha,\alpha)}\alpha^*,\varpi_\beta)= \frac{2}{(\alpha,\alpha)} \langle \alpha, \varpi_\beta \rangle= \frac{2}{(\alpha,\alpha)}  \langle \alpha^{\vee}, \check{\varpi}_\beta \rangle$$

% and we see that
% $$ (\sum n_\alpha \alpha^{\vee}, \varpi_\beta )=\frac{2}{(\alpha, \alpha)} n_\beta>0 \text{ if and only if } \langle \sum n_\alpha \alpha^{\vee}, \check{\varpi}_\beta \rangle = n_\beta>0. $$
% Hence we can in the following assume that $(\,,\,)$  is the 
% standard inner product on $\BQ^n$ and $\{\varpi_\alpha\}_{\alpha \in \Delta}$ are the usual fundamental weights of $\bG$.



%\todo[inline]{\textbf{Comment/Question}: Still hoping that these complexes are exact at the most positions to conclude somehow (maybe by weight reasons) that the spectral sequence degenerates at $E_2$ page. But I'm not sure how to continue. Some of the local cohomology groups will vanish by dimension reasons but it is at the moment not obvious which one. Maybe there is for each $q$ some $K \subset \Delta$ such that $H^q_{Y_I}(\sF,\CE)=0$ whenever $K \not \subset I$ and a generalized result of Theorem 2.5, ch. III in Period Domains by Dat, Orlik, Rapoport }





