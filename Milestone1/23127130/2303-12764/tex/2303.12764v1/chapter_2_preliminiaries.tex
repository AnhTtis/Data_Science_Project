%!TEX root = draft.tex
\section{Preliminaries}\label{s:preliminaries} 

\subsection{Local Cohomology}\label{s:localcohomology} 
In this subsection, we recall some facts about the local cohomology of topological spaces. For that purpose, we follow the theory described in \cite[Section 1]{H}. Let $X$ be a topological space and $Z \subset X$ a locally closed subset, i.e. there is an open subset $V$ such that  $Z$ is closed in $V$. Further, let $\CE \in \mathbf{Ab}(X)$ be an abelian sheaf on $X$. Then, $H^*_Z(\sF,\CE)$ be the right derived functors of $$\Gamma_Z(X,\CE):=\Ker\big(\Gamma(V,\CE)\rightarrow \Gamma(V\backslash Z,\CE)\big).$$ An essential property that we take advantage of is the following. 



% $\Gamma_Z(X,\CE)$ is the subgroup of $\CE(V)$ defined by the sections with support in $Z$. The definition of $\Gamma_Z(X,\CE)$ is independent of the chosen open subset $V$ and 
% \begin{align*} 
%     \mathbf{Ab}(X) &\longrightarrow \mathbf{Ab} \\
%     F &\mapsto \Gamma_Z(X,\CE)
% \end{align*}
% defines a left exact covariant functor. Thus, we let the \textit{local cohomology groups} $H^*_Z(\sF,\CE)$ be the right derived functors of $\Gamma_Z(X,-)$. 


\begin{proposition}\label{excision}\cite[Proposition 1.3]{H}
    Let $Z$ be locally closed in $X$, and let $V$ be open in $X$ and such that $Z \subseteq V \subset X$. Then, for any $\CE \in \mathbf{Ab}(X)$, 
    $$ H^i_Z(X,\CE) \cong H^i_Z(V,\CE|_V).$$
\end{proposition}
For two closed subsets $Z_1,Z_2 \subset X$ with $Z_1 \subset Z_2$, we let $$\Gamma_{Z_1/ Z_2}(X, \CE):=\Gamma_{Z_1}(X, \CE)/\Gamma_{Z_2}(X, \CE).$$ 
Then, $H^*_{Z_1/ Z_2}(X, \CE)$ denotes the right derived functor of $\Gamma_{Z_1/ Z_2}(X, -)$ as defined in \cite[Section 7, p. 349/350]{K}. It comes with the following property. 

\begin{lemma}\label{locrel}\cite[Lemma 7.7]{K}
    Let $Z_1 \supset Z_2$ be two closed subsets of a topological space $X$. Let $\CE$ be any abelian sheaf on $X$. Then, there is a natural isomorphism 
    $$ H^i_{Z_1/Z_2}(X,\CE) \longrightarrow H^i_{Z_1\backslash Z_2}(X\backslash Z_2, \CE)$$
    for all integers $i$.  
\end{lemma}


Next, we state a rather technical result which will be helpful afterwards. It seems somehow standard as it is for example used in \cite{MR} and \cite[Section 2.1]{FKT}, but we could not find a precise statement fitting our purposes.
    
\begin{lemma}\label{cousinspectral}
Let $X \supset Z_0 \supset Z_1 \supset \ldots  \supset Z_n$ be a filtration of $X$ by closed subsets and $\CE$ an abelian sheaf on $X$. Then, there is a 
spectral sequence 
$$E_1^{pq}= H^{p+q}_{Z_p/Z_{p+1}}(X, \CE)\Rightarrow H^{p+q}_{Z_0}(X,\CE),$$
where the morphisms on the $E_1$-page are the natural ones. 
\end{lemma}
\begin{proof}
We associate to $\CE$ a complex of flasque sheaves $\CG^\bullet(\CE)$ on $X$ together with an augmentation map $\CE \rightarrow \CG^0(\CE)$ such that 
$\CE \rightarrow \CG^\bullet(\CE)$ is a resolution of $\CE$ (cf. \cite{G}). As Kempf pointed out in \cite[Section 7, p. 350]{K}, we can use this resolution to compute the local cohomology groups in question. \\

The given filtration on $X$ naturally defines a filtration of complexes  
$$\Gamma_{Z_0}(X,\CG^\bullet(\CE)) \supset \Gamma_{Z_1}(X,\CG^\bullet(\CE)) \supset \ldots \supset \Gamma_{Z_n}(X,\CG^\bullet(\CE)) $$
from which we form the following quotient complexes 
\begin{equation}\label{quotient}
0 \rightarrow \Gamma_{Z_{j+1}}(X,\CG^\bullet(\CE)) \rightarrow \Gamma_{Z_j}(X,\CG^\bullet(\CE)) \rightarrow K_j^{\bullet} \rightarrow 0.
\end{equation}
Notice that by \cite[Section 7]{K}, one has $K_j^\bullet=\Gamma_{Z_j/Z_{j+1}}(X, \CG^\bullet(\CE))$ such that $$H^q(K_j^\bullet)=H^q_{Z_j/Z_{j+1}}(X, \CE).$$
Then, by the procedure explained in \cite[Section 3]{Ri}, we get an exact complex of complexes 
\begin{equation}\label{dcoffc}
0 \rightarrow \Gamma_{Z_0}(\sF,\CG^\bullet(\CF)) \rightarrow \tilde{K}_0^\bullet \rightarrow \tilde{K}^\bullet_1[1] \rightarrow \tilde{K}^\bullet_2[2] \rightarrow \ldots \rightarrow \tilde{K}^\bullet_n[n] \rightarrow 0
\end{equation}
where $\tilde{K}^\bullet_j$ is a complex quasi-isomorphic to $K^\bullet_j$. Moreover, the morphisms $\tilde{K}^\bullet_j \rightarrow \tilde{K}^\bullet_{j+1}[1]$ induces the natural homomorphisms
$H^q_{Z_j/Z_{j+1}}(X,\CE)\rightarrow H^{q+1}_{Z_{j+1}/Z_{j+2}}(X,\CE) $ which are given by the connecting homomorphisms coming from the long exact sequence in cohomology of (cf. (\ref{quotient})) followed by the quotient maps.\\

Thus, (\ref{dcoffc}) yields a double complex 
$$C^{\bullet, \bullet}:\tilde{K}_0^\bullet \rightarrow \tilde{K}^\bullet_1[1] \rightarrow \tilde{K}^\bullet_2[2] \rightarrow \ldots \rightarrow \tilde{K}^\bullet_n[n].$$ 
Then, by combining the properties mentioned above with the usual theory of spectral sequences associated to a double complex, we obtain the result we were looking for. 
\end{proof}

\subsection{Split reductive groups}\label{s:rootdatum} 
We recall the basic facts that come along with a split connected reductive group which will be essential throughout the whole thesis. \\

%Let $p$ be a prime and $K$ a finite field extension of $\BQ_p$.  \\

Let $K$ be a finite extension of $\BQ_p$ and $\bG$ a split connected reductive group over $K$. 
Any split maximal torus $\bT \subset \bG$ of rank $d$ defines the \textit{split pair} $(\bG,\bT)$ \textit{of rank $d$} to which we associate the root datum 
$$(X^*(\bT),\Phi(\bG,\bT), X_*(\bT),\Phi^\vee(\bG,\bT))$$ with the natural pairing 
\begin{align}
   \langle \text{ , } \rangle:X^*(\bT)_\BQ \times X_*(\bT)_\BQ &\longrightarrow \mathbb{\BQ}  \label{rootpairing}
\end{align}
(cf. \cite[Part II, 1.13]{J}). Furthermore, there exists an \textit{invariant inner product} on $\bG$ due to Totaro, abbreviated by $\inva$ (cf. \cite[Section 5.2.1]{CDHN}). That means 
we have a non-degenerate positive definite symmetric bilinear form $(\text{ , })$ on $X_*(\bT)_\BQ$ for all maximal tori $\bT$ (defined over $\overline{K}$) of $\bG$ such that 
the maps 
\begin{equation*} 
   X_*(\bT)_{\BQ} \rightarrow X_*(g \bT g^{-1})_{\BQ}, \, 
   X_*(\bT)_{\BQ} \rightarrow X_*(\tau \bT \tau^{-1})_{\BQ},
\end{equation*}
are isometries for all $g \in \bG(\overline{K})$ and $\tau \in \Gal(\overline{K}/K)$. A chosen $\inva$ on $\bG$, for any split pair $(\bG,\bT)$, together with the natural pairing (\ref{rootpairing}) induces an isomorphism of $\BQ$-vector spaces 
\begin{align*}
   X^*(\bT)_\BQ &\longrightarrow X_*(\bT)_\BQ,\\
   \chi &\mapsto \chi^*, 
\end{align*}
such that 
\begin{equation}\label{products} 
   (\chi^*, \mu)=\langle \chi, \mu \rangle
\end{equation} 
for all $\mu \in X_*(\bT)$. Furthermore, for $\alpha \in \Delta$ we have (cf. \cite[Ch. VI, §1.1., Lemma 2]{B})
\begin{equation}\label{dual}
   \alpha^{\vee}=\frac{2\alpha^*}{(\alpha^*,\alpha^*)}.
\end{equation}


For the rest of the subsection, we fix a split maximal torus $\bT$ of $\bG$ and an 
$\inva$ $(\text{ , })$  on $\bG$. Using $\Phi:=\Phi(\bG,\bT)$ for the root system and fixing a Borel subgroup $\bB$ inside $\bG$ containing $\bT$, we get a set of corresponding positive 
roots $\Phi^+ \subset \Phi$ and simple roots $\Delta \subset \Phi^+$ as explained in \cite[Section 16.3.1]{Sp}. We call such a tuple $(\bT, \bB)$ a \textit{Borel pair}. \\

There is the following relation between simple roots and coroots. 
\begin{lemma}\label{rootrel}\cite[Lemma 8.2.7]{Sp} For $\alpha, \beta \in \Delta$, $\alpha \neq \beta$, we have  $\langle \alpha, \beta^\vee \rangle \leq 0$.
\end{lemma}

Let $W=N_\bG(\bT)/\bT$ be the Weyl group of $\bG$ with longest element $w_0$ with respect to $\bB$. The natural action of $W$ on $\bT$ by conjugation induces an action on $X^*(\bT)$. We denote by $S:=\{s_\alpha\}_{\alpha \in \Delta} \subset W$, the \textit{simple reflections}, a set of generators of $W$ such that 
\begin{equation*}
   s_\alpha.\lambda=\lambda - \langle \lambda, \alpha^\vee \rangle \alpha 
\end{equation*}
for $\lambda \in X^*(\bT)_\BQ$ and $\alpha^\vee \in \Phi^\vee(\bG,\bT)$. %Additionally the simple reflections $s_\alpha$ let $\Phi$ invariant \cite[Section 7.4.1, Axiom RD2]{Sp}.  
For $w \in W$, we consider also the \textit{dot action} given by 
\begin{equation*} \label{dotaction} w \cdot \lambda=w.(\lambda+\rho)-\rho
\end{equation*}
where $\rho:=\frac{1}{2}\sum_{\alpha \in \Phi^+} \alpha $. The support $\supp(w)$ of an element $w \in W$ is the set of simple reflections contained in a (thus in any) reduced expression of $w$. \\

Each $I \subset \Delta$ defines a root system $\Phi_\RI \subset \Phi$ with positive 
roots $\Phi_\RI^+ \subset \Phi_\RI$ and a Weyl group $W_\RI \subset W$ generated by the $\{s_\alpha\}_{\alpha \in I}$ (cf. \cite[Part II, 1.7]{J}). 
We denote by $W^I$ the right \textit{Kostant representatives}, i.e.  the set of minimal length right coset representatives in $W_I\backslash W$. It can be described as (cf. \cite[(2.2)]{B1})
$$W^I=\{w \in W \mid l(s_\alpha w)>l(w) \text{ for all } \alpha \in I\}.$$ 
\begin{lemma}\label{Kostant}\cite[Section 0.3 (4)]{H2}
Let $w \in W$ and $I \subset \Delta$. Then, $w \in W^I$ if and only if $w^{-1}\alpha \in \Phi^+$  for all $\alpha \in \Phi_I^+$. 
\end{lemma}
For $w \in W$, we let 
\begin{equation}\label{maximalI}
   I(w):=\{\alpha \in \Delta \mid l(s_{\alpha }w)>l(w)\} \subset \Delta
\end{equation}
be the unique maximal subset such that $w \in W^{I(w)}$ (cf. \cite[p. 663]{OSt}). We have an inclusion preserving bijection (cf. \cite[Proposition 12.2]{MT}) 
\begin{align}\label{parabolicbijection}
    \CP(\Delta) &\longleftrightarrow   \{\text{parabolic subgroups  } \bP\supset \bB  \} \\
    I &\mapsto \bB W_I\bB =:\bP_I \nonumber
\end{align} 
where the subgroups $\bP_I$ denote the \textit{standard parabolic subgroups} of $\bG$ with respect to $\bB$, e.g. $\bP_\emptyset=\bB$, $\bP_\Delta=\bG$. Furthermore, each $\bP:=\bP_I$ admits a \textit{Levi decomposition} $\bP=\mathbf{L_P}\cdot \mathbf{U_P}$ (cf. \cite[Part II,1.8]{J}). 
Here, $\mathbf{L_P}$ denotes the standard \textit{Levi factor} containing $\bT$ and $ \mathbf{U_P}$ the \textit{unipotent radical} of $\bP$. Additionally, we let $\mathbf{U}^-_{\bP}$ be its \textit{opposite unipotent radical}. 
\begin{remark}\label{intersectionparabolic}
   Let $I,J \subset \Delta$. Since $W_I \cap W_J=W_{I \cap J}$, one sees that $\bP_I \cap \bP_J=\bP_{I \cap J}$. 
\end{remark}
% \begin{lemma} \label{parabolicintersection}
%    For $I \subset \Delta$, we have that 
%    $$\bP_I= \bigcap_{\alpha \in \Delta\backslash I}\bP_{\Delta\backslash\{\alpha\}}.$$ 
% \end{lemma}
% \begin{proof}
% Since $I\subset \Delta\backslash\{\alpha\}$ for each $\alpha \in \Delta\backslash I$, it follows immediately by the inclusion preserving bijection (\ref{parabolicbijection}) that $$\bP_I \subset \bigcap_{\alpha \in \Delta\backslash I}\bP_{\Delta\backslash\{\alpha\}}.$$
% On the other hand, all  $\bP_{\Delta\backslash\{\alpha\}}$ contain $\bB$. Hence, their intersection is a closed subgroup of $\bG$ containing $\bB$ and thus parabolic. Therefore, again by (\ref{parabolicbijection}), we have $$\bigcap_{\alpha \in \Delta\backslash I}\bP_{\Delta\backslash\{\alpha\}}=\bP_J$$
% for some $I \subseteq J \subseteq \Delta$. Then, we know that $\bP_J \subseteq \bP_{\Delta\backslash\{\alpha\}}$ for all $\alpha \in \Delta\backslash I$. By (\ref{parabolicbijection})

% $$J\subseteq \bigcap_{\alpha \in \Delta\backslash I}\Delta\backslash\{\alpha\}=\Delta \backslash \bigcup_{\alpha \in \Delta\backslash I} \{\alpha\} =I. $$ Hence, $I=J$.  
% \end{proof}
We define 
\begin{equation}\label{dominantweights}
   X^*(\bT)_I^+:=\Big\{\lambda \in X^*(\bT)\, \Big\vert \, \langle \lambda, \alpha^{\vee} \rangle \geq 0 \text{ for all } \alpha \in I \Big\}
\end{equation}
for $I \subset \Delta$  to be the set of \textit{$\bL_{\bP_{I}}$-dominant weights}. For $I=\Delta$, we just write $ X^*(\bT)^+$ and call it the set of \textit{dominant weights}. 

\begin{proposition}\label{parabolicweight}\cite[p. 502]{Le1}
   Let $\lambda \in X^*(\bT)^+$, $w \in W$ and  $I \subset \Delta$. If $w \in W^I$, then $w\cdot\lambda \in X^*(\bT)_I^+$.
\end{proposition}

\begin{remark}
   If $\lambda$ is additionally regular, i.e. $\langle \lambda +\rho, \alpha^\vee \rangle \neq 0$ for all $\alpha \in \Phi$ (cf. \cite[Section 1.8]{H2}), then also the converse holds (cf. \cite[Proposition 2.4]{B1}).
\end{remark}

The derived group $\bG_{\der}$ is a connected semi-simple subgroup of $\bG$ with maximal torus 
\begin{equation}\label{derivedtorus}
    \bT_\der:=\langle \Im(\alpha^\vee) \mid \alpha \in \Phi \rangle \subset \bT 
\end{equation}
(cf. \cite[Proposition 8.1.8/ Section 16.2.5]{Sp}). Moreover, $\bT_\der$ splits by \cite[Proposition 8.2.(c)]{Bor}. The natural map 
\begin{align} 
           X^*(\bT)&\longrightarrow X^*(\bT_\der) \label{torusmap}\\ 
            \lambda &\mapsto \lambda\circ \iota, \nonumber
\end{align}
induced by the inclusion $\iota$ (\ref{derivedtorus}), is injective after restriction to $\Phi$ (cf. \cite[Section 8.1, p. 135]{Sp}). Thus, we identify $\Phi$ with its image. Therefore, the split pair $(\bG_\der, \bT_\der)$ has the root datum 
$$(X^*(\bT_\der), \Phi, X_*(\bT_\der), \Phi^\vee)$$ (cf. \cite[Corollary 8.1.9]{Sp}) and we denote the associated pairing by $\langle \text{ , } \rangle_\der$.   
By semi-simplicity of $\bG_\der$, the simple roots $\Delta$ form a basis of $X^*(\bT_\der)_\BQ$ (cf. \cite[Part II, 1.6]{J}). Thus, we define
the dual basis 
\begin{equation}\label{dualbase}
   \{\varpi_\alpha \mid \alpha \in \Delta \} \subset X_*(\bT_\der)_\BQ,
\end{equation} i.e. $\langle \beta, \varpi_\alpha \rangle_\der = \delta_{\alpha,\beta}$ for all $\alpha, \beta \in \Delta$. Naturally, $\{\varpi_\alpha \}_{\alpha \in \Delta} \subset X_*(\bT)_\BQ$. 
By duality, the corresponding set of coroots $\{\alpha^\vee \mid \alpha \in \Delta\}$ forms a basis of $X_*(\bT_\der)_\BQ$ (cf. \cite[Part II, 1.6]{J}) and we analogously define the dual basis 
$$\{\check{\varpi}_\alpha \mid \alpha \in \Delta \} \subset X^*(\bT_\der)_\BQ$$
whose elements are known as \textit{fundamental weights}. A helpful observation for later is the following. 

\begin{lemma}\label{equivalentcond}
Let $\mu = \sum_{\alpha \in \Delta} n_\alpha \alpha^\vee \in X_*(\bT_\der)_\BQ \subset X_*(\bT)_\BQ$ with $n_\alpha \in \BQ$, and $\beta \in \Delta$. Then, $$(\mu, \varpi_\beta)>0 \text{ if and only if } \langle \check{\varpi}_\beta, \mu \rangle_\der >0.$$ 
\end{lemma}
\begin{proof} 
Let $\alpha \in \Delta$. We have by (\ref{dual}) and (\ref{products})
   $$ (\alpha^{\vee}, \varpi_\beta)=\big(\frac{2}{(\alpha^*,\alpha^*)}\alpha^*,\varpi_\beta\big)= \frac{2}{(\alpha^*,\alpha^*)} \langle  \alpha, \varpi_\beta \rangle.$$ %% = \frac{2}{(\alpha,\alpha)}\delta_{\alpha,\beta} $$%=\frac{2}{(\alpha,\alpha)}  \langle \check{\varpi}_\beta, \alpha^{\vee} \rangle$$
As the natural pairings are induced by the composition of a cocharacter with a character (cf. \cite[Part II, Section 1.3]{J}), we see that 
$$\langle  \alpha, \varpi_\beta \rangle=  \langle  \alpha, \varpi_\beta \rangle_\der.$$ Thus, 
$$ (\alpha^{\vee}, \varpi_\beta) = \frac{2}{(\alpha^*,\alpha^*)} \langle  \alpha, \varpi_\beta \rangle_\der=\frac{2}{(\alpha^*,\alpha^*)}\delta_{\alpha,\beta}.$$
Hence, 
\begin{equation*}
(\sum n_\alpha \alpha^{\vee}, \varpi_\beta )=\frac{2}{(\beta^*, \beta^*)} n_\beta> 0\text{ if and only if } \langle \check{\varpi}_\beta , \sum n_\alpha \alpha^{\vee} \rangle_\der= n_\beta>0. \qedhere
\end{equation*}
\end{proof}

Since the fundamental weights form a basis of $X^*(\bT_\der)_\BQ$, we notice that 
\begin{equation}\label{linearcombi}
   \alpha = \sum_{\beta \in \Delta} \langle \alpha, \beta^\vee{}\rangle_\der \check{\varpi}_\beta
\end{equation}
for $\alpha \in \Delta$.  After we fix an ordering on $\Delta=\{\alpha_1 < \alpha_2 < \ldots <  \alpha_r \}$, the \textit{Cartan matrix} is defined as 
\begin{equation}\label{cartanmatrix}
C \in \BQ^{\lb \Delta \rb \times \lb \Delta \rb } \text{ with } C_{ji}:=\langle \alpha_i, \alpha_{j}^\vee  \rangle_\der.
\end{equation}
Hence, by (\ref{linearcombi}), it is the base change matrix from $\{\alpha\}_{\alpha \in \Delta}$ to $\{\varpi_\alpha\}_{\alpha \in \Delta}$. For the inverse of $C$, we will need the following fact.
\begin{lemma}\label{inversecartan}
   Let $\Phi$ be irreducible. Then, all entries of $C^{-1}$ are positive rational numbers. 
\end{lemma}
\begin{proof}
   This is explained in \cite[Section 5,p. 19]{LG}.
\end{proof}
 With the definition of (\ref{dualbase}), we also obtain an alternative description of the standard parabolic subgroups of $\bG$. For a one-parameter subgroup $\mu \in X_*(\bG)$ defined over some field extension $L$ of $K$, we denote by $\bP(\mu)$ the parabolic subgroup of $\bG_L$ whose $\overline{K}$-valued points are given by 
\begin{equation*}
   \bP(\mu)(\ov K)=\big\{g \in \bG(\ov K)  \mid \lim_{t \rightarrow 0} \mu(t)g\mu(t)^{-1} \text{ exists in } \bG(\ov K) \big\}
\end{equation*}
(cf. \cite[Definition 2.3/Proposition 2.6]{M}). We have seen that $\langle \beta, \varpi_\alpha\rangle= \delta_{\alpha,\beta}$ for $\alpha, \beta \in \Delta$. Thus, \cite[Proof of Proposition 8.4.5/Lemma 15.1.2]{Sp} implies that $\bP_{\Delta\backslash\{\alpha\}}=\bP(\varpi_\alpha)$. We deduce from Remark \ref{intersectionparabolic} that 
\begin{equation*}
   \bP_I=\bigcap_{\alpha \not\in I}\bP_{\Delta\backslash\{\alpha\}}=\bigcap_{\alpha \not\in I}\bP(\varpi_\alpha)
\end{equation*}
for $I \subset \Delta$. \\

% Jantzen states in \cite[Introduction and Part II, Section 1]{J} that split reductive groups and constructions like Borel und Parabolic subgroups can be carried out over $\BZ$,
% and therefore, by base change, over any integral domain. This is based on the following theorem. 

% \begin{theorem}\cite[Exp. XXV, Corollary 1.3]{SGA3III}
%    Let $K$ be a field. Then, for any split connected reductive group $\bG$ over $K$ exists a reductive $\BZ$-group $\CG$ so that $$\CG\otimes_\BZ K \cong \bG.$$ 
% \end{theorem}
% As remarked in \cite[Exp. XXV, Section 1]{SGA3III} after the above statement, $\CG$ can be assumed to be split. \\

% In the case that $K$ is a local field with ring of integers $\CO_K$ and $\bG$ a split connected reductive group over $K$, this implies 
% that there is a split reductive group $\bG_0$ over $\CO_K$ such that $(\bG_0)_K \cong \bG$. Furthermore, $(\bG_0)_k$ and $\bG$ have the same root datum for $k$ the residue field of $\CO_K$. 
% We call $\bG_0$ a \textit{split reductive group model} of $\bG$ over $\bO_K$. \\

% Here $\bP(\mu)$ stands for the associated parabolic subgroup of $\bG_E$ with $\ov K$-valued points
% $$\bP(\mu)(\ov K)=\big\{g \in \bG(\ov K)  \mid \lim_{t \rightarrow 0} \mu(t)g\mu(t)^{-1} \text{ exists in } \bG(\ov K) \big\}.$$
% Last but not least, we  will consider an example that can be kept in mind for the upcoming chapters.  
% \begin{example}\label{GL_n}
%    Let $K=\BQ_p$, $n\in \BN$ and $\bG=\GL_n$ over $K$. We let $\bT$ be the algebraic subgroup of diagonal matrices. 
%    Then, we identify $X^*(\bT)$ with $\BZ^n$ by associating the character 
%          $$\lambda:(t_1,\ldots,t_n) \mapsto  \prod t_i^{\lambda_i}$$
%    to  $(\lambda_1,\ldots, \lambda_n) \in \BZ^n$. 
%    Similarly, $\BZ^n \cong X_*(\bT)$ by mapping $(\mu_1,\ldots, \mu_n) \in \BZ^n$ to the cocharacter 
%    $$ \mu:z \mapsto  (z^{\mu_1},\ldots z^{\mu_n}).$$
%    Then, the pairing (\ref{rootpairing}) is the usual inner product of $\BQ^n$. Furthermore,  
%    $$ \Phi=\Phi^\vee=\{e_i-e_j \mid 1 \leq i \neq j \leq n \} $$ 
%    with $e_i$ the $i$-th standard unit vector of $\BQ^n$. Hence, $$\rho=\frac{1}{2}(n-1,n-3,\ldots,-(n-3),-(n-1)) \in \BZ^n.$$ If we choose $\bB$ to be the algebraic subgroup 
%    of upper triangular matrices, then 
%    $$\Delta=\{\alpha_i:=e_i-e_{i+1} \mid 1 \leq i \leq n-1\}.$$ 
%    Moreover, $W=S_n$ and $s_i:=s_{\alpha_i}$ is the transposition $(i,i+1)$. Then, $W$ acts on $X_*(\bT)$ and $X^*(\bT)$ by permuting entries, respectively. Let $I \subset \Delta$ and $\Delta\backslash I=\{\alpha_{i_1}, \alpha_{i_2}, \ldots, \alpha_{i_r}\}$ with  $0=i_0<i_1 < i_2 <\ldots < i_r$. Then, $\bP_I$ is the algebraic subgroup such that $\bP_I(\overline{K})$ consists of matrices with $\GL_{i_{j+1}-i_{j}}(\overline{K})$-blocks along the main diagonal (ordered by the $i_j$), zeros below and arbitrary entries above. Furthermore, 
%    $$ X^*_I(\bT)^+=\{(\lambda_1, \ldots, \lambda_n) \in \BZ^n \mid \lambda_{i} \geq \lambda_{i+1} \text{ for all } \alpha_i \in I\}.$$ 

%    \noindent The derived subgroup $\bG_\der$ of $\bG$ is $\SL_n$, with $\bT_\der(\overline{K})=\bT(\overline{K}) \cap \SL_n(\overline{K}).$ In addition, 
%    $$\varpi_{\alpha_i}=\check{\varpi}_{\alpha_i}=\frac{1}{n}(n-i,\ldots,n-i,-i \ldots ,-i) \in \BZ^n$$
%    with $(\varpi_{\alpha_i})_i=n-i$ and $(\varpi_{\alpha_i})_{i+1}=-i$.
% \end{example}
% % We denote for by $\rho$ the half sum of positive roots. Then 
% % the so-called \textit{dot-action} of $W$ on $X^*(\bT)$ is given by $$w \cdot \lambda =w(\lambda + \rho)-\rho.$$

\subsection{The functor $\CF^G_P$}\label{s:FGP}

Let the ground field $K$ be a finite extension of $\BQ_p$ and $(\bG,\bT)$ a split pair over $K$ of rank $d$ (cf. section \ref{s:rootdatum}). Further, let $(\bT, \bB$) be a fixed Borel pair and $L$ a finite extension of $K$. Let $G=\bG(K)$ and $P=\bP(K)$ for some standard parabolic subgroup $\bP$ of $\bG$. As mentioned in \cite[p. 443]{ST2}, $G$ and $P$ are locally $K$-analytic groups.  \\

Orlik and Strauch defined in \cite{OSt} the bi-functor 

\begin{align*}\CF^G_P: \CO^{\fkp}_{\alg} \times \Rep^{\infty,\mathrm{adm}}_L(L_P) &\longrightarrow \Rep^{\ell a}_{L}(G)  \\
(M,V) &\mapsto \CF^G_P(M,V)
\end{align*}
which is contravariant in $M$ and covariant in $V$ (cf. \cite[Proposition 4.7]{OSt}). In case $V$ is the trivial representation $\textbf{1}$, we will write $\CF^G_P(M)$. We briefly explain the basics about this functors and start with the mentioned categories. \\

Let $V$ be a Hausdorff barelled locally convex $L$-vector space. Then, $C^{an}(G,V)$ is the \textit{locally convex $L$-vector space of locally $L$-analytic functions on $G$ with values in $V$} (see \cite[Section 2, p. 447]{ST2} for a detailed description). 
Further, $$D(G):=C^{an}(G,L)'$$ is the \textit{locally convex vector space of $L$-valued distributions on G} (cf. \cite[Section 2, Definition, p. 447]{ST2}). Additionally, with convolution as multiplication, it is an associative $L$-algebra  (cf. \cite[Proposition 2.3]{ST2}).
% A prominent class of elements of $D(G)$ is that of \textit{Dirac distributions} $\delta_g$, for $g \in G$, defined by $$\delta_g(f)=f(g).$$
%We will use analog definitions for $P$.  
%Note that $D(P)$ is a subalgebra of $D(G)$.

\begin{definition}\cite[Section 3, p. 451, Definition]{ST2} A \textit{locally analytic $G$-representation} $V$ (over $L$) is a Hausdorff barelled locally convex $L$-vector space $V$ equipped with a $G$-action by continuous linear endomorphisms such that, for each $v \in V$, the orbit map $\rho_v(g):=gv$ lies in $C^{an}(G,V)$.    We denote the category of such representations by $\Rep^{\ell a}_{L}(G)$. 


\end{definition}
% \begin{definition}\cite[Section 2.1, p. 103]{OSt}
%    A locally analytic $G$-representation $V$ is called \textit{strongly admissible} if $V$ is of compact type and $V'$ is a finitely generated $D(K)$-module for any compact open subgroup $K$ of $G$. 
% \end{definition} 
As in the algebraic or smooth case,   we also have the induction functor. 
\begin{definition}\cite[Section 2.2, p. 103]{OSt}
    Let $H$ be a closed subgroup of $G$ and $(V,\rho)$ a locally analytic representation of $H$. The \textit{locally analytic induced representation $\Ind^G_H(V)$} is defined as 
$$  \Ind^G_H(V)=\{f \in C^{an}(G,V) \mid  f(gh)=\rho(h^{-1})f(g) \, \forall h \in H, \forall g \in G \}.$$
The group $G$ acts on $\Ind^G_H(V)$ by $(g.f)(x)=f(g^{-1}x)$. 
\end{definition}


Over the complex numbers, the BGG category $\mathcal{O}$ and its parabolic version $\mathcal{O}^{\mathfrak{p}}$ provide powerful tools to investigate (infinite dimensional) representations of Lie algebras. A good reference for this topic is \cite{H2}. For our setting, this was considered in detail in \cite[Section 2.5]{OSt} by Orlik and Strauch.

\begin{definition}\cite[p. 106]{OSt} \label{catOp}
   By $\mathcal{O}^{\mathfrak{p}}$ we denote the full subcategory of 
   $\operatorname{Mod}U(\mathfrak{g})$ whose objects $M$ satisfy  the following conditions: 
\begin{enumerate}[leftmargin=11mm]
    \item[($\mathcal{O}^\fkp1$)] $M$ is a finitely generated $U(\fkg)$-module. 
    \item[($\mathcal{O}^\fkp2$)] Viewed as an $\fkl_{\RP,L}$-module, $M$ is the direct sum of finite dimensional simple modules.
    \item[($\mathcal{O}^\fkp3$)]$M$ is locally $\fku_{\RP,L}$-finite.
   \end{enumerate}       
\end{definition}

If $\bP$ is a Borel, we denote the category by $\mathcal{O}$. Notice  that $\mathcal{O}^{\fkg}$ is the 
category of all finite dimensional (semisimple) $U(\mathfrak{g})$-modules. Moreover, for a standard parabolic $\mathbf{Q} \supset \mathbf{P} $, we have that $\mathcal{O}^{\mathfrak{q}} \subset 
\mathcal{O}^{\mathfrak{p}}$. Hence, $\mathcal{O}^{\mathfrak{p}}$ is a full subcategory of $\mathcal{O}$ 
and contains all finite dimensional  $U(\mathfrak{g})$-modules. Additionally, $\mathcal{O}^{\mathfrak{p}}$ 
is an $L$-linear, abelian, artinian and noetherian category which is closed under taking submodules and quotients. 
Thus, the Jordan-Hölder series of an object of $\mathcal{O}^{\mathfrak{p}}$ lies in $\mathcal{O}^{\mathfrak{p}}$. \\

Letting $\operatorname{Irr}(\mathfrak{l}_{P,L})^{\operatorname{fd}}$ be the set of 
isomorphism classes of finite dimensional irreducible $\mathfrak{l}_{P,L}$-modules, we have for $M \in \mathcal{O}^{\mathfrak{p}}$ that 
\begin{equation*}
    M=\bigoplus_{\mathfrak{a} \in \operatorname{Irr}(\mathfrak{l}_{P,K})^{\operatorname{fd}}} 
    M_\mathfrak{a},
\end{equation*}
by property ($\mathcal{O}^\fkp2$) in Definition \ref{catOp}, with $M_\mathfrak{a} \subset M$ being the $\mathfrak{a}$-isotypic part of the representation 
$\mathfrak{a}$. Similiar to before there is an algebraic subcategory in $\mathcal{O}^{\mathfrak{p}}$.

\begin{definition}\cite[p. 106]{OSt}  Let $\mathcal{O}^{\mathfrak{p}}_{\mathrm{alg}}$ be the full subcategory of 
$\mathcal{O}^{\mathfrak{p}}$ with objects $M \in \mathcal{O}^{\mathfrak{p}}$ satisfying the following property: 
whenever $M_\mathfrak{a}\neq 0$, then,  $\mathfrak{a}$ is induced by a finite dimensional algebraic 
$\bL_{\bP,L}$-representation. 
\end{definition}

\begin{definition}\cite[Section 1.15]{H2}
   Let $M \in \CO$. The \textit{formal character} of $M$  is defined as 
   \begin{align*}
      \ch(M):\fkt^*_L &\longrightarrow \BZ^+  \\
             \lambda &\mapsto \dim_L(M_\lambda).
   \end{align*}
\end{definition}

Let $M \in \CO^{\fkp}_{\alg}$. Then, by the very definition of the category $\CO^{\fkp}_{\alg}$, there is a finite-dimensional representation $(W,\rho) \subset M$ of $\fkp_L$ which generates $M$ as $U(\fkg)$-module. We call such a tuple $(M,W)$ an \textit{$\CO^{\fkp}_{\alg}$-pair}. Hence, such a pair comes with a short exact sequence of $U(\fkg)$-modules
\begin{equation}\label{sesCatO} 
   0 \rightarrow \fkd \rightarrow U(\fkg) \otimes_{U(\fkp)} W \rightarrow M \rightarrow 0
\end{equation}
with $\fkd$ being the kernel of the natural map $ U(\fkg) \otimes_{U(\fkp)} W \rightarrow M$. 



\begin{example}\cite[Example 2.10]{OSt} \label{genVM} Let $I \subset \Delta$ such that $\bP=\bP_I$.  For  $\lambda \in X^*(\bT)_I^+$, there is a corresponding finite dimensional 
   irreducible algebraic $\bL_{\bP,L}$-representation $V_I(\lambda)$, 
   which can be viewed as a $\bP_L$-representation by letting $\bU_{\bP,L}$ act trivially on it. Then, 
   $$M_I(\lambda)=U(\fkg) \otimes_{U(\fkp)} V_I(\lambda)$$
   is the \textit{generalized parabolic Verma module} associated to $\lambda$, with simple quotient $L(\lambda)$ which both lie in $\CO^{\fkp}_{\alg}$. In case $I=\emptyset$ we omit the subscript. We have a surjection
               $$q_I:M(\lambda) \rightarrow M_I(\lambda)$$ 
   with the kernel being the image of $\bigoplus_{\alpha \in I} M(s_{\alpha} \cdot \lambda) \rightarrow M(\lambda)$ (cf. \cite[Proposition 2.1]{Le2}).
   Furthermore, for $J \subset I$, there is a transition map 
   \begin{equation}\label{transition}
     q_{J,I}:M_J(\lambda) \rightarrow M_I(\lambda)
   \end{equation}
  such that $q_I=q_{I,J}\circ q_J$ (cf. \cite[Section 2, p. 653]{OSch}).
\end{example}

 \begin{lemma}\label{Verma}\cite[Lemma 1]{B}
Let $\lambda, \mu \in \fkt_L^*$ and $M$ a $U(\fkg)$-submodule of $M(\lambda)$, such that $\ch(M)=\ch(M(\mu))$. Then, $M$ is isomorphic to $M(\mu)$. 
\end{lemma}

Next, we say a few words about the construction of the functor $\CF^G_P$. For this, let $V \in \Rep^{\infty,\, \mathrm{adm}}_L(L_P)$. By inflation, we consider $V$ as a representation of $P$. Equipping $V$ with the finest locally convex  $L$-vector space topology, it is of compact type and carries the structure of a locally analytic $P$-representation (cf. \cite[p. 117]{OSt}). For an $\CO^{\fkp}_{\alg}$-pair $(M,W)$, Orlik und Strauch consider $W'\otimes_L V$ as the projective (or inductive) tensor product which is complete and a locally analytic $P$-representation via the diagonal action (cf. \cite[p. 117]{OSt}). Then, they defined\footnote{\begin{equation*}
   \langle \, , \, \rangle_{C^{an}(G,V)}:D(G)\otimes_{D(P)} W \times \Ind^G_P(W'\otimes V) \longrightarrow C^{an}(G,L), \,
   (\delta \otimes w, f)\mapsto \Big[g \mapsto \Big(\delta \cdot_r\big(ev_w \circ f \big)\Big)(g)\Big]     
   \end{equation*}} (cf. \cite[(4.4.1), p. 117]{OSt})
\begin{align*}
   \CF^G_P(M,W,V)&:= \Ind^G_P(W'\otimes V)^\fkd \\
                 &:=\{f \in \Ind^G_P(W'\otimes V) \mid \langle \delta  , f \rangle_{C^{an}(G,V)} =0 \text{ for all } \delta \in \fkd \}.
\end{align*}

The definition is independent of the chosen $\CO^{\fkp}_{\alg}$-pair $(M,W)$ (cf. \cite[Section 4.6]{OSt}). Therefore, we write $\CF^G_P(M,V)$ for any $\CF^G_P(M,W, V)$. One of the most basic properties of the functor $\CF^G_P$ is the following. 
\begin{proposition}\label{exact}\cite[Proposition 4.9]{OSt} \begin{enumerate}[label=\roman*)]
   \item The bi-functor $\CF^G_P$ is exact in both arguments. 
   \item If $Q \supset P$ is a parabolic subgroup, $\fkq=\Lie(Q)$, and $M \in \CO^{\fkq}_{\alg}$, then 
               $$\CF^G_P(M,V)= \CF^G_Q\big(M,i^{L_Q}_{L_P(L_Q\cap U_P)}(V)\big)$$
   where $i^{L_Q}_{L_P(L_Q\cap U_P)}(V)=i^Q_P(V)$ denotes the corresponding induced representation in the category of smooth representations. 
\end{enumerate}
\end{proposition}

We devote the last part of this subsection to an application of the functor $\CF_P^G$. Let $I \subset \Delta$ and $P_I$ be the associated parabolic subgroup of $G$. Then, we notice that 
$$ \Ind^G_{P_I}(\textbf{1})=\CF_{P_I}^G(M_I(0))$$ 
where $0$ is the weight sent to the zero vector under the identification $X^*(\bT)\cong \BZ^d$. More generally, for $\lambda \in X^*(\bT)^+$ (cf. (\ref{dominantweights})), we have 
\begin{equation}\label{IGP}
   I^G_{P_I}(\lambda):=\Ind^G_{P_I}(V_I(\lambda)')=\CF_{P_I}^G(M_I(\lambda))
\end{equation}
since the $\CO_{\alg}^{\fkp_I}$-pair $(V_I(\lambda), M_I(\lambda))$ has trivial kernel $\fkd$ (cf. (\ref{sesCatO})). For $I \subset J \subset \Delta$, the morphism $q_{I,J}$ (cf. (\ref{transition})) induces by the functoriality of the functor $\CF^G_P$ a map 
$$ p_{J,I}:  I^G_{P_J}(\lambda)=\CF_{P_J}^G(M_J(\lambda), \textbf{1}) \xrightarrow{\CF^G_{P_J}(q_{I,J}, \, \mathrm{incl.})} \CF_{P_J}^G(M_I(\lambda), i^{P_J}_{P_I})\cong\CF_{P_I}^G(M_I(\lambda))=I^G_{P_I}(\lambda)$$
of locally analytic $G$-representations. Furthermore, the map $p_{J,I}$ is injective and has closed image (cf. \cite[p. 660]{OSch}). 
\begin{definition}\cite[p. 661]{OSch}
For $I\subset \Delta$, 
$$ V^G_{P_I}(\lambda):= I^G_{P_I}(\lambda)\Big/ \sum_{J \supsetneq I} I^G_{P_J}(\lambda) $$
is the \textit{twisted generalized Steinberg representation} associated to $\lambda$. 
\end{definition}
It has the following resolution in $\Rep^{\ell a}_{L}(G)$. 

\begin{theorem}\cite[Theorem 4.2]{OSch}
Let $\lambda \in X^*(\bT)^+$ and $I \subset \Delta$. Then, the following complex is a resolution of $V^G_{P_I}(\lambda)$ by locally analytic $G$-representations, 
\begin{align*}
   0 \rightarrow I^G_G(\lambda) \rightarrow \bigoplus_{\substack{I \subset K \subset \Delta \\ \lb \Delta \backslash K \rb = 1}} I^G_{P_K}(\lambda) &\rightarrow \bigoplus_{\substack{I \subset K \subset \Delta \\ \lb \Delta \backslash K \rb = 2}} I^G_{P_K}(\lambda) \rightarrow \ldots \\
   \ldots &\rightarrow \bigoplus_{\substack{I \subset K \subset \Delta \\ \lb K \backslash I \rb = 1}} I^G_{P_K}(\lambda) \rightarrow I^G_{P_I}(\lambda) \rightarrow V^G_{P_I}(\lambda)\rightarrow 0. 
\end{align*}
\end{theorem}

Here, the differentials $d_{K',K}: I^G_{P_{K'}}(\lambda) \rightarrow I^G_{P_{K}}(\lambda)$ are defined as follows (\cite[p. 660]{OSch}). We fix an ordering on $\Delta$. Let $K, K' \subset \Delta$ with $\lb K \rb= \lb K' \rb -1$ and $K'=\{\alpha_1 < \ldots < \alpha_r \}$. Then, 
$$ d_{K',K}= \begin{cases} (-1)^i p_{K',K}  &K'=K \cup \{\alpha_i\} \\
                                 0       & K \not\subset K' 
\end{cases}.$$
We like to stress a relative version which was shown in \cite[p. 663]{OSch} in the proof of the previous theorem. For this we follow the notion of \cite[p. 661]{OSch}. 

\begin{definition}\label{twistedSteinberg}
Let $\lambda \in X^*(\bT)^+$, $I \subset \Delta$ and $w \in W^I$. By Proposition \ref{parabolicweight}, we know that $w \cdot \lambda \in X^*(\bT)_I^+$. Then, we set  

\begin{align}
    I_{P_I}^G(w)&:=\Ind^G_{P_I}(V_I(w \cdot \lambda)')=\CF^G_{P_I}(M_I(w \cdot \lambda)) \label{twisted}, \\
    V_{P_I}^G(w)&:=I^G_{P_I}(w)\Big/\sum_{\substack{ J \supsetneq I \\ w \in W^{J}}} I_{P_{J}}^G(w). \nonumber
\end{align}
\end{definition}

\begin{corollary}\label{relativeresolution} Let $\lambda \in X^*(\bT)^+$, $I \subset \Delta$ and $w \in W^I$. Then, the following complex is acyclic 
   \begin{align*}
      0 \rightarrow I^G_{P_{I(w)}}(w) \rightarrow \ldots \rightarrow  \bigoplus_{\substack{I \subset K \subset I(w) \\ \lb K \backslash I \rb = 1}} I^G_{P_K}(w) \rightarrow I^G_{P_{I}}(w) \rightarrow V^G_{P_{I}}(w)\rightarrow 0. 
   \end{align*}
\end{corollary}

In \cite[Theorem 4.6]{OSch}, it was shown that the Jordan-Hölder factors of $V^G_B(\lambda)$ are of the form $\CF^G_{P_I}\Big(L(w\cdot \lambda), v^{P_I}_{P_J}\Big)$ for suitable $I,J \subset \Delta$ and $w \in W$. We will use Corollary \ref{relativeresolution} to get a similar statement for $V^G_B(w)$ which partially generalizes \cite[Theorem 4.6]{OSch}. For $w,v \in W$, we denote by $m(w,v) \in \BZ_{\geq 0}$ the \textit{multiplicity} of $L(v \cdot 0)$ in $M(w \cdot 0)$. It is well known that $m(w,v)>0$ if and only if $ w \leq v$ with respect to the Bruhat order $\leq$ on $W$. The multiplicities can be computed using Kazhdan-Lusztig polynomials (cf. \cite{BB} or \cite{BK}) which is in general only possible in a timely manner with the help of a computer. 

\begin{theorem}\label{multiplicities} \footnote{ 
   We have to assume the following( cf. \cite[Section 5]{OSt}). If the root sytem $\Phi(\bG,\bT)$ has irreducible components of type $B$, $C$, or $F_4$, we assume $p>2$, and if $\Phi(\bG,\bT)$ has irreducible components of type $G_2$, we assume that $p>3$.} Fix $w,v \in W$ and let $I_0:=I(w)$ and $I:=I(v)$, respectively, be as above. For a subset $J \subset \Delta$ with $J \subset I$, 
    let $v_{P_J}^{P_I}$ be the generalized smooth Steinberg representation of $L_{P_I}$. Then, the multiplicity of the irreducible $G$-representation 
    $\CF^G_{P_I}(L(v\cdot \lambda),v_{P_J}^{P_I})$ in $V^G_B(w)$ is 
    $$ \sum_{\substack{w' \in W  \\ \supp(w')=J \cap I_0}} (-1)^{\ell(w')+\lb J \cap I_0 \rb} m(w'w,v)$$ 
    and in this way we obtain all the Jordan-Hölder factors of $V^G_B(w)$. 
\end{theorem} 

\begin{proof} 
    We only have to slightly modify the proof of \cite[Theorem 4.6]{OSch}.  %check language 
    From the resolution for $V^G_B(w)$ by Corollary \ref{relativeresolution}, we obtain the multiplicity 
        $$\big[V_B^G(w): \CF^G_{P_I}(L(v\cdot \lambda),v_{P_J}^{P_I})\big]=\sum_{K \subset I_0}(-1)^{\lb K \rb}\big[I^G_{P_K}(w):\CF^G_{P_I}(L(v\cdot \lambda),v_{P_J}^{P_I})\big]$$
   of the simple object $\CF^G_{P_I}(L(v\cdot \lambda),v_{P_J}^{P_I})$ in $V^G_B(w)$. 
    By the arguments mentioned in loc. cit, it follows that $\big[I^G_{P_K}(w):\CF^G_{P_I}(L(v\cdot \lambda),v_{P_J}^{P_I})\big] \neq 0$ if only if  $K \subset J \cap I_0$. In that case we have 
    $$ \big[I^G_{P_K}(w):\CF^G_{P_I}(L(v\cdot \lambda),v_{P_J}^{P_I})\big] = \big[M_K(w \cdot \lambda): L(v\cdot \lambda)\big].$$
    From the character formula 
        $$\ch(M_K(w\cdot \lambda))= \sum_{w' \in W_K} (-1)^{\ell(w')} \ch(M(w'w\cdot \lambda)),$$ 
    (cf. \cite[Section 9.6, p. 189, Proposition]{H2}), we obtain 
    \begin{align*} \big[V_B^G(w): \CF^G_{P_I}(L(v\cdot \lambda),v_{P_J}^{P_I})\big] &= \sum_{K \subset J \cap I_0}(-1)^{\lb K \rb} \sum_{w' \in W_K} (-1)^{\ell(w')} \big[M(w'w\cdot \lambda):L(v \cdot \lambda)\big] \\
     &= \sum_{w' \in W} (-1)^{\ell(w')}\big[M(w'w\cdot \lambda):L(v \cdot \lambda)\big]  \sum_{\substack{K \subset J\cap I_0 \\ \supp(w') \subset K}} (-1)^{\lb K \rb}.
    \end{align*}
    Finally, we have 
    $$ \sum_{\supp(w') \subset K \subset J\cap I_0} (-1)^{\lb K \rb}=(-1)^{\supp(w')}(1-1)^{\lb (J \cap I_0) \backslash \supp(w')\rb}$$
    which is non-zero if and only if $\supp(w')=J \cap I_0$. Hence, the formula follows. \\

    The natural morphism $V^G_B(\lambda)\rightarrow V^G_B(w)$ is surjective for all $w \in W$ as it is induced by an injective morphism $M(w \cdot \lambda) \rightarrow M(\lambda)$ (cf. Lemma \ref{surjection}).  
    Therefore, \cite[Theorem 4.6]{OSch} implies that we obtain all Jordan-Hölder factors of $V^G_B(w)$ in this manner.  
\end{proof}


 




% \subsection{The BGG categories $\CO$ and $\CO^{\fkp}$}\label{s:catO} Let the ground field $K$ be a finite extension of $\BQ_p$ and $(\bG,\bT)$ a split pair over $K$ of rank $d$ (cf. section \ref{s:rootdatum}). Further, let $(\bT, \bB$) be a fixed Borel pair and $L$ a finite extension of $K$. \\



% Over the complex numbers, the BGG category $\mathcal{O}$ and its parabolic version $\mathcal{O}^{\mathfrak{p}}$ provide powerful tools to investigate (infinite dimensional) representations of Lie algebras. A good reference for this topic is \cite{H2}. 
% The goal of this chapter is to recall the adaption of these notions 
% to the case where the coefficient field $L$ is not algebraically closed. 
% For our setting, this was considered in detail in \cite[Section 2.5]{OSt} by Orlik and Strauch. 
% \begin{definition}\cite[p. 105]{OSt} \label{catO}
%    The category $\mathcal{O}$ is defined to be the full subcategory of 
%    $\operatorname{Mod} U(\mathfrak{g})$ whose objects $M$ satisfy the following conditions: 
%     \begin{enumerate}[leftmargin=9.5mm]
%         \item[($\mathcal{O}1$)]$M$ is a finitely generated $U(\mathfrak{g})$-module. 
%         \item[($\mathcal{O}2$)] $M$ is $\mathfrak{t}_L$-semisimple, i.e.  $M=\bigoplus_{\lambda \in \mathfrak{t}_L^*} 
%         M_\lambda$. \label{weightdec}
%         \item[($\mathcal{O}3$)]$M$ is locally $\mathfrak{b}_L$-finite, i.e. for each $v \in M$ one has that $U(\mathfrak{b}) 
%         \cdot v \subset M$ is a finite dimensional $L$-vector space. 
%     \end{enumerate}
% \end{definition}

% Here, for $\lambda \in \mathfrak{t}_L^*=\Hom_L(\mathfrak{t}_L,L)$, we denote  by 
% \begin{equation*}\label{Wactiont}
%     M_\lambda=\{v \in M \mid t \cdot v=\lambda(t)v \text{ for all } t \in \mathfrak{t}\}
% \end{equation*}
% the $\lambda$-eigenspace of $M$. Furthermore, by derivation we consider $X^*(\bT)$ as a subgroup of $\fkt_L^*$.
% \begin{definition}\cite[Section 1.15]{H2}
%    Let $M \in \CO$. The \textit{formal character} of $M$  is defined as 
%    \begin{align*}
%       \ch(M):\fkt^*_L &\longrightarrow \BZ^+  \\
%              \lambda &\mapsto \dim_L(M_\lambda).
%    \end{align*}
%    \end{definition}
   
%    \begin{remark}
%       It is also common to write 
%       $$\ch(M)=\sum_{\lambda \in \fkt^*_L}  \dim_L(M_\lambda) e(\lambda)$$
%       for the formal character of $M \in \CO$. Here, $e(\lambda)$ is the characteristic function which is 1 for $\lambda$ and zero else. 
%    \end{remark}

% Moreover, Orlik and Strauch defined a certain subcategory of $\mathcal{O}$ which will play an important role for upcoming sections. 

% \begin{definition}\cite[Definition 2.6]{OSt}
% Let $\mathcal{O}_{\mathrm{alg}}$ be the full subcategory of 
% $\mathcal{O}$ whose objects are $U(\fkg)$-modules $M$ such that all $\lambda$ appearing in $(\mathcal{O}2)$, for which $M_\lambda \neq 0$,
% are contained in $X^*(\bT) \subset \fkt_L^*$.
% \end{definition} 

% \begin{example}\cite[Example 2.7]{OSt}\label{VermaModule}  Let $\lambda \in \fkt_L^*$. The action of $\fkt_L$ on $L$ given by $\lambda$ defines the $\fkt_L$-module $L_\lambda$ 
% which extends uniquely to a $\fkb_L$-module. Then, 
% $$M(\lambda)=U(\fkg) \otimes_{U(\fkb)} L_\lambda$$
% is the  \textit{Verma module} corresponding to $\lambda$ and $L(\lambda) \in \CO$ is its unique simple quotient. Notice that $M(\lambda)$ and $L(\lambda)$, respectively, lies in $\CO_{\alg}$ if and only if $\lambda \in X^*(\bT)$. 
% \end{example}

% \begin{lemma}\label{Verma}\cite[Lemma 1]{B}
% Let $\lambda, \mu \in \fkt_L^*$ and $M$ a $U(\fkg)$-submodule of $M(\lambda)$, such that $\ch(M)=\ch(M(\mu))$. Then, $M$ is isomorphic to $M(\mu)$. 
% \end{lemma}

% There is also a parabolic version of the category $\CO$. To define it, let $\bP$ be a standard parabolic subgroup of $\bG$ with respect to $\bB$. 

% % $,itemsep=\the\smallskipamount}
% \begin{definition}\cite[p. 106]{OSt} \label{catOp}
%    By $\mathcal{O}^{\mathfrak{p}}$ we denote the full subcategory of 
%    $\operatorname{Mod}U(\mathfrak{g})$ whose objects $M$ satisfy  the following conditions: 
% \begin{enumerate}[leftmargin=11mm]
%     \item[($\mathcal{O}^\fkp1$)] $M$ is a finitely generated $U(\fkg)$-module. 
%     \item[($\mathcal{O}^\fkp2$)] Viewed as an $\fkl_{\RP,L}$-module, $M$ is the direct sum of finite dimensional simple modules.
%     \item[($\mathcal{O}^\fkp3$)]$M$ is locally $\fku_{\RP,L}$-finite.
%    \end{enumerate}       
% \end{definition}

% First, notice that $\mathcal{O}=\mathcal{O}^{\mathfrak{b}}$ and that $\mathcal{O}^{\fkg}$ is the 
% category of all finite dimensional (semisimple) $U(\mathfrak{g})$-modules. Moreover, for a standard 
% parabolic $\mathbf{Q} \supset \mathbf{P} $, we have that $\mathcal{O}^{\mathfrak{q}} \subset 
% \mathcal{O}^{\mathfrak{p}}$. Hence, $\mathcal{O}^{\mathfrak{p}}$ is a full subcategory of $\mathcal{O}$ 
% and contains all finite dimensional  $U(\mathfrak{g})$-modules. Additionally, $\mathcal{O}^{\mathfrak{p}}$ 
% is an $L$-linear, abelian, artinian and noetherian category which is closed under taking submodules and quotients. 
% Again, the Jordan-Hölder series of an object of $\mathcal{O}^{\mathfrak{p}}$ lies in $\mathcal{O}^{\mathfrak{p}}$. \\

% %\todo[inline]{\textbf{Question}: Quote Humphreys or is Orlik/Strauch enough?}
% Letting $\operatorname{Irr}(\mathfrak{l}_{P,L})^{\operatorname{fd}}$ be the set of 
% isomorphism classes of finite dimensional irreducible $\mathfrak{l}_{P,L}$-modules, we have for $M \in \mathcal{O}^{\mathfrak{p}}$ that 
% \begin{equation*}
%     M=\bigoplus_{\mathfrak{a} \in \operatorname{Irr}(\mathfrak{l}_{P,K})^{\operatorname{fd}}} 
%     M_\mathfrak{a},
% \end{equation*}
% by property ($\mathcal{O}^\fkp2$) in Definition \ref{catOp}, with $M_\mathfrak{a} \subset M$ being the $\mathfrak{a}$-isotypic part of the representation 
% $\mathfrak{a}$. Similiar to before there is an algebraic subcategory in $\mathcal{O}^{\mathfrak{p}}$.

% \begin{definition}\cite[p. 106]{OSt}  Let $\mathcal{O}^{\mathfrak{p}}_{\mathrm{alg}}$ be the full subcategory of 
% $\mathcal{O}^{\mathfrak{p}}$ with objects $M \in \mathcal{O}^{\mathfrak{p}}$ satisfying the following property: 
% whenever $M_\mathfrak{a}\neq 0$, then,  $\mathfrak{a}$ is induced by a finite dimensional algebraic 
% $\bL_{\bP,L}$-representation. 
% \end{definition}


% Then, $\mathcal{O}^{\mathfrak{b}}_{\mathrm{alg}}=\mathcal{O}_{\mathrm{alg}}$ and furthermore, $\mathcal{O}^{\mathfrak{p}}_{\mathrm{alg}}$ is 
% an abelian, artinian, noetherian category which is closed  under taking submodules and quotients.

% \begin{definition}\label{maximal}\cite[Definition 5.2]{OSt}
%    Let $M \in \CO$. A parabolic subalgebra $\fkp$ (and the corresponding parabolic subgroup $\bP$, respectively) is called \textit{maximal} for $M$  if $M \in \CO^\fkp$ and $M \notin \CO^\fkq$ for all parabolic 
%    subalgebras $\fkq$ strictly containing $\fkp$.    
% \end{definition}

% \begin{example}\cite[Example 2.10]{OSt} \label{genVM} Let $I \subset \Delta$ such that $\bP=\bP_I$.  For  $\lambda \in X^*(\bT)_I^+$, there is a corresponding finite dimensional 
%     irreducible algebraic $\bL_{\bP,L}$-representation $V_I(\lambda)$, 
%     which can be viewed as a $\bP_L$-representation by letting $\bU_{\bP,L}$ act trivially on it. Then, 
%     $$M_I(\lambda)=U(\fkg) \otimes_{U(\fkp)} V_I(\lambda)$$
%     is the \textit{generalized parabolic Verma module} associated to $\lambda$, which lies in $\CO^{\fkp}_{\alg}$. In case $I=\emptyset$ we omit the subscript. We have a surjection
%                 $$q_I:M(\lambda) \rightarrow M_I(\lambda)$$ 
%     with the kernel being the image of $\bigoplus_{\alpha \in I} M(s_{\alpha} \cdot \lambda) \rightarrow M(\lambda)$ (cf. \cite[Proposition 2.1]{Le2}).
%     Furthermore, for $J \subset I$, there is a transition map 
%     \begin{equation}\label{transition}
%       q_{J,I}:M_J(\lambda) \rightarrow M_I(\lambda)
%     \end{equation}
%    such that $q_I=q_{I,J}\circ q_J$ (cf. \cite[Section 2, p. 653]{OSch}).
% \end{example}

% \subsection{The category of locally analytic representations}\label{s:CatLocAna} Let $K$ and $L$ be fields as in the previous section and $G$ a locally $K$-analytic group. 
% In this subsection, we take a look at $\Rep^{\ell a}_{L}(G)$, the category of locally analytic representations of $G$ on a certain class of $L$-vector spaces introduced by Schneider and Teitelbaum in \cite{ST2}. \\

% We start by recalling some definitions concerning topological $L$-vector spaces.

% \begin{definition}
%    \begin{enumerate}[label=\roman*)]
%       \item A topological $L$-vector space $V$ is \textit{locally convex} if it has a fundamental system of open $0$-neighbourhoods consisting of $\CO_K$-submodules (cf. \cite[Section 1, p. 444]{ST2}). 
%       \item A locally convex $L$-vector space is \textit{barelled} if every closed lattice is open (cf. \cite[Section 1, p. 444]{ST2}).
%       \item A locally convex $L$-vector space is of \textit{compact type} if it is the inductive limit of countably many $L$-Banach spaces $(V_n)_{n \in \BN}$ with transition maps being injective and compact (cf. \cite[Section 1, p. 445]{ST2}).
%       \item A locally convex $L$-vector space is called an \textit{L-Fréchet space} if it is metrizable and complete (cf. \cite[Section 8,p. 46]{S}). 
%    \end{enumerate}
% \end{definition}
% \begin{theorem}\cite[Theorem 1.1]{ST2}
%    Any space $V$ of compact type is Hausdorff, complete, bornological and reflexive. Its dual is a Fréchet space and satisfies $V'=\varprojlim_{n} V_n'$. 
   
% \end{theorem}
% Let $V$ be a Hausdorff barelled locally convex $L$-vector space. Then, $C^{an}(G,V)$ is the \textit{locally convex $L$-vector space of locally $L$-analytic functions on $G$ with values in $V$} (see \cite[Section 2, p. 447]{ST2} for a detailed description). 
% Further, $$D(G):=C^{an}(G,L)'$$ is the \textit{locally convex vector space of $L$-valued distributions on G} (cf. \cite[Section 2, Definition, p. 447]{ST2}). Additionally, with convolution as multiplication, it is an associative $L$-algebra  (cf. \cite[Proposition 2.3]{ST2}). 
% A prominent class of elements of $D(G)$ is that of \textit{Dirac distributions} $\delta_g$, for $g \in G$, defined by $$\delta_g(f)=f(g).$$
% %We will use analog definitions for $P$.  
% %Note that $D(P)$ is a subalgebra of $D(G)$.

% \begin{definition}\cite[Section 3, p. 451, Definition]{ST2} A \textit{locally analytic $G$-representation} $V$ (over $L$) is a Hausdorff barelled locally convex $L$-vector space $V$ equipped with a $G$-action by continuous linear endomorphisms such that, for each $v \in V$, the orbit map $\rho_v(g):=gv$ lies in $C^{an}(G,V)$. We denote the category of such representations by $\Rep^{\ell a}_{L}(G)$. 


% \end{definition}

% \begin{definition}\cite[Section 2.1, p. 103]{OSt}
%    A locally analytic $G$-representation $V$ is called \textit{strongly admissible} if $V$ is of compact type and $V'$ is a finitely generated $D(K)$-module for any compact open subgroup $K$ of $G$. 
% \end{definition} 
% As in the algebraic or smooth case,   we also have the induction functor. 
% \begin{definition}\cite[Section 2.2, p. 103]{OSt}
%     Let $H$ be a closed subgroup of $G$ and $(V,\rho)$ a locally analytic representation of $H$. The \textit{locally analytic induced representation $\Ind^G_H(V)$} is defined as 
% $$  \Ind^G_H(V)=\{f \in C^{an}(G,V) \mid  f(gh)=\rho(h^{-1})f(g) \, \forall h \in H, \forall g \in G \}.$$
% The group $G$ acts on $\Ind^G_H(V)$ by $(g.f)(x)=f(g^{-1}x)$. 
% \end{definition}

% \subsection{The functor $\CF^G_P$}\label{s:FGP} We remain in the setting of section \ref{s:catO}. Let $G=\bG(K)$ and $P=\bP(K)$ for some standard parabolic subgroup $\bP$ of $\bG$. 
% As mentioned in \cite[p. 443]{ST2}, $G$ and $P$ are locally $K$-analytic groups. 
% We will introduce the functor $\CF_P^G$ defined by Orlik und Strauch in \cite{OSt}, which links the category $\CO^{\fkp}_{\alg}$ with the category $\Rep^{\ell a}_{L}(G)$ from the last two subsections. \\

% Due to the results of \cite{OSt}, we have at some point to make the following assumption. 

% \begin{assumption}\label{assumptionp}\cite[Assumption 5.1]{OSt}
% If the root sytem $\Phi(\bG,\bT)$ has irreducible components of type $B$, $C$, or $F_4$, we assume $p>2$, and if $\Phi(\bG,\bT)$ has irreducible components of type $G_2$, we assume that $p>3$. 
% \end{assumption}

% Let $M \in \CO^{\fkp}_{\alg}$. Then, by the very definition of the category $\CO^{\fkp}_{\alg}$, there is a finite-dimensional representation $(W,\rho) \subset M$ of $\fkp_L$ which generates $M$ as $U(\fkg)$-module. We call such a tuple $(M,W)$ an \textit{$\CO^{\fkp}_{\alg}$-pair}. Hence, such a pair comes with a short exact sequence of $U(\fkg)$-modules
% \begin{equation}\label{sesCatO} 
%    0 \rightarrow \fkd \rightarrow U(\fkg) \otimes_{U(\fkp)} W \rightarrow M \rightarrow 0
% \end{equation}
% with $\fkd$ being the kernel of the natural map $ U(\fkg) \otimes_{U(\fkp)} W \rightarrow M$. 

% By the following lemma, we see why it is helpful to restrict to the algebraic part of the category $\CO^{\fkp}$. 
% \begin{lemma}\cite[Lemma 3.2]{OSt}\label{liftOP}
%    The representation $\rho$ lifts uniquely to an algebraic $\bP_L$-representation on $W$ (which we denote again by $\rho$). 
% \end{lemma}

% Thus, we have a locally analytic representation of $P$ on the dual space $W'$ denoted by $\rho'$.  
% Then, Orlik and Strauch considered in \cite[p. 108, (3.2.2)]{OSt} the pairing 
% \begin{align}
% \langle \, , \, \rangle_{C^{an}(G,L)}:D(G)\otimes_{D(P)} W \otimes_L \Ind^G_P(W') &\longrightarrow C^{an}(G,L) \label{pairing}\\
% (\delta \otimes w) \otimes f &\mapsto \Big[g \mapsto \Big(\delta \cdot_r\big(f(\cdot)(w)\big)\Big)(g)\Big]     \nonumber 
% \end{align}
% with $\Big(\delta \cdot_r\big(f(\cdot)(w)\big)\Big)(g)= \delta\big(x \mapsto f(gx)(w)\big)$. Besides $D(P)$, we can also consider $U(\fkg)$ as a subring of $D(G)$, as explained in \cite[Section 2, p. 449/450]{ST2}, and similiarly $U(\fkp) \subset D(P)$.
% Then, it turns out that the canonical map 
% $$ U(\fkg)\otimes_{U(\fkp)} W  \mapsto D(G) \otimes_{D(P) }W$$ 
% is injective (cf. \cite[p.108]{OSt}). Therefore,  with the notation of (\ref{sesCatO}), we consider 
% $$\Ind^G_P(W')^\fkd:=\{f \in \Ind^G_P(W') \mid \langle \delta  , f \rangle_{C^{an}(G,L)}=0 \text{ for all } \delta \in \fkd \}$$
% from \cite[p.108, (3.2.3)]{OSt}. It is a $G$-equivariant subspace of $\Ind^G_P(W')$. 

% \begin{proposition}\cite[Proposition 3.3. (i)]{OSt}
%    The representation $\Ind^G_P(W')^\fkd$ is a strongly admissible locally analytic $G$-representation. In particular, the underlying topological vector space is reflexive. 
% \end{proposition}

% Moreover, $D(\fkg,P)$ denotes the subring of $D(G)$ generated by $U(\fkg)$ and $D(P)$. The following lemma explains which $D(\fkg,P)$-module structure we will use on any object $M \in \CO^{\fkp}_{\alg}$ from now on. 

% \begin{lemma}\cite[Corollary 3.6]{OSt}
% There is on any object $M \in \CO^{\fkp}_{\alg}$ a unique $D(\fkg,P)$-module structure with the following properties: 
% \begin{enumerate}[label=\roman*)]
%    \item The action of $U(\fkp)$, as a subring of $U(\fkg)$, coincides with the action of $U(\fkp)$ as a subring of $D(P)$. 
%    \item The Dirac distributions $\delta_g \in D(P)$ act like group elements $g \in P$ (the latter action given by Lemma \ref{liftOP}).
% \end{enumerate}
% Moreover, any morphism $M_1 \rightarrow M_2$ in $\CO^{\fkp}_{\alg}$ is automatically a homomorphism of $D(\fkg,P)$-modules. 
% \end{lemma}

% \begin{proposition}\label{repiso}\cite[Proposition 3.7]{OSt}
%    There is an isomorphism of $D(G)$-modules 
%    $$D(G) \otimes_{D(\fkg,\,P )} M \cong \Big(\Ind^G_P(W')^\fkd\Big)'.$$ 
% \end{proposition}
% Based on this, Orlik and Strauch defined the following contravariant functor  
% \begin{align*}
%    \CF_P^G:\CO^{\fkp}_{\alg} &\longrightarrow \Rep^{\ell a}_{L}(G) \\
%                M        &\mapsto (D(G) \otimes_{D(\fkg,\,P )} M)'.
% \end{align*}
% in \cite[Section 4.1]{OSt}. 
% \begin{proposition}\cite[Proposition 4.2]{OSt}
%    The functor $\CF^G_P$ is exact. 
% \end{proposition}

% They also gave an alternative description of this functor \cite[Section 3.8]{OSt} which we would like to recall. \\

% Let $\bG_0$ be a split reductive group model of $\bG$ over $\CO_L$ (cf. section \ref{s:rootdatum}) with Borel pair $(\bT_0, \bB_0)$ and parabolic $\bP_0$ containing $\bB_0$ such that the base change to $K$ yields the pair $(\bT,\bB)$ and $\bP$ respectively.
% Let $\pi \in \CO_K$ be an uniformizer. For any positive number $m \in \BN$, we consider the reduction map 
% \begin{equation*}
% p_m:\bG_0(\CO_K) \rightarrow \bG_0(\CO_K/(\pi^m)). 
% \end{equation*}
% We set $G_0=\bG_0(\CO_K)$ and define $P^m:=p_m^{-1}(\bP_{0}(\CO_K/(\pi^m)) \subset G_0$. Let $\Phi_{\fku_\bP^-}=\{{\alpha_1}, \ldots, {\alpha_r}\}$ be the set of roots appearing in $\fku_\bP^-$ (under the adjoint action of $\bT$) and $y_{\alpha_1}, \ldots, y_{\alpha_r}$ be a basis of the $L$-vector space $\fku_\bP^-$. Then, for $\epsilon \in \lb \overline{K^*} \rb$, the norm ${\lb \enspace\, \rb}_\epsilon$ on $U(\fku_\bP^-)$ is given by 
% \begin{equation}\label{normUp}
%    \Bigg \lb \sum_{(i_1,\ldots,i_r) \in \BN_0^r} a_{i_1,\ldots,i_r}y_{\alpha_1}^{i_1}\cdots  y_{\alpha_r}^{i_r} \Bigg\rb_\epsilon= \sup_{(i_1,\ldots,i_r) \in \BN_0^r}\Big\lb i_1! \cdots i_r! \cdot a_{i_1,\ldots,i_r}\Big\rb \epsilon^{i_1 + \ldots +i_r}.
%    \end{equation}
% Completing $U(\fku_{\bP}^-)$ with respect to ${\lb \enspace\, \rb}_\epsilon$ yields the $L$-Banach space
%  \begin{align}\label{completeUEA}
%    U(\fku_{\bP}^-)_\epsilon:=\Bigg\{&\sum_{(i_1,\ldots,i_r) \in \BN_0^r} a_{i_1,\ldots,i_r}y_{\alpha_1}^{i_1}\cdots  y_{\alpha_r}^{i_r}\, \bigg\vert \, a_{i_1,\ldots,i_r} \in L, \nonumber \\
%    &\lb i_1! \cdots i_r! \cdot a_{i_1,\ldots,i_r}\rb \epsilon^{i_1 + \ldots +i_r} \rightarrow 0 \text{ for } i_1+\ldots + i_r \rightarrow 0
%    \Bigg\}.
%    \end{align}
%    Let $m\in \BN$ and $\epsilon_m:=\lb \pi \rb^m$. We will write $U(\fku_{\bP}^-)_m$ for $U(\fku_{\bP}^-)_{\frac{1}{\epsilon_m}}$. For $M \in \CO^{\fkp}_{\alg}$, we have seen in (\ref{sesCatO}) that there is a short exact exact sequence 
%    \begin{equation*}
%       0 \rightarrow \fkd \rightarrow U(\fkg) \otimes_{U(\fkp)} W \rightarrow M \rightarrow 0
%    \end{equation*}
%    of $U(\fkg)$-modules with a finite dimensional $\fkp$-representation $W$  which can be lifted. By the PBW-Theorem, we know that 
%    $$ U(\fkg) \otimes_{U(\fkp)} W \cong U(\fku_{\bP}^-) \otimes_L W.$$ 
%    For that reason, we consider $\fkd$ as a submodule of $U(\fku_{\bP}^-) \otimes_L W$ and denote by $\fkd_m$ its topological closure in $U(\fku_{\bP}^-)_m\otimes_L W$. The latter object also has a natural  $P^m$-action induced by the action 
%       $$p.(x \otimes w)=\Ad(p)(x) \otimes w $$ 
%    of $P_0$ on  $U(\fkg) \otimes_{U(\fkp)} W$ (cf. \cite[p.113/114]{OSt}). Finally, this leads to the following identification. 
% \begin{proposition}\label{alternative}\cite[Corollary 3.12]{OSt}
%    Let $M \in \CO^{\fkp}_{\alg}$. With the preceeding notation we have that 
%    $$\CF_P^G(M)= \Big(\varprojlim_m \Ind^{G_0}_{P^m}\big(U(\fku_{\bP}^-)_m\otimes_L W/\fkd_m\big)\Big)'. $$
% \end{proposition} 

% Inspired by Proposition \ref{repiso}, Orlik und Strauch extended the functor $\CF^G_P$ to a bi-functor on $\CO^{\fkp}_{\alg} \times \Rep^{\infty,\mathrm{adm}}_L(L_P)$ (cf. \cite[Section 4.4]{OSt}). 
% For this, let $V \in \Rep^{\infty,\, \mathrm{adm}}_L(L_P)$. By inflation, we consider $V$ as a representation of $P$. 
% Equipping $V$ with the finest locally convex  $L$-vector space topology, it is of compact type and carries the structure of a locally analytic $P$-representation (cf. \cite[p. 117]{OSt}). For an $\CO^{\fkp}_{\alg}$-pair $(M,W)$, Orlik und Strauch consider $W \otimes_L V$ as the projective (or inductive) tensor product which is complete and a locally analytic $P$-representation via the diagonal action (cf. \cite[p. 117]{OSt}). Then, they defined  
% \begin{align*}
%    \CF^G_P(M,W,V)&:= \Ind^G_P(W'\otimes V)^\fkd \\
%                  &:=\{f \in \Ind^G_P(W'\otimes V) \mid \langle \delta  , f \rangle_{C^{an}(G,V)}=0 \text{ for all } \delta \in \fkd \}
% \end{align*}
% with the pairing $\langle \, , \, \rangle_{C^{an}(G,V)}$ being defined completely analogous to (\ref{pairing}). However, the definition is independent of the chosen $\CO^{\fkp}_{\alg}$-pair $(M,W)$ (cf. \cite[Section 4.6]{OSt}). Therefore, we write $\CF^G_P(M,V)$ for any $\CF^G_P(M,W, V)$. We recap some properties of the bi-functor $\CF^G_P$. 

% \begin{proposition}\label{functorial}\cite[Proposition 4.7]{OSt}
% $\CF^G_P$ is a bi-functor 
% \begin{align*} \CO^{\fkp}_{\alg} \times \Rep^{\infty,\mathrm{adm}}_L(L_P) &\longrightarrow \Rep^{\ell a}_{L}(G)  \\
% (M,V) &\mapsto \CF^G_P(M,V)
% \end{align*}
% which is contravariant in $M$ and covariant in $V$.

% \end{proposition}
% In case $V$ is the trivial representation $\textbf{1}$, we will write $\CF^G_P(M)$.

% % \begin{proposition}\label{functor\cite[Proposition 4.8]{OSt} \begin{enumerate}[label=\roman*)]
% %       \item For all $M \in \CO^{\fkp}_{\alg}$, and for all smooth admissible $L_P$-representations $V$, the $G$-representation $\CF^G_P(M,V)$ is admissible. 
% %       \item If $V$ is of finite length, then $\CF^G_P(M,V)$ is strongly admissible for all $M \in \CO^{\fkp}_{\alg}$. 
% %    \end{enumerate}
% % \end{proposition}

% \begin{proposition}\label{exact}\cite[Proposition 4.9]{OSt} \begin{enumerate}[label=\roman*)]
%       \item The bi-functor $\CF^G_P$ is exact in both arguments. 
%       \item If $Q \supset P$ is a parabolic subgroup, $\fkq=\Lie(Q)$, and $M \in \CO^{\fkq}_{\alg}$, then 
%                   $$\CF^G_P(M,V)= \CF^G_Q\big(M,i^{L_Q}_{L_P(L_Q\cap U_P)}(V)\big)$$
%       where $i^{L_Q}_{L_P(L_Q\cap U_P)}(V)=i^Q_P(V)$ denotes the corresponding induced representation in the category of smooth representations. 
%    \end{enumerate}
% \end{proposition}

% \begin{theorem}\label{irreducible}\cite[Theorem 5.8]{OSt}
%    Assume that Assumption \ref{assumptionp} holds. Let $M \in \CO_{\alg}$ be simple and assume that $\fkp$ is maximal for $M$ (cf. Definition \ref{maximal}). 
%    Let $V$ be a smooth and irreducible $L_P$-representation. Then, 
%    $\CF^G_P(M,V)$ is topologically irreducible as a $G$-representation. 
% \end{theorem}


% We devote the last part of this subsection to an application of the functor $\CF_P^G$. Let $I \subset \Delta$ and $P_I$ be the associated parabolic subgroup of $G$. Then, we notice that 
% $$ \Ind^G_{P_I}(\textbf{1})=\CF_{P_I}^G(M_I(0))$$ 
% where $0$ is the weight sent to the zero vector under the identification $X^*(\bT)\cong \BZ^d$. More generally, for $\lambda \in X^*(\bT)^+$ (cf. (\ref{dominantweights})), we have 
% \begin{equation}\label{IGP}
%    I^G_{P_I}(\lambda):=\Ind^G_{P_I}(V_I(\lambda)')=\CF_{P_I}^G(M_I(\lambda))
% \end{equation}
% since the $\CO_{\alg}^{\fkp_I}$-pair $(V_I(\lambda), M_I(\lambda))$ has trivial kernel $\fkd$ (cf. (\ref{sesCatO})). For $I \subset J \subset \Delta$, the morphism $q_{I,J}$ (cf. (\ref{transition})) induces, by Proposition \ref{functorial} and \ref{exact}, a map 
% $$ p_{J,I}:  I^G_{P_J}(\lambda)=\CF_{P_J}^G(M_J(\lambda), \textbf{1}) \xrightarrow{\CF^G_{P_J}(q_{I,J}, \, \mathrm{incl.})} \CF_{P_J}^G(M_I(\lambda), i^{P_J}_{P_I})\cong\CF_{P_I}^G(M_I(\lambda))=I^G_{P_I}(\lambda)$$
% of locally analytic $G$-representations. Furthermore, the map $p_{J,I}$ is injective and has closed image (cf. \cite[p. 660]{OSch}). 
% \begin{definition}\cite[p. 661]{OSch}
% For $I\subset \Delta$, 
% $$ V^G_{P_I}(\lambda):= I^G_{P_I}(\lambda)\Big/ \sum_{J \supsetneq I} I^G_{P_J}(\lambda) $$
% is the \textit{twisted generalized Steinberg representation}. 
% \end{definition}
% It has the following resolution in $\Rep^{\ell a}_{L}(G)$. 

% \begin{theorem}\cite[Theorem 4.2]{OSch}
% Let $\lambda \in X^*(\bT)^+$ and $I \subset \Delta$. Then, the following complex is a resolution of $V^G_{P_I}(\lambda)$ by locally analytic $G$-representations, 
% \begin{align*}
%    0 \rightarrow I^G_G(\lambda) \rightarrow \bigoplus_{\substack{I \subset K \subset \Delta \\ \lb \Delta \backslash K \rb = 1}} I^G_{P_K}(\lambda) &\rightarrow \bigoplus_{\substack{I \subset K \subset \Delta \\ \lb \Delta \backslash K \rb = 2}} I^G_{P_K}(\lambda) \rightarrow \ldots \\
%    \ldots &\rightarrow \bigoplus_{\substack{I \subset K \subset \Delta \\ \lb K \backslash I \rb = 1}} I^G_{P_K}(\lambda) \rightarrow I^G_{P_I}(\lambda) \rightarrow V^G_{P_I}(\lambda)\rightarrow 0. 
% \end{align*}
% \end{theorem}

% Here, the differentials $d_{K',K}: I^G_{P_{K'}}(\lambda) \rightarrow I^G_{P_{K}}(\lambda)$ are defined as follows. We fix an ordering on $\Delta$. Let $K, K' \subset \Delta$ with $\lb K \rb= \lb K' \rb -1$ and $K'=\{\alpha_1 < \ldots < \alpha_r \}$. Then, 
% $$ d_{K',K}= \begin{cases} (-1)^i p_{K',K}  &K'=K \cup \{\alpha_i\} \\
%                                  0       & K \not\subset K' 
% \end{cases}.$$
% We like to stress a relative version which was shown in \cite[p. 663]{OSch} in the proof of the previous theorem. For this we follow the notion of \cite[p. 661]{OSch}. 

% \begin{definition}\label{twistedSteinberg}
% Let $\lambda \in X^*(\bT)^+$, $I \subset \Delta$ and $w \in W^I$. By Proposition \ref{parabolicweight}, we know that $w \cdot \lambda \in X^*(\bT)_I^+$. Then, we set  

% \begin{align}
%     I_{P_I}^G(w)&:=\Ind^G_{P_I}(V_I(w \cdot \lambda)')=\CF^G_{P_I}(M_I(w \cdot \lambda)) \label{twisted}, \\
%     V_{P_I}^G(w)&:=I^G_{P_I}(w)\Big/\sum_{\substack{ J \supsetneq I \\ w \in W^{J}}} I_{P_{J}}^G(w). \nonumber
% \end{align}
% \end{definition}

% \begin{corollary}\label{relativeresolution} Let $\lambda \in X^*(\bT)^+$, $I \subset \Delta$ and $w \in W^I$. Then, the following complex is acyclic 
%    \begin{align*}
%       0 \rightarrow I^G_{P_{I(w)}}(w) \rightarrow \ldots \rightarrow  \bigoplus_{\substack{I \subset K \subset I(w) \\ \lb K \backslash I \rb = 1}} I^G_{P_K}(w) \rightarrow I^G_{P_{I}}(w) \rightarrow V^G_{P_{I}}(w)\rightarrow 0. 
%    \end{align*}
% \end{corollary}

% In \cite[Theorem 4.6]{OSch}, it was shown that the Jordan-Hölder factors of $V^G_B(\lambda)$ are of the form $\CF^G_{P_I}\Big(L(w\cdot \lambda), v^{P_I}_{P_J}\Big)$ for suitable $I,J \subset \Delta$ and $w \in W$. This is related to Theorem \ref{irreducible}. We will use Corollary \ref{relativeresolution} to get a similar statement for $V^G_B(w)$ which partially generalizes \cite[Theorem 4.6]{OSch}. \\

% For $w,v \in W$, we denote by $m(w,v) \in \BZ_{\geq 0}$
% the \textit{multiplicity} of $L(v \cdot 0)$ in $M(w \cdot 0)$. It is well known that $m(w,v)>0$ if and only if $ w \leq v$ with respect to the Bruhat order $\leq$ on $W$. 
% Moreover, the multiplicities can be computed using Kazhdan-Lusztig polynomials (cf. \cite{BB} or \cite{BK}) which is in general only possible in a timely manner with the help of a computer. 

% \begin{theorem}\label{multiplicities}
%    Assume that Assumption \ref{assumptionp} holds. Fix $w,v \in W$ and let $I_0:=I(w)$ and $I:=I(v)$, respectively, be as above. For a subset $J \subset \Delta$ with $J \subset I$, 
%     let $v_{P_J}^{P_I}$ be the generalized smooth Steinberg representation of $L_{P_I}$. Then, the multiplicity of the irreducible $G$-representation 
%     $\CF^G_{P_I}(L(v\cdot \lambda),v_{P_J}^{P_I})$ in $V^G_B(w)$ is 
%     $$ \sum_{\substack{w' \in W  \\ \supp(w')=J \cap I_0}} (-1)^{\ell(w')+\lb J \cap I_0 \rb} m(w'w,v)$$ 
%     and in this way we obtain all the Jordan-Hölder factors of $V^G_B(w)$. 
% \end{theorem} 

% \begin{proof} 
%     We only have to slightly modify the proof of \cite[Theorem 4.6]{OSch}.  %check language 
%     From the resolution for $V^G_B(w)$ by Corollary \ref{relativeresolution}, we obtain the multiplicity 
%         $$\big[V_B^G(w): \CF^G_{P_I}(L(v\cdot \lambda),v_{P_J}^{P_I})\big]=\sum_{K \subset I_0}(-1)^{\lb K \rb}\big[I^G_{P_K}(w):\CF^G_{P_I}(L(v\cdot \lambda),v_{P_J}^{P_I})\big]$$
%    of the simple object $\CF^G_{P_I}(L(v\cdot \lambda),v_{P_J}^{P_I})$ in $V^G_B(w)$. 
%     By the arguments mentioned in loc. cit, it follows that $\big[I^G_{P_K}(w):\CF^G_{P_I}(L(v\cdot \lambda),v_{P_J}^{P_I})\big] \neq 0$ if only if  $K \subset J \cap I_0$. In that case we have 
%     $$ \big[I^G_{P_K}(w):\CF^G_{P_I}(L(v\cdot \lambda),v_{P_J}^{P_I})\big] = \big[M_K(w \cdot \lambda): L(v\cdot \lambda)\big].$$
%     From the character formula 
%         $$\ch(M_K(w\cdot \lambda))= \sum_{w' \in W_K} (-1)^{\ell(w')} \ch(M(w'w\cdot \lambda)),$$ 
%     (cf. \cite[Section 9.6, p. 189, Proposition]{H2}), we obtain 
%     \begin{align*} \big[V_B^G(w): \CF^G_{P_I}(L(v\cdot \lambda),v_{P_J}^{P_I})\big] &= \sum_{K \subset J \cap I_0}(-1)^{\lb K \rb} \sum_{w' \in W_K} (-1)^{\ell(w')} \big[M(w'w\cdot \lambda):L(v \cdot \lambda)\big] \\
%      &= \sum_{w' \in W} (-1)^{\ell(w')}\big[M(w'w\cdot \lambda):L(v \cdot \lambda)\big]  \sum_{\substack{K \subset J\cap I_0 \\ \supp(w') \subset K}} (-1)^{\lb K \rb}.
%     \end{align*}
%     Finally, we have 
%     $$ \sum_{\supp(w') \subset K \subset J\cap I_0} (-1)^{\lb K \rb}=(-1)^{\supp(w')}(1-1)^{\lb (J \cap I_0) \backslash \supp(w')\rb}$$
%     which is non-zero if and only if $\supp(w')=J \cap I_0$. Hence, the formula follows. \\

%     The natural morphism $V^G_B(\lambda)\rightarrow V^G_B(w)$ is surjective for all $w \in W$ as it is induced by an injective morphism $M(w \cdot \lambda) \rightarrow M(\lambda)$ (cf. Lemma \ref{surjection}).  
%     Therefore, \cite[Theorem 4.6]{OSch} implies that we obtain all Jordan-Hölder factors of $V^G_B(w)$ in this manner.  
% \end{proof}


%!TEX root = draft.tex

\subsection{Period domains}\label{s:perdom}
\setenumerate[0]{leftmargin=5mm,itemsep=\the\smallskipamount}
%\todo[inline]{\textbf{TODO}: Recall the beginning of 1.3 in \cite{OS}}

In this section we give a brief introduction to our central object of study,  $p$-adic period domains (cf. \cite{O1} and \cite{CDHN}). 
For a more general setting and detailed presentation, we refer the reader to \cite{DOR}. \\

Let $F$ be an algebraically closed field of characteristic $p$ and $K_0 = \mathrm{Quot}(W(F))$, the quotient field of the ring of Witt vectors of $F$.
Let $K=\BQ_p$ with algebraic closure $\ov{K}$ and absolute Galois group $\Gamma_K=\Gal(\overline{K}/K)$. We denote by $C$ the $p$-adic completion of 
$\ov{K}$. Moreover, let $\sigma \in \Aut(K_0/K)$ be the Frobenius homomorphism
and $\bG$ a quasi-split connected reductive group over $K$.  

\subsubsection{Filtered isocrystals}
An \textit{isocrystal} over $F$ is a pair $(V, \Phi)$ with a finite-dimensional  $K_0$-vector space $V$ and a $\sigma$-linear bijective endomorphism $\Phi$ of $V$. 
Then, an \textit{isocrystal with $\bG$-structure} (also referred to as a $\bG$-isocrystal) is an exact faithful tensor functor $$ \Rep_K(\bG) \longrightarrow \mathrm{Isoc}(F)$$
from the category of finite-dimensional algebraic $K$-representations to the category of isocrystals over $F$. 
% In view of \cite[Remark  3.4]{RR} there exists a very concrete description of $\bG$-isocrystals which we will sketch.
 Every $\bG$-isocrystal is 
induced by an element $b \in \bG(K_0)$. Namely, to a finite-dimensional algebraic $K$-representation $(V, \rho)$ of $\bG$, we associate 
$$ N_b(V):=\big(V \otimes_K K_0, \rho(b)(\id_V \otimes\sigma )\big)$$
which defines an isocrystal over $F$  (cf. \cite[Remark  3.4]{RR}). The morphisms are mapped under $N_b$ as expected. Thus, $N_b$ is a $\bG$-isocrystal and 
$b,b' \in \bG(K_0)$ yield the same $\bG$-isocrystal if and only if there exists a $g \in \bG(K_0)$ 
such that $b'=gb\sigma(g)^{-1}$, i.e. if they are $\sigma$-conjugated. The set of $\sigma$-equivalence classes $[b]$ in $\bG(K_0)$ 
is denoted by $B(\bG)$ and was introduced by Kottwitz \cite{Ko1, Ko2}. In \cite[Section 3]{CDHN} the authors give several interpretations of this set. 
Additionally, the $\bG$-isocrystal $N_b$ comes along with its automorphism group $\bJ_b$. It is an algebraic group over $K$ with 
$$\bJ_b(A)=\{g \in \bG(K_0 \otimes_K A) \mid gb\sigma(g)^{-1}=b \}$$ 
for every $K$-algebra A. It depends only on $[b]$ in view of \cite[Section 2.1, p. 280]{RV}. 
As $\bG$ is quasi-split, we know by \cite[Section 6]{Ko1} that $\bJ_b$ is an inner form of a Levi subgroup of $\bG$; hence $\bJ_b$ is reductive. \\

Let $L$ be a field extension of $K_0$. A \textit{filtered isocrystal} $(V, \Phi, \CF^\bullet)$ over $L$ is an isocrystal $(V, \Phi)$ 
over $F$ with a $\BQ$-filtration $\CF^{\bullet}$ (decreasing, exhaustive and separated) on $V_L$. 
The filtered isocrystal over $L$ form a $K$-linear quasi-abelian tensor category $\mathrm{FilIsoc}^L_{F}(\Phi)$ (cf. \cite[Section VIII, p. 192]{DOR}).
Then, we say that a filtered isocrystal $(V,\Phi, \CF^\bullet)$ over $L$ is \textit{weakly admissible} if the inequality 
$$ \sum_i i\dim gr^i_{\CF^\bullet(V)}(N' \otimes_{K_0}L) \leq \ord_p \,\det(\Phi|N')$$
holds for every subisocrystal $N'$ of $(V,\Phi)$ and with equality in case $N'=(V, \Phi)$.\\
% By the tensor product theorem due to Faltings \cite{F} and Totaro \cite{T}, weak admissibility is stable under tensor products.
% Therefore, it is enough to check the weak admissibility for a single faithful representation. 

Any $1$-PS $\lambda:\BG_m \longrightarrow \bG_L$ defined over $L$ induces a $\mathbb{Z}$-graded $L$-vector space $$V_L=\bigoplus_{i \in \BZ} V_i^{\lambda}$$
for a finite-dimensional algebraic $K$-representation $(V,\rho)$, 
where the grading comes from the weight spaces $V_i^\lambda=\{v \in V_L \mid \rho(\lambda(s))v=s^iv \}$.
Thus, we naturally have a decreasing exhaustive separated $\BZ$-filtration $\CF_\lambda^\bullet(V)$ on $V_L$ given by
$$\CF^i_\lambda(V)=\bigoplus_{j \geq i} V_j^\lambda.$$ Therefore, a tuple $(b, \lambda ) \in \bG(K_0) \times X_*(\bG_L)$ defines a tensor functor 
\begin{align*}
    \Rep_K(\bG) &\longrightarrow \mathrm{FilIso}^L_{F}(\Phi)\\
    (V, \rho) &\mapsto (N_b(V),\CF_\lambda^{\bullet}(V)).
\end{align*}
Such a pair $(b, \lambda)$ is \textit{weakly admissible} if $(N_b(V),\CF_\lambda^{\bullet}(V))$ is weakly admissible for all $(V, \rho) \in \Rep_K(\bG)$.

% \subsubsection{Slope homomorphism and Newton map} \label{s:SlopeAndNewton}
% A technical issue, which will be important in the upcoming sections, is the \textit{slope homomorphism} $\nu_b$. Let $\BD$ be the algebraic pro-torus over $K$ with character group $\BQ$. 
% Kottwitz showed in \cite[Section 4]{Ko1} that there exists for $b \in \bG(K_0)$ a unique morphism $$\nu_b: \BD_{K_0} \longrightarrow \bG_{K_0}$$ 
%  which, by the Tannakian formalism, induces the tensor functor 
% \begin{align*}
% \Rep_K(\bG) &\longrightarrow \mathrm{Grad}(\mathrm{Vec}_{K_0}, \BQ) \\
% (V, \rho) &\mapsto \bigoplus_{i \in \BQ} V_i 
% \end{align*}
% from $\Rep_K(\bG)$ to the category of $\BQ$-graded ${K_0}$-vector spaces, where the grading comes from the 
% slope decomposition of $N_b$. Further, it has the properties 
% \begin{align}
%     \nu_{gb\sigma(g)^{-1}}&=g\nu_bg^{-1} \text{ for all }  g \in \bG( K_0), \label{propv_b1} \\
%     \nu_{\sigma(b)}&=\sigma(\nu_b) \label{propv_b2}
% \end{align}
% (cf. \cite[Section 4.4]{Ko1}). By employing both, one obtains $\nu_b=b\sigma(\nu_b)b^{-1}$ and thus, we have a well-defined map 
% \begin{align*}
%     B(\bG) &\longrightarrow [\Hom_{K_0}(\BD_{K_0}, \bG_{K_0})/\Int(\bG(K_0))]^{\sigma=1}, \\
%     [b] &\mapsto [v_b]
% \end{align*}
% the so called \textit{Newton map}. We will follow \cite[Section 3.2.1]{CDHN} and denote the codomain by $\sN(\bG)$, 
% which is also referred to as set of \textit{Newton vectors}. \\

% An element $b \in \bG(K_0)$ is \textit{basic} if $\nu_b$ factors through the center of $\bG$ which is equivalent to $\bJ_b$ being 
% an inner form of $\bG$ by \cite[Section 5.1/5.2]{Ko1}. Moreover, by \cite[Section 5.1]{Ko1}, $\nu_b$ is already defined over $K$. We say that $[b] \in B(\bG)$ is \textit{basic}  if it contains a basic element. 


\subsubsection{Parameterization of weakly admissible filtrations on isocrystals}

In the following, we fix together with $\bG$ an element $b \in \bG(K_0)$ and a conjugacy class $\{\mu\} \subset X_*(\bG)$ over $\ov K$. \\

The conjugacy class $\{\mu\}$ defines the Shimura field $E:=E(\bG, \{\mu\}) \subset \ov K$. It is the fixed subfield of $\ov K$
under the stabilizer $\Gamma_\mu$ of $\{\mu \}$ in $\Gamma_K $ and is a finite extension of $K$. 
As $\bG$ is quasi-split, $\{\mu \}$ contains an element $\mu$ defined over $E$ by \cite[Lemma 1.1.3]{Ko3}. Therefore, the associated flag variety $\sF:=\sF(\bG, \{\mu\})$, defined over $E$, can be identified as$$
\sF=\bG_E/\bP(\mu).$$ 
% Here $\bP(\mu)$ stands for the associated parabolic subgroup of $\bG_E$ with $\ov K$-valued points
% $$\bP(\mu)(\ov K)=\big\{g \in \bG(\ov K)  \mid \lim_{t \rightarrow 0} \mu(t)g\mu(t)^{-1} \text{ exists in } \bG(\ov K) \big\}.$$
Let us point out that the $\ov K$-valued points of $\sF$ are given by $$\{\mu\}/\sim$$ 
where $\sim$ is the par-equivalence relation explained in \cite[Section 4.1.2]{CDHN}, 
which identifies the elements of $\{\mu\}$ defining the same filtration on $\Rep_K(\bG)$. Hence, for a field extension $L$ of $E$, 
a point $x \in \sF(L)$ gives rise to a cocharacter $\mu_x \in \{\mu \}$ defined over $L$ up to par-equivalence (cf. \cite[Remark 6.1.6]{DOR} for more details).  \\

Setting $\br E:=E K_0$, we write $\br \sF$ for the adic analytification of $\sF_{\br E}$. According to \cite[Proposition 1.36\,i)]{RZ}, the set $\sF^{\wa}_b:=\sF(\bG, \{\mu \}, b)^{\wa}$ of weakly admissible filtrations with respect to $b$ in $\sF$, i.e. 
$$\sF^{\wa}_b(L)=\{x \in  \sF(L) \mid (b,\mu_x) \text{ weakly admissible} \}$$
for any field extension $L$ of $\br E$, has the structure of a partially proper open subset of $\br \sF$.
The space $\sF^{\wa}_b$ is the \textit{period domain} attached to the triple $(\bG, \{\mu\}, b)$. 
First, we note that $\sF^{\wa}_b$ only depends on $[b] \in B(\bG)$. Secondly, the natural 
action of $\bJ_b(K) \subset \bG(K_0)$ on $\br \sF$ restricts to an action on $\sF_b^{\wa}$ (cf. \cite[1.35 and 1.36\,i)]{RZ}). In the case that $b$ is $s$-decent\footnote{There exists a positive integer $s$ such that $sv_b$ factors through the quotient $\BG_{m,K_0}$  of $\BD_{K_0}$ and the equality $(b\sigma)^s=(s\nu_b)(p)\sigma^s$ holds in $\bG(K_0) \rtimes  \sigma^{\BZ}$. In our case, there exists always such an $s$ (cf. \cite[Remark 9.1.34]{DOR}).}, we can regard  $\sF^{\wa}_b$ as a partially proper open subset defined over $E_s:=E\BQ_{p^s}$ (cf. \cite[Proposition 1.36\,ii)]{RZ}). 
% Let $s$ be a positive integer. An element $b \in \bG(K_0)$ is \textit{s-decent} if $sv_b$ factors through
% the quotient $\BG_{m,K_0}$  of $\BD_{K_0}$ and the equality $$(b\sigma)^s=(s\nu_b)(p)\sigma^s$$
% holds in $\bG(K_0) \rtimes  \sigma^{\BZ}$. The equivalence class $[b] \in B(\bG)$ is \textsl{decent} if it contains
% an $s$-decent element for some positive integer $s$ (cf. \cite[Definition 1.8]{RZ}). By our assumptions on $\bG$ and the algebraically closedness of $F$, every class $[b] \in B(\bG)$ is decent (cf. \cite[Remark 9.1.34]{DOR}). In particular, there exists a positive integer $s$ and an $s$-decent element $b \in [b]$ such that $b  \in \bG(\BQ_{p^s})$ and $\nu_b$ is defined over $\BQ_{p^s}$ (cf. \cite[Corollary 1.9]{RZ}). 

\subsubsection{Existence of weakly admissible filtrations}\label{ss:existence}  

Let $\BD$ be the algebraic pro-torus over $K$ with character group $\BQ$. 
Kottwitz showed in \cite[Section 4]{Ko1} that there exists for $b \in \bG(K_0)$ a unique morphism $$\nu_b: \BD_{K_0} \longrightarrow \bG_{K_0}$$ which, by the Tannakian formalism, induces the tensor functor 
\begin{align*}
\Rep_K(\bG) &\longrightarrow \mathrm{Grad}(\mathrm{Vec}_{K_0}, \BQ) \\
(V, \rho) &\mapsto \bigoplus_{i \in \BQ} V_i 
\end{align*}
from $\Rep_K(\bG)$ to the category of $\BQ$-graded ${K_0}$-vector spaces, where the grading comes from the slope decomposition of $N_b$. That is the \textit{slope homomorphism}. It has the property that $\nu_b=b\sigma(\nu_b)b^{-1}$ 
and thus, we have a well-defined map 
\begin{align*}
    B(\bG) &\longrightarrow [\Hom_{K_0}(\BD_{K_0}, \bG_{K_0})/\Int(\bG(K_0))]^{\sigma=1}, \\
    [b] &\mapsto [v_b]
\end{align*}
the so called \textit{Newton map}. Let $\bB$ be a Borel subgroup in $\bG$ and $\bT$ a maximal torus contained in 
% An element $b \in \bG(K_0)$ is \textit{basic} if $\nu_b$ factors through the center of $\bG$ which is equivalent to $\bJ_b$ being 
% an inner form of $\bG$ by \cite[Section 5.1/5.2]{Ko1}. Moreover, by \cite[Section 5.1]{Ko1}, $\nu_b$ is already defined over $K$. We say that $[b] \in B(\bG)$ is \textit{basic}  if it contains a basic element. 
% Let $s$ be a positive integer. An element $b \in \bG(K_0)$ is \textit{s-decent} if $sv_b$ factors through 
% the quotient $\BG_{m,K_0}$  of $\BD_{K_0}$ and the equality $$(b\sigma)^s=(s\nu_b)(p)\sigma^s$$
% holds in $\bG(K_0) \rtimes  \sigma^{\BZ}$. The equivalence class $[b] \in B(\bG)$ is \textsl{decent} if it contains
% an $s$-decent element for some positive integer $s$ (cf. \cite[Definition 1.8]{RZ}). By our assumptions on $\bG$ and the algebraically closedness of $F$, every class $[b] \in B(\bG)$ is decent (cf. \cite[Remark 9.1.34]{DOR}). In particular, there exists a positive integer $s$ and an $s$-decent element $b \in [b]$ such that $b  \in \bG(\BQ_{p^s})$ and $\nu_b$ is defined over $\BQ_{p^s}$ (cf. \cite[Corollary 1.9]{RZ}). 
$\bB$. Further, let $X_*(\bT)_{\BQ}^+$ be the set of dominant rational cocharacters of $\bT$ with respect to $\bB$. The chosen $\bB$ induces a partial order $\leq$ on $X_*(\bT)_\BQ$ where $ \lambda' \leq \lambda$ if and only if 
$\lambda-\lambda'=\sum_{\alpha \in \Delta} n_\alpha \alpha^{\vee} $ with all $n_\alpha \in \BQ_{\geq 0}$. According to \cite[Section 2.1/2.2]{RV} (cf. \cite[Remark 3.3]{CDHN}), there is a unique $\mu \in \{\mu\}$ and a  unique representative $\nu_{[b]} \in [\nu_b]$ for $[b] \in B(\bG)$ lying in $X_*(\bT)_{\BQ}^+$.  \\

Rapoport and Viehmann associated in \cite[Definition 2.3]{RV}  the \textit{set of acceptable elements} to a conjugacy class $\{\mu\}$ by setting
$$A(\bG, \{\mu\}):=\{ [b] \in B(\bG) \mid v_{[b]} \leq \overline{\mu} \} $$ 
where $\overline{\mu} := \frac{1}{[\Gamma_K:\Gamma_{\mu}]} \sum_{\tau \in \Gamma_K/\Gamma_\mu} \tau(\mu) \in X_*(\bT)_\BQ^+$. 
We remark that this set is non-empty and finite by \cite[Lemma 2.5]{RV}. Finally, one obtains the following result. 

\begin{theorem}{\cite[Theorem 9.5.10]{DOR}} \label{necessary} The period domain $\sF(G,\{\mu \}, b)^{\wa}$ is non-empty if and only if $[b] \in A(\bG, \{\mu \})$. 
\end{theorem}

% \subsubsection{Geometric invariant theory}\label{GIT} We recall some notation from \cite[Section 5.2.3]{CDHN}. Let $[b] \in B(\bG)$ be decent 
% and fix an $\inva$ on $\bG$ (cf. section \ref{s:rootdatum}). 
% Then, there is an ample line bundle $\sL:=\sL_{\bG,\{\mu\}, [b],\, (\, , \,)}$ on $\sF_{\br E}$ together with the slope function $\mu^{\sL}(\, , \,)$ (cf. \cite[Definition 2.2]{M}) 
% which characterizes weakly admissible points. 

% \begin{theorem}[Totaro, {\cite[Theorem 3]{T}, \cite[Theorem 9.7.3]{DOR}}] Let $L/\br E$ be a field extension and $x \in \sF(L)$. Then, 
%     $x \in \sF^{\wa}_b(L)$ if and only if $\mu^{\sL}(x, \lambda)\geq 0$ for all $\lambda \in X_*(J_b)^{\Gamma_{\BQ_p}}$. 
% \end{theorem} 

% \subsubsection{Synopsis} The definition of a period domain involves a lot of data which we summarize at this point. 
% Therefore, we recap the notion of a local Shtuka datum from {\cite[Definiton 4.4]{CDHN}} which we adjust to the quasi-split case. 

% \begin{definition}\cite[Definiton 4.4]{CDHN} A \textit{local Shtuka datum} over $K$ is a triple $(\bG, \{\mu\}, [b])$ 
% consisting of a quasi-split connected reductive  group $\bG$ defined over $K$, a geometric
% conjugacy class $\{\mu\}$ of cocharacters of $\bG$ defined over $\ov K$ and a $\sigma$-conjugacy class $[b] \in A(\bG, \{\mu\}) \subset B(\bG)$. 
% \end{definition}

% Associated to a local Shtuka datum $(\bG, \{\mu\}, [b])$, we have seen 
% \begin{enumerate}
% \item the reductive  group $\bJ:=\bJ_b$ over $K$ for $b \in [b]$, 
% \item the Newton vector $[v_b] \in \sN(\bG)$,
% \item the Shimura field $E:=E(\bG,\{\mu\})$, 
% \item the flag variety $\sF:=\sF(\bG,\{\mu\})=\bG_E/\bP(\mu)$ over $E$, 
% \item the period domain $\sF^{\wa}_b:=\sF(\bG,\{\mu\}, b)^{\wa}$ over $\br E$ with a $\bJ(K$)-action. \\
% \end{enumerate}

% In view of \cite[Remark 4.5]{CDHN}, we can assume that $K=\BQ_p$.

% Set $Y:=\sF^\mathrm{rig} \backslash \sF^{\mathrm{wa}}$. Consider the topological exact squence of locally convex $E$-vector spaces with continuous $G$-action 
% \begin{align*}
% 0 \longrightarrow H^0_Y(\sF^\mathrm{rig}, \CE) \longrightarrow &H^0(\sF^\mathrm{rig}, \CE) \longrightarrow H^0(\sF^{\mathrm{wa}},\CE) \longrightarrow \\ &H^1_Y(\sF^\mathrm{rig}, \CE) \longrightarrow H^1(\sF^\mathrm{rig}, \CE) \longrightarrow H^1(\sF^{\mathrm{wa}},\CE) \longrightarrow  \ldots
% \end{align*}
% We see that by computing $H^*_Y(\sF^\mathrm{rig}, \CE)$ we can tackle $H^*(\sF^{\mathrm{wa}},\CE)$. 
% %\todo[inline]{\textbf{TODO}: Define everything for the fundamental complex}
% On $Y^{\mathrm{ad}}$ we have the following complex of sheaves (in the notation of \cite{CDHN}, chapter 6)

% \begin{equation}\label{fundamentalcomplex}
%     0 \longrightarrow \BZ \longrightarrow \bigoplus\limits_{\substack{I \subset \Delta \\ \lb \Delta \backslash I \rb = 1 }} \BZ_I \longrightarrow \bigoplus\limits_{\substack{I \subset \Delta \\ \lb \Delta \backslash I \rb = 2 }} \BZ_I \longrightarrow \ldots \longrightarrow \bigoplus\limits_{\substack{I \subset \Delta \\ \lb \Delta \backslash I \rb = \lb \Delta \rb -1 }} \BZ_I \longrightarrow \BZ_\emptyset
% \end{equation}
% which is by  \cite[Theorem 6.13]{CDHN} acyclic. Denoting by $i:Y^{\mathrm{ad}} \hookrightarrow \sF^{\mathrm{ad}}$ the closed embedding we have by \cite[Proposition 2.3]{SGA2}  
% \begin{equation*}
%     \Ext^*(i_*(\BZ_{Y^{\mathrm{ad}}}),\CE) \cong H^*_{Y^{\mathrm{ad}}}(\sF^{\mathrm{ad}},\CE). 
% \end{equation*}
% Applying $\Ext^*(i_*(\BZ_{Y^{\mathrm{ad}}}),-)$ to the complex \ref{fundamentalcomplex} we get the spectral sequence 
% \begin{equation}\label{spectral}
% E_1^{-p,q}=\Ext^q(\bigoplus\limits_{\substack{I \subset \Delta \\ \lb \Delta \backslash I \rb = p+1 }} \BZ_I, \CE) \Rightarrow \Ext^{-p+q}(i_*(\BZ_{Y^{\mathrm{ad}}}),\CE)=H^{-p+q}_{Y^{\mathrm{ad}}}(\sF^{\mathrm{ad}},\CE). 
% \end{equation}

% For the $E_1$-terms we have the following identification. 

% \begin{proposition} For all $I \subset \Delta$, there is an isomorphism 
%     \begin{equation*}
%         \Ext^*( \BZ_I, \CE) \cong \varprojlim_{n \in \BN} \Ind_{P_I^n}^{G_0} H^*_{Y_I(\epsilon_n)}(\sF,\CE).
%     \end{equation*}
% \end{proposition}
% \begin{proof}
%     \todo[inline]{\textbf{TODO}: The plan is to adapt the proof of Proposition 2.2.1 in \cite{OS}, but there are steps which I don't understand.}
% \end{proof}

% Hence analyzing \ref{spectral} yields for the rows $E_1^{\bullet, q}$ the complexes of $E$-Fréchet spaces
% \begin{align*}
%     E_1^{\bullet, q}: \varprojlim_{n \in \BN} \Ind_{P_\emptyset^n}^{G_0} H^q_{Y_I(\epsilon_n)}(\sF,\CE) \longrightarrow \varprojlim_{n \in \BN} \bigoplus\limits_{\substack{I \subset \Delta \\ \lb  I \rb = 1}} \Ind_{P_I^n}^{G_0} H^q_{Y_I(\epsilon_n)}(\sF,\CE) \longrightarrow &\varprojlim_{n \in \BN} \bigoplus\limits_{\substack{I \subset \Delta \\ \lb  I \rb = 2}} \Ind_{P_I^n}^{G_0} H^q_{Y_I(\epsilon_n)}(\sF,\CE) \\ \longrightarrow \ldots \longrightarrow \varprojlim_{n \in \BN} \bigoplus\limits_{\substack{I \subset \Delta \\ \lb \Delta \backslash I \rb = 2}} \Ind_{P_I^n}^{G_0} H^q_{Y_I(\epsilon_n)}(\sF,\CE) \longrightarrow &\varprojlim_{n \in \BN} \bigoplus\limits_{\substack{I \subset \Delta \\ \lb \Delta \backslash I \rb = 1}} \Ind_{P_I^n}^{G_0} H^q_{Y_I(\epsilon_n)}(\sF,\CE).
% \end{align*}
% Passing to the dual we get by Proposition \ref{Bi-Functor} the complex
% \begin{align*}
%     E_1^{\bullet, q}:\CF_{P_\emptyset}^G(H^q_{Y_\emptyset}(\sF,\CE)) \longleftarrow \bigoplus\limits_{\substack{I \subset \Delta \\ \lb  I \rb = 1}} \CF_{P_I}^G(H^q_{Y_I}(\sF,\CE)) \longleftarrow &\bigoplus\limits_{\substack{I \subset \Delta \\ \lb  I \rb = 2}} \CF_{P_I}^G(H^q_{Y_I}(\sF,\CE)) \\ \longleftarrow \ldots \longleftarrow \bigoplus\limits_{\substack{I \subset \Delta \\ \lb \Delta \backslash I \rb = 2}} \CF_{P_I}^G(H^q_{Y_I}(\sF,\CE)) \longleftarrow &\bigoplus\limits_{\substack{I \subset \Delta \\ \lb \Delta \backslash I \rb = 1}} \CF_{P_I}^G(H^q_{Y_I}(\sF,\CE)).
% \end{align*}

% From now on fix a conjugacy class $\{\mu\} \subset X_*(\bG)$ with Shimura field $E$ such that $\bP(\mu)$ is a Borel $\bB\subset \bG_E$.
% Then $\sF:=\bG_E/\bB$ is the complete flag variety defined over $E$ and we denote by $\sF^{wa}$ the period domain associated to the local Shtuka datum $(\bG,[1],\{\mu\})$ 
% (see e.g. \cite{CDHN}[Definition 4.4.]), hence associated to a trival isocrystal. \newline
% \\For a weight $\lambda \in X^{*}(\bT)$ let $\CL_\lambda$ be the sheaf such that for $U \subset \sF$ open 
% $$\CL_\lambda(U)=\{f \in \CO_{\bG}(\pi^{-1}(U)) \mid f(gb)=-\lambda(b)f(g) \text{ for all } g \in \bG(\bar{E}), b \in \bB(\bar{E})\}$$
% where $\pi:\bG\rightarrow \sF$ is the natural projection. For example $\CL_{2\rho}=\omega_{\sF}$ with $\rho$ the half 
% sum of all positive roots with respect to $\bB$. For $\lambda$ being dominant and $\CE:=\CL_\lambda \otimes \omega_{\sF}$ we have

% \begin{lemma}For $w \in W$ 
% $$H^i_{C(w)}(\sF,\CE)\cong\begin{cases} M(w\cdot\lambda) &i=n-l(w),\\ 0 &\text{else.} \end{cases}$$
% \end{lemma}

% \begin{proof}

% \end{proof}

% Recall that $\Delta$ is a set of simple roots with respect to $\bB$. For $I \subset \Delta$ we associate the  
% parabolic subgroup $\bP_I\supset \bB$ with Weyl group $W_I \subset W$, the subgroup generated by the $s_\alpha$ with $\alpha \in I$.
%  We denote by $W^I$ the set of minimal length right coset representatives of $W_I\backslash W$. 
%  Then following \cite{MR} we define for $w \in W^I$ $$C_I(w):=\bP_Iw\bB/\bB=\bigcup_{v \in W_I}C(vw)$$
% and similiar to before we have
% \begin{lemma}\label{C_I(w)} For $w \in W^I$ 
%     $$H^i_{C_I(w)}(\sF,\CE)\cong\begin{cases} M_I(w\cdot\lambda) &i=n-l(w),\\ 0 &\text{else.} \end{cases}$$
% \end{lemma}
% \begin{proof}
% \end{proof}

% For later use we follow  \cite{OSch} and set, with respect to lemma \ref{C_I(w)}, $$I^G_{P_I}(w):=\CF_{P_I}^G(H^i_{C_I(w)}(\sF,\CE)).$$

% Lets come back to \ref{spectral} applied to our setting, where we start with some observations. 

% \begin{lemma} For $I \subset \Delta$ 
%     $$ Y_I=\bigcup_{w \in W^I \cap \Omega_I} C_I(w). $$
% \end{lemma}
% \begin{proof}
% %As $\bP_I=\bigcap_{\alpha \notin I} \bP(\omega_\alpha)$, we can use the same argument applied to a Borel in the proof of proposition 4.1 in \cite{O1} to see that $Y_I=\bigcup_{w \in \Omega_I} \bP_I w\bB/\bB$. 

% By lemma \ref{} $Y_I=\bigcup_{w \in \Omega_I} C(w)$. Let $w \in \Omega_I$ and $v \in W_I$ with reduced expression $v=s_1s_2\ldots s_r$  where $s_i=s_{\alpha}$ for some $\alpha \in I$. 
% Then by \cite{Bo}[1.10] we have $$(vw\mu, \varpi_\beta)=(w\mu, v^{-1}\varpi_\beta)=(w\mu, \varpi_\beta)>0$$ for all $\beta \in \Delta \backslash I$.  Hence
% $W_Iw \subset \Omega_I$ for every $w \in \Omega_I$ and the claim follows. 
% \end{proof}

% From now on assume that $\bG=\mathbf{GL_n}$ for $n \in  \BN$ with simple roots $\alpha_i=e_i-e_{i+1} \in \BQ^n\cong X^*(\bT)_\BQ$.
% Then we identify $\mu$ under the isomorphism $X_*(T)_\BQ\cong \BQ^n$ with $(\mu_1,\ldots, \mu_n) \in \BQ^{n}$ where by assumption $\mu_i>\mu_{i+1}$ for all $i$ and $\mu=\sum n_\alpha \alpha^{\vee}$ with $n_\alpha\geq0$ all $\alpha$. 
% Additionally for $\alpha \in I$ we denote by $ \check{\varpi}_\alpha \in X^*(\bT)$ the fundamental weight i.e. 
% $$ \langle \alpha^{\vee}, \check{\varpi}_\beta \rangle =\delta_{\alpha,\beta}.$$
% Then 
%     $$(\alpha^{\vee}, \varpi_\beta) = (\alpha^{\vee}, \varpi_\beta)=(\frac{2}{(\alpha,\alpha)}\alpha^*,\varpi_\beta)= \frac{2}{(\alpha,\alpha)} \langle \alpha, \varpi_\beta \rangle= \frac{2}{(\alpha,\alpha)}  \langle \alpha^{\vee}, \check{\varpi}_\beta \rangle$$

% and we see that
% $$ (\sum n_\alpha \alpha^{\vee}, \varpi_\beta )=\frac{2}{(\alpha, \alpha)} n_\beta>0 \text{ if and only if } \langle \sum n_\alpha \alpha^{\vee}, \check{\varpi}_\beta \rangle = n_\beta>0. $$
% Hence we can in the following assume that $(\,,\,)$  is the 
% standard inner product on $\BQ^n$ and $\{\varpi_\alpha\}_{\alpha \in \Delta}$ are the usual fundamental weights of $\bG$.

% \begin{lemma} Let $I \subset \Delta$ and $w \in W^I \cap \Omega_I$. Then $w \in \Omega_\emptyset$. 
% \end{lemma}
% \begin{proof}
%     We have to show that $(w\mu, \varpi_\alpha ) >0$ for all $\alpha \in I$.
%     Therefor we write $\Delta \backslash I=\{\alpha_{i_1}, \ldots, \alpha_{i_k}\}$ 
%     with $i_l<i_{l+1}$ for all $l$. In the following we will frequently use that $\langle w\mu, \alpha \rangle \geq 0$ 
%     for all $\alpha \in I$ since $w \in W^I$ and $\mu$ lies in the positive Weyl chamber. Hence 
    
%     \begin{equation}\label{chamber}
%         \mu_{w^{-1}(j)}-\mu_{w^{-1}(j+1)}\geq0
%     \end{equation}
%     for all $a_j \in I$. Now assume there is an $\alpha_{j} \in I$ with $(w\mu, \varpi_{\alpha_j})\leq 0$. 
%     There are three possible cases for $j$ and we consider for each a system of inequalities \\

%     \underline{$j<i_1:$}
%     \begin{align*}
%     \text{I) }&0\geq (w\mu, \varpi_{\alpha_j})=\sum_{i=1}^{j} \mu_{w^{-1}(i)} \\
%     \text{II) }&0<(w\mu, \varpi_{\alpha_{i_1}})-(w\mu, \varpi_{\alpha_j})=\sum_{i=j+1}^{i_1} \mu_{w^{-1}(i)}.
%     \end{align*}

%     \underline{$i_l<j<i_{l+1}:$}
%     \begin{align*}
%         \text{I) }&0>(w\mu, \varpi_{\alpha_j})-(w\mu, \varpi_{\alpha_{i_{l}}})=\sum_{i=i_l+1}^{j} \mu_{w^{-1}(i)}, \\
%         \text{II) }&0<(w\mu, \varpi_{\alpha_{i_{l+1}}})-(w\mu, \varpi_{\alpha_j})=\sum_{i=j+1}^{i_{l+1}} \mu_{w^{-1}(i)}.
%     \end{align*}

%     \underline{$i_k<j:$}
%     \begin{align*}
%         \text{I) }&0>(w\mu, \varpi_{\alpha_j})-(w\mu, \varpi_{\alpha_{i_{k}}})=\sum_{i=i_k+1}^{j} \mu_{w^{-1}(i)}, \\
%         \text{II) }&0\leq\sum_{i=1}^n \mu_i - (w\mu, \varpi_{\alpha_j})= \sum_{i=j+1}^{n} \mu_{w^{-1}(i)}.
%     \end{align*}
%     In each case \ref{chamber} implies that $\mu_{w^{-1}(j)}-\mu_{w^{-1}(j+1)}< 0$ which is a contradiction to \ref{chamber}.
% \end{proof}


% %\todo[inline]{\textbf{Comment/Question}: Still hoping that these complexes are exact at the most positions to concludee somehow (maybe by weight reasons) that the spectral sequence degenerates at $E_2$ page. But I'm not sure how to continue. Some of the local cohomology groups will vanish by dimension reasons but it is at the moment not obvious which one. Maybe there is for each $q$ some $K \subset \Delta$ such that $H^q_{Y_I}(\sF,\CE)=0$ whenever $K \not \subset I$ and a generalized result of Theorem 2.5, ch. III in Period Domains by Dat, Orlik, Rapoport }






%For the character group $X^*(\bT)$ we have via derivation an injection 
%\begin{equation}\label{charsub}
%    \iota:X^*(\bT) \hookrightarrow{} \fkt_K^*.
%\end{equation}
%Similarly evaluating the derivation of an element of the cocharacter group $X_*(\bT)$ at one gives an injection  
%\begin{equation}\label{cocharsub}
%   \iota^\vee:X_*(\bT) \hookrightarrow{} \fkt_K.
%\end{equation}
%Therefore we can identitfy from now on $X^*(\bT)$ as subgroup of $\fkt_K^*$ respectively $X_*(\bT)$ as subgroup of $\fkt_K$.
%Moreover \ref{charsub} and \ref{cocharsub} induce isomorphism 
%\begin{align*}
%   &X^*(\bT)\otimes_\BZ K\cong \fkt_K^*, \\
%   &X_*(\bT)\otimes_\BZ K\cong \fkt_K
%\end{align*}
%which extends the pairing \eqref{pairing} to the pairing 
%\begin{align*}
%   \langle \text{ , } \rangle:\fkt^*_K \times \fkt_K %&\longrightarrow K \\ 
%   (\lambda, \chi) &\mapsto \lambda \circ \chi. \nonumber
%\end{align*}
%such that the natural action of $W$ on $\fkt^*$ is given as in %\ref{Waction}, i.e. 
%\begin{equation}
%\lambda(w_\alpha t w_\alpha)=\lambda(t)-\langle \lambda, \iota^\vee(\alpha^\vee) \rangle \iota(\alpha)(t)
%\end{equation}
%for $t \in \fkt$.
%\todo[inline, ]{\textbf{Comment:} Maybe it turns out that this last part is not necessary, but at the moment I think I need it for a proof in chapter \ref{s:CatO}.}
%\todo[inline, ]{\textbf{Question:} $\mathbf{T_K}$ or $\bT$ doesn't matter? look for quote of these facts?} 


