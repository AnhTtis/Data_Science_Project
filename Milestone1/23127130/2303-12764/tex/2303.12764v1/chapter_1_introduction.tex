%!TEX root = draft.tex
\section{Introduction}\label{s:Introduction} 

The origin of period domains lies in the work of Griffiths \cite{Gr1, Gr2}. He introduced them as certain open subspaces of generalized flag varieties over $\BC$ which parametrize polarized $\BR$-Hodge structures of a given type. Rapoport and Zink introduced period domains over $p$-adic fields \cite{RZ} in which we are interested in this paper. For this, let $K=\BQ_p$. Given a reductive group $\bG$ over $K$, a period domain over $K$ parametrizes weakly admissible filtrations on a $\bG$-isocrystal of a fixed Hodge type. It is an open admissible rigid-analytic subset of a generalized flag variety $\sF$ (cf. section \ref{s:perdom}). The prototype for a $p$-adic period domain over $K$ is Drinfeld's upper half space $\Omega^{(n+1)}$, which is the complement of all $K$-rational hyperplanes in the projective space $\BP^{n}_K$, i.e. 
$$ \Omega^{(n+1)}= \BP^{n}_K \backslash \bigcup_{ H \subsetneq K^{n+1}} \BP(H).$$ 
It arises from the trivial $\mathbf{GL}_{n+1}$-isocrystal inside the projective space $\sF=\BP^{n}_K$.  \\

Given an appropriate cohomology theory, it is a natural problem to determine cohomology groups for period domains. The starting point is the work of Drinfeld \cite{D}, who computed the first étale cohomology group of $\Omega^{(2)}$. Schneider and Stuhler \cite{SS} computed cohomology groups of $\Omega^{(n+1)}$ in the $p$-adic case for ``good'' cohomology theories. This includes the étale cohomology with torsion coefficients, not including $p$-torsion, and the de Rham cohomology. 
% The formula for $\ell$-adic cohomology with compact support is also valid over finite fields. This was shown by Orlik \cite{O3} where he determined the $\ell$-adic cohomology 
% with compact support for period domains over finite fields for $\GL_n$. He also conjectured that the result could be generalized to arbitrary reductive groups over finite and $p$-adic fields. 
% For finite fields this was proven to be true by Orlik in \cite{O4}. In the $p$-adic setting, computations were restricted to period domains in the case of 
% a basic isocrystal for quasi-split reductive groups. For these Orlik computed the étale cohomology with compact support 
% and torsion coefficients \cite{O1} . Notice that there are some restrictions on the torsion, such as not including $p$-torsion. 
% As a side product, he was able to find the $\ell$-adic cohomology with compact support in loc. cit. 
% Afterwards, Orlik determined the continuous $\ell$-adic cohomology with compact support \cite{O5}.
% A harder problem in the $p$-adic setting was to compute the $p$-torsion and $p$-adic étale cohomologies with compact support.
% This was recently solved by Colmez, Dospinescu, Hauseux and Niziol \cite{CDHN}. \\
So far, the only results for coherent sheaf cohomology are known for Drinfeld's upper half space over $p$-adic fields. After the work of Schneider and Stuhler, it was Schneider together with Teitelbaum, who made the beginning and considered at first coefficients in the canonical bundle \cite{ST1}. Shortly afterwards, Pohlkamp \cite{Po} computed the sheaf cohomology with respect to the structure sheaf. Finally, Orlik was able to generalize these results to arbitrary $\mathbf{GL}_{n+1}$-equivariant vector bundles on Drinfeld's upper half space over $p$-adic fields which are induced by restriction of a homogeneous vector bundle $\CE$ on $\BP^{n}_K$ \cite{O2}. It turns out that by Schneider and Teitelbaum \cite{ST1}, the space of global sections $\CE(\Omega^{(n+1)})$ is a reflexive $K$-Fréchet space with a continuous $\mathbf{GL}_{n+1}(K)$-action and its strong dual is a locally analytic $\mathbf{GL}_{n+1}(K)$-representation.\\
% Moreover, he could apply his methods to compute its pro-étale cohomology \cite{O6} and the coherent sheaf cohomology of $\Omega^{(n+1)}$ in the case of finite fields \cite{O7}. 
% It was Kuschkowitz, a student of Orlik, who computed the rigid cohomology of Drinfeld's upper halfspace over finite fields with similar methods \cite{Ku}. \\

The goal of this paper is to investigate sheaf cohomology of period domains over $p$-adic fields, other than $\Omega^{(n+1)}$, with coefficients in certain line bundles. For this, let $\bG$ be a split connected reductive group over $K$ with split maximal torus $\bT$. Further, let $\bB \supset \bT$ a Borel subgroup of $\bG$ associated to a cocharacter $\mu \in X_*(\bG)$ defined over $K$. Then, we consider the associated period domain $\sF^{\wa}$ which parametrizes weakly admissible filtrations on the trivial $\bG$-isocrystal of Hodge type $\mu$ inside the complete flag variety $\sF:=\bG/\bB$. We study the sheaf cohomology of these spaces with respect to the restriction of a homogeneous line bundle $\CE_\lambda:=\CL_\lambda \otimes \omega_{\sF}$ on $\sF$. Here, $\omega_{\sF}$ denotes the canonical bundle on $\sF$ and $\CL_\lambda$ a line bundle associated to a dominant weight $\lambda  \in X^*(\bT)$ wih respect to $\bB$ (cf. section \ref{s:Setup}). 

Let $G$ and $B$ be the $K$-valued points of $\bG$ and $\bB$, respectively. Furthermore, let $W$ be the Weyl group of $\bG$. For $w \in W$, we denote by $V_B^G(w)$ the twisted generalized locally analytic Steinberg representation of weight $w\cdot \lambda$ (cf. Definition \ref{twistedSteinberg}). It is a quotient of $\Ind^G_{B}({K_{w\cdot \lambda}}')$, the locally analytic induced representation of the dual of the simple algebraic $\bT$-representation of weight $w \cdot \lambda$.  Under the assumption of a hypothesis concerning the density of some local cohomology groups (cf. Assumption \ref{hypo1}), we prove the following result.

\begin{theorem}[Theorem \ref{theorem1}]\label{theorem1intro}
   Let $i_0:=\dim\sF-\lb\Delta\rb$. The homology of the (chain) complex 
   \begin{equation}\label{Cbullet}
      C_\bullet:  \bigoplus_{\substack{w \in \Omega_\emptyset \\ l(w)=\dim Y_\emptyset}}V^G_B(w) \leftarrow \ldots \leftarrow \bigoplus_{\substack{w \in \Omega_\emptyset \\ \ l(w)=1}} V_B^G(w) \leftarrow  V^G_B(\lambda) 
   \end{equation}
   starting in degree $i_0$ coincides with $H^*(\sF^{\wa},\CE_\lambda)'$, i.e. $H_*(C_\bullet)=H^*(\sF^{\wa},\CE_\lambda)'$.
\end{theorem}

Here, $\Delta$ is the set of simple roots of the root system of $\bG$ with respect to $\bB$. Further, $\Omega_\emptyset$ is a subset of $W$  defined by some numerical conditions (cf. (\ref{omegaI})) and $Y_\emptyset$ a union of Schubert cells in $\sF$ indexed by $\Omega_\emptyset$ which is closed in $\sF$ (cf. subsection \ref{s:GeoProp}). Moreover, $H^i(\sF^{\wa},\CE_\lambda)'$ denotes the strong dual of $H^i(\sF^{\wa},\CE_\lambda)$ in the sense of Schneider and Tei\-telbaum (cf. \cite[p. 50]{S}). \\

We briefly explain the idea of the proof. For this let be $Y:=\sF^{\ad} \backslash \sF^{\wa}$ with closed embedding $\iota:Y\hookrightarrow \sF^{\mathrm{ad}}$.  The main ingredient is the acyclic complex 
\begin{equation}\label{fundamentalcomplexintro}
   0 \longrightarrow \BZ \longrightarrow \bigoplus\limits_{\substack{I \subset \Delta \\ \lb \Delta \backslash I \rb = 1 }} \BZ_I \longrightarrow \bigoplus\limits_{\substack{I \subset \Delta \\ \lb \Delta \backslash I \rb = 2 }} \BZ_I \longrightarrow \ldots \longrightarrow \bigoplus\limits_{\substack{I \subset \Delta \\ \lb \Delta \backslash I \rb = \lb \Delta \rb -1 }} \BZ_I \longrightarrow \BZ_\emptyset \longrightarrow 0
\end{equation}
on the étale site of $Y$ (cf. \cite[Theorem 6.9]{CDHN}). By applying $\Ext^*(\iota_*(-),\CE_\lambda)$ to the complex (\ref{fundamentalcomplexintro}), we get the spectral sequence 
\begin{equation*}\label{spectralintro}
  \hat{E}_1^{-p,q}=\Ext^q(\bigoplus\limits_{\substack{I \subsetneq \Delta \\ \lb \Delta \backslash I \rb = p+1 }} \iota_*(\BZ_I), \CE_\lambda) \Rightarrow \Ext^{-p+q}(\iota_*(\BZ_{Y}),\CE_\lambda)=H^{-p+q}_{Y}(\sF^{\mathrm{ad}},\CE_\lambda). 
\end{equation*}
From this we find a spectral sequence $(E_r, d_r)_{r \geq 1}$ converging to $H^*(\sF^{\mathrm{wa}},\CE_\lambda)$.
We construct a second quadrant double chain complex $D_{\bullet, \bullet}$ (cf. (\ref{double})), where one of the two induced standard spectral sequences coincides with the strong dual of $(E_r, d_r)_{r \in \BN_{\geq 1}}$ and the other one yields the chain complex in Theorem \ref{theorem1intro}. A crucial point in the proof is that 
   $$H^i_{C_I(w)}(\sF,\CE_\lambda)\cong\begin{cases} M_I(w\cdot\lambda) &i=n-l(w),\\ 0 &\text{else} \end{cases}$$
for $I \subset \Delta$ and $w \in W^I$ (cf. Lemma \ref{C_I(w)})\footnote{Here, $C_I(w)$ is a generalized Schubert cell in $\sF$ (cf. Defintion \ref{defC_Iw}) and $M_I(w\cdot\lambda)$ is a generalized parabolic Verma module of highest weight $w\cdot\lambda$ (cf. Example \ref{genVM})}. Because of this fact we can identify each column of the $E_1$-page with the homology of a chain complex which leads to the construction of $D_{\bullet, \bullet}.$\\

For the analysis of (\ref{Cbullet}), we first noted that Orlik und Schraen described the Jordan-Hölder factors of $V^G_B(e)$  (cf. \cite[Theorem 4.6]{OSch}). We partially generalize this to all objects in  $C_\bullet$ (cf. (\ref{Cbullet})). 
% \begin{theorem}[{\cite[Theorem 4.6]{OSch}}]\footnote{If the root sytem $\Phi(\bG,\bT)$ has irreducible components of type $B$, $C$, or $F_4$, we assume $p>2$, and if $\Phi(\bG,\bT)$ has irreducible components of type $G_2$, we assume that $p>3$ ( cf. \cite[Section 5]{OSt}).\label{fn:note1} }Fix $w \in W$ and let $I:=I(w)$. For a subset $J \subset \Delta$ with $J \subset I$, let $v_{P_J}^{P_I}$ be the generalized smooth Steinberg representation of $L_{P_I}$. Then, the multiplicity of the irreducible $G$-representation $\CF^G_{P_I}(L(w\cdot \lambda),v_{P_J}^{P_I})$ in $V^G_B(e)$ is $$ \sum_{\substack{w' \in W  \\ \supp(w')=J }} (-1)^{\ell(w')+\lb J  \rb} m(w',w)$$ and we obtain in this way all the Jordan-Hölder factors of $V^G_B(e)$. Moreover, this multiplicity is non-zero if and only if $J \subset supp(w)$. 
% \end{theorem} 
\begin{theorem}[Theorem \ref{multiplicities}]\label{multiplicitiesintro}
   \footnote{If the root sytem $\Phi(\bG,\bT)$ has irreducible components of type $B$, $C$, or $F_4$, we assume $p>2$, and if $\Phi(\bG,\bT)$ has irreducible components of type $G_2$, we assume that $p>3$ (cf. \cite[Section 5]{OSt}).\label{fn:note1} } Fix $w,v \in W$ and let $I_0:=I(w)$ respectively $I:=I(v)$ (cf. (\ref{maximalI})). For a subset $J \subset \Delta$ with $J \subset I$, 
   let $v_{P_J}^{P_I}$ be the generalized smooth Steinberg representation of $L_{P_I}$. Then, the multiplicity of the irreducible $G$-representation 
   $\CF^G_{P_I}(L(v\cdot \lambda),v_{P_J}^{P_I})$ in $V^G_B(w)$ is 
   $$ \sum_{\substack{w' \in W  \\ \supp(w')=J \cap I_0}} (-1)^{\ell(w')+\lb J \cap I_0 \rb} m(w'w,v)$$ 
   and we obtain in this way all the Jordan-Hölder factors of $V^G_B(w)$. 
\end{theorem} 

From this we see that the homology of $C_\bullet$ is closely related to the Kazhdan-Lusztig conjecture. Therefore, we compute the Jordan-Hölder factors of the homology of $C_\bullet$ only in concrete examples (cf. Example \ref{exampleCohomology}). In this computations we make use of the fact that the morphism $p_{w',w}:V^G_B(w')\rightarrow V^G_B(w)$ in the differentials of $C_\bullet$ is surjective for $w',w \in W$ with $w' \leq w$ (cf. Lemma \ref{surjection}). \\

The paper is divided into two parts. The first half is devoted the preliminaries that we will use afterwards. In detail, we recall some basics about local cohomology in Section \ref{s:localcohomology}, about a split reductive group $\bG$ over a finite extension of $\BQ_p$ in Section \ref{s:rootdatum}, about  the functor $\CF^G_P$ in Section \ref{s:FGP} and last but not least about $p$-adic period domains in Section \ref{s:perdom}. In 
Section \ref{s:FGP} we also prove Theorem \ref{multiplicitiesintro}. 

In the second half, we first introduce our setup in Section \ref{s:Setup}. 
This includes the period domain $\sF^{\wa}$ inside the complete flag variety 
$\sF$ over $K$ and the line bundle $\CE_\lambda$ on $\sF$ associated to a dominant character $\lambda$ of $\bG$ with respect to the Borel pair $(\bT,\bB)$. 
In the next section, we make some geometric observations for the complement $Y$ of $\sF^{\wa}$ in $\sF^\ad$. In particular, (generalized) Schubert cells and unions of Schubert varieties will appear there. Then, in Section \ref{s:AlgLocCo}, we determine the algebraic local cohomology groups of $\sF$ 
with support in these (locally) closed subsets and coefficients in $\CE_\lambda$. We relate them to analytic local cohomology groups of $\sF^{\rig}$ in Section \ref{s:AnaLocCo}. In Section \ref{s:results}, we use this relation to deduce Theorem \ref{theorem1intro}. For this we use Orlik's fundamental complex (cf. \cite[Section 6.2.2]{CDHN}) and an induced spectral sequence in cohomology. Moreover, we determine the Jordan-Hölder factors of the dual of $H^*(\sF^{\wa}, \CE_\lambda)$ in examples with the help of the computer. 
%  In the last section we explain why our strategy does not automatically transfer to the general parabolic case. 

In the Appendix \ref{s:Appendix} we list the code we use to determine the Jordan-Hölder factors of the terms in the chain complexes in the examples given. 

% The Jordan-Hölder factors can also be found in the appendix. 

\subsection{Notations}
 Let $p$ be a prime and $K$ be a finite extension of $\BQ_p$. Further let $L$ be a complete extension of $\BQ_p$ with $K\subset L$. We let $\CO_K$ and $\CO_L$, respectively, be the ring of integers of $K$ and $L$, respectively. Moreover, let $\lb \text{\,\,\,} \rb$ be the absolute value of $K$ and $L$, respectively, such that $\lb p \rb = p^{-1}$. \\


% Let $p$ be a prime, $K$ a perfect field and $L$ a finite extension of $K$. 
% When $K$ is a finite extension of $\BQ_p$, we let $\CO_K$ respectively $\CO_L$ be the ring of integers of $K$ and $L$, respectively.
% Moreover, let $\lb \text{  \, } \rb$ be the absolute value of $K$ and $L$, respectively, such that $\lb p \rb = p^{-1}$.    \\


We use bold letters for algebraic group schemes over $K$, e.g. $\bG$, $\bB$. The corresponding groups of $K$-valued points are denoted by normal letters, e.g.  $G$, $B$ and the associated Lie algebras by Gothic letters, e.g. $\fkg$, $\fkb$. We write $U(\fkh)$ for the universal enveloping algebra of a Lie algebra $\fkh$ over $K$. \\

% To each of these algebraic groups we associate Lie algebras $\fkg=\Lie(\bG)$,
% $\fkb=\Lie(\bB)$,  $\fkt=\Lie(\bT)$, $\fkp=\Lie(\bP)$, 
% $\fkl_\RP=\Lie(\mathbf{L_P})$, $\fku_\RP=\Lie(\mathbf{U_P})$ respectively. 
We consider $L$ as the field of coefficients. The base change of a $K$-vector 
space or a scheme over $K$ to $L$ is indicated by $L$ in the subscript, e.g. $\mathfrak{g}_L=\mathfrak{g}\otimes_K L$. 
We make an exception when considering a universal enveloping algebra, i.e. we will write  $U(\fkh)$ for $U(\fkh)_L \cong U(\fkh_L)$. \\

We denote by $\Rep^{\infty,\, \mathrm{adm}}_L(H)$ the category of smooth admissible representations of a locally profinite group $H$ on $L$-vector spaces, as in \cite[Section 2.1]{BH2}. \\

For a locally convex $L$-vector space $V$, we denote by $V'$ the strong dual, i.e. the $L$- vector space of continuous linear forms equipped with
the strong topology of bounded convergence. \\ 

For an algebraic variety $X$ over $K$, we write $X^\mathrm{rig}$ for the rigid analytic variety and by $X^\ad$ the adic space attached to $X$, respectively. If $\CE$ is a sheaf on such a variety $X$, we also write $\CE$ for the associated sheaf on $X^{\rig}$, $X^{\ad}$ and its restriction to any subspace, respectively. 


\subsection{Acknowledgements} This paper is part of a PhD thesis of the Bergische Universität Wuppertal. My biggest thanks therefore goes to Sascha Orlik. He gave me the opportunity for this research project. It is a multifaceted topic, which somehow miraculously connects areas of mathematics that at first glance do not seem to be in contact with each other - wonderful. Moreover, he actively mentored me throughout the process. \\

I would also like to thank Roland Huber, Thomas Hudson, Henry July, Georg Linden and Sean Tilson for all the helpful mathematical conversations and the support throughout the process. \\

This research was conducted in the framework of the research training group GRK 2240: Algebro-Geometric Methods in Algebra, Arithmetic and Topology, which is funded by the DFG.
% Of course I would also like to thank my family and friends who always had an open ear for me. 
% I would especially like to mention my sister Karolin, as well as Corinna Hank, Georg Kempa, Bento Natura, Daniel Rehn and Christoph Schaller. Thank you very much. You are great. \\

% Last but not least, I would like to thank my father, who made it all possible. Enjoy your pension. You deserve it! 
%For the character group $X^*(\bT)$ we have via derivation an injection 
%\begin{equation}\label{charsub}
%    \iota:X^*(\bT) \hookrightarrow{} \fkt_K^*.
%\end{equation}
%Similarly evaluating the derivation of an element of the cocharacter group $X_*(\bT)$ at one gives an injection  
%\begin{equation}\label{cocharsub}
%   \iota^\vee:X_*(\bT) \hookrightarrow{} \fkt_K.
%\end{equation}
%Therefore we can identitfy from now on $X^*(\bT)$ as subgroup of $\fkt_K^*$ respectively $X_*(\bT)$ as subgroup of $\fkt_K$.
%Moreover \ref{charsub} and \ref{cocharsub} induce isomorphism 
%\begin{align*}
%   &X^*(\bT)\otimes_\BZ K\cong \fkt_K^*, \\
%   &X_*(\bT)\otimes_\BZ K\cong \fkt_K
%\end{align*}
%which extends the pairing \eqref{pairing} to the pairing 
%\begin{align*}
%   \langle \text{ , } \rangle:\fkt^*_K \times \fkt_K %&\longrightarrow K \\ 
%   (\lambda, \chi) &\mapsto \lambda \circ \chi. \nonumber
%\end{align*}
%such that the natural action of $W$ on $\fkt^*$ is given as in %\ref{Waction}, i.e. 
%\begin{equation}
%\lambda(w_\alpha t w_\alpha)=\lambda(t)-\langle \lambda, \iota^\vee(\alpha^\vee) \rangle \iota(\alpha)(t)
%\end{equation}
%for $t \in \fkt$.
%\todo[inline, ]{\textbf{Comment:} Maybe it turns out that this last part is not necessary, but at the moment I think I need it for a proof in chapter \ref{s:CatO}.}
%\todo[inline, ]{\textbf{Question:} $\mathbf{T_K}$ or $\bT$ doesn't matter? look for quote of these facts?} 


