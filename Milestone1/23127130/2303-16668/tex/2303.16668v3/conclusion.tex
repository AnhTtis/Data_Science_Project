\section{Conclusion}
\label{sec:conclusion}
We introduced FLANDERS, a novel FL filter robust to extreme untargeted model poisoning attacks, i.e., when malicious clients far exceed legitimate participants. 
FLANDERS serves as a pre-processing step before applying aggregation rules, enhancing robustness across diverse hostile settings.
\\
FLANDERS treats the sequence of local model updates sent by clients in each FL round as a matrix-valued time series. 
Then, it identifies malicious client updates as outliers in this time series using a matrix autoregressive forecasting model.
\\
Experiments conducted in several non-iid FL setups demonstrated that existing (secure) aggregation methods further improve their robustness when paired with FLANDERS. 
Moreover, FLANDERS allows these methods to operate even under extremely severe attack scenarios thanks to its ability to accurately filter out every malicious client \textit{before} the aggregation process takes place.
% Experiments %run on four datasets 
% %under several FL settings 
% demonstrated that FLANDERS exhibits greater consistency across the spectrum of attack settings compared to other methods. %matches the robustness of the most powerful baselines. 
% Furthermore, since FLANDERS is agnostic to the number of malicious clients and incorporates temporal dependencies by design, it remains highly effective even under extremely severe attacks, %(i.e., beyond $40\%$ malicious clients), 
% as opposed to existing defense strategies that either fail or are not even designed to face such scenarios.
\\
In the future, we will address the primary limitations of this work, as discussed in Section~\ref{sec:limitations}.
% as Krum and Bulyan.
% \\
% In the future, we will address some limitations of this work. 
% First, we expect to reduce the computational cost of FLANDERS further, especially for complex models with millions of parameters, leveraging the implementation tricks discussed in Appendix~\ref{app:defenses}. 
% Then, we plan to extend FLANDERS using more expressive MAR($p$) models ($p > 1$) and conduct a deeper analysis of its key parameters (e.g., $w$, $k$, and $\delta$).
%This work paves the way for several future directions. 
%In future work, we plan to extend FLANDERS using more complex MAR($p$) models and improve its scalability.
%In the future, we will address some of the limitations of this work. 
%Indeed, FLANDERS is more computationally-intensive than existing defense mechanisms, primarily when combined with complex models having millions of parameters. 
%Some implementation tricks can reduce the computational cost of FLANDERS, as reported in Appendix~\ref{app:defenses}. Still, we expect to improve its scalability further. 
%Finally, we plan to extend FLANDERS using more expressive MAR($p$) models and conduct a deeper analysis of its key parameters (e.g., $w$, $k$, and $\delta$).