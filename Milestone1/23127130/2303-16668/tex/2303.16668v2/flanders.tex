\documentclass{article}


% if you need to pass options to natbib, use, e.g.:
%     \PassOptionsToPackage{numbers, compress}{natbib}
% before loading neurips_2024

\PassOptionsToPackage{numbers, compress}{natbib} % added for number style
% ready for submission
\usepackage[preprint]{neurips_2024}


% to compile a preprint version, e.g., for submission to arXiv, add add the
% [preprint] option:
%     \usepackage[preprint]{neurips_2024}


% to compile a camera-ready version, add the [final] option, e.g.:
%     \usepackage[final]{neurips_2024}


% to avoid loading the natbib package, add option nonatbib:
%    \usepackage[nonatbib]{neurips_2024}
\usepackage[utf8]{inputenc} % allow utf-8 input
\usepackage[T1]{fontenc}    % use 8-bit T1 fonts
\usepackage{hyperref}       % hyperlinks
\usepackage{url}            % simple URL typesetting
\usepackage{booktabs}       % professional-quality tables
\usepackage{amsfonts}       % blackboard math symbols
\usepackage{nicefrac}       % compact symbols for 1/2, etc.
\usepackage{microtype}      % microtypography
\usepackage{xcolor}         % colors

\usepackage{nicefrac}       % compact symbols for 1/2, etc.
\usepackage{microtype}      % microtypography
\usepackage{xcolor}         % colors
\usepackage{amsmath}
\usepackage{amssymb}
\usepackage{bm}
\usepackage{bbm}
\usepackage{todonotes}
\usepackage{algorithm}
\usepackage{algpseudocode}
\usepackage{multicol}
\usepackage{caption}
\usepackage{multirow}
\usepackage{tabularx}
\usepackage{makecell}
\usepackage{graphicx}           % figures
\usepackage{duckuments}         % sample images
\usepackage{bold-extra}
\usepackage{wrapfig}

%%%%%%%%%%%%%%%%%%%%%%%%%%%%%%%%%%%%%%
% For subcaption and subfigure
\usepackage{caption}
\usepackage{subcaption}
\usepackage{svg}
%%%%%%%%%%%%%%%%%%%%%%%%%%%%%%%%%%%%%%

\newcommand{\argmax}{\text{arg\,max}}
\newcommand{\argmin}{\text{arg\,min}}
\newcommand{\argsup}{\text{arg\,sup}}
\newcommand{\softmax}{\text{softmax}}

\algrenewcommand\algorithmicrequire{\textbf{Input:}}
\algrenewcommand\algorithmicensure{\textbf{Output:}}
\algnewcommand\algorithmicforeach{\textbf{for each}}
\algdef{S}[FOR]{ForEach}[1]{\algorithmicforeach\ #1\ \algorithmicdo}


\title{Protecting Federated Learning from Extreme Model Poisoning Attacks via Multidimensional Time Series Anomaly Detection}

\author{
  Edoardo~Gabrielli \\%\thanks{Use footnote for providing further information about author (webpage, alternative address)---\emph{not} for acknowledging funding agencies.} \\
  Sapienza University of Rome\\
  Italy \\
  \texttt{edoardo.gabrielli@uniroma1.it} \\
  \And
  Dimitri~Belli \\
  ISTI-CNR, Pisa \\
  Italy \\
  \texttt{dimitri.belli@isti.cnr.it} \\
  \And
  Vittorio~Miori \\
  ISTI-CNR, Pisa \\
  Italy \\
  \texttt{vittorio.miori@isti.cnr.it} \\
  \And
  Gabriele~Tolomei \\
  Sapienza University of Rome \\
  Italy \\
  \texttt{tolomei@di.uniroma1.it} \\
  % \And
  % Coauthor \\
  % Affiliation \\
  % Address \\
  % \texttt{email} \\
}

%\renewcommand{\shortauthors}{Gabrielli et al.}


\begin{document}
\maketitle
%%
%% The abstract is a short summary of the work to be presented in the
%% article.
\begin{abstract}
Current defense mechanisms against model poisoning attacks in federated learning (FL) systems have proven effective up to a certain threshold of malicious clients. %(e.g., $25\%$ to $50\%$).
\\
In this work, we introduce FLANDERS, a novel pre-aggregation filter for FL resilient to large-scale model poisoning attacks, i.e., when malicious clients far exceed legitimate participants.
FLANDERS treats the sequence of local models sent by clients in each FL round as a matrix-valued time series. Then, it identifies malicious client updates as outliers in this time series by comparing actual observations with estimates generated by a matrix autoregressive forecasting model maintained by the server.
Experiments conducted in several non-iid FL setups show that FLANDERS significantly improves robustness across a wide spectrum of attacks when paired with standard and robust existing aggregation methods.
\end{abstract}


\newcommand{\Prob}{\mathbb{P}}
\newcommand{\R}{\mathbb{R}}
\newcommand{\Z}{\mathbb{Z}}
\newcommand{\E}{\mathbb{E}}
\newcommand{\insta}{\bm{x}}
\newcommand{\X}{X}
\newcommand{\dataset}{\mathcal{D}}
\newcommand{\train}{\dataset_{\text{train}}}
\newcommand{\test}{\dataset_{\text{test}}}
\newcommand{\features}{\mathcal{X}}
\newcommand{\labels}{\mathcal{Y}}
\newcommand{\hypspace}{\mathcal{H}}
\newcommand{\params}{\bm{\theta}}
\newcommand{\Params}{\bm{\Theta}}
\newcommand{\w}{\bm{\omega}}
\newcommand{\loss}{\ell}
\newcommand{\Loss}{\mathcal{L}}
\newcommand{\ind}{\mathbbm{1}}


% INTRODUCTION
\section{INTRODUCTION}
\label{sec:introduction}

% Diffusion of FL
Recently, {\em federated learning} (FL) has emerged as the leading paradigm for training distributed, large-scale, and privacy-preserving machine learning (ML) systems~\cite{mcmahan2017googleai,mcmahan2017aistats}. 
The core idea of FL is to allow multiple edge clients to collaboratively train a shared, global model without disclosing their local private training data.
%Specifically, an FL system consists of a central server and many edge clients; 
A typical FL round involves the following steps: {\em(i)} the server randomly picks some clients and sends them the current, global model; {\em(ii)} each selected client locally trains its model with its own private data; then, it sends the resulting local model to the server;\footnote{Whenever we refer to global/local model, we mean global/local model {\em parameters}.} {\em(iii)} the server updates the global model by computing an \emph{aggregation function}, usually the average (FedAvg), on the local models received from clients.
This process goes on until the global model converges. 
\\
% Security threats to FL
Although its advantages over standard ML, FL also raises security concerns~\cite{costa2022covert}. 
Here, we focus on \emph{untargeted model poisoning} attacks~\cite{bhagoji2019pmlr}, where an adversary attempts to tweak the global model weights by directly perturbing the local model's parameters of some infected clients before these are sent to the central server for aggregation.
In doing so, the adversary aims to jeopardize the global model \textit{indiscriminately} at inference time.
Such model poisoning attacks severely impact standard FedAvg; therefore, more robust aggregation functions must be designed to secure FL systems.
\\
% Limitations of existing Byzantine-resilient strategies
Unfortunately, existing defense mechanisms either rely on simple heuristics (e.g., Trimmed Mean and FedMedian by~\cite{yin2018icml}) or need strong and unrealistic assumptions to work effectively (e.g., foreknowledge or estimation of the number of malicious clients in the FL system, as for Krum/Multi-Krum~\cite{blanchard2017nips} and Bulyan~\cite{mhamdi2018pmlr}, which, however, cannot exceed a fixed threshold).
Furthermore, outlier detection methods using K-means clustering~\cite{shen2016acm} or spectral analysis like DnC~\cite{shejwalkar2021ndss} do not directly consider the temporal evolution of local model updates received.
Finally, strategies like FLTrust~\cite{cao2020fltrust} require the server to collect its own dataset and act as a proper client, thereby altering the standard FL protocol.
\\
% Description of the proposed method
This work introduces a novel pre-aggregation \textit{filter} robust to untargeted model poisoning attacks. Notably, this filter $(i)$ operates without requiring prior knowledge or constraints on the number of malicious clients and $(ii)$ inherently integrates temporal dependencies. 
The FL server can employ this filter as a preprocessing step before applying \textit{any} aggregation function, be it standard like FedAvg or robust like Krum or Bulyan.
Specifically, we formulate the problem of identifying corrupted updates as a multidimensional (i.e., matrix-valued) time series anomaly detection task. 
The key idea is that legitimate local updates, resulting from well-calibrated iterative procedures like stochastic gradient descent (SGD) with an appropriate learning rate, show \textit{higher predictability} compared to malicious updates. This hypothesis stems from the fact that the sequence of gradients (thus, model parameters) observed during legitimate training exhibit regular patterns, as validated in Section~\ref{subsec:intuition}.
\\
Inspired by the matrix autoregressive (MAR) framework for multidimensional time series forecasting~\cite{chen2021je}, we propose the FLANDERS ({\em \textbf{F}ederated \textbf{L}earning meets \textbf{AN}omaly \textbf{DE}tection for a \textbf{R}obust and \textbf{S}ecure}) filter.
The main advantage of FLANDERS over existing strategies like FLDetector~\cite{zhao2020multivariate} is its resilience to large-scale attacks, where $50\%$ or more FL participants are hostile.
We attribute such a capability to two key factors: $(i)$ FLANDERS works without knowing a priori the ratio of corrupted clients, and $(ii)$ it embodies temporal dependencies between intra- and inter-client updates, quickly recognizing local model drifts caused by evil players. Below, we summarize our main contributions:
\\
{$\bullet$}
We provide empirical evidence that the sequence of models sent by legitimate clients is more predictable than those of malicious participants performing untargeted model poisoning attacks.
\\
{$\bullet$} 
We introduce FLANDERS, the first pre-aggregation filter for FL robust to untargeted model poisoning based on multidimensional time series anomaly detection.
\\
{$\bullet$} 
We integrate FLANDERS into Flower,\footnote{\scriptsize{\url{https://flower.dev/}}} a popular FL simulation framework for reproducibility.
\\
{$\bullet$}  
We show that FLANDERS improves the robustness of the existing aggregation methods under multiple settings: different datasets, client's data distribution (non-iid), models, and attack scenarios.
\\
{$\bullet$} 
We publicly release all the implementation code of FLANDERS along with our experiments.\footnote{\scriptsize{\url{https://anonymous.4open.science/r/flanders_exp-7EEB}}}

% Paper's structure and organization
The remainder of the paper is structured as follows. 
Section~\ref{sec:background} covers background and preliminaries. 
In Section~\ref{sec:related}, we discuss related work.
Section~\ref{sec:problem} and Section~\ref{sec:method} describe the problem formulation and the method proposed. 
Section~\ref{sec:experiments} gathers experimental results. 
Finally, we conclude in Section~\ref{sec:conclusion}.

% BACKGROUND AND PRELIMINARIES
\section{BACKGROUND AND PRELIMINARIES}
\label{sec:background}
\newcommand\addtag{\refstepcounter{equation}\tag{\theequation}}
\subsection{Federated Learning}
\label{subsec:fl}
We consider a typical supervised learning task under a standard FL setting, which consists of a central server $S$ and a set of distributed clients $\mathcal{C}$, such that $|\mathcal{C}|=K$.
Each client $c\in \mathcal{C}$ has its own private training set $\dataset_c$, namely the set of its $n_c$ local labeled examples, i.e., $\dataset_c = \{\bm{x}_{c,i}, y_{c,i}\}_{i=1}^{n_c}$.
\\
The goal of FL is to train a global predictive model whose architecture and parameters $\params^*\in \R^d$ are shared across all clients by solving 
$\params^* = \text{argmin}_{\params} \Loss(\params) = \text{argmin}_{\params} \sum_{c=1}^K p_c \Loss_c(\params;\dataset_c),$ where $\Loss_c$ is the local objective function for client $c$. Usually, this is defined as the empirical risk calculated over the training set $\dataset_c$ sampled from the client's local data distribution:
$\Loss_c(\params;\dataset_c) = \frac{1}{n_c}\sum_{i=1}^{n_c} \loss(\params;(\insta_{c,i}, y_{c,i})),$ where $\loss$ is an instance-level loss, e.g., cross-entropy (classification) or squared error (regression). 
Each $p_c \geq 0$ specifies the relative contribution of each client. 
Since it must hold that $\sum_{c=1}^{K}p_c = 1$, two possible settings are: $p_c = 1/K$ or $p_c = n_c/n$, where $n = \sum_{c=1}^K n_c$. 

The generic federated round at each time $t$ is decomposed into the following steps and iteratively performed until convergence, i.e., for each $t\in \{1,2,\ldots,T\}$:
\\
{\em(i)} $S$ randomly selects a subset of clients ${\mathcal{C}}^{(t)}\subseteq \mathcal{C}$, so that $1 \leq |{\mathcal{C}}^{(t)}| \leq K$, and sends them the current, global model $\params^{(t)}$. At $t=1$, $\params^{(1)}$ is randomly initalized.
In the following, we assume that the number of clients picked at each round is constant and fixed, i.e., $|\mathcal{C}^{(t)}| = m,~\forall t\in \{1,2,\ldots, T\}$.
\\
{\em(ii)} Each selected client $c\in {\mathcal{C}}^{(t)}$ trains its local model $\params_c^{(t)}$ on its own private data $\dataset_c$ by optimizing the following objective, starting from $\params^{(t)}$: $\params_c^{(t)} = \text{argmin}_{\params^{(t)}}\Loss_c(\params^{(t)}; \dataset_c)$.
The value $\params_c^{(t)}$ is computed via gradient-based methods like stochastic gradient descent and sent to $S$. 
\\
{\em(iii)} $S$ computes $\params^{(t+1)} = \phi(\{\params_c^{(t)}~|~c\in \mathcal{C}^{(t)}\})$ as the updated global model, where $\phi: \R^{d^m} \mapsto \R^d $ is an \emph{aggregation function}; for example, $\phi = \frac{1}{m}\sum_{c\in \mathcal{C}^{(t)}} \params_c^{(t)}$, i.e., FedAvg or alike~\cite{lu2020spml}.


\subsection{The Attack Model: Federated Aggregation under Model Poisoning}
\label{subsec:byzantine}
The most straightforward aggregation function $\phi$ the server can implement is FedAvg, which computes the global model as the average of the local model weights received from clients. 
FedAvg is effective when all the FL participants behave honestly~\cite{dean2012nips,konecny2016nipsws,mcmahan2017aistats}.
Instead, this work assumes that an attacker controls a fraction $b=\lceil r*K \rceil, r\in [0,1]$ of the $K$ clients, i.e., $0\leq b\leq K$, known as malicious. 

\noindent{{\bf {\em Attacker's Goal.}}}
Inspired by many studies on poisoning attacks against ML~\cite{rubinstein2009imc,biggio2012icml,biggio2013icb,xiao2015icml,li2016nips,yang2017ndss,jagielski2018sp}, we consider the attacker's goal is to jeopardize the jointly learned global model \textit{indiscriminately} at inference time with \textit{any} test example. 
Such attacks are known as {\em untargeted}~\cite{fang2020usenix}, as opposed to {\em targeted} poisoning attacks, where instead, the goal is to induce prediction errors only for some specific test inputs (e.g., via so-called \textit{backdoor triggers} as shown by~\cite{bagdasaryan2020pmlr}).
\\
\noindent{{\bf {\em Attacker's Capability.}}}
Like Sybil attacks to distributed systems~\cite{douceur2002iptps}, the attacker can inject $b$ fake clients into the FL system or compromise $b$ honest clients.
The attacker can arbitrarily manipulate the local models sent by malicious clients to the server $S$.
More formally, let $x \in \mathcal{C}$ be one of the $b$ corrupted clients selected by the server on the generic $t$-th FL round; it first computes its legitimate local model $\params_x^{(t)}$ without modifying its private data $\dataset_x$; then it finds $\check{\params}_x^{(t)}$ by applying a {\em post hoc} perturbation $\bm{\varepsilon} \in \R^d$ to $\params_x^{(t)}$. For example, $\check{\params}_x^{(t)} = \params_x^{(t)} + \bm{\varepsilon}$, where $\bm{\varepsilon}\sim \mathcal{N}(\bm{\mu}, \bm{\Sigma})$ is a Gaussian noise vector. 
More advanced attack strategies have been designed, as discussed in Section~\ref{subsec:attacks}.
\\
\noindent{{\bf {\em Attacker's Knowledge.}}}
We assume the attacker knows its controlled clients' code, local training datasets, and local models. 
Moreover, we consider the worst-case scenario, where the attacker is \textit{omniscient}, i.e., it has full knowledge of the parameters sent by honest parties. This allows the attacker to crafting malicious local models that are close to legitimate ones, thus increasing the probability of being chosen by the server for the aggregation.
\\
\noindent{{\bf {\em Attacker's Behavior.}}}
In each FL round, the attacker, similar to the server, randomly selects $b$ clients to corrupt out of the $K$ available. Any of these $b$ malicious clients that happen to be among those selected by the server will poison their local models using one of the strategies outlined in Section~\ref{subsec:attacks}. 

This category of {\em untargeted model poisoning} attacks has been extensively explored in previous works~\cite{blanchard2017nips,bhagoji2019pmlr,fang2020usenix}.
It has been shown that including updates even from a single malicious client can wildly disrupt the global model if the server runs standard FedAvg~\cite{blanchard2017nips,yin2018icml}. 

% RELATED WORK
\section{REALATED WORK}
\label{sec:related}

Below, we describe the most popular defenses against model poisoning attacks on FL systems, which will serve as the baselines for our comparison.
A more comprehensive discussion is in~\cite{hu2021arxiv,barroso2023if}.

\noindent{{\bf {\em FedMedian}}~\cite{yin2018icml}{\bf.}}
The central server sorts the $j$-th parameters received from all the $m$ local models and takes the median of those as the value of the $j$-th parameter of the global model. This process is applied for all the model parameters, i.e., $\forall j\in \{1,\ldots,d\}$.
\\
\noindent{{\bf {\em Trimmed Mean}}~\cite{xie2018corr}{\bf.}}
This rule computes a model as FedMedian does, and then it averages the $k$ nearest parameters to the median.
If $b$ clients are compromised at most, this aggregation rule achieves an order-optimal error rate when $b \leq \frac{m}{2} - 1$.
\\
\noindent{{\bf {\em Krum}}~\cite{blanchard2017nips}{\bf.}}
It selects one of the $m$ local models received from the clients, which is most similar to \emph{all} other models, as the global model. The rationale behind this approach is that even if the chosen local model is poisoned, its influence would be restricted because it resembles other local models, potentially from benign clients.
A variant called Multi-Krum mixes Krum with standard FedAvg. 
\\
\noindent{{\bf {\em Bulyan}}~\cite{mhamdi2018pmlr}{\bf.}}
Since a single component can largely impact the Euclidean distance between two high-dimensional vectors, Krum may lose effectiveness in complex, high-dimensional parameter spaces due to the influence of a few abnormal local model weights. Bulyan iteratively applies Krum to select $\alpha$ local models and aggregates them with a Trimmed Mean variant to mitigate this issue.
\\
\noindent{{\bf {\em DnC}}~\cite{shejwalkar2021ndss}{\bf.}}
This robust aggregation uses spectral analysis to detect and filter outliers as proposed by~\cite{diakonikolas2017icml}. 
Similarly, DnC computes the principal component of the set of local updates sent by clients. Then, it projects each local update onto this principal component. Finally, it removes a constant fraction of the submitted model updates with the largest projections.
\\
\noindent{{\bf {\em FLDetector}}~\cite{zhang2022fldetector}{\bf.}} 
This method is the closest to our approach. It filters out malicious clients by measuring their consistency across $N$ rounds using an approximation of the integrated Hessian to compute the \textit{suspicious scores} for all clients. However, unlike our method, FLDetector struggles with large numbers of malicious clients, and cannot work with highly heterogeneous data.


% PROBLEM FORMULATION
\section{PROBLEM FORMULATION}
\label{sec:problem}

\subsection{Time Series of Local Models}
\label{subsec:ts-models}
At the end of each FL round $t$, the central server $S$ collects the updated local models $\{\params_c^{(t)}\}_{c\in \mathcal{C}^{(t)}}$ sent by the subset of selected clients.\footnote{A similar reasoning would apply if clients sent their local displacement vectors $\bm{u}_c^{(t)} = \params_c^{(t)} - \params^{(t)}$ or local gradients $\nabla{\Loss}_c^{(t)}$ rather than local model parameters $\params_c^{(t)}$.}
Without loss of generality, we assume that the number of clients picked at each round is constant and fixed, i.e., $|\mathcal{C}^{(t)}| = m,~\forall t\in \{1,2,\ldots, T\}$.
Hence, the server arranges the local models received at round $t$ into a $d\times m$ matrix $\Params_{t} = [\params_1^{(t)}, \ldots, \params_c^{(t)}, \ldots, \params_m^{(t)}]$, whose $c$-th column corresponds to the $d$-dimensional vector of updated parameters $\params_c^{(t)}$ sent by client $c$.
However, the subset of selected clients may be different at each FL round, i.e., $\mathcal{C}^{(t)} \neq \mathcal{C}^{(t')}$ for $t\neq t'$, although we have assumed their size ($m$) is the same. 
More generally, to track the local models sent by \textit{all} clients chosen across $1\leq w \leq T$ rounds, we can extend each $\Params_{t}$ into a $d\times h$ matrix, where $h = |\bigcup_{t=1}^w \mathcal{C}^{(t)}|$, such that $m \leq h \leq K$, with $K$ being the total number of clients. 
Notice that $h=m$ when $w=1$, whereas $h=K$ if the server selects all $K$ clients at least once over the $w$ rounds considered.
At every round $t$, $\Params_{t}$ contains $m$ columns corresponding to the local models sent by the clients \textit{actually} selected for that round. In contrast, the remaining $h-m$ ``fictitious'' columns refer to the unselected clients.
We fill these columns with the current global model at FL round $t$, i.e., $\params^{(t)}$, as if this were sent by the $h-m$ unselected clients. 
Note that this strategy is neutral and will not impact the aggregated global model computed by the server for the next round $(t+1)$, as this is calculated only from the updates received by the $m$ clients previously selected. 

\subsection{Predictability of Local Models Evolution: Legitimate vs. Malicious Clients}
\label{subsec:intuition}
We claim that legitimate local updates show \textit{higher predictability} compared to malicious updates. 
This hypothesis stems from the fact that the sequence of gradients (hence, model parameters) observed during legitimate training should exhibit ``regular'' patterns until convergence. 
\\
To demonstrate this behavior, we consider two clients $i,j \in \mathcal{C}$. Client $i$ is assumed to be a legitimate participant, while client $j$ acts maliciously. %alternates between legitimate and malicious behavior over several FL rounds. 
Furthermore, both clients are selected by the server over a sequence of consecutive $T$ FL rounds to train a global model on the \textit{MNIST} dataset.
Therefore, we examine the local models sent to the server at each round $t \in \{1,2,\ldots,T\}$ by clients $i$ and $j$. Specifically, these are $d$-dimensional vectors of real-valued parameters $\params_i^{(t)}, \params_j^{(t)} \in \R^d$, such that $\params_i^{(t)} = (\theta_{i,1}^{(t)}, \ldots, \theta_{i,d}^{(t)})$ and $\params_j^{(t)} = (\theta_{j,1}^{(t)}, \ldots, \theta_{j,d}^{(t)})$, respectively.
Next, we calculate the average time-delayed mutual information (TDMI) for each pair of observed local models $(\params_i^{(t)}, \params_i^{(t+\delta)})$ and $(\params_j^{(t)}, \params_j^{(t+\delta)})$ across $T=50$ rounds, where $\delta > 0$, for both client $i$ and $j$. This analysis aims to discern the time-dependent nonlinear correlation and the level of predictability between observations. 
We expect the TDMI calculated between pairs of models sent by the legitimate client $i$ to be higher than that computed between pairs sent by the malicious client $j$. Thus, the temporal sequence of local models observed from client $i$ is more predictable than that of client $j$.
\\
Let $\params_c^{(t)} = (\theta_{c,1}^{(t)}, \ldots, \theta_{c,d}^{(t)})$ denote the local model sent by the generic FL client $c\in \mathcal{C}$ as an instance of a $d$-dimensional vector-valued process. 
We define $\text{TDMI}(\params_c^{(t)}, \params_c^{(t+\delta)})$ as follows: 
\begin{equation}
\small
\text{TDMI}(\params_c^{(t)}, \params_c^{(t+\delta)}) =  \int p(\params_c^{(t)}, \params_c^{(t+\delta)})\times \log\Bigg(\frac{p(\params_c^{(t)}, \params_c^{(t+\delta)})}{p(\params_c^{(t)})p( \params_c^{(t+\delta)})}\Bigg) d\params_c^{(t)} d\params_c^{(t+\delta)},
\label{eq:tdmi2}
\end{equation}
where $p(\cdot)$ is the probability density function. 
If each realization $\theta_{c,k}^{(t)}$ in the vector $\params_c^{(t)}$ is \textit{independent} from each other, we can compute the average TDMI as follows~\cite{albers2012chaos}:
\begin{equation}
\small
\text{Avg}\Big[\text{TDMI}(\params_c^{(t)}, \params_c^{(t+\delta)})\Big] = \frac{1}{d}\Bigg[\sum_{k=1}^d \text{TDMI}(\theta_{c,k}^{(t)}, \theta_{c,k}^{(t+\delta)})\Bigg], 
\label{eq:avg-tdmi}
\end{equation}
where TDMI$(\theta_{c,k}^{(t)}, \theta_{c,k}^{(t+\delta)})$ calculates TDMI for the univariate case.
Indeed, the $d$ model parameters $\theta_{c,k}^{(\cdot)}$ should reasonably be independent. Under this assumption, the joint probability density function of the parameters factors into a product of individual probability density functions, forming a product measure on $d$-dimensional Euclidean space. Thus, according to Fubini's theorem~\cite{billingsley2017probability}, the integral of each parameter will be independent of the others because the parameters are independent.
\\
We consider the empirical distributions of the average TDMI between each pair of \textit{consecutive} models $(\params_c^{(t)},\params_c^{(t+1)})$, sent by the legitimate client ($c=i$) and the malicious client ($c=j$) when running one of the four attacks described in Section~\ref{subsec:attacks} (see Fig.~\ref{fig:avg-tdmi-density} in Appendix~\ref{app:legit-vs-malicious}). 
We call $\bar{\params_i}$ and $\bar{\params_j}$ the mean of these empirical distributions, respectively. Then, we run a one-tailed $t$-test against the null hypothesis $H_0: \bar{\params_i} = \bar{\params_j}$, where the alternative hypothesis is $H_a: \bar{\params_i} > \bar{\params_j}$. 
In all four cases, there is enough evidence to reject the null hypothesis at a confidence level $\alpha=0.01$ (see Tab.~\ref{tab:t-test} in Appendix~\ref{app:legit-vs-malicious}).
\\
This empirical evidence supports our claim and shows that the sequence of models sent by legitimate clients to the FL server is more predictable. 
Further discussion is provided in Appendix~\ref{app:legit-vs-malicious}.

\subsection{Poisoned Local Models as Matrix-Based Time Series Outliers}
\label{subsec:ts-outliers}
We, therefore, formulate our problem as a multidimensional time series anomaly detection task.
At a high level, we want to equip the central server with an anomaly scoring function that estimates the degree of each client picked at the current round being malicious, i.e., its {\em anomaly score}, based on the historical observations of model updates seen so far from the clients selected.
Such a score will be used to restrict the set of trustworthy candidate clients for the downstream aggregation method. 
Note that unselected clients will \textit{not} contribute to the aggregation; thus, there is no need to compute their anomaly score even if they were marked as suspicious in some previous rounds.
On the other hand, we must design a fallback strategy for ``cold start'' clients  -- i.e., FL participants who send their local updates for the first time or have not been selected in any of the previous rounds considered  -- whether honest or malicious.
\\
More formally, let $\hat{\Params}_t = f(\Params_{t-w:t-1};\hat{\bm{\Omega}})$ be the matrix of local model updates \textit{predicted} by the central server $S$ at the generic FL round $t$. 
The forecasting model $f$ depends on a set of parameters $\hat{\bm{\Omega}}$ estimated using the past $w$ model updates observed, i.e., $\Params_{t-w:t-1}$, where $1\leq w \leq t-1$. 
Also, the server sees the actual matrix $\Params_t$.
\\
The anomaly score $s_{c}^{(t)} \in \R$ of the generic client $c$ at round $t$ can thus be defined as follows:
\begin{equation}
\small
\label{eq:anomaly-score}
s_{c}^{(t)} =
\begin{cases}
\delta(\params^{(t)}_c, \hat{\params}^{(t)}_c),\text{ if }c\in \mathcal{C}^{(t)} \wedge \exists j\in\{1..w\}\text{ s.t. }c\in \mathcal{C}^{(t-j)} \\
\delta(\params^{(t)}, \params^{(t)}_c), \text{ if }c\in \mathcal{C}^{(t)} \wedge \nexists j\in\{1..w\}\text{ s.t. }c\in \mathcal{C}^{(t-j)}\\
\perp, \text{ if }c\notin \mathcal{C}^{(t)}.
\end{cases}
\end{equation}
The first condition refers to a client selected for round $t$, which also appeared at least once in the history. 
In this case, $\delta$ measures the distance between the observed vector of weights sent to the server ($\params^{(t)}_c$) and the predicted vector of weights ($\hat{\params}^{(t)}_c$) output by the forecasting model $f$.
The second condition, instead, occurs when a client is selected for the first time at round $t$ or does not appear in the previous $w$ historical matrices of observations used to generate predictions. Here, due to the cold start problem, we cannot rely on $f$ and, therefore, the most sensible strategy is to compute the distance between the current global model and the local update received by the new client. 
Finally, the anomaly score is undefined ($\perp$) for any client not selected for round $t$.
\\
The server will thus rank all the $m$ selected clients according to their anomaly scores (e.g., from the lowest to the highest).
Several strategies can be adopted to choose which model updates should be aggregated in preparation for the next round, i.e., to restrict from the initial set $\mathcal{C}^{(t)}$ to another (possibly smaller) set $\mathcal{C}_{*}^{(t)}\subseteq \mathcal{C}^{(t)}$ of trusted clients. 
For example, $S$ may consider only the model updates received from the top-$k$ clients ($1 \leq k \leq m$) with the smallest anomaly score, i.e., $\mathcal{C}_*^{(t)} = \{c\in \mathcal{C}^{(t)}~|~s_{c}^{(t)} \leq s_k^{(t)}\}$, where $s_k^{(t)}$ indicates the $k$-th smallest anomaly score at round $t$. 
Alternatively, the raw anomaly scores computed by the server can be converted into well-calibrated probability estimates. 
Here, the server sets a threshold $\rho \in [0,1]$ and aggregates the weights only of those clients whose anomaly score is below $\rho$, i.e., $\mathcal{C}_*^{(t)} = \{c\in \mathcal{C}^{(t)}~|~s_c^{(t)}\leq \rho\}$. 
In the former case, the number of considered clients ($k$) is bound apriori,\footnote{This may not be true if anomaly scores are not unique; in that case, we can simply enforce $|\mathcal{C}_*^{(t)}| = k$.} whereas the latter does not put any constraint on the size of final candidates $|\mathcal{C}_*^{(t)}|$.
Eventually, $S$ will compute the updated global model $\params^{(t+1)} = \phi(\{\params_c^{(t)}~|~c\in \mathcal{C}_*^{(t)}\})$, where $\phi$ is any aggregation function, e.g., FedAvg, Bulyan, or any other strategy.

Below, we describe how we use the matrix autoregressive (MAR) framework proposed by \cite{chen2021je} to implement our multidimensional time series forecasting model $f$, hence the anomaly score.


% PROPOSED METHOD
\section{PROPOSED METHOD: FLANDERS}
\label{sec:method}

\subsection{Matrix Autoregressive Model (MAR)}
\label{subsec:mar}
We assume the temporal evolution of the local models sent by FL clients at each round is captured by a matrix autoregressive model (MAR). 
In its most generic form, MAR($w$) is a $w$-order autoregressive model defined as follows:
\begin{equation}
\small
    \Params_t = {\bm A}_1 \Params_{t-1} {\bm B}_1 + \cdots + {\bm A}_w \Params_{t-w} {\bm B}_w + {\bm E}_t,
    \label{eq:mar-w}
\end{equation}
where $\Params_t$ is the $d\times h$ matrix of observations at time $t$, $\bm{\Omega} = \{{\bm A}_i, {\bm B}_i\}_{i=1}^w$ are $d\times d$ and $h\times h$ autoregressive coefficient matrices, and ${\bm E}_t$ is a $d\times h$ white noise matrix.
In this work, we consider the simplest MAR($1$)\footnote{Unless otherwise specified, whenever we refer to MAR, we assume MAR($1$).} Markovian forecasting model, i.e., ${\Params}_t = {\bm A} {\Params}_{t-1} {\bm B} + {\bm E}_t$, where the matrix of local updates at time $t$ depends only on the matrix observed at time step $t-1$, namely $w=1$.

Let $\widetilde{\Params}_t \approx \Params_t$ be the {\em predicted} matrix of observations at time $t$, according to a model $f$ parameterized by $\widetilde{\bm{\Omega}} = \{\widetilde{{\bm A}},\widetilde{{\bm B}}\}$, i.e., $\widetilde{\Params}_t = f(\Params_{t-1};\widetilde{\bm{\Omega}}) = \widetilde{{\bm A}} \Params_{t-1}\widetilde{{\bm B}}$.
If we have access to $l > 0$ historical observations, we can estimate the best coefficients ${\hat{\bm \Omega}} = \{\hat{{\bm A}}, \hat{{\bm B}}\}$ by solving the following objective: 
\begin{equation}
\small
    {\hat{\bm \Omega}} = \hat{{\bm A}}, \hat{{\bm B}} = \argmin_{\widetilde{{\bm A}}, \widetilde{{\bm B}}}
    \Big\{\sum_{j=0}^{l-1} ||{\Params}_{t-j} - \widetilde{{\bm A}}{\Params}_{t-j-1}\widetilde{{\bm B}}||^2_{\text F} \Big\},
    \label{eq:mar-opt}
\end{equation}
where $||\cdot||_{\text F}$ indicates the Frobenius norm. 
Note that $l$ impacts only the size of the training set used to estimate the optimal coefficients $\hat{\bm A}$ and $\hat{\bm B}$; it does not affect the order of the autoregressive model, which will remain MAR($1$) and \textit{not} MAR($l$).
The optimal coefficients $\hat{\bm A}$ and $\hat{\bm B}$ can thus be estimated via alternating least squares (ALS) optimization~\cite{koren2009ieeecomp}.
Further details are provided in Appendix~\ref{app:mar}.

\subsection{MAR-based Anomaly Score}
\label{subsec:mar-anomaly-score}
Our approach consists of two primary steps: $(i)$ \textit{MAR estimation} and $(ii)$ \textit{anomaly score computation}. 
\\
\noindent{{\bf {\em MAR Estimation.}}}
At the first FL round ($t=1$), the server sends the initial global model $\params^{(1)}$ to the set of $m$ selected clients $\mathcal{C}^{(1)}$ and collects from them the $d\times m$ matrix of updated models $\Params_1$. Hence, it computes the \textit{new} global model $\params^{(2)} = \phi(\{\params_{c}^{(1)}~|~c\in \mathcal{C}^{(1)}\})$, where $\phi = \text{Krum}$ or any other existing robust aggregation heuristic.
For any other FL round $t > 1$, the server can use the past $l > 0$ historical observations $\Params_{t-l:t-1}$ to estimate the best MAR coefficients $\hat{\bm{\Omega}} = \{\hat{\bm{A}},\hat{\bm{B}}\}$ according to~(\ref{eq:mar-opt}). 
In general, $1\leq l\leq t-1$; however, if we assume $l$ fixed at each round, the server will consider $\Params_{\max(1,t-l):t-1}$ past observations.
Again, independently of the value of $l$, MAR will learn to predict the current matrix of weights \textit{only} from the previously observed matrix.
Therefore, each matrix used for training $\{\Params_{t-j-1}\}_{j=0}^{l-1}$ has the same size $d\times m$, as it contains exclusively the $m$ updates received at the round $t-j-1$. 
Of course, employing a higher order MAR($w$) model with $w > 1$ would require extending each observed matrix to $d\times h$ ($m < h \leq K)$, as detailed in Section~\ref{subsec:ts-models}. This is to track the local updates sent by selected clients across multiple historical rounds.
\\
\noindent{{\bf {\em Anomaly Score Computation.}}}
At the generic FL round $t > 1$, we compute the anomaly score using the estimated MAR forecasting model as follows. 
Let $\Params_{t}$ be the matrix of observed weights. 
This matrix may contain one or more corrupted local models from malicious clients.
Then, we compute the $m$-dimensional anomaly score vector ${\bm s}^{(t)}$, where ${\bm s}^{(t)}[c] = s_c^{(t)}$, as in~(\ref{eq:anomaly-score}).
A critical choice concerns the function $\delta$ used to measure the distance between the observed vector of weights sent by each selected client and the vector of weights predicted by MAR.
In this work, we set $\delta (\bm{u}, \bm{v}) = ||\bm{u} - \bm{v}||_2^2$, where $||\cdot||_2^2$ is the squared $L^2$-norm.  
Other functions can be used (e.g., {\em cosine distance}), especially in high dimensional spaces, where $L^p$-norm with $p\in (0,1)$ has proven effective~\cite{aggarwal2001icdt}. 
Choosing the best $\delta$ is outside the scope of this work, and we leave it to future study.
\\
According to one of the filtering strategies discussed in Section~\ref{subsec:ts-outliers}, we retain only the $k$ clients with the smallest anomaly scores. The remaining $m-k$ clients are considered malicious; thus, they 
are discarded and do not contribute to the aggregation run by the server as if they were never selected.
\\
At the next round $t+1$, we may want to refresh our estimation of the MAR model, i.e., to update the coefficient matrices $\hat{{\bm A}}$ and $\hat{{\bm B}}$.
We do so by considering the latest observed $\Params_{t}$ and the other $l-1$ previous matrices of local updates, using the same sliding window of size $l$.
Since the observed matrix $\Params_{t}$ contains $m-k$ potentially malicious clients, we cannot use it as-is. Otherwise, \textit{if any of the spotted malicious clients is selected again at round $t+1$}, we may alter the estimation of $\hat{{\bm A}}$ and $\hat{{\bm B}}$ with possibly corrupted matrix columns.
To overcome this problem, we replace the original $\Params_{t}$ with $\Params'_{t}$ {\em before}, feeding it to train the new MAR model. 
Specifically, $\Params'_{t}$ is obtained from $\Params_{t}$ by substituting the $m-k$ anomalous columns either with the parameter vectors from the same clients observed at time $t-1$, which are supposed to be still legitimate \textit{or} the current global model.
The advantage of this solution is twofold. 
On the one hand, a client labeled as malicious at FL round $t$ would likely still be considered so at $t+1$ if it keeps perturbing its local weights, thus improving robustness.
On the other hand, our solution allows malicious clients to alternate legitimate behaviors without being banned, speeding up model convergence.
Notice that the two considerations above might not be valid if we updated the MAR model using the original, partially corrupted $\Params_{t}$. 
Indeed, in the first case, the distance between two successive poisoned models by the same client would reasonably be small. So, the client's anomaly score will likely drop to non-alarming values, thereby increasing the number of false negatives.
In the second case, the distance between a corrupted and a legitimate model would likely be large. 
Thus, a malicious client will maintain its anomaly score high even if it acts honestly, impacting the number of false positives.
\\
A comprehensive overview of how FLANDERS works along with its pseudocode is in Appendix~\ref{app:flanders}.


\subsection{Computational Complexity Analysis}
\label{subsec:complexity}
To train our MAR forecasting model, we first need to run the ALS algorithm, which estimates the coefficient matrices $\hat{{\bm A}}$ and $\hat{{\bm B}}$. 
ALS iteratively performs two steps to compute $\hat{{\bm A}}$ and $\hat{{\bm B}}$, respectively. 
At the generic $i$-th iteration, computing the $i$-th $d\times d$ matrix $\hat{{\bm A}}$ costs $O(d^3) + O(dm^2)$; similarly, computing the $i$-th $m\times m$ matrix\footnote{It is worth remarking that, using MAR($1$), $h = m$ and thus $\hat{{\bm B}}$ has size $m\times m$.} $\hat{{\bm B}}$ costs $O(m^3) + O(d^2m)$.
Thus, a single iteration costs $O(d^3) + O(dm^2) + O(m^3) + O(d^2m)$, namely $O(d^3)$ \textit{or} $O(m^3)$,\footnote{$O(d^{2.376})$ \textit{or} $O(m^{2.376})$ using the Coppersmith-Winograd algorithm~\cite{coppersmith1990jsc}.} 
depending what term dominates between $d$ (the number of weights) and $m$ (the number of clients). 
ALS performs the two steps above for $N$ iterations (e.g., in this work, we set $N=100$).
Second, the computation of the anomaly score costs $O(md)$ as it measures the distance between observed and predicted local updates from all the $m$ selected clients.
Third, every time the MAR model is refreshed, we need to rerun ALS; in the worst-case scenario, we might want to re-estimate $\hat{{\bm A}}$ and $\hat{{\bm B}}$ at every single FL round.
\\
It is worth noticing that performing ALS directly on high dimensional parameter space ($d$) in standard FL settings with loads of clients ($m$) may be unfeasible (e.g., when $d$ and $m$ range from $10^6$ to $10^9$).
To keep the computational cost tractable (and limit the impact of the curse of dimensionality), in our experiments, where $d \gg m$, we reduce dimensionality via random sampling on the parameter space, as proposed by~\cite{shejwalkar2021ndss}. Specifically, we sample $\tilde{d} < d$ model parameters for ALS, with $\tilde{d}$ set to $500$.

% EXPERIMENTS
\section{EXPERIMENTS}
\label{sec:experiments}

\subsection{Experimental Setup}
\noindent{{\bf {\em Datasets, Tasks, and FL Models.}}}
We consider four public datasets for image classification: {\em MNIST}, {\em Fashion-MNIST}, {\em CIFAR-10}, and {\em CIFAR-100}.
All datasets are randomly shuffled and partitioned into two disjoint sets: 80\% is used for training and 20\% for testing. 
We train a Multilayer Perceptron (MLP) on \textit{MNIST} and \textit{Fashion-MNIST} and a Convolutional Neural Network (CNN) on \textit{CIFAR-10} and \textit{CIFAR-100} (MobileNet~\cite{howard2017mobilenets}).
All models are trained by minimizing cross-entropy loss.
\\
The full details are available in Appendix~\ref{app:setup}.
\\
\noindent{{\bf {\em FL Simulation Environment.}}}
To simulate a realistic FL environment, we integrate FLANDERS into Flower \cite{beutel2022flower}. Moreover, we implement in Flower any defense baseline considered that the framework does not natively provide.
We set the number $K$ of FL clients to $100$, and we assume that the server selects {\em all} these clients in every FL round ($m=K$). In Appendix~\ref{app:client-selection}, we test when the server \textit{randomly} chooses $m<K$ clients. We suppose the attack starts at $t=1$; then, we monitor the performance of the global model until $t=T=50$. 
Note that we select the malicious clients randomly across all the participating ones; this implies that a single client can alternate between legitimate and malicious behavior over successive FL rounds.
\\
We describe our FL simulation environment in Appendix~\ref{app:fl-env}.
\\
\noindent{{\bf {\em Non-IID Local Training Data.}}}
We simulate non-iid clients' training data distributions following~\cite{hsu2019LDA}. We assume every client training example is drawn independently with class labels following a categorical distribution over $N$ classes parameterized by a vector $\bm{q}$ such that $q_i \geq 0, i \in [1,\ldots,N]$ and $||\bm{q}||_1 = 1$.
To synthesize a population of non-identical clients, we draw $\bm{q} \sim \text{Dir}(\alpha_D, \bm{p})$ from a Dirichlet distribution, where $\bm{p}$ characterizes a prior class distribution over $N$ classes, and $\alpha_D > 0$ is a {\em concentration} parameter controlling the identicalness among clients. 
With $\alpha_D \rightarrow \infty$, all clients have identical label distributions; on the other extreme, with $\alpha_D \rightarrow 0$, each client holds examples from only one class chosen at random. 
In our experiments, we set $\alpha_D = 0.5$.

\subsection{Attacks}
\label{subsec:attacks}
We assess the robustness of FLANDERS under the following well-known attacks.
For each attack, we vary the number $b = \lceil r * m\rceil, r\in [0,1]$ of malicious clients, where $r = \{0, 0.2, 0.6, 0.8\}$.
\\
\noindent{{\bf {\em Gaussian Noise Attack.}}} 
\noindent This attack randomly crafts the local models on the compromised clients. 
Specifically, the attacker samples a random value from a Gaussian distribution $\varepsilon \sim \mathcal{N}(0,\sigma^2)$ and sums it to all the $d$ learned parameters. 
We refer to this attack as GAUSS.
\\
\noindent{{\bf {\em ``A Little Is Enough'' Attack}}~\cite{baruch2019neurips}{\bf.}}
This attack shifts the aggregation gradient by carefully crafting malicious values that deviate from the correct ones as far as possible. 
We call this attack LIE.
\\
\noindent{{\bf {\em Optimization-based Attack}}~\cite{fang2020usenix}{\bf.}}
This attack is framed as an optimization task, aiming to maximize the distance between the poisoned aggregated gradient and the aggregated gradient under no attack. 
By using a halving search, one can obtain a crafted malicious gradient. 
We refer to this attack as OPT.
\\
\noindent{{\bf {\em AGR Attack Series}}~\cite{shejwalkar2021ndss}{\bf.}}
This improves the optimization program above by introducing perturbation vectors and scaling factors. Then, three instances are proposed: AGR-tailored, AGR-agnostic Min-Max, and Min-Sum, which maximize the deviation between benign and malicious gradients.
In this work, we experiment with AGR Min-Max, which we call AGR-MM.
\\
We detail the parameters of the attacks in Appendix~\ref{app:attacks}.

\subsection{Evaluation}
\label{subsec:eval}
We evaluate four key aspects of FLANDERS. 
Firstly, we test its ability to detect malicious clients against the best-competing filtering strategy, FLDetector.
Secondly, we measure the accuracy improvement of the global model when FLANDERS is paired with ``vanilla'' FedAvg and the most popular robust aggregation baselines: FedMedian, Trimmed Mean, Multi-Krum, Bulyan, DnC. 
Thirdly, we analyze the cost-benefit trade-off of FLANDERS.
Lastly, we test the robustness of our method against adaptive attacks. 
The complete experimental settings are illustrated in Appendix~\ref{app:defenses}.

\noindent{\textbf{\textit{Malicious Detection Accuracy}.}}
Tab.~\ref{tab:pr-20} shows the precision ($P$) and recall ($R$) of FLDetector and FLANDERS in filtering out malicious clients across different datasets and attacks, with $r=0.2$ (20\% evil participants in the FL systems). 
Remarkably, FLANDERS successfully detects \textit{all and only} malicious clients across every attack setting except for OPT, outperforming its main competitor, FLDetector.
This is further confirmed by Tab.~\ref{tab:increment-fld}, which shows that our method generally provides much higher protection than FLDetector when combined with standard FedAvg.
Similar findings are observed for different proportions of malicious clients ($r$) and are reported in Appendix~\ref{app:flanders-accuracy}.

\begin{table}[htb!]
\centering
\vspace{-4mm}
\caption{\textit{Precision} ($P$) and \textit{Recall} ($R$) of FLDetector and FLANDERS in detecting malicious clients across various datasets and attacks ($r=0.2$; $T=50$ rounds). Due to space limits, the results for $r=0.6$ and $0.8$ are provided in Tab.~\ref{tab:pr-60} and \ref{tab:pr-80} (Appendix~\ref{app:flanders-accuracy}).}
\label{tab:pr-20}
\scalebox{0.72}{
\begin{tabular}{|c|cc|cc|cc|cc||cc|cc|cc|cc|}
 \cline{2-17}
 \multicolumn{1}{c|}{} & \multicolumn{8}{c||}{\textbf{FLDetector}} & \multicolumn{8}{c|}{\textbf{FLANDERS}} \\
 \hline
 \multirow{2}{*}{\textbf{Dataset}} & \multicolumn{2}{c|}{GAUSS} & \multicolumn{2}{c|}{LIE} & \multicolumn{2}{c|}{OPT} & \multicolumn{2}{c||}{AGR-MM} & \multicolumn{2}{c|}{GAUSS} & \multicolumn{2}{c|}{LIE} & \multicolumn{2}{c|}{OPT} & \multicolumn{2}{c|}{AGR-MM} \\
 \cline{2-17}
                            & $P$   & $R$   & $P$   & $R$   & $P$    & $R$    & $P$   & $R$     & $P$   & $R$   & $P$   & $R$    & $P$  & $R$   & $P$  & $R$ \\
\hline
\textit{MNIST}          & $0.20$ & $0.20$ & $0.22$ & $0.22$ & ${\bf 0.20}$ & ${\bf 0.20}$ & $0.19$ & $0.19$  & ${\bf 1.0}$ & ${\bf 1.0}$ & ${\bf 1.0}$ & ${\bf 1.0}$ & $0.13$ & $0.13$ & ${\bf 1.0}$ & ${\bf 1.0}$ \\
\hline
\textit{Fashion-MNIST}  & $0.21$ & $0.21$ & $0.18$ & $0.18$ & ${\bf 0.21}$ & ${\bf 0.21}$ & $0.20$ & $0.20$  & ${\bf 1.0}$ & ${\bf 1.0}$ & ${\bf 1.0}$ & ${\bf 1.0}$ & $0.13$ & $0.13$ & ${\bf 1.0}$ & ${\bf 1.0}$ \\
\hline
\textit{CIFAR-10}       & $0.21$ & $0.21$ & $0.21$ & $0.21$ & $0.22$ & $0.22$ & $0.19$ & $0.19$  & ${\bf 1.0}$ & ${\bf 1.0}$ & ${\bf 1.0}$ & ${\bf 1.0}$ & ${\bf 0.58}$ & ${\bf 0.58}$ & ${\bf 1.0}$ & ${\bf 1.0}$ \\
\hline
\textit{CIFAR-100}      & $0.20$ & $0.20$ & $0.19$ & $0.19$ & $0.20$ & $0.20$ & $0.21$ & $0.21$  & ${\bf 1.0}$ & ${\bf 1.0}$ & ${\bf 1.0}$ & ${\bf 1.0}$ & ${\bf 1.0}$  & ${\bf 1.0}$  & ${\bf 1.0}$ & ${\bf 1.0}$ \\
\hline
\end{tabular}
}
\end{table}


\begin{table*}[htb!]
\centering
\vspace{-2mm}
\caption{Accuracy of the global model using FedAvg with FLDetector and FLANDERS ($r=0.2$). Due to space limits, the results for $r=0.6$ and $0.8$ are provided in Tab.~\ref{tab:increment-fld-60} and \ref{tab:increment-fld-80} (Appendix~\ref{app:flanders-accuracy}).}
\label{tab:increment-fld}
\scalebox{0.7}{
\begin{tabular}{|c|cccc|cccc|cccc|}
\hline
\multirow{2}{*}{\textbf{Attack}} & \multicolumn{4}{c|}{\textbf{FedAvg}} & \multicolumn{4}{c|}{\textbf{FLDetector + FedAvg}} & \multicolumn{4}{c|}{\textbf{FLANDERS + FedAvg}} \\
\cline{2-13}
                    &  GAUSS        &  LIE          &  OPT          &  AGR-MM       &  GAUSS        &  LIE          &  OPT          &  AGR-MM       &  GAUSS         &  LIE          &  OPT          &  AGR-MM      \\ 
\hline
\textit{MNIST}     & $0.18$ & $0.12$ & $0.63$ & $0.34$ & $0.20$ & $0.11$ & ${\bf 0.68}$ & $0.43$ & ${\bf 0.86}$ & ${\bf 0.83}$ & $0.62$ & ${\bf 0.85}$ \\
\textit{Fashion-MNIST} & $0.25$ & $0.10$ & $0.56$ & $0.16$ & $0.28$ & $0.10$ & ${\bf 0.60}$ & $0.17$ & ${\bf 0.69}$ & ${\bf 0.64}$ & $0.58$ & ${\bf 0.63}$ \\
\textit{CIFAR-10}     & $0.10$ & $0.10$ & $0.23$ & $0.10$ & $0.10$ & $0.10$ & $0.26$ & $0.12$ & ${\bf 0.38}$ & ${\bf 0.37}$ & ${\bf 0.28}$ & ${\bf 0.36}$ \\
\textit{CIFAR-100} & $0.01$ & $0.01$ & $0.01$ & $0.02$ & $0.01$ & $0.01$ & $0.01$ & $0.02$ & ${\bf 0.07}$ & ${\bf 0.05}$ & ${\bf 0.05}$ & ${\bf 0.06}$ \\\hline
\end{tabular}
}
\vspace{-2mm}
\end{table*}

\noindent{\textbf{\textit{Aggregation Robustness Lift}.}}
We proceed to evaluate the enhancement that FLANDERS provides to the robustness of the global model. 
Specifically, we measure the best accuracy of the global model under several attack strengths using all the baselines \textit{without} and \textit{with} FLANDERS as a pre-filtering strategy.
Tab.~\ref{tab:increment} shows that FLANDERS keeps high accuracy for the best global model under extreme attack ($r=0.8$) when used before \textit{every} aggregation method, including Multi-Krum and Bulyan, which would otherwise be inapplicable in such strong attack scenarios.
This is further emphasized in Tab.~\ref{tab:robustness-1} (see Appendix~\ref{app:agg-lift}), where Multi-Krum, when paired with FLANDERS, can effectively operate without \textit{any} performance degradation, even in the presence of $80\%$ malicious clients.

\begin{table*}[htb!]
\centering
\vspace{-2mm}
\caption{Accuracy of the global model using all the baseline aggregations without and with FLANDERS ($r=0.8$). Due to space limits, the results for \textit{CIFAR-100} are shown in Tab.~\ref{tab:increment-cifar100-80} (Appendix~\ref{app:agg-lift}).}
\label{tab:increment}
%\vspace{2mm}
\scalebox{0.7}{
\begin{tabular}{|c|cccc|cccc|cccc|}
\hline
\multirow{2}{*}{\textbf{Aggregation}} & \multicolumn{4}{c|}{\textit{MNIST}} & \multicolumn{4}{c|}{\textit{Fashion-MNIST}} & \multicolumn{4}{c|}{\textit{CIFAR-10}} \\
\cline{2-13}
              &  GAUSS        &  LIE          &  OPT          &  AGR-MM       &  GAUSS        &  LIE          &  OPT          &  AGR-MM       &  GAUSS         &  LIE          &  OPT          &  AGR-MM      \\ 
\hline
FedAvg      & $0.18$ & $0.11$ & $0.21$ & $0.11$ & $0.24$ & $0.10$ & $0.19$ & $0.10$ & $0.10$ & $0.10$ & $0.11$ & $0.10$  \\
+ FLANDERS  & ${\bf 0.75}$ & ${\bf 0.84}$ & ${\bf 0.84}$ & ${\bf 0.82}$ & ${\bf 0.68}$ & ${\bf 0.70}$ & ${\bf 0.66}$ & ${\bf 0.66}$ & ${\bf 0.33}$ & ${\bf 0.32}$ & ${\bf 0.32}$ & ${\bf 0.32}$  \\
\hline
FedMedian   & $0.34$ & $0.19$ & $0.13$ & $0.23$ & $0.29$ & $0.10$ & $0.17$ & $0.10$ & $0.13$ & $0.10$ & $0.10$ & $0.12$  \\
+ FLANDERS  & ${\bf 0.81}$ & ${\bf 0.84}$ & ${\bf 0.85}$ & ${\bf 0.81}$ & ${\bf 0.70}$ & ${\bf 0.71}$ & ${\bf 0.71}$ & ${\bf 0.68}$ & ${\bf 0.29}$ & ${\bf 0.29}$ & ${\bf 0.31}$ & ${\bf 0.28}$  \\
\hline
TrimmedMean & $0.15$ & $0.13$ & $0.20$ & $0.18$ & $0.21$ & $0.10$ & $0.23$ & $0.10$ & $0.10$ & $0.10$ & $0.10$ & $0.10$ \\
+ FLANDERS  & ${\bf 0.81}$ & ${\bf 0.83}$ & ${\bf 0.83}$ & ${\bf 0.85}$ & ${\bf 0.71}$ & ${\bf 0.70}$ & ${\bf 0.71}$ & ${\bf 0.69}$ & ${\bf 0.30}$ & ${\bf 0.29}$ & ${\bf 0.30}$ & ${\bf 0.29}$ \\
\hline
Multi-Krum  & $0.80$ & $0.12$ & $0.17$ & $0.25$ & $0.64$ & $0.10$ & $0.20$ & $0.10$ & $0.34$ & $0.10$ & $0.10$ & $0.10$ \\
+ FLANDERS  & ${\bf 0.87}$ & ${\bf 0.90}$ & ${\bf 0.88}$ & ${\bf 0.89}$ & ${\bf 0.69}$ & ${\bf 0.68}$ & ${\bf 0.72}$ & ${\bf 0.68}$ & ${\bf 0.38}$ & ${\bf 0.38}$ & ${\bf 0.39}$ & ${\bf 0.40}$ \\
\hline
Bulyan      & N/A & N/A & N/A & N/A & N/A & N/A & N/A & N/A & N/A & N/A & N/A & N/A \\
+ FLANDERS  & ${\bf 0.89}$ & ${\bf 0.85}$ & ${\bf 0.88}$ & ${\bf 0.82}$ & ${\bf 0.68}$ & ${\bf 0.66}$ & $0{\bf .68}$ & ${\bf 0.70}$ & ${\bf 0.40}$ & ${\bf 0.43}$ & ${\bf 0.40}$ & ${\bf 0.41}$ \\
\hline
DnC         & $0.21$ & $0.11$ & $0.17$ & $0.11$ & $0.25$ & $0.10$ & $0.14$ & $0.10$ & $0.10$ &$ 0.10$ & $0.10$ & $0.10$ \\
+ FLANDERS  & ${\bf 0.85}$ & ${\bf 0.87}$ & ${\bf 0.89}$ & ${\bf 0.87}$ & ${\bf 0.71}$ & ${\bf 0.69}$ & ${\bf 0.68}$ & ${\bf 0.68}$ & ${\bf 0.41}$ & ${\bf 0.40}$ & ${\bf 0.39}$ & ${\bf 0.40}$ \\
\hline
\end{tabular}
}
\end{table*}

For weaker attacks, e.g., $r=0.2$, FLANDERS still generally improves the accuracy of the global model, except when combined with Bulyan, which alone appears already robust enough to counter these mild attacks (see Tab.~\ref{tab:increment-20-mnist-fmnist} and \ref{tab:increment-20-cifar10-cifar100} in Appendix~\ref{app:agg-lift}). The complete results are in Appendix~\ref{app:agg-lift}.
\\
\noindent{\textbf{\textit{Cost-Benefit Analysis}.}}
Obviously, the robustness guaranteed by FLANDERS under extreme attack scenarios comes with costs, especially due to the MAR estimation stage. 
Fig.~\ref{fig:tradeoff} of Appendix~\ref{app:cost-benefit} depicts two scatter plots for the \textit{MNIST} and \textit{CIFAR-10} datasets, focusing on a specific attack scenario (AGR-MM). Each data point on a scatter plot represents a method under one of two attack strengths considered ($r=0.2$ and $r=0.6$). These data points are specified by two coordinates: the overall training time on the $x$-axis and the maximum accuracy of the global model on the $y$-axis.
\\
Overall, the take-home message is as follows. In scenarios with low attack strength ($r=0.2$), Bulyan demonstrates superior accuracy, while FLANDERS + FedAvg offers comparable performance with notably shorter training times. However, as the attack strength increases ($r=0.6$), Bulyan becomes impractical, FedAvg alone proves ineffective, and FLANDERS emerges as the optimal choice for achieving the best accuracy vs. cost trade-off.
\\
\noindent{\textbf{\textit{Robustness against Adaptive Attacks}.}}
We test the robustness of our method when malicious clients might exploit the knowledge that the FL server implements our FLANDERS filter. We consider two scenarios: one more plausible and the other most pessimistic. For the first setup, Table~\ref{tab:nonomn-adaptive} in Appendix~\ref{app:adaptive-attacks} shows that FLANDERS can cope with adaptive attackers who tentatively guess the subset of parameters used by the FL server to estimate the MAR forecasting model (\textit{non-omniscient} adaptive attack). In contrast, Table~\ref{tab:omn-adaptive} highlights that FLANDERS becomes unsurprisingly ineffective in the worst-case scenario, i.e., if malicious clients know \textit{exactly} which parameters are used by the FL server (\textit{omniscient} adaptive attack). However, this latter situation is quite unrealistic in practice.


% CONCLUSION & FUTURE WORK
\section{CONCLUSION}
\label{sec:conclusion}
We introduced FLANDERS, a novel FL filter robust to extreme untargeted model poisoning attacks, i.e., when malicious clients far exceed legitimate participants. 
FLANDERS serves as a pre-processing step before applying aggregation rules, enhancing robustness across diverse hostile settings.
\\
FLANDERS treats the sequence of local model updates sent by clients in each FL round as a matrix-valued time series. 
Then, it identifies malicious client updates as outliers in this time series using a matrix autoregressive forecasting model.
\\
Experiments conducted in several non-iid FL setups demonstrated that existing (secure) aggregation methods further improve their robustness when paired with FLANDERS. 
Moreover, FLANDERS allows these methods to operate even under extremely severe attack scenarios thanks to its ability to accurately filter out every malicious client \textit{before} the aggregation process takes place.
\\
In the future, we will address the primary limitations of this work, as discussed in Appendix~\ref{app:limitations}.




%%
%% The acknowledgments section is defined using the "acks" environment
%% (and NOT an unnumbered section). This ensures the proper
%% identification of the section in the article metadata, and the
%% consistent spelling of the heading.
% \begin{acks}
% To Robert, for the bagels and explaining CMYK and color spaces.
% \end{acks}

%%
%% If your work has an appendix, this is the place to put it.

%%
%% The next two lines define the bibliography style to be used, and
%% the bibliography file.
\bibliographystyle{plain}

\begin{thebibliography}{10}

\bibitem{mnist-ds}
{MNIST Dataset}.
\newblock [Online]. Available from: \url{http://yann.lecun.com/exdb/mnist/},
  1998.

\bibitem{cifar10-100-ds}
{CIFAR-10/CIFAR-100 Datasets}.
\newblock [Online]. Available from:
  \url{https://www.cs.toronto.edu/~kriz/cifar.html}, 2009.

\bibitem{pytorch-cifar-10}
{Training a Classifier}.
\newblock [Online]. Available from:
  \url{https://pytorch.org/tutorials/beginner/blitz/cifar10_tutorial.html},
  2023.

\bibitem{aggarwal2001icdt}
Charu~C Aggarwal, Alexander Hinneburg, and Daniel~A Keim.
\newblock {On the Surprising Behavior of Distance Metrics in High Dimensional
  Space}.
\newblock In {\em Proc. of ICDT '01}, pages 420--434. Springer, 2001.

\bibitem{albers2012chaos}
David~J Albers and George Hripcsak.
\newblock Using time-delayed mutual information to discover and interpret
  temporal correlation structure in complex populations.
\newblock {\em {Chaos: An Interdisciplinary Journal of Nonlinear Science}},
  22(1), 2012.

\bibitem{bagdasaryan2020pmlr}
Eugene Bagdasaryan, Andreas Veit, Yiqing Hua, Deborah Estrin, and Vitaly
  Shmatikov.
\newblock {How to Backdoor Federated Learning}.
\newblock In {\em Proc. of AISTATS '20}, volume 108, pages 2938--2948, virtual
  event, 2020. PMLR.

\bibitem{baruch2019neurips}
Gilad Baruch, Moran Baruch, and Yoav Goldberg.
\newblock {A Little Is Enough: Circumventing Defenses for Distributed
  Learning}.
\newblock In {\em Proc. of NeurIPS '19}, pages 8632--8642, 2019.

\bibitem{beutel2022flower}
Daniel~J. Beutel, Taner Topal, Akhil Mathur, Xinchi Qiu, Javier
  Fernandez-Marques, Yan Gao, Lorenzo Sani, Kwing~Hei Li, Titouan Parcollet,
  Pedro Porto~Buarque de~Gusmão, and Nicholas~D. Lane.
\newblock Flower: A friendly federated learning research framework, 2022.

\bibitem{bhagoji2019pmlr}
Arjun~Nitin Bhagoji, Supriyo Chakraborty, Prateek Mittal, and Seraphin Calo.
\newblock {Analyzing Federated Learning through an Adversarial Lens}.
\newblock In {\em Proc. of ICML '19}, pages 634--643. PMLR Press, 2019.

\bibitem{biggio2013icb}
Battista Biggio, Luca Didaci, Giorgio Fumera, and Fabio Roli.
\newblock {Poisoning Attacks to Compromise Face Templates}.
\newblock In {\em Proc of ICB '13}, pages 1--7. {IEEE}, 2013.

\bibitem{biggio2012icml}
Battista Biggio, Blaine Nelson, and Pavel Laskov.
\newblock {Poisoning Attacks against Support Vector Machines}.
\newblock In {\em Proc. of ICML '12}, pages 1467--1474. Omnipress, 2012.

\bibitem{billingsley2017probability}
Patrick Billingsley.
\newblock {\em Probability and Measure}.
\newblock John Wiley \& Sons, 2017.

\bibitem{blanchard2017nips}
Peva Blanchard, El~Mahdi El~Mhamdi, Rachid Guerraoui, and Julien Stainer.
\newblock {Machine Learning with Adversaries: Byzantine Tolerant Gradient
  Descent}.
\newblock In {\em Proc. of NeurIPS '17}, pages 118--128. Curran Associates
  Inc., 2017.

\bibitem{cao2020fltrust}
Xiaoyu Cao, Minghong Fang, Jia Liu, and Neil~Zhenqiang Gong.
\newblock {FLTrust: Byzantine-Robust Federated Learning via Trust
  Bootstrapping}.
\newblock {\em arXiv preprint arXiv:2012.13995}, abs/2012.13995, 2020.

\bibitem{chen2021je}
Rong Chen, Han Xiao, and Dan Yang.
\newblock {Autoregressive Models for Matrix-Valued Time Series}.
\newblock {\em Journal of Econometrics}, 222(1):539--560, 2021.

\bibitem{coppersmith1990jsc}
Don Coppersmith and Shmuel Winograd.
\newblock {Matrix Multiplication via Arithmetic Progressions}.
\newblock {\em Journal of Symbolic Computation}, 9(3):251--280, 1990.
\newblock Computational algebraic complexity editorial.

\bibitem{costa2022covert}
Gabriele Costa, Fabio Pinelli, Simone Soderi, and Gabriele Tolomei.
\newblock {Turning Federated Learning Systems into Covert Channels}.
\newblock {\em IEEE Access}, 10:130642--130656, 2022.

\bibitem{dean2012nips}
Jeffrey Dean, Greg Corrado, Rajat Monga, Kai Chen, Matthieu Devin, Mark Mao,
  Marc'aurelio Ranzato, Andrew Senior, Paul Tucker, Ke~Yang, et~al.
\newblock {Large Scale Distributed Deep Networks}.
\newblock In {\em Proc. of NeurIPS '12}, pages 1223--1231, 2012.

\bibitem{diakonikolas2017icml}
Ilias Diakonikolas, Gautam Kamath, Daniel~M. Kane, Jerry Li, Ankur Moitra, and
  Alistair Stewart.
\newblock {Being Robust (in High Dimensions) Can Be Practical}.
\newblock In {\em Proc. of ICML '17}, pages 999--1008. JMLR.org, 2017.

\bibitem{douceur2002iptps}
John~R. Douceur.
\newblock {The Sybil Attack}.
\newblock In {\em Proc. of IPTPS '02}, volume 2429 of {\em LNCS}, pages
  251--260. Springer, 2002.

\bibitem{fang2020usenix}
Minghong Fang, Xiaoyu Cao, Jinyuan Jia, and Neil Gong.
\newblock {Local Model Poisoning Attacks to Byzantine-Robust Federated
  Learning}.
\newblock In {\em Proc. of USENIX '20}, pages 1605--1622. USENIX Association,
  2020.

\bibitem{howard2017mobilenets}
Andrew~G. Howard, Menglong Zhu, Bo~Chen, Dmitry Kalenichenko, Weijun Wang,
  Tobias Weyand, Marco Andreetto, and Hartwig Adam.
\newblock Mobilenets: Efficient convolutional neural networks for mobile vision
  applications, 2017.

\bibitem{hsu2019LDA}
Tzu-Ming~Harry Hsu, Hang Qi, and Matthew Brown.
\newblock {Measuring the Effects of Non-Identical Data Distribution for
  Federated Visual Classification}, 2019.

\bibitem{hu2021arxiv}
Shengshan Hu, Jianrong Lu, Wei Wan, and Leo~Yu Zhang.
\newblock {Challenges and Approaches for Mitigating Byzantine Attacks in
  Federated Learning}.
\newblock {\em CoRR}, abs/2112.14468, 2021.

\bibitem{jagielski2018sp}
Matthew Jagielski, Alina Oprea, Battista Biggio, Chang Liu, Cristina
  Nita-Rotaru, and Bo~Li.
\newblock {Manipulating Machine Learning: Poisoning Attacks and Countermeasures
  for Regression Learning}.
\newblock In {\em Proc. of S\&P '18}, pages 19--35. IEEE, 2018.

\bibitem{konecny2016nipsws}
Jakub Kone{\v{c}}n{\'y}, H.~Brendan McMahan, Felix~X. Yu, Peter
  Richt{\'{a}}rik, Ananda~Theertha Suresh, and Dave Bacon.
\newblock {Federated Learning: Strategies for Improving Communication
  Efficiency}.
\newblock {\em CoRR}, abs/1610.05492, 2016.

\bibitem{koren2009ieeecomp}
Yehuda Koren, Robert Bell, and Chris Volinsky.
\newblock {Matrix Factorization Techniques for Recommender Systems}.
\newblock {\em Computer}, 42(8):30--37, Aug 2009.

\bibitem{li2016nips}
Bo~Li, Yining Wang, Aarti Singh, and Yevgeniy Vorobeychik.
\newblock {Data Poisoning Attacks on Factorization-Based Collaborative
  Filtering}.
\newblock In {\em Proc. of NeurIPS '16}, pages 1885--1893. Curran Associates,
  Inc., 2016.

\bibitem{li2022blades}
Shenghui Li, Li~Ju, Tianru Zhang, Edith Ngai, and Thiemo Voigt.
\newblock {Blades: A Simulator for Attacks and Defenses in Federated Learning},
  2022.

\bibitem{lu2020spml}
Yanyang Lu and Lei Fan.
\newblock {An Efficient and Robust Aggregation Algorithm for Learning Federated
  CNN}.
\newblock In {\em Proc. of SPML '20}, pages 1--7. ACM, 2020.

\bibitem{mcmahan2017aistats}
Brendan McMahan, Eider Moore, Daniel Ramage, Seth Hampson, and Blaise Aguera~y
  Arcas.
\newblock {Communication-Efficient Learning of Deep Networks from Decentralized
  Data}.
\newblock In {\em Proc. of AISTATS '17}, volume~54, pages 1273--1282. PMLR,
  2017.

\bibitem{mcmahan2017googleai}
Brendan McMahan and Daniel Ramage.
\newblock {Federated Learning: Collaborative Machine Learning without
  Centralized Training Data}.
\newblock {\em Google Research Blog}, 3, 2017.

\bibitem{mhamdi2018pmlr}
El~Mahdi~El Mhamdi, Rachid Guerraoui, and S{\'{e}}bastien Rouault.
\newblock {The Hidden Vulnerability of Distributed Learning in Byzantium}.
\newblock In Jennifer~G. Dy and Andreas Krause, editors, {\em Proc. of ICML
  '18}, volume~80, pages 3518--3527. {PMLR}, 2018.

\bibitem{barroso2023if}
Nuria Rodríguez-Barroso, Daniel Jiménez-López, M.~Victoria Luzón, Francisco
  Herrera, and Eugenio Martínez-Cámara.
\newblock {Survey on Federated Learning Threats: Concepts, Taxonomy on Attacks
  and Defences, Experimental Study and Challenges}.
\newblock {\em Information Fusion}, 90:148--173, 2023.

\bibitem{rubinstein2009imc}
Benjamin I.~P. Rubinstein, Blaine Nelson, Ling Huang, Anthony~D. Joseph,
  Shing{-}hon Lau, Satish Rao, Nina Taft, and J.~D. Tygar.
\newblock {ANTIDOTE: Understanding and Defending against Poisoning of Anomaly
  Detectors}.
\newblock In {\em Proc. of IMC '09}, pages 1--14. {ACM}, 2009.

\bibitem{shejwalkar2021ndss}
Virat Shejwalkar and Amir Houmansadr.
\newblock {Manipulating the Byzantine: Optimizing Model Poisoning Attacks and
  Defenses for Federated Learning}.
\newblock In {\em Proc. of NDSS '21}, 2021.

\bibitem{shen2016acm}
Shiqi Shen, Shruti Tople, and Prateek Saxena.
\newblock {Auror: Defending Against Poisoning Attacks in Collaborative Deep
  Learning Systems}.
\newblock In {\em Proc. of ACSAC'16}, pages 508--519. ACM, 2016.

\bibitem{fashionmnist-ds}
Han Xiao, Kashif Rasul, and Roland Vollgraf.
\newblock {Fashion-MNIST: A Novel Image Dataset for Benchmarking Machine
  Learning Algorithms}.
\newblock [Online]. Available from:
  \url{https://github.com/zalandoresearch/fashion-mnist}, 2017.

\bibitem{xiao2015icml}
Huang Xiao, Battista Biggio, Gavin Brown, Giorgio Fumera, Claudia Eckert, and
  Fabio Roli.
\newblock {Is Feature Selection Secure against Training Data Poisoning?}
\newblock In {\em Proc. of ICML '15}, volume~37, pages 1689--1698. JMLR.org,
  2015.

\bibitem{xie2018corr}
Cong Xie, Oluwasanmi Koyejo, and Indranil Gupta.
\newblock {Generalized Byzantine-Tolerant SGD}.
\newblock {\em CoRR}, abs/1802.10116, 2018.

\bibitem{yang2017ndss}
Guolei Yang, Neil~Zhenqiang Gong, and Ying Cai.
\newblock {Fake Co-Visitation Injection Attacks to Recommender Systems}.
\newblock In {\em Proc. of NDSS '17}. The Internet Society, 2017.

\bibitem{yin2018icml}
Dong Yin, Yudong Chen, Ramchandran Kannan, and Peter Bartlett.
\newblock {Byzantine-Robust Distributed Learning: Towards Optimal Statistical
  Rates}.
\newblock In {\em Proc. of ICML '18}, volume~80, pages 5650--5659. PMLR, 2018.

\bibitem{zhang2022fldetector}
Zaixi Zhang, Xiaoyu Cao, Jinyuan Jia, and Neil~Zhenqiang Gong.
\newblock {FLDetector: Defending Federated Learning Against Model Poisoning
  Attacks via Detecting Malicious Clients}.
\newblock In {\em Proc. of KDD '22}, pages 2545--2555. ACM, 2022.

\bibitem{zhao2020multivariate}
Hang Zhao, Yujing Wang, Juanyong Duan, Congrui Huang, Defu Cao, Yunhai Tong,
  Bixiong Xu, Jing Bai, Jie Tong, and Qi~Zhang.
\newblock {Multivariate Time-Series Anomaly Detection via Graph Attention
  Network}.
\newblock In {\em Proc. of ICDM '20}, pages 841--850, 2020.

\end{thebibliography}


% \newpage
% \section*{NeurIPS Paper Checklist}

% %%% BEGIN INSTRUCTIONS %%%
% % The checklist is designed to encourage best practices for responsible machine learning research, addressing issues of reproducibility, transparency, research ethics, and societal impact. Do not remove the checklist: {\bf The papers not including the checklist will be desk rejected.} The checklist should follow the references and precede the (optional) supplemental material.  The checklist does NOT count towards the page
% % limit. 

% % Please read the checklist guidelines carefully for information on how to answer these questions. For each question in the checklist:
% % \begin{itemize}
% %     \item You should answer \answerYes{}, \answerNo{}, or \answerNA{}.
% %     \item \answerNA{} means either that the question is Not Applicable for that particular paper or the relevant information is Not Available.
% %     \item Please provide a short (1–2 sentence) justification right after your answer (even for NA). 
% %    % \item {\bf The papers not including the checklist will be desk rejected.}
% % \end{itemize}

% % {\bf The checklist answers are an integral part of your paper submission.} They are visible to the reviewers, area chairs, senior area chairs, and ethics reviewers. You will be asked to also include it (after eventual revisions) with the final version of your paper, and its final version will be published with the paper.

% % The reviewers of your paper will be asked to use the checklist as one of the factors in their evaluation. While "\answerYes{}" is generally preferable to "\answerNo{}", it is perfectly acceptable to answer "\answerNo{}" provided a proper justification is given (e.g., "error bars are not reported because it would be too computationally expensive" or "we were unable to find the license for the dataset we used"). In general, answering "\answerNo{}" or "\answerNA{}" is not grounds for rejection. While the questions are phrased in a binary way, we acknowledge that the true answer is often more nuanced, so please just use your best judgment and write a justification to elaborate. All supporting evidence can appear either in the main paper or the supplemental material, provided in appendix. If you answer \answerYes{} to a question, in the justification please point to the section(s) where related material for the question can be found.

% % IMPORTANT, please:
% % \begin{itemize}
% %     \item {\bf Delete this instruction block, but keep the section heading ``NeurIPS paper checklist"},
% %     \item  {\bf Keep the checklist subsection headings, questions/answers and guidelines below.}
% %     \item {\bf Do not modify the questions and only use the provided macros for your answers}.
% % \end{itemize} 
 

% %%% END INSTRUCTIONS %%%


% \begin{enumerate}

% \item {\bf Claims}
%     \item[] Question: Do the main claims made in the abstract and introduction accurately reflect the paper's contributions and scope?
%     \item[] Answer: \answerYes{} % Replace by \answerYes{}, \answerNo{}, or \answerNA{}.
%     \item[] Justification: The primary claim of this work, highlighting the capability of our method (FLANDERS) to protect FL systems under \textit{extremely severe} attack settings, is supported by the experimental findings extensively discussed both in the main body and in the supplemental material. Furthermore, the assumption that the sequence of local models sent to the FL server by legitimate clients is more predictable than those sent by malicious ones has been statistically validated through a dedicated experiment, which is sufficiently general to support broader conclusions.
%     \item[] Guidelines:
%     \begin{itemize}
%         \item The answer NA means that the abstract and introduction do not include the claims made in the paper.
%         \item The abstract and/or introduction should clearly state the claims made, including the contributions made in the paper and important assumptions and limitations. A No or NA answer to this question will not be perceived well by the reviewers. 
%         \item The claims made should match theoretical and experimental results, and reflect how much the results can be expected to generalize to other settings. 
%         \item It is fine to include aspirational goals as motivation as long as it is clear that these goals are not attained by the paper. 
%     \end{itemize}

% \item {\bf Limitations}
%     \item[] Question: Does the paper discuss the limitations of the work performed by the authors?
%     \item[] Answer: \answerYes{} % Replace by \answerYes{}, \answerNo{}, or \answerNA{}.
%     \item[] Justification: We have discussed some limitations due to the computational cost of our proposed method, and have suggested possible techniques to enhance its feasibility in practice (see Section~\ref{subsec:complexity} in the main body), inspired by previous work~\cite{shejwalkar2021ndss}. 
%     We have also detailed a cost-benefit analysis of our method both in the main body (see Section~\ref{subsec:eval}) and, more extensively, in the Appendix~\ref{app:cost-benefit}. 
%     Finally, we have included a dedicated Section ``Limitations and Future Work'' in Appendix~\ref{app:limitations}, where we further elaborate on the main limitations of our work, including efficiency/feasibility, privacy, and benchmarking.
%     \item[] Guidelines:
%     \begin{itemize}
%         \item The answer NA means that the paper has no limitation while the answer No means that the paper has limitations, but those are not discussed in the paper. 
%         \item The authors are encouraged to create a separate "Limitations" section in their paper.
%         \item The paper should point out any strong assumptions and how robust the results are to violations of these assumptions (e.g., independence assumptions, noiseless settings, model well-specification, asymptotic approximations only holding locally). The authors should reflect on how these assumptions might be violated in practice and what the implications would be.
%         \item The authors should reflect on the scope of the claims made, e.g., if the approach was only tested on a few datasets or with a few runs. In general, empirical results often depend on implicit assumptions, which should be articulated.
%         \item The authors should reflect on the factors that influence the performance of the approach. For example, a facial recognition algorithm may perform poorly when image resolution is low or images are taken in low lighting. Or a speech-to-text system might not be used reliably to provide closed captions for online lectures because it fails to handle technical jargon.
%         \item The authors should discuss the computational efficiency of the proposed algorithms and how they scale with dataset size.
%         \item If applicable, the authors should discuss possible limitations of their approach to address problems of privacy and fairness.
%         \item While the authors might fear that complete honesty about limitations might be used by reviewers as grounds for rejection, a worse outcome might be that reviewers discover limitations that aren't acknowledged in the paper. The authors should use their best judgment and recognize that individual actions in favor of transparency play an important role in developing norms that preserve the integrity of the community. Reviewers will be specifically instructed to not penalize honesty concerning limitations.
%     \end{itemize}

% \item {\bf Theory Assumptions and Proofs}
%     \item[] Question: For each theoretical result, does the paper provide the full set of assumptions and a complete (and correct) proof?
%     \item[] Answer: \answerNA{} % Replace by \answerYes{}, \answerNo{}, or \answerNA{}.
%     \item[] Justification: FLANDERS leverages the theoretical foundations of the matrix autoregressive forecasting model (MAR) at its core, proposed by~\cite{chen2021je}. Furthermore, empirical evidence has statistically validated the claim regarding the higher predictability of legitimate models compared to malicious ones.
%     \item[] Guidelines:
%     \begin{itemize}
%         \item The answer NA means that the paper does not include theoretical results. 
%         \item All the theorems, formulas, and proofs in the paper should be numbered and cross-referenced.
%         \item All assumptions should be clearly stated or referenced in the statement of any theorems.
%         \item The proofs can either appear in the main paper or the supplemental material, but if they appear in the supplemental material, the authors are encouraged to provide a short proof sketch to provide intuition. 
%         \item Inversely, any informal proof provided in the core of the paper should be complemented by formal proofs provided in appendix or supplemental material.
%         \item Theorems and Lemmas that the proof relies upon should be properly referenced. 
%     \end{itemize}

%     \item {\bf Experimental Result Reproducibility}
%     \item[] Question: Does the paper fully disclose all the information needed to reproduce the main experimental results of the paper to the extent that it affects the main claims and/or conclusions of the paper (regardless of whether the code and data are provided or not)?
%     \item[] Answer: \answerYes % Replace by \answerYes{}, \answerNo{}, or \answerNA{}.
%     \item[] Justification: We have provided all the details to fully reproduce our experiments in the Appendix, and have released documented source code of our FLANDERS implementation at \url{https://anonymous.4open.science/r/flanders_exp-7EEB}.
%     \item[] Guidelines:
%     \begin{itemize}
%         \item The answer NA means that the paper does not include experiments.
%         \item If the paper includes experiments, a No answer to this question will not be perceived well by the reviewers: Making the paper reproducible is important, regardless of whether the code and data are provided or not.
%         \item If the contribution is a dataset and/or model, the authors should describe the steps taken to make their results reproducible or verifiable. 
%         \item Depending on the contribution, reproducibility can be accomplished in various ways. For example, if the contribution is a novel architecture, describing the architecture fully might suffice, or if the contribution is a specific model and empirical evaluation, it may be necessary to either make it possible for others to replicate the model with the same dataset, or provide access to the model. In general. releasing code and data is often one good way to accomplish this, but reproducibility can also be provided via detailed instructions for how to replicate the results, access to a hosted model (e.g., in the case of a large language model), releasing of a model checkpoint, or other means that are appropriate to the research performed.
%         \item While NeurIPS does not require releasing code, the conference does require all submissions to provide some reasonable avenue for reproducibility, which may depend on the nature of the contribution. For example
%         \begin{enumerate}
%             \item If the contribution is primarily a new algorithm, the paper should make it clear how to reproduce that algorithm.
%             \item If the contribution is primarily a new model architecture, the paper should describe the architecture clearly and fully.
%             \item If the contribution is a new model (e.g., a large language model), then there should either be a way to access this model for reproducing the results or a way to reproduce the model (e.g., with an open-source dataset or instructions for how to construct the dataset).
%             \item We recognize that reproducibility may be tricky in some cases, in which case authors are welcome to describe the particular way they provide for reproducibility. In the case of closed-source models, it may be that access to the model is limited in some way (e.g., to registered users), but it should be possible for other researchers to have some path to reproducing or verifying the results.
%         \end{enumerate}
%     \end{itemize}


% \item {\bf Open access to data and code}
%     \item[] Question: Does the paper provide open access to the data and code, with sufficient instructions to faithfully reproduce the main experimental results, as described in supplemental material?
%     \item[] Answer: \answerYes{} % Replace by \answerYes{}, \answerNo{}, or \answerNA{}.
%     \item[] Justification: We have provided a link to the anonymous repository (\url{https://anonymous.4open.science/r/flanders_exp-7EEB}), along with detailed instructions for running the code. The public datasets required are automatically downloaded as part of the code execution.
%     \item[] Guidelines:
%     \begin{itemize}
%         \item The answer NA means that paper does not include experiments requiring code.
%         \item Please see the NeurIPS code and data submission guidelines (\url{https://nips.cc/public/guides/CodeSubmissionPolicy}) for more details.
%         \item While we encourage the release of code and data, we understand that this might not be possible, so “No” is an acceptable answer. Papers cannot be rejected simply for not including code, unless this is central to the contribution (e.g., for a new open-source benchmark).
%         \item The instructions should contain the exact command and environment needed to run to reproduce the results. See the NeurIPS code and data submission guidelines (\url{https://nips.cc/public/guides/CodeSubmissionPolicy}) for more details.
%         \item The authors should provide instructions on data access and preparation, including how to access the raw data, preprocessed data, intermediate data, and generated data, etc.
%         \item The authors should provide scripts to reproduce all experimental results for the new proposed method and baselines. If only a subset of experiments are reproducible, they should state which ones are omitted from the script and why.
%         \item At submission time, to preserve anonymity, the authors should release anonymized versions (if applicable).
%         \item Providing as much information as possible in supplemental material (appended to the paper) is recommended, but including URLs to data and code is permitted.
%     \end{itemize}


% \item {\bf Experimental Setting/Details}
%     \item[] Question: Does the paper specify all the training and test details (e.g., data splits, hyperparameters, how they were chosen, type of optimizer, etc.) necessary to understand the results?
%     \item[] Answer: \answerYes{} % Replace by \answerYes{}, \answerNo{}, or \answerNA{}.
%     \item[] Justification: We have included all the details of our experiments in Appendix~\ref{app:setup}, \ref{app:fl-env}, \ref{app:attacks}, and \ref{app:defenses}.
%     \item[] Guidelines:
%     \begin{itemize}
%         \item The answer NA means that the paper does not include experiments.
%         \item The experimental setting should be presented in the core of the paper to a level of detail that is necessary to appreciate the results and make sense of them.
%         \item The full details can be provided either with the code, in appendix, or as supplemental material.
%     \end{itemize}

% \item {\bf Experiment Statistical Significance}
%     \item[] Question: Does the paper report error bars suitably and correctly defined or other appropriate information about the statistical significance of the experiments?
%     \item[] Answer: \answerNo{} % Replace by \answerYes{}, \answerNo{}, or \answerNA{}.
%     \item[] Justification: To report the average/median and min/max accuracy of the global model, we would have to run our experiments multiple times. Unfortunately, this is computationally infeasible. Indeed, given the massive number of combinations of datasets, attacks, attack power, and aggregation methods considered, this work already counts 1243 experiments involving an FL system with 100 clients running for 50 rounds. 
%     However, to validate our claim regarding the higher predictability of local models sent by legitimate clients, we have indeed conducted an appropriate statistical hypothesis test that reports the statistical significance of our findings (see Appendix~\ref{app:legit-vs-malicious}).
%     \item[] Guidelines:
%     \begin{itemize}
%         \item The answer NA means that the paper does not include experiments.
%         \item The authors should answer "Yes" if the results are accompanied by error bars, confidence intervals, or statistical significance tests, at least for the experiments that support the main claims of the paper.
%         \item The factors of variability that the error bars are capturing should be clearly stated (for example, train/test split, initialization, random drawing of some parameter, or overall run with given experimental conditions).
%         \item The method for calculating the error bars should be explained (closed form formula, call to a library function, bootstrap, etc.)
%         \item The assumptions made should be given (e.g., Normally distributed errors).
%         \item It should be clear whether the error bar is the standard deviation or the standard error of the mean.
%         \item It is OK to report 1-sigma error bars, but one should state it. The authors should preferably report a 2-sigma error bar than state that they have a 96\% CI, if the hypothesis of Normality of errors is not verified.
%         \item For asymmetric distributions, the authors should be careful not to show in tables or figures symmetric error bars that would yield results that are out of range (e.g. negative error rates).
%         \item If error bars are reported in tables or plots, The authors should explain in the text how they were calculated and reference the corresponding figures or tables in the text.
%     \end{itemize}

% \item {\bf Experiments Compute Resources}
%     \item[] Question: For each experiment, does the paper provide sufficient information on the computer resources (type of compute workers, memory, time of execution) needed to reproduce the experiments?
%     \item[] Answer: \answerYes{} % Replace by \answerYes{}, \answerNo{}, or \answerNA{}.
%     \item[] Justification: In Appendix~\ref{app:setup}, we have detailed the main characteristics of the machine on which we have run our experiments.
%     \item[] Guidelines:
%     \begin{itemize}
%         \item The answer NA means that the paper does not include experiments.
%         \item The paper should indicate the type of compute workers CPU or GPU, internal cluster, or cloud provider, including relevant memory and storage.
%         \item The paper should provide the amount of compute required for each of the individual experimental runs as well as estimate the total compute. 
%         \item The paper should disclose whether the full research project required more compute than the experiments reported in the paper (e.g., preliminary or failed experiments that didn't make it into the paper). 
%     \end{itemize}
    
% \item {\bf Code Of Ethics}
%     \item[] Question: Does the research conducted in the paper conform, in every respect, with the NeurIPS Code of Ethics \url{https://neurips.cc/public/EthicsGuidelines}?
%     \item[] Answer: \answerYes{} % Replace by \answerYes{}, \answerNo{}, or \answerNA{}.
%     \item[] Justification: Our work complies with the Code of Ethics.
%     \item[] Guidelines:
%     \begin{itemize}
%         \item The answer NA means that the authors have not reviewed the NeurIPS Code of Ethics.
%         \item If the authors answer No, they should explain the special circumstances that require a deviation from the Code of Ethics.
%         \item The authors should make sure to preserve anonymity (e.g., if there is a special consideration due to laws or regulations in their jurisdiction).
%     \end{itemize}


% \item {\bf Broader Impacts}
%     \item[] Question: Does the paper discuss both potential positive societal impacts and negative societal impacts of the work performed?
%     \item[] Answer: \answerYes{} % Replace by \answerYes{}, \answerNo{}, or \answerNA{}.
%     \item[] Justification: We have discussed some privacy considerations in Appendix~\ref{app:limitations-privacy}.
%     \item[] Guidelines:
%     \begin{itemize}
%         \item The answer NA means that there is no societal impact of the work performed.
%         \item If the authors answer NA or No, they should explain why their work has no societal impact or why the paper does not address societal impact.
%         \item Examples of negative societal impacts include potential malicious or unintended uses (e.g., disinformation, generating fake profiles, surveillance), fairness considerations (e.g., deployment of technologies that could make decisions that unfairly impact specific groups), privacy considerations, and security considerations.
%         \item The conference expects that many papers will be foundational research and not tied to particular applications, let alone deployments. However, if there is a direct path to any negative applications, the authors should point it out. For example, it is legitimate to point out that an improvement in the quality of generative models could be used to generate deepfakes for disinformation. On the other hand, it is not needed to point out that a generic algorithm for optimizing neural networks could enable people to train models that generate Deepfakes faster.
%         \item The authors should consider possible harms that could arise when the technology is being used as intended and functioning correctly, harms that could arise when the technology is being used as intended but gives incorrect results, and harms following from (intentional or unintentional) misuse of the technology.
%         \item If there are negative societal impacts, the authors could also discuss possible mitigation strategies (e.g., gated release of models, providing defenses in addition to attacks, mechanisms for monitoring misuse, mechanisms to monitor how a system learns from feedback over time, improving the efficiency and accessibility of ML).
%     \end{itemize}
    
% \item {\bf Safeguards}
%     \item[] Question: Does the paper describe safeguards that have been put in place for responsible release of data or models that have a high risk for misuse (e.g., pretrained language models, image generators, or scraped datasets)?
%     \item[] Answer: \answerNA{} % Replace by \answerYes{}, \answerNo{}, or \answerNA{}.
%     \item[] Justification: We have not released any new model.
%     \item[] Guidelines:
%     \begin{itemize}
%         \item The answer NA means that the paper poses no such risks.
%         \item Released models that have a high risk for misuse or dual-use should be released with necessary safeguards to allow for controlled use of the model, for example by requiring that users adhere to usage guidelines or restrictions to access the model or implementing safety filters. 
%         \item Datasets that have been scraped from the Internet could pose safety risks. The authors should describe how they avoided releasing unsafe images.
%         \item We recognize that providing effective safeguards is challenging, and many papers do not require this, but we encourage authors to take this into account and make a best faith effort.
%     \end{itemize}

% \item {\bf Licenses for existing assets}
%     \item[] Question: Are the creators or original owners of assets (e.g., code, data, models), used in the paper, properly credited and are the license and terms of use explicitly mentioned and properly respected?
%     \item[] Answer: \answerYes{} % Replace by \answerYes{}, \answerNo{}, or \answerNA{}.
%     \item[] Justification: We have properly cited all assets that we have used.
%     \item[] Guidelines:
%     \begin{itemize}
%         \item The answer NA means that the paper does not use existing assets.
%         \item The authors should cite the original paper that produced the code package or dataset.
%         \item The authors should state which version of the asset is used and, if possible, include a URL.
%         \item The name of the license (e.g., CC-BY 4.0) should be included for each asset.
%         \item For scraped data from a particular source (e.g., website), the copyright and terms of service of that source should be provided.
%         \item If assets are released, the license, copyright information, and terms of use in the package should be provided. For popular datasets, \url{paperswithcode.com/datasets} has curated licenses for some datasets. Their licensing guide can help determine the license of a dataset.
%         \item For existing datasets that are re-packaged, both the original license and the license of the derived asset (if it has changed) should be provided.
%         \item If this information is not available online, the authors are encouraged to reach out to the asset's creators.
%     \end{itemize}

% \item {\bf New Assets}
%     \item[] Question: Are new assets introduced in the paper well documented and is the documentation provided alongside the assets?
%     \item[] Answer: \answerYes{} % Replace by \answerYes{}, \answerNo{}, or \answerNA{}.
%     \item[] Justification: We have included an anonymous repository available at \url{https://anonymous.4open.science/r/flanders_exp-7EEB}, along with detailed documentation.
%     \item[] Guidelines:
%     \begin{itemize}
%         \item The answer NA means that the paper does not release new assets.
%         \item Researchers should communicate the details of the dataset/code/model as part of their submissions via structured templates. This includes details about training, license, limitations, etc. 
%         \item The paper should discuss whether and how consent was obtained from people whose asset is used.
%         \item At submission time, remember to anonymize your assets (if applicable). You can either create an anonymized URL or include an anonymized zip file.
%     \end{itemize}

% \item {\bf Crowdsourcing and Research with Human Subjects}
%     \item[] Question: For crowdsourcing experiments and research with human subjects, does the paper include the full text of instructions given to participants and screenshots, if applicable, as well as details about compensation (if any)? 
%     \item[] Answer: \answerNA{} % Replace by \answerYes{}, \answerNo{}, or \answerNA{}.
%     \item[] Justification: We have not made crowdsourcing experiments, nor have used human subjects.
%     \item[] Guidelines:
%     \begin{itemize}
%         \item The answer NA means that the paper does not involve crowdsourcing nor research with human subjects.
%         \item Including this information in the supplemental material is fine, but if the main contribution of the paper involves human subjects, then as much detail as possible should be included in the main paper. 
%         \item According to the NeurIPS Code of Ethics, workers involved in data collection, curation, or other labor should be paid at least the minimum wage in the country of the data collector. 
%     \end{itemize}

% \item {\bf Institutional Review Board (IRB) Approvals or Equivalent for Research with Human Subjects}
%     \item[] Question: Does the paper describe potential risks incurred by study participants, whether such risks were disclosed to the subjects, and whether Institutional Review Board (IRB) approvals (or an equivalent approval/review based on the requirements of your country or institution) were obtained?
%     \item[] Answer: \answerNA{} % Replace by \answerYes{}, \answerNo{}, or \answerNA{}.
%     \item[] Justification: We have not made crowdsourcing experiments, nor have used human subjects.
%     \item[] Guidelines:
%     \begin{itemize}
%         \item The answer NA means that the paper does not involve crowdsourcing nor research with human subjects.
%         \item Depending on the country in which research is conducted, IRB approval (or equivalent) may be required for any human subjects research. If you obtained IRB approval, you should clearly state this in the paper. 
%         \item We recognize that the procedures for this may vary significantly between institutions and locations, and we expect authors to adhere to the NeurIPS Code of Ethics and the guidelines for their institution. 
%         \item For initial submissions, do not include any information that would break anonymity (if applicable), such as the institution conducting the review.
%     \end{itemize}

% \end{enumerate}

\newpage
\appendix
\section*{Appendix}
\label{sec:appendix}
The (anonymous) GitHub repository with the code and the data to replicate the results discussed in this work is accessible at the following link: \url{https://anonymous.4open.science/r/flanders_exp-7EEB/}

%%% NOTE ON THE TERMINOLOGY
\section{A Note on the Terminology Used}
\label{app:terminology}
The type of model poisoning attacks we consider here are often referred to as \textit{Byzantine} attacks in the literature (\cite{blanchard2017nips,fang2020usenix,barroso2023if}). 
Although, in this work, we adhere to the taxonomy proposed by~\cite{barroso2023if}, the research community has yet to reach a unanimous consensus on the terminology. 
In fact, some authors use the word ``Byzantine'' as an umbrella term to broadly indicate \textit{any} attack involving malicious clients (e.g., targeted data poisoning like backdoor attacks \textit{and} untargeted model poisoning as in~\cite{hu2021arxiv}).
Therefore, to avoid confusion and hurting the feelings of some readers who have already debated on that and found the term inappropriate or disrespectful,\footnote{\small{\url{https://openreview.net/forum?id=pfuqQQCB34&noteId=5KAMwoI2cC}}} we have decided \textit{not} to use the word ``Byzantine'' to refer to our attack model.

%%% IMPACT OF MALICIOUS CLIENTS
\section{The Impact of Malicious Clients at each FL Round}
\label{app:impact}
Under our assumptions, the FL system contains $K$ clients, where $b$ of them are malicious and controlled by an attacker ($0\leq b\leq K$).
In addition, at each FL round, $m$ clients ($1 \leq m \leq K$) are selected, and thus, some of the $m$ model updates received by the central server may be corrupted.
The probability of this event can actually be computed by noticing that the outcome of the client selection at each round can be represented by a random variable $X\sim \text{Hypergeometric}(K, b, m)$, whose probability mass function is:
\[p_X(x) = \Pr(X=x) = \frac{\binom{b}{x} \binom{K-b}{m-x}}{\binom{K}{m}}.
\]
The chance that, at a single round, {\em at least one} of the $b$ malicious clients ends up in the list of $m$ clients randomly picked by the server is equal to:
\[\Pr(X\geq 1) = 1 - \Pr(X=0) = 1 - \frac{\binom{b}{0}\binom{K-b}{m-0}}{\binom{K}{m}}= 1 - \frac{ \binom{K-b}{m}}{\binom{K}{m}}.
\]
For example, if the total number of clients is $K=100$, $b=5$ of them are malicious, and $m=20$ must be drawn at each round, then $\Pr(X\geq 1) \approx 68\%$. 
In other words, there are about two out of three chances that at least one malicious client is selected at {\em every} FL round.

In our FL simulation environment, Flower, we can set a fixed proportion of malicious clients in the system (e.g., $20\%$). However, it is important to note that these clients may not remain constant across different FL rounds. In other words, a client who is selected in one round and acts legitimately could become malicious in another round of the FL process.

%%% INTUITIVE EXPLANATION OF FLANDERS
\section{The Intuition behind FLANDERS: Predictability of Legitimate vs. Malicious Local Models}
\label{app:legit-vs-malicious}

In this section, we elaborate further on our findings discussed in Section~\ref{subsec:intuition}.

We begin by considering again two clients $i,j \in \mathcal{C}$, where client $i$ is assumed to be a legitimate participant, while client $j$ acts maliciously. 
We assume that both clients are selected by the server over a sequence of consecutive $T$ FL rounds to train a global model on the \textit{MNIST} dataset.
Therefore, we examine the local models sent to the server at each round $t \in \{1,2,\ldots,T\}$ by clients $i$ and $j$.  
Specifically, these are $d$-dimensional vectors of real-valued parameters $\params_i^{(t)}, \params_j^{(t)} \in \R^d$, such that $\params_i^{(t)} = (\theta_{i,1}^{(t)}, \ldots, \theta_{i,d}^{(t)})$ and $\params_j^{(t)} = (\theta_{j,1}^{(t)}, \ldots, \theta_{j,d}^{(t)})$, respectively.

To mimic the behavior of a hypothetical optimal FLANDERS filter, and therefore avoid the propagation of poisoned local models sent by client $j$ across the $T$ rounds, we assume that, at each round $t$, the global model from which both clients start their local training process is polished from any malicious updates received at the previous round $t-1$.
Next, we calculate the average time-delayed mutual information (TDMI) for each pair of observed local models $(\params_i^{(t)}, \params_i^{(t')})$ and $(\params_j^{(t)}, \params_j^{(t')})$ across $T=50$ rounds, where $t' > t$, for both client $i$ and $j$.

Firstly, we consider the special case where $t'=t+1$ and compute the average TDMI between each pair of \textit{consecutive} local models sent by the legitimate client and the malicious client, when this runs one of the four attacks considered in this work, namely GAUSS, LIE, OPT, and AGR-MM. 
In Figure~\ref{fig:avg-tdmi-density}, we plot the empirical distributions of the observed average TDMI for the legitimate and malicious clients. 
\begin{figure*}[ht!]
    \centering
    \includegraphics[width=.6\textwidth]{./img/tdmi-density}
    \caption{Empirical distributions of average TDMI computed between each pair of \textit{consecutive} local models sent by the legitimate client $i$ $(\params_i^{(t)}, \params_i^{(t+1)})$ and the malicious client $j$ $(\params_j^{(t)}, \params_j^{(t+1)})$, when this runs one of the four attacks considered in this work, namely GAUSS, LIE, OPT, and AGR-MM.}
    \label{fig:avg-tdmi-density}
\end{figure*}

To validate our claim that legitimate models are more predictable than malicious ones, we compute the mean of the empirical distributions for the legitimate and malicious client, $\bar{\params_i}$ and $\bar{\params_j}^{atk}$, respectively, where $atk=\{$GAUSS, LIE, OPT, AGR-MM$\}$. Then, we run a one-tailed $t$-test against the null hypothesis $H_0: \bar{\params_i} = \bar{\params_j}^{atk}$, where the alternative hypothesis is $H_a: \bar{\params_i} > \bar{\params_j}^{atk}$. 
The results of these statistical tests are illustrated in Table~\ref{tab:t-test}, showing that, in all four cases, there is enough evidence to reject the null hypothesis at a confidence level $\alpha=0.01$.

\begin{table*}[htb!]
\centering
\caption{One-tailed $t$-test against the null hypothesis $H_0: \bar{\boldsymbol{\theta}_i} = \bar{\boldsymbol{\theta}_j}^{atk}$. Each cell contains the $p$-value for the statistical test corresponding to a specific attack. In all four cases, there is enough evidence to reject the null hypothesis at a confidence level $\alpha=0.01$ ($p$-value $\ll 0.01$).}
\label{tab:t-test}
\begin{tabular}{c|c|c|c|c|}
\cline{2-5}
& \multicolumn{4}{c|}{$atk$}\\
\cline{2-5}
& GAUSS & LIE & OPT & AGR-MM \\
\hline
\multicolumn{1}{|c|}{$H_0: \bar{\boldsymbol{\theta}_i} = \bar{\boldsymbol{\theta}_j}^{atk}$} & $5.67*10^{-38}$ & $8.33*10^{-5}$ & $6.70*10^{-18}$ & $4.20*10^{-12}$\\
\hline
\end{tabular}
\end{table*}

%%% MAR
\section{Matrix Autoregressive Model (MAR)}
\label{app:mar}
This work assumes the temporal evolution of the local models sent by clients at each FL round exhibits a bilinear structure captured by a \textit{matrix autoregressive model} of order $1$, i.e., a Markovian forecasting model denoted by MAR($1$) and defined as follows:
\[
{\bm \Params}_{t} = {\bm A}{\bm \Params}_{t-1}{\bm B} + {\bm E}_t,
\]
where ${\bm E}_t$ is a white noise matrix, i.e., its entries are iid normal with zero-mean and constant variance.
To approximate such a behavior, we consider a parametric forecasting model $f$, in the form:
\[
\widetilde{\bm{\Params}}_t = f(\Params_{t-1};\bm{\widetilde{\Omega}}) = \widetilde{{\bm A}} {\bm{\Params}}_{t-1}\widetilde{{\bm B}}\approx {\bm \Params}_{t},
\]
where $\widetilde{\bm{\Params}}_t$ is the {\em predicted} matrix of observations at time $t$ according to $f$ when parametrized by coefficient matrices $\bm{\widetilde{\Omega}} = \{\widetilde{{\bm A}},\widetilde{{\bm B}}\}$.
Thus, the key question is how to estimate the best model $f$, namely the best coefficient matrices $\hat{{\bm\Omega}} = \{\hat{{\bm A}},\hat{{\bm B}}\}$.
For starters, we define an instance-level loss function that measures the cost of approximating the true (yet unknown) data generation process with our model $f$ as follows:
\begin{equation}
\begin{split}
\loss(\bm{\widetilde{\Omega}};{\bm \Params}_{t}) = ||{\bm \Params}_{t} - \widetilde{{\bm \Params}}_{t}||^2_{\text F} = 
||{\bm \Params}_{t} - f(\Params_{t-1};\bm{\widetilde{\Omega}})||^2_{\text F} = 
||{\bm \Params}_{t} - \widetilde{{\bm A}}{\bm \Params}_{t-1}\widetilde{{\bm B}}||^2_{\text F} = 
\loss(\widetilde{{\bm A}},\widetilde{{\bm B}}; {\bm \Params}_{t}),
\end{split}
\label{eq:app-inst-loss}
\end{equation}
where $||\cdot||_{\text F}$ indicates the Frobenius norm of a matrix.
More generally, if we have access to $l > 0$ historical matrix observations, we can compute the overall loss function below:
\begin{equation}
\Loss(\widetilde{{\bm A}},\widetilde{{\bm B}}; {\bm \Params}_{t},l) = \sum_{j=0}^{l-1} \loss(\widetilde{{\bm A}},\widetilde{{\bm B}}; {\bm \Params}_{t-j}).
\label{eq:app-loss}
\end{equation}
Notice that $l$ here affects only the size of the training set \textit{not} the order of the autoregressive model. In other words, the forecasting model $f$ will still be MAR($1$) and not MAR($l$), i.e., the matrix of local updates at time $t$ (${\bm \Params}_{t}$) depends \textit{only} on the previously observed matrix at time step $t-1$ (${\bm \Params}_{t-1}$).

Eventually, the best estimates $\hat{\bm A}$ and $\hat{\bm B}$ can be found as the solutions to the following objective:
\begin{equation}
\label{eq:app-mar-opt}
\begin{split}
    \hat{{\bm\Omega}} = \hat{\bm A}, \hat{\bm B} = \argmin_{\widetilde{{\bm A}}, \widetilde{{\bm B}}}\Big\{\Loss(\widetilde{{\bm A}},\widetilde{{\bm B}}; {\bm \Params}_{t},l) \Big\} =  
    \argmin_{\widetilde{{\bm A}}, \widetilde{{\bm B}}}\Big\{\sum_{j=0}^{l-1} ||{\bm \Params}_{t-j} - \widetilde{{\bm A}}{\bm \Params}_{t-j-1}\widetilde{{\bm B}}||^2_{\text F} \Big\}.
\end{split}
\end{equation}
Before solving the optimization task defined in Eq.~(\ref{eq:app-mar-opt}) above, we replace $\widetilde{{\bm A}}$ with ${\bm A}$ and $\widetilde{{\bm B}}$ with ${\bm B}$ to ease the reading. Hence, we observe the following.
A closed-form solution to find $\hat{\bm A}$ can be computed by taking the partial derivative of the loss w.r.t. ${\bm A}$, setting it to $0$, and solving it for ${\bm A}$. In other words, we search for $\hat{\bm A}={\bm A}$, such that:
\begin{equation}
    \label{eq:app-partial-a}
\frac{\partial{\Loss({\bm A},{\bm B}; {\bm \Params}_{t},l)}}{\partial{{\bm A}}} = 0.
\end{equation}

Using Eq.~(\ref{eq:app-loss}) and Eq.~(\ref{eq:app-inst-loss}), the left-hand side of Eq.~(\ref{eq:app-partial-a}) can be rewritten as follows:
\begin{align}
\frac{\partial{\Loss({\bm A},{\bm B}; {\bm \Params}_{t},l)}}{\partial{{\bm A}}} & = \frac{\partial{\sum_{j=0}^{l-1} ||{\bm \Params}_{t-j} - {\bm A}{\bm \Params}_{t-j-1}{\bm B}||^2_{\text F}}}{\partial{{\bm A}}}=\nonumber\\
& = -2 \sum_{j=0}^{l-1}({\bm \Params}_{t-j} - {\bm A}{\bm \Params}_{t-j-1}{\bm B}^T){\bm B}{\bm \Params}^T_{t-1}=\nonumber\\
& = -2 \Bigg[\Bigg(\sum_{j=0}^{l-1} {\bm \Params}_{t-j}{\bm B}{\bm \Params}^T_{t-j-1}\Bigg) - {\bm A}\Bigg(\sum_{j=0}^{l-1}{\bm \Params}_{t-j-1}{\bm B}^T{\bm B}{\bm \Params}^T_{t-j-1}\Bigg)\Bigg].
\label{eq:app-partial}
\end{align}
If we set Eq.~(\ref{eq:app-partial}) to $0$ and solve it for ${\bm A}$, we find:
\[
-2 \Bigg[\Bigg(\sum_{j=0}^{l-1} {\bm \Params}_{t-j}{\bm B}{\bm \Params}^T_{t-j-1}\Bigg) - {\bm A}\Bigg(\sum_{j=0} ^{l-1}{\bm \Params}_{t-j-1}{\bm B}^T{\bm B}{\bm \Params}^T_{t-j-1}\Bigg)\Bigg] = 0 
\] \[ \iff \] \[
\Bigg(\sum_{j=0}^{l-1} {\bm \Params}_{t-j}{\bm B}{\bm \Params}^T_{t-j-1}\Bigg) - {\bm A}\Bigg(\sum_{j=0}^{l-1}{\bm \Params}_{t-j-1}{\bm B}^T{\bm B}{\bm \Params}^T_{t-j-1}\Bigg) = 0.
\]
Hence:

\begin{equation}
\begin{split}
\Bigg(\sum_{j=0}^{l-1} {\bm \Params}_{t-j}{\bm B}{\bm \Params}^T_{t-j-1}\Bigg)  
- {\bm A}\Bigg(\sum_{j=0}^{l-1}{\bm \Params}_{t-j-1}{\bm B}^T{\bm B}{\bm \Params}^T_{t-j-1}\Bigg) = 0
\end{split}
\end{equation}

\begin{equation}
\begin{split}
\Bigg(\sum_{j=0}^{l-1} {\bm \Params}_{t-j}{\bm B}{\bm \Params}^T_{t-j-1}\Bigg) = 
{\bm A}\Bigg(\sum_{j=0}^{l-1}{\bm \Params}_{t-j-1}{\bm B}^T{\bm B}{\bm \Params}^T_{t-j-1}\Bigg) \label{eq:app-A-imp}
\end{split}
\end{equation}

\begin{equation}
\begin{split}
{\bm A} = \Bigg(\sum_{j=0}^{l-1} {\bm \Params}_{t-j}{\bm B}{\bm \Params}^T_{t-j-1}\Bigg) 
\Bigg(\sum_{j=0}^{l-1}{\bm \Params}_{t-j-1}{\bm B}^T{\bm B}{\bm \Params}^T_{t-j-1}\Bigg)^{-1}.
\label{eq:app-A}
\end{split}
\end{equation}

Notice that Eq.~(\ref{eq:app-A}) is obtained by multiplying both sides of Eq.~(\ref{eq:app-A-imp}) by $\Big(\sum_{j=0}^{l-1}{\bm \Params}_{t-j-1}{\bm B}^T{\bm B}{\bm \Params}^T_{t-j-1}\Big)^{-1}$.

If we apply the same reasoning, we can also find a closed-form solution to compute $\hat{\bm B}$. That is, we take the partial derivative of the loss w.r.t. ${\bm B}$, set it to $0$, and solve it for ${\bm B}$:
\begin{equation}
    \label{eq:app-partial-b}
\frac{\partial{\Loss({\bm A},{\bm B}; {\bm \Params}_{t},l)}}{\partial{{\bm B}}} = 0.
\end{equation}
Eventually, we obtain the following:
\begin{equation}
\begin{split}
    \label{eq:app-B}
{\bm B} = \Bigg(\sum_{j=0}^{l-1} {\bm \Params}^T_{t-j}{\bm A}{\bm \Params}_{t-j-1}\Bigg) 
\Bigg(\sum_{j=0}^{l-1}{\bm \Params}^T_{t-j-1}{\bm A}^T{\bm A}{\bm \Params}_{t-j-1}\Bigg)^{-1}.
\end{split}
\end{equation}
We now have two closed-form solutions; one for ${\bm A}$ (see Eq.~(\ref{eq:app-A})) and one for ${\bm B}$ (see Eq.~(\ref{eq:app-B})).
However, the solution to ${\bm A}$ involves ${\bm B}$, and the solution to ${\bm B}$ involves ${\bm A}$. In other words, we must know ${\bm B}$ to compute ${\bm A}$ and vice versa.

We can use the standard Alternating Least Squares (ALS) algorithm (\cite{koren2009ieeecomp}) to solve such a problem. 
The fundamental idea of ALS is to iteratively update the least squares closed-form solution of each variable alternately, keeping the other fixed. At the generic $i$-th iteration, we compute:
\[
{\bm A}^{(i+1)} = \Bigg(\sum_{j=0}^{l-1} {\bm \Params}_{t-j}{\bm B}^{(i)}{\bm \Params}^T_{t-j-1}\Bigg) \Bigg(\sum_{j=0}^{l-1}{\bm \Params}_{t-j-1}({\bm B}^{(i)})^T{\bm B}^{(i)}{\bm \Params}^T_{t-j-1}\Bigg)^{-1}; \] \[
{\bm B}^{(i+1)} = \Bigg(\sum_{j=0}^{l-1} {\bm \Params}^T_{t-j}{\bm A}^{(i+1)}{\bm \Params}_{t-j-1}\Bigg) \Bigg(\sum_{j=0}^{l-1}{\bm \Params}^T_{t-j-1}({\bm A}^{(i+1)})^T{\bm A}^{(i+1)}{\bm \Params}_{t-j-1}\Bigg)^{-1};\nonumber
\]
ALS repeats the two steps above until some convergence criterion is met, e.g., after a specific number of iterations $N$ or when the distance between the values of the variables computed in two consecutive iterations is smaller than a given positive threshold, i.e., $d({\bm A}^{(i+1)}-{\bm A}^{(i)}) < \varepsilon$ and $d({\bm B}^{(i+1)}-{\bm B}^{(i)}) < \varepsilon$, where $d(\cdot)$ is any suitable matrix distance function and $\varepsilon \in \R_{>0}$.

Eventually, if ${\bm A}^{(\infty)}$ and ${\bm B}^{(\infty)}$ are the parameters of the MAR model upon convergence, we set $\hat{\bm A} = {\bm A}^{(\infty)}$ and $\hat{\bm B} = {\bm B}^{(\infty)}$ as the best coefficient matrices.

\section{How Does FLANDERS Work?}
\label{app:flanders}
%%% ALGORITHM
\label{subsec:algorithm}

%%% OVERVIEW OF FLANDERS
\subsection{A Step-by-Step Example}
\label{app:example}
In Figure~\ref{fig:flanders}, we depict how FLANDERS computes the anomaly score vector $\bm{s}^{(t)}$ at the generic FL round $t$. Specifically, the server $S$ uses its current MAR($1$) forecasting model $f$ whose best parameters $\hat{\bm{\Omega}} = \{\hat{\bm A}, \hat{\bm B}\}$ are estimated from $l$ previous historical observations of local models received in the previous rounds $t-l,\ldots, t-1$. It applies $f$ to the previously observed matrix $\bm{\Params}_{t-1}$ to get the next predicted matrix of local models $\bm{\hat{\Params}}_{t}$, i.e., $\bm{\hat{\Params}}_{t} = f(\bm{\Params}_{t-1};\hat{\bm{\Omega}}) = \hat{\bm A}\bm{\Params}_{t-1} \hat{\bm B}$. Then, it compares this predicted matrix $\bm{\hat{\Params}}_{t}$ with the actual matrix of local updates received $\bm{\Params}_{t}$. 
The final anomaly score vector is calculated by measuring the distance $\delta$ between each column of those two matrices, according to Eq.~(4) defined in the main paper.

It is worth remarking that, in general, only some clients are selected at every round. In particular, if a client $c_{\text{new}}$ -- whether it is honest or malicious -- is selected by the server for the first time at round $t$, the MAR forecasting model $f$ will not be able to make any prediction for it due to a cold start problem (i.e., the predicted matrix $\bm{\hat{\Params}}_{t}$ will not contain a column corresponding to $c_{\text{new}}$). 
In such a case, as already stated in the main submission, we must adopt a fallback strategy. Without any historical information for a client, the most sensible thing to do is to compute the distance $\delta$ between the local model update it sent and the current global model.
\begin{figure*}[htb!]
    \centering
    \includegraphics[width=.7\textwidth]{./img/flanders}
    \caption{Overview of FLANDERS.}
    \label{fig:flanders}
\end{figure*}

To better clarify how FLANDERS works, consider the following practical example.
\\
Suppose an FL system consists of a centralized server and $10$ clients $c_1,\ldots, c_{10}$; furthermore, at each round, $4$ of those clients are randomly chosen for training. 
At the very first round ($t=1$), let $\mathcal{C}^{(1)} = \{c_2, c_3, c_7, c_9\}$ be the set of $4$ clients selected by the server. Hence, the server sends the current global model $\params^{(1)} \in \R^d$ to each of those clients and collects the $d\times 4$ matrix of updated local models $\Params_1 = [\params_2^{(1)}, \params_3^{(1)}, \params_7^{(1)}, \params_9^{(1)}]$. 
Therefore, it computes the \textit{new} global model $\params^{(2)} = \phi(\{\params_{c}^{(1)}~|~c\in \mathcal{C}^{(1)}\})$.
Notice that, at this stage, no anomaly score can be computed as FLANDERS cannot take advantage of any historical observations of local model updates. As such, if one (or more) selected clients in the very first round are malicious, plain FedAvg may not detect those. To overcome this problem, FLANDERS should be paired with one of the existing robust aggregation heuristics at $t=1$. For example, $\phi = \{\text{Trimmed Mean, Krum, Bulyan}\}$.

At the next round ($t=2$), FLANDERS can start using past observations (i.e., only $\Params_1$) to estimate the best MAR coefficients $\hat{\bm{\Omega}} = \{\hat{\bm{A}},\hat{\bm{B}}\}$ according to Eq.~(\ref{eq:app-mar-opt}) above. 
Suppose $\mathcal{C}^{(2)} = \{c_1, c_3, c_6, c_9\}$ is the set of $4$ clients selected by the server at the second round.
Let $\Params_2 = [\params_1^{(2)}, \params_3^{(2)}, \params_6^{(2)}, \params_9^{(2)}]$ be the local models sent by the clients to the server.
Moreover, $\hat{\Params}_2 = f(\Params_1;\hat{\bm{\Omega}}) = [\hat{\params}_2^{(2)}, \hat{\params}_3^{(2)}, \hat{\params}_7^{(2)}, \hat{\params}_9^{(2)}]$ are the local models predicted by MAR, using the previous set of observations. 

It is worth noting that $\mathcal{C}^{(1)} \cap \mathcal{C}^{(2)} = \{c_3, c_9\}$; the other two clients, $c_1$ and $c_6$, are considered ``cold-start'' since they are selected for the first time or, in any case, they do not appear in the historical observations used by MAR for predictions (i.e., in the previous $w=1$ matrices).
Thus, we calculate the following anomaly scores, according to Eq.~(4) in the main body:
\begin{itemize}
\item $s_3^{(2)} = \delta(\params_3^{(2)}, \hat{\params}_3^{(2)})$ and $s_9^{(2)} = \delta(\params_9^{(2)}, \hat{\params}_9^{(2)})$, i.e., we measure the distance between the models actually sent by $c_3$ and $c_9$ and those predicted by MAR using the previous observations (\textit{first condition});
\item $s_1^{(2)} = \delta(\params^{(2)}, \params_1^{(2)})$ and $s_6^{(2)} = \delta(\params^{(2)}, \params_6^{(2)})$, i.e., we measure the distance between the models actually sent by $c_1$ and $c_6$ and the current global model at time $t=2$, as those clients were never picked before (\textit{second condition});
\item $s_2^{(2)} = s_4^{(2)} = s_5^{(2)} = s_7^{(2)} = s_8^{(2)} = s_{10}^{(2)} = \perp$, i.e., these clients were not selected at round $t=2$, and therefore they will not contribute to computing the new global model $\params^{(3)}$ anyway (\textit{third condition}).
\end{itemize}
Let us assume that $c_3$ is a malicious client controlled by an attacker. Moreover, suppose that this client started sending poisoned models since the very first round, i.e., $\params_3^{(1)}$ was already corrupted. 
In such a case, it is evident that if we had used plain FedAvg at $t=1$, this would have likely polluted the global model $\params^{(2)}$ and, therefore, FLANDERS might fail to recognize this as a malicious client due to a relatively low anomaly score $s_3^{(2)}$. As stated before, the consequences of this edge situation in which one or more malicious clients are picked at the beginning of the FL training can be mitigated by replacing FedAvg with one of the robust aggregation strategies available in the literature, such as Trimmed Mean, Krum, or Bulyan. 

However, suppose $c_3$ is correctly spotted as malicious at the end of round $t=2$ due to its high anomaly score $s_3^{(2)}$. Therefore, $c_3$ (actually, $\params_3^{(2)}$) will be discarded from the aggregation at the server's end. Hence, FedAvg can now safely be used; more generally, the server can restart running FedAvg from $t=2$ on, i.e., once FLANDERS can compute \textit{valid} anomaly scores.
The server can now compute the updated global model as $\params^{(3)} = \phi(\{\params_{c}^{(2)}~|~c\in \mathcal{C}_*^{(2)}\})$, where $\mathcal{C}_*^{(2)} \subset \mathcal{C}^{(2)}$ contains the clients with the $k$ smallest anomaly scores. 
For instance, if $k=2$ and $\mathcal{C}_*^{(2)} = \{c_1, c_9\}$, $\params^{(3)} = 1/2 * (\params_1^{(2)} + \params_9^{(2)})$.

At the next round ($t=3$), FLANDERS can use the previous two observations $\Params_1$ and $\Params_2$ to refine the estimation of the best MAR coefficients, again solving Eq.~(\ref{eq:app-mar-opt}).\footnote{In general, FLANDERS uses $l$ past observations $\Params_{\text{max}(1,t-l):t-1}$.} 
However, $\Params_2$ cannot be fed as-is to re-train MAR since one of its components -- i.e., the local model $\params_3^{(2)}$ sent by client $c_3$ -- has been flagged as malicious. 
Otherwise, the resulting updated MAR coefficients could be unreliable due to the propagation of local poisoned models.

To overcome this problem, FLANDERS replaces the local model marked as suspicious $\params_3^{(2)}$ with either one of the previously observed models from the same client that is supposedly legitimate \textit{or} the current global model. 
Since, in this example, we assume $c_3$ has been malicious from the first round, we use the latter approach. 
Specifically, we change $\Params_2$ with $\Params'_2 = [\params_1^{(2)}, \underline{\params_{}^{(2)}}, \params_6^{(2)}, \params_9^{(2)}]$, where $\params^{(2)}$ substitutes $\params_3^{(2)}$.
As discussed at the end of Section~5.2 of the main body, this fix allows FLANDERS to work even when a malicious client is picked for two or more rounds consecutively.

We can now compute $\hat{\Params}_3 = f(\Params'_2;\hat{\bm{\Omega}}) = [\hat{\params}_1^{(3)}, \hat{\params}_3^{(3)}, \hat{\params}_6^{(3)}, \hat{\params}_9^{(3)}]$ using the updated MAR, leveraging the previous set of amended observations. 
Let $\mathcal{C}^{(3)} = \{c_2, c_3, c_4, c_5\}$ denote the set of clients selected at round $3$ and $\Params_3 = [\params_2^{(3)}, \params_3^{(3)}, \params_4^{(3)}, \params_5^{(3)}]$ be the local models sent by the clients to the server. 
Thus, $\mathcal{C}^{(2)} \cap \mathcal{C}^{(3)} = \{c_3\}$. The remaining three clients -- $c_2$, $c_4$, and $c_5$ -- are treated as ``cold-start.'' Specifically, $c_4$ and $c_5$ are selected for the first time, while $c_2$, even though previously selected in the first round, was not chosen at the previous round ($2$). As $c_2$ was not part of the historical observations used by MAR($1$) to make predictions, we need to treat it as if it were a cold-start client.

Therefore, anomaly scores are updated as follows:
\begin{itemize}
\item $s_3^{(3)} = \delta(\params_3^{(3)}, \hat{\params}_3^{(3)})$, i.e., we measure the distance between the models actually sent by $c_3$ and those predicted by MAR using the previous observations (\textit{first condition});
\item $s_2^{(3)} = \delta(\params^{(3)}, \params_2^{(3)})$, $s_4^{(3)} = \delta(\params^{(3)}, \params_4^{(3)})$, and $s_5^{(3)} = \delta(\params^{(3)}, \params_5^{(3)})$ i.e., we measure the distance between the models actually sent by $c_2$, $c_4$, and $c_5$ and the current global model at time $t=3$, as those clients were never picked before or did not appear in the previous observations used to make predictions (\textit{second condition});
\item $s_1^{(3)} = s_6^{(3)} = s_7^{(3)} = s_8^{(3)} = s_9^{(3)} = s_{10}^{(3)} = \perp$, i.e., these clients were not selected at round $t=3$, and therefore they will not contribute to computing the new global model $\params^{(4)}$ anyway (\textit{third condition}).
\end{itemize}
Generally, the process above continues until the global model converges.


\subsection{FLANDERS' Pseudocode}
We present the pseudocode of a hypothetical FL server that integrates FLANDERS into the model aggregation stage. 
The main server-side loop is shown in Algorithm~\ref{alg:server}, whereas Algorithm~\ref{alg:flanders} details the computation of anomaly scores, which is at the heart of the FLANDERS filter.

\begin{algorithm}
    \caption{\texttt{\textsc{FLANDERS-Server}}}\label{alg:server}
    \begin{algorithmic}[1]
        \Require The aggregation function ($\phi$); the number of randomly selected clients at each FL round ($m$); the number of local models to keep as legitimate ($k$); the autoregressive order of MAR ($w$); the number of historical observations used to train MAR ($l$); the number of MAR training iterations ($N$); the number of total FL rounds ($T$).
        \Ensure The global model $\bm{\theta}^{(T)}$.%
 \Procedure{\texttt{FLANDERS-Server}}{$\phi$, $m$, $k$, $w$, $l$, $N$, $T$}
        \State $\bm{\theta}^{(1)} \gets $ A randomly initialized model
        
        \ForEach {$t \in \{1,2,\ldots T\}$}
            \State $\mathcal{C}^{(t)} \gets $ sample a subset of $m$ clients from $\mathcal{C}$
            \State Send the global model $\bm{\theta}^{(t)}$ to every  $c\in \mathcal{C}^{(t)}$
            \State $\bm{\Theta}_t \gets [\bm{\theta}^{(t)}_1, \ldots, \bm{\theta}^{(t)}_m]$ \Comment{Receive the $m$ local models trained by each $c\in \mathcal{C}^{(t)}$}
            \If{$t==1$}
            \State $\mathcal{C}^{(t)}_* \gets \texttt{\textsc{Fallback}}(\bm{\Theta}_{t})$ \Comment{At the very first round, use the designated fallback aggregation strategy (e.g., FedAvg)}
            \Else
            \State $\mathcal{C}^{(t)}_* \gets \texttt{\textsc{FLANDERS-Filter}}(\bm{\Theta}_{t-l:t-1}, \mathcal{C}^{(t)}, k, w, N)$ \Comment{Returns the set of clients classified as legitimate}
            \EndIf
            \State $\params^{(t+1)} \gets \phi(\{\params_c^{(t)}~|~c\in \mathcal{C}_*^{(t)}\})$
        \EndFor
    \EndProcedure
    \end{algorithmic}%
\end{algorithm}%
\begin{algorithm}
    \caption{\texttt{\textsc{FLANDERS-Filter}}}\label{alg:flanders}
    \begin{algorithmic}[1]
        \Require The tensor containing the local models of selected clients observed from $l$ past FL rounds ($\bm{\Theta}_{t-l:t-1}$); the clients selected at round $t$ ($\mathcal{C}^{(t)}$); the autoregressive order of MAR ($w$); the number of local models to keep as legitimate ($k$); the number of MAR training iterations ($N$).
        \Ensure The set of $k$ clients classified as legitimate in round $t$ $\mathcal{C}^{(t)}_{*}$.
 \Procedure{\texttt{FLANDERS-Filter}}{$\bm{\Theta}_{t-l:t-1}, \mathcal{C}^{(t)}, k, w, N$}
        \State $\hat{\bm{\Theta}}_{t} \gets \texttt{MAR}(\bm{\Theta}_{t-l:t-1}, w, N)$ \Comment MAR($w$) estimation via ALS training ($N$ iterations)
        \State $\bm{s}^{(t)} \gets \delta(\bm{\Theta}_t, \hat{\bm{\Theta}}_t)$ \Comment{Anomaly score vector}
        \State $\mathcal{C}^{(t)}_{*} \gets \{c \in \mathcal{C}^{(t)}~|~s_{c}^{(t)} \leq s_k^{(t)}\}$ \Comment{$s^{(t)}_k$ is the $k$-th smallest anomaly score}
        \State \textbf{return} $\mathcal{C}^{(t)}_{*}$
\EndProcedure
    \end{algorithmic}
    \end{algorithm}

%%% SETUP
\section{Datasets, Tasks, and FL Models}
\label{app:setup}

In Table~\ref{tab:setup}, we report the full details of our experimental setup concerning the datasets used, their associated tasks, and the models (along with their hyperparameters) trained on the simulated FL environment, i.e., Flower\footnote{\url{https://flower.dev/}} (see Appendix~\ref{app:fl-env}). Table~\ref{tab:hyp} shows the description of the hyperparameters.

\begin{table}[ht]
\centering
\caption{Experimental setup: datasets, tasks, and FL models considered.
}
\label{tab:setup}
\vspace{2mm}
\scalebox{0.8}{
\begin{tabular}{|c|c|c|c|c|c|}
\hline
\textbf{Dataset}        & \thead{\textbf{N. of Instances}\\\textbf{(training/test)}} & \textbf{N. of Features} & \textbf{Task} & \textbf{FL Model} & \textbf{Hyperparameters}\\ \hline
\thead{\textit{MNIST}~\cite{mnist-ds}} & $60,000$/$10,000$ & \thead{$28$x$28$\\(numerical)} & \thead{multiclass\\classification} & MLP & \thead{\{{\tt batch}=$32$; {\tt layers}=$2$;\\{\tt opt}=Adam; $\eta$=$10^{-3}$\}}\\ \hline
\thead{\textit{Fashion-MNIST}~\cite{fashionmnist-ds}} & $60,000$/$10,000$ & \thead{$28$x$28$\\(numerical)} & \thead{multiclass\\classification} & MLP & \thead{\{{\tt batch}=$32$; {\tt layers}=$4$;\\{\tt opt}=Adam; $\eta$=$10^{-3}$\}}\\ \hline
\thead{\textit{CIFAR-10}~\cite{cifar10-100-ds}} & $50,000$/$10,000$ & \thead{$32$x$32$x$3$\\(numerical)} & \thead{multiclass\\classification} & CNN & \thead{\{{\tt batch}=$32$; {\tt layers}=$6$;\\{\tt opt}=SGD; $\eta$=$10^{-2}$; $\mu$=$0.9$\}}\\ \hline
\thead{\textit{CIFAR-100}~\cite{cifar10-100-ds}} & $50,000$/$10,000$ & \thead{$32$x$32$x$3$\\(numerical)} & \thead{multiclass\\classification} &  CNN (MobileNet) & \thead{\{{\tt batch}=$32$; {\tt layers}=$28$;\\{\tt opt}=SGD; $\eta$=$10^{-2}$; $\mu$=$0.9$\}}\\ \hline
\end{tabular}
}
\end{table}

\begin{table}[ht!]
\centering
\caption{Description of hyperparameters.
}
\label{tab:hyp}
\vspace{2mm}
\scalebox{0.9}{
\begin{tabular}{|c|c|}
\hline
\textbf{Hyperparameter}        & \textbf{Description} \\ \hline
$\eta$ & learning rate\\
\hline
$\mu$ & momentum\\
\hline
{\tt opt} & optimizer: stochastic gradient descent (SGD); Adam\\
\hline
{\tt batch} & batch size\\
\hline
{\tt layers} & number of neural network layers\\
\hline
\end{tabular}
}
\end{table}

The MLP for the \textit{MNIST} dataset is a $2$-layer fully connected feed-forward neural network, whereas the MLP for the \textit{Fashion-MNIST} dataset is a $4$-layer fully connected feed-forward neural network. Both MLPs are trained by minimizing multiclass cross-entropy loss using Adam optimizer with batch size equal to $32$~\cite{li2022blades}.
The CNN used for \textit{CIFAR-10} is a $6$-layer convolutional neural network, while the CNN used for \textit{CIFAR-100} is the well-known MobileNet architecture~\cite{howard2017mobilenets}. Both CNNs are trained by minimizing multiclass cross-entropy loss via stochastic gradient descent (SGD) with batch size equal to $32$~\cite{pytorch-cifar-10}.

At each FL round, every client performs one training epoch of Adam/SGD, which corresponds to the number of iterations needed to ``see'' all the training instances once, when divided into batches of size $32$.
In any case, the updated local model is sent to the central server for aggregation.

We run our experiments on a machine equipped with an AMD Ryzen 9, 64 GB RAM, and an NVIDIA 4090 GPU with 24 GB VRAM.

%%% FL SIMULATION ENVIRONMENT
\section{FL Simulation Environment}
\label{app:fl-env}
To simulate a realistic FL environment, we integrate FLANDERS into Flower. Moreover, we implement in Flower every other defense baseline considered in this work that the framework does not natively provide.
Other valid FL frameworks are available (e.g., TensorFlow Federated\footnote{\scriptsize{\url{https://www.tensorflow.org/federated}}} and PySyft\footnote{\scriptsize{\url{https://github.com/OpenMined/PySyft}}}), but Flower turned out the most flexible.
\label{app:fl-env}
\begin{table}[ht!]
\centering
\caption{Main properties of our FL environment simulated on Flower.
}
\label{tab:fl-env}
\vspace{2mm}
\scalebox{0.85}{
\begin{tabular}{|l|c|}
\hline
\textbf{Total N. of Clients}        & $K = 100$ \\ \hline
\textbf{N. of Selected Clients (at each round)} & $m = K = 100$\\ \hline
\textbf{Ratio of Malicious Clients} & $r=\{0, 0.2, 0.6, 0.8\}$ \\ \hline
\textbf{Total N. of FL Rounds} & $T=50$ \\ \hline
\textbf{Autoregressive Order of MAR} & $w=1$ \\ \hline
\textbf{Historical Window Size of Past FL Rounds} & $l=2$ \\ \hline
\textbf{Non-IID Dataset Distribution across Clients} & $\alpha_D = 0.5$  \\ \hline
\end{tabular}
}
\end{table}

We want to remark that within Flower, only the \textit{number} of malicious clients in each FL round remains constant, while the framework manages their selection. This implies that a single client can alternate between legitimate and malicious behavior over successive FL rounds.

We report the main properties of our FL environment simulated on Flower in Table~\ref{tab:fl-env}.

%%% ATTACK SETTINGS
\section{Attack Settings}
\label{app:attacks}
Below, we describe the critical parameters for
each attack considered, which are also summarized in Table~\ref{tab:attacks}.

\noindent{{\bf {\em GAUSS}}{\bf.}} This attack has only one parameter: the magnitude $\sigma$ of the perturbation to apply. We set $\sigma=10$ for all the experiments.

\noindent{{\bf {\em LIE}}{\bf.}} This method has no parameters to set.

\noindent{{\bf {\em OPT}}{\bf.}} The parameter $\tau$ represents the minimum value that $\lambda$ can assume. Below this threshold, the halving search stops. As suggested by the authors, we set $\tau=10^{-5}$.

\noindent{{\bf {\em AGR-MM}}{\bf.}} In addition to the threshold $\tau$, AGR-MM uses the so-called perturbation vectors ($\nabla^p$) in combination with the scaling coefficient $\gamma$ to optimize. We set $\tau=10^{-5}$ and $\nabla^p = \{std\}$, which is the vector obtained by computing the parameters' inverse of the standard deviation. For Krum, we set $\nabla^p = \{uv\}$, which is the inverse unit vector perturbation.
\begin{table}[ht]
\centering
\caption{Key parameter settings for each attack strategy considered.
}
\label{tab:attacks}
\vspace{2mm}
\begin{tabular}{|c|c|}
\hline
\textbf{Attack}        & \textbf{Parameters} \\ \hline
{\bf {\em GAUSS}} & $\sigma=10$\\ \hline
{\bf {\em LIE}} & N/A\\ \hline
{\bf {\em OPT}} & $\tau=10^{-5}$\\ \hline
{\bf {\em AGR-MM}} & $\tau=10^{-5}; \nabla^p = \{uv, std\}; \gamma=5$\\ \hline
\end{tabular}
\end{table}

%%% DEFENSE SETTINGS
\section{Defense Settings}
\label{app:defenses}
Below, we describe the critical parameters for
each non-trivial baseline considered.

\noindent{{\bf {\em Trimmed Mean}}{\bf.}} 
The key parameter of this defense strategy is $\beta$, which is used to cut the parameters on the edges. In this work, we set $\beta = 0.2$.

\noindent{{\bf {\em FedMedian}}{\bf.}}
This method has no parameters to set.

\noindent{{\bf {\em Multi-Krum}}{\bf.}} 
We set the number of local models to keep for the aggregation (FedAvg) after the Krum filtering as $k=(b-m)$

\noindent{{\bf {\em Bulyan}}{\bf.}}
The two crucial parameters of this hybrid robust aggregation rule are $\alpha$ and $\beta$. The former determines the number of times Krum is applied to generate $\alpha$ local models; the latter is used to determine the number of parameters to select closer to the median.
In this work, we set $\alpha = m - 2 \cdot b, \beta = \alpha - 2 \cdot b$, where $b=\{0,20\}$.

\noindent{{\bf {\em DnC}}{\bf.}}
This is an iterative algorithm that has three parameters: we set $niters = 5$ as suggested by the authors, and the filtering fraction $c = 1$ to keep exactly one model (i.e., the one with the best anomaly score). Furthermore, we set $\widetilde{d} = 500$ on \textit{CIFAR-100} to sample $500$ local model weights.

\noindent{{\bf {\em FLDetector}}{\bf.}}
We set the window size $N=20$, and we let FLDetector know how many local models to keep, i.e., $k=(m-b)$, as we did for the other baselines.


\noindent{{\bf {\em FLANDERS}}{\bf.}}
For a fair comparison with other baselines, we set the sliding window size $l=2$, and the number of clients to keep at every round $k=(m-b)$, where $m$ is the number of clients selected at each round and $b$ the total number of malicious clients in the FL system. Furthermore, we use a sampling value $\widetilde{d}\leq d$ that indicates how many parameters we store in the history, the number $N$ of ALS iterations, and $\alpha$ and $\beta$ which are regularization factors. Random sampling is used to select the subset of parameters considered. This is a common strategy proposed in the literature~\cite{shejwalkar2021ndss} to lower the model size to train and save the server's memory. On the other hand, the regularization factors are needed when the model parameters, the number of clients selected, or $\widetilde{d}$ cause numerical problems in the ALS algorithm, whose number of iterations is set to $N=100$. We always set $\alpha=1$ and $\beta=1$, meaning that there is no regularization since, in our experience, it reduces the capability of predicting the right model. For this reason, we set $\widetilde{d}=500$.
Finally, we set the distance function $\delta$ used to measure the difference between the observed vector of weights sent to the server and the predicted vector of weights output by MAR to squared $L^2$-norm. 
In future work, we plan to investigate other distance measures, such as cosine or mutual information distance.

Table~\ref{tab:defenses} summarizes the values of the key parameters discussed above and those characterizing our method FLANDERS.
\begin{table*}[h]
\centering
\caption{Key parameter settings for each defense strategy considered.
}
\label{tab:defenses}
\vspace{2mm}
\scalebox{0.95}{
\begin{tabular}{|c|c|}
\hline
\textbf{Defense}        & \textbf{Parameters} \\ \hline
{\bf {\em FedAvg}} & N/A\\ \hline
{\bf {\em Trimmed Mean}} & $\beta=0.2$\\ \hline
{\bf {\em FedMedian}} & N/A\\ \hline
{\bf {\em Multi-Krum}} & $b=\{0,20\}; k=(m-b)$\\ \hline
{\bf {\em Bulyan}} & $b=\{0,20\}; \alpha = m - 2 \cdot b; \beta = \alpha - 2 \cdot b$\\ \hline
{\bf {\em DnC}} & $\widetilde{d}=500; niters=5; c=1$ \\ \hline
{\bf {\em FLDetector}} & $N=20; k=(m-b)$ \\
\hline
{\bf {\em FLANDERS}} & $l=2$; $k=(m-b)$; $\widetilde{d}=\{0, 500\}$; $N=100$; $\alpha=1$; $\beta=1$; $\delta = \text{squared } L^2$-norm  \\ \hline
\end{tabular}
}
\end{table*}

\section{Additional Results}
\label{app:further-results}


\subsection{Malicious Detection Accuracy}
\label{app:flanders-accuracy}

We evaluate the capability of FLANDERS to detect malicious clients accurately. 
Specifically, let $\mathcal{C}^{(t)}_{\text{mal}}\subseteq \mathcal{C}$ be the set of malicious clients selected at round $t$ by the FL server. 
Furthermore, let $\hat{\mathcal{C}}^{(t)}_{\text{mal}}\subseteq \mathcal{C}$ be the set of malicious clients identified by FLANDERS at round $t$. Thus, we measure \textit{Precision} ($P$) and \textit{Recall} ($R$) as usual, i.e., $P=\frac{tp}{tp + fp}$ and $R=\frac{tp}{tp + fn}$, where $tp,fp,fn$ stand for \textit{true positives}, \textit{false positives}, and \textit{false negatives}. 
Specifically, $tp = \sum_{t=1}^T |\mathcal{C}^{(t)}_{\text{mal}} \cap \hat{\mathcal{C}}^{(t)}_{\text{mal}}|$, $fp = \sum_{t=1}^T |\hat{\mathcal{C}}^{(t)}_{\text{mal}} \setminus \mathcal{C}^{(t)}_{\text{mal}}|$, and $fn = \sum_{t=1}^T |\mathcal{C}^{(t)}_{\text{mal}} \setminus \hat{\mathcal{C}}^{(t)}_{\text{mal}}|$.

Tables~\ref{tab:pr-60}, and \ref{tab:pr-80} illustrate the values for the precision ($P$) and recall ($R$) values of FLANDERS under extreme attack scenarios, integrate the findings already reported in Table~\ref{tab:pr-20} in the main body. 


\begin{table}[htb!]
\centering
\caption{\textit{Precision} ($P$) and \textit{Recall} ($R$) of FLDetector and FLANDERS in detecting malicious clients across various datasets and attacks ($r=0.6$; $T=50$ rounds).}
\label{tab:pr-60}
\scalebox{0.75}{
\begin{tabular}{|c|cc|cc|cc|cc||cc|cc|cc|cc|}
 \cline{2-17}
 \multicolumn{1}{c|}{} & \multicolumn{8}{c||}{\textbf{FLDetector}} & \multicolumn{8}{c|}{\textbf{FLANDERS}} \\
 \hline
 \multirow{2}{*}{\textbf{Dataset}} & \multicolumn{2}{c|}{GAUSS} & \multicolumn{2}{c|}{LIE} & \multicolumn{2}{c|}{OPT} & \multicolumn{2}{c||}{AGR-MM} & \multicolumn{2}{c|}{GAUSS} & \multicolumn{2}{c|}{LIE} & \multicolumn{2}{c|}{OPT} & \multicolumn{2}{c|}{AGR-MM} \\
 \cline{2-17}
                       & $P$   & $R$   & $P$   & $R$   & $P$    & $R$    & $P$   & $R$   & $P$   & $R$   & $P$   & $R$    & $P$  & $R$   & $P$  & $R$ \\
\hline
\textit{MNIST}         & $0.60$ & $0.60$ & $0.60$ & $0.60$ & $0.62$ & $0.62$ & $0.60$ & $0.60$   & ${\bf 1.0}$ & ${\bf 1.0}$ & ${\bf 1.0}$ & ${\bf 1.0}$ & ${\bf 1.0}$ & ${\bf 1.0}$ & ${\bf 1.0}$ & ${\bf 1.0}$ \\
\hline
\textit{Fashion-MNIST} & $0.60$ & $0.60$ & $0.59$ & $0.59$ & $0.60$ & $0.60$ & $0.61$ & $0.61$   & ${\bf 1.0}$ & ${\bf 1.0}$ & ${\bf 1.0}$ & ${\bf 1.0}$ & ${\bf 1.0}$ & ${\bf 1.0}$ & ${\bf 1.0}$ & ${\bf 1.0}$ \\
\hline
\textit{CIFAR-10}      & $0.60$ & $0.60$ & $0.61$ & $0.61$ & $0.60$ & $0.60$ & $0.60$ & $0.60$   & ${\bf 1.0}$ & ${\bf 1.0}$ & ${\bf 1.0}$ & ${\bf 1.0}$ & ${\bf 1.0}$ & ${\bf 1.0}$ & ${\bf 1.0}$ & ${\bf 1.0}$ \\
\hline
\textit{CIFAR-100}     & $0.60$ & $0.60$ & $0.60$ & $0.60$ & $0.60$ & $0.60$ & $0.61$ & $0.61$   & ${\bf 1.0}$ & ${\bf 1.0}$ & ${\bf 1.0}$ & ${\bf 1.0}$ & ${\bf 1.0}$  & ${\bf 1.0}$  & ${\bf 1.0}$ & ${\bf 1.0}$ \\
\hline
\end{tabular}
}
\end{table}


\begin{table}[htb!]
\centering
%\vspace{-4mm}
\caption{\textit{Precision} ($P$) and \textit{Recall} ($R$) of FLDetector and FLANDERS in detecting malicious clients across various datasets and attacks ($r=0.8$; $T=50$ rounds).}
\label{tab:pr-80}
\scalebox{0.75}{
\begin{tabular}{|c|cc|cc|cc|cc||cc|cc|cc|cc|}
 \cline{2-17}
 \multicolumn{1}{c|}{} & \multicolumn{8}{c||}{\textbf{FLDetector}} & \multicolumn{8}{c|}{\textbf{FLANDERS}} \\
 \hline
 \multirow{2}{*}{\textbf{Dataset}} & \multicolumn{2}{c|}{GAUSS} & \multicolumn{2}{c|}{LIE} & \multicolumn{2}{c|}{OPT} & \multicolumn{2}{c||}{AGR-MM} & \multicolumn{2}{c|}{GAUSS} & \multicolumn{2}{c|}{LIE} & \multicolumn{2}{c|}{OPT} & \multicolumn{2}{c|}{AGR-MM} \\
 \cline{2-17}
                            & $P$   & $R$   & $P$   & $R$   & $P$    & $R$    & $P$   & $R$     & $P$   & $R$   & $P$   & $R$    & $P$  & $R$   & $P$  & $R$ \\
\hline
\textit{MNIST}              & $0.80$ & $0.80$ & $0.80$ & $0.80$ & $0.80$ & $0.80$ & $0.80$ & $0.80$ & ${\bf 1.0}$ & ${\bf 1.0}$ & ${\bf 1.0}$ & ${\bf 1.0}$ & ${\bf 1.0}$ & ${\bf 1.0}$ & ${\bf 1.0}$ & ${\bf 1.0}$ \\
\hline
\textit{Fashion-MNIST}      & $0.80$ & $0.80$ & $0.80$ & $0.80$ & $0.79$ & $0.79$ & $0.80$ & $0.80$ & ${\bf 1.0}$ & ${\bf 1.0}$ & ${\bf 1.0}$ & ${\bf 1.0}$ & ${\bf 1.0}$ & ${\bf 1.0}$ & ${\bf 1.0}$ & ${\bf 1.0}$ \\
\hline
\textit{CIFAR-10}           & $0.80$ & $0.80$ & $0.80$ & $0.80$ & $0.80$ & $0.80$ & $0.80$ & $0.80$ & ${\bf 1.0}$ & ${\bf 1.0}$ & ${\bf 1.0}$ & ${\bf 1.0}$ & ${\bf 1.0}$ & ${\bf 1.0}$ & ${\bf 1.0}$ & ${\bf 1.0}$ \\
\hline
\textit{CIFAR-100}          & $0.80$ & $0.80$ & $0.80$ & $0.80$ & $0.80$ & $0.80$ & $0.80$ & $0.80$ & ${\bf 1.0}$ & ${\bf 1.0}$ & ${\bf 1.0}$ & ${\bf 1.0}$ & ${\bf 1.0}$  & ${\bf 1.0}$  & ${\bf 1.0}$ & ${\bf 1.0}$ \\
\hline
\end{tabular}
}
\end{table}


\subsection{Aggregation Robustness Lift}
\label{app:agg-lift}

In this section, we report the full results on the improved robustness of the global model against malicious attacks when FLANDERS is paired with existing aggregation strategies.

First of all, Tables~\ref{tab:increment-fld-60} and \ref{tab:increment-fld-80} illustrate the comparison between FLANDERS and its main competitor, FLDetector, when both are paired with standard FedAvg, under severe ($r=0.6$) and extreme ($r=0.8$) attack scenarios. These results complete those shown in Table~\ref{tab:increment-fld} of the main submission.

In Table~\ref{tab:increment-cifar100-80}, we report the results for the \textit{CIFAR-100} dataset, which, due to space limitations, were not included in Table~\ref{tab:increment} of the main submission.

\begin{table*}[htb!]
\centering
%\vspace{-2mm}
\caption{Accuracy of the global model using FedAvg with FLDetector and FLANDERS ($r=0.6$).}
\label{tab:increment-fld-60}
%\vspace{2mm}
\scalebox{0.7}{
\begin{tabular}{|c|cccc|cccc|cccc|}
\hline
\multirow{2}{*}{\textbf{Attack}} & \multicolumn{4}{c|}{\textbf{FedAvg}} & \multicolumn{4}{c|}{\textbf{FLDetector + FedAvg}} & \multicolumn{4}{c|}{\textbf{FLANDERS + FedAvg}} \\
\cline{2-13}
                    &  GAUSS        &  LIE          &  OPT          &  AGR-MM       &  GAUSS        &  LIE          &  OPT          &  AGR-MM       &  GAUSS         &  LIE          &  OPT          &  AGR-MM      \\ 
\hline
\textit{MNIST}          & $0.18$ & $0.12$ & $0.16$ & $0.15$ & $0.19$ & $0.12$ & $0.17$ & $0.12$ & ${\bf 0.78}$ & ${\bf 0.82}$ & ${\bf 0.79}$ & ${\bf 0.85}$ \\
\textit{Fashion-MNIST}  & $0.28$ & $0.10$ & $0.18$ & $0.10$ & $0.19$ & $0.10$ & $0.11$ & $0.10$ & ${\bf 0.69}$ & ${\bf 0.65}$ & ${\bf 0.65}$ & ${\bf 0.62}$ \\
\textit{CIFAR-10}       & $0.10$ & $0.10$ & $0.10$ & $0.10$ & $0.10$ & $0.10$ & $0.10$ & $0.10$ & ${\bf 0.36}$ & ${\bf 0.36}$ & ${\bf 0.34}$ & ${\bf 0.35}$ \\
\textit{CIFAR-100}      & $0.01$ & $0.01$ & $0.01$ & $0.01$ & $0.01$ & $0.01$ & $0.01$ & $0.01$ & ${\bf 0.08}$ & ${\bf 0.09}$ & ${\bf 0.08}$ & ${\bf 0.09}$ \\
\hline
\end{tabular}
}
\end{table*}

\begin{table*}[htb!]
\centering
%\vspace{-2mm}
\caption{Accuracy of the global model using FedAvg with FLDetector and FLANDERS ($r=0.8$).}
\label{tab:increment-fld-80}
%\vspace{2mm}
\scalebox{0.7}{
\begin{tabular}{|c|cccc|cccc|cccc|}
\hline
\multirow{2}{*}{\textbf{Attack}} & \multicolumn{4}{c|}{\textbf{FedAvg}} & \multicolumn{4}{c|}{\textbf{FLDetector + FedAvg}} & \multicolumn{4}{c|}{\textbf{FLANDERS + FedAvg}} \\
\cline{2-13}
                    &  GAUSS        &  LIE          &  OPT          &  AGR-MM       &  GAUSS        &  LIE          &  OPT          &  AGR-MM       &  GAUSS         &  LIE          &  OPT          &  AGR-MM      \\ 
\hline
\textit{MNIST}          & $0.18$ & $0.11$ & $0.21$ & $0.11$ & $0.15$ & $0.13$ & $0.23$ & $0.12$ & ${\bf 0.75}$ & ${\bf 0.84}$ & ${\bf 0.84}$ & ${\bf 0.82}$ \\
\textit{Fashion-MNIST}  & $0.24$ & $0.10$ & $0.19$ & $0.10$ & $0.19$ & $0.10$ & $0.03$ & $0.10$ & ${\bf 0.68}$ & ${\bf 0.70}$ & ${\bf 0.66}$ & ${\bf 0.66}$ \\
\textit{CIFAR-10}       & $0.10$ & $0.10$ & $0.11$ & $0.10$ & $0.10$ & $0.10$ & $0.10$ & $0.10$ & ${\bf 0.33}$ & ${\bf 0.32}$ & ${\bf 0.32}$ & ${\bf 0.32}$ \\
\textit{CIFAR-100}      & $0.01$ & $0.01$ & $0.01$ & $0.01$ & $0.01$ & $0.01$ & $0.01$ & $0.01$ & ${\bf 0.09}$ & ${\bf 0.11}$ & ${\bf 0.11}$ & ${\bf 0.10}$ \\
\hline
\end{tabular}
}
\end{table*}

\begin{table*}[htb!]
\centering
%\vspace{-2mm}
\caption{Accuracy of the global model on \textit{CIFAR-100} using all the baseline aggregation methods without and with FLANDERS ($r=0.8$). The best results are typed in boldface.}
\label{tab:increment-cifar100-80}
%\vspace{2mm}
\scalebox{0.9}{
\begin{tabular}{|c|cccc|}
\hline
\multirow{2}{*}{\textbf{Aggregation}} & \multicolumn{4}{c|}{\textit{CIFAR-100}} \\
\cline{2-5}
            &  GAUSS        &  LIE          &  OPT          &  AGR-MM      \\ 
\hline
FedAvg      & $0.01$ & $0.01$ & $0.01$ & $0.01$ \\
+ FLANDERS  & ${\bf 0.09}$ & ${\bf 0.11}$ & ${\bf 0.11}$ & ${\bf 0.10}$ \\
\hline
FedMedian   & $0.01$ & $0.01$ & $0.01$ & $0.01$ \\
+ FLANDERS  & ${\bf 0.10}$ & ${\bf 0.11}$ & ${\bf 0.11}$ & ${\bf 0.10}$ \\
\hline
TrimmedMean & $0.01$ & $0.01$ & $0.01$ & $0.01$ \\
+ FLANDERS  & ${\bf 0.11}$ & ${\bf 0.10}$ & ${\bf 0.11}$ & ${\bf 0.11}$ \\
\hline
Multi-Krum  & $0.08$ & $0.01$ & $0.01$ & $0.01$ \\
+ FLANDERS  & ${\bf 0.11}$ & ${\bf 0.10}$ & ${\bf 0.10}$ & ${\bf 0.11}$ \\
\hline
Bulyan      & N/A & N/A & N/A & N/A \\
+ FLANDERS  & ${\bf 0.11}$ & ${\bf 0.11}$ & ${\bf 0.11}$ & ${\bf 0.11}$ \\
\hline
DnC         & $0.0.1$ & $0.01$ & $0.01$ & $0.01$ \\
+ FLANDERS  & ${\bf 0.10}$ & ${\bf 0.11}$ & ${\bf 0.11}$ & ${\bf 0.12}$ \\
\hline
\end{tabular}
}
\end{table*}

Furthermore, Tables~\ref{tab:increment-20-mnist-fmnist}, \ref{tab:increment-20-cifar10-cifar100}, \ref{tab:increment-60-mnist-fmnist}, and \ref{tab:increment-60-cifar10-cifar100} illustrate the impact of FLANDERS on the global model's accuracy under light ($r=0.2$) and moderate ($r=0.6$) attack settings.

\begin{table*}[htb!]
\centering
%\vspace{-2mm}
\caption{Accuracy of the global model using all the baseline aggregation methods without and with FLANDERS on \textit{MNIST} and \textit{Fashion-MNIST} ($r=0.2$). The best results are typed in boldface.}
\label{tab:increment-20-mnist-fmnist}
%\vspace{2mm}
%\scalebox{0.8}{
\begin{tabular}{|c|cccc|cccc|}
\hline
\multirow{2}{*}{\textbf{Aggregation}} & \multicolumn{4}{c|}{\textit{MNIST}} & \multicolumn{4}{c|}{\textit{Fashion-MNIST}} \\
\cline{2-9}
              &  GAUSS         &  LIE          &  OPT          &  AGR-MM       &  GAUSS        &  LIE          &  OPT          &  AGR-MM      \\ 
\hline
FedAvg        &  $0.18$       &  $0.12$       &  ${\bf 0.63}$ &  $0.34$       &  $0.25$       & $0.10$        &  $0.56$       &  $0.16$       \\
+ FLANDERS    &  ${\bf 0.86}$ &  ${\bf 0.83}$ &  $0.62$       &  ${\bf 0.85}$ &  ${\bf 0.69}$ & ${\bf 0.64}$  &  ${\bf 0.58}$ &  ${\bf 0.63}$ \\
\hline
FedMedian     &  ${\bf 0.81}$ &  $0.57$       &  ${\bf 0.61}$ &  $0.63$       &  $0.70$       &  $0.63$       &  ${\bf 0.68}$ &  $0.62$       \\
+ FLANDERS    &  ${\bf 0.81}$ &  ${\bf 0.86}$ &  $0.58$       &  ${\bf 0.85}$ &  ${\bf 0.71}$ &  ${\bf 0.73}$ &  $0.64$       &  ${\bf 0.72}$ \\
\hline
TrimmedMean   &  $0.78$       &  $0.57$       &  ${\bf 0.76}$ &  $0.48$       &  $0.65$       &  $0.58$       &  $0.61$       &  $0.55$       \\
+ FLANDERS    &  ${\bf 0.83}$ &  ${\bf 0.82}$ &  $0.75$       &  ${\bf 0.77}$ &  ${\bf 0.69}$ &  ${\bf 0.69}$ &  ${\bf 0.63}$ &  ${\bf 0.70}$ \\
\hline
Multi-Krum    &  $0.73$       &  ${\bf 0.86}$ &  $0.83$       &  ${\bf 0.80}$ &  ${\bf 0.66}$ &  ${\bf 0.67}$ &  $0.65$       &  $0.64$       \\
+ FLANDERS    &  ${\bf 0.87}$ &  $0.85$       &  ${\bf 0.85}$ &  $0.77$       &  $0.65$       &  $0.64$       &  ${\bf 0.69}$ &  ${\bf 0.66}$ \\
\hline
Bulyan        &  $0.81$       &  ${\bf 0.88}$ &  ${\bf 0.85}$ &  $0.84$       &  ${\bf 0.67}$ &  ${\bf 0.72}$ &  ${\bf 0.70}$ &  ${\bf 0.76}$ \\
+ FLANDERS    &  ${\bf 0.83}$ &  ${\bf 0.88}$ &  $0.84$       &  ${\bf 0.87}$ &  $0.62$       &  $0.66$       &  $0.66$       &  $0.68$       \\
\hline
DnC           &  $0.20$       &  $0.14$       &  ${\bf 0.69}$ &  $0.30$       &  $0.25$       &  $0.18$       &  $0.59$       &  $0.17$       \\
+ FLANDERS    &  ${\bf 0.88}$ &  ${\bf 0.87}$ &  $0.65$       &  ${\bf 0.88}$ &  ${\bf 0.61}$ &  ${\bf 0.64}$ &  ${\bf 0.62}$ &  ${\bf 0.66}$ \\
\hline
\end{tabular}
%}
\end{table*}

\begin{table*}[htb!]
\centering
%\vspace{-2mm}
\caption{Accuracy of the global model using all the baseline aggregation methods without and with FLANDERS on \textit{CIFAR-10} and \textit{CIFAR-100} ($r=0.2$). The best results are typed in boldface.}
\label{tab:increment-20-cifar10-cifar100}
%\vspace{2mm}
%\scalebox{0.8}{
\begin{tabular}{|c|cccc|cccc|}
\hline
\multirow{2}{*}{\textbf{Aggregation}} & \multicolumn{4}{c|}{\textit{CIFAR-10}} & \multicolumn{4}{c|}{\textit{CIFAR-100}} \\
\cline{2-9}
              &  GAUSS         &  LIE          &  OPT          &  AGR-MM       &  GAUSS        &  LIE          &  OPT          &  AGR-MM      \\ 
\hline
FedAvg        &  $0.10$       &  $0.10$       &  $0.23$       &  $0.10$       &  $0.01$       &  $0.01$       &  $0.01$       &  $0.02$ \\
+ FLANDERS    &  ${\bf 0.38}$ &  ${\bf 0.37}$ &  ${\bf 0.28}$ &  ${\bf 0.36}$ &  ${\bf 0.07}$ &  ${\bf 0.05}$ &  ${\bf 0.05}$ &  ${\bf 0.06}$ \\
\hline
FedMedian     &  ${\bf 0.34}$ &  $0.24$       &  ${\bf 0.24}$ &  $0.21$       &  $0.01$       &  $0.04$       &  $0.01$       &  $0.03$ \\
+ FLANDERS    &  ${\bf 0.34}$ &  ${\bf 0.28}$ &  $0.17$       &  ${\bf 0.32}$ &  $0.10$       &  ${\bf 0.10}$ &  ${\bf 0.10}$ &  ${\bf 0.11}$ \\
\hline
TrimmedMean   &  ${\bf 0.35}$ &  $0.24$       &  ${\bf 0.23}$ &  $0.23$       &  $0.01$       &  $0.03$       &  $0.01$       &  $0.03$ \\
+ FLANDERS    &  ${\bf 0.35}$ &  ${\bf 0.33}$ &  $0.22$       &  ${\bf 0.35}$ &  $0.11$       &  ${\bf 0.10}$ &  ${\bf 0.10}$ &  ${\bf 0.10}$ \\
\hline
Multi-Krum    &  $0.36$       &  $0.35$       &  $0.27$       &  $0.34$       &  $0.07$       &  $0.06$       &  $0.07$       &  $0.07$ \\
+ FLANDERS    &  ${\bf 0.42}$ &  ${\bf 0.44}$ &  ${\bf 0.35}$ &  ${\bf 0.42}$ &  ${\bf 0.10}$ &  ${\bf 0.09}$ &  ${\bf 0.10}$ &  ${\bf 0.09}$ \\
\hline
Bulyan        &  $0.40$       &  $0.39$       &  $0.34$       &  $0.39$       &  ${\bf 0.10}$ &  ${\bf 0.10}$ &  ${\bf 0.10}$ &  ${\bf 0.08}$ \\
+ FLANDERS    &  ${\bf 0.43}$ &  ${\bf 0.43}$ &  ${\bf 0.36}$ &  ${\bf 0.42}$ &  $0.07$       &  $0.06$       &  $0.06$       &  ${\bf 0.08}$ \\
\hline
DnC           &  $0.11$        &  $0.10$       &  $0.27$       &  $0.15$       &  $0.01$       &  $0.01$       &  $0.01$       &  $0.02$ \\
+ FLANDERS    &  ${\bf 0.43}$       &  ${\bf 0.44}$ &  ${\bf 0.34}$ &  ${\bf 0.43}$ &  ${\bf 0.07}$ &  ${\bf 0.07}$ &  ${\bf 0.07}$ &  ${\bf 0.08}$ \\
\hline
\end{tabular}
%}
\end{table*}

\begin{table*}[htb!]
\centering
%\vspace{-2mm}
\caption{Accuracy of the global model using all the baseline aggregation methods without and with FLANDERS on \textit{MNIST} and \textit{Fashion-MNIST} ($r=0.6$). The best results are typed in boldface.}
\label{tab:increment-60-mnist-fmnist}
%\vspace{2mm}
%\scalebox{0.8}{
\begin{tabular}{|c|cccc|cccc|}
\hline
\multirow{2}{*}{\textbf{Aggregation}} & \multicolumn{4}{c|}{\textit{MNIST}} & \multicolumn{4}{c|}{\textit{Fashion-MNIST}} \\
\cline{2-9}
              &  GAUSS        &  LIE          &  OPT          &  AGR-MM       &  GAUSS        &  LIE          &  OPT          &  AGR-MM       \\ 
\hline
FedAvg      & $0.18$ & $0.12$ & $0.16$ & $0.15$ & $0.28$ & $0.10$ & $0.18$ & $0.10$ \\
+ FLANDERS  & ${\bf 0.78}$ & ${\bf 0.82}$ & ${\bf 0.79}$ & ${\bf 0.85}$ & ${\bf 0.69}$ & ${\bf 0.65}$ & ${\bf 0.65}$ & ${\bf 0.62}$  \\
\hline
FedMedian   & ${\bf 0.84}$ & $0.18$ & $0.12$ & $0.16$ & $0.70$ & $0.10$ & $0.13$ & $0.10$ \\
+ FLANDERS  & $0.73$ & ${\bf 0.81}$ & ${\bf 0.83}$ & $0{\bf .82}$ & ${\bf 0.72}$ & ${\bf 0.73}$ & ${\bf 0.71}$ & ${\bf 0.72}$ \\
\hline
TrimmedMean & $0.22$ & $0.12$ & $0.21$ & $0.13$ & $0.33$ & $0.10$ & $0.18$ & $0.10$ \\
+ FLANDERS  & ${\bf 0.76}$ & ${\bf 0.83}$ & ${\bf 0.83}$ & ${\bf 0.78}$ & ${\bf 0.72}$ & ${\bf 0.70}$ & ${\bf 0.71}$ & ${\bf 0.68}$ \\
\hline
Multi-Krum  & $0.85$ & $0.24$ & $0.27$ & $0.12$ & $0.65$ & $0.10$ & $0.11$ & $0.10$ \\
+ FLANDERS  & ${\bf 0.89}$ & ${\bf 0.84}$ & ${\bf 0.84}$ & ${\bf 0.88}$ & ${\bf 0.71}$ & ${\bf 0.70}$ & ${\bf 0.68}$ & ${\bf 0.67}$ \\
\hline
Bulyan      & N/A & N/A & N/A & N/A & N/A & N/A & N/A & N/A \\
+ FLANDERS  & ${\bf 0.86}$ & ${\bf 0.87}$ & ${\bf 0.81}$ & ${\bf 0.88}$ & ${\bf 0.65}$ & ${\bf 0.69}$ & ${\bf 0.66}$ & ${\bf 0.67}$ \\
\hline
DnC         & $0.19$ & $0.10$ & $0.33$ & $0.24$ & $0.28$ & $0.10$ & $0.12$ & $0.10$ \\
+ FLANDERS  & ${\bf 0.89}$ & ${\bf 0.84}$ & ${\bf 0.88}$ & ${\bf 0.85}$ & ${\bf 0.66}$ & ${\bf 0.65}$ & ${\bf 0.68}$ & ${\bf 0.66}$ \\
\hline
\end{tabular}
%}
\end{table*}

\begin{table*}[htb!]
\centering
%\vspace{-2mm}
\caption{Accuracy of the global model using all the baseline aggregation methods without and with FLANDERS on \textit{CIFAR-10} and \textit{CIFAR-100} ($r=0.6$). The best results are typed in boldface.}
\label{tab:increment-60-cifar10-cifar100}
%\vspace{2mm}
%\scalebox{0.8}{
\begin{tabular}{|c|cccc|cccc|}
\hline
\multirow{2}{*}{\textbf{Aggregation}} & \multicolumn{4}{c|}{\textit{CIFAR-10}} & \multicolumn{4}{c|}{\textit{CIFAR-100}} \\
\cline{2-9}
              &  GAUSS         &  LIE          &  OPT          &  AGR-MM       &  GAUSS        &  LIE          &  OPT          &  AGR-MM      \\ 
\hline
FedAvg      & $0.10$ & $0.10$ & $0.10$ & $0.10$ & $0.01$ & $0.01$ & $0.01$ & $0.01$ \\
+ FLANDERS  & ${\bf 0.36}$ & ${\bf 0.36}$ & ${\bf 0.34}$ & ${\bf 0.35}$ & ${\bf 0.08}$ & ${\bf 0.09}$ & ${\bf 0.08}$ & ${\bf 0.09}$ \\
\hline
FedMedian   & ${\bf 0.34}$ & $0.10$ & $0.11$ & $0.10$ & $0.01$ & $0.01$ & $0.01$ & $0.01$ \\
+ FLANDERS  & $0.31$ & ${\bf 0.31}$ & ${\bf 0.30}$ & ${\bf 0.33}$ & $0.11$ & ${\bf 0.10}$ & ${\bf 0.11}$ & ${\bf 0.10}$ \\
\hline
TrimmedMean & $0.12$ & $0.10$ & $0.10$ & $0.10$ & $0.01$ & $0.01$ & $0.01$ & $0.01$ \\
+ FLANDERS  & ${\bf 0.35}$ & ${\bf 0.28}$ & ${\bf 0.33}$ & ${\bf 0.29}$ & $0.10$ & ${\bf 0.11}$ & ${\bf 0.11}$ & ${\bf 0.12}$ \\
\hline
Multi-Krum  & $0.10$ & $0.35$ & $0.10$ & $0.13$ & $0.09$ & $0.01$ & $0.01$ & $0.01$ \\
+ FLANDERS  & ${\bf 0.41}$ & ${\bf 0.40}$ & ${\bf 0.40}$ & ${\bf 0.40}$ & ${\bf 0.12}$ & ${\bf 0.12}$ & ${\bf 0.11}$ & ${\bf 0.10}$ \\
\hline
Bulyan      & N/A & N/A & N/A & N/A & N/A & N/A & N/A & N/A \\
+ FLANDERS  & ${\bf 0.43}$ & ${\bf 0.41}$ & ${\bf 0.42}$ & ${\bf 0.42}$ & ${\bf 0.09}$ & ${\bf 0.08}$ & ${\bf 0.10}$ & ${\bf 0.08}$ \\
\hline
DnC         & $0.10$ & $0.10$ & $0.10$ & $0.11$ & $0.01$ & $0.01$ & $0.01$ & $0.01$ \\
+ FLANDERS  & ${\bf 0.42}$ & ${\bf 0.42}$ & ${\bf 0.42}$ & ${\bf 0.42}$ & ${\bf 0.09}$ & ${\bf 0.09}$ & ${\bf 0.09}$ & ${\bf 0.10}$ \\
\hline
\end{tabular}
%}
\end{table*}



Finally, in Table~\ref{tab:increment-0}, we also assess the impact of FLANDERS in an attack-free scenario (i.e., when $r=0$). In this setting, no clear winning strategy emerges. Sometimes, FLANDERS has a detrimental effect on the global model's accuracy with a standard aggregation mechanism (e.g., see FedAvg with the \textit{MNIST} dataset). In other instances, however, FLANDERS improves the global model's accuracy when paired with robust aggregation schemes (e.g., see Bulyan with the \textit{MNIST} dataset).

%\begin{table*}[htb!]
%\centering
%\caption{Accuracy of the global model using FedAvg coupled with FLDetector and FLANDERS on CIFAR-100 ($r = 0.2$).}
%\label{tab:increment-fld-cifar100}
%%\vspace{2mm}
%\scalebox{0.9}{
%\begin{tabular}{|c|cccc|}
%\hline
%\multirow{2}{*}{\textbf{Aggregation}} & \multicolumn{4}{c|}{\textit{CIFAR-100}} \\
%\cline{2-5}
%                    &  GAUSS         &  LIE          &  OPT          &  AGR-MM   \\ 
%\hline
%FedAvg              & $0.01$ & $0.01$ & $0.01$ & $0.02$ \\
%FLDetector + FedAvg & $0.01$ & $0.01$ & $0.01$ & $0.02$ \\
%FLANDERS + FedAvg   & ${\bf 0.07}$ & ${\bf 0.05}$ & ${\bf 0.05}$ & ${\bf 0.06}$ \\
%\hline
%\end{tabular}
%}
%\end{table*}

\begin{table*}[htb!]
\centering
%\vspace{-2mm}
\caption{Accuracy of the global model using all the baseline aggregation methods without and with FLANDERS ($r=0$). The best results are typed in boldface.}
\label{tab:increment-0}
%\vspace{2mm}
%\scalebox{0.9}{
\begin{tabular}{|c|c|c|c|c|}
\hline
\multirow{1}{*}{\textbf{Aggregation}} & \multicolumn{1}{c|}{\textit{MNIST}} & \multicolumn{1}{c|}{\textit{Fashion-MNIST}} & \multicolumn{1}{c|}{\textit{CIFAR-10}} & \multicolumn{1}{c|}{\textit{CIFAR-100}} \\
\cline{2-5}
\hline
FedAvg                  & ${\bf 0.86}$ & $0.63$ & $0.35$ & ${\bf 0.06}$ \\
+ FLANDERS              & $0.83$ & ${\bf 0.68}$ & ${\bf 0.36}$ & $0.05$ \\
\hline
FedMedian               & ${\bf 0.84}$ & ${\bf 0.71}$ & $0.31$ & ${\bf 0.10}$ \\
+ FLANDERS              & $0.78$ & $0.68$ & ${\bf 0.33}$ & ${\bf 0.10}$ \\
\hline
TrimmedMean             & ${\bf 0.82}$ & $0.69$ & ${\bf 0.33}$ & ${\bf 0.12}$ \\
+ FLANDERS              & $0.80$ & ${\bf 0.70}$ & $0.32$ & $0.11$ \\
\hline
MultiKrum               & $0.72$ & ${\bf 0.65}$ & $0.34$ & $0.05$ \\
+ FLANDERS              & ${\bf 0.80}$ & $0.64$ & ${\bf 0.44}$ & ${\bf 0.08}$ \\
\hline
Bulyan                  & $0.85$ & $0.65$ & ${\bf 0.43}$ & $0.05$ \\
+ FLANDERS              & ${\bf 0.90}$ & ${\bf 0.70}$ & $0.42$ & ${\bf 0.06}$ \\
\hline
DnC                     & $0.81$ & $0.62$ & ${\bf 0.44}$ & $0.06$ \\
+ FLANDERS              & ${\bf 0.86}$ & ${\bf 0.64}$ & ${\bf 0.44}$ & ${\bf 0.07}$ \\
\hline
\end{tabular}
%}
\end{table*}





\begin{table}[htb!]
\centering
\caption{Accuracy of the global model using Multi-FLANDERS + Multi-Krum over various numbers of malicious clients.}
\label{tab:robustness-1}
%\vspace{2mm}
\scalebox{0.75}{
\begin{tabular}{|c|cccc|cccc|cccc|cccc|}
\hline
\multirow{2}{*}{\textbf{Attack}} & \multicolumn{4}{c|}{\textit{MNIST}} & \multicolumn{4}{c|}{\textit{Fashion-MNIST}} & \multicolumn{4}{c|}{\textit{CIFAR-10}} & \multicolumn{4}{c|}{\textit{CIFAR-100}} \\
\cline{2-17}
        & 0      & 20     & 60     & 80     & 0      & 20     & 60     & 80     & 0      & 20     & 60     & 80     & 0      & 20     & 60     & 80     \\ 
\hline
GAUSS   & $0.80$ & $0.87$ & $0.89$ & $0.87$ & $0.64$ & $0.65$ & $0.71$ & $0.69$ & $0.44$ & $0.42$ & $0.41$ & $0.38$ & $0.08$ & $0.10$ & $0.12$ & $0.11$ \\
LIE     & $0.80$ & $0.85$ & $0.84$ & $0.90$ & $0.64$ & $0.64$ & $0.70$ & $0.68$ & $0.44$ & $0.44$ & $0.40$ & $0.38$ & $0.08$ & $0.09$ & $0.12$ & $0.10$ \\
OPT     & $0.80$ & $0.85$ & $0.84$ & $0.88$ & $0.64$ & $0.69$ & $0.68$ & $0.72$ & $0.44$ & $0.35$ & $0.40$ & $0.39$ & $0.08$ & $0.10$ & $0.11$ & $0.10$ \\
AGR-MM  & $0.80$ & $0.77$ & $0.88$ & $0.89$ & $0.64$ & $0.66$ & $0.67$ & $0.68$ & $0.44$ & $0.42$ & $0.40$ & $0.40$ & $0.08$ & $0.09$ & $0.10$ & $0.11$ \\
\hline
\end{tabular}
}
%\vspace{-4mm}
\end{table}


In Figure~\ref{fig:accuracy-over-time} we compare the accuracy of the global model when using FLANDERS (left) and when using FedAvg (right). The left figure shows how the evolution of the accuracy over multiple rounds remains stable and similar to the one without any attack (dashed line), while on the right the accuracy drops irremediably.

\begin{figure}[htb!]%{0.4\textwidth}
     \centering
     \includegraphics[width=0.8\textwidth]{./img/accuracy-over-time-mnist-5}
     \caption{FedAvg with FLANDERS (left) vs. "vanilla" FedAvg (right). Accuracy of the global model in each FL round under all attack strategies on the \textit{MNIST} dataset, with $80\%$ of malicious clients. Attack starts at round $t=3$.}
     \label{fig:accuracy-over-time}
     %\vspace{-6mm}
\end{figure}

\subsection{Cost-Benefit Analysis}
\label{app:cost-benefit}
Below, we report the cost-benefit analysis described in Section~\ref{subsec:eval}. 
Specifically, Figure~\ref{fig:tradeoff} shows the trade-off between the global model's accuracy and its training time on the \textit{MNIST} and \textit{CIFAR-10} datasets when using FedAvg and Bulyan aggregation without and with FLANDERS as a filter.
In general, FLANDERS introduces an overhead in the training time, which is, however, compensated by the increased level of accuracy.


\begin{figure}[htb!]
     \centering
     \includegraphics[width=\textwidth]{./img/tradeoff}
     \caption{Accuracy vs. training time of FedAvg and Bulyan compared with their corresponding versions with FLANDERS as a filter for the \textit{MNIST} (left) and \textit{CIFAR-10} (right) datasets.}
     \label{fig:tradeoff}
\end{figure}

\subsection{Robustness against Adaptive Attacks}
\label{app:adaptive-attacks}
In this section, we further validate the robustness of FLANDERS against adaptive attacks. We consider a scenario where malicious clients are aware that the FL server uses our method as a pre-aggregation filter. 
Specifically, we focus on two levels of knowledge. The first scenario assumes that malicious clients tentatively guess the subset of parameters ($\widetilde{d}$) used by the FL server to estimate the MAR forecasting model. We refer to this setting as \textit{non-omniscient}.
The second, more challenging as well as unrealistic scenario assumes that malicious clients know \textit{exactly} which parameters are used by the FL server. We call this second scenario \textit{omniscient}. Obviously, the latter penalizes FLANDERS way more than the former. 
\\
Specifically, we perform our experiments over $T=20$ rounds with $m=20$ clients, of which $r=0.2$ ($b=5$) are malicious. The attacker constructs a matrix $M = b \times \widetilde{d}$ using the local models generated by the corrupted clients. This matrix $M$ is then passed as input to the same forecasting model, MAR, that the server uses to determine the legitimacy of local models. The attacker, instead, substitutes the legitimate parameters with those estimated by MAR, exploiting the fact that these estimations do not perform like a legitimate local model. This substitution ultimately hurts the accuracy of the global model once the parameters are aggregated. 
\\
As introduced above, we first assume that the attacker is \textit{non-omniscient}, meaning it does not know which parameters the server has selected for the MAR estimation. Instead, the attacker selects the last layer as $\widetilde{d}$. 
Afterward, we consider an \textit{omniscient} attacker who exploits the knowledge of the parameters selected by the server. 
In Table~\ref{tab:nonomn-adaptive}, we show the results of the non-omniscient scenario, where FLANDERS + Multi-Krum outperforms all other baselines on all three datasets. Table~\ref{tab:omn-adaptive}, on the other hand, refers to the omniscient scenario and demonstrates a different pattern, where FedAvg and Multi-Krum alone perform better than when coupled with FLANDERS.
This may be because FLANDERS consistently filters out legitimate local models in favor of corrupted ones. When using FedAvg, the impact of corrupted parameters is mitigated by averaging a larger number of legitimate models and because the corrupted models' parameter values are not too different, unlike in methods like OPT. On the other hand, Multi-Krum selects parameters with more nearby neighbors, and with only $b=5$ corrupted clients, legitimate models likely still have more and closer neighbors, effectively defending against our adaptive attack.

\begin{table}[htb!]
\centering
\vspace{-2mm}
\caption{Accuracy of the global model using FedAvg and Multi-Krum, with and without FLANDERS, under the \textit{\textbf{non-}omniscient} adaptive attack.}
\label{tab:nonomn-adaptive}
\begin{tabular}{|c|c|c|c|}
    \hline
    Strategy & \textit{MNIST} & \textit{Fashion-MNIST} & \textit{CIFAR-10} \\
    \hline
    FedAvg                  & $0.84$ & $0.68$ & $0.45$ \\
    FLANDERS + FedAvg       & $0.82$ & $0.67$ & $0.43$ \\
    %\hline
    Multi-Krum              & $0.87$ & ${\bf 0.72}$ & $0.41$ \\
    FLANDERS + Multi-Krum   & ${\bf 0.90}$ & ${\bf 0.72}$ & ${\bf 0.47}$ \\
    \hline
\end{tabular}
\vspace{-4mm}
\end{table}

\begin{table}[htb!]
\centering
\caption{Accuracy of the global model using FedAvg and Multi-Krum, with and without FLANDERS, under the \textit{omniscient} adaptive attack.}
\label{tab:omn-adaptive}
\begin{tabular}{|c|c|c|c|}
    \hline
    Strategy & \textit{MNIST} & \textit{Fashion-MNIST} & \textit{CIFAR-10} \\
    \hline
    FedAvg                  & $0.78$ & ${\bf 0.68}$ & ${\bf 0.25}$ \\
    FLANDERS + FedAvg       & $0.65$ & $0.61$ & $0.10$ \\
    %\hline
    Multi-Krum              & ${\bf 0.86}$ & ${\bf 0.68}$ & $0.23$ \\
    FLANDERS + Multi-Krum   & $0.73$ & $0.60$ & $0.10$ \\
    \hline
\end{tabular}
\end{table}

\vspace{-2mm}
\subsection{Random Client Selection}
\label{app:client-selection}
So far, we have assumed that the FL server selects \textit{all} available clients at each round, i.e., $|\mathcal{C}^{(t)}| = m = K~\forall t\in \{1,2, \ldots, T\}$.
In this section, we investigate the scenario where the number of clients selected at each round remains fixed ($|\mathcal{C}^{(t)}| = m$), but now the FL server chooses a \textit{random} subset of the available clients, i.e., $1 \leq m < K$.
\\
We experiment with $K=10$ clients, of which $b=2$ are malicious (running the AGR-MM attack). Each client has a non-iid sample of the \textit{MNIST} dataset, where we fix $k=max(1,K*c - b)$. At each FL round, $m=K*c$ clients are randomly chosen ($c \in \{0.2,0.5,1.0\}$). 
\begin{table}[htb!]
    \centering
    \caption{Accuracy of the global model on the \textit{MNIST} dataset using FLANDERS + FedAvg, with $b=2$ attackers out of $K=10$ clients, under AGR-MM attack, and with variable clients selected for training each round ($c$).}
    \begin{tabular}{|c|ccc|}
    \hline
    $c$ & ${\bf 0.2}$ & ${\bf 0.5}$ & ${\bf 1.0}$ \\
    \hline
    Accuracy & $0.75$ & $0.83$ & $0.92$ \\
    \hline
    \end{tabular}
    \label{tab:random-client-selection}
\end{table}
\\
The results in Table~\ref{tab:random-client-selection} show that the accuracy of the global model improves as the ratio of sampled clients increases, while FLANDERS remains robust to AGR-MM attacks.



\section{Limitations and Future Work}
\label{app:limitations}
The limitations of this work can be categorized into the following areas, which we intend to address in a forthcoming extension of this research.

\subsection{Efficiency/Feasibility}
As we discussed in Section~5.3 of the main body and Appendix~\ref{app:defenses} above, FLANDERS may suffer from a high computational cost that could limit its deployment in practice.
This concern holds particularly true for \textit{cross-device} FL configurations encompassing millions of edge devices. Conversely, the impact on \textit{cross-silo} FL scenarios would be notably less pronounced. 
However, as we have introduced FLANDERS as a versatile and robust aggregation approach applicable to diverse FL setups (cross-silo and cross-device), there are implementation techniques available to mitigate its complexity. For instance, methods like random parameter sampling~\cite{shejwalkar2021ndss} can be employed, and we have already incorporated them.
Still, we plan to enhance the scalability of FLANDERS further in future work. 
For example, we could replace the standard matrix inversion algorithm with the more efficient Coppersmith-Winograd algorithm~\cite{coppersmith1990jsc} and find the optimal frequency for re-estimating FLANDERS' parameters, instead of performing it during each FL round.

\subsection{Potential Privacy Leakage}
\label{app:limitations-privacy}
In the standard FL setup, the central server must access the local model updates sent by each client (e.g., even to perform a simple FedAvg). 
Therefore, our approach, FLANDERS, does not need additional permission nor violate any privacy constraints beyond what any other FL server could already do. 
Indeed, the most effective robust aggregation schemes, such as Krum and Bulyan, like FLANDERS, assume that the central server is a \textit{trusted} entity. 
However, if this assumption does not hold, the scenario will change. For instance, if the server operates as an ``honest but curious'' entity, thoroughly examining the local model parameters received for training the outlier detection model (such as MAR) could unveil sensitive details that the server might exploit, potentially inferring information about each client's local data distribution.


\subsection{Benchmarking}
We extensively validate FLANDERS with an exhaustive set of experiments. We compared it against six robust baselines amongst the most powerful at the time of writing, along with standard FedAvg.
\\
In addition, robust federated aggregation is a hot research topic; keeping pace with the massive body of work that has been flourishing is challenging.
Hence, we might have missed considering some other methods. 
\textit{\textbf{However, we believe that the value of our work still stands. In this regard, our contribution is clear: We are the first to frame the problem of detecting untargeted model poisoning attacks on FL as a matrix-valued time series anomaly detection task and to propose a method effective under severe attack settings, as opposed to existing baselines.}} 
We attribute such a capability to two key factors: $(i)$ FLANDERS operates without knowing a priori the proportion of corrupted clients, and $(ii)$ it naturally embodies temporal dependencies between intra- and inter-client updates, quickly recognizing local model drifts caused by evil players.
\\
Ideally, the research community should agree on a standard framework (i.e., a benchmark) as a test bed for evaluating any new FL robust aggregation scheme where all the existing methods are implemented. This way, no baseline would be missing. 
As a first step in this direction, to encourage reproducibility, we have implemented our method \textit{as well as \textbf{all} the considered baselines} in a real-world FL simulation environment (Flower).
\\
Finally, FLANDERS relies on several hyperparameters that can impact its performance. For instance, we initially set the matrix autoregressive model order $w$ to 1 but can explore larger values to capture broader temporal dependencies. Also, a more in-depth examination of Table~\ref{tab:defenses} parameters (e.g., $l$, $k$, and $\delta$) could yield insights.

\end{document}
\endinput
%%
%% End of file `sample-sigconf.tex'.