\begin{abstract}

We present the InterviewBot that dynamically integrates conversation history and customized topics into a coherent embedding space to conduct 10 mins hybrid-domain (open and closed) conversations with foreign students applying to U.S. colleges for assessing their academic and cultural readiness.
To build a neural-based end-to-end dialogue model, 7,361 audio recordings of human-to-human interviews are automatically transcribed, where 440 are manually corrected for finetuning and evaluation.
To overcome the input/output size limit of a transformer-based encoder-decoder model, two new methods are proposed, context attention and topic storing, allowing the model to make relevant and consistent interactions.
Our final model is tested both statistically by comparing its responses to the interview data and dynamically by inviting professional interviewers and various students to interact with it in real-time, finding it highly satisfactory in fluency and context awareness.



%We introduce a real-time end-to-end dialogue system that conducts 10-min interviews with foreign students applying to colleges in the US to assess their academic and cultural readiness.
% To handle speaker diarization errors from the automatic transcription, a multi-task learning model is first trained on large pseudo-generated data, created by statistical heuristics observed in 440 interviews that we annotate, then finetuned on our annotation, achieving the F1-score of 93.8\%.
% This diarization model is applied to auto-correct the remaining 6,921 dialogues that are used to train encoder-decoder models in our dialogue system.
%To train and evaluate our model, 7,361 audio recordings are collected and transcribed, consisting of 6,921 raw conversations, and 440 manually annotated conversations.
%To overcome the input size constraint of transformer-based encoder-decoder models, two new methods are proposed, context attention and topic storing, which integrate the context in the conversation history and essential topics brought up during the interview into one embedding space, thus, allowing the model to make relevant and consistent interactions.
%The effectiveness of these methods is clearly illustrated by our ablation studies. 
%We conducted static evaluation on generated responses, as well as real-time evaluation by creating a text-based interface for people to interact with the model and requesting for scoring the model for overall satisfaction of the interactions.
%We invite professional interviewers and various students to test our dialogue system, who find it highly satisfactory in fluency and context awareness.

%In this paper, we introduce an Interviewbot that integrates conversation history and critical interview questions (generated on the fly) as context intended to conduct hybrid-domain (open and closed domains) conversations with high school applicants to U.S. colleges. 
%Our model is based on the Facebook Blenderbot, while overcoming its lack of ability to handle arbitrary context and long input (> 128 tokens). 
%In addition, to prepare the experiment dataset, we build a context-based multi-task joint model to perform a speaker diarization task to automaticaly correct diarization errors in the text transcription dataset.
%To achieve that we pseudo-generated more conversations to mimic the characteristics of the true interview data, using external data sources. 
%Our speaker diarization task achieves 0.938 in F1 score. 
%The Interviewbot model is trained by pseudo-generated data and fine-tuned by the golden applicant interview data. 
%To evaluate, we set up an mock interview interface to interact with real users. 
%The users are asked to give feedback on several aspects and an overall impression of the conversations. 
%Based on the evaluation, our interviewbot has achieved high satisfaction of users with fluency and context awareness. 
\end{abstract}