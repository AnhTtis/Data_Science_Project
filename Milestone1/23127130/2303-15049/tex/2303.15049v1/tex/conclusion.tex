\section{Conclusion}
\label{sec:conclusion}

Our InterviewBot is a model-based dialogue system equipped with contextual awareness and topic sensitivity that conducts college admission interviews. 
Questions covering diverse topics and discussions in extended follow-ups are carried along the conversations, which have been assessed by professional interviewers and student volunteers. 
The average satisfaction score of 3.5 projects prevailing deployment of the InterviewBot for thousands of college applicants, especially for international students. 

With promising future applications, however, the current version of the InterviewBot has two major limitations. 
First, the early ending in Table~\ref{tab:interviewbot-error-analysis} still happens, where an ending utterance gets generated after an insufficient amount of turns, in which case, the interview may not cover critical aspects of the\LN applicant.
Second, the bot makes good follow-ups to various topics; however, it needs to derive deeper discussions with more details.

In the future work, the main focus is to enrich the follow-up discussions on topics or sub-topics during interactions by training the InterviewBot with more structured topic-flow materials. This task would indirectly alleviate the early ending issue by deepening the discussions on certain topics. 

% We introduced InterviewBot, a real-time end-to-end dialogue system conducting admission interviews with students applying to colleges.
% The main contribution is that a context and topic-sensitive dialogue model is established. 
% It can encode arbitrarily long utterances and context, as well as identifying topical questions to conduct engaging conversations. 
% Our InterviewBot is currently deployed to interact with selective users who have given satisfactory scores.
% To the best of our knowledge, it is the first end-to-end chatbot that is deployed to a real-time system.

%In this paper, we proposed Interviewbot, which is developed on top of the Blenderbot from Facebook. 
%Interviewbot has the ability to integrate contextual information, including previous utterances and previously generated key questions, which enables it to have contextual awareness.
%We made our contribution to developing a pipeline to process noisy conversation data and train a dialogue system for interviewing high school students. 
%First we developed a text-based joint diarization model to process data from one-channel recordings. 
%A context-aware conversational interviewbot is developed to take arbitrary long utterances and fuse information from different perspectives to be condensed representation to generate responses. 
%The current version of Interviewbot is deployed to interact with both interviewees and students obtaining an average rating of 3.47, which is promising for future deployment in real applications.

%Coping with a real industry situation, a context-based joint diarization model is also developed to process data from one-channel recordings. 

