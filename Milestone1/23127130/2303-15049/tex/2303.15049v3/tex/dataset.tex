\section{Materials and Methods}
\label{sec:materials_methods}

\subsection{Interview Dataset}
\label{sec:dataset}

\noindent Audio recordings of 7,361 interviews are automatically transcribed with speaker identification by the online tool RevAI,\footnote{\url{https://www.rev.ai}} where 440 are manually corrected on speaker ID assignment for finetuning and evaluation of our models (Table~\ref{tab:annotated-data}).
Each recording contains an average of a $\approx$15-min long dialogue between an interviewer and an interviewee.
The interviews were conducted by 67 professionals in 2018 - 2022.
The largest age group of interviewees is 18-years-old with 59.3\%, followed by 17-years-old with 29.4\%.
The male-to-female ratio is 1.2:1.
The major country of origin is China with 81.4\% followed by Belgium with 10.5\%, alongside 37 other countries.
Table \ref{tab:annotated-data} provides detailed demographics of the interviewees.


All recordings are transcribed into text and speakers are identified automatically.
For speech recognition, three tools from Amazon,\footnote{\url{https://aws.amazon.com/transcribe}} Google,\footnote{\url{https://cloud.google.com/speech-to-text}} and RevAI\footnote{\url{https://www.rev.ai}} are assessed on 5 recordings for speaker diarization, achieving the F1-scores of 66.3\%, 50.1\%, and 72.7\%, respectively.\footnote{The same metric as in Table~\ref{tab:diarization-results} is used for this evaluation.}


\begin{table}[htbp!]
\caption{Distributions of our data. D: num of dialogues, U: avg-num of utterances per dialogue, S1/S2: avg-num of tokens per utterance by interviewer/interviewee. \texttt{TRN}/\texttt{DEV}/\texttt{TST}: training/development/evaluation (annotated) sets. \texttt{RAW}: unannotated set (auto-transcribed).}
\vspace{0.5em}
\centering\small { %\resizebox{\columnwidth}{!}{
%\begin{tabular}{cr\CSP{1.8}r|r\CSP{1.8}r\CSP{1.8}r\CSP{1.8}r||r\CSP{1.8}r\CSP{1.8}r\CSP{1.8}r|r\CSP{1.8}r\CSP{1.8}r}
\begin{tabular}{crrrr}
\toprule
 & \multicolumn{1}{c}{\bf D} & \multicolumn{1}{c}{\bf U} & \multicolumn{1}{c}{\bf S1} & \multicolumn{1}{c}{\bf S2} \\
\midrule
\tt TRN &   140 & 43.8 & 39.3 & 64.0 \\
\tt DEV & 150 & 45.0 & 36.2 & 60.3 \\
\tt TST &   150 & 44.3 & 37.8 & 61.3 \\
\midrule
\tt RAW & 6,921 & 40.4 & 41.5 & 67.6 \\
\bottomrule
\end{tabular}}
\label{tab:annotated-data}
\end{table}


%\subsection{Interviewee Demographics}
%\label{app:interviewee-demographics}


\noindent Figure~\ref{fig:interviewee-age-demographics} shows the distribution of the ages of applicants. 
Most interviewees are between 17~to~19, which is an accurate reflection of the ages of high school students applying to colleges.
Figure~\ref{fig:interviewee-country-demographics} shows the distribution of the applicants' countries of origin.
There are 38 countries in total.
The majority of applicants come from China.
Other major countries are Belgium, Bangladesh, Canada, India, and Belarus.
The gender distribution of applicants is shown in Figure \ref{fig:interviewee-gender-demographics}. 
The numbers of male and female applicants are close, with the exclusion of applicants not providing gender information.

\begin{figure}[htbp!]
\centering
\includegraphics[width=0.8\columnwidth]{img/age_demo.pdf}
\caption{The interviewee's age demographics.}
\label{fig:interviewee-age-demographics}
\end{figure}
