\begin{figure}[t]
    \centering
    \includegraphics[width=\columnwidth]{imgs/applications2.png}
    % \vspace{-0.5cm}
    \caption{
    \textbf{Scene Composition.} \textbf{Top:} The outcome of rendering 2500 materials and ficus objects together in single scenes.  \textbf{Bottom:} An AR application showcase of \methodname with four chairs in a real-world setting using an AR/VR headset. See the \textbf{supplementary video} for more details.
    }
    \vspace{-0.5cm}
    \label{fig:app}
\end{figure}