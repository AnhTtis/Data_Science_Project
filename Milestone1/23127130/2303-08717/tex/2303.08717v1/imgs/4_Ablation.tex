\begin{figure}[t]
    \centering
    % \vspace{0.2cm}
    \includegraphics[width=1.0\linewidth]{imgs/reflection.pdf}
    % \vspace{-0.3cm}
    \caption{
    % \albert{Update caption}
    \textbf{Left:} We compare how PSNR varies as the number of texels per face increases for both synthetic and unbounded scenes.
    It is possible to trade-off disk space for higher quality renderings by increasing the number of texels.
    \textbf{Right:} We visualize color as a function of elevation (y-axis) and azimuth (x-axis) for a surface point x, y, z on the Materials scene from the Synthetic dataset, for different embedding dimensions. 
    % A larger embedding dimensionality allows our factorized representation to better approximate the ground truth reflectance.
    % \textbf{Rendering Quality \textit{vs.} Disk Space.}
    % We provide information on disk space and PSNR as the number of texels per face increases for both synthetic and unbounded scenes.
    % In this fashion, we show it is possible to trade-off disk space for higher quality renderings by increasing the number of texels.
    }
    \vspace{-0.4cm}
    \label{fig:ablation}
\end{figure}