\section{Experiments}\label{sec:experiments}
\vspace{-0.2cm}
\begin{table}[t!]
    \footnotesize
    \centering
    \setlength{\tabcolsep}{1.5pt}
    \begin{tabular}{c"c"c"c"c}
        \thickhline
        Device         & Type    & OS              & GPU                     & Browser \\ 
        \thickhline
        % iPhone 12       & Phone   & iOS             & Apple GPU               & Chrome  \\
        Samsung S21    & Phone   & Android 13      & Mali G78                & Chrome  \\
        Motorola G9    & Phone   & Android 11      & Adreno 610              & Chrome    \\
        Galaxy S6      & Tablet  & Android 13       & Mali G72 MP3               & Firefox   \\
        Dell           & Laptop  & Windows 10      & Integrated GPU           & Firefox  \\
        Gaming Lap.    & Laptop  & Windows 10      & NVIDIA GF RTX 2060      & Firefox   \\
        Desktop             & Desktop & Ubuntu 18.04    & NVIDIA GF RTX 3090      & Chrome \\
        Meta Quest P. & Headset & Oculus & Adreno 650 & $-$ \\
        \thickhline
    \end{tabular}
    % \vspace{-0.2cm}
     % \vspace{0.07cm}
    \caption{\textbf{Testing Devices.}
    % \sara{update tablet}
    We compare \methodname against other methods on a set of representative devices with a wide range of compute capabilities. 
    Our devices include low- to high-end mobile phones, tablets, laptops, desktop computers, and a VR headset. % \juan{are we keeping the G9?}
    % We show in this table all devices with corresponding characteristics.
    }
    \vspace{-0.45cm}
    \label{tab:0_hdw_specs_tb}
\end{table}

% \begin{table}[]
% \footnotesize
% \centering
% \setlength{\tabcolsep}{2.5pt}
% \begin{tabular}{lllll}\thickhline
% Device         & Type    & OS              & GPU                     & Power \\ \hline
% Iphone 12      & Phone   & iOS             & Apple GPU               & 7W    \\
% Motorola G9    & Phone   & Android 11      & Adreno 610              & 7W    \\
% Lenovo Tablet  & Tablet  & Android 9       & Adreno 505              & 10W   \\
% Dell           & Laptop  & Windows 10      & Integrated GPU          & 15W  \\
% Gaming Lap.    & Laptop  & Windows 10      & NVIDIA GF RTX 2060      & 90W   \\
% PC             & Desktop & Ubuntu 18.04    & NVIDIA GF RTX 3090      & 450W \\\thickhline
% \end{tabular}
% \caption{\textbf{Hardware specs} – of the devices used in our rendering experiments. The power is the max GPU power for discrete NVIDIA cards, and the combined max CPU and GPU power for integrated GPUs.}
% \label{tab:0_hdw_specs_tb}
% \end{table}

% \begin{table}[]
% \footnotesize
% \centering
% \setlength{\tabcolsep}{1pt}
% \begin{tabular}{llllll}\thickhline
% Device                                                        & Browser & Type    & OS                 & GPU                     & Power \\ \hline
% \begin{tabular}[c]{@{}l@{}}Motorola G9\\ Power\end{tabular}   & Chrome  & Phone   & Android 11         & Adreno 610              & 7W    \\
% \begin{tabular}[c]{@{}l@{}}Lenovo \\ Yoga Tablet\end{tabular} & Chrome  & Tablet  & Android Pie (9.0)  & Adreno 505              & 10W   \\
% Dell                                                          & Firefox & Laptop  & Windows 10 Pro     & Integrated GPU          & 15 W  \\
% \begin{tabular}[c]{@{}l@{}}Gaming \\ Laptop\end{tabular}                                                 & Firefox & Laptop  & Windows 10 Pro     & NVIDIA GeForce RTX 2060 & 90W   \\
% PC                                                            & Chrome  & Desktop & Ubuntu 18.04.6 LTS & NVIDIA GeForce RTX 3090 & 450W \\\thickhline
% \end{tabular}
% \caption{\textbf{Hardware specs} – of the devices used in our rendering experiments. The power is the max GPU power for discrete
% NVIDIA cards, and the combined max CPU and GPU power for
% integrated GPUs.}
% \label{tab:0_hdw_specs_tb}
% \end{table}

\input{results/desktop}

% \subsection{Datasets}
%Next, we conduct a comprehensive empirical study on the capabilities of \methodname by analyzing its performance on multiple scenes and on a variety of devices.
Next, we conduct an extensive study of \methodname performance on multiple scenes and devices.
\vspace{-0.5cm}
\paragraph{Datasets.}
We experiment on both synthetic and real data by using two standard datasets:
\vspace{-0.2cm}
\begin{itemize}
\item \textbf{Synthetic 360° dataset}~\cite{barron2022mip}, eight synthetic scenes with intricate geometries and non-Lambertian materials.
Each scene has 100 views for training and 200 views for testing, both at a resolution of $800$px$\times800$px.
\vspace{-0.2cm}
\item \textbf{Tanks and Temples (T\&T) dataset}~\cite{tandt}, an unbounded dataset consisting of hand-held 360° captures of four large-scale scenes. 
%with camera poses estimated using COLMAP~\cite{colmap}.
We use the dataset configuration of~\cite{nerf++}.
%The majority of the background content is covered by the unit sphere by normalizing such that all cameras are inside the sphere of radius $\nicefrac{1}{8}$. 
% $\displaystyle \frac{1}{8}$
% \item \textbf{Unbounded 360° dataset}~\cite{mildenhall2021nerf}, which includes indoor and outdoor scenes displaying a central object or area and a detailed background.
% Each scene has between 100 and 330 views, whose camera poses were estimated by COLMAP SfM~\cite{colmap}.
% Following~\cite{chen2022mobilenerf}, we use five scenes from this dataset. % (Bicycle Flower Garden Stump Treehill)
\vspace{-0.5cm}
\end{itemize}


\paragraph{Devices.}
Our main goal is to test our method on hardware-constrained devices such as mobile phones, tablets, and VR headsets. 
Nonetheless, for completeness, we also test \methodname on more powerful laptops and desktops. 
In total, we test \methodname on seven devices reported in \Table{0_hdw_specs_tb}. 
\vspace{-0.1cm}



\subsection{Implementation details} 
\vspace{-0.1cm}
% Here we provide full details of our implementation.

\noindent\textbf{Pre-trained NeRFs.}
For the pre-trained NeRF models $R$, we use MipNeRF~\cite{barron2021mipnerf} (in the Synthetic 360° dataset), % MipNeRF-360~\cite{mildenhall2021nerf} (in the Unbounded 360° dataset), 
and NeRF++~\cite{nerf++} (in the T\&T dataset). 

\noindent\textbf{Meshing.}
Obtaining the mesh $\chi$ from a pre-trained NeRF requires distilling the learnt geometry. %  learnt by the NeRF.
% sampling the densities learned by $R$ on a volumetric grid of side $K$.
For the Synthetic 360° dataset, we run Marching Cubes~\cite{lorensen1987marching} on a density grid of side $K = 256$, except for the \textit{ficus} object, in which we use $K = 512$ to capture finer geometric details.
For the T\&T dataset, we use a grid with $K = 512$.
% For the Unbounded 360° dataset, we follow~\cite{nerfstudio}, and run Marching Cubes on a Truncated Signed Distance Function, constructed from Mip-NeRF-360's depth maps.
We remove small connected components resulting from noisy estimates, and decimate the meshes to around $400$k faces. % a maximum of $250$k triangles.
Finally, we enclose unbounded scenes within a dome. %  semi-sphere of radius $1$ and add an approximate ground plane in order to obtain a closed mesh.

% \begin{figure*}[t]
%     % \centering
%     \includegraphics[width=\linewidth]{imgs/2_SynQualitative.png}
%     \caption{
%     \textbf{Qualitative Results} - \sara{TODO}
%     }
%     \label{fig:SynQualitative}
% \end{figure*}

\begin{figure}[t]
    % \centering
    \includegraphics[width=0.47\textwidth]{imgs/synth.png}
    % \vspace{-0.1cm}
    \caption{
    \textbf{Qualitative Results on Synthetic Scenes.}
    We show renderings of two synthetic scenes to compare \methodname and MobileNeRF~\cite{chen2022mobilenerf}. 
    In both scenes, we can see how \methodname accurately more reproduces light effects like shininess and reflections. Notice that MobileNeRF fails to model such scene producing holes and high frequency noise.
    }
    \vspace{-0.5cm}
    \label{fig:SyntheticQualitative}
\end{figure}

\noindent\textbf{Pseudo-Image Generation.}
Training the Factorized NeLF via Eq.~\eqref{eq:fnelf_loss} requires the set $\mathcal{I}_{\text{pseu}}$ of pseudo-images.
% Once a rough mesh has been extracted from the NeRF model, we render the set of pseudo-images $\mathcal{I}_{\text{pseu}}$ which will be used for training \methodname.
We obtain these images by using the NeRF to render $10$k images from random camera poses for each scene. % , and using each scene's NeRF  to perform rendering.
% We render $10$k images for all scenes. 
% , except for \textit{ficus} and \textit{mic}, for which we render $14$k images. % for ficus and mic and $10K$ for all other objects with randomly sampled camera poses.
% We acquire the NeLF pseudo data for training by randomly sampling the camera positions and orientations from the trained NeRF .

\noindent\textbf{Factorized NeLF Training.}
We implement the position- and direction-dependent MLPs ($L_{\text{pos}}$ and $L_{\text{dir}}$) by following~\cite{r2l}, and thus employ intensive residual blocks~\cite{resnet} and deep architectures of $88$ layers. %  and deep MLPs for each .
We train these MLPs with hard-ray sampling~\cite{r2l} and learning rate warm-up strategies.
We train with a batch size of $200$k rays for 2.5 days on one NVIDIA A100 GPU.

\noindent\textbf{Baking Light Field Embeddings.}
For baking $\boldsymbol\beta$, we uniformly sample elevation and azimuth angles in $[0$-$180^{\circ}]$ and $[0$-$360^{\circ}]$, respectively. % on the surface of a sphere.
We use $1024$ samples for synthetic scenes and $2048$ for real scenes.
When quantizing $\mathbf{u}$, $\mathbf{v}$, $\mathbf{w}$, and $\boldsymbol\beta$ for their storage as PNG files, we perform per-channel min-max normalization.
% It takes approximately 50 mins to generate all textures.
Unless otherwise stated, we use $18 = \lceil\nicefrac{6\times6}{2}\rceil$ texels per triangle face. 
% \sara{yes or not?}



\noindent\textbf{Rendering with Shaders.}
Since \methodname requires no MLP queries, we implement it in a simple fragment shader.
This implementation allows deploying \methodname across not only various devices, but also different graphics frameworks. 
Notably, this implementation allows us to deploy on a VR headset (Meta Quest Pro), which runs Unity shaders.
% This combination is computed in the vertex shader.
The shader computes color by combining the position- and direction-dependent embeddings according to Eq.~\eqref{eq:factorized-nelf}, where each embedding is queried from % $\mathbf{u}$, $\mathbf{v}$, $\mathbf{w}$, and $\boldsymbol\beta$ in 
its corresponding texture map.
In turn, each texture map is a $2\times4$ grid stored in a $4$-channeled PNG image, %  (\ie the PNG), 
thus fully accounting for our default embedding dimension $D = 32 = 2\times4\times4$.
We obtain the position embeddings by indexing $\mathbf{M}_{\mathbf{u},\mathbf{v},\mathbf{w}}$ with the fragment's $uv$ coordinates, while the direction embeddings are obtained by indexing $\mathbf{M}_{\boldsymbol\beta}$ with the azimuth and elevation angles.
Since texture values are 8-bit quantized, we map to the original range by reverting the per-channel min-max normalization. %  to approximate the original values ($32$ values for each $\mathbf{u}, \mathbf{v}, \mathbf{w},$ and $\boldsymbol\beta$).
% We compute the fragment's color as the dot product of $(\mathbf{u}, \mathbf{v}, \mathbf{w})$ and $\boldsymbol\beta$, followed by a sigmoid function.
% Our use of light fields allows use to dispose of alpha compositing, and thus we set alpha values to~$1$. 
We provide a full implementation of our shaders in the \SM.

% \begin{figure*}[t]
%     % \centering
%     \includegraphics[width=\linewidth]{imgs/comparison.png}
%     \caption{
%     \textbf{Qualitative Results} - \sara{TODO}
%     }
%     \label{fig:RealQualitative}
% \end{figure*}

\begin{figure*}[t]
    \centering
    \includegraphics[width=\linewidth]{imgs/real.png}
    % \vspace{-0.6cm}
    \caption{
    \textbf{Qualitative Results of Real Scenes.}
    We show renderings of challenging real scenes using \methodname, MobileNeRF~\cite{chen2022mobilenerf}, and SNeRG~\cite{hedman2021snerg}. 
    Our framework can generate sharper and more accurate renders. 
    In contrast, both MobileNeRF and SNeRG struggle with foreground and background artifacts, and fail to reconstruct fine details such as the writings on the train (bottom images). 
    }
    \vspace{-0.5cm}
    \label{fig:RealQualitative}
\end{figure*}

 
\subsection{Main Results}
We compare against two methods intended for fast rendering, SNeRG~\cite{hedman2021snerg} and MobileNeRF~\cite{chen2022mobilenerf}. % with the capacity to render in real-time on devices. % , in \Table{1_Rendering_speed_tb}.
\vspace{-0.5cm}
\paragraph{Quantitative Results.}
The first three rows of \Table{desktop} show how \methodname can achieve real-time rendering when deployed on a mobile.
For both datasets, we find that, while SNeRG achieves reasonable disk usage and photo-metric quality, it is ultimately incapable of interactive rendering.
\underline{On the Synthetic 360° dataset}, we find that \methodname can render at over 54 FPS, outperforming Mobile-NeRF by more than 30\%.
Furthermore, such fast rendering speed comes at a negligible drop in performance of $\sim$1 dB. %  with respect to Mobile-NeRF.
Moreover, these benefits over MobileNeRF are accompanied by simpler meshes (\ie faces and vertex quantity).
% Furthermore, 
\underline{On the realistic Unbounded 360° dataset}, our method outperforms competing approaches by even larger margins than on synthetic data.
In particular, we improve upon MobileNeRF \textit{both} in rendering quality and speed.
That is, while SNeRG and MobileNeRF achieve a PSNR of 14.0 and 15.6, respectively, \methodname attains 17.9.
Impressively, this outstanding gain in photo-metric quality of over 2 dB is accompanied by an also impressive superiority in rendering speed of over 10 FPS over MobileNeRF~(22.79 \textit{vs.} 33.46 FPS).
% Once again, \methodname's superiority over MobileNeRF come with the added benefits of simpler, \ie smaller, meshes.
% From the methods we mention here, only \methodname and MobileNeRF use polygons to model scenes.
% As such, the size of the meshes used by these methods has an impact on their cost, storage and performance.
% We underscore how, despite using substantially smaller meshes than MobileNeRF to model real scenes, \methodname still outperforms this competitor by significant margins both in speed. %  (\Table{1_Rendering_speed_tb}) and quality (\Table{desktop}).
% The gains we achieve can be attributed to our light field-based formulation, that allows us to work with smaller meshes that thus possess smaller expressive power.
The gains we achieve can be attributed to our light field-based formulation, that allows us to work with smaller meshes while still maintaining high directional expressivity.
% \vspace{-0.1cm}
% The last seven rows of \Table{desktop} report an upper-bound to the performance of all methods, by evaluating on a desktop computer at a higher resolution of 8k$\times$4k pixels.
The bottom rows of \Table{desktop} report the performance of all methods in a non-constrained scenario, which is a desktop.
%, by evaluating on a desktop computer at a higher resolution of 8k$\times$4k pixels.
When provided with such larger computational resources, our method's speed improves both in absolute and relative terms.
Specifically, while the gap in mobiles w.r.t. MobileNeRF was $\sim$30\% (on the Synthetic dataset), this gap jumps to over 460\% on the desktop.
This increased gap is even more pronounced in the Unbounded dataset: the gap in mobiles was $\sim$45\% (from 22 FPS to 33), while the gap in desktop becomes over 1,500\%.
The larger computational resources also allow us to test a larger version of \methodname, which achieves remarkable PSNRs of 30.1 and 18.0 in the Synthetic and Unbounded datasets, respectively, while also remaining much faster than MobileNeRF.
The last three rows of \Table{desktop} report foundational methods (\ie NeRF, Mip-NeRF and NeRF++) as reference of high quality, although these methods are inefficient for rendering.
\vspace{-0.5cm}

\paragraph{Qualitative Results.} We next compare \methodname performance against the other real-time rendering methods. 
%We can gain a better understanding of rendering quality by looking at qualitative results of \methodname compared to other methods for real-time NeRF rendering. 
We show 
%several of these 
results for both synthetic (\Figure{SyntheticQualitative}) and real scenes (\Figure{RealQualitative}).
Note \methodname higher quality, particularly in real scenes, where our method preserves sharper object boundaries, and crisper surface details. 
\methodname allows real-time rendering while preserving strong image quality.



% \begin{table}[t!]
%     \footnotesize
%     \centering
%     \setlength{\tabcolsep}{1.5pt}
%     \begin{tabular}{l"c|c|c"c|c|c}
%         \thickhline
%          & \multicolumn{3}{c"}{Synthetic 360°} & \multicolumn{3}{c}{Unbounded 360°} \\
%         Device & SNeRG & M-NeRF & \methodname & SNeRG & M-NeRF & \methodname \\
%         \thickhline
%         % iPhone 12 & N/A & 57.9 & \best{60.0}$^{\star}$ & N/A & 38.2 & \best{60.0}$^{\star}$ \\
%         Samsumg S21 & N/A & 57.9 & \best{60.0}$^{\star}$ & N/A & 38.2 & \best{60.0}$^{\star}$ \\
%         G9 & N/A & 10.6 & \best{22.8} & N/A & 3.3 & \best{16.0} \\
%         Yoga & N/A & 5.7 & \best{13.8} & N/A & 2.0 & \best{9.2} \\
%         Dell & N/A & 49.8 & \best{83.3} & N/A & 16.9 & \best{62.7} \\
%         Gaming & 192.7 & 473.40 & \best{625.1} & 42.3 & 169.5 & \best{503.5} \\
%         Desktop & 688.9 & 1,075.2 & \best{1,820.2} & 83.9 & 478.5 & \best{1,648.0} \\
%         Headset & N/A & - & \best{74.0}$^{\star}$ &  N/A & - & \best{74.0}$^{\star}$ \\
%         \thickhline
%         GPU (MB) & 5,707.3 & 538.4 & \best{239.3} & 3,388.3 & 1,063.0 & \best{397.5} \\
%         Disk (MB) & \best{87.0} & 125.75 & 143.5 & 324.8 & 311.3 & \best{233.5} \\
%         \thickhline
%         PSNR (dB) & 30.4 & \best{30.9} & 29.7 & 14.0 & 15.6 & \best{17.9} \\
%         \thickhline
%     \end{tabular}
%     % \vspace{-0.1cm}
%     \caption{\textbf{On-device Performance.} 
%     \sara{needs to update} We compare the performance of \methodname against MobileNeRF (M-NeRF)~\cite{chen2022mobilenerf} and SNeRG~\cite{hedman2021snerg} across devices. 
%     The first seven rows report rendering speed in frames-per-second~(FPS).
%     In notation, $``^{\star}"$ means the device's FPS limit was reached, $``-"$ means missing implementation, and ``N/A'' means the method failed to run.
%     % For SNeRG,  denotes the method was unable to run due to out-of-memory errors. %  generated when running, precluding rendering in the specified device. 
%     % SNeRG is unable to run on resource-constrained devices, which we denote by .
%     %Note how \methodname provides significantly larger FPS than the competitors, particularly on real scenes. 
%     The second set of rows reports GPU memory and disk space requirements.
%     Finally, we report average PSNR for each dataset.
%     Note \methodname provides the fastest rendering across scenes with comparable memory and disk requirements.
%     For synthetic scenes, our FPS gains come with limited cost to quality.
%     In unbounded scenes, our method improves \textit{both} PSNR and FPS by sizable margins. 
%     Moreover, we highlight that \methodname is capable of real-time rendering even on VR headsets. % , and achieves real-time rendering speeds. 
%     }
%     \label{tab:1_Rendering_speed_tb}
% \end{table}

\begin{table}[t!]
    \footnotesize
    \centering
    \setlength{\tabcolsep}{1.5pt}
    \begin{tabular}{l"c|c|c"c|c|c}
        \thickhline
         & \multicolumn{3}{c"}{Synthetic 360°} & \multicolumn{3}{c}{Unbounded 360°} \\
        Device & SNeRG & M-NeRF & \methodname & SNeRG & M-NeRF & \methodname \\
        \thickhline
        % iPhone 12 & N/A & 57.9 & \best{60.0}$^{\star}$ & N/A & 38.2 & \best{60.0}$^{\star}$ \\
        Samsung S21 & $\dagger$ & \textcolor{black}{41.7} & \textcolor{black}{\best{54.7}} & $\dagger$ & \textcolor{black}{22.8} & \textcolor{black}{\best{33.5}} \\
        G9 & $\dagger$ & \textcolor{black}{9.7} & \textcolor{black}{\best{10.4}} & $\dagger$ & \textcolor{black}{3.5} & \textcolor{black}{\best{7.7}} \\
        \textcolor{black}{Galaxy S6 } & N/A & \textcolor{black}{18.1} & \textcolor{black}{\best{26.6}} & $\dagger$ & \textcolor{black}{6.2} & \textcolor{black}{\best{20.4}} \\
        Dell & $\dagger$ & \textcolor{black}{49.8} & \textcolor{black}{\best{75.3}} & $\dagger$ & \textcolor{black}{16.9} & \textcolor{black}{\best{54.0}} \\
        Gaming & \textcolor{black}{197.9}& \textcolor{black}{496.0} & \textcolor{black}{\best{697.3}} & \textcolor{black}{46.0} & \textcolor{black}{169.9} & \textcolor{black}{\best{516.6}} \\
        Desktop & \textcolor{black}{502.1} & \textcolor{black}{762.3} & \textcolor{black}{\best{1,013.2}} & \textcolor{black}{141.1} & \textcolor{black}{389.8} & \textcolor{black}{\best{925.4}} \\
        Headset & $\dagger$ & $-$ & \best{74.0}$^{\star}$ &  $\dagger$ & $-$ & \best{74.0}$^{\star}$ \\
        % \thickhline
        % GPU (MB) & 5,707.3 & 538.4 & \best{239.3} & 3,388.3 & 1,063.0 & \best{397.5} \\
        % Disk (MB) & \best{87.0} & 125.75 & 143.5 & 324.8 & 311.3 & \best{233.5} \\
        % \thickhline
        % PSNR (dB) & 30.4 & \best{30.9} & 29.7 & 14.0 & 15.6 & \best{17.9} \\
        \thickhline
    \end{tabular}
    \vspace{0.07cm}
    \caption{\textbf{On-device rendering speed (FPS).} 
    % \sara{needs to update} 
    We compare the rendering speed of \methodname against MobileNeRF (M-NeRF)~\cite{chen2022mobilenerf} and SNeRG~\cite{hedman2021snerg} across devices. 
    % The first seven rows report rendering speed in frames-per-second~(FPS).
    Conventions: $``^{\star}"$ means the device's FPS limit was reached, $``-"$ means missing implementation, and ``$\dagger$" means the method failed to run.
    % For SNeRG,  denotes the method was unable to run due to out-of-memory errors. %  generated when running, precluding rendering in the specified device. 
    % SNeRG is unable to run on resource-constrained devices, which we denote by .
    %Note how \methodname provides significantly larger FPS than the competitors, particularly on real scenes. 
    % The second set of rows reports GPU memory and disk space requirements.
    % Finally, we report average PSNR for each dataset.
    % Note \methodname provides the fastest rendering across scenes with comparable memory and disk requirements.
    % For synthetic scenes, our FPS gains come with limited cost to quality.
    % In unbounded scenes, our method improves \textit{both} PSNR and FPS by sizable margins. 
    For all devices and datasets, \methodname provides the fastest rendering speeds, usually by a large margin.
    Furthermore, we highlight that \methodname is capable of real-time rendering even on VR headsets. % , and achieves real-time rendering speeds. 
    }
    \vspace{-0.6cm}
    \label{tab:1_Rendering_speed_tb}
\end{table}



\subsection{Rendering Speed on Devices}
% This table reports render speed in frames-per-second (FPS), % GPU memory consumption and disk storage (MB), and PSNR for assessing render quality. 
% compare against methods with capabilities of real-time rendering on devices, mainly MobileNeRF~\cite{chen2022mobilenerf} and SNeRG~\cite{hedman2021snerg}. 
% Here, rendering speed was measured 
We measure frames-per-second (FPS) in seven devices~(via the same procedure as~\cite{chen2022mobilenerf}), and report results in \Table{1_Rendering_speed_tb}.
Our results illustrate how \methodname significantly outperforms other methods in rendering speed. 
The advantages in performance are particularly large in unbounded scenes. 
In this scenario, \methodname is, on average $2.6\times$ faster than MobileNeRF. %  while maintaining higher quality.
% Namely, in this scenario, \methodname is, on average $2.6\times$ faster than MobileNeRF. %  while maintaining higher quality.
For synthetic scenes, our method provides sizable speed gains of 35\%. %  still come only with a minor loss in quality for synthetic scenes.
Importantly, we find that \methodname is capable of real-time rendering in a VR headset even reaching the device's limit of 74 FPS. %  (\Table{1_Rendering_speed_tb}). 
% Furthermore, speed gains come with minor to no loss in quality, as evidenced by the PSNR row. %  in \Table{1_Rendering_speed_tb}. 
% Furthermore, speed gains come only with a minor loss in quality for synthetic scenes as evidenced by the PSNR. %  in \Table{1_Rendering_speed_tb}. 
% \methodname achieves these results while requiring less dense meshes (\Table{4_Polygon_count_tb}).

% Notably, this implementation allows us to deploy our method on a VR headset (Meta Quest Pro), which runs Unity shaders. 
% For both synthetic and real scenes, we find that \methodname is capable of rendering in the headset in real time, even reaching the device's capped limit of 74 FPS (\Table{1_Rendering_speed_tb}). 
% The flexibility of our framework allows us to easily deploy on any device that supports a standard graphics pipeline.
\begin{comment}
\sara{Update}  \subsection{Rendering quality}

\begin{table}[t!]
    \footnotesize
    \centering
    \setlength{\tabcolsep}{2pt}
    \begin{tabular}{l"c|c|c"c|c|c}
        \thickhline
        & \multicolumn{3}{c"}{Synthetic 360°} & \multicolumn{3}{c}{Unbounded 360°} \\
        Method     & PSNR $\uparrow$      & SSIM  $\uparrow$     & LPIPS $\downarrow$     & PSNR  $\uparrow$    & SSIM  $\uparrow$     & LPIPS  $\downarrow$    \\\hline
        NeRF \cite{mildenhall2021nerf}      &   31.00  &    0.947        &     0.081       &   18.98         & 0.712        &  0.573     \\
        Mip-NeRF \cite{barron2021mipnerf}      &   33.09  &    0.961        &     0.043       &   -         & -        &  -     \\
        NeRF++  \cite{nerf++}   &   -             &   -             &    -            &   20.08  & 0.758 &  0.412     \\
        ADOP  \cite{ruckert2022adop}   &   -             &   -             &    -            &   23.53*  & - &  0.151*     \\\hline
        SNeRG   \cite{hedman2021snerg}  &    30.38        &   \best{0.950} & \best{0.050}     &   14.04           & 0.528 &  0.548     \\
        M-NeRF  \cite{chen2022mobilenerf}   &    \best{30.90} &   0.947         &   0.062         &   15.59           & 0.474        &  0.535     \\
        Ours       &    29.66        &   0.939         &  0.068         &   \best{17.91}    & \best{0.542}        &  \best{0.513}     \\
        % Ours$\bigstar$  &    28.67   &   0.927         &  0.079          &   18.31         & 0.561        &  0.488   \\
        \thickhline
    \end{tabular}
    % \vspace{-0.2cm}
    \caption{\textbf{Rendering Quality.} \sara{ADOP does not have truck scene but it has lighthouse???} In this table we present common measures of image quality for renderings from \methodname and several other non-real and real time NeRF pipelines on edge devices. We present this table as reference to show that \methodname achieves comparable image quality to more complex NeRF architectures, while being significantly faster. Note that we extract numbers directly from each corresponding paper, and omit values (with -) that were not originally reported by the methods. 
    }
    % \vspace{-0.2cm}
    \label{tab:3_Quantitative_tb}
\end{table}
% \begin{table}[t!]
    \footnotesize
    \centering
    \begin{tabular}{l"c|c"c|c}
        \thickhline
        & \multicolumn{2}{c"}{Synthetic 360°} & \multicolumn{2}{c}{Unbounded 360°}\\
        Method & M-NeRF & \methodname & M-NeRF & \methodname\\
        \thickhline
        Vertices & 494,289 &  \best{99,539} & 332,678 & \best{117,751}\\
        Faces    & 224,341 & \best{205,693} & 543,047 & \best{244,847}\\
        \thickhline                                            
    \end{tabular}
    % \vspace{-0.2cm}
    \caption{\textbf{Average Mesh Size.} We report the average number of vertices and triangle faces for meshes used by \methodname and MobileNeRF. 
    The meshes used by our method are substantially smaller than those used by MobileNeRF. %  is more economical in mesh requirements size.
    }
    % \vspace{-0.4cm}
    \label{tab:4_Polygon_count_tb}
\end{table}

We compare the rendering quality of \methodname against other methods via common quantitative measures (PSNR, SSIM~\cite{SSIM} and LPIPS~\cite{LPIPS}). %  commonly used for testing image quality. 
We report this comparison in \Table{3_Quantitative_tb}. % , along with equivalent ones for several NeRF methods. 
When compared to more complex NeRF architectures, \methodname could be expected to suffer small declines in quality. 
However, as \Table{3_Quantitative_tb} illustrates, our method still achieves comparable performance. 
We argue that, given the substantial gains in rendering speed provided by our method, the magnitude of these drops is a manageable. 

We can gain a better understanding of rendering quality by looking at qualitative results of \methodname compared to other methods for real-time NeRF rendering. 
We show several of these results for both synthetic \sara{Update} (\Figure{SyntheticQualitative}) and real scenes (\Figure{RealQualitative}).
We can observe the higher quality of \methodname renderings, particularly in real scenes, where our method manages to preserve sharper object boundaries, and crispier surface details. 
\methodname not only allows fast real-time rendering, but also preserves strong image quality.

From the methods we mention here, only our \methodname and MobileNeRF use polygons to model scenes.
As such, the size of the meshes used by these methods has an impact on their cost, storage and performance.
\sara{Update} \Table{4_Polygon_count_tb} compares the sizes of the meshes used by these methods.
Notably, our method uses smaller meshes, and thus is more economical for storage. %  is more economical in mesh requirements size. 
We underscore how, despite using substantially smaller meshes than MobileNeRF to model real scenes, \methodname still outperforms this competitor by significant margins both in speed (\Table{1_Rendering_speed_tb}) and quality (\Table{3_Quantitative_tb}).
The gains we achieve can be attributed to our light field-based formulation, that allows us to work with smaller meshes that thus possess smaller expressive power.
\end{comment}

\subsection{Quality \textit{vs.} Representation Size}
Two main factors affect the size of our representation: the texel count and the dimensionality of the light field embeddings ($D$ in Eq.~\eqref{eq:dir-pos}).
Here we study how these factors, in turn, affect the photo-metric quality of \methodname.
% The amount of texels tells us how big or small the image applied as a texture is.
\underline{Texel count.} 
We examine the effect that varying the number of texels assigned to each face in the mesh has on rendering quality and disk space. %  by adjusting the number of texels assigned to each triangle face. 
\Figure{ablation} (left) and \Figure{ablation2} show that, for both datasets, an increased texel count is accompanied by a drastic rise in quality of the reconstruction and disk space. %  for both datasets. 
% The disk space for synthetic scenes is lower than that of real scenes, due to a lower average number of triangle faces per scene.
\underline{Embedding dimensionality.} 
We now examine the effect that varying the dimensionality of the light field embeddings has on the ability to represent light effects.
% \Figure{ablation} (right) reports reflectance maps for various dimensionalities.
% With respect to light effects, our results suggest that the capacity of \methodname to model challenging view-dependent effects, such as lighting, is rather robust, providing reasonable performance even with low dimensionalities.
% \methodname is capable of modeling challenging view-dependent effects even with low dimensionality embeddings, but larger embeddings allow it to model more complex reflections.
Reflectance maps in \Figure{ablation} (right) demonstrate that \methodname can model challenging view-dependent effects even with low-dimensional embeddings, but larger embeddings enable more complex reflections. 
Together, our experiments demostrate that exchanging disk space for renderings of higher quality, and suggest that increasing the texel count can improve rendering quality.


\begin{figure}[t]
    \centering
    % \vspace{0.2cm}
    \includegraphics[width=1.0\linewidth]{imgs/reflection.pdf}
    % \vspace{-0.3cm}
    \caption{
    % \albert{Update caption}
    \textbf{Left:} We compare how PSNR varies as the number of texels per face increases for both synthetic and unbounded scenes.
    It is possible to trade-off disk space for higher quality renderings by increasing the number of texels.
    \textbf{Right:} We visualize color as a function of elevation (y-axis) and azimuth (x-axis) for a surface point x, y, z on the Materials scene from the Synthetic dataset, for different embedding dimensions. 
    % A larger embedding dimensionality allows our factorized representation to better approximate the ground truth reflectance.
    % \textbf{Rendering Quality \textit{vs.} Disk Space.}
    % We provide information on disk space and PSNR as the number of texels per face increases for both synthetic and unbounded scenes.
    % In this fashion, we show it is possible to trade-off disk space for higher quality renderings by increasing the number of texels.
    }
    \vspace{-0.4cm}
    \label{fig:ablation}
\end{figure}
\begin{figure}[t]
    % \centering
    \includegraphics[width=\linewidth]{imgs/AblatQuali.png}
    \vspace{-0.5cm}
    \caption{
    \textbf{Qualitative Effect of Varying Number of Texels.} 
    We show renderings of synthetic and real scenes with an increasing number of texels per triangle face.
    Higher texels per face result in crispier renderings, although it is possible to appreciate directional effects such as specular highlights even with low texel counts.
    }
    \vspace{-0.5cm}
    \label{fig:ablation2}
\end{figure}
\subsection{Compositional scenes}

In \Figure{app}, we showcase the practical application of \methodname for scene composition.
Our approach enables efficient rendering of 2500 materials and ficus objects in single scenes at 130 FPS each on a desktop.
Additionally, we demonstrate an AR application that uses an AR/VR headset to render four chairs in real-time in a real-world setting.
% This exemplifies our approach's ability to seamlessly blend virtual and real environments, creating an immersive visual experience for users.
% Our method's effectiveness in this context unlocks new possibilities for powerful applications.
This exemplifies our approach's ability to seamlessly blend virtual and real environments, unlocking new possibilities for immersive visual experiences.
For this chapter, fix a prime $p$. We first discuss deformations of coalgebras from $\F_{p}$
to the $p$-adic integers and further to the $p$-completed sphere $\S_{p}^{\wedge}$ which leads
us to the question of how coalgebras behave with respect to $p$-completion. We introduce the
notion of a $p$-complete coalgebra and show that this is well behaved with respect to the
deformation theory discussed in the previous chapter. We then use this to iterate
Proposition~\ref{witt} and prove our main results, namely the existence of Witt Vectors
and spherical Witt Vectors for formally \'etale coalgebras. Then we specialize to the case
of homology coalgebras, show that for a finite space $X$ the coalgebra $\F_{p}[X]$ is formally
\'etale, and answer our initial question about the relation between $\S[X]^{\wedge}_{p}$
and $\F_{p}[X]$

\subsection{Coalgebras and $p$-completion}

We have seen that the functors that interest us are all \textit{nilcomplete}. For a nilcomplete
functor $X:\rm{CAlg}^{\rm{cn}} \to \cl{S}$ and a connective $\bb{E}_{\infty}$-ring $R$, we can construct
lifts from $X(\pi_{0}R)$ to $X(R)$ inductively along the Postnikov tower
\[ \dots \to \tau_{\leq2}R \to \tau_{\tau\leq 1}R \to \tau_{\leq0} R =\pi_{0}R.\]
This is however not quite enough to obtain our goal of lifting from $\F_{p}$ to the
$p$-completed sphere, we first need to pass to $\Z_{p}= \pi_{0}\S_{p}^{\wedge}$.
Explicitly, this means constructing lifts against the tower
\[\dots \to \Z/p^{3}\to \Z/p^{2}\to \Z/p\to \F_{p}\]
which is clearly presents a different problem. With the machinery developed thus far, we can already
prove the following for a general deformation problem.

\begin{proposition}\label{liftpgen}
  Let $X: \rm{CAlg}^{\rm{cn}} \to \cl{S}$ be a cohesive functor and $A\in X(\F_{p})$
  such that $T_{X_{A}}\simeq 0$. Then there exists a unique lift of $A$ to a point in
  $\flim_{n}X(\Z/p^{n})$.
\end{proposition}
\begin{proof}
  Set $A_{0}= A$, we inductively construct lifts against the tower of square zero extensions
  \[\dots \to \Z/p^{3} \to \Z/p^{2}\to \F_{p}.\]
  Suppose we have already constructed lifts $A_{k}$ for $k\le n$ for some $n$.
  Applying Proposition~\ref{bc} inductively, we get that
  \[T_{X_{A_{n}}}^{\F_{p}} \simeq T^{\F_{p}}_{X_{A_{0}}} \simeq 0.\]
  Thus, since $\Z/p^{n+1}\to \Z/p^{n}$ is a square zero extension with fiber $\F_{p}$,
  Proposition~\ref{deformations} implies that the fiber
  \[X_{A_{n}}^{\Z/p^{n+1}}=\rm{fib}_{A_{n}}(X(\Z/p^{n+1})\to \Z/p^{n})\]
  is contractible and we find an essentially unique lift $A_{n+1}$. This proves the claim.
\end{proof}
 Of course, for an arbitrary functor $X:\rm{CAlg}^{\rm{cn}} \to \cl{S}$ the natural map
$X\to \flim_{n}X(\Z/p^{n})$ might not be an equivalence, meaning that in this generality
we can only construct pro-$p$ objects of $X$ using this inductive method.
In fact, we have that $\rm{cCAlg}_{\Z_{p}}\neq  \flim_{n} \rm{cCAlg}_{\Z/p^{n}}$. To remedy
this problem we show that this limit admits a description via \textit{$p$-complete} coalgebras.
To do this, we first recall some facts about $p$-complete modules.

\begin{definition}
Let $R$ be an $\bb{E}_{\infty}$-ring, then $M \in \rm{Mod}_{R}$ is called
$p$-\textit{complete} if the limit
\[ \lim \left(\dots \rar{\cdot p} M \rar{\cdot p}M \right)\]
vanishes. We denote the full subcategory spanned by the $p$-complete modules by $(\rm{Mod}_{R})_{p}^{\wedge}$.
\end{definition}

\begin{remark}
The inclusion $(\rm{Mod}_{R})_{p}^{\wedge} \rari{} \rm{Mod_{R}}$ admits a left adjoint which takes a module $M$
to its \textit{$p$-completion} given by the limit
\[ \lim \left( \dots \to M/p^{2} \to M/p \right).\]
In fact, $M$ is $p$-complete if and only if the natural map $M \to \lim M/p^{n}$ is an equivalence.
This inherits a natural $R^{\wedge}_{p}$-module structure, thus $p$-completion also gives
an equivalence of categories $(\rm{Mod}_{R})^{\wedge}_{p} \simeq (\rm{Mod}_{R^{\wedge}_{p}})^{\wedge}_{p}$ which
allows us to identify these in what follows.\\
The tensor product of $p$-complete modules is in general not $p$-complete. However, the
category $(\rm{Mod}_{R})_{p}^{\wedge}$ admits a symmetric monoidal structure given by the formula
 \[ M \otimes_{(\rm{Mod}_{R})_{p}^{\wedge}} N := ( M \otimes N )^{\wedge}_{p}.\]
 With this monoidal structure the $p$-completion functor $\rm{Mod}_{R}\to (\rm{Mod}_{R})_{p}^{\wedge}$
 is strong monoidal, while the inclusion is only lax monoidal.
\end{remark}

 \begin{definition}
   Let $R$ be an $\bb{E}_{\infty}$-ring. We define the $\infty$-category of $p$-complete
   $R$-coalgebras is given by.
   \[ {(\rm{cCAlg}_{R})}^{\wedge}_{p}:= \rm{cCAlg}({(\rm{Mod}_{R})}^{\wedge}_{p}).\]
 \end{definition}

 \begin{warning}
   Let $R$ be a $\bb{E}_{\infty}$-ring. Notice that by our definition a $p$-complete $R$-coalgebra
   is the same as a $p$-complete $R^{\wedge}_{p}$-coalgebra and so we do not differentiate between
   the two notions.
   However, this is \textit{not} the same as an $R^{\wedge}_{p}$-coalgebra whose underlying
   spectrum is $p$-complete. The process of $p$-completion does refine to a functor
   $\rm{cCAlg}_{R} \to (\rm{cCAlg}_{R^{\wedge}_{p}})^{\wedge}_{p}$,
   but it does not factor through the category $\rm{cCAlg}_{R^{\wedge}_{p}}$.
 \end{warning}

 We now show check that the assignment $R \mapsto \rm{cCAlg}_{R}^{\rm{cn}}$ is subject to the machinery
 of deformation theory.

 \begin{lemma}\label{conil2}
   The following statements hold:
   \begin{enumerate}
     \item   Suppose we have a pullback diagram of connective $\bb{E}_{\infty}$-rings
   \[\begin{tikzcd}
	R\p & S\p \\
	R & S
	\arrow[from=1-1, to=2-1]
	\arrow[from=2-1, to=2-2]
	\arrow[from=1-2, to=2-2]
	\arrow[from=1-1, to=1-2]
\end{tikzcd}\]
such that the map $\pi_{0}R \to \pi_{0}S$ is surjective. Then the natural map
\[ (\rm{cCAlg}_{R\p}^{\rm{cn}})^{\wedge}_{p} \to (\rm{cCAlg}_{R}^{\rm{cn}})^{\wedge}_{p}\times_{(\rm{cCAlg}_{S}^{\rm{cn}})^{\wedge}_{p}} (\rm{cCAlg}_{S\p}^{\rm{cn}})^{\wedge}_{p}\]
is an equivalence.
     \item For every connective $\bb{E}_{\infty}$-ring $R$, the natural map
           \[ (\rm{cCAlg}_{R}^{\rm{cn}})^{\wedge}_{p} \to\flim_{n} (\rm{cCAlg}_{\tau_{\le n}R}^{\rm{cn}})^{\wedge}_{p}\]
           is an equivalence.
   \end{enumerate}
 \end{lemma}
 \begin{proof}
   Ad 1.: Arguing as in the proof of Proposition~\ref{Mod}, it suffices to show that the
   strong monoidal functor
   \begin{align*}
    (\rm{Mod}_{R\p})^{\wedge}_{p} \to (\rm{Mod}_{R})^{\wedge}_{p}\times_{(\rm{Mod}_{S})^{\wedge}_{p}} (\rm{Mod}_{S\p})^{\wedge}_{p}
   \end{align*}
   is an equivalence. Indeed, given a point $(M,N,h)$ in the pullback, the $R\p$-module $M \times_{M \otimes_{R} S}N$
   is again $p$-complete since $p$-completion commutes with limits. Thus, the inverse functor of
   Proposition~\ref{Mod} also induces a functor on the categories of $p$-complete modules. Moreover,
   we have that
   \[ ((M\times_{M\otimes_{R}S}N)\otimes_{R\p} R)^{\wedge}_{p} \simeq M^{\wedge}_{p} \simeq M\]
   \[ ((M \times_{M\otimes_{R}}N)\otimes_{R\p}S\p)^{\wedge}_{p}\simeq N^{\wedge}_{p} \simeq N,\]
   where the first equivalences hold by Proposition~\ref{Mod}, and the latter since $M$ and $N$ are
   to be $p$-complete. Finally, for $M\in (\rm{Mod}_{R\p})^{\wedge}_{p}$, we compute that
   \[ (M \otimes_{R\p} R)^{\wedge}_{p}\times_{(M \otimes_{R\p} S)^{\wedge}_{p}}(M \otimes_{R\p}S\p)^{\wedge}_{p}
     \simeq \left( M \otimes_{R\p} R \times_{M\otimes_{R\p} S} M \otimes_{R\p} S\p\right)^{\wedge}_{p}
   \simeq M^{\wedge}_{p} \simeq M,\]
 where we have again used the result of Proposition~\ref{Mod} and the fact that $p$-completion commutes
 with limits.\\
 Ad 2: This uses the exact same arguments applied to the equivalence of Corollary~\ref{nilcomplete}.
 \end{proof}

 \begin{corollary}
   For any $n\in \bb{N}$, the functor
   \[ \rm{CAlg}^{\rm{cn}} \to \cl{S} \qquad R \mapsto [(\rm{cCAlg}_{R}^{\rm{cn}})^{\wedge}_{p}]^{\Delta^{n}}\]
   is coherent and nilcomplete.
 \end{corollary}

 We now prove the crucial $p$-completeness result for $\Z_{p}$-modules. As before
 this will enable us to deduce the same result for coalgebras and allow us to tackle the
 actual problem of comparing coalgebras over $\F_{p}$, $\Z_{p}$ and $\S_{p}^{\wedge}$.
\begin{proposition}\label{pcomp}
  Let $\rm{Mod}^{\wedge}_{\Z_p} \subseteq \rm{Mod}_{\Z_{p}}$ denote the full subcategory spanned by the
  $p$-complete $\Z_{p}$-module spectra. Then the natural map
  \[ \rm{Mod}_{\Z_{p}} \to \flim_{n} \rm{Mod}_{\Z/p^{n}} \quad N \mapsto (N\otimes_{\Z_{p}}\Z/p^{n})\]
  restricts to a strong monoidal equivalence
  \[(\rm{Mod}_{\Z_{p}})^{\wedge}_{p} \simeq \flim_{n}\rm{Mod}_{\Z/p^{n}}. \]
\end{proposition}
\begin{proof}
  The functor admits a right adjoint which takes $(M_{n})\in \flim_{n}\rm{Mod}_{\Z/p^{n}}$ to the limit
  $\lim_{n}M_{n}$ taken in the category of $\Z_{p}$-modules. Since $p$-complete modules are closed under
  limits, the essential image of this functor is contained in $\rm{Mod}_{\Z_{p}}^{\wedge}$. Moreover,
  if $M\in \rm{Mod}_{\Z_{p}}^{\wedge}$, then we have that
  \[ \flim_{n}(M \otimes_{\Z_{p}} \Z/p^{n}) \simeq \flim_{n} M/p^{n} \simeq M^{\wedge}_{p}\simeq M.\]
  Hence, the counit of the adjunction is an equivalence on $p$-complete modules.
  Conversely, given $(N_{k})\in \flim_{k}\rm{Mod}_{\Z/p^{k}}$ write $N= \lim_{k}N$. We want
  to show that, for every $n$ the natural map
  \[ N \otimes_{\Z_{p}} \Z/p^{n}\rar{\sim}N_{n}\]
  is an equivalence. Since $N \otimes_{\Z_{p}}Z/p^{n}\simeq N/p^{n}$ and limits are exact, we have an equivalence
  \[N \otimes_{\Z_{p}}\Z/p^{n}\simeq \lim_{k >n}(N_{k}\otimes_{\Z_{p}}\Z/p^{n}).\]
  Thus, the unit of the adjunction may be written as
  \[ \lim_{k>n}(N_{k} \otimes_{\Z_{p}}\Z/p^{n}) \to \lim_{k>n}(N_{k}\otimes_{\Z/p^{k}}\Z/p^{n})\simeq N_{n}\]
  and so has fiber given by
  \[ F_{n}:=\lim_{k>n}\left(N_{k}\otimes_{\Z/p^{k}}\rm{fib}(\Z/p^{k}\otimes_{\Z_{p}}\Z/p^{n}\to \Z/p^{n}) \right).\]
  Now we compute the fiber of $\Z/p^{k}\otimes_{\Z_{p}}\Z/p^{n}\to \Z/p^{n}$ as the module
  \[ \rm{Tor}^{\Z_{p}}(\Z/p^{k}, \Z/p^{n})[1]\simeq \Z/p^{n}[1].\]
  The reduction map $\Z/p^{k}\to \Z/p^{k-1}$ is induced by the map of projective resolutions
\[\begin{tikzcd}
	{\Z_p} & {\Z_p} \\
	{\Z_p} & {\Z_p}
	\arrow["{\cdot p^k}", from=1-1, to=1-2]
	\arrow["\id", from=1-2, to=2-2]
	\arrow["{\cdot p}"', from=1-1, to=2-1]
	\arrow["{\cdot p^{k-1}}"', from=2-1, to=2-2],
\end{tikzcd}\]
hence, on Tor it induces the multiplication by $p$ map
\[ \Z/p^{n}=\rm{Tor}^{\Z_{p}}(\Z/p^{k}, \Z/p^{n})\rar{\cdot p} \rm{Tor}^{\Z_{p}}(\Z/p^{k-1}, \Z/p^{n}) =\Z/p^{n}.\]
Thus, if we have $k\p > k > n$ such that $k\p -k > n$, the transition map
\[ F_{k\p}=N_{k\p} \otimes \rm{Tor}^{\Z_{p}}(\Z/p^{k}, \Z/p^{n})\to N_{k} \otimes \rm{Tor}^{\Z_{p}}(\Z/p^{k-1}, \Z/p^{n})= F_{k}\]
vanishes since the Tor-groups are $p^{n}$-torsion. Choosing a cofinal subset $S\subseteq \bb{N}_{>n}$ such that
$\abs{k\p -k}> n$ for any distinct $k\p,k\in S$, we see that
\[ \lim_{k>n} F_{k}\simeq \lim_{k\in S} F_{k} \simeq 0 \]
vanishes. Thus, since limits are exact, the map $N \otimes_{\Z_{p}} \Z/p^{n}\rar{\sim}N_{n}$ is an equivalence.\\
To see that the functor $\rm{Mod}_{\Z_{p}}^{\wedge} \to \flim_n \rm{Mod}_{\Z/p^{n}}$ is strong monoidal,
we observe that since cofibers and limits are exact, we have for each $n$ equivalences
\begin{align*}
  (M \otimes_{\Z_{p}} N)^{\wedge}_{p} \otimes_{\Z_{p}}\Z/p^{n} &\simeq \lim_{k}(M/p^{k} \otimes_{\Z_{p}}N/p^{k})/p^{n}\\
                                              &\simeq \lim_{k}\left((M/p^{n} \otimes_{\Z_{p}} N/p^{n})\otimes_{Z_{p}}\Z/p^{k}\right) \\
  &\simeq ((N\otimes_{\Z_{p}}\Z/p^{n}) \otimes_{\Z_{p}} (M \otimes_{\Z_{p}}\Z/p^{n}))^{\wedge}_{p}.
\end{align*}
This proves the claim.
\end{proof}

\begin{corollary}\label{pcomp1}
  We have an equivalence of categories
  \[ (\rm{cCAlg}_{\Z_{p}})_{p}^{\wedge} \rar{\sim} \flim_{n} \rm{cCAlg}_{\Z/p^{n}} \quad A \mapsto (A\otimes_{\Z_{p}}\Z/p^{n})\]
  with inverse taking a system of coalgebras $(B_{n})$ to the limit $\lim_{n}B_{n}$ taken in the
  category of ($p$-complete) $\Z_{p}$-modules, equipped with the induced $p$-complete
  $\Z_{p}$-coalgebra structure.
\end{corollary}
\begin{proof}
This follows from Proposition~\ref{pcomp}, arguing as in the proof of Proposition~\ref{Mod}.
\end{proof}

\begin{corollary}\label{obliftzp}
  Let $X(\blank)= (\rm{cCAlg}_{\blank}^{\rm{cn}})^{\Delta^{0}}$ and $A\in X(\F_{p})$ such that $T_{X_{A}}\simeq 0$.
  Then the space of lifts of $A$ to a $p$-complete $\Z_{p}$-coalgebra is contractible
\end{corollary}
 \begin{proof}
 Combine Proposition~\ref{liftpgen} and Corollary~\ref{pcomp1}.
 \end{proof}

\begin{corollary}\label{mapliftzp}
  Let $\varphi: B\to A$ be a map of connective, formally \'etale $\F_{p}$-coalgebras. Then the space of
  lifts of $\varphi$ to a map of $p$-complete $\Z_{p}$-coalgebras $B\p \to A\p$ is contractible.
\end{corollary}
\begin{proof}
    Let $ \cl{X}(\blank)=\rm{cCAlg}_{\blank}^{\rm{cn}}$. By Proposition~\ref{etalchar} the natural map
    \[ T_{\cl{X}^{\Delta^{1}}_{\varphi}} \to T_{\cl{X}^{\Delta^{0}}_{B}}\]
    is an equivalence, but since $B$ is formally \'etale we have $T_{\cl{X}^{\Delta^{0}}_{B}} \simeq 0$.
    Hence, the claim follows by applying Proposition~\ref{liftpgen} to the functor $\cl{X}^{\Delta^{1}}$
    and using Corollary~\ref{pcomp1}.
\end{proof}

Having shown this, we can now construct a functor which is analogous to the classical
Witt-Vectors, which allow us to pass from \'etale $\F_{p}$-algebras to $\Z_{p}$-algebras.

\begin{theorem}
  Let $\cl{C}\subseteq (\rm{cCAlg}_{\Z_{p}}^{\rm{cn}})^{\wedge}_{p}$ denote the full subcategory spanned by those
  coalgebras $A$ for which $A\otimes_{\Z_{p}} \F_{p}$ is formally \'etale. Then the base change functor
  \[ \cl{C} \to \rm{cCAlg}_{\F_{p}}^{\rm{cn}, \rm{f\acute{e}t}}  \qquad A \mapsto A\otimes_{\Z_{p}}\F_{p}\]
  is fully faithful and essentially surjective. In particular, the quasi inverse defines a functor
  \[ W_{p}: \rm{cCAlg}_{\F_{p}}^{\rm{cn,f\acute{e}t}} \to (\rm{cCAlg}_{\Z_{p}}^{\rm{cn}})^{\wedge}_{p}\]
  which is fully faithful and satisfies $W_{p}(A)\otimes_{\Z_{p}}\F_{p} \simeq A$ for every connective, formally
  \'etale $\F_{p}$-coalgebra $A$.
\end{theorem}

\begin{proof}
  Combine Corollary~\ref{obliftzp} and Corollary~\ref{mapliftzp}.
\end{proof}

We now turn our attention to the leap from $\Z_{p}$ to $\S_{p}^{\wedge}$. The following proposition shows that,
for an arbitrary cohesive and nilcomplete functor, a $\Z_{p}$-valued point which has vanishing $\F_{p}$-tangent
complex admits a unique lift to a $\S_{p}^{\wedge}$-valued point. This is surprising, as we do not
actually require any information about the $\Z_{p}$-tangent complex, everything is determined by
what happens modulo $p$.

\begin{proposition}\label{spherelift}
  Let $X: \rm{CAlg}^{\rm{cn}} \to \cl{S}$ be a cohesive and nilcomplete functor and let $A \in X(\Z_{p})$
  such that $T_{X_{A\otimes_{\Z_{p}}\F_{p}}}\simeq 0$. Then $A$ admits an essentially unique lift to $X(\S_{p}^{\wedge})$.
\end{proposition}

\begin{proof}
  We inductively construct lifts against the Postnikov Tower
  \[ \dots \to \tau_{\leq2} \S_{p}^{\wedge}  \to \tau_{\leq 1} \S_{p}^{\wedge} \to \tau_{\leq 0} \S_{p}^{\wedge} \simeq \Z_{p}. \]
  Write $A=A_{0},~S_{n}= \tau_{\leq n}\S_{p}^{\wedge},~ M_{n} = \pi_{n}S_{n}$ and assume we have already constructed
  a unique lift $A_{n}$ to $X(S_{n})$. Consider the square zero extension
  \[ M_{n+1}[n+1] \to S_{n+1}\to S_{n}.\]
  Since $M_{n+1} = \pi_{n+1}S_{n+1}$ is concentrated in a single degree, the $S_{n}$-action factors
  through $S_{0}=\Z_{p}$. Moreover, since $\pi_{n+1}S_{n+1}$ is of finite $p$-torsion, the action
  further factors through $\Z/p^{k}$ for some $k\geq 0$. Thus, Proposition~\ref{bc} implies that
  we have an equivalence
  \[ T_{X_{A_{n}}}^{M_{n+1}[n+1]} \simeq \Sigma^{n}T_{X_{A_{n}}}^{M_{n+1}} \simeq T_{X_{A_{n} \otimes_{S_{n}} \Z/p^{k}}}^{M_{n+1}}.\]
  Arguing as in Proposition~\ref{cofib} with respect to the square zero extension
  \[ \F_{p} \to \Z/p^{k}\to \Z/p^{k-1},\]
  we see that we have a cofiber sequence
  \[  T^{M_{n+1}\otimes_{\Z/p^{k}}\F_{p}}_{X_{A_{n} \otimes_{S_{n}} \Z/p^{k-1}}}
    \to T_{X_{A_{n} \otimes_{S_{n}} \Z/p^{k}}}^{M_{n+1}}
    \to T^{M_{n+1}\otimes_{\Z/p^{k}}\Z/p^{{k-1}}}_{X_{A_{n} \otimes_{S_{n}} \Z/p^{k-1}}}.\]
  For the left hand term, Proposition~\ref{bc} gives the equivalence
  \[ T_{X_{A_{n}\otimes_{S_{n}}\Z/p^{k-1}}}^{M_{n+1}\otimes_{\Z/p^{k}}\F_{p}}
    \simeq T_{X_{A \otimes_{\Z_{p}}\F_{p}}}^{{M_{n+1}\otimes_{\Z/p^{k}}\F_{p}}}
    \simeq T_{X_{A\otimes_{\Z_{p}}\F_{p}}}\otimes_{\F_{p}}( M_{n+1}\otimes_{\Z/p^{k}}\F_{p} ) \simeq 0,\]
  where we have used that, since $M_{n+1}$ is finitely generated, the $\F_{p}$-module
  $M_{n+1}\otimes_{\Z/p^{k}}\F_{p}$ is perfect. For the right hand term we
  replace $M_{n+1}$ with $M_{n+1} \otimes_{\Z/p^{k}}\Z/p^{k-1}$ and repeat the argument,
  inductively yielding equivalences
  \[ T^{M_{n+1}}_{X_{A_{n}\otimes_{S_{n}}\Z/p^{k}}}
    \simeq T^{M_{n+1}\otimes_{\Z/p^{k}}\Z/p^{{k-1}}}_{X_{A_{n-1} \otimes_{S_{n-1}} \Z/p^{k-1}}}
  \simeq \cdots \simeq T^{M_{n+1}\otimes_{\Z/p^{k}} \F_{p}}_{X_{A \otimes_{\Z_{p}}\F_{p}}} \simeq 0.\]
In total, this shows that $T_{X_{A_{n}}}^{M_{n+1}[n+1]} \simeq 0$, and hence $A_{n}$ admits an essentially
unique lift to $X(S_{n+1})$. Thus, the fiber over $A$ of the map
\[ X(\S_{p}^{\wedge})\simeq \flim_{n}X(S_{n})\to X( \Z_{p})\]
is contractible and we are done.
  \end{proof}

  \begin{lemma}\label{pcomparison}
    Write $\cl{X}(\blank)=\rm{cCAlg}^{\rm{cn}}_{\blank}$ and $\cl{Y}(\blank)=
    (\rm{cCAlg}^{\rm{cn}}_{\blank})^{\wedge}_{p}$. Then the $p$-completion map $f:\cl{X}\to \cl{X}\p$
    induces an equivalence
    \[ T^{M}_{(\cl{X}^{\Delta^{n}})_{\xi}} \to  T^{M}_{(\cl{Y}^{\Delta^{n}})_{f(\xi)}}\]
        for every $\F_{p}$-module $M$, $n\in \bb{N}$ and $\xi \in \cl{X}(\F_{p})^{\Delta^{n}}$.
  \end{lemma}
  \begin{proof}
    For any $\F_{p}$-algebra $R$ the $p$-completion map gives an equivalence
    $\rm{Mod}_{R}\rar{\sim} (\rm{Mod}_{R})^{\wedge}_{p}$, since multiplication by some power of $p$
    is nullhomotopic over $\F_{p}$. In particular, this applies to the split square zero
    extension $\F_{p}\oplus M$ for any $M \in \rm{Mod}_{\F_{p}}$ and so the natural map
    $\cl{X}(\F_{p}\oplus M) \to \cl{Y}(\F_{p}\oplus M)$ is an equivalence as well.
    Consequently, we also obtain natural equivalences between the fibers
    \[ (\cl{X}^{\Delta^{n}})_{\xi}^{\F_{p}\oplus M} \to  (\cl{Y}^{\Delta^{n}})_{f(\xi_)}^{\F_{p}\oplus M},\]
    which induces the equivalence of spectra
    \[ T^{M}_{(\cl{X}^{\Delta^{n}})_{\xi}} \to  T^{M}_{(\cl{Y}^{\Delta^{n}})_{f(\xi)}}\]
      as claimed.
  \end{proof}

  \begin{corollary}\label{obliftsp}
    Let $X(\blank)=(\rm{cCAlg}^{\rm{cn}}_{\blank})^{\Delta^{0}}$ and $A \in X(\F_{p})$ such that
    $T_{X_{A}}\simeq 0$, then the space of lifts of $A$ to a $p$-complete $\S_{p}^{\wedge}$-coalgebra
    is contractible.
  \end{corollary}

  \begin{proof}
    Write $Y(\blank)= ((\rm{cCAlg}^{\rm{cn}}_{\blank})^{\wedge}_{p})^{\Delta^{0}}$. Then by Lemma~\ref{pcomparison}
    we have an equivalence $T_{X_{A}}\simeq T_{Y_{A}} \simeq 0$. Hence, we can apply Proposition~\ref{obliftzp} to
    obtain an essentially unique lift $A\p\in Y(Z_{p})$. Further applying Proposition~\ref{spherelift}
    to $A\p$ yields our claim.
  \end{proof}
  Thus, we can pointwise lift $\F_{p}$-coalgebras with vanishing tangent complex to $\S_{p}^{\wedge}$. If
  we moreover consider \textit{formally \'etale coalgebras}, we can make this lifting functorial
  in a coalgebraic analogue of the \textit{Spherical Witt Vectors} construction for
  $\bb{E}_{\infty}$-algebras over $\F_{p}$.

\begin{corollary}\label{mapliftsp}
  Let $\varphi:B\to A$ be a map of $\F_{p}$-coalgebras such that $A$ and $B$ are formally \'etale.
  Then the space of lifts of $\varphi$ to a map $\varphi\p: B\p \to A\p$ of $p$-complete
  $\S_{p}^{\wedge}$-coalgebras is contractible.
\end{corollary}

\begin{proof}
  Let $ \cl{X}(\blank)=\rm{cCAlg}_{\blank}^{\rm{cn}}$ and $\cl{Y}(\blank) =
  (\rm{cCAlg}_{\blank}^{\rm{cn}})^{\wedge}_{p}$. By Proposition~\ref{mapliftzp} the map $\varphi$ admits
  an essentially unique lift to a point $\psi \in \cl{Y}(\Z_{p})^{\Delta^{1}}$. Moreover, Lemma~\ref{pcomparison}
  yields an equivalence $T_{\cl{X}^{\Delta^{1}}_{\varphi}}\simeq T_{\cl{Y}^{\Delta^{1}}_{\varphi}}$. Since both $A$ and $B$ are
  formally \'etale Proposition~\ref{etalchar} gives equivalences
  \[ T_{\cl{X}^{\Delta^{1}}_{\varphi}} \rar{\sim} T_{\cl{X}^{\Delta^{0}}_{B}} \simeq 0\]
  Hence, we can apply Proposition~\ref{spherelift} to the functor $\cl{Y}^{\Delta^{1}}$ and the point
  $\psi \in \cl{Y}^{\Delta^{1}}$, proving the claim.
\end{proof}

\begin{theorem}\label{wittsp}
  Denote by $\cl{C}\subseteq (\rm{cCAlg}_{\S_{p}^{\wedge}}^{\rm{cn}})^{\wedge}_{p} $ the full subcategory spanned by those
  coalgebras $A$ such that $A\otimes_{\S_{p}^{\wedge}}\F_{p}$ is formally \'etale. Then the base change functor
  \[ \cl{C} \to \rm{cCAlg}_{\F_{p}}^{\rm{cn}, \rm{f\acute{e}t}} \qquad A \mapsto A \otimes_{\S_{p}^{\wedge}} \F_{p}\]
  is fully faithful and essentially surjective.
\end{theorem}
\begin{proof}
  Combine Corollary~\ref{obliftsp} and Corollary~\ref{mapliftsp}.
\end{proof}

\begin{remark}
  In the setting of Theorem~\ref{wittsp} the quasi-inverse to $\blank \otimes_{\S^{\wedge}_{p}}\F_{p}$ defines
  a fully faithful functor
  \[ W_{\S_{p}^{\wedge}}: \rm{cCAlg}_{\F_{p}}^{\rm{cn}, \rm{f\acute{e}t}}
    \to (\rm{cCAlg}_{\S_{p}^{\wedge}}^{\rm{cn}})^{\wedge}_{p}\]
  which satisfies $W_{\S_{p}^{\wedge}}(A)\otimes_{\S^{\wedge}_{p}}\F_{p} \simeq A$ for every connective, formally \'etale
  $\F_{p}$-coalgebra $A$. We call $W_{\S_{p}^{\wedge}}(A)$ the \textit{spherical Witt vectors} of $A$.
\end{remark}


\subsection{Homology coalgebras}

As observed in Example~\ref{homology}, for every space $X$ and every $\bb{E}_{\infty}$-ring $R$, the
$R$-homology $R[X]$ carries a natural $R$-coalgebra structure, which is a stronger invariant than its
underlying $R$-module. We now want to apply our results and see what can be said about the deformation
theoretic behavior of homology coalgebras. To do this, we first need to compute the cotangent complex of the
$\F_{p}$-cohomology.

\begin{definition}
  A space $X\in \cl{S}$ is called $p$-finite if the following conditions hold:
  \begin{enumerate}
    \item The space $X$ is truncated.
    \item The set $\pi_{0}X$ is finite.
    \item For each $n\geq 1$ and $x\in X$, we have that $\pi_{n}(X,x)$ is a finite $p$-group.
  \end{enumerate}
  We denote the full subcategory of $\cl{S}$ spanned by the $p$-finite spaces as $\cl{S}_{p}$ and call
 $\cl{S}^{\vee}_{p} =: \rm{Pro}(\cl{S}_{p})$ the category of $p$-\textit{profinite} spaces.
\end{definition}

\begin{remark}
We can regard $\cl{S}_{p}^{\vee}$ as the category of ``formal limits'' of $p$-finite spaces $\varprojlim X_{\alpha}$.
As such there is a functor $\cl{S}^{\vee}_{p}\to \cl{S}$ which takes a formal limit to the actual limit in $\cl{S}$.
This functor admits a left adjoint given by $Y \mapsto \flim_{Y_{\alpha} \to Y} Y_{\alpha}$, where the limit runs over all maps
from a $p$-finite space $Y_{\alpha}$ to $Y$.
\end{remark}

\begin{lemma}
  Let $X$ be a space and $\flim X_{\alpha}$ be its $p$-profinite completion. Then the natural map
  of cohomology rings
  \[ \fcolim \F_{p}^{X_{\alpha}} \to \F_{p}^{X} \]
  is an equivalence.
\end{lemma}
\begin{proof}
  This is immediate since the Eilenberg-MacLane spaces $K(\F_{p},n)$ are $p$-finite.
\end{proof}

\begin{proposition}[Mandell, Lurie]\label{coetal}
  Let $X$ be a space, then the $\F_{p}$-cohomology $\F_{p}^{X}$ is a formally \'etale $\F_{p}$-algebra.
\end{proposition}
\begin{proof}
  Since the functor $R \mapsto L_{R/\F_{p}}$ commutes with colimits, the claim follows from the fact that
  $L_{\F_{p}^{X}/\F_{p}}\simeq 0$ for every $p$-finite space $X$ which is proven
  in~\cite[][Proposition 2.4.12]{dag8}.
\end{proof}

Thus we obtain the following result about the homology coalgebra of a finite space $X$
with coefficients in a connective $\F_{p}$-algebra $R$:

\begin{corollary}\label{goal}
  Let $X$ be a finite space and $R$ be an $\F_{p}$-algebra, then $R[X]$ is a formally
  \'etale $R$-coalgebra.
\end{corollary}
\begin{proof}
  From Proposition~\ref{coetal} we get that
  \[ L_{R^{X}/R}\simeq L_{\F_{p}^{X}/\F_{p}}\otimes_{\F_{p}}R \simeq 0.\]
  Since $X$ is finite, the coalgebra $R[X]$ is dualizable with dual given by $R^{X}$, so the claim
  follows from Proposition~\ref{dualetal}.
\end{proof}

Moreover, for the case $R=\F_{p}$, we can use Theorem~\ref{wittsp} to give a partial answer to our
initial question about lifts of the coalgebra $\F_{p}[X]$.

\begin{corollary}
  Let $X$ be a finite space, then $\F_{p}[X]$ admits a unique lift to a $p$-complete $\S_{p}^{\wedge}$-coalgebra
  given by $W_{\S_{p}^{\wedge}}(\F_{p}[X]) \simeq (\S[X])^{\wedge}_{p}$. Moreover, for any other finite space $Y$
  the natural map
  \[\rm{Map}_{(\rm{cCAlg}_{\S_{p}^{\wedge}})^{\wedge}_{p}}((\S[Y])^{\wedge}_{p}, (\S[X])^{\wedge}_{p})
    \to \rm{Map}_{\rm{cCAlg}_{\F_{p}}}(\F_{p}[Y], \F_{p}[X])\]
  is a homotopy equivalence.
\end{corollary}
\begin{proof}
 Combine Corollary~\ref{goal} and Theorem~\ref{wittsp}.
\end{proof}

\section{Where to go from here}

We finish our discussion by explaining some of the shortcomings of our results and sketch a possible
way to proceed towards a coalgebraic analogue of Mandell's Theorem. The first missing puzzle piece is
the cotangent complex of a coalgebra $A$, which we have been unable to give a solid definition of.
The second and more important one is the relation to the \textit{coalgebra Frobenius}. We conjecture
that the class of \textit{perfect} coalgebras defined via this map give examples of non-dualizable
formally \'etale coalgebras. In particular, this conjecture would imply that the $\F_{p}$-homology
of \textit{any} space $X$ is formally \'etale.

\subsection{The cotangent complex of a coalgebra}
One of the first questions that arose during this project turned out to be one of the most subtle and
tricky ones, namely:

\begin{question}
  What is the cotangent complex of a coalgebra $A$?
\end{question}

Clearly, the existence of a single spectrum controlling the deformation theory of $A$ would be immensely
useful. However, it is not immediately clear what the universal property of such a spectrum should be,
i.e.~which space of derivations it should (co)represent.
Some inspiration can be gleamed from Proposition~\ref{cotangentder}. There we had seen that, for
$\varphi: B \to A$ a map of $R$-coalgebras with $A$ dualizable and $M$ an $R$-module, we have an equivalence
\[ \rm{Der}_{\varphi}(B, C_{A}(M)) \simeq \rm{Map}_{A^{\vee}}(L_{A^{\vee}/R}, \varphi^{\vee}_{\pt}\rm{map}_{R}(B, M)).\]
To get rid of the dependence on the second coalgebra $B$ one is tempted to take $B=R$ such that
$\rm{map}_{R}(B,M)\simeq M$. However, not every coalgebra $A$ admits a map $R\to A$, much less a canonical
one. The only natural choice for a map that is not the initial map would yield the following:

\begin{definition}[Preliminary 1.]
  Let $R$ be an $\bb{E}_{\infty}$-ring and $A\in \rm{cCAlg}_{R}$. The cotangent complex of $A$, if it exists,
  is the $R$-module $L_{A}$ corepresenting the functor
  \[ \rm{Mod}_{R}\to \rm{Mod}_{R} \qquad M \mapsto \rm{der}_{\id}(A, C_{A}(M))\]
\end{definition}

There are however several problems with this. Firstly, it is entirely unclear from the definition
whether $L_{A}$ vanishing would actually imply $A$ being formally \'etale. Moreover, in the dualizable
case it would lead to the rather awkward formula
\[ L_{A} \simeq L_{A^{\vee}/R}\otimes_{A^{\vee}}A.\]
Although somewhat plausible, this again gives us little information about what can actually be
deduced in the case that $L_{A}\simeq 0$.
This leaves us with several options, lest we accept that there is no good notion of one singular
cotangent complex. For one we could work with \textit{coaugmented} coalgebras, namely coalgebras
together with a map $R \to A$. For the purpose of understanding homology coalgebras this would correspond
to considering pointed spaces instead of just spaces, an entirely acceptable compromise, but beyond the
scope of this paper. \\
A different  approach would be to give up on the idea of corepresentability
and instead hope for a colimit preserving functor. For example, the functor
\[ \rm{Mod}_{R}\to \rm{Mod}_{R} \qquad M \mapsto C_{A}(M):=\rm{cofib}( A \rar{\eps} \Omega^{\infty}_{A}M).\]
seems to have no chance of preserving limits, but since colimits of coalgebras are formed underlying,
colimits are not out of the race. This leads us to the following idea:

\begin{definition}[Preliminary 2]\label{dream}
  Let $R$ be an $\bb{E}_{\infty}$-ring and $A\in \rm{cCAlg}_{R}$. We say that $A$ admits a cotangent
  complex $L_{A}:= C_{A}(R)$ if the functor $C_{A}(\blank):\rm{Mod}_{R} \to \rm{Mod}_{R}$ commutes
  with colimits. In this case we have $C_{A}(M)\simeq L_{A}\otimes M$ for every $ M \in \rm{Mod}_{R}$
\end{definition}

This definition is highly speculative, as the only coalgebras we know to admit a cotangent complex
in this sense are the formally \'etale coalgebras, for which the functor $C_{\blank}(A)$ is constant.
Conversely, if $A$ admits a cotangent complex then $L_{A}$ vanishes if and only if $A$ is formally
\'etale. Hence, the spectrum $L_{A}$ is precisely the obstruction to $A$ being formally \'etale,
which is the kind of conceptual clarity we are looking for.
While we lose any direct comparison to the cotangent complex of $A^{\vee}$ this is not entirely surprising,
since the property of being formally \'etale is defined very differently for $A^{\vee}$.
This leaves us with the following:

\begin{question}\label{cotangentdream}
  Let $R$ be an $\bb{E}_{\infty}$-ring. Does every $A \in \rm{cCAlg}_{R}$ admit a cotangent complex in the sense
  of Definition~\ref{dream}?
\end{question}

Regardless of the answer, the takeaway should be that the modules
$C_{A}(M)$ are exactly the obstruction towards $A$ being formally \'etale. Moreover, while the functor
$A\mapsto C_{A}(M)$ is very complicated, the dependence on $M$ should be relatively tame. That is,
for fixed $A$ it should be possible to describe the functor $M \mapsto C_{A}(M)$ in terms of a
formula involving $C_{A}(R)$. However, because $C_{A}(M)$ no longer has a direct relation to any
space of derivations or tangent complex, we cannot leverage results like Proposition~\ref{structure}
to obtain such a formula. We understand this as an indication that for these questions, the formalism may
have reached its limit.

\subsection{The Frobenius}
The most lacking thing about our results is the class of coalgebras that we can currently apply them to.
As of now, we are unable to give examples of formally \'etale coalgebras which are not dualizable. In
particular, we cannot describe the deformation theory of $R[X]$ for spaces $X$ which are not finite.
Attempts to reduce to the dualizable case all seem to fail for the following reason: Even though
we may write $X= \fcolim_{i}X_{i}$ where each $X_{i}$ is finite, giving the formula
$R[X]= \fcolim_{i}R[X_{i}]$, there is no reason why the functor
$\Omega^{\infty}_{\blank}(M): \rm{cCAlg}_{R}\to \rm{cCAlg}_{R}$ should commute with colimits.
Indeed, write $f_{M}:R\to R\oplus M$ for inclusion, then by definition
$\Omega^{\infty}_{\blank}(M) = f_{M,!} f^{\pt}_{M}$. The functor $f^{\pt}_{M}$ commutes with colimits,
and from Proposition~\ref{present} and the converse of the adjoint functor theorem we can deduce
that $f_{M,!}$ commutes with $\kappa$-filtered colimits for some regular cardinal $\kappa$. Thus, the class
of formally \'etale coalgebras is closed under $\kappa$-filtered colimits, but $\kappa$ is, in general, not countable.
% Closely related is the fact the notion of compactness is strangely behaved for coalgebras. For example,
% one can show that $\bb{Q}$ is not a compact object of $\rm{cCAlg}_{\Q}$, see~\cite[][Warning 1.2.15.]{ellII}.
% In particular, this means that
% \[ \rm{cSpec}(\fcolim_{i}\S[X_{i}])(\bb{Q})\neq \fcolim_{i}\rm{cSpec}(\S[X_{i}])(\Q),\]
% so we cannot deduce things about the cospectrum of infinite spaces in this way either. \\
This goes to show that the deformation theory of non-dualizable coalgebras is richer and more
interesting than that of the Ind-completion of dualizable coalgebras and requires additional input.
One contender for this additional input is the \textit{Coalgebra Frobenius} constructed by
Nikolaus:

\begin{theorem}[Nikolaus]
  Let $\cl{C} = (\rm{cCAlg}^{\rm{cn}}_{\S^{\wedge}_{p}})^{\wedge}_{p}$, then there exists a natural transformation
  $\psi_{p}:\id_{\cl{C}}\to \id_{\cl{C}}$ which on an object $A\in \cl{C}$ is given by the composition
  \[ \psi_{p}: A \rar{\Delta_{A}^{\otimes p}} (A^{\otimes p})^{hC_{p}} \rar{\rm{can}} (A^{\otimes p})^{tC_{p}} \rar{\sim} A,\]
  where the final map is the inverse of the \textit{Tate Diagonal}, see~\cite[][Theorem III.1.7]{tch}.
\end{theorem}

Given this map, we are naturally led to define \textit{perfect} coalgebras as follows:

\begin{definition}
  We say that $A \in  (\rm{cCAlg}^{\rm{cn}}_{\S^{\wedge}_{p}})^{\wedge}_{p}$ is \textit{perfect} if the coalgebra
  Frobenius $\psi_{p}: A\to A$ is a homotopy equivalence. We denote the full subcategory spanned by
  the perfect coalgebras by $(\rm{cCAlg}^{\rm{cn}}_{\S^{\wedge}_{p}})^{\wedge ,\rm{perf}}_{p} \subseteq
  (\rm{cCAlg}^{\rm{cn}}_{\S^{\wedge}_{p}})^{\wedge}_{p}$.
\end{definition}

\begin{example}\label{frobchains}
  Let $X$ be any space. Then $(\S[X])^{\wedge}_{p}$ is a perfect coalgebra since we have that
  \[\S[X]^{\wedge}_{p} \simeq (\S_{p}^{\wedge}[\colim_{X}\pt])^{\wedge}_{p} \simeq (\colim_{X} \S_{p}^{\wedge})^{\wedge}_{p}.\]
  On $\S_{p}^{\wedge}$ the map $\psi_{p}$ is necessarily given by the identity, because $\S_{p}^{\wedge}$
  is the terminal $p$-complete $\S_{p}^{\wedge}$-coalgebra. Thus, by naturality $\psi_{p}$ is given
  by the identity on $(\S[X])^{\wedge}_{p}$ as well.
\end{example}

We conjecture that this Frobenius map is related to the deformation theory of coalgebras in a similar
way to the Algebra Frobenius, in that it provides a sufficient condition for a coalgebra to be formally
\'etale.

\begin{conjecture}\label{frobcof}
  Let $A \in (\rm{cCAlg}^{\rm{cn}}_{\S^{\wedge}_{p}})^{\wedge}_{p}$ and write $A\p= A\otimes_{\S^{\wedge}_{p}}\F_{p}$.
  Then for any $M \in \rm{Mod}_{\F_{p}}^{\rm{cn}}$, the coalgebra Frobenius $\psi_p:A\to A$ induces the zero map
  on the $R$-module  $C_{A\p}(M) = \rm{cofib}(A\p \rar{\eta_{A\p}} \Omega^{\infty}_{A}(M))$.
\end{conjecture}

\begin{corollary}
  If Conjecture~\ref{frobcof} holds, then the base change functor
  \[ (\rm{cCAlg}^{\rm{cn}}_{\S^{\wedge}_{p}})^{\wedge ,\rm{perf}}_{p} \to \rm{cCAlg}_{\F_{p}}^{\rm{cn}}
  \qquad A \mapsto A\otimes_{\S_{p}^{\wedge}}\F_{p}\]
is fully faithful and factors through the full subcategory
$\rm{cCAlg}_{\F_{p}}^{\rm{cn}, \rm{f\acute{e}t}}\subseteq \rm{cCAlg}_{\F_{p}}^{\rm{cn}}$.
\end{corollary}
\begin{proof}
  Since $\psi_{p}:A\rar{\sim} A$ is an equivalence it induces an equivalence on $A\otimes_{\S_{p}^{\wedge}}\F_{p}$ and
  thus on $C_{A\otimes_{\S_{p}^{\wedge}}\F_{p}}(M)$ as well. However, since it also induces the zero map on the latter
  we get that $C_{A\otimes_{\S_{p}^{\wedge}}\F_{p}}(M)\simeq 0$. Thus, $A\otimes_{\S_{p}^{\wedge}}\F_{p}$ is formally \'etale and the
  claim follows from Theorem~\ref{wittsp}.
\end{proof}

Combining this with Example~\ref{frobchains} would allow us to fully answer our initial question about
homology coalgebras.

\begin{corollary}\label{dream2}
  If Conjecture~\ref{frobcof} holds, then for any space $X$ the $\F_{p}$-chains $\F_{p}[X]$
  are formally \'etale. In particular $\F_{p}[X]$ admits a unique and functorial lift to a $p$-complete
  $\S_{p}^{\wedge}$-coalgebra given by $\S[X]^{\wedge}_{p}= W_{\S_{p}^{\wedge}}(\F_{p}[X])$.
\end{corollary}

The fact that Conjecture~\ref{frobcof} needs to be checked for every connective $\F_{p}$-module should
be understood as an extension of our failure to find a cotangent complex. Indeed, if $\F_{p}[X]$ admits
a cotangent complex in the sense of Definition~\ref{dream}, then to obtain Corollary~\ref{dream2} it
would suffice to show that $\psi_{p}$ induces the zero map on $C_{A\otimes_{\S_{p}^{\wedge}}\F_{p}}(\F_{p})
= L_{A\otimes_{\S_{p}^{\wedge}}\F_{p}}$. However, even for this specific module the conjecture is difficult
to attack from our present position. The problem is the tricky right adjoint
$\rm{cCAlg}_{\F_{p}\oplus \F_{p}}\to \rm{cCAlg}_{\F_{p}}$ appearing in the definition of
$C_{A\otimes_{\S^{\wedge}_{p}}\F_{p}}(\F_{p})$. Because there is no known formula for this functor, attempts to verify
the conjecture have thus far been unsuccessful in all non-trivial cases. This warrants further investigation
of the coalgebra Frobenius and Conjecture~\ref{dream2}.


% \begin{figure*}[t]%[t]% !]
    \centering
    \includegraphics[width=0.7\textwidth]{supplmat/imgs/meshes2.png}
    \caption{\textbf{Meshes used by \methodname.} 
  }
    \label{fig:meshes}
\end{figure*}

% \subsection{Additional Results}
%\vspace{2pt}\noindent\textbf{View-dependent Effects.}
%To study the impact of viewpoint in \methodname we report reflectance maps in Figure~\ref{fig:reflectance_maps}. \albert{I guess this should go now, it is in the main paper}
% \begin{figure}[t]
    \centering
    \includegraphics[width=0.7\columnwidth]{imgs/ViewDirPpt.pdf}
    % \vspace{-0.5cm}
    \caption{
    \albert{remove}
   
    We show color as a function of elevation (y-axis) and azimuth (x-axis) for one point on the surface of the \textit{Materials} scene from the Synthetic dataset. 
    The first row is a pseudo-GT (extracted from MipNeRF), and subsequent rows are \methodname's output when varying the light field embedding dimension $D$. 
    }
    \vspace{-0.2cm}
    \label{fig:reflectance_maps}
\end{figure}
\textbf{Additional Results.} Finally, we report detailed qualitative and quantitative results, validation of view-dependent effects, sensitivity to geometry variations and photo-metric quality depending on the dimensionality $D$ of \methodname in the \SM.

\begin{comment}
\vspace{2pt}\noindent\textbf{Qualitative and Quantitative Results.}
Due to space constraints, we comprehensively report qualitative and quantitative results in the Appendix.

\vspace{2pt}\noindent\textbf{Comparison against RGB-textured mesh.}
Our approach is essentially a mesh representation whose texture can model view-dependent effects.
Thus, for completeness, we compare \methodname against a naive RGB-texturized mesh in the Appendix.

\vspace{2pt}\noindent\textbf{Sensitivity to geometry variations.}
For a fixed mesh $\chi$, \methodname distills the color information of the scene into a NeLF.
In the paper, we construct $\chi$ to accurately approximate the scene's geometry.
In the Appendix, we explore how degraded geometries affect the photo-metric quality achieved by \methodname.

\vspace{2pt}\noindent\textbf{Sensitivity to embedding size.}
In the Appendix, we report how photo-metric quality is affected by the dimensionality $D$ of the light field embeddings, as defined in Eq.~\eqref{eq:dir-pos}.

\end{comment}

% \begin{table}[]
    \centering
        \caption{Zero-shot task performance of \texttt{base/large} models after parameter-efficient training.$LwA$/$DA$ indicates adapter types, corresponding to (rows h/f in Table \ref{tab:ablations}). }
    \label{tab:performance}
    \begin{adjustbox}{max width=\textwidth}
        \begin{tabular}{lccccccccc}
% \hline & \multicolumn{4}{c}{ Pre-training Task } & & \multicolumn{2}{c}{ Zero-Shot Performance } \\
% \cline { 2 - 5 } \cline { 6 - 7 } 
\toprule
% \multicolumn{4}{c}{Components} &  \multicolumn{5}{c}{Flickr30k True 0-shot (1k test set)} \\
\multicolumn{4}{c}{Model (591k Training Pairs)} &  & \multicolumn{2}{c}{Flickr} & & \multicolumn{2}{c}{ImageNet V2} \\
\cmidrule(l{0.5em}r{0.5em}){2-4}  \cmidrule(l{0.5em}r{0.5em}){5-8} \cmidrule(l{0.5em}r{0.5em}){9-10}
& Configuration & \# Trainable & \% Trained &  TR@1 & IR@1 & TR@5 & IR@5 & Acc-1 & Acc-5 \\
\midrule
 % (a) &   LilT-tiny & 736.45 K & 7.37\% & 15.7 & 12.4 & 37.4 & 31.56 & 5.61 & 14.54  \\
 % (c) &   LilT-small & 5.19 M & 10.28\% & 37.6 & 27.38 & 66.9 & 54.96 & 10.92 & 23.99  \\
 (a) &   LilT$_{DA}$-base & 14.65 M & 7.51\% & 47.6 & 34.46 & 74.1 & 64.92 & 12.94 & 28.39  \\
 (b) &   LilT$_{DA}$-large & 25.92 M & 4.06\% & 57.6 & 42.18 & 82.2 & 72.38 & 13.97 & 30.89  \\ \midrule
  (c) &   LilT$_{LwA}$-base & 14.67 M & 7.01\% & 56.8 & 41.7 & 81.1 & 70.74 & 12.18 & 27.78  \\
   (d) &   LilT$_{LwA}$-large & 51.18 M & 7.43\% & 63.5 & 50.7 & 88.5 & 79.14 & 14.05 & 31.31 \\
 \midrule
 % (b) & LiT-tiny & 4.45 M & 44.57\% & 25.0 & 16.62 & 49.2 & 40.06 & 10.06 & 22.15  \\
 % (g) & LiT-small & 28.73 M & 56.98\% & 37.3 & 26.46 & 66.6 & 54.78 & \textbf{15.14} & 29.16  \\
 (e) & LiT-base & 109.28 M & 56.01\% & 44.1 & 29.64 & 72.1 & 59.94 & 15.0 & 29.44 \\
 % (d) & CLIP-tiny & 9.99 M & 100.0\% & 34.6 & 25.1 & 62.0 & 51.8 & 7.46 & 18.43  \\
 % (h) & CLIP-small & 50.42 M & 100.0\% & 50.4 & 38.34 & 76.8 & 66.56 & 10.64 & 24.42  \\
 (f) & CLIP-base & 195.13 M & 100.0\% & 56.1 & 44.3 & 81.7 & 71.98 & 12.29 & 28.44  \\
\bottomrule
% (e) & $\checkmark$ & & $\checkmark$ & & & $-$ & $-$ & & & &\\
% (f) & $\checkmark$ & & & $\checkmark$ & & $-$ & $-$ & & & &\\
% (g) & & $\checkmark$ & $\checkmark$ & & & $-$ & $-$ & & & &\\
% \hline (h) & $\checkmark$ & & $\checkmark$ & $\checkmark$ & & $-$ & $-$ & & & &\\
% (i) & $\checkmark$ & $\checkmark$ & & $\checkmark$ & & $-$ & $-$ & & & &\\
% (j) & $\checkmark$ & $\checkmark$ & $\checkmark$ & & & $-$ & $-$ & & & &\\
% (k) & & $\checkmark$ & $\checkmark$ & $\checkmark$ & & $-$ & $-$ & & & &\\
% (k) & & $\checkmark$ & $\checkmark$ & $\checkmark$ & & $-$ & $-$ & & & &\\
% \hline (l) & $\checkmark$ & $\checkmark$ & $\checkmark$ & $\checkmark$ & & $-$ & $-$ & & & &\\
% \hline
\end{tabular}
    \end{adjustbox}
\end{table}
% \begin{itemize}
%     \item We use MipNerF for  Realistic Synthetic 360° dataset and Nerf++ for 360° Unbounded Tanks and Temples dataset.
%     \item We sample 256x256x256 density points for Realistic Synthetic 360° dataset except ficus (512x512x512) and 512x512x512 for for 360° Unbounded Tanks and Temples dataset
%     \item We decimated all scene up to 250.000, for real scene we aggregate a semi sphere and a plane to recreate the floor of the scene. We remove small components inn all the scenes.
%     \item We generate 10,000 images for all objects except ficus, mic (because low number of intersected points) in which we obtained 14,000. Random poses for Realistic Synthetic 360° dataset and interpolation rnaodm poses.
%     the psnr for the first stages are these ... TABLE
%     \item Following~\cite{r2l}, we employ intensive residual blocks~\cite{resnet} and deep MLPs for each $G_{\text{pos}}$ and $G_{\text{dir}}$ network. We utilize hard ray and warm up strategies.
%     \item  In \cite{resnet}, residual connections were demonstrated to be essential for enabling the much greater network depth, and the same holds here for learning the light field.
%     \item training time is 1 day on one A100, batch size 200,000.
%     \item We extract a png for each uvwb component. we normalized min max per channel. 
%     \item time to create texture map 50 min. 
%     \item 0-360 azimuth and elevation, 1024x1024 synthetic dataset and 2048x2048 real scenes. floor approximation for angle query.
%     \item texel per triangle used in all experiments except plot X is 5x5/2.
%     \item We implement a fragment shader to combine uvw with direction.
%     \item uvw and beta are stored in a single texture each (4 textures total), each encoded as a 2x4 grid of matrices
    
%     \item direction is calculated in a vertex shader as the vector from camera to vertex, resulting in a direction vector interpolated from all face vertices which is then input to the fragment shader
%     \item the shader converts direction from cartesian to spherical coordinates, and uses the resulting [0,1] normalized elevation and azimuth to index the beta texture
%     \item fragment UV coordinates are used to index the uvw feature textures.
%     % \item Since we use triangular meshes, 
%     \item Since textures are stored as discretized 8-bit integers, we map to their original values by applying the inverse of the min-max normalization per-channel (32 values each for uvw, beta)
%     \item We compute the final color for each fragment by performing a dot product between the uvw and beta before applying a sigmoid function
%     \item No alpha compositing is required since our pipeline takes into account all transparencies with the light field formulation. Thus fragment alpha is set to 1.
    
    
% \end{itemize}

% In \cite{resnet}, residual connections were demonstrated to be essential for enabling the much greater network depth, and the same holds here for learning the light field.



% \begin{table}[t!]
    \footnotesize
    \centering
    \setlength{\tabcolsep}{1.5pt}
    \begin{tabular}{c"c"c"c"c}
        \thickhline
        Device         & Type    & OS              & GPU                     & Browser \\ 
        \thickhline
        % iPhone 12       & Phone   & iOS             & Apple GPU               & Chrome  \\
        Samsung S21    & Phone   & Android 13      & Mali G78                & Chrome  \\
        Motorola G9    & Phone   & Android 11      & Adreno 610              & Chrome    \\
        Galaxy S6      & Tablet  & Android 13       & Mali G72 MP3               & Firefox   \\
        Dell           & Laptop  & Windows 10      & Integrated GPU           & Firefox  \\
        Gaming Lap.    & Laptop  & Windows 10      & NVIDIA GF RTX 2060      & Firefox   \\
        Desktop             & Desktop & Ubuntu 18.04    & NVIDIA GF RTX 3090      & Chrome \\
        Meta Quest P. & Headset & Oculus & Adreno 650 & $-$ \\
        \thickhline
    \end{tabular}
    % \vspace{-0.2cm}
     % \vspace{0.07cm}
    \caption{\textbf{Testing Devices.}
    % \sara{update tablet}
    We compare \methodname against other methods on a set of representative devices with a wide range of compute capabilities. 
    Our devices include low- to high-end mobile phones, tablets, laptops, desktop computers, and a VR headset. % \juan{are we keeping the G9?}
    % We show in this table all devices with corresponding characteristics.
    }
    \vspace{-0.45cm}
    \label{tab:0_hdw_specs_tb}
\end{table}

% \begin{table}[]
% \footnotesize
% \centering
% \setlength{\tabcolsep}{2.5pt}
% \begin{tabular}{lllll}\thickhline
% Device         & Type    & OS              & GPU                     & Power \\ \hline
% Iphone 12      & Phone   & iOS             & Apple GPU               & 7W    \\
% Motorola G9    & Phone   & Android 11      & Adreno 610              & 7W    \\
% Lenovo Tablet  & Tablet  & Android 9       & Adreno 505              & 10W   \\
% Dell           & Laptop  & Windows 10      & Integrated GPU          & 15W  \\
% Gaming Lap.    & Laptop  & Windows 10      & NVIDIA GF RTX 2060      & 90W   \\
% PC             & Desktop & Ubuntu 18.04    & NVIDIA GF RTX 3090      & 450W \\\thickhline
% \end{tabular}
% \caption{\textbf{Hardware specs} – of the devices used in our rendering experiments. The power is the max GPU power for discrete NVIDIA cards, and the combined max CPU and GPU power for integrated GPUs.}
% \label{tab:0_hdw_specs_tb}
% \end{table}

% \begin{table}[]
% \footnotesize
% \centering
% \setlength{\tabcolsep}{1pt}
% \begin{tabular}{llllll}\thickhline
% Device                                                        & Browser & Type    & OS                 & GPU                     & Power \\ \hline
% \begin{tabular}[c]{@{}l@{}}Motorola G9\\ Power\end{tabular}   & Chrome  & Phone   & Android 11         & Adreno 610              & 7W    \\
% \begin{tabular}[c]{@{}l@{}}Lenovo \\ Yoga Tablet\end{tabular} & Chrome  & Tablet  & Android Pie (9.0)  & Adreno 505              & 10W   \\
% Dell                                                          & Firefox & Laptop  & Windows 10 Pro     & Integrated GPU          & 15 W  \\
% \begin{tabular}[c]{@{}l@{}}Gaming \\ Laptop\end{tabular}                                                 & Firefox & Laptop  & Windows 10 Pro     & NVIDIA GeForce RTX 2060 & 90W   \\
% PC                                                            & Chrome  & Desktop & Ubuntu 18.04.6 LTS & NVIDIA GeForce RTX 3090 & 450W \\\thickhline
% \end{tabular}
% \caption{\textbf{Hardware specs} – of the devices used in our rendering experiments. The power is the max GPU power for discrete
% NVIDIA cards, and the combined max CPU and GPU power for
% integrated GPUs.}
% \label{tab:0_hdw_specs_tb}
% \end{table}

% % \begin{table}[t!]
%     \footnotesize
%     \centering
%     \setlength{\tabcolsep}{1.5pt}
%     \begin{tabular}{l"c|c|c"c|c|c}
%         \thickhline
%          & \multicolumn{3}{c"}{Synthetic 360°} & \multicolumn{3}{c}{Unbounded 360°} \\
%         Device & SNeRG & M-NeRF & \methodname & SNeRG & M-NeRF & \methodname \\
%         \thickhline
%         % iPhone 12 & N/A & 57.9 & \best{60.0}$^{\star}$ & N/A & 38.2 & \best{60.0}$^{\star}$ \\
%         Samsumg S21 & N/A & 57.9 & \best{60.0}$^{\star}$ & N/A & 38.2 & \best{60.0}$^{\star}$ \\
%         G9 & N/A & 10.6 & \best{22.8} & N/A & 3.3 & \best{16.0} \\
%         Yoga & N/A & 5.7 & \best{13.8} & N/A & 2.0 & \best{9.2} \\
%         Dell & N/A & 49.8 & \best{83.3} & N/A & 16.9 & \best{62.7} \\
%         Gaming & 192.7 & 473.40 & \best{625.1} & 42.3 & 169.5 & \best{503.5} \\
%         Desktop & 688.9 & 1,075.2 & \best{1,820.2} & 83.9 & 478.5 & \best{1,648.0} \\
%         Headset & N/A & - & \best{74.0}$^{\star}$ &  N/A & - & \best{74.0}$^{\star}$ \\
%         \thickhline
%         GPU (MB) & 5,707.3 & 538.4 & \best{239.3} & 3,388.3 & 1,063.0 & \best{397.5} \\
%         Disk (MB) & \best{87.0} & 125.75 & 143.5 & 324.8 & 311.3 & \best{233.5} \\
%         \thickhline
%         PSNR (dB) & 30.4 & \best{30.9} & 29.7 & 14.0 & 15.6 & \best{17.9} \\
%         \thickhline
%     \end{tabular}
%     % \vspace{-0.1cm}
%     \caption{\textbf{On-device Performance.} 
%     \sara{needs to update} We compare the performance of \methodname against MobileNeRF (M-NeRF)~\cite{chen2022mobilenerf} and SNeRG~\cite{hedman2021snerg} across devices. 
%     The first seven rows report rendering speed in frames-per-second~(FPS).
%     In notation, $``^{\star}"$ means the device's FPS limit was reached, $``-"$ means missing implementation, and ``N/A'' means the method failed to run.
%     % For SNeRG,  denotes the method was unable to run due to out-of-memory errors. %  generated when running, precluding rendering in the specified device. 
%     % SNeRG is unable to run on resource-constrained devices, which we denote by .
%     %Note how \methodname provides significantly larger FPS than the competitors, particularly on real scenes. 
%     The second set of rows reports GPU memory and disk space requirements.
%     Finally, we report average PSNR for each dataset.
%     Note \methodname provides the fastest rendering across scenes with comparable memory and disk requirements.
%     For synthetic scenes, our FPS gains come with limited cost to quality.
%     In unbounded scenes, our method improves \textit{both} PSNR and FPS by sizable margins. 
%     Moreover, we highlight that \methodname is capable of real-time rendering even on VR headsets. % , and achieves real-time rendering speeds. 
%     }
%     \label{tab:1_Rendering_speed_tb}
% \end{table}

\begin{table}[t!]
    \footnotesize
    \centering
    \setlength{\tabcolsep}{1.5pt}
    \begin{tabular}{l"c|c|c"c|c|c}
        \thickhline
         & \multicolumn{3}{c"}{Synthetic 360°} & \multicolumn{3}{c}{Unbounded 360°} \\
        Device & SNeRG & M-NeRF & \methodname & SNeRG & M-NeRF & \methodname \\
        \thickhline
        % iPhone 12 & N/A & 57.9 & \best{60.0}$^{\star}$ & N/A & 38.2 & \best{60.0}$^{\star}$ \\
        Samsung S21 & $\dagger$ & \textcolor{black}{41.7} & \textcolor{black}{\best{54.7}} & $\dagger$ & \textcolor{black}{22.8} & \textcolor{black}{\best{33.5}} \\
        G9 & $\dagger$ & \textcolor{black}{9.7} & \textcolor{black}{\best{10.4}} & $\dagger$ & \textcolor{black}{3.5} & \textcolor{black}{\best{7.7}} \\
        \textcolor{black}{Galaxy S6 } & N/A & \textcolor{black}{18.1} & \textcolor{black}{\best{26.6}} & $\dagger$ & \textcolor{black}{6.2} & \textcolor{black}{\best{20.4}} \\
        Dell & $\dagger$ & \textcolor{black}{49.8} & \textcolor{black}{\best{75.3}} & $\dagger$ & \textcolor{black}{16.9} & \textcolor{black}{\best{54.0}} \\
        Gaming & \textcolor{black}{197.9}& \textcolor{black}{496.0} & \textcolor{black}{\best{697.3}} & \textcolor{black}{46.0} & \textcolor{black}{169.9} & \textcolor{black}{\best{516.6}} \\
        Desktop & \textcolor{black}{502.1} & \textcolor{black}{762.3} & \textcolor{black}{\best{1,013.2}} & \textcolor{black}{141.1} & \textcolor{black}{389.8} & \textcolor{black}{\best{925.4}} \\
        Headset & $\dagger$ & $-$ & \best{74.0}$^{\star}$ &  $\dagger$ & $-$ & \best{74.0}$^{\star}$ \\
        % \thickhline
        % GPU (MB) & 5,707.3 & 538.4 & \best{239.3} & 3,388.3 & 1,063.0 & \best{397.5} \\
        % Disk (MB) & \best{87.0} & 125.75 & 143.5 & 324.8 & 311.3 & \best{233.5} \\
        % \thickhline
        % PSNR (dB) & 30.4 & \best{30.9} & 29.7 & 14.0 & 15.6 & \best{17.9} \\
        \thickhline
    \end{tabular}
    \vspace{0.07cm}
    \caption{\textbf{On-device rendering speed (FPS).} 
    % \sara{needs to update} 
    We compare the rendering speed of \methodname against MobileNeRF (M-NeRF)~\cite{chen2022mobilenerf} and SNeRG~\cite{hedman2021snerg} across devices. 
    % The first seven rows report rendering speed in frames-per-second~(FPS).
    Conventions: $``^{\star}"$ means the device's FPS limit was reached, $``-"$ means missing implementation, and ``$\dagger$" means the method failed to run.
    % For SNeRG,  denotes the method was unable to run due to out-of-memory errors. %  generated when running, precluding rendering in the specified device. 
    % SNeRG is unable to run on resource-constrained devices, which we denote by .
    %Note how \methodname provides significantly larger FPS than the competitors, particularly on real scenes. 
    % The second set of rows reports GPU memory and disk space requirements.
    % Finally, we report average PSNR for each dataset.
    % Note \methodname provides the fastest rendering across scenes with comparable memory and disk requirements.
    % For synthetic scenes, our FPS gains come with limited cost to quality.
    % In unbounded scenes, our method improves \textit{both} PSNR and FPS by sizable margins. 
    For all devices and datasets, \methodname provides the fastest rendering speeds, usually by a large margin.
    Furthermore, we highlight that \methodname is capable of real-time rendering even on VR headsets. % , and achieves real-time rendering speeds. 
    }
    \vspace{-0.6cm}
    \label{tab:1_Rendering_speed_tb}
\end{table}


% \begin{table}[t!]
\footnotesize
\centering
\setlength{\tabcolsep}{3pt}
\begin{tabular}{l"lll"lll}\thickhline
Dataset      & \multicolumn{3}{c"}{Synthetic 360°} & \multicolumn{3}{c}{Unbounded 360°} \\
Method       & Ours    & SNeRG    & \shortstack{Mobile \\ NeRF}    & Ours  & SNeRG          & \shortstack{Mobile \\ NeRF}         \\\hline
GPU memory   &   239.3      &    5,707.3      &      538.38         &      397.5   &   3,388.3  &   1,063.0                 \\
Disk storage &   143.5      &    87.0      &         125.75      &    233.47   & 324.8  &    311.3               \\\thickhline
\end{tabular}
\caption{\textbf{Resources} – memory and disk storage (MB).}
\label{2_Resources}
\end{table}


% \begin{table}[t!]
    \footnotesize
    \centering
    \setlength{\tabcolsep}{2pt}
    \begin{tabular}{l"c|c|c"c|c|c}
        \thickhline
        & \multicolumn{3}{c"}{Synthetic 360°} & \multicolumn{3}{c}{Unbounded 360°} \\
        Method     & PSNR $\uparrow$      & SSIM  $\uparrow$     & LPIPS $\downarrow$     & PSNR  $\uparrow$    & SSIM  $\uparrow$     & LPIPS  $\downarrow$    \\\hline
        NeRF \cite{mildenhall2021nerf}      &   31.00  &    0.947        &     0.081       &   18.98         & 0.712        &  0.573     \\
        Mip-NeRF \cite{barron2021mipnerf}      &   33.09  &    0.961        &     0.043       &   -         & -        &  -     \\
        NeRF++  \cite{nerf++}   &   -             &   -             &    -            &   20.08  & 0.758 &  0.412     \\
        ADOP  \cite{ruckert2022adop}   &   -             &   -             &    -            &   23.53*  & - &  0.151*     \\\hline
        SNeRG   \cite{hedman2021snerg}  &    30.38        &   \best{0.950} & \best{0.050}     &   14.04           & 0.528 &  0.548     \\
        M-NeRF  \cite{chen2022mobilenerf}   &    \best{30.90} &   0.947         &   0.062         &   15.59           & 0.474        &  0.535     \\
        Ours       &    29.66        &   0.939         &  0.068         &   \best{17.91}    & \best{0.542}        &  \best{0.513}     \\
        % Ours$\bigstar$  &    28.67   &   0.927         &  0.079          &   18.31         & 0.561        &  0.488   \\
        \thickhline
    \end{tabular}
    % \vspace{-0.2cm}
    \caption{\textbf{Rendering Quality.} \sara{ADOP does not have truck scene but it has lighthouse???} In this table we present common measures of image quality for renderings from \methodname and several other non-real and real time NeRF pipelines on edge devices. We present this table as reference to show that \methodname achieves comparable image quality to more complex NeRF architectures, while being significantly faster. Note that we extract numbers directly from each corresponding paper, and omit values (with -) that were not originally reported by the methods. 
    }
    % \vspace{-0.2cm}
    \label{tab:3_Quantitative_tb}
\end{table}
% \begin{table}[t!]
    \footnotesize
    \centering
    \begin{tabular}{l"c|c"c|c}
        \thickhline
        & \multicolumn{2}{c"}{Synthetic 360°} & \multicolumn{2}{c}{Unbounded 360°}\\
        Method & M-NeRF & \methodname & M-NeRF & \methodname\\
        \thickhline
        Vertices & 494,289 &  \best{99,539} & 332,678 & \best{117,751}\\
        Faces    & 224,341 & \best{205,693} & 543,047 & \best{244,847}\\
        \thickhline                                            
    \end{tabular}
    % \vspace{-0.2cm}
    \caption{\textbf{Average Mesh Size.} We report the average number of vertices and triangle faces for meshes used by \methodname and MobileNeRF. 
    The meshes used by our method are substantially smaller than those used by MobileNeRF. %  is more economical in mesh requirements size.
    }
    % \vspace{-0.4cm}
    \label{tab:4_Polygon_count_tb}
\end{table}