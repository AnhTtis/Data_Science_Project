\begin{table}[t!]
    \footnotesize
    \centering
    \setlength{\tabcolsep}{1.5pt}
    \begin{tabular}{c"c"c"c"c}
        \thickhline
        Device         & Type    & OS              & GPU                     & Browser \\ 
        \thickhline
        % iPhone 12       & Phone   & iOS             & Apple GPU               & Chrome  \\
        Samsung S21    & Phone   & Android 13      & Mali G78                & Chrome  \\
        Motorola G9    & Phone   & Android 11      & Adreno 610              & Chrome    \\
        Galaxy S6      & Tablet  & Android 13       & Mali G72 MP3               & Firefox   \\
        Dell           & Laptop  & Windows 10      & Integrated GPU           & Firefox  \\
        Gaming Lap.    & Laptop  & Windows 10      & NVIDIA GF RTX 2060      & Firefox   \\
        Desktop             & Desktop & Ubuntu 18.04    & NVIDIA GF RTX 3090      & Chrome \\
        Meta Quest P. & Headset & Oculus & Adreno 650 & $-$ \\
        \thickhline
    \end{tabular}
    % \vspace{-0.2cm}
     % \vspace{0.07cm}
    \caption{\textbf{Testing Devices.}
    % \sara{update tablet}
    We compare \methodname against other methods on a set of representative devices with a wide range of compute capabilities. 
    Our devices include low- to high-end mobile phones, tablets, laptops, desktop computers, and a VR headset. % \juan{are we keeping the G9?}
    % We show in this table all devices with corresponding characteristics.
    }
    \vspace{-0.45cm}
    \label{tab:0_hdw_specs_tb}
\end{table}

% \begin{table}[]
% \footnotesize
% \centering
% \setlength{\tabcolsep}{2.5pt}
% \begin{tabular}{lllll}\thickhline
% Device         & Type    & OS              & GPU                     & Power \\ \hline
% Iphone 12      & Phone   & iOS             & Apple GPU               & 7W    \\
% Motorola G9    & Phone   & Android 11      & Adreno 610              & 7W    \\
% Lenovo Tablet  & Tablet  & Android 9       & Adreno 505              & 10W   \\
% Dell           & Laptop  & Windows 10      & Integrated GPU          & 15W  \\
% Gaming Lap.    & Laptop  & Windows 10      & NVIDIA GF RTX 2060      & 90W   \\
% PC             & Desktop & Ubuntu 18.04    & NVIDIA GF RTX 3090      & 450W \\\thickhline
% \end{tabular}
% \caption{\textbf{Hardware specs} – of the devices used in our rendering experiments. The power is the max GPU power for discrete NVIDIA cards, and the combined max CPU and GPU power for integrated GPUs.}
% \label{tab:0_hdw_specs_tb}
% \end{table}

% \begin{table}[]
% \footnotesize
% \centering
% \setlength{\tabcolsep}{1pt}
% \begin{tabular}{llllll}\thickhline
% Device                                                        & Browser & Type    & OS                 & GPU                     & Power \\ \hline
% \begin{tabular}[c]{@{}l@{}}Motorola G9\\ Power\end{tabular}   & Chrome  & Phone   & Android 11         & Adreno 610              & 7W    \\
% \begin{tabular}[c]{@{}l@{}}Lenovo \\ Yoga Tablet\end{tabular} & Chrome  & Tablet  & Android Pie (9.0)  & Adreno 505              & 10W   \\
% Dell                                                          & Firefox & Laptop  & Windows 10 Pro     & Integrated GPU          & 15 W  \\
% \begin{tabular}[c]{@{}l@{}}Gaming \\ Laptop\end{tabular}                                                 & Firefox & Laptop  & Windows 10 Pro     & NVIDIA GeForce RTX 2060 & 90W   \\
% PC                                                            & Chrome  & Desktop & Ubuntu 18.04.6 LTS & NVIDIA GeForce RTX 3090 & 450W \\\thickhline
% \end{tabular}
% \caption{\textbf{Hardware specs} – of the devices used in our rendering experiments. The power is the max GPU power for discrete
% NVIDIA cards, and the combined max CPU and GPU power for
% integrated GPUs.}
% \label{tab:0_hdw_specs_tb}
% \end{table}
