% \begin{table}[t!]
%     \footnotesize
%     \centering
%     \setlength{\tabcolsep}{1.5pt}
%     \begin{tabular}{l"c|c|c"c|c|c}
%         \thickhline
%          & \multicolumn{3}{c"}{Synthetic 360°} & \multicolumn{3}{c}{Unbounded 360°} \\
%         Device & SNeRG & M-NeRF & \methodname & SNeRG & M-NeRF & \methodname \\
%         \thickhline
%         % iPhone 12 & N/A & 57.9 & \best{60.0}$^{\star}$ & N/A & 38.2 & \best{60.0}$^{\star}$ \\
%         Samsumg S21 & N/A & 57.9 & \best{60.0}$^{\star}$ & N/A & 38.2 & \best{60.0}$^{\star}$ \\
%         G9 & N/A & 10.6 & \best{22.8} & N/A & 3.3 & \best{16.0} \\
%         Yoga & N/A & 5.7 & \best{13.8} & N/A & 2.0 & \best{9.2} \\
%         Dell & N/A & 49.8 & \best{83.3} & N/A & 16.9 & \best{62.7} \\
%         Gaming & 192.7 & 473.40 & \best{625.1} & 42.3 & 169.5 & \best{503.5} \\
%         Desktop & 688.9 & 1,075.2 & \best{1,820.2} & 83.9 & 478.5 & \best{1,648.0} \\
%         Headset & N/A & - & \best{74.0}$^{\star}$ &  N/A & - & \best{74.0}$^{\star}$ \\
%         \thickhline
%         GPU (MB) & 5,707.3 & 538.4 & \best{239.3} & 3,388.3 & 1,063.0 & \best{397.5} \\
%         Disk (MB) & \best{87.0} & 125.75 & 143.5 & 324.8 & 311.3 & \best{233.5} \\
%         \thickhline
%         PSNR (dB) & 30.4 & \best{30.9} & 29.7 & 14.0 & 15.6 & \best{17.9} \\
%         \thickhline
%     \end{tabular}
%     % \vspace{-0.1cm}
%     \caption{\textbf{On-device Performance.} 
%     \sara{needs to update} We compare the performance of \methodname against MobileNeRF (M-NeRF)~\cite{chen2022mobilenerf} and SNeRG~\cite{hedman2021snerg} across devices. 
%     The first seven rows report rendering speed in frames-per-second~(FPS).
%     In notation, $``^{\star}"$ means the device's FPS limit was reached, $``-"$ means missing implementation, and ``N/A'' means the method failed to run.
%     % For SNeRG,  denotes the method was unable to run due to out-of-memory errors. %  generated when running, precluding rendering in the specified device. 
%     % SNeRG is unable to run on resource-constrained devices, which we denote by .
%     %Note how \methodname provides significantly larger FPS than the competitors, particularly on real scenes. 
%     The second set of rows reports GPU memory and disk space requirements.
%     Finally, we report average PSNR for each dataset.
%     Note \methodname provides the fastest rendering across scenes with comparable memory and disk requirements.
%     For synthetic scenes, our FPS gains come with limited cost to quality.
%     In unbounded scenes, our method improves \textit{both} PSNR and FPS by sizable margins. 
%     Moreover, we highlight that \methodname is capable of real-time rendering even on VR headsets. % , and achieves real-time rendering speeds. 
%     }
%     \label{tab:1_Rendering_speed_tb}
% \end{table}

\begin{table}[t!]
    \footnotesize
    \centering
    \setlength{\tabcolsep}{1.5pt}
    \begin{tabular}{l"c|c|c"c|c|c}
        \thickhline
         & \multicolumn{3}{c"}{Synthetic 360°} & \multicolumn{3}{c}{Unbounded 360°} \\
        Device & SNeRG & M-NeRF & \methodname & SNeRG & M-NeRF & \methodname \\
        \thickhline
        % iPhone 12 & N/A & 57.9 & \best{60.0}$^{\star}$ & N/A & 38.2 & \best{60.0}$^{\star}$ \\
        Samsung S21 & $\dagger$ & \textcolor{black}{41.7} & \textcolor{black}{\best{54.7}} & $\dagger$ & \textcolor{black}{22.8} & \textcolor{black}{\best{33.5}} \\
        G9 & $\dagger$ & \textcolor{black}{9.7} & \textcolor{black}{\best{10.4}} & $\dagger$ & \textcolor{black}{3.5} & \textcolor{black}{\best{7.7}} \\
        \textcolor{black}{Galaxy S6 } & N/A & \textcolor{black}{18.1} & \textcolor{black}{\best{26.6}} & $\dagger$ & \textcolor{black}{6.2} & \textcolor{black}{\best{20.4}} \\
        Dell & $\dagger$ & \textcolor{black}{49.8} & \textcolor{black}{\best{75.3}} & $\dagger$ & \textcolor{black}{16.9} & \textcolor{black}{\best{54.0}} \\
        Gaming & \textcolor{black}{197.9}& \textcolor{black}{496.0} & \textcolor{black}{\best{697.3}} & \textcolor{black}{46.0} & \textcolor{black}{169.9} & \textcolor{black}{\best{516.6}} \\
        Desktop & \textcolor{black}{502.1} & \textcolor{black}{762.3} & \textcolor{black}{\best{1,013.2}} & \textcolor{black}{141.1} & \textcolor{black}{389.8} & \textcolor{black}{\best{925.4}} \\
        Headset & $\dagger$ & $-$ & \best{74.0}$^{\star}$ &  $\dagger$ & $-$ & \best{74.0}$^{\star}$ \\
        % \thickhline
        % GPU (MB) & 5,707.3 & 538.4 & \best{239.3} & 3,388.3 & 1,063.0 & \best{397.5} \\
        % Disk (MB) & \best{87.0} & 125.75 & 143.5 & 324.8 & 311.3 & \best{233.5} \\
        % \thickhline
        % PSNR (dB) & 30.4 & \best{30.9} & 29.7 & 14.0 & 15.6 & \best{17.9} \\
        \thickhline
    \end{tabular}
    \vspace{0.07cm}
    \caption{\textbf{On-device rendering speed (FPS).} 
    % \sara{needs to update} 
    We compare the rendering speed of \methodname against MobileNeRF (M-NeRF)~\cite{chen2022mobilenerf} and SNeRG~\cite{hedman2021snerg} across devices. 
    % The first seven rows report rendering speed in frames-per-second~(FPS).
    Conventions: $``^{\star}"$ means the device's FPS limit was reached, $``-"$ means missing implementation, and ``$\dagger$" means the method failed to run.
    % For SNeRG,  denotes the method was unable to run due to out-of-memory errors. %  generated when running, precluding rendering in the specified device. 
    % SNeRG is unable to run on resource-constrained devices, which we denote by .
    %Note how \methodname provides significantly larger FPS than the competitors, particularly on real scenes. 
    % The second set of rows reports GPU memory and disk space requirements.
    % Finally, we report average PSNR for each dataset.
    % Note \methodname provides the fastest rendering across scenes with comparable memory and disk requirements.
    % For synthetic scenes, our FPS gains come with limited cost to quality.
    % In unbounded scenes, our method improves \textit{both} PSNR and FPS by sizable margins. 
    For all devices and datasets, \methodname provides the fastest rendering speeds, usually by a large margin.
    Furthermore, we highlight that \methodname is capable of real-time rendering even on VR headsets. % , and achieves real-time rendering speeds. 
    }
    \vspace{-0.6cm}
    \label{tab:1_Rendering_speed_tb}
\end{table}

