% CVPR 2023 Paper Template
% based on the CVPR template provided by Ming-Ming Cheng (https://github.com/MCG-NKU/CVPR_Template)
% modified and extended by Stefan Roth (stefan.roth@NOSPAMtu-darmstadt.de)

\documentclass[10pt,twocolumn,letterpaper]{article}

%%%%%%%%% PAPER TYPE  - PLEASE UPDATE FOR FINAL VERSION
% \usepackage[review]{cvpr}      % To produce the REVIEW version
\usepackage{cvpr}              % To produce the CAMERA-READY version
%\usepackage[pagenumbers]{cvpr} % To force page numbers, e.g. for an arXiv version

% Include other packages here, before hyperref.
\usepackage{graphicx}
\usepackage{amsmath}
\usepackage{amssymb}
\usepackage{booktabs}

%%%%%%%%% customize
\usepackage{multirow}
% \usepackage{algorithm}
\usepackage[ruled]{algorithm2e}
%%%%%%%%% customize

\usepackage[accsupp]{axessibility}

% It is strongly recommended to use hyperref, especially for the review version.
% hyperref with option pagebackref eases the reviewers' job.
% Please disable hyperref *only* if you encounter grave issues, e.g. with the
% file validation for the camera-ready version.
%
% If you comment hyperref and then uncomment it, you should delete
% ReviewTempalte.aux before re-running LaTeX.
% (Or just hit 'q' on the first LaTeX run, let it finish, and you
%  should be clear).
\usepackage[pagebackref,breaklinks,colorlinks]{hyperref}


% Support for easy cross-referencing
\usepackage[capitalize]{cleveref}
\crefname{section}{Sec.}{Secs.}
\Crefname{section}{Section}{Sections}
\Crefname{table}{Table}{Tables}
\crefname{table}{Tab.}{Tabs.}


%%%%%%%%% PAPER ID  - PLEASE UPDATE
\def\cvprPaperID{18} % *** Enter the CVPR Paper ID here
\def\confName{CVPR}
\def\confYear{2023}


\begin{document}

%%%%%%%%% TITLE - PLEASE UPDATE
\title{Are Data-driven Explanations Robust against Out-of-distribution Data?}

\author{Tang Li \qquad Fengchun Qiao \qquad Mengmeng Ma \qquad Xi Peng\\
University of Delaware\\
% Newark, DE 19716, USA\\
{\tt\small \{tangli, fengchun, mengma, xipeng\}@udel.edu}
% For a paper whose authors are all at the same institution,
% omit the following lines up until the closing ``}''.
% Additional authors and addresses can be added with ``\and'',
% just like the second author.
% To save space, use either the email address or home page, not both
}
\maketitle

%%%%%%%%% ABSTRACT
\begin{abstract}
As black-box models increasingly power high-stakes applications, a variety of data-driven explanation methods have been introduced.
Meanwhile, machine learning models are constantly challenged by distributional shifts.
A question naturally arises: Are data-driven explanations robust against out-of-distribution data?
Our empirical results show that even though predict correctly, the model might still yield unreliable explanations under distributional shifts.
How to develop robust explanations against out-of-distribution data?
To address this problem, we propose an end-to-end model-agnostic learning framework Distributionally Robust Explanations (DRE).
The key idea is, inspired by self-supervised learning, to fully utilizes the inter-distribution information to provide supervisory signals for the learning of explanations without human annotation.
Can robust explanations benefit the model’s generalization capability?
We conduct extensive experiments on a wide range of tasks and data types, including classification and regression on image and scientific tabular data.
Our results demonstrate that the proposed method significantly improves the model's performance in terms of explanation and prediction robustness against distributional shifts.
\end{abstract}

%%%%%%%%% BODY TEXT
\section{Introduction}
\label{sec:intro}
\vspace{-10pt}
%%%%%%%%% BODY TEXT

\section{Introduction}
\label{section:introduction}
%% 1. why should someone care?

%The advent of advanced interactive computer vision systems~\cite{hololens} and recent progress in vision-language and multi-modal models~\cite{} opens doors for such next generation of assistive agents. 
% We envision that the future assistive agents would build up on these visual and language reasoning capabilities of today and empower users to achieve goals in their everyday lives. In particular, such agents would be able to reason about \emph{unseen} human goals... 
% We posit that such agents would require the ability to understand user goals described in natural language at high-level i.e., without complete details about as well as unseen user goals. 

%Recent progress in augmented reality systems~\cite{hololens, magicleap}, as well as vision-language and multi-modal models~\cite{}, opens doors for the next generation of assistive agents. 
Inspired by recent progress in visual systems~\cite{MagicLeap, ungureanu2020hololens}, we consider an assistive egocentric agent capable of reasoning about daily activities. When invoked via natural language commands, for e.g., while baking a cake, the agent understands the steps involved in baking, tracks progress through the various stages of the task, detects and proactively prevents mistakes by making suggestions. Such an agent would empower users to learn new skills and accomplish tasks efficiently.
% One could envision invoking such an agent merely through natural language descriptions of tasks similar to how present day assistants such as Alexa, Siri etc.~\cite{voice_assistants} are invoked. 
%We envision such agents to empower users in daily life by  invoking them naturally through 

%% 2. Why is it challenging? 
%While recent progress in vision-language and multi-modal models~\cite{} opens doors for such next generation of assistive agents, various challenges remain in making such agents a reality. 
%To make such agents a reality, 

Developing such an egocentric agent capable of tracking and verifying everyday tasks based on their natural language specification is challenging for multiple reasons. First, such an agent must reason about various ways of doing a \emph{multi-step} task specified in natural language. This entails decomposing the task into relevant actions, state changes, object interactions as well as any necessary causal and temporal relationships between these entities. Secondly, the agent must ground these entities in egocentric observations to track progress and detect mistakes. Lastly, to truly be useful, such an agent must support tracking and verification for a combination of tasks and, ideally, even unseen tasks. These three challenges -- causal and temporal reasoning about task structure from natural language, visual grounding of sub-tasks, and compositional generalization -- form the core goals of our work.

% %% 3. What are we doing? What is our approach?
% \aks{I think this is a matter of preference, but I personally don't like related work in intro. I would make this paragraph be about EgoTV and NSG. Starting with something like - "To this end, we propose...", ie, your next paragraph.}
% \nk{+1, we should move parts of this para to lit review and delete the rest.}
% Recent research on language modeling enables decomposing tasks into multiple steps from natural language descriptions~\cite{llm_zero_shot_planning,proscript}. However, such \emph{task decompositions} cannot directly be leveraged for task tracking in egocentric agents because of lack of grounding into the visual observations or context. In parallel, the computer vision community has advanced action recognition~\cite{}, object detection and tracking~\cite{}, hand object interaction and object state change detection~\cite{ego_4d,change_it,}, step classification in procedural tasks~\cite{}, and even vision language reasoning~\cite{nsvqa,nscl,star_situated_reasoning,clevrer}, which may help with the grounding challenge. However, majority of current research on identifying actions, objects, steps, or state changes does not account for the overall task structure. Likewise, predominant research on vision language understanding~\cite{} and multi-modal grounding~\cite{} does not consider the temporal and causal constraints that emerge in task tracking and verification. We therefore focus on the order-aware visual grounding problem in our work, with an eye towards compositional generalization to scale usability of these agents. In particular, we aim to achieve visual grounding of the actions and objects corresponding to each step or sub-task obtained from the task description decomposition in an order-aware manner.

%% 4. What are our results/contributions?
As our first contribution, we propose a benchmark -- \emph{\textbf{Ego}centric \textbf{T}ask \textbf{V}erification} (\etv \inlineimg{figures/TV}) -- and a corresponding dataset in the AI2-THOR~\cite{ai2thor} simulator. % \emoji{tv}
Given a natural language (NL) task description and a corresponding egocentric video of an agent, the goal of \etv is to verify whether the task was successfully completed in the video or not.
\etv contains multi-step tasks with \emph{ordering} constraints on the steps and \emph{abstracted} NL task descriptions with omitted low-level task details inspired by the needs of real-world assistants. We also provide splits of the dataset focused on different generalization aspects, e.g., unseen visual contexts, compositions of steps, and tasks (see Figure~\ref{figure:dataset}).
% Next, we create splits of the dataset focused on different aspects of generalization, ranging from generalization to unseen visual context to unseen compositions of steps and tasks. Figure~\ref{figure:dataset} shows an example task and overview of generalization splits from \etv. Succeeding at \etv tasks requires decomposing tasks into partially-ordered steps from the NL description and order-aware visual grounding of these steps into the video. 

Our second contribution is a novel approach for order-aware visual grounding~--~\emph{\textbf{N}euro-\textbf{S}ymbolic \textbf{G}rounding} (NSG), capable of compositional reasoning and generalizing to unseen tasks owing to its ability to leverage abstract NL descriptions and compositional structure of tasks (task decomposition, ordering).~In contrast, state-of-the-art vision-language models~\cite{coca,clip,videoclip,clip_hitchiker} struggle to ground NL descriptions in egocentric videos, and do not generalize to unseen tasks.~NSG outperforms these models by~$\mathbf{33.8}\%$~on compositional generalization and~$\mathbf{32.8}\%$~on abstractly described task verification. Finally, to evaluate \nsg on real-world data, we instantiate \etv on the CrossTask~\cite{cross_task} instructional video dataset. %Specifically, we synthetically create videos with mistakes in CrossTask. 
We find that it also outperforms state-of-the-art models at task verification on CrossTask. We hope that the \etv~benchmark and dataset will enable future research on egocentric agents capable of aiding in everyday tasks.

% We experiment with many for the \etv tasks. We find that while these models generalize well to unseen visual context, they struggle to perform grounding from abstracted task descriptions and to generalize to new compositions of tasks. To deal with these challenges, we take inspiration from recent research on and develop . ~\rd{unclear why neurosymbolic models would do well on abstraction.} 

% To summarize, our main contributions are:~1)~\etv: a benchmark and synthetic dataset to systematically study egocentric task verification.
% 2)~\nsg: a novel neuro-symbolic approach to enable the core reasoning capability for \etv -- order-aware visual grounding. We demonstrate \nsg's capability on our synthetic \etv dataset as well as a real-world dataset derived from CrossTask. We will release both of these datasets and our models for future research on egocentric task tracking and verification. 


% Assistive agents require the ability to track actions and state changes from an egocentric perspective for effective assistance in day-to-day tasks. For example, an agent helping a user prepare a recipe would need to both generate the steps of the recipe (\textit{plan generation}) and track the user's actions to ensure the plan is executed correctly (\textit{plan verification}). We formulate this as a Video Entailment task~\cite{violin_dataset,9710490} \rd{should we call our task video-based goal entailment?}, wherein, given an egocentric video of an agent (or human) performing a task (\textit{premise}) and a NL task description (\textit{hypothesis}), the objective is to learn a model to track whether the given task was successfully executed in the video. 
% An ideal model should also be able to seamlessly generalize to novel compositions (of actions and objects) unseen during training. \rd{add a line about what we mean by abstraction and why is it important.} To this end, we generate a novel Vision-Language dataset on the AI2-THOR simulator~\cite{ai2thor} to study compositional and abstraction-based generalization. Our dataset provides effective evaluation measures in a controlled setting, while closely reflecting the diversity of real-world events. We implement and train a variety of end-to-end models based on existing state-of-the-art approaches. We empirically demonstrate that neural models suffer from overfitting and cannot effectively generalize to novel compositions of actions, objects, and scenes. 
% To address this problem, we propose an end-to-end Neuro-Symbolic (NeSy) framework that performs plan generation and verification. At the heart of our approach is the hypothesis that symbolic reasoning models are good at generalization and capturing compositional substructure, while neural models are good at learning representations from sensory data~\cite{10.5555/3326943.3327039,nscl,clevrer}. \rd{summarize contributions in a bulleted list.} \rd{also add a line about the main result e.g., x\% improvement as compared to end-to-end models}. 

% \rd{we also evaluate NeSy with real-world data: add briefly about CrossTask experiments.}

% % \fbox{\begin{minipage}{\linewidth}
% % \textbf{Problem Statement}

% % Given:
% % (i) Premise: Egocentric video of an agent performing a task.
% % (ii) Hypothesis: NL description of the task.

% % Learn: A model to track whether the premise entails the hypothesis. The output of the model is True if the given task is executed successfully in the video.
% % \end{minipage}}

% \textbf{Contributions:} 
% \begin{itemize}
%     \item We generate a benchmark video-language dataset to study compositional and abstraction-based generalization.
%     \item We evaluate the performance of a variety of state-of-the-art models and show that these (baseline) models cannot effectively generalize to novel compositions of actions.
%     \item We propose a novel end-to-end NeSy approach that significantly outperforms the baselines on some compositional generalization splits while performing on par with them on the rest.
%     \item We also evaluate our NeSy approach with real-world data showing similar performance improvements.
% \end{itemize}

%-------------------------------------------------------------------------

\section{Related work}
\label{sec:relate}
{\bf Explainable machine learning.}
A suite of techniques has been proposed to reveal the decision-making process of modern {\it black-box} ML models.
One direction is intrinsic to the model design and training, rendering an explanation along with its output, {\it e.g.}, attention mechanisms~\cite{vaswani2017attention} and joint training~\cite{hind2019ted, chen2019looks}.
A more popular way is to give insight into the learned associations of a model that are not readily interpretable by design, known as post-hoc methods.
Such methods usually leverage backpropagation or local approximation to offer saliency maps as explanations, {\it e.g.}, Input Gradient~\cite{simonyan2013deep}, Grad-CAM~\cite{selvaraju2017grad}, LIME~\cite{ribeiro2016should}, and SHAP~\cite{lundberg2017unified}.
Recent works have shed light on the downsides of post-hoc methods.
The gradient-based explanations ({\it e.g.}, Input Gradient) are consistent with sample-based explanations ({\it e.g.}, LIME) with comparable fidelity but have much lower computational cost~\cite{ross2017right}.
Moreover,  only the Input Gradient and Grad-CAM methods passed the sanity checks in~\cite{adebayo2018sanity}.
In our work, we incorporate gradient-based methods into optimization to calculate explanations efficiently.



%-------------------------------------------------------------------------
{\bf Out-of-distribution generalization.}
% Aiming to address the {\it distributional shifts} problems, OOD generalization tasks usually assume the training and testing data follow different distributions whose labels are available.
To generalize machine learning models from training distributions to unseen distributions, existing methods on OOD generalization can be categorized into four branches:
(1) Data augmentation. \cite{volpi2018generalizing, peng2018jointly, shankar2018generalizing, ma2022multimodal} enhance the generalization performance by increasing the diversity of data through augmentation.
(2) Distribution alignment. \cite{li2018domain, bahng2020learning, qiao2023topology} align the features across source distributions in latent space to minimize the distribution gaps.
(3) Meta learning. \cite{li2018learning, dou2019domain, peng2022out, ma2021smil} using meta-learning to facilitate fast-transferable model initialization.
(4) Invariant learning. \cite{arjovsky2019invariant, krueger2021out, peng2017reconstruction} learn invariant representations that are general and transferable to different distributions.
However, recent works~\cite{gulrajani2020search, koh2021wilds} show that the classic empirical risk minimization (ERM)~\cite{vapnik1999nature} method has comparable or even outperforms the aforementioned approaches.
We argue that this is because the existing approaches barely have constraints on explanations, the model would still recklessly absorb any correlations identified in the training data.




%-------------------------------------------------------------------------
{\bf Explanation-guided learning.}
Several recent works have attempted to incorporate explanations into model training to improve predictive performance.
\cite{rieger2020interpretations, stammer2021right} match explanations with human annotations based on domain knowledge, to alleviate the model's reliance on background pixels.
\cite{guo2019visual, wang2020self, pillai2022consistent} align explanations between spatial transformations to improve image classification and weakly-supervised segmentation.
\cite{chen2019robust, han2021explanation, cugu2022attention} synchronize the explanations of the perturbed and original samples to enhance the robustness of models.
However, acquiring ground truth explanations is prohibitively labor-intensive~\cite{wang2020self} or even impossible due to subjectivity in real-world tasks~\cite{roscher2020explainable}.
Furthermore, image transformations are insufficient to address different data types and the general {\it naturally-occurring} distributional shifts, there is no one-to-one correlation between samples from different distributions to provide supervision.
%-------------------------------------------------------------------------



% Different from the aforementioned approaches, our method introduces {\it distributional explanation consistency} constraints to regularize the model training.
% Such regularization plays an important role in narrowing the search space of explanations, {\it i.e.}, the associations that the model relies on to make predictions.
% As the result, our method would significantly alleviate the model's reliance on {\it spurious correlations}, therefore enhancing its generalization capability.
% Furthermore, our explanations are learned utilizing inter-distribution information in a self-supervised manner, no additional annotation is required.
% Probably~\cite{pillai2022consistent} is the closest to our work, since it leverages the self-supervised learning ideas, using affine transformation to generate positive pairs of explanations for contrastive learning.
% However, spatial transformation is only one kind of distributional shift, it can not be generalized to 





%------------------------------------------------------------------------
\section{Methods}
\label{sec:method}
\section{Method}
Our method, {\moniker}, extends the volume rendering equation to accurately reconstruct the geometry and appearance robust to hazy conditions.
Our key idea is to introduce a series of important biases in the network architecture along with regularizers in the loss function that together underpin physically based scattering phenomena.

\subsection{Preliminary on Neural Radiance Fields}\label{sec:nerf}
Neural Radiance Fields (NeRFs)~\cite{mildenhall2020nerf} map a 3D sample point \(\p\) into a color $\mathbf{c}$ and volume density $\sigma$.
Considering only emission from classic volume rendering~\cite{kajiya1984ray,tagliasacchi2022volume}, the expected color ${C}(\r)$ of a camera ray $\r(t)=\mathbf{o} + t\mathbf{d}$ with the near and far boundary $t_n$ and $t_f$ can be written as
\begin{gather}
	{C}(\r, \mathbf{d})=\int_{t_n}^{t_f}T(t)\sigma(\r(t))c(\r(t), \mathbf{d}) \ dt \;\textrm{with} \label{eq:nerf}\\
    T(t)=\mathrm{exp}\left( - \int_{t_n}^{t}\sigma(\r(t')) \ dt'\right),
	\label{eq:occlusion}
\end{gather}
where \(T(t)\) is the accumulated transmittance between the ray section \(t_{n}\) to \(t \).
The predicted pixel value is then compared to the ground truth $\widehat{C}(\r,\d)$ for optimization.

\subsection{3D Haze Formation}\label{sec:rte_haze}
To address the 3D dehazing problem, we propose an alternative rendering equation to the image formation model.
We start from the radiative transfer equation (RTE)~\cite{chandrasekhar2013radiative,van1999multiple}, which describes the behavior of light in a medium that absorbs, scatters and emits radiation.
Assuming, a ray \(\r\left( t \right) = \mathbf{o} + t\d\) hits a surface point at \(\r\left( t_{0} \right)\), the incident radiance at the near image plane \(t_{n}\) can be divided into three parts~\cite{pharr2016physically}:
{\small
\begin{align}
C(\r, \d) &= C_{\textrm{emission}}(\r) + C_{\textrm{surface}}(\r) + C_{\textrm{in-scattering}}(\r)\nonumber\\
C_{\textrm{emission}}(\r, \d) &=
\int_{t_{n}}^{t_{0}}\epsilon\left(\r\left( t\right),\d\right)T_{\sigma_{t}}\left( t\right)dt\nonumber\\
C_{\textrm{surface}}(\r, \d) & =C_e\left(\r\left( t_{0} \right),\d\right)T_{\sigma_{t}}\left( t_{0}\right)\nonumber\\
C_{\textrm{in-scattering}}(\r, \d) &=
\int_{t_{n}}^{t_{0}}c_{\textrm{s}}\left( \r\left( t \right), \d \right)\sigma_{s}\left(\r\left( t \right)\right)T_{\sigma_{t}}\left( t \right)dt,\nonumber
\end{align}
}
where \(\epsilon\) is the emission, \(C_{e}\) is the outgoing radiance at the surface intersection, \(c_{\textrm{s}}\left(\r\left( t \right), \d \right)\) is the in-scattered light and \(\sigma_{s}\) is the scattering coefficient.
In particular, the transmittance here is computed from the attenuation coefficient \(\sigma_{t}\), \ie,
\(T_{\sigma_{t}}\left( t\right)=\exp\left( -\int_{t_{n}}^{t}\sigma_{t}(t')dt' \right)\),
where \(\sigma_{t}=\sigma_{a} + \sigma_{s}\) including the absorption and out-scattering effect.
For common haze formation, the participating particles are considered non-luminous~\cite{narasimhan2003contrast}, therefore we can drop the emission part, which leads to
{
\small
\begin{align}
\begin{split}
C(\r,\d)= {} & C_e(\r\left( t_{0} \right),\d)T_{\sigma_{t}}\left( t_{0} \right)+\\
&\int_{t_{n}}^{t_{0}}c_{\textrm{s}}\left( \r\left( t \right), \d \right)\sigma_{s}\left(\r\left( t \right)\right)T_{\sigma_{t}}\left( t \right)dt.
\end{split}
\label{eq:RTE_Haze}
\end{align}
}

Following NeRF~\cite{mildenhall2020nerf}, we represent the surface as a continuous density field with emission \(\epsilon\left(\r\left(t\right), \d\right)\coloneqq c\left(\r\left( t \right),\d\right)\sigma\left(\r\left( t \right)\right)\).
Meanwhile, the absorption part in the attenuation \(\sigma_{t}\) can be interpreted as the surface density \(\sigma\), since the volume density $\sigma$ is equal to absorption coefficient $\sigma_{a}$ in that they both determine the probability of a photon or a ray terminating at a given location.
As a result, we can write the rendering equation as
{\small
\begin{align}\begin{split}
    C(\r,\d)=&
    \underbrace{\int_{t_{n}}^{t_{0}}c(\r(t),\d)\sigma(t)T_{\sigma+\sigma_{s}}\left( t \right)dt}_{C_{\textrm{Surface}}} +\\
    &\underbrace{\int_{t_{n}}^{t_{0}}c_{s}(\r(t))\sigma_{s}(t)T_{\sigma+\sigma_{s}}\left( t \right)dt}_{C_{\textrm{Haze}}}.
    \label{eq:3D_haze_formation}
\end{split}
\end{align}
}
\cref{eq:3D_haze_formation} formally disentangles the surface and haze, represented by \(\left\{ c, \sigma \right\}\) and \(\left\{  c_{s}, \sigma_{s} \right\}\) respectively, in a principled manner.
Once successfully optimized (see the next Section), the clear-view surfaces can be recovered using \(\left\{ c, \sigma \right\}\):
\begin{equation}
C(\r,\d)=
\int_{t_{n}}^{t_{0}}c(\r(t),\d)\sigma(t)T_{\sigma}\left( t \right)dt\label{eq:clear_view}.
\end{equation}

\begin{figure}[t!]
\centering \includegraphics[width=\linewidth]{images/architecture.pdf}
\makeatother
\caption{\textbf{\moniker{} architecture.} Given a set of hazy images, our method augments the existing NeRF pipeline (gray) with a haze module (yellow), which explicitly models the scattering phenomenon using atmospheric light and scattering coefficient. During training, we render the hazy reconstruction as a composition of surface and haze, which is compared to the input hazy images to optimize the learnable parameters (in green) jointly. During inference, we use the surface module (gray) to render clear views.}
\vspace{-0.5cm}\label{fig:architecture}
\end{figure}

\subsection{Haze-aware Neural Radiance Field}\label{sec:dehaze_nerf}
Given multiple images of a hazy scene, we aim to jointly optimize for the surface appearance and geometry, \(\left\{ c, \sigma \right\}\) as well as the haze's scattering coefficient and in-scattered light (atmospheric light),  \(\left\{c_{s}, \sigma_{s} \right\}\) based on the enhanced scattering-aware rendering equation~\cref{eq:3D_haze_formation}.
However, the effects of these variables are interdependent. In order to correctly disentangle them, our model adopts suitable architecture designs and training regularizers to capture the distinct physical properties of haze and surface.
An overview of \moniker{} is illustrated in \cref{fig:architecture}.

\paragraph{Architecture.} Now we introduce inductive biases to match the physical properties of haze and surface.
For clarity, we highlight the quantities directly modeled by neural networks in \nn{green}.

\emph{Modeling a Surface.} Recall our goal is to learn the surface appearance and geometry, \(\left\{ c, \sigma \right\}\).
Similar to previous works~\cite{mildenhall2020nerf}, we model the appearance \(\cnet\left( \p, \d \right)\) with an MLP, which takes the sample location \(\p\) and viewing direction \(\d\) as inputs.
However, in order to encourage volume density \(\sigma\) to form a well-defined solid surface, instead of directly learning the volume density, we adopt the reparameterization of the volume density using signed distance function (SDF), \(\sdf\left( \r\left( t \right) \right)\in \R\), as proposed in NeuS~\cite{wang2021neus,wang2022hfs}.
The modified surface volume density \(\sigma\left( \r\left( t \right) \right) \), referred to as opaque density, can be parameterized as \(\sdf\left( \r\left( t \right) \right)\):
\begin{equation}
\sigma\left( \r\left( t \right) \right) = s\left( \Phi_{s}\left( \sdf\left( \r\left( t \right) \right)\right) -1 \right)\nabla \sdf\left( \r\left( t \right) \right)\mathbf{d},\label{eq:hfneus-sigma}
\end{equation}
where $\Phi_{s}(x)$ is the sigmoid function $\Phi_s(x) = (1 + e^{-sx})^{-1}$, whose derivative is a bell-shaped density function centered at 0 and has a learnable standard deviation of \(\nicefrac{1}{s}\).
We derive the discrete approximate following~\cite{mildenhall2020nerf,tagliasacchi2022volume}.
It samples $n$ points $\left\{ \p_{i}=\mathbf{o}+t_n\mathbf{d}|n=1,...,N,t_n<t_{n+1} \right\}$ along the ray.
The approximate pixel color of the ray is computed based on quadrature rule~\cite{max1995optical}, yielding
\begin{align}\begin{gathered}
C_{\textrm{surface}}(\r,\d) = \sum_{n=1}^{N}\frac{\sigma^{n}}{\sigma_{t}^{n}} T_{t}^{n}\alpha_{t}^{n}\nn{c}^{n} \textrm{ with } T_{t}^{n}=\prod_{m=1}^{n-1}\left(1 - \alpha_{t}^{m}\right) \label{eq:C_surface},
\end{gathered}\end{align}
where \(\alpha_{t}\) denotes the discrete \(\alpha\)-compositional weight defined as~\cite{wang2021neus,wang2022hfs}
\begin{equation}
 \resizebox{1\hsize}{!}{
 $
    \alpha_{t}^{n}=\textsc{clamp}\left( 1-\exp\left( -\sigma_{t}^{n}\delta^{n} \right),0, 1 \right) \textrm{ with } \delta^{n}=t^{n+1}-t^{n}\label{eq:alpha}\nonumber,$}
\end{equation}
where \(\sigma_{t}^{n}=\sigma^{n}+\nn{\sigma_{s}}^{n}\) denotes the total attenuation at sample \(n\), including the attenuation due to surface occlusion and the out-scattering.

\emph{Modeling Haze.} We use a low-frequency prior to compute the scattering coefficient and atmospheric light, \(\left\{c_{s}, \sigma_{s} \right\}\), since these components usually vary slowly in a common hazy scenes~\cite{li2015simultaneous}.
In practice, we use a small band-limited \textsc{MLP}~\cite{lindell2022bacon} for the scattering coefficient \(\sigma_{s}\) to capture inhomogenous haze.
Analogous to \cref{eq:C_surface}, the haze color can be approximated as
% \begin{equation}
% \begin{gathered}
% C_{\textrm{haze}}(\r) = \sum_{i=1}^{n}\frac{\nn{\sigma_{s}}^{n}}{\sigma_{t}^{n}} T_{t}^{n}\alpha_{t}^{n}\nn{c_{s}}^{n}.\label{eq:C_haze}
% \end{gathered}
% \end{equation}
\begin{equation}
\begin{gathered}
C_{\textrm{haze}}(\r) = \sum_{n=1}^{N}\frac{\nn{\sigma_{s}}^{n}}{\sigma_{t}^{n}} T_{t}^{n}\alpha_{t}^{n}\nn{c_{s}}^{n}.\label{eq:C_haze}
\end{gathered}
\end{equation}
During optimization, the color for an arbitrary input hazy image can be written as $C = C_{\textrm{surface}} + C_{\textrm{haze}}$.
At test time, we can reconstruct the clear-view color by discretizing \cref{eq:clear_view}, namely:
\begin{gather}
 C_{\textrm{clear}}\left( \r,\d \right) = \sum_{n=1}^{N}T_{\sigma}^{n}\alpha^{n} \nn{c}^{n}, \label{eq:clear_view_discrete}\\
 \resizebox{1\hsize}{!}{
 $T_{\sigma}^{n} = \prod_{j=1}^{n-1}\left(1 - \alpha^{j}\right)\, \textrm{and }\, \alpha^{n} = \textsc{clamp}\left( 1 - \exp\left( -\sigma^{n}\delta^{n} \right),0, 1 \right).\nonumber$}
\end{gather}
\paragraph{Optimization.} While the inductive biases separate the high-frequency surface appearance and geometry from the low-frequency color and density of the scattering medium, we introduce further regularizers to guide the optimization process to converge to more plausible clear-view geometry and color.

\emph{Koschmieder Consistency.}
Given an accurate depth map \(D\), assuming globally constant scattering coefficient \(\bar{\sigma}_{s}\) and airlight \(\bar{c}_{s}\), the relation between a clear-view image \(C_{\textrm{clear}}\) and the hazy image \(C\) can be described by the Koschmieder law~\cite{israel1959koschmieders} as
\begin{equation}
\resizebox{0.88\hsize}{!}{
\(C(\r)=C_{\textrm{clear}}(\r)\exp(-\bar{\sigma}_{s} D(\r))+\bar{c}_{s}(1-\exp(-\bar{\sigma}_{s} D(\r)))\).
}\label{eq:koschmieder}
\end{equation}
This model is widely adopted as the basis for image-based single and multiview dehazing.
The Koschmider model is an approximation of our rendering equation~\cref{eq:3D_haze_formation} under the assumption of
spatially-invariant (i.e., homogeneous) scattering coefficient and an ideal surface
\begin{align}
C_{\textrm{surface}}\left( \r \right) & \approx C_{\textrm{clear}}(\r)\exp(-\bar{\sigma}_{s} D(\r)) = \tilde{C}_{\textrm{surface}}\left( \r \right)\\
C_{\textrm{haze}}\left( \r \right) & \approx \bar{c}_{s}(1-\exp(-\bar{\sigma}_{s} D(\r)) = \tilde{C}_{\textrm{haze}}\left( \r \right),
\end{align}

We promote this relation with
%
\begin{align}
&\loss_{\textrm{2D}} = \left\|C_{\textrm{surface}}\left( \r \right) -  \tilde{C}_{\textrm{surface}}\left( \r \right)\right\|_{1} \\+
&\left\| C_{\textrm{haze}}\left( \r \right)\! - \!\tilde{C}_{\textrm{haze}}\left( \r \right)\right\|_{1} \!\!+\!
 \left\| C\! -\! \tilde{C}_{\textrm{surface}}\left( \r \right)\! -\! \tilde{C}_{\textrm{haze}}\left( \r \right)\right\|_{1}\!, \nonumber
\end{align}
%
where \(\bar{\sigma}_{s}\) and \(\bar{c}_{s}\) are the average over the samples on the ray, while
the depth value \(D\left( \r \right)\) is computed via the learned surface geometry~\cite{mildenhall2020nerf,yu2022monosdf} by accumulating over ray-length over all the samples on a ray:
\begin{equation}
    D\left( \r \right) = \sum_{n=1}^{N} T_{\sigma}^{n}\alpha^{n}t^{n}.
\end{equation}

\emph{Color Prior.}
Without knowing the original image, the heavily attenuated color in the hazy image can be explained by the haze but also by a dull surface color.
In order to reconstruct plausible clear-view colors, we adopt the popular 2D prior widely used in image-based dehazing methods, Dark Channel Prior (DCP)~\cite{he2010single}, which arises from the observation, that for most pixels in a natural haze-free image, the minimum of three color channels is close to zero.
We apply this prior to the estimated clear image \(C_{\textrm{clear}}\)
\begin{align}
DC(C_{\textrm{clear}})\left(\x\right)&=\underset{\y\in\Omega\left(\x\right)}{\min}\left(\underset{c\in\left\{r,g,b\right\}}{\min}C_\textrm{clear}^{c}\left(\y\right)\right),
\label{eq:DCP_definition}\\
\loss_{\textrm{dcp}}&=\frac{1}{K}\sum\limits_{k=1}^{K}\Vert DC\left(C_{\textrm{clear}}\right)\Vert_{1}.
\label{eq:loss_dcp}
\end{align}


\subsection{Implementation Details}
We adopt the same setting as that in HF-NeuS~\cite{wang2021neus} wherever possible.
This includes the MLPs for the surface SDF, \(\sdf\) and the view-dependent surface color, \(\cnet\), as well as the sampling strategy, the background composition, and learning rate schedule.

\paragraph{Loss.}
Our loss is composed of several terms:
\begin{equation}
    \loss = \loss_{\textrm{color}} + \lambda\loss_{\textrm{eikonal}} + \alpha\loss_{\textrm{dcp}} + \beta\loss_{\textrm{2D}},\label{eq:total_loss}
\end{equation}
where \(\loss_{\textrm{dcp}}\) and \(\loss_{\textrm{2D}}\) are the regularizations introduced in \cref{sec:dehaze_nerf},
while the photo-consistency loss, $\loss_{\textrm{color}}$, is the standard NeRF loss, and the eikonal loss, \(\loss_{\textrm{eikonal}}\), is commonly used to regularize SDF~\cite{gropp2020implicit},
\begin{align}
    \loss_{\textrm{color}}& = \frac{1}{K}\sum_{k=1}^{K}\left\|\widehat{C}_{k}(\r,\d) - C_{k}(\r,\d)\right\|_{1},\\
    \loss_{\textrm{eikonal}} &= \frac{1}{KN}\sum_{k}^{K}\sum_{n}^{N}(\|\nabla f({\mathbf{r}}_{k}(t_n))\|_2 - 1)^2,
\label{eq:loss_color}
\end{align}
where $\widehat{C}_{k}(\r,\d)$ is the pixel color. $N$ and $K$ denote the total sampling points on a ray and the total number of rays sampled per training batch.

Finally, because of the surface representation using SDF, we can optionally adopt the object masks for supervision~\cite{yariv2021volume,wang2021neus,wang2022hfs}.
Specifically, given the object mask, \(M\), the mask loss $\loss_{\textrm{mask}}$ for a sampled ray $k$ is defined as
\begin{equation}
    \loss_{\textrm{mask}} = \text{BCE}(M_k, \hat{O}_k),\label{eq:mask_loss}
\end{equation}
where $\hat{O}_k = \sum_{i=1}^{N}T_{\sigma}^{i}\alpha^{i}$ is the total weight for the clear-view surface color along the camera ray, and $\text{BCE}$ is the binary cross entropy loss.



%------------------------------------------------------------------------
\section{Experiments}
\label{sec:exp}
\begin{table*}
\begin{center}
\caption{Comparison with \sota\ methods on the public crowd analysis benchmarks: \jhu, ShanghaiTech, UCF, and \nwpu. 
The best results are shown in \first{red}. The second-best results are shown in \second{blue}. 
}
\vspace{\tablegap}
\resizebox{0.95\textwidth}{!}{
\begin{tabular}{l c c c c c c c c c c c c c}
\toprule
 \multirow{2}{*}{Method} & \multirow{2}{*}{Venue} &\multicolumn{2}{c}{\jhu} &\multicolumn{2}{c}{\shha} &\multicolumn{2}{c}{\shhb} &\multicolumn{2}{c}{\ucf} &\multicolumn{2}{c}{\qnrf} &\multicolumn{2}{c}{\nwpu}\\[0.2ex]
 \cmidrule(lr){3-4}\cmidrule(lr){5-6}\cmidrule(lr){7-8}\cmidrule(lr){9-10}\cmidrule(lr){11-12}\cmidrule(lr){13-14}
& & MAE$\downarrow$ & MSE$\downarrow$ & MAE$\downarrow$ & MSE$\downarrow$ & MAE$\downarrow$ & MSE$\downarrow$ & MAE$\downarrow$ & MSE$\downarrow$ & MAE$\downarrow$ & MSE$\downarrow$ & MAE$\downarrow$ & MSE$\downarrow$\\[0.2ex]
\midrule\midrule
TopoCount \cite{abousamra2021localization}	& AAAI'21	& {60.9}	& {267.4}	& {61.2}	& {104.6}	& {7.8}	& {13.7}	& {184.1}	& {258.3}	& {89.0}	& {159.0}	& {107.8}	& {438.5}	\\[0.2ex]
SUA \cite{meng2021spatial}	& ICCV'21	& {80.7}	& {290.8}	& {68.5}	& {121.9}	& {14.1}	& {20.6}	& {-}	& {-}	& {130.3}	& {226.3}	& {111.7}	& {443.2}	\\[0.2ex]
ChfL \cite{shu2022crowd}	& CVPR'22	& {57.0}	& {235.7}	& {57.5}	& {94.3}	& {6.9}	& {11.0}	& {-}	& {-}	& {80.3}	& {137.6}	& {76.8}	& {343.0}	\\[0.2ex]
MAN \cite{lin2022boosting}	& CVPR'22	& {53.4}	& \second{209.9}	& {56.8}	& {90.3}	& {-}	& {-}	& {-}	& {-}	& {77.3}	& {131.5}	& {76.5}	& {323.0}	\\[0.2ex]
GauNet \cite{cheng2022rethinking}	& CVPR'22	& {58.2}	& {245.1}	& {54.8}	& {89.1}	& {6.2}	& {9.9}	& {186.3}	& {256.5}	& {81.6}	& {153.7}	& {-}	& {-}	\\[0.2ex]
CLTR \cite{liang2022end}	& ECCV'22	& {59.5}	& {240.6}	& {56.9}	& {95.2}	& {6.5}	& {10.6}	& {-}	& {-}	& {85.8}	& {141.3}	& {74.3}	& {333.8}	\\[0.2ex]
CrwodHat \cite{wu2023boosting}	& CVPR'23	& \second{52.3}	& {211.8}	& {51.2}	& {81.9}	& \first{5.7}	& {9.4}	& {-}	& {-}	& {75.1}	& \second{126.7}	& {68.7}	& \second{296.9}	\\[0.2ex]
STEERER \cite{han2023steerer}	& ICCV'23	& {54.3}	& {238.3}	& {54.5}	& {86.9}	& {5.8}	& \second{8.5}	& {-}	& {-}	& {74.3}	& {128.3}	& \second{63.7}	& {309.8}	\\[0.2ex]
PET \cite{liu2023point}	& ICCV'23	& {58.5}	& {238.0}	& \second{49.3}	& \second{78.8}	& {6.2}	& {9.7}	& {-}	& {-}	& {79.5}	& {144.3}	& {74.4}	& {328.5}	\\[0.2ex]
\rowcolor{black!10}\method\	& 	& \first{47.3}	& \first{198.9}	& \first{47.4}	& \first{75.0}	& \first{5.7}	& \first{8.2}	& \first{160.8}	& \first{225.0}	& \first{68.9}	& \first{125.6}	& \first{57.8}	& \first{221.2}	\\[0.2ex]
\bottomrule
\end{tabular}
}
\vspace{\tablegap}
\label{table: crowd counting performance}
\end{center}
\end{table*}
\begin{table*}[!t]
    \begin{center}
    
    \resizebox{\textwidth}{!}{
    \begin{tabular}{l|cc|ccccc|c}
\toprule

Model & VLM & Additional Backbone & General & Earth Monit. & Medical Sciences & Engineering & Agri. and Biology & Mean \\
\midrule\midrule
\textit{Random (LB)} & - & - & \phantom{0}\textit{1.17} & \phantom{0}\textit{7.11} & \textit{29.51} & \textit{11.71} & \phantom{0}\textit{6.14} & \textit{10.27} \\
\textit{Best supervised (UB)} & - & - & \textit{48.62} & \textit{79.12} & \textit{89.49} & \textit{67.66} & \textit{81.94} & \textit{70.99} \\
\midrule
ZSSeg~\citep{xu2022simple} & CLIP ViT-B/16 & ResNet-101 & 19.98 & 17.98 & \underline{41.82} & 14.0\phantom{0} & 22.32 & 22.73 \\
ZegFormer~\citep{ding2022decoupling} & CLIP ViT-B/16 & ResNet-101 & 13.57 & 17.25 & 17.47 & 17.92 & \underline{25.78} & 17.57 \\
X-Decoder~\citep{zou2023generalized} & UniCL-T & Focal-T & 22.01 & 18.92 & 23.28 & 15.31 & 18.17 & 19.8\phantom{0} \\
OpenSeeD~\citep{zhang2023simple} & UniCL-B & Swin-T & 22.49 & 25.11 & \textbf{44.44} & 16.5\phantom{0} & 10.35 & 24.33 \\
SAN~\citep{xu2023side} & CLIP ViT-B/16 & - & \underline{29.35} & \underline{30.64} & 29.85 & \textbf{23.58} & 15.07 & \underline{26.74} \\

\hlrow & & & \textbf{38.69} & \textbf{35.91} & 28.09 & \underline{20.34} & \textbf{32.57} & \textbf{31.96} \\
\hlrow\multirow{-2}{*}{\ours (ours)} & \multirow{-2}{*}{CLIP ViT-B/16} & \multirow{-2}{*}{-} & \textcolor{ForestGreen}{(+9.34)} & \textcolor{ForestGreen}{(+5.27)} & \color{gray}{(-16.35)} & \color{gray}{(-3.24)} & \textcolor{ForestGreen}{(+6.79)} & \textcolor{ForestGreen}{(+5.22)} \\
\midrule
OVSeg~\citep{liang2022open} & CLIP ViT-L/14 & Swin-B & 29.54 & 29.04 & \textbf{31.9\phantom{0}} & 14.16 & \underline{28.64} & 26.94 \\
SAN~\citep{xu2023side} & CLIP ViT-L/14 & - & \underline{36.18} & \underline{38.83} & \underline{30.27} & \underline{16.95} & 20.41 & \underline{30.06} \\
\hlrow & & & \textbf{44.69} & \textbf{39.99} & 24.70 & \textbf{20.20} & \textbf{38.61} & \textbf{34.70} \\
\hlrow\multirow{-2}{*}{\ours (ours)} & \multirow{-2}{*}{CLIP ViT-L/14} & \multirow{-2}{*}{-} & \textcolor{ForestGreen}{(+8.51)} & \textcolor{ForestGreen}{(+1.16)} & \color{gray}{(-7.2)} & \textcolor{ForestGreen}{(+3.25)} & \textcolor{ForestGreen}{(+9.97)} & \textcolor{ForestGreen}{(+4.64)} \\
        \bottomrule
    \end{tabular}
    }

    \vspace{-5pt}        
    \caption{\textbf{Quantitative evaluation on MESS~\citep{blumenstiel2023mess}.} MESS includes a wide range of domain-specific datasets, which pose significant challenges due to their substantial domain differences from the training dataset. We report the average score for each domain. Please refer to the supplementary material for the results of all 22 datasets. \textit{Random} is the result of uniform distributed prediction which represents the lower-bound, while \textit{Best supervised} represents the upper-bound performance for the datasets.}
    \label{tab:mess}
    \vspace{-20pt}
    \end{center}
\end{table*}


\section{Experiments}
\subsection{Datasets and Evaluation}
We train our model on the COCO-Stuff~\cite{caesar2018coco}, which has 118k densely annotated training images with 171 categories, following \cite{liang2022open}. We employ the mean Intersection-over-Union (mIoU) as the evaluation metric for all experiments. For the evaluation, we conducted experiments on two different sets of datasets~\cite{zhou2019semantic,everingham2009pascal,mottaghi2014role}: a commonly used in-domain datasets~\cite{ghiasi2022scaling}, and a multi-domain evaluation set~\cite{blumenstiel2023mess} containing domain-specific images and class labels. 

\vspace{-10pt}
\paragraph{Datasets for standard benchmarks.} 
For in-domain evaluation, we evaluate our model on ADE20K~\cite{zhou2019semantic}, PASCAL VOC~\cite{everingham2009pascal}, and PASCAL-Context~\cite{mottaghi2014role} datasets. ADE20K has 20k training and 2k validation images, with two sets of categories: A-150 with 150 frequent classes and A-847 with 847 classes~\cite{ding2022decoupling}. PASCAL-Context contains 5k training and validation images, with 459 classes in the full version (PC-459) and the most frequent 59 classes in the PC-59 version. PASCAL VOC has 20 object classes and a background class, with 1.5k training and validation images. We report PAS-20 using 20 object classes. We also report the score for PAS-$20^b$, which defines the ``background" as classes present in PC-59 but not in PAS-20, as in \citet{ghiasi2022scaling}.

\vspace{-10pt}
\paragraph{Datasets for multi-domain evaluation.}
We conducted a multi-domain evaluation on the MESS benchmark~\cite{blumenstiel2023mess}, specifically designed to stress-test the real-world applicability of open-vocabulary models with 22 datasets. The benchmark includes a wide range of domain-specific datasets from fields such as earth monitoring, medical sciences, engineering, agriculture, and biology. Additionally, the benchmark contains a diverse set of general domains, encompassing driving scenes, maritime scenes, paintings, and body parts. We report the average scores for each domain in the main text for brevity. For the complete results and details of the 22 datasets, please refer to the supplementary material.

\subsection{Implementation Details}
We train the CLIP image encoder and the cost aggregation module with per-pixel binary cross-entropy loss. We set $d_F=128$, $N_B=2$, $N_U=2$ for all of our models. We implement our work using PyTorch~\cite{paszke2019pytorch} and Detectron2~\cite{wu2019detectron2}. AdamW~\cite{loshchilov2017decoupled} 
optimizer is used with a learning rate of $2\cdot10^{-4}$ for our model  and $2\cdot10^{-6}$ for the CLIP, with weight decay set to $10^{-4}$. The batch size is set to 4. We use 4 NVIDIA RTX 3090 GPUs for training. All of the models are trained for 80k iterations. 

\subsection{Main Results}
\paragraph{Results of standard benchmarks.}
The evaluation of standard open-vocabulary semantic segmentation benchmarks is shown in Table~\ref{tab:main_table}. Overall, our method significantly outperforms all competing methods, including those~\cite{ghiasi2022scaling,liang2022open} that leverage additional datasets~\cite{chen2015microsoft,pont2020connecting} for further performance improvements. To ensure a fair comparison, we categorize the models based on the scale of the vision-language models (VLMs) they employ. First, we present results for models that use VLMs of comparable scale to ViT-B/16~\cite{dosovitskiy2020image}, and our model surpasses all previous methods, even achieving performance that matches or surpasses those using the ViT-L/14 model as their VLM~\cite{xu2023side}.
For models employing the ViT-L/14 model as their VLM, our model demonstrates remarkable results, achieving a 16.0 mIoU in the challenging A-847 dataset and a 23.8 mIoU in PC-459. These results represent a 29\% and 52\% increase, respectively, compared to the previous state-of-the-art.
We also present qualitative results of PASCAL-Context with 459 categories in Fig.~\ref{fig:qualitative}, demonstrating the efficacy of our proposed approach in comparison to the current state-of-the-art methods~\cite{ding2022decoupling, xu2022simple,liang2022open}. 


\begin{figure*}[t]
  \centering
    \subfloat[SAN]
{\includegraphics[width=0.1595\linewidth]{figures/fig4/pc_san.pdf}}\hfill
    \subfloat[\textbf{Ours}]
{\includegraphics[width=0.1595\linewidth]{figures/fig4/pc_ours.pdf}}\hfill
     \subfloat[GT]
 {\includegraphics[width=0.1595\linewidth]{figures/fig4/pc_gt.pdf}}\hfill
    \subfloat[SAN]
{\includegraphics[width=0.166\linewidth]{figures/fig4/mess_san.pdf}}\hfill
    \subfloat[\textbf{Ours}]
{\includegraphics[width=0.166\linewidth]{figures/fig4/mess_ours.pdf}}\hfill
    \subfloat[GT]
{\includegraphics[width=0.166\linewidth]{figures/fig4/mess_gt.pdf}}\hfill
\\
\vspace{-10pt}
\caption{\textbf{Qualitative comparison to SAN~\citep{xu2023side}.} We visualize the results of PC-459 dataset in (a-c). For (d-f), we visualize the results from the MESS benchmark~\citep{blumenstiel2023mess} across three domains: underwater (top), human parts (middle), and agriculture (bottom).} 
\label{fig:qualitative}
\vspace{-10pt}
\end{figure*}

\begin{table}[t]
    \centering
    \resizebox{0.48\textwidth}{!}{%
    \begin{tabular}{cl|cccccc}
    \toprule
        & Methods & A-847 & PC-459 & A-150 & PC-59 & PAS-20 & $\textnormal{PAS-20}^b$
        \\
        \midrule\midrule
        \textbf{(I)} & Feature agg. + Freeze & 3.1 & 8.7 & 16.6 & 46.8 & 92.3 & 69.7\\
        \textbf{(II)} & Feature agg. + F.T. & 5.6 & {12.8} & {23.6} & \underline{58.1} & \underline{96.3} & \underline{77.7}\\
        \midrule
        \textbf{(III)} & Cost agg. + Freeze& \underline{10.0} & \underline{14.5} & \underline{26.0} & 46.9 & 94.2 & 65.1\\
        \hlrow\textbf{(IV)} & Cost agg. + F.T. & \textbf{14.7} & \textbf{23.2} & \textbf{35.3} & \textbf{60.3} & \textbf{96.7} & \textbf{78.9}\\
        \bottomrule
    \end{tabular}%
    }
    \vspace{-5pt}
    \caption{\textbf{Quantitative comparison between feature and cost aggregation.} Cost aggregation acts as an effective alternative to direct fine-tuning of CLIP image encoder. \textit{F.T.: Fine-Tuning.}
    }
    \label{tab:feature-vs-cost}
    \vspace{-15pt}

\end{table}



\vspace{-10pt}
\paragraph{Results of multi-domain evaluation.}
In Table~\ref{tab:mess}, we present the qualitative results obtained from the MESS benchmark~\cite{blumenstiel2023mess}. This benchmark assesses the real-world performance of a model across a wide range of domains. Notably, our model demonstrates a significant performance boost over other models, achieving the highest mean score. It particularly excels in the general domain as well as in agriculture and biology, showing its strong generalization ability. However, in the domains of medical sciences and engineering, the results exhibit inconsistencies with respect to the size of the VLM. Additionally, the scores for medical sciences are comparable to random predictions. We speculate that CLIP may have limited knowledge in these particular domains~\cite{radford2021learning}.


\subsection{Analysis and Ablation Study}\label{sec:ablation}

\paragraph{Comparison between feature and cost aggregation.} We provide quantitative and qualitative comparison of two aggregation baselines, feature aggregation, and cost aggregation, in Table~\ref{tab:feature-vs-cost}. For both of baseline architectures, we simply apply the upsampling decoder and note that both methods share most of the architecture, but differ in whether they aggregate the concatenated features or aggregate the cosine similarity between image and text embeddings of CLIP.
\begin{table}[t]
    \centering
    \resizebox{0.48\textwidth}{!}{%
    \begin{tabular}{cl|cccccc}
    \toprule
        & Methods & A-847 & PC-459 & A-150 & PC-59 & PAS-20 & $\textnormal{PAS-20}^b$
        \\
        \midrule\midrule
        \textbf{(I)} & Feature agg. + Freeze & 3.1 & 8.7 & 16.6 & 46.8 & 92.3 & 69.7\\
        \textbf{(II)} & Feature agg. + F.T. & 5.6 & {12.8} & {23.6} & \underline{58.1} & \underline{96.3} & \underline{77.7}\\
        \midrule
        \textbf{(III)} & Cost agg. + Freeze& \underline{10.0} & \underline{14.5} & \underline{26.0} & 46.9 & 94.2 & 65.1\\
        \hlrow\textbf{(IV)} & Cost agg. + F.T. & \textbf{14.7} & \textbf{23.2} & \textbf{35.3} & \textbf{60.3} & \textbf{96.7} & \textbf{78.9}\\
        \bottomrule
    \end{tabular}%
    }
    \vspace{-5pt}
    \caption{\textbf{Quantitative comparison between feature and cost aggregation.} Cost aggregation acts as an effective alternative to direct fine-tuning of CLIP image encoder. \textit{F.T.: Fine-Tuning.}
    }
    \label{tab:feature-vs-cost}
    \vspace{-15pt}

\end{table}


For \textbf{(I)} and \textbf{(III)}, we freeze the encoders of CLIP and only optimize the upsampling decoder. Subsequently, in \textbf{(II)} and \textbf{(IV)}, we fine-tune the encoders of CLIP on top of \textbf{(I)} and \textbf{(III)}. Our results show that feature aggregation can benefit from fine-tuning, but the gain is only marginal. On the other hand, cost aggregation benefits significantly from fine-tuning, highlighting the effectiveness of cost aggregation for adapting CLIP to the task of segmentation. 

For the qualitative results in Fig.~\ref{fig:feature_cost}, we show the prediction results from \textbf{(II)} and \textbf{(IV)}. As seen in Fig.~\ref{fig:feature_cost}(c-d), we observe that feature aggregation shows overfitting to the seen class of ``bucket," while cost aggregation successfully identifies the unseen class ``birdcage." 

\begin{table}[t!]
\centering
\resizebox{\linewidth}{!}{
\begin{tabular}{ll|cccccc}
        \toprule
        &Components & A-847 & PC-459 & A-150 & PC-59 & PAS-20 & $\textnormal{PAS-20}^b$
         \\
        \midrule\midrule
        \textbf{(I)} & Feature Agg. & 5.6 & 12.8 & 23.6 & 58.1 & 96.3 & 77.7\\
        \midrule
        \textbf{(II)} & Cost Agg.  & 14.7& \underline{23.2}& 35.3& 60.3& \underline{96.7}&78.9\\
        \textbf{(III)} &\textbf{(II)} + Spatial agg.  & 14.9& 23.1& 35.9& 60.3& \underline{96.7}&79.5\\
        \textbf{(IV)} &\textbf{(II)} + Class agg.  & 14.7& 21.5& 36.6& 60.6& 95.5&80.5\\
        \textbf{(V)} &\textbf{(II)} + Spatial and Class agg. & \underline{15.5}& \underline{23.2}& \underline{37.0}& \underline{62.3}& \underline{96.7}&\underline{81.3}\\
        \hlrow\textbf{(VI)} &\textbf{(V)} + Embedding guidance  & \textbf{16.0} & \textbf{23.8}& \textbf{37.9}& \textbf{63.3}& \textbf{97.0}&\textbf{82.5}\\
        \bottomrule
\end{tabular}}
\vspace{-5pt}
\caption{\textbf{Ablation study for \ours.} We conduct ablation study by gradually adding components to the cost aggregation baseline.}
    \vspace{-10pt}
    \label{tab:ablation}
\end{table}


\begin{table}[h!]\scriptsize	
    \centering
    \begin{tabular}{c|c|c|c|c|c}
        \multicolumn{2}{c|}{} & Baseline & w/o normals & w/o viscosity & w/o coarea \\ \hline
        \multirow{4}{*}{Anchor}
            & $d_C$ & \textbf{0.21} & 0.61 & 0.55 & 0.72 \\
            & $d_H$ & \textbf{3.00} & 7.82 & 10.83 & 10.24 \\
            & $d_C^\too$ & 0.15 & 0.37 & 0.27 & 0.36 \\
            & $d_H^\too$ & 1.07 & 7.84 & 1.44 & 9.68 \\ \hline
        \multirow{4}{*}{Daratech}
            & $d_C$ & 0.26 & 0.24 & 0.24 & \textbf{0.23} \\
            & $d_H$ & 4.06 & 4.2 & 4.3 & \textbf{2.19} \\
            & $d_C^\too$ & 0.14 & 0.13 & 0.12 & 0.13 \\
            & $d_H^\too$ & 1.76 & 2.69 & 1.77 & 1.77 \\ \hline
        \multirow{4}{*}{DC}
            & $d_C$ & \textbf{0.15} & \textbf{0.15} & \textbf{0.15} & 0.34 \\
            & $d_H$ & \textbf{2.22} & 2.24 & 2.24 & 6.58 \\
            & $d_C^\too$ & 0.09 & 0.08 & 0.08 & 0.16 \\
            & $d_H^\too$ & 2.76 & 2.76 & 2.79 & 2.82 \\ \hline
        \multirow{4}{*}{Gargoyle}
            & $d_C$ & \textbf{0.17} & 0.58 & 0.47 & 0.59 \\
            & $d_H$ & \textbf{4.40} & 6.32 & 10.38 & 6.35 \\
            & $d_C^\too$ & 0.11 & 0.07 & 0.26 & 0.38 \\
            & $d_H^\too$ & 0.96 & 2.39 & 1.34 & 1.25 \\ \hline
        \multirow{4}{*}{Lord Quas}
            & $d_C$ & \textbf{0.12} & 0.12 & 0.12 & 0.58 \\
            & $d_H$ & 1.06 & 1.38 & \textbf{1.04} & 6.05 \\
            & $d_C^\too$ & 0.07 & 0.37 & 0.06 & 0.32 \\
            & $d_H^\too$ & 0.64 & 0.69 & 0.64 & 3.73 \\ \hline %
            
    \end{tabular} \vspace{5pt}
    \caption{Ablations study. We show the contribution of each component of VisCo Grids. Baseline is the full method. The remaining columns correspond to optimizing without normal loss, viscosity loss and coarea loss, respectively. We show results for each mesh of the benchmark \cite{williams2019deep}. The results justify the use of the different components in VisCo Grids.}
    \label{tab:ablations}
\end{table}
\vspace{-10pt}
\paragraph{Component analysis.}
Table~\ref{tab:ablation} shows the effectiveness of the main components within our architecture through quantitative results. 
First, we introduce the baseline models in \textbf{(I)} and \textbf{(II)}, identical to the fine-tuned baseline models from Table~\ref{tab:feature-vs-cost}.
We first add the proposed spatial and class aggregations to the cost aggregation baseline in \textbf{(III)} and \textbf{(IV)}, respectively. In \textbf{(V)}, we interleave the spatial and class aggregations. Lastly, we add the proposed embedding guidance to \textbf{(V)}, which becomes our final model.

As shown, we stress the gap between \textbf{(I)} and  \textbf{(II)}, which supports the findings presented in Fig.~\ref{fig:feature_cost}. Given that PAS-20 shares most of its classes with the training datasets\cite{xu2022simple}, the performance gap between \textbf{(I)} and \textbf{(II)} is minor. However, for challenging datasets such as A-847 or PC-459, the difference is notably significant, validating our cost aggregation framework for its generalizability.
We also highlight that as we incorporate the proposed spatial and class aggregation techniques, our approach \textbf{(V)} outperforms \textbf{(II)}, demonstrating the effectiveness of our design.
Finally, \textbf{(VI)} shows that our embedding guidance further improves performance across all the benchmarks.
Furthermore, we provide quantitative results of adopting the upsampling decoder in Table ~\ref{tab:conv-decoder}. The results show consistent improvements across all the benchmarks.


\begin{table}[!t]
\centering
\resizebox{\linewidth}{!}{
   \begin{tabular}{ll|cccccc|cc}
        \toprule
        &\multirow{2}{*}{Methods} & \multirow{2}{*}{A-847} & \multirow{2}{*}{PC-459} & \multirow{2}{*}{A-150} & \multirow{2}{*}{PC-59}& \multirow{2}{*}{PAS-20} & \multirow{2}{*}{$\textnormal{PAS-20}^b$} &\#param.  & Memory
         \\
         &&&&&&&&(M)&(GiB)
         \\
        \midrule\midrule
        \textbf{(I)} &Freeze & 10.4& 15.0& 31.8& 52.5& 92.2& 71.3& 5.8 & 20.0\\
        \textbf{(II)} &Prompt  & 8.8& 14.3 & 30.5& 55.8 & 93.2 & 74.7 & 7.0 & 20.9\\
        \textbf{(III)} &Full F.T.  & 13.6& 22.2& 34.0& 61.1& \textbf{97.3}& 79.7 & 393.2 & 26.8\\
        \textbf{(IV)} &Attn. F.T. & 15.7& \underline{23.7}& 37.1& \underline{63.1}& \underline{97.1}& 81.5 & 134.9 & 20.9\\
        \textbf{(V)} &QK F.T. & 15.3& 23.0& 36.3& 62.0& 95.9& 81.9 & 70.3 & 20.9\\
        \textbf{(VI)} &KV F.T. & \textbf{16.1}& \textbf{23.8}& \underline{37.6}& 62.4& 96.7& \underline{82.0} & 70.3 & 20.9\\
        \midrule
        \textbf{(VII)} & QV F.T. (Img.)  & 13.9& 22.8& 35.1& 62.0& 96.3& \underline{82.0} & 56.7 & 20.9\\
        \textbf{(VIII)} & QV F.T. (Txt.)  & 14.7& 22.2& 35.1& 60.0& 95.8& 80.3 & 19.9 & 20.0\\
        \hlrow \textbf{(IX)} & QV F.T. (Both) & \underline{16.0}& \textbf{23.8}& \textbf{37.9}& \textbf{63.3}& 97.0& \textbf{82.5} & 70.3 & 20.9\\
        \bottomrule       
\end{tabular}
}
    \vspace{-5pt}
\caption{\textbf{Analysis of fine-tuning methods for CLIP.} We additionally note the number of learnable parameters of CLIP and memory consumption during training. Our method not only outperforms full fine-tuning, but also requires smaller computation.}
\label{tab:finetuning-ablation}
\vspace{-10pt}
\end{table}

\begin{figure}[t]
  \centering
    \subfloat[CLIP]
{\includegraphics[width=0.4999\linewidth]{figures/fig_embedding/tsne_clip_final.pdf}}\hfill
     \subfloat[Fine-tuned CLIP]
 {\includegraphics[width=0.4999\linewidth]{figures/fig_embedding/tsne_clip_2_v3.pdf}}\hfill\\
        
\vspace{-5pt}
\caption{\textbf{Effects of fine-tuning CLIP.} We show the t-SNE~\cite{van2008visualizing} 
visualization of CLIP image embeddings based on its predictions. In contrast to (a), we observe well-grouped clusters in (b), showing the adaptation of CLIP to segmentation for both seen and unseen classes.} 
\label{fig:embedding_space}
\vspace{-10pt}
\end{figure}


\vspace{-10pt}
\label{finetune}
\paragraph{Analysis on fine-tuning of CLIP.}
In this section, we analyze the effects and methods of fine-tuning of the encoders of CLIP. In Table~\ref{tab:finetuning-ablation}, we report the results of different approaches, which include the variant \textbf{(I)}:~without fine-tuning, \textbf{(II)}:~adopting Prompt Tuning~\cite{zhou2022learning, jia2022visual}, \textbf{(III)}:~fine-tuning the entire CLIP, \textbf{(IV)}:~fine-tuning the attention layer only~\cite{touvron2022three}, \textbf{(V)}:~fine-tuning query and key projections only, \textbf{(VI)}:~fine-tuning key and value projections only, \textbf{(VII)}:~our approach for CLIP image encoder only, \textbf{(VIII)}:~our approach for text encoder only, and  \textbf{(IX)}:~our approach for both encoders. Note that both image and text encoders are fine-tuned in \textbf{(I-VI)}. Overall, we observed that fine-tuning enhances the performance of our framework. Among the various fine-tuning methods, fine-tuning only the query and value projection yields the best performance improvement while also demonstrating high efficiency. Additionally, as can be seen in \textbf{(VII-IX)}, fine-tuning both encoders leads to better performance compared to fine-tuning only one of them in our framework.

In Fig.~\ref{fig:embedding_space}, we show the t-SNE~\cite{van2008visualizing} visualization of the dense image embeddings of CLIP within the A-150~\cite{zhou2019semantic} dataset. We color the embeddings based on the prediction with text classes. From (a), we can observe that the clusters are not well-formed for each classes, due to the image-level training of CLIP. In contrast, we observe well-formed clusters in (b) for both seen and unseen classes, showing the adaptation of CLIP for the downstream task.

\vspace{-10pt}
\paragraph{Training with various datasets.}
In this experiment, we further examine the generalization power of our method in comparison to other methods~\cite{ding2022decoupling, xu2022simple} by training our model on smaller-scale datasets, which include A-150 and PC-59, that poses additional challenges to achieve good performance.  The results are shown in Table~\ref{tab:cross-dataset-ablation}. As shown, we find that although we observe some performance drops, which seem quite natural when a smaller dataset is used, our work significantly outperforms other competitors. These results highlight the strong generalization power of our framework, a favorable characteristic that suggests the practicality of our approach.

\begin{table}[!t]
    \centering
    
    \resizebox{\linewidth}{!}{
    \begin{tabular}{l|c|ccccccc}
    \toprule
        Methods & Training dataset & A-847 & PC-459 & A-150 & PC-59 & PAS-20 & $\textnormal{PAS-20}^b$
        \\
        \midrule\midrule
        ZegFormer & COCO-Stuff & 5.6 & \underline{10.4} & 18.0 & 45.5 & \underline{89.5} & 65.5\\
        ZSseg & COCO-Stuff & \underline{7.0} & 9.0 & \underline{20.5} & \underline{47.7} & 88.4 & \underline{67.9}\\
        \hlrow \ours (ours) & COCO-Stuff & \textbf{12.0} & \textbf{19.0} & \textbf{31.8} & \textbf{57.5} & \textbf{94.6} & \textbf{77.3}\\
        \midrule
        ZegFormer & A-150 & 6.8 & \underline{7.1} & \color{gray}{33.1} & 34.7 & 77.2 & 53.6 \\
        ZSseg & A-150 & \underline{7.6} & \underline{7.1} & \color{gray}{40.3} & \underline{39.7} & \underline{80.9} & \underline{61.1}\\
        \hlrow \ours (ours) & A-150 & \textbf{14.4} & \textbf{16.2} & \color{gray}{47.7} & \textbf{49.9} & \textbf{91.1} & \textbf{73.4} \\
        \midrule
        ZegFormer & PC-59 & \underline{3.8} & \underline{8.2} & \underline{13.1} & \color{gray}{48.7} & 86.5 & 66.8 \\
        ZSseg & PC-59 & 3.0 & 7.6 & 11.9 & \color{gray}{54.7} & \underline{87.7} & \underline{71.7}\\
        \hlrow \ours (ours) & PC-59 & \textbf{9.6} & \textbf{16.7} & \textbf{27.4} & \color{gray}{63.7} & \textbf{93.5} & \textbf{79.9} \\
        \bottomrule
    \end{tabular}
    }

    
    \vspace{-5pt}
    \caption{\textbf{Training on various datasets.} CLIP with ViT-B is used for all methods. Our model demonstrates remarkable generalization capabilities even on relatively smaller datasets. The scores evaluated on the same dataset used for training are colored in \textcolor{gray}{gray}.}
    \vspace{-10pt}
    \label{tab:cross-dataset-ablation}
\end{table}

\begin{table}[t!]
    \centering
    \caption{
    \textbf{Efficiency comparison of optimization algorithms.}
    R@1 scores evaluated on MSRVTT-7k for video retrieval are recorded.
    Multi-task learning simultaneously trains all tasks with even loss weights. 
    CG and FP are abbreviations of conjugate gradient and fixed-point optimization. 
    In terms of time costs, average training time per epoch is reported. 
    $^\dagger$ refers to our optimization algorithm which approximates $\nabla^2_w \aux$ as the identity matrix $\mathrm{I}$.}
    \begin{adjustbox}{width=\linewidth}
    \begin{tabular}{l |c| c  c}
        \toprule
        \textbf{Method}  & \textbf{Opt. Scheme}  & \textbf{R@1} &  \textbf{Time} \\
        \midrule
        \midrule
        Multi-task Learning   & 
        - &  
        26.1 \scriptsize(+0.0)    & 
        547 \scriptsize(+0.0\%) \\
        
        \textbf{MELTR} + Meta-Weight Net~\cite{shu2019meta}  & 
        ITD &  
        27.3 \scriptsize(\textcolor{red}{+1.2})  & 
        1,296 \scriptsize(\textcolor{red}{+136.9\%}) \\ 
        
        \textbf{MELTR} + StocBIO~\cite{ji2021bilevel} & 
        N/A  &  
        26.8 \scriptsize(\textcolor{red}{+0.7})   &   
        686 \scriptsize(\textcolor{red}{+25.4\%})\\
        
        \textbf{MELTR} + CG & 
        AID-CG &  
        28.0 \scriptsize(\textcolor{red}{+1.9})   &   
        624 \scriptsize(\textcolor{red}{+14.1\%})\\
        
        \textbf{MELTR} + AuxiLearn~\cite{navon2020auxiliary} &  
        AID-FP    &  
        27.9 \scriptsize(\textcolor{red}{+1.8})    &
        638 \scriptsize(\textcolor{red}{+16.6\%})      \\
        
        \textbf{MELTR} + \textbf{AID-FP-Lite}$^\dagger$ & 
        AID-FP &  
        28.5 \scriptsize(\textcolor{red}{+2.4})   &   
        574 \scriptsize(\textcolor{red}{+4.9\%})\\
        \bottomrule
    \end{tabular}
    \end{adjustbox}
    \label{tab:efficiency}
    \vspace{-3mm}
\end{table}

\vspace{-10pt}
\paragraph{Efficiency comparison.}
In Table~\ref{tab:efficiency}, we thoroughly compare the efficiency of our method to recent methods~\cite{ding2022decoupling,xu2022simple,liang2022open}. We measure the number of learnable parameters, the total number of parameters, training time, inference time, and inference GFLOPs. Our model demonstrates strong efficiency in terms of both training and inference. This efficiency is achieved because our framework does not require an additional mask generator~\cite{ding2022decoupling}.


%------------------------------------------------------------------------
\section{Conclusion}
\label{sec:conclusion}
\section{Conclusion}
In conclusion, we introduce a cost aggregation framework for open-vocabulary semantic segmentation, aggregating the cosine-similarity scores between image and text embeddings of CLIP. Through our \ours framework, we fine-tune the encoders of CLIP for its adaptation for the downstream task of segmentation. Our method surpasses the previous state-of-the-art in standard benchmarks and also in scenarios with a vast domain difference. The success in diverse domains underscores the promise and potential of our cost aggregation framework in advancing the field of open-vocabulary semantic segmentation.\\
\vspace{-10pt}\paragraph{Acknowledgement.} This research was supported by the MSIT, Korea (IITP-2023-2020-0-01819, RS-2023-00266509).



%------------------------------------------------------------------------
\section*{Acknowledgement}
We would like to thank the anonymous reviewers for their insightful feedback.
This work is partially supported by the General University Research (GUR) and the University of Delaware Research Foundation (UDRF).

%------------------------------------------------------------------------
%%%%%%%%% REFERENCES
{\small
\bibliographystyle{ieee_fullname}
\bibliography{egbib}
}

\end{document}
