% !TeX spellcheck = en_GB
% encoding: utf8
% !TEX encoding = utf8
% !TEX program = pdflatex

\documentclass{ieeetran}

\usepackage[cmex10]{amsmath}
\usepackage[utf8]{inputenc}
\usepackage[T1]{fontenc}
\usepackage{bm}
%\usepackage{plotstable}
\usepackage{graphicx}
\usepackage{float}
\usepackage{subcaption}
\usepackage[linesnumbered,ruled,vlined,onelanguage]{algorithm2e}
\usepackage{algorithmic}
\usepackage{mathtools}
%\usepackage[]{algorithmic}
\usepackage{array}

\usepackage{array}
\usepackage{makecell}
\usepackage{csvsimple}
\usepackage{booktabs}
\usepackage{xargs}
%\usepackage{color, colortbl}
\usepackage{footmisc}

\usepackage{ifthen}

% % %
% BibTeX styles
\bibliographystyle{IEEEtran}


%%%%%
%%  Comments 
%%%%%
\usepackage{xargs}



\newcommand{\fig}[1]{fig.~\ref{#1}}
\newcommand{\Fig}[1]{Fig.~\ref{#1}}

\usepackage{fourier} 
\usepackage{array}
\usepackage{makecell}
\usepackage{csvsimple}
\usepackage{booktabs}

\renewcommand\theadalign{bc}
\renewcommand\theadfont{\bfseries}
\renewcommand\theadgape{\Gape[4pt]}
\renewcommand\cellgape{\Gape[4pt]}
\renewcommand{\labelenumii}{\theenumii}
\renewcommand{\theenumii}{\theenumi.\arabic{enumii}.}
\usepackage{tabularx}
% compiler error without following two lines
\usepackage{caption,setspace}
%\captionsetup{font={sf,small,stretch=0.80},labelfont={bf,color=accessblue}}
%\usepackage{soul}
\usepackage{xargs}


\usepackage{hyperref} %<--- Load after everything else

%%%%%
%%   Definicja wymagań
%%%%
\newcommand{\req}[1]{\textbf{\texttt{R#1}}}
\newcommand\mycommfont[1]{\footnotesize\ttfamily\textcolor{blue}{//#1\\}}
\begin{document}
	
	\title{MeROS -- Metamodel for ROS-based Systems}
	%\title{Harmonizing complex tasks for robots with the TaskER framework}
	
	\author{Tomasz Winiarski$^{1}$, Member, IEEE
	\thanks{$^{1}$Warsaw University of Technology, Institute of Control and Computation Engineering, Poland
			{\tt\small tomasz.winiarski@pw.edu.pl}
	}}
	
		\maketitle
	
	\begin{abstract}
		
		The complexity of today's robot control systems implies difficulty in developing them efficiently and reliably. Systems engineering (SE) and frameworks come to help. The framework metamodels are needed to support the standardisation and correctness of the created application models. Although the use of frameworks is widespread nowadays, for the most popular of them, Robot Operating System (ROS) version~1, a~complete, contemporary metamodel, has been missing so far. This article proposes a~new metamodel for ROS (MeROS), which addresses both the running system and developer workspace. For compatibility with the latest versions of ROS~1, the metamodel includes the latest ROS~1 concepts such as nodelet, action, and metapackage. An essential addition to the original ROS concepts is the grouping concepts, which provide an opportunity to illustrate the decomposition of the system, as well as varying degrees of detail in its presentation. The metamodel is derived from the requirements and then verified on the practical example of the Rico assistive robot. The matter is described in the SysML language, supported by standard development tools to conduct projects in the spirit of SE.
	\end{abstract}
	


	
	\section{Introduction}
	\label{sec:intro}	
	
	
	The development of civilisation leads to an increase in the importance of robotics. In particular, it is determined by demographic changes, but also by the race for technological supremacy. Many modern robotic systems are complex. To create them as effectively and reliably as possible, it is necessary to follow systems engineering (SE), where metamodels play an essential role~\cite{bezivin2004search,schmidt2006model,kent2002model}.
	
	Robots, especially complex ones, are mostly software-controlled. Hence, in robotics SE is inextricably linked with software engineering, where the use of frameworks has been crucial for many years \cite{mnkandla2009software,shehory2014agent}.
	Many robotics frameworks have been developed so far \cite{inigo2012robotics,tsardoulias2017robotic,hentout2016survey}. Some steps towards standardisation have been made in the recent years, and ROS (Robot Operating System) has come to the fore. Stand-alone ROS~1 (ROS version 1)~\cite{quigley2009ros} is unsuitable for hard RT (Real Time) systems, so in large number of practical applications ROS~1 is integrated with Orocos \cite{bruyninckx2001open,bruyninckx2002orocos}. Over time, ROS~1 has evolved to, among other things, improve its performance. Known and crucial problems in the face of some contemporary applications (e.g. cybersecurity, RT performance) led to the development of a~new version of the framework, ROS~2 \cite{maruyama2016exploring,park2020real}.   
	
	According to~analyses of ROS metrics from~July 2021\footnote{\url{https://discourse.ros.org/t/2021-ros-metrics-report/24130/1}}, interest in ROS is growing, and~though the ROS~2 version is gaining popularity, ROS~1 is still more than twice as popular as ROS~2, even in terms of new package downloads, not to mention~existing applications.
	ROS~1 has evolved considerably from the initial distributions. Its metamodels that were created so far are now incomplete and outdated (sec.~\ref{sec:related-work}). According to ROS metrics, new ROS~1 distributions have practically replaced older one's, which lacked elements such as actions or nodelets. These indications point to the need to formulate an up-to-date and complete recent ROS~1 metamodel, which this article undertakes by presenting the new metamodel for ROS -- MeROS.
	
	The robotic models can be subdivided~\cite{de2021survey} into Platform Independent Models (PIM), e.g., \cite{zielinski2017variable,zielinski2010motion,tasker2020,earl2020}, and Platform Specific Models (PSM). The metamodels of ROS, including MeROS, belong to PSM and should answer to the component nature of ROS \cite{Figat:2022:RAS,wenger2016model}. It is convenient to use a~standardized language to describe the model. Hence, in this work, the Systems Modelling Language (SysML) \cite{omg-sysml16,Friedenthal:2015} is chosen as it is a~standard in SE.
	
	The following presentation starts from formulating the requirements (sec.~\ref{sec:requirements}) for the MeROS metamodel. These requirements are allocated to the metamodel that is described in sec.~\ref{sec:metamodel}. The way to present a~model of a~specific application based on MeROS is presented on practical example in sec.~\ref{sec:example}.  
	A comparison of the MeROS with similar metamodels is presented in sec.~\ref{sec:related-work}. The paper is finalized with conclusions (sec.~\ref{sec:conslusions}).
	
	
	\section{Metamodel requirements}
	\label{sec:requirements}
	
	The requirements formulation process for MeROS metamodel is multi-stage and iterative. In the beginning, the initial requirements were formulated based on: (i) literature review (both scientific and ROS wiki/community sources), (ii) author experience from supervising and supporting ROS-based projects, and finally, (iii) author experience from EARL \cite{earl2020} PIM development and its applications (e.g. \cite{tasker2020,karwowski2021hubero,en14206693-grav-comp}). Verification of draft versions of MeROS by its practical applications led to an iterative reformulation of requirements and MeROS itself. The article presents the final version of both MeROS metamodel and the requirements it originates from.
	
	MeROS requirements are depicted on a~number of dedicated SysML diagrams. The requirements are organised in a~tree-like nesting structure, with additional internal relations, and labelled following this structure. The general requirements are depicted in Fig.~\ref{fig:general_req}.
	
	\begin{figure}[htb]
		\centering
		\begin{center}
			{\includegraphics[width=\columnwidth]{img/requirement_pkg/general_req.png}}
		\end{center}
		\caption{General requirements.} 
		\label{fig:general_req}
	\end{figure}
	
	
	The SysML models have two main parts: behavioural [R1] and structural [R2]. The goal of MeROS is to cover ROS concepts [R3] and not change theirs labels as long as possible, to maintain conformity and intuitiveness. The ROS system is two-faced. While it is executed [R4], it has a~specific structure and behaviour, but from the developers' point of view, the workspace [R5] is the exposed aspect. The model should be compact and simple [R6] rather than unnecessarily elaborate and complicated. One of the assumptions that stands out MeROS from other ROS metamodels is conformity with the latest ROS~1 versions [R7]. Although the SysML-based MeROS is classified into PSM [R8], i.e., Platform Specific Models, it should be compatible with Non-ROS elements [R9].
	
	The structural aspects requirements are presented in Fig.~\ref{fig:structural_aspects_req}.
	
	\begin{figure}[htb]
		\centering
		\begin{center}
			{\includegraphics[width=\columnwidth]{img/requirement_pkg/structural_aspects_req.png}}
		\end{center}
		\caption{Structural aspects requirements.} 
		\label{fig:structural_aspects_req}
	\end{figure}
	
	A~vital addition to the original ROS concepts is the grouping of communicating methods [R2.1] and communicating components [R2.2]. The motivation for this is presented further on.
	
	The ROS concepts that MeROS models are grouped into four major classes (Fig.~\ref{fig:ros_concepts_req}): Communicating components [R3.1], Communication methods [R3.2], Workspace [R3.3], and Other [R3.4].
	
	\begin{figure}[htb]
		\centering
		\begin{center}
			{\includegraphics[width=\columnwidth]{img/requirement_pkg/original_ros_concepts_req.png}}
		\end{center}
		\caption{ROS concepts requirements.} 
		\label{fig:ros_concepts_req}
	\end{figure}
	
	Communicating components (Fig.~\ref{fig:communicating_components_req}) are: ROS Node [R3.1.1], ROS Nodelet [R3.1.2], ROS plugin [R3.1.3], and ROS library [R3.1.4]. The system demands two particular Nodes: ROS Master [R3.1.1.1] and rosout [R3.1.1.2].
	
	\begin{figure}[htb]
		\centering
		\begin{center}
			{\includegraphics[width=\columnwidth]{img/requirement_pkg/communicating_elements_req.png}}
		\end{center}
		\caption{Communicating components requirements.} 
		\label{fig:communicating_components_req}
	\end{figure}
	
	Communication concepts (Fig.~\ref{fig:communication_concepts_req}) cover three methods of ROS communication and relate two inter-component connections and data structures: ROS Topic [R3.2.1] with its Message [R3.2.1.1] and connection [R3.2.1.2], ROS Service [R3.2.2] with its data structure [R3.2.2.1] and connection [R3.2.2.2], and finally ROS Action [R3.2.3] with its data structure [R3.2.3.1] and connection [R3.2.3.2].
	
	\begin{figure}[htb]
		\centering
		\begin{center}
			{\includegraphics[width=\columnwidth]{img/requirement_pkg/communication_concepts_req.png}}
		\end{center}
		\caption{Communication concepts requirements.} 
		\label{fig:communication_concepts_req}
	\end{figure}
	
	
	Workspace concepts (Fig.~\ref{fig:workspace_concepts_req}) comprises ROS Package [R3.3.1] and Metapackage [R3.3.2].
	
	
	\begin{figure}[htb]
		\centering
		\begin{center}
			{\includegraphics[width=\columnwidth]{img/requirement_pkg/workspace_concepts_req.png}}
		\end{center}
		\caption{Workspace concepts requirements.} 
		\label{fig:workspace_concepts_req}
	\end{figure}
	
	
	To achieve intuitiveness MeROS presents Running system structure (Fig.~\ref{fig:running_system_req}) following rqt\_graph pattern [R4.1]. In particular, there are two ways to present communication utilising topics, with [R4.1.1]
	%	\footnote{exemplary graph \url{https://github.com/siemens/ros-sharp/issues/174}}
	and without [R4.1.2]
	% \footnote{exemplary graph \url{https://industrial-training-master.readthedocs.io/en/melodic/_source/session6/Using-rqt-tools-for-analysis.html}}
	dedicated communication blocks. The dedicated blocks are especially useful in the presentation when many communication components use the same topic both on the publisher and the subscribe side. In opposition, the expression of topics names on arrows connecting communicating components, i.e., without dedicated communication blocks, let to reduce the number of blocks needed to depict communication for many topics and a~low number of communicating components. Services [R4.2] and actions [R4.3] should be depicted as an addition to the presentation of the particular topics. It should be noted that rqt\_graph represents actions as a~number of topics. In MeROS the topics executing an action are aggregated, that reduces the number of depicted connections.
	
	\begin{figure}[htb]
		\centering
		\begin{center}
			{\includegraphics[width=\columnwidth]{img/requirement_pkg/running_system_req.png}}
		\end{center}
		\caption{Running system requirements.} 
		\label{fig:running_system_req}
	\end{figure}
	
	The compactness and simplicity [R6] and its nesting requirements are presented in Fig.~\ref{fig:compactness_and_simplicity_req}. 
	
	\begin{figure}[htb]
		\centering
		\begin{center}
			{\includegraphics[width=\columnwidth]{img/requirement_pkg/compactness_and_simplicity_req.png}}
		\end{center}
		\caption{Compactness and simplicity requirements.} 
		\label{fig:compactness_and_simplicity_req}
	\end{figure}
	
	A SysML project to develop and represent MeROS metamodel should consist of a~small number of packages [R6.1], but still, the packages should distinguish the major aspects of development: (i) metamodel requirements formulation, (ii) metamodel itself, and (iii) metamodel realizations/applications.
	Dedicated SysML stereotypes [R6.2] are introduced to MeROS to replace the direct block specialization representation on diagrams and improve the legibility and compactness of diagrams.
	The grouping of concepts [R6.3] has diverse aims. It enables presentation of a~system part in a~general, PIM like abstract way, on the logical level rather than a~detailed, PSM like implementation one. The aggregation reduces the number of objects represented on the diagram, to highlight the essential aspects, and stay compact and consistent in presentation.
	
	There are three elements in the requirements set that satisfy the specifics of latest ROS~1 versions (Fig.~\ref{fig:recent_ros_version_req}): ROS Nodelet [R3.1.2], ROS Action [R3.2.3], and ROS Metapackage [R3.3.2].
	
	\begin{figure}[htb]
		\centering
		\begin{center}
			{\includegraphics[width=\columnwidth]{img/requirement_pkg/recent_ros_version_req.png}}
		\end{center}
		\caption{Latest ROS 1 version requirements.} 
		\label{fig:recent_ros_version_req}
	\end{figure}
	
	
	
	\section{MeROS metamodel}
	\label{sec:metamodel}
	
	MeROS metamodel is formulated according to the requirements described in the previous section. Subsequently, the allocated requirements are mentioned in square brackets as [RX], where X is the requirement number. Sec.~\ref{sec:metamodel-composition} presents MeROS blocks structural composition, sec.~\ref{sec:metamodel-communication} describes inter-component communication. From the metamodel perspective, the Structural aspects [R2] are formulated in both sections, while behavioural [R1] in the latter.
	The diagrams comprise selected requirements being allocated to expose the MeROS metamodel development process. 
	
	The MeROS diagrams were created in the Enterprise Architect development tool within the SysML project [R8] organised in three packages [R6.1] (Fig.~\ref{fig:meros_project_packages_pkg}): (i) Requirement Model related to requirements formulation and analysis, (ii) MeROS -- the metamodel itself, (iii) Rico Controller -- the exemplary application of MeROS. The stereotypes are introduced in MeROS metamodel with the dedicated MeROS profile [R6.2].
	
	\begin{figure}[htb]
		\centering
		\begin{center}
			{\includegraphics[width=\columnwidth]{img/meros_project_packages_pkg.png}}
		\end{center}
		\caption{MeROS project packages, where Rico Controller is an exemplary realisation of MeROS metamodel.} 
		\label{fig:meros_project_packages_pkg}
	\end{figure}
	
	
	\subsection{Metamodel composition}
	\label{sec:metamodel-composition}
	
	The blocks reflect ROS concepts [R3], and their composition is depicted in bdd (block definition) diagrams. The metamodel is formulated in a~single package. Hence, Workspaces and Running Systems are composed into ROS System (Fig.~\ref{fig:ros_system_bdd}). Consequently, some concepts (e.g., Node) occur both in Workspaces and Running Systems.
	
	\begin{figure}[htb]
		\centering
		\begin{center}
			{\includegraphics[width=\columnwidth]{img/meros_pkg/ros_system_bdd.png}}
		\end{center}
		\caption{ROS system general composition -- bdd.} 
		\label{fig:ros_system_bdd}
	\end{figure}
	
	In MeROS, a~Communicating Component (Fig.~\ref{fig:communicating_components_bdd}) is a~crucial abstraction of a~number of ROS concepts to represent their standardized role regarding communication. It should be noted that behavioural aspects of a~particular model can be formulated by operation specification as an act (activity), sd (sequence), or stm (state machine) diagrams. The Intrasystem is one of the aggregates added to the base ROS concepts in MeROS. 
	
	\begin{figure}[htb]
		\centering
		\begin{center}
			{\includegraphics[width=\columnwidth]{img/meros_pkg/communicating_components_inherit_bdd.png}}
		\end{center}
		\caption{Communicating Component and specialised blocks -- bdd.} 
		\label{fig:communicating_components_bdd}
	\end{figure}
	
	Relations of Communicating Components are depicted in Fig.~\ref{fig:communication_blocks_bdd} and \ref{fig:communication_blocks2_bdd}. Besides standard ROS communication methods, the Non-ROS are also included to achieve interfaces with Non-ROS parts of the general system. An Action Data Structure comprises data used by three of five Topics composed in Action, i.e., goal, feedback and result.
	
	\begin{figure}[htb]
		\centering
		\begin{center}
			{\includegraphics[width=\columnwidth]{img/meros_pkg/communicating_component_bdd.png}}
		\end{center}
		\caption{Communicating Component relations -- topics and actions -- bdd.} 
		\label{fig:communication_blocks_bdd}
	\end{figure}
	
	\begin{figure}[htb]
		\centering
		\begin{center}
			{\includegraphics[width=\columnwidth]{img/meros_pkg/communicating_component2_bdd.png}}
		\end{center}
		\caption{Communicating Component relations -- services, aggregates, Non-ROS elements -- bdd.} 
		\label{fig:communication_blocks2_bdd}
	\end{figure}
	
	
	The Intrasystem (Fig.~\ref{fig:running_system_bdd}) composes a~number of Communicating Components Types and the connections between them. The Communication method concept is also introduced to aggregate Topic connections, Service connections, and Action connections. It should be noted that although MeROS could be classified as PSM, the initial, general system description with Communications Mediums and Intrasystems corresponds to PIM specification. Then, the detailing of these aggregates corresponds to the transition from PIM to PSM. Running System is a~specialisation of the Intrasystem and relates to the executed system. Hence, it composes specific Nodes, i.e., rosout and ROS master.
	%\twci{ROS master nie musi być w~wuruchomionym systemie 0..1. Podobnie rosout. One występują podczas uruchamiania po czym moga zostac wylaczone. Wtedy nei działa np. service ale truddniejsze cyberataki.}
	
	
	
	\begin{figure}[htb]
		\centering
		\begin{center}
			{\includegraphics[width=\columnwidth]{img/meros_pkg/running_system_bdd.png}}
		\end{center}
		\caption{Running System and Intrasystem compositions -- bdd.} 
		\label{fig:running_system_bdd}
	\end{figure}
	
	The Workspace (Fig.~\ref{fig:ros_workspace_bdd}) contains Packages that compose the files related to general ROS concepts as Node, communication data structures, etc. It should be noted that data structures of Topics composed into Actions are stored in Action Data Structures instead of Msg Data Structures. The Misc <<block>> relates to other ROS and Non-ROS files, e.g., roslaunch configuration, obligatory package.xml, obligatory CMakeLists, graphics. The Metapackage is introduce to allocate [R3.3.2]. 
	
	
	\begin{figure}[htb]
		\centering
		\begin{center}
			{\includegraphics[width=\columnwidth]{img/meros_pkg/ros_workspace_bdd.png}}
		\end{center}
		\caption{ROS Workspace composition -- bdd.} 
		\label{fig:ros_workspace_bdd}
	\end{figure}
	
	
	
	\subsection{Communication}
	\label{sec:metamodel-communication}
	
	This section depicts the behavioural and structural aspects of communication in the system. The diagrams complete presentation from figs.~\ref{fig:communication_blocks_bdd},~\ref{fig:communication_blocks2_bdd}. Here, the ibd (internal block) and sd (sequence) diagrams are used to present three modes of communication: Topic [R4.1] (sec.~\ref{sec:metamodel-topic}), Service [R4.2] (sec.~\ref{sec:metamodel-service}) and Action [R4.3] (sec.~\ref{sec:metamodel-action}).
	
	\subsubsection{Topic}
	\label{sec:metamodel-topic}
	
	
	Fig.~\ref{fig:topic_communication_1_ibd} presents the ibd of publishers and subscribers communication via topics. In this diagram dedicated communication block is used for each Topic [R4.1.1].
	
	\begin{figure}[htb]
		\centering
		\begin{center}
			{\includegraphics[width=\columnwidth]{img/meros_pkg/topic_communication_1_ibd.png}}
		\end{center}
		\caption{Topics with dedicated communication blocks -- ibd.} 
		\label{fig:topic_communication_1_ibd}
	\end{figure}
	
	The corresponding sequence diagram is depicted in Fig. \ref{fig:topic_communication_1_sd}.
	
	\begin{figure}[htb]
		\centering
		\begin{center}
			{\includegraphics[width=\columnwidth]{img/meros_pkg/topic_communication_1_sd.png}}
		\end{center}
		\caption{Topics with dedicated communication blocks -- sd.} 
		\label{fig:topic_communication_1_sd}
	\end{figure}
	
	Figs.~\ref{fig:topic_communication_2_ibd},~\ref{fig:topic_communication_2_sd} present an alternative approach depicting the system without dedicated communication blocks [R4.1.2]. 
	
	\begin{figure}[htb]
		\centering
		\begin{center}
			{\includegraphics[width=\columnwidth]{img/meros_pkg/topic_communication_2_ibd.png}}
		\end{center}
		\caption{Topics without dedicated communication blocks -- ibd.} 
		\label{fig:topic_communication_2_ibd}
	\end{figure}
	
	\begin{figure}[htb]
		\centering
		\begin{center}
			{\includegraphics[width=\columnwidth]{img/meros_pkg/topic_communication_2_sd.png}}
		\end{center}
		\caption{Topics without dedicated communication blocks -- sd.} 
		\label{fig:topic_communication_2_sd}
	\end{figure}
	
	\subsubsection{Service}
	\label{sec:metamodel-service}
	
	For each ROS Service there is at most one server and a~number of clients Fig.~\ref{fig:service_communication_ibd} and ~\ref{fig:service_communication_sd}. Service type communication is bidirectional and realizes RPC (remote procedure call).
	
	\begin{figure}[htb]
		\centering
		\begin{center}
			{\includegraphics[width=\columnwidth]{img/meros_pkg/service_communication_ibd.png}}
		\end{center}
		\caption{Service-based communication -- ibd.} 
		\label{fig:service_communication_ibd}
	\end{figure}
	
	\begin{figure}[htb]
		\centering
		\begin{center}
			{\includegraphics[width=\columnwidth]{img/meros_pkg/service_communication_sd.png}}
		\end{center}
		\caption{Service-based communication -- sd.} 
		\label{fig:service_communication_sd}
	\end{figure}
	
	
	\subsubsection{Action}
	\label{sec:metamodel-action}
	
	The general structure of ROS Action communication (fig~.\ref{fig:action_communication_compact_ibd}) is analogous to ROS Service.
	
	\begin{figure}[htb]
		\centering
		\begin{center}
			{\includegraphics[width=\columnwidth]{img/meros_pkg/action_communication_compact_ibd.png}}
		\end{center}
		\caption{Action based communication -- compact representation -- ibd.} 
		\label{fig:action_communication_compact_ibd}
	\end{figure}
	
	In ROS 1, an Action is based on several Topics (Fig.~\ref{fig:action_communication_topics_ibd}).
	
	\begin{figure}[htb]
		\centering
		\begin{center}
			{\includegraphics[width=\columnwidth]{img/meros_pkg/action_communication_topics_ibd.png}}
		\end{center}
		\caption{Action based communication -- Topics -- ibd.} 
		\label{fig:action_communication_topics_ibd}
	\end{figure}
	
	In practice, to present an action-related communication compactly (Fig.~\ref{fig:action_communication_compact_sd}) particular Topics can be generalized as a~request (for /goal and /cancel Topics) and a~response (for /status, /feedback and /result Topics). It should be noted that the diagram presents the Action communication sequence in simplified way.
	
	
	\begin{figure}[htb]
		\centering
		\begin{center}
			{\includegraphics[width=\columnwidth]{img/meros_pkg/action_communication_compact_sd.png}}
		\end{center}
		\caption{Action-based communication sequence -- simplified, compact presentation -- sd.} 
		\label{fig:action_communication_compact_sd}
	\end{figure}
	
	The detailed behaviour of Action server and Action client is specified by state machines \footnote{\url{http://wiki.ros.org/actionlib/DetailedDescription}}. Here, these two state machines are depicted in SysML stm diagrams. In the description, in addition to the original ROS wiki presentation, the Topics are directly mentioned both in transitions and states actions. Fig.~\ref{fig:action_server_stm} depicts Action server state machine.
	
	\begin{figure}[htb]
		\centering
		\begin{center}
			{\includegraphics[width=\columnwidth]{img/meros_pkg/action_server_stm.png}}
		\end{center}
		\caption{Action server -- stm.} 
		\label{fig:action_server_stm}
	\end{figure}
	
	The Action server transitions depend on the new messages sent by the Action client or internal Action server predicates. The Action client state machine (Fig.~\ref{fig:action_client_stm}) depends on server state provided by the Action server in /status Topic and internal Action client predicates. 
	
	
	
	\begin{figure}[htb]
		\centering
		\begin{center}
			{\includegraphics[width=\columnwidth]{img/meros_pkg/action_client_stm.png}}
		\end{center}
		\caption{Action client -- stm.} 
		\label{fig:action_client_stm}
	\end{figure}
	
	
	\section{MeROS application}
	\label{sec:example}
	
	This section presents key aspects of an example system development process incorporating MeROS. The example system was created within AAL INCARE project to control the Rico assistive robot (modified TIAGo platform) to execute transportation attendance tasks (Fig.~\ref{fig:herbatka_u_winiara}).
	
	\begin{figure}[htb]
		\centering
		\begin{center}
			{\includegraphics[width=\columnwidth]{img/herbatka_u_winiara.jpg}}
		\end{center}
		\caption{Transportation attendance by Rico robot \url{https://vimeo.com/670252925}} 
		\label{fig:herbatka_u_winiara}
	\end{figure}
	
	The part of the application scenario is conceptually presented in Fig.~\ref{fig:general_sd}. Here, the system and its behaviour are treated in a~general way.
	
	\begin{figure}[htb] 
		\centering
		\begin{center}
			{\includegraphics[width=\columnwidth]{img/rico_pkg/general_sd.png}}
		\end{center}
		\caption{Concept scenario -- sd.} 
		\label{fig:general_sd}
	\end{figure}
	
	In the following part of the description, the <<Running System>> \texttt{:Rico} and sequence diagram frame \texttt{motion execution} are presented in detailed way.
	The block definition diagram in Fig.~\ref{fig:rico_running_system_bdd} depicts the composition of <<RunningSystem>> \texttt{:Rico}.
	
	\begin{figure}[htb]
		\centering
		\begin{center}
			{\includegraphics[width=\columnwidth]{img/rico_pkg/rico_running_system_bdd.png}}
		\end{center}
		\caption{Rico Running System composition -- bdd.}
		\label{fig:rico_running_system_bdd}
	\end{figure}
	
	
	The <<Running System>> \texttt{:Rico} structure is depicted in Fig.~\ref{fig:rico_running_system_ibd}. Here, and in the following diagrams the \texttt{rosout} and \texttt{ROS master} <<Node>>s were omitted to make the diagrams more compact. The specific label is needed for <<CommMedium>>, e.g., <<CommMedium>> \texttt{:Move To to Robot Core}, because this <<CommMedium>> is described later on.
	
	\begin{figure}[htb]
		\centering
		\begin{center}
			{\includegraphics[width=\columnwidth]{img/rico_pkg/rico_running_system_ibd.png}}
		\end{center}
		\caption{Structure of Running System -- ibd.} 
		\label{fig:rico_running_system_ibd}
	\end{figure}
	
	<<CommMedium>> \texttt{:Move To to Robot Core} is depicted in Fig.~\ref{fig:move_to_2_core_cm_ibd}. It comprises three actions.
	
	\begin{figure}[htb]
		\centering
		\begin{center}
			{\includegraphics[width=\columnwidth]{img/rico_pkg/move_to_2_core_cm_ibd.png}}
		\end{center}
		\caption{Example Communication Medium -- ibd.} 
		\label{fig:move_to_2_core_cm_ibd}
	\end{figure}
	
	The part of the scenario generally described in Fig.~\ref{fig:general_sd} is depicted in detail in Fig.~\ref{fig:motion_execution_sd}. The presentation remains conceptual from the behavioural point of view, but it considers the particular parts of the <<RunningSystem>> \texttt{:Rico}.
	
	\begin{figure}[htb]
		\centering
		\begin{center}
			{\includegraphics[width=\columnwidth]{img/rico_pkg/motion_execution_sd.png}}
		\end{center}
		\caption{Motion execution -- sd.} 
		\label{fig:motion_execution_sd}
	\end{figure}
	
	Finally, on the most detailed level, the particular communication methods are specified (Fig.~\ref{fig:command_motion_sd}).
	
	\begin{figure}[htb]
		\centering
		\begin{center}
			{\includegraphics[width=\columnwidth]{img/rico_pkg/command_motion_sd.png}}
		\end{center}
		\caption{Command motion with detailed Communication Mediums presentation -- sd.} 
		\label{fig:command_motion_sd}
	\end{figure}
	
	The part of the <<Workspace>> \texttt{Rico} that includes previously mentioned elements is presented in Fig.~\ref{fig:rico_workspace_bdd}.
	
	\begin{figure}[htb]
		\centering
		\begin{center}
			{\includegraphics[width=\columnwidth]{img/rico_pkg/rico_workspace_bdd.png}}
		\end{center}
		\caption{Rico Workspace composition -- bdd.}
		\label{fig:rico_workspace_bdd}
	\end{figure}
	
	
	\section{Related work}
	\label{sec:related-work}
	
	Papers in the scope of the literature review are chosen based on an intensive study of the previous scientific work in robotic systems modelling. In particular, the survey \cite{de2021survey} is deeply analysed. As the qualification criterion, the occurrence of ROS 1 metamodel is chosen as well as  UML, or SysML language, to describe it. Six papers describing five metamodels met this criterion, all of which used UML. The metamodels are contrasted with the representative requirements to which MeROS is subjected (Tab.~\ref{tab:ros-spec-req-sota}).
	For clarification, the table refers to aspects of the metamodels, which are visualised in the analysed papers' diagrams.
	
	
	\begin{table}[htb]
		\centering
		\caption{MeROS requirements satisfaction in ROS specific metamodels.}
		\setlength{\tabcolsep}{3.5pt} % Default value: 6pt
		\renewcommand{\arraystretch}{1.0} % Default value: 1
		\begin{tabular}{c c c c c c c c c c c}
			\toprule
			Publication \textbackslash\  req. & \rotatebox{90}{R3.1.1 Node}  & \rotatebox{90}{R3.1.2 Nodelet}  & \rotatebox{90}{R3.1.3 ROS plugin} & \rotatebox{90}{R3.1.4 ROS library} & \rotatebox{90}{R3.2.1 Topic} & \rotatebox{90}{R3.2.2 Service} & \rotatebox{90}{R3.2.3 Action}&
			\rotatebox{90}{R3.3.1 Package} & \rotatebox{90}{R3.3.2 Metapackage}& \rotatebox{90}{R6.3 Grouping of concepts}\\
			\midrule
			Ecore \cite{wenger2016model}                                  & $+$ & $-$ & $-$ & $-$ & $+$ & $+$ & $+$ & $-$ & $-$ & $+$\\
			RosSystem\cite{garcia2019bootstrapping}                       & $-$ & $-$ & $-$ & $-$ & $+$ & $+$ & $+$ & $-$ & $-$ & $-$\\
			HyperFlex \cite{brugali2016hyperflex,gherardi2013variability} & $+$ & $-$ & $-$ & $-$ & $+$ & $+$ & $+$ & $+$ & $-$ & $-$\\
			RoBMEX \cite{ladeira2021robmex}                               & $+$ & $-$ & $-$ & $-$ & $+$ & $+$ & $-$ & $+$ & $+$ & $+$\\
			ROSMOD \cite{kumar2016rosmod}                                 & $+$ & $-$ & $-$ & $+$ & $-$ & $+$ & $-$ & $+$ & $-$ & $+$\\
			MeROS 
			& $+$ & $+$ & $+$ & $+$ & $+$ & $+$ & $+$ & $+$ & $+$ & $+$\\
			\bottomrule
		\end{tabular}
		\label{tab:ros-spec-req-sota}
	\end{table}	
	
	The authors of \cite{wenger2016model} present Ecore -- the ROS 1 metamodel as the central part of the ReApp workbench created to support the efficiency of software creation for robotic systems. The metamodel is specified in a~single, extensive structural diagram, with the ROS node being its central part. The diagram describes the aspects of running system and comprises all ROS communication methods. The nodes are composed into an AppNetwork concept [R6.3]. 
	
	In the paper \cite{garcia2019bootstrapping}, the authors propose two methods based on the created RosSystem metamodel. It aims at the automated generation of models from manually written artefacts through static code analysis and monitoring the execution of the running system. A~large part of the work is concentrated on the toolchain. This ROS metamodel is structurally specified in a~UML class diagram that emphasises communication methods.
	
	HyperFlex toolchain \cite{brugali2016hyperflex,gherardi2013variability} includes extensive and comprehensive metamodel addressing both ROS 1 and complementary Orocos. Formerly, these two frameworks were used together to take from RT properties of Orocos and elasticity of ROS. The presentation of HyperFlex is complex \cite{brugali2016hyperflex,gherardi2013variability}, both the running system and workspace are considered. Concepts such as nodelet or metapackage are missing due to the HyperFlex period of its foundation.
	
	RoBMEX \cite{ladeira2021robmex} was created as a~top-down methodology based on a~set of domain-specific languages that enhance the autonomy of ROS-based systems by allowing the creation of missions graphically and then generating automatically executable source codes conforming to the designed missions. Hence, the ROS metamodel was extended by the upper layer with mission/task specification. The metamodel is complex and inspiring and consists of the running and workspace parts. In the workspace aspect, the grouping concepts are introduced together with subpackages, classified as metapackage for the sake of generality.
	
	ROSMOD \cite{kumar2016rosmod} is the Robot Operating System Model-driven development tool suite,
	an integrated development environment for rapid prototyping component-based
	software for ROS. Its internal metamodel is complex and comprises a~number of standard ROS concepts and additional grouping concepts. Although the description is extensive, the ROSMOD was created in 2016, hence some current concepts are missing, like ROS actions or nodelets.
	
	
	\section{Conlusions}
	\label{sec:conslusions}
	
	Diagrams are an integral part of the description of component-based robot control systems. ROS comprises rqt\_graph tool that generates diagrams showing the structure of the running system. This capability is readily used by software developers (e.g., \cite{thale2020ros,bisi2018development,gupta2020design}) due to its ease of use. Unfortunately, despite its numerous advantages and configurability, this tool has many limitations. Hence, in parallel to automatically generated diagrams, others are needed, some of which are based on UML/SysML. The most comprehensive modelling solutions include explicitly defined metamodels. As a~novelty regarding previous works, this paper proposes a~complete and up-to-date metamodel for current versions of ROS~1 supported by profile to support the metamodel application in ROS applications models. The profile and other materials are accessible from MeROS project page\footnote{\url{https://www.robotyka.ia.pw.edu.pl/projects/meros/}}. In MeROS, the metamodel of original ROS concepts is extended by grouping concepts. It lets to present part of the system in a~PIM-like style instead of a~platform specific -- PSM.
	
	System development involves the use of a~number of tools organised in toolchains. The degree of tools interaction varies. In software engineering, the aim is to create clear procedures for system development, with an indication of the dependencies between the successive stages of the development process. In robotics, there have been many works dedicated to toolchains (SmartMDSD \cite{dennis2016smartmdsd}, RobotML \cite{robotml2}, \cite{dal2022formal}), nowadays the ROS is common middleware (e.g. \cite{wienke2012meta}, BRIDE \cite{bubeck2014bride}, HyperFlex \cite{brugali2016hyperflex}). MeROS is part of the toolchain used in the Robot Programming and Machine Perception Team at Warsaw University of Technology (WUT). At the forefront of the toolchain stays the modelling of the system with PIM using the EARL language \cite{earl2020}, derived from agent theory \cite{kornuta-bpan-2020, zielinski2010motion,zielinski2017variable}. In the intermediate stage, MeROS plays the major role of a~PSM. Finally, FABRIC  \cite{Seredynski-fabric-romoco-2019}, as well as alternative approaches \cite{winiarskimmar2015,figat2020robotic} are used to support code generation. Current work concerns deepening the integration of MeROS with the rest of the toolchain. The works are centred around two robotic platforms: the Velma service robot \cite{en14206693-grav-comp,Figat:2022:RAS} \footnote{\url{https://www.robotyka.ia.pw.edu.pl/robots/velma}} (Fig. \Ref{fig:velma}) and the assistive robot Rico \cite{tasker2020,karwowski2021hubero} \footnote{\url{https://www.robotyka.ia.pw.edu.pl/robots/rico}} (Fig.\ref{fig:rico}).
	
	\begin{figure}[htb]
		\centering
		\begin{subfigure}[b]{.43\columnwidth}
			\includegraphics[height=5.5cm]{img/velma.jpg}
			\caption{Velma.}
			\label{fig:velma}
		\end{subfigure}
		\begin{subfigure}[b]{.47\columnwidth}
			\includegraphics[height=5.5cm]{img/rico.jpg}
			\caption{Rico.}\label{fig:rico}
		\end{subfigure}
		\caption{Robotic platforms.}
		\label{fig:robots}
	\end{figure}
	
	Although ROS~1 still dominates among component-based robotic frameworks, one should expect many robotic platforms to migrate to ROS~2 in the future, not to mention the use of ROS 2 in newly developed systems. The ROS version change will entail a~corresponding adaptation of MeROS, which should not cause significant difficulties.
	
	
	\section*{Acknowledgment}
	The research was funded by the Centre for Priority Research Area Artificial Intelligence and Robotics of Warsaw University of Technology within the Excellence Initiative: Research University (IDUB) programme. 
	
	\bibliography{meros}
	
	
	
%	\begin{IEEEbiography}[{\includegraphics[width=1in,height=1.25in,clip,keepaspectratio]{img/autorzy/twiniarski.jpg}}]{Tomasz Winiarski} IEEE, INCOSE Member, M.Sc./Eng. (2002), PhD (2009) in control and robotics, from Warsaw University of Technology (WUT), assistant professor of WUT. He is a~member of Robotics Group as the head of Robotics Laboratory in Institute of Control and Computation Engineering (ICCE), Faculty of Electronics and Information Technology (FEIT). He is working on modelling and design of robots, and programming methods of robot control systems. The research targets service and social robots as well as didactic robotic platforms. His personal experience concerns development and modelling of robotic frameworks, manipulator position--force and impedance control, safety in robotic research. Recently, he was the head of the WUT group in AAL -- INCARE project "Integrated Solution for Innovative Elderly Care".
%	\end{IEEEbiography}
	
	
	
\end{document}
