\begin{figure*}[t]
    \centering
    \includegraphics[width=\textwidth]{Images/ProblemFormV2.png}
    \caption{A two node UAV network. At each time step, the link cost between $D_{B}$ and $D_{F}$ increases while the link cost between $D_{B}$ and $D_{F}$ is constant. The B.A.T.M.A.N. routing protocol will continue to select the route with the lower link cost, branch $(b)$. The proposed solution, branch $(a)$, would preemptively switch routes to manage network congestion and select an alternate route despite being unfavorable at the current time step.}
    \label{fig:SysMod}
\end{figure*} 

The initial system, shown in Fig.~\ref{fig:SysMod}, is a simple two-node UAV network, which will be extended to a multi-node heterogeneous UAV network in future work. The network employs the B.A.T.M.A.N. routing protocol for communication between UAVs. The aim of this study, illustrated branch $(a)$ in Fig.~\ref{fig:SysMod}, is to demonstrate the feasibility of an ML-aided B.A.T.M.A.N. protocol to improve network congestion by predicting when to switch routes, even if switching to a route with a higher cost is not immediately beneficial. 

\subsection{B.A.T.M.A.N. Protocol}
As a baseline, the network uses the B.A.T.M.A.N routing protocol~\cite{batman}. B.A.T.M.A.N. was designed to address the challenges of routing in mobile ad-hoc networks (MANETs), such as frequent topology changes and the lack of a central authority to coordinate routing. Rather than maintain information about the global network topology, B.A.T.M.A.N. only requires nodes to maintain information about the best next hop to its immediate neighbors. The network is flooded with originator messages (OGMs). OGMs routed through good paths are received by nodes quicker than those transmitted on poor quality routes, informing the nodes in the network which immediate neighbor has the best route to transmit across. The routing tables are configured by selecting the best next hop to the originator node \cite{b11}.

However, there are a few drawbacks to the B.A.T.M.A.N. routing algorithm. For one, if the network contains a substantial number of nodes, B.A.T.M.A.N. can generate a large amount of overhead, as each node must re-broadcast the OGM to its neighbors. This can lead to increased network congestion and reduced overall efficiency. Additionally, in scenarios where the source or destination of the packet is in motion, B.A.T.M.A.N. can suffer from higher delays, which is undesirable if the network topology is highly dynamic. Finally, the drawback we focus on in this work is that B.A.T.M.A.N. is a threshold-based routing protocol, demonstrated in Fig.~\ref{fig:SysMod}. As a result, the node will always choose the next hop with the best route, even if conditions on the current best route are degrading. Waiting to change routes until the threshold is met can cause bottlenecks in the network \cite{b12}. 
