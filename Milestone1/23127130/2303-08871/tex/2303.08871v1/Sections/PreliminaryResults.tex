\subsection{CIFAR10 Baseline}
We use the CIFAR10 dataset \cite{b20} to test the performance of the network emulator proposed in Section \ref{sec:FLinEMANE} and compare the results to a centralized ML approach. To accurately model a distributed UAV environment, we randomly sample the CIFAR10 dataset so that each NEM in the emulator is not training on an identical dataset. We train the two edge nodes for 10 epochs. The results are presented in Fig.~\ref{fig:FLEMANERes}, demonstrating that the FL emulator ('x' curve) approximated the traditional ML approach ('o' curve). Because the nodes are not training on an identical dataset, the FL training loss curve should approximate the behavior of the centralized ML model training curve, but should not identically match the curve. The results presented in Fig.~\ref{fig:FLEMANERes} indicate the system setup proposed in Section \ref{sec:FLinEMANE} works and is suitable for future integration with the ML model presented in Section \ref{sec:MLBat} in addition to expanding the number of NEMs included in the emulation. 

\subsection{Machine Learning Aided B.A.T.M.A.N.}
We attempt to hand generate a simple testing and training set consisting of 50 time samples of link costs across the two routes branching from $D_{B}$ as demonstrated in Fig.~\ref{fig:TrainSet}. We construct the LSTM to have two recurrent layers. The input sequence length is set to four. We use a batch size of five and train the model for ten epochs using the binary cross entropy loss function and the ADAM optimizer with a learning rate of 0.01. Initial testing accuracy results show the LSTM has an 100\% classification accuracy. This is due to the memoryless nature of the hand generated data set. Therefore, in this current state, it is not possible to accurately estimate the practicality of the proposed model.

\begin{figure}[!t]
\centering
    \includegraphics[width = \linewidth]{Images/Train_Test.png}
    \caption{Demonstration of Hand Generated Training Set}
    \label{fig:TrainSet}
\end{figure}

\begin{figure}[!t]
    \centering
    \includegraphics[width = \linewidth]{Images/Centralized_v_FL2Nodes_10_Epochs.png}
    \caption{Federated Learning in EMANE Results: FL Model Approximates Centralized Model Performance}
    \label{fig:FLEMANERes}
\end{figure}