Unmanned Aerial Vehicles (UAVs) are a rapidly developing technology that has been used in numerous applications, including transportation, traffic control, surveillance, search and rescue, and disaster management \cite{b3}. Although UAV technology has many advantages, numerous challenges still need to be addressed to implement networking protocols for UAV-based infrastructures \cite{b2}. For one, UAV networks are highly dynamic. They do not have a consistent topology making communication, control, and path planning protocols designed for less dynamic mobile ad-hoc networks (MANETs) less effective \cite{b3}. These challenges motivate a need for routing protocols catered to UAVs. Ideally, these protocols should be simple, have low overhead, and not require extensive global topology knowledge  \cite{b4}. Further, because of the dynamic nature of UAVs, these protocols should make decisions based on the expected network topology rather than just the current state of the network. These requirements make AI-based routing protocols for UAVs appealing.

AI-based routing protocols are not a new area of research. For example, in \cite{b5} a supervised feed-forward neural network (FFNN) was proposed to learn network traffic history to adaptively route packets and improve heterogeneous network control. Using the Open Shortest Path First (OSPF) routing algorithm, the model input was an array representing the number of packets that were forwarded through each node in the network, while the model's output was the interface to forward the packet along. Simulation results demonstrated the effectiveness of the proposed FFNN approach and outperformed the OSPF baseline. Similarly, in \cite{b6}, Boltzmann machines (RBM) were proposed where the input was characterized as the traffic pattern observed at each router. Like \cite{b5}, this approach outperforms the baseline OSPF routing algorithm. Finally, the research presented in \cite{b7} proposed a neural network (NN) trained at each link in the network, and the output of the NN was the likelihood of successful packet delivery if the packet was forwarded along that link. The authors propose using buffer capacity, number of successful packet transfers, and node popularity.

All of these approaches focus on applying common machine learning (ML) techniques to the AI-based routing protocol problem in homogeneous networks, which will not work UAV networks. However, an emerging ML technique, federated learning (FL) \cite{b21}, has yet to be explored as a solution for heterogeneous networks. In this work, we propose an FL-based approach to the AI-based routing protocol problem, specifically for UAV swarms. We narrow our focus to the Better Approach to Mobile Ad-hoc Networking (B.A.T.M.A.N.) protocol \cite{b8}, propose modifying the algorithm using a NN model, and characterize the dataset necessary for this problem. Finally, we present an FL emulation environment built on the Extendable Mobile Ad-hoc Network Emulator (EMANE) \cite{EMANE} that will be used for testing the proposed solution. 

The remainder of this paper is organized as follows: Section \ref{sec:SysMod} provides an overview of the system model. Section \ref{sec:ProposedSln} presents the proposed solution and describes FL in more detail. Section \ref{sec:PreLim} shows the preliminary results of FL setup in a network emulator and NN model, and Section \ref{sec:FutureWrk} discusses conclusions and future research directions.