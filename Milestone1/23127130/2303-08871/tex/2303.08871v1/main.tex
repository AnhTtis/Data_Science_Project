\documentclass[conference]{IEEEtran}
\IEEEoverridecommandlockouts
\usepackage{cite}
\usepackage{amsmath,amssymb,amsfonts}
\usepackage{algorithmic}
\usepackage{graphicx}
\usepackage{comment}
\graphicspath{ {./Images/} }
\usepackage{textcomp}
\usepackage{xcolor}
\usepackage{float}
\usepackage{subcaption}


\DeclareMathOperator*{\argmin}{arg\,min}

\def\BibTeX{{\rm B\kern-.05em{\sc i\kern-.025em b}\kern-.08em
    T\kern-.1667em\lower.7ex\hbox{E}\kern-.125emX}}

\bibliographystyle{IEEEtran}
\begin{document}

\title{WIP: Federated Learning for Routing in Swarm Based Distributed Multi-Hop Networks
}

\author{\IEEEauthorblockN{Martha Cash,
Joseph Murphy, Alexander Wyglinski}
\IEEEauthorblockA{Department of Electrical and Computer Engineering,
Worcester Polytechnic Institute\\
Worcester, MA\\
\{mcash, jrmurphy, alexw\}@wpi.edu}
}

\maketitle

\begin{abstract}
Unmanned Aerial Vehicles (UAVs) are a rapidly emerging technology offering fast and cost-effective solutions for many areas, including public safety, surveillance, and wireless networks. However, due to the highly dynamic network topology of UAVs, traditional mesh networking protocols, such as the Better Approach to Mobile Ad-hoc Networking (B.A.T.M.A.N.), are unsuitable. To this end, we investigate modifying the B.A.T.M.A.N. routing protocol with a machine learning (ML) model and propose implementing this solution using federated learning (FL). This work aims to aid the routing protocol to learn to predict future network topologies and preemptively make routing decisions to minimize network congestion. We also present an FL testbed built on a network emulator for future testing of the proposed ML aided B.A.T.M.A.N. routing protocol. 
\end{abstract}

\begin{IEEEkeywords}
Federated Learning, Routing, UAV Networks, B.A.T.M.A.N., Machine Learning
\end{IEEEkeywords}

\section{Introduction}\label{sec:Intro}
% Importance and appeal of children's drawings
Children's depictions of the human figure are highly expressive and varied.
As one of the very first subjects children attempt to draw, the representation begins as an almost unintelligible cloud of scribbles. 
As the child grows, their representation of the human figure becomes more developed and is extended to graphically represent many different types of characters: people, animals, and even personified objects (see Figure 1).

Who among us has not wished, either as a child or as an adult, to see such figures come to life and move around on the page?
Sadly, while it is relatively fast to produce a single drawing, creating the sequence of images necessary for animation is a much more tedious endeavor, requiring discipline, skill, patience, and sometimes complicated software.
As a result, most of these figures remain static upon the page.

% We built a system to animate them.
Inspired by the importance and appeal of the drawn human figure, we design and build a system to automatically animate it given an in-the-wild photograph of a child's drawing. 
Our system is fast, intuitive, and robust to much of the variation present in these types of drawings, making it well-suited to allow our target audience--children--to see their own characters coming to life.
The system is comprised of four stages: figure detection, segmentation masking, pose estimation/rigging, and animation. 
We describe each stage and identify common causes of failure in each. 
For object detection and pose estimation, we make use of existing computer vision models designed to detect human figures and joints in photographs; we fine-tune these models for use with children's drawings.
For segmentation, we present a straightforward, image processing-based method that, for animation purposes, is more useful and accurate than segmentation masks obtained from a fine-tuned object detection model.
During the animation step, we take advantage of the \textit{twisted perspective} commonly seen in children’s drawings to retarget motion capture data onto the character in a novel and appealing way.

% We use existing machine learning models. However, given the wide domain gap it's not clear how much fine-tuning data was needed. So we ran some experiments to find out and report it.
While our system leverages existing models and techniques, most are not directly applicable to the task due to the many differences between photographic images and simple pen and paper representations. 
To this end, we couple the presentation of our system with a set of experiments exploring the relationship between fine-tuning training set size and success rates.
We also include a perceptual study validating viewer preference for incorporating \textit{twisted perspective} into the motion retargeting step.

We validate the desirability and appeal of our system by building and publicly releasing a version of it as the \AD Demo \,\cite{animateddrawings}.
Launched in December 2021, this demo has been used by millions of people around the world to animate their children's drawings.
Inspired by this reception, our second contribution is The Amateur Drawings Dataset: \hjs{180,000 drawings and user-accepted annotations collected, with consent, through the demo. See Section \ref{sec:UI} for a description of how the annotations were generated.}
We believe this dataset will be a resource to researchers from various fields seeking to better understand the space of amateur drawings, evaluate new algorithms in this domain, or develop new drawing-based tools in general.

To summarize, our contributions are as follows:
\begin{enumerate}
    \item 
    We explore the problem of automatic sketch-to-animation for children's drawings of human figures and present a framework that achieves this effect. We also present a set of experiments determining the amount of training data necessary to achieve high levels of success and a perceptual study validating the usefulness of our motion retargeting technique.
    \item To encourage additional research in the domain of amateur drawings, we present a first-of-its-kind dataset of 180,000 user-submitted amateur drawings, along with user-accepted bounding box, segmentation mask, and joint location annotations.
\end{enumerate}

Upon acceptance of this paper, we plan to publicly release the Amateur Drawings Dataset, project code, and fine-tuned model weights.


\section{System Model}\label{sec:SysMod}
\begin{figure*}[t]
    \centering
    \includegraphics[width=\textwidth]{Images/ProblemFormV2.png}
    \caption{A two node UAV network. At each time step, the link cost between $D_{B}$ and $D_{F}$ increases while the link cost between $D_{B}$ and $D_{F}$ is constant. The B.A.T.M.A.N. routing protocol will continue to select the route with the lower link cost, branch $(b)$. The proposed solution, branch $(a)$, would preemptively switch routes to manage network congestion and select an alternate route despite being unfavorable at the current time step.}
    \label{fig:SysMod}
\end{figure*} 

The initial system, shown in Fig.~\ref{fig:SysMod}, is a simple two-node UAV network, which will be extended to a multi-node heterogeneous UAV network in future work. The network employs the B.A.T.M.A.N. routing protocol for communication between UAVs. The aim of this study, illustrated branch $(a)$ in Fig.~\ref{fig:SysMod}, is to demonstrate the feasibility of an ML-aided B.A.T.M.A.N. protocol to improve network congestion by predicting when to switch routes, even if switching to a route with a higher cost is not immediately beneficial. 

\subsection{B.A.T.M.A.N. Protocol}
As a baseline, the network uses the B.A.T.M.A.N routing protocol~\cite{batman}. B.A.T.M.A.N. was designed to address the challenges of routing in mobile ad-hoc networks (MANETs), such as frequent topology changes and the lack of a central authority to coordinate routing. Rather than maintain information about the global network topology, B.A.T.M.A.N. only requires nodes to maintain information about the best next hop to its immediate neighbors. The network is flooded with originator messages (OGMs). OGMs routed through good paths are received by nodes quicker than those transmitted on poor quality routes, informing the nodes in the network which immediate neighbor has the best route to transmit across. The routing tables are configured by selecting the best next hop to the originator node \cite{b11}.

However, there are a few drawbacks to the B.A.T.M.A.N. routing algorithm. For one, if the network contains a substantial number of nodes, B.A.T.M.A.N. can generate a large amount of overhead, as each node must re-broadcast the OGM to its neighbors. This can lead to increased network congestion and reduced overall efficiency. Additionally, in scenarios where the source or destination of the packet is in motion, B.A.T.M.A.N. can suffer from higher delays, which is undesirable if the network topology is highly dynamic. Finally, the drawback we focus on in this work is that B.A.T.M.A.N. is a threshold-based routing protocol, demonstrated in Fig.~\ref{fig:SysMod}. As a result, the node will always choose the next hop with the best route, even if conditions on the current best route are degrading. Waiting to change routes until the threshold is met can cause bottlenecks in the network \cite{b12}. 


\section{Proposed Solution}\label{sec:ProposedSln}
We propose two solutions. First, we integrate federated learning (FL) to a wireless drone swarm network and propose an ML model and data set for enhancing the B.A.T.M.A.N. routing protocol. Next, we demonstrate an FL simulation environment built on the EMANE emulation environment, which will be integrated with the proposed ML model in future work to investigate a larger UAV network, and to introduce movement among the UAVs, since this will be a feature compensated for by the ML model. 

\subsection{Machine Learning Model \& Dataset}\label{sec:MLBat}
We consider a supervised learning approach for this work. The objective of supervised learning is to learn a mapping, or function approximation, $\hat{\mathbb{F}}(\mathbf{x},\mathbf{y})$, between a set of samples, $x_{i} \in X$, and their labels, $y_{i} \in Y$, where $X$ and $Y$ are the sample space and label space, respectively. Ideally, $\hat{\mathbb{F}}(x,y)$ takes a set of new samples, $\mathbf{x^{*}}$ and produces the correct label, $\mathbf{y^{*}}$. The quality of the mapping is determined by the loss function, $L(y^{*}_{i}, \hat{y}_{i})$, where $y^{*}_{i}$ is the true label of the new sample, and $\hat{y}_{i}$ is the output of $\hat{\mathbb{F}}({x^{*}_{i}}, \cdot)$ \cite{b18}. An accurate function approximation is quantified by a low loss value. 
 
Since the B.A.T.M.A.N. routing protocol does not maintain a history of route conditions (\textit{i.e.} link cost, throughput), we need to modify the B.A.T.M.A.N. algorithm to include a memory element. The model should learn a history of the prior link costs for each route, and the route the node selected. These requirements make the long short-term memory (LSTM) model, a type of recurrent neural network (RNN) that is designed to learn long-term dependencies in sequential data, appropriate for this task \cite{b14}.

The input to the LSTM model is a two dimensional array of the history of the link cost at each neighboring route from $D_{B}$, (see Fig.~\ref{fig:SysMod}). However, this approach can be extended to $n$ dimensions for $n$ many neighbors in a more complex network. The corresponding labels are a history of the selected route for transmission.  Instead of feeding the entire history of the network to the model, we implement a windowing technique. For example, if the window size is set to 4, then four prior time steps are fed into the model for training. We can treat this as a classification problem and use the binary cross entropy loss function since we are training the LSTM to select which of the two routes from $D_{B}$ to transmit across. 

\subsection{Federated Learning Approach}
Traditional machine learning approaches are centralized, meaning a model is trained on a central dataset that is collected and stored in a central location. However, this approach assumes all devices on the network have the same computational capabilities and network resources, which is often not the case for UAV networks \cite{b2}. As a result, a FL-based distributed ML technique is employed in this research where training of a global model is performed on data distributed across many UAVs in various locations. For generality, we assume there are $J$ UAVs in the network. Each UAV for $j \in J$ observes a unique dataset $\mathbf{x_{j}} = \{x_{j1}, x_{j2}, \ldots, x_{jN}\}$. Since we use a supervised learning approach, we assume a single input sample, $X_{jn}$ corresponds to a single output $y_{jn} \in \{y_{j1}, y_{j2}, \ldots, y_{jN}\}$. The sets $\mathbf{x}_{j}$ and $\mathbf{y}_{j}$ are used to train the local ML model at each UAV. Let $\mathbf{w}_{j} \in \mathbf{w}$ denote the corresponding model parameters at the $j^{th}$ UAV. Then, the FL objective function can be employed, which is defined as~\cite{b19}:

\begin{equation}\label{eq:FLObj}
\argmin_{\mathbf{w} \in \mathbb{R}^{d}} F(\mathbf{w}) = \frac{1}{N} \sum_{j}^{J}\sum_{n=1}^{N_{j}}f(\mathbf{w}_{j}, x_{jn}, y_{jn}). 
\end{equation}

We solve (\ref{eq:FLObj}) via the following steps: First, all UAVs are initialized with random parameters. Each UAV trains on its respective training sets, $x_{jn}$ and $y_{jn}$. After one epoch of training, the FL parameters at the $j^{th}$ UAV are sent to a central server. Once all parameters are received, the central server aggregates the parameters according to the following expression:

\begin{equation}\label{eq:fedAvg}
w_{global} = \frac{1}{J}\sum_{j=1}^{J} \mathbf{w_{j}}
\end{equation}

The global parameters, $w_{global}$, are sent back to the UAVs, and the training process repeats for $K$ epochs, or until $F(\mathbf{w})$ has converged to the optimal parameters, $w^{*}$. Fig.~\ref{fig:FLEMANE} summarizes this approach. 

\subsection{Federated Learning in EMANE}\label{sec:FLinEMANE}
This works also aims to build an FL emulation environment, with the future goal of integrating the emulator with the proposed ML model to test the feasibility of the proposed solution. A system diagram of the emulator is shown in Fig.~\ref{fig:FLEMANE}. At the core of the emulator is EMANE, which allows for the creation of Network Emulation Modules (NEMs) to model different radio interface types. In turn, these can be incorporated into a real-time emulation running in a distributed environment and allow the direct integration of standard software, such as PyTorch, for handling ML tasks. 

For the results in this work, we construct a simulator with three NEMs, similar to the model setup in Fig.~\ref{fig:SysMod}. One NEM is designated as the central server. The remaining NEMs carry out the FL task. However, this could be generalized to $M$ nodes, see Fig.~\ref{fig:FLEMANE}. 

\begin{figure}[!t]
    \centering
    \includegraphics[width = \linewidth]{Images/FL_in_EMANE.png}
    \caption{Federated Learning Setup in EMANE}
    \label{fig:FLEMANE}
\end{figure}


\section{Preliminary Results}\label{sec:PreLim}
\subsection{CIFAR10 Baseline}
We use the CIFAR10 dataset \cite{b20} to test the performance of the network emulator proposed in Section \ref{sec:FLinEMANE} and compare the results to a centralized ML approach. To accurately model a distributed UAV environment, we randomly sample the CIFAR10 dataset so that each NEM in the emulator is not training on an identical dataset. We train the two edge nodes for 10 epochs. The results are presented in Fig.~\ref{fig:FLEMANERes}, demonstrating that the FL emulator ('x' curve) approximated the traditional ML approach ('o' curve). Because the nodes are not training on an identical dataset, the FL training loss curve should approximate the behavior of the centralized ML model training curve, but should not identically match the curve. The results presented in Fig.~\ref{fig:FLEMANERes} indicate the system setup proposed in Section \ref{sec:FLinEMANE} works and is suitable for future integration with the ML model presented in Section \ref{sec:MLBat} in addition to expanding the number of NEMs included in the emulation. 

\subsection{Machine Learning Aided B.A.T.M.A.N.}
We attempt to hand generate a simple testing and training set consisting of 50 time samples of link costs across the two routes branching from $D_{B}$ as demonstrated in Fig.~\ref{fig:TrainSet}. We construct the LSTM to have two recurrent layers. The input sequence length is set to four. We use a batch size of five and train the model for ten epochs using the binary cross entropy loss function and the ADAM optimizer with a learning rate of 0.01. Initial testing accuracy results show the LSTM has an 100\% classification accuracy. This is due to the memoryless nature of the hand generated data set. Therefore, in this current state, it is not possible to accurately estimate the practicality of the proposed model.

\begin{figure}[!t]
\centering
    \includegraphics[width = \linewidth]{Images/Train_Test.png}
    \caption{Demonstration of Hand Generated Training Set}
    \label{fig:TrainSet}
\end{figure}

\begin{figure}[!t]
    \centering
    \includegraphics[width = \linewidth]{Images/Centralized_v_FL2Nodes_10_Epochs.png}
    \caption{Federated Learning in EMANE Results: FL Model Approximates Centralized Model Performance}
    \label{fig:FLEMANERes}
\end{figure}

\section{Conclusion}\label{sec:FutureWrk}
This paper proposed a machine learning (ML) solution implemented via federated learning (FL) to modify the B.A.T.M.A.N. routing protocol for unmanned aerial vehicle networks. We presented a FL testbed built on the network emulator EMANE and used the CIFAR10 dataset to compare the FL testbed to a traditional centralized ML approach. The baseline results proved the testbed works as a proof of concept for future work. We also propose modifying the B.A.T.M.A.N. algorithm using a long short term memory model. However, current results do not accurately reflect the viability of this modification. Therefore, future work will need to generate a dataset from a simulation of the UAV swarm using EMANE to assess this approach better. 

\section*{Acknowledgment}
This research was sponsored by the DEVCOM Analysis Center and was accomplished  under  Cooperative  Agreement  Number  W911NF-22-2-0001. The views and conclusions contained in this document are  those  of  the  authors  and  should  not  be  interpreted  as representing the official policies, either expressed or implied, of  the  Army  Research  Office  or  the  U.S.  Government.  The U.S.  Government  is  authorized  to  reproduce  and  distribute reprints  for  Government  purposes,  notwithstanding  any copyright notation herein.

\bibliography{IEEEabrv,references}

\end{document}