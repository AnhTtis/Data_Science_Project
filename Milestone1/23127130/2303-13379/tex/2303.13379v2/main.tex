%%%%%%%%%%%%%%%%%%%%%%%%%%%%%%%%%%%%%%%%%%%%%%%%%%%%%%%
% A template for Wiley article submissions developed by 
% Overleaf for the Overleaf-Wiley pilot which ran 
% during 2017 and 2018.
% 
% This template is no longer supported, but is provided
% for historical reference. Last updated January 2019.
%
% Please note that whilst this template provides a 
% preview of the typeset manuscript for submission, it 
% will not necessarily be the final publication layout.
%
% Document class options:
% =======================
% blind: Anonymise all author, affiliation, correspondence
%        and funding information.
%
% lineno: Adds line numbers.
%
% serif: Sets the body font to be serif. 
%
% twocolumn: Sets the body text in two-column layout. 
% 
% num-refs: Uses numerical citation and references style 
%           (Vancouver-authoryear).
%
% alpha-refs: Uses author-year citation and references style
%             (rss).
%
% Using other bibliography styles:
% =======================
%
% To specify a different bibiography style
%
% 1) Do not use either num-refs or alpha-refs in documentclass.
% 2) Load natbib package with the options set as needed.
% 3) Use the \bibliographystyle command to specify the style
% 
% Included NJD styles are: 
%   WileyNJD-ACS
%   WileyNJD-AMA
%   WileyNJD-AMS
%   WileyNJD-APA
%   WileyNJD-Harvard
%   WileyNJD-VANCOUVER
%
% or you may upload an alternative .bst file 
% (if requested by the journal).
%
% Examples:
% =======================
%% Example: Using numerical, sort-by-authors citations.
% \documentclass[num-refs]{wiley-article}

%% Example: Using author-year citations and anonymising submission
% \documentclass[num-refs]{wiley-article}
\documentclass[alpha-refs]{wiley-article}

%% Example: Using unsrtnat for numerical, in-sequence citations
% \documentclass{wiley-article}
% \usepackage{natbib}
% \bibliographystyle{WileyNJD-APA}

%% Example: Using WileyNJD-AMA reference style and superscript
%%          citations, two-column and serif fonts for AIChE
% \documentclass[serif,twocolumn,lineno]{wiley-article}
% \usepackage[super]{natbib}
% \bibliographystyle{WileyNJD-AMA}
% \makeatletter
% \renewcommand{\@biblabel}[1]{#1.}
% \makeatother

% Add additional packages here if required
\usepackage{siunitx}
\usepackage{soul,xcolor} %strikethrough with color options
\usepackage{ulem}
% Update article type if known
\papertype{Review Article}

\title{Practical and Ethical Challenges of Large Language Models in Education: A Systematic Scoping Review}

% Include full author names and degrees, when required by the journal.
% Use the \authfn to add symbols for additional footnotes and present addresses, if any. Usually start with 1 for notes about author contributions; then continuing with 2 etc if any author has a different present address.
\author[1]{Lixiang Yan}
\author[1]{Lele Sha} 
\author[1]{Linxuan Zhao}
\author[1]{Yuheng Li}
\author[1]{Roberto Martinez-Maldonado} 
\author[1]{Guanliang Chen} 
\author[1]{Xinyu Li}
\author[1]{Yueqiao Jin}
\author[1]{Dragan Gašević}


% Include full affiliation details for all authors
\affil[1]{Centre for Learning Analytics at Monash, Faculty of Information Technology, Monash University, Clayton, Victoria, Australia}

\corraddress{Lixiang Yan, Centre for Learning Analytics at Monash, Faculty of Information Technology, Monash University, 20 Exhibition Walk, Clayton, VIC 3800, Australia}
\corremail{jimmie.yan@monash.edu}

\fundinginfo{This research was at least in part funded by the Australian Research Council (DP210100060) and Jacobs Foundation (Research Fellowship).}

% Include the name of the author that should appear in the running header
\runningauthor{Yan et al.}

\begin{document}
\setstcolor{red} % set the color for strikethrough
\begin{frontmatter}
\maketitle

\begin{abstract}

Educational technology innovations leveraging large language models (LLMs) have shown the potential to automate the laborious process of generating and analysing textual content. While various innovations have been developed to automate a range of educational tasks (e.g., question generation, feedback provision, and essay grading), there are concerns regarding the practicality and ethicality of these innovations. Such concerns may hinder future research and the adoption of LLMs-based innovations in authentic educational contexts. To address this, we conducted a systematic scoping review of 118 peer-reviewed papers published since 2017 to pinpoint the current state of research on using LLMs to automate and support educational tasks. The findings revealed 53 use cases for LLMs in automating education tasks, categorised into nine main categories: profiling/labelling, detection, grading, teaching support, prediction, knowledge representation, feedback, content generation, and recommendation. Additionally, we also identified several practical and ethical challenges, including low technological readiness, lack of replicability and transparency, and insufficient privacy and beneficence considerations. The findings were summarised into three recommendations for future studies, including updating existing innovations with state-of-the-art models (e.g., GPT-3/4), embracing the initiative of open-sourcing models/systems, and adopting a human-centred approach throughout the developmental process. As the intersection of AI and education is continuously evolving, the findings of this study can serve as an essential reference point for researchers, allowing them to leverage the strengths, learn from the limitations, and uncover potential research opportunities enabled by ChatGPT and other generative AI models.


% Please include a maximum of seven keywords
\keywords{large language models, pre-trained language models, artificial intelligence, education, systematic scoping review, GPT-3, BERT, ChatGPT}
\end{abstract}
\end{frontmatter}

% \section*{Author Bio}

% Lixiang Yan is a PhD candidate in the Faculty of Information Technology at Monash University. His research interests include artificial intelligence, multimodal learning analytics, collaborative learning, and applied machine learning. Roberto Martinez-Maldonado is a senior lecturer of learning analytics and human-computer interaction in the Faculty of Information Technology at Monash University. His current research is in the area of learning and teamwork analytics, in which he utilises his expertise in human-computer interaction, collaborative learning and artificial intelligence. Linxuan Zhao is a PhD candidate in the Faculty of Information Technology at Monash University. His research interests include audio analytics, multimodal learning analytics, and collaborative learning. Xinyu Li is a research fellow in the Faculty of Information Technology at Monash University. His main research areas include self-regulated learning analytics, the application of computer vision in collaborative learning and multimodal learning analytics. Yueqiao Jin is a research assistant at the Centre for Learning Analytics at Monash. Her research interests include human-centred learning analytics and artificial intelligence. Dragan Gašević is a distinguished professor of learning analytics in the Faculty of Information Technology and director of the Centre for Learning Analytics at Monash University. His research interests are learning analytics, educational data mining, self-regulated learning, and collaborative learning. 

\section*{Practitioner notes}
What is currently known about this topic
\begin{itemize}
    \item Generating and analysing text-based content are time-consuming and laborious tasks.    
    \item Large language models are capable of efficiently analysing an unprecedented amount of textual content and completing complex natural language processing and generation tasks.
    \item Large language models have been increasingly used to develop educational technologies that aim to automate the generation and analysis of textual content, such as automated question generation and essay scoring.
\end{itemize}
What this paper adds
\begin{itemize}
    \item A comprehensive list of 53 different educational tasks that could potentially benefit from LLMs-based innovations through automation.
    \item A structured assessment of the practicality and ethicality of existing LLMs-based innovations from seven important aspects using established frameworks. 
    \item Three recommendations that could potentially support future studies to develop LLMs-based innovations that are practical and ethical to implement in authentic educational contexts.
\end{itemize}
Implications for practitioners
\begin{itemize}
    \item Updating existing innovations with state-of-the-art models may further reduce the amount of manual effort required for adapting existing models to different educational tasks.
    \item The reporting standards of empirical research that aims to develop educational technologies using large language models need to be improved.
    \item Adopting a human-centred approach throughout the developmental process could contribute to resolving the practical and ethical challenges of large language models in education. 
\end{itemize}

% \begin{figure}[t]
%     % \begin{subfigure}{1\linewidth}
%     %   \centering
%     % %   \includegraphics[width=1\linewidth]{figs/fig_1_moti_textattn.pdf}  
%     % %   \includegraphics[width=1\linewidth]{figs/fig_1_moti_textattn_v2.pdf}  
%     %   \includegraphics[width=1\linewidth]{figs/fig_1_moti_textattn_v5.pdf}  
%     %   \vspace{-0.5cm}
%     %     \caption{Amount of attention added to each video clip from the source video and query text in the self-attention layers of Moment-DETR encoder.}
%     %     % \caption{Distribution of attention for source and query in Moment-DETR encoder}
%     %     % Visualization of video clip's self-attention score in Moment-DETR encoder.
%     %   \label{fig:fig1_text_attn_ex}
%     % \end{subfigure}%\hfill% or  or \hspace{0.3\textwidth}
%     \vspace{0.2cm}
%     % \begin{subfigure}{1\linewidth}
%       \centering
%     %   \includegraphics[width=1\linewidth]{figs/fig1_moti_negattn.pdf}  
%       \includegraphics[width=1\linewidth]{figs/fig1_moti_negattn_v3.pdf}  
%       \vspace{-0.4cm}
%     %   \caption{Correspondence of saliency scores on the relevance between video clips and the text query.}
%     % \caption{Predicted saliency scores against the video relevant positive query and video irrelevant negative query}
%       \label{fig:fig1_neg_attn_ex}
%     % \end{subfigure}%\hfill% or  or \hspace{0.3\textwidth}
%     \caption{
%     % 원준 원본
%     % (a) Comparison between attention scores of source and query for each video clip~(We sum the attention scores from video and text). 
%     % We observe that the attention scores are dominated by other clips in the source video. 
%     % Text queries do not account for much attention regardless of the relevance to the video clips.
%     % \textbf{(a)} Inspection of the query dependency in Moment-DETR encoder.
%     % % We visualize the attention score of video tokens in the transformer encoder and observe that text query accounts for only a low portion of attention.
%     % % This tendency occurs regardless of the relevance between the text query and video clips. 
%     % We visualize the attention score of video tokens in the transformer encoder and observe 1) text query only accounts for a low portion of attention, and 2) relevance between video-query pair does not affect the attention scores ratio of text.
%     \textbf{(b)} Comparison of highlight-ness when relevant and non-relevant queries are input.
%     As observed in , existing work only uses queries to play an insignificant role, thereby may not be capable of detecting false queries and considering the video-query relevance even when the problem in (a) is resolved. 
%     % \SE{} % 이 부분이 "not capable of" 란 용어가 세다는 피드백이 있는 듯 합니다. 이러한 능력이 없다는 것은 굉장히 강한 어조인거 같기는 하고, 이러한 경우들이 종종 있다거나 좀 약화시킬 필요가 있어보이긴 하네요.
%     On the other hand, our QD-DETR yields a query-dependent representation that the relevance between the source video and query text is updated in the saliency scores.
%     There is a large gap between positive and negative saliency scores, and scores are consistent since the clips are all highly correlated to others.
%     }
%     \label{fig:motivation_ex}
%     % \captionsetup{belowskip=13pt}
%     % \setlength{\belowcaptionskip}{-10pt}
% \end{figure}
\begin{figure}
    \centering
    \includegraphics[width=1\linewidth]{figs/fig1_moti_negattn_1111.pdf}
    % \includegraphics[width=1\linewidth]{figs/fig1_moti_negattn_1109.pdf}
    % \includegraphics[width=1\linewidth]{figs/fig1_moti_negattn_stat.pdf}
    \vspace{-0.6cm}
    \caption{
        % \SE{} % 수정 필요
        Comparison of highlight-ness~(saliency score) when relevant and non-relevant queries are given.
        We found that the existing work only uses queries to play an insignificant role, thereby may not be capable of detecting negative queries and video-query relevance; saliency scores for clips in ground-truth~(GT) moments are low and equivalent for positive and negative queries.
        % This also results in mispredicted moments when ground-truth~(GT) moment is dominated by clips unrelated to GT since their prediction is highly focused on the video.
        % \SE{} % 여기 한번 더 보면 좋을 듯 합니다. GT moment에 unrelated한 clip이 많으면? label이 틀렷을 경우를 말씀하시는건지?
        % As observed in saliency graph, existing work only uses queries to play an insignificant role, thereby may not be capable of detecting false queries and considering the video-query relevance.
        On the other hand, query-dependent representations of QD-DETR result in corresponding saliency scores to the video-query relevance and precisely localized moments.
        % On the other hand, our QD-DETR yields a query-dependent representation that the
        % saliency scores are in accordance with the relevance between the video and query.
        % text is in accordance with the saliency scores.
        % There is a large gap between positive and negative saliency scores, and scores are consistent since the clips are all highly correlated to others.
}
    \label{fig:motivation_ex}
\end{figure}


\section{Introduction}
% 원준 원본
% Along with the advance of digital devices and platforms, video is now one of the most desired data type for consumers. However, although the large information capacity of videos may be beneficial in many aspects, e.g., informative and entertaining, on the contrary perspective, videos are time-consuming, and hard to search for desirable moments. 
% This has led many creators to use extra manpower to crop and edit the video to generate highlight clips to gain the consumer’s attention.
Along with the advance of digital devices and platforms, video is now one of the most desired data types for consumers~\cite{apostolidis2021video,wu2017deep}.
% SE: Video aware deep learning application & survey papers?
Although the large information capacity of videos might be beneficial in many aspects, e.g., informative and entertaining, inspecting the videos is time-consuming, so that it is hard to capture the desired moments~\cite{anne2017localizing,apostolidis2021video}. 
% This has led many creators to use extra manpower to crop and edit the video to generate highlight clips to gain the consumer’s attention.


% On the other side, 
Indeed, the need to retrieve user-requested or highlight moments within videos is greatly raised.
Numerous research efforts were put into the search for the requested moments in the video~\cite{anne2017localizing, gao2017tall, liu2015multi, escorcia2019temporal} and summarizing the video highlights~\cite{zhang2016video, mahasseni2017unsupervised, badamdorj2022contrastive, wei2022learning}.
% Numerous research efforts were put into the search for the requested moments in the video~\cite{anne2017localizing, gao2017tall, liu2015multi, escorcia2019temporal}, summarizing the video to generate highlights was another popular topic~\cite{zhang2016video, mahasseni2017unsupervised, badamdorj2022contrastive, wei2022learning}.
Recently, Moment-DETR~\cite{momentdetr} further spotlighted the topic by proposing a QVHighlights dataset that enables the model to perform both tasks, retrieving the moments with their highlight-ness, simultaneously.

% 원준 원본
% To detect the desired moments, previous works employed transformer encoder-decoder architectural designs to fuse the text query into the video representations. Moment-DETR~\cite{mDETR} modified detection transformer to process capture the moment as a set, and UMT~\cite{umt} implemented transformer decoder as to output clip-wise saliency. 
% Yet to their outstanding breakthroughs in the literature of moment retrieval with the seminal architectures, their limitation is that the role of the given text query is insignificant in representing the query-conditioned video representation; the attention mechanism of moment DETR is not explicitly conditioned on the text query, and the text query is conditioned on multi-modal clips where the differences between the clips are smoothed after encoding process in UMT.



% \begin{figure}[t]
% \centering
%     \begin{subfigure}[l]{0.37\linewidth}
%       \centering
%       \vspace{0.20cm}
%     %   \includegraphics[width=1\linewidth]{figs/fig_1_moti_textattn.pdf}  
%     %   \includegraphics[width=1\linewidth]{figs/fig_1_moti_textattn_v2.pdf}  
%       \includegraphics[width=1\linewidth]{figs/fig1_moti_violin_a.pdf}  
%       \vspace{-0.60cm}
%     %   \caption{text attention}
%         \caption{Importance of queries in video representation}
%       \label{fig:fig1_text_attn}
%     \end{subfigure}%\hfill% or  or \hspace{0.3\textwidth}
%     \vspace{0.2cm}
%     \begin{subfigure}[r]{0.61\linewidth}
%       \centering
%     %   \includegraphics[width=1\linewidth]{figs/fig1_moti_negattn.pdf}  
%       \includegraphics[width=1\linewidth]{figs/fig1_moti_violin_b.pdf}  
%     %   \caption{neg attention}
%         % \caption{Relation between the highlight-ness and the relevance between videos and query texts.}
%         \caption{Highlight-ness~(saliency) histogram of positive and negative video-query pairs\SE{}}
%       \label{fig:fig1_neg_attn}
%     \end{subfigure}%\hfill% or  or \hspace{0.3\textwidth}
%     % \vspace{-0.2cm}
%     \caption{Overall statistics for attention scores in Fig.~\ref{fig:motivation_ex} in QVHighlights dataset. 
%     (a) For the attention scores that measure how much the text query is generally involved in video representation, we use violin plots to show the probability density. We plot the score for each layer in the encoder.
%     % (b) Using the histogram, we compare how the baseline and QD-DETR yield different salient scores given the positive and negative video-text pairs.
%     (b) Saliency histogram shows the distributional gap between positive and negative video-text query pairs of baseline~(Moment-DETR) and proposed QD-DETR.\SE{}
%     }
%     \label{fig:motivation}
%     % \captionsetup{belowskip=13pt}
%     % \setlength{\belowcaptionskip}{-10pt}
% \end{figure}

% \begin{figure}[t]
% \centering

%     \begin{subfigure}[r]{1\linewidth}
%       \centering
%       \hspace{-0.2cm}
%     %   \includegraphics[width=1\linewidth]{figs/fig1_moti_negattn.pdf}  
%       \includegraphics[width=1.1\linewidth]{figs/fig1_moti_violin_a_v2.pdf}  
%     %   \caption{neg attention}
%         % \caption{Relation between the highlight-ness and the relevance between videos and query texts.}
%         \vspace{-0.5cm}
%         % \caption{Saliency histogram of positive and negative video-query pairs}
%         \caption{We plot the histograms and its average value~(dotted line) to compare saliency scores when true and false text queries are given for each method. (left) Since the video representations do not include much textual information, both the true and false queries yield similar saliency scores. (Middle) Even when the video representation is enforced to be updated with the textual information, the issue is not much resolved. (Right) By extracting discriminative features in the text query, distributions are differentiated.
%         % \SE{} % R1@0.5 설명
%         Also, R1@0.5 indicates evaluation metric, Recall at 1 with IoU 0.5 threshold on QVhighlight \textit{val} set.
%         }
%       \label{fig:fig1_neg_attn}
%     \end{subfigure}%\hfill% or  or \hspace{0.3\textwidth}
%     \\
%     \begin{tabular}{cc}
%     \hspace{-0.2cm}
%         \begin{minipage}{.4\linewidth}
%             \begin{subfigure}[l]{1\linewidth}
%               \centering
%             %   \vspace{0.20cm}
%             %   \includegraphics[width=1\linewidth]{figs/fig_1_moti_textattn.pdf}  
%             %   \includegraphics[width=1\linewidth]{figs/fig_1_moti_textattn_v2.pdf}  
%               \includegraphics[width=1\linewidth]{figs/fig1_moti_violin_a.pdf}  
%               \vspace{-0.60cm}
%             %   \caption{text attention}
%                 \caption{Importance of queries in video representation}
%               \label{fig:fig1_text_attn}
%             \end{subfigure}%\hfill% or  or \hspace{0.3\textwidth}
%         \end{minipage}
        
%         \begin{minipage}{.6\linewidth}
%             \vspace{-0.2cm}
%             \caption{Overall statistics of Fig.~\ref{fig:motivation_ex} in QVHighlights dataset. 
%             (a) Saliency histogram shows the distributional gap between positive and negative video-text query pairs.
%             % (a) For the attention scores that measure how much the text query is generally involved in video representation, we use violin plots to show the probability density. We plot the score for each layer in the encoder.
%             % (b) Using the histogram, we compare how the baseline and QD-DETR yield different salient scores given the positive and negative video-text pairs.
%             % (b) Text ratio in self-attention layer to  of Moment-DETR
%             % (b) Ratio of text when representing video tokens in self-attention of Moment-DETR.
%             % (b) Magnitude of attention text query involved.
%             % (b) Attention score of video tokens
%             % (b) Magnitude of text query to refine the video tokens in self-attention layer of Moment-DETR.
%             (b) Probability density depicting the weight of the text query in attention score for video clips. Scores are from the self-attention layers in Moment-DETR encoder.
%             % (b) The text query ratio in attention score of video clips (Self-attention layer in Moment-DETR encoder). We use violin plots to show probability density.
%             % 텍스트 쿼리가, 비디오 피쳐에 얼만큼 attend 하는지
%             }
%         \end{minipage}
    
%     \end{tabular}
%     \vspace{-0.5cm}
%     \label{fig:moti}
%     % \captionsetup{belowskip=13pt}
%     % \setlength{\belowcaptionskip}{-10pt}
% \end{figure}


% \begin{figure}
%     \centering
%     % \includegraphics[width=1\linewidth]{figs/fig1_moti_negattn_1109.pdf}
%     \includegraphics[width=1\linewidth]{figs/fig1_moti_negattn_stat_v2.pdf}
%     \vspace{-0.8cm}
%     \caption{
%         Histogram of saliency when the positive and negative queries are given. We plot the histograms and its average value~(dotted line) to compare saliency scores when relevant~(positive) and irrelevant~(negative) text queries are given for each method. (Left) Since the video representations do not properly reflect textual information, both the positive and negative queries yield similar saliency scores. 
%         % (Middle) Even when the video representation is enforced to be updated with the textual information, the issue is not much resolved. 
%         (Right) By representing video clips in query-dependent manner, distributions are differentiated.
%     }
%     \vspace{-0.6cm}
%     \label{fig:motivation}
% \end{figure}


% One of the demanding task is moment retrieval task, which is detecting the desired moments from the given query, typically the text query.
When describing the moment, one of the most favored types of query is the natural language sentence~(text)\cite{anne2017localizing}. 
While early methods utilized convolution networks~\cite{zhang2020learning, gao2021fast, wang2020temporally}, recent approaches have shown that deploying the attention mechanism of transformer architecture is more effective to fuse the text query into the video representation.
% To handle these modalities, previous works simply employed the attention mechanism of transformer architecture to fuse the text query into the video representation.
For example, Moment-DETR~\cite{momentdetr} introduced the transformer architecture which processes both text and video tokens as input by modifying the detection transformer~(DETR), and UMT~\cite{umt} proposed transformer architectures to take multi-modal sources, e.g., video and audio. 
Also, they utilized the text queries in the transformer decoder.
Although they brought breakthroughs in the field of MR/HD with seminal architectures, they overlooked the role of the text query.
To validate our claim, we investigate the Moment-DETR~\cite{momentdetr} in terms of the impact of text query in MR/HD~(Fig.\ref{fig:motivation_ex}).
Given the video clips with a relevant positive query and an irrelevant negative query, we observe that the baseline often neglects the given text query when estimating the query-relevance scores, i.e., saliency scores, for each video clip.
% the output saliency score, i.e. query-relevance scores.
% Based on the observation, we traced the actual saliency prediction of the model against both the video-relevant query and the irrelevant dummy one where we find that the baseline often neglects the given text query when estimating the query-relevance scores of video clips.
% For example, in Fig.~\ref{fig:motivation_ex}, saliency scores are not affected even when the query is substituted with the dummy.
% % General statistics for Fig.~\ref{fig:motivation_ex} is shown in Fig.~\ref{fig:motivation}. 
% General statistics corresponding to Fig.~\ref{fig:motivation_ex} are also shown in Fig.~\ref{fig:motivation}.



% The limitation of the concrete baseline~\cite{momentdetr} is inspected in two different aspects; 1) Utilization of text-query in the encoding process and 2) the output saliency score, i.e. query-relevance scores.
% Firstly, we visualize the attention score when video clips are given as a query in self-attention. 
% We observe that the text queries have relatively small impacts compared to other video features, as shown in Fig.~\ref{fig:fig1_text_attn_ex}.
% That is, the text does not account for much in representing every video clip, although the goal of MR/HD is to detect query-relevant moments.
% Based on the observation, we traced the actual saliency prediction of the model against both the video-relevant query and the irrelevant dummy one where we find that the baseline often neglects the given text query when estimating the query-relevance scores of video clips.
% For example, in Fig.~\ref{fig:motivation_ex}, saliency scores are not affected even when the query is substituted with the dummy.
% % General statistics for Fig.~\ref{fig:motivation_ex} is shown in Fig.~\ref{fig:motivation}. 
% General statistics are also shown in Fig.~\ref{fig:motivation}.

% Consequently, in Fig.~\ref{fig:fig1_neg_attn_ex}~(b), we found that the baseline often neglects the given text query when estimating the query-relevance scores of video clips; 
% For example, 


% We validate the previous work sometimes neglects the given query when estimating the saliency of video clips.
% For example, there is an example that the saliency scores from positive and negative queries cannot be distinguishable, as shown in Fig.~\ref{fig:fig1_neg_attn_ex}.
% % 우리는 추가로 text attention을 추가도 해봤지만, 효과가 있긴 했으나, still 이슈가 있는 것을 확인하였다?
% % Still, we observe that assuring the high attendance of text queries does not resolve the overlap which motivates us to question the quality of the naive use of task-agnostic text representation~\cite{momentdetr, umt}.
% We found that introducing the text-attention for ensuring the high attendance of text queries relieve the overlap, but there still be a severe overlap.


% To validate their limitations, we inspect the impacts of text queries in the concrete baseline~\cite{momentdetr} with the two different aspects, 1) tendency of attention in self-attention layer and 2) saliency score, i.e. query-relevance scores. \SE{} % attention 이 갑자기 등장하는가?
% Firstly, we visualize the attention score when video clips are given as a query in self-attention. We observe the text queries have relatively low attention scores compared to the video features, as shown in Fig.~\ref{fig:fig1_text_attn_ex}.
% That is, the text does not account for much in representing every video clip, although the goal of MR/HD is to detect query-relevant moments.
% Based on this observation, we trace the actual saliency prediction of the model against both positive and negative text queries.
% We validate the previous work sometimes neglects the given query when estimating the saliency of video clips.
% For example, there is an example that the saliency scores from positive and negative queries cannot be distinguishable, as shown in Fig.~\ref{fig:fig1_neg_attn_ex}.
% % 우리는 추가로 text attention을 추가도 해봤지만, 효과가 있긴 했으나, still 이슈가 있는 것을 확인하였다?
% % Still, we observe that assuring the high attendance of text queries does not resolve the overlap which motivates us to question the quality of the naive use of task-agnostic text representation~\cite{momentdetr, umt}.
% We found that introducing the text-attention for ensuring the high attendance of text queries relieve the overlap, but there still be a severe overlap.



% Thus, we 
% query dependency를 높이기 위해 
% Cross-attention? text-attention? detailed explanation on text-attention should be needed?
% By handling these two issues, we find that more precise retrieval can be achieved.
% 
% 
%
% By projecting video-discriminative text features with high text attendance to source video, we f 
% We also find the need to improve the quality of query features since assuring high text attendance also results in...
% pairs are not finetuned to be discriminative that even the similarity within the pairs does not reflect the relevance between the query and the video clips.
% General statistics for Fig.~\ref{fig:motivation_ex} is shown in Fig.~\ref{fig:motivation}. 
% \SE{} % 이거 ??로 뜨는데, 위처럼 figure 그리면 label이 안되는걸까요
% \SE{}
% 형님 아래 사항 생각 좀 해보는게 좋을 거 같아요.
% fig 1. (a) 그림만 봤을 때 모든 clip에 대해 text attention이 일정이상 존재하긴 하니까, 뭔가 not assured to be conditioned가 와닿지 않는거 같아요.
% + 왜 text가 항상 attend 해야하나?
% not assured to be conditioned --> text shows relatively low affects compared to video 같이 실제 나타난 현상까지 같이 적으면 어떨까 싶어요.
% fig 1. (b) 덜 반영한다?

% \SU{}
% 일단 text가 attend 잘 되어야 한다는 것에 좀 궁금점이 생깁니다. 결국에는 text와 관련있는 frame들을 attend해서 higlight를 찾아야 하는게 아닐까요? 그리고, 현제 저희의 모델 구조상 text query가 Key와 Value로 거의 활용되고 있는데 그렇다면 결국에는 해당 모델은 text에 대한 attention이 전혀 없다고 봐도 무방하지 않을까요? 그런 면에서 text attention을 강조하는게 좀 걸리긴 합니다.

% Specifically, the text query is not assured to be explicitly conditioned on every clip of the video, and as the query texts are evenly treated, discriminative keywords may not be spotlighted.
% attention mechanism of Moment-DETR is not explicitly conditioned on the text query as shown in Fig~\ref{}(d), and in UMT, the text are only used for conditioning the queries while the video representation are refined itself by self-attention.

% \begin{figure}[t]
%     \begin{subfigure}{1\linewidth}
%       \centering
%     %   \includegraphics[width=1\linewidth]{figs/fig_1_moti_textattn.pdf}  
%     %   \includegraphics[width=1\linewidth]{figs/fig_1_moti_textattn_v2.pdf}  
%       \includegraphics[width=1\linewidth]{figs/fig_1_moti_textattn_v4.pdf}  
%       \vspace{-0.5cm}
%     %   \caption{text attention}
%         \caption{Distribution of attention scores in Moment-DETR encoder}
%       \label{fig:fig1_text_attn}
%     \end{subfigure}%\hfill% or  or \hspace{0.3\textwidth}
%     \vspace{0.2cm}
%     \begin{subfigure}{1\linewidth}
%       \centering
%     %   \includegraphics[width=1\linewidth]{figs/fig1_moti_negattn.pdf}  
%       \includegraphics[width=1\linewidth]{figs/fig1_moti_negattn_v2.pdf}  
%       \vspace{-0.5cm}
%     %   \caption{neg attention}
%         \caption{Saliency score against positive and negative text queries}
%       \label{fig:fig1_neg_attn}
%     \end{subfigure}%\hfill% or  or \hspace{0.3\textwidth}
%     \vspace{0.2cm}
%     \begin{subfigure}{1\linewidth}
%       \centering
%     %   \includegraphics[width=1\linewidth]{figs/fig1_moti_violin.pdf}  
%       \includegraphics[width=1\linewidth]{figs/fig1_moti_violin_v2.pdf}  
%       \vspace{-0.5cm}
%       \caption{violin}
%       \label{fig:fig1_violin}
%     \end{subfigure}%\hfill% or  or \hspace{0.3\textwidth}
%     \vspace{-0.2cm}
%     \caption{(a) 1. portion of text attention vs. video attention 2. relation with text query and content (e.g. fg, bg) of clip seems not to affect the attention score
%     (b) 1. high variability even though entire clips are highly correlated with the given text query 2. positive and negative query makes overlaps on saliency score distribution
%     (3) actual distribution on validation dataset.}
%     \label{fig:motivation}
%     % \captionsetup{belowskip=13pt}
%     % \setlength{\belowcaptionskip}{-10pt}
% \end{figure}

To this end, we propose Query-Dependent DETR~(QD-DETR) that produces query-dependent video representation.
% Our key focus is to ensure each clip in predicted moments is explicitly conditioned by the query, particularly on the video-descriptive portion of the text query.
% Our key focus is to ensure that query-relevant clips are predicted by enforcing each clip to be explicitly conditioned by the query.
%Our key focus is to ensure that the model prediction for each clip is highly relevant to the query.
Our key focus is to ensure that the model's prediction for each clip is highly dependent on the query.
% by enforcing each clip to be explicitly conditioned by the query. :)
% hmm...
% \SE {} % "query-relevant clips are predicted" 이 문장이 좀 애매한거 같습니다. relevant 클립을 놓지지 않고 찾는 것을 보장한다? 이런 느낌인지 아니면 높은 saliency 를 주는게 목적이다? model prediction이 query-relevance를 반영하는 것을 보장한다?
% Our key focus is to ensure that the model prediction reflects query-relevance of clips by enforcing each clip to be explicitly conditioned by the query.
First, to fully utilize the contextual information in the query, we revise the transformer encoder to be equipped with cross-attention layers at the very first layers.
% 상익's thought :  single video - query간의 관계만 고려 - 같은 word가 더 많이 쓰이는 것을 보고 
% 교수님's thought : neg pair 를 쓰면 쿼리를 보지 않고서는 video clip간만 고려하는 것이 사라짐. 왜냐면 0으로 내보내야 하기 때문. --> SE: relative difference 만 고려하다가, 
By inserting a video as the query and a text as the key and value of the cross-attention layers, our encoder enforces the engagement of the text query in extracting video representation.
% 원준 교수님 코멘트 반영해서 다시
Then, in order to not only inject a lot of textual information into the video feature but also make it fully exploited, we leverage the negative video-query pairs generated by mixing the original pairs.
Specifically, the model is learned to suppress the saliency scores of such  negative~(irrelevant) pairs.
Our expectation is the increased contribution of the text query in prediction since the videos will be sometimes required to yield high saliency scores and sometimes low ones depending on whether the text query is relevant or not.
% \SE{}
% learns to?
% By suppressing the saliency scores of the irrelevant video-query pairs, the model learns to spotlight only the video-specific discriminative words in the query.
% % \SE{} % ====================== 상익 수정 ========================
% However, this architectural design still lacks the capability of identifying the video-descriptive keywords in the query.
% % However, this architectural design still lacks in identifying proper query relevance.
% This is because the current training scheme only focuses on the interactions of video and clips within a single video while neglecting information shared throughout the entire video.
% % We argue the problem of the current training scheme that only focuses on distinguishing the clips in a single video while neglecting information shared throughout the entire video.
% Therefore, we leverage the negative video-query relationships to enhance the capability of identifying the contextual similarity of query and video clips.
% 
% 원준 원본 
% However, this architectural design heavily relies on the quality of the text query.
% Therefore, we leverage the negative video-query relationships to enable the model to emphasize key corresponding query features.
% By suppressing the saliency scores of the irrelevant video-query pairs, the model learns to spotlight only the video-specific discriminative words in the query.
% =========================================================
Lastly, to apply the dynamic criterion to mark highlights for each instance, we deploy a saliency token to represent the entire video and utilize it as an input-adaptive saliency criterion. 
With all components combined, our QD-DETR produces query-dependent video representation by integrating source and query modalities.
This further allows the use of positional queries~\cite{dabdetr} in the transformer decoder.
% Furthermore, we can exploit the advanced DETR decoder architectures using the positional information, e.g., DAB-DETR, since our encoded tokens consist of identical position representations from a single modality.
% \SE{} % ====================== 상익 수정 ========================
% Furthermore, we can exploit the advanced DETR decoder architectures using the positional information, e.g., DAB-DETR, since our video clip tokens consist of identical position representations from a single modality.
% 원준 원본
% It also enables the use of advanced DETR decoder architectures, e.g., DAB-DETR, for the first time, as these works exploit the position information within a single modality.
% =========================================================
Overall, our superior performances over the existing approaches validate the significance of the role of text query for MR/HD.
% Our extensive experiments on QVHighlights, TVSum, and Charades-STA datasets validate the significance of considering the role and the quality of text query.

% All components combined with dynamic anchor moments for the query of decoder, our FOQUE fosters the query-dependent video representation, thereby making the 
% All components combined, our modified transformer encoding process fosters the query-dependent video representation thereby achieving the state-of-the-art results on various benchmarks of moment-retrieval and highlight detection.
	
% -	Video Platform & Streamer & Consumer의 증가. 
% Video는 다른 데이터 타입보다 정보가 많아 유용하지만, 이는 다른 말로 해석하면 video를 보는 것은 time-consuming 하고, 원하는 것을 찾아보기에는 힘들 수 있음.
% 따라서, 많은 매체에서는 사람들의 더 많은 이목을 끌기 위해 highlight 비디오라는 것을 편집하여 공유도 함.
% 하지만, highlight video를 만들기 위해 사람의 노력이 필요한 현 시점에서, This spotlights the need to retrieve the user-requested / Highlight moments in the video.

% -	이전에도 이러한 문제를 해결하기 위해 (asdfasdf) for moment retrieval, (asdfasdf) for highlight detection 등이 제안 되었지만, 이들은 비디오의 특정 영역을 찾는다는 공통된 목적을 가지고 있으면서도, 데이터 셋의 한계로 인해 따로 연구되었음. 이를 문제 삼으며, 최근에는 두 task를 동시에 학습할 수 있는 dataset이 소개 되었는데, 컴퓨터비전에서 최근 각광을 받고 있는 Transformer 모델 도입과 함께 큰 발전을 거듭하고 있음.

% -	구체적으로, 이 두가지 task를 수행하기 위해서는 transformer를 두가지 방법으로 이용할 수 있는데, moment-DETR 처럼 moment 를 clip의 set 단위로 예측할 수 있고, UMT 처럼 clip-wise prediction을 할 수 있음. 하지만, 이들은 query를 condition이 아닌 video와 동등한 레벨로 취급하거나 [mDETR], 매 클립이 self-attention으로 mixing 된 후에 condition을 걸어주어 clip간의 차이를 확실하지 이용하지 못하였고, 또한, 확실하게 condition으로 주지 못하였고, video와 query 사이의 관계를 한정적으로만 이용하였다.

% -	따라서, we explore three different ways to fully exploit query information. First, we design one-way cross-attention layer to condition every clip with the query features. Then, we utilized the negative video-text pairs to better model the relationships between the video and the text embeddings. Lastly, we define the saliency token to be the video-query dependent saliency estimator.


















% ===================== neg pair 부분 ===========================
% Nevertheless, the current training scheme, only considering the given video-query pair, still disturbs the model from identifying proper query-relevance prediction.
% In detail, the model focus on learning the fine-grained discrepancy between video clips, while neglecting the information they share, which contains significant clues to understand the context of video.
% Therefore, we leverage the negative video-query relationships to enhance the capability of identifying the contextual similarity of query and video clips.
% Therefore, we leverage the negative video-query relationships by suppressing those pairs, so that enhance the capability of identifying the contextual similarity of query and video clips.
% We hypothsize the diversity in query-video pairs are insufficient to learn the general relationship between text query and video.
% Therefore, we leverage the negative video-query relationships by suppressing the saliency scores of the irrelevant video-query pairs.
% However, this architectural design still lacks in identifying proper query relevance.
% We argue that the current training scheme only focuses on learning the fine-grained discrepancy between clips in a single video, while neglecting the information they share, which contains significant clues to understand the context of the video.
% Therefore, we leverage the negative video-query relationships to enhance the capability of identifying the contextual similarity of query and video clips.
% However, this architectural design still lacks in identifying proper query relevance.
% We argue the problem of the current training scheme that only focuses on learning the fine-grained discrepancy between clips in a single video.
% That is, the current design neglects the information shared throughout the video, although it contains significant clues to understand the context of the video.
\section{Related Work}

 %Most MMLA studies so far have primarily focused on developing prototypes and testing the functionality of different combinations of sensors and analytics approaches \citep{shankar2018review, mu2020multimodal, noroozi2020multimodal}. 
In this section, we review the most recent systematic literature reviews on MMLA and related works that have identified several prominent logistical, privacy and ethical challenges that need to be addressed for this promising area to remain relevant and have an actual impact on educational practices.

%MMLA and multimodal data have received increased attention from the learning analytics and educational technology communities as a promising research direction that holds the potential to generate meaningful insights about teaching, and learning in partially and non-computer mediated educational contexts \citep{sharma2020multimodal, chango2022review}. 
% \citep{alwahaby2021evidence, yan2022scalability}. 

%\subsection{Practical Challenges}

%The practical challenges of MMLA are mainly associated with the lack of well-reported and large-scale studies that structurally assess the influence of MMLA innovations on actual educational practice. For example, \citet{alwahaby2021evidence} reviewed 100 MMLA articles and concluded that most of the empirical evidence presented in prior studies remains descriptive, correlational, or anecdotal, with little strong causal evidence regarding the impacts of MMLA innovations on real-world educational practices. 


%Consequently, although the alignment between MMLA innovations and learning design should be one of the foundations for developing MMLA innovations \citep{cukurova2020promise, ochoa_multimodal_2022}, such alignments are rarely considered or reported in the existing literature, as evidenced in recent reviews \citep{sharma2020multimodal, praharaj2021literature}. 
%Likewise, the lack of MMLA studies that closed the LA loop by providing feedback to students or insights to teachers during educational practices instead through post-hoc research-focused interviews or surveys also hindered the understanding of MMLA innovations' actual impacts on learning and teaching outcomes \citep{yan2022scalability}. %Therefore, in-depth insights on aligning MMLA innovations with learning designs, relevant theories, and educational stakeholders are urgently needed to ensure future MMLA studies do not deviate from the ultimate goals of learning analytics \citep{gavsevic2015let}.

\subsection{Logistical Challenges}
Most MMLA studies so far have primarily focused on developing prototypes and testing the functionality of different combinations of sensors and analytics approaches \citep{shankar2018review, mu2020multimodal, noroozi2020multimodal}. Yet, many concerns have been raised regarding the logistical challenges that can emerge when moving from controlled settings to in-the-wild MMLA deployments such as the added intrusiveness of sensing devices and complexity in their installation and orchestration \citep{chua2019technologies}. \citet{yan2022scalability} systematically reviewed these logistical issues and identified a relatively low level of technology readiness regarding existing MMLA innovations, resulting in heavy reliance on the onsite support of researchers or technicians. This  undermines the sustainability of these systems and unnecessarily increases the complexity of the learning situation from the teachers' perspective. While most of the sensing technologies used in MMLA research can be purchased off-the-shelf, implementing these technologies in authentic physical learning spaces often requires extensive technical background for tasks such as physical installation, system integration, and modalities synchronisation \citep{crescenzi2020multimodal, shankar2018review, mu2020multimodal}. There is also a trade-off between data quality and affordability as most of the MMLA innovations that rely on mature sensing technologies, such as location sensors, eye-trackers, and biometric sensors, can be financially unscalable due to the high unit prices \citep{yan2022scalability}. Although low-cost alternatives are emerging \citep[e.g.,][]{ochoa2018rap, saquib2018sensei}, these technologies remain in the prototype and validation stages and often sacrifice accuracy or portability for affordability. 

Likewise, the lack of MMLA studies that have closed the LA loop by providing some form of end-user interface to students or insights to teachers make it harder for educational stakeholders to weigh the benefits against the potential added complexity to their already rich educational ecologies \citep{yan2022scalability}. Although the alignment between MMLA innovations and learning design should be one of the foundations for developing MMLA innovations \citep{cukurova2020promise, ochoa_multimodal_2022}, such alignments are rarely considered or reported in the existing literature, as noted in recent literature reviews \citep{sharma2020multimodal, praharaj2021literature}. This can undermine teacher and student confidence, if they do not understand how the MMLA system aligns with their teaching practices or learning outcomes. 

All of these challenges are hallmarks of emerging HCI infrastructures that must be co-evolved with work practices. This in-the-wild MMLA deployment offered the opportunity to study how both educational and technical stakeholders learnt to work together to address the challenges.%Gaining insights regarding this trade-off between data quality and affordability could benefit educational researchers and practitioners when evaluating their budgets against the type of educational insights they are trying to capture. 

\subsection{Privacy Challenges}
As a research area that benefits from the data collection opportunities enabled by various sensing technologies, the privacy issues surrounding the adoption of MMLA innovations are the focus of critical debate. \citet{crescenzi2020multimodal} emphasised the need to consider the privacy implications of using sensing technologies to generate analytics about children's activity. Such implications have also been identified by students and teachers who have expressed concerns regarding the security of their data \citep{mangaroska2021challenges, kasepalu2021teachers}. These privacy implications of MMLA innovations have been under-investigated in the literature \citep{Alwahaby2022, yan2022scalability, Oviatt2018challenges}. Specifically, while most works published in MMLA  mention that informed consent was obtained from participants, none of the existing works has elaborated on the consenting strategies they adopted, which could contribute valuable insights regarding data security measures for protecting individual privacy and maximising data autonomy (e.g., individuals' autonomy of removing their data from the database) \citep{beardsley2020enhancing}. Additionally, while most of MMLA innovations endeavour to provide dashboards and visualisations for supporting educational practices, privacy issues regarding who has the right to see these visualisations  remain unclear, especially in the contexts of collaborative learning where, in most cases, individuals' personal trace data, even anonymised (e.g., masking students' identity with numbers or colours), could remain identifiable when used for provoking reflections at a group-level, since other students typically have the contextual knowledge to decode anonymised representations \citep{mangaroska2021challenges, Alwahaby2022}. Providing additional empirical evidence on educational stakeholders' perspectives of these privacy-related issues could potential benefit the on-going development of MMLA, and is a particular focus of this study.

\subsection{Ethical Challenges}
Beyond logistical and privcy issues, the potential biases in analytics, and cognitive dissonances that may be caused by the inconsistency between individuals' observations and generated insights, could also undermine the potential benefits of MMLA innovations \citep{ferguson2016guest,Oviatt2018challenges}. Such issues are vital as the accuracy of the existing MMLA-based predictive models and early-warning systems are far from suitable for practical deployment (e.g., rarely above 80\% accuracy), and these models have mostly been developed and evaluated based on relatively small sample sizes (i.e., with n < 50) \citep{yan2022scalability}. These small sample sizes combined with the poor reporting standards found in the existing MMLA literature could also mask potential algorithmic biases that may disadvantage certain minority groups of students as replicating these studies remain difficult without adequately reported methodologies \citep{luzardo2014estimation, yan2022scalability}. Additionally, \citet{Alwahaby2022} also highlighted the significant concerns regarding the need to enhance trust and data transparency within MMLA systems and \citet{yan2021footprints} suggested that more research needs to be done to assess the potential risk of making decisions with incomplete multimodal data.%Additionally, using unsupervised machine learning techniques to cluster and label students may also induce the potential risk of discrimination, where certain labels (e.g., at-risk)  could negatively impact learners' self-esteem and educators' expectations \cite{higgins2002stages}. 
Consequently, understanding the ethical practices of using these analytics is also essential but rarely considered in prior literature \citep{selwyn2019s} and requires the participation of key educational stakeholders such as students and educators \citep{Oviatt2018challenges}.  A large-scale in-the-wild study opens new opportunities to study approaches to these ethical challenges under more authentic conditions than has been reported to date. 

\subsection{Contribution to HCI and Research Question}
Against the literature reviewed above we formulate the following research question (RQ) that guided our study: 

\textit{\textbf{RQ:} What logistical, privacy and ethical challenges emerge from a complex MMLA, in-the-wild study that closes the analytics loop by providing direct feedback to students?}

In addressing this question, the contribution of this paper is a set of lessons learnt regarding how such challenges were, or could have been, addressed in the context of a two-year deployment of a MMLA system in an authentic educational scenario. The implications of this study should assist researchers, developers and designers in making informed decisions about the effective deployment of innovations that involve the use of ubiquitous computing technologies, sensing devices and artificial intelligence (AI) algorithms to augment teaching and learning in physical spaces. 

% Reviews I reckon you already cited in the Scalability paper and which I used in the intro
% \cite{sharma2020multimodal}
% \cite{crescenzi2020multimodal}
% \cite{chua2019technologies}
% \cite{alwahaby2021evidence}

% Note for Jimmie - Other SLR on MMLA reviews to be inlcuded: 
% \cite{noroozi2020multimodal}
% \cite{shankar2018review}
% \cite{praharaj2021literature}
% \cite{mu2020multimodal}
% \cite{yan2022scalability} %of course!
% \cite{chango2022review}

\section{Methods}
\label{sec:methods}
\subsection{Preliminary}
In this section, we introduce each component of MAPSeg (\hyperref[fig2]{Fig.2}) and how MAPSeg can serve as a unified solution to centralized, federated, and test-time UDA (\hyperref[fig:overview]{Fig.1b}). We deploy MAPSeg for domain adaptative 3D segmentation of heterogeneous medical images and it consists of three components: (1) 3D masked multi-scale autoencoding for self-supervised pre-training, (2) 3D masked pseudo-labeling for domain adaptive self-training, and (3) global-local feature collaboration to fuse global and local contexts for the final segmentation task. The hybrid cross-entropy and Dice loss (\hyperref[eq:L_seg]{Eq.1}) is often adopted for regular supervised segmentation training, and we employ it as the basic component of the objective functions for MAPSeg:
\begin{equation}
    \label{eq:L_seg}
    \mathcal{L}_{seg}(\hat{y},y) = -\frac{1}{n}\sum_i\sum_jy_{i,j}\log(\hat{y}_{i,j}) -\frac{2\sum y\hat{y}+\epsilon}{\sum y+\sum \hat{y}+\epsilon}
\end{equation}
where $n$ denotes the number of pixels, $y_{i,j}$ and $\hat{y}_{i,j}$ represent the ground truth label and predicted probability for the $i$th pixel to belong to the $j$th class, and $\epsilon$ is used to prevent zero-division. 

In the following sections, notations are defined as: $x$ and $y$ indicate the original image and label of the randomly sampled local patch; $X$ and $Y$ refer to downsampled global scan and label; the subscripts $s$ and $t$ refer to the source and target domains, respectively; the superscript $M$ indicates the image is masked (\eg, $x_t^M$ refers to a masked local patch from the target domain).

\begin{figure*}
\centering
\includegraphics[width=0.85\linewidth]{./figs/fig2-11.pdf}
\caption{Components of the proposed MAPSeg framework. (a) 3D multi-scale masked autoencoding. (b) 3D masked pseudo labeling in source and target domains. (c) 3D Global-local collaboration.} 
\label{fig2}
\end{figure*}
\subsection{3D Multi-Scale Masked Autoencoder (MAE)}
In this study, we propose a 3D variant of MAE using a 3D CNN backbone (\hyperref[fig2]{Fig.2a}). The detailed configuration can be found in Appendix \cref{sec:archite}. Training is jointly performed on two image sources with identical size ($96^3$ voxels): local patches $x$ randomly sampled from the volumetric scan, and the whole scan downsampled to the same size, denoted as $X$. 
Both $x$ and $X$ are masked before feeding into the MAE: $x$ is divided into non-overlapping 3D sub-patches with size $8^3$, of which 70\% are masked out randomly based on a uniform distribution (\hyperref[fig2]{Fig.2a}); The same procedure is applied to $X$ with patch size $4^3$ since it contains a larger field-of-view (FOV). The masked versions of $x$ and $X$ are denoted as $x^M$ and $X^M$, respectively. We train the MAE encoder and decoder to reconstruct $x/X$ based on $x^M/X^M$ using mean squared error on the masked-out regions as the objective function.

\subsection{3D Masked Pseudo-Labeling (MPL)}
MPL uses a teacher-student framework which is a standard strategy in semi-/self-supervised learning~\cite{grill2020bootstrap,NIPS2017_68053af2} to provide stable pseudo labels on an unlabeled target domain during training. 
After MAE pre-training, we keep the MAE encoder $g$ and append a segmentation decoder $h$ to build the segmentation model $f=h\circ g$ (\hyperref[fig2]{Fig.2b-c}). Given an input image $x_s$ and label $y_s$ from the source domain and an input image $x_t$ from the target domain, the teacher model $f_\theta$ takes as input the target image $x_t$ and generates pseudo labels $f_\theta(x_t)$, with gradient detached. The student model $f_\phi$ is then optimized by minimizing the segmentation loss between the predictions of $x_t^M$/$x_s^M$ and $f_{\theta}(x_t)$/$y_s$, which can be formulated as:  
\begin{equation}
\label{eq:L_mpl}
\mathcal{L}_{MPL} = \mathcal{L}_{Seg}(f_{\phi}(x_t^M),f_{\theta}(x_t))+\beta\mathcal{L}_{Seg}(f_{\phi}(x_s^M),y_s)
\end{equation}
where $\beta$ is the weight of source prediction and set as 0.5. 
The teacher model's parameters $\theta$ are then updated during training via exponential moving average (EMA) based on the student model's parameters $\phi$~\cite{NIPS2017_68053af2}.

\begin{equation}
\label{eq:ema_update}
\theta_{t+1} \gets \alpha \theta_{t} + (1-\alpha)\phi_t, 
\end{equation}
where $t$ and $t+1$ indicate training iterations and $\alpha$ is the EMA update weight. For model initialized from the large-scale MAE pretraining, we set $\alpha$ as 0.999 during the first 1,000 steps and 0.9999 afterwards. For model pretrained on small-scale source and target datasets (\eg, only dozens of scans), we set $\alpha$ as 0.99 during the first 1,000 steps, 0.999 during the next 2,000 steps, and 0.9999 for the remaining training. The teacher model $f_{\theta}$ is initialized with student model's parameters $\phi$ after some warm-up training (\eg, 1,000 iterations) on the source-domain data. 

\subsection{3D Global-Local Collaboration (GLC)}
Directly applying MPL for UDA segmentation with large domain shift (\eg, cross-modality/sequence) may lead to unreliable pseudo-label and disrupt the training. Therefore, we design a GLC module (\hyperref[fig2]{Fig.2c}) to improve pseudo-labeling by leveraging the spatial global-local contextual relations induced by the inherent anatomical distribution prior in medical images. With the image encoder pretrained to extract image features at both local and global levels during multi-scale MAE, we take advantage of the global-local contextual relations by concatenating local and global semantic features in the latent space and make prediction based on the fused features. We differ from previous study~\cite{Chen_2019_CVPR} by only applying GLC on the output of the encoder $g$ instead of all layers to save computation cost and employing a different regularization to prevent segmentation decoder from predicting solely based on local features. 

In GLC, a binary mask $M$ is used to indicate the corresponding location of the local patch $x$ inside the downsampled global volume $X$. The encoder $g$ takes as input $x$ and $X$ and generates the local latent feature $\chi_{loc} = g(x)$ as well as cropped and resized global latent feature $\chi_{glo}=\mathit{upsample}(M \odot g(X))$, where $\odot$ indicates cropping $g(X)$ based on $M$ followed by upsampling to match the spatial size of $\chi_{loc}$. Therefore, segmenting a local patch $x$ can be rewritten as $f(x)=h(\chi_{loc}\oplus\chi_{glo})$, where $\oplus$ is the concatenation along channel dimension (\hyperref[fig2]{Fig.2c}). In addition, $f$ is also trained on downsampled global volume $X$ with $\mathcal{L}_{Seg}(f(X),Y)$), in which the global latent feature $g(X)$ is duplicated and $f(X) = h(g(X)\oplus g(X))$, to prevent model from solely relying on local semantic features and encourage the encoder to extract meaningful semantic features from both local and global levels.

We also add a regularization term between the $\chi_{loc}$ and $\chi_{glo}$ to maintain their similarity following~\cite{Chen_2019_CVPR}. Instead of the $\mathcal{L}_2$ regularization used in~\cite{Chen_2019_CVPR}, we maximize the cosine similarity between the $\chi_{loc}$ and $\chi_{glo}$ as:
\begin{equation}
\mathcal{L}_{cos}(x, X) = 1 - \frac{\chi_{loc}\cdot\chi_{glo}}{\max(\| \chi_{loc} \|_2, \| \chi_{glo} \|_2, \epsilon)}
\end{equation}
where $\epsilon$ is used to prevent zero-division. The loss function for GLC calculated on the source data is formulated as: 
\begin{align}
\label{eq.L_gs}
\mathcal{L}_{GLC}^{S} &= \gamma(\mathcal{L}_{Seg}(f_{\phi}(X_s),Y_s)+\mathcal{L}_{Seg}(f_{\phi}(X_s^M),Y_s))
\nonumber\\
&+\delta(\mathcal{L}_{cos}(x_s, X_s) + \mathcal{L}_{cos}(x_s^M, X_s^M))
\end{align}
where $\gamma$ and $\delta$ are the weights of the auxiliary global loss and cosine similarity, and set as $\gamma=0.05$ and $\delta= 0.025$ in our experiments. Similarly, the GLC loss is also calculated on the target data based on pseudo-label $f_{\theta}(X_t)$ and formulated as:
\begin{align}
\label{eq.L_gt}
\mathcal{L}_{GLC}^{T} &= 2\gamma\mathcal{L}_{Seg}(f_{\phi}(X_t^M),f_{\theta}(X_t)) + 2\delta\mathcal{L}_{cos}(x_t^M, X_t^M)
\end{align}
Therefore, the overall loss function of GLC is:
\begin{align}
\label{eq.L_global}
\mathcal{L}_{GLC} &= \mathcal{L}_{GLC}^{S}+\mathcal{L}_{GLC}^{T}
\end{align}
With the regular fully-supervised segmentation loss on source data $\mathcal{L}_{FSS} = \beta\mathcal{L}_{Seg}(f_{\phi}(x_s),y_s)$, where $\beta$ is defined as in \hyperref[eq:L_mpl]{Eq.2}, the overall objective function $\mathcal{L}$ for centralized UDA is formulated as:
\begin{equation}
\label{eq.L_center}
\mathcal{L} = \mathcal{L}_{FSS}+\mathcal{L}_{MPL}+\mathcal{L}_{GLC}
\end{equation}
It is clear that \hyperref[eq.L_center]{Eq.8} requires centralized and synchronous access to source and target data. In the section \hyperref[sec.fuda]{3.5} and \hyperref[sec.ttuda]{3.6}, we demonstrate how MAPSeg can be adapted to federated (decentralized and synchronous access to data) and test-time (decentralized and asynchronous access to data) UDA scenarios. 

\subsection{Extension to Federated UDA}
\label{sec.fuda}
In reality, labeled source-domain data and unlabeled target-domain data are often collected at different sites. We consider a practical scenario where a server (\eg a major hospital) hosts potentially large amount of both labeled and unlabeled scans, and distributed clients (\eg clinics or imaging sites) possess only unlabeled images. This is an under-explored scenario as FL typically assumes either fully or partially labeled data from all clients. We extend MAPSeg to solve this federated multi-target UDA problem according to the details in Algorithm 1 of Appendix \cref{sec:recipe}. Specifically, the server updates the student model $f_\phi$ by minimizing the loss for the labeled source-domain data $D_S$:
\begin{align}
    \mathcal{L}_s 
    &= \beta(\mathcal{L}_{seg}(f_\phi(x_s), y_s)+\mathcal{L}_{seg}(f_\phi(x_s^M), y_s)) \nonumber\\
    &+\gamma(\mathcal{L}_{seg}(f_\phi(X_s), Y_s)+\mathcal{L}_{seg}(f_\phi(X_s^M), Y_s)) \nonumber\\
    &+ \delta(\mathcal{L}_{cos}(x_s, X_s) + \mathcal{L}_{cos}(x_s^M, X_s^M)) \label{eq:loss_server}
\end{align}
The clients update the student model $f_\phi$ by minimizing the loss for its own unlabeled target-domain data $D_T^k$:
\begin{align}
    \mathcal{L}_u
    &= \beta(\mathcal{L}_{seg}(f_\phi(x_t^M), f_\theta(x_t))+\mathcal{L}_{seg}(f_\phi(x_t), f_\theta(x_t))) \nonumber\\
    &+ \gamma(\mathcal{L}_{seg}(f_\phi(X_t^M), f_\theta(X_t))+\mathcal{L}_{seg}(f_\phi(X_t), f_\theta(X_t))) \nonumber\\
    &+ \delta(\mathcal{L}_{cos}(x_t, X_t) + \mathcal{L}_{cos}(x_t^M, X_t^M)) \label{eq:loss_client}
\end{align}
Comparing to the centralized UDA loss (\hyperref[eq.L_center]{Eq.8}), we decompose it into two components: fully supervised loss for server training (\hyperref[eq:loss_server]{Eq.9}) and self-supervised loss for client updates (\hyperref[eq:loss_client]{Eq.10}), which avoids the need for centralized data. After each local update, each client sends the EMA teacher model parameters $\theta$ to the server for aggregation following typical federated averaging\cite{mcmahan2017communication}.

\subsection{Extension to Test-time UDA}
\label{sec.ttuda}
Test-time UDA often involves two separate stages of training, including the source-only training at one center and the target-only finetuning at another site. In the federated UDA setting, \hyperref[eq:loss_server]{Eq.9} and \hyperref[eq:loss_client]{Eq.10} are jointly used to update the server model through synchronous federated averaging after each round. We can further ease the constraint of synchronous communication between source and target sites by training $f_\phi$ on the source data using \hyperref[eq:loss_server]{Eq.9} for some (\eg 1,000) warm-up steps before distributing the model parameters $\phi$ to the target site for initializing the teacher model $f_\theta$. On the target site, $f_\theta$ provides stable pseudo-labels to guide the self-supervised training with \hyperref[eq:loss_client]{Eq.10} and is updated by the EMA of $\phi$ following \hyperref[eq:ema_update]{Eq.3}. We find that in this asynchronous setting MAPSeg still performs well on the target-domain data, albeit with a minor performance tradeoff on the source-domain data (see \hyperref[tab:testtime]{Tab.3}).

%-------------------------------------------------------------------------
\subsection{Implementation Details} \label{section:2.1}

\noindent\textbf{Model architecture and implementation.} We implement the encoder backbone $g$ using 3D-ResNet-like CNN. The segmentation decoder $h$ is adapted from DeepLabV3~\cite{chen2017rethinking}. The framework is implemented using PyTorch. More details of the model and the training procedure are provided in Appendix \cref{sec:archite} and \cref{sec:recipe}. 

\noindent\textbf{Selecting the best model.} For choosing the best model during training, some studies choose to train for fixed iterations and use the last checkpoint. On the other hand, some of the previous UDA studies~\cite{8988158,Chen_Dou_Chen_Qin_Heng_2019} face a dilemma in selecting the best model during training by validating against a hold-out portion of target-domain labels, which is unrealistic as UDA assumes full absence of target labels. We demonstrate that MPL not only provides an efficient pathway to domain adaptative segmentation but also serves as an indicator of how well the model is being adapted to the target domain. We validate the model after each epoch and the best model is selected based on the score: 
$\mathit{Score}=\mathit{Dice}_{Src}-0.5\times\overline{\mathcal{L}_{Seg}}(f_{\phi}(x_t^M),f_{\theta}(x_t))$, where $\mathit{Dice}_{Src}$ is the Dice score on source-domain validation set and $\overline{\mathcal{L}_{Seg}}(f_{\phi}(x_t^M),f_{\theta}(x_t))$ is the mean of $\mathcal{L}_{Seg}(f_{\phi}(x_t^M),f_{\theta}(x_t))$ during the last training epoch. From \hyperref[eq:L_seg]{Eq.1}, it is clear that $ \lim_{\hat{y}\to y} \mathcal{L}_{seg}(\hat{y},y)=-1$, therefore, $Score$ has an upper bound of $1.5$. We demonstrate in \hyperref[tab:cardiac]{Tab.4} that the difference between validation using target labels versus $Score$ is acceptable (81.2 vs. 80.3). Even without accessing target labels for validation, MAPSeg still surpasses the previous SOTA results that use target labels for validation. It is worth noting that we only use target labels for validation in \hyperref[tab:cardiac]{Tab.4} for a fair comparison with previously reported results; other results presented use $Score$ for validation by default. For federated and test-time UDA, $\mathit{Score} = -\overline{\mathcal{L}_{Seg}}(f_{\phi}(x_t^M),f_{\theta}(x_t))$.

%%%%%%%%%%%%%%%%%%%%%%%%%%%%%%%%%%%%%%%%%%%%%%%
%%%%%%%        4. Results         %%%%%%%
%%%%%%%%%%%%%%%%%%%%%%%%%%%%%%%%%%%%%%%%%%%%%%%

\section{Results}
\label{sec:results}

\subsection{MOS prediction results}
\label{subsec:mos_results}
We first evaluate our MOS-prediction performance in comparison with other approaches. In particular, we compare against NISQA~\cite{mittag2019non}, which we modified to estimate human-accessed MOS. Originally, they estimate perceptual objective listening quality assessment (POLQA)~\cite{beerends2013perceptual} scores using a CNN and BLSTM architecture. We also compare against the PMOS model proposed in~\cite{dong2020pyramid}, which is identical in structure to our PMOS model. Finally, we include our proposed SE+PMOS approach~\cite{nayem2021incorporating} (no joint training), where our PMOS model is held fixed while the SE model is training using the embeddings from the PMOS encoder. 

We use four metrics to evaluate MOS-estimation performance: mean absolute error (MAE), epsilon insensitive root mean squared error (RMSE)~\cite{rec2012p}, Pearson’s correlation coefficient $\gamma$ (PCC), and Spearman’s rank correlation coefficient $\rho$ (SRCC). 

%    Later, both models are jointly-trained for fine tuning. Our proposed PMOS model is similar of \cite{nayem2021incorporating}, however, SE models are different in structure.

%%%%%%%%%%%%%%%%%%%%%%%%%%%%%%%%%%%%%%%%%%%%%%%%%%%%
% Table 1, MOS results
%%%%%%%%%%%%%%%%%%%%%%%%%%%%%%%%%%%%%%%%%%%%%%%%%%%%
\begin{table}[t!]

\centering
\caption{Performance comparison with MOS prediction models {comparing against the ground truth MOS obtained from human subjects}. Best results are shown in \textbf{bold}.}
\label{tab:mos_results}
% \vspace{-0.5em}
\resizebox{\columnwidth}{!}{%
\begin{tabular}{| l | c c c c | }
\cline{2-5}
   \multicolumn{1}{c|}{}         & {MAE}$\downarrow$ & {RMSE}$\downarrow$ & {PCC ($\gamma$)}$\downarrow$ & {SRCC ($\rho$)}$\downarrow$ \\ \hline
   
NISQA~\cite{mittag2019non}    & 0.62 ($\pm$0.18)        & 0.7 ($\pm$0.16)      & 0.71 ($\pm$0.14)           & 0.79 ($\pm$0.15)            \\
PMOS~\cite{dong2020pyramid}                      & 0.51 ($\pm$0.15)         & 0.57 ($\pm$0.12)          & 0.88 ($\pm$0.17)           & 0.88 ($\pm$0.14)           \\
SE+PMOS~\cite{nayem2021incorporating}                     & \textbf{0.45} ($\pm$0.08) & \textbf{0.52} ($\pm$0.09) & \textbf{0.9} ($\pm$0.12) & \textbf{0.91} ($\pm$0.1)           \\
Proposed                     & \textbf{0.45} ($\pm$0.08) & \textbf{0.52} ($\pm$0.09) & \textbf{0.9} ($\pm$0.12) & \textbf{0.91} ($\pm$0.1)         \\
\hline
\end{tabular}
}
% \vspace{-2em}
\end{table}

Table~\ref{tab:mos_results} shows the results, where our proposed approach and SE+PMOS clearly outperform the other MOS prediction models according to all metrics. MAE is minimized by $0.6$ compared to the original PMOS~\cite{dong2020pyramid} approach. There is also a $0.05$ reduction in RMSE. This justifies our proposed approach that combines MOS estimation and speech enhancement tasks. Note, however, that similar results are obtained for our proposed approach and the SE+PMOS approach, which suggests that joint training (e.g., fine tuning) may help speech enhancement more than MOS prediction.  




\subsection{Speech enhancement model}
\label{subsec:se_results}
%%%%%%%%%%%%%%%%%%%%%%%%%%%%%%%%%%%%%%%%%%%%%%%%%%%%
% Table 2, SE comparison results on COSINE & VOiCES
%%%%%%%%%%%%%%%%%%%%%%%%%%%%%%%%%%%%%%%%%%%%%%%%%%%%

% Please add the following required packages to your document preamble:
% \usepackage{multirow}
% \usepackage[table,xcdraw]{xcolor}
% If you use beamer only pass "xcolor=table" option, i.e. \documentclass[xcolor=table]{beamer}
\begin{table*}[t!]
\centering
\caption{Average results of the speech enhancement models in different performance metrics. Best results are shown in \textbf{bold}.}
\label{tab:results_cosineVoices}
\resizebox{\linewidth}{!}{%
\begin{tabular}{ | l | l | c c c c | c c c c | }
\cline{3-10}
\multicolumn{1}{l}{\multirow{2}{*}{}} &                    & \multicolumn{4}{ c |}{{COSINE}}                             & \multicolumn{4}{ c |}{{VOiCES}} 
\\ \hline
\multicolumn{1}{|l|}{{models}}                 & \multicolumn{1}{c|}{{loss func.}} & {PESQ}$\uparrow$ & {SI-SDR}$\uparrow$ & {ESTOI}$\uparrow$ & {MOS-LQO}$\uparrow$ & {PESQ}$\uparrow$ & {SI-SDR}$\uparrow$ & {ESTOI}$\uparrow$ & {MOS-LQO}$\uparrow$ \\ \hline
\multicolumn{1}{|l|}{{Mixture}}                                        & {-}                                                       & {1.46} & {0.53}   & {0.62}  & {4.04}    & {1.26} & {-1.3}   & {0.48}  & {2.74}    \\ \hline
\multicolumn{1}{|l|}{}                                                        & mse                                                              & 2.68          & 2.8             & 0.8            & 3.2              & 2.3           & 1.2             & 0.69           & 3.5              \\ 
\multicolumn{1}{|l|}{}                                                        & mos~\cite{fu2019learning}                                                              & 2.8           & 3.8             & 0.82           & 4.2              & 2.37          & 1.66            & 0.74           & 5.3              \\ 
\multicolumn{1}{|l|}{}                                                        & mse+sa                                                           & 2.72          & 3.1             & 0.82           & 4                & 2.35          & 1.6             & 0.7            & 3.8              \\ 
\multicolumn{1}{|l|}{}                                                        & mos+sa                                                           & 2.89          & 4.1             & 0.85           & 4.4              & 2.42          & 1.72            & 0.77           & 5.7              \\ 
\multicolumn{1}{|l|}{\multirow{-5}{*}{SE}}                                    & sdr~\cite{kawanaka2020stable}                                                              & 2.7           & 4.5             & 0.82           & 3.4                & 2.32          & 2.01            & 0.72           & 3              \\ \hline
\multicolumn{1}{|l|}{{ }}                                 & mse                                                              & 3.1           & 4               & 0.85           & 4.2              & 2.48          & 1.8             & 0.8            & 6                \\ 
\multicolumn{1}{|l|}{{}}                                 & mse+sa                                                           & 3.19          & 4.6             & 0.93           & 4.8              & 2.54          & 2.08            & 0.86           & 6.3              \\  
\multicolumn{1}{|l|}{\multirow{-3}{*}{SE+PMOS~\cite{nayem2021incorporating}}}        & mse+sa+mos                                                       & 3.19          & 4.5             & 0.92           & \textbf{5.1}     & 2.53          & 2.06            & 0.84           & \textbf{6.5}     \\ \hline
\multicolumn{1}{|l|}{}                                                        & pesq                                                             & \textbf{3.28} & 4.4             & 0.9            & 5                & \textbf{2.67} & 2.01            & 0.83           & 6.1              \\ 
\multicolumn{1}{|l|}{\multirow{-2}{*}{MetricGAN~\cite{fu2019metricGAN}} }                             & stoi                                                             & 3.19          & 4.3             & \textbf{0.94}  & 4.8              & 2.5           & 2               & \textbf{0.87}  & 5.8              \\ \hline
\multicolumn{1}{|l|}{SSEMS~\cite{zezario2019specialized}}                                                   & qnet ($\phi=0dB$)                                                       & 2.85          & 2.99            & 0.83           & 3                & 2.4           & 1.8             & 0.7            & 2.8              \\ \hline
\multicolumn{1}{|l|}{{Chi++\textsubscript{fQSM,bS}~\cite{nayem2021towards}}}                     &    dc+cls+sa                                                              & 2.9           & 3.3             & 0.84           & 3.4              & 2.44          & 1.78            & 0.7            & 3                \\ \hline
\multicolumn{1}{|l|}{}                                 & mse+sa                                                           & 3.25          & 4.8             & \textbf{0.94}  & 4.75             & 2.64          & 2.1             & \textbf{0.87}  & 6.2              \\ 
\multicolumn{1}{|l|}{\multirow{-2}{*}{Proposed}} & mse+sa+mos                                                       & 3.25          & \textbf{4.82}   & \textbf{0.94}  & 5.04             & 2.64          & \textbf{2.13}   & \textbf{0.87}  & 6.47             \\ \hline
\end{tabular}
}
\end{table*}
For speech enhancement, we compare against a baseline approach without an attention mechanism \cite{graves2013speech}. We denote this baseline approach as SE. Five separate loss functions are applied to optimize this approach, and they are MSE, MSE plus signal approximation, MOS, signal approximation with MOS, and SDR. To compute the MOS loss function, we utilize the SE loss function from \cite{fu2019learning} which leverages objective-MOS (oMOS) ratings learned from a speech assessment model~\cite{fu2018quality}. SDR~\cite{kawanaka2020stable} loss functions are proposed in literature previously with different enhancement architectures. For the SDR loss function, the SE model is optimized using the following cost function:
\begin{align}
    \mathcal{L}_{SDR} = \sum_{n=1}^N \mathcal{K}_{\theta}  \Big( 10 \log \frac{\Vert s^n\Vert^2}{\Vert s^n-\hat{s}^n\Vert^2} \Big)
\end{align}
where $\mathcal{K}_\theta(a)=\theta\cdot \tanh(\frac{a}{\theta})$, $\theta$ is a clipping parameter, $N$ is the mini-batch size, and $s^n$ and $\hat{s}^n$ are the n\textsuperscript{th} sample of the clean and estimated speech signal in time. We use $\theta=20$ in our training. We also compare against a generative adversarial network (GAN) approach that individually optimizes with PESQ and STOI~\cite{fu2019metricGAN}. We denote this model as MetricGAN. 
% They estimate the IRM conditioned on continuous space of the discriminator label based on either PESQ or STOI target label. 
They estimate the IRM for a speech mixture conditioned on a GAN discriminator that outputs evaluation scores in continuous space (i.e. scores between 0 and 1) based on either normalized PESQ or STOI target metrics. 
We compare our model with the ensemble-based Specialized Speech Enhancement Model Selection (SSEMS) approach~\cite{zezario2019specialized} that uses Quality-Net~\cite{fu2018quality} as its objective function in a black-box manner. Quality-Net is an oMOS approach that estimates the Perceptual Evaluation of Speech Quality (PESQ) score. The SSEMS approach uses an ensemble of enhancement models, each trained on audio at specific SNRs and speaker genders. During inference, it selects the output with the highest PESQ score. SSEMS uses a SNR threshold of $20$ dB, while we use a threshold of $0$ dB for balanced training and better performance. Additionally, we conduct a comparison with our initial approach that integrates MOS embeddings in speech enhancement, as presented in \cite{nayem2021incorporating}. This model is referred to as SE+PMOS, and it does not involve joint training or the QSM language model. We evaluate SE+PMOS with varying combinations of loss functions. %We compare against a quantized speech enhancement model which utilizes a spectral language model~\cite{nayem2021towards}. This model is motivated from chimera++~\cite{wang2018alternative} in structure with BLSTM layers and deep clustering (dc) loss.
%Traditional chimera++ model estimates a phase-sensitive mask which has been applied in the task of speech enhancement in non-speech noisy conditions with multi-talker speech~\cite{wichern2019wham, yang2019improved}. However, in \cite{nayem2021incorporating}, they estimate quantized speech signal, not mask; they use cross-entropy classification (cls) loss, and signal approximation loss altogether. They report best results using per-frequency quantized spectral model (fQSM) as language model for beam search (bS) with beam size $100$. We use this model as our comparison model denoting as Chi++\textsubscript{fQSM,bS}. 
All models are trained using the experimental setup that is previously mentioned. We modify the comparison models using the code provided by the original authors.

We assess speech enhancement performance using PESQ~\cite{rix2001perceptual}, scale-invariant SDR (SI-SDR)~\cite{le2019sdr}, and extended STOI (ESTOI)~\cite{jensen2016algorithm}. In the absence of actual human quality objective, we measure the predicted MOS score of the enhanced speech, using our proposed PMOS model, since we aim to improve human-assessed speech quality. We denote this metric as MOS listener quality objective (MOS-LQO). Table~\ref{tab:results_cosineVoices} shows the average results of the different enhancement models, according to each of the performance metrics on COSINE and VOiCES dataset. As the scores of the unprocessed mixtures show, the VOiCES corpus is  more challenging than the COSINE corpus. 
With the baseline SE model, we experiment with 5 different combination of loss functions. Using the MSE loss only in SE:mse, we see improvements in objective scores, except with MOS-LQO for the COSINE data. Then we apply a MOS loss $\mathcal{L}_{mos}$ as the sole objective criterion, as proposed in \cite{fu2019learning}. Our experimental results show that this approach results in an overall improvement of $1.4$ in MOS-LQO compared to SE:mse. %We apply MOS-LQO scores of enhanced speech to calculate MOS loss $\mathcal{L}_{mos}$ as the only objective criteria as proposed in \cite{fu2019learning}, which gives improves MOS-LQO by $1.4$ overall compared with SE:mse. 
Then we separately combine the signal approximation loss with the mse loss and MOS loss (e.g., mse+sa and mos+sa). In PESQ, we gain an average of $\ge0.05$ and $\ge0.07$ compared to the models that use only the MSE loss and only the MOS loss, respectively. Furthermore, the model trained with the mos+sa loss function achieves the highest MOS-LQO score of $4.4$ and $5.7$ among all five loss functions tested with the SE model in COSINE and VOiCES dataset, respectively. This result is on average $1.15$ MOS-LQO higher than that obtained with the mse+sa loss function. These scores suggest that $\mathcal{L}_{mse}$ and $\mathcal{L}_{sa}$ maximize the overall speech intelligibility, whereas $\mathcal{L}_{mos}$ guides the model towards perceptual speech quality. Note that in all these $\mathcal{L}_{mos}$ calculations, we use a separately trained PMOS model's output without joint learning.
Lastly, we apply the SDR loss function as proposed in \cite{kawanaka2020stable}, which is used as the pre-training stage for model training. We observe an average gain of $0.9$ in SI-SDR, however, it yields a poor score according to other metrics, especially a $0.7$ loss in MOS-LQO compared to SE with mse and sa loss terms. 

SE+PMOS is separately investigated with 3 combinations of loss functions, i.e. mse, mse+sa, and mse+sa+mos. Compared with SE models, SE+PMOS with mse loss achieves $0.9$ SI-SDR and $1.75$ MOS-LQO improvements on average, which shows the benefit of incorporating the PMOS model. The SE+PMOS:mse+sa model improves the performance further with an average of $0.14$ ESTOI gain over the SE:mse+sa model. The inclusion of the mos loss gives the best MOS-LQO scores of $5.1$ and $6.5$ over all the comparison models in noisy and reverberant conditions, respectively.

%%%%%%%%%%%%%%%%%%%%%%%%%%%%%%%%%%%%%%%%%%%%%%%%%%%%
% Table 3, SE comparison test results on CHiME 5+4 
%%%%%%%%%%%%%%%%%%%%%%%%%%%%%%%%%%%%%%%%%%%%%%%%%%%%

% Please add the following required packages to your document preamble:
% \usepackage{multirow}
\begin{table*}[t!]
\centering
\caption{Average testing results of the speech enhancement models on CHiME-5 and CHiME-4 datasets. Best results are shown in \textbf{bold}.}
\label{tab:results_chime}
\resizebox{\linewidth}{!}{%
\begin{tabular}{| l | l | c c c c c | c c c c c |}
\cline{3-12}
\multicolumn{1}{l}{\multirow{2}{*}{}} &                    & \multicolumn{5}{ c |}{{CHiME-5}}                             & \multicolumn{5}{ c |}{{CHiME-4}} 
\\ \hline
\multicolumn{1}{|l|}{models}                          & \multicolumn{1}{c|}{loss func.} & PESQ$\uparrow$          & SI-SDR$\uparrow$       & ESTOI$\uparrow$         & MOS-LQO$\uparrow$      & WER\%$\downarrow$         & PESQ$\uparrow$          & SI-SDR$\uparrow$        & ESTOI$\uparrow$         & MOS-LQO$\uparrow$      & WER\%$\downarrow$         \\ \hline
\multicolumn{1}{|l|}{Mixture}                & -                               & 1.7           & 2.4          & 0.52          & 3.8          & 152.1         & 1.96          & 2.86          & 0.6           & {4.6} & {33.7} \\ \hline
\multicolumn{1}{|l|}{SE}                              & mos+sa                          & {2.25} & {3.9} & {0.62} & {4}   & {96.4} & {2.32} & {5.22} & {0.63} & {5}   & {25.6} \\ \hline
\multicolumn{1}{|l|}{SE+PMOS}                         & mse+sa+mos                      & 2.37          & 6.1          & 0.67          & 4.4          & 84.5          & 2.45          & 7.6           & 0.7           & 5.8          & 22.6          \\ \hline
\multicolumn{1}{|l|}{\multirow{2}{*}{MetricGAN}}      & pesq                            & \textbf{2.44} & {6.3} & {0.65} & {4.1} & {94.8} & \textbf{2.51} & {7}    & {0.68} & {5.3} & {19.7} \\ 
\multicolumn{1}{|l|}{}                                & stoi                            & 2.39          & 6.2          & \textbf{0.71} & 4.1          & 91.3          & 2.45          & {6.45} & \textbf{0.73} & 5.6          & 21.5          \\ \hline
\multicolumn{1}{|l|}{\multirow{2}{*}{Proposed}} & mse+sa                          & 2.41          & 7.1          & {0.68} & 4.7          & \textbf{78.3} & 2.5           & {7.9}  & 0.72          & 5.76         & \textbf{18.1} \\ 
\multicolumn{1}{|l|}{}                                & mse+sa+mos                      & 2.41          & \textbf{7.3} & {0.68} & \textbf{4.9} & 79.4          & {2.5}  & \textbf{8.61} & \textbf{0.73} & \textbf{6}   & 18.9          \\ \hline
\end{tabular}}
\end{table*}
MetricGAN optimizes PESQ or STOI, therefore, it outperforms other comparison models in terms of PESQ and ESTOI, although the scores for the SE+PMOS approaches are higher according to the other evaluation metrics even though these metrics are not leveraged during training. 
SSEMS yields the lowest scores across all metrics compared with SE+PMOS and MetricGAN approaches, though we do parameter tuning for this model.
Chi++\textsubscript{fQSM,bS} estimates quantized speech, and the results show that it affects the traditional objective functions. This performs poorly compared with the SE+PMOS and MetricGAN approaches, however, on average, it outperforms SSEMS in all criteria, and the SE models in terms of PESQ. With the MOS-LQO criteria, it fails to produce good scores. This points out the importance of incorporating perceptual features during enhancement, which Chi++\textsubscript{fQSM,bS} clearly lacks.

We calculate the performance of our proposed model using two combinations of loss functions. 
Using only mse and sa loss terms, we achieve the highest ESTOI scores for both corpora, though these results are nearly identical to the model trained with all three loss terms. Using $\mathcal{L}$ (eq:\ref{eq:loss}) in our proposed model, we obtain the highest SI-SDR scores while maintaining similar PESQ and ESTOI performance as compared to the best-performing model. Specifically, our proposed model achieves the highest ESTOI score and an average PESQ score that is only $0.03$ less than that of the best performing MetricGAN:pesq model.
Contrasting with the Chi++\textsubscript{fQSM,bS} model, which uses spectral language model to estimate quantized speech, our proposed approach outperforms the quantized model according to all metrics, which proves the significance of joint learning.% to direct speech enhancement model towards perceptually better speech using a speech quality assessment model.
When comparing MOS-LQO scores, our proposed:mse+sa+mos model achieves better scores than the other models except the SE+PMOS:mse+sa+mos model with an average of only $0.05$ declination. Thus, the inclusion of a spectral language model helps the model proposed (e.g., mse+sa+mos) to estimate better quality speech according to the overall evaluation criteria. 
It is important to note that our proposed approach performs best according to SI-SDR in both noisy and reverberant environments, where this metric is not used by any of the approaches during optimization.  

We further examine our approaches using completely unseen corpora. We test models with the CHiME-5 and CHiME-4 corpora where the models are trained from the COSINE dataset according to the system setup mentioned in section~\ref{subsec:setup}. Table~\ref{tab:results_chime} shows the performance evaluated according to PESQ, SI-SDR, ESTOI, MOS-LQO, and word error rate (WER). To calculate WER, we use the conventional ASR baseline that is provided with CHiME-5 and CHiME-4 dataset. We investigate WER with both GMM based ASR and end-to-end ASR, however, we find that the end-to-end approach results in a higher error compared to the GMM baseline. This might happen due to larger data requirements of the end-to-end ASR system as mentioned in \cite{barker2018fifth}. Therefore, we use the GMM ASR approach to compare the WER performance of the enhancement models.
From the scores of mixtures, we find that CHiME-5 is more challenging than CHiME-4 with a $118.8\%$ higher WER and a $0.46$ lower SI-SDR. Our proposed approach yields the best MOS-LQO scores with $4.9$ with CHiME-5 and $6$ with CHiME-4 data. The proposed mse+sa model results in the lowest WER of $78.3$ and $18.1$ using CHiME-5 and CHiME-4, respectively. Note that the WER of the GMM baseline ASR for the CHiME-5 challenge is $72.8$ in binaural and $91.7$ in single array conditions. Here our approaches enhance monaural speech, a more challenging condition. Our proposed approach outperforms other comparison models in terms of SI-SDR with a $5.29$ average improvement compared to others. According to PESQ and ESTOI metrics, MetricGAN variants give the best performace, however, proposed model's performance is $0.02$  and $ 0.015$ lower according to PESQ and ESTOI, respectively, for the best performing MetricGAN models. Hence, our proposed approach is effective on out-of-vocabulary scenario trained by a comparable dataset.


% \nayem{*** Possibly add graphs of evaluation metrics vs SNRs.}

%%%%%%%%%%%%%%%%%%%%%%%%%%%%%%%%%%%%%%%%%%%%%%%%%%%%
% Table 3, DNSMOS results
%%%%%%%%%%%%%%%%%%%%%%%%%%%%%%%%%%%%%%%%%%%%%%%%%%%%
% \begin{table}[thb!]

% \centering
% \caption{Average MOS ratings of the speech enhancement modes on CHiME-4 and CHiME-5 datasets using DNSMOS P.835~\cite{reddy2022dnsmos}. Best results are shown in \textbf{bold}.}
% \label{tab:dnsmos_results}
% % \vspace{-0.5em}
% \resizebox{\columnwidth}{!}{%
% \begin{tabular}{| l | c c | }
% \cline{2-3}
%   \multicolumn{1}{c|}{}         & {CHiME-4} & {CHiME-5} \\ \hline
   
% Mixture   & 1.54 ($\pm$0.85)         & 1.3 ($\pm$1.1)                \\
% PMOS+SE                      & 4.28 ($\pm$0.9)       & 3.67 ($\pm$1.3)\\
% MetricGAN                    & 4.26 ($\pm$0.87) & 3.5 ($\pm$1.34)          \\
% Proposed                     & \textbf{4.32} ($\pm$0.8)& \textbf{3.8} ($\pm$1.41)            \\ \hline
% Clean                     & 4.67 ($\pm$1.2) & -      \\
% \hline
% \end{tabular}
% }
% % \vspace{-2em}
% \end{table}

%%%%%%%%%%%%%%%%%%%%%%%%%%%%%%%%%%%%%%%%%%%%%%%%%%%%
% Fig 3, DNSMOS results plot
%%%%%%%%%%%%%%%%%%%%%%%%%%%%%%%%%%%%%%%%%%%%%%%%%%%%

\begin{figure}[b!]
    \centering
\begin{tikzpicture}
	\begin{axis}[
	    cycle list/Dark2-4,
		boxplot/draw direction = y,
		boxplot/box extend=0.8,
% 		x=3em,
% 		x axis line style = {opacity=0.6},
		axis x line* = bottom,
		axis y line = left,
		enlarge y limits,
		ymajorgrids,
		xtick = {1, 2, 3, 4, 5, 6, 7, 8},
		xticklabel style = {align=center, font=\small, rotate=60, alias={xtick-\ticknum}},
		xticklabels = {Mixture, SE+PMOS, MetricGAN, Proposed, Mixture, SE+PMOS, MetricGAN, Proposed},
% 		xtick style = {draw=none}, % Hide tick line
		ylabel = {MOS},
		ytick = {1, 2, 3, 4, 5},
	]
	
	\addplot+[
        boxplot prepared={
        lower whisker=1, lower quartile=1.45,
        median=1.74,
        upper quartile=2.5, upper whisker=4.05, }, fill, draw=black]
        coordinates {}
        node[above, color=black] at
        (boxplot box cs: \boxplotvalue{median},.5)
        {\scriptsize \pgfmathprintnumber{\boxplotvalue{median}}};
    \addplot+[
        boxplot prepared={
        lower whisker=1.38, lower quartile=1.84,
        median=2.28,
        upper quartile=3.1, upper whisker=4.3, }, fill, draw=black]
        coordinates {}
        node[above, color=black] at
        (boxplot box cs: \boxplotvalue{median},.5)
        {\scriptsize \pgfmathprintnumber{\boxplotvalue{median}}};
    \addplot+[
        boxplot prepared={
        lower whisker=1.3, lower quartile=1.75,
        median=2.13,
        upper quartile=3.2, upper whisker=4.1, }, fill, draw=black]
        coordinates {}
        node[above, color=black] at
        (boxplot box cs: \boxplotvalue{median},.5)
        {\scriptsize \pgfmathprintnumber{\boxplotvalue{median}}};
    \addplot+[
        boxplot prepared={
        lower whisker=1.4, lower quartile=1.9,
        median=2.46,
        upper quartile=3.16, upper whisker=4.34, }, fill, draw=black]
        coordinates {}
        node[above, color=black] at
        (boxplot box cs: \boxplotvalue{median},.5)
        {\scriptsize \pgfmathprintnumber{\boxplotvalue{median}}};
        
    \addplot+[
        boxplot prepared={
        lower whisker=1.0, lower quartile=1.35,
        median=1.64,
        upper quartile=2.39, upper whisker=4.18, }, fill, draw=black]
        coordinates {}
        node[above, color=black] at
        (boxplot box cs: \boxplotvalue{median},.5)
        {\scriptsize \pgfmathprintnumber{\boxplotvalue{median}}};
    \addplot+[
        boxplot prepared={
        lower whisker=1.31, lower quartile=1.8,
        median=2.18,
        upper quartile=2.76, upper whisker=4.24, }, fill, draw=black]
        coordinates {}
        node[above, color=black] at
        (boxplot box cs: \boxplotvalue{median},.5)
        {\scriptsize \pgfmathprintnumber{\boxplotvalue{median}}};
    \addplot+[
        boxplot prepared={
        lower whisker=1.26, lower quartile=1.71,
        median=2.06,
        upper quartile=3.17, upper whisker=4.32, }, fill, draw=black]
        coordinates {}
        node[above, color=black] at
        (boxplot box cs: \boxplotvalue{median},.5)
        {\scriptsize \pgfmathprintnumber{\boxplotvalue{median}}};
    \addplot+[
        boxplot prepared={
        lower whisker=1.34, lower quartile=1.85,
        median=2.25,
        upper quartile=3.07, upper whisker=4.48, }, fill, draw=black]
        coordinates {}
        node[above, color=black] at
        (boxplot box cs: \boxplotvalue{median},.5)
        {\scriptsize \pgfmathprintnumber{\boxplotvalue{median}}};
        
	\end{axis}
	
	\path (0,0) coordinate (P);
    \draw [thick,decoration={brace,mirror,raise=5em},decorate] (xtick-0|-P) -- (xtick-3.5|-P) 
        node[midway,yshift=-6em]{CHiME-4};
    \draw [thick,decoration={brace,mirror,raise=5em},decorate] (xtick-4|-P) -- (xtick-7.5|-P) 
        node[midway,yshift=-6em]{CHiME-5};

    % \node[text width=3cm] at (1.54,0.5) 
    % {\scriptsize 1.54};

\end{tikzpicture}

\caption{MOS ratings of the speech enhancement modes on CHiME-4 and CHiME-5 datasets using DNSMOS P.835.}
    % \vspace{-2em}
\label{fig:dnsmos_results}
    % \vspace{-0.4cm}
\end{figure}

\subsection{Perceptual quality evaluation}
\label{subsec:dnsmos}

We finally evaluate our model using P.835 metric~\cite{reddy2022dnsmos} to measure perceptual quality. We calculate the DNSMOS score on a scale of $[1-5]$ ($1$ = worst, $5$ = best) for the mixture, PMOS+SE, MetricGAN, and our proposed models using the CHiME-4~\cite{vincent2017analysis} and CHiME-5~\cite{barker2018fifth} datasets (simulated and real-recording). Figure~\ref{fig:dnsmos_results} shows the scores. With CHiME-4, the original mixture scores range from $1.45$ to $2.5$ with a median of $1.74$. Our proposed model achieves a median MOS of $2.46$, which is higher than the others. Fon CHiME-5, the original mixture scores range from $1.0$ to $4.18$. Our proposed model outperforms the others with a median of $2.25$. Our proposed model and PMOS+SE have smaller standard deviations compared to MetricGAN. Overall, our proposed model improves noisy speech in both the acoustic and perceptual aspects. 




% \subsection{Listening results}
% \label{subsec:listening_results}

% We conduct an IRB-approved listening study using Amazon Mechanical Turk to conceive the perceptual quality of enhanced speech assessed by normal-hearing listeners. 

% This study follows the design structure of \cite{nayem2021towards} and figure~\ref{fig:survey} shows the actual listener study interface of a single question. The study is conducted as follows, the participant will listen to two audio signals, one is enhanced and the other is clean audio as reference.  Then they provide a preference score using a Likert scale. The scale ranges from $-3$ to $+3$, where $-3$ refers to a strong preference towards the first signal, $+3$ refers to a strong preference towards the second signal, and $0$ refers to no preference. Before providing a score, the participant can listen to the signals as many as times they like, where the scores are not limited to integer values. The two signals are randomly selected, and the participant listens to different audio clips in each question. The audio clips are chosen from the CHiME-5 and CHiME-4 corpus spoken by both males and females in equal proportion. Prior to actual survey questions, each participants has to pass eligibility test and make themselves familiar with the upcoming study session by going through a practice session. The structure of this practice session is similar to the actual study, however, speakers' voice and audio clips which participants hear in practice session are not used in the actual study. A tentative feedback is provided in the practice session to give a guideline to the participants, however, to avoid biases and leading answers, the feedback is provided in a form of range where the expected answer should reside.



%  \begin{figure}[thb!]
%     \centering
%     \includegraphics[width = 0.5\linewidth]{IEEEtran/figs/survey.png}
%     % \vspace{-2em}
%     \caption{A question of actual listener study interface conducted on MTurk.}
%     \label{fig:survey}
%     % \vspace{-2em}
% \end{figure}

% \nayem{***One paragraph on the statistics of the conducted study.}
% The study session contains total 30 questions, which is preceded by a practice session of 7 questions. Ten participants (9 male, 1 female) who are native English speakers over the age of 18 participated, where a headset/headphone was required to be worn. On average, participants took 14 minutes to complete the study, they were given $\$3$ monetary incentive.


\section{Discussion}
\label{sec:discuss}

Our proposed model outperforms all comparison models on SI-SDR metrics for both seen and unseen datasets, without optimization of any of the models (Table \ref{tab:results_cosineVoices}, \ref{tab:results_chime}). This means that our approach improves speech quality by minimizing the distortion ratio when separated from the noise component. Additionally, our models yield the best MOS-LQO ratings on real-world captured audios (CHiME datasets, Table \ref{tab:results_chime}). These results are consistent with the findings of \cite{zezario2022deep, nayem2021incorporating} that incorporating embeddings from a speech assessment model improves SE performance, and the results of \cite{braun2022effect} that using MOS loss during model optimization leads to higher MOS-LQO scores. Our proposed approach achieves PESQ and ESTOI scores that are only slightly lower than those of the best-performing model, with a difference of only $0.03$ and $0.01$, respectively. This indicates that speech quality and intelligibility metrics are closely related to the subjective speech quality metric (MOS-LQO), and that these metrics can be improved without explicit optimization. Furthermore, our proposed model achieves the best average DNSMOS scores with low standard deviations on CHiME datasets (Figure \ref{fig:dnsmos_results}), indicating that it is effective in a wide range of real-world noise levels. This is a desirable quality for an effective SE model to be effective not only in high SNRs and limited noisy environments, but also in large SNR ranges and real-world conditions such as those offered by the CHiME dataset.

When comparing our proposed model that uses mse+sa+mos loss to the PMOS+SE model (as shown in Table \ref{tab:results_chime}), we can observe significant improvements in all performance metrics. As both models use the same loss function, the improvements are attributed to the incorporation of LM and the joint learning method. Moreover, we found that these two models exhibit similar performance on the MOS prediction (Table \ref{tab:mos_results}), indicating that the benefits of joint learning mostly impact the enhancement part of the model.

An intriguing finding is that our proposed model shows a decline in WER\% when MOS loss is incorporated, especially for larger real-world recordings such as CHiME-5, with degradation up to $1.1$. Although our study is not primarily concerned with ASR performance, this suggests a potential trade-off between ASR accuracy and subjective speech quality scores. Further investigation is needed to comprehend this relationship.

Our proposed method demonstrates that training a speech enhancement (SE) model and a MOS-based speech assessment model jointly can lead to better speech quality measured by objective metrics such as perceptual quality, intelligibility, and MOS ratings. However, we acknowledge that our study's use of subjective MOS (sMOS) estimation instead of actual human listeners may introduce discrepancies between MOS-LQO and human-rated MOS, which could impact our findings. To address this limitation, we plan to conduct sMOS evaluation by human listeners in future work. Although we used the same MOS prediction model for all comparison models, we believe that incorporating human-rated sMOS evaluations will provide more robust insights into our proposed method's effectiveness.
For computing loss terms, we opt for the MSE loss function along with a bi-gram language model that considers only time-along transitions. Our aim is to keep the model simple and focus on the effectiveness of our approach. However, we acknowledge that using different loss functions for different loss components and employing a more complex language model that considers both temporal and spectral transition levels can be beneficial. We plan to explore these possibilities in our future work.



% !TEX root = root.tex

\section{Conclusions and Future Work}
\label{sec:5_discussion}
Our work unifies the SF-GPI and value composition to the continuous concurrent composition framework and allows reconstructing task policy from a set of primitives. The proposed method was extended to composition at the action component level. We demonstrate in the Pointmass environment that our multi-task agents can reconstruct the task policy from a set of primitives in real time and transfer the skills to solve unseen tasks while the single-task performance is competitive with SAC.
This flexible framework incorporates well with the reward-shaping techniques, such as entropy regularization, curiosity\cite{pmlr-v70-pathak17a}, etc. In addition, the task-agnostic property should benefit the autotelic framework \cite{colas2022autotelic} where agents can set goals and curriculum for themselves \cite{narvekar2020curriculum}. 

However, the primary concern at this stage is whether the proposed approach can scale to higher dimensional problems. Additionally, two important topics are left as future works. First, look for the corresponding value composition for DAC. A good starting point might be thinking of the MSF composition with weights evaluated by GPE. 
Second, the optimality of each composition method. One might start with bounding the loss incurred by the policy and value composition. 
 \section{Conclusion}
 In this paper, we have presented a tactile manipulation system that is able to rotate different objects without vision. We showed an end-to-end reinforcement learning framework to learn tactile dexterity over the proposed system. We carried out experiments both in simulation and real to demonstrate its effectiveness. Our work demonstrated that we are able to achieve tactile dexterity as humans in real for the first time. In the future, there are many promising future directions to investigate, such as exploring the use of a more dense contact sensor array and scaling up the system to solve more diverse tasks. We hope that our work can pave the way for more intelligent robot hands.

\printendnotes

\bibliography{reference.bib}

\end{document}