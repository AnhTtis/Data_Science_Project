\section{Introduction}

Advancements in generative artificial intelligence (AI) and large language models (LLMs) have fueled the development of many educational technology innovations that aim to automate the often time-consuming and laborious tasks of generating and analysing textual content (e.g., generating open-ended questions and analysing student feedback survey) \citep{kasneci2023chatgpt,wollny2021we,leiker2023prototyping}. LLMs are generative artificial intelligence models that have been trained on an extensive amount of text data, capable of generating human-like text content based on natural language inputs. Specifically, these LLMs, such as Bidirectional Encoder Representations from Transformers (BERT) \citep{devlin2018bert} and Generative Pre-trained Transformer (GPT) \citep{brown2020language}, utilise deep learning and self-attention mechanisms \citep{vaswani2017attention} to selectively attend to the different parts of input texts, depending on the focus of the current tasks, allowing the model to learn complex patterns and relationships among textual contents, such as their semantic, contextual, and syntactic relationships \citep{min2021recent,liu2023context}. As several LLMs (e.g., GPT-3 and Codex) have been pre-trained on massive amounts of data across multiple disciplines, they are capable of completing natural language processing tasks with little (few-shot learning) or no additional training (zero-shot learning) \citep{brown2020language,wu2023matching}. This could lower the technological barriers to LLMs-based innovations as researchers and practitioners can develop new educational technologies by fine-tuning LLMs on specific educational tasks without starting from scratch \citep{caines2023application,sridhar2023harnessing}. The recent release of ChatGPT, an LLMs-based generative AI chatbot that requires only natural language prompts without additional model training or fine-tuning \citep{openai2022chatgpt}, has further lowered the barrier for individuals without technological background to leverage the generative powers of LLMs.

Although educational research that leverages LLMs to develop technological innovations for automating educational tasks is yet to achieve its full potential (i.e., most works have focused on improving model performances \citep{kurdi2020systematic,ramesh2022automated}), a growing body of literature hints at how different stakeholders could potentially benefit from such innovations. Specifically, these innovations could potentially play a vital role in addressing teachers' high levels of stress and burnout by reducing their heavy workloads by automating punctual, time-consuming tasks \citep{carroll2022teacher} such as question generation \citep{kurdi2020systematic,bulut9automatic,oleny2023generating}, feedback provision \citep{cavalcanti2021automatic,nye2023generative}, scoring essays \citep{ramesh2022automated} and short answers \citep{zeng2023Curriculum}. These innovations could also potentially benefit both students and institutions by improving the efficiency of often tedious administrative processes such as learning resource recommendation, course recommendation and student feedback evaluation, potentially  \citep{zawacki2019systematic,wollny2021we,sridhar2023harnessing}. 

Despite the growing empirical evidence of LLMs' potential in automating a wide range of educational tasks, none of the existing work has systematically reviewed the practical and ethical challenges of these LLMs-based innovations. Understanding these challenges is essential for developing responsible technologies as LLMs-based innovations (e.g., ChatGPT) could contain human-like biases based on the existing ethical and moral norms of society, such as inheriting biased and toxic knowledge (e.g., gender and racial biases) when trained on unfiltered internet text data \citep{schramowski2022large}. Prior systematic reviews have focused on investigating these issues related to one specific application scenario of LLMs-based innovations (e.g.,  question generation, essay scoring, chatbots, or automated feedback) \citep{kurdi2020systematic,cavalcanti2021automatic,wollny2021we,ramesh2022automated}. The practical and ethical challenges of LLMs in automating different types of educational tasks remain unclear. Understanding these challenges is essential for translating research findings into educational technologies that stakeholders (e.g., students, teachers, and institutions) can use in authentic teaching and learning practices \citep{adams2021artificial}.

The current study is the first systematic scoping review that aimed to address this gap by reviewing the \textit{current state of research} on using LLMs to automate educational tasks and identify the \textit{practical} and \textit{ethical} challenges of adopting these LLMs-based innovations in authentic educational contexts. A total of 118 peer-reviewed publications from four prominent databases were included in this review following the Preferred Reporting Items for Systematic Reviews and Meta-Analyses (PRISMA) \citep{page2021prisma} protocol. An inductive thematic analysis was conducted to extract details regarding the different types of educational tasks, stakeholders, LLMs, and machine learning tasks investigated in prior literature. The practicality of LLMs-based innovations was assessed through the lens of technological readiness, model performance, and model replicability. Lastly, the ethicality of these innovations was assessed by investigating system transparency, privacy, equality, and beneficence. 

The contribution of this paper to the educational technology community is threefold: 1) we systematically summarise a comprehensive list of 53 different educational tasks that could potentially benefit from LLMs-based innovations through automation, 2) we present a structured assessment of the practicality and ethicality of existing LLMs-based innovations based on seven important aspects using established frameworks (e.g., the transparency index \citep{chaudhry2022transparency}), and 3) we propose three recommendations that could potentially support future studies to develop LLMs-based innovations to be practically and ethically implement in authentic educational contexts. As the intersection of LLMs and education is continuously evolving, the findings of this systematic scoping review can serve as an essential reference point for researchers, allowing them to leverage the strengths, learn from the limitations, and uncover potential opportunities for novel LLMs in supporting educational research and practice. Specifically, emerging works should carefully consider the practical and ethical challenges identified in this study while exploring the research opportunities enabled by ChatGPT and other generative AI models.
