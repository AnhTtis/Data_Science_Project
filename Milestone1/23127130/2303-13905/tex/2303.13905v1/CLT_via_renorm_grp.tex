\documentclass[12pt,a4paper,reqno]{amsart}

\synctex=1

\usepackage[margin=2.8cm,footskip=1cm]{geometry}
\setlength{\marginparwidth}{2.4cm}
\usepackage[utf8]{inputenc}
\usepackage[T1]{fontenc}
\usepackage[english]{babel}
\usepackage{amsmath}
\usepackage{amsfonts}
\usepackage{amssymb}
\usepackage{amsthm}
\usepackage{ucs}
\usepackage{graphicx}
\usepackage{xcolor}
\usepackage{dsfont}
\usepackage{tikz}
\usepackage{wasysym}
\usepackage{physics}
\usepackage{mathtools}
\usepackage{xifthen}
\usepackage[textwidth=2.5cm,textsize=tiny]{todonotes}
\usepackage{enumitem}
\usepackage{stmaryrd}
\usepackage{constants}
\usepackage[ruled]{algorithm2e}
\usepackage{tikz,pgfplots}
%\usepackage{chngcntr}
\usepackage[hidelinks]{hyperref}
\usepackage[font={small}]{caption}

%\graphicspath{{Pictures/}}


%%%%%%%%%%%%%%%%%%%%%%%%%%%%%%%%%%%%%%%%%%%%%%%%%%%%%%%%%%%%%%%%%%%%%%%%%%%%%%%%%%%%%%%%%%%%%%%%%%%

\makeatletter
\newcommand{\pushright}[1]{\ifmeasuring@#1\else\omit\hfill$\displaystyle#1$\fi\ignorespaces}
\newcommand{\pushleft}[1]{\ifmeasuring@#1\else\omit$\displaystyle#1$\hfill\fi\ignorespaces}
\makeatother


%%%%%%%%%%%%%%%%%  Macro  %%%%%%%%%%%%%%%%%%%%%%%%%%%%%%%%%%%%%%%%%%%%%%

\renewcommand{\norm}[1]{\|#1\|}
\newcommand{\normI}[1]{\left\|#1\right\|_{\scriptscriptstyle 1}}
\newcommand{\normsup}[1]{\left\|#1\right\|_{\scriptscriptstyle\infty}}
\newcommand{\normII}[1]{\|#1\|_{\scriptscriptstyle 2}}

\newcommand{\R}{\mathbb{R}}

\newcommand{\rmd}{\mathrm{d}}
\newcommand{\rmi}{\mathrm{i}}

\newcommand{\calN}{\mathcal{N}}
\newcommand{\calP}{\mathcal{P}}
\newcommand{\calQ}{\mathcal{Q}}


%%%%%%%%%%%%%%%%%%%%%%%%%%%%%%%%%%%%%%%%%%%%%%%%%%%%%%%%%%%%%%%%%%%%%%%%%%%%%%%%%%%%%%%%%%%%%%%%%%%

\theoremstyle{plain}
\newtheorem{theorem}{Theorem}[section]
\newtheorem{lemma}[theorem]{Lemma}
\newtheorem{proposition}[theorem]{Proposition}
\newtheorem{corollary}[theorem]{Corollary}
\newtheorem{conjecture}[theorem]{Conjecture}
\newtheorem{remark}{Remark}[section]
\newtheorem{properties}{Properties}
\newtheorem{claim}{Claim}


\theoremstyle{definition}
\newtheorem{definition}{Definition}[section]
\newtheorem{obs}{Observation}
\newenvironment{Obs}{\begin{obs}}{ \end{obs}}


%%%%%%%%%%%%%%%%%%%%%%%%%%%%%%%%%%%%%%%%%%%%%%%%%%%%%%%%%%%%%%%%%%%%%%%%%%%%%%%%%%%%%%%%%%%%%%%%%%%

\author{S\'{e}bastien Ott}
\address{D\'epartement de Math\'ematiques, Universit\'e de Fribourg,
	Chemin du Mus\'ee 23, 1700 Fribourg, Switzerland}
\email{ott.sebast@gmail.com}

\date{\today}

\title{A note on the renormalization group approach to the Central Limit Theorem}

\begin{document}

\begin{abstract}
	A proof of the Central Limit Theorem using a renormalization group approach is presented. The proof is conducted under a third moment assumption and shows that a suitable renormalization group map is a contraction over the space of probability measures with a third moment. This is by far not the most optimal proof of the CLT, and the main interest of the proof is its existence, the CLT being the simplest case in which a renormalization group argument should apply. None of the tools used in this note are new. Similar proofs are known amongst expert in limit theorems, but explicit references are not so easy to come by for non-experts in the field.
\end{abstract}

\maketitle

%%%%%%%%%%%%%%%%%%%%%%%%%%%%%%%%%%%%%%%%%%%%%%%%%%%%%%%%%%%%%%%%%%%%%%%%%%%%%%%%%%%%%%%%%%%%%%%%%%%

\section*{A word from the author}

I am neither a renormalization group expert, nor a limit theorem expert. So it is perfectly possible that the results described this note appeared somewhere else, or that I missed relevant references to the ``Renormalization group approach to CLT'' story. In both cases, I would be extremely grateful to receive pointers towards the relevant references.

\section*{Introduction/bibliographical review}

This note is about providing a simple renormalization group style proof of the classical Central Limit Theorem. The CLT is likely to be the most well known feat of probability theory, so I do not intend to say anything new about it. The purpose of this work is more to perform a ``sanity check'' for the renormalization group approach: if one can not get the method to work in the simplest instance it should apply to, one has little chances of success in much more involved situations. This note is not the first result about the renormalization road to the CLT: on the ``perturbative side'' (perturbation around the fixed point) I am aware of the rigorous works~\cite[pages 131-132]{Sinai-1992},~\cite[section 10.3]{Koralov+Sinai-2007}, which deal with the same renormalization group map, and ~\cite{Calvo+Cuchi+Esteve+Falceto-2010}, same but with as a goal the study of more general stable laws. A non-rigorous renormalization approach to that problem is also discussed in~\cite{Amir-2020}. A non-perturbative argument can be found as an application of the methods of~\cite{Neininger+Ruschendorf-2004} (see the slides~\cite{Neininger-2007} for the application to the CLT). The proof there is very similar to the one presented here: use of ideal metric (Zolotarev metric -introduced in~\cite{Zolotarev-1976,Zolotarev-1978}-, in~\cite{Neininger-2007} versus Fourier based metric here) to prove contraction properties of the renormalization map.

As said above, the approach taken here is not new: the renormalization group map is the one of \cite{Sinai-1992} (see equation (15.3) there) and is probably older, the contraction principle is included in the paper~\cite{Goudon+Junca+Toscani-2002}: equations (6), (7) are morally the contraction used here (which is a property of ideal metrics). The argument presented in~\cite{Neininger-2007} is the same as the one of the present note but with a different metric. It is worth noting that a contraction principle is underlying most approaches to the CLT, but the latter is usually formulated as follows: one has a sequence of operators, \((T_n)_{n\geq 1}\) (which acts by \(n\)-fold convolution and rescaling by \(\sqrt{n}\)), having the normal distribution as fixed point, and having ``law-dependent contraction constant'' going to \(0\) with \(n\), rather than iterating a fixed transform with uniform (over probability distribution) contraction constant.

As nothing is really new in the content of this text, I will work under a third moment assumption. The latter can be relaxed to a \(2+\epsilon\) moment assumption: the distance used (denoted \(\rmd_3\) in the text) is a particular case of the Fourier-based distances introduced in~\cite{Gabetta+Toscani+Wennberg-1995}. Replacing this distance by its \(\rmd_{2+\epsilon}\) version handles the extension (\(3\) is chosen for aesthetic reasons and to slightly simplify appendix~\ref{app:Metric_structure}). A review on these metrics and their applications can be found in~\cite{Carrillo+Toscani-2007}.
Making a ``whole space'' renormalization group argument with only a second moment assumption remains an open question as far as I know.

\section{Renormalization road to CLT}

\subsection{Framework}

Probability measures on \(\R\) will be denoted \(\nu,\mu\), the expectation under \(\nu\) will be denoted \(E_{\nu}\). Inside expected value, \(X\) will denote a random variable of the relevant law, and \((X,Y)\) a random vector of law \(\nu\otimes \mu\) when this case is considered. Let \(\calP_3=\calP_3(\R)\) be the set of probability measures on \(\R\) with finite third absolute moment. Denote
\begin{equation*}
	\calQ_3 := \{\nu\in \calP_3:\ E_{\nu}(X) = 0,\ E_{\nu}(X^2) =1 \},
\end{equation*}the set of centred, reduced probability measures with a third absolute moment. Equip \(\calQ_{3}\) with the Fourier-based distance
\begin{equation}
	\rmd_3(\nu,\mu) = \sup_{\xi\in \R^*} \frac{\big| \varphi_{\nu}(\xi) - \varphi_{\mu}(\xi) \big|}{|\xi|^3}
\end{equation}where \(\R^* = \R\setminus \{0\}\), and
\begin{equation}
	\varphi_{\nu}(\xi) = E_{\nu}\big(e^{\rmi X\xi}\big),
\end{equation}is the characteristic function of \(\nu\) (\cite{Carrillo+Toscani-2007} defines it with a \(-\) sign in the exponential).

\begin{lemma}
	\label{lem:Finite_metric_space_struct}
	\(\rmd_3\) is a finite distance on \(\calQ_3\).
\end{lemma}
This result can be imported form~\cite{Gabetta+Toscani+Wennberg-1995, Carrillo+Toscani-2007}, but a proof is included in Appendix~\ref{app:Metric_structure}. The goal will be to study convergence towards a normal distribution \(\calN(0,1)\). Denote \(\gamma\) the normal law:
\begin{equation}
	d\gamma(x) = \frac{1}{\sqrt{2\pi}} e^{-x^2/2} dx.
\end{equation}One obviously has \(\gamma \in \calQ_3\).

\subsection{Renormalization transform}

One then consider the following renormalization transformation on probability measures: \(T\nu\) is the law of \(2^{-1/2}(X+Y)\) where \(X,Y\) are independent random variables of law \(\nu\):
\begin{equation}
	E_{T\nu}(f) = E_{\nu\otimes \nu}\Big(f\big((X+Y)/\sqrt{2}\big)\Big).
\end{equation}In words: \(T\) maps \(\nu\) to the renormalized convolution of \(\nu\) with itself. Taking the sum is the ``coarse graining'' part of a renormalization step and dividing by \(\sqrt{2}\) is the ``rescaling'' part.

\begin{lemma}
	\label{lem:stability}
	If \(\nu\in \calQ_3\), then \(T\nu\in \calQ_3\).
\end{lemma}
\begin{proof}
	From the definition, one has that if \(\nu\) has a second moment, and \(E_{\nu}(X) = 0\), then \(E_{T\nu}(X) =0\), and \(E_{T\nu}(X^2) = E_{\nu}(X^2)\). Then,
	\begin{equation*}
		E_{T\nu}(|X|^3) = \frac{1}{2^{3/2}}E_{\nu\otimes \nu}(|X+Y|^3)\leq \frac{16}{2^{3/2}} E_{\nu}(|X|^3)<\infty.
	\end{equation*}
\end{proof}

The goal of this note is to study the CLT, so the main interest of this transformation is
\begin{equation}
	\label{eq:fixed_pt}
	T\gamma = \gamma,
\end{equation}which is an standard consequence of the stability of the Gaussian distribution.

The link with the CLT is as follows: if \(X_1,X_2,\dots\) form an i.i.d. sequence of law \(\nu\in \calQ_3\), then \(T^n \nu \) is the law of
\begin{equation*}
	\frac{1}{2^{n/2}}\sum_{k=1}^{2^n} X_k \equiv \frac{1}{\sqrt{N}}\sum_{k=1}^{N} X_k,
\end{equation*}where \(N= 2^n\).

\subsection{Contraction and CLT}

\begin{theorem}
	\label{thm:contraction_principle}
	The application \(T\) is a contraction on \((\calQ_3,\rmd_3)\) with contraction constant \(\leq 2^{-1/2}\). In particular (by~\eqref{eq:fixed_pt}),
	\begin{equation}
		\rmd_3(T^n \nu, \gamma) \leq 2^{-n/2} \rmd_3(\nu, \gamma).
	\end{equation}
\end{theorem}
\begin{proof}
	The proof is almost trivial. Let \(\mu,\nu\in \calQ_3\). First, notice that
	\begin{equation}
		\varphi_{T\nu}(\xi) = E_{\nu\otimes \nu}(e^{\rmi \xi (X+Y)/\sqrt{2}}) = \varphi_{\nu}(\xi/\sqrt{2})^2.
	\end{equation}
	 Then, writing the definition, one has
	\begin{align*}
		\rmd_3(T\nu, T\mu) &= \sup_{\xi\in \R^*} \frac{\big|\varphi_{\nu}(\xi/\sqrt{2})^2 - \varphi_{\mu}(\xi/\sqrt{2})^2 \big|}{2^{3/2}|\xi/\sqrt{2}|^3}\\
		&= 2^{-3/2} \sup_{\xi\in \R^*} \frac{\big|\varphi_{\nu}(\xi)^2 - \varphi_{\mu}(\xi)^2 \big|}{|\xi|^3}\\
		&= 2^{-3/2} \sup_{\xi\in \R^*} \frac{\big|\varphi_{\nu}(\xi) - \varphi_{\mu}(\xi) \big|}{|\xi|^3}\big|\varphi_{\nu}(\xi) + \varphi_{\mu}(\xi) \big|\\
		&\leq 2^{-1/2} \sup_{\xi\in \R^*} \frac{\big|\varphi_{\nu}(\xi) - \varphi_{\mu}(\xi) \big|}{|\xi|^3} = 2^{-1/2} \rmd_3(\nu,\mu),
	\end{align*}as \(\normsup{\varphi_{a}} = 1\) for any probability measure \(a\).
\end{proof}

\begin{corollary}[Central Limit Theorem]
	\label{cor:CLT}
	Let \(\nu\in \calQ_3\). Let \(X_1,X_2,\dots\) be an i.i.d. sequence of law \(\nu\). Then,
	\begin{equation}
		\frac{1}{\sqrt{N}}\sum_{k=1}^N X_k \xrightarrow{N\to\infty} \calN(0,1),
	\end{equation}where the convergence is \(\rmd_3\) distance (and therefore also in law).
\end{corollary}
The claim along the sequence \(N=1,2,4,8,\dots\) (or any geometric sequence) follows directly from Theorem~\ref{thm:contraction_principle}, Lemma~\ref{lem:stability}, and the fact that \(\rmd_3\) is a \emph{finite} distance on \(\calQ_3\) (by Lemma~\ref{lem:Finite_metric_space_struct}). Extending this to arbitrary sequences can be done in several ways, which shall not be exposed here. It is worth stressing out that extending the result to arbitrary sequences is never simpler than using stability of Gaussian and scaling+convolution properties of \(\rmd_3\) to obtain the CLT directly. Indeed,
\begin{equation}
\label{eq:ideal metric}
\begin{gathered}
	\rmd_3(\nu_1*\nu_2, \mu_1*\mu_2) \leq \rmd_3(\nu_1, \mu_1)+\rmd_3(\nu_2, \mu_2),\\
	\rmd_3([\nu]_{\lambda},[\nu]_{\lambda}) \leq \lambda^3 \rmd_3(\nu,\mu),
\end{gathered}
\end{equation}where \([\nu]_{\lambda}\) is the law of \(\lambda X, X\sim \nu\). The first point follows from \(|ab-cd|\leq |a-c||b| + |b-d||c|\), and the definition. Therefore, by the properties of the Gaussian,
\begin{equation*}
	\rmd_3([\nu^{*n}]_{n^{-1/2}}, \gamma) = \rmd_3([\nu^{*n}]_{n^{-1/2}}, [\gamma^{*n}]_{n^{-1/2}}) \leq \frac{1}{n^{3/2}} \rmd_3(\nu^{*n}, \gamma^{*n})\leq \frac{1}{n^{1/2}}.
\end{equation*}


\section{Concluding remarks}
\label{sec:conclusion}

The metric \(\rmd_3\) (and its generalizations \(\rmd_s\), see~\cite{Goudon+Junca+Toscani-2002}) as well as the Zolotarev metric mentioned in the introduction, belong to a class of metric on probability measures that are defined as the weighted supremum of difference between expectation values over a suitable class of test functions (apparently said to have a \emph{\(\zeta\)-structure}). When such a distance is constructed in a way that makes it \(\lambda\)-ideal (a suitable version of~\eqref{eq:ideal metric}, see~\cite[equations (6),(7)]{Goudon+Junca+Toscani-2002}) with \(\lambda>2\), one can perform the same contraction argument as presented here and in~\cite{Neininger-2007}. It is also clear that \(\lambda=2\) will not work so easily, so a different idea is required.


\section*{Acknowledgements}

Thanks to Nicolas Curien for suggesting me to contact Ralph Neininger, and to Ralph Neininger for pointers to~\cite{Neininger+Ruschendorf-2004,Neininger-2007}, and comments on the general picture.

The author is supported by the Swiss NSF grant 200021\_182237 and is a member of the NCCR SwissMAP.


\appendix

\section{Fourier based distance}
\label{app:Metric_structure}

This section contains a proof of Lemma~\ref{lem:Finite_metric_space_struct}. The proof is by no mean new, it is included for the reader convenience.

First, note that \(\rmd_{3}\) is a distance: symmetry is obvious, and separation follows from the fact that two probability measures are the same if and only if they have the same characteristic function. Triangular inequality follows from triangular inequality for the absolute value, and from \(\sup f+g \leq \sup f + \sup g\).

Then, show that \(\rmd_{3}\) is finite on \(\calQ_3\times \calQ_3\). Take \(\nu,\mu\in \calQ_3\). Then, as \(\nu,\mu\) have a third moment, \(\varphi_{\nu},\varphi_{\mu}\) are three times continuously differentiable and they admit a Taylor expansion at \(0\):
\begin{equation}
	\varphi_*(\xi) = 1 - \frac{\xi^2}{2} - \rmi\frac{E_{*}(X^3) \xi^3}{6} + h_*(\xi) \xi^3,
\end{equation}with \(h_*(\xi) \xrightarrow{\xi\to 0} 0\) bounded uniformly over \([-1,1]\) (as it is continuous over \(\R\)), \(*\in \{\nu,\mu\}\). So,
\begin{equation*}
	\rmd_3(\nu,\mu) \leq 6^{-1}\sup_{0<|\xi|<1} \big|- \rmi E_{\nu}(X^3) + 6h_{\nu}(\xi)  + \rmi E_{\mu}(X^3) - 6h_{\mu}(\xi) \big| + 2<\infty.
\end{equation*}



\bibliographystyle{plain}
\bibliography{BIGbib}



\end{document}