\section{TTC's and Picard groups}\label{section3}
While the Bondal-Orlov reconstruction (Theorem \ref{thm:bonorvreconstruction}) tells us that one can directly recover a smooth projective variety $X$ with ample (anti-)canonical bundle from the derived category $\der\simeq Perf(X)$,  there are plenty of smooth projective varieties which have non-isomorphic Fourier-Mukai partners, varieties $Y$ such that $\der\simeq D^{b}(Y)$, which implies that on a given derived category $\der$ there might be many nonequivalent tensor triangulated category structures. \\
However even in the case where our variety $X$ has ample (anti-)canonical bundle as in the hypothesis of the Bondal-Orlov reconstruction theorem, it is not immediate that there is only one possible tensor triangulated category structure. It is, in principle, possible that there might be one such structure $(\der,\boxtimes,\mathbbm{1})$ such that $D^{b}(Spc(\boxtimes, \mathbbm{1}))\not\simeq \der$ and so Bondal-Orlov does not apply. \\
In some sense our motivating question is whether Balmer's reconstruction implies Bondal-Orlov. In this section we will be looking into this and related ideas by exploring the possible tensor triangulated categories one can equip on $\der$ under the slightly more general hypothesis of $X$ having a big (anti-)canonical bundle. \\
We start by mentioning the following result by Liu and Sierra from \cite{liu2013recovering} that shows in particular that there are smooth projective varieties $X$ with ample anti-canonical bundle, so under the hypothesis of Bondal-Orlov, for which the derived category $\der$ admits a tensor triangulated category structure $(\boxtimes, \mathbbm{1})$ such that $Spc(\boxtimes, \mathbbm{1})\not\cong X$. \\
Recall that there are varieties $X$ that are known to have derived categories equivalent to the derived category of representations on a quiver (possibly with relations). For example, in the presence of a full strong exceptional collection $\{E_{i}\}$ then we have that $\der$ is equivalent to $D^{b}(mod-End(\bigoplus E_{i}))$, the derived category of finitely generated modules over the algebra $End(\bigoplus E_{i})$. This latter algebra on the other hand is equivalent to the path algebra of a quiver, and so we obtain an equivalence between the derived category of $X$ and the derived category of finite dimensional representations of a quiver $Q=(P_{n},E_{ij})$. \\
The important point here is that this derived category of representations of a quiver comes with a tensor triangulated category structure induced by the tensor product of representations. To recall, let $(V_{i},p_{ik})$ and $(W_{j},q_{js})$ two such representations, then the tensor product is given entry-wise: $(V_{i},p_{ik})\otimes_{rep} (W_{j},q_{js}):= (V_{i}\otimes W_{j}, p_{ik}\otimes q_{js})$.\\
Let us denote by $(D^{b}(repQ),\otimes^{\mathbb{L}}_{rep}, \mathbbm{1}_{rep})$ the resulting tensor triangulated category structure on $D^{b}(repQ)$ by deriving this tensor product and where $\mathbbm{1}_{rep}:=(k_{i},Id_{ij})$ is the representation given by putting $k$ on every vertex  and the identity morphism in each edge of the quiver. \\
Liu and Sierra consider quivers with relations satisfying a compatibility condition with the tensor product (\cite[Definition 1.2.5]{liu2013recovering}) and say that in this case the quiver has tensor relations. \\
\begin{thm}\cite[Theorem 2.1.5.1]{liu2013recovering}\label{thm:sierraspace}
Let $Q$ be a finite ordered quiver with tensor relations. Then $Spc((D^{b}(repQ),\otimes^{\mathbb{L}}_{rep}))$ is the discrete space $\{P_{n}\}$.
\end{thm}
They also describe completely the structure sheaf in this case.
\begin{thm}\cite[Theorem 2.2.4.1]{liu2013recovering}\label{thm:sierrastructuresheaf}
Let $Q$ be a finite ordered quiver with tensor relations. Then $\Ox_{Q}:=\Ox_{Spc(\otimes^{\mathbb{L}}_{rep})}$ is the constant sheaf of algebras $k$. So that for any open $W\subset Spc(\otimes_{rep}^{\mathbb{L}})$ we have $\Ox_{Q}(W)=k^{\oplus W}$.
\end{thm}
In particular, for $X=\mathbb{P}^{n}$ we have by a well-known result of Beilinson (\cite{beilinson1978coherent}) that $\der$ is equivalent to the category of representations of a quiver with $n+1$ vertices. Thus the derived category $\der$ has a tensor triangulated category structure $(\otimes^{\mathbb{L}}_{rep},\mathbbm{1}_{rep})$ such that $Spc((\otimes^{\mathbb{L}}_{rep},\mathbbm{1}_{rep}))\not\cong X=\mathbb{P}^{n}$. As $\mathbb{P}^{n}$ is a smooth projective variety with ample anti-canonical bundle, this previous result implies that the study of tensor triangulated category structures on $\der$ is not trivial even in the cases falling under the hypothesis of the Bondal-Orlov reconstruction theorem and might shed some light in the internal structure of the derived category in itself. \\ 
\\
In general the behavior of the dynamics of the Balmer spectrum and taking derived categories can be complex. As we know that the Balmer spectrum is a locally ringed space it has an abelian category of sheaves of modules which admits a tensor product then we can derive this category as usual, however the category of sheaves of modules is in general much more complicated than a category of coherent or even quasi-coherent sheaves. \\ 
\\
Having said that, let us put ourselves in the slightly more general situation of derived categories of varieties of general type. Recall a variety is of general type if its canonical bundle is big. In particular varieties with ample canonical bundle are of general type. \\
One alternative characterization of bigness for a variety is the following:
\begin{thm}\cite[Example 2.2.9]{lazarsfeld2017positivity}
A smooth projective variety is of general type if and only if, for any sheaf $\F\in Coh(X)$, there exists an integer $i_{0}$ depending on $\F$ such that the sheaf $\F\otimes_{X}\omega_{X}^{i}$ is generically globally generated for $i>> i_{0}$.
\end{thm}
As a consequence of the Kodaira lemma (cf. \cite[Prop 2.2.6]{lazarsfeld2017positivity}) we have the corollary: \\
\begin{cor}\label{cor:opengentype}
Let $X$ be a smooth projective variety of general type, then there exists an open sub-variety $X^{\ast}$ such that for any $\F\in Coh(X)$, there exists a positive integer $i_{0}$ for which for any $i>>i_{0}$, the sheaf $\F\mid_{X^{\ast}}\otimes_{X}\omega_{X^{\ast}}^{i}$ on $X^{\ast}$ is globally generated.
\end{cor}
Let us explain the previous corollary and the nature of the open sub-variety $X^{\ast}$. We recall some basic definitions.
\begin{defn}\label{defn:augmentedlinebundle}
Let $X$ be a projective variety and $\Ll$ a line bundle on $X$, the augmented base locus is the Zariski closed set
\[ B_{+}(\Ll):=\bigcap_{m\in\mathbb{N}} B(m\Ll-A) \]
Where $A$ is any ample line bundle, and for any line bundle $\Ll'$ the set $B(\Ll')$ is defined as the intersection of the base loci of multiples of the line bundle, that is
\[ B(\Ll'):= \bigcap_{m\in \mathbb{N}} Bs(m\Ll') \]
\end{defn}
In \cite{boucksom2014augmented} the following theorem characterizing the complement of the augmented base focus is proven:
\begin{thm}\label{thm:restrictionisample}
Let $\Ll$ be a big line bundle on a normal projective variety $X$ over an algebraically closed field. Then the complement $X\backslash B_{+}(\Ll)$ of the augmented base locus is the largest Zariski open subset $U\subseteq X\backslash B(\Ll)$ such that for all large and divisible $m(\Ll)\in \mathbb{Z}$ the restriction of the morphism 
\[ \phi_{m}:X\backslash B(\Ll) \dashrightarrow \mathbb{P}H^{0}(X,m\Ll) \]
to U is an isomorphism onto its image.
\end{thm}
The following couple important observations follow immediately from the definition, the fact that the augmented base locus is independent of the choice of ample line bundle, and Kodaira's decomposition of big line bundles. 
\begin{obs}\label{properideal}
\begin{enumerate}
    \item $B_{+}(\Ll)=\emptyset$ if and only if $\Ll$ is ample.
    \item $B_{+}(\Ll)\not=X$ if and only if $\Ll$ is big. 
\end{enumerate}
\end{obs}
From the remarks above and using Thomason's classification Theorem (\ref{thm:balmerthomasonclassification}) we know that as there exists a correspondence between closed subsets of the Balmer spectrum and radical tensor ideals in the tensor triangulated category then there exists a radical tensor ideal corresponding to the augmented base locus $B_{+}(\Ll)$ for any given line bundle $\Ll$. In particular the open sub-variety $X^{\ast}$ from Corollary \ref{cor:opengentype} is the complement of the augmented base locus, $X\backslash B_{+}(\omega_{X})$ 
 and corresponds to a $\dtee$-ideal generated by a single object ( using Theorem \ref{thm:roquierthomason} ) whose homological support gives back the closed subset $B_{+}(\omega_{X})$. \\
\begin{obs}
Let us denote by $I_{X^{\ast}}$ the $\dtee$-ideal corresponding to the open subvariety $X^{\ast}$. By our previous Remark \ref{properideal} we have that this ideal must be a proper $\dtee$-ideal of $\der$ and is the ideal $0$ precisely when the (anti-)canonical bundle is ample.
\end{obs}
We would like to understand the effect of the positivity of the canonical bundle ( in this case the fact that the variety is of general type ) on the tensor triangulated structure of the category. We know from Theorem \ref{thm:uniqueserrefunctor} that the Serre functor in a triangulated category is unique up to degree whenever it exists and so it is a property of the category and not extra data. In our concrete case we know furthermore that the Serre functor is isomorphic to $\_\dtee\omega_{X}[n]$ where $n\in\mathbb{N}$ is the dimension of the variety and $\omega_{X}$ is the dualizing sheaf of $X$. \\
Let us start with a definition mimicking that of spanning class:
\begin{defn}\label{defn:almostspanning}
Let $(\T,\otimes)$ be a tensor triangulated category, let $\I\subseteq \T$ be a thick subcategory and let us denote by $\pi:\T\to \T/\I$ the localization functor. We say that a collection of objects $\Omega\subset \T$ is an almost spanning class with respect to $\I$ if the following two conditions hold.
\begin{enumerate}
    \item If $X\in \T/\I$ is such that $Hom_{\T/\I}(\pi(B),X[j])=0$ for all $B\in \Omega$ and $j\in \mathbb{Z}$, then $X\cong 0$.
    \item If $X\in \T/\I$ is such that $Hom_{\T/I}(X[j],\pi(B))=0$ for all $B\in \Omega$ and $j\in \mathbb{Z}$, then $X\cong 0$.
\end{enumerate}
\end{defn}
It is immediate to see that the previous definition is equivalent to asking that the collection $\Omega$ maps through $\pi$ to a spanning class on the quotient $\T/\I$. When the thick subcategory in question is the 0 subcategory then the definition reduces to that of a spanning class as in Definition \ref{defn:spanningclass}. \\
Additionally when the triangulated category $\T/\I$ has a Serre functor, only one of the conditions in the definition is necessary as the Serre duality implies the other automatically. \\
We would like to generalize Theorem \ref{lemma:amplespanning} but for a big canonical bundle instead of an ample one and see that a big bundle induces an almost spanning class in the derived category with respect to a $\dtee$-ideal $\I$. \\
\begin{thm}\label{thm:almostspanning}
Let $X$ be a smooth projective variety of general type. Then the collection of tensor powers $(\omega^{\otimes i}_{X})_{i\in \mathbb{Z}}$ forms an almost spanning class with respect to the tensor ideal $I_{X^{\ast}}$ in the tensor triangulated category $(\der, \dtee)$.
\begin{proof}
We need to show that $\pi(\{\omega_{X}^{\otimes i}\})$ forms a spanning class in the quotient $\der/I_{X^{\ast}}$. As $I_{X^{\ast}}$ is the ideal corresponding to the open smooth subvariety $X^{\ast}$ from Corollary \ref{cor:opengentype} then we know that there is an isomorphism $Spc(\der/I_{X^{\ast}})\cong X^{\ast}$. Since $\omega_{X}$ restricted to $X^{\ast}$ is ample by the characterization of Theorem \ref{thm:restrictionisample}, we get that $\{\omega_{X}^{\dteee i}\mid_{X^{\ast}}\}$ forms an spanning class by Lemma \ref{lemma:amplespanning} of the derived category of $X^{\ast}$ which coincides with the quotient category $\der/I_{X^{\ast}}$ by Proposition \ref{prop:verdiertensorfunctor}.
\end{proof}
\end{thm}
The main key in our arguments is the fact that one can construct, as in the ample case, a resolution for any complex of coherent sheaves on $X^{\ast}$ in terms of tensor powers of the canonical bundle $\omega_{X^{\ast}}$ of $X^{\ast}$. With the advantage that one is able to have a concrete description of the derived category of this space in terms of a quotient of the derived category of the larger variety $X$.  \\
Explicitly for any complex $A$ of coherent sheaves over $X^{\ast}$ there is a resolution:
\[ \dots\to \oplus_{j_{0}}(\omega_{X^{\ast}}^{\otimes i_{0}})\to \dots \to \oplus_{j_{k}}(\omega_{X^{\ast}}^{\otimes i_{k}})\to A\to 0.\]
\\
Another thing to notice is that in the example given above for the non equivalent tensor triangulated category structures on $D^{b}(\mathbb{P}^{n})$, one immediate issue with the two given such structures was that the units were non-isomorphic. For this reason we should proceed to work with tensor triangulated categories with a fixed unit isomorphic to $\Ox_{X}$. \\
\begin{defn}\label{defn:invertible}
Let $(\T,\otimes,\mathbbm{1})$ be a TTC, an object $X\in \T$ is $\otimes$-invertible if there exists $X^{-1}\in \T$ such that $X\otimes X^{-1}\cong \mathbbm{1}$. We will denote by $Pic(\der, \boxtimes)$ the group of isomorphism classes of $\boxtimes$-invertible objects.
\end{defn}
We will make use of the following lemma:
\begin{lemma}\label{lemma:picardlemma}
Suppose $X$ is a smooth projective variety of general type of dimension $n$. If $\boxtimes$ is a tensor triangulated structure on $D^{b}(X)$ with unit $\Ox_{X}$, and $U$ is a $\boxtimes$-invertible object  such that $U\boxtimes I_{X^{\ast}} \subseteq I_{X^{\ast}}$. Then there is a natural equivalence between the functors induced by $U\boxtimes \_ $ and $U\dtee\_$ in $D^{b}(X)/I_{X^{\ast}}$.
\begin{proof}
By our previous discussion we know that any complex can be resolved in $\der/I_{X^{\ast}}$ by a resolution
\[ \dots\to \oplus_{j_{0}}(\omega_{X^{\ast}}^{\otimes i_{0}})\to \dots \to \oplus_{j_{k}}(\omega_{X^{\ast}}^{\otimes i_{k}})\to A\to 0.\]
As the Serre functor in $D^{b}(X^{\ast})$ is given by $\_\dtee\omega_{X^{\ast}}[n']$, where $n'$ is the dimension of $X^{\ast}$ and we know any exact equivalence must commute with it, if we let $U\widehat{\boxtimes}$ and $U\widehat{\dtee}$ denote the autoequivalences of $\der/I_{X^{\ast}}$ induced respectively by $U\boxtimes$ and $U\dteee$, then we have that 
\[(U\widehat{\boxtimes} \hat{A})\widehat{\dtee} \omega_{X^{\ast}}[n'] \cong U\widehat{\boxtimes} (\widehat{A}\widehat{\dtee} \omega_{X^{\ast}}[n']).\]
As $\Ox_{X}$ is a unit for both $\otimes_{X}$ and $\boxtimes$, and after shifting by $[-n']$ we deduce
\[ U\widehat{\dtee} \omega_{X^{\ast}} \cong U\widehat{\boxtimes} \omega_{X^{\ast}}. \]
From this, the exactness of $\dteee$ and $\boxtimes$, and the resolutions in terms of $\omega_{X^{\ast}}^{i}$, we obtain the isomorphisms
\[ U\widehat{\dteee}A \cong U\widehat{\boxtimes}A. \]
\end{proof}
\end{lemma}
\begin{obs}
Let us point out the slight abuse of notation of the autoequivalence $U\widehat{\dteee}$. This functor would formally be denoted by $\widehat{U}\otimes^{\mathbb{L}}_{\der/I_{X^{\ast}}}$  as it is induced by the object $\widehat{U}$ in the tensor triangulated category $(\der/I_{X^{\ast}},\otimes^{\mathbb{L}}_{\der/I_{X^{\ast}}})$, but as the only tensor ideal we are taking a quotient by in this section is $I_{X^{\ast}}$, we believe our notation is lighter without losing sight of which functors they represent. 
\end{obs}
We have the following corollary:
\begin{cor}\label{cor:subpicard}
Let $X$ be a variety of general type and let $\boxtimes$ a tensor triangulated category structure on $D^{b}(X)$ with unit $\Ox_{X}$. Then for any $\boxtimes$-invertible object $U$ such that $U\boxtimes I_{X^{\ast}}\subseteq I_{X^{\ast}}$, the equivalence $U\widehat{\boxtimes}:\der/I_{X^{\ast}}\to \der/I_{X^{\ast}}$ induced by $U\boxtimes\_$ is equivalent to an equivalence given by objects in the group $Pic(\der/I_{X^{\ast}},\widehat{\dteee})$ of invertible $\widehat{\dteee}$-objects.
\begin{proof}
From Lemma \ref{lemma:picardlemma} we have that if $U^{-1}$ is such that $U\boxtimes U^{-1} \cong \Ox_{X}$ then in the quotient $\der/I_{X^{\ast}}$, 
\[ U\widehat{\dteee} \widehat{U^{-1}}\cong U\widehat{\boxtimes}\widehat{U^{-1}} \cong \Ox_{X^{\ast}}. \]
As $(\der/I_{X^{\ast}},\widehat{\dteee})$ is a tensor triangulated category, we have that $\widehat{U}\in \der/I_{X^{\ast}}$ is a $\widehat{\dteee}$-invertible objects.
\end{proof}
\end{cor}
In Lemma \ref{lemma:picardlemma} and Corollary \ref{cor:subpicard} above, the ideal $I_{X^{\ast}}$ might not be a $\boxtimes$-tensor ideal and thus the quotient $\der/I_{X^{\ast}}$ does not necessarily carry a tensor triangulated category structure induced by $\boxtimes$. However, our result guarantees that after passing to the quotient, the equivalences induced by the functors $U\boxtimes \_$ are equivalent to equivalences given by invertible objects in $(\der/I_{X^{\ast}},\dtee)$ induced by the same object, under the condition that $I_{X^{\ast}}$ is stable by $U\boxtimes$. \\
In particular, we have:
\begin{cor}\label{cor:bigsubideal}
Let $X$ be a variety of general type and $\boxtimes$ a tensor triangulated structure on $\der$ with unit $\Ox_{X}$. If $I_{X^{\ast}}$ is a $\boxtimes$-ideal then the Picard group $Pic(\der/I_{X^{\ast}},\widehat{\boxtimes})$ is a subgroup of the Picard group $Pic(\der/I_{X^{\ast}},\widehat{\dtee})$.
\begin{proof}
The proof is as in the previous two, if $U$ is in  $Pic(\der/I_{X^{\ast}},\widehat{\boxtimes})$ then it induces an autoequivalence of $\der/I_{X^{\ast}}$ and so it commutes with the Serre functor on $D^{b}(X^{\ast})\simeq \der/I_{X^{\ast}}$. By writing a resolution for any complex $A$ in terms of direct sums of derived tensor powers of $\omega_{X^{\ast}}$ we can use the same argument than in the proof of Lemma \ref{lemma:picardlemma} and we arrive at the isomorphisms
\[ U\widehat{\dteee} A \cong U\widehat{\boxtimes} A.\]
\end{proof}
\end{cor}
\begin{obs}
Let us point out that in the results above we have chosen to work with varieties of general type, but the same argument applies to varieties with big anti-canonical bundle.
\end{obs}
The case when our variety has an ample (anti-)canonical bundle allows us to relate the Picard group of the full derived category to that of any other tensor triangulated category structure on it. \\
The following result follows from the previous argument.
\begin{cor}\label{cor:amplepicard}
Let X be a variety with ample (anti-)canonical bundle. Then if $\boxtimes$ is a tensor triangulated category structure on $D^{b}(X)$ with unit $\Ox_{X}$, the Picard group $Pic(D^{b}(X), \boxtimes)$ is isomorphic to a subgroup of $Pic(D^{b}(X),\otimes_{X})$.
\begin{proof}
We just need to notice that in this case the $\otimes_{X}$-ideal from Corollary \ref{cor:opengentype} is the 0 ideal and thus we can resolve any object $A\in D^{b}(X)$ by a sequence of powers of the Serre functor. By the same reasoning as above we see that 
\[ U\dteee A \cong U\boxtimes A.\]
\end{proof}
\end{cor}
One thing to note here is that although Bondal and Orlov had already classified the group of autoequivalences of a derived category of a variety with ample (anti-)canonical bundle, we are working without the condition of an equivalence between the derived category of the Balmer spectrum of $\boxtimes$ and the derived category $\der$, and as such it is not immediate from their result that the Picard group of $\boxtimes$ must involve invertible sheaves over $X$. \\
In other words, as $Spc(\boxtimes)$ is not necessarily isomorphic to $X$ then understanding the autoequivalences of $\der$ alone does not give us an immediate relationship to the Picard group of $\boxtimes$. \\
We should think of the following corollary as a monoidal version of the Bondal-Orlov reconstruction theorem:
\begin{cor}\label{cor:monoidalbondalorlov}
Let $X$ be as above, then if $\omega_{X}[n]$ is an invertible object for a tensor triangulated structure $\boxtimes$ on $\der$ with unit $\Ox_{X}$ then $\boxtimes$ and $\dtee$ coincide on objects.
\begin{proof}
As $\omega_{X}$ is $\boxtimes$-invertible, Corollary \ref{cor:amplepicard} tells us that for any $A\in \der$ we have
\[ \omega_{X}\dtee A \cong \omega_{X}\boxtimes A. \]
But we can resolve any other complex $B$ in terms of derived powers of the canonical sheaf, by the exactness of $\boxtimes$ we have then 
\[ B\dtee A \cong B\boxtimes A. \]
\end{proof}
\end{cor}
The nature of this result comes precisely from the fact that the tensor triangulated category structure $(\boxtimes, \mathbbm{1})$ does not necessarily come from a derived equivalence $\der\simeq D^{b}(Y)$, while the extra assumption on the unit is required the result is in this direction slightly more general than the original theorem. \\
\begin{cor}\label{cor:isopicard}
Let $X$ be a variety with ample (anti-)canonical bundle, suppose $(\Ox_{X},\boxtimes)$ is a tensor triangulated structure on $\der[X]$ such that $Pic(\boxtimes)\cong Pic(\der[X])$ via the assignment $U\mapsto U$ then $\boxtimes$ coincides with $\dtee$ on objects.
\begin{proof}
In this case if this morphism is an isomorphism, then $\omega_{X}$ is $\boxtimes$-invertible and the result follows from the previous corollary.
\end{proof}
\end{cor}
In fact if we are under the same hypothesis for $X$ then as soon as we are able to show that the generators of $Pic(\der,\dteee)$ are $\boxtimes$-invertible then by the previous corollary there must be an equivalence between $\boxtimes$ and $\dtee$.
\begin{exmp}
Let $X=\mathbb{P}^{n}$ be the projective space, in this case we know that $Pic(D^{b}(X))=\mathbb{Z}\oplus \mathbb{Z}$ corresponding to the line bundles plus their shifts. The result above then says that whenever there is a tensor triangulated structure $\boxtimes$ on $D^{b}(X)$ with unit $\Ox_{X}$ then the Picard group of this tensor structure must necessarily be a subgroup of $\mathbb{Z}\oplus \mathbb{Z}$. \\
If $\omega_{X}=\Ox_{X}(-n-1)$ is $\boxtimes$-invertible then we get that $\boxtimes$ coincides with $\dtee$. Similarly if $\Ox_{X}(-1)$ is $\boxtimes$-invertible.
\end{exmp}
One natural question to ask when working with Picard groups of tensor triangulated category structures is what is the relationship with line bundles on the associated space. In \cite[Proposition 4.4]{balmer2007gluing} Balmer and Favi prove the following result:
\begin{prop}\label{prop:balmergluing}
Let $X$ be a scheme and consider $Perf(X)$ its derived category of perfect complexes. Then there is a split short exact sequence of abelian groups 
\[ 0\to Pic(X)\to Pic(Perf(X),\dtee)\to C(X;\mathbb{Z})\to 0\]
where $C(X;\mathbb{Z})$ stands for the group of locally constant functions from $X$ to $\mathbb{Z}$.
\end{prop}
Again under the hypothesis of $X$ having an ample (anti-)canonical bundle, by using Proposition \ref{prop:balmergluing} we see that for a TTC $(\boxtimes,\Ox_{X})$, if $Spc(\boxtimes)$ is a scheme then the Picard group of $Spc(\boxtimes)$ must be a subgroup of the Picard group of $Pic(\der, \boxtimes)\leq Pic(\der,\dtee)$. So a line bundle in $Spc(\boxtimes)$ has to be $\dtee$-invertible. \\ 
\begin{obs}\label{obs:reconstructiontensorantieu}
From Bondal-Orlov's reconstruction original proof we know that it is actually possible to fully characterize line bundles up to a shift from purely categorical properties. Given the importance of the Picard group of the variety, we can ask whether it is possible to reconstruct the derived tensor product in $\der$ without having to pass through a reconstruction theorem. \\
In \cite{143661} Antieau sketches a construction in which by considering invertible objects ( in the sense of Bondal and Orlov ) one can define a collection of tensor products $\dteee_{U}$ by exploiting the resolution by derived tensor powers of $\omega_{X}$. \\
The idea is to pick an invertible object $U$ which are shown in \cite{bondal2001reconstruction} to be isomorphic to a shift of a line bundle in $X$, then by use of the resolution we only need to define the products $\omega_{X}^{\dtee i}[ni]\dteee_{U}A^{\bullet}$ for any object $A^{\bullet}$. As the Serre functor $S\simeq \_\dtee \omega_{X}[n]$ comes with the categorical structure alone then we can set these products to be simply $S^{i}(A^{\bullet})$. \\
These tensor products $\dteee_{U}$ have $U$ as unit and all of them have $X$ as Balmer spectrum.
\end{obs}
In general for a triangulated category $\T$ we have an action by $Aut(\T)$ on the collection $TTS(\T)$.\\
If $(\otimes,\mathbbm{1})\in TTS(\T)$ and $\phi\in Aut(\T)$ we have a tensor structure defined by \[ X\otimes_{\phi} Y:= \phi^{-1}(\phi(X)\otimes \phi(Y)). \]
And with unit given by $\phi^{-1}(\mathbbm{1})$. \\
We have now justified enough the following definition:
\begin{defn}\label{defn:tens}
Let $\T$ be a triangulated category, denote by $TTS(\T)$ the collection of equivalence classes of tensor triangulated category structures on $\T$. Where we consider two tensor triangulated category structures to be equivalent if there is a monoidal equivalence between the two of them. \\
\end{defn}
To keep some control and avoid counting structures coming from autoequivalences as we saw, we should at least fix the unit object.
\begin{defn}
Let $\T$ be a triangulated category and $U\in \T$ an object. Then the set $TTS_{U}(\T)$ is the set of equivalence classes of tensor triangulated structures on $\T$ where $U$ is the unit.
\end{defn}
It is this set the one we are mainly interested in classifying. \\
Let us finish by discussing the original Bondal-Orlov reconstruction theorem in terms of the results we have shown so far. \\
\begin{thm}\label{thm:newbondalorlov}
Let $X$ be a variety with ample (anti-)canonical divisor, and let $\boxtimes$ be a tensor triangulated structure on $\der$ with unit $\Ox_{X}$. Suppose $Spc(\boxtimes)$ is a smooth projective space with ample (anti-)canonical bundle and that there is an equivalence $\der\simeq D^{b}(Spc(\boxtimes))$, then $X\cong Spc(\boxtimes)$
\begin{proof}
In fact the only thing to note here is that as $Spc(\boxtimes)$ has ample (anti-)canonical bundle then $\omega_{X}$ has to be $\boxtimes$-invertible. Indeed, we recall that one can pick the equivalence $\der\simeq D^{b}(Spc(\boxtimes))$ to send $\omega_{X}$ to $\omega_{Spc(\boxtimes})$ and then the assertion follows by applying Corollary \ref{cor:amplepicard} to $Spc(\boxtimes)$ we obtain that $Pic(\der, \dtee)$ has to be isomorphic via the assignment $\Ll\mapsto \Ll$  to a subgroup of $Pic(\der, \boxtimes)$. Since $\omega_{X}$ is $\boxtimes$-invertible, by Corollary \ref{cor:monoidalbondalorlov} we obtain our result. 
\end{proof}
\end{thm}
\begin{obs}
We need to explain our choice of hypothesis here. On the first hand the assumption that $Spc(\boxtimes)$ is a smooth projective variety is necessary just as in the original Bondal-Orlov theorem formulation. We have added a couple more assumptions, however. We suppose that the (anti-)canonical bundle of $Spc(\boxtimes)$ is also ample to highlight the use of the monoidal structures in the theorem. This hypothesis is however not necessary as it can be directly deduced from the derived equivalence between the two spaces, just as in the original proof of Bondal and Orlov. Alternatively, we can formulate the theorem as follows:
\begin{thm}
Let $X$ be a variety with ample (anti-)canonical divisor, and let $\boxtimes$ be a tensor triangulated structure on $\der$ with unit $\Ox_{X}$. Suppose $Spc(\boxtimes)$ is a smooth projective space, and that we have an equivalence $\der\simeq D^{b}(Spc(\boxtimes))$, then $X\cong Spc(\boxtimes)$.
\end{thm}
Of more importance is perhaps the choice of unit, as we have seen that there are tensor triangulated category structures on the derived category of such a variety which will produce very different spaces under the Balmer reconstruction. This choice of unit allows us to keep some control in the classification of structures producing the same space. \\
A natural next step for future work would be to deal with the possible sort of objects which can be units for such a structure. \\
\end{obs}
\begin{obs}
We wish to point out that there is some nuance in the way in which Bondal-Orlov follows from our results as we make use of some important technical results from the original proof. We expect however that the discussion in this work has provided enough of a justification and motivation for looking at this problem in terms of monoidal structures. 
\end{obs}
We can close our discussion with the following theorem:
\begin{thm}\label{thm:uniquetens}
Let $X$ be a smooth projective variety with (anti-)canonical bundle. Consider a tensor triangulated category structure $\boxtimes$ on $\der$ such that $\Ox_{X}$ is its unit and $Spc(\boxtimes)$ is isomorphic to $X$, then $\boxtimes$ and $\dtee$ coincide on objects.
\end{thm}
This however does not fully classify $TTS_{\Ox_{X}}(\der)$ as we require Balmer's spectrum to be a Fourier-Mukai partner, but there is no reason to expect in general a relationship between the derived category of the Balmer spectrum and the original triangulated category. \\
The lack of morphisms between a space $X$ and the Balmer spectrum $Spc(\boxtimes)$ for some tensor triangulated structure, and thus of functors between the derived categories of these two spaces is one of the obstacles to being able to understand the possible structures $\boxtimes$.