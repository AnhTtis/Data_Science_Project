\section{Introduction}
In \cite{bondal2001reconstruction}, Bondal and Orlov showed that if $X$ is a smooth projective variety over $\mathbb{C}$ with ample (anti-)canonical bundle then its bounded derived category $\der$ completely recovers the space. More precisely, they showed that
\begin{thm}\cite[Theorem 2.5]{bondal2001reconstruction}\label{thm:bonorvreconstruction}
Let X be an irreducible smooth projective variety with ample (anti-)canonical bundle. If $\der\simeq D^{b}(Y)$ for some other smooth algebraic variety Y, then $X \cong Y$.
\end{thm}
This theorem came in contrast with the discovery by Mukai (\cite{mukai1987fourier}) that for an abelian variety $A$, there exists an equivalence as triangulated categories $D^{b}(A)\simeq D^{b}(\hat{A})$ between the bounded derived category of $A$ and the bounded derived category of its dual $\hat{A}$. \\
This observation sparked the study of what is now called Fourier-Mukai partners of a given variety $X$, those varieties which are triangulated equivalent to the bounded derived category of $X$. \\
Bondal and Orlov's reconstruction pointed out that a (birational) geometric condition on the variety can introduce some control on these derived equivalences and with this in mind Kawamata generalized this theorem for varieties with big (anti-)canonical bundle clarifying from a geometric point of view what is the role of this condition on the possible equivalence of derived categories. Namely he showed:
\begin{thm}\cite[Theorem 1.4]{kawamata2002d}
Let $X,Y$ be smooth projective varieties such that there is an equivalence \[\mathcal{F}:D^{b}(X)\overset{\simeq}{\longrightarrow} D^{b}(Y)\] as triangulated categories, then
\begin{enumerate}
    \item dim X = dim Y.
    \item If the canonical divisor $K_{X}$ is nef, so is $K_{Y}$ and there is an equality in the numerical Kodaira dimensions $\nu(X)$ and $\nu(Y)$. 
    \item If X is of general type, then X and Y are birational and furthermore, there is a smooth projective variety $p:Z\to X$, $q:Z\to Y$ such that $p^{\ast}K_{X} \simeq q^{\ast}K_{Y}$.
\end{enumerate}
\end{thm}
This theorem should be understood as a strong indication of a relationship between the birational geometry of a variety and its derived category. \\
On the other hand, Balmer showed in \cite{balmer2002presheaves,bondal2001reconstruction} that when equipped with the derived tensor product $\dtee$, the derived category of perfect complexes $Perf(X)$ of any coherent scheme $X$ can recover the space $X$ by what is now known as the Balmer spectrum $Spc(Perf(X),\dtee)$. The Balmer spectrum can be constructed for a general tensor triangulated category, a triangulated category equipped with a compatible monoidal structure, and produce a locally ringed space. \\
The existence of non isomorphic Fourier-Mukai partners $Y$ for a smooth variety $X$ implies using the Balmer spectrum construction that the bounded derived category $D^{b}(X)$ can be equipped with at least as many tensor triangulated category structures as non-isomorphic Fourier-Mukai partners, up to monoidal equivalence. \\
In other words, if $FM(X)$ is the set of isomorphism classes of Fourier-Mukai partners of $X$ and $TTS(X)$ is the set of equivalence classes of tensor triangulated category structures on the bounded derived category $D^{b}(X)$ there exists an injection
\begin{align*}
FM(X)&\rightarrow TTS(X)\\
Y&\mapsto (\otimes_{Y}^{\mathbb{L}}, \Ox_{Y})
\end{align*}
Where the pair $(\otimes_{Y}^{\mathbb{L}}, \Ox_{Y})$ denotes the tensor triangulated category structure given by the derived tensor product $\otimes_{Y}^{\mathbb{L}}$ with unit $\Ox_{Y}$. \\
Our main interest in this work is the study of this function, its surjectivity and the properties that one can deduce about possible tensor triangulated category structures outside of the image of this injection, all under the condition that the (anti-)canonical bundle of $X$ is big. \\
In Section~\ref{sec2} we give a brief general overview of the results we will need about general derived categories of quasi-coherent sheaves on a smooth projective variety, together with a reminder of the Balmer spectrum construction through Thomason's classification theorem. \\
In Section\ref{section3}, given a tensor triangulated category structure $(\der, \boxtimes,\mathbbm{1})$ with unit $\mathbbm{1}$ on a bounded derived category $\der$, we introduce the notion of almost spanning class with respect to a thick subcategory $I$ (Definition \ref{defn:almostspanning}) and we show (Theorem \ref{thm:almostspanning}) that if $X$ is a smooth projective variety of general type then there exists a proper tensor ideal $I_{X^{\ast}}$ of $(\der,\dtee,\Ox_{X})$ such that the set of tensor powers of $\omega_{X}$ forms an almost spanning sequence with respect to this ideal $I_{X^{\ast}}$.
This result is meant to highlight the more general behavior of almost spanning classes through the use of Thomason's classification theorem and properties of the Balmer spectrum. We see that this collection of objects can be used to prove the following theorem:
\begin{lemma}(Lemma \ref{lemma:picardlemma})
Suppose $X$ is a smooth projective variety of general type. If $\boxtimes$ is a tensor triangulated structure on $D^{b}(X)$ with unit $\Ox_{X}$, and $U$ is a $\boxtimes$-invertible object  such that $U\boxtimes I_{X^{\ast}} \subseteq I_{X^{\ast}}$. Then there is a natural equivalence between the functors induced by $U\boxtimes \_ $ and $U\dtee\_$ in $D^{b}(X)/I_{X^{\ast}}$.
\end{lemma}
When the $\dtee$-tensor ideal $I_{X^{\ast}}$ is also a $\boxtimes$-tensor ideal for a tensor triangulated category structure as described in the previous lemma, then we obtain that the Picard group of $\boxtimes$-invertible objects is a subgroup of the Picard group of $\dtee$-invertible objects (Corollary \ref{cor:bigsubideal}). This hypothesis holds true in particular when the (anti-)canonical bundle of $X$ is ample. \\
With this observation, our main corollary is the following monoidal version of the Bondal-Orlov reconstruction theorem:
\begin{cor}(Corollary \ref{cor:monoidalbondalorlov})
Let $X$ be a smooth projective variety with ample (anti-)canonical bundle, then if $\omega_{X}[n]$ is an invertible object for a tensor triangulated structure $\boxtimes$ on $\der$ with unit $\Ox_{X}$ then $\boxtimes$ and $\dtee$ coincide on objects.
\end{cor}
The results in this work were obtained as part of the author's PhD thesis at the Laboratoire J.A. Dieudonné at the Université Côte d'Azur. The author would like to thank his advisor Carlos Simpson for many discussions and to Ivo Dell'Ambrogio and Bertrand Toën for their careful and valuable comments on the thesis manuscript. The PhD thesis was partially financed by the CONACyT-Gobierno Francés 2018 doctoral scholarship. 
