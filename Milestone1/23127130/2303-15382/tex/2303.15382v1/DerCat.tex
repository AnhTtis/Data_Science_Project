\section{Derived categories and the Balmer reconstruction}\label{sec2}
Through the rest of this work we will be working exclusively with smooth projective varieties over $\mathbb{C}$. We will from now on omit the mention of the base field in our exposition. Some of the material presented in this section can be found in deeper detail in \cite{huybrechts2006fourier}. \\
The goal of this section is to introduce the basic results and notions we will be using for our results. \\
Let us start by recalling that if $X$ is a smooth projective variety then there exists an equivalence as triangulated categories between the derived category $Perf(X)$ of perfect complexes on $X$ and the bounded derived category $\der$. As a consequence of this whenever we work with such a variety we will at times make no distinction between these two categories. \\
One important feature of these categories is the existence of Serre functors, let us recall:
\begin{defn}
Let $\T$ be a triangulated category an autoequivalence $S:\T\to \T$  satisfying $Hom(A,B)\cong Hom(B,S(A))^{\ast}$ for all objects $A,B\in \T$, is called a Serre functor.
\end{defn}
\begin{exmp}\label{exmp:serrefunctor}
Specifically if for example the triangulated category is a derived category of a smooth projective scheme of dimension n, we have Grothendieck-Verdier duality which implies that for every pair of objects $M,N\in\der$, $Hom(M,N)=Hom(N,M\otimes \omega_{X}[n])^{\ast}$ where $\omega_{X}$ is the canonical bundle of X. 
\end{exmp}
This notion was first defined by Kapranov and Bondal in \cite{bondal1990representable}. The following two properties of the Serre functor are essential to our work:
\begin{lemma} \label{serrecommutes}(Proposition 1.3 \cite{bondal2001reconstruction})
Let $\T$ be a triangulated category with Serre functor S, and let $\psi:\T\to \T$ be any autoequivalence, then $\psi \circ S \cong S \circ \psi$.
\end{lemma}
\begin{prop}\label{thm:uniqueserrefunctor}(Proposition 3.4 \cite{bondal1990representable})
Let $\T$ be a triangulated category and let $S$ be a Serre functor in $\T$, then it is unique up to graded isomorphism.
\end{prop}
This latter proposition implies that whenever the Serre functor exists it is part of the data of the given category. In our case of interest, as one can write this functor using the derived tensor product $\dtee$ we have now some possible control on the monoidal structure $\dtee$ directly from the category without knowledge of $X$. \\
Another crucial notion we will use is that of spanning classes, we recall the definition:
\begin{defn}\label{defn:spanningclass}
A collection of objects $\{X_{i}\}\subseteq \T$ of a triangulated category is called a spanning class if:
\begin{enumerate}
    \item If $Hom(X_{i},D[j])=0$ for all $i,j$ then $D\simeq 0$
    \item If $Hom(D[j],X_i)=0$ for all $i,j$ then $D\simeq 0$
\end{enumerate}
\end{defn}
However, whenever the Serre functor exists in the triangulated category we see that only one of the conditions is necessary and the other will be automatically satisfied by use of the Serre functor isomorphism. \\
A general way to produce spanning classes in derived categories of abelian categories is from ample sequences:
\begin{defn}
We call a collection of objects of an abelian category $\mathcal{A}$, $\{ L_{i} \}\subset \mathcal{A}$ an ample sequence if the following conditions are met:
For $i<< 0$, and all $A\in \mathcal{A}$
\begin{enumerate}
    \item $Hom(L_{i},A)\otimes_{k} L_{i}\to A$ is surjective. 
    \item $Hom(A,L_{i})=0$
    \item $Ext^{j}(L_{i},A)=0$, $j\not=0$
\end{enumerate}
\end{defn}
As the name suggest, an important example of such sequences comes from collections of tensor powers of ample line bundles. The relation between the two notions of spanning class and ample sequence was shown by Bondal and Orlov in the following result:
\begin{lemma}\label{lemma:amplespanning}
Let $\mathcal{A}$ be an abelian category of finite homological dimension and let $\{L_{i}\}$ be an ample sequence, then the collection $\{L_{i}\}$ seen as objects of $\mathcal{D}(\mathcal{A})$ form a spanning class.
\end{lemma}
The following example illustrates how we should be exploiting the existence of ample sequences. 
\begin{exmp}\label{exmp:resolvebyample}
Let $X$ be a smooth projective variety with ample canonical bundle. Then the set $\{\omega_{X}^{\otimes_{X}} \}$ forms an ample sequence and so by the previous lemma it forms a spanning class in the derived category $\der$. \\
As a consequence, we see that any complex $\F$ of coherent sheaves can be resolved by tensor powers of the canonical bundle $\omega_{X}$. In other words, there exists a sequence:
\[ 0\to \oplus_{j_{0}}(\omega_{X}^{\otimes i_{0}})\to \dots \to \oplus_{j_{k}}(\omega_{X}^{\otimes i_{k}}) \to \F\to 0\]
\end{exmp}
\begin{obs}\label{obs:resolvebyautoeq}
We remark too that in general for a triangulated category $\T$ with a spanning class $\Omega\subset \T$, if $\phi:\T\to \T$ is an autoequivalence then the set $\phi(\Omega)$ is too a spanning class. \\
In the example above, this remark implies that one can resolve any complex $\F$ by tensor powers of sheaves of the form $\omega_{X}(i)[j]$ for a fixed $i,j\in \mathbb{Z}$.
\end{obs}
\subsection{Tensor triangulated geometry}
When dealing with derived categories of coherent sheaves on a variety one can equip this category with a monoidal structure given by the derived tensor product. One can axiomatize this sort of structure in what is known as a tensor triangulated category. \\
In this subsection we recall Balmer's spectrum construction which inputs a tensor triangulated category and outputs a locally ringed space which as we will see recovers a variety whenever we work with the derived category of perfect complexes on said variety.
\begin{defn}\label{defn:ttc}
A tensor triangulated category (TTC for short) $\T$ is a triangulated category together with the following data: \\
\begin{enumerate}
    \item A closed symmetric monoidal structure given by a functor $\otimes:\T\times \T\to \T$ additive and exact ( with respect to the k-linear structure ) on both entries.
    \item The internal Hom functor $\underline{hom}:\T\times\T\to \T$ sends triangles to triangles ( up to a sign ). 
    \item Coherent natural isomorphisms for each n and m in $\mathbb{Z}$, $r:x\otimes (y[n])\cong (x\otimes y)[n]$ and $l:(x[n])\otimes y \cong (x\otimes y)[n]$ compatible with the symmetry, associative and unit coherence morphisms from the symmetric monoidal category structure. (See for example \cite[Section 2.1.1]{dell2016triangulated} for the explicit diagrams).
\end{enumerate}
\end{defn}
We will refer to a TTC by the triple $(\T,\otimes,\mathbbm{1}_{\T})$ where $\otimes$ refers to the monoidal structure and $\mathbbm{1}_{\T}$ to the unit object. Often if there is no confusion or the unit plays no role we will omit it and write $(\T,\otimes)$ instead. \\
At times when we deal with a fixed underlying triangulated category $\T$ we will write $\otimes$ or $(\otimes,\mathbbm{1})$ to refer to a tensor triangulated category structure on $\T$.
Let us remark however that the functor $\otimes$ and unit $\mathbbm{1}_{\T}$ do not completely determine a tensor triangulated category since the compatibility conditions in the symmetric monoidal category structure can in principle change while maintaining the functor $\otimes$ and unit $\mathbbm{1}$. As we will explain in the following this does not represent a problem for our purposes.  \\
We proceed with a number of definitions.
\begin{defn}\label{def:thick}
Let $\T$ be a triangulated category, and $\I\subseteq \T$ a full triangulated subcategory, we say that it is thick if it is closed under direct summands. So that if $A\oplus B \in \I$ then $A,B\in \I$.
\end{defn}
\begin{defn}
Let $(\T,\otimes)$ be a TTC. We will say that a thick subcategory $\I\subset \T$ is a $\otimes-\text{ideal}$ if for every $A\in \T$ we have $A\otimes \I\subset \I$
\end{defn}
\begin{defn}
Let $(\T,\otimes)$ be a tensor triangulated category. Let $\I$ be a $\otimes$-ideal, we will say that it is prime if for any $A,B\in \T$ with $A\otimes B\in \I$ then $A\in \I$ or $B\in \I$.
\end{defn}
As in affine algebraic geometry we can define the spectrum of a tensor triangulated category. 
\begin{defn}
Let $(\T,\otimes,\mathbbm{1})$ be a tensor triangulated category, the set of all prime $\otimes$-ideals will be denoted by $Spc(\T,\otimes,\mathbbm{1}))$ (alternatively $Spc(\T)$, $Spc(\otimes,\mathbbm{1})$ or $Spc(\otimes)$ depending on which information is clear from context).
\end{defn}
Importantly, whenever the triangulated category $\T$ is non-zero we have that $Spec(\T,\otimes)\not=\emptyset$ for any tensor triangulated category structure $\otimes$ we can put on $\T$ (see \cite[Proposition 2.3]{balmer2005spectrum}). \\
To this set we will put a topology structure.
\begin{defn}
Let $(\T,\otimes, \mathbbm{1})$ be a TTC, the support of an object $A\in \T$, denoted $supp(A)$, is the set $\{\mathfrak{p}\in Spc(\T)\mid A\not\in \mathfrak{p}\}$.
\end{defn}
\begin{lemma}\cite[Lemma 2.6]{balmer2005spectrum}
The sets of the form $\mathcal{Z}(S):=\bigcap_{A\in S} supp(A)$, for a family of objects $S\subset \T$, form a basis for a topology on $Spc(\T)$.
\end{lemma}
An important result regarding this topology is the following, which restricts the kind of spaces we should be expecting from the construction.
\begin{thm}\cite[Propositions 2.15,2.18]{balmer2005spectrum}
For any TTC $(\T,\otimes, \mathbbm{1})$, the space $Spc(\T)$ is a spectral space in the sense of Hochster, meaning it is sober and has a basis of quasi-compact open subsets.
\end{thm}
Now that the topology on $Spc(\T)$ has been chosen, the next step is to equip this space with sheaf of rings which will act as the structure sheaf. \\
To a subset $Y\subset Spc(\T)$ we can assign a thick $\otimes$-ideal denoted by $\I_{Y}$ and defined as the subcategory supported on Y, meaning $\I_{Y}:=\{A\in \T\mid supp(A)\subset Y\}$. \\
Finally, with Y as above, we denote by $\mathbbm{1}_{T_{Y}}$ the image of the unit $\mathbbm{1}$ of $\T$ under the localization functor $\pi:\T\to \T/\I_{Y}$.
\begin{defn}\label{defn:structuresheafpsc}
Let $\T$ be a nonzero TTC and we define a structure sheaf $\mathcal{O}_{Spc(\T)}$ over $Spc(\T)$ as the sheaffification of the assignment $U\mapsto End(\mathbbm{1}_{T_{Z}})$ where $Z:=Spc(\T)\backslash U$, for an open subset $U\subset Spc(\T)$.
\end{defn}
It is not hard to see the assignment $Spc(F)$ respects composition of exact monoidal functors, so if $F:\T\to \T'$ is such a functor, we get a morphism of ringed spaces since for a closed $Z= Spc(\T)\backslash U$ we have $F(\I_{Z})\subset \I_{Z'}$ where $Z'=Spc(\T')\backslash Spc(F)^{-1}(U)$ which implies there is a morphism $\mathcal{O}_{\T}\to Spc(F)_{\ast}\mathcal{O}_{\mathcal{L}}$ and so $Spc:\mathbb{TTC}\to RS$ is a functor, and under nice conditions ( for example $\T$ being rigid ) this can be shown to be a functor $Spc:\mathbb{TTC}\to LRS$.  \\
With this construction in mind we can now describe the anticipated reconstruction theorem as described by Balmer. 
\begin{thm}\cite[Corollary 5.6]{balmer2005spectrum}\label{thm:homeoperf}
    Let $X$ be a quasi-compact and quasiseparated scheme. There is a homeomorphism \[f:X\overset{\cong}{\longrightarrow} Spc(Perf(X),\dtee).\]
\end{thm}
This homeomorphism follows from Thomason's classification theorem \cite[Theorem 3.15]{thomason_1997} which establishes a correspondence between certain subsets of a quasicompact and quasi-separated scheme $X$ and $\dtee$-ideals of $Perf(X)$. The following is a general version of this classification for tensor triangulated categories as presented by Balmer in \cite[Theorem 4.10]{balmer2005spectrum}
\begin{thm}\label{thm:balmerthomasonclassification}
Let $(\T,\otimes,U)$ be a TTC. Let $\mathscr{S}$ of those subsets $Y\subset Spc(\T)$ which are unions $Y=\bigcup_{i\in I} Y_{i}$ where $Y_{i}$ are closed subsets with quasi-compact complement for all $i\in I$. Let $\mathscr{R}$ be the set of radical $\otimes$-ideals of $\T$. Then there is an order-preserving bijection $\mathscr{S}\to \mathscr{S}$ given by the assignment which sends $Y$ to the subcategory $\T_{Y}:=\{A\in \T\mid supp(A)\subset Y\}$ and with inverse sending a radical $\otimes$-ideal $\I$ to the subset $S_{\I}:=\bigcup_{A\in \I} supp(A)$.
%Let $X$ be a quasicompact and quasi-separated scheme. Denote by $\mathscr{C}$ the set of $\dtee$-ideals of the derived category $Perf(X)$ of perfect complexes on X. \\
%Denote by $\mathscr{S}$ the set of subspaces $Y\subset X$ such that $Y=\bigcup Y_{\alpha}$ where the $Y_{\alpha}$ are closed subspaces such that $X\backslash Y$ is quasi-compact. \\
%There is a bijective correspondence between $\mathscr{C}$ and $\mathscr{S}$. \\
%The bijection is given by the assignment which sends a subspace $Y\in \mathscr{S}$ to the $\dtee$-ideal whose objects are those perfect complexes $E^{\bullet}$ such that $Supph(E^{\bullet}):=\bigcup Supp(H^{i}(E^{\ast}))$ is a subset of $Y$. On the other direction it sends a $\dtee$-ideal $\I\in \mathscr{S}$ to the subspace $Y=\bigcup_{E^{\bullet}\in \I} Supph(E^{\bullet})$.
\end{thm}
Here by radical $\otimes$-ideal we mean a $\otimes$-ideal $\I$ such that whenever $A^{\otimes n}$ is in $\I$ then $A$ is in $\I$. \\
In practice every $\otimes$-ideal is automatically a radical $\otimes$-ideal and it certainly depends on the monoidal structure one can put on the triangulated category $\T$. As pointed out by Balmer in \cite{balmer2005spectrum} this condition is satisfied as soon as the tensor triangulated category is rigid, meaning that every object is dualizable. \\
When $X$ is a variety the classification theorem can be specialized to a very simple form as pointed out by Rouquier in \cite{rouquier2003categories}.
\begin{thm}\label{thm:roquierthomason}
    Let $X$ be a variety, there is a correspondence between the set of closed subsets of $X$ and $\dtee$-ideals of finite type, those ideals generated by a single object. 
\end{thm}
Using the homeomorphism from Theorem \ref{thm:homeoperf} and the construction of the structure sheaf on $Spc(Perf(X),\dtee)$ from Definition \ref{defn:structuresheafpsc} we only need the following theorem to complete the reconstruction theorem of Balmer.
\begin{thm}\cite{balmer2002presheaves}\label{thm:balmerstructuresheaf}
Let $X$ be quasi-compact and quasiseparated scheme X. There is an isomorphism $\Ox_{X}\cong \Ox_{Spc(Perf(X),\Ox_{X})}$.
\end{thm}
The following proposition should inform us how localizations behave under taking $Spc$. 
\begin{prop}\cite[Proposition 3.11]{balmer2005spectrum}\label{prop:verdiertensorfunctor}
Let $\I\subset\T$ be thick $\otimes$-ideal, then the localization functor $\pi:\T\to \T/\I$ is an exact monoidal functor and induces an homeomorphism $Spc(\T/\I)\cong \{\mathfrak{p}\in Spc(\T)\mid \I\subset \mathfrak{p}\}$.
\end{prop}
In particular when combined with the classification theorem in the form of Theorem \ref{thm:roquierthomason} we see that open subvarieties $U$ of a variety $X$ are isomorphic to $Spc(Perf(X),\dtee)/\I_{Z}$ where $Z$ is the complement of $U$ in $X$.\\
We close this section with the following remark.
\begin{obs}
So far we have been dealing with tensor triangulated categories as described in the Definition \ref{defn:ttc}, meaning we require there to be a closed symmetric monoidal category structure on $\T$. However under closer inspection one sees that nowhere in the classification theorem nor in Balmer's construction one needs the full monoidal structure. \\
In fact so far we really only need the data of a functor $\otimes:\T\times \T\to \T$ covariant and exact in each variable, together with a unit object $\mathbbm{1}$ and isomorphisms corresponding to the symmetric, associative and unit conditions. In other words, if $(\T,\otimes,\mathbbm{1})$ and $(\T,\boxtimes,\mathbbm{1}')$ are two tensor triangulated categories with underlying triangulated category $\T$ such that $\otimes \simeq \boxtimes$ for every pair of objects in $\T$, and $\mathbbm{1}\simeq \mathbbm{1}'$then the Balmer spectra $Spc(\T,\otimes,\mathbbm{1})\cong Spc(\T,\boxtimes,\mathbbm{1}')$ as locally ringed spaces. The associators, unitors and braidings of the monoidal categories have no influence in the resulting space. \\
It is this that justifies our notation $(\T,\otimes,\mathbbm{1})$ as we have mentioned before. In the following we shall keep referring to tensor triangulated categories although our results apply for slightly more general but more awkward structures. 
\end{obs}
