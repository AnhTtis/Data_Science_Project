\documentclass[reprint]{revtex4-2}

\usepackage{graphicx}% Include figure files
\usepackage{dcolumn}% Align table columns on decimal point
\usepackage{bm}% bold math
\usepackage{svg} % import svg images
\usepackage{lipsum} 
\usepackage{amsmath}
\usepackage[acronym,shortcuts]{glossaries}
\usepackage{url}
\usepackage[breaklinks,colorlinks=true,allcolors=blue]{hyperref}

\newcommand{\chris}[1]{\textit{\textcolor{blue}{CP: #1}}}

\begin{document}

\preprint{APS/123-QED}

\title{Free-induction-decay magnetic field imaging with a microfabricated Cs vapor cell}


\author{D. Hunter\textsuperscript{1}} 
\author{C. Perrella\textsuperscript{2}}
\author{A. McWilliam\textsuperscript{1}}
\author{J. P. McGilligan\textsuperscript{1}}
\author{M. Mrozowski\textsuperscript{1}}
\author{S. J. Ingleby\textsuperscript{1}}
\author{P. F. Griffin\textsuperscript{1}}
\author{D. Burt\textsuperscript{3}}
\author{A. N. Luiten\textsuperscript{2}}
\author{E. Riis\textsuperscript{1}}

\affiliation{\textsuperscript{1}Department of Physics, SUPA, University of Strathclyde, Glasgow G4 0NG, United Kingdom}
\email{d.hunter@strath.ac.uk}
\affiliation{\textsuperscript{2}Institute for Photonics and Advanced Sensing (IPAS), and School of Physical Sciences, University of Adelaide, South Australia 5005, Australia}
\affiliation{\textsuperscript{3}Kelvin Nanotechnology, University of Glasgow, Glasgow G12 8LS, United Kingdom}

\date{\today}% It is always \today, today,
             %  but any date may be explicitly specified

\begin{abstract}
Magnetic field imaging is a valuable resource for signal source localization and characterization. This work reports an optically pumped magnetometer (OPM) based on the free-induction-decay (FID) protocol, that implements microfabricated cesium (Cs) vapor cell technology to visualize the magnetic field distributions resulting from various magnetic sources placed close to the cell. The slow diffusion of Cs atoms in the presence of a nitrogen (N$_{2}$) buffer gas enables spatially independent measurements to be made within the same vapor cell by translating a $175\,\mu$m probe beam over the sensing area. For example, the OPM was used to record temporal and spatial information to reconstruct magnetic field distributions in one and two dimensions. The optimal magnetometer sensitivity was estimated to be 0.43\,pT/$\sqrt{\mathrm{Hz}}$ within a Nyquist limited bandwidth of $500\,$Hz. Furthermore, the sensor's dynamic range exceeds the Earth's field of approximately $50\,\mu$T, which provides a framework for magnetic field imaging in unshielded environments.    
\end{abstract}

\maketitle

\section{Introduction}
Magnetic imaging encompasses a diverse range of research fields including biomedical science \cite{yang2021new, alem2015fetal, colombo2016four}, current density imaging \cite{hu2020sensitive, bason2022non}, and magnetorelaxometry in magnetic nanoparticles \cite{richter2010magnetorelaxometry, jaufenthaler2021pulsed}, to name a few. Such applications require sensitive instrumentation capable of resolving the spatial and temporal properties of the magnetic sources of interest. For example, living biological specimens have already been imaged at nanometer resolution with diamond nitrogen-vacancy (NV) centers under ambient conditions \cite{balasubramanian2008nanoscale,le2013optical}. The placement of NV centers in close proximity to the sample can greatly benefit such measurements; however, achieving competitive sensitivities is difficult without extensive effort toward fabricating crystals with low impurities \cite{taylor2008high}. Alkali-vapor based radio-frequency magnetometers are commonly implemented for defect detection in metallic objects \cite{deans2016electromagnetic, bevington2019enhanced}, a capability particularly suited for industrial and defense applications. These devices can reach fT-level sensitivities \cite{savukov2005tunable, lee2006subfemtotesla, chalupczak2012room}, especially when operated at higher frequencies where $1/f$ technical noise contributions are suppressed, e.g., laser frequency and intensity drifts. Despite their highly tunable bandwidth, the narrow magnetic resonance linewidth necessary for achieving high sensitivity operation limits the dynamic range. Thus, active or passive field compensation is widely used, inhibiting their compatibility in unshielded environments. \\   
\indent Recent years have seen emphasis placed on miniaturizing OPMs, particularly those operating in the spin-exchange relaxation-free (SERF) regime \cite{osborne2018fully, kitching2018chip}. These devices are already commercially available and have been widely used in the development of magnetoencephalography (MEG) applications; however, they require extensive magnetic shielding in order to function due to their limited dynamic range. Additionally, OPM array density is often constrained by crosstalk between neighbouring sensors, although recent advancements in compact bi-planar coil design and fabrication techniques can aid in alleviating these issues \cite{tayler2022miniature}. Therefore, most OPM-based MEG experiments provide centimeter-scale spatial resolution as each module is self-contained and requires field modulation to operate \cite{boto2018moving, tierney2019optically}. Deploying multiple sensors within a single vapor cell could be advantageous in such applications to increase the spatial resolution \cite{kim2019magnetocardiography}. This method requires fewer vapor cells and allows one to perform gradiometry within the same cell resulting in high common-mode noise rejection \cite{borna2020non}. \\
\indent The OPM modality adopted in this work is based on the FID measurement protocol \cite{hunter2018free, hunter2018waveform, grujic2015sensitive}, as this technique offers distinct benefits toward magnetic imaging applications. For example, as a total-field sensor, its dynamic range extends beyond the Earth's field ($\sim 50\,\mu$T) whilst maintaining $\mathrm{fT/\sqrt{Hz}}$ sensitivities competitive with zero-field OPM strategies \cite{gerginov2020scalar, limes2020portable}. Furthermore, the sensor bandwidth can extend across a wide range (several kHz), and is highly tunable given the flexibility in the digital signal processing (DSP) techniques that can be used to extract the magnetic field information \cite{hunter2018waveform, wilson2020wide}. This is in contrast to SERF sensors whose bandwidth is typically limited to below $200\,$Hz \cite{osborne2018fully}, owing to the long coherence times needed to reach optimal sensitivity. A major advantage of the FID modality lies in its accuracy \cite{hunter2022accurate}, as the pumping light is switched off during detection which significantly reduces the fictitious magnetic field generated by light shifts. These low systematics reduce the potential sensor background in magnetic imaging experiments. Moreover, the FID approach is extremely robust as it can easily operate in a free-running mode without the use of feedback loops. This enables direct measurement of the Larmor frequency with no prerequisite knowledge of magnetic field required.  \\
\indent This study employs micro-electro-mechanical-systems (MEMS) cells filled with Cs vapor and N$_2$ buffer gas, to image the magnetic field distribution produced from various current configurations in proximity to the cell. This is accomplished by translating the probe beam across the vapor cell and recording the magnetic field experienced by the alkali spins. In contrast to paraffin coated vapor cells, in which atoms sample the entire cell volume during measurement, this work exploits the slow alkali spin diffusion inherent to cells with buffer gas. First, the magnetometer's response is verified by applying a well-defined gradient across the sensing volume. Additionally, higher complexity magnetic field distributions produced by current flowing through a copper wire configured in different orientations are compared to theoretical expectations based on the Biot-Savart law. Furthemore, the sensor was able to track the full temporal and spatial characteristics of a magnetic source driven by an oscillating current, constructing a two-dimensional magnetic field image in a bias field of approximately $50\,\mu$T.

\section{FID Magnetometry}
\subsection{Experimental Setup}
A schematic representation of the experimental setup is depicted in Fig.~\ref{Experimental setup}(a). The sensor head consists of a MEMS Cs vapor cell, shown in Fig.~\ref{Experimental setup}(b), formed from a glass-silicon-glass anodically bonded stack. The 3~mm thick silicon wafer is water-jet cut to realize internal dimensions of $6\times6\times3\,\mathrm{mm^3}$ \cite{dyer}. Following the first glass-silicon bond, an aqueous solution of cesium azide (CsN$_3$) is deposited within the cells. The final bond takes place under vacuum with a back-filled N$_2$ environment. Following decomposition of the azide under ultra-violet light, the cell measures a total buffer gas pressure of approximately $220\,$torr. This was measured from the collisional broadening and shift in the optical spectrum with respect to a Cs reference cell \cite{hunter2018free}. \\
\indent The Cs MEMS cell is mounted between a set of printed circuit boards (PCBs) used to resistively heat the vapor to a temperature of $88\,^{\circ}$C. There is no current flowing through the heating element during detection to avoid stray magnetic field contributions. In order to suppress magnetic technical noise from the environment, the sensor head is enclosed inside a three-layer $\mu$-metal shield. It is housed within a set of three-axis field coils that are driven by a highly stable programmable current driver to provide complete control of the magnitude and direction of the bias field \cite{mrozowski2023ultra}. Each magnetic source used for imaging was also powered by the same current source except in instances where modulation was required. In this case, a low-noise waveform generator (Keysight 33600A) was used in series with a resistor with $15\,$ppm temperature stability. \\
\begin{figure}
	\centering	
        \includegraphics[scale = 0.95]{FID_imaging_experimental_setup.png}
	\caption{(a) Simplified experimental setup: GT, Glan-Thompson polarizer; NPBS, non-polarizing beamsplitter; PD, photodiode; M, mirror; L, lens; TS, translation stage; PBS, polarizing beamsplitter; DWP, dual-wavelength waveplate; HWP, half-wave plate; BPF, bandpass filter; WP, Wollaston prism; BPD, balanced photodetector. (b) Cs MEMS vapor cell with approximately $220\,$torr N\textsubscript{2} buffer gas and $6 \times 6 \times 3\,$mm$^3$ cavity dimensions. (c, d) Wires placed in a ``s" and ``cross" configuration used as a magnetic source for one- and two-dimensional (1D and 2D) imaging, respectively. The white arrows indicate the positive current flow paths.}
	\label{Experimental setup}
\end{figure}
\indent Optical pumping and probing were performed using separate co-propagating laser beams tuned to different atomic transitions. The pump laser (LD852-SEV600) is single frequency and can produce up to $600\,$mW of optical power. It is set to be resonant with the $F = 3 \longrightarrow F^{\prime}$ transition on the Cs $\mathrm{D_2}$ line. The optical line is collisionally broadened by the N$_{2}$ buffer gas to approximately $3.7\,$GHz. This transition is used to optically pump the atomic population from the $F = 3$ hyperfine ground state such that these atoms can subsequently contribute to the signal \cite{schultze2015improving}, while partially suppressing spin exchange through light narrowing \cite{scholtes2011light}. The pump light was modulated using an acousto-optic modulator (AOM) with approximately $65\,$mW peak optical power available, after beam conditioning and a fiber-coupling stage, which irradiates the vapor cell over a $3.1\,$mm ($1/\mathrm{e}^2$) beam diameter. The extinction ratio of the AOM reduces residual pump light to $<10\,\mu$W when off, limiting interaction with the atoms during detection. The spatial resolution of the magnetometer is governed by the waist of the probe beam which was focused down to $175\,\mathrm{\mu{m}}$ ($1/\mathrm{e}^2$), resulting in an elevated optical intensity within the vapor cell. Consequently, the probe laser (DBR895PN) frequency was $60\,\mathrm{GHz}$ blue-detuned from the $F = 4 \longrightarrow F^{\prime}$ transition of the Cs $\mathrm{D_1}$ line to avoid excessive residual optical pumping from broadening the magnetic resonance during detection. The direction of frequency detuning is not significant in this case given that most of the atoms occupy the $F=4$ ground state after optical pumping and the light is far from resonance. The significant detuning has the additional benefit of reducing light shift systematics to $0.5\,\mathrm{pT/\mu{W}}$, determined experimentally, for improved measurement accuracy. A probe power of around $1\,$mW was chosen to maximize the signal amplitude while maintaining photon shot-noise level performance. The Glan-Thompson polarizer sets a defined polarization for the probe beam by converting polarization noise into amplitude noise, including that originating from the optical fiber. The light is then split equally between the vapor cell and a monitor photodiode with a non-polarizing beamsplitter. This enables the probe intensity to be actively stabilized by adjusting the voltage supplied to the AOM with an analog proportional-integral-derivative controller (SRS SIM960). Prior to illuminating the vapor cell, a polarizing beamsplitter combines the pump and probe light which then traverse a dual-wavelength multi-order waveplate to convert the $852\,$nm light to circular polarization while the $895\,$nm beam remains linear. This optimizes the optical pumping efficiency whilst maximizing optical rotation of the near-resonant probe beam passing through the vapor. 

\subsection{Sensor Performance}
Larmor precession induces an oscillating birefringence that is detectable through optical rotation of the linearly polarized probe. A balanced polarimeter detects this rotation and cancels common-mode noise sources such as laser intensity and frequency noise, facilitating photon shot-noise level operation. The bandpass filter impedes residual pump light exiting the cell from reaching the detector. The ensuing analog signal is digitized by a data acquisition system consisting of a Picoscope (model 5444D) sampling at 125 MHz, which is downsampled, by averaging 25 successive data points, to 5 MHz for further processing. Figure \ref{FID magnetometer sensitivity}(a) shows a snapshot of a typical FID signal train captured by the polarimeter in the presence of a $49.4\,\mu$T bias field. The pump-probe cycle repetition period was set to $1\,$ms with close to $110\,\mu$s dedicated to optically pumping the alkali spins. Consequently, the magnetic field data is streamed at a $1\,$kHz sampling rate resulting in a Nyquist limited bandwidth of $500\,$Hz. \\
\indent The Larmor frequency was determined by fitting each FID trace to a damped sinusoidal model with the Levenberg-Marquardt algorithm \cite{hunter2018free}. The inset in Fig. \ref{FID magnetometer sensitivity}(a) shows two oscillations from a fitted FID signal, resulting in a Larmor frequency of $172.9\,$kHz. An optimized magnetometer sensitivity of $0.43\,$pT/$\sqrt{\mathrm{Hz}}$ was calculated from the sensitivity spectrum shown in Fig. \ref{FID magnetometer sensitivity}(b). This spectrum was computed using Welch's method \cite{welch1967use}, by averaging the magnetic field time series generated from 40 subsequent FID signals trains. The peak observed at $376\,$Hz was produced by sending current through the wire configuration depicted in Fig. \ref{Experimental setup}(d), later used as a magnetic source for imaging. Technical noise peaks can also be seen between $20-60\,$Hz as a result of magnetic noise in the environment penetrating the $\mu$-metal shield. \\
\begin{figure}
  \centering
  \includegraphics{FID_magnetometer_sensitivity.png}
  \caption{(a) Subsection of a FID signal train captured by the polarimeter in a bias field of $49.4\,\mu$T. The repetition cycle rate was set to $1\,$kHz with the alkali spins optically pumped for around $110\,\mu$s. A Larmor frequency of $172.9\,$kHz was recorded (see inset) by fitting each FID trace to a damped sinusoidal model \cite{hunter2018free}. (b) Optimal magnetic field amplitude spectral density (ASD) computed via Welch's method \cite{welch1967use}, using the extracted Larmor frequency values from a sequence of FID signal trains. The peak observed at $376\,$Hz originates from an oscillating current flowing through a wire used as a magnetic imaging source.}
  \label{FID magnetometer sensitivity}
\end{figure}
\indent The magnetometer's spatial resolution is primarily influenced by the probe beam waist within the interrogation region, with the lower bound constrained by the diffraction limit. However, the role of spin diffusion should also be considered which is dictated by the buffer gas pressure inside the cell. One can consider a buffer gas vapor cell as an array of locally independent sensors whose size is equivalent to the distance travelled by the atoms during a measurement. This is known as the diffusion crosstalk-free distance which can be estimated as, $\Delta{x} = \sqrt{2DT}$, where $D$ is the diffusion constant calculated for Cs atoms in a N$_{2}$ atmosphere of a specific pressure and temperature, and $T$ is the measurement time \cite{horsley2015high,dong2019spin}. This yields close to $300\,\mu$m when taking into account the $220\,$torr buffer gas pressure of the cell and the readout period of a single FID cycle. This is opposed to paraffin coated cells, in which the atoms sample the entire cell volume during the measurement interval, leading to magnetic gradient broadening of the measured resonance \cite{pustelny2006influence}. By contrast, the coherence time of buffer gas cells is affected by magnetic field variations local to the probe beam within $\Delta{x}$. This diffusion length can be reduced by shortening the measurement time, to the detriment of sensitivity, by analysing subsections of the FID signal using, for example, a Hilbert transform \cite{wilson2020wide, hunter2022accurate, ingleby2022digital}. Furthermore, increasing the buffer gas pressure can also lower $\Delta{x}$ by slowing the rate of spin diffusion. However, there is a trade-off with the elevated rate of spin destruction collisions with buffer gas atoms that negatively impact sensor precision. 

\section{Results}
\subsection{Magnetic Field Gradient Mapping}
An initial evaluation of the sensor head's suitability for magnetic imaging applications was conducted by measuring a well-defined first-order magnetic field gradient along the $x$-axis. Additionally, an offset bias field, $B_0 \approx 0.95\,\mu\mathrm{T}$, was applied along the same axis. The gradient field was produced by a single-turn counter-wound coil pair separated by approximately $19.7\,$mm, each with a radius close to $16\,$mm.  To demonstrate the effect of different gradient fields on the sensor output, the current supplied to the coil was varied between $\pm\,2.5\,$mA. The probe light was reflected by a mirror mounted on a translation stage as seen in Fig. \ref{Experimental setup}(a), enabling adjustment of the beam position along the $x$-axis of the vapor cell. The average magnetic field obtained from the measured Larmor frequency values of a $1\,$s FID signal train was computed at several $x$-axis positions as shown in Fig. \ref{Gradient magnetic field distribution}(a). \\
\begin{figure}[t]
  \centering
  \includegraphics{gradient_field_mapping.png}
  \caption{(a) OPM output (squares) when a first-order magnetic field gradient is applied along the $x$-axis using various coil supply currents as noted in the legend. The black data points represent the sensor background with no current supplied to the coil. The solid lines serve as a guide for the eye. (b) Measured magnetic field gradients (squares) at each supply current after background subtraction, and associated linear fits (dashed lines). The ratios between the experimental and theoretically predicted (see Eq. \ref{eqn: gradient}) gradients, in ascending order of current, were 0.934, 0.931, 0.943, and 0.937.}
  \label{Gradient magnetic field distribution}
\end{figure}
\indent The observations are compared to theoretical predictions based on calculating the field strength, $B(x)$, at a position, $x$, from the midpoint between two identical parallel plane coils whose current flows are in opposite directions. The gradient field distribution can be determined from the following analytical expression,
\begin{equation}
\label{eqn: gradient}
    \begin{split}
        B(x) = \frac{\mu_0 N I R^{2}}{2} & \biggl(\left[R^2 + (x-s/2)^2\right]^{-3/2} \\
          - & \left[R^2 + (x+s/2)^2\right]^{-3/2}\biggr),  
    \end{split}
\end{equation}
\noindent where $\mu_0$ is the magnetic permeability, $N$ is the number of coil turns, $I$ is the current, $R$ is the coil radius, and $s$ is the coil separation. \\
\indent Most of the field measured by the magnetometer is produced by $B_{0}$, that is generated by the coils along the $x$-axis. There are additional minor background variations observed in the absence of current, as denoted by the black data points in Fig. \ref{Gradient magnetic field distribution}(a). These can arise from multiple sources, including gradients produced when applying the bias field, and optical pumping effects. For example, optical pumping of the alkali spins within the cell is inhomogeneous due to the finite waist and Gaussian profile of the pump beam over the imaging area. This can be observed experimentally as a reduction in signal amplitude, and consequently sensitivity, as the probe beam samples lower intensity regions of the pump beam. This is a signature that the atomic population distribution over both hyperfine ground states is a function of position within the vapor cell due to variations in optical pumping efficiency. As a result, the Larmor frequency measurements will be weighted by these populations since the ground states have slightly different gyromagnetic factors. This is a source of heading error which can be compensated for analytically in the high spin polarization limit \cite{lee2021heading}.   \\
\indent Figure \ref{Gradient magnetic field distribution}(b) shows the measured magnetic field gradients after subtracting the sensor background. As expected, a linear relationship is clearly evident between the observed magnetic field and $x$-axis position of the probe beam for each supply current. Furthermore, all the experimental gradients coincide with the theoretical estimates to within $7\,\%$ and are in agreement with each other to $\pm\,0.6\,\%$, verifying the OPMs ability to spatially resolve magnetic field distributions. The observed offset likely results from inaccuracies in the theoretical predictions, since they are extremely sensitive to minor deviations in coil geometry.  

\vspace{-0.3cm}
\subsection{1D Magnetic Field Imaging}
\begin{figure}[t]
  \centering
  \includegraphics{wire_scan_1D.png}
  \caption{(a) 1D magnetic field mapping of the current configuration depicted in Fig. \ref{Experimental setup}(c). The positions of the copper wire are represented by vertical dashed-dotted lines, with the supply currents noted in the legend. The OPM readings (circles) are compared with theoretical predictions (dashed lines) based on the Biot-Savart law shown in Eq. \ref{eqn:BS}. (b) Magnetic field distribution (circles) after subtraction of the the sensor background and subsequent mean values. The slight asymmetry likely stems from deformations in the wire and interference produced by the external feedthroughs. Linear interpolation (solid lines) serves as a guide for the eye.}
  \label{Wire 1D scan}
\end{figure}
A local magnetic field source was placed close to the vapor cell to generate a more complex field distribution along the $x$-axis. A copper wire (32 awg) was positioned approximately $4.5\,$mm from the center of the vapor cell in an ``s" wire configuration as depicted in Fig. \ref{Experimental setup}(c), consisting of three adjacent wire segments spaced approximately $2.75\,$mm apart with alternating current flows. It was placed on the light exit face of the cell so that the pump and probe beams could interact uninterrupted with the vapor. The locations of the wires are indicated by the vertical dashed-dotted lines in Fig. \ref{Wire 1D scan}, showing the recorded data. No data could be collected at these positions as the probe beam was obstructed by the wire and did not reach the balanced photodetector. \\
\indent The magnetometer readings are denoted by circles in Fig. \ref{Wire 1D scan}(a) for various currents flowing through the wire ranging between $\pm\,4\,$mA. The sensor background measured at $0\,$mA (black circles) matched closely to the observations in Fig. \ref{Gradient magnetic field distribution}(a). The dominant contribution is the magnetic field gradient generated as a byproduct of the bias field. It can be seen that positive current flow in the central wire segment generates a magnetic field that opposes $B_{0}$, whereas the wires near the edges produce fields that add constructively. This is to be expected given the geometries indicated in Fig. \ref{Experimental setup}(c). In addition, there is an overall increase in the measured field with respect to the sensor background as the wire segments at the edges dominate. Also, as anticipated, reversing the direction of the current inverts the observed magnetic field distribution and the lowers the overall field. \\
\indent This behaviour is consistent with theoretical predictions (dashed lines) derived from the Biot-Savart law,
\begin{equation}\label{eqn:BS}
    \bm{B}(\bm{r}) = \frac{\mu_{0}}{4 \pi} \int \frac{I\,d{\bm{l}} \times \hat{\bm{r}} }{\bm{r} \cdot \bm{r}},
\end{equation}
where $I$ is the current along increment $d{\bm{l}}$, and $\bm{r}$ is the position with unit vector $\hat{\bm{r}}$. The contributions from the three wire segments $\bm{B}(\bm{r})_{1}$, $\bm{B}(\bm{r})_{2}$, and $\bm{B}(\bm{r})_{3}$ were calculated using Eq. \ref{eqn:BS} for arbitrary positions within the cell and averaged over the length of the vapor cell that the probe beam passes through, yielding $\overline{B}$,
\begin{equation}\label{eqn:BAvg}
    \overline{B}(x,y) = \frac{1}{L}\int_{0}^{L} \left\|\bm{B}(\bm{r})_1+\bm{B}(\bm{r})_2+\bm{B}(\bm{r})_3\right\|\,d\,z.
\end{equation}
The average is made as the probe beam samples the atomic vapor over the length, $L = 3\,$mm, of the cell with the atoms experiencing different field strengths along this path which our observation averages over. This effect of averaging could be minimized using vapor cells with shorter optical path lengths such as that described in Ref. \cite{hunter2018free}. The wire segments were assumed to be $4.5\,$mm from the center of the cell along the $z$-axis which aligns with experimental conditions. \\
\indent It can be seen from Fig. \ref{Wire 1D scan}(a) that there is a slight deviation of the experimental data from the theoretical trends, particularly toward the right edge of the cell, which is likely caused by interference from the external connecting wire. This is particularly evident from the asymmetry observed in Fig. \ref{Wire 1D scan}(b) where the sensor background and subsequent mean values have been subtracted, and more succinctly highlights the summation of fields produced by adjacent wire segments. It should also be noted that the theoretical predictions do not account for deformations, the most noticeable of which is in the central wire segment as seen in Fig. \ref{Experimental setup}(c). \\
\begin{figure*}[t]
  \centering
  \includegraphics{wire_scan_2D.png}
  \caption{(a) Sensor background measured at each probe beam location (grey dots). The magnetic field gradient is a byproduct of $B_{0}$ applied along the $x$-axis. (b) Theoretical magnetic field distribution based on the Biot-Savart law (see Eq. \ref{eqn:BS}) for the ``cross" current configuration depicted in Fig. \ref{Experimental setup}(d), assuming a $3\,$mA current and $B_{0} = 50\,\mu\mathrm{T}$. The dashed-dotted lines represent the position of the wire. (c) Magnetic field distribution produced with currents of $3.08\,$mA (left) and $-3.13\,$mA (right) passed through the wire. The sensor background was subtracted from the OPM recordings at each location. (d) Magnetic field distribution produced by a $0.55\,$mA root-mean-square current modulation at $376\,$Hz. The in-phase component, $B_{X}$, is calculated via software demodulation of the $1\,$s magnetic field time series measured at each probe beam position. No background subtraction was required in this case. The quadrature component yielded no useful information due to the phase response of the magnetometer. Natural-neighbour interpolation was used in each of the images seen here to enhance data visualization \cite{sibson1981brief}.}
  \label{Cross-wire 2D scan}
\end{figure*}

\subsection{2D Magnetic Field Imaging}
Thus far, the probe beam has been translated along only the $x$-axis to reconstruct a 1D magnetic field distribution. To facilitate 2D magnetic imaging, the mirror was tilted vertically to deflect the beam toward different positions within the vapor cell along the $y$-axis. Consequently, the beam passes through the cell at a slight angle, although this is insignificant given the narrow $3\,$mm optical path length. The probe beam's position was determined using a CMOS camera temporarily placed in front of the cell before each $y$-axis translation. Measuring the respective distances of the camera and vapor cell from the scanning mirror allowed the probe beam's position within the cell to be calculated.  The beam was translated $\pm\,2\,$mm (relative to the cell) along the $y$-axis in steps of $1\,$mm, and was moved in increments of $250\,\mu$m along the $x$-axis using the translation stage. \\
\indent The magnetic source in this case consisted of a wire arranged in a ``cross" configuration as depicted in Fig. \ref{Experimental setup}(d), in order to invoke magnetic field structure over two dimensions. A sinusoidal current at $376\,$Hz with a $0.55\,$mA root-mean-square amplitude, and separate offsets of $-3.13\,$mA and $3.08\,$mA were supplied to the magnetic source. Measurements were also conducted with no offset to determine the magnetometer background as in previous cases. The bias field along the $x$-axis was set to $B_{0} \approx 50\,\mu$T during these measurements to emulate Earth's field conditions, demonstrating the sensors compatibility with unshielded operation. \\
\indent Figure \ref{Cross-wire 2D scan} shows the resulting magnetic field distributions with natural-neighbour interpolation applied to improve data visualization \cite{sibson1981brief}. The dashed-dotted lines indicate the wire's position, and the dots represent the locations of the probe beam during each measurement. The sensor background is shown in Fig. \ref{Cross-wire 2D scan}(a), exhibiting a significant magnetic field gradient produced as a byproduct of the strong bias field generated along the $x$-axis. The static magnetic field component recorded by the magnetometer after background subtraction is shown in Fig. \ref{Cross-wire 2D scan}(c), with the current offset set to approximately $3.08\,$mA (left) and $-3.13\,$mA (right). This contributes to $<0.1\,\%$ of the overall field and accounts for around $6\,\%$ of the bias field variation across the cell. Depending on the direction of current flow, the magnetic field produced by the wire adds constructively or destructively with $B_{0}$, as indicated by the distinct red and blue portions of the image. The wire's configuration results in opposing directions of current flow, causing field cancellation at the central regions along both the $x$- and $y$-axes. It can also be seen by comparing the plots in Fig. \ref{Cross-wire 2D scan}(c) that reversing the direction of current flow inverts the magnetic field distribution in the anticipated manner. The standard deviation of the fluctuation observed when adding these two distributions was found to be approximately $1\,$nT, which is around $1.7\,\%$ of the total field variation contributed by the wires over the imaging area. Some of this discrepancy will originate from the small difference in the magnitudes of the applied currents. \\
\indent The behaviour shown in Fig. \ref{Cross-wire 2D scan}(c) is well represented by theoretical expectations based on the Biot-Savart law (see Fig. \ref{Cross-wire 2D scan}(b)). In this case, the contributions from two wire segments, $\bm{B}(\bm{r})_{1}$ and $\bm{B}(\bm{r})_{2}$, arranged in the ``cross" current configuration depicted in Fig. \ref{Experimental setup}(d) were calculated using Eq. \ref{eqn:BS}. The wire was placed roughly $4.5\,$mm from the center of the $3\,$mm thick vapor cell, consistent with the 1D scan. The theory assumed a current of $3\,$mA and a bias field of $50\,\mu$T along the $x$-axis. The experimental and theoretical magnetic field distributions show good correlation, as both provide similar patterns extending over a range of approximately $60\,$nT within the imaging area. The center of the theoretical distribution is significantly affected by geometrical variations, with around a $10\,$nT offset from the bias field of $50\,\mu\mathrm{T}$ observed. This was based on the best estimate of the wire positioning with respect to the vapor cell which was not clearly defined. \\        
\indent Using the oscillating component of the magnetic field recorded by the OPM, a 2D magnetic field distribution could also be constructed as shown in Fig. \ref{Cross-wire 2D scan}(d). In this case, the $1\,$s magnetic field time series collected at each probe beam position was software demodulated to extract the in-phase component, $B_{X}$, with respect to the applied $376\,$Hz current modulation. The demodulation phase was set to be equal to the reference (modulation) signal phase for consistency; however, the phase and frequency responses will not be completely flat within the Nyquist limit due to atomic decoherence \cite{hunter2018waveform}. The direction of the measured oscillating field relative to $B_0$ is encoded in the phase offset with respect to the applied modulation assuming a perfect magnetometer phase response. Clearly, the 2D magnetic field image is consistent with that observed from the static component (left) in Fig. \ref{Cross-wire 2D scan}(c). Furthermore, the applied root-mean-square current amplitude of $0.55\,$mA is around a factor of $5.6$ smaller than the current offset of $3.08\,$mA. Taking this into account, the relative amplitudes observed between the static and oscillating field strength components agree to within approximately $20\,\%$. The discrepancy can be attributed to the roll-off experienced by the magnetometer towards the Nyquist limit. In addition, there will be an associated phase lag which was not taken into account during demodulation that will reduce the demodulated amplitude. Performing these magnetic imaging experiments at higher frequency is advantageous as no background subtraction is required, and there is flexibility in selecting a frequency that resides in a clean part of the noise spectrum. Additionally, it extends the magnetometer's utility to imaging of high frequency magnetic sources.  

\section{Conclusion and outlook}
\vspace{-0.2cm}
 In summary, the magnetic field imaging capability of a FID magnetometer based on a Cs MEMS vapor cell with approximately $220\,$torr N$_{2}$ buffer gas was demonstrated. This was achieved by translating the readout beam across different regions of the sensor head and measuring the magnetic field at each location. The OPM response was validated using theoretical predictions of the field distributions produced by a gradient field and a wire magnetic source. Two-dimensional magnetic imaging was conducted at a high bias field of approximately $50\,\mu$T. This illustrates the magnetometer's extensive dynamic range and offers a pathway toward magnetic imaging in unshielded environments. Furthermore, the wide bandwidth of the sensor was exploited to simultaneously resolve the temporal and spatial magnetic field components produced by the wire. \\  
 \indent Aligning the probe beam through the cell in a double-pass arrangement would improve this magnetic imaging strategy since it eliminates light obstruction caused by the magnetic source, and elevates the sensitivity through increased optical rotation. This could be achieved by placing a reflector on the back surface of the cell prior to the imaging source, allowing the sensor head to be extremely close to the magnetic source of interest. Such a sensor would be an invaluable resource in PCB inspection and quality assurance of integrated circuits (ICs) \cite{holzl2012quality, hofer2012analyzing}. The current distribution of an IC has already been measured using NV centers \cite{kehayias2022measurement}, although these devices demonstrate limited sensitivity. The output could be compared to known distributions obtained from functional devices, to rapidly test ICs in a production line setting. The higher precision achievable with atomic magnetometers would be particularly valuable in battery diagnostics through non-invasive current mapping \cite{bason2022non}. These measurements could benefit from more sensitive instrumentation, and have already been performed using zero-field OPMs \cite{hu2020rapid}. These devices demonstrated a $5\,$nT dynamic range, hence require shielding and could only tolerate limited nickel concentrations; a problem that can be overcome using total-field sensors such as that considered in this work.  \\
 \indent There is also potential for improving homogeneity in the optical pumping dynamics across the cell by providing a more uniform pump beam intensity profile. Consequently, this ensures a consistent spatial sensitivity dependence, and reduces the overall systematics that contribute to the sensor background. A simple solution would be to expand the pump beam; however, this would result in a lower peak intensity and excess beam clipping at the edges of the cell, negatively impacting sensor sensitivity. Alternatively, a flat-top beam profile tailored to the cell's dimensions could be implemented, although this requires complex beam shaping and will likely result in optical losses. The most effective solution would be to translate both the pump and probe beams simultaneously by launching them through the same optical fiber for optimal optical pumping efficiency at every position. \\
 \indent Image reconstruction speeds could be accelerated using a spatial light modulator (SLM) or digital micromirror device (DMD) \cite{fang2020high}, as opposed to the slower process of probe beam translation. In this case, magnetic field distributions would be reconstructed with a single pixel detector, such as a photodetector, by applying spatially varying illumination patterns \cite{zhang2017hadamard}. The photodetector signal collected from each spatial light mode would result in FID data that can be subsequently analysed to extract Larmor frequency information. The magnetic image would be reconstructed by correlating symmetries in the spatial light modes with the corresponding magnetic field data. Accordingly, magnetic images could be formed more efficiently as DMDs can deliver refresh rates in the kHz regime. In this case, the measurement bandwidth would be limited by the repetition rate of the OPM. However, one could circumvent this issue by utilizing alternative DSP techniques, such as a Hilbert Transform \cite{wilson2020wide, hunter2022accurate, ingleby2022digital}, for quicker Larmor frequency extraction. Also, the spatial resolution becomes dependent on how many spatial modes are used. Thus, the probe beam width can be extended without sacrificing spatial resolution in order to enhance sensitivity performance. \\ 
\indent \emph{Data availability}---Data underlying the results presented in this manuscript are available in Ref. \cite{hunter2023Free}.

\section*{Acknowledgements}
The authors acknowledge funding from InnovateUK (ISCF-42186) and UKRI (EP/T001046/1). The authors would like to acknowledge support and funding from the INMAQS collaboration (EP/W026929/1). JPM gratefully acknowledges funding from a RAEng Research Fellowship. AM was supported by a Ph.D. studentship from the Defence Science and Technology Laboratory (Dstl).

\bibliography{references} 

\end{document}