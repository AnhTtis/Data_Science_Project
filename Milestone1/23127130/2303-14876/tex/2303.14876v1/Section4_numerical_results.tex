\section{Numerical results}\label{sec:result}
The present section shows numerical results for perturbed polytropic sheets due to test mass sheet. We first explain the local characteristics of gravitational amplification and then global ones in the sheets. We lastly assess the shrinkage of sheet thickness in the $z$-direction. 



\subsection{Maximum gravitational amplification and its location}\label{sec:numel_res_chi_M}

Figure \ref{fig:1D_Est} depicts numerical values of gravitational amplification in selected polytropic sheets perturbed by test mass. The characteristics of $\chi(z)$ are alike among the polytropic sheets between $n=0.01$ and $n=500$. The amplification is maximized only once along the $z$-direction and weak near $z=0$ and $z=z_\text{M}$. This feature is qualitatively the same as previously reported gravitational amplification in collisionless isochrone \citep{Gilbert_1970} and isothermal cylinder and sheet \citep{Ito_2023a}. 

In the polytropic sheets, the maximum amplification $\chi_\text{max}$ increases with index $n$ and asymptotically reaches $1.68$ (Figure \ref{fig:1D_n_EstM}). This value approximately equals the gravitational amplification obtained for the isothermal sheet model. The isothermal sheet may be considered the case of $n\to\infty$ in the polytropic sheets \citep[e.g.,][]{Horedt_2000}. On the one hand, the location of the maximum amplification gets closer to $z=0$ as $n$ increases (Figure \ref{fig:1D_n_rEstM}). This would reflect a feature of unperturbed polytropic sheets that the mass distribution is concentrated more near $z=0$ as $n$ increases (Figure \ref{fig:1D_rho}). 


\begin{figure}
	\centering
	\begin{tikzpicture}[scale=0.9]
	\begin{semilogxaxis}[ grid=major, xmax=5e0, xmin=3e-4, xlabel=\Large{$z$},ylabel=\Large{$\chi$}, legend pos=north west]
	\addplot [color = red ,mark=no,thick,solid ] table[x index=0, y index=3]{Poly1D_n1_r_Phi_Phio_Est.txt}; 
	     \addlegendentry{\large{$n=1$}}
	\addplot [color = orange ,mark=no,thick,densely dashed ] table[x index=0, y index=3]{Poly1D_n2_r_Phi_Phio_Est.txt}; 
	      \addlegendentry{\large{$n=2$}}
	\addplot [color = purple ,mark=no,thick,densely dotted ] table[x index=0, y index=3]{Poly1D_n5_r_Phi_Phio_Est.txt}; 
	      \addlegendentry{\large{$n=5$}}
	\addplot [color = blue ,mark=no,thick,dashed ] table[x index=0, y index=3]{Poly1D_n20_r_Phi_Phio_Est.txt}; 
     	\addlegendentry{\large{$n=20$}}
	\addplot [color = black ,mark=no,thick,dotted ] table[x index=0, y index=3]{Poly1D_n100_r_Phi_Phio_Est.txt};
	     \addlegendentry{\large{$n=100$}}
	\end{semilogxaxis}
	\end{tikzpicture}
\caption{Gravitational amplification $\chi$ due to test mass sheet in the polytropic sheet models.}
\label{fig:1D_Est}
\end{figure}




\begin{figure}
	\centering
	\begin{tikzpicture}[scale=0.9]
		\begin{semilogxaxis}[ grid=major,xlabel=\Large{$n$},ylabel=\Large{$\chi_\text{max}$},xmax=1000, ymax=1.8, legend pos=north east]
			\addplot [color = bblue ,mark=*,thick,solid ] table[x index=0, y index=3]{Poly1D_n_rM_Mtot_EstM_rEstM_EstEff_EstAVE.txt}; 
		\end{semilogxaxis}
	\end{tikzpicture}
	\caption{Maximum gravitational amplification $\chi_\text{max}$ against polytropic index $n$ for the polytropic sheet model.}
	\label{fig:1D_n_EstM}
\end{figure}

\begin{figure}
	\centering
	\begin{tikzpicture}[scale=0.9]
		\begin{semilogxaxis}[ grid=major,xlabel=\Large{$n$},ylabel=\Large{$r_{\chi.\text{max}}$}, legend pos=north east]
			\addplot [color = bblue ,mark=*,thick,solid ] table[x index=0, y index=4]{Poly1D_n_rM_Mtot_EstM_rEstM_EstEff_EstAVE.txt}; 
		\end{semilogxaxis}
	\end{tikzpicture}
	\caption{Location of $\chi_\text{max}$ against polytropic index $n$ for the polytropic sheet models.}
	\label{fig:1D_n_rEstM}
\end{figure}





%\begin{figure}
%	\centering
%	\begin{tikzpicture}[scale=0.9]
%		\begin{axis}[ grid=major,xlabel=\Large{$M_\text{tot}$},ylabel=\Large{$r_{\chi.\text{max}}$}, legend pos=north east]
%			\addplot [color = bblue ,mark=*,thick,solid ] table[x index=2, y index=4]{Poly1D_n_rM_Mtot_EstM_rEstM_EstEff_EstAVE.txt}; 
	%	\end{axis}
	%\end{tikzpicture}
	%\caption{Location of the maximum of $\chi$ against the total mass $M_\text{tot}$ of the polytropic %sheet models.}
%	\label{fig:1D_Mtot_rEstM}
%\end{figure}




\subsection{Average gravitational amplification}

The numerical value of $\chi_\text{ave}$ is shown in Figure \ref{fig:1D_Mtot_EstEff} against polytropic index $n$. As easily expected, the polytrope of $n=1$ takes a relatively high value of $\chi_\text{ave}$ since it does not depend on its unperturbed mass distribution and system size. The maximum of $\chi_\text{ave}$ occurs when $n\simeq 1.6$. This would be the outcome of the two effects; a moderately high index $n$ and slowly decaying density $\rho_\text{o}(z)$ in the perturbed density $(\propto n(-\phi_\text{o}(z))^{n-1}=n \rho_\text{o}^{1-1/n}(z))$ in equation \eqref{LE_1D_perturb_expan_delPhi}. For this, we first assume from the result of Section \ref{sec:n_0} that if the perturbed density  is close to zero then the amplification is not effective. For example, for very low $n(\ll 1)$, mass density distribution slowly diverges at large distances, but the perturbed density is still low at every height $z$ due to the low $n$. For higher $n(\gg 1)$, $\chi(z)$ can reach large values, but the average is low because of large polytrope height and rapid decay in the unperturbed density. We hence may expect the perturbed density to be more effective for amplification when $n$ is the order of 1.


\begin{figure}
	\centering
	\begin{tikzpicture}[scale=0.9]
		\begin{semilogxaxis}[ grid=major,xlabel=\Large{$n$},ylabel=\Large{$\chi_\text{ave}$}, legend pos=north east]
			\addplot [color = bblue ,mark=*,thick,solid ] table[x index=0, y index=6]{Poly1D_n_rM_Mtot_EstM_rEstM_EstEff_EstAVE.txt}; 
		\end{semilogxaxis}
	\end{tikzpicture}
	\caption{Average gravitational amplification $\chi_\text{ave}$ against  polytropic index $n$ for the polytropic sheet models.}
	\label{fig:1D_Mtot_EstEff}
\end{figure}


\subsection{Shortening of the polytropic sheets}

The \emph{bare} gravitational field due to test mass is constant $(=1/2)$ throughout polytropic sheets. The higher sheets can be deviated more largely from the original to the new equilibrium position because of low pressure. Polytropes with higher polytropic index hence experience more significant shortening in the $z$-direction (Figure \ref{fig:1D_Eta}). The ratio $\eta$ is well approximated by $\sim n^{0.492}$ for high indexes. This means that we need a very small value of $\epsilon$ for a high polytropic index. For example, to achieve 1$\%$ of shrinkage, or $\epsilon \delta z/ z_\text{M}=0.01$, the necessary value of $\epsilon$ is $\sim0.01 n^{-0.492}$; $\epsilon\sim10^{-5}$ for $n=500$. It is obvious that the linear approximation of potential $\phi(z)$ may break down for high $n$. This break-down could be reasonable by seeing the expanded form of $\phi(z)$ in equation \eqref{Eq.phi_expan}. The perturbation potential $\epsilon \delta \phi(z)$ increases like $\epsilon z/2$ as $z$ increases while $\phi_\text{o}(z)$ decreases. The linearization would successfully apply to polytropes with shorter heights. To hold the linear approximation for any $n$, polytropes must embedded in a pressurized medium. 

\begin{figure}
	\centering
	\begin{tikzpicture}[scale=0.9]
		\begin{loglogaxis}[ grid=major,xlabel=\Large{$n$}, legend pos=north west]
			\addplot [color = bblue ,mark=o,thick,solid, only marks] table[x index=0, y index=3]{Poly1D_n_Delx_ref1_Eta_ref2_epsil_ref3.txt}; 
			\addlegendentry{\large{$-\eta$}}
			\addplot [color = bred ,mark=no,thick ] table[x index=0, y index=4]{Poly1D_n_Delx_ref1_Eta_ref2_epsil_ref3.txt};
			\addlegendentry{\large{$0.39n^{0.492}$}}
		\end{loglogaxis}
	\end{tikzpicture}
	\caption{Negative of ratio $\eta$ of $\delta z$ to $z_\text{M}$ against  polytropic index $n$ for the polytropic sheet models. A guideline is depicted together as an approximation of $\eta$ for high index $n$.}
	\label{fig:1D_Eta}
\end{figure}

%\begin{figure}
%	\centering
%	\begin{tikzpicture}[scale=0.9]
%		\begin{loglogaxis}[ grid=major,xlabel=\Large{$n$},ylabel=\Large{$\epsilon_\text{max}$}, legend pos=north east]
%			\addplot [color = bblue ,mark=*,thick,solid ] table[x index=0, y index=5]{Poly1D_n_Delx_ref1_Eta_ref2_epsil_ref3.txt};  
%		\end{loglogaxis}
%	\end{tikzpicture}
%	\caption{Maximum value of  $\epsilon$ against  polytropic index $n$ for the polytropic sheet models to achieve $\epsilon \delta z/ z_\text{M}=0.01$.}
%	\label{fig:1D_epsilon}
%\end{figure}