\section{Conclusion}\label{sec:conclusion}

The present work discussed gravitational polarization of fields induced by test mass sheet instantaneously placed at $z=0$ in polytropic sheets of $0\leq n<\infty$. It showed that gravitational amplification occurs even in a \emph{finite} extent of equilibrium self-gravitating systems in addition to previously reported infinite collisional and collisionless systems.  We first obtained analytical results for polytropic sheets of $n=0$ and $n=1$. The former model is a special case in which neither system shrinkage nor gravitational amplification does occur. The latter model is the simplest model that possesses basic features of gravitational amplification. It also helps confirm our numerical results.

Our numerical results showed that gravitational amplification $\eta$ in polytropic sheets are qualitatively the same as those in collisionless isochrone, isothermal sheet, and isothermal cylinder. The amplification is one at $z=0$, gets greater with height, and reaches its maximum value. After passing the maximum, it then decreases with height and equals one at the maximum height.  We found in the polytropic sheets that the maximum of $\eta$ increases with polytropics index $n$ and approaches that of the isothermal sheet model. The height at which the maximum occurs gets shorter for a higher polytropic index. We also found that the average of gravitational amplification is maximized around at $n=1.6$ and it gets weaker as $n$ gets greater or less.  Lastly, we computed the shortening of the polytropic sheets. The ratio of the height change to maximum height becomes greater with index $n$. It is approximately proportional to $n$ when $n\gg1$. This analysis provided a constraint on the present linear approximation.

In the future paper, we will extend the present work to nonlinear cases. We may also apply the present method to polytropic cylinder and sphere. Especially, the latter model is unstable to radial perturbation. We will confine the isothermal sphere in a rigid wall or by a pressurized medium or polytrope. The present formulation, especially for $n=0$, can be readily arranged for such confined systems. Another interesting extension work would be to move test mass from the original equilibrium polytrope to the center of gravity rather than adding extra test mass. Such a process was already discussed in \cite{Gilbert_1970} for collisionless isochrone. The discussion can be useful to understand the discreteness of self-gravitating systems; it will help understand how the ``statistical term \citep{Gilbert_1968}'' affects gravitational fields due to test mass and how the fields disappear near the surface of self-gravitating systems. 

     