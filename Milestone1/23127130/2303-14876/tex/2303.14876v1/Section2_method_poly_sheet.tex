\section{Polytropic sheets and its perturbation due to test mass}\label{sec_poly_models}

The present section first explains equilibrium polytropic sheets and a perturbation method for the sheets due to test mass sheet. It then explains the effect of shortening on the sheets.

\subsection{Deriving the Lane-Emden equation for polytropic sheets}\label{Deriv_Poly_1D}

Polytropic sheet models are three-dimensional self-gravitating equilibrium model composed of particles interacting via the pair-wise Newtonian potential \citep[e.g.,][]{Horedt_2000}. The self-consistent mean-field potential $\Phi(\boldsymbol{r})$ is determined by mass density $\rho(\boldsymbol{r})$ via the Poisson equation
\begin{equation}
	\frac{\partial}{\partial \boldsymbol{r}}\cdot\left(\frac{\partial \Phi}{\partial \boldsymbol{r}}\right)=4\pi G\rho(\boldsymbol{r}),\label{Eq_Pois}
\end{equation}
where $G$ is the gravitational constant. In the equilibrium model, the hydrostatic equation must hold;
\begin{equation}
	\frac{\partial p}{\partial \boldsymbol{r}}+ \rho(\boldsymbol{r})\frac{\partial \Phi}{\partial \boldsymbol{r}}=0,\label{Eq_hydro}
\end{equation}
where $p(\boldsymbol{r})$ is the isotropic scalar pressure. We assume polytropic sheets to satisfy the polytropic relation
\begin{equation}
	p=K\rho^{1+1/n}, \label{Eq_poly}
\end{equation}
where $K$ is the polytropic constant and $n$ the polytropic index. Equations \eqref{Eq_Pois},  \eqref{Eq_hydro}, and \eqref{Eq_poly} are a set of closed equations for variables $\rho(\boldsymbol{r})$, $p(\boldsymbol{r})$, and $\Phi(\boldsymbol{r})$. 

With dimensionless variables
\begin{eqnarray}
	      \phi&\equiv&\frac{\Phi(\boldsymbol{r})}{K(n+1)\,\rho_\text{c}^{1/n}},\\
	\boldsymbol{\xi}&\equiv&\frac{\boldsymbol{r}}{L_\text{c}}\equiv\left(\frac{4\pi G}{K(n+1)}\rho_\text{c}^{1-1/n}\right)^{1/2}\boldsymbol{r},\label{dimless_var}
\end{eqnarray}
where $\rho_\text{c}$ is the central density of $\rho(r)$, equations \eqref{Eq_Pois},  \eqref{Eq_hydro}, and \eqref{Eq_poly} reduce to the Lane-Emden equation for polytropes of index $n$
\begin{equation}
	\frac{\partial}{\partial \boldsymbol{\xi}}\cdot\left(\frac{\partial \phi}{\partial \boldsymbol{\xi}}\right)-\left(-\phi\right)^n=0,\label{g_LE}
\end{equation}
where the following relations are used
\begin{eqnarray}
	\rho&=&\rho_\text{c}\left(-\phi\right)^n,\\
	   p&=&K\rho_\text{c}^{1+1/n}\left(-\phi\right)^{n+1}.
\end{eqnarray}

The polytropic sheets are stratified in a direction, say the $z$-direction. Equation \eqref{g_LE} for stratified sheets along the $z$-axis is written as
\begin{equation}
	\frac{\text{d}^2\phi}{\text{d}\,z^2}-\left(-\phi\right)^n=0.
	\label{LE_1D}
\end{equation} 
The boundary conditions (BCs) for equation \eqref{LE_1D} are 
\begin{equation}
	\phi(z=0)=-1, \qquad \phi'(z=0)= 0. \label{BC_LE_1D}
\end{equation}
Our interest is polytrope sheets of $0\leq n<\infty$ whose potential monotonically decreases with height and reaches zero at a finite height. We label the maximum height, or the first zero of the potential, as $z_\text{M}$ hereafter. Their mass density also has the same characteristics as the potential except for $n=0$; the density is uniform in the polytrope of $n=0$. Figure \ref{fig:1D_rho} shows mass density distributions at $z>0$ for selected polytropic indexes.
 
\begin{figure}
	\centering
	\begin{tikzpicture}[scale=0.9]
		\begin{semilogxaxis}[ grid=major, xmax=5, xmin=3e-4, xlabel=\Large{$z$},ylabel=\Large{$\rho$}, legend pos=south west]
			\addplot [color = red ,mark=no,thick,solid ] table[x index=0, y index=1]{Poly1D_n1_r_rho_DiffRho.txt}; 
			\addlegendentry{\large{$n=1$}}
			\addplot [color = orange ,mark=no,thick,densely dashed ] table[x index=0, y index=1]{Poly1D_n2_r_rho_DiffRho.txt}; 
			\addlegendentry{\large{$n=2$}}
			\addplot [color = purple ,mark=no,thick,densely dotted ] table[x index=0, y index=1]{Poly1D_n5_r_rho_DiffRho.txt}; 
			\addlegendentry{\large{$n=5$}}
			\addplot [color = blue ,mark=no,thick,dashed ] table[x index=0, y index=1]{Poly1D_n20_r_rho_DiffRho.txt}; 
			\addlegendentry{\large{$n=20$}}
			\addplot [color = black ,mark=no,thick,dotted ] table[x index=0, y index=1]{Poly1D_n100_r_rho_DiffRho.txt};
			\addlegendentry{\large{$n=100$}}
		\end{semilogxaxis}
	\end{tikzpicture}
	\caption{Mass density distribution of the polytropic sheet models.}
	\label{fig:1D_rho}
\end{figure}

\subsection{Perturbation method for the polytropic sheets}\label{Perturb_Poly1D}

The present mathematical formulation for gravitational amplification is the same as that for gravitational tides \citep{Chandrasekhar_1933b} except for the location of test mass. The \emph{bare} gravitational field strength due to test mass sheet is constant. Hence, even if it is placed outside a polytrope, it does not cause tidal effect. On the one hand, if it is placed at the center of gravity of the polytrope, it causes gravitational amplification. 

We use the perturbation method developed for the isothermal systems \citep{Ito_2023a}. Imagine that test mass sheet is added at $z=0$ on the $xy$-plane in a polytropic sheet model. Equation \eqref{LE_1D} is modified as
\begin{equation}
	\frac{\text{d}^2\phi}{\text{d}\,z^2}-\left(-\phi\right)^n=\epsilon\delta(z),
	\label{LE_1D_perturb}
\end{equation} 
where $\epsilon$ is a small parameter defined as
\begin{equation}
	\epsilon\equiv\dfrac{m_\text{p}}{\rho_\text{c}L_\text{c}},
\end{equation}
where $m_\text{p}$ is test mass.  Expand equation \eqref{LE_1D_perturb} with $\epsilon$ as
\begin{equation}
	\frac{\text{d}^2}{\text{d}\,z^2}\left(\phi_\text{o}+\epsilon\delta\phi\right)-\left(-\phi_\text{o}-\epsilon\delta\phi\right)^n=\epsilon\delta(z),
	\label{LE_1D_perturb_expan}
\end{equation} 
where $\phi_\text{o}$ is the unperturbed potential of polytropic sheet model and $\epsilon\delta\phi$ the potential deviated from $\phi_\text{o}$ due to test mass. At the order of $\epsilon$, we have a linearized Lane-Emden equation for $\delta\phi$;
\begin{equation}
	\frac{\text{d}^2 \delta\phi}{\text{d}\,z^2}+n\left(-\phi_\text{o}\right)^{n-1}\delta\phi=\delta(z).
	\label{LE_1D_perturb_expan_delPhi}
\end{equation} 

We introduce the following potential
\begin{equation}
	W(z)\equiv\delta\phi(z)-\frac{1}{2}|z|,\label{Eq.W}
\end{equation}
so that equation \eqref{LE_1D_perturb} does not include the delta function 
\begin{equation}
	\frac{\text{d}^2 W}{\text{d}\,z^2}+n\left(-\phi_\text{o}\right)^{n-1}\left(W+\frac{1}{2}|z|\right)=0.
	\label{LE_1D_perturb_reg}
\end{equation}
With equation \eqref{Eq.W}, there is no gravitational field due to $W(z)$ at $z=0$. Hence, the BC is
\begin{equation}
	W'(z=0)=0.\label{BC_1D_x_0}
\end{equation}
Another BC is determined so that the effective mass $m^{*}$, the partial sum of reconfigured sheet masses, is zero at $|z|=z_\text{M}$. The effective mass for the polytropic sheet is at $z$
\begin{eqnarray}
	m^{*}(z)&=&-2\int_{0}^{z}n\left(-\phi_\text{o}\left(z'\right)\right)^{n-1}\left(W\left(z'\right)+\frac{1}{2}|z'|\right)\text{d}z',\nonumber\\
	&=&2\left(W'(z)-W'(0)\right).
\end{eqnarray}
The mass $m^{*}(z)$ must be zero at $z=z_\text{M}$ since test mass $m_\text{p}$ is not included in the reconfigured masses. Accordingly, the BC is at $|z|=z_\text{M}$
\begin{equation}
	W'(|z|=z_\text{M})=0.\label{BC_1D_x_inf}
\end{equation}
Our fundamental numerical strategy is to find potential $W(z)$ by solving equation \eqref{LE_1D_perturb_reg} with two BCs \eqref{BC_1D_x_0} and \eqref{BC_1D_x_inf} after finding the unperturbed potential $\phi_\text{o}$ by solving equation \eqref{LE_1D} with BCs in equation \eqref{BC_LE_1D}.

\subsection{Polytropic sheet shortening and surface boundary condition}

Gravitational potential due to test mass sheet pulls polytropic sheets in the $z$-direction toward the test mass, which results in a shortening of sheet thickness. The discussion of polytrope shortening is essentially the same as that of distorted polytrope due to tidal effects \citep{Chandrasekhar_1933b}. We first introduce the new maximum height of polytropic sheets due to perturbation
\begin{equation}
	z_\text{s}\equiv z_\text{M}+\epsilon\delta z,\label{Eq.zs}
\end{equation} 
where $\delta z$ is the deviation from $z_\text{M}$ at order of $\epsilon$ and expected to take a negative value. Expand the new equilibrium potential
\begin{equation}
	\phi(z)=\phi_\text{o}(z)+\epsilon\delta\phi(z). \label{Eq.phi_expan}
\end{equation}
around at $z_\text{M}$ using equation \eqref{Eq.zs} as follows  
\begin{equation}
	\phi(z_\text{s})=\phi_\text{o}(z_\text{M})+\epsilon\left(\phi_\text{o}'(z_\text{M})\,\delta z +\delta\phi(z_\text{M})\right)+\mathcal{O}\left(\epsilon^2\right).
\end{equation}
The new equilibrium potential $\phi(z)$ must be zero at $z_\text{s}$. Since $\phi_\text{o}(z_\text{M})$ is zero, we have the condition
\begin{equation}
	\delta z=-\frac{\delta\phi(z_\text{M})}{\phi_\text{o}'(z_\text{M})}=-\frac{W(z_\text{M})+\frac{1}{2}z_\text{M}}{\phi_\text{o}'(z_\text{M})}.\label{Eq.delta_z}
\end{equation}
Using equation \eqref{Eq.zs}, we next expand the derivative of potential $\phi(z)$
\begin{equation}
	\phi'(z)=\phi_\text{o}'(z)+\epsilon\delta\phi'(z)
\end{equation}
to the order of $\epsilon$ as follows
\begin{equation}
	\phi'(z_\text{s})=\phi_\text{o}'(z_\text{M})+\epsilon\left(\phi_\text{o}''(z_\text{M})\,\delta z +\delta\phi'(z_\text{M})\right).
\end{equation}
With function $W(z)$ in equation \eqref{Eq.W}, another condition is given on the surface of the perturbed polytrope as
\begin{equation}
\phi_\text{o}''(z_\text{M})\,\delta z +W'(z_\text{M})=0.\label{Eq.delta_x2}
\end{equation}
With equation \eqref{Eq.delta_z}, we hence obtain the boundary condition on the surface
\begin{equation}
	-\frac{W(z_\text{M})+\frac{1}{2}z_\text{M}}{\phi_\text{o}'(z_\text{M})}\,\phi_\text{o}''(z_\text{M}) +W'(z_\text{M})=0.\label{Eq.gener_BC}
\end{equation}
Equation \eqref{Eq.gener_BC} is a generalized BC of that in equation \eqref{BC_1D_x_inf}. For polytropes of $1<n<\infty$, the term $\phi_\text{o}''(z_\text{M})$ is zero because of equation \eqref{LE_1D}. This means that the boundary condition on a shortened polytrope surface is not affected by the shortening, or $\delta z$. The effect of shortening should appear in distorted mass distribution when $\phi_\text{o}''(z_\text{M})$ is non-zero. Such an example is that polytropic sheets are inserted between two pressurized mediums. Another example is that mass density does not reach zero at $z_\text{M}$, which corresponds with the polytropic sheet of $n=0$ in the present work.





