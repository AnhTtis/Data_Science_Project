\section{Introduction}

Gravitational polarization of fields induced by test mass is an important but little-explored fundamental concept in statistical dynamics of self-gravitating systems. Polarization effects are generally known for dielectrics under external electric fields and for electrolytes or plasmas perturbed by a test-charge potential. In the systems, positive and negative charge distributions are reconfigured so that applied electric fields are weaken or even shielded. On the one hand, if test mass is added in a self-gravitating system, gravitational field due to test mass could be reinforced \citep{Miller_1966,Gilbert_1970}. The mechanism is simple; gravitational field due to test mass attracts ambient masses toward the test mass, the reconfigured mass distribution then can strengthen the field. Gravitational polarization could help understand the fundamental response of self-gravitating system to external perturbations. Examples are a deformation of polarized halo due to disc growth \cite{Murali_1998, Moody_1999}, perturbations in massive haloes \citep{Murali_1999},  stellar accretion and heavy central astrophysical objects \citep{Young_1980, Murali_1998, Quinlan_1995}, interaction between ``dressed'' particles/stars \citep{Heyvaerts_2010, Chavanis_2012}, and flyby of heavy objects \citep{Vesperini_2000}. These settings are realistic but complicated, so they are not suitable to assess \emph{pure} effects and properties of gravitational polarization. The simplest setting is to examine gravitation amplification is to add or move test mass to the center of gravity in a self-gravitating system so that the spatial symmetry is held through test-mass perturbation \citep{Gilbert_1970, Goodman_1984a}.

Discussing gravitational polarization of point-mass potential, however, is not a straightforward topic. First, self-gravitating systems are inhomogeneous and finite in size because of self-gravity. We can not resort to an infinite homogeneous approximation, which often allows us to use simple mathematical deductions. Such an approximation was applied to collisionless stellar systems \citep{Marochnik_1968} and cold dark matters \citep{Padmanabhan_1985} with the Maxwellian distribution function for stars and particles. The works suggested that gravitational potential due to test point mass behaves like $\cos(k_\text{J}r)/r$, where $k_\text{J}$ is the Jeans wave number, and $r$ the distance from test mass. The configuration, however, suffers from the problem of ``Jeans swindle'' \citep[e.g.,][]{Falco_2013} which is an inconsistent method unless we consider a cosmological setting. Second, particles and stars orbit obeying self-consistent Newtonian mean-field potentials in self-gravitating systems. This feature is especially important for stars in stellar systems, such as galaxies, globular clusters, and nuclear star clusters. Stars move around forming \emph{smooth} orbits that are not disturbed irregularly by other stars on dynamical time scales \citep[e.g.,][]{Binney_2011}. This implies that we need to know explicitly analytical expressions to describe orbits, such as orbital periods and isolating integrals. It is possible to obtain the expressions only for limited cases. \cite{Goodman_1984a} used a harmonic oscillator as an approximation of stellar cluster core. A successful discussion of gravitational amplification was made for a collisionless isochrone \citep{Gilbert_1968} where an emerging test-point mass is placed at the center of the isochrone holding the conservation of total mass. Gilbert showed that gravitational field due to test mass is strengthened with radius at small radii and reaches its maximum once. At larger radii, the effect of amplification disappears as the density approaches zero. Third, even with extensive numerical methods, the orbital effects can be handled only for weak perturbation. To overcome the problem of orbital effects above,  it is possible to use \cite{Kalnajs_1977}'s matrix method for realistic self-gravitating systems, such as the King models. However, the method applies only to weak perturbation problems. Even for such limited setting, the method needs exhaustive series expansions with sophisticated numerical schemes \citep{Murali_1999}. It appears that no further works are found with this method for gravitational polarization.  

For this situation, we recently discussed gravitational polarization in infinite collisional gaseous systems, namely the isothermal sheets, cylinder, and sphere \citep{Ito_2023a}. There are advantages in examining the models. (i) We can avoid the orbital effects of particles because of high collisionality in the systems. (ii) As discussed in \citep{Murali_1998}, gaseous (fluid) system could show similar response features to collisionless systems. (iii) Not only linear but non-linear analyses are easily executed because of simple mathematical structures of the Lane-Emden equation. In \citep{Ito_2023a}, we found in the isothermal sheets and cylinder that potentials due to test masses show the same qualitative characteristics as that in collisionless isochrone. On the one hand, the isothermal sphere showed an oscillatory gravitational field due to test point mass similar to that in infinite homogeneous collisionless systems with the Maxwellian distribution. The isothermal systems seems \emph{nice} models to discuss gravitational polarization compared to the previous results.  Hence, it would be reasonable to explore other collisional self-gravitating systems as well for further discussions.

All the previous works focusing on gravitational polarization examined only \emph{infinite} self-gravitating systems. Such systems are unrealistic in nature. It is important to examine whether gravitational polarization may be observed even in \emph{finite} systems as well. The present paper examines gravitational polarization in equilibrium self-gravitating polytropic sheets. The mass density and height of the sheet models are finite in the stratified direction for polytropic index of $0\leq n < \infty$ \citep[e.g.,][]{Horedt_2000}. The models are important to understand stratification and fragmentation of self-gravitating gaseous systems.  Also, they are stable against radial perturbation. The last feature may be less attractive for a statistical-dynamics point of view \citep{Campa_2009} since the sheet models do not show exotic collective features, such as negative specific heat. Yet, we believe that excluding those features can make easier our understanding of gravitational polarization. 

The goal of the present work is to show gravitational polarization in polytropic sheets. We assume that test mass sheet is placed perpendicularly to the stratified direction at the center of the sheets. We numerically solve a linearlized Lane-Emden equation perturbed by potential due to test mass. To account the finiteness of system size, we employ the method used for a tidal effect on polytropic spheres \citep{Chandrasekhar_1933a,Chandrasekhar_1933b}. Our numerical results show that the maximum of gravitational field gets greater with a higher polytropic index in polytropic sheets. In the limit $n>>1$, the maximum approaches that of the isothermal polytrope. We also show that the shortening of polytropic sheets is more significant for higher polytropic indexes. This result provides the limit of the linearlization approximation that we use for test-mass perturbation. 



The present paper is organized as follows. Section \ref{sec_poly_models} explains polytropic sheet models and the perturbation method to examine gravitational polarization. Section \ref{Sec_analy} shows analytical  results for polytropes of $n=0$ and $n=1$ to which our perturbation method is applied. It also introduces measures of gravitational amplification and shortening of polytropic sheets. Section \ref{sec:result} shows numerical results for $0\leq n<\infty$. Section \ref{sec:conclusion} is Conclusion.


