\section{Gravitational polarization in the polytropic sheet models of $n=0$ and $n=1$ }\label{Sec_analy}

The present section explains gravitational amplification of polytropic sheets of $n=0$ and $n=1$. We provide the explicitly analytical form of the amplification for the models using the perturbation method explained in Section \ref{sec_poly_models}. We also introduce measures of gravitational amplification and shortening of polytropic sheets.

\subsection{Gravitational polarization for $n=0$}\label{sec:n_0} 

The Lane-Emden equation for polytrope of $n=0$ is
\begin{equation}
	\frac{\text{d}^2\phi}{\text{d}\,z^2}-1=0.
\end{equation}
With the BCs in equation \eqref{BC_LE_1D}, the solution reads
\begin{equation}
	\phi_\text{o}(z)=-1+\frac{z^{2}}{2},
\end{equation}
and the maximum height is 
\begin{equation}
	z_\text{M}=\sqrt{2}.
\end{equation}
The equation for the perturbed polytrope of $n=0$ is the Laplace equation
\begin{equation}
	W''(z)=0,
\end{equation}
with the BCs  \eqref{BC_1D_x_0} and \eqref{Eq.gener_BC}. The solution reads 
\begin{equation}
	W(z)=-\frac{1}{\sqrt{2}}.
\end{equation}
This means that gravitational field due to test mass sheet is not amplified and that only the reference value of potential constant is determined as $-1/\sqrt{2}$. Also, no shortening of the polytropic sheets occurs;
\begin{equation}
	\delta z=0.
\end{equation}
With the above analysis, we confirm the consistency of the BC \eqref{Eq.gener_BC}. If the BC \eqref{BC_1D_x_inf} is used instead, $W(z)$ is undetermined.

\subsection{Gravitational polarization for $n=1$} 


The Lane-Emden equation for polytropic sheet of $n=1$ is
\begin{equation}
	\frac{\text{d}^2\phi}{\text{d}\,z^2}+\phi=0.
\end{equation}
With the BCs in equation \eqref{BC_LE_1D}, the solution reads
\begin{equation}
	\phi_\text{o}(z)=-\cos z.
\end{equation}
From equation \eqref{LE_1D_perturb_reg}, the linearized Lane-Emden equation for potential $W(z)$ is
\begin{equation}
	\frac{\text{d}^2 W}{\text{d}\,z^2}+\left(W+\frac{1}{2}|z|\right)=0.\label{eq.LE_n1}
\end{equation}
With two BCs \eqref{BC_1D_x_0} and \eqref{BC_1D_x_inf}, the solution is on $z>0$
\begin{equation}
	W(z)=\frac{1}{\sqrt{2}}\sin\left(z-\frac{\pi}{4}\right)-\frac{1}{2}z.
\end{equation}
In analogy with the standard electromagnetism, we introduce the following gravitational ``susceptibility''
\begin{equation}
	\chi(z)\equiv -2\delta\phi'(z).
\end{equation}
The quantity $\chi(z)$ is the ratio of the amplified to the \emph{bare} gravitational field due to test mass.  If $\chi(z)$ is greater than one, it means that gravitational field is amplified. It appears suitable as a measure of amplification, we hence call $\chi(z)$ the \emph{gravitational amplification} of fields due to test mass hereafter. For $n=1$, we have
\begin{equation}
	\chi(z)=\sqrt{2}\cos\left(z-\frac{\pi}{4}\right).
\end{equation}
The height $z_{\chi.\text{max}}$ at which gravitational amplification reaches the maximum is $\pi/4$. 
The maximum of gravitational amplification, $\chi_\text{max}$, is $\sqrt{2}$. We also introduce the average of $\chi(z)$
\begin{equation}
	\chi_\text{ave}\equiv\frac{1}{2z_\text{M}}\int^{z_\text{M}}_{-z_\text{M}}\chi\left(z'\right)\,\text{d}z'=\frac{\delta \phi(z_\text{M})-\delta\phi(0)}{z_\text{M}}.
\end{equation}
For $n=1$, $z_\text{M}$ is $\pi/2$. So, the value of $\chi_\text{ave}$ is
\begin{equation}
	\chi_\text{ave}=\frac{4}{\pi}.
\end{equation}

We next introduce the following quantity to examine the degree of shortening of polytropic sheets
\begin{equation}
	\eta\equiv\frac{\delta z}{z_\text{M}}=-\frac{W(z_\text{M})+\frac{1}{2}z_\text{M}}{z_\text{M}\,\phi_\text{o}'(z_\text{M})}.
\end{equation}
For $n=1$, the ratio reads
\begin{equation}
	\eta=-\frac{1}{2\sqrt{2}}.
\end{equation}

The above analytical results are important. First, they are used to confirm our numerical calculation. Second, unlike other indexes, the perturbed polytrope of $n=1$ does not explicitly depend on the unperturbed density as seen in equation \eqref{eq.LE_n1}. Polytropes with $n=1$ may give a hint of \emph{pure} gravitational-polarization effect. Comparing our result to the collisionless homogeneous systems \citep{Marochnik_1968,Padmanabhan_1985}, it seems that ``pure'' gravitational polarization causes a sinusoidal amplification. 
 
