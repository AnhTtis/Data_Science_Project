\documentclass[10pt, a4paper, onecolumn]{article}
\usepackage{geometry}
 \geometry{
 a4paper,
 total={170mm,257mm},
 left=20mm,
 top=20mm,
 }
\newcommand*\circled[4]{\tikz[baseline=(char.base)]{
    \node[shape=circle, fill=#2, draw=#3, text=#4, inner sep=2pt] (char) {#1};}}
%%% The "real" document content comes below...
\usepackage{authblk}
\title{ An analysis of $\mathbb{P}$-invariance and dynamical compensation properties from a control perspective}
\author[]{Akram Ashyani, PhD}
\author[]{Yu-Heng Wu}
\author[]{Huan-Wei Hsu}
\author[]{Torbj{\"o}rn E. M. Nordling, PhD$^*$} 

\affil[]{{Department of Mechanical Engineering}, {National Cheng Kung University}, {No. 1 University Rd.}, {Tainan} {701}, {Taiwan}
\\
$^*$Corresponding author: torbj{\"o}rn.nordling@nordlinglab.org
}
%{* Corresponding author: torbj{\"o}rn.nordling@nordlinglab.org}
%\cortext[cor1]{Corresponding author}

%\setlength\parskip{\baselineskip}
\usepackage{hyperref}
\usepackage[dvipsnames]{xcolor}
\usepackage{verbatim}
\usepackage[round]{natbib}
\usepackage{graphicx}
\usepackage{subcaption}
\usepackage{longtable}
\usepackage{amssymb}
\usepackage{array}
\usepackage{graphicx}
\usepackage{ragged2e}
\usepackage{refstyle}
\usepackage{makecell}
%\usepackage{slashbox} %What is this needed for?
\usepackage{rotating}
\usepackage{xcolor}
\usepackage{amsmath}
\definecolor{B}{RGB}{31,120,180}
\definecolor{LB}{RGB}{166,206,227}
\definecolor{G}{RGB}{51,160,44}
\definecolor{LG}{RGB}{178,223,138}
\graphicspath{{../Figures/}}
\newcolumntype{P}[1]{>{\centering\arraybackslash}p{#1}}
\newcolumntype{M}[1]{>{\centering\arraybackslash}m{#1}}
%This is for figure.
\setcitestyle{numbers}
\setcitestyle{square}
\usepackage{todonotes}
\usepackage{lipsum,calc,multicol}
\usepackage{mathrsfs}
\usepackage{titlesec}
\usepackage{hyperref}
\newcommand{\CommaPunct}{\mathpunct{\raisebox{0.5ex}{,}}}
\usepackage[toc,page]{appendix}

% For images 
\usepackage{graphicx}
\usepackage{caption}
\usepackage{subcaption}
% \usepackage{subfig}
\usepackage{graphicx}
\newsavebox{\bigimage}
% Bold the 'Figure #' in the caption and separate it from the title/caption with a period
% Captions will be left justified
\usepackage[aboveskip=1pt,labelfont=bf,labelsep=period,justification=raggedright,singlelinecheck=off]{caption}
\renewcommand{\figurename}{Fig}

%Remove ugly borders around clickable cross-references and hyperlinks
\hypersetup{
    colorlinks=true,
    pdfborder={0 0 0},
}


%\usepackage[demo]{graphicx}
\usepackage{floatrow}
\floatsetup[table]{capposition=top}

% For the references
%% Imports the package natbib
%\usepackage[sorting=none]{biblatex}
%\bibliographystyle{abbrvnat}
\usepackage{natbib}
% \setcitestyle{sort&compress,square, comma, numbers,authoryear}

\date{}


\begin{document}

\maketitle

\section*{Abstract}
%%% ChatGPT start
Dynamical compensation (DC) provides robustness to parameter fluctuations. As an example, DC enable control of the functional mass of endocrine or neuronal tissue essential for controlling blood glucose by insulin through a nonlinear feedback loop. 
% The dynamical compensation (DC) property provides robustness to parameters fluctuations.
%This property demonstrates how DC, by means of a nonlinear feedback loop, can control the functional mass of the endocrine or neuronal tissue essential for control of blood glucose by insulin. 
Researchers have shown that DC is related to structural unidentifiability and $\mathbb{P}$-invariance property, and $\mathbb{P}$-invariance property is a sufficient and necessary condition for the DC property. 
%Later, researchers have demonstrated that DC shares the same concept with structural unidentifiability and $\mathbb{P}$-invariancy, and more precisely, $\mathbb{P}$-invariance property is a sufficient and necessary condition for the DC property.
In this article, we discuss DC and $\mathbb{P}$-invariancy from an adaptive control perspective. 
An adaptive controller is a self-tuning controller used to compensate for changes in a dynamical system. 
%Adaptive controller is a self-tuning controller to compensate the system to achieve control objectives.
% Adaptive control is used to compensate for changes in dynamical system.
% The ideal adaptive system is one in which the control is so perfect that it is invariant to changing parameters, or state, and the change has no effect.
%We aim to demonstrate that DC can make a system invariant to a parameter and synthesize an ideal adaptive controller. 
% We aim to show that DC can be used to make a system invariant to a parameter and thus synthesis the ideal adaptive controller.
To design an adaptive controller with the DC property, it is easier to start with a two-dimensional dynamical model. 
We introduce a simplified system of ordinary differential equations (ODEs) with the DC property and extend it to a general form. 
The value of the ideal adaptive control lies in developing methods to synthesize DC to variations in multiple parameters.
%We introduced a simplified system of ordinary differential equations (ODEs) with DC property and further extended this simplified system to a general form.
%The value of ideal adaptive control in practical applications is developing methods for synthesis of DC to variation in multiple parameters.
%%% ChatGPT end
Then we investigate the stability of the system with time-varying input and disturbance signals, with a focus on the system's $\mathbb{P}$-invariance properties. 
%The system exhibits stable behavior around an interior equilibrium point, with the input and disturbance signals remaining constant during certain time intervals and changing at other intervals. 
%The study analyzes the system's response to different combinations of input and disturbance signals and examines the $\mathbb{P}$-invariance properties of the system's parameters. 
%The results show that one of the parameters has a $\mathbb{P}$-invariance property, while the other parameter does not.
This study provides phase portraits and step-like response graphs to visualize the system's behavior and stability properties.
\\
\textbf{Keywords} Dynamical compensation property; $\mathbb{P}$-invariance property; ordinary differential equations; adaptive proportional-integral feedback
%\end{itemize}
\\
\textbf{Mathematics Subject Classification 2010} 93B11, 93B52




\section{Introduction}
%% ChatGPT start
Dynamical compensation (DC) implies that the output of a system does not depend on a parameter for any input \citep{Karin2016}. 
% \cite{Karin2016} defined dynamical compensation (DC) property as ``For any input $u(t)$ and any constant $s$ the output of the system $y(t,s)$ does not depend on $s$".
For instance, in glucose homeostasis controlled by insulin, despite parameter variations, the glucose response remains identical. 
This definition of the DC property is a sufficient condition and implies that the parameter is structurally unidentifiable \citep{Villaverde2017Dynamic,Villaverde2017DynamicalAndDtructuralIdentifiability,Villaverde2022}. 
In 2017, a necessary and sufficient condition for the DC property was introduced using equivariances and partial differential equations, denoted as the $\mathbb{P}$-invariance property \citep{Sontag2017}. 
The $\mathbb{P}$-invariance property related to a parameter indicates that changing the parameter does not alter the system's behavior, which is useful in biological and medical models. 
This phenomenon is especially advantageous when a change in a parameter has no effect on the output, allowing the system's behavior to be predicted. 
%Ideally, the system should reserve stability without any bifurcation resulting in system's behavior \citep{Ashyani2016,Ashyani2016b,Ashyani2018}. \todo[inline]{What are you trying to say in this last sentence? Please rephrase, I don not understand}

Robustness, which refers to a system's ability to handle fluctuations, is critical in dynamical systems. 
Several studies on adaptation and homeostasis have demonstrated the robustness of biological systems, such as the robustness of bacterial chemotaxis \citep{Barkai:1997:Robustness-in-simple-biochemical-networks.:oa,Alon1999}. 
The application of DC and $\mathbb{P}$-invariance properties is also beneficial in epidemiological models \citep{Sauer2021,Browning2020}. 
Therefore, the DC property may be included in future robustness research. 
Karin et al. used the glucose homeostasis model to discuss the robustness and DC property of homeostasis \citep{Karin2016}.

Several mathematical models based on systems of differential equations have been developed to comprehensively analyze biological observations and identify all possible connections \citep{Ashyani2016,Ashyani2016b,Ashyani2018}. 
However, it is often more convenient to work with simpler models with fewer dimensions, as they are easier to interpret and analyze.
In this paper, we aim to simplify the original model in Karin et al. and include another feedback mechanism to derive an extended model in section \ref{mathematical_model}. 
We began by checking the system's stability in section \ref{sec:results} because the system must be stable to check the DC and $\mathbb{P}$-invariance properties. 
We use the phase portrait approach to verify the system's stability and obtain some results, for preferred stable situations, to compare the results of the DC and $\mathbb{P}$-invariance properties. 
Finally, in the numerical simulation in section \ref{Numerical_simulation}, we considered situations in which the system is stable at desired equilibrium points and demonstrate the impact of adaptive control and $\mathbb{P}$-invariance in the system when it is perturbed.
%%% ChatGPT end

%%%%%%%%%%%%%%%%%%%%%%%%%%%%%%%%%%%%%%%%%%%%%%%%%%%%%%%%%%%%%%%%%%%%%%%%%%%%%%%%
%\todo[inline]{P-invariance is a method used mostly for showing that there is not an equivariance family with the condition for p-invariancy and then it shows that related to that parameter the system is not p-invariance. But with DC property we can just show that it has DC property and there is not a way to show that it does not have DC property. This is because the DC property by Karin et. al. is sufficient condition but the Sontag way is a sufficient and necessary condition. Here we want to check these two ways in showing DC property in three different parameters. Then we see them as a control perspective.}
\section{Mathematical model}
\label{mathematical_model}
	In our study, as a starting point, we used the hormonal circuit reactions model stated in Karin et al. \cite{Karin2016};
	\begin{subequations}
\label{eq:dc_original_model}
	\begin{align}
\label{eq:dc_original_model_1}	\frac{dy}{dt} &= u_{0} + u(t) - sxy, \\
\label{eq:dc_original_model_2}	\frac{dx}{dt} &= pzy-x, \\
\label{eq:dc_original_model_3}	\frac{dz}{dt} &= z(y-y_{0}),
	\end{align}
	\end{subequations}
	where $s$ and $p$ are the feedback gains of $x$ and $z$, respectively.
	The output variable, $y$, is a regulated variable that is able to form a feedback loop with $x$ and $z$.
	The regulated variable $y$ controls the functional mass $z$ of tissue which secretes hormone $x$ in this circuit.
	The aim is to first simplify the model \ref{eq:dc_original_model}, containing Eqs. \ref{eq:dc_original_model_1}-\ref{eq:dc_original_model_3}, then control the system with adaptive proportional-integral feedback so that it has $\mathbb{P}$-invariance property. 
We also compared the differences between DC and $\mathbb{P}$-invariance property. 

	We simplified the model \ref{eq:dc_original_model} as, 
	\begin{subequations}
\label{eq:dc_simplified}
	\begin{align}
	\label{simp1}
	\frac{dy}{dt} &= u_{0} + u(t) - szy, \\
	\label{simp2}
	\frac{dz}{dt} &= z(y-y_{0}),
	\end{align}
	\end{subequations} 
where  $z$ is the feedback state, and $y$ is the output of the system.
	Our expectation is that the positive constant $s$ has the DC property, meaning that the output $y$ is invariant to the change of the parameter $s$. %, so the variable substitution should yield the result that $y$ is invariant to $s$.
	Hence we introduce $\tilde{z} = sz$ and substitute $z$ in Eq. \ref{simp1} and Eq. \ref{simp2} with $\tilde{z}$ resulting in
	\begin{subequations}
		\label{eq:dc_simplified_variable_substitution}
		\begin{align}
		\frac{dy}{dt} &= u_{0} + u(t) - \tilde{z}y, \\
		\frac{d\tilde{z}}{dt} &= \tilde{z}(y-y_{0}).
		\end{align}
	\end{subequations}
	The above Equations show that the output response $y$ remains the same when the value of $s$ changes.


%	We derived the extended model based on our simplified model to make an adaptive proportional-integral feedback model
By extending our simplified model with a DC property in parameter $s$, we created an adaptive proportional-integral feedback model
	\begin{subequations}\label{extend}
	\begin{align}
		\label{extend1}
		\frac{dy}{dt} &= by(t) + d(t) + sz(t)\big(lr(t)-y(t)\big) , \\
		\label{extend2}
		\frac{dz}{dt} &= -cz(t)\big(r(t)-y(t)\big).
	\end{align}
	\end{subequations}
%%% ChatGPT start
The system can be viewed as an open-loop exponential growth system $\frac{dy}{dt} = by(t)$, where $d(t)$ and $r(t)$ represent the disturbance and reference input, respectively. 
The error term is given by $r(t)-y(t)$ and $lr(t)-y(t)$, and the adaptive proportional-integral feedback is $sz(t)\big(lr(t)-y(t)\big)$, where $sz(t)$ is considered as the adaptive proportional-integral gain. 
In control theory, a reference input refers to an input signal that guides the system response. 
Typically, the goal is to make the response $y(t)$ track the reference input $r(t)$, such that the error term is zero ($r(t)-y(t)=0$) at the equilibrium point. 
Furthermore, since we consider this equation in the context of biological phenomena, all parameters are assumed positive. 
This implies that $b$, $s$, $l$, and $c$ are all positive, and for every $t>0$, all $y(t)$, $z(t)$, and $r(t)$ are positive. 
The block diagram of the adaptive proportional-integral system is illustrated in Fig. \ref{fig:block_diagram_adaptive_PI}.
%%% ChatGPT end
\begin{figure}[h]
	\begin{center}
		\includegraphics[width=0.8\columnwidth]{DC_adaptive_proportional_integral_feedback_block_diagram_RW20221109}
%Link: https://drive.google.com/open?id=1XFRt7gJd9EFldlWhWfOZJ6xSdru-ZQgB&authuser=10709076%40gs.ncku.edu.tw&usp=drive_fs
% DC\Block_diagram
	\end{center}
	\caption{Block diagram shows the adaptive proportional-integral feedback $ sz(t)\big(lr(t)-y(t)\big)$ where $sz(t)$ is the adaptive proportional-integral gain with two error terms  $r(t)-y(t)$ and $lr(t)-y(t)$. 
Term $d(t)$ represents the disturbance.}
	\label{fig:block_diagram_adaptive_PI}
\end{figure}	
	%Reflecting this model to biological systems, $y$, $z$, and $r$ stands for the amount of free protein. 	The protein $y$ and $r$ with affinity $l_{1}$ and $l_{2}$, which means the proportion when binding, are able to bind with $z$, which represents the transcription factor to this mechanism. 	Other parameters, $a$, $b$, $c$, $d$, $k_{1}$, $k_{2}$ represent constant, proportional feedback gain, strength of $z$, disturbance, adaptive integral feedback gain, integral feedback gain, respectively. 	We consider the term $k_1z$ and $k_2z$ are the adaptive integral feedback and the adaptive proportional feedback. 	We assumed that $k_{1}$ and $c$ have no connection with the dynamics, and the steady-state of $y$ is unaffected by $a$, $b$, $d$, $k_{2}$.

%%% ChatGPT start
To verify the DC property of our model, the system should be at an equilibrium point before being perturbed by any input. 
When a system is at an equilibrium point, its value does not change with time. 
We then triggered the system with a step-like input $r(t)$ to sketch the response $y(t)$. 
%Then with a step-like input $r(t)$ to the system we sketch the response $y(t)$.
We adjust the value of each parameter in Eqs. \ref{extend1} and \ref{extend2} to observe how they affect the system. 
%	Furthermore, we sought to explore the constraints, namely the region of attractions, which guided us to the phase portrait approach. 
The stability region is discovered by drawing the phase portrait.
%%% ChatGPT end

%%%%%%%%%%%%%%%%%%%%%%%%%%%%%%%%%%%%%%%%%%%%%%%%%%%%%%%%%%%%%%%%%%%%%%%%%%%%%%%%
	\section{Results}
	\label{sec:results}
%	Moreover, we modify the value of parameter $s$ and seek to show DC for this simplified model.
%	\todo[inline]{Add a Fig.  showing the case when the Karin2016 model and the Menolascina-Nordling model differs.}
%	\todo[inline]{Look in the literature for adaptive control and find an example of the same model as we have here. Write a block diagram of this system where the integral feedback is in the controller.}
%Reflecting this model to biological system, $y$, $z$, and $r$ stands for the amount of free protein.
%The protein $y$ is able to bind with $r$, where $l$ is their binding affinity, i.e. the proportion when binding, $z$ represents the transcription factor to this mechanism.                                                                                                                                                                                                                                                                                                                                                                                                                                                                                                                                                                                                                                                                                                                                                                                                                                                                                                                                                                                                                                                                                                                                                                                                                                                                                                                                                                                                                                                                                                                                                                                                                                                                                                                                                                                                                                                                                                                                                                                                                                                                                                                                                                                                                                                                                                                                                                                                                                                                                                                                                                                                                                                                                                                                                                                                                                                                    
Here, we began by checking stability of the system, and then we compared the differences between $\mathbb{P}$-invariance and DC property. 
Finally, we provided a numerical example to illustrate the result.
\subsection{Phase portrait and stability}
\label{append:dc_modified_phase}
Our goal is to discover the region of attraction by drawing the phase portrait. 
By setting the derivative terms in Eqs. \ref{extend1} and \ref{extend2} to zero, two equilibrium points can be obtained: %left-hand side of the Eq.s
\begin{align}
\label{equilibrium point1}
&E_1=(y_1,z_1)=\big(-\frac{d(t)}{b},0\big), \\
\label{equilibrium point2}
&E_2=(y_2,z_2)=\big(r(t),\frac{d(t)+br(t)}{sr(t)(1-l)}\big).
\end{align}
%%% ChatGPT start
Under the assumption that all parameters are non-negative and the signals $d(t)$ and $r(t)$ are positive at some timepoint(s) and non-negative at all other timepoints, we note the following:
Since $y_1$ is negative in the equilibrium point $E_1$, it is a biologically infeasible state of the system. 
%Therefore, understanding the behavior around $E_2$ is desirable. 
If $0<l \leq 1$, then both $z_2$ and $y_2$ are non-negative, making $E_2$ the equilibrium point of interest.
To ensure that $z_2$ remain finite, we first assume $0<l<1$. %\todo[inline]{Can you please later discuss what happen in the limit as l approach 1? It is case of interests.}
The local stability of a system can be analyzed by calculating the eigenvalues of the matrix of partial derivatives in equilibrium points, known as the Jacobian matrix. 
The matrix of partial derivatives for system \ref{extend} and its eigenvalues are shown below.
%%% ChatGPT end
\begin{align}
	\label{eq:dc_modified_Jacobian}
		\boldsymbol{J}(y,z) &= \begin{bmatrix}
	b-sz(t) & s\big(lr(t)-y(t)\big)\\
	cz(t) & -c\big(r(t)-y(t)\big)
	\end{bmatrix}, \\
	\lambda(y,z) &=\frac{1}{2}\Big(
	b-sz(t)-c\big(r(t)-y(t)\big) 
 \pm \scriptstyle \sqrt{\Big(b-sz(t)-c\big(r(t)-y(t)\big)\Big)^2-4c\Big(z(t)\big(sr(t)(1-l)\big)-b\big(r(t)-y(t)\big)\Big)} \Big).
\end{align}
The Jacobian matrix is presented in an algebraic structure to calculate the eigenvalues easier when analyzing the local stability of the individual equilibrium point.

{\bf{1) Local stability of $E_1$:}}
To investigate the local stability around $E_1$, we computed two eigenvalues.
\begin{align}
 &\lambda_1(y_1,z_1)=b,\hspace{5mm}
 \lambda_2(y_1,z_1)= -c\big(d(t)+b r(t)\big)/b.
\end{align}
As $b>0$ and $-c\big(d(t)+br(t)\big)/b<0$ this equilibrium point is a saddle point.

{\bf{2) Local stability of $E_2$:}}
For the equilibrium point $E_2$ the eigenvalues are
\begin{align}
\lambda_{1}=\frac{\tau+\sqrt{\tau^2-4\delta}}{2}, \hspace{5mm}
	\lambda_{2}=\frac{\tau-\sqrt{\tau^2-4\delta}}{2},
\end{align}
where
\begin{align}
\tau&=\mathrm{trace} \big(J(y_2,z_2)\big)=\frac{d(t)+blr(t)}{r(t)(l-1)},\\
\delta&=\det \big(J(y_2,z_2)\big)= c\big(d(t)+br(t)\big).
\end{align}
Three situations can happen:
\begin{enumerate}
\item[1)]$\tau^2-4\delta=0$,
\item[2)]$\tau^2-4\delta<0$,
\item[3)]$\tau^2-4\delta>0$.
\end{enumerate}
%%% ChatGPT start
In both (1) and (2), stability depends on $\tau$. 
Hence, if $\tau <0$, then $E_2$ is stable. 
Based on the assumption that parameters and variables are positive to be meaningful in biology and $l<1$, we have $\tau <0$, which means that $E_2$ is a stable equilibrium point. 
In situation (3), as $\delta >0$, it will result in $|\tau|>\sqrt{\tau^2-4\delta}$. 
Hence, if $\tau <0$, then $E_2$ is stable. 
Again, based on the assumption of having meaningful parameters in the equilibrium points, $l<1$, the equilibrium $E_2$ is stable. 
These findings demonstrate that as long as $E_2$ is meaningful in biology, it is a globally stable equilibrium point.
%%% ChatGPT end
%%%%%%%%%%%%%%%%%%%%%%%%%%%%%%%%%%%%%%%%%%%%%%%%%%%%%%%       P-invariance and DC property
\subsection{Influence of the adaptive controller on stability}
\label{subsec:dc_p-invariance_result}
Our aim is to investigate the influence of adaptive controller term on the stability of the system.
	Given a system
	\begin{align}
	\label{eq:dc_pinvariance_def_system}
	\dot{x} = f(x(t),u(t),p),\; y = g(x(t),u(t),p), \; x(0) = \gamma_{p};
	\end{align}
	if there exists an equivalent transformation
	\begin{subequations}
	\label{eq:dc_pinvariance_def}
	\begin{align}
	\label{p_invariancy_f} f(\eta_{p}(x),u,p) &= (\eta_{p})_*(x)f(x,u), \\
	\label{p_invariancy_g} g(\eta_{p}(x),u,p) &= g(x,u), \\
	\label{p_invariancy_eta} \eta_{p}(\gamma) &= \gamma_{p},
	\end{align}
	\end{subequations}
	where $\eta_*$ denotes the Jacobian matrix of transformation $\eta$, the system have the $\mathbb{P}$-invariance property \cite{Sontag2017}.
%%% ChatGPT start
By verifying the invariance of the system \ref{extend}, containing Eqs. \ref{extend1}-\ref{extend2}, we were able to discover the parameters that lead to the DC property. 
We also demonstrated the differences between the definitions of $\mathbb{P}$-invariance and the DC property. 
Based on the definition of the $\mathbb{P}$-invariance property and the relationship with the DC property, the DC property can be classified as an adaptive control strategy in the system \ref{extend}.
%%% ChatGPT end



\subsubsection{Verification of the $\mathbb{P}$-invariance property}
	%$\mathbb{P}$-invariance is a necessary condition for a system to demonstrate DC property \citep{Sontag2017}.
	We verify that the system is $\mathbb{P}$-invariant with respect to variation of $s$.
	In order to verify that the system \ref{extend} has the $\mathbb{P}$-invariance property, we introduced $x_{1}(t)$ and $x_{2}(t)$ as two state variables and $y(t)$ as the output variable of the system. 
For simplicity, we wrote the system \ref{extend} in $x_1$, $x_2$, and $y$ as,
	\begin{subequations}\label{eq:dc_extended_control}
		\begin{align}
		\label{eq:dc_extended_x1}
		\dot{x_{1}} &=  -cx_{1}\big(r(t)-x_{2}\big), \\
		\label{eq:dc_extended_x2}
		\dot{x_{2}} &=  bx_{2} + d(t)+sx_{1}\big(lr(t)-x_{2}\big), \\
		\label{eq:dc_extended_y}
		y &= x_{2}.
		\end{align}
	\end{subequations}
	The notation here is selected to be identical to the one used by \cite{Sontag2017}. 
%%%%%%%%%%%%%%%%%%%%%%%%%
%Before checking these parameters one by one we will check if we have $\mathbb{P}$-invariance for $s$.
We considered the possible equivariance $\eta_p(x_1,x_2)=\big(\alpha_p(x_1,x_2), \beta_p(x_1,x_2)\big)$. 
In this case, the condition $g\big(\eta_p(x),u,p\big)=g(x,u)$ means $\beta_p(x_1,x_2)=x_2$. 
Therefore, we have $\eta_p(x_1,x_2)=\big(\alpha_p(x_1,x_2), x_2\big)$. 
Hence, 
\begin{align}
(\eta_p)_*(x_1,x_2)&=
\begin{bmatrix}
\frac{\partial \alpha_p}{\partial x_1}(x_1,x_2) & \frac{\partial \alpha_p}{\partial x_2}(x_1,x_2)\\
\frac{\partial x_2}{\partial x_1}&\frac{\partial x_2}{\partial x_2}
\end{bmatrix}
=\begin{bmatrix}
\frac{\partial \alpha_p}{\partial x_1}(x_1,x_2) & \frac{\partial \alpha_p}{\partial x_2}(x_1,x_2)\\
0&1
\end{bmatrix}.
\end{align}
%Then we should check $f(\rho_p(x),u,p))=(\rho_p)_*(x)f(x,u)$ for the three desired parameters $k_1$, $k_2$ and $b$, so we should set $p$ to be $k_1$, $k_2$ and $b$.
As a result from equation \ref{p_invariancy_f}, for parameter $s$ our aim is to prove 
$ f(\eta_{s}(x),u,s) = (\eta_{s})_*(x)f(x,u)$.
It means:
\begin{align}
\begin{bmatrix}
-c\alpha_{s}(x_{1},x_{2})\big(r(t)-x_{2}\big)  \\
bx_{2}+d(t)+s\alpha_{s}(x_{1},x_{2})\big(lr(t)-x_{2}\big)
\end{bmatrix}
=
\begin{bmatrix}
\frac{\partial \alpha_s}{\partial x_1}(x_1,x_2) & \frac{\partial \alpha_s}{\partial x_2}(x_1,x_2)\\
0&1
\end{bmatrix}
\begin{bmatrix}
-cx_{1}\big(r(t)-x_{2}\big)
\\
bx_{2}+d(t)+x_{1}\big(lr(t)-x_{2}\big)
\end{bmatrix}.
\end{align}
Hence:
	\begin{align}
	\label{p-invariance_x1_k1}
	& -c\alpha_{s}(x_{1},x_{2})\big(r(t)-x_{2}\big) = \frac{\partial\alpha_{s}(x_{1},x_{2})}{\partial x_{1}}\Big(-cx_{1}\big(r(t)-x_{2}\big)\Big) 
+ \frac{\partial\alpha_{s}(x_{1},x_{2})}{\partial x_{2}}\Big(bx_{2}+d(t)+x_{1}\big(lr(t)-x_{2}\big)\Big),	\\
	\label{p-invariance_x2_k1}
		&bx_{2}+d(t)+s\alpha_{s}(x_{1},x_{2})\big(lr(t)-x_{2}\big)=
	bx_{2}+d(t)+x_{1}\big(lr(t)-x_{2}\big).
	\end{align}
By comparing the coefficients in  Eq. \ref{p-invariance_x1_k1} we have
\begin{align}
\frac{\partial\alpha_{s}(x_{1},x_{2})}{\partial x_{1}}&=\frac{\alpha_{s}(x_{1},x_{2})}{x_{1}},\\
\frac{\partial\alpha_{s}(x_{1},x_{2})}{\partial x_{2}}&=0.
\end{align}
From  Eq. \ref{p-invariance_x2_k1} we attained
\begin{align}
s\alpha_{s}(x_{1},x_{2})(lr(t)-x_{2})=x_{1}\big(lr(t)-x_{2}\big),
\end{align}
If $lr(t)-x_{2} \neq 0$, it means 
\begin{align}
\alpha_{s}(x_{1},x_{2})=\frac{x_1}{s}.
\end{align}
%Therefore,
%\begin{align}
%\label{alpha(p)_1_1}\frac{\partial\alpha_{p}(x_{1},x_{2})}{\partial x_{1}}&=\frac{1}{s},\\
%\label{alpha(p)_2_2}\frac{\partial\alpha_{p}(x_{1},x_{2})}{\partial x_{2}}&=0.
%\end{align}
%%% ChatGPT start
Therefore, there is a Jacobian matrix $\alpha_{s}(x_{1},x_{2})= x_1/s$ that can achieve the transformation of the system. 
The system could demonstrate $\mathbb{P}$-invariance when $s$ is the $\mathbb{P}$-invariance parameter.
%Hence, in this case $k$ is the $\mathbb{P}$-invariance parameter and system has $\mathbb{P}$-invariance property.
%
%\todo[inline]{Add a paragraph demonstrating that the system is $\mathbb{P}$-invariant in parameter $c$. Once you have shown this add verification of it's DC property in 3.2.2. I already added $c$ with $s$ to the Conclusion}
%We showed that the system is also $\mathbb{P}$-invariant in the parameter $c$.
%By introducing $\tilde{x}_1 = c x_1$ and $\tilde{s} = \frac{s}{c}$, and substituting them in the system \ref{eq:dc_extended_control}, we obtained
%	\begin{subequations}\label{eq:dc_extended_controlsub}
%		\begin{align}
%		\label{eq:dc_extended_x1sub}
%		\dot{\tilde{x}}_{1} &=  -\tilde{x}_{1}\big(r(t)-x_{2}\big), \\
%		\label{eq:dc_extended_x2sub}
%		\dot{x}_{2} &=  bx_{2} + d(t)+\tilde{s} \tilde{x}_{1}\big(lr(t)-x_{2}\big), \\
%		\label{eq:dc_extended_ysub}
%		y &= x_{2}.
%		\end{align}
%	\end{subequations}
%The $s$ in system \ref{eq:dc_extended_control} was replaced by another value $\tilde{s}$ that contain the effect of changing $c$. 
%Since the system is $\mathbb{P}$-invariant in parameter $s$ it is also $\mathbb{P}$-invariant in the parameter $c$.

%%%%%%%%%%%%%%%%%%%%%%%%%%%%%%%             P-invariance for b
%\todo[inline]{This part related to p-invriancy for varied b  should be like below that I wrote. Because it seems that h should be output so it is always x2}
Next, we investigated whether the parameter $b$ has the $\mathbb{P}$-invariance property, meaning that changing $b$ will not influence the behavior of $y(t)$.
%%% ChatGPT end
As a result from equation \ref{p_invariancy_f}, for parameter $b$ our aim is to prove 
$ f(\eta_{b}(x),u,b) = (\eta_{b})_*(x)f(x,u)$.
It means:
\begin{align}
\begin{bmatrix}
-c\alpha_{b}(x_{1},x_{2})\big(r(t)-x_{2}\big)  \\
bx_{2}+d(t)+s\alpha_{b}(x_{1},x_{2})\big(lr(t)-x_{2}\big)
\end{bmatrix}
=
\begin{bmatrix}
\frac{\partial \alpha_b}{\partial x_1}(x_1,x_2) & \frac{\partial \alpha_b}{\partial x_2}(x_1,x_2)\\
0&1
\end{bmatrix}
\begin{bmatrix}
-cx_{1}\big(r(t)-x_{2}\big)
\\
x_{2}+d(t)+sx_{1}\big(lr(t)-x_{2}\big)
\end{bmatrix}.
\end{align}
	Thus, it is essential to solve
	\begin{align}
	\label{p-invariance_b_1}
		&-c\alpha_{b}(x_{1},x_{2})\big(r(t)-x_{2}\big) = \frac{\partial\alpha_{b}(x_{1},x_{2})}{\partial x_{1}}\Big(-cx_{1}\big(r(t)-x_{2}\big) \Big) 
	+\frac{\partial\alpha_{b}(x_{1},x_{2})}{\partial x_{2}}\Big(x_{2}+d(t)+sx_{1}\big(lr(t)-x_{2}\b)\Big),
	\\
	\label{p-invariance_b_2}
		&bx_{2}+d(t)+s\alpha_{b}(x_{1},x_{2})\big(lr(t)-x_{2}\big)=
	x_{2}+d(t)+sx_{1}\big(lr(t)-x_{2}\big).
	\end{align}
By comparing the coefficients in Eq. \ref{p-invariance_b_1}, we have
\begin{align}
\label{p-invariance_b1}\frac{\partial\alpha_{b}(x_{1},x_{2})}{\partial x_{1}}&=\frac{\alpha_{b}(x_{1},x_{2})}{x_{1}},\\
\label{p-invariance_b2}\frac{\partial\alpha_{b}(x_{1},x_{2})}{\partial x_{2}}&=0.
\end{align}
From  Eq. \ref{p-invariance_b_2} we have
\begin{align}
  bx_2+s\alpha_{b}(x_{1},x_{2})\big(lr(t)-x_{2}\big)=
x_2+sx_{1}\big(lr(t)-x_{2}\big),
\end{align}
If $lr(t)-x_{2} \neq 0$, it means 
\begin{align}\label{b}
\alpha_{b}(x_{1},x_{2})=\frac{x_2+sx_{1}\big(lr(t)-x_{2}\big)-bx_2}{s\big(lr(t)-x_{2}\big)},
\end{align}
and yields
\begin{align}
\label{p-invariance_b1_1}\frac{\partial\alpha_{b}(x_{1},x_{2})}{\partial x_{1}}&=1,\\
\frac{\partial\alpha_{b}(x_{1},x_{2})}{\partial x_{2}}&=\frac{(1-b)\big(slr(t)\big)}{\Big(s\big(lr(t)-x_{2}\big)\Big)^2}.
\end{align}
	There is no solution of $\alpha_{b}(x_{1},x_{2})$ that can be obtained from Eq. \ref{b} and satisfies the two conditions in Eqs. \ref{p-invariance_b1} and \ref{p-invariance_b2}.
	This implies that the system is not $\mathbb{P}$-invariant in $b$.


%%%%%%%%%%%%%%%%%%%%%%%%%%%%%%%%%             P-invariance for c
%Next, we investigated whether the parameter $c$ has the $\mathbb{P}$-invariance property, meaning that changing $c$ will not influence the behavior of $y(t)$.
%%%% ChatGPT end
%As a result from equation \ref{p_invariancy_f}, for parameter $c$ our aim is to prove 
%$ f(\eta_{c}(x),u,b) = (\eta_{c})_*(x)f(x,u)$.
%It means:
%\begin{align}
%\begin{bmatrix}
%-c\alpha_{c}(x_{1},x_{2})\big(r(t)-x_{2}\big)  \\
%bx_{2}+d(t)+s\alpha_{c}(x_{1},x_{2})\big(lr(t)-x_{2}\big)
%\end{bmatrix}
%=
%\begin{bmatrix}
%\frac{\partial \alpha_c}{\partial x_1}(x_1,x_2) & \frac{\partial \alpha_c}{\partial x_2}(x_1,x_2)\\
%0&1
%\end{bmatrix}
%\begin{bmatrix}
%-x_{1}\big(r(t)-x_{2}\big)
%\\
%bx_{2}+d(t)+sx_{1}\big(lr(t)-x_{2}\big)
%\end{bmatrix}.
%\end{align}
%	Thus, it is essential to solve
%	\begin{align}
%	\label{p-invariance_c_1}
%		&-c\alpha_{c}(x_{1},x_{2})\big(r(t)-x_{2}\big) = \frac{\partial\alpha_{c}(x_{1},x_{2})}{\partial x_{1}}\Big(-x_{1}\big(r(t)-x_{2}\big) \Big) 
%	+\frac{\partial\alpha_{c}(x_{1},x_{2})}{\partial x_{2}}\Big(bx_{2}+d(t)+sx_{1}\big(lr(t)-x_{2}\b)\Big),
%	\\
%	\label{p-invariance_c_2}
%		&bx_{2}+d(t)+s\alpha_{c}(x_{1},x_{2})\big(lr(t)-x_{2}\big)=
%	bx_{2}+d(t)+sx_{1}\big(lr(t)-x_{2}\big).
%	\end{align}
%By comparing the coefficients in Eq. \ref{p-invariance_c_1}, we have
%\begin{align}
%\label{p-invariance_c1}\frac{\partial\alpha_{c}(x_{1},x_{2})}{\partial x_{1}}&=\frac{c \alpha_{c}(x_{1},x_{2})}{x_{1}},\\
%\label{p-invariance_c2}\frac{\partial\alpha_{c}(x_{1},x_{2})}{\partial x_{2}}&=0.
%\end{align}
%From  Eq. \ref{p-invariance_c_2} we have
%\begin{align}
%  s\alpha_{c}(x_{1},x_{2})\big(lr(t)-x_{2}\big)=
%sx_{1}\big(lr(t)-x_{2}\big),
%\end{align}
%If $lr(t)-x_{2} \neq 0$, it means 
%\begin{align}\label{c}
%\alpha_{c}(x_{1},x_{2})=\frac{sx_{1}\big(lr(t)-x_{2}\big)}{s\big(lr(t)-x_{2}\big)}=x_1,
%\end{align}
%and yields
%\begin{align}
%\label{p-invariance_c1_1}\frac{\partial\alpha_{c}(x_{1},x_{2})}{\partial x_{1}}&=1,\\
%\frac{\partial\alpha_{c}(x_{1},x_{2})}{\partial x_{2}}&=0.
%\end{align}
%Therefore, there is a Jacobian matrix $\alpha_{c}(x_{1},x_{2})= x_1$ that can achieve the transformation of the system. 
%The system could demonstrate $\mathbb{P}$-invariance when $c$ is the $\mathbb{P}$-invariance parameter.
%%	There is no solution of $\alpha_{c}(x_{1},x_{2})$ that can be obtained from Eq. \ref{c} and satisfies the two conditions in Eqs. \ref{p-invariance_c1} and \ref{p-invariance_c2}.
%%	This implies that the system is not $\mathbb{P}$-invariant in $c$.


%%%%%%%%%%%%%%%%%%%%%%%%%%%%%%%%  Consider z as output

	Next we verify that the system is not $\mathbb{P}$-invariant with respect to variation of $c$.
	In order to verify that the system \ref{extend} has the $\mathbb{P}$-invariance property, we introduced $x_{1}(t)$ and $x_{2}(t)$ as two state variables and $z(t)$ as the output variable of the system. 
For simplicity, we wrote the system \ref{extend} in $x_1$, $x_2$, and $z$ as,
	\begin{subequations}\label{eq:dc_extended_control}
		\begin{align}
		\label{eq:dc_extended_z1}
		\dot{x_{1}} &=  bx_{1} + d(t)+sx_{2}\big(lr(t)-x_{1}\big), \\
\label{eq:dc_extended_z2}
		\dot{x_{2}} &=  -cx_{2}\big(r(t)-x_{1}\big), \\
		\label{eq:dc_extended_z}
		z &= x_{2}.
		\end{align}
	\end{subequations}
	The notation here is selected to be identical to the one used by \cite{Sontag2017}. 

We considered the possible equivariance $\eta_p(x_1,x_2)=\big(\alpha_p(x_1,x_2), \beta_p(x_1,x_2)\big)$. 
In this case, the condition $g\big(\eta_p(x),u,p\big)=g(x,u)$ means $\beta_p(x_1,x_2)=x_2$. 
Therefore, we have $\eta_p(x_1,x_2)=\big(\alpha_p(x_1,x_2), x_2\big)$. 
Hence, 
\begin{align}
(\eta_p)_*(x_1,x_2)&=
\begin{bmatrix}
\frac{\partial \alpha_p}{\partial x_1}(x_1,x_2) & \frac{\partial \alpha_p}{\partial x_2}(x_1,x_2)\\
\frac{\partial x_2}{\partial x_1}&\frac{\partial x_2}{\partial x_2}
\end{bmatrix}
=\begin{bmatrix}
\frac{\partial \alpha_p}{\partial x_1}(x_1,x_2) & \frac{\partial \alpha_p}{\partial x_2}(x_1,x_2)\\
0&1
\end{bmatrix}.
\end{align}
%Then we should check $f(\rho_p(x),u,p))=(\rho_p)_*(x)f(x,u)$ for the three desired parameters $k_1$, $k_2$ and $b$, so we should set $p$ to be $k_1$, $k_2$ and $b$.
As a result from equation \ref{p_invariancy_f}, for parameter $c$ our aim is to prove 
$ f(\eta_{c}(x),u,c) = (\eta_{c})_*(x)f(x,u)$.
It means:
\begin{align}
& \nonumber
\begin{bmatrix}
b\alpha_{c}(x_{1},x_{2}) + d(t)+sx_{2}\big(lr(t)-\alpha_{c}(x_{1},x_{2})\big)\\
-cx_{2}\big(r(t)-\alpha_{c}(x_{1},x_{2})\big)
\end{bmatrix}
=  \\
&
\begin{bmatrix}
\frac{\partial \alpha_c}{\partial x_1}(x_1,x_2) & \frac{\partial \alpha_c}{\partial x_2}(x_1,x_2)\\
0&1
\end{bmatrix}
\begin{bmatrix}
bx_{1} + d(t)+sx_{2}\big(lr(t)-x_{1}\big)
\\
-x_{2}\big(r(t)-x_{1}\big)
\end{bmatrix}.
\end{align}
Hence:
	\begin{align}
	\nonumber & b\alpha_{c}(x_{1},x_{2}) + d(t)+sx_{2}\big(lr(t)-\alpha_{c}(x_{1},x_{2})\big) = \frac{\partial\alpha_{c}(x_{1},x_{2})}{\partial x_{1}}\Big(bx_{1} + d(t)+sx_{2}\big(lr(t)-x_{1}\big)\Big) \\
\label{p-invariance_x1_c} &
+ \frac{\partial\alpha_{c}(x_{1},x_{2})}{\partial x_{2}}\Big(-x_{2}\big(r(t)-x_{1}\big)\Big),	\\
	\label{p-invariance_x2_c}
		&-cx_{2}\big(r(t)-\alpha_{c}(x_{1},x_{2})\big)=
	-x_{2}\big(r(t)-x_{1}\big).
	\end{align}
By comparing the coefficients in  Eq. \ref{p-invariance_x1_c} we have
\begin{align}
\label{p-invariance_c1}\frac{\partial\alpha_{c}(x_{1},x_{2})}{\partial x_{1}}&=\frac{b\alpha_{c}(x_{1},x_{2}) + d(t)+sx_{2}\big(lr(t)-\alpha_{c}(x_{1},x_{2})\big)}{bx_{1} + d(t)+sx_{2}\big(lr(t)-x_{1}\big)},\\
\label{p-invariance_c2}\frac{\partial\alpha_{c}(x_{1},x_{2})}{\partial x_{2}}&=0.
\end{align}
From  Eq. \ref{p-invariance_x2_c} we attained
\begin{align}
c x_2 \alpha_{c}(x_{1},x_{2})=x_{2}\big((c-1)r(t)+x_1),
\end{align}
If $cx_{2} \neq 0$, it means 
\begin{align}
\label{c1} \alpha_{c}(x_{1},x_{2})=\frac{(c-1)r(t)+x_1}{c},
\end{align}
and yields
\begin{align}
\frac{\partial\alpha_{c}(x_{1},x_{2})}{\partial x_{1}}&=\frac{1}{c},\\
\frac{\partial\alpha_{c}(x_{1},x_{2})}{\partial x_{2}}&=0.
\end{align}
	There is no solution of $\alpha_{c}(x_{1},x_{2})$ that can be obtained from Eq. \ref{c1} and satisfies the two conditions in Eqs. \ref{p-invariance_c1} and \ref{p-invariance_c2}.
	This implies that the system is not $\mathbb{P}$-invariant in $c$.


%%%%%%%%%%%%%%%%%%%%%%%%%%%%%%%%                  DC property by Karin et al.
\subsubsection{Verification of the DC property}
As a demonstration to show that $\mathbb{P}$-invariance property is more general than DC property, we used the DC property definition by Karin et al. \citep{Karin2016} in the system \ref{eq:dc_extended_control}, containing Eqs. \ref{eq:dc_extended_x1}-\ref{eq:dc_extended_y}. 
By choosing $v_1=s x_1$ and $ v_2= x_2$, for $s \neq 0$ we have:
\begin{subequations}
\begin{align}
	\label{DC_k_1}
	\dot{v}_{1} &= -cv_1\big(r(t)-v_2\big), \\
	\dot{v}_{2} &= bv_2+d(t)+v_1\big(lr(t)-v_2\big),\\
y&=v_2.
	\end{align}
\end{subequations}
%%% ChatGPT start
Therefore, we can assume $s = 1$, which means DC property with respect to $s \neq 0$.

By choosing $v_1=x_1$ and $ v_2= b x_2$, for $s \neq 0$ we have:
\begin{subequations}
\begin{align}
	\label{DC_k_1}
	\dot{v}_{1} &= -cv_1\big(r(t)-\frac{1}{b}v_2\big), \\
	\dot{v}_{2} &= bv_2+bd(t)+bv_1\big(lr(t)-\frac{1}{b}v_2\big),\\
y&=v_2.
	\end{align}
\end{subequations}
%%% ChatGPT start
Therefore, we cannot assume $b = 1$, which means for $b$, we could not find a transformation to prove DC property, but as DC property is a sufficient condition, we cannot claim that it is not DC property with respect to $b$. %\todo[inline]{Where have you tried to do this? This sentence is confusing, please reformulate.}
%%% ChatGPT end
However, as $\mathbb{P}$-invariance property is a sufficient and necessary condition for DC, we proved that the system does not have DC for variation in $b$.


By choosing $v_1=c x_1$ and $ v_2= x_2$, for $c \neq 0$ we have:
\begin{subequations}
\begin{align}
	\label{DC_k_1}
	\dot{v}_{1} &= -cv_1\big(r(t)-v_2\big), \\
	\dot{v}_{2} &= bv_2+d(t)+\frac{s}{c}v_1\big(lr(t)-v_2\big),\\
y&=v_2.
	\end{align}
\end{subequations}
%%% ChatGPT start
%Therefore, we cannot assume $c = 1$, which means we are not sure whether the system has DC property with respect to $c \neq 0$.
Therefore, we cannot assume $c = 1$, which means for $c$, we could not find a transformation to prove DC property, but as DC property is a sufficient condition, we cannot claim that it is not DC property with respect to $c$.
However, as $\mathbb{P}$-invariance property is a sufficient and necessary condition for DC, we proved that the system does not have DC for variation in $c$.
%%%%%%%%%%%%%%%%%%%%%%%%%%%%%%%%%%%%%%%%%%%%%%%%%%%%%%%%%%

\section{Numerical simulation}
\label{Numerical_simulation}
%%% ChatGPT start
In this section, we discuss and exemplify the theoretical results of our research by numerical simulations. 
We verify the phase portrait, influence of the adaptive controller, and the $\mathbb{P}$-invariance property by using step-like responses for input $r(t)$ and disturbance $d(t)$. 
To investigate the $\mathbb{P}$-invariance property with respect to the parameters $s$ and $b$, the system \ref{extend} is first brought to its equilibrium. 
Next, by perturbing the system with a step-like response $d(t)$ and changes in $s$ and $b$ separately, we check whether it returns to the equilibrium or not.
%\todo[inline]{Make the updates to the values as discussed on Friday.}

As analyzed in Section \ref{append:dc_modified_phase}, equilibrium point $E_1$ is always a saddle point, and to have stability at equilibrium point $E_2$, the main condition is $0<l<1$. Therefore, we chose initial conditions such that all solutions converge to $E_2$, i.e., a stable equilibrium point.
%%% ChatGPT end
\begin{align}\label{parameters}
 b=0.3,\hspace{1mm} d(0)=0.01, \hspace{1mm} c=2,\hspace{1mm} r(0)=11,\hspace{1mm} l=0.7, s=0.25
\end{align}
Hence the system \ref{extend} is:
	\begin{subequations}\label{eq:extend_numeric}
	\begin{align}
		\frac{dy}{dt} &= 0.3y(t) + 0.01 + sz(t)\big(7.7-y(t)\big) , \\
		\frac{dz}{dt} &= -2z(t)\big(11-y(t)\big),
	\end{align}
	\end{subequations}
%%% ChatGPT start
The local stability of the system can be analyzed by calculating the eigenvalues at each equilibrium point. 
 When the real parts of the eigenvalues are negative, the equilibrium point is locally stable. 

We simulated the step-like response with initial input $r(0)=11$ and a single pulse with amplitude $5$ from time $0$ to $400$. %eason of no unit, scale time line.}}
% The first $50$ time unit, which is not shown in our plots, is for guiding the response to its equilibrium point with constant input $11$ for avoiding zero padding problem at time zero.
We verified the results with different parameters for $s$, $b$ and $c$.
%$s=0.25$ and $s=1.5$ while $b=0.3$, and two different $b=0.3$ and $b=3$ while $s=0.25$. 
The phase portrait for the original parameters in \ref{parameters} with different values of $s$, $b$ and $c$ is shown in Fig. \ref{fig:DC_Phase_portrait}.

For $s=0.25$, two red dots in Fig. \ref{fig:DC_Phase_portrait_original} represent the equilibrium points $E_1=(-0.033, 0)$ and $E_2=(11.000, 4.012)$, with eigenvalues
\begin{align}\label{parameters_eigenvalues}
\lambda(E_1)=\{0.300, -22.067\},\hspace{5mm}
 \lambda(E_2)=\{-0.351+ 2.549 i , -0.351- 2.549 i \}.
\end{align}
Since $E_2$ is a stable equilibrium point, all trajectories in its region of attraction approach it. 
\\
If we multiply $s$ by 6 times ($s=1.5$), we obtain the equilibrium points $E_1=(-0.033, 0)$ and $E_2=(11.000, 0.669)$ in Fig. \ref{fig:DC_Phase_portrait_s}, with eigenvalues 
\begin{align}\label{parameters_eigenvalues}
\lambda(E_1)=\{0.300, -22.067\},\hspace{5mm}
 \lambda(E_2)=\{-0.351+ 2.549 i , -0.351- 2.549 i \}.
\end{align}
Again, since $E_2$ is a stable equilibrium point, all trajectories in its region of attraction approach it. 
\\
In Fig. \ref{fig:DC_Phase_portrait_b}, we choose $b=0.6$, which is twice as large as the original one, and it alters the equilibrium points to $E_1=(-0.017, 0)$ and $E_2=(11.000, 8.012)$, with eigenvalues
\begin{align}\label{parameters_eigenvalues}
\lambda(E_1)=\{0.6, -22.033\},\hspace{5mm}
 \lambda(E_2)=\{-0.701+ 3.568 i , -0.701 - 3.568 i \}.
\end{align}
Since $E_2$ is a stable equilibrium point, all trajectories in its region of attraction approach it. \\
Finally, if we multiply $c$ by two ($c=4$), we get the equilibrium points $E_1=(-0.033, 0)$ and $E_2=(11.000, 4.012)$ in Fig. \ref{fig:DC_Phase_portrait_c}, with eigenvalues 
\begin{align}\label{parameters_eigenvalues}
\lambda(E_1)=\{0.300, -44.133\},\hspace{5mm}
 \lambda(E_2)=\{-0.351+ 3.622 i , -0.351- 3.622 i \}.
\end{align}
All trajectories in the region of attraction approach $E_2$ as it is a stable equilibrium point.
%%% ChatGPT end
\begin{figure}[tbh]
  \begin{minipage}{\linewidth}
\begin{subfigure}[b]{0.5\linewidth}
\begin{center}
			\includegraphics[trim={0 0 0 0cm},clip,width=\columnwidth]{Akram_2023_DC_phase_portrait_original.pdf}
		\subcaption{$s=0.25$, $b=0.3$, $c=2$
		\label{fig:DC_Phase_portrait_original}}
\end{center}
\end{subfigure}
     \begin{subfigure}[b]{0.5\linewidth}
\begin{center}
		\includegraphics[trim={0 0 0 0cm},clip,width=\columnwidth]{Akram_2023_DC_phase_portrait_change_s.pdf}
	\subcaption{ $s=1.5$
	\label{fig:DC_Phase_portrait_s}}
\end{center}
   \end{subfigure}
     \begin{subfigure}[b]{0.5\linewidth}
\begin{center}
		\includegraphics[trim={0 0 0 0cm},clip,width=\columnwidth]{Akram_2023_DC_phase_portrait_change_b.pdf}
	\subcaption{ $b=0.6$
	\label{fig:DC_Phase_portrait_b}}
\end{center}
   \end{subfigure}
     \begin{subfigure}[b]{0.5\linewidth}
\begin{center}
		\includegraphics[trim={0 0 0 0cm},clip,width=\columnwidth]{Akram_2023_DC_phase_portrait_change_c.pdf}
	\subcaption{ $c=4$
	\label{fig:DC_Phase_portrait_c}}
\end{center}
   \end{subfigure}
\caption{ %%% ChatGPT start
Phase portraits with different values of $s$, $b$ and $c$ show that the stable equilibrium $E_2$ has almost the same region of attraction in all cases but the trajectories differ. 
The region of attraction is determined by $E_1$.
(\subref{fig:DC_Phase_portrait_original}) The phase portrait for parameters \ref{parameters}. (\subref{fig:DC_Phase_portrait_s}) The phase portrait for parameters \ref{parameters}, except $s$ that is changed from $0.25$ to $1.5$. 
(\subref{fig:DC_Phase_portrait_b}) The phase portrait for parameters \ref{parameters}, except $b$ that is changed from $0.3$ to $0.6$.
(\subref{fig:DC_Phase_portrait_c}) The phase portrait for parameters \ref{parameters}, except $c$ that is changed from $2$ to $4$.
 \label{fig:DC_Phase_portrait}}
\end{minipage}
\end{figure}

After verifying the stability of the system, we investigated the $\mathbb{P}$-invariance property under different situations.
%According to the definition of DC, the system should be at the equilibrium point before making any changes to the parameter $s$, so in case $s=0.25$, equilibrium point $E_2$ is stable, and equilibrium $E_1$ is a saddle point. 
In all Figs. \ref{fig:step_like_response_s}, \ref{fig:step_like_response_b} and \ref{fig:step_like_response_c} $r(t)$ and $d(t)$ are the same and show the time-varying step-like response of the reference input $r(t)$ and disturbance $d(t)$
These inputs were also subject to additional noise from a standard normal distribution.
We tested different combinations of the reference $r(t)$ and disturbance $d(t)$ to exemplify the $\mathbb{P}$-invariance property under different scenarios. 
Both the input $r(t)$ and the disturbance $d(t)$ began with the starting values defined at \ref{parameters} and remained constant from time 0 until time 50.
The input $r(t)$ changes while disturbance $d(t)$ remains constant when time is between 50 and 150. 
Both the input $r(t)$ and disturbance $d(t)$ remain constant between time 150 and 200. 
In the time interval ($200, 300$), the disturbance $d(t)$ changes while the input $r(t)$ remains constant. 
When time is between ($300, 350$), both the input $r(t)$ and disturbance $d(t)$ change. 
Finally, both converge to a new amount ($r=13.75, d=5$) and remain constant in the time interval ($350, 400$). %\todo[inline]{Note: You should mention the equilibrium point and what the new amounts of $r$ and $d$ are?}

The purple trajectories in Fig. \ref{fig:DC_Phase_portrait} illustrates the starting position in the Figs. \ref{fig:step_like_response_s}, \ref{fig:step_like_response_b} and \ref{fig:step_like_response_c}. 
As a result, we are in a stable situation at the start, however it may take some time to achieve stability.
In all Figs. \ref{fig:step_like_response_s}, \ref{fig:step_like_response_b} and \ref{fig:step_like_response_c} $z(t)$ and $y(t)$ show the responses to the input and disturbance by $r(t)$ and $d(t)$. 
The green dashed line represents the residual of changes in $y(t)$, which remains zero when $s$ changes, but is non-zero when $b$ or $c$ changes. 
This is a consequence of the system having $\mathbb{P}$-invariance for parameter $s$, but not for $b$ and $c$. 




%{\color{red}{We showed that $y(t)$ is always tracking to $r(t)$ and the behavior remained unchanged even with the changing $s$ ($0.25$ to $5$), indicating that we have $\mathbb{P}$-invariance property. but while $b$ change residual Fig.  shows that there is differences among amount of $y(t)$}}

%By comparing the result with the original model in Fig.  \ref{fig:DC_noisy_step_response} C, it is obvious that we do not have $\mathbb{P}$-invariance property related to $b$.

%\begin{figure}[tbh]
%	\begin{subfigure}[b]{0.8\linewidth}
%		\begin{center}
%			\includegraphics[trim = 0 100 0 0, clip, width=\columnwidth]{Akram_2023_DC_noisy_step_input_change_s.pdf}
%	\subcaption{ step-like response
%	\label{fig:step_like_response}}
%		\end{center}
%	\end{subfigure}
%
%	\begin{subfigure}[b]{0.8\linewidth}
%		\begin{center}
%			\includegraphics[trim = 0 0 0 0, clip,width=\columnwidth]{Akram_2023_DC_noisy_step_input_change_b.pdf}
%	\subcaption{ s changes
%	\label{fig:response_s}}
%		\end{center}
%	\end{subfigure}
%	\begin{subfigure}[b]{0.8\linewidth}
%		\begin{center}
%			\includegraphics[trim = 0 0 0 0, clip,width=\columnwidth]{Akram_2023_DC_noisy_step_input_change_c.pdf}
%	\subcaption{ b changes
%	\label{fig:response_b}}
%		\end{center}
%	\end{subfigure}
%	\caption{
%Visualization of the impact of DC and lack there off on the output using time-varying step-like changes in the reference input $r(t)$ and disturbance $d(t)$ in different combinations. 
%% Time varying step-like response of the reference input $r(t)$ and disturbance $d(t)$ in differerent combination. 
%(\subref{fig:step_like_response}) Gaussian noise was added to the constant value of $r(t)$ and $d(t)$ during certain periods to ensure excitation. 
%%We first perturbed $r(t)$ while $d(t)$ remained constant, and at time $t=50$, a step response occurred in $r(t)$. 
%%Then, both $r(t)$ and $d(t)$ remained constant from $t=100$ to $t=150$. 
%%At $t=150$, $r(t)$ remained constant while $d(t)$ was perturbed, and a step response occurred at $t=200$. 
%%Finally, both $r(t)$ and $d(t)$ were perturbed at $t=250$ to observe changes while both were changing, and then an impulse response occurred at $t=300$. 
%(\subref{fig:response_s}) Comparison of the step response when $s$ is $0.25$ and $2.5$ for constant $b$. 
%The output $y(t)$ remained identical--the residual (green dashed line) equals zero. A hallmark of the system being  $\mathbb{P}$-invariant with regard to $s$. 
%(\subref{fig:response_b}) Comparison of the step response when $b$ is $0.3$ and $3$ for constant $s$. 
%The output $y(t)$ differs and the residual (green dashed line) is non-zero. A hallmark of the system not being $\mathbb{P}$-invariant with regard to $b$. \label{fig:DC_noisy_step_response}}
%\end{figure}

\begin{figure}[tbh]
			\includegraphics[trim = 0 0 0 0, clip, width=\columnwidth]{Akram_2023_DC_noisy_step_input_change_s.pdf}
	\caption{
Visualization of the impact of DC and lack there off on the output using time-varying step-like changes in the reference input $r(t)$ and disturbance $d(t)$ in different combinations. 
 Gaussian noise was added to the constant value of $r(t)$ and $d(t)$ during certain periods to ensure excitation. 
$y(t)$ and $z(t)$ show the comparison of the step response when $s$ is $0.25$ and $1.5$.
As we have started $y(t)$ and $z(t)$ with a distance from the equilibrium point, it takes a time to converge to the stable situation resulting at haing some residual between but then the output $y(t)$ remained identical--the residual (green dashed line) equals zero. 
A hallmark of the system being  $\mathbb{P}$-invariant with regard to $s$. 
 \label{fig:step_like_response_s}}
\end{figure}

\begin{figure}[tbh]
			\includegraphics[trim = 0 0 0 0, clip,width=\columnwidth]{Akram_2023_DC_noisy_step_input_change_b.pdf}
	\caption{
Visualization of the impact of DC and lack there off on the output using time-varying step-like changes in the reference input $r(t)$ and disturbance $d(t)$ in different combinations. 
 Gaussian noise was added to the constant value of $r(t)$ and $d(t)$ during certain periods to ensure excitation. 
$y(t)$ and $z(t)$ show the comparison of the step response when $b$ is $0.3$ and $0.6$.
The output $y(t)$ differs and the residual (green dashed line) is non-zero. 
A hallmark of the system not being $\mathbb{P}$-invariant with regard to $b$. \label{fig:step_like_response_b}}
\end{figure}

\begin{figure}[tbh]
			\includegraphics[trim = 0 0 0 0, clip,width=\columnwidth]{Akram_2023_DC_noisy_step_input_change_c.pdf}
	\caption{
Visualization of the impact of DC and lack there off on the output using time-varying step-like changes in the reference input $r(t)$ and disturbance $d(t)$ in different combinations. 
 Gaussian noise was added to the constant value of $r(t)$ and $d(t)$ during certain periods to ensure excitation. 
$y(t)$ and $z(t)$ show the comparison of the step response when $c$ is $2$ and $4$.
The output $y(t)$ differs and the residual (green dashed line) is non-zero. 
A hallmark of the system not being $\mathbb{P}$-invariant with regard to $c$.\label{fig:step_like_response_c}}
\end{figure}




%\clearpage
%%%%%%%%%%%%%%%%%%%%%%%%%%%%%%%%%%%%%%%%%%%%%%%%%%%%%%%%%%%%%%%%%%%%%%%%%%%%%%%%
\section{Conclusions}\label{sec:conclusion}
Our two state simplified and extended model based on Karin et al.'s work \cite{Karin2016} preserves the DC property when the parameter $s$ is changed. 
We have demonstrated this using the $\mathbb{P}$-invariance definition by Sontag \cite{Sontag2017}. 
With this approach, we have also shown no DC for the parameters $b$ and $c$, because the definition of $\mathbb{P}$-invariance is both sufficient and necessary.

Our example system is an exponential growth system with an adaptive proportional integral controller.
Exponential growth is a common feature of many physical systems, such as early stage of cell growth or disease spread. 
We have shown that our adaptive proportional integral feedback with DC in the control parameters $s$ can stabilize the system and ensure that the response tracks the reference input despite variation in the control parameters. 
The downside of this is that the closed loop systems behavior cannot be tuned by changing the gain of the controller as customary in e.g. PID-controllers. 
Moreover, we have demonstrated the stability of the system under a variety of conditions and plotted the phase portrait for a representative example. 
%Our adaptive controller can help to reduce fluctuations, thus achieving stability more quickly. 
%By changing the value of $s$, the adaptive controller can control the behavior of the system to achieve $\mathbb{P}$-invariance, thus controlling the output accordingly. 
In summary, we have demonstrated an adaptive controller with $\mathbb{P}$-invariance in it's parameter $s$.
%To summarize, we can use an adaptive controller to control the system so that it has $\mathbb{P}$-invariance.
%Our simplified and extended DC model offers a new perspective for analyzing the robustness of a system. 
%Based on our findings, a system with the DC property is not only robust but also stable when certain constraints are followed. 
%Furthermore, we have demonstrated the DC property into a two-equation system and presented an extended model that is more general than the original three-dimensional equation. 
This can be beneficial for designing robust controllers that can handle environmental fluctuations, in particular in Synthetic biology, as well as for understanding biological systems during modeling and analyzing.
%%% ChatGPT end


  %{\color{red}{If you remove the influence of the insulin in this model, how can you now explain it in biology as the insuli and glucose are related to each other. I think insuline allows body to use Glucose.}} {\color{red}{Rain: In the conclusion it said robustness. However, in the abstract, it highlight adaptive controller. It feels weird to me.}}

%\addtolength{\textheight}{-12cm}   % This command serves to balance the column lengths
                                  % on the last page of the document manually. It shortens
                                  % the textheight of the last page by a suitable amount.
                                  % This command does not take effect until the next page
                                  % so it should come on the page before the last. Make
                                  % sure that you do not shorten the textheight too much.
%%%%%%%%%%%%%%%%%%%%%%%%%%%%%%%%%%%%%%%%%%%%%%%%%%%%%%%%%%%%%%%%%%%%%%%%%%%%%%%%
\vspace{1cm}
\noindent \textbf{Data availability}
All data used is included in this article.
\\
\textbf{Acknowledgement}
The authors gratefully acknowledge valuable comments by Prof. Filippo Menolascina from the University of Edinburgh, UK. 
%We also would like to acknowledge the MOST AI Biomedical Research Center at NCKU (National Cheng Kung University).
\\
\textbf{Funding}
We would like to thank the Ministry of Science and Technology in Taiwan for their financial support (grants number MOST 105-2218-E-006-016-MY2, 105-2911-I-006-518, 107-2634-F-006-009, 110-2222-E-006-010, and 111-2221-E-006-186).



%%%%%%%%%%%%%%%%%%%%%%%%%%%%%%%%%%%%%%%%%%%%%%%%%%%%%%%%%%%%%%%%%%%%%%%%%%%%%%%%
%\todo[inline]{I changed the style of references because it had some problem}
%\bibliographystyle{IEEEtran}
%\bibliographystyle{unsrt}
%\bibliography{IEEEabrv,../LibraryAllReferences} %IEEEabrv defines abbreviations of journal names

%\bibliographystyle{apalike}
%\bibliography{../LibraryAllReferences}



%%%%%%%%%%%%%%%%%%%%%%%%%%%%%%%%%%%%%%%%%%%%%%%%%%%%%%%%%%%%%%%%%%%%%%%%%%%%%%%%



%\clearpage
\begingroup
\let\clearpage\relax 
\onecolumn

\titleformat{\section}
{\normalfont\Large\bfseries}{A\thesection}{1em}{}
\endgroup
{\small{
\clearpage
%\bibliographystyle{apalike}
%\bibliographystyle{plainnat}
%\bibliography{DBpaper_cite}
\bibliographystyle{unsrt}
%\bibliographystyle{ieeetr}
\bibliography{../LibraryAllReferences}}}
%\appendix
\section{Skew Equations}
We will justify and show the three equations used in Lemma \ref{skew rel} to narrow our search for these skew axial algebras. Although they do not provide much use to understanding how these algebras could be constructed, they do make the proof easier.

Suppose $v$ is an $\mu$-eigenvector of an axis, $x$, where $\mu\neq1$. Then the projection on that axis should be equal to 0; that is, $\lm_x(v)=0$. Coincidentally, nearly all of the eigenvectors in Lemma \ref{eigen a} and \ref{eigen b} satisfy that rule. However we have
\begin{equation*}
 0=\lm_b\left(-\frac{P}{\bt}a+Pb+c\right) = -\frac{P}{\bt}\lmf_1+P+\lmf_2.
\end{equation*}
Whence we get Equation (\ref{proof1}).

\begin{defn}
Let $x$ be a $\mon{\al,\bt}$-axis in $A$, $\lm\in \{1,0, \al, \bt\}$ and $v\in A$. We denote $[v]^x_\lm$ to be the component of $v$ in $ A_\lm(x)$. 
\end{defn}
\begin{lem}
Let $w:=\frac{1}{2}(b-c)$. We have $[a]^a_\bt=0$, $[b]^a_\bt=w$, $[c]^a_\bt=-w$, $[\sg]^a_\bt=0$. Further, $[ab]^a_\bt=\bt w$, $[ac]^a_\bt=-\bt w$, $[bc]^a_\bt=0$, $[a\sg]^a_\bt=0$, $[b\sg]^a_\bt=\dt^fw$, $[c\sg]^a_\bt=-\dt^fw$ and $[\sg^2]^a_\bt=0$.
\end{lem}
\proof
As $a\in A_1(a)$, it has no $\bt$-component in $A_\bt(a)$ and $[a]^a_\bt=0$. As $\sg\in A_{\{1,0,\al\}}(a)$, it has no $\bt$-component in $A_\bt(a)$ and $[\sg]^a_\bt=0$. We can express $b$ in terms of the eigenvectors of $\text{ad}_a$ in Lemma \ref{eigen a}. The reader can check
\[ b= \lm_1 a+ \frac{1}{\al}\left(\ep a+\frac{1}{2}(\al-\bt)(b+c)-\sg\right)+ \frac{1}{\al}\left(\gm a +\frac{1}{2}\bt(b+c)+\sg\right)+\frac{1}{2}(b-c).\]
Thus $[b]_\bt^a=w$. As $c=b^{\tu{a}}$, we get $[c]_\bt^a=-w$.

Let $x, y \in A_{\{0,1,\al\}}(a)$ and notice $x^2, xy\in A_{\{1,0,\al\}}(a)$ and so has no $\bt$-component in $A_\bt(a)$. Therefore $[\sg^2]^a_\bt=[a\sg]^a_\bt=0$. Also
\[ [bc]_\bt^a=P\left([a]_\bt^a+\frac{1}{\bt}[\sg]_\bt^a\right)=0.\]
Note that
\[ [ab]_\bt^a=[\sg]_\bt^a+\bt[a]_\bt^a+\bt[b]_\bt^a=\bt w\]
and 
\[ [b\sg]_\bt^a=(\al-\bt)[\sg]_\bt^a+\bt(\al-\bt)[a]_\bt^a+dt^f[b]_\bt^a=\dt^f w.\]
Applying $\tu{a}$, we get $[ac]_\bt^a$ and $[c\sg]_\bt^a$. \qed



Let $u:= (b -\al)a - \bt b=\sg -(\al-\bt)a$. As $A_\bt(b)=\{0\}$, we have that $u\in A_{\{1,0\}}(b)$. By Lemma \ref{Seress}, the following holds
\[b(au)=(ba)u.\]
Notice
\[ au = a(\sg -(\al-\bt)a)=(\dt -(\al-\bt))a+\frac{1}{2}\bt(\al-\bt)(b+c)+(\al-\bt)\sg\]
and so
\begin{eqnarray*}
[b(au)]_\bt^a &=& (\dt -(\al-\bt))[ab]_\bt^a+\frac{1}{2}\bt(\al-\bt)([b]_\bt^a+[bc]_\bt^a)+(\al-\bt)[b\sg]_\bt^a\\
& =& \left(\bt(\dt -(\al-\bt))+\frac{1}{2}\bt(\al-\bt)+(\al-\bt)\dt^f\right)w
\end{eqnarray*}
We also have 
\begin{eqnarray*}
[(ba)u]_\bt^a&=&[(\sg+\bt a +\bt b)(\sg -(\al-\bt)a)]_\bt^a\\
&=& [\sg^2]_\bt^a -(\al-2\bt)[a\sg]_\bt^a +\bt [b\sg]_\bt^a -\bt(\al-\bt)[a]_\bt^a - \bt(\al-\bt)[ab]_\bt^a\\
&=& (\bt\dt^f -\bt^2(\al-\bt)) w
\end{eqnarray*}
By Lemma \ref{Seress}, we have $0=(ba)u-b(au)$ moreover $0=[(ba)u]_\bt-[b(au)]_\bt$. Looking at the coefficient of $w$, we have
\begin{eqnarray*} 
0&=& (\bt\dt^f-\bt^2(\al-\bt))\\
& -& \left(\bt \dt -\bt(\al-\bt)+\frac{1}{2}\bt(\al-\bt)+(\al-\bt)\dt^f\right)\\
&=&-\bt^2(\al-\bt) -\bt\dt+\frac{1}{2}\bt(\al-\bt)-(\al-2\bt)\dt^f.
\end{eqnarray*}
Rearranging we get Equation (\ref{proof2}).

Let $v:=Pa+\frac{P}{\bt}\sg -\al c=c(b-\al)$. Notice that $v \in A_{\{1,0\}}(b)$. Again by Lemma \ref{Seress}, the following holds
\[b(av)=(ba)v.\]
We have
\begin{eqnarray*}
av &=& Pa +\frac{P}{\bt}\left(\dt a + \frac{1}{2}\bt(\al-\bt)(b+c) +(\al-\bt)\sg\right)\\
& -&\al(\bt a +\bt c +\sg)\\
&=&\left(P +\frac{P}{\bt}\dt -\al\bt\right)a+\left(\frac{1}{2}(\al-\bt)P\right)b\\
&+&\left(\frac{1}{2}(\al-\bt)P-\al\bt\right)c+\left(\frac{P}{\bt}(\al-\bt)-\al\right)\sg.
\end{eqnarray*}
Therefore
\begin{eqnarray*}
[b(av)]_\bt^a &=&\left(P +\frac{P}{\bt}\dt -\al\bt\right)[ab]_\bt^a+\left(\frac{1}{2}(\al-\bt)P\right)[b]_\bt^a\\
&+&\left(\frac{1}{2}(\al-\bt)P-\al\bt\right)[bc]_\bt^a+\left(\frac{P}{\bt}(\al-\bt)-\al\right)[b\sg]_\bt^a.\\
&=&\left(\bt \left(P +\frac{P}{\bt}\dt -\al\bt\right)+\dt^f\left(\frac{P}{\bt}(\al-\bt)-\al\right)\right)w
\end{eqnarray*}
We also have
\begin{eqnarray*}
[(ba)v]_\bt^a&=&\left[\left(\bt a +\bt b +\sg\right)\left(Pa+\frac{P}{\bt}\sg -\al c\right)\right]_\bt^a\\
&=&2P[a\sg]_\bt^a +\frac{P}{\bt}[\sg^2]_\bt^a -\al [c \sg]_\bt^a + \bt P [a]_\bt^a -\al\bt [ac]_\bt^a\\
&+&\bt P [ab]_\bt^a +P[b\sg]_\bt^a -\al\bt [bc]_\bt^a\\
&=&\left(\al \dt^f +\al\bt^2 +\bt^2 P  +P\dt^f\right)w
\end{eqnarray*}
By Lemma \ref{Seress}, $0=[b(av)]^a_\bt-[(ba)v]^a_\bt$ and looking at the coefficient of $w$, we get 
\begin{eqnarray*}
0&=&[b(av)]_\bt-[(ba)v]_\bt\\
&=&\left(\bt P +\dt P -\al\bt^2+\frac{1}{2}(\al-\bt)P+\frac{P}{\bt}(\al-\bt)\dt^f -\al\dt^f\right)\\
&-&\left(\bt^2P +P\dt^f+\al\dt^f +\al\bt^2 \right)\\ 
&=&\left(\frac{P}{\bt}\left[\bt^2 +\bt\dt+\frac{1}{2}\bt(\al-\bt)+(\al-2\bt)\dt^f-\bt^3\right]-2\al(\dt^f+\bt^2)\right).
\end{eqnarray*}
From Equation (\ref{proof2}), we get that
\begin{eqnarray*}
0&=&\frac{P}{\bt}\left[\bt^2 -\bt^2(\al-\bt) +\frac{1}{2}\bt(\al-\bt)-(\al-2\bt)\dt^f\right.\\
&+&\left.\frac{1}{2}\bt(\al-\bt)+(\al-2\bt)\dt^f-\bt^3\right]-2\al(\dt^f+\bt^2)\\
&=&\frac{P}{\bt}\left[\bt^2 -\bt^2(\al-\bt) +\bt(\al-\bt)-\bt^3\right]-2\al(\dt^f+\bt^2)\\
&=&\frac{P}{\bt}\al\bt\left[1-\bt\right]-2\al(\dt^f+\bt^2).
\end{eqnarray*}
Hence we get Equation (\ref{proof3}).

\section*{Acknowledgements}
I would like to thank Professor Sergey Shpectorov for his guidance throughout my PhD studies so far and pushing me to complete this paper. I would also like to thank my family for their continuing support. 

%\todo[inline]{The following appendix does not add anything necessary to the story in its current form}
%\section*{APPENDIX}
%
%	In our study, we simplified model stated in Karin et al. \cite{Karin2016};
%	\begin{subequations}
%\label{eq:dc_original_modelApp}
%	\begin{align}
%\label{eq:dc_original_modelApp_1}	\frac{dy}{dt} &= u_{0} + u(t) - sxy, \\
%\label{eq:dc_original_modelApp_2}	\frac{dx}{dt} &= pzy-x, \\
%\label{eq:dc_original_modelApp_3}	\frac{dz}{dt} &= z(y-y_{0}),
%	\end{align}
%	\end{subequations}
%	where $s$ and $p$ are the feedback gains of $x$ and $z$, respectively.
%	The output variable, $y$, is a regulated variable that is able to form a feedback loop with $x$ and $z$.
%	The regulated variable $y$ controls the functional mass $z$ of tissue which secretes hormone $x$ in this circuit.
%
%
%%	We derived the extended model based on our simplified model to make an adaptive proportional-integral feedback model
%By extending our simplified model with a DC property in parameter $s$, we created an adaptive proportional-integral feedback model with the following Equations.
%	\begin{subequations}\label{extendApp}
%	\begin{align}
%		\label{extendApp1}
%		\frac{dy}{dt} &= by(t) + d(t) + sz(t)\big(lr(t)-y(t)\big) , \\
%		\label{extendApp2}
%		\frac{dz}{dt} &= -cz(t)\big(r(t)-y(t)\big),
%	\end{align}
%	\end{subequations}
% We chose following initial conditions for the 2 dimentional system:
%\begin{align}
% b=0.3,\hspace{1mm} d(0)=0.01, \hspace{1mm} c=2,\hspace{1mm} r(0)=11,\hspace{1mm} l=0.7, s=0.25
%\end{align}
%Hence the system \ref{extendApp} is:
%	\begin{subequations}\label{eq:extend_numeric}
%	\begin{align}
%		\frac{dy}{dt} &= 0.3y(t) + 0.01 + sz(t)\big(7.7-y(t)\big) , \\
%		\frac{dz}{dt} &= -2z(t)\big(11-y(t)\big),
%	\end{align}
%	\end{subequations}
%
%We check 2 situations: 
%\\
%When $b$ remains unchanged as parameter \ref{parameters} but $s$ changes from $0.25$ to $2.5$. 
%\\
%When $s$ remains unchanged as parameter \ref{parameters} but $b$ changes from $b=0.3$ to $b=3$. 
%
%In order to draw both 2 dimentional and 3 dimentional in the same figure we chose following parameters for system \ref{eq:dc_original_model}:
%\begin{align}
% u_{0}=0.01,\hspace{1mm} s=0.25, \hspace{1mm} p=1.925,\hspace{1mm} y_{0}=11.
%\end{align}
%Therefore, the system is 
%	\begin{subequations}
%\label{eq:dc_original_modelEx}
%	\begin{align}
%\label{eq:dc_original_modelEx_1}	\frac{dy}{dt} &= 0.01 + u(t) - 0.25 xy, \\
%\label{eq:dc_original_modelEx_2}	\frac{dx}{dt} &= 1.925 zy-x, \\
%\label{eq:dc_original_modelEx_3}	\frac{dz}{dt} &= z(y-11),
%	\end{align}
%	\end{subequations}
%		%% Appendix
%%		\begin{appendices}
%%\appendix
%%			%\chapter{Network Inference}\label{chap:network_inference_appendix}
%%			\input{NI_tutorial_Article.tex}
%%		\end{appendices}

\end{document}
