\section{Introduction}
\label{sect:intro}
The MeerKAT radio telescope has an existing correlator-beamformer based on Field
Programmable Gate Arrays (FPGAs), which has been previously described
\cite{meerkat-cbf}. The MeerKAT Extension project is currently underway to
add more dishes with longer baselines\cite{mke}. Since the MeerKAT correlator depends on
a number of hardware components that have reached end-of-life (particularly
the Hybrid Memory Cube memory) and there were concerns that the design would
not scale up, a new correlator is being designed rather than
expanding the existing correlator.

The FPGA development process for MeerKAT was plagued by long compile times
(usually overnight), difficult-to-use tools, and rigid designs: each channel count
used a different design, and changes had to be manually copied between
designs. While FPGA development tools have since improved, it nevertheless
remains challenging to achieve high performance \cite{fgpa-vs-gpu}.

Graphics Processing Units (GPUs) offer an alternative with a mature ecosystem
and a more convenient development process. They have been used for some years
for the correlation (X) step in F-X correlators\cite{xgpu}, but the only GPU-based
channelizer (F step) of which we are aware is the Cobalt correlator (used by
LOFAR) \cite{cobalt,accel-compare}. There are also GPU-based spectrometers
such as at the Green Bank Telescope \cite{gbt-spectrometer} and
Atacama Compact Array \cite{kasi-spectrometer}, but these do not include the
delay correction needed for an F-X correlator.

We first developed a proof-of-concept
channelizer which implemented the data-path functionality of the MeerKAT
wide-band correlator. Partly based on the good results from this
proof-of-concept, we elected to pursue a fully GPU-based correlator for the
MeerKAT Extension. This paper describes the implementation and tuning of our
GPU-based channelizer.

Section~\ref{sect:background} describes the functionality included in our
channelizer, and summarizes the programming model for GPUs.
Section~\ref{sect:implementation} details the initial software
implementation. We then describe a significant optimization in
Sec.~\ref{sect:unzip}. We finish with results (Sec.~\ref{sect:results}) and
conclusions (Sec.~\ref{sect:conclusions}).
% Introduce FGPAs and GPUs, cite Romein's paper comparing difficulty
% Introduce MeerKAT, SKARAB, MeerKAT+
