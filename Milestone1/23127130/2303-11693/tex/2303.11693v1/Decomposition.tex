\section{Decomposition}\label{Decomposition}
In this section we give a proof of Theorem \ref{MainDecompTheorem}, but without the geometric inequality \eqref{GeometricInequality}. The proof consists of two parts. The first part is showing that for any non-vanishing holomorphic function $\psi$ with bounded frequency in a disc at least four times the radius of the original interval $I$, there exists a decomposition of $I$ into almost disjoint intervals $I_k$ on which we can control the oscillation of $\psi$. In particular, using Lemma \ref{lemmaByFoster} we can show that any holomorphic function with bounded frequency in an even larger disc must be equivalent to a polynomial of bounded degree on sufficiently small subintervals of $I$.

The second part of the proof is showing that for any function $\varphi\in\mathcal{A}(I,N,R)$, there exists a decomposition of $I$ such that $\varphi$ is equivalent to a monomial on each subinterval. Here we refer to a decomposition given in \cite{DW} for polynomials. Taking the intersection of the intervals from these two decomposition schemes gives the required decomposition.
Let us therefore begin with the decomposition for non-vanshing holomorphic functions with bounded frequency.
\begin{prop}\label{HolDecomProp}
Let $I=[a-r,a+r]$ be a compact interval, and let $\psi\in \mathcal{A}(I,N,4r)$ be non-vanishing on $D(a,4r)\subset \mathbb{C}$. Then for each $\varepsilon>0$ there exists a finite decomposition $I=\bigcup_{j=1}^{M_{N,\varepsilon}} I_j$ such that for each $j$
\begin{equation*}
    \left|\frac{\psi(x)}{\psi(y)}-1\right|<\varepsilon,
\end{equation*}
for any $x,y\in I_j$. The number of intervals $M_{N,\varepsilon}$ depends only on $N$ and $\varepsilon$.
\end{prop}
\begin{proof}
Without loss of generality, we may assume that $\sup_{z\in D(a,4r)}|\psi(z)|=1$. Since $\psi$ is of bounded frequency, it follows by \eqref{FrequencyDoublingLemma} in Proposition \ref{FreqPropProp} that
\[
\sup_{z\in D(x,2s)}|\psi(z)|\leq 2^{CN+1}\sup_{z\in D(x,s)}|\psi(z)|,
\]
for any $x\in I$, and $s<r$. In particular, for any $k\in\mathbb{N}$,
\[
1=\sup_{z\in D(a,4r)}|\psi(z)|\leq 2^{(k+2)(CN+1)}\sup_{z\in D(a,2^{-k}r)}|\psi(z)|.
\]
Fix some $k\in\mathbb{N}$, and divide the interval $I$ into $2^{k+1}$ subintervals $I_j$ of length $2^{-k}r$ with centres $a_j$. Then for each $j$ it follows that
\begin{equation}\label{BallEstimate}
    \begin{aligned}
   \sup_{z\in D(a_j,2^{-k}r)}|\psi(z)|\geq 2^{-(k+1)(CN+1)}\sup_{z\in D(a_j,2r)}|\psi(z)|&\geq 2^{-(k+1)(CN+1)}\sup_{z\in D(a,r)}|\psi(z)|
   \\&\geq 2^{-(k+3)(CN+1)}. 
\end{aligned}
\end{equation}


Define the function $
h(z)=-\log_2|\psi(z)|$ on $D(a,4r)$. This is a positive harmonic function as $\psi$ is a normalized non-vanishing holomorphic function on $D(a,4r)$. Moreover, by \eqref{BallEstimate},
\[
\inf_{z\in D(a_j,2^{-k}r)}h(z)\leq (k+3)(CN+1).
\]
By Harnack's inequality, we have
\begin{equation*} \frac{1-2^{-k}}{1+2^{-k}}h(a_j)\leq h(x)\leq \frac{1+2^{-k}}{1-2^{-k}}h(a_j),
\end{equation*}
for any $x\in D(a_j,2^{-k}r)\subset D(a_j,r)$. As such, we have
\[
\sup_{z\in D(a_j,2^{-k}r)}h(z)\leq \frac{(1+2^{-k})^2}{(1-2^{-k})^2}\inf_{z\in D(a_j,2^{-k}r)}h(z).
\]
Since $I_j\subset D(a_j,2^{-k}r)$, it follows that for any $x,y\in I_j$,
\begin{equation}\label{HarnackArgument}
\begin{aligned}
|h(x)-h(y)|
\leq& \sup_{z\in D(a_j,2^{-k}r)}h(z)-\inf_{z\in D(a_j,2^{-k}r)}h(z) 
\\
\leq& 2^{-k+4}\inf_{z\in D(a_j,2^{-k}r)}h(z)
\\
\leq& 2^{-k+4}(k+3)(CN+1).
\end{aligned}
\end{equation}
Thus, by choosing $k$ sufficiently large we can make the difference arbitrarily small.

Now let $\varepsilon>0$.
By the continuity of the exponential function there exists $\delta>0$, such that
\[
\left|\frac{\psi(x)}{\psi(y)}-1\right|=\left|2^{h(y)-h(x)}-1\right|<\varepsilon,
\]
whenever $|h(x)-h(y)|<\delta$. We see from \eqref{HarnackArgument} that this is achieved by Harnack's inequality by choosing $k=k(\varepsilon,N)\in\mathbb{N}$ such that
\[
2^{-k}(k+3)<\frac{\delta}{16(CN+1)},
\]
and decomposing $I$ into $2^{k+1}$ intervals of length $2^{-k}r$.
\end{proof}
\begin{remark}
From the proof it follows that the number of intervals $I_j$ is independent of the radius $r$. The only dependence on the length of the interval $I$ is in the size of the disc for which the function needs to have bounded frequency.
\end{remark}
As mentioned earlier, by Lemma \ref{lemmaByFoster} we note that any $\varphi\in \mathcal{A}(I,N,8r)$ can be written as $\varphi=\psi p$, where $p$ is a polynomial of bounded degree and $\psi\in\mathcal{A}(I,CN,4r)$. Proposition \ref{HolDecomProp} then gives that $\varphi$ is equivalent to the polynomial $p$ on each $I_j$ in the decomposition. Since a decomposition scheme for polynomials already exists, Theorem \ref{MainDecompTheorem} follows by combining these two decompositions. In order to continue, the following lemma is needed. 
\begin{lemma}[D1 in \cite{DW}]\label{D1Lemma}
Let $Q:J\to\R$ be a real polynomial of degree at most $N$. Then the interval $J$ can be decomposed into a finite number of open disjoint intervals,
\[
J=\bigcup_{j=1}^{M_{N}}I_j,
\]
so that on each $I_j$ there exist constants $c_j,C_j>0$ for which 
\[
c_j\left(A_j|t-a_j|^{k_j}\right)\leq|Q(t)|\leq C_j\left(A_j|t-a_j|^{k_j}\right),
\]
for all $t\in I_j$. Here $0\leq k_j\leq N$, $A_j\neq 0$, and the centres $a_j$ are the real parts of the zeros of $Q$ which are not contained in the interior of $I_j$. The constants $c_j,C_j$ and $M_{N}$ depend only on $N$.
\end{lemma}
With this decomposition lemma for polynomials, we are ready to prove the main theorem of this section.
\begin{theorem}\label{FirstDecompThm}
Let $N\in\mathbb{N}$ be a fixed integer, $I=[a-r,a+r]$ a compact interval, and consider $\varphi\in \mathcal{A}(I,N,8r)$. There exists a finite decomposition 
\[
I=\bigcup_{j=1}^{K_N}I_j,
\]
so that for each $j$ there exist constants $c_j,C_j>0$, $a_j\in\R\backslash I_j$, and $0\leq k_j\leq 2N$, such that 
\[
c_j|t-a_j|^{k_j}\leq |\varphi(t)|\leq C_j|t-a_j|^{k_j},
\]
for all $t\in I_j$. Moreover, the number of intervals $K_N$ depends only on $N$.
\end{theorem}
\begin{proof}
By Lemma \ref{lemmaByFoster}, there exists a polynomial of degree at most $2N$, and a non-vanishing function of bounded frequency $\psi$ of absolute value at most $1$, such that
\[
\varphi(x)=p(x)\psi(x),
\]
on the disc $D(a,4r)$. Let $\varepsilon=1/2$, and apply Proposition \ref{HolDecomProp} to the function $\psi$. Then there exists a finite number of intervals such that
\[
I=\bigcup_{j=1}^{M_{N}}I_j,
\]
on which
\begin{equation*}
    \left|\frac{\psi(x)}{\psi(y)}-1\right|\leq \frac{1}{2},
\end{equation*}
for any $x,y\in I_j$.
Fix a point $x_j\in I_j$, and consider the function $\psi_j$,
\begin{equation*}
    \psi_j(t)=\frac{\psi(t)}{\psi(x_j)},
\end{equation*}
which is well-defined as $\psi$ is non-vanishing. Moreover, we have
\begin{equation*}
    \left|\psi_j(t)-1\right|=\left|\frac{\psi(t)}{\psi(x_j)}-1\right|\leq\frac{1}{2}.
\end{equation*}
This necessarily implies that $1/2\leq|\psi_j(t)|\leq 3/2$ on $I_j$.

For each $I_j$, consider the polynomial $p_j$ given by
\[
p_j(t)=\psi(x_j)p(t),
\]
so that $\varphi=\psi_jp_j$ on $I_j$. Apply Lemma \ref{D1Lemma} with respect to the polynomial $p_j$ to decompose $I_j$ further into $I_{j,i}$. Then on each $I_{j,i}$ there exist constants $c_i,C_i$ such that
\[
c_i\leq \frac{|p_j(t)|}{A_i|t-a_i|^{k_i}}\leq C_i,
\]
by Lemma \ref{D1Lemma}.
However, this implies
\begin{align*}
    \frac{c_i}{2}\leq\frac{|\varphi(t)|}{A_i|t-a_i|^{k_i}}\leq \frac{3}{2}C_i,
\end{align*}
as $\varphi=\psi_jp_j$. The proof follows by relabeling the intervals $I_{j,i}$.
\end{proof}

For each function $\varphi\in \mathcal{A}_0^{d-1}(I,N,2^{d+3}r)$ it follows that $\varphi^{(d)}\in\mathcal{A}(I,\mathcal{N}_{N,d},8r)$ by applying  \eqref{BoundedFreqDerivative} $d$ times. Here $\mathcal{N}_{N,d}$ is a positive constant depending only on $N$ and $d$. Moreover, by property \eqref{NumberOfZeros} we control the number of zeros of $\varphi^{(d)}$, and thus the number of connected components of $I$ on which $\varphi^{(d)}$ is single-signed. The first part of Theorem \ref{MainDecompTheorem} then follows from Theorem \ref{FirstDecompThm} applied to $\varphi^{(d)}$ and taking the intersection of all intervals in the decomposition with all connected components of $I$ on which $\varphi^{(d)}$ is single-signed.


