\section{Introduction and main result}


The Fourier restriction operator $\mathcal{R}_\g$ associated to a curve $\g:I\to\R^d$ is defined as
\[
\mathcal{R}_\g(f)(t):=\widehat{f}(\g(t)),
\]
where $f\in\mathscr{S}(\R^d)$ and $t\in I$. In this paper we investigate uniform boundedness of the restriction operator $\mathcal{R}_\g:L^p(\mathbb{R}^d)\to L^q(I;\lambda_\g dt)$, where the latter is equipped with the affine arc length measure $\lambda_\g dt=|L_\g|^{\frac{2}{d^2+d}}dt$ and $L_\g=\det{(\g',\ldots,\g^{(d)})}$ denotes the torsion of the curve. The curves we consider are called \textit{simple} curves, and are of the form
\begin{equation}\label{defSimpCurve}
\g(t)=\left(t,t^2,\ldots, t^{d-1},\varphi(t)\right),
\end{equation}
where $\varphi$ belongs to certain class of functions, the so-called functions of bounded frequency.

The study of Fourier restriction of curves in dimension two can be traced back to Fefferman and Stein \cite{Fefferman}, Zygmund \cite{Zygmund}, and H\"{o}rmander \cite{Hormander}. The first uniform restriction estimate for curves is due to Sj\"{o}lin \cite{Sjolin}. He proved the existence of a restriction constant uniform over all $C^2$ convex plane curves for the range $1\leq p< 4/3$ and $3q\leq p'$, and showed that this range is optimal by considering the curve $\g(t)=(t,exp(-t^{-1})\sin(t^{-k}))$ for $k>q$. This curve acts as a counter-example to the boundedness of $\mathcal{R}_\g$ due to the rapid oscillation of the curve near $t=0$. The result of Sj\"{o}lin has been extended by Fraccaroli, who proved uniformity over all convex plane curves \cite{Fraccaroli}.

The first result in dimension three belongs to Prestini, who proved restriction in a reduced range \cite{Prestini}. The work of Prestini was extended to higher dimensions by Christ in \cite{Christ} for the range
\begin{equation}\label{ChristRange}
q\leq \frac{2}{d^2+d}p',\quad 1\leq p<\frac{d^2+2d}{d^2+2d-2},
\end{equation}
now known as the Christ range. The maximal conjectured range of $p$ and $q$ is
\begin{equation}\label{FullRange}
  1\leq p<\frac{d^2+d+2}{d^2+d},\quad 1\leq q\leq \frac{2}{d^2+d}p^{\prime}.
\end{equation}
where $p^{\prime}=p/(p-1)$ denotes the H\"{o}lder conjugate of $p$. The optimality of the range of $p$ follows from an argument by Arkhipov, Chubarikov and Karatsuba \cite{Arkhipov}, while a scaling argument can be used to derive the endpoint case for the range of $q$. 

The first result beyond the Christ range is due to Drury, who proved the restriction estimate in the full range \eqref{FullRange} for the moment curve in dimension three \cite{Drury}. 
He also pointed out the benefits of using the affine arc length measure rather than the Euclidean in the case of degenerate curves \cite{Drury-Affine}.

A well studied class of curves in uniform restriction is that of polynomial curves, meaning curves of the form $\g(t)=(p_1(t),\ldots,p_d(t))$ where $p_i$ are polynomials. 
A significant result is due to Dendrinos and Wright \cite{DW}. They proved restriction for the Christ range \eqref{ChristRange},
where the restriction constant only depends on $p,q,d$ and the maximal degree $N$. Stovall later extended the result to the full conjectured range \eqref{FullRange}. We emphasise that her result holds for any polynomial curve defined on $I=\mathbb{R}$ \cite{Stovall}.


In \cite{Chen}, Chen, Fan and Wang investigated simple curves of the form \eqref{defSimpCurve} for smooth functions $\varphi\in C^\infty(I)$ on a compact interval $I$. They proved a restriction estimate in the full range \eqref{FullRange}, under the additional condition that the torsion is bounded between two constants. The result is uniform in the sense that the constant only depends on $p$ and $d$. By considering a dyadic decomposition of the torsion, they extended the restriction estimate for the range
\[
1\leq p<\frac{d^2+d+2}{d^2+d},\quad 1\leq q< \frac{2}{d^2+d}p^{\prime}.
\]
However, the constant in the restriction now depends on a certain H\"{older} norm of the curve, and therefore is no longer uniform. Moreover, by building on the counter-example of Sj\"{o}lin, they were able to show that the estimate fails for $(d^2+d)q=2p'$ by considering the curve
\[
\g(t)=(t,\ldots,t^{d-1},e^{-t^{-\alpha}}\sin{t^{-\beta}}),
\]
for $\alpha>0$ and $2\beta>(d+1)\alpha$. This implies that a restriction estimate cannot hold for smooth simple curves at the critical line
$(d^2+d)q=2p'$. The result of Chen, Fan and Wang was extended to general smooth curves by Jesurum \cite{Jesurum}.


A class of curves where a restriction estimate could hold at the critical line ($(d^2+d)q=2p'$) is that of real analytic curves. By taking successive polynomial approximations of Sj\"{o}lin's example it is clear that even global analyticity does not suffice to obtain uniform bounds. Bounding some form of global oscillation of a family of curves becomes necessary.
In this article we consider a special class of real analytic simple curves, and prove a uniform restriction estimate for the full range \eqref{FullRange}.
Our class consists of real analytic curves of the form \eqref{defSimpCurve} where $\varphi$ has so-called bounded frequency. The frequency function goes back to the work of Agmon \cite{Agmon} and Almgren \cite{Almgren}, and has since become an important tool in the theory of elliptic PDEs. For a non-vanishing holomorphic function $u=\sum_{n=0}^\infty c_nz^n$ on $\Omega\subseteq \mathbb{C}$, we define the frequency function on $D(x,r)\subset \Omega$ as
\[
N_u(z_0,R)=2\frac{\sum_{n=1}^\infty n|c_n|^2R^{2n}}{\sum_{k=0}^\infty |c_k|^2R^{2k}}.
\]
If a function $u$ has frequency bounded by $N$ in a disc $D(x,r)$, then it behaves similar to a polynomial of degree $CN$ in every smaller disc centered at $x$.

Given a compact interval $I=[a-r,a+r]$, an integer $N\in\mathbb{N}$, and a positive constant $R>r$, we consider the classes of functions
\begin{align*}
    &\mathcal{A}(I,N,R):=\{\varphi: I\to \mathbb{R}: \exists \Phi\in\mathrm{Hol}(D(a,R))\text{ such that }\Phi|_I=\varphi,\text{ and } N_\Phi(a,R)\leq N\},\\
    &\mathcal{A}^{d-1}_0(I,N,R):=\{\varphi\in \mathcal{A}(I,N,R): \varphi(a)=\varphi'(a)=\ldots=\varphi^{(d-1)}(a)=0\}.
\end{align*}
Any simple curve \eqref{defSimpCurve} with $\varphi\in \mathcal{A}_0^{d-1}(I,N,R)$ is a smooth simple curve, so all restriction estimates in \cite{Chen} are valid for these curves. 
Our main result is the following.
\begin{theorem}\label{MainRestrictionTheorem}
Let $a\in\mathbb{R}$, and $r>0$ be fixed, and let $I$ denote the  bounded interval $I=[a-r,a+r]$. Then for any $N\in\mathbb{N}$, and any
\[
1\leq p< \frac{d^2+d+2}{d^2+d}
\]
there exists $C=C(N,d,p)>0$ such that for any simple curve
\[
\g(t)=\left(t,\ldots,t^{d-1},\varphi(t)\right),
\]
with $\varphi\in \mathcal{A}_0^{d-1}(I,N,2^{2d+3}r)$ and any $f\in L^p(\R^d)$ the restriction estimate
\begin{equation}
    \|\mathcal{R}_\g(f)\|_{L^{q}(I;\lambda_\g dt)}\leq C\|f\|_{L^p(\R^d)},
\end{equation}
holds for $q=2p'/(d^2+d)$.
\end{theorem}



We will use a decomposition scheme to prove Theorem \ref{MainRestrictionTheorem}. This was first done in \cite{DW} for polynomial curves. Here they showed that restriction can be reduced to having two properties of the curve;
\begin{enumerate}[i)]
    \item $\G(t_1,\ldots,t_d)=\sum_{i=1}^d\g(t_i)$ is $d!$-to-1.
    \item $|J_{\G}(t_1,\ldots,t_d)|\geq C\prod_{j=1}^d|L_\g(t_j)|^\frac1d\prod_{k>l}|t_k-t_l|$.
\end{enumerate}
Inequality $\mathrm{ii)}$ is known as \textit{the geometric inequality}. Uniform restriction estimates are then established by constructing a finite decomposition for polynomials curves on which these two properties hold. The same decomposition approach has successfully been used by de Dios Pont in \cite{DeDios1} to extend the uniform estimate of Stovall to complex polynomial curves. A further extension was later given by the same author in \cite{DeDios2}, where the decomposition was extended to include any algebraic extension of a $p$-adic field.

For our case, the torsion simplifies to $L_\g(t)=C_d \varphi^{(d)}(t)$, where $C_d$ is a positive constant only dependent on $d$. To prove Theorem \ref{MainRestrictionTheorem}, we will need the following decomposition theorem.
\begin{theorem}\label{MainDecompTheorem}
Let $a\in\mathbb{R}$, and $r>0$ be fixed, and let $I$ denote the  bounded interval $I=[a-r,a+r]$. For a fixed integer $N\in\mathbb{N}$, consider the function $\varphi\in\mathcal{A}_0^{d-1}(I,N,2^{d+3}r)$. There exists a finite decomposition,
\[
I=\bigcup_{j=1}^{M_{N}}I_j,
\]
into almost disjoint intervals such that $\varphi$ is single-signed on each $I_j$, there exist positive constants $c_j, C_j, K_j>0$, centres $a_j\in\mathbb{R}\backslash I_j$, and integers $k_j\leq 2N$ such that for each $t\in I_j$, 
\begin{equation}\label{PolyDecompEstimate}
    c_j|a_j-t|^{k_j}\leq |\varphi^{(d)}(t)|\leq C_j|a_j-t|^{k_j},
\end{equation}
and for each $(t_1,\ldots,t_d)\in I_j^d$,
\begin{equation}\label{GeometricInequality}
|J_{\Gamma}(t_1,\ldots,t_d)|\geq K_j\prod_{l=1}^d|\varphi^{(d)}(t_l)|^\frac{1}{d}\prod_{k>l}|t_k-t_l|.
\end{equation}
\end{theorem}
A consequence of \eqref{PolyDecompEstimate} and \eqref{GeometricInequality} is that $\G(t)=\sum\g(t_i)$ is injective on $I_j^d$ whenever $t_{\sigma(1)}<\ldots <t_{\sigma(d)}$ for any permutation $\sigma$ on $\{1,\ldots,d\}$.
There is nothing in the reduction argument of Dendrinos and Wright which requires the curve to be a polynomial curve. In fact, given Theorem \ref{MainDecompTheorem} the reduction argument provided in Section 3 of \cite{DW} can be used to achieve a uniform restriction estimate over the class $\mathcal{A}_0^{(d-1)}(I,N,R)$ in the Christ range \eqref{ChristRange}.
Finally, the result can be extend to the full range \eqref{FullRange} by following the ideas of Stovall from \cite{Stovall}. 


\subsection{Structure of the paper}


In Section \ref{Preliminaries} we give a brief discussion of the frequency function, focusing on properties relevant to the proof of Theorem \ref{MainDecompTheorem}. For a more detailed description we refer to \cite{Malinnikova}.

The proof of Theorem \ref{MainDecompTheorem} is divided into two parts. We first find a finite decomposition of $I$ such that \eqref{PolyDecompEstimate} holds on each subinterval $I_j$ in Section \ref{Decomposition}. The main idea here is to write any function in the class $\mathcal{A}(I,N,R)$ as the product of a polynomial of bounded degree and a holomorphic factor. The latter cannot oscillate too much in a small neighbourhood, and thus the original function is locally equivalent to a polynomial of bounded degree. 
In Section \ref{GeometricInequalitySection} we finalize the proof of Theorem \ref{MainDecompTheorem} by deducing the geometric inequality \eqref{GeometricInequality}. 
The proof is based on a recursive integral formula for the Jacobian involving the torsion and its minors, first provided in \cite{DW}. Through an induction argument we show that property \eqref{PolyDecompEstimate} implies \eqref{GeometricInequality} for any $C^d$ simple curve.

Finally, in Section \ref{Reduction} we
prove Theorem \ref{MainRestrictionTheorem} by combining Lemma $1$ in \cite{Chen} with the argument of Stovall found in Section $3$ and $4$ of \cite{Stovall}.
It suffices to prove Theorem \ref{MainRestrictionTheorem} for the range
\[
p\geq \frac{d^2+d}{d^2+d-1},
\]
as the Christ range follows directly from Section 3 of \cite{DW}.


