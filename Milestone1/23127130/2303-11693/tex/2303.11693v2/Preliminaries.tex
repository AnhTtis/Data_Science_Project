\section{Properties of the Frequency function}\label{Preliminaries}
The frequency function has been a useful tool in the theory of elliptic partial differential operators. If we consider a holomorphic function $u$ on a disc $D(z_0,R)$, the frequency function reduces to
\[
N_u(z_0,R)=2\frac{\sum_{n=1}^\infty n|c_n|^2R^{2n}}{\sum_{k=0}^\infty |c_k|^2R^{2k}},
\]
where $c_n$ is the $n^{\mathrm{th}}$ coefficient of the power series expansion of $u$ around $z_0$. If there exists $N>0$ such that $N_u(z_0,R)\leq N$, we say that $u$ is a holomorphic function of bounded frequency at the point $z_0$. For a fixed point $z_0\in\mathbb{C}$, the frequency function is a monotonically increasing function of $r$ as 
\[
\frac{\partial N_u(z_0,r)}{\partial r}>0.
\]

One of the key properties of a holomorphic function of bounded frequency is that it behaves similarly to a polynomial of bounded degree. A straightforward estimate gives that the frequency function of any polynomial $p$ of degree at most $N$ is bounded by $2N$. That is $N_p(z_0,R)\leq 2N$ for any centre $z_0\in\mathbb{C}$ and any radius $R>0$.
The following proposition lists properties of holomorphic functions of bounded frequency.
\begin{prop}\label{FreqPropProp}
Let $N\in\mathbb{N}$ and let $u$ be a holomorphic function on $D(z_0,R)$ such that $N_u(z_0,R)\leq N$. Then the following holds.
\begin{enumerate}[i)]
    \item Given $z\in D(z_0,R)$, there exists $C>0$ such that for any $0<r<R/2$ with $D(z,r)\subset D(z_0,R/2)$, we have
    \begin{equation}\label{FrequencyDoublingLemma}
        \sup_{\zeta\in D(z,r)}|u(\zeta)|\leq 2^{CN+1}\sup_{\zeta\in D\left(z,\frac{r}{2}\right)}|u(\zeta)|.
    \end{equation}
    \item If $u(0)=0$, then there exists a constant $C>0$ such that the frequency function of the derivative is bounded in a smaller disc, that is \begin{equation}\label{BoundedFreqDerivative}
    N_{u'}\left(z_0,\frac{R}{2}\right)\leq C(N+1).
    \end{equation}
    \item There exists a universal constant $1/2<c<3/4$ such that
    \begin{equation}\label{NumberOfZeros}
        \#\left\{z\in D\left(z_0,cR\right): u(z)=0\right\}\leq 2N.
    \end{equation}
    In particular $\#\left\{z\in D\left(z_0,R/2\right): u(z)=0\right\}\leq 2N$.
\end{enumerate}
\end{prop}
We omit the proof, and instead refer to \cite{Malinnikova} and \cite{Han} for \eqref{NumberOfZeros}.

From \eqref{NumberOfZeros} it follows that one can decompose any function of bounded frequency into the product of a polynomial of bounded degree and a non-vanishing holomorphic factor. Moreover, the non-vanishing holomorphic factor is also of bounded frequency in a smaller disc. We refer to the following lemma by Foster \cite{Foster}.
\begin{lemma}[Lemma $9$ in \cite{Foster}]\label{lemmaByFoster}
Assume that $N_u(z_0,R)\leq N$, and that $u=pf$, where $p$ is a polynomial and $f$ is a non-vanishing holomorphic function on $D(z_0,R/2)$. Then there exists a universal constant $C>0$ such that $N_f(z_0,R/2)\leq CN$.
\end{lemma}