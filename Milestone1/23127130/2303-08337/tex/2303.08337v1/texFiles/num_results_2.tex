\subsection{Preparing the initial state for the quasi-static force-free model}
\label{sec:initial}

This section discusses the preparation of the initial condition which will be used in all the rest numerical tests.
The quasi-static force-free model targets a post-thermal-quench plasma
in which the plasma beta is negligibly small. We prepare such an
initial force-free state via a free-boundary Grad-Shafranov
solver~\cite{Li-Tang2-SIAMJSC-2021} by zeroing out the plasma beta of
an original full-beta \SI{15}{\mega\ampere} ITER equilibrium, while holding the
poloidal magnetic flux fixed from the original full-beta \SI{15}{\mega\ampere}
free-boundary Grad-Shafranov equilibrium.  This frozen flux boundary
condition simply reflects the fact that on the time scale of the
thermal quench, which is anticipated to be on the order of a
millisecond and hence much shorter than the vacuum vessel wall time,
the vacuum vessel acts like a perfect flux conserver.  This initial
state thus has zero pressure gradient, \SI{15}{\mega\ampere} of toroidal plasma
current, and one x-point at the bottom. The details of the force-free
equilibrium solver can be found in Ref.~\cite{Li-Tang2-SIAMJSC-2021}.
The numerical solution, once transferred to the staggered grid of the
mimetic finite difference solver, has $(\nabla \times \B) \times \B$
that is not exactly zero but close to $10^{-3}$, and that introduces
an error at the first time step of the quasi-static model in
Equations~\eqref{eq:4.566} and~\eqref{eq:4.568} since the initial
velocity is nil. This gross violation of force-free condition creates
difficulties for the quasi-static force-free solver at the Newton
iteration of the first time step.

A workaround for this issue is to call a time-dependent model that
would decrease $(\nabla \times \B) \times \B$ and also compute a
velocity that would restore some balance in Equations~\eqref{eq:4.566}
and~\eqref{eq:4.568}.  This time-dependent model is as follows.
\begin{itemize}
\item In the plasma region:
\begin{align}
  m_i n_0 \frac{\partial {\V}_{i\perp} }{\partial t} - \nu m_i n_0 \nabla^2 \V_{i\perp} - \frac{1}{\mu_0} (\nabla\times{\B})\times{\B} = & ~0 \qquad \text{in} \quad \Omega^{P} , \label{eq:4.597} \\
  - \nabla^2 \Phi + \nabla\cdot\left[ {\V}_{i\perp}\times {\B}  \right] = & ~ 0 \qquad \text{in} \quad \Omega^{P}, \label{eq:4.598} \\
  \boldsymbol{\tau} - \nabla\Phi + {\V}_{i\perp}\times {\B}  = & ~ \mathbf{0} \qquad \text{in} \quad \Omega^{P}, \label{eq:4.599} \\
  \frac{\partial{\B}}{\partial t} + \nabla\times\boldsymbol{\tau} = & ~ \mathbf{0} \qquad \text{in} \quad \Omega^{P}. \label{eq:4.600}
\end{align}
\item In the wall region:
\begin{align}
  {\V}_{i\perp} = & ~ \mathbf{0} \qquad \text{in} \quad \Omega^{W} , \label{eq:4.602} \\
  - \nabla^2 \Phi = & ~ 0 \qquad \text{in} \quad \Omega^{W} \label{eq:4.603} \\
  \boldsymbol{\tau} - \nabla\Phi = & ~ \mathbf{0} \qquad \text{in} \quad \Omega^{W}, \label{eq:4.604} \\
  \frac{\partial{\B}}{\partial t} + \nabla\times\boldsymbol{\tau} = & ~ \mathbf{0} \qquad \text{in} \quad \Omega^{W} . \label{eq:4.605}
\end{align}
\item At the wall/plasma interface:
\begin{alignat}{3}
  &{\V}_{i\perp} = ~ 0 \qquad &&\text{on} \quad \Gamma^{PW} , \label{eq:4.606} \\
  &\Phi \qquad &&\text{continous across} \quad  \Gamma^{PW} ,\label{eq:phicont} \\
  &\boldsymbol{\tau} \times \mathbf{n}  \qquad &&  \text{continous across} \quad \Gamma^{PW}. \label{eq:4.608} \\
  &{\B} \cdot \mathbf{n} \qquad &&  \text{continous across} \quad \Gamma^{PW} , \label{eq:4.607} 
\end{alignat}
\item At the outer rectangular boundary $\partial\Omega$:
\begin{align}
  {\V}_{i\perp} & =  \mathbf{0},\\
  \Phi & =  {0},\\
  \boldsymbol{\tau} & =  \mathbf{0}, \\
 \B & = \B_0.
\end{align}
\end{itemize}
Here the resistivity diffusion is deliberately turned off. 
Note that unlike the quasi-static model, this time-dependent model has an inertia term in~\eqref{eq:4.597},
 and thus it supports the Alfven wave and is subject to the time step constraint due to the wave.
 A physics-based preconditioner for the small flow limit of a similar extended MHD model 
 was proposed in Ref.~\cite{Chacon2008}.
 This preconditioner is extended to our MFD solver for efficiently inverting the linearized system. 

 
Evolving the model produces a magnetic field that has a much smaller force balancing than the original field loaded into the MFD solver. 
In the final step of the initial condition preparation, the resulting velocity $\V_{i\perp,0}$ and magnetic field $\B_0$ from
this time-dependent model %, which is run for about 2000 Alfven times
are then used to update $\boldsymbol{\tau}_0$ the divergence-free
component of the electric field and $\Phi_0$ the electrostatic
potential such that we have
\begin{itemize}
\item in the plasma region:
\begin{align*}
  - \nabla^2 \Phi_0 - \nabla\cdot\left[ - {\V}_{i\perp,0}\times {\B_0} + \frac{\eta}{\mu_0}\left(\nabla\times{\B_0}\right) \right] = & ~ 0 \qquad \text{in} \quad \Omega^{P},  \\
  \boldsymbol{\tau}_0 - \nabla\Phi_0 + {\V}_{i\perp,0}\times {\B_0} - \frac{\eta}{\mu_0}\left(\nabla\times{\B_0}\right) = & ~ \mathbf{0} \qquad \text{in} \quad \Omega^{P}, 
\end{align*}
\item in the wall region:
\begin{align*}
  - \nabla^2 \Phi_0 - \nabla\cdot\left[ \frac{\eta}{\mu_0}\left(\nabla\times{\B_0}\right) \right] = & ~ 0 \qquad \text{in} \quad \Omega^{W}  \\
  \boldsymbol{\tau}_0 - \nabla\Phi_0 - \frac{\eta}{\mu_0}\left(\nabla\times{\B_0}\right) = & ~ \mathbf{0} \qquad \text{in} \quad \Omega^{W},
\end{align*}
\end{itemize}
This last step is necessary to guarantee the initial condition is consistent with the quasi-static model. 
Finally, we set a uniform initial density of $n_0 = \SI{1e+20}{\meter}^{-3}$ and that completes
the initial state $(n_0, {\V}_{i\perp,0}, \Phi_0, \boldsymbol{\tau}_0,
{\B_0} )$ for the quasi-static model.

%\QT{we could potentially add a plot of the maximum of JxB over time, just to show the approach works }
%\KL{How big is the change in B?} \ZJ{Probably one order of magnitude.}
%\XT{Zakariae, since we now have the halo current and wall force results, let's show them in section 5.4.}  
%\ZJ{I added the results in question.}