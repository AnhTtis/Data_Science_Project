\subsection{Full cold ITER VDE simulation}
\label{sec:vde}
In this section, we present the results of a full ITER VDE
simulation. We distinguish two settings for the blanket module:
isotropic resistivity and anisotropic resistivity.  In the istropic
case, a higher resistivity inhits both toroidal and poloidal current
flowing inside the blanket.
The wall current mostly resides in the vacuum vessel, driven by inductively
by the inductive electric fild.
For the anisotropic case, the poloidal
resistivity is sufficiently reduced that ploidal current can flow
inside the blanket but the toroidal current is still inhibited by a
higher toroidal resistivity. The poloidal current in the blanket now
provides the pathways for halo current that enters the blanket from
the plasma, poloidally tranverses the blanket, and then exit into the
vacuum vessel, and vice versa.

\subsubsection{Isotropic setting}
\label{sec:vde_isotropic}
\begin{comment}
\begin{figure}[ht]
\centering
\begin{subfigure}[b]{0.32\textwidth} 
\begin{center}
\includegraphics[width=\textwidth]{J_Phi_0000.png}
\caption{$t=\SI{0}{\second}$}
\label{fig:VDE_t0}
\end{center}
\end{subfigure}%
~
\begin{subfigure}[b]{0.32\textwidth} 
\begin{center}
\includegraphics[width=\textwidth]{J_Phi_0030.png}
\caption{$t=\SI{156}{\milli\second}$}
\label{fig:VDE_30dt}
\end{center}
\end{subfigure}%
~
\begin{subfigure}[b]{0.32\textwidth} 
\begin{center}
\includegraphics[width=\textwidth]{J_Phi_0050.png}
\caption{$t=\SI{260}{\milli\second}$}
\label{fig:VDE_50dt}
\end{center}
\end{subfigure}
\begin{subfigure}[b]{0.32\textwidth} 
\begin{center}
\includegraphics[width=\textwidth]{J_Phi_0060.png}
\caption{$t=\SI{312}{\milli\second}$}
\label{fig:VDE_60dt}
\end{center}
\end{subfigure}%
~
\begin{subfigure}[b]{0.32\textwidth} 
\begin{center}
\includegraphics[width=\textwidth]{J_Phi_0062.png}
\caption{$t=\SI{322.4}{\milli\second}$}
\label{fig:VDE_62dt}
\end{center}
\end{subfigure}%
~
\begin{subfigure}[b]{0.32\textwidth} 
\begin{center}
\includegraphics[width=\textwidth]{J_Phi_0063.png}
\caption{$t=\SI{327.6}{\milli\second}$}
\label{fig:VDE_63dt}
\end{center}
\end{subfigure}
\caption{The toroidal current $(\nabla \times \B)_\phi$ and magnetic field lines over time}
\label{fig:VDE_time}
\end{figure}
\end{comment}

{
\begin{figure}[ht]
\centering
\begin{subfigure}[b]{0.32\textwidth} 
\begin{center}
\includegraphics[width=\textwidth]{RJ_Phi_0000.png}
\caption{$t=\SI{0}{\second}$}
\label{fig:VDE_t0_2}
\end{center}
\end{subfigure}%
~
\begin{subfigure}[b]{0.32\textwidth} 
\begin{center}
\includegraphics[width=\textwidth]{RJ_Phi_0030.png}
\caption{$t=\SI{156}{\milli\second}$}
\label{fig:VDE_30dt_2}
\end{center}
\end{subfigure}%
~
\begin{subfigure}[b]{0.32\textwidth} 
\begin{center}
\includegraphics[width=\textwidth]{RJ_Phi_0050.png}
\caption{$t=\SI{260}{\milli\second}$}
\label{fig:VDE_50dt_2}
\end{center}
\end{subfigure}
\begin{subfigure}[b]{0.32\textwidth} 
\begin{center}
\includegraphics[width=\textwidth]{RJ_Phi_0060.png}
\caption{$t=\SI{312}{\milli\second}$}
\label{fig:VDE_60dt_2}
\end{center}
\end{subfigure}%
~
\begin{subfigure}[b]{0.32\textwidth} 
\begin{center}
\includegraphics[width=\textwidth]{RJ_Phi_0062.png}
\caption{$t=\SI{322.4}{\milli\second}$}
\label{fig:VDE_62dt_2}
\end{center}
\end{subfigure}%
~
\begin{subfigure}[b]{0.32\textwidth} 
\begin{center}
\includegraphics[width=\textwidth]{RJ_Phi_0063.png}
\caption{$t=\SI{327.6}{\milli\second}$}
\label{fig:VDE_63dt_2}
\end{center}
\end{subfigure}
\caption{The toroidal current $R (\nabla \times \B)_\phi$ and streamlines of the poloidal magnetic field over time.}
\label{fig:VDE_time_2}
\end{figure}
}


First, we consider the resistivity setting described in
Section~\ref{sec:regularization} and a fictitious viscosity
regularization term with $Re = 1000$. The VDE phenomenon is
illustrated in Figure~\ref{fig:VDE_time_2} where we notice that
vertical motion of the plasma column is correlated with the resistive
dissipation of the plasma current. The toroidal current,
$j_\phi:=(\nabla \times \B)_\phi$, in the blanket is mininal because
of the much higher material resistivity there.  In contrast, much
toroidal current can be inductively driven in the vacuum vessel, which
has a much lower resistivity that gives rise to the 500~ms wall time
for the ITER vacuum vessel.  The magnetic field lines are also
projected to the poloidal plane in Figure~\ref{fig:VDE_time_2}.  It is
noted that simultaneously the magnetic axis (the o-point of the
streamlines) progressively moves upward until it hits the rigid wall
and the toroidal current in the plasma chamber gets almost entirely
dissipated. The induced toroidal current in the vacuum vessel persists
till the end of the simulation ($t=\SI{327.6}{\milli\second}$), because the ITER vacuum
vessel has a wall time that is longer ($\SI{500}{\milli\second}$).

\pgfplotstableread{data_to_plot/dataRe1000_divB.txt}\Bdivtable
\pgfplotstablecreatecol[create col/expr={\thisrow{y} * 5)}]{nonnormB}\Bdivtable
\pgfplotstablecreatecol[create col/expr={0.0052*(\thisrow{x})}]{realtime}\Bdivtable

\begin{figure}[ht]
\centering
\begin{subfigure}[b]{0.5\textwidth} 
\begin{center}
\begin{tikzpicture}[scale=0.8,>=latex]
\begin{axis}[xlabel=\textsc{{Time (s)}}, scaled y ticks=base 10:14,
    y tick label style={
        /pgf/number format/fixed,
        /pgf/number format/precision=1
    },
     ylabel=\textsc{{$||\nabla \cdot \widetilde{\B}||_{\inf}$}} , ymajorgrids, xmajorgrids, axis background/.style={fill=gray!4}]
%ytick={3.981e-10,3.982e-10,3.983e-10,3.984e-10,3.985e-10,3.986e-10,3.987e-10,3.988e-10,3.989e-10}, yticklabels={3.981e-10,3.982e-10,3.983e-10,3.984e-10,3.985e-10,3.986e-10,3.987e-10,3.988e-10,3.989e-10},
\addplot+[color=teal,line width=1pt,mark=none] table[x=realtime,y] \Bdivtable;
\end{axis}
\end{tikzpicture}
\caption{normalized}
\label{fig:normBdiv_evolution}
\end{center}
\end{subfigure}%
~
\begin{subfigure}[b]{0.5\textwidth} 
\begin{center}
\begin{tikzpicture}[scale=0.8,>=latex]
\begin{axis}[xlabel=\textsc{{Time (s)}}, scaled y ticks=base 10:12, ytick scale label code/.code={$10^{-12}$}, ylabel=\textsc{{$||\nabla \cdot \B||_{\inf}$}} , ymajorgrids, xmajorgrids, axis background/.style={fill=gray!4}]
%ytick={3.981e-10,3.982e-10,3.983e-10,3.984e-10,3.985e-10,3.986e-10,3.987e-10,3.988e-10,3.989e-10}, yticklabels={3.981e-10,3.982e-10,3.983e-10,3.984e-10,3.985e-10,3.986e-10,3.987e-10,3.988e-10,3.989e-10},
\addplot+[color=teal,line width=1pt,mark=none] table[x=realtime,y=nonnormB] \Bdivtable;
\end{axis}
\end{tikzpicture}
\caption{non-normalized}
\label{fig:nonnormBdiv_evolution}
\end{center}
\end{subfigure}
\caption{Evolution of the infinity norm of the divergence of the magnetic field over time.}
\label{fig:Bdiv_evolution}
\end{figure}

This example is further used to investigate the divergence-free
property of the magnetic field of the MFD solver.
Figure~\ref{fig:Bdiv_evolution} depicts the evolution of the infinity
norm of the divergence of the magnetic field over
time. Figure~\ref{fig:nonnormBdiv_evolution} shows the non-normalized
quantity whereas Figure~\ref{fig:normBdiv_evolution} corresponds to
the normalized one.  Taking Figure~\ref{fig:normBdiv_evolution} as an
example, the initial divergence is not exactly zero but rather of the
order of $10^{-10}$.  Note that the initial condition is prepared
through several steps as described in Section~\ref{sec:initial}, which
makes it difficult to get an exactly discrete divergence-free field,
especially corresponding to the mimetic divergence operator.
Nevertheless, as time evolves, the divergence fluctuates between two
successive time steps by $\pm 10^{-15}$, which confirms the
divergence-free property of the magnetic field as stated in
Section~\ref{sec:mfd_discretization}.  Here the fluctuation is due to
an iterative nonlinear solver being used.  Nevertheless, such a small
error is more than enough for the tests considered here for avoiding
any potential numerical artifacts due to the divergence error.

{
\begin{figure}[ht]
\centering
\begin{subfigure}[b]{0.32\textwidth} 
\begin{center}
\includegraphics[width=\textwidth]{J_poloidal_halo0000.png}
\caption{$t=\SI{0}{\second}$}
\label{fig:VDE_halo_t0}
\end{center}
\end{subfigure}%
~
\begin{subfigure}[b]{0.32\textwidth} 
\begin{center}
\includegraphics[width=\textwidth]{J_poloidal_halo0015.png}
\caption{$t=\SI{44.19}{\milli\second}$}
\label{fig:VDE_halo_15dt}
\end{center}
\end{subfigure}%
~
\begin{subfigure}[b]{0.32\textwidth} 
\begin{center}
\includegraphics[width=\textwidth]{J_poloidal_halo0035.png}
\caption{$t=\SI{103.11}{\milli\second}$}
\label{fig:VDE_halo_35dt}
\end{center}
\end{subfigure}
\begin{subfigure}[b]{0.32\textwidth} 
\begin{center}
\includegraphics[width=\textwidth]{J_poloidal_halo0053.png}
\caption{$t=\SI{156.14}{\milli\second}$}
\label{fig:VDE_halo_53dt}
\end{center}
\end{subfigure}%
~
\begin{subfigure}[b]{0.32\textwidth} 
\begin{center}
\includegraphics[width=\textwidth]{J_poloidal_halo0063.png}
\caption{$t=\SI{185.60}{\milli\second}$}
\label{fig:VDE_halo_63dt}
\end{center}
\end{subfigure}%
~
\begin{subfigure}[b]{0.32\textwidth} 
\begin{center}
\includegraphics[width=\textwidth]{J_poloidal_halo0075.png}
\caption{$t=\SI{220.96}{\milli\second}$}
\label{fig:VDE_halo_75dt}
\end{center}
\end{subfigure}
\caption{The toroidal current $(\nabla \times \B)_\phi$ and streamlines of the poloidal current $\left(\nabla\times{\B}\right)_{\rm pol}$ over time. At t=$\SI{103.11}{\milli\second}$, a halo current (on the top left region in subfigure~\ref{fig:VDE_halo_35dt}) is observed to form.}
\label{fig:VDE_halocurrent_time}
\end{figure}
}

\subsubsection{Anisotropic setting}
\label{sec:vde_isotropic}
Lastly, we provide the results of a full ITER VDE simulation for the
case of anisotropic resistivity at the first wall and blanket module
(see Formulas~\eqref{eqn:eq34} and~\eqref{eqn:eq35} in
Section~\ref{sec:wall_model}) and a fictitious viscosity
regularization term with $Re = 100$.  The resistivity values tested
are as follows:
\begin{itemize}
    \item $\eta = \SI{1.71e-5}{\ohm\meter}$ in the plasma chamber (that corresponds to an electron temperature of $T_e = 10{\rm eV}$); %($\Omega^{(P)}$);  
    \item $\eta_t = \SI{4.4e-2}{\ohm\cdot\meter}$ in the blanket module;
    \item $\eta_p = \SI{1.71e-5}{\ohm\cdot\meter}$ in the blanket module;
    \item $\eta = \SI{1.3e-6}{\ohm\meter}$ in the vacuum vessel;
    \item $\eta = \SI{1.3e-3}{\ohm\meter}$ in the region outside the wall.
\end{itemize}

{
\begin{figure}[ht]
\centering
\begin{subfigure}[b]{0.32\textwidth} 
\begin{center}
\includegraphics[width=\textwidth]{jxB_phi_halo0000.png}
\caption{$t=\SI{0}{\second}$}
\label{fig:jxB_phi_halo_t0}
\end{center}
\end{subfigure}%
~
\begin{subfigure}[b]{0.32\textwidth} 
\begin{center}
\includegraphics[width=\textwidth]{jxB_phi_halo0015.png}
\caption{$t=\SI{44.19}{\milli\second}$}
\label{fig:jxB_phi_halo_15dt}
\end{center}
\end{subfigure}%
~
\begin{subfigure}[b]{0.32\textwidth} 
\begin{center}
\includegraphics[width=\textwidth]{jxB_phi_halo0035.png}
\caption{$t=\SI{103.11}{\milli\second}$}
\label{fig:jxB_phi_halo_35dt}
\end{center}
\end{subfigure}
\begin{subfigure}[b]{0.32\textwidth} 
\begin{center}
\includegraphics[width=\textwidth]{jxB_phi_halo0053.png}
\caption{$t=\SI{156.14}{\milli\second}$}
\label{fig:jxB_phi_halo_53dt}
\end{center}
\end{subfigure}%
~
\begin{subfigure}[b]{0.32\textwidth} 
\begin{center}
\includegraphics[width=\textwidth]{jxB_phi_halo0063.png}
\caption{$t=\SI{185.60}{\milli\second}$}
\label{fig:jxB_phi_halo_63dt}
\end{center}
\end{subfigure}%
~
\begin{subfigure}[b]{0.32\textwidth} 
\begin{center}
\includegraphics[width=\textwidth]{jxB_phi_halo0075.png}
\caption{$t=\SI{220.96}{\milli\second}$}
\label{fig:jxB_phi_halo_75dt}
\end{center}
\end{subfigure}
\caption{The toroidal component of the Lorentz Force $((\nabla \times \B) \times \B) _\phi$ and streamlines of the poloidal magnetic field over time.}
\label{fig:jxB_phi_halocurrent_time}
\end{figure}
}

The so called ``halo currents'' are illustrated in Figure~\ref{fig:VDE_halocurrent_time} where the toroidal current is given by the colormap and the poloidal current is presented  as streamlines.
We observe that the toroidal current is eliminated and the halo current flows only poloidally in the blanket to enter the vacuum vessel, where the electrical current can have a strong toroidal component.

{
\begin{figure}[ht]
\centering
\begin{subfigure}[b]{0.32\textwidth} 
\begin{center}
\includegraphics[width=\textwidth]{jxB_poloidal_halo0000.png}
\caption{$t=\SI{0}{\second}$}
\label{fig:jxB_poloidal_halo_t0}
\end{center}
\end{subfigure}%
~
\begin{subfigure}[b]{0.32\textwidth} 
\begin{center}
\includegraphics[width=\textwidth]{jxB_poloidal_halo0015.png}
\caption{$t=\SI{44.19}{\milli\second}$}
\label{fig:jxB_poloidal_halo_15dt}
\end{center}
\end{subfigure}%
~
\begin{subfigure}[b]{0.32\textwidth} 
\begin{center}
\includegraphics[width=\textwidth]{jxB_poloidal_halo0035.png}
\caption{$t=\SI{103.11}{\milli\second}$}
\label{fig:jxB_poloidal_halo_35dt}
\end{center}
\end{subfigure}
\begin{subfigure}[b]{0.32\textwidth} 
\begin{center}
\includegraphics[width=\textwidth]{jxB_poloidal_halo0053.png}
\caption{$t=\SI{156.14}{\milli\second}$}
\label{fig:jxB_poloidal_halo_53dt}
\end{center}
\end{subfigure}%
~
\begin{subfigure}[b]{0.32\textwidth} 
\begin{center}
\includegraphics[width=\textwidth]{jxB_poloidal_halo0063.png}
\caption{$t=\SI{185.60}{\milli\second}$}
\label{fig:jxB_poloidal_halo_63dt}
\end{center}
\end{subfigure}%
~
\begin{subfigure}[b]{0.32\textwidth} 
\begin{center}
\includegraphics[width=\textwidth]{jxB_poloidal_halo0075.png}
\caption{$t=\SI{220.96}{\milli\second}$}
\label{fig:jxB_poloidal_halo_75dt}
\end{center}
\end{subfigure}
\caption{The magnitude of the poloidal component of the Lorentz Force $||((\nabla \times \B) \times \B)_{\rm pol}||$ and streamlines of the poloidal magnetic field over time.}
\label{fig:jxB_poloidal_halocurrent_time}
\end{figure}
}

The electromagnetic force loading ($\mathbf{j}\times\mathbf{B}$) due
to the halo and eddy current in the blanket and vacuum vessel during a
VDE can also be quantified from our simulation. The toroidal and
poloidal components of the Lorentz force
($\mathbf{j}\times\mathbf{B}$) are given in
Figures~\ref{fig:jxB_phi_halocurrent_time}
and~\ref{fig:jxB_poloidal_halocurrent_time} respectively, with the
magnetic field lines at six different times of the simulation.  We
notice that the Lorentz force is initially concentrated along the
plasma/wall interface whereas later in time, this force becomes
stronger at the vacuum vessel and almost nil elsewhere.
