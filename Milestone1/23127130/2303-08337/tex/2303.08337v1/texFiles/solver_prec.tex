\section{Solver and preconditioning strategy}
\label{sec:precsolve}

This section discusses the details of the solver and block preconditioning strategy. 
In the most outer lever, a nonlinear solver based on Jacobian-free Newton-Krylov (JFNK) and inexact Newton is used. For the inner linear solver, a finite difference coloring Jacobian matrix is formed as a preconditioner 
for the Jacobian matrix inversion. Relying on the finite difference coloring approximated Jacobian turns out to be necessary in this work, since the mimetic formulations, as outlined in the previous section, involve many projections between different bases, and
implementing an analytical Jacobian for such a complicated system is not practical.

The state-of-art approach for inverting such a fully coupled system is to seek an effective preconditioner. 
Here we outline the details of the preconditioning strategy. 
For ease of presentation, continuous operators are used in the following discussion.  
Note that the Jacobian matrix corresponding to the quasi-static perpendicular plasma dynamics model takes the form:

{
\footnotesize
\begin{align}
& \mathbf{J} \, d\mathbf{U} = \nonumber \\
& %\begin{blockarray}{cccccc}
%d{\V}_{i\perp} & d{\Phi} & d \boldsymbol{\tau} & d{\B} & dn_i\\
\begin{bmatrix}
  \dfrac{1}{dt}\mathbf{I} + (\nabla \cdot \V_{0}) \mathbf{I} + {\V}_{0} \cdot \nabla & \mathbf{0} & \mathbf{0} & \mathbf{0} & n_{i,0} \nabla \cdot \quad+ (\nabla n_{i,0}) \cdot \quad \smallskip\\
  \mathbf{0} & \mathbf{I} & -\nabla \quad & - \frac{\eta}{\mu_0} (\nabla \times \quad ) + ({\V}_{0} \times \quad ) & - \mathbf{B}_0 \times \smallskip \\
  \mathbf{0} & \mathbf{0} & - \nabla^2 \quad & - \nabla \cdot (\frac{\eta}{\mu_0} \nabla \times \quad) + \nabla \cdot ({\V}_{0} \times \quad) & - \nabla \cdot ( {\B}_0 \times \quad ) \smallskip\\
  \mathbf{0} & \nabla \times \quad & \mathbf{0} & \dfrac{1}{dt}\mathbf{I} & \mathbf{0} \smallskip \\
  \mathbf{0} & \mathbf{0} & \mathbf{0} & \mathbf{C}_2 & \mathbf{C}_1
\end{bmatrix}
\begin{bmatrix}
d n_i \medskip\\
d \boldsymbol{\tau}\medskip\\
d\Phi \bigskip\\
d {\B}\medskip\\
d{\V}_{i\perp}
\end{bmatrix}
%\end{blockarray},
\label{eqn:4.581}
\end{align}}
where $\mathbf{C}_1$ and $\mathbf{C}_2$ are different in each domain, given by:
\begin{alignat}{3}
\mathbf{C}_1 := & ~\mathbf{C}_{1,p} \qquad &&\text{in} \quad \Omega^{P}, \label{eq:4.582} \\
\mathbf{C}_1 := & ~\mathbf{C}_{1,w} \qquad &&\text{in} \quad \Omega^{W}, \label{eq:4.583} \\
\mathbf{C}_2 := & ~\mathbf{C}_{2,p} \qquad &&\text{in} \quad \Omega^{P}, \label{eq:4.584} \\
\mathbf{C}_2 := & ~\mathbf{C}_{2,w} \qquad &&\text{in} \quad \Omega^{W}; \label{eq:4.585}
\end{alignat}
and
{\footnotesize
\begin{align}
& \mathbf{C}_{1,p} = ~
\begin{blockarray}{cccc}
d{\V}_r & d{\V}_{\phi} & d{\V}_{z} \medskip\\
\begin{block}{[ccc]c}
  \dfrac{\nu n_0 m_i}{r^2}\mathbf{I} - \nu n_0 m_i \nabla^2 & \dfrac{2 \nu n_0 m_i}{r^2} \dfrac{\partial}{\partial \phi} & \mathbf{0} &  d{\V}_r \medskip\\
  {\mathcal{P}}_{\mathrm{f\rightarrow v}}(\B)_{r,0} \mathbf{I} & {\mathcal{P}}_{\mathrm{f\rightarrow v}}(\B)_{\phi,0} \mathbf{I} & {\mathcal{P}}_{\mathrm{f\rightarrow v}}(\B)_{z,0} \mathbf{I} & d{\V}_{\phi} \bigskip\\
  \mathbf{0} & \mathbf{0} & - \nu n_0 m_i \nabla^2 & d{\V}_{z} \\
\end{block}
\end{blockarray} , \\
& \mathbf{C}_{1,w} =  ~ \mathbf{I} , \\
&\mathbf{C}_{2,p} = ~ \\
&\begin{blockarray}{cccc}
d{\B}_r & d{\B}_{\phi} & d{\B}_{z} \medskip\\
\begin{block}{[ccc]c}
  - \dfrac{\B_{\phi,0}}{r}\dfrac{\partial}{\partial \phi} - \B_{z,0} \dfrac{\partial}{\partial z} & \left( \dfrac{\partial \B_{\phi,0}}{\partial r} - \dfrac{1}{r}\dfrac{\partial \B_{r,0}}{\partial \phi} + \dfrac{2 \B_{\phi,0}}{r} \right) \mathbf{I} + \B_{\phi,0} \dfrac{\partial}{\partial r} & \left( \dfrac{\partial \B_{z,0}}{\partial r} - \dfrac{\partial \B_{r,0}}{\partial z} \right) \mathbf{I} + \B_{z,0} \dfrac{\partial}{\partial r} &  d{\V}_r \medskip\\
  \V_{r,0} \mathbf{I} & \V_{\phi,0} \mathbf{I} & \V_{z,0} \mathbf{I} & d{\V}_{\phi} \bigskip\\
  \left( \dfrac{\partial \B_{r,0}}{\partial z} - \dfrac{\partial \B_{z,0}}{\partial r} \right) \mathbf{I} + \B_{r,0} \dfrac{\partial}{\partial z} & \left( \dfrac{\partial \B_{\phi,0}}{\partial z} - \dfrac{1}{r}\dfrac{\partial \B_{z,0}}{\partial \phi} \right) \mathbf{I} + \B_{\phi,0} \dfrac{\partial}{\partial z} & - \dfrac{\B_{\phi,0}}{r}\dfrac{\partial}{\partial \phi} - \B_{r,0} \dfrac{\partial}{\partial r} & d{\V}_{z} \\
\end{block}
\end{blockarray} , \\
& \mathbf{C}_{2,w} =  ~ \mathbf{0}.
\end{align}}
Note that in the mimetic finite difference, $\mathbf{C}_{2,p}$ as well as its sub-blocks are not square matrices, since $\V$ and $\B$ live on different positions. 
We indicate the corresponding locations of sub-blocks in the full Jacobian using the notations of $d{\B}$ and $d{\V}$.

The linear system resulting from the linearization of the PDE is solved iteratively with a FGMRES solver preconditioned by a four-level block preconditioner that uses PETSc's fieldsplit interface. 
Our strategy is inspired from the recent coupled preconditioning work~\cite{joshaghani2019composable} that considered a four-field system.
In the first level, the following preconditioner is used
\begin{align}
  \mathbf{P}^{-1} &= \begin{bmatrix} \mathcal{KSP}(\mathbf{J}_{n_i}) & \mathbf{0} \\ \mathbf{0} & \mathbf{I}_{\boldsymbol{\tau} \Phi {\B} \V} \end{bmatrix}\begin{bmatrix} \mathbf{I}_{n_i} & - \mathbf{C}_3 \\ \mathbf{0} & \mathbf{I}_{\boldsymbol{\tau} \Phi {\B} \V} \end{bmatrix}\begin{bmatrix} \mathbf{I}_{n_i} & \mathbf{0} \\ \mathbf{0} & \mathcal{KSP}(\mathbf{J}_{\boldsymbol{\tau} \Phi {\B} \V}) \end{bmatrix}.
\end{align}
where $\mathbf{C}_{3}$ is the top right off-diagonal block of the Jacobian matrix
\begin{align}
  \mathbf{C}_{3} = 
    \left[\mathbf{0}\qquad  \mathbf{0} \qquad \mathbf{0} \qquad n_{i,0} \nabla \cdot \quad+ (\nabla n_{i,0}) \cdot \quad \right],
\end{align}
and $\mathcal{KSP}$ is used to denote the linear solver from PETSc.
Since the bottom left block of the Jacobian matrix is zero ($\mathbf{J}_{\boldsymbol{\tau} \Phi {\B} \V, n_i} = \mathbf{0}$), such a factorization is~\emph{exact} as long as we can invert $\mathbf{J}_{n_i}$ and $\mathbf{J}_{\boldsymbol{\tau} \Phi {\B} \V}$ exactly. 
Note that this factorization corresponds to the ``multiplicative'' option of fieldsplit in PETSc. 
In particular, we use the following options for the two remaining linear solvers:
\begin{itemize}
\item
$\mathcal{KSP}(\mathbf{J}_{n_i})$ : GMRES solver preconditioned with a block Jacobi preconditioner;
\item
$\mathcal{KSP}(\mathbf{J}_{\boldsymbol{\tau} \Phi {\B} \V})$ : FGMRES solver preconditioned with $\mathbf{P}^{-1}_{\boldsymbol{\tau} \Phi {\B} \V}$.
\end{itemize}
Now the problem is converted to looking for an efficient preconditioner $\mathbf{P}^{-1}_{\boldsymbol{\tau} \Phi {\B} \V}$ for $\mathbf{J}_{\boldsymbol{\tau} \Phi {\B} \V}$,
which is our second level of the block preconditioner. We use the following preconditioner based on the conventional Schur complement
\begin{align}
  \mathbf{P}^{-1}_{\boldsymbol{\tau} \Phi {\B} \V} &= \begin{bmatrix} \mathbf{I}_{\boldsymbol{\tau}} & - \mathbf{J}_{\boldsymbol{\tau}, \Phi {\B} \V} \\ \mathbf{0} & \mathbf{I}_{\Phi {\B} \V} \end{bmatrix}\begin{bmatrix} \mathcal{KSP}(\mathbf{J}_{\boldsymbol{\tau}}) &  \mathbf{0} \\ \mathbf{0} & \mathcal{KSP}(\mathbf{S}_{\Phi {\B} \V}) \end{bmatrix}\begin{bmatrix} \mathbf{I}_{\boldsymbol{\tau}} & \mathbf{0} \\ - \mathbf{J}_{\Phi {\B} \V, \boldsymbol{\tau}} & \mathbf{I}_{\Phi {\B} \V} \end{bmatrix}.
\end{align}
Note that $\mathbf{J}_{\boldsymbol{\tau}}$ can be trivially inverted since $\mathbf{J}_{\boldsymbol{\tau}}$ is an identity matrix. In addition, $\mathbf{J}_{\boldsymbol{\tau}}$ being an identity ensures that the Schur complement $\mathbf{S}_{\Phi {\B} \V}$  can be~\emph{exactly} formed  through a matrix multiplication. In particular, the Schur complement is 
\begin{equation}
\mathbf{S}_{\{ \Phi {\B} {\V}_{i\perp} \}} \, d\mathbf{U} =
\begin{bmatrix}
  - \nabla^2 \quad & - \nabla \cdot (\frac{\eta}{\mu_0} \nabla \times \quad) + \nabla \cdot ({\V}_{0} \times \quad) & - \nabla \cdot ( {\B}_0 \times \quad ) \smallskip\\
  \mathbf{0} & \dfrac{1}{dt}\mathbf{I} + \nabla \times (\frac{\eta}{\mu_0} \nabla \times \quad) - \nabla \times ({\V}_{0} \times \quad) & \nabla \times ( {\B}_0 \times \quad ) \smallskip \\
  \mathbf{0} & \mathbf{C}_2 & \mathbf{C}_1
\end{bmatrix}
\begin{bmatrix}
d\Phi \medskip\\
d {\B}\medskip\\
d{\V}_{i\perp}
\end{bmatrix} .
%\end{blockarray},
\label{eqn:EBV_Schur}
\end{equation}
Note that the zero sub-block in $\mathbf{S}_{\{ \Phi {\B} {\V}_{i\perp} \}}$ is a result of the property of the mimetic operator that guarantees the curl of the gradient operator is zero in the discrete level (up to machine precision). 
This shows another advantage of using the MFD discretization. 
Again, the overall problem is converted to finding an efficient preconditioner for $\mathbf{S}_{\{ \Phi {\B} {\V}_{i\perp} \}}$.
In the third level, the following preconditioner is used
\begin{align}
  \mathbf{P}^{-1}_{\Phi {\B} \V} &= \begin{bmatrix} \mathcal{KSP}(\mathbf{J}_{\Phi}) & \mathbf{0} \\ \mathbf{0} & \mathbf{I}_{{\B} \V} \end{bmatrix}\begin{bmatrix} \mathbf{I}_{\Phi} & -\mathbf{C}_4 \\ \mathbf{0} & \mathbf{I}_{{\B} \V} \end{bmatrix}\begin{bmatrix} \mathbf{I}_{\Phi} & \mathbf{0} \\ \mathbf{0} & \mathcal{KSP}(\mathcal{S}) \end{bmatrix},
\end{align}
where 
\begin{align}
  \mathbf{C}_{4} = 
  \left[- \nabla \cdot (\frac{\eta}{\mu_0} \nabla \times \quad) + \nabla \cdot ({\V}_{0} \times \quad) \qquad - \nabla \cdot ( {\B}_0 \times \quad ) \right],
\end{align}
and 
\begin{align}
 \mathcal{S} = 
 \left[
\begin{matrix}
  \dfrac{1}{dt}\mathbf{I} + \nabla \times (\frac{\eta}{\mu_0} \nabla \times \quad) - \nabla \times ({\V}_{0} \times \quad) & \nabla \times ( {\B}_0 \times \quad )  \\
  \mathbf{C}_2 & \mathbf{C}_1  
\end{matrix}\right].
\end{align}
The preconditioner $\mathbf{P}^{-1}_{\Phi {\B} \V}$ corresponds to the multiplicative option of two sub-block solvers, which is again~\emph{exact} since the bottom left block of ${\mathbf{S}}_{\Phi {\B} \V}$ is zero. 
The following options are used for the two remaining linear solvers
\begin{itemize}
\item $\mathcal{KSP}(\mathbf{J}_{\Phi})$ : GMRES solver preconditioned with BoomerAMG preconditioner from the hypre package. Note here that $\mathbf{J}_{\Phi} = - \nabla^2 $;
\item $\mathcal{KSP}(\mathcal{S})$ : A direct solver (SuperLU\_DIST) is used.
%\item
%$\mathcal{KSP}(\mathbf{S}_{\Phi {\B} \V})$ : No linear solver is called here, only a preconditioner $\mathbf{M}^{-1}_{\Phi {\B} \V}$ is applied directly on the Schur complement $\mathbf{S}_{\Phi {\B} \V}$ defined in~\eqref{eqn:EBV_Schur}. Note here that the flag used
%\begin{verbatim}
%-fieldsplit_TEBV_pc_fieldsplit_schur_precondition selfp
%\end{verbatim}
%means that the preconditioning for $\mathbf{S}_{\Phi {\B} \V}$ is generated from an explicitly-assembled approximation $\widehat{\mathbf{S}}_{\Phi {\B} \V} := \mathbf{J}_{\Phi {\B} \V} - \mathbf{J}_{\Phi {\B} \V, \boldsymbol{\tau}} (diag(\mathbf{J}_{\boldsymbol{\tau}}))^{-1} \mathbf{J}_{\boldsymbol{\tau}, \Phi {\B} \V}$. Since $\mathbf{J}_{\boldsymbol{\tau}}$ is just an identity matrix, the approximation is exact: $\widehat{\mathbf{S}}_{\Phi {\B} \V} = {\mathbf{S}}_{\Phi {\B} \V}$;
%\item $\mathbf{M}^{-1}_{\Phi {\B} \V}$ : a two-level multiplicative (since the bottom left block of ${\mathbf{S}}_{\Phi {\B} \V}$ is zero) fieldsplit preconditioner given by
%\item $\mathbf{C}_{4}$ : the top right off-diagonal block of the Schur complement matrix $\mathbf{S}_{\Phi {\B} \V}$
%through SUPERLU\_DIST the parallel direct solver package for LU factorization. 
%Note here that the flag
%\begin{verbatim}
%-fieldsplit_TEBV_fieldsplit_EBV_fieldsplit_BV_mat_superlu_dist_replacetinypivot
%\end{verbatim}
%which replaces tiny pivots in $\mathbf{S}_{\Phi {\B} \V / {\B} \V}$ is activated.
\end{itemize}

The advantage of our preconditioning  is that except for the sub-block solvers for $\mathbf{J}_{n_i}$ and $\mathbf{J}_{\Phi}$, the factorization and solver strategy is~\emph{exact}. 
This is an outcome of the proposed preconditioning strategy and the properties of the mimetic operators. In addition, the solver for $\mathbf{J}_{\Phi}$ is scalable due to algebraic multigrid being used. 
Overall, the nonlinear solver and the preconditioner perform very well in practice, which will be demonstrated in the numerical section. 

As expected,  majority of the computational time comes from the inversion of the operator $\mathcal{S}$. The coupling between $\B$ and $\V$ in $\mathcal{S}$ is non-conventional. In particular, we find that there is a strong off-diagonal coupling in this sub-system. Note that the commonly used physics-based preconditioning strategy through parabolization~\cite{Chacon2008, tang2022adaptive}  does not work since this sub-system does not propagate the wave. We leave it to the future study for an algorithmic scalable solver. Nevertheless, the above preconditioner is sufficient for the problems considered in this work. The typical walltime to take a single time step using the proposed solver is about half a minute on 128 CPUs.  In comparison, a director solver or an iterative solver preconditioned with a generic preconditioner such as block incomplete LU need several hours to invert the same system.  
