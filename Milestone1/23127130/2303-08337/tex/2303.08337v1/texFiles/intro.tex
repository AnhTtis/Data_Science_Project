\section{Introduction}
\label{sec:intro}

The macroscopic plasma motion can be described by the plasma momentum
equation in the magnetohydrodynamic (MHD) model~\cite{Goedbloed-Poedts-MHD-2004}
\begin{align}
\rho \left(\frac{\partial {\V}}{\partial t} +
     {\V}\cdot\nabla{\V}\right) = {\j}\times{\B} - \nabla p - \nabla\cdot\boldsymbol{\pi}, \label{eq:momentum}
\end{align}
where $\rho$ is the plasma mass density,
%\KL{units?} \ZJ{(in $\SI{}{\kilo\gram\cdot\meter}^{-3}$)},
${\V}$ the flow field,
%\ZJ{(in $\SI{}{\meter\cdot\second^{-1}}$)},
$p$ the plasma pressure,
%\ZJ{(in $\SI{}{\kilo\gram\cdot\meter^{-1}\cdot\second^{-2}}$ or $\SI{}{\pascal}$)},
$\boldsymbol{\pi}$ the viscosity tensor,
%\ZJ{(in $\SI{}{\kilo\gram\cdot\meter^{-1}\cdot\second^{-2}}$)},
${\j}\times{\B}$ the Lorentz force with ${\B}$ the magnetic field,
%\ZJ{(in $\SI{}{\kilo\gram\cdot\ampere^{-1}\cdot\second^{-2}}$ or $\SI{}{\tesla}$)},
and $\mu_0 {\j} = \nabla\times{\B}$ the plasma current
%\ZJ{(in $\SI{}{\kilo\gram\cdot\meter^{-1}\cdot\ampere^{-1}\cdot\second^{-2}}$)}
with $\mu_0$ the vacuum magnetic permeability.
%\ZJ{(in $\SI{}{\kilo\gram\cdot\meter\cdot\ampere^{-2}\cdot\second^{-2}}$ or $\SI{}{\henry\cdot\meter^{-1}}$)}.
A large class of practical problems considers
low-beta plasmas in which the plasma pressure is low compared with the
magnetic pressure. 
%The plasma beta is defined as the ratio of the plasma pressure and the magnetic pressure.\KL{remove sentence???}
The limiting case is $\nabla p \approx 0,$ so after
a possible transient period over which the plasma flow is damped by
viscosity, the plasma will settle into a steady state that is
approximately force-free,
\begin{align}
{\j} \times {\B} = 0, \label{eq:force-free-constraint}
\end{align}
and ${\V} \approx 0$ (hence $\boldsymbol{\pi}\approx 0$). This is a~\emph{force-free plasma} 
where the plasma current supports a force-free
magnetic field satisfying Eq.~(\ref{eq:force-free-constraint}).  The
general solution to Eq.~(\ref{eq:force-free-constraint}) is~\cite{ck-apj-1957}
\begin{align}
{\j} = \lambda {\B}. \label{eq:force-free-field}
\end{align}
A quasi-neutral plasma has $\nabla\cdot{\j}=0.$ Combining this
constraint with Eq.~(\ref{eq:force-free-field}), and making use of
$\nabla\cdot{\B}=0,$ one finds
\begin{align}
{\B}\cdot\nabla\lambda = 0, \label{eq:BdotLambda=0}
\end{align}
which says that $\lambda$ is a constant along a magnetic field line.
If the magnetic field is integrable with an irrational winding
number~\cite{boozer-rmp-2004}, $\lambda$ will be a function of the
flux surface label $\psi,$ i.e., $\lambda(\psi).$ This can be the case in
which the magnetic field obeys geometrical symmetry, say the toroidal
symmetry in a tokamak. For a three-dimensional magnetic field with no geometrical symmetry,
the magnetic field line is generally non-integrable, so the stochastic
field line fills an ergodic sub-volume in space, for which
Eq.~(\ref{eq:BdotLambda=0}) implies a constant $\lambda.$ If the
magnetic field produces globally stochastic field lines, one would
have $\lambda$ a global constant.

It is of interest to note that ${\j}=\lambda {\B}$ with $\lambda$ a
global constant is the celebrated Taylor state, which is the minimum
energy state for a zero-beta plasma under the constraint of conserved
magnetic
helicity~\cite{woltjer-pnas-1958,taylor-prl-1974,taylor-rmp-1986}. The
connection between magnetic field line topology and $\lambda$ indeed
underlies the dynamical processes by which the so-called magnetic or
Taylor relaxation is realized (see Ref.~\cite{tb-pop-2004b} for instance). Namely, when the
plasma becomes unstable to macroscopic MHD instabilities, flux
surfaces would be broken so that the magnetic field lines can become globally
stochastic, which then relax $\lambda$ to be a global
constant. For weak stochasticity, even small perpendicular current
associated with the plasma inertia, can produce significant modulation
in $\lambda$ along the magnetic field line due to the
Pfirsch-Schl\"{u}ter effect~\cite{tb-pop-2003}. As the MHD
instabilities die down and the flux surfaces reheal, like those in
laboratory confinement experiments with reversed field
pinch~\cite{Bonfiglio-etal-prl-2013}, spheromaks~\cite{tb-pop-2008},
spherical tokamak~\cite{tb-pop-2005a,tb-pop-2006a,tb-pop-2007}, and
tokamak disruptions~\cite{Izzo-pop-2021}, or in the solar
corona~\cite{Wiegelmann-Sakurai-LRSP-2012} and radio
lobes~\cite{tang-apj-2008}, one can end up with a force-free plasma
with $\lambda(\psi)$ a function of the flux surface label $\psi,$ also know
as a nonlinear force-free magnetic
field~\cite{Pevtsov-etal-ApjL-1994,Regnier-etal-AA-2002,DeRosa-etal-AA-2009,Valori-SolarPhysics-2012},
which despite the deviation from the constant $\lambda$ Taylor state,
retains the key feature of magnetic self-organization via a resonant
coupling between the helicity injection source and global magnetic
configuration~\cite{tb-prl-2005a,tb-prl-2005b,tang-prl-2007}. Further
evolution of such a nonlinear force-free plasma, for example, a
post-thermal-quench tokamak plasma undergoing a cold vertical
displacement event (VDE), is governed by the slow transport process,
namely the resistive decay of the plasma current. By slow, we refer
to a time scale much longer than the transit time of the Alfv\'{e}n
waves, which sets the time scale for the zero-beta plasma to
re-establish force-balance. This condition is easily satisfied in a
plasma with Lundquist number much greater than unity, which applies to
most cases of practical interest.
It is of interest to note that during the cold VDE after the plasma thermal quench,
the Ohmic heating power by a decaying plasma current, is mostly balanced by
radiative cooling, so the zero-beta or force-free plasma remains a good approximation
throughout the VDE~\cite{McDevitt_2023}.

The drastic time scale separation between Ohmic decay of the plasma
current and re-establishment of force-balance in a zero-beta plasma by
Alfv\'{e}n waves, suggests the utility of a quasi-static force-free
description of the plasma dynamics on transport time scale as opposed
to the full Alfv\'{e}n wave dynamics. 
%\QT{There are $\V$,  $\V_\perp$ and  $\V_{i\perp}$. The latter two are used without introduction. Can we explicit define them somewhere? If the latter two are the same, we should stick with one throughout the paper. Can we also say something why we care about ion density?}
 The most obvious and naive form is
\begin{align}
  \left(\nabla\times{\B}\right)\times{\B} & = 0 \label{eq:ff-default} \\
  \frac{\partial {\B}}{\partial t} & = - \nabla\times{\E} \label{eq:faraday-law} \\
  {\E} & = -{\V}\times{\B} + \frac{\eta}{\mu_0} \nabla\times{\B} \label{eq:Ohms-law}
\end{align}
with $\eta$ the plasma resistivity.
%\ZJ{(in $\SI{}{\kilo\gram\cdot\meter^{3}\cdot\ampere^{-2}\cdot\second^{-3}}$ or $\SI{}{\ohm\cdot\meter}$)}.
As shown later in~\ref{sec:nonuniqueness} as well as by many others 
(see~\cite{priest2012solar} for detailed discussions on force-free fields in astrophysics), 
 this model however is
not well posed without a proper regularization. The solution to such a
quasi-static model has long benefited from a MHD relaxation method
that Chodura and Schl\"{u}ter first introduced to find a 3D MHD
equilibrium~\cite{Chodura-Schluter-JCP-1981}.  The idea is to
introduce a fictitious drag coefficient $\epsilon$ so the
force-balance equation is regularized as
\begin{align}
\epsilon {\V} = \frac{1}{\mu_0}\left(\nabla\times{\B}\right)\times{\B} - \nabla p. \label{eq:ff-relaxation}
\end{align}
%\ZJ{There appears to be a $\frac{1}{\mu_0}$ missing before $\left(\nabla\times{\B}\right)\times{\B}$ }\\
This is to be solved in combination with the induction equation and the ideal Ohm's law
\begin{align}
  \frac{\partial {\B}}{\partial t} & = - \nabla\times{\E} \\
  {\E} & = -{\V}\times{\B}
\end{align}
and the constraint $\nabla\cdot{\B} = 0.$ Setting $\nabla p =0$ would
recover the force-free magnetic field as ${\V}$ is damped to zero by
the fictictious drag toward a steady state.  This ends up to be a
popular numerical model for studying force-free coronal magnetic
fields in solar
physics.~\cite{mikic-mcclymont-1994,McClymont-etal-SolarPhysics-1997,Roumeliotis-ApJ-1996,valori-etal-AA-2005}

Recent interests in the quasi-static MHD model for magnetic
confinement fusion have focused on the tokamak disruption modeling,
particularly the force-free evolution of a cold VDE after the initial
thermal quench. The basic physics is that once the thermal quench
drives the plasma beta to approximately zero, the vertical
force-balance of the plasma column is lost on the time scale that the
vertical position control coil current can be adjusted to provide the counter-acting vertical magnetic fields. The vertical
displacement of the entire current-carrying plasma column is driven by the
Ohmic current decay rate, the physics of which is described by a
finite $\eta$ in Eq.~(\ref{eq:Ohms-law}). This is a problem quite
amendable for a quasi-static treatment, and the MHD relaxation method
of Chodura and Schl\"{u}ter has been employed by replacing
Eq.~(\ref{eq:ff-default}) with Eq.~(\ref{eq:ff-relaxation}) in
Ref.~\cite{zakharov-li-pop-2015} and~\cite{kiramov-breizman-pop-2018}.

Here we revisit the formulation of the quasi-static force-free MHD
model and its numerical solution by implicit time
stepping. Specifically we will analyze the role of the fictitious
drag force in the regularization of the model and its physical
implication on the solution.  This will be followed by proposing an alternative
regularization of the model, which we find to have a more
straightforward physics interpretation.  In terms of spatial
discretization, a focus is on ensuring the divergence-free constraint
of the magnetic field, as failure to do so is known to spoil the
numerical solution~\cite{NicolaidesWang,JIANG1996104}.  Various 
approaches have been proposed in the literature, such as starting from
existing convergent discretizations and enhancing their numerical
accuracy by introducing appropriate divergence-free reconstructions
\cite{BALSARA20095040}, or employing divergence-free methods, like
divergence-free Discontinuous Galerkin methods \cite{COCKBURN2004588},
and stable finite element method \cite{HuMaXu}.
A particularly relevant and common approach in the compressible MHD literature is 
the constrained transport approach originally proposed in~\cite{evans1988simulation}, 
where a staggered formulation of the electric and magnetic fields is proposed 
to create specific mimetic finite difference operators that
result in a magnetic field which is divergence-free regarding a specific discrete divergence operator. 
The recent work attempts to generalize it to a non-staggered formulation such as 
finite volume~\cite{rossmanith2006unstaggered}, high-order finite difference~\cite{christlieb2014finite, christlieb2016high},
and the extension to a mapped curvilinear grid~\cite{christlieb2018high}.
There also exists abundant literature on spatial discretizations for the MHD
equations, see \cite{Schtzau2004MixedFE, CHACON2004143,
  SOVINEC2004355, SHADID20161, BADIA2013399, tang2022adaptive} and the references therein.

For the linear solver part, there has been a lot of research work
focusing on the design of efficient and scalable preconditioners for
solving linear systems stemming from MHD models.  Ref.~\cite{Cyr}
 proposes scalable block preconditioners for Newton-Krylov
solver that rely on the approximate block factorization (ABF) approach
and the recursive approximation of the Schur complement.
Ref.~\cite{Chacon2008, tang2022adaptive} follow the same line of thinking by employing the
physical-based ABF technique to devise a preconditioner for
fully-implicit Newton-Krylov solvers.  By expressing the matrix in
blocks corresponding to different unknowns, and approximating the
resulting Schur complements, they carry out a parabolization which
transforms ill-conditioned hyperbolic systems that are difficult to
solve into well-conditioned diagonally dominant parabolic operators
for which multigrid (MG) methods perform very well.  More recently, a
new family of recursive block LU preconditioners is proposed
in~\cite{BADIA2014562} for solving the thermally coupled inductionless
MHD equations, whereas in~\cite{MaHuHuXu} robust block
preconditioners, which satisfy the divergence-free condition exactly
when used in Krylov iterative methods, are developed for the
structure-preserving discretization of the incompressible MHD system.

In this paper, we use a Mimetic Finite Difference (MFD) method~\cite{LIPNIKOV20141163} for the
discretization of a quasi-static perpendicular plasma dynamics model
on a 3D structured grid.  There is a rich literature on the usage of this
method for solving diffusion
\cite{BrezziLipnikovShashkovSimoncini2007}, advection-diffusion
\cite{CangianiManziniRusso2009, BeiraodaVeigaDroniouManzini2011},
elasticity \cite{BeiraodaVeiga2010}, Stokes
\cite{BeiraodaVeigaGyryaLipnikovManzini2009} and porous media flow
problems \cite{LipnikovMoultonSvyatskiy2008}. As for the
time-integration, we use a fully implicit L-stable second order
diagonally implicit Runge--Kutta (DIRK) scheme \cite{PareschiRusso}. The preconditioner employed in
this article is a four-level block preconditioner, which is
created by combining separate preconditioners for individual fields
(as many splits as fields), that calls multigrid methods or exact
factorization on the separate fields.

%in \cite{lipnikov,manzini,shashkov} for Maxwell’s equations, equations of magnetic diffusion, and magnetostatics
%The mimetic discretization methodology has been previously used to solve Maxwell’s equations, equations of magnetic diffusion, and magnetostatics

The rest of this article is structured as follows.  Section~\ref{sec:model}
introduces the quasi-static perpendicular plasma dynamics model.
Section~\ref{sec:mfd} presents some basic elements of the mimetic
discretization methodology and presents the discrete equations of the
quasi-static model.  The numerical results of the mimetic
finite difference method applied to the quasi-static perpendicular
plasma dynamics model are shown in Section~\ref{sec:numres}.
It is followed by the conclusion section of Section~\ref{sec:conc}.
Some mathematical aspects of the models, such as well-posedness 
and the energy dissipation law, are included in the appendix.  

