{
\newcommand{\figWidth}{4cm}
\newcommand{\trimfig}[2]{\trimh{#1}{#2}{.0}{.0}{.0}{.0}}
\begin{figure}[htb]
\begin{center}
\begin{tikzpicture}[scale=1]
  \useasboundingbox (0.,1.2) rectangle (11,4.8);  % set the bounding box (so we have less surrounding white space)
  \draw(0.0,.5) node[anchor=south west,xshift=-4pt,yshift=+0pt] {\trimfig{Adiabatic}{\figWidth}};
 \node at (10.8,.7) {\small$\epsilon$};
 \node at (.3,4.6) {\small$N(\epsilon)$};
% background grid:
% \draw[step=1cm,gray] (0,0) grid (11,4.8);
\end{tikzpicture}
\end{center}
\caption{ $N(\varepsilon)$ as a function of $\varepsilon$ for $\rho = 1.1$ and a random set of weights. \label{fig: Adiabatic} }
\end{figure}
}