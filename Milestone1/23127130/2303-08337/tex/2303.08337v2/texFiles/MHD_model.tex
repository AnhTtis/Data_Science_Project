\section{Quasi-static force-free MHD model~\label{sec:model}}

\subsection{Regularized quasi-static force-free model\label{sec:regularization}}

The concept of a quasi-static force-free MHD model is based upon the
idea that any force imbalance introduced by the resistive decay of
plasma current in the Ohm's law of Eq.~(\ref{eq:faraday-law}), is
quickly removed by the Alfv\'{e}n wave dynamics, so at any given instance,
the plasma is approximately in a force-free state.  By dropping the
plasma inertia in Eq.~(\ref{eq:ff-default}), the Alfv\'{e}n wave
dynamics is deliberately removed, so maintaining a force-free
magnetic field comes from the solution of the perpendicular flow
\begin{align}
  {\V}_\perp \equiv {\V} - {\V}\cdot{\B}/|{\B}|
\end{align}
from
Eq.~(\ref{eq:Ohms-law}), subjected to the force-free constraint of
Eq.~(\ref{eq:ff-default}). The Faraday's law,
Eq.~(\ref{eq:faraday-law}), connects the constraint of
Eq.~(\ref{eq:ff-default}) to the solution of ${\V}_\perp$ from
Eq.~(\ref{eq:Ohms-law}). In other words, the Ohm's law is the equation
from which ${\V}_\perp$ is solved.  One obvious implication of this is
that since this is a time-dependent partial differential equation with
a constraint, one would need to solve the time-dependent equations with
implicit time stepping.

A more subtle implication, although well-known in the constrained
optimization problem such as saddle point problems~\cite{benzi2005numerical, bochev2005finite},
is the need for regularization. In the specific
case of the quasi-static force-free MHD model, the coupled system of
Eqs.~(\ref{eq:ff-default}, \ref{eq:faraday-law}, \ref{eq:Ohms-law})
has a null space in the solution of ${\V}_\perp.$ One can see this by
noting that if ${\V}_{\perp}$ is a solution, then ${\V}_{\perp} +
\delta{\V}_{\perp}$ is also a solution with
\begin{align}
\delta {\V}_{\perp} = - \frac{\nabla\varphi\times{\B}}{B^2} \label{eq:Delta-V}
\end{align}
for any $\varphi$ that satisfies
\begin{align}
{\B}\cdot\nabla\varphi = 0.
\end{align}
The underlying physics is that the electrostatic field, which can be
written as $-\nabla\varphi,$ does not contribute to $\nabla\times{\E}$
and hence has no effect on magnetic field evolution.  This results in
a degeneracy of the mathematical formulation that cannot be inverted
for $\V_{\perp}$ that is required in quasi-static evolution.
The formal derivation of such a null space is given in~\ref{sec:nonuniqueness}.

To gain insights into how the degeneracy can be removed,
we introduce an explicit treatment of the electrostatic potential $\Phi$
%\ZJ{(in $\SI{}{\kilo\gram\cdot\meter^{2}\cdot\ampere^{-1}\cdot\second^{-3}}$ or $\SI{}{\volt}$)}
via a Helmholtz decomposition of the electric field,
%\KL{unique when boundary conditions are specified plus $div h = 0$}   
\begin{align}
{\E} = - \nabla\Phi + \nabla\times \mathbf{h} = -\nabla\Phi + \boldsymbol{\tau}. 
\end{align}
Substituting this form of ${\E}$
%\ZJ{(in $\SI{}{\kilo\gram\cdot\meter\cdot\ampere^{-1}\cdot\second^{-3}}$ or $\SI{}{\volt\cdot\meter^{-1}}$)}
into Eq.~\eqref{eq:Ohms-law}, we find
\begin{align}
  \boldsymbol{\tau} = \nabla\Phi - {\V}_{\perp}\times {\B}
  + \frac{\eta}{\mu_0} \nabla\times{\B}. \label{eq:tau-def}
\end{align}
The Faraday's law, Eq.~\eqref{eq:faraday-law}, is now rewritten as
\begin{align}
\frac{\partial{\B}}{\partial t} =  - \nabla\times\boldsymbol{\tau}.
\end{align}
The final step is to come up with another equation to solve for $\Phi,$
which can be done in the usual way by taking the divergence of the electric field,
\begin{align}
  \nabla^2\Phi = - \nabla\cdot\left[
    - {\V}_{\perp}\times {\B} + \frac{\eta}{\mu_0} \nabla\times{\B}
    \right] \label{eq:qsp-Phi}
\end{align}
The boundary condition for $\Phi,$ in the case of ITER configuration, is simply
\begin{align}
\Phi |_{\partial\Omega} = 0, \label{eq:qsp-Phi-bdy}
\end{align}
where the boundary $\partial\Omega$ is the outer vacuum vessel wall,
which is assumed to be perfectly conducting on the time scale of a major
disruption.  So in all, we add one more unknown ($\Phi$) and one
additional equation, Eq.~\eqref{eq:qsp-Phi}, with its boundary
condition, Eq.~\eqref{eq:qsp-Phi-bdy}, to the quasi-static force-free
model.

A special null space in an axisymmetric tokamak plasma is particularly relevant. 
For an axisymmetric tokamak plasma with flux surface label $\psi,$ a
pure radial electric field
\begin{align}
{\E} = - \nabla \varphi(\psi) = -
\frac{\partial\varphi}{\partial\psi}\nabla\psi \label{eq:radial-E-field}
\end{align}
is unconstrained by Eq.~(\ref{eq:qsp-Phi}).  To see this, one can
substitute ${\V}_\perp+\delta {\V}_\perp$ with $\delta{\V}_\perp$ given in
Eq.~(\ref{eq:Delta-V}) having $\varphi=\varphi(\psi),$ for ${\V}_\perp$ in
Eq.~(\ref{eq:tau-def}) and Eq.~(\ref{eq:qsp-Phi}). The result is
\begin{align}
\boldsymbol{\tau} & = \nabla\left(\Phi - \varphi(\psi)\right) - {\V}_{\perp}\times {\B}
+ \frac{\eta}{\mu_0} \nabla\times{\B} \\
  \nabla^2\left(\Phi - \varphi(\psi)\right) & = - \nabla\cdot\left[
    - {\V}_{\perp}\times {\B} + \frac{\eta}{\mu_0} \nabla\times{\B}
    \right], 
\end{align}
which says that $\Phi-\varphi(\psi)$ is also a solution if $\Phi$ is a
solution.  This null space for the solution of ${\V}_\perp$ or
degeneracy of the force-free model with respect to a pure radial
electric field of the form in Eq.~(\ref{eq:radial-E-field}) needs to be
removed for numerical computation. The fictitious viscous drag in
Eq.~(\ref{eq:ff-relaxation}), first introduced by Chodura and
Schl\"{u}ter~\cite{Chodura-Schluter-JCP-1981}, precisely provides such
a regularization. Specifically the value of the fictitious viscous
drag coefficient $\epsilon$ picks a particular $\varphi(\psi).$ In other
words, the regularization of the quasi-static force-free model sets a
radial electrical field that is not constrained by the MHD model.
Although this peculiarity does not affect the force-free magnetic
field during the quasi-static evolution, one needs to be aware of the
regularization-induced radial electric field $\varphi(\psi).$ For
example, should one be interested in advancing the particle motion
using the electromagnetic field from this regularized quasi-static
model, the component of the pure radial electric field should be
removed as it is not {\em physically} constrained in the MHD model.

This artificialness in radial electric field motivates a more careful
look at the widely used $\epsilon {\V}$ regularization approach.  The
physical origin of the collisional damping of the flow field in single
fluid MHD, which is to be mimicked by $\epsilon {\V},$ is not collisional
friction, but the viscosity in Eq.~(\ref{eq:momentum}). This suggests that a more
physically sound regularization is to simply retain the viscosity while
ignoring the inertia, so
\begin{align}
\left(\nabla\times{\B}\right)\times{\B} = - \mu_0 \nabla\cdot\boldsymbol{\pi}.
\end{align}
For simplicity, we adopt the approximate form
\begin{align}
  \nabla\cdot\boldsymbol{\pi} = \nu \nabla^2{\V},
\end{align}
so the alternatively regularized force-free constraint is
\begin{align}
\left(\nabla\times{\B}\right)\times{\B} = - \mu_0 \nu \nabla^2{\V}.\label{eq:viscous-regularize}
\end{align}
%\ZJ{Should there not be a - sign in the previous equations?}
The formal derivation of the regularization term removing the null space is given in~\ref{sec:uniqueness}.
In addition, note that an energy dissipation law can be derived with this regularization term, 
see the derivation in~\ref{sec:magNRJ}.  
If ones desires, part of the plasma inertia, which is quadratic
in ${\V},$ can be retained as well,%\KL{why is it useful?}
\begin{align}
\left(\nabla\times{\B}\right)\times{\B} = \mu_0 \nu \nabla^2{\V} + \rho {\V}\cdot\nabla{\V}.\label{eq:viscous-inertial-reg}
\end{align}
For ${\V}$ with small amplitude, which scales with $\eta,$ the quadratic inertia term
has very little effect.
Finally, we should note that the force-free MHD model does not constrain ${\V}\cdot{\B}$ at all,
the regularized quasi-static force-free MHD model is supplemented with the constraint,
\begin{align}
{\B}\cdot{\V} = 0.
\end{align}

In summary, we consider two forms of regularized quasi-static force-free MHD model in the current work.
One uses the fictitious drag of Chodura and Schl\"{u}ter for regularization,
\begin{align}
\left(\nabla\times{\B}\right)\times{\B} = \epsilon {\V},\label{eq:drag-regularize}
\end{align}
while the other invokes the viscous damping regularization,
Eq.~(\ref{eq:viscous-regularize}) or
Eq.~(\ref{eq:viscous-inertial-reg}).  One of these regularized
force-balance
equations~(\ref{eq:viscous-regularize}, \ref{eq:viscous-inertial-reg}, \ref{eq:drag-regularize})
will be solved in tandem with
\begin{align}
  \frac{\partial{\B}}{\partial t} & = - \nabla \times \boldsymbol{\tau}, \\
  \boldsymbol{\tau} & = \nabla\Phi - {\V}\times{\B} + \frac{\eta}{\mu_0}\nabla\times{\B}, \\
  \nabla^2\Phi & = - \nabla\cdot\left[
    - {\V}\times {\B} + \frac{\eta}{\mu_0} \nabla\times{\B}
    \right], \\
  {\V}\cdot{\B} & = 0.
\end{align}






%%%%%%%%%%%%%%%%%%%%%%%%%%%%%%%%%%%%%%%%%%%%%%%%%%%%%%%%%%%%%%%%%%%%%
\subsection{Coupling to ITER blankets and vacuum vessel}
\label{sec:wall_model}

\begin{figure}[ht]
\begin{center}
\includegraphics[trim={10cm 0 1cm 0},clip, width=0.5\textwidth]{5-layer-setting.png}
\caption{ITER tokamak sub-domains in the poloidal plane. The level-set function shown in the color-map indicates different physical domains: 1 in the inner area of the plasma chamber, 2 between the separatrix and the wall, 0 at the rigid wall, -1 at the vacuum vessel and -2 outside the vacuum vessel. The above configuration is used in the simulations of the numerical section.} \label{fig:5layer}
\end{center}
\end{figure}

The plasma in the ITER tokamak reactor is enclosed by a chamber wall,
behind which are blanket modules secured on a stainless steel vacuum
vessel.  See Figure~\ref{fig:5layer} for the ITER's poloidal cross section. 
The vacuum vessel (in light blue) is continuous toroidally and poloidally, so
it is a good flux conserver with a wall time of about \SI{500}{\milli\second}. The
blanket modules (in ivory white) are attached to the vacuum vessel, and they are
constructed and arranged in such a way that a net toroidal current is
impeded. To a reasonable approximation, we will approximate the
entire vacuum vessel as toroidally symmetric conductor with a constant
resistivity, so that the wall time is \SI{500}{\milli\second}. This simplification ignores
the neutron shielding materials embedded in the vacuum vessel.
The electromagnetic field is evolved inside the vacuum vessel with the
standard Ohm's law of constant resistivity,
\begin{align}
  \frac{\partial {\B}}{\partial t} & = - \nabla\times{\E},\\
  {\E} & = \eta_{\rm vv} {\j} = \frac{\eta_{\rm vv}}{\mu_0} \nabla\times{\B}. 
\end{align}
Since the wall time depends on inductance as well, so we numerically compute the
current decay time in the vacuum vessel and match the \SI{500}{\milli\second} wall time to an effective
resistivity $\eta_{\rm vv}$ for ITER's vacuum vessel.

%\QT{Xianzhu, I moved Figure 1 here and can you use the figure to indicate the different region you are talking about? I think that would help a mathimatican reviewer}

For the first wall and blanket module section, we will deploy an anisotropic
resistivity that has the toroidal resistivity $\eta_t$ much greater than the
poloidal resistivity $\eta_p,$ so
\begin{align}
  \frac{\partial {\B}}{\partial t} & = - \nabla\times{\E}, \label{eqn:eq34} \\
       {\E} & =
  \frac{\eta_t}{\mu_0} \left(\nabla\times{\B}\right)_\phi +
  \frac{\eta_p}{\mu_0} \left[\nabla\times{\B} -
    \left(\nabla\times{\B}\right)_\phi\right]   \label{eqn:eq35}
\end{align}
with $()_\phi$ denoting the toroidal component.  The ratio of
$\eta_t$ and $\eta_p$ is chosen so that the toroidal current in the
blanket is suppressed and the halo current can flow poloidally in the
blankets to enter the vacuum vessel, where the electrical current can
have a strong toroidal component.



%%%%%%%%%%%%%%%%%%%%%%%%%%%%%%%%%%%%%%%%%%%%%%%%%%%%%%%%%%%%%%%%%%%%%
\subsection{Quasi-static perpendicular plasma dynamics model and its interface conditions}
For tokamak simulations, we consider the cylindrical coordinate of $(R,\phi,Z)$ for a direct mapping from the Cartesian coordinate $(x, y, z)$.  The structured staggered mesh under the cylindrical coordinate is  used in the current work.
The tokamak computational domain is $\Omega =  [R_{\min}, R_{\max}]\times [0, 2\pi]\times [Z_{\min}, Z_{\max}]$, which can be decomposed into two sub-domains: 
\begin{align*}
    \Omega := \Omega^{P} \cup \Omega^{W},
\end{align*}
where $\Omega^{P}$ corresponds to the tokamak’s plasma chamber whereas $\Omega^{W}$ includes the rigid wall region, the vacuum vessel and the area outside it (see Figure~\ref{fig:5layer} for details: $\Omega^{P}$ comprises the areas where the level-set function is positive, $\Omega^{W}$ corresponds to non-positive values of the level-set function). We use $\Gamma^{PW}$ to denote the interface between the two subdomains $\Omega^{P}$ and $\Omega^{W}$.
Note that in $\Omega^{W}$, there is no plasma and thus the plasma density ($n$) and velocity equations become
\begin{align*}
\frac{\partial n}{\partial t} = & ~ 0 \qquad \text{in} \quad \Omega^{W}, \\
  {\V}_\perp = & ~ \mathbf{0} \qquad \text{in} \quad \Omega^{W},
\end{align*}
while the fields satisfy the diffusion equation as discussed in Section~\ref{sec:wall_model}.

In the sequel, we consider the model introduced in
Section~\ref{sec:regularization} for the plasma region $\Omega^P$ with
some slight adjustments:
\begin{itemize}
    \item When compared to the perpendicular component, the parallel velocity component can be considered as negligible. Therefore, it is neglected in the density equation and not solved in the velocity equation.  
    \item For stabilization purposes, a viscosity term is added to the velocity equations in $R$ and $Z$ directions (see~\ref{sec:nonuniqueness}, \ref{sec:uniqueness} and~\ref{sec:magNRJ} for some discussions on its impact).
    \item The resistivity is assumed to be a constant in each sub-domain. 
   % \item The temperatures are not being solved which reduces the number of unknowns. 
%    \item The temperature gradient, the electron viscosity term and the Hall term ($(\nabla \times \mathbf{B}) \times \mathbf{B}$) are dropped from the generalized Ohm's law and the electrostatic potential equation. For these two equations, we also neglect for simplicity the thermal force contribution in the collisional drag, and further assume
%\begin{equation}
%\mathbf{R}_u \approx - \frac{m_e n_e}{\tau_e} \mathbf{u} = \frac{m_e}{e \tau_e} \mathbf{j}.
%\end{equation}
%Note that $\mathbf{R}_{ei}$ appears in conjunction with a factor of $\frac{1}{e n_e}$, leading to a resistive contribution in the magnetic field equation of the form
%\begin{equation}
%\frac{1}{e n_e} \mathbf{R}_{ei} = \frac{m_e}{e^2 n_e \tau_e} \mathbf{j} \eqqcolon \eta_0 \mathbf{j},
%\end{equation}
%for resistivity coefficient
%\begin{equation}
%\eta_0 \propto T^{-3/2},
%\end{equation}
%that can vary over space which we may for simplicity also instead consider to be constant. 
%The collisional drag is replaced by a resistive (MHD) term: $\mu_0 \nabla \times \mathbf{B}$ where $\mu_0$ is the resistivity which varies over space. 
\end{itemize}

An interface condition is needed for such a multi-domain interface problem.
For the fields, the following jump conditions should be naturally satisfied,   
\begin{alignat}{3}
  &[{\B} \cdot \mathbf{n}]=0 \qquad  &\text{on} \quad \Gamma^{PW} , \label{eqn:edge-face-continuity1} \\
  &[{\E} \times \mathbf{n}]=\mathbf{0}  \qquad & \text{on} \quad \Gamma^{PW}, \label{eqn:edge-face-continuity2} 
\\
  &\V_{i\perp} = \mathbf{0} \qquad & \text{on} \quad \Gamma^{PW}, \label{eqn:vertex-continuity2}
\end{alignat}
where $[\cdot]$ stands for the jump operator along the interface. 
These conditions are consistent with the absence of surface charge/current and the fact that there is no jump in the electrical field.

\begin{comment}
The remaining piece is to find a proper interface condition for velocity and possibly a more constrained interface condition on the fields. 
For that purpose, it is helpful to revisit the formal definition of curl, which is
\[
(\nabla \times \mathbf{F})\cdot \mathbf{n}\equiv \lim_{A\rightarrow 0} \frac 1 {|A|} \oint_C \mathbf{F}\cdot d\mathbf{r}
\]
where the line integral is calculated along the boundary $C$ of the area $A$ whose normal vector is $\mathbf{n}$.
In fact, this limit can be taken through restricting $A$ as a small surface on the interface $\Gamma$ (using Stokes' theorem). 
The definition of the curl leads to the following lemma
\begin{lemma} \label{lem:curljump}
On a smooth surface $\Gamma$ and $ \mathbf{n}_{1\rightarrow2}$ is a normal vector from $\Omega_1$ to $\Omega_2$, if a vector field satisfies $[\mathbf{F} \times \mathbf{n}_{1\rightarrow2}] = \mathbf{0}$, then 
\[
[(\nabla \times \mathbf{F})\cdot \mathbf{n}_{1\rightarrow2}]=0.
\]
\end{lemma}
\noindent We consider three potential options for the interface condition that satisfy equations~\eqref{eqn:edge-face-continuity1} and~\eqref{eqn:edge-face-continuity2}: 
\begin{enumerate}[label=(\roman*)]
\item \begin{alignat}{3}
  &[{\B} ]= \mathbf{0} \qquad  &\text{on} \quad \Gamma^{PW}  , \nonumber \\
  &[{\E}]= \mathbf{0}  \qquad & \text{on} \quad \Gamma^{PW}  , \nonumber \\
  &{\V}_\perp= \mathbf{0}  \qquad & \text{on} \quad \Gamma^{PW}. \nonumber
\end{alignat}
\item \begin{alignat}{3}
  &[{\B} ]= \mathbf{0} \qquad  &\text{on} \quad \Gamma^{PW}  , \nonumber \\
  &[{\E}\times\mathbf{n}_{P\rightarrow W}]= \mathbf{0}  \qquad & \text{on} \quad \Gamma^{PW}, \nonumber\\
  &\V_{i\perp}\cdot \B = 0 \qquad & \text{on} \quad \Gamma^{PW}. \nonumber 
\end{alignat}
\item \begin{alignat}{3}
  &[{\B}\cdot\mathbf{n} _{P\rightarrow W}]=0 \qquad  &\text{on} \quad \Gamma^{PW} , \nonumber \\
  &[{\E}\times\mathbf{n}_{P\rightarrow W}]= \mathbf{0}  \qquad & \text{on} \quad \Gamma^{PW} , \nonumber \\
  &\V_{i\perp} = \mathbf{0} \qquad & \text{on} \quad \Gamma^{PW}. \nonumber
\end{alignat}
\end{enumerate}
Further analyses are needed to choose the proper interface condition. For the purpose of analysis, we consider the following simplified model. In the plasma region, we have
\begin{alignat*}{3}
  (\nabla\times{\B})\times{\B}  = & ~\epsilon {\V}_\perp \qquad &\text{in} \quad \Omega^{P} , \\
  {\V}_\perp \cdot {\B} = & ~ 0 \qquad &\text{in} \quad \Omega^{P},  \\
  \frac{\partial{\B}}{\partial t} = & - \nabla\times{\E} \qquad &\text{in} \quad \Omega^{P},  \\
  {\E} = & - {\V}_\perp\times {\B} + \frac{\eta_P}{\mu_0}\left(\nabla\times{\B}\right) \qquad &\text{in} \quad \Omega^{P}. 
  \end{alignat*}
In the wall region, we have
\begin{alignat*}{3}
  {\V}_\perp = & ~ \mathbf{0} \qquad &\text{in} \quad \Omega^{W} , \\
  \frac{\partial{\B}}{\partial t} = & - \nabla\times{\E} \qquad &\text{in} \quad \Omega^{W} ,   \\
    {\E} = & ~\frac{\eta_W}{\mu_0}\left(\nabla\times{\B}\right) \qquad &\text{in} \quad \Omega^{W},  
\end{alignat*} 
with the resistivity $\eta_P\neq\eta_W$.

With option (i), we have ${\V}_\perp = \mathbf{0}$ on $\Gamma^{PW}$, thus 
\begin{equation}
    \E_P  =  \frac{\eta_P}{\mu_0}\left(\nabla\times{\B_P}\right) = \E_W  =  \frac{\eta_W}{\mu_0}\left(\nabla\times{\B_W}\right) \qquad  \text{on} \quad \Gamma^{PW}.  
\end{equation}
However, Lemma~\ref{lem:curljump} indicates that
\[
[(\nabla \times \mathbf{B})\cdot \mathbf{n}_{P\rightarrow W}]=0.
\]
Since $\eta_P\neq\eta_W$, the interface condition is possible if and only if
\[
(\nabla \times \mathbf{B}_P)\cdot \mathbf{n}_{P\rightarrow W}= (\nabla \times \mathbf{B}_W)\cdot \mathbf{n}_{P\rightarrow W}=0.
\]
%Therefore, the interface condition indicates $\E_P= \E_W$ and $\B_P = \B_W$, and both of them are perpendicular to $\mathbf{n}$. 
In other words, it leads to the following interface condition:
\begin{align}
%&\E_P\cdot \mathbf{n}_{P\rightarrow W}= \E_W\cdot \mathbf{n}_{P\rightarrow W} =  \B_P\cdot \mathbf{n}_{P\rightarrow W} =  \B_W\cdot \mathbf{n}_{P\rightarrow W}=0, \\
&\E_P\cdot \mathbf{n}_{P\rightarrow W}= \E_W\cdot \mathbf{n}_{P\rightarrow W} =0, \\
&[\E\times \mathbf{n}_{P\rightarrow W}] = \mathbf{0},\\
&[\B] = \mathbf{0},\\
&\V_{i\perp} = \mathbf{0}.
\end{align}
This appears to be over-determined.\KL{I am confused, conditions (i) imply conditions (44)-(47).}

 With option (ii), we have :
\begin{equation}
{\Big [}\dfrac{\eta}{\mu_0} (\nabla \times {\B}){\Big ]}  = {\big [} {\E} {\big ]}  + {\V}_{i\perp,P} \times {\B}_P.
\end{equation}
By computing the dot and cross products with $\mathbf{n}_{P\rightarrow W}$, we obtain: 
\begin{align}
\label{eqn:nonzeroetacurlBxn}
{\Big [}\dfrac{\eta}{\mu_0} (\nabla \times {\B}) \times \mathbf{n}_{P\rightarrow W}{\Big ]} &= {\Big [} {\E} \times \mathbf{n}_{P\rightarrow W} {\Big ]}  + \left( {\V}_{i\perp,P} \times {\B}_P\right) \times \mathbf{n}_{P\rightarrow W} = \left( {\V}_{i\perp,P} \times {\B}_P\right) \times \mathbf{n}_{P\rightarrow W}\\
{\Big [}\dfrac{\eta}{\mu_0} (\nabla \times {\B}) \cdot \mathbf{n}_{P\rightarrow W}{\Big ]} &= {\Big [} {\E} \cdot \mathbf{n}_{P\rightarrow W} {\Big ]} + \left( {\V}_{i\perp,P} \times {\B}_P\right) \cdot \mathbf{n}_{P\rightarrow W}
\end{align}
Since ${\big [} {\B} {\big ]}= \mathbf{0}$, then we can again apply Lemma~\eqref{lem:curljump} on ${\B}$ to get 
\begin{equation}
{\Big [}(\nabla \times \mathbf{B})\cdot \mathbf{n}_{P\rightarrow W}{\Big ]} = {0}.
\label{eqn:zerocurlB.n}
\end{equation}
We can also take a derivate in time to obtain 
\begin{equation}
- \dfrac{\partial}{\partial t} \left( {\big [} {\B} {\big ]}  \right) = {\Big [} - \dfrac{\partial {\B} }{\partial t} {\Big ]}  =
{\Big [} \nabla \times {\E} {\Big ]}   = \mathbf{0}.
\label{eqn:zerocurlE}
\end{equation} 


\KL{No conclusion for option (ii).} \ZJ{The conclusion for option (ii) was drawn later along with the conclusion for option (iii).}
 With option (iii), we have:
\begin{equation*}
{\Big [} \nabla \times {\E}{\Big ]}=- \dfrac{\partial}{\partial t} \left( {\big [} {\B} {\big ]} \right) ~ ; ~~
{\Big [} \nabla \times {\B}{\Big ]} = {\big [} \dfrac{\mu_0}{\eta} {\E} {\big ]} ~ ; ~~ {\Big [}\dfrac{\eta}{\mu_0} (\nabla \times {\B}){\Big ]}= {\big [} {\E} {\big ]}.
\end{equation*} 
Thus, the dot and cross products with $\mathbf{n}_{P\rightarrow W}$ yield: 
\begin{align}
{\Big [} (\nabla \times {\E}) \times \mathbf{n}_{P\rightarrow W} {\Big ]}&=- \dfrac{\partial}{\partial t} \left( {\big [} {\B} \times \mathbf{n}_{P\rightarrow W} {\big ]} \right) , \\
\label{eqn:zerocurlE.n}
{\Big [} (\nabla \times {\E}) \cdot \mathbf{n}_{P\rightarrow W} {\Big ]}&=- \dfrac{\partial}{\partial t} \left( {\big [} {\B}\cdot \mathbf{n}_{P\rightarrow W} {\big ]} \right) = {0} , \\
{\Big [} (\nabla \times {\B}) \times \mathbf{n}_{P\rightarrow W}{\Big ]} &= {\Big [} \dfrac{\mu_0}{\eta} {\E} \times \mathbf{n}_{P\rightarrow W} {\Big ]} ,\\
{\Big [} (\nabla \times {\B}) \cdot \mathbf{n}_{P\rightarrow W}{\Big ]}&= {\Big [} \dfrac{\mu_0}{\eta} {\E} \cdot \mathbf{n}_{P\rightarrow W} {\Big ]}, \\
\label{eqn:zeroetacurlBxn}
{\Big [}\dfrac{\eta}{\mu_0} (\nabla \times {\B}) \times \mathbf{n}_{P\rightarrow W}{\Big ]} &= {\Big [} {\E} \times \mathbf{n}_{P\rightarrow W} {\Big ]} = \mathbf{0} ,\\
{\Big [}\dfrac{\eta}{\mu_0} (\nabla \times {\B}) \cdot \mathbf{n}_{P\rightarrow W}{\Big ]} &= {\Big [} {\E} \cdot \mathbf{n}_{P\rightarrow W} {\Big ]}.
\end{align}

 It is noteworthy to mention that Equation~\eqref{eqn:zeroetacurlBxn} of option (iii) appears to be compatible with the MFD discretization where there is no jump in $\dfrac{\eta}{\mu_0} (\nabla \times {\B}) \times \mathbf{n}_{P\rightarrow W}$ across the interface (more precisely, this term is edge-based because it is discretized using the derived mimetic curl operator whose codomain is $\mathcal{E}_h$) whereas Equation~\eqref{eqn:nonzeroetacurlBxn} seems to indicate the opposite in option (ii), i.e. there is a jump in $\dfrac{\eta}{\mu_0} (\nabla \times {\B}) \times \mathbf{n}_{P\rightarrow W}$. 
 Also, note that interface conditions of option (ii) yield Equations~\eqref{eqn:zerocurlB.n} and~\eqref{eqn:zerocurlE} that count as 4 scalar continuity equations at the interface whereas those of option (iii) yield Equations~\eqref{eqn:zerocurlE.n} and~\eqref{eqn:zeroetacurlBxn} which count as 3 scalar equations. For the reasons mentioned above, we decide to retain option (iii) for the interface conditions.
\KL{Is there a potential interest in the above discussion on interface conditions in the community? Also, what is the point to consider the model problem
where two equations are well studied, so are the interface conditions ifor them and the only question is about $V_\perp$? It would be more beneficial to
discuss $[\tau \cdot n] =0$ vs $[\tau \times n]=0$}
\end{comment}

 Finally, the multi-domain quasi-static plasma model along with its interface condition that is considered in the current work is summarized as follows: 
\begin{itemize}
\item In the plasma region:
\begin{align}
  \frac{\partial n}{\partial t} + \nabla\cdot\left(n{\V}_\perp\right) = & ~0 \qquad \text{in} \quad  \Omega^{P}, \label{eq:4.565} \\
  -\left( \nu n_0 m_i \nabla^2 \V_{i\perp} + \dfrac{1}{\mu_0} (\nabla\times{\B})\times{\B} \right) \cdot \mathbf{e}_R= & ~0 \qquad \text{in} \quad \Omega^{P} , \label{eq:4.566} \\
  {\V}_\perp \cdot {\B} = & ~ 0 \qquad \text{in} \quad \Omega^{P}, \label{eq:4.567} \\
  -\left( \nu n_0 m_i \nabla^2 \V_{i\perp} + \dfrac{1}{\mu_0} (\nabla\times{\B})\times{\B} \right) \cdot \mathbf{e}_Z = & ~ 0 \qquad \text{in} \quad \Omega^{P}, \label{eq:4.568} \\
  - \nabla^2 \Phi - \nabla\cdot\left[ - {\V}_\perp\times {\B} + \frac{\eta}{\mu_0}\left(\nabla\times{\B}\right) \right] = & ~ 0 \qquad \text{in} \quad \Omega^{P}, \label{eq:4.569} \\
  \boldsymbol{\tau} - \nabla\Phi + {\V}_\perp\times {\B} - \frac{\eta}{\mu_0}\left(\nabla\times{\B}\right) = & ~ \mathbf{0} \qquad \text{in} \quad \Omega^{P}, \label{eq:4.570} \\
  \frac{\partial{\B}}{\partial t} + \nabla\times\boldsymbol{\tau} = & ~ \mathbf{0} \qquad \text{in} \quad \Omega^{P}. \label{eq:4.571}
\end{align}
\item In the wall region:
\begin{align}
\frac{\partial n}{\partial t} = & ~ 0 \qquad \text{in} \quad \Omega^{W}, \label{eq:4.572} \\
  {\V}_\perp = & ~ \mathbf{0} \qquad \text{in} \quad \Omega^{W} , \label{eq:4.573} \\
  - \nabla^2 \Phi - \nabla\cdot\left[ \frac{\eta}{\mu_0}\left(\nabla\times{\B}\right) \right] = & ~ 0 \qquad \text{in} \quad \Omega^{W} \label{eq:4.574} \\
  \boldsymbol{\tau} - \nabla\Phi - \frac{\eta}{\mu_0}\left(\nabla\times{\B}\right) = & ~ \mathbf{0} \qquad \text{in} \quad \Omega^{W}, \label{eq:4.575} \\
  \frac{\partial{\B}}{\partial t} + \nabla\times\boldsymbol{\tau} = & ~ \mathbf{0} \qquad \text{in} \quad \Omega^{W} . \label{eq:4.576}
\end{align}
\item At the wall/plasma interface:
\begin{alignat}{3}
%&n \qquad &&  \text{continous across} \quad  \Gamma^{PW} ,\label{eq:4.577} \\
  &{\V}_\perp = ~ 0 \qquad &&  \text{on} \quad  \Gamma^{PW} ,\label{eq:4.577} \\
%  & \Phi \qquad &&  \text{continous across} \quad  \Gamma^{PW} ,\label{eq:phicontnty} \\
  &\boldsymbol{\tau} \times \mathbf{n} \qquad  &&  \text{continous across} \quad \Gamma^{PW},  \label{eq:4.579} \\
  &{\B} \cdot \mathbf{n} \qquad && \text{continous across} \quad \Gamma^{PW} . \label{eq:4.578} 
\end{alignat}
\item At the outer rectangular boundary $\partial \Omega$:
\begin{alignat}{3}
   %n & = ~ n_{0}, \label{eq:4.580n} \\
   %{\V}_{I\perp} & = ~ \mathbf{0}, \label{eq:4.580V}\\
   \Phi & = ~ {0},  \label{eq:4.580} \\
   \boldsymbol{\tau} \times \mathbf{n} & = ~ \mathbf{0} .  \label{eq:4.580tau}  %\\
   %\B & = ~ \B_{0} \KL{\mbox{Is $B_0\times n$ used in the scheme}},
   %\dfrac{\partial \B}{\partial t} & = & ~ \mathbf{0} \qquad & \text{on} \quad \partial \Omega.\label{eq:4.580B}
\end{alignat}
\end{itemize}
%where the subscript $_0$ stands for the initial value on the boundary. 
Note that Equations~\eqref{eq:4.565} and ~\eqref{eq:4.572} express the plasma density continuity equation and that the field of $\boldsymbol{\tau}$ and $\Phi$ are considered instead of the electric field due to the reason described in Section~\ref{sec:regularization}.
