\def\hh{\hspace*{-3mm}}
\def\DIV{{\mathrm{Div}}_h}
\def\DIVt{\widetilde{\mathrm{Div}}_h}
\def\GRAD{{\mathrm{Grad}}_h}
\def\GRADt{\widetilde{\mathrm{Grad}}_h}
\def\CURL{{\mathrm{Curl}}_h}
\def\CURLt{\widetilde{\mathrm{Curl}}_h}

\section{Mimetic finite difference discretization}
\label{sec:mfd}
This section focuses on the MFD formulation and the full multi-domain quasi-static plasma model.
We consider a structured orthogonal mesh in 3D with hexahedral cells $c$ that form a subdivision of the computation domain. Let $|c|$, $|f|$ and $|e|$ be the volume of cell $c$, the area of face $f$ and the length of edges $e$, respectively.  
We use $\mathcal{N}_h$, $\mathcal{E}_h$, $\mathcal{F}_h$ and
$\mathcal{C}_h$ to denote the discrete node, edge, face and cell spaces,
respectively.  The MFD employed here is a staggered mesh method; the
velocity and electrostatic potential unknowns are defined at mesh
vertices ($\mathbf{V}_h, \Phi_h \in \mathcal{N}_h$), the electric
field unknowns are defined on mesh edges ($\boldsymbol{\tau}_h \in
\mathcal{E}_h$), the magnetic field unknowns are defined on mesh faces
($\mathbf{B}_h \in \mathcal{F}_h$) while the ion number density
unknowns are defined on mesh elements (${n_i}_h \in \mathcal{C}_h$).
We first introduce the primary and dual mimetic operators acting between the discrete spaces:
$$
\begin{array}{cccccccc}
\mathcal{N}_h & \longrightarrow & \mathcal{E}_h & \longrightarrow & \mathcal{F}_h & \longrightarrow & \mathcal{C}_h\\[-0.5ex]
& \hh\GRAD\hh && \hh\CURL\hh && \hh\DIV\hh & \\[1.8ex]
\mathcal{N}_h & \longleftarrow & \mathcal{E}_h & \longleftarrow & \mathcal{F}_h & \longleftarrow & \mathcal{C}_h \\[-0.5ex]
& \hh\DIVt\hh && \hh\CURLt\hh && \hh\GRADt\hh &
\end{array}
$$
and then use them to discretize
the quasi-static perpendicular plasma dynamics model.

\subsection{Primary and derived mimetic operators}
The MFD framework approximates  first-order operators using
coordinate-invariant formulas. We write the Stokes theorem for a finite-size
mesh object, either cell $c$ or face $f$ or edge $e$. For instance,
$$
  \frac{1}{|c|} \int_c \nabla\cdot \mathbf{B} \, {\rm d} V
  = \frac{1}{|c|} \oint_{\partial c} \mathbf{B} \cdot \mathbf{n}\, {\rm d} S,
$$
where $\mathbf{n}$ is the unit normal vector to the surface $\partial c$. 
This gives us the following definition of the primary mimetic divergence operator
$$
{\mathrm{Div}_h} \mathbf{B}_h := \frac{1}{|c|} \sum\limits_{f \in \partial c} \alpha_{c,f} |f|\, B_f,
  \qquad
  \text{where } B_f := \frac{1}{|f|} \int_f \mathbf{B}\cdot \mathbf{n}\, {\rm d} S.
$$
%Note that the discrete divergence operator is \underline{exact} for any field $\mathbf{B}$.
Here $|c|$ is the cell volume that has different formula in different coordinate systems.
Similarly, $|f|$ is the face area. $\alpha_{c,f}$ is the orientation factor, $\pm 1$.
% that also has different formula in different coordinate  systems. 

%For instance, the area in 2D $(R, Z)$-coordinates is calculated using the ``Cartesian'' 
%area:
%$$
%  |f|^{RZ} = |f|^{XY} \, r_f,
%$$
%where $r_f$ is the r-coordinate of the face centroid.

The discrete gradient and curl operators are defined in a similar fashion:
$$
  {\mathrm{Grad}_h} v_h := \frac{1}{|e|} \int_e \nabla v\, {\rm d} L
   = \frac{1}{|e|} (v_{2} - v_{1}),
$$
where vertices $v_1$ and $v_2$ are the endpoints of $e$
and
$$
  {\mathrm{Curl}_h} \mathbf{E}_h := \frac{1}{|f|} \int_f \nabla \times \mathbf{E}\, {\rm d} S
   = \frac{1}{|f|} \sum\limits_{e \in \partial f} \alpha_{f,e}\, |e|\, E_e,
  \qquad
  \text{with } E_e := \frac{1}{|e|} \int_e \mathbf{E}\cdot \boldsymbol{\tau}\, {\rm d} L,
$$
where $\boldsymbol{\tau}$ is the unit vector tangent to an edge $e$ and $\alpha_{f,e}$ is the orientation factor, $\pm 1$.
The three primary mimetic operators satisfy the following discrete identities:
\begin{equation}
  {\mathrm{Div}_h}\, {\mathrm{Curl}_h} = 0 
  \quad\mbox{and}\quad
  {\mathrm{Curl}_h}\, {\mathrm{Grad}_h} = 0.
  \label{eqn:prim_disc_ident}
\end{equation}


A set of dual operators can be defined by discrete analogs of integration by parts.
The dual operators preserve the duality property by design.
To simplify the presentation, we consider the Green's formulas in functional spaces 
where the boundary integrals are zeros.
An dual (injective) operator $\widetilde{\mathrm{Div}_h} : \mathcal{E}_h \rightarrow \mathcal{N}_h$ is (uniquely) defined via the 
discrete duality relationship:
$$
  [{\mathrm{Grad}_h}\, p_h,\, \mathbf{v}_h]_{\mathcal{E}_h} = -[p_h,\, \widetilde{\mathrm{Div}_h}\,\mathbf{v}_h]_{\mathcal{N}_h} ,
$$
where the brackets denote inner products 
in the aforementioned discrete spaces. 
%The inner products are accurate approximations of integrals.
%Their derivation is at the heart of the mimetic framework.
The other dual operators $\widetilde{\mathrm{Grad}_h} : \mathcal{C}_h \rightarrow \mathcal{F}_h$ and $\widetilde{\mathrm{Curl}_h} : \mathcal{F}_h \rightarrow \mathcal{E}_h$ are defined similarly:
\begin{align*}
  [{\mathrm{Div}_h}\, \mathbf{u}_h ,\, p_h]_{\mathcal{C}_h} & = -[\mathbf{u}_h,\, \widetilde{\mathrm{Grad}_h}\,p_h]_{\mathcal{F}_h}, \\
  [{\mathrm{Curl}_h}\, \mathbf{u}_h,\, \mathbf{v}_h]_{\mathcal{F}_h} &= [\mathbf{u}_h,\, \widetilde{\mathrm{Curl}_h}\,\mathbf{v}_h]_{\mathcal{E}_h}.
\end{align*}
The dual operators also satisfy discrete identities:
$$
  \widetilde{\mathrm{Div}_h}\, \widetilde{\mathrm{Curl}_h} = 0 
  \quad\mbox{and}\quad
  \widetilde{\mathrm{Curl}_h}\, \widetilde{\mathrm{Grad}_h} = 0.
$$
%In this case, the dual operator has a local stencil. 
%On a general mesh, the mass matrices are often irreducible and their inverses are dense matrices.
%Hence, the dual operator has global stencil, which requires some care in numerical schemes. 

\begin{comment}
\KL{Do we need this elliptic example? Citations and summary of properties in the introduction may be enough.}
Consider for example the elliptic problem $-\nabla\cdot (K \nabla p) = f$
with homogeneous boundary conditions.
The discrete equation is 
$$
  -\widetilde{\mathrm{Div}_h}\,{\mathrm{Grad}_h}\, p_h = f_h.
$$
Using the matrix form of the dual operator, we have
$$
  \mathbb{M}_n^{-1}\,({\mathrm{Grad}_h})^T\, \mathbb{M}_e \,{\mathrm{Grad}_h}\, p_h = f_h.
$$
The form acceptable for numerical treatment is obtained by multiplying both sides
of this equation by $\mathbb{M}_e$. Note that this form leads to a symmetric
positive definite matrix.
The above discrete equation uses $p_h$ defined at the nodes.
Another possible discretization of the same equation is $-{\mathrm{Div}_h}\,\widetilde{\mathrm{Grad}_h}\, p_h = f_h$, where $p_h$ is defined at the cells.
\end{comment}


%The first-order operators satisfy the duality relations that are given
%by various Green's formulas, such as:
%\begin{eqnarray*}
% && \int_\Omega p\, \nabla \cdot \mathbf{u} \,{\rm d}V 
%  = -\int_\Omega (\nabla p) \cdot \mathbf{u} \,{\rm V} 
%  + \oint_{\partial \Omega} p\,(\mathbf{u}\cdot \mathbf{n}) \,{\rm d}S.\\
%%
% &&
%  \int_\Omega \mathbf{u} \cdot (\nabla \times \mathbf{v}) \,{\rm d}V 
%  = \int_\Omega (\nabla \times  \mathbf{u}) \cdot \mathbf{v} \,{\rm d}V 
%  + \oint_{\partial \Omega} (\mathbf{u}\times \mathbf{v}) \cdot \mathbf{n} \,{\rm d}S.
%\end{eqnarray*}








\subsection{MFD discretization of the quasi-static perpendicular plasma dynamics model}
\label{sec:mfd_discretization}
To discretize the model above, we first need to define some projections and reconstructions between the spaces $\mathcal{N}_h$, $\mathcal{E}_h$, $\mathcal{F}_h$ and $\mathcal{C}_h$. 
For vectors that are discrete representations of continuous functions, integer indices, e.g., $(i,j,k)$, in the three directions correspond to values associated with vertices; integer indices in two directions, e.g. $(i,j+\frac{1}{2},k)$ correspond to values associated with edges; integer indices in one direction, e.g. $(i+\frac{1}{2},j,k+\frac{1}{2})$ correspond to values associated with faces; whereas non-integer indices in all directions, e.g. $(i+\frac{1}{2},j+\frac{1}{2},k+\frac{1}{2})$ correspond to values associated with cells. 

 For any vertex-based vector $\mathbf{W}$, face-based vector $\mathbf{X}$, cell-based vector $\mathbf{Y}$ and edge-based vector $\mathbf{Z}$, we define the cell-to-face projection ${\mathcal{P}}_{\mathrm c\rightarrow f}$ by:
\begin{align}
  {\mathcal{P}}_{\mathrm{c\rightarrow f}}(\mathbf{Y})_{i+\frac{1}{2},j+\frac{1}{2},k} = & ~ \frac{1}{2}\left((\mathbf{Y}_{i+\frac{1}{2},j+\frac{1}{2},k+\frac{1}{2}} + \mathbf{Y}_{i+\frac{1}{2},j+\frac{1}{2},k-\frac{1}{2}}) \cdot \mathbf{e}_Z \right) \mathbf{e}_Z, \label{eq:face-av-cell_1} \\
  {\mathcal{P}}_{\mathrm{c\rightarrow f}}(\mathbf{Y})_{i+\frac{1}{2},j,k+\frac{1}{2}} = & ~ \frac{1}{2}\left((\mathbf{Y}_{i+\frac{1}{2},j+\frac{1}{2},k+\frac{1}{2}} + \mathbf{Y}_{i+\frac{1}{2},j-\frac{1}{2},k+\frac{1}{2}})\cdot \mathbf{e}_{\phi} \right) \mathbf{e}_{\phi}, \label{eq:face-av-cell_2} \\
  {\mathcal{P}}_{\mathrm{c\rightarrow f}}(\mathbf{Y})_{i,j+\frac{1}{2},k+\frac{1}{2}} = & ~ \frac{1}{2}\left((\mathbf{Y}_{i+\frac{1}{2},j+\frac{1}{2},k+\frac{1}{2}} + \mathbf{Y}_{i-\frac{1}{2},j+\frac{1}{2},k+\frac{1}{2}}) \cdot \mathbf{e}_R \right) \mathbf{e}_R, \label{eq:face-av-cell_3}
\end{align}
%the cell-to-vertex projection ${\mathcal{P}}_{\mathrm{c\rightarrow v}}$ by:
%\begin{align}
%{\mathcal{P}}_{\mathrm{c\rightarrow v}}(\mathbf{Y})_{i,j,k} = & ~ ( \mathbf{Y}_{i+\frac{1}{2},j+\frac{1}{2},k+\frac{1}{2}} + \mathbf{Y}_{i-\frac{1}{2},j+\frac{1}{2},k+\frac{1}{2}} + \mathbf{Y}_{i+\frac{1}{2},j-\frac{1}{2},k+\frac{1}{2}} + \mathbf{Y}_{i-\frac{1}{2},j-\frac{1}{2},k+\frac{1}{2}} \nonumber \\ 
%& + \mathbf{Y}_{i-\frac{1}{2},j+\frac{1}{2},k-\frac{1}{2}} + \mathbf{Y}_{i+\frac{1}{2},j-\frac{1}{2},k-\frac{1}{2}} + \mathbf{Y}_{i-\frac{1}{2},j-\frac{1}{2},k-\frac{1}{2}} + \mathbf{Y}_{i+\frac{1}{2},j+\frac{1}{2},k-\frac{1}{2}})/8, \label{eq:vertex-av-cell}
%\end{align}
the edge-to-vertex projection ${\mathcal{P}}_{\mathrm{e\rightarrow v}}$ by:
\begin{align}
  {\mathcal{P}}_{\mathrm{e\rightarrow v}}(\mathbf{Z})_{i,j,k} = ~ & \dfrac{1}{2} \left( ( \mathbf{Z}_{i+\frac{1}{2},j,k} + \mathbf{Z}_{i-\frac{1}{2},j,k} ) \cdot \mathbf{e}_{R} \right)\mathbf{e}_{R} + \dfrac{1}{2} \left( (\mathbf{Z}_{i,j-\frac{1}{2},k} + \mathbf{Z}_{i,j+\frac{1}{2},k} ) \cdot   \mathbf{e}_{\phi} \right) \mathbf{e}_{\phi} \nonumber \\
  + ~ & \dfrac{1}{2} \left( (\mathbf{Z}_{i,j,k-\frac{1}{2}} + \mathbf{Z}_{i,j,k+\frac{1}{2}} ) \cdot \mathbf{e}_{Z} \right) \mathbf{e}_{Z}, \label{eq:vertex-av-edge}
\end{align}
the face-to-vertex projection ${\mathcal{P}}_{\mathrm{f\rightarrow v}}$ by:
\begin{align}
  {\mathcal{P}}_{\mathrm{f\rightarrow v}}(\mathbf{X})_{i,j,k} = ~ & \dfrac{1}{4} \left( ( \mathbf{X}_{i,j+\frac{1}{2},k+\frac{1}{2}} + \mathbf{X}_{i,j-\frac{1}{2},k+\frac{1}{2}} + \mathbf{X}_{i,j+\frac{1}{2},k-\frac{1}{2}} + \mathbf{X}_{i,j-\frac{1}{2},k-\frac{1}{2}} ) \cdot \mathbf{e}_{R} \right)\mathbf{e}_{R} \nonumber \\
  + ~ & \dfrac{1}{4} \left( (\mathbf{X}_{i-\frac{1}{2},j,k-\frac{1}{2}} +   \mathbf{X}_{i+\frac{1}{2},j,k-\frac{1}{2}} + \mathbf{X}_{i-\frac{1}{2},j,k+\frac{1}{2}} + \mathbf{X}_{i+\frac{1}{2},j,k+\frac{1}{2}}) \cdot   \mathbf{e}_{\phi} \right) \mathbf{e}_{\phi} \nonumber \\
  + ~ & \dfrac{1}{4} \left( (\mathbf{X}_{i-\frac{1}{2},j-\frac{1}{2},k} + \mathbf{X}_{i+\frac{1}{2},j-\frac{1}{2},k} + \mathbf{X}_{i-\frac{1}{2},j+\frac{1}{2},k} +   \mathbf{X}_{i+\frac{1}{2},j+\frac{1}{2},k}) \cdot \mathbf{e}_{Z} \right) \mathbf{e}_{Z}, \label{eq:vertex-av-face}
\end{align}
the vertex-to-edge reconstruction ${\mathcal{R}}_{\mathrm{v\rightarrow e}}$ by:
\begin{align}
  {\mathcal{R}}_{\mathrm{v\rightarrow e}}(\mathbf{W})_{i+\frac{1}{2},j,k} = & ~ \frac{1}{2} ~ {\Big (} \mathbf{e}_R \cdot (\mathbf{W}_{i+1,j,k} + \mathbf{W}_{i,j,k}) {\Big )} \mathbf{e}_R \label{eq:edge-av-vertex_1} , \\
    {\mathcal{R}}_{\mathrm{v\rightarrow e}}(\mathbf{W})_{i,j+\frac{1}{2},k} = & ~ \frac{1}{2} ~ {\Big (} \mathbf{e}_{\phi} \cdot (\mathbf{W}_{i,j+1,k} + \mathbf{W}_{i,j,k}) {\Big )} \mathbf{e}_{\phi} \label{eq:edge-av-vertex_2} , \\
      {\mathcal{R}}_{\mathrm{v\rightarrow e}}(\mathbf{W})_{i,j,k+\frac{1}{2}} = & ~ \frac{1}{2} ~ {\Big (} \mathbf{e}_Z \cdot (\mathbf{W}_{i,j,k+1} + \mathbf{W}_{i,j,k}) {\Big )} \mathbf{e}_Z, \label{eq:edge-av-vertex_3}
\end{align}
the vertex-to-face reconstruction ${\mathcal{R}}_{\mathrm{v\rightarrow f}}$ by:
\begin{align}
{\mathcal{R}}_{\mathrm{v\rightarrow f}}(\mathbf{W})_{i+\frac{1}{2},j+\frac{1}{2},k} = ~ \dfrac{1}{4} ~ {\Big (} & \mathbf{e}_Z \cdot ( \mathbf{W}_{i,j,k} + \mathbf{W}_{i,j+1,k} + \mathbf{W}_{i+1,j,k} + \mathbf{W}_{i+1,j+1,k}) {\Big )} \mathbf{e}_Z, \label{eq:face-av-vertex2}
\\
{\mathcal{R}}_{\mathrm{v\rightarrow f}}(\mathbf{W})_{i+\frac{1}{2},j,k+\frac{1}{2}} = ~ \dfrac{1}{4} ~ {\Big (} & \mathbf{e}_{\phi} \cdot (\mathbf{W}_{i,j,k} + \mathbf{W}_{i,j,k+1} + \mathbf{W}_{i+1,j,k} + \mathbf{W}_{i+1,j,k+1}) {\Big )} \mathbf{e}_{\phi} , \label{eq:face-av-vertex3}
\\
{\mathcal{R}}_{\mathrm{v\rightarrow f}}(\mathbf{W})_{i,j+\frac{1}{2},k+\frac{1}{2}} = ~ \dfrac{1}{4} ~ {\Big (} & \mathbf{e}_{R} \cdot ( \mathbf{W}_{i,j,k} + \mathbf{W}_{i,j+1,k} + \mathbf{W}_{i,j,k+1} + \mathbf{W}_{i,j+1,k+1}) {\Big )} \mathbf{e}_{R}. \label{eq:face-av-vertex4}
\end{align}
 Note that all these projection and reconstruction operators are second order accurate for the uniform orthogonal mesh considered.
\begin{remark}
These projection and reconstruction operators could be generalized for non-uniform meshes. 
\end{remark}
% KL{Only for a uniform orthogonal mesh. 
%Should we also add a remark on other projectors, including those with upwind flavor.} \XT{Zakariae, is the issue raised by Konstantin here resolved?}
%\ZJ{No, it is not. I do not know what Konstantin wanted to outline here about other projectors.} 

We use these projection and reconstruction operators as well as the
mimetic operators to discretize the quasi-static model:
\begin{itemize}
\item In the plasma region:
\begin{align}
  \frac{\partial n_{i,h}}{\partial t} + {\mathrm{Div}_h} \left({\mathcal{P}}_{\mathrm{c\rightarrow f}}({n_{i,h}}) {\mathcal{R}}_{\mathrm{v\rightarrow f}}({{\V}}_{i\perp})\right) = & ~0 \quad \text{in} \quad  \Omega^{P}, \label{eq:4.565_disc} \\
  - \nu n_0 m_i \left( \widetilde{\mathrm{Div}_h} ({\mathrm{Grad}_h} \V_{i\perp,R}) - \dfrac{1}{R^2}\V_{i\perp,R} - \dfrac{2}{R} {\mathcal{P}}_{\mathrm{e\rightarrow v}}( {\mathrm{Grad}_h} \V_{i\perp,\phi} ) \cdot \mathbf{e}_{\phi} \right) \quad & \nonumber \\
  - \left( {\mathcal{P}}_{\mathrm{e\rightarrow v}}( \underline{\widetilde{\mathrm{Curl}}}_h {\mathbf{B}}_h) \times {\mathcal{P}}_{\mathrm{f\rightarrow v}}({\mathbf{B}_h}) \right) \cdot \mathbf{e}_R = & ~0 \quad \text{in} \quad \Omega^{P} , \label{eq:4.566_disc} \\
  {\V}_{i\perp} \cdot {\mathcal{P}}_{\mathrm{f\rightarrow v}}( {{\B}}_h)  = & ~ 0 \quad \text{in} \quad \Omega^{P}, \label{eq:4.567_disc} \\
  - \nu n_0 m_i \left( \widetilde{\mathrm{Div}_h} ({\mathrm{Grad}_h} \V_{i\perp,Z}) \right) - \left( {\mathcal{P}}_{\mathrm{e\rightarrow v}}(\underline{\widetilde{\mathrm{Curl}}}_h {\mathbf{B}}_h) \times {\mathcal{P}}_{\mathrm{f\rightarrow v}}({\mathbf{B}_h}) \right) \cdot \mathbf{e}_Z= & ~ 0 \quad \text{in} \quad \Omega^{P}, \label{eq:4.568_disc} \\
  - \widetilde{\mathrm{Div}_h} ({\mathrm{Grad}_h}({\Phi}_h)) - \widetilde{\mathrm{Div}_h} \left[ - {\mathcal{R}}_{\mathrm{v\rightarrow e}} \left({{\V}}_{i\perp}\times {\mathcal{P}}_{\mathrm{f\rightarrow v}}({{\B}}_h)\right) + \widetilde{\mathrm{Curl}_h}({{\B}}_h) \right] = & ~ 0 \quad \text{in} \quad \Omega^{P}, \label{eq:4.569_disc} \\
  \boldsymbol{\tau}_h - {\mathrm{Grad}_h}({\Phi}_h) + {\mathcal{R}}_{\mathrm{v\rightarrow e}} \left({{\V}}_{i\perp}\times {\mathcal{P}}_{\mathrm{f\rightarrow v}}({{\B}}_h)\right) - \widetilde{\mathrm{Curl}_h}({{\B}}_h) = & ~ \mathbf{0} \quad \text{in} \quad \Omega^{P}, \label{eq:4.570_disc} \\
  \frac{\partial{\B}_h}{\partial t} + {\mathrm{Curl}_h}(\boldsymbol{\tau}_h) = & ~ \mathbf{0} \quad \text{in} \quad \Omega^{P}. \label{eq:4.571_disc}
\end{align}
\item In the wall region:
\begin{align}
\frac{\partial n_{i,h}}{\partial t} = & ~ 0 \quad \text{in} \quad \Omega^{W}, \label{eq:4.572_disc} \\
  {\V}_{i\perp} = & ~ \mathbf{0} \quad \text{in} \quad \Omega^{W} , \label{eq:4.573_disc} \\
  - \widetilde{\mathrm{Div}_h} ({\mathrm{Grad}_h}({\Phi}_h)) - \widetilde{\mathrm{Div}_h} \left[ \widetilde{\mathrm{Curl}_h}({{\B}}_h) \right] = & ~ 0 \quad \text{in} \quad \Omega^{W} \label{eq:4.574_disc} \\
  \boldsymbol{\tau}_h - {\mathrm{Grad}_h}({\Phi}_h) - \widetilde{\mathrm{Curl}_h}({{\B}}_h) = & ~ \mathbf{0} \quad \text{in} \quad \Omega^{W}, \label{eq:4.575_disc} \\
  \frac{\partial{\B}_h}{\partial t} + {\mathrm{Curl}_h}(\boldsymbol{\tau}_h) = & ~ \mathbf{0} \quad \text{in} \quad \Omega^{W} . \label{eq:4.576_disc}
\end{align}
\end{itemize}

 As mentioned earlier, the derived mimetic curl operator ${\widetilde{\mathrm{Curl}_h}}$ used in~\eqref{eq:4.569_disc} and~\eqref{eq:4.570_disc} includes the variable coefficient $\dfrac{\eta}{\mu_0}$ as $K$ in~\eqref{eqn:Kcurl}. On the other hand, the other derived mimetic curl operator $\underline{\widetilde{\mathrm{Curl}}}_h$ used in~\eqref{eq:4.566_disc} and~\eqref{eq:4.568_disc} includes only the constant $\frac{1}{\mu_0}$.
%\KL{We need to explain that usage of derived operators is the right way to go (e.g. correct appximation of Laplacian) in each case and bad alternative would be to use projectors and primary operators.}
The usage of derived mimetic operators in the discrete model above instead of the combination of projections/reconstructions and primary mimetic operators is motivated by the need to preserve important properties of the continuum problem (see \ref{sec:magNRJ} for magnetic energy dissipation), and also to simplify the Jacobian matrix by cancelling some off-diagonal blocks using discrete identities (see Section~\ref{sec:precsolve}). 

 The MFD discretization of the induction equation \eqref{eq:4.571_disc} and \eqref{eq:4.576_disc} lead to the following property.
 \begin{prop}
 Given a divergence-free initial magnetic field, the magnetic field solved from the above MFD system is divergence-free,
 when the primary divergence operator ${\mathrm{Div}_h}$ is considered.
 \end{prop}
 \noindent To see that, we note that for a discretization such as a backward Euler of the above system, the following identity holds
\begin{align*}
  {\mathrm{Div}_h} (\B_h^{n+1}) & = {\mathrm{Div}_h} (\B_h^n) - \Delta t \, {\mathrm{Div}_h} ( {\mathrm{Curl}_h} (\boldsymbol{\tau}_h^{n+1})) \\
  & =  {\mathrm{Div}_h} (\B_h^n).
\end{align*}
That being said,  the divergence-free property needs to satisfy two constraints: the initial magnetic field is divergence-free and 
the iterative solver has to be accurate enough. 
In practice, however, it is not easy to prepare a set of an initial condition with the magnetic field being divergence-free.
It is also not necessary to solve  the nonlinear system excessively accurately just for the divergence free property. 
An extra corrector stage using the solved $\boldsymbol{\tau}_h^{n+1}$ can be performed in order to avoid the impact of the iterative solver.
Such a strategy is commonly used in the constrained transport approach for ideal MHD~\cite{rossmanith2006unstaggered, christlieb2014finite, christlieb2016high}. 
We have experimented such a strategy but found its impact to the qualify of the solution is very minimal. 

 By the mimetic theory, we have the definitions of the derived divergence and curl operators given by Formulas~\eqref{eqn:DDiv-def} and~\eqref{eqn:DCurl-def} where the mass matrices $\mathbb{M}_n$, $\mathbb{M}_e$ and $\mathbb{M}_f$ are in the spaces of vertex-based, edge-based and face-based unknowns, respectively.
We have to think about the discrete operator ${\mathrm{Curl}_h}$ (resp. ${\mathrm{Grad}_h}$ and ${\mathrm{Div}_h}$) as a rectangular matrix
acting form the space of edge-based (resp. vertex-based and face-based) fields to the space of face-based (resp. edge-based and cell-based) fields.
We will present the detailed formulas for all the pieces in the above equations.

%need more thoughts: TODO \ZJ{I thought it would be helpful for the reader to have an idea of what the primary mimetic operators look like and provide them with the mass matrices that are necessary to build the derived mimetic operators (along with the primary ones). In this part, we also have the opportunity to show how the derived mimetic operators embed variable coefficients or material properties and to outline there could be different versions of the same mass matrix for different derived operators (for example here, $\mathbb{M}_e^{(3)}$ used in the definition of the derived mimetic divergence operator is different than $\mathbb{M}_e^{(1)}$ used for the derived mimetic curl operator that includes the coefficient $\dfrac{\mu_0}{\eta}$. And the latter matrix differs from $\mathbb{M}_e^{(2)}$ that defines the derived mimetic curl operator that includes only the permeability $\mu_0$). }

 Consider a cylindrical mesh.
The typical cell $c$ is shown in Figure~\ref{fig:cell3D} and the radius of the cell center is denoted by $R_c$.
\begin{figure}[ht]
\centering
\includegraphics[trim={1cm 0 0 .5cm},clip, width=10cm]{figs/plot.pdf}
\caption{A typical computational cell in the configuration space used in the current work.\label{fig:cell3D}} 
\end{figure}
 The definition of the primary curl operator for the top face of cell $c$ in Figure~\ref{fig:cell3D} is
$$
  {\mathrm{Curl}_h} \boldsymbol{\tau}_h 
  = \frac{1}{|f|} \sum\limits_{e \in \partial f} \alpha_{f,e}\,|e|\, {\tau}_e
  = \frac{1}{|f|} \left( h_R  \, {\tau}_1 + (R_c+ \frac{h_R}{2}) h_\phi\, {\tau}_2 - h_R  \, {\tau}_3 - (R_c- \frac{h_R}{2}) h_\phi \, {\tau}_4 \right).
$$
The definition of the primary gradient operator for the number $1$ edge in the top face of cell $c$ in Figure~\ref{fig:cell3D} is
$$
  {\mathrm{Grad}_h} {\Phi}_h = \frac{1}{|e|} \left({\Phi}(R_c + \dfrac{h_R}{2}) - {\Phi}(R_c - \dfrac{h_R}{2} ) \right) = \frac{1}{h_R} \left({\Phi}(R_c + \dfrac{h_R}{2}) - {\Phi}(R_c - \dfrac{h_R}{2} ) \right).
$$
The definition of the primary divergence operator for the cell $c$ in Figure~\ref{fig:cell3D} is
\begin{align*}
  &{\mathrm{Div}_h} \mathbf{B}_h = \frac{1}{|c|} \sum\limits_{f \in \partial c} \alpha_{c,f} |f|\, B_f\\ 
  & = \frac{1}{R_ch_R h_z h_\phi} \left( |f_{\rm right}| B_{\rm right} - |f_{\rm left}| B_{\rm left} + |f_{\rm up}| B_{\rm up} - |f_{\rm down}| B_{\rm down} + |f_{\rm front}| B_{\rm front} - |f_{\rm back}| B_{\rm back} \right).
\end{align*}
The mass matrices $\mathbb{M}_n$, $\mathbb{M}_e$ and $\mathbb{M}_f$ are defined similarly. First, we use the
additivity of integration to break the mass matrix into the sum of cell-based
matrices:
$$
  \mathbb{M}_e = \sum\limits_{c} {\cal N}_c \, \mathbb{M}_{e,c} \, {\cal N}_c^T,
$$
where ${\cal N}_c$ is the conventional assembly matrix.
The elemental matrix $\mathbb{M}_{e,c}$ affects the accuracy of the scheme, however, its selection does not break other mimetic properties such as the discrete exact identities.
The MFD framework says that the vector-matrix-vector product should approximate the integral of underlying 3D vector functions.
Recall that material properties (the coefficient $\dfrac{\eta}{\mu_0}$ where the resistivity $\eta$ may vary from a cell to another) are embedded in the derived operator $\widetilde{\mathrm{Curl}_h}$ via mass matrices.
In this case,
$$
  \mathbf{u}_h^T \,\mathbb{M}_{e,c}\, \mathbf{v}_h
  = \frac{\mu_0}{\eta} \int_c \mathbf{u} \cdot \mathbf{v} \, {\rm d} V + O(h) |c|
  = \frac{\mu_0}{\eta} \int_c \mathbf{u} \cdot \mathbf{v} \, r {\rm d} r {\rm d} z {\rm d} \varphi + O(h) |c|.
$$
By taking unitary functions $\mathbf{u}_h$ and $\mathbf{v}_h$ in the formula above, we could select $\mathbb{M}_{e,c}$ the be the diagonal matrix of size $12$, which is the number of edges in a cell, given by: 
\begin{equation}
\mathbb{M}_{e,c} = \mathbb{M}_{e,c}^{(1)} := \frac{\mu_0}{\eta} \frac{|c|}{4} \mathbb{I}_c.
\end{equation} 
Using the same line of thoughts, we conclude that the elemental matrix $\mathbb{M}_{f,c}$ could be also a scalar matrix of size 6, which is the number of faces in a cell, given by: 
\begin{equation}
\mathbb{M}_{f,c} = \frac{|c|}{2} \mathbb{I}_c.
\end{equation} 
Thus, we have for any face-based vector $\mathbf{X}$ and edge-based vector $\mathbf{Y}$: 
\begin{align}
\mathbb{M}_f \mathbf{X} & = \left[ \beta^f_{i} \mathbf{X}_{i} \right]_i ; \quad \beta^f_{i}:= \sum_{c_k \in \mathcal{C}(f_i)} \alpha^f_{c_k} ; \quad \alpha^{f}_{c_k} := \frac{|c_k|}{2} ; \\
\mathbb{M}_e^{(1)} \mathbf{Y} & = {\Big[} \beta^e_{l} \mathbf{Y}_{l} {\Big]}_l ; \quad \beta^e_{l}:= \sum_{c_k \in \mathcal{C}(e_l)} \alpha^e_{c_k} ; \quad \alpha^{e}_{c_k} := \frac{\mu_0}{\eta(c_k)} \frac{|c_k|}{4} ;
\end{align}
where $\mathcal{C}(f_l)$ denotes all cells sharing face $f_l$ and $\mathcal{C}(e_l)$ denotes all cells sharing edge $e_l$. 
Hence, we have all the ingredients to build the discrete derived curl operator $\widetilde{\mathrm{Curl}_h}$. To form the other discrete derived curl operator $\underline{\widetilde{\mathrm{Curl}}}_h$, it suffices to consider the following mass matrix for edge-based vectors:  
\begin{align}
\mathbb{M}_e^{(2)} \mathbf{Y} & = {\Big[} \underline{\beta}^e_{l} \mathbf{Y}_{l} {\Big]}_l ; \quad \underline{\beta}^e_{l} := \sum_{c_k \in \mathcal{C}(e_l)} \underline{\alpha}^e_{c_k} ; \quad \underline{\alpha}^{e}_{c_k} := \mu_0 \frac{|c_k|}{4} ;
\end{align}
As for the discrete derived gradient operator $\widetilde{\mathrm{Div}_h}$, the following mass matrices are needed. For any vertex-based vector $\mathbf{W}$ and edge-based vector $\mathbf{Y}$:  
\begin{align}
\mathbb{M}_n \mathbf{W} & = \left[ \beta^n_{j} \mathbf{W}_{j} \right]_j ; \quad \beta^n_{j} := \sum_{c_k \in \mathcal{C}(n_j)} \alpha^n_{c_k} ; \quad \alpha^{n}_{c_k} := \frac{|c_k|}{8} ; \\
\mathbb{M}_e^{(3)} \mathbf{Y} & = {\Big[} \hat{\beta}^e_{l} \mathbf{Y}_{l} {\Big]}_l ; \quad \hat{\beta}^e_{l} := \sum_{c_k \in \mathcal{C}(e_l)} \hat{\alpha}^e_{c_k}; \quad \hat{\alpha}^{e}_{c_k} := \frac{|c_k|}{4}.
\end{align}


%%%%%%%%%%%%%%%%%%%%%%%%%%%%%%%%%%%%%%%%%%%%%%%%%%%%%%%%%%%%%%%%%%%%%%
%\subsection{Divergence-free magnetic field} \label{sec:div-freeB}
%Consider the fully implicit backward Euler time discretization of the quasi-static perpendicular plasma dynamics model (other implicit schemes can be treated similarly),
%\begin{equation}\label{MFD-discrete}
%  \frac{\B_h^{n+1} - \B_h^n}{\Delta t} = - \, {\mathrm{Curl}_h} (\boldsymbol{\tau}_h^{n+1}).
%\end{equation}
%After solving for $\boldsymbol{\tau}_h^{n+1}$, it is possible to add a post-processing step to compute the magnetic field from the discrete 
%Faraday law:
%$$
%  \B_h^{n+1} = \B_h^n - \Delta t \, {\mathrm{Curl}_h} (\boldsymbol{\tau}_h^{n+1}).
%$$
%This formula looks like the explicit Euler time stepping.
%By applying the primary mimetic divergence operator, we obtain:
%$$
%  {\mathrm{Div}_h} (\B_h^{n+1}) = {\mathrm{Div}_h} (\B_h^n) - \Delta t \, {\mathrm{Div}_h} ( {\mathrm{Curl}_h} (\boldsymbol{\tau}_h^{n+1})).
%$$
%Hence, the divergence-free property of the magnetic field holds up to machine precision
%regardless of the value of iterative solver error in the electric field because of the discrete identities satisfied by the mimetic operators in~\eqref{eqn:prim_disc_ident}.

%%%%%%%%%%%%%%%%%%%%%%%%%%%%%%%%%%%%%%%%%%%%%%%%%%%%%%%%%%%%%%%%%%%%%
