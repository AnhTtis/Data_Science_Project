\section{Nonuniqueness of solutions to the model without the viscosity term}
\label{sec:nonuniqueness}
This section addresses the nonuniqueness of the solution to the model without the viscosity regularization term.
Consider the following model in a single domain
\begin{align}
  -\left(  \dfrac{1}{\mu_0} (\nabla\times{\B})\times{\B} \right) \cdot \mathbf{e}_R & = 0 & \qquad \text{in} \quad \Omega^{W}, \label{eqn:ap_v1}  \\
  {\V}_{i\perp} \cdot {\B} & = 0 & \qquad \text{in} \quad \Omega^{W}, \label{eqn:ap_v2}\\
  -\left(   \dfrac{1}{\mu_0} (\nabla\times{\B})\times{\B} \right) \cdot \mathbf{e}_Z & = 0 & \qquad \text{in} \quad \Omega^{W},  \label{eqn:ap_v3}\\
  \nabla^2 \Phi & = - \nabla\cdot\left[ - {\V}_{i\perp}\times {\B} + \frac{\eta}{\mu_0}\left(\nabla\times{\B}\right) \right] &\qquad \text{in} \quad \Omega^{W},    \\
  \boldsymbol{\tau} & = \nabla\Phi - {\V}_{i\perp}\times {\B} + \frac{\eta}{\mu_0}\left(\nabla\times{\B}\right) &\qquad \text{in} \quad \Omega^{W},   \\
  \frac{\partial{\B}}{\partial t} & = - \nabla\times\boldsymbol{\tau} &\qquad \text{in} \quad \Omega^{W},
\end{align}
under some proper boundary condition
\begin{align}
  \mathbf{V}_{i\perp} & = \mathbf{0} &\qquad \text{on} \quad \partial \Omega^{W}, \label{eqn:ap_bc_V}\\
  \Phi & = \Phi_{\rm bc} &\qquad \text{on} \quad \partial\Omega^{W}, \label{eqn:ap_bc_Phi}\\
  \boldsymbol{\tau} & = \boldsymbol{\tau}_{\rm bc} &\qquad \text{on} \quad \partial\Omega^{W}. \label{eqn:ap_bc_tau}
\end{align}
The equation for $n_i$ is dropped as it is not coupled to the four fields considered here. At first glance, it is
tempting to assume the above system is well-posed  since it is a closed system.  
We will however show the form of a possible null space that only needs to satisfy a mild constraint.

Let $(\boldsymbol{\tau} , \B, \V_{i\perp}, \Phi )$ and $(\boldsymbol{\tau} + \mathbf{Z}, \B, \V_{i\perp}  + \mathbf{W}, \Phi + \varphi)$ be two solutions of the above system. We then have:%\KL{I think this is not needed. A reader can compute the difference of two sets of equations.Go directly to (A.17)}
%\begin{align}
%  \left((\nabla\times{\B})\times{\B} \right) \cdot \mathbf{e}_R= & ~0 , \\
%  \left((\nabla\times{\B})\times{\B} \right) \cdot \mathbf{e}_Z = & ~ 0 ,  \\
%  {\V}_{i\perp} \cdot {\B} = & ~ 0 ,  \\
%  \left( {\V}_{i\perp} + \mathbf{W} \right) \cdot {\B} = & ~ 0 ,  \\
%  \nabla^2 {\Phi} = & - \nabla\cdot\left[
%    - {{\V}}_{i\perp}\times {{\B}} +  \frac{\eta}{\mu_0}\left(\nabla\times{{\B}}\right)
%    \right] \\
%  \nabla^2 {\Phi} + \nabla^2 {\varphi} = & - \nabla\cdot\left[
%    - {{\V}}_{i\perp}\times {{\B}} - \mathbf{W}\times {\B} + \frac{\eta}{\mu_0}\left(\nabla\times{{\B}}\right)
%    \right] , \\
%  {\boldsymbol{\tau}} = & \nabla \Phi - {{\V}}_{i\perp}\times {{\B}} +  \frac{\eta}{\mu_0}\left(\nabla\times{{\B}}\right) , \\
%  {\boldsymbol{\tau}} + \mathbf{Z} = & ~ \nabla \Phi + \nabla \varphi - {{\V}}_{i\perp}\times {{\B}} - \mathbf{W}\times {\B} +   \frac{\eta}{\mu_0}\left(\nabla\times{{\B}}\right) , \\
%  \frac{\partial{\B}}{\partial t} = & - \nabla\times {\boldsymbol{\tau}} , \\
%  \frac{\partial{\B}}{\partial t} = & - \nabla\times \left( {\boldsymbol{\tau}} + \mathbf{Z} \right) .
%\end{align}
%Thus,
\begin{align}
\label{eq:WdotB}
\mathbf{W} \cdot {\B} = & ~ 0 & \qquad \text{in} \quad \Omega^{W} ,  \\
\nabla^2 {\varphi} = & ~ \nabla \cdot \left( \mathbf{W}\times {\B} \right) & \qquad \text{in} \quad \Omega^{W},
\label{eq:LaplacianW} \\
\mathbf{Z} = & ~ \nabla \varphi - \mathbf{W}\times {\B} & \qquad \text{in} \quad \Omega^{W}, \\
\nabla\times  \mathbf{Z} = & - \nabla\times (\mathbf{W}\times {\B}) = ~ \mathbf{0} & \qquad \text{in} \quad \Omega^{W}, \label{eq:nil_curl}
\\
\mathbf{W}  = & ~ \mathbf{0} &\qquad \text{on} \quad \partial \Omega^{W}, \label{eqn:ap_bc_W} \\
\varphi = & ~ 0 &\qquad \text{on} \quad \partial\Omega^{W}, \label{eqn:ap_bc_varphi}\\
\mathbf{Z} = & ~ \mathbf{0} &\qquad \text{on} \quad \partial\Omega^{W} \label{eqn:ap_bc_Z}
.
\end{align}
\eqref{eq:nil_curl} implies that there exists a field $g \in \mathrm{H}^{1}(\Omega)$
 such that:
\begin{equation}
\label{eq:conservative_field_2}
\mathbf{W} \times \B = \nabla g ;
\end{equation}
That further implies that
\begin{equation}
\nabla \cdot (\mathbf{W} \times \B) = \nabla^2 g = \nabla^2 \varphi.
\end{equation}
The Dirichlet boundary condition for $\Phi$ suggests that $g$ is equal to $\varphi$, %\KL{it would be useful to mention earlier boundary conditions for $\phi$, $Z$ and $W$.} \ZJ{These boundary conditions were added in~\eqref{eqn:ap_bc_W},~\eqref{eqn:ap_bc_varphi} and~\eqref{eqn:ap_bc_Z}}
and thus $\nabla \varphi = \nabla g = \mathbf{W} \times \B$ and $\mathbf{Z} = 0$.
Moveover, Equations~\eqref{eq:WdotB} %and~\eqref{eq:LaplacianW} 
together with $\nabla \varphi = \mathbf{W} \times \B$ are equivalent to:
\begin{align}
\mathbf{W} & = ||{\B}||^{-2} ({\B} \times \nabla \varphi) \\
\nabla \varphi \cdot {\B} & = \mathbf{0}. \label{eqn:psiconstraint}
\end{align}
To prove the equivalence $\left\{ \begin{aligned} 
  \mathbf{W} \cdot {\B} & = ~ 0 \\
  \nabla \varphi &= \mathbf{W} \times \B
\end{aligned} \right. \Leftrightarrow \left\{ \begin{aligned} 
  \mathbf{W} & = ||{\B}||^{-2} ({\B} \times \nabla \varphi) \\
  \nabla \varphi \cdot {\B} & = \mathbf{0} 
\end{aligned} \right. $, it suffices to compute the dot and cross product with $\B$ for the bottom left side equation and the top right side equation.
This identifies at least one null space for the system, which only needs the scalar function $\varphi$ to satisfy the constraint~\eqref{eqn:psiconstraint}.
As pointed out in Section~\ref{sec:regularization}, such a constraint can be easily satisfied in the axisymmetric tokamak, as the magnetic field is given by~\eqref{eqn:Bfrompsi} 
and any function of $\varphi(\psi)$ satisfies~\eqref{eqn:psiconstraint}. 

% Inversely, let $(\boldsymbol{\tau} , \B, \V, \Phi )$ be a solution of the model and let $\varphi \in \mathrm{H}^{1}(\Omega^{P})$
% such that $\nabla \varphi \cdot \B = 0$.
% We have:
% \begin{align}
% - ||\B||^{-2} (\nabla \varphi \times \B) \cdot \B & = 0 , \\
% - ||\B||^{-2} (\nabla \varphi \times \B) \times \B & = \nabla \varphi, \\
%  - \nabla \cdot \left( ||\B||^{-2} (\nabla \varphi \times \B) \times \B \right) & = \nabla^2 \varphi .
% \end{align}
%Then for $\mathbf{W} := - ||\B||^{-2} (\nabla \varphi \times \B) $, we have:
%\begin{align}
%(\V + \mathbf{W}) \cdot \B & = \mathbf{W} \cdot\B = 0, \\
% \boldsymbol{\tau} - \nabla(\varphi + \Phi) + (\V + \mathbf{W}) \times \B - \frac{1}{ v_A} * \frac{\eta}{\mu_0}\left(\nabla\times{{\B}}\right) & = - \nabla \varphi +  \mathbf{W} \times \B = \mathbf{0}, \\
% \nabla^2(\varphi + \Phi) - \nabla\cdot\left[
%    {{\V}}_{i\perp}\times {{\B}} + \mathbf{W}\times {{\B}} - \frac{1}{ v_A} * \frac{\eta}{\mu_0}\left(\nabla\times{{\B}}\right)
%    \right] &= \nabla^2\varphi - \nabla\cdot\left[ \mathbf{W}\times {{\B}} \right] = 0 .
%\end{align}
%That shows that $(\boldsymbol{\tau}, \B, \V + \mathbf{W}, \Phi + \varphi)$ is also a solution of the model.


%\bigskip
%
%In what follows, we provide an algebraic proof that the system is ill-posed by showing that the Jacobian matrix is singular. Here we rely on the following well known proposition
%\begin{prop}
%Given a block $2\times 2$ matrix 
%\begin{align*}
%M = \left[\begin{matrix} A & B \\ C & D \end{matrix}\right]
%\end{align*}
%where $A$ and $D$ are square matrices and $A$ is invertible, the matrix $M$ is invertible if and only if its Schur complement $M/A := D - CA^{-1}B$ is invertible. 
%\end{prop}
%According to Formula~\eqref{eqn:4.581} (the ion density equation is again dropped), since the diagonal block for the $\boldsymbol{\tau}$ field is an identity matrix, the Jacobian is invertible if and only if $\mathbf{S}_{\{ \Phi, {\B}, {\V}_{i\perp} \}}$, the Schur complement for the $\{ \Phi, {\B}, {\V}_{i\perp} \}$ block, is invertible.
%The same reasoning applies to this latter matrix.
%By looking at Formula~\eqref{eqn:EBV_Schur}, since the $\Phi$-submatrix (laplacian operator) is clearly nonsingular, $\mathbf{S}_{\{ \Phi, {\B}, {\V}_{i\perp} \}}$ is invertible if and only if its bottom right diagonal block (corresponding to $\{\B , \mathbf{V}_{i \perp} \}$), which is also equal to $\mathbf{S}_{\{ {\B}, {\V}_{i\perp} \}}$ the Schur complement for the $\{ {\B},  {\V}_{i\perp} \}$ block, is invertible.
%
% Now, we consider that $\{\B , \mathbf{V}_{i \perp} \}$ diagonal block, i.e., $\mathbf{S}_{\{ {\B}, {\V}_{i\perp} \}}$  and show that submatrix is singular.
%For that purpose, we start by re-permuting the fields: $\{ {{\V}_{i\perp}}_{R} , {{\V}_{i\perp}}_{Z} , {{\V}_{i\perp}}_{\phi}, \B \}$. The corresponding $\{\mathbf{V}_{i \perp} , \B \}$ diagonal block then becomes:
%\begin{equation}
%\mathbf{S}_{\{ {\V}_{i\perp}, {\B} \}} \, d\mathbf{U} =
%\begin{bmatrix}
%  \mathbf{0} & \mathbf{0} & \mathbf{0} & {\Big [} - \mathbf{B}_0 \times (\nabla \times \quad) + (\nabla \times {\B}_0) \times \quad {\Big ]} \cdot {\mathbf{e}}_{R}  \smallskip \\
%  \mathbf{0} & \mathbf{0} & \mathbf{0} & {\Big [} - \mathbf{B}_0 \times (\nabla \times \quad) + (\nabla \times {\B}_0) \times \quad {\Big ]} \cdot {\mathbf{e}}_{Z}  \smallskip \\
%  {{\B}_0}_{R} & {{\B}_0}_{Z} & {{\B}_0}_{\phi} & {\V}_{0} \cdot \smallskip \\
%  \mathbf{W}_4 & \mathbf{W}_6 & \mathbf{W}_5 & \dfrac{1}{dt}\mathbf{I} + \nabla \times (\frac{\eta}{\mu_0} \nabla \times \quad) - \nabla \times ({\V}_{0} \times \quad)
%\end{bmatrix}
%\begin{bmatrix}
%{d{\V}_{i\perp}}_{R} \smallskip\\
%{d{\V}_{i\perp}}_{Z} \smallskip\\
%{d{\V}_{i\perp}}_{\phi} \smallskip\\
%d {\B}
%\end{bmatrix} ,
%%\end{blockarray},
%\label{eqn:BV_Schur_2}
%\end{equation}
%where
%\begin{align}
%  \nabla \times ( {\B}_0 \times {d{\V}_{i\perp}}) = \mathbf{W}_5 ~ {d{\V}_{i\perp}}_{\phi} + \mathbf{W}_4 ~ {d{\V}_{i\perp}}_{R} + \mathbf{W}_6 ~ {d{\V}_{i\perp}}_{Z}.
%\end{align}
%In~\eqref{eqn:BV_Schur_2}, we denote by $\mathbf{S}_{0}$ the top left $2\times2$ submatrix that has only zero entries, and by $\mathbf{S}_{1}$ the bottom right $2\times2$ submatrix.
%In $\mathbf{S}_{1}$, the top left block ${{\B}_0}_{\phi}$ is diagonal with all its diagonal values being nonzero, and $\mathbf{S}_{\B}$ the Schur complement for $\B$
%\begin{equation}
%\mathbf{S}_{\B}:= \dfrac{1}{dt}\mathbf{I} + \nabla \times (\frac{\eta}{\mu_0} \nabla \times \quad) - \nabla \times ({\V}_{0} \times \quad) - \mathbf{W}_5 {\B_0}^{-1}_{\phi} {\V}_{0} \cdot \quad ,
%\end{equation}
%is nonsingular, thus $\mathbf{S}_{1}$ is nonsingular and and its inverse is given by
%\begin{equation}
%\mathbf{S}_{1}^{-1}  =
%\begin{bmatrix}
%{\B_0}^{-1}_{\phi} + {\B_0}^{-1}_{\phi} ({\V}_{0} \cdot \quad ) \mathbf{S}^{-1}_{\B} \mathbf{W}_5 {\B_0}^{-1}_{\phi} & - {\B_0}^{-1}_{\phi} ({\V}_{0} \cdot \quad ) \mathbf{S}^{-1}_{\B} \smallskip \\
%- \mathbf{S}^{-1}_{\B} \mathbf{W}_5 {\B_0}^{-1}_{\phi} & \mathbf{S}^{-1}_{\B}
%\end{bmatrix}.
%\end{equation}
%The Schur complement for $\{{\mathbf{V}_{i \perp}}_{\phi} , \B \}$ fields inside $\mathbf{S}_{\{ {\V}_{i\perp} {\B} \}}$ is expressed as:
%\begin{align}
%\mathbf{S}_{0} - \mathbf{S}_{01} \mathbf{S}_{1}^{-1} \mathbf{S}_{10} = & -
%\begin{bmatrix}
%\mathbf{0} &  {\Big [} - \mathbf{B}_0 \times (\nabla \times \quad) + (\nabla \times {\B}_0) \times \quad {\Big ]} \cdot {\mathbf{e}}_{R} \smallskip \\
%\mathbf{0} & {\Big [} - \mathbf{B}_0 \times (\nabla \times \quad) + (\nabla \times {\B}_0) \times \quad {\Big ]} \cdot {\mathbf{e}}_{Z}
%\end{bmatrix} *  \nonumber \\
%& \begin{bmatrix}
%{\B_0}^{-1}_{\phi} + {\B_0}^{-1}_{\phi} ({\V}_{0} \cdot \quad ) \mathbf{S}^{-1}_{\B} \mathbf{W}_5 {\B_0}^{-1}_{\phi} & - {\B_0}^{-1}_{\phi} ({\V}_{0} \cdot \quad ) \mathbf{S}^{-1}_{\B} \smallskip \\
%- \mathbf{S}^{-1}_{\B} \mathbf{W}_5 {\B_0}^{-1}_{\phi} & \mathbf{S}^{-1}_{\B}
%\end{bmatrix} *
%\begin{bmatrix}
%{\B_0}_{R} &  {\B_0}_{Z} \smallskip \\
%\mathbf{W}_5  & \mathbf{W}_6
%\end{bmatrix} \nonumber \\
%= & -
%\begin{bmatrix}
%\mathbf{0} &  {\Big [} - \mathbf{B}_0 \times (\nabla \times \quad) + (\nabla \times {\B}_0) \times \quad {\Big ]} \cdot {\mathbf{e}}_{R} \smallskip \\
%\mathbf{0} & {\Big [} - \mathbf{B}_0 \times (\nabla \times \quad) + (\nabla \times {\B}_0) \times \quad {\Big ]} \cdot {\mathbf{e}}_{Z}
%\end{bmatrix} *  \nonumber \\
%& \begin{bmatrix}
%\left(\mathbf{S}_{1}^{-1} \mathbf{S}_{10}\right)_{0,0} & \left(\mathbf{S}_{1}^{-1} \mathbf{S}_{10}\right)_{0,1} \smallskip \\
%\mathbf{S}^{-1}_{\B} \mathbf{W}_5 - \mathbf{S}^{-1}_{\B} \mathbf{W}_5 {\B_0}^{-1}_{\phi} {\B_0}_{R} & \mathbf{S}^{-1}_{\B} \mathbf{W}_6 - \mathbf{S}^{-1}_{\B} \mathbf{W}_5 {\B_0}^{-1}_{\phi} {\B_0}_{Z}
%\end{bmatrix} \nonumber \\
%= & -
%\begin{bmatrix}
%{\Big [} - \mathbf{B}_0 \times (\nabla \times \quad) + (\nabla \times {\B}_0) \times \quad {\Big ]} \cdot {\mathbf{e}}_{R} \smallskip \\
%{\Big [} - \mathbf{B}_0 \times (\nabla \times \quad) + (\nabla \times {\B}_0) \times \quad {\Big ]} \cdot {\mathbf{e}}_{Z}
%\end{bmatrix} *  \nonumber \\
%& \begin{bmatrix}
%\mathbf{S}^{-1}_{\B} \mathbf{W}_5 - \mathbf{S}^{-1}_{\B} \mathbf{W}_5 {\B_0}^{-1}_{\phi} {\B_0}_{R} & \mathbf{S}^{-1}_{\B} \mathbf{W}_6 - \mathbf{S}^{-1}_{\B} \mathbf{W}_5 {\B_0}^{-1}_{\phi} {\B_0}_{Z}
%\end{bmatrix} .
%\end{align}
%where
%\begin{align}
%\left(\mathbf{S}_{1}^{-1} \mathbf{S}_{10}\right)_{0,0} & = \left( {\B_0}^{-1}_{\phi} + {\B_0}^{-1}_{\phi} ({\V}_{0} \cdot \quad ) \mathbf{S}^{-1}_{\B} \mathbf{W}_5 {\B_0}^{-1}_{\phi} \right) {\B_0}_{R} - {\B_0}^{-1}_{\phi} ({\V}_{0} \cdot \quad ) \mathbf{S}^{-1}_{\B} \mathbf{W}_5 ; \\
%\left(\mathbf{S}_{1}^{-1} \mathbf{S}_{10}\right)_{0,1} & = \left( {\B_0}^{-1}_{\phi} + {\B_0}^{-1}_{\phi} ({\V}_{0} \cdot \quad ) \mathbf{S}^{-1}_{\B} \mathbf{W}_5 {\B_0}^{-1}_{\phi} \right) {\B_0}_{Z} - {\B_0}^{-1}_{\phi} ({\V}_{0} \cdot \quad ) \mathbf{S}^{-1}_{\B} \mathbf{W}_6 .
%\end{align}
%We just proved that $\mathbf{S}_{0} - \mathbf{S}_{01} \mathbf{S}_{1}^{-1} \mathbf{S}_{10}$ is equal to the product of two low-rank matrices. \QT{check me} As a consequence, its product is not full rank. Thus, the Schur complement for $\{{\mathbf{V}_{i \perp}}_{\phi} , \B \}$ fields inside $\mathbf{S}_{\{ {\V}_{i\perp}, {\B} \}}$ is not invertible.
% We conclude that the Jacobian matrix are singular, Q.E.D.

\section{Regularization through the viscosity term}
\label{sec:uniqueness}
%In this section, we demonstrate the uniqueness of the solution to the model  when there is a viscosity term involved.
%For ease of presentation, we consider the full $\V$ formulation. 
%We proceed in a similar way to the previous section, by assuming that the model has a solution in the plasma region and showing the forms that all possible solutions can take.

This section show the aforementioned null space is removed through the regularization term.
 For ease of presentation, we consider the full $\V$ formulation. More specifically, 
 we consider to replace the equations~\eqref{eqn:ap_v1}--\eqref{eqn:ap_v3} with the following equation
 \begin{equation}
  \frac{1}{\mu_0} (\nabla\times{\B})\times{\B} = - \nu n_0 m_i \nabla^2 {\V}.
 \end{equation}
Let $(\boldsymbol{\tau} , \B, {\V}, \Phi )$ and $(\boldsymbol{\tau} + \mathbf{Z}, \B, {\V} + \mathbf{W}, \Phi + \varphi)$ again be two solutions of the new model. We then have:%\KL{AGain. I think this level of details is not needed, jump directly (B.10)}
%\begin{align}
%  \frac{1}{\mu_0} (\nabla\times{\B})\times{\B} = & ~ - \nu n_0 m_i \nabla^2 {\V}  ,  \\
%  \frac{1}{\mu_0} (\nabla\times{\B})\times{\B} = & ~ - \nu n_0 m_i \nabla^2 ({\V} + \mathbf{W}) ,  \\
%  \nabla^2 {\Phi} = & - \nabla\cdot\left[
%    - {\V} \times {\B} + \frac{\eta}{\mu_0}\left(\nabla\times{{\B}}\right)
%    \right] \\
%  \nabla^2 {\Phi} + \nabla^2 {\varphi} = & - \nabla\cdot\left[
%    - {\V} \times {\B} - \mathbf{W}\times {\B} + \frac{\eta}{\mu_0}\left(\nabla\times{{\B}}\right)
%    \right] , \\
%  \boldsymbol{\tau} = & ~ \nabla \Phi - {\V} \times {\B} + \frac{\eta}{\mu_0}\left(\nabla\times{\B}\right) , \\
%  \boldsymbol{\tau} + \mathbf{Z} = & ~ \nabla \Phi + \nabla \varphi - {\V} \times {{\B}} - \mathbf{W}\times {\B} + \frac{\eta}{\mu_0}\left(\nabla\times{{\B}}\right) , \\
%  \frac{\partial{\B}}{\partial t} = & - \nabla\times \boldsymbol{\tau} , \\
%  \frac{\partial{\B}}{\partial t} = & - \nabla\times \left( \boldsymbol{\tau} + \mathbf{Z} \right) .
%\end{align}
%Thus,
\begin{align}
\nabla^2 \mathbf{W} = & ~ \mathbf{0} , \label{eqn:bndryV} \\
\nabla^2 {\varphi} = & ~ \nabla \cdot \left( \mathbf{W}\times {\B} \right) , \label{eqn:bndryPsi} \\
\mathbf{Z} = & ~ \nabla \varphi - \mathbf{W}\times {\B} , \label{eqn:Z}\\
\nabla\times  \mathbf{Z} = & - \nabla\times (\mathbf{W}\times {\B}) = ~ \mathbf{0} .
\end{align}
The Dirichlet boundary condition implies that $\mathbf{W} = 0$, $\varphi=0$, and  $\mathbf{Z}=0$. Hence, we show the regularization term removes the null space.
This analysis can be easily extended to the case with the $\V_\perp$ case when we consider the axisymmetric case (i.e., $\partial/\partial \phi = 0$). 

\section{Matrix definition of the derived mimetic operators}
\label{sec:der_mim_def}
To obtain a matrix form of the dual operators, we note that 
inner products are represented by symmetric positive definite mass matrices: $\mathbb{M}_n$ for $\mathcal{N}_h$, $\mathbb{M}_e$ for $\mathcal{E}_h$, $\mathbb{M}_f$ for $\mathcal{F}_h$ and $\mathbb{M}_c$ for $\mathcal{C}_h$.
Hence,
\begin{align*}
  (\mathbb{M}_e\,{\mathrm{Grad}_h}\, p_h)^T\, \mathbf{v}_h 
  & = -(\mathbb{M}_n\, p_h)^T\, \widetilde{\mathrm{Div}_h}\,\mathbf{v}_h, \\
  (\mathbb{M}_c\,{\mathrm{Div}_h}\, \mathbf{u}_h)^T\, p_h 
  & = -(\mathbb{M}_f\, \mathbf{u}_h)^T\, \widetilde{\mathrm{Grad}_h}\,p_h, \\
  (\mathbb{M}_f\,{\mathrm{Curl}_h}\, \mathbf{u}_h)^T\, \mathbf{v}_h 
  & = (\mathbb{M}_e\, \mathbf{u}_h)^T\, \widetilde{\mathrm{Curl}_h}\,\mathbf{v}_h.
\end{align*}
These should hold true for any vectors $p_h$, $u_h$ and $\mathbf{v}_h$. 
As a consequence, the matrix definitions are given by
\begin{align}
  \widetilde{\mathrm{Div}_h} & = -\mathbb{M}_n^{-1}\,({\mathrm{Grad}_h})^T\, \mathbb{M}_e, \label{eqn:DDiv-def} \\
  \widetilde{\mathrm{Grad}_h} & = -\mathbb{M}_f^{-1}\,({\mathrm{Div}_h})^T\, \mathbb{M}_c, \label{eqn:DGrad-def} \\
  \widetilde{\mathrm{Curl}_h} & = \mathbb{M}_e^{-1}\,({\mathrm{Curl}_h})^T\, \mathbb{M}_f. \label{eqn:DCurl-def}
\end{align}
The mass matrices are diagonal on structured meshes that are considered in the current work. 

To discretize PDEs, the MFD framework builds upon three primary and three dual operators.
The four discrete spaces along with their primary or derived operators form a discrete de Rham complex. 
The coefficients of the PDEs are usually embedded in the definition of the derived operator. Consider, for example, the following formulas:
\begin{eqnarray}
&&
  \int_\Omega p\,\nabla \cdot {\mathbf u} \,{\rm d}V 
   = -\int_\Omega K^{-1} (K\, \nabla p) \cdot \mathbf{u} \,{\rm d}V 
  + \oint_{\partial \Omega} p\,(\mathbf{u}\cdot \mathbf{n}) \,{\rm d}S, \\
&&
\int_\Omega \mathbf{u} \cdot (\nabla \times \mathbf{v}) \,{\rm d}V 
  = \int_\Omega K^{-1} (K \nabla \times  \mathbf{u}) \cdot \mathbf{v} \,{\rm d}V 
  + \oint_{\partial \Omega} (\mathbf{u}\times \mathbf{v}) \cdot \mathbf{n} \,{\rm d}S, \label{eqn:Kcurl}
\end{eqnarray}
where $K$ is a positive definite tensor. 
These formulas represents the duality between the two first operators, ${\rm div}$ and 
$(K\,{\rm grad})$ for the first equation, ${\rm curl}$ and 
$(K\,{\rm curl})$ for the second one, using a weighted inner product for the vector fields (with $K^{-1}$ 
as the weight). 
The corresponding discrete operators will be in a duality relation with respect to such an inner product.

\section{Magnetic energy dissipation}
\label{sec:magNRJ}
%% Show with diffusion term a proper energy dissipation law can be achieved if we consider the energy as the total magnetic energy
In this section, we show that with a viscosity term and under a proper boundary condition, an energy dissipation law can be achieved if we consider the total magnetic energy.
 Let $(n_i, \boldsymbol{\tau} , \B, {\V}, \Phi )$ be the solution for the five-field model (with full $\V$) with the viscosity term. The total magnetic energy is given by
\begin{equation}
E_{\mathbf{B}} := \frac{1}{2 \mu_0} |\mathbf{B}|^2 .
\end{equation}
We then have:
\begin{align}
  \dfrac{d E_{\mathbf{B}}}{dt} = & \int_{\Omega} \frac{1}{\mu_0} \left( \mathbf{B} \cdot \dfrac{\partial {\B}}{\partial t} \right) dV , \\
   = & - \frac{1}{\mu_0} \int_{\Omega}  \left( \mathbf{B} \cdot (\nabla\times \boldsymbol{\tau}) \right) dV , \\
   = & - \frac{1}{\mu_0} \int_{\Omega}  \left( \boldsymbol{\tau} \cdot (\nabla\times {\B}) \right) dV - \oint_{\partial \Omega} (\mathbf{B}\times \boldsymbol{\tau}) \cdot \mathbf{n} \,{\rm d}S , \\
   = & - \frac{1}{\mu_0} \int_{\Omega}  \left(  ( - {\V} \times {\B} + \frac{\eta}{\mu_0} \nabla \times {\B}) \cdot (\nabla \times {\B} ) \right) dV , \\
   = & ~ \frac{1}{\mu_0} \int_{\Omega}  \left(  (  {\V} \times {\B}) \cdot (\nabla \times {\B} ) \right) dV - \frac{1}{\mu_0^2} \int_{\Omega} \eta |\nabla \times {\B}|^2 dV , \label{eqn:C6} \\ 
   = & - \frac{1}{\mu_0} \int_{\Omega}  \left(  (\nabla \times {\B} ) \times {\B}) \cdot {\V} \right) dV - \frac{1}{\mu_0^2} \int_{\Omega} \eta |\nabla \times {\B}|^2 dV, \label{eqn:C7} \\ 
   = & ~ \nu n_0 m_i \int_{\Omega}  \left(  {\V} \cdot \nabla^2 {\V} \right) dV - \frac{1}{\mu_0^2} \int_{\Omega} \eta |\nabla \times {\B}|^2 dV , \\ 
   = & - \nu n_0 m_i \int_{\Omega} |\nabla {\V}|^2 dV + \nu n_0 m_i \oint_{\partial \Omega}  \left( {\V} \cdot (\nabla {\V} \cdot \mathbf{n}) \right) dS - \frac{1}{\mu_0^2} \int_{\Omega} \eta |\nabla \times {\B}|^2 dV , \\ 
   = & - \nu n_0 m_i \int_{\Omega} |\nabla {\V}|^2 dV - \frac{1}{\mu_0^2} \int_{\Omega} \eta |\nabla \times {\B}|^2 dV , \\
   \leq & ~ 0 .
\end{align}
Here we drop the boundary integral terms assuming the proper boundary condition is imposed. 
\iffalse
\KL{It would be great to show that the discrete system has the same property due to definion of dual operators.}
\XT{Zakariae, can what Konstantin suggested here be shown?} 
\ZJ{I rewrote the discrete analogs of the above equations. I am not sure that the transition $(\V \times \B) \cdot (\nabla \times \B) = ((\nabla \times \B) \times \B) \cdot \V$ from~\eqref{eqn:C6} to~\eqref{eqn:C7} holds at the discrete level with the projection and reconstruction operators involved, i.e., \\ 
$\left[ \underline{\widetilde{\mathrm{Curl}}}_h\, {\mathbf{B}}_h , {\mathcal{R}}_{\mathrm{v\rightarrow e}} \left({{\V}}_{h}\times {\mathcal{P}}_{\mathrm{f\rightarrow v}}({{\B}}_h)\right) \right]_{\mathcal{E}_h} = \left[ {\V}_h , \left( {\mathcal{P}}_{\mathrm{e\rightarrow v}}( \underline{\widetilde{\mathrm{Curl}}}_h {\mathbf{B}}_h) \times {\mathcal{P}}_{\mathrm{f\rightarrow v}}({\mathbf{B}_h}) \right) \right]_{\mathcal{N}_h} $. See below.}
\\
At the discrete level, the total magnetic energy is expressed as:
\begin{equation}
E_{\mathbf{B}_h} := \frac{1}{2 \mu_0} \left[ \mathbf{B}_h , \mathbf{B}_h \right]_{\mathcal{F}_h} .
\end{equation}
We then have:
\begin{align}
\dfrac{d E_{\mathbf{B}_h}}{dt} = & \frac{1}{ \mu_0} \left[ \mathbf{B}_h , \dfrac{\partial {\B}_h}{\partial t} \right]_{\mathcal{F}_h} ,  \\
= & - \frac{1}{ \mu_0} \left[ \mathbf{B}_h , {\mathrm{Curl}_h}\, \boldsymbol{\tau}_h \right]_{\mathcal{F}_h} 
\\
= & - \frac{1}{ \mu_0} \left[ \underline{\widetilde{\mathrm{Curl}}}_h\, {\mathbf{B}}_h ,  \boldsymbol{\tau}_h \right]_{\mathcal{E}_h}
\\
= & - \frac{1}{ \mu_0} \left[ \underline{\widetilde{\mathrm{Curl}}}_h\, {\mathbf{B}}_h , \widetilde{\mathrm{Curl}}_h\, {\mathbf{B}}_h + {\mathrm{Grad}_h}({\Phi}_h) - {\mathcal{R}}_{\mathrm{v\rightarrow e}} \left({{\V}}_{h}\times {\mathcal{P}}_{\mathrm{f\rightarrow v}}({{\B}}_h)\right) \right]_{\mathcal{E}_h}
\\
= & - \frac{1}{ \mu_0} \left[ {\mathbf{B}}_h , {\mathrm{Curl}}_h\, \widetilde{\mathrm{Curl}}_h\, {\mathbf{B}}_h +  \cancelto{0}{ {\mathrm{Curl}}_h\, {\mathrm{Grad}_h}({\Phi}_h)} \quad \, \right]_{\mathcal{F}_h} 
\nonumber \\
& + \frac{1}{ \mu_0} \left[ \underline{\widetilde{\mathrm{Curl}}}_h\, {\mathbf{B}}_h , {\mathcal{R}}_{\mathrm{v\rightarrow e}} \left({{\V}}_{h}\times {\mathcal{P}}_{\mathrm{f\rightarrow v}}({{\B}}_h)\right) \right]_{\mathcal{E}_h}
.
\end{align}
\fi