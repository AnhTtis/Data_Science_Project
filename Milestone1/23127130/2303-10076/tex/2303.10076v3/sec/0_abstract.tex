% \begin{abstract}
% The ABSTRACT is to be in fully justified italicized text, at the top of the left-hand column, below the author and affiliation information.
% Use the word ``Abstract'' as the title, in 12-point Times, boldface type, centered relative to the column, initially capitalized.
% The abstract is to be in 10-point, single-spaced type.
% Leave two blank lines after the Abstract, then begin the main text.
% Look at previous \confName abstracts to get a feel for style and length.
% \end{abstract}

\begin{abstract}
The task of estimating 3D occupancy from surrounding-view images is an exciting development in the field of autonomous driving, following the success of Bird's Eye View (BEV) perception. This task provides crucial 3D attributes of the driving environment, enhancing the overall understanding and perception of the surrounding space. In this work, we present a simple attempt for 3D occupancy estimation, which is a CNN-based framework designed to reveal several key factors for 3D occupancy estimation, such as network design, optimization, and evaluation. In addition, we explore the relationship between 3D occupancy estimation and other related tasks, such as monocular depth estimation, stereo matching, and BEV perception (3D object detection and map segmentation), which could advance the study on 3D occupancy estimation. For evaluation, we propose a simple sampling strategy to define the metric for occupancy evaluation, which is flexible for current public datasets. Moreover, we establish a new benchmark in terms of the depth estimation metric, where we compare our proposed method with monocular depth estimation methods on the DDAD and Nuscenes datasets and achieve competitive performance. 
% The relevant code will be available in \href{https://github.com/GANWANSHUI/SimpleOccupancy}{https://github.com/GANWANSHUI/SimpleOccupancy}.
% We demonstrate this baseline work with extensive experiments to reduce the burden for future research in this field.
\end{abstract}