%Version 2.1 April 2023
% See section 11 of the User Manual for version history
%
%%%%%%%%%%%%%%%%%%%%%%%%%%%%%%%%%%%%%%%%%%%%%%%%%%%%%%%%%%%%%%%%%%%%%%
%%                                                                 %%
%% Please do not use \input{...} to include other tex files.       %%
%% Submit your LaTeX manuscript as one .tex document.              %%
%%                                                                 %%
%% All additional figures and files should be attached             %%
%% separately and not embedded in the \TeX\ document itself.       %%
%%                                                                 %%
%%%%%%%%%%%%%%%%%%%%%%%%%%%%%%%%%%%%%%%%%%%%%%%%%%%%%%%%%%%%%%%%%%%%%

%%\documentclass[referee,sn-basic]{sn-jnl}% referee option is meant for double line spacing

%%=======================================================%%
%% to print line numbers in the margin use lineno option %%
%%=======================================================%%

%%\documentclass[lineno,sn-basic]{sn-jnl}% Basic Springer Nature Reference Style/Chemistry Reference Style

%%======================================================%%
%% to compile with pdflatex/xelatex use pdflatex option %%
%%======================================================%%

%%\documentclass[pdflatex,sn-basic]{sn-jnl}% Basic Springer Nature Reference Style/Chemistry Reference Style


%%Note: the following reference styles support Namedate and Numbered referencing. By default the style follows the most common style. To switch between the options you can add or remove “Numbered” in the optional parenthesis. 
%%The option is available for: sn-basic.bst, sn-vancouver.bst, sn-chicago.bst, sn-mathphys.bst. %  
 
%%\documentclass[sn-nature]{sn-jnl}% Style for submissions to Nature Portfolio journals
%%\documentclass[sn-basic]{sn-jnl}% Basic Springer Nature Reference Style/Chemistry Reference Style
% \documentclass[sn-mathphys,Numbered]{sn-jnl}% Math and Physical Sciences Reference Style
%%\documentclass[sn-aps]{sn-jnl}% American Physical Society (APS) Reference Style
%%\documentclass[sn-vancouver,Numbered]{sn-jnl}% Vancouver Reference Style
%%\documentclass[sn-apa]{sn-jnl}% APA Reference Style 
%%\documentclass[sn-chicago]{sn-jnl}% Chicago-based Humanities Reference Style
%%\documentclass[default]{sn-jnl}% Default
% \documentclass[default,iicol]{sn-jnl}% Default with double column layout
\documentclass[default,iicol]{sn-jnl}% Default with double column layout

%%%% Standard Packages
%%<additional latex packages if required can be included here>

\usepackage{graphicx}%
\usepackage{multirow}%
\usepackage{amsmath,amssymb,amsfonts}%
\usepackage{amsthm}%
\usepackage{mathrsfs}%
\usepackage[title]{appendix}%
\usepackage{xcolor}%
\usepackage{textcomp}%
\usepackage{manyfoot}%
\usepackage{booktabs}%
\usepackage{algorithm}%
\usepackage{algorithmicx}%
\usepackage{algpseudocode}%
\usepackage{listings}%
\usepackage{geometry}

\usepackage{bbding}
\usepackage{times}
\usepackage{epsfig}


% \usepackage[numbers]{natbib} 



%%%%

%%%%%=============================================================================%%%%
%%%%  Remarks: This template is provided to aid authors with the preparation
%%%%  of original research articles intended for submission to journals published 
%%%%  by Springer Nature. The guidance has been prepared in partnership with 
%%%%  production teams to conform to Springer Nature technical requirements. 
%%%%  Editorial and presentation requirements differ among journal portfolios and 
%%%%  research disciplines. You may find sections in this template are irrelevant 
%%%%  to your work and are empowered to omit any such section if allowed by the 
%%%%  journal you intend to submit to. The submission guidelines and policies 
%%%%  of the journal take precedence. A detailed User Manual is available in the 
%%%%  template package for technical guidance.
%%%%%=============================================================================%%%%

%\jyear{2021}%

%% as per the requirement new theorem styles can be included as shown below
\theoremstyle{thmstyleone}%
\newtheorem{theorem}{Theorem}%  meant for continuous numbers
%%\newtheorem{theorem}{Theorem}[section]% meant for sectionwise numbers
%% optional argument [theorem] produces theorem numbering sequence instead of independent numbers for Proposition
\newtheorem{proposition}[theorem]{Proposition}% 
%%\newtheorem{proposition}{Proposition}% to get separate numbers for theorem and proposition etc.

\theoremstyle{thmstyletwo}%
\newtheorem{example}{Example}%
\newtheorem{remark}{Remark}%

\theoremstyle{thmstylethree}%
\newtheorem{definition}{Definition}%

\raggedbottom
%%\unnumbered% uncomment this for unnumbered level heads

\begin{document}

\title[Article Title]{A Simple Framework for 3D Occupancy Estimation in Autonomous Driving}

%%=============================================================%%
%% Prefix	-> \pfx{Dr}
%% GivenName	-> \fnm{Joergen W.}
%% Particle	-> \spfx{van der} -> surname prefix
%% FamilyName	-> \sur{Ploeg}
%% Suffix	-> \sfx{IV}
%% NatureName	-> \tanm{Poet Laureate} -> Title after name
%% Degrees	-> \dgr{MSc, PhD}
%% \author*[1,2]{\pfx{Dr} \fnm{Joergen W.} \spfx{van der} \sur{Ploeg} \sfx{IV} \tanm{Poet Laureate} 
%%                 \dgr{MSc, PhD}}\email{iauthor@gmail.com}
%%=============================================================%%

\author[1,2]{\fnm{Wanshui} \sur{GAN}}\email{wanshuigan@gmail.com}

\author[3]{\fnm{Ningkai} \sur{Mo}}\email{nk.mo19941001@gmail.com}
% \equalcont{These authors contributed equally to this work.}

\author[4]{\fnm{Hongbin} \sur{Xu}}\email{hongbinxu1013@gmail.com}
% \equalcont{These authors contributed equally to this work.}

\author[1,2]{\fnm{Naoto} \sur{Yokoya}}\email{yokoya@k.u-tokyo.ac.jp}
\equalcont{Corresponding author.}

\affil[1]{\orgname{The University of Tokyo}, \orgaddress{ \state{Tokyo}, \country{Japan}}}

\affil[2]{\orgname{RIKEN }, \orgaddress{\state{Tokyo}, \country{Japan}}}

\affil[3]{\orgname{Shenzhen Institute of Advanced Technology, Chinese Academy of Sciences}, \orgaddress{ \state{Shenzhen}, \country{China}}}

\affil[4]{\orgname{South China University of Technology}, \orgaddress{\state{Guangzhou}, \country{China}}}

%%==================================%%
%% sample for unstructured abstract %%
%%==================================%%

\abstract{The task of estimating 3D occupancy from surrounding-view images is an exciting development in the field of autonomous driving, following the success of Bird's Eye View (BEV) perception. This task provides crucial 3D attributes of the driving environment, enhancing the overall understanding and perception of the surrounding space. In this work, we present a simple framework for 3D occupancy estimation, which is a CNN-based framework designed to reveal several key factors for 3D occupancy estimation, such as network design, optimization, and evaluation. In addition, we explore the relationship between 3D occupancy estimation and other related tasks, such as monocular depth estimation and 3D reconstruction, which could advance the study of 3D perception in autonomous driving. For evaluation, we propose a simple sampling strategy to define the metric for occupancy evaluation, which is flexible for current public datasets. Moreover, we establish a benchmark in terms of the depth estimation metric, where we compare our proposed method with monocular depth estimation methods on the DDAD and Nuscenes datasets and achieves competitive performance. The relevant code is available in \href{https://github.com/GANWANSHUI/SimpleOccupancy}{https://github.com/GANWANSHUI/SimpleOccupancy}.}

% \begin{abstract}
The current study investigated possible human-robot kinaesthetic interaction using a variational recurrent neural network model, called PV-RNN, which is based on the free energy principle.
Our prior robotic studies using PV-RNN showed that the nature of interactions between top-down expectation and bottom-up inference is strongly affected by a parameter, called the meta-prior, which regulates the complexity term in free energy.
% The current study examines how the behaviours of robots alter by changing the meta-prior $w$ in human-robot kinaesthetic interaction.
The current study examines how changing the meta-prior $w$ in the interaction phase affects the counter force generated when an experimenter attempts to induce movement pattern transitions familiar to the robot through its prior training.
The study also compares the counter force generated when trained transitions are induced by a human experimenter and when untrained transitions are induced.
Our experimental results indicated that (1) the human experimenter needs more/less force to induce trained transitions when $w$ is set with larger/smaller values, (2) the human experimenter needs more force to act on the robot when he attempts to induce untrained as opposed to trained movement pattern transitions.
Our analysis of time development of essential variables and values in PV-RNN during bodily interaction clarified the mechanism by which gaps in actional intentions between the human experimenter and the robot can be manifested as reaction forces between them.


%% Hiroki writing 2022-11-4
%Current study investigates the dynamics of the latent states during human-robot kinaesthetic interaction using PV-RNN.
%We have achieved to observe and analyse the internal state of an RNN model based on the free energy principle, during real-time human-robot interaction.
%Essential characteristics observed in the previous study of this variational recurrent neural network model, PV-RNN, is that by changing a meta prior $w$, the balance between the top-down intention and the bottom-up perceptual reality changes.
%In the current study, we examined how changing the weighting parameter $w$ between accuracy and complexity in free energy principle affects the humanoid robot's behaviour through human-robot interaction. We have conducted some human-robot kinaesthetic interaction experiments with various $w$ and quantitatively analysed the latent variable and the force applied to the humanoid robot. We have observed that the force required to change the robot's intention has increased, both when the top-down intention was strengthened by changing the $w$ and when corresponding switch of its primitive was against the experience of the RNN during its training. The study confirms through quantitative analysis that by increasing or decreasing the $w$ in PV-RNN, humanoid robot leads or follows the human counterpart during the human-robot kinaesthetic interaction.

\begin{comment}
Comment from Jun #2
・最後にQualitativeな結果(インパクト)が欲しい
・Current study investigates the problem on~と書き出すのが一般的
・最初の一文と最後の一文を対応させる
・最後の一文はもう少しAbstractかつ包括的に
\end{comment}

\begin{comment}
Comment from Jun #1
We investigated how the kinaesthetic human-robot interaction can affect the internal state of a model based on the free energy principle. 
=> how the internal state is affected is not the most important point in this study. This part should be rewritten.

The key function of this variational recurrent neural network model, PV-RNN, is that by changing a meta prior $w$, it takes a balance between the "complexity” term and the ”accuracy” term which corresponds to a top-down intention and a bottom-up perceptual reality in the free energy principle, respectively. 
=> This is not key function of PV-RNN. It is an essential characteristics observed in the previous study. The grammar after $w$ is something strange. Rewrite these.

This research has conducted a human-robot interaction experiment with a robotic agent in a kinaesthetic sense.
=> The sentence is not good. "in a kinaesthetic sense" is grammatically wrong.
MODIFIED => "In the current study human-robot interaction experiments using the kinaesthetic sense were conducted."

We investigated that when human forces the agent to switch primitives from one to another, larger force was required both when the human intention is conflictive against the top-down the intention of the agent and when the agent has a stronger top-down intention by modifying the $w$.
=> You should write the essential results of the experiments rather than what we investigated and also how these results could contribute to the studies on human-robot interaction.
\end{comment}

\end{abstract} 

%%================================%%
%% Sample for structured abstract %%
%%================================%%

% \abstract{\textbf{Purpose:} The abstract serves both as a general introduction to the topic and as a brief, non-technical summary of the main results and their implications. The abstract must not include subheadings (unless expressly permitted in the journal's Instructions to Authors), equations or citations. As a guide the abstract should not exceed 200 words. Most journals do not set a hard limit however authors are advised to check the author instructions for the journal they are submitting to.
% 
% \textbf{Methods:} The abstract serves both as a general introduction to the topic and as a brief, non-technical summary of the main results and their implications. The abstract must not include subheadings (unless expressly permitted in the journal's Instructions to Authors), equations or citations. As a guide the abstract should not exceed 200 words. Most journals do not set a hard limit however authors are advised to check the author instructions for the journal they are submitting to.
% 
% \textbf{Results:} The abstract serves both as a general introduction to the topic and as a brief, non-technical summary of the main results and their implications. The abstract must not include subheadings (unless expressly permitted in the journal's Instructions to Authors), equations or citations. As a guide the abstract should not exceed 200 words. Most journals do not set a hard limit however authors are advised to check the author instructions for the journal they are submitting to.
% 
% \textbf{Conclusion:} The abstract serves both as a general introduction to the topic and as a brief, non-technical summary of the main results and their implications. The abstract must not include subheadings (unless expressly permitted in the journal's Instructions to Authors), equations or citations. As a guide the abstract should not exceed 200 words. Most journals do not set a hard limit however authors are advised to check the author instructions for the journal they are submitting to.}

\keywords{3D Occupancy Estimation, Depth Estimation, 3D Reconstruction, Volume Rendering}

%%\pacs[JEL Classification]{D8, H51}

%%\pacs[MSC Classification]{35A01, 65L10, 65L12, 65L20, 65L70}

\maketitle
   

\section{Introduction}
\label{sec:intro}
\begin{figure}[t]
\begin{center}
    \includegraphics[width=1\linewidth]{figures/teaser.pdf}
\end{center}
\vspace{-0.1in}
\caption{\textbf{{\em Foggy} vs {\em Clear} NeRF.} Our \ournerf gets rid of reconstruction errors manifested as foggy ``floaters" in the density volume without additional input or significant computational overhead. 
%
Below are density profiles along a given ray before and after our geometry correction procedure, where we discard density peaks corresponding to floaters.
}
\label{fig:teaser}
\vspace{-0.2in}
\end{figure}



%The emergence of 
Neural Radiance Fields (NeRFs)~\cite{mildenhall2020nerf}  %and its variants 
have made revolutionary contributions in %photo-realistic 
novel view synthesis~\cite{barron2021mip,barron2022mip}, 
autonomous driving~\cite{rematas2022urban,tancik2022block}, digital human~\cite{hong2022headnerf,zhao2022humannerf}, and 3D content generation~\cite{eg3d,poole2022dreamfusion,lin2022magic3d}.
%by leveraging a multi-layer perceptron (MLP) to implicitly model the mapping from input 5D coordinates (i.e., 3D coordinates $\mathbf{x} = (x,y,z)$ and 2D viewing directions $\mathbf{d}=(\theta,\phi)$) to volume density $\sigma$ and view-dependent emitted radiance color $\mathbf{c} = (r,g,b)$. 
%
%They then use traditional volume rendering mechanisms on the obtained continuous 5D function (i.e., MLP) to generate novel views. 
To date, unfortunately, most NeRF-based methods encounter challenges when tackling large-scale cluttered scenes (e.g., Fig.~\ref{fig:teaser}):
\begin{enumerate}[leftmargin=0.16in, topsep=2pt,itemsep=-1ex,partopsep=1ex,parsep=1ex]
\item Input observations used for NeRF are often too sparse  compared to forward-facing or synthetic looking-inward scenes;
%\item Recovering fine-grained objects within a large volume is challenging for NeRF; %in capturing details accurately.
\item View-dependent visual effects give rise to ambiguity, resulting in a ``foggy" density field as shown in Fig.~\ref{fig:teaser}. 
%
Such artifacts are particularly pronounced in indoor scenes strewn with view-dependent appearances, such as specular highlights, glossy surface reflections from man-made objects. 
\end{enumerate}

Despite attempts to enhance NeRF's rendering quality given suboptimal input, such as using 3D conical frustums~\cite{barron2021mip,barron2022mip}, physically-grounded augmentations~\cite{chen2022aug}, and misalignment correction~\cite{jiang2022alignerf},  these challenges have yet to be fully resolved.
%
Depth supervision~\cite{deng2022depth, wei2021nerfingmvs} or proxy geometry~\cite{xu2021scalable,wu2022scalable} images can help alleviate the challenges in handling large-scale with sparse input, at the expense of %but they come at the cost of requiring 
expensive pre-processing or additional input.
%
Another line of work~\cite{wang2021neus, oechsle2021unisurf, wang2022neuris} achieves better reconstruction of surface geometry by using signed distances instead of volume density as scene representation. However, they sacrifice the ability to synthesize photo-realistic novel views.

%We observe that NeRF has been suffering from foggy ``floater" artifacts in large-scale cluttered scenes.
%
%Such artifacts are particularly pronounced in indoor scenes strewn with view-dependent appearances from man-made objects. 
%
To address the above issues, we propose an extension to NeRF, dubbed as {\bf \ournerf}, which enforces effective {\em appearance} and {\em geometry} constraints conducive to accurate colors and 3D densities estimation. We believe \ournerf can contribute beyond novel view synthesis, such as NeRF object detection~\cite{hu2022nerf}, NeRF object segmentation~\cite{zhi2021place, liu2022unsupervised, fan2022nerf,ren2022neural}, and NeRF registration~\cite{goli2022nerf2nerf}, where the rooms for improvement are substantial if more accurate color and density estimation are available.

Correspondingly, there are two steps in \ournerf. First, for appearance correction, the view-independent and view-dependent color components are predicted from the underlying 3D scene, which is combined to produce the final color estimation (Fig.~\ref{fig:toaster}).
%
The view-independent component (diffuse color and shading) captures the overall scene color, while the view-dependent component (highlights or reflections) captures color variations due to changes in viewing angle.
%
\ournerf then discards these view-dependent appearances in the training views to prevent them from interfering with the density estimation.
%
Second, a simple and effective geometry correction procedure will be performed to further eliminate the foggy ``floaters" or density errors. This geometry correction procedure is based on an assumption in line with traditional ray tracing in computer graphics.
\begin{comment}
% xh: basically copying method
On the other hand, ClearNeRF performs a geometric correction procedure performed on each traced ray during inference to refine the density estimation and better tackle the floater artifacts. 
%
The geometry correction procedure assumes that there should only be one salient peak along each traced ray during NeRF inference. 
Only the salient peak closest to the ray origin (the camera center) corresponds to  true geometry while the others will be manifested as foggy floaters hovering in the density volume. 
%
This assumption is in line with traditional ray tracing in computer graphics where in the absence of noise, only one intersection per ray should be returned to indicate the closest ray-object intersection.
%
\end{comment}
%%%%%%%%%%%
%As shown in Fig.~\ref{fig:teaser}, when reconstructing an indoor scene with sparse input and highly view-dependent objects, NeRF produces severe floating artifacts due to its attempt to explain view-dependent appearances.
%
Experiments verify that our proposed \ournerf can effectively get rid of floater artifacts without additional input.% or significant computational overhead. 


In summary, our contributions include the following:
\begin{itemize}[leftmargin=0.16in, topsep=2pt,itemsep=-1ex,partopsep=1ex,parsep=1ex]
    \item We propose a concise method for decomposing view-independent and view-dependent appearance during NeRF training and eliminate the interference of view-dependent appearance.
    \item We propose a geometric correction procedure performed on each traced ray during inference to refine the density estimation and better tackle the floater artifacts.
    \item Extensive experiments and ablations verify the effectiveness of our core designs and results in improvements over the vanilla NeRF and other state-of-the-art alternatives.
    %without additional computational resources or other inputs.
\end{itemize}




\section{Related Work}
\label{sec:related_work}
\subsection{Co-Speech Gesture Synthesis}
The early approaches for generating co-speech gestures often involve creating linguistic rules to translate speech input into a sequence of pre-collected gesture segments, which are typically referred to as rule-based methods \cite{cassell1994rulefullbody,cassell2001beat,kipp2004gesture,kopp2006bml}. \citet{wagner2014rulereview} provide a comprehensive review of these methods. Rule-based methods produce interpretable and controllable results, but creating gesture datasets and rules requires significant effort. To alleviate the manual effort of designing rules in rule-based methods, data-driven approaches have gradually become predominant in this field. \citet{nyatsanga2023data_driven_gesture_survey} offer a thorough survey of these methods. Early data-driven approaches aim to directly learn mapping rules from data through statistical models \cite{neff2008videogesture,levine2009prosodygesture,levine2010gesturecontroller} and combine them with predefined gesture units for gesture generation. Later, the powerful modeling capability of deep neural networks makes it possible to train complex end-to-end models using raw speech-gesture data directly. One option is deterministic models, such as MLP \cite{kucherenko2020gesticulator}, CNN \cite{habibie2021videogesture}, RNN \cite{yoon2019robot,yoon2020trimodalgesture,bhattacharya2021affectivegesture,liu2022hierarchicalgesture}, and Transformer \cite{bhattacharya2021text2gestures}. Another choice is generative models, including flow-based models \cite{alexanderson2020stylegesture,ye2022styleflowgesture}, VAEs \cite{li2021audio2gesture,ghorbani2022zeroeggs}, and VQ-VAE \cite{yi2022talkshow,yazdian2022gesture2vec,liu2022vqgesturevideo}. Due to the inherent many-to-many relationship between speech and gesture, end-to-end models can generate natural-looking gestures but face challenges in ensuring content matching between speech and generated gestures \cite{yoon2022genea}. To address this issue, some neural systems aim to explicitly model both rhythm and semantics from the perspective of model structure \cite{kucherenko2021speech2properties2gestures,ao2022rhythmicgesticulator,liu2022disco} or training supervision strategy \cite{liang2022seeg}. Furthermore, hybrid systems, such as the combination of deep features and motion graphs \cite{zhou2022gesturemaster}, have been proposed to harness the advantages of different approaches. Recently, diffusion models \cite{sohldickstein2015diffusion,song2020improvedscore,ho2020ddpm} have demonstrated impressive results in image synthesis \cite{ramesh2022dalle2} and human motion generation \cite{tevet2022humanmotiondiffusion, zhang2022motiondiffuse}. Inspired by these works, our system adapts the latent diffusion model \cite{rombach2022latentdiffusion} for the co-speech gesture generation task and achieves appealing results.

\subsection{Style Control for Human Motion}
A typical approach to style control for human motion involves specifying a motion clip as a reference and transferring the reference clip's style to the source motion. This task is also known as \emph{style transfer}. Early works in motion style transfer integrate traditional machine learning techniques with manually defined features to infer motion styles \cite{hsu2005motion_style_translation,ma2010motion_style_transfer,xia2015realtime_motion_style_transfer,yumer2016spectral_motion_style_transfer}. Recently, deep learning-based methods have significantly enhanced motion quality. \citet{holden2016deepmotion} first propose a learning framework enabling motion style control through optimization in the motion manifold space. \citet{du2019stylemotioncvae} improve transfer efficiency by training a conditional VAE. \citet{mason2018few-shot_motion_style_transfer} use few-shot learning to generate stylized locomotion. \citet{aberman2020adain} employ a temporally invariant adaptive instance normalization (AdaIN) layer for target style injection, eliminating the need for paired data during training. \citet{wen2021stylemotionflow} achieve unsupervised style transfer using a flow model. \citet{jang2022motionpuzzle} introduce a method capable of controlling styles for individual body parts.

Previous co-speech gesture synthesis systems with style control can be categorized based on whether or not they require style labels. For methods needing labeled data, early works can only learn an individual style for one generator \cite{levine2010gesturecontroller,neff2008videogesture,ginosar2019stylegesture}. \citet{ahuja2022lowresource} propose a strategy that efficiently adapts the source generator to another speaker style using low-resource data. Some works learn a speaker style embedding space with labeled speaker-motion data, enabling gesture style control by sampling from this space \cite{ahuja2020stylegesture,yoon2020trimodalgesture,bhattacharya2021affectivegesture}. \citet{alexanderson2020stylegesture} aimat controlling fine-grained styles, such as gesturing speed and spatial scope, using preprocessed control signal-motion data. Their later work \cite{alexanderson2022diffusiongesture} utilizes a diffusion model for audio-driven motion synthesis, achieving label-based style control by training the model on labeled data. For methods not requiring style labels, \citet{habibie2022motionmatching} propose a motion matching framework to achieve flexible style control. Other studies achieve arbitrary style control by imitating an example given as a video \cite{liu2022hierarchicalgesture} or a motion clip \cite{ghorbani2022zeroeggs,ye2022styleflowgesture,kuriyama2022tokenizedgestures}.  In this work, we utilize a CLIP-based encoder to extract a style embedding from an arbitrary text prompt and incorporate it into the generator via an AdaIN layer, guiding the synthesis of stylized gestures. Our system supports fine-grained multimodal style prompts as opposed to label-based style control. It employs a self-supervised learning scheme and eliminates the need for labeled data. Additionally, we use an autoregressive model rather than a parallel model, making it potentially suitable for real-time applications.
\section{Method}
\label{sec: method}
% This section introduces the rendering pipeline of our proposed hierarchical compositional scene. 
% our pipeline consists of three processes, including decomposing the text into editable 3D layout, rendering the compositional views with local (object) NeRFs and global (scene) NeRF and the joint optimization on these hierarchical 3D representations.

% Note that the transformation between the object and the scene frame is defined by ${p}_o$ and ${D}_o$. 
%
% Next, we build a residual connection to add ${\sigma}_o$ and the referenced global color, and the rendering result will be used to calculate the SDS loss based on the global text.  
% Fig.~\ref{fig:framework} illustrates our pipeline, which consists of three main components, including the editable 3D scene layout based on multi-object text (Sec.~\ref{ssec:layout}), the scene rendering pipeline that composites the predictions from all local NeRFs (Sec.~\ref{ssec:render}), and the joint optimization on both local and global representation models (Sec.~\ref{sec:optimization}).
% To elaborate, our editable 3D scene layout represents a global frame of the scene by decomposing it into a set of local frames, where each is parameterized by a local NeRF, a 3D bounding box, and a corresponding local text prompt.
% For instance, the text prompt `A teddy bear and a stuffed monkey sit side by side' is interpreted as a 3D scene layout, as shown in Fig.~\ref{fig:framework}.  
% The whole 3D layout, \ie, scene frame, consists of two 3D bounding boxes, \ie local frames \#1 and \#2, with specific local text prompts, \ie, `a teddy bear' and `a stuffed monkey'. 
% %
% To render the scene view, we first calculate the ray-box intersections between the boxes and rays $({\boldsymbol{r}}_o, \boldsymbol{\phi}_d, {\boldsymbol{\theta}}_d)$, where the ${\boldsymbol{r}}_o$ is the ray origin and the $({\boldsymbol{r}}_o, \boldsymbol{\phi}_d)$ is its direction.
% Then, to infer each object's properties in local NeRFs, we sample the global points $({\boldsymbol{x}}_g, {\boldsymbol{y}}_g, {\boldsymbol{z}}_g)$ in the global frame within the ray-box intersection intervals and project them into the normalized local location $({\boldsymbol{x}}_l, {\boldsymbol{y}}_l, {\boldsymbol{z}}_l)$ in the local frame.
% %
% Given the local sampling points $({\boldsymbol{x}}_l, {\boldsymbol{y}}_l, {\boldsymbol{z}}_l)$, the implicit local NeRF ${\boldsymbol{\theta}}_l$ outputs four pseudo-color channels ${\boldsymbol{C}}_l$ and density $\boldsymbol{\sigma}$, which can be used to render a local view of the local frame to match its local text prompt.
% %
% We further calibrate the predicted pseudo-color $\boldsymbol{C}_l$ from local frames by adding the global embeddings ${\boldsymbol{emb}}_g$ to improve the global view consistency.
% Then, the calibrated predictions after composition are used to reconstruct the scene view by volumetric rendering along the rays.
% %
% Lastly, the rendered views based on local and global frames are guided by score distillation sampling loss $\nabla \mathcal{L}_{\text{SDS}}$~\cite{poole2022dreamfusion} to optimize all the learnable parameters. 
To resolve the issue of guidance collapse, our principal strategy is to \textit{decompose the scene into reusable components and compose/recompose them into a unified and consistent one}.
This enables flexible control over the generated content with direct use of prompts and box layouts, as illustrated in \cref{fig:teaser}.
%
Our proposed CompoNeRF confers several key benefits:
1) \textbf{Semantic Coherence}: It reliably creates 3D objects with detailed textures and global consistency, exemplified by authentic light interactions, such as reflections on the bed surface.
2) \textbf{Modularity and Reusability}: CompoNeRF functions as an ensemble of independently trained NeRF models. These can be efficiently stored and later retrieved from a cached dataset, enabling their reuse in various cases.
3) \textbf{Editability}: Our approach allows for flexible scene modification, such as interchanging the lamp for a vase filled with sunflowers or altering its scale, by simply adjusting the box dimensions for later finetuning. This feature enhances flexibility and creative possibilities. 


% Furthermore, the usage of layout boxes enables more flexible control over the generated content compared with the intricate sketch shape in Latent-NeRF\cite{metzer2022latent}. 
\begin{figure*}[t]
    \centering
    \includegraphics[width=0.9\linewidth]{figures/method.pdf}
    % \vspace{-12pt}
    \caption{\textbf{Framework Overview}.
The CompoNeRF model unfolds in three stages: 1) Editing 3D scene, which initiates the process by structuring the scene with 3D boxes and textual prompts; 2) Scene rendering, which encapsulates the composition/recomposition process, facilitating the transformation of NeRFs to a global frame, ensuring cohesive scene construction. Here, we specify design choices between density-based or color-based(without refining density) composition; 3) Joint Optimization, which leverages textual directives to amplify the rendering quality of both global and local views, while also integrating revised text prompts and NeRFs for refined scene depiction.
  % The model is structured into three components: Composition, Decomposition, and Recomposition. Composition deals with the foundational setup, detailed with choices for density-based and color-based composition. Decomposition utilizes the modularity of the CompoNeRF feature, caching each NeRF module offline for efficient recalibration. Recomposition reuses these cached NeRFs and adjusts the semantic context, providing a revised output with the inclusion of the offline NeRF enhancements.
    % Our model consists of two branches where the upper part is individual NeRFs, and the lower part denotes global calibration with our tailored composition model. The specific designs for density-based and color-based composition modules are highlighted. 
    % CompoNeRF consists of three parts: 1). The editable 3D scene layout configures the scene representations with 3D boxes and text prompts; 2).  The scene rendering includes the global calibration and the compositional process; 3). The joint optimization applies global and local text guidance on global and local render views.
    % The global frame (scene space) contains a set of local frames. Each is  represented by a local NeRF associated with a 3D box and text prompt defined by the editable 3D layout.
    % The scene view is volumetric rendered by sampling the points $({\boldsymbol{x}}_g, \boldsymbol{y}_g, \boldsymbol{z}_g)$ intersected with any local frame along the ray $(\boldsymbol{r}_o, {\boldsymbol{\phi}}_d, \boldsymbol{\theta}_d)$.
    % The sampling points are first inferred through the local NeRF with the local frame locations $({\boldsymbol{x}}_l, \boldsymbol{y}_l, \boldsymbol{z}_l)$ projected from the global location $({\boldsymbol{x}}_g, \boldsymbol{y}_g, \boldsymbol{z}_g)$.
    % And then, all the local predictions are calibrated by a global MLP with conditional input to render the scene view.
    % During the optimization, the text guidance is applied to both local views predicted by local frames only and global views predicted by the composition of all local frame predictions.
    }
    \label{fig:framework}
    % \vspace{-8pt}
\end{figure*}

\subsection{Preliminaries}
Defining individual object bounding boxes as \textit{local frames} and the overall scene coordinate system as the \textit{global frame}, we build the foundation of NeRF and diffusion processes.

\label{sec:background}
\noindent \textbf{3D Representation in Latent Space.}
Our methodology capitalizes on the state-of-the-art text-to-image generative model—Stable Diffusion as described by Rombach et al\cite{rombach2022high}.
We build upon the Latent-NeRF framework~\cite{metzer2022latent}, which computes latent colors for individual objects by considering their sample positions within a localized frame. Specifically, it maps a three-dimensional point in local coordinates \(\boldsymbol{x}_l = (x_l, y_l, z_l)\) to a volumetric density \(\boldsymbol{\sigma}_l\) and an associated color \(\boldsymbol{C}_l\), expressed as \((\boldsymbol{C}_l, \boldsymbol{\sigma}_l) = f_{\boldsymbol{\theta}_l}(x_l, y_l, z_l)\). Here, \(f\) represents a Multi-Layer Perceptron (MLP) characterized by parameters \(\boldsymbol{\theta}_l\).
 This NeRF-generated color is then assessed in the context of the Stable Diffusion model, using text prompts to guide NeRF toward spatially coherent inference with intricate context.
% to infer pseudo-color for each object using local NeRF.
% Specifically, the representation maps a point $\boldsymbol{x}_l = \left({x}_l, {y}_l, {z}_l\right)\in [-1, 1]$ in the local frame to its corresponding volumetric density $\boldsymbol{\sigma}_l$ and emitted color $\boldsymbol{C}_l$, \ie,  $\left(\boldsymbol{C}_l, {\boldsymbol{\sigma}_l}\right)=\boldsymbol{\theta}_{_l}\left({x_l}, {y}_l, {z}_l\right)$.
% The predicted pseudo-color is fed forward into the decoder of the Stable Diffusion model to obtain the final rendering result.

\noindent \textbf{Volume Rendering with Multiple Objects.}
% For each local frame $j$ with NeRF parameterized as $\theta_j$, we follow original NeRF design\cite{nerf} to integrate $(\boldsymbol{C}_l, \boldsymbol{\sigma}_l)$ of   sampled points from any hit ray $r_l=(\boldsymbol{o}_l, \boldsymbol{d}_l)$ by,
% For consistent scene rendering, object transmittance $T_k$ must be recalculated in the global frame based on independent properties inferred from local NeRFs. Hence, we sort predictions according to their distance to $\boldsymbol{o}_g$. 
% Similar to \cref{eq:volrend}, global color $\hat{\boldsymbol{C}}_g$ of ray $\boldsymbol{r}_g=(\boldsymbol{o}_g, \boldsymbol{d}_g)$ is predicted by the volumetric rendering integrating over $m$ objects,
We extend the volume rendering process to accommodate multiple objects by assigning each a local frame, denoted as $j$, with NeRF parameters $\boldsymbol{\theta}_{l, j}$. Drawing from the foundational NeRF approach \cite{nerf}, in each local frame, we integrate the color $\boldsymbol{C}_l$ and density $\boldsymbol{\sigma}_l$ for points $\boldsymbol{x}_l$ sampled along a ray $\boldsymbol{r}_l$, emanates from the camera origin $\boldsymbol{o}_l$ in direction $\boldsymbol{d}_l$. This is formalized in the predicted color integration for $\hat{\boldsymbol{C}}_l$ as:
{\setlength\abovedisplayskip{2pt}
\setlength\belowdisplayskip{2pt}
\begin{equation}
\label{eq:volrend}
{\hat{\boldsymbol{C}}_l}({\boldsymbol{r}_l})=\sum_{k=1}^{N} T_{l, k} \left(1-\exp \left(-\sigma_{l, k} \delta_k\right) \right) {\boldsymbol{C}}_{l,k},
\end{equation}}where $T_{l, k}=\exp \left(-\sum_{j=1}^{k-1} \sigma_{l,j} \delta_j\right)$ represents the transmittance to the $k$-th of total $N$ sample, calculated exponentially over the cumulative density along $\boldsymbol{r}_l$, and $\delta_k$ is the interval between adjacent samples.
%
To synthesize a coherent scene, we transition from processing individual local frames to a collective global frame. Within this global context, we reconcile object attributes inferred from their individual local NeRFs for refined $\boldsymbol{\sigma}_g, \boldsymbol{C}_g$ along with $T_{g, k}$. The samples $\boldsymbol{x}_g$ are ordered based on their spatial distances from the origin $\boldsymbol{o}_g$ following the coordinate transformation. We then express the volumetric rendering of a ray $\boldsymbol{r}_g$ integrating $m$ objects within the global frame as follows:
{
\setlength\abovedisplayskip{2pt}
\setlength\belowdisplayskip{2pt}
\begin{equation}
\label{eq:multi_volrend}
{\hat{\boldsymbol{C}}_g}({\boldsymbol{r}_g})=\sum_{k=1}^{m*N} T_{g, k} \left(1-\exp \left(-\sigma_{g, k} \delta_k\right) \right) {\boldsymbol{C}}_{g,k}. 
\end{equation}}

\noindent \textbf{Score Distillation Sampling.}
% During the SDS process, a noise image $\boldsymbol{X}_t$ is first generated by adding a sampled noise $\epsilon \sim \mathcal{N}(0, I)$ in noise level $t$ into a rendered view $\boldsymbol{X}$ from a NeRF.
To facilitate the conversion from text descriptions to 3D models, DreamFusion~\cite{poole2022dreamfusion} utilizes Score Distillation Sampling (SDS), leveraging the generative capabilities of a diffusion model, denoted as $\phi$, to guide the optimization of NeRF parameters, symbolized as $\boldsymbol{\theta}$.
%
Initially, SDS creates a noisy image $\boldsymbol{X}_t$ by infusing a randomly sampled noise $\epsilon$, which follows a normal distribution $\mathcal{N}(0, I)$, into a NeRF-rendered image $\boldsymbol{X}$ at a given noise level $t$.
The diffusion model $\phi$ then estimates the noise $\epsilon_\phi\left(\boldsymbol{X}_t, t, T\right)$ from this noisy image, conditioned by the noise level $t$ and an optional text prompt $T$. 
The key step in SDS involves calculating the gradient of the loss function, which measures the discrepancy between the estimated noise and the originally added noise:
{\setlength\abovedisplayskip{2pt}
\setlength\belowdisplayskip{2pt}
\begin{equation}
\label{eq:sds_loss}
\nabla_\theta \mathcal{L}_{\text{SDS}}(\boldsymbol{X}_t, T)=  w(t)\left(\epsilon_\phi\left(\boldsymbol{X}_t, t, T\right)-\epsilon\right),
\end{equation}}where $w(t)$ is a weighting function that adjusts the influence of the gradient based on the noise level. 
The gradients across all rendered views direct the update of $\boldsymbol{\theta}$, ensuring that the NeRF-generated images align with the text descriptions. Additionally, we incorporate the 'perturb and average' technique from SJC for more robust $\mathcal{L}_{\text{SDS}}$. For a comprehensive understanding of these methods, the reader is directed to the detailed explanations provided in \cite{poole2022dreamfusion,wang2022score}.

%
%
% \subsection{Editable 3D Scene Layout}
% \label{ssec:layout}
% The 3D scene layout explicitly combines language structures with 3D layouts in an editable way.
% Given the input text prompt $T$, the attribute-object pairs can be easily obtained based on user control.
% Note that the text prompt indicates the multi-object text prompt by default.
% % available for free in many structured representations, such as the constituency tree.
% As shown in Fig.~\ref{fig:framework}, we can extract multiple noun phrases with their binding attributes and map these local text prompts into corresponding regions.
% Specifically, we define the scene structure with $m$ local frames, each employs a local NeRF $\boldsymbol{\theta}_l$ as representation, the local text prompt $T_{l} \subseteq{T}$ and its spatial layout with 3D boxes $\mathbf{b} = \{\mathbf{p}, \mathbf{s}\} \in  \mathbb{R}^6$ of each object entity, where $\mathbf{p}=\{p_x, p_y, p_z\}$ refers to the center point and $\mathbf{s}=\{s_x, s_y, s_z\}$ denotes the box scale. 
% \textit{Our editable 3D layout is easy to be collected and edited with its simplicity, allowing for versatile and interactive user control by modifying the box's or text's properties to define a new scene}.
% Moreover, as depicted in Fig.~\ref{fig:teaser}, each component in a 3D scene layout can be replaced or re-composited with other trained local NeRFs, which is more friendly for flexible user editions compared with using only text prompts.
% We fine-tuned the new layout by global rendering, which enables scalable re-editing.
% Each relationship $r_k \in R$ is a triplet in a <subject-predictive object> format, where a subject node is. After we generate the scene graph from the complex prompts, we can sample the closest relationship with the 2d spatial layout as the initial 3D position. fine-tuned the new layout by global rendering, which enables scalable re-editing
%
% \subsection{Scene Rendering Pipeline}
% \label{ssec:render}
% In CompoNeRF, the scene images are rendered by a ray-casting approach following the design of NeRF.
% % Each ray to be cast is generated based on the camera pose, intrinsic, and transformation.
% The camera is defined by a pinhole camera model, casting a set of rays $(\boldsymbol{r}_o, \boldsymbol{\phi}_d, {\boldsymbol{\theta}}_d)=\boldsymbol{o}+t\boldsymbol{d}$ through each pixel on the frame of size $H \times W$, where the $\boldsymbol{r}_o \in  \mathbb{R}^3$ is the origin and the $(\boldsymbol{\phi}_d, \boldsymbol{\theta}_d)$ is the viewing direction.
% Along this ray, we sample all the points intersected with any layout box of local frames.
% For each hit sampled point, the color and volumetric density are computed through the local NeRF of the hit local frame.
% The ray color perdition is calculated by the differentiable integration applied on all the point-predicted colors and volumetric density along the ray.
%
% \noindent \textbf{Ray-box Intersection with Local Frames.}
% Given a ray $\boldsymbol{r}_i$, each box $\boldsymbol{b}_j$ of the local frame is applied with the AABB ray intersection test algorithm to check the intersections.
% When the ray $r_i$ is hit with a box $\boldsymbol{b}_j$ of the local frame, we use the entrance and exit points as near $\boldsymbol{t}_{in}$ and far $\boldsymbol{t}_{out}$ bounds to sample $N$ equidistant quadrature points, $
% \boldsymbol{t}_{i,j,n}=\frac{n-1}{N-1}\left(\boldsymbol{t}_{out}-\boldsymbol{t}_{in}\right)+\boldsymbol{t}_{in} , n \in \left[1, N\right]$
% % Despite each local frame only having a small number of hit rays compared to the scene, we observe that it is enough to represent each object accurately while maintaining short rendering times.
% Note that the coordinates of sampled points are first projected into normalized coordinates using the box scale of local frames to enable each local NeRF to learn the scale-independent representation.
% The bounding box $\mathbf{b}$ of the local frame in global coordinate can be transformed into a canonical bounding box by ${(\mathbf{b}} - \boldsymbol{p}) / \mathbf{s}$.
% Considering the rendering efficiency, we only calculate the valid points, interacted with the boxes, and set all the empty points with a constant background color.
%
% The appearance of a set object representations depends on its interaction with the scene and illumination which should be decided by the local frame location.
% To ensure the volumetric consistency, we only calibrate the emitted color with scene location, while the gradient still can be propagated.
% Since the overall color depends on both the global  positions $({x}_w, {y}_w, {z}_w)$ and ray directions $({\phi}_d, {\theta}_d)$, the global color embedding is learned based on both the positions and ray directions.
% Since the overall color depends on both the global  positions $({x}_w, {y}_w, {z}_w)$ and ray directions $({\phi}_d, {\theta}_d)$, the global color embedding is learned based on both the positions and ray directions.
% \subsection{The Proposed CompoNeRF}
% \subsubsection{Composition Module}
% CompoNeRF aims to composite multiple NeRFs to reconstruct multi-object scenes with both box and prompt guidance.
% %
% Our framework, as shown in \cref{fig:framework}, applies the AABB ray intersection test algorithm to check for intersections on each box in the global frame. We then samples $\boldsymbol{x}_g$ within the ray box intervals, and project them to $\boldsymbol{x}_l$ to infer  $\left(\boldsymbol{C}_l, {\boldsymbol{\sigma}_l}\right)$ in separate NeRF models. 
% %
% We then utilize volume rendering to obtain rendered views for each local frame respectively. 
% %
% After that, they would be passed on to our tailored composition Module to infer 
% $\left(\boldsymbol{C}_g, {\boldsymbol{\sigma}_g}\right)$
% for global rendering. 
% Next, we match local and global texts with their corresponding image outputs by SDS losses. 
% We also support recomposition by passing samples from cached models into $\boldsymbol{x}_l$ to continue the above process.
\begin{figure}[t!]
    \centering
    \includegraphics[width=\linewidth]{figures/abls.pdf}
    % \vspace{-22pt}
    % \caption{Ablation study on text guidance. (a) without local SDS losses. (b) without global SDS losses. (c) vanilla SDS losses without perturb and average scoring~\cite{wang2022score}. (d) full model.}
    \caption{\textbf{Design Impact Comparison: Density vs. Color-based Methods.} The top row illustrates the density-based approach's detailed rendering and quick convergence in the 'table wine' scene. The bottom row highlights the color-based method's enhancements and its drawbacks, such as geometric and shadow inaccuracies, particularly in close-up views and slow convergence.
    % \textbf{(a)} global text guidance(integrating local frames by \cref{eq:multi_volrend}) and global calibration(integrating local frames, then aligning the rendering result directly with the full text). 
    }
    \label{fig:abls}
    % \vspace{-20pt}
\end{figure}
\subsection{The Proposed CompoNeRF}
\subsubsection{Composition Module}
CompoNeRF is designed to composite multiple NeRFs to reconstruct scenes featuring multiple objects, utilizing guidance from both bounding boxes and textual prompts. Within our framework, depicted in \cref{fig:framework}, the Axis-Aligned Bounding Box (AABB) ray intersection test algorithm is applied to ascertain intersections across each box in the global frame. Subsequently, we sample points \(\boldsymbol{x}_g\) within the intervals of the ray-box and project them to \(\boldsymbol{x}_l\) to deduce the corresponding color \(\boldsymbol{C}_l\) and density \(\boldsymbol{\sigma}_l\) within individual NeRF models.
%
These properties are processed through our composition module to infer the global color \(\boldsymbol{C}_g\) and density \(\boldsymbol{\sigma}_g\), crucial for the global rendering.
%
Volume rendering techniques~\cite{kajiya1984ray} are then employed to procure the rendered views for both local and global frames. We propose dual SDS losses to ensure coherence between the image outputs and their corresponding textual descriptions. Additionally, our approach facilitates recomposition by channeling samples from cached models back into local frames along with the text revision, thereby streamlining the integration.

% As shown in \cref{fig:abls}(a), we verify its necessity by dropping $\nabla \mathcal{L}_{\text{SDS}_g}$. 
% %
% Compared with our full model, its layout does not fit our shared sense of a room, \ie, \emph{nightstand} is usually lower than \emph{bed}; \emph{lamp} needs a base to support it. Additionally,  it lacks global consistency, such as light reflection, to make it more realistic. 
% %
% Therefore, we leverage the full text semantics to ensure consistent global rendering across local frames. 
% %
% Instead of conditioning the global rendering view with the full prompt directly, we note that global calibration is necessary for geometry and color to be learned sufficiently.
% For example, we observe that geometric completeness and texture of \emph{nightstand} are not ideal. Although reflection appears around \emph{nightstand}, \emph{bed} is stripped of the light. 
% %
% Therefore, we opt to leverage the correlation between the rendering output of the combined NeRFs and the overall semantics to perform multi-object scene reconstruction.  
%

\noindent\textbf{Global Composition.}
The independent optimization of each local frame may inadvertently result in a lack of global coherence within the scene. To address this, our scene composition process is designed to integrate these frames, thereby achieving a more consistent result.
%
Before exploring the specifics of the module, it is imperative to discuss two critical design decisions within the composition module, as depicted in \cref{fig:framework}.
%
Upon integrating the properties inferred from \(\boldsymbol{x}_g\) into the composition module, they are fine-tuned through gradients derived from the global SDS loss.  This process leads to a critical consideration: the necessity and implications of refining the global density \(\boldsymbol{\sigma}_g\). This can be divided into two approaches: \textbf{1) Density-based:} The advantage of adjusting \(\boldsymbol{\sigma}_g\) is that it can adjust geometry, thus yielding a scene more congruent with the global text prompt. 
However, this comes at the cost of potentially compromising the optimal color \(\boldsymbol{C}_g\), as calibrating \(\boldsymbol{\sigma}_g\) introduces more uncertainty for subsequent color refinement as it requires prior density features $\boldsymbol{h}$ as shown at \cref{fig:compo}. 
\textbf{2) Color-based:} Conversely, directly employing \(\boldsymbol{\sigma}_l\) mitigates this uncertainty but at the expense of reduced geometric control, presenting a challenging balance to strike in the pursuit of precise scene composition.
% , which may lead to suboptimal outcomes.
%
After thorough experiments, exemplified in \cref{fig:abls}, we have opted for the density-based approach to refine \(\boldsymbol{\sigma}_g\)  prioritizing both \textbf{accuracy and efficiency}. The test revealed that it excels in rendering intricate details, such as enhanced wood grain textures and more naturally contoured 'salad', as accentuated by boxes. This method also demonstrated a swifter convergence rate. Conversely, while the color-based improved reflections and reduced flickering on the 'wine cup', it was plagued by issues such as sparse density, which adversely brings holes at the base of the 'cup' and the corner of the 'table'.
Furthermore, upon close examination, it becomes evident that shadow artifacts of 'wine' on the 'table' are pronounced, suggesting that its disadvantages outweigh its advantages.
%  in this context
% \textbf{Global Composition.}
% Each local frame is optimized independently, causing a lack of global connections for scene composition.
% Before delving into module details, there are two choices (see \cref{fig:framework}) on the composition module design we need to elaborate on first. 
% %
% In \cref{fig:framework}, by taking $\boldsymbol{x}_g$ into the composition module, their inferred properties are calibrated with gradients propagated from the global SDS loss. 
% However, it remains unclear whether $\boldsymbol{\sigma}_g$ should be refined or not. 
% %
% The trade-off on its usage is the density adjustment bringing a more reasonable layout and more geometric details that fit the global text prompt. While its potential downside is that $\boldsymbol{C}_g$ may not be optimal as $\boldsymbol{\sigma}_g$ has more uncertainty compared to $\boldsymbol{\sigma}_l$, bringing sub-optimal rendering results. 

% We choose the density-based method after comparing them with the experiment shown in \cref{fig:abls}. 
% %
% Specifically, we test both designs on the scene \emph{table wine} and discover that the density-based design provides more intrinsic details(as indicated by green boxes), \eg, enriched wood grains, and a more natural shape for \emph{salad} and has much faster convergence speed. In contrast, the color-based method enhances the reflection and smooths flickering on \emph{wine cup}, (as indicated by red boxes), but it suffers from 1) sparse density, resulting in poorly generated geometry at the base of  \emph{cup} and the wood \emph{table} corner. Additionally, shadow artifacts appeared on \emph{table} when viewed up close, outweighing benefits of the color-based method.

\begin{figure}[t!]
    \centering
    \includegraphics[width=\linewidth]{figures/compo_module.pdf}
    % \vspace{-24pt}
    % \caption{Ablation study on text guidance. (a) without local SDS losses. (b) without global SDS losses. (c) vanilla SDS losses without perturb and average scoring~\cite{wang2022score}. (d) full model.}
    \caption{\textbf{Detail of Composition module}: density-based design. 
    }
    \label{fig:compo}
    % \vspace{-18pt}
\end{figure}
\noindent\textbf{Network Design.}
The compositional framework of our network, as delineated in \cref{fig:compo}, is predicated on an architecture that employs a suite of MLPs, represented as \(\{\boldsymbol{\theta}_l\}_{l=1}^{m}\),  each dedicated to a distinct local frame. To harmonize \(\boldsymbol{\sigma}_l\) and \(\boldsymbol{C}_l\), we incorporate global MLPs, including density calibrator $f_{\boldsymbol{\theta}_{g_d}}$ and color calibrator $f_{\boldsymbol{\theta}_{g_c}}$.
%
A transformation module complements this system, tasked with maintaining the spatial coherence between the global and local frames. It governs the transformation of sampling points $\boldsymbol{x}$, ray directions $\boldsymbol{d}$, and adjacent sampling distances $\delta$. This module also orders the points $\{\boldsymbol{x}_{g,j}\}_j$ by their distance to the global camera origin $\boldsymbol{o}_g$, ensuring that each local point $\boldsymbol{x}_l$ is accurately matched with its corresponding global point $\boldsymbol{x}_g$ for subsequent volume rendering. 
%
The network design is:
{
\setlength\abovedisplayskip{4.5pt}
\setlength\belowdisplayskip{4.5pt}
\begin{align}
\label{eq:g_c_d}
{\boldsymbol{\sigma}_g}  &= \alpha_d f_{\boldsymbol{\theta}_{g_d}}({\boldsymbol{x}_g}) + \boldsymbol{\sigma}_l, \\
{\boldsymbol{C}_g}  &= \alpha_c f_{\boldsymbol{\theta}_{g_c}}(\boldsymbol{h}, {\boldsymbol{d}_g}) + \boldsymbol{C}_l. 
\end{align}}In contrast to the local frames, the global frame's color output $\boldsymbol{C}_g$ is inferred based on $\boldsymbol{h}$ and conditional on $\boldsymbol{d}_g$ to enable a view-dependent lighting effect.
% Denote the density features as $\boldsymbol{h}$. 
%
%
Residual learning is leveraged here, where \(\boldsymbol{\sigma}_l, \boldsymbol{C}_l\) serve as foundational elements that support the learning of global density \(\boldsymbol{\sigma}_g\) and color \(\boldsymbol{C}_g\). The parameters \(\alpha_d, \alpha_c\) are adjustable, allowing fine-tuning of the influence that local components exert on the global outputs.
%
It is imperative to acknowledge that in our color-based method, density calibration is intentionally excluded to concentrate solely on the refinement of color dynamics as shown at \cref{fig:framework}. This is achieved by conditioning the process on both spatial and directional global inputs \((\boldsymbol{x}_g, \boldsymbol{d}_g)\), as demonstrated in the following equations:
\begin{align}
\setlength\abovedisplayskip{4.5pt}
\setlength\belowdisplayskip{4.5pt}
\label{eq:g_c_c}
\boldsymbol{\sigma}_g = \boldsymbol{\sigma}_l, \quad
{\boldsymbol{C}_g} = \alpha_c f_{\boldsymbol{\theta}_{g_c}}({\boldsymbol{x}_g}, {\boldsymbol{d}_g}) + \boldsymbol{C}_l.
\end{align}
The integration of extra $\boldsymbol{x}_g$ aims to facilitate a fair comparison under same inputs with the density-based. It enhances the visual appeal of effects like the wine cup's reflection, as demonstrated in \cref{fig:abls}. However, this method is not without its compromises. It tends to produce artifacts and is characterized by a slower convergence rate. Additionally, this approach limits the ability to precisely control density, subsequently impacting the intricate geometric details.


\begin{figure*}[t!]
    \centering
    \includegraphics[width=\linewidth]{figures/sota.pdf}
    % \vspace{-24pt}
    \caption{\textbf{Qualitative comparison with other text-to-3D methods using multi-object text prompts}. Cases 1-3 demonstrate simpler settings characterized by compositions involving two objects. In contrast, Cases 4-8 delve into more intricate scenarios featuring compositions with more than two objects. Smaller images are presented to illustrate the generated local NeRFs(partially shown in Cases 4-8).}
    \label{fig:sota}
    % \vspace{-5pt}
\end{figure*}
%
% \begin{table*}[t!]
% \centering
% \resizebox{\textwidth}{!}
% {
% \begin{tabular}{cccccccc}
% \toprule
% Method            & \rotatebox{60}{table wine}  & \rotatebox{60}{teddy monkey} & \rotatebox{60}{computer mouse} & \rotatebox{60}{bed room}  & \rotatebox{60}{chess} & \rotatebox{60}{pisa tower} & \rotatebox{60}{astronaut} & \rotatebox{60}{tesla}  \\ \midrule
% LatentNeRF  & 21.55 & 27.38 & 17.13 & 21.86 & 31.19 & 24.31 & 27.07 & 25.16 \\
% SJC & 23.33 & 27.37 & 18.00 & 22.54 & 30.53 & \textbf{26.18 }& 27.84 & 23.55 \\
% CompoNeRF & \textbf{32.68} & \textbf{28.57}	 &\textbf{ 22.34} &\textbf{ 28.65} & \textbf{31.45} & \textbf{28.96} & 25.82 & 25.95 & 24.42 & \textbf{32.71} & \textbf{26.13 }& \textbf{26.38} & \textbf{30.98} & \textbf{33.37} \\
% \bottomrule
% \end{tabular}
% }
% \vspace{-10pt}
% \caption{Performance of our CompoNeRF in different 3D scenes. We use CLIP score \cite{parmar2023zero,zhang2023sine,wang2023imagen} as our evaluation metric, which is a common evaluation metric in text-to-image generation tasks to evaluate the similarity of the generated image to the text prompt. }
% \label{perclass}
% \end{table*}
%
\begin{table*}[t!]
% \scalebox{0.8}
\renewcommand{\arraystretch}{1.2}
\fontsize{4pt}{4pt}
\selectfont 
\centering
% \vspace{-8pt}
\resizebox{\textwidth}{!}
{
% \begin{tabular}{lcccccccc}
% \hline
% Method     & table\_wine    & tesla          & pyramid        & chess          & apple and banana      & astronaut      & glass\_balls   & Eiffel\_tower    \\ \hline
% LatentNeRF & 21.55          & 25.16          & 27.43          & 31.19          & 27.69          & 27.07          & 29.51          & 26.32          \\
% SJC        & 23.33          & 23.55          & 25.62          & 30.53          & 28.21          & 27.84          & 28.76          &27.41 \\
% \textbf{CompoNeRF(Ours)}     & \textbf{32.68} & \textbf{26.13} & \textbf{28.96} & \textbf{31.45} & \textbf{33.37} & \textbf{32.71} & \textbf{30.98} & \textbf{28.44}          \\ \hline
% \end{tabular}
\begin{tabular}{lcccccccc}
\hline
Method                   & Case 1         & Case 2         & Case 3         & Case 4         & Case 5         & Case 6         & Case 7         & Case 8         \\ 
\hlineB{1.1}
LatentNeRF               & 25.16          & 27.07          & 27.69          & 31.19          & 21.55          & 26.32          & 27.43          & 29.51          \\
SJC                      & 23.55          & 27.84          & 28.21          & 30.53          & 23.33          & 27.41          & 25.62          & 28.76          \\
\textbf{CompoNeRF (Ours)} & \textbf{26.13} & \textbf{32.71} & \textbf{33.37} & \textbf{31.45} & \textbf{36.06} & \textbf{28.44} & \textbf{28.96} & \textbf{30.98} \\ \hlineB{1.1}
\end{tabular}
}

% \vspace{-6pt}
\caption{\textbf{Performance comparison of our CompoNeRF in different 3D scenes}. For our evaluation metric, we utilize the average of CLIP scores~\cite{parmar2023zero,zhang2023sine,wang2023imagen} across different views, which serve to assess the similarity between the generated images and the global text prompt. }
\label{tb:perclass}
\end{table*}
% \cref{fig:framework} depicts the network architecture of the composition module. Denote $m$ as local MLP $\{\boldsymbol{\theta}_l\}_{l=1}^{m}$ for each local frame. Then, we introduce the global MLPs including density $\boldsymbol{\theta}_{g_d}$ and $\boldsymbol{\theta}_{g_c}$ calibrators to refine $\boldsymbol{\sigma}_l$ and $\boldsymbol{C}_l$. 
% %
% In detail, the network design is, 
% {
% % \setlength\abovedisplayskip{4.5pt}
% % \setlength\belowdisplayskip{4.5pt}
% \begin{align}
% \label{eq:g_c_d}
% {\boldsymbol{\sigma}_g}  &= \alpha_d \boldsymbol{\theta}_{g_d}({\boldsymbol{\sigma}_l}) + \boldsymbol{\sigma}_l, \\  
% {\boldsymbol{C}_g}  &= \alpha_c \boldsymbol{\theta}_{g_c}({\boldsymbol{C}_l},  {\boldsymbol{d}_g}) + \boldsymbol{C}_l, 
% \end{align}}
% %
% where residual $\boldsymbol{\sigma}_l, \boldsymbol{C}_l$ assist in learning $\boldsymbol{\sigma}_g$ and $\boldsymbol{C}_g$, while $\alpha_d, \alpha_c$ balance their contribution as learnable parameters.
% %
% Note that the color-based omits density calibration, and simply uses the shared color refinement.



% The 3D boxes are only used for the spatial configuration of local NeRFs, while the implicit representation of local NeRFs is inferred by the canonical samples inside the local frame without considering the global relationship across different objects.
% To relieve such location-dependent effects, we further calibrate the output color and density from the local NeRF with global coordinates $({\boldsymbol{x}}_g, {\boldsymbol{y}}_g, {\boldsymbol{z}}_g)$ and ray directions $\left({\boldsymbol{\phi}}_{d}, {\boldsymbol{\theta}}_{d}\right)$ as the conditional input.
% % to inject the global visual clues.
% %
% %
% Specifically, we adopt a shared MLP $\boldsymbol{\theta}_{g}$ to calibrate all the predicted object colors, that is,
% {\setlength\abovedisplayskip{4.5pt}
% \setlength\belowdisplayskip{4.5pt}
% \begin{align}
% \label{eq:MLP_dyn_2}
% {\boldsymbol{C}_g} = {\boldsymbol{C}_l} + \boldsymbol{emb}_{g} &= {\boldsymbol{C}_l} + \boldsymbol{\theta}_{g}({\boldsymbol{x}}_g, {\boldsymbol{y}}_g, {\boldsymbol{z}}_g, {\boldsymbol{\phi}}_{d}, {\boldsymbol{\theta}}_{d}),
% \end{align}}
% where ${\boldsymbol{C}_l}$ is the color predicted by the local NeRF.
% Therefore, the scene color can preserve the view-consistent behavior from the original architecture and add consistency across poses for the volumetric density.
% Since the color and density values share the same latent expression in $({\boldsymbol{x}}_l, {\boldsymbol{y}}_l, {\boldsymbol{z}}_l)$, we only calibrate the emitted scene color explicitly with the scene location, as the densities of local NeRFs also are implicitly adjusted during optimization.

% \noindent \textbf{Global and Local Volumetric Rendering.}
% After compositing all the interacted points, each ray $\boldsymbol{r}_i$ collects a set sampling points by $\{\boldsymbol{t}_{i,j,n} \}_{j=1, n=1}^{m_j, N}$, where $m_j$ is the number of the hit object.
% For each sampling point, the inference results with the respective 3D representations are the local color $\boldsymbol{c}_{l}$, global color $\boldsymbol{c}_{g}$, and density $\sigma$.

% In fact, the local view $\hat{C}_{l,j}$ of single object $j$ also can be rendered by the sampled points  belongs to the same local frames as shown at Fig.~\ref{fig:framework}.

\subsubsection{Recomposition}
Our architecture advances scene reconstruction by providing an intuitive interface for layout manipulation.  This capability is crucial for the reconfiguration of scene elements into novel scenes, as depicted in \cref{fig:framework}. Here, the input panel allows for adjustments in the attributes of bounding boxes, such as modifying the position and scale of the 'apple' bounding box prior to composition. The refinement process further involves sampling ray-box intervals from the global frame, leading to transformed coordinates with the corresponding ray samples that are then incorporated into the pipeline, as demonstrated in \cref{fig:compo}.
%
Each bounding box represents an individual NeRF, providing the flexibility to move, scale, or remove elements as needed. CompoNeRF's capabilities also extend to textual edits, exemplified by the transformation of 'wine' into 'juice'.
%
Since NeRFs have been well trained, we only finetune \(\theta_g, \theta_l\) to align text prompts to promote consistency of both local and global views.
%
Moreover, the NeRFs once retrained within the edited scene, are also structured to be decomposable and cacheable in future scene compositions.
% Our CompoNeRF architecture facilitates the seamless reconstruction of scenes leveraging existing models. It enables precise editing of bounding boxes parameterized by \(\{\boldsymbol{\theta}_l\}_{l=1}^{m}\), allowing for their reconfiguration into new layouts. Refer to \cref{fig:framework}, the input panel permits the modification of attributes such as the position and scale of the 'apple' node's bounding box prior to composition. The process is further refined by sampling from the updated ray-box intervals within the global frame, which are then projected onto \(\boldsymbol{x}_l\), ensuring a streamlined reconstruction that integrates the 'apple' effectively. This addition is executed with careful attention to color consistency, positioning the 'apple' adjacent to the 'French bread' to complement the scene's overall palette. Each bounding box represents an individual NeRF, which means they can be manipulated through moving, scaling, and removal operations. CompoNeRF also extends its editing prowess to textual modifications, as evidenced by the 'wine cup' now appearing filled with juice—a change propagated through both subtexts and the global test. 
% %
% Since NeRFs have been well trained, we only finetune $\theta_g, \theta_l$ to align text prompts to promote consistency of both local and global views . 
% %
% Moreover, the NeRFs, once retrained within the reimagined scene, are also structured to be decomposable and cacheable for subsequent scene compositions.

% , as shown in Fig.~\ref{fig:framework}.
% For each scene described by the multi-object text prompt $T$, we
% To enhance the guidance of local representations, we use the local text prompt $T_l \subseteq T$ of a single object to optimize the local NeRFs in local views.
% The scene views $\hat{\boldsymbol{X}}_g=\{\hat{\boldsymbol{C}}_{g,i}\}_{i=1}^{H\times W}$ is obtained from the predicted pixel values of $H \times W$ rays by compositing all the ray-box interaction values.
% Similarly, the rendered view $\hat{\boldsymbol{X}}_{l,j}$ of the local frame $\boldsymbol{\theta}_j$ without compositing other objects can be calculated by $\hat{\boldsymbol{C}}_{l,j}$, as depicted in Sec.~\ref{ssec:render}.
% We use the local color instead of the globally calibrated color to obtain a local view because the local NeRF should learn the object identity unrelated to its placed position, as the position can be different during user edition.
% % Compared to cropping the local region from a global view for training, separate rendering can avoid the undesired information from other objects brought by the occlusion and resolution adjustments.
% Formally, we employ the following loss as the learning objective,
\begin{figure*}[t!]
    \centering
    \includegraphics[width=\linewidth]{figures/editing.pdf}
    % \vspace{-23pt}
    \caption{\textbf{Scene Editing Outcome:} Demonstrated here are the stages of our recomposition, utilizing cached source scenes. Each NeRF is individually identified by colorful labels. These decomposed nodes are then positioned in the initial layout and subsequently calibrated to form the final composition. The detailed description of the ambient environment is underscored, enhancing the scene's realism.}
    \label{fig:app}
    % \vspace{-12pt}
\end{figure*} 
\subsubsection{Optimization}
\label{sec:optimization}
During optimization, our method employs dual text guidance to align rendering results with both global and local textual descriptions. The optimization objective is:
{
\small
\setlength\abovedisplayskip{2pt}
\setlength\belowdisplayskip{2pt}
\begin{equation}
\label{eqn:loss_f}
\mathcal{L}= {\alpha_g}\nabla\mathcal{L}_{\text{SDS}}(\hat{\boldsymbol{X}}_{g}, T) + {\alpha_l}\sum_{j=1}^{m} \nabla\mathcal{L}_{\text{SDS}}(\hat{\boldsymbol{X}}_{l,j}, T_{l,j}) + \beta\mathcal{L}_{\text{sparse}},\nonumber
\end{equation}
}where $T$ signifies the global text prompt, while $T_{l}$ pertains to a specific object within the global context. The hyperparameters $\alpha_{g}, \alpha_{l}$, and $\beta$ modulate the respective loss weights. 
% $\nabla \mathcal{L}_{\text{SDS}}$ is the score distillation sampling loss, as described in Sec.~\ref{sec:background}.
As suggested in~\cite{metzer2022latent}, we use $L_{\text{sparse}}$ included to penalize the binary entropy of local NeRFs' densities, thereby mitigating the issue of extraneous floating radiance.
Additionally, incorporating directional cues such as "front view" or "side view" into the input text, as suggested by \cite{poole2022dreamfusion,metzer2022latent} proves beneficial in specifying camera poses during the training phase, further enhancing the alignment of our generated scenes with the intended perspectives.
% Note that the global calibration in the scene frame can adaptively revise both $({C}_l, {\sigma})$ in local NeRF with $\nabla \mathcal{L}_{SDS}$ along with the back-propagating gradient.

\begin{table}[h]
\begin{center}
\small
% \setlength{\tabcolsep}{2pt}
\begin{tabular}{l | c c c }
\whline
\multirow{2}{*}{\makecell[c]{Model}} &  \multirow{2}{*}{\makecell[c]{Params \\ (M)}} & \multirow{2}{*}{\makecell[c]{MACs \\ (G)}}  & \multirow{2}{*}{\makecell[c]{Top-1 \\ (\%)}} \\
~ & ~ & ~ &  \\
\whline
DeiT-S \cite{deit} & 22 & 4.6 & \textbf{79.8} \\ 
MetaNeXt-Attn & 22 & 4.6 & 3.9  \\ 
ConvNeXt-S (\textit{iso.}) \cite{convnext} & 22 & 4.3 & 79.7 \\
\modelname{}-S (\textit{iso.}) & 22 & 4.2 & 79.7 \\
\hline
DeiT-B \cite{deit} & 87 & 17.6 & 81.8 \\
ConvNeXt-S (\textit{iso.}) \cite{convnext} & 87 & 16.9 & 82.0 \\
\modelname{}-S (\textit{iso.}) & 86 & 16.8 & \textbf{82.1}  \\
\whline
\end{tabular}
\end{center}
\vspace{-2mm}
\caption{\textbf{Comparison among ViT, isotropic ConvNeXt and \modelname{}.}  MetaNeXt-Attn is instantiated from MetaNeXt with token mixer of self-attention \cite{transformer}.}
\label{tab:iso}
\end{table}

\begin{table*}[h]
    \centering
    \setlength{\tabcolsep}{5pt}
    \begin{tabular}{l | c | c | c c | c c | c}
% \toprule
\whline
\multirow{2}{*}{\makecell[c]{Model}}   &  \multirow{2}{*}{\makecell[c]{Mixing \\ Type}}     & 
\multirow{2}{*}{\makecell[c]{Image \\ (size)}} & \multirow{2}{*}{\makecell[c]{Params \\ (M)}}  & \multirow{2}{*}{\makecell[c]{MACs \\ (G)}} & \multicolumn{2}{c|}{Throughput (img/second)} & \multirow{2}{*}{\makecell[c]{Top-1 \\ (\%)}}
\\
~ &  ~ & ~ & ~ & ~ & Train & Inference & ~ \\
\whline
DeiT-S \cite{deit}  & Attn  & $224^2$ & 22 & 4.6 & 1227 & 3781 & 79.8 \\
T2T-ViT-14 \cite{t2t}  & Attn   & $224^2$ & 22 & 4.8 & -- & -- & 81.5 \\
TNT-S \cite{tnt} & Attn & $224^2$ & 24 & 5.2 & -- & -- & 81.5 \\
Swin-T \cite{swin}  & Attn   & $224^2$ &  29 & 4.5 & 564 & 1768 & 81.3  \\
Focal-T \cite{focal_transformer} & Attn & $224^2$ & 29 & 4.9 & -- & -- & 82.2 \\
\hdashline
ResNet-50 \cite{resnet, resnetsb}   & Conv   & $224^2$ & 26 & 4.1 & 969 &  3149 & 78.4 \\
RSB-ResNet-50 \cite{resnet, resnetsb}   & Conv   & $224^2$ &  26 & 4.1 & 969 &  3149 & 79.8 \\
RegNetY-4G \cite{regnet, resnetsb}   & Conv   & $224^2$ & 21 & 4.0 & 670 & 2694 & 81.3 \\
FocalNet-T \cite{focalnet} & Conv & $224^2$ & 29 & 4.5 & -- & -- & 82.3 \\
\br 
ConvNeXt-T \cite{convnext}   & Conv   & $224^2$ & 29 &  4.5 & 575 & 2413 \textcolor{gray}{(1943)} & 82.1 \\
\br 
\modelname{}-T (Ours) & Conv & $224^2$ & 28 &  4.2 & 901 (+57\%) & 2900 (+20\%) & 82.3 (+0.2) \\
\whline
T2T-ViT-19 \cite{t2t}  & Attn   & $224^2$ & 39 & 8.5 & -- & -- & 81.9 \\
PVT-Medium \cite{pvt} & Attn & $224^2$ & 44 & 6.7 & -- & -- & 81.2 \\
Swin-S \cite{swin}  & Attn   & $224^2$ & 50 &  8.7 & 359  & 1131 & 83.0  \\
Focal-S \cite{focal_transformer} & Attn & $224^2$ & 51 & 9.1 & -- & -- & 83.5 \\
\hdashline
RSB-ResNet-101 \cite{resnet, resnetsb}   & Conv   & $224^2$ &  45 & 7.9 & 620 & 2057 & 81.3 \\
RegNetY-8G \cite{regnet, resnetsb}   & Conv   & $224^2$ & 39 &  8.0 & 689 & 1326 & 82.1 \\
FocalNet-S \cite{focalnet} & Conv & $224^2$ & 50 & 8.7 & -- & -- & 83.5 \\
\br 
ConvNeXt-S \cite{convnext}   & Conv   & $224^2$ & 50 & 8.7 & 361 & 1535 \textcolor{gray}{(1275)} & 83.1 \\
\br 
\modelname{}-S (Ours) & Conv & $224^2$ & 49 & 8.4 & 521 (+44\%) & 1750 (+14\%) & 83.5 (+0.4) \\
\whline
DeiT-B \cite{deit}  & Attn  & $224^2$ & 86 & 17.5 & 541 & 1608 & 81.8 \\
T2T-ViT-24 \cite{t2t}  & Attn & $224^2$ & 64 &  13.8 & -- & -- & 82.3 \\
TNT-B \cite{tnt} & Attn &  $224^2$ & 66 & 14.1 & -- & -- & 82.9 \\
PVT-Large \cite{pvt} & Attn &  $224^2$ & 62 & 9.8 & -- & -- & 81.7 \\
Swin-B \cite{swin}  & Attn  & $224^2$ & 88 &  15.4 & 271 & 843 & 83.5 \\
Focal-B \cite{focal_transformer} & Attn & $224^2$ & 90 & 16.0 & -- & -- & 83.8 \\
\hdashline
RSB-ResNet-152 \cite{resnet, resnetsb}   & Conv  & $224^2$ & 60 & 11.6 & 437 & 1457 & 81.8 \\
RegNetY-16G \cite{regnet, resnetsb}   & Conv  & $224^2$ & 84 &15.9 & 322 & 1100 & 82.2 \\
RepLKNet-31B \cite{replknet}  & Conv & $224^2$ & 79  & 15.3 & -- & -- & 83.5 \\
FocalNet-B \cite{focalnet} & Conv & $224^2$ & 89 & 15.4 & -- & -- & 83.9 \\
\br 
ConvNeXt-B \cite{convnext}   & Conv   & $224^2$ & 89  & 15.4 & 267 & 1122 \textcolor{gray}{(969)} & 83.8  \\
\br
\modelname{}-B (Ours) & Conv & $224^2$ & 87 & 14.9 & 375 (+40\%) & 1244 (+11\%) & 84.0 (+0.2) \\
\hline
ViT-Base/16 \cite{vit} & Attn & $384^2$ & 87 & 55.4 & 130 & 359 & 77.9 \\
DeiT-B \cite{deit} & Attn & $384^2$ & 86 & 55.4 & 131 & 361 & 83.1 \\
Swin-B \cite{swin} & Attn & $384^2$ & 88 & 47.1 & 104 & 296 & 84.5 \\
\hdashline
RepLKNet-31B \cite{replknet} & Conv & $384^2$ & 79 & 45.1 & -- & -- & 84.8 \\
\br
ConvNeXt-B \cite{convnext} & Conv & $384^2$ & 89 & 45.0 & 95 & 393 (\textcolor{gray}{337}) & 85.1 \\
\br
\modelname{}-B (Ours) & Conv & $384^2$ & 87 & 43.6 & 139 (+46\%) & 428 (+9\%) & 85.2 (+0.1) \\

\whline
\end{tabular}


    \caption{\label{tab:imagenet}
    \textbf{Performance of models trained on ImageNet-1K.} The throughputs are measured on an A100 GPU with batch size of 128 and full precision (FP32). Our environment is PyTorch 1.13.0 and NVIDIA CUDA 11.7.1. The better results of ``Channel First" and ``Channel Last"  memory layouts are reported. The numbers in \textcolor{gray}{gray} color are reported by ConvNeXt \cite{convnext}. In our environment, ConvNeXt achieves much higher throughput than the values reported in the paper \cite{convnext}.}
\end{table*}

\section{Experiment}
%%%%%%%%% Table: ADE20K uper
\begin{table}[t]
\centering
\setlength{\tabcolsep}{3.5pt}
\begin{tabular}{l|cccc}
\whline
\multirow{2}{*}{Backbone} & \multicolumn{3}{c}{UperNet}\\
\cline{2-5}
& Params (M) & MACs (G) & FPS & mIoU (\%) \\
    \whline
    Swin-T~\cite{swin}                & 60 & 945 & 20.6 & 45.8 \\
    ConvNeXt-T ~\cite{convnext}          & 60 & 939 & 20.6 & 46.7 \\
    \modelname{}-T  & 56 & 933 & 22.7 & \textbf{47.9} \\
	\hline
    Swin-S~\cite{swin}                & 81 & 1038 & 16.2 & 49.5 \\
    ConvNeXt-S ~\cite{convnext}          & 82 & 1027 & 16.8 & 49.6 \\
    \modelname{}-S & 78 & 1020 & 17.6 & \textbf{50.0} \\
	\hline
    Swin-B~\cite{swin}                & 121 & 1188 & 16.2 & 49.7 \\
     ConvNeXt-B ~\cite{convnext}          & 122 & 1170 & 15.7 & 49.9 \\
    \modelname{}-B  & 115 & 1159 & 17.5 & \textbf{50.6} \\
\whline
\end{tabular}

\caption{\textbf{Performance of Semantic segmentation with UperNet \cite{upernet} on ADE20K~\cite{ade20k} validation set.} Images are cropped to $512 \times 512$ for training. The MACs are measured with input size of $512 \times 2048$. }
\label{tab:upernet}
\normalsize
\end{table}

%%%%%%%%% Table: ADE20K
\begin{table}[h!]
\small
\centering
\setlength{\tabcolsep}{3.5pt}
\begin{tabular}{l|ccc}
\whline
\multirow{2}{*}{Backbone} & \multicolumn{3}{c}{Semantic FPN}\\
\cline{2-4}
& Params (M) & MACs (G) & mIoU (\%) \\
    \whline
    ResNet-50~\cite{resnet}                & 29 & 46 & 36.7 \\
    PVT-Small~\cite{pvt}          & 28 & 45 & 39.8 \\
	PoolFormer-S24  \cite{metaformer}                  & 23 & 39 & 40.3 \\
    \modelname{}-T & 28 & 44 & \textbf{43.1} \\
    \hline
    ResNet-101~\cite{resnet}      & 48 & 65 &  38.8\\
    ResNeXt-101-32x4d~\cite{resnext} & 47 & 65 & 39.7 \\
    PVT-Medium~\cite{pvt}         & 48 & 61 & 41.6 \\
     PoolFormer-S36 \cite{metaformer}                   & 35 & 48 & 42.0 \\
    PoolFormer-M36 \cite{metaformer}                   & 60 & 68 & 42.4 \\
     \modelname{}-S & 50 & 65 & \textbf{45.6} \\
    \hline
    PVT-Large~\cite{pvt}          & 65 & 80 & 42.1 \\
    % \hline
    ResNeXt-101-64x4d~\cite{resnext} & 86 & 104 & 40.2 \\
     PoolFormer-M48 \cite{metaformer}                  & 77 & 82 &  42.7 \\
     \modelname{}-B & 85 & 100 & \textbf{46.4} \\
\whline
\end{tabular}
\caption{\textbf{Performance of Semantic segmentation with Semantic FPN \cite{fpn} on ADE20K~\cite{ade20k} validation set.} Images are cropped to $512 \times 512$ for training. The MACs are measured with input size of $512 \times 512$. }
\label{tab:fpn}
\normalsize
\end{table}


\subsection{Image classification}
\myPara{Setup}
For the image classification task, ImageNet-1K \cite{imagenet_cvpr, imagenet_ijcv} is one of the most commonly-used benchmarks, which contains around 1.3 million images in the training set and 50 thousand images in the validation set. To fairly compared with the widely-used baselines, \eg  Swin \cite{swin} and ConvNeXt \cite{convnext}, we mainly follow the training hyper-parameters from DeiT \cite{deit} without distillation. Specifically, the models are trained by AdamW \cite{adamw} optimizer with a learning rate $lr = 0.001 \times \mathrm{batch size} / 1024$ ($lr=4e-3$ and $\mathrm{batch size} = 4096$ are used in this paper the same as ConvNeXt). Following DeiT, data augmentation includes standard random resized crop, horizontal flip, RandAugment \cite{randaugment}, Mixup \cite{mixup}, CutMix \cite{cutmix}, Random Erasing \cite{random_erasing} and color jitter. For regularization, label smoothing \cite{inception_v3}, stochastic depth \cite{stochastic_depth}, and weight decay are adopted. Like ConvNeXt, we also use LayerScale \cite{layerscale}, a technique to help train deep models. Our code is based on PyTroch \cite{pytorch} and timm \cite{rw2019timm} libraries. 


\myPara{Results} 
We compare \modelname{} with various state-of-the-art models, including attention-based and convolution-based models. As can be seen in Table \ref{tab:imagenet}, \modelname{} achieves highly competitive performance as well as enjoys higher speed. With similar model sizes and MACs, \modelname{} consistently outperforms ConvNeXt in terms of top-1 accuracy, and also exhibits higher throughput. For example, \modelname{}-T not only surpasses ConvNeXt-T by 0.2\%, but also enjoys $1.6 \times$/$1.2 \times $ training/inference throughputs than ConvNeXts, similar to those of ResNet-50. That is to say, \modelname{}-T enjoys both ResNet-50's speed and ConvNeXt-T's accuracy. 
Moreover,  following Swin and ConvNeXt, we also finetuned the model trained at the resolution of $224 \times 224$ to $384 \times 384$ for 30 epochs.  We can see that \modelname{} still obtains promising performance similar to that of ConvNeXt.


Besides the 4-stage framework \cite{vgg, resnet, swin}, another notable one is ViT-style \cite{vit} isotropic architecture which has only one stage. To match the parameters and MACs of DeiT, we construct \modelname{} (\textit{iso.}) following ConvNeXt \cite{convnext}. Specifically, for the small/base model, we set the embedding dimension as 384/768 and the block number as 18/18. Besides, we build a model called MetaNeXt-Attn which is instantiated from MetaNeXt block by specifying self-attention as token mixer. The aim of this model is to investigate whether it is possible to merge two residual sub-blocks of the Transformer block into a single one. The experiment results are shown in Table \ref{tab:iso}. It can be seen that \modelname{} can also perform well with the isotropic architecture, demonstrating \modelname{} exhibits good generalization across different frameworks. It is worth noting that MetaNeXt-Attn could not be trained to converge and only achieved an accuracy of 3.9\%. This result suggests that, unlike the token mixer in MetaFormer, the token mixer in MetaNeXt cannot be too complex. If it is, the model may not be trainable.


%%%%%%%%% Table: Ablation
\begin{table*}[h]
\centering
\setlength{\tabcolsep}{2pt}
\scalebox{1.0}{\begin{figure}
       \centering
        \setlength{\tabcolsep}{1pt}
        {\scriptsize
        \begin{tabular}{c c c c c c c }
            { Original } &
            \multicolumn{2}{c}{  } &
            \multicolumn{4}{c}{$\longleftarrow$ Object level variations $\longrightarrow$} \\
            \includegraphics[width=0.185\linewidth]{images/ablation/chair.jpg} &
            \multicolumn{2}{c}{  } &
            \includegraphics[width=0.185\linewidth]{images/ablation/1_only_prompt_mixing/bench.jpg} &
            \includegraphics[width=0.185\linewidth]{images/ablation/1_only_prompt_mixing/stool.jpg} &
            \includegraphics[width=0.185\linewidth]{images/ablation/1_only_prompt_mixing/armchair.jpg} &
            \includegraphics[width=0.185\linewidth]{images/ablation/1_only_prompt_mixing/saddle.jpg} \\
            \multicolumn{3}{c}{  } &
            \multicolumn{4}{c}{ Only Prompt Mixing } \\
            \multicolumn{3}{c}{ } &
            \includegraphics[width=0.185\linewidth]{images/ablation/2_with_self_attn_injection/bench.jpg} &
            \includegraphics[width=0.185\linewidth]{images/ablation/2_with_self_attn_injection/stool.jpg} &
            \includegraphics[width=0.185\linewidth]{images/ablation/2_with_self_attn_injection/armchair.jpg} &
            \includegraphics[width=0.185\linewidth]{images/ablation/2_with_self_attn_injection/saddle.jpg} \\
            \multicolumn{3}{c}{  } &
            \multicolumn{4}{c}{ + Attention-Based Shape Localization } \\
            \multicolumn{3}{c}{ } &
            \includegraphics[width=0.185\linewidth]{images/ablation/3_background_blending/bench.jpg} &
            \includegraphics[width=0.185\linewidth]{images/ablation/3_background_blending/stool.jpg} &
            \includegraphics[width=0.185\linewidth]{images/ablation/3_background_blending/armchair.jpg} &
            \includegraphics[width=0.185\linewidth]{images/ablation/3_background_blending/saddle.jpg} \\
            \multicolumn{3}{c}{  } &
            \multicolumn{4}{c}{ + Controllable Background Preservation } \\
        \end{tabular}
        }
    \vspace{1mm}
    \captionof{figure}{
    Ablating our full object variations pipeline. Original image was crated using the prompt ``A \emph{chair} with a dog on it''. 
    }
    \vspace{-10pt}
    \label{fig:ablation}
\end{figure}
}
\caption{\textbf{Ablation for \modelname{} on ImageNet-1K classification benchmark.} \modelname{}-T is utilized as the baseline for the ablation study. Top-1 accuracy on the validation set is reported. 
}
\label{tab:ablation}
\normalsize
\end{table*}


\subsection{Semantic segmentation}
\myPara{Setup} 
ADE20K~\cite{ade20k}, one of the commonly used scene parsing benchmarks, is used to evaluate our models on semantic segmentation task. ADE20K includes 150 fine-grained semantic categories, containing twenty thousand and two thousand images in the training set and validation set, respectively.
The checkpoints trained on ImageNet-1K \cite{imagenet_cvpr} at the resolution of $224^2$ are utilized to initialize the backbones. Following Swin \cite{swin} and ConvNeXt \cite{convnext}, we firstly evaluate \modelname{} with UperNet \cite{upernet}. The models are trained with AdamW \cite{adamw} optimizer with learning rate of 6e-5 and batch size of 16 for 160K iterations. Following PVT \cite{pvt} and PoolFormer \cite{metaformer}, \modelname{} is also evaluated with Semantic FPN \cite{fpn}. 
In common practices~\cite{fpn,chen2017deeplab}, for the setting of 80K iterations, the batch size is 16. Following PoolFormer \cite{metaformer}, we increase the batch size to 32 and decrease the iterations to 40K, to speed up training. The AdamW \cite{adam, adamw} optimized is adopted with a learning rate of 2e-4 and a polynomial decay schedule of 0.9 power. Our code is based on PyTorch \cite{pytorch} and mmsegmentation library \cite{mmseg2020}.

\myPara{Results}
For segmentation with UpNet \cite{upernet}, the results are shown in Table \ref{tab:upernet}. As can be seen, 
\modelname{} consistently outperforms Swin \cite{swin} and ConvNeXt \cite{convnext} for different model sizes. On the setting of Semantic FPN \cite{fpn} as shown in Table \ref{tab:fpn}, \modelname{} significantly surpasses other backbones, like PVT \cite{pvt} and PoolFormer \cite{metaformer}. These results show that \modelname{} also has a high potential for dense prediction tasks.   





\subsection{Ablation studies}
We conduct ablation studies on ImageNet-1K \cite{imagenet_cvpr, imagenet_ijcv} using \modelname{}-T as baseline from the following aspects.


\myPara{Branch} Inception depthwise convolution includes four branches, three convolutional ones, and identity mapping. When removing any branch of horizontal or vertical band kernel, performance significantly drops from 82.3\% to 81.9\%, demonstrating the importance of these two branches. This is because these two branches with band kernels can enlarge the receptive field of the model. For the branch of small square kernel size of $3\times 3$, removing it can also achieve up to 82.0\% top-1 accuracy and bring higher throughput. This inspires us that if we attach more importance to the model speed, the simple version of \modelname{} without the square kernel of $3\times 3$ can be adopted.  For the band kernel, Inception v3 mostly equips them in a sequential way. We find that this assembling method can also obtain similar performance and even a little speed up the model. A possible reason is that PyTorch/CUDA may have optimized sequential convolutions well, and we only implement the parallel branches at a high level (see Algorithm \ref{alg:code}). We believe the parallel method will be faster when it is optimized better. Thus, parallel method for the band kernels is adopted by default. 


\myPara{Band kernel size} It is found the performance can be improved from kernel size 7 to 11, but it drops when the band kernel size increases to 13. This phenomenon may result from the optimization difficulty and can be solved by methods like structural re-parameterization \cite{repvgg, replknet}. For simplicity, we just set the band kernel size as 11 by default. 

\myPara{Convolution branch ratio} When the ratio increases from $1/8$ to $1/4$, performance improvement can not be observed. Ma \etal \cite{shufflenet_v2} also point out that it is not necessary for all channels to conduct convolution. But when the ratio decreases to $1/16$, it brings a serious performance drop. It is because a smaller ratio would limit the degree of token mixing, resulting in performance drop. Thus, we set the convolution branch ratio as $1/8$ by default.









% \clearpage
%\setcounter{page}{1}
\maketitlesupplementary
The supplementary material for our work  \textit{SC-VAE: Sparse Coding-based Variational Autoencoder with Learned ISTA} is structured as follows:
%Sec. \ref{section1_s} provides the detailed information of the encoder and decode architecture of the SC-VAE model. 
%Sec. \ref{section2_s} shows the visualization of the dictionary atoms.
%Sec. \ref{section3_s} shows the training loss on the ImageNet dataset with different number of downsampling (upsampling) blocks ($d$) in the encoder (decoder) of the SC-VAE model.
%Sec. \ref{section4_s} shows the visualization results of an unofficial implementation of VIT-VQGAN \cite{yu2021vector}. 
%Sec. \ref{section5_s} shows additional manipulation and interpolation results on FFHQ dataset. 
%Sec. \ref{section6_s} shows additional image patches clustering results on FFHQ and ImageNet datasets. 
%Sec. \ref{section7_s} shows additional unsupervised image segmentation results.
Section \ref{section1_s} details the encoder and decoder architecture of the SC-VAE model. In Section \ref{section2_s}, the dictionary atoms are visualized. In Section \ref{section3_s}, we provide the training losses on the ImageNet dataset when varying the number of downsampling (upsampling) blocks ($d$) in the encoder (decoder) of the SC-VAE model. In Section \ref{section4_s}, the visualized reconstruction results of an unofficial implementation of VIT-VQGAN \cite{yu2021vector} are provided. 
We provide  additional manipulation and interpolation results on the FFHQ dataset in Section \ref{section5_s}, while  additional clustering results of image patches on both FFHQ and ImageNet  are provided in Section \ref{section6_s}. Supplementary unsupervised image segmentation results are given in Section \ref{section7_s}.

%Additional results on image patches clustering and unsupervised image segmentation on FFHQ and ImageNet datasets are then presented in Sec. 2 and Sec. 3, respectively.
\setcounter{section}{0}

\section{The Encoder and Decoder Architecture of SC-VAE} \label{section1_s}
The SC-VAE model's encoder and decoder architecture mirrors that of VQGAN \cite{esser2021taming}. Details about the architecture are provided in Table \ref{figure:encoder_decoder}.
%The encoder and decoder architecture in the SC-VAE model are the same as the architecture used in VQGAN \cite{esser2021taming}, which is described in Table \ref{figure:encoder_decoder}. 
$H$, $W$ and $C$ denote the height, width
and the number of channels of an input image, respectively.
$C'$ and $C''$ represent the number of channels of the feature maps that are produced as outputs by the intermediate layers of the encoder and decode network.
In our experiment, $C'$ and $C''$ were set to $128$ and $512$, respectively. $n$ denotes the number of dimensions of each latent representation, which was set to $256$.
The variable $d$ represents the number of blocks used for downsampling and upsampling. Therefore, we can calculate the height ($h$) and width ($w$) of the encoder's output feature maps by dividing the height ($H$) and width ($W$) of input images by $2$ raised to the power of $d$.

\begin{table}[thbp!]
\centering
\caption{High-level architecture of the encoder and decoder of the SC-VAE model. $H$, $W$, and $C$ refer to the height, width, and the number of channels of an input image. 
$C'$ and $C''$ represent the number of channels of the feature maps from intermediate layers in the encoder and decoder networks. $n$ denotes the number of dimensions of each latent representation, while $d$ represents the number of downsampling (upsampling) blocks. Note that $h=\frac{H}{2^{d}}$, $w=\frac{W}{2^d}$.} 
\resizebox{1\linewidth}{!}{%
\begin{tabular}{c|c}
  \toprule
   &  $x\in \mathbb{R}^{H\times W\times C} $\\
   &  2D Convolution $\rightarrow \mathbb{R}^{H\times W\times C'}$\\
   &  $d \times$\{Residual Block, Downsample Block\} $\rightarrow \mathbb{R}^{h\times w\times C''}$\\
   &  Residual Block $\rightarrow \mathbb{R}^{h\times w\times C''}$\\
  Encoder &  Non-Local Block $\rightarrow \mathbb{R}^{h\times w\times C''}$\\
   &  Residual Block $\rightarrow \mathbb{R}^{h\times w\times C''}$\\
   &  Group Normalization \cite{wu2018group} $\rightarrow \mathbb{R}^{h\times w\times C''}$ \\
   &  Swish Activation Function \cite{ramachandran2017searching} $\rightarrow \mathbb{R}^{h\times w\times C''}$\\
   &  2D Convolution $\rightarrow E(x) \in \mathbb{R}^{h\times w\times n}$\\
  \midrule
   & $\tilde{E}(x)\in \mathbb{R}^{h\times w\times n} $  \\
   &  2D Convolution $\rightarrow \mathbb{R}^{h\times w\times C''}$  \\
   &  Residual Block $\rightarrow \mathbb{R}^{h\times w\times C''}$\\
    & Non-Local Block $\rightarrow \mathbb{R}^{h\times w\times C''}$\\
  Decoder & Residual Block $\rightarrow \mathbb{R}^{h\times w\times C''}$\\
    & $d\times$\{Residual Block, Upsample Block\} $\rightarrow \mathbb{R}^{H\times W\times C'}$\\
   & Group Normalization \cite{wu2018group} $\rightarrow \mathbb{R}^{H\times W\times C'}$\\
    & Swish  Activation Function \cite{ramachandran2017searching}
    $\rightarrow \mathbb{R}^{H\times W\times C'}$\\
    & 2D Convolution $\rightarrow G(\tilde{E}(x)) \in \mathbb{R}^{H\times W\times C}$\\
  \bottomrule
\end{tabular}}
\label{figure:encoder_decoder}
\end{table}

\section{Visualization of Dictionary Atoms}
\label{section2_s}
Figure \ref{figure:dictionary_visualization} demonstrates the $512$ columns (atoms) of the pre-determined Discrete Cosine Transform (DCT) dictionary. Each atom is of dimension $256$, which corresponds to the size of $16 \times 16$ images when shaped.
%We reshape all atoms into an image with a $16\times 16$ resolution.

\begin{figure}[tbp]
\centering
\includegraphics[width=8cm]{./Figures/visualization_of_dictionary.png}
\caption{$512$ atoms of the Discrete Cosine Transform (DCT) dictionary. All atoms were reshaped into a $16 \times 16$ image.}
\label{figure:dictionary_visualization}
\end{figure}

\section{Training Losses}  \label{section3_s}
%Training losses of inherent noises around the 140th epoch under different auxiliary dataset sizes (K)
Figures \ref{figure:TLImagenet32x32}, \ref{figure:TLImagenet16x16}, \ref{figure:TLImagenet4x4} and \ref{figure:TLImagenet1x1} show  the training losses over $120,000$ training steps on the
ImageNet dataset.
The number of downsampling (upsampling) blocks ($d$) in the encoder (decoder) of the SC-VAE model are $3, 4, 6$ and $8$, respectively.
%with the number of downsampling (upsampling) blocks ($d=3,4,6$ and $8$, respectively) in the encoder (decoder) of the SC-VAE model. 
%As is shown in these figures, the LISTA networks of the SC-VAE models converge to a fixed point no matter which downsampling (upsampling) block $d$ is used. However, SC-VAE suffer from image reconstruction when increasing $d$.
As depicted in these figures, the LISTA networks within the SC-VAE models consistently converge to a stable point regardless of the chosen downsampling (upsampling) block $d$. However, increasing $d$ leads to worse image reconstructions ($\mathcal{L}_{rec}$) in SC-VAE.

\begin{figure}[tbp]
\centering
\includegraphics[width=7.5cm]{./Figures/Imagenet32x32.png}
\caption{The training losses over $120,000$ training steps on the ImageNet dataset. The number of  downsampling (upsampling) blocks ($d$) in the encoder (decoder) of the SC-VAE model was set to $3$ and the height ($h$) and width ($w$) of latent representations were $32$. (a) Total loss $\mathcal{L}_{SC-VAE}$. (b) Image reconstruction loss $\mathcal{L}_{rec}$. (c)The mean of latent representations reconstruction loss $\frac{1}{hw}\mathcal{L}_{latent}$.}
\label{figure:TLImagenet32x32}
\end{figure}

\begin{figure}[tbp]
\centering
\includegraphics[width=7.5cm]{./Figures/Imagenet16x16.png}
\caption{The training losses over $120,000$ training steps on the ImageNet dataset. The number of  downsampling (upsampling) blocks ($d$) in the encoder (decoder) of the SC-VAE model was set to $4$ and the height ($h$) and width ($w$) of latent representations were $16$. (a) Total loss $\mathcal{L}_{SC-VAE}$. (b) Image reconstruction loss $\mathcal{L}_{rec}$. (c) The mean of latent representations reconstruction loss $\frac{1}{hw}\mathcal{L}_{latent}$.}
\label{figure:TLImagenet16x16}
\end{figure}

\begin{figure}[tbp]
\centering
\includegraphics[width=7.5cm]{./Figures/Imagenet4x4.png}
\caption{The training losses over $120,000$ training steps on the ImageNet dataset. The number of  downsampling (upsampling) blocks ($d$) in the encoder (decoder) of the SC-VAE model was set to $6$ and the height ($h$) and width ($w$) of latent representations were $4$. (a) Total loss $\mathcal{L}_{SC-VAE}$. (b) Image reconstruction loss $\mathcal{L}_{rec}$. (c) The mean of latent representations reconstruction loss $\frac{1}{hw}\mathcal{L}_{latent}$.}
\label{figure:TLImagenet4x4}
\end{figure}

\begin{figure}[tbp]
\centering
\includegraphics[width=7.5cm]{./Figures/Imagenet1x1.png}
\caption{The training losses over $120,000$ training steps on the ImageNet dataset. The number of  downsampling (upsampling) blocks ($d$) in the encoder (decoder) of the SC-VAE model was set to $8$ and the height ($h$) and width ($w$) of latent representations were $1$. (a) Total loss $\mathcal{L}_{SC-VAE}$. (b) Image reconstruction loss $\mathcal{L}_{rec}$. (c) The mean of latent representations reconstruction loss $\frac{1}{hw}\mathcal{L}_{latent}$.}
\label{figure:TLImagenet1x1}
\end{figure}

%\noindent
%\noindent\textbf{Learnbale ISTA.} The architecture of our Learnable ISTA network is shown in Table 2.
%\noindent\textbf{Attention Network for $\alpha$ Estimation.}  Our neural network architecture follows the backbone of PixelCNN++ [52], which is a U-Net [48] based on a Wide ResNet [72]. We replaced weight normalization [49] with group normalization [66] to make the implementation simpler. Our 32 × 32 models use four feature map resolutions (32 × 32 to 4 × 4), and our 256 × 256 models use six. All models have two convolutional residual blocks per resolution level and self-attention blocks at the 16 × 16 resolution between the convolutional blocks [6].






% \begin{table*}[!htbp]
% \centering
% \caption{High-level architecture of the Learnable ISTA of our SC-VAE. Note that $k$ is the number of the unfolded ISTA block.} 
% \begin{tabular}{c}
%   \toprule
%   Learnable ISTA \\
%   \midrule
%   $E(x)\in \mathbb{R}^{h\times w \times n} $ \\
%   Filter Matrix $\rightarrow \mathbb{R}^{h\times w\times K}$ \\
%   $k\times$\{Shrinkage Function, Mutual Inhibition Matrix, Addition Operator\} $\rightarrow \mathbb{R}^{h\times w\times K}$\\
%   Shrinkage function$\rightarrow Z\in \mathbb{R}^{h\times w\times K}$\\
%   \bottomrule
% \end{tabular}
% \end{table*}

\section{Image Reconstruction}  \label{section4_s}
Reconstruction results from unofficial implementation\footnote{https://github.com/thuanz123/enhancing-transformers} of VIT-VQGAN \cite{yu2021vector} are presented in Figure \ref{figure:ViT-VQGAN_Visualization}.
%Figures \ref{figure:ViT-VQGAN_Visualization} shows visualizations from unofficial implementation\footnote{https://github.com/thuanz123/enhancing-transformers} of VIT-VQGAN \cite{yu2021vector}. 
VIT-VQGAN \cite{yu2021vector} achieved visually appealing results. However, similar to VQ-GAN \cite{esser2021taming} and RQ-VAE \cite{lee2022autoregressive}, it faced challenges in accurately reconstructing intricate details and complex patterns.
%as VQ-GAN\cite{esser2021taming} and RQ-VAE\cite{lee2022autoregressive}. 
Additionally, its generalization performance was inferior to that of our model.

\begin{figure}[tbp]
\centering
\includegraphics[width=7.0cm]{./Figures/ViT-VQGAN_Visualization.png}
\caption{Image reconstructions from an unofficial implementation of VIT-VQGAN \cite{yu2021vector} and the SC-VAE models trained
on ImageNet dataset. Original images in the top two rows are
from the validation set of ImageNet dataset. Two external images are shown in the last two rows to demonstrate the generalizability of different methods. The numbers denote the shape of
latent codes and the learned codebook (dictionary) size, respectively.
SC-VAE achieved improved image reconstruction compared to VIT-VQGAN \cite{yu2021vector}. Zoom in to see the details in the red square area.}
\label{figure:ViT-VQGAN_Visualization}
\end{figure}

\section{Image Generation}  \label{section5_s}
Additional interpolation and manipulation results can be found in Figures \ref{figure:image_interpolation_supple} and \ref{figure:image_manipulation_supple}, respectively.

\begin{figure}[tbp]
\centering
\includegraphics[width=7.0cm]{./Figures/image_interpolation_supple.png}
\caption{Interpolation between the sparse code vectors of two samples from the SC-VAE$^{\dag}$ model trained on FFHQ.}
\label{figure:image_interpolation_supple}
\end{figure}

\begin{figure*}[tbp]
\centering
\includegraphics[width=14.5cm]{./Figures/image_manipulation_supple4.png}
\caption{Manipulating sparse code vectors on FFHQ. 
Each block contains five seed images used to infer the latent sparse code vector in the SC-VAE$^{\dag}$ model.
The disentangled attributes associated with the $i$-th component of a sparse code vector $z$ and a traversal range are shown on the top of each block.}
\label{figure:image_manipulation_supple}
\end{figure*}



% \begin{figure}[tbp]
% \centering
% \includegraphics[width=8cm]{./Figures/IG_Age.png}
% \caption{IG-Age.}
% \label{figure:IG_Age}
% \end{figure}

% \begin{figure}[tbp]
% \centering
% \includegraphics[width=8cm]{./Figures/IG_sunglasses.png}
% \caption{IG-sunglasses.}
% \label{figure:IG_sunglasses}
% \end{figure}

% \begin{figure}[tbp]
% \centering
% \includegraphics[width=8cm]{./Figures/IG_Azimuth.png}
% \caption{IG-azimuth.}
% \label{figure:IG_azimuth}
% \end{figure}

% \begin{figure}[tbp]
% \centering
% \includegraphics[width=8cm]{./Figures/IG_Fringe.png}
% \caption{IG-Fringe.}
% \label{figure:IG_Fringe}
% \end{figure}


% \begin{figure}[tbp]
% \centering
% \includegraphics[width=8cm]{./Figures/IG_skin color.png}
% \caption{IG-skin color.}
% \label{figure:IG_skin color}
% \end{figure}


% \begin{figure}[tbp]
% \centering
% \includegraphics[width=8cm]{./Figures/image_interpolation_supple.png}
% \caption{Interpolation in the latent space between two samples from a model trained on FFHQ.}
% \label{figure:interpolation}
% \end{figure}

\section{Image Patches Clustering}  \label{section6_s}
%Figures \ref{figure:s1} and \ref{figure:s2} exhibit more image patches clustering outcomes for the FFHQ and ImageNet datasets, respectively. 
Figures \ref{figure:s1} and \ref{figure:s2} showcase additional qualitative results of image patches clustering on FFHQ and ImageNet datasets, respectively.
These results were obtained utilizing the pre-trained SC-VAE$^\curlyvee$ model specific to each dataset with a downsampling block $d=4$.
\begin{figure*}[h!]
\centering
\includegraphics[width=16cm]{./Figures/patches_cluster_ffhq_supple_50.png}
\caption{50 randomly selected image patch clusters from the validation set of the FFHQ dataset generated by clustering the learned sparse code vectors of the pre-trained SC-VAE$^\curlyvee$ model
using the K-means algorithm. Each row represents one cluster. Image patches with similar patterns were grouped together.}
\label{figure:s1}
\end{figure*}

\begin{figure*}[h!]
\centering
\includegraphics[width=16cm]{./Figures/imagenet_cluster_patches_V3.png}
\caption{50 randomly selected image patch clusters from the validation set of the ImageNet dataset generated by clustering the learned sparse code vectors of the pre-trained SC-VAE$^\curlyvee$ model
using the K-means algorithm. Each row represents one cluster. Image patches with similar patterns were grouped together.}
\label{figure:s2}
\end{figure*}

% \begin{figure*}[h!]
% \centering
% \includegraphics[width=16cm]{./Figures/segmentation_ffhq_supple3.png}
% \caption{FFHQ.}
% \label{figure:5}
% \end{figure*}

\section{Unsupervised Image Segmentation} \label{section7_s}
\subsection{Qualitative Analysis on FFHQ and ImageNet}
%Figures \ref{figure:s3} and \ref{figure:s4} contain additional qualitative unsupervised image segmentation results on FFHQ and ImageNet datasets, respectively. 
%We utilized two SCVAE models that were pre-trained on the training set of the FFHQ and ImageNet dataset, respectively. These models had a downsampling block of $d = 3$ and a sparsity penalty of $\lambda = 2$. 
%We employed two SC-VAE$^\curlywedge$ models that had been pre-trained on the training sets of the FFHQ and ImageNet datasets, respectively. These models had a downsampling block $d=3$.
Additional qualitative unsupervised image segmentation results on the FFHQ and ImageNet datasets can be found in Figures \ref{figure:s3} and \ref{figure:s4}, respectively. We utilized two SC-VAE$^\curlywedge$ models pre-trained on the training sets of FFHQ and ImageNet, each employing a downsampling block $d=3$.
\subsection{Quantitative comparisons to prior work}
%Figure \ref{figure:Flower_CUB} shows more qualitative results on  Flowers \cite{nilsback2008automated} and Caltech-UCSD Birds-200-2011 (CUB) \cite{WahCUB_200_2011}. Flowers \cite{nilsback2008automated} consists of $8,189$ images of $102$ classes of flowers, with segmentation masks obtained by an automated algorithm developed specifically for segmenting flowers in color photographs \cite{nilsback2007delving}. CUB \cite{WahCUB_200_2011} consists of $11,788$ images of $200$ classes of birds and segmentation masks. Flowers and CUB contain $1,020$ and $1,000$ test images, respectively.
%Figure \ref{figure:Flower_CUB} shows more qualitative results on  Flowers \cite{nilsback2008automated} and Caltech-UCSD Birds-200-2011 (CUB) \cite{WahCUB_200_2011} datasets.
Figure \ref{figure:Flower_CUB} displays additional qualitative results from the Flowers \cite{nilsback2008automated} and Caltech-UCSD Birds-200-2011 (CUB) \cite{WahCUB_200_2011} datasets.\\
\subsubsection{Evaluation Metrics}
\textbf{Intersection of Union (IoU).} %The IoU score measures the overlap of two regions A and B by calculating the ratio of intersection over union, according to
The IoU score quantifies the overlap between two regions. This is achieved by evaluating the ratio of their intersection to their union.
\begin{align}
    \textup{IoU}(A, B) = \frac{|A\cap B|}{|A\cup B|}. \nonumber
\end{align}
%where we use the inferred mask and ground-truth mask as $A$ and $B$ respectively for evaluation.\\
$A$ denotes the ground-truth mask, while $B$ denotes the inferred mask.\\
%as $B$ for assessment purposes.\\
\textbf{DICE score.} Similarly, the DICE score is defined as:
\begin{align}
    \textup{Dice}(A, B) = \frac{2|A\cap B|}{|A|+ |B|}.\nonumber
\end{align}
\noindent
Higher is better for both scores.\\
\subsubsection{Dataset Details}
\textbf{Flowers.} The Flowers \cite{nilsback2008automated} dataset consists of $8,189$ images across $102$ different flower classes. Additionally, it includes segmentation masks generated by an automated algorithm designed explicitly for color photograph flower segmentation \cite{nilsback2007delving}. 
%The images in this dataset have large scale, pose and light variations.\\
The dataset contains images that exhibit substantial variations in scale, pose, and lighting.
Flowers \cite{nilsback2008automated} contains $1,020$ test images.\\
\textbf{CUB.} The CUB \cite{WahCUB_200_2011} dataset contains $11,788$ images covering $200$ bird classes, along with their segmentation masks. 
%Each image is further annotated with $15$ part locations and $1$ bounding box. We use theprovided bounding box to extract a center square from the image, and scale it to $128\times 128$ pixels.
Every image comes with annotations for $15$ part locations, $312$ binary attributes, and $1$ bounding box. We utilized the given bounding box to crop a central square from the image. The CUB dataset includes $1,000$ test images.\\
\textbf{ISIC-2016.} The ISIC-2016 \cite{gutman2016skin} dataset is a public challenge dataset dedicated to Skin Lesion Analysis for Melanoma Detection. Derived from the extensive International Skin Imaging Collaboration (ISIC) archive, it represents a significant collection of meticulously curated dermoscopic images of skin lesions. Within this challenge, a subset of $900$ images is designated as training data, while $379$ images serve as testing data, aiming to provide representative samples for analysis.
%The ISIC-2016 \cite{gutman2016skin} dataset is a public challenge dataset of Skin Lesion Analysis Towards Melanoma Detection released with ISBI 2016. This dataset is based on the International Skin Imaging Collaboration (ISIC) Archive, which is the largest publicly available collection of quality controlled dermoscopic images of skin lesions. The challenge employs a subset of representative images with $900$ images as training data and $379$ images as testing data.

%For all experiments, we resized the input images into a resolution of $256\times 256$ and  generated a $32\times 32$ binary mask for each image utilizing the pre-trained SC-VAE$^\curlywedge$ on ImageNet dataset, a spectral clustering algorithm and boundary connectivity information. The inferred binary mask and ground truth mask were resized to $128\times 128$ to calculate the IoU and DICE scores.
For our experiments, we resized the input images into a resolution of $256\times 256$.
Subsequently, we generated a binary mask of size $32\times 32$ per image by employing the pre-trained SC-VAE$^\curlywedge$ on the ImageNet dataset, along with a spectral clustering algorithm and boundary connectivity information \cite{zhu2014saliency}. To compute the IoU and DICE scores, both the inferred binary mask and the ground truth mask were resized to $128\times 128$.
%\subsubsection{Baseline Methods}
\label{section3}
\begin{figure*}[h!]
\centering
\includegraphics[width=16cm]{./Figures/segmen_ffhq_supple3.png}
%\caption{Additional unsupervised image segmentation results. Images are from the validation set of the FFHQ dataset.}
\caption{Additional unsupervised image segmentation results. These results were generated by grouping sparse code vectors into $5$ categories per image, utilizing the pre-trained SC-VAE$^{\curlywedge}$ model and the K-means algorithm. Images are from the validation set of the FFHQ dataset.}
\label{figure:s3}
\end{figure*}

\begin{figure*}[h!]
\centering
\includegraphics[width=16cm]{./Figures/segmentation_imagenet_supple.png}
%\caption{Additional unsupervised image segmentation results by applying K-means algorithm to cluster sparse code vectors per image into $5$ categories using the SC-VAE$^{\curlywedge}$ model. Images are from the validation set of the ImageNet dataset.}
\caption{Additional unsupervised image segmentation results. These results were generated by grouping sparse code vectors into $5$ categories per image, utilizing the pre-trained SC-VAE$^{\curlywedge}$ model and the K-means algorithm. Images are from the validation set of the ImageNet dataset.}
\label{figure:s4}
\end{figure*}

\begin{figure*}[tbp]
\centering
\includegraphics[width=18cm]{./Figures/flower_cub_isic2016_supple2.png}
\caption{Additional unsupervised image segmentation results on Flowers \cite{nilsback2008automated} (\textit{Left Panel}), CUB \cite{WahCUB_200_2011} (\textit{Middle Panel}) and ISIC-2016 \cite{gutman2016skin} (\textit{Right Panel}). (a) input image. (b) ground truth mask. (c) and (e) segmentation results by clustering sparse code vectors per image into $2$ or $3$ classes using a spectral clustering algorithm. (d) and (f) boundary connectivity information \cite{zhu2014saliency}
was used to decide the foreground and background.}
\label{figure:Flower_CUB}
\end{figure*}

\clearpage
\clearpage
{
   \small
   \bibliographystyle{ieee_fullname}
   \bibliography{egpaper_arxiv_V2}
}


% \begin{appendices}

% \section{Section title of first appendix}\label{secA1}



%%=============================================%%
%% For submissions to Nature Portfolio Journals %%
%% please use the heading ``Extended Data''.   %%
%%=============================================%%

%%=============================================================%%
%% Sample for another appendix section			       %%
%%=============================================================%%

%% \section{Example of another appendix section}\label{secA2}%
%% Appendices may be used for helpful, supporting or essential material that would otherwise 
%% clutter, break up or be distracting to the text. Appendices can consist of sections, figures, 
%% tables and equations etc.

% \end{appendices}

%%===========================================================================================%%
%% If you are submitting to one of the Nature Portfolio journals, using the eJP submission   %%
%% system, please include the references within the manuscript file itself. You may do this  %%
%% by copying the reference list from your .bbl file, paste it into the main manuscript .tex %%
%% file, and delete the associated \verb+\bibliography+ commands.                            %%
%%===========================================================================================%%

% \clearpage

% \bibliographystyle{sn-nature}
\bibliographystyle{ieeenat_fullname}
\bibliography{sn-bibliography}
% common bib file

%% if required, the content of .bbl file can be included here once bbl is generated
%%\input sn-article.bbl

% \bibliographystyle{ieeenat_fullname}
% \bibliography{main}

% {
%     \small
%     \bibliographystyle{ieeenat_fullname}
%     \bibliography{main}
% }

\end{document}
