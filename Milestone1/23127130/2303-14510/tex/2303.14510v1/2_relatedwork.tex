\section{Related Work} \label{sec:relatedwork}

The section introduces the research related to the paper, which can be divided into three parts: (1) high-utility pattern mining; (2) top-$k$ utility itemset mining; and (3) targeted pattern mining.

\subsection{High-utility pattern mining}

Frequent itemset mining (FIM) \cite{aggarwal2014frequent,agrawal1994fast,han2000mining} has been extensively studied for decades. However, relying only on frequency cannot bring enough benefits to users. Factors such as quantity and profit should also be considered. For this reason, Chen \textit{et al.} \cite{chan2003mining} put forward a new task called high-utility itemset mining (HUIM). Since then, utility mining research has developed rapidly \cite{gan2021survey,lin2016efficient,song2016high,wu2021haop}. For the convenience of discussion, high-utility itemset mining algorithms are grouped into the following three categories:

\textbf{Apriori-based algorithms}: Since Agrawal \textit{et al.} \cite{agrawal1993mining} proposed the Apriori property in 1994, lots of algorithms based on Apriori have been published. For example, Liu \textit{et al.} \cite{liu2005two} introduced the prominent Two-Phase algorithm to handle the difficulty that the utility, unlike frequency, is neither monotone nor anti-monotone. That algorithm uses an overestimation of the utility called \textit{TWU} (Transaction Weighted Utilization) to find candidate itemsets in a first phase. Thereafter, in a second phase, the database is searched again to determine the exact utility value of each candidate itemset. The IIDS algorithm \cite{li2008isolated} is an improved version of Two-Phase that discards isolated items to shrink the search space. However, the common disadvantage of Apriori-like algorithms is that plenty of candidate patterns are generated, resulting in considerable computational costs and memory consumption.

\textbf{Tree-based algorithms}: Tseng \textit{et al.} \cite{tseng2010up} designed the UP-tree structure, a utility-pattern tree, and introduced the UP-Growth algorithm inspired by FP-Growth. Subsequently, other versions of tree-based algorithms \cite{song2014mining,tseng2012efficient} have been presented. In general, utilizing the UP-tree can prevent many meaningless database scans. When working with large-scale databases, however, this structure grows increasingly complex and occupies a massive amount of memory.

\textbf{Other structure-based algorithms}: HUI-Miner \cite{liu2012mining} utilizes a novel data structure known as a utility-list, which avoids the difficulty of generating numerous candidates. Moreover, the FHM algorithm \cite{fournier2014fhm} reduces the cost of join operations by using a tighter upper bound, which results in outperforming HUI-Miner. However, the join operation on lists of these algorithms takes time and memory. Thus, Zida \textit{et al.} \cite{zida2015efim} proposed the EFIM algorithm with high-utility database projection (HDP) and high-utility transaction merging (HTM) techniques to lower the expensive cost of database passes. The utility-list-based CoUPM algorithm for correlated utility-based pattern mining \cite{gan2019correlated}. In summary, these algorithms integrate various strategies to discover HUIs as efficiently as possible.

\subsection{Top-$k$ utility itemset mining}

Although the above algorithms are effective in finding the desired set of itemsets, the efficiency of mining is strongly related to the selection of the minimum utility threshold. However, it is not easy to identify an appropriate threshold. Many top-$k$ pattern mining algorithms were thus designed to directly discover the set of top-$k$ HUIs, rather than asking users to specify a utility threshold. Top-$k$ HUIM algorithms mainly consist of two types: the first is the two-phase algorithms, and the other is the one-phase algorithms.

\textbf{Two-phase algorithms}: The task of discovering the top-$k$ HUIs was proposed by Wu \textit{et al.} \cite{wu2012mining} with the TKU algorithm, which outperformed HUIM algorithms in terms of speed. The TKU algorithm is a two-phase algorithm. In the first phase, a UP-Tree is built, and promising top-$k$ HUIs are generated. Then, in the second phase, the desired top-$k$ HUIs are selected among them. TKU applies several strategies to filter unpromising candidates during the search \cite{tseng2015efficient} and achieve higher efficiency. Subsequently, REPT \cite{ryang2015top} was introduced with optimizations to record and pre-calculate the utility of items to prune the search space effectively and raise the minimum utility threshold. REPT uses a tree structure and pre-evaluation matrixes as tools to store utility information. However, these two-phase algorithms still generate large sets of candidates, which causes unreasonably long runtimes and high memory usage.

\textbf{One-phase algorithms}: For top-$k$ HUIM, the one-phase TKO algorithm \cite{tseng2015efficient} was developed to solve the shortcomings of two-phase algorithms. TKO takes advantage of the utility-list structure of HUI-Miner, and outperforms the TKU and REPT algorithms according to experiments \cite{tseng2015efficient}. Similarly, another one-phase algorithm called KHMC \cite{duong2016efficient} also discovers the top-$k$ HUIs by using the utility-list structure. In KHMC, an estimated utility co-occurrence pruning (EUCP) technique is applied, which is based on precalculating the TWU of 2-itemsets. Moreover, the algorithm also adds another pruning strategy named early abandoning to avoid completely constructing the lists of unpromising itemsets. Three threshold-raising strategies are able to significantly shrink the search space and enhance the algorithm's efficiency. The THUI algorithm \cite{krishnamoorthy2019mining} has better performance thanks to introducing the concept of Leaf Itemset Utility (LIU), a triangular matrix, which can be implemented with only a small amount of memory to store utility information. Besides, the LIU-E and LIU-LB threshold raising strategies also accelerate the mining speed of the algorithm. THUI greatly outperforms TKO and KHMC, especially for dense or large datasets.

In addition, there are various other top-$k$ pattern mining problems and variations, such as mining top-$k$ sequential patterns \cite{zhang2021tkus}, mining top-$k$ HUIs in data streams \cite{cheng2021etkds}, discover top-$k$ high-utility sequential patterns \cite{zhang2021tkus}, and mining top-$k$ HUIs with negative utility values \cite{sun2021mining}.


\subsection{Targeted pattern mining}

Those algorithms listed above are designed to find all itemsets that meet a single predetermined criterion. Target-oriented query algorithms give an alternative solution to this problem by filtering out unnecessary information. Rather than searching for numerous but mostly insignificant items, the user can enter any target and then discover patterns containing the desired items. Several target-oriented query algorithms based on frequency have been developed in earlier studies. These interactive methods are capable of returning results containing a target. Kubat \textit{et al.} \cite{kubat2003itemset} were among the first to address the issue of processing target queries in a transactional database. They implemented target query processing algorithms for association mining by creating itemset trees that can be progressively updated. Fournier-Viger \textit{et al.} \cite{fournier2013meit} developed the Memory Efficient Itemset Tree (MEIT) to further reduce memory requirements. The tree is optimized to perform incremental modifications when new transactions are inserted, and it employs a node-compression method. For multi-objective mining of big data, the guided FP-growth (GFP-growth) algorithm based on FP-Growth was proposed by Shabtay \textit{et al.} \cite{shabtay2018guided}. In particular, many experiments have illustrated the excellent performance of the algorithm on imbalanced data. Target-oriented mining has also been studied and applied to discover sequential patterns. The targeted mining algorithm for sequential patterns proposed by Chueh \textit{et al.} \cite{chueh2010mining} speeds up the search for the target itemsets by using the reversion of the original sequence and comparing the reversed sequence with the related itemsets. Furthermore, clustering analysis is applied to automatically set time partition values for the task of time-interval sequential pattern mining. A novel target-oriented sequential pattern mining approach was presented by Chand \textit{et al.} \cite{chand2012target}, which uses RFM (recency, frequency, and monetary) constraints. As a result, fewer database projections are done, and the space complexity is reduced. To remove some useless or irrelevant patterns in high utility sequential pattern mining \cite{zhang2021shelf,gan2021explainable}, the TUSQ algorithm \cite{zhang2021tusq} first introduced the concept of utility into target sequence queries. The algorithm does not focus on frequency like previous algorithms, but rather on utility. Recently, the TargetUM algorithm \cite{miao2021targeted} has been proposed to fill the gap and perform target-oriented mining in HUIM.

In general, the TargetUM algorithm provides an integrated approach for high-utility mining with a target query, which serves as the foundation for this research. However, there are no studies combining top-$k$ high-utility methods with target pattern queries. This paper introduces the problem of targeted utility mining with the concept of top-$k$ patterns to prevent the generation of large sets of HUIs and to accurately and quickly process target queries.
