\section{Introduction} \label{sec:introduction}

As digital systems are widely used, the era of big data has arrived, with massive amounts of data being generated. Therefore, figuring out how to mine useful information from complex data has become a major issue. Data mining software assists clients in finding correlations in millions or billions of records, allowing decision-makers to make more informed decisions faster \cite{fournier2022pattern,gan2019utility,gan2019survey}. This technology is influencing many areas of daily life (e.g., market analysis \cite{hemalatha2012market}, biomedicine \cite{peek2014technical}, and network security management \cite{chai2009analyzes}). A typical application is market basket analysis \cite{kuriakose2017efficient}. Market basket analysis primarily analyzes the consumption records of customers to uncover important correlations between products and services, allowing them to rearrange products on shelves or design a combination of promotional packages that users are interested in. For example, if a supermarket's sales records show that users often buy bread and milk at the same time, bundling them on the same shelf can increase sales and revenue.

Previous studies have proposed numerous frequent itemset mining algorithms (FIM) \cite{aggarwal2014frequent, kuriakose2017efficient}, among which the most famous are Apriori \cite{agrawal1994fast} and FP-growth \cite{han2000mining}. Traditional frequent pattern mining algorithms use only one metric, support, to find patterns that users are interested in. As a result of applying the frequent itemset model, itemsets are often found to have a high selling frequency but yield a low profit. But some luxury goods that are not widely sold are highly profitable and deserve attention. To address this issue, utility-oriented pattern mining (UPM) \cite{gan2020huopm,krishnamoorthy2015pruning, song2014mining,wu2021haop} algorithms have been proposed. Internal and external utilities are both included in the concept of utility. The internal utility usually refers to how many times a good appears, while the profit per unit generated by each good is known as the external utility. UPM has been extensively studied in recent decades \cite{gan2018survey,gan2021survey}, since it is beneficial to consider the profitability and relative importance of each item. Some representative UPM algorithms are HUI-Miner \cite{liu2012mining}, which is based on the utility-list structure; UP-Growth \cite{tseng2010up} based on the UP-tree structure; and EFIM \cite{zida2015efim} based on data projection.
 
Although high-utility itemset mining algorithms are able to collect some useful information in specific practical applications, they are not suitable for target-oriented mining tasks. The targeted high-utility itemset querying (TargetUM) algorithm \cite{miao2021targeted} was introduced to resolve this issue by giving consideration to user needs. Take consumer purchases as an example: people pay more attention to the items on their shopping list and less to other items. Therefore, retailers can offer different product recommendations according to the various needs of users as well as discounts on goods in stock. Hence, target-oriented high-utility itemset mining (THUIM) can filter out many irrelevant patterns in the data to precisely focus on those containing items of interest. However, it is complex and laborious to select the minimum utility threshold in these utility-driven data mining algorithms. In particular, if a user is not familiar with a database, he will not have a good understanding of the utilities of itemsets and the content of the transactions. When the utility threshold is set too high, only a small percentage of target high-utility patterns are captured, while a lot of irrelevant information will be generated for a low utility threshold, which has no value to the user. As a result, depending on the number of items and utility distribution in each database, different utility thresholds must be set. In general, decision makers will always repeatedly test and modify the utility threshold until they find a threshold value that will produce the appropriate number of target high-utility itemsets, which can take a long time and is inconvenient.

Consequently, the problem of targeted mining of top-$k$ high utility itemsets is formulated in this paper, for which the user can directly set the number of itemsets to be mined. An algorithm called TMKU (Targeted Mining of top-$k$ high Utility itemset) is designed to perform this task efficiently. TMKU has proved effective and efficient in experiments. The main contributions of this study are outlined below:

\begin{itemize}
	\item This paper considers two important aspects: utility mining and targeted pattern discovery. As far as we know, the TMKU algorithm is the first algorithm to incorporate the concept of top-$k$ mining into target-based utility mining.

	\item A trie structure is used to query the target itemsets and a new data structure, namely TopKMap is used to efficiently store and retrieve the top-$k$ target itemsets, which is beneficial to reduce the query time and search space.

	\item The TMKU algorithm utilizes several upper bounds. Three pruning strategies are used to reduce the search space. In addition, two target utility raising strategies are adopted to quickly raise the threshold and ensure that all top-$k$ THUIs are discovered.

	\item Extensive experiments were conducted on both real and synthetic datasets. The results demonstrate that TKUM is efficient and effective.	
\end{itemize}


The rest of this paper is composed of six sections. A brief review of related work is provided in Section \ref{sec:relatedwork}. Section \ref{sec:preliminaries} describes the top-$k$ THUIM problem and related definitions. Section \ref{sec:algorithm} introduces the proposed TMKU algorithm, as well as the main strategies and techniques applied. The experimental results are presented and evaluated in Section \ref{sec:experiment}. Ultimately, Section \ref{sec:conclusion} concludes this paper and provides an outlook on promising research directions.

