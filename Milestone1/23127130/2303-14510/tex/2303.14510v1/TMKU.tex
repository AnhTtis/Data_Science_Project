%\documentclass[a4paper,fleqn,longmktitle]{cas-sc}
\documentclass[a4paper,fleqn]{cas-sc}



\usepackage[numbers]{natbib}
\usepackage[boxed,commentsnumbered,ruled,linesnumbered]{algorithm2e}


%===========================================
\newtheorem{definition}{Definition}  %gws
\newtheorem{example}{Example}  %gws
\newtheorem{property}{Property}   % gws
\newtheorem{strategy}{Strategy}    % gws
\newtheorem{proof}{Proof}    % gws
\newtheorem{lemma}{Lemma}  %gws
\newtheorem{theorem}{Theorem}   % gws
\newtheorem{constraint}{Constraint}
\newtheorem{corollary}{Corollary}

\begin{document}

\shorttitle{Targeted Mining of Top-$k$ High Utility Itemsets} 

\shortauthors{Shan Huang et~al.}

\title [mode = title]{Targeted Mining of Top-$k$ High Utility Itemsets}                      





\author[1]{Shan Huang}
\ead{shuang9901@gmail.com}
\address[1]{College of Cyber Security, Jinan University, Guangzhou 510632, China}


\author[1,2]{Wensheng Gan}
\cormark[1]
\ead{wsgan001@gmail.com}
\address[2]{Pazhou Lab, Guangzhou 510330, China}
\cortext[cor1]{Corresponding author}


\author[1]{Jinbao Miao}
\ead{osjbmiao@gmail.com}

\author[3]{Xuming Han}
\ead{hanxuming@jnu.edu.cn}
\address[3]{College of Information Science and Technology, Jinan University, Guangzhou 510632, China}


\author[4]{Philippe Fournier-Viger}
\ead{philfv@qq.com}
\address[4]{College of Computer Science and Software Engineering, Shenzhen University, Shenzhen 518060, China}


\begin{abstract}
	Finding high-importance patterns in data is an emerging data mining task known as High-utility itemset mining (HUIM). Given a minimum utility threshold, a HUIM algorithm extracts all the high-utility itemsets (HUIs) whose utility values are not less than the threshold. This can reveal a wealth of useful information, but the precise needs of users are not well taken into account. In particular, users often want to focus on patterns that have some specific items rather than find all patterns. To overcome that difficulty, targeted mining has emerged, focusing on user preferences, but only preliminary work has been conducted. For example, the targeted high-utility itemset querying algorithm (TargetUM) was proposed, which uses a lexicographic tree to query itemsets containing a target pattern. However, selecting the minimum utility threshold is difficult when the user is not familiar with the processed database. As a solution, this paper formulates the task of targeted mining of the top-$k$ high-utility itemsets and proposes an efficient algorithm called TMKU based on the TargetUM algorithm to discover the top-$k$ target high-utility itemsets (top-$k$ THUIs). At the same time, several pruning strategies are used to reduce the memory consumption and execution time. Extensive experiments show that the proposed TMKU algorithm has good performance on real and synthetic datasets.
\end{abstract}



\begin{keywords}
	data science \sep data mining \sep utility mining \sep target itemset \sep targeted mining \sep top-$k$
\end{keywords}


\maketitle




%%%%%%%%%%%%%%%%%%%%%%%%%%%%%%%%%%%%%%%
% \begin{figure}[t]
%     % \begin{subfigure}{1\linewidth}
%     %   \centering
%     % %   \includegraphics[width=1\linewidth]{figs/fig_1_moti_textattn.pdf}  
%     % %   \includegraphics[width=1\linewidth]{figs/fig_1_moti_textattn_v2.pdf}  
%     %   \includegraphics[width=1\linewidth]{figs/fig_1_moti_textattn_v5.pdf}  
%     %   \vspace{-0.5cm}
%     %     \caption{Amount of attention added to each video clip from the source video and query text in the self-attention layers of Moment-DETR encoder.}
%     %     % \caption{Distribution of attention for source and query in Moment-DETR encoder}
%     %     % Visualization of video clip's self-attention score in Moment-DETR encoder.
%     %   \label{fig:fig1_text_attn_ex}
%     % \end{subfigure}%\hfill% or  or \hspace{0.3\textwidth}
%     \vspace{0.2cm}
%     % \begin{subfigure}{1\linewidth}
%       \centering
%     %   \includegraphics[width=1\linewidth]{figs/fig1_moti_negattn.pdf}  
%       \includegraphics[width=1\linewidth]{figs/fig1_moti_negattn_v3.pdf}  
%       \vspace{-0.4cm}
%     %   \caption{Correspondence of saliency scores on the relevance between video clips and the text query.}
%     % \caption{Predicted saliency scores against the video relevant positive query and video irrelevant negative query}
%       \label{fig:fig1_neg_attn_ex}
%     % \end{subfigure}%\hfill% or  or \hspace{0.3\textwidth}
%     \caption{
%     % 원준 원본
%     % (a) Comparison between attention scores of source and query for each video clip~(We sum the attention scores from video and text). 
%     % We observe that the attention scores are dominated by other clips in the source video. 
%     % Text queries do not account for much attention regardless of the relevance to the video clips.
%     % \textbf{(a)} Inspection of the query dependency in Moment-DETR encoder.
%     % % We visualize the attention score of video tokens in the transformer encoder and observe that text query accounts for only a low portion of attention.
%     % % This tendency occurs regardless of the relevance between the text query and video clips. 
%     % We visualize the attention score of video tokens in the transformer encoder and observe 1) text query only accounts for a low portion of attention, and 2) relevance between video-query pair does not affect the attention scores ratio of text.
%     \textbf{(b)} Comparison of highlight-ness when relevant and non-relevant queries are input.
%     As observed in , existing work only uses queries to play an insignificant role, thereby may not be capable of detecting false queries and considering the video-query relevance even when the problem in (a) is resolved. 
%     % \SE{} % 이 부분이 "not capable of" 란 용어가 세다는 피드백이 있는 듯 합니다. 이러한 능력이 없다는 것은 굉장히 강한 어조인거 같기는 하고, 이러한 경우들이 종종 있다거나 좀 약화시킬 필요가 있어보이긴 하네요.
%     On the other hand, our QD-DETR yields a query-dependent representation that the relevance between the source video and query text is updated in the saliency scores.
%     There is a large gap between positive and negative saliency scores, and scores are consistent since the clips are all highly correlated to others.
%     }
%     \label{fig:motivation_ex}
%     % \captionsetup{belowskip=13pt}
%     % \setlength{\belowcaptionskip}{-10pt}
% \end{figure}
\begin{figure}
    \centering
    \includegraphics[width=1\linewidth]{figs/fig1_moti_negattn_1111.pdf}
    % \includegraphics[width=1\linewidth]{figs/fig1_moti_negattn_1109.pdf}
    % \includegraphics[width=1\linewidth]{figs/fig1_moti_negattn_stat.pdf}
    \vspace{-0.6cm}
    \caption{
        % \SE{} % 수정 필요
        Comparison of highlight-ness~(saliency score) when relevant and non-relevant queries are given.
        We found that the existing work only uses queries to play an insignificant role, thereby may not be capable of detecting negative queries and video-query relevance; saliency scores for clips in ground-truth~(GT) moments are low and equivalent for positive and negative queries.
        % This also results in mispredicted moments when ground-truth~(GT) moment is dominated by clips unrelated to GT since their prediction is highly focused on the video.
        % \SE{} % 여기 한번 더 보면 좋을 듯 합니다. GT moment에 unrelated한 clip이 많으면? label이 틀렷을 경우를 말씀하시는건지?
        % As observed in saliency graph, existing work only uses queries to play an insignificant role, thereby may not be capable of detecting false queries and considering the video-query relevance.
        On the other hand, query-dependent representations of QD-DETR result in corresponding saliency scores to the video-query relevance and precisely localized moments.
        % On the other hand, our QD-DETR yields a query-dependent representation that the
        % saliency scores are in accordance with the relevance between the video and query.
        % text is in accordance with the saliency scores.
        % There is a large gap between positive and negative saliency scores, and scores are consistent since the clips are all highly correlated to others.
}
    \label{fig:motivation_ex}
\end{figure}


\section{Introduction}
% 원준 원본
% Along with the advance of digital devices and platforms, video is now one of the most desired data type for consumers. However, although the large information capacity of videos may be beneficial in many aspects, e.g., informative and entertaining, on the contrary perspective, videos are time-consuming, and hard to search for desirable moments. 
% This has led many creators to use extra manpower to crop and edit the video to generate highlight clips to gain the consumer’s attention.
Along with the advance of digital devices and platforms, video is now one of the most desired data types for consumers~\cite{apostolidis2021video,wu2017deep}.
% SE: Video aware deep learning application & survey papers?
Although the large information capacity of videos might be beneficial in many aspects, e.g., informative and entertaining, inspecting the videos is time-consuming, so that it is hard to capture the desired moments~\cite{anne2017localizing,apostolidis2021video}. 
% This has led many creators to use extra manpower to crop and edit the video to generate highlight clips to gain the consumer’s attention.


% On the other side, 
Indeed, the need to retrieve user-requested or highlight moments within videos is greatly raised.
Numerous research efforts were put into the search for the requested moments in the video~\cite{anne2017localizing, gao2017tall, liu2015multi, escorcia2019temporal} and summarizing the video highlights~\cite{zhang2016video, mahasseni2017unsupervised, badamdorj2022contrastive, wei2022learning}.
% Numerous research efforts were put into the search for the requested moments in the video~\cite{anne2017localizing, gao2017tall, liu2015multi, escorcia2019temporal}, summarizing the video to generate highlights was another popular topic~\cite{zhang2016video, mahasseni2017unsupervised, badamdorj2022contrastive, wei2022learning}.
Recently, Moment-DETR~\cite{momentdetr} further spotlighted the topic by proposing a QVHighlights dataset that enables the model to perform both tasks, retrieving the moments with their highlight-ness, simultaneously.

% 원준 원본
% To detect the desired moments, previous works employed transformer encoder-decoder architectural designs to fuse the text query into the video representations. Moment-DETR~\cite{mDETR} modified detection transformer to process capture the moment as a set, and UMT~\cite{umt} implemented transformer decoder as to output clip-wise saliency. 
% Yet to their outstanding breakthroughs in the literature of moment retrieval with the seminal architectures, their limitation is that the role of the given text query is insignificant in representing the query-conditioned video representation; the attention mechanism of moment DETR is not explicitly conditioned on the text query, and the text query is conditioned on multi-modal clips where the differences between the clips are smoothed after encoding process in UMT.



% \begin{figure}[t]
% \centering
%     \begin{subfigure}[l]{0.37\linewidth}
%       \centering
%       \vspace{0.20cm}
%     %   \includegraphics[width=1\linewidth]{figs/fig_1_moti_textattn.pdf}  
%     %   \includegraphics[width=1\linewidth]{figs/fig_1_moti_textattn_v2.pdf}  
%       \includegraphics[width=1\linewidth]{figs/fig1_moti_violin_a.pdf}  
%       \vspace{-0.60cm}
%     %   \caption{text attention}
%         \caption{Importance of queries in video representation}
%       \label{fig:fig1_text_attn}
%     \end{subfigure}%\hfill% or  or \hspace{0.3\textwidth}
%     \vspace{0.2cm}
%     \begin{subfigure}[r]{0.61\linewidth}
%       \centering
%     %   \includegraphics[width=1\linewidth]{figs/fig1_moti_negattn.pdf}  
%       \includegraphics[width=1\linewidth]{figs/fig1_moti_violin_b.pdf}  
%     %   \caption{neg attention}
%         % \caption{Relation between the highlight-ness and the relevance between videos and query texts.}
%         \caption{Highlight-ness~(saliency) histogram of positive and negative video-query pairs\SE{}}
%       \label{fig:fig1_neg_attn}
%     \end{subfigure}%\hfill% or  or \hspace{0.3\textwidth}
%     % \vspace{-0.2cm}
%     \caption{Overall statistics for attention scores in Fig.~\ref{fig:motivation_ex} in QVHighlights dataset. 
%     (a) For the attention scores that measure how much the text query is generally involved in video representation, we use violin plots to show the probability density. We plot the score for each layer in the encoder.
%     % (b) Using the histogram, we compare how the baseline and QD-DETR yield different salient scores given the positive and negative video-text pairs.
%     (b) Saliency histogram shows the distributional gap between positive and negative video-text query pairs of baseline~(Moment-DETR) and proposed QD-DETR.\SE{}
%     }
%     \label{fig:motivation}
%     % \captionsetup{belowskip=13pt}
%     % \setlength{\belowcaptionskip}{-10pt}
% \end{figure}

% \begin{figure}[t]
% \centering

%     \begin{subfigure}[r]{1\linewidth}
%       \centering
%       \hspace{-0.2cm}
%     %   \includegraphics[width=1\linewidth]{figs/fig1_moti_negattn.pdf}  
%       \includegraphics[width=1.1\linewidth]{figs/fig1_moti_violin_a_v2.pdf}  
%     %   \caption{neg attention}
%         % \caption{Relation between the highlight-ness and the relevance between videos and query texts.}
%         \vspace{-0.5cm}
%         % \caption{Saliency histogram of positive and negative video-query pairs}
%         \caption{We plot the histograms and its average value~(dotted line) to compare saliency scores when true and false text queries are given for each method. (left) Since the video representations do not include much textual information, both the true and false queries yield similar saliency scores. (Middle) Even when the video representation is enforced to be updated with the textual information, the issue is not much resolved. (Right) By extracting discriminative features in the text query, distributions are differentiated.
%         % \SE{} % R1@0.5 설명
%         Also, R1@0.5 indicates evaluation metric, Recall at 1 with IoU 0.5 threshold on QVhighlight \textit{val} set.
%         }
%       \label{fig:fig1_neg_attn}
%     \end{subfigure}%\hfill% or  or \hspace{0.3\textwidth}
%     \\
%     \begin{tabular}{cc}
%     \hspace{-0.2cm}
%         \begin{minipage}{.4\linewidth}
%             \begin{subfigure}[l]{1\linewidth}
%               \centering
%             %   \vspace{0.20cm}
%             %   \includegraphics[width=1\linewidth]{figs/fig_1_moti_textattn.pdf}  
%             %   \includegraphics[width=1\linewidth]{figs/fig_1_moti_textattn_v2.pdf}  
%               \includegraphics[width=1\linewidth]{figs/fig1_moti_violin_a.pdf}  
%               \vspace{-0.60cm}
%             %   \caption{text attention}
%                 \caption{Importance of queries in video representation}
%               \label{fig:fig1_text_attn}
%             \end{subfigure}%\hfill% or  or \hspace{0.3\textwidth}
%         \end{minipage}
        
%         \begin{minipage}{.6\linewidth}
%             \vspace{-0.2cm}
%             \caption{Overall statistics of Fig.~\ref{fig:motivation_ex} in QVHighlights dataset. 
%             (a) Saliency histogram shows the distributional gap between positive and negative video-text query pairs.
%             % (a) For the attention scores that measure how much the text query is generally involved in video representation, we use violin plots to show the probability density. We plot the score for each layer in the encoder.
%             % (b) Using the histogram, we compare how the baseline and QD-DETR yield different salient scores given the positive and negative video-text pairs.
%             % (b) Text ratio in self-attention layer to  of Moment-DETR
%             % (b) Ratio of text when representing video tokens in self-attention of Moment-DETR.
%             % (b) Magnitude of attention text query involved.
%             % (b) Attention score of video tokens
%             % (b) Magnitude of text query to refine the video tokens in self-attention layer of Moment-DETR.
%             (b) Probability density depicting the weight of the text query in attention score for video clips. Scores are from the self-attention layers in Moment-DETR encoder.
%             % (b) The text query ratio in attention score of video clips (Self-attention layer in Moment-DETR encoder). We use violin plots to show probability density.
%             % 텍스트 쿼리가, 비디오 피쳐에 얼만큼 attend 하는지
%             }
%         \end{minipage}
    
%     \end{tabular}
%     \vspace{-0.5cm}
%     \label{fig:moti}
%     % \captionsetup{belowskip=13pt}
%     % \setlength{\belowcaptionskip}{-10pt}
% \end{figure}


% \begin{figure}
%     \centering
%     % \includegraphics[width=1\linewidth]{figs/fig1_moti_negattn_1109.pdf}
%     \includegraphics[width=1\linewidth]{figs/fig1_moti_negattn_stat_v2.pdf}
%     \vspace{-0.8cm}
%     \caption{
%         Histogram of saliency when the positive and negative queries are given. We plot the histograms and its average value~(dotted line) to compare saliency scores when relevant~(positive) and irrelevant~(negative) text queries are given for each method. (Left) Since the video representations do not properly reflect textual information, both the positive and negative queries yield similar saliency scores. 
%         % (Middle) Even when the video representation is enforced to be updated with the textual information, the issue is not much resolved. 
%         (Right) By representing video clips in query-dependent manner, distributions are differentiated.
%     }
%     \vspace{-0.6cm}
%     \label{fig:motivation}
% \end{figure}


% One of the demanding task is moment retrieval task, which is detecting the desired moments from the given query, typically the text query.
When describing the moment, one of the most favored types of query is the natural language sentence~(text)\cite{anne2017localizing}. 
While early methods utilized convolution networks~\cite{zhang2020learning, gao2021fast, wang2020temporally}, recent approaches have shown that deploying the attention mechanism of transformer architecture is more effective to fuse the text query into the video representation.
% To handle these modalities, previous works simply employed the attention mechanism of transformer architecture to fuse the text query into the video representation.
For example, Moment-DETR~\cite{momentdetr} introduced the transformer architecture which processes both text and video tokens as input by modifying the detection transformer~(DETR), and UMT~\cite{umt} proposed transformer architectures to take multi-modal sources, e.g., video and audio. 
Also, they utilized the text queries in the transformer decoder.
Although they brought breakthroughs in the field of MR/HD with seminal architectures, they overlooked the role of the text query.
To validate our claim, we investigate the Moment-DETR~\cite{momentdetr} in terms of the impact of text query in MR/HD~(Fig.\ref{fig:motivation_ex}).
Given the video clips with a relevant positive query and an irrelevant negative query, we observe that the baseline often neglects the given text query when estimating the query-relevance scores, i.e., saliency scores, for each video clip.
% the output saliency score, i.e. query-relevance scores.
% Based on the observation, we traced the actual saliency prediction of the model against both the video-relevant query and the irrelevant dummy one where we find that the baseline often neglects the given text query when estimating the query-relevance scores of video clips.
% For example, in Fig.~\ref{fig:motivation_ex}, saliency scores are not affected even when the query is substituted with the dummy.
% % General statistics for Fig.~\ref{fig:motivation_ex} is shown in Fig.~\ref{fig:motivation}. 
% General statistics corresponding to Fig.~\ref{fig:motivation_ex} are also shown in Fig.~\ref{fig:motivation}.



% The limitation of the concrete baseline~\cite{momentdetr} is inspected in two different aspects; 1) Utilization of text-query in the encoding process and 2) the output saliency score, i.e. query-relevance scores.
% Firstly, we visualize the attention score when video clips are given as a query in self-attention. 
% We observe that the text queries have relatively small impacts compared to other video features, as shown in Fig.~\ref{fig:fig1_text_attn_ex}.
% That is, the text does not account for much in representing every video clip, although the goal of MR/HD is to detect query-relevant moments.
% Based on the observation, we traced the actual saliency prediction of the model against both the video-relevant query and the irrelevant dummy one where we find that the baseline often neglects the given text query when estimating the query-relevance scores of video clips.
% For example, in Fig.~\ref{fig:motivation_ex}, saliency scores are not affected even when the query is substituted with the dummy.
% % General statistics for Fig.~\ref{fig:motivation_ex} is shown in Fig.~\ref{fig:motivation}. 
% General statistics are also shown in Fig.~\ref{fig:motivation}.

% Consequently, in Fig.~\ref{fig:fig1_neg_attn_ex}~(b), we found that the baseline often neglects the given text query when estimating the query-relevance scores of video clips; 
% For example, 


% We validate the previous work sometimes neglects the given query when estimating the saliency of video clips.
% For example, there is an example that the saliency scores from positive and negative queries cannot be distinguishable, as shown in Fig.~\ref{fig:fig1_neg_attn_ex}.
% % 우리는 추가로 text attention을 추가도 해봤지만, 효과가 있긴 했으나, still 이슈가 있는 것을 확인하였다?
% % Still, we observe that assuring the high attendance of text queries does not resolve the overlap which motivates us to question the quality of the naive use of task-agnostic text representation~\cite{momentdetr, umt}.
% We found that introducing the text-attention for ensuring the high attendance of text queries relieve the overlap, but there still be a severe overlap.


% To validate their limitations, we inspect the impacts of text queries in the concrete baseline~\cite{momentdetr} with the two different aspects, 1) tendency of attention in self-attention layer and 2) saliency score, i.e. query-relevance scores. \SE{} % attention 이 갑자기 등장하는가?
% Firstly, we visualize the attention score when video clips are given as a query in self-attention. We observe the text queries have relatively low attention scores compared to the video features, as shown in Fig.~\ref{fig:fig1_text_attn_ex}.
% That is, the text does not account for much in representing every video clip, although the goal of MR/HD is to detect query-relevant moments.
% Based on this observation, we trace the actual saliency prediction of the model against both positive and negative text queries.
% We validate the previous work sometimes neglects the given query when estimating the saliency of video clips.
% For example, there is an example that the saliency scores from positive and negative queries cannot be distinguishable, as shown in Fig.~\ref{fig:fig1_neg_attn_ex}.
% % 우리는 추가로 text attention을 추가도 해봤지만, 효과가 있긴 했으나, still 이슈가 있는 것을 확인하였다?
% % Still, we observe that assuring the high attendance of text queries does not resolve the overlap which motivates us to question the quality of the naive use of task-agnostic text representation~\cite{momentdetr, umt}.
% We found that introducing the text-attention for ensuring the high attendance of text queries relieve the overlap, but there still be a severe overlap.



% Thus, we 
% query dependency를 높이기 위해 
% Cross-attention? text-attention? detailed explanation on text-attention should be needed?
% By handling these two issues, we find that more precise retrieval can be achieved.
% 
% 
%
% By projecting video-discriminative text features with high text attendance to source video, we f 
% We also find the need to improve the quality of query features since assuring high text attendance also results in...
% pairs are not finetuned to be discriminative that even the similarity within the pairs does not reflect the relevance between the query and the video clips.
% General statistics for Fig.~\ref{fig:motivation_ex} is shown in Fig.~\ref{fig:motivation}. 
% \SE{} % 이거 ??로 뜨는데, 위처럼 figure 그리면 label이 안되는걸까요
% \SE{}
% 형님 아래 사항 생각 좀 해보는게 좋을 거 같아요.
% fig 1. (a) 그림만 봤을 때 모든 clip에 대해 text attention이 일정이상 존재하긴 하니까, 뭔가 not assured to be conditioned가 와닿지 않는거 같아요.
% + 왜 text가 항상 attend 해야하나?
% not assured to be conditioned --> text shows relatively low affects compared to video 같이 실제 나타난 현상까지 같이 적으면 어떨까 싶어요.
% fig 1. (b) 덜 반영한다?

% \SU{}
% 일단 text가 attend 잘 되어야 한다는 것에 좀 궁금점이 생깁니다. 결국에는 text와 관련있는 frame들을 attend해서 higlight를 찾아야 하는게 아닐까요? 그리고, 현제 저희의 모델 구조상 text query가 Key와 Value로 거의 활용되고 있는데 그렇다면 결국에는 해당 모델은 text에 대한 attention이 전혀 없다고 봐도 무방하지 않을까요? 그런 면에서 text attention을 강조하는게 좀 걸리긴 합니다.

% Specifically, the text query is not assured to be explicitly conditioned on every clip of the video, and as the query texts are evenly treated, discriminative keywords may not be spotlighted.
% attention mechanism of Moment-DETR is not explicitly conditioned on the text query as shown in Fig~\ref{}(d), and in UMT, the text are only used for conditioning the queries while the video representation are refined itself by self-attention.

% \begin{figure}[t]
%     \begin{subfigure}{1\linewidth}
%       \centering
%     %   \includegraphics[width=1\linewidth]{figs/fig_1_moti_textattn.pdf}  
%     %   \includegraphics[width=1\linewidth]{figs/fig_1_moti_textattn_v2.pdf}  
%       \includegraphics[width=1\linewidth]{figs/fig_1_moti_textattn_v4.pdf}  
%       \vspace{-0.5cm}
%     %   \caption{text attention}
%         \caption{Distribution of attention scores in Moment-DETR encoder}
%       \label{fig:fig1_text_attn}
%     \end{subfigure}%\hfill% or  or \hspace{0.3\textwidth}
%     \vspace{0.2cm}
%     \begin{subfigure}{1\linewidth}
%       \centering
%     %   \includegraphics[width=1\linewidth]{figs/fig1_moti_negattn.pdf}  
%       \includegraphics[width=1\linewidth]{figs/fig1_moti_negattn_v2.pdf}  
%       \vspace{-0.5cm}
%     %   \caption{neg attention}
%         \caption{Saliency score against positive and negative text queries}
%       \label{fig:fig1_neg_attn}
%     \end{subfigure}%\hfill% or  or \hspace{0.3\textwidth}
%     \vspace{0.2cm}
%     \begin{subfigure}{1\linewidth}
%       \centering
%     %   \includegraphics[width=1\linewidth]{figs/fig1_moti_violin.pdf}  
%       \includegraphics[width=1\linewidth]{figs/fig1_moti_violin_v2.pdf}  
%       \vspace{-0.5cm}
%       \caption{violin}
%       \label{fig:fig1_violin}
%     \end{subfigure}%\hfill% or  or \hspace{0.3\textwidth}
%     \vspace{-0.2cm}
%     \caption{(a) 1. portion of text attention vs. video attention 2. relation with text query and content (e.g. fg, bg) of clip seems not to affect the attention score
%     (b) 1. high variability even though entire clips are highly correlated with the given text query 2. positive and negative query makes overlaps on saliency score distribution
%     (3) actual distribution on validation dataset.}
%     \label{fig:motivation}
%     % \captionsetup{belowskip=13pt}
%     % \setlength{\belowcaptionskip}{-10pt}
% \end{figure}

To this end, we propose Query-Dependent DETR~(QD-DETR) that produces query-dependent video representation.
% Our key focus is to ensure each clip in predicted moments is explicitly conditioned by the query, particularly on the video-descriptive portion of the text query.
% Our key focus is to ensure that query-relevant clips are predicted by enforcing each clip to be explicitly conditioned by the query.
%Our key focus is to ensure that the model prediction for each clip is highly relevant to the query.
Our key focus is to ensure that the model's prediction for each clip is highly dependent on the query.
% by enforcing each clip to be explicitly conditioned by the query. :)
% hmm...
% \SE {} % "query-relevant clips are predicted" 이 문장이 좀 애매한거 같습니다. relevant 클립을 놓지지 않고 찾는 것을 보장한다? 이런 느낌인지 아니면 높은 saliency 를 주는게 목적이다? model prediction이 query-relevance를 반영하는 것을 보장한다?
% Our key focus is to ensure that the model prediction reflects query-relevance of clips by enforcing each clip to be explicitly conditioned by the query.
First, to fully utilize the contextual information in the query, we revise the transformer encoder to be equipped with cross-attention layers at the very first layers.
% 상익's thought :  single video - query간의 관계만 고려 - 같은 word가 더 많이 쓰이는 것을 보고 
% 교수님's thought : neg pair 를 쓰면 쿼리를 보지 않고서는 video clip간만 고려하는 것이 사라짐. 왜냐면 0으로 내보내야 하기 때문. --> SE: relative difference 만 고려하다가, 
By inserting a video as the query and a text as the key and value of the cross-attention layers, our encoder enforces the engagement of the text query in extracting video representation.
% 원준 교수님 코멘트 반영해서 다시
Then, in order to not only inject a lot of textual information into the video feature but also make it fully exploited, we leverage the negative video-query pairs generated by mixing the original pairs.
Specifically, the model is learned to suppress the saliency scores of such  negative~(irrelevant) pairs.
Our expectation is the increased contribution of the text query in prediction since the videos will be sometimes required to yield high saliency scores and sometimes low ones depending on whether the text query is relevant or not.
% \SE{}
% learns to?
% By suppressing the saliency scores of the irrelevant video-query pairs, the model learns to spotlight only the video-specific discriminative words in the query.
% % \SE{} % ====================== 상익 수정 ========================
% However, this architectural design still lacks the capability of identifying the video-descriptive keywords in the query.
% % However, this architectural design still lacks in identifying proper query relevance.
% This is because the current training scheme only focuses on the interactions of video and clips within a single video while neglecting information shared throughout the entire video.
% % We argue the problem of the current training scheme that only focuses on distinguishing the clips in a single video while neglecting information shared throughout the entire video.
% Therefore, we leverage the negative video-query relationships to enhance the capability of identifying the contextual similarity of query and video clips.
% 
% 원준 원본 
% However, this architectural design heavily relies on the quality of the text query.
% Therefore, we leverage the negative video-query relationships to enable the model to emphasize key corresponding query features.
% By suppressing the saliency scores of the irrelevant video-query pairs, the model learns to spotlight only the video-specific discriminative words in the query.
% =========================================================
Lastly, to apply the dynamic criterion to mark highlights for each instance, we deploy a saliency token to represent the entire video and utilize it as an input-adaptive saliency criterion. 
With all components combined, our QD-DETR produces query-dependent video representation by integrating source and query modalities.
This further allows the use of positional queries~\cite{dabdetr} in the transformer decoder.
% Furthermore, we can exploit the advanced DETR decoder architectures using the positional information, e.g., DAB-DETR, since our encoded tokens consist of identical position representations from a single modality.
% \SE{} % ====================== 상익 수정 ========================
% Furthermore, we can exploit the advanced DETR decoder architectures using the positional information, e.g., DAB-DETR, since our video clip tokens consist of identical position representations from a single modality.
% 원준 원본
% It also enables the use of advanced DETR decoder architectures, e.g., DAB-DETR, for the first time, as these works exploit the position information within a single modality.
% =========================================================
Overall, our superior performances over the existing approaches validate the significance of the role of text query for MR/HD.
% Our extensive experiments on QVHighlights, TVSum, and Charades-STA datasets validate the significance of considering the role and the quality of text query.

% All components combined with dynamic anchor moments for the query of decoder, our FOQUE fosters the query-dependent video representation, thereby making the 
% All components combined, our modified transformer encoding process fosters the query-dependent video representation thereby achieving the state-of-the-art results on various benchmarks of moment-retrieval and highlight detection.
	
% -	Video Platform & Streamer & Consumer의 증가. 
% Video는 다른 데이터 타입보다 정보가 많아 유용하지만, 이는 다른 말로 해석하면 video를 보는 것은 time-consuming 하고, 원하는 것을 찾아보기에는 힘들 수 있음.
% 따라서, 많은 매체에서는 사람들의 더 많은 이목을 끌기 위해 highlight 비디오라는 것을 편집하여 공유도 함.
% 하지만, highlight video를 만들기 위해 사람의 노력이 필요한 현 시점에서, This spotlights the need to retrieve the user-requested / Highlight moments in the video.

% -	이전에도 이러한 문제를 해결하기 위해 (asdfasdf) for moment retrieval, (asdfasdf) for highlight detection 등이 제안 되었지만, 이들은 비디오의 특정 영역을 찾는다는 공통된 목적을 가지고 있으면서도, 데이터 셋의 한계로 인해 따로 연구되었음. 이를 문제 삼으며, 최근에는 두 task를 동시에 학습할 수 있는 dataset이 소개 되었는데, 컴퓨터비전에서 최근 각광을 받고 있는 Transformer 모델 도입과 함께 큰 발전을 거듭하고 있음.

% -	구체적으로, 이 두가지 task를 수행하기 위해서는 transformer를 두가지 방법으로 이용할 수 있는데, moment-DETR 처럼 moment 를 clip의 set 단위로 예측할 수 있고, UMT 처럼 clip-wise prediction을 할 수 있음. 하지만, 이들은 query를 condition이 아닌 video와 동등한 레벨로 취급하거나 [mDETR], 매 클립이 self-attention으로 mixing 된 후에 condition을 걸어주어 clip간의 차이를 확실하지 이용하지 못하였고, 또한, 확실하게 condition으로 주지 못하였고, video와 query 사이의 관계를 한정적으로만 이용하였다.

% -	따라서, we explore three different ways to fully exploit query information. First, we design one-way cross-attention layer to condition every clip with the query features. Then, we utilized the negative video-text pairs to better model the relationships between the video and the text embeddings. Lastly, we define the saliency token to be the video-query dependent saliency estimator.


















% ===================== neg pair 부분 ===========================
% Nevertheless, the current training scheme, only considering the given video-query pair, still disturbs the model from identifying proper query-relevance prediction.
% In detail, the model focus on learning the fine-grained discrepancy between video clips, while neglecting the information they share, which contains significant clues to understand the context of video.
% Therefore, we leverage the negative video-query relationships to enhance the capability of identifying the contextual similarity of query and video clips.
% Therefore, we leverage the negative video-query relationships by suppressing those pairs, so that enhance the capability of identifying the contextual similarity of query and video clips.
% We hypothsize the diversity in query-video pairs are insufficient to learn the general relationship between text query and video.
% Therefore, we leverage the negative video-query relationships by suppressing the saliency scores of the irrelevant video-query pairs.
% However, this architectural design still lacks in identifying proper query relevance.
% We argue that the current training scheme only focuses on learning the fine-grained discrepancy between clips in a single video, while neglecting the information they share, which contains significant clues to understand the context of the video.
% Therefore, we leverage the negative video-query relationships to enhance the capability of identifying the contextual similarity of query and video clips.
% However, this architectural design still lacks in identifying proper query relevance.
% We argue the problem of the current training scheme that only focuses on learning the fine-grained discrepancy between clips in a single video.
% That is, the current design neglects the information shared throughout the video, although it contains significant clues to understand the context of the video.
\section{Related Work} \label{sec:relatedwork}

The section introduces the research related to the paper, which can be divided into three parts: (1) high-utility pattern mining; (2) top-$k$ utility itemset mining; and (3) targeted pattern mining.

\subsection{High-utility pattern mining}

Frequent itemset mining (FIM) \cite{aggarwal2014frequent,agrawal1994fast,han2000mining} has been extensively studied for decades. However, relying only on frequency cannot bring enough benefits to users. Factors such as quantity and profit should also be considered. For this reason, Chen \textit{et al.} \cite{chan2003mining} put forward a new task called high-utility itemset mining (HUIM). Since then, utility mining research has developed rapidly \cite{gan2021survey,lin2016efficient,song2016high,wu2021haop}. For the convenience of discussion, high-utility itemset mining algorithms are grouped into the following three categories:

\textbf{Apriori-based algorithms}: Since Agrawal \textit{et al.} \cite{agrawal1993mining} proposed the Apriori property in 1994, lots of algorithms based on Apriori have been published. For example, Liu \textit{et al.} \cite{liu2005two} introduced the prominent Two-Phase algorithm to handle the difficulty that the utility, unlike frequency, is neither monotone nor anti-monotone. That algorithm uses an overestimation of the utility called \textit{TWU} (Transaction Weighted Utilization) to find candidate itemsets in a first phase. Thereafter, in a second phase, the database is searched again to determine the exact utility value of each candidate itemset. The IIDS algorithm \cite{li2008isolated} is an improved version of Two-Phase that discards isolated items to shrink the search space. However, the common disadvantage of Apriori-like algorithms is that plenty of candidate patterns are generated, resulting in considerable computational costs and memory consumption.

\textbf{Tree-based algorithms}: Tseng \textit{et al.} \cite{tseng2010up} designed the UP-tree structure, a utility-pattern tree, and introduced the UP-Growth algorithm inspired by FP-Growth. Subsequently, other versions of tree-based algorithms \cite{song2014mining,tseng2012efficient} have been presented. In general, utilizing the UP-tree can prevent many meaningless database scans. When working with large-scale databases, however, this structure grows increasingly complex and occupies a massive amount of memory.

\textbf{Other structure-based algorithms}: HUI-Miner \cite{liu2012mining} utilizes a novel data structure known as a utility-list, which avoids the difficulty of generating numerous candidates. Moreover, the FHM algorithm \cite{fournier2014fhm} reduces the cost of join operations by using a tighter upper bound, which results in outperforming HUI-Miner. However, the join operation on lists of these algorithms takes time and memory. Thus, Zida \textit{et al.} \cite{zida2015efim} proposed the EFIM algorithm with high-utility database projection (HDP) and high-utility transaction merging (HTM) techniques to lower the expensive cost of database passes. The utility-list-based CoUPM algorithm for correlated utility-based pattern mining \cite{gan2019correlated}. In summary, these algorithms integrate various strategies to discover HUIs as efficiently as possible.

\subsection{Top-$k$ utility itemset mining}

Although the above algorithms are effective in finding the desired set of itemsets, the efficiency of mining is strongly related to the selection of the minimum utility threshold. However, it is not easy to identify an appropriate threshold. Many top-$k$ pattern mining algorithms were thus designed to directly discover the set of top-$k$ HUIs, rather than asking users to specify a utility threshold. Top-$k$ HUIM algorithms mainly consist of two types: the first is the two-phase algorithms, and the other is the one-phase algorithms.

\textbf{Two-phase algorithms}: The task of discovering the top-$k$ HUIs was proposed by Wu \textit{et al.} \cite{wu2012mining} with the TKU algorithm, which outperformed HUIM algorithms in terms of speed. The TKU algorithm is a two-phase algorithm. In the first phase, a UP-Tree is built, and promising top-$k$ HUIs are generated. Then, in the second phase, the desired top-$k$ HUIs are selected among them. TKU applies several strategies to filter unpromising candidates during the search \cite{tseng2015efficient} and achieve higher efficiency. Subsequently, REPT \cite{ryang2015top} was introduced with optimizations to record and pre-calculate the utility of items to prune the search space effectively and raise the minimum utility threshold. REPT uses a tree structure and pre-evaluation matrixes as tools to store utility information. However, these two-phase algorithms still generate large sets of candidates, which causes unreasonably long runtimes and high memory usage.

\textbf{One-phase algorithms}: For top-$k$ HUIM, the one-phase TKO algorithm \cite{tseng2015efficient} was developed to solve the shortcomings of two-phase algorithms. TKO takes advantage of the utility-list structure of HUI-Miner, and outperforms the TKU and REPT algorithms according to experiments \cite{tseng2015efficient}. Similarly, another one-phase algorithm called KHMC \cite{duong2016efficient} also discovers the top-$k$ HUIs by using the utility-list structure. In KHMC, an estimated utility co-occurrence pruning (EUCP) technique is applied, which is based on precalculating the TWU of 2-itemsets. Moreover, the algorithm also adds another pruning strategy named early abandoning to avoid completely constructing the lists of unpromising itemsets. Three threshold-raising strategies are able to significantly shrink the search space and enhance the algorithm's efficiency. The THUI algorithm \cite{krishnamoorthy2019mining} has better performance thanks to introducing the concept of Leaf Itemset Utility (LIU), a triangular matrix, which can be implemented with only a small amount of memory to store utility information. Besides, the LIU-E and LIU-LB threshold raising strategies also accelerate the mining speed of the algorithm. THUI greatly outperforms TKO and KHMC, especially for dense or large datasets.

In addition, there are various other top-$k$ pattern mining problems and variations, such as mining top-$k$ sequential patterns \cite{zhang2021tkus}, mining top-$k$ HUIs in data streams \cite{cheng2021etkds}, discover top-$k$ high-utility sequential patterns \cite{zhang2021tkus}, and mining top-$k$ HUIs with negative utility values \cite{sun2021mining}.


\subsection{Targeted pattern mining}

Those algorithms listed above are designed to find all itemsets that meet a single predetermined criterion. Target-oriented query algorithms give an alternative solution to this problem by filtering out unnecessary information. Rather than searching for numerous but mostly insignificant items, the user can enter any target and then discover patterns containing the desired items. Several target-oriented query algorithms based on frequency have been developed in earlier studies. These interactive methods are capable of returning results containing a target. Kubat \textit{et al.} \cite{kubat2003itemset} were among the first to address the issue of processing target queries in a transactional database. They implemented target query processing algorithms for association mining by creating itemset trees that can be progressively updated. Fournier-Viger \textit{et al.} \cite{fournier2013meit} developed the Memory Efficient Itemset Tree (MEIT) to further reduce memory requirements. The tree is optimized to perform incremental modifications when new transactions are inserted, and it employs a node-compression method. For multi-objective mining of big data, the guided FP-growth (GFP-growth) algorithm based on FP-Growth was proposed by Shabtay \textit{et al.} \cite{shabtay2018guided}. In particular, many experiments have illustrated the excellent performance of the algorithm on imbalanced data. Target-oriented mining has also been studied and applied to discover sequential patterns. The targeted mining algorithm for sequential patterns proposed by Chueh \textit{et al.} \cite{chueh2010mining} speeds up the search for the target itemsets by using the reversion of the original sequence and comparing the reversed sequence with the related itemsets. Furthermore, clustering analysis is applied to automatically set time partition values for the task of time-interval sequential pattern mining. A novel target-oriented sequential pattern mining approach was presented by Chand \textit{et al.} \cite{chand2012target}, which uses RFM (recency, frequency, and monetary) constraints. As a result, fewer database projections are done, and the space complexity is reduced. To remove some useless or irrelevant patterns in high utility sequential pattern mining \cite{zhang2021shelf,gan2021explainable}, the TUSQ algorithm \cite{zhang2021tusq} first introduced the concept of utility into target sequence queries. The algorithm does not focus on frequency like previous algorithms, but rather on utility. Recently, the TargetUM algorithm \cite{miao2021targeted} has been proposed to fill the gap and perform target-oriented mining in HUIM.

In general, the TargetUM algorithm provides an integrated approach for high-utility mining with a target query, which serves as the foundation for this research. However, there are no studies combining top-$k$ high-utility methods with target pattern queries. This paper introduces the problem of targeted utility mining with the concept of top-$k$ patterns to prevent the generation of large sets of HUIs and to accurately and quickly process target queries.

\section{Preliminaries}\label{sec:preliminaries}

%We leverage the principle of the normalized cut algorithm~\cite{shi2000normalized_cut} (NCut) to identify potential instances for 3D pseudo masks, lifted to the high-dimensional 3D scenario by using a geometric primitive basis, as discussed in Section~\ref{sec:oversegmentation}.
\NEW{We employ the principle of the NCut approach for pseudo-generation similarly to \cite{wang2023cut}, but lift it to support high-resolution 3D segmentation by operating on segment-level geometric primitives from self-supervised 2D and 3D features.}

NCut maximizes similarities within partitions and dissimilarities across partitions by minimizing the cost of a graph cut. This is formalized as: 
%
\begin{equation}
    NCut(A,B) = \frac{cut(A,B)}{assoc(A,V)} + \frac{cut(A,B)}{assoc(B,V)},
\end{equation}
where $A$ and $B$ are disjoint bipartitions from a full graph $V$, $cut$ measures the degree of dissimilarity computed as the total weight of edges that have been removed, and  $assoc$ represents the total connection within the partition.
Normalizing the cut cost function with the size of the partitions can solve the problem of single outlier node removal, which is one of the biggest difficulties of other graph cut algorithms~\cite{244673}.

While the minimum solution of this problem is intractable for practical applications, it can be rewritten as a generalized eigenvalue problem with adjacency matrix $W$ and degree matrix $D$, where $D(i,i) = \Sigma_jW(i,j)$:
%
\begin{equation} \label{eq:general_eigenval}
    (D-W)v = \lambda D v,
\end{equation}
%
Finding the second smallest eigenvalue $\lambda$  and its corresponding eigenvector $v$ is a close approximation for the minimized cost. 
From $v$, we obtain foreground separation by taking all node activations where the eigenvector components were larger than their mean. 
This method has been shown to be effective on intensity images, but combining it with deep features has demonstrated even stronger potential in the image domain \cite{wang2022tokencut,lis2022attentropy,wang2023cut}. 
While this multiple foreground objects could be directly predicted with a single pass by taking the eigenvectors in order, it was shown in \cite{wang2022tokencut} that a greedy iterative approach produces better results.




\section{The TMKU Algorithm} \label{sec:algorithm}

The targeted mining of top-$k$ HUIs algorithm, abbreviated as TMKU, is presented at length in this section. Figure \ref{img1} is the framework diagram of the proposed TMKU algorithm, showing the basic steps of the whole algorithm. TMKU uses a trie structure as storage method similar to the TargetUM algorithm, and then utilizes the idea of filtering to dynamically change $\eta$ (the utility threshold) to select the desired top-$k$ THUIs. The TMKU algorithm enables a wide range of applications by adjusting the target pattern and $k$ to query different targets and the number of result itemsets in a short period of time. After presenting these concepts, we will explain how search space pruning is conducted as well as the processes of the TMKU algorithm in detail.

\begin{figure}
	\centering
	\includegraphics[scale=0.35]{pics/framework.pdf} 
	\caption{The framework of TMKU}
	\label{img1} 
\end{figure}


\subsection{Constructing the TP-tree and query mechanism}

For data analysis, the database must be scanned to process the information and obtain patterns that meet the user's requirements. When using the target-pattern tree as a data structure, the algorithm uses nodes to represent itemsets, and the pattern tree can be updated incrementally by inserting new nodes.

MEIT \cite{fournier2013meit} is an early study about this problem. It uses a compact node-compression mechanism to store information about transactions and answer queries about target itemsets, but it is not designed for utility mining. Later, the TargetUM algorithm was proposed for HUIM. The algorithm uses a trie tree to store HUIs from the whole database, and this was shown to reduce memory consumption. Therefore, the algorithm proposed in this paper also adopts the trie structure, which can reduce the overhead time for queries by using the common prefix of strings to improve efficiency. Then, after identifying all HUIs containing the target item or itemset, a map structure (TopKMap) is built to store the discovered target itemsets. The algorithm takes the utility value of the $k$-th itemset as the initial threshold, and constantly updates the TopKMap w.r.t. the minimum threshold until the top-$k$ target HUIs are discovered.

\begin{definition}
	\rm (Target pattern tree). There are seven elements describing each node in this tree. For a node $n$ (\textit{name}), its parent node (\textit{parent}) is recorded. In addition, the transaction weighted utilization (\textit{twu}), \textit{sumIu} which records the utility of the itemsets represented by the current node, and \textit{sumRu} which records the remaining utility of the itemsets as the current node, are the three elements related to the utility. When $n$ is the last item of a HUI, one variable (\textit{isEnd}) is set to true. If not, \textit{isEnd} is set to false. The last element indicates the link to the next node (\textit{link}) containing the same item.
\end{definition}

\begin{definition}
	\rm (Item header table). Each item has its own item header table. There is also a pointer to the first linked node corresponding to each item in the TP-tree. This table is useful for quickly locating the position of the item nodes.	
\end{definition}

When the initial HUI \{$f$, $c$, $d$\} is found, the node $f$ is first added into this tree since it has the smallest \textit{TWU} value. Moreover, the element information of $f$ is recorded. Then, the algorithm creates the item header table of the $f$ node. The other nodes are then inserted in the same way. However, each time a node is inserted, the algorithm needs to verify whether a node with the same name already exists. If it already exists, the node is not created again. But the inserted itemset will share a common prefix with the already existing node item. When the second HUI \{$f$, $c$, $d$, $b$\} is inserted into the TP-tree, it shares the prefix ($f$, $c$, $d$) with the first HUI \{$f$, $c$, $d$\}. Then, node $b$ is created, and its information is recorded to update the item header table. Furthermore, the update of the item header table for item $b$ requires linking the new node. Subsequently, the last $b$ becomes the updated trailing node. Other HUIs that are stored in the TP-tree must also be inserted using the same process until the construction of the TP-tree is finished. The construction of the TP-tree is illustrated in Figure \ref{img2}.

\begin{figure}
	\centering
	\includegraphics[scale=0.35]{pics/tptree.pdf} 
	\caption{The construction of TP-tree}
	\label{img2} 
\end{figure}


\subsection{Targeted utility raising strategy}

As mentioned before, if only the value of $k$ is set by the user, the task of top-$k$ HUIM is to find out the top-$k$ HUIs. Since the utility threshold is not given by the user, effective utility raising strategies need to be proposed to efficiently find the top-$k$ HUIs. In this paper, the TMKU algorithm adopts two strategies (SUR and RIU) to achieve this goal.

\begin{strategy}
	\label{sur}
	\rm (SUR strategy). The SUR strategy \cite{zhang2021tkus} is a classic threshold raising strategy, used by the TMKU algorithm when storing the results in the TopKMap. The TopKMap can store $k$ items and their utility values dynamically. 
\end{strategy}

At first, the TopKMap is empty, and then when each THUI is found, the designed TMKU algorithm needs to consider whether to add it to the TopKMap or not. If the utility value of THUI is greater than $\eta$, it can be included in the TopKMap. However, when the number of itemsets stored in the TopKMap is greater than the $k$ value set by the user, some itemsets in the TopKMap need to be gradually deleted and updated. Those itemsets with low utility values will be replaced by itemsets with high utility values.

\begin{strategy}
	\label{riu}
	\rm (RIU strategy). The RIU raising strategy \cite{tseng2015efficient} is primarily based on the real utilities of items. If the number of THUIs is more than $k$ and the $k$-th highest utility value after sorting HUIs is still higher than $\eta$, the value of $\eta$ should be changed to the $k$-th highest value.
\end{strategy}

The RIU threshold raising strategy plays an integral role in the TMKU algorithm. The true utility of each item can be identified by scanning THUIs. To raise the utility threshold more quickly, TMKU uses the true utility of the first item as the reference value for $\eta$. Finally, combining with the SUR strategy, the correct top-$k$ THUIs can be quickly mined.

\subsection{Pruning strategies}

The TP-tree and item header table have been presented, which speeds up processing. Furthermore, some utility threshold-raising strategies are also adopted into the TMKU algorithm, which were listed in the previous subsection. Some pruning strategies used in TMKU are introduced in this subsection. They allow the algorithm to considerably reduce the search space and time.

\begin{strategy}
	\label{lab:one}
	\rm If the sum of utilities of $X_i$ $<$ $\xi$, then $X_i$ is not a THUI.	
\end{strategy}

It was shown in prior work that if \textit{TWU($X_i$)} < $\xi$, $u(X_i)$ is also less than the threshold since $u(X_i)$ must be smaller than \textit{TWU($X_i$)}. Therefore, it is not necessary to check any itemsets containing $X_i$. The utility and remaining utility of each item are recorded in its utility-list. If the sum of the utilities of $X_i$ (\textit{sumIu}) is smaller than $\xi$ defined by the user, then $X_i$ is not the HUI that we are looking for. Conversely, it is a HUI. Moreover, the utility-list also records the sum of the remaining utilities called \textit{sumRu}, which can be cleverly used to prune these utility values to improve search efficiency.

\begin{strategy}
	\label{lab:two}
	\rm In the TMKU algorithm, let $X^\prime$ and $\xi$ be the candidate and current minimum utility value, respectively. Suppose that the sum of \textit{sumIu} and \textit{sumRu} of $X^\prime$ is smaller than $\xi$. Then the supersets and extended itemsets of $X^\prime$ do not need to be checked.
\end{strategy}

In other words, the sum of \textit{sumIu} and \textit{sumRu} respects the downward closure property. Therefore, no extension representing $X^\prime$ can be the required top-$k$ THUI. This pruning strategy can prevent exploring several itemsets. It is worth noting that if the sum of \textit{sumIu} and \textit{sumRu} of $X^\prime$ is greater than $\xi$, then the extension of $X^\prime$ is a potential HUI and a further check is needed.

In addition, we can find that more and more branches may be generated during the construction of the TP-tree. However, the task is to discover all the top-$k$ HUIs that contain a target itemset, so not all high-utility itemsets must be searched. The TMKU algorithm needs to determine if the target itemset is included before inserting a node to reduce the running time. Therefore, TMKU uses the item header table to cut off the useless branches in advance and correctly and quickly identify the high-utility itemsets containing the target pattern. Then, we can use the item header table to quickly find the nodes in the target pattern tree.

\begin{strategy}
	\label{lab:three}
	\rm When $T^\prime$ is set as the target pattern, the TMKU algorithm sorts these items of the target itemset in \textit{TWU} ascending order. At this point, the item $i$ with the largest \textit{TWU} value is selected, and the position of $i$ in the TP-tree and its branches is determined using the item header table. A bottom-up approach is used to query whether this HUI contains the target pattern. When determining whether item $j$ in the target pattern is in the branch, we can compare the \textit{TWU} of the current node with the value of $\textit{TWU}(j)$. If the current node's \textit{TWU} is smaller than $\textit{TWU}(j)$, the branch does not contain the target pattern and can be discarded directly. If the current node's \textit{TWU} is greater than $\textit{TWU}(j)$, further exploration is required. If equal, it is also necessary to check whether the current node's node name is $j$ to avoid replacing the same \textit{TWU} value items.
\end{strategy}

For example, supposing that \{$f$, $c$, $d$, $b$\} is a HUI, and \{$a$, $b$\} is denoted as $T^\prime$. First, we find that item $b$ whose \textit{TWU} is the highest in the target pattern. Then, we need to continue checking whether $a$ is in the HUI, and then compare the \textit{TWU} values of other items in the HUI with the \textit{TWU} of $a$ in turn. It is found that the \textit{TWU} of both $d$ and $c$ is greater than $a$, but the \textit{TWU} of $f$ is less than $a$. According to the above analysis, it is known that \{$a$, $b$\} is not in the HUI and the search is stopped.

\subsection{Proposed algorithm}

The TMKU task will be summarized as follows, based on the above introduction. To discover those itemsets containing the target itemset, Algorithm \ref{algo:THUI} focuses on accomplishing this task by constructing the TP-tree and the item header table. Algorithm \ref{algo:TMKU} introduces how to search the TP-tree to find the THUIs followed by mining the top-$k$ itemsets, and through some utility threshold raising strategies, users can get the desired results as soon as possible.

%%%%%%%%%%%%%%%%%   Construct TP-tree procedure   %%%%%%%%%%%%%%%%%%%
\begin{algorithm}[h]
	\small
	\caption{The construction procedure}
	\label{algo:THUI}
	\LinesNumbered
		\KwIn{$X^\prime$: the prefix of HUI; \textit{IUs}: the utility-list of $X^\prime$; \textit{RUs}: the remaining utility-list of $X^\prime$; \textit{TUs}: the \textit{TWU} list of $X^\prime$; $x$: the last item of HUI; $\alpha$: the utility of $x$; $\beta$: the remaining utility of $x$; $\theta$: the \textit{TWU} of $x$; \textit{mapItemNode} store the header node of each item-header table; \textit{mapItemLastNode}: store the tail nodes of each item-header table; $\xi$: the target utility threshold.}
		\KwOut{\textit{TP-tree}; item header table.}
		
        initialize \textit{TP-tree}, \textit{currentNode} = \textit{null}, \textit{parentNode} = \textit{null}, \textit{listNodes} = \textit{root.childs}\;
        insert a HUI into the TP-tree\;
        \For{\rm each item $e$ $\in$ $X^\prime$}{
           search the position of $e$\;
           \If{\textit{currentNode} == \textit{null}}{
            create a new node in the TP-tree and construct item header table of $e$;
       }
       update \textit{parentNode}, \textit{listNodes}\;
      }
      \textit{currentNode} = $x$\;
      \eIf{\textit{currentNode} != \textit{null}}{
      {update} \textit{currentNode.sumIu}, \textit{currentNode.sumRu}, \textit{currentNode.TWU}\;
  }{ 
    construct new item header table of $x$\;
}
\textit{currentNode.isEnd} = \textit{true};
\end{algorithm}

%%%%%%%%%%%%%%%%%%   TMKU procedure  %%%%%%%%%%%%%%%%%%%

\begin{algorithm}[h]
	\small
	\caption{The discovery procedure}
    \label{algo:TMKU}
	\LinesNumbered
	\KwIn{$k$: the number of THUIs; $T^\prime$: the itemset obtain target; $\xi$: the target utility threshold; $\eta$: the minimum utility threshold.}
	\KwOut{top-$k$ THUIs.}  
	
	\textit{posToMatch} = |$T^\prime$|, \textit{node} = the target item of last item\;
	\While{\textit{node} != \textit{null}}{
	    \If{\rm \textit{node.sumIu} + \textit{node.sumRu} $\geq$ $\xi$}{ 
           \textit{node.name} = the current \textit{HUI} $X$\;
           \textit{posToMatch} decrease 1, {update} \textit{currentNode} = \textit{node.parent}\;
           
           \While{\textit{currentNode} != \textit{null}}{
           	  \If{\textit{posToMatch} $\geq$ 0}{
           	      a new item $y$ = $T^\prime$[\textit{posToMatch}]\;
           	      \If{\textit{currentNode.TWU} < \textit{y.TWU}}{
           	      {break}\;
           	    }
           	      \If{\textit{currentNode.TWU} == \textit{y.TWU} \textit{AND}  \textit{currentNode} == $y$}{
           	      	
                  \textit{posToMatch} decrease 1\;
              }
             }
              {update} $x$ and currentNode\;
          }
           	\If{\textit{posToMatch} == -1}{
           		
           	\If{\textit{node.sumRu} $\geq$ $\xi$ \textit{AND} \textit{currentNode.isEnd} = $true$}{
              a THUI has been discovered\;
           }
           recursively explore all suffix nodes of $X$\;
         }
		}
   }
	
   initialize TopKMap = $\emptyset$\;
   \For{\rm  each entry itemset $t$ in THUIs}{
    	\eIf{\textit{t.utility} $\geq$ $\eta$}{
            TopKMap.add(t)\;}{ 
            skip $t$ from TargetUM\;}
        \If{\rm the number of \rm{TopKMap} $>$ $k$}{
          {update} $\eta$ as the $k$-th highest utility in TopKMap\;
          {update} TopKMap and keep $k$ target itemsets with the highest utilities in TopKMap\;
	  }
   	    }
    \textbf{return} top-$k$ THUIs
\end{algorithm}


\textbf{(1) The construction of TMKU}: this procedure can be divided into two parts: the construction of the TP-tree and the processing of the item header table. The construction of a TP-tree is the basis of the TMKU algorithm. There are several indispensable parameters need to be introduced, including its last item $x$, the list of three utilities (\textit{IUs}, \textit{Rus}, \textit{TUs}), and the three utility values of $x$. The \textit{mapItemNode} is available to store the header node of each item header table. During the construction of the TP-tree, the node used to store the last item in each itemset is called the tail node, and the tail node is continuously updated. Besides, the \textit{mapItemLastNode} is required to store the tail nodes of each item header table. First, the algorithm initializes the TP-tree, i.e., sets the initial values for the several parameters (line 1). Based on the HUIs obtained by applying HUI-Miner, once a new HUI is found, we should consider inserting it into the TP-tree (line 2). For each item $n$ in $X^\prime$ (line 3), the position of $n$ in this TP-tree can be found easily (line 4). When there are no $n$ nodes in the TP-tree, the TMKU algorithm creates a new node and stores its information in the item header table (lines 5--6). After that, the values of nodes \textit{parentNode} and \textit{listNodes} are updated (line 8). When all the items of $X^\prime$ have been processed, the TP-tree of $X^\prime$ is completely built (lines 3--8). Then, while $x$ previously existed (found by the binary search method) (line 10), the algorithm will update the information value of the current node (lines 11 to 12). And if it does not exist, an item header table of $x$ is created (line 14). Finally, the value of the \textit{currentNode.isEnd} variable is set to true (line 16).


\textbf{(2) The discovery of TMKU}: The procedure for processing the target query is a depth-first search. First, the procedure determines whether the received prefix appears. Therefore, the node is denoted the last item of $T^\prime$, and then the \textit{posToMatch} parameter is initialized to record the current two items (line 1). Then, each item of $T^\prime$ is traversed. If the sum of \textit{sumLu} and \textit{sumRu} is not less than $\xi$, according to Strategy \ref{lab:two}, $X$ may be a THUI, and further comparison with other items can be performed (lines 2--25). Moreover, according to Strategy \ref{lab:three}, if the \textit{TWU(\textit{currentNode})} is smaller than the current item of $T^\prime$, it can be discarded directly, and the comparison is aborted (lines 9--10). When the value of \textit{posToMatch} decreases to -1, one of the target itemsets is found (lines 19--20). At this point, the algorithm uses a recursive method to output this THUI. First, all suffix nodes of $X$ are explored, and from each node, the extension nodes are obtained (line 22). Then, the THUIs are stored by filtering out the high-utility itemsets that do not contain the target itemset. Finally, if the remaining suffix nodes meet the condition of Strategy \ref{lab:two}, they need to be explored.

After that, the algorithm obtains the target high-utility itemsets. Then, the task of the algorithm is to quickly mine the top-$k$ THUI. First, TopKMap is initialized to the empty set and the utility value of the first THUI is used to initialize $\eta$ (line 26). Then, the algorithm iterates through all the itemsets in the THUIs and if the utility value of the itemset is higher than $\eta$, the itemset is added to the TopKMap (lines 27--29). Otherwise, the THUI is removed (line 31). When the size of the TopKMap is over $k$ (line 33), $\eta$ is updated to the $k$-th highest utility in the TopKMap (line 34). Moreover, the algorithm updates the TopKMap and keeps the $k$ target itemsets with the highest utilities in the TopKMap (line 35). Finally, all the top-$k$ THUIs can be explored and output successfully (line 38).

\section{Experiments}
\label{sec:exp}
% logic:
% 1, Experiment Settings 

% 2, Benchmark results 
    % 1, VIP-Seg
    % 2, VSPW
    % 3, KITTI-STEP

% 3, Ablation studies and analysis. 
    % 1, improvements on baseline 
    % 2, design choices of temporal contarstive loss 
    % 3, design choices of label assigin stragety
    % 4, Effect of tube frames choices for CS loss 
    % 5, Effect of large window size / overlap inference. 
    % 6, Comparison with the different tracking choices. 
    % 7, increased GFLops/Parameters analysis. 
    % 8. FPS/Window Cruves.

% 4, visualization results. 
    % 1, comparison with strong baseline. 
    % 2, attention mask arcoss differnt tube. 
    

% Due to the unavailability of the test set, we report the results on the \textit{validation set}. 

% The former mainly focuses on mask proposal level as PQ~\cite{kirillov2019panoptic} with different window sizes, while the latter emphasizes pixel-level segmentation and tracking without any thresholds.
% KITTI-STEP has 21 and 29 sequences for training and testing, respectively. The training sequences are split into a training set (12 sequences) and a validation set (9 sequences).

\subsection{Experimental Settings}
\noindent
\textbf{Dataset.} We conduct experiments on five video datasets: VIPSeg~\cite{miao2022large}, VSPW~\cite{miao2021vspw}, KITTI-STEP~\cite{STEP}, and YouTube-VIS-19/21~\cite{vis_dataset}. We mainly conduct experiments on VIPSeg due to its scene diversity and long-length clips. The training, validation, and test sets of VIPSeg contain 2,806/343/387 videos with 66,767/8,255/9,728 frames, respectively. Although VSPW and VIPSeg share the same video clips, the training details are different since they are different tasks. Please refer to the \textit{supplementary material} for other datasets.


\noindent
\textbf{Evaluation Metrics.} For the VPS task, we adopt two metrics: $VPQ$~\cite{kim2020vps} and $STQ$~\cite{STEP}. The metric $STQ$ contains geometric mean of two items: Segmentation Quality ($SQ$) and Association Quality ($AQ$), where $ STQ = (SQ \times AQ)^{\frac{1}{2}}$. The former evaluates the pixel-level tracking, while the latter evaluates the pixel-level segmentation results in a video clip. For the VSS task, the Mean Intersection over Union (\textit{mIoU}) and mean Video Consistency ($mVC$)~\cite{miao2021vspw} are used for reference. For the VIS task, \textit{mAP} is adopted.


\noindent
\textbf{Implementation Details and Baselines.} We implement our models in PyTorch~\cite{pytorch_paper} with the MMDetection toolbox~\cite{chen2019mmdetection}. We use the distributed training framework with 16 V100 GPUs. Each mini-batch has one image per GPU. Following previous work, we use the image baseline pre-trained on COCO dataset~\cite{coco_dataset}. ResNet~\cite{resnet}, STDC~\cite{STDCNet}, and Swin Transformer~\cite{liu2021swin} are adopted as the backbone networks, which are pre-trained on ImageNet, and the remaining layers adopt the Xavier initialization~\cite{xavier_init}. 
For the detailed settings of other datasets, pretraining, and fine-tuning, please refer to the \textit{supplementary material}. To further verify the effectiveness of our approach, we build a stronger baseline by unifying Video K-Net with Mask2Former, where we replace the image encoder with Mask2Former. We term it Video K-Net+. We denote the extended Mask2Former-VIS for VPS as Mask2Former-VIS+.


%%%%%%%% VIP-SEG %%%%%%%%%%%
\begin{table}[!t]
	\centering
	\caption{\small \textbf{Results on VIPSeg-VPS~\cite{miao2022large} validation dataset.} We report VPQ and STQ for reference. Following Miao~\etal~\cite{miao2022large}, we report VPQ scores at different window sizes (1, 2, 4, 6). We report the results obtained from either an efficient or a strong backbone for comparison.}
	\label{tab:vipseg_results}
  \scalebox{0.65}{
    \begin{tabular}{ r|c|cccccc}
    \toprule[0.15em]
     Method& backbone & $VPQ^{1}$ & $VPQ^{2}$ & $VPQ^{4}$ & $VPQ^{6}$ & VPQ & STQ \\
    \toprule[0.15em]
    VIP-DeepLab~\cite{ViPDeepLab} & ResNet50 & 18.4 & 16.9 & 14.8 & 13.7 & 16.0 & 22.0 \\
    VPSNet~\cite{kim2020vps} & ResNet50 & 19.9 & 18.1 & 15.8 & 14.5 & 17.0 & 20.8 \\
    SiamTrack~\cite{woo2021learning_associate_vps} & ResNet50 & 20.0 & 18.3 & 16.0 & 14.7 & 17.2 & 21.1 \\
    Clip-PanoFCN~\cite{miao2022large} & ResNet50 & 24.3 & 23.5 & 22.4 & 21.6 & 22.9 & 31.5 \\
    Video K-Net~\cite{li2022videoknet} & ResNet50 & 29.5 & 26.5 & 24.5 & 23.7 & 26.1 & 33.1 \\
    Video K-Net+~\cite{cheng2021mask2former,li2022videoknet} & ResNet50 & 32.1 & 30.5 & 28.5 & 26.7 & 29.1 & 36.6  \\
    Video K-Net~\cite{li2022videoknet} & Swin-base & 43.3 & 40.5 & 38.3 & 37.2 & 39.8 & 46.3 \\
    \hline
    Tube-Link & STDCv1 & 32.1 & 31.3 & 30.1 & 29.1 & 30.6 & 32.0 \\
    Tube-Link & STDCv2 & 33.2  & 31.8 & 30.6 & 29.6  &  31.4 & 32.8 \\
    \hline
    Tube-Link & ResNet50 & 41.2 & 39.5  & 38.0 & 37.0 &  39.2 & 39.5 \\
    Tube-Link & Swin-base & 54.5 & 51.4 & 48.6 & 47.1 & 50.4 & 49.4 \\
    % Tube-Link & Swin-large &  \lxt{wait results} \\
    \bottomrule[0.2em]
    \end{tabular}
}
\end{table}


%%%%%% VIS-Youtube %%%%%%%%%
\begin{table}[t]
  \centering
   \caption{\small \textbf{Results on the YouTube-VIS datasets.} We report the mAP metric. \textdagger~adopt COCO video pseudo labels. Axial means using the extra Axial Attention~\cite{axialDeeplab}. Our method does not apply these techniques for simplicity.}
  \label{tab:ytvis}
  \scalebox{0.68}{
  \begin{tabular}{l c | c  | c }
    \toprule[0.2em]
    Method & Backbone  & YTVIS-2019 & YTVIS-2021 \\
    \toprule[0.2em]
VISTR~\cite{VIS_TR} & ResNet50 & 36.2 & -  \\
TubeFormer~\cite{kim2022tubeformer} & ResNet50 + Aixal & 47.5  & 41.2  \\
IFC~\cite{hwang2021video} & ResNet50 & 42.8 & 36.6 \\
SeqFormer~\cite{seqformer} & ResNet50 & 47.4 & 40.5  \\
Mask2Former-VIS~\cite{cheng2021mask2former_vis}& ResNet50 & 46.4 & 40.6 \\
IDOL~\cite{IDOL} & ResNet50 & 46.4 & 43.9\\
IDOL~\cite{IDOL} \textdagger & ResNet50 & 49.5 & -\\
VITA~\cite{heo2022vita} \textdagger & ResNet50 & 49.8 & 45.7  \\
Min-VIS~\cite{huang2022minvis} &ResNet50& 47.4 & 44.2 \\
% GenVIS~\cite{heo2022generalized} & ResNet50 & 51.3 & 46.3 \\
\hline
Tube-Link & ResNet50 & 52.8 & 47.9  \\% & - \\
\hline
SeqFormer~\cite{seqformer} & Swin-large  & 59.3 & 51.8 \\% & - \\
Mask2Former-VIS~\cite{cheng2021mask2former_vis} & Swin-large &  60.4 & 52.6 \\
IDOL~\cite{IDOL}  & Swin-large  & 61.5 & 56.1 \\ %& 42.6\\
IDOL~\cite{IDOL}  & Swin-large \textdagger  & 64.3 & -\\
VITA~\cite{heo2022vita} \textdagger & Swin-large & 63.0 & 57.5 \\ 
Min-VIS~\cite{huang2022minvis} & Swin-large & 61.6 & 55.3 \\
\hline
Tube-Link & Swin-large  & 64.6 & 58.4  \\
    \bottomrule[0.2em]
  \end{tabular}
}
\end{table}



%%%%%% VSPW and VIP-Seg VSS%%%%%%%%%
\begin{table}[t]
  \centering
    \caption{\small \textbf{Results on VSPW-VSS validation set}. $mVC_{c}$ means that a clip with $c$ frames is used.}
    \label{tab:vspw}
  \scalebox{0.68}{
  \begin{tabular}{l c c c c c }
    \toprule[0.2em]
    \textbf{VPSW} & Backbone & mIoU & $mVC_{8}$ &$mVC_{16}$  \\
    \toprule[0.2em]
    DeepLabv3+~\cite{deeplabv3plus} & ResNet101 & 35.7 & 83.5 & 78.4 \\
    TCB(PSPNet)~\cite{miao2021vspw,zhao2017pyramid} & ResNet101 & 37.5 & 86.9 & 82.1  \\
    Video K-Net (Deeplabv3+)~\cite{li2022videoknet,deeplabv3plus} & ResNet101  & 37.9 & 87.0 & 82.1 \\
    Video K-Net (PSPNet)~\cite{li2022videoknet,zhao2017pyramid} & ResNet101  & 38.0 & 87.2  & 82.3 \\
    MRCFA~\cite{sun2022mining} & MiT-B5 & 49.9 & 90.9  &  87.4  \\
    CFFM~\cite{sun2022vss} & MiT-B5 & 49.3 & 90.8 & 87.1 \\
    TubeFormer~\cite{kim2022tubeformer} & Axial-ResNet50x64  &  63.2 &  92.1 & 88.0 \\
    \hline
    Tube-Link & ResNet50 & 42.3 & 86.8 & 83.2 \\
    Tube-Link & Swin-large & 59.7 & 90.3 & 88.4 \\
    \bottomrule[0.2em]
  \end{tabular}
  }

\end{table}


\begin{table}[t]
  \centering
    \caption{\small \textbf{Results on VIP-Seg-VSS validation set}. $mVC_{c}$ means that a clip with $c$ frames is used.}
    \label{tab:vipseg_vss}
  \scalebox{0.68}{
  \begin{tabular}{l c c c c c }
    \toprule[0.2em]
    \textbf{VPSW} & Backbone & mIoU & $mVC_{8}$ &$mVC_{16}$  \\
    \toprule[0.2em]
    Video K-Net (Deeplabv3+)~\cite{li2022videoknet,deeplabv3plus} & ResNet101  & 38.3 & 88.0 & 83.1 \\
    Video K-Net (PSPNet)~\cite{li2022videoknet,zhao2017pyramid} & ResNet101  & 39.0 & 88.2  & 84.2 \\
    Mask2Former~\cite{cheng2021mask2former} &  ResNet50 & 38.4 & 87.5 & 82.5 \\
    Video K-Net+~\cite{cheng2021mask2former,li2022videoknet} &  Swin-base & 57.2 & 90.1 & 87.8  \\
    \hline
    Tube-Link & ResNet50 & 43.4 & 89.2 & 85.4 \\
    Tube-Link & Swin-base & 62.3 & 91.4 & 89.3 \\
    Tube-Link & Swin-large & 64.9 & 92.4 & 89.9 \\
    \bottomrule[0.2em]
  \end{tabular}
  }

\end{table}


\subsection{Benchmark Results}


%%%%%% KITTI-STEP %%%%%%
\begin{table}[t]
  \centering
   \caption{\small \textbf{Results on the KITTI val set.} OF refers to an optical flow network~\cite{teed2020raft}.}
  \label{tab:kitti_step}
  \scalebox{0.68}{
  \begin{tabular}{l c c || c c c c }
    \toprule[0.2em]
    \textbf{KITTI-STEP} & Backbone & OF & STQ & AQ & SQ & VPQ \\
    \toprule[0.2em]
    P + Mask Propagation & ResNet50 & \checkmark & 0.67 & 0.63 & 0.71 & 0.44 \\
    Motion-Deeplab~\cite{STEP}& ResNet50 &  & 0.58 & 0.51 & 0.67 & 0.40  \\
    VPSNet~\cite{kim2020vps}& ResNet50  & \checkmark & 0.56 & 0.52 & 0.61 & {0.43}  \\
    TubeFormer-DeepLab~\cite{kim2022tubeformer} & ResNet-50 + Axial &  & 0.70 & 0.64 &  0.76 & 0.51 \\
    Video K-Net~\cite{li2022videoknet} & ResNet50 &  & 0.71 & 0.70  & 0.71  &  0.46 \\
    Video K-Net~\cite{li2022videoknet} & Swin-base &  & 0.73 & 0.72 & 0.73 & 0.53 \\
    \hline
    Tube-Link & ResNet50 &  & 0.68 & 0.67 & 0.69 & 0.51 \\
    Tube-Link & Swin-base &  & 0.72 & 0.69 & 0.74 & 0.56 \\
    \bottomrule[0.2em]
  \end{tabular}
  }
  \vspace{-4mm}
\end{table}

% \lxt{will be changed by test set Figure Results Further. This figure will be merged into it as subfigure.}
\begin{figure}[t]
  \centering
   \includegraphics[width=0.80\linewidth]{./figs/teaser_trade_off.pdf}
   \caption{\small Tube-Link also achieves the best accuracy and speed trade-off on VIP-Seg dataset. FPS is measured on RTX GPU.}
   \label{fig:curve_trade_off_vipseg}
\end{figure}

\noindent
\textbf{[VPS] Results on VIPSeg.} 
We present the results of our Tube-Link method compared to previous works on the VIPSeg dataset in Tab.~\ref{tab:vipseg_results}. Our approach outperforms Video K-Net\cite{li2022videoknet} (under the same backbone) with 12\%-15\% VPQ and 7\%-10\% STQ improvements, respectively. Notably, our method with Swin-base~\cite{liu2021swin} backbone achieves new state-of-the-art results. 
%
We also evaluate our method using a lightweight backbone~\cite{STDCNet} for more efficient inference on video clips, and it achieves even better results than all previous methods with a larger ResNet50 backbone. 
%
These results demonstrate the effectiveness of our approach in exploiting temporal information.  Benefiting from the joint inference of subclips, our method achieves a much faster inference speed, as shown in Fig.~\ref{fig:curve_trade_off_vipseg}. 



\begin{table*}[h!]
    \footnotesize
	\centering
	\caption{\small \textbf{Ablation studies and comparative analysis on VIPSeg validation set with the ResNet50 backbone.} 
	}
    \subfloat[Ablation Study on Each Component.]{
    \label{tab:ablation_a}
	    \begin{tabularx}{0.43\textwidth}{c c c c c} 
		        				\toprule[0.15em]
    	baseline  & TCL & CTL & $\mathrm{VPQ_{th}}$ & VPQ \\
        \toprule[0.15em]
            Mask2Former-VIS+ (F) & - & - & 29.4 & 32.4 \\
            \hline
            Mask2Former-VIS+ (T) & - & - & 31.0 & 34.5\\
             & \checkmark & - & 34.6  & 36.8  \\  
          \rowcolor{gray!15}  & \checkmark & \checkmark & 35.1 & 37.5 \\  
        \bottomrule[0.1em]
	    \end{tabularx}
    } \hfill
    \subfloat[Design Choices of TCL.]{
    \label{tab:ablation_b}
		\begin{tabularx}{0.28\textwidth}{c c c} 
			\toprule[0.15em]
			Method & VPQ & STQ \\
			\midrule[0.15em]
            Dense Query~\cite{qdtrack} & 30.2  & 30.1  \\
            Sparse Query~\cite{li2022videoknet} & 34.5  & 35.1 \\
            \rowcolor{gray!15} Global Query(Ours) &  37.5  & 36.5 \\
			\bottomrule[0.1em]
		\end{tabularx}
    } \hfill
    \subfloat[Association Target Assign.]{
    \label{tab:ablation_c}
		\begin{tabularx}{0.24\textwidth}{c c c} 
			\toprule[0.15em]
			Method & VPQ & STQ  \\
			\midrule[0.15em]
			All-Masks~\cite{qdtrack} & 30.1 & 29.2 \\
			GT-Mask~\cite{li2022videoknet} & 35.6 & 35.9 \\
			\rowcolor{gray!15} Tube-Mask & 37.5 & 36.5 \\
			\bottomrule[0.1em]
		\end{tabularx}
    } \hfill
    \vspace{2mm}
    \subfloat[Input Sub-clip Size with Tube Window Size of 2 as Input.]{
     \label{tab:ablation_d}
	    \begin{tabularx}{0.30\textwidth}{c c c c} 
		        				\toprule[0.15em]
    		 Clip Size & STQ & VPQ & $\mathrm{VPQ_{th}}$  \\
    		\toprule[0.15em]
    	    T=1 & 34.5 & 35.6 & 30.2 \\
    	    \rowcolor{gray!15} T=2 & 36.5 & 37.5 & 35.1 \\
    	    T=2(ovl) & 35.9 & 37.3 & 35.0 \\
    	    T=3 &  36.4 & 37.0 & 35.3 \\
        	\bottomrule[0.1em]
	    \end{tabularx}
    } \hfill
    \subfloat[Tube-Window for Inference with Input Sub-clip Size 2 for Training.]{
     \label{tab:ablation_e}
	    \begin{tabularx}{0.30\textwidth}{c  c c c} 
		        				\toprule[0.15em]
    		 Window Size & STQ & VPQ  & $\mathrm{VPQ_{th}}$ \\
    		\toprule[0.15em]
    	    W=2 &  36.5 & 37.5 & 35.1 \\
    	    W=4 &  39.2 & 39.0 & 38.2 \\
    	   \rowcolor{gray!15} W=6 &  39.5 & 39.2 & 38.9 \\
    	    W=8 &  38.3 & 38.5 & 37.3 \\
        	\bottomrule[0.1em]
	    \end{tabularx}
    } \hfill
    \subfloat[Tracking Choices with the Default Setting of Tab.(d). ]{
     \label{tab:ablation_f}
	    \begin{tabularx}{0.35\textwidth}{c c c c} 
		        				\toprule[0.15em]
    		 Settings  &  STQ & VPQ & $\mathrm{VPQ_{th}}$ \\
    		 \toprule[0.15em]
    		  Extra Tracker~\cite{wangUnitrack,deepsort}& 33.9 & 36.6 & 34.1 \\
    		  RoI Features~\cite{qdtrack} & 34.5 & 35.9 & 34.5 \\
    		  Query Embedding~\cite{li2022videoknet}  & 33.1  & 36.0  & 33.0 \\
    	     \rowcolor{gray!15} Our Tube embedding & 36.5 & 37.5 & 35.1\\
        	\bottomrule[0.1em]
	    \end{tabularx}
    } \hfill
\end{table*}


\noindent
\textbf{[VIS] Results on YouTube-VIS-2019/2021.} In Tab.~\ref{tab:ytvis}, we compare our method with state-of-the-art VIS methods on the YouTube-VIS 2019 and 2021 datasets. Our method achieves a 3.0\% and 2.2\% mAP gain over VITA~\cite{heo2022vita} when using the ResNet50 backbone. Furthermore, compared with the Mask2Former-VIS baseline~\cite{cheng2021mask2former_vis}, our method achieves 4-5\% mAP gains on the two datasets with different backbones. Our method also outperforms the previous near-online method TubeFormer~\cite{kim2022tubeformer} by 5-6\% in terms of mAP on the two VIS datasets.


\noindent
\textbf{[VSS] Results on VSPW and VIP-Seg.} We further conduct experiments on VSPW dataset~\cite{miao2021vspw} for VSS to demonstrate the generalization of Tube-Link. As shown in Tab.~\ref{tab:vspw}, our method achieves over 4\% mIoU improvement compared to the Mask2Former baseline. Under the same ResNet101 backbone, our method achieves the best results. Using the Swin base backbone, our method achieves about 3.7\% mIoU gains over Video K-Net+ with consistent improvements on $mVC$. Our method with a lightweight backbone achieves comparable results to DeepLabv3+ with ResNet101, but with about four times faster inference speed (shown in Fig.~\ref{fig:curve}). Without using any additional techniques, our method also outperforms recent methods specifically designed for VSS~\cite{sun2022vss,sun2022mining}. In Tab.~\ref{tab:vipseg_vss}, we also compare the video semantic segmentation methods in recent VIPSeg datasets with higher-resolution images. Compared with previous state-of-the-art methods, our approaches also achieve state-of-the-art results.

% Moreover, compared with the previous state-of-the-art Tubeformer~\cite{kim2022tubeformer}, our method achieves a better 1.7\% mIoU.


\noindent
\textbf{[VPS] Results on KITTI STEP.} 
We further validate our method on KITTI STEP~\cite{STEP} and report the results in Tab.~\ref{tab:kitti_step}. Our method achieves 0.51 VPQ with the ResNet50 backbone, setting a new state-of-the-art result \textit{without} using temporal attention or optical flow warping. When using a strong Swin-base~\cite{liu2021swin} backbone, our method still achieves better results than Video K-Net~\cite{li2022videoknet} by 3\% VPQ and comparable results on STQ. It is worth noting that one can further improve the performance of Tube-Link by employing a better tracker design.

\subsection{Ablation Study and Visual Analysis}
\label{sec:ablation}
% 1, improvements on baseline 
% 2, design choices of temporal contarstive loss 
% 3, design choices of label assigin stragety
% 4, Effect of tube frames choices for CS loss 
% 5, Effect of large window size / overlap inference. 
% 6, Comparison with the different tracking choices. 
% 7, increased GFLops/Parameters analysis. 
% 8. FPS/Window Cruves.

% In this section, we present some \textit{\textbf{key}} ablations on component design and analysis using VIPSeg dataset with ResNet50 backbone. 
%The default setting used in our model is indicated in gray.
%More results are provided in the supplementary material. 

% \cavan{image part? or do you mean feature extractor or encoder}
% \cite{li2022videoknet} as the baseline by replacing its encoder with Mask2Former~\cite{cheng2021mask2former}
\noindent
\textbf{Improvements over Strong VPS Baseline.} 
In Tab.~\ref{tab:ablation_a}, we demonstrate the effectiveness of each component proposed in Sec.~\ref{sec:tb_framework}. 
The first row shows the results of the frame matching baseline. After adopting the tube matching, we obtain a gain of 1.6\% $\mathrm{VPQ_{th}}$ and 2.1\% on VPQ, even without any specific tracking design, which results in the same observation as shown in Tab.~\ref{tab:toy_exp}. Thus, we use Mask2Former-VIS+ (T, T=2) as our baseline by default, which achieves a strong starting point of 34.5 VPQ. $\mathrm{VPQ_{th}}$ refers to the VPQ for the thing class. This result shows the effectiveness of the na\"{i}ve framework. The addition of TCL further boosts performance, with a gain of 3.5\% on $\mathrm{VPQ_{th}}$ and 1.7\% on VPQ. Furthermore, adding CTL, which makes the association more consistent, improves $\mathrm{VPQ_{th}}$ by 1.5\%.


\noindent
\textbf{Ablation on Temporal Contrastive Loss.} We also compare our TCL design with previous works that use dense queries~\cite{qdtrack} or sparse queries~\cite{li2022videoknet} for matching. Both settings use only one frame, while our subclip size is two. As shown in Tab.~\ref{tab:ablation_b}, our method achieves the best results since tube matching encodes more temporal information. In particular, we observe 3.0\% VPQ improvements compared to the strong Video K-Net baseline.


\begin{figure}[t!]
	\centering
	\includegraphics[width=1.0\linewidth]{./figs/tube_link_vis_results_1st.pdf}
	\caption{\small Comparison results on VIP-Seg and YuoTube-VIS. Our method achieves consistent segmentation (shown in orange boxes) and better tracking results (shown in red boxes).}
	\label{fig:visulize}
\end{figure}

% \gl{R-50 and R50 should be consistent. The same as R-101 and R101.}
\begin{figure}[t]
  \centering
   \includegraphics[width=1.\linewidth]{./figs/both.pdf}
   \caption{\small Efficiency Analysis of Tube-Link. Left: Segmentation results (mIoU) of VSPW with different subclip sizes. Right: Inference speed (FPS) with different subclip sizes.}
   \label{fig:curve}
\end{figure}



\noindent
\textbf{Ablation on Association Target Assignment.} 
In Table \ref{tab:ablation_c}, we show the results of the ablation study on building association targets. We find that using a tube-level mask achieves the best results. Using the mask from one of the input subclips leads to inferior results. This is because the ground truth masks of a single frame are not aligned with the input global queries, where the global queries are learned from multiple frames using Equation \eqref{equ:sp_attention}.

\noindent
\textbf{Effect of Sub-clip Size for Training.} 
In Tab.~\ref{tab:ablation_d}, we investigate the impact of subclip size on training. Tube-Link becomes an online method when the subclip size is 1. As shown in the table, enlarging the subclip size improves the performance. We also examine overlapping during sampling, denoted as ovl, where two input subclips overlap at one frame. As shown in Tab.~\ref{tab:ablation_d}, enlarging the subclip size to 2 achieves significant improvement. However, we find that either frame overlapping or using a larger subclip size ($T=3$) does not bring extra gains. Adding more frames does not benefit temporal association learning, since most instances are similar within a subclip. Moreover, using more frames is not memory-friendly during training. Thus, the subclip size is set to 2. We can enlarge the size for inference, as shown in Tab.~\ref{tab:ablation_e}.
  
% \cavan{to what value and why? During training?}
% \cavan{you mean we can set the subcip size to 2 during training and expand it during inference? This point is not clearly articulated here.}
%  \cavan{where? In future work?}
% \cavan{not sure why we use `Moreover', it doesn't connect well to the previous sentence.}
% % Hence, we can enlarge the subclip size for more efficient inference and global consistency within each tube. 

\noindent
\textbf{Effect of Sub-clip Size for Inference.} 
\if 0
The global queries for each tube learn to perform temporal association via cross-attention within each subclip. Despite the subclip size is limited during the training due to the memory issues, we can expand it during the inference.
For example, the subclip size is 2 during training and is set to 6 for inference. As shown in Tab.~\ref{tab:ablation_e}, we prove that enlarging subclip size for inference improves the performance by a significant margin for all three metrics: STQ, VPQ and $\mathrm{VPQ_{th}}$. When the size is 8, the performance drops. This is because the global queries cannot handle larger subclips as the offline method. Besides the effectiveness, increasing subclip size can also lead to faster speed for each clip input due to full utilization of GPU memory, as shown in Fig.~\ref{fig:curve}.
\fi
%
During training, the subclip size is limited due to memory constraints, but we can expand it during inference to improve the performance. For instance, we use a subclip size of 2 during training and increase it to 6 during inference. Tab.~\ref{tab:ablation_e} shows that enlarging the subclip size for inference improves the performance considerably for all three metrics: STQ, VPQ, and $\mathrm{VPQ_{th}}$. However, when the subclip size is further increased to 8, the performance drops because the global queries are not designed to handle larger subclips. Increasing the subclip size can also speed up the inference process by utilizing the full GPU memory, as demonstrated in Fig.~\ref{fig:curve}.

% \cavan{`lead to a higher number of frames leads to faster speed'? Rephrase this sentence.}

\noindent
\textbf{Different Tracking Choices.} 
\if 0
In Tab.~\ref{tab:ablation_f}, we compare different tracking approaches that were used in previous studies~\cite{qdtrack,li2022videoknet,deepsort}. The default Tube Embedding works best in our framework. It does not require any association embedding head or the RoI crop operation on the VIPSeg dataset. Our Tube-Link only uses the learned tube-level embedding for the association.
\fi
In Tab.~\ref{tab:ablation_f}, we compare different tracking approaches used in previous studies~\cite{qdtrack,li2022videoknet,deepsort} with our Tube-Link. Our Tube-Link only uses the learned tube-level embedding for the association. We find that the default tube embedding works best in our framework, without requiring any association embedding head or RoI crop operation on the VIPSeg dataset.  

\subsection{Visualization and More Analysis}
\label{sec:vis_analysis}

\noindent
\textbf{GFLops and Parameter Analysis.} Compared with Mask2Former baseline, we only add one $\mathrm{Emb}$ head and one self-attention layer, introducing only 2.2\% GFLops and 1.4\% extra parameters with $720 \times 1280$ input. 

%\cavan{the font size is too small to be visible. You can use a common legend for both plots and place it underneath the plots}



% \subsection{Visualization and Analysis}
% \cavan{Do we really need this section? The `Speed and Accuracy with different Input Subclip Size' can be merged with `Effect of Sub-clip Size For Inference.' in the ablation study. `Visual Improvements on Baseline' can be merged with `Improvements over Strong VPS Baseline'.}

\noindent
\textbf{Speed and Accuracy with Different Input Subclip Size.} 
As shown in Table \ref{tab:ablation_e}, adding more frames improves the VPS results. To further analyze the speed-accuracy trade-off, we present a detailed comparison of different methods on the VSPW dataset in Fig.~\ref{fig:curve}. The left plot shows that enlarging the subclip size also improves the VSS results. The right plot illustrates that increasing the subclip size improves the single-frame baseline by 1.25-1.5\% for various backbones. Both performance and speed reach a plateau when the size increases to 6. The experiment justifies our choice of using an input subclip size of 6 for inference.

\noindent
\textbf{Visual Improvements on Baselines.} In Fig.~\ref{fig:visulize}, we present the visual comparison with several strong baselines (Video K-Net+ and Mask2Fomer-VIS) in VPS and VIS settings. The results are randomly sampled from a long clip. We achieve better results on both segmentation and tracking. More visual examples can be found in the supplementary material. 

 \section{Conclusion}
 In this paper, we have presented a tactile manipulation system that is able to rotate different objects without vision. We showed an end-to-end reinforcement learning framework to learn tactile dexterity over the proposed system. We carried out experiments both in simulation and real to demonstrate its effectiveness. Our work demonstrated that we are able to achieve tactile dexterity as humans in real for the first time. In the future, there are many promising future directions to investigate, such as exploring the use of a more dense contact sensor array and scaling up the system to solve more diverse tasks. We hope that our work can pave the way for more intelligent robot hands.
%%%%%%%%%%%%%%%%%%%%%%%%%%%%%%%%%%%%%%%



\printcredits

%% Loading bibliography style file
%\bibliographystyle{model1-num-names}
\bibliographystyle{cas-model2-names}

% Loading bibliography database
\bibliography{TMKU.bib}



\end{document}

