\section{Preliminaries and Problem Statement}
\label{sec:preliminaries}

This section introduces the redesigned data structure and briefly describes key definitions concerning the problem of targeted top-k high utility itemset mining. $\mathcal{D}$ denotes a transaction database represented as multiple transactions \{$t_1$, $t_2$, $t_3$, $\ldots$, $t_n$\}. Each transaction contains one or more items from a set denoted as \{$i_1$, $i_2$, $i_3$, $\ldots$, $i_m$\}. The transaction identifier of a transaction $t_\textit{id}$ is $id$. In the case where an itemset includes $l$ items, it is called an $l$-itemset. Each item in a transaction can be annotated with a positive integer value indicating a purchase quantity (also called internal utility value or count). For example, $t_1$ = \{$(b,4)$, $(d,3)$, $(e,1)$\} represents four units of an item $b$, three units of an item $d$, and one unit of an item $e$. The itemset $\{ b,d,e\}$ is called a 3-itemset.

\begin{minipage}{\textwidth}
	\begin{minipage}[t]{0.45\textwidth}
		\centering
		\makeatletter\def\@captype{table}\makeatother\setlength{\belowcaptionskip}{10pt}
		\caption{Example database}
		\label{Tab:example} 
		\begin{tabular}{ccc}
			\hline
			$T_{id}$ &   Transaction &  Count    \\ \hline
			$t_1$    & $\{b,d,e\}$    & \{4,3,1\} \\
			$t_2$    & $\{a,b,e\}$  & \{1,3,2\} \\
			$t_3$    & $\{c,d\}$    & \{3,1\} \\
			$t_4$    & $\{a,b,c,f,g\}$    & \{3,1,4,2,1\} \\
			$t_5$    & $\{a,b\}$    &  \{1,2\} \\
			$t_6$    & $\{b,c,d,f\}$     & \{1,3,4,2\} \\
			$t_7$    & $\{a,e,g\}$   &  \{4,1,2\} \\ 
			\hline
			\\
		\end{tabular}
	\end{minipage}
	\begin{minipage}[t]{0.45\textwidth}
		\centering
		\makeatletter\def\@captype{table}\makeatother\setlength{\belowcaptionskip}{10pt}
		\caption{High-utility itemsets w.r.t \textit{minUtil} = \$25}
		\label{Tab:hui}
		\begin{tabular}{cccc}
			\hline
			Itemset &   Utility   & Itemset &   Utility   \\\hline
			$\{f,c,d,b\}$    & \$27  &$\{c,d,b\}$    & \$25 \\
			$\{e,d,b\}$    &  \$26 & $\{d\}$    & \$32\\
			$\{e,b\}$    & \$27  & $\{d,b\}$    & \$43  \\
			$\{c,d\}$    & \$32  &$\{b\}$    & \$33\\			
			\hline
		\end{tabular}
	\end{minipage}
	\begin{minipage}{\textwidth}\vspace{-0.5cm} 
	\end{minipage}
\end{minipage}


\begin{definition}
	\rm (The utility of the itemset) There are two types of utility values in $\mathcal{D}$. On the one hand, in each transaction $t_\textit{id}$, the internal utility of an item $i \in t_\textit{id}$ is denoted as $iu(i,t_\textit{id})$. On the other hand, the external utility of an item $i$ is a positive number denoted as $eu(i)$ that is utilized to indicate its relative weight or importance. Thus, in the transaction $t_\textit{id}$, $u(i,t_\textit{id}$) represents the utility of item $i$, which is the product of $eu(i)$ and $iu(i,t_\textit{id})$, that is $u(i,t_\textit{id})$ = $eu(i)$  $\times$  $iu(i,t_\textit{id})$). For an itemset $X$ and a transaction $t_\textit{id}$ containing $X$  ($X \subseteq t_\textit{id}$), the utility of $X$ in $t_\textit{id}$ is calculated as  $u(X,t_\textit{id})$ = $\sum_{i_{\ell}\in X \wedge X \subseteq t_\textit{id}}u(i_{\ell},~ t_\textit{id})$.  The notation $u(X)$ denotes the utility of $X$ in $\mathcal{D}$ and it is calculated as the sum of the utility of $X$ in all transactions containing $X$, that is $u(X)$ = $\sum_{X\subseteq t_{id} \in \mathcal{D}}{u(x,t_{id})}$.
\end{definition}

For example, the external utility of all items from the example database are \{($a$: 1), ($b$: 3), ($c$: 2), ($d$: 4), ($e$: 2), ($f$: 1), ($g$: 2)\}. In Table \ref{Tab:example}, $iu(b,t_2)$ = \$3, $eu(b)$ = \$3, then $u(b,t_2)$ = $iu(b,t_2)$ $\times$ $eu(b)$ = 3 $\times$ \$3 = \$9.  $u(\{ae\},t_2)$ = $u(a,t_2)$ + $u(e,t_2)$ = 1 $\times$ \$1 + 2 $\times$ \$2 = \$5, $u(\{abc\},t_4)$ = $u(a,t_4)$ + $u(b,t_4)$ + $u(c,t_4)$ = 3 $\times$ \$1 + 1 $\times$ \$3 + 4 $\times$ \$2 = \$14. Since ${d}$ has appeared in $t_1$, $t_3$, and $t_6$, $u(d)$ = $u(d,t_1)$ + $u(d,t_3)$ + $u(a,t_6)$ = 3 $\times$ \$4 + 1 $\times$ \$4 + 4 $\times$ \$4 = \$32, and $u(\{ab\})$ = $u(\{ ab\},t_2)$ + $u(\{ ab\},t_4)$ + $u(\{ ab\},t_5)$ = $u(a,t_2)$ + $u(b,t_2)$ + $u(a,t_4)$ + $u(b,t_4)$ + $u(a,t_5)$ + $u(6,t_5)$ = 1 $\times$ \$1 + 3 $\times$ \$3 + 3 $\times$ \$1 + 1 $\times$ \$3 + 1 $\times$ \$1 + 2 $\times$ \$3 = \$23.


\begin{definition}
	\rm (The transaction-weighted utility of itemset) The  notation $tu(t_\textit{id})$ refers to the utility of the transaction $t_\textit{id}$ and is defined by the sum of the utilities of all the items in $t_\textit{id}$. In other words, $tu(t_\textit{id}$) = $\sum _{i_{\ell}\in t_\textit{id}}u(i_{\ell},t_\textit{id})$. The transaction-weighted utility of an itemset $X$ in $\mathcal{D}$ is denoted as \textit{TWU(X)} and is the sum of the utilities of $X$ in all the transactions containing $X$ in $\mathcal{D}$, that is \textit{TWU(X)} = $\sum_{X\subseteq t_{id} \wedge t_{id} \in \mathcal{D}}u(X,~ t_\textit{id})$ $ = $ $\sum_{X\subseteq t_\textit{id} \wedge t_\textit{id}\in \mathcal{D}}\sum_{i_{\ell}\in X}u(i_{\ell}, ~t_\textit{id})$.
\end{definition}


For example, $tu(t_{1})$ = \$26, and the transaction having the maximum utility is $t_6$ with $tu(t_{6})$. Moreover, the calculated transaction utility of each transaction is \{($t_{1}$: \$26), ($t_{2}$: \$14), ($t_{3}$: \$10), ($t_{4}$: \$18), ($t_{5}$: \$7), ($t_{6}$: \$27), ($t_{7}$: \$10)\}.

\begin{property}\label{GR}
	The \textit{TWU} complies with the property of downward closure. This means that if an itemset's $\textit{TWU(X)}$ is less than or equal to the utility threshold $xi$, any other itemsets containing $X$ are low-utility itemsets.
\end{property}

The \textit{TWU} values of all the items are determined by initially scanning the database $\mathcal{D}$. Property \ref{GR} states that when $\textit{TWU(X)}$ < $\xi$ for an item $X$, $X$ will be discarded to reduce the number of itemsets to be evaluated. To achieve the goal of fast mining, the paper uses a TWU ascending sort and processes each transaction in the transaction database according to that order as well. The transaction-weighted utility of all 1-itemsets is sorted as: $g$ (= \$28) $<$ $f$ (= \$45) $<$ $a$ (= \$49) $<$ $e$ (= \$50) $<$ $c$ (= \$55) $<$ $d$ (= \$63) $<$ $b$ (= \$92).

\begin{definition}
	\rm (Remaining utility) If there is an itemset $X$, and $X \subseteq t_\textit{id}$, the remaining items after $X$ in $t_\textit{id}$ is denoted as $t_\textit{id}/X$. The remaining utility \cite{liu2012mining} of itemset $X$ in the transaction $t_\textit{id}$ is the sum of the utilities of  $t_\textit{id}/X$ denoted as $ru(X,t_\textit{id})$, where $ru(X,t_\textit{id})$ = $\sum_{ i_{\ell} \in t_\textit{id}/X}u(i_{\ell}, ~t_\textit{id})$. Similarly, the remaining utility of $X$ in  $\mathcal{D}$ is denoted as $ru(X)$, where $ru(X)$ = $\sum_{X \subseteq t_\textit{id} \in\mathcal{D}}ru(X,t_\textit{id})$.
\end{definition} 


In Table \ref{Tab:example}, the remaining utility of $\{bd\}$ in $t_1$ is defined as $ru(\{bd\},t_1)$ = $u(e,t_1)$ = \$2.

\begin{definition}
	\rm (Utility-list) In HUI-Miner \cite{liu2012mining}, the utility-list was introduced for the utility mining task. There are three essential elements ($t_\textit{id}$, \textit{iutil}, and \textit{rutil}) in the utility-list. The $t_\textit{id}$ is the identifier of a transaction that contains $X$. Element \textit{iutil} represents the utility of $X_i$ in $t_\textit{id}$. Similarly, \textit{rutil} is the remaining utility of $X_i$ in $t_\textit{id}$.
\end{definition}


For example, the utility-list of $\{c\}$ is \{$t_3$, \$6, \$4\}, \{$t_4$, \$8, \$4\} and \{$t_6$, \$6, \$18\}. The utility-list of $\{f\}$ is \{$t_4$, \$2, \$2\} and \{$t_6$, \$2, \$0\}. Therefore, the utility-list of $\{c, f\}$ is \{$t_4$, \$10, \$2\} and \{$t_6$, \$8, \$0\}.

\begin{definition}
	\rm (Target high-utility itemset) Let \textit{targetUtil} be the target utility threshold specified by a user and denoted as $\xi$. $X$ is known as a HUI for the user if its utility value is not smaller than  $\xi$. A target itemset  set by the user is called $T^\prime$. This itemset is called target since it is what the user wants to focus on. If $ u(X)$ $> \xi $, $T^\prime$ $\subseteq$ $X$, then $X$ is further recognized as one of the target high-utility itemsets (THUIs) \cite{miao2021targeted}.
\end{definition}

For example, a user wants to look for HUIs related to the target itemset $T^\prime$ = $\{b,d\}$. From Table \ref{Tab:hui}, we can search the HUIs with $T^\prime$. As a result, the set of THUIs is \{$\{f,c,d,b\}$, $\{e,d,b\}$, $\{c,d,b\}$, $\{d,b\}$\}.


\begin{definition}
	\rm (Top-$k$ target high-utility itemset) Inspired by other top-$k$ pattern mining tasks, we define the top-$k$ target high-utility itemsets (top-$k$ THUIs) as the $k$ targeted itemsets that have the highest utility values. The result itemsets will be less than $k$ items if the itemsets containing the target pattern in the database are relatively rare and the $k$ set by the user is relatively large. When there are more than $k$ patterns with the same utility, the results will also show only the $k$ itemsets in the order of mining.
\end{definition}

For example, if a user only wants to find the top-3 itemsets that satisfy the previous condition, then we set $k$ to 3. Obviously, the top-3 target HUIs are $\{f,c,d,b\}$ = \$27, $\{e,d,b\}$ = \$26, and $\{d,b\}$ = \$43.


\textbf{Problem statement}: Most utility mining algorithms are designed to identify itemsets with a utility equal to or greater than a fixed utility threshold. However, in reality, the user does not want to find all HUIs. For instance, a user may only want the top-$k$ HUIs containing a particular item or itemset. This paper calls the designed pattern mining problem as the targeted mining of top-$k$ high utility itemsets.

Unlike other HUIM problems, this paper needs to face the challenge of how to better deal with the search space under the combination of the target pattern and the top-$k$ pattern. In other words, it needs to apply a reasonable and efficient data structure to store the itemset information and a correct and fast pruning strategy to improve the mining efficiency. So far, the relevant definitions and concepts have been described, and the next section will introduce the details of the proposed TMKU algorithm.