\section{Conclusion} \label{sec:conclusion}

Until now, no algorithm has been proposed to mine the targeted top-$k$ high utility itemsets. In this research, the TMKU algorithm is proposed for this task. The algorithm utilizes several data structures, namely the utility list, TP-tree, TopKMap, and two threshold-raising strategies (SUR and RIU) to achieve the task. In addition, several pruning strategies are applied in the TMKU algorithm to lower the execution time. Under this task, users can determine $k$ and $T^\prime$ according to their needs. After testing the TMKU algorithm on six datasets, it was found that TMKU is able to create the TP-tree completely, raise the threshold quickly, and obtain the top-$k$ THUIs efficiently. Moreover, TMKU has short processing time, stable memory usage, and excellent scalability. The TMKU algorithm presents a solution to a new problem in the field of data mining, and in the future, we will work on designing more advanced algorithms to explore this promising topic. Furthermore, to explore more applications of TMKU and extend it to more scenarios, such as using it in a distributed environment. Future research could also take into account visualization of mining results to boost interpretability.


\section*{Acknowledgment}

This research was supported in part by the National Natural Science Foundation of China (Nos. 62002136, 62272196, 61472049, and 61572225), Natural Science Foundation of Guangdong Province (No. 2022A1515011861), Guangzhou Basic and Applied Basic Research Foundation (No. 202102020277), the Young Scholar Program of Pazhou Lab (No. PZL2021KF0023), Engineering Research Center of Trustworthy AI, Ministry of Education (Jinan University), and Guangdong Key Laboratory for Data Security and Privacy Preserving..