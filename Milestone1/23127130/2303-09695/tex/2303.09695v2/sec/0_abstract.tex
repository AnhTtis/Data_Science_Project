\begin{abstract}
\vspace{-0.1in}
Garment pattern design aims to convert a 3D garment to the corresponding 2D panels and their sewing structure. 
Existing methods rely either on template fitting with  heuristics and prior assumptions,
or on model learning with complicated shape parameterization.
Importantly, both approaches do not allow for personalization of the output garment, which today has increasing demands.   
%
%Not only are the previous methods less effctive,but also a desirable panel personalization capability is totally missing. This leads to a gap between limited ready-made garments and diverse individual requirements.
%
To fill this demand, we introduce {\bf\em PersonalTailor:} a personalized 2D pattern design method, where the user can input specific constraints or demands (in language or sketch) for personal 2D pattern design from 3D point clouds.
%both 3D point cloud garment and the customer need (in form of language or sketch) can be jointly leveraged  for 2D panel fabrication tailored to personal requirements.
% To address this problem, we further introduce a novel {\bf\em PersonalTailor} method characterized by multi-modal panel embedding
PersonalTailor first learns disentangled multi-modal panel embeddings based on unsupervised cross-modal association and attentive fusion. It then predicts 
2D binary panel masks individually using a transformer encoder-decoder framework.
%PersonalTailor is characterized by multi-modal panel embedding and binary panel mask prediction using an encoder-decoder framework.
%More specifically, multi-modal panel embeddings are first learned based on unsupervised cross-modal association and attentive fusion, that are further decoded into binary panel masks individually using a transformer decoder design.
% \saura{More specifically, the panel embedding is indexed
% by panel position, naturally enabling 
% both standard and personalized fabrication.} 
%
Extensive experiments show that our PersonalTailor
excels on both personalized and standard garment pattern design tasks.
\end{abstract}