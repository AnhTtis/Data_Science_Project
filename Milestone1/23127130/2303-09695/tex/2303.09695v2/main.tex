% CVPR 2024 Paper Template; see https://github.com/cvpr-org/author-kit

\documentclass[10pt,twocolumn,letterpaper]{article}

%%%%%%%%% PAPER TYPE  - PLEASE UPDATE FOR FINAL VERSION
% \usepackage{cvpr}              % To produce the CAMERA-READY version
% \usepackage[review]{cvpr}      % To produce the REVIEW version
\usepackage[pagenumbers]{cvpr} % To force page numbers, e.g. for an arXiv version

% Import additional packages in the preamble file, before hyperref
\usepackage[table,xcdraw]{xcolor}


\usepackage{times}
\usepackage{epsfig}
\usepackage{graphicx}
\usepackage{amsmath}
\usepackage{amssymb}
\usepackage{pifont}% http://ctan.org/pkg/pifont
\usepackage{stmaryrd}
\usepackage{trimclip}
% \usepackage{soul}
% \usepackage{tikz}
\newcommand{\cmark}{\ding{51}}%
\newcommand{\xmark}{\ding{55}}%

% \usepackage{graphicx}
% \usepackage{amsmath}
% \usepackage{amssymb}
\usepackage{booktabs}
\usepackage{caption}
\usepackage{multirow}
\usepackage{multicol}% http://ctan.org/pkg/multicols



% Include other packages here, before hyperref.

% If you comment hyperref and then uncomment it, you should delete
% egpaper.aux before re-running latex.  (Or just hit 'q' on the first latex
% run, let it finish, and you should be clear).
\usepackage[pagebackref=true,breaklinks=true,letterpaper=true,colorlinks,bookmarks=false]{hyperref}
% \usepackage[latin1]{inputenc}
\usepackage[british]{babel}
\usepackage[all]{xy}
\usepackage{amscd}
\usepackage{amssymb}
\usepackage{amsthm}
\usepackage{enumitem}
\usepackage{mathrsfs,bbm}
\usepackage{xcolor,graphicx}
\usepackage{graphics}
\usepackage{soul}
\usepackage{comment}
\usepackage[all]{xy}
\usepackage{amscd}
\usepackage{amssymb,amsmath,latexsym}
\usepackage{amsthm}
\usepackage{enumitem}
\usepackage{mathrsfs,bbm}
\usepackage{dsfont}
\usepackage{tikz-cd}
\usepackage[T1]{fontenc}
\usepackage[utf8]{inputenc}  
 %
%%%%%%%%%%%%%%%%%%%%%%%%%%%%%%%%%%
%pagestyle
%%%%%%%%%%%%%%%%%%%%%%%%%%%%%%%%%%
%\pagestyle{plain}
\textwidth=430pt
\headsep=.7cm
\evensidemargin=15pt
\oddsidemargin=15pt
\leftmargin=0cm
\rightmargin=0cm
%%
%%%%%%%%%%%%%%%%%%%%%%%
\newcommand*\fixitem {\item[]%
  \refstepcounter{enumi}\hskip-\leftmargin\labelenumi\hskip\labelsep}
\newtheorem*{mainthm}{Main Theorem}
\newtheorem*{mainthm1}{Theorem}
\newtheorem*{maincor}{Corollary}
\usepackage[colorlinks=true]{hyperref}
\DeclareMathOperator{\Forall}{\forall}
\DeclareMathOperator{\Exists}{\exists}
\DeclareMathOperator{\ord}{ord}
\newcommand{\phiD}{\varphi_D}
\newcommand{\phiDI}{\varphi_{\mathbf{D}_I}}
\newcommand{\phiDIj}{\varphi_{\mathbf{D}_I (j)}}
\newcommand{\phiH}{\varphi_H}
\newcommand{\phiTimes}{\phiD \otimes \phiH}
\newcommand{\phiTimesDI}{\varphi_{\mathbf{D}_I} \otimes \phiH}
\newcommand{\R}{\mathscr{A}}
\newcommand{\X}{\mathscr{X}}
\newcommand{\Xf}{\mathscr{X}_{(k_0 ,i)}[r_0]}
\newcommand{\Xfr}{\mathscr{X}_{(k_0,i)}[r]}
\newcommand{\hotimes}{\widehat{\otimes}}
\newcommand{\C}{\mathbb{C}_p}
\newcommand{\V}{\mathscr{V}}
\newcommand{\B}{\mathscr{B}}
\newcommand{\dualD}{\mathfrak{D}}
\newcommand{\Dg}{\mathbf{D}}
\newcommand{\DD}{\mathcal{D}^0}
\newcommand{\DDg}{\mathcal{D}}
\newcommand{\DV}{\mathcal{D}}
\newcommand{\W}{\mathscr{W}_N}
\newcommand{\Ao}{\mathbf{A}^\circ}
\newcommand{\AoK}{\mathbf{A}^\circ_{\K}}
\newcommand{\AK}{\mathbf{A}_{/\K}}
\newcommand{\OOO}{\mathscr{A}^\circ}
\newcommand{\K}{\mathcal{K}} 
\newcommand{\OK}{\mathcal{O}_{\K}}
\newcommand{\varprojlog}[1]{\underleftarrow{\log\!^{#1}}}
\newcommand{\T}{\mathscr{T}}
\newcommand{\TT}{\mathbf{T}}
\newcommand{\VV}{\mathbf{V}}
\newcommand{\HH}{\mathcal{H}}
\newcommand{\hh}{\mathcal{H}^+}
\newcommand{\HG}[2]{\mathcal{H}_{#1}(#2)}
\newcommand{\hhl}{\mathcal{H}^{+,[l]}}
\newcommand{\hhj}{\mathcal{H}^{+,[j]}}
\newcommand{\hhjj}{\mathcal{H}^{+,[l,l']}}
\newcommand{\GS}{G_{\mathbb{Q},S}}
\newcommand{\Rf}{R_{(k_0 ,i)}[r_0]}
\newcommand{\Rfr}{R_{(k_0 ,i)}[r]}
\newcommand{\parT}{\langle T\rangle}
\newcommand{\Zf}{Z_{(k_0 ,i)}[r_0]}
\newcommand{\Zfr}{\mathscr{Z}_{(k_0 ,i)}[r]}
\newcommand{\ZFf}{\mathscr{Z}_{(k_0 ,i)}[r_0]}
\newcommand{\ZFfr}{\mathscr{Z}_{(k_0 ,i)}[r]}
\newcommand{\ZF}{\mathscr{Z}}
% It is strongly recommended to use hyperref, especially for the review version.
% hyperref with option pagebackref eases the reviewers' job.
% Please disable hyperref *only* if you encounter grave issues, 
% e.g. with the file validation for the camera-ready version.
%
% If you comment hyperref and then uncomment it, you should delete *.aux before re-running LaTeX.
% (Or just hit 'q' on the first LaTeX run, let it finish, and you should be clear).
\definecolor{cvprblue}{rgb}{0.21,0.49,0.74}
% \usepackage[pagebackref,breaklinks,colorlinks,citecolor=cvprblue]{hyperref}

%%%%%%%%% PAPER ID  - PLEASE UPDATE
\def\paperID{3} % *** Enter the Paper ID here
\def\confName{3DV\xspace}
\def\confYear{2024\xspace}

%%%%%%%%% TITLE - PLEASE UPDATE
\title{PersonalTailor: Personalizing 2D Pattern Design from 3D Garment Point Clouds}

%%%%%%%%% AUTHORS - PLEASE UPDATE
\author{Sauradip Nag$^{1}$
\and 
Anran Qi$^{2}$
\and
Xiatian Zhu$^{3,4}$
\and
Ariel Shamir$^{5}$
\and \newline
{\small $^1$ Independent Researcher} ~
{\small $^2$ University of Tokyo, Japan} ~ 
{\small $^3$ Surrey Institute for People-Centred Artificial Intelligence, UK} \\
{\small $^4$ CVSSP, University of Surrey, UK} ~
{\small $^5$ Reichman University, Israel}
}
\begin{document}
\twocolumn[{%
\renewcommand\twocolumn[1][]{#1}%
\maketitle
\vspace{-6mm}
\begin{center}
    \centering
    \captionsetup{type=figure}
    \includegraphics[width=0.95\linewidth]{img/fig1_v6.pdf}
    % \vspace{-4mm}
    \captionof{figure}{\textbf{Illustration of personalized 2D pattern design.} Given a 3D point cloud, our goal is to predict the 2D panels and sewing pattern {\bf(a)}, and allow its \emph{personalization} based on a user text or sketch prompt. This supports various functionalities including {\bf(b)} adding new panels (\eg, waistbands), {\bf(c)} removing panels (\eg, sleeves), {\bf(d)} changing the topology (skirt \textrightarrow pant), and {\bf(e)} creating new design. }
   \label{fig:teaser}
\end{center}%
}]


% \maketitle
\begin{abstract}
The current study investigated possible human-robot kinaesthetic interaction using a variational recurrent neural network model, called PV-RNN, which is based on the free energy principle.
Our prior robotic studies using PV-RNN showed that the nature of interactions between top-down expectation and bottom-up inference is strongly affected by a parameter, called the meta-prior, which regulates the complexity term in free energy.
% The current study examines how the behaviours of robots alter by changing the meta-prior $w$ in human-robot kinaesthetic interaction.
The current study examines how changing the meta-prior $w$ in the interaction phase affects the counter force generated when an experimenter attempts to induce movement pattern transitions familiar to the robot through its prior training.
The study also compares the counter force generated when trained transitions are induced by a human experimenter and when untrained transitions are induced.
Our experimental results indicated that (1) the human experimenter needs more/less force to induce trained transitions when $w$ is set with larger/smaller values, (2) the human experimenter needs more force to act on the robot when he attempts to induce untrained as opposed to trained movement pattern transitions.
Our analysis of time development of essential variables and values in PV-RNN during bodily interaction clarified the mechanism by which gaps in actional intentions between the human experimenter and the robot can be manifested as reaction forces between them.


%% Hiroki writing 2022-11-4
%Current study investigates the dynamics of the latent states during human-robot kinaesthetic interaction using PV-RNN.
%We have achieved to observe and analyse the internal state of an RNN model based on the free energy principle, during real-time human-robot interaction.
%Essential characteristics observed in the previous study of this variational recurrent neural network model, PV-RNN, is that by changing a meta prior $w$, the balance between the top-down intention and the bottom-up perceptual reality changes.
%In the current study, we examined how changing the weighting parameter $w$ between accuracy and complexity in free energy principle affects the humanoid robot's behaviour through human-robot interaction. We have conducted some human-robot kinaesthetic interaction experiments with various $w$ and quantitatively analysed the latent variable and the force applied to the humanoid robot. We have observed that the force required to change the robot's intention has increased, both when the top-down intention was strengthened by changing the $w$ and when corresponding switch of its primitive was against the experience of the RNN during its training. The study confirms through quantitative analysis that by increasing or decreasing the $w$ in PV-RNN, humanoid robot leads or follows the human counterpart during the human-robot kinaesthetic interaction.

\begin{comment}
Comment from Jun #2
・最後にQualitativeな結果(インパクト)が欲しい
・Current study investigates the problem on~と書き出すのが一般的
・最初の一文と最後の一文を対応させる
・最後の一文はもう少しAbstractかつ包括的に
\end{comment}

\begin{comment}
Comment from Jun #1
We investigated how the kinaesthetic human-robot interaction can affect the internal state of a model based on the free energy principle. 
=> how the internal state is affected is not the most important point in this study. This part should be rewritten.

The key function of this variational recurrent neural network model, PV-RNN, is that by changing a meta prior $w$, it takes a balance between the "complexity” term and the ”accuracy” term which corresponds to a top-down intention and a bottom-up perceptual reality in the free energy principle, respectively. 
=> This is not key function of PV-RNN. It is an essential characteristics observed in the previous study. The grammar after $w$ is something strange. Rewrite these.

This research has conducted a human-robot interaction experiment with a robotic agent in a kinaesthetic sense.
=> The sentence is not good. "in a kinaesthetic sense" is grammatically wrong.
MODIFIED => "In the current study human-robot interaction experiments using the kinaesthetic sense were conducted."

We investigated that when human forces the agent to switch primitives from one to another, larger force was required both when the human intention is conflictive against the top-down the intention of the agent and when the agent has a stronger top-down intention by modifying the $w$.
=> You should write the essential results of the experiments rather than what we investigated and also how these results could contribute to the studies on human-robot interaction.
\end{comment}

\end{abstract}    
\section{Introduction}
\label{sec:intro}
\begin{figure}[t]
\begin{center}
    \includegraphics[width=1\linewidth]{figures/teaser.pdf}
\end{center}
\vspace{-0.1in}
\caption{\textbf{{\em Foggy} vs {\em Clear} NeRF.} Our \ournerf gets rid of reconstruction errors manifested as foggy ``floaters" in the density volume without additional input or significant computational overhead. 
%
Below are density profiles along a given ray before and after our geometry correction procedure, where we discard density peaks corresponding to floaters.
}
\label{fig:teaser}
\vspace{-0.2in}
\end{figure}



%The emergence of 
Neural Radiance Fields (NeRFs)~\cite{mildenhall2020nerf}  %and its variants 
have made revolutionary contributions in %photo-realistic 
novel view synthesis~\cite{barron2021mip,barron2022mip}, 
autonomous driving~\cite{rematas2022urban,tancik2022block}, digital human~\cite{hong2022headnerf,zhao2022humannerf}, and 3D content generation~\cite{eg3d,poole2022dreamfusion,lin2022magic3d}.
%by leveraging a multi-layer perceptron (MLP) to implicitly model the mapping from input 5D coordinates (i.e., 3D coordinates $\mathbf{x} = (x,y,z)$ and 2D viewing directions $\mathbf{d}=(\theta,\phi)$) to volume density $\sigma$ and view-dependent emitted radiance color $\mathbf{c} = (r,g,b)$. 
%
%They then use traditional volume rendering mechanisms on the obtained continuous 5D function (i.e., MLP) to generate novel views. 
To date, unfortunately, most NeRF-based methods encounter challenges when tackling large-scale cluttered scenes (e.g., Fig.~\ref{fig:teaser}):
\begin{enumerate}[leftmargin=0.16in, topsep=2pt,itemsep=-1ex,partopsep=1ex,parsep=1ex]
\item Input observations used for NeRF are often too sparse  compared to forward-facing or synthetic looking-inward scenes;
%\item Recovering fine-grained objects within a large volume is challenging for NeRF; %in capturing details accurately.
\item View-dependent visual effects give rise to ambiguity, resulting in a ``foggy" density field as shown in Fig.~\ref{fig:teaser}. 
%
Such artifacts are particularly pronounced in indoor scenes strewn with view-dependent appearances, such as specular highlights, glossy surface reflections from man-made objects. 
\end{enumerate}

Despite attempts to enhance NeRF's rendering quality given suboptimal input, such as using 3D conical frustums~\cite{barron2021mip,barron2022mip}, physically-grounded augmentations~\cite{chen2022aug}, and misalignment correction~\cite{jiang2022alignerf},  these challenges have yet to be fully resolved.
%
Depth supervision~\cite{deng2022depth, wei2021nerfingmvs} or proxy geometry~\cite{xu2021scalable,wu2022scalable} images can help alleviate the challenges in handling large-scale with sparse input, at the expense of %but they come at the cost of requiring 
expensive pre-processing or additional input.
%
Another line of work~\cite{wang2021neus, oechsle2021unisurf, wang2022neuris} achieves better reconstruction of surface geometry by using signed distances instead of volume density as scene representation. However, they sacrifice the ability to synthesize photo-realistic novel views.

%We observe that NeRF has been suffering from foggy ``floater" artifacts in large-scale cluttered scenes.
%
%Such artifacts are particularly pronounced in indoor scenes strewn with view-dependent appearances from man-made objects. 
%
To address the above issues, we propose an extension to NeRF, dubbed as {\bf \ournerf}, which enforces effective {\em appearance} and {\em geometry} constraints conducive to accurate colors and 3D densities estimation. We believe \ournerf can contribute beyond novel view synthesis, such as NeRF object detection~\cite{hu2022nerf}, NeRF object segmentation~\cite{zhi2021place, liu2022unsupervised, fan2022nerf,ren2022neural}, and NeRF registration~\cite{goli2022nerf2nerf}, where the rooms for improvement are substantial if more accurate color and density estimation are available.

Correspondingly, there are two steps in \ournerf. First, for appearance correction, the view-independent and view-dependent color components are predicted from the underlying 3D scene, which is combined to produce the final color estimation (Fig.~\ref{fig:toaster}).
%
The view-independent component (diffuse color and shading) captures the overall scene color, while the view-dependent component (highlights or reflections) captures color variations due to changes in viewing angle.
%
\ournerf then discards these view-dependent appearances in the training views to prevent them from interfering with the density estimation.
%
Second, a simple and effective geometry correction procedure will be performed to further eliminate the foggy ``floaters" or density errors. This geometry correction procedure is based on an assumption in line with traditional ray tracing in computer graphics.
\begin{comment}
% xh: basically copying method
On the other hand, ClearNeRF performs a geometric correction procedure performed on each traced ray during inference to refine the density estimation and better tackle the floater artifacts. 
%
The geometry correction procedure assumes that there should only be one salient peak along each traced ray during NeRF inference. 
Only the salient peak closest to the ray origin (the camera center) corresponds to  true geometry while the others will be manifested as foggy floaters hovering in the density volume. 
%
This assumption is in line with traditional ray tracing in computer graphics where in the absence of noise, only one intersection per ray should be returned to indicate the closest ray-object intersection.
%
\end{comment}
%%%%%%%%%%%
%As shown in Fig.~\ref{fig:teaser}, when reconstructing an indoor scene with sparse input and highly view-dependent objects, NeRF produces severe floating artifacts due to its attempt to explain view-dependent appearances.
%
Experiments verify that our proposed \ournerf can effectively get rid of floater artifacts without additional input.% or significant computational overhead. 


In summary, our contributions include the following:
\begin{itemize}[leftmargin=0.16in, topsep=2pt,itemsep=-1ex,partopsep=1ex,parsep=1ex]
    \item We propose a concise method for decomposing view-independent and view-dependent appearance during NeRF training and eliminate the interference of view-dependent appearance.
    \item We propose a geometric correction procedure performed on each traced ray during inference to refine the density estimation and better tackle the floater artifacts.
    \item Extensive experiments and ablations verify the effectiveness of our core designs and results in improvements over the vanilla NeRF and other state-of-the-art alternatives.
    %without additional computational resources or other inputs.
\end{itemize}




\section{Related work}
% There is extensive recent work on speeding up analytical queries due to the need for consistent execution times in the face of the explosive growth in the volume of available data.
% In this section, we divide existing work into two categories: maintaining data freshness in MVs (\Cref{sec:server_side}) and optimizations for minimizing ad-hoc query latency (\Cref{sec:client_side}).

% \subsection{Maintaining Data Freshness in MVs}
% \label{sec:server_side}
% There exists a variety of data warehousing applications aimed at supporting low-latency analytical queries on fresh data.
% In particular, these applications require efficiency in the propagation of newly ingested data into downstream MVs.
 
\mypara{Efficient MV Refresh}
Incremental view maintenance (IVM) aims to update MVs to reflect newly ingested data, taking advantage of already computed results to perform the update in a manner more efficient than computing from scratch (full refresh)
~\cite{ahmad2012dbtoaster,mcsherry2013differential,armbrust2013generalized,zeng2016iolap, palpanas2002incremental, griffin1995incremental, agiwal2021napa, braun2015analytics}. 
There is an abundance of work in IVM, including incremental updates on duplicate values~\cite{griffin1995incremental}, non-distributive aggregate functions~\cite{palpanas2002incremental}, higher-order views~\cite{ahmad2012dbtoaster}, and sliding windows~\cite{braun2015analytics}. 
More recent works also investigate the scalability aspect of IVM, proposing scale-independent updates~\cite{armbrust2013generalized} and sampled views~\cite{zeng2016iolap}. Since \system is applicable to arbitrary SQL statements, \system is orthogonal to and is fully compatible with existing IVM techniques.

\mypara{MV Refresh Scheduling}
There exist works on scheduling the refresh of a MV set focusing on resolving cyclic dependencies~\cite{folkert2005optimizing}, minimizing weighted average staleness~\cite{golab2009scheduling}, and prioritizing MVs with the highest speedups on predicted future queries~\cite{ahmed2020automated}.
\system's scheduling to speed up the end-to-end refresh of the MV set is not addressed in existing works.

\mypara{DAG Workflow Scheduling}
The execution of workloads consisting of individual jobs with acyclic dependencies is a well-studied topic~\cite{apacheoozie,sparkdag,marchal2018parallel,bathie2020revisiting,baruah2022ilp}; many of these techniques can be applied to MV refresh runs studied in this paper.
Existing workflow scheduling systems such as Apache Oozie~\cite{apacheoozie}, Apache Airflow~\cite{airflow}, and Spark DAG scheduler~\cite{sparkdag} automate the execution of user-defined workflows following a topological order.
There are recent works aimed at finding more optimal execution orders in terms of peak memory usage~\cite{marchal2018parallel, bathie2020revisiting} and execution time on parallel platforms~\cite{baruah2022ilp}.
While \system is designed for use with MV refresh runs/workloads, our technique on joint scheduling and optimization can be reasonably applied to general workloads as a possible future direction.

% \paragraph{Incremental MV indexing}
% Update-optimized indices such as the log-structured merge-trees (LSM)~\cite{o1996log} are used for indexing MVs due to frequent updates induced by data ingestion~\cite{gupta2016mesa,agiwal2021napa}.
% \system is orthogonal to indexing: \system is capable of efficiently performing MV refresh runs regardless of whether the individual MVs are indexed or not.

% \subsection{Ad-hoc Query Latency Reduction}
% \label{sec:client_side}

% The minimization of ad-hoc analytical query response times is a well-studied topic due to latency being negatively correlated with the productivity of a data analyst during a data analysis session~\cite{liu2014effects}.
% Sessions are commonly conducted within visualization systems that contain a variety of optimization techniques to ensure that query response times fall within a certain latency tolerance.

% \mypara{Data prefetching}
% Data is often loaded into memory on a by-need basis in visualization systems to minimize interference with user-issued query computations~\cite{mani2017effective,xin2021enhancing,galakatos2017revisiting, yan2020auto, battle2016dynamic, crotty2016case, jalaparti2018netco}. 
% Query-time data retrieval can be significantly expedited by anticipating the data usage of the user in future queries and pre-loading the data into memory during the downtime between user queries (`think time'). SMART~\cite{mani2017effective} prefetches data for modified versions of current user-issued queries with different filters and dimensions. A-WARE~\cite{crotty2016case} maintains a sample store constantly refined through ingesting data based on speculations of future plots.
% ForeCache~\cite{battle2016dynamic} uses an SVM to predict the user's current analysis phase and accordingly prefetches data tiles partitioned based on different numerical values. NetCo predicts future queries via log analysis, and solves an ILP formulation to prefetch data to maximize the number of SLO-meeting queries~\cite{jalaparti2018netco}.
% In the case of MV refresh workloads, `think time' is nonexistent as individual MVs are refreshed back-to-back, rendering data prefetching techniques non-applicable.

\mypara{Intermediate Data Caching}
Some existing data visualization systems cache user-defined variables to support the typical incremental construction of data visualizations~\cite{zgraggen2016progressive, eichmann2020idebench} during data analysis sessions~\cite{jupyter, rstudio, colab}. 
Recent work proposes a management scheme for these cached variables under a memory constraint that greedily keeps variables with the highest estimated time savings based on predicted future user behavior via neural networks~\cite{xin2021enhancing}.
While useful for data visualization, a greedy approach to memory management fails to achieve satisfactory results compared to \system.

\mypara{Intermediate Result Reuse}

There exist works on storing intermediate results from computations to speedup future computations~\cite{yang2018intermediate, dursun2017revisiting, nagel2013recycling, michiardi2019memory, galakatos2017revisiting}.
Studied topics include the identification of reuse opportunities by finding overlaps in computation graphs of successive jobs~\cite{yang2018intermediate, michiardi2019memory},
selective storage under a space constraint with heuristics such as reuse probability~\cite{dursun2017revisiting}, expected savings~\cite{yang2018intermediate}, and recompute-storage cost difference~\cite{nagel2013recycling},
and rewriting incoming jobs to take advantage of stored intermediates~\cite{galakatos2017revisiting}.
These works share similarity with \system in their selection of items to store under a memory constraint, however, \system's problem setting requires it to uniquely consider the joint (re)ordering of job executions along with the selection of items.

% work that considers both job execution (re)order as well as intermediate result caching with a bounded amount of memory. but notably lack the joint aspect of \system and cannot be used to achieve immediate speedup on an incoming MV refresh run if no intermediates are stored beforehand. 

\mypara{Incremental Query Processing} Incremental processing (IQP) is useful for cases where not all data required for a query is immediately available. Similar to online aggregation~\cite{hellerstein1997online}, initial results of a query are computed on a subset of required data and progressively refined as the rest of the required data arrives in a predictable pattern~\cite{tang2019intermittent,wangtempura}. Tang et al. propose a dynamic programming formulation to pick intermediate states to store in memory given a limited memory budget~\cite{tang2019intermittent}. Tempura rewrites the query plan for more efficient execution based on predicted data arrival patterns~\cite{wangtempura}. While similarities exist between the problem setting of IQP and \system, such as management of bounded memory, \system notably includes additional joint optimization for the order of MV updates.

% \paragraph{Sampling}
% Sampling has seen wide use in visualization systems for reducing the computation time of ad-hoc queries by computing an approximate result over a subset of data as exact results are not always required by the user~\cite{crotty2016case, mani2017effective, zgraggen2014panoramicdata, kraska2021northstar, galakatos2017revisiting, kandula2016quickr}. 
% Commonly studied topics in sampling for ad-hoc queries include complex query sampling~\cite{kandula2016quickr}, rare event aggregation~\cite{kraska2021northstar, galakatos2017revisiting}, and maintaining consistency between related sampled visualizations~\cite{zgraggen2014panoramicdata}.
% Sampling server-side at the MV level compromises the assumptions of downstream applications and is thus not considered in \system.

% \paragraph{Progressive visualization}
% The latency tolerance for time-consuming queries can be circumvented by presenting a partially-computed visualization to the user within the tolerance, which is then incrementally refined until it is fully accurate~\cite{rahman2017ve, zgraggen2016progressive, crotty2015vizdom, kraska2021northstar, kamat2017infiniviz}.
% Example plots which benefit from progressive visualization include bar charts~\cite{kamat2017infiniviz} and heatmaps~\cite{rahman2017ve}.
% Similar to sampling, study on this topic is orthogonal to \system as pushing out partially-updated MVs compromises downstream assumptions.
\section{Method}
\label{sec:method}
\setlength{\belowdisplayskip}{5pt} \setlength{\belowdisplayshortskip}{2pt}

\setlength{\abovedisplayskip}{5pt} \setlength{\abovedisplayshortskip}{5pt}

\noindent \textbf{Overview:} 
Given a source 3D garment $G$ and the personalized requests $L$ from a customer, we aim to predict the corresponding 2D patterns (\saura{which can be stitched together to form} this garment) from a specific library of panels, % \saura{from the dataset}, 
their position in 3D, and their stitching structure.
Our key idea is to define an encoder-decoder network that could encode the multi-modal information from the point clouds and the user's instructions and decode it into the 2D panels structure.
An overview is depicted in Fig.~\ref{fig:overview}. 
\saura{We need to address two key challenges. First, as panel-level point-cloud segmentation is expensive to label, we want to learn the multimodal correspondence across the text/sketch and point cloud. Second, the multi-modal latent panel representations need to be disentangled from each other. Such a design allows us to directly manipulate the panel composition during editing, while visually mirroring these transformation results in producing the garments of different designs and topology as shown in Fig~\ref{fig:teaser}.}

% \saura{ There are two key challenges which guide our design. First, as panel-level point cloud segmentations are expensive to obtain, we would like to learn the multimodal correspondence across the text/sketch and point cloud. Second, the multi-modal latent panel representations need to be disentangled from each other. Such a design allows us to directly manipulate the panel composition during editing, while visually mirroring these transformation results in producing garments of different designs and topology as shown in Fig~\ref{fig:teaser}. }

To apply such learning based approach, we assume a given dataset $D=\{D_{train}, D_{test}\}$ including training and testing garment classes. 
Each subset is in the form of 
$\{P_{i}, L_{i}, Y_{i}\}_{i=1}^{N}$ where
$P_i$ is the garment point cloud,
$L_{i} = \{g_{i},p_{k=1}^{N_{i}}\}$ represents the garment class $g_{i}$ along with the related panel class labels $p_{k}$. 
% It is defined as $D=\{D_{train},D_{test}\}$ where 
% $D_{train/test} = \{P_{i}, L_{i}, Y_{i}\}_{i=1}^{N}$ where $L_{i} = \{g_{i},p_{k=1}^{N_{i}}\}$ represents the garment class $g_{i}$ along with the related panel labels $p_{k}$. 
$Y_{i} = \{m_{j=1}^{N_{i}}, v_{j=1}^{N_{i}}, c_{j=1}^{N_{i}}, r_{j=1}^{N_{i}}, t_{j=1}^{N_{i}}, s_{j=1}^{N_{i}}\}$ where $m_{j}$ is the 2D panel mask of this garment class, 
$v_{j}$  is an ordered list of 2D coordinates for the panel $m_{j}$, $c_{j}$ is the curvature of the edges for $m_{j}$, 
$r_{j}$ and $t_{j}$ are the rotation and  translation (Euler angles) used to define the panel 3D location around a particular body model, $s_{j}$ is the stitch information of $m_{j}$,
and $N_{i}$ is the total number of panels in a given garment $i$ design.  
More details on the ground-truth formulation are provided in supplementary file.
Note that, all the garment classes share the same set of panels,
% from the panels library,
and personalizing the panel configuration allows to form different garments.
%


\saura{Given the garment point-cloud  and the text/sketch panel prompts, each panel of the training data can be converted into a discrete feature map. However, directly combining local part level point-cloud features with its corresponding panel embedding is not trivial, since %the point-cloud and the prompt embedding are
they lie in different feature spaces without part correspondence.
%and the correspondence between these two features is not directly accessible. 
To align the point-clouds and the prompt embeddings of the garment, we propose to reduce the distribution gap  between the two modalities using optimal transport based cross-modal association. To further infuse the aligned point-cloud representation into the prompt embedding space, we perform cross-attention between the point-cloud features and the prompt features to obtain multi-modal information infused panel features.  
}

\begin{figure*}
    \centering
    \includegraphics[scale=0.18]{img/fig3_v3.pdf}
    % \vspace{-0.2in}
    \caption{(a) {\bf\em Unsupervised cross-modal association} between point-cloud local representation and semantic representation of user's prompt. 
    % Attribute-Part alignment where part-level point cloud embeddings are matched with multi-modal prompt embeddings, 
    (b) {\bf\em Binary panel mask decoder}: Except the standard inference, personalization can be flexibly supported, \eg, using an instruction mask.}
    \label{fig:decoder}
    % \vspace{-0.2in}
\end{figure*}

 
\subsection{Encoder: Multi-modal panel embedding}
\noindent {\bf Point-cloud local representation} \saura{Since a 3D garment is composed of multiple panels, it is useful to align the local patches of point-cloud (corresponding to panel or panel parts) with the panel prompts.} To facilitate panel extraction from 3D point-cloud (without panel-level annotation) in an unsupervised manner, we start by 
local patch analysis as in PointBERT \cite{yu2022point}.
% (corresponding to panels or panel parts).
%
Specifically, given a point cloud $P \in \mathbb{R}^{N \times 3}$, we first sample $g$ center points of $p$ via farthest point sampling (FPS). Then use $k$-nearest neighbor ($k$NN) to associate the nearest points in the point cloud to each center point, resulting in $g$ local patches $\{p_{i}\}_{i=1}^{g}$. We normalize the local patches by subtracting their center coordinates, disentangling the structure patterns and spatial coordinates. For efficiency, we employ a mini-PointNet to embed the patches $F_{loc} \in \mathbb{R}^{g \times D}$ with $D$ the embedding dimension.

\noindent {\bf Point-cloud global representation}
% To obtain attribute-level alignment we require a global representation of the garment point-cloud. 
% \saura{
% Our model reconstructs latent codes for each panel
% in a sewing pattern given the global latent code. This global bottleneck makes the model prone to relying on the overall shape of the garment and less likely to exploit its per-component structure. 
% Since, transformers are good at global feature extraction we }
\saura{
We leverage the overall shape of garment as contextual information for estimating the sewing patterns.}
To that end
we use a PointTransformer ($\mathbb{T}$) \cite{zhao2021point} to extract the global representation of point cloud, as the self-attention has a global view field.
% Formally, the PointTransformer $\mathbb{T}$ (w/o the classification head) has a stack of point-layer blocks, transition-up, transition-down in a U-Net style design.
We first obtain per-point features $F_{p} = \mathbb{T}(P) \in \mathbb{R}^{N \times D}$ with $D$ the feature dimension, then apply positional encoding. 
We finally aggregate $F_{p}$ into a single feature vector $F_{global} = \phi(F_{p})$ by average pooling $\phi(.)$.

% \vspace{-0.15in}
\paragraph{Personalization input representation} 
We consider (not limited to) both text prompt and sketch as the user instruction for personalization.
For text prompt, any prompt encoder (\eg, pre-trained CLIP\cite{radford2021learning} text encoder) can be used. \saura{We use a static prompt template of $Garment+ \{panel class\}$ where the panel class (denoted by $p_{i}$, $i \in \{1,..,K\}$) can be ``skirt-front'', ``jacket-sleeveless'' etc.}
For sketch, we use the SketchRNN \cite{ha2017neural}.
%
Formally, given a panel set $p$ required by the user, \saura{we obtain 512-D textual features (for textual prompt) for
$K$ panels from CLIP-encoder and the mean latent vector from Sketch-RNN's encoder %(for sketch prompt) 
for the
sketches of all the training garments belonging to $K$ panels. We then use a projection MLP to match the dimensions.}
We finally obtain the personalization feature $P_{loc} \in \mathbb{R}^{K \times D}$, where $K$ is the number of panels and $D$ is the feature dimension.

% \vspace{-0.2in}
\subsubsection{Cross-modal local association}
Without ground-truth annotation of panels in a point cloud,
it is necessary to perform cross-modal association between local panel features
$F_{loc}$ and personalization feature $P_{loc}$ representation in an {\em unsupervised} manner.
The goal is to achieve multi-modal information fusion at the panel level
with both high-level semantic information from the personalization input (\eg, text prompt or sketch) and low-level fine-grained geometry information from the point cloud.

In essence, associating $F_{loc}$  and $P_{loc}$ is a set-to-set matching problem. 
% We observe that several local patches of point cloud
% may be associated with a single panel. 
Without pairwise labeling, we adopt the Wasserstein distance between the two sets of feature distributions (see Fig. \ref{fig:decoder}(a)). 
We define the cost of matching as the normalized mean-squared error (MSE) between $P_{loc}$ and $F_{loc}$ \cite{yu2022dual}. 
We denote the cost of moving $P_{loc}^{g}$ to $F_{loc}^{k}$ as $\delta_{g,k} = MSE(\hat{P}^{g}_{loc}, \hat{F}^{k}_{loc})$, where $\hat{P}$/$\hat{F}$ denotes an individual feature from $P$/$F$.
To encourage accurate local association, we minimize the following optimal transport cost:
\begin{align}
    OPT(P_{loc}^{g},F_{loc}^{k}) = \sum_{g=1}^{G}\sum_{k=1}^{K}f_{g,k}\delta_{g,k} \;\; \\ \text{where} 
    \sum_{g=1}^{G}\sum_{k=1}^{K}f_{g,k} = min(\sum_{g=1}^{G}w_{g}^{v},\sum_{k=1}^{K}w_{k}^{v})
\end{align}
where $w_{g}$/$w_{k}$ refers to the moving weight
and $G$/$K$ to the size of $F_{loc}$/$P_{loc}$.
%
To ease optimization, we further derive a proxy normalized loss quantity as:
\begin{equation}
    \mathbb{W}_{D} = \frac{\sum_{g=1}^{G}\sum_{k=1}^{K}f_{g,k}\delta_{g,k}}{\sum_{g=1}^{G}\sum_{k=1}^{K}f_{g,k}}.
\end{equation}
More details are given in the supplementary file.  
% \vspace{-0.2in}
\subsubsection{Multi-modal attentive embedding}
After local alignment between point cloud and personalization input,
we further fuse the information across modalities. \saura{Motivated by the generality of cross-attention \cite{chen2021crossvit}, we leverage transformers \cite{vaswani2017attention} to fuse multi-modal features via cross-attention}.
% This is achieved by leveraging an attention mechanism \cite{vaswani2017attention}.
%
Concretely, each Transformer module consists of a self-attention layer and a feed forward network.
%
We obtain the multi-modal panel embedding $F_{cm}$ via:
\begin{equation}
    F_{cm} = \mathcal{T}_{c}(P_{loc}, F_{loc},F_{loc}) \in \mathbb{R}^{K \times D}, 
\end{equation}
where %$\mathcal{T}_{c}$ is the transformer layer with 
we set the query as $P_{loc}$ and key/value both as $F_{loc}$ respectively.
As a result, each element in $F_{cm}$ is linked particularly with a specific panel class \saura{$p_{i}$ ($i \in \{1,..,K\}$).}
This \saura{disentanglement of panel specific features} facilitates the realization of panel personalization,
as each panel can be manipulated individually.




\begin{table*}[ht]

\centering
\caption{\textbf{Evaluation of personalized pattern design} on Panel IOU for 6 garment transfer cases. x$\shortrightarrow$y indicates before (x) and after (y) personalization. Abbreviations J: Jacket, JP: Jumpsuit, T: Tee, D: Dress, JS: Jacket Sleeveless. }
\setlength\tabcolsep{5pt}
\begin{tabular}{cc|ccccccccc}
\hline
\multicolumn{1}{l|}{\multirow{2}{*}{\bf{Modality}}} & \multirow{2}{*}{\bf{Method}} & \multicolumn{7}{c}{\bf{Panel IOU for personalized edits}}                                                                                                                                                                                                                                                                                                                                                                                                                      \\ \cline{3-9} 
\multicolumn{1}{l|}{}                          &                         & \multicolumn{1}{c|}{\bf{Case 1}}                   & \multicolumn{1}{c|}{\bf{Case 2}}                   & \multicolumn{1}{c|}{\bf{Case 3}}                   & \multicolumn{1}{c|}{\bf{Case 4}}                               & \multicolumn{1}{c|}{\bf{Case 5}}                   & \multicolumn{1}{c|}{\bf{Case 6}}                   &             \\ \hline
\multicolumn{2}{c|}{\bf{Combinations}}                                        & \multicolumn{1}{c|}{\bf{J to T}}                  & \multicolumn{1}{c|}{\bf{T to J}}                  & \multicolumn{1}{c|}{\bf{JP to D}}                  & \multicolumn{1}{c|}{\bf{D to JP}}                   & \multicolumn{1}{c|}{\bf{J to JS}}                 & \multicolumn{1}{c|}{\bf{JS to J}}      & \multicolumn{1}{c}{\bf{Avg}}                    \\ \hline
\multicolumn{1}{c|}{\multirow{2}{*}{Text}}     & Baseline                &\multicolumn{1}{c|}{0.32$\shortrightarrow$0.39}                        &\multicolumn{1}{c|}{0.27$\shortrightarrow$0.40}                        &\multicolumn{1}{c|}{0.18$\shortrightarrow$0.32}                                        &\multicolumn{1}{c|}{0.16$\shortrightarrow$0.31}                        &\multicolumn{1}{c|}{0.11$\shortrightarrow$0.32}                        &\multicolumn{1}{c|}{0.16$\shortrightarrow$0.43}    & \multicolumn{1}{c}{0.20$\shortrightarrow$0.36}                 \\ \cline{2-9} 
                \multicolumn{1}{c|}{}          & Ours                    &\multicolumn{1}{c|}{\cellcolor[HTML]{F6DDCC}0.46$\shortrightarrow$0.52} & \multicolumn{1}{c|}{\cellcolor[HTML]{F6DDCC} 0.41$\shortrightarrow$0.53} &\multicolumn{1}{c|}{ \cellcolor[HTML]{F6DDCC} 0.29$\shortrightarrow$0.51} &\multicolumn{1}{c|}{\cellcolor[HTML]{F6DDCC} 0.19$\shortrightarrow$0.48}&\multicolumn{1}{c|}{\cellcolor[HTML]{F6DDCC} 0.25$\shortrightarrow$0.54} &\multicolumn{1}{c|}{\cellcolor[HTML]{F6DDCC} 0.25$\shortrightarrow$0.60}  &\multicolumn{1}{c}  {\cellcolor[HTML]{F6DDCC} 0.31$\shortrightarrow$0.53}     \\ \hline
\multicolumn{1}{c|}{\multirow{2}{*}{Sketch}}   & Baseline                & \multicolumn{1}{c|}{0.29$\shortrightarrow$0.33}                        & \multicolumn{1}{c|}{0.24$\shortrightarrow$0.32}                         & \multicolumn{1}{c|}{0.15$\shortrightarrow$0.39}                                      & \multicolumn{1}{c|}{0.12$\shortrightarrow$0.37}                        & \multicolumn{1}{c|}{0.11$\shortrightarrow$0.34}                        & \multicolumn{1}{c|}{0.18$\shortrightarrow$0.40}   &    \multicolumn{1}{c}  {0.19$\shortrightarrow$0.36}                        \\ \cline{2-9} 
\multicolumn{1}{c|}{}                          & Ours                    & \multicolumn{1}{c|}{\cellcolor[HTML]{F6DDCC} 0.45$\shortrightarrow$0.52}                        & \multicolumn{1}{c|}{\cellcolor[HTML]{F6DDCC} 0.41$\shortrightarrow$0.52}                          & \multicolumn{1}{c|}{\cellcolor[HTML]{F6DDCC} 0.28$\shortrightarrow$0.51}                        & \multicolumn{1}{c|}{\cellcolor[HTML]{F6DDCC} 0.18$\shortrightarrow$0.46}                                      & \multicolumn{1}{c|}{\cellcolor[HTML]{F6DDCC} 0.25$\shortrightarrow$0.55}                        & \multicolumn{1}{c|}{\cellcolor[HTML]{F6DDCC} 0.27$\shortrightarrow$0.56}      &    \multicolumn{1}{c}  {\cellcolor[HTML]{F6DDCC} 0.31$\shortrightarrow$0.52}                                       \\ \hline
\end{tabular}

\label{tab:personalization}
\end{table*}

\begin{figure*}[t]
    \centering
    \includegraphics[scale=0.52]{img/fig5_v3.pdf}
    \caption{\ann{Examples of garment class transfer cases. Given a 3D source garment, we use the desired panel attributes to transfer it to the target garment class. (a) Case 1: Jacket to Tee by text prompt (target garment's panel classes), 
    (b) Case 2: Tee to Jacket by text prompt, 
    (c) Case 3: Jumpsuit to Dress by sketch prompt (target garment's average panel silhouettes), and (d) Case 4: Dress to Jumpsuit by sketch prompt. 
    The topology changes of panel are highlighted.
    % Highlighted panels denote panels having topology changes from
% prediction.
}}
    \label{fig:editing}
    % \vspace{-0.5cm}
\end{figure*}
% \vspace{-0.05in}
\subsection{Decoder: Panel mask prediction}
% \saura{%Our objective is to obtain a 
% The panel-level decomposition of a given 3D garment forms the basis of our panel latent embedding based editing. 
% Previous works \cite{hertz2022spaghetti} have shown the ability by conditioning the shape generation on the parts partitioning, and their manipulation. Similarly, PersonalTailor utilizes this ability to compose different panel combinations to produce customized or standard garment designs.}
\saura{The panel-level decomposition of a given 3D garment forms the basis of our panel embedding based editing. This is similar in spirit with
shape generation by parts partitioning \cite{hertz2022spaghetti}.}
Given per-panel multi-modal embedding $F_{cm}$,
we predict the panel masks along with the stitching using a Transformer decoder $\mathcal{C}$ \cite{vaswani2017attention}.
Specifically, to exploit the panel's spatial information, we append positional encoding to $F_{cm} \in \mathbb{R}^{K \times D}$ with $K$ the number of panels. 
We set this embedding as the queries $Q$ of $\mathcal{C}$. 
We then apply self-attention on $F_{cm}$ for local interaction,
followed by cross-attention with the global feature $F_{global}$
to obtain the final panel-specific representation:
% \vspace{-0.05in}
\begin{equation}
    F_{comp} = \mathcal{C}(F_{global};Q) \in \mathbb{R}^{K \times D}.
\end{equation}

\noindent {\bf Prediction heads} For efficiency, three lightweight heads are built to decode $F_{comp}$. 
The \emph{garment placement head} outputs the stitching information per panel. This is realized by training an MLP to output the rotation $\hat{r} \in \mathbb{R}^{M \times 3}$ and translation $\hat{t} \in \mathbb{R}^{M \times 3}$:
\begin{equation}\footnotesize
    \hat{r} = \sigma(Pool(W_{r}*F_{comp})), \hat{t} = \sigma(Pool(W_{t}*F_{comp}))
\end{equation}
where $W_{t}/W_{r} \in \mathbb{R}^{D \times 3}$ denotes the weights
$pool$ the average pooling operation. 
The \emph{panel confidence head} predicts the confidence of each panel. It is realized by another MLP followed by sigmoid operation:
\begin{equation}
    \hat{p}_{c} = \sigma(W_{p_{c}}*F_{comp}) \in \mathbb{R}^{D \times 1}
\end{equation}
where $W_{p_{c}}$ is the weight. 
The \emph{panel mask head} first unsamples each query in $F_{comp}$ to a fixed dimension binary mask and then applies sigmoid as 
\begin{equation}
   \hat{m}= \sigma(\psi(F_{comp})) \in \mathbb{R}^{H \times W}
\end{equation}
where $\psi(.)$ is a series of up-convolution followed by ReLU except the last layer.
\saura{More specifically, given a 3D garment point-cloud, we predict the 2D binary panel masks $\hat{m}$ per query
position using the mask head. We then obtain most confident ones (predicted by panel
confidence head $\hat{p}_{c}$, thresholded at 0.5).
We select top-$k$ panels among the most confident panel masks ($k=14$ is set
empirically). The placement head $\hat{r} / \hat{t}$ is used for draping the garment.}

\noindent \textbf{Panel mask smoothing} 
The sewing panels of garments typically present smooth outlines \cite{korosteleva2022neuraltailor}.
To exploit this prior, we reformulate the predicted panel mask as a closed piece-wise curve with every piece (edge) constrained to be Bezier spline. 
% To understand the predicted panel mask and interpret the panel edge sets, 
Given the predicted mask $\hat{m}_{i}$,
we estimate the 2D starting vertex $\hat{v}_{i}$ and the curvature $\hat{c}_{i}$ of a panel edge as:
\begin{equation}
    \{\hat{v}_{i},\hat{c}_{i}\} := \tau(\mathcal{B}(\hat{m}_{i}))
\end{equation}
where $\mathcal{B}(\cdot)$ denotes VGG net with an MLP classifier followed by tanh activation $\tau(\cdot)$. 




\subsection{Model training}

% \noindent \textbf{Learning objective} 
% The panel mask head is a binary class prediction problem. 
%
%Following \cite{korosteleva2022neuraltailor} 
We use the dataset $D_{train}$ to train the model. Model input includes the point set, and the user instruction in form of the text of the ground-truth panel classes (\ie, textual prompt) % as text prompts, and
or the silhouettes of the ground-truth panels (\ie, visual prompt).
% for the sketch prompt. 
\saura{Since the user instruction is based on panel positions, we activate the ground-truth panel positions and set null (zero) vectors at other positions. At inference, this enables the removing-panel function.}
Model output includes the 2D panels and sewing patterns.
We adopt the MSE loss ($L_{place}$) for 
training the \emph{garment placement head}. 
%
To train the \emph{panel confidence head}, we assign each ground-truth panel position $j$ as $p_{h}(j) = 1$, otherwise 0. As panel position prediction is a multi-class multi-label problem, we use binary cross entropy loss ($\mathcal{L}_{conf}$). 
%
We train the \emph{panel mask head} using a binary cross-entropy loss ($\mathcal{L}_{mask}$). 
We assign each ground-truth panel position $j$ with its corresponding mask $m_{j} \in \mathbb{R}^{H \times W}$ to $p_{g}(j)$, otherwise an empty mask.
%
We use $L_1$ regression loss ($\mathcal{L}_{con}$) to estimate the curvature and vertex positions. 
%
We additionally use the local-association loss
($\mathcal{L}_{asso} = \mathbb{W}_{D}$)
% ($\mathcal{L}_{asso}$) 
to learn the cross-modal association. 
% which is defined by $\mathcal{L}_{asso} = \mathbb{W}_{D}$. 
The overall objective is the sum of all above loss terms.







% \vspace{-0.05in}
\subsection{Model inference and personalization}

Our model can support both standard and personalized pattern design.
% For both cases, given an unseen point cloud and its garment class, we output top-$k$ most confident panel masks.
The difference between the two settings 
lies in how to set the mask instruction $M$.
In the standard setting, 
given an unseen point cloud, we activate all the panels of $M$.
\saura{Although the training and testing garments share the same panel class set $p$, the panel combination that forms unseen garments is unknown.} Thus, for the textual prompt we use all $M$ panel classes ; For the visual prompt, we use the mean sketchRNN \cite{ha2017neural} embedding (\ie, the prototype of each of $M$ panels in all training classes). We output the top-$k$ most confident panel masks above a fixed threshold.
For {\em personalization}, we activate only the panels specified in the user's instruction mask \saura{(\ie, passing null vectors at other positions)} and output their 2D panels as shown in Fig~\ref{fig:decoder}(b).
%\arik{>>> no use of sketches in personalization?}

% % \vspace{-0.1in}
\subsection{Garment stitching}
With the panel masks predicted,
we can further infer the stitching information for edge sewing across the panels. To that end, we design a StitchGraph module leveraging a GNN $\mathcal{G}$ \cite{yang2021sketchgnn}. 
Given a set of panel vertices $\hat{v}_{i}$, panel curvature $\hat{c}_{i}$ and placement information ($\hat{r}_{i}$, $\hat{t}_{i}$) of the panel mask $\hat{m}_{i}$,
we predict the stitching signal $\hat{s}$:
\begin{equation}
    \hat{s} = \mathcal{G}(\hat{v}_{i},\hat{c}_{i}) 
\end{equation}
where the value of ``1'' indicates the two edges stitched and ``0'' otherwise. 
We train $\mathcal{G}$ by a binary cross entropy loss $\mathcal{L}_{stitch}$.
This stitching signal coupled with panel placement information $\hat{r}_{i},\hat{t}_{i}$ is used for draping the garment on to the human body. \saura{Note, we follow the same stitching evaluation setting (\eg, no optimization of the stitching parameters in 3D space for different body shapes and sizes) as NeuralTailor \cite{korosteleva2022neuraltailor} for facilitating comparison.}
% , thus we do not optimize the stitching parameters in 3D space based on different body shapes and size.}









\section{Experiments}



\begin{table*}[t]
\small
\centering
\caption{\textbf{Evaluation of panel-prediction quality} on seen and unseen garment classes. M-L2: Mask L2 ; P-L2 : Panel L2; R-L2: Rotation L2; T-L2: Translation L2 . $\dagger$ represents orderless-LSTM.} %$*$ denotes the models without data-filtering.}
\setlength{\tabcolsep}{8pt}
\begin{tabular}{@{}c|ccccc|ccccc@{}}
\toprule
                                   & \multicolumn{5}{c|}{\textbf{Seen classes}}                                                                                                                                                                                                      & \multicolumn{5}{c}{\textbf{Unseen classes}}                                                                                                                                                                                                     \\ \cmidrule(l){2-11} 
\multirow{-2}{*}{\textbf{Methods}}& \textbf{P-L2}                     & \textbf{\# Panels}                    & \textbf{\# Edges}                     & \textbf{R-L2}                       & \textbf{T-L2}           & \textbf{P-L2}                     & \textbf{\# Panels}                    & \textbf{\# Edges}                     & \textbf{R-L2}                       & \textbf{T-L2}                     \\ \midrule
Baseline-I                 & 3.92                                    &  \bf 99.9\%                                    &\bf 100.0 \%                                    & 0.06                             & 0.117                                       & 6.61                                    & 94.6\%                                     &  95.4\%                                     & 0.09                                    & 0.21                                    \\
Baseline-II                                                   & 4.3                                   & 99.4\%                                   &   99.7\%                                        & 0.08                                   & 1.46                                                             & 8.1                                   & 89.3\%                                   & 90.3\%                                   & 121                                   & 1.25                                   \\
%Baseline-III                                                 & 3.91                                   & 99.9 \%                                  & 99.9 \%                                    & 0.06                                   & 0.05                                                            & 6.3                                   & 93.9  \%                                    & 94.2 \%                                    & 0.07                                   & 0.18                                   \\
LSTM                                                       & 2.71                                  & 99.8\%                                & 99.9\%                                & \bf 0.004                                 & 0.32                                                                 & 14.7                                  & 6.5\%                                 & 53.2\%                                & 0.17                                  & 6.75                                  \\
LSTM$^{\dagger}$                                                    & 2.87                                  & 99.4\%                                & 99.9\%                                & \textbf{0.004}                                 & 0.33                                                                   & 12.94                                 & 2.7\%                                 & 59.0\%                                & 0.16                                  & 7.18                                  \\
Neural-Tailor                                                      & \textbf{1.5}                                   & 99.7\%                                & 99.7\%                                & 0.04                                  & 1.46                                                         & 5.2                                   & 83.6\%                                & 87.3\%                                & 0.07                                  & 3.22                                  \\
% Neural-Tailor*                                                    & 1.53                                  & 98.8\%                                & 99.6\%                                & 0.04                                  & 1.45                                                        & 7.96                                  & 73.1\%                                & 80.5\%                                & 0.08                                  & 3.57                                  \\
% Neural-Tailor*                                                    & 1.6/1.95                              & 98.6/97.5\%                           & 99.8/99.2\%                           & 0.07/0.07                             & 2.2/2.5                                                             & 6.2/6.4                               & 81.6/75.2\%                           & 88.5/88.2\%                           & 0.08/0.10                             & 3.9/4.5                               \\ \midrule
\midrule
\textbf{Ours w/ Text}                      & \cellcolor[HTML]{FFFFDB}2.80  & \cellcolor[HTML]{FFFFDB}\textbf{99.9\%}  & \cellcolor[HTML]{FFFFDB}{99.9\%}  & \cellcolor[HTML]{FFFFDB}0.04  & \cellcolor[HTML]{FFFFDB}\textbf{0.04}    & \cellcolor[HTML]{FFFFDB}\textbf{4.20}  & \cellcolor[HTML]{FFFFDB}\textbf{99.9\%}  & \cellcolor[HTML]{FFFFDB}\textbf{99.8\%}  & \cellcolor[HTML]{FFFFDB}\textbf{0.05}  & \cellcolor[HTML]{FFFFDB}\textbf{0.05}  \\
\textbf{Ours w/ Sketch}                      & \cellcolor[HTML]{FFFFDB}2.91  & \cellcolor[HTML]{FFFFDB}\textbf{99.9\%}  & \cellcolor[HTML]{FFFFDB}{99.9\%}  & \cellcolor[HTML]{FFFFDB}0.05  & \cellcolor[HTML]{FFFFDB}{0.06}    & \cellcolor[HTML]{FFFFDB}\textbf{4.33}  & \cellcolor[HTML]{FFFFDB}\textbf{99.9\%}  & \cellcolor[HTML]{FFFFDB}\textbf{99.9\%}  & \cellcolor[HTML]{FFFFDB}\textbf{0.06}  & \cellcolor[HTML]{FFFFDB}\textbf{0.07}  \\
%\textbf{Ours w/ Overlap}                      & \cellcolor[HTML]{FFFFDB}2.80  & \cellcolor[HTML]{FFFFDB}\textbf{99.9\%}  & \cellcolor[HTML]{FFFFDB}\textbf{99.9\%}  & \cellcolor[HTML]{FFFFDB}0.04  & \cellcolor[HTML]{FFFFDB}\textbf{0.04}    & \cellcolor[HTML]{FFFFDB}\textbf{4.20}  & \cellcolor[HTML]{FFFFDB}\textbf{99.9\%}  & \cellcolor[HTML]{FFFFDB}\textbf{99.8\%}  & \cellcolor[HTML]{FFFFDB}\textbf{0.05}  & \cellcolor[HTML]{FFFFDB}\textbf{0.05}  \\

 \bottomrule 
\end{tabular}

\label{tab:main_tab}
\end{table*}

\begin{figure*}[t]
    \centering
    \includegraphics [width=\linewidth]{img/fig4_v2.pdf}
    \caption{
    Comparing our method with NeuralTailor (\texttt{NT})
    on the unseen garment classes: ‘jacket sleeveless’, ‘skirt waistband’, ‘wb jumpsuit sleeveless’ and ‘dress’.
    {\em Metric}: the average Vertex L2 error.
    {\em Ground-truth}: dash thin lines.
    % Comparison of our method with NeuralTailor (\texttt{NT}) \cite{korosteleva2022neuraltailor} on the unseen garments from ‘jacket sleeveless’, ‘skirt waistband’, ‘wb jumpsuit sleeveless’ and ‘dress’ categories of the dataset \cite{korosteleva2021generating}. The numbers show the average Vertex L2 for the shown exemplars. The colored panels indicate predicted panels, and the dash thin lines indicate the ground-truth panels.
    }
    \label{fig:main_viz}
    % % \vspace{-0.2in}
\end{figure*}
\noindent \textbf{Dataset}
We evaluate the PersonalTailor on the 3D garments dataset with sewing patterns from \cite{korosteleva2021generating}. It contains 19 garment classes with $22,000$ 3D garment-sewing pattern pairs in total, covering the variations in t-shirts, jackets, pants, skirts, jumpsuits and dresses. 
There are 10627/722/729 samples for train/val/test
% The number of samples in train/val/test is 10627/722/729 
in the filtered version. 
Following NeuralTailor \cite{korosteleva2022neuraltailor}, the classes of panels are designed based on the panel's role and location around the body across all garment classes. For example, panels located around the back of human body are grouped in the ``back panels'' class. We follow the standard panel labels, data filtering and train/test splits of garment classes. There are 7 garment classes unseen to training and used for evaluation. 



\noindent \textbf{Evaluation metrics}
We use the same evaluation metrics as in \cite{korosteleva2022neuraltailor}. 
We evaluate the accuracy in predicting the number of panels within
every pattern (\ie, \#Panels) and the number of edges within every panel
(\ie, \#Edges). To estimate the quality of panel shape predictions, we use the average distance (L2 norm) between the vertices of predicted and ground
truth panels with curvature coordinates converted to panel space,
acting as panel masks in this comparison (Panel L2). Similarly,
we report L2 normalized differences of predicted panel rotations
(Rot L2) and translations (Transl L2) with the ground truth. The
quality of predicted stitching information is measured by a mean
precision (Precision) and recall (Recall) of predicted stitches.




\noindent \textbf{Implementation details}
For language encoding, we use CLIP \cite{radford2021learning} pretrained encoder. 
For sketch encoding, we use SketchRNN \cite{yang2021sketchgnn}. 
We follow the training scheme as \cite{korosteleva2022neuraltailor}.
We set the maximum number of panels $M=23$.
There are $g=12/8$ garment classes in training/testing set. We set the feature dimension for text and the global embedding $D = 512$. % is set as 512. 
% The number of codes $K$ in codebook is set as 2000. 
% For Stage-1 training, 
Our model is trained for 250 epochs using Adam optimizer with learning rate of $10e-5$ and batch size of 15. 
% For Stage-2 training, our model
The stitching GNN is trained for 50 epochs using SGD optimizer with learning rate of $10e-4$.
%
% Specifically, %We train our stitch prediction network
% it is trained by the edge vectors from ground truth panels and edges outputted by the prediction module under the text prompt scenario. 
%
Specifically, %We train our stitch prediction network
it is trained by the predicted edges.
%
The inference threshold for panel mask head is set as 0.5 and top-$k$ is set as 14. The code will be made publicly available upon acceptance.
% Our model is implemented in Pytorch and trained with batch size of 15 on a single NVIDIA 2080GTX GPU.

% \begin{figure}[t]
%     \centering
%     \includegraphics[scale=0.35]{img/final_model.png}
%     \caption{\textbf{Examples of garment personalization} 
%     % Based on user input via sketch/text prompt, we illustrate the customization 
%     %
%     % Garment personalization 
%     from (a) Pant Straight sides to Skirt 4 Panels, (b) Skirt 4 Panels to Pant Straight Sides, (c) Dress Sleeveless to Dress Waistband Sleeveless, (d) Dress Waistband Sleeveless to Dress Sleeveless respectively. }
%     \label{fig:customized}
% \end{figure}


\subsection{Personalized pattern design evaluation}
\noindent \textbf{Setting}  To quantitatively evaluate the performance of personalization,
% based on the user input prompts (\ie, text and sketch), 
we conduct 6 garment class transfer cases (case 1\&2: Tee $\leftrightarrow$ Jacket, case 3\&4: Jumpsuit$\leftrightarrow$ Dress, case 5\&6: Jacket $\leftrightarrow$ Jacket Sleeveless)
under both text and sketch prompt. We define the \textit{Panel IOU}  metric as the mean of panel-wise IOUs between predicted panels of the source garment class and the average panels of the target garment class. Formally, we use the desired input prompts to transfer the source garment class to the target garment class. Then we compare the \textit{Panel IOU} before and after personalization against the target class panel attributes. % in personalized query. 

\noindent \textbf{Baseline} Due to lacking of competing works or open-source alternatives, % in the literature, 
% we created our own baselines. More specifically, 
we created a personalization baseline by removing the prompt embedding and cross-modal embedding module (referred as \texttt{baseline}) from our PersonalTailor. 

\noindent \textbf{Quantitative results} 
The personalization results are reported in Tab.~\ref{tab:personalization}. It can be observed that (1) our method can achieve an average panel IOU of $53\%$ over 6 cases by text and $52\%$ by sketch, outperforming the baseline method by $13\%$/$16\%$ respectively. This is because the decoder of the baseline is randomly initialized lacking the semantic and structural information of the panel attributes. Thus, it has less personalization ability. (2) Our method yields a larger gain over the baseline before and after personalization under both text ($22\%$ \vs $16\%$) and sketch prompts  $21\%$ \vs $17\%$). 
This verifies our superior personalization ability.
% of our model design.
% This showcase that our method has better personalization ability.



\noindent \textbf{Qualitative/visual results}
We show the personalized garment transfer process of case 1\&2 by text prompt (target garment’s panel classes) in Fig.~\ref{fig:editing} (a,b), case 3\&4 by sketch prompt (target garment’s average panel silhouettes) in Fig.~\ref{fig:editing} (c,d).  Overall, it is shown that our method can support panel shape editing with complex topology changes from one garment class to another using personalized prompts, even for those unseen during training, \eg, Jumpsuit and Dress. Beyond topology change, it also supports adding 
new panels (Fig.~\ref{fig:teaser} (b)), removing panels (Fig.~\ref{fig:teaser} (c)), and creating a
new design % that is not included in the dataset 
(Fig.~\ref{fig:teaser} (e)).
We also observe that our method can achieve fine-grained panel shape editing by using sketch prompts. As shown in Fig.~\ref{fig:sketch_edit}, given a 3D jacket and different users' sketch prompts, our method can produce the panels aligned with the sketch's shape while preserving the intrinsic structure of the 3D shape. 
% We provide more illustration of personalized garment transfer in Fig.~\ref{fig:customized}. 
%\paragraph{Controllable garment personalization} \annie{figure 4 and 1} We demonstrate our framework is capable of controllable garment personalization. Given a source 3D garment, our PersonalTailor can accurately edit the 2D sewing panel shapes. Additionally, our model can also perform personalization by design choice during inference. From the results in Fig~\ref{fig:editing}, we can observe that PersonalTailor supports controllable editing on the 3D garment shape and topology with
%preserved intrinsic structure. And PerosnalTailor can edit garments with significant shape variations or transfer garments from one category to another by editing on the 2D panels via mask instructions, even for some categories that are not shown in the training set. As an added benefit, our network facilitates both text and sketch based editing to give more expressive power to the users.








% \begin{figure*}
%     \centering
%     \includegraphics[scale=0.43]{img/PTailor_main.png}
%     \caption{\textbf{Examples of PersonalTailor's output (unrefined)}
%     It is shown that our method works similarly well with text and sketch/visual prompts.
%     % As seen from the figure, both text and sketch prompt predicts similar mask prediction with textual prompt marginally better. 
%     }
%     \label{fig:ptailor_main}
% \end{figure*}

\begin{figure}[t]

    \centering
    \includegraphics[scale=0.32]{img/new_fig_6.png}
    % % \vspace{-0.1in}
    \caption{\ann{Illustration of fine-grained panel editing by sketch. Given a 3D garment and different users' sketches, our method can support fine-grained panel shape editing while preserving the intrinsic structure of the 3D garment.}}
  \label{fig:sketch_edit}
  % % \vspace{-0.2in}
\end{figure}
\subsection{Standard pattern design evaluation}

\noindent \textbf{Setting} In this setting, we evaluate the standard (non-personalized) pattern design.  
% We test the open-set scenario where the training and testing garment classes do not overlap, \ie $g_{train} \cap g_{test} = \phi$, whilst the panel classes may overlap $p_{train} \cap p_{test} \neq \phi$. 
We follow the same setting and dataset splits as proposed in NeuralTailor \cite{korosteleva2022neuraltailor}. More specifically, we evaluate on two settings:
\ann{(1) Training with seen classes and evaluating on unseen data of those seen classes, \ie closed-set setting; 
(2) Training with seen classes and evaluating on unseen classes, \ie open-set setting.} 

\noindent \textbf{Competitors} We considered the following competitors for comparison: 
(a) a competitive garment pattern prediction method Neural Tailor on filtered data \cite{korosteleva2021generating}, 
(c) an LSTM \cite{graves2012long} based garment pattern prediction ,
(d) an orderless LSTM \cite{yazici2020orderless} based garment pattern prediction, 
(e) \textit{Baseline-I} we created using GCN encoder and CNN decoder, 
(f) \textit{Baseline-II} we created using PointTransformer encoder \cite{zhao2021point} and Transformer decoder \cite{vaswani2017attention} with random initialized queries. %(g) a baseline termed as \textit{Baseline-III} created using GCN encoder and Transformer decoder without language. We evaluate all of the above techniques in a similar setup. 

\noindent \textbf{Results} The results are reported in Tab.~\ref{tab:main_tab}. 
\textbf{(I) Closed-set settings:} 
% Under this setting, t
The performance of some metrics (\eg, Edges/Panels) has almost saturated.
% by different methods.
In particular, NeuralTailor has the best Panel L2 result, indicating that learning vertex is better in the closed set than mask prediction.
However, our PersonalTailor achieves the best translation prediction, suggesting the importance of global information.
%
% In this settings, our PersonalTailor has near-to 100\% edge and panel accuracy indicating the superior model design. It is interesting to note that PersonalTailor has almost 8x better translation metric indicating the importance of global information. NeuralTailor has the best Panel L2 indicating that learning vertex has better generalization in the closed set than mask prediction.
\textbf{(II) Open-set settings:} %In contrast to the closed-set setting, 
Our method achieves the state-of-the-art in all the metrics, surpassing the competitors by a large margin. This indicates the superiority of \ann{personalized prompts} in open-set generalization.
% of class agnostic masks. %Additionally, we have tested our method on the non-overlapping setting which can be found in the supplementary file.\annie{check}.



% % \vspace{-0.1in}
\paragraph{Qualitative results} 
% We first present qualitative results for our PersonalTailor in Fig.~\ref{fig:main_viz},\ref{fig:ptailor_main} whilst comparing to our closest competitor NeuralTailor \cite{korosteleva2022neuraltailor} with unseen garments. 
We present qualitative results on unseen garments. 
It can be observed from Fig.~\ref{fig:main_viz} that our PersonalTailor predicts more accurate panels over the prior art NeuralTailor \cite{korosteleva2022neuraltailor}, due to our multi-modal embedding-based design enabling the prompt bring in additional semantic information about the garment's shape.  
We also show in Fig. \ref{fig:ptailor_main} that PersonalTailor works well with both text and sketch prompts.
% Our findings are also reflected in the Panel L2 metric for each garment where our model is superior in open-set scenario. %Additionaly, our model can also handle overlapping garment problem as seen in Fig~\ref{fig:overlap} where our model has significant better mask prediction than NeuralTailor.

\begin{figure}[t]
    \centering
    \includegraphics[scale=0.35]{img/final_model.png}
    \caption{\textbf{Examples of garment personalization} 
    % Based on user input via sketch/text prompt, we illustrate the customization 
    %
    % Garment personalization 
    from (a) Pant Straight sides to Skirt 4 Panels, (b) Skirt 4 Panels to Pant Straight Sides, (c) Dress Sleeveless to Dress Waistband Sleeveless, (d) Dress Waistband Sleeveless to Dress Sleeveless respectively. }
    \label{fig:customized}
\end{figure}





\subsection{Stitching prediction} 

We evaluate the stitching module design
by comparing with NeuralTailor \cite{korosteleva2022neuraltailor}.
As shown in Tab.~\ref{tab:stitch}, 
our GNN based design is clearly superior particularly
for unseen garment classes.
This validates the efficacy of our exploiting the structural information of panels.
%
We also show that using the edge vectors of ground-truth panels
for training is inferior than using the predicted for both methods,
as the former introduces some inconsistency with model inference.


% our stitch prediction network trained on the edges predictions outperforms the second best $3.3/0.5$ and $9.0/0.3$ points on Precision and Recall for Seen Type and Unseen Type respectively.} And the models  trained by edge prediction data (OurPred, NeuralTailorPred) perform better than that of ground truth edge vectors  (OurGT, NeuralTailorGT). The noised prediction data empowers the generalization ability of the networks.


\begin{table}[h]
\caption{\textbf{Evaluation of stitching prediction} on both seen and unseen garment classes.
$^*$: Trained by the edges of GT panels.
}
\label{tab:stitch}
\resizebox{\columnwidth}{!}{
\begin{tabular}{c|cc|cc}
\hline
\multirow{2}{*}{\textbf{Method}}                 & \multicolumn{2}{c|}{\textbf{Seen classes}}     & \multicolumn{2}{c}{\textbf{Unseen classes}}   \\ \cline{2-5}
                & \textbf{Precision}       & \textbf{Recall}          & \textbf{Precision}       & \textbf{Recall}          \\ \hline
NeuralTailor$^*$ \cite{korosteleva2022neuraltailor}  & 96.6\%          & 88.6\%          & 75.3\%          & 60.6\%          \\ \hline
NeuralTailor \cite{korosteleva2022neuraltailor} & 96.3\%          & 99.4\%          & 74.7\%          & 83.9\%          \\ \hline
Ours$^*$            & 74.9\%          & 65.0\%          & 76.8\%          & 73.0\%          \\ \hline
Ours          & \cellcolor[HTML]{F6DDCC}\textbf{99.9\%} & \cellcolor[HTML]{F6DDCC}\textbf{99.9\%} & \cellcolor[HTML]{F6DDCC}\textbf{85.8\%} & \cellcolor[HTML]{F6DDCC}\textbf{84.2\%} \\ \hline
\end{tabular}
}
\label{stitchingres}
\end{table}






\begin{figure*}
    \centering
    \includegraphics[scale=0.43]{img/PTailor_main.png}
    \caption{\textbf{Examples of PersonalTailor's output (unrefined)}
    It is shown that our method works similarly well with text and sketch/visual prompts.
    % As seen from the figure, both text and sketch prompt predicts similar mask prediction with textual prompt marginally better. 
    }
    \label{fig:ptailor_main}
\end{figure*}
\subsection{Ablation studies}
We conduct ablation studies to provide insights into each
component with text prompt. \saura{More in-depth ablations are provided in the \texttt{Supplementary}.}

\noindent \textbf{Impact of cross-modal alignment}
We evaluate the importance of cross-modal alignment.
To that end, we consider two down-stripped designs:
{\bf(1)} Removing the cross modal local association block (CMLA);
{\bf(2)} Removing the multi-modal transformer (MMT).
As shown in Tab.~\ref{tab:mmemb},
we find that without CMLA, a significant drop in the Panel L2 metric 
occurs, suggesting the importance of resolving the domain gap between the point cloud and semantic information from the prompt.
It is also shown that MMT is useful in terms of 
fusing information.

\begin{table}[h]
\centering
\small
\caption{Impact of multi-modal embedding.
CMLA: Cross modal local association;
MMT: Multi-modal transformer.
}
\label{tab:mmemb}
\setlength{\tabcolsep}{3pt}
\begin{tabular}{c|cc|cc}
\hline
\multirow{2}{*}{\textbf{Model}} & \multicolumn{2}{c|}{\textbf{Seen}}     & \multicolumn{2}{c}{\textbf{Unseen}}    \\ \cline{2-5} 
                                & \textbf{Panel L2} & \textbf{\# panels} & \textbf{Panel L2} & \textbf{\# panels} \\ \hline
 Ours                            & \cellcolor[HTML]{F6DDCC}\textbf{2.80}                & \cellcolor[HTML]{F6DDCC} \textbf{99.9\%}                 & \cellcolor[HTML]{F6DDCC}\textbf{4.20}               & \cellcolor[HTML]{F6DDCC}\textbf{99.9\%}  \\ \hline
w/o CMLA                         & 5.51               & 99.8\%                & 6.5                & 99.2\%                   \\
w/o MMT                         & 4.42                &  99.9\%                  & 5.32     
& 99.6\%                  \\ 
\hline


% \hline
 % \\ 
\hline
% w/o both                         & 41                & 42                 & 43   
% & 44 \\

\end{tabular}
% % \vspace{-0.2in}
\end{table}
% as the raw prompt features lack the power of expressively to predict the panel mask.

% in the pattern prediction performance in Tab.~\ref{tab:mmemb}. We first removed the cross modal local association block (CMLA) from our model and observed a significant drop of 2.7\%/2.3\% in the Panel L2 metric for Seen and Unseen setting respectively. This signifies the importance of resolving the domain gap between the point cloud and semantic information from the prompt. Once we remove the multi-modal transformer (MMT) we observe a drop of 1.6\%/1.1\% in the Panel L2 metric for Seen and Unseen setting respectively. This indicates that the raw prompt features lack the power of expressivity to predict the panel mask.



\noindent \textbf{Impact of point-cloud encoders} We evaluate our point-cloud encoder design including 
global encoder (GE) and local encoder (LE).
As shown in Tab.~\ref{tab:ptc}, we see that 
{\bf(1)} GE is useful particularly for unseen garment classes.
This is because global information plays an important role in estimating the cutting pattern and guiding the panel-mask decoder prediction.
% during cross-attention.
{\bf (2)} LE brings in further performance gain
due to extra part-level information introduced.
%
% ({\bf \color{red} W/O LE giving better Panel L2 on Unseen? Double check!})
% experimentally prove the necessity of our point cloud encoders in Tab.~\ref{tab:ptc}. Our variant without global encoder (GE) performs the worst by a margin of 0.6\%/1.3\% in Panel L2 metric for Seen and Unseen setting respectively. This is expected as global information plays an important role in estimating the cutting pattern and also guides the panel-mask decoder prediction using cross-attention. It additionally also affects the panel mask accuracy shown by a performance drop of 4.7\% in \# panels metric for Unseen setting  without global encoder (GE). The variant without local encoder (LE) however observes a slight drop of 0.5\%/0.7\% indicating fine-grained local structure is also important for the overall performance improvement. 






\begin{table}[t]
\centering
\small
\setlength{\tabcolsep}{3pt}
\caption{Impact of global and local point cloud encoders.
LE: Local Encoder; GE: Global Encoder.}
\label{tab:ptc}
\begin{tabular}{cc|cc|cc}
\hline
\multicolumn{2}{c|}{\textbf{Model} }      & \multicolumn{2}{c|}{\textbf{Seen}} & \multicolumn{2}{c}{\textbf{Unseen}} \\
\hline
 \textbf{LE}     & \textbf{GE}                       & \textbf{Panel L2}    & \textbf{\# Panels}      & \textbf{Panel L2}     & \textbf{\# Panels}     \\ \hline
\xmark          & \cmark                         & 3.20          & 99.4\%            & 4.90               & 95.2\%            \\
\cmark         & \xmark         & 3.30          & 99.9\%            & 5.51               & 96.3\%      \\
\hline
 \cmark        & \cmark    & \cellcolor[HTML]{F6DDCC}\textbf{2.80} & \cellcolor[HTML]{F6DDCC}\textbf{99.9\%} & \cellcolor[HTML]{F6DDCC}\textbf{4.20}  & \cellcolor[HTML]{F6DDCC}\textbf{99.9\%} \\ \hline
\end{tabular}
\end{table}



\noindent \textbf{Design choice of panel decoder}
We evaluate more choices of panel mask decoder
including (1) a CNN and (2) a Transformer decoder without positional embedding (Trans. w/o PE). 
We observe in Tab.~\ref{tab:dec} that (1) CNN is least performing for instruction based mask prediction as it loses out on the panel prediction performance due to lack of interaction among the panels. 
(2) Positional encoding is important as it predicts position specific masks.

\begin{table}[h]
\centering
\setlength{\tabcolsep}{3pt}
\caption{Ablation on the design choice of decoder.
Trans.: Transformer; PE: Positional Encoding.
}
\label{tab:dec}
\begin{tabular}{c|cc|cc}
\hline
\multirow{2}{*}{\textbf{Model}}       & \multicolumn{2}{c|}{\textbf{Seen}} & \multicolumn{2}{c}{\textbf{Unseen}} \\ \cline{2-5} 
                             & \textbf{Panel L2}    & \textbf{\# Panels}      & \textbf{Panel L2}     & \textbf{\# Panels}     \\ \hline
CNN                          & 5.49          & 93.2\%          & 6.92           & 91.7\%          \\
Trans. w/o PE        & 3.30          & 99.9\%            & 5.61           & 96.1\%    \\
\hline
Ours & \cellcolor[HTML]{F6DDCC}\textbf{2.80} & \cellcolor[HTML]{F6DDCC}\textbf{99.9\%} & \cellcolor[HTML]{F6DDCC}\textbf{4.20}  & \cellcolor[HTML]{F6DDCC}\textbf{99.9\%} \\ \hline
\end{tabular}
\end{table}

% To evaluate the expressive power of our panel mask decoder, we compare it against a CNN based baseline (CNN) and a vanilla transformer decoder without positional embedding (Trans. w/o PE). From the results in Tab.~\ref{tab:dec}, it is clear that the CNN based baseline  is not suitable for instruction based mask prediction as it loses out on the panel prediction performance due to lack of self-attention among the panels. This is improved by 2.7\%/2.7\% higher Panel L2 metric with the transformer decoder under Seen and Unseen settings respectively.  Besides, positional encoding is important as it predicts position specific masks which is reflected in the drop of 1.4\%/0.5\% in overall performance without positional encoding under Seen and Unseen settings.

\subsection{In-the-wild garment pattern design}
For more extensive evaluation, we qualitatively test on the garment captures from DeepFashion3D dataset \cite{zhu2020deep}.
We observe in Fig.~\ref{fig:wild} that our model makes better panel predictions than NeuralTailor \cite{korosteleva2022neuraltailor}. For example, NeuralTailor fails to perceive the sleeves
with the T-shirt (unseen to model training), whilst our model succeeds.
In the case of jeans, our model gives better panel uniformity than \cite{korosteleva2022neuraltailor}.
% We observe in Fig.~\ref{fig:wild} that our model makes more correct panel predictions and good guesses about the panel structure than NeuralTailor \cite{korosteleva2022neuraltailor}. For example, for garments not seen in the dataset, like the t-shirt which has sleeves, was not predicted by \cite{korosteleva2022neuraltailor} but succesfully predicted by our approach. In the case of jeans, our model has better panel uniformity than \cite{korosteleva2022neuraltailor} highlighting the effectivity of our model design. The quality of prediction on the in-the-wild garment scans can be improved further and thus bridging this sim-to-real gap is part of our future work.
\begin{figure}[h]
    \centering
    \includegraphics[scale=0.28]{img/3rd_party.png}
    \caption{\textbf{In-the-wild garment evaluation} Sewing patterns predicted by NeuralTailor \cite{korosteleva2022neuraltailor} and our PersonalTailor on examples from DeepFashion3D \cite{zhu2020deep}.}
    \label{fig:wild}
\end{figure}



\section{Human Evaluation}

In additional to benchmark based assessment,
we further provide human evaluation with a thoughtful user study.
In particular, we approached 20 professional tailors to request their 
preference on the significance of personalizing garment pattern design.
In particular, we asked them two questions: 
(1) If garment personalization is necessary? 
(2) Which prompt (text or sketch) is preferred?
As shown in Fig.~\ref{fig:hstud1}, 80\% of tailors consider 
automated personalization to be useful, as it saves them time in production and the cost of business.
Besides, 40\% prefer sketch over text for instruction,
whilst 5\% are concerned with sketch to be only for professional designers.
We also collected the tailors feedback on the functions of (a) adding new garment panels, (b) removing garment panels, (c) changing dress topology, and (d) creating new designs. As shown in Fig.~\ref{fig:hstud2}, removing garment panels is most popular, which is supported by our proposed method. 

% To evaluate the performance in a human perceptive level, we conduct thoughtful user studies in this section. Human subjects \ie professional Tailors evaluation is conducted to investigate the usefullness of customized garment pattern design. We have conducted this experiments on twenty human subjects and reported the scores in percentages out of hundred. As observed from Fig~\ref{fig:hstud1}(a), majority of the tailors prefer automated personalization to save time of production and cost of business. It is also interesting to note from Fig~\ref{fig:hstud1}(b) that majority of the tailors prefer sketch as a medium of customer input for the ease of production. However, 5\% of the tailors also raised concerns on the fact that sketch is only for professional designers and not so customer friendly. We also collected tailors feedback on the most requested mode of personalization among (a) adding new garment panels (b) removing panels (c) changing dress topology and (d) creating new designs. From the results in Fig~\ref{fig:hstud2}, it can be observed that majority of customers prefer removing garment panels which is also supported by our proposed solution. 

\begin{figure}[h]
    \centering
    \includegraphics[scale=0.38]{img/human_eval.png}
    % % \vspace{-0.1in}
    \caption{\textbf{Votes on garment personalization.}}
    \label{fig:hstud1}
    % % \vspace{-0.2in}
\end{figure}

\begin{figure}
    \centering
    \includegraphics[scale=0.23]{img/human_study_2.png}
    \caption{\textbf{Votes on garment personalization functions.}}
    \label{fig:hstud2}
\end{figure}

\section{Limitations and Future Work}
We presented a personalized 2D pattern design method for 3D garments,
featuring editing capabilities. Our model is 2D panel-aware, requires no panel annotation, and leverages the Transformer architecture to
form globally coherent 2D patterns of varied topology. The network was designed to allow editing at an interactive rate, where, as demonstrated, the user can interact with the model using a simple instruction (refer to Fig 2). However, one limitation is a relatively small set of panels and
garments, lacking multiple test domains for evaluation. Our mask based panel design cannot model complicated 2D patterns like pleats or darts which is another limitation.  As this is an under-studied area, many challenges (3D optimization, fitting to different body structures etc.) still exist for future work.

\section{Conclusion}\label{sec:conclusion}
In this work, we focus on addressing the fundamental challenge of OOD detection tasks, which is how to fully understand the semantic discrepancy between the ID/OOD samples. We reveal that the key to success in the realistic SCOOD task is to allocate as many ID samples in the unlabeled set correctly as possible. To this end, we propose a novel uncertainty-aware optimal transport scheme that introduces class-specific energy scores as guidance for effective label assignment. Experimental results show that our method achieves better performance than previous state-of-the-art methods on SCOOD benchmarks.

\textbf{Limitations.} In addition to temperature scaling, other techniques such as feature clipping applied in ReAct~\cite{sun2021react} also enhance the performance of energy score, so how to obtain an OOD score that best fits the SCOOD task can be further explored. Moreover, a setting highly related to SCOOD has been proposed in \cite{katz2022training} and formulated as a constrained optimization problem. We will also theoretically analyze these practical OOD settings in our feature work.

% \section*{Acknowledgments}
\textbf{Acknowledgments.} 
This work is supported by National Key R\&D Program of China under Grant 2020AAA0105701, National Natural Science Foundation of China (NSFC) under Grants 61872327, Major Special Science and Technology Project of Anhui, National Natural Science Foundation of China (62033012) and Ant Group through Ant Research Intern Program.

{
    \small
    \bibliographystyle{ieeenat_fullname}
    \bibliography{main}
}
\clearpage
%\setcounter{page}{1}
\maketitlesupplementary
The supplementary material for our work  \textit{SC-VAE: Sparse Coding-based Variational Autoencoder with Learned ISTA} is structured as follows:
%Sec. \ref{section1_s} provides the detailed information of the encoder and decode architecture of the SC-VAE model. 
%Sec. \ref{section2_s} shows the visualization of the dictionary atoms.
%Sec. \ref{section3_s} shows the training loss on the ImageNet dataset with different number of downsampling (upsampling) blocks ($d$) in the encoder (decoder) of the SC-VAE model.
%Sec. \ref{section4_s} shows the visualization results of an unofficial implementation of VIT-VQGAN \cite{yu2021vector}. 
%Sec. \ref{section5_s} shows additional manipulation and interpolation results on FFHQ dataset. 
%Sec. \ref{section6_s} shows additional image patches clustering results on FFHQ and ImageNet datasets. 
%Sec. \ref{section7_s} shows additional unsupervised image segmentation results.
Section \ref{section1_s} details the encoder and decoder architecture of the SC-VAE model. In Section \ref{section2_s}, the dictionary atoms are visualized. In Section \ref{section3_s}, we provide the training losses on the ImageNet dataset when varying the number of downsampling (upsampling) blocks ($d$) in the encoder (decoder) of the SC-VAE model. In Section \ref{section4_s}, the visualized reconstruction results of an unofficial implementation of VIT-VQGAN \cite{yu2021vector} are provided. 
We provide  additional manipulation and interpolation results on the FFHQ dataset in Section \ref{section5_s}, while  additional clustering results of image patches on both FFHQ and ImageNet  are provided in Section \ref{section6_s}. Supplementary unsupervised image segmentation results are given in Section \ref{section7_s}.

%Additional results on image patches clustering and unsupervised image segmentation on FFHQ and ImageNet datasets are then presented in Sec. 2 and Sec. 3, respectively.
\setcounter{section}{0}

\section{The Encoder and Decoder Architecture of SC-VAE} \label{section1_s}
The SC-VAE model's encoder and decoder architecture mirrors that of VQGAN \cite{esser2021taming}. Details about the architecture are provided in Table \ref{figure:encoder_decoder}.
%The encoder and decoder architecture in the SC-VAE model are the same as the architecture used in VQGAN \cite{esser2021taming}, which is described in Table \ref{figure:encoder_decoder}. 
$H$, $W$ and $C$ denote the height, width
and the number of channels of an input image, respectively.
$C'$ and $C''$ represent the number of channels of the feature maps that are produced as outputs by the intermediate layers of the encoder and decode network.
In our experiment, $C'$ and $C''$ were set to $128$ and $512$, respectively. $n$ denotes the number of dimensions of each latent representation, which was set to $256$.
The variable $d$ represents the number of blocks used for downsampling and upsampling. Therefore, we can calculate the height ($h$) and width ($w$) of the encoder's output feature maps by dividing the height ($H$) and width ($W$) of input images by $2$ raised to the power of $d$.

\begin{table}[thbp!]
\centering
\caption{High-level architecture of the encoder and decoder of the SC-VAE model. $H$, $W$, and $C$ refer to the height, width, and the number of channels of an input image. 
$C'$ and $C''$ represent the number of channels of the feature maps from intermediate layers in the encoder and decoder networks. $n$ denotes the number of dimensions of each latent representation, while $d$ represents the number of downsampling (upsampling) blocks. Note that $h=\frac{H}{2^{d}}$, $w=\frac{W}{2^d}$.} 
\resizebox{1\linewidth}{!}{%
\begin{tabular}{c|c}
  \toprule
   &  $x\in \mathbb{R}^{H\times W\times C} $\\
   &  2D Convolution $\rightarrow \mathbb{R}^{H\times W\times C'}$\\
   &  $d \times$\{Residual Block, Downsample Block\} $\rightarrow \mathbb{R}^{h\times w\times C''}$\\
   &  Residual Block $\rightarrow \mathbb{R}^{h\times w\times C''}$\\
  Encoder &  Non-Local Block $\rightarrow \mathbb{R}^{h\times w\times C''}$\\
   &  Residual Block $\rightarrow \mathbb{R}^{h\times w\times C''}$\\
   &  Group Normalization \cite{wu2018group} $\rightarrow \mathbb{R}^{h\times w\times C''}$ \\
   &  Swish Activation Function \cite{ramachandran2017searching} $\rightarrow \mathbb{R}^{h\times w\times C''}$\\
   &  2D Convolution $\rightarrow E(x) \in \mathbb{R}^{h\times w\times n}$\\
  \midrule
   & $\tilde{E}(x)\in \mathbb{R}^{h\times w\times n} $  \\
   &  2D Convolution $\rightarrow \mathbb{R}^{h\times w\times C''}$  \\
   &  Residual Block $\rightarrow \mathbb{R}^{h\times w\times C''}$\\
    & Non-Local Block $\rightarrow \mathbb{R}^{h\times w\times C''}$\\
  Decoder & Residual Block $\rightarrow \mathbb{R}^{h\times w\times C''}$\\
    & $d\times$\{Residual Block, Upsample Block\} $\rightarrow \mathbb{R}^{H\times W\times C'}$\\
   & Group Normalization \cite{wu2018group} $\rightarrow \mathbb{R}^{H\times W\times C'}$\\
    & Swish  Activation Function \cite{ramachandran2017searching}
    $\rightarrow \mathbb{R}^{H\times W\times C'}$\\
    & 2D Convolution $\rightarrow G(\tilde{E}(x)) \in \mathbb{R}^{H\times W\times C}$\\
  \bottomrule
\end{tabular}}
\label{figure:encoder_decoder}
\end{table}

\section{Visualization of Dictionary Atoms}
\label{section2_s}
Figure \ref{figure:dictionary_visualization} demonstrates the $512$ columns (atoms) of the pre-determined Discrete Cosine Transform (DCT) dictionary. Each atom is of dimension $256$, which corresponds to the size of $16 \times 16$ images when shaped.
%We reshape all atoms into an image with a $16\times 16$ resolution.

\begin{figure}[tbp]
\centering
\includegraphics[width=8cm]{./Figures/visualization_of_dictionary.png}
\caption{$512$ atoms of the Discrete Cosine Transform (DCT) dictionary. All atoms were reshaped into a $16 \times 16$ image.}
\label{figure:dictionary_visualization}
\end{figure}

\section{Training Losses}  \label{section3_s}
%Training losses of inherent noises around the 140th epoch under different auxiliary dataset sizes (K)
Figures \ref{figure:TLImagenet32x32}, \ref{figure:TLImagenet16x16}, \ref{figure:TLImagenet4x4} and \ref{figure:TLImagenet1x1} show  the training losses over $120,000$ training steps on the
ImageNet dataset.
The number of downsampling (upsampling) blocks ($d$) in the encoder (decoder) of the SC-VAE model are $3, 4, 6$ and $8$, respectively.
%with the number of downsampling (upsampling) blocks ($d=3,4,6$ and $8$, respectively) in the encoder (decoder) of the SC-VAE model. 
%As is shown in these figures, the LISTA networks of the SC-VAE models converge to a fixed point no matter which downsampling (upsampling) block $d$ is used. However, SC-VAE suffer from image reconstruction when increasing $d$.
As depicted in these figures, the LISTA networks within the SC-VAE models consistently converge to a stable point regardless of the chosen downsampling (upsampling) block $d$. However, increasing $d$ leads to worse image reconstructions ($\mathcal{L}_{rec}$) in SC-VAE.

\begin{figure}[tbp]
\centering
\includegraphics[width=7.5cm]{./Figures/Imagenet32x32.png}
\caption{The training losses over $120,000$ training steps on the ImageNet dataset. The number of  downsampling (upsampling) blocks ($d$) in the encoder (decoder) of the SC-VAE model was set to $3$ and the height ($h$) and width ($w$) of latent representations were $32$. (a) Total loss $\mathcal{L}_{SC-VAE}$. (b) Image reconstruction loss $\mathcal{L}_{rec}$. (c)The mean of latent representations reconstruction loss $\frac{1}{hw}\mathcal{L}_{latent}$.}
\label{figure:TLImagenet32x32}
\end{figure}

\begin{figure}[tbp]
\centering
\includegraphics[width=7.5cm]{./Figures/Imagenet16x16.png}
\caption{The training losses over $120,000$ training steps on the ImageNet dataset. The number of  downsampling (upsampling) blocks ($d$) in the encoder (decoder) of the SC-VAE model was set to $4$ and the height ($h$) and width ($w$) of latent representations were $16$. (a) Total loss $\mathcal{L}_{SC-VAE}$. (b) Image reconstruction loss $\mathcal{L}_{rec}$. (c) The mean of latent representations reconstruction loss $\frac{1}{hw}\mathcal{L}_{latent}$.}
\label{figure:TLImagenet16x16}
\end{figure}

\begin{figure}[tbp]
\centering
\includegraphics[width=7.5cm]{./Figures/Imagenet4x4.png}
\caption{The training losses over $120,000$ training steps on the ImageNet dataset. The number of  downsampling (upsampling) blocks ($d$) in the encoder (decoder) of the SC-VAE model was set to $6$ and the height ($h$) and width ($w$) of latent representations were $4$. (a) Total loss $\mathcal{L}_{SC-VAE}$. (b) Image reconstruction loss $\mathcal{L}_{rec}$. (c) The mean of latent representations reconstruction loss $\frac{1}{hw}\mathcal{L}_{latent}$.}
\label{figure:TLImagenet4x4}
\end{figure}

\begin{figure}[tbp]
\centering
\includegraphics[width=7.5cm]{./Figures/Imagenet1x1.png}
\caption{The training losses over $120,000$ training steps on the ImageNet dataset. The number of  downsampling (upsampling) blocks ($d$) in the encoder (decoder) of the SC-VAE model was set to $8$ and the height ($h$) and width ($w$) of latent representations were $1$. (a) Total loss $\mathcal{L}_{SC-VAE}$. (b) Image reconstruction loss $\mathcal{L}_{rec}$. (c) The mean of latent representations reconstruction loss $\frac{1}{hw}\mathcal{L}_{latent}$.}
\label{figure:TLImagenet1x1}
\end{figure}

%\noindent
%\noindent\textbf{Learnbale ISTA.} The architecture of our Learnable ISTA network is shown in Table 2.
%\noindent\textbf{Attention Network for $\alpha$ Estimation.}  Our neural network architecture follows the backbone of PixelCNN++ [52], which is a U-Net [48] based on a Wide ResNet [72]. We replaced weight normalization [49] with group normalization [66] to make the implementation simpler. Our 32 × 32 models use four feature map resolutions (32 × 32 to 4 × 4), and our 256 × 256 models use six. All models have two convolutional residual blocks per resolution level and self-attention blocks at the 16 × 16 resolution between the convolutional blocks [6].






% \begin{table*}[!htbp]
% \centering
% \caption{High-level architecture of the Learnable ISTA of our SC-VAE. Note that $k$ is the number of the unfolded ISTA block.} 
% \begin{tabular}{c}
%   \toprule
%   Learnable ISTA \\
%   \midrule
%   $E(x)\in \mathbb{R}^{h\times w \times n} $ \\
%   Filter Matrix $\rightarrow \mathbb{R}^{h\times w\times K}$ \\
%   $k\times$\{Shrinkage Function, Mutual Inhibition Matrix, Addition Operator\} $\rightarrow \mathbb{R}^{h\times w\times K}$\\
%   Shrinkage function$\rightarrow Z\in \mathbb{R}^{h\times w\times K}$\\
%   \bottomrule
% \end{tabular}
% \end{table*}

\section{Image Reconstruction}  \label{section4_s}
Reconstruction results from unofficial implementation\footnote{https://github.com/thuanz123/enhancing-transformers} of VIT-VQGAN \cite{yu2021vector} are presented in Figure \ref{figure:ViT-VQGAN_Visualization}.
%Figures \ref{figure:ViT-VQGAN_Visualization} shows visualizations from unofficial implementation\footnote{https://github.com/thuanz123/enhancing-transformers} of VIT-VQGAN \cite{yu2021vector}. 
VIT-VQGAN \cite{yu2021vector} achieved visually appealing results. However, similar to VQ-GAN \cite{esser2021taming} and RQ-VAE \cite{lee2022autoregressive}, it faced challenges in accurately reconstructing intricate details and complex patterns.
%as VQ-GAN\cite{esser2021taming} and RQ-VAE\cite{lee2022autoregressive}. 
Additionally, its generalization performance was inferior to that of our model.

\begin{figure}[tbp]
\centering
\includegraphics[width=7.0cm]{./Figures/ViT-VQGAN_Visualization.png}
\caption{Image reconstructions from an unofficial implementation of VIT-VQGAN \cite{yu2021vector} and the SC-VAE models trained
on ImageNet dataset. Original images in the top two rows are
from the validation set of ImageNet dataset. Two external images are shown in the last two rows to demonstrate the generalizability of different methods. The numbers denote the shape of
latent codes and the learned codebook (dictionary) size, respectively.
SC-VAE achieved improved image reconstruction compared to VIT-VQGAN \cite{yu2021vector}. Zoom in to see the details in the red square area.}
\label{figure:ViT-VQGAN_Visualization}
\end{figure}

\section{Image Generation}  \label{section5_s}
Additional interpolation and manipulation results can be found in Figures \ref{figure:image_interpolation_supple} and \ref{figure:image_manipulation_supple}, respectively.

\begin{figure}[tbp]
\centering
\includegraphics[width=7.0cm]{./Figures/image_interpolation_supple.png}
\caption{Interpolation between the sparse code vectors of two samples from the SC-VAE$^{\dag}$ model trained on FFHQ.}
\label{figure:image_interpolation_supple}
\end{figure}

\begin{figure*}[tbp]
\centering
\includegraphics[width=14.5cm]{./Figures/image_manipulation_supple4.png}
\caption{Manipulating sparse code vectors on FFHQ. 
Each block contains five seed images used to infer the latent sparse code vector in the SC-VAE$^{\dag}$ model.
The disentangled attributes associated with the $i$-th component of a sparse code vector $z$ and a traversal range are shown on the top of each block.}
\label{figure:image_manipulation_supple}
\end{figure*}



% \begin{figure}[tbp]
% \centering
% \includegraphics[width=8cm]{./Figures/IG_Age.png}
% \caption{IG-Age.}
% \label{figure:IG_Age}
% \end{figure}

% \begin{figure}[tbp]
% \centering
% \includegraphics[width=8cm]{./Figures/IG_sunglasses.png}
% \caption{IG-sunglasses.}
% \label{figure:IG_sunglasses}
% \end{figure}

% \begin{figure}[tbp]
% \centering
% \includegraphics[width=8cm]{./Figures/IG_Azimuth.png}
% \caption{IG-azimuth.}
% \label{figure:IG_azimuth}
% \end{figure}

% \begin{figure}[tbp]
% \centering
% \includegraphics[width=8cm]{./Figures/IG_Fringe.png}
% \caption{IG-Fringe.}
% \label{figure:IG_Fringe}
% \end{figure}


% \begin{figure}[tbp]
% \centering
% \includegraphics[width=8cm]{./Figures/IG_skin color.png}
% \caption{IG-skin color.}
% \label{figure:IG_skin color}
% \end{figure}


% \begin{figure}[tbp]
% \centering
% \includegraphics[width=8cm]{./Figures/image_interpolation_supple.png}
% \caption{Interpolation in the latent space between two samples from a model trained on FFHQ.}
% \label{figure:interpolation}
% \end{figure}

\section{Image Patches Clustering}  \label{section6_s}
%Figures \ref{figure:s1} and \ref{figure:s2} exhibit more image patches clustering outcomes for the FFHQ and ImageNet datasets, respectively. 
Figures \ref{figure:s1} and \ref{figure:s2} showcase additional qualitative results of image patches clustering on FFHQ and ImageNet datasets, respectively.
These results were obtained utilizing the pre-trained SC-VAE$^\curlyvee$ model specific to each dataset with a downsampling block $d=4$.
\begin{figure*}[h!]
\centering
\includegraphics[width=16cm]{./Figures/patches_cluster_ffhq_supple_50.png}
\caption{50 randomly selected image patch clusters from the validation set of the FFHQ dataset generated by clustering the learned sparse code vectors of the pre-trained SC-VAE$^\curlyvee$ model
using the K-means algorithm. Each row represents one cluster. Image patches with similar patterns were grouped together.}
\label{figure:s1}
\end{figure*}

\begin{figure*}[h!]
\centering
\includegraphics[width=16cm]{./Figures/imagenet_cluster_patches_V3.png}
\caption{50 randomly selected image patch clusters from the validation set of the ImageNet dataset generated by clustering the learned sparse code vectors of the pre-trained SC-VAE$^\curlyvee$ model
using the K-means algorithm. Each row represents one cluster. Image patches with similar patterns were grouped together.}
\label{figure:s2}
\end{figure*}

% \begin{figure*}[h!]
% \centering
% \includegraphics[width=16cm]{./Figures/segmentation_ffhq_supple3.png}
% \caption{FFHQ.}
% \label{figure:5}
% \end{figure*}

\section{Unsupervised Image Segmentation} \label{section7_s}
\subsection{Qualitative Analysis on FFHQ and ImageNet}
%Figures \ref{figure:s3} and \ref{figure:s4} contain additional qualitative unsupervised image segmentation results on FFHQ and ImageNet datasets, respectively. 
%We utilized two SCVAE models that were pre-trained on the training set of the FFHQ and ImageNet dataset, respectively. These models had a downsampling block of $d = 3$ and a sparsity penalty of $\lambda = 2$. 
%We employed two SC-VAE$^\curlywedge$ models that had been pre-trained on the training sets of the FFHQ and ImageNet datasets, respectively. These models had a downsampling block $d=3$.
Additional qualitative unsupervised image segmentation results on the FFHQ and ImageNet datasets can be found in Figures \ref{figure:s3} and \ref{figure:s4}, respectively. We utilized two SC-VAE$^\curlywedge$ models pre-trained on the training sets of FFHQ and ImageNet, each employing a downsampling block $d=3$.
\subsection{Quantitative comparisons to prior work}
%Figure \ref{figure:Flower_CUB} shows more qualitative results on  Flowers \cite{nilsback2008automated} and Caltech-UCSD Birds-200-2011 (CUB) \cite{WahCUB_200_2011}. Flowers \cite{nilsback2008automated} consists of $8,189$ images of $102$ classes of flowers, with segmentation masks obtained by an automated algorithm developed specifically for segmenting flowers in color photographs \cite{nilsback2007delving}. CUB \cite{WahCUB_200_2011} consists of $11,788$ images of $200$ classes of birds and segmentation masks. Flowers and CUB contain $1,020$ and $1,000$ test images, respectively.
%Figure \ref{figure:Flower_CUB} shows more qualitative results on  Flowers \cite{nilsback2008automated} and Caltech-UCSD Birds-200-2011 (CUB) \cite{WahCUB_200_2011} datasets.
Figure \ref{figure:Flower_CUB} displays additional qualitative results from the Flowers \cite{nilsback2008automated} and Caltech-UCSD Birds-200-2011 (CUB) \cite{WahCUB_200_2011} datasets.\\
\subsubsection{Evaluation Metrics}
\textbf{Intersection of Union (IoU).} %The IoU score measures the overlap of two regions A and B by calculating the ratio of intersection over union, according to
The IoU score quantifies the overlap between two regions. This is achieved by evaluating the ratio of their intersection to their union.
\begin{align}
    \textup{IoU}(A, B) = \frac{|A\cap B|}{|A\cup B|}. \nonumber
\end{align}
%where we use the inferred mask and ground-truth mask as $A$ and $B$ respectively for evaluation.\\
$A$ denotes the ground-truth mask, while $B$ denotes the inferred mask.\\
%as $B$ for assessment purposes.\\
\textbf{DICE score.} Similarly, the DICE score is defined as:
\begin{align}
    \textup{Dice}(A, B) = \frac{2|A\cap B|}{|A|+ |B|}.\nonumber
\end{align}
\noindent
Higher is better for both scores.\\
\subsubsection{Dataset Details}
\textbf{Flowers.} The Flowers \cite{nilsback2008automated} dataset consists of $8,189$ images across $102$ different flower classes. Additionally, it includes segmentation masks generated by an automated algorithm designed explicitly for color photograph flower segmentation \cite{nilsback2007delving}. 
%The images in this dataset have large scale, pose and light variations.\\
The dataset contains images that exhibit substantial variations in scale, pose, and lighting.
Flowers \cite{nilsback2008automated} contains $1,020$ test images.\\
\textbf{CUB.} The CUB \cite{WahCUB_200_2011} dataset contains $11,788$ images covering $200$ bird classes, along with their segmentation masks. 
%Each image is further annotated with $15$ part locations and $1$ bounding box. We use theprovided bounding box to extract a center square from the image, and scale it to $128\times 128$ pixels.
Every image comes with annotations for $15$ part locations, $312$ binary attributes, and $1$ bounding box. We utilized the given bounding box to crop a central square from the image. The CUB dataset includes $1,000$ test images.\\
\textbf{ISIC-2016.} The ISIC-2016 \cite{gutman2016skin} dataset is a public challenge dataset dedicated to Skin Lesion Analysis for Melanoma Detection. Derived from the extensive International Skin Imaging Collaboration (ISIC) archive, it represents a significant collection of meticulously curated dermoscopic images of skin lesions. Within this challenge, a subset of $900$ images is designated as training data, while $379$ images serve as testing data, aiming to provide representative samples for analysis.
%The ISIC-2016 \cite{gutman2016skin} dataset is a public challenge dataset of Skin Lesion Analysis Towards Melanoma Detection released with ISBI 2016. This dataset is based on the International Skin Imaging Collaboration (ISIC) Archive, which is the largest publicly available collection of quality controlled dermoscopic images of skin lesions. The challenge employs a subset of representative images with $900$ images as training data and $379$ images as testing data.

%For all experiments, we resized the input images into a resolution of $256\times 256$ and  generated a $32\times 32$ binary mask for each image utilizing the pre-trained SC-VAE$^\curlywedge$ on ImageNet dataset, a spectral clustering algorithm and boundary connectivity information. The inferred binary mask and ground truth mask were resized to $128\times 128$ to calculate the IoU and DICE scores.
For our experiments, we resized the input images into a resolution of $256\times 256$.
Subsequently, we generated a binary mask of size $32\times 32$ per image by employing the pre-trained SC-VAE$^\curlywedge$ on the ImageNet dataset, along with a spectral clustering algorithm and boundary connectivity information \cite{zhu2014saliency}. To compute the IoU and DICE scores, both the inferred binary mask and the ground truth mask were resized to $128\times 128$.
%\subsubsection{Baseline Methods}
\label{section3}
\begin{figure*}[h!]
\centering
\includegraphics[width=16cm]{./Figures/segmen_ffhq_supple3.png}
%\caption{Additional unsupervised image segmentation results. Images are from the validation set of the FFHQ dataset.}
\caption{Additional unsupervised image segmentation results. These results were generated by grouping sparse code vectors into $5$ categories per image, utilizing the pre-trained SC-VAE$^{\curlywedge}$ model and the K-means algorithm. Images are from the validation set of the FFHQ dataset.}
\label{figure:s3}
\end{figure*}

\begin{figure*}[h!]
\centering
\includegraphics[width=16cm]{./Figures/segmentation_imagenet_supple.png}
%\caption{Additional unsupervised image segmentation results by applying K-means algorithm to cluster sparse code vectors per image into $5$ categories using the SC-VAE$^{\curlywedge}$ model. Images are from the validation set of the ImageNet dataset.}
\caption{Additional unsupervised image segmentation results. These results were generated by grouping sparse code vectors into $5$ categories per image, utilizing the pre-trained SC-VAE$^{\curlywedge}$ model and the K-means algorithm. Images are from the validation set of the ImageNet dataset.}
\label{figure:s4}
\end{figure*}

\begin{figure*}[tbp]
\centering
\includegraphics[width=18cm]{./Figures/flower_cub_isic2016_supple2.png}
\caption{Additional unsupervised image segmentation results on Flowers \cite{nilsback2008automated} (\textit{Left Panel}), CUB \cite{WahCUB_200_2011} (\textit{Middle Panel}) and ISIC-2016 \cite{gutman2016skin} (\textit{Right Panel}). (a) input image. (b) ground truth mask. (c) and (e) segmentation results by clustering sparse code vectors per image into $2$ or $3$ classes using a spectral clustering algorithm. (d) and (f) boundary connectivity information \cite{zhu2014saliency}
was used to decide the foreground and background.}
\label{figure:Flower_CUB}
\end{figure*}

\clearpage
\clearpage
{
   \small
   \bibliographystyle{ieee_fullname}
   \bibliography{egpaper_arxiv_V2}
}

\end{document}
