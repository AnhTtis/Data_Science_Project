\documentclass[a4paper,UKenglish]{lipics-v2016}
%This is a template for producing LIPIcs articles. 
%See lipics-manual.pdf for further information.
%for A4 paper format use option "a4paper", for US-letter use option "letterpaper"
%for british hyphenation rules use option "UKenglish", for american hyphenation rules use option "USenglish"
% for section-numbered lemmas etc., use "numberwithinsect"
 
%\hideLIPIcs

\usepackage{microtype}%if unwanted, comment out or use option "draft"

%\graphicspath{{./graphics/}}%helpful if your graphic files are in another directory

\bibliographystyle{plainurl}% the recommended bibstyle

% Author macros::begin %%%%%%%%%%%%%%%%%%%%%%%%%%%%%%%%%%%%%%%%%%%%%%%%
\title{Randomized Rumor Spreading Revisited}
%\title{A Sample Article for the LIPIcs series\footnote{This work was partially supported by someone.}}
%\titlerunning{A Sample LIPIcs Article} %optional, in case that the title is too long; the running title should fit into the top page column

%% Please provide for each author the \author and \affil macro, even when authors have the same affiliation, i.e. for each author there needs to be the  \author and \affil macros
\author[1]{Benjamin Doerr}
\author[2]{Anatolii Kostrygin}
\affil[1]{Laboratoire d'Informatique (LIX), \'Ecole Polytechnique, Palaiseau, France\\
{\tt doerr@lix.polytechnique.fr}}
\affil[1]{Laboratoire d'Informatique (LIX), \'Ecole Polytechnique, Palaiseau, France\\
{\tt kostrygin@lix.polytechnique.fr}}
\authorrunning{B. Doerr and A. Kostrygin} %mandatory. First: Use abbreviated first/middle names. Second (only in severe cases): Use first author plus 'et. al.'

\Copyright{Benjamin Doerr and Anatolii Kostrygin}%mandatory, please use full first names. LIPIcs license is "CC-BY";  http://creativecommons.org/licenses/by/3.0/

\subjclass{F.2.2 Nonnumerical Algorithms and Problems}% mandatory: Please choose ACM 1998 classifications from http://www.acm.org/about/class/ccs98-html . E.g., cite as "F.1.1 Models of Computation". 
\keywords{Epidemic algorithm, rumor spreading, tight analysis}% mandatory: Please provide 1-5 keywords
% Author macros::end %%%%%%%%%%%%%%%%%%%%%%%%%%%%%%%%%%%%%%%%%%%%%%%%%
% test line	

%\documentclass[11pt,letterpaper]{article}

%\usepackage{showkeys}
%\usepackage[letterpaper,margin=1.1in]{geometry}
%\usepackage[a4paper]{geometry}
\usepackage{amsmath}
\usepackage{amsthm}
\usepackage{amssymb}
\usepackage{mathrsfs}
\usepackage{accents}
%\usepackage[normalem]{ulem}
\newcommand{\ubar}[1]{\underaccent{\bar}{#1}}
\newcommand{\N}{\mathbb{N}}
\newcommand{\Z}{\mathbb{Z}}
\newcommand{\R}{\mathbb{R}}

% \newcommand{\eps}{\epsilon}

\usepackage{graphicx}
\usepackage{color}

\usepackage{caption}

\clubpenalty10000
\widowpenalty10000
\displaywidowpenalty=10000

%\usepackage{enumerate} %(i), (ii), ... - enumeration
\renewcommand{\theenumi}{(\roman{enumi})}

\newcommand{\merk}[1]{\textbf{#1}}

%Import the natbib package and sets a bibliography  and citation styles
%\usepackage{natbib}
%\bibliographystyle{alpha}
%\setcitestyle{authoryear,open=(,close=)}

\newtheorem*{recall*}{Recall}

\theoremstyle{plain}
	\newtheorem*{claim*}{Claim}
	\newtheorem*{lemma*}{Lemma}
	\newtheorem*{observation*}{Observation}
%	\newtheorem{lemma}{Lemma}[section]
%	\newtheorem{corollary}[lemma]{Corollary}%[section]
%	\newtheorem{theorem}{Theorem}
	\newtheorem{proposition}{Proposition}
	\newtheorem{observation}{Observation}%[theorem]

\theoremstyle{definition}
	\newtheorem*{def*}{Definition}
	\newtheorem{defn}{Definition}
	
\theoremstyle{remark}
	\newtheorem*{remark*}{Remark}
%	\newtheorem{remark}{Remark}%[section]
	
\renewcommand{\Pr}{\mathbb{P}}

\newcommand{\eps}{\varepsilon}
\newcommand{\Expect}{\mathbb{E}}%{\operatorname{E}}
\newcommand{\E}{\Expect}%{\mathscr{E}}
\newcommand{\Pk}{p_k}%{\mathscr{P}_k}
\newcommand{\Ck}{c_k}%{\mathscr{C}_k}
\newcommand{\Cov}{\operatorname{Cov}}
\newcommand{\Var}{\operatorname{Var}}
\newcommand{\dist}{\operatorname{dist}}
\newcommand{\Geom}{\operatorname{Geom}}
\newcommand{\Bin}{\operatorname{Bin}}
\newcommand{\dominated}{\preceq}
\newcommand{\subdominated}{\succeq}

\newcommand{\tfn}{\tfrac1n}
\newcommand{\Odn}{O\left(\tfrac1n\right)}
\newcommand{\Okn}{O\left(\tfrac{k}{n}\right)}
\newcommand{\Oknn}{O\left(\tfrac{k}{n^2}\right)}
\newcommand{\Odu}{O\left(\tfrac1u\right)}
\newcommand{\OneE}{\left(1-\tfrac1e\right)}
\newcommand{\OneN}{\left(1-\tfrac1n\right)}
\newcommand{\OneNone}{\left(1-\tfrac1{n-1}\right)}
\newcommand{\Alk}{A_{\ell,k}}
\newcommand{\Olkn}{O\left(\tfrac{\ell+k}{n}\right)}
\newcommand{\Olku}{O\left(\tfrac{\ell+k}{u}\right)}
\newcommand{\AnX}{\Alk \AND X_j=0}
\newcommand{\Omn}{O\left(\tfrac{m}{n}\right)}
\newcommand{\Omnm}{O\left(\tfrac{m}{n-m}\right)}
\newcommand{\fun}{\tfrac{u}{n}}
\newcommand{\ftwok}{\tfrac1{2^k}}
\newcommand{\Odnsq}{O\left(\tfrac1{n^2}\right)}
\newcommand{\Ofun}{O\left(\tfrac{u}{n}\right)}

\newcommand{\Oepsq}{O\left(\eps^2\right)}

\newcommand{\Teg}{T_{eg}}

\newcommand{\ueps}{\ubar\eps}
\newcommand{\beps}{\bar\eps}

\newcommand{\AND}{\wedge}
\newcommand{\OR}{\vee}
\newcommand{\NOT}{\neg}

\newcommand{\miss}{miss}
\newcommand{\hit}{hit}
\newcommand{\called}{called}
\newcommand{\pushed}{pushed}
\newcommand{\accepted}{accepted}
\newcommand{\rejected}{rejected}

%\newcommand{\wlog}{w.l.o.g.}
\newcommand{\Wlog}{W.l.o.g.}

\renewcommand{\labelenumi}{\theenumi}

%\title{Randomized Rumor Spreading Revisited}
%
%\author{Benjamin Doerr and Anatolii Kostrygin\\
%Laboratoire d'Informatique (LIX)\\
%\'Ecole Polytechnique\\
%Palaiseau\\
%France\\
%{\tt doerr@mpi-inf.mpg.de, anatolii.kostrygin@gmail.com}
%}

\date{}

%Editor-only macros:: begin (do not touch as author)%%%%%%%%%%%%%%%%%%%%%%%%%%%%%%%%%%
%\EventEditors{John Q. Open and Joan R. Acces}
%\EventNoEds{2}
%\EventLongTitle{42nd Conference on Very Important Topics (CVIT 2016)}
%\EventShortTitle{CVIT 2016}
%\EventAcronym{CVIT}
%\EventYear{2016}
%\EventDate{December 24--27, 2016}
%\EventLocation{Little Whinging, United Kingdom}
%\EventLogo{}
%\SeriesVolume{42}
%\ArticleNo{23}
% Editor-only macros::end %%%%%%%%%%%%%%%%%%%%%%%%%%%%%%%%%%%%%%%%%%%%%%%

\begin{document}
%\pagenumbering{roman}
\maketitle

\begin{abstract}
  We develop a simple and generic method to analyze randomized rumor spreading processes in fully connected networks. In contrast to all previous works, which heavily exploit the precise definition of the process under investigation, we only need to understand the probability and the covariance of the events that uninformed nodes become informed. This universality allows us to easily analyze the classic push, pull, and push-pull protocols both in their pure version and in several variations such as messages failing with constant probability or nodes calling a random number of others each round. Some dynamic models can be analyzed as well, e.g., when the network is a $G(n,p)$ random graph sampled independently each round [Clementi et al.\ (ESA 2013)].
  
  Despite this generality, our method determines the expected rumor spreading time precisely apart from additive constants, which is more precise than almost all previous works. We also prove tail bounds showing that a deviation from the expectation by more than an additive number of  $r$ rounds occurs with probability at most $\exp(-\Omega(r))$. 

  We further use our method to discuss the common assumption that nodes can answer any number of incoming calls. We observe that the restriction that only one call can be answered leads to a significant increase of the runtime of the push-pull protocol. In particular, the double logarithmic end phase of the process now takes logarithmic time. This also increases the message complexity from the asymptotically optimal $\Theta(n \log\log n)$ [Karp, Shenker, Schindelhauer, V\"ocking (FOCS 2000)] to $\Theta(n \log n)$. We propose a simple variation of the push-pull protocol that reverts back to the double logarithmic end phase  and thus to the $\Theta(n \log\log n)$ message complexity.
\end{abstract}

%
%
%
%\title{The analysis of discrete time push\&pull rumor spreading protocols}
%\author{Benjamin~Doerr, Anatolii~Kostrygin}
%
%\begin{document}
%
%    \maketitle
%
%    \abstract{
%         In this work we take a large complete graph of size $n$ and consider on it the general epidemic protocol in which the probability to involve new node in the process depends only on the number of involved nodes.
%         The spreading time is the time until all nodes are involved into the process (often say "informed").
%         It is known that for many epidemic protocols the spreading time is concentrated in two regimes: growth regime, when $o(n)$ nodes are informed and this number grows fast; and shrinking regime, when it remains to involve only $o(n)$ nodes and this number shrinks fast.
%         We found the conditions implying such behavior of the epidemic protocol on complete graph.
%         Such conditions are expressed in terms of atomic probability of one node to be informed in current round and covariance of involving two nodes simultaneously.
%         This can be considered as a general method of analysis of large class of epidemic protocols on complete graphs, giving the expected spreading time up to some additive constant.
%
%         In the second part of the work we consider different examples of epidemic protocols in which such method can be useful.
%         Thus we introduce the random call model - a round-based process, in which each node calls one randomly chosen neighbor and communicates with it.
%         First we show that our method gives right results for well known push protocol and pull protocol based on the random call model.
%         Then we analyze the basic push\&pull protocol slightly refine the expected spreading time to $\log_3n + \log_2\ln n + O(1)$ instead of $\log_3n + O(\ln \ln n)$.
%
%         In the last part, we introduce the modification of the basic protocols called the \emph{fast single incoming call protocol}, in which the number of communications per node is limited by to in one round.
%         This property makes the protocol more scalable, and allows to reduce the time gap between rounds to be compared to the time of single rumor transaction.
%         However, the last causes the slowdown of the expected spreading time in terms of number of round ($\log_{3-2/e}n + \log_2\ln n + O(1)$ vs $\log_3n + \log_2\ln n + O(1)$).
%    }



\section{Introduction}

Randomized rumor spreading is one of the core primitives to disseminate information in distributed networks. It builds on the paradigm that nodes call random neighbors and exchange information with these contacts. This gives highly robust dissemination algorithms belonging to the broader class of gossip-based algorithms that, due to their epidemic nature, are surprisingly efficient and scalable. Randomized rumor spreading has found numerous applications, among others, maintaining the consistency of replicated databases~\cite{Demers87}, disseminating large amounts of data in a scalable manner~\cite{MatosSFOR12}, and organizing any kind of communication in highly dynamic and unreliable networks like wireless sensor networks and mobile ad-hoc networks~\cite{IwanickiS10}. Randomized rumor spreading processes are also used to model epidemic processes like viruses spreading over the internet~\cite{BergerBCS05}, news spreading in social networks~\cite{DoerrFF11}, or opinions forming in social networks~\cite{Kleinberg08}.

The importance of these processes not only has led to a huge body of experimental results, but, starting with the influential works of Frieze and Grimmett~\cite{FriezeG85} and Karp, Shenker, Schindelhauer, and V\"ocking~\cite{KarpSSV00} also to a large number of mathematical analyses of rumor spreading algorithms giving runtime or robustness guarantees for existing algorithms and, based on such findings, proposing new algorithms.

Roughly speaking, two types of results can be found in the literature, general bounds trying to give a performance guarantee based only on certain graph parameters and analyses for specific graphs or graph classes. In the domain of general bounds, there is the classic maximum-degree-diameter bound of~\cite{FeigePRU90} and more recently, a number of works bounding the rumor spreading time in terms of conductance or other expansion properties~\cite{MoskAoyamaS08,ChierichettiLP10stoc,Giakkoupis11,Giakkoupis14}, which not only greatly helped our understanding of existing processes, but could also be exploited to design new dissemination algorithms~\cite{CensorHillelHKM12,HillelS10,CensorHillelS11,Haeupler13}.
The natural downside of such general results is that they often do not give sharp bounds. It seems that among the known graph parameters, none captures very well how suitable this network structure is for randomized rumor spreading. Also, it has to be mentioned that these results mostly apply to the push-pull protocol.

The other research direction followed in the past is to try to prove sharper bounds for specific graph classes. This led, among others, to the results that the push-protocol spreads a rumor in a complete graph in time $\log_2 n + \ln n \pm \omega(1)$ with high probability $1-o(1)$ (whp.)~\cite{Pittel87} (and in time $\log_{2-p} n + \frac 1p \ln n \pm o(\log n)$ when messages fail independently with probability $p$), whereas the push-pull protocol does so in time $\log_3 n + O(\log\log n)$~\cite{KarpSSV00}. The push protocol spreads rumors in hypercubes in time $O(\log n)$ whp.~\cite{FeigePRU90}, determining the leading constant is a major open problem. For Erd\H os-R\'eny random graphs with edge probability asymptotically larger than the connectivity threshold, again a runtime of $\log_{2-p} n + \frac 1p \ln n \pm o(\log n)$ was shown for the push protocol allowing transmission errors with rate $p$~\cite{FountoulakisHP10}. For preferential attachment graphs, which are often used as model for real-world networks, it was proven that the push-protocol needs $\Omega(n^\alpha)$ rounds, $\alpha>0$ some constant, whereas the push-pull protocol takes time $\Theta(\log n)$ and $\Theta((\log n)/\log\log n)$ when nodes avoid to call the same neighbor twice in a row~\cite{ChierichettiLP09,DoerrFF11}. Even faster rumor spreading times were shown on Chung-Lu power-law random graphs~\cite{FountoulakisPS12}.

One weakness of all these results on specific graphs is that they very much rely on the particular properties of the protocol under investigation. Even in fully connected networks (complete graphs), the existing analyses for the basic push protocol~\cite{FriezeG85,Pittel87,DoerrK14}, the push protocol in the presence of transmission failures~\cite{DoerrHL13}, the push protocol with multiple calls~\cite{PanagiotouPS15}, and the push-pull protocol~\cite{KarpSSV00} all uses highly specific arguments that cannot be used immediately for the other processes. This is despite the fact that the global behavior of these processes is often very similar. For example, all processes mentioned have an exponential expansion phase in which the number of informed node roughly grows by a constant factor until a constant fraction of the nodes is informed. Clearly, this hinders a faster development of the field. Note that the typical analysis of a rumor spreading protocol in the papers cited above needs between six and eight pages of proofs.

\subsection*{Our Results}\label{sec:results}

In this work, we make a big step forward towards overcoming this weakness. We propose a \emph{general analysis method} for all symmetric and memoryless rumor spreading processes in complete networks. It allows to easily analyze all rumor spreading processes mentioned above and many new ones. The key to this generality is showing that the rumor spreading times for these protocols are determined by the probabilities $p_k$ of a new node becoming informed in a round starting with $k$ informed nodes together with a mild bound on the covariance on the indicator random variables of the events that new nodes become informed. Consequently, all other particularities of the protocol can safely be ignored.

Despite this generality, our method gives bounds for the expected rumor spreading time that are \emph{tight apart from an additive constant number of rounds}. Such tight bounds so far have only been obtained once, namely for the basic push protocol~\cite{DoerrK14}. 

Our method also gives \emph{tail bounds} stating that deviations from the expectation by an additive number of at least $r$ of rounds occur with probability at most $A' \exp(-\alpha'r)$, where $A',\alpha'>0$ are absolute constants. Such a precise tail bound was previously given only for the push protocol in~\cite{DoerrK14}. Note that our tail bounds imply the usual whp-statements, e.g., that overshooting the expectation by any $\omega(1)$ term happens with probability $o(1)$ only, and that a rumor spreading time of $O(\log n)$ can be obtained with probability $1-n^{-c}$, $c$ any constant, by making the implicit constant in the time bound large enough.

We use our method to obtain the following particular results. We only state the expected runtimes. In all cases, the above tail bounds are valid as well.

\textbf{Classic protocols, robustness:} We start by analyzing the three basic push, pull, and push-pull protocols. In the \emph{push protocol}, in each round each informed node calls a random node and sends a copy of the rumor to it. In the \emph{pull protocol}, in each round each uninformed node calls a random node and tries to obtain the rumor from it. In the \emph{push-pull protocol}, all nodes contact a random node and in each such contact the informed nodes send rumor to the communication partner. %More details on the protocols and an extensive account of the literature are given in Section~\ref{sec:classics}.

For these three protocols, both in the fault-free setting and when assuming that calls fail independently with probability $1-p$, our method easily yields the expected rumor spreading times given in Table~\ref{tab:results}. Note that all previous works apart from~\cite{DoerrK14} did not state explicitly a bound for the expected runtime. Note further that for half of the settings regarded in Table~\ref{tab:results} no previous result existed. In particular, we are the first to find that the double logarithmic shrinking phase observed by Karp et al.~\cite{KarpSSV00} for the push-pull protocol disappears when messages fail with constant probability $p$, and is instead replaced by an ordinary shrinking regime with the number of uninformed nodes reducing by roughly a factor of $(1-p)e^{-p}$ each round. This observation is not overly deep, but has the important consequence that the message complexity of the push-pull protocol raises from the theoretically optimal $\Theta(n \log\log n)$ value proven in~\cite{KarpSSV00} to an order of magnitude of $\Theta(n \log n)$ in the presence of a constant rate of transmission errors. Hence the significant superiority of the push-pull protocol over the push protocol in the fault-free setting reduces to a constant-factor advantage in the faulty setting.

%notes: whp means... $1-O(n^{-c})$, $c$ arbitrary constant for Karp et al., $1-O(n^{-h(n)}$ with $h(n) = o(1)$ arb. slow for DoerrHL12,
\begin{table}
{\small
	\centering
		\begin{tabular}{|p{1.4cm}|p{5.0cm}|p{5.8cm}|}\hline
		& no transmission failures & calls fail indep.\ with prob.\ $1-p \in (0,1)$\\\hline
		push  \newline protocol & $\E[T] = \log_2 n + \ln n \pm O(1)$ \newline $\lfloor \log_2 n \rfloor + \ln n - 1.116 \le \E[T] \le$\newline$\lceil \log_2 n \rceil + \ln n + 2.765 + o(1)$ \cite{DoerrK14} & $\E[T] = \log_{1+p} n + \tfrac 1p \ln n \pm O(1)$ \newline $T = \log_{1+p} n + \tfrac 1p \ln n \pm o(\log n)$ whp. \cite{DoerrHL13} \\\hline
		pull \newline protocol & $\E[T] = \log_2 n + \log_2 \ln n \pm O(1)$ \newline  & $\E[T] = \log_{1+p} n + \frac{1}{\ln\frac{1}{1-p}} \ln n \pm O(1)$ \newline  \\\hline
		push-pull protocol & $\E[T] = \log_3 n + \log_2 \ln n \pm O(1)$ \newline $T = \log_3 n \pm O(\log\log n)$ whp. \cite{KarpSSV00} & $\E[T] = \log_{1+2p} n + \frac{1}{p+\ln\frac{1}{1-p}} \ln n \pm O(1)$ \newline  \\\hline
					\end{tabular}
\caption{New and previous-best results for rumor spreading time $T$ of the classic rumor spreading protocols in complete graphs on $n$ vertices. The first line of each table entry contains the result that follows  from the method proposed in this work, the second line states the best previous result (if any). For all new bounds on the expected rumor spreading time, a tail bound of type $\Pr[|T \ge \Expect[T]| \ge r] \le A' \exp(-\alpha'r)$ with $A', \alpha' >0$ suitable constants follows as well from this work. In~\cite{DoerrK14}, such a bound was given for the rumor spreading time of the push protocol without transmission failures.}\label{tab:results}
}
\end{table}

\textbf{Multiple calls:} Panagiotou, Pourmiri, and Sauerwald~\cite{PanagiotouPS15} proposed a variation of the classic protocols in which the number of calls (always to different nodes) each node performs when active is a positive random variable $R$. They mostly assume that for each node, this random number is sampled once at the beginning of the process. For the case that $R$ has constant expectation and variance, they show that the rumor spreading time of the push protocol is $\log_{1+\E[R]} n + \frac 1 {\E[R]} \ln n  \pm o(\log n)$ with high probability and that the rumor spreading time of the push-pull protocol is $\Omega(\log n)$ with probability $1 - \eps$, $\eps > 0$. When $R$ follows a power law with exponent $\beta=3$, the push-pull protocol takes $\Theta(\frac{\log n}{\log\log n})$ rounds, and when $2 < \beta < 3$, it takes $\Theta(\log\log n)$ rounds.

The model of~\cite{PanagiotouPS15} makes sense when assuming that nodes have generally different communication capacities. To model momentarily different capacities, e.g., caused by being occupied with other communication tasks, we assume that the random variable is resampled for each node in each round. We also allow $R$ to take the value $0$. Again for the case $\E[R] = \Theta(1)$ and $\Var[R] = O(1)$, we show that the expected rumor spreading time of the push protocol is $\log_{1+\E[R]} n + \frac 1 {\E[R]} \ln n \pm O(1)$. The rumor spreading time of the push-pull protocol depends critically on the smallest value $\ell$ which $R$ takes with positive probability. If $\ell = 0$, that is, with constant probability nodes contact no other node, then there is no double exponential shrinking and the expected rumor spreading time is $\log_{1+2\E[R]}n + \tfrac1{\E[R]-\ln\Pr[R=0]}  \ln n \pm O(1)$. If nodes surely perform at least one call, then we have a double exponential shrinking regime and an expected rumor spreading time of $\log_{1+2\E[R]}n + \log_{1+\ell}\ln n \pm O(1)$.

\textbf{Dynamic networks:} We also show that our method is capable of analyzing dynamic networks when the dynamic is memory-less. Clementi et al.~\cite{ClementiCDFPS16} have shown that when the network in each round is a newly sampled $G(n,p)$ random graph, then for any constant $c$ the rumor spreading time of the push protocol is $\Theta(\log(n) / \min\{p,1/n\})$ with probability $1 - n^{-c}$. We sharpen this result for the most interesting regime that $p =a/n$, $a$ a positive constant. For this case, we show that the expected rumor spreading time is $\log_{2-e^{-a}} n + \frac{1}{1-e^{-a}} \ln(n) + O(1)$. 
%This bound is the same that we proved for the push protocol on the complete graph in the presence of transmission failures with probability $e^{-a}$, which could be interpreted in the way that the main difficulty stemming from this dynamic setting is the fact that nodes are isolated with probability $e^{-a} + O(1/n)$, whereas all other aspects like the small vertex degrees or different vertex degrees have at most an additive $O(1)$ influence on the rumor spreading time. 
Our tail bound $\Pr[|T - \Expect[T]| \ge r] \le A' \exp(-\alpha'r)$ for suitable constants $A', \alpha>0$ implies also the large deviation statement of~\cite{ClementiCDFPS16} (where for $\Theta(\log n)$ deviations in the lower tail the trivial $\log_2(n)$ lower bound holding with probability $1$ should be used). 
%=======
%\merk{wait for push result}
%\textbf{Dynamic networks:} As a proof of concept, we also show that our method is capable of analyzing dynamic networks when the dynamic is memory-less. As the result closest to ours, Clementi et al.~\cite{ClementiCDFPS16} have shown that when the network in each round is a newly sampled $G_{n,p}$ random graph for some fixed $p \ge 1/n$, then the rumor spreading time is $\Theta(\log n)$. To see how our method copes with more dependent network structures, we regard the case that the network in each round is a newly sampled $2$-regular simple graph. For this scenario, we show that the push protocol has an expected rumor spreading time of $\log_2 n + \log_4 n \pm O(1)$, the pull protocol takes $\log_2 n + \log_2 \ln n \pm O(1)$ rounds, and the push-pull protocol finishes after $\log_{5/2} n + \log_2 \ln n \pm O(1)$ rounds. Interestingly, the push protocol profits from the dynamicity of the network (compared to the complete graph), whereas the push-pull protocol needs longer by a constant factor.

\textbf{Answering single calls only:} We finally use our method to discuss an aspect mostly ignored by previous research. While in all protocols above (apart from the one of~\cite{PanagiotouPS15}) it is assumed that each node can call at most one other node per round, it is tacitly assumed in the pull and push-pull protocols that nodes can answer all incoming calls. For complete graphs on $n$ vertices, the classic balls-into-bins theory immediately gives that in a typical round there is at least one node that receives $\Theta(\frac{\log n}{\log\log n})$ calls. So unlike for the outgoing traffic, nodes are implicitly assumed to be able to handle very different amounts of incoming traffic in one round.

The first to discuss this issue are Daum, Kuhn, and Maus~\cite{DaumKM15} (also the SIROCCO 2016 best paper). Among other results, they show that if only one incoming call can be answered and if this choice is taken adversarially, then there are networks where a previously polylogarithmic rumor spreading time of the pull protocol becomes $\tilde \Omega(\sqrt n)$. If the choice which incoming call is answered is taken randomly, then things improve and the authors show that for any network, the rumor spreading times of the pull and push-pull protocol increase by at most a factor of $O(\frac{\Delta(G)}{\delta(G)} \log n)$ compared to the variant in which all incoming calls are answered. Subsequently, Ghaffari and Newport~\cite{GhaffariN16} showed that with the restriction to accept only one incoming call, the general performance guarantees for the push-pull protocol in terms of vertex expansion or conductance~\cite{Giakkoupis11,Giakkoupis14} do not hold. Kiwi and Caro~\cite{KiwiC17} showed that solving the problem of multiple incoming calls via a FIFO queue can lead to extremely long rumor spreading times.

With our generic method, we can easily analyze this aspect of rumor spreading on complete graphs. While for the pull protocol only the growth phase mildly slows down, giving a total expected rumor spreading time of $\E[T] = \log_{2-1/e} n + \log_2 \ln n \pm O(1)$, for the push-pull protocol also the double logarithmic shrinking phase breaks down and we observe a total runtime of $\E[T] = \log_{3-2/e} n + \frac 12 \ln n \pm O(1)$ and, similarly as for the push-pull protocol with transmission failures, an increase of the message complexity to $\Theta(n \log n)$. The reason, as our proof reveals,  is that when a large number of nodes are informed, then their push calls have little positive effect (as in the classic push-pull protocol), but they now also block other nodes' pull calls from being accepted. This problem can be overcome by changing the protocol so that informed nodes stop calling others when the rumor is $\log_{3-2/e} n$ rounds old. The rumor spreading time of this modified push-pull protocol is $\E[T] = \log_{3-2/e} n + \log_2 \ln n \pm O(1)$ and, when halted at the right moment, this process takes $\Theta(n \log\log n)$ messages.


\section{Outline of the Analysis Method}

As just discussed, the main advantages of our approach are its universality and the very tight bounds it proves. We now briefly sketch the main new ideas that lead to this progress. Interestingly, they are rather simpler than the ones used in previous works.

\subsection{Tight Bounds via a Target-Failure Calculus}

We first describe how we obtain estimates for the rumor spreading time that are \emph{tight apart from additive constants}. Let us take as example the classic push protocol. It is easy to compute that in a round starting with $k$ informed nodes, the expected number of newly informed nodes is $E(k) = k - \Theta(k^2/n)$. Hence roughly speaking the number of informed nodes doubles each round (which explains the $\log_2 n$ part of the $\log_2 n + \ln n \pm O(1)$ rumor spreading time), but there is a growing gap to truly doubling caused by (i) calls reaching already informed nodes and (ii) several calls reaching the same target. This weakening of the doubling process was a main difficulty in all previous works.

The usual way to analyze this weakening doubling process is to partition the rumor spreading process in phases and within each phase to uniformly estimate the progress. For example, Pittel~\cite{Pittel87} considers 7 phases. He argues first that with high probability the number if informed nodes doubles until $n_1 = o(\sqrt{n})$ nodes are informed. Then, until $n_2 = n / \log^2(n)$ nodes are informed, with high probability in each round the number of informed nodes increases by at least a factor of $2 (1 - \frac{1}{\log^2(n)})$. Consequently, this second phase lasts at most $\log_{2 (1 - \frac{1}{\log^2(n)})}(n_2/n_1)$ rounds. While this type of argument gives good bounds for phases bounded away from the middle regime with both $\Theta(n)$ nodes informed and uninformed, we do not see how this ``estimating a phase uniformly'' argument can cross the middle regime without losing a number $\omega(1)$ of rounds.

For this reason, we proceed differently. To prove upper bounds on rumor spreading times, for each number $k$ of informed nodes, we formulate a pessimistic round target $E_0(k)$ that is sufficiently below the expected number $E(k)$ of newly informed nodes. Here ``sufficiently below'' means that the probability $q(k)$ to fail reaching this target number of informed nodes is small, but not necessarily $o(1)$ as in all previous analyses. Using a restart argument, we observe that the random time needed to go from $k$ informed nodes to at least $E_0(k)$ informed nodes is stochastically dominated by $1$ plus a geometric random variable with parameter $1 - q(k)$, where all our geometric random variables count the number of failures until success (this is one of the two definitions of geometric distributions that are in use). In particular, the expected time to go from $k$ to at least $E_0(k)$ informed nodes is at most $1 + \frac{q(k)}{1-q(k)}$.

The second, again elementary, key argument is that when we define a sequence of round targets by $k_0 := 1$, $k_1 := E_0(k_0)$, $k_2 := E_0(k_1), \dots$ with suitably defined $E_0(\cdot)$, then the $k_i$ grow almost like $2^i$ (in the example of the classic push protocol). More precisely, there is a $T = \log_2 n \pm O(1)$ such that $k_T = \Theta(n)$. Hence together with the previous paragraph we obtain that the number of rounds to reach $k_T$ informed nodes is dominated by $T$ plus a sum of independent geometric random variables. This sum has expectation $\sum_{i=0}^{T-1}   \frac{q(k_i)}{1-q(k_i)} = O(\sum_{i=0}^{T-1} q(k_i))$, so it suffices that the sum of the failure probabilities $q(k_i)$ is a constant (unlike in previous works, where it needed to be $o(1)$). A closer look at this sum also gives the desired tail bounds.

Similarly, to prove matching lower bounds, we define optimistic round targets $E_0(k)$ such that a round starting with $k$ informed nodes finds it unlikely to reach $E_0(k)$ informed nodes. Since again we want to allow failure probabilities that are constant, we now have to be more careful and also quantify the probability to overshoot $E_0(k)$ by larger quantities. This will then allow to argue that when defining a sequence of round targets recursively as above, then the expected number of targets overjumped (and thus the expected number of rounds saved compared to the ``one target per round'' calculus), is only constant. 

We remark that a target-failure argument similar to ours was used already in~\cite{DoerrK14}, there however only to give an upper bound for the runtime of the push protocol in the regime from $n^s$, $s$ a small constant, to $\Theta(n)$ informed nodes, that is, the later part of the exponential growth regime of the push process, in which via Chernoff bounds very strong concentration results could be exploited. Hence the novelty of this work with respect to the target-failure argument is that this analysis method can be used (i)~also from the very beginning of the process on, where we have no strong concentration, (ii)~also for the exponential and double exponential shrinking regimes of rumor spreading processes, and (iii)~also for lower bounds.

\subsection{Uniform Treatment of Many Rumor Spreading Processes}

As discussed earlier, the previous works regarding different rumor spreading processes on complete graphs all had to use different arguments. The reason is that the processes, even when looking similar from the outside, are intrinsically different when looking at the details. As an example, let us consider the first few rounds of the push and the pull protocol. In the push protocol, we just saw that while there are at most $o(\sqrt n)$ nodes informed, then a birthday paradox type argument gives that with high probability we have perfect doubling in each round. For the pull process, in which each uninformed node calls a random node and becomes informed when the latter was informed, we also easily compute that a round starting with $k$ informed nodes creates an expected number of $(n-k)\frac kn = k - \frac{k^2}n$ newly informed nodes. However, since these are binomially distributed, there is no hope for perfect doubling. In fact, for the first constant number of rounds, we even have a constant probability that not a single node becomes informed.

The only way to uniformly treat such different processes is by making the analysis depend only on general parameters of the process as opposed to the precise definition. Our second main contribution is distilling a few simple conditions that (i)~subsume essentially all symmetric and time-invariant rumor spreading processes on complete graphs and (ii)~suffice to prove rumor spreading times via the above described target-failure method. All this is made possible by the observation that the target-failure method needs much less in terms of failure probabilities than previous approaches, in particular, it can tolerate constant failure probabilities. Consequently, instead of using Chernoff and Azuma bounds for independent or negatively correlated random variables (which rely on the precise definition of the process), it suffices to use Chebyshev's inequality as concentration result.

Consequently, to apply our method we only need to (i)~understand (with a certain precision) the probability $p_k$ that an uninformed node becomes informed in a round starting with $k$ informed nodes; recall that we assumed symmetry, that is, this probability is the same for all uninformed nodes, and (ii)~we need to have a mild upper bound on the covariance of the indicator random variables of the events that two nodes become informed. 

The probabilities $p_k$ usually are easy to compute from the protocol definition. Also, we do not know them precisely. For example, for the growth phase of the push protocol discussed above, it suffices to know that there are constants $a<2$ and $a'$ such that for all $k < n/2$ we have $\frac kn (1 - a \frac kn) \le p_k \le \frac kn (1 + a' \frac kn)$.  This (together with the covariance condition) is enough to show that the rumor spreading process takes $\log_2 n \pm O(1)$ rounds to inform $n/2$ nodes or more. The constants $a, a'$ have no influence on the final result apart from the additive constant number of rounds hidden in the $O(1)$ term. The covariances are also often easy to bound with sufficient precision, among others, because many in processes the events that two uniformed nodes become informed are independent or negatively correlated.

In our general analysis method, we profit from the fact that seemingly all reasonable rumor spreading processes in complete networks can be described via three regimes:

\emph{Exponential growth:} Up to a constant fraction $fn$ of informed nodes, $p_k = \gamma_n \frac kn (1 \pm O(\frac kn))$. The number of informed nodes thus increases roughly by a factor of $(1+\gamma_n)$ in each round, hence the expected time to reach $fn$ informed nodes or more is $\log_{1+\gamma_n} n \pm O(1)$.

\emph{Exponential shrinking:} From a certain constant fraction $u = n-k = gn$ of uninformed nodes on, the probability of remaining uninformed satisfies $1-p_{n-k} = e^{-\rho_n} \pm O(\frac un)$. This leads to a shrinking of the number of uninformed nodes by essentially a factor of $e^{-\rho_n}$ per round. Hence when starting with $gn$ informed nodes, it takes another $\frac 1 {\rho_n} \ln n \pm O(1)$ rounds in expectation until all are informed.

\emph{Double exponential shrinking:} From a certain constant fraction $u = n-k = gn$ of uninformed nodes on, the probability of remaining uninformed satisfies $1-p_{n-k} = \Theta((\frac un)^{\ell-1})$. Now the expected time to go from $gn$ uninformed nodes to no uninformed node is $\log_\ell \ln n \pm O(1)$.

Due to their different nature, we cannot help treating these three regimes separately, however all with the target-failure method. Hence the main differences between these regimes lie in defining the pessimistic estimates for the targets, computing the failure probabilities, and computing the number of intermediate targets until the goal is reached. All this only needs computing expectations, using Chebyshev's inequality, and a couple of elementary estimates.

\section{Precise Statement of the Technical Results}\label{sec:tech}

In this work, we consider only \emph{homogeneous rumor spreading processes} characterized as follows. We always assume that we have $n$ nodes. Each node can be either \emph{informed} or \emph{uninformed}. We assume that the process starts with exactly one node being informed. Uninformed nodes may become informed, but an informed node never becomes uninformed. We consider a discrete time process, so the process can be partitioned into \emph{rounds}. In each round each uninformed node can become informed. Whenever a round starts with $k$ nodes being informed, then the probability for each uninformed node to become informed is some number $p_k$, which only depends on the number $k$ of informed nodes at the beginning of the round.

The main insight of this work is that for such homogeneous rumor spreading processes we can mostly ignore the particular structure of the process and only work with the \emph{success probabilities} $p_k$ defined above and the \emph{covariance numbers} $c_k$ defined as follows. 

\begin{defn}[Covariance numbers]
For a given homogeneous rumor spreading process and $k \in [1..n-1]$ let $c_k$ be the smallest number such that whenever a round starts with $k$ informed nodes and for any two uninformed nodes $x_1, x_2$, the indicator random variables $X_1, X_2$ for the events that these nodes become informed in this round satisfy \[\Cov[X_1,X_2] \le c_k.\]
\end{defn}
Upper bound for these covariances imply upper bounds on the variance of the number of nodes newly informed in a round. If the latter is small, Chebyshev's inequality yields that the actual number of newly informed nodes deviates not a lot from its expectation (which is determined by $p_k$).

Our main interest is studying after how many round all nodes are informed.

%%%DEF Spreading Time
\begin{defn}[Rumor spreading times]
  Consider a homogeneous rumor spreading process. For all $t = 0, 1, \dots$ denote by $I_t$ the number of informed nodes at the end of the $t$-th round ($I_0 := 1$). Let $k \le m \le n$. By $T(k,m)$ we denote the time it takes to increase the number of informed nodes from $k$ to $m$ or more, that is,
    \[
        T(k,m) = \min\{ t-s | I_s = k \operatorname{ and } I_t \ge m \}.
    \]
    We call $T(1,n)$ the rumor spreading time of the process.
\end{defn}

As it turns out, almost all homogeneous rumor spreading processes can be analyzed via three regimes.

\subsection{Exponential Growth Regime} 

When not too many nodes are informed, in most rumor spreading processes we observe roughly a constant-factor increase of the number of informed nodes in one round, however, this increase becomes weaker with increasing number of informed nodes. %This behavior is made precise in the following definition.

\begin{defn}[Exponential growth conditions]
 Let $\gamma_n$ be bounded between two positive constants. Let $a, b, c \ge 0$ and $0 < f < 1$.
    We say that a homogeneous rumor spreading process satisfies the \emph{upper (respectively lower) exponential growth conditions} in $[1,fn[$ if for any $n \in \N$ big enough the following properties are satisfied for any $k < fn$.
    \begin{enumerate}
        \item $\Pk \ge \gamma_n \tfrac{k}{n} \cdot \left(1- a\tfrac{k}{n} - \tfrac{b}{\ln n}\right)$ (respectively $\Pk \le \gamma_n \tfrac{k}{n} \cdot \left(1+ a\tfrac{k}{n} + \tfrac{b}{\ln n}\right)$).
        \item $\Ck \le c\tfrac{k}{n^2}$.
    \end{enumerate}
    In the case of the upper exponential growth condition, we also require $af<1$.
\end{defn}

These growth conditions suffice to prove that in an expected time of at most (respectively at least) $\log_{1+\gamma_n} n \pm O(1)$ rounds a linear number of nodes becomes informed. Consequently, the decrease of the dissemination speed when more nodes are informed (quantified by the term $-a\frac kn$ in the upper exponential growth condition), which was a main difficulty in previous analyses, has only an $O(1)$ influence on the rumor spreading time.  

\begin{theorem}
  If a homogeneous rumor spreading process satisfies the upper (lower) exponential growth conditions in $[1,fn[$, then there are constants $A', \alpha'>0$ such that 
    \begin{align*}
    	&\Expect[T(1, fn)] \underset{(\ge)}{\le} \log_{1+\gamma_n} n \underset{(-)}{+} O(1), \\
    	&\Pr[T(1, fn) \underset{(\le)}{\ge} \log_{1+\gamma_n} n \underset{(-)}{+} r] \le A' \exp(-\alpha'r) \, \mbox{ for  all $r \in \N$}. 
    \end{align*}
  When the lower exponential growth conditions are satisfied, then also there is an $f' \in ]f,1[$ such that with probability $1-O\left(\tfrac1n\right)$ at most $f'n$ nodes are informed at the end of round $T(1,fn)$.
\end{theorem}

We note that the upper tail bound is tight apart from the implicit constants. This is witnessed, for example, by the pull protocol, where rounds starting with only a constant number of informed nodes have a constant probability of not informing any new node.

\subsection{Exponential Shrinking Regime} 

In a sense dual to the previous regime, in some rumor spreading processes (e.g., the push protocol as well as the pull and push-pull protocols in the presence of transmission failures) we observe that the number of uninformed nodes shrinks by a constant factor once sufficiently many nodes are informed. Again, the weaker shrinking at the beginning of this regime has only an $O(1)$ influence on the resulting rumor spreading times.

\begin{defn}[Exponential shrinking conditions]
Let $\rho_n$ be bounded between two positive constants. Let $0 < g < 1$, and $a, c \in \R_{\ge0}$.
	We say that a homogeneous rumor spreading process satisfies the \emph{upper (respectively lower) exponential shrinking conditions} if for any $n \in \N$ big enough, the following properties are satisfied for all $u = n-k \le gn$.
	\begin{enumerate}
		\item
			$1-p_k = 1-p_{n-u} \le e^{-\rho_n} + a\frac{u}{n}$\, (respectively $1-p_k = 1-p_{n-u} \ge e^{-\rho_n} - a\frac{u}{n}$).
		\item
			$c_k = c_{n-u} \le \frac{c}{u}$.
	\end{enumerate}
	For the upper exponential shrinking conditions, we also assume that $e^{-\rho_n} + ag < 1$.
\end{defn}


\begin{theorem}
	If a homogeneous rumor spreading process satisfies the upper (lower) exponential shrinking conditions, then there are $A' \alpha' > 0$ such that 
	\begin{align*}
		&\Expect[T(n-\lfloor g n \rfloor, n)] \underset{(\ge)}{\le} \tfrac1{\rho_n}\ln n \underset{(-)}{+} O(1),\\
		&\Pr[T(n-\lfloor g n \rfloor, n) \underset{(\le)}{\ge} \tfrac1{\rho_n}\ln n \underset{(-)}{+} r] \le A'\exp(-\alpha'r)\, \mbox{ for all $r \in \N$}.
  \end{align*}		
\end{theorem}

Again, the upper tail bound is tight apart from the constants as shown by the push protocol. Here, a round starting with $n-1$ informed nodes has a constant chance to not inform the remaining node.

\subsection{Double Exponential Shrinking Regime}

Protocols using pull operations in the absence of transmission failures display a faster reduction of the number if uninformed nodes.

\begin{defn}[Double exponential shrinking conditions]
	Let $g \in ]0,1]$, $\ell > 1$, and $a, c \in \R_{\ge0}$ such that $ag^{\ell-1} < 1$.
	We say that a homogeneous rumor spreading process satisfies the \emph{upper (respectively lower) double exponential shrinking conditions}
	if for any $n$ big enough the following properties are satisfied for all $u = n-k \in [1, gn]$.
	\begin{enumerate}
		\item $1-p_{n-u} \le a\left(\tfrac{u}{n}\right)^{\ell-1}$\, (respectively $1-p_{n-u} \ge a\left(\tfrac{u}{n}\right)^{\ell-1}$).
		\item $c_{n-u} \le c \tfrac{n}{u^2}$.
	\end{enumerate}
\end{defn}

\begin{theorem}
	If a homogeneous rumor spreading process satisfies the upper (lower) double exponential shrinking conditions, then there are $A', \alpha' > 0$ and $R$ (depending on $\alpha$) such that 
	\begin{align*}
		&\Expect[T(n-\lfloor g n \rfloor, n)] \underset{(\ge)}{\le} \log_\ell \ln n \underset{(-)}{+} O(1),\\
		&\Pr[T(n-\lfloor g n \rfloor, n) \ge \log_\ell \ln n + r] \le O(n^{-\alpha'r+A'})\, \mbox{ for all $r \in \N$},\\
		\mbox{(}&\Pr[T(n-\lfloor g n \rfloor, n) \le \log_\ell \ln n - R] \le O(n^{-1+2\ell\alpha})\mbox{)}.
  \end{align*}		
\end{theorem}

The last rounds of the push-pull protocol show that the upper tail bound is tight apart from the constants. The lower tail bound is clearly not best possible, but most likely good enough for most purposes.

\subsection{Connecting Regimes}

While often these above described three regimes suffice to fully analyze a rumor spreading process, occasionally it is necessary or convenient to separately regard a constant number of rounds between the growth and the shrinking regime. This is achieved by the following two lemmas.

\begin{lemma}
	Consider a homogeneous rumor spreading process. Let $0 < \ell < m < n$ and $0 < p < 1$. Suppose for any number $\ell \le k < m$, we have $p_{k} \ge p$. Then
  \begin{align*}
   	&\Expect[T(\ell,m)] \le \tfrac{n-\ell}{n-m} \cdot \tfrac1{p}\,,\\
   	&\Pr[T(\ell,m) > r] \le \tfrac{n-\ell}{n-m} \cdot (1-p)^r\, \mbox{ for all $r \in \N$}.
  \end{align*}
\end{lemma}

\begin{lemma}
    Let $f, p \in ]0,1[$ and $c > 0$.
    Suppose that for any $k < fn$ we have $p_k \le p$ and $c_k \le \tfrac{c}{n}$.
    Then there exists $f' \in ]f,1[$ such that with probability $1-O\left(\tfrac1n\right)$ at the end of some round the number of informed nodes will be between $fn$ and $f'n$.
\end{lemma}


\section{Applying the Above Technical Results}

In this section, we sketch how to use the above tools to obtain some of the results described in Section~\ref{sec:results}. Since it does not make a difference, to ease the notation we always assume that nodes call random nodes, that is, including themselves. The main observation is that computing the $p_k$ is usually very elementary. For the covariance conditions, often we easily observe a negative or zero covariance, but when this is not true, then things can become technical.

For the \emph{basic push, pull, and push-pull protocols}, we easily observe that all covariances to be regarded are negative or zero: Knowing that one uninformed node $x_1$ becomes informed in the current round has no influence on the pull call of another uninformed node $x_2$. When the protocol has push calls and $x_1$ was informed via a push call, then this event makes it slightly less likely that $x_2$ becomes informed via a push call, simply because at least one informed node is occupied with calling $x_1$.

The success probabilities $p_k$ are easy to compute right from the protocol definition. When $k$ nodes are informed, then the probabilities that an uninformed node becomes informed are 
\begin{equation*}
p_k = 
\begin{cases}
1 - (1-1/n)^k\, &\mbox{ for the push protocol,}\\
k/n\, &\mbox{ for the pull protocol,}\\
p_k = 1 - (1-1/n)^k \frac{n-k}n\, &\mbox{ for the push-pull protocol.}
\end{cases}
\end{equation*}
Using elementary estimates like $1 - k/n \le (1-1/n)^k \le 1 - k/n + k^2/2n^2$, we see that the push and pull protocols satisfy the exponential growth conditions with $\gamma_n = 1$, whereas the push-pull protocol does the same with $\gamma_n = 2$. The push protocol satisfies the exponential shrinking conditions with $\rho_n=1$. The pull and push-pull protocols satisfy the double exponential shrinking conditions with $\ell = 2$. All growth conditions are satisfied at least up to $k = n/2$ informed nodes and all shrinking conditions are satisfied at least for $u \le n/2$ uninformed nodes, so we do not need the intermediate lemmas. This proves our results given in Table~\ref{tab:results} for the fault-free case.

%The success probabilities $p_k$ are easy to compute right from the protocol definition. When $k$ nodes are informed, then the probability that an uninformed node becomes informed are $p_k = 1 - (1-1/n)^k$ for the push protocol, $p_k = k/n$ for the pull protocol, and $p_k = 1 - (1-1/n)^k \frac{n-k}n$ for the push-pull protocol. Using elementary estimates like $1 - k/n \le (1-1/n)^k \le 1 - k/n + k^2/2n^2$, we see that the push and pull protocols satisfy the exponential growth conditions with $\gamma_n = 1$, whereas the push-pull protocol does the same with $\gamma_n = 2$. The push protocol satisfies the exponential shrinking conditions with $\rho_n=1$. The pull and push-pull protocols satisfy the double exponential shrinking conditions with $\ell = 2$. All growth conditions are satisfied at least up to $k = n/2$ informed nodes and all shrinking conditions are satisfied at least for $u \le n/2$ uninformed nodes, so we do not need the intermediate lemmas.

\emph{Faulty communication:} The same arguments (with different constants $\gamma_n$ and $\rho_n$) suffice to analyze these protocols when messages get lost independently with probability $1-p$. The only structural difference is that now for the pull and push-pull protocols uninformed nodes remain uninformed with at least constant probability. For this reason, now all three protocols have an exponential shrinking phase.

The push-pull protocol with the restriction that nodes \emph{answer only a single incoming call} randomly chosen among the incoming calls is an example where the exponential growth and shrinking conditions are harder to prove. To compute the $p_k$ we assume that all $n$ calls have a random unique priority in $[1..n]$ and that the call with lowest priority number is accepted. For fixed priority, the probability of being accepted is easy to compute, and this leads to the success probability of a pull call. For the probability to become informed via a push call, the simple argument that the first incoming call is from an informed node with probability $k/n$ solves the problem. When showing the covariance conditions, we face the problem that it is indeed not clear if we have negative or zero covariance. The event that some node becomes informed increases the chance that this node received a push call. This push call cannot interfere with another node's pull call to an informed node. So it does have some positive influence on the probability of another uninformed node to become informed. Fortunately, for our covariance conditions allow some positive correlation. Because of this, very generally speaking, we can ignore certain difficulties to handle situations when they occur rare enough.

%<<<<<<< .mine
%\emph{Dynamic comunication graphs:} Being maybe the result where it is most surprising that bounds sharp apart from additive constants can be achieved, we now regard in more detail a problem regarded in~\cite{ClementiCDFPS15}. There, the performance of push rumor spreading in a group of $n$ agents was investigated when the actual communication network is changing in each round. As one such dynamic models, it was assumed that the communication graph in each round is a newly sampled $G_{np}$ random graph, that is, there is an edge independently with probability $p$ between any two vertices. 
%
%For the ease of presentation, let us assume that the edge probability equals $p = a/n$ for some constant $a>0$. This is clearly the most interesting case. For this case (and larger $p$), a  $\Theta(\log n)$ rumor spreading time is observed with probability $1-n^{-c}$, where $c$ is arbitrary (but has an influence on the constant hidden in the term $\Theta(\log n)$). Recalling that for $p=a/n$ the graph $G_{np}$ is not connected and has vertex degrees ranging from $0$ to $\Theta(\log(n)/\log\log(n))$, this result is not totally obvious. Also, observe that the random graph is not newly sampled for each action of a node, so there are dependencies that have to be taken into account.
%
%For this setting, we now show conduct a very precise analysis which in particular makes precise the influence of the graph density parameter $a$.
%\begin{theorem}\label{thm:random}
%  Let $T$ be the time the push protocol needs to inform $n$ nodes when in each round a newly sampled $G_{np}$, $p = a/n$, random graph represents the communication network. Then \[\Expect[T] = \log_{2-e^{-a}} n + \frac{1}{1-e^{-a}} \ln(n) \pm O(1).\] There are constants $A',\alpha'>0$ such that $\Pr[|T - \Expect[T]| \ge r] \le A' \exp(\alpha'r)$ holds for all $r \in \N$.
%\end{theorem}
%
%Recall that the probability that a vertex is isolated in this random graph is $e^{-a} + O(1/n)$. Clearly, an informed vertex being isolated necessarily fails to inform another vertex is this round. The rumor spreading time proven above is the same as the one for the case that the communication network is always a complete graph, but calls fail independently with probability $e^{-a}$. Hence in a sense the changing topology (with low vertex degrees) is not harmful apart from the effect that it creates isolated vertices with constant rate. We did not expect this. 
%
%To prove Theorem~\ref{thm:random}, we first observe that the covariance properties are fulfilled. By symmetry, we can assume that in a round starting with $k$ informed nodes, we first sample the random graph and decide for each node with neighbor it potentially calls in this round, and only then decide randomly which $k$ nodes are informed and have these call the random neighbor determined before. Conditioning on the outcome of random graph, neighbor choice, and on that nodes $x$ and $y$ are not informed, in the remaining random experiment the events ``$x$ becomes informed'' and ``$y$ becomes informed'' clearly are negatively correlated. 
%
%Estimating the probability $p_k$ for an uninformed node to become informed in a round starting with $k$ informed nodes, is slightly technical. Since it is unlikely that two neighbors of an uninformed node $x$ are connected by an edge, the main contribution to $p_k$ stems from the case that the informed neighbors of $x$ form an independent set. Conditioning on this outcome of the edges in $\{x\} \cup N(x)$, each informed neighbor of $x$ has a probability of roughly $1-e^{-a}/a$ of calling $x$, giving (again taking care of the dependencies) a probability of roughly $1-p_k \approx (1 - \frac an \frac{e^{-a}}{a})^k$ for the event that no informed node calls $x$. From this, we estimate $\frac kn (1-e^{-a}) (1 - \frac {k+O(1)}{2n}(1-e^{-a})) \le p_k \le \frac kn (1-e^{-a}+O(1/n))$, enabling us to use the argument for the exponential growth phase with $\gamma_n = 1 - e^{-a}$. Similar arguments, again taking some care for the dependencies that random graph imposes on the actions of informed neighbors, show that the upper exponential shrinking conditions are satisfied for $\rho_n = 1-e^{-a}$, whereas and the lower exponential shrinking conditions are satisfied with $\rho_n = 1 - e^{-a} + O(\log(n)^2 /n)$. 
%
%
%=======

\emph{Dynamic communication graphs:} Being maybe the result where it is most surprising that bounds sharp apart from additive constants can be obtained, we now regard in more detail a problem regarded in~\cite{ClementiCDFPS16}. There, the performance of push rumor spreading in a group of $n$ agents was investigated when the actual communication network is changing in each round. As one such dynamic models, it was assumed that the communication graph in each round is a newly sampled $G(n,p)$ random graph, that is, there is an edge independently with probability $p$ between any two vertices. 

For the ease of presentation, we assume that the edge probability equals $p = a/n$ for some constant $a>0$. This is clearly the most interesting case. For such (and larger) $p$, a rumor spreading time of $\Theta(\log n)$ was shown to hold with inverse-polynomial failure probability. Recalling that for $p=a/n$ the graph $G(n,p)$ is not connected and has vertex degrees ranging from $0$ to $\Theta(\log(n)/\log\log(n))$, this result is not obvious (as the proof in~\cite{ClementiCDFPS16} also indicates). Also, observe that the random graph is not newly sampled for each action of a node, so there are dependencies that have to be taken into account. 

For this setting, we now conduct a very precise analysis, which in particular makes precise the influence of the graph density parameter $a$.
\begin{theorem}\label{thm:random}
  Let $T$ be the time the push protocol needs to inform $n$ nodes when in each round a newly sampled $G(n,p)$, $p = a/n$, random graph represents the communication network. Then \[\Expect[T] = \log_{2-e^{-a}} n + \tfrac{1}{1-e^{-a}} \ln(n) \pm O(1)\] and there are constants $A',\alpha'>0$ such that $\Pr[|T - \Expect[T]| \ge r] \le A' \exp(\alpha'r)$ holds for all $r \in \N$.
\end{theorem}

Recall that a vertex is isolated with probability $e^{-a} + O(1/n)$. Clearly, an informed vertex when isolated necessarily fails to inform another vertex in this round. The rumor spreading time proven above is the same as the one for the case that the communication network is always a complete graph, but calls fail independently with probability $e^{-a}$. Hence in a sense the changing topology (with low vertex degrees) is not harmful apart from the effect that it creates isolated vertices with constant rate. We did not expect this. 

To prove Theorem~\ref{thm:random}, we first observe that the covariance properties are fulfilled. By symmetry, we can assume that in a round starting with $k$ informed nodes, we first sample the random graph and decide for each node which neighbor it potentially calls in this round, and only then decide randomly which $k$ nodes are informed and have these call the random neighbor determined before. Conditioning on the outcome of random graph, neighbor choice, and on that nodes $x$ and $y$ are not informed, in the remaining random experiment the events ``$x$ becomes informed'' and ``$y$ becomes informed'' clearly are negatively correlated. 

Estimating the probability $p_k$ for an uninformed node to become informed in a round starting with $k$ informed nodes, is slightly technical. Since it is unlikely that two neighbors of an uninformed node $x$ are connected by an edge, the main contribution to $p_k$ stems from the case that the informed neighbors of $x$ form an independent set. Conditioning on this outcome of the edges in $\{x\} \cup N(x)$, each informed neighbor of $x$ has an independent probability of roughly $(1-e^{-a})/a$ of calling $x$, giving (again taking care of the dependencies) a probability of roughly $1-p_k \approx (1 - \frac an \frac{1 - e^{-a}}{a})^k$ for the event that no informed node calls $x$. From this, we estimate $\frac kn (1-e^{-a}) (1 - \frac {k+O(1)}{2n}(1-e^{-a})) \le p_k \le \frac kn (1-e^{-a}+O(1/n))$, showing that the exponential growth conditions are satisfied with $\gamma_n = 1 - e^{-a}$. Similar arguments, again taking some care for the dependencies that the random graph imposes on the actions of informed neighbors, show that the upper exponential shrinking conditions are satisfied for $\rho_n = 1-e^{-a}$, whereas the lower exponential shrinking conditions are satisfied with $\rho_n = 1 - e^{-a} + O(\log(n)^2 /n)$. 

\section{Summary, Outlook}

In this work, we presented a general, easy-to-use method to analyze homogeneous rumor spreading processes on complete networks (including memoryless dynamic settings). Such processes are important in many applications, among others, due to the use of random peer sampling services in many distributed systems. Such processes also correspond to the fully mixed population model in mathematical epidemiology.

The two main strengths of our method are (i)~that it builds only on estimates for the probability and the covariance of the events that new nodes become informed---consequently, many processes can be analyzed with identical arguments (as opposed to all previous works), and (ii)~that it determines the expected rumor spreading time precise apart from additive constants (with tail bounds giving in most cases that deviations by an additive number $r$ of rounds occur with probability $\exp(-\Omega(r))$ only). The key to our results is distilling the right growth and shrinking conditions, which allow to describe essentially all previously regarded homogeneous processes, and to show, based on these conditions, that the usually present mild deviations from a perfect exponential growth or shrinking in total cost only a constant number of rounds.
%
%From the viewpoint of rumor spreading, this work leaves open two desires, namely overcoming the restrictions to complete networks and to processes without memory. For the former, random graphs might be a good first object of investigation as there similar rumor spreading times have been observed as in complete networks. Concerning the memory issue, it has been observed that already a mild use of memory (not calling the same neighbor twice in a row) can make a substantial difference, so potentially this is an interesting first object for further research.

From a broader perspective, this work shows that the traditional approach to randomized processes of splitting the analysis in several phases and then trying to understand each phase with uniform arguments might not be the ideal way to capture the nature of processes with a behavior changing continuously over time. While we demonstrated that the more careful round-target approach is better suited for homogeneous rumor spreading processes, one can speculate if similar ideas are profitable for other randomized algorithms or processes regarded in computer science.







\newpage
\appendix

\section*{APPENDIX}

This appendix contains material to be read at the reviewers' discretion. Since this appendix is much longer than the paper itself, to ease reading we not only give the parts left out in the body of the submission, but repeat (sometimes mildly reformulated) the technical parts of the body of the submission.

\section{Introduction}
\label{sec:introduction}
% \begin{itemize}
%     % Diffusion of FL
%     \item {\st{Diffusion of FL}}
%     % Security threats to FL
%     \item {\st{Security threats to FL with particular focus on model poisoning}}
%     % Limitations of existing countermeasures
%     \item {\st{Current countermeasures (e.g., KRUM) and their limitations}}
%     % Proposed method and its advantages
%     \item {\st{Intuitive description of the proposed method and its difference (i.e., advantages) w.r.t. state of the art}}
%     % Main contributions
%     \item {\st{Summary of the main contributions of this work}}
%     % Paper's structure and organization
%     \item {\st{Paper's structure and organization}}
% \end{itemize}

% Diffusion of FL
Recently, {\em federated learning} (FL) has emerged as the leading paradigm for training distributed, large-scale, and privacy-preserving machine learning (ML) systems~\cite{mcmahan2017googleai,mcmahan2017aistats}. 
The core idea of FL is to allow multiple edge clients to collaboratively train a shared, global model without disclosing their local private training data.
%Specifically, an FL system consists of a central server and many edge clients; 
A typical FL round involves the following steps: {\em(i)} the server randomly picks some clients and sends them the current, global model; {\em(ii)} each selected client locally trains its model with its own private data; then, it sends the resulting local model to the server;\footnote{Whenever we refer to global/local model, we mean global/local model {\em parameters}.} {\em(iii)} the server updates the global model by computing an \emph{aggregation function}, usually the average (FedAvg), on the local models received from clients.
% \begin{enumerate}
%     \item[{\em(i)}] the server sends the current, global model to the clients and appoints some of them for training;
%     \item[{\em(ii)}] each selected client locally trains its copy of the global model with its own private data; then, it sends the resulting local model back to the server;\footnote{Whenever we refer to global/local model, we mean global/local model {\em parameters}.}
%     \item[{\em(iii)}] the server updates the global model by computing an \emph{aggregation function} on the local models received from clients (by default, the average, also referred to as FedAvg~\cite{mcmahan2017aistats}).
% \end{enumerate}
This process goes on until the global model converges. %(e.g., after a certain number of rounds or other similar stopping criteria).
%\\
% The advantages of FL over the traditional, centralized learning paradigm are undoubtedly clear in terms of flexibility/scalability (clients can join/disconnect from the FL network dynamically), network communications (only model weights\footnote{We will use \textit{parameters} and \textit{weights} interchangeably.} are exchanged between clients and server), and privacy (each client's private training data is kept local at the client's end and not uploaded to the server).
\\
% Security threats to FL
%However, the growing adoption of FL also raises security concerns~\cite{costa2022covert}, particularly about its confidentiality, integrity, and availability.
Although its advantages over standard ML, FL also raises security concerns~\cite{costa2022covert}. %, particularly about its confidentiality, integrity, and availability~\cite{costa2022covert}.
% OLD, LONG VERSION
% Indeed, some work deals with privacy leakage that may expose the local data of some clients~\cite{melis2019sp}. 
% A large body of work, instead, investigates attacks that usually aim to detriment the predictive accuracy of the learned global model. For instance, \emph{data poisoning} attacks achieve this goal by letting an adversary pollute the training set of some corrupt FL clients with maliciously crafted examples~\cite{jagielski2018sp}.
% Similarly, in \emph{model poisoning} the attacker attempts to tweak the global model weights~\cite{bhagoji2019pmlr} by directly perturbing the local model's weights of some infected FL clients before these are sent to the central server for aggregation, usually via so-called Byzantine attacks. 
% It turns out that Byzantine model poisoning attacks severely impact standard FedAvg; therefore, more robust aggregation functions must be designed to make FL systems secure.
Here, we focus on \emph{untargeted model poisoning} attacks~\cite{bhagoji2019pmlr}, where an adversary attempts to tweak the global model weights %\footnote{We will use the terms \textit{parameters} and \textit{weights} interchangeably.} 
by directly perturbing the local model's parameters of some infected clients before these are sent to the central server for aggregation.
In doing so, the adversary aims to jeopardize the global model \textit{indiscriminately} at inference time.
Such model poisoning attacks severely impact standard FedAvg; therefore, more robust aggregation functions must be designed to secure FL systems.
\\
% In this paper, we focus on designing a novel robust aggregation scheme at the server's end to contrast the effect of Byzantine model poisoning attacks.
%
% Current countermeasures and their limitations
%Several countermeasures have been proposed in the literature to combat model poisoning attacks on FL systems.
% Some methods use simple statistics more robust than plain average to smooth the impact of malicious updates (e.g., Trimmed Mean and FedMedian~\cite{yin2018icml}). 
% Other defenses implement outlier detection techniques to discard malicious updates from the aggregation performed at the server's end. Those are either based on heuristics (e.g., Krum/Multi-Krum~\cite{blanchard2017nips} and Bulyan~\cite{mhamdi2018pmlr}) or data-driven approaches (e.g., K-means clustering~\cite{shen2016acm} or DnC via spectral analysis~\cite{shejwalkar2021ndss}). 
% Finally, some strategies rely on a centralized ``source of trust'' to spot potential malicious updates (e.g., FLTrust~\cite{cao2020fltrust}).
% Several countermeasures have been proposed in the literature to combat model poisoning attacks on FL systems, i.e., to discard possible malicious local updates from the aggregation performed at the server's end. 
% These techniques range from simple statistics more robust than plain average (e.g., Trimmed Mean and FedMedian~\cite{yin2018icml}) to outlier detection heuristics (e.g., Krum/Multi-Krum~\cite{blanchard2017nips} and Bulyan~\cite{mhamdi2018pmlr}) or data-driven approaches (e.g., spectral analysis via K-means clustering~\cite{shen2016acm} or spectral analysis), or methods based on ``source of trust'' (e.g., FLTrust~\cite{cao2020fltrust}).
% OLD, LONG VERSION
%Several countermeasures have been proposed in the literature to combat Byzantine model poisoning attacks on FL systems.
% Descriptive statistics
% For example, Trimmed Mean and FedMedian aggregate local model updates using more robust statistics than standard average~\cite{yin2018icml}.
%
% % Heuristics for outlier detection
% Many existing Byzantine-resilient strategies implement some outlier detection heuristics to discard the model updates sent by potentially malicious clients from the input of the aggregation function.
% One of the most popular heuristics is Krum~\cite{blanchard2017nips}.
% This strategy tries to mitigate the impact of Byzantine attacks by selecting as a global model the local model with the smallest sum of Euclidean distances to {\em all} the other local models.
% Although powerful, Krum requires the server to know (or, at least, estimate) the number of malicious FL clients upfront, which is generally impossible in a realistic attack scenario. %
% Moreover, Krum may become ineffective for complex, high-dimensional model parameter spaces due to the curse of dimensionality.
% Bulyan~\cite{mhamdi2018pmlr} tries to overcome this issue by combining Krum with a variant of Trimmed Mean.
% % Data-driven outlier detection
% Other strategies use data-driven outlier detection techniques -- e.g., via K-means clustering~\cite{shen2016acm} -- to spot potential malicious local model updates. 
% %For instance, Shen et al. propose to cluster local model updates with K-means and thus identify outliers.
%
% % Other techniques
% As far as the server is concerned, any local model received can be from a potential malicious client. 
% FLTrust~\cite{cao2020fltrust} assumes the server acts as a client, i.e., trains a local model on an additional {\em trustworthy} dataset at the server's end and compares it against all the local models from other clients. 
% This way, the server can rely on some ``source of trust'' when discarding potentially malicious clients.
%\\
% Limitations of existing Byzantine-resilient strategies
Unfortunately, existing defense mechanisms either rely on simple heuristics (e.g., Trimmed Mean and FedMedian by~\cite{yin2018icml}) or need strong and unrealistic assumptions to work effectively (e.g., foreknowledge or estimation of the number of malicious clients in the FL system, as for Krum/Multi-Krum~\cite{blanchard2017nips} and Bulyan~\cite{mhamdi2018pmlr}, which, however, cannot exceed a fixed threshold).
Furthermore, outlier detection methods using K-means clustering~\cite{shen2016acm} or spectral analysis like DnC~\cite{shejwalkar2021ndss} do not directly consider the temporal evolution of local model updates received.
Finally, strategies like FLTrust~\cite{cao2020fltrust} require the server to collect its own dataset and act as a proper client, thereby altering the standard FL protocol.
\\
% OLD, LONG VERSION
% Overall, existing Byzantine-resilient strategies are either simple heuristics (e.g., FedMedian) or, if they are more complex, they rely on strong and unrealistic assumptions to work effectively (e.g., knowing the number of malicious clients in the FL system in advance, as for Krum and alike).
% Furthermore, data-driven outlier detection methods do not consider the temporary evolution of local model updates received (e.g., K-means clustering). 
% Finally, strategies like FLTrust requires the server to collect its own dataset and act as a proper client, thereby altering the standard FL protocol.
%
% Description of the proposed method
This work introduces a novel pre-aggregation \textit{filter} robust to untargeted model poisoning attacks. Notably, this filter $(i)$ operates without requiring prior knowledge or constraints on the number of malicious clients and $(ii)$ inherently integrates temporal dependencies. 
The FL server can employ this filter as a preprocessing step before applying \textit{any} aggregation function, be it standard like FedAvg or robust like Krum or Bulyan.
Specifically, we formulate the problem of identifying corrupted updates as a multidimensional (i.e., matrix-valued) time series anomaly detection task. 
The key idea is that legitimate local updates, resulting from well-calibrated iterative procedures like stochastic gradient descent (SGD) with an appropriate learning rate, show \textit{higher predictability} compared to malicious updates. This hypothesis stems from the fact that the sequence of gradients (thus, model parameters) observed during legitimate training exhibit regular patterns, as validated in Section~\ref{subsec:intuition}. %until convergence. 
%This regularity may be more pronounced for smooth convex loss functions, but it can still be captured within an appropriate time window, even for more complex and convoluted loss surfaces. 
%We provide evidence of this claim in Appendix~B, where we show that the average mutual information (i.e., ``predictability''), calculated over pairs of legitimate model updates sent at different FL rounds, is significantly higher than the corresponding computation for a malicious client.
\\
Inspired by the matrix autoregressive (MAR) framework for multidimensional time series forecasting~\cite{chen2021je}, we propose the FLANDERS ({\em \textbf{F}ederated \textbf{L}earning meets \textbf{AN}omaly \textbf{DE}tection for a \textbf{R}obust and \textbf{S}ecure}) filter.
The main advantages of FLANDERS over existing strategies like FLDetector~\cite{zhao2020multivariate} are its resilience to large-scale attacks, where $50\%$ or more FL participants are hostile, and the capability of working under realistic non-iid scenarios.
We attribute such a capability to two key factors: $(i)$ FLANDERS works without knowing a priori the ratio of corrupted clients, and $(ii)$ it embodies temporal dependencies between intra- and inter-client updates, quickly recognizing local model drifts caused by evil players. Below, we summarize our main contributions:

\begin{itemize}
\item[{\em(i)}]
We provide empirical evidence that the sequence of models sent by legitimate clients is more predictable than those of malicious participants performing untargeted model poisoning attacks.
\\
\item[{\em(ii)}] 
We introduce FLANDERS, the first pre-aggregation filter for FL robust to untargeted model poisoning based on multidimensional time series anomaly detection.
\\
\item[{\em(iii)}] 
We integrate FLANDERS into Flower,\footnote{\scriptsize{\url{https://flower.dev/}}} a popular FL simulation framework for reproducibility.
\\
\item[{\em(iv)}] 
We show that FLANDERS improves the robustness of the existing aggregation methods under multiple settings: different datasets, client's data distribution (non-iid), models, and attack scenarios.
\\
\item[{\em(v)}] 
We publicly release all the implementation code of FLANDERS along with our experiments.\footnote{\scriptsize{\url{https://anonymous.4open.science/r/flanders_exp-7EEB}}}
\end{itemize}

% Paper's structure and organization
The remainder of the paper is structured as follows. %some related work and the current state-of-the-art solutions to security issues that FL entails. 
Section~\ref{sec:background} covers background and preliminaries. 
In Section~\ref{sec:related}, we discuss related work.
Section~\ref{sec:problem} and Section~\ref{sec:method} describe the problem formulation and the method proposed. % to tackle it. 
Section~\ref{sec:experiments} gathers experimental results. %, and Section~\ref{sec:limitations} discusses some limitations of this work.
Finally, we conclude in Section~\ref{sec:conclusion}.
 %discusses the limitations of this work and draws future research directions.
%reports conclusions and draws perspectives for future research directions.

%%%%%%% OLD %%%%%%%
%to overcome the resilience of Byzantine failures in distributed Stochastic Gradient Descent computations. 
% The strength of Krum is its time complexity, which is linear in the gradient dimension. 
% However, the robustness of the approach is guaranteed for gradient-based learning applications only when the majority of the clients are not compromised. 
% Besides, the aggregation mechanism of Krum, as well as that of similar methods, is robust from a coarse-grained perspective and does not provide solutions to errors and perturbations that may occur at inference time.
%A related approach to~\cite{blanchard2017nips} is the work of Su et al.~\cite{su2016dc}. Here, the authors propose an iterated approximate agreement to tackle a multi-layer scenario attacked by Byzantine agents. 
%However, the method works efficiently on the sole discrete context and it is inapplicable to continuous state environments.
%\gabri{Maybe, we should just talk about the main limitations of existing countermeasures without digging into their details (or, we can just mention Krum as this is the most popular one). I will move the description of all these methods to the Related Work section.}

%     \input{epidemic_protocols.tex}


\section{Main Analysis Technique}

As outlined earlier, in this work we attempt to develop a general analysis technique that covers a large class of rumor spreading problems in perfectly connected networks (complete graphs). To this aim, we define a general class of rumor spreading processes and then distill three regimes such that most rumor spreading processes regarded in the literature are covered by these regimes. For each regime, we prove rumor spreading times sharp apart from additive constants. We shall treat upper and lower bounds separately, so that in cases where only estimates in one direction are known, we still obtain this type of bound.

%
%
%
%
%\section{Epidemic Protocols}
%
%%%%%               SUBSECTION
%%%%%           PROTOCOL DEFINITION
%%%%%

\subsection{Homogeneous Rumor Spreading Processes}

We now characterize the class of rumor spreading processes we aim at analyzing.

\begin{defn} [Homogeneous rumor spreading process]
  We always assume that we have $n$ nodes. Each node can be either \emph{informed} or \emph{uninformed}. We assume that the process starts with exactly one node being informed. Uninformed nodes may become informed, but an informed node never can become uninformed. We consider a discrete time process, so the process can be partitioned into \emph{rounds}. In each round each uninformed node can become informed. Whenever a round starts with $k$ nodes being informed, then the probability for each uninformed node to become informed is some number $p_k$, which only depends on the number $k$ of the informed nodes at the beginning of the round.
\end{defn}

The above definition is relatively abstract and, in principle, could be simply phrased as a Markov process on the number $k \in [1..n]$ of informed nodes. We still find it natural to use the language of rumor spreading. We will discuss many rumor spreading processes covered by this definition in Sections~\ref{sec:classics},~\ref{sec:more examples},~and~\ref{sec:single incoming call}, so let us for the moment only remark that the definition covers all processes regarded in the literature as long as they are memoryless (the events in the current round depend only on which nodes are informed) and symmetric (only the numbers of informed and uninformed nodes is relevant, but not which nodes these are). We remark that our methods can be applied to suitable processes that are not memoryless, see Section~\ref{sec:single call-fast} for an example that is not memoryless due to the use of a time counter. %\merk{Asymptotics?}

The main insight of this work is that we can mostly ignore the particular structure of a rumor spreading process and only work with the \emph{success probabilities} $p_k$ and the \emph{covariance numbers} $c_k$ defines as follows.

\begin{defn}[Covariance numbers]
For a given homogeneous rumor spreading process and $k \in [1..n-1]$ let $c_k$ be the smallest number such that whenever a round starts with $k$ informed nodes and for any two uninformed nodes $x_1, x_2$, the indicator random variables $X_1, X_2$ for the events that these nodes become informed in this round satisfy \[\Cov[X_1,X_2] \le c_k.\]
\end{defn}

It turns out that essentially all homogeneous rumor spreading processes have an \emph{exponential growth phase}, which is roughly characterized by the fact that for suitable constants $f \in (0,1]$, $c \in \R$ and $\gamma_n > 0$ we have for all $k \in [1..fn-1]$ both $p_k = \gamma_n \frac k n (1 \pm O(\frac kn))$ and $c_k \le c \frac k{n^2}$.

This growth phase is followed by one of the following two shrinking regimes. (i) Exponential shrinking regime: For suitable constants $g > 0$, $c > 0$, and $\rho_n > 0$, we have for all $u \le gn$ that $1-p_{n-u} = e^{-\rho_n} \pm \Theta(\frac un)$ and $c_{n-u} \le \frac cu$. In particular, in a round starting with $u \le gn$ uninformed nodes, we expect the number of uninformed nodes to shrink by a factor of roughly $e^{-\rho_n}$. (ii) Double exponential shrinking regime: For suitable constants $g > 0$ and $\ell>1$, we have that for all $u \le gn$ both $1-p_{n-u} = \Theta((\frac un)^{\ell-1})$ and $c_{n-u} \le c \frac n {u^2}$. In particular, we expect the fraction of uninformed nodes to be raised to some positive power $\ell-1$.

In the following subsections, we shall analyze each of these regimes, treating separately upper and lower bound guarantees. The very rough analysis idea is the same in each subsection, so we present and discuss it in more detail in the following subsection and then are more brief in the remaining ones.

Before doing so, we define the rumor spreading time and show an elementary fact that will be convenient several times in the following.


%
%\subsection{Protocol Definition}
%
%Although the inner structure of different protocols may be significantly different, their behavior is often very close.
%Note also that the complete graph is the most symmetrical case, so the informed nodes are indistinguishable and so the uninformed ones.
%These arguments justify the following simplification of the rumor spreading protocols.
%
%%%%DEF Epidemic Protocol
%\begin{defn} [Homogeneous rumor spreading process]
%    Consider $n$ nodes.
%    Each of them can be in one of two states: informed or uninformed.
%    Suppose all nodes share a clock which count discrete rounds.
%    At the end of each round any uninformed node (not necessarily independently) can become informed with certain probability $p_k$ which does not depend on the choice of the node but only on the total number $k$ of the informed nodes at the beginning of the round.
%    This probability is the same for all uninformed nodes and depends only on their number.
%\end{defn}
%
%\merk{alternative definition}
%\begin{defn}[General protocol]
%    Consider $n$ nodes.
%    Suppose they share a clock which count discrete rounds.
%    Each of them can be in one of two states: informed or uninformed and may change its state only at the end of the round.
%\end{defn}
%\begin{defn}[Indicators of informing]
%    Consider some general protocol on $n$ nodes enumerated nodes with numbers $1\ldots n$.
%    Let at some round nodes $i_1, \ldots, i_k$ are informed ($k \le n$).
%    We denote by $X_i(i_1, \ldots, i_k)$ the random indicator variable, such that $X_i(k) = 1$ if and only if the $i$th node is informed at the end of the round when
%    We say that the protocol is \emph{homogeneous}, if for fixed $k \le n$, and for any uninformed node $i$, the probability $\Pk := \Pr[X_i(i_1, \ldots, i_k)=1]$ depends only on $k$.
%    In this case we use the shorter notation $X_i(k)$ or simply $X_i$ if $k$ is supposed to be fixed.
%    In addition let $$\Ck := \max_{i, j \text{ -- uninformed}} \Cov[X_i(k), X_j(k)]$$.
%\end{defn}
%The idea is to split the idea to generalize the protocol by hiding all inner details and the notation of the homogeneous protocol, in which we add some conditions on the symmetry.
%Also it looks natural to introduce here $X_i$, $\Pk$, $\Ck$.
%\merk{end of alternative definition}
%
%Note that here the rumor is born in some uninformed nodes each round instead of being passed from informed ones to their peers.
%All we know are probabilities that the rumor appears in current node in current round and dependencies between them.
%This means that all inner such as calls and accepting (cf. the abstract) are hidden in the corresponding indicator random variables.
%The homogeneity means that we can consider the process as the evolution of the random variable $I_t$ denoting the number of informed nodes at the end of the $t$-th round.
%The common problem in the studying of different rumor spreading processs is to determine the rumor spreading time defined as follows.

%\merk{Warning: Definition of $X_i$ and $I_t$ removed, since most likely not overly necessary here}



%%%DEF Spreading Time
\begin{defn}[Rumor spreading times]
  Consider a homogeneous rumor spreading process. For all $t = 0, 1, \dots$ denote by $I_t$ the number of informed nodes at the end of the $t$-th round ($I_0 := 1$). Let $k \le m \le n$. By $T(k,m)$ we denote the time it takes to increase the number of informed nodes from $k$ to $m$ or more, that is,
    \[
        T(k,m) = \min\{ t-s | I_s = k \operatorname{ and } I_t \ge m \}.
    \]
    We call $T(1,n)$ the rumor spreading time of the process.
\end{defn}


%
%In the rest of the section we will describe the most common behaviors of the homogeneous rumor spreading process.
%First of all we remark that usually there are two regimes of the protocol which determine the rumor spreading time.
%
%\begin{enumerate}
%    \item
%		\emph{Growth regime.}
%		Very few nodes are informed (i.e. $k \le O(n)$).
%		In this case there are approximatively $n \cdot p_k$ newly informed nodes each round started from $k$ informed nodes, where $p_k$ is the probability of one node to be informed.
%		In most of the practical examples the growth regime is \emph{exponential}, because if $p_k \sim k/n$, then the number of informed nodes grows exponentially yielding a logarithmic term in the rumor spreading time.
%    \item
%	    \emph{Shrinking regime.}
%		Most of the nodes are informed, so it is more natural to watch the uninformed ones.
%		If the probability is almost a constant between 0 and 1, then the number of uninformed nodes roughly multiplies by this constant each round what corresponds to an \emph{exponential shrinking} regime and to another logarithmic term in the rumor spreading time.
%		Another common behavior is characterized by the squaring of the fraction of uninformed nodes and yields a double logarithmic term in the spreading time.
%		For this reason it is called \emph{double exponential shrinking}.
%\end{enumerate}
%
%Even when we aim at proving bounds for the spreading time that are sharp apart from additive constant, we usually do not need the growth and shrinking regime to cover the whole process. The following lemma implies for most rumor spreading protocols that when $\ell, m$ are integers with $\ell = \Theta(n)$ and $m = \Theta(m)$ as well as $n-\ell = \Theta(n)$ and $n-m=\Theta(n)$, then going from $\ell$ informed nodes to at least $m$ informed nodes only takes an expected constant number of rounds.

%The really interesting fact is that to determine the expected spreading time up to an additive constant it is enough to know the behavior of the protocol only when the number of informed or uninformed nodes is $o(n)$.
%The following lemma explains why the most of rumor spreading protocols pass the regime when the numbers of informed and uninformed nodes are of the same order in only constant number of rounds.


Most homogeneous rumor spreading processes have the property that when a constant fraction of the nodes is informed, then each uninformed node has a constant positive probability of becoming informed in one round. In this situation, the following lemma allows to argue that an expected constant number of rounds suffices to go from any constant fraction of informed nodes to any constant fraction of uninformed nodes. This will be convenient in some the following proofs of upper bounds for rumor spreading times, namely when the growth or shrinking conditions are not strong enough near to the middle point of $n/2$ informed nodes.

\begin{lemma}\label{lem:general-connect}
	Consider a homogeneous rumor spreading process. Let $0 < \ell < m < n$ and $0 < p < 1$. Suppose for any number $\ell \le k < m$, we have $p_{k} \ge p$.
    Then
    \renewcommand{\theenumi}{(\roman{enumi})}%
    \begin{enumerate}
	    \item $\Pr[T(\ell, m) > r] \le \tfrac{n-\ell}{n-m} \cdot (1-p)^r$.
    	\item $\Expect[T(\ell,m)] \le \frac{n-\ell}{n-m} \cdot \frac1{p}\,.$
    \end{enumerate}
%    In particular, when $\ell$ and $ n-m = \Omega(n)$, then an expected constant number of rounds suffices to proceed from $\ell$ informed nodes to $m$ or more informed nodes.
\end{lemma}



\begin{proof}
  Let $q := 1-p$. We regard a dummy process which coincides with the given process until the number of informed nodes is at least $m$. If there are at least $m$ nodes informed, then the dummy process shall be such that each uniformed node in each round independently becomes informed with probability $p$. Obviously, $T(\ell,m)$ is the same for both processes, so we consider the dummy process in the following.

In this dummy process, by the memorylessness of our rumor spreading process, an uninformed node remains uninformed for $r$ rounds with probability at most $q^r$. Hence the expected number $U_r$ of uninformed nodes after $r$ rounds is $\Expect[U_r] \le  (n-\ell) q^r$ and Markov's inequality gives
    \[
        \Pr[T(\ell,m) > r] = \Pr[ U_r > (n-m) ] < \frac{n-\ell}{n-m} \cdot q^r.
    \]
    Hence
    \begin{align*}
    	\Expect[T(\ell,m)]
    	& = \sum_{r=0}^\infty \Pr[ T(\ell,m) > r ]
			< \frac{n-\ell}{n-m} \sum_{r=0}^\infty q^r
			= \frac{n-\ell}{n-m} \cdot \frac1{1-q}\,\,.
    \end{align*}
%
%
%
%  \merk{in other parts we use $X_i$ instead of $U_i$.}
%	
%	Let $r \in \N$. Consider some enumeration $1, \ldots, n-\ell$ of the uninformed nodes. For $i \in [1..n-\ell]$, introduce a random indicator variable $U_i^r = 1$ if and only if the $i$th node stays uninformed for $r$ consecutive rounds. As the probability that an uninformed node remains uninformed is at most $q$ (independently for all future rounds), we have $\Pr[U_i^r=1] \le q^r$.
%    Let $U^r := \sum_i U_i^r$ denote the number of uninformed nodes after $r$ rounds.
%    Clearly, $\Expect[U^r] \le (n-\ell) q^r$. By Markov's inequality, we have
%    \[
%        \Pr[ U^r > (n-m) ] < \frac{n-\ell}{n-m} \cdot q^r.
%    \]
%    Note that this probability is also the probability that $T(\ell,m) > r$.
%    Hence
%    \begin{align*}
%    	\Expect[T(\ell,m)]
%    	& = \sum_{r=0}^\infty \Pr[ T(\ell,m) > r ]
%			< \frac{n-\ell}{n-m} \sum_{r=0}^\infty q^r
%			= \frac{n-\ell}{n-m} \cdot \frac1{1-q}\,\,.
%    \end{align*}
\end{proof}

Similarly to the lemma above, the following lemma will be convenient in some of the proofs of lower bounds for rumor spreading times, again when the growth and shrinking conditions do not cover the whole process. In this case, the following lemma allows to argue that an arbitrarily small, but still constant fraction of uninformed nodes will be reached at some time.

\begin{lemma}\label{lem:general-connect-lower}
    Let $f, p \in ]0,1[$ and $c > 0$.
    Suppose that for any $k < fn$ we have $p_k \le p$ and $c_k \le \tfrac{c}{n}$.
    Then there exists $f' \in ]f,1[$ such that with probability $1-O\left(\tfrac1n\right)$ at the end of some round the number of informed nodes will be between $fn$ and $f'n$.
\end{lemma}
\begin{proof}
    Suppose $k < fn$.
    Denote by $X(k)$ the number of newly informed nodes in a round starting with $k$ informed nodes.
    Since $p_k \le p$, we have $\E[X(k)] \le pn(1-f) \le pn$.
    Then by Lemma~\ref{lem:prelim:variance} we have $\Var[X(k)] \le (p+c)n$.
    Let $f' \in ]f+p(1-f),1[$.
    Applying Chebyshev's inequality, we compute
    \begin{align*}
        & \Pr[k+X(k) \ge f'n] \le \Pr[X(k) \ge (f'-f)n] \\
        & \le \Pr[X(k) \ge \E[X(k)] + n(f'-f-p(1-f))] \\
        & \le \tfrac{\Var[X(k)]}{n^2(f'-f-p(1-f))^2}
            = \tfrac{p+c}{n(f'-f-p(1-f))^2} = O\left(\tfrac1n\right).
    \end{align*}
    Therefore, the probability that the process ``jumps over'' the interval $[fn,f'n]$ is $O\left(\tfrac1n\right)$.
\end{proof}

% \input{exp_growth.tex}

\subsection{Exponential Growth Regime. Upper Bound}
\label{section:exp-growth-upper}

In this section and the following, we analyze the runtime of a homogeneous rumor spreading process in the regime where the number of informed nodes roughly grows by a constant factor until a linear number $fn$ of nodes is informed. Not surprisingly, this implies that the process takes a logarithmic time to inform a linear number of nodes.

The challenge in the following analysis, which was also faced by previous works, is that in most rumor spreading processes the dissemination speed reduces when more nodes are informed. So it is not true that for all $k \in [1,fn]$, a round starting with $k$ informed nodes ends with an expected number of $k + \gamma k$ nodes, where $\gamma$ is some constant, but rather that we only expect $E_k = \gamma k (1 - \Theta(k/n))$ newly informed nodes. This non-linearity also implies that a round starting with an \emph{expected} number of $k$ nodes does not end with an expected number of $k + E_k$ informed nodes, but less. So we also need to argue that the number of newly informed nodes a round ends with is strongly concentrated around its expectation, and that thus, we can assume that with sufficiently high probability we end up not too far below the expectation (which gives another small loss over the idealized multiplicative increase of the number of informed nodes).

We overcome these difficulties as follows. (i) We formulate an \emph{exponential growth condition} that is satisfied by essentially all homogeneous rumor spreading processes showing an exponential growth regime. The key observation, which allows us to treat many protocols with this single analysis is that it is not necessary that the actions of the nodes show particular independences. It suffices that a relatively mild covariance condition is satisfied. (ii) We then use (throughout the whole regime from the first informed node to a linear number of informed nodes) a simple phase-target argument. (a) We define for each number $k$ of initially informed nodes a \emph{round target} $E_0(k)$ such that a round starting with $k$ informed nodes with (sufficiently high) probability $1-q_k$ ends with $E_0(k)$ informed nodes. Hence the expected time to go from $k$ to $E_0(k)$ or more informed nodes is $t_k = 1 + \frac{q_k}{1-q_k}$. (b) From this, we define a sequence of target $k_0 = 1, k_1 = E_0(k_0), k_2 = E_0(k_1), \dots, k_J = \Theta(n)$ and argue that the time to reach $k_J$ informed nodes is just the sum of the expected times $t_{k_j}$. By defining the round targets in a suitable manner, we ensure that $J = \log_{1+\gamma}(n) + \Theta(1)$ and that the sum of the $t_{k_j}$ is $J + \Theta(1)$. We note that the phase-target argument was also used in~\cite{DoerrK14}, there however only for the push-protocol and only in the regime from $n^s$, $s$ a small constant, to $\Theta(n)$ informed nodes. Consequently, due to the large number of active nodes acting independently, the phase failure probabilities where ignorable small.

In principle, all the arguments outlined above are very elementary and use nothing more advanced than expectations and Chebyshev's inequality. Hence the main technical progress of this work is formulating an exponential growth condition (including the covariance condition) that allows these elementary arguments in a way that the deviations from the idealized ``multiply-by-$\gamma$'' world in the end all disappear in the $\Theta(1)$ term of the dissemination time. These technicalities also appear in some of the following calculations, which therefore, while all not difficult, are at times slightly lengthy. Since arguments similar to the ones in this section are used throughout this work, we give all details in this section and will be more brief in the following ones.

We start in this section with proving an upper bound for the runtime given that we have suitable lower bounds for the probability that an uninformed node becomes informed. In the following section, we prove a lower bound for the runtime given that we have suitable upper bounds on the speed of the progress. These bounds will match apart from additive constants if the growth factor $\gamma$ is identical.


\subsubsection{Exponential Growth Conditions}

Throughout this section, we assume that we regard a homogeneous epidemic protocol which satisfies the following \emph{upper exponential growth conditions} including a covariance condition.

%\merk{add the same text to other sections}

%\merk{A: should we hence split exp-growth and covariance conditions? B: not now, but a some time yes}

\begin{defn} [upper exponential growth conditions]\label{def:upper-exp-growth-conditions}
    Let $\gamma_n$ be bounded between two positive constants.
    Let $a, b, c \ge 0$ and $0 < f < 1$ with $af<1$.
    We say that a homogeneous epidemic protocol satisfies the \emph{upper exponential growth conditions} in $[1,fn[$ if for any $n \in \N$ big enough the following properties are satisfied for any $k < fn$.
	\renewcommand{\theenumi}{(\roman{enumi})}%
    \begin{enumerate}
        \item $\Pk \ge \gamma_n \tfrac{k}{n} \cdot \left(1- a\tfrac{k}{n} - \tfrac{b}{\ln n}\right)$.
        \item $\Ck \le c\tfrac{k}{n^2}$.
    \end{enumerate}
\end{defn}
%\merk{we need only to define somewhere $\Pk$ and $\Ck$}


The main result of this section is that the upper exponential growth conditions imply that the number of informed nodes multiplies by, essentially, $1+\gamma_n$ in each round, and that the expected number of rounds until $fn$ nodes are informed, is at most $\log_{1+\gamma_n}n + O(1)$.

\begin{theorem}[upper bound for the spreading time]\label{th:exp-growth-upper}
    Consider a homogeneous epidemic protocol satisfying the upper exponential growth conditions in $[1,fn[$.
    Then there exist constant $A', \alpha'$ such that
    \begin{eqnarray*}
    	&& \Expect[T(1, fn)] \le \log_{1+\gamma_n} n + O(1), \\
		&& \Pr[T(1,fn) > \log_{1+\gamma_n} n + r] \le A' e^{-\alpha' r} \, \mbox{for any $r \in N$}.
    \end{eqnarray*}
%	In addittion, there exist $\lambda, \nu > 0$ such that for any integer $r > 0$ we have
%	$$\Pr[T(1,fn) > \log_{1+\gamma_n} n + r] \le \lambda e^{-\nu r}.$$
\end{theorem}

\subsubsection{Round Targets and Failure Probabilities}\label{subsection:exp-growth-upper - round targets}
Let us introduce the random variable $X(k)$ being equal to the number of newly informed nodes in a round having $k$ informed nodes at the beginning.
Since $\Expect[X(k)] = p_k (n-k)$, the exponential growth conditions imply $\Expect[X(k)] \ge E(k)$, where
\begin{equation}
	E(k) := \gamma_n k \left(1 - (a+1)\tfrac{k}{n} - \tfrac{b}{\ln n}\right). \notag
\end{equation}

Using Chebyshev's inequality we can show that the value of $X(k)$ is concentrated around its expected value.
Lemma~\ref{lem:exp-growth-failure} hence claims that with good probability, $X(k)$ attains at least the \emph{target value}
\begin{equation}
    E_0(k) := E(k) - Ak^B, \label{eq:def-E0-upper}
\end{equation}
where $A > 0$ and $B \in ]0.5, 1[$ are some constants chosen uniformly for all values of $k$ and $n$.
There are no special conditions on $B$, so we suppose that $B$ is fixed from now on, e.g., to 3/4.
We will, in the following, choose $A$ small enough to ensure that the $-Ak^B$ term has a sufficiently small influence on the general bevahior of  $E_0(k)$.

\begin{lemma}\label{lem:exp-growth-E0-increase}
%\merk{A: we don't need here $f' < f$. B: But maybe it makes still sense...}
There exist $f' > 0$ and $A'>0$ such that for $n$ big enough, the following conditions are satisfied.
\begin{itemize}
	%\item $E(k) \ge \gamma_n k/2$,
	\item $E(\cdot)$ is increasing up to $f'n$, that is, for all $i < j \le f'n$ we have $E(i) < E(j)$;
	\item When $A$ in equation~\eqref{eq:def-E0-upper} satisfies $0 < A < A'$, then also $E_0(\cdot)$ is increasing up to $f'n$;
	\item $E_0(k) > 0$ for all $k \in [1,f'n[$.
\end{itemize}
\end{lemma}
\begin{proof}
	The first claim follows from the second, so let us regard the derivative of $E_0(k)$,
    \[
    	E_0'(k) = \gamma_n - 2\gamma_n(a+1)\tfrac{k}{n} - \gamma_n\tfrac{b}{\ln n} - ABk^{-1+B}.
    \]
    We see that, for any $f' < \tfrac1{2(a+1)}$, any $A > 0$ small enough, and any $n$ large enough, $E_0'(k)$ is positive for all $k \in [1, f'n[$. Therefore, to satisfy the first two parts of the claim, we pick any $f' \in ]0, \tfrac1{2(a+1)}[$ and then any $A' < \tfrac1B \gamma_n(1-2(a+1)f')$.

    To show that $E_0(k) > 0$ for all $k \in [1,f'n[$, it suffices to check this for $k=1$.
    By possibly lowering $A'$ further, we obtain for $n$ large enough that
    \begin{equation}
    	E_0(1) = \gamma_n \left(1 - \tfrac{a+1}{n} - \tfrac{b}{\ln n}\right) - A > 0 \notag.
    \end{equation}
\end{proof}
We assume in the following that $f$ in Definition~\ref{def:upper-exp-growth-conditions} satisfies $f < f'$ and that $A$ in~\eqref{eq:def-E0-upper} was chosen in $]0,A'[$.
%Also, we choose $A$ such that $0 < A < A'$.
%Finally, let us assume that $n$ is sufficiently large.

\begin{lemma}\label{lem:exp-growth-failure}
    For any $k < fn$,
    \begin{equation}
        \Pr[X(k) \le E_0(k)] \le \min\left\{q(k), \tfrac1{1+1/q(1)}\right\} \notag,
    \end{equation}
    where $q(k) := \tfrac{\gamma_n+c}{A^2} \cdot k^{-2B+1}$.
\end{lemma}
\begin{proof}
    By the exponential growth conditions, $\Expect[X(k)] \ge E(k)$.
    Applying Chebyshev's inequality, we compute
    \begin{align*}
        & \Pr[X(k) \le E_0(k)]
    	   = \Pr\left[X(k) \le E(k) \cdot \left(1-\tfrac{Ak^B}{E(k)}\right)\right] \\
        & \le \Pr\left[ X(k) \le \Expect[X(k)]\cdot\left(1-\tfrac{Ak^B}{E(k)}\right) \right] \\
    	& = \Pr\left[ X(k) \le \Expect[X(k)] - Ak^B \cdot \tfrac{\Expect[X(k)]}{E(k)} \right] \\
        & \le \tfrac{\Var[X]}{(Ak^B)^2} \cdot \tfrac{E(k)^2}{\Expect[X(k)]^2}.
    \end{align*}
    From the covariance condition, it follows that $\Var[X(k)] \le \Expect[X(k)] + ck$.
	Using $E(k) / \Expect[X(k)] \le 1$ once, we obtain
    \begin{align*}
        \Pr[X(k) \le E_0(k)]
        & \le \left(1 + \tfrac{ck}{\Expect[X(k)]}\right) \cdot \tfrac{E(k)}{\Expect[X(k)]}
    	   		\cdot \tfrac{E(k)}{A^2k^{2B}} \\
    	& \le \left(1 + \tfrac{ck}{E(k)}\right) \cdot \tfrac{E(k)}{A^2k^{2B}} \\
        & = \tfrac{E(k)+ck}{A^2k^{2B}} \le \tfrac{\gamma_nk + ck}{A^2k^{2B}}.
    \end{align*}
    One can see that for small values of $k$, $q(k)$ might be more than one.
	To avoid such a trivial bound for the failure probability, it suffices to replace Chebyshev's inequality in the proof by the Cantelli's inequality (see Lemma~\ref{lem:prelim:Cantelli}) and bound the probability by $\tfrac1{1+1/q(k)}$.
	To finish the proof we note that $q(k)$ is decreasing in $k$, so
	$\Pr[X(k) \le E_0(k)] \le \min\left\{q(k), \tfrac1{1+1/q(1)}\right\}$.
\end{proof}
%\begin{corollary}\label{cor:exp-growth-error-prob}
%	$\Pr[X(k) \le E_0(k)] \le \min\left\{q(k), \tfrac1{1+1/q(1)}\right\}$.
%\end{corollary}

%One can see that for small values of $k$, $q(k)$ might be more than one.
%To avoid this, it suffices to replace the Chebyshev inequality in the proof by the Cantelli's.
%Then, taking into account that $q(k)$ is decreasing in $k$, we obtain the following.

\subsubsection{The Phase Calculus}\label{subsection:exp-growth-upper - phases}

Having just defined round targets for all numbers $k$ of initially informed nodes and the probabilities that these targets are not achieved within a round, we now proceed to define the sequence $k_j$ of round targets which we aim at satisfying one after the other, ideally within one round per target.
%
%
%
%\merk{B: I'll make this text a little more crisp at some time}
%In the previous section we proved that if at the beginning of the round there were $k$ informed nodes, then at the end there will be at least $k + E_0(k)$ informed nodes, with probability at least $1-q(k)$.

We define recursively
\begin{equation}
    k_0 = 1, \quad k_{j+1} := k_j + E_0(k_j) \notag.
\end{equation}

%\merk{B: I'll write some text here}
\begin{lemma}\label{lem:exp-growth-expk-upper}
	After possibly lowering $A'$ from Lemma~\ref{lem:exp-growth-E0-increase}, there exist $\alpha > 0$ and $J = \log_{1+\gamma_n} n + O(1)$ such that
	\begin{equation}
		fn > k_j \ge \alpha(1+\gamma_n)^j \notag,
	\end{equation}
	for all $j \le J$.
	In particuar, $k_J = \Theta(n)$.
%	In addition, $fn \in [k_{J-1}, k_J[$.
\end{lemma}
%\merk{A: we do need $k_j \le fn$ to guarantee that $k_j > 0$ (see Lemma~\ref{lem:exp-growth-E0-increase}). We also need this condition for Lemma~\ref{lem:exp-growth-ETj-upper}}
%\merk{B: not yet proof-read}
\begin{proof}
	By definition of $k_j$,
	\begin{equation}
		k_j = k_{j-1} + E_0(k_{j-1})
		= k_{j-1}
			\left(1 + \gamma_n - \gamma_n(a+1)\tfrac{k_{j-1}}{n}-\gamma_n\tfrac{b}{\ln n} - Ak_{j-1}^{-1+B}\right)\notag.
	\end{equation}
	Let $\Gamma_n := 1+\gamma_n-\gamma_n\tfrac{b}{\ln n}$.
	Then,
	\begin{equation}
		k_j = \Gamma_n k_{j-1} \left(1 - \gamma_n\tfrac{a+1}{\Gamma_n}\cdot\tfrac{k_{j-1}}{n}
			- \tfrac{A}{\Gamma_n}\cdot k_{j-1}^{-1+B}\right) \notag.
	\end{equation}
	%\merk{A: we have simpler bound for $\Gamma_n$, i.e., we can avoid the $\tfrac1{1+\gamma_n}$ factor.}
	Clearly, $\Gamma_n \ge (1+\gamma_n)(1-\tfrac{b}{\ln n})$.
	By our assumption on $\gamma_n$, $\Gamma_n$ is bounded from above by a constant and is at least $1+\gamma_n/2$ for $n$ big enough.
	Let hence $\tilde{a} := \gamma_n\tfrac{a+1}{1+\gamma_n/2}$ and $\tilde{A} := \tfrac{A}{1+\gamma_n/2}$.
	Then, for any big $n$,
	\begin{equation}
		k_j \ge (1+\gamma_n) \left(1 -
			%\tfrac1{1+\gamma_n}\cdot
			\tfrac{b}{\ln n}\right)
			k_{j-1} \left(1 - \tilde{a}\tfrac{k_j-1}{n} - \tilde{A}k_{j-1}^{-1+B}\right) \notag.
	\end{equation}
    We assume that $A$ (resp. $\tilde{A}$) and $f$ are small enough such that the expression in the brackets is positive.
    Since $k_0 = 1$, by induction we obtain for all $j$ that
    %\merk{(we need here $k_i > 0$, for all $i < j$, or we must keep $f$ in lemma)}
	\begin{equation}
        k_j \ge (1+\gamma_n)^j (1-\tfrac{b}{\ln n})^j
        	\prod_{i=0}^{j-1}\left(1-\tilde{a}\tfrac{k_i}{n} - \tilde{A}k_i^{-1+B}\right) \notag.
    \end{equation}
    By choosing $f$ and $A$ small enough, we can assume that $k_i > 0$ for all $i < j$.
    \begin{equation}
        k_j \ge (1+\gamma_n)^j (1-\tfrac{b}{\ln n})^j
        	\left(1 - \tilde{a}\sum_{i=0}^{j-1} \tfrac{k_i}{n}
        	- \tilde{A}\sum_{i=0}^{j-1} k_i^{-1+B}\right) \notag.
    \end{equation}

    %\merk{A: the following modification on $J$ ensures that $k_j < fn$.}

    Let $J := \log_{1+\gamma_n}(fn) - \Delta r$ for some positive $\Delta r = O(1)$ determined later.
    For $j \le J$ we have $k_j \le (1+\gamma_n)^j$ by construction, and thus $k_j \le fn$.
    Also we have $(1-\tfrac{b}{\ln n})^j = \Theta(1)$.
    In particular this term is at least $2\alpha$ for some $\alpha > 0$ and all $n$ big enough.

    We show by induction on $j$ that $k_j \ge \alpha(1+\gamma_n)^j$ for all $j \le J$.
    The base for $j=0$ and $k_0=1$ is obvious.
    Let $1 \le j \le J$  and let $k_i \ge \alpha(1+\gamma_n)^i$ for all $i<j$.
    By construction, we have $k_i \le (1+\gamma_n)^i$.
    Therefore,
    \begin{align}
    	k_j & \ge 2\alpha (1+\gamma_n)^j
    		\left(1-\tfrac{\tilde{a}}{n}\sum_{i=0}^{j-1}(1+\gamma_n)^i
    		- \tilde{A}\alpha^{-1+B}\sum_{i=0}^{j-1} (1+\gamma_n)^{i(-1+B)}\right) \notag \\
    	& \ge 2\alpha (1+\gamma_n)^j
    		\left(1 - \tilde{a}\cdot\tfrac{(1+\gamma_n)^{-\Delta r}}{\gamma_n}
    		- \tilde{A}\alpha^{-1+B}\cdot\tfrac1{1-(1+\gamma_n)^{B-1}}\right) \notag.
    \end{align}
    By choosing $\Delta r$ large enough and $\tilde{A}$ (resp. $A$) small enough, we can bound the last two expressions by $1/4$, and obtain
    $$k_j \ge 2\alpha(1+\gamma_n)^j (1-1/4-1/4) = \alpha(1+\gamma_n)^j.$$
\end{proof}

By Lemma~\ref{lem:exp-growth-E0-increase}, the $k_j$ form a non-decreasing sequence.
We say that our homogeneous rumor spreading process is in phase $j$ for $j \in \{0,\ldots,J-1\}$, if the number of informed nodes is in $[k_j, k_{j+1}[$.

%\merk{A: we need the next condition for Lemma~\ref{lem:exp-growth-ETj-upper}.}
%Also w.l.o.g. we assume that $J$ is chosen such that $k_J < fn$.
%\merk{A: we enumerate phases from $0$, since the first phase is $[k_0, k_1[$!}

\begin{lemma}\label{lem:exp-growth-ETj-upper}
	If our process is in phase $j<J$, then the number of rounds to leave phase $j$ is stochastically dominated by $1 + \Geom(1-Q_j)$, where $Q_j := \min\left\{q(k_j), \tfrac1{1+1/q(1)}\right\}$.
%	In addition, $\sum_j Q_j = O(1)$.
%	at most
%	\begin{equation}
%		1 + \tfrac{Q_j}{1-Q_j}
%		\text{, where }
%		Q_j := \min\left\{q(k_j), \tfrac1{1+1/q(1)}\right\} \notag.
%	\end{equation}
\end{lemma}
\begin{proof}
	By Lemma~\ref{lem:exp-growth-E0-increase} we have $k + E_0(k) \ge k_j + E_0(k_j) = k_{j+1}$ for any $k_j \le k < fn$.
	By Lemma~\ref{lem:exp-growth-failure},
	\begin{equation}
		\Pr[k+X(k) \le k_{j+1}]
		< \Pr[k+X(k) \le k+E_0(k)]
		< \min\left\{q(k), \tfrac1{1+1/q(1)}\right\} \notag.
	\end{equation}
	Since $q(k)$ is decreasing,
	\begin{equation}
    	\max_{k_{j+1} > k \ge k_j} \Pr[k + X(k) < k_{j+1}] \le Q_j \notag,
	\end{equation}
	and this is an upper bound for the probability to stay in phase $j$ for one round.
	We can thus bound the number of rounds taken to leave phase $j$ by a random variable with geometric distribution $\Geom(1-Q_j)$.
%	Consequently, the expected number of rounds to leave this phase is at most $1 + \tfrac{Q_j}{1-Q_j}$.
\end{proof}

\begin{lemma}\label{lem:exp-growth-sum-Qj}
	$\sum_{j=0}^{J-1} Q_j = O(1)$.
\end{lemma}
\begin{proof}
    We apply the estimate for $q(k_j)$ from Lemma~\ref{lem:exp-growth-failure} and the bounds for $k_j$ from Lemma~\ref{lem:exp-growth-expk-upper}.
    Therefore,
    \begin{align}
        \sum_{j=0}^{J-1} Q_j
        & \le \sum_{j=0}^{J-1} q(k_j)
        	\le \tfrac{\gamma_n+c}{A^2} \cdot \sum_{j=0}^{J-1} k_j^{-2B+1} \notag \\
        & \le \tfrac{\gamma_n+c}{A^2} \cdot \alpha^{-2B+1} \cdot \sum_{j=0}^{J-1} (1+\gamma_n)^{j(-2B+1)} \notag.
    \end{align}
    The last sum is a decreasing geometric series as $B > 0.5$.
    So, $\sum_j Q_j = O(1)$.
\end{proof}

Now we can prove the main result of this section.

\begin{proof}[Proof of Theorem~\ref{th:exp-growth-upper}]
	By Lemma~\ref{lem:exp-growth-expk-upper}, there exists $J = \log_{1+\gamma_n} n + O(1)$ such that $k_J = \Theta(n)$.
	In the following we assume that $J \le \log_{1+\gamma_n} + \tau$ for some constant $\tau$.
	%So there exists some $0 < \tilde{f} < 1$ such that $k_J > \tilde{f}n$ for all $n$ big enough.
	The phase method allows us to bound the number of rounds until at least $k_J$ nodes are informed.
    We denote by the random variable $T_j$ the number of rounds spent in the $j$th phase.
    By Lemma~\ref{lem:exp-growth-ETj-upper}, $T_j$ is stochastically dominated by $1 + \Geom(1-Q_j)$.
    With Lemma~\ref{lem:exp-growth-sum-Qj}, we compute
    \begin{align}
        \Expect[T(1,k_J)]
        & \le \sum_{j=0}^{J-1} \Expect[T_j] \le \sum_{j=0}^{J-1} (1+\tfrac{Q_j}{1-Q_j}) \notag \\
        & = J + \sum_{j=0}^{J-1} \tfrac{Q_j}{1-Q_j} \le J + \tfrac1{1-Q_0}\sum_{j=0}^{J-1} Q_j \notag \\
        & = J + O(1). \notag
    \end{align}
    Since $Q_j$ is bounded by a geometric sequence, Lemma~\ref{lemma:sum geometrical-2} claims that there exist $A'_1, \alpha'_1$ such that $$\Pr[T(1,k_J) > J + r/2] \le A'_1e^{-\alpha'_1r}.$$
    If $k_J < fn$, then we observe that for all $k \in [k_J, fn[$, $p_k$ satisfies the conditions of Lemma~\ref{lem:general-connect}. Therefore, $T(k_J, fn) = O(1)$ and there exist $A'_2, \alpha'_2$ such that $\Pr[T(k_J, fn) > r/2] \le A'_2e^{-\alpha'_2r}$.
    Combining bounds for $T(1,k_J)$ and $T(k_J,fn)$ we obtain the following.
    \begin{eqnarray*}
    	&& \Expect[T(1,fn)] \le \Expect[T(1,k_J)] + \Expect[T(k_J,fn)]
    			\le \log_{1+\gamma_n} n + O(1), \\
    	&& \Pr[T(1,fn) > \log_{1+\gamma_n}n + r] \le A' e^{-\alpha' r},\,
    	\mbox{where $A' := (A'_1+A'_2)e^{\alpha' \tau}$ and $\alpha' := \min\{\alpha'_1,\alpha'_2\}$}.
    \end{eqnarray*}
\end{proof}

%\begin{corollary}
%	For any $0 < s < 1$, there exists $c > 0$ such that
%	$$\Pr\left[T(\lceil n^s \rceil, fn) > (1-s)\log_{1+\gamma_n}n + c\right] < \tfrac1n.$$
%\end{corollary}
%\begin{proof}
%	Again, there exists $J = \log_{1+\gamma_n} + O(1)$ such that $k_J = \Theta(n)$.
%	Obviously, $T(k_{j_0}, k_J)$ is stochastically dominated by $\sum_{j=j_0}^{J-1} T_j$, where $T_j$ are defined in the proof of Theorem~\ref{th:exp-growth-upper}.
%	Since by the proof of Lemma~\ref{lem:exp-growth-ETj-upper}, $T_j$ is stochastically dominated by a random variable having a distribution $1+\Geom(1-Q_j)$, $T$ is bounded by $J-j_0+\Geom(1-\sum_{j=j_0}^{J-1}Q_j)$.
%\end{proof}

%\begin{corollary}
%	There exist $A, \alpha > 0$ such that for any integer $r > 0$ we have
%	$$\Pr[T(1,fn) \le \E[T(1,fn)] + r] \le Ae^{-\alpha r}.$$
%\end{corollary}
%\begin{proof}
%	By Lemma~\ref{lemma:sum geometrical-2} and the proof of Theorem~\ref{th:exp-growth-upper}, we have for some $A', \alpha' > 0$
%	$$\Pr[T(1,k_J) > J + r] \le A'e^{-\alpha'r}.$$
%	If $k_J < fn$, then we observe that for all $k \in [k_J, fn[$, $p_k$ satisfies the conditions of Lemma~\ref{lem:general-connect}, and, thus, from~\eqref{eq:interconnection} it follows that
%	$$\Pr[T(k_J,fn) > r] \le O(1)\cdot \left(1-\max_{k \in [k_J, fn[} p_k\right)^r.$$
%	The claim of corollary directly follows from the fact that both probabilities above are exponentially small in $r$.
%\end{proof}
%\merk{the last equation is a very little inaccurate, since we don't know that we start at $k_J$}
%===========================================================================================
%================           LOWER   BOUND         ==========================================
%===========================================================================================

\subsection{Exponential Growth Regime. Lower Bound}

In this section, we prove a lower bound for an exponential growth regime. We formulate a condition matching the upper bound condition and show that this leads to a lower bound on the rumor spreading time that matches the upper bound apart from a constant number of rounds. We use again the target-phase method.

This is the first time that the target-phase argument is used to prove a lower bound. In the work closest to ours,~\cite{DoerrK14}, only the classic push protocol was regarded. Consequently, there, the simple argument that the number of nodes can at most double each round was sufficient to obtain a lower bound for the growth regime. Such an argument, e.g., is not possible for the classic pull protocol.

The main difference to the upper bound proof lies in the final argument. In the upper bound proof, the failure to reach a round target simply resulted in that we had to try again to reach this target. For the lower bound, a failure is that the process gains more than one phase in one round, resulting in that the time usually spent in these now skipped phases is spared. Arguing that the total time spared by such events is only $O(1)$ needs a slightly more complicated book-keeping of the failure events and a slightly more complicated final argument.

\subsubsection{Exponential Growth Conditions}

We formulate the lower exponential growth condition in an analoguous way as the upper one. In particular, the covariance condition is identical.

\begin{defn} [lower exponential growth conditions] \label{def:lower-exp-growth-conditions}
    Let $\gamma_n$ be bounded between two positive constants and let $a, b, c \ge 0$ and $0 < f < 1$.
    We say that a homogeneous epidemic protocol satisfies the \emph{lower exponential growth conditions} in $[1,fn[$ if for any $n \in \N$ big enough, the following properties are satisfied for any $k < fn$.
	\renewcommand{\theenumi}{(\roman{enumi})}%
    \begin{enumerate}
        \item $p_k \le \gamma_n \tfrac{k}{n} \cdot \left(1 + a\tfrac{k}{n} + \tfrac{b}{\ln n}\right)$.
        \item $c_k \le c\tfrac{k}{n^2}$.
    \end{enumerate}
\end{defn}

These conditions imply the following lower bounds on the rumor spreading time.
\begin{theorem}%[lower bound for the spreading time]
\label{th:exp-growth-lower}
	Consider a homogeneous epidemic protocol satisfying the lower exponential growth conditions in $[1,fn[$. Then there are constant $A', \alpha'>0$ such that 
	\begin{eqnarray*}
    &&\Expect[T(1, fn)] \ge \log_{1+\gamma_n} n - O(1),\\
    &&\Pr[T(1, fn) \le \log_{1+\gamma_n} n - r] \le A' \exp(-\alpha'r)\, \mbox{ for all $r \in \N$}.
  \end{eqnarray*}
  In addition there exists $f' \in ]f,1[$ such that with probability $1-O\left(\tfrac1n\right)$ there are at most $f'n$ informed nodes after $T(1,fn)$ rounds.
\end{theorem}

\subsubsection{Round Targets and Failure Probabilities}

As above, we consider a round with $k$ informed nodes initially.
We define $X(k)$ to be the number of newly informed nodes in this round.
Since  $\Expect[X(k)] = \Pk (n-k)$, the exponential growth conditions give $\Expect[X(k)] \le E(k)$ with
\begin{equation}
	E(k) := \gamma_n k \left(1 + a\tfrac{k}{n} + \tfrac{b}{\ln n}\right). \notag
\end{equation}
Note that we could replace the $a$ above by $a-1$, giving an expression closer resembling the corresponding one from the previous section. Since all these constants do not matter, we preferred the simpler version without the extra~$-1$.

Like in the previous section we introduce
\begin{equation}
	E_0(k) := E(k) + Ak^B, \label{eq:expgrowth-def-E0-lower}
\end{equation}
where $A > 0$ and $B \in ]0.5, 1[$ are some constants chosen uniformly for all values of $k$ and $n$.
Unlike in Section~\ref{section:exp-growth-upper}, it is obvious that $E(k)$ and $E_0(k)$ are increasing.

Note that we can freely replace $f$ in the definition of the lower exponential growth conditions by a smaller constant $f'$, since showing $\E[T(1,f'n)] \ge \log_{1+\gamma_n}(n) - O(1)$ in Theorem~\ref{th:exp-growth-lower} would immediately imply $\E[T(1,fn)] \ge \log_{1+\gamma_n}(n) - O(1)$. Consequently, let us assume that $f$ is small enough such that for any $n$ sufficiently large and $k<fn$,
\begin{equation}
    E(k) \le 2\gamma_n k. \label{eq:exp-growth-lower-42}
\end{equation}

%\merk{the following with need some cleaning}
%Moreover, there exist $f \in ]0, 1[$ and $A' > 0$ such that
%\begin{equation}
%    E(k) \le 2\gamma_n k, \label{eq:exp-growth-lower-42}
%\end{equation}
% or any $A < A'$ and for any $n$ large enough. To ease our notation we assume in the following that $f < f'$, $n$ is sufficiently large and we choose $A$ such that $0 < A < A'$.
%As it was made for the upper bound, we suppose $A$ to be as small as possible to provide the correct cut of the interval $[1,fn]$ into phases.
%Also we suppose that $f < f'$ defined in the following obvious lemma.
%We pick the values of constants such that $E_0(k) \ge 0$, for any $1 \le k < fn$ for some $f \in ]0, 1[$.
%\merk{todo: check the conditions.}
%\begin{lemma}\label{lem:exp-growth-increasing}
%    $k + E_0(k)$ increases on $k$.
%\end{lemma}
%\begin{proof}
%	As the probabilities of becoming informed are the same for all nodes,
%	$$\Expect[X(k)] = \sum_{i=1}^{n-k} \Pr[X_i=1] \ge (n-k) \tfrac{k}{n} \cdot \gamma_n \left(1 - a\tfrac{k}{n} - \tfrac{b}{\ln n}\right).$$
%	The claim follows from the disclosure of the parenthesis.
%\end{proof}

%Let us introduce $E_0(k) := E(k) - Ak^B$ for some chosen values of $A > 0$ and $B \in ]0.5, 1[$.
%We say that the round is successful if it informs at least $E_0(k)$ new nodes, otherwise we say about round "failure".
%We suppose that $B$ is fixed since now on e.g. 3/4, so in the following we will discuss only the choice of $A$.
%
%The motivation of $E_0(k)$ is the following.
%Lemma~\ref{lem:exp-growth-failure} uses Chebyshev's \merk{(Cantelli?)} inequality to prove that the probability of failure is of order $O(k^{1-2B})$.
%Then, introducing the sequence $k_{j+1} := k_j + E_0(k_j)$, we show in Lemma~\ref{lem:exp-growth-kj} that its elements grow exponentially.
%Lemma~\merk{lem:exp-growth-prob-next-phase} argues that the number of informed nodes overcomes the next element $k_{j+1}$ in one round with probability at least $1-O(k_j^{1-2B})$.
%The two last result yield the upper bound for the number of rounds in the exponential growth phase.
%Our goal is to prove that $X(k)$ is at least $E_0(k)$ using the Chebyshev's inequality.
%So the role of $Ak^B$ is some kind of threshold: if it is relatively big then the "failure" probability is relatively small, but we can guarantee the relatively slow progress of the epidemic process.
%On the other hand, the smaller values of $Ak^B$ will make the successful steps longer, but the probabilities of the "failures" will increase.
%\merk{reformulate, difficult to understand, because the phases are not introduced yet}

%\merk{modify this lemma - we need $E(k) \le 2\gamma_n k$}
%\begin{lemma}\label{lem:exp-growth-E0-increase}
%There exists some $0 < f' < 1$ such that if $n$ is big enough, then for any $k \in [1, f'n[$
%\begin{itemize}
%	\item $E(k) \le 2\gamma_n k$,
%	\item $E(k)$ and $E_0(k)$ are increasing on $k$.
%\end{itemize}
%\end{lemma}
%\merk{Formally $E(k)$ and $E_0(k)$ are both increasing for any values of $k$.}

%\merk{we need the probability to go further than the next phase}
%\begin{lemma}\label{lem:exp-growth-failure}
%    Let the exponential growth conditions are satisfied.
%    For any $k < fn$ and $n$ big enough, $\Pr[X(k) \ge E_0(k)] \ge q(k)$, where $$q(k) = \tfrac{2\gamma_n + c}{A^2} \cdot k^{1-2B}.$$
%\end{lemma}
%\begin{proof}
%    By the exponential growth conditions, $\Expect[X(k)] \le E(k)$.
%    Then, applying Chebyshev's inequality, we get the following.
%    \begin{align*}
%        \Pr&[X(k) \ge E_0(k)]
%    	   = \Pr\left[X(k) \ge E(k) \cdot \left(1+\tfrac{Ak^B}{E(k)}\right)\right] \\
%        & \le \Pr\left[X(k) \ge \Expect[X(k)] + Ak^B\right]
%            \le \tfrac{\Var[X]}{(Ak^B)^2} \notag
%    \end{align*}
%    From the the covariance condition of the exponential growth, it follows that $\Var[X(k)] \le \Expect[X(k)] + ck$.
%    Therefore, taking into account that $E(k) \le 2\gamma_n k$,
%    \begin{equation}
%        \Pr[X(k) \le E_0(k)]
%        \le \tfrac{\Expect[X(k)] + ck}{(Ak^B)^2}
%        \le \tfrac{2\gamma_n + c}{A^2} \cdot k^{1-2B} \notag
%    \end{equation}
%\end{proof}

The following lemma will later be used to argue that an unexpectedly fast progress is unlikely. Different from the upper bound analysis in the previous section, we now need a failure probability for different excessive progresses (quantified by the parameter $h$ below).

\begin{lemma}\label{lem:exp-growth-failure-lower}
    For any $k < fn$ and $h = 0, 1, 2, \ldots$,
    \begin{equation}
    	\Pr[X(k) \ge E(k) + Ak^B(1+\gamma_n)^h]
    	\le q_h(k) := \tfrac{2\gamma_n + c}{A^2} \cdot \tfrac{k^{-2B+1}}{(1+\gamma_n)^{2h}} \notag.
    \end{equation}
\end{lemma}
\begin{proof}
    By the exponential growth conditions, $\Expect[X(k)] \le E(k)$.
    By the covariance condition and~\eqref{eq:exp-growth-lower-42},
    \begin{equation}
        \Var[X(k)] \le E(k) + n^2 c_k \le k(2\gamma_n+c) \notag.
    \end{equation}
    Applying Chebyshev's inequality, we obtain
    \begin{align}
    	\Pr&[X(k) \ge E(k) + Ak^B(1+\gamma_n)^h] \notag \\
    	& \le \Pr[X(k) \ge \Expect[X(k)] + Ak^B(1+\gamma_n)^h] \notag \\
    	& \le \tfrac{\Var[X(k)]}{(Ak^B)^2(1+\gamma_n)^{2h}} \notag \\
    	& \le \tfrac{2\gamma_n+c}{A^2} \cdot k^{-2B+1} \cdot \tfrac1{(1+\gamma_n)^{2h}} \notag.
    \end{align}
\end{proof}
%\erk{the same trick as in Lemma~\ref{lem:exp-growth-failure} seems to be impossible, as now $\Expect[X(k)] \le E(k)$.}
%\merk{think if needed}
%Finally we remark here that $q_h(k) < 1$, for any $k > k_0$, where $k_0$ depends only on the process parameters and the choice of constants $A, B$.

%\merk{todo. Think about Cantelli's inequality. It is easier to start from $k_0$ big enough to ensure all probabilities to be less than 1}

%In the case when $q(k) \ge 1$, one can replace Chebyshev's inequality by the Cantelli's one to obtain the following.
%\begin{corollary}\label{cor:exp-growth-error-prob}
%	$\Pr[X(k) \le E_0(k)] \le min\left\{q(k), \tfrac1{1+1/q(1)}\right\}$.
%%    \begin{equation}
%%        \Pr[X(k) \le E_0(k)] \le min\left\{q(k), \tfrac1{1+1/q(1)}\right\} \notag
%%    \end{equation}
%\end{corollary}

%\begin{remark*}
%    Let us denote by $q(k) := \tfrac{\gamma_n}{A^2} \cdot k^{1-2B}$, the obtained above bounds for the failure probability.
%    For smaller values of $k$, $q(k)$ might be more than one.
%    In this case it suffices to apply the Cantelli inequality \merk{(ref?!)} instead of Chebyshev's one and to replace $q$ by $q' \leftarrow \tfrac1{1+1/q}$.
%    One can easily see that $q' < 1$ and $q' < q$.
%\end{remark*}

\subsubsection{The Phase Calculus}
Like in Section~\ref{section:exp-growth-upper}, we define the sequence $k_j$ recursively by
\begin{equation}
	k_0 = 1, \quad k_{j+1} := k_j + E_0(k_j) \notag,
\end{equation}
and obtain the following exponential growth behavior.

%In previous section we proved that if at the beginning of the round there were $k$ informed nodes, then at the end their number will be at least $k + E_0(k)$ with probability at least $1-q(k)$.
%Let us hence introduce the sequence $k_j$, defined recursively as $k_0 = 1$ and $k_{j+1} := k_j + E_0(k_j)$.
%Clearly, if we choose $A$ small enough, then $E_0(k_j)$ still be positive for any $1 \le k < fn$, and $k_j$ is an increasing sequence.
%So it splits $[1, fn]$ into consecutive intervals called phases.
%Moreover, from Lemma~\ref{lem:exp-growth-E0-increase} it follows that $k + E_0(k)$ is an increasing function, in particular $\forall k \ge k_j$, $k + E_0(k) \ge k_j + E_0(k_j) = k_{j+1}$.
%So by Lemma~\ref{lem:exp-growth-failure}, with probability at least $1-q(k)$ the process leaves the current phase in current round.
%Let us consider
%\begin{equation}
%    Q_{j+1} := \max_{k_{j+1} > k \ge k_j} \Pr[k + X(k) < k_{j+1}],
%\end{equation}
%the upper bound for the probability to leave the $j$th phase in one round.
%$q(k)$ is decreasing function, so $Q_{j+1} \le min\left\{q(k_j), \tfrac1{1+1/q(1)}\right\}$ (the second term raises from Corollary~\ref{cor:exp-growth-error-prob} to avoid problems with smaller values of $k_j$).

%So, we can bound the time spent in each phase by the geometrically distributed variable $\Geom(1-Q_{j+1})$.
%The expected number of rounds to leave this phase is at most $ 1 + Q_{j+1}/(1-Q_{j+1})$.
%Therefore, the expected exponential growth time is at most $r + \sum_{j=0}^r Q_{j+1}/(1-Q_{j+1})$, where $r$ is the number of phases in the exponential growth regime.
%To find $r$ we need to study the behavior of $k_j$ first.
%\begin{lemma}\label{lem:exp-growth-expk-lower}
%    There exists some $f \in ]0, 1[$ and $\Delta r > 0$, such that for any $n$ big enough, the sequence $k_j$ splits the interval $[1, fn]$ into at most $\log_{1+\gamma_n} n - \Delta r$ phases by the sequence $k_j$.
%    Moreover,
%    $$k_j \ge k_0 \cdot (1+\gamma_n)^j \cdot O(1), \quad \forall j \le \log_{1+\gamma_n}n - \Delta r.$$
%    %$r := \log_{1+\gamma_n} n + const$, $const$ to determine.
%    %Then for any $j \le r$, $k_j \le (1+\gamma_n)^j \cdot O(1)$.
%\end{lemma}
%\merk{can also write: there exists $const > 0$ such that $$k_j \ge k_0 \cdot (1+\gamma_n)^j \cdot const, \quad \forall j \le \log_{1+\gamma_n}n - \Delta r,$$ but it seems to be to many constants for one lemma.}
%
%\merk{the constructions look more ugly with $k_0 \ne 1$. Can we remove it and multiply just in the end?..}

\begin{lemma}\label{lem:exp-growth-expk-lower}
	By taking $A$ small enough in~\eqref{eq:expgrowth-def-E0-lower}, there exist $\alpha > 0$ and $J = \log_{1+\gamma_n}n - O(1)$ such that for all $j < J$
	\begin{equation}
		(1+\gamma_n)^j \le k_j \le \alpha(1+\gamma_n)^j \text{ and } \; \text k_j < fn \notag.
	\end{equation}
	%In particular, $k_J = \Theta(n)$.
\end{lemma}
%\merk{it is hard to understand without looking the proof that this lemma affects the choice of $A$.}
\begin{proof}
    %Indeed, we observe first that $k_j \ge (1+\gamma_n)^j$.
    Note that %$k_J = \Omega(n)$, since 
    $k_j \ge (1+\gamma_n)^j$ is immediate from the definitions and a simple induction.
    So it remains to show the upper bound on the $k_j$.
%    \merk{A: not sure that I understood correctly the sentence above from your comments...}
    Clearly, by definition of $k_j$,
    \begin{equation}
        k_j \le (1+\gamma_n) (1+\tfrac{b}{\ln n}) k_{j-1} \left(1 + a\tfrac{k_{j-1}}{n}\right)
        	\left(1 + Ak_{j-1}^{-1+B}\right) \notag.
	\end{equation}
    Since $k_0 = 1$, by induction we obtain
	\begin{equation}
        k_j \le (1+\gamma_n)^j (1+\tfrac{b}{\ln n})^j \prod_{i=0}^{j-1}\left(1 + a\tfrac{k_i}{n}\right)
        	\prod_{i=0}^{j-1}\left(1 + Ak_i^{-1+B}\right). \notag %\label{eq:expk-1}
	\end{equation}
	Let $J := \log_{1+\gamma_n}n - \Delta r$ for some $\Delta r = O(1)$ determined later.
    If $j < J$, then $(1-\tfrac{b}{\ln n})^j = \Theta(1)$.
    In particular, it is at most $\tfrac\alpha4$ for some $\alpha > 0$ and any $n$ big enough.
	By the fact that $1+x \le e^x$ for any $x > 0$, we have
	\begin{equation}
		k_j \le \tfrac\alpha4 (1+\gamma_n)^j
		\exp\left(\sum_{i=0}^{j-1}a\tfrac{k_i}{n}\right)
		\cdot \exp\left(\sum_{i=0}^{j-1}Ak_i^{-1+B}\right) \label{eq:exp-growth-upper-eq1}.
	\end{equation}
%	Since $(1+\tfrac{b'}{\ln n})^j \to c := \exp(\tfrac{b'}{\ln(1+\gamma_n)})$, when $j \to \log_{1+\gamma_n} n$, we know that if $j \le \log_{1+\gamma_n}n - 1$, then $(1+\tfrac{b'}{\ln n})^j \le c$.
	We prove the claim of lemma by induction on $j$. Assume that for some $j < J$ we have $k_i \le \alpha(1+\gamma_n)^i$ for any $i < j$.
%	We can assume $j \ge 2$.
	Since $k_i \ge (1+\gamma_n)^i$ for all $i$, both sums in~\eqref{eq:exp-growth-upper-eq1} can be bounded by geometric series.
	Therefore,
	\begin{equation}
		k_j \le \tfrac\alpha4 (1+\gamma_n)^j
		\exp\left(\sum_{i=0}^{j-1}\tfrac{a}{n} \cdot \alpha(1+\gamma_n)^i\right)
		\cdot \exp\left(\sum_{i=0}^{j-1}A(1+\gamma_n)^{i(-1+B)}\right) \notag.
	\end{equation}
	Since $j < J$, by choosing $\Delta r$ large enough and $A$ small enough, we can bound both sums by any positive constant, in particular by $\ln 2$.
	Therefore, for any $j < J$,
	\begin{equation}
		k_j \le \tfrac\alpha4(1+\gamma_n)^j \exp(\ln2) \cdot \exp(\ln 2) = \alpha(1+\gamma_n)^j\notag.
	\end{equation}
%	Observe that the second summands in both brackets form the geometric series.
%	Thus,
%	\begin{equation}
%		\tfrac{a'c}{n}\sum_{i=1}^{j-1} (1+\gamma_n)^i
%		\le \tfrac{a'c}{n}(1+\gamma_n)^j/\gamma_n
%		\le \tfrac12 \notag,
%	\end{equation}
%	if $j \le \log_{1+\gamma_n} n - \Delta r$ for the appropriate choice of $\Delta r$.
%	On the other hand,
%	\begin{equation}
%		A' \sum_{i=0}^{j-1} (1+\gamma_n)^{i(B-1)}
%		\le A' \tfrac1{1-(1+\gamma_n)^{B-1}}
%		\le \tfrac12 \notag,
%	\end{equation}
%	if $A'$ (resp. $A'$) is small enough.
%	Finally, by Lemma~\ref{lemma:Bernoulli}, both products are at most 2, so $k_j \le 4k_0c(1+\gamma_n)^j$.
\end{proof}

By definition, the $k_j$ form a non-decreasing sequence.
Like in Section~\ref{section:exp-growth-upper}, we say that the rumor spreading process is in phase $j$ for $j = 0, \ldots, J-1$, if the number of informed nodes is in $[k_j, k_{j+1}[$.

%\merk{A: Indeed, we don't need $j < J-h$ for the next lemma. But if $j+h \ge J$, the corresponding phase is not well defined, so we say "the probability that the number of informed nodes is at least $k_{j+h}$ at the end of the round".}
\begin{lemma}\label{lem:exp-growth-jump-prob}
	Let $h \ge 2$.
	If the process is in phase $j < J$ at the beginning of one round, then the probability that the number of informed nodes is at least $k_{j+h}$ at the end of the round, is at most $q_{h-2}(k_j)$.
\end{lemma}
\begin{proof}
	For $1 \le k \le k_{j+1}$, we have $$k + E(k) + Ak^B \le k_{j+1} + E(k_{j+1}) + Ak_{j+1}^B = k_{j+2}.$$
	Since $k_{j+h} \ge (1+\gamma_n)^{h-2} k_{j+2}$, we have
	\begin{equation}
		k_{j+h}
		\ge (1+\gamma_n)^{h-2} \left(E(k) + Ak^B + k\right)
		\ge k + E(k) + Ak^B(1+\gamma_n)^{h-2} \notag.
	\end{equation}

    By Lemma~\ref{lem:exp-growth-failure-lower}, the maximum probability to have at least $k_{j+h}$ informed nodes at the end of the round is
    %``jump" from phase $j$ to phase $j+h$ (or further) is
    \begin{align}
        & \max_{k\in [k_j, k_{j+1}[} \Pr[k+X(k) \ge k_{j+h}] \notag \\
    	& \le \max_{k\in [k_j,k_{j+1}[} \Pr[k+X(k)\ge k+E(k)+Ak^B(1+\gamma_n)^{h-2}] \notag \\
    	& \le \max_{k\in [k_j,k_{j+1}[} q_{h-2}(k) \le q_{h-2}(k_j) \notag.
    \end{align}
    The last inequality follows from the fact that since $B > 1/2$, $q_{h-2}(\cdot)$ decreases.
%    The last expression is increasing on $k$, so the sought probability is at most $q_h(k_j)$.
\end{proof}

%\begin{lemma}\label{lem:exp-growth-sum-Qj}
%	Let $r = \log_{1+\gamma_n} n - \Delta r$ satisfies the conditions of Lemma~\ref{lem:exp-growth-expk}.
%	Then, $\sum_{j=1}^r Q_j = O(1)$.
%\end{lemma}
%\begin{proof}
%    Indeed, $\sum Q_j \le \sum q(k_j)$.
%    Then we apply the estimate for $q(k_j)$ from Lemma~\ref{lem:exp-growth-failure} and the bounds for $k_j$ from Lemma~\ref{lem:exp-growth-expk}.
%    Therefore,
%    \begin{align}
%        \sum_{j=1}^r Q_j
%        & \le \left(1+\tfrac{2c}{\gamma_n}\right) \cdot \tfrac{\gamma_n}{A^2} \cdot \sum_{j=1}^r k_j^{1-2B} \notag \\
%        & \le O(1) \cdot \sum_{j=1}^r (1+\gamma_n)^{j(1-2B)}.
%    \end{align}
%    The last sum is the sum of decreasing geometric series, as $B > 0.5$.
%    So, $\sum_j Q_j = O(1)$.
%\end{proof}%

With Lemma~\ref{lem:exp-growth-expk-lower}~and~\ref{lem:exp-growth-jump-prob}, we can now prove Theorem~\ref{th:exp-growth-upper}.%\merk{B: I'll write some clever text here some time}

\begin{proof}[Proof of Theorem~\ref{th:exp-growth-lower}]
	Let $S$ be the set of visited phases, e.g., if the process does not jump over any phase, then $S = \{0,\ldots,J-1\}$.
	By $\tau_j$ we denote the number of rounds spent in the $j$th phase.
	So the spreading time $T(k_0, k_J) = \sum_{j\in S} \tau_j$.
	We do not know the size of $S$, so in order to bound the spreading time below, let us introduce the random variable $\Delta_j$ which is equal to the length of the jump from the $j$th phase when the process leaves it.
	Let also $d_j := \Delta_j - \tau_j$.
	Since $\sum_{j\in S} \Delta_j = J$, we have $T(k_0, k_J) = J - \sum_{j\in S} d_j$.
%	Let us study the distribution of $d_j$.
	By definition, for $j \in S$ and $h > 0$, we have $\Pr[d_j \ge h] \le \Pr[\Delta_j \ge h+1]$.
%	\begin{equation}
%		\Pr[d_j \ge h]
%		= \sum_{t \ge 1} \Pr[\Delta_j \ge h + t \text{ and } \tau_j = t]
%		\le \sum_{\Delta h \ge 1} \Pr[\Delta_j \ge h + t] \notag
%	\end{equation}
	Then, by Lemma~\ref{lem:exp-growth-failure-lower}~and~\ref{lem:exp-growth-jump-prob},
	\begin{equation*}
		\Pr[d_j \ge h]
		\le q_{h-1}(k_j)
		\le \tfrac{2\gamma_n+c}{A^2} \tfrac{k_j^{-2B+1}}{(1+\gamma_n)^{2h-2}}.
	\end{equation*}
%	\begin{align}
%		\Pr&[d_j \ge h]
%		\le \sum_{t \ge 1}
%			\tfrac{2\gamma_n+c}{A^2}\cdot\tfrac{k_{j-1}^{-2B+1}}{(1+\gamma_n)^{h+t-1}} \notag \\
%		& \le \tfrac{2\gamma_n+c}{A^2} \tfrac{k_{j-1}^{-2B+1}}{(1+\gamma_n)^h}
%			\sum_{t \ge 0} (1+\gamma_n)^t
%		\le k_{j-1}^{-2B+1} \cdot \tfrac{O(1)}{(1+\gamma_n)^h} \notag.
%	\end{align}
  The above argument shows that $T(k_0, k_J)$ stochastically dominates $J - D$, where $D$ is the sum of independent non-negative integer random variables $D = \sum_{j=0}^{J-1} D_j$ satisfying $\Pr[D_j \ge h] \le \tfrac{2\gamma_n+c}{A^2} \tfrac{k_j^{-2B+1}}{(1+\gamma_n)^{2h-2}}$ for all $h \ge 1$. Let $R_h := \{(r_0, \dots, r_{J-1}) \in \Z^J_{\ge 0} \mid \sum_{j=0}^{J-1} r_i = h\}$ for all $h \ge 1$. We compute
  \begin{align*}
  \Pr[D \ge h] & \le \sum_{r \in R_h} \prod_{j = 0}^{J-1} \Pr[D_j \ge r_j] \\
  &\le (1+\gamma_n)^{-2h} \sum_{r \in R_h} \prod_{j \in [0..J-1], r_j >0} \tfrac{2\gamma_n+c}{A^2} \tfrac{k_j^{-2B+1}}{(1+\gamma_n)^{-2}}\\
  &\le (1+\gamma_n)^{-2h} \sum_{M \subseteq [0..J-1]} \prod_{j \in M} \tfrac{2\gamma_n+c}{A^2} \tfrac{k_j^{-2B+1}}{(1+\gamma_n)^{-2}}\\
  &\le (1+\gamma_n)^{-2h} \prod_{j \in [0..J-1]} \bigg(1+\tfrac{2\gamma_n+c}{A^2} \tfrac{k_j^{-2B+1}}{(1+\gamma_n)^{-2}}\bigg)\\
  &\le (1+\gamma_n)^{-2h} \exp\bigg(\sum_{j \in [0..J-1]} \tfrac{2\gamma_n+c}{A^2} \tfrac{k_j^{-2B+1}}{(1+\gamma_n)^{-2}}\bigg)\\
  &\le (1+\gamma_n)^{-2h} O(1),
\end{align*}
	where the last estimate uses  Lemma~\ref{lem:exp-growth-expk-lower}. This proves that tail bound statement. For the claim on the expected rumor spreading time, we compute
	\begin{align*}
		\Expect[D] \le \sum_{h \ge 1} \Pr[D \ge h] \le \sum_{h \ge 1} (1+\gamma_n)^{-2h} O(1) = O(1).
	\end{align*}
  Finally, by Lemma~\ref{lem:general-connect-lower}, there exists $f' \in ]f,1[$ such that with probability $1-O\left(\tfrac1n\right)$ there are at most $f'n$ informed nodes after $T(1,fn)$ rounds.
\end{proof} 

%\input{exp_growth_old.tex}

{\sloppy
 %-----------------------------------------------------------------------------------
%-----------------      Exponential Shrinking Regime. Upper Bound       -----------------------
%-----------------------------------------------------------------------------------

\subsection{Exponential Shrinking Regime. Upper Bound}

We now regard the regime that at most $gn$, $g$ a small constant, nodes are not informed, and that in each round each of these nodes has an approximately constant chance of becoming informed. From a very distant point of view, this part of the process vaguely resembles the exponential growth regime with time running backwards, but the details are too different to simply transfer our previous results to this setting.

We start in this section with the upper bound on the runtime. Throughout this section, we assume that our homogeneous epidemic protocol satisfies the following \emph{upper exponential shrinking conditions} including the covariance condition.

\begin{defn}[upper exponential shrinking conditions] \label{def:upper-exp-shrinking-conditions}
    Let $\rho_n$ be bounded between two positive constants.
	Let $0 < g < 1$ and $a, c \in \R_{\ge0}$ such that $e^{-{\rho_n}} + ag < 1$.
	We say that a homogeneous epidemic protocol satisfies \emph{the upper exponential shrinking conditions} if for any $n \in \N$ big enough, the following properties are satisfied, for all $u = n-k \le gn$.
	\renewcommand{\theenumi}{(\roman{enumi})}%
	\begin{enumerate}
		\item
			$1-p_k = 1-p_{n-u} \le e^{-{\rho_n}} + a\frac{u}{n}$;
		\item
			$c_k = c_{n-u} \le \frac{c}{u}$.
	\end{enumerate}
\end{defn}
Let us note that in this section we study the number of uninformed nodes $u := n-k$ instead of $k$, i.e., the number of informed ones.
We will show that $u$ shrinks by almost a constant factor each round.
So the main result of the section is the following theorem.

\begin{theorem}[upper bound for spreading time] \label{th:exp-shrinking-upper}
	Consider a homogeneous epidemic protocol satisfying the upper exponential shrinking conditions.
  Then there are constant $A', \alpha' > 0$ such that
	\begin{eqnarray*}
		&& \Expect[T(n-\lfloor g n \rfloor, n)] \le \tfrac1{\rho_n}\ln n + O(1), \\
		&& \Pr[T(n-\lfloor g n \rfloor, n) > \tfrac1{\rho_n}\ln n + r] \le A' e^{-\alpha r}\,
			\mbox{ for all $r \in \N$}.
	\end{eqnarray*}
\end{theorem}

We first note that the upper exponential shrinking conditions imply that nodes remain uninformed with at most a constant probability. Hence Lemma~\ref{lem:general-connect} shows that we reach any constant fraction of uninformed nodes in expected constant time. For this reason, we may conveniently assume that \emph{$g$ is an arbitrarily small constant} in the following. We shall also always assume that \emph{$n$ is large enough}.

The proof below follows the general principle established in this work, that is, we define for each number $u$ of uninformed nodes a suitable target $E_0(u)$ such that with sufficiently high probability $1-q(u)$ (following from the covariance condition and Chebyshev's inequality), one round started with at most $u$ uninformed nodes ends with at most $E_0(u)$ uninformed nodes. The choice of $E_0(u)$ is such that the sequence $u_0 = gn, u_1 = E_0(u_0), u_2 = E_0(u_1), \dots$ within $J = \frac 1{\rho_n} \ln(n) + O(1)$ steps reaches a constant $u_J$ and such that failure probabilities $q(u_i)$, $i = 0, \dots, J-1$, imply that only an expected constant number of rounds in addition to $J$ are needed to reach at most $u_J$ nodes. For the constant number of  $u_J$ or less remaining uninformed nodes, we use the simple waiting time argument that each of them needs an expected constant number of rounds to be informed, adding another constant number of rounds to the expected spreading time.


\subsubsection{Round Targets and Failure Probabilities}

Let us introduce the random variable $Y(u)$ being equal to the number of uninformed nodes at the end of a round started with $u$ uninformed ones.
Since $\Expect[Y(u)] = u(1-p_{n-u})$, the exponential shrinking conditions imply that
\begin{equation}
	\Expect[Y(u)] \le E(u) := u\left(e^{-{\rho_n}} + a\tfrac{u}{n}\right) \notag.
\end{equation}

As before, the Lemma~\ref{lem:exp-shrinking-failure-upper} shows that with good probability, $Y(u)$ is less than the $\emph{target value}$
\begin{equation}
	E_0(u) := E(u) + A u^{1-B} \label{eq:def-E0-upper-expshr},
\end{equation}
where $A > 0$ and $0 < B < 1/2$ are some constants chosen uniformly for all values of $u$ and $n$.
In addition we will choose $g$ and $A$ small enough (relative to $g$) to ensure that for all $u \le gn$, the target value $E_0(u)$ is less than $u$ (see Lemma~\ref{lem:exp-shrinking-E0-decrease}) and that the "chain" of consequent target values forms an exponentially decreasing sequence (see Lemma~\ref{lem:exp-shrinking-expk-upper}).

\begin{lemma}\label{lem:exp-shrinking-E0-decrease}
	Assume that $g$ and $A$ are sufficiently small constants. Then for all $u \in [1,gn]$, we have $E_0(u) < u$.
\end{lemma}
\begin{proof}
	Indeed, it suffices to show that
	\begin{equation}
		\tfrac{E_0(u)}{u} = e^{-{\rho_n}} + a\tfrac{u}{n} + Au^{-B} < 1 \notag.
	\end{equation}
	Since $u \in [1, gn]$, we have
	\begin{equation}
		\tfrac{E_0(u)}{u} \le e^{-{\rho_n}} + ag + A \notag.
	\end{equation}
	Clearly there exist positive $A$ and $g$ small enough such that the expression above is less than 1.
\end{proof}

We assume in the following that $g$ and $A$ are small enough to make the assertion of the lemma above true. We compute the target failure probabilities as follows.

\begin{lemma} \label{lem:exp-shrinking-failure-upper}
	For any $1 \le u < gn$,
	\begin{equation}
		\Pr[Y(u) \ge E_0(u)] \le q(u) := \tfrac{(1+a)e^{-{\rho_n}}+c}{A^2} \cdot \tfrac1{u^{1-2B}}\notag.
	\end{equation}
\end{lemma}
\begin{proof}
	Like in the proofs of Lemma~\ref{lem:exp-growth-failure}~and~\ref{lem:exp-growth-failure-lower}, using Chebyshev's inequality and taking into account $E(u) \ge \Expect[Y(u)]$, we compute
	\begin{align}
		\Pr&[Y(u) \ge E_0(u)]
		\le \Pr\left[Y(u) \ge \Expect[Y(u)] + Au^{1-B}\right]
		\le \tfrac{\Var[Y(u)]}{(Au^{1-B})^2} \notag.
	\end{align}
	From Lemma~\ref{lem:prelim:variance} and the covariance condition it follows that
	\begin{equation}
		\Var[Y(u)] \le \Expect[Y(u)] + cu \le \E[Y(u)] + cu \notag.
	\end{equation}
	Therefore,
	\begin{align}
		\Pr[Y(u) \ge E_0(u)]
			% \le \tfrac{\Expect[Y(u)] + cu}{A^2u^{2-2B}}
			\le \tfrac{E(u) + cu}{A^2u^{2-2B}}
			\le \tfrac{(1+a)e^{-{\rho_n}}+c}{A^2} \cdot \tfrac1{u^{1-2B}} \notag.
	\end{align}
\end{proof}

%\merk{The Cantelli inequality is needed only to show that $q(u) < 1$, for any $u$. Should we use it?}
%
%One can see that for smaller values of $u$, $q(u)$ might be more than one.
%To avoid this, it suffices to replace Chebyshev's inequality by Cantelli's one.
%Then, taking into account that $q(u)$ is decreasing in $u$, we obtain the following.
%\merk{check if we do need Cantelli's inequality}
%\begin{corollary} \label{cor:exp-shrinking-error-prob}
%	$\Pr[Y(u) \ge E_0(u)] \le \min\left\{q(u), \tfrac1{1+1/q(1)}\right\}$.
%\end{corollary}

\subsubsection{The Phase Calculus}

Let us define the sequence $u_j$ recursively by
\begin{equation}
	u_0 = gn, \quad u_{j+1} := E_0(u_j) \notag.
\end{equation}
The next observation follows from the definition.

\begin{observation}\label{obs:exp-shrinking-expk-upper}
	For any $j \ge 1$ we have $u_j \ge u_0e^{-j{\rho_n}}$.
	In particular, for any $j \le \tfrac1{\rho_n} \ln n$ we have $u_j \ge \tfrac{u_0}{n}$.
\end{observation}

%\begin{lemma}\label{lem:exp-shrinking-expk-upper}
%	There exists $J \le \tfrac1{\rho_n}\ln n + O(1)$ such that
%	\begin{equation}
%		1 \le u_j \le 2u_0e^{-j{\rho_n}} \text{ for all } j < J \text{ and } u_J < 1 \notag.
%	\end{equation}
%\end{lemma}
%\merk{A: in the proof of the next lemma we need $g$ to be small. I don't now how to mention it (see the remark at the end of the proof).}
\begin{lemma}\label{lem:exp-shrinking-expk-upper}
	By choosing $A$ in~\eqref{eq:def-E0-upper-expshr} and $g$ sufficiently small, we can assume that for all $j \le \tfrac1{\rho_n} \ln n$, we have $u_j \le 2u_0 e^{-j{\rho_n}}$.
\end{lemma}
\begin{proof}
	For $j=0$, there is nothing to prove.
	Consider $1 \le j \le \tfrac1{\rho_n} \ln n$ and assume that for all $i < j$ we have $u_i \le 2u_0 e^{-i{\rho_n}}$.
	We will show that $u_j \le 2u_0e^{-j{\rho_n}}$.
%	Let us proof by induction on $j$ that if $u_{j-1} \ge 1$ and for all $i < j$, $u_i \ge \alpha a^{-j{\rho_n}}$, then $u_j \ge \alpha u_0 e^{-(j+1){\rho_n}}$.
%	(The base for $j=0$ is trivial.)
	By definition,
	\begin{align}
		u_j & = u_0 e^{-j{\rho_n}}
			\prod_{i=0}^{j-1}\left(1 + ae^{\rho_n} \tfrac{u_i}{n} + Ae^{\rho_n} u_i^{-B}\right) \notag\\
		& \le u_0 e^{-j{\rho_n}}
			\prod_{i=0}^{j-1} \exp\left(ae^{\rho_n} \tfrac{u_i}{n} + Ae^{\rho_n} u_i^{-B}\right) \notag\\
		& \le u_0 e^{-j{\rho_n}} \exp\left(\sum_{i=0}^{j-1}ae^{\rho_n} \tfrac{u_i}{n}
			+ \sum_{i=0}^{j-1}Ae^{\rho_n} u_i^{-B}\right) \label{eq:exp-shrinking-upper-eq1}.
	\end{align}
	We estimate separately the two sums.
	Since $u_i \le 2u_0 e^{-i{\rho_n}}$ for $i<j$, the first sum can be bounded by a geometric series:
	\begin{equation}
		\sum_{i=0}^{j-1}ae^{\rho_n} \tfrac{u_i}{n}
		\le \tfrac{ae^{\rho_n}}{n} \sum_{i=0}^{j-1} 2u_0e^{-i{\rho_n}}
		\le ae^{\rho_n} \cdot \tfrac{2u_0}{n} \cdot \tfrac1{1-e^{-{\rho_n}}} \notag.
	\end{equation}
	This expression is proportional to $\tfrac{u_0}{n} = g$, so by choosing $g$ small enough, we can bound it by $\tfrac{\ln2}{2}$.
	For the second sum we use Observation~\ref{obs:exp-shrinking-expk-upper} and obtain
	\begin{align}
		\sum_{i=0}^{j-1}Ae^{\rho_n} u_i^{-B}
		& \le Ae^{\rho_n} \sum_{i=0}^{j-1} u_0^{-B}e^{i{\rho_n} B}
			\le Ae^{\rho_n} u_0^{-B} \frac{e^{j{\rho_n} B}}{e^{{\rho_n} B}-1} \notag \\
		& \le Ae^{\rho_n} \left(\tfrac{n}{u_0}\right)^B \tfrac1{e^{{\rho_n} B}-1}
			\le Ae^{\rho_n} g^{-B} \tfrac1{e^{{\rho_n} B}-1} \label{eq:exp-shrinking-upper-eq42}.
	\end{align}
	By taking $A$ small enough, the result is also at most $\tfrac{\ln2}{2}$.
%	Indeed, we must only show that $u_0^{-B} e^{j{\rho_n} B}$ is at most a constant.
%	By observation above, $\tfrac{u_0}{n} \le u_{j-1}$ and by the induction hypothesis, $u_{j-1} \le 2u_0e^{-(j-1){\rho_n}}$.
%	Then $e^{j{\rho_n}} \le 2 n e^{\rho_n}$.
%	Therefore,
%	\begin{equation}
%		u_0^{-B} e^{j{\rho_n} B} \le 2^B\left(\tfrac{n}{u_0}\right)^B e^{{\rho_n} B}
%		= 2^B g^{-B} e^{{\rho_n} B} = O(1) \notag.
%	\end{equation}
	Substituting the sums in~\eqref{eq:exp-shrinking-upper-eq1} by their bounds of $\tfrac{\ln 2}{2}$, we obtain
	\begin{equation}
		u_j \le u_0 e^{-j{\rho_n}} \exp\left(\tfrac{\ln2}{2} + \tfrac{\ln2}{2}\right)
		= 2u_0 e^{-j{\rho_n}} \notag.
	\end{equation}
\end{proof}
We assume in the following that $A$ and $g$ are as in Lemma~\ref{lem:exp-shrinking-expk-upper}.
%\begin{remark}
%	In the proof above we assume that $\tfrac{u_0}{n} = g$ is small enough.
%	This does not have any impact on the main theorem by the following simple argument.
%	We construct the sequence $u_j$ from $u_0$ such that $\tfrac{u_0}{n}$ is small enough.
%	By Lemma~\ref{lem:general-connect} the number of rounds between $gn$ and $u_0$ uninformed nodes is $O(1)$ for any choice of $u_0$ such that $\tfrac{u_0}{n} = \Theta(1)$.
%\end{remark}
Combining the lemma above with the definition of $q(u)$ in Lemma~\ref{lem:exp-shrinking-failure-upper}, one can easily see the following.
\begin{corollary}\label{cor:exp-shrinking-nphases-upper}
	There exists $J \le \tfrac1{\rho_n} \ln n$ such that (i) $q(u_J) < \tfrac12$ and (ii) $u_J = O(1)$.
\end{corollary}
%\merk{A: I don't see why we need $\tfrac12$ instead of $1$.}
%\begin{proof}
%	For $j=0$, there is nothing to prove.
%	Let $j \ge 1$ be such that $u_{j-1} \ge 1$ and for all $i < j$, $u_i \le 2u_0 e^{-i{\rho_n}}$.
%	We show that $u_j \le 2u_0e^{-j{\rho_n}}$.
%%	Let us proof by induction on $j$ that if $u_{j-1} \ge 1$ and for all $i < j$, $u_i \ge \alpha a^{-j{\rho_n}}$, then $u_j \ge \alpha u_0 e^{-(j+1){\rho_n}}$.
%%	(The base for $j=0$ is trivial.)
%	By definition,
%	\begin{align}
%		u_j & = u_0 e^{-j{\rho_n}}
%			\prod_{i=0}^{j-1}\left(1 + ae^{\rho_n} \tfrac{u_i}{n} + Ae^{\rho_n} u_i^{-B}\right) \notag\\
%		& \le u_0 e^{-j{\rho_n}}
%			\prod_{i=0}^{j-1} \exp\left(ae^{\rho_n} \tfrac{u_i}{n} + Ae^{\rho_n} u_i^{-B}\right) \notag\\
%		& \le u_0 e^{-j{\rho_n}} \exp\left(\sum_{i=0}^{j-1}ae^{\rho_n} \tfrac{u_i}{n}
%			+ \sum_{i=0}^{j-1}Ae^{\rho_n} u_i^{-B}\right) \label{eq:exp-shrinking-upper-eq1}.
%	\end{align}
%	We estimate separately the two sums.
%	Since $u_i \le 2u_0 e^{-i{\rho_n}}$, the first sum is bounded by a geometric series as follows.
%	\begin{equation}
%		\sum_{i=0}^{j-1}ae^{\rho_n} \tfrac{u_i}{n}
%		\le \tfrac{ae^{\rho_n}}{n} \sum_{i=0}^{j-1} 2u_0e^{-i{\rho_n}}
%		\le ae^{\rho_n} \cdot \tfrac{2u_0}{n} \cdot \tfrac1{1-e^{-{\rho_n}}} \notag.
%	\end{equation}
%	The bound is proportional to $\tfrac{u_0}{n}$, so by choosing $g$ small enough, we can bound it by any positive constant, in particular by $\tfrac{\ln2}{2}$.
%	The second sum can be bounded by the simple argument that $u_j \ge u_0e^{-j{\rho_n}}$ for any $j$.
%	Thus
%	\begin{equation}\sum_{i=0}^{j-1}Ae^{\rho_n} u_i^{-B}
%		\le Ae^{\rho_n} \sum_{i=0}^{j-1} u_0^{-B}e^{i{\rho_n} B}
%		\le Ae^{\rho_n} u_0^{-B} \frac{e^{j{\rho_n} B}}{e^{{\rho_n} B}-1} \notag.
%	\end{equation}
%	There exists an $A$ small enough such that the result is also at most $\tfrac{\ln2}{2}$.
%	Indeed, we must only show that $u_0^{-B} e^{j{\rho_n} B}$ is at most a constant.
%	By the induction hypothesis, $1 \le u_{j-1} \le 2u_0e^{-(j-1){\rho_n}}$.
%	Then $e^{j{\rho_n}} \le 2 u_0 e^{\rho_n}$.
%	Therefore, $u_0^{-B} e^{j{\rho_n} B} \le 2^B e^{{\rho_n} B} = O(1)$.
%	
%	Substituting the sums in~\eqref{eq:exp-shrinking-upper-eq1} for their bounds of $\tfrac{\ln 2}{2}$, we get
%	\begin{equation}
%		u_j \le u_0 e^{-j{\rho_n}} \exp\left(\tfrac{\ln2}{2} + \tfrac{\ln2}{2}\right)
%		= 2u_0 e^{-j{\rho_n}} \notag.
%	\end{equation}
%	Finally, we observe that since $u_0 = gn$, the smallest index $j$ such that $u_j \ge 1$ is at most $\tfrac1{\rho_n} \ln n + O(1)$.
%	This completes the proof of Lemma~\ref{lem:exp-shrinking-expk-upper}.
%\end{proof}

By Lemma~\ref{lem:exp-shrinking-E0-decrease}, $u_j$ form a decreasing sequence.
We say that the rumor spreading process is in \emph{phase} $j$, $j \in \{0, \ldots, J-1\}$, if the number of informed nodes is in $[u_{j+1}, u_j[$.

%\merk{A: The "failure" means that $Y(u) \ge E_0(u)$, i.e., to stay in phase $\Rightarrow u_{j+1}$ is included.}

%\begin{lemma} \label{lem:exp-shrinking-ETj-upper}
%	If the process is in phase $j$ ($j < J$), then the expected number of rounds to leave phase $j$ is at most
%	\begin{equation}
%		1 + \tfrac{Q_j}{1-Q_j},
%		\text{ where } Q_j := \min\left\{q(u_{j+1}), \tfrac1{1+1/q(1)}\right\} \notag.
%	\end{equation}
%\end{lemma}
\begin{lemma}\label{lem:exp-shrinking-ETj-upper}
	If the process is in phase $j < J$, then the number of rounds to leave phase $j$ is stochastically dominated by $1 + \Geom(1-Q_j)$, where $Q_j := q(u_{j+1})$.
	%at most $1 + \tfrac{Q_j}{1-Q_j}$, where $Q_j := q(u_{j+1})$.
\end{lemma}
\begin{proof}
	Consider a round with $u$ uninformed nodes.
	By definition, the process leaves the phase $j$ if $Y(u) < u_{j+1} = E_0(u_j)$.
	Since $E_0(u)$ is an increasing function, the upper bound for the probability to stay in phase $j$ in current round is the following.
	\begin{equation}
		\max_{u \in [u_{j+1},u_j[} \Pr[Y(u) \ge E_0(u_j)]
		\le \max_{u \in [u_{j+1},u_j[} \Pr[Y(u) \ge E_0(u)]
		\le q(u_{j+1}) \notag.
	\end{equation}
%	Thus, by Corollary~\ref{cor:exp-shrinking-error-prob}, the probability to stay in phase $j$ is bounded by
%	\begin{equation}
%		\max_{u \in [u_{j+1},u_j[} \min\left\{q(u), \tfrac1{1+1/q(1)}\right\} \notag.
%	\end{equation}
	So the number of rounds to leave phase $j$ is stochastically dominated by $1 + \Geom(1-Q_j)$.
%	Therefore, the expected number of rounds to leave phase $j$ is at most $1 + \tfrac{Q_j}{1-Q_j}$.
\end{proof}

\begin{lemma} \label{lem:exp-shrinking-sum-Qj-upper}
	$\sum_{j=0}^{J-1} Q_j = O(1)$.
\end{lemma}
\begin{proof}
	%By definition, $Q_j \le q(u_{j+1})$.
	By Lemma~\ref{lem:exp-shrinking-failure-upper}, we have
	\begin{equation}
		\sum_{j=0}^{J-1} Q_j \le O(1) \cdot \sum_{j=1}^J \tfrac1{u_j^{1-2B}} = O(1) \notag,
	\end{equation}
	where the last equality follows as in~\eqref{eq:exp-shrinking-upper-eq42}, using that $J \le \tfrac1{\rho_n} \ln n$.
%	From Lemma~\ref{lem:exp-shrinking-expk-upper} it follows that $u_j$ is bounded by a decreasing geometric sequence.
%	Since $B < 1/2$, $\sum Q_j$ is bounded from above by the sum of an increasing geometric series.
%	Such sum is of order of its last term, i.e.,
%	\begin{equation}
%		\sum_{j=0}^{J-1} Q_j = O(Q_{J-1}) = O\left(\tfrac1{u_J^{1-2B}}\right) \notag.
%	\end{equation}
%	To conclude the proof, it remains to note that $u_J = O(1)$.
\end{proof}
%\begin{proof}
%	We apply the estimate for $q(u_j)$ from Lemma~\ref{lem:exp-shrinking-failure-upper} and the bounds for $u_j$ from Lemma~\ref{lem:exp-shrinking-expk-upper}.
%	Therefore,
%	\begin{align}
%		\sum_{j=0}^{J-1} Q_j
%		& \le \sum_{j=0}^{J-1} q(u_j)
%			\le \sum_{j=0}^{J-1} \tfrac{(1+a)e^{-{\rho_n}}+c}{A^2} \cdot \tfrac1{u_j^{1-2B}} \notag \\
%		& \le O(1) \cdot \sum_{j=0}^{J-1} \tfrac1{(2u_0)^{1-2B}} \cdot e^{j{\rho_n}(1-2B)} \notag \\
%		& \le O(1) \cdot \tfrac1{(2u_0)^{1-2B}} \cdot \tfrac{e^{J{\rho_n}(1-2B)}}{1-e^{{\rho_n}(1-2B)}} \notag \\
%		& = O(1) \cdot \tfrac1{u_J^{1-2B}} = O(1) \notag.
%	\end{align}
%	\merk{check the last equality \ldots}
%\end{proof}

Now we can proof the main result of this section, i.e., Theorem~\ref{th:exp-shrinking-upper}.
\begin{proof}[Proof of Theorem~\ref{th:exp-shrinking-upper}]
%    \merk{A: paragraph added to avoid $g'$ in the statement of the theorem.}
    First, let $g' > 0$ be smaller than $g$.
    Then,
    \begin{equation}
        \Expect[T(n-\lfloor gn\rfloor, n)]
        \le \Expect[T(n-\lfloor gn \rfloor, n-\lceil g'n \rceil)]
        + \Expect[T(n- \lfloor g'n \rfloor, n)] \notag.
    \end{equation}
    By Lemma~\ref{lem:general-connect}, the exponential shrinking conditions imply that $\Expect[T(n-\lfloor gn \rfloor, n-\lceil g'n \rceil)]$ is at most a constant.
    In addition there exist $A'_0, \alpha'_0 > 0$ such that $\Pr[T(n-\lfloor gn \rfloor, n-\lceil g'n \rceil) > r/3] \le A'_0 e^{-\alpha'_0 r}$.
    We can hence assume that $g$ is small enough so that all Lemma~\ref{lem:exp-shrinking-E0-decrease}~and~\ref{lem:exp-shrinking-expk-upper} are satisfied.

	We denote by the random variable $T_j$ the number of rounds spent in phase $j$.
	With Corollary~\ref{cor:exp-shrinking-nphases-upper} and Lemma~\ref{lem:exp-shrinking-sum-Qj-upper}, we compute
	\begin{align}
		\Expect[T(n-\lfloor gn \rfloor, n-\lceil u_J \rceil)]
		& \le \sum_{j=0}^{J-1} \Expect[T_j]
			\le \sum_{j=0}^{J-1} \left(1 + \tfrac{Q_j}{1-Q_j}\right) \notag \\
		& = J + \sum_{j=0}^{J-1} \tfrac{Q_j}{1-Q_j}
			\le J + \tfrac1{1-Q_J} \cdot \sum_{j=0}^{J-1} Q_j \notag \\
		& = J + O(1) \notag.
	\end{align}
	Since $Q_j$ form a geometrical sequence, it follows from Lemma~\ref{lemma:sum geometrical-2} that there exist $A', \alpha' > 0$ such that
	\begin{equation}
		\Pr[T(n-\lfloor gn \rfloor, \lceil u_J \rceil) > J+r/2] \le A' e^{-\alpha' r}. \label{eq:1/th-30}
	\end{equation}
%	Then let $q = 1-\min_{k\in[n-u_J,n]}p_k$.
%	\merk{check if the sum should be up to $J$ or up to $J-1$ (also in Exponential Growth).}
	For the last at most $u_J$ uninformed nodes, we argue as follows.
	Consider one uninformed node.
	From the exponential shrinking conditions it follows that the expected number of rounds until this node is informed is at most $O(1)$.
	So, $\Expect[T(n-\lfloor u_J\rfloor, n)] \le u_J \cdot O(1) = O(1)$.
	Finally,
	\begin{equation}
		\Expect[T(n-\lfloor gn \rfloor, n)]
		\le \Expect[T(n-\lfloor gn \rfloor, n-\lceil u_J \rceil)]
			+ \Expect[T(n-\lfloor u_J\rfloor, n)]
		\le \tfrac1{\rho_n} \ln n + O(1) \notag.
	\end{equation}
	
	To prove the tail bound statement, let $q = 1-\min_{k\in[n-u_J,n]}p_k$.
	Now we consider the epidemic protocol with $m = O(1)$ uninformed nodes.
	Since an uninformed node stays uninformed for $r/2$ rounds with probability at most $q^{r/2}$, we have $\Pr[T(n-m,n) > r/2] \le m \cdot q^{r/2}$.
	Combining the last inequation with~\eqref{eq:1/th-30}, we obtain
	\[
		\Pr[T(n-\lfloor gn \rfloor, n) > J + r]
		\le (u_J+A') \exp\left(-r\cdot\min\right\{\alpha', \tfrac{\ln q}2\left\}\right) \,.
	\]
	Since $u_J = O(1)$, the tail bound statement directly follows as in the proof of Theorem~\ref{th:exp-growth-upper}.
\end{proof}
%\merk{check if we need to add a constant number of rounds to inform a constant number of last uninformed nodes.}
%\begin{corollary}
%	There exist $A, \alpha > 0$ such that for any integer $r > 0$ we have
%	$$\Pr[T(n-\lfloor gn \rfloor,n) \le \E[T(n-\lfloor gn \rfloor,n)] + r] \le Ae^{-\alpha r}.$$
%\end{corollary}
%\begin{proof}
%	We observe first that by Lemma~\ref{lemma:sum geometrical-2}, there exist some $A', \alpha' > 0$ such that
%	$$\Pr[T(n-\lfloor gn \rfloor, \lceil u_J \rceil) > J+r] \le A'e^{-\alpha'r}.$$
%	Then let $q = 1-\min_{k\in[n-u_J,n]}p_k$.
%	Now we consider the epidemic protocol with $m = O(1)$ uninformed nodes.
%	Since an uninformed node stays uninformed for $r$ rounds with probability at most $q^r$, we have $\Pr[T(n-m,n) > r] \le m \cdot q^r$.
%	Since $u_J = O(1)$, the claim of corollary directly follows from the fact that both probabilities above are exponentially small in $r$.
%\end{proof}

%-----------------------------------------------------------------------------------
%-----------------             Lower       BOUND             -----------------------
%-----------------------------------------------------------------------------------

\subsection{Exponential Shrinking Regime. Lower Bound}

\subsubsection{Exponential Shrinking Conditions}
\begin{defn}[lower exponential shrinking conditions] \label{def:upper-exp-shrinking-conditions}
    Let $\rho_n$ be bounded between two positive constants.
	Let $0 < g < 1$ and $a, c \in \R_{\ge0}$.
	We say that a homogeneous epidemic protocol satisfies \emph{the lower exponential shrinking conditions} if for any $n \in \N$ big enough, the following properties are satisfied, for all $u \le gn$ (resp. $k \in [n-\lfloor gn \rfloor, n]$).
	\renewcommand{\theenumi}{(\roman{enumi})}%
	\begin{enumerate}
		\item
			$1 - p_k = 1-p_{n-u} \ge e^{-{\rho_n}} - a\frac{u}{n}$;
		\item
			$c_k = c_{n-u} \le \frac{c}{u}$.
	\end{enumerate}
\end{defn}

\begin{theorem}[lower bound of spreading time] \label{th:exp-shrinking-lower}
	Consider a homogeneous epidemic protocol satisfying the lower exponential shrinking conditions (see definition above). There is a constant $g' \in ]0, 1[$ and further constants $A', \alpha'>0$ such that for any positive $g < g'$,
	\begin{align*}
		&\Expect[T(n-\lfloor gn \rfloor, n)] \ge \tfrac1{\rho_n}\ln n + O(1),\\
		&\Pr[T(n-\lfloor gn \rfloor, n) \le \tfrac1{\rho_n}\ln n - r] \le A' \exp(-\alpha'r)\, \mbox{ for all $r \in \N$}.\\
	\end{align*}
\end{theorem}

\subsubsection{Round Targets and Failure Probabilities}

Let $Y(u)$ be the number of uninformed nodes at the end of the round with $u$ uninformed ones.
%Since $\Expect[Y(u)] = u(1-p_{n-u})$, the exponential shrinking conditions imply that
From the exponential shrinking conditions it follows that
\begin{equation}
	\Expect[Y(u)] \ge E(u) := u \left(e^{-{\rho_n}} - a\tfrac{u}{n}\right) \notag.
\end{equation}
We define the \emph{target value} in the same way as for the upper bound.
\begin{equation}
	E_0(u) := E(u) - Au^{1-B}, \label{eq:def-E0-lower-expshr}
\end{equation}
where $A>0$ and $B \in ]0,1/2[$ are some constants chosen uniformly for all values of $u$ and $n$.
In addition $A$ is required to be small enough to satisfy Lemma~\ref{lem:exp-shrinking-expk-lower}.

%\begin{lemma}
%	$E_0(u)$ is increasing.
%\end{lemma}
%\begin{proof}
%	\merk{todo}
%\end{proof}

\begin{lemma}~\label{lem:exp-shrinking-failure-lower}
	For any $u > gn$ and $u \in \N$,
	\begin{equation}
		\Pr[Y(u) \le E_0(u)] \le q(u) := \tfrac{e^{-{\rho_n}}+c}{A^2} \cdot \tfrac1{u^{1-2B}} \notag.
	\end{equation}
\end{lemma}
\begin{proof}
	As before, using Chebyshev's inequality and taking into account that $E(u) \le \Expect[Y(u)]$, we compute
	\begin{align}
		\Pr&[Y(u) \le E_0(u)]
			= \Pr\left[Y(u) \le E(u) \cdot \left(1-\tfrac{Au^{1-B}}{E(u)}\right) \right] \notag \\
		& \le \Pr\left[Y(u) \le \Expect[Y(u)]-Au^{1-B}\cdot\tfrac{\Expect[Y(u)]}{E(u)}\right] \notag \\
		& \le \tfrac{\Var[Y(u)]}{(Au^{1-B})^2} \cdot \tfrac{E(u)^2}{\Expect[Y(u)]^2} \notag.
	\end{align}
    From covariance condition, it follows that $\Var[Y(u)] \le \Expect[Y(u)] + cu$.
    Therefore,
    \begin{align}
        \Pr[Y(u) \le E_0(u)]
        & \le \left(1 + \tfrac{cu}{\Expect[Y(u)]}\right) \cdot \tfrac{E(u)}{\Expect[Y(u)]}
    	   		\cdot \tfrac{E(u)}{(Au^{1-B})^2} \notag \\
    	& \le \left(1 + \tfrac{cu}{E(u)}\right) \cdot \tfrac{E(u)}{(Au^{1-B})^2} \notag \\
        & = (E(u)+cu) \cdot \tfrac1{(Au^{1-B})^2}
        	\le \tfrac{e^{-{\rho_n}}+c}{A^2} \cdot \tfrac1{u^{1-2B}} \notag.
    \end{align}
\end{proof}

\subsubsection{The Phase Calculus}

We define the sequence $u_j$ recursively by
\begin{equation}
	u_0 := gn, \qquad u_{j+1} := E_0(u_j) \notag.
\end{equation}
The next observation follows from the definition.
\begin{observation} \label{obs:exp-shrinking-expk-lower}
	For any $j \ge 0$ we have $u_j \le u_0 e^{-j{\rho_n}}$.
\end{observation}

\begin{lemma}\label{lem:exp-shrinking-expk-lower}
	By choosing $A$ in~\eqref{eq:def-E0-lower-expshr} and $g$ sufficiently small, we can assume that for all $j \le \tfrac1{\rho_n} n$, we have $u_j \le \tfrac12 u_0 e^{-j{\rho_n}}$.
\end{lemma}
\begin{proof}
	For $j=0$, there is nothing to prove.
	Consider $1 \le j \le \tfrac1{\rho_n} \ln n$ and assume that for all $i<j$ we have $u_i \ge \tfrac12 u_0 e^{-i{\rho_n}}$.
	We will show that $u_j \ge \tfrac12 u_0 e^{-j{\rho_n}}$.
	By definition,
	\begin{align}
		u_j & = u_0e^{-j{\rho_n}} \prod_{i=0}^{j-1}
			\left(1-e^{\rho_n} a\tfrac{u_i}{n} - A\tfrac1{u_i^B}\right) \notag \\
		& \ge u_0e^{-j{\rho_n}} \left(1
			-\tfrac{e^{\rho_n} a}{n}\sum_{i=0}^{j-1}u_i - A\sum_{i=0}^{j-1}\tfrac1{u_i^B}\right) \notag
	\end{align}
	Like in the proof of Lemma~\ref{lem:exp-shrinking-expk-upper}, we estimate separately the two sums.
	Using Observation~\ref{obs:exp-shrinking-expk-lower}, we obtain for the first sum that
	\begin{equation}
		\tfrac{e^{\rho_n} a}{n}\sum_{i=0}^{j-1}u_i
		\le e^{\rho_n} a \tfrac{u_0}{n} \sum_{i \ge 0} e^{-i{\rho_n}}
		= \tfrac{e^{\rho_n} a}{1-e^{-{\rho_n}}} \cdot \tfrac{u_0}{n}
		= g \cdot O(1) \notag.
	\end{equation}
	By the hypothesis of induction, for any $i < j$, $u_i \ge \tfrac12 u_0 e^{-i{\rho_n}}$.
	Since $j < \tfrac1{\rho_n} \ln n$,
	\begin{align}
		A \sum_{i=0}^{j-1}\tfrac1{u_i^B}
		& \le \tfrac{A}{2^B u_0^B} \sum_{i=0}^{j-1} e^{-i{\rho_n} B}
			\le \tfrac{A}{2^B u_0^B} \cdot \tfrac{e^{j{\rho_n} B}}{e^{{\rho_n} B} - 1} \notag \\
		& = \tfrac{A}{2^B(e^{{\rho_n} B}-1)} \cdot \tfrac{n^B}{u_0^B}
			= \tfrac{A}{2^B(e^{{\rho_n} B}-1)} \cdot g^{-B}
			= Ag^{-B} \cdot O(1) \notag.
	\end{align}
	Then, by choosing $A$ and $g$ small enough, we can bound both sums by 1/4, so that
	\begin{equation}
		u_j \ge u_0 e^{-j{\rho_n}} \left(1 - \tfrac14 - \tfrac14\right)
		\ge \tfrac12 u_0 e^{j-{\rho_n}} \notag.
	\end{equation}
\end{proof}

Having $u_j$ bounded from above and below, one can easily see the following.
\begin{corollary} \label{cor:exp-shrinking-nphases-lower}
	There exists $J = \tfrac1{\rho_n} \ln n + O(1)$ such that $u_J > 1$ for any $n$ big enough.
\end{corollary}

By definition, the $u_j$ form a non-decreasing sequence.
We say that the rumor spreading process is in phase $j$, $j \in \{0,\ldots,J-1\}$, if the number of informed nodes is in $[u_{j+1}, u_j[$.

\begin{lemma}
	If the process is in phase $j < J-1$, then the probability that it "leapfrogs" phase $j+1$ (i.e., proceeds to phase $j+2$ or further in current round) is at most $q(u_j)$.
\end{lemma}
\begin{proof}
    Consider a round with $u \in [u_{j+1}, u_j[$ uninformed nodes.
    The protocol jumps over the phase $j+1$, if at the end of current round $Y(u) < u_{j+2} = E_0(u_{j+1})$.
    Since $E_0$ is increasing,
	\begin{equation}
		\Pr[u < u_{j+2}]
		%= \Pr[u < E_0(u_{j+1})]
		\le Pr[u < E_0(u)]
		\le q(u) \notag.
	\end{equation}
	Since $q(u)$ is a decreasing function, the upper bound for the probability to jump over phase $j+1$ is  the following.
	\begin{equation}
		\max_{u \in [u_{j+1}, u_j[} \Pr[u < u_{j+2}]
		\le q(u_{j+1}) \notag.
	\end{equation}
\end{proof}

Now we can proof the main result of this section, i.e., Theorem~\ref{th:exp-shrinking-lower}.

\begin{proof}[Proof of Theorem~\ref{th:exp-shrinking-lower}]
    Let $\tau$ be the first round $t$ (of this shrinking phase) in which the process leapfrogs a phase. Let $\tau = \infty$ if such an event does not occur. 
    By Corollary~\ref{cor:exp-shrinking-nphases-lower}, the interval $[1,gn]$ is cut into at least $J = \tfrac1{\rho_n} \ln n + O(1)$ phases.
    Clearly, if $\tau < J$, then $T(n-\lfloor gn \rfloor, n) \ge \tau$, and if $\tau \ge J$, then $T(n-\lfloor gn \rfloor, n) \ge J$.

    If $\tau = J-t$, then the process in phase $J-t$, that is, from some number $u$ of uninformed nodes belonging to phase $J-t$, makes an exceptionally large progress from. Since $q(u)$ is a decreasing function, we have $\Pr[\tau = J-t] \le q(u_{J-t})$. Consequently, using the fact that $q(u_j)$ forms a decreasing geometric sequence, we obtain %.using $u_{J-t} \ge O(1) \cdot u_J \cdot e^{{\rho_n} t}$, %\merk{double-check the previous
    \begin{align*}
    \Pr[T(n-\lfloor gn \rfloor, n) \le J-t] \le \Pr[\tau \le J-t] \le q(u_0) + q(u_1) + \ldots + q(u_{J-t}) = O(q(u_{J-t})).
    \end{align*}
    Then, using $u_{J-t} \ge O(1) \cdot u_J \cdot e^{{\rho_n} t}$, we compute
    \begin{align*}
    \Pr[T(n-\lfloor gn \rfloor, n) \le J-t]
    & \le O(q(u_{J-t})) \le O(1) u_{J-t}^{-2B+1} \\
    & \le O(1) (u_J e^{\rho_n t})^{-2B+1} \le O(1) \exp(-\Omega(t)).
		\end{align*}
    Applying Lemma~\ref{lem:exp-shrinking-failure-lower}, we obtain
    \begin{align*}
        \Expect[T(n-\lfloor gn \rfloor, n)]
        &\ge J \Pr[\tau > J] + \sum_{t=1}^{J-1} t\cdot\Pr[\tau = t]
        = J - \sum_{t=1}^{J-1} t \Pr[\tau = J-t] \\
        &\ge J - \sum_{t=1}^{J-1} t q(u_{J-t})
        \ge J - \tfrac{e^{\rho_n}+a+c}{A^2} \cdot \sum_{t=1}^{J-1} \tfrac{t}{u_{J-t}^{1-2B}}.
    \end{align*}
    Since $B < 1/2$ and $u_{J-t} \ge O(1) \cdot u_J \cdot e^{{\rho_n} t}$, the sum above converges.
    Therefore,
    \begin{align}
    	\Expect[T(n-\lfloor gn \rfloor, n)] \ge J + O(1) \notag.
    \end{align}
%        \ge J - O(1) \sum_{t=1}^{J-1} \tfrac t{(u_0 e^{-(J-t){\rho_n}})^{1-2B}}
%        = J - \tfrac{O(1)}{(u_0 e^{-J{\rho_n}})^{1-2B}} \sum_{t=1}^{J-1} te^{-t(1-2B)}
%        = J - O(1)
%    \end{align}
\end{proof} 
}

 %\input{exp_shrinking.tex}

 %%%%                 SUBSECTION
%%%%       DOUBLE EXPONENTIAL SHRINKING REGIME
%%%%
\subsection{Double Exponential Shrinking Regime. Upper Bound.}

In the following two sections we consider the regime in which uninformed nodes remain uninformed with probability proportional to the fraction uninformed nodes, or, more generally, some positive power $\ell-1$ there of.
Such a regime often occurs in protocols using pull operations.
We show that the \emph{fraction} of uninformed nodes is raised to the $\ell$-th power each round and that such a regime informs the last $gn$ nodes ($g$ a small constant) in a double logarithmic number of rounds.

We discuss the upper bound on the runtime first.
Throughout this section, we assume that our homogeneous epidemic protocol satisfies the following \emph{upper double exponential shrinking conditions} including the covariance condition.
%\merk{here are the key differences of this section from the exponential growth and shrinking}
%\begin{itemize}
%    \item
%        The main parameter here is $\eps := \tfrac{u}{n}$ - the fraction of informed nodes.
%        In particular the using of $Y(u)$ from previous section creates an ambiguity in the notation: $Y(\eps n)$ and $E(\eps)$.
%    \item
%        Some of steps are too trivial to formulate the corresponding lemmas, e.g. $\eps_{j+1} = a^{\frac{\ell^j-1}{\ell-1}} \eps_0^{\ell^j}$.\item
%        We use very loose bound for the variance.
%    \item
%        The same for the error probability $q$ which now does not depend on $\eps$.
%    \item
%        We do need the fast finishing conditions, because the standard phase construction is applicable until there are at least $\sqrt{n}$ (\merk{see if it the right bound}) uninformed nodes.
%\end{itemize}

%%%DEF - The double exponential shrinking conditions
\begin{defn}[upper double exponential shrinking conditions] \label{def:dexp}
	Let $g, \alpha \in [0,1]$, $\ell > 1$, and $a, c \in \R_{\ge0}$ such that $ag^{\ell-1} < 1$.
	We say that a homogeneous epidemic protocol satisfies \emph{the upper double exponential shrinking conditions}
	if for any $n$ big enough, the following properties are satisfied for all $u \in [n^{1-\alpha}, gn]$.
	%$\eps \in [n^{-\alpha},g]$ such that $\eps n \in \N$.
	\renewcommand{\theenumi}{(\roman{enumi})}%
	\begin{enumerate}
		\item $1-p_{n-u} \le a\left(\tfrac{u}{n}\right)^{\ell-1}$.
		\item $c_{n-u} \le c \tfrac{n}{u^2}$.
	\end{enumerate}
\end{defn}

Similarly to the exponential shrinking regime we argue with the number $u$ of uninformed nodes rather than the number $k$ of informed ones.
To ease the notation in the double exponential shrinking regime we use the \emph{fraction} $\eps := \tfrac{u}{n}$ of uninformed nodes instead of the absolute number $u$.
Thus, the double exponential shrinking conditions turns into the following bounds, valid for all $\eps \in [n^{-\alpha},g]$ with $\eps n \in \N$.
\renewcommand{\theenumi}{(\roman{enumi})}%
\begin{enumerate}
	\item $1-p_{n(1-\eps)} \le a\eps^{\ell-1}$.
	\item $c_{n(1-\eps)} \le \eps^{-2}\tfrac{c}{n}$.
\end{enumerate}

In the definition above, we cover the rounds starting with a number of uninformed nodes between $n^{1-\alpha}$ and $gn$. While, by taking $\alpha=1$ this would allow to analyze the process until all nodes are informed, it turns out that the crucial part is reduce the number of uninformed nodes from $\Theta(n)$ to $n^{1-\alpha}$ for an arbitrarily small constant $\alpha$. For $u \in [1,n^{1-\alpha}]$, the double exponential shrinking conditions can be relaxed: the covariance condition is no longer needed and it is sufficient to bound uniformly the probability of a node to stay uninformed by $n^{-\tau}$, for some $\tau < 1$.

The main result of the section is the following theorem.

%For the remainder of the process, the following notion will often be more convenient.
%
%\merk{Seems that we can drop the part with fast finishing. Here is the old construction.}
%
%We say that the protocol satisfies the \emph{fast finishing conditions} from $n^{-\alpha}$ on (for some $0<\alpha<1$) if there exists a constant $\tau > 0$ such that for any $\eps \le n^{-\alpha}$, we have $1-p_{n(1-\eps)} \le n^{-\tau}$. The following observation follows right from the definitions. It allows to freely increase the switching point between the adjacent exponential shrinking and fast finishing regimes.
%
%\begin{observation}\label{obs:double-exp-finishing}
%  If a homogeneous protocol satisfies the upper double exponential shrinking conditions in $[n^{-\alpha},g]$ and the fast finishing conditions from $n^{-\alpha}$ on, then it also satisfies the fast finishing conditions from $n^{-\beta}$ on for any constant $0 < \beta \le \alpha$.
%\end{observation}
%
%Once we have reached a regime with fast finishing conditions, we only need an expected constant number of rounds to get all nodes informed.
%
%\begin{lemma}\label{lem:double-exp-finishing}
%	Let the protocol satisfy the fast finishing conditions from $n^{-\alpha}$ on.
%	Then the expected spreading time $\E[T(n-u_0,n)]$ is $O(1)$ for any $u_0 \le n^{1-\alpha}$.
%\end{lemma}
%\begin{proof}
%	Consider one of the $u_0 \le n^{1-\alpha}$ initially uninformed nodes.
%	For any $R\ge1$, this node remains uninformed for $R$ rounds with probability at most $n^{-\tau R}$.
%	By the union bound, the probability that for $R$ rounds at least one node remains uninformed is at most $P_R := \min\{1,n^{-\tau R + 1-\alpha}\}$.
%	Therefore, $\E[T(n-u_0,n)] \le 1+\sum_{R\ge1} P_R = O(1)$.
%\end{proof}
%\merk{end of fast finishing}

%The main result of this section is that the double exponential shrinking conditions (possibly with the fast finishing conditions in the end) give a sharp runtime of $\log_\ell \ln n + O(1)$.

\begin{theorem}\label{th:double-exp-shrinking-upper}
	Consider a homogeneous epidemic protocol satisfying the upper double exponential shrinking conditions in $[n^{-\alpha},g]$.
	Suppose further that there exists $\tau > 0$ such that $1-p_{n-u} \le n^{-\tau}$ for all $u \le n^{1-\alpha}$.
	
	Then there exist constant $A', \alpha' > 0$ such that
	\begin{eqnarray*}
		&& \Expect[T(\lceil (1-g)n \rceil,n)] \le \log_\ell \ln n + O(1), \\
		&& \Pr[T(\lceil (1-g)n \rceil,n) \ge \log_\ell \ln n + r] \le O(n^{-\alpha' r+A'})\, \mbox{ for all $r \in \N$}.
	\end{eqnarray*}
\end{theorem}

%\begin{theorem}\label{theorem:double-exp-shrinking-upper}
%	Consider a homogeneous epidemic protocol disseminating a rumor on $n$ nodes.
%	If the upper double exponential shrinking conditions are satisfied in $[n^{-\alpha},g]$ together with the fast finishing condition from $n^{-\alpha}$ on, and if $g < (2\bar a)^{1/(\ell-1)}$, \merk{B: added this} then
%	\begin{equation}
%		\Expect[T(\lceil (1-g)n \rceil,n)] \le \log_\ell \ln n + O(1) \notag.
%    \end{equation}
%\end{theorem}
%\begin{remark}
%	In reality, we can slightly relax the double exponential shrinking condition: if for any $\eps \in [n^{-\alpha},g]$, (i) and (ii) are satisfied and in addition, for any $\eps < n^{-\alpha}$, $1-p_{n(1-\eps)} \le n^{-\tau}$, for some $\tau > 0$ and $0 < \alpha < 1$, then we have the same upper bound on the runtime of the protocol.
%	\begin{equation}
%		\Expect[T(\lceil (1-g)n \rceil,n)] \le \log_\ell \ln n + O(1) \notag.
%	\end{equation}
%\end{remark}

\subsubsection{Round Targets and Failure Probabilities}

%\merk{B: hier koennte man formal korrekt erst $E(\eps)$ fuer alle $\eps$ definieren und dann $y(\eps)$ nur fuer die $\eps$ mit $\eps n$ ganzzahlig}
Let the random variable $y(\eps)$ denote to the fraction of uninformed nodes at the end of a round started with $\eps n$ uninformed ones.
%Since $\Expect[y(\eps)] \ge \eps n (1-p_{n(1-\eps)})$,
The double exponential shrinking conditions state that
\begin{equation}
    \Expect[y(\eps)] \le E(\eps) := a\eps^\ell \notag.
\end{equation}

\begin{lemma}\label{lem:double-exp-shrinking-variance}
	$\Var[y(\eps)] \le \tfrac{1+c}{n}$.
\end{lemma}
\begin{proof}
	Indeed, $\Var[y(\eps)] = \tfrac1{n^2}\Var[Y(\eps)]$, where $Y(\eps) := ny(\eps)$ is the number of uninformed nodes at the end of the round.
	By Lemma~\ref{lem:prelim:variance},
	\begin{equation}
		\Var[Y(\eps)] \le \Expect[Y(\eps)] + (n\eps)^2 c_{n(1-\eps)}
		\le n + cn\notag.
	\end{equation}
\end{proof}

The next lemma states that with good probability, $y(\eps)$ is less than the \emph{target value} $2E(\eps)$.

%\merk{A: "for any fraction of uninformed nodes" allows not to mention that $\eps n \in \N$.}
\begin{lemma}\label{lem:double-exp-shrinking-failure-upper}
    For any fraction of uninformed nodes $\eps \in [n^{-\alpha}, g]$,
    \begin{equation}
        \Pr[y(\eps) \ge 2E(\eps)] \le q := \tfrac{(1+c)}{a^2}n^{2\alpha\ell-1} \notag.
    \end{equation}
\end{lemma}
\begin{proof}
    Applying Chebyshev's inequality and taking into account that $E(\eps) \ge \Expect[y(\eps)]$, we compute
    \begin{equation}
        \Pr[y(\eps) \ge 2E(\eps)]
        \le \Pr[y(\eps) \ge \Expect[y(\eps)] + E(\eps)]
        \le \tfrac{\Var[y(\eps)]}{E(\eps)^2} \notag.
    \end{equation}
    By Lemma~\ref{lem:double-exp-shrinking-variance} and since $\eps \ge n^{-\alpha}$,
    \begin{equation}
    	\Pr[y(\eps) \ge 2E(\eps)]
    	\le \tfrac{1+c}{n} \cdot \tfrac1{(a\eps^\ell)^2}
    	\le \tfrac{1+c}{a^2} n^{2\alpha\ell-1}\notag.
    \end{equation}
%    Since $\Expect[y(\eps)] \le 1$ and $c_{n(1-\eps)} \le \eps^{-2}c/n$,
%    we have
%    \begin{equation}
%        \Var[y(\eps)] \le \Expect[y(\eps)] + (\eps n)^2 c_{n(1-\eps)} \le 1 + c \notag.
%    \end{equation}
%    Since, $E(\eps) = a\eps^\ell$, the claim of lemma follows directly.
\end{proof}
Our choice to analyze the double exponential shrinking regime only up to $n^{1-\alpha}$ uninformed nodes allows us to define $q$ independent of $\eps$.
Since the double exponential shrinking conditions imply the second assumption of Theorem~\ref{th:double-exp-shrinking-upper}, without loss of generality we may assume that $\alpha< \tfrac1{2\ell}$, and that consequently $q=n^{-\Theta(1)}$.

%\merk{A: do we need later $q \cdot \log_\ell \ln n < 1$?}

%%%  PHASE CALCULUS
\subsubsection{The Phase Calculus}

Let us define the sequence $\eps_j$ recursively by
\begin{equation}
    \eps_0 := g, \quad \eps_{j+1} := 2E(\eps_j) \notag.
\end{equation}
The following observation can be obtained by a simple induction.
\begin{observation}\label{obs:double-exp-shrinking-epsj}
    For all $j \ge 0$, $\eps_j = (2a)^{\frac{\ell^j-1}{\ell-1}} g^{\ell^j}$.
    In particular, the $\eps_j$ form a decreasing sequence if $g < (2a)^{-\tfrac1{\ell-1}}$.
\end{observation}
In the following we assume that $g$ is small enough to ensure that the $\eps_j$ decrease.
Applying logarithm twice to the previous equation one can also see the following.
%\begin{observation} \label{obs:double-exp-shrinking-nphases}
%	There exists $J = \log_\ell \ln n + O(1)$ such that $n^{-\alpha} \in [\eps_J, \eps_{J-1}[$.
%\end{observation}
%\merk{todo: discuss $\to n$ should be large enough}
\begin{corollary} \label{cor:double-exp-shrinking-nphases}
There exists $J = \log_\ell \ln n + O(1)$ such that for any $n$ big enough
\begin{equation}
    n^{-\alpha} < \eps_J \le \left(\tfrac{n^{-\alpha}}{2a}\right)^{1/\ell} \notag.
\end{equation}
%	There exists $J = \log_\ell \ln n + O(1)$ and $0 < \alpha' < \alpha$ such that $\eps_J \in ]n^{-\alpha}, n^{-\alpha'}]$.
\end{corollary}
\begin{proof}
	From Observation~\ref{obs:double-exp-shrinking-epsj} we see that the biggest $J$ such that $\eps_J > n^{-\alpha}$ is equal to $\log_\ell \ln n + O(1)$.
	Since $\eps_{J+1} < n^{-\alpha}$, we have $\eps_J < \left(\tfrac{n^{-\alpha}}{2a}\right)^{1/\ell}$.
%	Hence for any $0 < \alpha' < \alpha/\ell$ we have $\eps_J < n^{-\alpha'}$.
\end{proof}
%Then by applying double logarithm to the previous equation, one can see that there exists $J = \log_\ell \ln n + O(1)$ such that $\eps_J > n^{-\alpha}$, but $\eps_{J+1} < n^{-\alpha}$.

We say that the process is in phase $j$ if the fraction $\eps$ of uninformed nodes is in $]\eps_{j+1}, \eps_j]$.
\begin{lemma} \label{lem:double-exp-shrinking-ETj-upper}
    If the process is in phase $j$, $j < J$, then the number of rounds to leave phase $j$ is stochastically dominated by $1 + \Geom(1-q)$.
%    at most $1 + \tfrac{q}{1-q}$.
\end{lemma}
\begin{proof}
    Consider a round starting with $\eps n$ uninformed nodes.
    By construction, the process leaves the phase $j$ if $y(\eps) \le \eps_{j+1} = 2E(\eps_j)$.
    Since $E(\cdot)$ is an increasing function, an upper bound for the probability to stay in phase $j$ in the current round is
    \begin{equation}
        \max_{\eps\in]\eps_{j+1},\eps_j]} \Pr[y(\eps) > 2E(\eps_j)]
        \le \max_{\eps\in]\eps_{j+1},\eps_j]} \Pr[y(\eps) \ge 2E(\eps)]
        \le q \notag.
    \end{equation}
    Hence, the number of rounds the process spends in phase $j$ is stochastically dominated by a random variable with distribution $1+\Geom(1-q)$.
%    , which has an expectation of $1+q/(1-q)$.
\end{proof}

Let us now prove the main theorem of the section.

\begin{proof}[Proof of Theorem~\ref{th:double-exp-shrinking-upper}]
	From Lemma~\ref{lem:general-connect} it follows that for any $g' < g$ we have $\Expect[T(n-\lfloor gn \rfloor, n - \lceil g'n \rceil)] = O(1)$.
	So without loss of generality we can assume that $g < (2a)^{-\tfrac1{\ell-1}}$ that is required by Observation~\ref{obs:double-exp-shrinking-epsj} and, thus, by Corollary~\ref{cor:double-exp-shrinking-nphases}.
	Let the random variable $T_j$ denote the number of rounds spent in phase $j$.
%    By Lemma~\merk{ref?!}, the number of rounds since the process enters in the fast finishing regime until all nodes are informed is at most $O(1)$.
    With Corollary~\ref{cor:double-exp-shrinking-nphases} as well as Lemma~\ref{lem:double-exp-shrinking-failure-upper}~and~\ref{lem:double-exp-shrinking-ETj-upper}, we compute
    \begin{eqnarray}
        && \Expect[T(n-\lfloor gn \rfloor, n - \lceil\eps_Jn\rceil)]
        	\le \sum_{j=0}^{J-1} \Expect[T_j]
	        \le J \left(1 + \tfrac{q}{1-q}\right)
        	= \log_\ell \ln n + O(1) \label{eq:2/th-42} \\
        && \Pr\left[T(n-\lfloor gn \rfloor, n - \lceil \eps_Jn \rceil) > J+r\right]
        	\le J q^{-r} = n^{-\Omega(r)} \,. \label{eq:1/th-42}
    \end{eqnarray}
    By Corollary~\ref{cor:double-exp-shrinking-nphases}, $\eps_J < \left(\tfrac{n^{-\alpha}}{2a}\right)^{1/\ell}$.
    Consequently, there exists $\alpha' \in ]0,\alpha[$ such that $\eps_J < n^{-\alpha'}$ for any $n$ large enough.
    Without loss of generality we can assume that for any $u \le n^{1-\alpha'}$ we have $1-p_{n-u} \le n^{-\tau}$ (for $u \in [n^{1-\alpha}, n^{1-\alpha'}]$ it follows from the double exponential shrinking condition).
    Now suppose $u_0 \le n^{1-\alpha'}$ and consider $T(n-u_0,n)$.
    %Since for any $u \le n^{1-\alpha}$ we have $1-p_{n-u} \le n^{-\tau}$, this
	By the argument above, any of the $u_0$ uninformed nodes stays uninformed for $r \ge 1$ rounds with probability at most $n^{-\tau r}$.
	Then by the union bound, we have $\Pr[T(n-u_0, n) > r] \le P_r := \min\{1,n^{-\tau r + 1-\alpha'}\}$, that together with~\eqref{eq:1/th-42} proves the tail bound statement.
	
	Finally, $\E[T(n-u_0,n)] \le 1+\sum_{r\ge1} P_r = O(1)$, for any $u_0 \le n^{1-\alpha}$.
	Then, together with~\eqref{eq:2/th-42} it proves that $\E[T(n-\lfloor gn \rfloor, n)] \le \log_\ell \ln n + O(1)$.
%	To finish the proof we recall that from Observation~\ref{obs:double-exp-shrinking-nphases} it follows that $\eps_Jn \le n^{1-\alpha'}$.
\end{proof}
%\merk{this is the relaxed version of the corollary. In my opinion the stronger one is better (see below).}
%\begin{corollary}
%	There exist $A, \alpha > 0$ such that for any integer $r > 0$ we have
%	$$\Pr[T(n-\lfloor gn \rfloor,n) \le \E[T(n-\lfloor gn \rfloor,n)] + r] \le Ae^{-\alpha r}.$$
%\end{corollary}
%\begin{proof}
%	By construction, the interval $[n-\lfloor gn \rfloor, n - \lceil \eps_Jn \rceil]$ is cut into $J$ phases.
%	The number of rounds spent in each phase is stochastically dominated by $1+\Geom(1-q)$, where $q = n^{-\Theta(1)}$.
%	Consequently, for any integer $r > 0$ we have
%	$$\Pr\left[T(n-\lfloor gn \rfloor, n - \lceil \eps_Jn \rceil) > J+r\right] < J n^{-\Theta(1)\cdot r} \,.$$
%	Since $J = O(\log\log n)$, there exist $A', \alpha'$ such that $Jn^{-\Theta(1)\cdot r} \le A'e^{-\alpha'r}$.
%	Then, in the proof of Theorem~\ref{th:double-exp-shrinking-upper} we showed that for any $u_0 \le \lfloor \eps_J n \rfloor$ we have
%	$$\Pr[T(n-u_0, n) > r] \le \min\left\{1,n^{-\Theta(1)\cdot r}\right\}.$$
%	The claim of statement follows directly.
%\end{proof}

% \merk{stronger version of the corollary}
%\begin{corollary}
%	There exist $A, \alpha > 0$ such that for any integer $r > 0$ we have
%	$$\Pr[T(n-\lfloor gn \rfloor,n) \ge \E[T(n-\lfloor gn \rfloor,n)] + r] \le n^{-\alpha r + A}.$$
%\end{corollary}
%
%%\merk{I would prefer the version below without constants $A, \alpha$.}
%%\begin{corollary}
%%	For any integer $r > 0$ we have
%%	$$\Pr[T(n-\lfloor gn \rfloor,n) \le \E[T(n-\lfloor gn \rfloor,n)] + r] \le n^{-\Theta(1)\cdot r}.$$
%%\end{corollary}
%
%\begin{proof}
%	By construction, the interval $[n-\lfloor gn \rfloor, n - \lceil \eps_Jn \rceil]$ is cut into $J$ phases.
%	The number of rounds spent in each phase is stochastically dominated by $1+\Geom(1-q)$, where $q = n^{-\Theta(1)}$.
%	Consequently, for any integer $r > 0$ we have
%	$$\Pr\left[T(n-\lfloor gn \rfloor, n - \lceil \eps_Jn \rceil) > J+r\right] < J n^{-\Theta(1)\cdot r} \,.$$
%	Note that, since $J = O(\log\log n)$, we have $Jn^{-\Theta(1)\cdot r} = n^{-\Theta(1)\cdot r}$.
%	Then, in the proof of Theorem~\ref{th:double-exp-shrinking-upper} we showed that for any $u_0 \le \lfloor \eps_J n \rfloor$ we have
%	$$\Pr[T(n-u_0, n) > r] \le \min\left\{1,n^{-\Theta(1)\cdot r}\right\}.$$
%	The claim of statement follows directly.
%\end{proof}




%****************************************************************
%****************************************************************
%****************************************************************




\subsection{Double Exponential Shrinking Regime. Lower Bound.}

We now prove that under lower bound conditions comparable to the upper bound conditions of the previous section, we obtain a lower bound on the runtime equaling our upper bound apart from an additive constant.

\subsubsection{Double Exponential Shrinking Conditions}

Throughout this section, we assume that the following \emph{lower double exponential shrinking conditions} are satisfied.
%%%DEF - The double exponential shrinking conditions

%\merk{A: maybe it is reasonable to add a condition $ag^{\ell-1} < 1$ to ensure that the lower bound for the probability to stay uninformed is less than 1.}
\begin{defn}[lower double exponential shrinking conditions] \label{def:dexp}
	Let $g, \alpha \in ]0,1]$ and $\ell > 1$.
	Let $a, c \in \R_{\ge0}$.
    %Suppose $ag^{\ell-1} < 1$.
	We say that a homogeneous epidemic protocol satisfies \emph{the lower double exponential shrinking conditions} if for any $n$ big enough, the following properties are satisfied for all $u \in [n^{1-\alpha},gn]$.
	\renewcommand{\theenumi}{(\roman{enumi})}%
	\begin{enumerate}
		\item $1-p_{n-u} \ge a\left(\tfrac{u}{n}\right)^{\ell-1}$.
		\item $c_{n-u} \le c\tfrac{n}{u^2}$.
	\end{enumerate}
\end{defn}

Similarly to the upper double exponential shrinking conditions, we work mostly with the fraction $\eps := \tfrac{u}{n}$ of uninformed nodes instead of the absolute number~$u$.
Thus, the double exponential shrinking conditions turns into the following bounds, valid for all $\eps \in [n^{-\alpha},g]$ with $\eps n \in \N$.
\renewcommand{\theenumi}{(\roman{enumi})}%
\begin{enumerate}
	\item $1-p_{n(1-\eps)} \ge a\eps^{\ell-1}$.
	\item $c_{n(1-\eps)} \le \eps^{-2}\tfrac{c}{n}$.
\end{enumerate}

%The main result of this section is that the double exponential shrinking conditions (possibly with the fast finishing conditions in the end) give a sharp runtime of $\log_\ell \ln n + O(1)$.
The main result of this section is the following theorem.
%\begin{theorem}\label{th:double-exp-shrinking-lower}
%	Consider a homogeneous epidemic protocol satisfying the lower exponential shrinking condition.
%    Then, there is a constant $\alpha > 0$ such that for any $0 < g' \le g$,
%	\begin{equation}
%		\Expect[T(n-\lceil g'n \rceil, n-\lfloor n^{1-\alpha} \rfloor)] \ge \log_\ell \ln n + O(1) \notag.
%    \end{equation}
%\end{theorem}

\begin{theorem}\label{th:double-exp-shrinking-lower}
	Consider a homogeneous epidemic protocol satisfying the lower double exponential shrinking conditions in the interval $[n^{1-\alpha},gn]$. Let $r$ be a sufficiently large constant (possibly depending on $\alpha$). Then,
	\begin{align*}
		&\Expect[T(n-\lceil gn \rceil, n-\lfloor n^{1-\alpha} \rfloor)] \ge \log_\ell \ln n + O(1),\\
		&\Pr[T(n-\lceil gn \rceil, n-\lfloor n^{1-\alpha} \rfloor) \le \log_\ell \ln n - r] \le O(n^{-1+2\alpha\ell}),
	\end{align*}
\end{theorem}

\subsubsection{Round Targets and Failure Probabilities}
Let again $y(\eps)$ denote the fraction of uninformed nodes at the end of a round started with $\eps n$ uninformed ones.
%Since for the lower bound we have that $\Expect[y(\eps)] \le \eps n (1-p_{n(1-\eps)})$,
The double exponential shrinking conditions state that
\begin{equation}
    \Expect[y(\eps)] \ge E(\eps) := a\eps^\ell \notag.
\end{equation}

The next lemma gives that with good probability, $y(\eps)$ is at least the \emph{target} value $E(\eps)/2$.

%\merk{A: "for any fraction of uninformed nodes" allows not to mention that $\eps n \in \N$.}
\begin{lemma}\label{lem:double-exp-shrinking-failure-lower}
    For any fraction of uninformed nodes $\eps \in [n^{-\alpha}, g]$,
    \begin{equation}
        \Pr\left[y(\eps) \le \tfrac12E(\eps)\right] \le \tfrac{4+4c}{a^2 \eps^2 n} \le q := \tfrac{4+4c}{a^2}n^{2\alpha\ell-1} \notag.
    \end{equation}
\end{lemma}
\begin{proof}
    Applying Chebyshev's inequality and taking into account that $\Expect[y(\eps)] \ge E(\eps)$, we compute
    \begin{equation}
        \Pr[y(\eps) \le \tfrac12E(\eps)]
        \le \Pr\left[y(\eps) \le \Expect[y(\eps)] - \tfrac12E(\eps)\right]
        \le 4 \cdot \tfrac{\Var[y(\eps)]}{E(\eps)^2} \notag.
    \end{equation}
    By the same arguments like in Lemma~\ref{lem:double-exp-shrinking-variance}, $\Var[y(\eps)] \le \tfrac{1+c}{n}$.
    Since $\eps \ge n^{-\alpha}$, we have $E(\eps) \ge an^{-\alpha\ell}$, and the claim of the lemma directly follows.
\end{proof}
%Similarly to Lemma~\ref{lem:double-exp-shrinking-failure-upper}, the \emph{failure} probability $q$ does not depend on $\eps$.
%We also assume that $\alpha < \tfrac1{2\ell}$ to ensure that $q < 1$.

Similarly to the upper bound, our choice to analyze the double exponential shrinking regime only up to $n^{1-\alpha}$ uninformed nodes allows us to define $q$ independent of $\eps$.
We also assume that $\alpha< \tfrac1{2\ell}$ so that $q=n^{-\Theta(1)}$.

%%%  PHASE CALCULUS
\subsubsection{The Phase Calculus}

Let us define the sequence $\eps_j$ recursively by
\begin{equation}
    \eps_0 := g, \quad \eps_{j+1} := \tfrac12E(\eps_j) \notag.
\end{equation}
The next observation follows from the definition by a simple induction.
The $\eps_j$ are decreasing simply because $\eps_{j+1}=\tfrac12E(\eps_j) < \Expect[y(\eps_j)] \le \eps_j$.
Note that $y(\eps) \le \eps$ with probability one for any homogeneous protocol.
\begin{observation}
	For all $j \ge 1$,
    $\eps_j = (a/2)^{\frac{\ell^j-1}{\ell-1}} g^{\ell^j}$.
    The $\eps_j$ form a decreasing sequence.% if $g < (a/2)^{-\tfrac1{\ell-1}}$.
\end{observation}
In the rest of the section we assume that $g < (a/2)^{-\tfrac1{\ell-1}}$.
Applying logarithm twice to the previous equation one can also see the following.
\begin{observation}\label{obs:double-exp-shrinking-nphases-lower}
    There exists $J = \log_\ell \ln n + O(1)$ such that $\eps_J > n^{-\alpha}$.%, but $\eps_{J+1} \le n^{-\alpha}$.
\end{observation}
%Then by applying double logarithm to the previous equation, one can see that there exists $J = \log_\ell \ln n + O(1)$ such that $\eps_J > n^{-\alpha}$, but $\eps_{J+1} < n^{-\alpha}$.

As before, we say that the process is in phase $j$ if the fraction $\eps$ of uninformed nodes is in $]\eps_{j+1}, \eps_j]$.
%Since $q$ does not depends on epsilon, the following lemma is also obvious.
\begin{lemma}\label{lem:double-exp-shrinking-ETj-lower}
    If the process starts in phase $j$, $j < J$, then the probability that after one round it is in phase $j+2$ or higher is at most $q$.
\end{lemma}
%\merk{A: should we provide such a proof?}
\begin{proof}
    Consider a round starting with $\eps n$ uninformed nodes, where $\eps \in ]\eps_{j+1},\eps_j]$.
    By construction, the process leapfrogs phase $j+1$ if $y(\eps) \le \eps_{j+2} = \tfrac12E(\eps_{j+1})$.
    Since $E(\cdot)$ is an increasing function, an upper bound for the probability to jump over phase $j+1$ is
    \begin{equation}
        \max_{\eps\in]\eps_{j+1},\eps_j]} \Pr[y(\eps) \le \tfrac12E(\eps_{j+1})]
        \le \max_{\eps\in]\eps_{j+1},\eps_j]} \Pr[y(\eps) \le \tfrac12E(\eps)]
        \le q \notag.
    \end{equation}
\end{proof}

\begin{proof}[Proof of Theorem~\ref{th:double-exp-shrinking-lower}]
  Consider the rumor spreading process starting with $\eps_0 n = gn$ uninformed nodes. By Lemma~\ref{lem:double-exp-shrinking-ETj-lower}, with probability at least $(1-q)^J \ge 1 - Jq$, the process visits each phase $j \in [0..J-1]$, which naturally takes at least $J-1$ rounds. Consequently, by definition of $J$ in Observation~\ref{obs:double-exp-shrinking-nphases-lower}, we have
	\begin{align*}
		\Expect[T(n-\lceil gn \rceil, n-\lfloor n^{1-\alpha} \rfloor)]
			&\ge \Expect[T(n-\lceil n\eps_0 \rceil, n-\lfloor n\eps_J\rfloor)] \\
			&\ge (J-1)(1-Jq) = \log_\ell \ln n + O(1) \notag.
	\end{align*}
	The large-deviation statement follows immediately from adding the failure probabilities $\frac{4+4c}{a^2\eps_j^2 n}$, $j = 0, \dots, J-1$, from Lemma~\ref{lem:double-exp-shrinking-failure-lower}.
\end{proof}
%\begin{proof}[old proof]
%	Without loss of generality we can assume that $g' = g$.
%    The proof uses the same arguments as the proof of Theorem~\ref{th:exp-shrinking-lower} for the exponential shrinking.
%    By definition of $\eps_j$ and Observation~\ref{obs:double-exp-shrinking-nphases-lower}, we have
%    \begin{equation}
%    	\Expect[T(n-\lceil gn \rceil, n-\lfloor n^{1-\alpha} \rfloor)]
%    		\ge \Expect[T(n-\lceil n\eps_0 \rceil, n-\lfloor n\eps_J\rfloor)] \notag.
%    \end{equation}
%    Suppose $\tau$ be the smallest round $t$ in which the process leapfrogs a phase ($\tau = \infty$, if this never happens).
%    The spreading time is hence at least $\min\{J, \tau\}$.
%    By Lemma~\ref{lem:double-exp-shrinking-ETj-lower}, $\Pr[\tau=t] \le q = O\left(n^{2\alpha\ell-1}\right)$ for any $t$.
%    Therefore,
%    \begin{align}
%        \Expect&[T(n-\lceil gn \rceil, n-\lfloor n^{1-\alpha} \rfloor)] \notag \\
%%            \ge \Expect[T(n-\lceil gn \rceil, n-\lfloor n\eps_J\rfloor)] \notag \\
%		& \ge J \Pr[\tau > J] + \sum_{t=1}^{J-1} t\cdot\Pr[\tau = t] \notag \\
%		& = J - \sum_{t=1}^{J-1} qt
%			= \log _\ell \ln n + O(1) \notag.
%    \end{align}
%\end{proof} 
 %\input{double_exp_shrinking.tex}

 \section{Application of our Method to the Classic Protocols}\label{sec:classics}

In this section, we define the classic push, pull, and push-pull protocols, give some background information on them, and show how the methods developed above easily give very sharp (tight apart from additive constants) rumor spreading times. For this, we easily convince ourselves that all three protocols satisfy the exponential growth conditions. The push protocol satisfies the exponential shrinking conditions, whereas the pull and push-pull protocols both satisfy the double exponential shrinking conditions. For all these conditions, we can show for the upper and lower bound part of the conditions the same value for the critical parameter $\gamma_n$, ${\rho_n}$, and~$\ell$), which is why we then obtain sharp estimates for the rumor spreading times.

We stick to the usual convention that for rumor spreading in complete graphs we allow that nodes call themselves, that is, the random communication partner is chosen uniformly at random from all nodes. By replacing all $(1-\tfrac 1n)$ terms with $(1-\tfrac 1 {n-1})$, the elementary proofs below can easily be transformed to the setting where nodes only call random neighbors in the complete graph.


%\merk{this too relaxed. do we need it somewhere?}
%Formally, in Theorem~\ref{th:example-push},~\ref{th:example-pull},~and~\ref{th:example-push-n-pull} and in Proposition~\ref{prop:example-push-w-failures},~\ref{prop:example-pull-w-failures},~\and~\ref{prop:example-push-n-pull-w-failures}, one should also verify the conditions of Lemma~\ref{lem:general-connect} to ensure that the transition from growth to shrinking regimes takes at most $O(1)$ rounds.
%This seems to be obvious for the protocols discussed in this section, so we don't provide the proofs.

\subsection{Push Protocol}

The push protocol appeared in the computer science literature first in the works of Frieze and Grimmett~\cite{FriezeG85} (as a technical tool to analyze the all-pairs shortest path problem on complete digraphs with random edge weights) and, under the name \emph{rumor mongering}, Demers et al.~\cite{Demers87}, the first work that proposed rumor spreading as a robust and scalable method to maintain consistency in replicated databases. In the push protocol, in each round each node knowing the rumor calls a random neighbor and gossips the rumor to it.

The push protocol is the most intensively studied rumor spreading process. It has been proven that with high probability it disseminates a rumor known to a single node to all others in time logarithmic in the number $n$ of nodes when the communication networks is a complete graph (see below), a random graph in the $G(n,p)$ model with $p \ge (1+\eps) \ln(n)/n$, that is, only very slightly above the connectivity threshold, or a hypercube~\cite{FeigePRU90}, or a random regular graph~\cite{FountoulakisP10} (and this list is not complete).

For the complete graph, Frieze and Grimmett~\cite{FriezeG85} show (among other results) that with high probability, the rumor spreading time is $\log_2 n + \ln n \pm o(\log n)$. This estimate was sharpened by Pittel~\cite{Pittel87}, who proved that for any $h = \omega(1)$, the rumor spreading time with high probability is $\log_2 n + \ln n \pm h(n)$. The first explicit bound for the expected runtime, $\lfloor \log_2 n \rfloor + \ln n - 1.116 \le E[S_n] \le \lceil \log_2 n \rceil + \ln n + 2.765 + o(1)$ was shown in~\cite{DoerrK14}. All these works are relatively technical (see, e.g., the 9-pages proof of~\cite{Pittel87}) and heavily exploit particular properties of the push process (e.g., a birthday paradox argument for the first $\log_2(o(\sqrt n))$ calls and a reduction to the coupon collector process for the last roughly $\ln n$ rounds in~\cite{DoerrK14}).

With the methods developed in this work, we only need to show that the push protocol satisfies the exponential growth and shrinking conditions (with $\gamma_n = 1$ and ${\rho_n}=1$), which is very easy. This reproves the bound of~\cite{DoerrK14} cited above apart from the additive constants, but with a, as we believe, much simpler proof.



%In the push protocol only informed nodes make calls and forward the rumor.
%\begin{def*}[Basic push protocol]
%    Let a graph $G$ given and its vertex $v$ initially knows the rumor.
%    %At some discrete instances of time called rounds
%    At each round, each node knowing the rumor chooses uniformly at random a neighbor in $G$ and transfers the rumor to it.
%\end{def*}

\begin{theorem}\label{th:example-push}
	The expected rumor spreading time of the push protocol on the complete graph with $n$ vertices is $\log_2 n + \ln n \pm O(1)$.
\end{theorem}

\begin{proof}
	Consider one round of the protocol. Let $x_1, x_2$ be two different uninformed nodes. Let $X_1$ and $X_2$ be the indicator random variables for events that $x_1$ resp.~$x_2$ become informed. Clearly, if we condition on that $x_1$ becomes informed, then it is slightly less likely that $x_2$ becomes informed. Consequently, $\Cov[X_1,X_2] < 0$ and the covariance part of the exponential growth and shrinking conditions is satisfied.
%	 one and another gets the rumor correspondingly.
%	It is easy to see that the corresponding random indicator variables are negatively correlated.
%	Indeed, if one of the nodes gets informed, the probability that another one also gets informed is less than in the unconditioned case, because some calls are already wasted to inform the first node.
%%	As this observation doesn't depend on the number of informed nodes, it remains only to bound the probability of one node to get informed.

	Therefore, it remains to analyze the probability $p_k$ of an uninformed node to become informed.
	
	For the exponential growth regime, suppose that $k$ nodes are informed.	An uninformed node remains uninformed when all informed nodes fail to call it.
%	The probability that nobody calls this node in current round is $\left(1-\tfrac1n\right)^k$.
	Consequently, it becomes informed with probability $p_k = 1 - \left(1-\tfrac1n\right)^k$. With the estimates
	\[\tfrac{k}{n} - \tfrac{k^2}{2n^2} \le p_k \le \tfrac{k}{n}\]
	we see that the protocol satisfies the exponential growth conditions with parameter $\gamma_n = 1$. More precisely, we can take $\gamma_n=1$, $f=1$, $b=0$ and $c=0$ is both the upper and lower bound exponential growth condition. Taking $a=1$ satisfies the upper exponential growth condition, taking $a=0$ suffices for the lower exponential growth condition.
%	\[
%		p_k = 1 - \left(1-\tfrac1n\right)^k
%		\le \tfrac{k}{n} + O\left(\tfrac{k^2}{n^2}\right),
%	\]
%	so the protocol satisfies the exponential growth conditions with parameter $\gamma_n = 1$.
	
	For the exponential shrinking conditions, suppose that there are $u$ uninformed nodes.	Again, the probability for a node to stay uninformed is $1-p_{n-u} = \left(1-\tfrac1n\right)^{n-u}$.
By Corollary~\ref{cor:prelim:(1-1/n)^(n-u)}, for any $u<n$ we have the following estimate.
\[\tfrac1e \le 1-p_{n-u} \le \tfrac1e + \tfrac2e\cdot\tfrac{u}{n}\]
	The push protocol hence satisfies the exponential shrinking conditions (from $gn := \tfrac12 n$ uninformed nodes on) with parameter ${\rho_n}=1$.
	
	By Theorems~\ref{th:exp-growth-upper},~\ref{th:exp-growth-lower},~\ref{th:exp-shrinking-upper},~and~\ref{th:exp-shrinking-lower}, the expected rumor spreading time of the push protocol is $\log_2 n + \ln n \pm O(1)$.
\end{proof}

\subsection{Pull Protocol}

The pull protocol is dual to the push protocol in the sense that now in each round, each uninformed node calls a random neighbor and becomes informed if the latter was informed. We are not aware of a convincing practical motivation for this protocol, however, it has been very helpful in proving performance guarantees for other protocols, e.g., in~\cite{Giakkoupis11}. Note that the duality between the two protocols immediately shows that the probability that the push protocol in $t$ rounds moves a rumor initially present at a node $u$ to a node $v$ equals the probability that the pull protocol gets the rumor from $v$ to $u$ in $t$ rounds, but this does not imply that both protocols have the same rumor spreading times (as also Theorems~\ref{th:example-push} and~\ref{th:example-pull} show).

We are not aware of any performance guarantees proven for the pull protocol. Some existing results for the push protocol obviously can be transformed into results for the pull protocol via the duality and union bounds. For complete graphs, we do not see how this would give bounds stronger than $\Theta(\log n)$.

Interestingly, the expansion phase of the pull protocol (when viewed from a distance) resembles the expansion phase of the push protocol---the probability that an uninformed node becomes informed in a round starting with $k$ informed nodes is $p_k = \tfrac kn$ and thus, for small $k$, very close to the $\tfrac kn - \Theta(\tfrac{k^2}{n^2})$ probability of the push protocol. Nevertheless, the precise processes are very different. For example, in the push protocol we almost surely observe a perfect doubling of the number of informed nodes as long as $o(\sqrt n)$ nodes are informed. For the pull protocol, the number of newly informed nodes in the first round is binomially distributed with parameters $n-1$ and $\frac 1n$, so the probability for a perfect doubling is asymptotically equal to $\tfrac 1e$. For this reason, the existing analyses of the push protocol cannot easily be transferred to the pull protocol. This is different for our method, which ignored many details of the process and only relies on the rough characteristics $p_k$ and $c_k$ of the process. We show below that the similar values of $p_k$ lead to the same $\log_2 n \pm O(1)$ time it takes to inform a constant fraction of the nodes. From that point on, the double exponential shrinking conditions are obvious, leading to a double logarithmic remaining time.

%Unlike the push protocol, in the pull protocol only uninformed nodes make calls trying to hit their informed neighbors and "pull" the rumor from them.
%
%\begin{def*}[Basic pull protocol]
%    Let a graph $G$ given and its vertex $v$ initially knows the rumor.
%%    At some discrete instances of time called rounds
%	Each round, each uninformed node calls one of its neighbors chosen uniformly at random.
%    If the called node knows the rumor, it transfers the rumor to its interlocutor.
%\end{def*}

\begin{theorem}\label{th:example-pull}
	The expected rumor spreading time of the pull protocol on the complete graph with $n$ vertices is $\log_2 n + \log_2 \ln n \pm O(1)$.
\end{theorem}

\begin{proof}
  Clearly, the events that uniformed nodes become informed are mutually independent. Hence the covariance conditions are exponential growth and double exponential shrinking regimes are satisfied.
	
	An uninformed node becomes informed if its call reaches an informed node. Hence for all $k \in [1..n-1]$, we have  $p_k = k/n$. This shows that both the upper and lower exponential growth conditions are satisfied with parameter $\gamma_n=1$ (and $f=1$, $a=0$, $b=0$, $c=0$).
	
	For the same reason, the probability $1-p_{n-u}$ that an uninformed node remains uninformed when $u$ nodes are uninformed, is $1- p_{n-u} = 1 - \tfrac{n-u}{n} = \tfrac un$. Consequently, the upper and lower double exponential shrinking conditions are satisfied with $\ell = 2$ (and $g=1$, $\alpha = 0$, $a=1$, and $c=0$).
	
%	Suppose then that there are $\eps n$ ($0<\eps<1$) uninformed nodes.
%	The probability that an uninformed node stays uninformed in current round is equal to $\eps$, so the protocol satisfies the double exponential shrinking conditions with parameter $\ell = 2$.

	By Theorems~\ref{th:exp-growth-upper},~\ref{th:exp-growth-lower},~\ref{th:double-exp-shrinking-upper},~and~\ref{th:double-exp-shrinking-lower}, the expected rumor spreading time is $\log_2 n + \log_2 \ln n \pm O(1)$.
\end{proof}

\subsection{Push-Pull Protocol}\label{sec:push-pull}

In the push-pull protocol, both informed and uninformed nodes contact a random neighbor in each round. If one of the two partners of such a conversation is informed, then also the other one becomes informed. The push-pull protocol is popular for a number of reasons.

The push-pull protocol (called \emph{anti-entropy} there) was found to be very reliable in the first experimental work on epidemic algorithms~\cite{Demers87}. The seminal paper by Karp et al.~\cite{KarpSSV00} proved that the push-pull protocol disseminates a rumor in a complete graph in $\log_3 n \pm O(\log\log n)$ rounds with high probability. This not only is faster than the push and pull protocols, but it allows implementations using only few messages per node. The just mentioned rumor spreading time stems from an exponential growths phase of length roughly $\log_3 n$ and a double exponential shrinking phase. Hence by making informed nodes stop their activity after the exponential growth phase, the total number of messages can be reduced massively.

The push-pull protocol was also investigated in models for social networks. Clearly, when modeling human communication, say people randomly meeting at parties and chatting, a push-pull spreading mechanism makes sense. However, also from the algorithmic viewpoint, it was observed that in graphs with a non-concentrated degree distribution the push-pull protocol greatly outperforms the push and pull protocols. This was first made precise by Chierichetti, Latanzi, and Panconesi~\cite{ChierichettiLP09}, who showed that the push-pull protocol spreads a rumor in a preferential attachment graph~\cite{BarabasiA99,BollobasR03} in time $O(\log^2 n)$, whereas both the push and the pull protocols need time $\Omega(n^\alpha)$ for some constant $\alpha > 0$ to inform all nodes. The precise rumor spreading time of $\Theta(\log n)$ of the push-pull protocol was shown in~\cite{DoerrFF11} (see also~\cite{DoerrFF12acm}). There is was also proven that the rumor spreading time reduces to $\Theta(\frac{\log n}{\log\log n})$ when the communication partners are chosen randomly but with the previous partner excluded. This first sublogarithmic rumor spreading time was quickly followed up by other fast rumor spreading times in networks modeling social networks, e.g.,~\cite{FountoulakisPS12,DoerrFF12,MehrabianP14}.

The push-pull protocol also performs well and admits strong theoretical analyses when the network has certain general expansion properties like a good vertex expansion~\cite{GiakkoupisS12,Giakkoupis14} or a low conductance~\cite{MoskAoyamaS06,ChierichettiLP10stoc,Giakkoupis11}.

%
%If we allow the push protocol and the pull protocol to run simultaneously, we get the push-pull protocol.
%\begin{def*}[Push-pull protocol]
%    Let a graph $G$ given and its vertex $v$ initially knows the rumor.
%%    Suppose we also have a clock shared by all nodes counting the rounds since the rumor appeared.
%    Each round each node chooses one of its neighbors in $G$ uniformly at random and communicates with it ("calls it").
%    If two nodes communicate in current round and one of them is informed, then another one also gets informed at the end of the round.
%\end{def*}

%As before let the random indicator variables $X_i$ are such that $X_i = 1$ if and only if uninformed node $i$ gets informed at current round.
%It is easy to see that $X_i = \max\{pull(i),pushed(i)\}$, where
%\begin{itemize}
%	\item $pull(i) = 1$ if and only if uninformed node $i$ makes successful \emph{pull} call, i.e., hits some informed node;
%	\item $pushed(i) = 1$ if and only if uninformed node $i$ is called by an informed node, we name such call a \emph{push} call.
%\end{itemize}

%As the basic push-pull protocol on the complete graph is a homogeneous epidemic protocol, it suffices to show that the it satisfies the corresponding exponential growth and exponential shrinking conditions.

%Before the proofs we will introduce some useful notations.
%We say that the call is \emph{successful} if it causes a new informed node, otherwise it is \emph{failed}
%If the call is made by an informed node, then it is a \emph{push} call, otherwise it is a \emph{pull} one.
%The following indicator random variables are used to reduction of formulas.
%\begin{itemize}
%	\item $pull(i)=1$ if and only if the $i$-th node calls an informed node.
%	\item $pushed(i)=1$ if and only if the $i$-th node is called by an informed node.
%	\item $X_i=1$ if and only if the $i$-th uninformed node becomes informed at current round.
%\end{itemize}
%Obviously, $X_i = \max\{pull(i),pushed(i)\}$.
%It is also easy to see that $X_i$ are pairwise negatively correlated.
%This observation is sufficient for the covariance for both exponential growth and shrinking regimes.

%\begin{lemma}~\label{lem:push-pull-cov}
%	For any $i \ne j$, $\Cov[X_i,X_j] \le 0$.
%\end{lemma}
%\begin{proof}
%	Since the protocol is uniform, it suffices to show that $\Cov[X_1, X_2] \le 0$.
%%	The covariation of binary random variables is equal to
%%	$$\Cov[X_i,X_j] = \Pr[X_i=X_j=1] - \Pr[X_i=1]Pr[X_j=1].$$
%%	Since the protocol is uniform, it suffices to show that $\Pr[X_i=1|X_j=1] \le \Pr[X_i=1]$, for any $i \ne j$.
%	Equivalently, we can show that $\Pr[X_1=1|X_2=1] \le \Pr[X_1=1]$.
%	Indeed,
%	\begin{align}
%		\Pr&[X_1=1|X_2=1] \notag\\
%		& = \Pr[X_1=1|pull(2)=1] \cdot \Pr[pull(2)=1|X_2=1] \notag \\
%		& \qquad + \Pr[X_1=1|pull(2)=0,X_2=1] \cdot \Pr[pull(2)=0|X_2=1]. \label{eq:push-pull-1}
%	\end{align}
%	
%	Obviously the event ``$pull(2)=1" \AND ``X_2=1$" is nothing but ``$pull(2)=1$".
%	Then we see that $\Pr[X_1=1|pull(2)=1] = \Pr[X_1=1]$, because the outcoming call of the uninformed node have no influence on the informing any other node.	
%	It is also easy to see that $\Pr[X_1=1|pull(2)=0,X_2=1] \le \Pr[X_1=1]$, because since one of two nodes is called by some informed node, there are less chances of another one to be informed.
%	So we can substitute the long conditional probabilities in~\eqref{eq:push-pull-1} as follows.
%	\begin{align*}
%		\Pr&[X_1=1|X_2=1] \\
%		& \le \Pr[X_1=1] \cdot \left(\Pr[pull(2)=1|X_2=1]+\Pr[pull(2)=0|X_2=1]\right) \\
%		& = \Pr[X_1=1]
%	\end{align*}
%\end{proof}

\begin{theorem}\label{th:example-push-n-pull}
	The expected rumor spreading time of the push-pull protocol on the complete graph with $n$ vertices is $\log_3n + \log_2\ln n \pm O(1)$.
\end{theorem}

\begin{proof}
  We again discuss the covariance condition first. Consider one round of the protocol. Let $x_1$, $x_2$ be two different uninformed nodes. For $i = 1,2$, let $X_i$ be the indicator random variable for the event that $x_i$ becomes informed in this round, $Y_i$ the indicator random variable for the event that $x_i$ is called by an informed node, and $Z_i$ the indicator random variable for event that $x_i$ calls an informed node. Clearly, $X_i = \max\{Z_i,Y_i\}$.

  We show $\Cov[X_1,X_2] \le 0$, and thus all covariance conditions, by showing that $\Pr[X_1=1\mid X_2=1] \le \Pr[X_1=1]$. We have
    \begin{align}
        \Pr[X_1=&1 \mid X_2=1]
        = \Pr[X_1=1 \mid X_2=1 \AND Z_2=1] \cdot \Pr[Z_2=1 \mid X_2=1] \notag \\
        & + \Pr[X_1=1 \mid X_2=1 \AND Z_2=0] \cdot \Pr[Z_2=0 \mid X_2=1] \label{eq:push-pull-1}.
    \end{align}
    Since the intersection of events $Z_2=1 \AND X_2=1$ is equivalent to the single event $Z_2=1$ and the outgoing call of the uninformed node cannot inform any node, we have
    \begin{equation}
        \Pr[X_1=1 \mid X_2=1  \AND Z_2=1] = \Pr[X_1=1 \mid Z_2=1] = \Pr[X_1=1]. \label{eq:push-pull-2}
    \end{equation}
    When $Z_2=0 \AND X_2=1$ holds, then $x_2$ becomes informed via a push call, which is not available anymore to inform $x_1$. Hence
    \begin{equation}
        \Pr[X_1=1 \mid Z_2=0 \AND X_2=1] \le \Pr[X_1=1]. \label{eq:push-pull-3}
    \end{equation}
    From~\eqref{eq:push-pull-1} to~\eqref{eq:push-pull-3} we obtain $\Pr[X_1=1 \mid X_2=1] \le \Pr[X_1=1]$.
	%\begin{align*}
%		\Pr&[X_1=1 \mid X_2=1] \\
%		& \le \Pr[X_1=1] \cdot \left(\Pr[pull(2)=1 \mid X_2=1]+\Pr[pull(2)=0 \mid X_2=1]\right) \\
%		& = \Pr[X_1=1]
%	\end{align*}

	An uninformed node remains uninformed if it is not called by any informed node and it calls an uninformed node itself. Hence $p_k = 1 - \left(1-\tfrac1n\right)^k\cdot\tfrac{n-k}{n}$.
%    \[
%    	1 - \Pr[pull(1)=0]\cdot\Pr[pushed(1)=0]
%    	= 1 - \left(1-\tfrac1n\right)^k\cdot\tfrac{n-k}{n}.
%    \]
    Using the estimates from Lemma~\ref{lem:prelim:(1-1/n)^k} we obtain
    \[
    	2\tfrac{k}{n} - \tfrac{3k^2}{2n^2} \le p_k \le 2\tfrac{k}{n}% + \tfrac{k^2}{n^2}
    \]
    and see that the protocol satisfies the exponential growth conditions with $\gamma_n = 2$.



%    For the double exponential shrinking conditions, suppose that there are $u$ uninformed nodes.
    Likewise, the probability $1-p_{n-u}$ that an uninformed node stays uninformed in a round starting with $u$ uninformed nodes is equal to $\tfrac{u}{n} \left(1-\tfrac1n\right)^{n-u}$.
    With Corollary~\ref{cor:prelim:(1-1/n)^(n-u)}, we estimate \[\tfrac1e\cdot\tfrac{u}{n} \le 1-p_{n-u} \le \tfrac{u}{n}.\]
    Therefore, the protocol satisfies the double exponential shrinking conditions with $\ell = 2$.

    By Theorems~\ref{th:exp-growth-upper},~\ref{th:exp-growth-lower},~\ref{th:double-exp-shrinking-upper},~and~\ref{th:double-exp-shrinking-lower}, the expected rumor spreading time is $\log_3 n + \log_2 \ln n \pm O(1)$.
\end{proof}


%\begin{lemma}\label{lem:push-pull-growth}
%	Let us consider one round of push-pull protocol with $k$ informed nodes.
%	Then $\Pr[X_1=1] = 2\tfrac{k}{n} + O\left(\tfrac{k^2}{n^2}\right)$, for any $k < gn$.
%%    The basic push-pull protocol satisfies the exponential growth conditions with parameter $P(n)=2$.
%\end{lemma}
%\begin{proof}
%%	Let there are $k<gn$ informed nodes for some $g\in]0,1[$.
%%	It suffices to evaluate the probability of being informed $\Pr[X_i=1]$ for the $i$-th uninformed node.
%	Indeed, some node stays uninformed in current round if it is not called by any informed node and its own pull call goes to an uninformed one.
%	Therefore,
%    \begin{align*}
%	    \Pr&[X_1=1] = 1 - \Pr[X_1=0] \\
%	    & = 1 - \Pr[pull(1)=0]\cdot\Pr[pushed(1)=0] \\
%	    & = 1 - \left(1-\tfrac1n\right)^k\cdot\tfrac{n-k}{n}.
%    \end{align*}
%    To finish the proof we recall that $\left(1-\tfrac1n\right)^k = 1 - \tfrac{k}{n} + O\left(\tfrac{k^2}{n^2}\right)$.
%    Then by the simple computation we see that $\Pr[X_1=1] = 2\tfrac{k}{n} - O\left(\tfrac{2k^2}{n^2}\right)$.
%%    By the simple analysis, we see that $1-\frac{k}{n} \le (1-\frac1n)^k \le 1-\frac{k}{n}+\frac{k^2}{2n^2}$.
%%    \[
%%    	2\tfrac{k}{n} - \tfrac{2k^2}{n^2} \le \Pr[X_i=1] \le 2\tfrac{k}{n},
%%    \]
%%    and the exponential growth conditions are satisfied.
%\end{proof}
%
%\begin{lemma}\label{lem:push-pull-shrinking}
%	Let us consider one round of push-pull protocol with $\eps n$ uninformed nodes.
%	Then $\Pr[X_1=0] = O(\eps)$.
%%    The basic push-pull protocol satisfies the double exponential shrinking conditions with parameter $\ell = 2$.
%\end{lemma}
%\begin{proof}
%%    Let us consider the $i$-th uninformed node in a round in which $\eps n$ nodes remain uninformed.
%%    Our goal is to prove that the probability that this node remains uninformed at the end of the round is upper and lower bounded by $a\eps^\ell$ for some values of $a$.
%    Indeed, a node stays uninformed if and only if it misses its own outgoing call it is not called by any informed node.
%    Therefore,
%    \[
%        \Pr[X_1=0] = \eps \left(1-\tfrac1n\right)^{(1-\eps)n} = O(\eps).
%%        \sim \eps \left(\tfrac1e+O\left(\tfrac1n\right)\right) \left(1-\tfrac1n\right)^{-\eps n}.
%    \]
%%    So, if $\eps \le g$ for some $g\in]0,1[$, then $\eps/e \le \Pr[X_i=0] \le \eps$, that finishes the proof.
%\end{proof}
%
%\begin{proof}[Proof of Theorem~\ref{th:example-push-n-pull}]
%	Lemma~\ref{lem:push-pull-cov} claims that $X_i$ are pairwise negatively correlated.
%	Together with Lemma~\ref{lem:push-pull-growth} it implies the exponential shrinking conditions with parameter $\gamma_n=2$.
%	Finally, Lemma~\ref{lem:push-pull-shrinking} claims that the push-pull protocol satisfies the double exponential shrinking conditions with parameter $\ell = 2$.
%	Therefore the expected rumor spreading time is $\log_3n + \log_2\ln n + O(1)$.
%\end{proof}

\section{Robustness, Multiple Calls, and Dynamic Graphs}\label{sec:more examples}

In this section, we apply our analysis method to settings (i)~in which calls fail independently with constant probability, (ii)~in which nodes are allowed to call a random number of other nodes instead of one as proposed in~\cite{PanagiotouPS15}, and (iii)~to a simple dynamic graph setting.

\subsection{Transmission Failures}

One key selling point for randomized rumor spreading, and more generally gossip-based algorithms, is that all these algorithms due to the intensive use of independent randomness are highly robust against all types of failures. In this subsection, we analyze the performance of the three classic protocols in the presence of independent transmission failures, that is, when calls are successful only with probability $p < 1$. Not unexpectedly, we can show that the rumor spreading times only increase by constant factors. However, we also observe a structural change, namely that the extremely fast double exponential shrinking previously seen with the pull and push-pull protocols is replaces by the slower single exponential shrinking regime. This has the important implication that the message complexity of the simple push-pull protocol (where messages are counted as in~\cite{KarpSSV00} and the protocol is assumed to stop when a suitable time limit is reached) increases from the theoretically optimal value of $\Theta(n \log\log n)$ to $\Theta(n \log n)$, see the remark following the proof of Theorem~\ref{th:example-push-n-pull-w-failures}.

While the robustness of randomized rumor spreading is consistently emphasized in the literature, only relatively few proven guarantees for this phenomenon exist. All results model communication failures by assuming that each call independently with probability $1-p$ fails to reach its target. The usual assumption is that the protocol does not take notice of such events. Els\"asser and Sauerwald~\cite{ElsasserS09} show for any graph $G$ that if the push protocol spreads a rumor with probability $1-O(1/n)$ to all nodes in time $T$, then the push protocol with failures succeeds in informing all nodes with probability $1 - O(1/n)$ in time $\tfrac 6p T$. This was made more precise for complete graphs in~\cite{DoerrHL13}, for which a rumor spreading time of $\log_{1+p} + \tfrac 1p n \pm o(\log n)$ was shown to hold with high probability. The same result also holds for random graphs in the $G(n,p')$ model when the edge probability $p'$ is $\omega(\log(n)/n)$, that is, asymptotically larger than the connectivity threshold~\cite{FountoulakisHP10}. To the best of our knowledge, these few results are all that is known in terms of proven guarantees for the classic rumor spreading protocols in the presence of failures.

We now use the methods developed in this work to obtain very sharp estimates for the runtimes of the classic protocols on complete graphs when calls fail independently with probability $1-p$, $p < 1$. As in Sections~\ref{sec:classics}, the growth or shrinking conditions valid in each case are easily proven, showing again the versatility of our approach.



%In the three classical examples above we assumed that all communications are reliable.
%Our method lets to analyze the protocols above if we suppose that each call can be failed independently with constant probability $1-p$.
%The expected runtimes for push, pull and push-pull protocols are provided in Proposition~\ref{prop:example-push-w-failures},~\ref{prop:example-pull-w-failures},~and~\ref{prop:example-push-n-pull-w-failures} correspondingly.

\begin{theorem}\label{th:example-push-w-failures}
    The expected rumor spreading time for the push protocol with success probability $p$ on the complete graph of size $n$ is equal to
    \[
        \log_{1+p} n + \tfrac1p\ln n \pm O(1).
    \]
\end{theorem}
\begin{proof}
	With the same argument as in the proof of Theorem~\ref{th:example-push}, we see that the covariances regarded in the covariance conditions are all negative.
	
	Consider an uninformed node in a round started with $k$ informed nodes. The probability that it becomes informed in this round is $p_k = 1 - (1-\tfrac{p}{n})^k$. By Lemma~\ref{lem:prelim:(1-1/n)^k}, we estimate
	\[
		\tfrac{pk}{n} - \tfrac{p^2k^2}{2n^2} \le p_k \le \tfrac{pk}{n}
	\]
	for all $k < n$ and see that the protocol satisfies the exponential growth conditions in $[1,n[$ with $\gamma_n = p$.
%	\begin{equation}
%		1 - \left(1-\tfrac{p}{n}\right)^k
%		= \tfrac{pk}{n} + O\left(\tfrac{p^2 k^2}{n^2}\right) \notag.
%	\end{equation}
	
	Similarly, the probability that an uninformed node in a round starting with $u := n-k$ uninformed nodes stays uninformed, is $1-p_{n-u} = \left(1-\tfrac{p}{n}\right)^{n-u}$. By Corollary~\ref{cor:prelim:(1-p/n)^(n-u)}, we estimate
	\[
		e^{-p} \le 1-p_{n-u} \le e^{-p} (1+\tfrac{2pu}{n})
	\]
	for all $u < n$ and thus have the exponential shrinking conditions with ${\rho_n} = p$ for all $u \le n/2$.
	
	By Theorems~\ref{th:exp-growth-upper},~\ref{th:exp-growth-lower},~\ref{th:exp-shrinking-upper},~and~\ref{th:exp-shrinking-lower}, the expected rumor spreading time is $\log_{1+p} n + \tfrac1p\log n \pm O(1)$.
\end{proof}

The result above and its proof are valid for $p=1$ and then coincide with Theorem~\ref{th:example-push}. For the pull protocol and the push-pull protocol, we observe a substantial change of the process when transmission errors occur. In this case, an uninformed node stays uninformed with probability at least $1-p$, so the double exponential shrinking conditions cannot be satisfied. Instead, we observe that the single exponential shrinking conditions are satisfied.

\begin{theorem}\label{th:example-pull-w-failures}
	The expected rumor spreading time of the pull protocol with success probability $p<1$ on the complete graph of size $n$ is equal to
	\begin{equation}
		\log_{1+p}n + \tfrac1{\ln \frac{1}{1-p}} \ln n \pm O(1) \notag.
	\end{equation}
\end{theorem}

\begin{proof}
	As in the proof of Theorem~\ref{th:example-pull}, the events that uninformed nodes become informed are mutually independent. Hence all covariance conditions are satisfied with $c=0$. The probability that an uninformed node becomes informed in a round starting with $k$ informed nodes is $p_k = p\frac kn$, hence the protocol satisfies the exponential growth conditions in $[1,n[$ with $\gamma_n = p$.
	
	Similarly, the probability that an uninformed node remains uninformed in a round starting with $u$ uninformed nodes is \[1 - p_{n-u} = 1 - p\tfrac{n-u}{n} = 1 - p + p \tfrac un = \exp(-\ln \tfrac 1 {1-p}) + p \tfrac un.\] Consequently, the protocol satisfies the exponential shrinking conditions with ${\rho_n} = \ln\tfrac1{1-p}$ for all $u \le gn$, $g$ any constant smaller than $1$.
	
	By Theorems~\ref{th:exp-growth-upper},~\ref{th:exp-growth-lower},~\ref{th:exp-shrinking-upper},~and~\ref{th:exp-shrinking-lower}, the expected rumor spreading time is $\log_{1+p} n + \tfrac1{\ln(1/(1-p))} \ln n \pm O(1)$.
\end{proof}

\begin{theorem}\label{th:example-push-n-pull-w-failures}
	The expected rumor spreading time for the push-pull protocol with success probability $p<1$ on the complete graph of size $n$ is equal to
	\begin{equation}
		\log_{2p+1}n + \tfrac1{p + \ln \frac{1}{1-p}} \ln n \pm O(1) \notag.
	\end{equation}
\end{theorem}
\begin{proof}
	Using the same arguments as for the push-pull protocol without failures, we observe that the covariances are at most zero, so all covariance conditions are satisfied.
	Consider an uninformed node in a round starting with $k$ informed nodes.
	The probability that this node does not inform itself via its pull call is $1-p\tfrac{k}{n}$.
	The probability that it is not successfully called by an informed node is $\left(1-\tfrac{p}{n}\right)^k$.
	Hence $p_k = 1 - \left(1-p\tfrac{k}{n}\right) \left(1-\tfrac{p}{n}\right)^k$ and Corollary~\ref{cor:prelim:(1-p/n)^k} gives
	\[
		2p\tfrac{k}{n} - \tfrac{3p^2k^2}{2n^2} \le p_k \le 2p\tfrac{k}{n}.
	\]
%	\begin{equation}
%		1 - \left(1-p\tfrac{k}{n}\right) \left(1-\tfrac{p}{n}\right)^k
%		= 2p\tfrac{k}{n} + p^2 O\left(\tfrac{k^2}{n^2}\right) \notag.
%	\end{equation}
	Thus the protocol satisfies the exponential growth conditions in $[1,\tfrac 23n[$ with $\gamma_n = 2p$.

    Likewise, the probability $1-p_{n-u}$ that an uninformed node stays uninformed in a round starting with $u$ uninformed nodes is equal to
    $\left(1-p\tfrac{n-u}{n}\right) \left(1-\tfrac{p}{n}\right)^{n-u}$.
    With Corollary~\ref{cor:prelim:(1-p/n)^(n-u)} we estimate
    \[
        (1-p)e^{-p} + pe^{-p}\cdot\tfrac{u}{n}
        \le 1-p_{n-u}
        \le (1-p)e^{-p} + 3pe^{-p}\cdot\tfrac{u}{n}.
    \]
%	\begin{equation}
%		\left(1-p\tfrac{n-u}{n}\right) \left(1-\tfrac{p}{n}\right)^{n-u}
%		= \exp\left(-p+\ln(1-p)\right) + O\left(\tfrac{u}{n}\right) \notag.
%	\end{equation}
%	Since $u < n/2$, for any $n$ big enough we have
%	\[
%		e^{-p}(1-p)\left(1-\tfrac{7p^2+4p^3u}{8n}\right)
%		\le 1-p_{n-u}
%		\le e^{-p}(1-p) + e^{-p}(3p+p^2)\tfrac{u}{n}.
%	\]
	Therefore, the protocol satisfies the exponential growth conditions with ${\rho_n} = p + \ln\tfrac1{1-p}$.
    Thus by Theorems~\ref{th:exp-growth-upper},~\ref{th:exp-growth-lower},~\ref{th:exp-shrinking-upper},~and~\ref{th:exp-shrinking-lower}, the expected spreading time is equal to
    $\log_{p+1}n + \tfrac1{p + \ln(1/(1-p))} \ln n \pm O(1)$.
%	\begin{equation}
%		\log_{p+1}n + \tfrac1{p + \ln(1/(1-p))} \ln n + O(1) \notag.
%	\end{equation}
\end{proof}

The fact that in the presence of transmission failures the double exponential shrinking regime ceases to exist has an important implication on the message complexity. In their seminal paper~\cite{KarpSSV00}, Karp et al.\ show that any address-oblivious rumor spreading algorithm that informs all nodes of the complete graph with at least constant probability needs $\Omega(n \log\log n)$ message transmissions in expectation (we refer to that paper for a discussion of the tricky question how to count messages in algorithms performing pull calls).

This optimal order of magnitude is attained by the push-pull protocol when nodes stop sending a rumor that is older than $\log_3 n + O(\log\log n)$ rounds. As Karp et al.\ remark, relying on such a time stamp is risky. A mild underestimate of the true rumor spreading time leaves a constant fraction of the nodes uninformed. A mild overestimate of the rumor spreading time by $\eps \log n$ rounds leads to the situation that for $\eps \log n$ rounds a constant fraction of the nodes knows and pushes the rumor, which implies a message complexity of $\Omega(n \log n)$. For this reason, Karp et al.\ propose the more complicated median-counter algorithms which is robust against a moderate number of adversarial node failures and against moderate deviations from the uniform choice of the nodes to contact.

Our above analysis of the push-pull protocol in the presences of transmission faults shows that not only an unexpected deviation from the ideal fault-free push-pull protocol leads to an increased message complexity, but even a perfectly anticipated faulty behavior. While we know the expected rumor spreading time very precisely (and we could with the same arguments also show a tail bound stating that our upper bound for the expectation is exceeded by $\lambda$ with probability $\exp(-\Omega(\lambda))$ only), the ``transmit until time limit reached'' approach still leads to a message complexity of $\Omega(n \log n)$ due to the missing double exponential shrinking phase. As our analysis shows, after an expected number of $\log_{2p+1} n$ iterations, a constant fraction of the nodes are informed. However, it takes another $\tfrac1{p + \ln \frac{1}{1-p}} \ln n + O(1)$ rounds in the exponential shrinking regime until all nodes are informed. Hence when using the simple ``transmit until time limit reached'' approach to limit the number of messages, the exponential shrinking regime alone would see $\Omega(n \log n)$ push calls by the $\Omega(n)$ informed nodes.

It is not clear how to overcome this difficulty. The median-counter algorithm of Karp et al.\ for constant-probability transmission failures also seems to require $\Omega(n \log n)$ messages (see the comment right before Theorem~3.1 in~\cite{KarpSSV00}).

\subsection{Multiple Calls}

In this section, we analyze rumor spreading protocols in which in each round each node when active calls a random number $R$ of nodes. This was proposed by~\cite{PanagiotouPS15} to model different data processing speeds of nodes. Unlike in~\cite{PanagiotouPS15}, we assume that each node in each round resamples the number of nodes it may call. This allows to model changing data processing speed as opposed to nodes having generally different speeds.

Consider a random integer variable $R$ taking values in $[0,n[$.
We say that a rumor spreading protocol is an $R$-protocol if in each round it respects the following call procedure.
Each node which can make calls in current round samples independently a new value $r$ from $R$.
Then it calls $r$ different neighbors chosen uniformly at random.

In this section we consider the $R$-push protocol and the $R$-push-pull protocol and prove the statements similar to Theorem~1.1, 1.2, and 1.3 from~\cite{PanagiotouPS15}.
Note that by putting $R\equiv1$, we obtain the classic push and push-pull protocols.
\begin{theorem}\label{th:multiple-push}
    Assume that $R$ is a distribution with $\E[R] = \Theta(1)$ and $\Var[R]=O(1)$.
    Then the expected spreading time for the $R$-push protocol on the complete graph of size $n$ is equal to
    \[
        \log_{1+\E[R]}n + \tfrac1{\E[R]}\ln n \pm O(1).
    \]
\end{theorem}
\begin{proof}
    Consider a round of the protocol started from $k$ informed nodes.
    Let $x_1$ and $x_2$ be two different uninformed nodes and let $X_1$ and $X_2$ be the indicator random variables for events that $x_1$ resp. $x_2$ become informed.
    Suppose that node $y$ is informed.
    The probability that $x_1$ and $x_2$ are both called by $y$ is at most
    \[
        \sum_{j\ge2} \Pr[R=j] \cdot \binom{j}{2} \cdot \tfrac1{n(n-1)}
        \le \tfrac1{n^2} \sum_{j\ge2} j^2 \cdot \Pr[R=j]
        \le (\Var[R] + \E[R]^2) \cdot \tfrac1{n^2}
        = O\left(\tfrac1{n^2}\right).
    \]
%    The last estimate follows from $\E[R] = O(1)$ and $\Var[R] = O(1)$.
    Since there are $k$ informed nodes, the probability that $x_1$, $x_2$ are both called by the same node (not necessary $y$) is $k\cdot O\left(\tfrac1{n^2}\right)$.
    In addition, if we condition on the event that $x_1$ and $x_2$ are not called by the same node, then the probability that they both get informed is slightly less than $p_k^2 = \Pr[X_1=1]^2$.
    Therefore, $\Cov[X_1,X_2] \le k\cdot O\left(\tfrac1{n^2}\right)$ for any $k < n$ which corresponds to the covariance condition for both exponential growth and exponential shrinking.

    Now let us study the probability $p_k$.
    Since the probability that $x$ does not belong to a random set of $j$ nodes is equal to
    $$\left(1-\tfrac1n\right)\left(1-\tfrac1{n-1}\right)\ldots\left(1-\tfrac1{n-j+1}\right) = \tfrac{n-j}{n},$$
    the probability that $y$ does not call $x$ is equal to $\sum_{j\ge0} \Pr[R=j] \cdot \tfrac{n-j}{n} = 1 - \tfrac{\E[R]}{n}$.
    Therefore the probability $p_k$ that $x$ gets informed in current round is equal to
    \begin{equation}
        1 - \left(1-\tfrac{\E[R]}{n}\right)^k. \label{eq:multiple-1}
    \end{equation}
    With Corollary~\ref{cor:prelim:(1-p/n)^k} we estimate
    \begin{equation}
        \E[R]\cdot\tfrac{k}{n} - \E[R]^2\cdot\tfrac{k^2}{2n^2}
        \le p_k
        \le \E[R]\cdot\tfrac{k}{n}, \label{eq:multiple-1}
    \end{equation}
    for any $k \le n/\E[R]$.
    Therefore, the protocol satisfies the exponential growth conditions in $[1,n/\E[R]]$ with $\gamma_n = \E[R]$.

    Similarly, the probability that an uninformed node stays uninformed in a round starting with $u:=n-k$ uninformed nodes, is $1-p_{n-u} = \left(1-\tfrac{\E[R]}{n}\right)^{n-u}$.
    By Corollary~\ref{cor:prelim:(1-p/n)^(n-u)}, for all $u \le n/\E[R]$ we estimate
    \begin{equation}
        e^{-\E[R]}
        \le 1-p_{n-u}
        \le e^{-\E[R]} \left(1+2\E[R]\tfrac{u}{n}\right). \label{eq:multiple-4}
    \end{equation}
    Therefore, the protocol satisfies the exponential shrinking conditions in $[n(1-1/\E[R]),n]$ with ${\rho_n} = \E[R]$.

    We note that the intervals for the exponential growth and shrinking regime does not intersect if $\E[R] > 2$.
    However, we still be able to bound the expected spreading time.
    From~\eqref{eq:multiple-1} it follows that $p_{n/\E[R]} = 1-\tfrac1e+o(1)$ and $p_{n(1-1/\E[R])} = 1 - e^{1-\E[R]} + o(1)$.
    Since $p_k$ increases, it is bounded uniformly for any $k \in \left[\tfrac{n}{\E[R]}, n-\tfrac{n}{\E[R]}\right]$.
    Hence, by Lemma~\ref{lem:general-connect}, we have
    $\E\left[T\left(\tfrac{\E[R]}{n},n-\tfrac{\E[R]}{n}\right)\right] = O(1)$.
    So by Theorems~\ref{th:exp-growth-upper}~and~\ref{th:exp-shrinking-upper}, the expected rumor spreading time is at most $\log_{1+\E[R]} n + \tfrac1{\E[R]}\log n \pm O(1)$.

    Similarly, by Lemma~\ref{lem:general-connect-lower}, there exists some $f' \in \left]1-\tfrac1{\E[R]},1\right[$ such that with probability $1-O\left(\tfrac1n\right)$ the number of informed nodes after some round will belong to $\left[n-\tfrac{n}{\E[R]},f'n\right]$.
    Then by Theorems~\ref{th:exp-growth-lower}~and~\ref{th:exp-shrinking-lower}, the expected rumor spreading time is at least $\log_{1+\E[R]} n + \tfrac1{\E[R]}\log n \pm O(1)$.
\end{proof}

\begin{theorem}\label{th:multiple-push-pull}
    Assume that $R$ is a distribution with $\E[R] = \Theta(1)$ and $\Var[R]=O(1)$.
    Let $\ell$ be the smallest nonnegative integer such that $\Pr[R=\ell] > 0$ and we suppose that $\Pr[R=\ell] = \Theta(1)$.
    Then the expected spreading time for the $R$-push-pull protocol on the complete graph of size $n$ is at most
    \begin{align*}
        & \log_{1+2\E[R]}n + \tfrac1{\E[R]-\ln\Pr[R=0]} \cdot \ln n \pm O(1), & \ell = 0; \\
        & \log_{1+2\E[R]}n + \log_{1+\ell}\ln n \pm O(1), & \ell > 0.
    \end{align*}
\end{theorem}
\begin{proof}
	As usual, we discuss the covariance condition first.
	Consider one round of the protocol started from $k$ informed nodes.
	Let $x_1$, $x_2$ be two different uninformed nodes.
	For $i = 1,2$, let $X_i$ be the indicator random variables for event that $x_i$ becomes informed in this round, $Y_i$ the indicator random variable for the event that $x_i$ is called by an informed node, and $Z_i$ the indicator random variable for event that $x_i$ calls an informed node.
	Since $Y_i$ coincides with $X_i$ for the push protocol from the proof of Theorem~\ref{th:multiple-push}, we have $\Cov[Y_1,Y_2] \le k \cdot O\left(\tfrac1{n^2}\right)$.
	In addition $Z_i$ are pairwise independent and also independent from $Y_i$.
	Since $X_i = \max\{Z_i,Y_i\}$ we have also $\Cov[X_1,X_2] \le k \cdot O\left(\tfrac1{n^2}\right)$ for any $k < n$.
	Therefore, the covariance condition is satisfied for exponential growth and both exponential and double exponential shrinking conditions.
	
	Let us study $\Pr[Z_1=0]$.
	If node $x_1$ calls $j$ different nodes in current round, then the probability that it does not hit informed node is $\left(1-\tfrac{k}{n}\right)\ldots\left(1-\tfrac{k}{n-j+1}\right)$.
	Summing over all possible values of $j$ we obtain the following.
	\begin{equation}
		\Pr[Z_1=0] = \sum_{j=0}^{n-k} \Pr[R=j] \cdot \left(1-\tfrac{k}{n}\right)\ldots\left(1-\tfrac{k}{n-j+1}\right). \label{eq:multiple-3}
	\end{equation}
	Recall that that $\sum_{j=0}^n j\cdot\Pr[R=j] = \E[R]$ and $\sum_{j=0}^n j^2\cdot\Pr[R=j] = \Var[R] + \E[R]^2 = O(1)$.
	Using estimate from Corollary~\ref{cor:prelim:(1-p/n)^k}, we compute for any $k\le\tfrac{n}{2}$
	\begin{align*}
		\Pr[Z_1=0]
		& \le \sum_{j=0}^{n-k} \Pr[R=j] \cdot \left(1-\tfrac{k}{n}\right)^j \\
		& \le \sum_{j=0}^{n/k}\Pr[R=j] \cdot \left(1-j\tfrac{k}{n}+j^2\tfrac{k^2}{2n^2}\right)
			+ \sum_{j=n/k+1}^{n-k} \Pr[R=j] \\
		& = \sum_{j=0}^{n/k}\Pr[R=j] - \tfrac{k}{n}\sum_{j=0}^{n/k}j\cdot\Pr[R=j]
			+ \tfrac{k^2}{2n^2}\sum_{j=0}^{n/k}j^2\cdot\Pr[R=j]
			+ \sum_{j=n/k-1}^{n-k}\Pr[R=j] \\
		& \le 1 - \tfrac{k}{n}\left(\E[R]-\sum_{j=n/k-1}^n j\cdot\Pr[R=j]\right)
			+ \tfrac{k^2}{n^2}\sum_{j=0}^{n-k}j^2\cdot\Pr[R=j] \\
		& \le 1 - \E[R] \cdot \tfrac{k}{n} + \tfrac{k^2}{n^2} \sum_{j=n/k-1}^n j^2\cdot\Pr[R=j]
			+ \tfrac{k^2}{n^2}\sum_{j=0}^{n-k}j^2\cdot\Pr[R=j] \\
		& \le 1 - \E[R] \cdot \tfrac{k}{n} + 2(\Var[R]+\E[R]^2) \cdot \tfrac{k^2}{n^2}.
	\end{align*}
	For any $k\le\tfrac{n}{2}$ we can similarly bound $\Pr[Z_i=0]$ from below using Bernoulli's inequality.
	\begin{align*}
		\Pr[Z_1=0]
		& \ge \sum_{j=0}^{n-k} \Pr[R=j]\left(1-k\cdot\tfrac{j}{n-j}\right) \\
		& \ge \sum_{j=0}^{n-k} \Pr[R=j]\left(1-\tfrac{jk}{n}\left(1+2\tfrac{j}{n}\right)\right) \\
		& = 1 - \E[R]\cdot \tfrac{k}{n} + O(1) \cdot \tfrac{k^2}{n^2}
	\end{align*}
	By \eqref{eq:multiple-1}, we estimate $\Pr[Y_1=0] = 1-\E[R]\cdot\tfrac{k}{n} \pm O(1) \cdot \tfrac{k^2}{n^2}$.
	Since $Y_1$ and $Z_1$ are independent, we have
	$$\Pr[X_1=1] = 1-\Pr[Y_1=0]\cdot\Pr[Z_1=0].$$
	Therefore, $p_k = 2\E[R]\cdot\tfrac{k}{n} \pm O(1) \cdot \tfrac{k^2}{n^2}$ for any $k \le \min\left\{\tfrac{n}{2}, \tfrac{n}{\E[R]}\right\}$.
	Hence the protocol satisfies the exponential growth conditions with $\gamma_n = 2\E[R]$ for any $k \le \min\left\{\tfrac{n}{2}, \tfrac{n}{\E[R]}\right\}$.
	
	Now we discuss the shrinking conditions.
	We consider a round started from $u:=n-k$ uninformed nodes.
	Similarly to~\eqref{eq:multiple-3}, we have
	\[
		\Pr[Z_1=0] = \sum_{j\ge0} \Pr[R=j] \cdot \tfrac{u}{n} \cdot \tfrac{u-1}{n-1} \cdot \ldots \cdot \tfrac{u-j+1}{n-j+1}.
	\]
	Assume first that $\Pr[R=0] > 0$, i.e., $\ell=0$.
	Since $x_1$ might not call in current round, there is at least a constant probability, that it stays uninformed.
	With~\eqref{eq:multiple-4} and estimate
	$$\Pr[R=0] \le \Pr[Z_1=0] \le \Pr[R=0] + \Pr[R\ge1] \cdot \tfrac{u}{n},$$
	we see that $\Pr[X_1=0] = \Pr[R=0]\cdot e^{-\E[R]} \pm O(1)\cdot\tfrac{u}{n}$ for any $u \le \min\left\{\tfrac{n}{2}, \tfrac{n}{\E[R]}\right\}$.
	In this case the protocol satisfies the exponential shrinking conditions with ${\rho_n} = \E[R] - \ln\Pr[R=0]$.
	Applying Lemma~\ref{lem:general-connect}~and~\ref{lem:general-connect-lower} in the similar way as in the proof of Theorem~\ref{th:multiple-push}, one can see that by Theorems~\ref{th:exp-growth-upper},~\ref{th:exp-growth-lower},~\ref{th:exp-shrinking-upper},~and~\ref{th:exp-shrinking-lower}, the expected rumor spreading time is $\log_{1+2\E[R]}n + \tfrac1{\E[R]-\ln\Pr[R=0]}\ln n \pm O(1)$.
	
	Finally, suppose that $\Pr[R=0] = 0$, and let $\ell$ be the smallest integer such that $\Pr[R=\ell] > 0$.
	In this case we can easily estimate the probability that $x_1$ stays uninformed.
	From below we have
	\[
		\Pr[X_1=0] \ge \Pr[Y_1=0] \cdot \Pr[R=\ell] \cdot \tfrac{u^\ell}{n^\ell}
		\ge e^{-\E[R]} \cdot \Pr[R=\ell] \cdot \tfrac{u^\ell}{n^\ell}.
	\]
	From above, $\Pr[X_1=0] \le \Pr[Z_1=0] \le \tfrac{u^\ell}{n^\ell}$.
	Hence the protocol satisfies the double exponential shrinking conditions with parameter $1+\ell$.
	Again, by Theorems~\ref{th:exp-growth-upper},~\ref{th:exp-growth-lower},~\ref{th:double-exp-shrinking-upper},~and~\ref{th:double-exp-shrinking-lower} and Lemmas~\ref{lem:general-connect}~and~\ref{lem:general-connect-lower}, the expected rumor spreading time is $\log_{1+2\E[R]}n + \log_{1+\ell}\ln n \pm O(1)$.
\end{proof}


% Random 2-Regular Graphs. Nice stuff, don't delete!!!

%On the other hand, it is quite obvious that the $G(n,p)$ model can be analyzed with our methods. To show that our methods also allow the analysis of dynamic graph models with more dependencies, we regard in this section the model where in each round independently the network $G_t$ is chosen as a random $2$-regular (simple) graph. Regular random graphs are notorious for the inherent dependencies which already make sampling them highly non-trivial. For this reason, it is quite clear that the classic rumor spreading analysis approach of trying to understand the distribution of the number of newly informed nodes will be tedious. For our method, however, we only need to take the local view of understanding how likely it is that a new node becomes informed. For the covariance condition, while we believe it to be true, we do not need to show that the events of two uninformed nodes becoming informed are negatively correlated (or independent). Since the covariance conditions allow a mild positive covariance, we may conveniently ignore rare events like the two nodes sharing a neighbor and may thus assume that the calls affecting the two nodes are disjoint and thus independent.
%
%Nevertheless, it turns out that the rumor spreading process in this type of dynamic graphs is different from the one in complete graphs. This is visible from the rumor spreading times proven below, but also from the observation that even in the pull protocol the events that two uniformed nodes become informed are not independent.
%
%In this subsection, due to the small node degrees, we skip the assumption that nodes may also call themselves, but assume that contacts are chosen uniformly at random from all neighbors.
%
%\begin{theorem}
%  Consider a dynamic graph setting where the network in each round $t$ independently is a $2$-regular random (simple) graph $G_t$. Then the rumor spreading times $T$ of the three classic protocols are as follow.
%  \begin{itemize}
%	  \item Push protocol: $\E[T] = \log_2 n + \log_4 n \pm O(1)$.
%	  \item Pull protocol: $\E[T] = \log_2 n + \log_2 \ln n \pm O(1)$.
%	  \item Push-pull protocol: $\E[T] = \log_{5/2} n + \log_2 \ln n \pm O(1)$.
%  \end{itemize}
%\end{theorem}
%
%\begin{proof}
%  We defer the proof of the covariance conditions to the very end. Consider a round $t$ starting with $k$ informed nodes. Consider a fixed uninformed node $x$. Let $A_i$, $i = 0,1,2$, be the event that $i$ of its $2$ neighbors are informed. Since in a $2$-regular random graph the two neighbors of $x$ form a random $2$-set of the nodes different from $x$, we easily compute
%\begin{align*}
%	\Pr[A_0] &= \tfrac{n-1-k}{n-1} \tfrac{n-2-k}{n-2},\\
%	\Pr[A_1] &= 2 \tfrac{n-1-k}{n-1} \tfrac{k}{n-2},\\
%	\Pr[A_2] &= \tfrac{k}{n-1} \tfrac{k-1}{n-2}\,.	
%\end{align*}
%  For the push protocol, we compute $p_k = \frac 12 \Pr[A_1] + \frac 34 \Pr[A_2] = \frac{k}{n-1}(1 - \frac14 \frac{k-1}{n-2})$. Assuming $n$ to be sufficiently large, the exponential growth condition (apart from the covariance condition) is satisfied in the whole range $k \in  [1,n[$ with $\gamma_n=1$. Rewriting the expression for $p_k$, we see that the probability to remain uninformed in a round starting with $u$ uninformed nodes, is $1-p_{n-u} = \frac 14 + \frac 12 \frac{u-1}{n-2} + \frac 14 \frac{(u-1)(u-4)}{(n-1)(n-2)}$. Hence for, say $u \le n/2$, the exponential shrinking conditions are satisfied with ${\rho_n} = \ln 4$. Apart from the covariance conditions, this shows our claim for the push protocol.
%
%  For the pull protocol, we have $p_k = \frac 12 \Pr[A_1] + \Pr[A_2] = \frac{k}{n-1}$ and consequently $1 - p_{n-u} = \frac{u-1}{n-1}$, showing the exponential growth and double exponential shrinking conditions to be satisfied in overlapping ranges with $\gamma_n = 1$ and $\ell = 1$.
%
%  For the push-pull protocol, we have $p_k = \frac 34 \Pr[A_1] + \Pr[A_2] = \frac 32 \frac{k}{n-1}(1 - \frac 23 \frac{k-1}{n-2})$ and consequently $1 - p_{n-u} = \frac{u-1}{n-1} + \frac 12 \frac{u+1}{n-1} - \frac 12 \frac{(u+1)(u-1)}{(n-1)(n-2)}$, showing the exponential growth and double exponential shrinking conditions to be satisfied in overlapping ranges with $\gamma_n = \frac 32$ and $\ell = 1$.
%
%  It remains to show the covariance conditions. Note first that the covariance condition of the exponential growth conditions implies the other covariance conditions, so it suffices to show the former.
%
%  Let $x_1, x_2$ be two uninformed nodes and let $X_1,X_2$ be the indicator random variables for the events of becoming informed in the current round. We have $\Cov[X_1,X_2] = \Pr[X_2](\Pr[X_1 \mid X_2] - \Pr[X_1]) = O(\frac kn)(\Pr[X_1 \mid X_2] - \Pr[X_1])$, so it suffices to show $\Pr[X_1 \mid X_2] - \Pr[X_1] \le c/n$ for some constant $c$.
%
%  Let $B$ be the event that $x_1$ and $x_2$ have distance at least $3$ in $G_t$. Since $x_2$ has at most $4$ other nodes in distance $2$ or closer, $\Pr[B] \ge 1 - \frac 4 {n-1}$. By $\Pr[X_1 \mid X_2] - \Pr[X_1] \le \Pr[\neg B] + \Pr[X_1 \mid X_2 \AND B] - \Pr[X_1 \mid B]$, we only need to consider the case that $B$ holds. In this case, the targets of the calls of $x_1$ and its neighbors are independent of the event~$X_2$. Consequently, the only correlation among $X_2=1$ and $X_1=1$ stems from  the fact that $X_2=1$ has an influence on where the informed nodes are in $G_t$. More formally, denoting by $A_i^j$ the event that $x_j$ has exactly $i$ of its two neighbors informed, we have that $X_2=1$ has an influence on the distribution of $(A^2_i)_i$ which in turn has an influence on the distribution $(A^1_i)_i$. However, regardless which event $A^2_{i_2}$ we condition on, the probability distribution of $(A^1_{i_1})_{i_1}$ is only mildly affected. The precise expression for $\Pr[A^1_{i_1} \mid A^2_{i_2}]$ can be obtained from the one for $\Pr[A_{i_1}]$ above by replacing $n$ by $n-2$ and $k$ by $k-i_2$. Consequently, $\Pr[A^1_{i_1} \mid A^2_{i_2}] = \Pr[A_{i_1}] \pm O(1/n)$ for all $i_1,i_2 \in \{0,1,2\}$. Therefore $\Pr[X_1 \mid X_2 \AND B] = \Pr[X_1 \mid B] + O(1/n)$ and hence $\Pr[X_1 \mid X_2] - \Pr[X_1] = O(1/n)$ as aimed at. This shows the covariance condition of the exponential growth conditions in the whole range $k \in [1,n[$, and thus also the other two covariance conditions.
%\end{proof}

\subsection{Dynamic Graphs}\label{sec:dynamic graphs}

We now show that our method can also be applied to certain dynamic graph settings, that is, when the network structure may be different in each round. While it is generally agreed upon that dynamic problem settings are highly relevant for practical applications, it is still not so clear what is a good theoretical model for dynamicity. For rumor spreading problems, the only work regarding dynamic graphs~\cite{ClementiCDFPS16} considers the two models (i) that in each round independently the network is a $G(n,p)$ random graph and (ii)~that each possible edge has its own independent two-state Markov chain describing how it changes between being present and not (edge-Markovian dynamic graphs). For both models, it is proven that the push protocol informs all nodes in logarithmic time with high probability (when the parameters are chosen reasonably).

It is clear that the edge-Markovian model due to the time-dependence cannot be analyzed with our methods. For the other result, we now show that our method quite easily gives a very precise analysis. We only treat the case of $\Theta(1/n)$ edge probabilities, as this seems to be the most interesting one (the graph is not connected, but has nodes with degrees varying between $0$ and $\Theta(\log(n)/\log\log(n))$; when $p \ge (1+\eps)/n$, a giant component encompassing a linear number of nodes exists). 

To make the model precise, we assume that in each round independently, before the communication starts, the communication graph is sampled as $G(n,p)$ random graph, where $p = a/n$ for some positive constant $a$. That is, between any two nodes there is an edge, independently, with probability $a/n$. In the communication part of the round, each informed node chooses a communication partner uniformly at random from its neighbors in the communication graph and sends a copy of the rumor to it. Isolated informed nodes, naturally, do not communicate in this round.

%
%
%In this section we consider the rumor spreading on the dynamic Erd\H{o}s-R\'enyi Graph $G(n,p)$ where $p = \tfrac{a}{n}$ for some constant $a$.
%Before each round we resample the graph and though we ensure the property of the symmetry which is critical in our analysis.
%The choice of $p = \tfrac an$ is caused by the fact that for each node there is a finite probability to have no neighbors in current round, so the behavior of the protocol is significantly different from the classic rumor spreading on the complete graph.

%\subsubsection*{Push Protocol}
%
%\begin{lemma}\label{lem:Erdos-Renyi-call-prob}
%    Consider one round of Erdos-Renyi push protocol.
%    Let $E$ be the set of edges of the communication graph.
%    Then we have
%    \[
%        \Pr[y \to x | xy \in E] = \tfrac{1-e^{-a}}{a} + O\left(\tfrac1n\right).
%    \]
%\end{lemma}
%\begin{proof}
%    The probability that $y$ calls $x$ is equal to $\tfrac1{\deg y}$.
%    Since we know that $xy \in E$, then $\deg y - 1$ has the binomial distribution with parameters $n-1$ and $p=\tfrac{a}{n}$.
%    Therefore,
%    \begin{align*}
%        \Pr[y \to x | xy \in E]
%        & = \sum_{i=0}^{n-1} \tfrac1{i+1} \cdot \Pr[\deg y=i+1|xy \in E] \\
%        & = \sum_{i=0}^{n-1} \tfrac1{i+1} \cdot {n-1\choose i} \cdot \left(\tfrac{a}{n}\right)^i \cdot \left(1-\tfrac{a}{n}\right)^{n-i-1} \\
%        & = \tfrac1a \sum_{i=0}^{n-1} \tfrac{a^{i+1}}{(i+1)!} \cdot \left(1-\tfrac1n\right) \cdot \ldots \left(1-\tfrac{i}{n}\right) \\
%        & = \tfrac{1-e^{-a}}{a} + O\left(\tfrac1n\right).
%    \end{align*}
%\end{proof}
%
%\begin{observation}\label{obs:Erdos-Renyi-call-prob-w/o-triangles}
%	For any set of nodes $\{x_1, \ldots, x_\ell\}$ not containing $x$ we have
%    \[
%        \Pr[y \to x | xy \in E, \{x_1y, \ldots, x_\ell y\} \cap E = \emptyset]
%        \ge \Pr[y \to x | xy \in E].
%    \]
%\end{observation}
%
%\begin{lemma}\label{lem:Erdos-Renyi-push-pk}
%    Consider one round of Erdos-Renyi push protocol started with $k$ informed nodes.
%    The probability that an uninformed node $x$ is called (i.e., informed) is equal to
%    $\tfrac{k}{n} \cdot \left(1-e^{-a} + O\left(\tfrac1n\right)\right).$
%\end{lemma}
%%\begin{proof}[this proof contains a mistake \ldots]
%%    By union bound, the probability that node $x$ is called is at most
%%    \[
%%        k \cdot \Pr[y \to x | xy \in E] \cdot \tfrac1n = \tfrac{k}{n} \cdot \left(1-e^{-a}+ O\left(\tfrac1n\right)\right).
%%    \]
%%
%%	Suppose that $x$ does not belong to any cycle of length 3, i.e., all neighbors of $x$ do not have common edges.
%%	Consequently the degrees of all neighbors of $x$ are independent.
%%    Observe that for any set of nodes $\{x_1, \ldots, x_\ell\}$ not containing $x$ we have
%%    \[
%%        \Pr[y \to x | xy \in E, \{x_1y, \ldots, x_\ell y\} \cap E = \emptyset]
%%        \ge \Pr[y \to x | xy \in E].
%%    \]
%%    Since the probability that there is no cycle of length 3 containing $x$, we have
%%    Let us consider all edges in $E$ which are adjacent to $x$ or to neighbors of $x$.
%%    Such subgraph is a tree if
%%    By union bound we can show that the corresponding probability is at least $1-O\left(\tfrac1n\right)$.
%%    Let nodes $y_1, \ldots, y_k$ are informed.
%%    Therefore,
%%    \begin{align*}
%%    	\Pr[X=1]
%%    	& = \E[X] = \E[X|x \notin \text{cycles of len. 3}] \cdot \left(1-O\left(\tfrac1n\right)\right) \\
%%    	& = \left(1-O\left(\tfrac1n\right)\right)
%%    		\cdot \E\left[1-\left(1-\Pr[y\to x|xy \in E, x \notin \text{cycles of len. 3}]\right)^{\deg x}\right] \\
%%    	& \ge \left(1-O\left(\tfrac1n\right)\right)
%%    		\cdot \E\left[1-\left(1-\Pr[y\to x|xy \in E]\right)^{\deg x}\right].
%%    \end{align*}
%%    Clearly, $\deg x = X_1 + \ldots + X_k$, where $X_i$ is the random indicator variable for an event ``$xy_i \in E$''.
%%    Therefore, the probability that node $x$ is called is at least
%%    \begin{align*}
%%    	\left(1-O\left(\tfrac1n\right)\right)
%%    		\cdot \left(1 - \prod_{i=1}^k\E\left[\left(1-\Pr[y\to x|xy \in \E]\right)^{X_i}\right]\right).
%%    \end{align*}
%%    Since by Lemma~\ref{lem:Erdos-Renyi-call-prob} we have $\E\left[\left(1-\Pr[y\to x|xy \in E]\right)^{X_i}\right]
%%    	= 1 - \tfrac{1-e^{-a}}{n} + O\left(\tfrac1{n^2}\right)$, we compute
%%    \begin{align*}
%%    	\Pr[X=1]
%%    	& \ge \left(1-O\left(\tfrac1n\right)\right)
%%    		\cdot \left(1 - \left(1-\tfrac{1-e^{-a}}{n}+O\left(\tfrac1{n^2}\right)\right)^k\right) \\
%%    	& = \left(1-O\left(\tfrac1n\right)\right)
%%    		\cdot \left( 1 - \left(1-\tfrac{k}{n}\left(1-e^{-a}\right)\right)
%%    			+ O\left(\tfrac{k}{n^2}\right)\right) \\
%%    	& = \tfrac{k}{n} \cdot \left(1-e^{-a} + O\left(\tfrac1n\right)\right).
%%    \end{align*}
%%\end{proof}
%
%\begin{proof}
%    By the union bound and Lemma~\ref{lem:Erdos-Renyi-call-prob} we have
%    \[
%    	\Pr[X=1] \le k \Pr[xy \in E] \cdot \Pr[x \leftarrow y | xy \in E]
%    	\le \tfrac{k}{n}\left(1-e^{-a}\right) + k\cdot O\left(\tfrac1{n^2}\right).
%    \]
%    
%    Although the degrees of nodes in the Erdos-Renyi graph are not independent, we can consider the informed neighbors of $x$ independently if we suppose that there is no edges between all such neighbors.
%    Formally, let event $A$ be ``there is no cycle of length 3 in graph $G$ formed by $x$ and 2 informed nodes''.
%    Clearly, $\Pr[A] \ge 1 - k^2 \cdot \tfrac{a^3}{n^3}$.
%    Therefore,
%    \begin{equation}
%    	\Pr[X=1] = 1 - \Pr[X=0]
%    	\ge 1 - \Pr[\NOT A] - \Pr[X=0 \AND A]
%    	\ge 1 - k^2 \cdot \tfrac{a^3}{n^3} - \Pr[X=0 \AND A]. \label{eq:Erdos-Renyi-eq1}
%    \end{equation}
%    Suppose that $\deg_{inf} x = \ell$.
%    If we condition on $A$, informed neighbors $\{y_1, \ldots, y_\ell\}$ of $x$ have no common edges.
%    Thus the events $\Pr[x \leftarrow y_i | xy_i \in E; A]$ for $i = 1,\ldots,\ell$ are independent.
%    Using Observation~\ref{obs:Erdos-Renyi-call-prob-w/o-triangles}, we estimate
%    \begin{align}
%    	\Pr[X=0 \AND A]
%    	& \le \sum_{\ell=0}^k \Pr[\deg_{inf} x = \ell]
%    		\cdot \Pr[X=0 | \deg_{inf} x = \ell \AND A] \notag\\
%    	& \le \sum_{\ell=0}^k \tbinom{k}{\ell} \left(\tfrac{a}{n}\right)^\ell \left(1-\tfrac{a}{n}\right)^{k-\ell} \cdot (1-p_0)^\ell \notag\\
%    	& = \left(1-\tfrac{a}{n}p_0\right)^{k}
%    		= \left(1-\tfrac1{n}\left(1-e^{-a}\right)+O\left(\tfrac1{n^2}\right)\right)^k.
%    		\label{eq:Erdos-Renyi-eq2}
%    \end{align}
%    %Therefore, $\Pr[X=1] \ge 1 - k^2\tfrac{a^3}{n^3} - \left(1-\tfrac1{n}\left(1-e^{-a}\right)+O\left(\tfrac1{n^2}\right)\right)^k$.
%    Substituting in~\eqref{eq:Erdos-Renyi-eq1} $\Pr[X=0 \AND A]$ by its bound from~\eqref{eq:Erdos-Renyi-eq2}, it is easy to see that there exists $f \in ]0,1[$ such that for any $k < fn$ we have $\Pr[X=1] \ge \tfrac{k}{n}\left(1-e^{-a}\right) + k\cdot O\left(\tfrac1{n^2}\right)$.
%\end{proof}
%
%%\begin{observation}
%%    $\sum_{\ell=0}^k {k\choose\ell} (pq)^\ell(1-p)^{k-\ell} = (1-p(1-q))^k$.
%%\end{observation}
%
%\subsection*{Covariance (briefly)}
%Suppose nodes $x$ and $y$ are uninformed.
%Denote by $X$ (resp. $Y$) the random indicator variables for the events $x$ (resp. $y$) gets informed in current round.
%To ease the notation, we will denote the corresponding positive events by the same capital letters.
%In addition we introduce the event $close$: $\dist(x,y) \le 3$ and the complementary event $far$: $\dist(x,y) \ge 4$.
%
%\begin{lemma*}
%	$\Cov[X,Y] \le \Pr[close] \cdot \Pr[X] = \Pr[close] \cdot \Pr[Y]$.
%\end{lemma*}
%\begin{proof}
%	Since $X$ and $Y$ are random indicator variables, we have
%	$$\Cov[X,Y] \le \Pr[X \AND Y] - \Pr[X]\cdot\Pr[Y].$$
%	Then we cut the probability space according to events $far/close$ as follows.
%	$$\Pr[X \AND Y] = \Pr[close]\cdot\Pr[X \AND Y|close] + \Pr[far]\cdot\Pr[X \AND Y|far].$$
%	First, we note that $\Pr[X \AND Y|close] \le \Pr[X|close] \le \Pr[X]$.
%	The last inequality follows from the observation that the event $close$ privileges some potentially successful calls to go to $Y$ instead of $Y$.
%	
%	Second, one can see that $\Pr[far] \le 1$ and $\Pr[X \AND Y|far] \le \Pr[X]\cdot\Pr[Y]$, since 
%\end{proof}



%\subsubsection{Push Protocol}
%\subsubsection{Push Protocol}
%\subsubsection{Push Protocol}
%\begin{itemize}
%	\item we define $p_0 := \Pr[x \leftarrow y | xy \in E]$
%	\item Lem. $p_0 = \tfrac{1-e^{-a}}{a} + O\left(\tfrac1n\right)$
%	\item Obs. $p_0 \le \Pr[x \leftarrow y | xy \in E, \{x_1y, \ldots, x_\ell y\} \cap E = \emptyset]$
%	\item Th. $\Pr[int \to x] \sim \left(1-\tfrac{a}{n}p_0\right)^k$ ($k$ nodes informed)
%	\item Cor. exp shrinking
%	\item Cor. exp growth
%	\item $T = \ldots$
%\end{itemize}


We introduce the following notation. We consider one round and aim at showing the exponential growth and shrinking conditions. Let $E$ be the set of edges of the communication graph $G(n,\tfrac an)$ of this round. We write $xy \in E$ as shorthand for $\{x,y\} \in E$. We write $x \to y$ to denote the event that $x$ calls $y$. By $\deg_{\inf} x$ we denote the number of informed neighbors of $x$.


%In addition, we write $inf \to x$ if $x$ is called by at least one informed node.
%Similarly, $x \to inf$ means that $x$ calls an informed node.

%Consider uninformed node $x$ and informed node $y$.
%Suppose that $xy \in E$.
%Let us study the probablity $p_0 := \Pr[y \to x \mid xy \in E]$.
%The following observation follows from the definition of $p_0$.

%\begin{lemma}\label{lem:Erdos-Renyi-call-prob}
%    \[
%        p_0 = \tfrac{1-e^{-a}}{a} + O\left(\tfrac1n\right).
%    \]
%\end{lemma}
%\begin{proof}
%    The probability that $y$ calls $x$ is equal to $\tfrac1{\deg y}$.
%    Since we know that $xy \in E$, then $\deg y - 1$ has the binomial distribution with parameters $n-1$ and $p=\tfrac{a}{n}$.
%    Therefore,
%    \begin{align*}
%        \Pr[y \to x | xy \in E]
%        & = \sum_{i=0}^{n-1} \tfrac1{i+1} \cdot \Pr[\deg y=i+1|xy \in E] \\
%        & = \sum_{i=0}^{n-1} \tfrac1{i+1} \cdot {n-1\choose i} \cdot \left(\tfrac{a}{n}\right)^i \cdot \left(1-\tfrac{a}{n}\right)^{n-i-1} \\
%        & = \tfrac1a \sum_{i=0}^{n-1} \tfrac{a^{i+1}}{(i+1)!} \cdot \left(1-\tfrac1n\right) \cdot \ldots \left(1-\tfrac{i}{n}\right) \\
%        & = \tfrac{1-e^{-a}}{a} + O\left(\tfrac1n\right).
%    \end{align*}
%\end{proof}
%\begin{observation}\label{obs:Erdos-Renyi-call-prob-w/o-triangles}
%	For any set of nodes $\{x_1, \ldots, x_\ell\}$ not containing $x$ we have
%    \[
%        \Pr[y \to x | xy \in E, \{x_1y, \ldots, x_\ell y\} \cap E = \emptyset]
%        \ge \Pr[y \to x | xy \in E].
%    \]
%\end{observation}
%\merk{this lemma replaces previous one\ldots}
\begin{lemma}\label{lemma: Erdos-Renyi - call probability}
	Consider an uninformed node $x$ and an informed node $y$.
	Let $\ell \le n/2$ and let $A_\ell$ be the event that $\{y_1y, \ldots, y_\ell y\} \cap E = \emptyset$.
	Then
	$$\Pr[y \to x \mid xy \in E \AND A_\ell] = \tfrac{1-e^{-a}}{a} + (\ell+1) \cdot O\left(\tfrac1n\right).$$
\end{lemma}
\begin{proof}
    Assume that $xy \in E$. Then the number of other neighbors of $y$, that is,  the random variable $\deg y - 1$, has a binomial distribution with parameters $n-2-\ell$ and $\tfrac{a}{n}$.
    The probability that $y$ calls $x$ is equal to $\tfrac1{\deg y}$.
	Using the fact that $\binom{m+1}{k+1} = \tfrac{k+1}{m+1} \binom{m}{k}$, we compute
	\begin{align*}
		\Pr[y \to x & \mid xy \in E \AND A_\ell]
		= \sum_{i=0}^{n-2-\ell} \tfrac1{i+1} \binom{n-2-\ell}{i} \left(\tfrac an\right)^i \left(1-\tfrac an\right)^{n-2-\ell-i} \\
		& = \tfrac na \cdot \tfrac1{n-2-\ell+1}
			\cdot \sum_{i=0}^{n-2-\ell} \binom{n-2-\ell+1}{i+1} \left(\tfrac an\right)^{i+1} \left(1-\tfrac an\right)^{n-2-\ell+1-(i+1)} \\
		& = \tfrac1a \cdot \left(1-\tfrac{\ell+1}{n-\ell-1}\right)
			\cdot \left(1 - \Pr[\Bin(n-2-\ell+1,\tfrac an)=0]\right) \\
		& = \tfrac1a \cdot \left(1-\tfrac{\ell+1}{n-\ell-1}\right)
			\cdot \left(1 - \left(1-\tfrac an\right)^{n-\ell-1}\right) \\
		& = \tfrac{1-e^{-a}}{a} + (\ell+1) \cdot O\left(\tfrac 1n\right),
	\end{align*}
	where above we denoted by $\Bin(m,p)$ a random variable having a binomial distribution with parameters $m$ and $p$.
\end{proof}


\begin{lemma} \label{lemma: Erdos-Renyi - growth and shrinking}
	Consider one round starting with $k<n$ informed nodes.
	The probability $1-p_k$ that an uninformed node $x$ stays uninformed in this round is at most
	$(1 - \tfrac{1-e^{-a}}{n})^k + k \cdot O(\tfrac1{n^2})$.
\end{lemma}

\begin{proof}
	Let $A$ be the event that $G\left(n,\tfrac an\right)$ contains no triangle formed by $x$ and two other informed nodes.
	By the first moment method, $\Pr[A] \ge 1 - k^2\cdot\tfrac{a^3}{n^3}$. Let $X$ be the indicator random variable for the event that $x$ is called by an informed node. Then
	\begin{align*}
		\Pr[X=0] \le \Pr[\NOT A] + \Pr[X=0 \AND A] \le k^2\tfrac{a^3}{n^3} + \Pr[X=0 \AND A].
	\end{align*}
	We compute $\Pr[X=0 \AND A]$ by conditioning on $\deg_{\inf} x$, which has a binomial distribution with parameters $k$ and $\tfrac an$.
	In addition, we observe that the conditioning on $A$ makes the actions of the informed neighbors of $x$ independent (in the probability space composed of the random actions of the nodes and the not yet determined random edges). Hence
	\[
		\Pr[X=0 \mid \deg_{\inf}x = \ell \AND A]
		= \left(1-\Pr[y \to x \mid xy \in E \AND A_{\ell-1}]\right)^\ell
		\le \left(1-\tfrac{1-e^{-a}}{a} + O\left(\tfrac1n\right)\right)^\ell
	\]
	by Lemma~\ref{lemma: Erdos-Renyi - call probability}.
%	The last inequality follows from the fact that in the construction of $A_{\ell-1}$, $\{y_1, \ldots, y_{\ell-1}\}$ are the other neighbors of $x$.
%	event $A \AND \deg_{\inf} x = \ell$ is nothing but $A_\ell$ introduced in Lemma~\ref{lemma: Erdos-Renyi - call probability}.
%	Then $\Pr[X=0 \mid A_l] \le \left(\Pr[y \to x \mid A_\ell]\right)^k$
%	Then, using Lemma~\ref{lemma: Erdos-Renyi - call probability}, w
We compute.
	\begin{align*}
		\Pr[X=0 \AND A]
		& = \sum_{\ell=0}^k \Pr[\deg_{inf} x = \ell] \cdot \Pr[A \mid \deg_{inf} x = \ell] \cdot \Pr[X=0 \mid \deg_{\inf}x = \ell \AND A] \\
		& \le \sum_{\ell=0}^k \binom kl \left(\tfrac an\right)^\ell \left(1-\tfrac an\right)^{k-\ell}
			\cdot 1 %\left(1-\tfrac an\right)^{\ell^2}
			\cdot \left(1-\tfrac{1-e^{-a}}{a} + O\left(\tfrac1n\right)\right)^\ell \\
		& \le \left[\tfrac an\left(1-\tfrac{1-e^{-a}}{a} + O\left(\tfrac1n\right)\right)
			+ 1 - \tfrac an\right]^k \\
		& = \left(1 - \tfrac{1-e^{-a}}{n}\right)^k + k \cdot O\left(\tfrac1{n^2}\right).
	\end{align*}
%	$\deg_{\inf} x$ which has a binomial distribution with parameters $k$ and $\tfrac an$.
%	
%	\merk{---------------------}
%	
%	Clearly, $\Pr[X=0] \ge \Pr[X=0 \AND A]$.
%	Therefore,
%	\begin{align*}
%		\Pr[x=0 \AND A] = \sum_{\ell=0}^k \Pr[\deg_{inf} x = \ell] \cdot \Pr[A | \deg_{inf} x = \ell] \cdot \Pr[X=0 | \deg_{inf} x = \ell \AND A] \\
%		\ge \sum_{\ell=0}^k \binom kl \left(\tfrac an\right)^\ell \left(1-\tfrac an\right)^{k-\ell} \cdot \left(1-\tfrac an\right)^{\ell^2} \left(1-p(\ell)\right)^\ell \\
%		= \sum_{\ell=0}^k \binom kl \left(\tfrac an\right)^\ell \left(1-\tfrac an\right)^{k-\ell} \cdot \left(1-\tfrac an\right)^{\ell^2} \left(1-p_0\right)^\ell \left(1 + \ell^2O\left(\tfrac1n\right)\right).
%	\end{align*}
\end{proof}

\begin{lemma} \label{lemma: Erdos-Renyi - growth}
	Consider one round starting with $k < n$ informed nodes. The probability $p_k$ that an uninformed node $x$ becomes informed in the current round is at most
	$\tfrac kn \cdot \left(1-e^{-a} + O\left(\tfrac1n\right)\right)$.
\end{lemma}
\begin{proof}
	Consider an uninformed node $x$ and an informed node $y$.
	Applying Lemma~\ref{lemma: Erdos-Renyi - call probability} with $\ell = 0$, we compute
	\[
		\Pr[y \to x] = \Pr[xy \in E] \cdot \Pr[y \to x \mid xy \in E]
		= \tfrac an \cdot \left(\tfrac{1-e^{-a}}{a} + O\left(\tfrac1n\right)\right).
	\]
	A union bound over the $k$ informed nodes proves the claim.
\end{proof}

\begin{lemma}\label{lemma: Erdos-Renyi - shrinking}
%	Let $g \in ]0,1[$.
	Consider one round starting with $k = \Omega(n)$ informed nodes.
	The probability $1-p_k$ that an uninformed node $x$ stays uninformed in current round is at least
	$\left(1 - \tfrac{1-e^{-a}}{n}\right)^k - O\left(\tfrac{\log^2 n}{n}\right)$.
\end{lemma}
\begin{proof}
	Let again $A$ denote the event that $G\left(n,\tfrac an\right)$ contains no cycle of length 3 formed by $x$ and two other informed nodes, and let $X$ be the indicator random variable for the event that $x$ becomes informed.
	Then $\Pr[X=0] \ge \Pr[X=0 \AND A]$.
	Similar to the proof of Lemma~\ref{lemma: Erdos-Renyi - growth and shrinking}, we compute $\Pr[X=0]$ by conditioning on the number $\deg_{\inf} x$ of its informed neighbors.
	\begin{align*}
		\Pr[X=0 \AND A]
		& = \sum_{\ell=0}^k \Pr[\deg_{\inf} x = \ell]
			\cdot \Pr[A \mid \deg_{\inf} x = \ell] \cdot \Pr[X=0 \mid \deg_{\inf}x =\ell \AND A] \\
		& = \sum_{\ell=0}^{k} \binom kl \left(\tfrac an\right)^\ell \left(1-\tfrac an\right)^{k-\ell}
			\cdot \left(1-\tfrac an\right)^{\ell^2}
			\cdot \left(1 - \tfrac{1-e^{-a}}{a} - (\ell+1) \cdot O\left(\tfrac 1n\right)\right)^\ell
	\end{align*}
	To simplify the notation, we denote
	$x_\ell := \binom kl \left(\tfrac an\right)^\ell \left(1-\tfrac an\right)^{k-\ell}$ and $q := 1 - \tfrac{1-e^{-a}}{a}$.
%	Since $k = \Theta(n)$, we bound $\sum_{\ell=0}^k$ by $\sum_{\ell=0}^{c\log n}$ for some constant $c > 0$.
	Then
		\begin{align*}
		\Pr[X=0 \AND A]
		& \ge \sum_{\ell=0}^{c\log n} x_\ell
			\cdot \left(1-\tfrac an\right)^{\ell^2}
			\cdot \left(q - \ell \cdot O\left(\tfrac 1n\right)\right)^\ell \\
		& \ge \sum_{\ell=0}^{c\log n} x_\ell \cdot \left(1-\tfrac an\right)^{c^2\log^2n} \left(q-O\left(\tfrac{\log n}{n}\right)\right)^\ell \\
%		& \ge \sum_{\ell=0}^{c\log n} \binom kl \left(\tfrac an\right)^\ell \left(1-\tfrac an\right)^{k-\ell} \cdot \left(1-a\tfrac{c^2\log^2n}{n}\right) (1-p_0)^\ell \left(1-\tfrac{c^2\log^2n}{n}\right) \\
		& \ge \left(1-O\left(\tfrac{\log^2n}{n}\right)\right)\sum_{\ell=0}^{c\log n} x_\ell q^\ell.
	\end{align*}
%	\begin{align*}
%		\Pr[X=0]
%		& \ge \sum_{\ell=0}^{c\log n} \binom kl \left(\tfrac an\right)^\ell \left(1-\tfrac an\right)^{k-\ell}
%			\cdot \left(1-\tfrac an\right)^{\ell^2}
%			\cdot \left(1 - \tfrac{1-e^{-a}}{a} - \ell \cdot O\left(\tfrac 1n\right)\right)^\ell \\
%		& \ge \sum_{\ell=0}^{c\log n} \binom kl \left(\tfrac an\right)^\ell \left(1-\tfrac an\right)^{k-\ell} \cdot \left(1-\tfrac an\right)^{c^2\log^2n} \left(1-p_0-\tfrac{c\log n}{n}\right)^\ell \\
%		& \ge \sum_{\ell=0}^{c\log n} \binom kl \left(\tfrac an\right)^\ell \left(1-\tfrac an\right)^{k-\ell} \cdot \left(1-a\tfrac{c^2\log^2n}{n}\right) (1-p_0)^\ell \left(1-\tfrac{c^2\log^2n}{n}\right) \\
%		& \ge \sum_{\ell=0}^{c\log n} \binom kl \left(\tfrac an\right)^\ell \left(1-\tfrac an\right)^{k-\ell} (1-p_0)^\ell \\
%		& \qquad + O\left(\tfrac{\log^2n}{n}\right) \sum_{\ell=0}^{c\log n} \binom kl \left(\tfrac an\right)^\ell \left(1-\tfrac an\right)^{k-\ell} (1-p_0)^\ell
%	\end{align*}
	By Lemma~\ref{lemma: log degree}, there exists $c > 0$ such that $\sum_{\ell=c\log n}^{k} x_\ell q^\ell \le \tfrac1n$.
%	\[
%		\sum_{\ell=c\log n}^{k} x_\ell q^\ell
%		\le \sum_{\ell=c\log n}^{k} \binom kl \left(\tfrac an\right)^\ell \left(1-\tfrac an\right)^{k-\ell} (1-p_0)^\ell \le \tfrac1n.
%	\]
	Since $\sum_{\ell=0}^{k} x_\ell q^\ell = \left(1-\tfrac{1-e^{-a}}{n}\right)^k$, we have 
	\begin{align*}
		\Pr[X=0 \AND A] \ge \left(1-O\left(\tfrac{\log^2n}{n}\right)\right) \left(1-\tfrac{1-e^{-a}}{n}\right)^k.
	\end{align*}
%	\[
%		\sum_{\ell=0}^{c\log n} \binom kl \left(\tfrac an\right)^\ell \left(1-\tfrac an\right)^{k-\ell} (1-p_0)^\ell \ge \sum_{\ell=0}^{k} \binom kl \left(\tfrac an\right)^\ell \left(1-\tfrac an\right)^{k-\ell} (1-p_0)^\ell - \tfrac1n.
%	\]
%	Using this bound we obtain that 
%	\begin{align*}
%		\Pr[X=0] \ge \left(1-\tfrac{1-e^{-a}}{n}\right)^k - O\left(\tfrac{\log^2 n}{n}\right).
%	\end{align*}
\end{proof}

\begin{lemma} \label{lemma: Erdos-Renyi - covariance}
	Consider a round starting with $k$ informed nodes. Let $x_1$ and $x_2$ be two uninformed nodes.
	Then the corresponding random indicator variables $X_1$ and $X_2$ for the events of these becoming informed are negatively correlated.
\end{lemma}
\begin{proof}
	By symmetry, we can assume that in this round we first generate the random communication graph, then we let each node choose a potential communication partner (uniformly among its neighbors), and then we decide randomly which $k$ nodes are informed, and finally those nodes which are informed actually call the potential partner chosen before. In this joint probability space, let $x_1$ and $x_2$ be two nodes. We condition in the following on (i) the outcome of the random graph, (ii) the outcome of the potential communication partners, and (iii) $x_1$ and $x_2$ being uninformed. In other words, all randomness is already decided except which set $I$ of $k$ nodes different from $x_1$ and $x_2$ is informed. 

	Let $S_1$ and $S_2$ be the sets of nodes having chosen $x_1$ and $x_2$ as potential partner. Now we have $X_1=1$ if and only if $S_1 \cap I \ne \emptyset$.	Similarly, $X_2 = 1$ is equivalent to $S_2 \cap I \ne \emptyset$.
	Since $S_1 \cap S_2 = \emptyset$ by construction, $X_1$ and $X_2$ are negatively correlated.
\end{proof}

\begin{theorem}
	The expected rumor spreading time is $\log_{2-e^{-a}}n + \tfrac1{1-e^{-a}}\ln n \pm O(1)$.
	In addition, there are constant $A' \alpha' > 0$ such that for any $r \in \N$ we have $\Pr[|T - \E[T]| \ge r] \le A' e^{-\alpha' r}$.
\end{theorem}
\begin{proof}
	By Lemma~\ref{lemma: Erdos-Renyi - covariance}, the covariance conditions are satisfied for both exponential growth and exponential shrinking.

	From Lemma~\ref{lemma: Erdos-Renyi - growth and shrinking} together with Corollary~\ref{cor:prelim:(1-p/n)^k} it follows that for any $k < n$ we have
	$$p_k \ge \tfrac kn \left(1-e^{-a}\right) - \tfrac{k^2}{2n^2}\left(1-e^{-a}\right)^2 - k\cdot O\left(\tfrac1{n^2}\right).$$
	Combining this with Lemma~\ref{lemma: Erdos-Renyi - growth}, we see that the process satisfies the exponential growth conditions with $\gamma_n = 1-e^{-a}$ in interval $[1,fn]$ for any constant $0 < f < 1$.
    
    For $k = \Theta(n)$, Lemma~\ref{lemma: Erdos-Renyi - growth and shrinking} and Lemma~\ref{lemma: Erdos-Renyi - shrinking} yield that
    \[
    	\left(1 - \tfrac{1-e^{-a}}{n}\right)^k - O\left(\tfrac{\log^2 n}{n}\right)
    	\le 1-p_k
    	\le \left(1 - \tfrac{1-e^{-a}}{n}\right)^k + k \cdot O\left(\tfrac1{n^2}\right).
    \]
    Substituting $k$ by $n-u$ and applying Corollary~\ref{cor:prelim:(1-p/n)^(n-u)}, we obtain for any $u < n$ that
    \[
    	\exp\left(-1+e^{-a}\right) - O\left(\tfrac{\log^2 n}{n}\right)
    	\le 1-p_{n-u}
    	\le \exp\left(-1+e^{-a}\right) \left(1 + 2\left(1-e^{-a}\right)\tfrac un\right) + O\left(\tfrac1n\right).
    \]
    Therefore, the protocol satisfies the upper exponential shrinking conditions with $\rho_n = 1-e^{-a}$ and the lower exponential shrinking conditions with $\rho_n = 1-e^{-a} + O\left(\tfrac{\log^2n}{n}\right)$ in the interval $[n-gn,n]$ for any $0 < g < 1$.

	Since the intervals of exponential growth and exponential shrinking overlap, it follows from Theorems~\ref{th:exp-growth-upper},~\ref{th:exp-growth-lower},~\ref{th:exp-shrinking-upper},~and~\ref{th:exp-shrinking-lower} that the expected spreading time $\E[T]$ is equal to $\log_{1-e^{-a}}n + \tfrac1{1-e^{-a}}\ln n \pm O(1)$ and $\Pr[|T-\E[T]| \ge r] \le A' e^{-\alpha' r}$ for suitable constants $A', \alpha' > 0$.
\end{proof}

%\subsubsection{Pull Protocol}
%\subsubsection*{Push-Pull Protocol}
%
%Idea. The goal is to estimate the probability that node $x$ gets informed simultaneously by push and pull call.
%The naive way to model such situation is the following.
%\begin{enumerate}
%	\item Create the random graph.
%	\item If $\deg x > 0$, make the pull call.
%	\item Modelize the push calls of all informed neighbors of $x$.
%\end{enumerate}
%In this case "the destiny" of $x$'s pull call depends on the ratio between number of informed and uninformed neighbors of $x$.
%Such object is very hard to calculate.
%Hence we will study the following model which seems to be equivalent.
%
%\begin{enumerate}
%	\item Throw a coin w.p. $1-\left(1-\tfrac an\right)^{n-1}$, i.e., probability that $\deg x > 0$.
%		Continue only if we won.
%	\item
%		Make an $x$'s pull call uniformly at random. And add the corresponding edge to the graph.
%	\item
%		Fill the remaining graph.
%	\item
%		Modelize the push calls of all informed neighbors of $x$.
%\end{enumerate}
%In such model it is relatively easy to compute the conditional probability such as $\Pr[push|pull]$.

 \section{Limited Incoming Calls Capacity} \label{sec:single incoming call}

For all the protocols discussed above the nodes are allowed to be called several times in one round.
For some processes such as protocols considered in Section~\ref{sec:dynamic graphs}, the number of calls received by each node is at most constant.
However in most of rumor spreading processes such number can be unbounded.
For example, consider the basic push-pull protocol from Section~\ref{sec:push-pull} on the complete graph with $n$ vertices.
Since each round all nodes make calls, the maximum number of incoming calls received by the same node in one round is the same as the maximum load of a bin in the well-known problem of throwing uniformly and independently at random $n$ balls into $n$ bills, i.e., $\frac{\log n}{\log\log n} \cdot (1+o(1))$.
Such phenomenon can impact the scalability of the rumor spreading process: typically the time gap between rounds is bounded, but each round with high probability there is at least one node which have to finish $\omega(1)$ transactions.

The simplest solution is to limit the incoming ``capacity'' of nodes, i.e., the number of calls they can reply in one round.
In this section we propose a \emph{single incoming call} setting -- any node can reply to only one incoming call per round chosen uniformly at random among all received calls in current round.
All other calls are considered ``dropped", i.e., they cannot transfer the rumor.
Therefore, each node participates in at most two rumor transactions per round, whatever is the size of the network.

On the other hand, we expect the noticeable slowdown for the protocols based on the single incoming call setting compared to the usual unlimited ``capacity'' setting.
Thus we will show in Section~\ref{sec:single-push-pull} that the single incoming call push-pull protocol satisfies the single exponential shrinking conditions instead of double exponential shrinking and the corresponding expected rumor spreading time is equal to $\log_{3-2/e}n + \tfrac12\ln n \pm O(1)$.
In Section~\ref{sec:single-pull} we argue that since $\Theta(n)$ nodes are informed, the push calls of informed nodes becomes inefficient and they are responsible for such considerable slowdown.
Finally, in Section~\ref{sec:single call-fast} we combine a single incoming call push-pull protocol with pull protocol and provide a not memoryless process with spreading time
$\log_{3-2/e}n + \log_2\ln n + O(1)$.

Before proceeding to the computations, we observe that the following setting is equivalent to the single incoming call model.
In each round we choose uniformly at random a permutation $\sigma \in S_n$.
The element $\sigma_n$ is the \emph{order} of the outgoing call of node $x_i$, we write $ord_i = \sigma_i$.
Each node accepts the call with the lowest order among its received incoming calls.
We call such construction the \emph{ordered calls} setting.
%Before getting started we make some easy observations.
%%First we will reformulate the notation of randomly accepting of a call.
%
%%One can wonder what is the probability that one fixed call is accepted.
%%We show that it is $1-1/e + O(1/n)$.
%Let us enumerate the vertices of the complete graph from $1$ to $n$ and consider the $i$-th node introducing the events $C_i$: ``the $i$-th node has been called in current round" and $A_i$: ``the outgoing call of the $i$-th node is accepted".
%%\begin{itemize}
%%	\item $C_i$, if the $i$-th node has been called in current round;
%%	\item $A_i$, if the outgoing call of the $i$-th node is accepted;
%%\end{itemize}
%
%\begin{observation} \label{obs:called/accepted}
%	$\Pr[C_i] = \Pr[A_i] = 1-(1-1/n)^n$.
%\end{observation}
%\begin{proof}
%	Indeed, $\Pr[C_i] = 1-(1-1/n)^n$ is trivial.
%	To prove the second part we use the equivalence from Observation~\ref{obs:equivalence}.
%	Thus, the outgoing call of the $i$th node can have any number $j$ with probability $\frac1n$.
%	This call is accepted if none of the first $j-1$ calls hits the same node before.
%	Therefore,
%	\[
%		\Pr[A_i] = \sum_{j=1}^n \tfrac1n \left(1-\tfrac1n\right)^{j-1} = 1-\left(1-\tfrac1n\right)^n.
%	\]
%\end{proof}
%
%We note that that the events $C_i$ and $A_j$ are not independent.
%Thus, if one call is accepted, it occupies one spot decreasing the probability of other calls to be accepted.
%%Formally we can show the following.
%
%%By the difference from the independent call model, there is another one source of the dependency: if one of the calls has been accepted, it occupies one spot and makes the other calls less likely to be accepted.
%%Let us hence analyze the correlation of $A_i$ and $C_i$.


\subsection{Single Incoming Call Push-Pull Protocol}\label{sec:single-push-pull}

%\begin{def*}[Independent Call Model]
%    Suppose we also have a clock shared by all nodes counting the rounds since the rumor appeared.
%    Each round each node chooses one of its neighbors in $G$ uniformly at random and communicates with it ("calls it").
%    If $x$ communicates with $y$, one of them knows the rumor and the second don't, then it tells the rumor to the other one.
%\end{def*}

%\begin{def*}[Single incoming call push-pull protocol]
%    Let a graph $G$ given and its vertex $v$ initially knows the rumor.
%    Each round the nodes set up their calls according to the single incoming call model.
%    Within each accepted call the rumor passes from informed to uninformed node.
%\end{def*}
%
%We observe that the following setting is equivalent to the single incoming call model.
%In each round we choose uniformly at random a permutation $\sigma \in S_n$.
%The element $\sigma_n$ is the \emph{order} of the outgoing call of node $x_i$, we write $ord_i = \sigma_i$.
%Each node accepts the call with the lowest order among its received incoming calls.
%We call this construction the \emph{ordered calls model}.

\begin{theorem} \label{th:single push-pull}
	The expected spreading time for the single incoming call push-pull protocol is $\log_{3-2/e} n + \tfrac12\ln n + O(1)$.
\end{theorem}
%\merk{A: Attention! the rate of the exponential shrinking has been changed}

In this section we keep the notation from the previous ones, i.e. $X_i$ is the random indicator variable corresponding to the event ``uninformed node $x_i$ gets informed in considered round''.
Since all considered protocols are uniform, we denote by $p_k$ the probability $\Pr[X_i=1]$ for the round started with $k$ informed nodes and any $i$.
In addition we denote by $Y_i, Z_i$ the indicator random variables for the following events.
\begin{itemize}
    \item [$Y_i$] ``Node $i$ is called and the first incoming call comes from an informed node.''
    \item [$Z_i$] ``The outgoing call of node $i$ is accepted by an informed node.''
\end{itemize}

%\begin{def*}[Single incoming call push-pull protocol]
%    Let a graph $G$ given and its vertex $v$ initially knows the rumor.
%    Each round each node chooses independently and uniformly at random one node and calls it.
%	These choices describe a directed \emph{call graph} $C_t$.
%    Then each called node chooses independently and uniformly at random among its incoming edges (i.e. calls) only one and \emph{accepts} it.
%    Not accepted calls are rejected.
%    The remained structure is called a \emph{communication graph} $G_t$.
%    At the end of the round all neighbors of informed nodes in $G_t$ get the rumor and become informed.
%\end{def*}

\begin{lemma}
    Suppose that the fraction $f$ of nodes is informed.
    Suppose node $i$ is uninformed.
    Then
    \begin{equation}
        p_{fn} = 2f\left(1-\tfrac1e\right) - f^2\left(1-\tfrac1e\right)^2 + f\cdot O\left(\tfrac1n\right) \label{eq:single-prob-informed}.
    \end{equation}
\end{lemma}
\begin{proof}
%    We introduce the following events.
%    \begin{enumerate}[(1)]
%        \item Node $i$ is called and the first incoming call comes from informed node.
%        \item The outgoing call of node $i$ is accepted by informed node.
%    \end{enumerate}
    First, we compute the probabilities of the events corresponding to $Y_i$ and $Z_i$.
    Since each node makes a call in the round, the probability that node $x_i$ is not called is equal to $(1-\tfrac1n)^n$.
    Therefore,
    \[
        \Pr[Y_i=1]
        = f \left(1-\left(1-\tfrac1n\right)^n\right)
        = f \left(1-\tfrac1e\right) + f\cdot O\left(\tfrac1n\right).
    \]

    To compute $\Pr[Z_i=1]$ we will use the ordered call model.
    Suppose that $ord_i = \ell$.
    Then, the outgoing call of node $x_i$ is accepted if all calls with orders less than $\ell$ do not call the same node.
    Since the probability that the outgoing call of node $x_i$ has order $\ell$ is equal to $\tfrac1n$, we compute
    \[
        \Pr[Z_i=1]
        = f \sum_{\ell=1}^n \tfrac1n \left(1-\tfrac1n\right)^{\ell-1}
        = f \left(1-\left(1-\tfrac1n\right)^n\right)
        = f \left(1-\tfrac1e\right) + f\cdot O\left(\tfrac1n\right).
    \]


    Since $X_i = \max\left\{Y_i,Z_i\right\}$, it remains to compute the probability of the event $Y_i=Z_i=1$.
    Suppose that $ord_i = \ell$.
    Since the outgoing call of node $x_i$ is accepted, all calls with order less than $\ell$ should go away from the $x_i$'s target, i.e., they can have only $n-1$ possible targets.
    We also remark that node $x_i$ calls informed node, so it cannot call itself.
    Thus the probability that nobody calls node $x_i$ is equal to $\left(1-\tfrac1{n-1}\right)^{i-1}\left(1-\tfrac1n\right)^{n-i}$.
    Therefore,
    \begin{align*}
        \Pr[Z_i=1|Y_i=1, \; ord_i = \ell]
        & = f\left(1 - \left(1-\tfrac1{n-1}\right)^{i-1} \left(1-\tfrac1n\right)^{n-i}\right) \\
        & = f\left(1 - \left(1-\tfrac1n\right)^n + O\left(\tfrac1n\right)\right).
    \end{align*}
    Since the probability above is independent of $\ell$, we obtain immediately that node
    \begin{align*}
        \Pr[Y_i=Z_i=1]
        %& = 2f\left(1-\left(1-\tfrac1n\right)^n\right) \\
        & = f^2\left(1-\left(1-\tfrac1n\right)^n\right)^2 + f^2 \cdot O\left(\tfrac1n\right) \\
        & = f^2\left(1-\tfrac1e\right)^2 + f^2 \cdot O\left(\tfrac1n\right).
    \end{align*}
    The claim of lemma follows by including-excluding formula.
\end{proof}

\begin{lemma}\label{lem:single-conditional-prob}
    There exists $c \ge 0$ such that for any uninformed nodes $x_i \ne x_j$ we have
    \begin{equation}
        \Pr[X_i=1|X_j=1] \le \Pr[X_i=1] + \tfrac{c}{n}. \label{eq:single-prob-conditional}
    \end{equation}
\end{lemma}
\begin{proof}
    We say that nodes $x_i$ and $x_j$ \emph{interact} if one calls another or if they both call the same node.
    Clearly, $\Pr[x_i, x_j \text{ interact}|X_j=1] = O\left(\tfrac1n\right)$.
    Since we need to bound $\Pr[X_i=1|X_j=1]$ up to $O(\tfrac1n)$, without loss of generality we assume for the rest of the proof that nodes $x_i$ and $x_j$ do not interact.
    We say that a call interacts with a node $x_j$ if its target coincides with $x_j$ or with $x_j$'s target (by convention a call does not interact with it source).
    Denote by $I_j$ the number of calls interacting with node $x_j$ and observe that since $x_i$ and $x_j$ don't interact, no node can interact with both $x_i$ and $x_j$.
    We split the probability $\Pr[X_i=1|X_j=1]$ conditioning on the values of $I_j$ as follows.
    \begin{align*}
        \Pr[X_i=1|X_j=1] = \sum_{k=1}^n \Pr[X_i=1|X_j=1,I_j=k] \cdot \Pr[I_j=k|X_j=1].
    \end{align*}
    Our goal is to study $\Pr[X_i=1|X_j=1,I_j=k]$.
    Since $k$ nodes interact with $x_j$, there are $n-k-1$ independent calls going uniformly to $n-2$ remaining targets (except $x_j$ and $x_j$'s target).
    In addition at least $n(f - \tfrac{k+1}{n})$ of calls are made by informed nodes.
    By these two observations we deduce
    \begin{align*}
        \Pr[Y_i=1|X_j=1,I_j=k]
        & = \left(f - \tfrac{k+1}{n}\right) \left(1-(1-\tfrac1{n-2})^{n-k-1}\right) \\
        & = f \left(1-\left(1-\tfrac1n\right)^n\right) + kO\left(\tfrac1n\right)
        = f \left(1-\tfrac1e\right) + k\cdot O\left(\tfrac1n\right).
    \end{align*}
    By the similar analysis we obtain that
    \begin{align*}
        \Pr[Z_i=1|X_j=1,I_j=k] & = f \left(1-\tfrac1e\right) + k\cdot O\left(\tfrac1n\right); \\
        \Pr[Y_i=Z_i=1|X_j=1,I_j=k] & = f^2 \left(1-\tfrac1e\right)^2 + k\cdot O\left(\tfrac1n\right).
    \end{align*}
    Therefore, $\Pr[X_i=1|X_j=1,I_j=k] = \Pr[X_i=1] + k \cdot O\left(\tfrac1n\right)$.
    Since $\Expect[I_j|X_j=1] = O(1)$, we sum up by $k$ and obtain
    \begin{align*}
        \Pr[X_i=1|X_j=1]
        & = \Pr[X_i=1] + \sum_{k=1}^n kO\left(\tfrac1n\right) \cdot \Pr[I_j=k|X_j=1] \\
        & = \Pr[X_i=1] + O\left(\tfrac1n\right) \Expect[I_j|X_j=1]
        = \Pr[X_i=1] + O\left(\tfrac1n\right).
    \end{align*}
\end{proof}

\begin{proof}[Proof of Theorem~\ref{th:single push-pull}]
    Consider a round started with $k$ informed nodes.
    Substituting $f$ by $k/n$ in~\eqref{eq:single-prob-informed}, we obtain the probability part of the exponential growth conditions.
    \[
        p_k = 2\left(1-\tfrac1e\right)\cdot\tfrac{k}{n} + k^2 \cdot O\left(\tfrac1{n^2}\right).
    \]
    Multiplying~\eqref{eq:single-prob-conditional} by $p_k$ we get the covariance condition.
    Therefore the protocol satisfies the exponential growth conditions with $\gamma_n = 2(1-\tfrac1e)$.

    Denote by $u:=n-k$ the number of uninformed nodes.
    Substituting $f$ by $1-\tfrac{u}{n}$ in~\eqref{eq:single-prob-informed}, we compute
    \[
        \Pr[X_i=0] = 1-\Pr[X_i=1]
        = \tfrac1{e^2} + O\left(\tfrac1n\right).
    \]
    Since the covariance condition follows from Lemma~\ref{lem:single-conditional-prob}, the protocol satisfies the exponential shrinking conditions with ${\rho_n} = 2$.
    Therefore the expected spreading time is equal to $\log_{3-2/e} n + \tfrac12\ln n + O(1)$.
\end{proof}

%\begin{lemma} \label{lem:single-growth}
%	Consider one uninformed node.
%	Let at the beginning of current round there are $k$ informed nodes.
%	Then the probability that this node becomes informed in current round is
%	$2\left(1-\tfrac1e\right) \cdot \tfrac{k}{n} + O\left(\tfrac{k^2}{n^2}\right)$.
%\end{lemma}
%\begin{proof}
%	Indeed, the $i$-th uninformed node can be informed by one of the two ways.
%	In the pull mechanism it calls informed node and its call is accepted, totally with probability $\Pr[A_i]\cdot\tfrac{k}{n}$.
%	To be informed by the push mechanism if it is called and the accepted incoming call comes from informed node; the corresponding probability is $\Pr[C_i]\cdot\tfrac{k}{n}$.
%	The probability that two ways take place in the same round for the same node is at most $O\left(\tfrac{k^2}{n^2}\right)$ because the $i$th node should both call and be called by informed one.
%	So, applying Observation~\ref{obs:called/accepted}, we deduce that
%	\[
%		\Pr[X_i(k)=1] = 2\left(1-\tfrac1e\right) \cdot \tfrac{k}{n} + O\left(\tfrac{k^2}{n^2}\right).
%	\]
%\end{proof}
%
%\begin{lemma} \label{lem:single-shrinking}
%	Consider one uninformed node in round with $u$ uninformed ones.
%	Then the probability that this node stays uninformed in current round is $\tfrac1e + \Ofun$.
%\end{lemma}
%\begin{proof}
%	Suppose, that the current node is called.
%	It stay uninformed only if its accepted incoming call was made by uninformed node, with probability $\fun$ so the contribution of this case is $\Ofun$.
%	
%	By Observation~\ref{obs:called/accepted}, probability that nobody calls current node equals to $\left(1-\frac1n\right)^n = \frac1e + O\left(\frac1n\right)$.
%	Being not called, the current node stays uninformed if it calls an uninformed node with probability $1-\tfrac{u}{n-1}$. So the uncalled node stays uninformed with total probability $1-O\left(\tfrac{u}{n}\right)$.
%	
%	Two cases above are excluding, so we obtain immediately that the total probability of staying uninformed is $\tfrac1e + O\left(\frac{u}{n}\right)$.
%\end{proof}
%%\texttt{\merk{some formalization}
%%\begin{remark*}
%%	The following formal procedure is equivalent to the single incoming call push-pull model.
%%	\begin{itemize}
%%		\item Generate uniformly at random an integer vector $v \in [1;n]^n$.
%%		\item Generate two permutations $\sigma, \tau \in S_n$, also uniformly at random.
%%		\item We say that the node $\sigma_i$ calls $v_i$ with the priority $\tau_i$.
%%	\end{itemize}
%%\end{remark*}
%%Here by a random call $c$ we understand a triplet $(v_c, \sigma_c, \tau_c)$.}
%
%Next we need to prove that the covariance $\Cov[X_i,X_j]$ is small.
%This means that the knowledge of the fate of one node cannot strongly influence another one.
%In order to do so, we will show that almost surely one node "interacts" with only few peers in one round.
%
%\begin{def*}
%	We say that the outgoing call of the $i$th node \emph{interacts} with the $j$th node ($i \ne j$) if one of the following holds:
%	\begin{enumerate} [(i)]
%		\item $i$ calls $j$;
%		\item $i$ and $j$ call the same node.
%	\end{enumerate}
%\end{def*}
%\begin{observation} \label{obs:interact-probability}
%	Consider the $i$th node and a random call $c$, which is not emitted by $i$.
%	The probability that $i$ and $c$ interact is $\tfrac2n + \Odnsq$.
%\end{observation}
%\begin{observation} \label{obs:interact-same probability}
%	Let the call $c$ interacts with the $i$th node, which doesn't call itself in current round.
%	Then the probabilities that $c$ calls to the $i$th node or to the node called by the $i$th one are both equal to $\tfrac12$.
%\end{observation}
%
%We denote by $I_i$ the number of calls interacting in current round with the $i$th node.
%
%\begin{observation} \label{obs:interact-conditioning}
%	$\quad$
%	\begin{enumerate} [(i)]
%		\item $\Pr[I_i=k|I_j=\ell] \le \Pr[I_i = k]$, for any $\ell \ge 2$.
%		\item $\Pr[I_i=k|I_j=\ell] = \Pr[I_i = k] + \Odn$, for any $\ell \in \{0,1\}$.
%	\end{enumerate}
%\end{observation}
%\begin{proof}
%	Indeed, by Observation~\ref{obs:interact-probability}, $\E[I_j] = 2$, so $(i)$ holds for any $\ell \ge 2$.
%	It is easy to see the explicit expression for $\Pr[I_i=k]$:
%	\begin{equation} \notag
%		\Pr[I_i=k] = \binom{n-1}{k} \cdot \left(\tfrac2n\right)^k \cdot \left(1-\tfrac2n\right)^{n-k-1}.
%	\end{equation}
%	With probability $\Odn$ the nodes $i$ and $j$ don't interact, so we can assume that any call interacts with at most one of them.
%	So,
%	\begin{align*} \notag
%		\Pr[I_i=k|I_j=\ell]
%		&= \binom{n-1-\ell}{k} \cdot\left(\tfrac{2}{n-2}\right)^k
%			\cdot \left(1-\tfrac2{n-2}\right)^{n-k-1-\ell} + \Odn \\
%		&= \Pr[I_i=k] + \Odn, \text{  for any } \ell \in \{0,1\}
%	\end{align*}
%\end{proof}
%
%\begin{lemma}
%	If $\Pr[X_i=1|X_j=1] \ge \Pr[X_i=1]$, then their difference is of order $\Odn$.
%\end{lemma}
%\begin{proof}
%	So, $\Pr[X_i=1] = \sum_{k\ge0} \Pr[X_i=1|I_i=k] \cdot \Pr[I_i=k]$.
%	In addition, the probability that one of nodes $i$, $j$ calls itself is $\Odn$, so w.l.o.g. we can suppose that none of them calls itself.
%	
%	The $i$th node can be informed by three independent scenarios and for each one we will compute the contribution to the probability $\Pr[X_i=0|I_i=k]$, supposing that in current round there are $u < n$ uninformed nodes.
%	
%	By Observation~\ref{obs:interact-same probability}, the probability that the $i$th node is not called is $\ftwok$.
%	In this case all $k$ calls go to the target of the $i$'s call so the probability that the outgoing call of the $i$th node is accepted is $\tfrac1{k+1}$, and the probability that the target is informed is $1-\fun$.
%	
%	Consequently, the $i$th node is called with probability $1-\ftwok$.
%	The second scenario is that the $i$th node accepts the call from informed node with corresponding probability $1-\fun$.
%	
%	The last case is that the $i$th node is called, but accepts a call from uninformed node.
%	So, to be informed, the $i$th node must call an informed node (the probability of such event is $1-\fun$).
%	Moreover, this call should be accepted.
%	Let us denote by $\ell \in [0,k-1]$ the number of the "concurrents" of the $i$'s outgoing call.
%	The probability that the $i$'s call is accepted is $\tfrac1{\ell+1}$ and, as the probability to have exactly $\ell$ "concurrents" is $\ftwok \binom{k}{\ell}$, the total probability is the following.
%	
%	\begin{align}
%		\Pr[X_i=1&|I_i=k]
%		= \ftwok \cdot \tfrac1{k+1}\left(1-\fun\right)
%			+ \left(1-\ftwok\right) \cdot \left(1-\fun\right) \notag \\
%		& + \left(1-\ftwok\right) \fun
%				\cdot\sum_{\ell=0}^{k-1}\ftwok\binom{k}{\ell}
%				\cdot\tfrac1{\ell+1}\cdot\left(1-\fun\right) \label{eq:cov-2}
%	\end{align}
%	
%	Now we want to condition the probability in~\eqref{eq:cov-2} to the events $X_j=1$ and $I_j=k'$.
%	Indeed,
%	\begin{equation}\label{eq:cov-4}
%		\Pr[X_i=1|X_j=1] = \sum_{k'} \Pr[X_i=1|X_j=1, \; I_j=k'] \cdot \Pr[I_j=k'|X_j=1].
%	\end{equation}
%	We can also split the probability $\Pr[X_i=1|X_j=1, \; I_j=k']$ by the number $k$ of calls interacting with the $i$th node.
%	Therefore,
%	\begin{align}
%		\Pr&[X_i=1|X_j=1, \; I_j=k'] \notag \\
%		& = \sum_k \Pr[X_i=1|X_j=1, \; I_j=k', \; I_i=k]
%			\cdot \Pr[I_i=k|I_j=k', \; X_j=1].
%	\end{align}
%	Obviously, $\Pr[I_i=k|I_j=k', \; X_j=1] = \Pr[I_i=k|I_j=k']$.
%	Then, applying Observation~\ref{obs:interact-conditioning}, we get
%	\begin{align}
%		\Pr&[X_i=1|X_j=1, \; I_j=k'] \notag \\
%		& \le \sum_k \Pr[X_i=1|X_j=1, \; I_j=k', \; I_i=k]
%			\cdot \Pr[I_i=k] + \Odn. \label{eq:cov-3}
%	\end{align}
%	Comparing the probability $\Pr[X_i=1|X_j=1, \; I_j=k', \; I_i=k]$ with $\Pr[X_i=1|I_i=k]$, one can obtain the similar expression as in~\eqref{eq:cov-2}.
%	The only difference is that knowledge of the "fate" of the $j$th node "occupies" at most two explicit  nodes -- the target of the $j$'s outgoing call and the source from the first $j$'s incoming call.
%	This shrinks the probability of the $i$th node to call uninformed node or to be called by uninformed one from $\fun$ to $\fun + \Odn$.
%	So, $\Pr[X_i=1|X_j=1, \; I_j=k' \; I_i=k] = \Pr[X_i=1|I_i=k] + \Odn$.
%	Or, applying the last result to~\eqref{eq:cov-3}, we obtain that
%	\begin{align*}
%		\Pr& [X_i=1|X_j=1, \; I_j=k'] \\
%		& \le	\sum_{k,k'} \left(\Pr[X_i=1|I_i=k] + \Odn\right) \cdot \Pr[I_i=k] + \Odn
%		& = \Pr[X_i=1] + \Odn.
%	\end{align*}
%	The claim of lemma immediately follows from result above and~\eqref{eq:cov-4}.
%\end{proof}
%
%\begin{corollary} $\quad$
%	\begin{enumerate}
%		\item $\Cov[X_i,X_j] = \Oknn$ in the exponential growth regime;
%		\item $\Cov[X_i,X_j] = \Odn$ in the exponential shrinking regime.
%	\end{enumerate}
%\end{corollary}
%\begin{proof}
%	Indeed, as $X_i$ and $X_j$ are both binary variables,
%	\begin{align*}
%		\Cov[X_i,X_j]
%		& = \Pr[X_i=X_j=1] - \left(\Pr[X_j=1]\right)^2 \\
%		& = \Pr[X_j=1] \left(\Pr[X_i=1|X_j=1] - \Pr[X_i=1]\right).
%	\end{align*}
%	The claim obviously follows from the fact that in the exponential growth regime $\Pr[X_i=1] = O\left(\tfrac{k}{n}\right)$, where $k$ is the number of currently informed nodes.
%\end{proof}
%
%%So, $\Pr[X_j=0] = \sum_k \Pr[X_j=0|I_j=k] \cdot \Pr[I_j=k] + \Odn$.
%%The term $\Odn$ comes from the case if the $j$th node calls itself with probability $\tfrac1n$.
%%
%%Let us found explicitly the probability $\Pr[X_j=0|I_j=k]$, supposing that in current round there are $u < n$ uninformed nodes.
%%The $j$th node can stay uninformed by two independent scenarios and for each one we will compute the contribution to the probability.
%%First, it may call an uninformed node with probability $\fun$.
%%In this case the $j$th node can only be informed receiving the push call from informed node.
%%As there are $k$ calls interacting with $j$th node, the probability that it is not called is $\ftwok$.
%%Or, the $j$th node is called with probability $1-\ftwok$, but with probability $\fun$ the call comes from uninformed node.
%%Therefore, the contribution of the first scenario is
%%$\fun \cdot \left(\left(1-\ftwok\right)\fun + \ftwok\right)$.
%%
%%The second scenario realizes if the $j$th node calls some informed node (with probability $1-\fun$).
%%Suppose that $\ell \le k$ node also call the same node.
%%Then one of them should have the higher priority than the outgoing call of the $j$th node.
%%The corresponding probability is $\tfrac{\ell}{\ell+1}$.
%%If $\ell<k$ then the remaining $k-\ell$ calls hit the $j$th node and the accepted one should be made by an uninformed node with probability $\fun$.
%%Summing over $\ell$ we obtain the contribution of the second scenario.
%%Therefore,
%%\begin{align}
%%	Pr[X_j=0|I_j=k] & = \fun \cdot \left[\left(1-\ftwok\right)\fun + \ftwok\right] \notag\\
%%	& + \left(1-\fun\right) \cdot \left[\sum_\ell\binom{k}{\ell}\cdot\ftwok\cdot\tfrac{\ell}{\ell+1}\cdot\fun + O\left(\ftwok\right)\right]. \label{eq:cov-1}
%%\end{align}
%%
%%\begin{lemma}
%%	$\Cov[X_i,X_j]$ is small.
%%\end{lemma}
%%\begin{proof}
%%	As $X_i$ and $X_j$ are both binary variables,
%%	$$\Cov[X_i,X_j] = \Pr[X_i=X_j=0] - \left(\Pr[X_j=0]\right)^2.$$
%%	We can already compute the probability that $X_j=0$.
%%	So let us compare it to the joint probability.
%%	Indeed,
%%	\begin{align*}
%%		\Pr[X_i=&X_j=0] = \sum_\ell \Pr[X_i=X_j=0|I_j=\ell] \cdot \Pr[I_j=\ell] \\
%%		& = \sum_\ell \Pr[X_i=0|X_j=0 \AND I_j=\ell] \cdot \Pr[X_j=0|I_j=\ell] \cdot \Pr[I_j=\ell].
%%	\end{align*}
%%	We can apply the same trick to $\Pr[X_i=0|X_j=0 \AND I_j=\ell]$:
%%	\begin{align*}
%%		\Pr[X_i=0|X_j=0\AND I_j=\ell]
%%		= & \sum_k \Pr[X_i=0|X_j=0\AND I_j=\ell \AND I_i=k] \\
%%		& \times \Pr[I_i=k|X_j=0\AND I_j=\ell].
%%	\end{align*}
%%	It is easy to see that
%%	\[
%%		\Pr[I_i=k|X_j=0\AND I_j=\ell] = \Pr[I_i=k|I_j=\ell]
%%		= O\left(\left(1-\tfrac{\ell}{n}\right)^\ell \cdot \tfrac{2^k}{k!}\right).
%%	\]
%%	Comparing the probability $\Pr[X_i=0|X_j=0\AND I_j=\ell \AND I_i=k]$ with $\Pr[X_i=0|I_i=k]$, one can obtain the similar expression as in~\eqref{eq:cov-1}.
%%	The only two differences are that any probabilities:
%%	\begin{itemize}
%%		\item we suppose that with probability $\Odn$ the $i$th and $j$th nodes don't interact in current round;
%%		\item the probability $\fun$ of call uninformed node or be called by uninformed one transforms into $\fun + \Odn$, as some uninformed nodes can be "occupied" by the $j$th one.
%%	\end{itemize}
%%	Therefore, $\Pr[X_i=0|X_j=0\AND I_j=\ell \AND I_i=k] = \Pr[X_i=0|I_i=k] + \Odn$.
%%	So, $\Cov[X_i=X_j=0] \le \Odn$, if $u \ge gn$.
%%\end{proof}
%%\begin{remark}
%%	It seems that the last proof can be easily generalized to the case of exponential growth\ldots
%%\end{remark}


\subsection{Single Incoming Call Pull-Only Protocol}\label{sec:single-pull}
We showed that the the single call push-pull protocol is significantly slower than the classic push-pull protocol.
Although protocol based on the single incoming call setting cannot be faster than the classic independent call model, we can make it noticeably faster using the following trick.
Let us consider one round of the exponential shrinking phase with $u$ uninformed nodes.
In such round there are $n-u$ push calls, each one hits uninformed node with small probability $\fun$.
On the other hand, each of $u$ pull calls touches some informed node with probability $1-\fun$.
One can conclude that push calls ``spam'' the network: they ``occupy'' other informed nodes making them inaccessible for pull calls of uninformed nodes.
This observation is verified in the following theorem.
%Indeed, let us introduce the single incoming call pull protocol.

%\begin{def*}[Single incoming call pull protocol]
%    Let a graph $G$ given and its vertex $v$ initially knows the rumor.
%    Each round only uninformed nodes make their calls.
%    To get informed, an uninformed node must call informed node and its call should be accepted according to the single incoming call model.
%\end{def*}

\begin{theorem} \label{th:single pull}
	The spreading time for the single incoming call pull protocol is $\log_{2-1/e} n + \log_2\ln n + O(1)$.
\end{theorem}
\begin{proof}
	Consider one round of the protocol.
    Clearly, if $x_1$ becomes informed it ``occupies'' one informed node which cannot inform any other node in current round.
    Thus, if we condition on that $X_1=1$, then it is slightly less likely that $x_2$ becomes informed. Consequently, $\Cov[X_1,X_2] < 0$ and the covariance part of the exponential growth and double exponential shrinking conditions is satisfied.

    Again, the call with order $\ell$ is accepted with probability $\left(1-\tfrac1n\right)^{\ell-1}$.
	Since in the round started with $k$ informed nodes only $n-k$ nodes perform calls, $ord_i$ is uniformly distributed in $\{1,\ldots,n-k\}$.
    Since the probability to call an informed node is $\tfrac{k}{n}$, we compute
    \begin{equation}
        p_k = \tfrac{k}{n}\sum_{\ell=1}^{n-k} \tfrac1{n-k}\left(1-\tfrac1n\right)^{\ell-1}
        = \tfrac{k}{n-k}\left(1-\left(1-\tfrac1n\right)^{n-k}\right) \label{eq:single-pull-prob}.
    \end{equation}
    By Corollary~\ref{cor:prelim:(1-1/n)^(n-u)}, we have
    \[
        \left(1-\tfrac1e\right)\tfrac{k}{n} - 4\tfrac{k^2}{n^2}
        \le p_k
        \le \left(1-\tfrac1e\right)\tfrac{k}{n} + 2\left(1-\tfrac1e\right)\tfrac{k^2}{n^2}.
    \]
    So the protocol satisfies the exponential growth conditions with parameter $\gamma_n = 1-\tfrac1e$.

    If we denote by $u$ the number of uninformed nodes, from~\eqref{eq:single-pull-prob} follows the following expression.
    \[
        1-p_{n-u} = \tfrac{n-u}{u} \left(1-\left(1-\tfrac1n\right)^u\right).
    \]
    With Lemma~\ref{lem:prelim:(1-1/n)^k}, we estimate $\tfrac{u}{n} \le 1-p_{n-u} \le \tfrac{3u}{2n}$.
    The protocol hence satisfies the double exponential shrinking conditions with $\ell=2$.

    Therefore, the expected spreading time is equal to $\log_{2-1/e} n + \log_2\ln n + O(1)$.
\end{proof}

%\begin{lemma}
%	The probability that a uninformed node stays uninformed in one round is $\tfrac32\eps + \Oepsq$.
%%	The single incoming call pull protocol satisfies the double exponential shrinking conditions with parameter $\ell = 2$.
%\end{lemma}
%\begin{proof}
%	Let us consider one round of the single incoming call pull protocol with $\eps n$ uninformed nodes.
%	In current round only $\eps n$ nodes will make calls.
%	By the same idea as in the proof of Observation~\ref{obs:called/accepted}, the probability that one call is accepted is
%	\begin{align*}
%		\sum_{i=1}^{\eps n} & \tfrac1{\eps n} \cdot \left(1-\tfrac1n\right)^{i-1}
%			= \tfrac1{\eps n} \tfrac{1-\left(1-\tfrac1n\right)^{\eps n}}{1-\left(1-\tfrac1n\right)} \\
%		& = \tfrac1\eps \left( 1 - \left(1-\tfrac1n\right)^{\eps n} \right)
%			= 1 - \tfrac\eps2 + O\left(\eps^2+\tfrac{\eps}{n} \right)
%	\end{align*}
%	The node becomes informed if and only if it calls informed node with probability $1-\eps$ and its call is accepted.
%	Therefore the probability that this node stays uninformed in current round is $\tfrac32\eps + O\left(\eps^2\right)$.
%	
%	The covariance condition is proved in the lemma above, so the single incoming call pull protocol satisfies the double exponential shrinking conditions.
%\end{proof}

%Although such slowdown is the price of dealing with only one incoming call per node per round and seems to be unavoidable, we would like to design the single incoming call process having the double exponential shrinking regime instead of exponential shrinking one.
%The reason of the loss of rapidity in the shrinking regime of the single call push-pull protocol compared to its classical version raises from the pull calls.
%Indeed, when the most of the nodes are informed, the most of the calls are push ones, but their efficiency decreases and they only spam the network.
%So let us study the single incoming call version of the classic pull protocol, which shouldn't be subject to such problem.

%To understand this we recall that the probability of a single call to be accepted is a constant in $]0,1[$ which doesn't depend on the number of current round.
%Therefore the probability of one node to stay uninformed in current round is non-negligible for any number of uninformed nodes.
%On the other hand, we know that the pull protocol is much more efficient, as it performs a double exponential shrinking regime.
%Therefore is looks reasonable to consider the \emph{single incoming call pull protocol}, i.e. the protocol acting on the single incoming call model with the constraint that only uninformed nodes are allowed to make calls.

%\begin{def*}[Single incoming call push-pull protocol]
%    Let a graph $G$ given and its vertex $v$ initially knows the rumor.
%    Each round each uninformed node calls one neighbor uniformly and independently at random.
%    Then each called node chooses one uniformly and independently at random one incoming call among all received ones, and sends the rumor via this call.
%\end{def*}
%\begin{lemma}\label{lem:single pull - covariance}
%	The pull calls in this setting are pairwise negatively correlated.
%\end{lemma}
%\begin{proof}
%	Indeed, let the indicator random variable $Y_i=1$ if and only if the $i$-th uninformed node stays uninformed in current round.
%	Then $\Cov[Y_i,Y_j] = \Cov[X_i,X_j]$, where $X_i = 1-Y_i$.
%	Moreover, we will show that $X_i$ are pairwise negatively correlated.
%	Indeed, we compare $\Pr[X_i=X_j=1]$ to $\Pr[X_i=1]\cdot\Pr[X_j=1]$.
%	Let the event $A_i$ be "the outgoing call of the $i$-th node is accepted.
%	Then $\Pr[X_i=1] = (1-\eps)\Pr[A_i]$.
%	Observation~\ref{obs:correlation} can be easily extended to the single call push protocol, so $A_i$ are negatively correlated.
%	Once both calls of $i$-th and $j$-th node are accepted, the probability that they touch different informed nodes is $(1-\eps)\left(1-\eps-\tfrac1n\right)$.
%	Therefore,
%	\[
%		\Pr[X_i=X_j=1] \le (1-\eps)\left(1-\eps-\tfrac1n\right) \Pr[A_i]\Pr[A_j] \le \Pr[X_i=1]\cdot\Pr[X_j=1],
%	\]
%	what proves that $X_i$ are pairwise negatively correlated.
%\end{proof}

%At this moment we described two single incoming protocols: the push-pull protocol having the faster exponential growth regime than the pull protocol which provides a double exponential shrinking regime.
%Let us construct a \emph{fast single incoming call protocol}, which actions as push-pull protocol when there are few informed nodes and as pull protocol when there are few uninformed ones.
%Obviously, nodes cannot know, how many of them are already informed.
%On the other hand, we can join to the rumor its "date of birth", so each informed node will know how many rounds the process lasts and stop pushing if the rumor is too old.

\subsection{Push-Pull Protocol with Transition Time}\label{sec:single call-fast}

Comparing Theorems~\ref{th:single push-pull}~and~\ref{th:single pull} we see that push-pull protocol still be more efficient until $\Theta(n)$ nodes are informed.
Suppose now that we join to the rumor a counter which increases by one each round, so that each informed node knows the ``age'' of the rumor.
Then the single incoming call push-pull protocol \emph{with transition time $R>0$} acts as follows.
While the age of the rumor is at most $R$, it acts as a single incoming call push-pull protocol.
After $R$ rounds of rumor spreading, all informed nodes stop calling simultaneously, so the protocol acts as the single incoming call pull protocol until nodes are informed.

\begin{theorem}
	The expected rumor spreading time of the single incoming call push-pull protocol with the transition time $R = \lceil \log_{3-2/e} n \rceil$ on the complete graph with $n$ vertices is $\log_{3-2/e} n + \log_2\ln n + O(1)$.
\end{theorem}
\begin{proof}
    In the proof of Theorem~\ref{th:single pull} we showed that the single incoming call pull protocol satisfies the double exponential shrinking conditions for all $k \in [gn,n]$ for some $0 < g < 1$.
    Denote by $I_t$ the number of informed nodes after $t$ rounds.
    Let $t := \max\{R,t'\}$, where $t'$ is the smallest time such that $I_{t'} \ge gn$.
    By construction, after round $t$ the transition protocol acts as the pull protocol.
    Therefore,
    \[
        \E[T(1,n)] \le \E[t] + \E[T(fn,n)] \le \E[t] + \log_2\ln n + O(1).
    \]
    It is easy to see that the transition protocol satisfies the conditions of Lemma~\ref{lem:general-connect} with $\ell = fn$, $m = gn$ for any $0 < f < g < 1$.
    Thus, $\E[t] \le \E[T(1,fn)] + O(1)$ for any constant $0 < f < 1$, i.e., it suffices to analyse the spreading time until $fn$ informed nodes.

    Let us consider a single incoming call push-pull protocol.
    In the proof of Theorem~\ref{th:single push-pull} we showed that the single incoming call push-pull protocol satisfies the exponential growth conditions with $\gamma_n = 2-\tfrac2e$.
    In Section~\ref{subsection:exp-growth-upper - phases} we introduced a sequence $k_j$ splitting the interval $[1,fn]$ into phases such that most of the rounds the rumor spreading process moves to exactly the next phase.
    Lemma~\ref{lem:exp-growth-expk-upper} claims that the biggest number of phase $J = \log_{1+\gamma_n}n + O(1)$.
    Since $\gamma_n = 2-\tfrac2e$, we have $J = R + O(1)$.
    To simplify the proof we suppose that $R \le J$ and $fn \le k_R$.In the proof of Theorem~\ref{th:exp-growth-upper} we showed that $T(1,k_R) \le R + \Delta r$, where $\Delta r$ is stochastically dominated by a random variable with distribution $\Geom(1-q)$ for some constant $q<1$.
    By construction, $\Delta r$ is the number of rounds during which the process stayed it the same phase.
    Therefore, after at the end of round $R$ when the protocol switches from push-pull to pull-only, we have $I_R \ge k_{R-\Delta r}$.
    By Lemma~\ref{lem:exp-growth-expk-upper}, we have $k_{R-\Delta r} \ge \tfrac{fn}{(3-2/e)^{\Delta r}}$.

    Consider now the single incoming call pull protocol.
    Let a sequence $k'_j$ defines the phases for the single incoming call pull protocol.
    Suppose that $R'$ is such that $k_{R'} \ge fn$ and that $I_R$ belongs to the phase $i$ of the single incoming call pull protocol.
    Since the single incoming call pull protocol satisfies the exponential growth conditions with $\gamma_n = 1-1/e$, we have $R'-i = \tfrac{3-2/e}{2-1/e} \Delta r + O(1)$.
    Therefore,
    \[
        \E[T(I_R,fn)] \le \E[T(k_i', k'_{R'})] \le \tfrac{3-2/e}{2-1/e} \Delta r + O(1).
    \]
    Summing over all possible values of $\Delta r$ we compute
    \[
        \E[T(1,fn)] \le r + \sum_{s=0}^R \Pr[\Delta r = s] \cdot \left(\tfrac{3-2/e}{2-1/e} s + O(1) \right)
        = R + O(1)
    \]
    Since $\Delta r$ is dominated by a random variable with distribution $\Geom(1-q)$, we have $\E[T(1,fn)] \le R + O(1)$.
    Therefore, $\E[T(1,n)] \le \log_{3-2/e} n + \log_2\ln n + O(1)$.

    To prove the lower bound we consider the following protocol.
    Suppose that any node knows the total number of informed nodes.
    The protocol acts as the single incoming call push-pull protocol until there are at least $fn$ informed nodes for some $0 < f < 1$.
    Then the protocol acts as the single incoming call pull protocol.
    Since we proved Theorems~\ref{th:single push-pull}~and~\ref{th:single pull}, the expected spreading time of such protocol is at least $\log_{3-2/e}n + \log_2\ln n + O(1)$.
    It is also easy to see that such protocol spreads the rumor slightly quicker that the protocol with the fixed transition time, so the expected spreading time is bounded from below by the same expression.
\end{proof}

%\merk{A: I kept the old proof for compare. It is shorter, but seems less formal.}
%\begin{proof}[old proof]
%	Each round the fast single incoming call protocol behaves as one of two protocols studied above.
%	Thus, the double exponential shrinking also guaranteed whatever the rumor is older or younger than $R$ rounds.
%	Our goal is to prove that there exists some $f \in ]0,1[$, such that the expected time until at least $fn$ nodes are informed $\E[\Teg] = R + O(1)$.
%%	The last observation can potentially cause some problems, because for any $f \in ]0,1[$, the probability that after $R$ rounds at least $fn$ nodes are informed is some function of $f$, if $n$ is big enough.
%%	So, we cannot guarantee to at least a constant fraction of informed nodes after $R$ rounds.
%	
%	Let us recall how the exponential growth works.
%	We introduce the monotonically increasing sequence $k_i$ that splits the interval $[0,n]$ into phases.
%	By Lemma~\ref{lem:exponentialk} we know that $k_i$ increases almost exponentially, in particular, $k_R \sim fn$ for some $f \in ]0,1[$.
%	Let us consider the phase-transition process, i.e. the evolution of the phase which the number of informed nodes belongs to.
%	This process is stochastically dominated by the Markov process which either stays in the same phase ("failed" round) or moves to the next phase each round.
%	Let us consider such Markov process until the $R$-th phase is reached (i.e. at least $(f-o(1))n$ nodes are informed).
%	By the proof of Lemma~\ref{prop:UBgrowth}, we know that the process takes $R + \Delta r$ rounds where $\Delta r \sim \Geom(1-q)$ for some constant $q < 1$.
%	On the one hand, if the process took $R + \Delta r$ rounds, then the first $R$ rounds of such Markov process can have at most $\Delta r$ fails at should stop at phase $k_{R-\Delta r} \sim \tfrac{fn}{(3-2/e)^{\Delta r}}$.
%	After the $R$th round the protocol changes its behavior, it acts like a pull protocol.
%	Since this we can redefine the phase boundaries $k_i'$ corresponding to the pull protocol rate, and started from $k_i' = k_{R-\Delta r}$, i.e. where the protocol have been stopped after $R$ rounds.
%	So one can see that the remaining number $t_{\Delta r}$ of rounds until at least $k_R$ nodes are informed is stochastically dominated by $\alpha\cdot\Delta r + \Geom(1-q')$, where $\alpha = \tfrac{3-2/e}{2-1/e} + O\left(\tfrac1n\right)$ and $q'<1$ is some constant.
%	So we can easily compute the expected number time of the exponential growth regime $T_{eg}$.
%	\begin{align*}
%		\E[T_{eg}]
%		& \le R + \sum_{\Delta r} q^{\Delta r} \cdot \E[t_{\Delta r}] \\
%		& = R + \sum_{\Delta r} q^{\Delta r} \cdot (\alpha \cdot \Delta r + O(1)) \\
%		& = R + O(1)
%	\end{align*}
%	The double exponential shrinking is already proved, so the expected rumor spreading time is at most $R + \log_2 \ln n + O(1) = \log_{3-2/e} n + \log_2 \ln n + O(1)$.
%	The lower bound for the spreading time is obvious.
%\end{proof} 

% \bibliography{bibliography}


%\bibliographystyle{alphaabbr}%plain,unsrt
\bibliography{bibrumors}




\end{document}
