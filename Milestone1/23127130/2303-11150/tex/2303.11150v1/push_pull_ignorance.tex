\section{Limited Incoming Calls Capacity} \label{sec:single incoming call}

For all the protocols discussed above the nodes are allowed to be called several times in one round.
For some processes such as protocols considered in Section~\ref{sec:dynamic graphs}, the number of calls received by each node is at most constant.
However in most of rumor spreading processes such number can be unbounded.
For example, consider the basic push-pull protocol from Section~\ref{sec:push-pull} on the complete graph with $n$ vertices.
Since each round all nodes make calls, the maximum number of incoming calls received by the same node in one round is the same as the maximum load of a bin in the well-known problem of throwing uniformly and independently at random $n$ balls into $n$ bills, i.e., $\frac{\log n}{\log\log n} \cdot (1+o(1))$.
Such phenomenon can impact the scalability of the rumor spreading process: typically the time gap between rounds is bounded, but each round with high probability there is at least one node which have to finish $\omega(1)$ transactions.

The simplest solution is to limit the incoming ``capacity'' of nodes, i.e., the number of calls they can reply in one round.
In this section we propose a \emph{single incoming call} setting -- any node can reply to only one incoming call per round chosen uniformly at random among all received calls in current round.
All other calls are considered ``dropped", i.e., they cannot transfer the rumor.
Therefore, each node participates in at most two rumor transactions per round, whatever is the size of the network.

On the other hand, we expect the noticeable slowdown for the protocols based on the single incoming call setting compared to the usual unlimited ``capacity'' setting.
Thus we will show in Section~\ref{sec:single-push-pull} that the single incoming call push-pull protocol satisfies the single exponential shrinking conditions instead of double exponential shrinking and the corresponding expected rumor spreading time is equal to $\log_{3-2/e}n + \tfrac12\ln n \pm O(1)$.
In Section~\ref{sec:single-pull} we argue that since $\Theta(n)$ nodes are informed, the push calls of informed nodes becomes inefficient and they are responsible for such considerable slowdown.
Finally, in Section~\ref{sec:single call-fast} we combine a single incoming call push-pull protocol with pull protocol and provide a not memoryless process with spreading time
$\log_{3-2/e}n + \log_2\ln n + O(1)$.

Before proceeding to the computations, we observe that the following setting is equivalent to the single incoming call model.
In each round we choose uniformly at random a permutation $\sigma \in S_n$.
The element $\sigma_n$ is the \emph{order} of the outgoing call of node $x_i$, we write $ord_i = \sigma_i$.
Each node accepts the call with the lowest order among its received incoming calls.
We call such construction the \emph{ordered calls} setting.
%Before getting started we make some easy observations.
%%First we will reformulate the notation of randomly accepting of a call.
%
%%One can wonder what is the probability that one fixed call is accepted.
%%We show that it is $1-1/e + O(1/n)$.
%Let us enumerate the vertices of the complete graph from $1$ to $n$ and consider the $i$-th node introducing the events $C_i$: ``the $i$-th node has been called in current round" and $A_i$: ``the outgoing call of the $i$-th node is accepted".
%%\begin{itemize}
%%	\item $C_i$, if the $i$-th node has been called in current round;
%%	\item $A_i$, if the outgoing call of the $i$-th node is accepted;
%%\end{itemize}
%
%\begin{observation} \label{obs:called/accepted}
%	$\Pr[C_i] = \Pr[A_i] = 1-(1-1/n)^n$.
%\end{observation}
%\begin{proof}
%	Indeed, $\Pr[C_i] = 1-(1-1/n)^n$ is trivial.
%	To prove the second part we use the equivalence from Observation~\ref{obs:equivalence}.
%	Thus, the outgoing call of the $i$th node can have any number $j$ with probability $\frac1n$.
%	This call is accepted if none of the first $j-1$ calls hits the same node before.
%	Therefore,
%	\[
%		\Pr[A_i] = \sum_{j=1}^n \tfrac1n \left(1-\tfrac1n\right)^{j-1} = 1-\left(1-\tfrac1n\right)^n.
%	\]
%\end{proof}
%
%We note that that the events $C_i$ and $A_j$ are not independent.
%Thus, if one call is accepted, it occupies one spot decreasing the probability of other calls to be accepted.
%%Formally we can show the following.
%
%%By the difference from the independent call model, there is another one source of the dependency: if one of the calls has been accepted, it occupies one spot and makes the other calls less likely to be accepted.
%%Let us hence analyze the correlation of $A_i$ and $C_i$.


\subsection{Single Incoming Call Push-Pull Protocol}\label{sec:single-push-pull}

%\begin{def*}[Independent Call Model]
%    Suppose we also have a clock shared by all nodes counting the rounds since the rumor appeared.
%    Each round each node chooses one of its neighbors in $G$ uniformly at random and communicates with it ("calls it").
%    If $x$ communicates with $y$, one of them knows the rumor and the second don't, then it tells the rumor to the other one.
%\end{def*}

%\begin{def*}[Single incoming call push-pull protocol]
%    Let a graph $G$ given and its vertex $v$ initially knows the rumor.
%    Each round the nodes set up their calls according to the single incoming call model.
%    Within each accepted call the rumor passes from informed to uninformed node.
%\end{def*}
%
%We observe that the following setting is equivalent to the single incoming call model.
%In each round we choose uniformly at random a permutation $\sigma \in S_n$.
%The element $\sigma_n$ is the \emph{order} of the outgoing call of node $x_i$, we write $ord_i = \sigma_i$.
%Each node accepts the call with the lowest order among its received incoming calls.
%We call this construction the \emph{ordered calls model}.

\begin{theorem} \label{th:single push-pull}
	The expected spreading time for the single incoming call push-pull protocol is $\log_{3-2/e} n + \tfrac12\ln n + O(1)$.
\end{theorem}
%\merk{A: Attention! the rate of the exponential shrinking has been changed}

In this section we keep the notation from the previous ones, i.e. $X_i$ is the random indicator variable corresponding to the event ``uninformed node $x_i$ gets informed in considered round''.
Since all considered protocols are uniform, we denote by $p_k$ the probability $\Pr[X_i=1]$ for the round started with $k$ informed nodes and any $i$.
In addition we denote by $Y_i, Z_i$ the indicator random variables for the following events.
\begin{itemize}
    \item [$Y_i$] ``Node $i$ is called and the first incoming call comes from an informed node.''
    \item [$Z_i$] ``The outgoing call of node $i$ is accepted by an informed node.''
\end{itemize}

%\begin{def*}[Single incoming call push-pull protocol]
%    Let a graph $G$ given and its vertex $v$ initially knows the rumor.
%    Each round each node chooses independently and uniformly at random one node and calls it.
%	These choices describe a directed \emph{call graph} $C_t$.
%    Then each called node chooses independently and uniformly at random among its incoming edges (i.e. calls) only one and \emph{accepts} it.
%    Not accepted calls are rejected.
%    The remained structure is called a \emph{communication graph} $G_t$.
%    At the end of the round all neighbors of informed nodes in $G_t$ get the rumor and become informed.
%\end{def*}

\begin{lemma}
    Suppose that the fraction $f$ of nodes is informed.
    Suppose node $i$ is uninformed.
    Then
    \begin{equation}
        p_{fn} = 2f\left(1-\tfrac1e\right) - f^2\left(1-\tfrac1e\right)^2 + f\cdot O\left(\tfrac1n\right) \label{eq:single-prob-informed}.
    \end{equation}
\end{lemma}
\begin{proof}
%    We introduce the following events.
%    \begin{enumerate}[(1)]
%        \item Node $i$ is called and the first incoming call comes from informed node.
%        \item The outgoing call of node $i$ is accepted by informed node.
%    \end{enumerate}
    First, we compute the probabilities of the events corresponding to $Y_i$ and $Z_i$.
    Since each node makes a call in the round, the probability that node $x_i$ is not called is equal to $(1-\tfrac1n)^n$.
    Therefore,
    \[
        \Pr[Y_i=1]
        = f \left(1-\left(1-\tfrac1n\right)^n\right)
        = f \left(1-\tfrac1e\right) + f\cdot O\left(\tfrac1n\right).
    \]

    To compute $\Pr[Z_i=1]$ we will use the ordered call model.
    Suppose that $ord_i = \ell$.
    Then, the outgoing call of node $x_i$ is accepted if all calls with orders less than $\ell$ do not call the same node.
    Since the probability that the outgoing call of node $x_i$ has order $\ell$ is equal to $\tfrac1n$, we compute
    \[
        \Pr[Z_i=1]
        = f \sum_{\ell=1}^n \tfrac1n \left(1-\tfrac1n\right)^{\ell-1}
        = f \left(1-\left(1-\tfrac1n\right)^n\right)
        = f \left(1-\tfrac1e\right) + f\cdot O\left(\tfrac1n\right).
    \]


    Since $X_i = \max\left\{Y_i,Z_i\right\}$, it remains to compute the probability of the event $Y_i=Z_i=1$.
    Suppose that $ord_i = \ell$.
    Since the outgoing call of node $x_i$ is accepted, all calls with order less than $\ell$ should go away from the $x_i$'s target, i.e., they can have only $n-1$ possible targets.
    We also remark that node $x_i$ calls informed node, so it cannot call itself.
    Thus the probability that nobody calls node $x_i$ is equal to $\left(1-\tfrac1{n-1}\right)^{i-1}\left(1-\tfrac1n\right)^{n-i}$.
    Therefore,
    \begin{align*}
        \Pr[Z_i=1|Y_i=1, \; ord_i = \ell]
        & = f\left(1 - \left(1-\tfrac1{n-1}\right)^{i-1} \left(1-\tfrac1n\right)^{n-i}\right) \\
        & = f\left(1 - \left(1-\tfrac1n\right)^n + O\left(\tfrac1n\right)\right).
    \end{align*}
    Since the probability above is independent of $\ell$, we obtain immediately that node
    \begin{align*}
        \Pr[Y_i=Z_i=1]
        %& = 2f\left(1-\left(1-\tfrac1n\right)^n\right) \\
        & = f^2\left(1-\left(1-\tfrac1n\right)^n\right)^2 + f^2 \cdot O\left(\tfrac1n\right) \\
        & = f^2\left(1-\tfrac1e\right)^2 + f^2 \cdot O\left(\tfrac1n\right).
    \end{align*}
    The claim of lemma follows by including-excluding formula.
\end{proof}

\begin{lemma}\label{lem:single-conditional-prob}
    There exists $c \ge 0$ such that for any uninformed nodes $x_i \ne x_j$ we have
    \begin{equation}
        \Pr[X_i=1|X_j=1] \le \Pr[X_i=1] + \tfrac{c}{n}. \label{eq:single-prob-conditional}
    \end{equation}
\end{lemma}
\begin{proof}
    We say that nodes $x_i$ and $x_j$ \emph{interact} if one calls another or if they both call the same node.
    Clearly, $\Pr[x_i, x_j \text{ interact}|X_j=1] = O\left(\tfrac1n\right)$.
    Since we need to bound $\Pr[X_i=1|X_j=1]$ up to $O(\tfrac1n)$, without loss of generality we assume for the rest of the proof that nodes $x_i$ and $x_j$ do not interact.
    We say that a call interacts with a node $x_j$ if its target coincides with $x_j$ or with $x_j$'s target (by convention a call does not interact with it source).
    Denote by $I_j$ the number of calls interacting with node $x_j$ and observe that since $x_i$ and $x_j$ don't interact, no node can interact with both $x_i$ and $x_j$.
    We split the probability $\Pr[X_i=1|X_j=1]$ conditioning on the values of $I_j$ as follows.
    \begin{align*}
        \Pr[X_i=1|X_j=1] = \sum_{k=1}^n \Pr[X_i=1|X_j=1,I_j=k] \cdot \Pr[I_j=k|X_j=1].
    \end{align*}
    Our goal is to study $\Pr[X_i=1|X_j=1,I_j=k]$.
    Since $k$ nodes interact with $x_j$, there are $n-k-1$ independent calls going uniformly to $n-2$ remaining targets (except $x_j$ and $x_j$'s target).
    In addition at least $n(f - \tfrac{k+1}{n})$ of calls are made by informed nodes.
    By these two observations we deduce
    \begin{align*}
        \Pr[Y_i=1|X_j=1,I_j=k]
        & = \left(f - \tfrac{k+1}{n}\right) \left(1-(1-\tfrac1{n-2})^{n-k-1}\right) \\
        & = f \left(1-\left(1-\tfrac1n\right)^n\right) + kO\left(\tfrac1n\right)
        = f \left(1-\tfrac1e\right) + k\cdot O\left(\tfrac1n\right).
    \end{align*}
    By the similar analysis we obtain that
    \begin{align*}
        \Pr[Z_i=1|X_j=1,I_j=k] & = f \left(1-\tfrac1e\right) + k\cdot O\left(\tfrac1n\right); \\
        \Pr[Y_i=Z_i=1|X_j=1,I_j=k] & = f^2 \left(1-\tfrac1e\right)^2 + k\cdot O\left(\tfrac1n\right).
    \end{align*}
    Therefore, $\Pr[X_i=1|X_j=1,I_j=k] = \Pr[X_i=1] + k \cdot O\left(\tfrac1n\right)$.
    Since $\Expect[I_j|X_j=1] = O(1)$, we sum up by $k$ and obtain
    \begin{align*}
        \Pr[X_i=1|X_j=1]
        & = \Pr[X_i=1] + \sum_{k=1}^n kO\left(\tfrac1n\right) \cdot \Pr[I_j=k|X_j=1] \\
        & = \Pr[X_i=1] + O\left(\tfrac1n\right) \Expect[I_j|X_j=1]
        = \Pr[X_i=1] + O\left(\tfrac1n\right).
    \end{align*}
\end{proof}

\begin{proof}[Proof of Theorem~\ref{th:single push-pull}]
    Consider a round started with $k$ informed nodes.
    Substituting $f$ by $k/n$ in~\eqref{eq:single-prob-informed}, we obtain the probability part of the exponential growth conditions.
    \[
        p_k = 2\left(1-\tfrac1e\right)\cdot\tfrac{k}{n} + k^2 \cdot O\left(\tfrac1{n^2}\right).
    \]
    Multiplying~\eqref{eq:single-prob-conditional} by $p_k$ we get the covariance condition.
    Therefore the protocol satisfies the exponential growth conditions with $\gamma_n = 2(1-\tfrac1e)$.

    Denote by $u:=n-k$ the number of uninformed nodes.
    Substituting $f$ by $1-\tfrac{u}{n}$ in~\eqref{eq:single-prob-informed}, we compute
    \[
        \Pr[X_i=0] = 1-\Pr[X_i=1]
        = \tfrac1{e^2} + O\left(\tfrac1n\right).
    \]
    Since the covariance condition follows from Lemma~\ref{lem:single-conditional-prob}, the protocol satisfies the exponential shrinking conditions with ${\rho_n} = 2$.
    Therefore the expected spreading time is equal to $\log_{3-2/e} n + \tfrac12\ln n + O(1)$.
\end{proof}

%\begin{lemma} \label{lem:single-growth}
%	Consider one uninformed node.
%	Let at the beginning of current round there are $k$ informed nodes.
%	Then the probability that this node becomes informed in current round is
%	$2\left(1-\tfrac1e\right) \cdot \tfrac{k}{n} + O\left(\tfrac{k^2}{n^2}\right)$.
%\end{lemma}
%\begin{proof}
%	Indeed, the $i$-th uninformed node can be informed by one of the two ways.
%	In the pull mechanism it calls informed node and its call is accepted, totally with probability $\Pr[A_i]\cdot\tfrac{k}{n}$.
%	To be informed by the push mechanism if it is called and the accepted incoming call comes from informed node; the corresponding probability is $\Pr[C_i]\cdot\tfrac{k}{n}$.
%	The probability that two ways take place in the same round for the same node is at most $O\left(\tfrac{k^2}{n^2}\right)$ because the $i$th node should both call and be called by informed one.
%	So, applying Observation~\ref{obs:called/accepted}, we deduce that
%	\[
%		\Pr[X_i(k)=1] = 2\left(1-\tfrac1e\right) \cdot \tfrac{k}{n} + O\left(\tfrac{k^2}{n^2}\right).
%	\]
%\end{proof}
%
%\begin{lemma} \label{lem:single-shrinking}
%	Consider one uninformed node in round with $u$ uninformed ones.
%	Then the probability that this node stays uninformed in current round is $\tfrac1e + \Ofun$.
%\end{lemma}
%\begin{proof}
%	Suppose, that the current node is called.
%	It stay uninformed only if its accepted incoming call was made by uninformed node, with probability $\fun$ so the contribution of this case is $\Ofun$.
%	
%	By Observation~\ref{obs:called/accepted}, probability that nobody calls current node equals to $\left(1-\frac1n\right)^n = \frac1e + O\left(\frac1n\right)$.
%	Being not called, the current node stays uninformed if it calls an uninformed node with probability $1-\tfrac{u}{n-1}$. So the uncalled node stays uninformed with total probability $1-O\left(\tfrac{u}{n}\right)$.
%	
%	Two cases above are excluding, so we obtain immediately that the total probability of staying uninformed is $\tfrac1e + O\left(\frac{u}{n}\right)$.
%\end{proof}
%%\texttt{\merk{some formalization}
%%\begin{remark*}
%%	The following formal procedure is equivalent to the single incoming call push-pull model.
%%	\begin{itemize}
%%		\item Generate uniformly at random an integer vector $v \in [1;n]^n$.
%%		\item Generate two permutations $\sigma, \tau \in S_n$, also uniformly at random.
%%		\item We say that the node $\sigma_i$ calls $v_i$ with the priority $\tau_i$.
%%	\end{itemize}
%%\end{remark*}
%%Here by a random call $c$ we understand a triplet $(v_c, \sigma_c, \tau_c)$.}
%
%Next we need to prove that the covariance $\Cov[X_i,X_j]$ is small.
%This means that the knowledge of the fate of one node cannot strongly influence another one.
%In order to do so, we will show that almost surely one node "interacts" with only few peers in one round.
%
%\begin{def*}
%	We say that the outgoing call of the $i$th node \emph{interacts} with the $j$th node ($i \ne j$) if one of the following holds:
%	\begin{enumerate} [(i)]
%		\item $i$ calls $j$;
%		\item $i$ and $j$ call the same node.
%	\end{enumerate}
%\end{def*}
%\begin{observation} \label{obs:interact-probability}
%	Consider the $i$th node and a random call $c$, which is not emitted by $i$.
%	The probability that $i$ and $c$ interact is $\tfrac2n + \Odnsq$.
%\end{observation}
%\begin{observation} \label{obs:interact-same probability}
%	Let the call $c$ interacts with the $i$th node, which doesn't call itself in current round.
%	Then the probabilities that $c$ calls to the $i$th node or to the node called by the $i$th one are both equal to $\tfrac12$.
%\end{observation}
%
%We denote by $I_i$ the number of calls interacting in current round with the $i$th node.
%
%\begin{observation} \label{obs:interact-conditioning}
%	$\quad$
%	\begin{enumerate} [(i)]
%		\item $\Pr[I_i=k|I_j=\ell] \le \Pr[I_i = k]$, for any $\ell \ge 2$.
%		\item $\Pr[I_i=k|I_j=\ell] = \Pr[I_i = k] + \Odn$, for any $\ell \in \{0,1\}$.
%	\end{enumerate}
%\end{observation}
%\begin{proof}
%	Indeed, by Observation~\ref{obs:interact-probability}, $\E[I_j] = 2$, so $(i)$ holds for any $\ell \ge 2$.
%	It is easy to see the explicit expression for $\Pr[I_i=k]$:
%	\begin{equation} \notag
%		\Pr[I_i=k] = \binom{n-1}{k} \cdot \left(\tfrac2n\right)^k \cdot \left(1-\tfrac2n\right)^{n-k-1}.
%	\end{equation}
%	With probability $\Odn$ the nodes $i$ and $j$ don't interact, so we can assume that any call interacts with at most one of them.
%	So,
%	\begin{align*} \notag
%		\Pr[I_i=k|I_j=\ell]
%		&= \binom{n-1-\ell}{k} \cdot\left(\tfrac{2}{n-2}\right)^k
%			\cdot \left(1-\tfrac2{n-2}\right)^{n-k-1-\ell} + \Odn \\
%		&= \Pr[I_i=k] + \Odn, \text{  for any } \ell \in \{0,1\}
%	\end{align*}
%\end{proof}
%
%\begin{lemma}
%	If $\Pr[X_i=1|X_j=1] \ge \Pr[X_i=1]$, then their difference is of order $\Odn$.
%\end{lemma}
%\begin{proof}
%	So, $\Pr[X_i=1] = \sum_{k\ge0} \Pr[X_i=1|I_i=k] \cdot \Pr[I_i=k]$.
%	In addition, the probability that one of nodes $i$, $j$ calls itself is $\Odn$, so w.l.o.g. we can suppose that none of them calls itself.
%	
%	The $i$th node can be informed by three independent scenarios and for each one we will compute the contribution to the probability $\Pr[X_i=0|I_i=k]$, supposing that in current round there are $u < n$ uninformed nodes.
%	
%	By Observation~\ref{obs:interact-same probability}, the probability that the $i$th node is not called is $\ftwok$.
%	In this case all $k$ calls go to the target of the $i$'s call so the probability that the outgoing call of the $i$th node is accepted is $\tfrac1{k+1}$, and the probability that the target is informed is $1-\fun$.
%	
%	Consequently, the $i$th node is called with probability $1-\ftwok$.
%	The second scenario is that the $i$th node accepts the call from informed node with corresponding probability $1-\fun$.
%	
%	The last case is that the $i$th node is called, but accepts a call from uninformed node.
%	So, to be informed, the $i$th node must call an informed node (the probability of such event is $1-\fun$).
%	Moreover, this call should be accepted.
%	Let us denote by $\ell \in [0,k-1]$ the number of the "concurrents" of the $i$'s outgoing call.
%	The probability that the $i$'s call is accepted is $\tfrac1{\ell+1}$ and, as the probability to have exactly $\ell$ "concurrents" is $\ftwok \binom{k}{\ell}$, the total probability is the following.
%	
%	\begin{align}
%		\Pr[X_i=1&|I_i=k]
%		= \ftwok \cdot \tfrac1{k+1}\left(1-\fun\right)
%			+ \left(1-\ftwok\right) \cdot \left(1-\fun\right) \notag \\
%		& + \left(1-\ftwok\right) \fun
%				\cdot\sum_{\ell=0}^{k-1}\ftwok\binom{k}{\ell}
%				\cdot\tfrac1{\ell+1}\cdot\left(1-\fun\right) \label{eq:cov-2}
%	\end{align}
%	
%	Now we want to condition the probability in~\eqref{eq:cov-2} to the events $X_j=1$ and $I_j=k'$.
%	Indeed,
%	\begin{equation}\label{eq:cov-4}
%		\Pr[X_i=1|X_j=1] = \sum_{k'} \Pr[X_i=1|X_j=1, \; I_j=k'] \cdot \Pr[I_j=k'|X_j=1].
%	\end{equation}
%	We can also split the probability $\Pr[X_i=1|X_j=1, \; I_j=k']$ by the number $k$ of calls interacting with the $i$th node.
%	Therefore,
%	\begin{align}
%		\Pr&[X_i=1|X_j=1, \; I_j=k'] \notag \\
%		& = \sum_k \Pr[X_i=1|X_j=1, \; I_j=k', \; I_i=k]
%			\cdot \Pr[I_i=k|I_j=k', \; X_j=1].
%	\end{align}
%	Obviously, $\Pr[I_i=k|I_j=k', \; X_j=1] = \Pr[I_i=k|I_j=k']$.
%	Then, applying Observation~\ref{obs:interact-conditioning}, we get
%	\begin{align}
%		\Pr&[X_i=1|X_j=1, \; I_j=k'] \notag \\
%		& \le \sum_k \Pr[X_i=1|X_j=1, \; I_j=k', \; I_i=k]
%			\cdot \Pr[I_i=k] + \Odn. \label{eq:cov-3}
%	\end{align}
%	Comparing the probability $\Pr[X_i=1|X_j=1, \; I_j=k', \; I_i=k]$ with $\Pr[X_i=1|I_i=k]$, one can obtain the similar expression as in~\eqref{eq:cov-2}.
%	The only difference is that knowledge of the "fate" of the $j$th node "occupies" at most two explicit  nodes -- the target of the $j$'s outgoing call and the source from the first $j$'s incoming call.
%	This shrinks the probability of the $i$th node to call uninformed node or to be called by uninformed one from $\fun$ to $\fun + \Odn$.
%	So, $\Pr[X_i=1|X_j=1, \; I_j=k' \; I_i=k] = \Pr[X_i=1|I_i=k] + \Odn$.
%	Or, applying the last result to~\eqref{eq:cov-3}, we obtain that
%	\begin{align*}
%		\Pr& [X_i=1|X_j=1, \; I_j=k'] \\
%		& \le	\sum_{k,k'} \left(\Pr[X_i=1|I_i=k] + \Odn\right) \cdot \Pr[I_i=k] + \Odn
%		& = \Pr[X_i=1] + \Odn.
%	\end{align*}
%	The claim of lemma immediately follows from result above and~\eqref{eq:cov-4}.
%\end{proof}
%
%\begin{corollary} $\quad$
%	\begin{enumerate}
%		\item $\Cov[X_i,X_j] = \Oknn$ in the exponential growth regime;
%		\item $\Cov[X_i,X_j] = \Odn$ in the exponential shrinking regime.
%	\end{enumerate}
%\end{corollary}
%\begin{proof}
%	Indeed, as $X_i$ and $X_j$ are both binary variables,
%	\begin{align*}
%		\Cov[X_i,X_j]
%		& = \Pr[X_i=X_j=1] - \left(\Pr[X_j=1]\right)^2 \\
%		& = \Pr[X_j=1] \left(\Pr[X_i=1|X_j=1] - \Pr[X_i=1]\right).
%	\end{align*}
%	The claim obviously follows from the fact that in the exponential growth regime $\Pr[X_i=1] = O\left(\tfrac{k}{n}\right)$, where $k$ is the number of currently informed nodes.
%\end{proof}
%
%%So, $\Pr[X_j=0] = \sum_k \Pr[X_j=0|I_j=k] \cdot \Pr[I_j=k] + \Odn$.
%%The term $\Odn$ comes from the case if the $j$th node calls itself with probability $\tfrac1n$.
%%
%%Let us found explicitly the probability $\Pr[X_j=0|I_j=k]$, supposing that in current round there are $u < n$ uninformed nodes.
%%The $j$th node can stay uninformed by two independent scenarios and for each one we will compute the contribution to the probability.
%%First, it may call an uninformed node with probability $\fun$.
%%In this case the $j$th node can only be informed receiving the push call from informed node.
%%As there are $k$ calls interacting with $j$th node, the probability that it is not called is $\ftwok$.
%%Or, the $j$th node is called with probability $1-\ftwok$, but with probability $\fun$ the call comes from uninformed node.
%%Therefore, the contribution of the first scenario is
%%$\fun \cdot \left(\left(1-\ftwok\right)\fun + \ftwok\right)$.
%%
%%The second scenario realizes if the $j$th node calls some informed node (with probability $1-\fun$).
%%Suppose that $\ell \le k$ node also call the same node.
%%Then one of them should have the higher priority than the outgoing call of the $j$th node.
%%The corresponding probability is $\tfrac{\ell}{\ell+1}$.
%%If $\ell<k$ then the remaining $k-\ell$ calls hit the $j$th node and the accepted one should be made by an uninformed node with probability $\fun$.
%%Summing over $\ell$ we obtain the contribution of the second scenario.
%%Therefore,
%%\begin{align}
%%	Pr[X_j=0|I_j=k] & = \fun \cdot \left[\left(1-\ftwok\right)\fun + \ftwok\right] \notag\\
%%	& + \left(1-\fun\right) \cdot \left[\sum_\ell\binom{k}{\ell}\cdot\ftwok\cdot\tfrac{\ell}{\ell+1}\cdot\fun + O\left(\ftwok\right)\right]. \label{eq:cov-1}
%%\end{align}
%%
%%\begin{lemma}
%%	$\Cov[X_i,X_j]$ is small.
%%\end{lemma}
%%\begin{proof}
%%	As $X_i$ and $X_j$ are both binary variables,
%%	$$\Cov[X_i,X_j] = \Pr[X_i=X_j=0] - \left(\Pr[X_j=0]\right)^2.$$
%%	We can already compute the probability that $X_j=0$.
%%	So let us compare it to the joint probability.
%%	Indeed,
%%	\begin{align*}
%%		\Pr[X_i=&X_j=0] = \sum_\ell \Pr[X_i=X_j=0|I_j=\ell] \cdot \Pr[I_j=\ell] \\
%%		& = \sum_\ell \Pr[X_i=0|X_j=0 \AND I_j=\ell] \cdot \Pr[X_j=0|I_j=\ell] \cdot \Pr[I_j=\ell].
%%	\end{align*}
%%	We can apply the same trick to $\Pr[X_i=0|X_j=0 \AND I_j=\ell]$:
%%	\begin{align*}
%%		\Pr[X_i=0|X_j=0\AND I_j=\ell]
%%		= & \sum_k \Pr[X_i=0|X_j=0\AND I_j=\ell \AND I_i=k] \\
%%		& \times \Pr[I_i=k|X_j=0\AND I_j=\ell].
%%	\end{align*}
%%	It is easy to see that
%%	\[
%%		\Pr[I_i=k|X_j=0\AND I_j=\ell] = \Pr[I_i=k|I_j=\ell]
%%		= O\left(\left(1-\tfrac{\ell}{n}\right)^\ell \cdot \tfrac{2^k}{k!}\right).
%%	\]
%%	Comparing the probability $\Pr[X_i=0|X_j=0\AND I_j=\ell \AND I_i=k]$ with $\Pr[X_i=0|I_i=k]$, one can obtain the similar expression as in~\eqref{eq:cov-1}.
%%	The only two differences are that any probabilities:
%%	\begin{itemize}
%%		\item we suppose that with probability $\Odn$ the $i$th and $j$th nodes don't interact in current round;
%%		\item the probability $\fun$ of call uninformed node or be called by uninformed one transforms into $\fun + \Odn$, as some uninformed nodes can be "occupied" by the $j$th one.
%%	\end{itemize}
%%	Therefore, $\Pr[X_i=0|X_j=0\AND I_j=\ell \AND I_i=k] = \Pr[X_i=0|I_i=k] + \Odn$.
%%	So, $\Cov[X_i=X_j=0] \le \Odn$, if $u \ge gn$.
%%\end{proof}
%%\begin{remark}
%%	It seems that the last proof can be easily generalized to the case of exponential growth\ldots
%%\end{remark}


\subsection{Single Incoming Call Pull-Only Protocol}\label{sec:single-pull}
We showed that the the single call push-pull protocol is significantly slower than the classic push-pull protocol.
Although protocol based on the single incoming call setting cannot be faster than the classic independent call model, we can make it noticeably faster using the following trick.
Let us consider one round of the exponential shrinking phase with $u$ uninformed nodes.
In such round there are $n-u$ push calls, each one hits uninformed node with small probability $\fun$.
On the other hand, each of $u$ pull calls touches some informed node with probability $1-\fun$.
One can conclude that push calls ``spam'' the network: they ``occupy'' other informed nodes making them inaccessible for pull calls of uninformed nodes.
This observation is verified in the following theorem.
%Indeed, let us introduce the single incoming call pull protocol.

%\begin{def*}[Single incoming call pull protocol]
%    Let a graph $G$ given and its vertex $v$ initially knows the rumor.
%    Each round only uninformed nodes make their calls.
%    To get informed, an uninformed node must call informed node and its call should be accepted according to the single incoming call model.
%\end{def*}

\begin{theorem} \label{th:single pull}
	The spreading time for the single incoming call pull protocol is $\log_{2-1/e} n + \log_2\ln n + O(1)$.
\end{theorem}
\begin{proof}
	Consider one round of the protocol.
    Clearly, if $x_1$ becomes informed it ``occupies'' one informed node which cannot inform any other node in current round.
    Thus, if we condition on that $X_1=1$, then it is slightly less likely that $x_2$ becomes informed. Consequently, $\Cov[X_1,X_2] < 0$ and the covariance part of the exponential growth and double exponential shrinking conditions is satisfied.

    Again, the call with order $\ell$ is accepted with probability $\left(1-\tfrac1n\right)^{\ell-1}$.
	Since in the round started with $k$ informed nodes only $n-k$ nodes perform calls, $ord_i$ is uniformly distributed in $\{1,\ldots,n-k\}$.
    Since the probability to call an informed node is $\tfrac{k}{n}$, we compute
    \begin{equation}
        p_k = \tfrac{k}{n}\sum_{\ell=1}^{n-k} \tfrac1{n-k}\left(1-\tfrac1n\right)^{\ell-1}
        = \tfrac{k}{n-k}\left(1-\left(1-\tfrac1n\right)^{n-k}\right) \label{eq:single-pull-prob}.
    \end{equation}
    By Corollary~\ref{cor:prelim:(1-1/n)^(n-u)}, we have
    \[
        \left(1-\tfrac1e\right)\tfrac{k}{n} - 4\tfrac{k^2}{n^2}
        \le p_k
        \le \left(1-\tfrac1e\right)\tfrac{k}{n} + 2\left(1-\tfrac1e\right)\tfrac{k^2}{n^2}.
    \]
    So the protocol satisfies the exponential growth conditions with parameter $\gamma_n = 1-\tfrac1e$.

    If we denote by $u$ the number of uninformed nodes, from~\eqref{eq:single-pull-prob} follows the following expression.
    \[
        1-p_{n-u} = \tfrac{n-u}{u} \left(1-\left(1-\tfrac1n\right)^u\right).
    \]
    With Lemma~\ref{lem:prelim:(1-1/n)^k}, we estimate $\tfrac{u}{n} \le 1-p_{n-u} \le \tfrac{3u}{2n}$.
    The protocol hence satisfies the double exponential shrinking conditions with $\ell=2$.

    Therefore, the expected spreading time is equal to $\log_{2-1/e} n + \log_2\ln n + O(1)$.
\end{proof}

%\begin{lemma}
%	The probability that a uninformed node stays uninformed in one round is $\tfrac32\eps + \Oepsq$.
%%	The single incoming call pull protocol satisfies the double exponential shrinking conditions with parameter $\ell = 2$.
%\end{lemma}
%\begin{proof}
%	Let us consider one round of the single incoming call pull protocol with $\eps n$ uninformed nodes.
%	In current round only $\eps n$ nodes will make calls.
%	By the same idea as in the proof of Observation~\ref{obs:called/accepted}, the probability that one call is accepted is
%	\begin{align*}
%		\sum_{i=1}^{\eps n} & \tfrac1{\eps n} \cdot \left(1-\tfrac1n\right)^{i-1}
%			= \tfrac1{\eps n} \tfrac{1-\left(1-\tfrac1n\right)^{\eps n}}{1-\left(1-\tfrac1n\right)} \\
%		& = \tfrac1\eps \left( 1 - \left(1-\tfrac1n\right)^{\eps n} \right)
%			= 1 - \tfrac\eps2 + O\left(\eps^2+\tfrac{\eps}{n} \right)
%	\end{align*}
%	The node becomes informed if and only if it calls informed node with probability $1-\eps$ and its call is accepted.
%	Therefore the probability that this node stays uninformed in current round is $\tfrac32\eps + O\left(\eps^2\right)$.
%	
%	The covariance condition is proved in the lemma above, so the single incoming call pull protocol satisfies the double exponential shrinking conditions.
%\end{proof}

%Although such slowdown is the price of dealing with only one incoming call per node per round and seems to be unavoidable, we would like to design the single incoming call process having the double exponential shrinking regime instead of exponential shrinking one.
%The reason of the loss of rapidity in the shrinking regime of the single call push-pull protocol compared to its classical version raises from the pull calls.
%Indeed, when the most of the nodes are informed, the most of the calls are push ones, but their efficiency decreases and they only spam the network.
%So let us study the single incoming call version of the classic pull protocol, which shouldn't be subject to such problem.

%To understand this we recall that the probability of a single call to be accepted is a constant in $]0,1[$ which doesn't depend on the number of current round.
%Therefore the probability of one node to stay uninformed in current round is non-negligible for any number of uninformed nodes.
%On the other hand, we know that the pull protocol is much more efficient, as it performs a double exponential shrinking regime.
%Therefore is looks reasonable to consider the \emph{single incoming call pull protocol}, i.e. the protocol acting on the single incoming call model with the constraint that only uninformed nodes are allowed to make calls.

%\begin{def*}[Single incoming call push-pull protocol]
%    Let a graph $G$ given and its vertex $v$ initially knows the rumor.
%    Each round each uninformed node calls one neighbor uniformly and independently at random.
%    Then each called node chooses one uniformly and independently at random one incoming call among all received ones, and sends the rumor via this call.
%\end{def*}
%\begin{lemma}\label{lem:single pull - covariance}
%	The pull calls in this setting are pairwise negatively correlated.
%\end{lemma}
%\begin{proof}
%	Indeed, let the indicator random variable $Y_i=1$ if and only if the $i$-th uninformed node stays uninformed in current round.
%	Then $\Cov[Y_i,Y_j] = \Cov[X_i,X_j]$, where $X_i = 1-Y_i$.
%	Moreover, we will show that $X_i$ are pairwise negatively correlated.
%	Indeed, we compare $\Pr[X_i=X_j=1]$ to $\Pr[X_i=1]\cdot\Pr[X_j=1]$.
%	Let the event $A_i$ be "the outgoing call of the $i$-th node is accepted.
%	Then $\Pr[X_i=1] = (1-\eps)\Pr[A_i]$.
%	Observation~\ref{obs:correlation} can be easily extended to the single call push protocol, so $A_i$ are negatively correlated.
%	Once both calls of $i$-th and $j$-th node are accepted, the probability that they touch different informed nodes is $(1-\eps)\left(1-\eps-\tfrac1n\right)$.
%	Therefore,
%	\[
%		\Pr[X_i=X_j=1] \le (1-\eps)\left(1-\eps-\tfrac1n\right) \Pr[A_i]\Pr[A_j] \le \Pr[X_i=1]\cdot\Pr[X_j=1],
%	\]
%	what proves that $X_i$ are pairwise negatively correlated.
%\end{proof}

%At this moment we described two single incoming protocols: the push-pull protocol having the faster exponential growth regime than the pull protocol which provides a double exponential shrinking regime.
%Let us construct a \emph{fast single incoming call protocol}, which actions as push-pull protocol when there are few informed nodes and as pull protocol when there are few uninformed ones.
%Obviously, nodes cannot know, how many of them are already informed.
%On the other hand, we can join to the rumor its "date of birth", so each informed node will know how many rounds the process lasts and stop pushing if the rumor is too old.

\subsection{Push-Pull Protocol with Transition Time}\label{sec:single call-fast}

Comparing Theorems~\ref{th:single push-pull}~and~\ref{th:single pull} we see that push-pull protocol still be more efficient until $\Theta(n)$ nodes are informed.
Suppose now that we join to the rumor a counter which increases by one each round, so that each informed node knows the ``age'' of the rumor.
Then the single incoming call push-pull protocol \emph{with transition time $R>0$} acts as follows.
While the age of the rumor is at most $R$, it acts as a single incoming call push-pull protocol.
After $R$ rounds of rumor spreading, all informed nodes stop calling simultaneously, so the protocol acts as the single incoming call pull protocol until nodes are informed.

\begin{theorem}
	The expected rumor spreading time of the single incoming call push-pull protocol with the transition time $R = \lceil \log_{3-2/e} n \rceil$ on the complete graph with $n$ vertices is $\log_{3-2/e} n + \log_2\ln n + O(1)$.
\end{theorem}
\begin{proof}
    In the proof of Theorem~\ref{th:single pull} we showed that the single incoming call pull protocol satisfies the double exponential shrinking conditions for all $k \in [gn,n]$ for some $0 < g < 1$.
    Denote by $I_t$ the number of informed nodes after $t$ rounds.
    Let $t := \max\{R,t'\}$, where $t'$ is the smallest time such that $I_{t'} \ge gn$.
    By construction, after round $t$ the transition protocol acts as the pull protocol.
    Therefore,
    \[
        \E[T(1,n)] \le \E[t] + \E[T(fn,n)] \le \E[t] + \log_2\ln n + O(1).
    \]
    It is easy to see that the transition protocol satisfies the conditions of Lemma~\ref{lem:general-connect} with $\ell = fn$, $m = gn$ for any $0 < f < g < 1$.
    Thus, $\E[t] \le \E[T(1,fn)] + O(1)$ for any constant $0 < f < 1$, i.e., it suffices to analyse the spreading time until $fn$ informed nodes.

    Let us consider a single incoming call push-pull protocol.
    In the proof of Theorem~\ref{th:single push-pull} we showed that the single incoming call push-pull protocol satisfies the exponential growth conditions with $\gamma_n = 2-\tfrac2e$.
    In Section~\ref{subsection:exp-growth-upper - phases} we introduced a sequence $k_j$ splitting the interval $[1,fn]$ into phases such that most of the rounds the rumor spreading process moves to exactly the next phase.
    Lemma~\ref{lem:exp-growth-expk-upper} claims that the biggest number of phase $J = \log_{1+\gamma_n}n + O(1)$.
    Since $\gamma_n = 2-\tfrac2e$, we have $J = R + O(1)$.
    To simplify the proof we suppose that $R \le J$ and $fn \le k_R$.In the proof of Theorem~\ref{th:exp-growth-upper} we showed that $T(1,k_R) \le R + \Delta r$, where $\Delta r$ is stochastically dominated by a random variable with distribution $\Geom(1-q)$ for some constant $q<1$.
    By construction, $\Delta r$ is the number of rounds during which the process stayed it the same phase.
    Therefore, after at the end of round $R$ when the protocol switches from push-pull to pull-only, we have $I_R \ge k_{R-\Delta r}$.
    By Lemma~\ref{lem:exp-growth-expk-upper}, we have $k_{R-\Delta r} \ge \tfrac{fn}{(3-2/e)^{\Delta r}}$.

    Consider now the single incoming call pull protocol.
    Let a sequence $k'_j$ defines the phases for the single incoming call pull protocol.
    Suppose that $R'$ is such that $k_{R'} \ge fn$ and that $I_R$ belongs to the phase $i$ of the single incoming call pull protocol.
    Since the single incoming call pull protocol satisfies the exponential growth conditions with $\gamma_n = 1-1/e$, we have $R'-i = \tfrac{3-2/e}{2-1/e} \Delta r + O(1)$.
    Therefore,
    \[
        \E[T(I_R,fn)] \le \E[T(k_i', k'_{R'})] \le \tfrac{3-2/e}{2-1/e} \Delta r + O(1).
    \]
    Summing over all possible values of $\Delta r$ we compute
    \[
        \E[T(1,fn)] \le r + \sum_{s=0}^R \Pr[\Delta r = s] \cdot \left(\tfrac{3-2/e}{2-1/e} s + O(1) \right)
        = R + O(1)
    \]
    Since $\Delta r$ is dominated by a random variable with distribution $\Geom(1-q)$, we have $\E[T(1,fn)] \le R + O(1)$.
    Therefore, $\E[T(1,n)] \le \log_{3-2/e} n + \log_2\ln n + O(1)$.

    To prove the lower bound we consider the following protocol.
    Suppose that any node knows the total number of informed nodes.
    The protocol acts as the single incoming call push-pull protocol until there are at least $fn$ informed nodes for some $0 < f < 1$.
    Then the protocol acts as the single incoming call pull protocol.
    Since we proved Theorems~\ref{th:single push-pull}~and~\ref{th:single pull}, the expected spreading time of such protocol is at least $\log_{3-2/e}n + \log_2\ln n + O(1)$.
    It is also easy to see that such protocol spreads the rumor slightly quicker that the protocol with the fixed transition time, so the expected spreading time is bounded from below by the same expression.
\end{proof}

%\merk{A: I kept the old proof for compare. It is shorter, but seems less formal.}
%\begin{proof}[old proof]
%	Each round the fast single incoming call protocol behaves as one of two protocols studied above.
%	Thus, the double exponential shrinking also guaranteed whatever the rumor is older or younger than $R$ rounds.
%	Our goal is to prove that there exists some $f \in ]0,1[$, such that the expected time until at least $fn$ nodes are informed $\E[\Teg] = R + O(1)$.
%%	The last observation can potentially cause some problems, because for any $f \in ]0,1[$, the probability that after $R$ rounds at least $fn$ nodes are informed is some function of $f$, if $n$ is big enough.
%%	So, we cannot guarantee to at least a constant fraction of informed nodes after $R$ rounds.
%	
%	Let us recall how the exponential growth works.
%	We introduce the monotonically increasing sequence $k_i$ that splits the interval $[0,n]$ into phases.
%	By Lemma~\ref{lem:exponentialk} we know that $k_i$ increases almost exponentially, in particular, $k_R \sim fn$ for some $f \in ]0,1[$.
%	Let us consider the phase-transition process, i.e. the evolution of the phase which the number of informed nodes belongs to.
%	This process is stochastically dominated by the Markov process which either stays in the same phase ("failed" round) or moves to the next phase each round.
%	Let us consider such Markov process until the $R$-th phase is reached (i.e. at least $(f-o(1))n$ nodes are informed).
%	By the proof of Lemma~\ref{prop:UBgrowth}, we know that the process takes $R + \Delta r$ rounds where $\Delta r \sim \Geom(1-q)$ for some constant $q < 1$.
%	On the one hand, if the process took $R + \Delta r$ rounds, then the first $R$ rounds of such Markov process can have at most $\Delta r$ fails at should stop at phase $k_{R-\Delta r} \sim \tfrac{fn}{(3-2/e)^{\Delta r}}$.
%	After the $R$th round the protocol changes its behavior, it acts like a pull protocol.
%	Since this we can redefine the phase boundaries $k_i'$ corresponding to the pull protocol rate, and started from $k_i' = k_{R-\Delta r}$, i.e. where the protocol have been stopped after $R$ rounds.
%	So one can see that the remaining number $t_{\Delta r}$ of rounds until at least $k_R$ nodes are informed is stochastically dominated by $\alpha\cdot\Delta r + \Geom(1-q')$, where $\alpha = \tfrac{3-2/e}{2-1/e} + O\left(\tfrac1n\right)$ and $q'<1$ is some constant.
%	So we can easily compute the expected number time of the exponential growth regime $T_{eg}$.
%	\begin{align*}
%		\E[T_{eg}]
%		& \le R + \sum_{\Delta r} q^{\Delta r} \cdot \E[t_{\Delta r}] \\
%		& = R + \sum_{\Delta r} q^{\Delta r} \cdot (\alpha \cdot \Delta r + O(1)) \\
%		& = R + O(1)
%	\end{align*}
%	The double exponential shrinking is already proved, so the expected rumor spreading time is at most $R + \log_2 \ln n + O(1) = \log_{3-2/e} n + \log_2 \ln n + O(1)$.
%	The lower bound for the spreading time is obvious.
%\end{proof} 