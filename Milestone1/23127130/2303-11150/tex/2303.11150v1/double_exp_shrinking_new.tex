%%%%                 SUBSECTION
%%%%       DOUBLE EXPONENTIAL SHRINKING REGIME
%%%%
\subsection{Double Exponential Shrinking Regime. Upper Bound.}

In the following two sections we consider the regime in which uninformed nodes remain uninformed with probability proportional to the fraction uninformed nodes, or, more generally, some positive power $\ell-1$ there of.
Such a regime often occurs in protocols using pull operations.
We show that the \emph{fraction} of uninformed nodes is raised to the $\ell$-th power each round and that such a regime informs the last $gn$ nodes ($g$ a small constant) in a double logarithmic number of rounds.

We discuss the upper bound on the runtime first.
Throughout this section, we assume that our homogeneous epidemic protocol satisfies the following \emph{upper double exponential shrinking conditions} including the covariance condition.
%\merk{here are the key differences of this section from the exponential growth and shrinking}
%\begin{itemize}
%    \item
%        The main parameter here is $\eps := \tfrac{u}{n}$ - the fraction of informed nodes.
%        In particular the using of $Y(u)$ from previous section creates an ambiguity in the notation: $Y(\eps n)$ and $E(\eps)$.
%    \item
%        Some of steps are too trivial to formulate the corresponding lemmas, e.g. $\eps_{j+1} = a^{\frac{\ell^j-1}{\ell-1}} \eps_0^{\ell^j}$.\item
%        We use very loose bound for the variance.
%    \item
%        The same for the error probability $q$ which now does not depend on $\eps$.
%    \item
%        We do need the fast finishing conditions, because the standard phase construction is applicable until there are at least $\sqrt{n}$ (\merk{see if it the right bound}) uninformed nodes.
%\end{itemize}

%%%DEF - The double exponential shrinking conditions
\begin{defn}[upper double exponential shrinking conditions] \label{def:dexp}
	Let $g, \alpha \in [0,1]$, $\ell > 1$, and $a, c \in \R_{\ge0}$ such that $ag^{\ell-1} < 1$.
	We say that a homogeneous epidemic protocol satisfies \emph{the upper double exponential shrinking conditions}
	if for any $n$ big enough, the following properties are satisfied for all $u \in [n^{1-\alpha}, gn]$.
	%$\eps \in [n^{-\alpha},g]$ such that $\eps n \in \N$.
	\renewcommand{\theenumi}{(\roman{enumi})}%
	\begin{enumerate}
		\item $1-p_{n-u} \le a\left(\tfrac{u}{n}\right)^{\ell-1}$.
		\item $c_{n-u} \le c \tfrac{n}{u^2}$.
	\end{enumerate}
\end{defn}

Similarly to the exponential shrinking regime we argue with the number $u$ of uninformed nodes rather than the number $k$ of informed ones.
To ease the notation in the double exponential shrinking regime we use the \emph{fraction} $\eps := \tfrac{u}{n}$ of uninformed nodes instead of the absolute number $u$.
Thus, the double exponential shrinking conditions turns into the following bounds, valid for all $\eps \in [n^{-\alpha},g]$ with $\eps n \in \N$.
\renewcommand{\theenumi}{(\roman{enumi})}%
\begin{enumerate}
	\item $1-p_{n(1-\eps)} \le a\eps^{\ell-1}$.
	\item $c_{n(1-\eps)} \le \eps^{-2}\tfrac{c}{n}$.
\end{enumerate}

In the definition above, we cover the rounds starting with a number of uninformed nodes between $n^{1-\alpha}$ and $gn$. While, by taking $\alpha=1$ this would allow to analyze the process until all nodes are informed, it turns out that the crucial part is reduce the number of uninformed nodes from $\Theta(n)$ to $n^{1-\alpha}$ for an arbitrarily small constant $\alpha$. For $u \in [1,n^{1-\alpha}]$, the double exponential shrinking conditions can be relaxed: the covariance condition is no longer needed and it is sufficient to bound uniformly the probability of a node to stay uninformed by $n^{-\tau}$, for some $\tau < 1$.

The main result of the section is the following theorem.

%For the remainder of the process, the following notion will often be more convenient.
%
%\merk{Seems that we can drop the part with fast finishing. Here is the old construction.}
%
%We say that the protocol satisfies the \emph{fast finishing conditions} from $n^{-\alpha}$ on (for some $0<\alpha<1$) if there exists a constant $\tau > 0$ such that for any $\eps \le n^{-\alpha}$, we have $1-p_{n(1-\eps)} \le n^{-\tau}$. The following observation follows right from the definitions. It allows to freely increase the switching point between the adjacent exponential shrinking and fast finishing regimes.
%
%\begin{observation}\label{obs:double-exp-finishing}
%  If a homogeneous protocol satisfies the upper double exponential shrinking conditions in $[n^{-\alpha},g]$ and the fast finishing conditions from $n^{-\alpha}$ on, then it also satisfies the fast finishing conditions from $n^{-\beta}$ on for any constant $0 < \beta \le \alpha$.
%\end{observation}
%
%Once we have reached a regime with fast finishing conditions, we only need an expected constant number of rounds to get all nodes informed.
%
%\begin{lemma}\label{lem:double-exp-finishing}
%	Let the protocol satisfy the fast finishing conditions from $n^{-\alpha}$ on.
%	Then the expected spreading time $\E[T(n-u_0,n)]$ is $O(1)$ for any $u_0 \le n^{1-\alpha}$.
%\end{lemma}
%\begin{proof}
%	Consider one of the $u_0 \le n^{1-\alpha}$ initially uninformed nodes.
%	For any $R\ge1$, this node remains uninformed for $R$ rounds with probability at most $n^{-\tau R}$.
%	By the union bound, the probability that for $R$ rounds at least one node remains uninformed is at most $P_R := \min\{1,n^{-\tau R + 1-\alpha}\}$.
%	Therefore, $\E[T(n-u_0,n)] \le 1+\sum_{R\ge1} P_R = O(1)$.
%\end{proof}
%\merk{end of fast finishing}

%The main result of this section is that the double exponential shrinking conditions (possibly with the fast finishing conditions in the end) give a sharp runtime of $\log_\ell \ln n + O(1)$.

\begin{theorem}\label{th:double-exp-shrinking-upper}
	Consider a homogeneous epidemic protocol satisfying the upper double exponential shrinking conditions in $[n^{-\alpha},g]$.
	Suppose further that there exists $\tau > 0$ such that $1-p_{n-u} \le n^{-\tau}$ for all $u \le n^{1-\alpha}$.
	
	Then there exist constant $A', \alpha' > 0$ such that
	\begin{eqnarray*}
		&& \Expect[T(\lceil (1-g)n \rceil,n)] \le \log_\ell \ln n + O(1), \\
		&& \Pr[T(\lceil (1-g)n \rceil,n) \ge \log_\ell \ln n + r] \le O(n^{-\alpha' r+A'})\, \mbox{ for all $r \in \N$}.
	\end{eqnarray*}
\end{theorem}

%\begin{theorem}\label{theorem:double-exp-shrinking-upper}
%	Consider a homogeneous epidemic protocol disseminating a rumor on $n$ nodes.
%	If the upper double exponential shrinking conditions are satisfied in $[n^{-\alpha},g]$ together with the fast finishing condition from $n^{-\alpha}$ on, and if $g < (2\bar a)^{1/(\ell-1)}$, \merk{B: added this} then
%	\begin{equation}
%		\Expect[T(\lceil (1-g)n \rceil,n)] \le \log_\ell \ln n + O(1) \notag.
%    \end{equation}
%\end{theorem}
%\begin{remark}
%	In reality, we can slightly relax the double exponential shrinking condition: if for any $\eps \in [n^{-\alpha},g]$, (i) and (ii) are satisfied and in addition, for any $\eps < n^{-\alpha}$, $1-p_{n(1-\eps)} \le n^{-\tau}$, for some $\tau > 0$ and $0 < \alpha < 1$, then we have the same upper bound on the runtime of the protocol.
%	\begin{equation}
%		\Expect[T(\lceil (1-g)n \rceil,n)] \le \log_\ell \ln n + O(1) \notag.
%	\end{equation}
%\end{remark}

\subsubsection{Round Targets and Failure Probabilities}

%\merk{B: hier koennte man formal korrekt erst $E(\eps)$ fuer alle $\eps$ definieren und dann $y(\eps)$ nur fuer die $\eps$ mit $\eps n$ ganzzahlig}
Let the random variable $y(\eps)$ denote to the fraction of uninformed nodes at the end of a round started with $\eps n$ uninformed ones.
%Since $\Expect[y(\eps)] \ge \eps n (1-p_{n(1-\eps)})$,
The double exponential shrinking conditions state that
\begin{equation}
    \Expect[y(\eps)] \le E(\eps) := a\eps^\ell \notag.
\end{equation}

\begin{lemma}\label{lem:double-exp-shrinking-variance}
	$\Var[y(\eps)] \le \tfrac{1+c}{n}$.
\end{lemma}
\begin{proof}
	Indeed, $\Var[y(\eps)] = \tfrac1{n^2}\Var[Y(\eps)]$, where $Y(\eps) := ny(\eps)$ is the number of uninformed nodes at the end of the round.
	By Lemma~\ref{lem:prelim:variance},
	\begin{equation}
		\Var[Y(\eps)] \le \Expect[Y(\eps)] + (n\eps)^2 c_{n(1-\eps)}
		\le n + cn\notag.
	\end{equation}
\end{proof}

The next lemma states that with good probability, $y(\eps)$ is less than the \emph{target value} $2E(\eps)$.

%\merk{A: "for any fraction of uninformed nodes" allows not to mention that $\eps n \in \N$.}
\begin{lemma}\label{lem:double-exp-shrinking-failure-upper}
    For any fraction of uninformed nodes $\eps \in [n^{-\alpha}, g]$,
    \begin{equation}
        \Pr[y(\eps) \ge 2E(\eps)] \le q := \tfrac{(1+c)}{a^2}n^{2\alpha\ell-1} \notag.
    \end{equation}
\end{lemma}
\begin{proof}
    Applying Chebyshev's inequality and taking into account that $E(\eps) \ge \Expect[y(\eps)]$, we compute
    \begin{equation}
        \Pr[y(\eps) \ge 2E(\eps)]
        \le \Pr[y(\eps) \ge \Expect[y(\eps)] + E(\eps)]
        \le \tfrac{\Var[y(\eps)]}{E(\eps)^2} \notag.
    \end{equation}
    By Lemma~\ref{lem:double-exp-shrinking-variance} and since $\eps \ge n^{-\alpha}$,
    \begin{equation}
    	\Pr[y(\eps) \ge 2E(\eps)]
    	\le \tfrac{1+c}{n} \cdot \tfrac1{(a\eps^\ell)^2}
    	\le \tfrac{1+c}{a^2} n^{2\alpha\ell-1}\notag.
    \end{equation}
%    Since $\Expect[y(\eps)] \le 1$ and $c_{n(1-\eps)} \le \eps^{-2}c/n$,
%    we have
%    \begin{equation}
%        \Var[y(\eps)] \le \Expect[y(\eps)] + (\eps n)^2 c_{n(1-\eps)} \le 1 + c \notag.
%    \end{equation}
%    Since, $E(\eps) = a\eps^\ell$, the claim of lemma follows directly.
\end{proof}
Our choice to analyze the double exponential shrinking regime only up to $n^{1-\alpha}$ uninformed nodes allows us to define $q$ independent of $\eps$.
Since the double exponential shrinking conditions imply the second assumption of Theorem~\ref{th:double-exp-shrinking-upper}, without loss of generality we may assume that $\alpha< \tfrac1{2\ell}$, and that consequently $q=n^{-\Theta(1)}$.

%\merk{A: do we need later $q \cdot \log_\ell \ln n < 1$?}

%%%  PHASE CALCULUS
\subsubsection{The Phase Calculus}

Let us define the sequence $\eps_j$ recursively by
\begin{equation}
    \eps_0 := g, \quad \eps_{j+1} := 2E(\eps_j) \notag.
\end{equation}
The following observation can be obtained by a simple induction.
\begin{observation}\label{obs:double-exp-shrinking-epsj}
    For all $j \ge 0$, $\eps_j = (2a)^{\frac{\ell^j-1}{\ell-1}} g^{\ell^j}$.
    In particular, the $\eps_j$ form a decreasing sequence if $g < (2a)^{-\tfrac1{\ell-1}}$.
\end{observation}
In the following we assume that $g$ is small enough to ensure that the $\eps_j$ decrease.
Applying logarithm twice to the previous equation one can also see the following.
%\begin{observation} \label{obs:double-exp-shrinking-nphases}
%	There exists $J = \log_\ell \ln n + O(1)$ such that $n^{-\alpha} \in [\eps_J, \eps_{J-1}[$.
%\end{observation}
%\merk{todo: discuss $\to n$ should be large enough}
\begin{corollary} \label{cor:double-exp-shrinking-nphases}
There exists $J = \log_\ell \ln n + O(1)$ such that for any $n$ big enough
\begin{equation}
    n^{-\alpha} < \eps_J \le \left(\tfrac{n^{-\alpha}}{2a}\right)^{1/\ell} \notag.
\end{equation}
%	There exists $J = \log_\ell \ln n + O(1)$ and $0 < \alpha' < \alpha$ such that $\eps_J \in ]n^{-\alpha}, n^{-\alpha'}]$.
\end{corollary}
\begin{proof}
	From Observation~\ref{obs:double-exp-shrinking-epsj} we see that the biggest $J$ such that $\eps_J > n^{-\alpha}$ is equal to $\log_\ell \ln n + O(1)$.
	Since $\eps_{J+1} < n^{-\alpha}$, we have $\eps_J < \left(\tfrac{n^{-\alpha}}{2a}\right)^{1/\ell}$.
%	Hence for any $0 < \alpha' < \alpha/\ell$ we have $\eps_J < n^{-\alpha'}$.
\end{proof}
%Then by applying double logarithm to the previous equation, one can see that there exists $J = \log_\ell \ln n + O(1)$ such that $\eps_J > n^{-\alpha}$, but $\eps_{J+1} < n^{-\alpha}$.

We say that the process is in phase $j$ if the fraction $\eps$ of uninformed nodes is in $]\eps_{j+1}, \eps_j]$.
\begin{lemma} \label{lem:double-exp-shrinking-ETj-upper}
    If the process is in phase $j$, $j < J$, then the number of rounds to leave phase $j$ is stochastically dominated by $1 + \Geom(1-q)$.
%    at most $1 + \tfrac{q}{1-q}$.
\end{lemma}
\begin{proof}
    Consider a round starting with $\eps n$ uninformed nodes.
    By construction, the process leaves the phase $j$ if $y(\eps) \le \eps_{j+1} = 2E(\eps_j)$.
    Since $E(\cdot)$ is an increasing function, an upper bound for the probability to stay in phase $j$ in the current round is
    \begin{equation}
        \max_{\eps\in]\eps_{j+1},\eps_j]} \Pr[y(\eps) > 2E(\eps_j)]
        \le \max_{\eps\in]\eps_{j+1},\eps_j]} \Pr[y(\eps) \ge 2E(\eps)]
        \le q \notag.
    \end{equation}
    Hence, the number of rounds the process spends in phase $j$ is stochastically dominated by a random variable with distribution $1+\Geom(1-q)$.
%    , which has an expectation of $1+q/(1-q)$.
\end{proof}

Let us now prove the main theorem of the section.

\begin{proof}[Proof of Theorem~\ref{th:double-exp-shrinking-upper}]
	From Lemma~\ref{lem:general-connect} it follows that for any $g' < g$ we have $\Expect[T(n-\lfloor gn \rfloor, n - \lceil g'n \rceil)] = O(1)$.
	So without loss of generality we can assume that $g < (2a)^{-\tfrac1{\ell-1}}$ that is required by Observation~\ref{obs:double-exp-shrinking-epsj} and, thus, by Corollary~\ref{cor:double-exp-shrinking-nphases}.
	Let the random variable $T_j$ denote the number of rounds spent in phase $j$.
%    By Lemma~\merk{ref?!}, the number of rounds since the process enters in the fast finishing regime until all nodes are informed is at most $O(1)$.
    With Corollary~\ref{cor:double-exp-shrinking-nphases} as well as Lemma~\ref{lem:double-exp-shrinking-failure-upper}~and~\ref{lem:double-exp-shrinking-ETj-upper}, we compute
    \begin{eqnarray}
        && \Expect[T(n-\lfloor gn \rfloor, n - \lceil\eps_Jn\rceil)]
        	\le \sum_{j=0}^{J-1} \Expect[T_j]
	        \le J \left(1 + \tfrac{q}{1-q}\right)
        	= \log_\ell \ln n + O(1) \label{eq:2/th-42} \\
        && \Pr\left[T(n-\lfloor gn \rfloor, n - \lceil \eps_Jn \rceil) > J+r\right]
        	\le J q^{-r} = n^{-\Omega(r)} \,. \label{eq:1/th-42}
    \end{eqnarray}
    By Corollary~\ref{cor:double-exp-shrinking-nphases}, $\eps_J < \left(\tfrac{n^{-\alpha}}{2a}\right)^{1/\ell}$.
    Consequently, there exists $\alpha' \in ]0,\alpha[$ such that $\eps_J < n^{-\alpha'}$ for any $n$ large enough.
    Without loss of generality we can assume that for any $u \le n^{1-\alpha'}$ we have $1-p_{n-u} \le n^{-\tau}$ (for $u \in [n^{1-\alpha}, n^{1-\alpha'}]$ it follows from the double exponential shrinking condition).
    Now suppose $u_0 \le n^{1-\alpha'}$ and consider $T(n-u_0,n)$.
    %Since for any $u \le n^{1-\alpha}$ we have $1-p_{n-u} \le n^{-\tau}$, this
	By the argument above, any of the $u_0$ uninformed nodes stays uninformed for $r \ge 1$ rounds with probability at most $n^{-\tau r}$.
	Then by the union bound, we have $\Pr[T(n-u_0, n) > r] \le P_r := \min\{1,n^{-\tau r + 1-\alpha'}\}$, that together with~\eqref{eq:1/th-42} proves the tail bound statement.
	
	Finally, $\E[T(n-u_0,n)] \le 1+\sum_{r\ge1} P_r = O(1)$, for any $u_0 \le n^{1-\alpha}$.
	Then, together with~\eqref{eq:2/th-42} it proves that $\E[T(n-\lfloor gn \rfloor, n)] \le \log_\ell \ln n + O(1)$.
%	To finish the proof we recall that from Observation~\ref{obs:double-exp-shrinking-nphases} it follows that $\eps_Jn \le n^{1-\alpha'}$.
\end{proof}
%\merk{this is the relaxed version of the corollary. In my opinion the stronger one is better (see below).}
%\begin{corollary}
%	There exist $A, \alpha > 0$ such that for any integer $r > 0$ we have
%	$$\Pr[T(n-\lfloor gn \rfloor,n) \le \E[T(n-\lfloor gn \rfloor,n)] + r] \le Ae^{-\alpha r}.$$
%\end{corollary}
%\begin{proof}
%	By construction, the interval $[n-\lfloor gn \rfloor, n - \lceil \eps_Jn \rceil]$ is cut into $J$ phases.
%	The number of rounds spent in each phase is stochastically dominated by $1+\Geom(1-q)$, where $q = n^{-\Theta(1)}$.
%	Consequently, for any integer $r > 0$ we have
%	$$\Pr\left[T(n-\lfloor gn \rfloor, n - \lceil \eps_Jn \rceil) > J+r\right] < J n^{-\Theta(1)\cdot r} \,.$$
%	Since $J = O(\log\log n)$, there exist $A', \alpha'$ such that $Jn^{-\Theta(1)\cdot r} \le A'e^{-\alpha'r}$.
%	Then, in the proof of Theorem~\ref{th:double-exp-shrinking-upper} we showed that for any $u_0 \le \lfloor \eps_J n \rfloor$ we have
%	$$\Pr[T(n-u_0, n) > r] \le \min\left\{1,n^{-\Theta(1)\cdot r}\right\}.$$
%	The claim of statement follows directly.
%\end{proof}

% \merk{stronger version of the corollary}
%\begin{corollary}
%	There exist $A, \alpha > 0$ such that for any integer $r > 0$ we have
%	$$\Pr[T(n-\lfloor gn \rfloor,n) \ge \E[T(n-\lfloor gn \rfloor,n)] + r] \le n^{-\alpha r + A}.$$
%\end{corollary}
%
%%\merk{I would prefer the version below without constants $A, \alpha$.}
%%\begin{corollary}
%%	For any integer $r > 0$ we have
%%	$$\Pr[T(n-\lfloor gn \rfloor,n) \le \E[T(n-\lfloor gn \rfloor,n)] + r] \le n^{-\Theta(1)\cdot r}.$$
%%\end{corollary}
%
%\begin{proof}
%	By construction, the interval $[n-\lfloor gn \rfloor, n - \lceil \eps_Jn \rceil]$ is cut into $J$ phases.
%	The number of rounds spent in each phase is stochastically dominated by $1+\Geom(1-q)$, where $q = n^{-\Theta(1)}$.
%	Consequently, for any integer $r > 0$ we have
%	$$\Pr\left[T(n-\lfloor gn \rfloor, n - \lceil \eps_Jn \rceil) > J+r\right] < J n^{-\Theta(1)\cdot r} \,.$$
%	Note that, since $J = O(\log\log n)$, we have $Jn^{-\Theta(1)\cdot r} = n^{-\Theta(1)\cdot r}$.
%	Then, in the proof of Theorem~\ref{th:double-exp-shrinking-upper} we showed that for any $u_0 \le \lfloor \eps_J n \rfloor$ we have
%	$$\Pr[T(n-u_0, n) > r] \le \min\left\{1,n^{-\Theta(1)\cdot r}\right\}.$$
%	The claim of statement follows directly.
%\end{proof}




%****************************************************************
%****************************************************************
%****************************************************************




\subsection{Double Exponential Shrinking Regime. Lower Bound.}

We now prove that under lower bound conditions comparable to the upper bound conditions of the previous section, we obtain a lower bound on the runtime equaling our upper bound apart from an additive constant.

\subsubsection{Double Exponential Shrinking Conditions}

Throughout this section, we assume that the following \emph{lower double exponential shrinking conditions} are satisfied.
%%%DEF - The double exponential shrinking conditions

%\merk{A: maybe it is reasonable to add a condition $ag^{\ell-1} < 1$ to ensure that the lower bound for the probability to stay uninformed is less than 1.}
\begin{defn}[lower double exponential shrinking conditions] \label{def:dexp}
	Let $g, \alpha \in ]0,1]$ and $\ell > 1$.
	Let $a, c \in \R_{\ge0}$.
    %Suppose $ag^{\ell-1} < 1$.
	We say that a homogeneous epidemic protocol satisfies \emph{the lower double exponential shrinking conditions} if for any $n$ big enough, the following properties are satisfied for all $u \in [n^{1-\alpha},gn]$.
	\renewcommand{\theenumi}{(\roman{enumi})}%
	\begin{enumerate}
		\item $1-p_{n-u} \ge a\left(\tfrac{u}{n}\right)^{\ell-1}$.
		\item $c_{n-u} \le c\tfrac{n}{u^2}$.
	\end{enumerate}
\end{defn}

Similarly to the upper double exponential shrinking conditions, we work mostly with the fraction $\eps := \tfrac{u}{n}$ of uninformed nodes instead of the absolute number~$u$.
Thus, the double exponential shrinking conditions turns into the following bounds, valid for all $\eps \in [n^{-\alpha},g]$ with $\eps n \in \N$.
\renewcommand{\theenumi}{(\roman{enumi})}%
\begin{enumerate}
	\item $1-p_{n(1-\eps)} \ge a\eps^{\ell-1}$.
	\item $c_{n(1-\eps)} \le \eps^{-2}\tfrac{c}{n}$.
\end{enumerate}

%The main result of this section is that the double exponential shrinking conditions (possibly with the fast finishing conditions in the end) give a sharp runtime of $\log_\ell \ln n + O(1)$.
The main result of this section is the following theorem.
%\begin{theorem}\label{th:double-exp-shrinking-lower}
%	Consider a homogeneous epidemic protocol satisfying the lower exponential shrinking condition.
%    Then, there is a constant $\alpha > 0$ such that for any $0 < g' \le g$,
%	\begin{equation}
%		\Expect[T(n-\lceil g'n \rceil, n-\lfloor n^{1-\alpha} \rfloor)] \ge \log_\ell \ln n + O(1) \notag.
%    \end{equation}
%\end{theorem}

\begin{theorem}\label{th:double-exp-shrinking-lower}
	Consider a homogeneous epidemic protocol satisfying the lower double exponential shrinking conditions in the interval $[n^{1-\alpha},gn]$. Let $r$ be a sufficiently large constant (possibly depending on $\alpha$). Then,
	\begin{align*}
		&\Expect[T(n-\lceil gn \rceil, n-\lfloor n^{1-\alpha} \rfloor)] \ge \log_\ell \ln n + O(1),\\
		&\Pr[T(n-\lceil gn \rceil, n-\lfloor n^{1-\alpha} \rfloor) \le \log_\ell \ln n - r] \le O(n^{-1+2\alpha\ell}),
	\end{align*}
\end{theorem}

\subsubsection{Round Targets and Failure Probabilities}
Let again $y(\eps)$ denote the fraction of uninformed nodes at the end of a round started with $\eps n$ uninformed ones.
%Since for the lower bound we have that $\Expect[y(\eps)] \le \eps n (1-p_{n(1-\eps)})$,
The double exponential shrinking conditions state that
\begin{equation}
    \Expect[y(\eps)] \ge E(\eps) := a\eps^\ell \notag.
\end{equation}

The next lemma gives that with good probability, $y(\eps)$ is at least the \emph{target} value $E(\eps)/2$.

%\merk{A: "for any fraction of uninformed nodes" allows not to mention that $\eps n \in \N$.}
\begin{lemma}\label{lem:double-exp-shrinking-failure-lower}
    For any fraction of uninformed nodes $\eps \in [n^{-\alpha}, g]$,
    \begin{equation}
        \Pr\left[y(\eps) \le \tfrac12E(\eps)\right] \le \tfrac{4+4c}{a^2 \eps^2 n} \le q := \tfrac{4+4c}{a^2}n^{2\alpha\ell-1} \notag.
    \end{equation}
\end{lemma}
\begin{proof}
    Applying Chebyshev's inequality and taking into account that $\Expect[y(\eps)] \ge E(\eps)$, we compute
    \begin{equation}
        \Pr[y(\eps) \le \tfrac12E(\eps)]
        \le \Pr\left[y(\eps) \le \Expect[y(\eps)] - \tfrac12E(\eps)\right]
        \le 4 \cdot \tfrac{\Var[y(\eps)]}{E(\eps)^2} \notag.
    \end{equation}
    By the same arguments like in Lemma~\ref{lem:double-exp-shrinking-variance}, $\Var[y(\eps)] \le \tfrac{1+c}{n}$.
    Since $\eps \ge n^{-\alpha}$, we have $E(\eps) \ge an^{-\alpha\ell}$, and the claim of the lemma directly follows.
\end{proof}
%Similarly to Lemma~\ref{lem:double-exp-shrinking-failure-upper}, the \emph{failure} probability $q$ does not depend on $\eps$.
%We also assume that $\alpha < \tfrac1{2\ell}$ to ensure that $q < 1$.

Similarly to the upper bound, our choice to analyze the double exponential shrinking regime only up to $n^{1-\alpha}$ uninformed nodes allows us to define $q$ independent of $\eps$.
We also assume that $\alpha< \tfrac1{2\ell}$ so that $q=n^{-\Theta(1)}$.

%%%  PHASE CALCULUS
\subsubsection{The Phase Calculus}

Let us define the sequence $\eps_j$ recursively by
\begin{equation}
    \eps_0 := g, \quad \eps_{j+1} := \tfrac12E(\eps_j) \notag.
\end{equation}
The next observation follows from the definition by a simple induction.
The $\eps_j$ are decreasing simply because $\eps_{j+1}=\tfrac12E(\eps_j) < \Expect[y(\eps_j)] \le \eps_j$.
Note that $y(\eps) \le \eps$ with probability one for any homogeneous protocol.
\begin{observation}
	For all $j \ge 1$,
    $\eps_j = (a/2)^{\frac{\ell^j-1}{\ell-1}} g^{\ell^j}$.
    The $\eps_j$ form a decreasing sequence.% if $g < (a/2)^{-\tfrac1{\ell-1}}$.
\end{observation}
In the rest of the section we assume that $g < (a/2)^{-\tfrac1{\ell-1}}$.
Applying logarithm twice to the previous equation one can also see the following.
\begin{observation}\label{obs:double-exp-shrinking-nphases-lower}
    There exists $J = \log_\ell \ln n + O(1)$ such that $\eps_J > n^{-\alpha}$.%, but $\eps_{J+1} \le n^{-\alpha}$.
\end{observation}
%Then by applying double logarithm to the previous equation, one can see that there exists $J = \log_\ell \ln n + O(1)$ such that $\eps_J > n^{-\alpha}$, but $\eps_{J+1} < n^{-\alpha}$.

As before, we say that the process is in phase $j$ if the fraction $\eps$ of uninformed nodes is in $]\eps_{j+1}, \eps_j]$.
%Since $q$ does not depends on epsilon, the following lemma is also obvious.
\begin{lemma}\label{lem:double-exp-shrinking-ETj-lower}
    If the process starts in phase $j$, $j < J$, then the probability that after one round it is in phase $j+2$ or higher is at most $q$.
\end{lemma}
%\merk{A: should we provide such a proof?}
\begin{proof}
    Consider a round starting with $\eps n$ uninformed nodes, where $\eps \in ]\eps_{j+1},\eps_j]$.
    By construction, the process leapfrogs phase $j+1$ if $y(\eps) \le \eps_{j+2} = \tfrac12E(\eps_{j+1})$.
    Since $E(\cdot)$ is an increasing function, an upper bound for the probability to jump over phase $j+1$ is
    \begin{equation}
        \max_{\eps\in]\eps_{j+1},\eps_j]} \Pr[y(\eps) \le \tfrac12E(\eps_{j+1})]
        \le \max_{\eps\in]\eps_{j+1},\eps_j]} \Pr[y(\eps) \le \tfrac12E(\eps)]
        \le q \notag.
    \end{equation}
\end{proof}

\begin{proof}[Proof of Theorem~\ref{th:double-exp-shrinking-lower}]
  Consider the rumor spreading process starting with $\eps_0 n = gn$ uninformed nodes. By Lemma~\ref{lem:double-exp-shrinking-ETj-lower}, with probability at least $(1-q)^J \ge 1 - Jq$, the process visits each phase $j \in [0..J-1]$, which naturally takes at least $J-1$ rounds. Consequently, by definition of $J$ in Observation~\ref{obs:double-exp-shrinking-nphases-lower}, we have
	\begin{align*}
		\Expect[T(n-\lceil gn \rceil, n-\lfloor n^{1-\alpha} \rfloor)]
			&\ge \Expect[T(n-\lceil n\eps_0 \rceil, n-\lfloor n\eps_J\rfloor)] \\
			&\ge (J-1)(1-Jq) = \log_\ell \ln n + O(1) \notag.
	\end{align*}
	The large-deviation statement follows immediately from adding the failure probabilities $\frac{4+4c}{a^2\eps_j^2 n}$, $j = 0, \dots, J-1$, from Lemma~\ref{lem:double-exp-shrinking-failure-lower}.
\end{proof}
%\begin{proof}[old proof]
%	Without loss of generality we can assume that $g' = g$.
%    The proof uses the same arguments as the proof of Theorem~\ref{th:exp-shrinking-lower} for the exponential shrinking.
%    By definition of $\eps_j$ and Observation~\ref{obs:double-exp-shrinking-nphases-lower}, we have
%    \begin{equation}
%    	\Expect[T(n-\lceil gn \rceil, n-\lfloor n^{1-\alpha} \rfloor)]
%    		\ge \Expect[T(n-\lceil n\eps_0 \rceil, n-\lfloor n\eps_J\rfloor)] \notag.
%    \end{equation}
%    Suppose $\tau$ be the smallest round $t$ in which the process leapfrogs a phase ($\tau = \infty$, if this never happens).
%    The spreading time is hence at least $\min\{J, \tau\}$.
%    By Lemma~\ref{lem:double-exp-shrinking-ETj-lower}, $\Pr[\tau=t] \le q = O\left(n^{2\alpha\ell-1}\right)$ for any $t$.
%    Therefore,
%    \begin{align}
%        \Expect&[T(n-\lceil gn \rceil, n-\lfloor n^{1-\alpha} \rfloor)] \notag \\
%%            \ge \Expect[T(n-\lceil gn \rceil, n-\lfloor n\eps_J\rfloor)] \notag \\
%		& \ge J \Pr[\tau > J] + \sum_{t=1}^{J-1} t\cdot\Pr[\tau = t] \notag \\
%		& = J - \sum_{t=1}^{J-1} qt
%			= \log _\ell \ln n + O(1) \notag.
%    \end{align}
%\end{proof} 