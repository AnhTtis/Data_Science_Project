\section{Datasets and Splits}
\label{sec:app_expt_setup}

\begin{table}[!b]
     \vspace{-0.9cm}
    \begin{center}
      \centering
     \captionof{table}{Two-step and five-step dataset splits for the \cincd experiments. The number of novel classes $\classes\tst$ and the number of unlabelled images $\numdatast$ in $\data\tst$ for each task $\task\tst$ are reported.}
       \begin{tabular}{l|rrrr|rrrrrrrrrr}
        \toprule
        
        \multirow{3}{*}{Splits} & \multicolumn{4}{c|}{Two-step} & \multicolumn{10}{c}{Five-step} \\
         & \multicolumn{2}{c}{$\task\tsone$} & \multicolumn{2}{c|}{$\task\tstwo$}   & \multicolumn{2}{c}{$\task\tsone$} & \multicolumn{2}{c}{$\task\tstwo$} & \multicolumn{2}{c}{$\task\tsthree$} & \multicolumn{2}{c}{$\task\tsfour$} & \multicolumn{2}{c}{$\task\tsfive$} \\
        
         & $\classes\tsone$ & $\vert\data\tsone\vert$ & $\classes\tstwo$ & $\vert\data\tstwo\vert$ &  $\classes\tsone$ & $\vert\data\tsone\vert$ & $\classes\tstwo$ & $\vert\data\tstwo\vert$ & $\classes\tsthree$ & $\vert\data\tsthree\vert$ & $\classes\tsfour$ & $\vert\data\tsfour\vert$ & $\classes\tsfive$ & $\vert\data\tsfive\vert$ \\
        \hline
        
        C10 & 5 & 25.0k & 5 & 25.0k & 2 & 10.0k & 2 & 10.0k & 2 & 10.0k & 2 & 10.0k & 2 & 10.0k \\
        C100 & 50 & 25.0k & 50 & 25.0k & 20 & 10.0k & 20 & 10.0k & 20 & 10.0k & 20 & 10.0k & 20 & 10.0k \\
        T200 & 100 & 50.0k & 100 & 50.0k & 40 & 20.0k & 40 & 20.0k & 40 & 20.0k & 40 & 20.0k & 40 & 20.0k \\
        B200 & 100 & 2.4k & 100 & 2.4k & 40 & 0.9k & 40 & 0.9k & 40 & 0.9k & 40 & 0.9k & 40 & 0.9k \\
        H683 & 342 & 14.5k & 341 & 16.3k & 137 & 6.3k & 137 & 5.4k & 137 & 6.1k & 137 & 6.8k & 135 & 6.3k \\
        \bottomrule
      \end{tabular} 
       % \vspace{-.2in}

       \vspace{-0.9cm}
       \label{tab:data_details}
    \end{center}
\end{table}

We conduct experiments on five datasets, which are: CIFAR-10 (C10), CIFAR-100 (C100), TinyImageNet-200 (T200), CUB-200 (B200) and Herbarium-683 (H683). The Tab.~\ref{tab:data_details} presents the detailed splits for the two adopted task sequences (two-step and five-step) on the five data sets~\cite{krizhevsky2009learning,le2015tiny,wah2011caltech,Tan2019TheHC}. For a task sequence of $T=2$, the total classes and their corresponding instances in the dataset are equally divided into two splits (\eg, for C100, 100 classes / 2 tasks = 50 novel classes per task). Similarly, for a task sequence of $T=5$, the same method is used to divide the classes and their corresponding instances into five splits (\eg, for C100, 100 classes / 5 tasks = 20 novel classes per task).

The experimental results on C10, C100, and T200 provide an indication of the performance of the studied \cincd methods in common image recognition tasks, while the results on B200 and H683 show their performance in fine-grained image recognition tasks. Moreover, the evaluation on H683 offers insights into the performance of the studied methods in long-tailed task sequences and also when the downstream dataset is quite different from the internet-scale images.



