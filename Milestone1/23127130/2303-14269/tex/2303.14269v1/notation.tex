
%\usepackage[margin = 1in]{geometry}

%\usepackage{palatino}
% \usepackage{newpxtext}
%\usepackage{lmodern}



%\usepackage{fullpage,times}

\usepackage{mathrsfs}
\usepackage{amsmath}
\usepackage[english]{babel}
\usepackage[toc,page]{appendix} 
\usepackage{hyperref}
%\geometry{letterpaper}                   	 
\usepackage{graphicx,wrapfig,lipsum,xcolor}	
%\usepackage[labelformat=empty]{caption}
% 
\usepackage{yfonts}	 		
\usepackage{amssymb}
\usepackage{amsmath}
\usepackage{amsthm}
\usepackage{scalerel}
\usepackage{verbatim}
\usepackage[lined,boxed,ruled,norelsize,algo2e,linesnumbered]{algorithm2e}
%  






%%%%%%%%%%%
\usepackage{dsfont}
\usepackage{amsmath}
\usepackage{mathrsfs}
\usepackage{amssymb, graphicx, amsmath, amsthm}

\def \ind{\mathds{1}}
\def \N{\mathbb{N}}
\def \R{\mathbb{R}}
\def \D{\mathbb{D}}
\def \G{\mathbb{G}}
\def \U{\mathbb{U}}
\def \calA{\mathcal{A}}
\def \calB{\mathcal{B}}
\def \calC{\mathcal{C}}
\def \calV{\mathcal{V}}
\def \calH{\mathcal{H}}
\def \calE{\mathcal{E}}
\def \calN{\mathcal{N}}
\def \calS{\mathcal{S}}
\def \calX{\mathcal{X}}
\def \calG{\mathcal{G}}
\def \calF{\mathcal{F}}
\def \calP{\mathcal{P}}
\def \calM{\mathcal{M}}
\def \calZ{\mathcal{Z}}
\def \calR{\mathcal{R}}
\def \calK{\mathcal{K}}
\def \calL{\mathcal{L}}
\def \calO{\mathcal{O}}
\def \pbb{\mathbb{P}} 
\def \E{\mathbb{E}}
\def \LL{\mathscr{L}}
\DeclareMathOperator*{\cntr}{center}
\DeclareMathOperator*{\diam}{diam}
\DeclareMathOperator*{\disc}{disc}
\DeclareMathOperator*{\add}{additive}
\DeclareMathOperator*{\op}{op}
\DeclareMathOperator*{\dist}{dist} 
\DeclareMathOperator*{\sep}{sep}
\DeclareMathOperator*{\var}{var}
\DeclareMathOperator*{\vect}{vec}
\DeclareMathOperator*{\aut}{Aut}
\DeclareMathOperator*{\argmin}{arg\,min}
\DeclareMathOperator*{\tr}{tr}
\DeclareMathOperator*{\argmax}{arg\,max}
\usepackage{algorithm}
\usepackage{algorithmic}
\usepackage{tikz} 
\renewcommand{\algorithmicrequire}{\textbf{Input:}}
\renewcommand{\algorithmicensure}{\textbf{Output:}}
%\newtheorem{definition}{Definition}
\newtheorem{example}{Example}
%\newtheorem{theorem}{Theorem}
\newtheorem{claim}{Claim}
%\newtheorem{lemma}{Lemma}
%\newtheorem{corollary}{Corollary}
%\newtheorem{remark}{Remark}
%\newtheorem{proposition}{Proposition}
\newtheorem{conjecture}{Conjecture}
\usepackage{amsthm}
%%%%%%%%%%%

\allowdisplaybreaks


%bold
\def \ba{\bold{a}}
\def \bh{\bold{h}}
\def \bK{\bold{K}}
\def \bx {\bold{x}}
\def \bX {\bold{X}}
\def \by {\bold{y}}
\def \bY {\bold{Y}}
\def \bg {\bold{g}} 
\def \bG {\bold{G}}
\def \bz {\bold{z}}
\def \bZ {\bold{Z}}
\def \bj {\bold{j}}
\def \bz {\bold{z}}
\def \bJ {\bold{J}}

\def \fraku {\mathfrak{u}}


%%%%%%%%%%

%special

\def \ff{\hat{f}}
\def \calRR{\hat{\mathcal{R}}}
