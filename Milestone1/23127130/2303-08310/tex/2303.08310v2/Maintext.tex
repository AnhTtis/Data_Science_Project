
%%%%%%%%%%%%%%%%%%%%%%%%%%%%%%%%%%%%%%%%%%%%%%%%%%%%%%%%%%%%%%%%%%%%%
%% This is a (brief) model paper using the achemso class
%% The document class accepts keyval options, which should include
%% the target journal and optionally the manuscript type.
%%%%%%%%%%%%%%%%%%%%%%%%%%%%%%%%%%%%%%%%%%%%%%%%%%%%%%%%%%%%%%%%%%%%%
\documentclass[journal=jacsat,manuscript=article]{achemso}

%%%%%%%%%%%%%%%%%%%%%%%%%%%%%%%%%%%%%%%%%%%%%%%%%%%%%%%%%%%%%%%%%%%%%
%% Place any additional packages needed here.  Only include packages
%% which are essential, to avoid problems later.
%%%%%%%%%%%%%%%%%%%%%%%%%%%%%%%%%%%%%%%%%%%%%%%%%%%%%%%%%%%%%%%%%%%%%
\usepackage{chemformula} % Formula subscripts using \ch{}
\usepackage[T1]{fontenc} % Use modern font encodings

%%%%%%%%%%%%%%%%%%%%%%%%%%%%%%%%%%%%%%%%%%%%%%%%%%%%%%%%%%%%%%%%%%%%%
%% If issues arise when submitting your manuscript, you may want to
%% un-comment the next line.  This provides information on the
%% version of every file you have used.
%%%%%%%%%%%%%%%%%%%%%%%%%%%%%%%%%%%%%%%%%%%%%%%%%%%%%%%%%%%%%%%%%%%%%
%%\listfiles

%%%%%%%%%%%%%%%%%%%%%%%%%%%%%%%%%%%%%%%%%%%%%%%%%%%%%%%%%%%%%%%%%%%%%
%% Place any additional macros here.  Please use \newcommand* where
%% possible, and avoid layout-changing macros (which are not used
%% when typesetting).
%%%%%%%%%%%%%%%%%%%%%%%%%%%%%%%%%%%%%%%%%%%%%%%%%%%%%%%%%%%%%%%%%%%%%
\newcommand*\mycommand[1]{\texttt{\emph{#1}}}

%%%%%%%%%%%%%%%%%%%%%%%%%%%%%%%%%%%%%%%%%%%%%%%%%%%%%%%%%%%%%%%%%%%%%
%% Meta-data block
%% ---------------
%% Each author should be given as a separate \author command.
%%
%% Corresponding authors should have an e-mail given after the author
%% name as an \email command. Phone and fax numbers can be given
%% using \phone and \fax, respectively; this information is optional.
%%
%% The affiliation of authors is given after the authors; each
%% \affiliation command applies to all preceding authors not already
%% assigned an affiliation.
%%
%% The affiliation takes an option argument for the short name.  This
%% will typically be something like "University of Somewhere".
%%
%% The \altaffiliation macro should be used for new address, etc.
%% On the other hand, \alsoaffiliation is used on a per author basis
%% when authors are associated with multiple institutions.
%%%%%%%%%%%%%%%%%%%%%%%%%%%%%%%%%%%%%%%%%%%%%%%%%%%%%%%%%%%%%%%%%%%%%
\author{Xi Yang}
\affiliation[First University]
{Frontiers Science Center for Nano-optoelectronics and State Key Laboratory for Mesoscopic Physics, School of Physics, Peking University, Beijing 100871, China}
\author{Shui-Jing Tang}
\affiliation[First University]
{Frontiers Science Center for Nano-optoelectronics and State Key Laboratory for Mesoscopic Physics, School of Physics, Peking University, Beijing 100871, China}
\author{Jia-Wei Meng}
\affiliation[First University]
{Frontiers Science Center for Nano-optoelectronics and State Key Laboratory for Mesoscopic Physics, School of Physics, Peking University, Beijing 100871, China}
\author{{Pei-Ji Zhang}}
\affiliation[First University]
{Frontiers Science Center for Nano-optoelectronics and State Key Laboratory for Mesoscopic Physics, School of Physics, Peking University, Beijing 100871, China}
\author{You-Ling Chen}
\email{ylchen@semi.ac.cn}
\affiliation[First University]
{State Key Laboratory on Integrated Optoelectronics, Institute of
Semiconductors, Chinese Academy of Sciences, Beijing 100083, China}
\author{Yun-Feng Xiao}
%\altaffiliation{A shared footnote}
\email{yfxiao@pku.edu.cn}
%\phone{+123 (0)123 4445556}
%\fax{+123 (0)123 4445557}
\affiliation[First University]
{Frontiers Science Center for Nano-optoelectronics and State Key Laboratory for Mesoscopic Physics, School of Physics, Peking University, Beijing 100871, China}
\alsoaffiliation[Second University]
{Collaborative Innovation Center of Extreme Optics, Shanxi University, Taiyuan 030006, China}
\alsoaffiliation[Third University]
{Peking University Yangtze Delta Institute of Optoelectronics, Nantong 226010, China}
\alsoaffiliation[Fourth University]
{National Biomedical Imaging Center, Peking University, Beijing 100871, China}
%\author{Susanne K. Laborator}
%\email{s.k.laborator@bigpharma.co}
%\affiliation[BigPharma]
%{Lead Discovery, BigPharma, Big Town, USA}
%\author{Kay T. Finally}
%\affiliation[Unknown University]
%{Department of Chemistry, Unknown University, Unknown Town}
%\alsoaffiliation[Second University]
%{Department of Chemistry, Second University, Nearby Town}

%%%%%%%%%%%%%%%%%%%%%%%%%%%%%%%%%%%%%%%%%%%%%%%%%%%%%%%%%%%%%%%%%%%%%
%% The document title should be given as usual. Some journals require
%% a running title from the author: this should be supplied as an
%% optional argument to \title.
%%%%%%%%%%%%%%%%%%%%%%%%%%%%%%%%%%%%%%%%%%%%%%%%%%%%%%%%%%%%%%%%%%%%%
\title[An \textsf{achemso} demo]
  {Phase-transition microcavity laser}
  %\footnote{A footnote for the title}}

%%%%%%%%%%%%%%%%%%%%%%%%%%%%%%%%%%%%%%%%%%%%%%%%%%%%%%%%%%%%%%%%%%%%%
%% Some journals require a list of abbreviations or keywords to be
%% supplied. These should be set up here, and will be printed after
%% the title and author information, if needed.
%%%%%%%%%%%%%%%%%%%%%%%%%%%%%%%%%%%%%%%%%%%%%%%%%%%%%%%%%%%%%%%%%%%%%
\abbreviations{IR,NMR,UV}

%\newcommand\keywords[1]{\textbf{Keywords}: #1}

\keywords{Microcavity laser, phase transition, liquid crystal, thermal sensing}

%%%%%%%%%%%%%%%%%%%%%%%%%%%%%%%%%%%%%%%%%%%%%%%%%%%%%%%%%%%%%%%%%%%%%
%% The manuscript does not need to include \maketitle, which is
%% executed automatically.
%%%%%%%%%%%%%%%%%%%%%%%%%%%%%%%%%%%%%%%%%%%%%%%%%%%%%%%%%%%%%%%%%%%%%
\begin{document}

%%%%%%%%%%%%%%%%%%%%%%%%%%%%%%%%%%%%%%%%%%%%%%%%%%%%%%%%%%%%%%%%%%%%%
%% The "tocentry" environment can be used to create an entry for the
%% graphical table of contents. It is given here as some journals
%% require that it is printed as part of the abstract page. It will
%% be automatically moved as appropriate.
%%%%%%%%%%%%%%%%%%%%%%%%%%%%%%%%%%%%%%%%%%%%%%%%%%%%%%%%%%%%%%%%%%%%%


%%%%%%%%%%%%%%%%%%%%%%%%%%%%%%%%%%%%%%%%%%%%%%%%%%%%%%%%%%%%%%%%%%%%%
%% The abstract environment will automatically gobble the contents
%% if an abstract is not used by the target journal.
%%%%%%%%%%%%%%%%%%%%%%%%%%%%%%%%%%%%%%%%%%%%%%%%%%%%%%%%%%%%%%%%%%%%%
\begin{abstract} 

Liquid-crystal microcavity lasers have attracted considerable attention because of their extraordinary tunability and sensitive response to external stimuli, and they operate generally within a specific phase.
Here, we demonstrate a liquid-crystal microcavity laser operated in the phase transition, 
in which the reorientation of liquid-crystal molecules occurs from aligned to disordered states.
A significant wavelength shift of the microlaser is observed, resulting from the dramatic changes in the refractive index of liquid-crystal microdroplets during the phase transition. 
This phase-transition microcavity laser is then exploited for sensitive thermal sensing, enabling two-order-of-magnitude enhancement in sensitivity compared with the nematic-phase microlaser operated far from the transition point. Experimentally, we demonstrate an exceptional sensitivity of -40 nm/K and an ultrahigh resolution of 320 \textmu K. The phase-transition microcavity laser features compactness, softness, and tunability, showing great potential for high-performance sensors, optical modulators, and soft matter photonics.
\end{abstract}

%%%%%%%%%%%%%%%%%%%%%%%%%%%%%%%%%%%%%%%%%%%%%%%%%%%%%%%%%%%%%%%%%%%%%
%% Start the main part of the manuscript here.
%%%%%%%%%%%%%%%%%%%%%%%%%%%%%%%%%%%%%%%%%%%%%%%%%%%%%%%%%%%%%%%%%%%%%
\section{Introduction}

Optical microcavities with high quality factors and small mode volumes  
significantly enhance the light-matter interaction, 
facilitating a wide variety of applications in nonlinear optics,\cite{lai2020earth,li2018whispering,strekalov2016nonlinear,del2007optical}
quantum optomechanics,\cite{aspelmeyer2014cavity,zhang2021optomechanical} non-Hermitian physics,\cite{ozdemir2019parity,cao2015dielectric}
as well as high-performance lasers and sensors.\cite{toropov2021review,liu2020nonlinear,basiri2019precision} 
Generally, the emergence of new materials may improve performance and expand many functionalities of optical microcavities.\cite{liu2022emerging} For example, lithium niobate with a large electro-optic coefficient and a second-order nonlinear coefficient, greatly promotes the development of microcavity-based  electro-optic modulators,\cite{wang2018nanophotonic} frequency combs,\cite{zhang2019broadband,wang20222} and nonlinear frequency converters.\cite{zhu2021integrated,lu2020toward,lin2019broadband,zheng2019high}
Halide perovskite featuring the intrinsic gain inspires the integrated microcavity lasers toward applications of all-optical switching,\cite{huang2020ultrafast} color display,\cite{gao2018lead} and light-emitting diodes. \cite{miao2020microcavity,zhang2021halide}

Liquid crystal (LC) has been recently introduced into optical microcavities, creating new opportunities in tunable lasers and high-performance sensors.\cite{mysliwiec2021liquid,coles2010liquid,humar2016liquid,ma2022self}
The long-range order and the fluid nature of LC molecules lead to its unique properties not being presented in any other material.
The exceptional tunability enables a wide range of applications of LC microlasers in tunable light sources,\cite{xiang2016electrically,chen2022over,adamow2020electrically} light field modulation,\cite{zhang2020tunable,papivc2021topological,ali2022demand} and laser display.\cite{zhan20213d,wang2021programmable} The stimulus-response allows LC microlasers for high-performance sensing of temperature\cite{franklin2021bioresorbable,humar2017biomaterial}  and   humidity\cite{hu2020snr}, as well as  biomolecules\cite{gong2021topological,wang2021applications} and ions\cite{duan2019detection,wang2018detecting}. In these applications, the LC microlasers are tuned generally by changing the orientation of LC molecules within a specific phase. 
Theoretically, LC microlasers operated in the vicinity of phase transition are expected to enhance significantly their tunability and sensitivity,\cite{franklin2021bioresorbable,yang2021operando} but yet to be demonstrated experimentally.

Here we demonstrate a phase-transition microcavity laser using LC microdroplets supporting whispering-gallery modes (WGMs). 
It is observed experimentally that the lasing wavelength experiences a dramatic shift, which reaches the maximum when the LC molecules approach the transition point between two different phases.
This phase-transition microlaser is further utilized for thermal sensing.
The thermal sensitivity in the proximity of phase transition is enhanced by two orders of magnitude without significant changes in background fluctuations, achieving a peak sensitivity of -40 nm/K and an ultrahigh resolution of 320 \textmu K. This sensor featuring phase transition is also compatible with sensitivity-enhanced techniques such as optical field confinement and nonlinear enhancement.\cite{jiang2020whispering,liu2020nonlinear}


\section{Results and discussion}


The schematic diagram of the phase-transition microcavity laser  
is illustrated in Figure 1, in which a LC microdroplet is formed in an aqueous environment due to its hydrophobic properties. 
The rod-like LC molecules are self-assembled with their long axis aligned along the radial direction of the microdroplet by confining surfaces.\cite{humar2009electrically} 
When the LC microdroplet doped with laser dye is pumped by a pulsed laser, multiple WGMs are excited to achieve laser emission (Figure 1a). 
The lasing spectrum depends highly on the LC phases with the specific spatial ordering of molecules, which can be controlled by temperature (Figures 1b and 1c).\cite{yang2014fundamentals}
(i) In the nematic phase, the radially aligned LC molecules 
slightly tilt with increasing temperature, causing a small blueshift of lasing wavelength due to the negative thermo-optic coefficient.
(ii) In the isotropic phase, the LC molecules have no orientational order and can point in any direction, in which the lasing wavelength shows no evident changes.
(iii) In the vicinity of the transition point between two phases, the LC molecules convert from aligned to disordered states, which induces a dramatic change in the refractive index of the LC microdroplet and thus a sharp shift of lasing wavelength. 


\captionsetup[figure]{labelfont={bf}, name={Figure}, labelsep=period}
\begin{figure*}[htbp]
\centering
\includegraphics[width=12cm]{Figure_1.pdf}
\caption{\label{fig1}
Concept of the phase-transition microcavity laser. (a) Schematics of the characterization system. The LC microdroplet supporting WGMs is pumped by a pulsed laser, and its lasing spectra are collected by an objective. Inset: thermal-induced spectrum variation of the phase-transition microlaser. (b) Spatial ordering of LC molecules from nematic to isotropic phases.
(c) Dependence of the lasing wavelength on temperature with (blue solid curve) and without (red dashed line) the LC phase transition. 
}
\end{figure*}



Experimentally, liquid crystal 7CB molecules are self-assembled into dye-doped microdroplet  cavities with a diameter of about 15 \textmu m (see Methods in Supporting Information). The surface of the microdroplet is modified by sodium dodecyl sulfate (SDS) to form the radial configuration of LC molecules. The concentration of SDS should be high enough to keep the radial configuration of LC molecules and also be lower to minimize the influence on phase transition. Their spatial ordering inside the microdroplet can be observed from the bright-field and polarized optical images (insets of Figures 2a, 2b, and S1). Typically, a nematic LC microdroplet indicates a cross pattern in the polarized optical image and creates a point topological defect in the center of the droplet under bright-field image (insets of Figure 2a), while these features disappear in the isotropic phase (insets of Figure 2b). 

To characterize the phase transition, the LC microdroplet is pumped by a pulsed laser (532 nm, 1.8 ns) to generate lasing at around 615 nm.
The lasing spectra at the nematic and isotropic phases are presented in Figures 2a and 2b, respectively. It is observed that the transverse-magnetic (TM) lasing modes dominate in the nematic phase, since these modes with electric field oscillating along the long axis of the LC molecules experience a larger dielectric constant and thus higher quality factors than the transverse-electric (TE) modes whose electric field oscillate along the short axis of LC molecules. By contrast, both TM and TE modes lase in the isotropic phase due to the vanish of birefringence (Figure S2).  
It is also found that both lasing threshold and free spectral ranges (FSRs) vary significantly with the phase transition as shown in Figures 2c and 2d, respectively. 
The lasing threshold achieved in nematic LC microlaser (8 \textmu J/mm$^{2}$) is much lower than that in the isotropic phase (110 \textmu J/mm$^{2}$), which is attributed to the phase-transition-induced decrease of refractive index (Figure S3).
The FSR increases statistically from 3.7 THz to 4.1 THz, in which the FSRs of the same mode family (lasing modes with same radial and azimuthal mode numbers, as well as different angular mode number) of a LC microlaser are measured (Figure 2d).   


\begin{figure*}[htbp]
\centering
\includegraphics[width=12cm]{Figure_2.pdf}
\caption{\label{fig2}
Characterization of phase-transition microcavity laser. Typical lasing spectra of LC microdroplet with nematic (a) and isotropic phase (b). 
The radial ($q$) and angular ($l$) mode numbers are labeled with $\text{TM}^{q}_{l}$ and $\text{TE}^{q}_{l}$.\cite{gorodetsky2006geometrical} 
The insets show the corresponding  bright-field images (top) and polarized optical images (bottom). Scale bars, 5 \textmu m. (c) Threshold curves of nematic and isotropic LC microdroplets. (d) Statistics of the FSR from the same LC microdroplet in the nematic and isotropic phases.
}
\end{figure*}


The phase-transition microlaser is then exploited for thermal sensing. To explore the sensing performance at different phases, the temperature around the microdroplet is tuned precisely by the stage-top incubator with a proportional-integral-derivative (PID) controller. 
The lasing spectra are recorded and shown in Figure 3a, which depend highly on the environment temperature.  
It is found that the lasing wavelengths show slight blueshifts with increasing temperature, and then experience a sharp variation when approaching the phase transition point (42.7 $^{\circ}$C). The transition process from nematic to isotropic phase is observed by the lasing behaviour of microcavity.
This unique property promises an ultrahigh sensitivity for thermal measurement.      

A four-parameter model is employed to analyze this phase-transition behavior. 
The microlaser with TM modes features electric field oscillating along the long axis of the LC molecules, which corresponds to the extraordinary refractive index (${n_e}$) of the nematic LC microdroplet.
This refractive index ${n_e}$ depends highly on the temperature and can be expressed as (See Supporting Information)\cite{li2004temperature}

 
\begin{equation}
    {n_e}(T)=A-BT+\frac{2(\Delta{n})_0}{3}(1-\frac{T}{T_C})^\beta
\end{equation}
where ${A}$ and ${B}$ are the material constants that can be obtained by fitting the temperature-dependent average refractive index $<n>=(n_e+2n_o)/3$ ($n_o$ is the ordinary refractive index). $(\Delta{n})_0$ and ${\beta}$ are the birefringence at ${T}$=0 and material constant, respectively, which can be obtained from fitting the temperature-dependent birefringence $\Delta{n}$=${n_e}$-${n_o}$. ${T_C}$ is the phase-transition temperature (i.e., cleaning point).
It indicates that refractive index ${n_e}$ decreases with increasing temperature ${T}$, and the variation of ${n_e}$ is dominant by the last term when the LC microlaser is operated in the vicinity of the phase transition.
This enables a negative thermo-optic coefficient, which reaches the maximum near the phase transition (Figure S2). 
On the other hand, birefringence vanishes in the isotropic phase (i.e., $n_e=n_o$), and the refractive index decreases linearly with temperature.  
Therefore, the dependence of resonant wavelengths on temperature is calculated and shown in Figure 3b, in good agreement with the experimental results in Figure 3a.

To evaluate the sensitivity of the microlaser-based thermal sensor, the lasing wavelength is extracted from the spectrum by Gaussian fitting the experimental results.\cite{schubert2020monitoring} 
The dependence of lasing wavelength on temperature is shown in Figure 3c (blue circles). It is found that the lasing wavelength blueshifts faster and reaches the maximum of -18 nm when the temperature approaches the transition point.  
The sensitivity ${S}=-\Delta\lambda/\Delta{T}$ is shown in Figure 3c (orange circles). In the nematic phase, the sensitivity is only -0.4 nm/K, while it boosts to -40 nm/K in the vicinity of phase transition.
It shows two-order-of-magnitude improvement in sensitivity, which is ascribed to the reorientation of LC molecules from aligned to disordered states during the phase transition.
Moreover, the temporal response of this thermal sensor is studied by changing the temperature repeatedly (Figure 3d, grey curve).  
It is found that the evolution of lasing wavelength follows instantaneously the environmental temperature. 
Note that the phase-transition temperature of the microlaser can be tuned by tailoring different LC materials to satisfy the application requirement.\cite{van2012combinatorial}
The ultrahigh sensitivity in the vicinity of phase transition and the real-time response of LC microlaser make it quite suitable for monitoring the subtle fluctuation at a specific temperature. 


\begin{figure*}[htbp]
\centering
\includegraphics[width=12cm]{Figure_3.pdf}
\caption{\label{fig3}
Phase-transition-enhanced thermal sensing. (a) Experimental lasing spectra of LC microlaser at different temperatures. (b) Theoretical resonant wavelengths at different temperatures obtained from four-parameter model\cite{li2004temperature}. (c) Wavelength shift of a specific lasing mode (blue circles) and the corresponding sensitivity in thermal sensing (orange circles) extracted from a.
The error bar is obtained from five measurements. Solid curves, fitting results.  
(d) Temporal response of the lasing wavelength when operated near the phase-transition point (blue dots). The real-time temperature is indicated with the grey curve. 
}
\end{figure*}

 
To analyze the resolution for thermal sensing, the stability of the lasing wavelengths is studied. Its lasing spectra are monitored for 5 minutes at different temperatures, which are tuned from the nematic phase to approach the phase transition (Figure S4).
The fluctuations of lasing wavelengths in three representative temperatures are presented in Figure 4a. At temperatures of 32.8 $^\circ$C, 41.9 $^\circ$C, and 42.6 $^\circ$C, 
the lasing wavelengths remain nearly unchanged.
The fluctuations can be derived with the standard deviations of 9 pm, 17 pm, and 13 pm, indicating a minor change in noise floor.      
According to the thermal response in Figure 3c, the signal-to-noise ratio (SNR) of the LC microlaser at different operating temperatures is shown in Figure 4b.   
It is found that the SNR increases from 50 at 35 $^\circ$C to 2000 at 42.6 $^\circ$C, showing a 40-fold SNR improvement in the vicinity of the LC phase transition.
Accordingly, the temperature resolution of the LC microdroplet for thermal sensing can be derived, which reaches as low as 320 \textmu K in the vicinity of phase transition.


\begin{figure*}[htbp]
\centering
\includegraphics[width=12cm]{Figure_4.pdf}
\caption{\label{fig4}
SNR enhancement of the phase-transition microlaser for thermal sensing. (a) Stability of lasing wavelength when the LC microlaser operated at three representative temperatures. The standard deviations (s.d.) of lasing wavelengths are indicated. 
(b) SNR as a function of the operating temperature of the LC microlaser.
Circles, experimental data. Curve, fitting result. 
}
\end{figure*}

In conclusion, we have demonstrated a phase-transition microcavity laser using dye-doped LC microdroplets. 
The two phases of LC molecules and their transition process in the microcavity laser are observed experimentally, unveiling a dramatic wavelength shift of the microlaser in the vicinity of phase transition. 
This phase-transition microlaser is then applied for thermal sensing, exhibiting several new characteristics. First, the thermal sensitivity is enhanced by two-order-of-magnitude due to the phase transition, while background noises show no significant changes. Second, a temperature sensitivity of -40 nm/K and a resolution of 320 \textmu K are demonstrated, showing a distinct advantage compared to other works using the optical microcavity (Table S1).  
The stand-alone phase-transition microlaser holds great potential for high-precision temperature measurement in the life sciences such as intracellular thermal sensing.\cite{martino2019wavelength,tang2021laser,schubert2020monitoring}



%%%%%%%%%%%%%%%%%%%%%%%%%%%%%%%%%%%%%%%%%%%%%%%%%%%%%%%%%%%%%%%%%%%%%
%% The "Acknowledgement" section can be given in all manuscript
%% classes.  This should be given within the "acknowledgement"
%% environment, which will make the correct section or running title.
%%%%%%%%%%%%%%%%%%%%%%%%%%%%%%%%%%%%%%%%%%%%%%%%%%%%%%%%%%%%%%%%%%%%%
\begin{acknowledgement}

This project is supported by the National Key R$\&$D Program of China (Grant 2018YFB2200401), the National Natural Science Foundation of China (Grants 11825402, 12293051, 62205007, 62105006, 11674390). X. Yang is supported by the China Postdoctoral Science Foundation (Grant 2021M700208). S.-J. Tang is supported by the China Postdoctoral Science Foundation (Grants 2021T140023 and 2020M680187).

\end{acknowledgement}

%%%%%%%%%%%%%%%%%%%%%%%%%%%%%%%%%%%%%%%%%%%%%%%%%%%%%%%%%%%%%%%%%%%%%
%% The same is true for Supporting Information, which should use the
%% suppinfo environment.
%%%%%%%%%%%%%%%%%%%%%%%%%%%%%%%%%%%%%%%%%%%%%%%%%%%%%%%%%%%%%%%%%%%%%

%%%%%%%%%%%%%%%%%%%%%%%%%%%%%%%%%%%%%%%%%%%%%%%%%%%%%%%%%%%%%%%%%%%%%
%% The appropriate \bibliography command should be placed here.
%% Notice that the class file automatically sets \bibliographystyle
%% and also names the section correctly.
%%%%%%%%%%%%%%%%%%%%%%%%%%%%%%%%%%%%%%%%%%%%%%%%%%%%%%%%%%%%%%%%%%%%%
\providecommand{\latin}[1]{#1}
\makeatletter
\providecommand{\doi}
  {\begingroup\let\do\@makeother\dospecials
  \catcode`\{=1 \catcode`\}=2 \doi@aux}
\providecommand{\doi@aux}[1]{\endgroup\texttt{#1}}
\makeatother
\providecommand*\mcitethebibliography{\thebibliography}
\csname @ifundefined\endcsname{endmcitethebibliography}
  {\let\endmcitethebibliography\endthebibliography}{}
\begin{mcitethebibliography}{53}
\providecommand*\natexlab[1]{#1}
\providecommand*\mciteSetBstSublistMode[1]{}
\providecommand*\mciteSetBstMaxWidthForm[2]{}
\providecommand*\mciteBstWouldAddEndPuncttrue
  {\def\EndOfBibitem{\unskip.}}
\providecommand*\mciteBstWouldAddEndPunctfalse
  {\let\EndOfBibitem\relax}
\providecommand*\mciteSetBstMidEndSepPunct[3]{}
\providecommand*\mciteSetBstSublistLabelBeginEnd[3]{}
\providecommand*\EndOfBibitem{}
\mciteSetBstSublistMode{f}
\mciteSetBstMaxWidthForm{subitem}{(\alph{mcitesubitemcount})}
\mciteSetBstSublistLabelBeginEnd
  {\mcitemaxwidthsubitemform\space}
  {\relax}
  {\relax}

\bibitem[Lai \latin{et~al.}(2020)Lai, Suh, Lu, Shen, Yang, Wang, Li, Lee, Yang,
  and Vahala]{lai2020earth}
Lai,~Y.-H.; Suh,~M.-G.; Lu,~Y.-K.; Shen,~B.; Yang,~Q.-F.; Wang,~H.; Li,~J.;
  Lee,~S.~H.; Yang,~K.~Y.; Vahala,~K. Earth rotation measured by a chip-scale
  ring laser gyroscope. \emph{Nat. Photonics} \textbf{2020}, \emph{14},
  345--349\relax
\mciteBstWouldAddEndPuncttrue
\mciteSetBstMidEndSepPunct{\mcitedefaultmidpunct}
{\mcitedefaultendpunct}{\mcitedefaultseppunct}\relax
\EndOfBibitem
\bibitem[Li \latin{et~al.}(2018)Li, Jiang, Zhao, and Yang]{li2018whispering}
Li,~Y.; Jiang,~X.; Zhao,~G.; Yang,~L. Whispering gallery mode microresonator
  for nonlinear optics. \emph{arXiv} \textbf{2018}, 1809.04878\relax
\mciteBstWouldAddEndPuncttrue
\mciteSetBstMidEndSepPunct{\mcitedefaultmidpunct}
{\mcitedefaultendpunct}{\mcitedefaultseppunct}\relax
\EndOfBibitem
\bibitem[Strekalov \latin{et~al.}(2016)Strekalov, Marquardt, Matsko, Schwefel,
  and Leuchs]{strekalov2016nonlinear}
Strekalov,~D.~V.; Marquardt,~C.; Matsko,~A.~B.; Schwefel,~H.~G.; Leuchs,~G.
  Nonlinear and quantum optics with whispering gallery resonators. \emph{J.
  Opt.} \textbf{2016}, \emph{18}, 123002\relax
\mciteBstWouldAddEndPuncttrue
\mciteSetBstMidEndSepPunct{\mcitedefaultmidpunct}
{\mcitedefaultendpunct}{\mcitedefaultseppunct}\relax
\EndOfBibitem
\bibitem[Del’Haye \latin{et~al.}(2007)Del’Haye, Schliesser, Arcizet,
  Wilken, Holzwarth, and Kippenberg]{del2007optical}
Del’Haye,~P.; Schliesser,~A.; Arcizet,~O.; Wilken,~T.; Holzwarth,~R.;
  Kippenberg,~T.~J. Optical frequency comb generation from a monolithic
  microresonator. \emph{Nature} \textbf{2007}, \emph{450}, 1214--1217\relax
\mciteBstWouldAddEndPuncttrue
\mciteSetBstMidEndSepPunct{\mcitedefaultmidpunct}
{\mcitedefaultendpunct}{\mcitedefaultseppunct}\relax
\EndOfBibitem
\bibitem[Aspelmeyer \latin{et~al.}(2014)Aspelmeyer, Kippenberg, and
  Marquardt]{aspelmeyer2014cavity}
Aspelmeyer,~M.; Kippenberg,~T.~J.; Marquardt,~F. Cavity optomechanics.
  \emph{Rev. Mod. Phys.} \textbf{2014}, \emph{86}, 1391--1452\relax
\mciteBstWouldAddEndPuncttrue
\mciteSetBstMidEndSepPunct{\mcitedefaultmidpunct}
{\mcitedefaultendpunct}{\mcitedefaultseppunct}\relax
\EndOfBibitem
\bibitem[Zhang \latin{et~al.}(2021)Zhang, Peng, Kim, Monifi, Jiang, Li, Yu,
  Liu, Liu, Al{\`u}, and Yang]{zhang2021optomechanical}
Zhang,~J.; Peng,~B.; Kim,~S.; Monifi,~F.; Jiang,~X.; Li,~Y.; Yu,~P.; Liu,~L.;
  Liu,~Y.-X.; Al{\`u},~A.; Yang,~L. Optomechanical dissipative solitons.
  \emph{Nature} \textbf{2021}, \emph{600}, 75--80\relax
\mciteBstWouldAddEndPuncttrue
\mciteSetBstMidEndSepPunct{\mcitedefaultmidpunct}
{\mcitedefaultendpunct}{\mcitedefaultseppunct}\relax
\EndOfBibitem
\bibitem[{\"O}zdemir \latin{et~al.}(2019){\"O}zdemir, Rotter, Nori, and
  Yang]{ozdemir2019parity}
{\"O}zdemir,~{\c{S}}.~K.; Rotter,~S.; Nori,~F.; Yang,~L. Parity-time symmetry
  and exceptional points in photonics. \emph{Nat. Mater.} \textbf{2019},
  \emph{18}, 783--798\relax
\mciteBstWouldAddEndPuncttrue
\mciteSetBstMidEndSepPunct{\mcitedefaultmidpunct}
{\mcitedefaultendpunct}{\mcitedefaultseppunct}\relax
\EndOfBibitem
\bibitem[Cao and Wiersig(2015)Cao, and Wiersig]{cao2015dielectric}
Cao,~H.; Wiersig,~J. Dielectric microcavities: model systems for wave chaos and
  non-Hermitian physics. \emph{Rev. Mod. Phys.} \textbf{2015}, \emph{87},
  61--111\relax
\mciteBstWouldAddEndPuncttrue
\mciteSetBstMidEndSepPunct{\mcitedefaultmidpunct}
{\mcitedefaultendpunct}{\mcitedefaultseppunct}\relax
\EndOfBibitem
\bibitem[Toropov \latin{et~al.}(2021)Toropov, Cabello, Serrano, Gutha, Rafti,
  and Vollmer]{toropov2021review}
Toropov,~N.; Cabello,~G.; Serrano,~M.~P.; Gutha,~R.~R.; Rafti,~M.; Vollmer,~F.
  Review of biosensing with whispering-gallery mode lasers. \emph{Light Sci.
  Appl.} \textbf{2021}, \emph{10}, 42\relax
\mciteBstWouldAddEndPuncttrue
\mciteSetBstMidEndSepPunct{\mcitedefaultmidpunct}
{\mcitedefaultendpunct}{\mcitedefaultseppunct}\relax
\EndOfBibitem
\bibitem[Liu \latin{et~al.}(2020)Liu, Chen, Tang, Vollmer, and
  Xiao]{liu2020nonlinear}
Liu,~W.; Chen,~Y.-L.; Tang,~S.-J.; Vollmer,~F.; Xiao,~Y.-F. Nonlinear sensing
  with whispering-gallery mode microcavities: From label-free detection to
  spectral fingerprinting. \emph{Nano Lett.} \textbf{2020}, \emph{21},
  1566--1575\relax
\mciteBstWouldAddEndPuncttrue
\mciteSetBstMidEndSepPunct{\mcitedefaultmidpunct}
{\mcitedefaultendpunct}{\mcitedefaultseppunct}\relax
\EndOfBibitem
\bibitem[Basiri-Esfahani \latin{et~al.}(2019)Basiri-Esfahani, Armin, Forstner,
  and Bowen]{basiri2019precision}
Basiri-Esfahani,~S.; Armin,~A.; Forstner,~S.; Bowen,~W.~P. Precision ultrasound
  sensing on a chip. \emph{Nat. Commun.} \textbf{2019}, \emph{10}, 132\relax
\mciteBstWouldAddEndPuncttrue
\mciteSetBstMidEndSepPunct{\mcitedefaultmidpunct}
{\mcitedefaultendpunct}{\mcitedefaultseppunct}\relax
\EndOfBibitem
\bibitem[Liu \latin{et~al.}(2022)Liu, Bo, Chang, Dong, Ou, Regan, Shen, Song,
  Yao, Zhang, Zou, and Xiao]{liu2022emerging}
Liu,~J.; Bo,~F.; Chang,~L.; Dong,~C.-H.; Ou,~X.; Regan,~B.; Shen,~X.; Song,~Q.;
  Yao,~B.; Zhang,~W.; Zou,~C.-L.; Xiao,~Y.-F. Emerging material platforms for
  integrated microcavity photonics. \emph{Sci. China-Phys. Mech. Astron.}
  \textbf{2022}, \emph{65}, 104201\relax
\mciteBstWouldAddEndPuncttrue
\mciteSetBstMidEndSepPunct{\mcitedefaultmidpunct}
{\mcitedefaultendpunct}{\mcitedefaultseppunct}\relax
\EndOfBibitem
\bibitem[Wang \latin{et~al.}(2018)Wang, Zhang, Stern, Lipson, and
  Lon{\v{c}}ar]{wang2018nanophotonic}
Wang,~C.; Zhang,~M.; Stern,~B.; Lipson,~M.; Lon{\v{c}}ar,~M. Nanophotonic
  lithium niobate electro-optic modulators. \emph{Opt. Express} \textbf{2018},
  \emph{26}, 1547--1555\relax
\mciteBstWouldAddEndPuncttrue
\mciteSetBstMidEndSepPunct{\mcitedefaultmidpunct}
{\mcitedefaultendpunct}{\mcitedefaultseppunct}\relax
\EndOfBibitem
\bibitem[Zhang \latin{et~al.}(2019)Zhang, Buscaino, Wang, Shams-Ansari, Reimer,
  Zhu, Kahn, and Lon{\v{c}}ar]{zhang2019broadband}
Zhang,~M.; Buscaino,~B.; Wang,~C.; Shams-Ansari,~A.; Reimer,~C.; Zhu,~R.;
  Kahn,~J.~M.; Lon{\v{c}}ar,~M. Broadband electro-optic frequency comb
  generation in a lithium niobate microring resonator. \emph{Nature}
  \textbf{2019}, \emph{568}, 373--377\relax
\mciteBstWouldAddEndPuncttrue
\mciteSetBstMidEndSepPunct{\mcitedefaultmidpunct}
{\mcitedefaultendpunct}{\mcitedefaultseppunct}\relax
\EndOfBibitem
\bibitem[Wang \latin{et~al.}(2022)Wang, Jia, Chen, Cheng, Ni, Guo, Li, Liu,
  Hao, Ning, Zhao, Lv, Huang, Xie, and Zhu]{wang20222}
Wang,~X.; Jia,~K.; Chen,~M.; Cheng,~S.; Ni,~X.; Guo,~J.; Li,~Y.; Liu,~H.;
  Hao,~L.; Ning,~J.; Zhao,~G.; Lv,~X.; Huang,~S.-W.; Xie,~Z.; Zhu,~S.-N. 2
  $\mu$m optical frequency comb generation via optical parametric oscillation
  from a lithium niobate optical superlattice box resonator. \emph{Photonics
  Res.} \textbf{2022}, \emph{10}, 509--515\relax
\mciteBstWouldAddEndPuncttrue
\mciteSetBstMidEndSepPunct{\mcitedefaultmidpunct}
{\mcitedefaultendpunct}{\mcitedefaultseppunct}\relax
\EndOfBibitem
\bibitem[Zhu \latin{et~al.}(2021)Zhu, Shao, Yu, Cheng, Desiatov, Xin, Hu,
  Holzgrafe, Ghosh, Shams-Ansari, Puma, Sinclair, Reimer, Zhang, and
  Lon{\v{c}}ar]{zhu2021integrated}
Zhu,~D.; Shao,~L.; Yu,~M.; Cheng,~R.; Desiatov,~B.; Xin,~C.; Hu,~Y.;
  Holzgrafe,~J.; Ghosh,~S.; Shams-Ansari,~A.; Puma,~E.; Sinclair,~N.;
  Reimer,~C.; Zhang,~M.; Lon{\v{c}}ar,~M. Integrated photonics on thin-film
  lithium niobate. \emph{Adv. Opt. Photonics} \textbf{2021}, \emph{13},
  242--352\relax
\mciteBstWouldAddEndPuncttrue
\mciteSetBstMidEndSepPunct{\mcitedefaultmidpunct}
{\mcitedefaultendpunct}{\mcitedefaultseppunct}\relax
\EndOfBibitem
\bibitem[Lu \latin{et~al.}(2020)Lu, Li, Zou, Al~Sayem, and Tang]{lu2020toward}
Lu,~J.; Li,~M.; Zou,~C.-L.; Al~Sayem,~A.; Tang,~H.~X. Toward 1\% single-photon
  anharmonicity with periodically poled lithium niobate microring resonators.
  \emph{Optica} \textbf{2020}, \emph{7}, 1654--1659\relax
\mciteBstWouldAddEndPuncttrue
\mciteSetBstMidEndSepPunct{\mcitedefaultmidpunct}
{\mcitedefaultendpunct}{\mcitedefaultseppunct}\relax
\EndOfBibitem
\bibitem[Lin \latin{et~al.}(2019)Lin, Yao, Hao, Zhang, Mao, Wang, Chu, Wu,
  Fang, Qiao, Fang, Bo, and Ya]{lin2019broadband}
Lin,~J.; Yao,~N.; Hao,~Z.; Zhang,~J.; Mao,~W.; Wang,~M.; Chu,~W.; Wu,~R.;
  Fang,~Z.; Qiao,~L.; Fang,~W.; Bo,~F.; Ya,~C. Broadband quasi-phase-matched
  harmonic generation in an on-chip monocrystalline lithium niobate microdisk
  resonator. \emph{Phys. Rev. Lett.} \textbf{2019}, \emph{122}, 173903\relax
\mciteBstWouldAddEndPuncttrue
\mciteSetBstMidEndSepPunct{\mcitedefaultmidpunct}
{\mcitedefaultendpunct}{\mcitedefaultseppunct}\relax
\EndOfBibitem
\bibitem[Zheng \latin{et~al.}(2019)Zheng, Fang, Liu, Cheng, and
  Chen]{zheng2019high}
Zheng,~Y.; Fang,~Z.; Liu,~S.; Cheng,~Y.; Chen,~X. High-Q exterior
  whispering-gallery modes in a double-layer crystalline microdisk resonator.
  \emph{Phys. Rev. Lett.} \textbf{2019}, \emph{122}, 253902\relax
\mciteBstWouldAddEndPuncttrue
\mciteSetBstMidEndSepPunct{\mcitedefaultmidpunct}
{\mcitedefaultendpunct}{\mcitedefaultseppunct}\relax
\EndOfBibitem
\bibitem[Huang \latin{et~al.}(2020)Huang, Zhang, Xiao, Wang, Fan, Liu, Zhang,
  Qu, Ji, Han, Ge, Kivshar, and Song]{huang2020ultrafast}
Huang,~C.; Zhang,~C.; Xiao,~S.; Wang,~Y.; Fan,~Y.; Liu,~Y.; Zhang,~N.; Qu,~G.;
  Ji,~H.; Han,~J.; Ge,~L.; Kivshar,~Y.; Song,~Q. Ultrafast control of vortex
  microlasers. \emph{Science} \textbf{2020}, \emph{367}, 1018--1021\relax
\mciteBstWouldAddEndPuncttrue
\mciteSetBstMidEndSepPunct{\mcitedefaultmidpunct}
{\mcitedefaultendpunct}{\mcitedefaultseppunct}\relax
\EndOfBibitem
\bibitem[Gao \latin{et~al.}(2018)Gao, Huang, Hao, Sun, Zhang, Zhang, Duan,
  Wang, Jin, Zhang, Kildishev, Qiu, Song, and Xiao]{gao2018lead}
Gao,~Y.; Huang,~C.; Hao,~C.; Sun,~S.; Zhang,~L.; Zhang,~C.; Duan,~Z.; Wang,~K.;
  Jin,~Z.; Zhang,~N.; Kildishev,~A.~V.; Qiu,~C.-W.; Song,~Q.; Xiao,~S. Lead
  halide perovskite nanostructures for dynamic color display. \emph{ACS Nano}
  \textbf{2018}, \emph{12}, 8847--8854\relax
\mciteBstWouldAddEndPuncttrue
\mciteSetBstMidEndSepPunct{\mcitedefaultmidpunct}
{\mcitedefaultendpunct}{\mcitedefaultseppunct}\relax
\EndOfBibitem
\bibitem[Miao \latin{et~al.}(2020)Miao, Cheng, Zou, Gu, Zhang, Guo, Peng, Xu,
  He, Zhang, Cao, Li, Wang, Huang, and Wang]{miao2020microcavity}
Miao,~Y.; Cheng,~L.; Zou,~W.; Gu,~L.; Zhang,~J.; Guo,~Q.; Peng,~Q.; Xu,~M.;
  He,~Y.; Zhang,~S.; Cao,~Y.; Li,~R.; Wang,~N.; Huang,~W.; Wang,~J. Microcavity
  top-emission perovskite light-emitting diodes. \emph{Light Sci. Appl.}
  \textbf{2020}, \emph{9}, 89\relax
\mciteBstWouldAddEndPuncttrue
\mciteSetBstMidEndSepPunct{\mcitedefaultmidpunct}
{\mcitedefaultendpunct}{\mcitedefaultseppunct}\relax
\EndOfBibitem
\bibitem[Zhang \latin{et~al.}(2021)Zhang, Shang, Su, Do, and
  Xiong]{zhang2021halide}
Zhang,~Q.; Shang,~Q.; Su,~R.; Do,~T. T.~H.; Xiong,~Q. Halide perovskite
  semiconductor lasers: materials, cavity design, and low threshold. \emph{Nano
  Lett.} \textbf{2021}, \emph{21}, 1903--1914\relax
\mciteBstWouldAddEndPuncttrue
\mciteSetBstMidEndSepPunct{\mcitedefaultmidpunct}
{\mcitedefaultendpunct}{\mcitedefaultseppunct}\relax
\EndOfBibitem
\bibitem[Mysliwiec \latin{et~al.}(2021)Mysliwiec, Szukalska, Szukalski, and
  Sznitko]{mysliwiec2021liquid}
Mysliwiec,~J.; Szukalska,~A.; Szukalski,~A.; Sznitko,~L. Liquid crystal lasers:
  the last decade and the future. \emph{Nanophotonics} \textbf{2021},
  \emph{10}, 2309--2346\relax
\mciteBstWouldAddEndPuncttrue
\mciteSetBstMidEndSepPunct{\mcitedefaultmidpunct}
{\mcitedefaultendpunct}{\mcitedefaultseppunct}\relax
\EndOfBibitem
\bibitem[Coles and Morris(2010)Coles, and Morris]{coles2010liquid}
Coles,~H.; Morris,~S. Liquid-crystal lasers. \emph{Nat. Photonics}
  \textbf{2010}, \emph{4}, 676--685\relax
\mciteBstWouldAddEndPuncttrue
\mciteSetBstMidEndSepPunct{\mcitedefaultmidpunct}
{\mcitedefaultendpunct}{\mcitedefaultseppunct}\relax
\EndOfBibitem
\bibitem[Humar(2016)]{humar2016liquid}
Humar,~M. Liquid-crystal-droplet optical microcavities. \emph{Liq. Cryst.}
  \textbf{2016}, \emph{43}, 1937--1950\relax
\mciteBstWouldAddEndPuncttrue
\mciteSetBstMidEndSepPunct{\mcitedefaultmidpunct}
{\mcitedefaultendpunct}{\mcitedefaultseppunct}\relax
\EndOfBibitem
\bibitem[Ma \latin{et~al.}(2022)Ma, Li, Pan, Ji, Jiang, Zheng, Wang, Wang, Li,
  and Lu]{ma2022self}
Ma,~L.-L.; Li,~C.-Y.; Pan,~J.-T.; Ji,~Y.-E.; Jiang,~C.; Zheng,~R.; Wang,~Z.-Y.;
  Wang,~Y.; Li,~B.-X.; Lu,~Y.-Q. Self-assembled liquid crystal architectures
  for soft matter photonics. \emph{Light Sci. Appl.} \textbf{2022}, \emph{11},
  270\relax
\mciteBstWouldAddEndPuncttrue
\mciteSetBstMidEndSepPunct{\mcitedefaultmidpunct}
{\mcitedefaultendpunct}{\mcitedefaultseppunct}\relax
\EndOfBibitem
\bibitem[Xiang \latin{et~al.}(2016)Xiang, Varanytsia, Minkowski, Paterson,
  Storey, Imrie, Lavrentovich, and Palffy-Muhoray]{xiang2016electrically}
Xiang,~J.; Varanytsia,~A.; Minkowski,~F.; Paterson,~D.~A.; Storey,~J.~M.;
  Imrie,~C.~T.; Lavrentovich,~O.~D.; Palffy-Muhoray,~P. Electrically tunable
  laser based on oblique heliconical cholesteric liquid crystal. \emph{Proc.
  Natl. Acad. Sci. U.S.A.} \textbf{2016}, \emph{113}, 12925--12928\relax
\mciteBstWouldAddEndPuncttrue
\mciteSetBstMidEndSepPunct{\mcitedefaultmidpunct}
{\mcitedefaultendpunct}{\mcitedefaultseppunct}\relax
\EndOfBibitem
\bibitem[Chen \latin{et~al.}(2022)Chen, Zheng, Yang, Li, Jin, Li, Wang, and
  Jiang]{chen2022over}
Chen,~Y.; Zheng,~C.; Yang,~W.; Li,~J.; Jin,~F.; Li,~X.; Wang,~J.; Jiang,~L.
  Over 200 $^\circ$C broad-temperature lasers reconstructed from a blue-phase
  polymer scaffold. \emph{Adv. Mater.} \textbf{2022}, \emph{34}, 2206580\relax
\mciteBstWouldAddEndPuncttrue
\mciteSetBstMidEndSepPunct{\mcitedefaultmidpunct}
{\mcitedefaultendpunct}{\mcitedefaultseppunct}\relax
\EndOfBibitem
\bibitem[Adamow \latin{et~al.}(2020)Adamow, Szukalski, Sznitko, Persano,
  Pisignano, Camposeo, and Mysliwiec]{adamow2020electrically}
Adamow,~A.; Szukalski,~A.; Sznitko,~L.; Persano,~L.; Pisignano,~D.;
  Camposeo,~A.; Mysliwiec,~J. Electrically controlled white laser emission
  through liquid crystal/polymer multiphases. \emph{Light Sci. Appl.}
  \textbf{2020}, \emph{9}, 19\relax
\mciteBstWouldAddEndPuncttrue
\mciteSetBstMidEndSepPunct{\mcitedefaultmidpunct}
{\mcitedefaultendpunct}{\mcitedefaultseppunct}\relax
\EndOfBibitem
\bibitem[Zhang \latin{et~al.}(2020)Zhang, Yuan, Qiao, Barshilia, Wang, Chang,
  and Chen]{zhang2020tunable}
Zhang,~Y.; Yuan,~Z.; Qiao,~Z.; Barshilia,~D.; Wang,~W.; Chang,~G.-E.;
  Chen,~Y.-C. Tunable microlasers modulated by intracavity spherical
  confinement with chiral liquid crystal. \emph{Adv. Opt. Mater.}
  \textbf{2020}, \emph{8}, 1902184\relax
\mciteBstWouldAddEndPuncttrue
\mciteSetBstMidEndSepPunct{\mcitedefaultmidpunct}
{\mcitedefaultendpunct}{\mcitedefaultseppunct}\relax
\EndOfBibitem
\bibitem[Papi{\v{c}} \latin{et~al.}(2021)Papi{\v{c}}, Mur, Zuhail, Ravnik,
  Mu{\v{s}}evi{\v{c}}, and Humar]{papivc2021topological}
Papi{\v{c}},~M.; Mur,~U.; Zuhail,~K.~P.; Ravnik,~M.; Mu{\v{s}}evi{\v{c}},~I.;
  Humar,~M. Topological liquid crystal superstructures as structured light
  lasers. \emph{Proc. Natl. Acad. Sci. U.S.A.} \textbf{2021}, \emph{118},
  e2110839118\relax
\mciteBstWouldAddEndPuncttrue
\mciteSetBstMidEndSepPunct{\mcitedefaultmidpunct}
{\mcitedefaultendpunct}{\mcitedefaultseppunct}\relax
\EndOfBibitem
\bibitem[Ali \latin{et~al.}(2022)Ali, Elston, Lin, and Morris]{ali2022demand}
Ali,~T.; Elston,~S.~J.; Lin,~J.-D.; Morris,~S.~M. On-demand polarization
  controllable liquid crystal laser. \emph{Adv. Mater. Technol.} \textbf{2022},
  2200674\relax
\mciteBstWouldAddEndPuncttrue
\mciteSetBstMidEndSepPunct{\mcitedefaultmidpunct}
{\mcitedefaultendpunct}{\mcitedefaultseppunct}\relax
\EndOfBibitem
\bibitem[Zhan \latin{et~al.}(2021)Zhan, Xu, Zhou, Yan, Yao, and
  Zhao]{zhan20213d}
Zhan,~X.; Xu,~F.-F.; Zhou,~Z.; Yan,~Y.; Yao,~J.; Zhao,~Y.~S. 3D laser displays
  based on circularly polarized lasing from cholesteric liquid crystal arrays.
  \emph{Adv. Mater.} \textbf{2021}, \emph{33}, 2104418\relax
\mciteBstWouldAddEndPuncttrue
\mciteSetBstMidEndSepPunct{\mcitedefaultmidpunct}
{\mcitedefaultendpunct}{\mcitedefaultseppunct}\relax
\EndOfBibitem
\bibitem[Wang \latin{et~al.}(2021)Wang, Gong, Zhang, Qiao, Yuan, Gong, Chang,
  Tu, and Chen]{wang2021programmable}
Wang,~C.; Gong,~C.; Zhang,~Y.; Qiao,~Z.; Yuan,~Z.; Gong,~Y.; Chang,~G.-E.;
  Tu,~W.-C.; Chen,~Y.-C. Programmable rainbow-colored optofluidic fiber laser
  encoded with topologically structured chiral droplets. \emph{ACS Nano}
  \textbf{2021}, \emph{15}, 11126--11136\relax
\mciteBstWouldAddEndPuncttrue
\mciteSetBstMidEndSepPunct{\mcitedefaultmidpunct}
{\mcitedefaultendpunct}{\mcitedefaultseppunct}\relax
\EndOfBibitem
\bibitem[Franklin \latin{et~al.}(2021)Franklin, Ueltschi, Carlini, Yao, Reeder,
  Richards, Van~Duyne, and Rogers]{franklin2021bioresorbable}
Franklin,~D.; Ueltschi,~T.; Carlini,~A.; Yao,~S.; Reeder,~J.; Richards,~B.;
  Van~Duyne,~R.~P.; Rogers,~J.~A. Bioresorbable microdroplet lasers as
  injectable systems for transient thermal sensing and modulation. \emph{ACS
  Nano} \textbf{2021}, \emph{15}, 2327--2339\relax
\mciteBstWouldAddEndPuncttrue
\mciteSetBstMidEndSepPunct{\mcitedefaultmidpunct}
{\mcitedefaultendpunct}{\mcitedefaultseppunct}\relax
\EndOfBibitem
\bibitem[Humar \latin{et~al.}(2017)Humar, Dobravec, Zhao, and
  Yun]{humar2017biomaterial}
Humar,~M.; Dobravec,~A.; Zhao,~X.; Yun,~S.~H. Biomaterial microlasers
  implantable in the cornea, skin, and blood. \emph{Optica} \textbf{2017},
  \emph{4}, 1080--1085\relax
\mciteBstWouldAddEndPuncttrue
\mciteSetBstMidEndSepPunct{\mcitedefaultmidpunct}
{\mcitedefaultendpunct}{\mcitedefaultseppunct}\relax
\EndOfBibitem
\bibitem[Hu \latin{et~al.}(2020)Hu, Wu, Li, Chen, Forsberg, and He]{hu2020snr}
Hu,~S.; Wu,~S.; Li,~C.; Chen,~R.; Forsberg,~E.; He,~S. SNR-enhanced
  temperature-insensitive microfiber humidity sensor based on upconversion
  nanoparticles and cellulose liquid crystal coating. \emph{Sens. Actuators, B}
  \textbf{2020}, \emph{305}, 127517\relax
\mciteBstWouldAddEndPuncttrue
\mciteSetBstMidEndSepPunct{\mcitedefaultmidpunct}
{\mcitedefaultendpunct}{\mcitedefaultseppunct}\relax
\EndOfBibitem
\bibitem[Gong \latin{et~al.}(2021)Gong, Qiao, Yuan, Huang, Wang, Wu, and
  Chen]{gong2021topological}
Gong,~C.; Qiao,~Z.; Yuan,~Z.; Huang,~S.-H.; Wang,~W.; Wu,~P.~C.; Chen,~Y.-C.
  Topological encoded vector beams for monitoring amyloid-lipid interactions in
  microcavity. \emph{Adv. Sci.} \textbf{2021}, \emph{8}, 2100096\relax
\mciteBstWouldAddEndPuncttrue
\mciteSetBstMidEndSepPunct{\mcitedefaultmidpunct}
{\mcitedefaultendpunct}{\mcitedefaultseppunct}\relax
\EndOfBibitem
\bibitem[Wang \latin{et~al.}(2021)Wang, Xu, Noel, Chen, and
  Liu]{wang2021applications}
Wang,~Z.; Xu,~T.; Noel,~A.; Chen,~Y.-C.; Liu,~T. Applications of liquid
  crystals in biosensing. \emph{Soft Matter} \textbf{2021}, \emph{17},
  4675--4702\relax
\mciteBstWouldAddEndPuncttrue
\mciteSetBstMidEndSepPunct{\mcitedefaultmidpunct}
{\mcitedefaultendpunct}{\mcitedefaultseppunct}\relax
\EndOfBibitem
\bibitem[Duan \latin{et~al.}(2019)Duan, Li, Li, and Yang]{duan2019detection}
Duan,~R.; Li,~Y.; Li,~H.; Yang,~J. Detection of heavy metal ions using
  whispering gallery mode lasing in functionalized liquid crystal
  microdroplets. \emph{Biomed. Opt. Express} \textbf{2019}, \emph{10},
  6073--6083\relax
\mciteBstWouldAddEndPuncttrue
\mciteSetBstMidEndSepPunct{\mcitedefaultmidpunct}
{\mcitedefaultendpunct}{\mcitedefaultseppunct}\relax
\EndOfBibitem
\bibitem[Wang \latin{et~al.}(2018)Wang, Zhao, Xu, Wang, Zhang, Liu, Liu, and
  Li]{wang2018detecting}
Wang,~Y.; Zhao,~L.; Xu,~A.; Wang,~L.; Zhang,~L.; Liu,~S.; Liu,~Y.; Li,~H.
  Detecting enzymatic reactions in penicillinase via liquid crystal
  microdroplet-based pH sensor. \emph{Sens. Actuators, B} \textbf{2018},
  \emph{258}, 1090--1098\relax
\mciteBstWouldAddEndPuncttrue
\mciteSetBstMidEndSepPunct{\mcitedefaultmidpunct}
{\mcitedefaultendpunct}{\mcitedefaultseppunct}\relax
\EndOfBibitem
\bibitem[Yang \latin{et~al.}(2021)Yang, Chen, Cao, Duan, Chen, Yu, and
  Xiao]{yang2021operando}
Yang,~D.-Q.; Chen,~J.-h.; Cao,~Q.-T.; Duan,~B.; Chen,~H.-J.; Yu,~X.-C.;
  Xiao,~Y.-F. Operando monitoring transition dynamics of responsive polymer
  using optofluidic microcavities. \emph{Light: Science $\&$ Applications}
  \textbf{2021}, \emph{10}, 128\relax
\mciteBstWouldAddEndPuncttrue
\mciteSetBstMidEndSepPunct{\mcitedefaultmidpunct}
{\mcitedefaultendpunct}{\mcitedefaultseppunct}\relax
\EndOfBibitem
\bibitem[Jiang \latin{et~al.}(2020)Jiang, Qavi, Huang, and
  Yang]{jiang2020whispering}
Jiang,~X.; Qavi,~A.~J.; Huang,~S.~H.; Yang,~L. Whispering-gallery sensors.
  \emph{Matter} \textbf{2020}, \emph{3}, 371--392\relax
\mciteBstWouldAddEndPuncttrue
\mciteSetBstMidEndSepPunct{\mcitedefaultmidpunct}
{\mcitedefaultendpunct}{\mcitedefaultseppunct}\relax
\EndOfBibitem
\bibitem[Humar \latin{et~al.}(2009)Humar, Ravnik, Pajk, and
  Mu{\v{s}}evi{\v{c}}]{humar2009electrically}
Humar,~M.; Ravnik,~M.; Pajk,~S.; Mu{\v{s}}evi{\v{c}},~I. Electrically tunable
  liquid crystal optical microresonators. \emph{Nat. Photonics} \textbf{2009},
  \emph{3}, 595--600\relax
\mciteBstWouldAddEndPuncttrue
\mciteSetBstMidEndSepPunct{\mcitedefaultmidpunct}
{\mcitedefaultendpunct}{\mcitedefaultseppunct}\relax
\EndOfBibitem
\bibitem[Yang and Wu(2014)Yang, and Wu]{yang2014fundamentals}
Yang,~D.-K.; Wu,~S.-T. \emph{Fundamentals of liquid crystal devices}, 2nd ed.;
  John Wiley \& Sons: Chichester, United Kingdom, 2014\relax
\mciteBstWouldAddEndPuncttrue
\mciteSetBstMidEndSepPunct{\mcitedefaultmidpunct}
{\mcitedefaultendpunct}{\mcitedefaultseppunct}\relax
\EndOfBibitem
\bibitem[Gorodetsky and Fomin(2006)Gorodetsky, and
  Fomin]{gorodetsky2006geometrical}
Gorodetsky,~M.~L.; Fomin,~A.~E. Geometrical theory of whispering-gallery modes.
  \emph{IEEE J. Sel. Top. Quantum Electron.} \textbf{2006}, \emph{12},
  33--39\relax
\mciteBstWouldAddEndPuncttrue
\mciteSetBstMidEndSepPunct{\mcitedefaultmidpunct}
{\mcitedefaultendpunct}{\mcitedefaultseppunct}\relax
\EndOfBibitem
\bibitem[Li \latin{et~al.}(2004)Li, Gauza, and Wu]{li2004temperature}
Li,~J.; Gauza,~S.; Wu,~S.-T. Temperature effect on liquid crystal refractive
  indices. \emph{J. Appl. Phys.} \textbf{2004}, \emph{96}, 19--24\relax
\mciteBstWouldAddEndPuncttrue
\mciteSetBstMidEndSepPunct{\mcitedefaultmidpunct}
{\mcitedefaultendpunct}{\mcitedefaultseppunct}\relax
\EndOfBibitem
\bibitem[Schubert \latin{et~al.}(2020)Schubert, Woolfson, Barnard, Dorward,
  Casement, Morton, Robertson, Appleton, Miles, Tucker, Pitt, and
  Gather]{schubert2020monitoring}
Schubert,~M.; Woolfson,~L.; Barnard,~I.~R.; Dorward,~A.~M.; Casement,~B.;
  Morton,~A.; Robertson,~G.~B.; Appleton,~P.~L.; Miles,~G.~B.; Tucker,~C.~S.;
  Pitt,~S.~J.; Gather,~M.~C. Monitoring contractility in cardiac tissue with
  cellular resolution using biointegrated microlasers. \emph{Nat. Photonics}
  \textbf{2020}, \emph{14}, 452--458\relax
\mciteBstWouldAddEndPuncttrue
\mciteSetBstMidEndSepPunct{\mcitedefaultmidpunct}
{\mcitedefaultendpunct}{\mcitedefaultseppunct}\relax
\EndOfBibitem
\bibitem[Van~der Werff \latin{et~al.}(2012)Van~der Werff, Robinson, and
  Kyratzis]{van2012combinatorial}
Van~der Werff,~L.~C.; Robinson,~A.~J.; Kyratzis,~I.~L. Combinatorial approach
  for the rapid determination of thermochromic behavior of binary and ternary
  cholesteric liquid crystalline mixtures. \emph{ACS Comb. Sci.} \textbf{2012},
  \emph{14}, 605--612\relax
\mciteBstWouldAddEndPuncttrue
\mciteSetBstMidEndSepPunct{\mcitedefaultmidpunct}
{\mcitedefaultendpunct}{\mcitedefaultseppunct}\relax
\EndOfBibitem
\bibitem[Martino \latin{et~al.}(2019)Martino, Kwok, Liapis, Forward, Jang, Kim,
  Wu, Wu, Dannenberg, Jang, and Yun]{martino2019wavelength}
Martino,~N.; Kwok,~S.~J.; Liapis,~A.~C.; Forward,~S.; Jang,~H.; Kim,~H.-M.;
  Wu,~S.~J.; Wu,~J.; Dannenberg,~P.~H.; Jang,~S.-J.; Yun,~S.-H.
  Wavelength-encoded laser particles for massively multiplexed cell tagging.
  \emph{Nat. Photonics} \textbf{2019}, \emph{13}, 720--727\relax
\mciteBstWouldAddEndPuncttrue
\mciteSetBstMidEndSepPunct{\mcitedefaultmidpunct}
{\mcitedefaultendpunct}{\mcitedefaultseppunct}\relax
\EndOfBibitem
\bibitem[Tang \latin{et~al.}(2021)Tang, Dannenberg, Liapis, Martino, Zhuo,
  Xiao, and Yun]{tang2021laser}
Tang,~S.-J.; Dannenberg,~P.~H.; Liapis,~A.~C.; Martino,~N.; Zhuo,~Y.;
  Xiao,~Y.-F.; Yun,~S.-H. Laser particles with omnidirectional emission for
  cell tracking. \emph{Light Sci. Appl.} \textbf{2021}, \emph{10}, 23\relax
\mciteBstWouldAddEndPuncttrue
\mciteSetBstMidEndSepPunct{\mcitedefaultmidpunct}
{\mcitedefaultendpunct}{\mcitedefaultseppunct}\relax
\EndOfBibitem
\end{mcitethebibliography}


\end{document}
