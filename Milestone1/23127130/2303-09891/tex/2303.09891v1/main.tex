\documentclass[fleqn,10pt]{wlscirep}
\usepackage[utf8]{inputenc}
\usepackage[T1]{fontenc}
\usepackage{xcolor}
\title{The role of fluid's viscoelasticity on the self-assembly of particle chains in simple shear flow}

\author[1]{Matthew G. Smith}
\author[2]{Graham M. Gibson}
\author[1]{Andreas Link}
\author[1]{Thomas Franke}
\author[1,*]{Manlio Tassieri}
\affil[1]{Division of Biomedical Engineering, James Watt School of Engineering, University of Glasgow, Glasgow G12 8LT, U.K.}
\affil[2]{School of Physics and Astronomy, University of Glasgow, Glasgow G12 8QQ, U.K.}
\affil[*]{manlio.tassieri@glasgow.ac.uk}


%\keywords{Keyword1, Keyword2, Keyword3}

\begin{abstract}
Flow-Induced Self-Assembly (FISA) is the phenomena of particle chaining in viscoelastic fluids while experiencing shear flow. FISA has many applications across many fields including material science, food processing and the biomedical field. Although this phenomena is currently not fully understood it is accepted that the shear-thinning nature of the complex fluid plays a role in enhancing FISA. In this work, a bespoke cone and plate shear cell is used to provide new insights on the FISA dynamics. In particular, we have fine tuned the applied shear rates to investigate the particle chaining phenomenon across a Weissenberg number of $1$, on a particle suspension made by a viscoelastic fluid characterised by a clear transition between a Newtonian-like behaviour (at relatively low shear rates) and a shear-thinning one (at relatively high shear rates). 
This has allowed us to reveal phenomena never reported previously in literature before; e.g., we have corroborated that for Weissenberg number higher than $1$, the shear thinning nature of the viscoelastic fluid strongly enhances FISA and that particle chains break apart when a constant shear is applied for sufficiently long-time (i.e. much longer than the fluids' longest relaxation time). We speculate that this latter point could be explained by a ``\textit{disentanglement}'' process of the polymer chains constituting the viscoelastic fluid.

\end{abstract}
\begin{document}

\flushbottom
\maketitle
% * <john.hammersley@gmail.com> 2015-02-09T12:07:31.197Z:
%
%  Click the title above to edit the author information and abstract
%
\thispagestyle{empty}

%\noindent Please note: Abbreviations should be introduced at the first mention in the main text – no abbreviations lists. Suggested structure of main text (not enforced) is provided below.

\section*{Introduction}
Flow-Induced Self-Assembly (FISA) of single particles into long chains while subjected to shear flow is a phenomenon that has been discussed at length since its first description in the $1977$ paper by Michele \textit{et al.}\cite{michele1977}. FISA phenomena occur frequently across a variety of applications, e.g.: (i) in material science, it is well documented that the inclusion of micro and nano particles in polymer melts can greatly enhance the final mechanical properties of products~\cite{vaia2001}; (ii) in food processing, the addition of soft microspheres or microgel droplets can be used to encapsulate phytonutrients for targeted delivery in the gut \cite{shewan2013}; and  (iii) in microfluidics, particle alignment is often required to enhance processes such as counting, analysis and separation \cite{delGiudice2013particle}. 

Currently, the exact mechanism that causes micro-particles to align in simple shear flow is unclear, and the focus of the debate between different schools of thought is mainly on the relative contribution to the driving force governing the phenomena by the elastic and the viscous forces generated during flow, due to the viscoelastic nature of complex fluids. It follows that, most of the arguments have been developed around the relative value assumed by the Weissenberg number ($Wi$), which is a dimensionless parameter used in rheology studies to describe the ratio between the elastic and the viscous forces; as further explicated later in this manuscript for the case of simple shear flow. For a general overview of the field, an up-to-date review has been masterly drawn by D'Avino and Maffettone \cite{d2015particle}, whose highlights are summarised hereafter for the convenience of the reader. The aforementioned work by Michele \textit{et al.}\cite{michele1977} reported for the first time glass beads forming into long chains and aggregations in a viscoelastic media, under both oscillatory and pipe flows; suggesting that (i) the alignment of particles could be related to the fluid's normal stresses (a measure of the fluid's elastic character) and that (ii) a critical Weissenberg value of $10$ is necessary for the alignment to occur. Subsequent studies by Petit and Noetinger \cite{petit1988shear}, and Lyon \textit{et al}. \cite{lyon2001structure} further corroborated Michele's observation in the case of string formation. Conversely, a more recent study by Won and Kim \cite{won2004alignment} suggests that the shear-thinning nature of the suspension fluid is the driving force for FISA, while normal stresses facilitate migration. Furthermore, Scirocco \textit{et al}. \cite{scirocco2004effect} found that a critical Weissenberg number (as low as $1$) is not solely responsible for string formation as they observed no alignment in Boger fluids (i.e., a viscoelastic fluid with a constant viscosity value). Interestingly, by varying the gap distance between their parallel plate flow cell, Scirocco \textit{et al}. \cite{scirocco2004effect} also found that FISA is a phenomena that occurs within the bulk of the fluid, rather than being a wall effect. However, in contrast to these findings, other studies \cite{pasquino2010effect,pasquino2013migration} observed single particles migration towards the walls, where they would assemble and form strings in the flow direction, when suspended in weakly viscoelastic liquids (i.e., $Wi<<1$). Nonetheless, a recent study by Pasquino \textit{et al}. \cite{pasquino2014} has shown that FISA occurs in both the bulk of the fluid and at the walls of the system; thus implying that such phenomenon is a convoluted function of specific parameters of the system being analysed, such as fluids' viscoelastic properties and flow cell geometries.

In this work, a bespoke, counter-rotating cone and plate shear cell has been used to analyse the effects that fluid's viscoelasticity has on FISA. This has been investigated by exploring shear rates that spanned across the critical value of $Wi=1$ of a water based solution of Polyacrylamide (PAM), whose frequency-dependent viscoelastic moduli have been determined by means of microrheology measurements performed with optical tweezers. In agreement with previous works in literature, we have observed that particle chains form in the bulk of the fluid and in the flow direction. Moreover, we show that for $Wi>1$, where the fluid shows its shear-thinning behaviour, FISA is significantly enhanced; nevertheless, a significant alignment was also observed at $Wi<1$. Nonetheless, as a means of novelty, we report for the first time in literature evidence of a spontaneous reduction in particles' chain length at relatively long times, which is not associated with migration, but could be associated to mechanical `\textit{degradation}' of the inter-molecular interactions between PAM molecules.


\section*{Materials and Methods} 

\subsection*{Particle Suspension}
A dilute solution of particle suspension was prepared by dispersing $5.2\mu$m diameter polystyrene beads (Bangs Laboratories), at a final concentration of $0.02\%$ w/v, in a water based solution of Polyacrylamide (PAM – molecular weight $18$M, Polysciences Inc.) at a final concentration of $0.07\%$ wt. 

\subsection*{Optical tweezers rig}
Microrheology measurements were performed by using an OT system based on a continuous wave, diode pumped solid state (DPSS) laser (Ventus, Laser Quantum), which provided up to $3$ W at $1,064$ nm. A nematic liquid crystal spatial light modulator (SLM) (BNS, XY series $512\times512$) was used to create and arrange the desired optical trap. The laser entered a custom-made inverted microscope that uses a microscope objective lens (Nikon, $100$x, $1.3$ NA) to both focus the trapping beam and to image the thermal fluctuations of a $4.74\mu$m diameter silica bead (Bangs Laboratories), at room temperature $\sim20^o$C. The sample was mounted on a motorized microscope stage (ASI, MS-$2000$). A complementary metal-oxide semiconductor (CMOS) camera (Dalsa, Genie HM $1024$ GigE) acquired high-speed images of a reduced field-of-view. These images were processed in real-time at up to $\sim 1$ kHz to calculate the center of mass of the bead by using a particle tracking software developed in LabVIEW (National Instruments), running on a standard desktop PC \cite{bowman2014red,gibson2008measuring}.


\subsection*{A Theoretical Background of Microrheology with Optical Tweezers}

Microrheology is an branch of Rheology (the study of the flow of matter), and it is focused on the characterisation of the viscoelastic properties of complex fluids by using sample volumes in the micro-litre range; thus making microrheological methods ideal candidates for measuring rare or precious samples, with a clear advantage over classical bulk rheology approaches that require millilitres of sample volume. Microrheology techniques are categorised into either ``passive'' or ``active'' depending on whether the tracer particle, suspended in the target fluid, is driven by thermal fluctuations within the fluid, or by means of an external force, respectively.  

Developed in the 1970s \cite{Ashkin1970}, optical tweezers (OT) utilise a monochromatic laser beam, focused through a microscope objective with a high numerical aperture, to optically trap in three dimensions a micron sized particle, suspended in a fluid; a schematic representation is presented in Fig\ref{fig:MicroPAM}-(A). Once trapped, the particle `feels' a harmonic potential, therefore the restoring force exerted on the particle is linearly proportional to the distance from the center of the trap, provided the displacement is within the bead diameter, and it is of the order of a few $\mu$N. In this work, passive microrheology with OT (MOT) has been used to measure the viscoelastic properties of a PAM solution at a concentration of $0.07\%$ wt. The thermal fluctuations of an optically trapped particle were analysed by means of the theoretical framework developed by Tassieri \textit{et al}.\cite{Tassieri2015_MOT, tassieri2016microrheology,Tassieri2019}, which is here summarised for convenience of the reader.

The Brownian motion of an optically trapped particle can uncover the viscoelastic properties of the suspending fluid when its trajectory is analysed by means of a generalised Langevin equation (Eqn.\ref{eqn:Langevin}) -- as first established by Mason and Weitz \cite{Mason1995} for the case of a freely diffusing particle -- which in this case reads: 
\begin{equation}
    m\vec{a}(t)=\vec{f_R}(t)-\int_{0}^{t}\xi(t-\tau)\vec{v}(\tau)d\tau-\kappa\vec{r}(t),
    \label{eqn:Langevin}
\end{equation}
where $m$ is the mass of the particle, $\vec{a}(t)$ is its acceleration, $\vec{v}(\tau)$ is its velocity, $\vec{r}(t)$ is its position, $\vec{f_R}(t)$ is the Gaussian white noise term used for modelling the stochastic forces acting on the particle, and $\xi(t)$ is the generalised time-dependent memory function accounting for the viscoelastic nature of the fluid.

As described by by Tassieri \textit{et al}.\cite{Tassieri2015_MOT, tassieri2016microrheology,Tassieri2019, Smith2021}, Eqn.\ref{eqn:Langevin} can be solved for the fluid's complex modulus ($G^*(\omega)$) \textit{via} either the normalised mean squared displacement (NMSD) $\Pi(\tau)=\langle\Delta r^2(\tau)\rangle / 2\left\langle r^2\right\rangle$ or the normalised position autocorrelation function (NPAF) $A(\tau)=\left\langle \vec{r}(t)\vec{r}(t+\tau)\right\rangle / \left\langle r^2\right\rangle$, which are simply related to each another as:
\begin{equation}
    \Pi(\tau)=\frac{\langle \Delta r^2(\tau)\rangle_{t_0}}{2\langle r^2\rangle_{eq.}}\equiv \frac{\langle [r(t_0+\tau)-r(t_0)]^2\rangle_{t_0}}{2\langle r^2\rangle_{eq.}} =1-A(\tau),
    \label{eqn:NMSD}
\end{equation}
where $\tau$ is the lag-time ($t-t_0$) and the brackets $\langle...\rangle_{t_0}$ represent an average over all initial times $t_0$.
The fluid's complex modulus can then be expressed as:
\begin{equation}
\label{eqn:G*OT}
	G^*(\omega)\frac{6\pi a}{\kappa}=\left(\frac{1}{i\omega \hat{\Pi}(\omega)}-1\right)\equiv \left(\frac{1}{i\omega\hat{A}(\omega)}-1\right)^{-1}\equiv \frac{\hat{A}(\omega)}{\hat{\Pi}(\omega)}
\end{equation}
where $G^*(\omega)$ is a complex number, whose real and imaginary parts define the elastic ($G'(\omega)$) and the viscous ($G''(\omega)$) moduli of the fluid, respectively; $a$ is the particle radius, $\kappa$ is the optical trap stiffness, $\hat{\Pi}(\omega)$ and $\hat{A}(\omega)$ are the Fourier transforms of the NMSD and the NPAF, respectively. One should note that, in order to obtain Eqn.\ref{eqn:G*OT}, the inertial term ($m\omega^{2}$) present in the original works\cite{Tassieri2010,Preece2011} has been here neglected, because for micron-sized particles it only becomes significant at frequencies of the order of MHz. From Eqn.\ref{eqn:G*OT} it is a trivial step to derive the complex viscosity ($\eta^*(\omega)$) of the fluid:
\begin{equation}
    |\eta^*(\omega)| = \frac{\sqrt{G'^2(\omega)+G''^2(\omega)}}{\omega}.
\end{equation}

\begin{figure}[t!]
    \centering
    \includegraphics[trim={0 0 0 0},clip,width=\textwidth]{Figure1.pdf}
    \caption{(A) A picture of the bespoke shear cell mounted on a microscope stage with two driving motors. (B) An exploded schematic representation of the shear cell showing the transparent cone (grey area) and plate (blue area) geometries. (C) Calibration curve comparing the rotational frequency of the cone and the plate ($\omega_{c/p}$) versus the rotational frequency of the motors ($\omega_M$). The conversion factor for the cone and the plate was $0.263$ and $0.205$, respectively. (D) Example of a typical frame captured during experiments and the same image post processing. In red are the single beads, dimers and one trimer identified by using the particle tracking software developed in LabVIEW (National Instruments) for this work, running on a standard desktop PC.}
    \label{fig:Calibration}
\end{figure}

\subsection*{Shear Cell}
The shear cell used in this work had a bespoke cone and plate design, both of which were transparent allowing the use of an optical microscope to capture images and the generation of a uniform shear rate along the diameter of the shear cell. An image and an exploded schematic representation of the setup is shown in Fig.\ref{fig:Calibration}-(A,B). The shear cell is positioned on top of a microscope stage and is composed of several individual parts including: the 3D printed base, the shear cell mount, the plate body, and the cone, which slot together creating a chamber for the fluid being analysed. As shown in Fig.\ref{fig:Calibration}-(B), the base of the shear cell (black) is 3D printed that allows the setup to fit within the microscope stage and hence accurately position the shear cell within the optical path of the microscope. The mount for the shear cell is a ring that screws into the 3D printed base and houses a bearing around its internal diameter. On this bearing sits the plate body allowing the plate to rotate freely. The plate body also has a bearing on which the cone body sits. The fluid is applied between the cone and the plate, which in Fig.\ref{fig:Calibration}-(B) is schematically represented by the blue area. The rotation of the cone and the plate was driven by two independent stepper motors via two nitrile O-rings as shown in Fig.\ref{fig:Calibration}-(A). Motor control was provided by two individual Arduino boards, each driving an A4988 stepper motor controller, interfaced with a Labview program that allowed us to control motor speed. 

Prior to performing the measurements, a calibration of the relative speed between the two electric motors and the related cone and plate geometries was performed.
This consisted in varying the rotational frequency of the motors and measuring those of the related geometries, which resulted to be much lower because of the relatively high gear ratio.
The rotational frequency of the driven cone/plate ($\omega_{c/p}$) can be calculated using the Eqn.\ref{eqn:RotatFreq}, which is based on the diameter ratio between motor and the cone/plate, $d_m$ and $d_{c/p}$ respectively.

\begin{equation}
    \omega_{c/p} = \frac{d_m}{d_{c/p}}\omega_M,
    \label{eqn:RotatFreq}
\end{equation}
where $\omega_M$ is the rotational frequency of the motor shaft. 
%\begin{figure}[h]
%    \centering
%    \includegraphics[width=\textwidth]{ShearCellDiagram.pdf}
%    \caption{Diagram showing the relation of cone/plate rotational frequency ($\omega$c/p) to the ratio of disk diameters, for the motor (dm) and the cone/plate (dc/p), and to the rotational frequency of the motor shaft ($\omega$m). }
%    \label{fig:ShearCellDiagram}
%\end{figure}
% Could probably replace this figure with just the equation

Calibration was measured in three configurations: (i) with either the cone or the plate stationary, (ii) with both rotating in the same direction and (iii) with both rotating in opposite directions. The purpose of measuring in each configuration was to make sure that the rotation of the cone did not influence the rotation of the plate and \textit{vice versa}. In each configuration the time required for the cone and the plate to complete 10 revolutions was measured. The rotational frequency of both the cone and the plate were graphed against the rotational frequency of the motors, shown in Fig.\ref{fig:Calibration}-(C), and the gradient of the line is the inverse of the gear ratio.

 The shear cell was counter rotated, and the shear rates explored during the measurements are those reported in Table\ref{tab:ShearRate}, which shows also the rotational frequency of the cone driving motor ($\omega_{cM}$) and the plate driving motor ($\omega_{pM}$) required to achieve an equal rotational frequency of the cone and the plate ($\omega_{c/p}$) for each of the shear rate examined:
\begin{equation}
    \dot{\gamma} = \frac{\omega_{c/p}}{\alpha},
\end{equation}
where $\dot{\gamma}$ is the shear rate and $\alpha$ is the angle of the cone (i.e. $2^\circ$)\cite{schmid1973counter}.

\begin{table}[ht]
\centering
\begin{tabular}{|l|l|l|l|}
\hline
$\dot{\gamma }$ $[s^{-1}]$ & $\omega_{c/p}$ [Hz] & $\omega_{cM}$ [Hz] & $\omega_{pM}$ [Hz] \\
\hline
0.08 & 0.16 & 0.61 & 0.78 \\
\hline
0.12 & 0.24 & 0.91 & 1.17 \\
\hline
0.20 & 0.40 & 1.52 & 1.95 \\
\hline
0.32 & 0.64 & 2.43 & 3.12 \\
\hline
0.49 & 0.98 & 3.73 & 4.78 \\
\hline
0.78 & 1.56 & 5.93 & 7.61 \\
\hline
\end{tabular}
\caption{\label{tab:ShearRate} Table of shear rates and rotational frequencies explored in this work.}
\end{table}

The sample, $750\mu$l of $0.07\%$wt PAM-bead solution, was inserted between the cone and the glass coverslip (plate) and then a constant shear rates was applied for $30$ minutes.
Images were acquired using a Dalsa Teledyne Genie camera at $960$fps with an exposure time of $400\mu$s. The high frame rate was achieved by reducing the region of interest of the camera from $1400 \times 1024$ pixels to $752 \times 100$ pixels. The window was aligned in the direction of flow, so that chaining could be observed through the major axis of the window. A Labview program has been developed to allow us to run the camera at $960$fps, but capture images at set intervals in milliseconds, thus improving significantly the performance of the image acquisition and reducing the time taken for analysis. The acquisition rate of the Labview program was set to $1.2$s for all experiments, thus returning $1500$ frames for each measurement; which were performed in triplicates. Fig.\ref{fig:Calibration}-(D) shows a typical image frame captured both before and after image processing.

Image analysis was carried out using a Labview program that was able to identify particles, frame by frame, measuring properties such as pixel area and elongation to count the number of single beads, dimers, trimers, tetras and pentas. The numbers of each chain length per frame could then be used to analyse the chaining phenomenon.


\section*{Results and Discussion}

\subsection*{Microrheology with Optical Tweezers of a Polyacrylamide solution}

The rheological properties of the PAM solution described above were measured by means of microrheology measurements performed with optical tweezers and here reported in Fig.\ref{fig:MicroPAM}. The trajectory of a suspended particle was captured at circa $1$kHz for circa $10$mins. The NPAF of the particle trajectory was analysed by using i-Rheo MOT \cite{Tassieri2012}, an algorithm based on the analytical method developed by Evans \textit{et al}.\cite{evans2009direct} for evaluating the Fourier transform of any generic function, sampled over a finite time window, without the need for Laplace transforms or fitting functions. For more detail about the principles underpinning i-Rheo MOT, the reader is advised to read these references \cite{tassieri2016rheo,Smith2021,Tassieri2019,Tassieri2018}.
\begin{figure}[ht]
    \centering
    \includegraphics[trim={0 0 0 0},clip,width=\textwidth]{Figure2.pdf}
    \caption{(A) x and y components of the trajectory of an optically trapped particle. (B) Normalised Position Autocorrelation Function (NPAF) versus lag-time ($\tau$) calculated using the x-component of the trajectory shown in (A). (C) A schematic representation of an optically trapped bead within a harmonic potential, $E(\vec{r})$, where $\kappa$ is the trap stiffness and $\vec{r}$ is the bead position from the trap centre. (D) The viscoelastic moduli and the complex viscosity of $0.07\%$wt PAM solution. The vertical dotted lines represent the shear rates used in the experiments.}
    \label{fig:MicroPAM}
\end{figure}

From Fig.\ref{fig:MicroPAM}-(B), it is clear the existence of a characteristic time $\lambda$ identified by the inverse value of the characteristic frequency at which the two moduli crossover (i.e., $\lambda^{-1}_D=\omega_c=0.32$ rad/s). Interestingly, at frequencies lower than $\omega_c$ the complex viscosity shows a Newtonian (i.e., frequency-independent) behaviour with the viscous modulus $G''(\omega)$ dominating the elastic one $G'(\omega)$ for small frequencies.
Whereas, at frequencies higher than $\omega_c$ the complex viscosity shows a power law behaviour, with both the moduli growing with frequency as circa $\omega^{0.5}$, and therefore $\eta^*(\omega)\propto \omega^{-0.5}$ revealing the shear-thinning nature of the solution.
The characteristic time $\lambda$ can be used to evaluate the Weissenberg number ($Wi$), which is a dimensionless number defined as the ratio between the elastic and the viscous forces\cite{ferry1980}:
\begin{equation}
    Wi = \frac{Elastic Forces}{Viscous Forces} = \dot{\gamma}\lambda,
\end{equation}
where $\dot{\gamma}$ is the shear rate.

Additionally, a $97\%$wt glycerol/water mixture with beads concentration of $0.02\%$ w/v was used as a control. This ratio of glycerol/water mixture was chosen based on the fact that its Newtonian viscosity of $\eta=0.765$ Pa$\cdot$s closely matches the one of the PAM solution at the plateau value.

\subsection*{Accumulation of Particles}

FISA phenomenon can be analysed by calculating the accumulation of chains, of a given length, over each successive image frame. The analysis can be thought of as a series of `bins' into which different chain lengths are added. For example, The number of chains in each bin can easily be summed over time to produce information on the long time effect of shear rate on the chaining process occurring within the sample. As mentioned earlier, the image analysis software separated the particles/chains identified in each image frame into $5$ different `bins': i.e., singles, dimers, trimers, tetras and pentas. The curves shown in Fig.\ref{fig:Accumulation}-(A) are the total number of particles collected in each `bin' over time, at each imposed shear rate.

\begin{figure}[ht]
    \centering
    \includegraphics[trim={0 50 0 50},clip,width=\textwidth]{Figure3.pdf}
    \caption{(A) Accumulation of particles for each chain length \textit{versus} time for shear rates ranging from $0.08-0.78$s$^{-1}$. Each curve has been down-sampled for easier identification and the dashed black line indicates a linear growth. (B) Mean rate of accumulation (MROA), normalised by the maximum MROA for each particle chain length \textit{versus} the Wiessenberg number. The colours associated with each curve represent a particle chain length as for by the outline of the single, dimer, trimer, tetra and penta images on the right side. The outputs shown in this figure were obtained in triplicates.}
    \label{fig:Accumulation}
\end{figure}
%Could add this as a secondary figure??

From Fig.\ref{fig:Accumulation}-(A) it is clear and expected that the accumulation of particles increases with time. However, it is interesting to notice that the curves settle into clear ``bands'' depending on the relative value of the Weissenberg number. Indeed, from Fig.\ref{fig:Accumulation}-(A) it is clear that the accumulation curves overlay on each other at relatively low shear rates (i.e., for $0.08-0.32$s$^{-1}$ or equivalently for Wi$<1$), but they branch off at the highest shear rates measured (i.e., for $0.49-0.78$s$^{-1}$ or equivalently for Wi$>1$).
This is apparent for the dark blue symbols in Fig.\ref{fig:Accumulation}-(A) representative of the pentas; specifically, for the shear rates at $0.49$ and $0.78$s$^{-1}$, respectively.
Moreover, it is interesting to notice that the increase in accumulation for the longer chains at the highest shear rates, seems to be reflected by a decrease of the accumulation rate of single particles at the same shear rates.
This observation is corroborated by the analysis of the mean rate of accumulation (MROA) of the curves shown in Fig.\ref{fig:Accumulation}-(A). In particular, the MROA is evaluated by taking the mean value of the time derivative of the curves shown in Fig\ref{fig:Accumulation}-(A) and normalise it by its maximum value. In Fig.\ref{fig:Accumulation}-(B) we report the MROA \textit{versus} Wi, and from this we can observe that the MROA of single particles decreases with increasing Wi, whereas the MROA for dimers stays relatively constant across the same range of explored Wi. In contrary, the MROA for the longer chain particles is low and constant for Wi values of less than $1$, with a steep increase after this value.

Notice that, in these experiments the concentration of individual polystyrene beads is kept constant, and therefore as chains of different length start to form, the number of single particles decreases.
It follows that, at low Wi, particles are not able to form long chains and therefore the MROA for single particles is high; whereas, as the shear rate increases to a point where the Wi is greater than $1$, the shear-thinning behaviour of the PAM solution enhances the chaining process, which produces a significant drop in the MROA of the singles, while the MROA of the trimers, tetras and pentas increases rapidly. 

\subsection*{Relative Population of Chains}
A complementary method to analyse the progression of FISA within the sample at different shear rates, is by identifying the relative population of each chain length in the image frame. In order to achieve this, the percentage of particles was calculated by taking the ratio between each chain length and the total number of particles identified in each frame, as shown in Fig.\ref{fig:Percentage}. 

\begin{figure}[ht]
    \centering
    \includegraphics[trim={0 0 0 0},clip,width=\textwidth]{Figure4.pdf}
    \caption{Relative population of particle chains (singles, dimers, trimers, tetras and pentas) \textit{versus} time. (A and B) Particle suspension in glycerol/water mixture at a shear rates of $0.08$ and $0.78$s$^{-1}$, respectively. (C-H) PAM solutions at shear rates of $0.08$, $0.12$, $0.20$, $0.32$, $0.49$ and $0.78$s$^{-1}$, respectively. Note that, the shaded areas are the standard deviation associated with the experimental triplicates and each curve has been smoothed by using a moving average window of $30$s.}
    \label{fig:Percentage}
\end{figure}

In particular, in Fig.\ref{fig:Percentage}-(A-B) are reported the relative population of particle chains for suspensions made with glycerol/water mixture and measured at the two extremes of the range of explored shear rates (i.e., $0.08$ and $0.78$s$^{-1}$). From these diagrams it is apparent that no significant changes occur and that the mean percentage of single particles identified in the image frames stays well above $80\%$; whereas, for all the other particle chain sizes, it remains well below $20\%$. Thus confirming that chaining doesn't occur in Newtonian fluids.
Interestingly, this is not the case of particle suspensions in the viacoelastic solution employed in this work. Indeed, as shown in Fig.\ref{fig:Percentage}-(C-F), there is an initial fall of the mean percentage of single particles whose magnitude increases with the applied shear rate. This is followed by a corresponding rise in the mean percentage of longer chains, all within the first $5$ minutes from the start of the experiments. Then the percentage of single particles begins to climb back to an almost steady value for the duration of the experiment, which is higher than $80\%$ for Wi$\leq 1$ and lower than $80\%$ for Wi$> 1$.
A similar, but opposite behaviour is seen for the longer particle chains, where an initial increase of their population is observed over the same time scale (i.e., $5$min), followed by a decrease towards a steady value.
One could argue that, a possible explanation of such dynamic process could be related to the migration of single particles and chains from/into the focal plane; however, the complementary of these processes between the single particles and the chains (i.e., the decrease/increase in single particles is complemented by the increase/decrease in the percentage of dimers, trimers, tetras and pentas) infers (i) that the total number of particles is constant during the measurement and (ii) that longer chain particles may breaking down back to single particles.  

\subsection*{Alignment Factor}
An additional method to analyse the progression of FISA is by means of the alignment factor ($A_f$), as described by Pasquino \textit{et al}. \cite{pasquino2013migration}, which is defined as:
\begin{equation}
    A_f = \frac{\sum_{L=1}^{L_{max}}N_LL^2}{\sum_{L=1}^{L_{max}}N_LL},
\end{equation}
where $L$ is the chain length and $N_L$ is the number of chains of a given length $L$ in an image frame.
As suggested by its name, $A_f$ is a measure of bead alignment in a given image frame and will always have a value $\geq 1$, where $A_f = 1$ is achievable \textit{if and only if} there were solely single beads in the image frame. As stated by Pasquino \textit{et al}. \cite{pasquino2013migration}, $A_f$ bares a resemblance to the weight average molecular weight of polymer chains and as such chains of longer length have greater impact on $A_f$. 

\begin{figure}[ht]
    \centering
    \includegraphics[trim={0 0 0 0},clip,width=\textwidth]{Figure5.pdf}
    \caption{(A) Alignment factor curves (averaged over triplicates) for each shear rate measured including the measurements performed on the $97\%$ glycerol/water mixture (Gly) as control (dotted lines). The red dashed lines represent the fitting function calculated by using Eq.\ref{eqn:asym2sig}. Notice that, each curve has been smoothed by using a moving average window of $60$s width. (B) The fitting parameters $y_0$ and $A$ \textit{versus} the Weissenberg number. (C) The fitting parameters $x_c$ (left axis), $w_2$ and $w_3$ (right axis) \textit{versus} Wi. The standard deviation of the fitting function is depicted as error bars in (B) and (C). (D) The normalised number of particles \textit{versus} time, used to analyse the migration of particles away from the focal plane. Here, each curve has been smoothed by using a moving average window of $30$s width.}
    \label{fig:AlignmentFactor}
\end{figure}
%Need error bars

In Fig.\ref{fig:AlignmentFactor}-(A) we report the alignment factor for the same set of experiments analysed earlier in Fig.\ref{fig:Accumulation} and \ref{fig:Percentage}.
From Fig.\ref{fig:AlignmentFactor}-(A) it can be seen that at relatively short times, all curves show an increase of $A_f$ up to a maximum value, whose amplitude increases proportionally to the shear rate, while its abscissa is inversely proportional to $\dot{\gamma}$ for Wi$\leq 1$ and remains almost constant for Wi$> 1$. After reaching a maximum, all the curves tend to exponentially decrease towards a steady-state value at long times.
Notably, this is the first time in literature that such behaviour of $A_f$ is reported, as it was expected that $A_f$ would have increased monotonically until a plateau value at long times, as described by Pasquino \textit{et al}. \cite{pasquino2013migration,pasquino2014}. However, it must be said that, their focal plane was positioned at the wall of the shear cell and therefore particle migration\cite{pasquino2010effect,pasquino2013migration} towards the plate continued to supply the area with new beads. In this work, the focal plane was placed at the centre of the fluid chamber and therefore we again posit that the decrease of $A_f$ at long-times may be due to either (i) particle migration away from the centre of the channel (this being a possible explanation that is not supported by the experimental evidence reported in this work), or (ii) longer chain particles breaking down to single particles (which is the thesis we support).

%Additionally, the overshoot $A_f$ increases with increasing shear rate for the 0.07$\%$ PAM, with a significantly greater increase in this value between a shear rate of 0.32s$^{-1}$ and 0.49s$^{-1}$. From Fig.\ref{fig:MicroPAM}-(B), a shear rate of 0.32s$^{-1}$ corresponds to the crossover of G' and G'' and hence the point when 0.07$\%$ PAM starts to exhibit shear thinning behaviour. However, even at shear rates much lower than the crossover (0.08-0.20s$^{-1}$) there is still alignment occurring within the fluid that cannot be explained by the shear thinning quality of the sample. Comparing these curves to those acquired for the purely viscous Glycerol, at the highest and lowest shear rates analysed, gives an insight into what could be causing this alignment at lower shear rates. As shown in Fig.\ref{fig:AlignmentFactor}-(A), the $A_f$ of the PAM when subjected to a shear rate of 0.08s$^{-1}$ (solid dark blue line) is greater than that of the Glycerol when subjected to the same shear rate. Consequently, it is clear that the material properties must affect alignment of the particles. However, as previously described in Fig.\ref{fig:MicroPAM}-(B), 0.08s$^{-1}$ is far below the crossover of G' and G" and therefore at this shear rate the PAM is not experiencing shear-thinning while alignment of particles is occurring albeit only slightly more than the Glycerol. One could argue that because the Glycerol shows the same alignment, when subjected to 0.78s$^{-1}$, as the PAM sample at 0.08 and 0.12s$^{-1}$, then the viscoelasticity of the fluid below the crossover does not play a significant role in alignment. Nonetheless, to reach an $A_f$ value that is comparable to the PAM at the lowest shear rate of 0.08s$^{-1}$, the glycerol requires a shear rate that is almost 10 times greater, implying that the viscoelasticity of the PAM significantly enhances alignment even when it is not in its shear-thinning regime. 

In order to better understand the temporal behaviour of $A_f$, we performed a best fit of the curves by means of the following asymmetric double sigmoid function\cite{chen2017thermogravimetric} (also known as piecewise logistic function):
\begin{equation}
    y = y_0 + A\frac{1}{1+e^{-\frac{x-x_c+w_1/2}{w_2}}}\left(1-\frac{1}{1+e^{-\frac{x-x_c-w_1/2}{w_3}}}\right),
    \label{eqn:asym2sig}
\end{equation}
where $y_0$ is the steady state value, $A$ is the amplitude, $x_c$ is the peak center, $w_1$ is the curve width, $w_2$ and $w_3$ are shape parameters.
For the fitting curves shown in Fig.\ref{fig:AlignmentFactor}-(A), it was found that $w_1=0$ for all fits; whereas, the remaining parameters were different from zero and they are plotted against Wi in Fig.\ref{fig:AlignmentFactor}-(B) and (C). 
From these figures, it can be seen that $y_0$ is almost constant for Wi$\leq 1$, with a significant increase for Wi$>1$. Whereas, the amplitude $A$ shows a progressive increase with Wi; although, it could be argued that there is an initial shallow increase for Wi$<1$ and then a relatively sharp increase for Wi$>1$.
From Fig.\ref{fig:AlignmentFactor}-(C) one will notice that $x_c$ (the peak centre) behaves rather erratically when plotted against the Weissenberg number. Indeed, it initially decreases for Wi$\leq 1$ (which corresponds to a reduction in time for the peak to occur in Fig.\ref{fig:AlignmentFactor}-(A)); however, as the Weissenberg number exceeds $1$, there is first a sharp increase in $x_c$ and then it starts decreasing again.
Interestingly, a similar behaviour is shown by $w_2$.
Finally, $w_3$ is the `shape parameter' for the curve after its peak, and an increase in its value would suggest an increase of the characteristic ``\textit{relaxation time}'' of $A_f$. Thus suggesting that the degradation of the particle chains is controlled by a different physical process than the one one governing the generation of particle chains.

Overall, the above analysis of the fitting parameters indicates a clear change in behaviour of the particle chaining phenomena as the Weissenberg number exceeds $1$; with a sharp enhancement of the chaining process due to the shear-thinning nature of the fluid rather that the existence of an elastic component because of its viscoelastic nature.

\subsection*{Migration of Particles}

Both the analysis of the relative chain length population and the $A_f$ have raised the same question: is the decrease in FISA at long-times due to the migration of longer chain particles away from the focal plane or is it due to the break down of these same particle chains back to single particles? 

In order to address this question, we have followed the approach introduced by Mirsepassi \textit{et al}. \cite{mirsepassi2012particle}, whereby one could reveal the existence of migration during flow by normalising the change in the number of particles in each frame (independently on whether they are single or belonging to chains of different length) to the average number of particles in each image determined over the entire measurement duration. In Fig.\ref{fig:AlignmentFactor}-(D) we report the results of such analysis with a moving average window of $30$s width. Notably, despite the existence of significant fluctuations, all curves fluctuates around a constant value equal to $1$, which suggests that the average number of particles in the bulk of the PAM solution does not change during the measurements; hence, migration of particles out of the focal plane does not occur. This result further corroborate our hypothesis that the reduction in FISA at long-times is likely to be due to the longer particle chains breaking down back into shorter chain lengths. 

This poses a further question: what would cause the particles chains to break down? A possible cause would involve the mechanical ``\textit{degradation}'' of the PAM when exposed to constant shear rate over a long period of time. There has been significant work across disciplines investigating the mechanical degradation of PAM \cite{mansour2014situ,xiong2018polyacrylamide,rho1996degradation,al2013rheology,vazquez2001shear}, albeit at significantly greater shear rates than those utilised here and therefore it is unlikely that the PAM used in this study experiences fission of the polymer chains as described in literature. A more likely scenario is related to the ``\textit{disentanglement}'' process of the polymer chain, as described by Vazquez \textit{et al.} \cite{vazquez2001shear}. This phenomenon would occur on time scales much longer than $\lambda$ and would produce a drop in the intrinsic viscosity of the PAM, thus leading to the breakdown of the longer chain particles.
%Certainly, significantly more work specifically targeting this phenomenon is needed to fully explain this long time reduction in $A_f$.

\section*{Conclusions} 

In this work we have studied the flow-induced self-assembly (FISA) process of particles suspended into both Newtonian and viscoelastic fluids. This has been achieved by means of a bespoke shear cell that has allowed us to monitor the dynamics of the particle chain `generation' and for the first time in literature `degradation'. 
In particular, the adoption of a viscoelastic fluid characterised by a clear transition between a Newtonian-like behaviour (i.e., with a constant viscosity) at relatively low frequencies (or equivalently at low shear rates) and a shear-thinning character at relatively high frequencies (or shear rates) and the fine tuning of the applied shear rates, have allowed us to investigate the particle chaining phenomenon across a Weissenberg number of $1$, thus revealing phenomena never reported previously in literature. In particular, this study has led to the following key findings: (i) we have corroborated that particles suspended into Newtonian fluids do not show FISA enhancement at different shear rates; (ii) FISA within viscoelastic fluids is significantly greater than in Newtonian fluids, when compared over the same range of shear rates; (iii) for Weissenberg number higher than $1$, the shear thinning nature of the viscoelastic fluid strongly enhances FISA; (iv) particle chains break apart when a constant shear is applied for sufficiently long-time (i.e. much longer than the fluids' longest relaxation time). This latter point could be explained by a ``\textit{disentanglement}'' process of the polymer chains constituting the viscoelastic fluid.

\section*{Acknowledgements}
This work was supported by the EPSRC CDT in Intelligent Sensing and Measurement, Grant Number EP/L$016753$/$1$.

\section*{Author contributions statement}

M.T. conceived the experiment(s), M.G.S. conducted the experiment(s), M.G.S. and M.T. analysed the results, M.G.S. wrote the manuscript. All authors reviewed the manuscript. 

\bibliography{References}


\end{document}