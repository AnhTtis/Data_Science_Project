% NOTE: Compile with XeLaTeX (or other png/pdf-compatible system)
%\documentclass[preprint,12pt]{elsarticle}
%\documentclass[10pt]{elsarticle}
%\documentclass[final,5p,times,twocolumn,sort,authoryear]{elsarticle} %% ... FOR THE REAL ARTICLE
\documentclass[final,5p,times,twocolumn,sort, compress]{elsarticle} %% ... FOR THE REAL ARTICLE
\parindent=0pt % disables indentation
% packages:
\usepackage{amsmath}
\usepackage{amssymb}
\usepackage{siunitx}
%\sisetup{separate-uncertainty = true, range-phrase = --, output-decimal-marker = {.}, group-separator = {,}, group-four-digits, per-mode = symbol, table-unit-alignment = left, range-units=single }
\sisetup{separate-uncertainty = true, range-phrase = --, range-units=single, output-decimal-marker = {.}, group-separator = {.}}
\graphicspath{{pics/}}						% set path of pics

\usepackage{ngerman}
%\usepackage{url}
\usepackage[hyphens]{url}
\usepackage{hyperref}  
\hypersetup{colorlinks = true, allcolors = black}
\usepackage{lineno}
\usepackage[11pt]{moresize}
\usepackage{graphicx}
\usepackage[dvipsnames]{xcolor}
\usepackage{booktabs}
\usepackage{multirow}
\usepackage{booktabs}
\usepackage{float}
% \usepackage[toc,page]{appendix}
\usepackage{appendix}
% %%%%%%
% % for nomenclature:
\usepackage{framed} % Framing content
%% works when adding this command
\immediate\write18{%
	makeindex -s nomencl.ist -o \jobname.nls -t \jobname.nlg \jobname.nlo%
}


\usepackage{multicol} % Multiple columns environment

\usepackage{nomencl}
% \renewcommand*\nompreamble{\begin{multicols}{3}}
%	\renewcommand*\nompostamble{\end{multicols}}
%\makenomenclature
%\renewcommand{\nomgroup}[1]{%
%	\ifthenelse{\equal{#1}{S}}{\item[\textbf{Subscripts}]}{%
%			\ifthenelse{\equal{#1}{H}}{\item[\textbf{Superscripts}]}{%
%		\ifthenelse{\equal{#1}{V}}{\item[\textbf{Variables}]}{}}} }
%\setlength{\nomitemsep}{-\parsep}
%\RequirePackage{ifthen}

%\renewcommand{\theequation}{\thesubsection.\arabic{equation}}


\makenomenclature
\setlength{\nomitemsep}{-\parskip}
\renewcommand*\nompreamble{\begin{multicols}{3}}
	\renewcommand*\nompostamble{\end{multicols}}
\setlength{\nomitemsep}{-0.4em} 
\newlength{\nomgroupstartsep}
\setlength{\nomgroupstartsep}{1em}
\usepackage{etoolbox}
\renewcommand\nomgroup[1]{%
	\itemsep\nomgroupstartsep%
	\item[\bfseries
	\ifstrequal{#1}{S}{Subscripts}{%
		\ifstrequal{#1}{H}{Superscripts}{%
			\ifstrequal{#1}{V}{Variables}{%
				\ifstrequal{#1}{A}{Abbreviations}{}}}}%
	]
	\itemsep\nomitemsep% restore spacing
}
%\RequirePackage{ifthen}

% \makenomenclature

\def\sectionautorefname{Sec.}
\def\subsectionautorefname{Sec.}
\def\subsubsectionautorefname{Sec.}
\def\figureautorefname{Fig.}
\def\equationautorefname{Eq.}
\def\tableautorefname{Tab.}
\def\appendixautorefname{}

\usepackage[tableposition=top]{caption}

%%%%%

\def\v#1{\mbox{\boldmath $\mathrm{ #1} $}}
\def\r#1{\mathrm{ #1 }}
\def\todo#1{\textcolor{red}{#1} }
%\linenumbers

\selectlanguage{english}

\journal{Journal of Process Control}

\newcommand{\abbpfad}{./04_Forecasting_Methods/}



%\usepackage{lineno}
%\linenumbers

%\bibliographystyle{apalike}
\begin{document}
	
	\begin{frontmatter}
		
		%% Title, authors and addresses
		
		%% use the tnoteref command within \title for footnotes;
		%% use the tnotetext command for theassociated footnote;
		%% use the fnref command within \author or \address for footnotes;
		%% use the fntext command for theassociated footnote;
		%% use the corref command within \author for corresponding author footnotes;
		%% use the cortext command for theassociated footnote;
		%% use the ead command for the email address,
		%% and the form \ead[url] for the home page:
		%% \title{Title\tnoteref{label1}}
		%% \tnotetext[label1]{}
		%% \author{Name\corref{cor1}\fnref{label2}}
		%% \ead{email address}
		%% \ead[url]{home page}
		%% \fntext[label2]{}
		%% \cortext[cor1]{}
		%% \address{Address\fnref{label3}}
		%% \fntext[label3]{}
		
		\title{Control-oriented modeling of a $ \r{LiBr/H_2O} $ absorption heat pumping device \\ and experimental validation}
		
		%% use optional labels to link authors explicitly to addresses:
		%% \author[label1,label2]{}
		%% \address[label1]{}
		%% \address[label2]{}
		
		\author[BEST,IRT]{Sandra Staudt}
%		\ead{sandra.zlabinger@best-research.eu}
		\author[BEST]{Viktor Unterberger}
%		\ead{viktor.unterberger@best-research.eu}
		\author[BEST,IRT]{Markus~G{\"o}lles\corref{cor1}}
		\ead{markus.goelles@best-research.eu}
		\author[IWT]{Michael~Wernhart}
%		\ead{michael.wernhart@tugraz.at}
		\author[IWT]{Ren{é}~Rieberer}
%		\ead{rene.rieberer@tugraz.at}
		\author[BEST,IRT]{Martin~Horn}
%		\ead{martin.horn@tugraz.at}
		
		

		
		\cortext[cor1]{Corresponding author: Inffeldgasse 21/B, 8010 Graz, Austria}
		\address[BEST]{BEST - Bioenergy and Sustainable Technologies GmbH, Inffeldgasse 21/B, 8010 Graz, Austria}
		\address[IRT]{Institute of Automation and Control, Graz University of Technology, Inffeldgasse 21/B, 8010 Graz, Austria}
		\address[IWT]{Institute of Thermal Engineering, Graz University of Technology, Inffeldgasse 25/B, 8010 Graz, Austria}
		
		
		
		\begin{abstract}
			%
			
			Absorption heat pumping devices (AHPDs, comprising absorption heat pumps and chillers) are devices that use thermal energy instead of electricity to generate heating and cooling, thereby facilitating the use of  waste heat and renewable energy sources such as solar or geothermal energy.
%			 in heating and cooling systems. 
			 Despite this benefit, widespread use of AHPDs is still limited. One reason for this is partly unsatisfactory control performance under varying operating conditions, which can result in poor modulation and part load capability. A promising approach to tackle this issue is using dynamic, model-based control strategies, whose effectiveness, however,  strongly depend on the model being used. 
			This paper therefore focuses on the derivation of a viable dynamic model to be used for such model-based control strategies for AHPDs such as state feedback or model-predictive control. 
			The derived model is experimentally validated, showing good modeling accuracy. Its modeling accuracy is also compared to alternative model versions, that contain other heat transfer correlations, as a benchmark.
%			For this, different models, consisting of existing and new approaches to model heat transfer in AHPDs, are compared and experimentally validated, showing that the new approaches yield better modeling accuracy. 
			Although the derived model is mathematically simple, it does have the structure of a nonlinear differential-algebraic system of equations. To obtain an even simpler model structure,  linearization at an operating point  is discussed to derive a model in linear state space representation.	The experimental validation shows that the linear model does have slightly worse steady-state accuracy,   but that the dynamic accuracy seems to be almost unaffected by the linearization.	The presented new  modeling approach is considered suitable to be used as a basis for the design of advanced, model-based control strategies, ultimately aiming to improve the modulation and part load capability of AHPDs.
			
			%The experimental validation shows that the linearized models have slightly worse static accuracy, but the system dynamics can still be represented well. 
			
			%Two simple, physically motivated, experimentally validated models with only nine state variables are presented - one nonlinear differential-algebraic model and one linear state-space model, the latter being derived from the nonlinear model by means of linearization in a reference operating point and further simplifications. Detailed experimental validation results are discussed for steady-state and transient effects of variations of manipulable input variables.
			%Both models proved to describe the main system dynamics accurately and are able to describe not only effects of variations of inlet temperatures but also of flow rates of external hot, cool and chilled water circuits, as well as of variations of the rich solution flow rate. They represent viable models for model-based control strategies which shall be used to improve modulation- and part load capability of AHPDs.
			
			%
			%further showed good steady-state accuracy over a wide operating range, the linear model on the other side 
			
			
			
			% Text of abstract
			
		\end{abstract}
		
		%%%Graphical abstract
		%\begin{graphicalabstract}
		%%\includegraphics{grabs}
		%\end{graphicalabstract}
		%
		%%%Research highlights
		%\begin{highlights}
		%\item Research highlight 1
		%\item Research highlight 2
		%\end{highlights}
		
		\begin{keyword}
			
			
			absorption \sep heat pump \sep chiller \sep model  \sep experimental validation \sep LiBr 
			%% keywords here, in the form: keyword \sep keyword
			%\PACS \TODO{PACS}         %% PACS codes here, in the form: \PACS code \sep code
			%\MSC[2008] \TODO{MSC} %% MSC codes here, in the form: \MSC code \sep code
			%% or \MSC[2008] code \sep code (2000 is the default)
		\end{keyword}
		
	\end{frontmatter}
	
	%% \linenumbers
	
	%% main text
	
	%%%%%%%%%%%%%%%%%%%%%%%%%%%%%%%%%%%%%%%%%%%%%%%%%%%%%%%%%%%%%%%%%%%%%%%%%%%%%%%%%%%%%%%
	%%%%%%%%%%%%%%%%%%%%%%%%%%%%%%%%%%%%%%%%%%%%%%%%%%%%%%%%%%%%%%%%%%%%%%%%%%%%%%%%%%%%%%%
		\nomenclature[v00]{$\Delta t_\r{d}$ }{dead time  [\si{\second }]}
	\nomenclature[v01]{$\epsilon$}{effectiveness {[-]}} 
	%\nomenclature[v02]{$\phi_{\mathrm{\dot{Q}}}$ }{ scaling factor [-]}
	\nomenclature[v03]{$\phi_{\mathrm{sub}}$ }{ subcooling fraction [-]}
	\nomenclature[v04]{$\rho$}{density [\si{\kilogram \per \meter \cubed}]}
	\nomenclature[v05]{$\tau$ }{time constant  [\si{\second }]}
	\nomenclature[v06]{$\xi$  }{ mass fraction of LiBr in solution  [$\frac{\text{kg}_{\r{LiBr}}}{\text{kg}_{\r{solution}}}$]}
	
	\nomenclature[v07]{$c_{p}$ }{specific heat capacity  [\si{\joule \per \kilogram \per \kelvin }]}
	\nomenclature[v08]{$h$}{specific enthalpy [\si{\joule \per \kilogram }]}
		\nomenclature[v09]{$H$}{enthalpy [\si{\joule}]}
		
	\nomenclature[v13]{$m$  }{mass  [\si{\kilogram}]}
	\nomenclature[v14]{$\dot{m}$ }{mass flow rate  [\si{\kilogram \per \second}]}
	\nomenclature[v15]{$p$}{pressure [\si{\pascal}]}
	\nomenclature[v16]{$\dot{Q}$  }{heat flow rate [\si{\watt}]}
	\nomenclature[v18]{$t$ }{time  [\si{\second }]}
	\nomenclature[v19]{$T$  }{temperature [\si{\kelvin}]}

	
	\nomenclature[v21]{$\mathit{TTD}$  }{terminal temperature difference [\si{\kelvin}]}
%		\nomenclature[v21]{$u$}{input variable   }
%			\nomenclature[v22]{$\v{u}$ }{input variable vector  }
%\nomenclature[v22]{$ \v{u} $}{input variable vector}
			\nomenclature[v23]{$U$}{internal energy   [\si{\joule }]}

	\nomenclature[v32]{$\mathit{UA}$ }{$\mathit{UA}$ value  [\si{\watt \per \kelvin }]}
	\nomenclature[v33]{$\dot{V}$ }{volume flow rate [\si{\meter \cubed \per \second}]}
%	\nomenclature[v34]{$\v{x}$ }{state variable vector}
%	\nomenclature[v35]{$\v{y}$ }{output variable vector}
%	\nomenclature[v34]{${x}$ }{ state variable}
%	\nomenclature[v35]{$\v{x}$ }{state variable vector  }
%	\nomenclature[v36]{$y$ }{output variable  }
%	\nomenclature[v37]{$\v{y}$ }{ouput variable vector  }
%	\nomenclature[v38]{$z$  }{algebraic variable  }
%	\nomenclature[v39]{$\v{z}$  }{algebraic variable vector  }
	
	
	
	
	\nomenclature[s01]{A}{absorber}
	\nomenclature[s02]{C}{condenser}
	\nomenclature[s03]{E}{evaporator}
	\nomenclature[s04]{G }{generator}
	\nomenclature[s05]{GRh }{gas room on high pressure side}
	\nomenclature[s06]{GRl }{gas room on low pressure side}
	\nomenclature[s07]{h}{ high}
	\nomenclature[s08]{$\mathrm{H_2O}$ }{water}
	\nomenclature[s09]{HX}{heat exchanger }
	
	\nomenclature[s11]{in }{inflowing}
	\nomenclature[s12]{l}{low}
	\nomenclature[s13]{LiBr}{lithium bromide}
	\nomenclature[s14]{meas}{measured }
%	\nomenclature[s15]{new}{new system}
	\nomenclature[s16]{out}{outflowing}
	\nomenclature[s17]{PSo}{poor solution }
	\nomenclature[s18]{rec}{recirculating}
%	\nomenclature[s19]{ref}{reference system}
	
	\nomenclature[s21]{Ref}{refrigerant}
	\nomenclature[s22]{ROP}{at reference operating point }
	\nomenclature[s23]{RSo}{rich solution }
	\nomenclature[s24]{sat}{saturated }
	\nomenclature[s25]{SHX}{solution heat exchanger}
		\nomenclature[s26]{ss}{steady state}
				\nomenclature[s27]{sumps}{in sumps}
	\nomenclature[s28]{W}{water in external circuits}
	
	
	%\nomenclature[s]{tot }{ total (LiBr and $\mathrm{H_2O}$)}
	%\nomenclature[s]{0}{initial }
	
	
	
	\nomenclature[h]{l}{liquid }
	\nomenclature[h]{v}{vapor }
	
	
	\section{Introduction}
	\label{sec:intro}
	
	
	Heat pumping devices (HPDs, comprising heat pumps and chillers) are  devices that can transfer thermal energy from a low to a high temperature level. This requires an energetic input, either in the form of mechanical work (usually supplied by an electric motor) in the case of compression HPDs or in the form of high-temperature thermal energy in the case of absorption HPDs (AHPDs). 
%	\cite{Herold2016}. 
	Therefore, AHPDs can use waste heat and renewable energy sources such as solar or geothermal energy instead of electricity to generate heating and cooling in a resource-efficient manner. 
	
	
	Despite this benefit, AHPDs are still not a very common technology since operators are often put off by the increased complexity of AHPDs (e.g. more in- and output variables compared to compression HPS) and a lack of dynamic control strategies that allow good control performance over a wide operating range and that consider the coupled dynamics of different system variables   \cite{Goyal2019}. 	
	Current control strategies for AHPDs typically use SISO PI controllers or rely on simple ON/OFF operation, which may be sufficient for many current AHPD applications with rather constant operating conditions. However, they can reach their limits when AHPDs are integrated into modern, more complex energy systems where renewable, volatile energy sources play an increasingly important role. 
%	Here, new, systematically-adaptable control strategies can be beneficial by, e.g., enhancing an AHPD's modulation and part load capability.
%	 and where it is favorable to use dynamic control strategies  that facilitate operation under volatile operating conditions \cite{Goyal2019}. 
%	 satisfactory operation also under varying, more dynamic operating conditions. 
%Therefore, a goal of current research on AHPD control is to develop a multi-variable, dynamic model-based control strategy for AHPDs (such as state feedback control or MPC) to 	extend their operating range.	
Here, one way to further advance AHPD control is to design dynamic model-based control strategies for AHPDs, e.g. \cite{Goyal2019a},  (such as state feedback or model-predictive control)  to extend their operating range.	
	This, however, first requires a suitable control-oriented, dynamic AHPD model - preferably in the form of a state-space model - which is the focus of this paper. The modeling approaches discussed here refer to AHPDs with the working fluid lithium bromide/water ($\r{LiBr/H_2O}$).
	 In the following, first an overview of available models from  literature is given, followed by the contribution this paper aims to make and its structure. 

	\subsection{Available models for $ {LiBr/H_2O} $ AHPDs}
	\label{subsec:available_models}
%	\todo{BORG!!!}
	
%	The literature on modeling of $ \r{LiBr/H_2O} $ AHPD contains several reasonable approaches, including some that can be used as a basis for a control-oriented AHPD model. An overview shall be given subsequently:
%	Available models for AHPD can first of all be divided into black-box models on the one hand and physically based models on the other hand.
%	In the former case, the authors of \cite{Lazrak2016} designed a dynamic artificial neural network model and successfully showed selected experimental validation results. In \cite{Chow2002} a neural network approach is used to develop a model for steady-state performance correlations. Such black-box approaches have the disadvantage that they are not easily scalable though and thus have to be re-trained with a large number of (measurement) data to derive a robust model for a new machine, even if the design is similar and only the size differs from the reference machine.
%	Physically based models, in turn, are easier to scale and allow for parametrization without measurement data. They are significantly more prevalent in the literature on AHPD modeling and can roughly be subdivided into three groups:
%	
%	First, there are static models, e.g.  \cite{Nia2015, Florides2003, Kim2008, Rathod2019, Hosseinirad2019},  which can be very useful in the dimensioning and design process and can also be suitable for high-level system control with rather long time constants, as shown in \cite{Rathod2019, Hosseinirad2019} but are not viable for low-level control of the actual AHPD since they do not contain the AHPD's dynamics. 
%	
%	Second, there are dynamic, discretized models, i.e. the components of the AHPD are spatially discretized in the model, e.g. \cite{Fu2006, Bittanti2010,Vinther2015}. These are excellent models for very detailed simulation studies but are too complex and computationally expensive to be used for model-based control due to the high number of state variables and therefore the high model order.  In \cite{Vinther2015}, however, the authors discuss the possibility to use the presented high-order, nonlinear model, which was implemented in Dymola\textsuperscript{\textregistered} \cite{DassaultSystemes}, as a basis for a simpler, linear model by employing built-in functions of Dymola\textsuperscript{\textregistered} and Matlab\textsuperscript{\textregistered} \cite{MathWorks2021}  to first linearize the model and then reduce the model order, which is a reasonable approach to derive a control-oriented model in case a complex one is already available in Dymola\textsuperscript{\textregistered}. 
%	
%	Third, there are simpler dynamic models based on so-called lumped-component approaches, e.g. \cite{Jeong1998, Kohlenbach2008,  Evola2013, Marc2015,  Sabbagh2018,Ochoa2016,Zinet2012,delaCalle2016, Castro2020}, where  individual components are modeled as lumped nodes with concentrated properties like temperature, density etc., which allows for a significantly lower model order compared to the discretized models mentioned before. Also combinations of lumped-component and discretized models, e.g. \cite{Misenheimer2017, Xu2016, Shin2009}, or  lumped-component and black-box models, e.g. \cite{Borg2012}, exist.
%	Within the group of lumped-component models the model complexity can roughly be assessed on the one hand by the used correlations for heat transfer and on the other hand by the number of state variables. Models in \cite{Jeong1998,Kohlenbach2008, Evola2013, Marc2015, Sabbagh2018} use constant heat transfer coefficients while \cite{Ochoa2016, delaCalle2016,Zinet2012, Castro2020} use more complex $ Nusselt $ correlations. 
%	%The model order ranges from app. 12 \cite{Jeong1998} to 20 \cite{Marc2015}, depending on which mass and energy stores are considered. 
%	The model order ranges from app. 12 \cite{Jeong1998} to 27 \cite{delaCalle2016} state variables, depending on which mass and energy stores are considered in the model. 
%	In principle, these lumped-component models (\cite{Jeong1998, Kohlenbach2008, Zinet2012, Evola2013, Marc2015, delaCalle2016, Ochoa2016, Sabbagh2018}) can be a good basis for control-oriented AHPD models. However, these models from literature are either not or not fully validated \cite{Jeong1998,Ochoa2016, Sabbagh2018, Zinet2012, delaCalle2016} so that it is not clear which model accuracy can be expected, or they do not describe dynamic effects of all manipulable input variables \cite{ Jeong1998, Kohlenbach2008, Evola2013, Marc2015, Ochoa2016}. In particular, one of the input variables, the flow rate of the so-called rich solution,  is either assumed to be constant \cite{ Jeong1998, Kohlenbach2008, Evola2013, Marc2015} or not considered in the validation process \cite{Sabbagh2018, Jeong1998, Ochoa2016, delaCalle2016, Zinet2012, Castro2020}. 
%	Since this is an input variable that shall be used as manipulated variable for model-based control though, it is important that a control-oriented model can describe its dynamic effects. 

The literature on modeling of $ \r{LiBr/H_2O} $ AHPDs contains several reasonable approaches, including some that can be used as a basis for a control-oriented AHPD model. An overview shall be given subsequently:
Available models for AHPDs can first of all be divided into black-box models on the one hand and physically based models on the other hand.
In the former case, the authors of \cite{Lazrak2016} developed a dynamic artificial neural network model and successfully showed selected experimental validation results. In \cite{Chow2002} a neural network approach is used to model steady-state correlations between desired cooling capacity, selected disturbance variables and the corresponding necessary driving energy. Such black-box approaches have the disadvantage that they are not easily scalable though and thus have to be re-trained with a large number of (measurement) data to derive a robust model for a new machine, even if the AHPD design is similar and only the size differs from the reference machine.
Physically-based models, on the other hand, are easier to scale. 
% and allow for parameterization without measurement data. 
They are also significantly more prevalent in the literature on AHPD modeling and can roughly be subdivided into three groups:

First, there are steady-state models, e.g.  \cite{Furukawa1987, Florides2003, Kim2008, PuigArnavat2010, Nia2015, Albers2018,  Rathod2019, Hosseinirad2019, Albers2021, Prieto2022},  which can be very useful in the dimensioning and design process, for feedforward-control of AHPD and for system-level control of energy systems with AHPD as shown in \cite{Rathod2019, Hosseinirad2019, Albers2021} but are not viable for dynamic model-based control methods since they do not capture the AHPD's dynamics. 

Second, there are dynamic, discretized models, i.e. the components of the AHPD are spatially discretized in the model, e.g. \cite{Fu2006, Bittanti2010,Vinther2015, Wernhart2020}. These are excellent models for very detailed simulation studies but are usually too complex and computationally expensive to be used for model-based control due to the high number of state variables and therefore the high model order.  In \cite{Vinther2015}, however, the authors discuss the possibility to use the presented high-order, nonlinear model, which was implemented in Dymola\textsuperscript{\textregistered} \cite{DassaultSystemes}, as a basis for a simpler, linear model by employing built-in functions of Dymola\textsuperscript{\textregistered} and Matlab\textsuperscript{\textregistered} \cite{MathWorks2021}  to first linearize the model and then reduce the model order, which is a reasonable approach to derive a control-oriented model in case a complex one is already available in Dymola\textsuperscript{\textregistered}. 

Third, there are simpler dynamic models based on so-called lumped-component approaches, e.g. \cite{Jeong1998, Kohlenbach2008,  Evola2013, Marc2015,  Sabbagh2018,Ochoa2016,Zinet2012,delaCalle2016, Castro2020}, where  individual components are modeled as lumped nodes with concentrated properties like temperature, density etc., which allows for a significantly lower model order compared to the discretized models mentioned before. Also combinations of lumped-component and discretized models, e.g. \cite{Misenheimer2017, Xu2016, Shin2009}, or  lumped-component and black-box models, e.g. \cite{Borg2012}, exist.
Within the group of lumped-component models the model complexity can roughly be assessed on the one hand by the used correlations for heat transfer and on the other hand by the number of state variables. Models in \cite{Jeong1998,Kohlenbach2008, Evola2013, Marc2015, Sabbagh2018} use constant heat transfer coefficients while \cite{Ochoa2016, delaCalle2016,Zinet2012, Castro2020} use more complex $ Nusselt $ correlations. 
%The model order ranges from app. 12 \cite{Jeong1998} to 20 \cite{Marc2015}, depending on which mass and energy stores are considered. 
The model order ranges from app. 12 \cite{Jeong1998} to 27 \cite{delaCalle2016} state variables, depending on which mass and energy stores are considered in the model. 
In principle, these lumped-component models (\cite{Jeong1998, Kohlenbach2008, Zinet2012, Evola2013, Marc2015, delaCalle2016, Ochoa2016, Sabbagh2018}) can be a good basis for control-oriented AHPD models. However, these models from literature are either not or not fully validated \cite{Jeong1998,Ochoa2016, Sabbagh2018, Zinet2012, delaCalle2016} so that it is not clear which model accuracy can be expected, or they do not describe dynamic effects of all manipulable input variables \cite{ Jeong1998, Kohlenbach2008, Evola2013, Marc2015, Ochoa2016}. In particular, one of the input variables, the flow rate of the so-called rich solution,  is either assumed to be constant \cite{ Jeong1998, Kohlenbach2008, Evola2013, Marc2015} or not considered in the validation process \cite{Sabbagh2018, Jeong1998, Ochoa2016, delaCalle2016, Zinet2012, Castro2020}. 
But since this is an input variable that shall be used as manipulated variable for model-based control, it is important that a control-oriented model can describe its dynamic effects. 
	
	
	
	
	
	\subsection{Research contribution}
%Though many reasonable dynamic modeling approaches for $ \r{LiBr/H_2O} $ AHPD exist, they are either considered too complex to be used for model-based control or they do not sufficiently investigate (either neglect or do not experimentally validate) the dynamic effects of all input variables that are relevant for control purposes. 

%In addition, there does not yet seem to be a consistent approach to the choice of dynamics, i.e., the state variables and/or dead time effects to be considered in the model. This may be related to the fact that so far most models have been used for simulation  rather than for control purposes,  so that model order and complexity were less significant.
	%but it needs to be clarified for control-oriented models. 
	The aim of this paper is to discuss control-oriented modeling approaches for $ \r{LiBr/H_2O} $ AHPDs that describe the main dynamics of the AHPD's output variables, being the outlet temperatures and heat flow rates in the hydraulic circuits, for variations of all relevant input variables, and compare it to measurement data from an AHPD at a laboratory test bench as well as to alternative model versions as a benchmark.
%	existing modeling approaches from literature as well as to measurement data from an AHPD at a laboratory test bench.  
%	This will be done for for lithium-bromide/water ($ \r{LiBr/H_2O} $) AHPD 
	The main research contributions can be summarized as: 
	\begin{enumerate}
		%	\setlength\itemsep{-0.5em}
%		\item Discussion of most relevant mass and energy stores 
%		for a simple model containing only the most relevant  state variables
		\item Derivation of a control-oriented AHPD model, taking into account varying operating conditions, especially variations of the rich solution flow rate
		\item Investigation of the effect of linearization at an operating point on modeling accuracy
%		Linearization around operating point for an even simpler control-oriented model
		\item Extensive experimental validation for an AHPD with a nominal cooling capacity of \SI{15}{\kilo \watt} and comparison to alternative model versions as a benchmark.
	\end{enumerate}

	
	
	
	\subsection{Structure of paper}
	
	\autoref{sec:process_description} gives a process description for $\r{LiBr/H_2O}$ AHPDs in general and presents the specific AHPD that is used as a basis for the present work.  \autoref{sec:Sim_model} describes the new AHPD model and its linearization at a reference operating point. In \autoref{sec:benchmark_models}  two benchmark models are described which are used for comparison in the validation section \autoref{sec:validation} where all presented models are compared to each other and to measurement data to analyze the models' steady-state and dynamic accuracy. Lastly, a conclusion and outlook is given in \autoref{sec:conclusion}.
	
	
	\begin{table*}[h!]
		%\begin{table}
		\begin{framed}
			\printnomenclature
		\end{framed}
	\end{table*} 
	
	
	
	%%%%%%%%%%%%%%%%%%%%%%%%%%%%%%%%%%%%%%%%%%%%%%%%%%%%%%%%%%%%%%%%%%%%%%%%%%%%%%%%%%%%%%%
	%%%%%%%%%%%%%%%%%%%%%%%%%%%%%%%%%%%%%%%%%%%%%%%%%%%%%%%%%%%%%%%%%%%%%%%%%%%%%%%%%%%%
	
	
	\section{Process description}
	\label{sec:process_description}
	
	
	The models discussed in this paper are intended for $ \r{LiBr/H_2O} $ AHPDs  as depicted in  
	\autoref{fig:AHPD_Process}. The schematic layout in 	\autoref{fig:AHPD_Process}  gives an overview over relevant fluid streams (indicated by $ \dot{m}_{\r{subscript}} $, e.g. $  \dot{m}_{\r{PSo,SHX,in}}$) and fluid stores (indicated by $ m_{\r{subscript}} $, e.g. $ m_{\r{PSo,G}} $). For easier readability the following notation will be used throughout this paper: The two superscripts  l and v for liquid and vaporous fluid are only added when the corresponding fluid stream  consists of a liquid and a vaporous phase. In all other cases, when the fluid's state of aggregation is evident, the superscript will be omitted. 
	
	\begin{figure*}[h!]
		\centering
		\input{pics/AHPS_modelling_V6b.pdf_tex}
		\caption{Scheme of investigated $\r{LiBr/H_2O}$ AHPD and relevant mass and energy flows ($\dot{m}_{\r{subscript}}$) and stores (${m}_{\r{subscript}}$)}
		\label{fig:AHPD_Process}%
	\end{figure*}
	
	The most relevant components are the four main components generator, condenser, evaporator and absorber, each consisting of a heat exchanger (HX) and a fluid reservoir (sump),  the solution heat exchanger (SHX), the solution pump, the refrigerant recirculation pump, the solution expansion valve (SEV) and the refrigerant expansion valve (REV), where the SEV and REV can either be valves or simply pipes with reduced cross-sections.
%	 In some AHPD, the SEV and REV are not valves per se but merely pipes with a smaller cross-section to induce a pressure reduction.}
	 Inside the AHPD, there are two interlinked circuits - a refrigerant circuit, containing $\r{H_2O} $ as refrigerant, and a solution circuit, containing $ \r{LiBr/H_2O} $, a mixture of $\r{H_2O} $ as refrigerant and LiBr as solvent. In addition, three external hydraulic circuits are connected to the AHPD for heat rejection (cooling water circuit) and heat supply (hot and chilled water circuit).
	% The hot water circuit is connected to the generator, the cool water circuit to the absorber and condenser (ususally in series), and the chilled water circuit to the evaporator.
	% Furthermore, the AHPD is often 
	The process has seven adjustable input variables - the inlet temperatures and  flow rates in the three hydraulic circuits, as well as the  flow rate of the rich solution.
%	 and of the recirculated refrigerant in the evaporator, the latter usually being approximately constant though due to constant speed recirculation pumps.
	  One or more of the hydraulic circuits' outlet temperatures and/or a heat flow rate are usually the controlled variables of interest,  
%	Output variables considered in this paper are outlet temperatures and transferred heat  flows in the hydraulic circuits since a subset of these are usually the control variables of interest, 
	in detail depending on the specific application.
	
	The process can be described as follows. Starting at the generator, solution rich in refrigerant (rich solution) entering the generator ($ \dot{m }_{\r{RSo,SHX,out}} $) is heated up by the heat flow $ \dot{Q}_\r{G} $ and part of the absorbed refrigerant desorbs so that the solution leaving the HX ($ \dot{m }_{\r{PSo,HX,G,out}} $)  contains less refrigerant (poor solution). The desorbed refrigerant vapor ($ \dot{m }_{\r{Ref,GRh}} $) accumulates in the room between generator and condenser (high pressure gas room) and is condensed in the condenser ($ \dot{m }_{\r{Ref,HX,C,out}} $), thereby releasing the heat flow $ \dot{Q}_\r{C} $, and then flows into the condenser sump ($ {m }_{\r{Ref,C}} $). From here, it flows through the refrigerant expansion valve, where it expands to a lower pressure level, partly evaporates and cools down. The liquid part ($ \dot{m }^\r{l}_{\r{Ref,E,in}} $) of the now cold refrigerant flows into the sump of the evaporator ($ {m }_{\r{Ref,E}} $). The recirculation pump pumps refrigerant  over the evaporator HX ($ \dot{m }_{\r{Ref,E,rec}} $), where it partly evaporates by means of the heat flow $ \dot{Q}_\r{E} $. The liquid refrigerant leaving the evaporator HX ($ \dot{m }^\r{l}_{\r{Ref,E,HX,out}} $) flows back into the sump while the evaporated refrigerant ($ \dot{m }^\r{v}_{\r{Ref,E,HX,out}} $) and the vaporous refrigerant entering the evaporator ($ \dot{m }^\r{v}_{\r{Ref,E,in}} $) accumulate in the low pressure gas room. 
	%Here refrigerant and solution circuit meet again.
	In the absorber, poor solution flows over the HX ($ \dot{m}_{\r{PSo,A,in}} $), where it is cooled down by the heat flow $ \dot{Q}_\r{A} $ and absorbs this vaporous refrigerant ($ \dot{m}_{\r{Ref,GRl}} $).  The now rich solution ($ \dot{m}_{\r{RSo,A,HX,out}} $) flows into the absorber sump ($ m_{\r{RSo,A}} $).
	The solution pump then conveys the rich solution from the absorber ($ \dot{m}_{\r{RSo,A,out}} $) to the SHX ($ \dot{m}_{\r{RSo,SHX,in}} $) and into the generator ($ \dot{m}_{\r{RSo,SHX,out}} $),
	and raises its pressure to the high pressure level. In the SHX, heat is transferred from the poor solution to the rich solution, enhancing system efficiency. The poor solution flows from the generator sump ($ {m }_{\r{PSo,G}} $) through the SHX and the solution expansion valve, where it expands back to low pressure level, and finally into the absorber.   A detailed description of the working principle and other system designs can be taken from \cite{Herold2016}.
	When the AHPD is used as a chiller, the heat flow $ \dot{Q}_{\r{E}} $ is utilized and $ \dot{Q}_{\r{A}} $ and $ \dot{Q}_{\r{C}} $ have to be recooled by e.g. a cooling tower. When the AHPD is used as a heat pump, the heat flows $ \dot{Q}_{\r{A}} $ and $ \dot{Q}_{\r{C}} $ are utilized and $ \dot{Q}_{\r{E}} $ can be supplied by low temperature energy sources like e.g. waste or ambient heat. In either case the heat flow $ \dot{Q}_{\r{G}} $ has to be supplied by some high-temperature heat source. 
	
	
	The proposed models in this paper are developed on the basis of and validated for the $\r{LiBr/H_2O} $ AHPD type \textit{WEGRACAL \textsuperscript{\textregistered} Maral 1} of the company EAW with a nominal cooling capacity of app. \SI{15}{\kilo \watt}  \cite{EAW2017} and  open plate heat exchangers in the generator, condenser, evaporator and absorber. 
	
	
	\section{Control-oriented AHPD model}
	\label{sec:Sim_model}
	
	In this section the modeling of $\r{LiBr/H_2O} $ AHPDs for control purposes will be discussed. 
%	, with three model versions, differing in the way the heat transfer is modeled in the individual HX. 
	First, the basic modeling approach is explained in \autoref{subsec:methodology}, followed by a discussion of relevant mass and energy stores in \autoref{subsec:rel_storage}. 
%	In \autoref{subsec:versions} the three different model versions to be compared are presented, differing in the way the heat transfer is modeled in the individual HX. Moreover, the linearization at an operating point is discussed. 
 Then, in \autoref{subsec:gen_con} to  \autoref{subsec:pumps}  the modeling of the AHPD's individual components is explained which are ultimately combined to form the complete AHPD model in \autoref{subsec:nonlin_complete}.
%	Finally, in \autoref{subsec:gen_con} to \autoref{subsec:nonlin_complete} the models of the individual AHPD components are presented. 
	
		For easier readability the following notation will be used from now on: When two variables with the same sub- and superscript are multiplied, they are joined in a square bracket with the common sub- and superscripts, so that e.g. $ \dot{m}^\r{l}_{\r{Ref,HX,E,out}} h^\r{l}_{\r{Ref,HX,E,out}}  $ becomes
	$ [\dot{m}h]^\r{l}_{\r{Ref,HX,E,out}}  $. 
	
	
	
%	
	\subsection{Basic modeling approach}
	\label{subsec:methodology}
%	
%	\todo{ANDERS}
	
	For the modeling in this paper lumped-component approaches are used, i.e.,  individual components are not modeled as spatially discretized components but considered as lumped nodes with concentrated properties, which allows for a rather simple model with few state variables. 
	%State variables in this model are stored mass and internal energies of selected fluid reservoirs. 
	Only those state variables, i.e., mass and energy stores,  that result in a dynamic effect in the relevant time range are considered, which will be discussed in \autoref{subsec:rel_storage}.
	The dynamic effects of these relevant mass and energy stores are described with transient mass and energy balances, i.e., ordinary first-order differential equations. In addition, algebraic equations are necessary to complete the model. These algebraic equations describe, e.g., heat transfer in the heat exchangers or correlations between physical quantities like, e.g., saturation pressure and temperature or enthalpy, temperature and mass fraction. 
	Due to the strong couplings between these algebraic equations and their generally non-linear character, they cannot easily be given as an explicit function of the input and state variables only but will be given in implicit form instead. 
	
	Moreover, the following modeling assumptions are used:
	\begin{itemize}
				\setlength\itemsep{-0.4em}
\item The refrigerant is pure water.
\item The refrigerant and solution leaving the generator, condenser and evaporator HXs are saturated, except the vaporous refrigerant leaving the generator which is assumed to have the same temperature as the entering rich solution. 
\item The expansion in the REV and SEV is isenthalpic.
\item All liquids are incompressible fluids. 
\item Heat losses are neglected. 
\item The power of the pumps is negligible.
	\end{itemize}
	
%	\todo{Übergang bzgl. heat transfer correlations}
	
	
	
	
		\subsection{Discussion of relevant mass and energy stores}
	\label{subsec:rel_storage}

	For a simple model 	it is advisable to  assess which mass and energy stores result in  dynamic effects
%	, i.e. lag or dead time effects 
	that occur on a time scale relevant for the latter control problem. 
	These include first-order delay effects, caused by mass or energy storage in a vessel,  and dead time effects, caused by the plug-flow of a fluid through a pipe or HX. 
	For this, dynamic effects with time constants $ \tau $ of app. \SIrange{5}{500}{\second} and dead times $ \Delta t_\r{d}  >$ \SI{5}{\second}  shall be considered relevant. 
	%In addition, the position, resp. conenction, of individual mass and energy stores in relation to in- and output variables and to other mass and energy stores can also be decisive, since the dynamic effect of two stores connected in series might be approximated as one storage with little effect on modeling accuracy, whereas this is not possible when an in- or output variable is present between the stores.
	%Additionally, also the position/connection of individual mass and energy stores in relation to in- and output variables as well as to other mass and energy stores can be decisive  since the dynamic effect of two stores in series might be approximated as one storage with little effect on modeling accuracy, while this is not possible if an in- or output is interconnected.
	%In the end, a good trade-off between model simplicity and accuracy must be found. 
%In addition, there does not yet seem to be a consistent approach to the choice of dynamics, i.e., the state variables and/or dead time effects to be considered in the model. This may be related to the fact that so far most models have been used for simulation  rather than for control purposes,  so that model order and complexity were less significant.
So far, in the literature there does not seem to be a consistent approach to the choice of dynamics, which may be related to the fact that most models have been used for simulation  rather than for control purposes,  so that model order and complexity were less significant.
	Therefore, the individual mass and energy stores, are discussed below, based on the AHPD from \autoref{sec:process_description}:
	\begin{itemize}
		\setlength\itemsep{-0.4em}
		\item The room filled with vaporous refrigerant (high and low pressure gas room) contains a very small amount of refrigerant  and results in a first-order delay effect with time constants $\tau < \SI{1}{\s}  $. It is therefore modeled by means of algebraic equations in the form of steady-state mass and energy balances.
		\item Similarly, the low energy storage capability of the five HXs' metal plates in combination with relatively high heat transfer rates result in first-order delay effects with negligible time constants $\tau < \SI{5}{\s}  $ for the HX's metal plates. 
		\item The fluid in the main components' sumps have relevant dynamic effects with $ \SI{5}{\s}< \tau < \SI{500}{\s}$. They are approximated as first-order delay elements and modeled by first-order mass and energy balances. 
		\item The mass of refrigerant and solution at the surface of the main components' HXs is difficult to assess due to unknown void fractions of vaporous refrigerant in the two-phase liquids, but assumed to be sufficiently small in comparison to the mass in the sumps to neglect their first-order delay and dead time effects.
		\item The dead time effect of the water in the main components' HXs strongly depends on the operating conditions in the hydraulic circuits, so that for low volume flow rates  $\Delta t_\r{d}$ can reach \SI{5}{\s}. Additionally, the temperature sensors that measure in- and outlet temperatures are usually not mounted directly at the in- and outlet of the HXs but further away (due to constructural reasons), which results in additional dead times (up to \SI{20}{\s} for the investigated AHPD). For a more manageable control-oriented model these dead time effects are modeled in such a way that the main components' HXs are approximated as lumped nodes without dead time and instead the dead time effects are added to the hydraulic circuits by means of two volume-flow-dependent  dead time elements at the in- and outlet of each hydraulic circuit. These  are not considered part of the AHPD model itself, but are discussed with the test bench setup in \autoref{subsec:test bench} (\autoref{eq:deadtime_in} and \autoref{eq:deadtime_out}).
		\item The solution in the SHX and in the pipes between absorber and generator has relevant dynamic effects with volume-flow-dependent dead times of up to app. \SI{60}{\s} on both the poor and the rich solution side. These dead times occur ``in the middle'' of the AHPD as opposed to at the in- or outlet as discussed above and investigations showed that they can be approximated as first-order delay elements for a more manageable control-oriented model and still yield good dynamic modeling accuracy for the outlet temperatures and heat flow rates in the three hydraulic circuits.  
%		Note that often, e.g. \cite{Jeong1998,Kohlenbach2008, Marc2015, Sabbagh2018, Bittanti2010}, these dynamic effects in the SHX are neglected altogether, which is why model version V1 (see \autoref{tab:model_version}) does not contain any SHX dynamics (neither lag delay nor dead time elements). 
\item Similarly, the refrigerant in the pipe between condenser and evaporator would lead to relevant dead times. However, since this pipe basically connects two sumps, this dead time effect is neglected and instead the refrigerant is modeled as part of the consecutive evaporator sump, which in return is modeled as first-order delay element.  The same applies to the refrigerant in the pipes of the recirculation stream in the evaporator. Details on this simplification will be discussed in \autoref{subsubsec:REV}.
		\item The metal of the AHPD's shell has a high mass and energy storage capability but the heat transfer rate between fluid and shell is considerably lower than in the HXs. Its dynamic effect is therefore considered too slow to be considered in a control-oriented model.	Similarly, also heat losses to the ambient are neglected since they change with changing shell and ambient temperature, both being too slow to be considered. The effect of this simplification can easily be compensated by adding integral action to the model-based controller to be designed. 
%		The effects of this modeling simplification in a model-based controller, these modeling simplifications will be \todo{taken care of} by a controller with integral action.
	\end{itemize}
	
%	\subsection{Model versions to be compared}
%			\label{subsec:versions}
%			
%			As mentioned at the beginning, three different model versions, differing in terms of the way the five heat exchangers are modeled, are compared in this paper. 
%	Heat transfer, in general, is a complex phenomenon that is usually modeled by means of semi-empirical heat transfer correlations when using lumped-component approaches, of which the following three will be used for the models in this paper (details on these heat transfer correlations will be given in \autoref{subsec:gen_con}-\ref{subsec:SHX}):
%	First, heat transfer can be modeled by means of the overall heat transfer coefficient $ \mathit{UA}$ as a HX parameter, which describes the heat transfer rate in dependence of the temperature difference between primary and secondary side of a HX, e.g. \cite{Jeong1998,Kohlenbach2008, Evola2013, Marc2015, Sabbagh2018, Ochoa2016, Zinet2012, delaCalle2016}. Second, it can be modeled with the effectiveness $ \epsilon $ as a HX parameter, which describes the ratio between the actual and the maximum possible heat flow rate of a HX, e.g. \cite{Jeong1998, Kohlenbach2008} \todo{more?}. Third, the terminal temperature differences $ \mathit{TTD} $ on either end of the HX can be used instead of modeling the transferred heat flow itself, e.g. \cite{Unterberger2020, Michel2013}.  
%	Other modeling approaches like calculating the heat transfer by means of $ Nusselt $-correlations, e.g. \cite{Ochoa2016, Zinet2012, delaCalle2016}, are considered to complex for use in control-oriented models and will not be further discussed here.
%
%	Different modeling approaches for the main component's HX and for the SHX are compared. They are combined to form three different model versions - V1, V2 and V3 - which are summarized in  \autoref{tab:model_version}.
%	The base case, V1, represents a modeling approach that is quite common in literature: 
%	Both the heat transfer in the main component's HX as well as in the SHX is
%	 modeled by means of constant heat transfer coefficients $ \mathit{UA}$, like e.g. in AHPD models in \cite{Jeong1998,Kohlenbach2008, Evola2013, Marc2015, Sabbagh2018} for the main component's HX and \cite{Evola2013, Marc2015, Sabbagh2018} for the SHX. Model version V2 and V3 contain the proposed new modeling approaches:
%	In model version V2 heat transfer in the main components' HX is still modeled in the same way but the heat transfer in the SHX is modeled by means of terminal temperature differences $ \mathit{TTD} $, similar to an approach that was successfully used for modeling plate heat exchangers in hydraulic systems in \cite{Unterberger2020}, extended by terms to consider the mass and energy storage capability of the solution inside the SHX and the piping between generator and absorber.
%	In model version V3 the heat transfer in the main components' HX is modeled by means of variable heat transfer coefficients $ \mathit{UA}$ for generator and absorber and variable effectiveness $ \epsilon $ for evaporator and condenser, i.e. $ \mathit{UA}$ and $ \epsilon $ vary with operating conditions like e.g. mass flow rates. The SHX is also modeled by the $ \mathit{TTD} $ approach mentioned above.
%	
%	
%	
%	\begin{table}[H]
%		\centering
%%		\setlength{\tabcolsep}{2.5pt} %% default is 6pt
%		\begin{tabular}{@{}ccc@{}}
%			\toprule
%			model version & main components' HX               & SHX                \\ \midrule
%			V1            & constant $ \mathit{UA}$         & constant $ \mathit{UA}$    \\
%			V2            & constant $ \mathit{UA}$        & variable $ \mathit{TTD} $\\
%			V3            & variable  $ \mathit{UA}$ /$\epsilon$ & variable $ \mathit{TTD} $ \\ \bottomrule
%		\end{tabular}
%		\caption{Overview over different model versions to be compared}%
%		\label{tab:model_version}
%	\end{table}
%	
%	To make the models' descriptions in the following section clearer, the remark \textit{V1, V2} or \textit{V3} is added to all equations that are only valid for a specific model version. Equations that apply equally to all three versions are not marked separately. 
	
	\subsection{Generator and condenser}
	\label{subsec:gen_con}
	
	Generator and condenser each consist of a plate heat exchanger and a fluid sump and are housed in the same shell, whose gas room is filled with refrigerant vapor at high pressure (high pressure gas room). This subsystem's inlet streams are the rich solution flow from the SHX, the cooling water flow coming from the absorber and the hot water flow.
	
	\subsubsection{Heat exchangers}
	
	
		
	
	The incoming rich solution flows over the generator HX, is heated up by the incoming hot water flow and part of the absorbed refrigerant is desorbed. 
	%	The resulting heat flow $ \dot{Q}_\r{G} $ depends on  operating conditions like mass flow rates and inlet temperatures on either side and is modeled 
A very common way to model heat transfer in lumped-component AHPD models, e.g.  \cite{Jeong1998,Kohlenbach2008, Evola2013, Marc2015, Sabbagh2018, Ochoa2016}, is to use the HX's 
so-called $ \mathit{UA}$-value, which is the product of the HX's overall heat transfer coefficient $ U $ and  its heat transfer area $ A $. This HX parameter can be used to describe the heat transfer rate in dependence of the temperature difference between primary and secondary side of a HX. Usually, in thermal engineering the so-called logarithmic mean temperature difference,  is used for that - a modeling approach that can, however, result in singular behavior \cite{Herold2016} (e.g., when the two temperature differences at the HX's ends are equal) and is mathematically quite complex. Therefore, instead, many AHPD models make use of polynomial approximations of the logarithmic mean temperature difference, e.g. \cite{ Sabbagh2018}, or a linear approximation, i.e., the arithmetic mean temperature difference,  e.g. \cite{Kohlenbach2008,  Ochoa2016}, which is also used for the modeling in this paper. This simplification is deliberately accepted due to the mentioned possibility of singular behavior of the logarithmic mean temperature difference. 
%, e.g. \cite{Jeong1998,Kohlenbach2008, Evola2013, Marc2015, Sabbagh2018, Ochoa2016, Zinet2012, delaCalle2016}.
% which is also adapted for the model in this thesis.  
	%For the generator the heat flow 
	$ \dot{Q}_{\r{G}} $ is therefore calculated with the generator HX's $ \mathit{UA}$-value $\mathit{UA} _\r{G}$ and  the (arithmetic) mean  temperature difference $\overline{\Delta T}_{\r{HX,G}}$ as
	\begin{equation}
		\label{eq:Qdot_G}
		\dot{Q}_\r{G}=\mathit{UA} _\r{G} \overline{\Delta T}_{\r{HX,G}}
	 \end{equation} 
%	with 
	%the mean HX temperature difference $\overline{\Delta T}_{\r{HX,G}}$
	\begin{equation}
		\label{eq:dT_G}
		\overline{\Delta T}_{\r{HX,G}}=\frac{T_{\r{W,G,in}}+T_{\r{W,G,out}}}{2}-\frac{T_{\r{RSo,SHX,out}}+T_{\r{PSo,HX,G,out}}}{2}
	 \end{equation} 
	consisting of the in- and outlet temperatures  of the solution, $ T_{\r{RSo,SHX,out}} $ and $ T_{\r{PSo,HX,G,out}} $, and of the hot water, $ T_{\r{W,G,in}} $ and $ T_{\r{W,G,out}} $. 
	Lumped-component models from  literature, see e.g. \cite{Jeong1998,Kohlenbach2008, Evola2013, Marc2015, Sabbagh2018}, often use constant $\mathit{UA}  $-values which, however, cannot capture the influence of varying mass flow rates in the HX. Instead, for the modeling in this paper, $ \mathit{UA} _\r{G} $ is described by 
	%however, $ \mathit{UA} _\r{G} $ is not assumed to be constant. Instead, $ \mathit{UA} _\r{G} $ is described by 
	\begin{equation}
		\label{eq:UA_G}
		\mathit{UA} _\r{G}=   \r{K_{G1} }+ \r{K_{G2}} \dot{m}_{\r{W,G}} + \r{K_{G3}} \dot{m}_{\r{RSo,SHX,out}} + \r{ K_{G4}} \dot{m}_{\r{Ref,GRh}} 
	 \end{equation} 
	where $ \mathit{UA} _\r{G} $  increases with an increasing mass flow rate of hot water $ \dot{m}_{\r{W,G}} $, rich solution $\dot{m}_{\r{RSo,SHX,out}} $ and desorbed refrigerant $  \dot{m}_{\r{Ref,GRh}} $. 
	The parameters $ \r{K_{G1} }  $ to $  \r{K_{G4} } $ can be found empirically from measurement data or by means of more complex modeling approaches, like $ Nusselt $-correlations (see e.g. \cite{Ochoa2016, Zinet2012, delaCalle2016} for shell and tube heat exchangers; to the authors' knowledge there are no validated $ Nusselt $-correlations for two-phase heat transfer in open plate heat exchangers for $\r{LiBr/H_2O} $ yet.). 
	The parameters in this paper for the investigated AHPD were derived from measurement data at different operating points and are listed in \autoref{sec:param_empirical}.
	

%Note that only the temperature of the (liquid) solution is considered here since it has a higher impact on the heat transfer than the temperature of the vaporous refrigerant.

	The heat flow $ \dot{Q}_\r{G} $ causes the hot water to cool down on the one hand and the solution to heat up and release part of the absorbed refrigerant on the other hand - a process that is of course subject to mass and energy conservation. 
	Since no energy storage is considered in the HX (see \autoref{subsec:rel_storage}) it is described by the following steady-state mass and energy balances: The mass flow of poor solution and refrigerant leaving the generator HX, $ \dot{m}_{\r{PSo,HX,G,out}} $ and $ \dot{m}_{\r{Ref,GRh}} $ must equal the entering mass flow of rich solution $ \dot{m}_{\r{RSo,SHX,out}} $:
\begin{equation}
	\label{eq:m_bal_HX_G}
0  	\left( =\frac{\r{d}m_{\r{So,HX,G}}}{\r{d}t} \right)= \dot{m}_{\r{RSo,SHX,out}}-\dot{m}_{\r{PSo,HX,G,out}}- \dot{m}_{\r{Ref,GRh}}
 \end{equation} 
Also the amount of LiBr leaving and entering the HX, $ [\dot{m} \xi]_{\r{PSo,HX,G,out}} $ and $  [\dot{m} \xi]_{\r{RSo,SHX,out}} $, must be equal:
\begin{equation}
	\label{eq:LiBr_bal_HX_G}
	0\left( =\frac{\r{d}m_{\r{LiBr,HX,G}}}{\r{d}t} \right)= [\dot{m} \xi]_{\r{RSo,SHX,out}} - [\dot{m} \xi]_{\r{PSo,HX,G,out}}
 \end{equation} 

Similarly, the energy balances on either side (hot water side and solution side) are as follows. Note that for the energy balances the stored enthalpy $ H $ instead of the internal energy $ U $ is considered since all liquids in this model are treated as incompressible fluids for which the difference between enthalpy and internal energy is negligible \cite{Moran2006}. 
\begin{equation}
		\label{eq:E_bal_HX_G_Sol}
\begin{split}
		0\left( = \frac{\r{d}H_{\r{So,HX,G}}}{\r{d}t} \right) = &[\dot{m} h]_{\r{RSo,SHX,out}}-[\dot{m} h]_{\r{PSo,HX,G,out}} \\
		& - [\dot{m} h]_{\r{Ref,GRh}} + \dot{Q}_\r{G}
\end{split}
 \end{equation} 
\begin{equation}
	\label{eq:E_bal_HX_G_W}
	0 \left( = \frac{\r{d}H_{\r{W,HX,G}}}{\r{d}t} \right)=\dot{m}_{\r{W,G}} (h_{\r{W,G,in}}-h_{\r{W,G,out}}) - \dot{Q}_\r{G}
 \end{equation} 
with the mass flow rate and specific enthalpy of the entering rich solution $[\dot{m} h]_{\r{RSo,SHX,out}}$, the leaving poor solution, $ [\dot{m} h]_{\r{PSo,HX,G,out}}$, the desorbed refrigerant $[\dot{m} h]_{\r{Ref,GRh}}  $ and of the entering and leaving hot water $\dot{m}_{\r{W,G}} (h_{\r{W,G,in}}-h_{\r{W,G,out}})$. 
Depending on whether volume or mass flow rates are used as input variables,
the  property functions below for density of $ \r{LiBr/H_2O} $ 	$ \rho_{\r{LiBr/H_2O}} $ and of liquid water $ \rho^{\r{l}}_{\r{H_2O}} $  can be used to calculated the mass flow rates  $\dot{m} _{\r{RSo,SHX,out}} $  and  $ \dot{m} _{\r{W,G}}  $ from corresponding volume flow rates if necessary.
% for \autoref{eq:m_bal_HX_G} to \autoref{eq:E_bal_HX_G_W}. 
\begin{equation}
	\label{eq:rho_LiBrH2O}
	\rho_{\r{LiBr/H_2O}}=\r{R_1} - \r{R_2} \xi+ \r{R_3} \xi^2 + \r{R_4} T 
 \end{equation} 
\begin{equation}
	\label{eq:rho_H2O}
	\rho^{\r{l}}_{\r{H_2O}}=\r{F_1} + \r{F_2} T- \r{F_3}  T^ 2
 \end{equation} 
 All  property  functions in this paper are simplified functions based on more complex  functions from  \cite{Yuan2005}  for $ \r{LiBr/H_2O} $ and from \cite{Wagner2002} for water. Their parameters are listed in \autoref{sec:param_substance_prop}.

The correlation between the specific enthalpies in \autoref{eq:E_bal_HX_G_Sol} and  \autoref{eq:E_bal_HX_G_W}, and the fluids' temperatures and 
%for  $ \r{LiBr/H_2O} $ also the fluids' 
mass fractions is  described by the following property functions for the solution, $h_{\r{LiBr/H_2O}}(T,\xi)$, the vaporous refrigerant, $h^\r{v}_{\r{H_2O}}(T)$, and the (liquid) hot water, $ h^\r{l}_{\r{H_2O}}(T) $:
\begin{equation}
	\label{eq:h_LiBrH2O}
	h_{\r{LiBr/H_2O}}(T,\xi)=-\r{A_1} - \r{A_2} \xi+\r{A_3} \xi^2 + \r{A_4}T - \r{A_5} \xi T 
 \end{equation} 
\begin{equation}
	\label{eq:h_v_H2O}
	h^\r{v}_{\r{H_2O}}(T)=\r{D_1} + \r{D_2}  T
 \end{equation} 
\begin{equation}
	\label{eq:h_l_H2O}
	h^\r{l}_{\r{H_2O}}(T)=-\r{C_1} + \r{C_2}  T
 \end{equation} 



% the equation for density can be used for conversion. 
%The mass flow rates  $\dot{m} _{\r{RSo,SHX,out}} $  and  $ \dot{m} _{\r{W,G}}  $ are calculated from the corresponding volume flow rates as input variables with the  property  functions for density of $ \r{LiBr/H_2O} $ 	$ \rho_{\r{LiBr/H_2O}} $ and of liquid water $ \rho^{\r{l}}_{\r{H_2O}} $.


For the temperatures of the desorbed refrigerant, $ T_{\r{Ref,GRh}} $, and of the poor solution $ T_{\r{PSo,HX,G,out}} $ the following simplifying assumptions can be made: 
First, it is assumed that the vaporous refrigerant leaving the generator HX has the same temperature as the entering rich solution since it flows upwards towards the rich solution inlet (see \autoref{fig:AHPD_Process}), i.e.
\begin{equation}
	\label{eq:T_Ref_HX_G_out}
	T_{\r{Ref,GRh}}= T_{\r{RSo,SHX,out}}
 \end{equation} 
%With equation \autoref{eq:E_bal_G_Sol} to \autoref{eq:LiBr_bal_G} and the correlation $\dot{m}=\rho \dot{V}  $ for  $\dot{m} _{\r{RSo,SHX,out}} $  and $ \dot{m} _{\r{W,G}}  $  there are eleven equations for 13 unknowns. To determine all unknowns two more equations are therefore necessary: 
Secondly, it is assumed that the poor solution leaving the generator HX is saturated, e.g. \cite{Jeong1998,Kohlenbach2008, Evola2013, Marc2015, Sabbagh2018, Ochoa2016}. Its temperature and LiBr mass fraction therefore correlate with the high pressure via the  property  function for saturation pressure of $\r{LiBr/H_2O} $, 
\begin{equation}
	\label{eq:p_high_G}
\r{	ln}(p_{\r{high}})=\r{	ln}(p_{\r{sat,\r{LiBr/H_2O}}}(T_{\r{PSo,HX,G,out}},\xi_{\r{PSo,HX,G,out}}))
 \end{equation} 
with
\begin{equation}
	\label{eq:p_sat_LiBrH2O}
	\r{	ln}(p_{\r{sat,\r{LiBr/H_2O}}}(T,\xi))=-\r{B_{1}}+\r{B_{2}} \xi - \r{B_{3} }\xi ^2 + \r{B_{4}} T
 \end{equation} 


The high pressure, however, also correlates with the condensation temperature in the condenser. Consequently, the  processes at the generator HX and the condenser HX are coupled by the high pressure. At the condenser HX surface the vaporous refrigerant is cooled down to condensation temperature, condenses and flows into the sump. 
It is common, see, e.g., \cite{Jeong1998,Kohlenbach2008, Evola2013, Marc2015, Sabbagh2018, Ochoa2016},  to assume, that the refrigerant leaves the condenser HX at saturation temperature. This means, its temperature correlates with the high pressure via the  property  function for saturation pressure of water: 
\begin{equation}
	\label{eq:p_high_C}
	\r{	ln}(p_{\r{high}})=\r{	ln}(p_{\r{sat,\r{H_2O}}}(T_{\r{Ref,HX,C,out}}))
 \end{equation} 
\begin{equation}
	\label{eq:p_sat_H2O}
	\r{	ln}(p_{\r{sat,H2O}}(T))=-\r{E_{1}}+ \r{E_{2}} T
 \end{equation} 

%The remaining modeling of the condenser HX is also very similar to the one of the generator and will only be briefly discussed. It is also based on heat transfer correlations, mass and energy balances and substance property equations.

The transferred heat flow $ \dot{Q}_\r{C} $ at the condenser HX could now also be modeled by an approach  similar to the generator, based on $ \mathit{UA} $, but it was found that the following approach based on the HX's effectiveness $ \epsilon $ as a HX parameter resulted in a simpler AHPD model with better accuracy.  $ \epsilon $  describes the ratio between the actual and the maximum possible heat flow rate of a HX and is used, e.g., in \cite{Jeong1998, Kohlenbach2008}. 
%is modeled by means of the effectiveness $ \epsilon_\r{C} $ as a HX parameter, which describes the ratio between the actual and the maximum possible heat flow rate of a HX. $ \dot{Q}_\r{C} $ could also be modeled by an approach with $ \r{UA} $ similar to the generator above but the effectiveness appraoch resulted in a simpler model with better accuracy. 
For the condenser, the maximum possible heat flow rate would be obtained if the cooling water outlet temperature reached the temperature of the refrigerant at the HX surface, i.e., if $ T_{\r{W,AC,out}} $ equals $ T_{\r{Ref,HX,C,out}} $. Therefore  $ \dot{Q}_\r{C} $ is modeled by 
\refstepcounter{equation}\label{myeqn1}% Correctly mark and label equation
\begin{equation}
	\label{eq:Qdot_C_epsilon}
	\dot{Q}_\r{C}=\epsilon_\r{C}~ \dot{m}_{\r{W,AC}}~ c_{p,\r{W}} \left(T_{\r{Ref,HX,C,out}}- T_{\r{W,C,in}}\right) 
 \end{equation} 

with the constant specific heat capacity $ c_{p,\r{W}} $ ($ c_{p,\r{W}} =C_2$ ) and where the effectiveness $\epsilon_\r{C}  $ is modeled as a function of the cooling water mass flow rate $ \dot{m}_{\r{W,AC}}  $:
\begin{equation}
	\label{eq:epsilon_C}
	\epsilon_\r{C}=\r{K_{C,1} }- \r{K_{C,2} }\dot{m}_{\r{W,AC}} 
 \end{equation} 


Moreover, the following mass and energy balances apply:
\begin{equation}
	\label{eq:m_bal_HX_C}
	0= \dot{m}_{\r{Ref,GRh}}-\dot{m}_{\r{Ref,HX,C,out}}
 \end{equation} 
\begin{equation}
	\label{eq:E_bal_HX_C_Ref}
	0= [\dot{m}h]_{\r{Ref,GRh}} -[\dot{m}h]_{\r{Ref,HX,C,out}}- \dot{Q}_\r{C}
 \end{equation} 
\begin{equation}
		\label{eq:E_bal_HX_C_W}
	0=\dot{m}_{\r{W,AC}} (h_{\r{W,C,in}}-h_{\r{W,C,out}}) + \dot{Q}_\r{C}
 \end{equation} 

The correlation between specific enthalpies and temperatures is again described by the  property  function \autoref{eq:h_v_H2O} for vaporous and \autoref{eq:h_l_H2O} for liquid refrigerant and the cooling water.

\subsubsection{Fluid sumps}

Both in the generator and the condenser, the liquid fluid leaving the HXs flows into the corresponding fluid sump. These sumps  are approximated as continuously stirred tank reactors, i.e., their mass and energy storage capability is modeled by lumped differential mass and energy balances. 
For the generator sump, the mass and energy balances are therefore 
\begin{equation}
	\label{eq:dm_dt_G}
	\frac{\r{d}m_{\r{PSo,G}}}{\r{d}t}= \dot{m}_{\r{PSo,HX,G,out}}- \dot{m}_{\r{PSo,SHX,in}}
 \end{equation} 
\begin{equation}
		\label{eq:dm_dt_G_LiBr}
	\frac{\r{d}m_{\r{LiBr,G}}}{\r{d}t}=  [\dot{m} \xi]_{\r{PSo,HX,G,out}} - \dot{m}_{\r{PSo,SHX,in}} \frac{m_{\r{LiBr,G}}}{m_{\r{PSo,G}}} 
 \end{equation} 
\begin{equation}
			\label{eq:dE_dt_G}
	\frac{\r{d}H_{\r{PSo,G}}}{\r{d}t}=[\dot{m} h]_{\r{PSo,HX,G,out}}- \dot{m}_{\r{PSo,SHX,in}} \frac{H_{\r{PSo,G}}}{m_\r{{PSo,G}}} 
 \end{equation} 

with the stored poor solution mass and enthalpy $ m_{\r{PSo,G}} $ and  $ H_{\r{PSo,G}} $, the stored LiBr mass  $ m_{\r{LiBr,G}} $ and the mass flow rate of the poor solution flowing out of the sump $\dot{m}_{\r{PSo,SHX,in}}  $.

For the condenser sump, measurement data of the investigated AHPD showed that the mass inside the fluid sump hardly ever changes, therefore $ \dot{m}_{\r{Ref,C,out}}  $ is set equal to	$ \dot{m}_{\r{Ref,HX,C,out}}  $.
% it is assumed that the mass inside the sump is constant, which means that $\dot{m}_{\r{Ref,E,in}} $  equals $ \dot{m}_{\r{Ref,cond}} $. Measurement data confirm the validity of this assumption. 
The mass and energy balances for the condenser are therefore   
\begin{equation}
	\label{eq:dm_dt_C}
	0   = \dot{m}_{\r{Ref,HX,C,out}}-\dot{m}_{\r{Ref,C,out}} 
 \end{equation} 
\begin{equation}
		\label{eq:dE_dt_C}
	\frac{\r{d}H_{\r{Ref,C}}}{\r{d}t}=  [\dot{m}h]_{\r{Ref,HX,C,out}}  - \dot{m}_{\r{Ref,C,out}} \frac{H_\r{{Ref,C}}}{m_{\r{Ref,C}}} 
 \end{equation} 
with the stored refrigerant mass and enthalpy $ m_{\r{Ref,C}} $ and  $ H_{\r{Ref,C}} $ and the mass flow rate of the refrigerant flowing out of the sump $\dot{m}_{\r{Ref,C,out}}  $.


	\subsection{Absorber and evaporator}
	\label{subsec:abs_eva}

	The modeling of the absorber and evaporator is very similar to the modeling of the generator and condenser described in the section before. Therefore, the description here will mainly be restricted to listing the equations and discussing differences where applicable. 
	
		\subsubsection{Heat exchangers}
		
		
		The heat and mass transfer at the absorber HXs is modeled by
	\begin{equation}
		\label{eq:Qdot_A}
		\dot{Q}_\r{A}=\mathit{UA}_\r{A} \overline{\Delta T}_{\r{HX,A}}
	 \end{equation} 
	\begin{equation}
		\label{eq:dT_A}
		\overline{\Delta T}_{\r{HX,A}}=\frac{T_{\r{PSo,SHX,out}}+T_{\r{RSo,HX,A,out}}}{2}-\frac{T_{\r{W,A,in}}+T_{\r{W,A,out}}}{2}
	 \end{equation} 
%\begin{equation}
%	\label{eq:UA_A_c}
%		\text{\small{V1,V2:~~~}}	
%	\mathit{UA} _\r{A}=\mathit{UA} _\r{A,const.}=const.
% \end{equation} 
\begin{equation}
\mathit{UA}_\r{A} =\r{K_{A1}} + \r{K_{A2}} \dot{m}_{\r{W,AC}} + \r{K_{A3}}  \dot{m}_{\r{PSo,A,in}} + \r{K_{A4}} \dot{m}_{\r{Ref,GRl}} 
	\label{eq:UA_A}
 \end{equation} 
	\begin{equation}
		\label{eq:m_bal_HX_A}
		0= \dot{m}_{\r{PSo,A,in}}-\dot{m}_{\r{RSo,HX,A,out}} + \dot{m}_{\r{Ref,GRl}} 	
	 \end{equation} 
	\begin{equation}
				\label{eq:LiBr_bal_HX_A}
		0=  [\dot{m}\xi]_{\r{PSo,A,in}}-[\dot{m}\xi]_{\r{RSo,HX,A,out}} 
	 \end{equation} 
	\begin{equation}
				\label{eq:E_bal_HX_A_Sol}		
		0=  [\dot{m}h]_{\r{PSo,A,in}}-[\dot{m}h]_{\r{RSo,HX,A,out}} + [\dot{m}h]_{\r{Ref,GRl}} 	- \dot{Q}_\r{A}
	 \end{equation} 
	\begin{equation}
				\label{eq:E_bal_HX_A_W}				
		0=\dot{m}_{\r{W,AC}} (h_{\r{W,A,in}}-h_{\r{W,A,out}}) + \dot{Q}_\r{A}
	 \end{equation} 

One difference between the modeling of the generator and the absorber is that the solution leaving the HX of the absorber is assumed to be subcooled rather than saturated. Thus, it is assumed that the absorption process is not ideal and requires a slightly larger temperature difference. 
	To model the subcooling process, it is assumed that a small fraction of the absorber heat flow  $\phi_\r{sub}\dot{Q}_\r{A}$ accounts for subcooling the solution from saturation temperature $T_{\r{RSo,HX,A,out,sat}}$ to the actual outlet temperature  $T_{\r{RSo,HX,A,out}}$, while the rest of  $\dot{Q}_\r{A}$ accounts for the absorption process without subcooling. For the AHPD used in this paper $\phi_\r{sub}$ is estimated to be app. \SI{8}{\percent} on average by means of measurement data analysis. Note that this value merely affects steady-state accuracy and can also be set to \SI{0}{\percent} if it cannot be determined, as suggested in, e.g., \cite{Ochoa2016, Kohlenbach2008, Misenheimer2017, Evola2013, Sabbagh2018}, without noticeably affecting the model's dynamic accuracy, i.e., how well the model can reproduce the AHPD's transient behavior.
	The saturation temperature is calculated with the   property  function for saturation pressure of $\r{LiBr/H_2O}$,  \autoref{eq:p_sat_LiBrH2O}, and
	%\begin{equation}
	%	ln(p_l) = B_{l1}+B_{l2} x_{\r{RSo,HX,A,out}} + B_{l3} x_{\r{RSo,HX,A,out}}^2 + B_{l4} T_{\r{RSo,HX,A,out,sat}}.
	% \end{equation} 
	\begin{equation}
		\label{eq:p_low_A}
		\r{	ln}(p_{\r{low}})=\r{	ln}(p_{\r{sat,\r{LiBr/H_2O}}} (T_{\r{RSo,HX,A,out,sat}},\xi_{\r{RSo,HX,A,out}}))
	 \end{equation} 
	The specific enthalpy of the rich solution leaving the HX is then calculated with
	\begin{equation}
		\label{eq:h_RSo_HX_A_out}
		h_{\r{RSo,HX,A,out}}=h_{\r{RSo,HX,A,out,sat}}-\frac{  \phi_\r{sub}\dot{Q}_\r{A}}{\dot{m}_{\r{RSo,HX,A,out}}} 
	 \end{equation} 
	
	with $ h_{\r{RSo,HX,A,out,sat}}=h(T_{\r{RSo,HX,A,out,sat}}, \xi_{\r{RSo,HX,A,out}}) $ and $ h_{\r{RSo,HX,A,out}}=h( T_{\r{RSo,HX,A,out}},\xi_{\r{RSo,HX,A,out}} ) $ from \autoref{eq:h_LiBrH2O}. 
	
	
	
		The heat and mass transfer at the evaporator HX is modeled by
%\begin{equation}
%	\label{eq:Qdot_E_UA}
%		\text{\small{V1,V2:~~~}}	
%		\dot{Q}_\r{E}=UA_\r{E} \overline{\Delta T}_{\r{HX,E}}
%	 \end{equation} 
%\begin{equation}
%	\label{eq:dT_E}
%		\text{\small{V1,V2:~~~}}	
%		\overline{\Delta T}_{\r{HX,E}}=\frac{T_{\r{W,E,in}}+T_{\r{W,E,out}}}{2}-\frac{T_{\r{Ref,rec}}+T_{\r{Ref,HX,E,out}}}{2}
%	 \end{equation} 
\begin{equation}
	\label{eq:Qdot_E_epsilon}
		\dot{Q}_\r{E}=	\epsilon_\r{E} ~\dot{m}_{\r{W,E}} ~c_{p,\r{W}}~ \left(T_{\r{W,E,in}}-T_{\r{Ref,rec}}\right) 
	 \end{equation} 
\begin{equation}
			\label{eq:epsilon_E}
		\epsilon_\r{E}=\r{K_{E1}} -\r{K_{E2}} \dot{m}_{\r{W,E}}
	 \end{equation} 
	\begin{equation}
				\label{eq:m_bal_HX_E}
		0=\dot{m}_{\r{Ref,rec}}-\dot{m}^\r{l}_{\r{Ref,HX,E,out}}-\dot{m}^\r{v}_{\r{Ref,HX,E,out}}
	 \end{equation} 
	\begin{equation}
		\label{eq:E_bal_HX_E_Ref}
		0=[\dot{m}h]_{\r{Ref,rec}} 
		-[\dot{m}h]^\r{l}_{\r{Ref,HX,E,out}} 
		-[\dot{m}h]^\r{v}_{\r{Ref,HX,E,out}} +	\dot{Q}_\r{E}
	 \end{equation} 
	\begin{equation}
				\label{eq:E_bal_HX_E_W}
		0=\dot{m}_{\r{W,E}} (h_{\r{W,E,in}}-h_{\r{W,E,out}}) - \dot{Q}_\r{E}
	 \end{equation} 
	\begin{equation}
		\label{eq:p_low_E}
		\r{	ln}(p_{\r{low}})=\r{	ln}(p_{\r{sat,\r{H_2O}}}(T_{\r{Ref,HX,E,out}}))
	 \end{equation} 
The correlation between specific enthalpies and temperatures is again described by the  property  functions \autoref{eq:h_LiBrH2O} for $ \r{LiBr/H_2O} $, \autoref{eq:h_v_H2O} for vaporous refrigerant and \autoref{eq:h_l_H2O} for liquid refrigerant and water in the hydraulic circuits. Similarly, equation \autoref{eq:rho_H2O} for the density of water can be used to calculate mass flow rates from corresponding  volume flow rates.
% used as input variables to the model.  
	
		\subsubsection{Fluid sumps}
	

For the mass balance of the two remaining sumps it has to be considered that the overall mass of refrigerant and LiBr inside the AHPD remains constant. Therefore, one differential LiBr and one differential overall mass balance can be omitted and replaced by corresponding  steady-state balances over all sumps. 
For the absorber sump only the differential overall mass and energy balances are required and the transient LiBr balance is replaced by a steady-state LiBr balance: 
%In combination with the transient LiBr  balance in the generator sump \autoref{eq:dm_dt_G_LiBr} this fully describes the dynamics of the stored LiBr in the absorber sump. 
%Instead of a transie
% since the dynamics of the stored LiBr are already defined through the differential LiBr balance in the generator sump \autoref{eq:dm_dt_G_LiBr}. Instead, the conservation of LiBr mass inside the AHPD is expressed by a steady-state mass balance. 
%The corresponding balances for the absorber sump are then
\begin{equation}
	\label{eq:dm_dt_A}
	\frac{\r{d}m_{\r{RSo,A}}}{\r{d}t}= [\dot{m}]_{\r{RSo,HX,A,out}} - \dot{m}_{\r{RSo,A,out}} 
 \end{equation} 
\begin{equation}
	\label{eq:LiBr_bal}
	0=m_{\r{LiBr,A}}+ m_{\r{LiBr,G}}-m_{\r{LiBr,sumps}}
 \end{equation} 
\begin{equation}
	\label{eq:dE_dt_A}
	\frac{\r{d}H_{\r{RSo,A}}}{\r{d}t}= [\dot{m}h]_{\r{RSo,HX,A,out}} - \dot{m}_{\r{RSo,A,out}} \frac{H_{\r{RSo,A}}}{m_{\r{RSo,A}}}
 \end{equation} 
where  $m_{\r{LiBr,sumps}}$ is the (constant) LiBr mass  in all sumps. 

For the evaporator sump only the transient energy balance is required and the transient mass balance can be replaced by a steady-state balance: 
\begin{equation}
	\label{eq:total_bal}
	0=m_{\r{Ref,E}}+m_{\r{Ref,C}}+m_{\r{RSo,A}}+m_{\r{PSo,G}}-m_{\r{LiBr+H_2O,sumps}}
 \end{equation} 
\begin{equation}
	\label{eq:dE_dt_E}
	\frac{\r{d}H_{\r{Ref,E}}}{\r{d}t}=  [\dot{m}h]^\r{l}_{\r{Ref,E,in}}+[\dot{m}h]^\r{l}_{\r{Ref,HX,E,out}} -\dot{m}_{\r{Ref,rec}} \frac{H_{\r{Ref,E}}}{m_{\r{Ref,E}}}
 \end{equation} 
where $m_{\r{LiBr+H_2O,sumps}}$ is the (constant)  total fluid mass (LiBr and refrigerant) in all sumps. 

	



	\subsection{Solution heat exchanger (SHX)}
	\label{subsec:SHX}
	
	Similar to the modeling of the heat transfer in the HXs of the main components in  \autoref{subsec:gen_con} and \ref{subsec:abs_eva}, it is common to also model the heat transfer in the SHX with a constant $ \mathit{UA}$-value, e.g. \cite{Evola2013, Marc2015, Sabbagh2018}. From measurement data analysis of the investigated AHPD it was found that this approach can reproduce the steady-state behavior well, but not the dynamic  response to variations of the flow rate of rich solution, as will be shown later in \autoref{subsec:dynamic_val}. 
	Instead, it is suggested to model the heat transfer in the SHX based on the terminal temperature differences  ($ \mathit{TTD} $), also referred to as approach temperature difference, on either end of the SHX instead of modeling the transferred heat flow itself. Such a modeling approach is discussed  in \cite{Unterberger2020}, where it is used to model the operating behavior of a plate heat exchanger with varying mass flow rates. It should be mentioned that energy conservation cannot be guaranteed with this approach. However, it can reproduce the dynamic effects of volume flow variations very well and  small steady-state deviations in the energy balance can typically be tolerated in control-oriented models as long as the basic  dynamics are still represented well and the system remains stable. 
%	, as described in  \cite{Unterberger2020} for 
%	see e.g. \cite{Unterberger2020, Michel2013}.  
	Based on \cite{Unterberger2020} the $ \mathit{TTD} $ at steady-state for the hot end of the SHX, $ \mathit{TTD}_{\r{h}} $, can then be modeled by 
	\begin{equation}
		\label{eq:delta_T_SHX_h}
		\begin{split}
			\mathit{TTD}_{\r{h}}&=
			T_{\r{PSo, SHX,in}} -T_{\r{RSo,SHX,out,ss}} 	\\
			&=f_{\r{h}} \left([\dot{m}c_p]_{\r{RSo}},[\dot{m}c_p]_{\r{PSo}}  \right) \Delta T_{\r{SHX,in}} 
		\end{split}
	 \end{equation} 
	where $ T_{\r{RSo,SHX,out,ss}} $ corresponds to the steady-state outlet temperature of the rich solution and $  \Delta T_{\r{SHX,in}} $ to the SHX's inlet temperature difference $ (T_{\r{PSo,SHX,in}} -T_{\r{RSo,SHX,in}})  $.
	%\begin{equation}
	%	\label{eq:delta_T_SHX_h}
	%	\Delta T_{\r{SHX,in}} = T_{\r{PSo, SHX,in}} -T_{\r{RSo, SHX,in}} 
	% \end{equation} 
	In \cite{Unterberger2020} a polynomial  of degree four is used for $f_{\r{h}} $. Here, however, a linear correlation is chosen, since the mass flow rates of rich and poor solution are usually very similar, so that a linear approximation is considered sufficient, which yields
	\begin{equation}
		f_{\r{h}} =\r{K_{h1}}+\r{K_{h2}}[\dot{m}c_p]_{\r{RSo}}-\r{K_{h3}} [\dot{m}c_p]_{\r{PSo}}
	 \end{equation} 
	Parameters  $ \r{K_{h1}}$ to $ \r{K_{h3}} $ can be determined empirically or by means of other more detailed models, such as discretized models. The specific heat capacities $ c_{p,\r{RSo}} $ and $ c_{p,\r{PSo}} $ are calculated by the corresponding  property  function:
	\begin{equation}
			\label{eq:cp_LiBrH2O}
			c_{p,\r{LiBr/H_2O}}(\xi)=\left( \frac{\partial h_{\r{LiBr/H_2O}}}{\partial T} \right)_{p,\xi}=  \r{A_4} - \r{A_5} \xi
		 \end{equation} 
	Similarly, the $ \mathit{TTD} $ at the cooler end of the SHX, $ \mathit{TTD}_{\r{c}} $, can be modeled by 
	\begin{equation}
		\label{eq:delta_T_SHX_l}
		\begin{split}
			\mathit{TTD}_{\r{c}} &=
			T_{\r{PSo,SHX,out,ss}} -T_{\r{RSo, SHX,in}} \\
			%		T^*_{\r{RSo,SHX,out}} -T_{\r{PSo, SHX,in}}=
			&= f_{\r{c}} \left([\dot{m}c_p]_{\r{RSo}},[\dot{m}c_p]_{\r{PSo}}  \right)	\Delta T_{\r{SHX,in}}
		\end{split}
	 \end{equation} 
%	with 
	\begin{equation}
		f_{\r{c}} =	\r{K_{c1}}-\r{K_{c2}}[\dot{m}c_p]_{\r{RSo}}+\r{K_{c3}} [\dot{m}c_p]_{\r{PSo}}
	 \end{equation} 
	where $ T_{\r{PSo,SHX,out,ss}} $ corresponds to the steady-state outlet temperature of the poor solution and  $ \r{K_{c1}}$ to $ \r{K_{c3}} $ to the corresponding parameters. 
	
	
	
%	%%%%%%%%%
%	Similar to the modeling of the heat transfer in the HX of the main components in  \autoref{subsec:gen_con} and \ref{subsec:abs_eva}, it is common to model the heat transfer in the SHX with a constant overall heat transfer coefficient $\mathit{UA}$, e.g. \cite{Evola2013, Marc2015, Sabbagh2018}. Therefore, in model version V1, the following heat transfer correlation is used for the SHX.
%\begin{equation}
%		\label{eq:Qdot_SHX}
%			\text{\small{V1:~~~}}	
%		\dot{Q}_\r{SHX}=UA_\r{SHX} \overline{\Delta T}_{\r{SHX}}
%	 \end{equation} 
%	with the mean HX temperature difference $\overline{\Delta T}_{\r{SHX}}$
%\begin{equation}
%		\label{eq:T_mean_SHX}
%		\begin{split}
%			\text{\small{V1:~~~}}	
%		\overline{\Delta T}_{\r{SHX}}=&\frac{T_{\r{PSo,SHX,in}}+T_{\r{PSo,SHX,out}}}{2}- \\
%		&\frac{T_{\r{RSo,SHX,in}}+T_{\r{RSo,SHX,out}}}{2}
%				\end{split}
%	 \end{equation} 
%	with the in- and outlet temperatures of the  poor solution $ T_{\r{PSo,SHX,in}}$ and $ T_{\r{PSo,SHX,out}}$ and of the rich solution $ T_{\r{RSo,SHX,in}}$ and $ T_{\r{RSo,SHX,out}}$. 
%%	Outlet temperatures can then be determined by the energy balances on the rich and poor solution side of the SHX, \autoref{eq:E_bal_SHX_RSo} and \autoref{eq:E_bal_SHX_PSo},  in combination with the substance property data function for specific enthalpy of $ \r{LiBr/H_2O} $, \autoref{eq:h_LiBrH2O}.
%	The energy balance on either side (rich and poor solution) are 
%\begin{equation}
%		\label{eq:E_bal_SHX_RSo}
%			\text{\small{V1:~~~}}	
%		0=\dot{m}_{\r{RSo,SHX,in}} (h_{\r{RSo,SHX,in}}-h_{\r{RSo,SHX,out}})+\dot{Q}_\r{SHX}
%	 \end{equation} 
%\begin{equation}
%		\label{eq:E_bal_SHX_PSo}
%			\text{\small{V1:~~~}}	
%		0=\dot{m}_{\r{PSo, SHX, in}} (h_{\r{PSo,SHX,in}}-h_{\r{PSo,SHX,out}})-\dot{Q}_\r{SHX}
%	 \end{equation} 
%	where the correlation between enthalpies, temperatures and mass fractions is again described by \autoref{eq:h_LiBrH2O}. 
%%	 the substance property data function for specific enthalpy of $ \r{LiBr/H_20} $
%	%While this is a common approach in literature, e.g. \cite{Evola2013, Marc2015, Sabbagh2018}, these papers either stated that the flow rate of rich solution does not change \cite{Marc2015, Evola2013} or its effects were at least not validated experimentally \cite{Sabbagh2018}. 
%	%Model version V1 with the heat transfer correlation \autoref{eq:Qdot_SHX1} 
%	This approach shows good steady-state accuracy but is not able to describe the dynamic response to variations of the flow rate of rich solution correctly, as will be shown later in \autoref{sec:validation}. Therefore, for model version V2 and V3 the following modeling approach is suggested instead: Here, the heat transfer in the SHX is modeled based on an approach presented in \cite{Unterberger2020}, where the terminal temperature differences ($ \mathit{TTD} $) on both ends of the HX are modeled by empirical equations. It should be mentioned that energy conservation cannot be guaranteed with this approach but
%	%However, while this is usually a basic requirement for thermodynamic models, 
%	small deviations can typically be tolerated in control-oriented models as long as the basic system dynamics are still captured well and the system remains stable. 
%	%y can be compensated by adding integral action to the model-based controller to be designed.
%	In \cite{Unterberger2020} it is suggested to model the $ \mathit{TTD} $ at steady-state as a function of the inlet temperature difference  $ \Delta T_{\r{in}} $ and the heat capacity flow rates of the fluids on both sides, $[\dot{m}c_p]_{\r{Fluid1}}$ and $[\dot{m}c_p]_{\r{Fluid2}}$, as
%\begin{equation}
%		\label{eq:TTD}
%			\text{\small{V2,V3:~~~}}	
%			\mathit{TTD} =f\left([\dot{m}c_p]_{\r{Fluid1}}, [\dot{m}c_p]_{\r{Fluid2}}\right) \Delta T_{\r{in}} 
%	 \end{equation} 
%	In \cite{Unterberger2020} a polynomial  of degree four is used for $f\left([\dot{m}c_p]_{\r{Fluid1}}, [\dot{m}c_p]_{\r{Fluid2}}\right)$. Here, however, a linear correlation is chosen, since the mass flow rates of rich and poor solution are usually very similar, so that a linear approximation is legit. 
%	Therefore, for the modeling of the SHX, the following correlations are used: The $ \mathit{TTD} $ at steady-state at the high temperature end of the SHX,  $  \mathit{TTD}_{\r{h}} $ , is modeled by
%\begin{equation}
%		\label{eq:delta_T_SHX_h}
%		\begin{split}
%				\text{\small{V2,V3:~~~}}	
%			\mathit{TTD}_{\r{h}}&=
%			T_{\r{PSo, SHX,in}} -T_{\r{RSo,SHX,out,ss}} \\
%			%		T^*_{\r{RSo,SHX,out}} -T_{\r{PSo, SHX,in}}=
%			&=	\left(\r{K_{h1}}+\r{K_{h2}}[\dot{m}c_p]_{\r{RSo}}-\r{K_{h3}} [\dot{m}c_p]_{\r{PSo}}\right) \Delta T_{\r{SHX,in}} 
%		\end{split}
%	 \end{equation} 
%	where $ T_{\r{RSo,SHX,out,ss}} $ corresponds to the steady-state outlet temperature of the rich solution and $  \Delta T_{\r{SHX,in}} $ to the SHX's inlet temperature difference $ (T_{\r{PSo,SHX,in}} -T_{\r{RSo,SHX,in}})  $. Parameters  $ \r{K_{h1}}$ to $ \r{K_{h3}} $ have to be determined empirically or by means of other more detailed models, like e.g. discretized models. Similarly, the $ \mathit{TTD} $ at steady-state at the low temperature end of the SHX,  $  \mathit{TTD}_{\r{l}} $ , is modeled by
%\begin{equation}
%		\label{eq:delta_T_SHX_l}
%		\begin{split}
%				\text{\small{V2,V3:~~~}}	
%			\mathit{TTD}_{\r{l}} &=
%			T_{\r{PSo,SHX,out,ss}} -T_{\r{RSo, SHX,in}} \\
%			%		T^*_{\r{RSo,SHX,out}} -T_{\r{PSo, SHX,in}}=
%			&=	\left(\r{K_{l1}}-\r{K_{l2}}[\dot{m}c_p]_{\r{RSo}}+\r{K_{l3}} [\dot{m}c_p]_{\r{PSo}}\right) \Delta T_{\r{SHX,in}}
%		\end{split}
%	 \end{equation} 
%	where $ T_{\r{PSo,SHX,out,ss}} $ corresponds to the steady-state outlet temperature of the poor solution and  $ \r{K_{l1}}$ to $ \r{K_{l3}} $ to the corresponding parameters. The specific heat capacities $ c_{p,\r{RSo}} $ and $ c_{p,\r{PSo}} $ are calculated by the corresponding  property  function,
%\begin{equation}
%		\label{eq:cp_LiBrH2O}
%						\text{\small{V2,V3:~~~}}	
%		c_{p,\r{LiBr/H_2O}}(\xi)=\left( \frac{\partial h_{\r{LiBr/H_2O}}}{\partial T} \right)_{p,\xi}=  \r{A_4} - \r{A_5} \xi
%	 \end{equation} 
%		%%%%%%%%%
	
	In order to also consider the SHX's dynamics in a simple way, the model from \cite{Unterberger2020} is extended by adding first-order  delay elements with mass-flow-dependent time constants on the rich and the poor solution side:
%	 in the form of the following differential energy balances:
		\begin{equation}
		\small
		\label{eq:T_bal_SHX_RSo}
		\frac{\r{d}H_{\r{RSo,SHX}}}{\r{d}t}= \frac{\r{d}[mh]_{\r{RSo,SHX}}}{\r{d}t}=\dot{m}_{\r{RSo}} (h_{\r{RSo,SHX,out,ss}}-h_{\r{RSo,SHX}})
	 \end{equation} 
	\begin{equation}
		\small
		\label{eq:T_bal_SHX_PSo}
		\frac{\r{d}H_{\r{PSo,SHX}}}{\r{d}t}=\frac{\r{d}[mh]_{\r{PSo,SHX}}}{\r{d}t}=\dot{m}_{\r{PSo}} (h_{\r{PSo,SHX,out,ss}}-h_{\r{PSo,SHX}})
	 \end{equation} 
	
	%\begin{equation}
	%	\label{eq:T_bal_SHX_RSo}
	%	\frac{\r{d}T_{\r{RSo,SHX,out}}}{\r{d}t}=\frac{\dot{m}_{\r{RSo}}}{m_{\r{RSo,SHX}}} (T_{\r{RSo,SHX,out,ss}}-T_{\r{RSo,SHX,out}})
	% \end{equation} 
	%\begin{equation}
	%	\label{eq:T_bal_SHX_PSo}
	%	\frac{\r{d}T_{\r{PSo,SHX,out}}}{\r{d}t}=\frac{\dot{m}_{\r{PSo}}}{m_{\r{PSo,SHX}}} (T_{\r{PSo,SHX,out,ss}}-T_{\r{PSo,SHX,out}})
	% \end{equation} 
	
	where $ h_{\r{RSo,SHX,out,ss}} $ and $ h_{\r{PSo,SHX,out,ss}} $ are the specific enthalpies correlating with $ T_{\r{RSo,SHX,out,ss}} $ and $ T_{\r{PSo,SHX,out,ss}} $ from \autoref{eq:delta_T_SHX_h} and \autoref{eq:delta_T_SHX_l}, and $ h_{\r{RSo,SHX,out}} $ and $ h_{\r{PSo,SHX,out}} $ are the specific enthalpies stored in the fluid in the SHX. Again, the correlations between specific enthalpies, temperatures and mass fractions are described by the corresponding property function \autoref{eq:h_LiBrH2O}. The masses of the solution in the SHX,  $ m_{\r{RSo,SHX}} $ and  $ m_{\r{PSo,SHX}} $, can be assumed constant with sufficient accuracy. The modeling scheme of the SHX dynamics is also depicted in \autoref{fig:SHX_dyn}, for the rich solution side as an example. 
		\begin{figure}[h!]
		\centering
				\def\svgwidth{250pt}
				\footnotesize
		\input{pics/SHX_dyn.pdf_tex}
		\caption{Modeling scheme of SHX dynamics for the rich solution}
		\label{fig:SHX_dyn}%
	\end{figure}

	As mentioned before, energy conservation cannot be guaranteed with this approach since it does not contain an energy balance that correlates the transferred energy on the rich and poor solution side but directly gives the two outlet temperatures instead (otherwise the model would be overdetermined).  For the investigations in this paper, the error of such a steady-state energy balance over the SHX was in the range of \SIrange{0}{12}{\%}, depending on the operating point, which is considered sufficiently small for a control-oriented model.
	
	
	\subsection{Expansion valves}
	\label{subsec:expansion_valves}
	
	In the two expansion valves (REV and SEV), the solution and refrigerant, respectively, expand from high to low pressure level.
%	 The mass flow rate through the valves is 
%	separate the high and low pressure components  
%		The pressure difference across the two expansion valves - the solution and the refrigerant expansion valve (REV and SEV) -  determine the mass flow rates between the high pressure components generator and condenser and the low pressure components absorber and evaporator.
		%The two valves, the refrigerant and the solution expansion valves (REV and SEV), cause a pressure drop in the fluids and 
	Since heat losses are neglected, the flow through the REV and SEV is modeled as an adiabatic process, i.e., the expansion process is assumed to be isenthalpic, which is a very common assumption in the literature on modeling of $ \r{LiBr/H_2O} $ AHPDs \cite{Jeong1998, Kohlenbach2008,  Evola2013, Marc2015,  Sabbagh2018,Ochoa2016,Zinet2012,delaCalle2016, Castro2020}. Both valves have a constant opening and are thus modeled as such.
	
	%\todo{die folgenden Abschnitte kürzen}
	
	\subsubsection{Solution expansion valve (SEV)}
	
%	The SEV is positioned after the SHX and separates the high and low pressure components of the 
%	expands the poor solution from high to low pressure. 
%	In the SEV the poor solution expands from high to low pressure. 
	
The expansion is assumed to be isenthalpic, i.e.
		\begin{equation}
			\label{eq:h_SEV}
	h_{\r{PSo,SHX,out}}=	h_{\r{PSo,A,in}}
	 \end{equation} 
	
	And assuming that the poor solution entering the SEV is sufficiently subcooled by the preceding SHX \cite{Kohlenbach2008, Jeong1998, Evola2013, Marc2015, Sabbagh2018}, no refrigerant desorbs during this process, so that the temperature before and after the SEV can be assumed to be the same, i.e.
			\begin{equation}
				\label{eq:T_SEV}
		T_{\r{PSo,SHX,out}}=	T_{\r{PSo,A,in}}
	 \end{equation} 
	
The mass flow rate through the SEV, $ \dot{m}_{\r{PSo,A,in}}  $, correlates with the pressure difference $ p_{\r{high}}-p_{\r{low}} $, the filling level in the generator and the poor solution's density. And although existing models from literature \cite{Kohlenbach2008,  Evola2013, Marc2015,  Sabbagh2018,Ochoa2016, Castro2020}  usually do consider these three influencing factors, it was found that the following much simpler approach yields satisfactory modeling accuracy too and is therefore favored over more complex approaches. 
In this paper $ \dot{m}_{\r{PSo,A,in}} $ is modeled in such a way that it linearly scales with the stored mass in the generator sump only, i.e.
%modeled by a simple linear correlation, that only takes into account the filling level in the generator sump and assumes a constant pressure loss coefficient $\r{ K_{SEV}} $,
\begin{equation}
	\label{eq:mdot_PSo}
	\dot{m}_{\r{PSo,A,in}} =\r{ K_{SEV} }m_{\r{PSo,G}}
 \end{equation} 
where $ \r{ K_{SEV}}  $ is determined empirically.
	
	
%	\todo{In the machine serving as basis for the work described in this paper \cite{EAW2017} there is no explicit SEV. Instead, the pressure reduction between high and low pressure level is induced merely by the flow through the SHX. But since the pressure level does not significantly influence the heat transfer in the SHX, both the SHX and the SEV are nevertheless modeled as described above, i.e. as if they were two consecutive components.}
	
	\subsubsection{Refrigerant expansion valve (REV)}
	\label{subsubsec:REV}
	
	The mass flow rate through the REV $ \dot{m}_{\r{Ref,E,in}} $ can basically be modeled in the same way as described for the SEV in  \autoref{eq:mdot_PSo}. However, as discussed in \autoref{subsec:abs_eva},
	 measurement data of the AHPD serving as basis for the work described in this paper indicate that the level in the condenser sump hardly ever changes. Therefore $ \dot{m}_{\r{Ref,E,in}} $ and correspondingly also  $ \dot{m}_{\r{Ref,C,out}} $ is simply set equal to  $ \dot{m}_{\r{Ref,HX,C,out}}$. 
	
	As described above, also the expansion through the REV is modeled as isenthalpic process. And since the refrigerant entering the REV is liquid and saturated or at least very close to saturation, the expansion through the REV causes part of the refrigerant to evaporate, so that the stream entering the evaporator consists of a vaporous part $ \dot{m}^\r{v}_{\r{Ref,E,in}} $ and a liquid part $ \dot{m}^\r{l}_{\r{Ref,E,in}} $, which are determined by means of steady-state mass and energy balances: 
	\begin{equation}
		\label{eq:m_bal_REV}
		0= \dot{m}^\r{v}_{\r{Ref,E,in}} 
		+\dot{m}^\r{l}_{\r{Ref,E,in}} -	\dot{m}_{\r{Ref,E,in}}
	 \end{equation} 
	%\begin{equation}
	%	0= \dot{m}^\r{v}_{\r{Ref,E,in}} 
	%	h^\r{v}_{\r{Ref,sat,GRl}}+\dot{m}^\r{l}_{\r{Ref,E,in}} 
	%	h^\r{l}_{\r{Ref,sat,GRl}}-
	%	\dot{m}_{\r{Ref,C,out}}\frac{E_{\r{Ref,C}}}{m_{\r{Ref,C}}}
	% \end{equation} 
	\begin{equation}
			\label{eq:E_bal_REV}
		0= [\dot{m}h]^\r{v}_{\r{Ref,E,in}} + [\dot{m}h]^\r{l}_{\r{Ref,E,in}} -
		\dot{m}_{\r{Ref,E,in}}\frac{H_{\r{Ref,C}}}{m_{\r{Ref,C}}}
	 \end{equation} 
	
	where $h^\r{v}_{\r{Ref,E,in}}$ and $h^\r{l}_{\r{Ref,E,in}}$ are the specific enthalpies of vaporous and liquid, saturated refrigerant.  The correlation between specific enthalpies and temperatures are again expressed by the corresponding  property  functions \autoref{eq:h_v_H2O} and \autoref{eq:h_l_H2O}. 
%	\begin{equation}
%		h^\r{v}_{\r{Ref,E,in}}=\r{D_1} + \r{D_2}  T_{\r{Ref,E,in}}
%	 \end{equation} 
%	\begin{equation}
%		h^\r{l}_{\r{Ref,E,in}}=\r{C_1} + \r{C_2}  T_{\r{Ref,E,in}}
%	 \end{equation} 
	and the temperature of refrigerant $T_{\r{Ref,E,in}}$ corresponds to the saturation temperature at low pressure, which is expressed by the property function for saturation pressure of water,  \autoref{eq:p_sat_H2O}.
%	\begin{equation}
%		\label{eq:p_sat_H2O}
%		\r{	ln}(p_{\r{sat,H2O}})=-\r{E_{1}}+ \r{E_{2}} T_{\r{Ref,E,in}}.
%	 \end{equation} 
		 
		 Note that since the predominant part of $ \dot{m}_{\r{Ref,E,in}}  $ (in general $> \SI{95}{\%}  $) flows directly into the evaporator sump, the assumption above to set  $ \dot{m}_{\r{Ref,C,out}} $ equal to  $ \dot{m}_{\r{Ref,HX,C,out}}$, i.e., to set $\frac{\r{d}m_{\r{Ref,C}}}{\r{d}t}=  0$, is always legit for AHPDs of this construction type  due to the following reason:  As already discussed in \autoref{subsec:rel_storage}, the condenser and evaporator sumps can be interpreted as two first-order delay elements in series with only the REV in between, which does not interact with any HXs or other components though. Even if the assumption $\frac{\r{d}m_{\r{Ref,C}}}{\r{d}t}= 0$ didn't apply, this would only mean that the individual time constants of the two first-order delay elements would be slightly wrong, but  the effect on their combined dynamics, and hence on the overall AHPD model, would be negligible. The same applies to AHPDs where $ \dot{m}_{\r{Ref,E,in}}  $ does not flow directly into the sump but first flows over the HX surface. In this case it would still account for a significantly smaller share compared to the recirculated $ \dot{m}_{\r{Ref,rec}} $ (in the investigated AHPD $ < \SI{5}{\%} $) and would therefore also have a negligible effect on the heat transfer at the evaporator HX.
%		  even for AHPD where the level in the condenser does change  it is legit to set  $ \dot{m}_{\r{Ref,C,out}} $ equal to  $ \dot{m}_{\r{Ref,HX,C,out}}$ due to the following reason: The predominant part of $ \dot{m}_{\r{Ref,E,in}}  $ (in general $> \SI{95}{\%}  $) flows directly into the evaporator sump. Therefore, these two sumps can be interpreted as two lag delay elements in series, 
%	
	\subsection{Solution and refrigerant pump}
	\label{subsec:pumps}
	In this paper, the pump work is neglected which is a common and valid simplification \cite{Ochoa2016, Kohlenbach2008, Marc2015} for $ \r{LiBr/H2O} $ AHPD. Since the volume flow rates, and not the pump speed, are used as input variables for the models in this paper, no correlation describing the relationship between speed, pressure difference and flow rate is necessary. If required, such a correlation can be taken from e.g. \cite{Vinther2015}.
	
	
	
	%
	%\subsection{leftovers}
	%The nonlinear  model consists of individual submodels, that describe the most relevant AHPD components, like HXs, sumps, valves etc. These submodels are based on the first law of thermodynamics,  i.e. mass and energy balances, in combination with a lumped component approach. 
	%Therefore, individual components are not modeled as spatially discretized components but considered as lumped nodes with concentrated properties like temperature, mass fraction etc., which allows for a simple model with few state variables. 
	%Manipulable input variables considered for this model are inlet temperatures and volume flow rates of all three hydraulic circuits, i.e. $T_{\r{W,G,in}}$,  $ \dot{V}_{\r{W,G}} $, $ T_{\r{W,AC,in}}$,  $ \dot{V}_{\r{W,AC}}$, $ T_{\r{W,E,in}}$ and  $ \dot{V}_{\r{W,E}} $, as well as the volume flow rate of rich solution, i.e $ \dot{V}_{\r{RSo}} $ (at the outlet of the SHX). The volume flow rate of recirculated refrigerant in the evaporator $ \dot{V}_{\r{Ref,rec}} $ is also an input variable, but not manipulable since the circulation pump is operated at constant speed.
	%As is common in the field of AHPD modeling, modeling simplifications are applied, which are described in \autoref{subsec:Modeling_simplifications}. Furthermore the simplified substance property equations that are used are described in \autoref{subsec:substance_prop}. The following sections~\autoref{subsec:Fluid_sumps} to~\autoref{subsec:expansion_valves} focus on the detailed description of submodels for the fluid sumps and HXs of main components, the SHX and the expansion valves. 
	%
	%
	%For easier readability the following notation will be used in the next sections: When two variables with the same sub- and superscript are used, they are joined in a square bracket with the common sub- and superscripts, so that e.g. $ \dot{m}^\r{l}_{\r{Ref,HX,E,out}} h^\r{l}_{\r{Ref,HX,E,out}}  $ becomes
	%$ [\dot{m}h]^\r{l}_{\r{Ref,HX,E,out}}  $.
	
	\subsection{Overall AHPD model }
	%\todo{Noch machen!!}
	\label{subsec:nonlin_complete}

The overall AHPD model consists of the previously discussed  \autoref{eq:Qdot_G} to \autoref{eq:E_bal_REV}. 
%The equations for modeling the individual AHPD components, \autoref{eq:Qdot_G} to \autoref{eq:E_bal_REV}, are now merged to form the overall AHPD model. 
%, consisting of \autoref{eq:Qdot_G}	to \autoref{eq:E_bal_REV}. 
Note, that the property functions  \autoref{eq:rho_LiBrH2O} to \autoref{eq:h_l_H2O}, \autoref{eq:p_sat_LiBrH2O}, \autoref{eq:p_sat_H2O} and \autoref{eq:cp_LiBrH2O} occur multiple times throughout the overall model  to describe correlations between different fluid
 properties like temperature, pressure, mass fraction, specific enthalpy, and density for individual fluid streams.  
The resulting model  then consist of first-order, ordinary coupled differential equations describing the system's dynamics, and coupled algebraic equations describing, e.g., heat transfer or fluid property correlations. 
%The strong coupling between the equations kann daran erkannt werden, dass eine Vielzahl der Modellvariablen in mehreren Gleichungen vorkommen. Aus physikalische Sicht lässt sich die starke Kopplung dadurch begründen, dass AHPD ein Kreisprozess ist, bei dem die Ausgangsgörßen eines Subprozesses wiederum die Einangsgrößen eines oder mehrerer anderer Subprozesse sind.
The strong coupling between the equations is reflected in the large number of model variables that occur in equations of different subsystems. From a physical point of view, the strong coupling can be explained by the fact that the process underlying AHPDs is a circular process, in which the output variables of one sub-process are in turn the input variables of one or more other sub-processes.  
Since both differential and algebraic equations are in general nonlinear, the resulting system is  a system of nonlinear differential-algebraic equations. As such it can be very useful as a simulation model to test different control strategies in simulation, but it is too complex for many model-based control approaches. 

For a simpler model structure, the model can be linearized at a reference operating point  (ROP). The ROP consists of values for all model variables and is determined using the proposed nonlinear model, by simulating the steady-state values for a set of input variables, i.e., inlet temperatures and mass flow rates in the three hydraulic circuits and the volume flow rate of rich solution. In a first step, all equations are linearized at the chosen ROP. In the next step, the system of linearized, coupled algebraic equations can  be solved and eliminated by inserting the solution into the system of differential equations, eventually yielding a model in the typical linear state-space representation: 
\begin{equation}
	\begin{split}
\dot{\v{x}}= \v{A} \v{x} + \v{B} \v{u}\\
\v{y}= \v{C} \v{x} + \v{D} \v{u}
	\end{split}
 \end{equation} 
where the state variables $ \v{x} $ are the stored masses and energies, the input variables $ \v{u} $ are the inlet temperatures and mass flow rates in the three hydraulic circuits and the volume flow rate of rich solution, the output variables $ \v{y} $ are the outlet temperatures and heat flow rates in the three hydraulic circuits, and $ \v{A} $, $ \v{B} $, $ \v{C} $ and $ \v{D} $ are the corresponding system matrices whose entries depend on the chosen ROP.  
Details on the linearization process for this AHPD model can be taken from \cite{Zlabinger2020}. 
To investigate the effect of the linearization on the model's accuracy, both the nonlinear and a linearized model version will be experimentally validated in \autoref{sec:validation}. Additionally, the new modeling approach will be compared to alternative modeling approaches as benchmark which will be explained in the next section. 

%	In summary, the overall AHPD models to be compared - V1, V2 and V3 - consist of the following equations: All three models contain {\autoref{eq:Qdot_G}-\ref{eq:dT_G}}, 	\autoref{eq:m_bal_HX_G}-\ref{eq:E_bal_HX_G_W}, 	\autoref{eq:T_Ref_HX_G_out}-\ref{eq:p_high_G}, 	\autoref{eq:p_high_C}, 	\autoref{eq:m_bal_HX_C}-\ref{eq:dT_A}, 	\autoref{eq:m_bal_HX_A}-\ref{eq:h_RSo_HX_A_out},	\autoref{eq:m_bal_HX_E}-\ref{eq:dE_dt_A} and 	\autoref{eq:h_SEV}-\ref{eq:E_bal_REV}, as well as the  property  functions 	\autoref{eq:h_LiBrH2O}-\ref{eq:rho_H2O}, \autoref{eq:p_sat_LiBrH2O} and	\autoref{eq:p_sat_H2O}. Model version V1 additionally contains 	\autoref{eq:UA_G_c}, 	\autoref{eq:Qdot_C_UA}-\ref{eq:UA_C_c}, 	\autoref{eq:UA_A_c},
%	\autoref{eq:Qdot_E_UA}-\ref{eq:dT_E} and 	\autoref{eq:Qdot_SHX}-\ref{eq:E_bal_SHX_PSo}, while model version V2 additionally contains 	\autoref{eq:UA_G_c}, 	\autoref{eq:Qdot_C_UA}-\ref{eq:UA_C_c}, \autoref{eq:UA_A_c},
%	\autoref{eq:Qdot_E_UA}-\ref{eq:dT_E} and \autoref{eq:delta_T_SHX_h}-\ref{eq:T_bal_SHX_PSo}, and model version V3 additionally contains \autoref{eq:UA_G}, \autoref{eq:Qdot_C_epsilon}-\ref{eq:epsilon_C}, \autoref{eq:UA_A}, \autoref{eq:Qdot_E_epsilon}-\ref{eq:epsilon_E} and \autoref{eq:delta_T_SHX_h}-\ref{eq:T_bal_SHX_PSo}.	

%	substance:
%\autoref{eq:h_LiBrH2O}
%\autoref{eq:h_v_H2O}
%\autoref{eq:h_l_H2O}
%\autoref{eq:rho_LiBrH2O}
%\autoref{eq:rho_H2O}
%\autoref{eq:p_sat_LiBrH2O}
%\autoref{eq:p_sat_H2O}
%
%all: 
%\autoref{eq:Qdot_G}
%\autoref{eq:dT_G}
%\autoref{eq:m_bal_HX_G}
%\autoref{eq:LiBr_bal_HX_G}
%\autoref{eq:E_bal_HX_G_Sol}
%\autoref{eq:E_bal_HX_G_W}
%\autoref{eq:T_Ref_HX_G_out}
%\autoref{eq:p_high_G}
%\autoref{eq:p_high_C}
%\autoref{eq:m_bal_HX_C}
%\autoref{eq:E_bal_HX_C_Ref}
%\autoref{eq:E_bal_HX_C_W}
%\autoref{eq:dm_dt_G}
%\autoref{eq:dm_dt_G_LiBr}
%\autoref{eq:dE_dt_G}
%\autoref{eq:dm_dt_C}
%\autoref{eq:dE_dt_C}
%\autoref{eq:Qdot_A}
%\autoref{eq:dT_A}
%\autoref{eq:m_bal_HX_A}
%\autoref{eq:LiBr_bal_HX_A}
%\autoref{eq:E_bal_HX_A_Sol}
%\autoref{eq:E_bal_HX_A_W}
%\autoref{eq:p_low_A}
%\autoref{eq:h_RSo_HX_A_out}
%\autoref{eq:m_bal_HX_E}
%\autoref{eq:E_bal_HX_E_Ref}
%\autoref{eq:E_bal_HX_E_W}
%\autoref{eq:p_low_E}
%\autoref{eq:dm_dt_E}
%\autoref{eq:dE_dt_E}
%\autoref{eq:LiBr_bal}
%\autoref{eq:total_bal}
%\autoref{eq:dE_dt_A}
%
%
%
%
%V1:
%\autoref{eq:UA_G_c}
%\autoref{eq:Qdot_C_UA}
%\autoref{eq:UA_C_c}
%\autoref{eq:dT_C}
%\autoref{eq:UA_A_c}
%\autoref{eq:Qdot_E_UA}
%\autoref{eq:dT_E}
%\autoref{eq:Qdot_SHX}
%\autoref{eq:T_mean_SHX}
%\autoref{eq:E_bal_SHX_RSo}
%\autoref{eq:E_bal_SHX_PSo}
%\autoref{eq:h_SEV}
%\autoref{eq:T_SEV}
%\autoref{eq:mdot_PSo}
%\autoref{eq:m_bal_REV}
%\autoref{eq:E_bal_REV}
%
%V2: 
%\autoref{eq:UA_G_c}
%\autoref{eq:Qdot_C_UA}
%\autoref{eq:UA_C_c}
%\autoref{eq:dT_C}
%\autoref{eq:UA_A_c}
%\autoref{eq:Qdot_E_UA}
%\autoref{eq:dT_E}
%\autoref{eq:delta_T_SHX_h}
%\autoref{eq:delta_T_SHX_l}
%\autoref{eq:cp_LiBrH2O}
%\autoref{eq:T_bal_SHX_RSo}
%\autoref{eq:T_bal_SHX_PSo}
%
%
%V3: 
%\autoref{eq:UA_G}
%\autoref{eq:Qdot_C_epsilon}
%\autoref{eq:epsilon_C}
%\autoref{eq:UA_A}
%\autoref{eq:Qdot_E_epsilon}
%\autoref{eq:epsilon_E}
%\autoref{eq:delta_T_SHX_h}
%\autoref{eq:delta_T_SHX_l}
%\autoref{eq:cp_LiBrH2O}
%\autoref{eq:T_bal_SHX_RSo}
%\autoref{eq:T_bal_SHX_PSo}	


	
	
%	 a linearized model version 
%	To investigate the effect of the linearization on the model's accuracy, in addition to the three models V1, V2 and V3, also a linearized version of one of the models, namely V3, will be investigated in the experimental validation in the next section. Investigation of the linearized version V3 is chosen over V1 or V2 since V3 is more complex - the linearization will hence have the biggest impact on the modeling accuracy of V3. 
	

\section{Benchmark modeling approaches for comparison}
\label{sec:benchmark_models}
Although some approaches from  literature on AHPD models served as the basis for the model developed in this work, it does differ in the way heat transfer is modeled in the HXs, as mentioned earlier in the corresponding sections. 
%Although  modeling approaches from AHPD literature served as a basis for the model developed in this paper, it does differ in the way the heat transfer in the HX is modeled as already mentioned in the corresponding sections. 
In order to compare the new modeling approaches for AHPD  HXs  to  other approaches, two benchmark model versions  are now presented in this section, in each of which either the HXs of the main components and/or the SHX is modeled in a different way compared to the AHPD model from this paper. In the following, first an overview of the model versions to be compared is given, followed by a more detailed description of the two benchmark model versions (\emph{V1} and \emph{V2} in \autoref{tab:model_version}). 
% some aspects differ, especially the modeling of the heat transfer in the heat exchangers. 

%The developed AHPD model from \autoref{subsec:nonlin_complete}  

%The AHPD model from \autoref{subsec:nonlin_complete} contains modeling approaches for the AHPD's HXs that slightly differ from modeling approaches in the literature. Therefore, two variants are now presented in this section, in each of which either the heat exchangers of the main components or the SHX is modeled in a different way compared to the simulation model . 

The base case
%, which will be referred to as version V1, 
represents the AHPD model developed in this paper. It will be referred to as \emph{base-a}, and its linearized version as \emph{base-b}.
%as described in \autoref{subsec:gen_con} to \autoref{subsec:pumps} and summarized in \autoref{subsec:nonlin_complete} .  
In the first benchmark model version, which will be referred to as version \emph{V1}, the HXs in the AHPD's main components generator, condenser, evaporator and absorber, are  modeled by means of constant $ \mathit{UA}$-values instead of variable $ \mathit{UA}$-values and effectiveness $ \epsilon $ as it is the case for the base model, since this is a very common approach in AHPD modeling, e.g. \cite{Jeong1998,Kohlenbach2008, Evola2013, Marc2015, Sabbagh2018}. The SHX is modeled in the same way as for the base model. 
In the second benchmark model version, which will be referred to as version \emph{V2}, also the heat transfer in the SHX is  modeled by means of a constant $ \mathit{UA} $-value  similar to AHPD models in  \cite{Evola2013, Marc2015, Sabbagh2018}  instead of  terminal temperature differences $ \mathit{TTD} $.
The modeling of the remaining AHPD components (sumps, valves, pumps) is the same for all  model versions. The model versions to be compared, including the linearized version of the base model, are summarized in  \autoref{tab:model_version}.


\begin{table}[h!]
	\centering
	%		\setlength{\tabcolsep}{2.5pt} %% default is 6pt
		\caption{Overview over different model versions to be compared}%
	\begin{tabular}{@{}ccc@{}}
		\toprule
		model version & main components' HXs               & SHX                \\ \midrule
		\emph{V1}            & constant $ \mathit{UA}$        & variable $ \mathit{TTD} $\\
		\emph{V2}               & constant $ \mathit{UA}$         & constant $ \mathit{UA}$    \\
				\emph{base-a} (nonlin.)         & variable  $ \mathit{UA}$  or $\epsilon$ & variable $ \mathit{TTD} $ \\ 
		\emph{base-b} (lin.)         & not applicable & not applicable \\ \bottomrule
	\end{tabular}
	\label{tab:model_version}
\end{table}


Specifically, the following equations are now replaced for the two benchmark model versions:
For the modeling of the generator, in versions \emph{V1} and \emph{V2}, \autoref{eq:UA_G} describing the variable $ \mathit{UA}$-value $ \mathit{UA}_\r{G} $ in dependence of the mass flow rates at the HX is replaced by a constant $\mathit{UA}_\r{G,const} $. 
% as in,  e.g.,   \cite{Kohlenbach2008, Evola2013, Marc2015}. 
 Similarly, for the modeling of the absorber, \autoref{eq:UA_A} describing the variable $\mathit{UA}_\r{A} $ is also replaced by a constant $\mathit{UA}_\r{A,const} $.
For the  modeling of the condenser \autoref{eq:Qdot_C_epsilon} and \autoref{eq:epsilon_C} describing the transferred heat flow $ \dot{Q}_\r{C} $ based on the variable  effectiveness $ 	\epsilon_\r{C} $, is replaced by
\begin{equation}
	\label{eq:Qdot_C_UA}
	\dot{Q}_\r{C}=\mathit{UA}_\r{C.const} \overline{\Delta T}_{\r{HX,C}}
 \end{equation} 
with the constant $ \mathit{UA}$-value $UA_\r{C,const}$ and  the   arithmetic mean temperature difference $\overline{\Delta T}_{\r{HX,C}}$
\begin{equation}
	\label{eq:dT_C}
	\overline{\Delta T}_{\r{HX,C}}=T_{\r{Ref,HX,C,out}}-\frac{T_{\r{W,C,in}}+T_{\r{W,C,out}}}{2}
 \end{equation} 
consisting of  the in- and outlet temperatures  of the cooling water $ T_{\r{W,C,in}} $ and $ T_{\r{W,C,out}} $ and the condensation temperature of refrigerant, i.e., the temperature of the refrigerant leaving the HX, $ T_{\r{Ref,HX,C,out}}  $. 
And finally for the modeling of the evaporator, \autoref{eq:Qdot_E_epsilon} and \autoref{eq:epsilon_E} describing the transferred heat flow $ \dot{Q}_\r{E} $ based on the variable  effectiveness $ 	\epsilon_\r{E} $, is replaced by
\begin{equation}
	\label{eq:Qdot_E_UA}
	\dot{Q}_\r{E}=\mathit{UA}_\r{E,const} \overline{\Delta T}_{\r{HX,E}}
 \end{equation} 
with the constant $ \mathit{UA}$-value $UA_\r{E,const}$ and  the  arithmetic mean temperature difference $\overline{\Delta T}_{\r{HX,E}}$
\begin{equation}
	\label{eq:dT_E}
	\overline{\Delta T}_{\r{HX,E}}= \frac{T_{\r{W,E,in}}+T_{\r{W,E,out}}}{2}-\frac{T_{\r{Ref,rec}}+T_{\r{Ref,HX,E,out}}}{2}
 \end{equation} 

In version \emph{V2}, also the modeling of the SHX is adapted. Instead of modeling the SHX by means of the terminal temperature difference approach, it is also modeled by means of a constant $ \mathit{UA}$-value $ \mathit{UA}_{\r{SHX,const}} $. Moreover, the energy storage capability of the fluid in the SHX is neglected, which are both common approaches in the literature, e.g. \cite{Evola2013, Marc2015, Sabbagh2018}.  
\autoref{eq:delta_T_SHX_h} to \autoref{eq:T_bal_SHX_PSo} are therefore replaced by 
\begin{equation}
	\label{eq:Qdot_SHX}
	\dot{Q}_\r{SHX}=UA_\r{SHX} \overline{\Delta T}_{\r{SHX}}
 \end{equation} 
with the mean HX temperature difference $\overline{\Delta T}_{\r{SHX}}$
\begin{equation}
	\label{eq:T_mean_SHX}
	\begin{split}
	\overline{\Delta T}_{\r{SHX}}=& \frac{T_{\r{PSo,SHX,in}}+T_{\r{PSo,SHX,out}}}{2}-\\
	&\frac{T_{\r{RSo,SHX,in}}+T_{\r{RSo,SHX,out}}}{2}
	\end{split}
 \end{equation} 
with the in- and outlet temperatures of the  poor solution $ T_{\r{PSo,SHX,in}}$ and $ T_{\r{PSo,SHX,out}}$ and of the rich solution $ T_{\r{RSo,SHX,in}}$ and $ T_{\r{RSo,SHX,out}}$. 
%	Outlet temperatures can then be determined by the energy balances on the rich and poor solution side of the SHX, \autoref{eq:E_bal_SHX_RSo} and \autoref{eq:E_bal_SHX_PSo},  in combination with the substance property data function for specific enthalpy of $ \r{LiBr/H_2O} $, \autoref{eq:h_LiBrH2O}.

The steady-state energy balances on either side (rich and poor solution) are 
\begin{equation}
	\label{eq:E_bal_SHX_RSo}
	0=\dot{m}_{\r{RSo,SHX,in}} (h_{\r{RSo,SHX,in}}-h_{\r{RSo,SHX,out}})+\dot{Q}_\r{SHX}
 \end{equation} 
\begin{equation}
	\label{eq:E_bal_SHX_PSo}
	0=\dot{m}_{\r{PSo, SHX, in}} (h_{\r{PSo,SHX,in}}-h_{\r{PSo,SHX,out}})-\dot{Q}_\r{SHX}
 \end{equation} 
where the correlation between enthalpies, temperatures and mass fractions is again described by \autoref{eq:h_LiBrH2O}. 

These two model versions \emph{V1} and \emph{V2} will be used as benchmark for the new modeling approach in the course of the experimental validation in the next section. 



	\section{Experimental validation}
	\label{sec:validation}
	
	For  validation, the new model from \autoref{sec:Sim_model} (nonlinear model \emph{base-a} in \autoref{tab:model_version}), the linearized version thereof based on the linearization procedure described in \cite{Zlabinger2020} (\emph{base-b}), and the two benchmark model versions (\emph{V1} and \emph{V2})  are implemented in MATLAB Simulink\textsuperscript{\textregistered} \cite{MathWorks2020}. Simulation results are compared to each other and to 
	measurement data from the AHPD described in \autoref{sec:process_description} to illustrate the  degree of accuracy that can be expected from these models. This will first be done for steady-state results in \autoref{subsec:static_val} and then for transient operation  in \autoref{subsec:dynamic_val}. Note that for the visualization of the validation results in this section, variables will often not be given in SI-units like \si{\kelvin} and \si{\m \cubed \per \second}, but rather in more manageable units like \si{\celsius} and \si{\m \cubed \per \hour}.
	%To compare both steady-state and dynamic accuracy of the model versions \emph{V1} to V3
	%In this section, experimental validation results for variations of all seven manipulable input variables are shown for the nonlinear model from  \autoref{sec:Sim_model} and the linear model from  \autoref{sec:controller_model}.  To illustrate the degree of accuracy that can be expected for these two types of model outlet temperatures in hot, cool and cold water circuit are compared under steady-state (\autoref{subsec:static_val}) and transient (\autoref{subsec:dynamic_val}) conditions.
	%For the simulation model a high steady-state and dynamic accuracy are both desirable in order to obtain a realistic model of the AHPD. The focus for the controller design model, however, is primarily high dynamic accuracy, while steady- state deviations are less important, since they can be compensated relatively easy by adding integral action to the model-based controller to be designed.
	% Although steady-state accuracy is definitely advantageous, it is less important than 
	
	The ROP for the linearized model used for this validation is listed in \autoref{tab:ROP}. This is a reasonable operating point but it is deliberately chosen to not be exactly in the middle of the investigated operating range to highlight the effect of the linearization. 
%	 and is chosen close to a standard operating point of the investigated AHPD, taken from the corresponding data sheet \cite{EAW2017}. The input variable range investigated for validation, however, is deliberately chosen to be much wider, so that the system's behavior well outside this standard operating point can be investigated too. 
	 The constant $\mathit{UA} $-values for model version \emph{V1} and \emph{V2} are also determined based on measurement data from this operating point (see \autoref{sec:param_empirical}).
	
	
%	\begin{table}[h!]
%		\setlength{\tabcolsep}{5pt} %% default is 6pt
%		\centering
%		\begin{tabular}{ccccccc}
%			\toprule
%			$ T_{\r{W,G,in}}$ &  $ \dot{V}_{\r{W,G}} $ & $ T_{\r{W,AC,in}}$ & $ \dot{V}_{\r{W,AC}}$ & $ T_{\r{W,E,in}}$ & $ \dot{V}_{\r{W,E}} $ & $ \dot{V}_{\r{RSo}} $ \\
%			\midrule
%			\si{\celsius}	& \si{\meter\cubed \per \hour}  & \si{\celsius} &\si{\meter\cubed \per \hour} &\si{\celsius} &\si{\meter\cubed \per \hour}  &\si{\litre\per \hour} \\
%			\midrule
%			80	&1.2 &  29& 6.2 & 14 & 2.2 & 450 \\
%			\bottomrule
%		\end{tabular}
%		\caption{Reference operating point (ROP) for experimental validation}%
%		\label{tab:ROP}
%	\end{table}
%	
	\begin{table}[h!]
		\setlength{\tabcolsep}{5pt} %% default is 6pt
		\centering
				\caption{Reference operating point (ROP) for experimental validation}%
		\begin{tabular}{ccccccc}
			\toprule
			$ T_{\r{W,G,in}}$ &  $ \dot{m}_{\r{W,G}} $ & $ T_{\r{W,AC,in}}$ & $ \dot{m}_{\r{W,AC}}$ & $ T_{\r{W,E,in}}$ & $ \dot{m}_{\r{W,E}} $ & $ \dot{V}_{\r{RSo}} $ \\
			\midrule
			\si{\celsius}	& \si{\kg \per \hour}  & \si{\celsius} &\si{\kg  \per \hour} &\si{\celsius} &\si{\kg  \per \hour}  &\si{\litre\per \hour} \\
			\midrule
			80	&1200 &  29& 6200 & 14 & 2200 & 450 \\
			\bottomrule
		\end{tabular}
		\label{tab:ROP}
	\end{table}
	
	%?? Offene Fragen:
	%\begin{itemize}
	%	\item Wärmeverlust?
	%	\item Totzeiten?
	%	\item Messgeräte?
	%\end{itemize}
	
	\subsection{Test bench setup}
	\label{subsec:test bench}
	
	The AHPD described in \autoref{sec:process_description} is installed at a test bench that allows individual adjustment of all seven  input variables. Flow rates are adjusted by means of speed controlled pumps and measured by means of magnetic-inductive flow meters \cite{EndressHauser}, where the sensors for the flow rates in the hydraulic circuits are positioned at the AHPD outlet and the sensor for the rich solution between SHX and generator.
	 Inlet temperatures are adjusted by means of three-way-valves and in- and outlet temperatures are measured by means of Pt100 elements \cite{PMR2021}.
	As discussed in \autoref{subsec:rel_storage} the position of these temperature sensors  influences the throughput time between in- and outlet temperature sensor. Since this is due to the AHPD's hydraulic integration rather than the AHPD construction itself, this effect is not considered in the AHPD model. In order to still consider this effect for the validation, the measurement data must be adjusted accordingly. 
%	these sensors for in- and outlet temperatures are not mounted directly at the machine, but at a certain distance from it due to constructural reasons.
%	, which inevitably results in a volume-flow-dependent dead time between temperature measurement and  inlet (respectively outlet) temperature. Therefore,
	% In order to seperate this peculiarity of the test bench from the investigation of the dynamic behavior of the actual AHPD, 
%	measurement data of in- and outlet temperatures are adjusted so that this dead time effect is considered.
%		In addition, the flow through the HX also adds small dead times.  
		For that, the measurement signals of the in- and outlet temperatures,  $ T_{\r{W,i,in,meas}} $ and $ T_{\r{W,i,out,meas}} $, are shifted by the respective dead times $ \Delta t_{\r{d,i,in}}  $ and $ \Delta t_{\r{d,i,out}} $ by
	\begin{equation}
		\label{eq:deadtime_in}
		T_{\r{W,i,in}}(t)=T_{\r{W,i,in,meas}}(t-\Delta t_{\r{d,i,in}} )
	 \end{equation} 
	\begin{equation}
				\label{eq:deadtime_out}
		T_{\r{W,i,out}}(t)=T_{\r{W,i,out,meas}}(t+\Delta t_{\r{d,i,out}} )
	 \end{equation} 
	
	where the index i denotes the subscripts G, AC and E for the hot, cooling and chilled water circuit respectively. The individual dead times are determined based on the corresponding volume flow rates and the fluid volume inside the HXs and the pipe segments between the sensors. 
	%Heat losses that occur along these pipe segments are neglected for simplicity.
	
	
	\subsection{Steady-state validation}
	\label{subsec:static_val}
	
	For the evaluation of the steady-state accuracy of the four models, all seven  input variables are varied individually and simulation results are compared to steady-state measurement data. 
	\begin{figure*}
		\centering
		\includegraphics[width=2.6 \columnwidth, angle=90]{static_validation_4.pdf}%
		\caption{Comparison of measured and simulated output variables in steady-state }%
		\label{fig:stat_val}%
	\end{figure*}
	%for model-based control on component level steady-state accuracy is 
	% In this section steady-state measurement data are compared to steady-state simulation results from the nonlinear and the linear model.
	%For this purpose, input variables were held constant until the output variables stopped changing and reached their steady-state values. 
	\autoref{fig:stat_val} shows the effect of the seven  input variables $ T_{\r{W,G,in}}$,  $ \dot{V}_{\r{W,G}} $, $ T_{\r{W,AC,in}}$,  $ \dot{V}_{\r{W,AC}}$, $ T_{\r{W,E,in}}$,  $ \dot{V}_{\r{W,E}} $ and $ \dot{V}_{\r{RSo}} $, which were each varied over their entire operating range, on the three heat flow rates $ \dot{Q}_{\r{G}} $, $ \dot{Q}_{\r{AC}} $ and $ \dot{Q}_{\r{E}} $  (Subfig. 1a-7c) and the corresponding outlet temperatures $ T_{\r{W,G,out}} $, $ T_{\r{W,AC,out}} $  and  $ T_{\r{W,E,out}}$ (Subfig. 1d-7f), which are chosen as output variables. 
%	The cooling coefficient of performance for these operating points ranges from 0,4 to 0,79. 
%	and the corresponding three heat flows $ \dot{Q}_{\r{E}} $, $ \dot{Q}_{\r{AC}} $ and $ \dot{Q}_{\r{G}} $  (Subfig. 1d-7f), which are chosen as output variables. 
	%  added to support easier understanding of correlations depicted in Subfig. 1d-7f.
	The figure is arranged in such a way that each subfigure shows the correlation between one input and one output variable, when the remaining input variables are kept constant at the chosen ROP from  \autoref{tab:ROP}. Input variables are plotted on the x-axes, where the individual ROP values are marked with vertical gray dashed lines. Output variables are plotted on the y-axes. Simulation results from model version \emph{V1} are plotted in dashed-dotted green, from  \emph{V2} in solid yellow, from \emph{base-a} in dashed bright blue, from the linearized version \emph{base-b} in dotted dark blue and measurement data with gray markers. The following reading example is intended to facilitate the interpretation of the figure: For the marked opearting point (red in \autoref{fig:stat_val}) $ T_{\r{W,G,in}} = \SI{90}{\celsius}$, while the remaining six input variables have the same values as listed in \autoref{tab:ROP}. The AHPD's measured ouput variables for this operating point are 	
	\mbox{$ \dot{Q}_{\r{G}}= \SI{21.3}{ \kilo \watt} $}, 
		\mbox{$ \dot{Q}_{\r{AC}}= \SI{34.9}{ \kilo \watt} $},  
		\mbox{$ \dot{Q}_{\r{E}}= \SI{14.5}{ \kilo \watt} $}, 
		\mbox{$ T_{\r{W,G,out}} = \SI{74.4}{\celsius}$}, 
		\mbox{$ T_{\r{W,AC,out}}= \SI{33.9}{\celsius} $}  and  
		\mbox{$ T_{\r{W,E,out}}= \SI{8.4}{\celsius}$} (Subfig. 1a-1f). 
	
	
	
	% and the linear model are plotted in dotted dark blue and dashed bright blue, while measurement results are indicated with green markers. 
	%Subfigure XXX, e.g., shows the effect of the hot water inlet temperature $ T_{\r{W,G,in}} $ on the chilled water outlet temperature  $ T_{\r{W,E,out}} $, when the remaining six input variabels have the values from table~\autoref{tab:ROP}.
	%Although steady-state accuracy is less important for models that should be used in model-based control as it si for e.g. simulation models, it should still be briefly discussed. 
	
	For variations of inlet temperatures (Subplots 1a-1f, 3a-3f and 5a-5f) it can be seen that the effects on all output variables follow a linear correlation and that the four models yield very similar simulation results, which correspond sufficiently well with measurement data.
	Correlations between volume flow rates in the hydraulic circuits and output variables  (Subfig. 2a-2f, 4a-4f, 6a-6f) are slightly nonlinear though. Accordingly, the nonlinear model versions correspond better with measurement data than the linearized model version, whose accuracy becomes worse for operating points further away from the ROP. 
	% In Subfig. 2d,  it can be seen that this can even result in physically infeasible outlet temperatures for operating points far away from the ROP: For $ \dot{V}_{\r{W,G}} > \SI{2,8}{\cubic \meter \per \hour}$ the modeled outlet temperature of the linearized model lies above the inlet temperature. Although this phenomenon at first seems like a strong contraindication against the linearized model, one should note that the ROP can always be chosen to best fit the considered operating range and that any model-based control strategy the model will be used for will most likely also contain integral action. For an application, where the complete $ \dot{V}_{\r{W,G}}$ range depicted in  \autoref{fig:stat_val}  will be used, it would therefore be better to choose an ROP in the middle of this specific operating range instead. 
	% Then, the difference between the linear model and the real machine will be less pronounced at high flow rates.  
	The correlations between the rich solution flow rate and output variables  (Subfig. 7a-7f) are only weakly nonlinear - the nonlinear models and the linearized one give similar results here with satisfactory accuracy. Small deviations are noticeable though. 
%	  The modeling accuracy for variations of the rich solution flow rate is satisfactory for all models, but small deviations are noticeable.
	
	
	Interestingly, the model with constant $ \mathit{UA} $-values for all HXs (model version \emph{V2}) gives only slightly worse results than the model with variable heat transfer correlations (\emph{base-a}) when looking at the results in  \autoref{fig:stat_val}. If, however, operating points are considered where more than one input variable is varied from the ROP, the difference between the individual model versions becomes more apparent and \emph{base-a} yields more accurate steady-state results than \emph{V1} and \emph{V2}, which is depicted in \autoref{fig:stat_val_2}. On the y axis the relative absolute error of heat flows, $RAE_{\dot{Q}}$, is depicted which is calculated by
	\begin{equation}
		RAE_{\dot{Q}}=\frac{1}{3} \left(
		\left|\frac{\Delta\dot{Q}_{\r{G}}}{\dot{Q}_\r{G,meas}} \right|+
		\left|\frac{\Delta\dot{Q}_\r{AC}}{\dot{Q}_\r{AC,meas}} \right|+
		\left|\frac{\Delta\dot{Q}_\r{E}}{\dot{Q}_\r{E,meas}} \right| \right)
	 \end{equation} 
	where $ \Delta\dot{Q}  $ corresponds to the difference between simulated and measured heat flows and $\dot{Q}_\r{meas} $ to the measured heat flows. The diagram shows the $RAE_{\dot{Q}}$ for the three nonlinear model versions for four individual operating points, where all volume flows are varied simultaneously while the remaining input variables (inlet temperatures) are  kept constant at the ROP from \autoref{tab:ROP}. The first operating point at the left end of the scale corresponds to the volume flow rates of the ROP and the last operating point at the right end of the scale to the maximum values of the volume flow rates. It can be seen, that the $RAE_{\dot{Q}}$ of version \emph{base-a} is app. \SI{5}{\%} for all operating points but reaches app. \SI{16}{\%} and \SI{17}{\%} for \emph{V1} and \emph{V2} for the last operating point.
	The difference becomes even more apparent when extreme operating conditions are considered where very high heat flows occur. However, it should be considered that the ROP and therefore also the constant $ \mathit{UA}$-values for model versions \emph{V1} and \emph{V2} would be chosen in the middle of the expected operating range when using the model for model-based control of a real application. Therefore, also the deviations would be smaller than depicted here. Nevertheless, the base model  is able to reflect the plant behavior better in a larger operating range than model versions \emph{V1} and \emph{V2}.
		\begin{figure}[h!]
		\centering
		\includegraphics[width=0.85 \columnwidth]{RAE_Qdot_V2.pdf}%
		\caption{Relative absolute error of heat flows ($RAE_{\dot{Q}}$) of four exemplary operating points with simultaneous variation of $\dot{V}_{\r{W,G}}$, $\dot{V}_{\r{W,AC}}$, $\dot{V}_{\r{W,E}}$ and $\dot{V}_{\r{RSo}}$ }%
		\label{fig:stat_val_2}%
	\end{figure}
	%The validation results in  \autoref{fig:stat_val}  can be interpreted as follows:  
	%\begin{itemize}
	%		\setlength\itemsep{-0.5em}
	%	\item The overall agreement between nonlinear model and measurement data is satisfactory, although modeling accuracy for the chilled water outlet temperature (Subfig. 1f-6f) is slightly worse than for hot and cool water outlet temperatures (Subfig. 1d-6e). 
	%	\item The correlations between input and output temperatures each appear to be of linear nature. Accordingly, the nonlinear and the linear model provide similar results here (Subplots 1d-1f, 3d-3f and 5d-5f).
	%	\item The correlations between volume flow rates in the hydraulic circuits and the outlet temperatures are noticeably nonlinear. Accordingly, it can be clearly seen that the linear and nonlinear model agree well near the ROP but diverge for operating points further away from the ROP (Subfig. 2d-2f, 4d-4f, 6d-6f). 
	%	\item  In Subfig. 2d,  it can be seen that this can even result in physically infeasible outlet temperatures for operating points far away from the ROP: For $ \dot{V}_{\r{W,G}} > \SI{3}{\cubic \meter \per \hour}$ the modeled outlet temperature lies above the inlet temperature. Although this phenomenon at first seems like a strong contraindication against the linear model, one should note that the ROP can always be chosen to best fit the considered operating range. For an application, where the complete $ \dot{V}_{\r{W,G}}$ range depicted in  \autoref{fig:stat_val}  will be used, it would therefore be better to choose an ROP in the middle of this specific operating range instead. Then, the divergence between the linear model and the real machine will be less pronounced at high flow rates. 
	%	\item The correlations between the rich solution flow rate and outlet temperatures are only weakly nonlinear. Accordingly, the nonlinear and linear models give similar results here. Modeling accuracy of both models is satisfactory, but deviations are noticeable in the chilled water circuit for high solution flow rates and in the hot water circuit for low solution flow rates. 
	%\end{itemize}
	
	
	%These correlations also apply to the corresponding heat flows (Subfig. 1a-7c). 
	In summary, the steady-state accuracy of the nonlinear models exceeds the one of the linear model for  variations of $ \dot{V}_{\r{W,G}}$, $ \dot{V}_{\r{W,AC}}$ and $ \dot{V}_{\r{W,E}}$ and is equally good for variations of $ T_{\r{W,G,in}}$, $ T_{\r{W,AC,in}}$,  $ T_{\r{W,E,in}}$ and $ \dot{V}_{\r{RSo}} $. Among the three nonlinear model versions, the base model  yields the best results when considering the entire operating range, followed by \emph{V1} and \emph{V2}, indicating that variable heat transfer correlations yield better results than constant $ \mathit{UA} $-values. Nevertheless, all four models are considered suitable to be used for the design of advanced, model-based control strategies in terms of their steady-state accuracy.  Deviations are considered sufficiently small to be compensated by adding integral action to the model-based controller to be designed.
	%Both models are evaluated as acceptable, since also the deviations between the linear model and the (nonlinear) real AHPD are considered sufficiently small to be easily compensated by adding integral action to the model-based controller to be designed.
	%However, the steady-state deviations between the linear model and the (nonlinear) real AHPD are still considered sufficiently small and can easily be compensated by adding integral action to the model-based controller to be designed.
	%Nevertheless, it is of course reasonable to select the ROP in such a way that the values for hot, cool and chilled water flow rates lie in the middle of the planned operating range.
	%, since the linear model can only approximate the nonlinear steady-state effects of these three input variables. 
	%In case that linearization in one ROP might not be sufficient though, the platform-independence of the linear model would also allow for online linearization at the current operating point. 
	
	
	
	
	\subsection{Dynamic validation}
	\label{subsec:dynamic_val}
	
	In this section measurement data from transient operating conditions are compared to corresponding simulation results from the four models. For this, three exemplary test runs are discussed, in which one input variable at a time is changed in a step-like manner, while the other  input variables are again kept constant at the selected ROP from \autoref{tab:ROP}. 
		The three test runs were chosen deliberately since they reflect the wide variety of dynamic responses to different input variable
	variations.
	%. To illustrate the different time constants resulting from steps in different input variables, three exemplary step response test runs are discussed - 
	%first with a step in the hot water inlet temperature $ T_{\r{W,G,in}} $, second with a step in the chilled water volume flow rate $ \dot{V}_{\r{W,E}} $ and third with a step in the rich solution volume flow rate $ \dot{V}_{\r{RSo}} $.
	
	
	\subsubsection{Step in the hot water inlet temperature $ T_{\r{W,G,in}} $}
	\label{subsubsec:T_W_G_in}
	
	\autoref{fig:step_T_W_G_in} shows a step-like change in the hot water inlet temperature  $ T_{\r{W,G,in}} $ and the corresponding step response for the three outlet temperatures over time. The upper left plot shows the measured volume flow rates of hot, cooling and chilled water $ \dot{V}_{\r{W,G}} $, $ \dot{V}_{\r{W,AC}} $ and $ \dot{V}_{\r{W,E}} $. The lower left plot shows the measured volume flow rate of the rich solution $ \dot{V}_{\r{RSo}} $. The plots on the right side show in- and outlet temperatures in the hot, cooling, and chilled water circuit. Outlet temperatures are again plotted in solid  gray for measured values, in dashed-dotted green  for simulated values with version \emph{V1}, in solid yellow for  \emph{V2}, in dashed bright blue for \emph{base-a} and in dotted dark blue for the linearized version \emph{base-b}. Note that the rectangular fluctuations in the cooling water inlet temperature are linked to the fact that the corresponding three-way-valve at the test bench only allows discrete valve positions.  
	
	
	\begin{figure*}
		\centering
		\includegraphics[width=2.1\columnwidth, angle=0]{T_W_G_step_4.pdf}%
		\caption{Measured and simulated step response for step in hot water inlet temperature $ T_{\r{W,G,in}} $ }%
		\label{fig:step_T_W_G_in}%
	\end{figure*}
	
	At $ t=\SI{0}{\minute} $ the hot water inlet temperature $ T_{\r{W,G,in}} $  is increased from $ \SI{70}{\celsius} $ to $ \SI{80}{\celsius} $. As a result the hot water  outlet temperature  $ T_{\r{W,G,out}} $ increases too. At first, the outlet temperature increases as fast as the inlet temperature but after app. \SI{30}{\s} the slope gets flatter and the new steady-state value is reached after app. \SIrange{20}{25}{\min}.
	All four models can reproduce this step response from $ T_{\r{W,G,in}} $ to $ T_{\r{W,G,out}} $  sufficiently well. 
			Due to cross-coupling effects within the AHPD also the outlet temperatures of the other two hydraulic circuits change after the step. 
	The cooling water outlet temperature $ T_{\r{W,AC,out}} $ increases sharply and quickly reaches its new steady-state value within the first minute. All four models show similar results and agree well with measurement data, although the simulated temperatures do increase slightly faster than the measured one. 
	%\todo{liegt am Volumenstrom der armen Lösung!!} 
		The chilled water outlet temperature $ T_{\r{W,E,out}} $ starts decreasing app. \SI{45}{\s}  after the step and reaches steady-state after app. \SI{30}{\min}, which is also reproduced well by all four models.	The steady-state difference between simulated and measured values of $ T_{\r{W,E,out}} $ is considered acceptable for the use in model-based control, as already discussed in  \autoref{subsec:static_val}. 
%	 This delayed decrease is actually the effect of a non-minimum phase dynamic which is more pronounced in the measured outlet temperatures but is also represented well enough by all models. 
	%The delayed reaction of  $ T_{\r{W,E,out}} $ can be explained by the locations of evaporator and generator. The evaporator is located, so to speak, at the opposite end of the AHPD as the generator. This means, that a variation of an input variable that directly acts upon the generator (like $ T_{\r{W,G,in}} $) has to travel through the whole AHPD until its effect shows in the evaporator. Furthermore, the step in $ T_{\r{W,G,in}} $ results in a non-minimum phase response in  $ T_{\r{W,E,out}} $ which can be seen in the measured values in  \autoref{fig:step_T_W_G_in}. The models can represent this behavior too but less pronounced so that it is not clearly visible in the plot. 
	%Overall, all models reproduce the transient behavior of  $ T_{\r{W,E,out}} $ well. 

	
	
	
	
	
	\subsubsection{Step in the chilled water volume flow rate $ \dot{V}_{\r{W,E}} $}
	\label{subsubsec:Vdot_W_E}
	\autoref{fig:Vdot_W_E} shows a test run similar to the one described before. Here, the chilled water volume flow rate  $ \dot{V}_{\r{W,E}} $ is changed in a step-like manner from $ \SI{4,5}{\meter \cubed \per \hour} $ to $   \SI{1,5}{\meter \cubed \per \hour}$. 
	After the step $ T_{\r{W,E,out}} $ first decreases sharply and then slowly settles at its new steady-state value after app. \SI{25}{\min}. 
	%All models represent the step response from $ \dot{V}_{\r{W,E}} $ to $ T_{\r{W,E,out}} $ well. 
	 $ T_{\r{W,AC,out}} $ also decreases sharply within the first minute and changes only slightly afterwards. 
	%All models give satisfactory simulation results. 
	%step response of the cool water outlet temperature $ T_{\r{W,AC,out}} $ 
	%is of similar shape and is simulated accurately by both models. 
	In the hot water circuit, $ T_{\r{W,G,out}} $ first slightly decreases and then increases to  reach its new steady-state value after app.  \SI{30}{\min}. All models describe this transient behavior well. It can also be seen that steady-state deviations are higher for operating points that are further away from the ROP for the linearized model \emph{base-b}, as discussed in  \autoref{subsec:static_val}.
%	the first operating point before the step since $ \dot{V}_{\r{W,E}} $ is further away from the ROP than for the second operating point.
	\begin{figure*}
		\centering
		\includegraphics[width=2.1\columnwidth, angle=0]{Vdot_W_E_step_4.pdf}%
		\caption{Measured and simulated step response for step in chilled water volume flow rate $ \dot{V}_{\r{W,E}} $ }%
		\label{fig:Vdot_W_E}%
	\end{figure*}
	
	 Furthermore, especially in the first part of the test run, before the step, it can be seen that the measured cooling and chilled water outlet temperatures $ T_{\r{W,AC,out}} $ and $ T_{\r{W,E,out}} $ slightly oscillate with a period time of app. \SI{2}{\min}. This is a phenomenon that has been observed in some test runs with this AHPD and is not caused by fluctuating input variables, but by a fluctuating poor solution flow rate, which in return is probably related to rising gas bubbles inside the AHPD. From a large number of test runs, however, it was found that these oscillations never lead to instability or other serious operating problems.
	In the test run shown in  \autoref{fig:Vdot_W_E}, the oscillations were more intense at the first operating point (before the step) and subsided after the step. Since this phenomenon is very complex and its effect on the outlet temperatures are considered to be sufficiently small, it is not considered in any  of the models. 
	
	
	\subsubsection{Step in the rich solution volume flow rate $ \dot{V}_{\r{RSo}} $ }
	
	%\todo{darauf hinweisen, dass das dynamik allein das Problem für \emph{V1} nicht löst}
	
	\autoref{fig:Vdot_RSo_step} shows a test run where the rich solution volume flow rate $ \dot{V}_{\r{RSo}} $ is changed in a step-like manner from $ \SI{185}{\litre \per \hour} $ to $   \SI{450}{\litre \per \hour} $. 
	\begin{figure*}
		\centering
		\includegraphics[width=2.1\columnwidth, angle=0]{Vdot_RSo_step_5.pdf}%
		\caption{Measured and simulated step response for step in rich solution volume flow rate $ \dot{V}_{\r{RSo}} $}%
		\label{fig:Vdot_RSo_step}%
	\end{figure*}
	%After the step, all three outlet temperatures change and the transferred heat flows increase, i.e. the temperature differences between water in- and outlet increase. 
	After the step the measured $ T_{\r{W,G,out}} $ first decreases relatively fast and then rises again slightly to reach the new steady-state value after app. \SI{3}{\min}. 
	Unlike the previously discussed test runs, the models now provide relatively different results for this step: While the model versions \emph{base-a}, \emph{base-b} and \emph{V1}  (with variable $ \mathit{TTD} $ and dynamics in the SHX model) are able to reproduce the main dynamics well, the shape of the step response from \emph{V2} (with constant $ \mathit{UA} $ and no dynamics in the SHX model)  does not fit the measured curve well, since \emph{V2}  gives a temperature change in the wrong direction for $ T_{\r{W,AC,out}} $ and $ T_{\r{W,E,out}} $, which can lead to serious control performance degradation (e.g. oscillations or instability) if such a model is used for dynamic model based control strategies. 
%	and the temperature decrease in $ T_{\r{W,G,out}} $ is too fast. 
	
However, also  \emph{base-a}, \emph{base-b} and \emph{V1}  show a flaw when comparing measured and simulated $ T_{\r{W,AC,out}} $:  After the step, the measured $ T_{\r{W,AC,out}} $ first increases fast, then a pronounced temperature drop can be seen after app. \SI{30}{\s} before it increases again and reaches the new steady-state value after app. \SI{5}{\min}. 
%Now for one thing, it can be seen that, again, simulation results from model version \emph{V1} are worse than those of \emph{V2} and V3 since right after the step, simulation results from \emph{V1} show a temperature change in the wrong direction. 
The models do reproduce the main dynamics well, but not the temperature drop shortly after the step. Taking a closer look at the solution circuit, it becomes clear that this temperature drop is linked to a drop in the poor solution flow rate:  \autoref{fig:Vdot_PSo} shows the measured rich and poor solution flow rate (measured at the entry of the generator and absorber respectively), as well as the simulated poor solution volume flow rate (using  \autoref{eq:rho_LiBrH2O} to calculate $ \dot{V}_{\r{PSo}} $ from the simulated $ \dot{m}_{\r{PSo}} $) during the discussed test run. For the sake of better visualization, only the simulation results of \emph{base-a} are shown since the other models yield virtually identical results for $ \dot{V}_{\r{PSo}} $. 
	It can be seen that the measured $ \dot{V}_{\r{PSo}} $ significantly drops at the same time as $ T_{\r{W,AC,out}} $ drops. During this drop, less poor solution arrives in the absorber, which leads to a decrease of  absorber performance, which in turn is reflected in the drop in  $ T_{\r{W,AC,out}} $. 
	The drop in $ \dot{V}_{\r{PSo}} $ is again assumed to be connected to gas bubbles rising from the absorber to the generator under these operating conditions which is not considered in any of the models. Therefore, 
	the corresponding  drop in $ T_{\r{W,AC,out}} $ cannot be reproduced either. 
	%	The measured $ T_{\r{W,AC,out}} $ first increases fast, then a pronounced temperature drop can be seen after app. \SI{30}{\s} before it increases again and reaches the new steady-state value after app. \SI{5}{\min}. Now for one thing, it can be seen that, again, simulation results from model version \emph{V1} are worse than those of \emph{V2} and V3 since right after the step, simulation results from \emph{V1} show a temperature change in the wrong direction. Model version \emph{V2} and V3 reproduce the main dynamics well, but not the temperature drop shortly after the step. Taking a closer look at the solution circuit, it becomes clear that this temperature drop is linked to a drop in the poor solution flow rate:  \autoref{fig:Vdot_PSo} shows the measured rich and poor solution flow rate, as well as the simulated poor solution volume flow rate (using  \autoref{eq:rho_LiBrH2O} to calculate $ \dot{V}_{\r{PSo}} $ from the simulated $ \dot{m}_{\r{PSo}} $) during the discussed test run. For the sake of better visualization, only the simulation results of V3 are shown since the other models yield virtually identical results for $ \dot{V}_{\r{PSo}} $. 
%	It can be seen that the measured $ \dot{V}_{\r{PSo}} $ significantly drops at the same time as $ T_{\r{W,AC,out}} $ drops. During this drop, less poor solution arrives in the absorber, which leads to a decrease of  absorber performance, which in turn is reflected in the drop in  $ T_{\r{W,AC,out}} $. 
%	The drop in $ \dot{V}_{\r{PSo}} $ is again assumed to be connected to gas bubbles rising from the absorber to the generator under these operating conditions which is not considered in any of the models. Therefore, 
%	the corresponding  drop in $ T_{\r{W,AC,out}} $ cannot be reproduced either. 
	The main dynamics (the time constant of the step responses in \autoref{fig:Vdot_RSo_step}) are still represented well by \emph{base-a}, \emph{base-b} and \emph{V1}  though and such sudden, large changes in the rich solution flow rate are unlikely to occur under realistic operation conditions. Nevertheless, during control design  it can still be useful to consciously consider this deviation 
	between the modeled and the real system 
	and limit the rate of change for the rich solution flow rate if it is used as a manipulated variable. 
	
	%Nevertheless, it makes sense to take this slight deviation between the modeled and the real system into account when designing the controller, so that when $ \dot{V}_{\r{RSo}} $ is used as a manipulated variable its rate of change could  be limited.
	
	%The main dynamics, i.e. the time constant for this step, however, can be described satisfactorily. 
	%In  \autoref{fig:Vdot_PSo} it can also be seen that the simulated and the measured volume flow rates have in principle similar dynamic properties (similar time constant), but the simulated values do not show the distinct drop at minute , which is the reason why the drop cannot be seen in the simulated cooling water outlet temperatures either. Therefore, neither of the two models can represent this complex phenomenon. 
	
	%Nevertheless, it makes sense to take this deviation between the modeled and the real system into account when designing the controller, so that when $  \dot{V}_{\r{RSo}}  $ is used as a manipulated variable its rate of change should be limited. 
	%(e.g. by limiting the allowed manipulation rate% or by imposing a stronger penalty in LQR approaches
	
	\begin{figure}[h!]
		\centering
		\includegraphics[width=0.95\columnwidth, angle=0]{Vdot_PSo.pdf}%
		\caption{Rich and poor solution volume flow rate after step in  $ \dot{V}_{\r{RSo}} $}%
		\label{fig:Vdot_PSo}%
	\end{figure}
%	Finally, the investigation of the step response in the chilled water circuit leads to similar results. Model version V1 again wrongly yields a pronounced non-minimum phase system response, while \emph{V2} and V3  describe the main system dynamics well. Only the intermediate slight increase at $ t=\SI{30}{\second} $ which is also related to the previously discussed drop in $ \dot{V}_{\r{PSo}} $, cannot be represented by any of the models.
	
	It should be mentioned that the authors also investigated a modeling approach with constant $ \mathit{UA} $ in the SHX as in version \emph{V2} in combination with additional first-order delay elements as in version \emph{base-a} and \emph{V1}. Although the dynamic modeling accuracy does improve compared to \emph{V2}, the basic problem of first yielding temperature changes in the wrong direction for $ T_{\r{W,AC,out}} $ and $ T_{\r{W,E,out}} $ remains. 
%	wrongly modeling a non-existing strong non-minimum phase system behavior remains. 
	Therefore, for the modeling of the SHX the approach based on variable $ \mathit{TTD} $ should be favored over the approach with constant $ \mathit{UA} $ if $ \dot{V}_{\r{RSo}} $ is expected to vary. 
%	of the base model and V1  (with variable $ \mathit{TTD} $) should definitely be favored over the modeling approach of \emph{V2} (with 
	% Due to the previously discussed drop in the poor solution flow rate and the resulting reduced absorber capacity, less refrigerant is absorbed. Consequently the pressure in the low pressure vessel increases and thus also the evaporation temperature increases, which ultimately leads to the slight increase in $ T_{\r{W,E,out}} $ at $ t=\SI{30}{\second} $ .
	
	%Both models can correctly represent the delayed response to the step change, as well as the time constant of the decreasing chilled water outlet temperature $ T_{\r{W,E,out}} $, 
	
	\subsubsection{Steps in other input variables}
	%\newline
	%The experimental validation in this section shows, that both models have satisfactory steady-state and dynamic accuracy. 
	The authors also investigated step responses for variations in the remaining four  input variables and observed similar dynamic accuracy as discussed above. 
	Furthermore, it was found that time constants for variations in the cooling water circuit are much smaller ($<$ \SI{1}{\min}) than for the hot and chilled water circuit.  This is explained by the fact that the cooling water circuit is connected to two components - to the absorber and to the condenser - and therefore,
	% variations in the cooling water circuit directly act upon two components and 
	their effects show much faster throughout the whole AHPD.
	%  compared to variations in the cold or the hot water circuit.
	% on one component in the high-pressure range, and one component in the low-pressure range, at a time, and thus changes act more quickly throughout the AHPD. 
	%Step responses for changes in  $ \dot{V}_{\r{W,G}} $ have similar time constants as described in sec. \autoref{subsubsec:T_W_G_in}, step responses for change in  $ T_{\r{W,E,in}} $ have similar time constants as described in sec. \autoref{subsubsec:Vdot_W_E}
	
	
	% time to reach the equilibrium state again after the stepwise change of an input variable depends on the respective input and output variable, three jump experiments are shown as examples.
	%
	%To illustrate that the time to reach steady-state again after a step-wise change of an input variables depends on the 
	%  constants  steps in different input variables result 
	
	%Moreover, such sudden, large changes in the rich solution flow rate are unlikely to occur under realistic operation conditions. 
	%When designing the controller it can be still be useful to consciously consider the deviation between the modeled and the real system and limit the rate of change for rich solution flow rate if it is used as a manipulated variable. 
	
	
	
	%Use:
	%
	%for $ \dot{m}_{W,Eva} $: 3.Mai 2019
	%for $ \dot{V}_{RSo} $: 15. März 2019
	%for $ T_{W,G,in} $: 18. März 2019
	
	
	
	\section{Conclusion and outlook}
	\label{sec:conclusion}
	%main results:
	%- linearization makes steady-state accuracy worse but does not impair dynamic accuracy
	%- models with variable heat transfer yield better steady-state accuracy over wider operating range
	%- new model (with TTD + dynamics) yield much better results for steps in Vdot_RSo. Nevertheless, not perfect due to unmodeled changes in Vdot_PSo - not sure whether this applies to all machines. 
	%While AHPD have the potential to use waste heat and renewable energies like solar or geothermal energy instead of electricity to generate heat and cold, their 
	% state-of-the-art controllers of AHPD do not yet allow for 
	The aim of this paper was to find a model suitable to be used for the design of dynamic model-based control of AHPDs. First, an assessment of mass and energy stores showed that the most relevant dynamic effects are caused by the fluid in the sumps and in the SHX. Three model versions, differing in the way the heat transfer is modeled in the main components' HXs on the one hand and in the SHX on the other hand,  were then presented and investigated. The experimental validation results indicated that models with variable heat transfer correlations (model version \emph{base-a}) yield better steady-state accuracy over a wider operating range than models that use constant $ \mathit{UA} $-values in the main components' HXs (version \emph{V1} and \emph{V2} ). 
	%However, the  steady-state accuracy of all three models is considered sufficient for use in controller design. 
	Furthermore, it was shown that the common approach to model heat transfer in the SHX with constant $ \mathit{UA} $-values (\emph{V2}) does provide satisfactory steady-state accuracy and can also represent dynamic load changes of almost all input variables well but it cannot model dynamic changes of the rich solution volume flow rate $ \dot{V}_\r{RSo} $  appropriately. Instead,  it is recommended to model heat transfer in the SHX by means of an approach based on terminal temperature differences (as used in \emph{base-a} and \emph{V1}).
	
	Experimental validation also revealed a flaw that affects all model versions: For the step-like change in  $ \dot{V}_\r{RSo} $ a fluctuation in the poor solution flow rate $ \dot{V}_\r{PSo} $ - probably caused by rising gas bubbles - was observed in the investigated AHPD. None of the models consider this phenomenon, which should be taken into account when using $ \dot{V}_\r{RSo} $  as manipulated variable in a later control problem. 
	
	Another interesting finding of this contribution is that the linearization of the presented model
	% is presented and simulations results are compared to those of the nonlinear base model and measurement data. 
	only slightly impairs the model's steady-state accuracy for variations of the volume flow rates in the hydraulic circuits, but not significantly for variations of the other input variables. Additionally, the dynamic accuracy, i.e., how well the system's main dynamics can be reproduced, appears to be almost unaffected by the linearization process so that the linearized model  can be considered highly suitable to be used for model-based control. It has the very simple mathematical structure of a linear state-space model and can describe dynamic and steady-state effects of variations of all input variables sufficiently well.
	
	
	% 
	% \subsubsection{Model scalability}
	% The models can be scaled for AHPD of different size but similar design compared to the reference AHPD from \autoref{subsec:EAW_AHPD} by scaling selected parameters. Substance property data function parameters from \autoref{sec:param_substance_prop} can be adopted without change. Parameters concerning the total refrigerant and LiBr amount
	% %, $ m_{\r{tot,0}} $ and $ m_{\r{LiBr,0}} $, 
	% and also the amount of solution stored in the SHX (see \autoref{sec:param_mass}) have to be changed according to manufacturer's information. 
	% %The amount of refrigerant in the condenser can be scaled using the cooling capacity ratio between new and reference machine, $ \phi_\r{\dot{Q}_\r{E}} =\frac{\dot{Q}_\r{E,new}}{\dot{Q}_\r{E,ref}}$. Similarly, 
	% The constant $ \mathit{UA} $ values (see \autoref{sec:param_empirical}) for model versions \emph{V1} and \emph{V2} can be scaled using the cooling capacity ratio between new and reference machine, $ \phi_{\dot{Q}} =\frac{\dot{Q}_\r{E,new}}{\dot{Q}_\r{E,ref}}$
	% . For more complex heat transfer correlations, like e.g.  \autoref{eq:UA_G} for operating point dependant $ \mathit{UA}_\r{G} $, parameters have to be scaled in such a way that the ratio between the resulting new $ \mathit{UA}$ value $ \mathit{UA}_\r{G,new}$ and the reference value $ \mathit{UA}_\r{G,ref} $ again corresponds to $ \phi_{\dot{Q}} $. For \autoref{eq:UA_G} this means
	% \begin{equation}
	% 	\label{eq:UA_G_new}
	% 	\begin{split}
	% 		UA_\r{G,new}&\overset{!}{=}  \phi_{\dot{Q}}UA_\r{G,ref}  \\
	% 		&= 
	% 		\phi_{\dot{Q}}\r{K_{G1} }+ \r{K_{G2}} \dot{m}_{\r{W,G,new}} +\\&~~~~ \r{K_{G3}} \dot{m}_{\r{RSo,SHX,out, new}} + \r{ K_{G4}} \dot{m}_{\r{Ref,GRh,new}}
	% 	\end{split}
	%  \end{equation} 
	% % Specifically, this means for equation 1, for example:
	% 
	% The remaining parameters from \autoref{sec:param_empirical} can be scaled according to the same principle.  
	
	% of the linear model is similar to the nonlinear one for variations of inlet temperatures and volume flow rate of rich solution and worse for variations of volume flow rate in the hydraulic circuits. However, it is still considered good enough for use in model-based control. The dynamic accuracy, i.e. how well the the shape of the system response can be modeled, seems to be almost unaffected by the linearization process so that the linearized model version V3 is considered highly suitable to be used for model-based control. 
	
	
	
	%The first one consists of nonlinear, coupled differential and algebraic equations and is based on mass and energy balances, simple heat transfer correlations and substance property equations. 
	%The second one was derived from the nonlinear model by means of linearization in a reference operating point and elimination of algebraic equations, resulting in the favorable structure of a linear state-space model. 
	%It consists merely of linear ordinary differential equations and linear output equations. 
	%Despite their simple mathematical structure, both models can reproduce the main system dynamics of the investigated AHPD sufficiently well. Furthermore, both describe the effects of variations of all manipulable input variables which makes them generic models that can be used for various AHPD applications, regardless of which variables are chosen as manipulated and as controlled variables. 
	%Steady-state accuracy was better for the nonlinear model but still sufficiently good for the linear model while dynamic accuracy was found to be equally good for both models. Both models are considered viable models for model-based control of AHPDs at component-level. 
	%The linear model, however, allows the use of a wider range of model-based control methods due to its simpler mathematical structure which makes it even more promising.
	
	
	%\vspace{5mm} %5mm vertical space
	%In summary, both models showed very similar dynamic accuracy which indicates that the linearization of the nonlinear model (\autoref{subsec:linearization}) does not impair its dynamic accuracy. Both models agree well with measurement data after steps in inlet temperatures and volume flow rates of external hydraulic circuits, show minor deviations after steps in the rich solution volume flow rate, but still represent the main system dynamics very well. Therefore, both models are considered viable models for use in model-based control.  
	
	
	%With this work it was shown, that it is possible to use simple, low-order mathematical models to accurately describe the main system dynamics of an AHPD which is an important step for the design of model-based control strategies. 
	%The nonlinear model can therefore also be used as a simulation model in the computer-aided controller design process. 
	%In any case, it is definitely noteworthy that the AHPD, which is commonly seen as an extremely nonlinear system \todo{LIT}, can be described sufficiently well with a linear state-space model with merely 9 state variables. \todo{+ Relativierender Satz, dass bei uns ss-accuracy nicht so wichtig ist?}
	As a next step, the scalability of the suggested models shall be investigated for AHPDs of different sizes.
	%	 and similar design compared to the AHPDs used for validation in this paper \cite{EAW2017}. 
	First investigations showed that it should be possible to re-parameterize the model with only basic information like cooling capacity in at least one operating point as well as refrigerant and LiBr filling amount. 
	Moreover, the authors plan to use the linearized model  to design a model-based controller to extend the operating range of AHPDs, and in the course of this, to use the developed nonlinear model  as a simulation model for a virtual test bench to test different control settings. 
	
	%Since the linear model has the favorable structure of a linear state-space model, the linear model is therefore considered to be 
	
	%Both models fulfill the specified requirements
	%TODO
	
	
	\section{Acknowledgments}
%	The content of this publication was developed in cooperation with our project partners \textit{SOLID Solar Energy Systems GmbH}, \textit{EAW Energieanlagenbau GmbH Westenfeld}, \textit{Pink GmbH} and \textit{AEE - Institut für Nachhaltige Technologien} and received funding from the Austrian Climate and Energy Fund under the Grant No. 865095  in the framework of the \textit{Energieforschungsprogramm 2017} program, from the Horizon 2020 program under Grant No. 792276 and from the COMET program, which is managed by the Austrian Research Promotion Agency (FFG) and is co-financed by the Republic of
%	Austria and the Federal Provinces of Vienna, Lower Austria and Styria.
%	
	The research leading to these results received funding from the Austrian Climate and Energy Fund under the Grant No. 865095  in the framework of the \textit{Energieforschungsprogramm 2017} program, from the Horizon 2020 program under Grant No. 792276 and from the COMET program, which is managed by the Austrian Research Promotion Agency (FFG) and is co-financed by the Republic of
	Austria and the Federal Provinces of Vienna, Lower Austria and Styria.
	
	%\label{sec:references}
	
	% Literatureinbindung:
	%\bibliographystyle{elsarticle-harv} 
	\bibliographystyle{elsarticle-num} 
	% \setcitestyle{number}
	%\setcitestyle{authoryear,open={((},close={))}}
	\bibliography{Lit_ModellingPaper}
	
	%\clearpage
	%\listoftodos
	\newpage
	\appendix
	

	\section{Empirical parameters for submodels}
	\label{sec:param_empirical}
	
	\autoref{tab:param_empirical} summarizes the empirically determined parameters for the modeling approaches in \autoref{sec:Sim_model} and \autoref{sec:benchmark_models}. 
	
	\begin{table}[H]
		\centering
				\caption{Empirical parameters for submodels}%
		\begin{tabular}{lll}
			\toprule
			submodel                                    & \multicolumn{2}{c}{parameter}       \\
			\midrule
			$\mathit{UA}_{\r{G}}$, const. for \emph{V1} and \emph{V2} & $\mathit{UA}_{\r{G,const.}}$   & 2.594E+03 \\
			\midrule
			$\mathit{UA}_{\r{C}}$, const. for \emph{V1} and \emph{V2} & $\mathit{UA}_{\r{C,const.}}$   & 1.731E+03 \\
			\midrule
			$\mathit{UA}_{\r{E}}$, const. for \emph{V1} and \emph{V2} & $\mathit{UA}_{\r{E,const.}}$   & 4.337E+03 \\
			\midrule
			$\mathit{UA}_{\r{A}}$, const. for \emph{V1} and \emph{V2} & $\mathit{UA}_{\r{A,const.}}$   & 7.759E+03 \\
			\midrule
			$\mathit{UA}_{\r{SHX}}$, const. for \emph{V2}      & $\mathit{UA}_{\r{SHX}}$ & 2.803E+03 \\
			\midrule
			$\mathit{UA}_{\r{G}}$, for base model               & $\r{K_{G1}}$            & 2.523E+02 \\
			& $\r{K_{G2}}$            & 7.238E+02 \\
			& $\r{K_{G3}}$            & 5.003E+03 \\
			& $\r{K_{G4}}$            & 1.642E+05 \\
			\midrule
			$\epsilon_{\r{C}}$, for base model                  & $\r{K_{C1}}$            & 8.836E-01 \\
			& $\r{K_{C2}}$            & 1.034E-01 \\
			\midrule
			$\epsilon_{\r{E}}$, for base model            & $\r{K_{E1}}$            & 1.046E+00 \\
			& $\r{K_{E2}}$            & 1.964E-01 \\
			\midrule
			$\mathit{UA}_{\r{A}}$, for base model           & $\r{K_{A1}}$            & 1.673E+03 \\
			& $\r{K_{A2}}$            & 2.483E+02 \\
			& $\r{K_{A3}}$            & 1.812E+03 \\
			& $\r{K_{A4}}$            & 6.038E+04 \\
			\midrule
			$\mathit{TTD}_{\r{h}}$, for \emph{V1} and base model       & $\r{K_{h1}}$          & 1.109E-02 \\
			& $\r{K_{h2}}$          & 7.704E-04 \\
			& $\r{K_{h3}}$          & 4.409E-04 \\
			\midrule
			$\mathit{TTD}_{\r{l}}$, for \emph{V1} and base model    & $\r{K_{l1}}$          & 7.319E-02 \\
			& $\r{K_{l2}}$          & 2.637E-04 \\
			& $\r{K_{l3}}$          & 4.150E-04 \\
			\midrule
			$\dot{m}_{\r{PSo,A,in}}$                    & $K_\r{SEV}$               & 2.579E-02   \\
			\bottomrule
		\end{tabular}
		\label{tab:param_empirical}
	\end{table}
	%
	%\begin{table}[h!]
	%	\centering
	%	\begin{tabular}{lll}
	%		\toprule
	%		submodel                                                                                       & \multicolumn{2}{c}{parameters} \\
	%		\midrule
	%		$\epsilon_{\r{G}}$, \autoref{eq:eff_G}                                    & $\r{K_{G1}}$     & 1,011E+00   \\
	%		& $\r{K_{G2}}$     & 6,278E-01   \\
	%		& $\r{K_{G3}}$     & 9,410E-01   \\
	%		& $\r{K_{G4}}$     & 9,488E-01   \\
	%		& $\r{K_{G5}}$     & 3,065E+01   \\
	%		\midrule
	%		$\epsilon_{\r{C}}$, \autoref{eq:eff_C}                                    & $\r{K_{C1}}$     & 8,728E-01   \\
	%		& $\r{K_{C2}}$     & 1,109E-01   \\
	%		\midrule
	%		$\epsilon_{\r{E}}$, \autoref{eq:eff_E}                                    & $\r{K_{E1}}$     & 1,052E+00   \\
	%		& $\r{K_{E2}}$     & 2,041E-01   \\
	%		\midrule
	%		$UA_{\r{A}}$, \autoref{eq:UA_A}                    & $\r{K_{A1}}$     & 9,924E+03   \\
	%		& $\r{K_{A2}}$     & 2,371E+02   \\
	%		& $\r{K_{A3}}$     & 1,977E+03   \\
	%		& $\r{K_{A4}}$     & 1,549E+04   \\
	%		& $\r{K_{A5}}$     & 9,646E+04   \\
	%		\midrule
	%		$f_{\r{Rso}}$,  \autoref{eq:f_SHX_RSo} & $\r{K_{RSo1}}$   & 1,109E-02   \\
	%		& $\r{K_{RSo2}}$   & 7,704E-04   \\
	%		& $\r{K_{RSo3}}$   & 4,409E-04   \\
	%		\midrule
	%		$f_{\r{Pso}}$,  \autoref{eq:f_SHX_PSo} & $\r{K_{PSo1}}$   & 7,319E-02   \\
	%		& $\r{K_{PSo2}}$   & 2,637E-04   \\
	%		& $\r{K_{RSo3}}$   & 4,150E-04   \\
	%		\midrule
	%		$\dot{m}_{\r{PSo,G,out}}$, \autoref{eq:mdot_PSo}                             & $K_{SEV}$        & 2,579E-02   \\
	%		\bottomrule
	%	\end{tabular}
	%	\caption{Empirical parameters for submodels}%
	%	\label{tab:param_empirical}
	%\end{table}
	
	
	
	\section{Parameters for mass stores}
	\label{sec:param_mass}
	
	\autoref{tab:mass} summarizes stored fluid masses  used  for the modeling approaches in \autoref{sec:Sim_model} and \autoref{sec:benchmark_models}. 
	
	\begin{table}[H]
		\centering
				\caption{Stored fluid masses}%
		\begin{tabular}{lc}
			\bottomrule
			fluid                                       & mass [kg] \\
			\midrule
%			total LiBr  $m_{\r{LiBr}}$                & 21.63    \\
%			total $\r{H_2O}$ and LiBr  $m_{\r{LiBr+H_2O}}$ & 60.00   \\
%					  LiBr in sumps $m_{\r{LiBr,sumps}}$   &  21.63       \\
%			$\r{H_2O}$ and LiBr in sumps $m_{\r{LiBr+H_2O,sumps}}$   & 60.00   \\
		  LiBr in sumps $m_{\r{LiBr,sumps}}$   & 15.98     \\
			$\r{H_2O}$ and LiBr in sumps $m_{\r{LiBr+H_2O,sumps}}$  & 49.36   \\
		refrigerant in condenser sump $m_{\r{Ref,C}}$    & 4.15    \\
			rich solution in SHX $m_{\r{RSo,SHX}}$     & 5.32        \\
		poor solution in SHX $m_{\r{PSo,SHX}}$  & 5.32     \\
			\bottomrule
		\end{tabular}
		\label{tab:mass}
	\end{table}
	
	
%		\begin{table}[H]
%		\centering
%		\begin{tabular}{lc}
%			\bottomrule
%			fluid                                       & mass [kg] \\
%			\midrule
%			total mass of LiBr  $m_{\r{LiBr}}$                & 21,63    \\
%			mass of $\r{H_2O}$ and LiBr in sumps $m_{\r{LiBr+H_2O,sumps}}$  & 60,00   \\
%			mass of refrigerant in condenser $m_{\r{Ref,C}}$    & 4,15    \\
%			mass of rich solution in SHX $m_{\r{RSo,SHX}}$      & 5,32        \\
%			mass of poor solution in SHX $m_{\r{PSo,SHX}}$      & 5,32     \\
%			mass of LiBr in sumps $m_{\r{LiBr,sumps}}$      & 15,98     \\
%			\bottomrule
%		\end{tabular}
%		\caption{Mass storage values}%
%		\label{tab:mass}
%	\end{table}
	
	\newpage
	
	\section{Parameters for  property  functions}
	\label{sec:param_substance_prop}
	
	\autoref{tab:substance} summarizes the parameters for the used  property  functions and the range for mass fraction of $ \r{LiBr} $  in the solution $ \xi ~[-] $, temperature $ T~[\si{\kelvin}] $ and pressure $ p~[\si{\pascal}]$, that were used for parametrization. 
	%The given range indicates the values that were used to fit the more complex substance property equations. 
	The parameters for functions for saturation pressure  ($\r{ln}(p_{\r{sat,\r{LiBr/H_2O}}})$ and $\r{ln}(p_{\r{sat,\r{H_2O}}})$) are given for two individual ranges, because different parameters are used for high and low pressure components. 
	As discussed in \autoref{sec:Sim_model} all  property  functions used in this paper are simplified functions based on more complex  property  functions from  \cite{Yuan2005}  for $ \r{LiBr/H_2O} $ and from \cite{Wagner2002} for water. 
	%% to force table to be at top of the page
	%\makeatletter
	%\setlength{\@fptop}{0pt}
	%\makeatother
	
	
	%\begin{table}[h!]
	%	\begin{tabular}{lcll}
	%		\toprule
	%		property                            & range                           & \multicolumn{2}{c}{parameters} \\
	%		\midrule
	%		$ h_{\r{LiBr/H_2O}} $               & $ 0,45<\xi< 0,6$                & $A_1$        & 6,892E+05       \\
	%		& $ 293,15<T< 363,15$             & $A_2$        & 7,001E+05       \\
	%		&                                 & $A_3$        & 1,738E+06       \\
	%		&                                 & $A_4$        & 3,617E+03       \\
	%		&                                 & $A_5$        & 2,827E+03       \\
	%			\midrule
	%		$\r{ln}(p_{\r{sat,\r{LiBr/H_2O}}})$ & $ 0,45<\xi< 0,6$                & $B_1$        & 1,226E+01       \\
	%		& $ 293,15<T< 328,15$             & $B_2$        & 1,042E+01       \\
	%		& $p< 2200$                       & $B_3$        & 1,944E+01       \\
	%		&                                 & $B_4$        & 6,237E-02       \\
	%				\cmidrule{2-4}
	%		& $ 0,45<\xi< 0,6$                & $B_1$        & 6,804E+00       \\
	%		& $ 323,15<T< 363,15$             & $B_2$        & 7,405E+00       \\
	%		& $5000<p< 14000$                 & $B_3$        & 1,483E+01       \\
	%		&                                 & $B_4$        & 4,647E-02       \\
	%			\midrule
	%$ \rho_{\r{LiBr/H_2O}} $            & $ 0,45<\xi< 0,6$                & $R_1$        & 1,349E+03       \\
	%& $ 293,15<T< 363,15$             & $R_2$        & 2,274E+02       \\
	%&                                 & $R_3$        & 1,856E+03       \\
	%&                                 & $R_4$        & 5,569E-01       \\
	%			\midrule
	%		$ h^l_{\r{H_2O}} $                  & $ 278,15<T< 368,15$             & $C_1$        & 1,143E+06       \\
	%		&                                 & $C_2$        & 4,186E+03       \\
	%			\midrule
	%		$ h^v_{\r{H_2O}} $                  & $ T_{\r{sat}}<T< T_{\r{sat}}+5$ & $D_1$        & 2,009E+06       \\
	%		& $800<p< 15000$                  & $D_2$        & 1,803E+03       \\
	%				\midrule
	%		$\r{ln}(p_{\r{sat,\r{H_2O}}})$      & $ 277,15<T< 293,15$             & $E_1$        & 1,158E+01       \\
	%		&                                 & $E_2$        & 6,599E-02       \\
	%				\cmidrule{2-4}
	%		& $ 303,15<T< 323,15$             & $E_1$        & 7,591E+00       \\
	%		&                                 & $E_2$        & 5,266E-02      \\
	%					\midrule	
	%			$ \rho^{\r{l}}_{\r{H_2O}} $         & $ 278,15<T< 368,15$             & $\r{F_1}$      & 7.397E+02     \\
	%		&                                 & $\r{F_2}$      & 1.984E+00     \\
	%		&                                 & $\r{F_3}$      & 3.760E-03     \\
	%		\bottomrule
	%	\end{tabular}
	%	\caption{Parameters for substance property equations}%
	%			\label{tab:substance}
	%\end{table}
	
	
	\begin{table}[H]
				\caption{Parameters for  property  functions}%
		\begin{tabular}{lcll}
			\toprule
			property                            & range                           & \multicolumn{2}{c}{parameters} \\
			\midrule
			$ h_{\r{LiBr/H_2O}} $               & $ 0.45<\xi< 0.6$                & $\r{A_1}$     & 6.892E+05     \\
			& $ 293.15<T< 363.15$             & $\r{A_2}$      & 7.001E+05     \\
			&                                 & $\r{A_3}$      & 1.738E+06     \\
			&                                 & $\r{A_4}$      & 3.617E+03     \\
			&                                 & $\r{A_5}$      & 2.827E+03     \\
			\midrule
			$\r{ln}(p_{\r{sat,\r{LiBr/H_2O}}})$ & $ 0.45<\xi< 0.6$                & $\r{B_1}$      & 1.226E+01     \\
			& $ 293.15<T< 328.15$             & $\r{B_2}$      & 1.042E+01     \\
			& $p< 2200$                       & $\r{B_3}$      & 1.944E+01     \\
			&                                 & $\r{B_4}$      & 6.237E-02     \\
			\cmidrule{2-4}
			& $ 0.45<\xi< 0.6$                & $\r{B_1}$      & 6.804E+00     \\
			& $ 323.15<T< 363.15$             & $\r{B_2}$      & 7.405E+00     \\
			& $5000<p< 14000$                 & $\r{B_3}$      & 1.483E+01     \\
			&                                 & $\r{B_4}$      & 4.647E-02     \\
			\midrule
			$ \rho_{\r{LiBr/H_2O}} $            & $ 0.45<\xi< 0.6$                & $\r{R_1}$      & 1.349E+03     \\
			& $ 293.15<T< 363.15$             & $\r{R_2}$      & 2.274E+02     \\
			&                                 & $\r{R_3}$      & 1.856E+03     \\
			&                                 & $\r{R_4}$      & 5.569E-01     \\
			\midrule
			$ h^{\r{l}}_{\r{H_2O}} $            & $ 278.15<T< 368.15$             & $\r{C_1}$      & 1.143E+06     \\
			&                                 & $\r{C_2}$      & 4.186E+03     \\
			\midrule
			$ h^{\r{v}}_{\r{H_2O}} $            & $ T_{\r{sat}}<T< T_{\r{sat}}+5$ & $\r{D_1}$      & 2.009E+06     \\
			& $800<p< 15000$                  & $\r{D_2}$      & 1.803E+03     \\
			\midrule
			$\r{ln}(p_{\r{sat,\r{H_2O}}})$      & $ 277.15<T< 293.15$             & $\r{E_1}$      & 1.158E+01     \\
			&                                 & $\r{E_2}$      & 6.599E-01     \\
			\cmidrule{2-4}
			& $ 303.15<T< 323.15$             & $\r{E_1}$      & 7.591E+00     \\
			&                                 & $\r{E_2}$      & 5.266E-02     \\
			\midrule
			$ \rho^{\r{l}}_{\r{H_2O}} $         & $ 278.15<T< 368.15$             & $\r{F_1}$      & 7.397E-01     \\
			&                                 & $\r{F_2}$      & 1.984E-03     \\
			&                                 & $\r{F_3}$      & 3.760E-06     \\       
			\bottomrule
		\end{tabular}
		\label{tab:substance}
	\end{table}
	
	
	
	
\end{document}