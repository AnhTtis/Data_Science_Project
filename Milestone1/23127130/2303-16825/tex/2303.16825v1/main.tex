\documentclass[%
reprint,
superscriptaddress,
%groupedaddress,
%unsortedaddress,
%runinaddress,
%frontmatterverbose,
%preprint,
%preprintnumbers,
%nofootinbib,
%nobibnotes,
%bibnotes,
amsmath,
amssymb,
aps,
%pra,
prx,
%prb,
%rmp,
%prstab,
%prstper,
floatfix,
]{revtex4-2}
\usepackage{graphicx}% Include figure files
\usepackage{dcolumn}% Align table columns on decimal point
\usepackage{bm}% bold math
\usepackage{hyperref}% add hypertext capabilities
\usepackage{xr}
%\usepackage{color}
%\externaldocument{Al_graphene_SM}
\usepackage{lipsum}
\usepackage{braket}
%for including svg graphics
%\usepackage{graphicx}
\usepackage{xcolor}
\begin{document}
\preprint{APS/123-QED}
\bibliographystyle{unsrtnat} 
%=======================================================================================================================================
\title{Strong coupling between a microwave photon and a singlet-triplet qubit}
%%%%%
\author{J.\,H.~Ungerer}
\altaffiliation{Equal contributions.}
%\email{jannhinnerk.ungerer@unibas.ch}
\affiliation{
Department of Physics, University of Basel, Klingelbergstrasse 82 CH-4056, Switzerland
}
\affiliation{
Swiss Nanoscience Institute, University of Basel, Klingelbergstrasse 82 CH-4056, Switzerland
}
\author{A.~Pally\footnotemark[1]}
\altaffiliation{Equal contributions.}
\affiliation{
Department of Physics, University of Basel, Klingelbergstrasse 82 CH-4056, Switzerland
}
\author{A.~Kononov}
\affiliation{
Department of Physics, University of Basel, Klingelbergstrasse 82 CH-4056, Switzerland
}

\author{S.~Lehmann}
\affiliation{Solid State Physics and NanoLund, Lund University, Box 118, S-22100 Lund, Sweden}

\author{J.~Ridderbos}
\altaffiliation{Current address: MESA Institute for Nanotechnology, University of Twente, P.O. Box 217, 7500 AE Enschede, The
Netherlands}
\affiliation{
Department of Physics, University of Basel, Klingelbergstrasse 82 CH-4056, Switzerland
}

\author{C.~Thelander}
\affiliation{Solid State Physics and NanoLund, Lund University, Box 118, S-22100 Lund, Sweden}

\author{K.A.~Dick}
\affiliation{Centre for Analysis and Synthesis, Lund University, Box 124, S-22100 Lund, Sweden}

\author{V.F.~Maisi}
\affiliation{Solid State Physics and NanoLund, Lund University, Box 118, S-22100 Lund, Sweden}

\author{P.~Scarlino}
\affiliation{Institute of Physics and Center for Quantum Science and Engineering, Ecole Polytechnique Fédérale de Lausanne, CH-1015 Lausanne, Switzerland}


\author{A.~Baumgartner}
\homepage{www.nanoelectronics.unibas.ch}
\affiliation{
Department of Physics, University of Basel, Klingelbergstrasse 82 CH-4056, Switzerland
}
\affiliation{
Swiss Nanoscience Institute, University of Basel, Klingelbergstrasse 82 CH-4056, Switzerland
}

\author{C.~Sch{\"o}nenberger}
\homepage{www.nanoelectronics.unibas.ch}
\affiliation{
Department of Physics, University of Basel, Klingelbergstrasse 82 CH-4056, Switzerland
}
\affiliation{
Swiss Nanoscience Institute, University of Basel, Klingelbergstrasse 82 CH-4056, Switzerland
}
\date{\today}
%=======================================================================================================================================
% ABSTRACT
%=======================================================================================================================================

\begin{abstract}
Tremendous progress in few-qubit quantum processing has been achieved lately using superconducting resonators coupled to gate voltage defined quantum dots. While the strong coupling regime has been demonstrated recently for odd charge parity flopping mode spin qubits, first attempts towards coupling a resonator to even charge parity singlet-triplet spin qubits have resulted only in weak spin-photon coupling strengths.
Here, we integrate a zincblende InAs nanowire double quantum dot with strong spin-orbit interaction in a magnetic-field resilient, high-quality resonator.
In contrast to conventional strategies, the quantum confinement is achieved using deterministically grown wurtzite tunnel barriers without resorting to electrical gating.
Our experiments on even charge parity states and at large magnetic fields, allow us to identify the relevant spin states and to measure the
spin decoherence rates and spin-photon coupling strengths.
Most importantly, at a specific magnetic field, we find an anti-crossing between the resonator mode in the single photon limit and a singlet-triplet qubit with an electron spin-photon coupling strength of $g = 114 \pm 9$ MHz, reaching the strong coupling regime in which the coherent coupling exceeds the combined qubit and resonator linewidth.
\end{abstract}
\maketitle
%%%% INTRO %%%%
%\section{Introduction}
Spin qubits in semiconductors are promising candidates for scalable quantum information processing due to long coherence times and fast manipulation~\cite{hanson2007spins,zwanenburg2013silicon,vandersypen2017interfacing,chatterjee2021semiconductor}.
For the qubit readout, circuit quantum electrodynamics based on superconducting resonators\cite{childress2004mesoscopic}, allows a direct and fast measurement of qubit states and their dynamics~\cite{petersson2012circuit}.
Recently, resonators were used to achieve charge-photon~\cite{stockklauser2017strong,mi2017strong}, spin-photon~\cite{mi2018coherent,samkharadze2018strong,landig2018coherent} as well as coherent coupling of distant charge~\cite{van2018microwave} and spin qubits~\cite{borjans2020resonant,harvey2022coherent}, enabling coherent information exchange between distant qubits.
However, the small electric and magnetic moments of individual electrons require complicated device architectures such as micromagnets, and a large number of surface gates that render scaling up to more complex architectures challenging. These approaches typically achieve a relatively weak electron spin-photon coupling on the order of $\sim 10 - 30$\,MHz.
In addition to single electron spin-qubits, also spin qubits based on two electrons in a double quantum dot (DQD), e.g. in a singlet-triplet qubit have been demonstrated~\cite{petta2005coherent}.
Spin qubits based on two electrons typically offer a large hybridization of the spin and charge degree of freedom compared to single-electron spin qubits in principle allowing even stronger coupling strengths.
So far, however, the experimentally achieved coupling strengths in such systems~\cite{landig2019microwave,bottcher2022parametric} remained well below the strong coupling limit in which the coherent coupling rate exceeds both, the cavity mode decay rate and the qubit linewidth.

Here, we demonstrate that the strong coupling regime between a singlet-triplet qubit and a superconducting resonator can be reached.
We achieve this strong coupling by carefully designing the resonator and by using a DQD defined by in-situ grown tunnel barriers in a semiconductor with a large spin-orbit interaction.
The tunnel barriers consist of InAs segments in the wurtzite crytsal-phase with an atomically sharp interface to the zincblende bulk of the nanowire (NW)~\cite{lehmann2013general}.
We make use of the large spin-orbit interaction in these wires~\cite{nilsson2018tuning} to define a singlet-triplet qubit at a finite magnetic field in which the $T_{1,1}^+$ and $S_{2,0}$ states hybridize, forming a quantum two-level system.
For the qubit spectroscopy, we use a magnetic-field resilient resonator based on NbTiN that can be operated in magnetic fields exceeding 2\,T~\cite{samkharadze2016high,ungerer2023performance}.
At $B\sim 1.7$\,T, the singlet-triplet qubit hybridizes with the resonator with a record-high electron spin-photon coupling strength reaching the strong coupling limit.
\section{Device characterization}
\begin{figure}
    \centering
    \includegraphics[width=\linewidth]{Figs/Fig1.pdf}
    \caption{\textbf{Coupled resonator-qubit system} (a) False colored SEM-image of the device. The NW (green) is divided into two segments by an in-situ grown tunnel barrier (red), thus forming the DQD system. The NW ends are contacted by two Ti/Au contacts (S,D) and the NW segements can be electrically tuned by two Ti/Au sidegates SG$_R$ (purple) and SG$_L$ (yellow).
    A high-impedance, half-wave resonator is connected to SG$_R$. Top gates (orange) are kept at a constant voltage of $-0.05$~V. The magnetic field is applied in-plane at an angle of $\sim60^\circ$ %$123.6^\circ$
    to the NW. The arrows illustrate an even charge configuration with the two degenerate DQD states $T_{1,1}^+$ and $S_{2,0}$. (b) Schematic of the crystal-phase defined DQD. The conduction band of wurzite and zincblende are offset by $\sim 100$ meV, resulting in a tunnel barrier between the zincblende segments. The intrinsic spin-orbit interaction enables spin-rotating tunneling between these segments. (c) Energy levels of an even charge configuration as a function of magnetic field $B$ at a fixed positive detuning $\epsilon$ between the dot levels. %of $\epsilon=2/2\pi$ GHz.
    At finite magnetic fields, $T_{1,1}^+$ (blue) hybridizes with $S_{2,0}$ (red) defining a singlet-triplet qubit with an energy splitting given by the spin-orbit interaction strength $\Delta_\mathrm{SO}$.}
    \label{fig:Fig1}
\end{figure}
The resonator-qubit system is shown in Figure~\ref{fig:Fig1}a), including a false-colored SEM-image of the crystal-phase defined NW DQD. The DQD is hosted in the $280$\,nm and $380$\,nm long zincblende segments (green), separated by $30$\,nm long wurtzite (red) tunnel barriers with a conduction band offset of $\sim$100\,meV~\cite{nilsson2016single}, as illustrated in Fig.~\ref{fig:Fig1}(b).
A high-impedance, half-wave coplanar-waveguide resonator is capacitively coupled to the DQD at its voltage anti-node via a sidegate.
In addition, the same sidegate can be used to tune the DQD charge states using a dc voltage ($V_R$) applied at the resonator voltage node. The DQD state is probed by reading out the resonator rf-transmission.
We extract the bare resonance frequency of the resonator $\omega_r/2\pi=5.25308\pm0.00003$\,GHz and the bare decay rate $\kappa/2\pi=34.4$ MHz.
The resonator design and fitting are described in detail in the methods section~\ref{app:resonator}.

In the following, we prepare the DQD in  an even charge configuration in the many-electron regime (see methods~\ref{app:evenodd}), described by a two-electron Hamiltonian given in the methods section~\ref{app:Hamiltonian}.
Figure \ref{fig:Fig1}(c) shows the eigenvalues of this Hamiltonian as a function of external magnetic field $B$ at a fixed DQD detuning.
At zero magnetic field, the detuning renders the singlet $S_{2,0}$ the ground state, for which both electrons reside in the same dot.
Without spin-rotating tunneling, this, and the $S_{1,1}$ state, with the electrons distributed to different dots, form a charge qubit~\cite{vanderwiel2002}. The subscripts describe the dot electron occupation of the left and right dot respectively.
By applying an external magnetic field, the Zeeman effect can lower the energy of the triplet $T_{1,1}^+$ state, that becomes the ground state for sufficient high magnetic fields.
However, the intrinsic spin-orbit interaction hybridizes the $S_{2,0}$ and $T_{1,1}^+$ states, with the two new eigenstates of the avoided crossing forming a singlet-triplet qubit shown schematically in Figs.~\ref{fig:Fig1}(a) and (b).

\begin{figure}
    \centering
    \includegraphics[width=\linewidth]{Figs/Fig2.pdf}
    \caption{\textbf{Dispersive sensing of the DQD at $\mathbf{B=0}$.}
    (a) Charge stability diagram of the device, in which the resonator phase $\varphi$ is measured as a function of the SG voltages $V_R$ and $V_L$.
    The negative slopes of the interdot transitions are due to the strong cross-capacitance of the larger gate $SG_R$. A zoom on the interdot transition pointed out by the red rectangle is shown in (b). (c) Resonator transmission amplitude $\left|S_{21}\right|$ versus probe frequency $
    \omega_\mathrm{p}$ and detuning $\epsilon$ along the white line in (b). At the charge degeneracy point of the DQD, we find a dispersive shift of $\chi=6.9$\,MHz with respect to the bare resonance frequency. The black line is a fit to the data.}
    \label{fig:Fig2}
\end{figure}
Fig.~\ref{fig:Fig2}(a) shows the charge stability diagram of the DQD detected as a shift in the transmission phase $\varphi$ of the resonator, plotted as a function of the two gate voltages $V_L$ and $V_R$ at a fixed probe frequency of $5.253$ GHz, close to resonance.
%i.e detuned from the bare resonace frequency by $80$ kHz.
We observe a slanted honeycomb pattern, in which the inter-dot transition lines exhibit a negative slope due to the specific gate geometry (see Fig.~\ref{fig:Fig1}(a)), which results in the right gate ($V_R$) coupling stronger to both dots than the left ($V_L$).
Using a capacitance model~\cite{van2002electron,scarlino2022situ}, we extract the gate-to-dot capacitances $C_{R2} = 2.5\pm0.4$\,aF, $C_{L2} =1.65\pm0.08$\,aF, $C_{R1}=10.1\pm0.6 $\,aF and $C_{L1} =2.0\pm0.2$\,aF.

In Fig.~\ref{fig:Fig2}(c) we show the resonator response while varying the probe frequency $\omega_p$ and changing the detuning $\epsilon$ along the white line in Fig.~\ref{fig:Fig2}(b).
An electron can now reside on either of the two tunnel-coupled dots thus forming a charge qubit.
At the inter-dot transition (IDT), close to charge degeneracy, the electrical dipole moment of the charge qubit interacts with the resonator, resulting in a dispersive shift of the latter's resonance frequency.
%Cut this part if necessary
By fitting the parameters of a Jaynes-Cummings Hamiltonian (see methods section~\ref{app:JC}) to this particular IDT, we extract the dispersive shift $\chi_0 = 4.6$ MHz, an inter-dot tunnel coupling $t_c = 6.9\pm0.1$ GHz, charge-photon coupling $g_c = 235\pm3$ MHz, and charge qubit linewidth $\gamma = 7.9\pm0.6$~GHz.
\section{Strong spin-photon coupling}
\label{sec:anti-crossing}
\begin{figure}
    \centering
    \includegraphics[width=\linewidth]{Figs/Fig3.pdf}
    \caption{\textbf{Strong spin-photon coupling.} (a) Anti-crossing of the resonator and the qubit found when plotting the resonator transmission amplitude $\left|S_{21}\right|$ as a function of detuning $\epsilon$ and probe frequency $\omega_p$ at a magnetic field $B = 1.67$ T. The dashed white curves are fits to a Jaynes-Cummings model (Eq.~\eqref{eq:JComega} in methods~\ref{app:JC}).
    (b) Dressed resonator linewidth $\delta_{\omega}$ extracted from Fig.~\ref{fig:Fig4}(a) with a fit to the Jaynes-Cummings model (Eq.~\eqref{eq:JClinewidth} in methods~\ref{app:JC}). 
    (c) Anti-crossing of the resonator and the singlet-triplet qubit measured by varying the magnetic field $B$ and the probe frequency $\omega_p$, the constant detuning of $\sim 1.65$\,GHz. %$\epsilon_B/2\pi$ 
    (d) Dressed resonator linewidth $\delta_{\omega}$ extracted from Fig.~\ref{fig:Fig4}(c), analogous to Fig.~\ref{fig:Fig4}(b).
    The red points were excluded from the fit, as they present a superposition of the hybridized qubit-photon state.
    Based on the dilution refrigerator attenuation, we estimate the average number of photons in the resonator during this measurement to be $<0.2$~\cite{palacios2010superconducting,weissl2015kerr}.}
    \label{fig:Fig4}
\end{figure}
When investigating the magnetic-field dependence of IDTs similar to the one shown in  Fig.~\ref{fig:Fig2}(b), we observe two qualitatively different behaviors which we identify as even and odd charge parity configuration as described in the methods section~\ref{app:evenodd}. 
In the following, we investigate one single IDT with an even charge parity.

As illustrated in Fig.~\ref{fig:Fig1}(c), the DQD can be operated as a singlet-triplet qubit when placed into a magnetic field. The qubit frequency $\omega_q=\Delta_{SO}/\hbar$ can be brought into resonance with the cavity frequency $\omega_r$ at $B\approx 1.7$~T, as discussed in more detail below. At the resonance condition ($\omega_q\sim\omega_r$), an anti-symmetric (bonding) and a symmetric (anti-bonding) qubit-photon superposition are formed.
Consistently, at a field of $B \approx 1.7$\,T, we observe an anti-crossing between the resonator and the singlet-triplet qubit. %, as shown in Fig~\ref{fig:Fig4}.
Figure~\ref{fig:Fig4}(a) shows the anti-crossing as a function of the detuning $\epsilon$ at constant magnetic field $B=1.67$\,T, from which we extract the resonance frequency $\omega_{\Psi_\pm}$ and linewidth $\delta_\omega$ by fitting a Lorentzian (amplitude and phase) to each trace of fixed $\epsilon$.
Then, we simultaneously fit the transition frequencies (dashed, white curves in Fig.~\ref{fig:Fig4}(a)) and linewidths (solid, black curve in Fig.~\ref{fig:Fig4}(b)) to the Jaynes-Cummings model as described in the methods section~\ref{app:JC}.
This allows us to extract a spin-photon coupling $g_{\epsilon}/2\pi = 75 \pm 4$ MHz, qubit linewidth $\gamma_\epsilon/2\pi=135\pm30\rm{\,MHz}$ and tunnel rate $t_\epsilon=1.10\pm 0.05$\,GHz for Figs.~\ref{fig:Fig4}(a,b).
We note that the observed anti-crossing occurs at a finite detuning $\epsilon^\prime\sim4.8$\,GHz and hence DQD polarization, which reduces the resonator-qubit coupling strength as $g_\epsilon=g_0\sin\theta$, with $\sin \theta=2t_\epsilon/\sqrt{(2t)^2+{\epsilon^\prime}^2}$, where $\theta$ is the mixing angle~\cite{childress2004mesoscopic,petersson2012circuit,frey2012dipole}.
After correcting for this mixing angle we obtain a spin-photon coupling strength $g_0/2\pi=178\pm6\rm{MHz}$.

In Fig.~\ref{fig:Fig4}(c), we show the same anti-crossing as a function of $B$ at a fixed detuning of $\sim 1.65$\,GHz.
To extract the spin-photon coupling strength and qubit linewidth from this second measurement, we characterize the effective qubit transition frequency around the minimum $t_0=t(B_0)$ by $\omega_q(B)=\sqrt{(2t_0)^2+(\alpha_B(B-B_0))^2}$, where we introduce the heuristic scaling factor $\alpha_B$ .
With this additional free parameter, we fit the Jaynes-Cummings model (dashed, white curves in Fig. 3(c) and solid,
black curve in Fig. 3(d)) and extract a spin-photon coupling strength of $g_B/2\pi = 114 \pm 9$ MHz and linewidth $\gamma_B/2\pi=190\pm24\rm{\,MHz}$ for Fig.~\ref{fig:Fig4}(c,d).
The larger value of the coupling strength in the magnetic-field sweep compared to the detuning sweep is attributed to the smaller mixing angle and reflected by the larger splitting at the anti-crossing in Figure~\ref{fig:Fig4}(c) compared to Figure~\ref{fig:Fig4}(a). 

These extracted coupling strengths demonstrate that the strong coupling limit between a superconducting resonator and a singlet-triplet qubit can be reached: In the strong coupling regime, a single photon coherently hybridizes with a two-level system. This limit is reached if the vacuum Rabi $2g$ splitting exceeds the linewidth $\gamma+\kappa/2$ of the hybridized bonding and anti-bonding states~\cite{blais2021circuit}.
For our device we find a ratio $2g_\epsilon/(\gamma_\epsilon+\kappa/2)=1.0\pm0.2$ for the detuning sweep (Fig~\ref{fig:Fig4}(a,b)) and $2g_B/(\gamma_B+\kappa/2)=1.1\pm0.2$ for the magnetic field sweep (Fig.~\ref{fig:Fig4}(c,d)) reaching the strong coupling regime in both cases. % which is the main finding of this manuscript.
After accounting for the mixing angle and using the larger extracted linewidth $\gamma\equiv\max (\gamma_B,\gamma_\epsilon)=\gamma_B$, we find $2g_{0}/(\gamma+\kappa/2)=1.7 \pm0.2$, now very clearly in the strong coupling regime.
\section{Magnetospectroscopy}
\begin{figure}
    \centering
    \includegraphics[width=\linewidth]{Figs/Fig4.pdf}
    \caption{\textbf{Magnetospectroscopy of the singlet-triplet qubit.} a) Dispersive shift $\chi$ as a function of the magnetic field $B$ and detuning $\epsilon$. The white dashed line is a fit of the effective two-electron Hamiltonian (Eq.~\eqref{app:Hamiltonian}) to the data. b) Extracted tunnel rate $2t/2\pi$ (black), qubit-photon coupling times hundred $100 \cdot g/2\pi$ (blue) and qubit linewidth $\gamma/2\pi$ (red). The bare resonator frequency is indicated by the dashed black line. (c) Two-electron energy level diagrams at various magnetic fields with the corresponding field strength indicated in (a) and (b) by the given symbols. A constant offset of $20$\,GHz and $30$\,GHz was subtracted from the energy levels at 1.65\,T and 2.0\,T, respectively. 
    We estimate the average photon number in these experiments as $\sim 1$~\cite{palacios2010superconducting,weissl2015kerr}.}
    \label{fig:Fig3}
\end{figure}
\label{sec:magnetospectroscopy}
To further characterize the spin-orbit eigenstates and and to independently verify the character of the singlet-triplet qubit, we now study the magnetic field evolution of the IDT from 0\,T up to 2\,T.
%We measure the complex transmission $S_{2,1}$ through the resonator as a function of magnetic field and DQD detuning. 
As described in the method section~\ref{app:resonator}, we measure the resonator shift $\chi$ as a function of the detuning $\epsilon$ and the magnetic field, as plotted in Fig.~\ref{fig:Fig3}(a).
A finite $\chi\neq 0$ occurs at the IDT when tunneling between the dots is allowed and hence if the DQD obtains a non-zero dipole moment.
As described in the methods section~\ref{app:Hamiltonian}, we model the DQD by an effective two electron Hamiltonian which allows us to fit the gate voltage and field dependence of the IDT (white dashed line in Fig.~\ref{fig:Fig2_app}(a)).
We find that the magneto-dispersion of the IDT is well fitted using the following fit parameters namely the spin-conserving singlet and triplet tunnel rates $t_c^S/2\pi\approx29$\,GHz, and
$t_c^T/2\pi\approx37$\,GHz, the singlet-triplet coupling rate
$t_{\rm{SO}}/2\pi\approx 5$\,GHz, the electron g-factors of the right and left dots, $g_R\approx 1.8$ and $ g_L\approx 2.8$, as well as the singlet-triplet energy splitting $\Delta_{\rm{ST}}/2\pi\approx 79$\,GHz.
These fit parameters are consistent with parameters obtained previously in this material system~\cite{fasth2007direct,nadj2010disentangling,nilsson2018tuning,trif2008spin,junger2020magnetic}.
We note, however, that the fit is under-determined and therefore, it does not provide accurate numbers.
Nonetheless, the model gives a qualitative, physical understanding of the system and allows us to establish which DQD levels interact with the resonator.

We can gain additional independent information on the system by also using the other IDT characteristics. Especially, the resonator provides an absolute energy scale allowing for a quantitative analysis of the interaction between the DQD and the resonator~\cite{borjans2021probing} complementing the preceding DQD Hamiltonian fit. As described in the methods section~\ref{app:JC}, by fitting a Jaynes-Cummings model to both the resonator shift $\chi$ and the dressed resonator linewidth $\delta_\omega$ simultaneously, we extract the resonator decay $\kappa$ as well as the qubit tunnel amplitude $t$, the qubit linewidth $\gamma$, and the qubit-photon coupling strength $g$ as a function of $B$, which we plot in Figure~\ref{fig:Fig3}(b).

Using the fits to both the 2-electron Hamiltonian model and the Jaynes-Cummings model in the 2-level approximation, allows to directly identify several regimes, in each of which the qubit has a different spin-character.
Fig.~\ref{fig:Fig3}(c) shows the corresponding DQD level structure based on the fit parameters as a function of $\epsilon$ for different magnetic field.

At a low magnetic fields around $B=0.1$\,T, the triplet states (blue curves) are Zeeman split and the ground-state curvature is dominated by the anti-crossing between $S_{1,1}$ and $S_{2,0}$ (red curves).
We find a singlet charge qubit in the weak coupling limit, i.e. for which the linewidth exceeds the charge-photon coupling by a factor of hundred.
The formation of a double-dip structure in $\chi (\epsilon)$ between $B\sim 0.03$\,T and $B\sim0.3$\,T is explained by an interaction between the three states $S_{2,0}$, $S_{1,1}$ and $T_{1,1}^+$ as described in the supplementary material.
Traces of $\chi(\epsilon)$ with a double-dip structure cannot be described by the Jaynes-Cummings Hamiltonian and are therefore not analysed quantitatively.

As $B$ is increased, the triplet state $T_{1,1}^+$ becomes the ground state for $\epsilon<0$ as shown in the second panel of Fig.~\ref{fig:Fig3}(c) for $B=0.7$\,T.
The spin-orbit interaction couples the singlet and triplet states, leading to an anti-crossing between $S_{2,0}$ and $T_{1,1}^+$.
A singlet-triplet qubit is created with $\omega_q=\Delta_\mathrm{SO}=2t_\mathrm{SO}$~\cite{mutter2021all,jirovec2021singlet}.
In this regime, at larger $B$, the resonance condition between $S_{2,0}$ and $T_{1,1}^+$ occurs at larger $\epsilon$, because the energy of the bare $T_{1,1}^+$ state decreases with larger $B$ and the energy of $S_{2,0}$ decreases with larger $\epsilon$.
Therefore, the IDT is observed at larger $\epsilon$ for increasing $B$.

Consistent with the interpretation of the formation of a singlet-triplet qubit, we measure an approximately constant tunneling rate $t$ between $B\sim0.5$\,T and $B\sim1.1$\,T.
In this regime, we extract the average spin-orbit tunneling to be $\bar{t}_{\rm{so}}=4.0\pm 0.3$\,GHz. % where the error bar is the root variance.
At $B\approx 1.3$\,T, $\chi$ becomes positive. This is interpreted as a drop of the tunnel rate below the resonator frequency, $2t<\omega_r^0$.
This decline in $t$ is not captured by our simplified Hamiltonian model and we speculate that changes in the orbital structure of a many-electron DQD could be the reason.

At a magnetic field of $B\approx 1.7$\,T, the resonator shift $\chi$ becomes positive again and we observe a resonant interaction between the resonator and the singlet-triplet qubit leading to the anti-crossing as discussed in section \ref{sec:anti-crossing}.
As seen in the level structure in Fig.~\ref{fig:Fig3}(c) at $B=1.65$\,T, because the IDT happens at elevated magnetic-field strength and detuning, the triplet state $T_{2,0}^+$ becomes relevant.
This results in a level repulsion between $T_{2,0}^+$ and $T_{1,1}^+$ and hence leads to a reduced splitting between the $S_{2,0}$ level and the $T_{1,1}^+$ states.
In Fig.~\ref{fig:Fig3}(c), this is illustrated by the smaller level gap (black arrow) compared to the one at $B=0.7$\,T.
 
The level structure at very large magnetic fields is plotted exemplarity for $B\approx2$\,T in the right panel of Fig.~\ref{fig:Fig3}(c).
In this regime, the ground-state of the DQD at the IDT is formed by a superposition of the $T_{2,0}^+$ and the $T_{1,1}^+$ states.
%, forming a charge qubit with triplet spin character.
Comparing this very-large magnetic-field regime with the lower field regimes, we find that the curve of Fig.~\ref{fig:Fig3}(a) turns back towards lower $\epsilon$ for increasing $B$. 
This situation can be understood easily by noting that the spin-polarized triplets $T_{2,0}^+$ and $T_{1,1}^+$ form a charge qubit similar to the singlets at low field.
While the transition is increasingly dominated by the triplet-charge qubit for increasing $B$, $\left|\chi\right|$ becomes smaller at the IDT, because the anti-crossing between the triplet states $T_{2,0}^-$ and $T_{1,1}^-$ happens at much larger frequencies $2t_c^T\gg2t_\mathrm{SO}$.
Hence, the triplet charge qubit has a larger frequency detuning from the resonator frequency than the singlet-triplet qubit, leading to a smaller resonator shift.

This large number of detailed findings justify the use and the parameters of the two-electron Hamiltonian introduced above, which, in turn, directly allows us to identify the singlet-triplet spin qubit, for which we find the strong coupling limit to the electromagnetic cavity. We note that the qubit linewidth $\gamma$ and qubit-photon coupling strength are both related to the qubit rate as discussed in the supplementary material.
%To confirm their linear relation, we plot $\gamma$ and $g$ against $t$ in Figure~\ref{fig:A2} in the appendix. 
An intuitive explanation is that the tunnel rate in our experiments increases as the qubit becomes more charge like and hence is more susceptible to charge noise.
Another possible explanation is that the qubit linewidth is limited by qubit relaxation which scales proportional to the tunnel rate to the contacts.

%%%%%%%%%%%%%%%
%%%%%%%%%%%%%%%
%%%%SUMMARY%%%%
%%%%%%%%%%%%%%%
%%%%%%%%%%%%%%%
\section{Conclusion and Outlook}
In summary, we demonstrate a semiconductor nanowire DQD device with crystal-phase defined tunnel barriers that can be operated as different types of qubits, coupled to a high-impedance, high magnetic field resilient electromagnetic resonator. As the main result, we find
a singlet-triplet qubit for which we extract the relevant qubit parameters, especially a record high electron spin-photon coupling of $g/2\pi=114$\,MHz in the single photon limit, thus reaching the strong coupling regime $2g\geq\gamma+\kappa/2$.

Our experiments demonstrate that deterministically grown tunnel barriers allow for a reduced number of gate lines, and that, mediated by intrinsic spin-orbit interaction, singlet-triplet qubits can reach the strong coupling limit for low photon numbers, similar to flopping mode spin qubits~\cite{yu2023strong,burkard2021semiconductor}. This finding is potentially applicable to other promising platforms with strong spin-orbit interactions, like holes in Ge~\cite{jirovec2021singlet}. Technologically, the large electron spin-photon coupling found in our experiments might become crucial for the implementation of two-qubit gates between distant spin qubits a milestone on the way towards scalable quantum computers.
Moreover, our nanowire platform without depletion gates results in a significantly reduced gate-induced photon-leakage in the absence of on-chip filtering~\cite{petersson2012circuit,mi2017circuit,harvey2020chip}.
An optimized gate design with resonators with larger impedance~\cite{scarlino2022situ} therefore presents an ideal platform to investigate new phenomena in the ultrastrong coupling regime~\cite{forn2019ultrastrong,scarlino2022situ}.

We acknowledge fruitful discussions with Simon Zihlmann, Roy Haller, Andrea Hofmann, Stefano Bosco, Romain Maurand and Antti Ranni and support in setting-up the experiments by Fabian Opliger, Roy Haller, Luk Yi Cheung, and Deepankar Sarmah.
This research was supported by the Swiss Nanoscience Institute (SNI), the Swiss National Science Foundation through grant 192027, the NCCR Quantum Science and Technology (NCCR-QSIT) and the NCCR Spin Qubit in Silicon (NCCR-Spin). We further acknowledge funding from the European Union’s Horizon 2020 research and innovation programme, specifically the FET-open project AndQC, agreement No 828948 and the FET-open project TOPSQUAD, agreement No 847471. Furthermore, we acknowledge funding by NanoLund and the Knut \& Alice Wallenberg Foundation (KAW). All data in this publication are available in
numerical form at: \url{https://doi.org/10.5281/zenodo.7777840}.
\section{Methods}
\subsection{Resonator characterization and analysis}
\label{app:resonator}
The resonator is fabricated from a thin-film NbTiN, sputtered onto a Si/SiO\textsubscript{2} (1000\,\textmu m/100\,nm) substrate~\cite{ungerer2023performance}.
The large sheet kinetic inductance of the NbTiN film of $L_\mathrm{sq}\approx 90$\,pH together with the narrow center conductor width of $\sim380$\,nm, and the large distance to the ground plane of $\sim 35$\,\textmu m results in the impedance value of $2.1$ k$\Omega$.
The resonator can be dc biased using a bias line which contains a meandered inductor ensuring sufficient frequency detuning between the half-wave resonance used in the experiment and a second, low quality resonance mode at a lower frequency that forms due to the finite inductance of the bias line~\cite{harvey2020chip}.
An optical microscopy image of a similar resonator is shown in Fig.\ref{fig:FigA1}(a) in the extended data.
One of the two resonator voltage anti-nodes is galvanically connected to gate SG$_R$ shown in Fig.~\ref{fig:Fig1}(c) of the main text.

We measure the transmission $S_{21}(\omega)$ of the resonator, to which we simultaneously fit the amplitude and phase of a Lorentzian:
\begin{align}
    \left|S_{21}\right|(\omega)&=a_0\cdot\frac{\frac{\delta_\omega}{2}}{((\omega-\omega_r)^2+\frac{\delta_\omega^2}{4})},\label{eq:amp}\\
    \varphi(\omega)&=-\arctan\left(\frac{\omega^2 - \omega_r^2}{\frac{\delta_\omega}{2} \omega}\right)+\varphi_0.
    \label{eq:phase}
\end{align}
From the fit (see Fig.~\ref{fig:FigA1}(b) in the extended data), we extract the resonator decay rate $\delta_\omega/2\pi$ and resonance frequency $\omega_r/2\pi$ which we identify as the bare resonator decay rate $\kappa/2\pi=\delta_\omega/2\pi=34.4\pm0.1$\,MHz and resonance frequency $\omega_r/2\pi=5.25308\pm0.00003$\,GHz, respectively.
In addition, we extract the transmission amplitude $a_0$ and phase offset $\varphi_0$ to calibrate our system. Gate voltages are converted to DQD detuning by using the lever arm $\alpha \approx 0.21$ eV/V $\approx 51$ THz/V extracted from the Jaynes-Cummings fit to the anti-crossing shown in Fig.~\ref{fig:Fig4}(a).
This calibration allows us to convert a measured complex transmission $S_{21} (\omega)$ into a decay rate $\delta_\omega$ and the frequency shift $\chi=\omega_r-\omega_r^0$ by numerically solving Eqs.~\eqref{eq:amp} and~\eqref{eq:phase} which results in Fig.~\ref{fig:Fig3}a) in the main text and Fig.~\ref{fig:Fig_app_shift_linewidth}(a,b) in the extended data.
\subsection{Charge parity determination\label{app:evenodd}}
We measure the phase $\varphi$ of the resonator transmission $S_{21}(\omega)$ as a function of detuning $\epsilon$ and magnetic field $B$ at a readout-frequency $\omega_{\rm{ro}}/2\pi=5.253$\,GHz close to the bare resonator frequency.
A change in $\varphi$ reflects the dispersive interaction between the resonator and two anticrossing levels of the DQD~\cite{frey2012dipole,crippa2019gate}.
Therefore, the non-zero phase response of the resonator tracks the position of the IDT along the detuning axis. 
The comparison of the magnetic field dependence of the position of the IDT to a Hamiltonian model of the DQD systems allows one to determine the charge parity~\cite{crippa2019gate,ezzouch2021dispersively}.
Figures \ref{fig:Fig2_app} (a) and (b) in the extended data show two typical low field IDT characteristics.

For an odd number of electrons (Fig.~\ref{fig:Fig2_app}(b)), the DQD resonance gate voltage $V_R$, at which the IDT is observed, disperses linearly with magnetic field starting from zero. 
This can be understood considering the Zeeman-splitting of the unpaired electron energy levels and two non-equal Landé g-factors of the two dots.
Fig.~\ref{fig:Fig2_app}(c) shows the energy level diagram of a one-electron Hamiltonian including Zeeman-splitting with a g-factor difference of 1.0 and spin-orbit interaction $t_{SO}/2\pi = 5$~GHz at a magnetic field of $B=0.5$~T (green, dashed line in Fig.~\ref{fig:Fig2_app}(b). The one-electron Hamiltonian is explicitly discussed in the supplementary material.
The arrow points out the center of the IDT (largest curvature of the groundstate~\cite{park2020adiabatic}) which corresponds to the largest dipole moment of the DQD and thus to the largest change in $\varphi$.
This point shifts with B towards increasingly negative values.

For an even number of electrons in the DQD at zero field (Fig.~\ref{fig:Fig2_app}(a), a single dip in phase is observed, but at a low magnetic fields, $B\approx60$\,mT, a double dip structure emerges as a function of $\epsilon$ (see supplementary material for details).
This double-dip originates from an interaction between $S_{2,0}$, $S_{1,1}$ and $T_{1,1}^+$ as explained in detail in the supplementary material.
The dependence of the IDT on magnetic field for an even number of electrons can be understood using an effective two electron Hamiltonian including spin-orbit interaction as described in more detail in section~\ref{app:Hamiltonian}.
In Fig.~\ref{fig:Fig2_app}(c), we plot the energy levels at a magnetic field $B = 0.15$~T.
In contrast to the odd filling, starting at zero field, the arrow marking the center of the IDT barely changes, consistent with our measurement. 
The double dip vanishes when further increasing the magnetic field, because of an increasing occupation of the polarized triplet states.
Once the Zeeman energy of the triplet state $\ket{T_{1,1}^+}$ becomes comparable to the singlet charge tunneling $t_{\rm{c}}^{\rm{S}}$, the position of the IDT as a function of $B$ disperses towards larger $\epsilon$~\cite{schroer2012radio,malinowski2018spin,ezzouch2021dispersively}.
This transition is marked by the white dashed line at $0.2$~T in~\ref{fig:Fig2_app}(a).

Based on the good qualitative agreement between our data and the one electron and two electron Hamiltonian, respectively, we can clearly identify the even and odd charge parities.
\subsection{Jaynes-Cummings model}
\label{app:JC}
In the regime of only two DQD levels being relevant, we model the DQD Hamiltonian as an effective two-level system (qubit) interacting with a single photon in the resonator.
The combined system is described by the Jaynes-Cummings model~\cite{shore1993jaynes}.
In which, the one-photon excitation energy from the ground state is given by~\cite{blais2004cavity}
\begin{equation}
\omega_{\psi_\pm}=\omega_r^0\pm\frac{1}{2}\sqrt{4g^2+(\omega_r^0-\omega_q)^2}\text{,}\label{eq:JComega}
\end{equation}
with the qubit frequency $\omega_q=\sqrt{\left(2t(B)\right)^2+\epsilon^2}$~\cite{vanderwiel2002} and the effective qubit-photon coupling strength $g=g_0\cdot2t/\omega_q$, where $g_0$ is the bare qubit-photon coupling accounting for the mixing angle~\cite{blais2004cavity,stockklauser2017strong}.
In the experiments, we detect the transitions from the ground state to the predominantly photon-like dressed state $\ket{\psi_-}$.
Its linewidth is given by
\begin{align}
\delta_\omega&=\left|\langle \psi_- |g,1\rangle \right|^2\kappa+\left|\langle\psi_-|e,0\rangle\right|^22\gamma\\
&=\cos^2\left(\theta\right)\kappa+\sin^2 \left(\theta\right) 2\gamma,\label{eq:JClinewidth}
\end{align}
where $\theta=\frac{1}{2}\tan^{-1}\left(\frac{2g}{\omega_q-\omega_r^0 }\right)$~\cite{blais2004cavity}.

In order to extract the qubit tunneling rate $t$, and linewidth $\gamma$ as well as qubit-photon coupling strength $g$, we simultaneously fit $\chi(\epsilon)$ and $\delta_\omega(\epsilon)$ using Eq.~\eqref{eq:JComega} and Eq.~\eqref{eq:JClinewidth}. An exemplary fit is shown in Fig.~\ref{fig:Fig2_app}(e,f) in the extended data.
\subsection{Effective two-electron Hamiltonian model}
\label{app:Hamiltonian}
We model an effective two-electron Hamiltonian in the presence of spin-orbit interaction and magnetic field.
We write the Hamiltonian in the basis of singlet and triplet states $\left\{\ket{S_{1,1}},\ket{S_{2,0}},\ket{T_{1,1}^{\pm,0}},\ket{T_{2,0}^{\pm,0}}\right\}$, with the subscripts indicating the charge distribution in the DQD. The Hamiltonian reads
\begin{equation}
\mathcal{H}=\mathcal{H}_0^S+\mathcal{H}_0^T+\mathcal{H}_Z+\mathcal{H}_{\mathrm{so}},
\end{equation}
with the spin quantum-number conserving Hamiltonians
{\footnotesize
\begin{align*}
\mathcal{H}_0^S/\hbar&=-\epsilon\ket{S_{2,0}}\bra{S_{2,0}}+t_c^S\ket{S_{1,1}}\bra{S_{2,0}}+\text{h.c.}\text{,}\\
\mathcal{H}_0^T/\hbar&=(\Delta_{\rm{ST}}-\epsilon) \sum_{\pm,0}\ket{T_{2,0}^{\pm,0}}\bra{T_{2,0}^{\pm,0}}+t_c^T\sum_{\pm,0}\ket{T_{1,1}^{\pm,0}}\bra{T_{2,0}^{\pm,0}}+\text{h.c.}
\end{align*}
}%
\normalsize
Here, $t_c^{S,T}$ are the tunnel rates between the two singlets, and between the two triplet states respectively, and $\Delta_\mathrm{ST}$ is the single dot singlet-triplet splitting that separates the $T_{2,0}$ states from the $S_{2,0}$ states.
The Zeeman Hamiltonian is given by
\begin{equation}
\mathcal{H}_Z/\mu_B =B\sum_\pm\left(\pm \frac{g_l+ g_r}{2}\ket{T_{1,1}^\pm}\bra{T_{1,1}^\pm}\pm g_l\ket{T_{2,0}^\pm}\bra{T_{2,0}^\pm}\right),
\end{equation}
%\normalsize
where $g_l$ ($g_r$) is the Landé g-factor of the left (right) dot. Because of the large intrinsic spin-orbit interaction in the NW, we include the spin-orbit Hamiltonian that couples the singlet and triplet states with opposite charge configuration using the spin-orbit tunnel rate $t_{\rm{SO}}$ as
%\footnotesize
\begin{equation*}
\mathcal{H}_\mathrm{SO}/\hbar=t_\mathrm{SO}\left(\ket{T^0_{1,1}}\bra{S_{2,0}}+\sum_{\pm}\pm \ket{T_{1,1}^\pm}\bra{S_{2,0}}\right)+\text{h.c.}
\end{equation*}
%\appendix
\section*{Bibliography}
\bibliography{ref}
\clearpage
\section{Extended data}
\begin{figure}[htbp!]
    \centering
    \includegraphics[width=\linewidth]{Figs/FigA1.pdf}
    \caption{\textbf{Resonator of a similar device.}
    a) Optical microscope image of a similar device including the resonator. b) Resonance curve of the resonator in phase (blue) and magnitude (red), as well as a fit to the data (black). c) Same false-colored SEM image of the device as in Fig.~\ref{fig:Fig1}(a). Scanning electron micrograph of the resonator center conductor of a similiar device}
    \label{fig:FigA1}
\end{figure}
\begin{figure}[htbp!]
    \centering
    \includegraphics[width=\linewidth]{Figs/FigA2.pdf}
    \caption{\textbf{Dispersive shift and dressed linewidth extraction.}
    Extracted dispersive shift $\chi$ (a) and dressed resonator linewidth $\delta_\omega$ (b) as a function of detuning $\epsilon$ of the even configuration at $0.448$ T.}
    \label{fig:Fig_app_shift_linewidth}
\end{figure}
\begin{figure}[htbp!]
    \centering
    \includegraphics[width=\linewidth]{Figs/FigA3.pdf}
    \caption{\textbf{Dispersive read-out at low magnetic field.}
    Resonator phase in dependence of the right gate voltage $V_R$ and magnetic field $B$ for even (a) and odd (b) occupation of the DQD. For the odd occupation the IDT shifts to lower $V_R$ from $B=0$. The IDT of the even occupation stays nearly independent of magnetic field until around $0.2$~T (blue dashed line), from where it starts moving to more positive $V_R$. Energy level diagram for the odd (c) and even (d) configuration at $0.5$~T  and $0.15$~T (green dashed line). The arrow marks the transition the resonator is sensitive to, where the ground state energy level has maximum curvature.}
    \label{fig:Fig2_app}
\end{figure}
\clearpage
\end{document}