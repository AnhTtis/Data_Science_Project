\documentclass[%
reprint,
superscriptaddress,
%groupedaddress,
%unsortedaddress,
%runinaddress,
%frontmatterverbose,
%preprint,
%preprintnumbers,
%nofootinbib,
%nobibnotes,
%bibnotes,
amsmath,
amssymb,
aps,
%pra,
prx,
%prb,
%rmp,
%prstab,
%prstper,
floatfix,
]{revtex4-2}
\usepackage{graphicx}% Include figure files
\usepackage{dcolumn}% Align table columns on decimal point
\usepackage{bm}% bold math
\usepackage{hyperref}% add hypertext capabilities
\usepackage{xr}
%\usepackage{color}
%\externaldocument{Al_graphene_SM}
\usepackage{lipsum}
\usepackage{braket}
%for including svg graphics
%\usepackage{graphicx}
\usepackage{xcolor}
%\usepackage{lineno}
%\usepackage{breqn}%linebreak in equations
%\linenumbers
\newcommand{\corr}[1]{\textcolor{black}{#1}}

\begin{document}
\preprint{APS/123-QED}
\bibliographystyle{unsrtnat} 
%=======================================================================================================================================
\title{Strong coupling between a microwave photon and a singlet-triplet qubit}
%%%%%
\author{J.\,H.~Ungerer}
\altaffiliation{Equal contributions.}
%\email{jannhinnerk.ungerer@unibas.ch}
\affiliation{
Department of Physics, University of Basel, Klingelbergstrasse 82 CH-4056, Switzerland
}
\affiliation{
Swiss Nanoscience Institute, University of Basel, Klingelbergstrasse 82 CH-4056, Switzerland
}
\author{A.~Pally\footnotemark[1]}
\altaffiliation{Equal contributions.}
\affiliation{
Department of Physics, University of Basel, Klingelbergstrasse 82 CH-4056, Switzerland
}
\author{A.~Kononov}
\affiliation{
Department of Physics, University of Basel, Klingelbergstrasse 82 CH-4056, Switzerland
}

\author{S.~Lehmann}
\affiliation{Solid State Physics and NanoLund, Lund University, Box 118, S-22100 Lund, Sweden}

\author{J.~Ridderbos}
\altaffiliation{Current address: MESA Institute for Nanotechnology, University of Twente, P.O. Box 217, 7500 AE Enschede, The
Netherlands}
\affiliation{
Department of Physics, University of Basel, Klingelbergstrasse 82 CH-4056, Switzerland
}

\author{P.\,P. Potts}
\affiliation{
Department of Physics, University of Basel, Klingelbergstrasse 82 CH-4056, Switzerland
}

\author{C.~Thelander}
\affiliation{Solid State Physics and NanoLund, Lund University, Box 118, S-22100 Lund, Sweden}
\author{K.A.~Dick}
\affiliation{Centre for Analysis and Synthesis, Lund University, Box 124, S-22100 Lund, Sweden}
\author{V.F.~Maisi}
\affiliation{Solid State Physics and NanoLund, Lund University, Box 118, S-22100 Lund, Sweden}
\author{P.~Scarlino}
\affiliation{Institute of Physics and Center for Quantum Science and Engineering, Ecole Polytechnique Fédérale de Lausanne, CH-1015 Lausanne, Switzerland}
\author{A.~Baumgartner}
\homepage{www.nanoelectronics.unibas.ch}
\affiliation{
Department of Physics, University of Basel, Klingelbergstrasse 82 CH-4056, Switzerland
}
\affiliation{
Swiss Nanoscience Institute, University of Basel, Klingelbergstrasse 82 CH-4056, Switzerland
}
\author{C.~Sch{\"o}nenberger}
\homepage{www.nanoelectronics.unibas.ch}
\affiliation{
Department of Physics, University of Basel, Klingelbergstrasse 82 CH-4056, Switzerland
}
\affiliation{
Swiss Nanoscience Institute, University of Basel, Klingelbergstrasse 82 CH-4056, Switzerland
}
\date{\today}
%=======================================================================================================================================
% ABSTRACT
%=======================================================================================================================================

\begin{abstract}
Tremendous progress in few-qubit quantum processing has been achieved lately using superconducting resonators coupled to gate voltage defined quantum dots. While the strong coupling regime has been demonstrated recently for odd charge parity flopping mode spin qubits, first attempts towards coupling a resonator to even charge parity singlet-triplet spin qubits have resulted only in weak spin-photon coupling strengths. Here, we integrate a zincblende InAs nanowire double quantum dot with strong spin-orbit interaction in a magnetic-field resilient, high-quality resonator. In contrast to conventional strategies, the quantum confinement is achieved using deterministically grown wurtzite tunnel barriers without resorting to electrical gating. Our experiments on even charge parity states and at large magnetic fields, allow to identify the relevant spin states and to measure the spin decoherence rates and spin-photon coupling strengths. Most importantly, we find an anti-crossing between the resonator mode in the single photon limit and a singlet-triplet qubit with an electron spin-photon coupling strength of \corr{$g/2\pi = 139 \pm 4$ MHz. Combined with the resonator decay rate $\kappa/2\pi=19.8\pm0.2$\,MHz and the qubit dephasing rate $\gamma/2\pi=116\pm7$\,MHz, our system achieves the strong coupling regime in which the coherent coupling exceeds qubit and resonator linewidth. These results pave the way towards large-scale quantum system based on singlet-triplet qubits.}
\end{abstract}
\maketitle
%%%% INTRO %%%%
%\section{Introduction}
Spin qubits in semiconductors are promising candidates for scalable quantum information processing due to long coherence times and fast manipulation~\cite{hanson2007spins,zwanenburg2013silicon,vandersypen2017interfacing,chatterjee2021semiconductor}.
For the qubit readout, circuit quantum electrodynamics based on superconducting resonators \cite{childress2004mesoscopic}, allows a direct and fast measurement of qubit states and their dynamics~\cite{petersson2012circuit}.
Recently, resonators were used to achieve charge-photon~\cite{stockklauser2017strong,mi2017strong}, spin-photon~\cite{mi2018coherent,samkharadze2018strong,landig2018coherent} as well as coherent coupling of distant charge~\cite{van2018microwave} and spin qubits~\cite{borjans2020resonant,harvey2022coherent}, enabling coherent information exchange between distant qubits.
However, the small electric and magnetic moments of individual electrons require complicated device architectures such as micromagnets, and a large number of surface gates that render scaling up to more complex architectures challenging. These approaches typically achieve a relatively weak electron spin-photon coupling on the order of $\sim 10 - 30$\,MHz.
In addition to single electron spin qubits, also spin qubits based on two electrons in a double quantum dot (DQD), e.g. in a singlet-triplet qubit have been demonstrated~\cite{petta2005coherent}.
Spin qubits based on two electrons typically offer a large hybridization of the spin and charge degree of freedom compared to single-electron spin qubits in principle allowing even stronger coupling strengths.
So far, however, the experimentally achieved coupling strengths in such systems~\cite{landig2019microwave,bottcher2022parametric} remained well below the strong coupling limit in which the coherent coupling rate exceeds both, the cavity mode decay rate and the qubit linewidth.

Here, we demonstrate that the strong coupling regime between a singlet-triplet qubit and a \corr{single photon in a} superconducting resonator can be reached.
We achieve this strong coupling by carefully designing the resonator and by using a DQD defined by in-situ grown tunnel barriers in a semiconductor with a large spin-orbit interaction.
The tunnel barriers consist of InAs segments in the wurtzite crystal-phase with an atomically sharp interface to the zincblende bulk of the nanowire (NW)~\cite{lehmann2013general}.
\corr{These crystal-phase barriers are highly reproducible and render the need of barrier gates obsolete, simplifying integration with superconducting resonators and making the nanowires a viable prototype for scalable quantum computing architectures.}

In this work, we make use of the large spin-orbit interaction in these nanowires~\cite{nilsson2018tuning} to define a singlet-triplet qubit at a finite \corr{in-plane} magnetic field in which the $T_{1,1}^+$ and $S_{2,0}$ states hybridize, forming a quantum two-level system.
Incorporating \corr{a NW with} a magnetic-field resilient resonator based on NbTiN~\cite{samkharadze2016high,ungerer2023performance} allows us to measure an avoided crossing between the singlet-triplet qubit and a single-photon excitation of the resonator at a \corr{ magnetic-field strength of $B=300$\,mT.
The measured coupling strength is very large compared to previously reported electron spin-photon coupling~\cite{mi2018coherent,samkharadze2018strong,landig2018coherent}, which enables us to reach the strong coupling regime.}
\corr{In addition, by analyzing the response of the hybridized resonator-qubit system for varying magnetic-field strengths, we perform qubit spectroscopy~\cite{borjans2021probing,mi2017high,ibberson2021large}.
This allows us to identify the specific spin states and to quantitatively extract the relevant device properties.}
\section{Device characterization}
\begin{figure}
    \centering
    \includegraphics[width=\linewidth]{Figs/Fig1.pdf}
    \caption{\textbf{Coupled resonator-qubit system} (a) False colored SEM-image of \corr{device A}. The NW (green) is divided into two segments by an in-situ grown tunnel barrier (red), thus forming the DQD system. The NW ends are contacted by two Ti/Au contacts (S,D) and the NW segements can be electrically tuned by two Ti/Au sidegates SG$_R$ (purple) and SG$_L$ (yellow).
    A high-impedance, half-wave resonator is connected to SG$_R$. Top gates (orange) are kept at a constant voltage of \corr{$-0.28$~V}. The magnetic field is applied in-plane at \corr{an angle $\alpha$ with respect to the NW axis, as illustrated by the grey arrow.} The white arrows illustrate an even charge configuration with the two degenerate DQD states $T_{1,1}^+$ and $S_{2,0}$. (b) Schematic of the crystal-phase defined DQD. The conduction band of wurzite and zincblende are offset by $\sim 100$ meV, resulting in a tunnel barrier between the zincblende segments. The intrinsic spin-orbit interaction enables spin-rotating tunneling between these segments. (c) Energy levels of an even charge configuration as a function of magnetic field $B$ at a fixed positive detuning $\varepsilon$ between the dot levels. %of $\varepsilon=2/2\pi$ GHz.
    At finite magnetic fields, $T_{1,1}^+$ (blue) hybridizes with $S_{2,0}$ (red) defining a singlet-triplet qubit with an energy splitting given by the spin-orbit interaction strength $\Delta_\mathrm{SO}$.}
    \label{fig:Fig1}
\end{figure}
\corr{Details about the NW properties and their growth can be found in the supplementary. The resonator-qubit system  of device A is shown in Fig.~\ref{fig:Fig1}(a), including a false-colored SEM-image of the crystal-phase defined NW DQD. We report similar experiments for two devices, A and B, with B discussed in the SI material. They are measured in a dilution refrigerator with a base temperature of $70$~mK.} The DQD forms in the \corr{490}~nm and \corr{370}~nm long zincblende segments (green), separated by $30$\,nm long wurtzite (red) tunnel barriers with a conduction band offset of $\sim$100\,meV~\cite{nilsson2016single}, as illustrated in Fig.~\ref{fig:Fig1}(b).
A high-impedance, half-wave coplanar-waveguide resonator is capacitively coupled to the DQD at its voltage anti-node via a sidegate.
In addition, the same sidegate can be used to tune the DQD charge states using a dc voltage ($V_R$) applied at the resonator voltage node. The DQD state is probed by reading out the resonator rf-transmission.
We extract the bare resonance frequency of the resonator \corr{$\omega_0/2\pi=5.1705\pm0.0003$\,GHz} at zero magnetic field and the bare decay rate \corr{$\kappa|_{B=0}/2\pi=27.3\pm0.6$ MHz.} %\kappa/2\pi=19.8\pm0.6$ at 300 mT
The resonator design and fitting are described in detail in methods section~\ref{app:resonator} and \ref{app:IO}. %\corr{Device B is shown and discussed in the supplementary material. In the following we focus exclusively on device A.}

In the following, we prepare the DQD in an even charge configuration in the many-electron regime (see methods~\ref{app:evenodd}), described by a two-electron Hamiltonian given in methods section~\ref{app:Hamiltonian}.
Figure \ref{fig:Fig1}(c) shows the eigenvalues of this Hamiltonian as a function of external magnetic field $B$ at a fixed DQD detuning.
At zero magnetic field, the detuning renders the singlet $S_{2,0}$ the ground state, for which both electrons reside in the same dot.
Without spin-rotating tunneling, this, and the $S_{1,1}$ state, with the electrons distributed to different dots, form a charge qubit~\cite{vanderwiel2002}. The subscripts describe the dot electron occupation of the left and right dot, respectively.
By applying an external magnetic field, the Zeeman effect lowers the energy of the triplet $T_{1,1}^+$ state, that becomes the ground state for sufficiently high magnetic fields. \corr{In the presence of a spin-rotating tunneling $t=\Delta_\mathrm{SO}/2$ induced by the intrinsic spin-orbit interaction $\Delta_\mathrm{SO}$, the energy levels of the hybridized $S_{2,0}$ and $T_{1,1}^+$ states are split.
The two new eigenstates of the avoided crossing form a singlet-triplet qubit shown schematically in Figs.~\ref{fig:Fig1}(a) and (b).}
%However, the intrinsic spin-orbit interaction hybridizes the $S_{2,0}$ and $T_{1,1}^+$ states, with the two new eigenstates of the avoided crossing forming a singlet-triplet qubit shown schematically in Figs.~\ref{fig:Fig1}(a) and (b).

\begin{figure}
    \centering
    \includegraphics[width=\linewidth]{Figs/Fig2.pdf}
    \caption{\textbf{Dispersive sensing of the DQD at $\mathbf{B=0}$.}
    (a) Charge stability diagram of the device \corr{at $B = 600$~mT applied at $\alpha=$ 164° with respect to the NW}, in which the resonator phase $\varphi$ is measured as a function of the SG voltages $V_R$ and $V_L$.
    %\corr{The negative slopes of the interdot transitions are due to the strong cross-capacitance of the larger gate $SG_R$}. 
    A zoom on the interdot transition pointed out by the \corr{green} rectangle is shown in (b) \corr{and (c) at $B = 0$~T and $B = 300$~mT with $\alpha= 57$°, respectively}. (d) \corr{Resonator transmission $(A/A_0)^2$} versus probe frequency $
    \omega_\mathrm{p}$ and detuning $\varepsilon$ (illustrated by the white line in (b)).
    At the charge degeneracy point of the DQD, \corr{we find a dispersive shift of $21\pm2$~MHz with respect to the bare resonance frequency. At small positive detuning a triplet state crosses the IDT, leading to a suppressed resonator transmission.}}
    \label{fig:Fig2}
\end{figure}
Figure \ref{fig:Fig2}(a) shows the charge stability diagram of \corr{device A} at a magnetic field of 600 mT with the angle $\alpha = 164$° with respect to the NW axis (See Fig.~\ref{fig:Fig1}(a)) detected as a shift in the transmission phase $\varphi$ of the resonator, plotted as a function of the two gate voltages $V_L$ and $V_R$ at a fixed probe frequency of \corr{$\omega_p/2\pi=5.174$ GHz}, close to resonance.
%i.e detuned from the bare resonace frequency by $80$ kHz.
We observe a \corr{characteristic honeycomb pattern of the charge stability diagram of a DQD.}
%The charge stability diagram of device B is discussed in S (ref).}
Using a capacitance model~\cite{van2002electron,scarlino2022situ}, we extract the gate-to-dot capacitances \corr{$C_{R2} = 44 \pm2$\,aF, $C_{L2} =2.0\pm0.2$\,aF, $C_{R1}=5\pm 2 $\,aF and $C_{L1} =4.6\pm0.2$\,aF for device A}.

\corr{We now focus on one particular inter-dot transition (IDT) marked by a green rectangle in Fig.~\ref{fig:Fig2}(a). The same IDT is shown in Fig.~\ref{fig:Fig2}(b) and (c) at $B=0$~T and $B= 300$~mT respectively, with $\alpha = 57$°. In Fig.~\ref{fig:Fig2}\corr{(d)} we show the normalized transmission $(A/A_0)^2$ at $B=0$~T,} while varying the probe frequency $\omega_p$ and relative detuning $\varepsilon_\mathrm{rel}$, \corr{illustrated by the} white line in Fig.~\ref{fig:Fig2}(b).
An electron can now reside on either of the two tunnel-coupled dots, constituting a charge qubit.
At the IDT, close to charge degeneracy, the electrical dipole moment of the charge qubit interacts with the resonator, resulting in a dispersive shift of the resonance frequency.
%Cut this part if necessary
By fitting \corr{input-output theory (see methods section~\ref{app:IO}) to this particular IDT, we extract the inter-dot tunnel coupling $t|_{B=0}/2\pi = 5.1\pm1.0$ GHz, the charge-photon coupling $g_0|_{B=0}/2\pi = 353\pm72$ MHz, and the charge qubit linewidth $\gamma|_{B=0}/2\pi = 1.7\pm0.7$~GHz.}
\section{Strong spin-photon coupling}
\label{sec:anti-crossing}
\begin{figure}[!htpb]
    \centering
    \includegraphics[width=\linewidth]{Figs/Fig3.pdf}
    \caption{\textbf{Strong spin-photon coupling.} (a) \corr{Anti-crossing of the resonator and the qubit found when plotting the resonator transmission as a function of detuning $\varepsilon_\mathrm{rel}$ and probe frequency $\omega_p$ at a magnetic field of $B = 300$\,mT and $\alpha=57$°. The solid white curves are the eigenstate energies from fits to a Jaynes-Cummings model (Eq.~\eqref{eq:JComega} in methods~\ref{app:JC}).
    The faint double-peak structure at $\varepsilon\approx 0$ is an unambiguous signature of the strong coupling regime, $g>\kappa,\gamma$~\cite{blais2021circuit}.    
   (b,c) Cross sections at the detunings indicated by colored bars in (a).
   The solid lines stem from fit to input-output theory.
   (b) Double-peak structure at $\omega_q\sim\omega_0$ (see text). The larger noise floor for $\omega_p\sim\omega_0$ (grey data) is attributed to the bare resonator which is visible in spectroscopy because of a finite coupling between DQD and leads resulting in an odd DQD occupation for a short fraction of time during data acquisition.
   (c) Transmission for $\omega_q\gg \omega_0$, corresponding to the bare resonator.
   (d) Simulation using input-output theory with the parameters extracted from the input-output fit to (b).
   For these measurements, given the input-power $P_\mathrm{in}=-133$\,dBm, the average number of photons is $n<0.25$ (see methods section~\ref{app:Nphoton}).}}
    \label{fig:Fig3}\label{fig:Fig3review}
\end{figure}
When investigating the magnetic-field dependence of IDTs similar to the ones shown in  Fig.~\ref{fig:Fig2}(b,c), we observe two qualitatively different behaviors which we identify as even and odd charge parity configurations described in methods section~\ref{app:evenodd}. 
In the following, we investigate a single IDT, \corr{shown in Fig.~\ref{fig:Fig2}(c)}, with an even charge parity.

\corr{As illustrated in Fig.~\ref{fig:Fig1}(c), the DQD can be operated as a singlet-triplet qubit when applying a magnetic field.
The qubit frequency $\omega_q$ can be brought into resonance with the cavity frequency $\omega_0$ at \corr{$B\gtrsim200$\,mT}, as discussed in more detail below. At the resonance condition ($\omega_q\sim\omega_0$), an anti-symmetric (bonding) and a symmetric (anti-bonding) qubit-photon superposition state are formed.
The corresponding resonances can spectroscopically be discriminated only if the splitting $2g$ between them is larger than the dressed states' linewidth $\gamma+\kappa/2$~\cite{blais2021circuit}. In particular, the hybrid system is considered strongly coupled if the qubit-photon coupling strength $g$ exceeds $\gamma$ and $\kappa$~\cite{blais2021circuit}.}

\corr{In Fig.~\ref{fig:Fig3review}(a), we plot a spectroscopic measurement of the resonator where the singlet-triplet qubit is tuned into resonance by applying an electrostatic detuning $\varepsilon_\mathrm{rel}$ relative to the configuration at which $S_{2,0}$ and $T_{1,1}^+$ would be fully degenerate in the absence of a a spin-rotating tunneling.}
%While in Fig.~\ref{fig:Fig3review}(a) we tune the qubit transition frequency using the gate potentials, it can also be tuned by changing the magnetic field strength as demonstrated in Fig.~S3 in the supplementary.}
\corr{Consistent with strong coupling, we observe an avoided crossing between the resonator and the qubit.
At the points where the bare qubit frequency $\omega_q$ and resonator frequency $\omega_0$ (dashed, white curves) are degenerate, instead of crossing, they anti-cross.
And in Fig.~\ref{fig:Fig3}(a), a faint double peak structure is visible at around $\varepsilon_\mathrm{rel}\sim 0$ as $2g>\kappa/2+\gamma$, signature of the strong coupling regime~\cite{blais2021circuit}.}

\corr{For a quantitative analysis, we fit Lorentzians to the transmission of
 each trace of constant $\varepsilon_\mathrm{rel}$, we extract the transition frequencies $\omega_{\pm}$ of the dressed states.
 These are fitted to the Jaynes-Cummings model (solid, white curves in Fig.~\ref{fig:Fig3}(a)) described in methods section~\ref{app:JC}.
From this fit, we extract the tunnel rate $t|_{B=300\,{\rm mT}}/2\pi=\Delta_\mathrm{so}|_{B=300\,{\rm mT}}/4\pi=2.54\pm0.03$\,GHz and bare spin-photon coupling strength $g_{0}^\mathrm{JC}|_{B=300\,{\rm mT}}/2\pi = 123\pm16$ MHz. The extracted tunnel rate allows to plot the qubit transition frequency $\omega_q=\sqrt{(\Delta_\mathrm{so}/\hbar)^2+(\varepsilon_\mathrm{rel})^2}$ in Fig.~\ref{fig:Fig3review}(a) and to identify the resonance condition $\omega_q=\omega_0$ at a small electrostatic detuning $\varepsilon_\mathrm{rel}/2\pi=\pm 1.0$\,GHz. We evaluate the effective coupling strength  $g=g_0\cdot2t/\omega_q$ at the finite detuning $\varepsilon_\mathrm{rel}/2\pi=- 1.0$\,GHz and obtain $g^\mathrm{JC}|_{\epsilon_\mathrm{rel}/2\pi=-1\mathrm{\,GHz, }}/2\pi = 121\pm16$~MHz, as the spin-photon coupling strength on resonance condition.}

\corr{In Fig.~\ref{fig:Fig3review}(b), we plot a line trace at this detuning value as indicated in Fig.~\ref{fig:Fig3review}(a). Despite the large noise, the double peak structure is also clearly visible and stands in stark contrast to the bare resonator transmission at large detuning (see corresponding linetrace in Fig.~\ref{fig:Fig3review}(c)).
Using Eq.~\eqref{eq:transmission} derived from input-output theory described in the supplementary, we fit these data at 300\,mT and extract the spin-photon coupling strength $g_{\varepsilon_\mathrm{rel}=-\mathrm{1\,GHz}}/2\pi=139\pm4$\,MHz and qubit dephasing $\gamma/2\pi=116\pm7$\,MHz where we used the bare resonator decay $\kappa|_{B=300\sf\,mT}/2\pi=19.8\pm0.6$\,MHz. This value agrees well with the one obtained from the Jaynes-Cummings model. Using the values from input-output theory we model the whole anti-crossing using input-output theory in Fig.~\ref{fig:Fig3}(d), observing a very good agreement with the measurement.}

\corr{All together, this measurement therefore clearly demonstrates that the strong coupling regime between a single microwave photon and a singlet-triplet qubit is reached.}
\section{Magnetospectroscopy}
\begin{figure}
    \centering
    \includegraphics[width=\linewidth]{Figs/Fig4.pdf}
    \caption{\textbf{Magnetospectroscopy of the singlet-triplet qubit.} a) Dispersive shift $\chi$ as a function of the magnetic field $B$ \corr{at an angle of $\alpha=130$°} and detuning $\varepsilon$. The white dashed line is a fit of the effective two-electron Hamiltonian (Eq.~\eqref{app:Hamiltonian}) to the data. b) Extracted tunnel rate $2t/2\pi$ (black), qubit-photon coupling $\corr{g_0}/2\pi$ (\corr{blue}) and qubit linewidth $\gamma/2\pi$ (\corr{purple}). The bare resonator frequency is indicated by the dashed black line. \corr{Shaded areas indicate the errorbars which originate from the uncertainty of the gate lever arm, which was independently measured.} (c) Two-electron energy level diagrams at various magnetic fields with the corresponding field strength indicated in (a) and (b) by the given symbols. For clarity a constant offset of 10 GHz, 20 GHz and 30 GHz was added to the energy levels at 300 mT, 410 mT and 600 mT. Given the input power $P_\mathrm{in}=-128$\,dBm, the average photon number is $n< 0.8$ in these experiments (see methods section~\ref{app:Nphoton}).}
    \label{fig:Fig4}
\end{figure}
\label{sec:magnetospectroscopy}
To explicitly identify and characterize the spin-orbit eigenstates and to independently verify the character of the singlet-triplet qubit, we now study the magnetic field evolution of the IDT from 0 up to \corr{$900$\,~mT applied at the angle $\alpha=130$°}.
%We measure the complex transmission amplitude  $S_{2,1}$ through the resonator as a function of magnetic field and DQD detuning. 
We measure the amplitude $A$ and phase $\varphi$ of the signal transmitted through the resonator as function of detuning $\varepsilon$ and magnetic field strength $B$.
A non-zero \corr{$\varphi$} occurs at the IDT when tunneling between the dots is allowed resulting in a non-zero DQD dipole moment.
As described in methods section~\ref{app:Hamiltonian}, we model the DQD by an effective two electron Hamiltonian which allows us to fit the gate voltage and field dependence of the IDT (white dashed line in \corr{Fig.~\ref{fig:Fig4}(a)}).
We find that the magneto-dispersion of the IDT is well described using the following fit parameters: the spin-conserving singlet and triplet tunnel rates \corr{$t_c^S/2\pi\approx8.5$\,GHz, and
$t_c^T/2\pi\approx3.2$\,GHz, the singlet-triplet coupling rate
$t_{\rm{SO}}/2\pi=\Delta_\mathrm{SO}/4\pi\approx 2.9$\,GHz, the electron g-factors of the right and left dots, $g_R\approx 1$ and $ g_L\approx 8$, as well as the single dot singlet-triplet energy splitting $\Delta_{\rm{ST}}/2\pi\approx 47$\,GHz.}
These fit parameters are consistent with parameters obtained previously in this material system~\cite{fasth2007direct,nadj2010disentangling,nilsson2018tuning,trif2008spin,junger2020magnetic}.
We note, however, that the fit is under-determined and therefore, it does not provide accurate numbers.
Nonetheless, the model gives a qualitative, physical understanding of the system and allows us to establish which DQD levels interact with the resonator.

Independently, we gain quantitative information about the system by considering the functional dependence of the amplitude $A$ and phase $\varphi$. % on the detuning $\varepsilon$. for values of fixed magnetic field $B$.
This is possible because the resonator provides an absolute energy scale allowing for a quantitative analysis of the interaction between the DQD and the resonator \corr{and hence to perform qubit spectroscopy~\cite{borjans2021probing,mi2017high,ibberson2021large}.
This spectroscopy complements} the preceding DQD Hamiltonian fit.
As described in methods \corr{section~\ref{app:IO}}, by fitting \corr{input-output theory to $\varphi$ and $A$} simultaneously, we extract the qubit tunnel amplitude $t$, the qubit linewidth $\gamma$, and the qubit-photon coupling strength $g$ as a function of $B$, which we plot in Fig.~\ref{fig:Fig4}(b). \corr{Here, we assume $\gamma$ as constant in detuning $\varepsilon$.}

Using the fits to both, the 2-electron Hamiltonian model and \corr{input-output theory} in the 2-level approximation, allows us to directly identify several regimes, in each of which the qubit has a different spin-character.
Fig.~\ref{fig:Fig4}(c) shows the corresponding DQD level structure based on the fit parameters as a function of $\varepsilon$ for different magnetic field.

At a low magnetic fields around \corr{$B = 20$\,mT}, the triplet states (blue curves) are Zeeman split and the ground-state curvature is dominated by the anti-crossing between $S_{1,1}$ and $S_{2,0}$ (red curves).
We find a singlet charge qubit in the weak coupling limit, i.e. the linewidth exceeds the charge-photon coupling by a factor of five.
The formation of an \corr{asymmetric double-dip} structure in $\varphi (\varepsilon)$ between \corr{$B\sim 0.01$\,T and $B\sim0.18$\,T} is explained by an interaction between the three states $S_{2,0}$, $S_{1,1}$ and $T_{1,1}^+$ as described in the supplementary material.
Traces of \corr{$\varphi(\varepsilon)$} with an \corr{asymmetric} double-dip structure cannot be described by \corr{a two-level input-output model} and are therefore not analysed quantitatively here. \corr{At $B\approx 50$\,mT, $\varphi$ becomes positive. Which we interpreted as a drop of the tunnel rate below the resonator frequency, $2t<\omega_0$}

As $B$ is increased, the triplet state $T_{1,1}^+$ becomes the ground state for $\varepsilon<0$, as shown in the second panel of Fig.~\ref{fig:Fig4}(c) for \corr{$B=300$\,mT.}
The spin-orbit interaction couples the singlet and triplet states, leading to an anti-crossing between $S_{2,0}$ and $T_{1,1}^+$, which constitutes a singlet-triplet qubit with $\omega_q=\Delta_\mathrm{SO}=2t_\mathrm{SO}$~\cite{mutter2021all,jirovec2021singlet}.
In this regime, at larger $B$, the resonance condition between $S_{2,0}$ and $T_{1,1}^+$ occurs at larger $\varepsilon$, because the energy of the bare $T_{1,1}^+$ state decreases with larger $B$ and the energy of $S_{2,0}$ decreases with larger $\varepsilon$.
Therefore, the IDT is observed at larger $\varepsilon$ for increasing $B$.

Consistent with the interpretation of the formation of a singlet-triplet qubit, we measure an approximately constant tunneling rate $t$ between \corr{$B\sim0.18$\,T and $B\sim0.36$\,T.}
In this regime, we extract the average spin-orbit tunneling rate to be \corr{$\bar{t}_{\rm{so}}=1.94\pm 0.02$\,GHz}. % where the error bar is the root variance.
%At $B\approx 1.3$\,T, $\chi$ becomes positive. This is interpreted as a drop of the tunnel rate below the resonator frequency, $2t<\omega_0$.
%This decline in $t$ is not captured by our simplified Hamiltonian model and we speculate that changes in the orbital structure of a many-electron DQD could be the reason.

\corr{At a magnetic field of \corr{$B\approx 370$\,mT,} the resonator phase $\varphi$ starts to vanish due to the the triplet state $T_{2,0}^+$ becoming relevant. The triplet state results in a level repulsion between $T_{2,0}^+$ and $T_{1,1}^+$ and hence leads to a reduced energy gap between the $S_{2,0}$ level and the $T_{1,1}^+$ level. In Fig.~\ref{fig:Fig4}(c), this is illustrated by the smaller energy gap (black arrow) at $B=410$\,mT compared to the one at $B=300$\,mT. Due to the The reduced energy gap, the resonator-qubit coupling on resonance ($\omega-q=\omega_0$ is and hence is the signal in  $\varphi$.}
 
The level structure at large magnetic fields is plotted exemplary for \corr{$B=600$\,mT} in the right panel of Fig.~\ref{fig:Fig4}(c).
In this regime, the ground-state of the DQD at the IDT is formed by a superposition of the $T_{2,0}^+$ and $T_{1,1}^+$ states.
%, forming a charge qubit with triplet spin character.
We find that the curve of Fig.~\ref{fig:Fig4}(a) turns back towards lower $\varepsilon$ for increasing $B$, which can be understood by noting that the spin-polarized triplets $T_{2,0}^+$ and $T_{1,1}^+$ form a charge qubit similar to the singlets at low field.
While the transition is increasingly dominated by the triplet-charge qubit for increasing $B$, $\varphi$ becomes \corr{negative} at the IDT, because the anti-crossing between the triplet states $T_{2,0}^+$ and $T_{1,1}^+$ occurs at much larger frequencies, \corr{$2t_c^T > 2t_\mathrm{SO},\omega_0$.
Hence, the triplet charge qubit frequency does not cross the resonator frequency, leading to a negative phase shift.}

\corr{At fields $B>700$~mT the dispersion turns to higher $\varepsilon$ again. Which is not accounted for in our model. A possible explanations to this discrepancy is that the magnetic field not only affects the detuning $\varepsilon$ of the DQD but also the total energy. This results in the lead to dot transitions starting to influece the IDT at high magnetic fields. Nevertheless, the data is well described at the magnetic field strengths we investigate in detail.}

This large number of detailed findings justify the parameters of the two-electron Hamiltonian introduced above, which, in turn, directly allows us to identify the singlet-triplet spin qubit, for which we find the strong coupling limit to the electromagnetic cavity. 
%We note that the qubit linewidth $\gamma$ and qubit-photon coupling strength are both related to the qubit rate.
%To confirm their linear relation, we plot $\gamma$ and $g$ against $t$ in Fig.~\ref{fig:A2} in the appendix. 
%An intuitive explanation is that the tunnel rate in our experiments increases as the qubit becomes more charge like and hence is more susceptible to charge noise.
%Another possible explanation is that the qubit linewidth is limited by qubit relaxation which scales proportional to the tunnel rate to the contacts. \corr{This argument is further strengthened by the comparison of device A and device B. For device A a more negative voltage ($V_{TG_A}-0.28$ V and $V_{TG_B}=-0.05$ V) was applied on the top gates on top of the tunnel barriers to the leads. Consequently, we observed a smaller qubit linewidth compared to device B (see supplementary information for the data on device B).

Note, that the extracted qubit linewidth is larger in Fig.~\ref{fig:Fig4}(b) compared to the strong-coupling in Fig.~\ref{fig:Fig3}. This is caused by applying the magnetic field at different angles in the two measurements.%~\cite{pally2023inprep}.
%%%%%%%%%%%%%%%
%%%%%%%%%%%%%%%
%%%%SUMMARY%%%%
%%%%%%%%%%%%%%%
%%%%%%%%%%%%%%%
\section{Conclusion and Outlook}
In summary, we demonstrate a semiconductor nanowire DQD device with crystal-phase defined tunnel barriers that can be operated as different types of qubits, coupled to a high-impedance, high magnetic field resilient electromagnetic resonator. As the main result, we find
a singlet-triplet qubit for which we extract the relevant qubit parameters, especially \corr{a large electron spin-photon coupling of $g/2\pi=139$\,MHz in the single photon limit, reaching the strong coupling regime $g>\gamma,\kappa$.}

Our experiments demonstrate that deterministically grown tunnel barriers allow for a reduced number of gate lines, and that, mediated by intrinsic spin-orbit interaction, singlet-triplet qubits can reach the strong coupling limit for low photon numbers, similar to flopping mode spin qubits~\cite{yu2023strong,burkard2021semiconductor}. This finding is potentially applicable to other promising platforms with strong spin-orbit interactions, like holes in Ge~\cite{jirovec2021singlet}.
Our nanowire platform without depletion gates results in a significantly reduced gate-induced photon-leakage in the absence of on-chip filtering~\cite{petersson2012circuit,mi2017circuit,harvey2020chip}. \corr{And, since DQD parameters (such as charging energy and individual tunnel rates) can be set deterministically in the NW growth, multiple NWs with optimal and essentially identical characteristics properties can be obtained simultaneously~\cite{nilsson2017parallel} and possibly integrated on the same substrate~\cite{ram2021high}.
This drastically simplifies the search for an optimal gate regime and renders further gates, such as the top gates in our device, unnecessary.}
An optimized gate design with resonators with larger impedance~\cite{scarlino2022situ} therefore presents an ideal platform to investigate new phenomena in the ultrastrong coupling regime~\cite{forn2019ultrastrong,scarlino2022situ}. Additionally, the large electron spin-photon coupling found in our experiments will be crucial for the implementation of two-qubit gates between distant spin qubits, a milestone on the way towards scalable quantum computers.
%\corr{In the future, a scalable quantum computing architecture based on crystal-phase defined barriers might be incorporated using template-assisted selective epitaxy (“TASE”) in which nanowires are used as crystal seeds for creating a two-dimensional crystal.}

We acknowledge fruitful discussions with Simon Zihlmann, Roy Haller, Andrea Hofmann, Stefano Bosco, Romain Maurand and Antti Ranni, and support in setting-up the experiments by Fabian Op\corr{p}liger, Roy Haller, Luk Yi Cheung, and Deepankar Sarmah.
This research was supported by the Swiss Nanoscience Institute (SNI), the Swiss National Science Foundation through grant 192027, the NCCR Quantum Science and Technology (NCCR-QSIT), the NCCR Spin Qubit in Silicon (NCCR-Spin) and the Eccellenza Professorial Fellowship PCEFP2\textunderscore194268. We further acknowledge funding from the European Union’s Horizon 2020 research and innovation programme, specifically the FET-open project AndQC, agreement No 828948 and the FET-open project TOPSQUAD, agreement No 847471. Furthermore, we acknowledge funding by NanoLund and the Knut \& Alice Wallenberg Foundation (KAW). PS acknowledges support from the SNSF through grant 200418 and the SERI through grant 589025. All data in this publication are available in
numerical form at: \url{https://doi.org/10.5281/zenodo.7777840}.
\section{Methods}
\subsection{Resonator characterization and analysis}
\label{app:resonator}
The resonator is fabricated from a thin-film NbTiN \corr{(thickness $\sim$10\,nm)}, sputtered onto a Si/SiO\textsubscript{2} (500\,\textmu m/100\,nm) substrate~\cite{ungerer2023performance}.
These resonators can be operated for in-plane fields exceeding $5$\,T~\cite{samkharadze2016high,ungerer2023performance}.
The large sheet kinetic inductance of the used NbTiN film of $L_\mathrm{sq}\approx 90$\,pH combined with the narrow center conductor width of $\sim380$\,nm, and the large distance to the ground plane of $\sim 35$\,\textmu m results in an impedance of $2.1$ k$\Omega$.
The resonator can be dc biased using a bias line which contains a meandered inductor ensuring sufficient frequency detuning between the half-wave resonance used in the experiment and a second, low quality resonance mode at a lower frequency that forms due to the finite inductance of the bias line~\cite{harvey2020chip}.
\corr{A scanning electron micrograph of the resonator center-conductor is shown in Fig.\ref{fig:FigA1}(b) in the extended data.}
One of the two resonator voltage anti-nodes is galvanically connected to gate SG$_R$ shown in Fig.~\ref{fig:Fig1}(c) of the main text.
\subsection{Charge parity determination\label{app:evenodd}}
We measure the phase $\varphi$ and amplitude 
$A$ of the resonator as a function of detuning $\varepsilon$ and magnetic field $B$ at a probe-frequency $\omega_{\rm{p}}/2\pi=5.253$\,GHz, close to the bare resonator frequency.
A change in $\varphi$ reflects the dispersive interaction between the resonator and two anticrossing levels of the DQD~\cite{frey2012dipole,crippa2019gate}.
Therefore, the non-zero phase response of the resonator tracks the position of the IDT along the detuning axis. 
The comparison of the magnetic field dependence of the IDT position to a Hamiltonian model of the DQD allows one to determine the charge parity~\cite{crippa2019gate,ezzouch2021dispersively}.
Figures \ref{fig:Fig2_app} (a) and (b) in the extended data show two typical low field IDT characteristics \corr{of device B}.

For an odd number of electrons (Fig.~\ref{fig:Fig2_app}(b)), the DQD resonance gate voltage $V_R$, at which the IDT is observed, disperses linearly with magnetic field starting from zero. 
This can be understood considering the Zeeman-splitting of the unpaired electron energy levels and two non-equal Landé g-factors of the two dots.
Fig.~\ref{fig:Fig2_app}(c) shows the energy level diagram of a one-electron Hamiltonian including Zeeman-splitting with a g-factor difference of 1.0 and spin-orbit interaction $t_{SO}/2\pi = 5$~GHz at a magnetic field of $B=500$\,mT (green, dashed line in Fig.~\ref{fig:Fig2_app}(b). The one-electron Hamiltonian is explicitly discussed in the supplementary material.
The arrow points out the center of the IDT (largest curvature of the groundstate~\cite{park2020adiabatic}) which corresponds to the largest dipole moment of the DQD and thus to the largest change in $\varphi$.
This point shifts with $B$ towards increasingly negative values.

For an even number of electrons in the DQD at zero magnetic field (Fig.~\ref{fig:Fig2_app}(a)), a single dip in phase is observed, but at a low magnetic fields, $B\approx60$\,mT, a double dip structure emerges as a function of $\varepsilon$ (see supplementary material for details).
This double-dip originates from an interaction between $S_{2,0}$, $S_{1,1}$ and $T_{1,1}^+$ as explained in detail in the supplementary material.
The dependence of the IDT on magnetic field for an even number of electrons can be understood using an effective two electron Hamiltonian including spin-orbit interaction described in more detail in section~\ref{app:Hamiltonian}.
In Fig.~\ref{fig:Fig2_app}(c), we plot the energy levels at a magnetic field $B = 0.15$~T.
In contrast to the odd filling, starting at zero magnetic field, the arrow marking the center of the IDT barely changes, consistent with our measurement. 
The double dip vanishes when further increasing the magnetic field, because of an increasing occupation of the polarized triplet states.
Once the Zeeman energy of the triplet state $\ket{T_{1,1}^+}$ becomes comparable to the singlet charge tunneling $t_{\rm{c}}^{\rm{S}}$, the position of the IDT as a function of $B$ disperses towards larger $\varepsilon$~\cite{schroer2012radio,malinowski2018spin,ezzouch2021dispersively}.
This transition is marked by the white dashed line at $0.2$~T in~\ref{fig:Fig2_app}(a).

Based on the good qualitative agreement between our data and the one electron and two electron Hamiltonian, respectively, we can clearly identify the even and odd charge parities.

\subsection{Jaynes-Cummings model}
\label{app:JC}
In the regime of only two DQD levels being relevant, we model the DQD Hamiltonian as an effective two-level system (qubit) interacting with a single \corr{mode} in the resonator.
The combined system is described by the Jaynes-Cummings model~\cite{shore1993jaynes} \corr{
\begin{equation}
    \label{eq:hamiltonJC}
    \hat{H}/\hbar = \omega_0\hat{a}^\dagger\hat{a}+\frac{\omega_q}{2}\hat{\sigma}_z+g\left(\hat{a}\hat{\sigma}^\dagger+\hat{a}^\dagger\hat{\sigma}\right),
\end{equation}
where $\hat{a}$ is the photon annihilation operator, $\hat{\sigma}$ the qubit lowering operator, and $\hat{\sigma}_z$ the Pauli z-matrix in the qubit subspace. The qubit frequency is given by $\omega_q=\sqrt{\left(2t\right)^2+\varepsilon^2}$~\cite{vanderwiel2002} with the effective qubit-photon coupling strength $g=g_0\cdot2t/\omega_q$ accounting for the mixing angle~\cite{blais2004cavity,stockklauser2017strong}, where $g_0$ is the bare qubit-photon coupling.}
\corr{An excitation from the ground state has the transition frequency}~\cite{blais2004cavity}
\begin{equation}
\omega_{\pm}=\corr{\frac{\omega_0+\omega_q}{2}}\pm\frac{1}{2}\sqrt{4g^2+(\omega_0-\omega_q)^2}\text{.}\label{eq:JComega}
\end{equation}

\subsection{Input-Output theory}
\label{app:IO}
\corr{To derive the response of the resonator, we use the equations of motion \cite{gardiner2004quantum}
\begin{equation}
    \label{eq:ainoutgen}
    \begin{aligned}
        &\partial_t{\langle \hat{a}\rangle}(t) = -i\omega_0 \hat{a}(t)-ig\langle \hat\sigma\rangle (t)-\frac{\kappa}{2}\langle \hat{a}\rangle(t)\\&\hspace{2cm}-\sqrt{\kappa_1}\langle \hat{b}_{{\rm in}, 1}\rangle(t)-\sqrt{\kappa_2}\langle \hat{b}_{{\rm in}, 2}\rangle(t),\\&
        \partial_t \langle \hat{\sigma}\rangle (t) = -i\omega_q\langle \hat{\sigma}\rangle (t)+ig\langle \hat{a} \hat{\sigma}_z\rangle(t)-\gamma\langle \hat{\sigma}\rangle (t).
    \end{aligned}
\end{equation}
%where $\kappa$ denotes the line-width of the cavity and $\gamma$ the dephasing rate of the DQD. 
The input couplings are denoted by $\kappa_{j}$ and the operators $\hat{b}_{{\rm in}, j}(t)$ capture a coherent drive in port $j$. In our experiments $\kappa_1\approx\kappa_2\approx\kappa/2$ as the resonator is symmetrically coupled and operates in the strongly over-coupled regime. The output of the cavity can be computed from the input-output relation \cite{gardiner2004quantum}
\begin{equation}
    \label{eq:inout}
    \langle \hat{b} _{{\rm out},j}\rangle(t) = \langle \hat{b}_{{\rm in},j}\rangle(t)+ \sqrt{\kappa_j}\langle \hat{a}\rangle(t).
\end{equation}
To solve these equations, we approximate \cite{wong:2017,schondorf:2018}
\begin{equation}
	\label{eq:mean_field}
	\langle\hat{a} \hat{\sigma}_z\rangle(t)\rightarrow \langle \hat{a}\rangle (t)\langle \hat{\sigma}_z\rangle,
\end{equation}
where $\langle \hat{\sigma}_z\rangle$ is evaluated at steady state and captures the difference between the population of the excited qubit state and the ground state, accounting for operation at larger temperatures or drive strengths. In our experiments, we operate in the linear regime, $\langle \hat{\sigma}_z\rangle= -1$.}

\corr{To compute the transmission amplitude, we solve Eqs.~\eqref{eq:ainoutgen} and \eqref{eq:inout} upon Fourier transformation and set $\langle \hat{b}_{{\rm in},2}\rangle(t)=0$. This results in the transmission amplitude
\begin{equation}
    \label{eq:transmission}
    \tau(\omega) = -\frac{\langle \hat{b} _{{\rm out},2}\rangle(\omega)}{\langle \hat{b} _{{\rm in},1}\rangle(\omega)}= \sqrt{\kappa_1\kappa_2}A(\omega),
\end{equation}
where the minus sign accounts for the phase difference of $\pi$ between the input and the output port ($\lambda/2$ resonator) and
\begin{equation}
    \label{eq:aomega}
    A(\omega) = \frac{\gamma+i(\omega_q-\omega)}{[\kappa/2+i(\omega_0-\omega)][\gamma+i(\omega_q-\omega)]-g^2\langle \hat{\sigma}_z\rangle}.
\end{equation}
In the main text, the absolute value squared of this quantity normalized by its maximal value is shown.}

%\corr{In input-output theory~\cite{gardiner2004quantum}, as derived in the supplementary, the transmission amplitude of the coupled resonator-qubit hybrid system is given by}
%\begin{equation}\label{eq:A_io}
%A(\omega)=\frac{\sqrt{\kappa_L\kappa_R} \left[\gamma+i\left(\omega_q-\omega\right)\right]}{\left[\frac{\kappa}{2}+i\left(\omega_0-\omega\right)\right]\left[\gamma+i\left(\omega_q-%\omega\right)\right]-g^2m_z}.
%\end{equation}
%Here, $m_z=p_e-p_g$ is the difference between the population of the excited qubit state $p_e$ and the population of the ground state $p_g$ and in principle accounts for operation at larger temperatures or drive strengths. In our experiments, we operate in the linear regime, $m_z=-1$. Furthermore $\kappa_L$ and $\kappa_R$ are the coupling rates of the resonator to the left and right port. In our experiments $\kappa_L\approx\kappa_R\approx\kappa/2$ as the resonator is symmetrically coupled and operates in the strongly over-coupled regime.
The phase of the transmitted signal is given by \corr{
\begin{equation}
	\label{eq:transphasegen}
 \begin{aligned}
	&\varphi(\omega) = -\arctan(\Lambda),\\&\Lambda =\frac{-2(\omega_q-\omega)g^2\langle\hat{\sigma}_z\rangle-2(\omega_0-\omega)[\gamma^2+(\omega_q-\omega)^2]}{\kappa[\gamma^2+(\omega_q-\omega)^2]-2\gamma g^2\langle \hat{\sigma}_z\rangle}.
 \end{aligned}
\end{equation}
%\begin{align}\label{eq:phiIO}
%&\varphi\left(\omega_p\right)=-\mathrm{atan}\Big\{\\
%&\nonumber\left[-2\left(\omega_q-\omega_p\right)g^2m_z-2\left(\omega_0-\omega_p\right)\left(\gamma^2+\left(\omega_q-\omega_p\right)^2\right)\right]/\\
%&\nonumber\left[\kappa\left(\gamma^2+\left(\omega_q-\omega_p\right)^2\right)-\gamma g^2m_z/2\right]\Big\}.
%\end{align}
%where the minus sign in front of the arcus tangens accounts for the phase difference of $\pi$ between the input and the output port ($\lambda/2$ resonator). 
As examples, the phase and amplitude of the bare resonance in Coulomb blockade is simultaneously fit in Fig.~\ref{fig:FigA1}(a) and in Fig.~\ref{fig:Fig_app_shift_linewidth} the same is done for a linecut of Fig.~\ref{fig:Fig4}(a) at $0.25$~T.}
\subsection{Estimation of the photon number}
\label{app:Nphoton}
\corr{Similarly, we may obtain $\langle \hat{a}\rangle(t)$ by solving Eqs.~\eqref{eq:ainoutgen}. Using $\langle \hat{b}_{{\rm in},1}\rangle (t)=\exp(-i\omega_p t)\sqrt{P_{\rm in}/ \omega_p}$, where $P_{\rm in}$ denotes the power in the input field, we find
\begin{equation}
    \label{eq:aoft}
    \langle \hat{a}\rangle (t) = -\sqrt{\frac{\kappa_1 P_{\rm in}}{\hbar\omega_p}}e^{-i\omega_pt}A(\omega_p).
\end{equation}
In the low-drive regime we consider here, we estimate the photon number as
\begin{equation}
    \label{eq:phtonnum}
    n = |\langle \hat{a}\rangle |^2 = \frac{\kappa_1 P_{\rm in}}{\hbar\omega_p}|A(\omega_p)|^2,
\end{equation}
where we approximate $\kappa_1\simeq \kappa/2$.}

%\textcolor{red}{TO BE REPLACED BY PATRICK In Fig.~\ref{fig:Fig3review}, the input-power at the device level is estimated as -133\,dBm by subtracting the calibrated attenuation of the cryostat wiring from the input-power at room temperature. Therefore, we estimate the average number of photons $<0.5$~\cite{palacios2010superconducting,weissl2015kerr} which is an upper bound as internal losses of the resonator and expected reflections at the device level (e.g. wire bonds) are ignored.}
\subsection{Effective two-electron Hamiltonian model}
\label{app:Hamiltonian}
We model an effective two-electron Hamiltonian in the presence of spin-orbit interaction and magnetic field.
We write the Hamiltonian in the basis of singlet and triplet states $\left\{\ket{S_{1,1}},\ket{S_{2,0}},\ket{T_{1,1}^{\pm,0}},\ket{T_{2,0}^{\pm,0}}\right\}$, with the subscripts indicating the charge distribution in the DQD. The Hamiltonian reads
\begin{equation}
\mathcal{H}=\mathcal{H}_0^S+\mathcal{H}_0^T+\mathcal{H}_Z+\mathcal{H}_{\mathrm{so}},
\end{equation}
with the spin quantum-number conserving Hamiltonians
\begin{align}\footnotesize
\mathcal{H}_0^S/\hbar=&-\varepsilon\ket{S_{2,0}}\bra{S_{2,0}}+t_c^S\ket{S_{1,1}}\bra{S_{2,0}}+\text{h.c.}\text{,}\\
\mathcal{H}_0^T/\hbar=&(\Delta_{\rm{ST}}-\varepsilon) \sum_{\pm,0}\nonumber\ket{T_{2,0}^{\pm,0}}\bra{T_{2,0}^{\pm,0}}\\
&+t_c^T\sum_{\pm,0}\ket{T_{1,1}^{\pm,0}}\bra{T_{2,0}^{\pm,0}}+\text{h.c.}\nonumber
\end{align}
\normalsize
Here, $t_{c}^{S,T}$ are the tunnel rates between the two singlets, and between the two triplet states respectively, and $\Delta_\mathrm{ST}$ is the single-dot singlet triplet splitting that separates the $T_{2,0}$ states from the $S_{2,0}$ states.
The Zeeman Hamiltonian is given by
\begin{equation}
\footnotesize
\mathcal{H}_Z/\mu_B =B\sum_\pm\left(\pm \frac{g_L+ g_R}{2}\ket{T_{1,1}^\pm}\bra{T_{1,1}^\pm}\pm g_L\ket{T_{2,0}^\pm}\bra{T_{2,0}^\pm}\right),
\end{equation}
%\normalsize
where $g_L$ ($g_R$) is the Landé g-factor of the left (right) dot. Because of the large intrinsic spin-orbit interaction in the NW, we include the spin-orbit Hamiltonian that couples the singlet and triplet states with opposite charge configuration using the spin-orbit tunnel rate $t_{\rm{SO}}$ as
%\footnotesize
\begin{equation}
\mathcal{H}_{\rm{SO}}/\hbar=t_{\rm{SO}}\left(\ket{T^0_{1,1}}\bra{S_{2,0}}+\sum_{\pm}\pm \ket{T_{1,1}^\pm}\bra{S_{2,0}}\right)+\text{h.c.}
\end{equation}
%\appendix
\section*{Bibliography}
\bibliography{ref}
\clearpage
\section{Extended data}
\begin{figure}[htbp!]
    \centering
    \includegraphics[width=\linewidth]{Figs/FigA1.pdf}
    \caption{\textbf{Resonator of a device A.}
    \corr{(a) Resonance curve of the resonator in Coulomb blockade in amplitude $A/A_0$ (blue) and phase $\varphi$ (red). The black lines are simultaneous fits to the data using input-ouput theory. (b) Scanning electron micrograph of the resonator center conductor.}}
    \label{fig:FigA1}
\end{figure}
\begin{figure}[htbp!]
    \centering
    \includegraphics[width=\linewidth]{Figs/FigA3.pdf}
    \caption{\textbf{Dispersive read-out at low magnetic field.}
    Resonator phase in dependence of the right gate voltage $V_R$ and magnetic field $B$ for even (a) and odd (b) occupation of the DQD \corr{of device B}. For the odd occupation the IDT shifts to lower $V_R$ from $B=0$. The IDT of the even occupation stays nearly independent of magnetic field until around $0.2$~T (white dashed line), from where it starts moving to more positive $V_R$. Energy level diagram for the even (c) and odd (d) configuration at $0.15$~T  and $0.5$~T (green dashed line). The arrow marks the transition the resonator is sensitive to, where the ground state energy level has maximum curvature.}
    \label{fig:Fig2_app}
\end{figure}
\begin{figure}[htbp!]
    \centering
    \includegraphics[width=\linewidth]{Figs/FigA2v2.pdf}
    \caption{\textbf{Fit of input-output theory to the data.}
    \corr{Input-output theory (solid lines) simoultaneously fitted to the resonator amplitude $A$ (a) and the resonator phase $\varphi$ (b) as a function of detuning $\varepsilon$ of the even configuration at $0.25$ T.}}
    \label{fig:Fig_app_shift_linewidth}
\end{figure}
\clearpage
\end{document}