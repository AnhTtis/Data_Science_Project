\documentclass[review,hidelinks,onefignum,onetabnum]{siamart220329}

\usepackage{amssymb}
\usepackage{latexsym}
%%%%%%%%%%%%%%%%%%
\usepackage{amstext,bbm,amsbsy} %amsthm
\usepackage{mathtools}
\usepackage{comment}
\usepackage{tabularx}
\usepackage{multirow}
\usepackage{url}

\usepackage{subcaption}

\usepackage{tikz}
\usetikzlibrary{shapes.geometric, arrows}
\usepackage{adjustbox}
\usepackage{siunitx}


% SIAM Shared Information Template
% This is information that is shared between the main document and any
% supplement. If no supplement is required, then this information can
% be included directly in the main document.


% Packages and macros go here
\usepackage{lipsum}
\usepackage{amsfonts}
\usepackage{graphicx}
\usepackage{epstopdf}
\usepackage{algorithmic}
\ifpdf
  \DeclareGraphicsExtensions{.eps,.pdf,.png,.jpg}
\else
  \DeclareGraphicsExtensions{.eps}
\fi

% Prevent itemized lists from running into the left margin inside theorems and proofs
\usepackage{enumitem}
\setlist[enumerate]{leftmargin=.5in}
\setlist[itemize]{leftmargin=.5in}

% Add a serial/Oxford comma by default.
\newcommand{\creflastconjunction}{, and~}

% Used for creating new theorem and remark environments
\newsiamremark{remark}{Remark}
\newsiamremark{hypothesis}{Hypothesis}
\crefname{hypothesis}{Hypothesis}{Hypotheses}
\newsiamthm{claim}{Claim}

% Sets running headers as well as PDF title and authors
\headers{Area formula for spherical polygons via prequantization}{Albert Chern and Sadashige Ishida}

% Title. If the supplement option is on, then "Supplementary Material"
% is automatically inserted before the title.
\title{Area formula for spherical polygons via prequantization 
% \thanks{Submitted to the editors April 12th, 2023. \rev{Revision submitted to the editors March 28th, 2024. 
% }
% % \funding{This project was funded in part by the European Research Council (ERC Consolidator Grant 101045083 \emph{CoDiNA})}
% }
}

% Authors: full names plus addresses.
\author{Albert Chern
\thanks{University of California San Diego
  (\email{alchern@ucsd.edu}).}
\and Sadashige Ishida\thanks{Institute of Science and Technology Austria 
  (\email{sadashige.ishida@ist.ac.at}).}
% \and Jane E. Smith\footnotemark[3]
}

\usepackage{amsopn}
\DeclareMathOperator{\diag}{diag}


%%% Local Variables: 
%%% mode:latex
%%% TeX-master: "ex_article"
%%% End: 


% Optional PDF information
\ifpdf
\hypersetup{
  pdftitle={Generalized Hopf Formula, Learning, and Optimal Control - LQR direction},
  pdfauthor={todo}
}
\fi


\definecolor{myorange}{rgb}{0.883, 0.473, 0.180}
\definecolor{myblue}{rgb}{0.277, 0.621, 0.969}
\definecolor{mygreen}{rgb}{0.324, 0.680, 0.230}

\newcommand{\be}{\begin{equation}}
\newcommand{\ee}{\end{equation}}
\newcommand{\benn}{\begin{equation*}}
\newcommand{\eenn}{\end{equation*}}

\newtheorem{prop}{Proposition}[section]
\newtheorem{lem}{Lemma}[section]
\newtheorem{rem}{Remark}[section]
\newtheorem{cor}{Corollary}[section]
\newtheorem{thm}{Theorem}[section]
\newtheorem{defn}{Definition}
\newtheorem{example}{Example}[section]
\DeclareMathOperator*{\argmin}{arg\,min}
\DeclareMathOperator*{\argmax}{arg\,max}

\def\bx {\boldsymbol{x}}
\def\by {\boldsymbol{y}}
\def\bv {\boldsymbol{v}}
\def\bu {\boldsymbol{u}}
\def\bd {\boldsymbol{d}}
\def\bp {\boldsymbol{p}}
\def\bq {\boldsymbol{q}}
\def\bb {\boldsymbol{b}}
\def\ba {\boldsymbol{a}}
\def\bz {\boldsymbol{z}}
\def\bc {\boldsymbol{c}}
\def\R {\mathbb{R}}
\def\N {\mathbb{N}}
\def\E {\mathbb{E}}
\def\dom {\mathrm{dom~}}
\def\prox {\mathrm{prox}}
\def\inter {\mathrm{int~}}
\def\epi {\mathrm{epi~}}
\def\ri {\mathrm{ri~}}
\def\conv {\mathrm{conv}~}
\def\cone {\mathrm{cone}~}
\def\cobar {\overline{\mathrm{co}}~}
\def\co {\mathrm{co~}}
\def\L {\mathcal{L}}
\def\cl {\mathrm{cl~}}
\def\gr {\mathrm{gr~}}
\def\unitsim {\Lambda}
\def\Rn {\R^n}
\def\gmRn {\Gamma_0(\R^n)}
\def\Laplacian{\Delta_{\boldsymbol x}}
\def\ext{V}
\def\indexRiccati{j}
\def\pdhgdual{\boldsymbol{w}}
\def\pdhgdualscal{w}
\def\HHJ {H}
\def\JHJ {J}
\def\LNN {L}
\def\JNN {\tilde{J}}
\def\SLO {S_{LO}}
\def\ind {I}
\def\proj{\Pi}


%%%%%%% macro for LQR %%%%%%%%
\newcommand{\Cpp}{C_{pp}} % p^2 term in H
\newcommand{\Cxp}{C_{xp}} % xp term in H
\newcommand{\Cxx}{C_{xx}} % x^2 term in H
\newcommand{\Sxx}{P} % x^2 term in S
\newcommand{\Sx}{\mathbf{q}} % x term in S
\newcommand{\Sc}{r} % const term in S

\newcommand{\update}[1]{\textcolor{red}{#1}} 

%%% USE THESE MACROS FOR REVISIONS %%%%
\newcommand{\updateone}[1]{\textcolor{black}{#1}}  % reviewer 1
\newcommand{\updatetwo}[1]{\textcolor{black}{#1}} % reviewer 2
\newcommand{\updatethree}[1]{\textcolor{black}{#1}} % reviewer 3
\newcommand{\updateeditor}[1]{\textcolor{black}{#1}} % editor

%%%%%%%%%%%%%%%%%%%%%%%%%%

% MACROS
\def\lossfunc{\mathcal{L}}
\def\NNfunc{f}
\def\weight{\theta}
\def\weightvec{{\boldsymbol{\weight}}}
\def\NNinput{\mathbf{z}}
\def\NNoutput{\mathbf{y}}
\def\regfunc{R}
\def\param{\lambda}
\def\HJmom{\mathbf{p}}
\def\HJx{\mathbf{x}}
\def\HJt{t}
\def\HJu{\mathbf{u}}
\def\Hamiltonian{\HHJ}
\def\HJIC{\JHJ}
\def\LPinspace{\R^M}
\def\LPoutspace{\R^m}
\def\weightspace{\Rn}
\def\HJstatespace{\weightspace}
\def\numt{N}
\def\LQRxx{\mathcal{Q}}
\def\LQRuu{\mathcal{R}}
\def\LQRxu{\mathcal{N}}
\def\LQRTC{\mathcal{Q}_f}
\def\LQRA{A} % linear dynamics
\def\LQRB{B}
% MACROS FOR LEARNING PROBLEMS
\def\MLregmat{\Gamma}
\def\MLreg{\gamma}    
\def\MLbasismat{\Phi}
\def\MLbasis{\phi}
\def\MLcentervec{\weightvec^0}
\def\MLcenter{\weight^0}
\def\MLcenternewvec{\tilde\weightvec^0}
\def\MLcenternew{\tilde\weight^0}
\def\MLy{\by}

\begin{document}
\nolinenumbers
\renewcommand{\thefootnote}{\fnsymbol{footnote}}

\footnotetext[1]{Division of Applied Mathematics, Brown University, Providence, RI 02912, USA (paula\_chen@alumni.brown.edu, zongren\_zou@brown.edu, jerome\_darbon@brown.edu, george\_karniadakis@brown.edu).}
\footnotetext[2]{Department of Mathematics, UCLA, Los Angeles, CA 90025, USA (\email{tingwei@math.ucla.edu}).}
\footnotetext[3]{Pacific Northwest National Laboratory, Richland, WA 99354, USA}
\footnotetext[4]{Paula Chen, Tingwei Meng, and Zongren Zou contributed equally to this work.}
\footnotetext[5]{Corresponding author.}
\renewcommand{\thefootnote}{\arabic{footnote}}


\maketitle %modify in ex_shared.tex

% REQUIRED
\begin{abstract}
Hamilton-Jacobi partial differential equations (HJ PDEs) have deep connections with a wide range of fields, including optimal control, differential games, and imaging sciences. By considering the time variable to be a higher dimensional quantity, HJ PDEs can be extended to the multi-time case. In this paper, we establish a novel theoretical connection between specific optimization problems arising in machine learning and the multi-time Hopf formula, which corresponds to a representation of the solution to certain multi-time HJ PDEs. Through this connection, we increase the interpretability of the training process of certain machine learning applications by showing that when we solve these learning problems, we also solve a multi-time HJ PDE and, by extension, its corresponding optimal control problem. As a first exploration of this connection, we develop the relation between the regularized linear regression problem and the Linear Quadratic Regulator (LQR). We then leverage our theoretical connection to adapt standard LQR solvers (namely, those based on the Riccati ordinary differential equations) to design new training approaches for machine learning. Finally, we provide some numerical examples that demonstrate the versatility and possible computational advantages of our Riccati-based approach in the context of continual learning, post-training calibration, transfer learning, and sparse dynamics identification. 
\end{abstract}

% REQUIRED
\begin{keywords}
Multi-time Hamilton-Jacobi PDEs; Hopf formula; machine learning; linear quadratic regulator; linear regression; Riccati equation
\end{keywords}

% REQUIRED
\begin{MSCcodes}
35F21, 49N05, 49N10, 68T05, 35B37
\end{MSCcodes}


% for tikz figures
\tikzstyle{box} = [rectangle, rounded corners, minimum width=0cm, minimum height=1cm,text centered, draw=black, fill=none]
\tikzstyle{boxsmall} = [rectangle, rounded corners, minimum width=0cm, minimum height=0cm,text centered, draw=black, fill=none]
\tikzstyle{nobox} = [rectangle, rounded corners, minimum width=0cm, minimum height=1cm,text centered, draw=none, fill=none]
\tikzstyle{doublearrow} = [thick,<->,>=stealth]
\tikzstyle{dottedarrow} = [thick,<->,>=stealth,dotted]



\section{Introduction}
\label{sec:introduction}
% \begin{itemize}
%     % Diffusion of FL
%     \item {\st{Diffusion of FL}}
%     % Security threats to FL
%     \item {\st{Security threats to FL with particular focus on model poisoning}}
%     % Limitations of existing countermeasures
%     \item {\st{Current countermeasures (e.g., KRUM) and their limitations}}
%     % Proposed method and its advantages
%     \item {\st{Intuitive description of the proposed method and its difference (i.e., advantages) w.r.t. state of the art}}
%     % Main contributions
%     \item {\st{Summary of the main contributions of this work}}
%     % Paper's structure and organization
%     \item {\st{Paper's structure and organization}}
% \end{itemize}

% Diffusion of FL
Recently, {\em federated learning} (FL) has emerged as the leading paradigm for training distributed, large-scale, and privacy-preserving machine learning (ML) systems~\cite{mcmahan2017googleai,mcmahan2017aistats}. 
The core idea of FL is to allow multiple edge clients to collaboratively train a shared, global model without disclosing their local private training data.
%Specifically, an FL system consists of a central server and many edge clients; 
A typical FL round involves the following steps: {\em(i)} the server randomly picks some clients and sends them the current, global model; {\em(ii)} each selected client locally trains its model with its own private data; then, it sends the resulting local model to the server;\footnote{Whenever we refer to global/local model, we mean global/local model {\em parameters}.} {\em(iii)} the server updates the global model by computing an \emph{aggregation function}, usually the average (FedAvg), on the local models received from clients.
% \begin{enumerate}
%     \item[{\em(i)}] the server sends the current, global model to the clients and appoints some of them for training;
%     \item[{\em(ii)}] each selected client locally trains its copy of the global model with its own private data; then, it sends the resulting local model back to the server;\footnote{Whenever we refer to global/local model, we mean global/local model {\em parameters}.}
%     \item[{\em(iii)}] the server updates the global model by computing an \emph{aggregation function} on the local models received from clients (by default, the average, also referred to as FedAvg~\cite{mcmahan2017aistats}).
% \end{enumerate}
This process goes on until the global model converges. %(e.g., after a certain number of rounds or other similar stopping criteria).
%\\
% The advantages of FL over the traditional, centralized learning paradigm are undoubtedly clear in terms of flexibility/scalability (clients can join/disconnect from the FL network dynamically), network communications (only model weights\footnote{We will use \textit{parameters} and \textit{weights} interchangeably.} are exchanged between clients and server), and privacy (each client's private training data is kept local at the client's end and not uploaded to the server).
\\
% Security threats to FL
%However, the growing adoption of FL also raises security concerns~\cite{costa2022covert}, particularly about its confidentiality, integrity, and availability.
Although its advantages over standard ML, FL also raises security concerns~\cite{costa2022covert}. %, particularly about its confidentiality, integrity, and availability~\cite{costa2022covert}.
% OLD, LONG VERSION
% Indeed, some work deals with privacy leakage that may expose the local data of some clients~\cite{melis2019sp}. 
% A large body of work, instead, investigates attacks that usually aim to detriment the predictive accuracy of the learned global model. For instance, \emph{data poisoning} attacks achieve this goal by letting an adversary pollute the training set of some corrupt FL clients with maliciously crafted examples~\cite{jagielski2018sp}.
% Similarly, in \emph{model poisoning} the attacker attempts to tweak the global model weights~\cite{bhagoji2019pmlr} by directly perturbing the local model's weights of some infected FL clients before these are sent to the central server for aggregation, usually via so-called Byzantine attacks. 
% It turns out that Byzantine model poisoning attacks severely impact standard FedAvg; therefore, more robust aggregation functions must be designed to make FL systems secure.
Here, we focus on \emph{untargeted model poisoning} attacks~\cite{bhagoji2019pmlr}, where an adversary attempts to tweak the global model weights %\footnote{We will use the terms \textit{parameters} and \textit{weights} interchangeably.} 
by directly perturbing the local model's parameters of some infected clients before these are sent to the central server for aggregation.
In doing so, the adversary aims to jeopardize the global model \textit{indiscriminately} at inference time.
Such model poisoning attacks severely impact standard FedAvg; therefore, more robust aggregation functions must be designed to secure FL systems.
\\
% In this paper, we focus on designing a novel robust aggregation scheme at the server's end to contrast the effect of Byzantine model poisoning attacks.
%
% Current countermeasures and their limitations
%Several countermeasures have been proposed in the literature to combat model poisoning attacks on FL systems.
% Some methods use simple statistics more robust than plain average to smooth the impact of malicious updates (e.g., Trimmed Mean and FedMedian~\cite{yin2018icml}). 
% Other defenses implement outlier detection techniques to discard malicious updates from the aggregation performed at the server's end. Those are either based on heuristics (e.g., Krum/Multi-Krum~\cite{blanchard2017nips} and Bulyan~\cite{mhamdi2018pmlr}) or data-driven approaches (e.g., K-means clustering~\cite{shen2016acm} or DnC via spectral analysis~\cite{shejwalkar2021ndss}). 
% Finally, some strategies rely on a centralized ``source of trust'' to spot potential malicious updates (e.g., FLTrust~\cite{cao2020fltrust}).
% Several countermeasures have been proposed in the literature to combat model poisoning attacks on FL systems, i.e., to discard possible malicious local updates from the aggregation performed at the server's end. 
% These techniques range from simple statistics more robust than plain average (e.g., Trimmed Mean and FedMedian~\cite{yin2018icml}) to outlier detection heuristics (e.g., Krum/Multi-Krum~\cite{blanchard2017nips} and Bulyan~\cite{mhamdi2018pmlr}) or data-driven approaches (e.g., spectral analysis via K-means clustering~\cite{shen2016acm} or spectral analysis), or methods based on ``source of trust'' (e.g., FLTrust~\cite{cao2020fltrust}).
% OLD, LONG VERSION
%Several countermeasures have been proposed in the literature to combat Byzantine model poisoning attacks on FL systems.
% Descriptive statistics
% For example, Trimmed Mean and FedMedian aggregate local model updates using more robust statistics than standard average~\cite{yin2018icml}.
%
% % Heuristics for outlier detection
% Many existing Byzantine-resilient strategies implement some outlier detection heuristics to discard the model updates sent by potentially malicious clients from the input of the aggregation function.
% One of the most popular heuristics is Krum~\cite{blanchard2017nips}.
% This strategy tries to mitigate the impact of Byzantine attacks by selecting as a global model the local model with the smallest sum of Euclidean distances to {\em all} the other local models.
% Although powerful, Krum requires the server to know (or, at least, estimate) the number of malicious FL clients upfront, which is generally impossible in a realistic attack scenario. %
% Moreover, Krum may become ineffective for complex, high-dimensional model parameter spaces due to the curse of dimensionality.
% Bulyan~\cite{mhamdi2018pmlr} tries to overcome this issue by combining Krum with a variant of Trimmed Mean.
% % Data-driven outlier detection
% Other strategies use data-driven outlier detection techniques -- e.g., via K-means clustering~\cite{shen2016acm} -- to spot potential malicious local model updates. 
% %For instance, Shen et al. propose to cluster local model updates with K-means and thus identify outliers.
%
% % Other techniques
% As far as the server is concerned, any local model received can be from a potential malicious client. 
% FLTrust~\cite{cao2020fltrust} assumes the server acts as a client, i.e., trains a local model on an additional {\em trustworthy} dataset at the server's end and compares it against all the local models from other clients. 
% This way, the server can rely on some ``source of trust'' when discarding potentially malicious clients.
%\\
% Limitations of existing Byzantine-resilient strategies
Unfortunately, existing defense mechanisms either rely on simple heuristics (e.g., Trimmed Mean and FedMedian by~\cite{yin2018icml}) or need strong and unrealistic assumptions to work effectively (e.g., foreknowledge or estimation of the number of malicious clients in the FL system, as for Krum/Multi-Krum~\cite{blanchard2017nips} and Bulyan~\cite{mhamdi2018pmlr}, which, however, cannot exceed a fixed threshold).
Furthermore, outlier detection methods using K-means clustering~\cite{shen2016acm} or spectral analysis like DnC~\cite{shejwalkar2021ndss} do not directly consider the temporal evolution of local model updates received.
Finally, strategies like FLTrust~\cite{cao2020fltrust} require the server to collect its own dataset and act as a proper client, thereby altering the standard FL protocol.
\\
% OLD, LONG VERSION
% Overall, existing Byzantine-resilient strategies are either simple heuristics (e.g., FedMedian) or, if they are more complex, they rely on strong and unrealistic assumptions to work effectively (e.g., knowing the number of malicious clients in the FL system in advance, as for Krum and alike).
% Furthermore, data-driven outlier detection methods do not consider the temporary evolution of local model updates received (e.g., K-means clustering). 
% Finally, strategies like FLTrust requires the server to collect its own dataset and act as a proper client, thereby altering the standard FL protocol.
%
% Description of the proposed method
This work introduces a novel pre-aggregation \textit{filter} robust to untargeted model poisoning attacks. Notably, this filter $(i)$ operates without requiring prior knowledge or constraints on the number of malicious clients and $(ii)$ inherently integrates temporal dependencies. 
The FL server can employ this filter as a preprocessing step before applying \textit{any} aggregation function, be it standard like FedAvg or robust like Krum or Bulyan.
Specifically, we formulate the problem of identifying corrupted updates as a multidimensional (i.e., matrix-valued) time series anomaly detection task. 
The key idea is that legitimate local updates, resulting from well-calibrated iterative procedures like stochastic gradient descent (SGD) with an appropriate learning rate, show \textit{higher predictability} compared to malicious updates. This hypothesis stems from the fact that the sequence of gradients (thus, model parameters) observed during legitimate training exhibit regular patterns, as validated in Section~\ref{subsec:intuition}. %until convergence. 
%This regularity may be more pronounced for smooth convex loss functions, but it can still be captured within an appropriate time window, even for more complex and convoluted loss surfaces. 
%We provide evidence of this claim in Appendix~B, where we show that the average mutual information (i.e., ``predictability''), calculated over pairs of legitimate model updates sent at different FL rounds, is significantly higher than the corresponding computation for a malicious client.
\\
Inspired by the matrix autoregressive (MAR) framework for multidimensional time series forecasting~\cite{chen2021je}, we propose the FLANDERS ({\em \textbf{F}ederated \textbf{L}earning meets \textbf{AN}omaly \textbf{DE}tection for a \textbf{R}obust and \textbf{S}ecure}) filter.
The main advantages of FLANDERS over existing strategies like FLDetector~\cite{zhao2020multivariate} are its resilience to large-scale attacks, where $50\%$ or more FL participants are hostile, and the capability of working under realistic non-iid scenarios.
We attribute such a capability to two key factors: $(i)$ FLANDERS works without knowing a priori the ratio of corrupted clients, and $(ii)$ it embodies temporal dependencies between intra- and inter-client updates, quickly recognizing local model drifts caused by evil players. Below, we summarize our main contributions:

\begin{itemize}
\item[{\em(i)}]
We provide empirical evidence that the sequence of models sent by legitimate clients is more predictable than those of malicious participants performing untargeted model poisoning attacks.
\\
\item[{\em(ii)}] 
We introduce FLANDERS, the first pre-aggregation filter for FL robust to untargeted model poisoning based on multidimensional time series anomaly detection.
\\
\item[{\em(iii)}] 
We integrate FLANDERS into Flower,\footnote{\scriptsize{\url{https://flower.dev/}}} a popular FL simulation framework for reproducibility.
\\
\item[{\em(iv)}] 
We show that FLANDERS improves the robustness of the existing aggregation methods under multiple settings: different datasets, client's data distribution (non-iid), models, and attack scenarios.
\\
\item[{\em(v)}] 
We publicly release all the implementation code of FLANDERS along with our experiments.\footnote{\scriptsize{\url{https://anonymous.4open.science/r/flanders_exp-7EEB}}}
\end{itemize}

% Paper's structure and organization
The remainder of the paper is structured as follows. %some related work and the current state-of-the-art solutions to security issues that FL entails. 
Section~\ref{sec:background} covers background and preliminaries. 
In Section~\ref{sec:related}, we discuss related work.
Section~\ref{sec:problem} and Section~\ref{sec:method} describe the problem formulation and the method proposed. % to tackle it. 
Section~\ref{sec:experiments} gathers experimental results. %, and Section~\ref{sec:limitations} discusses some limitations of this work.
Finally, we conclude in Section~\ref{sec:conclusion}.
 %discusses the limitations of this work and draws future research directions.
%reports conclusions and draws perspectives for future research directions.

%%%%%%% OLD %%%%%%%
%to overcome the resilience of Byzantine failures in distributed Stochastic Gradient Descent computations. 
% The strength of Krum is its time complexity, which is linear in the gradient dimension. 
% However, the robustness of the approach is guaranteed for gradient-based learning applications only when the majority of the clients are not compromised. 
% Besides, the aggregation mechanism of Krum, as well as that of similar methods, is robust from a coarse-grained perspective and does not provide solutions to errors and perturbations that may occur at inference time.
%A related approach to~\cite{blanchard2017nips} is the work of Su et al.~\cite{su2016dc}. Here, the authors propose an iterated approximate agreement to tackle a multi-layer scenario attacked by Byzantine agents. 
%However, the method works efficiently on the sole discrete context and it is inapplicable to continuous state environments.
%\gabri{Maybe, we should just talk about the main limitations of existing countermeasures without digging into their details (or, we can just mention Krum as this is the most popular one). I will move the description of all these methods to the Related Work section.}
\section{Generalized Hopf formula}\label{sec:Hopf}
In this section, we provide some mathematical background on the single- and multi-time Hopf formulas. Specifically, we review the well-known connections between the Hopf formula, the solution to the HJ PDEs, and the solution to the corresponding optimal control problems. We then present a novel theoretical connection between the Hopf formula and certain learning problems. Through this connection, we establish that when we solve certain learning problems, we are actually evaluating the solution to certain HJ PDEs and their corresponding optimal control problems, and vice versa.

\subsection{Introduction to the Hopf formula}

The single-time HJ PDE is 
\begin{equation}\label{eqt:singletimeHJPDE}
    \begin{dcases}
    \frac{\partial S(\HJx,\HJt)}{\partial \HJt} + \Hamiltonian(\nabla_\HJx S(\HJx,\HJt)) = 0 & \HJx\in \HJstatespace, \HJt > 0, \\
    S(\HJx,0) = \HJIC(\HJx) & \HJx\in\HJstatespace,
    \end{dcases}
\end{equation}
where $\Hamiltonian:\Rn\to \R$ is the Hamiltonian and $\HJIC:\Rn\to \R$ is the initial condition. Assume that $\Hamiltonian$ and $\HJIC$ are convex. Then, the viscosity solution to the single-time HJ PDE~\eqref{eqt:singletimeHJPDE} is given by the Hopf formula \cite{hopf1965hopfformula}:
\begin{equation}\label{eqt:singletime_Hopf}
\begin{aligned}
S(\HJx,\HJt) & = \sup_{\HJmom\in\HJstatespace} \{\langle \HJx,\HJmom\rangle - \HJt \Hamiltonian(\HJmom) - \HJIC^*(\HJmom)\} = -\inf_{\HJmom\in\HJstatespace} \{ \HJt \Hamiltonian(\HJmom) + \HJIC^*(\HJmom)-\langle \HJx,\HJmom\rangle\},
\end{aligned}
\end{equation}
where $f^*$ denotes the Fenchel-Legendre transform of the function $f$; i.e., $f^*(p) = \sup_{x\in\Rn} \{\langle x,p\rangle - f(x)\}$.

The Hopf formula~\eqref{eqt:singletime_Hopf} also solves the following optimal control problem: 
\begin{equation}\label{eqt:optimal_control_standardform}
\min_{\bx(\cdot)} \left\{\int_{0}^{\HJt}L(\HJu(s)) ds + \HJIC(\bx(\HJt))\colon \dot\bx(s) = f(\HJu(s)) \forall s\in(0,\HJt], \bx(0) = \HJx\right\},
\end{equation}
where the running cost $L$ and the source term $f$ of the dynamics are related to the Hamiltonian $\Hamiltonian$ by $\Hamiltonian(\HJmom) = \sup_{\HJu\in\Rn} \{\langle -f(\HJu), \HJmom\rangle - L(\HJu)\}$ and, in this context, we interpret $\HJIC$ to be the terminal cost.


A natural generalization of this formulation to the multi-time case is as follows. Let $\Hamiltonian_1,\dots, \Hamiltonian_\numt$ be convex Hamiltonians, such that $\dom \Hamiltonian_i = \Rn$ for all $i = 1, \dots, \numt$, and let $\HJIC$ be a convex initial condition. Then, the multi-time HJ PDE is given by
\begin{equation}\label{eqt:multitimeHJPDE}
    \begin{dcases}
    \frac{\partial S(\HJx,\HJt)}{\partial \HJt_i} + \Hamiltonian_i(\nabla_\HJx S(\HJx,\HJt)) = 0 \text{ for } i\in\{1, \dots, \numt\} & \HJx\in \HJstatespace, \HJt_1, \dots, \HJt_\numt > 0, \\
    S(\HJx,0, \dots, 0) = \HJIC(\HJx) & \HJx\in\HJstatespace,
    \end{dcases}
\end{equation}
and the solution to the multi-time HJ PDE~\eqref{eqt:multitimeHJPDE} can be represented by the following generalized (multi-time) Hopf formula \cite{lions1986hopf}:
\begin{equation} \label{eqt:multitime_Hopf}
\begin{aligned}
S(\HJx,\HJt_1,\dots, \HJt_\numt) & = \sup_{\HJmom\in\HJstatespace} \left\{\langle \HJx, \HJmom\rangle - \sum_{i=1}^\numt \HJt_i\Hamiltonian_i(\HJmom) - \HJIC^*(\HJmom)\right\} \\
&= -\inf_{\HJmom\in\HJstatespace} \left\{ \sum_{i=1}^\numt \HJt_i\Hamiltonian_i(\HJmom) + \HJIC^*(\HJmom) -\langle \HJx, \HJmom\rangle \right\}.\\
\end{aligned}
\end{equation}



Moreover, the generalized Hopf formula~\eqref{eqt:multitime_Hopf} also solves an optimal control problem in the form of~\eqref{eqt:optimal_control_standardform}
with terminal time $\HJt = \sum_{j=1}^\numt \HJt_j$ and where the running cost $L$ and the source term $f$ of the dynamics are defined piecewise by $L(s, \HJu) = L_i(\HJu)$ and $f(s, \HJu) = f_i(\HJu)$, respectively, for $s \in \left(\sum_{j=1}^{i-1}\HJt_j, \sum_{j=1}^{i}\HJt_j\right]$ and $i= 1, \dots, \numt$. The piecewise running costs $L_i$ and piecewise source terms $f_i$ of the dynamics are related to the Hamiltonians $\Hamiltonian_i$ by $\Hamiltonian_i(\HJmom) = \sup_{\HJu\in\Rn} \{\langle -f_i(\HJu), \HJmom\rangle - L_i(\HJu)\}$.




\subsection{Connection between the Hopf formula and learning problems}\label{sec:general_connection_Hopf}

\begin{figure}[htbp]
    \centering   
    \begin{adjustbox}{width=\textwidth}
\begin{tikzpicture}[node distance=2cm]
    \node (min) [nobox, yshift=-0.2cm] {$\min$};
    \node (minarg) [boxsmall, right of=min, xshift=-1.5cm, yshift=-0.37cm, draw=cyan!60, fill=cyan!5] {$\text{}_{\weightvec\in\weightspace}$};
    \node (sum) [nobox, right of=min, xshift=-0.5cm] {$\sum_{i=1}^\numt$};
    \node (param) [box, right of=sum, xshift=-1.25cm, draw=magenta!60, fill=magenta!5]{$\param_i$};
    \node (loss) [box, right of=param, xshift=-0.2cm, draw=green!60, fill=green!5] {$\lossfunc_i(\mathcal{A}F(\bz_i;\HJmom), \by_i)$};
    \node (plus) [nobox, right of=loss, xshift=1.03cm] {$+$};
    \node (regLeft) [nobox, right of=loss, xshift=1.1cm] {};
    \node (regularization) [box, right of=regLeft, xshift=0.1cm, draw=blue!60, fill=blue!5] {$\regfunc(\weightvec)$};
    \node (regRight) [nobox, right of=regularization, xshift=-0.1cm] {};
    
    \node (sup) [nobox, below of=min, xshift=-0cm] {$\min$};
    \node (suparg) [boxsmall, right of=sup, xshift=-1.5cm, yshift=-0.4cm, draw=cyan!60, fill=cyan!5] {$\text{}_{\HJmom\in\HJstatespace}$};
    \node (sum2) [nobox, below of=sum] {$\sum_{i=1}^\numt$};
    \node (equal) [nobox, left of=sup, xshift=1.4cm] {$=$};
    \node (S) [box, left of=equal, xshift=0.45cm, draw=red!60, fill=red!5] {$S(\HJx, \HJt_1, \dots, \HJt_n)$};
    \node (minus) [nobox, left of=S, xshift=0.5cm] {$-$};
    \node (time) [box, below of=param, draw=magenta!60, fill=magenta!5]{$\HJt_i$};
    \node (Hamiltonian) [box, below of=loss, draw=green!60, fill=green!5] {$\Hamiltonian_i(\HJmom)$};
    \node (plus2) [nobox, right of=Hamiltonian, xshift=-0.44cm] {$+$};
    \node (IC) [box, below of=regLeft, draw=blue!60, fill=blue!5] {$\HJIC^*(\HJmom)$};
    \node (linear) [box, below of=regRight, draw=blue!60, fill=blue!5] {$-\langle \HJx, \HJmom\rangle$};

    \node (OCmin) [nobox, below of=sup, xshift=0.2cm] {$\min$};
    \node (OCS) [box, below of=S, xshift=0cm, draw=red!60, fill=red!5] {$S(\HJx, \HJt_1, \dots, \HJt_n)$};
    \node (leftbracket) [nobox, right of=OCmin, xshift=-1.3cm] {$\Bigg\{$};
    \node (OCequal) [nobox, below of=equal] {$=$};
    \node (OCint) [nobox, below of=sum2, xshift=0.05cm] {$\int_{0}^{\sum_{i=1}^{\numt}}$};
    \node (OCintupperbd) [boxsmall, below of=time, yshift=0.15cm, draw=magenta!60, fill=magenta!5] {${}^{\HJt_i}$};
    \node (OCHamiltonian) [box, below of=Hamiltonian, draw=green!60, fill=green!5] {$L(\HJu(s)) $};
    \node (OCds) [nobox, right of=OCHamiltonian, xshift=-1cm] {$ds$};
    \node (OCplus) [nobox, right of=OCds, xshift=-1.6cm] {$+$};
    \node (OCIC) [box, below of=IC, draw=blue!60, fill=blue!5] {$\HJIC\left(\bx\left(\sum_{i=1}^{\numt}\HJt_i\right)\right)$};
    \node (colon) [nobox, right of=OCIC, xshift=-0.5cm] {:};
    \node(OCdynamics) [box, right of=OCIC, xshift=2.4cm, draw=green!60, fill=green!5] {$\dot\bx(s) = f(\HJu(s)), \forall s \in \left(0, \sum_{i=1}^{\numt}\HJt_i\right]$};
    \node(comma) [nobox, right of=OCdynamics, xshift=0.8cm, yshift=-0.11cm] {,};
    \node (terminalposition) [box, right of=OCdynamics, xshift=1.8cm, draw=blue!60, fill=blue!5] {$\bx\left(0\right) = \HJx$};
    \node (rightbracket) [nobox, right of=terminalposition, xshift=-0.9cm] {$\Bigg\}$};

    \node (LPequal) [nobox, above of=equal] {$=$};
    \node (minloss) [box, above of=S, draw=red!60, fill=red!5] {$\min_{\weightvec\in\weightspace}\mathcal{L}(\weightvec)$};
    
    \draw [doublearrow] (param) -- (time);
    \draw [doublearrow] (loss) -- (Hamiltonian);
    \draw [dottedarrow] (regularization) -- (IC);
    \draw [dottedarrow] (regularization) -- (linear);
    \draw [doublearrow] (minarg) -- (suparg);
    \draw [doublearrow] (S) -- (OCS);
    \draw [doublearrow] (time) -- (OCintupperbd);
    \draw [dottedarrow] (IC) -- (OCIC);
    \draw [dottedarrow] (Hamiltonian) -- (OCHamiltonian);
    \draw [dottedarrow] (Hamiltonian) -- (OCdynamics);
    \draw [dottedarrow] (linear) -- (terminalposition);
    \draw[dottedarrow](minloss) -- (S);
\end{tikzpicture}
\end{adjustbox}

    \caption{(See Section~\ref{sec:Hopf}) Mathematical formulation describing the connection between a regularized learning problem (\textbf{top}), the multi-time Hopf formula (\textbf{middle}), and an optimal control problem (\textbf{bottom}). The content of this illustration matches that of Figure~\ref{fig:intro_connection_in_words} by replacing each term in Figure~\ref{fig:intro_connection_in_words} with its corresponding mathematical expression. The colors indicate the associated quantities between each problem. The solid-line arrows denote direct equivalences. The dotted arrows represent additional mathematical relations.}
    \label{fig:connection_multitimeHopf}
\end{figure}

In this section, we connect the Hopf formulas~\eqref{eqt:singletime_Hopf} and~\eqref{eqt:multitime_Hopf} with certain learning problems. Consider a learning problem with data points $\{(\bz_i,\by_i)\}_{i=1}^\numt \subset \LPinspace\times \LPoutspace$. The goal of the learning problem is to find a function $F(\bz;\weightvec)$ with input $\bz\in\LPinspace$ and unknown parameter $\weightvec\in\weightspace$, such that $\mathcal{A}F(\bz_i;\weightvec)$ approximately equals $\by_i$ at every $\bz_i$. Here $\mathcal{A}$ is an operator acting on the function $F$. 
For instance, $\mathcal{A}$ could be the identity operator (as in the regression problem \cite{weisberg2005applied}) or a differential operator (as in PINNs \cite{raissi2019physics}).
Then, the learning problem is given by the following optimization problem:
\begin{equation}\label{eqt:general_learning}
\min_{\weightvec\in \weightspace} \sum_{i=1}^\numt \lambda_i\lossfunc_i(\mathcal{A}F(\bz_i;\weightvec), \by_i) + \regfunc(\weightvec).
\end{equation}
The above loss function consists of two parts: each of $\lossfunc_i(\mathcal{A}F(\bz_i;\weightvec), \by_i)$ is a data fitting term at $(\bz_i, \by_i)$ (where $\lossfunc_i(\ba,\bb)$ is a function measuring the discrepancy between $\ba$ and $\bb$) and $\regfunc$ is a regularization term. In this paper, we assume the function $\weightvec \mapsto \lossfunc_i(\mathcal{A}F(\bz_i;\weightvec), \by_i)$ is convex for all $i=1,\dots, \numt$.

Then, the connection between the learning problem~\eqref{eqt:general_learning}, the Hopf formulas~\eqref{eqt:singletime_Hopf} and~\eqref{eqt:multitime_Hopf}, and the optimal control problem~\eqref{eqt:optimal_control_standardform} is illustrated in Figure~\ref{fig:connection_multitimeHopf}. 
Specifically, if there is only one data point ($\numt =1$), the learning problem~\eqref{eqt:general_learning} is related to the single-time Hopf formula~\eqref{eqt:singletime_Hopf} by setting $\Hamiltonian(\HJmom) = \lossfunc_1(\mathcal{A}F(\bz_1;\HJmom), \by_1)$, $\HJt = \param_1$, and $\regfunc(\HJmom) = \HJIC^*(\HJmom) - \langle \HJx,\HJmom\rangle$. Hence, when we solve these single-point learning problems, we simultaneously evaluate the solution to the HJ PDE~\eqref{eqt:singletimeHJPDE} at the point $(\HJx,\HJt)$, or equivalently, we solve the corresponding optimal control problem~\eqref{eqt:optimal_control_standardform}. Conversely, when we solve the HJ PDE~\eqref{eqt:singletimeHJPDE}, the spatial gradient $\nabla_\HJx S(\HJx, \HJt)$ of the solution gives the minimizer $\weightvec^*$ to the single-point learning problem~\eqref{eqt:general_learning}.

If there are $\numt > 1$ data points, then the learning problem~\eqref{eqt:general_learning} is related to the multi-time Hopf formula~\eqref{eqt:multitime_Hopf} by setting $\Hamiltonian_i(\HJmom) = \lossfunc_i(\mathcal{A}F(\bz_i;\HJmom), \by_i)$, $\HJt_i = \param_i$, and $\regfunc(\HJmom) = \HJIC^*(\HJmom) - \langle \HJx,\HJmom\rangle$. 
Hence, solving these multi-point learning problems is equivalent to evaluating the solution to the HJ PDE~\eqref{eqt:multitime_Hopf} at the point $(\HJx,\HJt_1, \dots, \HJt_\numt)$, which is also equivalent to solving the corresponding optimal control problem~\eqref{eqt:optimal_control_standardform} with terminal time $\HJt = \sum_{j=1}^\numt \HJt_j$ and piecewise running costs $L_i$ and dynamics $f_i$ related via $\Hamiltonian_i(\HJmom) = \sup_{\HJu\in\Rn} \{\langle -f_i(\HJu), \HJmom\rangle - L_i(\HJu)\}$. Similarly to the single-time case, when we solve the multi-time HJ PDE~\eqref{eqt:multitimeHJPDE}, the spatial gradient $\nabla_\HJx S(\HJx, \HJt_1, \dots, \HJt_\numt)$ of the solution gives the minimizer $\weightvec^*$ to the multi-point learning problem~\eqref{eqt:general_learning}.








\section{Linear Quadratic Regulator}\label{sec:LQR}
In this section, we develop our theoretical connection from Section~\ref{sec:general_connection_Hopf} in the specific case where the optimal control problem~\eqref{eqt:optimal_control_standardform} corresponds to the LQR problem  \cite{TrentelmanControlLinearSystems, anderson2007optimal}. We show that solving certain LQR problems is equivalent to solving learning problems with linear models, quadratic data fitting losses, and quadratic regularization (i.e., an $\ell_2$-regularized linear regression problem). Although broader classes of learning problems are of interest, we restrict to linear regression problems as a starting point for demonstrating the potential of our theoretical connection. Specifically, we leverage our theoretical connection to show how established techniques for solving HJ PDEs (e.g., the Riccati ODEs \cite{mceneaney2006max}) can be reused to solve this class of learning problems. 

\subsection{Introduction to the Linear Quadratic Regulator and Riccati equation}\label{sec:intro_LQR}

The finite-horizon, continuous-time LQR is given by
\begin{multline}\label{eqt:LQR_general_control}
    S(\HJx,\HJt) = \min_{\updateone{\HJu(\cdot)}} \Bigg\{\int_0^\HJt \left(\frac{1}{2}\bx(s)^T\LQRxx\bx(s) + \frac{1}{2}\HJu(s)^T\LQRuu\HJu(s) + \bx(s)^T\LQRxu\HJu(s)\right) ds \\ + \frac{1}{2}\bx(\HJt)^T\LQRTC\bx(\HJt): 
    \dot\bx(s) = \LQRA\bx(s) + \LQRB\HJu(s) \forall s\in(0,t], \bx(0) = \HJx\Bigg\},
\end{multline}
where $\LQRxx, \LQRTC \in\R^{n\times n}$ and $\LQRuu\in\R^{m\times m}$ are symmetric positive definite, $\LQRxu\in\R^{n\times m}$, $\LQRA\in\R^{n\times n}$, and $\LQRB\in\R^{n\times m}$. The corresponding HJ PDE is 
\begin{equation}\label{eqt:LQR_HJPDE}
    \begin{dcases}
    \frac{\partial S(\HJx,\HJt)}{\partial \HJt} + \Hamiltonian(\HJx,\nabla_\HJx S(\HJx,\HJt)) = 0 & \HJx\in\R^n, \HJt > 0, \\
    S(\HJx,0) = \HJIC(\HJx)  & \HJx\in\R^n,
    \end{dcases}
\end{equation}
where the initial data of the HJ PDE is given by the terminal cost $\HJIC(\HJx) := \frac{1}{2}\HJx^T\LQRTC\HJx$ of the optimal control problem and the Hamiltonian $\Hamiltonian$ is defined by
\begin{equation}
    \begin{aligned}
    \Hamiltonian(\HJx,\HJmom) & = \sup_{\HJu\in\R^m} \langle -f(\HJx,\HJu), \HJmom \rangle - L(\HJx,\HJu) \\
     &=  -\langle \LQRA\HJx, \HJmom\rangle - \frac{1}{2}\langle \HJx, \LQRxx\HJx\rangle + \frac{1}{2}\langle \LQRB^T\HJmom + \LQRxu^T\HJx, \LQRuu^{-1}(\LQRB^T\HJmom + \LQRxu^T\HJx) \rangle,
    \end{aligned}
\end{equation}
where $f(\HJx,\HJu) = \LQRA\HJx + \LQRB\HJu$ is the source term of the dynamics and $L(\HJx,\HJu) = \frac{1}{2}\HJx^T\LQRxx\HJx + \frac{1}{2}\HJu^T\LQRuu\HJu + \HJx^T\LQRxu\HJu$ is the running cost. Note that because of the spatial dependence in the Hamiltonian $\Hamiltonian$, the Hopf formula cannot be applied directly to the above LQR problem without additional assumptions. In Sections~\ref{sec:LQR_1ptregression} and~\ref{sec:LQR_multiptregression}, we discuss some assumptions under which $\Hamiltonian$ is independent of the spatial variable $\HJx$ and the corresponding learning problems that can be solved via our connection through the Hopf formula (see, for example, Figure~\ref{fig:connection_LQR}).


It is well-known that this LQR problem can be solved using the Riccati equation as follows \cite{mceneaney2006max}. Define $\Cpp = \LQRB \LQRuu^{-1}\LQRB^T$, $\Cxx = -\LQRxu\LQRuu^{-1}\LQRxu^T + \LQRxx$, and $\Cxp = \LQRA - \LQRB \LQRuu^{-1}\LQRxu^T$. 
Then, the solution is also given by
$S(\HJx,\HJt) = \frac{1}{2} \HJx^T \Sxx(\HJt)\HJx$, 
where the function $\Sxx: [0,\infty)\to\R^{n\times n}$ takes values in the space of symmetric positive definite matrices and solves the following Riccati equation:
\begin{equation} \label{eqt: odeP}
{\small
    \begin{dcases}
    \dot{\Sxx}(\HJt) =  -\Sxx(\HJt)^T\Cpp\Sxx(\HJt) + \Sxx(\HJt)^T\Cxp + \Cxp^T\Sxx(\HJt) + \Cxx &\HJt\in(0,+\infty),\\
    \Sxx(0) = \LQRTC.
    \end{dcases}
    }
\end{equation}

When $\Hamiltonian$ does not depend on $\HJx$, we can use our connection to modify the corresponding LQR problem to consider different HJ PDEs and, hence, to accommodate different learning problems. For example, adding lower order terms in the running cost $L$ and/or the source term $f$ of the dynamics corresponds to adding lower order terms in the Hamiltonian of the HJ PDE or, equivalently, in the data fitting term of the learning problem. 
Similarly, adding lower order terms in the 
terminal cost $\HJIC$ corresponds to adding lower order terms in the initial condition of the HJ PDE or, equivalently, in the regularization term of the learning problem. \updatetwo{In Appendix~\ref{appendix:general_LQR}, we present a more general LQR problem with lower order terms that is more amenable to forming our connection to linear regression problems.}


\subsection{Connection to single-point regularized linear regression problems}\label{sec:LQR_1ptregression}


In this section, we establish a relation between the linear regression problem with only one data point and the LQR problem~\eqref{eqt:LQR_general_control} with $\LQRA = \LQRxx = 0$ and $\LQRxu = 0$. Note that, to do this, we also add in some lower order terms to the original LQR problem~\eqref{eqt:LQR_general_control} \updatetwo{(e.g., see Appendix~\ref{appendix:general_LQR})}.
In this case, the LQR problem becomes
    \begin{multline}\label{eqt:optctrl_1pt}
        S(\HJx,\HJt) = \min_{\updateone{\HJu(\cdot)}}  \Bigg\{\int_0^\HJt \left(\frac{1}{2}\HJu(s)^T\LQRuu\HJu(s) -\ba^T\HJu(s)\right)ds \\ + \frac{1}{2}\bx(\HJt)^T\LQRTC\bx(\HJt)
        + \bb^T \bx(\HJt)\colon 
         \dot\bx(s) = \LQRB\HJu(s) \forall s\in(0,\HJt], \bx(0) = \HJx\Bigg\},
    \end{multline}
    and the corresponding HJ PDE is given by~\eqref{eqt:LQR_HJPDE}, where $\HJIC(\HJx) = \frac{1}{2}\HJx^T\LQRTC\HJx + \bb^T \HJx$ is the initial data/terminal cost whose Fenchel transform is given by
    $$\HJIC^*(\HJmom) = \sup_{\HJx\in\Rn} \langle \HJx, \HJmom\rangle - \HJIC(\HJx) = \frac{1}{2}\left\|\LQRTC^{-1/2} (\HJmom - \bb)\right\|_2^2$$
    and the Hamiltonian $\Hamiltonian$ is given by 
    $$
    \Hamiltonian(\HJmom) 
    = \sup_{\HJu\in\R^m} \langle -\LQRB \HJu, \HJmom\rangle - \frac{1}{2}\HJu ^T\LQRuu\HJu + \ba^T \HJu = \frac{1}{2}\left\|\LQRuu^{-1/2}(\LQRB^T\HJmom- \ba)\right\|_2^2, $$
    where $f(\HJu) = \LQRB\HJu$ is the source term of the dynamics and $L(\HJu) = \frac{1}{2}\HJu^T\LQRuu\HJu$ is the running cost. Then, using the single-time Hopf formula~\eqref{eqt:singletime_Hopf}, we have that the solution to the HJ PDE is given by
    \begin{equation}\label{eqt:hopf_1ptregression}
        S(\HJx,\HJt) = \sup_{\HJmom\in\Rn} \left\{\langle \HJx, \HJmom\rangle - \frac{\HJt}{2}\left\|\LQRuu^{-1/2}( \LQRB^T \HJmom - \ba)\right\|_2^2 - \frac{1}{2}\left\|\LQRTC^{-1/2} (\HJmom - \bb)\right\|_2^2 \right\}.
    \end{equation}
    In this case, we can compute the maximizer in the above Hopf formula explicitly, which can be done numerically using the method of least squares.
   
    Alternatively, this LQR problem can also be solved via the Riccati ODEs, which are given by 
    \begin{equation}
    \dot{\Sxx}(\HJt) =  -\Sxx(\HJt)^T\LQRB\LQRuu^{-1}\LQRB^T\Sxx(\HJt), \quad 
    \dot{\Sx}(\HJt) = -\Sxx(\HJt)^T\LQRB\LQRuu^{-1}(\LQRB^T\Sx(\HJt) - \ba),
    \end{equation}
    $$\dot{\Sc}(\HJt) = -\frac{1}{2}\left\|\LQRuu^{-1/2}(\LQRB^T\Sx(\HJt)- \ba)\right\|_2^2$$
    with initial conditions $\Sxx(0) = \LQRTC$, $\Sx(0) = \bb$, and $\Sc(0) = 0$.


The above Hopf formula~\eqref{eqt:hopf_1ptregression} is related to the learning problem~\eqref{eqt:general_learning} with quadratic data fidelity term and quadratic regularization.
Specifically, let $\HJt = \param$ and $\HJmom = \weightvec$. Then, solving the above maximization problem~\eqref{eqt:hopf_1ptregression}  is equivalent to minimizing the following loss function with respect to $\weightvec = [\weight_1, \dots, \weight_n]^T$:
    \begin{equation}\label{eqt:loss_1ptregression}
        \lossfunc(\weightvec) = \frac{\param}{2}\left\|\LQRuu^{-1/2} (\LQRB^T \weightvec - \ba)\right\|_2^2 + \frac{1}{2}\left\|\LQRTC^{-1/2} (\weightvec - (\bb + \LQRTC\HJx))\right\|_2^2.
    \end{equation}
This loss function corresponds to a one-point linear regression problem, where the regularization term is given by $\frac{1}{2}\left\|\LQRTC^{-1/2} (\weightvec - (\bb + \LQRTC\HJx))\right\|_2^2$ and the data fitting term is given by $\frac{\param}{2}\left\|\LQRuu^{-1/2} (\LQRB^T \weightvec - \ba)\right\|_2^2$; i.e., set $\numt = 1$ in Figure~\ref{fig:connection_LQR}.
The minimizer $\weightvec^*$ of $\lossfunc$ is related to the solution of the Riccati equation via
\begin{equation}\label{eqt:singlept_minimizer}
\weightvec^* (=\HJmom^*) = \nabla_\HJx S(\HJx,\param) = \Sxx(\param)\bx + \Sx(\param),
\end{equation}
where $\HJmom^*$ is the minimizer in the Hopf formula~\eqref{eqt:hopf_1ptregression}.
Note that if we only need to recover the minimzer $\weightvec^*$, then we only need to solve two ODEs (namely, the ODEs for $\Sxx, \Sx$) since~\eqref{eqt:singlept_minimizer} does not depend on $\Sc$.




\subsection{Connection to multi-point regularized linear regression problems}\label{sec:LQR_multiptregression}

\begin{figure}[htbp]
    \centering
\begin{adjustbox}{width=\textwidth}
\begin{tikzpicture}[node distance=2cm]
    \node (min) [nobox, yshift=-0.2cm] {$\min$};
    \node (minarg) [boxsmall, right of=min, xshift=-1.5cm, yshift=-0.37cm, draw=cyan!60, fill=cyan!5] {$\text{}_{\weightvec\in\weightspace}$};
    \node (sum) [nobox, right of=minarg, xshift=-0.5cm] {$\sum_{i=1}^\numt$};
    \node (param) [box, right of=sum, xshift=-0.7cm, draw=magenta!60, fill=magenta!5] {$\param_i$};
    \node (loss) [box, right of=param, xshift=0.5cm, draw=green!60, fill=green!5] {$\frac{1}{2}\left\|\LQRuu_i^{-1/2}(\LQRB_i^T\weightvec - \ba_i)\right\|_2^2$ };
    \node (plus) [nobox, right of=loss, xshift=1.2cm] {$+$};
    \node (regLeft) [nobox, right of=loss, xshift=3.5cm] {};
    \node (regularization) [box, right of=regLeft, xshift=-0.2cm, draw=blue!60, fill=blue!5] { $\frac{1}{2}\left\|\LQRTC^{-1/2} (\weightvec - (\bb + \LQRTC\HJx))\right\|_2^2$};
    \node (regRight) [nobox, right of=regularization, xshift=-0.1cm] {};
    
    \node (sup) [nobox, below of=min, xshift=-0cm] {$\min$};
    \node (suparg) [boxsmall, right of=sup, xshift=-1.5cm, yshift=-0.37cm, draw=cyan!60, fill=cyan!5] {$\text{}_{\HJmom\in\HJstatespace}$};
    \node (sum) [nobox, below of=sum] {$\sum_{i=1}^\numt$};
    \node (time) [box, below of=param, xshift=-0cm, draw=magenta!60, fill=magenta!5] {$\HJt_i$};
    \node (equals) [nobox, left of=sup, xshift=1.45cm] {$=$};
    \node (S) [box, left of=equals, xshift=0.5cm, draw=red!60, fill=red!5] {$S(\HJx, \HJt_1, \dots, \HJt_\numt)$};
    \node (minus2) [nobox, left of=S, xshift=0.5cm] {$-$};
    \node (Hamiltonian) [box, below of=loss, draw=green!60, fill=green!5] {$\frac{1}{2}\left\|\LQRuu_i^{-1/2}(\LQRB_i^T\HJmom- \ba_i)\right\|_2^2$};
    \node (plus2) [nobox, right of=Hamiltonian,xshift=0.65cm] {$+$};
    \node (IC) [box, below of=regLeft, draw=blue!60, fill=blue!5] {$\frac{1}{2}\left\|\LQRTC^{-1/2}(\HJmom - \bb)\right\|_2^2$};
    \node (linear) [box, below of=regRight, draw=blue!60, fill=blue!5] {$-\langle \HJx, \HJmom\rangle$};

    \node (OCmin) [nobox, below of=sup, yshift=-0.3cm] {$\min$};
    \node (OCminarg) [boxsmall, right of=OCmin, xshift=-1.5cm, yshift=-0.37cm, draw=cyan!60, fill=cyan!5] {$\text{}_{\HJu(\cdot)}$};
    \node (OCS) [box, below of=S, xshift=0cm, yshift=-0.35cm, draw=red!60, fill=red!5] {$S(\HJx, \HJt_1, \dots, \HJt_\numt)$};
    \node (OCequals) [nobox, below of=equals, xshift=0cm, yshift=-0.35cm] {$=$};
    \node (OCint) [nobox, below of=time, xshift=-1.1cm] {$\sum_{i=1}^\numt \int_{\sum_{j=1}^{i-1}\HJt_j}^{\sum_{j=1}^{i}} $};
    \node (OCt) [boxsmall, right of=OCint, xshift=-0.9cm, yshift=0.2cm, draw=magenta!60, fill=magenta!5] {$\text{}^{\HJt_j}$};
    \node (OCHamiltonian) [box, below of=Hamiltonian, xshift=-0cm, draw=green!60, fill=green!5] {$\left(\frac{1}{2}\HJu(s)^T\LQRuu_i\HJu(s) -\ba_i^T\HJu(s)\right)$};
    \node (OCds) [nobox, right of=OCHamiltonian, xshift=0.4cm] {$ds$};
    \node (OCplus) [nobox, right of=OCds, xshift=-1.6cm] {$+$};
    \node (OCIC) [box, below of=IC, draw=blue!60, fill=blue!5] {$\frac{1}{2}\bx(T_\numt)^T\LQRTC\bx(T_\numt) + \bb^T \bx(T_\numt)$};
    \node (OCdynam) [box, right of=OCIC, xshift=3.05cm, draw=green!60, fill=green!5] {$\dot\bx(s) = \LQRB_i\HJu(s)\forall s\in \left(T_{i-1}, T_i\right]$};
    \node (OCinitialposition) [box, right of=OCdynam, xshift=1.5cm, draw=blue!60, fill=blue!5] {$\bx(0) = \HJx$};
    \node (leftbracket) [nobox, left of=OCint, xshift=0.8cm] {$\Bigg\{$};
    \node (colon) [nobox, right of=OCIC, xshift=0.55cm] {$:$};
    \node (comma) [nobox, right of=OCdynam, xshift=0.45cm, yshift=-0.11cm] {,};
    \node (rightbracket) [nobox, right of=OCinitialposition, xshift=-1.05cm] {$\Bigg\}$};

    \node (LPequal) [nobox, above of=equal] {$=$};
    \node (minloss) [box, above of=S, draw=red!60, fill=red!5] {$\min_{\weightvec\in\weightspace}\mathcal{L}(\weightvec)$};
    \draw [dottedarrow] (S) -- (minloss);
    
    \draw [doublearrow] (loss) -- (Hamiltonian);
    \draw [dottedarrow] (regularization) -- (IC);
    \draw [dottedarrow] (regularization) -- (linear);
    \draw [doublearrow] (minarg) -- (suparg);
    \draw [doublearrow] (OCS) -- (S);
    \draw [doublearrow] (OCt) -- (time);
    \draw [dottedarrow] (OCHamiltonian) -- ++(0,1.5cm);
    \draw [dottedarrow] (OCIC) -- (IC);
    \draw [dottedarrow] (OCdynam) -- (Hamiltonian);
    \draw [doublearrow] (time) -- (param);
    \draw [dottedarrow] (linear) -- (OCinitialposition);
    \draw [dottedarrow] (suparg) -- (OCminarg);
\end{tikzpicture}
\end{adjustbox}
    \caption{(See Section~\ref{sec:LQR}) Mathematical formulation describing the connection between a regularized linear regression problem  (\textbf{top}), the multi-time Hopf formula (\textbf{middle}), and a piecewise LQR problem with $\LQRA = \LQRxx = 0$ and $\LQRxu = 0$ on each piece (\textbf{bottom}). Note that $T_i = \sum_{j=1}^i t_j$. The content of this illustration is a special case of the connection in Figure~\ref{fig:connection_multitimeHopf} using quadratic data fitting losses and quadratic regularization. The colors indicate the associated quantities between each problem. The solid-line arrows denote direct equivalences. The dotted arrows represent additional mathematical relations.}
    \label{fig:connection_LQR}
\end{figure}

If the running cost is piecewise quadratic with $\LQRxx = 0, \LQRxu = 0$ on each piece and the dynamics are piecewise linear with $\LQRA = 0$ on each piece, then we get a more general piecewise LQR problem in the form of~\eqref{eqt:optimal_control_standardform}
with piecewise running costs $L_i(\HJu) = \frac{1}{2}\HJu ^T\LQRuu_i\HJu - \ba_i^T \HJu$, where each $\LQRuu_i\in \R^{m\times m}$ is symmetric positive definite, piecewise dynamics $f_i(\HJu) = \LQRB_i \HJu$, terminal cost $\HJIC(\HJx) = \frac{1}{2}\HJx^T\LQRTC \HJx + \bb^T\HJx$, and terminal time $\HJt = \sum_{j=1}^\numt \HJt_j$. Recall that this optimal control problem corresponds to the multi-time HJ PDE~\eqref{eqt:multitimeHJPDE} with Hamiltonians $\Hamiltonian_i(\HJmom) = \sup_{\HJu\in\R^m} \{\langle -f_i(\HJu), \HJmom\rangle - L_i(\HJu)\}$ and initial data $\HJIC$. Then, the solution to this LQR problem and corresponding HJ PDE is given by the following  multi-time Hopf formula:
\begin{equation} \label{eqt:piece_LQR_Hopf}
S(\HJx,\HJt_1, \dots, \HJt_\numt) = \sup_{\HJmom\in\Rn} \left\{\langle \HJx, \HJmom\rangle - \sum_{i=1}^\numt \frac{\HJt_i}{2}\|\LQRuu_i^{-1/2}(\LQRB_i^T\HJmom- \ba_i)\|_2^2 - \frac{1}{2}\|\LQRTC^{-1/2}(\HJmom - \bb)\|_2^2\right\}.
\end{equation} 


Define $T_i$ to be
\begin{equation}
T_i = \sum_{j=1}^i t_j.
\end{equation}
Then, the multi-time HJ PDE and piecewise LQR problem can also be solved using the following Riccati equation: 
$S(\HJx,\HJt_1,\cdots, \HJt_N) = \frac{1}{2} \HJx^T \Sxx(T_N)\HJx + \Sx(T_N)^T\HJx + \Sc(T_N)$, where the function $\Sxx:[0,\infty)\to\R^{n\times n}$ takes values in the space of symmetric positive definite matrices and solves the following piecewise Riccati equation:
\begin{equation} \label{eqt: odeP_piecewise}
    \begin{dcases}
    \dot{\Sxx}(s) =  -\Sxx(s)^T\LQRB_i\LQRuu_i^{-1}\LQRB_i^T\Sxx(s) &s\in \left(T_{i-1}, T_i\right)\\
    \Sxx(0) = \LQRTC,
    \end{dcases}
\end{equation}
the function $\Sx:[0,\infty)\to\Rn$ solves the following piecewise linear ODE:
\begin{equation} \label{eqt: odeq_piecewise}
    \begin{dcases}
    \dot{\Sx}(s) = -\Sxx(s)^T\LQRB_i\LQRuu_i^{-1}(\LQRB_i^T\Sx(s) - \ba_i)&s\in \left(T_{i-1}, T_i\right),\\
    \Sx(0) = -\bb,
    \end{dcases}
\end{equation}
and the function $\Sc:[0,\infty)\to\R$ solves the following piecewise ODE:
\begin{equation} \label{eqt: oder_piecewise}
    \begin{dcases}
    \dot{\Sc}(s) = -\frac{1}{2}\left\|\LQRuu_i^{-1/2}(\LQRB_i^T\Sx(s)- \ba_i)\right\|_2^2 &s\in \left(T_{i-1}, T_i\right),\\
    \Sc(0) = 0.
    \end{dcases}
\end{equation}

The above multi-time Hopf formula~\eqref{eqt:piece_LQR_Hopf} can also be regarded as a linear regression problem with multiple data points, as illustrated in Figure~\ref{fig:connection_LQR}. Let  $\frac{1}{2}\|\LQRTC^{-1/2}(\HJmom - \bb)\|_2^2$ be the the regularization term and  $\frac{\HJt_i}{2}\|\LQRuu_i^{-1/2}(\LQRB_i^T\HJmom- \ba_i)\|_2^2$ be the data fitting term at the $i$-th data point. 
Then, the multi-time LQR problem~\eqref{eqt:piece_LQR_Hopf} is equivalent to the learning problem $\min_{\weightvec}\lossfunc(\weightvec)$, where the loss function $\lossfunc:\Rn\to\R$ is given by
\begin{equation} \label{eqt:regression_multidata}
\lossfunc(\weightvec) = \sum_{i=1}^\numt \frac{\param_i}{2}\left\|\LQRuu_i^{-1/2}(\LQRB_i^T\weightvec - \ba_i)\right\|_2^2 + \frac{1}{2}\left\|\LQRTC^{-1/2} (\weightvec - (\bb + \LQRTC\HJx))\right\|_2^2.
\end{equation}
In Section~\ref{subsec:method_1}, we discuss a specific example of the learning problem~\eqref{eqt:regression_multidata}, which is more readily recognizable as the standard linear regression problem. The minimizer $\weightvec^*$ of the learning problem~\eqref{eqt:regression_multidata} (and $\HJmom^*$ of the Hopf formula~\eqref{eqt:piece_LQR_Hopf}) is given by 
\begin{equation}\label{eqt:multipt_minimizer}
\weightvec^* (=\HJmom^*) = \nabla_{\HJx} S(\HJx,\HJt_1,\dots, \HJt_N) = \Sxx\left(T_N\right)\HJx + \Sx\left(T_N\right).
\end{equation}
\updatethree{For more details, we refer readers to~\cite{bardi1984hopf}. Note that one could also use the Pontryagin maximum principle to compute the gradient of the solution to the HJ PDE, which gives the same result as in~\eqref{eqt:multipt_minimizer}.}



\section{Method}
\label{s:method}

We consider the 3D euclidean space $\Real^3$ with points $p=(x,y,z)\in\Real^3$. We discretize the unit cube $\gC=[0,1]^3$ with a 3D voxel grid $\gG=\set{p_I}$, with nodes $p_I$ indexed by $I=(i,j,k)$, $i,j,k\in [n]=\set{1,\ldots,n}$, \ie, $p_I=(x_{ijk},y_{ijk},z_{ijk})$. We denote by $h=n^{-1}$, and by $N=n^3$ the total number of nodes.   
We represent our reconstructed surface as a zero level of a scalar function $f$ defined over the cube $\gC$. $f$ is defined by prescribing its values at the grid's nodes $f_I\in\Real$ and trilinear interpolating in each voxel. We will denote by $f(p)$ the interpolated value at point $p$. 

Given an input point cloud consisting of $m$ points $q_k\in\Real^3$ with or without (unit norm) normals $n_k\in \Real^3$, $k\in [m]$, our goal is to compute $f$ so that its zero level set approximates the unknown surface, \ie, 
\begin{equation}
    \gS = \set{p\in\gC \ \vert \ f(p)=0}.
\end{equation}
Our approach to compute $f$ is to minimize a loss function of the form
\begin{equation}
    \gL = \gL_{\text{data}} + \gL_{\text{prior}}
\end{equation}
where 
\begin{equation}\label{e:loss_data}
    \gL_{\text{data}} = \frac{\lambda_{\text{p}}}{m}\sum_{k=1}^m \abs{f(q_k)}^2 + \frac{\lambda_{\text{n}}}{m}\sum_{k=1}^m \norm{\nabla f(q_k) - n_k}^2
\end{equation}
where $\norm{\cdot}$ is the standard euclidean norm in $\Real^3$, $\nabla f(p) \in \Real^3$ is the gradient of $f$ sampled at point $p$. Note that $\nabla f$ is defined in interior of voxels, which is generically where the input points $q_k$ resides. $\gL_{\text{data}}$ is the standard data loss encouraging the zero level to pass through the input points $q_k$, and its normals (defined by gradients of $f$) to coincide with input normals $n_k$. 

The prior, $\gL_{\text{prior}}$, is the main contribution of this work, where we combine two novel losses,
\begin{equation}
    \gL_{\text{prior}} = \lambda_{\text{v}} \gL_{\text{viscosity}} + \lambda_{\text{c}} \gL_{\text{coarea}}
\end{equation}
Intuitively, the viscosity loss optimizes for a smooth Signed Distance Function (SDF) solutions, avoiding auxiliary bad minima of the Eikonal equation, while the coarea loss strives to minimize the area of the zero level surface. Our loss has $4$ hyper-parameters $\lambda_{\text{p}},\lambda_{\text{n}},\lambda_{\text{v}},\lambda_{\text{c}}$. We provide more details on these priors next. 


\subsection{Viscosity Loss}\label{ss:viscosity_loss}
The goal of the viscosity loss is to make $f$ approximate an SDF over $\gC$. Given boundary conditions asking $f$ to vanish on some closed compact surface $\gS$, the SDF solves the Eikonal equation PDE, \ie, $\norm{\nabla f(p)}=1$, in a certain well defined sense (viscosity). This motivated some previous work to directly optimize the Eikonal loss \citep{gropp2020implicit,sitzmann2020implicit}
\begin{equation}\label{e:loss_eikonal}
    \gL_{\text{eikonal}} = \int_\gC \Big (\norm{\nabla f(p)}-1\Big )^2 dp
\end{equation}
\begin{wrapfigure}[14]{r}{0.28\textwidth}\vspace{-15pt}
  \begin{center}
    \includegraphics[width=0.25\textwidth]{figs/illustrations/eikonl_1d.png}
  \end{center}
  \caption{Two global minimizers of the Eikonal loss over a grid in 1D. Top solution is not an SDF. }\label{fig:eikonal_1d}
\end{wrapfigure}
Unfortunately, the Eikonal loss has many undesirable minima which are not SDFs. Figure \ref{fig:eikonal_1d} shows a 1D example: both depicted solutions (denoted $f$) vanish at the input points $q_1,q_2$ (black points) and globally minimize the Eikonal loss over the grid (grid points are shown in blue). The INR works mentioned above use neural networks for representing $f$ which injects an inductive bias avoiding these bad minima, however on grids, minimizing \eqref{e:loss_eikonal} cannot avoid these solutions. See, \eg, middle column in Figure \ref{fig:teaser}. 

More classical Eikonal solvers do work with grids however use mostly fast marching or sweeping methods \citep{osher1988fronts,sethian1996fast,zhao2005fast,chacon2012fast}. Namely, use a special discretization of the Eikonal equation favoring the viscosity  solution of the Eikonal \cite{rouy1992viscosity}, and update node values according to a moving front \cite{sethian1996fast}. Since this discretization is up-wind (will only propagate values in one direction) and requires choosing the maximal among its solution, its success in adaptation to a loss is not clear. 

We use a different approach to build a loss favoring SDF solutions over grids motivated by the vanishing viscosity method \cite{crandall1983viscosity}. Namely, adding to the Eikonal PDE a small perturbation of the Laplacian of $f$ (denoted by $\Delta f$), \ie, $\norm{\nabla f(p)}-1 - \eps\Delta f(p)=0$, makes the PDE semi-linear elliptic \citep{calder2018lecture}, and hence with suitable boundary conditions it is uniquely solvable inside $\gS$ with a smooth solution, approaching the viscosity positive distance function to the boundary as $\eps\too 0$. Similarly, for $1-\norm{\nabla f(p)} - \eps \Delta f(p)=0$ the solution approaches the negative distance function inside the domain. 
Motivated by the vanishing viscosity principle we suggest the following viscosity loss:
\begin{equation}\label{e:loss_viscosity_eikonal}
\gL_{\text{viscosity}} = \int_\gC \Big((\norm{\nabla f (p)}-1)\mathrm{sign}(f(p)) - \eps \Delta f(p)\Big)^2 dp
\end{equation}
We discretize this loss over the grid $\gG$ by replacing the first order derivatives and second order derivatives with symmetric finite  differences, \ie,
\begin{align*}
    D_x f_I=D_x f_{i,j,k} = \frac{f_{i+1,j,k}-f_{i-1,j,k}}{2h}, \quad D^2_x f_I = D^2_x f_{i,j,k}=\frac{f_{i+1,j,k}-2f_{i,j,k}+f_{i-1,j,k}}{h^2}
\end{align*}
and similarly for $D_y$ and $D_z$. We use these discrete operators to approximate the gradient $\widehat{\nabla} f(p_I) = (D_x f_I, D_y f_I, D_z f_I)$ and Laplacian $\widehat{\Delta}f(p_I) = D_x^2f_I + D_y^2 f_I + D_z^2 f_I$. The discretized viscosity loss now takes the form
\begin{equation}
    \widehat{\gL}_{\text{viscosity}} = \frac{1}{N}\sum_{I} \Big((\|\widehat{\nabla} f (p_I)\|-1)\mathrm{sign}(f(p_I)) - \eps \widehat{\Delta} f(p_I)\Big)^2
\end{equation}



\subsection{Coarea loss}\label{ss:coarea_loss}
The coarea loss is approximating the area of the zero level set, and therefore incorporating it in the optimization pushes the reconstructed surface to be economic in area. 

First, similarly to  \citep{yariv2021volume} we use the centered Laplace CDF
\begin{equation}
   \Psi\beta(s)= \begin{cases}
   \frac{1}{2}\exp\parr{\frac{s}{\beta}} & s\leq 0 \\ 1-\frac{1}{2}\exp\parr{-\frac{s}{\beta}} & s\geq  0
   \end{cases}
\end{equation} to transform the SDF $f$ to a smooth approximation of the indicator function:
\begin{equation}
    \chi_\beta(p)=\Psi\beta (-f(p))
\end{equation}
As $\beta\too 0$, $\chi_\beta$ converges to an indicator function leading to $1$ inside $\gS$ and $0$ outside. The coarea loss is defined as 
\begin{equation}
    \gL_{\text{coarea}} = \int_\gC \norm{\nabla \chi_\beta (p)} dp
\end{equation}
To understand why this loss approximates the area of $\gS$ we can use the coarea formula \citep{rindler2018calculus}:
\begin{equation}\label{e:coarea}
    \int_\gC \norm{\nabla \chi_\beta(p)}dp = \int_{-\infty}^{\infty} \mathrm{area}(\chi_\beta^{-1}(s))ds,
\end{equation}
where $\chi_\beta^{-1}(s)=\set{p\ \vert \ \chi_\beta(p)=s}$ is the preimage of the value $s$. Since $\chi_x(p)\in [0,1]$ the r.h.s.~integral can be restricted to the interval $[0,1]$, and therefore the coarea loss averages the area of the level sets of $\chi_\beta$. Next,  $$\chi_\beta^{-1}(s)= \set{p\ \vert \ \Psi\beta (-f(p)) = s } = \{p\ \vert \ f(p) = -\Psi\beta^{-1} (s) \} = f^{-1}(-\Psi\beta^{-1} (s)),$$
\begin{wrapfigure}[11]{r}{0.28\textwidth}\vspace{-20pt}
  \begin{center}
  \includegraphics[width=0.25\textwidth]{figs/semi.png}
  \end{center}
  \caption{Reconstruction of a semisphere point cloud (white dots) without (left) and with (right) coarea loss. }\label{fig:coarea_semisphere}
\end{wrapfigure}

which shows that the level set $s\in (0,1)$ of $\chi_\beta$ is the level set $-\Psi\beta^{-1}(s)$ of the SDF $f$. As $\beta\too 0$, $-\Psi\beta^{-1}(s)\too 0$ for all $s\in (0,1)$ (and uniformly in $(\eps,1-\eps)$ for fixed $\eps>0$). Therefore the average of the level set area (\ie, the r.h.s.~of \eqref{e:coarea}) converges to the area of $f^{-1}(0)=\gS$. Figure \ref{fig:teaser} (right) shows how removing the coarea loss introduces an extraneous zero level set, and hence results in an undesired surface part. Figure \ref{fig:coarea_semisphere} shows a comparison of a reconstruction of semisphere with and without coarea. In the experiments section we provide more ablation tests with the coarea and viscosity losses.

To discretize the coarea loss we let $w_I$ denote the centers of grid's voxels, and note that $\nabla \chi_\beta(w_I) = \Phi_\beta(-f(w_I))\nabla f(w_I)$, where 
\begin{equation*}
    \Phi_\beta(s) = \frac{1}{2\beta}\exp\parr{\frac{\abs{s}}{\beta}}
\end{equation*}
is the PDF of the Laplace distribution, and $\nabla f(w_I)$ is computed as a linear combination of the voxel's corner values $f_{I_1},\ldots,f_{I_8}$, see more details in the Appendix. We end up with the discretized loss:
\begin{equation}
    \widehat{\gL}_{\text{coarea}} = \frac{1}{N}\sum_{I}\Phi_\beta(-f(w_I))\norm{\nabla f(w_I)}
\end{equation}
This loss is usually incorporated with a small hyper-parameter $\lambda_{\text{c}}$ with the purpose of eliminating redundant surface parts.



\section{Numerical examples}\label{sec:numerics}
In this section, we apply the Riccati-based methodology presented in Section~\ref{sec:method} to four test problems from machine learning to demonstrate the versatility and potential computational advantages of our new approach.  
In each example, we use RK4 with double precision to solve the Riccati ODEs when applying our Riccati-based methodology. The connections between the examples presented in this section, the Hopf formula, and  the corresponding optimal control problems can be found in Table~\ref{tab:learning_problems}. 
We note that in this work we use two metrics to evaluate our results quantitatively: the $\ell_1$-norm (defined as $\|x\|_1 = \sum_{i=1}^n |x_i|$ for $x\in\Rn$) for the finite-dimensional minimizer and the $L^2$-norm (defined as $\|f\|_2 = \left(\int_{x\in\Rn} |f(x)|^2 dx\right)^{1/2}$ for $f:\Rn\to \R$) for the inferred functions. The $L^2$-norm for functions is approximated using trapezoidal rule with a uniform grid.
Supplementary details of the numerical experiments can be found in Appendix \ref{sec:details}. Code for all examples will be made publicly available at \url{https://github.com/ZongrenZou/HJPDE4SciML}.
\subsection{Function approximation in continual learning}\label{subsec:example_1}
In this section, we test our Riccati-based approach (as described in Section~\ref{subsec:example_1}) on a pedagogical function approximation example in continual learning \cite{parisi2019continual, kirkpatrick2017overcoming, van2019three} to demonstrate its computational and memory advantages.
Under the continual learning framework, data is accessed in a stream and the trainable model parameters are updated incrementally as new data becomes available.
In some cases, the historical data may also become inaccessible after new data is received, which can often lead to catastrophic forgetting \cite{kirkpatrick2017overcoming, parisi2019continual}, which refers to the abrupt degradation in performance of learned models on previous tasks upon training on new tasks. 
In this example, we show how our Riccati-based approach naturally coincides with the continual learning framework, while also inherently avoiding catastrophic forgetting even if the historical data is inaccessible.


Our example set-up is as follows. Our goal is to regress the function $y(\tau) = 0.1 \tau + 0.05\sin(10\tau)$ given noisy data $\{(\tau_i, y_i\approx y(\tau_i))\}_{i=1}^N\subset\R\times\R$.
Following the continual learning framework, we assume that a new noisy data point is available every $\Delta \tau= 0.01$ (i.e., $\tau_i = (i-1)\Delta \tau\in [0, 10]$, $i = 1, \dots, N$).
We regress $y(\tau)$ using the linear model $y(\tau) = \sum_{k=1}^n \weight_k \phi_k(\tau)$, where $n=10$ and the basis functions are given by
\begin{equation}\label{eq:continual:basis}
    \{\phi_k(\tau)\}_{k=1}^n = \{1, \tau, \tau^2, \tau^3, \sin(\tau), \sin(5\tau), \sin(8\tau), \sin(9\tau), \sin(10\tau), \sin(12\tau)\}.
\end{equation}
The coefficients $\weightvec = [\weight_1, \dots, \weight_n]^T$ of our linear model are learned by minimizing the following loss function:
\begin{equation}\label{eq:continual:loss}
    \mathcal{L}(\weightvec) = \frac{1}{2}\sum_{i=1}^N \lambda_i \left|\sum_{k=1}^{n} \weight_k\phi_k(\tau_i) - y_i \right|^2 + \frac{1}{2}\sum_{k=1}^{n} \MLreg_k|\weight_k|^2,
\end{equation}
where $\lambda_i, i=1,...,N$ and $\MLreg_k, k=1,...,n$ are weights for the data loss and $\ell_2$ regularization terms, respectively, and we update our learned coefficients every time a new data point is available. In our numerical experiments, we use additive Gaussian noise with zero mean and standard deviation $0.01$, and we set $\lambda_i=1, \forall i$ and $\MLreg_k=0.1, \forall k$. 


Although this loss function~\eqref{eq:continual:loss} could be minimized using conventional machine learning techniques (e.g., the method of least squares), these methods typically require access to and training on the entire dataset $\{(\tau_i, y_i)\}_{i=1}^N$, which conflicts with the assumptions in continual learning.
However, note that this loss function~\eqref{eq:continual:loss} is of the form~\eqref{eq:loss_function}. Thus, we can instead apply our Riccati-based approach (as described in Sections~\ref{subsec:method_1} and~\ref{subsec:method_2}) to solve this learning problem. In particular, the methodology described in Section~\ref{subsec:method_2} matches the data streaming paradigm of continual learning; we incrementally update our learned coefficients by considering each new data point as the time evolution of a corresponding multi-time HJ PDE. 
As a result, in contrast to conventional machine learning methods, our Riccati-based approach does not require storage of previous data points and its memory and computational complexity is constant for each added data point. As such, our Riccati-based methodology may be well-suited to online learning applications.
Additionally, since this time evolution of the HJ PDE (i.e., the addition of a new data point) requires knowledge about the solution to the HJ PDE at the previous time (i.e., the results of the previous training), our Riccati-based approach also inherently avoids catastrophic forgetting.


\begin{figure}[htbp]
    \begin{subfigure}[b]{\textwidth}
        \centering
        \includegraphics[width = 0.5\textwidth]{continual_learning/fig1_evolution.png}
        \caption{}
    \end{subfigure}
    \begin{subfigure}[b]{\textwidth}
        \centering
        \includegraphics[width = 0.3\textwidth]{continual_learning/fig1_200.png}
        \includegraphics[width = 0.3\textwidth]{continual_learning/fig1_400.png}
        \includegraphics[width = 0.3\textwidth]{continual_learning/fig1_800.png}
        \caption{}
    \end{subfigure}
    \caption{Evolution and continual learning of the function approximation learned using our Riccati-based approach as more data becomes available. (a) shows the evolution of the learned coefficients $\weight_k, k=1,...,n$ as more data is incorporated into the model; the horizontal dotted lines denote the exact reference values. (b) shows the inferences of $y$ after the 200th, 400th, and 800th noisy data point becomes available. Our Riccati-based approach allows us to incrementally update the learned coefficients as more data becomes available without requiring access to the previous data or re-training on the entire dataset, which provides advantages in both memory and computations over conventional learning methods. }
    \label{fig:continual}
\end{figure}

The results of applying our Riccati-based method to solve this learning problem~\eqref{eq:calibration:loss} are shown in Figure~\ref{fig:continual} and Table~\ref{tab:continual}. 
Figure~\ref{fig:continual}(a) depicts the evolution of the coefficients $\{\weight_k\}_{k=1}^n$ as more data points become available. We observe that the learned coefficients $\weight_k, k=1,...,n$ converge to their true values as more data points are incorporated into the model.
Figure~\ref{fig:continual}(b) displays three inferences at  different times $\tau = 2, 4, 8$, which demonstrate that our Riccati-based approach is capable of real-time inferences without catastrophic forgetting, even though each inference is made using only one data point.
Note that due to an appropriate choice of basis functions, our learned model is able to provide accurate extrapolations as well. 

Table~\ref{tab:continual} displays the numerical errors of the minimizer $\weightvec^*$ of the loss function~\eqref{eq:continual:loss} obtained using the Riccati-based methodology from Section~\ref{sec:method} after the last data point becomes available at $\tau=10$. We observe that the accuracy of our approach increases as we decrease the step size $h$ of RK4, which indicates that the errors of $\weightvec^*$ stem from the accuracy of RK4 in solving the corresponding Riccati ODEs.
The reference solution is obtained by minimizing~\eqref{eq:continual:loss} directly using the method of least squares and assuming that all $N=1000$ data points are accessible. 



\begin{table}[ht]
    \footnotesize
    \centering
    \begin{tabular}{|c|c|c|c|}
    \hline
         & $h=0.2$ & $h=0.1$ & $h=0.01$ \\
       \hline
       %$\ell^2$ error of $c^*$ & $6.4290\times 10^{-7}$ & $3.5060\times10^{-9}$ & $2.2395\times10^{-12}$\\
       %\hline
       %$L_2$ relative error of $c^*$ & $5.7623\times 10^{-6}$ & $3.1424\times 10^{-8}$ & $2.0073\times 10^{-11}$  \\
       %\hline
       $\ell_1$ error of $\weightvec^*$ & $1.1210\times 10^{-6}$ & $6.0924\times10^{-9}$ & $4.2222\times10^{-12}$\\
       \hline
       $\ell_1$ relative error of $\weightvec^*$ & $7.3622\times 10^{-6}$ & $4.0012\times10^{-8}$ & $2.7729\times10^{-11}$\\
       \hline
    \end{tabular}
\caption{Errors in computing the minimizer $\weightvec^*$ of the function approximation loss~\eqref{eq:continual:loss} using our Riccati-based approach. We use RK4 to solve the Riccati ODEs~\eqref{eqt:sequentialRiccatiODEs} and \eqref{eqt:regression_1Riccati} with double precision and various step sizes $h$. The reference is obtained by using the method of least squares to minimize the loss function~\eqref{eq:continual:loss} directly. }
    \label{tab:continual}
\end{table}

\subsection{1D steady-state reaction-diffusion equation and post-training calibration}\label{subsec:example_2}
In this example, we use our Riccati-based approach to apply post-training calibrations when solving a PDE. Specifically, we leverage the methodology in Section~\ref{subsec:method_2} to add or remove data and the methodology in Section~\ref{subsec:method_3} to enforce the boundary conditions of the PDE without retraining the entire learned model.
Consider the following 1D steady-state reaction-diffusion equation:
\begin{equation}\label{eq:reaction}
\begin{dcases}
  D\frac{\partial^2 u}{\partial x^2}(x) + \kappa u(x) = f(x), x\in [0, 1],\\
  u(0) = u(1) = 0,
\end{dcases}
\end{equation}
where $D=0.01$ is the diffusion coefficient, $\kappa=-1$, and $f(x)$ is the source term of which noisy measurements $\{(x_i, f_i\approx f(x_i))\}_{i=1}^N\subset\R\times\R$ are available.
We consider the scenario where regular training has been employed but with either insufficient data or sufficient data with outliers, both of which yield inaccurate inferences of the solution. 
We further assume that extra information is provided after the regular training and seek to perform post-training calibrations to incorporate this extra information into the already-trained models without losing information from the original training.
In the literature, it is well-established that post-training calibrations can significantly increase the performance of deployed machine learning methods \cite{psaros2023uncertainty, zou2022neuraluq}.
However, designing computationally efficient methods for performing these post-calibrations is still of great interest.

In this example, we solve this PDE~\eqref{eq:reaction} by reformulating the PDE as an optimization problem \cite{raissi2019physics, sirignano2018dgm, han2018solving}.
We use a linear model to approximate the solution, i.e. $u(x) = \sum_{k=1}^n \weight_k\phi_k(x)$, where $n=21$ and $\{\phi_k(x)\}_{k=1}^n=\{1\}\cup\{\sin(2l\pi x),$ $ \cos(2l\pi x)\}_{l=1}^{(n-1)/2}$ are the truncated Fourier basis functions on $[0, 1]$. 
We learn the coefficients $\weightvec = [\weight_1, \dots, \weight_n]^T$ of the linear model by minimizing the following loss:
\begin{equation}\label{eq:calibration:loss}
\begin{aligned}
    \mathcal{L}(\weightvec) & = \frac{1}{2}\sum_{i=1}^N \lambda_i \Bigg|D\sum_{k=1}^{n} \weight_k\frac{\partial^2 \phi}{\partial x^2}(x_i) + \kappa \sum_{k=1}^{n} \weight_k \phi_k(x_i) - f_i \Bigg|^2 \\
    & + \frac{1}{2}\lambda_b\left|\sum_{k=1}^{n} \weight_k \phi_k(0) - 0\right|^2 + \frac{1}{2}\lambda_b\left|\sum_{k=1}^{n} \weight_k \phi_k(1) - 0\right|^2 + \frac{1}{2}\sum_{k=1}^{n} \MLreg_k|\weight_k|^2,
\end{aligned}
\end{equation}
where $\lambda_i, i=1,...,N$, $\lambda_b$, and $\MLreg_k, k=1,..,n$ are balancing weights for the PDE residual, the boundary conditions, and the regularization term, respectively. In our numerical experiments, we assume the exact solution to be $u(x) = \sin^3(2\pi x)$ (and $f$ to be defined by~\eqref{eq:reaction}, accordingly) and the noise to be additive Gaussian with zero mean and standard deviation $0.1$. For the regular training, we set $\lambda_i=1$, $\lambda_b=1$, and $\MLreg_k=1$ and apply our Riccati-based approach (see Section~\ref{subsec:method_1}) to get our original estimate of the minimizer $\weightvec^*$ of the loss function~\eqref{eq:calibration:loss}. Using our Riccati-based approach yields an  $\ell_1$ error of $8.9950\times 10^{-10}$ and relative $\ell_1$ error of $1.4288\times 10^{-9}$ in $\weightvec^*$, where the reference is obtained by minimizing~\eqref{eq:calibration:loss} directly using the method of least squares.


In the leftmost column of Figure~\ref{fig:calibration:1}, we see that the accuracy of both $u$ and $f$ as inferred by the regular training is impaired by a lack of data around the highest peak and lowest valley of the exact functions. 
To compensate, we first calibrate our model by adding some new noisy measurements of $f$ in these regions where the data points are sparse. This calibration uses the methodology described in Section~\ref{subsec:method_2}, and the results are shown in the middle column of Figure~\ref{fig:calibration:1}. 
Next, we note that the inferred $u$ still disagrees with the exact solution at the boundary points. 
Hence, we further calibrate our model by increasing the value of the boundary weight $\lambda_b$ from $1$ to $10$ to enforce the boundary conditions of the PDE. This calibration is done using the methodology described in Section~\ref{subsec:method_3} and the results of this second calibration are presented in the rightmost column of Figure~\ref{fig:calibration:1}. In both cases, we observe that the calibrations successfully improve the accuracy of the learned model. \updatethree{These results are also reflected in the relative $L^2$ errors shown in Table~\ref{tab:calibration:1}.}


\begin{table}[ht]
    \footnotesize
    \centering
    \begin{tabular}{|c|c|c|c|}
    \hline
         & Regular training & First calibration & Second calibration \\
       \hline
       relative $L^2$  error of $f$ & $66.40\%$ & $7.15\%$ & $6.49\%$\\
       \hline
      relative $L^2$  error of $u$ & $57.29\%$ & $7.35\%$ & $5.57\%$\\
       \hline
    \end{tabular}
\caption{Errors in solving the 1D steady-state reaction-diffusion equation~\eqref{eq:reaction} and regressing $f$ at different stages of training, using our Riccati-based approach. Regular training is done with insufficient data and provides inaccurate inferences. The first calibration adds new measurements of $f$, and the second calibration increases the weights $\lambda_b$ of the boundary conditions in \eqref{eq:calibration:loss}. Both calibration steps successfully improve the accuracy of our inferences without requiring retraining on or access to the previous data. Qualitative results can be found in Figure~\ref{fig:calibration:1}.}
\label{tab:calibration:1}
\end{table}

\begin{figure}[ht]
    \begin{subfigure}[b]{\textwidth}
        \centering
        \includegraphics[width = 0.3\textwidth]{calibration/f_1.png}
        \includegraphics[width = 0.3\textwidth]{calibration/f_2.png}
        \includegraphics[width = 0.3\textwidth]{calibration/f_3.png}
        \caption{}
    \end{subfigure}
    \begin{subfigure}[b]{\textwidth}
        \centering
        \includegraphics[width = 0.3\textwidth]{calibration/u_1.png}
        \includegraphics[width = 0.3\textwidth]{calibration/u_2.png}
        \includegraphics[width = 0.3\textwidth]{calibration/u_3.png}
        \caption{}
    \end{subfigure}
    \caption{Results of solving the 1D steady-state reaction-diffusion equation~\eqref{eq:reaction} with noisy measurements of the source term $f$ in the domain and noiseless measurements of the solution $u$ on the boundary. (a) results for $f$; (b) results for $u$. \textbf{Left}: results of regular training; \textbf{middle}: calibrating the results with some additional noisy measurements of $f$; \textbf{right}: calibrating the results further by enforcing the boundary conditions by increasing the value of $\lambda_b$ in the loss function~\eqref{eq:calibration:loss}. The regular training uses our Riccati-based method in Section~\ref{subsec:method_1} to minimize~\eqref{eq:calibration:loss}, while the calibrations use the adaptations of our method in Section~\ref{subsec:method_2}. Calibrations are employed without re-training or access to the data from the previous training, which demonstrates the advantages in both memory storage and computational complexity of our Riccati-based approach over conventional machine learning methods.}
    \label{fig:calibration:1}
\end{figure}

Note that our Riccati-based approach allows us to perform each of these calibration steps using only the new or changed values in that step and the results of the previous training step. In other words, each step of this training process (including the original training and each subsequent calibration step) is done without sharing data between training steps, which exactly matches the framework of federated learning \cite{li2020federated}. Thus, our methodology may be relevant to distributed training or collaborative learning applications, where data privacy is of concern.



Next, we discuss another post-training calibration technique.
In this case, we assume the data for the regular training is sufficient but contains outliers due to large noise. We again assume the regular training is performed using the Riccati-based approach from Section~\ref{subsec:method_1}. Then, we eliminate these outliers using the methodology described in Section~\ref{subsec:method_2}, which only requires knowledge about the outliers to be removed and the results of the regular training.
The results of this post-training outlier removal show that eliminating these points successfully improves the accuracy of the learned models (see Appendix \ref{appendix:1}, Figure~\ref{fig:calibration:2}). \updatethree{Finally, in Appendix~\ref{appendix:1}, Figure~\ref{fig:example_2:3}, we also provide a large-scale continual learning example for solving~\eqref{eq:reaction} to further demonstrate the performance of our Riccati-based approach in large data settings.}



\subsection{Poisson equation using PINNs and transfer learning}\label{subsec:example_3}
In this example, we demonstrate the versatility of our Riccati-based method by combining it with existing machine learning techniques to fit the last layer of a PINN. We also show that when we perform hyper-parameter tuning by solving the associated Riccati ODEs (Section~\ref{subsec:method_3}), we not only provide the solution to the updated problem but also a continuum of solutions along a 1D curve on the Pareto front of the data fitting losses and regularization.
Consider the 2D Poisson equation with Dirichlet boundary conditions, which is given by
\begin{equation}\label{eq:poisson}
\begin{dcases}
  \frac{\partial^2 u}{\partial x^2}(x,y) + \frac{\partial^2 u}{\partial y^2}(x,y) = f(x,y) & (x, y) \in \Omega,\\
  u(x, y) = 0 & (x, y) \in \partial \Omega,
\end{dcases}
\end{equation}
where $\Omega := [0, 1]^2$ and $f$ is a source term. We solve this equation using transfer learning and PINNs. Consider the scenario where we only have access to measurements $\{(x_i, y_i, f_i = f(x_i, y_i))\}_{i=1}^N$ of $f$ at limited sampling points. Transfer learning compensates for this lack of knowledge by transferring the knowledge from models pre-trained on similar problems to  solve this new problem of interest. Here, we learn the linear model $u(x,y) = \sum_{k=1}^n\weight_k\phi_k(x,y)$, where each basis function $\phi_k(x, y), k=1, \dots, n$ is the PINN solution of a neural network 
pre-trained to solve the 2D Poisson equation~\eqref{eq:poisson} with similar source terms. 
Transfer learning using pre-trained neural networks as basis functions, as we do here, has recently grown in popularity in the scientific machine learning community \cite{desai2021one, zou2023hydra, goswami2022deep} and has been shown to provide efficient yet accurate inferences even given very limited data \cite{zou2023hydra}. 
We learn the coefficients $\weightvec$ of our linear model by minimizing the following PINN-type loss:
\begin{equation}\label{eq:loss:poisson}
    \mathcal{L}(\weightvec) = \frac{1}{2}\sum_{i=1}^N \lambda_i \left|\sum_{k=1}^n \weight_k\left(\frac{\partial^2\phi_k}{\partial x^2} + \frac{\partial^2\phi_k}{\partial y^2}\right)(x_i, y_i) - f_i \right|^2 + \frac{1}{2}\sum_{k=1}^n \MLreg_k|\weight_k|^2,
\end{equation}
where $\lambda_i=1, i=1,...,N$ and $\MLreg_k = \MLreg, k=1,...,n$ are weights for the data fitting and $\ell_2$-regularization terms, respectively. Note that minimizing~\eqref{eq:loss:poisson} with respect to $\weightvec$ is equivalent to fitting the last layer of a neural network given previous, pre-trained  nonlinear layers as the basis functions.

In our numerical experiments, we use transfer learning to solve the 2D Poisson equation~\eqref{eq:poisson}  with source term  $f(x, y) = \sin(2.5\pi x)\sin(2.5\pi y)$ using $n=100$ basis functions and $N=100$ random measurements of $f$.
We use the multi-head PINN method \cite{zou2023hydra} to obtain the basis functions, which correspond to the shared nonlinear hidden layers of the pre-trained PINN solutions to~\eqref{eq:poisson} with source terms $f(x, y) = \sin(k\pi x)\sin(k\pi y), k=1,2,3,4$. \updatetwo{We note that the boundary condition is hard-encoded in the basis functions, and hence, in \eqref{eq:loss_function}, no penalty term involving the boundary condition is included.} We then solve the learning problem~\eqref{eq:loss:poisson} using our Riccati-based method from Section~\ref{subsec:method_1}.


In Table~\ref{tab:poisson}, we compare the errors of the minimizer $\weightvec^*$ of~\eqref{eq:loss:poisson} and the solution $u$ of the PDE~\eqref{eq:poisson} as we decrease the weight $\MLreg (=\MLreg_k,\forall k)$ of the regularization term from 1 to  1e-5. The reference for $\weightvec^*$ is obtained from minimizing~\eqref{eq:loss:poisson} directly for each value of $\MLreg$ using the method of least squares. The reference for $u$ is computed using a finite difference method with a five-point stencil \updatetwo{and a $257\times257$ uniform grid on $\Omega$} to solve~\eqref{eq:poisson}. The same grid is used to evaluate our trained models. 
We originally minimize~\eqref{eq:loss:poisson} using $\MLreg = 1$ and the Riccati-based approach in Section~\ref{subsec:method_1}. We then compute the solutions for the other values of $\MLreg$ by incrementally decreasing $\MLreg$ by a factor of 10 and using the methodology from Section~\ref{subsec:method_3} to reuse the results of training with the previous value of $\MLreg$ to compute the solution for the new value of $\MLreg$. Consequently, we see that the error in $\weightvec^*$ increases as we decrease $\MLreg$ due to error accumulation from repeated applications of RK4. However, the error of $u$ generally decreases as we decrease $\MLreg$ with the lowest error being achieved when $\MLreg=$ 1e-4.


\begin{table}[ht]
    \footnotesize
    \centering
    \begin{adjustbox}{width=\textwidth}
    \begin{tabular}{|c|c|c|c|c|c|c|}
    \hline
         & $\MLreg = \updatetwo{10^0}$ & $\MLreg = \updatetwo{10^{-1}}$ & $\MLreg =\updatetwo{10^{-2}} $ & $\MLreg =\updatetwo{10^{-3}} $ & $\MLreg = \updatetwo{10^{-4}}$ & $\MLreg=\updatetwo{10^{-5}}$  \\
         \hline
       $\ell_1$ error of $\weightvec^*$ & $5.9707\times 10^{-10}$ & $4.2905\times 10^{-8}$ & $1.0725\times 10^{-6}$ & $1.7235 \times 10^{-5}$ & $2.0490\times 10^{-4}$ & $5.8749\times 10^{-3}$\\
       \hline
       $L^2$ relative error of $u$ & $6.5247\%$ & $5.6793\%$ & $3.0390\%$ & $1.7280\%$ & $1.1662\%$ & $1.4736\%$ \\
       \hline
    \end{tabular}
    \end{adjustbox}
    \caption{Errors of the minimizer $\weightvec^*$ of~\eqref{eq:loss:poisson} and the solution $u$ to the 2D Poisson equation~\eqref{eq:poisson} using transfer learning and our Riccati-based approach. The reference for $\weightvec^*$ is given by minimizing~\eqref{eq:loss:poisson} with the corresponding value of $\MLreg$ directly using the method of least squares, and the reference for $u$ is given by solving~\eqref{eq:poisson} using a finite difference method. Since we decrease $\MLreg$ incrementally from 1 to 1e-5, the error of $\weightvec^*$ accumulates due to successive applications of RK4.} 
    \label{tab:poisson}
\end{table}

From the results in Table~\ref{tab:poisson}, we see that our choice of the hyper-parameter $\MLreg$ can greatly influence the accuracy of our learned model. Note that since we fix $\lambda_i = 1, \forall i$ and 
$\MLreg_k = \MLreg, \forall k$, we can view~\eqref{eq:loss:poisson} as a bi-objective loss, where the two objectives are the weighted data fitting term $\frac{1}{2}\sum_{i=1}^N \left|\sum_{k=1}^n \weight_k\left(\frac{\partial^2\phi_k}{\partial x^2} + \frac{\partial^2\phi_k}{\partial y^2}\right)(x_i, y_i)- f_i\right|^2$ and the regularization term $ \frac{1}{2}\sum_{k=1}^n|\weight_k|^2$. 
To better understand the effects of tuning $\MLreg$, in Figure~\ref{fig:poisson}, we explore the Pareto front of these two objectives. Traditional scalarization-based approaches for computing the Pareto front typically rely on discrete samplings of the Pareto front corresponding to discrete choices of $\MLreg$ \cite{Jin2008pareto}.
While our Riccati-based methodology from Section~\ref{subsec:method_3} also recovers discrete points on the Pareto front corresponding to particular choices of $\MLreg$, note that when we change $\MLreg$, e.g., from $\MLreg=1$ to $\MLreg=0.1$, we also recover a one-dimensional curve along the Pareto front corresponding to every  $\MLreg\in[0.1, 1]$. We obtain this 1D curve theoretically via the flow of solutions obtained from the corresponding Riccati ODEs and numerically via the intermediate steps of RK4.
The left plot of Figure~\ref{fig:poisson} shows the 1D curve along the Pareto front recovered by our Riccati-based approach (although note that in this example, the Pareto front is also one-dimensional and hence is equivalent to the exposed 1D curve), where the flow of solutions corresponding to decreasing $\MLreg$ is represented by the arrows.
Thus, although in general our methodology cannot compute the entire Pareto front, every time we change the value of the hyper-parameters, our approach recovers a continuous 1D curve along the Pareto front. In the right plot of Figure~\ref{fig:poisson}, we also visualize how the $L^2$ error of our learned solution $u$ changes as we decrease $\MLreg$. 





\begin{figure}[htbp]
    \begin{subfigure}[b]{\textwidth}
        \centering
        \includegraphics[width = 0.4\textwidth]{poisson/pareto_front.png}
        \includegraphics[width = 0.4\textwidth]{poisson/error.png}
    \end{subfigure}
    \caption{Results of changing the regularization weight $\MLreg$ when solving the 2D Poisson equation using PINNs and transfer learning. We incrementally decrease the value of $\MLreg$ by evolving the corresponding Riccati ODEs backwards in time. As a result, we obtain a flow of solutions, the direction of which is represented by the arrows in each figure. In the left figure, this flow of solutions gives us that each change in the value of $\MLreg$ from $\hat\MLreg$ to $\tilde\MLreg$ results in the recovery of every point of the Pareto front along the one-dimensional curve parameterized by $\MLreg\in[\hat\MLreg, \tilde\MLreg]$.}
    \label{fig:poisson}
\end{figure}

\subsection{Identifying the dynamics of the Kraichnan-Orszag system from data}\label{subsec:example_4}
In this example, we demonstrate the versatility of our Riccati-based approach by showing how it can be combined with existing methods to solve more general problems (see Section~\ref{subsec:method_4}).
Consider the Kraichnan-Orszag (K-O) system \cite{wan2006multi, zou2022neuraluq, zhang2023discovering}
\begin{equation}\label{eq:ko}
\begin{dcases}
  \frac{dx_1}{d\tau} = x_2x_3,\\
  \frac{dx_2}{d\tau} = x_1x_3,\\
  \frac{dx_3}{d\tau} = -2x_1x_2,
\end{dcases}
\end{equation}
with initial conditions $x_1(0) = 1, x_2(0) = 0.8, x_3(0) = 0.5$. Our goal is to identify the dynamics (the right-hand side of~\eqref{eq:ko}) of the K-O system  using measurements of $x_i$ and $\frac{d x_i}{d\tau}, i=1,2,3$ at different times. We identify the dynamics by learning the linear models $\frac{dx_i}{d\tau} = \sum_{k=1}^n\weight_k^i\phi_k, i = 1, 2, 3$. Following the general framework of the SINDy method \cite{brunton2016discovering}, we use the following quadratic basis functions ($n = 10$) for the dynamics:
\begin{equation*}\label{eq:ko:basis}
    \{\MLbasis_k(x_1, x_2, x_3)\}_{k=1}^n = \{1, x_1, x_2, x_3, x_1^2, x_2^2, x_3^2, x_1x_2, x_2x_3, x_1x_3\},
\end{equation*}
and impose $\ell_1$-regularization on $\weightvec$ to promote sparse identification of the dynamics. 
Then, we learn each $\weightvec^i$ independently and in parallel by minimizing the loss functions
\begin{equation}\label{eq:ko:loss}
    \mathcal{L}_i(\weightvec^i) = \frac{1}{2}\sum_{j=1}^N \lambda_j \left[\left(\frac{dx_i}{d\tau}\right)_j - \sum_{k=1}^{n} \weight^i_k \phi_k((x_1)_j, (x_2)_j, (x_3)_j)\right]^2 + \sum_{k=1}^n \MLreg_k|\weight_k^i|,
\end{equation}
where $\mathcal{L}_i$ denotes the loss function for equation $i$, $(x_i)_j$ and $(\frac{dx_i}{dt})_j$ denote the measurements of $x_i$ and $\frac{dx_i}{d\tau}$, respectively, at time $\tau_j$, $i=1,2,3,j=1,...,N$. Note that \eqref{eq:ko:loss} corresponds to setting $R = \|\cdot\|_1$ and $\HJx = 0$ in the linear regression problem~\eqref{eqt:loss_fn_general_reg}. Thus, solving this learning problem is equivalent to evaluating the solution to the corresponding multi-time HJ PDE at $(0, \lambda_1, \dots, \lambda_n)$. 
In our numerical experiments, we generate data points for training and testing by solving~\eqref{eq:ko} numerically for $x_1, x_2, x_3\in[0, 10]$ \updatetwo{using MATLAB \textit{ode45} \cite{MATLAB} and then} using a central finite difference scheme to approximate the time derivatives. We set $\lambda_j=1, \forall j$ and $\MLreg_k=0.1, \forall k$.

Instead of the sparse regression techniques employed by SINDy, 
we use PDHG to minimize~\eqref{eq:ko:loss} (see Section~\ref{subsec:method_4}). Each iteration of PDHG involves minimizing a loss function of the form~\eqref{eq:ko:loss}, but with $\ell_2$-regularization instead of $\ell_1$. Hence, this sub-problem can be solved using our Riccati-based methods. As discussed in Section~\ref{subsec:method_4}, note that we only need to apply RK4 for the first iteration of PDHG, and every subsequent iteration can be solved using a change of bias. As such, we do not suffer from any error accumulation related to repeated applications of RK4.
In Table~\ref{tab:sindy:1}, we see that we do indeed recover a sparse identification of the dynamics. However, we also incorrectly identify non-zero coefficients for the basis functions $x_2$ and $x_3$. We note that this misidentification may be the result of a lack of unique identifiability of the system from the data points sampled. In fact, Table~\ref{tab:sindy:3} shows that the errors in the solution $x_1,x_2,x_3$ of the identified system versus the solution of the true system~\eqref{eq:ko} are relatively small, which corroborates that identifiability may have been an issue.



\begin{table}[ht]
    \footnotesize
    \centering
    \begin{tabular}{|c|c|c|c|c|c|c|c|c|c|c|}
    \hline
         & $1$ & $x_1$ & $x_2$ & $x_3$ & $x_1^2$ & $x_2^2$ & $x_3^2$ & $x_1x_2$ & $x_2x_3$ & $x_1x_3$ \\
         \hline
         $\weightvec^{1, *}$ & $0$ & $0$ & $0$ & $0$ & $0$ & $0$ & $0$ & $0$ & $0.9931$ & $0$\\
         \hline
         $\weightvec^{2, *}$ & $0$ & $0$ & $0$ & $0.0160$ & $0$ & $0$ & $0$ & $0$ & $0$ & $0.9761$\\
         \hline
         $\weightvec^{3, *}$ & $0$ & $0$ & $-0.0165$ & $0$ & $0$ & $0$ & $0$ & $-1.9777$ & $0$ & $0$\\
         \hline
    \end{tabular}
\caption{Results of sparse identification of the K-O system~\eqref{eq:ko} using PDHG to minimize the $\ell_1$-regularized losses~\eqref{eq:ko:loss}. The true solution is 0 for all entries, except $\weightvec^{1, *} = 1$ for $x_2x_3$, $\weightvec^{2, *} = 1$ for $x_1x_3$, and $\weightvec^{3, *} = -2$ for $x_1x_2$. We recover the dynamics reasonably well, albeit with some slight misidentification of $\weightvec^{2, *}, \weightvec^{3, *}$.}
    \label{tab:sindy:1}
\end{table}



\begin{table}[ht]
    \footnotesize
    \centering
    \begin{tabular}{|c|c|c|c|}
    \hline
         & $x_1$ & $x_2$ & $x_3$ \\
       \hline
       relative $L^2$ error (\%) & $0.2144$ & $0.3718$ & $0.3152$\\
       \hline
    \end{tabular}
    \caption{Errors of the solution $x_1, x_2, x_3$ of the system identified using PDHG compared to the true solution of the K-O system. The reference is obtained by numerically solving the true system~\eqref{eq:ko} using a central finite difference scheme. These errors indicate that the errors in the system identification in Table~\ref{tab:sindy:1} may be due to a lack of unique identifiability of the system using the given data points.}
    \label{tab:sindy:3}
\end{table}




\section{Conclusion}\label{sec:conclusion}
In this work, we focus on addressing the fundamental challenge of OOD detection tasks, which is how to fully understand the semantic discrepancy between the ID/OOD samples. We reveal that the key to success in the realistic SCOOD task is to allocate as many ID samples in the unlabeled set correctly as possible. To this end, we propose a novel uncertainty-aware optimal transport scheme that introduces class-specific energy scores as guidance for effective label assignment. Experimental results show that our method achieves better performance than previous state-of-the-art methods on SCOOD benchmarks.

\textbf{Limitations.} In addition to temperature scaling, other techniques such as feature clipping applied in ReAct~\cite{sun2021react} also enhance the performance of energy score, so how to obtain an OOD score that best fits the SCOOD task can be further explored. Moreover, a setting highly related to SCOOD has been proposed in \cite{katz2022training} and formulated as a constrained optimization problem. We will also theoretically analyze these practical OOD settings in our feature work.

% \section*{Acknowledgments}
\textbf{Acknowledgments.} 
This work is supported by National Key R\&D Program of China under Grant 2020AAA0105701, National Natural Science Foundation of China (NSFC) under Grants 61872327, Major Special Science and Technology Project of Anhui, National Natural Science Foundation of China (62033012) and Ant Group through Ant Research Intern Program.



\section*{Acknowledgments}
P.C. is supported by the SMART Scholarship, which is funded by the Under Secretary of Defense/Research and Engineering (USD/R\&E), National Defense Education Program (NDEP) / BA-1, Basic Research. J.D., G.E.K., and Z.Z. are supported by the MURI/AFOSR FA9550-20-1-0358 project. We also acknowledge the support by award DOE-MMICS SEA-CROGS DE-SC0023191.


\bibliographystyle{siamplain}
\bibliography{references}

\appendix
\section{Details of the methodology}
\subsection{Algorithm for deleting one data point} \label{appendix:method_delete_data}
Here, we provide details for the algorithm for deleting one data point from Section~\ref{subsec:method_2}.
Removing the $j$-th data point corresponds to removing the term $\frac{1}{2}\lambda_{j}\|\Phi_{j} \weightvec - \MLy_{j}\|_2^2 $ in the loss function~\eqref{eq:loss_function} or, equivalently, removing the Hamiltonian $\frac{1}{2}\|\Phi_{j} \weightvec - \MLy_{j}\|_2^2$ from the multi-time HJ PDE and removing the pieces $L_{j}(s, \HJu) = \frac{1}{2}\HJu^T\HJu - \MLy_{j}$ and $f(s,\HJu) = \Phi_{j}^T\HJu$ from the running cost and dynamics, respectively, of the corresponding piecewise LQR problem.
Hence, numerically, we can remove the $j$-th data point by solving the following Riccati ODE
\begin{equation}\label{eqt:regression_2Riccati}
    \begin{dcases}
    \dot{\tilde\Sxx}(\HJt) =  -\tilde\Sxx(\HJt)^T\MLbasismat_{j}^T\MLbasismat_{j}\tilde\Sxx(\HJt) &\HJt<\lambda_j,\\
    \dot{\tilde\Sx}(\HJt) = -\tilde\Sxx(\HJt)^T\MLbasismat_{j}^T(\MLbasismat_{j}\tilde\Sx(\HJt) - \MLy_{j})&\HJt<\lambda_j,
    \end{dcases}
\end{equation}
with terminal condition $\tilde\Sxx(\lambda_j) = \Sxx\left(T_N\right)$ and $\tilde\Sx(\lambda_j) = \Sx\left(T_N\right)$, where $\Sxx\left(T_N\right)$ and $\Sx\left(T_N\right)$ are obtained from solving the learning problem~\eqref{eq:loss_function} with all $N$ data points. Then, the solution to the new learning problem with the $j$-th point removed is given by~\eqref{eqt:sec42_newoptimizer}, where $\tilde\Sxx = \tilde\Sxx(0)$ and $\tilde \Sx = \tilde \Sx(0)$ are the solution to~\eqref{eqt:regression_2Riccati}.


\subsection{Algorithm for tuning the regularization weights}\label{appendix:hyperparam_tuning}
Here, we provide details for the algorithm for deleting one data point from Section~\ref{subsec:method_3}.
We consider the case where we change each regularization parameter $\MLreg_k$ to $\tilde\MLreg_k$. This change can be regarded as two steps: first, we change all parameters $\MLreg_k$ for the indices $k$ such that $\tilde\MLreg_k > \MLreg_k$, and then we change the other parameters. Define the index set $\mathcal{K} $ to be $\mathcal{K} = \{k\colon \tilde\MLreg_k > \MLreg_k\}$. 

The first step is equivalent to adding the term $\sum_{k\in \mathcal{K}}\frac{\tilde\MLreg_k - \MLreg_k}{2}(\weight_k - \MLcenter_k)^2$ to the loss function~\eqref{eq:loss_function}. We can interpret this as adding an $(N+1)$-th Hamiltonian $\frac{1}{2}\weightvec^T\MLregmat_{+}\weightvec$ with corresponding time variable $\HJt_{N+1} = 1$ to the multi-time HJ PDE, where $\MLregmat_+$ is a diagonal matrix whose $k$-th diagonal element is $\tilde\MLreg_k - \MLreg_k$ if $k\in\mathcal{K}$ and $0$ otherwise.
Therefore, the solution to this new multi-time HJ PDE can be solved by the following Riccati equation:
\begin{equation}\label{eqt:riccati_increase_regweight}
\begin{dcases}
\dot{\Sxx}_+(\HJt) =  -\Sxx_+(\HJt)^T\MLregmat_+\Sxx_+(\HJt) &\HJt\in \left(0,1\right),\\
\dot{\Sx}_+(\HJt) = -\Sxx_+(\HJt)^T\MLregmat_+\Sx_+(\HJt)&\HJt\in \left(0, 1\right),
\end{dcases}
\end{equation}
with initial condition $\Sxx_+(0)$ and $\Sx_+(0)$, which are the corresponding solutions to the Riccati equations before changing the weights $\MLreg_k, k\in\mathcal{K}$. In other words, we set $\Sxx_+(0) = \Sxx\left(T_N\right)$ and $\Sx_+(0) = \Sx\left(T_N\right)$, where $\Sxx(T_N), \Sx(T_N)$ are obtained from solving the original learning problem~\eqref{eq:loss_function} with the original values of $\MLreg_k$.

The second step is equivalent to removing the term $\sum_{k\not\in\mathcal{K}}\frac{\MLreg_k - \tilde\MLreg_k}{2}(\weight_k - \MLcenter_k)^2$ from the loss function~\eqref{eq:loss_function}. This is equivalent to solving a single-time HJ PDE with a terminal condition at time $1$ and Hamiltonian $\frac{1}{2}\weightvec^T\MLregmat_{-}\weightvec$, where $\MLregmat_-$ is a diagonal matrix whose $k$-th diagonal element is $\MLreg_k - \tilde\MLreg_k$ if $k\not\in\mathcal{K}$ and $0$ otherwise.
Then, the solution can be obtained by solving the following Riccati equation:
\begin{equation}\label{eqt:riccati_decrease_reg_weight}
\begin{dcases}
\dot{\Sxx}_-(\HJt) =  -\Sxx_-(\HJt)^T\MLregmat_-\Sxx_-(\HJt) &\HJt\in \left(0,1\right),\\
\dot{\Sx}_-(\HJt) = -\Sxx_-(\HJt)^T\MLregmat_-\Sx_-(\HJt)&\HJt\in \left(0, 1\right),
\end{dcases}
\end{equation}
with terminal condition $\Sxx_-(1) = \Sxx_+(1)$ and $\Sx_-(1) = \Sx_+(1)$, where $\Sxx_+, \Sx_+$ are obtained from the solution to~\eqref{eqt:riccati_increase_regweight}.

Finally, the minimizer of the new learning problem after changing all of the weights $\MLreg_k$ in the regularization term is given by ${\Sxx}_-(0) \tilde\MLregmat \MLcentervec+ {\Sx}_-(0)$, where ${\Sxx}_-, {\Sx}_-$ are obtained from the solution to~\eqref{eqt:riccati_decrease_reg_weight} and $\tilde\MLregmat$ is a diagonal matrix whose $k$-th diagonal element is the new regularization parameter $\tilde\MLreg_k$.




\section{Additional results for example 2}\label{appendix:1}
In Section~\ref{subsec:example_2}, we consider a 1D steady-state linear reaction-diffusion equation and discuss three different types of post-training calibrations: adding new data points to compensate for a lack of knowledge in the regular training, enforcing the fitting of some data points by increasing the weights $\lambda_i$ of their respective terms in the loss function~\eqref{eq:calibration:loss}, and removing some data points so that their  effects are eliminated. In this section, we present the results for the last case. In Figure~\ref{fig:calibration:2}, we remove two outliers one-by-one and observe that their removal does successfully increase the accuracy of the learned model. Again, using our Riccati-based approach, the removal of these points is done using only knowledge about the point to be removed and the results of the previous training step. 

\begin{figure}[ht]
    \begin{subfigure}[b]{\textwidth}
        \centering
        \includegraphics[width = 0.3\textwidth]{calibration/f_1_new.png}
        \includegraphics[width = 0.3\textwidth]{calibration/f_2_new.png}
        \includegraphics[width = 0.3\textwidth]{calibration/f_3_new.png}
        \caption{}
    \end{subfigure}
    \begin{subfigure}[b]{\textwidth}
        \centering
        \includegraphics[width = 0.3\textwidth]{calibration/u_1_new.png}
        \includegraphics[width = 0.3\textwidth]{calibration/u_2_new.png}
        \includegraphics[width = 0.3\textwidth]{calibration/u_3_new.png}
        \caption{}
    \end{subfigure}
    \caption{Results of solving the 1D steady-state reaction-diffusion equation~\eqref{eq:reaction} with noisy measurements of the source term $f$ in the domain and noiseless measurements of the solution $u$ on the boundary. (a) results for $f$; (b) results for $u$. \textbf{Left}: results of regular training using our Riccati-based method in Section~\ref{subsec:method_1}; \textbf{middle} and \textbf{right}: calibrating the results of regular training by eliminating two outlier measurements of $f$ using the methodology described in Section~\ref{subsec:method_2}. Calibrations are employed without re-training or access to the data from the previous training.}
    \label{fig:calibration:2}
\end{figure}


\end{document}
