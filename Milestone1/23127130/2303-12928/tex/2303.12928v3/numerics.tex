In this section, we apply the Riccati-based methodology presented in Section~\ref{sec:method} to four test problems from machine learning to demonstrate the versatility and potential computational advantages of our new approach.  
In each example, we use RK4 with double precision to solve the Riccati ODEs when applying our Riccati-based methodology. The connections between the examples presented in this section, the Hopf formula, and  the corresponding optimal control problems can be found in Table~\ref{tab:learning_problems}. 
We note that in this work we use two metrics to evaluate our results quantitatively: the $\ell_1$-norm (defined as $\|x\|_1 = \sum_{i=1}^n |x_i|$ for $x\in\Rn$) for the finite-dimensional minimizer and the $L^2$-norm (defined as $\|f\|_2 = \left(\int_{x\in\Rn} |f(x)|^2 dx\right)^{1/2}$ for $f:\Rn\to \R$) for the inferred functions. The $L^2$-norm for functions is approximated using trapezoidal rule with a uniform grid.
Supplementary details of the numerical experiments can be found in Appendix \ref{sec:details}. Code for all examples will be made publicly available at \url{https://github.com/ZongrenZou/HJPDE4SciML}.