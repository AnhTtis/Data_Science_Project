\documentclass[journal=jctcce,manuscript=article]{achemso}

\usepackage[version=3]{mhchem} 
\usepackage{amsmath,amssymb,amsthm, amsfonts, mathtools, dsfont} 
\usepackage{bbm,bm,tensor, braket}
\usepackage{eqnarray,array,enumerate}
\usepackage{csquotes}
\usepackage{siunitx}
\usepackage{booktabs}
\usepackage{graphicx,wrapfig, caption, float, subcaption, epstopdf, setspace}
\usepackage{hyperref}
\usepackage{longtable}
\usepackage[section]{placeins}
\usepackage{multicol, multirow}
\usepackage{pgfplots}
\pgfplotsset{compat=1.9}
\usepgfplotslibrary{fillbetween}
\usepackage{algorithm, algpseudocode}

\usepackage{cleveref}
\newcommand{\crefrangeconjunction}{--}
\setcounter{secnumdepth}{1}

\newcommand*{\citen}[1]{%
  \begingroup
    \romannumeral-`\x % remove space at the beginning of \setcitestyle
    \setcitestyle{numbers}%
    \cite{#1}%
  \endgroup   
}
\usepackage{mathabx}
\def\br{\ensuremath\bm{r}}
%%%%% Document
\author{Arno Förster}
\email{a.t.l.foerster@vu.nl}
\affiliation{Theoretical Chemistry, Vrije Universiteit, De Boelelaan 1083, NL-1081 HV, Amsterdam, The Netherlands}

\author{Erik van Lenthe}
\email{vanlenthe@scm.com}
\affiliation{Software for Chemistry and Materials NV, NL, 1081HV, Amsterdam, The Netherlands}

\author{Edoardo Spadetto}
\affiliation{Software for Chemistry and Materials NV, NL, 1081HV, Amsterdam, The Netherlands}

\author{Lucas Visscher}
\affiliation{Theoretical Chemistry, Vrije Universiteit, De Boelelaan 1083, NL-1081 HV, Amsterdam, The Netherlands}


\title{Two-component $GW$ calculations: Cubic scaling implementation and comparison of partially self-consistent variants}

\keywords{GW, Spin-orbit coupling, Vertex corrections, Ionization potential, Heavy elements}

\DeclareUnicodeCharacter{2212}{-}
\begin{document}

\begin{abstract}
We report an all-electron, atomic orbital (AO) based, two-component (2C) implementation of the $GW$ approximation (GWA) for closed-shell molecules. Our algorithm is based on the space-time formulation of the GWA and uses analytical continuation of the self-energy, and pair-atomic density fitting (PADF) to switch between AO and auxiliary basis. By calculating the dynamical contribution to the $GW$ self-energy at a quasi-one-component level, our 2C $GW$ algorithm is only about a factor of two to three slower than in the scalar relativistic case. Additionally, we present a 2C implementation of the simplest vertex correction to the self-energy, the statically screened $G3W2$ correction. Comparison of first ionization potentials of a set of 60 molecules with heavy elements (a subset of the SOC81 set) calculated with our implementation against results from the WEST code reveals mean absolute deviations of around 140 meV for $G_0W_0$@PBE and 150 meV for $G_0W_0$@PBE0. These are most likely due to technical differences in both implementations, most notably the use of different basis sets, pseudopotential approximations, different treatment of the frequency dependency of the self-energy and the choice of the 2C-Hamiltonian. However, how much each of these differences contribute to the observed discrepancies is unclear at the moment. Finally, we assess the performance of some (partially self-consistent) variants of the GWA for the calculation of first IPs for a set of 81 molecules with heavy elements (SOC81). Quasi-particle self-consistent $GW$ (qs$GW$) and eigenvalue-only self-consistent $GW$ (ev$GW$) agree best with vertical experimental reference values, even though they systematically overestimate the IPs. For the most accurate $GW$ variants, we further show that the perturbative $G3W2$ correction worsens the agreement with experiment and that explicit treatment of spin-orbit effects at the 2C level is crucial for systematic agreement with experiment. 
\end{abstract}

\maketitle

\section{Introduction}
Due to its favorable price-to-performance ratio, the $GW$ approximation (GWA)\cite{Hedin1965, martin2016} ($G$: single-particle Green's function, $W$: screened electron-electron interaction) is one of the most popular methods for the calculation of charged excitations in finite systems.\cite{Reining2018,Golze2019} Over the last decade, the GWA has been implemented into a large number of electronic structure codes\cite{Ren2012, Caruso2012, Caruso2013,
VanSetten2013,
Kaplan2015,
Kaplan2016,
Bruneval2016a,
Foerster2011a,Koval2014,
Mejia-Rodriguez2021,Forster2020b,Forster2021a,
Wilhelm2016a, Wilhelm2018, Wilhelm2021,
Ke2011} and $GW$ implementations for massively parallel architectures,\cite{Govoni2015, Wilhelm2016a, DelBen2019, DelBen2019a, Yu2022} low-order scaling implementations,\cite{Wilhelm2018, Wilhelm2021, Forster2020b, Forster2021a, Duchemin2021a} effectively linear scaling stochastic formulations,\cite{Vlcek2017, Vlcek2018} fragment-based approaches\cite{Fujita2018, Fujita2019, Winter2021, Amblard2022} or embedding techniques\cite{Romanova2020, Weng2021,Tolle2021} have enabled applications of the $GW$ method to large biomolecules,\cite{Forster2021a,Forster2022c} nanostructures\cite{BorinBarin2022, Amblard2022, Yu2022} or interfaces.\cite{Yu2022} 

A large numbers of studies has by now contributed to a thorough understanding of the impact of technical aspects of these implementations, like the choice of single-particle basis, pseudopotential (PP) approximations, or frequency treatment,\cite{VanSetten2015, Maggio2016, Govoni2018, Gao2019,Bruneval2020,Forster2021a} as well as the performance of various $GW$ approaches for the first ionization potentials (IP) and electron affinities (EA) of weakly correlated organic molecules.\cite{Bruneval2009, Marom2012, Bruneval2013, Ren2015, Knight2016, Rangel2016, Caruso2016, Forster2022} More recently, the GWA has also been benchmarked for core excitations\cite{Golze2018, VanSetten2018a, Golze2020, Yao2022, Li2022a} and strongly correlated systems like open-shell molecules\cite{Mansouri2021} or transition metal compounds with partially filled 3$d$ shells.\cite{Korbel2014, Berardo2017, Hung2017, Shi2018, Byun2019, Rezaei2021, Wang2022} Fully self-consistent $GW$ (sc$GW$) calculations are relatively expensive, technically demanding, and not necessarily very accurate for the calculation of IPs EAs.\cite{Marom2012, Caruso2016, Knight2016} Instead, the much cheaper perturbative $G_0W_0$ approach\cite{Hybertsen1985, Hybertsen1986} or its eigenvalue-only self-consistent extension (ev$GW$) are typically the method of choice. Despite their often excellent accuracy, these methods fail when the KS orbitals for which the $GW$ corrections are evaluated are qualitatively wrong.\cite{Bruneval2013, Knight2016, Forster2022c} In the quasi-particle self-consistent $GW$ method (qs$GW$),\cite{Faleev2004, VanSchilfgaarde2006, Kotani2007} the frequency dependent and non-Hermitian $GW$ self-energy is mapped self-consistently to an effective static and Hermitian non-local potential which is a functional of the non-interacting single-particle Green's function. Therefore, the results are strictly independent of the KS density functional which is used as starting-point for the calculation.\cite{Forster2021a, Forster2022c} The available benchmark data suggest that for molecules qs$GW$ is at least as accurate as $G_0W_0$.\cite{Ke2011, Gui2018, Forster2022} 

Less is known about the accuracy of the GWA for molecules containing heavier elements. One reason for this is that for those systems only a limited number of accurate first-principle results are available.\cite{Akinaga2017, Shee2018} Another reason is that comparison to experimental data is complicated by spin-orbit coupling (SOC) whose explicit treatment requires to implement the GWA in a 2-component (2C) framework. While Aryasetiawan and coworkers have generalized Hedin's equation to spin-dependent interactions\cite{Aryasetiawan2008, Aryasetiawan2009} more than a decade ago, only a few 2C implementations of the GWA for molecules have been realized so far.\cite{Kuhn2015, Scherpelz2016, Holzer2019, Franzke2022, Holzer2023} The probably most systematic study of SOC effects in molecules has been performed by Scherpelz and Govoni\cite{Scherpelz2016} who have compiled a set of 81 molecules containing heavy elements (referred to as SOC81 in the following).\cite{Scherpelz2016} They performed two-component (2C) $GW$@PBE\cite{Perdew1996} and $GW$@PBE0\cite{Adamo1999, Ernzerhof1999} calculations for this set using the WEST code\cite{Govoni2015, Yu2022} and found that SOC can shift scalar relativistic (1C) first ionization potentials by up to 400 meV for molecules containing Iodine.\cite{Scherpelz2016} Interestingly, they observed that the 1C results were often closer to experiment than the 2C ones. Also, the fact that $GW$@PBE and $GW$@PBE0 are not necessarily very accurate for molecules\cite{Caruso2016,Knight2016,Wang2021,Zhang2022} suggests that the good performance of those methods for these systems might at least partially be due to fortuitous error cancellation. The accuracy of $G_0W_0$ calculations based on starting points with a higher fraction of exact exchange has however not been systematically investigated for molecules containing heavy elements. Also, little is known about the performance of partially self-consistent approaches.

In efforts to improve over the $GW$ approximation, also the role of higher order terms in the expansion of the electronic self-energy in terms of $W$ (vertex corrections), has been assessed over the last years for small and medium molecules.\cite{Ren2015, Ma2019a, Vlcek2019, Pavlyukh2020, Bruneval2021a, Wang2021, Wang2022a, Forster2022, Mejuto-Zaera2022a} The available results suggest that they generally fail to improve consistently over the best available $GW$ variants when they are combined with QP approximations.\cite{Kutepov2017, Kutepov2017a, Forster2022} However, they can remove some of the starting point dependence of $G_0W_0$\cite{Wang2021, Wang2022a} and often tremendously improve the description of electron affinities.\cite{Vlcek2019, Forster2022a} With the exception of one recent study which focused on first-row transition metal oxides,\cite{Wang2022a} the available benchmark results are limited to charged valence excitations in mostly organic molecules. It is not known how these methods perform for molecules containing heavier elements, where electron correlation effects and screening effects might be stronger.

In this work, we address some of these open questions. We present systematic benchmarks of 2C-GWA at different levels of self-consistency, ranging from $G_0W_0$ to qs$GW$. We also investigate the effect of the statically screened $G3W2$ term\cite{Forster2022} on the QP energies in a 2C framework. Our calculations are performed using a newly developed 2C (qs)$GW$ implementation, a generalization of our atomic orbital based qs$GW$ and $G_0W_0$ algorithms.\cite{Forster2020b,Forster2021a} Our 2C implementation retains the same favorable scaling with system size and increases the prefactor of the calculations by only a factor of two compared to the 1C case. This relatively small increase in computational effort is achieved by calculating the dynamical contributions to the electron self-energy at a  quasi-one-component level. Therefore, our new implementation also allows us to describe SOC effects in large molecules. All other quantities, including the polarizability, are treated at the full 2C level without any further approximations.

The remainder of this paper is organized as follows: In section~\ref{sec::theory}, we review the 2C-$GW$ working equations and give a detailed overview of our implementation. After describing the details of our calculations in section~\ref{sec::computationalDetails}, we report the results of our detailed benchmark calculations in section~\ref{sec::theory}: First, to assess the influence of the different technical parameters in both implementations, we compare $G_0W_0$@PBE0 IPs for SOC81 to the ones from Scherpelz and Govoni\cite{Scherpelz2016}. We then use our new implementation to calculate the first ionization potentials of the molecules in the SOC81 database using some of the most accurate available $GW$ approaches: qs$GW$, eigenvalue-only self-consistent $GW$ (ev$GW$) and $G_0W_0$ based on hybrid starting points with different fractions of exact exchange. Finally, section~\ref{sec::conclusions} summarizes and concludes this work.

\section{\label{sec::theory}Theory}
\subsection{$GW$ approximation and $G3W2$ correction}
The central object of this work is the $GW + G3W2$ self-energy,
\begin{equation}
\label{self-energy-full}
    \Sigma^{GW + G3W2}(1,2) = 
    \Sigma_H(1,2) + \Sigma^{GW}(1,2) + 
    \Sigma^{G3W2}(1,2) \;.
\end{equation}
Here, 
\begin{equation}
    \Sigma_H(1,2) = v_H(1)\delta(1,2)
    = -i \delta(1,2)\int d3\; v_c(1,3)G(3,3^+) \;,
\end{equation}
with the Hartree-potential $v_H$,
\begin{equation}
\label{gw-self-energy-full}
    \Sigma^{GW}(1,2) = i G(1,2)W(1,2)  
\end{equation}
and    
\begin{equation}
\label{g3w2-self-energy-full}
\Sigma^{G3W2}(1,2) = -
\int d3 d4 G(1,3)W(1,4)G(3,4)G(4,2) W(3,2) \;.
\end{equation}
Space, spin, and imaginary time indices are collected as $1 = (\bm{r}_1,\sigma_1,i\tau_1)$. $W$ is the screened Coulomb interaction which is obtained by the Dyson equation \begin{equation}
\label{screened-coulomb-2}
   W(1,2) = W^{(0)}(1,2) + \int d3 d4 W^{(0)}(1,3)
   P^{(0)}(3,4)W(4,2) \;.
\end{equation}
Here, 
\begin{equation}
    W^{(0)}(1,2) = v_c(\bm{r}_1,\bm{r}_2)\delta_{\sigma,\sigma'}\delta(t_1-t_2) \;,
\end{equation}
is the bare Coulomb interaction and $P^{(0)}$ is the polarizability in the random phase approximation (RPA),
\begin{equation}
\label{chi0}
    P^{(0)}(1,2) = -i G(1,2)G(2,1) \;.
\end{equation}
Finally, $G$ is the interacting single-particle Green's function which is connected to its non-interacting counterpart $G^{(0)}$ by a Dyson equation with the electronic self-energy \eqref{self-energy-full} as its kernel,
\begin{equation}
\label{dyson}
    G(1,2) = G^{(0)}(1,2) + \int d3 d4 G^{(0)}(1,3) \Sigma(3,4) G(4,2) \;.
\end{equation}
If necessary, one can transform all quantities to imaginary frequency using the Laplace transform\cite{Rieger1999}
\begin{equation}
\label{TtoW}
    f(i\omega) = -i \int d\tau F(i\tau) e^{i\omega \tau} \;.
\end{equation}
The self-consistent solution of \cref{gw-self-energy-full,screened-coulomb-2,chi0,dyson} is referred to as $GW$ approximation. 

Typically, \eqref{dyson} is approximated. To this end, one defines an auxiliary Green's function $G^{(s)}$ which is related to $G^{(0)}$ by  
\begin{equation}
\label{dyson-aux}
    G^{(s)} = G^{(0)}(1,2) + \int d3 d4 G^{(0)}(1,3) v_{Hxc}(3,4) G^{(s)}(4,2) \;,
\end{equation}
where $v_{Hxc}$ is a (potentially local) generalized Kohn-Sham\cite{Hohenberg1964,Kohn1965, Seidl1996} Hartree-exchange-correlation potential. $G$ is then obtained from $G^{(s)}$ by 
\begin{equation}
\label{dyson-approx}
    G(1,2) = G^{(s)}(1,2) + \int d3 d4 G^{(s)}(1,3) \left[\Sigma_{Hxc}(3,4) - v_{Hxc}(3,4)\right]G(4,2) \;.
\end{equation}
In the basis of molecular orbitals (MO) $\left\{\phi_k\right\}$, $G^{(s)}$ is diagonal, 
\begin{equation}
\label{g_time-ordered_mo}
G^{(s)}_{pp'} = \Theta(i\tau)G^{>}_{pp'}(i\tau) - \Theta(-i\tau)G^{<}_{pp'}(i\tau) \;,
\end{equation}
with greater and lesser propagators being defined as 
\begin{equation}
\label{g_g_mo}
G^{>}_{pp'}(i\tau) = -i \Theta(\epsilon_p) e^{-\epsilon_p\tau}
\end{equation}
and
\begin{equation}
\label{g_l_mo}
G^{<}_{pp'}(i\tau) = -i \Theta(-\epsilon_p) e^{-\epsilon_p\tau} \;.
\end{equation}
Here, it is understood that all QP energies $\epsilon_k$ and KS eigenvalues $\epsilon^{KS}_k$ are measured relative to the chemical potential $\mu$ which we place in the middle of the HOMO-LUMO gap. $\Theta$ is the Heaviside step-function and $p,q,r,s \dots$ denote spinors. Under the assumption that the KS eigenstates are a good approximation to the $GW$ eigenstates, the off-diagonal elements of the operator $\Sigma_{Hxc} - v_{Hxc}$ in \eqref{dyson-approx} can be neglected. This leads to
\begin{equation}
\label{g0w0-equation}
   \left[\Sigma_{xc}\right]_{pp}(\epsilon_p) - \left[v_{xc}\right]_{pp} 
   = \epsilon_p - \epsilon^{KS}_p \;,
\end{equation}
Solving this equation as a perturbative correction is referred to as $G_0W_0$, while in ev$GW$, \cref{gw-self-energy-full,screened-coulomb-2,chi0,g0w0-equation} are solved self-consistently instead. Splitting the operator $\Sigma_{Hxc} - v_{Hxc}$ in \eqref{dyson-approx} into Hermitian and anti-Hermitian part and discarding the latter one, the solution of \eqref{dyson-approx} can be restricted to its QP part only.\cite{Layzer1963, Sham1966, Huser2013, Nakashima2021} Restricting the self-energy further to its static limit, a single-particle problem similar to the KS equations is obtained, 
\begin{equation}
\label{DysonqsGW}
    \sum_q\left\{\left[\Sigma^H_{Hxc}\right]_{pq} - \left[v_{Hxc} \right]_{pq}
    \right\}\phi_q(\br) = \left(\epsilon_p - \epsilon^{KS}_p\right) \phi_p(\br) \;,
\end{equation}
where $\Sigma^H = \frac{1}{2}\left(\Sigma + \Sigma^{\dagger}\right)$ denotes the Hermitian part of the self-energy. Solving \cref{gw-self-energy-full,screened-coulomb-2,chi0,DysonqsGW} self-consistently is referred to as the qs$GW$\cite{Faleev2004, VanSchilfgaarde2006, Kotani2007} approximation.\bibnote{It should be understood that in practice one solves 
\begin{equation}
\label{DysonqsGW2}
    \sum_q\left\{
    \left[\Sigma^H_{Hxc}\right]_{pq} - 
    \left[\Sigma^{H^{(n-1)}}_{Hxc}\right]_{pq}
    \right\}\phi_q(\br) = \left(\epsilon_p - \epsilon^{(n-1)}_p\right) \phi_p(\br) 
\end{equation}
in the $n$th iteration, which reduces to \eqref{DysonqsGW} for $n=1$.} There are many possible ways to construct the qs$GW$ Hamiltonian.\cite{Kotani2007, Shishkin2007, Kutepov2012, Kutepov2017b, Friedrich2022} In our implementation, we use the expression 
    \begin{equation}
    \label{ksf2}
        \left[\Sigma^{(GW)}\left(\left\{\epsilon_n\right\}\right)\right]_{pq}= 
        \begin{cases}
        \left[\Sigma^{(GW)}(\epsilon_p)\right]_{pq} & p = q \\ 
        \left[\Sigma^{(GW)}(\tilde{\epsilon})\right]_{pq} & \text{else} \;. 
        \end{cases}
    \end{equation}
with $\tilde{\epsilon} = 0$. If, as in our implementation\cite{Forster2021a}, the self-energy on the real frequency axis is calculated via analytical continuation (AC), eq.~\eqref{ksf2} is numerically more stable\cite{Forster2021a,Lei2022} than constructions of the qs$GW$ Hamiltonian in which also the off-diagonal elements are evaluated at the QP energies.\cite{Kotani2007,Kaplan2016}

% \subsubsection{Choice of the qs$GW$ Hamiltonian} 
% In practice, there are different possible choices for the frequency values at which the matrix elements of the Hermitian part of the self-energy are evaluated. Kotani, van Schilfgaarde and Faleev (KSF) suggested that the self-energy should be evaluated at the positions of the QP energies of the previous iteration,
% \begin{equation}
% \left[\Sigma^H_{Hxc}\right]_{nm} = 
% \frac{1}{2}\left[\Sigma^{(GW)}\left(\left\{\epsilon^{QP^{(n-1)}}_n\right\}\right)\right]_{nm}
%     + \frac{1}{2}\left[\Sigma^{{(GW)}^*}\left(\left\{\epsilon^{QP^{(n-1)}}_n\right\}\right)\right]_{nm} \;.
% \end{equation}
% For the diagonal elements, this results in a zeroth-order expansion of \eqref{g0w0-equation} around the QP energies from the previous iteration, 
% \begin{equation}
% \epsilon_n
%        \left[\Sigma^H_{Hxc}\right]_n(\epsilon^{QP^{n-1}}_n) - \left[\Sigma^{H^{n-1}}_{Hxc}\right]_n 
%    =  - \epsilon^{QP^{n-1}}_n \;.
% \end{equation}
% This choice seems to be justified when the qs$GW$ equations are close to convergence and the differences between KS eigenvalues from two subsequent iterations are small. Friedrich and coworkers have noticed that for large differences between QP energies from two subsequent iterations this might lead to convergence towards an unphysical solution.\cite{Aguilera2015,Friedrich2022} Therefore, one might choose the QP energies at which the self-energy is evaluated as the solutions of \eqref{g0w0-equation}.\cite{Friedrich2022} This can be done by linearizing the QP equations around the QP energies from the previous iterations,\cite{Shishkin2007} or by solving the non-linear equations \eqref{g0w0-equation} and calculate the self-energy at the resulting QP energies. In this work we refer to this variant as qs$GW$-II. Typically, there is little difference between the qs$GW$ and qs$GW$-II results, even though qs$GW$-II typically results in slightly higher ionization potentials. However, qs$GW$-II does not help in cases in which the qs$GW$ equations are difficult to converge since such issues are typically related to multiple QP solutions.\cite{Forster2021a}

% Another important consideration from the viewpoint of numerical stability are the energies at which the off-diagonal elements of the self-energy are evaluated. As we will describe in the following, we calculate the self-energy on the real axis by AC. It is well known, that AC is most accurate close to the Fermi energy while for core states or high-lying virtuals it produces at best unreliable results. Therefore, if the qs$GW$ equations are implemented with AC of the self-energy,\cite{Forster2021a,Lei2022} the expression
% \begin{equation}
% \label{ksf2}
%     \left[\Sigma^{(GW)}\left(\left\{\epsilon_n\right\}\right)\right]_{nm}(\epsilon_n) = 
%     \begin{cases}
%     \left[\Sigma^{(GW)}(\tilde{\epsilon})\right]_{nm} & n = m \\ 
%     \left[\Sigma^{(GW)}(\tilde{\epsilon})\right]_{nm} & \text{else} \;. 
%     \end{cases}
% \end{equation}
% is numerically more stable than variants at which also the off-diagonal elements are evaluated at the QP energies\cite{Kotani2007,Kaplan2016}. We note, that Kutepov and co-workers suggested a construction of the qs$GW$ Hamiltonian which by-passes AC entirely and exclusively relies on imaginary axis data.\cite{Kutepov2012,Kutepov2017b} 

% In our implementation\cite{Forster2021a}, we use \eqref{ksf2} with $\tilde{\epsilon} = 0$ since it leads to the best agreement with experiment. Notice, that KSF have suggested to use \eqref{ksf2} with $\tilde{\epsilon} = \epsilon_F$. The variant by Kutepov\cite{Kutepov2012,Kutepov2017b}  could potentially be a suitable alternative for qs$GW$ implementations which rely on AC. However, since the construction \eqref{ksf2} is however numerically stable, we did not consider the variant from ref.~\citen{Kutepov2012,Kutepov2017b} herein.

\subsection{Kramers-restricted two-component formalism}
Recently, an 2C implementation of the GWA for Kramers-unrestricted systems has been implemented by Holzer with $\mathcal{O}\left( N^4\right)$ scaling with system size.\cite{Holzer2023} In this work we will focus on application to closed-shell molecules with no internal or external magnetic fields. This allows us to simplify the treatment considerably as it possible to define a Kramers-restricted set of spinors in which pairs of spinors are related by time-reversal symmetry.

We expand each molecular spinor in a primary basis of atomic orbitals (AO), $\left\{\chi_{\mu}\right\}_{\mu = 1, \dots, N_{\text{bas}}}$, as
\begin{equation}
\label{spinors}
 \phi_k(\br)  =   \left( \begin{array}{c} \phi^\uparrow_k(\br) \\ \phi^\downarrow_k(\br) \end{array} \right)
  = \sum_{\mu} \left( \begin{array}{c} b_{k \uparrow \mu} \chi_\mu(\br) \\ b_{k \downarrow \mu} \chi_\mu(\br) \end{array} \right) 
  = \sum_{\mu} \left( \begin{array}{c} (b_{k\uparrow\mu}^{R} + i b_{k\uparrow\mu }^{I}) \chi_\mu(\br) \\ (b_{k\downarrow\mu}^{ R} + i b_{k\downarrow \mu}^{I}) \chi_\mu(\br) \end{array} \right) \;,
\end{equation}
where $\uparrow$ ($\sigma=\frac{1}{2}$) and $\downarrow$ ($\sigma=-\frac{1}{2}$) denote the different projections of spin on the $z$-axis. Each spinor $\phi_{k}$ can be related by the time-reversal symmetry or Kramers' operator $\hat{K}$ to a Kramers' partner $\phi_{\bar{k}}$ with the same energy, $\epsilon_{k}=\epsilon_{\bar{k}}$, 
\begin{equation}
\label{kramers-symmetry}
    \hat{K} \phi_{k} =
    \begin{pmatrix}
    \phi_{k}^{\uparrow} (\br) \\ 
    \phi_{k}^{\downarrow} (\br) \\
    \end{pmatrix} = 
     \begin{pmatrix}
    -\phi_{k}^{\downarrow^*} (\br) \\ 
    \phi_{k}^{\uparrow*} (\br) \\
    \end{pmatrix}   = 
      \begin{pmatrix}
    -\phi_{k}^{\downarrow,R} (\br) 
    + i \phi_{k}^{\downarrow,I} (\br)  \\ 
    \phi_{k}^{\uparrow,R} (\br) 
    - i \phi_{k}^{\uparrow,I} (\br)  \\ 
    \end{pmatrix} 
    = \phi_{\bar{k}}\;.
\end{equation}
Using quaternion algebra it is possible to reduce the dimension of matrices that need to be considered to half the original size\cite{Saue1999}. Alternatively, one may keep the full dimension, but use the spinor pairing to define matrices as either real or imaginary. We will take the latter approach in this work. Denoting pairs of spinors with $\left(p, \bar{p} \right)$, noting that $\hat{K} \phi_{\bar{p}}=-\phi_p$ and transforming a purely imaginary diagonal operator $A$ that obeys $A_{pp}=A_{\bar{p}\bar{p}}$ and $A_{pp}=-A^*_{pp}$ we can deduce
\begin{equation}
\label{Kramersrelation}
\begin{aligned}
    A_{\mu\nu, \uparrow \uparrow} = & \sum_{p}  b_{p \uparrow \mu } A_{pp} b^*_{p \uparrow \nu} + 
    \sum_{\bar{p}} 
    b_{\bar{p} \uparrow \mu } A_{\bar{p}\bar{p}} b^*_{\bar{p} \uparrow \nu} = 
    \sum_{\bar{p}} 
    b^*_{\bar{p} \downarrow \mu } A_{\bar{p}\bar{p}}  b_{\bar{p} \downarrow \nu} + 
    \sum_{p} 
    b^*_{p \downarrow \mu } A_{pp} b_{p \downarrow \nu} = - A_{\mu\nu, \downarrow \downarrow}^* \\
    A_{\mu\nu, \downarrow \uparrow} = & \sum_{p}  b_{p \downarrow \mu } A_{pp} b^*_{p \uparrow \nu} + \sum_{\bar{p}} b_{\bar{p} \downarrow \mu } A_{\bar{p}\bar{p}} b^*_{\bar{p} \uparrow \nu} = 
    - \sum_{p}  
    b^*_{\bar{p} \uparrow \mu } 
    A_{\bar{p}\bar{p}} b_{\bar{p} \downarrow \nu} 
    - \sum_{\bar{p}} 
    b^*_{p \uparrow \mu } A_{pp} b_{p \downarrow \nu} = A_{\mu\nu, \uparrow \downarrow}^* \;.
\end{aligned}
\end{equation}
Is is convenient to split this operator into real and imaginary components, and we use the character of the MO coefficient products to label real (superscript R) and imaginary (superscript I) parts of the operator,  
\begin{equation}
   A^{R}_{\mu \nu, \sigma \sigma'} =  
   \sum_{p}  b^R_{p \sigma \mu } A_{pp} b^R_{p \sigma' \nu} + 
   \sum_{\bar{p} } b^R_{\bar{p} \sigma \mu } A_{\bar{p}\bar{p}} b^R_{\bar{p} \sigma' \nu} + 
   \sum_{p} b^I_{p \sigma \mu } A_{pp} b^I_{p \sigma' \nu} + 
   \sum_{\bar{p} } b^I_{\bar{p} \sigma \mu } A_{\bar{p}\bar{p}} b^I_{\bar{p} \sigma' \nu}
\end{equation}
and 
\begin{equation}
   A^{I}_{\mu \nu, \sigma \sigma'} =  
   \sum_{p}  b^R_{p \sigma \mu } A_{pp} b^I_{p \sigma' \nu} + 
   \sum_{\bar{p} } b^R_{\bar{p} \sigma \mu } A_{\bar{p}\bar{p}} b^I_{\bar{p} \sigma' \nu} - 
   \sum_{p} b^I_{p \sigma \mu } A_{pp} b^R_{p \sigma' \nu} - 
   \sum_{\bar{p} } b^I_{\bar{p} \sigma \mu } A_{\bar{p}\bar{p}} b^R_{\bar{p} \sigma' \nu} \;.
\end{equation}
The  time-ordered single-particle Green's function is an example of such an operator which therefore in AO basis obeys the relations
\begin{equation}
\label{greensKramers}
\begin{aligned}
    G^{\lessgtr}_{\mu \nu, \upuparrows}(i\tau) = & - G^{\lessgtr*}_{\mu \nu, \downdownarrows}(i\tau) \\
    G^{\lessgtr}_{\mu \nu, \updownarrows}(i\tau) = & G^{\lessgtr*}_{\mu \nu, \downuparrows}(i\tau) \;.
\end{aligned}
\end{equation}
Convenient is sometimes also to re-express these quantities in a spin matrix basis. We then get (denoting the
unit matrix as 0, and the Pauli spin matrices as x, y and z)
\begin{equation}
\label{greensGWbasisU}
\begin{aligned}
    G^{\lessgtr^0}_{\mu\nu}(i\tau) 
    = & G^{\lessgtr}_{\mu \nu, \upuparrows}(i\tau)  + G^{\lessgtr}_{\mu \nu, \downdownarrows} (i\tau)
    = & 2 G^{\lessgtr^R}_{\mu \nu, \upuparrows}(i\tau)  , \\
   G^{\lessgtr^x}_{\mu\nu}(i\tau) 
    = & G^{\lessgtr}_{\mu \nu, \updownarrows}(i\tau)  + G^{\lessgtr}_{\mu \nu, \downuparrows} (i\tau)
    = & 2 G^{\lessgtr^I}_{\mu \nu, \updownarrows}(i\tau)   ,\\
   G^{\lessgtr^y}_{\mu\nu}(i\tau) 
    = & i G^{\lessgtr}_{\mu \nu, \updownarrows}(i\tau)  - i G^{\lessgtr}_{\mu \nu, \downuparrows} (i\tau)
    = & 2i G^{\lessgtr^R}_{\mu \nu, \updownarrows}(i\tau)  ,\\
   G^{\lessgtr^z}_{\mu\nu}(i\tau) 
    = & G^{\lessgtr}_{\mu \nu, \upuparrows}(i\tau)  - G^{\lessgtr}_{\mu \nu, \downdownarrows} (i\tau) 
    = & 2 G^{\lessgtr^I}_{\mu \nu, \upuparrows}(i\tau)  \;,
\end{aligned}
\end{equation}
which more clearly shows the relation to 1-component theories in which only the first Green's function has a non-zero value.

\subsubsection{Polarizability in imaginary time} 
We next consider the polarizability\cite{Aryasetiawan2008,Aryasetiawan2009,Sakuma2011}. Whereas in the complete formalism of Aryasetiawan and Biermann\cite{Aryasetiawan2008} 
the polarizability includes the response of the charge density to magnetic fields as well as the induction of current densities, both of these are considered strictly zero in a Kramers-restricted formalism.  
We can then define the relevant part of the polarizability in AO basis as
\begin{equation}
\label{4terms}
P^{(0)}_{\mu\nu\sigma,\kappa\lambda\sigma'}(i\tau) = 
i\Theta(\tau) 
G^{>}_{\mu\kappa,\sigma\sigma'}(i\tau)
G^{<}_{\nu\lambda,\sigma'\sigma}(-i\tau)
+ i\Theta(-\tau) 
G^{<}_{\mu\kappa,\sigma\sigma'}(i\tau)
G^{>}_{\nu\lambda,\sigma'\sigma}(-i\tau) \;.
\end{equation} 
Due to the symmetry $P^{(0)}(i\tau) = P^{(0)}(-i\tau)$, we can focus on the first term which we split in terms of real (R) and imaginary (I) components
\begin{equation}
\begin{aligned}
G^>_{\mu \kappa, \sigma \sigma'}(i\tau)
G^<_{\nu \lambda, \sigma' \sigma}(-i\tau)
= & 
G^{>^R}_{\mu \kappa, \sigma \sigma'}(i\tau)
G^{<^R}_{\nu \lambda, \sigma' \sigma}(-i\tau) - 
G^{>^I}_{\mu \kappa, \sigma \sigma'}(i\tau)
G^{<^I}_{\nu \lambda, \sigma' \sigma}(-i\tau) \\ 
+ &
i G^{>^I}_{\mu \kappa, \sigma \sigma'}(i\tau)
G^{<^R}_{\nu \lambda, \sigma' \sigma}(-i\tau) + 
i G^{>^R}_{\mu \kappa, \sigma \sigma'}(i\tau)
G^{<^I}_{\nu \lambda, \sigma' \sigma}(-i\tau)  \;.
\end{aligned}
\end{equation}
Kramers symmetry implies
\begin{equation}
\label{kramers1}
\sum_{\sigma, \sigma' = \uparrow,\downarrow}
i G^{>^I}_{\mu \sigma, \kappa  \sigma'}(i\tau)
G^{<^R}_{\nu \sigma', \lambda \sigma}(-i\tau) + 
i G^{>^R}_{\mu \sigma,\kappa  \sigma'}(i\tau)
G^{<^I}_{\nu \sigma',  \lambda \sigma}(-i\tau) = 0 \;,
\end{equation} 
as well as 
\begin{equation}
\label{kramers2}
\begin{aligned}
P^{(0)}_{\mu\nu\uparrow,\kappa\lambda\uparrow}(i\tau) = & 
P^{(0)}_{\mu\nu\downarrow,\kappa\lambda\downarrow}(i\tau) \\
P^{(0)}_{\mu\nu\uparrow,\kappa\lambda\downarrow}(i\tau) = & 
P^{(0)}_{\mu\nu\downarrow,\kappa\lambda\uparrow}(i\tau) \;.
\end{aligned}
\end{equation}
We proof these relations in appendix~\ref{app::B}.
Already in the primary AO basis this would reduce the number of matrix elements that are to be calculated considerably. 
Further efficiency can however be gained by expanding the polarizability and the Coulomb potential in a basis of auxiliary functions $\left\{f_{\alpha}\right\}_{\alpha = 1, \dots, N_{\text{aux}}}$
with products of primary basis functions being expressed as
\begin{equation}
\label{fitting}
    \chi_\mu(\br) \chi_\nu(\br) = \sum_{\alpha} c_{\mu \nu \alpha}  f_{\alpha}(\br) \;.
\end{equation}

To calculate the fitting coefficients, we use the pair-atomic density fitting (PADF) method\cite{Watson2003, Krykunov2009, Merlot2013, Merlot2014, Wirz2017, Ihrig2015} in the implementation of ref.~\citen{Spadetto2023}. The following working equations are however completely general and can be implemented using any type of density fitting (DF). For instance, global density fitting using the overlap kernel\cite{Dunlap1979} (also known as RI-SVS) or the attenuated Coulomb kernel\cite{Feyereisen1993, Jung2005} which have already been used to achieve low-scaling $GW$ implementations\cite{Wilhelm2018,Wilhelm2021} would be suitable choice as well.

For the polarizability we can eliminate the explicit dependence on spin in the transformation to the auxiliary basis and work with the spin-summed form

\begin{equation}
P^{(0)}_{\alpha\beta}(i\tau)  = 
\sum_{\sigma \sigma' = \uparrow, \downarrow}
c_{\mu \nu \alpha}  
P^{(0)}_{\mu\kappa\sigma,\nu\lambda\sigma'}(i\tau)
c_{\kappa \lambda \beta} \;.
\end{equation}
Likewise we define spin-independent representations of the Coulomb potential and screened interaction in the auxiliary basis as
\begin{align}
\label{pol_aux}
v_{\alpha \beta} = & \int d \br d \br' f_{\alpha}(\br)v_c(\br,\br')f_{\beta}(\br') \\
W_{\alpha \beta} (i\tau) = & \int d \br d \br' f_{\alpha}(\br)W(\br,\br',i\tau)f_{\beta}(\br') \;,
\end{align}
Our final expression for the polarizability is 
\begin{equation}
\label{final}
\begin{aligned}
P^{(0)}_{\alpha\beta}(i\tau) = -2i 
c_{\mu \nu \alpha}  & 
\left\{
G^{>^R}_{\mu\kappa,\uparrow\uparrow}(i\tau)
G^{<^R}_{\nu\lambda,\uparrow\uparrow}(i\tau) - 
G^{>^I}_{\mu\kappa,\uparrow\uparrow}(i\tau)
G^{<^I}_{\nu\lambda,\uparrow\uparrow}(i\tau)  \right. \\
& \left. \quad + 
G^{>^R}_{\mu\kappa , \uparrow\downarrow}(i\tau)
G^{<^R}_{\nu\lambda , \uparrow\downarrow}(i\tau) - 
G^{>^I}_{\mu\kappa,\uparrow\downarrow}(i\tau)
G^{<^I}_{\nu\lambda, \uparrow\downarrow}(i\tau)
\right\}
c_{\kappa \lambda \beta} \;,
\end{aligned}
\end{equation}
or equivalently 
\begin{equation}
\label{final_alternate}
\begin{aligned}
P^{(0)}_{\alpha\beta}(i\tau) = -\frac{1}{2}i 
c_{\mu \nu \alpha}  & 
\left\{
G^{>^0}_{\mu \kappa }(i\tau)
G^{<^0}_{\nu \lambda }(i\tau) - 
G^{>^x}_{\mu \kappa }(i\tau)
G^{<^x}_{\nu \lambda }(i\tau)  \right. \\
& \left. \quad - 
G^{>^y}_{\mu \kappa}(i\tau)
G^{<^y}_{\nu  \lambda}(i\tau) - 
G^{>^z}_{\mu \kappa}(i\tau)
G^{<^z}_{\nu \lambda}(i\tau) 
\right\}
c_{\kappa \lambda \beta} \;.
\end{aligned}
\end{equation}
The first term in this expression is equivalent in the spin-restricted 1C formalism.\cite{Forster2020b} Evaluation of \eqref{final} or \eqref{final_alternate} is therefore exactly four times more expensive than in a scalar relativistic calculation. \Cref{final} can be implemented with quadratic scaling with system size using PADF.\cite{Forster2020b}

\subsubsection{Polarizability in imaginary frequency and MO basis}
The AO based implementation of the polarizability is advantageous for rather large molecules only and it is not suitable for the molecules in the SOC81 database typically containing just a few often heavy atoms. We therefore also implement the polarizability in MO space. In the following, we will use $i,j \dots$ to label occupied, and $a,b \dots$ to label virtual orbitals. Using \cref{g_time-ordered_mo} and these indices, \cref{4terms} becomes 
\begin{equation}
    P^{(0)}_{aiai}(i\tau) = -i\Theta(\tau) e^{-(\epsilon_a - \epsilon_i)\tau} - i\Theta(-\tau) e^{-(\epsilon_i - \epsilon_a)\tau} 
\end{equation}
in the MO basis. Using \eqref{TtoW}, the corresponding expression on the imaginary frequency axis is 
\begin{equation}
\label{P_mo}
    P^{(0)}_{aiai}(i\omega) =  - \frac{1}{\epsilon_a - \epsilon_i - i\omega} - \frac{1}{\epsilon_a - \epsilon_i + i\omega} \;.\\
\end{equation}
Using the last equation on the \emph{r.h.s.} of \eqref{spinors} and \eqref{fitting}, we can write down a transformation from the auxiliary basis to the MO basis as 
\begin{equation}
\label{transform1}
    \phi_i^\dagger(\br) \phi_a(\br) = \sum_{\alpha} c_{i a \alpha}  f_{\alpha}(\br) 
\end{equation}
with
\begin{equation}
\label{transform2}
    \begin{aligned}
         c_{i a \alpha}  = &  \sum_{\mu \kappa} (b_{i \uparrow \mu}^* b_{a \uparrow \kappa} + b_{i \downarrow \mu}^* b_{a  \downarrow \kappa} )c_{\mu \nu \alpha}    = c^R_{i a \alpha} + i c^I_{i a \alpha} \\
    = & \sum_{\mu \kappa} (b_{i \uparrow \mu}^R b_{a \uparrow \kappa}^R + b_{i \uparrow \mu}^I b_{a \uparrow \kappa}^I + b_{i \downarrow \mu}^R b_{a \downarrow \kappa}^R + b_{i \downarrow \mu}^I b_{a \downarrow \kappa}^I ) c_{\mu \nu \alpha}   \\
  & + i \sum_{\mu \kappa} (b_{i \uparrow \mu}^R b_{a \uparrow \kappa}^I - b_{i \uparrow \mu}^I b_{a \uparrow \kappa}^R + b_{i \downarrow \mu}^R b_{a \downarrow \kappa}^I - b_{i \downarrow \mu}^I b_{a \downarrow \kappa}^R ) c_{\mu \nu \alpha} \;.
    \end{aligned}
\end{equation}
Using this expression, \cref{P_mo} becomes
\begin{equation}
\begin{aligned}
    P^{(0)}_{\alpha\beta}(i\omega) = & c_{ai\alpha}P^{(0)}_{aiai}(i\omega)c_{ai\beta} \\ 
    = & 
    2 \left\{c^R_{i a \alpha} \text{Re} P^{(0)}_{aiai} -  
    c^I_{i a \alpha} \text{Im} P^{(0)}_{aiai}\right\} c^R_{i a \beta} +
    2 \left\{c^R_{i a \alpha} \text{Im} P^{(0)}_{aiai} +  
    c^I_{i a \alpha} \text{Re} P^{(0)}_{aiai}\right\} c^I_{i a \beta} \;.    
\end{aligned}
\end{equation}

\subsubsection{Screened interaction and self-energy}
If necessary, the polarizability is transformed to the imaginary frequency axis where the screened interaction is calculated in the basis of auxiliary functions using \cref{screened-coulomb-2}, 
\begin{equation}
\label{wFitting}
W_{\alpha\beta}(i\omega) = v_{\alpha\beta} + \sum_{\gamma \delta}v_{\alpha \gamma} P^{(0)}_{\gamma \delta}(i\omega) W_{ \delta \gamma}(i\omega) \;.
\end{equation}
For the evaluation of the self-energy, we partition the screened Coulomb interaction as
\begin{equation}
    \widetilde{W} = W - v \;.
\end{equation}
This allows us to use different approximations for the dynamical and static contributions to the self-energy. To evaluate the self-energy on the imaginary frequency axis, we first define the time-ordered self-energy\cite{VanLeeuwen2015} 
\begin{equation}
\Sigma_{xc}(i\tau) = \Sigma_x + \Theta(\tau)\Sigma_c^{>}(i\tau) - \Theta(-\tau)\Sigma_c^{<}(i\tau) \;.
\end{equation} 
Here, the greater and lesser components of the self-energy are given by 
\begin{equation}
\label{sigma_c}
    \left[\Sigma_c^{\lessgtr}\right]_{\mu\nu,\sigma\sigma'}(i\tau) = 
    iG^{\lessgtr}_{\kappa \lambda, \sigma \sigma'}(i\tau)
c_{\mu \kappa \alpha}
\widetilde{W}_{\alpha\beta} (i\tau) c_{\nu \lambda \beta}\;,
\end{equation}
and the singular contribution (Fock term) as 
 \begin{equation}
\label{sigma_x}
    \left[\Sigma_x\right]_{\mu\nu,\sigma\sigma'} = 
    iG^{<}_{\kappa \lambda, \sigma \sigma'}(i\tau \rightarrow 0^-)
c_{\mu \kappa \alpha}
v_{\alpha\beta} c_{\nu \lambda \beta}\;.
\end{equation}   

\paragraph{Dynamical contribution}

In the basis of Pauli matrices, \eqref{sigma_c} can be expanded as
\begin{equation}
\label{sigma_c_expanded}
\left[\Sigma_c^{\lessgtr}\right]_{\mu\nu}(i\tau)  = 
i \left( \begin{array}{cc} 
 G^{\lessgtr^0}_{\kappa \lambda}(i\tau) +   G^{\lessgtr^z}_{\kappa \lambda}(i\tau) & G^{\lessgtr^x}_{\kappa \lambda}(i\tau) - i G^{\lessgtr^y}_{\kappa \lambda}(i\tau) \\
 G^{\lessgtr^x}_{\kappa \lambda}(i\tau) + i G^{\lessgtr^y}_{\kappa \lambda}(i\tau) & G^{\lessgtr^0}_{\kappa \lambda}(i\tau) -   G^{\lessgtr^z}_{\kappa \lambda}(i\tau)
  \end{array} \right)
c_{\mu \kappa \alpha}
\widetilde{W}_{\alpha\beta} (i\tau) c_{\nu \lambda \beta}.
\end{equation}

In the correlation part of the self-energy we only calculate the contribution due to $G^{\lessgtr^0}$, i.e., $G^{\lessgtr^x}$,$G^{\lessgtr^y}$, $G^{\lessgtr^z}$ are set to zero. Therefore, using \eqref{greensGWbasisU}, eq.~\eqref{sigma_c_expanded} reduces to 
\begin{equation}
\label{sigma_c_expanded_d0}
\left[\Sigma_c^{\lessgtr}\right]_{\mu\nu}(i\tau)  = 
2i \left( \begin{array}{cc} 
G^{\lessgtr^{R}}_{\kappa \lambda,\upuparrows}(i\tau) & 0\\
 0 & G^{\lessgtr^{R}}_{\kappa \lambda,\upuparrows}(i\tau) 
  \end{array} \right)
c_{\mu \kappa \alpha}
\widetilde{W}_{\alpha\beta} (i\tau) c_{\nu \lambda \beta} \;.
\end{equation}
This quantity has the form as in the 1C formalism and in the same way as in our 1C implementation.\cite{Forster2020b} Notice also, that $G^{\lessgtr^{R}}$ has a prefactor of $-i$ due to the definitions \cref{g_g_mo,g_l_mo}. We Fourier transform \eqref{sigma_c} to the imaginary frequency axis using \cref{TtoW}, for which we follow the treatment of Liu et al.\cite{Liu2016} From there, the self-energy is transformed back to the MO basis and analytically continued to real frequencies using the algorithm by Vidberg and Serene.\cite{Vidberg1977} For details on the AC for $G_0W_0$ and qs$GW$ we refer to our previous work.\cite{Forster2020b, Forster2021a}
% and using the definitions \cref{greensGWbasisU}, we obtain
% \begin{equation}
% G^{\lessgtr^0}_{\kappa \lambda}(i\tau) =  
%              2 G^{\lessgtr^{R}}_{\mu \nu, \upuparrows}(i\tau)  
% \end{equation}

% and write the self-energy as 
% \begin{equation}
% \label{sigma_ir}
% \begin{aligned}
% \left[\Sigma_c^{\lessgtr}\right]_{\mu\nu}(i\tau) = &
% \sum_{\sigma = \uparrow,\downarrow} \left[
% G^{\lessgtr^{R}}_{\mu \nu, \sigma\sigma}(i\tau)
% + i 
% G^{\lessgtr^{I}}_{\mu \nu, \sigma\sigma}(i\tau)
% \right] 
% c_{\mu \kappa \alpha}
% \widetilde{W}_{\alpha\beta} (i\tau) 
% c_{\nu \lambda \beta} \\
% = & 
% \left[\Sigma_c^{\lessgtr^R}\right]_{\mu\nu}(i\tau) + i 
% \left[\Sigma_c^{\lessgtr^I}\right]_{\mu\nu}(i\tau) \;. 
% \end{aligned}
% \end{equation}

% \begin{equation}
% \begin{aligned}
%     \Sigma_c (i\omega) = & - \frac{i}{2}\int d\tau 
%     \cos(\omega\tau)
%     \left[
%     \Sigma_c(i\tau) + \Sigma_c(-i\tau)
%     \right] + \frac{1}{2}\int d\tau \sin(\omega\tau)
%     \left[
%     \Sigma_c(i\tau) - \Sigma_c(-i\tau)
%     \right] \\
%     = & 
%     -i \int^{\infty}_0 d\tau
%         \left[\Sigma^{>}_c(i\tau) - \Sigma^{<}_c(-i\tau)
%     \right] \cos (\omega \tau) + \int_0^{\infty} d\tau 
%         \left[\Sigma^{>}_c(i\tau) + \Sigma^{<}_c(-i\tau)
%     \right] \sin (\omega \tau) \\ 
%     = &    -i \int^{\infty}_0 d\tau
%         \left[\Sigma^{>^R}_c(i\tau) + i \Sigma^{>^I}_c(i\tau) - 
%               \Sigma^{<^R}_c(-i\tau) - i\Sigma^{<^I}_c(-i\tau)
%     \right] \cos (\omega \tau)  \\ 
%     & + \int_0^{\infty} d\tau 
%         \left[\Sigma^{>^R}_c(i\tau) + i\Sigma^{>^I}_c(i\tau) + 
%               \Sigma^{<^I}_c(-i\tau) + i\Sigma^{<^I}_c(-i\tau)
%     \right] \sin (\omega \tau) 
%     \end{aligned}
% \end{equation}
% where we have used \eqref{sigma_ir} in the last line. Therefore, our final expression for the dynamical contribution to the self-energy on the imaginary axis is 
% \begin{equation}
%     \Sigma(i\omega) = \Sigma^R(i\omega) + i \Sigma^I(i\omega) \;,
% \end{equation}
% with 
% \begin{equation}
%     \Sigma^R(i\omega) = \int^{\infty}_0 \sin(\omega \tau) 
%     \left[
%     \Sigma^{>^R}_c(i\tau) + \Sigma^{<^R}_c(-i\tau)
%     \right] d\tau + 
%     \int^{\infty}_0 \cos(\omega \tau) 
%     \left[
%     \Sigma^{>^I}_c(i\tau) - \Sigma^{<^I}_c(-i\tau)
%     \right] d\tau 
% \end{equation}
% and
% \begin{equation}
%     \Sigma^I(i\omega) = \int^{\infty}_0 \sin(\omega \tau) 
%     \left[
%     \Sigma^{>^I}_c(i\tau) + \Sigma^{<^I}_c(-i\tau)
%     \right] d\tau - 
%     \int^{\infty}_0 \cos(\omega \tau) 
%     \left[
%     \Sigma^{>^R}_c(i\tau) - \Sigma^{<^R}_c(-i\tau)
%     \right] d\tau \;.
% \end{equation}
% Notice, that $\Sigma^R(i\tau)$ is zero in the 1C case. 

\paragraph{Hartree-exchange contribution}
\Cref{sigma_x} is recovered from \eqref{sigma_c_expanded} by replacing $\widetilde{W}(i\tau)$ with $v_c$ and using $D = G^<(i\tau \rightarrow 0^-)$ instead of $G^<(i\tau)$. The resulting expression is identical to the ones typically implemented in 2C-Hartree--Fock codes,\cite{Armbruster2008, Desmarais2019}
\begin{equation}
\label{sigma_x_expanded}
\left[\Sigma_x\right]_{\mu\nu} = 
\left( \begin{array}{cc} 
 D^{^0}_{\kappa \lambda} +   
 D^{^z}_{\kappa \lambda} & 
 D^{^x}_{\kappa \lambda} 
 - i D^{^y}_{\kappa \lambda} \\
 D^{^x}_{\kappa \lambda}
 + i D^{^y}_{\kappa \lambda} & 
 D^{^0}_{\kappa \lambda} -   
 D^{^z}_{\kappa \lambda}
  \end{array} \right)
c_{\mu \kappa \alpha}
v_{\alpha\beta} c_{\nu \lambda \beta}
\end{equation}
where the different components of $D$ are obtained as the $i\tau \rightarrow 0$ limit of \cref{greensGWbasisU}.
% \begin{align}
% D^{^0}_{\kappa \lambda} &=   
% \sum_i^{occ}b_{i \uparrow \kappa}b_{i \uparrow\lambda}^\dagger +  
%  \sum_i^{occ}b_{i \downarrow \kappa}b_{i \downarrow \lambda}^\dagger \\
% D^{^x}_{\kappa \lambda}(i\tau) &=   
% \sum_i^{occ} b_{i \uparrow \kappa} b_{i \downarrow \lambda}^{\dagger}  +  
%  \sum_i^{occ}b_{i \downarrow \kappa} b_{i \uparrow \lambda}^{\dagger} \\
% D^{^y}_{\kappa \lambda}(i\tau) &= i 
%  \sum_i^{occ} b_{i \uparrow \kappa} b_{i \downarrow\lambda}^{\dagger}  -  
%  \sum_i^{occ}b_{i \downarrow \kappa}b_{i \uparrow\lambda}^{\dagger} \\
% D^{^z}_{\kappa \lambda}(i\tau) &=   
% \sum_i^{occ} b_{i \uparrow \kappa} b_{i \uparrow\lambda}^{ \dagger} -  
% \sum_i^{occ} b_{i \downarrow\kappa}b_{i \downarrow\lambda}^{\dagger} \;.
% \end{align}
In qs$GW$, we also need to evaluate the block-diagonal Hartree-contribution to the self-energy,
\begin{equation}
\label{sigma_H_expanded}
\left[\Sigma_H\right]_{\mu \nu} =    
\left( \begin{array}{cc} 
 D^{^0}_{\kappa \lambda}  & \\  & 
 D^{^0}_{\kappa \lambda} 
  \end{array} \right)
c_{\mu \nu \alpha}
v_{\alpha\beta} 
c_{\kappa \lambda \beta}
\end{equation}
The full qs$GW$ Hamiltonian is then constructed according to \cref{ksf2} and \cref{DysonqsGW2} is solved in the MO basis from the previous iteration. The new set of MO expansion coefficients and QP energies is then used to evaluate \cref{greensGWbasisU} in the next iteration.

\subsubsection{The $G3W2$ Correction}
As explained in ref.~\citen{Forster2022}, We evaluate the contribution of the $G3W2$ term to the self-energy as a perturbative correction to the solution of the GWA. Relying on the assumption that $GW$ already gives rather accurate QP energies we expand $\Sigma^{G3W2}$ around the $GW$ QP energies and obtain
\begin{equation}
\label{g3w2_qs}
    \epsilon^{GW + G3W2}_p = 
    \epsilon^{GW}_p + \Sigma_{pp}^{G3W2}(\epsilon^{GW}_{p}) \;,
\end{equation}
at zeroth order where $\Sigma_{pp}^{G3W2}$ is evaluated using the $GW$ QP energies obtained from the solution of \eqref{g0w0-equation} or \eqref{DysonqsGW}. We restrict ourselves to the statically screened $G3W2$ self-energy which is obtained from \eqref{g3w2-self-energy-full} by replacing both $W(1,2)$ with $W(1,2)\delta(t_1 - t_2)$.\cite{Forster2022} In terms of $G^{(s)}$ and in a basis of single-particle states (In case of $G_0W_0$ or ev$GW$ this would be the basis of KS states, in case of qs$GW$ the basis of qs$GW$ eigenstates), 
this term becomes\cite{Forster2020}
\begin{equation}
\label{SigmaG3W2-static}
\Sigma_{pp}^{G3W2}(\epsilon_{p}) = 
\sum^{occ}_{i}\sum^{virt}_{ab}
\frac{W(i\omega=0)_{paib} W(i\omega=0)_{aibp}}
{\epsilon_a + \epsilon_b  - \epsilon_i - \epsilon_{p}}   
- \sum^{occ}_{ij}\sum^{virt}_{a}
\frac{W(i\omega=0)_{piaj} W(i\omega=0)_{iajp}}
{\epsilon_a - \epsilon_i - \epsilon_j + \epsilon_{p}} \;,
\end{equation}
with
\begin{equation}
\label{screened stuff}
    W(i\omega=0)_{pqrs} = \int d\br d\br' 
    \phi_p(\br)
    \phi^\dagger_q(\br)
    W(\br,\br',i\omega=0)
    \phi_r(\br')
    \phi^\dagger_s(\br') \;.
\end{equation}
Using the transformation \cref{transform1,transform2} we write \eqref{screened stuff} as 
\begin{equation}
W(i\omega = 0)_{pqrs} = \sum_{\alpha} d_{pq\alpha}
c_{rs \beta} \;,
\end{equation}
with
\begin{equation}
d_{pq\alpha} = 
\sum_{\beta}
c_{pq \beta} 
W(i\omega = 0)_{\alpha \beta} \;.
\end{equation}
When complex matrix algebra is used, inserting this transformation into \eqref{SigmaG3W2-static} increases the computational effort by a factor of 16 (notice that the denominator is always real) compared to the 1C case. To reduce the computational effort, we use real matrix algebra and define the intermediates
\begin{equation}
\label{intermediates}
\begin{aligned}
W^{R/I,R/I}_{pqrs} = &
    \sum_{\alpha} d^{R/I}_{pq\alpha}
c^{R/I}_{rs \beta} \\
e_{pqrs} = & W^{R,R}_{pqrs}
- W^{I,I}_{pqrs} \\
f_{pqrs} = & W^{R,I}_{pqrs}
+ W^{I,R}_{pqrs} \;.
\end{aligned}
\end{equation}
The final self-energy correction \eqref{SigmaG3W2-static} is then evaluated as 
\begin{equation}
\label{SigmaG3W2-static_imp}
\Sigma_{pp}^{G3W2}(\epsilon_{p}) = 
\sum^{occ}_{i}\sum^{virt}_{ab}
\frac{e_{paib} e_{aibp} - f_{paib} f_{aibp}}
{\epsilon_a + \epsilon_b  - \epsilon_i - \epsilon_{p}}   
- \sum^{occ}_{ij}\sum^{virt}_{a}
\frac{e_{piaj} e_{iajp} - f_{piaj} f_{iajp}}
{\epsilon_a - \epsilon_i - \epsilon_j + \epsilon_{p}} \;.
\end{equation}
Here, the by far most expensive step is the calculation of the first four intermediates defined in the first equation of \eqref{intermediates}. Therefore, evaluating \eqref{SigmaG3W2-static_imp} is by a factor of four more expensive than the corresponding 1C expression.

\section{\label{sec::computationalDetails}Computational Details}
\subsection{Choice of 2C-Hamiltonian}
The 2C $GW$ equations have been implemented in a locally modified development version of the Slater Type orbital (STO) based ADF engine\cite{adf2022} within the Amsterdam modelling suite (AMS2022).\cite{Ruger2022} In principle, the implementation is independent of the choice of the particular choice of the 2C Hamiltonian. In the work, we use the zeroth-order regular approximation (ZORA) Hamiltonian by van Lenthe et al,\cite{VanLenthe1993, VanLenthe1994, VanLenthe1996} which can be written as\cite{VanLenthe1996} 
\begin{equation}
\label{full2C}
\hat{h}_1^{ZORA}(\br) = \hat{h}_1^{ZORA,SR} (\br)+\hat{h}_{1}^{ZORA,SO}(\br) \;.
\end{equation}
The first term,
\begin{equation}
\label{zora-sr}
\hat{h}_1^{ZORA,SR}(\br) = v_{ext}(\br)+ \vec{p} \frac{c^2}{2c^2 - v_{ext}(\br)} \vec{p} 
\end{equation}
describes scalar relativistic effects and we use this Hamiltonian in all 1C calculations. The second term 
\begin{equation}
\label{zora-so}
\hat{h}_{1}^{ZORA,SO}(\br) = \frac{c^2}{\left(2c^2 - v_{ext}(\br)\right)^2}   
\vec{\sigma} 
\cdot 
\left(\nabla v_{ext}(\br) 
\times \vec{p} \right) 
\end{equation}
accounts for SOC. We employ the Hamiltonian \eqref{full2C} in all of the following 2C calculations. We also tested two Hamiltonians obtained from an exact transformation of the 4-component Dirac equation to 2-components (X2C and RA-X2C, respectively. In the latter variant, a regular approach to calculate the transformation matrix is used).\cite{Dyall1997, Kutzelnigg2005} In the X2C and RA-X2C method implemented in ADF, first the 4-component Dirac equation for a model potential (MAPA) of the molecule is calculated for the given basis set, using the modified Dirac equation (MDE) by Dyall\cite{Dyall1994} for X2C, or using the regular approach\cite{Sadlej1994} to the modified Dirac equation (RA-MDE) for RA-X2C. In the basis set limit the MDE and the RA-MDE should yield same results for the model potential (MAPA) but using a finite basis set, the results for MDE and RA-MDE will differ.\cite{Visscher1999} In a next step, these 4-component equations are transformed to a 2C form\cite{VanLenthe1996b} We found, that the particular choice of 2C Hamiltonian (ZORA, X2C or RA-X2C) only affects the final ionization potentials (IP) by a few 10 meV.

\subsection{Basis Sets}
In all calculations, we expand the spinors in \eqref{spinors} in all-electron STO basis sets of triple- and quadruple-$\zeta$ quality (TZ3P and QZ6P, respectively).\cite{Forster2021} The STO type basis sets in ADF are restricted to a maximum angular momentum of $l=3$, which complicates reaching the basis set limit for individual QP energies.\cite{Bruneval2020, Stuke2020} This is especially true for heavier elements with occupied $d$- or $f$-shells where higher angular momenta functions are needed to polarize the basis.\cite{Jensen2013} 

The numerical atomic orbital (NAO) based BAND engine\cite{TeVelde1991, Philipsen2022} of AMS can be used with basis functions of arbitrary angular momenta. To obtain converged QP energies we therefore augment our TZ3P and QZ6P basis sets and calculate scalar relativistic QP energies. In the choice of the higher angular momenta functions we follow the construction of the Sapporo-DKH3-(T,Q)ZP-2012 basis sets\cite{Noro2012,Noro2013} for all elements in the fourth to the sixth row of the periodic table. In the following we denote these basis sets as TZ3P+ and QZ6P+. Except for the Lanthanides, where the highest angular momenta are $l=5$ and $l=6$, the augmented TZ (QZ) basis set typically contains basis functions with angular momentum up to $l=4$ ($l=5$) for elements beyond the third row. The basis set definitions are included in the supporting information.

We then calculate all QP energies as follows: We first calculate complete basis set (CBS) limit extrapolated scalar relativistic QP energies with the BAND code using the expression
\begin{equation}
    \label{cbsExtra}
    \epsilon_n^{GW, \text{scalar}}(CBS) = \epsilon_n^{GW, \text{scalar}}(QZ6P+) - 
    \frac{\epsilon_n^{GW, \text{scalar}}(QZ6P+) - \epsilon_n^{GW, \text{scalar}}(TZ3P+)}{1 - \frac{N^{QZ}_{bas}}{N^{TZ}_{bas}}} \;,
\end{equation}
where $\epsilon_n^{GW, \text{scalar}}(QZ6P+)$ ($\epsilon_n^{GW, \text{scalar}}(TZ3P+)$) denotes the value of the QP energy calculated with QZ6P+ (TZ3P+) and $N^{QZ}_{bas}$ and $N^{TZ}_{bas}$ denote the respective numbers of basis functions (in spherical harmonics so that there are e.g. 5 $d$ and 7 $f$ functions). This expression is commonly used for the extrapolation of $GW$ QP energies to the complete basis set limit for localized basis functions.\cite{VanSetten2015} Spin-orbit corrections $\Delta_n^{2C}$ are then calculated with ADF using the QZ6P basis set,
\begin{equation}
   \Delta_n^{2C}(QZ6P) = \epsilon_n^{GW, \text{scalar}}(QZ6P) -  \epsilon_n^{GW, \text{2C}}(QZ6P)
\end{equation}
The corresponding QP energies are then obtained as 
\begin{align}
\label{gw_2C}
   \epsilon_n^{GW, \text{2C}}(CBS) = & \epsilon_n^{GW, \text{scalar}}(CBS) + \Delta_n^{2C}(QZ6P) \\
\label{g3w2_2C}
   \epsilon_n^{GW + G3W2, \text{2C}}(CBS) = & \epsilon_n^{GW, \text{2C}}(CBS) + \Sigma^{G3W2}_{nn}(QZ6P) \;.
\end{align}
This choice is well justified since the major part correction to the KS QP energies comes from the scalar relativistic part of the $GW$ correction. The spin-orbit correction and the $G3W2$ corrections are typically of the order of only a few hundred meV in magnitude (also see explicit values in the supporting information). Therefore, even relatively large errors in these quantities while only have a minor effect on the final results. Furthermore, the $G3W2$ contribution is expected to converge faster to the CBS limit than the $GW$ contribution.\cite{Peterson1993,Peterson1994, Irmler2019,Irmler2019a}  

\subsection{Technical Details}
We perform $G_0W_0$ calculations using PBE, PBE0 and BHLYP\cite{Becke1993} orbitals and eigenvalues. The latter functional contains 50 \% of exact exchange which is typically the optimal fraction for $G_0W_0$ QP energies for organic molecules.\cite{Bruneval2013, Bruneval2015, Zhang2022} ev$GW$ and qs$GW$ calculations are performed starting from PBE0 orbitals and eigenvalues. In all calculations we set the numerical quality to \emph{VeryGood}. The auxiliary bases used to expand 4-point correlation functions are automatically generated from products of primary basis functions. For this, we use a variant of an algorithm introduced in ref.~\citen{Ihrig2015} which has recently been implemented in ADF and BAND.\cite{Spadetto2023} The size of the auxiliary basis in this approach can be tuned by a single threshold which we set to $\epsilon_{aux}=1 \times 10^{-10}$ in all partially self-consistent calculations and to $\epsilon_{aux}=1 \times 10^{-8}$ for $G_0W_0$. This corresponds to a very large auxiliary basis which is typically around 12 times larger than the primary basis and eliminates PADF errors for relative energies of medium molecules almost completely.\cite{Spadetto2023} 

Imaginary time and imaginary frequency variables are discretized using non-uniform bases $\mathcal{T} = \left\{\tau_{\alpha}\right\}_{\alpha = 1, \dots N_{\tau}}$ and $\mathcal{W} = \left\{\omega_{\alpha}\right\}_{\alpha = 1, \dots N_{\omega}}$ of sizes $N_{\tau}$ and $N_{\omega}$, respectively, tailored to each system. More precisely, \eqref{TtoW} is implemented as
\begin{eqnarray}
    \overline{F}(i\omega_{\alpha}) = &  \Omega^{(c)}_{\alpha\beta} \overline{F}(i\tau_\beta) \\
    \underline{F}(i\omega_{\alpha}) = &  \Omega^{(s)}_{\alpha\beta} \underline{F}(i\tau_\beta) \;, 
\end{eqnarray}
where $\overline{F}$ and $\underline{F}$ denote even and odd parts of $F$. The transformation from imaginary frequency to imaginary time only requires the (pseudo)inversion of $\Omega^{(c)}$ and $\Omega^{(s)}$, respectively. Our procedure to calculate $\Omega^{(c)}$ and $\Omega^{(s)}$  as well as $\mathcal{T}$ and $\mathcal{W}$ follows Kresse and coworkers.\cite{Kaltak2014,Kaltak2014a,Liu2016} The technical specifications of our implementation have been described in the appendix of ref.~\citen{Forster2021}.

\subsection{\label{sec::computationalDetails-convergence}Convergence acceleration}

For the molecules in the SOC81 set, we have found that the ev$GW$ equations converge within 5-8 iterations within an accuracy of a few meV when the DIIS implementation of ref.~\citen{Veril2018} is used. All ev$GW$ results presented in this work have been obtained using this DIIS implementation with a convergence criterion of 3 meV.

On the other hand, using our own DIIS implementation of ref.~\citen{Forster2021a} the qs$GW$ equations often do not convergence for the systems in the SOC81 set. As discussed in the literature,\cite{Forster2021,Monino2022} this issue is related to multiple QP solutions which seem to occur frequently in systems containing heavy elements (also see discussion in section~\ref{sec::4.1}). More sophisticated DIIS algorithms might offer a solution to this problem.\cite{Pokhilko2022} For the present work, we have found that a linear mixing strategy with adaptive mixing parameter $\alpha_{mix}$ leads to stable convergence of the qs$GW$ equations after typically around 15 iterations. Specifically, we start the self-consistency cycle with $\alpha^{(0)}_{mix} = 0.3$. In case the SCF error decreases, we use the mixing parameter $\alpha^{(n)}_{mix} = \max \left\{1.2 \times \alpha^{(n-1)}_{mix},0.5\right\}$ in the $n$th iteration. In case the SCF error increases, we reset the mixing parameter to $\alpha^{(0)}_{mix}$.

\section{\label{sec::results}Results}

\subsection{\label{sec::4.1}Selection of Molecules from SOC81}

\begin{table}[hbt!]
    \centering
    \begin{tabular}{ll
S[table-format=3.2]%
S[table-format=3.2]%
S[table-format=3.2]%
S[table-format=3.2]%
S[table-format=3.2]%
S[table-format=3.2]%
S[table-format=3.2]%
S[table-format=3.2]}
\toprule
& & 
\multicolumn{4}{c}{{$G_0W_0$@PBE}}  &  
\multicolumn{4}{c}{{$G_0W_0$@PBE0}} \\
\multicolumn{2}{c}{{ADF}}  &  
\multicolumn{2}{c}{{BAND}} &
\multicolumn{2}{c}{{ADF}}  &  
\multicolumn{2}{c}{{BAND}} \\
\\ \cline{3-10}
Index & Name 
& {TZ3P} & {QZ6P}  & {TZ3P+} & {QZ6P+}  
& {TZ3P} & {QZ6P}  & {TZ3P+} & {QZ6P+} \\ 
\\
    \midrule 
   1 &     {\ce{AgBr}} &   8.76 &   8.85  &   9.05 &   9.03  &   9.01 &   9.13  &   9.22 &   9.17  \\ 
   2 &      {\ce{AgCl}} &   9.04 &   9.27  &   9.22 &   9.28  &   9.42 &   9.62  &   9.45 &   9.60 \\ 
   3 &      {\ce{AgI}} &   8.34 &   8.36  &   8.51 &   8.43  &   8.53 &   8.54  &   8.77 &   8.68 \\ 
  34 &     {\ce{CsCl}} &   7.75 &   7.66  &   7.54 &   7.77  &   7.88 &   8.10  &   7.92 &   8.15 \\ 
  35 &      {\ce{CsF}} &   7.93 &   8.03  &   7.90 &   8.33  &   8.63 &   8.80  &   8.58 &   8.99 \\ 
  36 &      {\ce{CsI}} &   6.81 &   6.78  &   6.92 &   6.80  &   7.08 &   7.12  &   7.20 &   7.08 \\ 
  37 &      {\ce{CuF}} &   9.45 &   9.57  &   9.76 &   9.44  &   9.33 &   9.64  &   9.57 &   9.76 \\ 
  44 &       {\ce{KI}} &   7.00 &   6.97  &   7.11 &   7.03  &   7.27 &   7.31  &   7.39 &   7.32 \\ 
  45 &      {\ce{Kr2}} &  12.87 &  13.03  &  13.18 &  13.14  &  13.09 &  13.29  &  13.62 &  13.44 \\ 
  46 &     {\ce{KrF2}} &  12.33 &  12.34  &  12.42 &  12.40  &  12.97 &  12.96  &  13.07 &  13.06 \\ 
  59 &     {\ce{RbBr}} &   7.29 &   7.26  &   7.44 &   7.51  &   7.60 &   7.71  &   7.83 &   7.84 \\  
  60 &     {\ce{RbCl}} &   7.09 &   7.70  &   7.21 &   7.70  &   7.95 &   8.14  &   7.99 &   8.13 \\
  61 &      {\ce{RbI}} &   6.90 &   6.84  &   7.00 &   6.92  &   7.17 &   7.19  &   7.30 &   7.22 \\ 
  69 &    {\ce{SrBr2}} &   9.07 &   9.17  &   9.27 &   9.31  &   9.43 &   9.56  &   9.81 &   9.72 \\ 
  70 &    {\ce{SrCl2}} &   9.49 &   9.66  &   9.49 &   9.71  &   9.90 &  10.06  &  10.12 &  10.13 \\ 
  71 &     {\ce{SrI2}} &   8.49 &   8.51  &   8.59 &   8.64  &   8.80 &   8.83  &   9.07 &   8.94 \\ 
  72 &      {\ce{SrO}} &   5.51 &   6.20  &   5.51 &   6.02  &   5.72 &   5.95  &   5.90 &   6.01 \\ 
  74 &     {\ce{TiI4}} &   8.58 &   8.54  &   8.66 &   8.63  &   8.97 &   9.02  &   9.09 &   9.15 \\
  76 &    {\ce{ZnCl2}} &  10.81 &  11.00  &  10.82 &  10.92  &  11.20 &  11.38  &  11.21 &  11.34 \\ 
  77 &     {\ce{ZnF2}} &  12.58 &  12.47  &  12.58 &  12.38  &  13.19 &  13.11  &  13.22 &  13.10 \\ 
  78 &     {\ce{ZnI2}} &   9.23 &   9.22  &   9.33 &   9.36  &   9.52 &   9.56  &   9.67 &   9.71 \\     
    \bottomrule 
    \end{tabular}
\caption{\label{tab::removed_I} $G_0W_0$@PBE and $G_0W_0$@PBE0 QP energies for a subset of SOC81 for different basis sets. All values are in eV.} 
\end{table}

The aim of this section is to benchmark the accuracy of different partially self-consistent $GW$ variants for the calculation of vertical IPs. This requires QP energies which are well converged with respect to all technical parameters, most notably the size of the single-particle basis and necessitates an extrapolation of the QP energies to the CBS limit.\cite{VanSetten2015} For this reason we excluded some molecules from our benchmarks for which we consider CBS limit extrapolation as unreliable for several reasons. These systems are collected in table~\ref{tab::removed_I}. These are the molecules containing the transition metals \ce{Ti} \ce{Cu}, \ce{Zn} and \ce{Ag} (with the exception of \ce{ZnBr2} and \ce{TiBr4}), the systems containing the group I and II elements \ce{Rb}, \ce{Sr} and \ce{Cs}, and \ce{Kr}. Additionally, we also excluded \ce{KI}. 

Many systems containing transition metals or group I and II elements are challenging for many $GW$ implementations which rely on AC of the self-energy.\cite{VanSetten2015, Govoni2018, Forster2021} This is due to the fact that the spectral weights of a single QP excitations are distributed over multiple peaks.\cite{Govoni2018} In such cases multiple solutions of the nonlinear QP equation may be found. Even if this is not the case, AC will typically be numerically unstable and result in large errors. This might then also cause erratic jumps in QP energies when changing the basis set from TZ3P (QZ6P) to TZ3P+ ( QZ6P+). This issue has for instance been observed for \ce{MgO}, \ce{AgCl} or \ce{Cu2} in the GW100 set\cite{VanSetten2015} and has also been investigated in ref.~\citen{Scherpelz2016} for SOC81. It should be noted that such issues are most pronounced for $G_0W_0$ methods based on GGA starting points.\cite{Govoni2018} This is also reflected in our results. For instance, for \ce{AgCl}, \ce{RbCl} and \ce{SrO}, we obtained differences of up to 0.7 eV between the TZ3P(+) and QZ6P(+) $G_0W_0$@PBE IPs. This most likely indicates that we find different solutions in our TZ3P and QZ6P calculations, respectively. However, with $G_0W_0$@PBE0 the differences between the TZ and QZ results are of the order of 0.1-0.2 eV only.

After elimination of the systems shown in table\ref{tab::removed_I}, a subset of 60 molecules from SOC81 remains on which we will focus in the following comparison. 


\subsection{\label{sec::results_west}Comparison to WEST}

\subsubsection{Scalar relativistic Ionization potentials}

\renewcommand*{\arraystretch}{0.4}
\sisetup{
  round-mode          = places, % Rounds numbers
  round-precision     = 2, % to 2 places
}

\noindent\begin{longtable}[c]{ll
S[table-format=3.2]%
S[table-format=3.2]%
S[table-format=3.2]%
S[table-format=3.2]%
}
\caption{\label{tab::IPs_WEST}Scalar $G_0W_0$@PBE and $G_0W_0$@PBE0 ionization potentials (IP) for the SOC81 database and deviations to the respective results from WEST. All values are in eV.} \\
\toprule
& & \multicolumn{2}{c}{IP} & \multicolumn{2}{c}{{$\Delta_{WEST}$}} \\
Index & Name & {$G_0W_0$@PBE} & {$G_0W_0$@PBE0} & {$G_0W_0$@PBE} & {$G_0W_0$@PBE0} \\
\midrule\endfirsthead\toprule 
& & \multicolumn{2}{c}{IP} & \multicolumn{2}{c}{{$\Delta_{WEST}$}} \\
Index & Name & {$G_0W_0$@PBE} & {$G_0W_0$@PBE0} & {$G_0W_0$@PBE} & {$G_0W_0$@PBE0} \\
\midrule\endhead\bottomrule\midrule%
\multicolumn{6}{r}{{Continued on next page}} \\ \bottomrule
\endfoot\bottomrule\endlastfoot
  4 &   {\ce{Al2Br6}} &  10.35 &  10.74 &  -0.03 &  -0.04 \\ 
  5 &    {\ce{AlBr3}} &  10.50 &  10.86 &  -0.03 &  -0.05 \\ 
  6 &     {\ce{AlI3}} &   9.23 &   9.55 &  -0.21 &  -0.23 \\ 
  7 &    {\ce{AsBr3}} &   9.74 &  10.13 &  -0.09 &  -0.04 \\ 
  8 &    {\ce{AsCl3}} &  10.46 &  10.81 &  -0.21 &  -0.19 \\ 
  9 &     {\ce{AsF3}} &  12.24 &  12.71 &  -0.25 &  -0.18 \\ 
 10 &     {\ce{AsF5}} &  14.36 &  15.31 &  -0.15 &   0.03 \\ 
 11 &     {\ce{AsH3}} &  10.24 &  10.45 &  -0.09 &  -0.10 \\ 
 12 &     {\ce{AsI3}} &   9.11 &   9.20 &   0.12 &  -0.19 \\ 
 13 &      {\ce{Br2}} &  10.38 &  10.60 &   0.05 &   0.01 \\ 
 14 &     {\ce{BrCl}} &  10.62 &  11.16 &  -0.17 &   0.10 \\ 
 15 & {\ce{C10H10Ru}} &   6.90 &   {--} &  -0.10 &   {--} \\ 
 16 &   {\ce{C2H2Se}} &   8.35 &   8.72 &  -0.13 &  -0.00 \\ 
 17 &   {\ce{C2H6Cd}} &   8.72 &   9.04 &  -0.10 &  -0.05 \\ 
 18 &   {\ce{C2H6Hg}} &   9.03 &   9.35 &   0.04 &   0.15 \\ 
 19 &   {\ce{C2H6Se}} &   8.19 &   8.31 &   0.02 &  -0.10 \\ 
 20 &   {\ce{C2H6Zn}} &   9.44 &   9.69 &   0.06 &   0.04 \\ 
 21 &   {\ce{C2HBrO}} &   8.99 &   9.35 &   0.01 &   0.07 \\ 
 22 &   {\ce{C4H4Se}} &   8.75 &   9.03 &   0.15 &   0.17 \\ 
 23 &     {\ce{CF3I}} &  10.28 &  10.58 &  -0.24 &  -0.23 \\ 
 24 &  {\ce{CH3HgBr}} &   9.56 &  10.00 &  -0.16 &  -0.11 \\ 
 25 &  {\ce{CH3HgCl}} &  10.12 &  10.63 &  -0.18 &  -0.13 \\ 
 26 &   {\ce{CH3HgI}} &   8.72 &   9.07 &  -0.36 &  -0.31 \\ 
 27 &     {\ce{CH3I}} &   9.29 &   9.61 &  -0.28 &  -0.17 \\ 
 28 &      {\ce{CI4}} &   8.73 &   9.15 &  -0.18 &  -0.13 \\ 
 29 &    {\ce{CaBr2}} &   9.70 &  10.05 &  -0.10 &  -0.19 \\ 
 30 &     {\ce{CaI2}} &   8.92 &   9.13 &  -0.17 &  -0.35 \\ 
 31 &    {\ce{CdBr2}} &  10.12 &  10.48 &  -0.09 &  -0.14 \\ 
 32 &    {\ce{CdCl2}} &  10.73 &  11.22 &  -0.18 &  -0.17 \\ 
 33 &     {\ce{CdI2}} &   9.21 &   9.49 &  -0.20 &  -0.27 \\ 
 38 &    {\ce{HgCl2}} &  10.60 &  11.06 &  -0.33 &  -0.29 \\ 
 39 &       {\ce{I2}} &   9.48 &   9.53 &   0.07 &  -0.11 \\ 
 40 &      {\ce{IBr}} &   9.61 &   9.78 &  -0.20 &  -0.26 \\ 
 41 &      {\ce{ICl}} &   9.92 &  10.13 &  -0.22 &  -0.24 \\ 
 42 &       {\ce{IF}} &  10.20 &  10.60 &  -0.35 &  -0.20 \\ 
 43 &      {\ce{KBr}} &   7.74 &   8.05 &   0.18 &  -0.16 \\ 
 47 &    {\ce{LaBr3}} &  10.07 &  10.13 &   0.17 &  -0.28 \\ 
 48 &    {\ce{LaCl3}} &  10.67 &  11.15 &  -0.06 &  -0.11 \\ 
 49 &     {\ce{LiBr}} &   8.82 &   9.12 &  -0.13 &  -0.23 \\ 
 50 &      {\ce{LiI}} &   8.16 &   8.30 &  -0.19 &  -0.35 \\ 
 51 &    {\ce{MgBr2}} &  10.42 &  10.78 &  -0.07 &  -0.13 \\ 
 52 &     {\ce{MgI2}} &   9.38 &   9.74 &  -0.24 &  -0.23 \\ 
 53 &   {\ce{MoC6O6}} &   8.41 &   {--} &  -0.14 &   {--} \\ 
 54 &     {\ce{NaBr}} &   7.88 &   8.45 &  -0.18 &  -0.22 \\ 
 55 &      {\ce{NaI}} &   7.64 &   7.75 &  -0.01 &  -0.37 \\ 
 56 &     {\ce{OsO4}} &  11.75 &  12.33 &   0.01 &  -0.08 \\ 
 57 &     {\ce{PBr3}} &   9.51 &   9.87 &  -0.09 &  -0.05 \\ 
 58 &    {\ce{POBr3}} &  10.63 &  11.02 &   0.08 &   0.01 \\ 
 62 &     {\ce{RuO4}} &  11.32 &  12.00 &  -0.13 &  -0.19 \\ 
 63 &    {\ce{SOBr2}} &  10.00 &  10.59 &  -0.17 &   0.02 \\ 
 64 &    {\ce{SPBr3}} &   9.39 &   9.77 &  -0.06 &  -0.05 \\ 
 65 &    {\ce{SeCl2}} &   9.06 &   9.45 &  -0.18 &  -0.08 \\ 
 66 &     {\ce{SeO2}} &  10.88 &  11.61 &  -0.15 &   0.00 \\ 
 67 &   {\ce{SiBrF3}} &  11.74 &  11.94 &  -0.04 &  -0.16 \\ 
 68 &    {\ce{SiH3I}} &   9.69 &   9.95 &  -0.24 &  -0.22 \\ 
 73 &    {\ce{TiBr4}} &   9.89 &  10.56 &  -0.09 &  -0.01 \\ 
 75 &    {\ce{ZnBr2}} &  10.47 &  10.87 &  -0.05 &  -0.03 \\ 
 79 &    {\ce{ZrBr4}} &  10.20 &  10.80 &  -0.06 &   0.02 \\ 
 80 &    {\ce{ZrCl4}} &  11.14 &  11.72 &  -0.21 &  -0.21 \\ 
 81 &     {\ce{ZrI4}} &   9.19 &   9.44 &   0.02 &  -0.18 \\ 
\midrule 
MD  & & &  & -0.11 & -0.12 \\
MAD & & &  & 0.14 & 0.15 \\
MAX & & &  & 0.36 & 0.37 \\
\bottomrule 
\end{longtable} 

\begin{table}[hbt!]
    \centering
    \begin{tabular}{lll}
    \toprule 
    & WEST & ADF/BAND \\
    \midrule 
     Single-particle basis     &  Plane-wave & Slater type orbital \\
     All-electron              &  No         & Yes \\ 
     Frequency treatment       &  Contour deformation & Analytical continuation \\
     QP equations              &  Secant method & Bisection \\
     Relativistic Hamiltonian  &  2C-pseudopotentials & ZORA \\
     2C self-energy            &  Static part only & Static and Dynamic part \\
     \bottomrule
    \end{tabular}
    \caption{Comparison of the implementations of 2C-$G_0W_0$ in WEST and ADF/BAND.}
    \label{tab::technical_details}
\end{table}

In this section~\ref{sec::results_west}, we compare our results for the remaining molecules in the SOC81 to the ones calculated by Scherpelz et al.\cite{Scherpelz2016} with the WEST code.\cite{Govoni2015, Yu2022}
Table~\ref{tab::IPs_WEST} shows our scalar relativistic IPs using $G_0W_0$@PBE and $G_0W_0$@PBE0 and the deviations to the results from WEST from ref.~\citen{Scherpelz2016}. ADF tends to predict lower IPs than WEST, independent of the starting point of the $G_0W_0$ calculation. Interestingly, with MADs of 140 meV for $G_0W_0$@PBE and 150 meV for $G_0W_0$@PBE0, the deviations are around twice as large as the ones we obtained for the GW100 database.\cite{Govoni2018,Forster2021} 

Several technical aspects of the $GW$ implementations in ADF/BAND and WEST which are summarized in table~\ref{tab::technical_details} might contribute to these discrepancies. As discussed in the preceding section, these are mainly related to the different frequency treatments in both codes as well as differences in the single-particle basis. Importantly, WEST is based on PPs while we used all-electron basis sets in all ADF and BAND calculations. As already discussed extensively by Scherpelz and Govoni,\cite{Scherpelz2016} the choice of the PP and the partitioning of core, semi-core and valence electrons might heavily affect the values of the IPs. For instance, in ref.~\citen{Scherpelz2016}, it was shown that using different valence configurations for Iodine might induce changes in IPs of the order of one eV. 

In all-electron calculations, this issue is completely avoided. However, possible issues might arise from inconsistencies in the augmentation of the TZ3P and QZ6P basis sets with additional high-$l$ functions. While it can be verified by comparison of TZ3P (QZ6P) results to their TZ3P+ ( QZ6P+) counterparts that adding any higher angular momenta functions will improve the quality of the AO basis, the effect is typically more pronounced on the TZ than on the QZ level. This might then lead to larger inaccuracies in the CBS limit extrapolation than in plane-wave based implementations.

\subsubsection{Changes in Ionization Potentials due to Spin-Orbit Coupling} 

\begin{figure}[hbt!]
    \centering
    \includegraphics[width=0.7\textwidth]{pics/so_splitting_west.png}
    \caption{Comparison of the IP shift due to spin-orbit coupling as calculated with ADF compared to WEST for $G_0W_0$@PBE and $G_0W_0$@PBE0. All values are in eV.}
    \label{fig::so_splitting_west}
\end{figure}

In fig.~\ref{fig::so_splitting_west} we plot the difference between the first IP in the scalar and the 2C relativistic case calculated with WEST (x-axis) against the one calculated with ADF. Overall, we find good agreement between both implementations. WEST tends to predict a larger shift due to SO coupling than ADF, especially for $G_0W_0$@PBE. This can be explained with the different division of scalar and spin–orbit relativistic effects in both codes (see table~\ref{tab::technical_details}). In particular, the division between scalar relativistic and SOC effects is not unique and depends on the method of separation.\cite{Visscher1999} 

\subsection{\label{sec::results_exp}Comparison to experiment}

\renewcommand*{\arraystretch}{0.4}
\sisetup{
  round-mode          = places, % Rounds numbers
  round-precision     = 2, % to 2 places
}

\noindent\begin{longtable}[c]{ll
S[table-format=3.2]%
S[table-format=3.2]%
S[table-format=3.2]%
S[table-format=3.2]%
S[table-format=3.2]%
S[table-format=3.2]%
}
\caption{\label{tab::IPs}First ionization potentials (IP) for the SOC81 database calculated with different 2C $GW$ methods. All values are in eV.} \\
\toprule
& & \multicolumn{3}{c}{$G_0W_0$} & & & \\ \cline{3-5}
Index & Name & {PBE} & {PBE0} & {BHLYP} & {ev$GW$@PBE0} & {qs$GW$} & {exp.} \\
\midrule\endfirsthead\toprule 
& & \multicolumn{3}{c}{$G_0W_0$} & & & \\ \cline{3-5}
Index & Name & {PBE} & {PBE0} & {BHLYP} & {ev$GW$@PBE0} & {qs$GW$} & {exp.} \\
\midrule\endhead\bottomrule\midrule%
\multicolumn{8}{r}{{Continued on next page}} \\ \bottomrule
\endfoot\bottomrule\endlastfoot
  4 &   {\ce{Al2Br6}} &  10.33 &  10.70 &  10.95 &  11.06 &  11.06 &  10.97 \\ 
  5 &    {\ce{AlBr3}} &  10.48 &  10.82 &  11.03 &  11.16 &  11.15 &  10.91 \\ 
  6 &     {\ce{AlI3}} &   9.13 &   9.40 &   9.65 &   9.83 &   9.82 &   9.66 \\ 
  7 &    {\ce{AsBr3}} &   9.72 &  10.05 &  10.24 &  10.38 &  10.39 &  10.21 \\ 
  8 &    {\ce{AsCl3}} &  10.45 &  10.81 &  11.11 &  11.17 &  11.15 &  10.90 \\ 
  9 &     {\ce{AsF3}} &  12.24 &  12.71 &  13.11 &  13.18 &  13.19 &  13.00 \\ 
 10 &     {\ce{AsF5}} &  14.35 &  15.29 &  15.78 &  16.14 &  16.14 &  15.53 \\ 
 11 &     {\ce{AsH3}} &  10.24 &  10.45 &  10.64 &  10.76 &  10.77 &  10.58 \\ 
 12 &     {\ce{AsI3}} &   8.95 &   8.94 &   9.27 &   9.18 &   9.26 &   9.00 \\ 
 13 &      {\ce{Br2}} &  10.24 &  10.45 &  10.49 &  10.69 &  10.66 &  10.51 \\ 
 14 &     {\ce{BrCl}} &  10.52 &  11.05 &  11.05 &  11.18 &  11.18 &  11.01 \\ 
 15 & {\ce{C10H10Ru}} &   6.89 &   7.12 &   7.44 &   7.44 &   7.40 &   7.45 \\ 
 16 &   {\ce{C2H2Se}} &   8.35 &   8.71 &   8.88 &   8.94 &   8.95 &   8.71 \\ 
 17 &   {\ce{C2H6Cd}} &   8.72 &   9.04 &   9.32 &   9.45 &   9.46 &   8.80 \\ 
 18 &   {\ce{C2H6Hg}} &   9.04 &   9.38 &   9.64 &   9.67 &   9.71 &   9.32 \\ 
 19 &   {\ce{C2H6Se}} &   8.18 &   8.30 &   8.53 &   8.63 &   8.63 &   8.40 \\ 
 20 &   {\ce{C2H6Zn}} &   9.44 &   9.69 &   9.90 &  10.06 &  10.05 &   9.40 \\ 
 21 &   {\ce{C2HBrO}} &   8.98 &   9.34 &   9.61 &   9.61 &   9.61 &   9.10 \\ 
 22 &   {\ce{C4H4Se}} &   8.75 &   9.03 &   9.19 &   9.26 &   9.22 &   8.86 \\ 
 23 &     {\ce{CF3I}} &  10.04 &  10.31 &  10.58 &  10.56 &  10.52 &  10.45 \\ 
 24 &  {\ce{CH3HgBr}} &   9.47 &   9.89 &  10.09 &  10.31 &  10.35 &  10.16 \\ 
 25 &  {\ce{CH3HgCl}} &  10.10 &  10.60 &  10.82 &  11.13 &  11.17 &  10.84 \\ 
 26 &   {\ce{CH3HgI}} &   8.54 &   8.84 &   9.05 &   9.26 &   9.24 &   9.25 \\ 
 27 &     {\ce{CH3I}} &   9.06 &   9.35 &   9.46 &   9.54 &   9.50 &   9.52 \\ 
 28 &      {\ce{CI4}} &   8.52 &   8.94 &   9.23 &   9.24 &   9.23 &   9.10 \\ 
 29 &    {\ce{CaBr2}} &   9.61 &   9.94 &  10.19 &  10.36 &  10.32 &  10.35 \\ 
 30 &     {\ce{CaI2}} &   8.73 &   8.89 &   9.10 &   9.26 &   9.20 &   9.39 \\ 
 31 &    {\ce{CdBr2}} &  10.02 &  10.36 &  10.60 &  10.71 &  10.68 &  10.58 \\ 
 32 &    {\ce{CdCl2}} &  10.71 &  11.19 &  11.40 &  11.72 &  11.71 &  11.44 \\ 
 33 &     {\ce{CdI2}} &   8.99 &   9.23 &   9.47 &   9.63 &   9.60 &   9.57 \\ 
 38 &    {\ce{HgCl2}} &  10.55 &  11.00 &  11.25 &  11.48 &  11.49 &  11.50 \\ 
 39 &       {\ce{I2}} &   9.21 &   9.23 &   9.42 &   9.49 &   9.44 &   9.35 \\ 
 40 &      {\ce{IBr}} &   9.38 &   9.51 &   9.79 &   9.85 &   9.80 &   9.85 \\ 
 41 &      {\ce{ICl}} &   9.68 &   9.86 &  10.14 &  10.08 &  10.17 &  10.10 \\ 
 42 &       {\ce{IF}} &   9.90 &  10.29 &  10.53 &  10.52 &  10.52 &  10.62 \\ 
 43 &      {\ce{KBr}} &   7.66 &   7.94 &   8.11 &   8.40 &   8.36 &   8.82 \\ 
 47 &    {\ce{LaBr3}} &  10.04 &  10.05 &  10.49 &  10.61 &  10.57 &  10.68 \\ 
 48 &    {\ce{LaCl3}} &  10.66 &  11.14 &  11.54 &  11.66 &  11.66 &  11.29 \\ 
 49 &     {\ce{LiBr}} &   8.73 &   9.02 &   9.20 &   9.43 &   9.39 &   9.44 \\ 
 50 &      {\ce{LiI}} &   7.98 &   8.08 &   8.30 &   8.46 &   8.47 &   8.44 \\ 
 51 &    {\ce{MgBr2}} &  10.34 &  10.65 &  10.81 &  11.04 &  11.01 &  10.85 \\ 
 52 &     {\ce{MgI2}} &   9.16 &   9.49 &   9.68 &   9.84 &   9.78 &  10.50 \\ 
 53 &   {\ce{MoC6O6}} &   8.38 &   8.67 &   9.00 &   8.88 &   8.84 &   8.50 \\ 
 54 &     {\ce{NaBr}} &   7.79 &   8.34 &   8.54 &   8.85 &   8.82 &   8.70 \\ 
 55 &      {\ce{NaI}} &   7.48 &   7.53 &   7.78 &   7.98 &   7.91 &   8.00 \\ 
 56 &     {\ce{OsO4}} &  11.74 &  12.31 &  12.71 &  12.93 &  12.70 &  12.35 \\ 
 57 &     {\ce{PBr3}} &   9.49 &   9.84 &  10.03 &  10.16 &  10.14 &   9.99 \\ 
 58 &    {\ce{POBr3}} &  10.54 &  10.90 &  11.19 &  11.28 &  11.28 &  11.03 \\ 
 62 &     {\ce{RuO4}} &  11.32 &  12.00 &  12.55 &  12.60 &  12.54 &  12.15 \\ 
 63 &    {\ce{SOBr2}} &   9.97 &  10.54 &  10.76 &  10.88 &  10.82 &  10.54 \\ 
 64 &    {\ce{SPBr3}} &   9.38 &   9.75 &   9.95 &  10.08 &  10.08 &   9.89 \\ 
 65 &    {\ce{SeCl2}} &   9.03 &   9.43 &   9.65 &   9.70 &   9.74 &   9.52 \\ 
 66 &     {\ce{SeO2}} &  10.87 &  11.61 &  12.35 &  12.09 &  12.09 &  11.76 \\ 
 67 &   {\ce{SiBrF3}} &  11.63 &  11.81 &  12.05 &  12.17 &  12.13 &  12.46 \\ 
 68 &    {\ce{SiH3I}} &   9.48 &   9.70 &   9.88 &  10.00 &   9.95 &   9.78 \\ 
 73 &    {\ce{TiBr4}} &   9.83 &  10.48 &  10.77 &  10.86 &  10.82 &  10.59 \\ 
 75 &    {\ce{ZnBr2}} &  10.34 &  10.74 &  10.84 &  11.03 &  11.00 &  10.90 \\ 
 79 &    {\ce{ZrBr4}} &  10.16 &  10.71 &  10.93 &  11.01 &  10.99 &  10.86 \\ 
 80 &    {\ce{ZrCl4}} &  11.13 &  11.70 &  12.13 &  12.32 &  12.32 &  11.94 \\ 
 81 &     {\ce{ZrI4}} &   9.05 &   9.25 &   9.56 &   9.62 &   9.54 &   9.55 \\ 
\midrule 
MAD & & 0.56 & 0.25 & 0.19 & 0.22 & 0.22 & \\
\bottomrule 
\end{longtable} 

\begin{figure}[hbt!]
    \centering
    \includegraphics[width=0.7\textwidth]{pics/boxes_fr.png}
    \caption{Distribution of the deviations of IPs (in eV) obtained with different 2C methods to the experimental reference values}
    \label{fig::boxes_fr}
\end{figure}

In this section~\ref{sec::results_exp}, we compare the different (partially self-consistent) $GW$ variants against experimental IPs. Table~\ref{tab::IPs} shows the first IPs calculated at the 2C level using \eqref{gw_2C} with five different flavors of $GW$: $G_0W_0$ based on PBE, PBE0 and BHLYP orbitals and eigenvalues ($G_0W_0$@PBE, $G_0W_0$@PBE0, $G_0W_0$@BHLYP respectively), ev$GW$ using PBE0 orbitals and eigenvalues (ev$GW$@PBE0) and qs$GW$. The last row shows the MAD with respect to experimental vertical IPs which we show in the last column of table~\ref{tab::IPs} for comparison. For the corresponding values including the $G3W2$ correction we refer to the supporting information. MADs of all considered methods are shown in table~\ref{tab::mads}. The Deviations to experiment are also visualized in figure~\ref{fig::boxes_fr}.

Since we take into account SO effects and since our IPs are complete basis set limit extrapolated, vertical experimental IPs are a reliable reference. Besides errors due to the technical parameters discussed in section~\ref{sec::results_west}, other potential sources of uncertainty are the neglect of vibronic effects in our calculations, as well as errors in experimental geometries. Due to the lack of high-quality data from other \emph{ab initio} calculations, these experimental reference values are however the most suitable for our purpose. 

Consistent with previous benchmark studies\cite{Bruneval2013, Rangel2016, Knight2016, Forster2022} for GW100 and another dataset of 24 organic acceptor molecules (ACC24),\cite{Richard2016a} the partially self-consistent variants as well as $G_0W_0$@BHLYP give the best IPs. BHLYP contains 50 \% of exact exchange which is typically optimal for small molecules.\cite{Bruneval2015, Zhang2022} As shown in figure~\ref{fig::boxes_fr} and also consistent with the results for GW100 and ACC24, $G_0W_0$@PBE and $G_0W_0$@PBE0 underestimate the reference values while both partially self-consistent variants overestimate them. With a mean signed deviation of almost zero eV, no clear trend in any direction can be observed for $G_0W_0$@BHLYP.

\begin{figure}[hbt!]
    \centering
    \includegraphics[width=0.7\textwidth]{pics/cross_deviations.pdf}
    \caption{MADs (lower triangle) and maximum deviations in eV of first IPs calculated with different 2-component methods to each other and to experiment of IPs.}
    \label{fig::cross_deviations_fr}
\end{figure}

It is also instructive to compare the performance of different 2C-$GW$ methods amongst each other. On the lower triangle, figure~\ref{fig::cross_deviations_fr} shows the MADs between different methods to each other and to experiment while the upper triangle shows the respective maximum deviations. As one would expect, the agreement of $G_0W_0$ to qs$GW$ becomes better with increased fraction of exact exchange in the KS reference. While $G_0W_0$@PBE has a MAD of 0.69 eV to qs$GW$, this number reduces to 0.15 eV for $G_0W_0$@BHLYP. Especially ev$GW$@PBE0 is in exceptionally close agreement to qs$GW$, with a MAD of 0.03 eV and a maximum deviation of 0.21 eV. This shows that ev$GW$@PBE0 can be used as a suitable approximation to qs$GW$ for the calculation of QP energies when the latter is difficult to converge (Also see section ~\ref{sec::computationalDetails-convergence}). 

\subsubsection{Shift of ionization potentials due to spin-orbit coupling }

\begin{figure}[hbt!]
    \centering
    \includegraphics[width=0.7\textwidth]{pics/so_splitting.png}
    \caption{Differences in 2C QP energies to 1C QP energies with $G_0W_0$ using different starting points (x-axis) compared to qs$GW$. All values are in eV.}
    \label{fig::so_splitting}
\end{figure}

In figure~\ref{fig::so_splitting} we compare the difference in the first IPs for the SOC81 set due to SOC among different $GW$ methods. On the x-axis, we plot the qs$GW$ IPs and on the y-axis the $G_0W_0$ ones for different starting points. A higher amount of exact exchange in the underlying exchange-correlation functional increases the difference between the IPs at the 1C and the 2C level. This can be explained by considering the more (less) pronounced relativistic contraction of the lower (upper) components of a degenerate orbital set that is split by the spin-orbit interaction\cite{Pyykko1979}. The ionization takes place from the upper, more diffuse, orbitals in which the exchange interaction is decreased as compared to the orbitals obtained with a scalar relativistic method. These changes in the exchange interaction induced by relativity are incompletely captured by an approximate exchange density functional approximation resulting in a too small spin-orbit splitting. Employing some non-local exchange, as done in DFT with hybrid functionals, or employing qsGW is required to obtain the full magnitude of this subtle effect of relativity. 

Consequently, qs$GW$ shows the largest spin-orbit splitting out of all $GW$ methods. Figure~\ref{fig::so_splitting}, however, also shows that qs$GW$ incorrectly predicts a large spin-orbit shift of around 0.2 eV for one molecule, \ce{OsO4}. This obtained shift is independent of the basis set and of the particular algorithm used to converge the qs$GW$ equations. The reason seems to be that the qs$GW$ equations converge to a wrong solution. Whether this issue is due to our particular implementation of the qs$GW$ method, for instance the AC or the partial neglect of SOC in the dynamical part of the self-energy is not clear at the moment.

% \begin{figure}[hbt!]
%     \centering
%     \includegraphics[width=0.7\textwidth]{pics/boxes_pbe0.png}
%     \caption{Distribution of the deviations of IPs (in eV) obtained with 1C-$G_0W_0$@PBE0, 2C-$G_0W_0$@PBE0 and 2C-$G_0W_0$@PBE0 + $G3W2$ to experimental reference values}
%     \label{fig::boxes_pbe0}
% \end{figure}

% \begin{figure}[hbt!]
%     \centering
%     \includegraphics[width=0.7\textwidth]{pics/boxes_qsGW.png}
%     \caption{Distribution of the deviations of IPs (in eV) obtained with 1C-qs$GW$, 2C-qs$GW$ and 2C-qs$GW$ + $G3W2$ to experimental reference values}
%     \label{fig::boxes_qsgw}
% \end{figure}

\begin{figure}[hbt!]
    \centering
    \includegraphics[width=1.0\textwidth]{pics/boxes_both.png}
    \caption{Distribution of the deviations to experimental reference values of IPs (in eV) obtained with a) 1C-$G_0W_0$@PBE0, 2C-$G_0W_0$@PBE0 and 2C-$G_0W_0$@PBE0 + $G3W2$ and b) 1C-qs$GW$, 2C-qs$GW$ and 2C-qs$GW$ + $G3W2$.}
    \label{fig::boxes_both}
\end{figure}

\begin{table}[hbt!]
    \centering
    \begin{tabular}{lccccc}
    \toprule 
    & \multicolumn{3}{c}{$G_0W_0$@} & & \\ \cline{2-4}
    &  PBE & PBE0 & BHLYP &  ev$GW$ &  qs$GW$\\ 
    \midrule
1C-$GW$        & 0.48 &     0.18 &     0.23 &     0.31 &     0.31 \\ 
2C-$GW$        & 0.56 &     0.25 &     0.19 &     0.22 &     0.22 \\ 
2C-$GW + G3W2$ & 0.50 &     0.18 &     0.22 &     0.30 &     0.29 \\ 
    \bottomrule
    \end{tabular}
    \caption{Mean absolute deviations (MAD) to experiment for the sOC81 set for different 1C-$GW$, 2C-$GW$ and 2C-$G3W2$ for different starting points and different levels of partial self-consistency. All values are in eV.}
    \label{tab::mads}
\end{table}

Generally, the SOC correction is negative, i.e. reduces the scalar relativistic IPs. This means, in case of $G_0W_0$@PBE0 the scalar relativistic results are in better agreement with experiment than the 2C ones. This is shown in the first two boxes in figure~\ref{fig::boxes_both}a). On the other hand, for the accurate partially self-consistent approaches it is crucial to take into account SOC, as shown in figure~\ref{fig::boxes_both}b) for qs$GW$. These observations are also reflected in the MADs shown in table~\ref{tab::mads}

\subsubsection{Effect of the perturbative $G3W2$ correction}

The perturbative inclusion of the $G3W2$ term increases the first IPs. In contrast, in ref.~\citen{Forster2022} it was shown that the $G3W2$ term tends to decrease the IPs in the ACC24 set. As shown in figure~\ref{fig::boxes_both}, in case of $G_0W_0$@PBE0 the inclusion of this contribution improves agreement with experiment, while for qs$GW$ it worsens it. Typically, the contribution of the $G3W2$ term to the IP is only of the order of about 0.1 eV. However, in some qs$GW$ calculations , we observe very large $G3W2$ shifts of up to 0.4 eV. Overall, our results indicate that the perturbative $G3W2$ correction should not be used in combination with an accurate $GW$ method for small molecules containing heavy elements.

\section{\label{sec::conclusions}Conclusions}
We have presented an all-electron, AO based 2C implementation of the GWA for closed-shell molecules in the ADF\cite{adf2022} and BAND\cite{Philipsen2022} engines of AMS\cite{Ruger2022}. As in our 1C $GW$ implementation,\cite{Forster2020b} we leverage the space-time formulation of the GWA, AC of the self-energy, and the PADF approximation to transform between the representations of 4-point correlation functions in the AO and the auxiliary basis to achieve formally cubic scaling with system size.\cite{Forster2020b} The AO-based implementation of the 2C-GWA is particularly efficient: The evaluation of the polarizability is only four times slower than in a 1C calculation. We furthermore only consider the 1-component contribution to the Green's function to evaluate the dynamical part of the self-energy. All in all, this leads to a 2C algorithm which is only about two to three times more expensive than its 1C counterpart. 

While the effect of SOC can faithfully be estimated by combining a 2C DFT calculation with a scalar relativistic $GW$ calculation,\cite{Scherpelz2016} the new implementation will be particularly useful to calculate optical excitations within the 2C-BSE@$GW$ method.

To verify the correctness of our implementation we have calculated the first IPs of the 81 molecules in the SOC81 dataset\cite{Scherpelz2016} and compared our results to the ones from the WEST code for a subset of 60 systems.\cite{Scherpelz2016} For $G_0W_0$@PBE and $G_0W_0$@PBE0 we found MADs to the WEST results of 140 and 150 meV, respectively, whereby the IPs calculated with ADF/BAND are typically lower than the ones from WEST. The discrepancy between both codes is almost twice as large as for the GW100 set.\cite{VanSetten2015, Govoni2018, Forster2021} However, reaching agreement between codes for SOC81 is more challenging  than for GW100 due to the relativistic effects and the presence of heavy elements which are more prone to errors due to incomplete single particle basis and PPs. As for the GW100 database,\cite{VanSetten2015} further benchmark results using different types of single-particle basis, for instance Gaussian type orbitals, will be necessary to clarify the origin of the discrepancies between both codes. 

Finally, we have used the new implementation to assess the accuracy of $G_0W_0$ based on different starting points and of partially self-consistent approaches for the first IPs of the molecules in the SOC81 set. As for other commonly used datasets like GW100\cite{VanSetten2015} of ACC24\cite{Richard2016a}, ev$GW$ and qs$GW$ give the best results, even though they overestimate the experimental vertical ionization energies. The explicit description of SOC in a 2C framework is crucial to reach good accuracy. In our implementation, however, the perturbative $G3W2$ correction worsens the agreement with experiment and should not be used in conjunction with partially self-consistent schemes. In general, it would be interesting to see if the deviations to experiment of our $GW$ results can be reproduced with other 2C implementations.

In our benchmarks, we restricted ourselves to 60 out of the 81 molecules in the SOC81 benchmark set since we observed issues with out AC treatment of the self-energy for the remaining molecules. We emphasize that this issue is in no way related to our 2C implementation. It is however important to address, since systems containing heavy elements, including transition metal compounds where problems with AC are ubiquitous, will be among the targets of 2C implementations. AC can be avoided by using analytical integration of the self-energy\cite{Bruneval2012, VanSetten2013, Bintrim2021} or contour deformation (CD) techniques.\cite{Lebegue2003, Govoni2015, Scherpelz2016, Golze2018} AC of the screened interaction can also be combined with CD of the self-energy\cite{Friedrich2019, Duchemin2020} to compute a single-matrix element of the self-energy in the MO basis with cubic scaling with system size. This technique is therefore suitable for $G_0W_0$ and also for ev$GW$ or BSE@$GW$ calculations where Hedin shifts\cite{Pollehn1998, Li2022a} or other rigid scissor-like shifts of the KS spectrum\cite{Vlcek2018b, Holzer2019, Wilhelm2021} can be employed to avoid the explicit calculation of all diagonal elements of the self-energy. Since in qs$GW$ the full self-energy matrix is needed, such an algorithm would scale as $\mathcal{O}\left(N^5\right)$ with system size and is therefore only suitable for small molecules. Together with the already mentioned convergence problems, this is in principle a strong argument against the use of qs$GW$ for such systems.

\appendix 

\section{\label{app::B}Proof of Eqs. 29 and 30}
In this appendix we proof \cref{kramers1,kramers2}, which are valid under Kramers symmetry. We employ relation \cref{kramers-symmetry} to first proof \eqref{kramers2}. In real space,
\begin{equation}
\begin{aligned}
    P^{(0)}(\br \uparrow,\br'\uparrow,i\tau) = & 
    -i \sum_{ia} e^{-\left(\epsilon_a - \epsilon_i\right)\tau}
    \phi_{i}^{\uparrow} (\br)
    \phi_{i}^{\uparrow^*} (\br')
    \phi_{a}^{\uparrow} (\br')
    \phi_{a}^{\uparrow^*} (\br) \\ 
 = & 
    -i \sum_{ia} e^{-\left(\epsilon_a - \epsilon_i\right)\tau}
    \phi_{i}^{\downarrow^*} (\br)
    \phi_{i}^{\downarrow} (\br')
    \phi_{a}^{\downarrow^*} (\br')
    \phi_{a}^{\downarrow} (\br) \\ 
    = & P^{(0)}(\br' \downarrow,\br\downarrow,i\tau) = 
    P^{(0)}(\br \downarrow,\br'\downarrow,i\tau)
\end{aligned}
\end{equation}
with the last equality due to the symmetry of $P^{(0)}$. In the same way, we also show the identity 
\begin{equation}
\begin{aligned}
    P^{(0)}(\br \uparrow,\br'\downarrow,i\tau) = & 
    -i \sum_{ia} e^{-\left(\epsilon_a - \epsilon_i\right)\tau}
    \phi_{i}^{\uparrow} (\br)
    \phi_{i}^{\downarrow^*} (\br')
    \phi_{a}^{\downarrow} (\br')
    \phi_{a}^{\uparrow^*} (\br) \\ 
   = & 
    -i \sum_{ia} e^{-\left(\epsilon_a - \epsilon_i\right)\tau}
    \phi_{i}^{\downarrow^*} (\br)
    \phi_{i}^{\uparrow} (\br')
    \phi_{a}^{\uparrow^*} (\br')
    \phi_{a}^{\downarrow} (\br) \\ 
    = & P^{(0)}(\br' \uparrow,\br\downarrow,i\tau) = 
    P^{(0)}(\br \downarrow,\br'\uparrow,i\tau) \;.
\end{aligned}
\end{equation}
After transformation to the AO basis, these are the identities in \eqref{kramers2}. 

\Cref{kramers1},  
\begin{equation}
\sum_{\sigma, \sigma' = \uparrow,\downarrow}
i G^{>^I}_{\mu \kappa, \sigma  \sigma'}(i\tau)
G^{<^R}_{\nu \lambda, \sigma'  \sigma}(-i\tau) + 
i G^{>^R}_{\mu \kappa,  \sigma  \sigma'}(i\tau)
G^{<^I}_{\nu \lambda, \sigma'  \sigma}(-i\tau) = 0 \;.
\end{equation} 
follows from the cancellation of terms in the sums due to the identities
\begin{align}
\label{appendixB_1}
    G^{>^I}_{\mu\kappa,\uparrow\uparrow}(i\tau)
G^{<^R}_{\nu\lambda, \uparrow\uparrow}(-i\tau) = &
- G^{>^I}_{\mu\kappa, \downarrow \downarrow}(i\tau)
G^{<^R}_{\nu \lambda, \downarrow\downarrow}(-i\tau) \\ 
\label{appendixB_2}
G^{>^R}_{\mu\kappa,\uparrow\uparrow}(i\tau)
G^{<^I}_{\nu\lambda, \uparrow\uparrow}(-i\tau) = &
- G^{>^R}_{\mu\kappa, \downarrow \downarrow}(i\tau)
G^{<^I}_{\nu \lambda, \downarrow\downarrow}(-i\tau) \\ 
\label{appendixB_3}
G^{>^I}_{\mu \kappa, \uparrow \downarrow}(i\tau)
G^{<^R}_{\nu \lambda, \downarrow  \uparrow}(-i\tau) = &
- G^{>^I}_{\mu \kappa, \downarrow \uparrow}(i\tau)
G^{<^R}_{\nu \lambda, \uparrow \downarrow}(-i\tau) \\ 
\label{appendixB_4}
G^{>^R}_{\mu \kappa, \uparrow \downarrow}(i\tau)
G^{<^I}_{\nu \lambda, \downarrow  \uparrow}(-i\tau) = &
- G^{>^R}_{\mu \kappa, \downarrow \uparrow}(i\tau)
G^{<^I}_{\nu \lambda, \uparrow \downarrow}(-i\tau) \;,
\end{align}
These relations follow directly from \cref{greensGWbasisU}, as in each of the four terms there is exactly one sign change upon applying Kramers' symmetry. 

\section{Computational timings}

\sisetup{
  round-mode          = places, % Rounds numbers
  round-precision     = 0, % to 0 places
}
\begin{table}[hbt!]
    \centering
    \begin{tabular}{ll
S[table-format=5.0]%
S[table-format=5.0]%
S[table-format=5.0]%
S[table-format=5.0]%
}
    \toprule 
    & & \multicolumn{2}{c}{{TZ3P}} & \multicolumn{2}{c}{{QZ6P}}  \\ \cline{3-6}
    & & {1C} & {2C} & {1C} & {2C} \\
    \midrule
     {$N_{\text{bas}}$}   & & \multicolumn{2}{c}{1566} & \multicolumn{2}{c}{2895} \\
     {Total}      & {[core h]} &  41  & 82   & 728    &  1995  \\
     {$P^{(0)}$ } & {[core h]}&  14  & 53   & 409    &  1655   \\
     {$W$}        & {[core h]} &   4  &  4   &  30    &    30  \\
     {$\Sigma$}   & {[core h]} &  21  & 20   & 205    &   213  \\
     \midrule
     {first IP}   & {[eV]} & {6.09} & {5.81} &   {6.13} & {5.78} \\ 
     \bottomrule
    \end{tabular}
    \caption{Computational timings and first IP of \ce{Ir(ppy)3} for different basis sets at the 1C and 2C level using $G_0W_0$@PBE0.}
    \label{tab::appendix_timings}
\end{table}

In this appendix we compare the computational timings of 1C and 2C $GW$ calculations in our implementation. We report here timings for Tris(2-phenylpyridine)iridium [\ce{Ir(ppy)3}], a molecule with 320 electrons which is widely used in organic light-emitting diodes (OLEDs) due to its high quantum yields, enabled by thermally activated delayed fluorescence (TADF).\cite{Samanta2017} Timing results for the full complex at the TZ3P and QZ6P level using the ADF engine are shown in table~\ref{tab::appendix_timings}. We note, that systems like \ce{Ir(ppy)3} which contain many first- and second-row atoms are suitable for AO-based implementations since they can exploit sparsity in the AO basis. For clusters of heavy elements, for instance the \ce{Pb14Se13} cluster considered in ref.~\citen{Scherpelz2016}, MO-based implementations are more suitable, even though their asymptotic scaling with system size is less favourable. 

As one would expect from the equations in section~\ref{sec::theory}, independently of the basis set the calculation of the polarizability is four times slower in the 2C case, while the timings for the other most time-consuming parts of a $G_0W_0$ calculation remain the same. In the QZ calculations, the timings are dominated by the calculation of the polarizability and therefore the 2C calculation is slower compared to the 1C calculation than for the TZ calculations.


\begin{acknowledgement}
Edoardo Spadetto acknowledges funding from the European Union's Horizon 2020 research and innovation program under grant agreement No 956813 (2Exciting).
\end{acknowledgement}

%%%%%%%%%%%%%%%%%%%%%%%%%%%%%%%%%%%%%%%%%%%%%%%%%%%%%%%%%%%%%%%%%%%%%

\begin{suppinfo}
All Quasiparticle energies calculated in this work. All basis set files.
\end{suppinfo}

%%%%%%%%%%%%%%%%%%%%%%%%%%%%%%%%%%%%%%%%%%%%%%%%%%%%%%%%%%%%%%%%%%%%%

\bibliography{all}

%%%%%%%%%%%%%%%%%%%%%%%%%%%%%%%%%%%%%%%%%%%%%%%%%%%%%%%%%%%%%%%%%%%%%

\begin{tocentry}
\includegraphics[width=\textwidth]{pics/toc.jpeg}
\end{tocentry}

%%%%%%%%%%%%%%%%%%%%%%%%%%%%%%%%%%%%%%%%%%%%%%%%%%%%%%%%%%%%%%%%%%%%%

\end{document}
